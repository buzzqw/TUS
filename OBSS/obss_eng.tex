\documentclass[a4paper,twoside,openany]{book}
\usepackage{quoting}
\usepackage{tcolorbox}
\usepackage{tikz}
\usetikzlibrary{shadows}
\usepackage{multicol}
\usepackage{tocloft}
\usepackage{lmodern}
\usepackage{caption}
\usepackage[utf8]{inputenc}
%\usepackage[utf8x]{inputenc}
\usepackage[T1]{fontenc}  %\usepackage[B1,T1]{fontenc}
\usepackage{setspace}
\usepackage[a4paper]{geometry}
\geometry{verbose,tmargin=2cm,bmargin=2cm,lmargin=2cm,rmargin=2cm}  %std
\setcounter{secnumdepth}{-1}
\usepackage{booktabs}
\usepackage{url}
\usepackage[italian]{babel}
\usepackage{setspace}
\usepackage{graphicx}
\usepackage{amssymb}
\usepackage{makeidx}
%\usepackage[allfiguresdraft]{draftfigure}  %senza figure, deve rimanere alla riga 24
\usepackage{multirow}
\usepackage{titlesec}
\usepackage[unicode=true, bookmarks=true,
pdftitle={OBSS},pdfauthor={Andres Zanzani},
breaklinks=false,pdfborder={0 0 1},backref=section,colorlinks=false]
{hyperref}
\hypersetup{colorlinks=true,linkcolor=blue,pdfcreator={LaTeX}}
\usepackage{bookmark}
\usepackage{yfonts}
\usepackage{lettrine}
\usepackage{calligra}
\renewcommand{\LettrineFontHook}{\calligra}
\usepackage{accanthis}
\usepackage{auncial}
\usepackage{fontspec}
\usepackage{ragged2e}

\setmainfont[Path=../altro/fonts/,BoldItalicFont=DejaVuSerif-BoldItalic.ttf,ItalicFont=DejaVuSerif-Italic.ttf,BoldFont=ReadexPro-bold.ttf,Ligatures=TeX,Scale=0.94]{ReadexPro-Regular.ttf} %togliere questa e fontspec per usare pdflatex

\usepackage{wrapfig}
\usepackage{fancyhdr}
\usepackage{tcolorbox}
\tcbuselibrary{skins}
\tcbset{colback=brown!10, fonttitle=\scshape}
\usepackage{imakeidx}
\usepackage{cancel}
\def\CountIndexOccurrences#1{%
	\expandafter\newcount\csname #1\endcsname%
	\expandafter\newcount\csname #1\endcsname%
	\def\indexentry##1##2{\expandafter\advance\csname #1\endcsname 1}%
	\IfFileExists{#1.idx}{\input{#1.idx}}{}%
}
\CountIndexOccurrences{OBSS}
\CountIndexOccurrences{Incantesimi}
\CountIndexOccurrences{Mostruario}
\CountIndexOccurrences{Tabelle}
\CountIndexOccurrences{OggettiMagici}
\def\TotalBox#1{\vfill%
	\fbox{Ci sono \expandafter\the\csname #1\endcsname\ voci in questo indice}\par}
\makeindex[columns=4, title=Indice Analitico, intoc=true]
\makeindex[columns=3, name=Tabelle, title=Lista delle Tabelle, intoc=true]
\makeindex[columns=4, name=Incantesimi, title=Lista degli Incantesimi, intoc=true]
\makeindex[columns=4, name=Mostruario, title=Lista dei Mostri, intoc=true]
\makeindex[columns=3, name=OggettiMagici, title=Lista degli Oggetti Magici, intoc=true]
\usetikzlibrary{shapes.misc,calc}
\definecolor{lightgray}{gray}{0.95}
\usetikzlibrary{shapes.misc,calc}
\definecolor{lightgray}{gray}{0.95}
\usepackage{fancyhdr}
\pagestyle{fancy}
\fancyhf{} 
\fancyhead[LE,RO]{\leftmark}
\fancyhead[RE,LO]{}
\fancyfoot[C]{\thepage}
\renewcommand{\sectionmark}[1]{\markboth{#1}{}}


\fancypagestyle{plain}{%
	%% Clear all headers and footers
	\fancyhf{}
	%% Right headers on odd pages
	\fancyhead[RO]{%
		\rotatebox{90}{
			\begin{tikzpicture}[overlay,remember picture]
				\node[fill=lightgray,text=black,
				font=\footnotesize,
				inner ysep=12pt, inner xsep=20pt,
				rounded rectangle,anchor=east,minimum width=7cm,
				xshift=-60mm,yshift=-21mm, text height=0.4cm]
				at ($ (current page.north east) + (-1cm,-0cm) + (-4*\thesection cm,0cm) $)
				{\sffamily\itshape\small\nouppercase{\leftmark}};
			\end{tikzpicture}
		}
	}
	%% Left headers on even pages
	\fancyhead[LE]{%
		\rotatebox{90}{
			\begin{tikzpicture}[overlay,remember picture]
				\node[fill=lightgray,text=black,
				font=\footnotesize,
				inner ysep=12pt, inner xsep=20pt,
				rounded rectangle,anchor=east,minimum width=7cm,
				xshift=-60mm,yshift=-4mm, text height=0.4cm]
				at ($ (current page.north west) + (1cm,0cm) + (-4*\thesection cm,0cm) $)
				{\sffamily\itshape\small\nouppercase{\leftmark}};
			\end{tikzpicture}
		}
	}
	\renewcommand{\headrulewidth}{0pt}
	\renewcommand{\footrulewidth}{0pt}
}
\pagestyle{plain}
\fancyfoot[C]{\thepage}
\renewcommand{\sectionmark}[1]{\markboth{#1}{}}
\usepackage{xltabular}
\usepackage{tabularx}
\usepackage{pdfpages}
\usepackage{hyperref}
\usepackage{tikz}
\usepackage[absolute,overlay]{textpos}
\usepackage{etoolbox}
\usepackage{soul}
\raggedbottom
\usepackage{array}
\newcolumntype{k}[1]{>{\centering\let\newline\\\arraybackslash\hspace{0pt}}m{#1}}
\newcolumntype{R}[1]{>{\raggedleft\let\newline\\\arraybackslash\hspace{0pt}}m{#1}}
\newcolumntype{D}[1]{>{\centering}m{#1}}
\newcolumntype{M}[1]{>{\centering\arraybackslash}m{#1}}
\titleformat{\section}{\filcenter\huge\bfseries\accanthis}{\thesection}{1em}\textsc{}
\titleformat{\subsection}{\Large\bfseries\accanthis}{\thesubsection}{1em}\textsc{}
\titleformat{\subsubsection}{\normalsize\bfseries\accanthis}{\thesubsubsection}{1em}\textsc{}
\def\changemargin#1#2{\list{}{\rightmargin#2\leftmargin#1}\item[]}
\let\endchangemargin=\endlist
\setcounter{tocdepth}{3}
\newtcolorbox{narratore}{
	enhanced, % enable advanced settings
	%left = 3mm,
	%width=0.45\textwidth,
	left = 9mm, % pushes text away from the left edge by 10mm
	sharp corners, % disables rounded corners
	rounded corners = southeast, % "round" the bottom right corner
	arc is angular, % make the "round" corner an angle
	arc = 3mm, % controls corner cut
	boxrule=0.6pt, % sets box line thickness
	underlay={%
		\path[fill=black] ([yshift=3mm]interior.south east)--++(-0.4,-0.1)--++(0.1,-0.2); % triangle
		\path[draw=black,shorten <=-0.05mm,shorten >=-0.05mm] ([yshift=3mm]interior.south east)--++(-0.4,-0.1)--++(0.1,-0.2); % triangle edge
		\path[fill=gray!50!black,draw=none] (interior.south west) rectangle node[brown!10]{\Huge\bfseries ?!} ([xshift=8mm]interior.north west);
	},
	drop fuzzy shadow }

\newtcolorbox{enfasi}{
	enhanced,
	arc=5pt,
	boxrule=0.3pt
}

\usepackage{zref-savepos,graphicx}
\newcommand{\filltopageendgraphics}[2][]{%\filltopageendgraphics[width=.5\linewidth]{image-a}
	\par
	\zsaveposy{top-\thepage}% Mark (baseline of) top of image
	\vfill
	\zsaveposy{bottom-\thepage}% Mark (baseline of) bottom of image
	\smash{\includegraphics[keepaspectratio=true,height=\dimexpr\zposy{top-\thepage}sp-\zposy{bottom-\thepage}sp\relax,#1]{#2}}%
	\par
}

%\usepackage[nospace]{varioref}
\newcommand\Yuge{\fontsize{42}{50}\selectfont}
\usepackage{accanthis}
\usepackage[framemethod=TikZ]{mdframed}

\begin{document}
\def \versione {0.85f} \fontsize{10}{12}\selectfont

\cleardoublepage \thispagestyle{empty} \tikz[remember picture,overlay]  \node[opacity=1] at (current page.center){\includegraphics[width=21cm,height=\pdfpageheight]{copertina.png}}; \begin{textblock*}{20cm}(10cm,7cm)\Huge {Old Bell School System}\\ \end{textblock*} \begin{textblock*}{22cm}(13.5cm,8cm) \Large {\textbf{(OBSS)}}\\ \end{textblock*} \begin{textblock*}{13cm}(9cm,11cm) \Huge{\color{black} \calligra\Huge{Fantasy Adventure Game}} \end{textblock*} \newpage~\thispagestyle{empty}  \newpage~\thispagestyle{empty} %deve rimanere su riga 180 cvv.bat

\newcommand\invisiblesection[1]{%
\refstepcounter{section}%
\addcontentsline{toc}{section}{\protect\numberline{\thesection}#1}%
\sectionmark{#1}}

%Define centered and automatically spaced columns for tabularx.
\newcolumntype{Q}{>{\centering\arraybackslash}X}

%Define left-aligned wrapping and automatically spaced columns for tabularx.
\newcolumntype{L}{>{\raggedright\arraybackslash}X}

%Define left-aligned wrapping columns for tabular.
\newcolumntype{h}{>{\raggedright\arraybackslash}p}

\newcommand{\riga}{\rule{\textwidth}{0.4pt}}

\bigskip
Dedicated to the only woman ever loved, the one who accompanies me in my dreams every day\\

Never give up on your dreams, persevere until they become real.

\vspace{\fill}
\begin{center}\textbf{\versione} - \today\end{center}
\thispagestyle{empty}


\newpage~\thispagestyle{empty}%%\newpage~\thispagestyle{empty}


\pagebreak

{\Huge \begin{center} Old Bell School System \end{center}}

\bigskip

\begin{center} {\LARGE Player and Storyteller's Manual}\\ \end{center}

{\large \begin{center} Fantasy Roleplaying Game Guide and Rules \end{center}}

\begin{center} of \end{center}

{\LARGE \begin{center} Andres Zanzani \end{center}}

\vspace{2cm}

\begin{center}
\includegraphics[keepaspectratio,width=0.50\textwidth]{immagini/copertina_old_scratch.png}
\end{center}

\vfill

\begin{mdframed}[roundcorner=10pt]

\medskip

\textbf{Playtesting}: Fabrizio Bonetti, Emanuele Pezzi, Leonardo Pezzi, Nicola Ricottone, Marco Valmori, Edoardo Zanzani, Isotta Zanzani, Federica Angeli, Samuele Mazzotti, Simona Bissi, Lucia Dolcini, Carlo Dall'Ara, DefinitelyNonMirko, Dario Galassi, Stefano Mannino, Francesco Converso.

\bigskip

\begin{flushleft}\textbf{Terms of use}: OBSS, Old Bell School System, is a registered trademark of Andres Zanzani (azanzani@gmail.com), licensed with Attribution-ShareAlike 4.0 International (CC BY-SA 4.0). See details in the Licensing chapter.
\end{flushleft}

\vspace{0.5cm}

\begin{center}
\includegraphics[keepaspectratio,width=0.25\textwidth]{immagini/CC_BY-SA_icon.svg.png}
\end{center}

\medskip

\end{mdframed}

\cleardoublepage
\pagebreak
\pagebreak
\setcounter{page}{1}
\pagenumbering{Roman}

\begin{multicols}{2}
{\small \tableofcontents{}}

\end{multicols}

\vfill

\begin{changemargin}{0.3cm}{0.3cm}\begin{tcolorbox}
"May you make all your Saving Throws!" Frank Mentzer, Spring 1985. Master Player's Book
\end{tcolorbox}\end{changemargin}

\pagebreak~
\pagebreak~

%\cleardoublepage
\setcounter{page}{1}
\pagenumbering{arabic}

\section{Introduction}

\begin{changemargin}{0.3cm}{0.3cm}\begin{tcolorbox}[enhanced,arc=5pt,boxrule=0.3pt]{You can find out more about a person in an hour of playing than in a year of conversation. (Plato)}\end{tcolorbox}\end{changemargin}\medskip

\smallskip

\begin{multicols}{2}
\lettrine[lines=2, lhang=0.33, loversize=0.25, findent=1.5em]{O}{BSS} is a tabletop role-playing game (RPG or Role Play Game in English) in which players create characters who will live fantastic and amazing adventures. The Narrator will take care of unraveling the story based on the choices of the characters. As in a narrative game, each character will actively contribute to the story with their choices, decisions and actions.

If you are a gamer you will soon realize that it is not an easy world, things are not given away nor are they simple. There will be more times they want to kill you and rob you than they want to buy you a drink.

This is a fantasy world full of monsters of the most varied abilities, yes strong, witty, charismatic wise! Always be prepared!

You will decide everything about your character, from his appearance, to his name, to his abilities and partly also what he possesses. Will he want to be a heartthrob pirate or a shy knight, a steppe barbarian or a sorcerer? Gold, honor, treasure, looting, your character's adventures will be punctuated by choices, battles and revelry. Your story will be sung by minstrels for centuries to come.

If you are the Narrator, however, you govern the world, the story, the adventure. Your role is to illustrate the scenario in which the players move and make decisions. Will you lead them into the depths of the earth in search of the forgotten Tome of Atmos or challenge the great Dragons for the crown of Omniscience?

Your task is not easy, use imagination, common sense and the main rule: \textbf{have fun}. When you're in difficulty, don't look for the precise rule, use your greatest ally: your imagination, add a pinch of common sense and try to amaze the players. The purpose is always and only one, to have fun together and grow, as players, as characters, as friends.

\begin{center}
\includegraphics[keepaspectratio,width=0.95\linewidth]{immagini/dice.png}

\emph{Typical RPG dice set}
\end{center}

In addition to this manual you will also need some dice, the classics used in role-playing games.
You will need the d4, d6, d8, d10, d12. They indicate a 4-sided die, a 6-sided die (you must have at least 3 or 4 of these!), an 8-sided die, a 10-sided die (they are usually sold in pairs, in order to obtain the d100) and the solitaire 12-sided die. Whenever you are asked to roll a die it will be written with the notation XdZ, that is, roll a die with Z faces X times. E.g. 4d6 indicates rolling the 6-sided die 4 times.

Even some miniatures may be necessary, otherwise even free snacks or chocolate eggs may be sufficient.

Inside this manual you will find everything you need, such as rules, to play, you (you) will need imagination, friendship, dice, a few sheets of paper and fun (sorry, chips and drinks are not included in the manual!)

Draw and use a map whenever the description and situation require accurate details and precise positioning, otherwise close your eyes and use your imagination, be accompanied by the voice of the Narrator and reconstruct the structures and locations in your mind.

Create and play the character that you like best, that you feel is yours and that makes you have fun, don't look for combinations of feats and abilities that give you more power otherwise sooner or later the character will get boring. The more you play, the more your character will gain experience and you will also play him better.

The Narrator will take care of telling you how much experience your character has gained, based on how you played, how you collaborated and helped the group, how much you contributed to the fun. He will keep you engaged in dangerous, often deadly battles, he will put your character to the test and as a group you will be able, perhaps not always, to resolve the intricate situations that the Narrator has prepared. Remember that the Storyteller always has the final say in any discussion.

This manual is complete, meaning inside you will find everything (except the dice!) to start playing!

You will also find many rules, yet many situations will have to be handled using the first rule: have fun. Common sense, experience and trust in the Storyteller will resolve any situation.

Whether you decide to be the Narrator or you decide to play a character, you need to read the following chapters carefully.
It's important that you have a good understanding of the basic rules and, above all, that you know where to look for everything when you need it!

\emph{For experts...}

At the base of OBSS is the dissatisfaction in playing the 5th edition of the famous role-playing game. The 5th flattens out the characters too much and the rules system, although truly efficient, does not allow the diversification, even if often exaggerated, that you could have in Pathfinder.

I needed a middle ground, a game still based on the d20 but which took the best of what had already been created and added what I liked from the countless role-playing games I studied and played. Don't try to recreate the 5ed or Pathfinder classes, you won't succeed, nor does OBSS want to lend itself to this job! In OBSS classes do not exist and the characters acquire depth and abilities depending on what they learn to do. The skills are dictated by the chosen profession and do not lead to exaggerated scores. The combat does not reach the epic complexity of Pathfinder nor the flatness of the 5th edition, but rather tries to be quick and tactical, effective and spectacular. The Golden Rules, critical management, give that extra something that allows players to have fun every time a die is rolled.

Magic takes up the standard canons of 5e but revisits them profoundly. So many spells have lost focus and the concept of boosting a spell using a higher slot is no longer there. The Golden Rules also apply to spells and this allows you to diversify the outcome, adding more tension to each spell.

The approach to alignment has completely changed, now becoming a fundamental aspect for character construction; no longer two skimpy letters (LB, CB...) but a choice based on character, morality and ethics.
The deities, sorry Patroni, have a dirtier and more direct role, read them carefully, they are not the usual gods.
The monsters are those of the 5th ed, modified to be tougher as there is no longer "bounded accuracy" you have better results on attack rolls and saving throws.

The License grants anyone the ability to create and produce wonderful adventures and expansions for OBSS.

The system wants to be more lethal than 5ed, more wounds, more suffering, without the concept of short or long rest or recovery of hit points based on hit dice. Enough hero characters anyway.

Good role-playing and teamwork will always guarantee excellent results, even in spells! Always participation and identification.

Finally, but I won't say it quietly, OBSS refers to the OSR movement, that is, it would like to be played according to those principles. Read the \hyperlink{OSR}{burning} chapter (page \pageref{burning}).

\medskip

\begin{center}
Happy reading and have fun!
\end{center}

\begin{flushright}
Andres Zanzani
\end{flushright}

\end{multicols}

\vfill

\begin{center}
\includegraphics[keepaspectratio,width=0.85\textwidth]{immagini/Dragon_by_Henry_Justice_Ford_grey.png}

\emph{The End of the Dragon - Henry Justice Ford}
\end{center}

\vspace{1cm}

\begin{changemargin}{0.3cm}{0.3cm}\begin{tcolorbox}
Although the masculine form of the appellation is typically used when listing the level titles of various character types, these names can easily be changed to feminine ones if desired. This is fantasy, what's in a name? In all cases \st{except some}(\emph{correction mine}) gender makes no difference to ability! (Gary Gygax, Advanced Dungeon \& Dragons Player Handbook)
\end{tcolorbox}\end{changemargin}


\pagebreak{}

\subsection{Common Terms}\label{Termini Comuni}

\begin{multicols}{2}

\lettrine[lines=2, lhang=0.33, loversize=0.25, findent=1.5em]{T}{i} list some terms\index{Common terms} and concepts that you will find repeated several times in the book.

\textbf{Roundings}: \index{Roundings}always down unless otherwise specified but with a minimum of 1. E.g. 7/2 = 3, 9/4=2, 1/2=1

\textbf{feats}: \index{feats}are the particular abilities that the character has learned to use. They are often similar to magical abilities, allowing particular actions and even subverting the rules at times. They are taken at level passes.

\textbf{Action}: \index{Action}is what you do in an interval of time. Everything the character does is measured in Actions. Fighting, casting spells, lock-picking, drinking potions, moving... in each round you can perform 3 Actions.

\textbf{Bonus}: \index{Bonus}any modifier due to external, environmental, magical, circumstance factors or that the Storyteller decides is a bonus or penalty to be applied to the die roll or difficulty in the test.

\textbf{Check/Test}: \index{Check}\index{Test} a check (or check) is the roll of 3d6 plus the score indicated by the characteristic and skill involved, modifiers given by skill and skill may be applied. circumstances.

\textbf{Class}: In OBSS there are no classes. Each character is built based on what he can do. So you won't find the word Class in the manual.

\textbf{Critical Hit}\index{Critical Hit}: when you roll 6s multiple times on the attack roll. Every two 6s, even following an Explosion of a 6, you apply only the extra weapon die to the damage caused.

\textbf{Magic Test}\index{Magic Test}: the Magic Test can be due to particular situations, for example when the character is injured or distracted, but it can also be requested by the player.

The Magic Test allows the character to go further in casting the spell and try to draw on and exploit more magic.

Depending on the results you could get advantages or disadvantages.

\textbf{Casting Spells under attack, threat, distraction..}:\index{Concentration Test}\index{Casting Spells under attack, threat, distraction..}\index{Distracted} when a spellcaster wants to use a spell but is disturbed, attacked, injured or otherwise distracted while casting a spell then he must make a magic check.

\textbf{Difficulty Class (DC)}:\index{Difficulty Class} \index{DC}indicates how difficult it is to succeed in a test. Can be used for skills (swimming..) as well as knowledge (poisons..). In spells it is the difficulty in resisting spells. Indicates what value to reach to pass and succeed in the test.

\textbf{Competence} \index{Competence}(skill)\index{Skill}: competence tells us what we know and its value indicates the degree of knowledge of the same. May it be studying a language, climbing, noticing little things.

\textbf{Weapon Proficiency (AC) (melee or ranged)} \index{Weapon Proficiency} is your ability to hit your opponent with melee weapons (swords, maces, fists..) or shooting/ranged (throwing daggers, bows, crossbows..)

\textbf{Magical Expertise (CM)}: \index{Magic Expertise}\index{CM}is your ability to use magic, the higher this value the more effective the spells will be, the more you will have available, the more you can throw.

\begin{center}
\includegraphics[keepaspectratio,width=0.4\textwidth]{immagini/spiritomagia2.png}
\end{center}

\textbf{Defense}: \index{Defense}Defense means the total value obtained from 10 + Shield + Armor + Dexterity + various and any bonuses. It represents the ability to not get hit, the higher it is the more difficult it is to get hit.

\textbf{+1d6 or -1d6}: it is a bonus or penalty to a check. Add or subtract a die roll of 6 to the check. The maximum penalty brings the number of dice rolled to 0 and the maximum bonus to +3d6.\index{Maximum bonus value}

\textbf{Distance}:\index{Distance} the distance, when it concerns combat, is measured in 1 meter squares.

\textbf{Devotee}\index{Devotee}: a character who has bonded with a Patron and has at least 3 Traits in common.

\textbf{Follower}: a character who has bonded with a Patron with 2 Traits in common

\textbf{Explosion of 6}:\index{Explosion of 6} when you make the Attack Roll, Saving Throw, Expertise Check, Magic Check, Initiative (read the specifications in the dedicated chapter) or in any case every time it is indicated that the explosion of 6 is valid, which means that for every die rolled that results in a 6, the die must be marked and withdrawn. The result of the new roll must also be added and if you roll a 6 you continue to reroll as long as you continue to roll 6s.

\textbf{Initiative}: \index{Initiative}is a Dexterity or Intelligence check. Establishes the order of actions in combat. Whoever has the highest test score goes first.

\textbf{Level}:\index{Level} The Level indicates the competence and power achieved by the character. It can indicate when the enemy is \emph{strong}.

\textbf{Spell level}: indicates the scale (from 1 to 9) of the spell's magical power.

\textbf{Enchanter, Magician:} \index{Enchanter} indicates any user of magic in any capacity.

\textbf{Melee}: \index{Melee}Melee means contact combat, hand-to-hand, sword to sword, or when your character fights with a weapon that has no range (bow, crossbows, slingshots). ...) against an opponent.
Any creature that the character can reach with his non-missile weapon is considered melee. A large creature (or one with a long weapon) might be in melee with the character but not vice versa.

\textbf{Movement}: \index{Movement}movement represents the ability to move. A Move Action represents the character moving. The higher the Move value, the more meters a creature can move.


\begin{center}
\includegraphics[width=0.40\textwidth]{immagini/merlin.png}

\emph{Merlin dictating his prophecies to his scribe. Robert de Boron's Merlin en prose (written ca 1200)}
\end{center}

\textbf{Narrator}:\index{Narrator} is the person who leads the adventure, establishes the rules and controls the elements of the story. The duty of every Storyteller is to entertain, be correct and use common sense. The Storyteller has the final say in all matters.

\textbf{Ability Test}\index{Ability Test}: it is a Competence test that uses the value of a Characteristic, such as Strength, Charisma, as a bonus.

\textbf{Patron}:\index{Patron} or deity. The Patron is a superior being who can grant powers and grant advantages.

\textbf{Penalties/Malus} \index{Penalties}: like the bonus, penalties or malus are values, numbers, which indicate unfavorable circumstances, penalizing spells or anything else that makes the test more difficult. Unfortunately, unlike bonuses, penalties, unless otherwise specified, always add to each other.

\textbf{PC, Character}: \index{Character}is the creature that is guided, managed, \emph{roleted} by the player.

\textbf{NPC}: \index{PNG}non-player character. They are particular characters, important or not, that the Narrator keeps to lead the adventure.

\textbf{Experience Points/PX}: \index{Experience Points} \index{PX} every time you solve difficulties, riddles, face monsters or find treasures, you play your character well and have fun you gain experience. These points accumulated over time establish the level and therefore the abilities of the character.

\textbf{Ability Scores}: \index{Ability Scores} \index{Statistics} also abbreviated to ability or statistics. Each character has 6 Characteristics: Strength (STR), Dexterity (DEX), Intelligence (INT), Wisdom (WIS) and Charisma (CH). The higher the score, the greater the value or ability of the character in that specific area.


%\medskip
%\begin{wrapfigure}{c}{0.5\textwidth}
%\begin{center}
% \includegraphics[width=0.35\textwidth]{immagini/Sakramentarz_tyniecki_02.png}
% 
% \emph{Sakramentarz Tyniecki: Majuskuła "V".}
%\end{center}

%\medskip

\textbf{Fate Points}:\index{Fate Points} \index{Beginner's Luck} or Beginner's Luck are points available that the player can transform into d6s to add to saving throws or attack rolls or rolls Skills. They are called Beginner's Luck because their number decreases as the character's level increases.

\textbf{Hit Points (Hit Points)}:\index{Hit Points} \index{Hit Points} indicate the creature's vital energy, resistance and luck in resisting wounds. As long as the creature has 1 hit point it will fight at its best, without problems (but it may also decide to run away rather than die!).

With each level up you gain a certain number of Hit Points, established by the rules. Each wound subtracts from this accumulation of energy and when you reach 0 (zero) Hit Points you faint, unable to act. If you are further wounded and your Hit Points drop to 10 + double your Constitution then you die.

\textbf{Damage Reduction (DR)}: \index{Damage Reduction} \index{DR} Some creatures have innate resistance to damage and wounds. This resistance is denoted as DR.

\textbf{Damage Resistance (DR)}, \textbf{Resistance}: \index{Damage Resistance}\index{DR}: A creature may have resistance to one type of damage. In this case it is considered to automatically halve the damage taken before applying any saving throws.

\textbf{Round}:\index{Round} combat or actions are divided into rounds. A round represents a time unit of approximately 10 seconds. During the round each creature has the opportunity to act according to its initiative and perform up to 3 Actions.

\textbf{Critical Success/Magic Critical Failure}\index{Magic Critical Success} \index{Magic Critical Failure}: in the case in which the player passes the Magic Test with critical points (two 1s or two 6s). Magical Critical Success leads to spectacular changes in the spell, otherwise bad things could happen to the caster.

\textbf{Saving Throw (Save)}:\index{Saving Throw} \index{Save}When a creature is subjected to a particular effect, a saving throw is often granted to mitigate or negate the effects. The saving throw is an action that takes no time or actions.

Saving throws involve reflexes and dodging (Reflexes), resisting poisons/diseases or changes to the body (Fortitude), or resisting mental attacks and effects that affect agency and will (Will).

\textbf{Critical Success/Critical Failure on Saving Throw}\index{Magic Critical Success on Saving Throw} \index{Critical Failure on Saving Throw}: depending on the spell in case of a critical success on the saving throw (Critical Success Successful save and at least two 6's rolled) the effects are halved further while in case of a critical failure (you fail the save and roll two 1's or one 1 and two 2's) you suffer even more damage.

\begin{center}
\includegraphics[width=0.5\textwidth]{immagini/Jan_Steen2.png}

\emph{Jan Havicksz. Steen}
\end{center}

\textbf{Attack Roll (TC)}:\index{Attack Roll} \index{TC}is an Attack check (Weapon Proficiency + Strength/Dexterity + feat + ability data from the weapons list...) against Defense (armor + shield + feat + magic...). The attack roll can be melee (i.e. for creatures close to your weapon, at melee range) or ranged (for bows, crossbows, but also thrown daggers...). Read the combat chapter carefully.

\textbf{Stroke}: \index{Stroke}indicates a component of the character. Each character chooses 5 Traits to compose and build her personality.

\textbf{Round}: \index{Round}is 10 minutes, i.e. 60 rounds

\textbf{One is bad luck}: \index{One is bad luck}if you roll a 1 with the data you remove 1 from the total result. This does not mean that a rolled 6 becomes a 5, the explosion of the 6 remains... except that you remove 1 from the final result. Said differently, 1 is equal to 0.

\begin{changemargin}{0.3cm}{0.3cm}\begin{enfasi}{
The game of D\&D (and \emph{OBSS}) has no losers or winners, it only has players who love to exercise their imagination. Players and the DM (\emph{Narrator}) share in creating adventures in fantastic lands where heroes abound and magic really works. In a sense, the game of D\&D has no rules, only suggestions of rules. No rule is inviolate, particularly if a new or changed rule will encourage creativity and imagination. The important thing is to enjoy the adventure. (Tom Moldvay, 03/12/1980)
}\end{enfasi}\end{changemargin}

\medskip

In the Manual you will find different types of boxes, each one has a precise meaning:

\medskip


\begin{changemargin}{0.3cm}{0.3cm}\begin{enfasi}{Example of box containing a quote or motivational phrase}\end{enfasi}\end{changemargin}

\begin{changemargin}{0.3cm}{0.3cm}\begin{tcolorbox}[title = Information for the player]Box containing information and clarifications for the Player.\end{tcolorbox}\end{changemargin}

\begin{changemargin}{0.3cm}{0.3cm}\begin{narratore}Box containing indications and suggestions for the Narrator\end{narratore}\end{changemargin}

\end{multicols}

\vfill


\begin{center}
\includegraphics[width=0.45\textwidth]{immagini/cavaliere2.png}
\end{center}



\pagebreak

\section{Breeds}\index{Breeds}

\begin{changemargin}{0.3cm}{0.3cm}\begin{enfasi}{The true voyage of discovery does not consist in finding new territories, but in possessing other eyes, seeing the universe through the eyes of another, of hundreds of others: to observe the hundred universes that each of them observes, that each of them is. (Marcel Proust)

\medskip

It is not the most intelligent of the species that survives; it's not even the strongest; the species that survives is the one that is able to adapt and adapt better to changes in the environment in which it finds itself. (Leon C. Megginson)}\end{enfasi}\end{changemargin}\medskip

\begin{multicols}{2}

\lettrine[lines=2, lhang=0.33, loversize=0.25, findent=1.5em]{Y}{eru} is a multifaceted world rich in diversity, cultural, natural and creatures.
It is the creatures that make the planet vital and rich, each one nourishes, contributes and enriches the knowledge of all the others.


\subsection{Humans}\index{Humans}\label{umani}

Humans with their desire for discovery, power, glory and violence and reproductive capacity are the dominant race.

The physical characteristics of humans are extremely varied like clothing, cultural and food traditions. The lifestyles can be the most disparate and original and everything only makes the character more \emph{human}.

Leave racism out of Yeru, there are enough wars already, there's no point in creating new ones just because someone uses axes instead of swords.

Humans were the race created by Ljust and Calicante together so that with their chaotic, changing and vital drive they could do and undo, continually starting from scratch and continuously improving.

\begin{center}
\includegraphics[height=0.4\textwidth]{immagini/uomovitruviano2.png}

\emph{Vitruvian Man - Leonardo da Vinci}
\end{center}


\textbf{Racial modifiers}: +1 to one characteristic of your choice

\textbf{Physical characteristics}: height 150-185 cm, 50-130 kg, life expectancy 65 years (50 + 2d10 years)

\textbf{Dimensions}: Medium

\textbf{Speed}: 9m

\textbf{Languages}: Common

\textbf{Advantage}: +1 Feat at first level. The first point assigned to AC or CM is doubled.

\subsection{Elves}\index{Elves}\label{elfi}

Elves are the race created directly by Ljust to lead the world with the elegance, intelligence and foresight of an immortal race.

After millennia of peace and life throughout Yeru, after natural and architectural beauty had spread harmoniously throughout the world, the creation of the new races and their expansionist drive led the elves to review their mission.

Suddenly from bearers of beauty, culture, passion and stimulus for the arts they became more reclusive, the fear that all that was beautiful could be lost arose in them.

They thus decided to isolate themselves more and more to preserve every form of art and beauty, from the written word to the arts, protected from those who could despise or make them ugly.

Their desire to preserve creation materializes in a pure fight against everything and everyone, against any creature that wants to live and create something new. Cattalm had poisoned their blood so profoundly that not even they had noticed.

What followed was a period of extremely nefarious and violent centuries dominated by the absolute will of the elves to destroy everything using any means.
All nations and civilizations paid a very high price both in terms of lives and in terms of a return to barbarism.

It took almost 500 years and the annihilation of entire nations and civilizations until all the other peoples decided to form a united front against the elves in a last-ditch attempt at salvation.

In what is called the Week of Hatred, hundreds of thousands of creatures and almost all the elves perished.

A radical purge followed, every elf found was killed and so on for almost another century.

Just under 100 years ago a new elven Queen, Licenea, undertook a journey to the court of every nation, risking lynching numerous times. She managed to obtain a peace treaty that could safeguard the very few remaining elves.

The surviving elves, although \emph{elderly}, have not lost their hatred, their blood remains tinged with Cattalm and they brood behind a treaty which they say prevents them from fulfilling their true destiny.

Fortunately the new elves, but not all, do not have this visceral hatred, their blood has not been stained by Cattalm and they would like to live a normal life in contact with all the other creatures even if they are well aware of how they are seen and treated by all others.

They are the young elves who not only want to preserve beauty but to live an almost infinite life where it itself is wonder.

Elves are generally taller and slimmer than humans. The eyes are always with light shades.

Elves value the written word and magic. They are a rational race, guided by a sharp mind and excellent senses, an interest in the extraordinary and in knowledge.

\begin{center}
\includegraphics[height=0.6\linewidth]{immagini/elf2-ai.png}
\end{center}

\textbf{Racial modifiers}: +1 Intelligence, +1 Dexterity, -1 Charisma

\textbf{Physical characteristics}: height 165-195 cm, 50-110 kg, life expectancy 60d100+ years

\textbf{Dimensions}: Medium

\textbf{Speed}: 9m

\textbf{Languages}: Elvish

\textbf{Advantage}: Twilight vision of 18 meters

\subsection{Dwarves}\index{Dwarves}\label{nani}

The dwarves are a stoic and severe race accustomed to the purest communism, without a true concept of property but of pure community of goods according to the idea that each dwarf works for the community and not for himself.

Dwarves are compact and set, reaching a maximum height of around 140cm with a stocky build that gives them a massive appearance. Both males and females proudly wear long hair, and men often decorate their beards with various kinds of clips and intricate braids. It is also true that bald dwarves are common, but not without beards. Dwarf women have no beards or excess hair. Sex is free and socialist.

Dwarves are guided by honor, tradition, and communism. They are often seen as grumpy, but they have a strong feeling of friendship and justice, they respect those who work hard and are committed to the community and the group.

Dwarves are the race created by Erondil with the help of Atmos.

They judge the Elves harshly because they were unable to complete and indeed betrayed the dictates of Creation and therefore they feel the task, the burden and the honor of forging creation and within creation the beauty and majesty of Erondil.

\begin{center}
\includegraphics[height=0.6\linewidth]{immagini/nana2-ai.png}
\end{center}

\textbf{Racial modifiers}: +1 Constitution, +1 Wisdom, -1 Dexterity

\textbf{Physical characteristics}: height 100-140 cm, 45-90 kg, life expectancy 450 years (400 + 1d100 years)

\textbf{Dimensions}: Medium

\textbf{Speed}: 6m

\textbf{Languages}: Dwarven

\textbf{Advantage}: Twilight vision of 18 meters

\subsection{Gnomes}\index{Gnomes}\label{gnomi}

Gnomes are small beings but full of energy and life. Gnomes are a race that appeared just over 1000 years ago, no one knows exactly where from.
In a short time, thanks to their innate curiosity, tenacity and inventiveness, they managed to create populous and rich cities, almost always within virgin forests.

Gnomes are deeply linked to nature, their relationship is almost symbiotic, a Gnome will never give up the sight of a tree and building with what nature provides.

The Gnomes have a deep respect for nature, the environment and animals, their perfectly functional and modern cities are built and created in the forest and never destroying it but rather enriching it.

Many Gnomes are inventors and builders capable of extraordinary imagination and ingenuity. Many of their inventions help and support the entire community and their social life is rich and supportive.

The most curious Gnomes often leave their communities, which are not closed to anyone who respects nature, and embark on a life of adventure to discover wonders and new works of ingenuity that they can pass on to the community.

A gnome forced to stay away from a natural environment suffers from the situation by becoming sad and apathetic, his need for nature is something physical and innate.

Gnomes get along well with anyone who loves nature and doesn't abuse it.

\textbf{Racial modifiers}: +1 Intelligence, +1 Charisma, -1 Strength

\textbf{Physical characteristics}: height 70-110 cm, 30-50 kg, life expectancy 650 years (600 + 1d100 years)

\textbf{Dimensions}: Small

\textbf{Speed}: 6m

\textbf{Languages}: Gnomish, Silvanus

\textbf{Advantage}: Druidic Artifice 1 per day.


\subsection{Half-Elf}\index{Half-Elf}\label{mezzelfo}


To an elf there is nothing more impure than a half-elf. No half-elf is born by the will of an Elf. Every half-elf is a child of violence. At least that's what the elves keep saying.

There are also rare half-elves born from romantic relationships. While usually short-lived, even by human standards, these secret encounters usually lead to the birth of half-elves, a race that descends from two cultures but is heir to neither. Half-elves can breed with each other, but even these {pure-blooded} half-elves are viewed as bastards by the elves.

Many elves see in a half-elf a betrayal of the original beauty, of the purity of creation.
Very few see a gesture of love and gift towards an increasingly ugly world.

Half-elves are taller than humans but shorter than elves. They inherit the slender build and attractive features of their elven lineage, but the color of their skin is normally dictated by their human side. Their eyes tend to be human-like in shape, but feature an exotic range of colors from amber and purple to emerald green and dark blue.

Half-elves understand loneliness and know that character is often more a product of life experience than race.

\medskip

\begin{center}
\includegraphics[height=0.6\linewidth]{immagini/half-elf-ai.png}
\end{center}


\textbf{Racial modifiers}: +1 to one Characteristic of your choice

\textbf{Physical characteristics}: height 160-185 cm, 50-100 kg, life expectancy 210 years (180 + 5d10 years)

\textbf{Dimensions}: Medium

\textbf{Speed}: 9m

\textbf{Languages}: Common or Elvish

\textbf{Advantage}: Twilight vision of 9 meters

\index{Half-orc}

\subsection{Half-orc}\label{mezzorco}

In the eyes of civilized cultures, half-orcs are monstrosities, the result of perversion and violence and are rarely the result of romantic unions, as such they are usually forced to grow up fast and hard, continually struggling to protect themselves or make a name for themselves. Some half-orcs spend their entire lives proving to full-blooded orcs that they are just as ferocious as them.

Half-orcs average 6 feet tall, with a powerful physique and greenish or gray skin. In males the canines often grow quite long until they protrude from their mouths and these \emph{fangs}, combined with a sometimes broad forehead and somewhat pointed ears, give them that well-known \emph{bestial} appearance.
In females the orc traits are much less marked and this often causes violence on the part of human males.

Despite these obvious orcish traits, half-orcs are as diverse as their human parents.

While within the orc tribes they must continually earn the respect of the \emph{purebloods}, things are no better in human society. Mocked, mocked, excluded and abandoned, half-orcs often find refuge in crime.

The orcs were created directly by Cattalm with the help of Calicante. Much of their creator's chaotic and destructive tendencies remain in the nature of half-orcs.

Half-orcs are constant victims of prejudice.

\begin{center}
\includegraphics[height=0.6\linewidth]{immagini/half-orc2-ai.png}
\end{center}

\textbf{Racial modifiers}: +2 Strength -1 Charisma

\textbf{Physical characteristics}: height 160-210 cm, 60 - 140 kg, life expectancy 70 years (50 + 5d10 years)

\textbf{Dimensions}: Medium

\textbf{Speed}: 9m

\textbf{Languages}: Common or Orcish

\textbf{Advantage}: Twilight vision of 9 meters

\subsection{Nibali}\index{Nibali}\label{nibali}

The Nibali are a race magically created to be slaves to the great spellcasters of the north.

Legend says that the terrible ice wizards, starting from a couple of humans (after thousands of them had died atrociously in previous experiments) managed to create by manipulating with magic, a more robust, stronger, more intelligent and at the same time more docile and disciplined with the honor that every child generated would be absolutely identical physically to the father or mother.

These things happened more than 2000 years ago and the kingdom of eternal evil collapsed under its own inability to evolve and understand the new challenges (and probably also thanks to the intervention of some Patron).

The Nibali continued to prosper and, taking advantage of what the ice kingdom had left them, they created one of the most modern, democratic and civilized civilizations in the world.

For many, the extreme efficiency and dedication of the Nibali is hateful, a yoke that leaves no room for personal freedom, for the Nibali it is just a natural way of progressing.

All Nibali are equal to each other with the same sex but the fact that they cannot have children with other races does not make them a closed or racist people, on the contrary, absorbing the best of each culture makes them better and also excellent diplomats. What really distinguishes one Nibali from another is the hairstyle, the tattoos, the clothing, being oneself. Respect for others and the Law are inextricably linked to their nature and yet there is nothing freer than a Nibali.

For a Nibali, rules and laws must promote peace and freedom, they must be just and those who maintain them must be understanding and wise. For a Nibali, freedom is not doing what you want but the right to do what you must.

The male Nibali is bald and has bright blue skin, and his eyes are purple. The women have amber skin, brown hair with blonde highlights, green eyes.

\textbf{Racial modifiers:} +1 Constitution, +1 Intelligence, - 1 Wisdom

\textbf{Physical characteristics}: height 183cm males, 170cm females, 50 - 120kg, life expectancy 130 years

\textbf{Dimensions}: Medium

\textbf{Speed}: 9m

\textbf{Languages}: Common

\textbf{Advantage}: Every round you reduce your Bleed damage by 1.


\subsection{Different}\index{Different}\label{diverso}\hypertarget{different}{}

Blessed or cursed, the Different are not like us. A Different is the result of a corrupt union. If the Patrons cannot act directly in Yeru, or at least this is what Gradh tries to avoid, they often instead use their powers to create a lineage loyal to them.

A Different is faithful to his Patron and cannot do otherwise. Luckily they are sterile with humans, otherwise they would have already dominated the world.

A Different is more robust and more intelligent. Unfortunately their hectic life is marked by a short duration. Usually a Different person does not exceed 50 years of age.

A Different is marked, somewhere on his body there is the symbol, a birthmark, of his Patron. Almost all (15/18) the Different have 3 or more concentric golden circles on the left wrist which can indicate the Patron (or Patrons in very rare cases) of whom they are \emph{children}.

Different is an attribute that can be given to any race. The racial modifiers are replaced with those of the Different and the life expectancy is halved (in the example below it refers to a Human). The original racial advantages remain valid and the Special one for the Different is added.

%\begin{center}
% \includegraphics[width=0.75\linewidth]{immagini/diverso.png}
%\end{center}

\textbf{Racial modifiers}: +1 to two Characteristics of your choice

\textbf{Physical characteristics}: height as original breed, life expectancy halved compared to original breed

\textbf{Dimensions}: as original breed

\textbf{Speed}: as original breed

\textbf{Languages}: as original breed

\textbf{Special}: Must identify a Patron and have at least 3 common Traits. He gains access to the Trait 5 sum power even if he has fewer points.


\subsection{Sornelian}\index{Sornelian}\index{Furryman}\label{sornelian}\hypertarget{Sornelian}{}

The genesis of the Sornelian is due to \hyperlink{efrem}{Efrem} (page \pageref{efrem}), its only known victory in the Thousand Year War. Efrem decided that nature should have a greater say in human affairs and decided that anthropomorphic creatures should be created to rebalance the excessive power of humanoid creatures.

A Sornelian has a head similar to that of an anthropomorphic animal but the body is more like a humanoid biped. Depending on the animal the Sornelian may also have fur, feathers, scales and claws. The size of a Sornelian depends greatly on the original animal ranging from small to large. The size does not grant specific advantages. The anthropomorphic appearance of a Sornelian is as varied as the animals they resemble.

\textbf{Racial modifiers}: +1 to one Characteristic of your choice

\textbf{Physical characteristics}: life expectancy depends on the longevity of the species, usually around 60+6d10 years.

\textbf{Dimensions}: depends on the original species, from 50cm to 270cm, from small to large.

\textbf{Speed}: 6 metres

\textbf{Languages}: Common. He gains +1d6 on checks to interact with animals of his bloodline.

\textbf{Special}: Upon creation the player chooses 2 abilities from those listed that best characterize his Sornelian. Some example animals are indicated in parentheses. The player can deal with the Storyteller to best create his Sornelian.

- \emph{Climber} (bear, cat, lizard, squirrel). You have hooked claws, sharp nails, or a serpentine tail. You have a climbing speed equal to your Speed.

- \emph{Predator} (bear, feline). Your natural attacks cause 1d6 lethal damage, and you are considered proficient in using them. You can use the Dark and Hatchet Weapon List to determine your abilities with natural attacks.

- \emph{Flying} (bat, eagle, owl, crow). You have vestigial wings. When you fall from at least 3 meters you can use a Reaction to glide and land safely, as a Feather Fall spell (p. \pageref{featherfall}), without taking damage from the fall. When you make a Long or High Jump check, you roll an extra 1d6.

- \emph{Runner} (deer, dog, horse, velociraptor). When you perform the Run Action you complete your Movement 3 times. Your Speed ​​is 30 feet.

- \emph{Swimmer} (crocodile, dolphin, frog, shark). You can hold your breath for up to an hour, you have a swimming speed equal to your Speed. You have Cold Damage Reduction of 4.

- \emph{Nocturnal creature} (cat, lizard, bat, dolphin, owl). You have low-light vision 30 feet.

- \emph{Excellent Senses (hearing, sight, smell...)} (dog, bat, owl). You have a +4 bonus on sense-based Awareness checks.

\begin{center}
\includegraphics[width=0.6\linewidth]{immagini/arpia-ai.png}

\emph{Harpy, bird woman, usually very angry...}

\end{center}

\subsection{Golian}\index{Golian}\index{Giant men}\label{golian}\hypertarget{golian}{}

The Golians, like the Sornelians, descend from the will of \hyperlink{erondil}{Erondil} (page \pageref{erondil}) and \hyperlink{gaya}{Gaya} (page \pageref{gaya} ) or from the desire to have creatures that could represent the majestic giants, their little children.

Golians have physical characteristics that resemble the giants of their family lines. Some Golians have gray or almost marbled skin like stone giants, others shoot sparks when they snap their fingers like fire giants, still others have blue skin like sky giants.


\begin{center}
\includegraphics[height=0.7\linewidth]{immagini/Herakles_Farnese_MAN_Napoli_Inv6001_n01.png}\\

\emph{Farnese Hercules, National Archaeological Museum of Naples}

\end{center}

\textbf{Racial modifiers:} +2 to Strength, -1 to one Characteristics of your choice

\textbf{Physical characteristics}: approximately 180/210cm tall. Life expectancy approximately 80 years (60+2d10)

\textbf{Dimensions}: medium size

\textbf{Speed}: 9 metres

\textbf{Languages}: Common, Giant of their lineage.

\textbf{Large Form}: starting from CM+CA at least 5 the Golian has the ability to enlarge and become large once a day every (CM+CA)/5. This transformation lasts 1 minute, Strength-based attack rolls and damage increase by 1d6, movement speed increases by 3 feet, reach increases to 6 feet.

\textbf{Stable}. You are considered Large to withstand tests of being grabbed or pushed.

\textbf{Special}: Each Golian descends from a line of giants and inherits peculiar powers from this. The indicated power is usable (CM+CA)/3, rounded up, per day.

- \emph{Cloud Giant}. A step in the sky. For the cost of two Actions you magically teleport up to 10 meters to an unoccupied space you can see.

- \emph{Fire Giant}. Burning embers. When you hit a target in melee you can deal 1d10 fire damage to that target. Cost 1 Reaction.

- \emph{Frost Giant}. Deep frost. When you hit a target in melee you can deal 1d6 cold damage and the creature's movement speed decreases by 10 feet until the end of your next round. Cost 1 Reaction.

- \emph{Hill Giant}. Angry blow. When you critically hit a creature your size or smaller in melee, it falls prone. It is a Reaction that costs 1 Action.

- \emph{Stone Giant}. Stone Skin. When you take damage you can harden your skin to the point of stone. Reduce damage taken by (AC or CM + Constitution)/2. Cost 1 Reaction.

- \emph{Storm Giant}. Thunder Resonance. When you take melee damage you can send out a shockwave that deals 1d10 Sound damage to the person who caused you damage. Cost 1 Reaction.

\subsection{Sulian}\index{Sulian}\label{sulian}\hypertarget{sulian}{}

The origin of the Sulians is not clear, some say they descend from elemental spirits, other less insistent rumors say that they are the children of \hyperlink{ledyal}{Ledyal or Laydel} (page \pageref{ledyal}) due of their changing appearance and character.

The power, energy and vitality of the elements flows in the Sulians, whether it be a single type or multiple elements.

The Sulians are very similar to humans but the primordial energy that characterizes them can be seen flowing in their eyes and often on their skin.

\textbf{Racial modifiers:} +1 to one Characteristic of your choice

\textbf{Physical characteristics}: approximately 150-190cm tall. Life expectancy approximately 180 years (160+2d10)

\textbf{Dimensions}: medium size

\textbf{Speed}: 9 metres

\textbf{Languages}: Common. They can understand the elemental language of their bloodline but cannot speak it.

\textbf{Special}: Each Sulian descends from one or more elemental lines and from this they inherit unique powers and abilities. At the first point of AC or CM assigned and then every eighth point assigned in total (CM+AC=1,8,16...), the Sulian upgrades his elemental bloodline and selects a power or unlocks another elemental line present in him to choose different powers.

The indicated power is usable (CM+CA)/3 per day.

- \emph{Primordial Discharge}: the Sulian can discharge part of his elemental energy at the cost of 1 Reaction when he is hit or hits in melee. The damage is equal to 2d6 times this power is selected.

- \emph{Access to the Magic List}: through this power the Sulian can access an Elemental List. Each time he takes this power he spontaneously knows up to 3 spells on that list with a maximum spell level equal to the times he took this power on the same list -1 (the first time he only casts cantrips). 
The Sulian does not make Magic Tests nor can he be considered distracted when he casts the spell. For possible factors, the CM is considered to be equal to the sum of CM+AC and Adept of Magic has been taken a number equal to the times this power has been taken.

- \emph{Elemental Resistance}: through this power the Sulian acquires Resistance to the chosen element.



\begin{center}
\includegraphics[height=0.75\linewidth]{immagini/Undine_Rising_from_the_Waters.png}\\

\emph{Undine Rising from the Waters, approx. 1880–1892, by Chauncey Bradley Ives (1810–1894), in the Yale University Art Gallery}

\end{center}

\end{multicols}

\index{Breeds}\index{Breeds}
\begin{changemargin}{0.3cm}{0.3cm}\begin{tcolorbox}[title = Note on Benefits]
The player, in agreement with the Storyteller, can choose an advantage or power different from the one indicated as long as it is consistent with the character's story.
\end{tcolorbox}\end{changemargin}

\begin{changemargin}{0.3cm}{0.3cm}\begin{tcolorbox}[title = Note on Breeds]
No description of a race will ever be able to harness and subdue a character. Each player is free to create the character of their favorite race (granted by the Narrator) and describe it, frame it, feel it, make it come alive as they like.

Don't limit yourself to the descriptions proposed here, they are meant to be just ideas, don't feel limited in your choices because your race says this or that.
Create the most beautiful and complete characters possible.

Each character is alive and is a person and as such will always be different from each other, each fantastic in a different way despite any race and prejudice.
\end{tcolorbox}\end{changemargin}

\begin{changemargin}{0.3cm}{0.3cm}\begin{tcolorbox}[title = Note on Sex]\index{Sex}
In case you were so obtuse, I reiterate that there is no difference in abilities or characteristics based on sex. Each player is invited to play the character of the sex (or not) that she prefers.

If the topic is not fun for you, clarify it with the Narrator, he will be able to orchestrate the adventure in a suitable way.
\end{tcolorbox}\end{changemargin}

\pagebreak

\section{Special Features}


\begin{changemargin}{0.3cm}{0.3cm}\begin{enfasi}{Having eyes is not enough to see (anonymous)}\end{enfasi}\end{changemargin}\medskip

\begin{multicols}{2}


\lettrine[lines=2, lhang=0.33, loversize=0.25, findent=1.5em]{O}{every} creature is special and unique and yet there are beings even more unique and special due to their characteristics. These are the peculiarities of some of these.

\subsection{Light Vision}\index{Light Vision}\label{visionecrepuscolare}

What for many is darkness, for those who have \hypertarget{light vision}{twilight vision} is seeing well as long as there is a minimum source of light.

Low-light vision is color vision.
A spellcaster with low-light vision can read a scroll as long as he has even the dullest candle nearby as a source of light.

Characters with low-light vision can see outward on moonlit nights as if they were in daylight.
In the absolute lack of light, twilight vision does not help, it remains impenetrable pitch darkness.

\subsection{Darkvision}\index{Darkvision}\label{scurovisione}

Darkvision is the extraordinary ability to see without light sources, up to a maximum distance indicated for each creature.

Darkvision is black and white only (it does not allow the creature to distinguish colors). It doesn't allow characters to see anything they couldn't otherwise see: Invisible objects are still Invisible, and Illusions are still visible as they appear to be.

Likewise, darkvision makes a creature susceptible to gaze attacks normally. The presence of light does not alter darkvision.
Making a Survival check to look for traps or a visual-only Awareness check takes a 1d6 penalty.

\subsection{Nose}\index{Nose}\label{fiuto}

This special quality allows a creature to use its sense of smell to detect hidden or approaching enemies and to follow their tracks. Creatures with a sense of smell can identify familiar odors by smell as humans do by sight.

The creature can detect creatures within 20 feet by smell. If the opponent is downwind, the range increases to 18 meters; if it is upwind, the radius decreases to a distance of 3 meters.
Stronger odors, such as smoke, garbage or decaying bodies, can be detected at double the above range.

When a creature detects a scent, the exact location of its source is not revealed, only its presence within range. The creature can use an Action to locate the direction the smell is coming from. When within melee range of the source, it pinpoints its location.

A creature with scent can track using smell, making a Track check to find and follow a trail. The typical DC of a fresh track is 10 (regardless of the surface the track is on). The DC increases or decreases depending on the intensity of the track, the number of creatures leaving it, and the time that has passed since it was left. For every hour that passes the DC increases by 2.

\begin{center}
\includegraphics[width=0.9\linewidth]{immagini/mostro.png}

\emph{John D. Batten}
\end{center}

Otherwise, this ability follows the rules of the Survival skill. Creatures that track by scent ignore the effects of tracking surfaces and poor visibility.

Water, and especially running water, denies creatures' ability to track.

Some strong odors can easily mask others. The presence of such an odor makes it impossible to precisely locate or identify a creature by scent; the base DC of the Survival skill for following tracks in the presence of covering odors changes from 10 to 20.

\subsection{Blind View}\index{Blind View}\label{vistacieca}

Using a variety of senses other than sight, such as the perception of vibrations, a sensitive sense of smell, keen hearing, or sonar, a creature with blindsight moves and fights as well as a creature with sight.

Invisibility and darkness are irrelevant, although the creature with blindsight must have line of effect to notice a given creature or object.

A creature with cover still has its Defense advantage.

The range of the ability is indicated in the creature's description. The creature generally does not need to make Awareness checks to notice creatures within range of its blindsight.

Unless otherwise indicated, blindsight is always active and the creature must take no action to activate it. Some forms of blindsight must be activated as an immediate action. In this case, it is indicated in the creature's description.

An ethereal creature is not visible to blindsight.


\begin{center}
\includegraphics[height=0.6\linewidth]{immagini/grabroid.png}

\medskip

\emph{Grabroid. Also known as Grabbers. Tremors (Film)}
\end{center}

\subsection{Telluric Sense}\index{Telluric Sense}\label{sensotellurico}
A creature with Tell Sense is sensitive to vibrations in the ground and can automatically detect anything in contact with the ground within the radius specified by the Tell Sense.

Aquatic Creatures with Telluric Sense (echolocation) can sense the location of creatures in contact with water.

The range of the ability is specified in the creature's descriptive text.

\end{multicols}

\vfill

\begin{center}
\includegraphics[width=0.8\linewidth]{immagini/argus2.png}

\emph{Argus Panoptes Guarding the Heifer (Io), Red Figure pitcher, c. 460 BC Museum of Fine Arts, Boston}
\end{center}

\pagebreak

\section{The Features}\index{Features}


\begin{changemargin}{0.3cm}{0.3cm}\begin{enfasi}{Living is not breathing: it is acting, it is making use of the organs, the senses, the faculties, of all those parts of ourselves through which we have the feeling of existing. (Jean-Jacques Rousseau)}\end{enfasi}\end{changemargin}\medskip

\begin{multicols}{2}

\lettrine[lines=2, lhang=0.33, loversize=0.25, findent=1.5em]{Each} character has 6 Characteristics (also called Statistics) which represent his basic attributes and constitute his potential talent and ability innate.

While it is not common for a character to make a check using only one of their Ability, Ability scores affect virtually every aspect of the character's abilities and proficiencies.


\subsection{Description of Features}\label{decrizionedellecaratteristiche}

The Characteristics score is not everything in a character, much less in a monster.

The more \emph{instinctive} and aggressive monsters will probably have negative Intelligence scores, but that doesn't make them \emph{stupid}, they simply act according to their natural patterns. At the same time, creatures with low Constitution values ​​are not about to die but are only \emph{fragile}.

\subsubsection{Strength}\index{Strength}\label{forza}

\begin{changemargin}{0.3cm}{0.3cm}\begin{enfasi}{
Ah, it is excellent to possess the strength of a giant, but to use it like a giant is tyranny! (William Shakespeare, Isabella: from “Measure for Measure”, act II, scene II)
}\end{enfasi}\end{changemargin}

Strength measures physical power, athleticism, and the limits of brute force you can express. Strength applies to melee damage and hand-fired weapons.

A Strength check can be used for any attempt to lift, push, pull, or break something, to push your body into a space, or any other application of brute force.

A monster with Strength -4 is not close to dying, it simply has very little strength (imagine giving a Strength value to a mouse or a squirrel if not to a small spider..)

A character with a Strength score of -5 is dead.


\subsubsection{Dexterity}\index{Dexterity}\label{destrezza}

\begin{changemargin}{0.3cm}{0.3cm}\begin{enfasi}{
Tired barking. Strength means nothing in life. Knowing how to dodge is what matters. (Daniel Pennac)
}\end{enfasi}\end{changemargin}

Dexterity measures agility, reflexes, balance and coordination; determines Defense and attack rolls with thrown weapons.

A Dexterity check can be used for any attempt to move nimbly, to avoid losing your balance or pickpocketing.

A character with a Dexterity score of -5 is unable to move and is completely immobile (but not unconscious).

\subsubsection{Constitution}\index{Constitution}\label{costituzione}

\begin{changemargin}{0.3cm}{0.3cm}\begin{enfasi}{
A little health every now and then is the best remedy for the sick. (Friedrich Nietzsche)
}\end{enfasi}\end{changemargin}

The Constitution measures health, vigor and vital force as well as resistance to effort.

A Constitution check can be used for attempts to push you beyond your body's normal limits and for tests of endurance and endurance.

A character with Constitution -5 no longer has control of his body and is dead.

\subsubsection{Intelligence}\index{Intelligence}\label{intelligenza}

\begin{changemargin}{0.3cm}{0.3cm}\begin{enfasi}{
Strength without intelligence collapses under its own weight. (Horace)
}\end{enfasi}\end{changemargin}

Intelligence measures mental acumen, the accuracy of memories, and the ability to reason.
An Intelligence check comes into play when you need to rely on logic, education, memory, or deductive skills.

Your Intelligence checks measure your ability to remember information about spells, magical items, esoteric symbols, magical lore, the planes of existence, and the inhabitants of those planes. Rummaging through ancient scrolls in search of a fragment of knowledge may require an Intelligence check.

A character with an Intelligence score of -5 is comatose.

\subsubsection{Wisdom}\index{Wisdom}\label{saggezza}

\begin{changemargin}{0.3cm}{0.3cm}\begin{enfasi}{
Strength does not come from physical ability. It comes from an indomitable will. (Mahatma Gandhi)}\end{enfasi}\end{changemargin}

Wisdom reflects your attunement to the world around you and represents insight, intuition, willpower and common sense.

A Wisdom check reflects an effort to interpret body language, understand someone's feelings, notice details of the environment, or heal an injured person.

A character with a Wisdom score of -5 is incapable of rational thought and is unconscious.

\subsubsection{Charisma}\index{Charisma}\label{carisma}

\begin{changemargin}{0.3cm}{0.3cm}\begin{enfasi}{
Kogami, do you know what charisma is?

– As I see it, it is an innate attitude, like that of a hero or a leader.

– [...] There are three elements that identify charisma: the innate nature of heroes and prophets, the ability to instill well-being in others with just their presence and a culture that allows you to have a brilliant conversation on any topic. (Psycho Pass)
}\end{enfasi}\end{changemargin}

Charisma measures your ability to interact effectively with others. It includes factors such as confidence and eloquence, it can represent a charming or authoritative personality.

A Charisma check may be called for when you try to influence or entertain other people, when you try to make an impression or tell a lie, or when you have to navigate a complicated social situation.

\begin{center}
\includegraphics[width=0.55\linewidth]{immagini/dice4.png}
\end{center}

Typical situations in which Charisma is used include attempts to fool a guard, defraud a merchant, earn money by gambling, pass off as someone else through a disguise, allay someone's suspicions with false reassurances, or maintain an imperturbable face while tells a blatant lie.

A character with a Charisma score of -5 is unconscious.

\subsubsection{Read the Feature scores}\index{Read the Feature scores}\label{leggereipunteggidellecaratteristiche}

Each Ability score typically ranges from 0 to 3, a good Ability score is 1, 2 is excellent, 0 is "normal", 3 is judged \emph{exceptional}.

A score of -1 is considered weak, a -2 is subnormal, a -3 is severely problematic, a -4 almost leads to non-use of the characteristic, a -5 is appropriate for him to just stay in bed (if he is not already in a coffin).

\subsubsection{Optional - Age of the character}\index{Optional - Age of the character}\hypertarget{age of the character}{} \label{etadelpersonaggio}

The character's age affects physical and mental characteristics.

\begin{tabular}{llllll}
Period & STR & DEX & COS & INT & WIS\\
\hline
Young & & & +1 & & -1 \\
\hline
Adult & & & -1 & & +1\\
\hline
Mature & & & -1 & & +1 \\
\hline
Elder & -2 & -1 & -1 & +1 & +1 \\
\hline
Venerable &-1 & -1 & -1 & -1 & +1 \\
\end{tabular}

\medskip

The indicated modifiers are cumulative.

\subsection{Feature scores} \hypertarget{assignment.feature.scores}{}\label{assegnazionepunteggicaratteristica}

Ability scores play an important but not critical role. The player must understand that a low score does not mean having a bad character, but rather he will have more fun playing him by leveraging his specific Feats, abilities and abilities, using ingenuity and wit. Multiple systems for pulling features are presented.

Personally I suggest the \textbf{Basic Mode} approach. In OBSS the characters are not heroes, they are not the chosen ones who stand up as defenders of the planet. The characters are normal people often involved in situations that are borderline if not beyond survival.

The undoubted advantage of putting the values ​​in order of the characteristics is that it allows you to mix up the schemes and avoid \emph{builds} done on the table.

It is likely that the results you hoped for will not be achieved or even features that you were not interested in will be achieved. That's fine. Change your mind, be inspired by the values ​​obtained! Have fun with the new character, build something new and different, let yourself be amazed.

The \textbf{Ability rolls are made in order}, so the first roll is for Strength, then Dexterity, Constitution, Intelligence, Wisdom and finally Charisma.

Finally, remember that OBSS is a role-playing game where character death happens, even more often than in other role-playing games. Create good, concrete characters and let the adventure shape the details.

\textbf{\emph{Racial modifiers cannot raise or lower scores beyond +4/-4}}.

\subsubsection{Basic mode}\index{Features - Basic mode}\label{modalitabase}

The player rolls 3d6 for each characteristic and in order, he can only reroll a 1 per triplet (3d6) once. He then rolls a seventh triplet which he can replace with another triplet. For each characteristic rolled, check the sum of the dice rolled with the \textbf{Table: Characteristics Roll}.

\subsubsection{Traditional Mode}\index{Features - Traditional Mode}\label{modalitadellatradizione}

Each player rolls 4d6 6 times and adds the best 3 results each time. The result obtained is checked with the \textbf{Table: Characteristics Roll} and assigned in order.

\subsubsection{Optional mode (for cowards!)}\index{Features - Optional mode for cowards}\label{modalitapericodardi}

Each player distributes 5 points among the 6 Characteristics, each Characteristic must have a minimum score of -1 and a maximum of 2 before racial modifiers.

\medskip

\begin{changemargin}{0.3cm}{0.3cm}\begin{tcolorbox}[title = Let's take a look at the features of Tups]

\textbf{First triplet}: \cancel{1},1,4,3 total 8. Strength is -1

\textbf{Second}: 5,6,6 total 17. Dexterity is +2

\textbf{Third}: \cancel{1},2,1,4 total 7. Constitution is -1

\textbf{Fourth}: 6,6,6 total 18. Intelligence is +3

\textbf{Fifth}: 3,4,2 total 9. Wisdom is +0

\textbf{Sixth}: 3,4,4 total 11. Charisma is +0

\textbf{Seventh}: 3,5,2 total 10. Which replaces Strength (from -1 to +0)

As a Human \hyperlink{different}{Different}, Tups gets +1 Constitution and +1 Intelligence

\end{tcolorbox}\end{changemargin}

\subsubsection{Table: Characteristics Roll}\index[Tables]{Characteristics Roll Table}

The sum of the dice rolled for the Characteristics should be compared to this table to determine the actual Characteristic values.\\

\begin{tabularx}{0.45\textwidth}{lX|lX}
\textbf{Val. pulled}& \textbf{Features}&\textbf{Val. pulled}& \textbf{Features}\\
\toprule
3 (or less)&-3&13-14-15&+1\\
4-5&-2&16-17&+2\\
6-7-8&-1&18 (or more)&+3\\
9-10-11-12&+0&&\\
\end{tabularx}

\medskip

\textbf{Remember to apply racial modifiers!}


\subsection{Increase Features}\label{aumentarelecaratteristiche}\hypertarget{increase Features}{}

Through the \hyperlink{supreme}{Supreme} Feat (page \pageref{supreme}) you can increase a Characteristic by one point, up to a maximum value of 4 + the racial bonus, or penalty, of the characteristic .

The Ultimate Feat can be taken whenever the sum of CM+AC is at least 4 points higher than the previous time it was taken.

To increase beyond this value requires magical items or spells. The increase in Characteristics has a retroactive effect only for increases in Constitution, affecting the maximum Hit Points.

The increase in Characteristics immediately applies the modifier to Saving Throws, Attack Rolls and Initiative, the increase in Intelligence has repercussions in the next level in the number of Feats acquired.

\begin{center}
\includegraphics[width=0.8\linewidth]{immagini/guerrieroispirato.png}

\emph{Brian Boru, High King of Ireland}
\end{center}

\end{multicols}

\vfill

\begin{changemargin}{0.3cm}{0.3cm}\begin{narratore}
Players will still complain about the Characteristics rolls, it's normal, especially the more inexperienced players. Try to make him understand that he shouldn't just look at the Characteristics but see the general whole of the character. Suggest Feats that can help him overcome his characteristics.
\end{narratore}\end{changemargin}

\medskip

\begin{changemargin}{0.3cm}{0.3cm}\begin{tcolorbox}[title = The Character sucks! - Low Features]\index{Low Features}
Having low Characteristics is not the death of the character! Instead, try to play so that it is not necessary to roll dice or take tests! Strive to be witty, intuitive, proactive, clever... in short, everything that can help you resolve the situation without necessarily having to roll dice. In OBSS the Storyteller rewards players who describe and get excited about what the character does!
\end{tcolorbox}\end{changemargin}


\pagebreak

\begin{multicols}{2}

\section{Hit Points}\index{Hit Points}\index{Hit Points}\index{HP}

\begin{changemargin}{0.3cm}{0.3cm}\begin{enfasi}{Whoever doesn't value life doesn't deserve it. (Leonardo da Vinci)}\end{enfasi}\end{changemargin}


\lettrine[lines=2, lhang=0.33, loversize=0.25, findent=1.5em]{I}{ Hit Points} represent the character's vital energy but also the character's skill, luck, ability to resist and fight. As long as the character/opponent has at least 1 Hit Point (HP) he will fight and fight to the best of his ability.

- Each character starts with 4 Hit Points at first level + the Constitution score.

- At each level beyond the first, he gains 1d4 Hit Points + his Constitution score.

Each point taken in Weapon Proficiency increases the Hit Points taken by 3. Additional Feats can increase the Hit Points.

Mark the maximum Hit Points you have on the sheet and indicate the current value each time you lose or recover them. Always note on the sheet what the current amount of hit points is, after each hit or damage. Maximum Hit Points are the amount of Hit Points when the character is \emph{perfectly healthy}.\\

\begin{changemargin}{0.3cm}{0.3cm}\begin{tcolorbox}[title = I'm about to die!]\index{I'm about to die!}
ESCAPE! Retreat, hide, exit the fight. There is no glory in being dead. Better a retreat than a TPK (Total Party Kill or death of the entire group).
\end{tcolorbox}\end{changemargin}

\medskip

The \textbf{Hit Points are recovered} in several ways:\index{Hit Points Recovery}

- for each night of rest (at least 8 hours) you recover in Hit Points the value of Constitution + 2*AC + CM (with a minimum of 1)\index{HP recovery while sleeping}

- through healing magic (spells, potions... or other magical effects)

- competence \hyperlink{first aid}{First Aid} (page \pageref{first aid}), through more or less long treatments

Hit Points can also be \textbf{temporary}\index{Temporary Hit Points} or added or removed temporarily from your current ones.

A spell that grants +10 temporary hit points will raise your current hit points by 10, if you take 8 damage you will have 2 temporary hit points left. If you instead suffer 13 damage, in addition to losing all your temporary hit points, you will also suffer 3 hit points \emph{normal}.

- When you gain temporary Hit Points you must choose whether the effect replaces the previous one. Temporary Hit Points do not stack and cannot exceed half your maximum Hit Points. A healing spell restores normal HP, not lost temporary HP.

- At the end of the effect that grants temporary hit points, they disappear, leaving the creature with its previous hit points.

- Unless otherwise stated, Temporary Hit Points disappear one hour after they were added.

- Temporary Hit Points are removed first when you are wounded.

A weapon or effect that causes nonlethal damage means it causes \hyperlink{nonlethal wound recovery}{temporary wounds}\label{feritetemporanee}.

\section{Fate Points}\index{Fate Points}\index{Beginner's Luck}

\begin{changemargin}{0.3cm}{0.3cm}\begin{enfasi}{If fate is against us, so much the worse for him. (motto of the 1st Carabinieri Parachute Regiment "Tuscania")}\end{enfasi}\end{changemargin}

\lettrine[lines=2, lhang=0.33, loversize=0.25, findent=1.5em]{I}{n} a not easy world Beginner's Luck helps those who have no experience.
Each character has a number of Fate Points equal to (20 - Level)/5, with a minimum of 1. Fate Points are counted per game session. You recover a Fate Point every time you roll at least three 1s in a test.\index{Recover Fate Points}

At each session they are reset and recalculated, it follows that Fate Points are not accumulated between one game session and another. E.g. a level 6 character has: 20-6 = 14/5 = 3 (round to the nearest integer) Fate Points to use in the session.

A Fate Point is used as an Immediate or Reaction Action and the character can use Fate Points to:

\smallskip

- add 1d6 to a Saving Throw, Attack Roll, or Expertise Check. To be declared before rolling the dice. The added die can explode according to the Golden Rules. 1 or more Fate Point

\smallskip{}

- reroll 1d6 (which maybe made 1...). 2 Fate Point

\smallskip

- recover 3 Hit Points even if they are negative. 1 Fate Point

\smallskip

- negate a weapon critical roll immediately. 1 Fate Point\\

Using all the Fate Points available it is possible to retract a test, accepting only the new result obtained. Does not use Actions.

\subsection*{Optional - Chaos Points}\index{Optional - Chaos Points}

One way to add tension is to manage a set of Fate Points shared between characters and opponents instead of those of the individual player. A container, a small bowl, is placed in the center of the table with a number of d6s inside equal to the number of characters.

Each player is free to take one die at a time and use it as if they were Fate Points.

The chaos is given by the fact that these dice are then moved to another container that the Narrator, always maximum one at a time per opponent, will use at \emph{his benefit}. Once the Storyteller has used the die he puts it back in the players' container.

\end{multicols}

\pagebreak

\section{The Traits}\index{Traits}\hypertarget{Traits}{}\label{tratti}

\begin{changemargin}{0.3cm}{0.3cm}\begin{enfasi}{Whoever therefore knows how to do good and does not do it, commits a sin. (James the Just 4.17, Epistle of James)
\smallskip

It is a natural right to satiate one's soul with revenge. (Attila)
\smallskip

East Sularus Oth Mithas. ("My honor is my life", Oath of the Knights of Solamnia)}\end{enfasi}\end{changemargin}\medskip

\begin{multicols}{2}

\index{Trades}
\lettrine[lines=2, lhang=0.33, loversize=0.25, findent=1.5em]{I}{n} OBSS there is no clear distinction between good and evil, law and chaos, between what is right and what's wrong.

In OBSS there are Traits, aspects and character nuances that contribute to the character's background, help the player to role play better and can provide him with guidelines to more correctly interpret the character he wanted to create.

A Trait is a detail that helps better frame the character, outlines the main characters, granting them different nuances.

\textbf{Each player chooses 5 Traits for their character at character creation.} These will be the \emph{moral, ethical and behavioral compasses} that will guide the character in acting and making choices.

\smallskip

\begin{changemargin}{0.3cm}{0.3cm}\begin{tcolorbox}[title = Choose Traits] %box player
The Traits are not the character, they do not block him or fix him eternally in time. A character is always constantly evolving and so is his character, morals, behavior and desires. Don't be rigid but use the Traits to give you suggestions to inspire you.
\end{tcolorbox}\end{changemargin}

\smallskip

\textbf{At first level choose a Trait most characteristic for the character, this will have a value of 1, the other 4 Traits will have a value of 0.}

As time and adventures pass, the Traits will increase in value or can be replaced, in agreement between the Narrator and the player based on how played, by other Traits. \textbf{The higher the Trait value, the more present and permeating it is in the character's choices}.

During the adventures, the Narrator, following particular scenes and acting, will be able to increase a character trait by one point, or a fraction of a point.

For example, following a particular situation and adventure climax, the Narrator could grant everyone or someone the Courageous Trait or give a +1 to Courageous to those who already have this Trait. For Traits not taken, the base value in points of -1 is considered, i.e. the first point is used to take the Trait and the following ones to emphasize them.

While it is \emph{relatively} easy to acquire new Traits it is extremely difficult to change existing ones. Talk about it with the Narrator, he will be able to prepare situations and adventures that will help you understand how to evolve the character and possibly change the chosen Traits.

In the form you will find \textbf{check} to be placed next to the Traits, these are marked following actions suitable for increasing the value of the Trait; once 10 checks have been reached, the Trait will increase by 1 point and a new ten will begin to be scored again.

During the adventure, the Narrator will tell you when to score, or cancel, partial points. \textbf{In principle it is assumed that a character acquires at least one Trait point per level.}

Each particularly important action where the character has followed a Trait brings the character closer to the \textbf{Patron} (or Patrons) competent for that Trait.

As the value of the sum of Traits in common with the Patron increases, the character will be able to acquire powers, regardless of whether he is a believer (Follower or Devotee) or not of that Patron.

- At \textbf{5} points you can start to feel the presence of a Patron

- At \textbf{10} points you can feel the closeness of a Patron

- At \textbf{15} points you are tied to Patrono

- At \textbf{20} points you are a Patron champion


It is not necessary to believe in a Patron to feel close to him, to be tied to him or his champion, it is simply one's nature (one's Traits) that is similar to the Patron, whether one wants it or not.

Since the goal of a Patron is to have his Traits dominate over others, having people of high status and power who are so akin to him will come in handy in the 1000 year judgment. Use the Traits and the bond that the Patron will establish with you to your advantage.

To identify the most similar Patron, check your highest value Trait on the \hyperlink{patrontract link table}{Patron Link - Trait Table} (page \pageref{patronotract link table}) and identify the Patron with whom you share it, if the Trait was shared between multiple Patrons, check the other Traits and choose the Patron based on the similarity. Then check in \hyperlink{cosmologia}{Cosmologia} (page \pageref{patrons}) the powers granted by the Patron. This check should be done every time the Trait value increases.

The Narrator is free to insert new Traits as he wishes or requested by the players, it is suggested to attribute these new Traits also to the Patrons.

\end{multicols}

\textbf{Trait Table}\index[Tables]{Trait Table}

\begin{multicols}{6}
{\small
\begin{flushleft}
Habitual\\
Accumulator\\
Easy going\\
Loving\\
Reliable\\
Aggressive\\
Cheerful\\
Haughty\\
Unselfish\\
Ambitious\\
Friendly\\
Anarchist\\
Anxious\\
Nonconformist\\
Unfriendly\\
Apathetic\\
Angry\\
Arrogant\\
Careful\\
Bold\\
Austere\\
Reckless\\
Greedy\\
Belligerent\\
Borious\\
Brutal\\
Jester\\
Liar\\
Good\\
Grumpy\\
Joker\\
Computer\\
Calm\\
White\\
Chaotic\\
Charitable\\
Casino player\\
Chaste\\
Bad\\
Cynical\\
Clement\\
Stubborn\\
Coward\\
Combative\\
Compassionate\\
Competitive\\
Comprehensive\\
Conformist\\
Confusing\\
Checked\\
Brave\\
Correct\\
Corrupt\\
Courteous\\
Creative\\
Gullible\\
Dark\\
Curious\\
Weak\\
Decided\\
Determined\\
Devoted\\
Distrustful\\
Disciplined\\
Dishonest\\
Dishonorable\\
Messy\\
Detached\\
Destructive\\
Docile\\
Double agent\\
Educated\\
Selfish\\
Emotional\\
Empath\\
Enthusiastic\\
Expansive\\
Outgoing\\
Exuberant\\
False\\
Imaginative\\
Fatalist\\
Cold\\
Furious\\
Gallant\\
Jealous\\
Generous\\
Joyful\\
Right\\
Idealist\\
Immature\\
Immoral\\
Awkward\\
Impartial\\
Impassive\\
Impatient\\
Impetuous\\
Implacable\\
Imprudent\\
Uncertain\\
Unsatisfiable\\
Inconstant\\
Indifferent\\
Undisciplined\\
Indomitable\\
Indulgent\\
Industrious\\
Childish\\
Misleading\\
Naive\\
Insensitive\\
Insolent\\
Integer\\
Intolerant\\
Enterprising\\
Introvert\\
Jealous\\
Hypocritical\\
Angry\\
Ironic\\
Unreasonable\\
Irritable\\
Instinctive\\
Complaining\\
Fair\\
Legal\\
Lethargic\\
Liberal\\
Licentious\\
Quarrelsome\\
Talkative\\
Moody\\
Lustful\\
Rude\\
Melancholy\\
Mischievous\\
Evil\\
Martyr\\
Masochist\\
Maternal\\
Wacky\\
Petty\\
Meticulous\\
Merciful\\
Measured\\
Mild\\
Moderate\\
Modest\\
Worldly\\
Moderate\\
Narcissist\\
Negligent\\
Nervous\\
Neutral\\
Non-violent\\
Shady\\
Honorable\\
Tidy\\
Observer\\
Hostile\\
Stubborn\\
Optimistic\\
Pacific\\
Paranoid\\
Passionate\\
Bungling\\
Scary\\
Patient\\
Perfectionist\\
Prickly\\
Crybaby\\
Planner\\
Pompous\\
Pragmatic\\
Caring\\
Bossy\\
Conceited\\
Foresight\\
Protective\\
Provocateur\\
Prudent\\
Angry\\
Rational\\
Reactionary\\
Rebel\\
Thoughtful\\
Rigid\\
Relaxed\\
Reserved\\
Respectful\\
Resolute\\
Know-it-all\\
Sadistic\\
Sadomasochist\\
Sarcastic\\
Skeptical\\
Joking\\
Outspoken\\
Evasive\\
Reckless\\
Grumpy\\
Grumpy\\
Simple\\
Sensitive\\
Serious\\
Unrestrained\\
Safe\\
Silent\\
Sincere\\
Unfair\\
Snob\\
Sober\\
Sociable\\
Dreamer\\
Solitary\\
Suspicious\\
Carefree\\
Ruthless\\
Spontaneous\\
Naive\\
Stoic\\
Extravagant\\
Superb\\
Superficial\\
Susceptible\\
Tenacious\\
Shy\\
Traitor\\
Traditionalist\\
Calm\\
Sad\\
Crook\\
Humble\\
Vain\\
Valiant\\
Vengeful\\
Violent\\
Fickle\\

\end{flushleft}}
\end{multicols}

%evaluate motivations, a motivation table

If a player does not role the character's Traits he will not have the character acquire experience points.

\vfill

\begin{center}
\includegraphics[height=0.35\linewidth]{immagini/troll.png}
\end{center}

\begin{changemargin}{0.3cm}{0.3cm}\begin{enfasi}{If a traveler doesn't bring back something to share, he's not a \textit{Hero}  but an imposter, an egotist without wisdom. (The Hero's Journey, Christopher Vogler)}\end{enfasi}\end{changemargin}

\pagebreak

\section{Skills}\index{Skills}

\begin{changemargin}{0.3cm}{0.3cm}\begin{enfasi}{
Whoever says that something is impossible should not disturb whoever is doing it. (Albert Einstein)

\medskip
You haven't truly understood something until you can explain it to your grandmother. (Albert Einstein)}\end{enfasi}\end{changemargin}\medskip

\begin{multicols}{2}

\lettrine[lines=2, lhang=0.33, loversize=0.25, findent=1.5em]{L}{e} Skills represent what you know and what you know how to do. Their scores represent how well the skill is known and therefore the higher the value, the more expert one is.

\subsection{Basic Skills}\index{Basic Skills}\label{competenzebase}

\begin{changemargin}{0.3cm}{0.3cm}\begin{enfasi}{ 
Studying is for losers! (Lobo) }\end{enfasi}
\end{changemargin}

Each character has an initial profession, a life and work path that led him to learn certain skills or what he did (and continues to do if he wants) before engaging in dangerous adventures.

Some Professions and their related skills are listed, the character acquires these skills with the score indicated in the table.

The initial profession and the skills acquired must be marked on the sheet. In agreement with the Narrator it is possible to select different skills and also choose different professions!

\end{multicols}

\textbf{Table: List of Professions and related Skills}\index[Tables]{Table List of Professions and related Skills}\index{Professions}

\medskip

\begin{tabularx}{0.95\textwidth}{llllll}
\textbf{\textbf{Profession}}& \textbf{1 point} & \textbf{2 points} & \textbf{2 points} & \textbf{3 points}\\
\toprule
\textbf{Acolyte}& Occult& History or Geography& Arcana& Religion\\
\textbf{Alchemist}& Evaluate&Nature& Herbalism& Arcana\\
\textbf{Breeder}& Survival&Tracking& Managing animals&Nature \\
\textbf{Apprentice magician}& History and Geography&Occult&Myths and Legends&Arcana\\
\textbf{Gets it right}& Evaluate&Deceive&Perceive Emotions&Diplomacy\\
\textbf{Librarian}& Nature and Geography&Local traditions&Religion and Arcana&History\\
\textbf{Woodcutter}& Using rope&Nature&Orientation&Survival\\
\textbf{Hunter}& Stealth&Tracking&Survival& Nature\\
\textbf{Caravaners}& History or Geography&Evaluating&Riding&Orientation\\
\textbf{Theatrical}& Perceiving emotions& Languages&Entertaining&Acrobatics\\
\textbf{Herbalist}& Myths&Geography&Nature&Herbalism\\
\textbf{Card Player}& Perceive Emotions&Evaluate&Entertain&Deceive\\
\textbf{Guard}& Sense Emotions&Knowledge Law&Ride&Intimidate\\
\textbf{Guide}& Myths&Dungeons&Nature&Geography\\
\textbf{Pickpocket} & Disable Devices&Escape Artist&Stealth&Fairy Hands\\
\textbf{Thug}& Survival&Ride&Evaluate&Stealth\\
\textbf{Innkeeper}& First aid&Evaluating&Perceiving Emotions&Diplomacy\\
\textbf{Merchant}& Languages&Local Traditions&Evaluate&Deceive\\
\textbf{Miner}& Using ropes&Evaluating&Orientation&Dungeon\\
\textbf{Fisher}& Orientation&Swimming&Using ropes&Nature\\
\textbf{Soldier}& Swimming&Animal handling&Athletics&Riding\\
\textbf{Cart driver}& Local Traditions & Orientation & Handling Animals & Riding\\
\textbf{Medicine man}& Myths&Nature&Herbal medicine&First aid\\
\textbf{Forest ranger}& Myths&Herbal medicine&Riding & Nature\\
\textbf{Farmer}& Survival &Herbalism& Managing Animals & Nature\\
\end{tabularx}

\bigskip

\begin{multicols}{2}

A profession is not expressed in just 4 skills but these are the ones that will most come into use during the adventures, the Narrator will be helped by your profession to understand how your character can resolve situations and how he will interact with other characters.

Below is the \textbf{Skills list table} from which to choose for any new professions or customizations of the same.

\begin{changemargin}{0.3cm}{0.3cm}\begin{enfasi}{
Although undoubtedly the desire to know is natural for all men, the desire to learn is not something for everyone... (Richard de Bury)
}\end{enfasi}\end{changemargin} \medskip

\subsubsection{Customize Skills and Profession}\index{Customize Skills and Profession}

For each new profession that you create, 4 skills taken from this list will be associated, one skill will start with a score of 1, two skills will start with a score of 2 and the most specific and professional one will start with a score of 3.

In agreement with the Narrator it is also possible to change the order of the Skills for the Professions already listed, making the character more capable in some skills rather than others.

\subsubsection{Character's Skills and Background}\label{quintacompetenza}\index{Character's Skills and Background}

When creating the character, the player can decide to take a +1 to an already known skill or take a new skill, linked to the character's history, with a score of 1.\\

The player \textbf{increases the score of a Characteristic that is linked to the Profession or background by 1} up to the maximum value of 4+racial modifications. It could be Intelligence for an Apprentice Wizard, but if this one is a bodybuilder as a hobby it could also be Strength.

\begin{changemargin}{0.3cm}{0.3cm}\begin{tcolorbox}[title = Profession ???]
Don't underestimate the choice of profession! Not everything can be solved with axes or magic. Knowing how to untangle knots, follow tracks, recognize herbs or diseases make the character an expert, create a profession. You must not define the character only based on the skills he has but based on what and how well he knows how to do it. A low level but experienced survival character will always be more useful than an expert fighter when it comes to crossing a desert.\end{tcolorbox}\end{changemargin}

\end{multicols}

\medskip

\textbf{Table: List of skills and related usage feature}\index[Tables]{Table List of skills and related usage feature}

\medskip

\begin{tabular}{llllll}
\textbf{Strength} & \textbf{Dexterity} & \textbf{Intelligence} & \textbf{Wisdom} & \textbf{Charisma}\\
\toprule
Climbing & Acrobatics & Arcana & Riding & Diplomacy \\
Athletics & Escape Artist & Crafts &\emph{Awareness} & Entertaining \\
Intimidate & Stealth & Knowledge & Animal Handling & Deception \\
Swimming & Fairy hands & Deactivating devices &Nature & Local traditions \\
& Using rope & Herbalism &Orientation & \\
& & Falsifying &Perceiving Emotions & \\
& & Evaluate &First aid &\\
& & &Following tracks &\\ 
& & &Survival &\\
\end{tabular}

\medskip

The \textbf{Knowledge} must be explained on which topic it concerns: Dungeons, Law, Languages, Plans, Occult, Architecture and Engineering, Nobility and Heraldry, Myths and Legends, Religion, History, Geography ...

\begin{multicols}{2}

For each \textbf{level after the first} distribute a number of points equal to half the Intelligence score +1, [(Int/2)+1], with a minimum of 1 point, among the skills already known or perfected in the adventure or learned from scratch.

\textbf{No Basic or Active proficiency can have a score higher than character level+3.}\index{Maximum Proficiency score}


\subsubsection{Awareness}\label{consapevolezza}\index{Awareness}

A skill that all characters have is \textbf{Awareness}, which is the ability to perceive the environment around them. This skill has a fixed score equal to 1/3 of the character's level (rounded up) plus Wisdom. The character cannot assign points to this skill, but can choose certain skills to raise his score.

Rather than using Awareness to search for information, players should ask questions, investigate, snoop, infer hypotheses and compare notes and not simply ask for an Awareness roll to find something.

\subsubsection{Learn new skills, professions}\label{apprenderenuovecompetenze}

A character can learn a new skill or improve it with study/practice of at least 4 hours a day for at least 4 months with a teacher who has a higher proficiency score than the character is aiming for. After this period of time the player can assign a point to the basic skill for which he applied.

To learn a new profession, you must spend at least 6 months for 6 hours a day with those who practice that profession. After 6 months the character acquires the 4 skills of the profession. Any Skills already known will increase by 1 point.

\subsubsection{Skills and their areas of use}\label{competenzeambitidiutilizzo}

The Skills and their usual uses are briefly described. The number of Actions necessary to carry out the typical test is also indicated, more complex uses require more time and Actions.

The Actions necessary for the test may vary depending on the character's ability and the complexity of the test.

In any case, always remember to carefully evaluate how the player claims to carry out the actions to understand their duration and effects.

Skills with a \textbf{*} suffer the penalties due to the \hyperlink{equipment.armor.shields}{armor} worn (page \pageref{armor equipment}).\\

\textbf{Acrobatics* (DES)}: This skill is used to maintain balance on narrow or precarious surfaces, to dive, roll, do somersaults, somersaults, overcome obstacles as well as fall and not get hurt. 1 Action.

\textbf{Arcana (INT)}: With this proficiency you are an expert in magic and spells, magical objects and are able to identify the spells that are cast. 1 Action.

\textbf{Climb* (FOR)}: With this skill you can climb vertical surfaces, from city walls to rock faces. It is linked to the Movement Action. With 8 points the Climb movement is only halved.

\textbf{Crafts (INT)}: It is necessary to specify the type of Craft in which you are competent. You are competent, but not at Profession level, in a form of craftsmanship.

\textbf{Escape Artist (DES)}: With this skill you can free yourself from bindings and handcuffs. 1 Action for every 10 of DC. With 6 points the time is 1 Action every 15 DC, with 12 it is 1 Action every 20 DC.

\textbf{Athletics* (FOR)}: With this skill one is an expert athlete, capable of prodigious jumps and exceptional feats of Strength. 1 Action.

\textbf{Ride (SAG)}: With this skill you can ride professionally and give commands to your mount. 1 Action.

\textbf{Awareness (SAG)}: to search, notice, notice. It's something active. 2 Shares. \textbf{Using 1 Action imposes a -1d6 penalty on the check}.

\textbf{Knowledge of Architecture and Engineering (INT)}: You are an expert builder and know how to evaluate the structure of buildings. You also know how to recognize architectural styles and create interior and furnishing projects. 1 Action.

\textbf{Dungeon Knowledge (INT)}: With this proficiency you have knowledge of Aberrations, caves, underground exploration, Oozes. 1 Action.

\textbf{Knowledge of Geography (INT)}: With this skill you have knowledge about the climate, population, land, territories, nations and borders. 1 Action.

\textbf{Knowledge of Law (INT)}: With this skill you know the Law of a region. You are an expert in knowing the rules and quibbles. You know how to cite cases and you know other guessers and judges. 2 Shares.

\textbf{Language Knowledge (INT)}: Each point in this skill allows you to learn a new written and spoken language. A good Language score helps you understand unfamiliar languages ​​and make yourself understood. It is also used to understand complex texts. Variable cost.

\textbf{Knowledge of Myths and Legends (INT)}: You have a real passion for traditional and more remote myths and legends. Learn about locations, history and legendary creatures. 1 Action.

\textbf{Knowledge of Nobility and Heraldry (INT)}: Know noble lines, families, rumors, coats of arms, personalities and major possessions and treasures. It also applies to famous and important people. 1 Action.

\textbf{Knowledge of the Planes (INT)}: With this proficiency you are an expert on the Planes and their inhabitants. 1 Action.

\textbf{Occult Knowledge (INT)}: With this proficiency you are an expert in the occult, fiendish creatures. 1 Action.

\textbf{Religious Knowledge (INT)}: With this skill you have knowledge about Patrons, mythology, Celestials, Undead, sacred symbols, ecclesiastical tradition, liturgical celebrations and celebrations. 1 Action.

\textbf{Knowledge of History (INT)}: With this skill you have knowledge of History such as wars, migrations, colonies, foundations of cities, important events.. 1 Action.

\textbf{Diplomacy (CAR)}: With this skill you can resolve disputes, and gather valuable information and rumors from people. Competence is also used to negotiate effectively with the right etiquette and conduct suited to the disputed situation. Variable cost.

\textbf{Disable Devices (INT)}: With this skill you can disarm traps and open locks, sabotage simple mechanical devices, such as catapults, wagon wheels or doors. 1 Action for every 10 of DC. With 6 points the time is 1 Action every 15 DC, with 12 points it is 1 Action every 20 DC.

\textbf{Herbalism (INT)}: With this skill you have knowledge of how to recognize and prepare potions and natural poisons. The score applies to checks to brew potions. Recognize Natural Potions 1 Action per 10 of DC. With 6 points the time is 1 Action every 15 DC, with 12 points it is 1 Action every 20 DC.

\textbf{Forgery (INT)}: With this skill you know how to forge art objects, maps, signatures... Variable cost.

\textbf{Handle Animals (SAG)}: With this skill you can train and tame animals. 1 minute every 5 of DC. With 6 points the time is 1 minute every 10 DC, with 12 it is 1 minute every 15 DC.

\textbf{Intimidate (FOR)}: Intimidate is based on physical approach to convince the interested party. 2 Shares. With a score of 12 it costs 1 Action.

\textbf{Deceive (CAR)}: The Deceive skill can be used to Deceive (thus telling lies) or Persuade (adapting the truth) in order to convince the interested party of one's words. Variable cost.

\textbf{Entertain (CAR)}: With this skill you are an expert in artistic expression, from singing to acting, from dancing to playing musical instruments. It is necessary to specify the form of entertainment. Variable cost.

\textbf{Fairy Hands* (DEX)}: With this skill you can pickpocket, draw a hidden weapon, or perform other actions without being noticed. 1 Action.

\textbf{Stealth* (DEX)}: With this skill you are able to move without causing noise or hiding in shadows. 1 Action.

\textbf{Nature (SAG)}: With this skill you have knowledge of Animals, Fairies, seasons and cycles, weather, plants. 1 Action.

\textbf{Swim* (STR)}: With this proficiency you are able to swim, even in stormy waters. Without skill you know how to stay afloat in placid water. Linked to the Movement Action.

\textbf{Orientation (SAG)}: With this skill you have a sense of direction and orientation making it impossible to get lost regardless of the environment you find yourself in. 2 Shares.

\textbf{Sense Emotions (SAG)}: With this skill you can understand if someone is lying or you can guess their true intentions. 1 Action.

\textbf{First Aid (SAG)}: With this skill you can treat wounds and illnesses. Variable cost.

\textbf{Following Tracks (SAG)}: With this skill you know how to follow tracks left in the environment. 1 Action for every 10 of DC. With 6 points the time is 1 Action every 15 DC, with 12 points it is 1 Action every 20 DC.

\textbf{Survival (WIS)}: With this skill you can survive and find your way in the wilderness. The proficiency is also used to actively search for traps and pits. 1 minute to search for traps in 3x3 metres, with a score of 6 it costs 3 rounds, with a score of 12 it costs 1 round, with a score of 18 it costs 1 Action.

\textbf{Local traditions (CAR)}: With this skill you have knowledge of the (best known) inhabitants, customs, legends, laws, personalities, traditions. It is necessary to specify a geographic region where the knowledge is applicable. 1 Action.

\textbf{Use Rope (DES)}: With this skill you are expert in ties and knots to secure and secure objects or people. 2 Shares.

\textbf{Evaluate (INT)}: With this skill you can estimate the monetary value of an object. 1 Action for every 5 of DC. With 6 points the time is 1 Action every 10 DC, with a score of 12 it is 1 Action every 20 DC.

\medskip

\subsubsection{Optional - Don't use Basic Skills}\index{Optional - Don't use Basic Skills}\label{nonusarecompetenze}

Have players choose their profession and do not record any scores or base proficiency values.
Think with an open mind and understand, both you Narrator and you Player, for each situation who has the profession and skills that best suit the test.
The test, if relevant to the profession, is resolved with a 3d6+Wisdom+1/2LV, if it is not relevant the Narrator will reduce the bonus given by the level, using the most appropriate Characteristic. Better yet, the outcome will be decided based on the description of how the test is carried out.

\subsection{Active Skills}\index{Active Skills}\label{competenzeattive}

\textbf{The character takes 1 point, at each level, to distribute among the Active Skills or attribute it to the Basic Skills.}.

\medskip

The \textbf{Active Skills} are: Magical Proficiency, Weapon Proficiency, Saving Throws (Reflexes, Fortitude, Willpower).

- \textbf{Magical Expertise (CM)}: \index{CM}\index{Magical Expertise} indicates the ability and competence in casting a spell.

- \textbf{Weapon Proficiency (AC)}: \index{AC}\index{Weapon Proficiency} is the ability and skill to fight with a melee or ranged weapon.

- \textbf{Saving Throws} are increased by the chosen Skills.

Attributing the Active Skills point to \textbf{Basic Skills} means distributing 4 additional points on at least 3 Basic Skills as desired.\index{Increasing Basic Skills}

\begin{changemargin}{0.3cm}{0.3cm}\begin{enfasi}{There is only one way to train: the right way. (Carl Lewis)
\medskip

Wang Chi: Are you ready?

Jack Burton: I was born ready! (Big Trouble in Chinatown, Film 1986)
}\end{enfasi}\end{changemargin}


\subsubsection{Saving Throws}\index{Saving Throws}\label{tirisavellza}

The \textbf{Saving Throws} (abbreviated to ST) are used when the character is subjected to an effort, either physical or mental resistance or exceptional agility. The scoring of saving throws is based on the chosen Feats. More physical Feats will tend to improve the character's resistance aspect, more athletic or attention Feats will increase reflexes, purely mental Feats will strengthen the character's will.

The \textbf{Fortitude Saving Throw} indicates how much you are able to withstand physical suffering or attacks against your vitality and health. The \textbf{Constitution} score is added to the value of Fortitude saving throws.

The \textbf{Will Saving Throw} indicates resistance against mental influence and other magical effects, which wants to modify your free will in choices and actions. A score of \textbf{Wisdom} is added to the value of Will saving throws.

The \textbf{Reflex Saving Throw} indicates how agile and ready you are to avoid obstacles or spells. The score of \textbf{Dexterity} is added to the value of Reflex saving throws.

When a saving throw is asked it means making a check on the required Active Proficiency, be it Willpower, Fortitude or Reflexes.
The test will be performed by rolling 3d6 + the value of the required Active Competence or the score in the Willpower, Reflex or Fortitude saving throw + the value of the Characteristic linked to the type of Active Competence (Wisdom, Dexterity or Constitution) + Feats + magical bonuses (objects which affect the saving throw) and various modifiers present.


\begin{changemargin}{0.3cm}{0.3cm}\begin{tcolorbox}[title = Non-standard saving throws]
It is possible that saving throws with different modifiers are required, i.e. a Fortitude save with a Strength modifier or a Will save with a Charisma modifier. The Storyteller will tell you when a different modifier applies.
\end{tcolorbox}\end{changemargin}

If you roll \textbf{three times 6 on the saving throw} the same is successful, regardless of the final result.\index{Three times 6 on the saving throw}

\subsubsection{Weapon Proficiency}\label{competenzaarmi}

The \textbf{Weapon Proficiency} (abbreviated to \textbf{CA}) indicates the ability and Feat in using a weapon. Proficiency is directly reflected in checks to hit the opponent with weapons.

The \textbf{Attack roll for melee weapons}\index{Melee weapons} is resolved with a Weapon Proficiency check (\textbf{AC}) + \textbf{Strength} + any Feats + bonuses from the Weapons List, magical bonuses and modifiers against the opponent's Defense (Dexterity + armor/shields/modifiers).

The \textbf{Attack roll with ranged weapons} \index{Range weapons} (bows, crossbows, throwing daggers, javelins, stones..) is resolved with a Weapons Proficiency check (\textbf{AC }) + \textbf{Dexterity} + bonus from Weapons List + any abilities, magic bonuses and modifiers against the opponent's Defense (Dexterity + armour/shields/modifiers).

When assigning a point to \textbf{CA} it must always be specified which group of weapon you take, if you do not declare it then it is as if you took it in the Simple Weapons group.
Check the list \hyperlink{list.weapons}{Weapons by Homogeneous Type} (page \pageref{list.weapons}).\index{Homogeneous type}

The character can decide to assign his point to a type of weapon that he already knows, thus improving his ability and talent in using it or learn another type of weapon.

The higher the score in a type of weapon, the more easily he can take advantage of advantages in it, but he will know fewer weapons.

If the player has not assigned any points in the \textbf{CA} he can use only the weapons grouped as Simple Weapons without penalty on hitting.

A character who uses a weapon present in the Weapon Lists he knows or in the Simple Weapons will always apply his Weapon Proficiency (AC) value to the attack roll, only when he uses an unknown weapon will he have penalties (-1d6).

The \textbf{Simple Weapons} are: Dagger, Light Mace, Club, Spiked Mace, Staff, Crossbow (Light), Javelin\index{Simple Weapons}

Using a \textbf{Weapon without the appropriate proficiency} in the group it belongs to imposes a -1d6 on the attack roll.\index{Weapon without proficiency}

To be able to use \textbf{Light Armour} and \textbf{Light Shields} you must have at least one point in Weapon Proficiency or Strength at least -1.\index{Light Armour}\index{Light Shields}

To be able to use \textbf{Medium Armours} and \textbf{Medium Shields} you must have at least 2 points in Weapon Proficiency.\index{Medium Armours}

With at least 3 points in Weapon Proficiency and 1 in Strength you can use \textbf{Heavy Armour} and \textbf{Heavy Shields} without penalty.\index{Heavy Armour}\index{Heavy Shields}

Using \textbf{Armor without the appropriate proficiency} prevents you from using the Dexterity value in Defense and the bonus given by the armor to Defense is reduced by 1.\index{Armor without proficiency}

Using a \textbf{Shield without the appropriate proficiency} worsens the attack roll by 1 and the shield grants a maximum bonus to Defense of 1.\index{Shield without proficiency}

\subsubsection{Magical Expertise}\label{competenzamagica}

The \textbf{Magical Expertise} (abbreviated to \textbf{CM}) allows the character to be able to know more spells, more powerful, more effective and more easily.

A character with high \textbf{Magic Proficiency} can manipulate more spells and with better results.

The Magical Expertise value establishes, together with the Adept of Magic Feat, the maximum level of spells that can be cast.

\begin{changemargin}{0.3cm}{0.3cm}\begin{tcolorbox}[title = Tups reaches the 4th level!]

Tups has reached the 4th level! Here's how he distributed the Active Feats points.

\textbf{1 level}: +1 Weapon Proficiency, Feats: Devoted Armor (+2 Will, +1 Reflex), Weapon Focus (+1 Reflex, +1 Fortitude)

\textbf{2nd level}: +1 Magical Expertise, Feat: Powerful Strikes (+2 Fortitude)

\textbf{3rd level}: +1 Magical Expertise, Feat: Faithful (+2 Will, +1 Fortitude)

\textbf{4th level}: +1 Weapon Proficiency, Feat: Cautious Spellcaster (+2 Reflexes, +1 Fortitude)

\textbf{\emph{Total}}: +2 AC, +2 CM, +5 Reflex save, +5 Fortitude save, +4 Will save
\end{tcolorbox}\end{changemargin}

Each point awarded in Basic Feats or Weapon or Magical Proficiency allows you to benefit from +1 in the relevant test (Attack Roll, Magical Proficiency)

\subsubsection{Optional - Feats as Active Skills}\index{Optional - Skills as Active Skills}

Upon the player's request, the Storyteller can grant the ability to use the Active Expertise point not to increase Magical or Weapon Expertise, but to select a new Skill, respecting the requirements.

\end{multicols}

%\begin{center}
%\includegraphics[width=0.25\linewidth]{immagini/giavellottiragazzo4.png}
%\end{center}


\pagebreak

\section{Let's build the character}\index{Character}

\begin{changemargin}{0.3cm}{0.3cm}\begin{enfasi}{
Never forget who you are, because the world certainly won't forget it. Transform who you are into your strength, so it can never be your weakness. Make it armor, and it can never be used against you. (Tyrion Lannister)
}\end{enfasi}\end{changemargin}\medskip

OBSS is a tough, dangerous, deadly system but also full of satisfaction. Your characters are not heroes, they are not chosen ones. They are unfortunate people who find themselves in businesses where perhaps they will survive and it will be at the expense of some comrades. You don't choose the adventure but it drags you impetuously into it. Be strong, courageous, witty but not reckless. 

Survive and claim the Law of the Prize and you will see that as the levels go by you will acquire extraordinary Feats and abilities!. \emph{Spes ultima goddess}!

\begin{multicols}{2}

\lettrine[lines=2, lhang=0.33, loversize=0.25, findent=1.5em]{C}{ome} first prepare the card and a piece of paper in front of you where you can take notes and notes.

To create a character, try answering these questions, they will help you imagine and shape it.

- Imagine what it looks like

- What is the main character trait

- What are his tics, ways of doing things, habits

- What are its primary objectives

- A curious thing, a funny thing, an embarrassing thing and a typical expression of the character

- What he is good at, what he is committed to, what he is denied

- The three flaws and the three main strengths of the character

\begin{center}
\includegraphics[width=0.6\linewidth]{immagini/Leonidas_I_of_Sparta.png}

%\includegraphics[width=\columnwidth]{immagini/Leonidas_I_of_Sparta.png}

\emph{Leonidas of Sparta}
\end{center}

He grew up in a family, in a clan, a vagabond, on the street.. what brought him and what choices did he make to get to this point?

What is his typical fighting style and strategy? Magic, Sword, from the rear... cheering on your teammates... running away...

And no less important: what is the purpose of him? what made him leave home, from his security... from a normal life and take up that of an adventurer?

To begin, read the chapter on Races and identify your character's race.

Always remember that this is a cruel world, full of risks, traps and monsters, but also opportunities that can make you powerful and very rich.

Recover some d6 and roll!

Consult the chapter of \hyperlink{assignment of scores.feature}{Features} to understand how lucky you were (page \pageref{assignment of feature scores}).

If you have an Intelligence of 2 or more, choose another \hyperlink{languages}{language} (page \pageref{languages}) spoken/written in addition to the Municipality, if you have 3 you can choose 2 more languages.

And if the Characteristics values ​​didn't turn out as you expected then let the chaos guide you and create something different but equally fun and magnificent.

Move on to Active Skills, here you have 1 point to distribute between Weapons Expertise and Magical Expertise.

Weapon Proficiency helps you hit better. Magical Proficiency is the only thing that allows you to use magic. Also remember that the points in Weapon Skills must be declared to which \hyperlink{list.weapons}{List Weapons} (page \pageref{list.weapons}) they have been applied to.

If you have no points in Weapon Proficiency you can only use the \hyperlink{simple.weapons}{simple weapons} (page \pageref{simpleweaponslist}) without incurring penalties on the attack roll.

Basic Skills are assigned based on the character's Profession. Choose it with attention and care, in addition to the skills required by the chosen profession you can take a fifth given from your background or increase the score in one.
Based on your background and chosen profession you increase a characteristic by 1, up to a maximum of 4 + racial modifier.

Hit Points are equal to 1d4 + Constitution + 3 if you put 1 point into Weapon Proficiency (AC).

At this point choose the \hyperlink{traits}{Traits} (page \pageref{traits}). Do it carefully, you are building your character and the Traits outline the character with strong brushstrokes. Remember that they will be fundamental for choosing the \hyperlink{patroni}{Patroni} (page \pageref{patroni}).

On the card, in the Traits mirror, where Patron is, mark with an X the Traits that connect you to the Patron, whether you have chosen it or not.

Finally, remember that a \emph{solitary} and \emph{cynical} character sounds good in a story where he is the only protagonist, but here we play in a \textbf{group}, do not take Traits in obvious opposition to the others or however, don't play \emph{asshole}, otherwise the character will naturally be separated from the other characters.

Choose the \hyperlink{disadvantages}{disadvantage} (page \pageref{disadvantages of role}) role and if you want also disadvantages and \hyperlink{advantages}{advantages} (page \pageref{advantages}). Remember to play it, otherwise it's not fun and you won't get Experience Points.

If you have put points into Magical Expertise, also remember to take the Adept of Magic Skill otherwise you will be able to choose between very few spells.

At this point you must choose which Spells you know.
In your Tome of Magic you can write a number of spells equal to 2 + your spell ability modifier, of which you can know, or cast, 2 + half the value of your spell ability modifier.

Go to \hyperlink{feats}{Feats} (page \pageref{feats}), at the first level you choose two, pay attention to the prerequisites and also to any feats that your race grants you.

It is the Feats you choose that increase your saving throw score. Remember that saving throws determine your ability to resist trauma and magic.

Choose \hyperlink{equipment}{equipment} (p. \pageref{equipment}), \hyperlink{equipment.armor.shields}{ armor} (p. \pageref{armor equipment}), \hyperlink{weapon equipment}{weapons} (page \pageref{weapon equipment}), backpack, two torches, some food rations.. a soft toy.. whatever you think is essential for the adventure.
Then update the part of the sheet relating to Defense, noting what bonus the armor and shield worn gives you. Remember that you start with 100 gp, spend it carefully!

Step into the role, allow yourself to play this extraordinary character. If you ever get bored of playing it and want to try something different, talk to the Narrator, he will be able to advise you and suggest the best path.
Furthermore, you have the advantage that in OBSS classes do not exist, the character grows, evolves and learns based on what you do and experience. You can prepare the \emph{build} at the table but you will never be sure that your character will evolve as you thought. Let him live and grow!

\begin{center}
\includegraphics[width=0.9\linewidth]{immagini/Alexander_and_Bucephalus_-_Battle_of_Issus_mosaic.png}
\emph{Alexander the Great}
\end{center}

Finally, remember the Law of the Prize \index{Law of the Prize}. Yeru is ferocious, often evil, even more so he will want to kill you, yet for those who survive there is the Law of Reward, a law that not even the Patrons can violate. The Law is quite simple in its basic concept \emph{To those who survive will go the treasures and the glory}.

\subsection{Level Up}\index{Level}\index{Level Up}\label{avanzamentodilivello}\index{Level Up}

\begin{changemargin}{0.3cm}{0.3cm}\begin{enfasi}{
But there are things that cannot be understood with reflection, you have to live them. (The Neverending Story, Michael Ende)
}\end{enfasi}\end{changemargin}


Every time the Narrator confirms that you have leveled up, various operations must be carried out to update the character sheet.

\emph{First take the card, pencil and eraser and the dice (at least the d4).}

- Update Experience Points

- Update the Level by increasing it by 1

- Distribute 1 point between Weapon Proficiency and Magical Proficiency

- Increase your hit points by 1d4+Constitution and add another 3 if you gave 1 point in Weapon Proficiency

- If you have assigned a point in Weapon Proficiency, decide whether to take a new List of Weapons or deepen your knowledge of an already learned list.

- Check if you acquire a new Feat. You can take a new one or improve an already learned Feat, pay attention to the prerequisites. Remember that you get a new Feat at all levels except 5-10-15-20.

- Update the saving throw scores based on the new Feats taken.

- Update the score of the attack rolls based on the new value of the Weapon and Feat Proficiency and bonuses given by the Weapon List.

- Distribute (Int/2)+1, with a minimum of 1 point, among the Basic Feats known or learned during the adventures. Check your Awareness score.

- Updated Fate Points score (20-level)/5

- Increase a Trait score as the Storyteller tells you. Check if you have reached a sufficient score to acquire Trait-related powers.

- Check the maximum level of spell that can be cast and the available Magic Points based on the new Magical Expertise score, the Adept of Magic Feat and the Characteristic score.

- If you have increased your Magical Expertise you learn 2 new spells from the Tome of Magic or by sacrificing one you can learn two cantrips (0 level spells). You can also replace the spells you learn with ones from the Tome.

- Update the second part of the sheet based on the new Magical Expertise score

As you may have noticed, the Skill scores are reduced, a few points are taken to distribute at a time.
As players you have the opportunity to prefer a specialized approach, i.e. \emph{to bet} on a few specific skills or to dilute the points across several skills to know a bit of everything and not have penalties in the tests (the test is done only with 1d6 + Feature if you have no points in the Skill).

A suggestion is also to use Feats, and in particular Expert, which gives you a +2 bonus on Skill checks.

\begin{changemargin}{0.3cm}{0.3cm}\begin{narratore}
The \textbf{perceived} power level of characters in OBSS is lower than that of other RPGs. In OBSS the aim is to explore and understand this changing and crazy world. The weakness of the character is only a perception and in fact you will soon realize the true power of the character. Play as a group and you will survive because remember that this is a mean, spiteful and deadly world with \textbf{selfish}.
\end{narratore}\end{changemargin}

%\subsection{Tips for having fun and surviving OBSS adventures}\index{Guidelines for players}\label{suggerimentigiocatori}

\subsection{How to Survive and Have Fun}\index{Guidelines for Players}\label{suggerimentigiocatori}

\begin{changemargin}{0.3cm}{0.3cm}\begin{enfasi}{
- We need a plan.

- Since when do heroes need plans? (Final Fantasy XIII)

\medskip

I'm crazy about successful plans! (Colonel John \emph{Hannibal} Smith, A-Team)}
\end{enfasi}\end{changemargin}\medskip

\begin{itemize}

\item
Every fight is potentially lethal. Decide rationally and approach it carefully. Learn to escape, don't be afraid to survive.

\item
Not everything is on the card. A character's sheet is his perimeter but does not define what he can or cannot do. Brainstorm and be creative, alternative, curious but not suicidal or reckless.

\item
Not everything can be solved with a giving roll. Ask the right questions, talk to your classmates and carefully describe what you intend to do. The Storyteller rewards accurate descriptions. Describing how and what you do can avoid having to take the test!.

\item
Low characteristics are just low characteristics and not the character. Make use of your skills and abilities and make sure you have to roll as few dice as possible to solve problems.

\item
Improvise, adapt and reach the purpose! (Tom Highway - Gunny, Film). Or like some of my players preferred \emph{Improvise, \textbf{Deceive} and achieve the goal}.

\item
Live your character fully. It amplifies his history and brings his past into the present. Help your classmates get to know you and the Storyteller to craft better stories around your stories.

\item
One thing that no one can ever take away from you is being heroic, intelligent, resolute, stubborn, stubborn but not stupid. Live the adventure to the full but never be afraid to survive.

\item
Describe what you do realistically, you will help the Narrator and the companions around you. It's definitely better than saying \emph{I'll do an Awareness test}. Exalted in describing the most important actions, the Storyteller will take them into account.

%\item
%And until you can say "\emph{I'm bad, pissed off and tired. I'm someone who eats barbed wire, pisses napalm and can put a ball in the ass of a flea at 200 meters}." (Tom Highway - Gunny, Film) then stay in your place and don't be a braggart, there's always someone bigger and angrier than you.

\item
Always remember that the greater the danger, the greater the experience gained. The deeper the dungeon, the greater the treasures and experience gained!

\item
The aim is to have fun, entertain and savor the challenge. Don't create a character who is against other characters or always causes annoyance and problems. Mediate your desire with the needs of the group, because you will always survive and \textbf{only as a group} and never only as an individual.

\item
Think before you act, but don't make others wait. Use the time between your rounds to plan how best to act.

\item
If you have difficulty understanding or imagining something, ask the Narrator for more information and clarification, he will only please him.

\item
Embrace failure. Failing with style is much better than a boring victory.

\item
Always make your character care about more than his life.

\item
Don't be afraid to argue with other characters, but always make sure you don't get personal with players.


\end{itemize}


\end{multicols}


%\vfill
\smallskip

\begin{changemargin}{0.3cm}{0.3cm}\begin{enfasi}{
The candle lit on both sides lasts half as long. (Anonymous)
}\end{enfasi}\end{changemargin}

\begin{center}
\includegraphics[width=0.45\linewidth]{immagini/threasure2.png}
\end{center}

\pagebreak

\section{Skill Rules}\index{Skill Rules}\index{Skills}

\begin{changemargin}{0.3cm}{0.3cm}\begin{enfasi}{
The law must be brief, so that those who are ill-practiced can more easily remember it. (Lucius Anneus Seneca)}\end{enfasi}\end{changemargin}

\begin{multicols}{2}

\lettrine[lines=2, lhang=0.33, loversize=0.25, findent=1.5em]{L}{\textbf{e}} \textbf{tests (i checks), for the Skills or Characteristics, are carried out by rolling 3d6, the score of the Competence (basic or active) and of the connected Characteristic and any magical and circumstance bonuses or Skills are added to the result of the dice; the result obtained must be communicated to the Narrator, who will compare it with the difficulty ( DC) of the test}.

When you have to establish a difficulty, start by thinking that the test must be reported by a person \emph{normal}. Don't think \emph{if I had to do it then the test would be impossible}, \emph{if Arsenio Lupine does the test the test is very easy}. Start from the assumption that the difficulty must contain all the circumstantial elements.

Think about if it's raining, there's little light, the character is running, he's injured, he's doing things in a hurry and also the complexity of the thing he has to do, jumping a 3 meter ditch isn't like jumping a 3 meter ditch in the dark, without shoes, in the rain and chased and with pockets full of coins...

Deciphering an ancient writing may be a walk in the park for an expert linguist, but for a \emph{normal person} who has no idea what the test may be facing is simply impossible. This \emph{impossible} is your DC, the difficulty of the test.

And don't be scared if the characters fail the tests, it will make the adventure more interesting and allow the Narrator to introduce facts, clues and new adventures.

\begin{changemargin}{0.3cm}{0.3cm}\begin{narratore}
Avoid asking for a test if the players declare \textbf{how} they carry out the test, how and where they look, what dialogue they create to intimidate the target. Evaluate carefully how the player describes what he does because this is already the test . It's not just to speed up the game, it serves to stimulate players to think fully and immerse themselves in the character and the environment.

It will make the game more dynamic and all players will participate in the situation and collaborate by declaring what and how they act. Always use common sense and save dice rolls! Rolling a die creates the possibility of failure!
\end{narratore}\end{changemargin}



%\medskip
%\begin{center}
%\includegraphics[width=0.8\linewidth]{immagini/master2.png}
%
%\emph{The Master of the Gamblers}
%\end{center}

\medskip

\textbf{When you have to make a test for a Basic Competence in which you are not prepared, i.e. you have no points, you must roll only 1d6 + score of the connected Characteristic}.\index{Test competence without proficiencies}

When you write -1d6 it means that you roll one die less (or two if it is -2d6), vice versa if it says +1d6 you roll one die of 6 more and add it.

The table below serves to relate the difficulty to the minimum skill needed to succeed in the test with an average roll (a score of 10 rolling 3d6). Use these indications to get an idea of ​​the difficulty scales.

The Narrator will not tell you give me a test at difficulty 10, but he will say that the test does not present particularly difficult elements.

%\begin{center}
%\includegraphics[width=0.9\linewidth]{immagini/difficulty.png}
%
%\emph{A City on a Rock, long attributed to Goya, is now thought to have been painted by 19th-century artist Eugenio Lucas Velázquez. Excellent proof of forgery}
%\end{center}
%\medskip

\medskip

\textbf{Table: Difficulty class}\index[Tables]{Difficulty class table}

\medskip
\begin{tabularx}{0.45\textwidth}{llll}
\textbf{Diff.} & \textbf{Description} & \textbf{Level}\\
\textbf{DC}&\textbf{difficulty}& \textbf{Proficiency}\\
\toprule
5 & ​​Extremely easy & Nothing\\
10 & Easy & Poor\\
15 & Normal & Normal\\
20 & Hard & Good\\
25 & Very difficult & Excellent\\
30 & Heroic & Excellent\\
35 & Almost Impossible & Amazing\\
40 & Impossible & Epic\\
\end{tabularx}

\bigskip

If you have to make a test on a Characteristic you must roll 3d6 and add the Characteristic score and other modifiers. Communicate this result to the Narrator who will compare it with the difficulty (DC).

\subsection{The Golden Rules}\index{The Golden Rules}

Unless otherwise specified, three basic rules apply to all proficiency tests (Basic, Active) \index{Basic Rules} called \textbf{Golden Rules}:\index{Golden Rules}

\begin{itemize}
\item
The \textbf{6 explode}, that is, if in the 3d6 test a die rolls a six, add the result and reroll, and if it rolls a 6 again, add the result and reroll again and again..
\item
The \textbf{1 is bad luck}, if you roll a 1 you remove 1 from the sum of the rolled dice (and therefore the die that rolled a 1 counts as zero)
\item
\textbf{Trust your luck}. For every 4 points between Competence (Basic or Active) and Characteristic that you fail to add in the test, you roll an extra 6 (Attack Roll, Saving Throw, Competence checks). This value cannot be subtracted from the score given by Skills or magic items.

\begin{center}\textbf{\emph{Corollary}}\end{center}\index{Golden Rules Corollary}

These notes count towards the initial roll of 3d6.

\item \textbf{Rolling 3 times 6 is a success}, both in Proficiency Checks, Saving Throws and Attack Rolls regardless of the final result.\index{Rolling three times 6}

\item \textbf{Rolling 3 times 1 is a failure}, both in Proficiency Checks, Saving Throws and Attack Rolls regardless of the final result. \index{Roll 1 three times}

\item \textbf{Rolling 6 twice} is an omen of good fortune (Critical Success) if you succeed in the test \index{Critical Success}

\item\textbf{Rolling 2 times 1 or 1 time 1 and twice 2} is a harbinger of misfortune (Critical Failure) if you fail the check \index{Critical Failure}

\end{itemize}

Use the \textbf{Golden Rules} to your advantage! Dare, try, take risks when the situation does not allow other solutions!

\begin{changemargin}{0.3cm}{0.3cm}\begin{tcolorbox}[title = Not just the card!]{
Don't necessarily look for the solution in the sheet. Use your ability to imagine, to solve, to intuition to get out of and resolve situations. The card represents only a small part of what your character can do.
}\end{tcolorbox}\end{changemargin}

\subsection{Pass or Fail the test}\index{Pass or Fail the test by a lot}\label{superareofallirelaprova}\index{Critical success in the tests}

The test is passed when the 3d6 are rolled and the relevant Competence and Characteristic as well as the various modifiers are added, the result is equal to or greater than the DC established by the Narrator.

If the result is lower than the difficulty, the test is failed.

Whenever the test is \textbf{passed with a critical}, i.e. the test is successful and at least two 6's have been rolled, the Narrator can decide to give more information, grant a bonus to subsequent actions (+1).. any what can enhance how easily the test was passed.\index{Pass the test with a Critical}

Conversely, if the test fails \textbf{and two 1's or a 1 and two 2's were rolled} the Narrator could describe how miserably the test failed and how the terrible result influences the Action and subsequent ones.

A test can be repeated\index{Repeat a test}\index{Redo a test} until the conditions that allow the test to be repeated change.

Think about how competent a character is in order to avoid any test with an obvious outcome.

\begin{changemargin}{0.3cm}{0.3cm}\begin{narratore}
If the test can be repeated until success without problems or interruptions then do not do the test, describe the attempts, the difficulties encountered and declare success.
\end{narratore}\end{changemargin}

\subsection{Awareness}\index{Awareness}\label{consapevolezza2}

Awareness is one of those skills that comes into play very often.

Make sure that the questions and reasoning of the characters reveal the clues, an Awareness check can be made every time there is something to look for that is not obvious, something that must be looked for otherwise it is not immediately perceptible or intuitable, something that players want to find and that is there but they don't ask the right question.

\begin{changemargin}{0.3cm}{0.3cm}\begin{narratore}
Don't let the tests rule your game. \textbf{Let the players play}, let them act, let them participate and based on what they say decide whether they passed the test or not.

If they tell you \emph {I convince the guard to let us pass} have them make an Intimidate (or Diplomacy) check, if instead they engage in a convincing dialogue you can consider that the check was made with a positive outcome (or negative if they were unable to argue!) Reward the HOW more than the WHAT.
\end{narratore}\end{changemargin}

\subsection{The Evidence}\index{Opposing Evidence}\label{proveopposte}\index{Opposing Evidence}

\subsubsection{Proficiency Tests pitted against an opponent}\index{Trials pitted against an opponent}

There are situations in which the character must perform an Opposed Test against an opponent, for example Stealth to move silently behind a guard, steal from the merchant's pockets, intimidate the orc into giving him directions, push an opponent...

In this case the character performs the indicated test whose \textbf{difficulty (DC) is equal to 10} + the Characteristic score + Proficiency or Saving Throw + contingent modifiers (bonuses/penalties). 

Whoever obtains the highest value wins, in case of a tie, the winner is whoever has the highest value in the Competence, then in the Characteristic and finally the possible \emph{opponent}. \index{DC Static in opposed tests}\index{Opposed tests}\\

\textbf{Some examples of Test Contrapposte}

- Deceiving someone: Deceiving Vs Perceiving Emotions

- Dressing up to look like someone else: Entertainment Vs Awareness

- Creating a fake map: Falsifying vs Evaluating

- Stealth: Expertise Vs Awareness, as long as unseen

- Intimidate: Intimidate vs. Will save (with Charisma modifier)

- Stealing: Fairy Hands vs. Awareness, or Fairy Hands if possessed 

- Untying yourself from ropes: Using Ropes Vs Escape Artist

- Arm wrestling: Fortitude save (with Strength modifier)

\subsubsection{Opposed Characteristic Tests}

Whenever the \textbf{Test} or \textbf{Opposed Test} concerns a \textbf{Characteristic} and not a Competence, make the test (3d6+) by adding the most suitable Saving Throw to the most suitable Characteristic.

\medskip

\textbf{Table: Opposed Trials and Modifiers}\index{Table of Opposed Trials and Modifiers}\label{Tabella Prove Contrapposte e Modificatori}

\begin{tabularx}{0.45\textwidth}{Xl}
\toprule
\textbf{Contrasted Trial} & \textbf{TS} \\
Strength & Temper \\
Dexterity & Reflexes \\
Constitution & Temper \\
Intelligence, Wisdom, Charisma & Will \\
\end{tabularx}

\medskip

It is possible that Contrasting Tests may be requested with different modifiers indicated. Those shown in the table above are typical usage examples. It is possible to make an opposed Strength check, making a Fortitude saving throw and adding the Strength score to understand who wins in a weightlifting competition.

\subsubsection{Non-opposed, static tests}

If the Test is pitted against a \textit{static opponent}, i.e. not a creature with Characteristics and Skills, but a lock, something to push... then the test is performed comparing 3d6 + the Characteristic concerned + the Active Skill (TS/CM/CA) or Basic Skill (Disable Devices...) most suitable against the difficulty (\textbf{DC}) established by the Storyteller.

\medskip

\begin{center}
\includegraphics[width=0.6\linewidth]{immagini/Foster_Bible_Pictures.png}

\emph{Bible Pictures and What They Teach Us}
\end{center}


\subsection{Advantages and Disadvantages, Bonuses and Penalties'} \index{Advantages}\index{Bonuses}\index{Malus}\index{Penalties}\index{Disadvantages}\label{vantaggi}

\begin{changemargin}{0.3cm}{0.3cm}\begin{enfasi}{Audentes fortune iuvat (\emph{Fortune favors the bold}, Virgil) }\end{enfasi}\end{changemargin}


Depending on the circumstances there may be bonuses or penalties in the tests.

The modifier in \textbf{dynamic tests}\index{Dynamic tests} is to be used when the test is made by rolling 3d6, in this case bonuses or penalties (-1, +2...) or even roll more or less dice (+1d6, -2d6), until you roll no dice (with a 3d6 penalty)!.

If the accumulated penalties bring the test dice below zero, only the value of the Competence and Characteristics is counted.

This means \textbf{fixed value tests} \index{Fixed value tests} when the value does not depend on the dice roll (e.g. Defence), in this case the score increases/decreases by the indicated value.

Try to always stay between these bonus and penalty values, otherwise you can directly say that the test has succeeded or failed.

The player can request to carry out the test even if the result is certain.

\medskip

\textbf{Table: Modifiers, Advantages and Disadvantages}:\index[Tables]{Table of Modifiers, Advantages and Disadvantages}

\medskip

\begin{tabular}{llll}
\multirow{2}*{\textbf{Advantage / Disadvantage}} & \multicolumn{2}{c}{\textbf{Trials}}\\
\cmidrule(lr){2-3} & \textbf{Dynamic} & \textbf{Fixed} \\
\toprule
Light Bonus & +1& +1\\
Normal Bonus & +2 & +2\\
Strong bonus & +1d6 & +4\\
Very strong bonus & +2d6 & +8\\
Light Disadvantage & -1 & -1\\
Normal Disadvantage & -2 & -2\\
Strong Disadvantage & -1d6 & -4\\
Very strong disadvantage & -2d6 & -8\\
\end{tabular}

\begin{changemargin}{0.3cm}{0.3cm}\begin{narratore}
The bonuses and penalties in the 3d6 roll have more \emph{effect} than in the check made with the d20. Try to always stay within $\pm2$ and only in particular situations of effective and strong advantage or disadvantage apply bonuses or greater penalties.
\end{narratore}\end{changemargin}

\subsubsection{Time factor}\index{Time factor}\label{fattoretempo}

\textbf{If a character is not in difficulty or pressure}\index{Without time problems}\index{Getting a 10} in carrying out the test he can get a 10 (+ Characteristic + Skills + Feats..) , that is, consider that he rolled 10 on the dice. The action takes 10 rounds. \label{prendere10}

\textbf{If the character does not have pressing time limits}, i.e. he can dedicate at least 10 minutes to work on it (60 rounds) he can consider taking 14. That is, as if he had made the test and rolled 14 with 3d6. \label{prendere14}

\textbf{If time becomes a factor not to be considered}, i.e. the character has at least 1 hour to think and work and has no penalty or risk considering having rolled 18 (but there is no explosion of dice or success critical even if the total is 18).\label{prendere18}\\

If you want to take these values, ask the Narrator, he will tell you if based on the situation, urgency, danger of what surrounds you you can take the score. Breaking open a door in a dungeon asking for 10 requires extreme cold blood and recklessness. Taking 10/14/18 should not be given for knowledge tests.

\begin{changemargin}{0.3cm}{0.3cm}\begin{narratore}
I recommend everyone to read the excellent article by Lorenzo Bertini \href{https://dietroschermo.wordpress.com/2022/03/10/elogio-del-10-e-del-20}{Elogio del 10 e del 20} for a critical and intelligent examination of the success and failure of the tests.
\end{narratore}\end{changemargin}

\subsubsection{Helping Another}\label{aiutarealtro}

\index{Helping another} You can help a friend in a test by giving him support and suggestions. The helper must perform the \textbf{same check} with a bonus of +1d6, if he succeeds he gets no effects but gives a +1 to the companion's check. If he makes a critical success (successful check and at least two 6s rolled) then the bonus is +2.

Multiple characters can help the same character; bonuses of this type can be cumulated up to a bonus equal to a quarter of the difficulty to be beaten (e.g. +6 in the case of difficulty 25).\index{Helping another}

\textbf{In the case of tests based on Competencies, the person helping must have assigned at least one point in the Competence involved}.

The Narrator will evaluate the possibility of more than one character providing help by considering spaces, ways and times (it is not easy to help someone thread a thread through the eye of a needle).

\subsection{Evidence made by the Narrator}\label{provefattedalnarratore}

Avoid doing the tests yourself instead of the Players. Be descriptive but don't go and tell the Player that \emph{could} be proof of something. If it is necessary to carry out tests secretly from the player, do not roll any dice but add the Characteristic value and the Competence score or the value of the character's Saving Throw in question to 10 and compare the result with the difficulty of the test.

\subsection{To roll or not to roll dice}\label{tirarenontiraredadi}

Do not roll dice for tests that have no chance of failing, for tests that do not have or generate \textbf{problems} if they are failed or can be retried without problems. Have the dice rolled whenever the test could result in \textbf{spectacular}, \textbf{failure} or trigger further scenes. Make the player enjoy success or fear critical failure.

\subsubsection{Optional - Partial Success}\index{Partial Success}\index{Trial with Risk}\index{Optional - Partial Success}\hypertarget{Partial Success}{}\label{successoparziale}

A \textbf{Trial with Risk} is requested in particularly tense and urgent tests in which the final result is more important than the risk involved. This request must be made before rolling the dice.

If the test fails by 1 it can be considered successful even if with a slight problem, if it fails by 2 it brings with it a serious problem if it fails by 4 it is successful with a critical problem, if it fails by more than 4 the test is still not successful. Applied to skills such as Knowledge, you can decide to provide information that is incomplete or partially true and false, or even if it involves opening a lock you could break the lock pick in the lock!

\subsection{Group tests}\label{provedigruppo}\hypertarget{group tests}{}\index{Group tests}

There are situations in which the group must make a competence test but the result must be unique, in this case if at least half of the group succeeds in the test it is successful.

\subsection{Examples of competence tests}\label{esempiprovecompetenza}\hypertarget{examples of competence tests}{}\index{Examples of competence tests}

\textbf{Atypical tests}\index{Atypical tests}. The player is invited to find uses, solutions, approaches that go beyond the most obvious tests. Be creative and describe to the Storyteller the wonderful action you want to do and how to do it! Based on your description of the action, he will decide what to try and how difficult it may be.

\medskip

For \textbf{recognize a magic object}\index{Recognize a magic object} and its abilities, a test of \textbf{Arcana} is required at difficulty 20 to get general indications on the powers and areas of use, only with a result of at least 30 on the test, you can learn its details, magical bonuses and charges. \textbf{10 minutes}. With Arcana score 6 it costs 5 minutes, with 12 it costs 1 minute, with Arcana 18 it costs 1 Round.

\medskip

\textbf{Recognize a spell}\index{Recognize a spell} while it is being cast is a \textbf{Arcana} DC check equal to 10 + the spell's level. It costs \textbf{Reaction}. If done in conjunction with casting a Counterspell, it costs no Reaction.

\medskip

For \textbf{recognize a monster}, a particular creature, you make a Knowledge check. Check out the \hyperlink{recognizing monsters}{Recognizing Monsters} chapter in the Monstrorium (page \pageref{recognizing monsters}). Costs 1 Action.

\medskip

\textbf{Acrobatics}\index{Acrobatics} \emph{Armor penalties}

A successful Acrobatics check with DC 15 allows the character to halve the damage when falling from less than 30 feet (\textbf{Reaction}).

\medskip

The steps below 50cm are considered difficult terrain and those within 1.5m are doubly difficult terrain beyond that is climbing.

See also the \hyperlink{walls}{Table: Walls}, page. \pageref{walls}.

\textbf{Climbing/Climbing} \index{Climbing}\index{Climbing}\label{Arrampicarsi} \emph{Penalty due to Armour.}

\medskip

Using a rope\index{Climbing a short one}\index{Climbing a rope}, climbing or scrambling is equivalent to moving in \textbf{doubly difficult terrain}.

If the test fails, the Action is consumed without moving. If you fail on a critical failure you lose your grip and fall, you can make a Reflex saving throw at the same difficulty to grab onto something, if you also fail the saving throw you fall all the way to the bottom. The difficulties indicated add up.\\

\begin{tabularx}{0.45\textwidth}{Xl}
\textbf{Surface Example} & \textbf{DC}\\
\toprule
Movement only halved & -2d6\\
Slippery surface&+5\\
Rough wall with handholds, protruding bricks&+10\\
A tree, a knotless rope&+15\\
A wall with a few protruding bricks &+20\\
A wall with very few holds&+25\\
A smooth natural wall without grips&+30\\
You can lean against 2 opposite walls&-8\\
You can lean against 2 corner walls&-4\\
You can use a&-8 string\\
\end{tabularx}

\medskip

To \textbf{identify a potion or natural poison}\index{Identify Poison}\index{Herbalism} \index{Identify Potion} you need a check of \textbf{Herbalism} at DC 12 + rarity factor of the plant, or the saving throw it grants in case of poisons.

It costs 1 Action for every 10 DC. With a 6 in Herbalism the time is 1 Action every 15 DC, with 12 points it is 1 Action every 20 DC. If you fail the check with a critical failure you have come into contact with/ingested part of the potion and suffer its effects.

\medskip

\textbf{Intimidate}\index{Intimidate}. The character uses \textbf{2 Actions} and performs an opposed Will save check with a bonus given by Charisma.
If the saving throw fails, the opponent has -1 to attack rolls and -1 to defense until the end of his next round against the one who intimidated him. The opponent must have Intelligence equal to or greater than -3. The saving throw takes a +2 bonus or -2 disadvantage per size difference. On a critical success the modifier becomes -2.

If the person attempting the Intimidate check fails with a critical failure he suffers the same penalties as if he had been intimidated.

\medskip

\textbf{Tame an animal} is a \textbf{Handle Animal} check at DC 12+2*GS of the animal. 1 minute every 3 of DC. With 6 points the time is 1 minute every 6 DC, with 12 it is 1 minute every 10 DC. The creature must have Intelligence -3 or higher.

\medskip

\textbf{Stealth} \index{Hide} \emph{Penalty due to Armour.}\index{Move silently}

Stealth gathers the abilities to move silently, hide in shadows, go unseen, and all those actions that require you not to be seen or heard.

Trying to move silently does not cost Actions, it is \emph{included} in the Move Action used to move. However, the terrain is treated as difficult and if it already was, it becomes doubly difficult. Moving at full speed while trying not to make any noise imposes a 2d6 penalty on your Stealth check.

Although trying to hide is a seemingly simple activity, only those trained in Stealth have a greater chance of not being noticed.

Using \textbf{1 Action} you can try to hide from your opponents' sight. It is not possible to hide if the environment does not allow it, even though your test may be high you cannot hide if there is not something that can hide or conceal you. To hide behind a creature it must be at least 3 sizes larger than you (otherwise the creature only provides cover).

\medskip

\textbf{Swimming}\index{Swimming} \emph{Penalty due to Armour}

In calm water DC 10, in rough water DC 15, in very rough water DC 20, stormy water DC 25. The check is required to stay afloat or swim. Swimming in the water is considered \textbf{difficult terrain}.

\medskip

A possible test on the \textbf{Profession} is made with 3d6+Wisdom+half the level.

\medskip

\textbf{First Aid}\hypertarget{First Aid}{}\label{prontosoccorso}\index{First Aid}. A successful check (DC 15) recovers 1d4 Hit Points \textbf {after a fight} or grants a +2 to a Fortitude save to resist a poison. To be done within 1 Turn of the end of the fight. Cost \textbf{2 minutes}. With a score of 6 it costs 1 minute. With a score of 12 it costs 3 rounds, with a score of 18 it costs 1 round.

A successful check (base DC 12) reduces the damage by 1 from \hyperlink{bleeding}{\textbf{Bleeding}}. For each Bleeding value above 1 the difficulty increases by 2. Cost \textbf{2 Actions}. A 1 minute treatment grants 1 success, no check. Each successful check reduces the bleeding by one additional point.

A successful check (base DC 13) for \textbf{taking care of a patient for 8 hours} recovers double the Hit Points ((2*AC+Constitution+CM)*2 with a minimum of 4) and grants a new Fortitude saving throw to eradicate natural diseases or poisons already in progress.
If carried out during rest hours, the person administering the treatment will be fatigued.

\medskip

\textbf{Athletics (Jumping)}\index[Tables]{Jumping Table} \emph{Penalty due to Armour.} \textbf{1 Action}
%\begin{tabular}{lc|lc}
%\textbf{Long Jump} & \textbf{DC}&\textbf{High Jump} & \textbf{DC}\\
%\multicolumn{2}{c}{Length} &\multicolumn{2}{c}{Height} \\
%\toprule
%1.5 m & 5 & 0.02 m & 4\\
%3 m & 10 &0.5 m & 8\\
%5 m & 15 & 1 m & 12\\
%7 m & 20 & 1.5 m & 16\\
%+1.5 m & +5 &+0.5 m & +4\\
%\end{tabular}

The \textbf{long jump distance} is equal to 30cm per result obtained in the test, rounding to the nearest integer. E.g. if in the jumping test I score 11, the jump will be 30cm*11=330cm=3 meters long, with a 16 in the test it is 30cm*16=480cm=5m.

The \textbf{high jump distance} is equal to 10cm per result obtained in the test.

In a \textbf{long jump} the highest point of the jump is equal to 1/3 of the length jumped. If you do a 3 meter long jump mid-jump you are 1 meter high.

If you don't have at least 3 meters of run-up you skip half. In the long jump you jump at the maximum of your movement and in the high one half.

Going down or going up within 50 cm is difficult terrain, between 50 and 150 cm it is doubly difficult terrain, beyond that it is falling or climbing. Falling damage (page \pageref{falls}): 1d6 damage every 3 meters of fall. Acrobatics DC 15 to halve damage (for falls within 9m).\index{Descent and Ascend}

\medskip

\textbf{Survival}\index{Survival}

\smallskip

\textbf{Chasing a creature}:

\begin{tabular}{ll}
Basic difficulty & DC 10\\
\toprule
If the ground is very soft& DC +5\\
If the ground is soft& DC +10\\
If the ground is stable& DC +15\\
If the ground is hard& DC +20\\
Every 3 creatures chased & DC -1\\
Depending on size & DC $\pm4$\\
Every 24 hours passed / Low visibility&DC +2\\
Every hour of rain&DC +4\\
%Low visibility&DC +2\\
Try to hide your tracks&DC +5\\
\end{tabular}\\

Survival can be used instead of \textbf{Disable Devices} with a -2d6 to disable traps or locks 1 Action per DC.

For every four points obtained in the Survival test over 13 the character is able to \textbf{procure food} for himself and another person as long as he is in an environment capable of supporting life.\\

The proof of \textbf{Value}\index{Value} is based on the rarity of the item, DC 12 + 2 common, 4 uncommon, 8 rare, 12 very rare, 16 legendary. \textbf{3 Shares}. With a score of 6 it costs 2 Actions, with a 12 it costs 1 Action.

\subsection{Languages}\index{Languages}\hypertarget{Languages}{}\label{linguaggi}

In Yeru each culture is the guardian of its own language. Any character with at least Intelligence -2 speaks the language of her culture, on 0 she writes it. For every point equal to or greater than 2, speak and write another language which will be chosen when creating the character.
A member of a race may very well have a first language other than that of their own race if the background justifies it (e.g. a dwarf who grew up among a goblin tribe).

For every point spent in the Language Proficiency he speaks and writes another language.

Some languages ​​marked with \textbf{*} can only be spoken by creatures belonging to that species or cultural group.

\end{multicols}

\textbf{Language Table}\index[Tables]{Language Table}

\medskip

{\small \begin{tabular}{lll|llll}
\textbf{Cultural field}& \textbf{Spoken} & \textbf{Written}&\textbf{Cultural field}& \textbf{Spoken} & \textbf{Written}\\
\toprule
%Human & Common& Common& Dwarven& Dwarven& Dwarven\\
%Elven& Elven & Elven & Gnomic& Gnomic & Gnomic \\
%Gnoll & Gnoll & Goblinoid & Giants& Giant & Giant\\
%Orc & Orc & Orc & Sentient Sea Creatures & Aquan & Elven\\
Sentient Sea Creatures & Aquan & Elven & Sentient Birds & Ictun & Elven\\
Woodland Dwellers*& Silvanus& Silvanus & Druidic* & Druidic & -\\
Celestial & Celestial & Celestial & Infernal & Infernal & Infernal\\
Abyssal & Abyssal& Abyssal& Dragons& Draconic & Draconic\\
Fire Elementals* & Ignan & - & Earth Elementals*& Tremun &-\\
Water Elementals* & Aquan & - & Air Elementals*& Ictun &-\\
Undead & Exspiran & - & Underground & Depths & Depths\\
%Sign Language*& Signs & Signs&&&\\
\end{tabular}}

\medskip

\textbf{Telepathy}\index{Telepathy} is a means of speaking with any creature that has language and Intelligence at least -3. There is no constraint of language, telepathy acts as a universal translator.

\begin{multicols}{2}

\medskip

\begin{changemargin}{0.3cm}{0.3cm}\begin{tcolorbox}[title = Tests Tests and Tests!]
To be cynical, a role-playing game is all about testing, whether it's to be able to make a jump, to hit someone, to avoid a trap or a spell...!
You have to be smarter and smarter. Trials can often be avoided or faced with advantage. Play with wit, use your imagination, be creative! The Narrator will only be happy and you will be satisfied!!!
\end{tcolorbox}\end{changemargin}

\begin{center}
\includegraphics[width=0.8\linewidth]{immagini/Pieter_Bruegel_the_Elder-The_Tower_of_Babel.png}

\emph{The Tower of Babel, Pieter Bruegel the Elder.}
\end{center}

{\small 

\begin{changemargin}{0.3cm}{0.3cm}\begin{narratore}
How to carry out the Tests, how to manage the results, like the matches, determines the type of game. Listen to the player and sense his enthusiasm, try to understand the final intentions, the objective and the purpose of the test.

An involved and participatory player spreads his enthusiasm to the other players too! Listen carefully to the proposals they make to you even if they seem \emph{not very sensible} or \emph{simply crazy}, nothing stops you from warning about the potential danger of the choices, don't dampen your enthusiasm!. 

If you don't find a rule suitable for the situation, use \textbf{common sense}, similarity to other actions, get involved in describing the facts, be theatrical when necessary! The spirit of the group will certainly benefit!

If the players then simply tell you I'll do a \emph{Awareness test} or \emph{convince the guard}, follow them in their intentions, let them do the simple test, but at the same time do everything to involve them further.

\textbf{There is no one rule for everything! Fun and common sense must never be missing}!

\end{narratore}\end{changemargin}}

\end{multicols}

\pagebreak

\section{Social Combat}\index{Social Combat}

\begin{changemargin}{0.3cm}{0.3cm}\begin{enfasi}{

To formulate dialectics in a clear way, it is necessary to consider it, without paying attention to objective truth (which is the object of logic), simply as the art of obtaining reason, which will certainly be much easier if one is objectively right. (Arthur Schopenhauer)

}\end{enfasi}\end{changemargin}\medskip


\begin{multicols}{2}

By Social Combat we mean the attempt by the characters to convince, force or deceive the NPCs or in any case creatures forced by the Narrator to do or say things they do not want.

It may happen that players try to bribe a guard, to obtain information in a diplomatic or intimidating way, to obtain a higher pay, to deceive a merchant or more simply whenever the \emph{confrontation} or \emph{confrontation } it is not by weapons but by words.

Although social combat can concern a multitude of situations, what all the tests have in common is the method with which one wants to obtain the final result, not through weapons but by trying to \emph{convince} the opponent.

In these situations two distinct approaches can be followed, on the one hand the Narrator evaluates the result based on what the players say, on the other this system creates the rules as if it were a fight to establish who wins in the final test.

Each Narrator chooses the approach he prefers, let's say that based on experience with the system and the role-playing game in general he might prefer one system or the other. For a neutral approach, using rules may be more suitable.

Depending on whether the player uses more or less coercive methods, the opponent will resist accordingly.
The player will make an opposed Intimidate, Diplomacy or Deception check and the opponent will attempt to resist with a Will saving throw with a Strength or Charisma bonus.
If you must resist an Intimidate-based compulsion, counter with a Will saving throw with a Strength bonus.

Based on the NPC's level, the Narrator will establish how many consecutive successes are necessary to convince him. Generally speaking, one success is needed for every 3 levels of the NPC.\index{Success needed to convince} The number of successes can be modified based on the beliefs, promises, pacts, interpersonal relationships that the opponent has regarding the situation.

If you win all the tests you will win \emph{fight} and you will get the information or what you requested. In the event of a critical success, i.e. in addition to passing the test at least two 6's have been rolled, two successes will be counted.

If the test fails, it can be tried again with a -1 penalty if the consequences of the failure do not lead to a subsequent scene.

If the failure is critical, i.e. in addition to failing the test, at least two 1s or one 1 and two 2s were rolled, then not only is the test failed but it will not be possible to make further attempts and the opponent will become even less friendly. The Narrator will most likely decide the evolution of the situation based on the original request and scene.

In case of intimidation the player's target will most likely become hostile, in case of Deception it is possible that, feeling deceived, he will lie or say nothing. In the case of diplomacy, silence or a polite denial is more likely.

The Storyteller must use this evidence, positive or negative, to evolve the scene and enrich the adventure. Information that was not intended to be given could be incomplete or partial, in any case it will allow the players and the Narrator to conduct the game better.
The Narrator must not think that giving the information is a problem, in the end the players have earned it.

\end{multicols}

\vfill

\begin{center}
\includegraphics[width=0.45\linewidth]{immagini/Greuter_Socrates.png}

\emph{Socrates and His Students. Johann Friedrich Greuter, 17th century.}
\end{center}



\pagebreak

\section{Armed Combat}\index{Armed Combat}

\begin{changemargin}{0.3cm}{0.3cm}\begin{enfasi}{
Si vis pacem, para bellum (\emph{If you want peace, prepare for war}, Vegetius, book III, Epitoma rei militaris)
\medskip

It doesn't matter how you fall, but whether and how you get up (anonymous)

\medskip
I'm not a hero. No and I never will be. I'm just a bad guy who gets paid to beat up worse guys than him. (Deadpool)

\medskip

An eye for an eye ... and the world goes blind (Mahatma Gandhi, Editor's note: His traits abhorred violence!)} \end{enfasi}\end{changemargin}\medskip

\begin{multicols}{2}

\lettrine[lines=2, lhang=0.33, loversize=0.25, findent=1.5em]{I}{l} combat is among the main phases of an adventure and is when the brave and courageous show off their mastery with weapons or magic.

\bigskip

The fight is divided into 2 phases:\index{Fight}
\begin{itemize}
\item verification of the initiative
\item resolution of actions (movement, attack, various actions..)
\end{itemize}

\begin{center}
\includegraphics[width=0.7\linewidth]{immagini/Achildbookofwarriors.png}

\emph{A child's book of warriors (1907), William Canton}
\end{center}

\subsection{The Initiative}\index{Initiative}\label{iniziativa}

Initiative is a check (3d6) of Dexterity or Intelligence and related Feats you may have.

The player chooses the Characteristic he prefers. If Dexterity is chosen, the reflexes will determine the character's reaction, while Intelligence will guide the ability to grasp the opponent's tactics and anticipate them.

Whoever has the highest initiative among players and enemies goes first and then the others act in decreasing order, declaring Actions and executing them. In the event of an equal-scoring Initiative, the player with the highest Characteristic score acts first, otherwise the battle will take place at the same time. The initiative is valid for the entire battle and is withdrawn when the opponent changes.\index{Equal initiative}

\begin{changemargin}{0.3cm}{0.3cm}\begin{narratore} %box narrator
Try to make the combat flow naturally. Don't interrupt the flow of actions, but rather involve the players (and enemies) in the following actions by describing their effects. I recommend reading the article \href{https://theangrygm.com/manage-combat-like-a-dolphin/}{How to Manage Combat Like a Dolphin} to understand the method in detail.
\end{narratore}\end{changemargin}


\textbf{The Golden Rules do not apply even in the Initiative Test.}\index{Initiative and Golden Rules}

\subsubsection{Action Resolution}\index{Action Resolution}\label{risoluzionedelleazioni}

\begin{changemargin}{0.3cm}{0.3cm}\begin{enfasi}{
It's not true that we have little time: the truth is that we waste a lot of it. (Lucius Anneus Seneca)
}
\end{enfasi}\end{changemargin}\medskip

From fastest to slowest is the resolution of Actions.

The Narrator will ask the fastest, the one with the highest initiative, to declare his Actions and act, he will then continue to ask and make the other players and enemies act.

In this way the choice of action takes place when it is the player's round who will also be able to act based on the Actions and resolutions that have already taken place.


\begin{center}
\includegraphics[width=0.9\linewidth]{immagini/Arthur-Pyle_Two_Knights.png}
\emph{Howard Pyle, from the 1903 edition of The Story of King Arthur and His Knights}
\end{center}


\subsubsection{Time (Rounds, Minutes and Turns)}\index{Round}\label{iltempo}

\begin{changemargin}{0.3cm}{0.3cm}\begin{enfasi}{Hesitation is the death of advantage (Magic, V.E. Schwab)} \end{enfasi}\end{changemargin}\medskip

A \textbf{round} lasts about 10 seconds, which is enough time to act, run, talk... fight. A Minute is therefore 6 rounds, and a Turn lasts 10 Minutes (or 60 rounds).\index{Round and Turn, duration}

Rounds are used in fight scenes or where tension must remain constantly high and each Action corresponds to an evolution of the situation.

\subsubsection{Item and Ability reactivation time}\index{Item and Ability activation time}\label{temporiattivazioneoggetti}\index{Magic item recharge}

Unless otherwise specified, an object or Feat that requires a certain number of uses per day \emph{e.g. once a day} it recharges completely at dawn after use.


%\begin{center}
% \includegraphics[width=0.6\linewidth]{immagini/hjford-fight.png}

% \emph{\\Fairy book - Fairytale illustration, Henry Justice Ford}
%\end{center}

\subsection{Shares in the Round}\index{Shares in the Round}\index{Share}\label{azioninelround}

%\begin{changemargin}{0.3cm}{0.3cm}\begin{enfasi}{
%A true man of action immediately sees so many things to do before him that he will never lack work and will succeed. (Fyodor Dostoevsky)
%} \end{enfasi}\end{changemargin}\medskip

A character can perform up to 3 Actions, 1 Immediate Action, and 1 Reaction Action per Round.

If \textbf{in the round} of the \textbf{initiative roll} the character in the same test rolled a \textbf{critical success}, twice 6, he will be able to use one more Reaction or Immediate Action, if he has rolled at least \textbf{two critical successes} his great reactivity allows him to perform one more Action.\index{Critical in the Initiative}. If instead he rolls a \textbf{critical failure} the extreme slowness will prevent him from using the Reaction or Immediate Action, if he rolls three 1s he will take one less Action.\index{Critical failure in the initiative}\index{Initiative , critical}

Actions can be performed in any order you prefer.

The table below shows the main Actions that a character can do, they are guidelines to follow. In the chapter dedicated to combat and examples of use of Feats, other Actions and their relative costs in Actions are listed.

\textbf{An Action cannot be interrupted by another Action, but can be followed by a Reaction Action or an Immediate Action}. \index{Stop Shares}\index{Shares, Stop}

If a character wants to make multiple attacks by moving around the battlefield, he can use an Action to make an attack, use a Move Action to move up to all of his available movement, and use a final Attack Action to make a last single attack , this second attack counts as a multiple attack with the related penalties.

It is possible \textbf{delay} one or more Actions\index{Delay Actions} to wait for the scenes to unfold. The character who delays his Action acts first among the subjects who act in that initiative value, in subsequent rounds he will continue to act in the new initiative order. In this way the player voluntarily delays his initiative to fit into the initiative order in another place.

A player who declares that he is waiting for a certain situation to be able to act is equivalent to carrying out one or more \textbf{Prepared Actions}\index{Prepared Actions}. In this case the character (or enemy) acts \textbf{after} the triggering Action with his Actions but remains in his initiative order at the end of the round.

If the character has already performed all the Actions then he will be able to act in the round only with an Immediate Action and outside of his initiative only through a Reaction, if available. The Reaction Action always activates after the Triggering Action.

\bigskip

\textbf{Table: Actions per Round}\index[Tables]{Table of Actions per Round}

\medskip


%\begin{xltabular}{0.45\textwidth}{Xc}
\begin{tabularx}{\columnwidth}{Lc}
\textbf{What we do} & \textbf{Shares}\\
\toprule
Perform an attack & 1\\
Perform two attacks & 2\\
Perform more than two attacks & 3\\
Cast a Spell* & 2\\
Perform a Move Action* & 1\\
Shoot & 1\\
Standing up from prone & 2\\
Help someone & 2\\
Exchanging a dialogue with someone* & 3\\
{\small Exchanging a few words with someone*} & 0\\
Look for something in the backpack & 2\\
Take something from your belt or ready & 1\\
Using a hand held object & 1\\
Drinking a potion held on the belt& 1\\
Draw or Sheathe Weapon & 1\\
Using a magic item & 2\\
Perform test on a skill* & 1\\
Recognize a creature & 1\\
Breaking down a door with shoulder/kicking & 1\\
Force door with crowbar & 2\\
Hiding & 1\\
Concentrate on a Spell & 1\\
Mounting or dismounting & 1\\
Action \textbf{I}immediate - Action \textbf{R}eaction & I - R\\
Drinking a hand-held potion & I\\
Throwing a hand-held object & R\\
Throw yourself prone & R\\
Recognize a Spell & R\\
\end{tabularx}

\medskip

\textbf{Attack Action} means both the use of melee weapons and the use of thrown or shooting weapons such as bows, crossbows or throwing daggers. In the case of thrown weapons, each throw/throw counts as an attack.

The character who performs an Attack Action and Casts a Spell in the same round is considered Distracted or must perform a Magic Test to cast the spell.

\textbf{Move Action*}: a Move Action is an Action dedicated to moving. You can move up to your full movement (30 feet for humans, 20 feet for dwarves..) per Action used.

\textbf{Casting a Spell*}: usually 2 Actions are required. The number of Actions required is indicated in the spell description. In the Magic chapter the \hyperlink{piumagieround}{rules} are specified (page \pageref{piumagieround}) for casting multiple spells in the round.

\textbf{Exchange a dialogue with someone*}: A dialogue can last a few seconds if not minutes. The Storyteller will evaluate how long this lasts.

\textbf{Exchanging a few words with someone*}: As long as there are very few words or a look does not consume Actions, if this becomes more complex then use Actions. The objective is not to interrupt the flow of the Actions with a dense dialogue but still allow interaction between the characters.

\textbf{Perform a test on a skill*}: if they yield a fraction of the round they cost 1 Action, otherwise 2 or more. Check the costs reported in the \hyperlink{examples of skills tests}{Examples of skills tests}.

An Action of \textbf{Reaction (R)} \index{Reaction Action}can be performed freely even outside your own round. This Action is usually due to particular Feats or situations. Unless otherwise indicated, a Reaction Action occurs immediately after the cause that triggers it.

An Action \textbf{Immediate (I)} \index{Immediate Action} can be performed freely in your round, before or after your Action. An Immediate Action is usually granted by particular Feats.

It is possible, unless specifically described in the Feat, to perform only one Immediate Action and one Reaction Action per round.

\medskip

This \textbf{list is not complete}, take it as a guideline to establish the weight of the characters' decisions and actions. An Action lasts approximately 3 seconds.

The \textbf{order} with which the Actions are executed is not important except for logical and physical correlation. The Move Action can be between other Actions (move, attack/spells/other action, move).

A character could attack, move and attack again, this second attack would have the penalties described in multiple attacks.
\smallskip


\begin{center}
\includegraphics[width=0.8\linewidth]{immagini/Perseus_Fighting_Phineus_and_his_Companions.png}

\emph{Luca Giordano: Perseus turning Phineas and his Followers to Stone}
\end{center}

\subsection{Optional - Initiative Variant}\index{Optional - Initiative Variant}\hypertarget{initiative Variant}{}\label{varianteiniziativa}

This variant of the initiative aims to stimulate the diversification of actions based on the situations that are faced from time to time.

Initiative is a value that is calculated round by round based on the Actions performed.

At the beginning of the round there is a declaration of the Actions that you want to undertake, starting from those who have the lowest Dexterity or Intelligence.

Up to 10 Action Points (AP) can be used per Round. \textbf{Whoever uses the fewest Action Points goes first}

If AP used is equal, the person with the highest Dexterity or Intelligence acts first, if the opponent is equal, in the case of teammates we reach an agreement.

The table below shows some typical actions and related costs.

\end{multicols}

\begin{tabularx}{0.95\textwidth}{Xl|Xl}
\textbf{What is done} & \textbf{PA}&\textbf{What is done} & \textbf{PA}\\
\toprule
Attacking with Light Weapon / Bare Hands & 3 &Attacking with Medium / Missile Weapon & 4\\
Attack with Large Weapon & 5 & Use Magic Item & 6\\
Move within 2m-3m (Move 6m-9m) & 1 &Sprint within 4m-6m (Move 6m-9m) & 1\\
Rise from prone & 4& Mount or dismount & 4\\
Take something in your backpack & 8& Use a hand-held object & 2\\
Drinking a potion held on your belt & 4&Concentrating on a spell & 3\\
Draw / Sheathe the weapon & 3&Fall prone & R\\
Breaking down a door with your shoulder/kicking & 5&Forcing door with crowbar & 6\\
Drinking a hand-held potion& 1&Throwing a hand-held object& R\\
Take something from belt or ready & 3 &Recognize a Spell& R\\

\end{tabularx}

\begin{multicols}{2}

- Skills, providing Help, cost 3 Action Points per Action indicated.

- Exchanging dialogue can be free or cost AP depending on how much you talk.

- Spells cost 3 AP per Component used (V,S,M), except when the casting time is longer than 1 round.

- The Speed ​​spell grants 4 more Action Points. These APs are not counted for the verification of the initiative of those who act first.

- The Slow spell subtracts 4 Action Points from the available ones. These PAs are added to verify the initiative of who acts first. E.g. I use 4 Action Points, out of a maximum of 6 (10-4 for Slowness), to verify the initiative I used 4+4 points.

Moving around, deciding which weapon or spell to use determine not only when you act but the tactic you want to pursue. The collaboration between the characters becomes fundamental and does not slow down the flow of actions.

\end{multicols}

\subsection{Movement}\index{Movement}\label{movimento}


\begin{changemargin}{0.3cm}{0.3cm}\begin{enfasi}{A slower piece of furniture cannot be reached by a faster one; since the one that follows must arrive at the point that the one that followed occupied and where it is no longer (when the second one arrives); in this way the former always maintains an advantage over the latter. (Zeno's Paradox)}
\end{enfasi}\end{changemargin}

\begin{multicols}{2}

The movement of a character is given by his size and race and by what he carries, by weights, encumbrances but also spells and magical objects.

The Movement written in the character's race is the indication of how many meters per Action (of Movement) the character can make.

A creature or character could also decide to move faster than usual or by running (Dash Action).

The Dash Action is a particular Movement Action, it consists of running for that Action.
If you perform an Action of \textbf{Sprint} \index{Sprint}, the meters traveled are doubled (2x9 meters for a human), for a dwarf (Movement 6m) it means covering 12 metres, in one Action.
It is also possible to do multiple Dash Actions, up to 3 in a round, i.e. run your movement 6 times.

The character who takes a Dash Action the casting of spells.

It is not possible to move even 1 meter if you do not spend Movement Actions.

These clarifications make sense and must be used when it comes to fighting and the location on the territory, map, is fundamental. During normal movements, while riding or walking freely without danger, the normal management of clockwise movement is used.

When we talk about \textbf{\emph{square}} \index{Square}to indicate a distance or an influence we mean a map square of 1 meter x 1 meter.

In the case of diagonal movement\index{Diagonal movement}\index{Moving to the side} a distance of 1.5 meters per square is counted, in the case of rounding on the last square it is done by default, i.e. it goes back to 'last crossed.

\textbf{If you move on \emph{difficult} terrain, you travel half the available movement so a human covers 4 meters per Move Action (each square crossed counts as two).}

The dimensions and relative spaces occupied by creatures of different \hyperlink{sizeanddimensions}{size} are indicated in the Monstrorium (page \pageref{sizeanddimensions}).


\subsection{Distance}\index{Distance}\label{distanza}

By \textbf{Touch distance} \index{Touch distance} \index{Touch} means a distance that allows you to touch the opponent, therefore no more than one meter for medium-sized creatures without long weapons or with flow rate. Touch distance is melee distance if long weapons are not used.

\textbf{Melee distance} \index{Melee Distance} \index{Melee} means a distance that allows melee combat (within 1 meter around the character, or within 2 meters in the case of a weapon long). In monsters this distance is indicated by the range, for thrown weapons it is called range.

If not indicated on the opponent's card, the \textbf{capacity} is equal to half the occupied space rounded up. A hill giant, enormous in size (3x3 squares on the map), has 2 squares of reach, meaning it hits creatures within 2 squares/meters of it.\index{Size and melee distance}\index{Reach}

\begin{changemargin}{0.3cm}{0.3cm}\begin{tcolorbox}[title = Combat Range Examples]
E.g. for a creature armed with a spear, the reach is 2 or the melee distance is 2 meters because the weapon is long. For a gnome armed with a hammer, or with bare hands, the melee range is 3 feet. Range indicates how far you can hit in melee.
\end{tcolorbox}\end{changemargin}

At melee range, a Medium creature can have a maximum of 8 Medium creatures.

\end{multicols}

\vfill

\begin{center}
\includegraphics[width=0.5\linewidth]{immagini/camminata.png}
\end{center}

\pagebreak

\subsection{Life and Death}\index{Dying}\label{morire}

\begin{changemargin}{0.3cm}{0.3cm}\begin{enfasi}{Whoever does not know death, does not know life. (Grand Hotel, 1932 film)

\medskip

The deserving Game Master never willfully kills player characters. He presents opportunities for hasty and careless players to do everything themselves. (Gary Gygax)}\end{enfasi}\end{changemargin}\medskip

\begin{multicols}{2}

Weapon damage is calculated as the sum of the weapon's die, Strength (or Dexterity if indicated by Feat) whether positive or negative, bonuses given by the Weapons List, bonuses given by Feats, bonuses given by the weapon and circumstantial bonuses. \index{How to calculate weapon damage}\index{Weapon damage}

When a character reaches 0 (zero) Hit Points he is considered unconscious\index{Unconscious}, i.e. Helpless and Unable to do anything. A magical cure (Spell, Potion...) will bring him conscious and his Hit Points healed. A First Aid check, 3 Actions, at DC 12 will bring him to 1 hit point. After an hour, if nothing has happened to change the situation, the character can make a Fortitude saving throw at DC 15, if he succeeds he returns to 1 hit point, if he fails he goes to -1 and becomes dying.\index{Zero HP, recovery from}

A dying character has negative Hit Points (-1 or less) and is unconscious and \hyperlink{dying}{near death}. He will continue to lose 1 Hit Point per round, the value will not reach double the Constitution +10 and the character will die if he is not healed.

A cure spell (spell or potion) of any level will bring him to 1 hit point, subsequent cures will work normally.

A test of \hyperlink{first aid}{First Aid}, 3 Actions, at difficulty 12 plus the value of negative Hit Points will bring the character to 0 Hit Points, i.e. unconscious. Each subsequent time the character drops below 0 Hit Points, the difficulty of the First Aid check increases by 2.

\begin{changemargin}{0.3cm}{0.3cm}\begin{tcolorbox}[title = Tups is dying]
E.g. Tups is seriously injured and currently has -6 Hit Points, Jade decides to try to heal him (after moving him to a safer place). Jade attempts a First Aid check (3 Actions) to stabilize her partner, her difficulty in the test is 12 + 6 or she must pass with First Aid DC 18 to bring him back to 0 Hit Points (unconscious)

A subsequent First Aid check at DC 12 (3 Actions) can bring him to 1 hit point and a magical cure will heal him by the declared amount.
\end{tcolorbox}\end{changemargin}

A dying character who suffers further damage, such as enemies attacking the body or spells aimed at him or the area, continues to take Hit Points at the risk of dying.

\textbf{Conditions} \index{Mental Conditions}\textbf{mental type} such as Fascinated, Confused but not Dominated, end when the character becomes dying.

\begin{center}
\includegraphics[width=0.8\linewidth]{immagini/Nuremberg_chronicles.png}

\emph{The Dance of Death (1493) by Michael Wolgemut, Nuremberg Chronicle of Hartmann Schedel}
\end{center}

If an attack or spell brings the character directly to -(10+CON*2), the character dies\index{Immediate Death}\index{Massive Damage} without the possibility of being healed.

When a character returns to positive Hit Points after having gone negative, he loses half of his remaining Magic Points with a reduction of at least 10 Magic Points and becomes further \hyperlink{Fatigued}{\textbf{Fatigued}} (page \pageref{fatigued}).

When a character reaches negative Hit Points equal to 10+double his Constitution score he is \hyperlink{dead}{\textbf{dead}} [-(10+(CON*2))].

A character with nonlethal hit points of 0 or less faints until normal hit points have returned to 1.

\begin{changemargin}{0.3cm}{0.3cm}\begin{tcolorbox}[title = The character's death]
He tries to understand why he died, what the causes are, the mistakes he made. What are the choices that brought him there. Every character who dies is a personal wound but also an experience and awareness. Treasure it, both you and the whole group. If something hasn't worked, try to understand it together, without accusing or blaming each other but with the awareness that everyone can improve.
\end{tcolorbox}\end{changemargin}

E.g. If he has Constitution 2 he will die at -[10+4]=-14 Hit Points, if he has Constitution 0 he will die at -10 Hit Points, if he has Constitution -2 he will die at -[10-4]=-6 Hit Points. In case of Constitution values ​​equal to or lower than -3 the character dies at -5 Hit Points.

If a character's non-lethal damage reaches negative hit points equal to 20+4*Constitution, the character is dead.\hypertarget{temporary hit pointsdeath}{}

\begin{changemargin}{0.3cm}{0.3cm}\begin{narratore}
Describe the character's fall with pathos and emotion, make clear the suffering experienced. Emphasize the fall to the ground, the gushing blood, the gasps. Be theatrical.
If you are dealing with easily impressionable players then it is better to reduce the \emph{gore}.
\end{narratore}\end{changemargin}

A dead character cannot benefit from normal or magical healing, and cannot be brought back to life by a spell. Only a Patron has enough power to return the soul to the body and bring the creature back to life. The animate dead spell can reanimate a body, but as undead.

\subsubsection{Optional - Recover from 0 hit points} \index{Recovery} \index{Unconscious}\index{Optional - Recover from 0 hit points}

\begin{changemargin}{0.3cm}{0.3cm}\begin{enfasi}{
Reports of my death are greatly exaggerated. (Samuel Clemens)
}\end{enfasi}\end{changemargin}\medskip

\textbf{If you want a less lethal system you can apply this optional rule.}

Every round after having gone to 0 hit points or less, therefore fainting or dying, the character must make a Fortitude saving throw at difficulty 15, if he succeeds he regains consciousness and goes to 1 hit point.

If he fails the saving throw he can make another one at DC +1 compared to the previous one the following round. When the difficulty reaches 18 (i.e. 3 failed tests in a row) the character dies.

As soon as the check succeeds (within 3 failures) the character returns to 1 hit point and is fatigued. Each time he drops below 0 Hit Points, the initial difficulty (15) increases by 3.

\subsubsection{Feature point recovery}\index{Feature point recovery}\label{recuperopunticcaratteristica}

Any lost Ability points are regained at a rate of 1 point per day, unless designated as a permanent loss.

\subsubsection{Natural hit point recovery}\index{Natural hit point recovery}\label{recuperopuntiferitanaturale}

Resting 8 hours, in 24 hours, recovers the Constitution score + 2x Weapon Proficiency + Magical Proficiency per day in Hit Points, with a minimum of 1.

\subsubsection{Non-lethal hit point recovery}\index{Non-lethal hit point recovery}\index{Non-lethal hit point recovery}\label{recuperopuntiferitanonletali}\hypertarget{non-lethal hit point recovery}{}

Every hour you recover your Constitution value, with a minimum of 1 Hit Point.

\subsubsection{Maximum Hit Points}\index{Maximum Hit Point Recovery}\index{Maximum Hit Points}\label{puntiferitamassimi}

Unless otherwise indicated, every time the character suffers damage that lowers the maximum hit points, in addition to lowering them, he must also subtract them from the current hit points. A character when healed cannot exceed its current Maximum Hit Points.

Every 8 hours of rest, in 24 hours, you recover 1d4 + Constitution in Maximum Hit Points, with a minimum of 1.

\end{multicols}

\vfill

\begin{center}

%\includegraphics[width=0.7\linewidth]{immagini/caravaggioSalomeLondon.png}

%\emph{Salome with the head of the Baptist is a painting by Caravaggio made in oil on canvas (91x106 cm) between 1607 and 1610.\\ It is preserved in the National Gallery in London.}
\includegraphics[width=0.43\linewidth]{immagini/giantdeath.png}

\emph{Henry Justice Ford}

\end{center}

\pagebreak

\subsection{Attack Roll and Defense}\index{Attack Roll}\index{Defense}\label{tiropercolpireedifesa}

\begin{changemargin}{0.3cm}{0.3cm}\begin{enfasi}{Always apply the right force, never too much, never too little. (Kano Jigoro)}\end{enfasi}\end{changemargin}\medskip

\begin{multicols}{2}

The \textbf{Attack Roll} is a check against the opponent's Defense.

If the attacker uses:

\begin{itemize}
\item \textbf{Melee or Contact Weapons}: the attacker must make a \textbf{Attack Roll (TC)}= 3d6 + Weapon Proficiency + Strength + any bonuses given by the Weapons List + Feats + magical bonuses of the weapon and circumstantial factors (environment, curses..)

\item
\textbf{Range Weapons}: the attacker must make an Attack Roll (TC) = 3d6 + Weapon Proficiency + Dexterity + any bonuses given by the Weapons List + Feats + magical bonuses of the weapon and circumstantial factors (environment , curses..). Applies to bows, crossbows, drawn daggers, javelins...

\item
\textbf{Spell}: the attacker must make an Attack Roll (TC) = 3d6 + Weapon Proficiency + Spell ability modifier + any Feats and circumstantial modifiers.
\end{itemize}

Whoever defends himself has a \textbf{Defense} equal to: 10 + Dexterity + Shield + Armor + magic bonuses + Feats and circumstantial bonuses, for monsters the Defense value is already indicated.

The player can decide to give up part of the bonus given by Weapon Proficiency to have a better Defense score. These points will not be available in the next attack (see Other actions and situations).

\begin{center}
\includegraphics[width=0.9\linewidth]{immagini/Coypel_Charles-Antoine_-_Fury_of_Achilles_-_1737.png}
\emph{Charles Antoine Coypel - Fury of Roland - 1737}
\end{center}

\subsection{Defense and Attack}\index{Defence}\index{Attack}\label{difesaeattacco}

\begin{changemargin}{0.3cm}{0.3cm}\begin{enfasi}{The defense is always legitimate (anonymous victim)}\end{enfasi}\end{changemargin}\medskip

Each attack roll compares Defense.

If the \textbf{Attack Roll} is equal to or greater than the Defense value the opponent has been hit and the damage of the wound will be established, given by the weapon die + Strength score and other factors such as magic bonuses, List of 'Weapons and Feats. 

If the Attack Roll (TC) is lower than the Defense then the opponent will have parried, dodged, avoided... The choice is left to the player (or Narrator), once the attack is avoided no wounds will be suffered.

There are situations that can benefit the Defense such as covers, hiding places, trenches, doors, companions whose size is much larger than one's own. Consult the paragraphs relating to \hyperlink{covers}{Hides and Covers} to understand the advantage they can give.

There are occasions when it is not important to penetrate the defense and hurt the opponent but simply touch him.

Other times the opponent is surprised and cannot defend himself completely.

If it is \textbf{it is enough to touch the opponent} the attack roll has a +1d6 bonus since it is not necessary to deliver the blow but only to touch it.\index{Touch the opponent}. In the manual it is called Touch Attachment.\index{Touch Attachment}\label{attaccoatocco}\hypertarget{Touch Attachment}{}

If \textbf{the opponent is surprised} or does not expect the attack, the Defense and Reflex saving throw will have a -4 penalty. This is the value of \textbf{Surprise Defense}.

\textbf{The Golden Rules also apply for the Attack Roll. The d6s explode if you roll a 6 on the die, make 1 bad luck (worth zero) and rely on luck (i.e. remove 4 points between Weapon Proficiency and Strength or Dexterity to add 1d6 to the Attack Roll, not from the bonuses given by the List of Weapons or Skills or magic items).}

If modifiers and circumstances cause the damage dealt to be zero or negative, you will still deal 1 damage.
This rule applies to weapon damage modifiers which cannot bring the total damage to be less than 1, if there are magical protections or damage reductions this can become zero and therefore you will not hurt the opponent (but if it becomes negative Don't worry about it!).

First of all, remember that for every 6 rolled (in the 3d6 of the attack roll) you must roll another one and continue rolling as long as you keep rolling 6s on the die. 

If you hit, \textbf{every two 6s rolled} (counting those on the attack roll and subsequent ones resulting from rolling 6s), the weapon does extra damage or a Critical Roll. Re-roll only the weapon damage die, without any other modifiers, for every two 6s rolled on the attack roll.\index{Critical Roll}

You can \textbf{subtract 4} or multiples from your attack to roll an extra d6. The choice is to be made in the most desperate situations where only luck can resolve the duel. The value is taken from the Weapon Proficiency or Strength or Dexterity score, not from scores given by Feats, Weapon Lists or magical bonuses.

\textbf{The fact of rolling a Critical Roll is not a guarantee of having hit, you must always overcome the Defense except if you rolled all 6s with the first three dice.}

The basic rules of the Feats also apply to the Attack Roll. Defense is a fixed value and as such uses modifiers for fixed value checks.

\begin{center}
\includegraphics[width=0.9\linewidth]{immagini/critico.png}

\emph{Henry Justice Ford}
\end{center}

\subsection{Throw 3 times 1}\index{Throw 2 times 1 or 2 times 2 and once 1}\label{tiraretrevolteuno}\index{Throw 3 times 1}

If you roll 1s three times you miss, regardless of the final result.

If you missed and rolled at least two 1s or a 1 and two 2s the Storyteller could decide bad things about your attack (you drop your weapon, you hit a friend, your weapon breaks, you get hurt, you fall, appears \hyperlink{devilpit}{Pit Devil} to mock you...).

\begin{changemargin}{0.3cm}{0.3cm}\begin{narratore}
OBSS wants to be fun to play, it wants players to have fun and see the results obtained from the dice (and obviously from their choices). The Golden Rules and Damage Explosion really want to remove the veneer from the dice and make you have fun. A player will appreciate, even more if he is experienced, how the rolls of the dice are not just a number but rather open up the possibility of making a difference. Ask the player to describe the critical hit and have him recite it in the glory of his power!
\end{narratore}\end{changemargin}

\subsection{Throw 3 times 6}\index{Throw 3 times 6}\label{tiraretrevoltesei}

If you roll 6s three times in your first 3 attack rolls, you will take your opponent regardless of the final result of the attack roll. In addition to being sure of having made a Critical Roll, the Storyteller could decide to apply some further descriptive (or effective) effect.

\subsection{Critical Shot}\index{Critical Shot}\index{Critical Damage}\label{tirocritico}

Each time you hit, you roll \textbf{additional weapon-only damage} for every two times you roll a 6 on your attack roll, this damage is also called \textbf{critical damage}. If you made two Critical Rolls it means you have to roll 2 more weapon dice.

\begin{changemargin}{0.3cm}{0.3cm}\begin{tcolorbox}[title = Example Critical Roll]
E.g. I roll 6 4 5, I roll an additional 6, I roll an additional 6, I roll an additional 4: as damage you roll 2 times the damage of the weapon, once because I hit, once because you rolled 6 three times (if I had rolled an additional 6 would have been Weapon + Strength + bonus/Feat + 3{*}Weapon die).
\end{tcolorbox}\end{changemargin}

\subsection{Optional - Variant Critical Shot}\index{Variant Critical Shot - Optional}\index{Optional - Variant Critical Shot}\hypertarget{variant Critical Shot}{}\label{tirocriticovariante}

The player may have less preference for chance and manage criticals based on the character's \emph{Feat} in using the weapon.
An alternative method is to grant a critical roll for every multiple of 6 in which the attack roll is higher than the defense, regardless of the number of 6s rolled.

The choice to use this variant of the critical roll must be made at the time of character creation and in agreement with the Narrator.

\begin{changemargin}{0.3cm}{0.3cm}\begin{tcolorbox}[title = Variant Critical Shot]
E.g. Attack roll 21, the opponent's Defense is 13. I hit him with a margin of 8, i.e. I add 1 more weapon damage
If the attack roll had been 26, 2 Critical damage or two weapon dice would have been added.
It is the Narrator who communicates how many critics have been obtained.
\end{tcolorbox}\end{changemargin}

\subsection{Optional - Critical Shot Actions}\index{Optional - Critical Shot Actions}\label{OpzionaleAzioniTiroCritico}\hypertarget{Optional Critical Shot Actions}{}

This Option allows for combat that is less focused on damage but more on maneuvers and tactics. The fight is understood as a continuous exchange of actions and reactions which can also have repercussions in subsequent rounds

The player takes into account the Critical Rolls he rolls and does not apply to damage, in three rounds at a time, starting the count again at the end of the third round or when the count is zero.

Each round can scale one or more accumulated Critical Rolls to perform Critical Actions. The use of Critical Actions must be against the opponent against whom the critical actions were performed. The list offers more Critical Actions per sum of Critical Rolls used. You cannot have more than 3 critical rolls accumulated. Activating these Critical Actions costs a Reaction.

\textbf{1 Critical Shot}: you cause critical weapon damage; you get +4 to hit until the end of the next round; +4 Defense until the end of the next round; the opponent has -4 to hit until the end of the next round.

\textbf{2 Critical Shot}: you cause two critical weapon damage; the opponent misses you with the first useful melee attack; you can move and/or move your opponent one meter; reduce the damage of a melee attack by double the maximum critical damage it causes; until the end of the next round the opponent has half movement.

\textbf{3 Critical Shot}: you have one more move Action; the opponent loses the first Action he takes by the end of the next round; you and/or your opponent can move half your movement, your opponent cannot move until the end of the next round.\\

These Critical Actions can be described as taking advantage of the opponent's distraction, throwing dirt in their eyes, forcing them to move with weapons shots...

The player is invited to suggest new Critical Actions which will be subject to evaluation by the Storyteller.

The system is also compatible and usable with \textbf{Optional - Critical Shot Variant}.


\begin{center}
\includegraphics[width=0.7\linewidth]{immagini/esplosionedanno.png}

\emph{Henry Justice Ford}
\end{center}

\subsection{Damage Explosion}\index{Damage Explosion}\label{esplosionedeldanno}

Every time you get the maximum value from the weapon's die roll (in the classic d8 for the long sword for example you get 8 and it is therefore the maximum value of the die), you reroll the die and add the value (of the die alone) again.

In the case of weapons with multiple dice (for example 2d4, the maximum value must be obtained as the sum of the two dice, i.e. 8). There is no damage explosion for weapons with maximum damage less than or equal to 6.

Some weapons have a different damage burst. In the weapons table where EDX is marked (e.g. ED9), the value of the weapon.

This is a characteristic of a few extremely lethal weapons.

The damage explosion does not explode again, even if you roll the maximum die with the added die it does not explode again.

Dice rolls added thanks to the Critical Roll (obtained by rolling at least two 6s) do not have the benefit of the damage explosion. If the die of the weapon rolled thanks to the Critical Roll makes the maximum, do not reroll the die. When rolling damage, declare which die is for the weapon and which is for Critical Rolls.

\subsection{Multiple attacks}\index{Multiple attacks}\label{attacchimultiplimischia}\hypertarget{multiplemelee attacks}{}

With an Action the character can perform a single attack.

With two Actions the character can make up to two attack rolls. \textbf{If he wants to make 3 or more attacks he must use 3 Actions}.

Each individual arrow, dart, dagger, or ranged weapon fired counts as one attack.\index{Multiple attacks with ranged weapons}

The first attack action has no penalty while the second attack action has -5 to hit. Subsequent attack rolls will add -5 to hit, so a third attack will have -10 and a fourth attack -15...

If the cumulative hit penalty becomes greater than the attack roll, no further attacks can be made.

If I have Weapon Proficiency 5, Strength 1, +2 to perform as a bonus from the Weapons List and +1 to hit given by a Feat, +2 because I flank and +1 for magic weapon the first attack roll will be 3d6+12, the second will be 3d6+7, the third 3d6+2. It is not possible to make a fourth attack as the hit bonus would become negative.

Any dynamic hitting bonuses, e.g. +1d6, apply only to the first attack roll and not to the calculation of the bonus for calculating the number of multiple attacks. In the example case the attack roll becomes 4d6+12, 3d6+7/+2.

The player can declare to make attacks on different targets. Each attack can be interspersed with a Move Action, as long as it has enough Actions.

\subsubsection{Optional - Multiple attacks variant}\index{Optional - Multiple attacks variant}\label{varianteattacchimultipli}

The player who has established his hit bonus makes a single attack roll.
If he hits, he rolls the damage and for every multiple of 6 of the hit bonus he adds a critical roll. This attack consumes 2 Actions and is the only attack that can be made in the round.

This variant serves to speed up the game by making a single attack roll. This variant is not compatible with Optional - Critical Shot Variant.

\begin{center}
\includegraphics[width=0.9\linewidth]{immagini/archer.png}

\emph{Scythian archers in ancient attic vase painting}
\end{center}

\subsection{Throwned Weapons}\index{Multiple Attacks Thrown Weapons}\label{armidatiro}\index{Multiple Weapons}\index{Throwned Weapons}

Thrown weapons are all weapons with a range, meaning they can be thrown or launch projectiles. The main throwing weapons are bows, crossbows, slingshots but also daggers, javelins and spears if they are thrown.

The damage bonus given by Strength applies automatically to slings, daggers, javelins... that is, with all weapons that are thrown with strength, bows apply this bonus only if they are of the composite type, crossbows never apply it.

Dexterity only modifies the attack roll.

\textbf{Projectiles launched from Bows, Slingshots, Magic Crossbows are not considered magical.\\
In the case of magical projectiles these add their magical bonus to the attack roll and damage}

In each throwing weapon the range is marked, that is, within what distance it is possible to shoot the projectile without penalty. Each thrown weapon can hit within three times its listed range.

If the target is within the indicated range there is no hit penalty, if the target is between the first and second increments the hit penalty is -1d6. If the target is between the second and third increments the hit penalty is -2d6.

A dagger thrown within 6 meters has no penalty, but thrown between 6 and 12 meters has a -1d6 to hit, at a distance between 12 and 18 meters a -2d6 to hit, beyond that it cannot be thrown.

%\begin{center}
% \includegraphics[width=0.75\linewidth]{immagini/fenice.png}
%
% \emph{Henry Justice Ford}
%\end{center}


\subsection{Light Arms} \index{Light Arms}\label{armileggere}

these weapons are light and suitable for \hyperlink{two-handed fighting}{two-handed fighting}.

\subsection{Versatile Weapons} \index{Versatile Weapons}\label{armiversatili}

weapons with the Versatile feat can use Dexterity instead of Strength on attack rolls. Strength is always used on damage.

\subsection{Long Weapon} \index{Long Weapon}\label{armalunga}

the long weapon gives the right to hit further away, i.e. 2 meters, and grants a bonus to the attack roll of +2. This bonus remains valid until the opponent enters range of your melee.

If the opponent also has a long weapon, do not consider the bonus (they are both in their own melee area).

\subsubsection{Long weapon at short range} \index{Long weapon at short range}\label{armalungabrevedistanza}

You can use a long weapon in combat at ranges closer than 2 meters with a -4 to attack roll, except for the staff.

\begin{changemargin}{0.3cm}{0.3cm}\begin{tcolorbox}[title = Long Weapon Fighting]
E.g. Tups armed with a long sword faces a brigand armed with a long spear. Tups has initiative 15, the brigand 12.

Tups, taking advantage of his agility, gets under the brigand, hitting him powerfully. The brigand, finding himself in melee with Tups, is unable to exploit his long weapon which actually penalizes him.

He uses an Action to move two meters away and then attacks with a +2 bonus because the opponent is far away.

As a third action he moves another 9 meters away and shouts curses at Tups.

Tups is at this point 11 meters from his opponent, he decides to charge, thus opening his defense but obtaining a bonus to hitting.

He charges the brigand, hitting him and landing on him, with one last action he decides to improve his Defense (Combat Mastery).

The very wounded brigand tries to hit him, trusting that his difficulty in using a long weapon so close is balanced by the penalties given by Tups' running. Tups is hit and the brigand throws his spear to the ground and pulls out a short dagger and also goes on the defensive.

\end{tcolorbox}\end{changemargin}

\subsection{Double Weapon} \index{Double Weapon}\label{armadippia}

a double weapon is a weapon that is dangerous from both ends. It can be used as a single weapon, or, incurring the penalties of fighting with two weapons, as two weapons.

Unless specified, a dual weapon used in two-weapon combat is equivalent to using two medium weapons.


\subsection{Two-Weapon Fighting}\index{Two-Weapon Fighting}\hypertarget{Two-Handed Fighting}{}\label{combattimentoduemani}

Attacks made with the secondary weapon are considered multiple attacks.
If I attack for the first time, regardless of whether it is with the primary or secondary weapon, this will have a full bonus to attack roll, the other attacks will accumulate -5 to hit.

Strength damage bonus on secondary weapon is halved. If the secondary weapon is not \textbf{Light} the attack roll has an additional -3 to hit (e.g. 0,-8,-10,-18..).

It is possible to use the secondary weapon to improve Defense by one point but you cannot make attacks with that weapon.

\begin{center}
\includegraphics[width=0.9\linewidth]{immagini/twoweapon.png}
\end{center}

\subsection{Load} \index{Load}\label{carica}

the opponent must be within 2 Move Actions (18 or 12 meters usually) and no closer than 3 meters away. You must run until you are within melee range.

You get a +1d6 to attack roll, -4 to Defense until the end of the round, the attack following the first takes a -10 to hit and a possible subsequent one -15.20...

Movement and attack costs 2 Actions. No other penalties are considered for having raced beyond those indicated.

The Charge action brings you into melee with your opponent. The attack, if made with a long weapon, still has a +2 bonus on the attack roll and hits from a distance, then ends up in contact with the opponent.

%\begin{center}
% \includegraphics[width=0.9\linewidth]{immagini/carica.png}

% \emph{A Connecticut Yankee in King Arthur's Court / Samuel Clemens. New York : Charles L. Webster \& Co., 1889}
%\end{center}

\subsubsection{Counter-Charging Weapon}\index{Counter-Charging Weapon}\label{controcarica}

an attack roll made with a weapon with the countercharge feat when used against a charging opponent/mount inflicts a critical roll and hits first, unless the opponent has a long weapon or greater reach than the countercharger, in this case the attack is regulated by initiative rolls.

\subsubsection{Prepare a long/counter-charge weapon against a charge} \index{Prepare a long weapon against a charge}\label{prepararearmalungacontrocarica}

Only a long weapon or one with the countercharge trait can be used against a charge. Readying the weapon costs a Reaction.

\subsubsection{Charge with Counter-Charging Weapon} \index{Counter-Charging}\label{caricaarmadacontrocarica}

if the attack roll is successful when you use a weapon with the countercharge feat to charge an opponent it inflicts a critical roll.


\begin{center}
\includegraphics[width=0.9\linewidth]{immagini/pilum.png}

\emph{Roman soldiers armed with Pilum, ready for a counter-charge.}
\end{center}

\subsection{Attacks with splash weapons} \index{Splash weapons}\index{Holy water}\index{Storn oil}\label{attacchiarmidaspargimento}\hypertarget{spread}{}

Spread weapons are those that \emph{spread} their contents wherever they fall, for example burning oil/Holy water... A splash weapon has a range of 6 meters\index{Throwing splash weapons}\index{Range of splash weapons}.

If the attack misses (by at least 5), roll a d8 and consult this diagram to understand where the ball fell:

\medskip

\begin{tabularx}{0.30\textwidth}{ccc}
1& 2& 3\\
4 &\textbf{X}& 5\\
6 &7 &8\\
&\textbf{0}&\\
\end{tabularx}

\smallskip

\textbf{X} is considered the target of the thrown object. \textbf{0} the point of origin of the launch.

If the roll misses by 5 or more, roll a 2d4 to determine along the direction indicated by the previous d8 how many meters it fell away from the target, i.e. count the meters from the target.

For example, with the roll of the d8 I get a 5 and then rolling the 2d6 I get a 4, which means that the bottle fell to the right of the target at 4 metres.

It is also possible that the bottle was thrown at one's feet (e.g. I do 7 and then 6... I could have thrown it at a friend or behind me!).


\subsection{Unprepared -- Caught by Surprise}\index{Unprepared}\index{Surprised}\label{coltidisorpresa}

if the characters are caught by surprise, i.e. they do not expect to be attacked, this first round must be considered a surprise round. When surprised you have a -4 to Defense and Reflex saving throws.

You will not be able to react, you will not use Actions or Reactions unless explicitly permitted; from the next round you will be able to declare the initiative and act normally. The same considerations apply to opponents.

To evaluate whether a character is surprised, make a Reflex Saving Throw, comparing it with the Stealth check of the opponents. If the Saving Throw is lower, the character is surprised. If the character is at attention and expecting an ambush, grant +4 to the Saving throw.

When characters and enemies are both caught by surprise, to evaluate who is actually surprised, make a Reflex saving throw, anyone who rolls more than 15 is not surprised.

\subsection{Combat Magic}\index{Combat Magic}\label{magiaincombattimento}

the caster who casts a spell while in combat (has an opponent in melee or is targeted from a distance) is considered distracted. 

Casting a spell while in combat imposes a -2 Defense.


\subsection{Attack or Defense Modifiers} \index{Attack and Defense Modifiers}\label{modificatoriattaccodifesaparticolari}

The best tip that can be given in managing the most chaotic combat situations is to think of them like a film, evaluate the cinematic nature of the situation.

It's not a question of miniatures, spaces, squares... it's a question of fun and visualization of the scene. Unorthodox solutions for unorthodox situations.

Grant a bonus or penalty ($\pm 1-2$) unless otherwise indicated) whenever the player has an advantage or disadvantage and similarly to the opponent.\\

\end{multicols}

\begin{tabularx}{0.98\textwidth}{l|X|X}
\multicolumn{2}{c}{\textbf{Attacker}}&\multicolumn{1}{c}{\textbf{Defender}}\\
\textbf{Mod}.&\multicolumn{1}{c}{\emph{Situation}}&\multicolumn{1}{c}{\emph{Situation}}\\
\toprule
\textbf{-1} & Dim light & \\
\hline
\textbf{-2} & Dazzled, Entangled & Grabbed, You cast a spell while under attack \\
\hline
\textbf{-4} & Prone, Long Weapon at close range & Surprised, Prone, Kneeling, Sitting, Restricted, Stunned, Grasped by a wall, Pinned\\
\hline
\textbf{-1d6} & Restricted, Frightened, Thrown weapon against opponent in melee, Weapon unknown, Target invisible but Spotted, Grasped to a wall, Blocked & \\
\hline
%\textbf{+1} & & \\ 
%\hline
\textbf{+2} & Line, pos. Overhead, Shoulder Attack, Long Weapon & Light Cover\\
\hline
\textbf{+4} & & Medium coverage\\
\hline
\textbf{+1d6} & Invisible, Charge & \\
\hline
\textbf{+8} & & Full coverage\\

\end{tabularx}

\medskip

\begin{multicols}{2}

When you write -1d6 it means that you roll one die less (or two if it is -2d6), equally if it says +1d6 you roll one die at 6 more and add it.

When the penalty is to Defense, treat each -1d6 as a -4 to Defense.

\textbf{In principle in combat a light bonus is +1, medium +2, high +1d6 (or +4), a very high bonus is +2d6 (or +8), vice versa for penalties}.

\medskip

The bonuses are not added to each other but the one with the highest value is used. If an opponent is above the character, behind him and charging, he has a bonus to hit of +1d6, given by the charge. %evaluate for optional

The penalties are added to each other. If the character is surprised and prone he has -8 to Defense.\index{Cumulative bonuses}\index{Cumulative penalties}

\begin{changemargin}{0.3cm}{0.3cm}\begin{narratore}
Always remember that the aim is to have fun, at the expense (for the Narrator) of some monsters, do not be rigid but dynamic and adapt to situations.
\end{narratore}\end{changemargin}

\subsection{Optional - Cumulative bonuses}\index{Optional - Cumulative bonuses}\label{Opzionale - Bonus comunulativi}

Add +1 to the Major Bonus for each additional bonus present.


%\begin{center}
% \includegraphics[width=0.55\linewidth]{immagini/vantaggio.png}
% 
% \emph{Henry Justice Ford}
%\end{center}

\subsection{Other actions and situations} \label{AltreAzioni}\index{Other actions and situations}

\subsubsection{Attack with bare hands} \index{Punch}\index{Kicks} \index{Fighting}\label{attaccomaninude}

two weapons that no one will ever lack are their punches and kicks.

If you did not take the list of weapons \emph{Empty Fist} a punch or kick will do 1d2 + Strength non-lethal damage. Only with the Empty Fist Weapons List can you become a martial artist.

\subsubsection{Helping another}\index{Helping}\label{aiutare}

you can help a teammate attack or defend yourself in melee combat by distracting or interfering with your opponent. You can make a melee attack (1 Action) against an opponent who has already engaged in battle with an ally.

You make an attack roll against your opponent's Defense with a 1d6 bonus. If the attack hits, no damage is done but the partner gets a +1 bonus on attack rolls with the next attack (by the end of the next round) against that opponent or a +1 bonus on Defense against the next one. that opponent's attack (your choice) within the next round. If the helper rolls a critical roll then the person being helped will have a +2 bonus.

Multiple characters can help the same ally; bonuses of this type are cumulative (maximum 4 on medium size), as long as the opponent is surrounded.


\subsubsection{Getting up from prone}\index{Getting up from prone}\label{alzarsidaprono}

costs two Actions. The player can make an Acrobatics check if he rolls 13 or more and it costs 1 Action to get up. If you fail the check you cannot take any further actions that round and remain prone.

When your Acrobatics score reaches 6, getting up from prone costs 1 Action. With Acrobatics 8 it costs an Immediate Action.

When you are prone you can crawl\index{Crawling}\index{On all fours} or move on all fours. The terrain is considered difficult and you are still considered prone until you get up.


\begin{center}
\includegraphics[width=0.9\linewidth]{immagini/colpodigrazia.png}
\emph{Beheading of Saint John the Baptist. St. John's Co-Cathedral in Valletta (Malta, Caravaggio)}
\end{center}

\subsubsection{Coup de Grace} \index{Coup de Grace}\label{colpodigrazia}

costs 3 Actions, you can use a melee weapon to deal a finishing blow to an incapacitated or defenseless (unconscious or trapped) target. You can also use a bow or crossbow, as long as you are adjacent to the target.

The attacker automatically hits and deals three critical hits.

\subsubsection{Targeted Shots}\index{Targeted Shots}\label{tirimirati}\index{Aiming at specific parts}

OBSS does not allow you to make targeted shots with any weapon or spell, unless it specifies it.

When you hit the target you hit it generically, without the possibility of specifying whether to the head, leg or something else, the same concept applies in the case of hits to objects, e.g. if you aim at a door hinge you hit the whole door. This does not prevent the Storyteller from considering appropriate consequences.

\subsubsection{Non-lethal damage}\index{Non-lethal damage}\label{dannononletale}

non-lethal damage is a form of damage caused by particular weapons or when the purpose is to make the opponent faint and not kill him.

Non-lethal damage is treated like normal damage except that it recovers more quickly and falling below zero causes fainting rather than approaching death.

\index{Non-lethal damage with unsuitable weapon} \label{dannononletalearmanonidonea}

If you want to deal non-lethal damage with a weapon not designed for non-lethal damage you have a -1d6 on your attack roll.

\subsubsection{Without Competence}\index{Without Competence}\label{senzacompetenza}

using a weapon without adequate proficiency, i.e. not having the Weapon List to which the weapon belongs, imposes a -1d6 on the attack roll.

You can't use a weapon's Versatile ability if you don't know how to use it. A Simple Weapon can be used without penalty even without specific expertise.

\subsubsection{Throwing weapons} \index{Throwing weapons}\label{lanciarearmi}

a sword or in any case a weapon not made to be thrown, without Range, can still be thrown at the opponent.

The attack roll takes -1d6 and the weapon does a lower category of damage (the long sword does 1d6, a short sword 1d4..). The launch range is 3 meters.

\subsubsection{Powerful Blows}\index{Powerful Blows}\label{colpipotenti}

the player can freely add +1 to damage by subtracting 2 from the attack roll (Weapon Proficiency requirement +1). You cannot remove more than Weapon Proficiency/4 from your attack roll.

\subsubsection{Flank} \index{Flank}\label{fiancheggiare}

if two characters are around the same target but are not next to each other they get +2 to attack or defense rolls (they choose which bonus to take).

There can be at most 4 characters around a medium-sized creature who get the flanking bonus. The type of bonus is chosen round by round, if not declared it is worth +2 to the attack roll.

If by drawing a hypothetical line that connects the two characters it crosses completely the opponent's square then there is a flanking situation.

\bigskip

Flanking Example\index{Flanking Examples}

\medskip

\begin{tabularx}{0.45\textwidth}{lll}
\toprule
A & G & D\\
B & \textbf{X} & E\\
C & H & F\\
\end{tabularx}

\bigskip

In this scheme the flanking is taken by the pairs: A-F, B-E, C-D, G-H

\bigskip

If the creature can face multiple creatures at the same time, they will not benefit from the flanking bonus.

\subsubsection{Combat Mastery} \index{Combat Mastery}\label{maestriacombattimento}

The player can freely add +1 to Defense by subtracting 2 from Attack Rolls as long as he attacks in the round.

Conversely, he can take a -2 Defense to raise his attack roll by +1 and therefore improve his attack. This option is only usable if you make at least one attack.

You cannot remove/add more than Weapon Proficiency/4 to the Attack/Defense roll, improving Defense or TC does not consume Actions. 

You can instead use an Action to better prepare for subsequent attacks from your opponents. Until the end of your next round you have +2 Defense.

\subsubsection{Precise Shot} \index{Precise Shot}\label{colpopreciso}

the player, using 2 Actions, makes only one attack (and must not have made any in the round). On this single attack he gains a +2 bonus on attack rolls.

\subsubsection{Aiming (sniper)} \index{Aiming (sniper)}\label{cecchino}

you dedicate 2 Actions per round to aiming at a target. You have a bonus on your attack roll of +1 on the first round, +2 on the second round and finally on the third and final round of Aiming +4. 

You cannot use Move Actions while aiming.

\subsubsection{Using a thrown weapon aiming at an opponent engaged in combat} \index{Throwned weapon against an opponent engaged in combat}\label{usarearmalancioinmischia}

it's not easy to aim correctly and not hit someone else, you have a -1d6 on your attack roll. The bonus is canceled if there is a difference of 2 or more sizes between the creatures involved.

\subsubsection{Using a thrown weapon under threat} \index{Using a thrown weapon under threat}\label{usarearmalanciosottominaccia}

Using a thrown weapon such as a bow, crossbow, or dagger (that you want to throw) while threatened in melee imposes a -1d6 chance on your attack roll.

\subsubsection{Weapon too large}\index{Weapon too large} \label{armatroppogrande}

The size indicated in the weapon table (see \hyperlink{sizeofaweapon}{Dimensions of Weapons}) refers to a medium creature. For a small-sized creature, the size must be considered a larger category, so a short sword, which is small in size, used by a small-sized creature is considered a medium-sized weapon.

Likewise, a large weapon, such as a two-handed greatsword, becomes a medium-sized weapon in the hands of a giant.

This does not change its damage or the type of damage the weapon causes.

A creature can use a weapon of its own size size or one size smaller with one hand, and must use two hands to wield a weapon of one size larger.

If the weapon is not among those usable, for example a Halberd (large weapon) for a small creature the penalty on the attack roll is -1d6. Likewise, a small weapon is not two-handed for a medium-sized creature.

In the weapons table the size is marked as P (small), M (medium), G (large), E (huge). A \emph{larger} version of a weapon increases the weapon's damage by one category (1d4->1d6, 1d6->1d8, 1d8->1d10, 1d10->2d6, 2d6->2d8, 2d8-> 2d10, 2d10->3d6...)


\begin{center}
\includegraphics[width=0.65\linewidth]{immagini/angelospadone.png}
\end{center}


\subsubsection{Using a weapon with two hands} \index{Using a weapon with two hands}\label{usarearmaconduemani}

a one-handed weapon that can (but does not have to) be used two-handed increases the damage die when used two-handed.

E.g. Longsword for a medium creature can cause 1d8 to one hand or 1d10 to two hands. A shortsword cannot be wielded two-handed by a medium creature.

If the weapon must be held in two hands because it is too large for one's size, this modifier is not considered (e.g. a two-handed greatsword for a medium-sized creature).

The EDX value if different from the maximum damage of the weapon increases by 2 (Katana will cause 2d6 damage and have ED11) when used with two hands.

\subsubsection{Total defence} \index{Total defence}\label{difesatotale}

It costs 2 Actions, if you move the terrain is difficult, you gain +4 to Defense.

\subsubsection{Disengage} \index{Disengage}\label{disingaggiare}

costs 1 Action, you move 1 meter and do not cause attacks of opportunity.\index{Taking a step}

\subsubsection{Fighting in the dark}

Fighting in low light conditions involves difficulties summarized in this diagram.

\medskip

\begin{tabular}{lcc}
\textbf{View} & \multicolumn{2}{c}{\textbf{Condition}}\\
& Dim Light & Darkness\\
\toprule
Normal & -2 Defense & Invisib. (page \pageref{invisibility})\\
Twilight & Normal & Invisib. (page \pageref{invisibility})\\
\end{tabular}


\subsection{Optional - The Only Rule}\index{Optional - The Only Rule}\hypertarget{the only rule}{}\label{lunicaregola}

This option is intended to simplify the management of any opposing test, whether relating to Basic or Active Skills.

When a creature or character has an advantage or disadvantage, roll 1d6 in addition to the check. If he has two advantages roll 2d6, if he has three advantages roll 3d6...
On the check it adds or subtracts, in case of bonus or penalty, the highest value among the rolled dice. For these dice the explosion of the result is not valid.

A Disadvantage cancels out an Advantage if present.

If the advantage/disadvantage is relative to a static value (such as Defense) then this increases by 2 for each accumulated advantage/disadvantage.

\begin{center}
\includegraphics[width=0.9\linewidth]{immagini/alfieri37.png}
\end{center}


\subsection{Optional Combat Maneuvers}\label{azioniopzionaliincombattimento}

These combat actions are at the discretion of the Storyteller who may or may not grant them. Each maneuver counts as an Attack Action.

\medskip

\subsubsection{Disarm*}\index{Disarm}\label{disarmare}

make an Opposed Weapon Proficiency Test + Dexterity or Strength (3d6+AC+Str or Dex)

If the person attempting the maneuver fails and gets a critical failure, he or she loses the weapon. He costs 2 Actions.

\subsubsection{Fake*} \index{Fake}\label{finta}

make an Opposite Test of Weapons Proficiency + Deception (the one who is feinting) versus Weapons Proficiency + Perceiving Emotions (the one being feinted). If the check succeeds, the opponent has a -2 to Defense until the end of the next round.

If the person attempting the maneuver fails and gets a critical failure, he gets -2 to Defense until the end of the next round. Costs 1 Action.

\subsubsection{Pushing an opponent*} \index{Pushing an opponent}\label{spingereavversario}\hyperlink{pushing an opponent}{}\index{Pushing an opponent}

it is an Opposed Test of Strength. Those with a larger size gain a bonus of +1d6 per size difference.

The winner can push or pull the opponent up to 0.5 meters in the direction he wants for success in the test (up to the maximum of his movement). E.g. if you win the test of 7 you move your opponent up to 3 meters. It costs 2 Actions, or the Reaction if the one who resists the push is vice.

\subsubsection{Grab an opponent*}\index{Grab an opponent}\label{afferrareunavversario}

it is an Opposed Test of Strength. Those with a larger size gain a bonus of +1d6 per size difference. If the person who succeeds in the maneuver obtains a critical success, he is considered to have \hyperlink{blocked}{Blocked} the opponent.

It costs 2 Actions to do and hold and free yourself from the hold. The person who grasps is considered to have also grasped and has at least one hand occupied in grasping.
Moving a grabbed creature requires \hyperlink{push opponent}{Push opponent}.

Each contender can attack the other with a small weapon or natural weapons. Defense has a -2 penalty and you are considered Distracted.

\subsubsection{Knocking down an opponent*} \index{Knocking down an opponent}\label{farecadereavversario}

it is an Opposed Test of Strength or Dexterity, each contender chooses the one he prefers.

For each additional leg/paw and size difference the contestant gets a bonus of +1 on the check.

If the person attempting the maneuver fails and gets a critical failure, it is they who fall.

It costs 2 Actions. Whoever does less in the test falls to the ground prone.

\subsubsection*{Optional - Universal Maneuver Management}\index{Optional - Universal Maneuver Management}

In order to neutrally manage any unexpected maneuver or action in combat, an approach can be used that benefits both the player and the opponents.

You declare what type of action you want to take in combat. If the attack roll is successful then the person who suffers the action decides between suffering the desired effects of the action or suffering the damage of the attack. If the person performing the action rolls a critical then it imposes the chosen effect of the action.

There are obvious limits to the type of action taken which at the Storyteller's discretion could be overcome by a certain number of critical rolls made.


\subsubsection{Change your size*}\index{Change your size}\label{modificatedimensioni}

if the character changes size \index{Change the size} his Defense changes accordingly\\

\begin{tabular}{ll|ll}
\textbf{Bounty} & \textbf{Defense}& \textbf{Bounty} & \textbf{Defense}\\
\toprule
Very Small & +8 & Large & -1\\
Tiny & +4 & Huge & -2\\
Tiny & +2 & Gargantuan & -4\\
Small & +1 & Colossal & -8\\
Average & +0 &&\\
\end{tabular}


\subsection{Mounts}\index{Horseback Combat}\index{Horse}\label{cavalcature}

\begin{changemargin}{0.3cm}{0.3cm}\begin{enfasi}{
- And you can find another wife!

- Ah, yes. but the trouble is that she took away my rifle and my horse! Too bad, she was so beautiful, I was fond of her. I gave her a few whippings, but she didn't pay any attention.

- Who, your wife?

- No, my mare. It's easy to find another wife, but I'll never find a mare like hers again. (Red Shadows, 1939 film)}\end{enfasi}\end{changemargin}\medskip

A mount has 2 Actions and they are usually used to move or react and obey your commands.

A mount acts in your round, and you decide when it takes its Actions versus yours. Don't roll initiative, use yours.

Attacks towards a character on horseback (or mount in general), unless otherwise stated, aim at the rider and not the horse.


\subsubsection{Situations and rules}\label{cavallosituazioniregole}

\begin{itemize}
\item
Whenever the mount is hit the rider must make a Ride check at DC 15 or be unhorsed.

If the mount is a war mount trained for combat, the Ride check has difficulty 12.

\item
Fighting from an elevated position grants a +2 to your attack roll if your opponent is not at your height.

\item
Getting on or off your mount costs 1 Action if you have the Ride skill, otherwise 2 Actions.

\item
If a spell or situation abruptly moves your mount against your will, you must make a DC 15 Reflex save or a Ride check (DC 15) or be thrown from your horse.

\item
If you are thrown from your horse you fall prone and suffer 1d6 damage.
\end{itemize}


\subsubsection{Control a Mount}\label{controllocavalcatura}

While mounted, you have two choices: you give orders to your mount or you allow it to act on its own.

Particularly intelligent mounts tend to favor autonomous action rather than being commanded.

You can only control a mount if it has been trained to accept a rider. Trained horses, war horses, mules, and similar creatures are presumed to have received such training.

By spending 1 of your Actions you can make the mount perform 2 of these Actions: Move, Attack, Disengage. 

If the mount is intelligent it could act and move as it prefers despite the rider's indications. It may flee from combat, attack and devour a badly wounded foe, or otherwise act against its rider's will.

\end{multicols}

\vfill

\begin{center}
\includegraphics[width=0.27\linewidth]{immagini/napoleone.png} 

\smallskip

\emph{Jacques-Louis David, Bonaparte crossing the Great Saint Bernard, 1801, Malmaison Castle}
\end{center}


%\vfill

%\begin{center}
% \includegraphics[width=0.7\linewidth]{immagini/fauchard.png}
%\end{center}

\pagebreak

\section{Hides and covers} \index{Hides}\index{Covering}\hypertarget{covers}{}

\begin{changemargin}{0.3cm}{0.3cm}\begin{enfasi} Where there is a lot of light, the shadow is blacker. (Johann Wolfgang von Goethe) \end{enfasi}\end{changemargin}\medskip

\begin{multicols}{2}

\lettrine[lines=2, lhang=0.33, loversize=0.25, findent=1.5em]{N}{on} the adversary always appears in front of us, he can often be hidden or even invisible.

He could be hidden behind a low wall or barrels, if not behind a gigantic, muscular familiar.
What if he was behind us and we didn't even notice him?

\subsection{Coverage}\index{Coverage}\label{copertura}

If the target is known to exist but is hidden in some way then it is said that he has \textbf{coverage}.

\begin{itemize}
\item
If the target \textbf{has more than half} (but not total) of the \textbf{visible} surface then the coverage is defined as \textbf{light}, i.e. it has +2 to Defense. This can be the case of a creature behind another creature of the same size or 1 size larger.

This may be the case of an archer standing behind a 1 meter wall.


\begin{center}
\includegraphics[width=0.9\linewidth]{immagini/hide.png}
\emph{British Soldiers Hiding From Boer Fire At The Battle Of Majuba Hill.}
\end{center}

\item
If the target has \textbf{less than half} (but at least a third) of the \textbf{visible} surface then the coverage is defined as \textbf{average}, i.e. it has +4 Defense. This may be the case of a creature behind another creature 2 sizes larger.

It may be the case of an enemy armed with a crossbow who leans out just enough to keep the crossbow leaning against the wall and shoot (chest, shoulders, arms and head visible).

\item
If the target knows where it is but \textbf{hides completely} only looking out to check or shoot an arrow every now and then, behind a wall, window, door, table, a creature larger than itself (at least 3 sizes) .. then the coverage is defined as \textbf{complete}, i.e. it has +8 to Defense.

\end{itemize}

Half the cover bonus also applies to \textbf{Saving Throws} against Spells that have a \textbf{area effect} (e.g. Fireballs exploding around...).\index{Cover on Saving Throws }

\subsection{Invisibility}\index{Invisibility} \hypertarget{invisibility}{}\label{invisibilita}

If an opponent is invisible or you don't know where he is, follow the Invisibility rules.

\begin{center}
\includegraphics[width=0.8\linewidth]{immagini/brickwall.png}

\emph{is there anyone in front of this wall?}
\end{center}

Even if you are invisible, this does not mean that you cannot be perceived differently through other senses, such as smell, hearing or touch. Invisibility makes a creature undetectable by sight but does not itself make a creature undetectable or immune to critical rolls or bursts of damage.

A creature that is blinded or fighting an invisible creature or fighting in complete darkness, without darkvision, can make an Awareness check, 1 Action, at difficulty 20, or 2 Actions at Difficulty 15, for \textbf{detect} a creature if within 20 feet of it.\index{Spot Targets}

Depending on the distance of the invisible creature or what it does there are different modifiers to the Awareness check to detect it.

\medskip

\textbf{Table: Awareness DC Modifiers for Detecting Invisible Creatures}\index[Tables]{Table Awareness Modifiers for Detecting Invisible Creatures}

\medskip

\begin{tabularx}{0.45\textwidth}{ll}
\textbf{The Invisible Creature...} & \textbf{Mod.}\\
\toprule
It moved & -4\\
She ran or charged & -8 \\
Use Stealth & check+10\\
It's still and doesn't make any noises & +8\\
For every meter over 6 meters & +2\\
You have Light/Medium/Full coverage. & +4/8/12\\
\end{tabularx}

\medskip

These modifiers are cumulative with each other.

If the invisible creature attacked in melee and did not move, it is considered \textbf{automatically detected}.

If the spot check succeeds, the observer has a feeling that there is something there, but cannot see it or target it accurately with an attack.

Whoever attacks a creature for her \textbf{invisible but detected} has a -1d6 on the attack roll, the creature that attacks the one who does not see it has +1d6 on the attack roll.

A blinded \index{Blinded} creature takes a -2 penalty on Strength and Dexterity-based Proficiency checks and automatically fails any Awareness checks that depend on sight.

Attacking an undetected target means attacking a random \emph{square} on the map. Always allow attack rolls, whether there is an opponent in that square or not. If the target is in that square you modify its Defense by +8, if the \emph{square} is empty the attack roll will not hit anyone and you will inform the character that nothing has been hit.

\subsubsection{Notes on invisibility}

If an invisible character picks up a visible object, the object remains visible. An invisible creature can pick up a small visible object and hide it on itself (putting it in a pocket or under its cloak, clenching it in its fist) and effectively making it invisible.

Someone might sprinkle some flour on an invisible object to at least keep track of its location (until the flour falls off completely or is blown away).

Invisible creatures leave footprints. Their tracks can be followed without problems. Footprints in sand, mud, or other soft surfaces can give enemies clues to the invisible creature's location, making it detectable.

An invisible creature in the water moves the liquid, revealing its position. The invisible creature still remains difficult to hit and enjoys the benefits of medium coverage (+4 to Defense).

An invisible lit torch still gives off light (as does an invisible object subject to light magic).

Invisible creatures can't use gaze attacks. Invisibility does not affect being targeted by a divination spell.

\end{multicols}

%\vspace{4cm}

\vfill

\begin{center}
\includegraphics[keepaspectratio,width=0.75\textwidth]{immagini/impronteneve.png}

\emph{Can help find an invisible wolf...}
\end{center}

\pagebreak


\section{List of Weapons by Homogeneous Type}\index{List of Weapons}\index{Homogeneous Type}\hypertarget{list.weapons}{}\label{lista.armi}

\begin{changemargin}{0.3cm}{0.3cm}\begin{enfasi}{Strength does not reside in a Sword, but in the arms of a brave man. (The Legend of Zelda: Twilight Princess)} \end{enfasi}\end{changemargin}\medskip

\begin{multicols}{2}

\lettrine[lines=2, lhang=0.33, loversize=0.25, findent=1.5em]{O}{gni} every time you assign a point to Weapon Proficiency you can decide whether to continue improving on an already known List of Weapons or learn a new one, if the use is not declared it is assigned to the Simple Weapons List.

On the sheet, note which List of Weapons you assign the Weapon Proficiency point to.

To reassign a Weapons Proficiency point to another list requires at least 4 hours of training for 4 months.

Using a weapon without the proper proficiency imposes a -1d6 on the attack roll.

All Weapon Lists grant, unless otherwise written, these cumulative benefits when the score in the Weapon List reaches the indicated value:


\begin{itemize}\index{Common Bonuses List of Weapons}

\item 6 points: If you face someone using a weapon on this list you are immediately able to understand their Weapon Proficiency ability.

\item 10 points: if you hit the same opponent with at least two attacks in the round, the second attack causes 1 critical damage if it has no generated

\item 14 points: if you hit the same opponent with at least two attacks in the round you can move one meter without using Actions.

\item 18 points: when you make an Attack Roll you also consider the 5 for the Critical count (but do not reroll the die).

\item 20 points: when you make an Attack Roll you also consider the 5 for the Critical count and reroll the die.

\end{itemize}

Points awarded in a Weapon List do not add to attack rolls! You must check the score in the Weapon List with any bonuses that the same list lists.

The bonuses indicated in the Weapon Lists apply only when fighting with the weapons indicated in the same list.

The benefits shown are cumulative unless otherwise indicated.

\subsection{Bows} \index{Bows} Long Bow, Short Bow, Long Composite Bow, Short Composite Bow\label{listaarmiarchi}



\begin{itemize}

\item 4 points: add the Strength value to the damage, even if the bow is not composite. On a short bow you can add up to +1 damage, on a long bow up to +2 damage.
\item 5 points: reduce the penalty for shooting beyond the standard range by 1d6.
\item 7 points: Your mastery of using the bow in combat is such that you suffer no penalty when shooting arrows at enemies in melee or light cover.
\item 9 points: the first shot you make shoots two arrows. The attack roll starts at a -5 penalty.
\item 11 points: you shoot an extra arrow with a -5 penalty on your attack roll, the penalty does not stack with the multi attack. (Attack Roll, TC-5, TC-5, TC -10...).
\item 16 points: The first arrow that hits in the round adds critical damage.

\end{itemize}

\begin{center}
\includegraphics[width=0.9\linewidth]{immagini/arma-arco.png}
\end{center}

\subsection{Armors}\index{Armor List} \label{listaarmature}

This List only grants the cumulative bonuses listed here when wearing Armor.

\begin{itemize}

\item 1 point: halve the time it takes to put on and take off armor
\item 2 points: the Defense granted by the armor increases by 1 point, sleeping in medium armor does not cause fatigue
\item 3 points: Proficiency Penalty decreases by 1 point, sleeping in heavy armor does not cause fatigue
\item 4 points: Movement penalty decreases by 1 meter, armor Defense increases by 1 point
\item 5 points: decrease the Critical Rolls taken per melee attack by 1, the Skill penalty decreases by 1 point, the Movement penalty decreases by 1 meter
\item 6 points: the reduction of the critical roll suffered also applies to ranged attacks. Wearing armor no longer forces you to take the Magic Test
\item 7 points: cancel the Penalty to Competence and Movement
\end{itemize}

\subsection{Light Weapons}\index{Light Weapons} Short Sword, Light Mace, Rapier, Scimitar, One-Handed Axe, Dagger\label{listaarmileggere}

\begin{itemize}

\item 4 points: You can use Dexterity instead of Strength on attack rolls.
\item 5 points: You can draw the weapon as part of the Move Action.
\item 7 points: you can draw the weapon as an Immediate Action.
\item 9 points: increase the weapon's damage die by one step. If the damage die becomes 8 or more the weapon gains EDX on the maximum value of the die.
\item 11 points: increase the weapon's damage die by one step. EDX is reduced by 1.
\item 16 points: Using a Reaction avoids the first attack of the round in melee range and can make a response attack.


\end{itemize}

\subsection{Double Weapons} \index{Dual Weapons} Double Great Axe, Double Flail, Double-Bladed Sword, Urgrosh\label{listaarmidoppie}

\begin{itemize}
\item 4 points: your proficiency in the use of these weapons makes you extremely versatile, giving you the possibility at the start of your round to choose whether to be defensive or offensive, increasing your attack roll or defense by 1 at the end of the next round . It doesn't cost Actions.
\item 5 points: by taking -4 to attack rolls on the first attack you make in the round you get +4 to Defense until the end of the next round.
\item 7 points: using a non-light double weapon does not add the additional -3 to the attack roll.
\item 9 points: your technique leaves no points uncovered, for each successful attack roll in the round you get +1 to Defense until the end of the following round.
\item 11 points: hit wildly with your weapon. The first hit is equivalent to two hits.
\item 16 points: every time you hit with a critical roll you can deliver, without using Actions, a blow with the other end of the weapon. This attack roll cannot itself cause critical attacks and is -1d6 to the roll.

\end{itemize}

\subsection{Graceful Weapons}\index{Graceful Weapons} Rapier, Scimitar, Glaive\\\label{listaarmiaggraziate}

\begin{center}
\includegraphics[width=0.7\linewidth]{immagini/sciabole.png}
\end{center}

\begin{itemize}
\item 4 points: your style is very similar to a dance. You can use your Charisma or Dexterity value on the attack roll.
\item 5 points: you can use your Perform score in place of Weapon Proficiency on Attack Rolls.
\item 7 points: you know how to hit where it really hurts. The first critical hit adds an additional critical hit.
\item 9 points: The weapon die increases by one category.
\item 11 points: using a Reaction you can try to intercept your opponent's attacks. Using a Reaction adds +2 to Defense until the end of the round.
\item 16 points: Your dance stops your opponent from tackling you. Force the opponent in melee with you to attack only you until the end of the next round. 1 Reaction.

\end{itemize}

\subsection{Weapons of Death}\index{Weapons of Death} Light Pike, Heavy Pike, Scythe, Sickle\label{listaarmidelamorte}


\begin{center}
\includegraphics[width=0.7\linewidth]{immagini/scythe-types.png}

\emph{Eric Sloane. A Museum of Early American Tools.}

\end{center}

\begin{itemize}
\item 4 points: you can perform a Coup de Grace with the cost of 1 Action.
\item 5 points: The first critical hit you land on your opponent adds an additional critical hit.
\item 7 points: increase the weapon's damage die by one step.
\item 9 points: the first critical hit you land on your opponent adds 2 additional critical hits.
\item 11 points: increase the weapon's damage die by one step.
\item 16 points: increase the weapon's damage die by one step.

\end{itemize}

\subsection{Stun weapons}\index{Stun weapons} Empty Fist, Truncheon, Spiked Gauntlet\label{listaarmistordimento}

\begin{itemize}
\item 4 points: An unaware opponent if hit with these weapons (during the surprise round) must make a Fortitude save DC 15 or be Slowed 1/1r.
\item 5 points: For each Critical Roll the opponent must make a Fortitude save at DC 13 or be weakened 1/1r.
\item 7 points: Double your Strength damage bonus. The 4-point skill's saving throw becomes 19.
\item 9 points: For each Critical Roll the opponent must make a Fortitude save at DC 17 or be weakened 1/1r.
\item 11 points: Your stun weapon does 1d6 more nonlethal damage. The skill's saving throw at 4 and 9 points becomes 23
\item 16 points: Whenever you hit an opponent with critical damage, a teammate in melee with that opponent can use a Reaction to make an attack against them.

\end{itemize}

\subsection{Thrownable Weapons} One-handed Axe, Javelin, Trident. Slingshot, Dagger\index{Throwing weapons}\label{listarmitiro}

You gain the \textbf{Devastating Shot} ability: you can throw one of your weapons with such force that it does two additional critical damage but your accuracy suffers -1d6 on the attack roll. It costs 2 Actions.



\begin{itemize}
\item 4 points: you have become extremely precise in throwing your weapon, you have a +1 to hit and a +1 to damage.
\item 5 points: the first Critical Shot you make on your opponent adds an additional critical hit.
\item 7 points: your skill allows you to have no downtime after throwing a weapon you can instantly draw another one without consuming actions.
\item 9 points: the first attack roll throws 2 weapons. You start with a -5 to attack roll.
\item 11 points: Reduce the range penalty above standard by 1d6.
\item 16 points: you have become extremely precise in throwing your weapon, you have a +4 to hit and a +4 to damage.
\end{itemize}


\subsection{Lethal weapons} Katana, Machete\index{Lethal weapons}\label{listarmiletali}

%\begin{center}
%\includegraphics[width=0.6\linewidth]{immagini/katana3.png}
%
%\emph{Katana}
%\end{center}


\begin{itemize}

\item 4 points: against surprised opponents add your Weapon Expertise to the damage.
\item 5 points: the first Critical Shot you make on your opponent adds an additional critical hit.
\item 7 points: increase the weapon's damage die by one step. If this causes the weapon to have the d8 as damage die, it also acquires EDX equal to 8.
\item 9 points: The first critical hit you land on your opponent adds two critical hits.
\item 11 points: EDX earnings. It is applied only by doing the maximum damage with the die, if the weapon already has an EDX (for example because with the previous bonus it reached 1d8 damage) this decreases by 1.
\item 16 points: increase the weapon's damage die by one step.
\end{itemize}


\subsection{Auctions} \index{Auctions}Javelin, Estoc, Trident, Halberd\label{listaarmiaste}


\begin{center}
\includegraphics[width=0.8\linewidth]{immagini/alabarda2.png}

\emph{Halberds}
\end{center}


\begin{itemize}

\item 4 points: if you make at least one critical roll with the attack roll you can leave the weapon in the opponent's body, penalizing him with -1 Dexterity. The weapon when removed deals critical damage.
\item 5 points: You can make an attack of opportunity against opponents who cross your melee zone, for free.
\item 7 points: You can use the long weapon in melee within one meter without penalty. The skill's damage at 4 points becomes 2 critical damage.
\item 9 points: The damage of the 4-point skill becomes 4 critical damage.
\item 11 points: the range if absent becomes 3 metres, if present you double it.
\item 16 points: using a Reaction you can follow the opponent while maintaining the current melee distance.

\end{itemize}


\subsection{Crossbows}\index{Crossbows}Light crossbow, Heavy crossbow, One-handed crossbow\label{listaarmibalestr}


\begin{center}
\includegraphics[width=0.9\linewidth]{immagini/arma-balestra.png}
\end{center}


\begin{itemize}

\item 4 points: You gain the Rapid Shot Skill.
\item 5 points: the first Critical Shot you make on your opponent adds an additional critical hit.
\item 7 points: each Action you dedicate to aiming, up to a maximum of 2, grants you +2 to hit.
\item 9 points: the first Critical Shot you make on your opponent adds two critical hits in addition, it does not cumulate with the advantage at point 5.
\item 11 points: The first hit against an opponent adds an additional critical hit.
\item 16 points: reduce the penalty for shooting beyond the standard range by 1d6.

\end{itemize}


\subsection{Spears} \index{Spears}Halberd, Urgrosh, Infantryman's Spear, Naginata, Spear Glaive, Spear, Brandistocco\label{listarmilance}

\begin{itemize}
\item 4 points: Used against a charge or while charging, as long as it has the Counter Charge ability, you deal additional critical damage if you hit.
\item 5 points: you can also use it against opponents at a distance of 1 meter without penalty.
\item 7 points: used against a charge or while charging, as long as it has the Counter Charge ability, you deal two additional critical damage
\item 9 points: Use 3 Actions. Make an attack roll at -5 and compare the result to the Defense of all creatures in melee to see if you hit them.
\item 11 points: Your spear's range becomes 3 meters.
\item 16 points: You use 2 Actions and make a single attack roll. If it hits, you deal 3 additional critical hits.
\end{itemize}

\begin{center}
\includegraphics[width=0.85\linewidth]{immagini/arma-asta.png}

\emph{1 Landsknecht skewer; 2 Pike; 3 Spear; 4 Hunting skewer; 5 Buttfire; 6 Glaive; 7 Partisan; 8 Halberd; 9 Halberd; 10 Roncone; 11 Mazzapicchio; 12 Berdica}\end{center}

\subsection{Rotating Balls} Flail, Heavy Flail, Double Flail, Spiked Chain, Whip\label{listaarmipallerotanti}


\begin{center}
\includegraphics[width=0.6\linewidth]{immagini/mazzafrusto.png}
\end{center}

\begin{itemize}
\item 4 points: if the attack roll is successful you can make a further TC (without consuming Actions) at -5 against an opponent in melee with you who is not the opponent already hit.
\item 5 points: If you hit your opponent twice in the round, the second attack roll generates additional critical damage.
\item 7 points: the impact of your shots is enough to stun enemies. If you hit your opponent with a Critical Roll, they will suffer -4 Defense until the end of your next round.
\item 9 points: You can use a Reaction and use your weapon to try to deflect an attack roll aimed at you onto a creature within melee range of you. 
\item 11 points: the precision and skill in swinging your weapon is such that it confuses the enemy's defense, you ignore the protection (Defense) given by the shield.
\item 16 points: You can use a Reaction and use your weapon to try to protect a creature in melee with you. The creature gets +4 Defense.
\end{itemize}

\subsection{Empty Fist} Punches and Kicks\index{Empty Fist}\hypertarget{Empty Fist}{}\label{listarmipugnonudo}

\textbf{Empty Fist}: Each time you take this skill the damage increases following this progression: 1d6 (taken the list 2 times), 1d8 (3), 2d6 (5), 2d8 (7), 2d10 (9), 3d6 (11), 3d8 (13), 3d10 (15), 4d6 (17).

The player can also decide to do non-lethal damage without incurring any penalties; he can apply the Strength or Dexterity value to the damage as he wishes.

\begin{itemize}
\item 1 point: your fists do lethal damage (1d4). You can use your Strength or Dexterity value on attack rolls and damage rolls.
\item 4 points: Wisdom of the Empty Hand. You can use Wisdom on Hit and Damage in place of Strength or Dexterity. Multiple attack penalties become -4 instead of -5.
\item 5 points: the Base Defense score goes from 10 to 11.
\item 9 points: solitary shot. If you make only one attack in the round and no move actions. If the hit hits, it adds 2 additional critical hits.
\item 11 points: You gain a bonus to hit and damage equal to double the characteristic used to determine this bonus.

\end{itemize}

See \hyperlink{magical equivalences}{Vulnerability, Resistance and Immunity} to find out how magical your strike is.

\subsection{Skull Breaker} \index{Skull Breaker}Scourge, Big Club, War Maul, War Hammer, Light Mace, Heavy Mace, Spiked Mace, Club
\label{listaarmirompicranio}

\begin{center}
\includegraphics[width=0.9\linewidth]{immagini/arma-mazza3.png}
\end{center}


\begin{itemize}
\item 4 points: you have become so skilled that you can control the force of your blows, you can do non-lethal damage without penalty to hit (otherwise -1d6 on attack roll).

You can choose to reduce the attack roll by 4 to increase the damage by 8 (does not stack with Power Strikes).
\item 5 points: the first Critical Shot you make on your opponent adds an additional critical hit.
\item 7 points: your shots daze the enemy. Each successful critical roll lowers the defense by 1 point. Duration 1 minute starting from the first successful critical hit.
\item 9 points: increase the weapon's damage die by one step.
\item 11 points: the first Critical Shot you make on your opponent adds two additional critical hits. It does not cumulate with the advantage in point 5.
\item 16 points: using a Reaction every time you hit with a Critical Roll you can make another attack roll with the same score against a different opponent as long as it is within melee range.

\end{itemize}

\subsection{Shields}\index{Shields} Light, Medium, Heavy Shields\label{listaarmiscudi}

You are a master in the use of shields, even as a weapon.

You can use the shield as a weapon, a small shield does 1d4 damage (B/T), a medium shield does 1d6 damage (B/T), a heavy shield does 1d8 damage (B/T).
You have no penalty for hitting with the shield, for you the shield is not an improvised weapon. This Weapon List does not have the 6 point bonus and the 18 point bonus common to other Weapon Lists.

Your technique effectively mixes defense and attack. You can throw your shield with a range of 20 feet.


\begin{itemize}
\item 1 point: you are proficient in all shield types. You are not constrained by the Strength 1 limit on Heavy Shields.
\item 2 points: the Defense bonus when using the shield increases by 1 and every 4 times you take the proficiency. Using the shield as a weapon does not cause you to lose the Defense bonus given by the shield.
\item 3 points: the Magical Proficiency penalty given by the shield decreases by one die
\item 4 points: the penalty on the attack roll decreases by 1.

\begin{center}
\includegraphics[width=0.9\linewidth]{immagini/scudotorre.png}
\emph{Henry Justice Ford. Heavy Shield}
\end{center}

\item 5 points: increases the damage category of the shield by 1 and every 4 additional points in the list (9,13,17..). 
\item 8 points: every ally adjacent (within 1 meter) to you has +1 Defense. You can throw your shield to defend a teammate by granting them +2 Defense, to use as a reaction. The shield falls to the ground where you defended your partner. You can throw your shield with a range of 30 feet. The Magical Proficiency penalty given by the shield decreases by one die.
\item 12 points: you can throw your shield as if it were a weapon with a range of 12 metres. If you hit and get a Critical Roll when throwing the shield it returns to your hands at the end of the round. Each ally adjacent (within 1 meter) to you has +2 Defense
\item 16 points: if an opponent makes at least your attack rolls and misses, you gain a shield attack against him in the round. Costs 1 Reaction.
\item 18 points: the thrown shield has a range of 18 meters and returns to your hands, if not prevented. This allows you to make multiple attacks even from throwing with the same shield. You can throw your shield to defend a teammate by granting them +4 Defense, to use as a reaction. The shield falls to the ground where you defended your partner. 

These bonuses cannot be applied if you use more than one shield.

\end{itemize}

\subsection{Axes and Hatchets}\index{Axes and Hatchets} One-handed Axe, Battle Axe, Hammer Axe, Double Great Axe, Sornelian natural attacks\label{listaasce}

\smallskip

Call \emph{Axes} in Table: Weapon List

\begin{center}
\includegraphics[width=0.9\linewidth]{immagini/scurieaccette.png} 
\end{center}

\begin{itemize}

\item 4 points: The fury of your attacks is such that you gain +2 to on-hit damage.
\item 5 points: the first Critical Triro you perform on your opponent adds an additional critical hit.
\item 7 points: the wounds you cause are so deep that they cause bleeding. Each successful attack increases your bleed by 1, up to a maximum of bleed 5.
\item 9 points: each critical hit you cause increases the Bleeding by 2, up to a maximum of 10.
\item 11 points: the wounds you cause are so deep that you cause a lot of bleeding. The maximum Bleeding value increases to 15.
\item 16 points: consume 3 Actions, make a single attack roll that you compare against all creatures in a cone equal to your movement to see if you hit them. At the end of the attack you are at the bottom of the cone.

\end{itemize}

\subsection{Swords}\index{Swords} Short Sword, Long Sword, Two-Handed Greatsword, Bastard Sword, Double-Bladed Sword, Broad Sword, Katana, Double-Bladed Sword

\begin{itemize}

\item 4 points: Your mastery of sword technique gives you +1 to damage and attack rolls.
\item 5 points: the first Critical Shot you make on your opponent adds an additional critical hit.
\item 7 points: Your mastery of sword technique gives you +2 to damage and attack rolls.
\item 9 points: the first successful hit in the round adds a critical hit.
\item 12 points: you have reached the pinnacle of mastery with the sword your blows are precise and difficult to predict you get +1 to damage, attack roll and defense. The EDX of the sword if present is lowered by 1.
\item 16 points: Your sword's damage die increases by one category.

The hand not holding the sword must be free.

\end{itemize}

\begin{center}
\includegraphics[width=0.95\linewidth]{immagini/arma-tipi-di-spade.png}

\emph{A Saber, B Scimitar, C One-handed sword, D Broadsword, D Rapier, E Rapier, F Longsword, G One-and-a-half-handed or bastard sword, H Two-handed broadsword}
\end{center}


\subsection{Swords and Shields}\index{Swords and Shields} Short Sword, Long Sword, Large Sword, Small Shield, Medium Shield\label{listaarmispadescudi}

\begin{itemize}

\item 4 points: your mastery of the sword and shield technique gives you +1 to Defense and Attack Rolls.
\item 5 points: if you score two consecutive hits with the sword you can make an attack roll, without accumulating further multi-attack penalties, with the shield by consuming a reaction.
\item 7 points: your mastery of the sword and shield technique gives you +2 to Defense and Attack Roll.
\item 9 points: Using a Reaction you can use your shield to protect a creature in melee with you. His Defense increases by 2 points.
\item ​​11 points: Your sword's damage die increases by one category.
\item 16 points: add the Shield Defense value to the Reflex saving throws.

\end{itemize}

The character must hold the sword in one hand and the shield in the other.


\subsection{Simple Weapons} Dagger, Light Mace, Club, Spiked Mace, Staff, Crossbow (Light), Javelin.\index{Simple Weapons}\hypertarget{simple.weapons}{}\label{listaarmisemplice}

\medskip

This subdivision can also be chosen by those who have not assigned points to Weapons Proficiency. This Weapon List does not grant specific bonuses.

\subsection{Additional Weapons List of Weapons}\index{Additional Weapons List of Weapons}\label{listaarmiinpiuliste}

When a character uses a weapon present in multiple known Weapon Lists, he can apply only one combat technique (one Weapon List) per opponent, he does not accumulate the advantages of any other lists.

By using 2 Actions he can concentrate and move on to using the bonuses resulting from the application of a different Weapon List.

\end{multicols}

\vfill


%\begin{center}
%\includegraphics[width=0.4\linewidth]{immagini/brancastle.png}
%
%\emph{Bran Castle, Transylvania}
%\end{center}

\pagebreak

\subsection{Optional - List of Weapon Maneuvers}\hypertarget{List of Weapon Talents}{}\label{elencotalentiarmi}\index{List of Weapon Maneuvers}

\begin{changemargin}{0.3cm}{0.3cm}\begin{enfasi}{
Honesty and Justice, Heroic Courage, Compassion, Kind Courtesy, Complete Sincerity, Honor, Duty and Loyalty (The Seven Principles of Bushido)
}\end{enfasi}\end{changemargin}

\begin{multicols}{2}

The more competent the character becomes with weapons, the more he is able to exploit attack opportunities and carry out weapon maneuvers. Whenever the character makes at least two weapon attacks in the round and \textbf{neither of them hits}, it is possible to consult the Weapon Maneuvers list to understand which maneuver can be used.

Each Maneuver has indicated which situation activates it (Activator) and which is the Effect. A Critical Effect can also be indicated, i.e. the Effect that occurs when a critical failure is obtained in at least one attack roll. As long as the Trigger is always respected, the player can choose between the Effect and the Critical Effect.

The Trigger can specify an odd or even value that is compared to the attack roll.

The Weapon Maneuvers are grouped by level, i.e. the minimum Weapon Proficiency score to be able to use those maneuvers, the player can choose between all the Weapon Maneuvers accessible to him and that can be activated.\\


\textbf{Weapon Maneuvers level 6}

\medskip

Name: \textbf{Missed Opportunity}\\
Trigger: You missed\\
\emph{Effect}: You can drink a potion held on your belt\\
\emph{Critical Effect}: the potion can be administered to a companion within melee range.

\smallskip

Name: \textbf{Real fake}\\
Trigger: You rolled a draw\\
\emph{Effect}: you didn't want to miss it but at least it was a feint. Until the end of your next round, add your Intelligence or Wisdom score to your Defense against the same opponent\\
\emph{Critical Effect}: on your next attack add your Intelligence and Wisdom scores to your attack roll against the same opponent

\smallskip

Name: \textbf{Smart Shot}\\
Trigger: You rolled a draw\\
\emph{Effect}: the companion at your side who fights against the same opponent as you gains a bonus on attack rolls equal to your Intelligence by the end of his next round\\
\emph{Critical Effect}: as above but also applies to an additional companion.

\smallskip

Name: \textbf{Continue minor}\\
Trigger: you rolled an odd\\
\emph{Effect}: Compare the attack roll of your last attack with a creature within melee range of you, if it is enough to hit it you deal damage equal to your Strength\\
\emph{Critical Effect}: compare the attack roll of your last attack with a creature within melee range of you, if it is enough to hit it you deal damage equal to twice your Strength

\smallskip

Name: \textbf{Distracted by noise}\\
Trigger: you rolled an odd\\
\emph{Effect}: the clangor of battle distracts an opponent, choose a grappled companion, he frees himself if he is Grabbed or Blocked\\
\emph{Critical Effect}: as above, he can also take a step (1 meter) in the direction he prefers \\


\textbf{Weapon Maneuvers level 8}

\medskip

Name: \textbf{Deep Breath}\\
Trigger: You missed\\
\emph{Effect}: You focus too much and miss an opportunity to hit, but you gain +2 to hit on all melee attacks by the end of your next round\\
\emph{Critical Effect}: as above and if you hit you get additional critical damage

\smallskip

Name: \textbf{Unbalanced}\\
Trigger: You missed with an odd\\
\emph{Effect}: Your trip caused you to miss but you gain +2 to Defense until the end of your next round\\
\emph{Critical Effect}: in your next round you have one less Action, but the first melee attack you make automatically misses

\smallskip

Name: \textbf{Perplexed}\\
Trigger: You missed with an odd\\
\emph{Effect}: you can't decide how and where to hit him. You make a knowledge check to better understand your opponent\\
\emph{Critical Effect}: even one of your companions fighting against the same creature gets the chance to make the same check

\smallskip

Name: \textbf{Misstep}\\
Trigger: You missed with a tie\\
\emph{Effect}: you've made a mistake. By the end of your next round the terrain is considered difficult, you have +4 to hit\\
\emph{Critical Effect}: By the end of your next round you cannot move. If you hit you get two critical damage

\smallskip

Name: \textbf{I missed!!!}\\
Trigger: You missed with a tie\\
\emph{Effect}: It was all a ploy, you missed on purpose. By the end of your next round you can make one additional attack without stacking multiple attack penalties.\\
\emph{Critical Effect}: check your attack roll with that of another melee opponent, if you hit you also cause critical damage\\

\textbf{Weapon Maneuvers level 10}

\medskip

Name: \textbf{On the side}\\
Trigger: You missed\\
\emph{Effect}: You have moved to your opponent's flank. Move one meter around the opponent\\
\emph{Critical Effect}: move up to 3 meters around the opponent, perform one less Action by the end of the next round

\smallskip

Name: \textbf{Opening}\\
Trigger: You missed with an odd\\
\emph{Effect}: you missed to allow a teammate to hit better. A companion who attacks the same opponent as you gains +4 to attack rolls by the end of your next round. \\
\emph{Critical Effect}: two companions gain the opening described in Effect and until the end of your next round, you have -4 to attack rolls

\smallskip

Name: \textbf{Bravado}\\
Trigger: You missed with an odd\\
\emph{Effect}: The opponent is intimidated by your combat mastery. By the end of your next round, the first attack you make gets a -1d6 penalty\\
\emph{Critical Effect}: as above but -2d6, by the end of your next round you have -4 to attack rolls

\smallskip

Name: \textbf{Smooth}\\
Trigger: You missed with a tie\\
\emph{Effect}: You just missed but was enough to hurt your opponent. Opponent increases Bleeding rating by 1\\
\emph{Critical Effect}: as above but Bleeding is 2, you hit yourself with the weapon and cause damage equal to your Strength

\smallskip

Name: \textbf{Testing your strengths}\\
Trigger: You missed with a tie\\
\emph{Effect}: You preferred to evaluate your opponent's capabilities. The first attack roll that hits by the end of your next round automatically deals 1 critical damage\\
\emph{Critical Effect}: as above but 2 critical damage, next round you have one less Action.\\

\smallskip

\textbf{Weapon Maneuvers level 12}

\medskip

Name: \textbf{Tenacious}\\
Trigger: You missed\\
\emph{Effect}: don't give up and insist. Each attack you make against this opponent by the end of the next round gains a cumulative +2 to hit until you miss\\
\emph{Critical Effect}: Each attack that hits by the end of the next round you make against this opponent gains cumulative critical damage, you have -2 on the attack roll

\smallskip

Name: \textbf{Persevere}\\
Trigger: You missed with an odd\\
\emph{Effect}: The first attack within the next round will automatically miss, the next attack will cause 2 additional critical damage if it hits\\
\emph{Critical Effect}: as above, but 3 critical damage

\smallskip

Name: \textbf{Battlecry}\\
Trigger: You missed with an odd\\
\emph{Effect}: you missed, it's true, but you took the opportunity to encourage your teammates. By the end of the round all your companions gain +1d6 to attack rolls on their first attack.\\
\emph{Critical Effect}: as above but +2d6, but take one less Action on your next round

\smallskip

Name: \textbf{Eureka}\\
Trigger: You missed with a tie\\
\emph{Effect}: the blow was used to understand how to hit him. The first successful attack by the end of your next round deals critical damage with the weapon die maximized\\
\emph{Critical Effect}: the next round you perform one less Action, until the end of your next round you and a companion you draw on the first attack, if successful, add 1 critical damage

\smallskip

Name: \textbf{Savage Attack}\\
Trigger: You missed with a tie\\
\emph{Effect}: Your fury is such that you hit something anyway. Compare your attack roll with a creature within melee range of you\\
\emph{Critical Effect}: until the end of the next round you have -4 Defense, but you get +1d6 on attack rolls and critical damage if you hit

\end{multicols}


\begin{changemargin}{0.3cm}{0.3cm}\begin{narratore}
Players, in agreement with the Storyteller, can create their own personalized Weapon Maneuvers based on the character's story and style.
\end{narratore}\end{changemargin}

\vfill

\begin{center}
%\includegraphics[width=0.12\linewidth,angle=90]{immagini/Bushido_Calligraphy.png}
\includegraphics[width=0.082\linewidth]{immagini/Bushido_Calligraphy.png}

\emph{Transcription in kanji of} bushido
\end{center}



\pagebreak

\section{Feats}\index{Feats}\hypertarget{Feats}{}\label{abilita}

\begin{changemargin}{0.3cm}{0.3cm}\begin{enfasi}{Martyrdom is the only way for a man to become famous if he has no Feat (George Bernard Shaw, The Devil's Disciple)}\end{enfasi}\end{changemargin}\medskip

\begin{multicols}{2}

\lettrine[lines=2, lhang=0.33, loversize=0.25, findent=1.5em]{T}{he} Feats are peculiar abilities, the result of training or particular gifts. Feats always have a practical effect.

Feats make up a good part of what the character can do, they must be chosen with attention and care. It is by choosing the Feats that you establish the character's style and capabilities, whether you want him to be more of a warrior or a magician or a healer... or any combination and peculiarity.

\textbf{At first level you get two Feats}. Subsequently you take a Feat at all levels except 5,10,15,20. this can be a Feat already known or a new Feat learned during adventures.

It is possible that Requirements are indicated under the name of the Feat, in this case they must be respected to take the Feat in question.
Any subsequent requirements are indicated from time to time.

Don't take Feats based on power, strength, combination with others but because they are in line with the character's history.
Choosing a jumble of Feats just because they are strong does not make a character powerful but unbalanced, don't be a power-player at any cost.

\medskip

\textbf{Feats must be taken based on the character's evolutionary path, based on what has been experienced and learned during the adventures.}

\medskip

It is possible to change a chosen Feat, still respecting the requirements, by retraining for at least 4 months for 4 hours a day.

The abilities provided by the Feats unless otherwise described are cumulative or if it is the same bonus the greater one applies. If not explicitly stated, an Ability cannot be taken multiple times.

\subsection{Saving Throws and Feats}\label{tirisalvezzaedabilita}

Each Feat, except those that directly modify saving throws, grants bonuses to saving throws that stack with each other, even when the same Feat is taken multiple times.

When you choose a Feat, also pay attention to which Saving Throws it increases!

\subsection{Add new Feats}\label{aggiungereabilita}

This list can never be exhaustive given the imagination of the players! However, try to understand if what the player wants is a Feat or Competence, to have an ability or to know how to do something particular.
Evaluate the prerequisites and the advantages it grants carefully, always try to be balanced, rather grant advantages to scaling, i.e. taking the Feat several times.

Also remember to note the bonuses related to saving throws. Usually a concrete and practical Feat grants a bonus of +3 divided between 2 saving throws, a more generic Feat grants 2 points to be divided between a single saving throw or two.

\subsection*{Adept of Magic}\index{Adept of Magic}\hypertarget{magic school}{}\label{adeptodellamagia}

\textbf{Requirement}: Magical Proficiency 1

\textbf{Saving Throws}: +1 to two saving throws of your choice.

Through this Feat you know a new Magic List.

A spellcaster can take the Magic Adept Feat multiple times and apply it to a new Magic List or to an already known one.

By taking this Feat several times and always selecting the same School it is possible to access higher spellcasting levels.

\subsection*{Phoenix Wings}\index{Phoenix Wings}\label{alidellafenice}

\textbf{Requirement}: Empty Fist List 2, Silver Crane 1

\textbf{Saving Throws}: +2 Reflexes, +1 Fortitude

Your fighting style emphasizes long range strikes such as flying kicks and punches.

The \textbf{first time} you take this Feat your melee distance with the Empty Fist List becomes 2 meters.

The \textbf{second time} you take this Feat, requirement List Empty Fist 6, Silver Crane 3, Iron Fist 1, your melee distance becomes 3 meters.

The \textbf{third time} you take this Feat, requirement List Empty Fist 9, Silver Crane 4, Iron Fist 2, your melee distance becomes 4 meters.

Until the opponent reaches your character's melee range, the character will have a +2 bonus to hit, as if he were using a long weapon.

\subsection*{Extension}\index{Extension}\label{Allungo}

\textbf{Requirement}: Weapon Proficiency +2

\textbf{Saving Throws}: +1 Will, +2 Fortitude

You use a Reaction in conjunction with your Attack Action to gain 2m reach with your strike. If the opponent does not have long weapons or 2m reach you also gain +2 to attack rolls.

\subsection*{Animalia}\index{Animalia}\label{amimalia}

\textbf{Requirement}: Follower or Devotee of Ephrem or Shayalia, Magical Expertise 2.

\textbf{Saving Throws}: +2 Will, +1 Fortitude

You gain the ability to transform into a known creature. Cost 2 Actions.

Your healing spells also work on normal and magical Animals or Plants.

The \textbf{first time} you take this Feat you can transform into a creature with these characteristics:\\
\textbf{Type of Creatures}: Beasts

\textbf{Transformation time}: 1 minute per sum Shared with the Patron, with minimum use of 1 minute

\textbf{Challenge Rating}: within one third of your Magical Expertise score + half the times you took the Animalia Feat.

\textbf{Characteristics}: the physical ones, Defense, Saving Throws and attack forms are of the animal. The hit points remain those of the character

\textbf{Spells}: You cannot cast spells in the new form.

\textbf{Equipment}: Equipment is absorbed into the new form but none have any effect.\\

The \textbf{second time} you take this Feat you can also transform into a creature with these characteristics:\\

\textbf{Requirement}: Magical Proficiency 4

\textbf{Type of Creatures}: Plants and Slimes

\textbf{Transformation time}: 1 minute per sum Shared with the Patron, with minimum use of 1 minute

\textbf{Challenge Rating}: within one third of your Magical Expertise score + half the times you took the Animalia Feat.

\textbf{Characteristics}: the Character chooses whether the Physical Characteristics, Defense, Saving Throws are their own or those of the animal. The hit points remain those of the character

\textbf{Spells}: You cannot cast spells in the new form

\textbf{Equipment}: Equipment is absorbed into the new form. The magic one has no effect. Armor and Shields apply the magical bonus to the creature's Defense. Items' spell-like abilities cannot be activated.\\

The \textbf{third time} you take this Feat you can also transform into a creature with these characteristics:\\

\textbf{Requirement}: Magical Proficiency 10

\textbf{Type of Creatures}: Elementals

\textbf{Transformation time}: 1 minute per sum Shared with the Patron, with minimum use of 1 minute

\textbf{Challenge Rating}: within one third of your Magical Expertise score + half the times you took the Animalia Feat.

\textbf{Characteristics}: the Character chooses whether the Physical Characteristics, Defense, Saving Throws are their own or those of the animal. The hit points remain those of the character

\textbf{Spells}: You can cast spells in the new form as long as they have only Verbal components

\textbf{Equipment}: Equipment is absorbed into the new form. The magical one continues to have effect if possible. Armor and Shields apply the magical bonus to the creature's Defense, and item spell-like abilities cannot be activated.\\

The \textbf{fourth time} you take this Feat you can also transform into a creature with these characteristics:\\

\textbf{Requirement}: Magical Proficiency 16

\textbf{Type of Creatures}: Monstrosities

\textbf{Transformation time}: 1 minute per sum Shared trait with the Patron, with minimum use of 1 minute.

\textbf{Challenge Rating}: within one third of your Magical Expertise score + half the times you took the Animalia Feat.

\textbf{Characteristics}: the Character chooses whether the Physical Characteristics, Defense, Saving Throws are their own or those of the animal. The hit points remain those of the character.

\textbf{Spells}: you can cast spells in the new form as long as they have only Verbal and Somatic components

\textbf{Equipment}: Equipment is absorbed into the new form. The magical one continues to have effect if possible. Armor and Shields apply the magical bonus to the creature's Defense and any magical abilities can be activated.\\

\textbf{Basic rules for transformation}

It costs 2 Actions to change shape and before switching from one form to another it is necessary to return to normal form.

The character retains his own Traits, personality, Feats (but the new form does not necessarily allow him to use them) and mental characteristics.

If the creature has a proficiency that the character also has and the creature's bonus is higher than the character's, then use the creature's bonus instead of your own. If the creature has additional or lair actions, the character cannot use them.

Any actions requiring his hands are limited to the capabilities of his new form. The transformation does not interrupt the character's concentration on a spell he has already cast and does not prevent him from performing actions that are part of a spell already cast, such as Call Lightning.

The attack forms are always those of the creatures.

He acquires the characteristics and abilities of the new form, such as senses, movement, languages ​​(but it is not certain that he can speak other languages ​​besides that of the animal). At the character's discretion, the acquired form may not be complete but for example only the claws, the face (and bite)...

When transformed you can channel your Magic Points to improve the transformation, for each Magic Point consumed in the round you get a +1 to Attack Rolls, damage with attacks, Defense and Saving Throws. The ability must be declared at the start of the round as an Immediate Action that lasts until the start of your next round. 


\subsection*{Pet / Pet}\index{Pet}\label{famiglio-abilita}

\textbf{Requirement}: Magical Proficiency 1

\textbf{Saving Throws}: +1 Will, +1 Fortitude

You earn a natural animal. This pet has a Challenge Rating equal to half your Wisdom. You can teach your pet basic actions and make him do simple tasks.

The \textbf{second time} you take this Feat you earn a \hyperlink{familiar}{Familiar} (p. \pageref{family}).


\begin{center}
\includegraphics[width=0.9\linewidth]{immagini/animalia3.png}
\emph{Henry Justice Ford}
\end{center}

\subsection*{Armor of the Devoted}\index{Armour of the Devoted}\label{armaturadeldoveto}

\textbf{Requirement}: Sum of Traits in common 2 (sum of Traits in common with the Patron), being Devoted or Follower

\textbf{Saving Throws}: +2 Will, +1 Reflexes

Constant training with your armor allows you to wear light armor without having to make a Magic Test.

The \textbf{second time} you take this Feat, common Trait sum requirement 6, perform the Magic Test without additional dice given by medium armor.

The \textbf{third time} you take this Feat, trait sum requirement 8, perform the Magic Test with only 1 additional die given by heavy armor.

The \textbf{fourth time} you take the Feat, trait sum requirement in common with the Patron 12, perform the Magic Test without additional dice given by heavy armor.

\subsection*{Armor of the Enchanted Mountain}\index{Armor of the Enchanted Mountain}\label{armaturamontagnaincantata}

\textbf{Requirement}: Empty Fist weapon list, Weapon Proficiency 1, Magical Proficiency 1, Wisdom 1

\textbf{Saving Throws}: +2 Fortitude, +1 Will

Constant training in spirit and body allows you to harden your skin and make it more difficult to hurt. To take advantage of these bonuses you do not have to bring armor or shields or objects that improve Defense.

The \textbf{first time} you take this Feat your Defense is 10 + Dexterity + 1/3 points in Empty Fist.

The \textbf{second time} you take this Feat you can add your Wisdom value to your Defense, up to a maximum of 1 point (even if you have higher Wisdom).

The \textbf{third time} you take this Feat you can add the entire value of Wisdom to Defense.

The \textbf{fourth time} you take this Feat, Empty Fist requirement 12, your Defense is 10 + Dexterity + Wisdom + 1/2 of the points in Empty Fist.

The \textbf{fifth time} you take this Feat, Empty Fist requirement 16, you gain damage resistance (DR) of 5/-

If you are surprised you only have -2 to Defense and Reflex saving throws.

\subsection*{Horse Archer}\index{Horse Archer}\label{arcereacavallo}

\textbf{Requirement}: Weapons Proficiency 1

\textbf{Saving Throws}: +1 Reflex, +1 Fortitude

the penalties of shooting arrows from horseback decrease by 2 each time you take this Feat.

The standard penalties are -4 and -6 depending on whether you trot (movement x2) or canter (movement x3)

%\begin{center}
%\includegraphics[width=0.9\linewidth]{immagini/horsearcher.png}
%\emph{Assyrian Archer}
%\end{center}


\subsection*{Weapon Focus}\index{Weapon Focus}\label{armafocalizzata}

\textbf{Requirement}: Weapons Proficiency 1

\textbf{Saving Throws}: +1 Reflexes, +1 Fortitude

Choose a weapon. You gain a +1 to Initiative and Attack Rolls when using this weapon you have proficiency with.

\subsection*{Weapon Artist}\index{Weapon Artist}\label{artistadellafuca}

\textbf{Requirement}: Weapons Proficiency 2

\textbf{Saving Throws}: +1 Will, +1 Fortitude

Choose a Weapon List, on these weapons you get +1 to hit.

The Feat can be taken multiple times and the Weapon List score must be 4 times the score for this Feat.

If you take \textbf{4 times} this Feat on the same Weapon List the hit bonuses are reduced to +1, instead of +4, but you make two attack rolls for the first two attacks of the round and choose the roll to keep.

\subsection*{Swirling Attack}\index{Swirling Attack}\label{attaccoturbinante}

\textbf{Requirement}: Weapons Proficiency 12

\textbf{Saving Throws}: +2 Reflexes, +1 Fortitude

Using 3 Actions you can make a single attack (with a 1d6 penalty on attack rolls) against all melee opponents around you.

\subsection*{Magic Battery}\index{Magic Battery}\label{batteriamagica}

\textbf{Requirement}: Magical Proficiency 3

\textbf{Saving Throws}: +2 Will, +1 Fortitude

You have a particular connection to the magic that lingers Yeru.

The first time you take this Feat you increase your available Magic points by 3.

Subsequent times you increase your Magic Points by a value equal to the previous increase +1.

%\begin{center}
%\includegraphics[width=0.9\linewidth]{immagini/Historia_Mundi_Naturalis.png}
%\emph{Woodcut illustration from an edition of Pliny the Elder's Naturalis Historia (1582)}
%\end{center}

\subsection*{Extended Battery}\index{Extended Battery}\label{batteriaestesa}

\textbf{Requirement}: Magical Proficiency 1

\textbf{Saving Throws}: +1 Fortitude, +1 Will

You can better handle the mental stress of casting spells.

The \textbf{first time} that you take this Feat has the effects of \hyperlink{when they have few magic points}{When they have few magic points} (see page \pageref{spells when they have few magic points}) are activated at 60\% of the use of Magic Points.

The \textbf{second time} you take this Feat the effects of \emph{When you have few magic points} are activated at 70\% of the use of Magic Points.

The \textbf{third time} you take this Feat the effects of \emph{When you have few magic points} are activated at 80\% of the use of Magic Points.

The \textbf{fourth time} you take this Feat the effects of \emph{When you are low on magic points} no longer apply.

\subsection*{Powerful blows}\index{Powerful blows}\label{colpipoderosi}

\textbf{Requirement}: Weapons Proficiency 1

\textbf{Saving Throws}: +2 Fortitude

Your style emphasizes powerful shots.

You gain +1 to damage with a Weapon List.

\subsection*{Sneak Strike}\index{Sneak Strike}\index{Back Attack}\label{attaccoallespalle}\label{colpofurtivo}

\textbf{Requirement}: Weapons Proficiency 3

\textbf{Saving Throws}: +2 Reflexes, +1 Will

When the opponent is melee attacked from behind, the first successful attack of the combat with a melee weapon causes two additional critical damage.

The \textbf{second time} you take this Feat, Weapon Proficiency requirement 6, causes 3 additional critical damage.

The \textbf{third time} you take this Feat, Weapon Proficiency requirement 10, causes 4 additional critical damage.

The \textbf{fourth} that you take this Feat, Weapon Proficiency requirement 12, causes 5 additional critical damage.

\subsection*{Weakening Strike}\index{Weakening Strike}\label{colpoindebolente}

\textbf{Requirement}: Sneak Strike 3, Weapon Proficiency 12

\textbf{Saving Throws}: +2 Reflexes, +1 Will

Weakening Strike is an advanced form of stealth strike. Each Weakening Strike lowers Strength or Dexterity (player's choice) by the number of times you have taken Sneak Strike.

The opponent is allowed a Fortitude saving throw with a DC equal to the attack roll. It causes the additional damage of the Sneak Strike or the loss of ability points.


\begin{center}
\includegraphics[width=0.8\linewidth]{immagini/teseo.png}

\emph{Henry Justice Ford - Backstab!}
\end{center}

\subsection*{Death Blow}\index{Death Blow}\label{colpomortale}

\textbf{Requirement}: Weapons Proficiency 5

\textbf{Saving Throws}: +2 Reflexes, +1 Will

Make the attack roll with a -1d6 penalty, if you hit you cause 3 critical damage. Subsequent attack rolls start at -10 to hit.


\subsection*{Paralyzing Shot}\index{Paralysing Shot}\label{colpoparalizzante}

\textbf{Requirement}: Weakening Strike, Sneak Strike 4, Weapon Proficiency 18

\textbf{Saving Throws}: +2 Reflexes, +1 Fortitude

You dedicate 2 Actions per Round, for 5 rounds, to studying an opponent you can threaten. In the sixth round using 2 Actions you make a melee or ranged attack. The opponent must make a Fortitude save with a DC equal to the roll or be paralyzed for 3d6 rounds.

\subsection*{Fighting Blindly}\index{Fighting Blindly}\label{combattereallacieca}

It is the ability to attack opponents who are not clearly perceptible.

\textbf{Requirement}: Awareness 2

\textbf{Saving Throws}: +2 Reflexes, +1 Will

An opponent with light cover gets no bonus to Defense, with medium cover he has a +2 to Defense, with full cover he has a +6 to Defense.

An invisible melee attacker gains no advantage when hitting the character in melee. 

The \textbf{second time} you take the Feat, Awareness requirement at 3, reduces the bonus to Defense from creatures with full cover by an additional two.

You do not need to make Acrobatics checks to move at full speed while blinded.

The penalty to attack rolls against invisible creatures is -2.

\emph{Zatoichi Level}, the \textbf{third time} you take the Feat, Awareness requirement at 5, in melee an invisible creature has no advantage against you nor do you have a penalty against it.

\subsection*{Two-Weapon Fighting}\index{Two-Weapon Fighting}\index{Two-Weapon Fighting}\label{duearmi}

\textbf{Requirement}: Dexterity 2, Strength 1, Weapon Proficiency 2

\textbf{Saving Throws}: +2 Reflexes, +1 Fortitude

The \textbf{first time} that you take this Feat the constant and continuous training allows you to reduce the multiattack penalty given by the attack with the secondary weapon. When you attack with your secondary weapon you gain a -4 hit penalty instead of -5 if the weapon is light.

\textbf{Requirement} Dexterity 3, Weapon Proficiency 12

The \textbf{second time} if the secondary weapon is not light you do not accumulate the additional -3 to hit.

\textbf{Requirement} Weapon Proficiency 18

The \textbf{third time} the first attack made with the secondary weapon does not cumulate the penalty of multiple attacks.

\subsection*{Concentrated}\index{Concentrated}\label{concentratp}

\textbf{Requirement}: Magical Proficiency 2

\textbf{Saving Throws}: +1 Fortitude, +1 Will

Choose a Magic List, the saving throw DC of your spells in that list increases by 1.

The Feat can be taken multiple times on the same Magic List or on other lists and the total must be less than CM/4.

\subsection*{Instinctive knowledge}\index{Instinctive knowledge}\label{conoscenzaistintiva}

\textbf{Requirement}: Knowledge 1

\textbf{Saving Throws}: +2 Will, +1 Fortitude

You never forget an enemy.

You have an instinctive ability to remember and evaluate an enemy. When you take this Feat you can make a \hyperlink{recognize monsters}{Recognize a Monster} check (page \pageref{recognize monsters}) using a Reaction.

\subsection*{Create Magic Items}\index{Create Magic Items}\label{creaoggettimagici}

\textbf{Requirement}: Magical Proficiency 6

\textbf{Saving Throws}: +1 Fortitude, +1 Will

Through this Feat the caster is able to infuse a spell up to level 3 into a magical object.

\subsection*{Create Greater Magic Items}\index{Create Greater Magic Items}\label{creaoggettimagicisuperiori}

\textbf{Requirement}: Create Magical Items, Magical Expertise 12

\textbf{Saving Throws}: +1 Fortitude, +1 Will

Through this Feat the caster is able to infuse a spell up to level 5 into a magical object.

\begin{center}
\includegraphics[width=0.8\linewidth]{immagini/oggettimagiciuomo.png}

\emph{Henry Purcell - King Arthur}
\end{center}

\subsection*{Create Wonderful Magic Items}\index{Create Wonderful Magic Items}\label{creaoggettimagicimeravigliosi}

\textbf{Requirement}: Create Greater Magical Item, Magical Expertise 16

\textbf{Saving Throws}: +1 Fortitude, +1 Will

Through this Feat the caster is able to infuse a spell up to level 8 into a magical object.


\subsection*{Create Mythical Magic Items}\index{Create Mythical Magic Items}\label{creaoggettimagicimitici}

\textbf{Requirement}: Create Wondrous Magical Item, Magical Expertise 18

\textbf{Saving Throws}: +1 Fortitude, +1 Will

Through this Feat the caster is able to infuse a spell up to level 9 into a magical object.

\subsection*{Loaded dice}\index{Loaded dice}\label{daditruccati}

\textbf{Requirement}: Magical Proficiency 6

\textbf{Saving Throws}: +1 Fortitude, +1 Reflex

You can increase a die in the Magic Test by 1, within the value of 6.

\subsection*{Coordinated Damage}\index{Coordinated Damage}\label{dannocoordinato}

\textbf{Requirement}: Weapon Proficiency 8, Wisdom 2

\textbf{Saving Throws}: +2 Will

Your experience in managing allies allows you to maximize the effectiveness of your attacks.

You can coordinate the attacks of two of your allies, who are within melee range of each other, so that the damage caused by one hits the other's enemy and vice versa. It costs 2 Actions to perform this coordination.

The \textbf{second time} you take this Feat, requirement Weapon Proficiency 4, Intelligence 2, you can coordinate and exchange the damage of three allies as long as they are within melee distance of each other. Cost 2 Actions.

Attack rolls must be successful in order to apply damage to the other opponent.

\subsection*{Blade Dance}\index{Blade Dance}\label{danzadellalama}

\textbf{Requirement}: Weapons List: Graceful Weapons at 2, Dexterity or Charisma 1, Perform 1

\textbf{Saving Throws}: +2 Reflexes, +1 Fortitude

When using Graceful Weapons you can replace Strength damage alone in melee attacks with half your Charisma or Dexterity value.

The \textbf{second time}, requirement Weapons Grace 4, Perform 3, that you take the Feat you can use Charisma as a weapon damage modifier, ignoring Strength damage.

The \textbf{third time}, requirement Weapons Grace 7, Perform 5, that you take the Feat you can use Dexterity or Charisma as a weapon damage modifier, ignoring Strength damage.

The second and third benefits are not cumulative.

\subsection*{Typist}\index{Typist}\label{dattilografo}

\textbf{Requirement}: Magical Proficiency 1

\textbf{Saving Throws}: +1 Fortitude, +1 Will

You are extremely quick at copying new spells into your Tome of Magic. The time to copy a spell increases from 1 hour to 30 minutes per page (a spell takes up a number of pages equal to its level). The cost in inks goes from 10 gp per page to 5 gp per page.

\subsection*{Deciphering magical writings}\index{Deciphering magical writings}\label{decifrarescrittimagici}

\textbf{Requirement}: Magical Proficiency 1

\textbf{Saving Throws}: +1 Fortitude, +1 Will

You have a +1d6 bonus to understanding the contents of a scroll and casting the spell within. The bonus also applies to the check to copy a spell into your Tome of Magic.

\subsection*{Defend Mount}\index{Defend Mount}\label{difenderecavalcatura}

\textbf{Requirement}: Ride 1

\textbf{Saving Throws}: +1 Fortitude, +1 Reflex

Whenever your mount is hit, you can make a Ride check to negate the hit. Your Ride check must be greater than your opponent's attack roll

The Feat can only be used once per round, for a single attack, it costs the Reaction.

\subsection*{Defense ready}\index{Defense ready}\label{difesapronta}

\textbf{Requirement}: Weapons Proficiency 2

\textbf{Saving Throws}: +2 Reflexes

You are always alert and careful when you risk your life. You have +4 Defense against attacks of opportunity, from behind, or from flanking.

\subsection*{Distill potions}\index{Distill potions}\label{distillarepozioni}

\textbf{Requirement}: Magical Proficiency 1

\textbf{Saving Throws}: +1 Fortitude, +1 Will

Proficiency in brewing potions.

You gain a +1d6 bonus to Herbal Knowledge and to distilling and creating potions and natural poisons.

\subsection*{Double portion}\index{Double portion}\label{doppiaporzione}

\textbf{Requirement}: Two-Weapon Fighting, Weapon Proficiency 4

\textbf{Saving Throws}: +2 Fortitude, +1 Reflex

Constant training with two weapons allows you to fully apply the Strength damage bonus to your secondary weapon.

\subsection*{Psychic Energy}\index{Psychic Energy}\label{energiapsichica}

\textbf{Requirement}: Strength 1, Wisdom 2, Weapon Proficiency 1, Magical Proficiency 1

\textbf{Saving Throws}: +2 Will, +1 Fortitude

After years of training, meditation and internship in Nanda Parbat you are able to harvest your Chi Energy.

Every day after at least 6 hours of rest and 2 hours of meditation/training fill your body with Chi energy equal to Weapon Expertise + Magical Expertise + Wisdom / 2

The \textbf{second time} you take this Feat, requirement Strength 1, Wisdom 2, Weapon Proficiency 4, Magical Proficiency 4

You recover 2 Chi points for every hour you don't do any demanding activities.

\subsection*{Psychic Strike}\index{Psychic Strike}\label{colpopsichico}

\textbf{Requirement}: Psychic Energy, Dexterity 2

\textbf{Saving Throws}: +2 Will, +1 Fortitude

You concentrate your Chi in your hands.
You can concentrate a number of Chi points equal to your Wisdom.
With a successful Touch Attack, you discharge energy that round, causing 1d6 Force damage per point of Chi used.

The \textbf{second time} you take this Feat, requirement Psychic Strike, Wisdom 3, Weapon Proficiency 7

You can use up to double your Wisdom score to enhance Psychic Strike.

\subsection*{Psychic Ray}\index{Psychic Ray}\label{raggiopsichico}

\textbf{Requirement}: Psychic Strike, Wisdom 3, Weapon Proficiency 5

\textbf{Saving Throws}: +2 Reflexes, +1 Will

You can make a ranged attack within 30 feet using psychic energy.
The attack, on Touch, deals 1d6 force damage per point of Psychic spent focused on the damage.

It is possible to focus one or more Psychic points to increase the distance by 9 meters each time.
You cannot use more total Chi points (for distance and damage) than your Wisdom.

The \textbf{second time} you take this Feat requires Psychic Strike, Wisdom 3, Weapon Proficiency 9

You can use up to double your Wisdom score to upgrade your Psychic Beam.

\subsection*{Elementalist}\index{Elementalist}\label{elementalista}

\textbf{Requirement}: At least 2 Elemental Magic Lists

\textbf{Saving Throws}: +1 Will, +1 Fortitude

You are able to swap the elements present in your spells. You can replace one type of elemental energy damage with damage caused by an element from a known Magic List. The casting time of the spell increases by 1 Action, if the total casting time exceeds 3 Actions it is not possible to use this Feat on the spell.

\subsection*{Expert}\index{Expert}\label{esperto}

\textbf{Requirement}: Characteristic linked to at least -1

\textbf{Saving Throws}: +1 to two saving throws of your choice.

You are an expert in a topic. Each time you take this Feat you gain +1 on checks against a Skill of your choice.

The \textbf{second time} you take this Feat add +2 to the check. You can take 10 on the check using 5 rounds instead of 10 (see page \pageref{take10}).

The \textbf{third time} you take this Feat add 1d6 to the check. You can get 14 on the test taking 5 minutes instead of 10.

The \textbf{fourth time} you take this Feat treat the total dice rolled as 10 if he rolled a 4-9.

The bonuses are cumulative but refer to always returning to the same skill. It cannot be used on Awareness (see Perceptual).

\begin{center}
\includegraphics[width=0.75\linewidth]{immagini/distillare.png}

\emph{The Alchemist Discovering Phosphorus. Joseph Wright of Derby (1771-1795)}
\end{center}

\subsection*{Quick Draw}\index{Quick Draw}\label{estrazionerapida}

You are extremely quick to draw your weapon.

\textbf{Requirement}: Weapons Proficiency 1

\textbf{Saving Throws}: +1 Reflexes, +1 Will

You can extract a weapon that is not too big for you with the cost of a Reaction.

The \textbf{second time} you take this Feat you can put away your current weapon and draw another one as a Move Action.

The \textbf{third time} you take this Feat you can put away your current weapon and draw another one without using actions.

\subsection*{Infuriate}\index{Infuriate}\label{fareinfuriare}

Your dialectical Feats are incredible.

\textbf{Requirement}: Weapon Proficiency 2 and Charisma or Strength 2

\textbf{Saving Throws}: +2 Will, +1 Fortitude

You spend 2 Actions defaming and railing against an opponent. The target must make an opposed Will save check against your Perform or Intimidate proficiency check or lose the Dexterity bonus (Saving Throws, Attack Rolls, and Defense) until the end of your next round.

The opponent may not understand your language but must have Intelligence of -3 or more.

\subsection*{Loyal}\index{Loyal}\label{fedele}

\textbf{Requirement}: Magical Expertise 1, Sum Value of Common Traits 2

\textbf{Saving Throws}: +2 Will, +1 Fortitude

Your connection with the Patron is strong and energetic. Each time you take this Feat you can increase your Magic Points by the sum of the Traits in common with the Patron.

Each time you take this Feat, the value of the sum of Traits in common with your Patron must be less than or equal to four times the number of times you have taken this Feat. This Feat does not stack with the Magic Battery Feat.

\subsection*{Ferocy}\index{Ferocy}\label{ferocia}

\textbf{Requirement}: Weapons Proficiency 1

\textbf{Saving Throws}: +2 Fortitude, +1 Will

Your anger is such that it temporarily defeats death.

When you drop below 0 hit points you don't faint and you start losing 1 hit point per round.

A creature with Ferocity faints when it has a negative Hit Point score equal to double its Constitution points and dies when its Hit Points drop to the negative score equal to three times its Constitution score+5 (CON*3+5)

The \textbf{second time} you take this Feat, Weapon Proficiency requirement 4, you can have your Strength increase by 2 in combat and gain 6 temporary Hit Points for 10 minutes. At the end of the fight your fatigue level increases by 1 for 10 minutes.

The \textbf{third time} you take this Feat, Weapon Proficiency requirement 7, you can have your Strength increase by 3 in combat and gain 12 temporary Hit Points for 10 minutes. At the end of the fight your fatigue level increases by 2 for 20 minutes.

The \textbf{fourth time} you take this Feat, Weapon Proficiency requirement 11, you can have your Strength increase by 4 in combat and gain 24 temporary Hit Points for 10 minutes. At the end of the fight your fatigue level increases by 3 for 30 minutes.

The player can choose only one rank of Ferocity to use in the fight (2, 3, 4).

\subsection*{Shayalia's daughter}\index{Shayalia's daughter}\label{figliadishayalia}

Your connection with nature is strong and concrete

\textbf{Requirement}: Devotee or Follower of Shayalia

\textbf{Saving Throws}: +1 Fortitude, +2 Will

The \textbf{first time} you take this Feat you gain a +2 on Nature checks and a +2 on saving throws against natural poisons.

The \textbf {second time} that you take this Feat, requirement sum of Traits in common 6, you get a +4 on Nature checks and a +4 on Saving Throws against effects, even magical, caused by Animals or Plants.

The \textbf{third time} you take this Feat, requirement sum Traits in common 12, you are always under the effect of the Sanctuary spell towards any non-magical animal.

The \textbf{fourth time} you take this Feat, Animalia requirement taken 4 times, you can transform into any creature as long as it is not a fiend or dragon.

\subsection*{Fake Death}\index{Fake Death}\label{fintamorte}

You are able to simulate death by slowing the heart.

\textbf{Requirement}: Constitution 0

\textbf{Saving Throws}: +2 Fortitude, +1 Will

As a Reaction Action you are able to fall to the ground (drop!) dead. Only a DC 20 First Aid check can reveal that you are alive.

The effect lasts a maximum of 2 minutes. The fake death cannot be repeated within 10 minutes of each other.

\subsection*{Dancing Scourge}\index{Dancing Scourge}\label{flagellodanzante}

\textbf{Requirement}: Weapon Proficiency 1, use a weapon from the Rolling Balls List

\textbf{Saving Throws}: +1 Fortitude, +1 Will

When you use your Spinning Ball weapon you have a +1 bonus on attack rolls and +1 on defense.

\subsection*{Forged in Fury}\index{Forged in Fury}\label{forgiatonellafuria}

\textbf{Requirement}: Weapons Proficiency 5

\textbf{Saving Throws}: +1 Fortitude, +1 Reflex

When you make a critical roll with a melee attack, i.e. you have rolled at least 2 6s, you are considered to have rolled an extra 6 towards the total critical roll count

\subsection*{Lucky}\index{Lucky}\label{fortunato}

\textbf{Saving Throws}: +1 Fortitude, +1 Reflex

Once per day you can have the Storyteller reroll 1d6 of a check (Attack Rolls, Feat Checks, Saving Throws) and take the lower value of the two rolls.

\subsection*{Elemental Form}\index{Elemental Form}\label{formaelementale}

\textbf{Requirement}: Follower or Devotee of Erondil, Gaya, Ephrem or Shayalia. Must have Magic Lists on an Element. Magical Expertise 6

\textbf{Saving Throws}: +1 Fortitude, +1 Will

The \textbf{first time} you take this Feat when you transform into an animal your animal attack damage can do elemental type damage.

The \textbf{second time} you take this Feat, Magical Proficiency 11, as long as you are in animal form you are resistant to the same type of elemental damage you cause. 

The \textbf{third time} you take this Feat, Magical Proficiency requirement 14, your animal attacks deal 2d6 more damage than the chosen elemental type.

The elemental damage must be from a known Elemental Magic List.

If you are a Devotee or Follower of Gaya or Erondil it is not necessary to transform into an animal, the damage type applies to your melee attacks.

\subsection*{Arrow called, arrow delivered}\index{Arrow called, arrow delivered}\label{frecciachiamata}

\textbf{Requirement}: Weapons Proficiency 2

\textbf{Saving Throws}: +2 Reflexes

You can shoot 1 arrow, once per day, as a reaction, with no hit penalties given by multiattack.

\subsection*{Fury}\index{Fury}\label{furia}

\textbf{Requirement}: Weapons Proficiency 1

\textbf{Saving Throws}: +2 Fortitude, +1 Will

Your fighting style is one of blind, murderous fury. Add +1d6 to damage to each successful melee attack and your opponents gain +1d6 on hits towards you. You can decide to activate this Feat round by round. Costs 1 Immediate Action.

\subsection*{Juggler}\index{Juggler}\label{giocoliere}

\textbf{Requirement}: Dexterity 2

\textbf{Saving Throws}: +2 Reflexes

You have a natural talent for manipulating objects.

Any Acrobatics check that involves handling objects or balance has a +2 bonus.

You can throw a second dagger as an immediate action following the throw attack action of a dagger, this dagger has a -3 to attack roll. Any third dagger thrown has the normal penalty of -5 (and -10.. and so on).

%\begin{center}
%\includegraphics[width=0.9\linewidth]{immagini/Early_Egyptian_juggling_art.png}
%
%\emph{This ancient wall painting appears to depict jugglers.}
%\end{center}

\subsection*{Warrior of Magic}\index{Warrior of Magic}\label{guerrierodellamagia}

\textbf{Requirement}: Weapon Proficiency 2, Magical Proficiency 2

\textbf{Saving Throws}: +1 Will, +1 Reflexes

You don't just follow the path of magic or even that of the sword, your style embraces both in a slash of pure magic.

The \textbf{first time} you take this Feat you are able to discharge a spell at melee range with your weapon. You make the attack roll and if you hit, in addition to the damage from the attack you also discharge the spell. You must succeed in a Magic Test. In this way you can only make one attack with the weapon. It costs 3 Actions.

The \textbf{second time} you take this Feat, requirement Weapon Proficiency 3, Magical Proficiency 3, by consuming 3 Actions you are able to discharge a spell that is not personal or touch with a ranged weapon. You must succeed in a Magic Test.

The \textbf{third time} you take this Feat, requirement Weapon Proficiency 6, Magical Proficiency 3, consuming 3 Actions you are able to combine up to two melee attacks with the discharge of a spell. The rules for the first time apply but you don't have to take the Magic Test.

You cannot discharge spells higher than level 3 with this Feat and the spell casting time cannot exceed 2 Actions.

\subsection*{Silver Crane}\index{Silver Crane}\label{grudargento}

\textbf{Requirement}: Empty Fist List 3

\textbf{Saving Throws}: +2 Reflexes, +1 Will

Your unarmed fighting style is based on agility and counterattacking.

The \textbf{first time} you take this Feat your Defense increases by 1.

The \textbf{second time} you take this Feat, Empty Fist List requirement 4, your Initiative increases by 2 (only with unarmed attacks).

The \textbf{third time} you take this Feat, Empty Fist List requirement 9 and Dexterity 2, you have a bonus on Reflex and Fortitude saving throws of 2 (cumulative).

The \textbf{fourth time} you take this Feat, Empty Fist List requirement 11, your Defense and Initiative increase by 2 (cumulative).

The \textbf{fifth time} you take this Feat, Empty Fist List requirement 13 and Dexterity 3, you have a bonus on Reflex and Will saving throws of 2 (cumulative).

The bonuses are active even if you are not fighting.

\subsection*{I said FALL!}\index{I said FALL!}\label{hodettocadi}

\textbf{Requirement}: Weapons Proficiency 4

\textbf{Saving Throws}: +2 Fortitude, +1 Will

If you hit an opponent 3 times within 2 rounds, they must make a Fortitude save with a DC equal to the attack roll for the last attack or fall prone. The saving throw has a 1d6 modifier per size difference.

\subsection*{The Patron is with me}\index{The Patron is with me}\label{ilpatronoeconme}

\textbf{Requirement}: Devoted, Sum Common Traits with Patron 2

\textbf{Saving Throws}: +1 Will, +1 Reflexes

The \textbf{first time} you take this Feat 1 time per day you can reroll a die rolled in the Magic Test for spell casting. 

The \textbf{second time} you take this Feat, requirement sums Common Traits with the Patron 6, 2 times per day you can reroll up to 2 dice rolled in the Magic Test for spell casting.

The \textbf{third time} you take this Feat, requirement sums Common Traits with the Patron 12, 3 times per day you can reroll up to 3 dice rolled in the Magic Test for spell casting. 

The Feat can also be declared after the dice have been rolled. Any new value obtained with the new roll must be kept or this Feat is used again. 

\subsection*{The Patron is my Weapon}\index{The Patron is my Weapon}\label{ilpatronoelamiaarma}\hypertarget{the patron and my weapon}{}\ref{ilpatronoelamiaarma}

\textbf{Requirement}: Sum of common traits with Patron 1

\textbf{Saving Throws}: +1 Will, +1 Reflexes

The \textbf{first time} you take this Feat you have a +1 to Attack Roll and Damage when using your Patron's favored Weapon.

The \textbf{second time} you take this Feat, common Trait sum requirement 5, Weapon Proficiency 1, the penalty for multiple attacks with the Patron's favored weapon becomes -4.

The \textbf{third time} you take this Feat, requirement sums Traits common with the Patron 10, Weapon Proficiency 2, add +1d6 to the attack roll when you make the third attack with the Patron's weapon.

The \textbf{fourth time} you take this Feat, requirement sums Common Traits with the Patron 15, Weapon Proficiency 3, you increase the damage of your Patron's weapon by one rank.

The \textbf{fifth time} you take this Feat, requirement sums Common Traits with the Patron 19, Weapon Proficiency 4, you gain an additional +1 to Attack Roll and +1 to Damage. The first successful attack in the round with the Patron's weapon always causes critical damage.

\subsection*{Iaijutsu}\index{Iaijutsu}\label{iaijutsu}

\textbf{Requirement}: Weapons Proficiency 2

\textbf{Saving Throws}: +2 Reflexes, +1 Will

For every -1d6 to your attack roll you gain a +4 to your Initiative and vice versa.
The bonus must be used by the end of your next round. The declaration must be made every round that you intend to use when checking initiatives.

\subsection*{Improvise}\index{Improvise}\label{improvvisare}

\textbf{Requirement}: Weapons Proficiency 1

\textbf{Saving Throws}: +1 Fortitude, +1 Reflex

Any object that can be defined as an improvised weapon is not improvised for you.
You do not suffer a hit penalty when using an improvised weapon. You gain +1 to damage when using an improvised weapon.

\subsection*{Combat Spellcaster}\index{Combat Spellcaster}\label{incantatoredacombattimento}

\textbf{Requirement}: Magical Proficiency 1

\textbf{Saving Throws}: +1 Fortitude, +1 Will

When you are Distracted you can roll one fewer die on the Magic Test.

The \textbf{second time}, Magical Proficiency requirement 6, that you take this Feat when you are Distracted you can roll one fewer die on the Magic Test.

\subsection*{Prudent Enchanter}\index{Prudent Enchanter}\label{incantatoreprudente}

\textbf{Requirement}: Magical Proficiency 8

\textbf{Saving Throws}: +2 Reflexes, +1 Fortitude

When a hostile creature first enters a space within 3 feet of you, you can use a Reaction to cast a cantrip without any enhancements or Magic Check.

This Feat does not affect the fact that you are distracted when casting a subsequent spell.

\subsection*{Immunity to poisons}\index{Immunity to poisons}\label{immunitaaiveleni}\index{Mithridatism}

\textbf{Requirement}: Constitution 1

\textbf{Saving Throws}: +2 Fortitude, +1 Will

The body becomes accustomed to poisons, the character gains a +2 saving throw on poisons.

The \textbf{second time} you take the Feat you become immune to natural poisons. You can't get drunk normally anymore.

The \textbf{third time} you have a +1d6 on saving throws to magical poisons and suffer the effects of toxic fumes (but you can still suffocate).

\subsection*{Laying on of hands} \index{Laying on of hands}\label{imposizionedellemani}

\textbf{Requirement}: Magical Proficiency 3, Common Traits 3

\textbf{Saving Throws}: +2 Will, +1 Fortitude

If your Traits are in common with a positive Patron you can channel positive energy (healing/harmful effect on undead), if they are in common with a neutral or evil Patron you can channel negative energy (harmful/healing effect on undead). Usable a number of times per day equal to (sum of Traits in common with the Patron)/2. Healing/damaging effect equal to 1d6+Wisdom

The \textbf{second time}, score requirement sum Traits in common 6, that you take this Feat the effect increases by 2d6.

The \textbf{third time}, score requirement sum Traits in common 12, that you take this Feat the effect increases by 3d6.

The \textbf{fourth time}, sum score requirement Common Traits 18: that you take this Feat the effect increases by 4d6.

The energy comes from the hands (it doesn't matter if there are gloves) and is applied with a Touch Attack. Use 2 Actions. Fortitude save DC 10 + add Traits in common with the Patron + Wisdom to avoid the effect.

\subsection*{Channel energy} \index{Channel energy}\label{incanalareenergia}

\textbf{Requirement}: Magical Proficiency 1, Common Traits 3

\textbf{Saving Throws}: +2 Will, +1 Fortitude

You are able to channel the positive or negative energy of your Patron.

If your Traits are in common with a positive Patron you can channel positive energy (healing/harmful on undead), if they are in common with a neutral or evil Patron you can channel negative energy (harmful/healing on undead). Usable an even number of times (sum of Traits in common with the Patron)/2. Healing/damaging effect equal to 1d6+Wisdom. You affect 1 creature.

The \textbf{second time}, requirement sum of Traits in Common 6, that you take this Feat increases the effect by 1d6. You influence up to 2 creatures.

The \textbf{third time}, requirement sum of Traits in Common 12, that you take this Feat increases the effect by 2d6. You influence up to 4 creatures.

The \textbf{fourth time}, Common Traits sum requirement 18, that you take this Feat increases the effect by 3d6. You influence up to 6 creatures. 

The energy comes from your body and affects yourself and two or more creatures within 20 feet of you, closest first. Reflex save DC 10 + add Traits in common with the Patron + Wisdom to avoid the effect. Use 2 Actions.\index{Channel energy into undead}


\begin{center}
\includegraphics[width=0.65\linewidth]{immagini/Portrait_of_V_Greatrakesv2.png}

\emph{Portrait of V. Greatrakes laying on his hands, window, in right-hand corner showing several successful cures, possibly. By W. Faithorne }
\end{center}


\subsection*{Channel selective energy}\index{Channel selective energy}\label{incanalareenergiaselettiva}

\textbf{Requirement}: Channel energy, Channel energy ranged

\textbf{Saving Throws}: +2 Will, +1 Fortitude

You can exclude a creature from Channel Energy or Channel Energy for each time it has taken this Ability.

%\begin{center}
%\includegraphics[width=0.8\linewidth]{immagini/kameame.png}
%\emph{Kamehameha!}
%\end{center}

\subsection*{Instill Courage}\index{Instill Courage}\label{infonderecoraggio}

\textbf{Requirement}: Charisma 2, Perform 1

\textbf{Saving Throws}: +2 Will, +1 Fortitude

Through your performance, singing, ballet, oratory... you are able to instill courage in companions who can hear or see you, within a radius of 6 meters.

The first time you take this Feat your companions have a +1 bonus to attack rolls and damage rolls in combat.

The \textbf{second time} you take this Feat, Perform requirement 4, you can decide to infuse up to 2 of these bonuses. +2 TC, +2 Defense, +2 Damage, +2 Will Save. Your companions must be within 12 meters radius.

The \textbf{third time} you take this Feat, Perform requirement 12, you can choose to infuse up to 2 of these bonuses. +1d6 Attack Roll, +4 Defense, +4 Damage, +1d6 Save. Your companions must be within 24 meters radius.

Activating, maintaining or changing the Feat's effect requires 2 Actions. You can maintain the Feat for a number of rounds, even non-consecutive, equal to the Entertain score x 3 per day. Creatures must continue to see/hear the performance to remain affected.

\subsection*{Imbue Magical Energy}\index{Imbue Magical Energy}\label{infondereenergiamagica}

\textbf{Requirement}: Weapons Proficiency 1, Magical Proficiency 2

\textbf{Saving Throws}: +1 Reflexes, +1 Fortitude

You know how to manipulate magical energies instinctively and infuse them into weapons. It costs 1 Action to imbue the weapon with magic.

The \textbf{first time} you take this Feat you can use two Magic Points and channel them into your weapon. For the duration of 6 rounds your weapon becomes a +1 magical weapon, if it already has magical abilities the effect does not work.

The \textbf{second time} you take this Feat, Magical Expertise requirement 4, you can use four Magic Points and a weapon you come into contact with becomes a +2 weapon for 6 rounds, if it is already enchanted with an additional bonus of + 1 up to a maximum of +3.

The \textbf{third time} you take this Feat, Magical Expertise requirement 8, you can use six Magic Points and a weapon you come into contact with becomes a +3 weapon for 6 rounds, if it is already enchanted with a +2 bonus or lower makes it +4.

\subsection*{Imbue Greater Magical Energy}\index{Imbue Greater Magical Energy}\label{infondereenergiamagicasuperiore}

\textbf{Requirement}: Weapons Proficiency 4, Magical Proficiency 6

\textbf{Saving Throws}: +1 Reflexes, +1 Fortitude

You know how to infuse the weapon with magical energy to give it fantastic abilities. It costs 1 Action to activate the infusion of magic into the weapon. The weapon must be magical.

The \textbf{first time} you take this Feat using one Magic Point per round you can make your weapon flaming or electrified or change the shape. Each successful hit deals an additional 1d6 points of fire or electricity damage, or you can change the weapon's form. If you stop paying the Magic Point it returns to its previous form and stops causing additional damage.

The \textbf{second time} you take this Feat using two Magic Points per round can make a weapon you come into contact with extremely dangerous. Each successful hit causes 1 additional critical damage. Magical Proficiency Requirement 7.

The \textbf{third time} you take this Feat using three Magic Points per round can grant a weapon you come into contact with both of the previous Feats.

Feats are not cumulative, you must choose which one to apply round by round.

\subsection*{Instill Fear}\index{Instill Fear}\label{infonderepaura}

\textbf{Requirement}: Charisma 2

\textbf{Saving Throws}: +2 Will, +1 Fortitude

Through your performance, singing, ballet, oratory... you are able to instill fear in opponents who can hear you, within a radius of 6 meters.

The first time you take this Feat your enemies have a -1 penalty to attack rolls and damage in combat.

The \textbf{second time} you take this Feat, Perform requirement 4, the strength of your art attacks enemies and you can select two effects from: -2 Attack Roll, -2 Combat Damage, -2 Defense, -2 on the Will saving throw. Your enemies must be within 12 meters radius.

The \textbf{third time} you take this Feat, requirement Perform 12, the strength of your art attacks enemies and you can select two effects from: -1d6 Attack Roll, -4 Defense, -4 Damage, -1d6 Save. Your enemies must be within 24 meters radius.

The opponent is allowed a DC Will save of 10+CHA+Perform score. A creature that succeeds on the saving throw is immune to new manifestations of this power that day.

Activating, maintaining or changing the Feat's effect requires 2 Actions. You can maintain the Feat for a number of rounds, even non-consecutive, equal to the Entertain score x 3 per day. Creatures must continue to see/hear the performance to remain affected.

\subsection*{Improved initiative}\index{Improved initiative}\label{iniziativamigliorata}

\textbf{Requirement}: Intelligence or Dexterity 1

\textbf{Saving Throws}: +2 Reflexes

You increase initiative by +1. The Feat can be taken up to 2 times and the bonus stacks.

\subsection*{My skin}\index{My skin}\label{La mia pelle}

\textbf{Requirement}: Weapons Proficiency 1

\textbf{Saving Throws}: +3 Fortitude

You have an almost symbiotic relationship with your armor.

The \textbf{first time} you take this Feat increases the Defense granted by the armor you wear by 1.

The \textbf{second time} you take this Feat, Weapon Proficiency requirement 6, the Defense granted by the armor you wear increases by 2.

\subsection*{My death your death}\index{My death your death}\label{lamiamortelatuamorte}

\textbf{Requirement}: Weapon Proficiency 1, Strength 1

\textbf{Saving Throws}: +2 Fortitude, +1 Will

For each individual combat opponent you can make the first hit of the fight cause additional damage equal to double the Weapon Proficiency. The opponent gains a bonus to attack rolls and damage equal to your Weapon Proficiency. It must be declared before the attack roll.

\subsection*{My Head is Harder}\index{My Head is Harder}\label{lamiatestaepiudura}

\textbf{Requirement}: Weapons Proficiency 1

\textbf{Saving Throws}: +1 Fortitude, +1 Will

Your Skull Breaking Weapon does +2 damage

\subsection*{Versatile Litany}\index{Versatile Litany}\label{itaniaversatile}

\textbf{Requirement}: Entertaining Feat 6

\textbf{Saving Throws}: +1 Will, +1 Reflexes

Through your performance you can choose to instill courage or fear in creatures within 30 feet of you. Each round you can decide to apply up to 2 modifiers between: bonus of +1d6 on the attack roll or +4 on the defense or -1d6 on the attack roll or -4 on the defense. 

The opponent is allowed a DC Will save of 10+CHA+Perform score. A creature that succeeds on the saving throw is immune to new manifestations of this power that day.

Activating and maintaining the Feat requires 2 Actions. You can maintain the Feat for a number of rounds, even non-consecutive, equal to your Perform score per day. Creatures must continue to see/hear the performance to remain affected.

\subsection*{The shield is my friend}\index{The shield is my friend}\label{loscudoemioamico}

\textbf{Requirement}: Weapons Proficiency 3

\textbf{Saving Throws}: +1 Fortitude, +1 Reflex

The \textbf{first time} taking this Feat, Magical Proficiency requirement 3, you can use light without having to make a Magic Test.

The \textbf{second time} you take this Feat, Weapon Proficiency requirement 5, the penalty to the attack roll given by the shield decreases by 1.

\subsection*{Powerful Spells}\index{Powerful Spells}\label{magiepotenti}

\textbf{Requirement}: Magical Proficiency 5

\textbf{Saving Throws}: +2 Will

Your spells are extraordinarily effective.

Choose a Magic List, gain a +1d6 Magic Check when casting spells from this school. The Feat can be taken multiple times but the total must be less than or equal to CM/4.

\subsection*{Spring}\index{Spring}\label{Molla}

\textbf{Requirement}: Strength 0

\textbf{Saving Throws}: +1 Reflexes, +1 Fortitude

You can ignore the 3 meter run-up requirement before a jump.

The \textbf{second time} you take this Feat when you make a long or high jump check you roll 1d6 more.

\subsection*{Human mountain}\index{Human mountain}\label{montagnaumana}

\textbf{Requirement}: Constitution 1

\textbf{Saving Throws}: +3 Fortitude

Maybe you were once frail and weak, now you are a mountain of muscles.

When you take this Feat you increase the hit points taken per level by 1.

The \textbf{second time} you take this Feat increases the Hit Points taken per level by 1.

The \textbf{third time} you take this Feat increases the die to roll Hit Points (from d4 to d6).

Bonuses are cumulative and retroactive to previous levels, except for the hit die increase.

The \textbf{fourth time} you take this Feat increases by one size (P > M > G).


\begin{center}
\includegraphics[width=0.65\linewidth]{immagini/elcolosso.png}

\emph{The Colossus (also known as The Giant), is known in Spanish as El Coloso.}
\end{center}


\subsection*{Mental Wall}\index{Mental Wall}\label{muromentale}

\textbf{Requirement}: Wisdom +1

\textbf{Saving Throws}: +2 Will, +1 Fortitude

Your mind is trained against those who want to influence it. Each time you take this Feat you gain +1 on saving throws against spells from the Enchantment Magic List.

\subsection*{Clinical Eye}\index{Clinical Eye}\label{occhioclinico}

\textbf{Requirement}: Weapons Proficiency 3

\textbf{Saving Throws}: +2 Reflexes

You are able to deal critical damage to creatures normally immune to critical damage (roll 6s multiple times and damage explodes).

\subsection*{Hawkeye}\index{Hawkeye}\label{occhiodifalco}

\textbf{Requirement}: Weapons Proficiency 3

\textbf{Saving Throws}: +2 Reflexes, +1 Will

The penalty for rolls between the first and second increments has no penalty

The \textbf{second time} you take this Feat, the penalty for rolls up to the third range increment is 1d6.

The \textbf{third time} you take this Feat you are able to extend your roll even further and bring it to a fifth increment with a -2d6 penalty to hit. You have no penalties within the first 3 increments while you have -1d6 to hit between the third and fourth increments.\\

\subsection*{Opportunist}\index{Opportunist}\label{opportunista}\hypertarget{opportunist}{}

\textbf{Requirement}: Weapons Proficiency 2

\textbf{Saving Throws}: +2 Reflexes, +1 Will

You can attempt to melee an opponent who \textbf{exits} or \textbf{through} a melee area that you threaten or uses a thrown weapon against you in your melee area. The Feat can be used once per round as a Reaction. This attack is also called an attack of opportunity in the manual, and there are several creatures that do not react to it.

\subsection*{Save}\index{Save}\label{parata}

\textbf{Requirement}: Weapon Proficiency 3 or Empty Fist 2

\textbf{Saving Throws}: +1 Reflexes, +1 Will

The \textbf{first time} you take this Feat You use a Reaction to increase your Defense by 1. 

The \textbf{second time} you take this Feat, Weapon Proficiency requirement 6 or Empty Fist 4, using a Reaction increases your Defense by 2.

The \textbf{third time} you take this Feat, Weapon Proficiency requirement 9 or Empty Fist 6, you gain a Reaction that you can only use to use the Parry Feat.

Using the Parry Feat can be declared even after you know how much you have been hit. 

\subsection*{Quick step}\index{Quick step}\label{passofelpato}

\textbf{Requirement}: Stealth 1

\textbf{Saving Throws}: +1 Reflexes, +1 Fortitude

Your step is naturally silent.

The \textbf{first} time you take this Feat the penalty to move at full speed using Stealth becomes -1d6.

The \textbf{second time} you take this Feat, requirement Dexterity 3, Stealth 8, you have no penalty to move at full speed.

\subsection*{Fast Pass}\index{Fast Pass}\label{passoveloce}

\textbf{Requirement}: Dexterity 1

\textbf{Saving Throws}: +2 Reflexes, +1 Fortitude

Your pace is naturally quick.
If you have 6m movement you switch to 7m movement, if you have 9m movement you switch to 10m movement.

Every additional \textbf{twice} you take the Feat your movement increases by 1 meter per Move Action, up to a maximum of +3 meters per round.

\subsection*{Safe Pass}\index{Safe Pass}\label{passosicuro}

\textbf{Requirement}: Wisdom 1

\textbf{Saving Throws}: +2 Fortitude, +1 Reflex\\

It is the ability not to be slowed down in a hostile environment. It is necessary to declare on which environment the Feat is taken. In these environments the natural terrain is not difficult. As long as you move in the chosen environment you have a +1 on Initiative checks.\\

\begin{tabular}{l|l}
\textbf{Environment} & \textbf{Environment}\\
\toprule
Jungle & Aquatic \& Coastal\\
Swamp & Hill \& Forest \\
Plain & Desert \\
Mountains & Glaciers \& Tundra \\
Urban & Underground \\
\end{tabular}\\

Every time you take this Feat again you choose a different environment and add to the previous one or specialize on the same one.\\

The \textbf{second time} you take this Feat on the same terrain you gain a specific ability depending on the terrain.\\

\emph{Jungle / Forest / Hill / Plains}: Your movement increases by 1 meter on this terrain\\
\emph{Costal / Aquatic}: swim speed equal to your movement\\
\emph{Swamp}: +2 on saving throws vs. Poison\\
\emph{Desert}: Fire damage reduction equal to level\\
\emph{Mountain / Glaciers / Tundra}: Cold damage reduction equal to level\\
\emph{Underground}: Low-light vision 9 meters\\
\emph{Urban}: +1 Language, +1 choice in two Knowledge

\subsection*{Leathery leather}\index{Leathery leather}\label{pellecoriacea}

\textbf{Requirement}: Constitution 2

\textbf{Saving Throws}: +2 Fortitude

Your skin is extremely durable. You take 1 less damage when hit by slashing weapons.

The \textbf{second time} you take this Feat, Weapon Proficiency requirement 6, you take 1 less damage when hit by slashing, piercing, bludgeoning weapons. Reduce the Bleeding condition by 1 when acquired.

The \textbf{third time} you take this Feat, Weapon Proficiency requirement 12 and Constitution 3, you take 1 less damage when hit by slashing, piercing, or bludgeoning weapons. You take 1 less damage when hit by magic. Reduce the Bleeding condition by 1 when acquired.

The \textbf{fourth time} you take this Feat, Weapon Proficiency requirement 16, you ignore 1 critical roll when hit by slashing, piercing, or bludgeoning weapons and take 1 less damage when hit by magic. Reduce the Bleeding condition by 1 when acquired.

Bonuses are cumulative.

\subsection*{Perceptive}\index{Perceptive}\label{percettivo}

\textbf{Requirement}: Wisdom 0

\textbf{Saving Throws}: +1 Reflexes, +1 Will

Your Awareness and attention to detail is above average.
You gain a +1 bonus on Awareness checks. The Feat can be taken a maximum of 3 times.

\subsection*{Truly evil person}\index{Truly evil person}\label{personaveramentemalvagia}

\textbf{Requirement}: Weapons Proficiency 1

\textbf{Saving Throws}: +1 Reflexes, +1 Will

Twice a day you add your Weapon Proficiency value to the damage of an opponent you want to hit in melee. The Feat must be declared before the attack roll. It costs one Action.

\subsection*{The bigger they are, the more noise they make when they fall}\index{The bigger they are, the more noise they make when they fall}\index{Giant Killers}\label{piusonogrossipiufannorumore}

\textbf{Requirement}: Weapons Proficiency 1

\textbf{Saving Throws}: +2 Fortitude, +1 Will

When you attack a creature at least 2 sizes larger than you, you deal +1 additional damage for every 2 points of Weapon Proficiency. If it is only one size larger, add 1 additional damage for every 3 Weapon Proficiency points.

\subsection*{Polyglot}\index{Polyglot}\label{poliglotta}

\textbf{Requirement}: at least Intelligence -1, at character creation

\textbf{Saving Throws}: +2 Will

You have an extraordinary ability to learn languages. Give 2 points to Language Knowledge and know two more languages.

\subsection*{Patron Power}\index{Patron Power}\label{poteredelpatrono}

\textbf{Requirement}: Sum of Traits in common with Patron 1, being Devoted

\textbf{Saving Throws}: +1 Fortitude, +2 Will

Your faith in the Patron knows no limits or collapses of trust.

Once a day for a single test, as a reaction before carrying out the test, you use the sum of the traits common to the Patron as a unique positive modifier. You can use this Feat on saving throws, attack rolls, and Feat checks.

If all three checks succeed it is likely that it is a Patron Manifestation.

\subsection*{First Blood}\index{First Blood}\label{primosangue}

\textbf{Requirement}: Weapons Proficiency 1

\textbf{Saving Throws}: +1 Fortitude, +1 Will

The first attack roll against a new opponent has a bonus of +1d6.

\subsection*{Continue}\index{Continue}\label{proseguire}

\textbf{Requirement}: Weapons Proficiency 1

\textbf{Saving Throws}: +1 Fortitude, +1 Will

If you kill the opponent with your last Attack Action, in melee, you can perform a bonus attack action with the same modifiers as the last Attack Action performed and attack another enemy as long as it is within melee range, if you kill this creature with one hit, you can't make other attacks on other creatures.

The \textbf{second time} you take this Feat requirement Continue, Weapon Proficiency 6

If you kill your opponent with your last Attack Action, you can take a bonus attack action with the same modifiers as your last Attack Action with the weapon and attack another enemy within 1 meter. If you kill it you can continue with a further bonus attack (and you move within 1 meter) with the next creature and so on.

Each bonus attack beyond the first has a -2 to hit and a -1 to cumulative damage.

\subsection*{Iron Fist}\index{Iron Fist}\label{pugnodiferro}

\textbf{Requirement}: Empty Fist List 3

\textbf{Saving Throws}: +2 Fortitude, +1 Will

Your unarmed combat technique is extremely precise and powerful.

The \textbf{first time} you take this Feat the damage caused by your punches (and kicks) and the attack roll increases by 1. Your blows are treated as silver weapons.

The \textbf{second time} you take this Feat, Empty Fist requirement 6. Damage +2, Attack Roll +1. Your hits are treated as a +1 weapon.

The \textbf{third time} you take this Feat, Empty Fist requirement 9. Damage +1, Attack Roll +2. Your shots are considered an adamantium weapon.

The \textbf{fourth time} you take this Feat, Empty Fist requirement 12. Damage +2, Attack Roll +1. Your hits are treated as a +2 weapon.

The \textbf{fifth time} you take this Feat, Empty Fist requirement 15. Damage +1, Attack Roll +2. Your hits are treated as a +3 weapon.

The \textbf{sixth time} you take this Feat, Empty Fist Requirement 18. Damage +2, Attack Roll +1. Your hits are treated as a +4 weapon.

Bonuses acquired are cumulative except for the magic level of the shot.

\subsection*{Power Punch}\index{Power Punch}\label{pugnopotente}

\textbf{Requirement}: Empty Fist List 3

\textbf{Saving Throws}: +1 Fortitude, +2 Will

Consume 2 Actions. You make a single attack roll with a -5 penalty. If you hit, in addition to the damage and a critical damage, the opponent who must be a maximum of two sizes larger than you must make a Fortitude saving throw with a DC equal to Weapon Proficiency + Empty Fist List x2 or be pushed 3 meters into a direction of your choice. If he fails the saving throw he takes an additional 2 critical damage.

\subsection*{This is my dagger}\index{This is my dagger}\label{questoeilmiopugnale}

\textbf{Requirement}: Weapons Proficiency 1

\textbf{Saving Throws}: +2 Fortitude, +1 Reflex

When you deal critical damage with your dagger, you add your Weapon Proficiency to the damage. The Feat can be used 1 time per opponent and is automatically applied to the first critical damage done.

\subsection*{This is my weapon!}\index{This is my weapon!}\label{questaelamiaarma}

\textbf{Requirement}: Weapons Proficiency 1

\textbf{Saving Throws}: +2 Fortitude, +1 Will

Every time you hit the same opponent, starting from the second round, you deal additional damage (Max +1 per combat round, even if you hit him multiple times in the round) up to a maximum of +5. The first time you don't hit your opponent in the round the bonus goes back to +0. The bonus can only be held on one opponent at a time.

The \textbf{second time} you take this Feat you can miss the opponent with one hit and not lose the benefits.

\subsection*{Magic Roots}\index{Magic Roots}\label{radicimagiche}

\textbf{Requirement}: Magical Proficiency 1

\textbf{Saving Throws}: +2 Will, +1 Fortitude

As long as you are affected by your spell, using an Action your weapon gains +1 to hit and damage and is considered a magical weapon until the end of the round.

\subsection*{Retaliation}\index{Retaliation}\label{rappresaglia}

\textbf{Requirement}: Weapons Proficiency 1

\textbf{Saving Throws}: +2 Will, +1 Fortitude

Seeing your friends hurt fills you with anger.
When a companion (or yourself) drops below half hit points you gain a +1 to attack rolls and saving throws. The maximum duration of the effect is 1 minute (6 rounds) per day and must be consecutive. The player chooses whether or not to activate the Feat and the injured teammate must be within 9 meters.
You can take this Feat up to 3 times, each time the bonus to the attack roll and saving throw increases by 1.

\subsection*{Stone resistance}\index{Stone resistance}\label{resistenzadellapietra}

\textbf{Requirement}: Constitution 0

Over time you have trained your Constitution to withstand shocks, transformations, poisons and anything else that wanted to modify your body. The first time you take this Feat you gain a +2 bonus on your Fortitude save. The bonus is cumulative, +2 the first time, +1 the \textbf{second}, +1 the \textbf{third}.

The fourth time you take this Feat you can choose to automatically succeed on a Fortitude save once per day as a reaction. It must be declared and does not cause the saving throw to be rolled.

\subsection*{Detect Magic}\index{Detect Magic}\label{rilevareilmagico}\index{Eyes of Magic}

\textbf{Requirement}: Magical Proficiency 1

\textbf{Saving Throws}: +1 Will, +1 Fortitude

If you can see it you also know if it is magical. It costs one Action to activate the magical sight and lasts one round.

The \textbf{second time}, Magical Expertise requirement 1, that you take the Feat to activate magical sight costs the Reaction.


%\begin{center}
%\includegraphics[width=0.55\linewidth]{immagini/streghegoya.png}
%
%\emph{The Witches' Sabbath (Goya, 1798)}
%\end{center}

\subsection*{Lightning reflexes}\index{Lightning reflexes}\label{riflessifulminei}

\textbf{Requirement}: Dexterity 0

Over time you have trained your reflexes to dodge and anticipate any obstacle. The first time you take this Feat you gain a +2 bonus on Reflex saving throws. The bonus is cumulative, +2 the first time, +1 the \textbf{second}, +1 the \textbf{third}.

The \textbf{fourth time} you take this Feat you can decide to automatically succeed on a Reflex saving throw once per day as a reaction. It must be declared and does not cause the saving throw to be rolled.

\subsection*{Pure Blood}\index{Pure Blood}\label{sanguepuro}

\textbf{Requirement}: Animalia, Devotee of Ephrem or Shayalia

\textbf{Saving Throws}: +1 Will, +2 Fortitude

With this Feat, each of your natural attacks while in animal form causes 1 additional damage and is considered a +1 magical attack. By focusing on your step you can leave the footprints of an animal that you can transform into and the terrain is considered difficult.

The \textbf{second time} you take this Feat, Magical Expertise 8, when you use Animalia's Feat you can perform a partial transformation or take the type of Movement or Senses of the creature you transform into. When you use the Animalia Ability you can select a creature with a Challenge Rating increased by 1.

The \textbf{third time} you take this Feat, Magical Proficiency 12, when you transform into an animal, you can use your Magical Proficiency in place of Weapon Proficiency on natural attacks. When you use the Animalia Ability you can select a creature with a Challenge Rating increased by 1. When you use the Animalia Ability you can perform a partial transformation or take the Movement or Senses type of the creature you transform into.


Feats two and three are cumulative.


\subsection*{Knowledgeable}\index{Knowledgeable}\label{sapiente}

\textbf{Requirement}: Magical Proficiency 4

\textbf{Saving Throws}: +2 Will

Your interest and connection with magic is unparalleled. You can know one more spell (while respecting the maximum level constraints that can be chosen).

The Feat can be taken again as long as the Magical Expertise value is at least 4 times the times this Feat was taken. Therefore with a minimum Magical Competence value of 4, 8, 12..

\begin{center}
\includegraphics[width=0.6\linewidth]{immagini/turning-undead-six.png}
\end{center}

\subsection*{Turn the undead}\index{Turn the undead}\label{scacciarenonmorti}

\textbf{Requirement}: Sum of common traits 2

\textbf{Saving Throws}: +2 Will, +1 Fortitude

By focusing on the power of your Patron, you channel positive energy and ward off or destroy the undead.

Roll 1d6 + sum of Traits in common with the Patron, this total is your Divine Power.

Starting from the weakest undead around you, within 30 feet, check the undead's Divine Power score and Challenge Rating.

If the Divine Power is at least double the Challenge Rating, the undead is destroyed and double the Challenge Rating is subtracted from the Divine Power value.

If the undead is not destroyed then it makes a DC Will save equal to 10 + Divine Power to resist the turning. If the saving throw fails the undead is turned away, if it succeeds it is unaffected. Whether you succeed or fail, subtract the Challenge Rating from your Divine Power score before checking out a new undead.

The Feat can be used a number of times per day equal to Wisdom but an undead can only be affected once per day by your effect.

An undead that is turned is under \hyperlink{fear condition}{Fear} for 1d4 rounds, a destroyed undead is reduced to dust and divine energy.

A Devotee of Sixiser, instead of turning and destroying, can dominate the undead for 2d4 rounds or 1 real hour respectively.

A Thaft Devotee gains +1d6 to Divine Power.


\subsection*{Dodging traps}\index{Dodging traps}\label{schivaretrappole}

\textbf{Requirement}: Dexterity 2

\textbf{Saving Throws}: +2 Reflexes, +1 Fortitude

The \textbf{first time} you take the Feat you get a +1d6 bonus on your saving throw to avoid the effect of traps.

The \textbf{second time} you take the Feat, Weapon Proficiency requirement 5, even if the trap does not grant a saving throw your natural propensity to avoid damage grants you a Reflex saving throw to halve the damage.

It is also possible to use this Feat, use a Reaction, to avoid Sneak Attack (Reflex save higher than the opponent's Attack roll).

%The \textbf{third time} you take the Feat Requirements Weapon Proficiency 9, the saving throw if successful allows you to avoid any effects of the trap, if physically possible.

\subsection*{Wonderful Dodge}\index{Wonderful Dodge}\label{schivataprodigioso}

\textbf{Requirement}: Dexterity 3

\textbf{Saving Throws}: +2 Reflexes

As a Reaction to an Attack Action you can add +1 to your Defense. You can use the Feat up to 3 times per day.

The \textbf{second time} you take the Feat, Weapon Proficiency requirement 4, an opponent does not take the flanking hit bonus against you.

The \textbf{third time} you take the Feat, requirement Weapon Proficiency 8, Dexterity 4, an opponent does not take the attack bonus to attack you from behind.

You can use the Feat even after it is known how much you have been hit. 

\subsection*{Second leather}\index{Second leather}\label{secondapelle}\hypertarget{Second leather}{}

\textbf{Requirement}: Weapons Proficiency 1

\textbf{Saving Throws}: +2 Fortitude

Constant use of armor allows you to wear them without major penalties.

the penalties on proficiency checks given by the armor decrease by 1.

The \textbf{second time} that you take this Feat, Weapon Proficiency requirement 6, the penalty on proficiency checks decreases by an additional 1. The penalty on movement penalties decreases by 1 meter. You can sleep in medium armor without being fatigued.

The \textbf{third time} you take this Feat, Weapon Proficiency requirement 11, the penalty on proficiency checks decreases by an additional 1. The penalty on movement penalties decreases by an additional 1 meter. You can sleep in heavy armor without being fatigued.

\subsection*{Hound}\index{Hound}\label{segugio}

\textbf{Requirement}: Intelligence 1, Wisdom 1, Weapon Proficiency 1

\textbf{Saving Throws}: +1 Reflexes, +1 Will

You have a natural talent for following people

With two Actions you focus on a target you can see and as long as you see it you stay focused. All your Actions involving that target have a +1 bonus. Maintaining focus costs 1 action per round.

The \textbf{second} time you take this Feat, Weapon Proficiency requirement 10, the bonus increases to +2.

The \textbf{third} time you take this Feat, Weapon Proficiency requirement 16, the bonus increases to +3.

The bonus can be used on attack rolls, saving throws caused by the opponent, and competence checks, not on damage.

\subsection*{Without Trace}\index{Without Trace}\label{senzatraccia}

\textbf{Requirement}: Safe passage

\textbf{Saving Throws}: +2 Will, +1 Reflexes

The ability to leave no footprints in the chosen environment. Each time you take this Feat you can choose a different environment (see Sure Step Feat) whose Feat you took. The difficulty of the Tracking check to chase you increased by 10.

\subsection*{Black siphon}\index{Black siphon}\label{sifonenero}

\textbf{Requirement}: Magic Proficiency 6, Tazher Adept, Common Trait points 6

\textbf{Saving Throws}: +1 Fortitude, +2 Will

By increasing by half, rounded up, the Magic Points used in the spell, which must be instantaneous in effect and cause Hit Point damage, you regain an amount of Hit Points equal to half the creature that lost the most.

The casting of the spell if of 2 or less Actions becomes 1 round.

\subsection*{Unlucky}\index{Unlucky}\label{sfortunato}

\textbf{Requirement}: Lucky, at least 6 points in the sum of Traits

\textbf{Saving Throws}: +1 Fortitude, +1 Will

Once per day you can turn a 6 rolled by the Storyteller (Attack Rolls, Feat Checks, Saving Throws) into a 1.

\subsection*{Shoot and Run}\index{Shoot and Run}\label{sparaescappa}

\textbf{Requirement}: List of crossbows 3

\textbf{Saving Throws}: +1 Fortitude, +1 Reflex

While performing a Move Action you can reduce the reload time of your crossbow by 1 Action. In the case of light or one-handed crossbows you can therefore reload it while moving, in the case of heavy crossbows reduce the loading time by 1 action.

\subsection*{Specialist}\index{Specialist}\label{specialista}

\textbf{Requirement}: Magical Proficiency 3

\textbf{Saving Throws}: +2 Fortitude

Choose a spell you know, the Magic Points spent to cast this spell decrease by 1.

The Feat can be taken multiple times on different spells and even on the same one as long as the Magic Points to cast the spell are greater than or equal to 50\% of the original cost. 

\subsection*{Stay down!}\index{Stay down!}\label{staigiu}

\textbf{Requirement}: Weapons Proficiency 3

\textbf{Saving Throws}: +2 Fortitude, +1 Will

When your attack causes two critical rolls on an opponent, the force of the blow is enough to knock him prone. The opponent must make a Fortitude saving throw (DC equal to the last 10 + attack roll with the weapon that caused the last critical roll) or fall prone. The Feat works on creatures of the same size or smaller than the character.

The \textbf{second time} you take the Ability you can also affect creatures of a larger size.

The \textbf{third time} you take the Feat you can also affect creatures two sizes larger. Grade 3 cannot be combined with grade 2.

\subsection*{Supreme}\index{Supreme}\label{supremo}\hypertarget{Supreme}{}

\textbf{Requirement}: the sum of CM+CA is at least 4 points higher than the previous time it was taken.

\textbf{Saving Throws}: +1 to two saving throws of your choice

Through the Supreme Feat it is possible to increase a Characteristic by 1 point respecting the rules of \hyperlink{increase characteristics}{Increase characteristics} (page \pageref{increase characteristics}).

\subsection*{Tactical}\index{Specialist}\label{tattico}

\textbf{Requirement}: Weapon Proficiency 1, Intelligence 1

\textbf{Saving Throws}: +1 Fortitude, +1 Will

You have an almost instinctive ability to manage and predict the outcome of fights.

The \textbf{first time} you take this Feat you can exchange, round by round, the outcome of the Initiative between you and a companion respectively at melee range. Cost 1 Action.

The \textbf{second time} you take this Feat, requirement Intelligence 2, Weapon Proficiency 6, you can exchange initiative, round for round, between three of your companions who are within melee range of each other. Cost 1 Action.

\subsection*{Storm of Fury}\index{Storm of Fury}\label{tempestadifuria}

\textbf{Requirement}: Empty Fist List 2, Dexterity 1, Strength 1

\textbf{Saving Throws}: +2 Reflexes, +1 Will

When you use this Feat you can declare that you use Storm of Fury as your only action (3 Actions).

Make a single attack roll with -1d6 and if you hit you cause a number of critical damage equal to Weapon Proficiency/4.

\subsection*{Hollow head}\index{Hollow head}\label{testacava}

\textbf{Requirement}: List of crossbows 4

\textbf{Saving Throws}: +2 Fortitude

You can give a deadly effect to your projectiles.

Your crossbow bolt increases by one damage size.

\subsection*{Precise shot}\index{Precise shot}\label{tiropreciso}

\textbf{Requirement}: Dexterity 3, Weapon Proficiency 1

\textbf{Saving Throws}: +2 Reflexes

You gain +1 to hit and +1 to damage and attack rolls, with thrown weapons, bows, or crossbows, with targets within 30 feet.

%\begin{center}
%\includegraphics[width=0.7\linewidth]{immagini/kenilguerriero.png}
%\end{center}

\subsection*{Quick shot}\index{Quick shot}\label{tirorapido}

\textbf{Requirement}: Dexterity 3, Accurate Shot, Weapon Proficiency 2

\textbf{Saving Throws}: +2 Reflexes

When using a bow, crossbow, or throwing a weapon, the multiple attack penalties are lower.

Each projectile fired beyond the first takes a cumulative -4 to attack roll (rather than -5). The first hit has a normal attack roll, the second has a -4, the third a -8...

\subsection*{Toccata and Fugue}\index{Toccata and Fugue}\label{toccataefuga}

\textbf{Saving Throws}: +2 Reflexes

Your attacks have a base penalty of -5 and you can take an Action of 1 more move. You cannot perform more than one bonus move action in this manner. It costs an Immediate Action.

\subsection*{Pitying touch}\index{Pitying touch}\label{toccopietoso}

\textbf{Requirement}: Good Patron, Laying of Hands, Traits in common with Patron 3

\textbf{Saving Throws}: +2 Will, +1 Fortitude

Your touch soothes not only wounds but also suffering and pain. Whenever you use the Lay on Hands Feat you can also add this Feat as an Immediate Action.

By using the Laying on Hands you can, forgoing the indicated number of healing d6s, remove the following conditions:

\textbf{2d6} Score Common Traits 3: Fatigued (down 1 rank)

\textbf{3d6} Score Common Traits 6: Stunned - Confused - Scared

The \textbf{second time} you take the Feat, common Trait sum requirement 11, remove the following conditions:

\textbf{4d6} Score Common Traits 12: Sick - Poisoned - Paralyzed - With Maximum Hit Points reduced (2d6 recovery) - Fatigued (reduce by two ranks)

You can also, by completely forgoing healing dice, regenerate limbs or remove the Blinded or Deaf condition.

\subsection*{One with the magic}\index{One with the magic}\label{tuttunoconlamagia}

\textbf{Requirement}: Adept of Magic

\textbf{Saving Throws}: +1 to two saving throws of your choice

Each time you take the At One with Magic Feat you must determine which Characteristic it connects to.
Your Characteristic has a +1 value for determining spell effects.

\subsection*{One arm, one weapon}\index{One arm, one weapon}\label{unbracciounarma}

\textbf{Requirement}: Weapons Proficiency, 2

\textbf{Saving Throw}: +1 Fortitude, +1 Will

Choose a Weapon List. Strength damage applied by weapons from that list increases by 1.

The Feat can be taken multiple times and the Weapon List score must be 4 times higher than the number of times you have taken it.

If you take \textbf{4 times} this Feat on the same Weapon List the damage bonuses are reduced to +4 but you roll the damage twice and choose the better result. Does not apply on burst damage or critical damage.

\subsection*{One shot one dead}\index{One gesture one dead}\label{ungestounmorto}

\textbf{Requirement}: Magical Expertise 1, Adept of Magic 1

\textbf{Saving Throws}: +1 Reflexes

The \textbf{first time} you take this Feat you gain a +1 on attack rolls to spells that require an attack roll.

The \textbf{second time} you get a +1 to attack roll for each time you took the Adept of Magic Feat.

\subsection*{One body, one mind, one spirit}\index{One body, one mind, one spirit}\label{USCMS}\index{USCMS}

\textbf{Saving Throws}: +1 of your choice

Assign one point to Weapon Proficiency or Magical Proficiency. This Feat can be taken a maximum of 2 times.

In the manual you will also find this Feat under the name \textbf{USCMS}.

\subsection*{Vampire}\index{Vampire}\label{abilitavampiro}

\textbf{Requirement}: Smell of blood (Benefits)

\textbf{Saving Throws}: +2 Fortitude, +1 Will

Your bloodlust becomes a cure. The bloodlust bonus can increase up to +5.

If the bonus increases from +3 to +4 or +5 you can, by swallowing the opponent's blood, heal yourself for 1d6 using 2 Actions

\subsection*{Iron Will}\index{Iron Will}\label{volontaferrea}

\textbf{Requirement}: Wisdom 0

Over time you have trained your will to resist any weakness and fear.

The first time you take this Feat you gain a +2 bonus on Will saving throws. The bonus is cumulative, +2 the first time, +1 the second, +1 the third.

The \textbf{fourth time} you take this Feat you can decide to automatically succeed on a Will save once per day before rolling the dice. It costs a Reaction.

\end{multicols}

\vfill

%\begin{center}

%\includegraphics[width=0.7\linewidth]{immagini/Granblue.Fantasy.full.2108782.png}

%\filltopageendgraphics[width=0.7\linewidth]{images/Granblue.Fantasy.full.2108782.png}

%\emph{around a fire, victorious another day!}
%\end{center}

\pagebreak

\begin{multicols}{2}


\subsection{Feat Grouping by Style}

To facilitate the transition for those coming from other role-playing games with classes, the Feats for the more canonical classes are divided here.

They are clearly just indications, in OBSS the character can be built as one prefers and as the story he experiences is instructing him.

These are suggestions to facilitate the construction of a character for those who are new to the Old Bell School System, it is obvious that it is possible to draw from all the groupings presented!

\end{multicols}

\bigskip

\begin{multicols}{3}

{\small


\begin{flushleft}
\subsubsection*{Warrior}

I lengthen\\
Weapon Focus\\
Whirling Attack\\
Powerful blows\\
Kill shot\\
Fighting Blindly\\
Dance of the Blades\\
Defense ready\\
Dancing Scourge\\
I said FALL!\\
Iaijutsu\\
My Head is Harder\\
Parry\\
Leathery Skin\\
Truly evil person\\
Continue\\
This is my weapon!\\
Stone resistance\\
Lightning reflexes\\
Second skin\\
Stay down!\\
One arm, one weapon\\

\subsubsection*{Barbarian}

Ferocity\\
Forged in Fury\\
Fury\\
Instinctive knowledge\\
My death your death\\
Human mountain\\
One arm, one weapon\\
Iron Will\\

\subsubsection*{Thief}

Sneak Strike\\
Weakening Strike\\
Paralyzing Strike\\
Quick extraction\\
Enrage\\
Arrow called, arrow delivered\\
Juggler\\
Improvise\\
Opportunist\\
First Blood\\
This is my dagger\\
Dodging traps\\
Wonderful dodge\\
Touch and run\\

\subsubsection*{Paladin}

The Patron is my Weapon\\
Laying on of hands\\
Channeling energy\\
Channeling selective energy\\
Mental wall\\
Retaliation\\
Pitiful touch\\

\subsubsection*{Bard}

Loaded Dice\\
Coordinated damage\\
Lucky\\
Instill Courage\\
Instill Fear\\
Versatile Litany\\
Polyglot\\
Tactical\\
Bad lucky\\

\subsubsection*{Ranger}

Horse Archer\\
Two-Weapon Fighting\\
Defend Mount\\
Double portion\\
Clinical Eye\\
Hawk eye\\
Safe step\\
The bigger they are, the more noise they make when they fall\\
Hound\\
Without a Trace\\
Shoot and Run\\
Hollow Head\\
Accurate shooting\\
Quick Shot\\

\subsubsection*{Druid}

Adept of Magic\\
Animalia\\
Elemental Form\\
Distilling potions\\
Shayalia's daughter\\
The Patron is with me\\
Pure Blood\\

\subsubsection*{Cleric}

Adept of Magic\\
Devotee's Armor\\
Typist\\
Loyal\\
The Patron is with me\\
Power of the Patron\\
Turning the Undead\\
Specialist\\
One with the magic\\

\subsubsection*{Wizard/Sorcerer}

Adept of Magic\\
Pet / Familiar\\
Magic Battery\\
Extended Battery\\
Creating Magical Items\\
Crafting Greater Magic Items\\
Crafting Wondrous Magic Items\\
Crafting Mythical Magic Items\\
Typist\\
Deciphering magical writings\\
Elementalist\\
The Patron is with me\\
The Patron is my Weapon\\
Powerful spells\\
Detecting Magic\\
Wise\\
Specialist\\
One with the magic\\
One gesture kills one\\

\subsubsection*{Monaco}

Wings of the Phoenix\\
Armor of the Enchanted Mountain\\
Psychic Energy\\
Psychic Strike\\
Psychic Ray\\
Silver Crane\\
Spring\\
Human mountain\\
Mental Wall\\
Quick Step\\
Iron fist\\
Storm of Fury\\

\subsubsection*{Gish (Warrior/Mage)}

Devotee's Armor\\
Magic Warrior\\
The Patron is my Weapon\\
Combat Spellcaster\\
Cautious Enchanter\\
Infuse Superior Energy\\
Infuse Greater Magical Energy\\
The shield is my friend\\
Magical roots\\
One with the magic\\


\end{flushleft}
}

\end{multicols}



\pagebreak

\section{Family}\index{Family}\label{famiglio}\hypertarget{Family}{}

\medskip

\begin{changemargin}{0.3cm}{0.3cm}\begin{enfasi}{
Mr. Wing's nephew: Listen mister, there are three rules to follow, though.

Rand: Oh, yeah? And what would they be?

Mr. Wing's nephew: Keep him away from light, he hates strong light, especially sunlight. He would die. And keep him away from water, don't let him get wet. But the most important thing, the rule that he must never forget is that even if he cries, even if he makes a scene and begs her, she must never, never feed him after midnight. Understood? (Gremlins, Film, 1984)} 
\end{enfasi}\end{changemargin}\medskip

\begin{multicols}{2}

\lettrine[lines=2, lhang=0.33, loversize=0.25, findent=1.5em]{The}{ pets} are animals chosen by the character, through the Familiar Feat, to help him in his adventures and for company . A familiar has a special bond with its master.

A familiar is a normal animal but is treated as a magical creature for the purposes of determining any effects that depend on its type.

Only a normal, unmodified animal can become a familiar.

A familiar grants Special Abilities to its master, these Special Abilities apply only when the master and the familiar are within 100 m of each other.

A special 4-hour ritual in the animal's native environment is necessary to make it become a familiar.

If a familiar is dismissed, lost, or dies, it can be replaced a week later with a special ritual that costs 2 points of the character's temporary Constitution. The ritual takes 8 hours to complete.

\medskip


\textbf{Table: Pet Types}\index[Tables]{Family Types Table}

\medskip

\begin{tabularx}{0.45\textwidth}{lX}
\textbf{Familiar} & \textbf{Ability acquired by the master}\\
\toprule
Owl & +2 on Arcana checks\\
Raven & +2 on Intimidate checks\\
\emph{Dobi}& +2 on Saving Throw vs. Enchantment\\
Weasel & +1 on Intelligence checks\\
Hawk & +2 Sight Awareness\\
Cat & +2 on Stealth checks\\
Owl & +2 Hearing Awareness\\
Otter & +2 on Swim checks\\
Lizard & +2 on Survival checks\\
Bat & +2 on Acrobatics checks\\
Rat & +1 to Saving Throw vs. Disease\\
Hedgehog & +1 to Will save\\
Toad & +2 to Save vs Poison\\
Monkey & +2 on Fairy Hands, Escape Artist checks\\
\emph{Topi} & become Topi's familiar!!!\\
Fox & +1 Reflex Saving Throw\\
\end{tabularx}

\bigskip

Use the base statistics of a creature of the familiar's species, making the following changes.

\medskip{}

\textbf{Attacks}: Use the master's Weapon Proficiency if higher. Use the pet's Dexterity or Strength modifier, whichever is higher, to calculate the pet's attack bonus with Natural Attacks. The damage is the same as that of a normal creature of the familiar's species. The familiar acts in the master's round.

\medskip

\begin{center}
\includegraphics[width=0.65\linewidth]{immagini/donnadrago2.png}

\emph{Henry Justice Ford}
\end{center}

\medskip

\textbf{Defense}: the familiar has a Defense equal to that of the standard animal plus a bonus due to the owner's Magical Expertise. See Familiar Feats table.

\medskip

\textbf{Saving Throw}: For each saving throw, use the familiar's saving throws or the master's saving throws, whichever is better. The familiar applies its Ability values ​​as a bonus on saving throws and takes none of the bonuses its master may have.

\textbf{Familiar Actions}\index{Familiar Actions}: Commanding a familiar takes 1 Action. The familiar performs 2 Actions per round. Without commands the Familiar does nothing except defend itself and attack those who attack it.


\textbf{Description of Familiar Abilities}

All familiars possess Special Abilities (or attribute them to their masters) depending on their master's Magical Expertise score. The Special abilities listed in the table are cumulative.


\end{multicols}


\textbf{Table: Familiar Skills and Bonuses}\index[Tables]{Familiar Skills Table}

\medskip

\begin{tabularx}{0.95\textwidth}{cccX}
\textbf{Master's CM} & \textbf{Defense Bonus} & \textbf{Intelligence Bonus} & \textbf{Special}\\
\toprule
1-2 & +1 & 0 & Alert, Share Spells, \\
& & & Empathic Bond\\
3-4 & +1 & +1 & Send Touch Spells\\
5-6 & +2 & +1 & Talk to Animals of Your Kind\\
7-8 & +3 & +1 & Talk to the Master\\
9-10 & +3 & +2 & -\\
11-12 & +4 & +2 & See through Familiar\\
13-14 & +5 & +2 & Improved Sending Touch Spells\\
15-16 & +5 & +3 & -\\
17-18 & +6 & +3 & -\\
19-20 & +7 & +3 & -\\
\end{tabularx}

\begin{multicols}{2}


\textbf{Master's Magical Expertise}: the number indicated here is the Magical Expertise value of the familiar's master, divided into bands.

\textbf{Defense Bonus}: the indicated bonus is to be added to the pet's Defense.

\textbf{Intelligence Bonus}: the indicated bonus is added to the pet's Intelligence score.

\textbf{Special}: the special abilities acquired by the familiar (and/or the master).

\emph{\textbf{Alert}}: when the familiar is within arm's reach of the master, he gains +1 on Awareness checks

\emph{\textbf{Sharing Spells}}: At his discretion, the master can cast any Spell that affects himself on his familiar.

The master can cast spells on his familiar even if they normally have no effect on creatures of the familiar's type (magical creatures).

\emph{\textbf{Empathic Bond}}: The master has an empathic bond with his pet up to a distance of 1 km. The master cannot see through the familiar's eyes, but can communicate with it telepathically. Due to the limited nature of the bond, only generic emotions can be communicated (fear, nervous, calm, joy...).

\emph{\textbf{Send Touch Spells}}: Your pet can cast Touch Spells for you. If the master and familiar are within 30 feet when the master casts a Touch Range Spell, he can designate his familiar as \emph{spell deliverer}.

The familiar can transmit the Spell just like the master. The familiar uses one of its Actions to make an attack.

\emph{\textbf{Talking with the Master}}: the familiar and the master can communicate verbally, as if they were using a common language. Other creatures or animals are unable to understand their conversation except by using magical aids. The capacity works within 50m and they must be heard.

\emph{\textbf{Speak with Animals of Its Kind}}: the familiar is able to communicate with animals of its generic species: bats with bats, mice with rodents, cats with felines, hawks and owls, and crows with birds , snakes and lizards with reptiles, toads with amphibians, monkeys with other primates, weasels with stoats and mustelids... Communication is limited by the Intelligence of the creatures the familiar communicates with.

\emph{\textbf{See Through Familiar}}: The master can see through the familiar. Activating this Skill costs 1 Immediate Action. The familiar must be within 50 meters.

\emph{\textbf{Transmit Improved Touch Spells}}: The familiar can transmit Touch Spells for him. If the master and the familiar are within 60 feet when the master casts a Touch Range Spell, he can designate his familiar as \emph{the one who delivers the Spell}.

\emph{No matter how special}, intelligent and unique a familiar remains an animal and as such cannot use magic objects or scrolls, it can use a potion if it has the ability to drink it. A particularly intelligent familiar could perform simple and straightforward tasks.

\end{multicols}

\begin{center}
\includegraphics[width=0.39\linewidth]{immagini/familiar.png}

\emph{Henry Justice Ford, a great familiar...}
\end{center}



\pagebreak

\section{Other Special Skills}


\begin{multicols}{2}

\lettrine[lines=2, lhang=0.33, loversize=0.25, findent=1.5em]{Q}{este} Abilities are not selectable by the player, but can be innate in creatures.

\subsection{Ethereal}\index{Ethereal}\label{etereo}

A creature that has become Ethereal is located in the Ethereal Plane which is superimposed on the Material Plane.

An ethereal creature is Invisible, without substance, and capable of moving in any direction, even up and down. An ethereal creature can move through solid objects, including other living creatures. An ethereal creature can see and hear what happens on the Material Plane, but everything appears gray and insubstantial. The sight and hearing of an ethereal creature on the Material Plane are limited to a distance of 30 feet.

Spells, if not appropriately formulated and modified, do not act on ethereal creatures. An ethereal creature has damage resistance to light or void, and ignores all other forms of energy.

An ethereal creature cannot attack a material creature, and spells cast while in the ethereal condition can only affect ethereal elements. Some creatures or material objects have attacks or special effects that also work on the Ethereal Plane. An ethereal creature treats all other ethereal creatures as if all were material.

\subsection{Damage Resistance}\index{Damage Resistance}\label{resistenzaaldanno}

Certain creatures or protections grant the ability to Resist a type of Damage.

Being Damage Resistant automatically means you halve the damage you take before applying any other protection or saving throw.

Damage Resistance can also take on values. When Damage Resistance: Electricity is written, the subject automatically halves electricity damage, if Damage Resistance: Electricity 10 is written, it means that it reduces electricity damage by 10 points before applying the saving throw or other bonuses.

A creature with a Fire Resistance halves (reduces) all damage it takes from flames, magical or otherwise unless otherwise noted.

There may be abilities or spells that ignore this Resistance. Multiple equal resistances do not add up, since two objects give me fire resistance I do not reduce the damage by a quarter, only one is applied.
If an ability ignores damage resistance it will pass resistance even if I have two or more sources of resistance.

\subsection{Damage Reduction - DR}\index{Damage Reduction}\label{resistenzaaldannodr}\hypertarget{damage reduction}{}

Certain creatures or Abilities grant the supernatural ability to resist damage from certain types of weapons or up to a certain amount (per attack).

It usually takes the value of XX/ZZ or how much damage (XX) is ignored if you are not attacked with (ZZ). Ignoring damage also means that effects related to the attack don't work, such as poisons on the weapon.

\begin{center}
\includegraphics[width=0.8\linewidth]{immagini/morteachille.png}

\emph{Paris shot Achilles with an arrow - Pieter Paul Rubens - Date 1630-1632}
\end{center}


Certain weapons, particularly magical ones, can ignore the DR \index{Ignore the DR}

\medskip

\textbf{Projectiles (arrows, darts, rocks) fired from \emph{magic thrusters} are NOT considered magical.}\index{Magic arrows}

\subsection{Magic Resistance}\index{Magic Resistance}\label{resistenzaallamagia}

Magic Resistance can be indicated in two different ways.

It can be indicated with a dice, e.g. \emph{Magic Resistance. The deva has +1d6 on saving throws against spells and other magical effects}. In this case the bonus applies as indicated.

Or followed by a number and level, e.g. \emph{Magic Resistance: 3lv}. In this case the creature is not affected by spells of that level or lower. A spell is considered to be one level higher for each magic critical obtained in the Magic Test.

Even if the target of the magic is not affected by direct effects, it is still affected by indirect effects, for example it can fall into the pit created by a Disintegration spell.

Magic Resistance cannot be lowered even by the creature that possesses it.

\subsection{Immunity to damage}\index{Immunity to damage}\label{immunitaaldanno}

It is extremely rare but there are creatures or magical effects that make you immune to a form of damage, be it physical (weapon damage..) or magical (various forms of energy).

A creature immune to a form of damage takes no damage from that attack. A creature that instead has the ability to have its own irresistible damage, meaning that it cannot be reduced by resistance, will only partially penetrate the creature's immunity, making it only resistant to that damage.

A creature that rolls \emph{Damage Immunity Void, Poison; weapons +2} means that he does not suffer damage from Void or Poison and to wound him you need a weapon with a +3 or higher magic bonus, or a character who attacks with natural weapons and is level 12 or higher or who has taken the Empty Fist Weapon List at least 6 times. 

See the diagram of \hyperlink{magical weapons equivalence}{Magical weapons equivalences} (page \pageref{magical weapons equivalence})

\subsection{Vulnerability to Damage}\index{Vulnerability to Damage}\label{vulnerabilitadanno}

Certain creatures or spells make certain effects more effective, causing greater damage to the vulnerable subject.

Being Vulnerable to a specific type of Damage automatically means double the damage received before applying any other protection or saving throw.

A creature with a Vulnerability to Fire doubles all damage taken then makes the saving throw indicated by the spell or effect if possible.

\subsection{Fear}\index{Fear}\label{paura}

Spells, Magic Items, and certain creatures can affect characters with the Fear effect. A creature with Fear can't suppress its aura if it is innate unless otherwise described. The difficulty with which to make the Will saving throw is always marked. A creature immune to Fear can't be frightened whether the source is natural or magical.

\textbf{Scared}\label{spaventato}\index{Scared}

A frightened creature has -1d6 on attack rolls, saving throws, and proficiency checks as long as the source of its fear is visible. A frightened creature cannot willingly approach the source of its fear.

\subsection{Paralysed}\index{Paralysed}\label{paralizzato}

There are several methods to Paralyze a creature, both magical and natural. While natural ones often have systems for freeing oneself later, magical systems can provide for freeing oneself from paralysis or not, perhaps only after a certain period of time.

A paralyzed character cannot perform Actions or Reactions or speak, melee attacks against her have a +2d6 bonus. The creature is aware of its surroundings and does not drop objects. The creature automatically fails Reflex saving throws. The creature loses its Dexterity bonus to Defense.

\end{multicols}

\vfill

\begin{center}
\includegraphics[width=0.4\linewidth]{immagini/the-scream.png}

\emph{The Scream (original title: Skrik)\\Edvard Munch - Date 1893–1910}
\end{center}

\pagebreak

\section{The Magic}\index{Magic}\label{lamagia}

\begin{changemargin}{0.3cm}{0.3cm}\begin{enfasi}{
The magic is not in the pendulum, but in those who use it. (NCIS - Crime Unit)\\

You will not let the one who practices magic live. (Book of Exodus)(Always depending on your Traits...)\\

A wizard is never late, Frodo Baggins. Nor in advance. He arrives precisely when he means to. (Gandalf, The Lord of the Rings - The Fellowship of the Ring. J.R.R. Tolkien)} \end{enfasi}\end{changemargin} \medskip

\begin{multicols}{2}

\lettrine[lines=2, lhang=0.33, loversize=0.25, findent=1.5em]{M}{agic} permeates game worlds, and its most common form is that of a spell. This chapter provides the rules for casting spells.

\medskip

\subsection{What is a Spell?}\index{What is a Spell}\index{Spell definition}

A spell is a manifestation of power. Each spell is the fruit of power and knowledge, the enchanter is a superior intermediary who channels the power of the Patrons. When casting a spell, a character composes gestures, words and uses objects that do nothing but connect him to the source, the Patron.

Spells can manifest protective weapons or barriers, they can inflict damage or heal life energies. Countless spells have been created throughout Yeru's history, many of which have been forgotten. Some may still be hidden within the pages of dusty spell tomes within ancient ruins or locked away in the minds of dead deities. Or they might one day be reinvented by a character who has enough power and ability to do so.

\subsection{How to make magic! (In summary)}\index{How to make magic! (In summary)}

Your character must have taken at least one Magic List by investing the first point in Magical Expertise or by having taken the Adept of Magic Feat.

Access to the Magic List allows you to access and therefore be able to learn the spells that belong to it. By assigning a point to Magical Proficiency you can access the Universal List.

Magical Expertise allows you to have more Magic Points, more spells and also make your spells harder to resist and together with the Adept of Magic Feat you can access higher level spells.

However, don't worry, this chapter contains everything you need to know!

\subsection{The characteristics of the spells}\index{The characteristics of the spells}\label{caratteristicheincantesimi}

The description of each spell begins with a block of information that includes the level, the Magic Lists to which it belongs, casting time, range, components and duration of the spell. The rest of the description informs us of the spell's effect.

When a character casts any spell, the following basic rules are used regardless of the spell's effect.

\medskip

\begin{center}

\includegraphics[width=0.7\linewidth]{immagini/Hex32.png}

\emph{The Witchcraft Art of Jacques de Gheyn II}

\end{center}


\subsubsection{Casting Time}\index{Spell Casting Time}\label{magietempodilancio}\index{Spells, Casting Actions}

Most spells can be cast with two Actions. Some spells require an Immediate Action, a Reaction Action, or much longer to cast.

\textbf{Immediate Action}

A spell cast with an Immediate Action is particularly quick. You can use an Immediate Action during your round to cast the spell that is Immediate, as long as you have not already taken an Immediate Action during your round. During the same round you cannot cast another spell, unless it is a 0-level spell (called Cantrips).

\textbf{Reactions}

Some spells can be cast as reactions. These spells take a fraction of a second to create and can be cast in response to an event. If a spell can be cast as a reaction, the spell's description tells you exactly when you can do so. You must have a Reaction Action available and not have already used it.

\textbf{Longer Launch Time}

Certain spells take longer to cast: minutes or even hours. When you cast a spell with a casting time longer than two Actions, each round after the first is considered used in casting the spell. For those rounds it's as if you have to maintain Concentration, to determine any effects.

In the final round, when the casting time is exhausted, you roll the spell with the initiative you rolled at the beginning of the formulation and use an Action to cast the spell.

\subsubsection{Magic Lists}\hypertarget{magic schools}{} \index{Magic Lists}\label{magielistadimagia}\index{Spells, Magic Lists}

Magical traditions throughout Yeru have formalized spells over the millennia into lists consistent in type and effect. There are therefore lists that concern Fire or other elements, illusions, healing energies...

The Lists presented here only include those codified and taught in magic schools. Ancient legends tell of further lists created, curated, and circulated in small circles or sects. One of these secret lists is that of the Gnome Devotees of Shayalia, a purely natural list that mixes the traditional List of Animals and Plants with some spells from the Element Lists.
Other more obscure lists are the demonic or Aboleth ones, some others are linked to belonging to groups of Devotees. Other, more nefarious lists corrupt the souls of the characters by also imposing Traits. These lists will normally be closed to the character but it is not certain that with the increase in Magical Expertise he himself will not create new lists of spells.

Magic Lists help describe spells; they do not have their own rules, although some rules may refer to these lists. The related characteristic is indicated next to the name of each List.

\begin{itemize}
\item
\emph{Abjuration} (INT) concerns spells of a protective nature, although it also contains some for aggressive use. These spells create magical barriers, negate harmful effects, or banish creatures to other planes of existence.

\item
\emph{Water} (DES) are the spells that act on the water and cold elements and to a minimal extent also on healing

\item
\emph{Air} (CHA) concerns spells that manipulate and use air and also electricity.

\item
\emph{Enchantment} (CAR) concerns spells that act on the minds of others, influencing or controlling their behavior. These spells can make enemies consider you a friend or even control another creature as if it were a puppet.

\item
\emph{Animals and Plants} (WIS) these are spells that act on animals and plants, natural or magical.

\item
\emph{Heal} (WIS) concerns spells that allow you to recover physical and mental energy and eliminate weaknesses and poisons.

\item
\emph{Divination} (WIS) concerns spells that reveal time-lost, forgotten information, visions of the future, the location of hidden objects, the truth behind illusions or images of distant people and places.

\item
\emph{Conjuration} (INT) involves spells that transport objects and creatures from one place to another. Some spells summon creatures or objects to the caster's side, while others allow the caster to teleport from one location to another. Some summons create items or effects out of thin air.

\emph{Fire} (STR) The most dangerous spells are in here with everything needed to burn and incinerate.

\begin{center}
\includegraphics[width=0.65\linewidth]{immagini/Leonids-1833.png}

\emph{The most famous depiction of the famous 1833 Leonids \hyperlink{meteor shower}{Meteor Storm}}
\end{center}


\emph{Illusion} (WIS) involves spells that deceive the senses and minds of others. They make people see things that don't exist, they don't notice things that exist, they make them hear fake noises or remember things that never happened. Some illusions create ghostly images that anyone can see.

\item
\emph{Invocation} (CON) involves spells that manipulate magical energy to produce a desired effect.

\item
\emph{Necromancy} (CON) involves spells that manipulate the energies of life and death. These spells can grant an additional pool of life force, drain the life energy from another creature, create undead, or even bring the dead back to life (if granted).

\emph{In OBSS only a Patron has enough power to bring a dead person back to life}.

\item
\emph{Earth} (CON) Spells that act and move the earth

\item
\emph{Transmutation} (DEX) involves spells that change the properties of a creature, object, or environment.

\item
\emph{Universal} some spells are cornerstones of magic itself and as such accessible to all spellcasters. To access the spells contained in this Magic List you must have at least one point in Magical Expertise. The maximum level of spells that can be cast is equal to the number of times the Adept of Magic Feat has been taken, with a minimum of 1.

\end{itemize}

\begin{changemargin}{0.3cm}{0.3cm}\begin{narratore}{Make it clear that each Magic List characterizes the character in a unique way. The Magic Lists taken give a different depth and role to each other.
}\end{narratore}
\end{changemargin}


\subsubsection{Range}\index{Range}\label{magiegittata}\index{Spells, Range}

The target of a spell must be within the spell's range. For a spell like Arcane Bolt, the target is a creature. For a spell like fireball, the target is the point in space from which the fireball explodes. Most spells have a range in meters. Some spells can target only a creature (including you) with which you are in physical contact. Other spells, such as the shield spell, only affect you: these spells have a range of \emph{personal}. A spell that has \emph{an ally} as its area of ​​effect can also be cast on yourself.

Spells that create cones or lines of effect that originate from you also have personal range\index{Personal Range}, indicating that you are the point of origin of the spell's effect (see \emph{Areas of Effect} later in this chapter).

\subsubsection{Casting Spells in Armor}\index{Casting Spells in Armor}\label{magielanciareincantesimiinarmatura}\index{Spells, in Armor}

Given the mental concentration and precise gestures required, the armor distracts and unbalances the flows. The Magic Test in casting the spell is mandatory and is modified as indicated in the \hyperlink{armorandmagic}{armor}{armor} section (page \pageref{armorandmagic}).

\subsubsection{Optional - Armor Enchantments}\index{Optional - Armor Enchantments}

The armor blocks magical flows and does not allow correct channeling.
This option means that all spells cast by the caster become Contact Ranged, i.e. they can only be discharged through the caster's hand. No Magic Checks are required for wearing armor.

\subsubsection{Duration}\index{Spell Duration}\label{magiedurata}\index{Spells, Duration}

A spell's duration is the length of time it persists. Duration can be expressed in rounds, minutes, hours, or even years. Some spells specify that their effects last until the spell is dispelled or destroyed. A spell can be interrupted by your spellcaster as an immediate action.\index{Interrupting your own spell}

If a magical critical doubles the duration, it is always understood as referring to the initial duration. E.g. if the duration is 2 hours after the first doubling it becomes 4 hours, with the second it becomes 6 hours and then 8 hours..\index{Magic critical success on duration}

\begin{itemize}

\item
\emph{Instant}

Many spells are instantaneous. The spell harms, heals, creates, or alters a creature or object so that it cannot be dispelled, as its magic exists only for an instant.

\item

\emph{Concentration}\index{Concentration}\index{Spells, Concentration Duration}

Some spells require you to maintain concentration to keep their magic active. If you cannot maintain concentration, the spell will end. If a spell must be maintained through concentration, this is indicated under Duration, the spell specifies how long you can maintain concentration on it. You can end your concentration at any time by using a Reaction.

Normal activities, such as moving and attacking, do not interfere with concentration. Maintaining concentration costs 1 Action per round.
\end{itemize}

\subsubsection{Components}\index{Components}\label{magiecomponenti}\index{Spells, Components}

The components of a spell are the material requirements you must meet to cast it. Each spell's description indicates whether it requires verbal (V), somatic (S), or material (M) components. If you are unable to provide one or more of the spell's components, you will not be able to cast it.

Most spells require you to chant mystical words. The words, rhythm, cadence and resonance allow for harmony with the Patron who provides the magic.

\textbf{Somatic (S)}

The gesture of casting a spell can include forced gesturing or intricate series of gestures. If a spell requires a somatic component, the caster must be free to use at least one hand to perform these gestures.

\medskip

\textbf{Material (M)}\index{Spell Components}

Casting certain spells requires particular objects, specified in parentheses under the components heading. The character must obtain that specific component before he can cast the spell.

If a spell indicates that the material component is consumed by the spell, the caster must supply this component with each casting of the spell.
A spellcaster must have a free hand to access these components, but it can be the same hand used to cast the somatic components.

\begin{changemargin}{0.3cm}{0.3cm}\begin{narratore}
For more immediate management of the components you can replace the components indicated by consuming an equivalent in gold coins equal to 100 * the square of the spell level (€5) in gem powder.
\end{narratore}\end{changemargin}

\subsubsection{Recover from dying}\index{Recover from dying}\label{magieessereucciso}\index{Spells, Incapacitated}

If you drop to zero or below zero Hit Points you lose half your remaining Magic Points, with a minimum of 10 Magic Points lost. All spells you are concentrating on are interrupted.

\subsubsection{Targets}\index{Targets}\label{magiebersagli}\index{Spells, Targets}

A normal spell requires you to choose one or more targets that are affected by its magic. The spell description tells you whether the spell targets creatures, objects, or a point of origin to generate an area of ​​effect. Unless the spell has a perceptible effect, a creature may never realize that it has been the target of a spell. An effect such as crackling lightning is overt, but a more subtle effect, such as attempting to read a creature's thoughts, is usually unnoticed unless the spell says otherwise.

Casting a spell is an action that does not go unnoticed. A Stealth check at difficulty 15 or casting the spell as if you were distracted allows you to conceal the casting if it doesn't happen right in front of the observer.

\subsubsection*{Clear Trajectory Towards Target}\index{Spells, see target}

\textbf{To target a creature or object}, you must see it and have a clear path towards it, and therefore it \textbf{cannot be behind complete cover}. If you place an area of ​​effect somewhere you can't see and an obstruction, such as a wall, is between you and that point, the origin point is created on your closest side of the obstruction (a Ball of Fire behind a closed door explodes upon contact with the door on your side and does not manifest beyond the door).\index{Magic see the target}

\subsubsection*{Take Yourself as Target}\index{Self as Target}\index{Spells, Self as Target}

If a spell targets a creature of your choice or an ally, you can also choose yourself, unless the creature must be hostile or it is specified that it cannot be you. If you are within the area of ​​effect of a spell cast by you, you will also be affected.

%\begin{center}
% \includegraphics[width=0.6\linewidth]{immagini/tarothanged.png}
%
% \emph{Tarot - The Hanged Man}
%\end{center}

\subsubsection{Areas of Effect}\index{Area of ​​Effect Spells}\label{magieareedieffetto}\index{Spells, Area of ​​Effect}

Spells like Searing Wave and Cone of Cold cover an area, allowing him to target multiple creatures at once.

A spell's description specifies its area of ​​effect, which usually falls into one of five shapes: cylinder, cone, cube, line, or sphere. Each area of ​​effect has a point of origin, a place from which the spell's energy manifests. The rules for each shape specify how to place its origin point. Usually the point of origin is a point in space, but some spells have an area whose origin is a creature or object. The origin point must always be valid.

\begin{center}
\includegraphics[width=0.55\linewidth]{immagini/3dforme.png}

\emph{Cone, Sphere, Cylinder, Cube}
\end{center}

\begin{itemize}
\item \emph{\textbf{Cylinder}}: The point of origin of a cylinder is the center of a circle of specific radius, as indicated in the spell description. The circle must be on the floor or level with the spell's effect. The energy in a cylinder expands in straight lines from the point of origin to the perimeter of the circle, forming the base of the cylinder. The spell's effect then starts from bottom to top or from top to bottom, up to a distance equal to the height of the cylinder. The cylinder's point of origin is included in its area of ​​effect.

\item

\emph{\textbf{Cone}}: A cone extends in a direction of your choice from its point of origin. The diameter of a cone at a given point along its length is equal to the distance of that point from the point of origin. The area of ​​effect of a cone specifies its maximum length. The cone's point of origin is not included in its area of ​​effect, unless you choose otherwise. 

E.g. A 9 meter Cold Cone is 9 meters wide at the end and stretches 9 meters from the point of origin, 3 meters away from the point of origin it is 3 meters wide.

\item
\emph{\textbf{Cube}}: select the origin point of a corner of the cube. The dimensions of the cube are expressed as the length of each of its edges. The cube's point of origin is not included in its area of ​​effect, unless you decide otherwise.

\item
\emph{\textbf{Line}}: A line extends from its point of origin in a straight path along its entire length and covers an area defined by its width. The line's point of origin is not included in its area of ​​effect, unless you choose otherwise. Unless otherwise specified, a line is one square wide.

\item
\emph{\textbf{Sphere}}: select the point of origin of a sphere, which must be valid (see Range and Targets) and the sphere will extend from that point until it encounters an insurmountable obstacle or its expressed size in the radius. The size of the sphere is indicated as the radius in meters that extends from that point. The sphere's point of origin is included in its area of ​​effect.

A fireball that is spawned in a 9x9m room will take up a good portion of it, and in a 6x6m room it will fill it all. In a 3x3 m room, if it has the opportunity to exit through a door or window, it will continue its explosion until it reaches a radius of 6 metres. A fireball in a 10x10 foot corridor will saturate it 20 feet back and forth from the point of origin.

\end{itemize}

\subsubsection{Spell Rarity}\index{Spell Rarity}\label{magieraritaincantesimi}\index{Spells, Spell Rarity}

On some Spells the Rarity is indicated, that is, how likely it is to find this spell or how much it can be known. The rarity depends not only on the level of the spell itself, obviously the most powerful spells are also the rarest, but also on how commonly they are widespread and known in the list. The Storyteller will use this scale on 3d6 to evaluate what can be found most easily: Common (1-14) - Uncommon (15) - Rare (16) - Very Rare (17) - Legendary (18).
%Common (70\%) - Uncommon (23\%) - Rare (4\%) - Very Rare (2\%) - Legendary (1\%), (1-70,71-93,94-97 ,98-99,100)

\subsubsection{Combining Magical Effects}\index{Combining Magical Effects}\label{magiecombinareeffettimagici}\index{Spells, Combining Effects}

The effects of different spells stack until their durations overlap. However, effects from the same spell or that give the same bonus cast multiple times on the same target do not combine. Instead, the most powerful spell among those cast, the one of the highest level and, all things being equal, the one that has obtained the most Magic Criticals will be applied as long as the durations overlap.

In the case of instantaneous spells, the effects act individually if they act in the same initiative segment. E.g. If I am hit by a bolt of lightning with initiative segment 4 and then by another bolt of lightning with initiative segment 8 I will make two separate saving throws with related damage management, if they were in the same initiative segment I would only suffer the more powerful one (see above ).

\subsection{Basic Rules}\index{Basic Magic Rules}\label{magieregoledibase}\index{Spells, Basic Rules}

\begin{itemize}

\item
When casting his first spell, the spellcaster chooses whether to use Intelligence as a modifier to the Magical Competence check, the Characteristic linked to the first Magic List taken or, if he is a Devotee, he can choose the Characteristic indicated by the Patron.\index{Spells , CM proof modifier}

Once the choice has been made it is no longer possible to change it. This modifier is called \textbf{spell ability modifier}.\index{Spell ability modifier}
\item
When the character assigns the first point of Magical Expertise, he knows a Magic List.
\item
Each time the character learns a new Magic List, either through the Adept of Magic Feat or by Magical Expertise score, he learns 2 first-level spells + ability modifier + 2 cantrips from that Magic List or Universal.
\item
Every time a spellcaster gains a point in Magical Expertise, he learns two new spells that he has available in his Tome and that are within the maximum castable level.
\item
Each time you gain a point in Magical Expertise you may forgo learning a spell of level 1 or higher to learn two cantrips (level 0) that you know.
\item
Every time the caster acquires a point in Magical Expertise, it is possible to forget a number of spells equal to the CM score and replace them with others available in the Tome, as long as they are within the maximum castable level.
\item
The number of spells that can be cast per day depends on the caster's ability. See \textbf{Magic Points and Magical Expertise Table}.
\item
A Follower gains +1d6 on Magic Checks in the schools favored by the Patron. She can use the Patron's favored energy in your spells.
\item
A Devotee adds +1d6 to Magic Tests in the Patron's favored schools and may ignore a die rolled on the Magic Test. He must use the Patron's preferred energy in his spells.
\item
The term \textbf{learned}\index{Spells, Learned} means a spell present in the Tome of Magic that has been memorized and can be cast whenever desired.

The term \textbf{known}\index{Spells, Known} refers to a spell present in the Tome of Magic which however has not been learned, i.e. it has not been memorized and cannot be cast when desired.
\end{itemize}

\subsubsection{Access to Magic Lists}\index{Access to Magic Lists}

The enchanter can cast spells only if they belong to a known Magic List or the Universal List and if they are learned and therefore memorized among those present in the Tome of Magic.

A spellcaster learns a new Magic List when she assigns her first point of Magical Expertise and every 4 total points of Magical Expertise thereafter.

Further access to the Magic Lists occurs through the choice of the Adept of Magic Feat.

\subsubsection{Maximum castable spell level}\hypertarget{schools and levels}{}\index{Spell Level by Skill}\label{magieaccessoallelistedimagia}\index{Spells, Maximum castable spell level}\index{Maximum spell level throwable}

While Magical Competence indicates the study and dedication to Magic in the most abstract form, it is only the Adept of Magic Feat that allows you to understand how good you are at formulating spells. 

To establish the specific maximum castable level for each Magic List it is necessary to know the Magical Expertise value and how many times the Magic Adept Feat has been taken in that Magic List.

\textbf{-} If Adept of Magic has been taken 1 time the maximum level of spell that can be cast is CM/2. Ex. CM=6, (6/2)=3 spell level.

\textbf{-} If Adept of Magic has been taken more than once, add to CM the times you have taken Adept of Magic -1 and divide the result by 2. E.g. CM=9, Adept of Magic taken 4 times, (9+3)/2=6, or CM=13, Adept of Magic 3 times, (13+2)/2=7 spell level.

If Adept of Magic was not taken in that list then the maximum level of spells that can be cast is 1.


\subsubsection{Optional - Ultimate Magic}\index{It's over 9000!}\index{Optional - Ultimate Magic}
\label{opzionalemagiasuprema}\hypertarget{optionalsupreme magic}{}

- \textbf{Supreme Magic}\index{Supreme Magic}\hypertarget{Supreme Magic}{}: if you want high-level spellcasters to dominate magic, make sure that for every 6 points of Magical Expertise the player can add 1d6 in the Magic Test and ignore a rolled die.

\noindent- \textbf{Magic for all}\index{Magic for all}\label{magiapertutti}: instead of every 4 points awarded in Magical Expertise the spellcaster learns a Magic List every 2.

\subsubsection{Magic Test}\index{Magic Test}\index{Magic Critical Success}\index{Magic Critical Failure}\label{magieprovadimagia}\index{Spells, Magic Test}

Casting a spell is not always enough, many times it is necessary for it to work well and indeed to act beyond its normal expectations. The caster can decide to call upon more energy when casting the spell, i.e. make a \emph{\textbf{Magic Check}} and trust in his abilities.

The caster rolls 3d6+1d6 for every two Magic Lists (rounded up) known plus any bonuses or Feats.

The spellcaster can ignore a 1 rolled on the Magic Test for each time he has taken Adept of Magic in the Magic List of the spell he is casting.

If in the set of rolled dice there are \textbf{at least two 1s} or \textbf{one 1 and two 2s} bad things will have happened, this case is called \textbf{Magic Critical Failure}\index{Failure Magic Critical}, the spell does not manifest and \textbf{Magic Points are spent}.

To check how many magical critical failures have been made, first check how many pairs of 1s are present, then check if there is another 1 left to associate with a 1 or two 2s.


\begin{changemargin}{0.3cm}{0.3cm}\begin{narratore}
Grant a +1d6 on the Magic Test when the character expertly and fluently recites the casting of the spell. If he says €5711 € {I throw a fireball} he will get no advantages but if with transportation he declaims €5712 € {By all the hellfires may Nedraf drag you to hell with his holy flames. Burn unworthy. Fireball!} then yes!.
\end{narratore}\end{changemargin}

Once you have checked the absence of critical failure, if there are at least two 6s in the dice roll you will have obtained \textbf{Magic Critical Success}\index{Magic Critical Success}, as for the Golden Rules you will continue to roll a die for every 6 done or what you are going to do. Count the 6 you get, every two is a Magical Critical Success! Any 1's rolled following the critical success do not count towards the critical failure. \textbf{For each critical success the spell's saving throw DC increases by 1}.

Any result other than a magical critical success or a magical critical failure will cause the spell to manifest without any particular effect unless explicitly requested.

When required to pass or make a Magic Test it is sufficient not to make a Magical Critical Failure.

By paying twice the cost of the spell you can roll 2d6 more, by paying quadruple you can roll 4 more (and then x8 and +6 dice...).\index{Magic Check, more dice}

A spellcaster can also willfully fail a Magic Test.


%\medskip
%
%\begin{center}
% \includegraphics[width=0.65\linewidth]{immagini/Arthur-Pyle_The_Enchanter_Merlin.png}
%
% \emph{Merlin. Howard Pyle, The Story of King Arthur and His Knights (1903)}
%\end{center}


\subsubsection{Magic Test Critical Failure}\index{Magic Test Critical Failure}\label{magiefallimentocriticonellaprovadimagia}\index{Spells, Magic Test Failure}

If the Magic Test had a Magical Critical Failure, roll 3d6 and consult the following table. For each additional Magic Critical Failure to the first, roll 1d6 less, until you only roll 1d6.

%\end{multicols}

\medskip

\textbf{Table: Magic Critical Failure Effects}\index[Tables]{Magic Test Critical Failure Effects Table}

\medskip

\begin{flushleft}
\begin{tabularx}{0.48\textwidth}{lX}
\hline
1 & Increase your Fatigued status by 2 ranks\\
2 & For 1 day you are no longer able to channel magical energies. You cannot cast spells unless you make a critical magical success on the Magic Test\\
3 & You exhibit minor body modification\\
4 & You are hit by a thundering column of Light and Void. In a 10-foot radius centered on you, anyone must make a DC 15 Reflex saving throw to halve or take 1d6 points of damage per spell level\\
5 & ​​For 3 rounds you are under the influence of the Confusion spell\\
6 & You are paralyzed for 3 rounds\\
7 & You are teleported within 3d10 meters in a random direction\\
8 & You become invisible and unable to speak for 6 rounds\\
9 & Only you are shrouded in impenetrable magical darkness for 6 rounds\\
10 & You can't speak well, you stammer. Each spellcasting forces you to pass a Magic Test. Duration 3 rounds\\
11 & The next spell you cast has effects that are minimized if possible\\
12 & Your heartbeat is like the beating of a drum, it can be heard within 50 meters\\
13 & All your body hair falls out, luckily it can grow back\\
14 & Emit a noisy, pestilential flatulence. A 1m x 50cm illuminated sign above your head points at you and mocks you\\
15 & Every object you hold in your hand falls to the ground\\
16 & Gain 2d6 Magic Points\\
17 & An anvil falls, 3d6 damage Reflex save DC 15 to halve, on one random creature, excluding you, within twenty feet\\
18 & All creatures, except you, within a 20-foot radius centered on you take 1d10 unresistible damage
\end{tabularx}

\end{flushleft}
%\begin{multicols}{2}

\subsubsection{Magic Points}\index{Magic Points}\label{magiepuntimagia}\index{Spells, Magic Points}

Depending on the score in Magical Expertise the caster has a certain amount of Magic Points available.

\textbf{Spells cost in Magic Points equal to their level}\index{Magic Points and spell cost}

%\textbf{Spells cost in Magic Points equal to the spell level +1.}\index{Spells, Cost in Magic Points}

Every time you cast a spell, the cost is subtracted from the Magic Points available for the day.
In the case of Cantrips, these do not consume Magic Points but it is necessary to have at least 1 Magic Point remaining.

The caster has a \textbf{bonus} to Magic Points equal to his spell ability modifier.

Magic Points are all recovered with 8 hours of rest. \index{Spells, Recovery of Magic Bridges}

\medskip

\textbf{Table: Magical Expertise (MC) and Magic Points (MP)}\index[Tables]{Table Magical Expertise (CM) and Magic Points (MP)}

\medskip

\begin{tabularx}{0.45\textwidth}{XX|XX|XX}
\textbf{CM} & \textbf{PM}&\textbf{CM} & \textbf{PM}&\textbf{CM} & \textbf{PM}\\
\hline
1& 2 &8& 27 &15&58\\
2& 4 &9& 36 &16&62\\
3& 8 &10&41 &17&71\\
4& 10 &11&43 &18&76\\
5& 16 &12&47 &19&82\\
6& 19 &13&50 &20&89\\
7& 23 &14&54 &20+&prev.+ 4\\
\end{tabularx}

%\begin{tabularx}{0.45\textwidth}{XX|XX}
%\textbf{Inc. Level} & \textbf{Magic Points}&\textbf{Inc. Level} & \textbf{Magic Points}\\

%1& 2 & 6& 9\\
%2& 3 & 7& 10\\
%3& 5 & 8& 11\\
%4& 6 & 9& 13\\
%5& 7&&\\
%\end{tabularx}


%\begin{tabularx}{0.45\textwidth}{XX|XX}
%\textbf{Comp. Magic} & \textbf{Magic Points}&\textbf{Comp. Magic} & \textbf{Magic Points}\\
%1 & 4 &2 &6\\
%3 & 14 &4 &17\\
%5 & 27 &6 &32\\
%7 & 38 &8 &44\\
%9 & 57 &10 &64\\
%11& 73 &12 &73\\
%13 &83 &14 &83\\
%15 &94 &16 &94\\
%17 &107 &18 &114\\
%19 &123 &20 &133\\
%\end{tabularx}


\begin{changemargin}{0.3cm}{0.3cm}\begin{narratore}
If you want a more realistic approach, for each degree of Fatigue you recover 20\% less than the maximum Magic Points per night's rest.
\end{narratore}\end{changemargin}

\subsubsection{When you have few Magic Points}\index{When you have few Magic Points}\hypertarget{when you have few magic points}{}\label{magiequandosihannopochipuntimagia}\index{Spells, Few Magic Points}

When the caster falls below the \textbf{ 50\% of the Magic Points} available, any further spell casting must be done by passing a Magic Test.

\subsubsection*{Optional - Spells as Rituals}\index{Optional - Spells as Rituals}\index{Rituals, Spells}

Especially at the first levels, it can be very annoying not to have learned a spell even though you have it available in the Tome of Magic.

With this optional rule the spellcaster can cast a spell, within 3rd level, that is present in his Tome of Magic or even that he has learned, extending its casting time to 1 hour per Magic Point cost. If a spell is cast in this way, no Magic Points are used, but a Magic Test must be passed.

\subsubsection*{Optional - The Vice of Magic}\index{Optional - The Vice of Magic}

If you want an approach that reduces the amount of spells cast by spellcasters, establish that the cost in Magic Points of each spell is equal to the spell's Level + the cost itself x the times it has already been cast that day, but avoid the Test of Magic for \emph{When you have few Magic Points}.

\subsubsection{Automagic Critical Success}\index{Automagic Critical Success}\index{Nova}\label{magienova}\index{Spells, Automatic Critical Success}

The caster can decide to additionally spend the \textbf{double the normal Magic Points} of the spell to automatically have a \textbf{Magic Critical Success}.
The choice can be made several times and each time the cost of the spell doubles compared to the previous one. The declaration of wanting to use the Automagic Critical Success must be declared before carrying out, and passing, the Magic Test.

The casting time of a spell enhanced in this manner increases by 1 Action.

E.g. Fireball, I want it to score 2 magic critical successes, I pay 3 Magic Points to cast it, plus 6 for the first Magic Critical Success plus 12 for the second Magic Critical Success, and possibly 24 for a third Magic Critical Success. In this case, all the Magic Points used are always paid regardless of the result of the Magic Test.

You cannot spend more than half your current Magic Points to enhance a spell.

\subsubsection{The Tome of Magic}\index{Tome of Magic}\index{The Tome of Magic}\label{magietomodellamagia}\index{Spells, Tome of Magic}

If the Patrons are the source of magic, it is only the application of ancient rites and formulas that allows this raw energy to be manifested in a form and expression that we call spell.

Every user of magic has one or more \textbf{Tome} of spells, don't just think of a large ancient tome bound in leather, different cultures have developed over time the ability to inscribe the runes of spells on cards, sticks, plates stone, tattoos... take your pick when creating your character.
This choice will not prevent you from copying spells worth \textbf{Tomes} made differently (tobacco leaves, liquids of knowledge...) for you it will always be easy (Arcana DC 12 test) to understand if you are dealing with a Tome of some kind.

When the character learns a Magic List for the first time, he writes into the Tome of Magic a number of spells equal to the +2 spell ability modifier and 2 cantrips from the same list. These spells will be first level or from the Universal List. Any other spells he wants to learn he will have to find and write them down in his Tome.

Each spell occupies a number of pages in the Tome equal to its level, with a minimum of one; copying a spell page takes 1 hour of work and 10 gp of precious ink.\index{Copying Spells on the Tome}

A Tome (book) of spells costs 10 gp per page.

\medskip
\begin{center}
\includegraphics[width=0.7\linewidth]{immagini/spellbook.png}
\end{center}

\medskip

A spellcaster can copy into his Tome spells that belong to a Magic List known to him (or Universal) and the maximum copyable level is one level higher than his maximum castable level (see \hyperlink{schools and levels}{Magic Lists}) .

If the spell is more than two levels higher or from an unknown Magic List the caster must make a Magic Check and gain a Magical Critical Success. If the character is a Devotee and the spell belongs to a Magic List known to him and preferred by the Patron, then the Magic Test is performed only if the spell is three or more levels higher than the maximum castable level.

If he does not achieve at least one Magical Critical Success he cannot attempt to copy that spell until the next point of Magical Proficiency gained. If he rolls a Magical Critical Failure, bad things will happen to the Tome and 1d4 random spells will be erased from the Tome itself.

The source of new spells can be another tome, staff, scroll... in short, anything that the previous spellcaster used to store spells. A magical object (magic staff, ring, rod... wand...) is not suitable as a source from which to copy the spell it contains, it must be copied from the equivalent tome or scroll of another spellcaster. A spell when copied to the new Tome vanishes from the original Tome.

During the adventures your enchanter will be able to copy many and numerous spells on his Tome but he will not be able to learn them immediately. When the character acquires a new point of Magical Expertise, he will be able to forget a spell learned to replace it with a spell present in his Tome that is from a known Magic List and learn the new spells.

\begin{changemargin}{0.3cm}{0.3cm}\begin{tcolorbox}[title = Choose Spells]
Spells are not learned alone, they are not chosen from a ready-made list. Each spell is a precious treasure that must be found and learned.

You will have to undertake perilous adventures, pay mercenaries, search for ancient tomes and reveal the darkest and most forgotten secrets in order to learn new spells.

Each spell is like a magical object, a true treasure to seek and obtain!
\end{tcolorbox}\end{changemargin}


\begin{changemargin}{0.3cm}{0.3cm}\begin{narratore}
Spells become full-fledged magical objects and rewards. Harness your characters' thirst for knowledge and power to build interesting adventures that revolve around ancient tomes and legendary lost spells.
\end{narratore}\end{changemargin}

\subsubsection{Studying spells}\index{Studying spells}\label{magiestudiareincantesimi}\index{Spells, Studying}

The character who wants to cast spells must review the ancient formulas in his Tome every day. This is quite quick, taking only 3 minutes per Magical Proficiency.

If the caster has not reviewed the spells as soon as he wakes up or before casting them, he must make a Magic Test for each spell until he has reviewed them.

\subsubsection{Changing Spells}\index{Changing Spells}\label{Cambiare gli Incantesimi}

Through a long and difficult magical rite the caster can replace a learned spell with a known spell present in the Tome. After 8 hours of ritual the caster performs a Magic Test and only if he succeeds can he change up to 1d4 spells, if the Magic Test achieves a Critical Success then he can change up to 1d4+4 spells. If the check fails critically, the caster forgets 1d4 spells.

\subsubsection{Attack Roll with Spells}\index{Attack Roll with Spells}\label{magietiropercolpireconlemagie}\index{Spells, Attack Roll}

Several spells must be cast and hit an opponent to work.

When the spell tells you to make a \emph{Spell attack roll} (ranged or melee) you must make an attack roll against your opponent's Defense.

This attack roll is made with 3d6+ \textbf{Weapon Proficiency} + €5796{Spell ability modifier} + \textbf{Feats} and €5798{miscellaneous modifiers}.

It is also possible that a \textbf{Touch Spell Attack Roll} is required, i.e. the attack is made with a bonus of +1d6, as for Touch Attack.\index{Spell Touch Attack}

Spell or weapon attack rolls accumulate multiple attack penalties.\index{Multiple spell attack penalty}

\medskip

When the magic is area-effect it is not necessary to make an attack roll except to reach difficult and specified areas, i.e. you aim at a well-defined area and you want to avoid hitting someone with an area-effect spell.

\subsubsection{Optional - Spell Attack Roll}\index{Optional - Spell Attack Roll}

If you want to make it easier for spellcasters to hit, you can decide that the attack roll is based not on weapon proficiency but on magical proficiency.

\subsubsection{The Explosion of 6 in Magic}\index{Explosion of 6 in Magic}\label{magieesplosionedelsei}\index{Magic Critical}

Even in the Magic Test the 6s explode, the 6s rolled in the Magic Test are rerolled, and rerolled again if necessary.

Keep track of how many criticals (two 6s rolled) you roll, it could help you get \emph{special effects} in the spell! Remember that for each magical critical the DC of the saving throw increases by 1.

\subsubsection{Saving Throw - Resist the spell}\index{Saving Throw - Resist the spell}\index{Saving Throw Spells}\label{magietirosalvezza}\hypertarget{spell saving throw}{}

The T\textbf{iro Salvation} based on what is required by the spell has difficulty (DC) equal to \textbf{10} + \textbf{Magical Expertise} + \textbf{characteristic modifier for spell} + € €5817{2 x Feats taken in that Magic List} +€5818{1 for each Magical Critical Success} in the Magic Test.

When you cast a spell, for example Thunderbolt, you impose a Reflex saving throw to try to avoid it and if in the Magic Test you had obtained at least one Magical Critical Success you would have done 9d6 damage and the DC of the saving throw would have increased by 1.

In the description of the spell it is written if it is necessary and which saving throw must be performed.

\begin{changemargin}{0.3cm}{0.3cm}\begin{tcolorbox}[title = Tups launches a Thunderbolt!]
Tups who has Intelligence 4, Magical Expertise 6, and has taken Adept of Magic 1 time in the Air list, casts the Lightning spell. The difficulty (DC) of the Reflex saving throw will be equal to 10 + 6 (CM) + 4 (characteristic modifier for spell, Intelligence) + 2 x 1 (Adept of Magic taken 1 time in the Aria List) or 10+6+4 +2x1 = 22 to halve damage. If he had made a Magic Test and it had a magical critical success the DC would have become 23.
\end{tcolorbox}\end{changemargin}

If you are the one who has to resist a spell, the Narrator will not tell you to make a saving throw at difficulty 18, he is the one who compares your roll with the difficulty, he will be able to tell you that the test is complex, difficult or easy...

\begin{itemize}

\item
If you roll 6 3 times on your saving throw you succeed, regardless of the total, and get €5,820 {Saving Critical Success}.

\item
If the saving throw is successful and you roll at least two 6's you get \textbf{Saving Critical Success}€5823{Spell Critical Success}.

\item
If you roll 3 1s on your saving throw you fail, regardless of the total, and get €5825{Save Failure}.€5826{Three 1s on magic saving throws}

\item
If the saving throw fails and you roll at least two 1s or one 1 and two 2s on the dice roll you get \textbf{Savement Critical Failure}\index{Spell Critical Failure}. \index{Two 1s on magical saving throws}

\end{itemize}

It is also possible that in the description of the spell it is reported what happens in case of Success or Critical Failure of the saving throw.

For \textbf{monsters} or in any case for a spell casting given by innate magical abilities, if not specified the \textbf{DC of the saving throw is equal to 10 + 2 x spell level + Intelligence or Wisdom whichever it is best}.\index{DC Monster Spell Saving Throw}\index{Monster Spell Difficulty}

\subsubsection{Optional - Saving Throw Value}\index{Optional - Saving Throw Value}

The saving throw calculation system aims to reward the character who invests in his own knowledge and specialization, while making the calculation more complex for the player.

An alternative is to set the saving throw DC to 10+2*Spell Level + Ability Modifier per spell +1 per Spell Critical.

\subsubsection{Distracted - Problems casting the spell}\index{Distracted - Problems casting the spell}\index{Distracted}\label{magiedistratto}

If the caster is severely \textbf{Distracted}, trying to hide the casting of magic, is impeded, disturbed, is bleeding, threatened, is under attack while trying to cast a spell, other than a cantrip, he must make a \textbf{Magic Test}.

For each critical or magical critical roll suffered in the round the Magic Test is made with an additional 1d6.\index{Critical damage if spell is cast}

\subsubsection{Concentration}\index{Hit while concentrating}\index{Concentration}\label{magieconcentrazione}

You lose concentration on a spell if you cast another spell that requires concentration. You can't concentrate on two spells at once. Breaking concentration costs a Reaction.

If you are hit\index{Hit while concentrating} while concentrating on a spell you must make a Magic Check and gain at least 1 Magical Critical Success or lose concentration.

Also in this case you can pay the additional Magic Point cost to ignore 1 or 2 (\hyperlink{effective spells}{Effective Spells} page \pageref{effective spells}).

While you are concentrated you can only cast Cantrips, maintaining Concentration costs 1 Action per round.

\subsubsection{Optional - Multiple Concentrations}\index{Optional - Multiple Concentrations}

Every 6 points of CM you can maintain concentration on an additional spell, without being limited to cantrips alone. If concentration is interrupted, all spells held in concentration are lost.
For each spell you maintain concentration on, you pay 1 Action.

\subsubsection{Keeping the magic}\index{Keeping the magic}\label{magieconservare}

The caster can cast the spell (usually 2 Actions) and hold it in his fist, without manifesting it. To do so he must cast the spell, then he can hold it for up to 1 round per point of ability modifier from spells +2 rounds per number of times he has taken Adept of Magic in the Magic List belonging to the spell.

To hold the spell, the caster must remain Concentrated (cost 1 Action per round) and pay 1 Magic Point per round.
To cast the stored spell, simply roll initiative and use 1 Action. You cannot cast additional spells other than cantrips as long as you retain a spell.

\subsubsection{Influenced by multiple spells}\index{Influenced by multiple spells}\label{magieinfluenzatodapiumagie}

When a character is affected by \textbf{two or more magical effects} that give the same type of bonus, penalty or damage in the same initiative segment (protection against fire, bonus to Defense or ST..., multiple acid balls ), only the one with the highest saving throw or bonus is taken into account

A character who takes 2 Fireballs in the same Initiative segment will only make the saving throw for the most powerful one, regardless of whether it is the one with the greatest damage. If he takes a Fireball at two different times in the same round he will make two separate saving throws taking relative damage.

\subsubsection{Rules for summoned creatures}\index{Rules for summoned creatures}\index{Summoning creatures}

These rules apply to all magically summoned creatures.

The summoned creature acts in your round, it does not have to roll initiative, but rather uses yours and is friendly towards you and your companions.

A summoned creature has 2 actions per round, if no orders are specified at the time of summoning the creature defends itself and counterattacks whoever attacked it.

The summoned creature understands the commands given to it to the best of its mental ability. To change the order you must use an Action.

\subsubsection{Attempting multiple spells in the same round}\index{Attempting multiple spells in the same round}\index{Multiple spells in the same round}\hypertarget{feathersround}{}\label{piumagieround}

It is not possible to cast multiple spells per round even if the sum of Actions allows it. Some dark rites and esoteric practices allow you to try to cast even more spells at great risk, as long as they always involve 3 Actions per round. You must have at least 3 Magical Proficiency.

The caster casting the first spell normally must make a Magic Test. If he succeeds at a Magical Critical Success then he is able to cast the second spell, if the Magic Test does not obtain a Magical Critical Success then it is considered as a Magical Critical Failure, with the appropriate effects.

\subsubsection{Altering Spells}\index{Altering Spells}\label{magiealteraremagie}

The caster can modify spells in several ways. These possibilities add versatility to the caster and it is advisable for the player to always have them present in the most critical situations.

- \textbf{Effective magic}\index{Effective magic}\label{magieefficaci}\hypertarget{effective magic}{}: by paying the cost of the spell once more in Magic Points you can ignore a die rolled in the Magic Test, by paying double you can ignore two, by paying four times you can ignore 3. Effective Spells can also be used by a companion of the caster by sacrificing the indicated cost in Magic Points and using the same number of Actions. Immediate Action to be declared after the Magic Test.

- \textbf{Ethereal magic}\index{Ethereal magic}: by increasing the Magic Points spent in the spell by 3 your spells have full effect on ethereal or incorporeal creatures. Immediate Action.

- \textbf{Magic Sacrifice}\index{Magic Sacrifice}: the spellcaster by reducing his Maximum Hit Points by 4 acquires 1 Magic Point to be used when casting a spell. You can't sacrifice more than half your current Hit Points at a time. Immediate Action.

- \textbf{Merciful Magic}\index{Merciful Magic}: by increasing the Magic Points spent by 3, spells inflict temporary damage.
Spells that deal damage of a particular type (such as fire) deal temporary damage of the same type. Immediate Action.

- \textbf{Targeted Magic}\index{Targeted Magic}: for each time you have taken Adept of Magic in that list beyond the first you can make a creature of your choice immune to the effect of the spell that you threw. Cost 1 Magic Point per excluded creature. 1 Action.

- \textbf{Far Magic}\index{Far Magic}: by increasing the Magic Points used by 1 you increase the casting distance of the spell by up to 9 meters per time you have taken Adept of Magic in that list. 1 Action.

- \textbf{Increase time}\index{Increase casting time} from 2 Actions to 3 Actions decreases by 1 in Magic Points spent for spell casting, with a minimum cost of 1 Magic Point.

- \textbf{Collaborative Spells}\index{Collaborative Spells}: only one other spellcaster, sacrificing half the Magic Points used by the companion casting the spell, using the same number of Actions, can grant +1d6 to the Test of Companion magic. Collaborative Magic can be combined with Effective Magic. Magical Proficiency Requirement 3. Action Reaction.

- \textbf{Circle of Power}\index{Circle of Power}: multiple spellcasters who are all Devotees or Followers of the same Patron can collaborate so that one of them is better at casting a spell.
Each spellcaster by sacrificing half the Magic Points of the spell cast by the companion can grant +1d6 to the companion's Magic Test, up to a maximum of +7d6. The casting time of a spell via Circle of Power becomes at least 1 Turn.
One or more companions can alternately use Effective Magic. Magical Proficiency Requirement 5.


The possibilities granted by Altering Magic can be combined with each other.

\textbf{Slight changes} \index{Slight changes to spells} at the manifestation of the spell can be agreed with the Storyteller for a cost of additional Magic Points or with a successful Magic Test.

\subsubsection{Attempting Spells with Impediments}\index{Attempting Spells with Impediments} \index{Impedations}\label{magieconimpedimenti}

The casting of a spell is linked to particular and unique gestures and words. When the character finds himself in a situation where he cannot gesture or speak then he can attempt to cast the spell anyway even if it becomes much more difficult.

The Magic Points required for casting spells are tripled if she cannot gesture and are further tripled if she cannot speak, it is also necessary in any case to pass a Magic Test.

If the spell also has material components, these must still be provided (placed within 30 cm of the caster) or it is not possible to cast the spell.

\subsubsection{Spell Objective Definitions}\index{Spell Objectives}\label{magiedefinizioniobiettivi}

In the spells listed below you will often find references to the types of subjects and targets that can be influenced as well as to different types of energy and elements.

- The \textbf{Creatures} \textbf{Natural} are Insects, Reptiles, Beasts, Humanoids, Plants, Aquatic Creatures, Monstrosities, Slimes.

- The \textbf{Creatures} \textbf{Magical} are: Fiends (Devils and Demons), Fairies, Spirits, Undead, Giants, Celestials, Elementals, Constructs, Aberrations (everything that is alien or unnatural) and the Dragons.

If a Natural Creature has magical powers then it is also considered a Magical Creature. A more complete description of these categories can be found in the Monstrorium Chapter.

- \textbf{Energy} includes: Force, Fire, Light, Sound, Electricity, Positive Energy, Negative Energy, Cold, Vacuum.

\subsubsection{Energy, Light and Void Damage}

The damage caused by \textbf{Light}\index{Light} is half fire and half positive energy, i.e. resistance to fire or positive energy only applies to half of the damage caused by the attack.

The damage caused by \textbf{Empty}\index{Empty} is half from cold and half from negative energy, any protections apply to the respective halves of the damage.

\textbf{negative energy} alone damages \index{Negative Energy} the living and cures the undead, \textbf{positive energy} alone\index{Positive Energy} damages the undead but does not cure the living (at the Storyteller's discretion, one round's exposure could be equivalent to a Lesser Restoration spell), see also descriptions of the Planes. A target takes full damage from Light or Void if it has no inherent resistances.

A special case is the \textbf{Healing positive energy}\index{Healing positive energy} which heals the living and damages the undead. This energy is that of Lay on Hands, Channel Energy, and Healing spells.\index{Positive energy on undead}

\subsection{List Magic}


\begin{changemargin}{0.3cm}{0.3cm}\begin{enfasi}{
I wanted, and I always wanted, and I wanted very strongly (Vittorio Alfieri, 06/09/1783, Letter to Ranieri de' Calzabigi)
}\end{enfasi}\end{changemargin}


The study of magic and the in-depth knowledge of the Magic Lists leads the enchanter to learn aspects of it that are not always known. The greater the capacity in a Magic List, the more the caster will be able to exploit it better than any general user.

The abilities presented depend on the number of times the caster has taken Adept of Magic in that specific magic list.\\

\emph{\textbf{Abjuration List}}

\textbf{2: Minor Shield.} By using a Reaction you are able to channel the magical energies that pervade you, manifesting protection. Until the end of this round you have +1 to Defense.

\textbf{3: Greater Protection.} By using a Reaction you are able to channel the magical energies that pervade you, manifesting protection. Choose up to 2 creatures within 6 meters, they get +2 Defense or +1 on Saving Throws until the end of the round.

\emph{\textbf{Water List}}

\textbf{2: Deep water.} By using a Reaction you gain resistance 5 to cold and fire until the end of the round.

\textbf{3: Clear waters.} By using a Reaction you can touch a creature and help it free itself from poisons and toxins. A new saving throw is allowed (if possible) to lose the poisoned condition.

\emph{\textbf{Air List}}

\textbf{2: In the clouds.} By using a Reaction you are able to cast the Feather Fall spell on yourself without using magic points.

\textbf{3: Shock.} Your hand manifests a crackle of electricity, the next spell you cast in the round that has an attack roll deals 1d8 more electricity damage. It costs a Reaction

\textbf{Enchantment List}

\textbf{2: Distraction.} When a creature you can observe within 30 feet of you makes a weapon or spell attack, you can use a Reaction to distract it. The creature has -2 to attack rolls.

\textbf{3: Major Distraction.} When a creature you can observe within 30 feet of you makes a weapon or spell attack, you can use a Reaction to distract it. Roll 1d6, if the result is 3-4-5 the creature has -2 on the attack roll, if the result is 6 the target of the attack is random.

\emph{\textbf{List of Animals and Plants}}

\textbf{2: Bark.} Using a Reaction makes your skin harder and more resistant. You have damage reduction of 2 until the end of the round.

\textbf{3: Claws.} Using an Action makes your natural attacks even sharper for that round. Each natural attack caused by Bleed 1, stacks up to Bleed 5.

\emph{\textbf{Care List}}

\textbf{2: Hot hand.} Using a Reaction, the first healing spell you cast in the round on a single subject heals a number of additional Hit Points equal to the level of the spell itself.

\textbf{3: Benevolent Spirit.} Using an Action you channel the residual energy of one of your spells to heal you. In that round, each healing spell cast causes you to recover 1 hit point.

\emph{\textbf{Divination List}}

\textbf{2: Premonition.} Using a Reaction he has a fleeting prediction of future events. Until the end of the round you have a +1 on Reflex saving throws.

\textbf{3: Blind Spot.} Using a Reaction you can touch a creature, until the end of your next round it has a +2 to attack roll.

\emph{\textbf{Summon List}}

\textbf{2: Hollow hand.} With a Reaction, you can make an object of volume L disappear and reappear whenever you want. You cannot hold more than three objects in this way.

\textbf{3: Cautious step.} With a Reaction you make the next Move Action not cause attacks of opportunity.

\emph{\textbf{Fire List}}

\textbf{2: Red throat.} ​​With a Reaction you spit a jet of fire into a square attached to you. The terrain is considered difficult and crossing or standing on it causes 1d6 Fire Hit Points. Lasts until the end of the next round.

\textbf{3: Napalm.} With a Reaction you touch a weapon. The weapon is engulfed in flames, dealing an additional 1d6 Fire damage until the end of your next round.

\emph{\textbf{Illusion List}}

\textbf{2: Prestidigitation.} You can use the Prestidigitation spell with a Reaction.

\textbf{3: Abundance} With a Reaction you can create an inorganic object of volume 1 or less worth 1 gp or less. The item persists until this ability is used again

\emph{\textbf{Invocation List}}

\textbf{2: Hope.} With a Reaction you can light up your hand until the end of your next round. The hand illuminates only your little picture and is a dim light in the next little picture.

\textbf{3: Best Wish.} With a Reaction you touch a creature, bestowing good fortune on it. The creature has a +1 to attack rolls, defense rolls, or saving throws of its choice until the end of your next round.

\emph{\textbf{Necromancy List}}

\textbf{2: Black Blood.} By using a Reaction until the end of your next round you ignore the fatigued condition.

\textbf{3: Dead Blood.} Using an Immediate Action you can touch a creature. This gets +2 on Fortitude saves and -1 on Reflex saves until the end of your next round.

\emph{\textbf{Earth List}}

\textbf{2: Glue.} You are able to cast the Repair spell as a reaction without spending Magic Points.

\textbf{3: Titan.} Using an Action, each time you cast a spell from the Earth List, as long as you are in contact with solid earth you recover a number of Hit Points equal to the level of the spell cast.

\emph{\textbf{Transmutation List}}

\textbf{2: Sharing.} By using a Reaction you touch a creature, the creature gains an extra Reaction.

\textbf{3: Transition.} With a Reaction you alter your presence in the space. Roll 1d6, if you roll 6 until the end of the round you become ethereal.

\emph{\textbf{Universal List}}

\textbf{2: Hearing.} You have a +4 on checks to recognize spells cast.

\textbf{3: Sight.} With a Reaction you can cast the Detect Magic spell without using Magic Points.

\textbf{4: Know.} You can cast the Identify spell with a Reaction, without using Magic Points.

\end{multicols}

%\vfill
%
%\begin{center}
%\includegraphics[width=0.27\linewidth]{immagini/Voynich_Manuscript.png}
%
%\smallskip
%
%\emph{A page from Voynich's manuscript, still undeciphered.}
%\end{center}


\pagebreak


\section{The Spells}

\begin{multicols}{2}

\medskip

\begin{changemargin}{0.3cm}{0.3cm}\begin{tcolorbox}[title = More special effects!]
The spells listed are those of the 5th edition plus some of my proposals and other revisions. If you have any suggestions for the Storyteller to handle unexpected criticism, talk to him! The spirit of collaboration must always be constructive.
\end{tcolorbox}\end{changemargin}

~

%\begin{changemargin}{0.3cm}{0.3cm}\begin{narratore}\index{Optional - Alternative to Magical Critical Success damage}
%An alternative to the effects of Magical Critical Success could be that with spells that cause direct damage or cure, instead of the additional die, half the value of the die rounded up is directly added. So 1d10 becomes 6, 1d8 becomes 5, 1d6 becomes 4, 1d4 becomes 3. Ex. 1d10 + 1d8 damage per critical effect becomes 1d10 + 5.
%\end{narratore}\end{changemargin}


\medskip\textbf{Help}\index[Spells]{Help}\\
\textbf{School}: Healing, Necromancy\\
\textbf{Level}: 2, Uncommon\\
\textbf{Launch Time}: 2 Shares\\
\textbf{Range}: 9 metres\\
\textbf{Components}: V, S, M (a thin strip of white fabric)\\
\textbf{Duration}: 1 hour for Magical Expertise\\
Your spell increases the toughness and resolve of your allies. Choose up to three creatures within range. For the duration, each target's maximum hit points and current hit points increase by 5.\\
\textbf{For each Magical Critical Success obtained} in the Magic Test the target's Hit Points increase by an additional 5 points

\medskip\textbf{Alarm}\index[Spells]{Alarm}\\
\textbf{School}: Abjuration\\
\textbf{Level}: 1, Municipality\\
\textbf{Launch Time}: 1 minute\\
\textbf{Range}: 9 metres\\
\textbf{Components}: V, S, M (a bell and a piece of fine silver thread)\\
\textbf{Duration}: 2 hours for Magical Expertise (maximum 24 hours)\\
Set up an alarm against unwanted intrusions. Choose a door, window, or area within range that is no larger than a 20-foot cube. Until the spell ends, you will be warned by an alarm whenever a creature of Tiny size or larger comes into contact with or enters the protected area. When you cast the spell, you can designate creatures that will not set off the alarm. You also choose whether the alarm is audible or just mental. A mental alarm, if you are within 1.5 kilometers of the protected area, warns you with a noise in your mind. The noise can wake you up if you are sleeping. An audible alarm sounds a bell for 10 seconds, audible within 60 feet.\\
\textbf{For each Magical Critical Success obtained} in the Magic Test the duration increases by 2 hours.

\medskip\textbf{Deadly Hallucination}\index[Spells]{Deadly Hallucination}\\
\textbf{School}: Illusion\\
\textbf{Level}: 4, Uncommon\\
\textbf{Launch Time}: 2 Shares\\
\textbf{Range}: 36 metres\\
\textbf{Components}: V, S\\
\textbf{Duration}: Instant\\
You tap into the nightmares of a creature within range and that you can see, and you create an illusory manifestation of its deepest fears, visible only to that creature. The target must make a Will saving throw.\\
On a failed save, the target is frightened for 1 minute and takes 4d10 damage. \\
\textbf{For each Magical Critical Success obtained} in the Magic Test the damage increases by 2d10

\medskip\textbf{Alter Self}\index[Spells]{Alter Self}\\
\textbf{School}: Transmutation\\
\textbf{Level}: 2, Municipality\\
\textbf{Launch Time}: 2 Shares\\
\textbf{Range}: Personal\\
\textbf{Components}: V, S\\
\textbf{Duration}: 1 minute for Magical Expertise\\
Take on a different form. When you cast this spell, you choose one of the following options, the effect of which lasts for the spell's duration. For the spell's duration, you can end one option to gain the benefits of another.\\
Aquatic Adaptation. You adapt your body to an aquatic environment, developing gills and webbed toes. You can breathe underwater and gain swim speed equal to your movement speed.\\
\emph{Natural Weapons}. Develops claws, fangs, spikes, horns, or a different natural weapon of your choice. Your unarmed strikes deal 1d6 bludgeoning, piercing, or slashing damage, as appropriate for the chosen natural weapon with which you are proficient. Finally, the natural weapon is magical and you receive a +1 bonus on attack and damage rolls made when using it.\\
\emph{Appearance Change}. Transform your appearance. Decide on your outward appearance, including your height, weight, facial features, the sound of your voice, the length of your hair, your complexion, and any quirks you desire. You can appear as a member of another race, although none of your statistics change. Furthermore, you cannot appear as a creature of a different size than yourself, and your basic form remains the same; if you are bipedal, you cannot use this spell to become quadrupedal, for example.\\
At any time during the spell's duration, you can use two Actions to change your appearance in this way again.\\
\textbf{For each Magical Critical Success obtained} in the Magic Test you can alter another subject or double the duration.

\medskip\textbf{Friendship with Animals}\index[Spells]{Friendship with Animals}\\
\textbf{School}: Animals and Plants\\
\textbf{Level}: 1, Uncommon\\
\textbf{Launch Time}: 2 Shares\\
\textbf{Range}: 9 metres\\
\textbf{Components}: V, S, M (some food)\\
\textbf{Duration}: 24 hours\\
This spell allows you to convince a natural beast that you mean no harm to it. Choose a beast within range that you can see. This must see and hear you. If the beast's Intelligence is 4 or more, the spell fails. Otherwise, the beast must succeed on a Will save or be charmed by you for the spell's duration. If you or one of your companions harms the target, the spell ends.\\
\textbf{For each Magical Critical Success obtained} in the Magic Test you can act on an additional beast.

\medskip\textbf{Anathema}\index[Spells]{Anathema}\\
\textbf{School}: Enchantment\\
\textbf{Level}: 1, Municipality\\
\textbf{Launch Time}: 1 minute\\
\textbf{Range}: 9 metres\\
\textbf{Components}: V, S, M (a drop of your blood)\\
\textbf{Duration}: 1 minute\\
Up to three creatures of your choice that you can see, and that are within range, must make a Will saving throw. Any target that fails this save and makes an attack roll or saving throw before the spell ends must roll a d4 and subtract the resulting number from the attack roll or saving throw.\\
\textbf{For each Magical Critical Success obtained} in the Magic Test you can target an additional creature.

\medskip\textbf{Messenger Animal}\index[Spells]{Messenger Animal}\\
\textbf{School}: Animals and Plants\\
\textbf{Level}: 2, Municipality\\
\textbf{Launch Time}: 2 Shares\\
\textbf{Range}: 9 metres\\
\textbf{Components}: V, S, M (a little food)\\
\textbf{Duration}: 24 hours\\
Through this spell, you use an animal to deliver a message. Choose a beast that is Tiny within range and that you can see, such as a squirrel, jay, or bat. You specify a location, which you must have visited in the past, and a recipient who matches a generic description, such as \emph{a man or woman wearing the uniform of the city guard} or \emph{a red-haired dwarf wearing a fedora}. Also speak a message of up to twenty-five words. The target beast travels for the duration of the spell to the specified location, covering approximately 45 miles in 24 hours for a flying messenger, or 25 miles for other animals. When the messenger arrives at its destination, it delivers the message to the creature you described, replicating the sound of your voice. The messenger speaks only to a creature matching the description you provide. If the messenger fails to reach the destination before the spell ends, the message is lost, and the beast returns to where you cast the spell.\\
\textbf{For each Magical Critical Success obtained} in the Magic Test the duration of the spell increases by 8 hours

\medskip\textbf{Animate Dead}\index[Spells]{Animate Dead}\\
\textbf{School}: Necromancy\\
\textbf{Level}: 3, Municipality\\
\textbf{Launch Time}: 1 minute\\
\textbf{Range}: 3 metres\\
\textbf{Components}: V, S, M (a drop of blood, a piece of meat and a pinch of bone dust)\\
\textbf{Duration}: Instant\\
This spell creates an undead minion. Choose a pile of bones or a Medium or Small humanoid corpse within range. Your spell imbues the target with a nefarious semblance of life, reviving them as an undead creature. The target becomes a skeleton if you choose bones or a zombie if you choose a corpse. During each of your rounds, you can use an Action to mentally command any creature you create with this spell that is within 60 feet of you (if you control multiple creatures, you can command all or just some of them at the same time, by sending the same command to all). Decide what action the creature will take and where it will move during its next round, or give it a general command, such as to guard a particular room or hallway. If you don't send any commands, the creature simply defends itself from hostile creatures. Once an order has been received, the creature will continue to carry out it until its completion. The creature is under your control for 24 hours, after which it will stop following commands you give it. To maintain control over the creature for another 24 hours, you must cast this spell on it again before the current 24-hour period ends. This use of the spell reaffirms your control over up to four creatures you have animated with this spell, rather than animating a new one.\\
\textbf{For each Magical Critical Success obtained} in the Magic Test you animate or reassert control over two undead creatures. Each of these creatures must come from a different corpse or pile of bones.

\medskip\textbf{Animate Objects}\index[Spells]{Animate Objects}\\
\textbf{School}: Transmutation\\
\textbf{Level}: 5, Municipality\\
\textbf{Launch Time}: 1 minute\\
\textbf{Range}: 36 metres\\
\textbf{Components}: V, S\\
\textbf{Duration}: Concentration, maximum 1 minute\\
Objects come to life at your command. Choose up to ten nonmagical items within range that are not worn or carried. Medium targets count as two items, Large targets count as four items, Huge targets count as eight items. You cannot animate objects larger than Huge. Each target animates and becomes a creature under your control until the spell ends or until it is reduced to 0 hit points.\\
As an action you can mentally command any creature you have created with this spell that is within 500 feet of you (if you control multiple creatures, you can command only some or all of them at the same time, giving the same command to each). You decide what action the creature will take and where it will move during its next round, or you can issue a generic command, such as to guard a particular room or hallway. If you don't issue commands, the creature will simply defend itself from hostile creatures. Once an order is given, the creature will continue to follow it until it has completed its task.\\
\textbf{For each Magical Critical Success obtained} in the Magic Test the maximum duration doubles.
\bigskip

\end{multicols}

\textbf{Animated Object Statistics}
\bigskip

\begin{tabular}{llllll}
Size&Hit Points&Defense&AC, Damage&Strength&Dexterity\\
\toprule
Lowercase &20 &18&8, {1d4+4} &-3 &4\\
Small &25 &16 &6, {1d8+2} &-2 &2\\
Average &40 &13 &5, {2d6+1} &0 &1\\
Large &50 &10 &6, {2d10+2}&2 &0\\
Huge &80 &10 &8, {2d12+4}&4 &-2\\
\end{tabular}

\bigskip

\begin{multicols}{2}

An animated object is a construct with Defense, Hit Points, attacks, Strength, and Dexterity based on its size. His Intelligence and Wisdom scores are -3, while Charisma is -4. It has movement 9 meters; If the object has no legs or other appendages that it can use to move, it has 0 movement, but it has flying movement 30 feet and can float. If the object is anchored to a surface or a larger object, such as a chain attached to a wall, its speed is 0. It has blindsight to a range of 30 feet and is blind beyond that distance.\\
When the animated object drops to 0 Hit Points, it reverts to its normal object form, and all excess damage is dealt to its original form.\\
If you command an object to attack, it can make a single melee attack against a creature within 3 feet of it. Makes an attack with AC and damage determined by size (see table). The Storyteller may determine that depending on its shape, an object may instead deal slashing or piercing damage.\\
\textbf{For each Magical Critical Success obtained} in the Magic Test you can animate two additional objects.

\medskip\textbf{Anti-Detection}\index[Spells]{Anti-Detection}\\
\textbf{School}: Abjuration\\
\textbf{Level}: 3, Uncommon\\
\textbf{Launch Time}: 2 Shares\\
\textbf{Range}: Contact\\
\textbf{Components}: V, S, M (a pinch of diamond dust worth 25 gp scattered on the target, which the spell consumes)\\
\textbf{Duration}: 8 hours\\
For the duration, you hide the target you have been in contact with from divination magic. The target can be a willing creature or a place or object that occupies a space equivalent to a cube no more than 10 feet in edge. The target cannot become the target of any divination magic or be sensed by magical scrying senses.

\medskip\textbf{Dislike/Like}\index[Spells]{Dislike/Like}\\
\textbf{School}: Enchantment\\
\textbf{Level}: 8, Rare\\
\textbf{Launch Time}: 1 hour\\
\textbf{Range}: 18 metres\\
\textbf{Components}: V, S, M (or a piece of alum dipped in vinegar for the antipathy effect or a drop of honey for the sympathy effect)\\
\textbf{Duration}: 10 days\\
This spell attracts or repels creatures of your choice. Take a target within range, whether a Huge or smaller object or creature, or an area no larger than a 200-foot cube. Then specify a species of intelligent creature, such as red dragons, goblins, or vampires. You invest the target with an aura that attracts or repels specified creatures for the duration. Choose dislike or like as your aura effect.\\
Dislike. The enchantment causes creatures of the type you indicate to feel a strong urge to leave the area and avoid the target. When such a creature can see the target or comes within 60 feet of it, the creature must succeed at a Will save or become frightened. The creature remains frightened as long as it can see the target or remains within 60 feet of it. While frightened by the target, the creature must use its movement to move to the nearest safe place from which it can no longer see the target. If the creature moves more than 60 feet away from the target and can't see it, the creature is no longer frightened, but it becomes frightened again if it can see the target again or moves within 60 feet of it.\\
Sympathy. The enchantment causes specified creatures to feel a strong urge to approach the target if they are within 60 feet of it or can see it. When such a creature can see the target or comes within 60 feet of it, the creature must succeed at a Will save or use its movement each round to enter the area or move within range of the target. Once the creature has done so, it can no longer voluntarily move away from the target. If the target damages or otherwise harms the affected creature, it can make a Will saving throw to end the effect, as described below.\\
Ending the Effect. If an affected creature ends its round while it is farther than 60 feet from the target or cannot see it, the creature makes a Will saving throw. On a successful save, the creature is no longer affected by the target and recognizes the feeling of revulsion or attraction as magical. Additionally, a creature subject to the spell is entitled to another Will save every 24 hours the spell lasts. A creature that saves against this effect is immune to it for 1 minute, after which it can suffer it again.

\medskip\textbf{Energy Weapon}\index[Spells]{Energy Weapon}\index{Lightsaber}\\
\textbf{School}: Air, Water, Earth, Fire\\
\textbf{Level}: 1, Very Rare\\
\textbf{Cast Time}: 1 Action\\
\textbf{Range}: Contact\\
\textbf{Components}: V, S, M (Fairy hair)\\
\textbf{Duration}: 6 rounds, Concentration\\
When you cast the spell in contact with a weapon, it acquires powers depending on the Magic List used and is considered magical, as if it had a bonus of +1.
If Energy Weapon is thrown using the Air List the weapon becomes filled with electricity, in case of Water the weapon becomes extremely cold, in case of Earth acid flows from the weapon, in case of Fire it becomes flaming. Whichever List is used, the effect is such that the weapon causes 1d6 additional damage of the indicated type per successful hit.
A weapon can only have one Energy Weapon effect active at a time.
For each List of Magic possessed you can add an elemental effect and add 1d6 damage of the chosen type. Every round using 1 action it is possible to change the type of damage.\\
\textbf{For every two Magic Critical Success obtained} in the Magic Test the damage increases by +1d6.

\medskip\textbf{Magic Weapon}\index[Spells]{Magic Weapon}\\
\textbf{School}: Transmutation\\
\textbf{Level}: 2, Municipality\\
\textbf{Cast Time}: 1 Immediate Action\\
\textbf{Range}: Contact\\
\textbf{Components}: V, S\\
\textbf{Duration}: 10 minutes\\
You cast the spell on contact with a nonmagical weapon. Until the spell ends, the weapon becomes a magical weapon with a +1 bonus on attack and damage rolls.\\
\textbf{For each Magical Critical Success obtained} in the Magic Test the bonus increases to +1.

\medskip\textbf{Spiritual Weapon}\index[Spells]{Spiritual Weapon}\\
\textbf{School}: Invocation\\
\textbf{Level}: 2, Municipality\\
\textbf{Launch Time}: 2 Shares\\
\textbf{Range}: 18 metres\\
\textbf{Components}: V, S\\
\textbf{Duration}: 3 minutes, Concentration\\
At a point within range, you create a floating spectral weapon, which remains for the duration or until you cast this spell again. When you cast the spell, you can make a melee spell attack against a creature within 3 feet of the weapon with a bonus to hit equal to Magical Expertise/4. If you hit, the target takes force damage equal to 1d4 + your spellcasting ability modifier + Magical Expertise/4. During your round, as an action, you can move the weapon 20 feet and make the attack against a creature within 3 feet of the weapon. The weapon can take any form you want, perhaps similar to the Patron. He is considered to have a magical bonus equal to Magical Expertise/4. \\
The bonuses granted by Magical Expertise/4 can be replaced by the sum of the Traits in common with the Patron/4.\\
\textbf{For each Magical Critical Success obtained} in the Magic Test the damage increases by 2.

\medskip\textbf{Magic Armor}\index[Spells]{Magic Weapon}\\
\textbf{School}: Abjuration\\
\textbf{Level}: 1, Uncommon\\
\textbf{Launch Time}: 2 Shares\\
\textbf{Range}: Contact\\
\textbf{Components}: V, S, M (a piece of worked leather)\\
\textbf{Duration}: 8 hours\\
You cast the spell on contact with a willing creature not wearing armor. A protective magical force surrounds the target until the spell ends. The target's Defense becomes 13 + Dexterity +1/6 Magical Proficiency. The spell ends if the target wears armor or interrupts the spell with an action.\\
\textbf{For each Magical Critical Success obtained} in the Magic Test the Defense increases by 1.

\medskip\textbf{Druidic Artifice}\index[Spells]{Cantrip - Druidic Artifice}\\
\textbf{School}: Universal\\
\textbf{Level}: 0, Uncommon\\
\textbf{Launch Time}: 2 Shares\\
\textbf{Range}: 9 metres\\
\textbf{Components}: V, S\\
\textbf{Duration}: Instant\\
By whispering to the nature spirits, you create one of the following effects within range:

- You create a tiny, harmless sensory effect that predicts what the weather will be like where you are for the next 24 hours. The effect might manifest as a golden sphere for clear skies, a cloud for rain, snowflakes for snow, and so on. The effect lasts for 1 round.\\

- Immediately cause a flower, seed or similar plant to bloom.\\

- Create an instant and harmless sensory effect, such as falling leaves, a puff of wind, the sound of a small animal, or the faint stench of a skunk. The effect must fit into a 1 meter cube.\\

- Instantly light or extinguish a candle, torch or small campfire.\\

This spell can only be cast by Followers or Devotees of Ephrem, Erondil, Gaya, Shayalia.

\medskip\textbf{Magical Aura of the Arcanist}\index[Spells]{Magical Aura of the Arcanist}\\
\textbf{School}: Illusion\\
\textbf{Level}: 2, Uncommon\\
\textbf{Launch Time}: 2 Shares\\
\textbf{Range}: Contact\\
\textbf{Components}: V, S, M (a small square of silk)\\
\textbf{Duration}: 24 hours\\
You place an illusion on a creature or object you are in contact with, so that divination spells reveal false information about it. The target can be a willing creature or an object that is not carried or worn by another creature. When you cast this spell, choose one or both of the following effects. The effect lasts for the duration. If you cast this spell on the same creature or object every day for 30 days, placing the same effect each time, the illusion will remain until it is dispelled.\\
\emph{False Aura}. You change how the target results in spells and magical effects, such as detect magic, that detect magical auras. You can make a normal item appear magical, a magical item appear nonmagical, or change the item's magical aura so that it appears to belong to a Magic List of your choice. When you use this effect on an object, you can cause the false magic to be apparent to any creature that manipulates it.\\
\emph{Mask}. You change how the target results in spells and magical effects that identify the creature's type or Traits, such as symbol spell activation. Choose a creature type or Trait, and other spells and magical effects treat the target as if it were a creature of that type or Trait, rather than the original one.

\medskip\textbf{Holy Aura}\index[Spells]{Holy Aura}\\
\textbf{School}: Abjuration\\
\textbf{Level}: 8, Municipality\\
\textbf{Launch Time}: 2 Shares\\
\textbf{Range}: Personal\\
\textbf{Components}: V, S, M (a tiny reliquary worth at least 1000 gp containing a sacred relic, such as a piece of fabric from a Devotee's robe or a fragment of parchment from a religious text)\\
\textbf{Duration}: Concentration, 1 minute\\
You radiate divine light that gathers in a faint 30-foot radius radiance around you. When you cast the spell, creatures you choose in this radius glow dimly with a 3-foot radius and have {+2d6} on all saving throws, while other creatures have {-2d6} on attack rolls against them. them until the spell ends. Additionally, when a demon or undead hits a target creature with a melee attack, the aura glows with a bright light and the creature must succeed at a Fortitude save or be blinded until the spell ends.

\medskip\textbf{Beneficial Berries}\index[Spells]{Beneficial Berries}\\
\textbf{School}: Animals and Plants\\
\textbf{Level}: 2, Municipality\\
\textbf{Launch Time}: 2 Shares\\
\textbf{Range}: Contact\\
\textbf{Components}: V, S, M (a sprig of mistletoe, up to 8 berries on which the spell works)\\
\textbf{Duration}: Instant\\
You enchant up to 2d4 berries in your hand which are infused with magic for the duration. A creature can use 1 Immediate Action to eat a berry. Eating a berry restores 1 hit point and provides nourishment, but not water, enough to fuel a creature for a day. Only the first berry is effective on the day.\\
The berries lose their effectiveness if they are not consumed within 8 hours of casting the spell. \\
\textbf{For each Magical Critical Success obtained} in the Magic Test the berries last one more day or you enchant one more berry (up to a total maximum of 8).

\medskip\textbf{Solar Flare}\index[Spells]{Solar Flare}\index{Yamato Wave Cannon}\\
\textbf{School}: Invocation\\
\textbf{Level}: 6, Uncommon\\
\textbf{Launch Time}: 2 Shares\\
\textbf{Range}: Personal (18 meter line)\\
\textbf{Components}: V, S, M (a magnifying glass)\\
\textbf{Duration}: Concentration, maximum 1 minute\\
A bright beam of light explodes from your hand in a line 1 meter wide and 18 meters long. Each creature in the line must make a Fortitude saving throw. On a failed save, the creature takes 6d8 Light damage and is blinded until your next round. If the save is successful, it takes half damage and is not blinded. Undead and oozes have -1d6 on this saving throw. You can create a new line of luminosity by spending 2 Actions during any of your rounds until the spell ends.\\
For the duration, a particle of bright light shines in your hand. It produces light in a radius of 9 meters and dim light for a further 9 meters. This light is considered sunlight.\\
\textbf{In case of two Magical Critical Successes obtained} the spell ends after the first ray but the line is 6 meters wide, 108 meters long, the Light damage becomes 12d8.

\medskip\textbf{Heroes' Banquet}\index[Spells]{Heroes' Banquet}\\
\textbf{School}: Summon\\
\textbf{Level}: 6, Uncommon\\
\textbf{Launch Time}: 10 minutes\\
\textbf{Range}: 9 metres\\
\textbf{Components}: V, S, M (a gem-encrusted bowl worth at least 500 gp, which the spell consumes)\\
\textbf{Duration}: Instant\\
Create a magnificent banquet, including delicious food and drinks. The feast is consumed in 1 hour and disappears at the end of this period, but the beneficial effects will not be felt until the end of the hour. Up to twelve other creatures can
attend the banquet. A creature that participates in the banquet gains several benefits. The creature is cured of all diseases and poisons, becomes immune to poison and being frightened, and has +2d6 on all Will saving throws. Its maximum hit points increase by 2d10, and it heals the same amount of current hit points. These benefits last 24 hours.\\
\textbf{In case of two Magical Critical Successes obtained} in the Magic Test the bowl is not consumed.

\medskip\textbf{Blade Barrier}\index[Spells]{Blade Barrier}\\
\textbf{School}: Invocation\\
\textbf{Level}: 6, Municipality\\
\textbf{Launch Time}: 2 Shares\\
\textbf{Range}: 18 metres\\
\textbf{Components}: V, S\\
\textbf{Duration}: 10 minutes \\
You create a vertical wall of spinning blades made of magical energy, sharp as razors. The wall appears within range and remains for the duration. You can create a straight wall up to 100 feet long, 20 feet high, and 3 feet thick, or a circular wall up to 60 feet in diameter, 20 feet high, and 3 feet thick. The wall provides three-quarters cover to creatures behind it, and its space is difficult terrain. \\
When a creature enters the wall's area for the first time in a round or begins its round there, the creature must make a Reflex saving throw. On a failed save, the creature takes 6d10 slashing damage, or half as much on a successful one.\\
A spellcaster who is within one meter of the Blade Barrier is considered distracted.

\medskip\textbf{Snakes Sticks}\index[Spells]{Snakes Sticks}\\
\textbf{School}: Animals and Plants\\
\textbf{Level}: 3, Uncommon\\
\textbf{Launch Time}: 2 Shares\\
\textbf{Range}: 18 metres\\
\textbf{Components}: V, S, M (several sticks and a drop of snake venom)\\
\textbf{Duration}: Concentration up to 1 minute\\
You transform 1d4 sticks, +1 for each time you took the Animal and Plant Magic List, into venomous snakes. The snakes always act in unison in your round and perform the same Action against the same opponent.

These snakes, considered tiny objects, have Defense 13 and 10 Hit Points. If they drop below 0 Hit Points they become sticks again but broken.

With an Action you can command the snakes to attack. Make an attack roll as per the melee spell attack for each Serpent against a creature within 3 feet of them. Each snake it hits causes 1 piercing damage and forces a Fortitude save at DC 14; if the save fails, the creature takes 2d4 poison damage, or half as much on a successful one.

With one Action you can command snakes to move up to 20 feet.\\
\textbf{Each Critical Magical Success obtained} in the Magic Test creates a new snake.

\medskip\textbf{Cruel Mockery}\index[Spells]{Cantrip - Cruel Mockery}\\
\textbf{School}: Enchantment\\
\textbf{Level}: 0, Municipality\\
\textbf{Cast Time}: 1 Action\\
\textbf{Range}: 18 metres\\
\textbf{Components}: V\\
\textbf{Duration}: Instant\\
You unleash a series of insults wrapped in a subtle spell against a creature within range and that you can see. If the target can hear you (though it doesn't have to understand you), it must succeed on a Will save or take 1d4 damage and have -1d6 on its next attack roll before the end of its next round.\\
The damage of the spell increases by 1d4 when you reach CM 5, CM 11 and CM 17, but it costs 2 Actions to cast it enhanced and 2 Magic Points, it is also necessary to have taken Adept of Magic in this Magic List a number of times equal to the enhancements that you want to apply.\\
\textbf{Every 2 Magical Critical Success obtained} in the Magic Test affects another creature.

\medskip\textbf{Bless Water}\index[Spells]{Bless Water}\\
\textbf{School}: Universal\\
\textbf{Level}: 2, Municipality\\
\textbf{Launch Time}: 10 Minutes\\
\textbf{Range}: Contact\\
\textbf{Components}: V, S, M (25 gold coins offered to the church)\\
\textbf{Duration}: Instant\\
Bless up to one liter of liquid, enough to create 5 bottles of Holy Water.\\
You must be a Follower or Devoted to cast this spell.\\
\textbf{For each Magical Critical Success obtained} in the Magic Test you bless an additional liter of liquid.\index{Blessed Water, create}

\medskip\textbf{Blessing}\index[Spells]{Blessing}\\
\textbf{School}: Universal\\
\textbf{Level}: 1, Municipality\\
\textbf{Launch Time}: 2 Shares\\
\textbf{Range}: 9 metres\\
\textbf{Components}: V, S, M (a splash of Holy Water)\\
\textbf{Duration}: 1 minute\\
Bless up to three creatures within range, chosen by you. Targets gain +1 on saving throws and attack rolls.\\
Multiple blessings, even from different Patrons, do not add up. You must be a Follower or Devoted to cast this spell.\\
\textbf{For each Magical Critical Success obtained} in the Magic Test you can add a creature as a target.

\medskip\textbf{Blessing of Life}\index[Spells]{Blessing of Life}\\
\textbf{School}: Abjuration\\
\textbf{Level}: 3, Rare\\
\textbf{Launch Time}: 2 Shares\\
\textbf{Range}: 9 metres\\
\textbf{Components}: V, S\\
\textbf{Duration}: 1 minute, Concentration\\
This spell grants hope and vitality. Choose up to 6 creatures within range. For the duration, each target has +2 on Will saves and regains 1 hit point per round.\\
\textbf{If you get 2 Magical Critical Successes and also have the Healing List} each round the chosen creatures recover 1 more hit point.

\medskip\textbf{Catalm's Blessing}\index[Spells]{Catalm's Blessing}\\
\textbf{School}: Enchantment, Fire\\
\textbf{Level}: 3, Very Rare\\
\textbf{Launch Time}: 2 Shares\\
\textbf{Range}: 18 metres\\
\textbf{Components}: V, S, M (a splash of vinegar)\\
\textbf{Duration}: Instant\\
You call down Cattalm's wrath upon your opponent. The target creature takes 4d6 fire damage, must make a Will save or suffer a -1d6 penalty on the first subsequent proficiency check or attack roll or saving throw, and the caster increases his Fate Point pool by one .\\
\textbf{For every two Magic Critical Success obtained} in the Magic Test you can influence another creature.

\medskip\textbf{Greater Blessing}\index[Spells]{Greater Blessing}\\
\textbf{School}: Invocation\\
\textbf{Level}: 2, Uncommon\\
\textbf{Launch Time}: 1 Minute\\
\textbf{Range}: 18 metres\\
\textbf{Components}: V, S, M (a splash of Holy Water, 10 gold coins)\\
\textbf{Duration}: 1 hour\\
Bless a creature of your choice. The creature within the duration can add 1d6 to a roll before knowing whether the check (TC/TS/Check) was successful or not. This bonus can be used 2 times per hour. You must be a Follower or Devoted to cast this spell.\\
\textbf{For each Magical Critical Success obtained} in the Magic Test you can add a creature as a target or add one hour to the duration.

\medskip\textbf{Supreme Blessing}\index[Spells]{Supreme Blessing}\\
\textbf{School}: Invocation\\
\textbf{Level}: 3, Rare\\
\textbf{Cast Time}: 1 Reaction\\
\textbf{Range}: 27 metres\\
\textbf{Components}: V, S, M (a splash of Holy Water, 25 gold)\\
\textbf{Duration}: Instant\\
Bless a creature of your choice. The creature can reroll two dice from a single check before knowing whether the check was successful or not. The creature chooses whether to take the new rolls gained or keep the old ones. You must be a Follower or Devoted to cast this spell.\\
\textbf{For each Magical Critical Success obtained} on the magic check the creature gets a +1 bonus on the check.

\medskip\textbf{Hold Monsters}\index[Spells]{Hold Monsters}\\
\textbf{School}: Enchantment\\
\textbf{Level}: 5, Municipality\\
\textbf{Launch Time}: 2 Shares\\
\textbf{Range}: 27 metres\\
\textbf{Components}: V, S, M (a small straight piece of iron)\\
\textbf{Duration}: 1 minute\\
Choose a creature within range and that you can see. The target must succeed on a Will save or be paralyzed for the duration. This spell has no effect on undead or constructs. At the end of each of its rounds, the target can make another Will saving throw. On a success, the spell ends for that target.\\
\textbf{For each Magical Critical Success obtained} in the Magic Test you can add a creature as a target as long as they are within 30 feet of each other.

\medskip\textbf{Block Person}\index[Spells]{Block Person}\\
\textbf{School}: Enchantment\\
\textbf{Level}: 2, Municipality\\
\textbf{Launch Time}: 2 Shares\\
\textbf{Range}: 18 metres\\
\textbf{Components}: V, S, M (a small straight piece of iron)\\
\textbf{Duration}: 1 minute\\
Choose a humanoid within range and that you can see. The spell has no effect on creatures with CR 4 or higher. The target must succeed on a Will save or be paralyzed for the duration.\\
\textbf{For each Magical Critical Success obtained} in the Magic Test you can add a creature as a target as long as they are within 30 feet of each other.

\medskip\textbf{Block Person Advanced}\index[Spells]{Block Person Advanced}\\
\textbf{School}: Enchantment\\
\textbf{Level}: 4, Uncommon\\
\textbf{Launch Time}: 2 Shares\\
\textbf{Range}: 18 meters, radius 6 meters\\
\textbf{Components}: V, S, M (a small straight piece of silver)\\
\textbf{Duration}: 1 minute\\
You block up to 2d4 CR (or levels) of creatures within 60 feet of you in a 20-foot radius. You start by blocking the creatures with the lowest CR and subtracting the CR from the 2d4 rolled, proceed until you have no more points to block the creatures. Targets must succeed on a Will save or be paralyzed for the duration.\\
\textbf{For each Magical Critical Success obtained} in the Magic Test you can add 2 points to the 2d4 rolled.

\medskip\textbf{Magic Mouth}\index[Spells]{Magic Mouth}\\
\textbf{School}: Illusion\\
\textbf{Level}: 2, Municipality\\
\textbf{Launch Time}: 1 minute\\
\textbf{Range}: 9 metres\\
\textbf{Components}: V, S, M (a small piece of honeycomb and jade dust worth at least 10 gp, which the spell consumes)\\
\textbf{Duration}: Until dissolved\\
You implant a message into a ranged object, which message is spoken when the activation condition is met. Choose an object that you can see and that is not being worn or carried by another creature. Then say your message, which must be 25 words or less, but can be spread out over a period of up to 10 minutes. Finally, determine the circumstance that will activate the spell, so that it conveys your message.\\
When the circumstance occurs, a magical mouth appears on the object and recites the message with your voice and at the same volume at which you spoke it. If the object you choose has a mouth or something that resembles a mouth (for example, the mouth of a statue), the magical mouth appears so that the words appear to come from the object's mouth. When you cast this spell, you can cause the spell to end after conveying its message, or to linger and repeat the message each time the condition is triggered.\\
The trigger circumstance can be as broad or detailed as you like, but must be based on visible or audible conditions occurring within 30 feet of the object. For example, you could instruct the mouth to speak when any creature approaches within 30 feet of the object or when a silver bell rings within 30 feet of it.

\medskip\textbf{Life Bubble}\index[Spells]{Life Bubble}\\
\textbf{School}: Aria, Abjuration\\
\textbf{Level}: 4, Uncommon\\
\textbf{Launch Time}: 1 minute\\
\textbf{Range}: 9 metres\\
\textbf{Components}: V, S, M (silver and diamond dust per 100 gp that are consumed)\\
\textbf{Duration}: 1 hour for Magical Expertise\\
You can create up to 6 bubbles surrounding the creatures you designate.
The total duration is 1 hour per point in Magical Expertise divided as desired among the creatures in the bubbles.
This bubble allows subjects to breathe freely, even underwater or in a vacuum, and makes them immune to harmful gases and vapors, including inhaled diseases and poisons and spells such as nauseating fog and deadly fog. The bubble protects subjects from extreme temperatures (but not causing damage each round) and extreme pressures.

Life Bubble does not provide protection from negative or positive energy (e.g. on the Negative and Positive Energy planes), the ability to see in low visibility conditions (such as in smoke or fog), nor the ability to move or act normally in conditions that impede movement (such as underwater).

\medskip\textbf{Feather Fall}\index[Spells]{Feather Fall}\label{cadutapiuma}\\
\textbf{School}: Air\\
\textbf{Level}: 1, Municipality\\
\textbf{Casting Time}: 1 Reaction, which you take when you or a creature within 60 feet of you falls\\
\textbf{Range}: 18 metres\\
\textbf{Components}: V, M (a small feather or piece of feather)\\
\textbf{Duration}: 1 minute\\
Choose up to five creatures within range. A falling creature's rate of descent decreases to 60 feet per round until the spell ends. If the creature lands before the spell ends, it takes no falling damage and can land on its feet; for that creature the spell ends.\\
\textbf{For each Magical Critical Success obtained} in the Magic Test you can move sideways 1 meter or affect another creature.

\medskip\textbf{Calm Emotions}\index[Spells]{Calm Emotions}\\
\textbf{School}: Enchantment\\
\textbf{Level}: 2, Municipality\\
\textbf{Launch Time}: 2 Shares\\
\textbf{Range}: 18 metres\\
\textbf{Components}: V, S\\
\textbf{Duration}: Concentration, maximum 1 minute\\
You try to suppress strong emotions in a group of people. Each humanoid in a 20-foot-radius sphere centered on a point you choose within range must make a Will saving throw; if it wishes, a creature can choose to fail this saving throw. If a creature fails its saving throw, choose one of these two effects. \\
\emph{Appease}. You can suppress any effect that makes the target Fascinated or Frightened. When this spell ends, the suppressed effects resume, as long as their duration has not expired in the meantime.\\
\emph{Indifference}. You can make a target indifferent to a creature of your choice, towards which it is hostile. This indifference ends if the target is attacked or damaged by a spell or if it sees one of her friends being damaged. When the spell ends, the creature becomes hostile again, unless the Storyteller determines otherwise.

\medskip\textbf{Walk on Water}\index[Spells]{Walk on Water}\\
\textbf{School}: Water\\
\textbf{Level}: 3, Municipality\\
\textbf{Launch Time}: 2 Shares\\
\textbf{Range}: 9 metres\\
\textbf{Components}: V, S, M (a piece of cork)\\
\textbf{Duration}: 1 hour\\
This spell grants the ability to move through liquid surfaces (such as water, acid, mud, snow, quicksand, or lava) as if they were harmless solid ground (creatures that pass through molten lava can still take heat damage or melt into the acid ). Up to ten willing creatures within range that you can see gain this ability for the duration. If your target is submerged in liquid, the spell returns the target to the surface of the liquid at a rate of 30 feet per round.\\
\textbf{For each Magical Critical Success obtained} in the Magic Test the spell lasts 1 hour longer or you affect another creature.

\medskip\textbf{Walking on the Wind}\index[Spells]{Walking on the Wind}\\
\textbf{School}: Air\\
\textbf{Level}: 6, Uncommon\\
\textbf{Launch Time}: 1 minute\\
\textbf{Range}: 9 metres\\
\textbf{Components}: V, S, M (fire and holy water)\\
\textbf{Duration}: 8 hours\\
For the duration, you and up to ten other willing creatures within range that you can see take on gaseous form, becoming clouds. While in cloud form, a creature has a flying speed of 300 feet and resistance to damage from nonmagical weapons. Returning to normal form takes 1 minute, during which the creature is incapacitated and cannot move. Until the spell ends, a creature can revert to cloud form, which requires a one-minute transformation. If a creature is in cloud form and flying when the effect ends, the creature descends 60 feet per round per minute until it lands safely. If it fails to land after 1 minute, the creature will fall the remaining distance.

\medskip\textbf{Charming People}\index[Spells]{Charming People}\\
\textbf{School}: Enchantment\\
\textbf{Level}: 1, Municipality\\
\textbf{Launch Time}: 2 Shares\\
\textbf{Range}: 9 metres\\
\textbf{Components}: V, S\\
You try to charm a humanoid within range and that you can see. He must make a Will save, and has +1d6 if he is fighting you or your allies. On a failed save, he is charmed by you until the spell ends or until you or your allies do something harmful to him. The fascinated creature considers you a friendly acquaintance. When the spell ends, the creature is aware that it has been charmed by you. Whenever the creature is threatened by you or a friend, it can re-roll the saving throw with a +2 bonus.\\
\textbf{For each Magical Critical Success obtained} in the Magic Test you can add a creature as a target. When you cast the spell, the target creatures must be within 30 feet of each other.

\medskip\textbf{Anti-Magic Field}\index[Spells]{Anti-Magic Field}\\
\textbf{School}: Abjuration\\
\textbf{Level}: 8, Rare\\
\textbf{Launch Time}: 2 Shares\\
\textbf{Range}: Personal (3 meter radius sphere)\\
\textbf{Components}: V, S, M (a pinch of iron powder or iron file)\\
\textbf{Duration}: Concentration, maximum 1 hour\\
You are surrounded by an invisible sphere of anti-magic 10 feet in radius. This area is separated from the magical energy that permeates the multiverse. Spells cannot be cast inside the sphere, the summoned creatures disappear and even magical objects become normal. Until the spell ends, the sphere moves with you, centered on you. Spells and other magical effects, except those created by an artifact or Patron, are suppressed within the sphere and cannot penetrate it. A slot spent casting a suppressed spell is expended. While an effect is suppressed, it does not function, but the time it spends suppressed counts towards its duration.\\

\emph{Effects with Target}. Spells and other magical effects, such as Arcane Bolt and charm person, that target a creature or object within the sphere have no effect on that target.\\
\emph{Areas of Magic}. The area of ​​another spell or magical effect, such as fireball, cannot extend into the sphere. If the sphere overlaps an area of ​​magic, the portion of that area covered by the sphere is suppressed. For example, flames generated by a wall of fire are suppressed within the sphere, creating a hole in the wall if the overlap is large enough. Spells. Any spell or other magical effect active on a creature or object within the sphere is suppressed as long as the creature or object is within the sphere.\\
\emph{Magic Items}. The properties and powers of magical items are suppressed by the sphere. For example, a +1 longsword inside the sphere functions as a nonmagical longsword. The properties and powers of magical weapons are suppressed if they are used against a target within the sphere or wielded by an attacker within the sphere. If a magical weapon or magical ammunition leaves the sphere entirely (for example, if you shoot a magical arrow or hurl a magical spear at a target outside the sphere), the item's magic is no longer suppressed as soon as it leaves the sphere .\\
\emph{Travel Magic}. Teleportation and planar travel do not work within the sphere, whether the sphere is the destination or starting point of this magical journey. Within the sphere, a portal to another location, world, or plane of existence, as well as an extradimensional space such as that created by the rope trick spell, remains closed.\\
\emph{Creatures and Objects}. Within the sphere, a creature or object summoned or created by magic temporarily fades from existence. The creature or object instantly reappears once the space it occupied is no longer within the sphere.\\
\emph{Dispel magic}. Spells and magical effects such as dispel magic have no effect on the sphere. Likewise, spheres created by other antimagic field spells do not cancel each other out.

\medskip\textbf{Disguise Self}€6399[Spells]{Disguise Self}\\
\textbf{School}: Illusion\\
\textbf{Level}: 1, Municipality\\
\textbf{Launch Time}: 2 Shares\\
\textbf{Range}: Personal\\
\textbf{Components}: V, S\\
\textbf{Duration}: 1 hour\\
You change your appearance, along with that of your clothes, armor, weapons, and other items you wear, until the spell ends or you take an action to end the spell. You can appear 30 centimeters shorter or taller, thin, fat or somewhere in between. You cannot change your physical shape, so you must adopt a form that has the same distribution of limbs. For everything else, the illusion is limited only by your imagination.\\
The changes brought about by this spell are not capable of withstanding physical inspection. For example, if you use this spell to add a hat to your outfit, objects pass through the hat, and anyone who touches it would feel nothing and would end up touching your head and hair. If you use this spell to appear thinner than you are, the hand of a person who tries to touch you will bounce off you, while at first sight it appears to stop in mid-air. To distinguish your disguise, a creature can use 2 actions to inspect your appearance and must succeed at a +4 Awareness check against the spell's saving throw DC.

\medskip\textbf{Hut}\index[Spells]{Hut}\\
\textbf{School}: Invocation\\
\textbf{Level}: 3, Uncommon\\
\textbf{Launch Time}: 1 minute\\
\textbf{Range}: Personal (3 meter radius hemisphere)\\
\textbf{Components}: V, S, M (a diamond chip worth 50 gp that the spell consumes)\\
\textbf{Duration}: 8 hours\\
A 10-foot radius half-sphere of motionless force forms around and above you, remaining stationary for the duration. The spell ends if you leave the area. Eight creatures of Medium or smaller size can enter the dome with you. The spell fails if the area includes a larger creature or more than nine creatures. Creatures and objects inside the dome can pass through it freely when you cast this spell. All other creatures and objects must make a Fortitude saving throw or be unable to pass through it that round. Spells and other magical effects cannot extend beyond the dome or pass through it if they are Cantrips. The atmosphere inside the space is comfortable and dry, whatever the climate outside.\\
Until the spell ends you can command the interior lighting to be full, dim, or dark. The dome is opaque from the outside, any color you choose, but is transparent from the inside.\\
\textbf{For each Magical Critical Success obtained} in the Magic Test the spell lasts 2 hours longer.

\medskip\textbf{Enhanced Ability}\index[Spells]{Enhanced Ability}\\
\textbf{School}: Transmutation\\
\textbf{Level}: 2, Municipality\\
\textbf{Launch Time}: 2 Actions\\
\textbf{Range}: Contact\\
\textbf{Components}: V, S, M (fur or feather of a beast)\\
\textbf{Duration}: maximum 10 minutes\\
You grant a magical buff to a creature you are in contact with. Choose one of the following effects; the target gains that effect until the spell ends.\\
\emph{Fox's Cunning}. The target has +1d6 on Intelligence and Strength checks\\
\emph{Forza del Toro}. The target has +1d6 on Strength checks, and its Encumbrance ability doubles.\\
\emph{Grace of Luminous Energy}. The target has +1d6 on Dexterity checks. Furthermore, if he is not incapacitated, he takes no damage from falls of 6 meters or less.\\
\emph{Bear Resistance}. The target has +1d6 on Constitution checks. He also gains 2d6 temporary hit points, which are lost at the end of the spell.\\
\emph{Wisdom of the Owl}. The target has +1d6 on Wisdom checks. \\
\emph{Splendor of the Eagle}. The target has +1d6 on Charisma checks.\\
\textbf{For each Magical Critical Success obtained} in the Magic Test you can target an additional creature

\medskip\textbf{Flesh in Stone - Stone in Flesh}\index[Spells]{Flesh in Stone}\index[Spells]{Stone in Flesh}\\
\textbf{School}: Earth\\
\textbf{Level}: 6, Uncommon - Rare\\
\textbf{Launch Time}: 2 Shares\\
\textbf{Range}: 18 metres\\
\textbf{Components}: V, S, M (a pinch of lime, water and earth)\\
\textbf{Duration}: Permanent\\
You try to turn a creature within range that you can see into stone. If the target's body is made of flesh, the creature becomes slowed 1/6r and must make a Fortitude saving throw. If she fails her save she becomes slowed for 1/10 minutes and her flesh begins to harden. If the saving throw succeeds, the creature suffers no further effects. The creature that fails the initial saving throw the next round must make a new Fortitude saving throw. If the save is successful there are no further effects.
If she fails her saving throw, she is turned to stone and remains petrified for the duration.\\
If the creature is physically harmed while petrified, it suffers deformities similar to the damage done to the stone if it returns to its original state.\\
The \emph{Stone to Flesh} spell returns a creature to flesh as long as it has not been transformed for more than a year. The dispel magic spell cannot negate its effects.

\medskip\textbf{Blindness/Deafness}\index[Spells]{Blindness/Deafness}\\
\textbf{School}: Necromancy\\
\textbf{Level}: 2, Municipality\\
\textbf{Launch Time}: 2 Shares\\
\textbf{Range}: 9 metres\\
\textbf{Components}: V\\
\textbf{Duration}: 1 minute, Concentration\\
You can blind or deafen an enemy. Choose a creature within range and that you can see. The target must make a Fortitude saving throw. On a failed save, the target is blinded or deafened (your choice) for the duration.\\
\textbf{For every two Magical Critical Successes obtained} in the Magic Test you can add another target in range. If you score 3 Magical Critical Successes the target is affected by the spell all day.

\medskip\textbf{Advanced Blindness/Deafness}\index[Spells]{Advanced Blindness/Deafness}\\
\textbf{School}: Necromancy\\
\textbf{Level}: 3, Uncommon\\
\textbf{Launch Time}: 2 Shares\\
\textbf{Range}: 36 metres\\
\textbf{Components}: V,S,M (cerumen or a piece of black cloth)\\
\textbf{Duration}: 10 minutes\\
You can blind or deafen an enemy. Choose a creature within range and that you can see. The target must make a Fortitude saving throw. On a failed save, the target is blinded or deafened (your choice) for the duration.\\
\textbf{For each Magical Critical Success obtained} in the Magic Test you can target an additional creature.

\medskip\textbf{Conceal}\index[Spells]{Conceal}\\
\textbf{School}: Transmutation\\
\textbf{Level}: 7, Rare\\
\textbf{Launch Time}: 2 Shares\\
\textbf{Range}: Contact\\
\textbf{Components}: V, S, M (a dust composed of diamond, emerald, ruby ​​and sapphire dust worth at least 50,000 gp, which the spell consumes)\\
\textbf{Duration}: Until dissolved \\
With this spell, a willing creature or object can be hidden, impossible to detect for the duration. By casting this spell and coming into contact with a target, the target becomes invisible and cannot be targeted by divination spells, nor sensed by scrying sensors created by divination spells.\\
If the target is a creature, it falls into a state of suspended animation. For him, time ceases to flow, and he does not age. \\
You can set a condition for the spell to end early. The condition can be anything you want, but it must occur or be visible within 1.5 kilometers of the target. Examples include \emph{at the next judgment of the Patrons} or \emph{when the tarrasque awakens}. This spell also ends if the target takes damage.

\medskip\textbf{Magic Circle}\index[Spells]{Magic Circle}\\
\textbf{School}: Abjuration\\
\textbf{Level}: 3, Municipality\\
\textbf{Launch Time}: 1 minute\\
\textbf{Range}: 3 metres\\
\textbf{Components}: V, S, M (Holy water or powdered silver and iron worth at least 100 gp, which the spell consumes)\\
\textbf{Duration}: 1 hour\\
You create a cylinder of magical energy 10 feet in radius and 20 feet high, centered on a point on the ground within range that you can see. Bright runes appear wherever the cylinder intersects with the floor or other surface.\\
Choose one or more of the following creature types: celestial, elemental, fey, demon, or undead. The circle affects a creature of the chosen type in the following ways:\\

- The creature cannot knowingly enter the cylinder through any nonmagical means. If the creature attempts to use teleportation or planeswalking to do so, it must first succeed at a Will saving throw.\\

- The creature has -1d6 on attack rolls against targets inside the cylinder.\\

- Targets inside the cylinder cannot be charmed, frightened, or possessed by the creature. When you cast this spell, you can choose to have the magic work in the opposite direction, preventing a creature of the specified type from leaving the cylinder and protecting targets outside.\\

\textbf{For each Magical Critical Success obtained} in the Magic Test you can increase the duration by 1 hour.

\medskip\textbf{Circle of Death}\index[Spells]{Circle of Death}\\
\textbf{School}: Invocation\\
\textbf{Level}: 6, Very Rare\\
\textbf{Launch Time}: 2 Shares\\
\textbf{Range}: 45 metres\\
\textbf{Components}: V, S, M (a black pearl crushed to dust worth at least 500 gp)\\
\textbf{Duration}: Instant\\
A sphere of negative energy, 60 feet in radius, erupts at a point within range. Each creature in that area must make a Fortitude saving throw. A target takes 8d6 void damage on a failed save, or half as much damage on a successful one. \\
\textbf{For each Magical Critical Success obtained} in the Magic Test the damage increases by 4d6.\\
\textbf{Saving Throw Success/Critical Failure}: In case of a critical failure the damage is doubled, in case of a critical success the damage is further halved

\medskip\textbf{Teleportation Circle}\index[Spells]{Teleportation Circle}\\
\textbf{School}: Summon\\
\textbf{Level}: 5, Uncommon\\
\textbf{Launch Time}: 1 minute\\
\textbf{Range}: 3 metres\\
\textbf{Components}: V, M (rare chalks and inks infused with precious gems worth at least 50 gp, which the spell consumes)\\
\textbf{Duration}: 1 round\\
As you cast the spell, you draw a 10-foot-diameter circle on the floor, inscribed with sigils that connect your location to a permanent teleportation circle of your choice, whose sequence of sigils you know, and which is on the same plane of existence you are on. A glowing portal opens within the circle you draw and remains open until the end of your next round. Any creature that enters the portal instantly reappears within 3 feet of the destination circle or in non-space
nearest occupied, if it cannot appear within 1 meter of it.\\
Many large temples, guilds, and other important locations have permanent teleportation circles engraved somewhere in their vicinity. Each of these circles has a unique sigil sequence: a series of magical runes arranged in a precise pattern.\\ When you gain the ability to cast this spell, you learn the sigil sequences of
two destinations on the Material Plane, determined by the Storyteller. Throughout your adventures you can learn new seal sequences. You can memorize a sequence of seals after studying it for at least 1 minute.\\
You can create a permanent teleportation circle by casting this spell in the same place every day for one year. You do not have to use the teleportation circle when casting the spell this way.\\
\textbf{NOTE}: Teleporting from Curyan to Tiya and vice versa has only a 5% success rate.

\medskip\textbf{Clairvoyance}\index[Spells]{Clairvoyance}\\
\textbf{School}: Divination\\
\textbf{Level}: 3, Municipality\\
\textbf{Launch Time}: 10 minutes\\
\textbf{Range}: 1.5 kilometres\\
\textbf{Components}: V, S, M (a focus worth at least 100 gp, whether a jeweled horn for hearing or a glass eye for seeing)\\
\textbf{Duration}: Concentration, maximum 10 minutes\\
You create an invisible sensor in a place that is familiar to you and within range (a place you have already visited or seen before) or in an obvious but unfamiliar place (such as behind a door or a corner, or in the middle of a grove of trees). The sensor remains in place for the duration, and cannot be attached or otherwise interacted with. When you cast this spell, you choose to see or hear. You can use the direction chosen by the sensor, as if you were in its space. With two actions, you can switch from hearing to hearing and vice versa. A creature that can see the sensor (a creature with invisibility or true seeing) perceives it as an intangible, luminous orb the size of your fist.\\
\textbf{For each Magical Critical Success obtained} in the Magic Test the duration increases by 10 minutes or the range increases by 500m.

\medskip\textbf{Clone}\index[Spells]{Clone}\\
\textbf{School}: Necromancy\\
\textbf{Level}: 8, Uncommon\\
\textbf{Range}: Contact\\
\textbf{Components}: V, S, M(a diamond worth at least 1000 gp and at least 1 pound of flesh of the creature to be cloned, which the spell consumes, and a vessel worth at least 2000 gp that has a sealable lid and is large enough to hold a Medium creature, such as a large urn, a coffin, a mud-filled pit in the ground, or a crystal container filled with salt water)\\
\textbf{Duration}: Instant\\
This spell produces an inert duplicate of a living creature as a safeguard from death. This clone is formed inside a sealed container and reaches maximum size and maturity after 120 days; you can also decide that the clone is a younger version of the same creature. It remains inert and survives indefinitely, as long as the container remains undisturbed.\\
At any time after the clone has matured, if the original creature dies, its soul transfers to the clone, as long as the soul is free and willing to return. The clone is physically identical to the original and has the same personality, memories and characteristics, but none of the original's equipment. The physical remains of the original creature, if they still exist, become inert and cannot be brought back to life, as the creature's soul is elsewhere. \\
\textbf{This spell is not selectable if Patrons are active}

\medskip\textbf{Accurate Strike}\index[Spells]{Cantrip - Accurate Strike}\\
\textbf{School}: Divination\\
\textbf{Level}: 0, Municipality\\
\textbf{Launch Time}: 2 Shares\\
\textbf{Range}: 9 metres\\
\textbf{Components}: Y\\
\textbf{Duration}: 1 round\\
You hold out your hand and point your finger at a target within range. Your magic gives you a brief understanding of the target's defenses. By the end of the next round you gain +1d6 on your first attack roll against that target.\\
\textbf{For each Magical Critical Success obtained} the bonus lasts for an additional round.


\medskip\textbf{Blinding Strike}\index[Spells]{Blinding Strike}\\
\textbf{School}: Invocation\\
\textbf{Level}: 3, Rare\\
\textbf{Cast Time}: 1 Immediate Action\\
\textbf{Range}: personal\\
\textbf{Components}: V\\
\textbf{Duration}: 1 minute\\
The target hit by the strike takes an extra 3d8 Light damage and must succeed on the Fortitude save or become blinded until the spell ends. At the end of each of its rounds, the blinded target repeats the saving throw, ending the spell on itself on a success.\\
\textbf{For each Magical Critical Success obtained} in the Magic Test you inflict +1d8 Light damage.


\medskip\textbf{Fiery Strike}\index[Spells]{Fiery Strike}\\
\textbf{School}: Fire\\
\textbf{Level}: 5, Municipality\\
\textbf{Launch Time}: 2 Shares\\
\textbf{Range}: 18 metres\\
\textbf{Components}: V, S, M (pinch of sulphur)\\
\textbf{Duration}: Instant\\
A vertical column of divine fire descends from the sky and strikes the location you specify. Each creature in a 10-foot-radius, 40-foot-tall cylinder centered on a point within range must make a Reflex saving throw. A creature takes 8d6 Light damage on a failed save, or half as much damage on a successful one.\\
\textbf{For each Magic Critical Success obtained} in the Magic Test the Light damage increases by 4d6.\\
\textbf{Saving Throw Success/Critical Failure}: In case of a critical failure the damage is doubled, in case of a critical success the damage is further halved

\medskip\textbf{Flaming Strike}\index[Spells]{Flaming Strike}\\
\textbf{School}: Invocation\\
\textbf{Level}: 1, Rare\\
\textbf{Cast Time}: 1 Immediate Action\\
\textbf{Range}: personal\\
\textbf{Components}: V\\
\textbf{Duration}: 1 minute\\
The target hit by the shot takes an extra 1d6 Fire damage. Each round he must make a Fortitude save or take 1d6 fire damage, this effect ends after one minite or when the saving throw succeeds.\\
To be able to cast the spell you must pass a Magic Test if you are fighting.\\
\textbf{For each Magical Critical Success obtained} in the Magic Test you inflict +1d6 Fire damage.

\medskip\textbf{Shimmering Strike}\index[Spells]{Shining Strike}\\
\textbf{School}: Invocation\\
\textbf{Level}: 2, Uncommon\\
\textbf{Cast Time}: 1 Immediate Action\\
\textbf{Range}: personal\\
\textbf{Components}: V\\
\textbf{Duration}: 1 minute\\
The target hit by the blow takes 2d6 Light damage and becomes visible for the spell's duration. Additionally, the creature radiates light in a 1 meter radius.\\
\textbf{For each Magical Critical Success obtained} in the Magic Test you deal an additional 1d6 Light damage.

\medskip\textbf{Command}\index[Spells]{Command}\\
\textbf{School}: Enchantment\\
\textbf{Level}: 1, Municipality\\
\textbf{Launch Time}: 2 Shares\\
\textbf{Range}: 18 metres\\
\textbf{Components}: V\\
\textbf{Duration}: 1 round\\
You speak a one-word command to a creature within range and that you can see. The target must succeed on a Will save or follow the command within its next round. The spell has no effect if the target is undead, if he does not understand your language, or if your command would cause him harm. Below are some typical commands and their effects. You can issue commands other than those described here, in which case the Storyteller will determine the target's behavior. If the target cannot follow your command, the spell ends.\\

- \emph{Come closer}. The target moves towards you the shortest and most direct route, ending its round if it comes within 1 meter of you.

- \emph{Stopped}. The target doesn't move and then ends its round. A flying creature stays in place as long as it can. If you have to move to stay in the air, fly the minimum distance necessary to do so.\\

- \emph{Throwaway}. The target throws whatever it is holding and then ends its round.

- \emph{Escape}. The target spends its round moving away from you by the fastest means available to it.\\

- \emph{Stripe}. The target falls prone and then ends its round.\\

\textbf{For each Magical Critical Success obtained} in the Magic Test you can act on an additional creature. At the time you cast the spell, the target creatures must be within 30 feet of each other and perform the same command.

\medskip\textbf{CTRLC+CTRLV (Copy Paste)}\index[Spells]{CTRLC+CTRLV (Copy Paste)}\\
\textbf{School}: Universal\\
\textbf{Level}: 1, Very Rare\\
\textbf{Launch Time}: 2 Shares\\
\textbf{Range}: Personal\\
\textbf{Components}: V, S, M (three small ceramic cubes bearing the letter C, the letter V and the CTRL glyph)\\
\textbf{Duration}: 1 minute for Magical Expertise\\
This spell allows you to copy text from one source to another. In the case of a non-magical source this can be a book, a scroll, runes on a slab or a staff. The destination that must be placed on the source will copy the symbols in the shape and size up to its capacity, for a maximum of 1 (destination) page per minute.

If the writing is a spell, therefore on a Tome or Scroll, the rules and limitations required for copying Spells on the Tome must still be respected. This spell allows you to avoid the Magic Test in case of a Spell of a level higher than the maximum allowed. Once a spell is copied, this spell ends.

\medskip\textbf{Language Comprehension}\index[Spells]{Language Comprehension}\\
\textbf{School}: Divination\\
\textbf{Level}: 1, Municipality\\
\textbf{Launch Time}: 2 Shares\\
\textbf{Range}: Personal\\
\textbf{Components}: V, S, M (a pinch of salt and soot)\\
\textbf{Duration}: 1 hour\\
For the duration, you understand the literal meaning of any spoken language you hear.\\
\textbf{For each Magical Critical Success obtained} in the Magic Test the duration doubles. With three critical successes you are also able to read.

\medskip\textbf{Writing Comprehension}\index[Spells]{Writing Comprehension}\\
\textbf{School}: Divination\\
\textbf{Level}: 2, Uncommon\\
\textbf{Launch Time}: 2 Shares\\
\textbf{Range}: Personal\\
\textbf{Components}: V, S, M (a pinch of silver and dry ink)\\
\textbf{Duration}: 1 hour\\
For the duration you understand any non-magical written language you see. You must be in contact with the surface on which the words are written. It takes you 1 minute to read a page of text. This spell does not decode secret messages in a text or glyph, such as an arcane sigil, that is not part of a written language.\\
\textbf{For each Magical Critical Success obtained} in the Magic Test the duration doubles.

\medskip\textbf{Compulsion}\index[Spells]{Compulsion}\\
\textbf{School}: Enchantment\\
\textbf{Level}: 4, Uncommon\\
\textbf{Launch Time}: 2 Shares\\
\textbf{Range}: 9 metres\\
\textbf{Components}: V, S\\
\textbf{Duration}: Concentration, maximum 1 minute\\
Creatures of your choice within range that you can see and hear you must make a Will saving throw. A target automatically succeeds on the saving throw if it cannot be charmed. Until the spell ends, you can use an action during each of your rounds to point in a horizontal direction relative to you. Each target affected by the spell must use as much of its movement as possible during its next round to move in that direction. The target cannot take any actions before moving. After moving this way, the target can make another Will save to attempt to end the effect.\\
A target cannot be forced to move into an obviously lethal hazard, such as flames or pits.

\medskip\textbf{Communion}\index[Spells]{Communion}\\
\textbf{School}: Divination\\
\textbf{Level}: 5, Rare\\
\textbf{Launch Time}: 1 minute\\
\textbf{Range}: Personal\\
\textbf{Components}: V, S, M (incense and a vial of Holy Water)\\
\textbf{Duration}: 1 minute\\
You communicate with your Patron and ask him up to three questions that can be answered with a yes or no. You must ask the questions before the spell ends. You will receive the correct answer to each question. Divine creatures are not necessarily omniscient, so you may receive \emph{unclear} as an answer to a question regarding information not pertaining to the Patron's knowledge. In the event that a one-word response would be misleading or contrary to the Patron's interests, the Storyteller may instead give a short sentence as a response.\\
If you cast the spell two or more times before the new dawn has risen there is a cumulative 25\% chance that for each casting after the first you will get no response. The Storyteller makes this roll in secret.\\
\textbf{This spell is not selectable if the Patrons are not active}

\medskip\textbf{Communion with Nature}\index[Spells]{Communion with Nature}\\
\textbf{School}: Divination\\
\textbf{Level}: 5, Very Rare\\
\textbf{Launch Time}: 1 minute\\
\textbf{Range}: Personal\\
\textbf{Components}: V, S\\
\textbf{Duration}: Instant\\
For an instant you become one with nature and gain insight into the surrounding area. In outdoor environments, the spell gives you information about the land within 5 kilometers of you. In caves and other natural underground environments, the range is limited to 100 meters. The spell does not work in places where nature has been replaced by buildings, such as in dungeons and towns.\\
Immediately learn about up to three topics of your choice on one of the following subjects, relevant to the area:

- land and bodies of water\\
- plants, minerals, animals and prevalent populations\\
- powerful celestials, elementals, fey, demons or undead\\
- influences from other planes of existence\\
- buildings\\
\textbf{For each Magical Critical Success obtained} in the Magic Test you learn an additional topic.

\medskip\textbf{Confusion}\index[Spells]{Confusion}\\\hypertarget{inconfusion}{}\label{incconfusione}
\textbf{School}: Enchantment\\
\textbf{Level}: 4, Municipality\\
\textbf{Launch Time}: 2 Shares\\
\textbf{Range}: 27 metres\\
\textbf{Components}: V, S, M (three walnut shells)\\
\textbf{Duration}: 1 minute\\
This spell attacks and bends the minds of creatures, generating illusions and provoking uncontrolled actions. When you cast this spell, each creature in a 10-foot-radius sphere centered on a point you choose within range must succeed at a Will saving throw or suffer its effects. A target affected by the spell cannot take reactions and must roll a d10 at the start of each of its rounds to determine its behavior for that round.

\medskip

\begin{tabularx}{0.45\textwidth}{lX}
\hline
d10 & Behavior\\
1 & The creature uses all of its movement to move in a random direction. To determine direction, roll a d8 assigning each face a compass point. There
creature will not take any actions this round. \\
2-6 & The creature can't move or attack this round.\\
7-8 & The creature uses its 2 Actions (and no others) to make a melee attack against a randomly determined creature within its reach. If there is no creature within range, the creature will do nothing this round.\\
9-10 & The creature can act and move normally.\\
\end{tabularx}

\medskip

At the end of each of its rounds, a target affected by the spell can make a Will saving throw. If he succeeds, the effect ends for him. \\
\textbf{For each Magical Critical Success obtained} in the Magic Test the radius of the sphere increases by 1 meter.

\medskip\textbf{Contagious Confusion}\index[Spells]{Contagious Confusion}\\
\textbf{School}: Enchantment\\
\textbf{Level}: 8, Very Rare\\
\textbf{Launch Time}: 10 minutes\\
\textbf{Range}: Contact\\
\textbf{Components}: V, S, M (tooth powder)\\
\textbf{Duration}: 1 minute\\
This spell attacks and bends the minds of creatures, generating illusions and provoking uncontrolled actions. Once you have cast this spell you then have one minute to touch the first creature. This creature can make a Will save to negate the effects.

Any creature touched by the first creature transmits the Confusion effect, with a saving throw like the first creature, the duration of the effect on this creature will be one minute.

If the caster does not touch a creature within one minute then he himself will be the victim of the confusion spell, without the possibility of a saving throw.

\medskip\textbf{Cone of Cold}\index[Spells]{Cone of Cold}\\
\textbf{School}: Water\\
\textbf{Level}: 5, Municipality\\
\textbf{Launch Time}: 2 Shares\\
\textbf{Range}: Personal (18 meter cone)\\
\textbf{Components}: V, S, M (a small crystal or glass cone)\\
\textbf{Duration}: Instant\\
A blast of cold air erupts from your hands. Each creature in a 60-foot cone must make a Fortitude saving throw. A creature takes 8d8 cold damage on a failed save, or half as much damage on a successful one. A creature killed by this spell becomes an ice statue until it thaws.\\
\textbf{For each Magical Critical Success obtained} in the Magic Test the damage increases by 4d8\\
\textbf{Saving Throw Success/Critical Failure}: In case of a critical failure the damage is doubled, in case of a critical success the damage is further halved

\medskip\textbf{Knowledge of Legends}\index[Spells]{Knowledge of Legends}\\
\textbf{School}: Divination\\
\textbf{Level}: 5, Municipality\\
\textbf{Launch Time}: 10 minutes\\
\textbf{Range}: Personal\\
\textbf{Components}: V, S, M (incense worth at least 250 gp, which the spell consumes, and four strips of ivory worth at least 50 gp)\\
\textbf{Duration}: Instant\\
Name or describe a person, place or object. The spell brings to your mind a brief summary of the most important knowledge on the topic you named. If the thing you named has no legendary relevance, you get no information. The more information you have on the topic, the more precise and detailed the information you will receive. The information you receive will be accurate, but perhaps hidden in metaphorical language.

\medskip\textbf{Contagion}\index[Spells]{Contagion}\\
\textbf{School}: Necromancy\\
\textbf{Level}: 5, Uncommon\\
\textbf{Launch Time}: 2 Shares\\
\textbf{Range}: Contact\\
\textbf{Components}: V, S\\
\textbf{Duration}: 7 days\\
Through contact you can inflict diseases. Make a melee attack against a creature within range. If you hit, you infect the creature with a disease of your choice from those described below. At the end of each of the target's rounds, it must make a Fortitude saving throw. After failing three of these saving throws, the disease's effects last for the duration, and the creature no longer makes saving throws. After succeeding at three of these saving throws, the creature recovers from the disease, and the spell ends. \\
Since this spell induces a natural disease in its target, any effect that removes disease or improves the effects of disease applies to it.\\
- \emph{Putrid Meat}. The creature's skin rots. The creature has -1d6 on Charisma checks and any damage is doubled.\\
- \emph{Blinding Weakness}. Pain grips the creature's mind as its eyes turn milky white. The creature has -1d6 on Wisdom checks and Will saving throws, and is blinded.\\
- \emph{Lurid Fever}. A devastating fever wreaks havoc on the creature's body. The creature has -1d6 on Strength checks and Fortitude saving throws, and on attack rolls that use Strength.\\
- \emph{Fitments}. The creature is overcome by tremors. The creature has -1d6 on Dexterity checks, Reflex saving throws, and attack rolls that use Dexterity.\\
- \emph{Mind Fire}. The creature's mind is gripped by fever. The creature has -1d6 on Intelligence checks and Will saving throws, and behaves as if it were under the effect of the confusion spell in combat.\\
- \emph{Slime Death}. The creature begins to bleed incessantly. The creature has -1d6 on Constitution checks and Fortitude saving throws. Additionally, whenever the creature takes damage, it is stunned until the end of its next round.\\


\medskip\textbf{Contingency}\index[Spells]{Contingency}\\
\textbf{School}: Invocation\\
\textbf{Level}: 6, Uncommon\\
\textbf{Launch Time}: 10 minutes\\
\textbf{Range}: Personal\\
\textbf{Components}: V, S, M (a figurine of yourself carved from ivory and decorated with gems worth at least 1,500 gp)\\
\textbf{Duration}: 10 days\\
Choose a spell of Level 4 or lower that you can cast, that has a casting time of 2 Actions, and that can target you. You cast that spell (called a contingent spell) as part of the contingency casting, expending both of your spell slots, but without the contingent spell taking effect. Instead, it will take effect when a certain circumstance occurs. Describe this circumstance as you cast the two spells. For example, a contingency cast in conjunction with breathing underwater might stipulate that breathing underwater takes effect when you are submerged in water or similar liquid.\\
The contingency spell takes effect immediately after the circumstance first occurs, whether you want it to or not, and then the contingency ends. The contingent spell affects only you, although it can normally target others as well. You can only use one contingency spell at a time. If you cast this spell again, another contingency spell's effect on you ends. Furthermore, the contingency ends for you if the material component is no longer on your person.\\
\textbf{For each Magical Critical Success obtained} in the Magic Test the contingency lasts 10 days longer.

\medskip\textbf{Counterspell}\index[Spells]{Counterspell}\\
\textbf{School}: Abjuration\\
\textbf{Level}: 3, Municipality\\
\textbf{Casting Time}: 1 Reaction, which you take when you see a creature/object within 18 meters manifest a spell\\
\textbf{Range}: 18 metres\\
\textbf{Components}: S \\
\textbf{Duration}: Instant\\
You use a Reaction Action to make an Arcana check at DC 13. If the check succeeds you understand whether you can cancel the spell's effect with Counterspell. The spell negated must be Level 2 or lower, regardless of whether it is cast by a spellcaster or item. Each Magic Critical success or enhancement gained from the original spell raises the spell's level by 1.\\
\textbf{For every two Magical Critical Success obtained} in the Magic Test you can cancel a spell of a higher level.

\medskip\textbf{Control Water}\index[Spells]{Control Water}\\
\textbf{School}: Water\\
\textbf{Level}: 4, Municipality\\
\textbf{Launch Time}: 2 Actions\\
\textbf{Range}: 90 metres\\
\textbf{Components}: V, S, M (a drop of water and a pinch of powder)\\
\textbf{Duration}: Concentration, maximum 10 minutes\\
Until the spell ends, you control any open water within your chosen area up to a 100-foot cube. When you cast this spell you can choose any of the following effects. As an action during your round, you can repeat the same effect or choose a different one.\\

- \emph{Flooding}. Cause the level of all water in the area to rise up to 20 feet. If the area includes a coast, the water floods the land. If you choose an area within a large body of water, you instead create a 20-foot-high wave that travels from one side of the area to the other before breaking. Any Huge or smaller vehicle in the wave's path is transported to the other side. Any Huge or smaller vehicle hit by water has a 25\% rollover rate.\\
The water level remains high until the spell ends or you choose a different effect. If this effect produced a wave, the wave repeats at the start of your next round, as long as the flooding effect lasts.\\
- \emph{Splitting the Waters}. Cause the water in the area to move to the side to create a gap. The rift extends across the spell's area, and the divided water forms a wall on both sides of the rift. The rift remains until the spell ends or you choose a different effect. The water will then slowly return to fill the gap over the next round, until it has risen to its normal level.\\
- \emph{Redirect the Flow}. Cause the flowing water in the area to move in a direction of your choosing, even if the water has to go over obstacles, up walls, or in other unlikely directions. The water in the area moves as you direct it, but once it reaches beyond the area of ​​the spell, it resumes its flow based on ground conditions. The water continues to move in the direction you choose until the spell ends or you choose a different effect.\\
- \emph{Turbines}. This effect requires a body of water that covers a 50-foot square and has a depth of 25 feet. Cause a whirlwind to form in the center of the area. The whirlwind produces a vortex 1 meter wide at the base, up to 15 meters wide at the top and 7 meters high. Any creature or object in the water within 25 feet of the whirlpool is pulled 10 feet toward it. A creature can swim away from the vortex by making a Swim check against the spell's saving throw DC.

When a creature enters the vortex for the first time during a round or begins its round there, it must make a Fortitude saving throw. On a failed save, the creature takes 2d8 bludgeoning damage and is caught in the vortex until the spell ends. On a successful save, the creature takes half as much damage, and is not caught in the vortex. A creature caught in the whirlpool can use 3 Actions to try to swim away from the whirlpool as described above, but has -1d6 on Dexterity (Athletics) checks to do so. The first time during each round that an object enters the vortex, the object takes 2d8 bludgeoning damage; this damage is taken each round the object remains in the vortex.

\medskip\textbf{Control Weather}\index[Spells]{Control Weather}\\
\textbf{School}: Water, Air\\
\textbf{Level}: 8, Very Rare\\
\textbf{Launch Time}: 10 minutes\\
\textbf{Range}: Personal (1.5 kilometer radius)\\
\textbf{Components}: V, S, M (burnt incense and some earth and wood mixed in water)\\
\textbf{Duration}: Concentration, maximum 8 hours \\
For the duration, take control of the weather within 7.5 kilometers of you. To cast this spell you must be outside. Moving to a place where you don't have an open view of the sky ends the spell early. When you cast this spell, it changes the current weather conditions, determined by the Storyteller based on the season and latitude. You can change precipitation, temperature and wind. It takes 1d4 x 10 minutes for the new condition to take effect. Once the condition takes effect, you can change it again. When the spell ends, the weather will gradually return to normal.\\
When you change the weather conditions, find the current condition on the following table and change it one stage, up or down. When you change the wind, you can also change its direction.

\medskip

\emph{Precipitation}

- 1 Clear

- 2 Some clouds

- 3 Overcast or mist on the ground

- 4 Rain, hail or snow

- 5 Torrential rain, heavy hailstorm, blizzard\\

\emph{Temperature}

- 1 Unbearable heat

- 2 Hot

- 3 Lukewarm

- 4 Fresh

- 5 Cold

- 6 Polar cold\\

\emph{Wind}

- 1 Calm

- 2 Moderate wind

- 3 Moderate wind

- 4 Fortunate

- 5 Storm\\

\textbf{For each Magical Critical Success obtained} in the Magic Test the duration increases by 8 hours.

\medskip\textbf{Dust}\index[Spells]{Dust}\\
\textbf{School}: Enchantment\\
\textbf{Level}: 5, Rare\\
\textbf{Launch Time}: 1 minute\\
\textbf{Range}: 18 metres\\
\textbf{Components}: V\\
\textbf{Duration}: 30 days\\
You impose a magical command on a creature within range that you can see, forcing it to perform a certain task or forbidding it from carrying out an action or course of activity of your choosing. If the creature can understand you, it must succeed on a Will save or be charmed by you for the duration. While the creature is charmed by you, it takes 3d10 points of damage whenever it acts directly contrary to your instructions, but no more than once per day. A creature that cannot understand you ignores the effects of this spell. You can issue any command of your choice, except an activity that would result in certain death. Should you utter a suicidal command, the spell will end.\\
You can end the spell using an action. remove curse, greater restoration, or wish also ends it.\\
\textbf{If you get at least two Criticals} in the Magic Test the duration is 1 year. If you roll 3 Criticals the spell lasts until ended by one of the spells mentioned above.

\medskip\textbf{Create Food and Water}\index[Spells]{Create Food and Water}\\
\textbf{School}: Summon\\
\textbf{Level}: 3, Municipality\\
\textbf{Launch Time}: 2 Shares\\
\textbf{Range}: 9 metres\\
\textbf{Components}: V, S\\
\textbf{Duration}: Instant\\
You create food and water in range containers, enough to sustain up to five humanoids or 2 mounts for 24 hours. The food is bland but nutritious and will rot after 24 hours if not consumed, as will the water.\\
\textbf{For each Magical Critical Success obtained} in the Magic Test you create food for 3 more people or 1 mount.

\medskip\textbf{Create Beer}\index[Spells]{Create Beer}\\
\textbf{School}: Summon\\
\textbf{Level}: 0, Rare\\
\textbf{Cast Time}: 2 Actions or more\\
\textbf{Range}: 9 metres\\
\textbf{Components}: V, S, M (brewer's yeast, malt, water)\\
\textbf{Duration}: 1 hour\\
Create a mug of beer, 0.5 liters. The quality and type of beer depends on the yeast, malt and water used.
The longer the casting time of the spell, the higher the alcohol content, with a casting time of two actions the alcohol content is 4.3, if 1 action is used the beer generated is non-alcoholic, each action spent increases the alcohol content by 0.3 vol up to a maximum of 12.5 vol.
After an hour the beer vanishes, when consumed after an hour any alcoholic effects of the beer on the people who drank it also end.\\
\textbf{For each Magical Critical Success obtained} in the Magic Test you increase the duration by one liter or one hour.

\medskip\textbf{Create or Destroy Water}\index[Spells]{Create or Destroy Water}\\
\textbf{School}: Water\\
\textbf{Level}: 1, Municipality\\
\textbf{Launch Time}: 2 Shares\\
\textbf{Range}: 9 metres\\
\textbf{Components}: V, S, M (a drop of water to create water or a few grains of salt to destroy it)\\
\textbf{Duration}: Instant\\
You make or break water.\\
\emph{Create Water}. Create up to 40 liters of clear water from your hands that spray up to 30 feet. Alternatively, the water falls as rain into a 30-foot cube within range, extinguishing exposed flames in the area.\\
The spell cannot be used on magical flames.\\
\emph{Destroy Water}. Destroy up to 40 liters of water in an open throw container. Alternatively, you can destroy the fog in a 30-foot cube within range. When used on a water elemental the spell deals 4d6 points of damage with a Fortitude saving throw for half.\\
\textbf{For each Magical Critical Success obtained} in the Magic Test you create or destroy an additional 40 liters of water, or the dimensions of the cube increase by 1 meter of edge in case of fog.\\
The water is drinkable and quenches thirst if drunk within one round of creation.

\medskip\textbf{Create Undead}\index[Spells]{Create Undead}\\
\textbf{School}: Necromancy\\
\textbf{Level}: 6, Uncommon\\
\textbf{Launch Time}: 2 Shares\\
\textbf{Range}: 3 metres\\
\textbf{Components}: V, S, M (a clay pot full of graveyard soil, a clay pot full of brackish water, and a black onyx worth 50 gp for each corpse)\\
\textbf{Duration}: Instant\\
You can only cast this spell at night. Choose up to three Medium or Small humanoid corpses within range. Each corpse becomes a ghoul under your control (the Storyteller has the game statistics of these creatures). During your round, with two Actions, you can mentally command any creature you animate with this spell, if the creature is within 120 feet of you (if you control multiple creatures, you can command all or just one of them at the same time by imparting the same command). You decide what action the creature will take and where it will move during its next round, or you can issue a generic command, such as to guard a specific room or corridor. If you don't issue commands, creatures will simply defend themselves from hostile creatures. Once a command is received, the creature will continue to follow it until the task is complete. The creature is under your control for 24 hours, after which it will stop responding to commands you give it. To maintain control of the creature for another 24 hours, you must cast this spell on the creature before the current 24-hour period ends. This use of the spell reasserts your control over up to three creatures you animated with this spell, rather than animating new ones.\\
\textbf{If you roll a Critical} in the Magic Test you can revive or reassert control over four ghouls. With two Crits you can animate or reassert control over five
ghoul or two ghasts or wights. With three Criticals you can animate or reassert control over six ghouls, three ghasts or wights, or two mummies.

\medskip\textbf{Creation}\index[Spells]{Creation}\\
\textbf{School}: Illusion\\
\textbf{Level}: 5, Rare\\
\textbf{Launch Time}: 1 minute\\
\textbf{Range}: 9 metres\\
\textbf{Components}: V, S, M (a tiny piece of material of the same type of object you intend to create) \\
\textbf{Duration}: Special\\
You grab chunks of shadow matter from the plane of Shadow to create, within range, nonliving objects of plant matter: soft goods, rope, wood, or the like. You can also use this spell to create mineral objects such as stone, crystal, or metal. The created object cannot be larger than a 1 meter cube, and the object must be of a shape and material that you have seen before.\\
The durability depends on the material of the object. If the object is made of multiple materials, use the shortest duration.
\medskip
Material Table - Duration
\medskip

\begin{tabularx}{0.45\textwidth}{lX}
\hline
Plant matter &1 day\\
Stone or crystal &12 hours\\
Precious metals &1 hour\\
Gems &10 minutes\\
Adamantium or mithral &1 minute\\
\end{tabularx}
\medskip

Using any material created by this spell as a material component of another spell will cause the new spell to fail.\\
\textbf{For each Magical Critical Success obtained} in the Magic Test the cube increases by 1 meter of edge.

\medskip\textbf{Growth of Spikes}\index[Spells]{Growth of Spikes}\\
\textbf{School}: Animals and Plants\\
\textbf{Level}: 2, Municipality\\
\textbf{Launch Time}: 2 Actions\\
\textbf{Range}: 45 metres\\
\textbf{Components}: V, S, M (seven sharp thorns or seven twigs, each of them pointed at one end)\\
\textbf{Duration}: 10 minutes\\
The terrain in a 20-foot radius centered on a point within range twists and generates very sharp spikes and spines. For the duration, the area becomes difficult terrain. When a creature enters or moves within the area, it takes 2d4 points of damage for every 3 feet it travels.
The transformation of the terrain is so well disguised that it seems natural. Any creature that did not see the area when the spell was cast must make an Awareness check against the spell's saving throw DC to recognize the danger posed by the terrain before entering it.

\medskip\textbf{Plant Growth}\index[Spells]{Plant Growth}\\
\textbf{School}: Animals and Plants\\
\textbf{Level}: 3, Uncommon\\
\textbf{Cast Time}: 2 Actions or 8 hours\\
\textbf{Range}: 45 metres\\
\textbf{Components}: V, S\\
\textbf{Duration}: Instant\\
This spell channels vitality into plants within a specific area. There are two possible uses for this spell, conferring immediate or long-term benefits. If you cast this spell taking 1 action, choose a point within range. All normal plants within a 100-foot radius centered on that point become thick and bushy. A creature that passes through the area quadruples its movement cost.\\
You can exclude from its effects one or more areas of any size within the spell's area.\\
If you cast this spell over the course of 8 hours, you nourish the land. All plants within a 750 meter radius centered on a point within range become super productive for 1 year. Vegetables produce double the normal amount of food at harvest.\\
\textbf{If you get two Magical Critical Successes} you get the effects of the 8 hours of casting even if the spell was cast with 2 Actions.

\medskip\textbf{Invisible Chef}\index[Spells]{Invisible Chef}\\
\textbf{School}: Summon\\
\textbf{Level}: 1, Municipality\\
\textbf{Launch Time}: 2 Shares\\
\textbf{Range}: 18 metres\\
\textbf{Components}: V, S, M (a wooden ladle and a few drops of olive oil, the food you want cooked)\\
\textbf{Duration}: 2 hours\\
This spell creates an almost invisible force only bounded by a light aura (of the color of your choice) capable and competent in cooking. Together with the cook there is also a set of pots and pans as well as crockery and a small camp stove.\\
Based on the ingredients available or edible vegetables within a radius of 100 meters (the chef does not go hunting) the chef will cook the best of the ingredients, preparing excellent food for up to 4 people. The spell does not create food or water, this must be available when the spell is cast. \\
Once the ingredients are available within two hours, the invisible chef will prepare the food. It is also possible to rush the execution but at the expense of quality.\\
None of the pans, dishes or fires may be used except by the invisible cook.\\
\textbf{If he gets two Magical Critical Successes} the Cook is summoned with food needed to feed 2 people

\medskip\textbf{Cure Light Wounds}\index[Spells]{Cure Light Wounds}\\
\textbf{School}: Water, Care\\
\textbf{Level}: 1, Municipality\\
\textbf{Launch Time}: 2 Shares\\
\textbf{Range}: Contact\\
\textbf{Components}: V, S\\
\textbf{Duration}: Instant\\
Your hand fills with positive healing energy, a creature you touch regains a number of hit points equal to 1d8 + spell ability modifier. This spell when used on an undead, attack roll with touch spell, damages it by the same amount.\\
Unless otherwise stated, this spell cannot be used on animals or plants.\\
\textbf{For each Magical Critical Success obtained} in the Magic Test you heal 1d6 more Hit Points.\\
If the spellcaster and the healed creature are both Followers of the same Patron, the spell heals 1d8 more.\\
If the spellcaster and the healed creature are both Devotees of the same Patron, any value on the die equal to 1,2,3 will be considered a 4.

\medskip\textbf{Cure Wounds Series}\index[Spells]{Cure Wounds Series}\\
\textbf{School}: Care\\
\textbf{Level}: 3, Uncommon\\
\textbf{Launch Time}: 2 Shares\\
\textbf{Range}: Contact\\
\textbf{Components}: V, S\\
\textbf{Duration}: Instant\\
Your hand fills with positive healing energy, a creature you touch regains a number of hit points equal to 3d8 + 2*spell ability modifier. This spell when used on an undead, attack roll with a touch spell, damages it by the same amount.\\
Unless otherwise stated, this spell cannot be used on animals or plants.\\
\textbf{For each Magical Critical Success obtained} in the Magic Test you heal 1d6 more Hit Points.\\
If the spellcaster and the healed creature are both Followers of the same Patron, the spell heals 1d8 more.\\
If the spellcaster and the healed creature are both Devotees of the same Patron, any value on the die equal to 1,2,3 will be considered a 4.

\medskip\textbf{Cure Critical Wounds}\index[Spells]{Cure Critical Wounds}\\
\textbf{School}: Care\\
\textbf{Level}: 5, Uncommon\\
\textbf{Launch Time}: 2 Shares\\
\textbf{Range}: Contact\\
\textbf{Components}: V, S\\
\textbf{Duration}: Instant\\
Your hand fills with positive healing energy, a creature you touch regains a number of hit points equal to 5d8 + 3*spell ability modifier. This spell when used on an undead, attack roll with touch spell, damages it by the same amount.\\
Unless otherwise stated, this spell cannot be used on animals or plants.\\
\textbf{For each Magical Critical Success obtained} in the Magic Test you heal 1d6 more Hit Points.\\
If the spellcaster and the healed creature are both Followers of the same Patron, the spell heals 1d8 more.\\
If the spellcaster and the healed creature are both Devotees of the same Patron, any value on the die equal to 1,2,3 will be considered a 4.

\medskip\textbf{Cure Mass Wounds}\index[Spells]{Cure Mass Wounds}\\
\textbf{School}: Care\\
\textbf{Rarity}: Uncommon\\
Like Cure Wounds but heal up to 4 creatures, within a 30-foot radius.\\
You use three more Magic Points than the selected Cure Wounds.\\
\textbf{For each Magical Critical Success obtained} in the check you heal one more creature.\\
If the spellcaster and the healed creature are both Followers of the same Patron, the spell heals 1d8 more.\\
Unless otherwise stated, this spell cannot be used on animals or plants.\\
If the spellcaster and the healed creature are both Devotees of the same Patron, any value on the die equal to 1,2,3 will be considered a 4.

\medskip\textbf{Fire Bolt}\index[Spells]{Fire Bolt}\\
\textbf{School}: Fire\\
\textbf{Level}: 1, Municipality\\
\textbf{Launch Time}: 2 Shares\\
\textbf{Range}: 36 metres\\
\textbf{Components}: V, S\\
\textbf{Duration}: Instant\\
You hurl a fiery spark at a creature or object within range. Make a ranged spell attack against the target. On a hit, the target takes 1d10 fire damage. A flammable object affected by this spell catches fire if it is not being worn or carried.\\
The damage of the spell increases by 1d8 when you reach CM 5, CM 11 and CM 17 but it costs 2 Actions to cast it enhanced and 2 Magic Points, it is also necessary to have taken Adept of Magic in this Magic List a number of times equal to the enhancements that they want to apply.\\
\textbf{For every two Magical Critical Success obtained} in the Magic Test you cast an additional spark.

\medskip\textbf{Tracking Bolt}\index[Spells]{Tracing Bolt}\\
\textbf{School}: Invocation\\
\textbf{Level}: 1, Uncommon\\
\textbf{Launch Time}: 2 Shares\\
\textbf{Range}: 36 metres\\
\textbf{Components}: V, S\\
\textbf{Duration}: 1 round\\
A flash of light travels toward a creature within range, chosen by you. Make a ranged spell attack against the target. If you hit, the target takes 4d6 Light damage, and your next attack roll made against it before the end of your next round has +1d6 to its TC, thanks to the mystical dim light that will continue to shine around the target until then. \\
\textbf{For each Magical Critical Success obtained} in the Magic Test the damage increases by 1d6.

\medskip\textbf{Irresistible Dance}\index[Spells]{Irresistible Dance}\\
\textbf{School}: Enchantment\\
\textbf{Level}: 8, Legendary\\
\textbf{Launch Time}: 2 Shares\\
\textbf{Range}: 9 metres\\
\textbf{Components}: V\\
\textbf{Duration}: 1 minute\\
Choose a creature within range and that you can see. The target begins a comical dance on the spot: flailing its legs, stomping its feet and hopping for the duration. Creatures that cannot be charmed are immune to this spell.\\
A dancing creature must use 2 move actions to dance without leaving its space and has -1d6 on Reflex saving throws and attack rolls. While the target is subject to this spell, other creatures have +1d6 on attack rolls against it. By spending 2 Actions the dancing creature can make a new Will save to regain control of itself. If he succeeds, the spell ends. While dancing he considers himself distracted\\
\textbf{If you get 2 Magical Critical Successes} the duration increases by 1 hour

\medskip\textbf{Arcane Bolt}\index[Spells]{Arcane Bolt}\\
\textbf{School}: Universal\\
\textbf{Level}: 1, Municipality\\
\textbf{Launch Time}: 2 Shares\\
\textbf{Range}: 36 metres\\
\textbf{Components}: V, S\\
\textbf{Duration}: 1 Turn, Concentration\\
You create a glowing bolt of magical force. Throwing one or more already summoned darts costs 1 Action. The bolt hits a creature you can see within range, chosen by you. A bolt deals 1d4 + 1 force damage to its target, and you can direct it to hit one or more creatures.\\
You create an additional bolt when you reach CM 3, CM 5, CM 7, and CM 9. The damage increases by 2 each time you take Adept of Magic on the Universal List up to a maximum of 5 increases.\\
\textbf{For each Magical Critical Success obtained} in the Magic Test the spell creates an additional bolt.

\medskip\textbf{Occult Bolt}\index[Spells]{Cantrip - Occult Bolt}\\
\textbf{School}: Invocation\\
\textbf{Level}: 1, Municipality\\
\textbf{Cast Time}: 1 Action\\
\textbf{Range}: 36 metres\\
\textbf{Components}: V, S\\
\textbf{Duration}: Instant\\
A beam of crackling energy hurtles toward a creature within range. Make a ranged spell attack against the target. On a hit, the target takes 1d8 force damage.\\
The damage of the spell increases by 1d8 when you reach CM 5, CM 11 and CM 17 but it costs 2 Actions to cast it enhanced and 2 Magic Points, it is also necessary to have taken Adept of Magic in this Magic List a number of times equal to the enhancements that they want to apply.\\
\textbf{Every 2 Magical Critical Success obtained} in the Magic Test you create another beam of energy.

\medskip\textbf{Wish}\index[Spells]{Wish}\\
\textbf{School}: Summon\\
\textbf{Level}: 9, Legendary\\
\textbf{Launch Time}: 2 Shares\\
\textbf{Range}: Personal\\
\textbf{Components}: V,S,M (gems ​​for 20000 gp)\\
\textbf{Duration}: Instant\\
Wish is the most powerful spell a mortal creature can cast. By simply speaking out loud and consuming the gems held in your hand, you can alter the very foundations of reality to suit your needs. \\
The basic use of this spell is to reproduce the effect of any other spell of level 8 or lower. You do not have to meet any of the spell's requirements, including expensive material components. The spell simply takes effect.\\
Alternatively, you can create one of the following effects of your choice:\\
- You create an item worth up to 25,000 gp that is not a magic item. The object cannot be larger than 300 feet in any dimension, and appears in an unoccupied space on the terrain.\\
- You allow up to twenty creatures you can see to regain all their Hit Points, and end all effects on them described by the greater restoration spell.\\
- Grant resistance to a damage type of your choice to up to ten creatures you can see.\\
- Give up to ten creatures you can see immunity to a single spell or other magical effect for 8 hours. For example, you could make you and all of your companions immune to the lich's life-drain attack.\\
- You undo any recent event by forcing any rolls made in the last round (including your last round) to be rerolled. Reality reshapes itself to accommodate the new result. You can make the new roll have +2d6 or -2d6, you can choose whether to use the original roll or the new roll. You may also be able to achieve more than the goals in the examples above.\\

\medskip
Define your desires as much as possible to the Storyteller. The Narrator has great latitude in deciding what happens in these cases; the greater the desire, the greater the chance that something will go wrong. The spell may simply fail, the desired effect may only partially manifest, or you may suffer unexpected consequences, depending on how you made the wish. The stress of casting this spell to create any effect other than reproducing another spell weakens you.\\
After you withstand the stress, each time you cast a spell, until you finish a night's rest, you will take 1d10 Void damage per level/2 of the spell. This damage cannot be reduced or diminished in any way. Additionally, your Constitution drops to -3, if not already -3 or lower, for 2d4 days.\\
For each day you spend resting and doing nothing but light activity, your remaining recovery time decreases by 2 days.
Roll 1d100, if you roll 1 to 33\% you will never be able to cast wish again due to the stress suffered, 34\%-66\% you age 5 years, 67\%-99\% no effect happens particular, 100\% you immediately recover the stress of the launch.\\
\textbf{In case of 2 magical Critical Successes obtained} you do not suffer side effects from the casting of Wish.

\medskip\textbf{Phantom Steed}\index[Spells]{Phantom Steed}\\
\textbf{School}: Illusion\\
\textbf{Level}: 3, Municipality\\
\textbf{Launch Time}: 1 minute\\
\textbf{Range}: 9 metres\\
\textbf{Components}: V, S\\
\textbf{Duration}: 1 hour\\
A Large, quasi-real horse-like creature appears on the field in an unoccupied space of your choice within range. You decide the appearance of the creature, and it appears equipped with saddle, bit and bridle. Any equipment created by the spell vanishes in a cloud of smoke if it is brought more than 10 feet away from the steed. For the duration, you or a creature of your choice can ride the steed. The creature uses racehorse statistics, except that it has a speed of 100 feet and can travel 9 miles in an hour, or 10 miles at a fast pace. When the spell ends, the steed gradually begins to vanish, giving the rider 1 minute to dismount. The spell ends if you use an action to end it or if the steed takes damage.\\
\textbf{For each Magical Critical Success obtained} in the Magic Test the duration increases by one hour or you create an additional mount.

\medskip\textbf{Floating Disk}\index[Spells]{Floating Disk}\\
\textbf{School}: Summon\\
\textbf{Level}: 1, Municipality\\
\textbf{Launch Time}: 2 Shares\\
\textbf{Range}: 9 metres\\
\textbf{Components}: V, S, M (a drop of mercury)\\
\textbf{Duration}: 1 hour\\
This spell creates a slightly concave, perfectly circular, horizontal plane of force 3 feet in diameter and 1 inch thick that floats 3 feet above the ground in an unoccupied space of your choice within range that you can see. The disc remains active for the duration, and can support 250 pounds. If more weight is placed on it, the spell ends and everything on it falls to the ground. As long as you are within 20 feet of it, the disk is immobile. If you move more than 6 meters away from it, the disc follows you so that it always remains 6 meters away from you. It can move across uneven terrain, up and down stairs, slopes, and the like, but cannot overcome changes in altitude of 10 feet or more. For example, the disk cannot cross a 10-foot-deep moat, nor could it leave the moat if it were created at the bottom of it. The disk can be grabbed by the caster and moved manually. If you move more than 100 feet away from the disk (usually because it cannot get around an obstacle in following you) the spell ends.\\
\textbf{For each Magical Critical Success obtained} in the Magic Test the duration doubles.

\medskip\textbf{Disintegration}\index[Spells]{Disintegration}\\
\textbf{School}: Transmutation\\
\textbf{Level}: 6, Uncommon\\
\textbf{Launch Time}: 2 Shares\\
\textbf{Range}: 18 metres\\
\textbf{Components}: V, S, M (a magnet and a pinch of powder)\\
\textbf{Duration}: Instant\\
A thin green beam shoots from your pointing finger at a target that is within range and that you can see. The target can be a creature, an object, or a creation of magical force, such as a wall created by wall of force. A creature targeted by this spell must make a Fortitude saving throw. The target takes 10d6 + 40 force damage on a failed save, half as much damage on a successful one. If this damage reduces the target to 0 hit points, it is disintegrated. A disintegrated creature and everything it is wearing and carrying, except magical items, are reduced to a pile of fine gray dust. The creature can only be brought back to life through the intervention of a Patron\\
This spell automatically disintegrates nonmagical objects or a creation of magical force that is Large or smaller. If the target is a nonmagical object or creation of Huge or larger strength, this spell disintegrates a portion of it equal to a 10-foot cube. Magical items ignore this spell.\\
\textbf{For each Magical Critical Success obtained} in the Magic Test the damage increases by 4d6.\\
\textbf{Saving Throw Success/Critical Failure}: In case of a critical failure the damage is doubled, in case of a critical success the damage is further halved

\medskip\textbf{Dispel Good and Evil}\index[Spells]{Dispel Good and Evil}\\
\textbf{School}: Abjuration\\
\textbf{Level}: 5, Rare\\
\textbf{Launch Time}: 2 Shares\\
\textbf{Range}: Personal\\
\textbf{Components}: V, S, M (Holy water or silver and iron powder)\\
\textbf{Duration}: Concentration, 1 minute \\
A bright energy surrounds you and protects you from fey, undead, and creatures from places beyond the Material Plane. For the duration, celestials, elementals, fey, demons, and undead have -1d6 on attack rolls against you. You can end the spell early using one of the following special functions.\\
\emph{Break Enchantment}. As an action, you can make contact with a creature charmed, frightened, or possessed by a celestial, elemental, fey, demon, or undead. The creature you are in contact with is no longer fascinated, frightened, or possessed by these creatures.\\
\emph{Leave}. As an action, make a melee attack against a celestial, elemental, fey, demon, or undead within your reach. If you hit it, you can attempt to send the creature back to its home plane. The creature must succeed on a Will save or be sent back to its home plane (if it is not already there). If not on their home plane, undead are sent back to the Shadow World and fey to the First World.

\medskip\textbf{Dispel Spells}\index[Spells]{Dispel Spells}\hypertarget{Dispel Spells}{}\\
\textbf{School}: Abjuration\\
\textbf{Level}: 3, Municipality\\
\textbf{Launch Time}: 2 Shares\\
\textbf{Range}: 36 metres\\
\textbf{Components}: V, S\\
\textbf{Duration}: Instant\\
Choose a creature, object, or magical effect within range. Any spell of level 3 or lower on the target ends. Each magical critical success on a spell increases its level by one. If cast on an object that manifests a spell, it is deactivated for 10 minutes.\\
\textbf{For each Magical Critical Success obtained} in the Magic Test the dispelable level increases by 1. In case of 3 critical successes an effect can be permanently dispelled on a non-artifact object.

\medskip\textbf{Advanced dispel magic}\index[Spells]{Advanced dispel magic}\hypertarget{advanced dispel magic}{}\\
\textbf{School}: Abjuration\\
\textbf{Level}: 5, Rare\\
\textbf{Launch Time}: 3 Actions\\
\textbf{Range}: 36 metres\\
\textbf{Components}: V, S, M (diamond dust worth 200 gp)\\
\textbf{Duration}: Instant\\
Choose a creature, object, or magical effect within range. Any spell of level 5 or lower on the target ends. Each magical critical success on a spell increases its level by one. If cast on an object that manifests a spell, it is deactivated for 10 minutes.\\
\textbf{For each Magical Critical Success obtained} in the Magic Test the dispelable level increases by 1.

\medskip\textbf{Destroy undead}\index[Spells]{Destroy undead}\\
\textbf{School}: Care\\
\textbf{Level}: 3, Uncommon\\
\textbf{Launch Time}: 2 Shares\\
\textbf{Range}: 36 metres\\
\textbf{Components}: V, S, M (a relic of a Devotee of Thaft or Sumkjt)\\
\textbf{Duration}: Instant\\
Choose an undead within 120 feet. A ray of light shoots out from your hand, enveloping the creature. The undead makes a Fortitude save to halve 4d12 points of positive energy damage. \\
\textbf{For each Magical Critical Success obtained} in the Magic Test the damage increases by 1d12.

\medskip\textbf{Finger}\index[Spells]{Finger}\\
\textbf{School}: Enchantment\\
\textbf{Level}: 0, Rare\\
\textbf{Cast Time}: 1 Immediate Action\\
\textbf{Range}: 18 metres\\
\textbf{Components}: Y\\
\textbf{Duration}: 3 rounds\\
Give the finger (or raspberry or umbrella gesture) to the opponent who must be able to see it\\
This must make a Will saving throw, if it succeeds nothing happens.
If he fails the saving throw by 5 or more he is humiliated, for the next 2 rounds he has a penalty of 2 on attack rolls, saving throws and on competence checks.\\
If he fails his saving throw by 3 or 4, he is mortified and has a 2 penalty on attack and proficiency rolls until the end of the next round.\\
If he fails the save by 2 or 1, he is punished, until the end of the next round he has a penalty of 2 on attack or defense rolls (choice of target).\\
\textbf{For each Magical Critical Success obtained} in the Magic Test you can influence another creature that can see the gesture.

\medskip\textbf{Finger of Death}\index[Spells]{Finger of Death}\\
\textbf{School}: Necromancy\\
\textbf{Level}: 6, Municipality\\
\textbf{Launch Time}: 2 Shares\\
\textbf{Range}: 18 metres\\
\textbf{Components}: V, S\\
\textbf{Duration}: Instant\\
You send a blast of negative energy to a creature within range and that you can see, causing it profound pain. The target must make a Fortitude saving throw. The target takes 7d8 + 30 Void damage on a failed save, or half as much damage on a successful one.\\
A humanoid killed by this spell reanimates as a zombie under your permanent command at the start of your next round, and will follow your verbal commands to the best of its ability.
\textbf{Saving Throw Success/Critical Failure}: In case of a critical failure the damage is doubled, in case of a critical success the damage is further halved

\medskip\textbf{Divination}\index[Spells]{Divination}\\
\textbf{School}: Divination\\
\textbf{Level}: 6, Rare\\
\textbf{Launch Time}: 2 Shares\\
\textbf{Range}: Personal\\
\textbf{Components}: V, S, M (incense and a sacrificial offering appropriate to your religion, whose total value is 25 gp, which will be consumed by the spell)\\
\textbf{Duration}: Instant\\
Your magic and a votive offering connect you with a Patron or a Patron's servant. You can ask him a single question about a specific goal, event or activity that needs to happen within 7 days. The Narrator gives a truthful answer. The reply could be a short phrase, a cryptic rhyme, or an omen. \\
The spell does not take into account any possible circumstances that could change the outcome, such as the casting of additional spells or the loss or arrival of an ally.\\
If you cast the spell two or more times before finishing the long day, there is a cumulative 25\% chance that for each casting after the first you will get an erroneous reading. The Storyteller makes this roll in secret.

\medskip\textbf{Dominate Beasts}\index[Spells]{Dominate Beasts}\\
\textbf{School}: Enchantment, Animals and Plants\\
\textbf{Level}: 4, Very Rare - Common\\
\textbf{Launch Time}: 2 Shares\\
\textbf{Range}: 18 metres\\
\textbf{Components}: V, S\\
\textbf{Duration}: Concentration, maximum 1 minute\\
You try to charm a beast within range that you can see. It must succeed on a Will save or be charmed for the duration, receiving +1d6 to the roll if you or your allies are fighting it.\\
While the beast is charmed, as long as the two of you are on the same plane of existence you maintain a telepathic link with it. You can use this telepathic link to send commands to the creature while you are conscious (requires 1 action), which it will obey as best it can. You can specify a simple, generic course of action, such as \emph{Attack that creature}, \emph{Run over there}, or \emph{Take that item}. If the creature completes the command and receives no further direction from you, it will defend and preserve itself to the best of its ability.\\
You can spend 2 of your actions to take full and precise control of the target. Until the end of your next round, the target will only take actions you decide, and will not do anything that you don't allow it to do. During this time, you can also have the target use a Reaction Action, but this requires the use of your reaction.\\
Each time the target takes damage, it makes a new Will saving throw against the spell. If the saving throw succeeds, the spell ends.\\
\textbf{For each Magical Critical Success obtained} in the Magic Test the duration doubles up to a maximum of 8 hours.

\medskip\textbf{Dominate Monsters}\index[Spells]{Dominate Monsters}\\
\textbf{School}: Enchantment\\
\textbf{Level}: 8, Uncommon\\
\textbf{Launch Time}: 2 Shares\\
\textbf{Range}: 18 metres\\
\textbf{Components}: V, S\\
\textbf{Duration}: Concentration, maximum 1 hour\\
You try to charm a creature within range that you can see. It must succeed on a Will save or be charmed for the duration, receiving +1d6 to the roll if you or your allies are fighting it.\\
While the creature is charmed, as long as the two of you are on the same plane of existence you maintain a telepathic link with it. You can use this telepathic link to send commands to the creature while you are conscious (requires 1 action), which it will obey as best it can. You can specify a simple, generic course of action, such as \emph{Attack that creature}, \emph{Run over there}, or \emph{Take that item}. If the creature completes the command and receives no further direction from you, it will defend and preserve itself to the best of its ability.\\
You can spend two of your Actions to take full and precise control of the target. Until the end of your next round the creature will only carry out actions decided by you, and will not do anything that you do not allow it to do. During this time, you can also have the creature use a Reaction Action, but this requires the use of your reaction. Each time the target takes damage, it makes a new Will saving throw against the spell. If the saving throw succeeds, the spell ends.\\
\textbf{For each Magical Critical Success obtained} in the Magic Test the duration doubles up to a maximum of 8 hours.

\medskip\textbf{Dominate People}\index[Spells]{Dominate People}\\
\textbf{School}: Enchantment\\
\textbf{Level}: 5, Uncommon\\
\textbf{Launch Time}: 2 Shares\\
\textbf{Range}: 18 metres\\
\textbf{Components}: V, S\\
\textbf{Duration}: Concentration, maximum 1 minute\\
You try to charm a humanoid within range that you can see. It must succeed on a Will save or be charmed for the duration, receiving +1d6 to the roll if you or your allies are fighting it.\\
While the target is charmed, as long as the two of you are on the same plane of existence you maintain a telepathic link with it. You can use this telepathic link to send commands to the target while you are conscious (requires 1 action), which it will obey as best it can. You can specify a simple, generic course of action, such as \emph{Attack that creature}, \emph{Run over there}, or \emph{Take that item}. If the target completes the order and receives no further guidance from you, he will defend himself to the best of his ability.
You can spend 2 Actions to take full and precise control of the target. Until the end of your next round, the target will only take actions you decide, and it won't do anything you don't allow it to do. During this time, you can also have the target use a Reaction Action, but this requires the use of your reaction. Each time the target takes damage, it makes a new Will saving throw against the spell. If the saving throw succeeds, the spell ends.\\
\textbf{For each Magical Critical Success obtained} in the Magic Test the duration doubles up to a maximum of 8 hours.

\medskip\textbf{Heroism}\index[Spells]{Heroism}\\
\textbf{School}: Enchantment\\
\textbf{Level}: 1, Uncommon\\
\textbf{Launch Time}: 2 Shares\\
\textbf{Range}: Contact\\
\textbf{Components}: V, S\\
\textbf{Duration}: 1 minute\\
A willing creature you are in contact with is infused with courage. Until the spell ends, the creature is immune to being frightened and, at the start of each of its rounds, it gains temporary hit points equal to your Intelligence value or spell modifier. When the spell ends, the target loses all remaining temporary Hit Points from this spell.\\
\textbf{For each Magical Critical Success obtained} in the Magic Test you can influence another creature.

\medskip\textbf{Exile}\index[Spells]{Exile}\\
\textbf{School}: Abjuration\\
\textbf{Level}: 4, Municipality\\
\textbf{Launch Time}: 2 Shares\\
\textbf{Range}: 18 metres\\
\textbf{Components}: V, S, M (an object despised by the target)\\
\textbf{Duration}: 1 minute\\
You try to send a creature within range and that you can see into another plane of existence. The target must succeed on a Will save or be exiled. If the target is native to the plane of existence you are on, you exile the target to a harmless demiplane. While there, the target is incapacitated. The target remains there until the spell ends, when it reappears in the space it left or in the nearest unoccupied space if its original space is now occupied. If the target is native to a different plane of existence than the one you are on, the target vanishes with a soft bang, returning to its home plane. If the spell ends before 1 minute has passed, the target reappears in the space it left or in the nearest unoccupied space if its original space is occupied.\\
\textbf{For each Magical Critical Success achieved} in the Magic Test you can affect another creature, or the creature is banished for a week.

\medskip\textbf{Sun Blast}\index[Spells]{Sun Blast}\\
\textbf{School}: Invocation\\
\textbf{Level}: 8, Rare\\
\textbf{Launch Time}: 2 Shares\\
\textbf{Range}: 45 metres\\
\textbf{Components}: V, S, M (fire and a piece of sunstone)\\
\textbf{Duration}: Instant\\
Intense sunlight illuminates a 60-foot radius centered on a point within range, chosen by you. All creatures within the light must make a Fortitude saving throw. On a failed save, a creature takes 12d6 Light damage and is blinded for 1 minute. If she succeeds, she takes half damage and is not blinded by the spell. Undead and oozes have -2d6 to this saving throw. A creature blinded by this spell makes another Fortitude saving throw at the end of each of its rounds. If she succeeds on the saving throw, she is no longer blinded.
In its area, this spell dispels any darkness cast by a spell.\\
\textbf{For each Magical Critical Success obtained} in the Magic Test the damage increases by 6d6.

\medskip\textbf{Enraptured}\index[Spells]{Enraptured}\\
\textbf{School}: Enchantment\\
\textbf{Level}: 2, Municipality\\
\textbf{Launch Time}: 2 Shares\\
\textbf{Range}: Personal\\
\textbf{Components}: V, S\\
\textbf{Duration}: 1 minute\\
You weave a series of misleading words, causing creatures of your choice within range who can see and hear you to make a Will saving throw. Any creature that cannot be charmed succeeds on the saving throw automatically, and if you or your companions are fighting a creature, it has a +1d6 to the saving throw. On a failed save, the target has -1d6 on Awareness checks made to sense any creature other than you until the spell ends or the target can no longer hear you.
The spell ends if you are incapacitated or can no longer speak.

\medskip\textbf{Summon Animals}\index[Spells]{Summon Animals}\\
\textbf{School}: Animals and Plants\\
\textbf{Level}: 3, Uncommon\\
\textbf{Launching Time}: 3 Actions\\
\textbf{Range}: 18 metres\\
\textbf{Components}: V, S\\
\textbf{Duration}: 1 Turn, Concentration\\
You summon magical spirits that take on the appearance of beasts and appear in unoccupied spaces within range that you can see. Choose one of the following options to determine what appears:\\

- A beast of challenge rating 2 or lower\\
- Two beasts of challenge rating 1 or lower\\
- Four beasts of challenge rating 1/2 or lower\\
- Eight beasts of challenge rating 1/4 or lower\\

Each beast is also considered magical and disappears when it drops to 0 hit points or when the spell ends. \\
The summoned creatures are friendly towards you and your companions. .\\
\textbf{For each Magic Critical Success obtained} in the Magic Test, two more beasts of a lower rank or 1 more beast of a higher rank than the one initially chosen will appear in the Magic Test.

\medskip\textbf{Summon Mount}\index[Spells]{Summon Mount}\\
\textbf{School}: Animals and Plants\\
\textbf{Level}: 2, Municipality\\
\textbf{Launch Time}: 10 minutes\\
\textbf{Range}: 9 metres\\
\textbf{Components}: V, S\\
\textbf{Duration}: 1 hour\\
You summon a spirit that takes the form of an unusually intelligent, strong, and loyal mount, forming a lasting bond with it. Appearing in an unoccupied space within range, the steed assumes the form of your choice, such as a warhorse, pony, camel, moose, or mastiff (the Storyteller may give you the ability to summon steeds as well). other types of animals). The steed has the statistics of its chosen form, though it is a celestial, fey, or demon type (your choice) instead of its normal type. Additionally, if your steed has Intelligence -3 or less, its Intelligence becomes -2, and it gains the ability to understand one language of your choice from those you can speak. Your steed serves as your mount, both in and out of combat, and you have an instinctive bond with it, allowing you to fight as one.\\
When the steed drops to 0 hit points, it disappears, leaving no physical form behind. you can dismiss the steed at any time with an action, making it disappear. In either case, casting this spell again summons the same steed, restored to full hit points.\\
You cannot have more than one steed bound by this spell at a time. As an action, you can free the steed from this bond at any time, causing it to disappear.\\
\textbf{For each Magical Critical Success obtained} in the Magic Test the spell lasts 1 hour longer.

\medskip\textbf{Summon Elemental}\index[Spells]{Summon Elemental}\\
\textbf{School}: Air, Water, Earth, Fire\\
\textbf{Level}: 5, Rare\\
\textbf{Launch Time}: 1 minute\\
\textbf{Range}: 27 metres\\
\textbf{Components}: V, S, M (burnt incense for air, malleable clay for earth, sulfur and phosphorus for fire, or water and sand for water) \\
\textbf{Duration}: 1 Turn, Concentration\\
You summon an elemental minion. Choose an area within range that is composed of water, air, fire, or earth and fills a 10-foot cube. An elemental of challenge rating 5 or lower appropriate for your chosen area appears in an unoccupied space within 10 feet of it. The elemental disappears when it drops to 0 hit points or the spell ends.
Each Magic List can only summon its own specific Elemental\\
\textbf{For two Magic Critical Success obtained} in the Magic Test the challenge rating of the summoned elemental increases by 1

\medskip\textbf{Summon Minor Elementals}\index[Spells]{Summon Minor Elementals}\\
\textbf{School}: Air, Water, Earth, Fire\\
\textbf{Level}: 4, Uncommon\\
\textbf{Launch Time}: 1 minute\\
\textbf{Range}: 27 metres\\
\textbf{Components}: V, S\\
\textbf{Duration}: 1 Turn, Concentration\\
You summon elementals that will appear in unoccupied spaces within range and that you can see. Choose one of the following options to decide what appears:\\

- An elemental of challenge rating 2 or lower\\
- Two elementals of challenge rating 1 or lower\\
- Four elementals of challenge rating 1/2 or less\\
- Eight elementals of challenge rating 1/4 or less\\

\medskip
A summoned elemental disappears when it drops to 0 hit points or the spell ends.\\
Each Magic List can only summon its own specific Elemental\\
\textbf{For each Magical Critical Success obtained} in the Magic Test, two more elementals of a lower rank or 1 more elemental of a higher rank than the one initially chosen will appear in the Magic Test.

\medskip\textbf{Instant Summons}\index[Spells]{Instant Summons}\\
\textbf{School}: Summon\\
\textbf{Level}: 6, Rare\\
\textbf{Launch Time}: 1 minute\\
\textbf{Range}: Contact\\
\textbf{Components}: V, S, M (a sapphire worth 1000 gp)\\
\textbf{Duration}: Until dissolved \\
You come into contact with an object weighing 5 kilos or less and whose largest dimension does not exceed 180 centimeters. The spell leaves a mark on the surface of the object and invisibly engraves its name on the sapphire used as a material component. Each time you cast this spell, you must use a different sapphire.\\
At any time thereafter, you can use 2 Actions to speak the item's name and shatter the sapphire. The object instantly appears in your hand regardless of the physical or planar distance between you, and the spell ends.\\
If another creature is holding or carrying the item, shattering the sapphire will not transport the item to you, but instead you will learn who the creature is in possession of and approximately where it is currently located.\\
Dispel magic, or a similar effect successfully applied to the sapphire, ends the spell's effect.

\medskip\textbf{Craft}\index[Spells]{Craft}\\
\textbf{School}: Transmutation\\
\textbf{Level}: 4, Municipality\\
\textbf{Launch Time}: 10 minutes\\
\textbf{Range}: 36 metres\\
\textbf{Components}: V, S\\
\textbf{Duration}: Instant\\
Convert raw materials into finished products of the same material. For example, you can make a small wooden bridge from a pile of trees, a rope from a pile of hemp, and clothes from flax or wool. Choose raw materials that you can see within range. You can craft one Large or smaller item (contained in a 10-foot cube, or eight connected 3-foot cubes) given enough raw materials. If you are working with metal, stone, or other mineral substances, the crafted object cannot be larger than Medium (contained in a single 3-foot cube). The quality of the items created by this spell is commensurate with the quality of the raw materials.\\
You cannot create or transmute magical creatures or objects with this spell. You also cannot use it to craft items that normally require a high level of craftsmanship, such as jewelry, weapons, glass, or armor, unless you have proficiency with the type of craftsman's tools used to craft these items. In case of a critical test in the Magic Test, more volumes can be processed or produced with greater quality.

\medskip\textbf{Fatal}\index[Spells]{Fatal}\\
\textbf{School}: Illusion\\
\textbf{Level}: 9, Rare\\
\textbf{Launch Time}: 2 Shares\\
\textbf{Range}: 36 metres\\
\textbf{Components}: V, S\\
\textbf{Duration}: Concentration, maximum 1 minute\\
By tapping into the innermost fears of a group of creatures, you create illusory creatures in their minds, visible only to them. Each creature in a 30-foot-radius sphere centered on a point of your choice within range must make a Will saving throw. On a failed save, the creature becomes frightened for the duration. The illusion sinks into the creature's most intimate fears, manifesting its worst nightmares as an implacable threat. At the end of each of the frightened creature's rounds, it must succeed on a Will save or take 4d10 points of damage. If the saving throw succeeds, the spell ends for that creature.

\medskip\textbf{Divine Favor}\index[Spells]{Divine Favor}\\
\textbf{School}: Invocation\\
\textbf{Level}: 1, Uncommon\\
\textbf{Cast Time}: 1 Immediate Action\\
\textbf{Range}: Personal\\
\textbf{Components}: V, S\\
\textbf{Duration}: 1 minute\\
Your prayers empower you and your weapon. Until the spell ends, when it hits, your weapon deals an additional 1d4 Light damage.\\
\textbf{For each Magical Critical Success obtained} in the Magic Test your weapon causes +1 additional Light damage.

\medskip\textbf{Wound}\index[Spells]{Wound}\\
\textbf{School}: Necromancy\\
\textbf{Level}: 6, Uncommon\\
\textbf{Launch Time}: 2 Shares\\
\textbf{Range}: 18 metres\\
\textbf{Components}: V, S\\
\textbf{Duration}: Instant\\
You unleash a virulent disease on a creature you can see within range. The target must make a Fortitude saving throw. The target takes 14d6 void damage on a failed save, or half as much damage on a successful one. If the target fails the saving throw, its maximum hit points are reduced for 1 hour by an amount equal to the void damage taken. Any effect that removes a disease allows the character's maximum Hit Points to return to normal before that time elapses.

\medskip\textbf{Stop Time}\index[Spells]{Stop Time}\\
\textbf{School}: Transmutation\\
\textbf{Level}: 9, Very Rare\\
\textbf{Launching Time}: 2 Actions\\
\textbf{Range}: Personal\\
\textbf{Components}: V\\
\textbf{Duration}: Instant\\
You briefly stop the flow of time for everyone except you. Time doesn't pass for other creatures, while you make 1d4 + 1 rounds in a row, during which you can take actions and move as usual. This spell ends if any of the actions you use during this period, or any effects you create during this period, affect a creature other than you or an object worn or carried by someone other than you. Additionally, the spell ends if you move to a location more than 300 meters away from where you cast it.\\
\textbf{For each Magical Critical Success obtained} in the Magic Test the duration increases by 1 round. If you have two Critical Magical Successes you can exclude another creature from stopping time.

\medskip\textbf{Everlasting Flame}\index[Spells]{Everlasting Flame}\\
\textbf{School}: Universal\\
\textbf{Level}: 2, Legendary\\
\textbf{Launch Time}: 2 Shares\\
\textbf{Range}: Contact\\
\textbf{Components}: V, S, M (ruby dust worth 75 gp, which the spell consumes)\\
\textbf{Duration}: 1 day\\
A flashlight-like glow emanates from an object you are in contact with. The effect appears to be that of a normal flame, but it does not produce heat or require oxygen. An everlasting flame can be hidden or hidden but cannot be dampened or extinguished.

\medskip\textbf{Holy Flame}\index[Spells]{Cantrip - Holy Flame}\\
\textbf{School}: Universal\\
\textbf{Level}: 0, Municipality\\
\textbf{Cast Time}: 1 Action\\
\textbf{Range}: 18 metres\\
\textbf{Components}: V, S\\
\textbf{Duration}: Instant\\
A torch-like glow descends upon a creature within range that you can see. The target must succeed on a Reflex save or take 1d8 Light damage. The target does not receive the benefit of cover on this saving throw.\\
The damage of the spell increases by 1d8 when the sum of the Traits in common with the Patron reaches 5, 11 and 17, but it costs 2 Actions to cast it enhanced and 2 Magic Points.\\
\textbf{For every two Magical Critical Success obtained} in the Magic Test, an extra flame descends and must hit a different target within range.

\medskip\textbf{Acid Blast}\index[Spells]{Cantrip - Acid Blast}\\
\textbf{School}: Summon\\
\textbf{Level}: 0, Municipality\\
\textbf{Cast Time}: 1 Action\\
\textbf{Range}: 18 metres\\
\textbf{Components}: V, S\\
\textbf{Duration}: Instant\\
Hurl a bubble of acid. Choose one creature within range or two creatures within range that are within 3 feet of each other. The target must succeed on a Reflex save or take 1d6 acid damage.\\
The damage of the spell increases by 1d8 when you reach CM 5, CM 11 and CM 17, but it costs 2 Actions to cast it enhanced and 2 Magic Points, it is also necessary to have taken Adept of Magic in this Magic List a number of times equal to the enhancements that you want to apply.\\
\textbf{For every two Magical Critical Success obtained} in the Magic Test you throw one more acid bubble within range.

\medskip\textbf{Gust of Wind}\index[Spells]{Gust of Wind}\\
\textbf{School}: Air\\
\textbf{Level}: 2, Municipality\\
\textbf{Launch Time}: 2 Shares\\
\textbf{Range}: Personal (18 meter line)\\
\textbf{Components}: V, S, M (a legume seed)\\
\textbf{Duration}: Concentration, maximum 1 minute\\
A line of strong wind 60 feet long and 10 feet wide explodes away from you in a direction of your choice for the duration of the spell. Any creature that begins its round inside the line must succeed on a Fortitude save or be pushed 10 feet away from you, following the direction of the line.\\
Any creature on the line must spend double movement to approach you.\\
The gust disperses gases or vapors, extinguishes candles, torches and similar unprotected flames in the area. Protected flames, such as those from lanterns, flail, and have a 50\% chance of dying out. As 1 Action during each of your rounds, before the spell ends, you can change the direction in which the line projects away from you.\\
A missile weapon that passes through a gust of wind has a 50% miss rate.

\medskip\textbf{Ethereal Form}\index[Spells]{Ethereal Form}\\
\textbf{School}: Transmutation\\
\textbf{Level}: 7, Rare\\
\textbf{Launch Time}: 2 Shares\\
\textbf{Range}: Personal\\
\textbf{Components}: V, S\\
\textbf{Duration}: Maximum 8 hours\\
You enter the border regions of the Ethereal Plane, the area that overlaps with your current plane. You remain on the Ethereal Edge for the duration or until you use an action to end the spell. If you move up or down, the movement cost is doubled, but if you move horizontally the movement cost is doubled per move action. You can see and hear the plane you came from, but everything there appears gray to you, and you can't see more than 60 feet away.\\
While you are on the Ethereal Plane, you can only interact with other creatures on that plane. Creatures that are not on the Ethereal Plane cannot perceive or interact with you, unless a special ability or magic gives them the ability to do so.\\
You ignore all objects and effects that are not on the Ethereal Plane, thus being able to pass through objects you perceive on the plane you came from. When the spell ends, you immediately return to the plane you came from at the spot you currently occupy. If you occupy the same space as a solid object or creature when this happens, you are immediately moved to the nearest unoccupied space you can occupy and take 6 force damage for every foot (or fraction thereof) you are moved. This spell has no effect if you cast it while you are already in the Ethereal Plane or on a plane that does not border it, such as one of the Outer Planes.\\
\textbf{For each Magical Critical Success obtained} in the Magic Test you can bring another creature with you.

\medskip\textbf{Gasous Form}\index[Spells]{Gasous Form}\\
\textbf{School}: Transmutation\\
\textbf{Level}: 3, Uncommon\\
\textbf{Launch Time}: 2 Shares\\
\textbf{Range}: Contact\\
\textbf{Components}: V, S, M (a piece of gauze and a wisp of smoke)\\
\textbf{Duration}: Concentration, maximum 1 hour\\
You transform a willing creature, along with whatever it is wearing and carrying, into a vaporous cloud for the duration. The spell ends if the creature drops to 0 hit points. Incorporeal creatures ignore this effect. While in this form, the target's only method of movement is a flight speed of 10 feet. The target can enter and occupy another creature's space. The target has resistance to nonmagical damage, and has +1d6 on Fortitude and Reflex saving throws. The target can pass through small holes, narrow passages, and even simple holes, although it treats liquids as solid surfaces. The target cannot fall and remains floating in the air even if stunned or otherwise incapacitated.\\
While in the form of a vaporous cloud, the target cannot speak or manipulate objects, and any objects it is wearing or carrying cannot be thrown, used, or otherwise employed. The target can't attack or cast spells.\\
\textbf{For every two Magical Critical Successes obtained} in the Magic Test you can influence another creature.


\medskip\textbf{Animal Shapes}\index[Spells]{Animal Shapes}\\
\textbf{School}: Animals and Plants\\
\textbf{Level}: 8, Rare\\
\textbf{Launch Time}: 2 Shares\\
\textbf{Range}: 9 metres\\
\textbf{Components}: V, S\\
\textbf{Duration}: 24 hours\\
You magically transform other creatures into beasts. Choose any number of willing creatures within range and that you can see. You transform each target into the form of a Large or smaller beast with a challenge rating of 4 or lower. On subsequent rounds, you can use 2 Actions to transform subject creatures into new forms.\\
The transformation lasts for each target for the duration of the spell, or until that target drops to 0 hit points or dies. You can choose a different shape for each target. The target's game statistics are replaced by the statistics of the chosen beast, with the exception of the Intelligence, Wisdom and Charisma scores and traits which remain those of the target.
target. The target takes on the hit points of its new form, and when it returns to its normal form, it returns to the number of hit points it had before transforming. If it transforms again because it has dropped to 0 hit points, the excess damage is applied to its original form. As long as the excess damage doesn't reduce the creature's normal form to 0 hit points, it isn't unconscious. The creature is limited in the actions it can perform by the nature of its new form, and cannot speak or cast spells.\\
The target's equipment merges into the new form. The target cannot activate, wield, or otherwise benefit from its equipment.

\medskip\textbf{Crush}\index[Spells]{Crush}\\
\textbf{School}: Invocation\\
\textbf{Level}: 2, Municipality\\
\textbf{Launch Time}: 2 Shares\\
\textbf{Range}: 18 metres\\
\textbf{Components}: V, S, M (a metal fragment)\\
\textbf{Duration}: Instant\\
A loud, very intense rumble erupts from a point within range of your choice. Each creature in a 10-foot-radius sphere centered on that point must make a Fortitude saving throw. A creature takes 3d8 sonic damage on a failed save, or half as much damage on a successful one. A creature made of inorganic material, such as stone, crystal, or metal, has -1d6 on its saving throw. A nonmagical object that is not worn or carried also takes damage if it is in the spell's area.\\
\textbf{For each Magical Critical Success obtained} in the Magic Test the damage increases by 1d8.\\
\textbf{Saving Throw Success/Critical Failure}: In case of a critical failure the damage is doubled, in case of a critical success the damage is further halved

\medskip\textbf{Acid Arrow}\index[Spells]{Acid Arrow}\\
\textbf{School}: Water, Earth\\
\textbf{Level}: 2, Municipality\\
\textbf{Launch Time}: 2 Shares\\
\textbf{Range}: 27 metres\\
\textbf{Components}: V, S, M (a powdered rhubarb leaf and a python stomach)\\
\textbf{Duration}: Instant\\
A glowing green arrow hurtles towards a target within range and explodes with a spray of acid. Make a ranged spell attack against the target. On a hit, the target takes 4d4 acid damage immediately and 2d4 acid damage at the end of its next round. If you miss, the arrow sprays the target with acid, dealing half initial damage and dealing no damage at the end of the target's next round.\\
\textbf{For each Magic Critical Success obtained} in the Magic Test the damage increases by 2d4.

\medskip\textbf{Lightning Bolt}\index[Spells]{Lightning Bolt}\\
\textbf{School}: Air\\
\textbf{Level}: 3, Municipality\\
\textbf{Launch Time}: 2 Shares\\
\textbf{Range}: Personal (30 meter line)\\
\textbf{Components}: V, S, M (a piece of fur and a rod of amber, crystal or glass)\\
\textbf{Duration}: Instant\\
You explode a bolt of lightning that forms a line 100 feet long and 3 feet wide from where you are in a direction you choose. Each creature in the line must succeed at a Reflex saving throw. The creature takes 8d6 lightning damage on a failed save, or half as much damage on a successful one.\\
Lightning ignites flammable objects in the area that are not being worn or carried.\\
If the lightning is thrown against hard worked stone it bounces with an exit angle equal to the entry angle (\textbackslash|/) (180-entry angle). Lightning thrown into water creates a 10-foot radius sphere of electricity where it enters.\\
\textbf{For each Magical Critical Success obtained} in the Magic Test the damage increases by 2d6.\\
\textbf{Saving Throw Success/Critical Failure}: In case of a critical failure the damage is doubled, in case of a critical success the damage is further halved

\medskip\textbf{Chained Lightning}\index[Spells]{Chained Lightning}\\
\textbf{School}: Air\\
\textbf{Level}: 6, Rare\\
\textbf{Launch Time}: 2 Shares\\
\textbf{Range}: 45 metres\\
\textbf{Components}: V, S, M (some fur; a piece of amber, glass or crystal rod; and three silver pins)\\
\textbf{Duration}: Instant\\
You create a bolt of electricity that strikes a target you can see within range, chosen by you. From this a further bolt is generated which hits the nearest target within 6 metres. The process continues until 7 targets have been hit or there are no new ranged opponents left. A target can be a creature or object of at least Medium size and can be the target of a single bolt. A target must make a Reflex saving throw. The target takes 8d6 lightning damage on a failed save, or half as much damage on a successful one.\\
\textbf{For each Magical Critical Success obtained} in the Magic Test the bolt reaches out to a further target.\\
\textbf{Saving Throw Success/Critical Failure}: On a critical failure the damage is doubled, on a critical success the damage is further halved

\medskip\textbf{Mislead}\index[Spells]{Mislead}\\
\textbf{School}: Illusion\\
\textbf{Level}: 5, Uncommon\\
\textbf{Launching Time}: 2 Actions\\
\textbf{Range}: Personal\\
\textbf{Components}: S\\
\textbf{Duration}: 1 hour\\
You become invisible at the same time that an illusory double of you appears where you are. The doppelganger remains for the duration of the spell, but the invisibility ends if you attack or cast a spell. You can use 2 Actions to make the illusory double move up to double your speed and make it gesture, speak and behave in any way you want.\\
You can see through his eyes and hear through his ears as if you were in the space he is in. During each of your rounds, with an Action, you can switch from using his senses to using yours, or vice versa. While you are using his senses, you are blinded and deafened to your surroundings.

\medskip\textbf{Force Cage}\index[Spells]{Force Cage}\\
\textbf{School}: Invocation\\
\textbf{Level}: 6, Rare\\
\textbf{Launch Time}: 2 Shares\\
\textbf{Range}: 30 metres\\
\textbf{Components}: V, S, M (ruby dust worth 1,500 gp)\\
\textbf{Duration}: 1 hour\\
An immobile, invisible cubic prison composed of magical force appears around an area you choose within range. The prison can be a cage or a solid box, your choice. A prison in the form of a cage can be 20 feet on a side and composed of 5-inch bars spaced 5 inches apart, providing complete coverage for the creatures inside. A box-shaped prison can be 10 feet on a side, creating a solid barrier that prevents any matter from passing through it and blocking any spells cast from inside or outside the area. When you cast this spell, any creature that is completely inside the cage is trapped. Creatures only partially in the cage area, or those too large to fit, are pushed away from the center of the area until they are completely outside.\\
A creature inside the cage cannot leave it through nonmagical means. If the creature attempts to use teleportation or interplanar travel to leave the cage, it must first make a Will saving throw. If it succeeds, the creature can use that magic to escape the cage. On a failed save, the creature cannot leave the cage and wastes the use of the spell or effect. The cage also extends into the Ethereal Plane, thus blocking ethereal travel.\\
This spell cannot be dispelled by €7355{dispelmagic}{Dispel Magic} but only with €7356{advanced dispelmagic}{Advanced Dispel Magic}.

\medskip\textbf{Magic Jar}\index[Spells]{Magic Jar}\\
\textbf{School}: Necromancy\\
\textbf{Level}: 6, Very Rare\\
\textbf{Launch Time}: 1 minute\\
\textbf{Range}: Personal\\
\textbf{Components}: V, S, M (a gem, crystal, reliquary, or some other ornamental container worth at least 500 gp)\\
\textbf{Duration}: Until dissolved\\
Your body enters a catatonic state as your soul abandons it and enters the container you use as a material component. While your soul occupies the container, you are aware of your surroundings as if you were in the space of the container. You cannot move or use reactions. The only action you can take is to project your soul up to 30 meters away, out of the container, returning to your living body (and ending the spell) or attempting to possess a humanoid body.\\
You can attempt to possess any humanoid within 100 feet of you that you can see (creatures protected by protection from good and evil or magic circle spells cannot be possessed). The target must make a Will save, and on a failed save, your soul enters the target's body, while the target's soul remains trapped in the container. If you succeed, the target resists your attempts to possess it, and you cannot attempt to possess it again until 24 hours have passed.\\
Once you possess a creature's body, you can control it. Your game statistics are replaced by the creature's statistics, except for your Traits and your Intelligence, Wisdom, and Charisma scores. Maintain the benefits provided by Skills. If the target has any Skills, you can't use any of them.\\
Meanwhile, the possessed creature's soul can sense the container's surroundings using its senses, but cannot move or perform any actions.\\
While in possession of a body, you can use 2 Actions to return from the host body to the container if you are within 100 feet of it, returning the host creature's soul to its body. If the host body dies while you are inside it, the creature dies, and you must make a Will save against your spell save DC. If you succeed, you return to the container, if it is within 30 meters of you. Otherwise, you will die.\\
If the container is destroyed or the spell ends, your soul immediately returns to your body. If your body is more than 100 feet away or if it dies while you try to return to it, your soul will die too. If another creature's soul is in the container when it is destroyed, the creature's soul returns to its body, if the body is alive and within 100 feet, otherwise, the creature dies. When the spell ends, the container is destroyed.

\medskip\textbf{Glyph of Warding}\index[Spells]{Glyph of Warding}\\
\textbf{School}: Abjuration\\
\textbf{Level}: 3, Municipality\\
\textbf{Launch Time}: 2 Shares\\
\textbf{Range}: Contact\\
\textbf{Components}: V. S, M (incense and powdered diamond worth at least 200 gp, which the spell consumes)\\
\textbf{Duration}: Until dispelled or activated \\
When you cast this spell, you inscribe a glyph that harms other creatures on a surface (such as a table or a section of floor or wall) or inside an object that can be closed (such as a book, scroll, or chest). to hide the glyph. If you choose a surface, the glyph can cover a surface area no larger than 10 feet in diameter. If you choose an object, that object must stay in place; if the object is moved more than 10 feet from where the spell was cast, the glyph is broken, and the spell ends without being activated.\\
The glyph is nearly invisible and can be found with an Awareness check against your spells' saving throw DC. You decide what activates the glyph when the spell is cast.\\
For glyphs inscribed on a surface, typical activation includes touching or standing over the glyph, removing another object covering the glyph, moving within a certain distance of the glyph, or manipulating the object on which the glyph is inscribed. glyph. For glyphs inscribed on an object, typical activation includes opening the object, moving within a certain distance of the object, or seeing or reading the glyph. Once the glyph has been activated, the spell ends.\\
You can better define the activation so that the spell activates only under certain circumstances or according to certain physical peculiarities (such as height or weight), species of creature (for example, the ward could act against aberrations or elves dark), or specific Traits. You can also set conditions to prevent the glyph from being triggered, such as saying a password.\\
When inscribing the glyph choose explosive runes or spell glyph.

\medskip

- \emph{Spell Glyph}. You can insert a prepared spell of level 2 or lower into the glyph by casting it as part of creating the glyph. The spell must target a single creature or an area. The spell that is inserted has no immediate effect if cast in this way. When the glyph is activated, the entered spell is cast. If the spell has a target, it targets the creature that activated the glyph. If the spell affects an area, the area is centered on that creature. If the spell summons hostile creatures or creates harmful objects or traps, they appear as close as possible to the intruder and attack. If the spell requires concentration, this is maintained until the end of its normal duration.\\

- \emph{Explosive Runes}. When activated, the glyph belches magical energy in a 20-foot radius sphere centered on the glyph. The sphere propagates around the corners. Each creature in the area must make a Reflex saving throw. A creature takes 5d8 acid, lightning, fire, cold, or sonic damage on a failed save (your choice when creating the glyph), or half as much damage on a successful save.\\
\textbf{For each Magical Critical Success obtained} in the magic check the damage of the explosive rune glyph increases by 1d8.

\medskip\textbf{Orb of Invulnerability}\index[Spells]{Orb of Invulnerability}\\
\textbf{School}: Abjuration\\
\textbf{Level}: 6, Municipality\\
\textbf{Launch Time}: 2 Actions\\
\textbf{Range}: Personal (3 meter radius)\\
\textbf{Components}: V. S, M (a glass or crystal ball that shatters when the spell ends) \\
\textbf{Duration}: Concentration, maximum 1 minute\\
A motionless, faintly shimmering barrier rises in a 10-foot radius around you and remains there for the duration.\\
Any spells of Level 4 (excluding higher results thanks to magic critical) or lower cast from outside the barrier cannot affect creatures or objects inside it. These spells are suppressed if they target creatures and objects within the barrier or affect the area the barrier is on.\\
\textbf{For every two Magic Critical Success obtained} in the Magic Test you can block a higher level of spell.

\medskip\textbf{Kyrin's Acorn Crab}\index[Spells]{Kyrin's Acorn Crab}\\
\textbf{School}: Animals and Plants\\
\textbf{Level}: 2, Uncommon\\
\textbf{Cast Time}: 1 Action\\
\textbf{Range}: 50 metres\\
\textbf{Components}: V, S, M (9 acorns that are consumed, a piece of rubber)\\
\textbf{Duration}: 1 minute for Magical Expertise, Concentration\\
You enchant 9 acorns of magical energy and they begin to swirl 30 centimeters above your shoulder.
Each round, by spending 1 Action, you can throw up to 5 acorns at one or more targets. Make a single attack roll with ranged spells, with a bonus equal to the number of times you took Animal and Plant Lists, per target regardless of how many acorns you throw at it. Each acorn deals 1d4 bludgeoning damage if it hits.\\
\textbf{For each Magical Critical Success obtained} in the Magic Test you can enchant two more acorns.

\medskip\textbf{Kyrin's Crab of Fire Acorns}\index[Spells]{Kyrin's Crab of Fire Acorns}\\
\textbf{School}: Animals and Plants, Fire\\
\textbf{Level}: 3, Rare\\
\textbf{Cast Time}: 2 Action\\
\textbf{Range}: 50 metres\\
\textbf{Components}: V, S, M (9 acorns that are consumed, a piece of rubber)\\
\textbf{Duration}: 1 minute for Magical Expertise, Concentration\\
You enchant 9 acorns of magical energy and they begin to swirl 30 centimeters above your shoulder.
Each round, by spending 1 Action, you can throw up to 5 acorns at one or more targets. Make a single attack roll with ranged spells, with a bonus equal to the number of times you took Animal and Plant Lists or Fire, per target regardless of how many acorns you throw at it. Each acorn deals 1d4 bludgeoning damage + 1d4 fire damage if it hits.\\
\textbf{For each Magical Critical Success obtained} in the Magic Test you can enchant two more acorns.

\medskip\textbf{Kyrin's Lemon Crumble}\index[Spells]{Kyrin's Lemon Crumble}\\
\textbf{School}: Animals and Plants, Earth\\
\textbf{Level}: 3, Rare\\
\textbf{Launching Time}: 2 Actions\\
\textbf{Range}: 30 metres\\
\textbf{Components}: V, S, M (at least 9 drops of lemon, one bottle)\\
\textbf{Duration}: 1 round for Magical Expertise, Concentration\\
Enchant a bottle with at least 9 drops of lemon inside.
Each round, by spending 1 Action, you can spray up to 2 drops of lemon, of the 9 total, against one or more targets within 30 meters. Make a single attack roll with ranged spells, with a bonus equal to the number of times you took Animal and Plant or Earth Lists, per target regardless of how many drops you roll at it. Each drop deals 1d6+1 acid damage if it hits.\\
\textbf{For each Magical Critical Success obtained} in the Magic Test you can create two extra lemon drops.

\medskip\textbf{Kyrin's Chestnut Crab}€7421[Spells]{Kyrin's Chestnut Crab}\\
\textbf{School}: Animals and Plants\\
\textbf{Level}: 5, Very Rare\\
\textbf{Cast Time}: 1 Action\\
\textbf{Range}: 60 metres\\
\textbf{Components}: V, S, M (9 browns that are consumed, a piece of rubber)\\
\textbf{Duration}: 1 minute for Magical Expertise, Concentration\\
You cast 9 browns of magical energy and they begin to swirl 2 feet above your shoulder.
Each round, by spending 1 Action, you can throw up to 5 browns at one or more targets. Make a single attack roll with ranged spells, with a bonus equal to the number of times you took Animal and Plant Lists, per target regardless of how many acorns you throw at it. Each acorn deals 2d8+4 bludgeoning damage if it hits\\
\textbf{For each Magical Critical Success obtained} in the Magic Test you can enchant two more browns.

\medskip\textbf{Cry of pain}\index[Spells]{Cry of pain}\\
\textbf{School}: Necromancy\\
\textbf{Level}: 1, Rare\\
\textbf{Cast Time}: 1 Reaction\\
\textbf{Range}: personal\\
\textbf{Components}: V\\
\textbf{Duration}: Tinstant\\
As a reaction action you let out a cry of pain when hit in melee. The creature that hit you must make a Fortitude save or take 2d4 void damage.\\
\textbf{For each Magical Critical Success obtained} in the Magic Test you cause 1d4 more damage.

\medskip\textbf{Healing}\index[Spells]{Healing}\\
\textbf{School}: Care\\
\textbf{Level}: 6, Rare\\
\textbf{Launching Time}: 2 Actions\\
\textbf{Range}: 18 metres\\
\textbf{Components}: V, S\\
\textbf{Duration}: Instant\\
Choose a creature within range and that you can see. a wave of positive healing energy overwhelms the creature, causing it to regain 70 hit points. The spell also ends any blindness, deafness, and disease (even magical) that afflicts the target. This spell causes 50 hit points of damage to an undead on a touch spell attack roll.\\
\textbf{For each Magical Critical Success obtained} in the Magic Test the amount healed increases by 20.

If the spellcaster and the healed creature are both \textbf{Followers} of the same Patron, the spell heals 90 Hit Points.

If the spellcaster and the healed creature are both \textbf{Devotees} of the same Patron, the spell restores full Hit Points.

\medskip\textbf{Mass Healing}\index[Spells]{Mass Healing}\\
\textbf{School}: Care\\
\textbf{Level}: 9, Legendary\\
\textbf{Launch Time}: 2 Actions\\
\textbf{Range}: 18 metres\\
\textbf{Components}: V, S\\
\textbf{Duration}: Instant\\
A stream of healing energy flows from you to the wounded creatures around you. You restore up to 700 Hit Points, divided as you wish among any creature within range and that you can see (with a maximum of 70 Hit Points per creature). Creatures healed by this spell are also cured of all diseases and any effects that cause them to be blinded or deafened. This spell can inflict up to 120 hit points of damage to an undead. Fortitude save to negate the effect.

If the caster and the healed creature are both \textbf{Followers} of the same Patron, the assigned healing increases by 20\%

If the caster and the healed creature are both \textbf{Devotees} of the same Patron, the assigned healing increases by 50\%

\medskip\textbf{Guide}\index[Spells]{Guide}\\
\textbf{School}: Divination\\
\textbf{Level}: 0, Municipality\\
\textbf{Cast Time}: 1 Reaction\\
\textbf{Range}: 3 metres\\
\textbf{Components}: V, S\\
\textbf{Duration}: 1 Round\\
You cast the spell on contact with a willing creature. Once, before the spell ends, the target can roll a d4 and add the rolled result to a proficiency check of his choice. He can roll the die before or after making the Proficiency check. The spell then ends. You can't cast Guidance on the same creature at intervals less than 1 hour.

\medskip\textbf{Anti-Life Shell}\index[Spells]{Anti-Life Shell}\\
\textbf{School}: Animals and Plants\\
\textbf{Level}: 5, Uncommon\\
\textbf{Launching Time}: 2 Actions\\
\textbf{Range}: Personal (3 meter radius)\\
\textbf{Components}: V, S\\
\textbf{Duration}: maximum 1 hour\\
A barrier of light extends up to a 10-foot radius around you, moving with you and remaining centered on you, keeping creatures other than undead or constructs at bay. The barrier remains for the duration. \\
The barrier prevents a subject creature from passing through it in any way. An affected creature can cast spells or make ranged or reach weapon attacks through the barrier. If you move so that a subject creature is forced through the barrier, the spell ends.

\medskip\textbf{Identify}\index[Spells]{Identify}\hypertarget{spellidentify}{}\\
\textbf{School}: Universal\\
\textbf{Level}: 1, Municipality\\
\textbf{Launch Time}: 1 minute\\
\textbf{Range}: Contact\\
\textbf{Components}: V, S, M (a gem worth at least 10 gp and an owl feather that the spell consumes)\\
\textbf{Duration}: Instant\\
Choose an object that you must remain in contact with throughout the casting of the spell. If it is a magical object or other object imbued with magic, make an Arcana check at DC 30 with a +2d6 bonus. If you succeed, you learn its properties and how to use them and how many charges it has, if any.\\
You learn if any spells are affecting the item and what they are. If the item was created by a spell, you learn which spell created it. If, however, you remain in contact with a creature during the execution, you learn whether any spells are acting on it and what they are.\\
\textbf{Only if you get a Magical Critical Success} you learn if the item is \hyperlink{cursed objectsid}{cursed}.

\medskip\textbf{Minor Illusion}\index[Spells]{Minor Illusion}\\
\textbf{School}: Universal\\
\textbf{Level}: 0, Municipality\\
\textbf{Launch Time}: 2 Shares\\
\textbf{Range}: 9 metres\\
\textbf{Components}: S, M (a piece of fleece)\\
\textbf{Duration}: 1 minute\\
You create an image of an object or a sound within range for the duration of the spell. The illusion ends if you end it with an action or cast this spell again.\\
If you create a sound, its volume can range from that of a whisper to a scream. It can be your voice, someone else's voice, the roar of a lion, the beating of drums, or any other sound you choose. The sound continues unabated for the duration, or you can produce different sounds at different times before the spell ends.\\
If you create an image of an object (such as a chair, a muddy footprint, or a small chest), it can't be larger than a 3-foot cube. The image cannot produce sounds, lights, smells or any other sensory effects. Physical interaction with the object reveals it as an illusion, because things can pass through it.\\
A creature that uses 3 Actions to examine the sound or image can determine that it is an illusion with a successful Intelligence (Investigation) check against your spell's saving throw DC. If a creature recognizes the illusion for what it is, the illusion fades for them.

\medskip\textbf{Programmed Illusion}\index[Spells]{Programmed Illusion}\\
\textbf{School}: Illusion\\
\textbf{Level}: 6, Uncommon\\
\textbf{Launch Time}: 2 Shares\\
\textbf{Range}: 36 metres\\
\textbf{Components}: V, S, M (a piece of fleece and jade dust worth at least 25 gp)\\
\textbf{Duration}: Until dissolved\\
You create, at range, the illusion of an object, creature, or some other visible phenomenon that activates when a specific condition is met. Until then the illusion is imperceptible. It can't be larger than a 30-foot cube, and you decide when you cast the spell, how the illusion behaves, and what sounds it makes. The scheduled performance can last up to 5 minutes. When the conditions you specify are met, the illusion manifests and behaves in the way you describe. Once the illusion has finished its performance, it disappears and remains dormant for 10 minutes. After this period, the illusion can be activated again.\\
The trigger condition can be as general or detailed as you like, although it must be based on visible or audible conditions occurring within 30 feet of the area. For example, you could create an illusion of yourself that appears and warns anyone who tries to open a trapped door, or you could set the illusion to activate only when a creature says the right word or phrase.\\
Physical interaction with the image reveals it as an illusion, as things pass through it. A creature that uses 3 Actions to examine the image can determine that it is an illusion with a successful Intelligence (Investigation) check against the spell's saving throw DC. If a creature recognizes the illusion for what it is, it can see through the image, and any sounds made by the image sound artificial to it.

\medskip\textbf{Major Image}\index[Spells]{Major Image}\\
\textbf{School}: Illusion\\
\textbf{Level}: 3, Municipality\\
\textbf{Launch Time}: 2 Shares\\
\textbf{Range}: 36 metres\\
\textbf{Components}: V, S, M (a piece of fleece)\\
\textbf{Duration}: Concentration, maximum 1 minute for Magical Expertise\\
You create an image of an object, creature, or some other visible phenomenon no larger than a 20-foot cube. The image appears at a point within range that you can see and remains there for the duration of the spell. The image appears completely real, and includes sounds, smells and the temperature appropriate to the thing depicted. You can't generate enough heat or cold to cause damage, nor a sound loud enough to deal sonic damage or deafen a creature, or an odor that would make a creature sick (such as a troglodyte's stench). As long as you remain within range of the illusion, you can use an action to make the image move anywhere else within range.\\
When the image changes position, you can alter its appearance so that its movements appear natural. For example, if you create an image of a creature and move it, you can alter the image to appear to be walking. Likewise, you can employ the illusion to produce different sounds at different times, even making it carry on a conversation.\\
Physical interaction with the image reveals it as an illusion, as things pass through it. A creature that uses 3 Actions to examine the image can determine that it is an illusion with a successful Intelligence (Investigation) check against your spell's saving throw DC. If a creature recognizes the illusion for what it is, the creature can see through it, and for that creature all other sensory qualities vanish.\\
\textbf{If you get a Magical Critical Success} in the Magic Test the spell lasts until it is dispelled, without requiring your concentration.

\medskip\textbf{Projected Image}\index[Spells]{Projected Image}\\
\textbf{School}: Illusion\\
\textbf{Level}: 7, Uncommon\\
\textbf{Launch Time}: 2 Shares\\
\textbf{Range}: 750 kilometres\\
\textbf{Components}: V, S, M (a small reproduction of you made of materials worth at least 5 gp)\\
\textbf{Duration}: 1 day\\
You create an illusory copy of yourself that remains for the duration. The copy can appear anywhere within range that you have already seen, ignoring any obstacles in the way. The illusion reproduces your appearance and sounds but is intangible. If the illusion takes damage, it disappears, and the spell ends.\\
You can use 2 Actions to make this illusion move up to double your speed and make it gesture, speak and behave in any way you want. Imitates your behavior perfectly.\\
You can see through her eyes and hear through her ears as if you were in the space where she is. During each of your rounds, with an Action, you can switch from using his senses to using yours, or vice versa. While you are using her senses, you are blinded and deafened to your surroundings.
Physical interaction with the image reveals it as an illusion, as things pass through it. A creature that uses 3 Actions to examine the image can determine that it is an illusion with a successful Awareness check against the spell's saving throw DC. If a creature recognizes the illusion for what it is, it can see through the image, and any sounds made by the image sound artificial to it.

\medskip\textbf{Silent Image}\index[Spells]{Silent Image}\\
\textbf{School}: Illusion\\
\textbf{Level}: 1, Municipality\\
\textbf{Launching Time}: 2 Actions\\
\textbf{Range}: 36 metres\\
\textbf{Components}: V, S, M (a piece of fleece)\\
\textbf{Duration}: Concentration, maximum 3 minutes for Magical Expertise\\
You create an image of an object, creature, or some other visible phenomenon no larger than a 10-foot cube. The image appears at a point you can see within range and remains for the duration of the spell. The image is purely visual; it is not accompanied by sounds, smells or other sensory effects. You can use an action to make the image move anywhere else within range. When the image changes position, you can alter its appearance so that its movements appear natural. For example, if you create an image of a creature and move it, you can alter the image to appear to be walking.\\
Physical interaction with the image reveals it as an illusion, as things pass through it. A creature that uses 3 Actions to examine the image can determine that it is an illusion with an Awareness check against your spell's saving throw DC. If a creature recognizes the illusion for what it is, the creature can see through it.

\medskip\textbf{Mirror Image}\index[Spells]{Mirror Image}\\
\textbf{School}: Illusion\\
\textbf{Level}: 2, Municipality\\
\textbf{Launch Time}: 2 Shares\\
\textbf{Range}: Personal\\
\textbf{Components}: V, S\\
\textbf{Duration}: 1 minute\\
2d4 illusory duplicates of yourself appear in your space. Until the spell ends, the duplicates move with you and imitate your actions, switching places in a way that makes it impossible to determine which is the real image. You can use 1 Action to dismiss the illusory duplicates.\\
Every time a creature catches you it actually hits an illusory image.
If a creature makes multiple attacks per round, it can dispel an image for each successful attack. If you are hit by an area spell, all images vanish.\\
A creature that cannot see, or relies on senses other than sight (such as blindsight), or that can distinguish illusions as false (such as true seeing), ignores the effects of this spell.\\
\textbf{For each Magical Critical Success obtained} in the Magic Test you create an additional duplicate image up to a total maximum of 8 images.

\medskip\textbf{Imprison}\index[Spells]{Imprison}\\
\textbf{School}: Abjuration\\
\textbf{Level}: 9, Rare\\
\textbf{Launch Time}: 2 Shares\\
\textbf{Range}: 9 metres\\
\textbf{Components}: V, S, M (a depiction on fleece or a statuette engraved with the target's features, and a special component that varies depending on the version of the spell you choose, worth at least 500 gp per Target Wound Die)\\
\textbf{Duration}: Until dissolved\\
You create magical binds to bind a creature within range and that you can see. The target must succeed on a Will save or be bound by the spell; if he succeeds, he is immune to the spell if he casts it again. While under this spell, the creature does not need to breathe, eat, or drink, and it does not age. Divination spells cannot locate or sense the target.\\
When you cast this spell, choose one of the following forms of imprisonment.\\

- \emph{Chaining}. Heavy chains, well welded to the ground, keep the target anchored. The target is restrained until the spell ends, and cannot move or be moved in any way until then. The special component for this version of the spell is a chain of precious metal.\\
- \emph{Minimum Insulation}. The target shrinks to 1 inch in height and is encased in a gem or similar object. Light can pass through the gem normally (allowing the target to see out and other creatures to see in), but nothing else can pass through it, not even via teleportation or planar travel. The gem cannot be cut or shattered while the spell remains in effect. The special component for this version of the spell is a large, transparent gem, such as corundum, diamond, or ruby.\\
- \emph{Confined Prison}. The spell transports the target to a tiny demiplane closed to teleportation and planar travel. The demiplane can be a maze, a cage, a tower, or any other enclosed structure of your choice. The special component for this version of the spell is a miniature representation of the prison made of jade.\\
- \emph{Burial}. The target is buried deep within the earth in a sphere of magical force large enough to contain the target. Nothing can pass through the sphere, nor can any creature teleport or use planar travel to enter or exit it. The special component for this version of the spell is a small mithral sphere.\\
- \emph{Drowsiness}. The target falls asleep and cannot be awakened. The special component for this version of the spell consists of rare soporific herbs.\\

\medskip
\emph{\textbf{End the spell}}. When casting the spell, in any of its versions, you can specify a condition that will end the spell and free the target. The condition can be as specific or elaborate as you wish, but the Storyteller must agree that the condition is reasonable and likely to come true. Conditions can be based on a creature's name, identity, or Patron, but are based on perceivable actions or qualities and not on intangibles such as level, skills, or hit points.\\
A dispel magic spell can end the spell only if cast by a character with at least Magical Proficiency 18, who targets the prison or the material component used to create it.\\
You can use a particular special component to create only one prison at a time. If you cast the spell again using the same component, the target of the spell's first casting is immediately freed from its binding.

\medskip\textbf{Wither}\index[Spells]{Wither}\\
\textbf{School}: Necromancy\\
\textbf{Level}: 4, Uncommon\\
\textbf{Launch Time}: 2 Shares\\
\textbf{Range}: 9 metres\\
\textbf{Components}: V, S\\
\textbf{Duration}: Instant\\
Necromantic energy envelops a creature of your choice within range and that you can see, draining it of sap and vitality. The target must make a Fortitude saving throw. On a failed save, the target takes 8d8 void damage, or half as much damage on a successful save. The spell has no effect on undead or constructs.\\
If the target is a nonmagical plant that is not also a creature, such as a tree or bush, it makes no saving throw and withers and dies instantly.\\
\textbf{For each Magical Critical Success obtained} in the Magic Test the damage increases by 1d8.\\
\textbf{Saving Throw Success/Critical Failure}: In case of a critical failure the damage is doubled, in case of a critical success the damage is further halved

\medskip\textbf{Detect Good and Evil}\index[Spells]{Detect Good and Evil}\\
\textbf{School}: Divination\\
\textbf{Level}: 1, Municipality\\
\textbf{Launch Time}: 2 Shares\\
\textbf{Range}: Personal\\
\textbf{Components}: V, S\\
\textbf{Duration}: 1 round for Magical Proficiency\\
For the duration, you learn whether there is an aberration, celestial, elemental, fey, demon, or undead, within 30 feet of you, and its location. Likewise, you learn whether within 30 feet of you is a place or object that has been magically consecrated or desecrated.\\
the spell can penetrate most barriers, but is blocked by 1 foot of stone, 1 inch of base metal, a thin sheet of lead, or 3 feet of wood or earth.\\
\textbf{For each Magical Critical Success obtained} in the Magic Test the duration doubles.\\
\textbf{Note}: This spell has no effect on creatures that follow Traits. At the Storyteller's discretion it can be used to identify the Patron of a Follower or Devotee.

\medskip\textbf{Detect Magic}\index[Spells]{Detect Magic}\\
\textbf{School}: Universal\\
\textbf{Level}: 1, Municipality\\
\textbf{Launch Time}: 2 Shares\\
\textbf{Range}: Personal\\
\textbf{Components}: V, S\\
\textbf{Duration}: 1d4 +1 round for Magical Expertise\\
For the duration, you sense the presence of magic within 30 feet of you. You can use 1 Action to see a faint aura extend around any visible creature or object in the area that carries magic. With two Actions you also learn the Magic List, if it has one.\\
The spell can penetrate most barriers, but is blocked by 1 foot of stone, 1 inch of base metal, a thin sheet of lead, or 3 feet of wood or earth.\\
\textbf{For each Magical Critical Success obtained} in the Magic Test the duration doubles.

\medskip\textbf{Detect Diseases and Poisons}\index[Spells]{Detect Diseases and Poisons}\\
\textbf{School}: Divination\\
\textbf{Level}: 1, Uncommon\\
\textbf{Launch Time}: 2 Shares\\
\textbf{Range}: Personal\\
\textbf{Components}: V, S, M (a yew leaf)\\
\textbf{Duration}: 1 round for Magical Expertise\\
For the duration, you sense the presence and location of poisons, poisonous creatures, and diseases within 30 feet of you. You can also identify the type of poison, poisonous creature or disease. The spell can penetrate most barriers, but is blocked by 1 foot of stone, 1 inch of base metal, a thin sheet of lead, or 3 feet of wood or earth.\\
\textbf{For each Magical Critical Success obtained} in the Magic Test the duration doubles.

\medskip\textbf{Detect Thoughts}\index[Spells]{Detect Thoughts}\\
\textbf{School}: Divination\\
\textbf{Level}: 2, Rare\\
\textbf{Launch Time}: 2 Shares\\
\textbf{Range}: Personal\\
\textbf{Components}: V, S, M (a piece of copper)\\
\textbf{Duration}: 1 minute\\
For the duration, you can read the thoughts of certain creatures. When you cast this spell and with two more Actions in each subsequent round until the spell ends, you can focus your mind on any creature you can see that is within 30 feet of you. If the creature you chose has an Intelligence score of -3 or less or speaks no language, the creature ignores the effect.\\
Initially, you learn only the creature's surface thoughts: the most recurring ones. As an action, you can either shift your attention to the thoughts of another creature or attempt to probe deeper into the same creature's mind. If you probe deeper, the target must make a Will saving throw. If he fails, you gain insight into his reasoning (if any), his emotional state, and anything prevalent in his thoughts (such as worry, love, or hate). If the save is successful, the spell ends. However, the target knows you are probing its mind, and unless you shift your attention to the mind of another creature, on its round the creature can use its 2 Action to make an opposed Will saving throw. ; if he wins, the spell ends.\\
Questions posed verbally to the target creature, of course, shape its train of thought, so this spell is particularly effective in interrogations.
You can also use this spell to detect the presence of thinking creatures you cannot see. When you cast this spell or with 2 Actions in its duration, you can search for thoughts within 30 feet of you. The spell can penetrate barriers, but is blocked by two feet of stone, two inches of metal other than lead, or a thin sheet of lead. You can't detect a creature with Intelligence -3 or less, or a creature that speaks no language. Once you detect a creature's presence in this way, you can read its thoughts for the duration of the spell as long as it remains within range, as described above, even if you cannot see it.
While you have this spell active for casting other spells you will be Distracted.

\medskip\textbf{Inflict Wounds}\index[Spells]{Inflict Wounds}\\
\textbf{School}: Necromancy\\
\textbf{Level}: 1, Municipality \\
\textbf{Launch Time}: 2 Shares\\
\textbf{Range}: Contact\\
\textbf{Components}: V, S\\
\textbf{Duration}: Instant\\
Make a melee spell attack against a creature within range. On a hit, the target takes 3d10 void damage, Fortitude save for half.\\
\textbf{For each Magic Critical Success obtained} in the Magic Test the damage increases by 1d8.

\medskip\textbf{Enlarge/Reduce}\index[Spells]{Enlarge/Reduce}\\
\textbf{School}: Transmutation\\
\textbf{Level}: 2, Municipality\\
\textbf{Launch Time}: 2 Shares\\
\textbf{Range}: 9 metres\\
\textbf{Components}: V, S, M (a pinch of powdered iron)\\
\textbf{Duration}: 1 minute\\
Cause a creature or object within range that you can see to grow or shrink for the spell's duration. Choose a creature or object that is neither worn nor carried. If the target is unwilling, it can make a Fortitude saving throw; if it succeeds, the spell has no effect. If the target is a creature, everything it is wearing and carrying changes size with it. Any object dropped by a creature affected by this spell immediately returns to its normal size.\\

- \emph{Enlarge}. The target's size doubles in all dimensions, and its weight is multiplied by eight. This growth increases his size by one category: from Medium to Large, for example. If there is not enough room for the target to double its size, the creature or object becomes as large as the available space allows. Until the spell ends, the target has +1d6 on Strength-based actions and Fortitude saving throws. The target's weapons grow to match the new size. While these weapons are enlarged, the target's attacks with them will deal an additional category of damage.\\
- \emph{Reduce}. The target's size is halved in all dimensions, and its weight is reduced to one-eighth. This growth decreases its size by one category: from Medium to Small, for example. Until the spell ends, the target has -1d6 on Strength-based actions and Fortitude saving throws. The target's weapons shrink to match the new size. While these weapons are shrunken, the target's attacks with them will deal one category of damage less (but without reducing the weapon's damage below 1).\\
\textbf{For every two Criticals obtained} in the Magic Test the creature increases by another size, or affects another creature within 6 meters of the first.

\medskip\textbf{Giant Insect}\index[Spells]{Giant Insect}\\
\textbf{School}: Animals and Plants\\
\textbf{Level}: 4, Uncommon\\
\textbf{Launch Time}: 2 Shares\\
\textbf{Range}: 9 metres\\
\textbf{Components}: V, S\\
\textbf{Duration}: 10 minutes\\
For the duration of the spell, you transform up to ten centipedes, three spiders, five wasps, or one scorpion within range into giant versions of their natural form. A centipede becomes a giant centipede, a spider becomes a giant spider, a wasp becomes a giant wasp, and a scorpion becomes a giant scorpion. Each creature obeys your voice commands and, in combat, takes action each round during your round. The Storyteller has the statistics of these creatures, and it will always be The Storyteller who resolves their actions and movements. A creature remains in its giant form for the duration, until it drops to 0 hit points, or until you use an action to end the effect on it.\\
The Storyteller can allow you to choose different targets. For example, if you transform a bee, its giant version might have the same stats as the giant wasp.

\medskip\textbf{Death Ward}\index[Spells]{Death Ward}\\
\textbf{School}: Abjuration\\
\textbf{Level}: 4, Uncommon\\
\textbf{Launch Time}: 2 Shares\\
\textbf{Range}: Contact\\
\textbf{Components}: V, S\\
\textbf{Duration}: 8 hours\\
You cast the spell on contact with a creature. Grant the target protection from death. The first time the target drops to 0 hit points as a result of damage taken, the target instead drops to 1 hit point and the spell ends. If the spell is still active when the target is subjected to an effect that would kill it instantly without dealing damage, that effect is instead negated on the target and the spell ends.\\
\textbf{For every two Magic Critical Success obtained} in the Magic Test the spell protects once more.

\medskip\textbf{Intermittent}\index[Spells]{Intermittent}\\
\textbf{School}: Transmutation\\
\textbf{Level}: 3, Uncommon\\
\textbf{Launch Time}: 2 Shares\\
\textbf{Range}: Personal\\
\textbf{Components}: V, S\\
\textbf{Duration}: 1 round for Magical Proficiency\\
Roll 1d6 at the end of each of your rounds for the duration of this spell. If you roll an odd number, you vanish from your current plane of existence and reappear on the Ethereal Plane (the spell fails and the casting is wasted if you were already on that plane). At the start of your next round, and when the spell ends, if you were on the Ethereal Plane, you return to an unoccupied space of your choice that you can see, within 10 feet of the space you vanished from. If no unoccupied spaces are available within this range, you appear in the nearest unoccupied space (randomly determined if more than one space is available). You can end the spell with an action.\\
While on the Ethereal Plane, you can see and hear the plane you come from, which you perceive in shades of gray, but you still cannot perceive anything more than 60 feet away. You can only interact with creatures that are on the Ethereal Plane. Creatures not there can neither sense nor interact with you unless they have the ability to do so.

\medskip\textbf{Catalm's Slap}\index[Spells]{Catalm's Slap}\\
\textbf{School}: Summon\\
\textbf{Level}: 1, Uncommon\\
\textbf{Casting Time}: 1 Reaction, which you can take in response to damage dealt to you by a creature within 60 feet of you that you can see\\
\textbf{Range}: 18 metres\\
\textbf{Components}: V, S\\
\textbf{Duration}: Instant\\
You point your finger, and the creature that harmed you is momentarily engulfed in fiendish flames. The creature must make a Reflex saving throw. He takes 2d10 fire damage on a failed save, or half as much damage on a successful one.\\
\textbf{For each Magical Critical Success obtained} in the Magic Test the damage increases by 1d6.

\medskip\textbf{Hinder}\index[Spells]{Hinder}\\
\textbf{School}: Animals and Plants\\
\textbf{Level}: 1, Municipality\\
\textbf{Launch Time}: 2 Shares\\
\textbf{Range}: 27 metres\\
\textbf{Components}: V, S\\
\textbf{Duration}: 1 minute\\
Vines and crushing branches sprout from the ground in a 20-foot square from a point within range. For the duration, these plants turn the soil in the area into difficult soil.\\
A creature in the area when you cast this spell must succeed on a Fortitude save or be entangled by these plants until the spell ends. A creature entangled by plants can use two actions to make a new saving throw. If he overcomes it, he is free. When the spell ends, the summoned plants vanish.

\medskip\textbf{Reversal of Gravity}\index[Spells]{Reversal of Gravity}\\
\textbf{School}: Transmutation\\
\textbf{Level}: 7, Rare\\
\textbf{Launch Time}: 2 Shares\\
\textbf{Range}: 30 metres\\
\textbf{Components}: V, S, M (a magnet and a wire)\\
\textbf{Duration}: Concentration, maximum 1 minute
This spell reverses gravity in a cylinder with a radius of 50 feet, 100 feet high, centered at a point within range. When you cast this spell, all creatures and objects that are not somehow anchored to the ground fall upward and reach the top of the area. A creature can attempt a Reflex saving throw to grab a fixed object within reach, to avoid falling this way if it succeeds.\\
If you encounter a solid object (the ceiling) during this fall, the falling objects and creatures impact you as they would during a normal fall. If an object or creature reaches the top of the area without hitting anything, it remains there, swaying slightly, for the duration.\\
At the end of the duration, objects and creatures hit fall back down.

\medskip\textbf{Send}\index[Spells]{Send}\\
\textbf{School}: Invocation\\
\textbf{Level}: 3, Municipality\\
\textbf{Launch Time}: 2 Shares\\
\textbf{Range}: Unlimited\\
\textbf{Components}: V, S, M (a small piece of copper wire)\\
\textbf{Duration}: 1 round\\
You send a short message of 25 words or less to a creature you are familiar with. The creature hears the message in its mind, recognizes you as the sender, and can respond to you in a similar way. The spell allows creatures with an Intelligence score of at least -2 to understand the meaning of your message even if they do not understand your language.\\
You can send the message across any distance and even to other planes of existence, but if the target is on a different plane than you, there is a 5\% chance that the message will not arrive.\\
\textbf{For each Magical Critical Success obtained} in the Magic Test you increase the message by 25 words or the duration by one round.

\medskip\textbf{Invisibility}\index[Spells]{Invisibility}\\
\textbf{School}: Illusion\\
\textbf{Level}: 2, Municipality\\
\textbf{Launch Time}: 2 Shares\\
\textbf{Range}: Contact\\
\textbf{Components}: V, S, M (an eyelash wrapped in gum arabic)\\
\textbf{Duration}: 1 minute for Magical Expertise\\
You cast the spell on contact with a creature. The target becomes invisible until the spell ends. Whatever the target is wearing or carrying becomes invisible as long as it remains on the target. The spell ends for the target that attacks or casts a spell.\\
\textbf{For each Magical Critical Success obtained} in the Magic Test you can choose an additional target creature or double the duration.

\medskip\textbf{Greater Invisibility}\index[Spells]{Greater Invisibility}\\
\textbf{School}: Illusion\\
\textbf{Level}: 4, Uncommon\\
\textbf{Launch Time}: 2 Shares\\
\textbf{Range}: Contact\\
\textbf{Components}: V, S\\
\textbf{Duration}: 1 minute\\
You cast the spell on contact with a creature. The target becomes invisible until the spell ends. Anything worn or carried by the target becomes invisible as long as it remains on the target.\\
Performing spells or attack actions does not cause you to become visible.

\medskip\textbf{Invoke Lightning}\index[Spells]{Invoke Lightning}\\
\textbf{School}: Air\\
\textbf{Level}: 3, Municipality\\
\textbf{Cast Time}: 1 round\\
\textbf{Range}: 36 metres\\
\textbf{Components}: V, S\\
\textbf{Duration}: Concentration, maximum 10 minutes\\
A storm cloud appears in the form of a 10-foot-tall cylinder with a radius of 60 feet, centered on a point you can see, 100 feet above you. The spell automatically fails if you cannot see the point in the air where the storm cloud will appear (for example, if you are in a room that cannot accommodate the cloud). When you cast the spell, choose a point you can see within range. Lightning will strike from the cloud at that point. Each creature within 3 feet of that point must make a Reflex saving throw. A creature takes 3d10 lightning damage on a failed save, or half as much damage on a successful one. During each of your rounds until the spell ends, you can use two Actions to call down another bolt in this way, targeting the same or different point.\\
If you are outside in stormy conditions when you cast this spell, the spell gives you control of the existing storm rather than creating a new one. Under these conditions, the spell's damage increases by 1d10. \\
\textbf{For each Magical Critical Success obtained} in the Magic Test the damage increases by 1d8.

\medskip\textbf{Labyrinth}\index[Spells]{Labyrinth}\\
\textbf{School}: Summon\\
\textbf{Level}: 8, Rare\\
\textbf{Launch Time}: 2 Actions\\
\textbf{Range}: 18 metres\\
\textbf{Components}: V, S\\
\textbf{Duration}: maximum 10 minutes\\
Banish a creature within range and that you can see into a labyrinthine demiplane. The target remains there for the duration of the spell or until it escapes the maze. The target can take 3 Actions to attempt to escape. When he does so, he makes a DC 25 Intelligence check. If he succeeds, he flees, and the spell ends (a minotaur or goristro demon automatically succeeds).\\
When the spell ends, the target reappears in the space it left or, if that space is occupied, in the nearest unoccupied space.\\
\textbf{For each Magical Critical Success obtained} in the Magic Test the duration increases by 10 minutes. With two Magical Critical Successes you can affect another creature.

\medskip\textbf{Laydel's Tear}\index[Spells]{Laydel's Tear}\\
\textbf{School}: Invocation\\
\textbf{Level}: 2, Very Rare/Common\\
\textbf{Cast Time}: 2 Action/1 Action\\
\textbf{Range}: 36 metres\\
\textbf{Components}: V, S, M (an enchanter's tear)\\
\textbf{Duration}: Instant\\
The caster imbues a tear with magic that he throws at the opponent, an attack roll with ranged spells is required.
The creature suffers 1d6+2d6 of damage. To establish the type of damage, consult the table with the values ​​of the first d6 rolled.

\medskip

\begin{tabular}{l|l}
\textbf{1d6}&\textbf{Energy}\\
\hline
1 &Fire\\
2 &Electricity\\
3 &Cold\\
4 &Sound\\
5 &Empty\\
6 &Come on\\
\end{tabular}

\medskip

The damage the target suffers is equal to the Energy type that results from the first d6. If the first die is a 6 and one of the other dice is also a 6 then roll 1d6 again and add to the damage.

For a Devotee of Laydel this spell is Common and has a casting time of 1 Action plus he can continue to roll additional d6s of damage as long as he continues to roll 6s on that die.

\medskip\textbf{Fiery Blade}\index[Spells]{Fiery Blade}\\
\textbf{School}: Fire\\
\textbf{Level}: 2, Municipality\\
\textbf{Cast Time}: 1 Immediate Action\\
\textbf{Range}: Personal\\
\textbf{Components}: V, S, M (a sumac leaf)\\
\textbf{Duration}: Concentration, maximum 10 minutes \\
You create a fiery blade in your hand. The blade is similar in size and shape to a scimitar, and stays put for durability. If you let go of the blade, it disappears, but you can create another one with an Action. You can use 2 Actions to make a melee attack with the flame blade. On a hit, the target takes 3d6 fire damage. The fiery blade casts bright light in a 10-foot radius and dim light for an additional 10 feet.\\
\textbf{For every two Criticals obtained} in the Magic Test the damage increases by 1d6.

\medskip\textbf{Flamethrower}\index[Spells]{Flamethrower}\\
\textbf{School}: Fire\\
\textbf{Level}: 2, Rare\\
\textbf{Launch Time}: 2 Actions\\
\textbf{Range}: Personal\\
\textbf{Components}: V, S, M (a 30 cm iron pipe, some beans)\\
\textbf{Duration}: 1 minute, Concentration\\
A small flame appears at the end of the metal tube you hold in your hand. The flame remains there for the duration of the spell during which you must remain focused and does not harm you or your equipment. The flame produces bright light in a 1 meter radius and dim light in a 1 meter radius. The spell ends if you interrupt it with an action or cast it again.\\
With a ranged spell attack roll and spending 1 action you can extend the flame up to 30 feet to hit a target. If you hit, the target takes 2d6 fire damage, if you hold the target you have a +2 to hit the next round.\\
\textbf{For each Critical obtained} in the Magic Test the damage increases by 1d6.

\medskip\textbf{Telepathic Bond}\index[Spells]{Telepathic Bond}\\
\textbf{School}: Divination\\
\textbf{Level}: 5, Rare\\
\textbf{Launching Time}: 2 Actions\\
\textbf{Range}: 9 metres\\
\textbf{Components}: V, S, M (pieces of eggshells from two different species of creatures)\\
\textbf{Duration}: 1 hour\\
You establish a telepathic link between up to eight willing creatures within range of your choice, psychically linking each creature to the others for the duration of the spell. Creatures with an Intelligence score of -3 or less ignore this spell. Until the spell ends, targets can communicate telepathically through this bond, whether or not they share a common language. Communication is possible at any distance, but cannot extend across different planes of existence.\\
\textbf{For each Magical Critical Success obtained} in the Magic Test the duration increases by 1 hour.

\medskip\hypertarget{slow}{\textbf{slow}}\index[Spells]{slow}\\
\textbf{School}: Transmutation\\
\textbf{Level}: 3, Uncommon\\
\textbf{Launch Time}: 2 Actions\\
\textbf{Range}: 36 metres\\
\textbf{Components}: V, S, M (a drop of molasses) \\
\textbf{Duration}: 1 minute, Concentration\\
You change the flow of time around up to 1d4 creatures of your choice in a 20-foot cube within range. Each target must succeed at a Will save or take one less action per round.\\
\textbf{For each Magical Critical Success obtained} in the Magic Test you can influence one additional creature.\\
\textbf{Critical Failure Saving Throw}: On a critical failure you are slowed by an additional action.

\medskip\textbf{Kyrin Land Read}\index[Spells]{Kyrin Land Read}\index{Echolocation}\\
\textbf{School}: Earth\\
\textbf{Level}: 2, Uncommon\\
\textbf{Cast Time}: 1 Round\\
\textbf{Range}: Personal (30 meter range)\\
\textbf{Components}: V, S\\
\textbf{Duration}: Instant\\
You place your hands on the earth and once you have cast the spell you have a fleeting vision of the environment around you within a spherical radius of 30 metres.
You can sense the position and relative shape of creatures and structures resting on the ground.\\
\textbf{For each Magical Critical Success obtained} in the Magic Test the radius increases by 10 meters.

\medskip\textbf{Levitation}\index[Spells]{Levitation}\\
\textbf{School}: Air\\
\textbf{Level}: 2, Municipality\\
\textbf{Launch Time}: 2 Shares\\
\textbf{Range}: 18 metres\\
\textbf{Components}: V, S, M (either a small leather thong or a piece of gold cord bent into the shape of a cup with a long stem at the end)\\
\textbf{Duration}: 10 minutes \\
A creature or object within range that you can see, chosen by you, rises vertically up to 20 feet and remains suspended for the duration of the spell. The spell can levitate a target weighing up to 500 pounds. An unwilling creature that succeeds on a Fortitude save ignores the effect.\\
The target can only move by pushing or pulling towards a fixed object or surface within reach (for example, a wall or ceiling). During your round you can change the target's altitude up to 20 feet in either direction. If you are the target, you can move up or down as part of your movement. Otherwise you can use 1 Action to move the target, which must remain within the spell's range. When the spell ends, if still in the air, the target floats gently to the ground.\\
While under the influence of this spell you are considered Distracted in spellcasting.\\
\textbf{For each Magical Critical Success obtained} in the Magic Test you can move 1 meter sideways or influence another creature.

\medskip\textbf{Magic Reading}\index[Spells]{Magic Reading}\\
\textbf{School}: Universal\\
\textbf{Level}: 1, Municipality\\
\textbf{Cast Time}: 1 Action\\
\textbf{Range}: Contact\\
\textbf{Components}: V, S, M (a fragment of an enchanted item)\\
\textbf{Duration}: 1 minute, until used\\
You grant the ability to read a scroll or magical writing to a target. For the duration of 1 minute or until used once, whichever comes first, the creature automatically manages to comprehend a magical scroll or cast the contents of the scroll according to the criteria and rules for casting spells from a scroll.
\textbf{For each Magical Critical Success obtained} in the Magic Test you can read or understand one more scroll.

\medskip\textbf{Freedom of Movement}\index[Spells]{Freedom of Movement}\\
\textbf{School}: Abjuration\\
\textbf{Level}: 4, Municipality\\
\textbf{Launch Time}: 2 Shares\\
\textbf{Range}: Contact\\
\textbf{Components}: V, S, M (a strip of leather, wrapped around an arm or similar appendage)\\
\textbf{Duration}: 1 hour\\
You cast the spell on contact with a willing creature. For its duration, the target's movement ignores difficult terrain, and spells or other magical effects cannot reduce its speed or cause the target to be paralyzed or restrained.\\
The target can use two Actions to automatically free itself from any nonmagical restraints, such as handcuffs or a creature that grips it. Finally, being underwater incurs no penalty to the target's movement or attacks.\\
\textbf{For two Magic Critical Success obtained} in the Magic Test you can influence another creature.

\medskip\textbf{Languages}\index[Spells]{Languages}\\
\textbf{School}: Divination\\
\textbf{Level}: 3, Municipality\\
\textbf{Launch Time}: 2 Shares\\
\textbf{Range}: Contact\\
\textbf{Components}: V, M (a small clay model of a ziggurat)\\
\textbf{Duration}: 1 hour\\
This spell gives the creature you were in contact with at the time you cast the spell the ability to understand any spoken language it hears. Additionally, when the target speaks, any creature that knows at least one language and can hear the target understands what it says.\\
\textbf{For each Magical Critical Success obtained} in the Magic Test the duration doubles or you affect another creature.

\medskip\textbf{Locate Animals and Plants}\index[Spells]{Locate Animals and Plants}\\
\textbf{School}: Animals and Plants\\
\textbf{Level}: 2, Uncommon\\
\textbf{Launch Time}: 2 Shares\\
\textbf{Range}: Personal\\
\textbf{Components}: V, S, M (a piece of hound fur) \\
\textbf{Duration}: Instant\\
Describe or name a specific type of beast or plant. By focusing on the voice of nature in your surroundings, you learn the direction and distance to the nearest creature or plant of that species, if there are any within 7.5 kilometers.\\
\textbf{For each Magical Critical Success obtained} in the Magic Test you increase the controlled area by 1 km

\medskip\textbf{Locate Creature}\index[Spells]{Locate Creature}\\
\textbf{School}: Divination\\
\textbf{Level}: 4, Municipality\\
\textbf{Launch Time}: 2 Shares\\
\textbf{Range}: Personal\\
\textbf{Components}: V, S, M (a piece of hound fur)\\
\textbf{Duration}: Concentration, maximum 1 hour\\
Describe or name a creature that is familiar to you. You sense the direction of the creature's location, as long as that creature is within 1,000 feet of you. If the creature moves, you also know the direction of its movement.\\
The spell can locate a specific creature known to you, or the closest creature of a species (such as human or unicorn), provided you have seen such a creature up close (within 30 feet) at least once. If the creature you describe or name has a different form, for example is under the effects of the polymorph spell, this spell will not be able to locate the creature.\\
This spell can't locate a creature if a stream of running water at least 10 feet wide blocks a direct path between you and the creature.\\
\textbf{For each Magical Critical Success obtained} in the Magic Test increases the distance by another 300m.

\medskip\textbf{Locate Object}\index[Spells]{Locate Object}\\
\textbf{School}: Divination\\
\textbf{Level}: 2, Municipality\\
\textbf{Launch Time}: 2 Shares\\
\textbf{Range}: Personal\\
\textbf{Components}: V, S, M (a forked twig)\\
\textbf{Duration}: Concentration, maximum 10 minutes \\
Describe or name an object that is familiar to you. You sense the direction of the object's location, as long as that object is within 300 meters of you. If the object moves, you also know the direction of its movement.\\
The spell can locate a specific object known to you, as long as you have seen it up close (within 30 feet) at least once. Alternatively, the spell can locate the nearest object of a particular type, such as certain types of clothing, jewelry, furniture, tools, or weapons.\\
This spell cannot locate an object if any thickness of lead, even a thin sheet, blocks a direct path between you and the object.\\
\textbf{For each Magical Critical Success obtained} in the Magic Test you double the duration.

\medskip\textbf{Talkativeness}\index[Spells]{Talkativeness}\\
\textbf{School}: Transmutation\\
\textbf{Level}: 8, Rare\\
\textbf{Launch Time}: 2 Shares\\
\textbf{Range}: Personal\\
\textbf{Components}: V\\
\textbf{Duration}: 1 hour\\
Until the spell ends, when you make a Charisma-based check you can replace the number rolled with 15. Furthermore, no matter what you say, the magic or analysis that determines whether you are telling the truth will always indicate that you are being honest.\ \
\textbf{For each Magical Critical Success obtained} in the Magic Test you double the duration.

\medskip\textbf{Light}\index[Spells]{Light}\\
\textbf{School}: Universal\\
\textbf{Level}: 1, Municipality\\
\textbf{Launch Time}: 2 Shares\\
\textbf{Range}: Contact\\
\textbf{Components}: V, M (a firefly or some phosphorescent moss)\\
\textbf{Duration}: 30 minutes of real game time\\
You cast the spell upon contact with an object no larger than 10 feet in any direction. Until the spell ends, the object radiates bright light in a 10-foot radius and dim light for an additional 20 feet. The light can be any color you want. Completely covering the object with something opaque blocks the light. If a target object is held or worn by a hostile creature, that creature must succeed on a Reflex saving throw to avoid the spell. A creature affected by the spell must make a Fortitude save or be blinded until the end of the next round. You cannot have more than one Light spell active at a time, subsequent casting extinguishes the previous Light.\\
\textbf{For each Critical obtained} in the Magic Test the duration doubles.

\medskip\textbf{Daylight}\index[Spells]{Daylight}\\
\textbf{School}: Invocation\\
\textbf{Level}: 3, Municipality\\
\textbf{Launch Time}: 2 Shares\\
\textbf{Range}: 18 metres\\
\textbf{Components}: V, S\\
\textbf{Duration}: 1 hour of real game time\\
A sphere of light with a radius of 20 feet expands from a point of your choice within range. The sphere radiates bright light and dim light for an additional 12 meters. If you choose a point on an object that you are holding or that is not being worn or carried, light radiates from the object and moves with it. Completely covering an object with something opaque, like a vase or helmet, blocks the light. If any part of this spell's area overlaps with the area of ​​darkness created by a spell of level 3 or lower, the spell that created the darkness is dispelled. The light created is considered sunlight.\\
\textbf{Note}: Devotees of Ljust or Sumkjr gain +1 on saving throws while enlightened by this spell

\medskip\textbf{Dancing Lights}\index[Spells]{Trick - Dancing Lights}\\
\textbf{School}: Invocation\\
\textbf{Level}: 1, Uncommon\\
\textbf{Launch Time}: 2 Shares\\
\textbf{Range}: 36 metres\\
\textbf{Components}: V, S, M (a piece of phosphorus or haunted wood, or an earthworm)\\
\textbf{Duration}: 10 minutes of real game time\\
You create up to four torch-sized lights within range, causing them to appear as torches, lanterns, or glowing orbs that float in the air for the duration of the spell. You can also combine the four lights into a single, vaguely humanoid, Medium-sized light shape. Whichever shape you choose, each light emits a dim light in a 10-foot radius. As 1 move action on your round, you can move the lights up to 60 feet to a new point within range.\\
A light must be within 20 feet of another light created with this spell, and lights vanish if they exceed the spell's range.\\
\textbf{For each Critical obtained} in the Magic Test the duration increases by 1 hour.

\medskip\textbf{Luminescence}\index[Spells]{Luminescence}\\
\textbf{School}: Invocation\\
\textbf{Level}: 1, Uncommon\\
\textbf{Launch Time}: 2 Shares\\
\textbf{Range}: 18 metres\\
\textbf{Components}: V\\
\textbf{Duration}: 1 minute of real game time \\
All objects in a 20-foot cube within range are surrounded by blue, green, or purple light (your choice). Any creature in the area when the spell is cast is also surrounded by light if it fails a Reflex saving throw. For the duration of the spell, affected objects and creatures emit a dim light with a 10-foot radius. Any attack roll against a subject creature or object is +1d6 if the attacker can see it, and the creature or object cannot benefit from invisibility.

\medskip\textbf{Hot Wave}\index[Spells]{Hot Wave}\\
\textbf{School}: Fire\\
\textbf{Level}: 1, Municipality\\
\textbf{Launch Time}: 2 Actions\\
\textbf{Range}: Personal (3 meter cone)\\
\textbf{Components}: V, S\\
\textbf{Duration}: Instant\\
Keep your hands closed in front of you, a powerful searing wave is generated from each of your punches. Each creature in a 10-foot cone must make a Reflex saving throw. A creature takes 1d4 points of magical proficiency damage, to a maximum of 5d4, fire damage on a failed save, or half as much on a successful one. The heat ignites flammable objects in the area that are not being worn or carried.\\
\textbf{For each Magical Critical Success obtained} in the Magic Test the damage increases by 1d4.\\
\textbf{Saving Throw Success/Critical Failure}: In case of a critical failure the damage is doubled, in case of a critical success the damage is further halved


\medskip\textbf{Arcane Hand}\index[Spells]{Arcane Hand}\\
\textbf{School}: Invocation\\
\textbf{Level}: 5, Uncommon\\
\textbf{Launch Time}: 2 Shares\\
\textbf{Range}: 36 metres\\
\textbf{Components}: V, S, M (an eggshell and a snakeskin glove)\\
\textbf{Duration}: Concentration, 1 minute\\
You create a Large hand, composed of transparent, luminous energy, in an unoccupied space within range that you can see. The hand remains for the duration of the spell, and moves at your command, imitating the movements of your hand.\\
The hand is an object that has Defense 25 and Hit Points equal to your maximum Hit Points. It has Strength 4 and Dexterity 0. The hand does not fill its space.
When you cast the spell and as 2 Actions during your subsequent rounds, you can move your hand up to 60 feet and then generate one of the following effects.\\

- \emph{Grabbing Hand}. The hand attempts to grab a Huge or smaller creature within 3 feet of it. To resolve the grab action you use Hand Strength. If the target is Medium or smaller, you have +1d6 on the check. While the hand grips the target, you can use an Action to have the hand crush the target. When you do so, the target takes bludgeoning damage equal to 2d6 + your Intelligence or Wisdom\\
- \emph{Hand of Strength}. The hand attempts to push a 3-foot creature in a direction of your choice. Make a Fortitude saving throw with a spell modifier opposed to the target. If the target is Medium or smaller, you have +1d6 on the check. If you win the contest, the hand pushes the target 1 meter plus 1 meter multiplied by the Intelligence or Wisdom value (minimum 1 meter). The hand moves with the target to stay within 1 meter of it.\\
- \emph{Interposed Hand}. The hand comes between you and a creature of your choice until you give it a different command. The hand moves to stay between you and the target, giving you half cover against the target. The target cannot move through the hand's space if its Strength score is equal to or lower than the hand's Strength score. If its Strength score is higher than the hand's Strength score, the target can move through the hand space, but treats that space as difficult terrain. \\
- \emph{Clenched Fist}. The hand hits a creature or object within 3 feet of it. Make a melee spell attack using your hand. On a hit, the target takes 4d8 force damage.\\

\medskip
\textbf{For each Magical Critical Success obtained} in the Magic Test the damage of the clenched fist option increases by 1d8 and the damage of the grasping hand option increases by 1d6.

\medskip\textbf{Magic Hand}\index[Spells]{Cantrip - Magic Hand}\\
\textbf{School}: Summon\\
\textbf{Level}: 0, Municipality\\
\textbf{Launch Time}: 2 Shares\\
\textbf{Range}: 9 metres\\
\textbf{Components}: V, S\\
\textbf{Duration}: 1d4 rounds +1 per point of Magical Expertise\\
A floating ghostly hand appears at a point within range, chosen by you. The hand remains for the duration of the spell or until interrupted with an action. The hand vanishes if it is more than 30 feet away from you or if you cast the spell again.\\
The Actions needed to move and use the magic hand are the same as those you would use to use your hand. You can use your hand to manipulate an object, open an unlocked door or container, insert or retrieve an object from an open container, or pour out the contents of a vial. You can move your hand 30 feet each time you use it. The hand cannot attack, activate magic items, or carry objects with Encumbrance greater than 2.\\
\textbf{For each Magical Critical Success achieved} in the Magic Test the Encumbrance lifted increases by 1 or doubles the duration.

\medskip\textbf{Magic Mark}\index[Spells]{Cantrip - Magic Mark}\\
\textbf{School}: Universal\\
\textbf{Level}: 0, Municipality\\
\textbf{Launch Time}: 2 Shares\\
\textbf{Range}: Contact\\
\textbf{Components}: V, S\\
\textbf{Duration}: Permanent\\
This spell allows you to inscribe a personal rune or mark on an object. The writing cannot be longer than 15 cm. The writing can be visible or invisible depending on how you decide when casting the spell.
A Detect Magic or Read Magic spell shows the writing if invisible.
If the writing is placed on a creature it disappears within a month.\\
\textbf{For each Magical Critical Success obtained} in the Magic Test write an additional logo.

\medskip\textbf{Message}\index[Spells]{Cantrip - Message}\\
\textbf{School}: Transmutation\\
\textbf{Level}: 0, Municipality\\
\textbf{Launch Time}: 2 Shares\\
\textbf{Range}: 36 metres\\
\textbf{Components}: V, S, M (a small piece of copper wire)\\
\textbf{Duration}: 1 round\\
You point your finger at a creature within range and whisper a short message. The target (and only the target) hears the message and may respond in a whisper that only you can hear.\\
You can also cast this spell through solid objects if you are familiar with the target and know that it is behind the barrier. Magical silence, 1 foot of stone, 1 inch of normal metal, a thin sheet of lead, or 3 feet of wood block the spell. The spell does not have to follow a straight line, and can freely go around corners or through gaps.\\
\textbf{For each Magical Critical Success obtained} in the Magic Test the spell lasts 1 more round.

\medskip\textbf{Metamorphosis}\index[Spells]{Metamorphosis}\\
\textbf{School}: Animals and Plants\\
\textbf{Level}: 4, Municipality\\
\textbf{Launch Time}: 2 Shares\\
\textbf{Range}: 18 metres\\
\textbf{Components}: V, S, M (a caterpillar cocoon)\\
\textbf{Duration}: 1 hour \\
This spell transforms a creature within range that you can see into a new form. An unwilling creature must succeed at a Will save to avoid the effect. Shapeshifters automatically succeed on their saving throw. The spell has no effect on a target with 0 hit points. \\
The transformation lasts for the duration of the spell or until the target drops to 0 hit points or dies. The new form can be that of any beast whose challenge rating is half the spell's Magical Expertise score (or Trait sum if Devoted of Shayalia) of the caster. The target's game statistics, including mental ability scores, are replaced by the chosen beast's statistics. However, he retains his traits and personalities.
The target retains the same Hit Points and regains 1d12 Hit Points in its new form. When he returns to his normal form, the creature retains the hit points he currently has. If he reaches 0 or less Hit Points in the new form then he returns to normal and any effects also affect the current form.\\
The creature is limited in the actions it can perform by the nature of its new form, and cannot speak, cast spells, or perform any other actions that require hands or speech. The target's equipment merges into the new form. The creature cannot activate, use, wield, or benefit in any way from its equipment.

\medskip\textbf{Pure Metamorphosis}\index[Spells]{Pure Metamorphosis}\\
\textbf{School}: Animals and Plants\\
\textbf{Level}: 9, Rare\\
\textbf{Launch Time}: 2 Shares\\
\textbf{Range}: 9 metres\\
\textbf{Components}: V, S, M (a drop of mercury, a small pile of gum arabic, and a puff of smoke) \\
\textbf{Duration}: 1 hour \\
Choose a nonmagical creature or object that you can see and within range. The spell has no effect on a target with 0 hit points. You transform the creature into a different creature, the creature into an object, or the object into a creature (the object must not be worn or carried by another creature). The transformation lasts for the duration of the spell or until the target drops to 0 hit points or dies. If you concentrate on this spell for the entire duration, the transformation becomes permanent.\\
Shapeshifters ignore this spell. An unwilling creature can make a Will saving throw and ignores this spell's effect on a successful one.\\

- \emph{Creature in Creature}. If you transform a creature into another species of creature, the new form can be that of any species you want, whose challenge rating is equal to or lower than your Magical Expertise score (or Common Traits sum if Devoted of Shayalia) . The target's game statistics, including mental ability scores, are replaced by the new form's statistics. However, he retains his traits and personality. \\
The target retains the same Hit Points and regains 1d12 Hit Points in its new form. When it returns to its normal form, the creature retains its current hit points. If it reaches 0 or fewer Hit Points in the new form then it returns to normal and any effect also affects the current form. The creature is limited in the actions it can perform by the nature of its new form, and cannot speak, cast spells, or perform any other actions that require hands or speech, unless the new form is capable of performing these actions. The target's equipment merges into the new form. The creature cannot activate, use, wield, or benefit in any way from its equipment. \\

- \emph{Object into Creature.} You can transform an object into any type of creature, as long as the creature's size is no larger than the object's size and the creature's challenge rating is 9 or less. The creature is friendly towards you and your companions. It works during your rounds. You decide what actions it will take and how it moves. The Storyteller has the creature's statistics and will resolve all of its actions and movements.
If the spell becomes permanent, you lose control of the creature. Depending on how you treated her, she may remain friendly towards you.\\

- \emph{Object Creature}. If you transform a creature into an object, it transforms along with whatever it is wearing or carrying. The creature's statistics become those of the object, and after the spell ends and the creature returns to its normal form, it has no memory of its time in object form.

\medskip\textbf{Arcane Mirage}\index[Spells]{Arcane Mirage}\\
\textbf{School}: Illusion\\
\textbf{Level}: 7, Rare\\
\textbf{Launch Time}: 10 minutes\\
\textbf{Range}: Sight\\
\textbf{Components}: V, S\\
\textbf{Duration}: 10 days\\
Make a piece of terrain within range, in a square area up to 1.5 kilometers, look, sound, and smell like some other type of terrain. However, the general lay of the land remains the same. Open fields or a road can be transformed into a swamp, hills, a crevasse, or some other type of difficult or impassable terrain. A pond can be transformed into a grassy clearing, a precipice into a gentle slope, a rock-strewn ravine into a wide, smooth road.\\
Likewise, you can change the appearance of structures, or add ones where there are none. The spell does not disguise, conceal, or add creatures.\\
The illusion includes auditory, visual, tactile, and olfactory elements, so it can turn clear terrain into difficult terrain (or vice versa) or otherwise prevent movement in the area. Any piece of illusory terrain (such as a stone or a staff) that is removed from the spell's area immediately vanishes. Creatures with true vision can see beyond illusion and discern the true shape of the land; however, the other elements of the illusion remain, so although the creature is aware of the illusion's presence, it can still physically interact with it.\\
\textbf{With three Magical Critical Successes obtained} in the Magic Test the duration is permanent.

\medskip\textbf{Modify Memory}\index[Spells]{Modify Memory}\\
\textbf{School}: Enchantment\\
\textbf{Level}: 5, Very Rare\\
\textbf{Launch Time}: 2 Shares\\
\textbf{Range}: 9 metres\\
\textbf{Components}: V, S\\
\textbf{Duration}: Concentration, maximum 1 minute\\
You attempt to reshape another creature's memories. A creature you can see must make a Will saving throw. If you are fighting it, the creature has +1d6 on its saving throw. On a failed save, the target becomes fascinated by you for the spell's duration. The charmed target is incapacitated and unaware of its surroundings, although it can still hear you. If it takes damage or becomes the target of another spell, this spell ends, and none of the target's memories are affected.\\
While the target is charmed by this spell, you can affect the target's memories of an event he or she experienced in the last 24 hours that lasted no more than 10 minutes. You can permanently erase all memories of the event, allow the target to remember the event with perfect clarity and detailed detail, change the memory of details of the event, or create the memory of another event. You must be able to speak to the target to describe how their memories will be affected, and they must be able to understand your language in order for the changed memories to become lodged in their memory. If the spell ends before you finish describing the altered memories, the creature's memory is not affected. Otherwise, the modified memories set in when the spell ends.\\
A changed memory does not necessarily affect the creature's behavior, particularly if its memories contradict the creature's natural inclinations, Traits, or faith. An illogically modified memory, such as implanting the memory of how much the creature loves to immerse itself in acid, is removed, as if it were a bad dream. The Storyteller may deem a modified memory too senseless to have any effect on a creature. A greater remove curse or restoration spell cast on the target restores its true memories.\\
\textbf{For each Magical Critical Success achieved} in the Magic Test you can alter a target's memories of an event that took place up to 7 days ago, 30 days ago, 1 year ago, or any point in the creature's past.

\medskip\textbf{Spider Moves}\index[Spells]{Spider Moves}\\
\textbf{School}: Transmutation\\
\textbf{Level}: 2, Uncommon\\
\textbf{Launch Time}: 2 Shares\\
\textbf{Range}: Contact\\
\textbf{Components}: V, S, M (a drop of bitumen and a spider)\\
\textbf{Duration}: 10 minutes \\
You cast the spell on contact with a willing creature. Until the spell ends, the creature gains the ability to move up, down, and along vertical surfaces or while standing upside down on the ceiling, keeping its hands free. The target also gains climb speed equal to its movement speed. The creature subject to the spell is considered distracted when casting other spells.

\medskip\textbf{Move the Ground}\index[Spells]{Move the Ground}\\
\textbf{School}: Earth\\
\textbf{Level}: 6, Uncommon\\
\textbf{Launch Time}: 2 Shares\\
\textbf{Range}: 36 metres\\
\textbf{Components}: V, S, M (an iron shovel and a small bag containing a mix of soil types - clay, compost and sand)\\
\textbf{Duration}: Concentration, maximum 2 hours\\
Choose an area on the ground within range, no larger than 40 feet on a side. For durability, you can reshape dirt, sand, or clay in the area any way you want. You can raise or lower the altitude of the area, create or fill a moat, erect or lower a wall, or form a pillar. The extent of these changes cannot exceed half the largest dimension of the area. So, if you operate on a 12 meter square, you can create a pillar 6 meters high, raise or lower the altitude of the terrain by 6 meters, dig a ditch 6 meters deep, and so on. It takes 10 minutes to complete these changes. At the end of every 10 minutes spent concentrating on the spell, you can choose a new area of ​​terrain to work on.\\
Because the transformation of the terrain occurs slowly, creatures in the area usually cannot become trapped or injured by the shifting terrain. The spell cannot manipulate natural stone or stone buildings. Rocks and structures move to adapt to the new terrain. If the way you shape the terrain would make a structure unstable, it could collapse. Likewise, this spell does not directly affect plant growth. The loosened earth carries with it any plants present.

\medskip\textbf{Wall of Force}\index[Spells]{Wall of Force}\\
\textbf{School}: Invocation\\
\textbf{Level}: 5, Municipality\\
\textbf{Launch Time}: 2 Shares\\
\textbf{Range}: 36 metres\\
\textbf{Components}: V, S, M (a pinch of dust produced by crushing a transparent gem)\\
\textbf{Duration}: Concentration\\
An invisible wall of force forms at a point you choose within range. The wall appears in any orientation you want, as a horizontal or vertical barrier, or at an angle. It can float in the air or rest on a solid surface. You can give it the shape of a hemispherical dome or sphere with a radius of up to 10 feet, or give it the appearance of a flat surface made up of up to ten 10-foot by 10-foot panels. Each panel must be contiguous with another panel. In any form, the wall is 30 inches thick and remains for the duration of the spell. If the wall cuts off a creature's space, when it appears, the creature is pushed to one side of the wall (at your discretion). Nothing can physically pass through the wall, anyone beyond the wall has complete coverage. It is immune to all damage and cannot be dispelled by dispel magic. However, the wall is instantly destroyed by the disintegrate spell. The wall also extends into the Ethereal Plane, preventing ethereal travelers from crossing it.

\medskip\textbf{Wall of Fire}\index[Spells]{Wall of Fire}\\
\textbf{School}: Fire\\
\textbf{Level}: 4, Municipality\\
\textbf{Launch Time}: 2 Shares\\
\textbf{Range}: 36 metres\\
\textbf{Components}: V, S, M (a small piece of phosphorus)\\
\textbf{Duration}: 1 minute\\
You create a wall of fire on a solid surface within range. You can create a wall up to 18 meters long, up to 6 meters high and 30 centimeters thick, or a circular wall 6 meters in diameter, 6 meters high and 30 centimeters thick. The wall is opaque and remains for the duration of the spell. \\
When the wall appears, each creature in its area must make a Reflex saving throw. A creature takes 5d8 fire damage on a failed save, or half as much on a successful one. One side of the wall, selected by you when you cast this spell, deals 5d8 fire damage to each creature that ends its round within 10 feet of that side or inside the wall. A creature takes the same damage when it enters the wall for the first time in a round. The other side of the wall deals no damage.\\
\textbf{For each Magic Critical Success obtained} in the Magic Test the damage increases by 2d8.

\medskip\textbf{Wall of Ice}\index[Spells]{Wall of Ice}\\
\textbf{School}: Water\\
\textbf{Level}: 6, Municipality\\
\textbf{Launch Time}: 2 Shares\\
\textbf{Range}: 36 metres\\
\textbf{Components}: V, S, M (a small piece of quartz)\\
\textbf{Duration}: 10 minutes\\
You create a wall of ice on a solid surface within range. You can create a hemispherical dome or sphere with a radius of up to 10 feet, or you can create a flat surface made of up to ten 10-foot square panels. Each panel must be contiguous with at least one other panel. In each form, the wall is 1 foot thick and remains for the duration of the spell. \\
If the wall passes through a creature's space when it appears, the creature is pushed to one side of the wall (your choice) and must make a Reflex saving throw. On a failed save, the creature takes 10d6 cold damage, or half as much damage on a successful one.\\
The wall is an object that can be damaged and broken. Each 10-foot section has Defense 12 and 30 Hit Points, and is vulnerable to fire damage. Reducing a 10-foot section to 0 hit points destroys it and leaves a breeze of freezing wind in the space the wall occupied. A creature that moves through this breeze of icy wind for the first time in a round must make a Fortitude saving throw. On a failed save, the creature takes 5d6 cold damage, or half as much damage on a successful save.\\
\textbf{For each Magical Critical Success obtained} in the Magic Test the damage increases by 2d8.

\medskip\textbf{Stone Wall}\index[Spells]{Stone Wall}\\
\textbf{School}: Invocation\\
\textbf{Level}: 5, Municipality\\
\textbf{Launch Time}: 2 Shares\\
\textbf{Range}: 36 metres\\
\textbf{Components}: V, S, M (a small block of granite)\\
\textbf{Duration}: 10 minutes\\
A nonmagical solid stone wall forms at a point within range, chosen by you. The wall is 15 centimeters thick and is made up of 10 panels measuring 3 by 3 metres. Each panel must be contiguous with at least one other panel. Alternatively, you can create 3 x 6 meter panels that are just 7.5 centimeters thick.\\
If the wall passes through a creature's space when it appears, the creature is pushed to one side of the wall (your choice). If the creature is surrounded on all sides by the wall (or the wall and another solid surface), the creature can make a Reflex saving throw. If she succeeds, she can use her Reaction Action to move at her speed so that she is no longer trapped in the wall.\\
The wall can be any shape you want, though it can't take up the same space as a creature or object. The wall may also not be vertical or rest on a plane. It must, however, blend with and be supported by existing stone. Thus, you can use this spell to bridge a chasm or create a ramp.\\
If you create such a non-vertical wall, longer than 20 feet, you need to halve the size of each panel to create supports. You can roughly shape the stone to create battlements, battlements, and so on. The wall is an object made of stone that can be damaged and broken. Each panel has Defense 15, Hardness 15 and 15 Hit Points per 2.5 centimeters of thickness. Reducing a panel to 0 Hit Points destroys it and may cause connected panels to collapse, at the Storyteller's discretion. If you maintain concentration on this spell for its entire duration, the wall becomes permanent and cannot be dispelled. Otherwise, the wall disappears when the spell ends.

\medskip\textbf{Prismatic Wall}\index[Spells]{Prismatic Wall}\\
\textbf{School}: Abjuration\\
\textbf{Level}: 9, Rare\\
\textbf{Launch Time}: 2 Shares\\
\textbf{Range}: 18 metres\\
\textbf{Components}: V, S\\
\textbf{Duration}: 10 minutes\\
A plane of bright, multicolored lights forms an opaque vertical wall, up to 90 feet wide, 30 feet high, and 1 inch thick, centered on a point within range and that you can see. Alternatively, you can shape the wall into a sphere, up to 30 feet in diameter, centered on a point within range of your choice. The wall remains fixed in place for the duration of the spell. If you place the wall so that it passes through the space occupied by a creature, the spell fails and the spell slot is wasted. The wall radiates bright light up to a range of 18 meters and dim light for an additional 18 meters. You and any creatures you designate when you cast the spell can pass through and remain near the wall without danger. If another creature that can see the wall moves within 20 feet of it or begins its round there, it must succeed on a Fortitude save or be blinded for 1 minute. The wall consists of seven layers, each a different color. When a creature tries to dive or pass through the wall, it does so one layer at a time, through all layers of the wall. As it enters or passes through each layer, the creature must succeed at a Reflex save or suffer the properties of each layer, one at a time, as described below.\\
The wall can be destroyed, one layer at a time, in order from red to violet, in a way specific to each layer. Once a layer is destroyed, it will be destroyed for the duration of the spell. A cancellation rod destroys a Prismatic Wall, but an anti-magic field has no effect on it.\\

- \emph{1. Red}. The target takes 10d6 fire damage on a failed save, or half as much damage on a successful one. As long as this layer exists, nonmagical ranged attacks cannot pass through the wall. The layer can be destroyed by dealing 25 Cold damage to it.\\
- \emph{2. Orange}. The target takes 10d6 acid damage on a failed save, or half as much damage on a successful one. As long as this layer exists, magical ranged attacks cannot pass through the wall. The layer can be destroyed by a strong wind. 3. Yellow. The target takes 10d6 lightning damage on a failed save, or half as much damage on a successful one. This layer can be destroyed by dealing 60 Force damage to it.\\
- \emph{4. Green}. The target takes 10d6 poison damage on a failed save, or half as much damage on a successful one. A Pass Door spell, or another spell of equal or higher level that can open a portal on a solid surface, destroys this layer.\\
- \emph{5. Blue}. The target takes 10d6 cold damage on a failed save, or half as much damage on a successful one. The layer can be destroyed by dealing at least 25 Fire damage to it.\\
- \emph{6. Indigo}. If the saving throw fails, the target is restrained. He must then make a Fortitude saving throw at the start of each of his rounds. If you succeed at the saving throw three times, the spell ends. If you fail your save three times, you are permanently turned to stone and become victim to the petrified condition. Successes and failures do not have to be consecutive; keep track of both until the target has obtained three of the same type. As long as this layer exists, spells cannot be cast through the wall. The layer is destroyed by the bright light emanating from the daylight spell or a similar higher-level spell.\\
- \emph{7. Violet}. If the saving throw fails, the target is blinded. It must then make a Will saving throw at the start of your next round. On a successful save, the blindness ends. If the saving throw fails, the creature is transported to another plane of existence of the Storyteller's choice and is no longer blinded (usually, a creature that is not on its home plane is exiled to it, while other creatures are usually cast into the Astral or Ethereal planes). This layer is destroyed by the dispel magic spell or a similar spell of equal or higher level that can end spells and magical effects.

\medskip\textbf{Wall of Thorns}\index[Spells]{Wall of Thorns}\\
\textbf{School}: Animals and Plants\\
\textbf{Level}: 6, Uncommon\\
\textbf{Launching Time}: 2 Actions\\
\textbf{Range}: 36 metres\\
\textbf{Components}: V, S, M (a handful of plugs)\\
\textbf{Duration}: maximum 10 minutes\\
You create a wall of tough, malleable, entangled bushes filled with sharp thorns. The wall appears within range on a solid surface and remains for the duration of the spell. The wall you can create can be up to 60 feet long, up to 10 feet high, and up to 3 feet thick, or a circle that is 20 feet in diameter, up to 20 feet high, and 3 feet thick. The wall blocks the line of sight.\\
When the wall appears, each creature in its area must make a Reflex saving throw. On a failed save, a creature takes 7d8 piercing damage, or half as much damage on a successful one. A creature can move through the wall, albeit slowly and painfully. For every 3 feet the creature moves through the wall, it must expend 20 feet of movement. Additionally, the first time a creature enters the wall during a round or ends its round inside it, the creature must make a Reflex saving throw. It takes 7d8 slashing damage on a failed save, or half as much damage on a successful one.\\
\textbf{For each Magical Critical Success obtained} in the Magic Test the damage increases by 2d8.

\medskip\textbf{Wall of Wind}\index[Spells]{Wall of Wind}\\
\textbf{School}: Air\\
\textbf{Level}: 3, Uncommon\\
\textbf{Launching Time}: 2 Actions\\
\textbf{Range}: 36 metres\\
\textbf{Components}: V, S, M (a tiny fan and a feather of exotic origins)\\
\textbf{Duration}: 1 minute\\
A wall of strong wind rises from the terrain at a point of your choice within range. You can create a wall up to 15 meters long, 3 meters high and 30 centimeters thick. You can shape the wall in any way you want as long as it makes a continuous path on the ground. The wall remains for the duration of the spell. When the wall appears, each creature within its area must make a Fortitude saving throw. A creature takes 3d8 bludgeoning damage on a failed save, or half as much damage on a successful one. The strong wind keeps away haze, smoke and other gases. Small or smaller flying creatures cannot pass through the wall. Light materials dragged into the wall fly upwards. Arrows, bolts, and other normal ammunition are deflected and automatically miss their targets (boulders thrown by giants and siege engines, and similar ammunition, ignore their effects). Creatures in gaseous form cannot pass through it.\\
\textbf{For each Magical Critical Success obtained} in the Magic Test the duration increases by 1 minute.

\medskip\textbf{Incendiary Cloud}\index[Spells]{Incendiary Cloud}\\
\textbf{School}: Fire\\
\textbf{Level}: 8, Rare\\
\textbf{Launching Time}: 2 Actions\\
\textbf{Range}: 45 metres\\
\textbf{Components}: V, S\\
\textbf{Duration}: 1 minute\\
A cloud of swirling smoke crisscrossed by incandescent lapilli forms in a 20-foot radius sphere centered on a point within range. The cloud spreads around the corners and is in dim light. It remains for the duration of the spell or until a wind of moderate or greater speed (at least 9 miles per hour) disperses it.\\
When the cloud appears, each creature within it must make a Reflex saving throw. A creature takes 10d8 fire damage on a failed save, and half as much damage on a successful one. A creature must also make a saving throw when it first enters the area or ends its round there.\\
At the start of each of your rounds, the cloud moves 10 feet away from you in a direction of your choice.

\medskip\textbf{Nauseating Fog}\index[Spells]{Nauseating Fog}\\
\textbf{School}: Water, Air\\
\textbf{Level}: 3, Uncommon\\
\textbf{Launch Time}: 2 Actions\\
\textbf{Range}: 27 metres\\
\textbf{Components}: V, S, M (a rotten egg or smelly cabbage leaves)\\
\textbf{Duration}: 10 minutes\\
You create, at a point within range, a 20-foot radius sphere composed of a foul, yellow gas. The fog rolls around corners and its area is in dim light. The fog lingers in the air for the duration. Any creature that is completely within the mist at the start of its round must make a Fortitude saving throw against poison. If the saving throw fails, the creature spends 2 actions that round vomiting and staggering. Creatures that don't need to breathe or that are immune to poison automatically succeed on the saving throw. A moderate wind (at least 9 mph) disperses the fog after 4 rounds. A strong wind (at least 18 mph) disperses it after 1 round.

\medskip\textbf{Deadly Fog}\index[Spells]{Deadly Fog}\\
\textbf{School}: Water, Air\\
\textbf{Level}: 5, Rare\\
\textbf{Launch Time}: 2 Shares\\
\textbf{Range}: 36 metres\\
\textbf{Components}: V, S\\
\textbf{Duration}: 10 minutes \\
You create a 20-foot-radius sphere of poisonous yellow-green mist centered at a point of your choice within range. Fog rolls around corners. It remains for the duration of the spell or until a strong wind disperses the fog, ending the spell. Its area is in dim light. When a creature enters the spell's area for the first time in a round or begins its round there, that creature must make a Fortitude saving throw. The creature takes 5d8 poison damage on a failed save, or half as much damage on a successful one. Creatures are affected even if they hold their breath or have no need to breathe. The fog moves 10 feet away from you at the start of each of your rounds, moving along the surface of the ground. The vapours, being heavier than air, tend to descend downwards, even insinuating themselves into the openings.\\
\textbf{For each Magical Critical Success obtained} in the Magic Test the damage increases by 2d8.

\medskip\textbf{Fog Cloud}\index[Spells]{Fog Cloud}\\
\textbf{School}: Water, Air\\
\textbf{Level}: 1, Municipality\\
\textbf{Launch Time}: 2 Shares\\
\textbf{Range}: 36 metres\\
\textbf{Components}: V, S\\
\textbf{Duration}: 1 hour\\
You create a 20-foot radius sphere of mist centered on a point within range. The sphere propagates around the corners, and its area is in dim light. It remains for the duration of the spell or until a wind of moderate or greater speed (at least 9 miles per hour) disperses it.\\
\textbf{For each Magical Critical Success obtained} in the Magic Test the radius of the mist increases by 6 meters.

\medskip\textbf{Arcane Eye}\index[Spells]{Arcane Eye}\\
\textbf{School}: Divination\\
\textbf{Level}: 4, Municipality\\
\textbf{Launch Time}: 2 Shares\\
\textbf{Range}: 9 metres\\
\textbf{Components}: V, S, M (a piece of bat fur)\\
\textbf{Duration}: Concentration, maximum 1 hour\\
You create a magical, invisible eye within range, which floats in the air for the spell's duration.\\
You mentally receive visual information from the eye, which has normal vision and darkvision out to 30 feet. The eye can look in all directions. As a move action, you can move the eye 30 feet in any direction. There is no limit to how far the eye can travel, but it cannot enter another plane of existence. A solid barrier blocks the eye's movement, but it can pass through an opening as small as 1 inch in diameter.

\medskip\textbf{Thundering Wave}\index[Spells]{Thundering Wave}\\
\textbf{School}: Air\\
\textbf{Level}: 1, Municipality\\
\textbf{Launch Time}: 2 Shares\\
\textbf{Range}: Personal (3 meter cube)\\
\textbf{Components}: V, S\\
\textbf{Duration}: Instant\\
A wave of thunderous force projects from you. Each creature in a 2-foot-radius sphere that originates from you must make a Fortitude saving throw. On a failed save, a creature takes 2d8 sonic damage and is moved 10 feet away from you. On a successful save, the creature takes half damage and is not driven away. Additionally, unanchored objects that are wholly within the area are pushed 10 feet away from you by the spell's effect, and the spell produces a thunderous boom audible up to 300 feet away.\\
\textbf{For each Magical Critical Success obtained} in the Magic Test the damage increases by 1d8.

\medskip\textbf{Darkness}\index[Spells]{Darkness}\\
\textbf{School}: Invocation\\
\textbf{Level}: 1, Municipality\\
\textbf{Launch Time}: 2 Shares\\
\textbf{Range}: 18 metres\\
\textbf{Components}: V, M (bat hair and a pinch of bitumen or a piece of coal)\\
\textbf{Duration}: 10 minutes\\
The magical darkness spreads from a point within range, chosen by you, to fill a 10-foot-radius sphere for the spell's duration. Darkness spreads around the corners. A creature with darkvision cannot see in this darkness, and nonmagical light cannot illuminate it.\\
If the spot you choose is on an object you are carrying or one that is not being worn or carried, darkness emanates from the object and moves with it. Completely covering the source of the darkness with an opaque object, such as a vase or helmet, blocks the darkness.\\
If any part of this spell's area overlaps with the area of ​​light created by a spell of level 2 or lower, the spell that created the light is dispelled.

\medskip\textbf{Eithne's Mudball}\index{Eithne's Mudball}\\
\textbf{School}: List of the Earth\\
\textbf{Level}: 1, Uncommon\\
\textbf{Launch Time}: 2 Shares\\
\textbf{Range}: 24 metres\\
\textbf{Components}: Y\\
\textbf{Duration}: Instant\\
The caster mimes the gesture of throwing a stone with a slingshot towards the target and makes an attack roll with ranged spells.
If the attack roll is successful, the target takes 2d6 bludgeoning damage and must make a Reflex saving throw. If the saving throw fails, the target's movement decreases by 2 meters per Action for 1 minute.\\
\textbf{For each Magical Critical Success obtained} in the Magic Test you throw one more stone.

\medskip

\begin{changemargin}{0.3cm}{0.3cm}\begin{enfasi}{
I spread out to avoid area spells (said by a player to avoid a Fireball)
}\end{enfasi}\end{changemargin}

\medskip\textbf{Fireball}\index[Spells]{Fireball}\\
\textbf{School}: Fire\\
\textbf{Level}: 3, Municipality\\
\textbf{Launch Time}: 2 Shares\\
\textbf{Range}: 45 metres\\
\textbf{Components}: V, S, M (a tiny ball of bat guano and sulphur)\\
\textbf{Duration}: Instant\\
A beam of yellow light starts from your pointing finger towards a point within range chosen by you and then explodes with a thunderous roar and transforms into a tongue of flame.\\
Each creature in a 20-foot-radius sphere centered at that point must make a Reflex saving throw. A creature takes 8d6 fire damage on a failed save, or half as much damage on a successful one.\\
The fire spreads and occupies all the available volume within 6 meters of the explosion point. Fire ignites flammable objects in the area that are not being worn or carried.\\
\textbf{For each Magical Critical Success obtained} in the Magic Test the base damage increases by 2d6.\\
\textbf{Saving Throw Success/Critical Failure}: In case of a critical failure the damage is doubled, in case of a critical success the damage is further halved

\medskip\textbf{Delayed Fireball}\index[Spells]{Delayed Fireball}\\
\textbf{School}: Fire\\
\textbf{Level}: 7, Rare\\
\textbf{Launch Time}: 2 Shares\\
\textbf{Range}: 45 metres\\
\textbf{Components}: V, S, M (a large ball of bat guano and sulphur)\\
\textbf{Duration}: Concentration, 1 minute\\
A beam of yellow light starts from your pointing finger, to condense for the duration of the spell in the form of a luminous ball at a point within range, chosen by you. When the spell ends, either because your concentration is broken or because you decide to end it, the ball explodes with a quiet roar and transforms into a jet of flame that spreads around corners. Each creature in a 20-foot-radius sphere centered at that point must make a Reflex saving throw. A creature takes fire damage equal to the total accumulated damage on a failed save, or half as much damage on a successful one. The spell's base damage is 12d6. If the ball has not yet detonated at the end of your round, the damage increases by 1d6.\\
If the glowing ball is touched before the spell ends, the creature that touches it must make a Reflex saving throw. If the save is failed, the spell ends immediately, causing the ball to erupt in flame. On a successful save, the creature can throw the ball up to 40 feet away. When it hits a creature or solid object, the spell ends and the ball explodes.\\
The fire damages objects in the area and ignites flammable objects that are not being worn or carried.\\
\textbf{For each Magic Critical Success obtained} in the Magic Test the damage increases by 1d6.\\
\textbf{Saving Throw Success/Critical Failure}: In case of a critical failure the damage is doubled, in case of a critical success the damage is further halved.

\medskip\textbf{Speak with Animals}\index[Spells]{Speak with Animals}\\
\textbf{School}: Animals and Plants\\
\textbf{Level}: 1, Municipality\\
\textbf{Launch Time}: 2 Actions\\
\textbf{Range}: Personal\\
\textbf{Components}: V, S\\
\textbf{Duration}: 10 minutes\\
For the duration of the spell, you gain the ability to understand and communicate verbally with beasts. Many beasts' knowledge and awareness are limited by their intellect, but at a minimum, beasts can provide you with information about nearby locations and monsters, including those they can sense or have sensed in days past. At the Storyteller's discretion, you may be able to convince a beast to do you a small favor.\\
\textbf{For each Magical Critical Success obtained} in the Magic Test the duration doubles.

\medskip\textbf{Speak with the Dead}\index[Spells]{Speak with the Dead}\\
\textbf{School}: Necromancy\\
\textbf{Level}: 3, Rare\\
\textbf{Launch Time}: 2 Shares\\
\textbf{Range}: 3 metres\\
\textbf{Components}: V, S, M (lit incense)\\
\textbf{Duration}: 10 minutes\\
You grant a semblance of life and Intelligence to a corpse of your choice within range, allowing it to answer questions you ask it. The corpse must still have a mouth and cannot be undead. The spell fails if the corpse has already been the target of this spell within the last 10 days. Until the spell ends, you can ask the corpse up to five questions. The corpse knows only what it already knew in life, including spoken languages. The answers are usually short, cryptic or repetitive, and the corpse is under no obligation to give you truthful answers if you are hostile to it or it recognizes you as its enemy. This spell does not return the creature's soul to the body, but only the spirit that moves it. As a result, the corpse cannot learn new information, understands nothing of what has happened since it died, and cannot make judgments about future events.

\medskip\textbf{Speak with Plants}\index[Spells]{Speak with Plants}\\
\textbf{School}: Animals and plants\\
\textbf{Level}: 3, Rare\\
\textbf{Launch Time}: 2 Shares\\
\textbf{Range}: Personal (9 meter range)\\
\textbf{Components}: V, S\\
\textbf{Duration}: 10 minutes\\
Imbue plants within 30 feet of you with sentience and limited mobility, giving them the ability to communicate with you and execute simple commands. You can question plants about events that occurred in the spell's area over the past day, obtaining information about passing creatures, the weather, and more. You can also transform difficult terrain produced by plant growth (such as bushes and dense undergrowth) into ordinary terrain for the duration of the spell.\\
Or you can transform normal terrain containing plants into difficult terrain, which remains for the duration of the spell, causing, for example, vines and branches to slow down pursuers. \\
At the Storyteller's discretion, the vegetables may also perform other tasks on your behalf. The spell does not allow plants to uproot and move, but they can freely move branches, stems, and stalks. If a plant creature is in the area, you can communicate with it as if you spoke the same language, but you gain no spell-like ability to affect it. This spell can cause plants created by the entangle spell to release an entangled creature.

\medskip\textbf{Divine Word}\index[Spells]{Divine Word}\\
\textbf{School}: Invocation\\
\textbf{Level}: 7, Very Rare\\
\textbf{Cast Time}: 1 Immediate Action\\
\textbf{Range}: 9 metres\\
\textbf{Components}: V\\
\textbf{Duration}: Instant\\
You speak a divine word, infused with the power of your Patron. Choose any number of creatures within range and that you can see. Each creature that can hear you must make a Will saving throw. If the saving throw fails, the creature suffers an effect based on its current hit points:\\

- 100 Hit Points or less: deafened for 1 minute\\
- 40 hit points or less: deafened and blinded for 10 minutes\\
- 30 hit points or less: blinded, deafened and stunned for 1 hour\\
- 20 hit points or less: killed instantly\\

Regardless of its current hit points, a celestial, elemental, fey, or demon that fails its saving throw is forced to return to its home plane (if it is not already there) and cannot return to your current plane before they are 24 hours have passed, unless the wish spell is used.

\medskip\textbf{Healing Word}\index[Spells]{Healing Word}\\
\textbf{School}: Care\\
\textbf{Level}: 1, Uncommon\\
\textbf{Cast Time}: 1 Reaction Action\\
\textbf{Range}: 18 metres\\
\textbf{Components}: V\\
\textbf{Duration}: Instant\\
A creature you choose within range that you can see regains hit points equal to 1d4 + your spellcasting ability modifier. This spell deals the same amount of damage to an undead.\\
\textbf{For each Magic Critical Success obtained} in the Magic Test the healing increases by 1d4.\\
If the spellcaster and the healed creature are both Followers of the same Patron, the spell heals 1d4 more.\\
If the spellcaster and the healed creature are both Devotees of the same Patron, the spell heals 2d4 more.

\medskip\textbf{Mass Healing Word}\index[Spells]{Mass Healing Word}\\
\textbf{School}: Care\\
\textbf{Level}: 3, Rare\\
\textbf{Cast Time}: 1 Immediate Action\\
\textbf{Range}: 18 metres\\
\textbf{Components}: V\\
\textbf{Duration}: Instant\\
As you speak words of healing, up to six creatures within range that you can see, chosen by you, regain hit points equal to 1d4 + your spell ability modifier. This spell deals the same amount of damage to undead.\\
\textbf{For each Magical Critical Success obtained} in the Magic Test the healing increases by 1d4.\\
If the spellcaster and the healed creature are both Followers of the same Patron, the spell heals 1d4 more.\\
If the spellcaster and the healed creature are both Devotees of the same Patron, the spell heals 2d4 more.

\medskip\textbf{Word of Power Stun}\index[Spells]{Word of Power Stun}\\
\textbf{School}: Enchantment\\
\textbf{Level}: 8, Uncommon\\
\textbf{Cast Time}: 1 Immediate Action\\
\textbf{Range}: 18 metres\\
\textbf{Components}: V\\
\textbf{Duration}: 1 minutes\\
You speak a word of power that can overwhelm the mind of a creature within range and that you can see, leaving them confused. If the target has 150 hit points or less, it is stunned. Otherwise, the spell has no effect.

\medskip\textbf{Word of Power Kill}\index[Spells]{Word of Power Kill}\\
\textbf{School}: Enchantment\\
\textbf{Level}: 9, Rare\\
\textbf{Cast Time}: 1 Immediate Action\\
\textbf{Range}: 18 metres\\
\textbf{Components}: V\\
\textbf{Duration}: Instant\\
You speak a word of power that causes a creature you can see within range to die instantly. If the creature you choose has 100 hit points or less, it dies. Otherwise the spell has no effect.

\medskip\textbf{Word of the Retreat}\index[Spells]{Word of the Retreat}\\
\textbf{School}: Summon\\
\textbf{Level}: 6, Rare\\
\textbf{Launch Time}: 2 Shares\\
\textbf{Range}: 1 metre\\
\textbf{Components}: V\\
\textbf{Duration}: Instant\\
You and up to five willing creatures within 3 feet of you are instantly teleported to a previously designated safe location, called a sanctuary. You, and any creatures teleported with you, reappear in the unoccupied space closest to the location indicated when you prepared this shrine (see below). If you cast this spell without first preparing a shrine, the spell has no effect.\\
You must indicate a sanctuary, which is dedicated or strongly connected to your Patron. If you attempt to cast the spell to take you to an area that is not dedicated by your Patron, the spell has no effect.

\medskip\textbf{Door Pass}\index[Spells]{Door Pass}\\
\textbf{School}: Earth\\
\textbf{Level}: 5, Uncommon\\
\textbf{Launch Time}: 2 Shares\\
\textbf{Range}: 9 metres\\
\textbf{Components}: V, S, M (a pinch of sesame seeds)\\
\textbf{Duration}: 1 hour\\
For the duration of the spell, a passage appears at a point within range that you can see, on a surface of wood, wall, or stone (such as a wall, ceiling, or floor) of your choice. Choose the size of the opening: a maximum of 1 meter wide, 2 meters high and 6 meters deep. The passage does not create instability in the structure surrounding it.\\
When the opening disappears, any creatures or objects still in the passage created by the spell are ejected safely into the unoccupied space closest to the surface on which you cast the spell.

\medskip\textbf{Pass Without Traces}\index[Spells]{Pass Without Traces}\\
\textbf{School}: Earth, Animals and Plants\\
\textbf{Level}: 2, Municipality\\
\textbf{Launch Time}: 2 Actions\\
\textbf{Range}: Personal\\
\textbf{Components}: V, S, M (ashes of a burnt mistletoe leaf and a spruce twig)\\
\textbf{Duration}: Concentration, 1 hour
For the duration of the spell your trail cannot be followed except by magical means. The creature receiving this bonus leaves no traces or other signs of its passage.\\
\textbf{For each Magical Critical Success obtained} in the Magic Test you can include another creature in the spell's benefits.

\medskip\textbf{Veiled Step}\index[Spells]{Veiled Step}\\
\textbf{School}: Summon\\
\textbf{Level}: 2, Rare\\
\textbf{Cast Time}: 1 Action\\
\textbf{Range}: Personal\\
\textbf{Components}: V\\
\textbf{Duration}: Instant\\
Quickly shrouded in a silver haze, you teleport up to 30 feet to an unoccupied space you can see.\\
If you are a Lynx devotee, the spell has a casting time of 1 Immediate Action and the rarity is Uncommon.\\
\textbf{If you roll two Magical Critical Successes} in the Magic Test you can switch with a willing creature.

\medskip\textbf{Fast Pass}\index[Spells]{Fast Pass}\\
\textbf{School}: Transmutation\\
\textbf{Level}: 1, Very Rare\\
\textbf{Launching Time}: 2 Actions\\
\textbf{Range}: Contact\\
\textbf{Components}: V, S, M (a hare's paw)\\
\textbf{Duration}: 1 hour\\
A creature's movement increases by 3 feet until the spell ends. \\
\textbf{For each Magical Critical Success obtained} in the Magic Test you can target an additional creature.

\medskip\textbf{Fear}\index[Spells]{Fear}\\
\textbf{School}: Illusion\\
\textbf{Level}: 3, Uncommon\\
\textbf{Launch Time}: 2 Actions\\
\textbf{Range}: Personal (9 meter cone)\\
\textbf{Components}: V, S, M (a white feather or the heart of a chicken)\\
\textbf{Duration}: 1 minute\\
You project an illusory image of a creature's worst fears. Each creature in a 30-foot cone must succeed on a Will save or drop whatever it is holding and be frightened for the spell's duration.\\
While frightened by this spell, a creature must, on each of its rounds, take the Dash action and move away from you by the safest route, unless it has no room to move. If the creature ends its round in a place where it has no line of sight of you, it can make a Will saving throw; if it succeeds, the spell ends for that creature.

\medskip\textbf{Bark Skin}\index[Spells]{Bark Skin}\\
\textbf{School}: Animals and Plants\\
\textbf{Level}: 2, Municipality\\
\textbf{Launch Time}: 2 Actions\\
\textbf{Range}: Contact\\
\textbf{Components}: V, S, M (a handful of oak bark)\\
\textbf{Duration}: 1 hour\\
The skin of the target you are in contact with when you cast the spell becomes rough and bark-like in appearance until the spell ends, and the target's Defense cannot be lower than 16, regardless of the armor it uses. is wearing.

\medskip\textbf{Stoneskin}\index[Spells]{Stoneskin}\\
\textbf{School}: Earth\\
\textbf{Level}: 4, Uncommon\\
\textbf{Launch Time}: 2 Actions\\
\textbf{Range}: Contact\\
\textbf{Components}: V, S, M (diamond dust worth 100 gp, which the spell consumes)\\
\textbf{Duration}: 1 hour\\
You cast the spell upon contact with a willing creature, whose skin turns into a substance as hard as stone. Roll 1d4+half the CM value, the resulting sum is the number of times a melee or ranged weapon attack is canceled (regardless of whether it hits or not).\\
\textbf{For each Magical Critical Success obtained} in the Magic Test you increase the attacks canceled by 1.

\medskip\textbf{Plague of Insects}\index[Spells]{Plague of Insects}\\
\textbf{School}: Animals and Plants\\
\textbf{Level}: 5, Rare\\
\textbf{Launching Time}: 2 Actions\\
\textbf{Range}: 90 metres\\
\textbf{Components}: V, S, M (a few grains of sugar, a few grains of wheat, a little lard)\\
\textbf{Duration}: 10 minutes\\
A swarm of hungry locusts fills a 20-foot-radius sphere centered at a point you choose within range. The sphere propagates around the corners. The sphere remains for the duration of the spell, and its area is in dim light. The area of ​​the sphere is difficult terrain.\\
When the area appears, each creature within it must make a Fortitude saving throw. A creature takes 4d10 damage on a failed save, or half as much damage on a successful one. A creature must also make this saving throw when it first enters the spell's area during a round or if it ends its round there.\\
\textbf{For each Magical Critical Success obtained} in the Magic Test the damage increases by 2d8.

\medskip\textbf{Stone in Mud - Mud in Stone}\index[Spells]{Stone in Mud}\index[Spells]{Mud in Rock}\\
\textbf{School}: Earth\\
\textbf{Level}: 5, Uncommon - Very Rare\\
\textbf{Launching Time}: 2 Actions\\
\textbf{Range}: 45 metres\\
\textbf{Components}: V, S, M (water and clay)\\
\textbf{Duration}: Instant\\
This spell turns any type of natural rock into an equal volume of mud. The magic stone is not affected by the spell. The spell affects up to 2 cubes measuring 3x3x3 meters. The depth of the mud created cannot exceed 3 metres. Creatures unable to fly, levitate, or otherwise move away from the mud sink up to their waist or chest; creatures are entangled and the terrain becomes doubly difficult. Creatures large enough to walk across the bottom of the mud puddle can wade through the area as difficult terrain.

If Mudstone is thrown onto the ceiling of a cave or tunnel, the mud splashes onto the floor and expands to form a pool 3 feet deep. The falling mud and ensuing landslide deals 8d6 bludgeoning damage to anyone directly beneath the area if they don't halve the damage with a Reflex save.

Castles and large stone buildings are generally immune to the spell's effects, as Turn Stone to Mud does not reach deep enough to undermine their foundations. However, other smaller buildings often rest on foundations shallow enough to be damaged or even destroyed by the spell's effects.

The mud remains until a dispel magic or mud to stone spell is successfully used, which restores its substance, but not necessarily its form. Natural evaporation turns the mud into normal soil over several days depending on exposure to sun, wind and natural drying.
If a creature is in the mud at the time of the Mud to Stone spell it can make a Reflex saving throw to free itself otherwise a Strength check of DC 22 or 30 damage is required to break the stone.\\
\textbf{For each Magical Critical Success obtained} in the Magic Test you influence an additional cube of 3x3x3 meters.

\medskip\textbf{Pyroexpert}\index[Spells]{Pyroexpert}\\
\textbf{School}: Fire\\
\textbf{Level}: 2, Uncommon\\
\textbf{Launch Time}: 2 Shares\\
\textbf{Range}: 18 metres\\
\textbf{Components}: V, S, M (a match that is consumed)\\
\textbf{Duration}: Instant\\
The caster chooses an area with a fire, at least 1 meter in edge, within range that is directly visible to him. By extinguishing the flames he can create fireworks or smoke.

- \emph{Fireworks}. The target fire explodes in a dazzling display of flame and color. Each creature within 10 feet of the target must succeed on a Fortitude save or become blinded until the end of the next round.

- \emph{Smoking}. Thick, black smoke billows from the target fire and spreads in a 20-foot radius, moving around corners. The smoke area is heavily darkened and provides medium coverage. The smoke persists for 1 minute or until a strong wind disperses it.

\medskip\textbf{Shimmer Dust}\index[Spells]{Shimmer Dust}\\
\textbf{School}: Fire, Air\\
\textbf{Level}: 2, Uncommon\\
\textbf{Launch Time}: 2 Actions\\
\textbf{Range}: 36 metres\\
\textbf{Components}: V, S, M (silver dust)\\
\textbf{Duration}: 1 round for Magical Expertise\\
In a sphere 3 meters in diameter, anyone who finds themselves is covered in shimmering, luminous dust. The cloud outlines the creatures present, even the invisible ones, and anyone remaining in the area must make a Reflex Saving Throw at the start of the round or be blinded for the round. The dust disappears naturally after duration or if carried away by even a light wind.

\medskip\textbf{Dimensional Door}\index[Spells]{Dimensional Door}\\
\textbf{School}: Summon\\
\textbf{Level}: 4, Municipality\\
\textbf{Launching Time}: 2 Actions\\
\textbf{Range}: 150 metres\\
\textbf{Components}: V\\
\textbf{Duration}: Instant\\
You teleport from your current location to anywhere else within range. You arrive exactly where you want. It can be a place you can see, one you can visualize, or one you can describe by indicating distance and direction, such as \emph{30 meters down} or \emph{90 meters up to the northwest at an angle of 45 degrees}.\\
You may carry with you objects whose weight does not exceed your Encumbrance capacity. You may also carry a willing creature your size or smaller with equipment up to the limit of its carrying capacity. The creature must be within 3 feet of you when you cast this spell. \\
If you arrive at a location already occupied by an object or creature, you and the creature traveling with you each take 4d6 force damage, and the spell fails to teleport you.\\
\textbf{For every two Magic Critical Success obtained} in the Magic Test you can bring an additional creature.

\medskip\textbf{Healing Prayer}\index[Spells]{Healing Prayer}\\
\textbf{School}: Care\\
\textbf{Level}: 2, Municipality\\
\textbf{Launch Time}: 10 minutes\\
\textbf{Range}: 9 metres\\
\textbf{Components}: V\\
\textbf{Duration}: Instant\\
Up to six creatures within range that you can see, chosen by you, each regain hit points equal to 2d6 + your spell ability modifier. This spell deals the same amount of damage to undead.\\
\textbf{For each Magical Critical Success obtained} in the Magic Test the healing increases by 1d8.

\medskip\textbf{Omen}\index[Spells]{Omen}\\
\textbf{School}: Divination\\
\textbf{Level}: 2, Municipality\\
\textbf{Launch Time}: 1 minute\\
\textbf{Range}: Personal\\
\textbf{Components}: V, S, M (specially marked sticks, bones or similar objects worth at least 25 gp)\\
\textbf{Duration}: Instant\\
By throwing sticks inlaid with gems, rolling dragon bones, stacking elaborate cards, or employing some other divination tool, you receive an omen from an otherworldly entity regarding the outcome of a specific course of action you intend to take in the next 30 minutes. The Narrator chooses from the following omens:\\

- Prosperity, for positive results\\
- Calamity, for negative results\\
- Prosperity and calamity, for both positive and negative results\\
- Nothing, for the results which are neither particularly positive nor negative\\

The spell does not take into account any possible circumstances that could change the outcome, such as the casting of additional spells or the loss or arrival of an ally. If you cast the spell two or more times before the new sun has risen, there is a cumulative 25\% chance that for each casting after the first you will get an erroneous reading. The Storyteller makes this roll in secret.

\medskip\textbf{Prestidigitation}\index[Spells]{Cantrip - Prestidigitation}\\
\textbf{School}: Universal\\
\textbf{Level}: 0, Municipality\\
\textbf{Launching Time}: 2 Actions\\
\textbf{Range}: 3 metres\\
\textbf{Components}: V, S\\
\textbf{Duration}: Maximum 1 hour\\
This spell is a minor magic trick that novice spellcasters employ for practice. You create one of the following magical effects within range:\\

- Create a harmless, instant sensory effect such as a shower of sparks, a breath of wind, a faint musical note, or a strange smell.\\
- Instantly light or extinguish a candle, torch or small campfire.\\
- Instantly clean or soil an object no larger than 30cm on a side.\\
- Cool, heat or season a 30 cm cube of non-living material for 1 hour.\\
- Make a color, small sign or symbol appear on an object or surface for 1 hour.\\
- You create a nonmagical trinket or illusory image that enters your hand and remains until the end of your next round.\\

If you cast this spell multiple times, you can keep up to three non-instantaneous effects active at a time, and you can end one of these effects with an action.\\
\textbf{For each Magical Critical Success obtained} in the Magic Test you can activate an additional magical effect.


\medskip\textbf{Prediction}\index[Spells]{Prediction}\\
\textbf{School}: Divination\\
\textbf{Level}: 9, Uncommon\\
\textbf{Launch Time}: 1 minute\\
\textbf{Range}: Contact\\
\textbf{Components}: V, S, M (a hummingbird feather)\\
\textbf{Duration}: 8 hours\\
You cast the spell on contact with a willing creature to grant it a limited ability to see into the immediate future. For the duration, the target cannot be surprised and has +1d6 on attack rolls, ability checks, and saving throws. Additionally, for the duration, other creatures have -1d6 on attack rolls against the target. The spell ends immediately if you cast it again before its duration ends.

\medskip\textbf{Produce Flame}\index[Spells]{Cantrip - Produce Flame}\\
\textbf{School}: Fire\\
\textbf{Level}: 0, Municipality\\
\textbf{Cast Time}: 1 Action\\
\textbf{Range}: Personal\\
\textbf{Components}: V, S\\
\textbf{Duration}: 10 minutes\\
A small flame appears in your hand. The flame remains there for the duration of the spell and does not harm you or your equipment. The flame produces dim light in a 1 meter radius. The spell ends if you interrupt it with an action or cast it again.\\
You can also use the flame to attack, although doing so ends the spell. When you cast this spell, or as an action on a subsequent round, you can hurl the flame at a creature within 30 feet of you. Make a ranged spell attack. On a hit, the target takes 1d8 fire damage.\\
The damage of the spell increases by 1d8 when you reach CM 5, CM 11 and CM 17, but it costs 2 Actions to cast it enhanced and 2 Magic Points, it is also necessary to have taken Adept of Magic in this Magic List a number of times equal to the enhancements that you want to apply.\\
\textbf{For each Magical Critical Success obtained} in the Magic Test you can attack one additional creature without ending the spell.


\medskip\textbf{Prohibition}\index[Spells]{Prohibition}\\
\textbf{School}: Abjuration\\
\textbf{Level}: 6, Uncommon\\
\textbf{Launch Time}: 10 minutes\\
\textbf{Range}: Contact\\
\textbf{Components}: V, S, M (a splash of Holy Water, rare incense, and a powdered ruby ​​worth 1000 gp)\\
\textbf{Duration}: 1 day\\
You create a magical travel ward that protects up to 4,000 square meters of floor space, up to a height of 30 feet above the ground. For the duration of the spell, creatures cannot teleport into the area or use passageways, such as the one created by the portal spell, to enter the area. The spell protects the area from planar travel, and thus prevents creatures from accessing the area via the Astral Plane, Ethereal Plane, or Shadow Plane, or the planar shift spell.\\
Additionally, the spell damages creature types you choose when casting. Choose one or more of the following: celestials, elementals, fey, demons, and undead. When a selected creature enters the spell's area for the first time in a round or begins its round here, the creature takes 5d10 Light or Void damage (your choice, when you cast the spell). \\
When you cast this spell, you can set a password. A creature that says the password while entering the spell's area takes no damage from it.\\
The spell's area cannot overlap with the area of ​​another forbidding spell. If you cast ban every day for 30 days in the same place, the spell will last until it is dispelled, and the material components will be consumed during the last casting.

\medskip\textbf{Astral Projection}\index[Spells]{Astral Projection}\\
\textbf{School}: Necromancy\\
\textbf{Level}: 9, Very Rare\\
\textbf{Launch Time}: 2 Actions\\
\textbf{Range}: 3 metres\\
\textbf{Components}: V, S, M (For each creature affected by this spell, you must provide a hyacinth worth at least 1000 gp and an elegantly carved silver ingot worth at least 100 gp, all of which are consumed by the spell)\\
\textbf{Duration}: Special\\
You and up to eight other willing creatures within range project your astral bodies into the Astral Plane (the spell fails and the casting is wasted if you are already in that plane). The material body you leave behind is unconscious and in a state of suspended animation; it does not need food or water and does not age.\\
Your astral body looks every bit like your mortal form, replicating your in-game stats and items. The main difference is the addition of a silver cord that extends from your shoulder blades 30 centimeters behind you, then becoming invisible. The cord is your connection to your material body. As long as this connection remains intact, you will be able to return home. If the cord is cut (an event that only happens when a specific effect indicates it) your body and soul are separated, killing you instantly.\\
Your astral form can travel freely through the Astral Plane and pass through portals that lead from there to other planes. If you enter a new plane or return to the plane you were on when you cast the spell, your body and items are transported along the silver cord, allowing you to reenter your body upon entering the new plane. Your astral form is a separate incarnation. Any damage or other effects that apply to it do not affect your physical body, nor appear there upon your return.\\
The spell ends for you and your companions when you use an action to end it. When the spell ends, the creature it affects returns to its physical body, and awakens. The spell may also have an early end for you or one of your companions. A dispel magic spell successfully used on the astral or physical body ends the spell for that creature. If the creature's original body or astral form drops to 0 hit points, the spell ends for that creature. If the spell ends and the silver cord is intact, the cord drags the creature's astral form back to its body, ending its state of suspended animation.\\
If you are returned to your body prematurely, your companions must remain in their astral form and find their own way back to their bodies, usually dropping to 0 Hit Points.

\medskip\textbf{Protection from Good and Evil}\index[Spells]{Protection from Good and Evil}\\
\textbf{School}: Abjuration\\
\textbf{Level}: 1, Municipality\\
\textbf{Launch Time}: 2 Shares\\
\textbf{Range}: Contact\\
\textbf{Components}: V, S, M (Holy water or powdered silver and iron, which the spell consumes worth 5 gp)\\
\textbf{Duration}: 10 minutes\\
Until the spell ends, a willing creature in contact with you at the time of casting is protected from certain types of creatures: aberrations, celestials, elementals, fey, demons, and undead.\\
Protection confers several benefits. Creatures of those types have -1d6 on attack rolls against the target. The target cannot be Charmed, frightened, or possessed by them. If the target is already charmed, frightened, or possessed by such a creature, the target has +1d6 on any new saving throw against the effect.\\
\textbf{This spell is not usable if you use Traits. The Narrator can grant the same effects towards the Followers and Patrons of other Patrons}

\medskip\textbf{Protection from Energy}\index[Spells]{Protection from Energy}\\
\textbf{School}: Abjuration\\
\textbf{Level}: 3, Municipality\\
\textbf{Launch Time}: 2 Shares\\
\textbf{Range}: Contact\\
\textbf{Components}: V, S\\
\textbf{Duration}: 10 minutes\\
You cast the spell on contact with a willing creature. For the duration of the spell, the target has resistance to a damage type of your choice: acid, cold, fire, lightning, or sonic. You can sacrifice the duration of the spell, ending it, to completely negate the damage taken from an energy source.\\
\textbf{For each Magical Critical Success obtained} in the Magic Test you can influence another person or double the duration.

\medskip\textbf{Protection from Minor Energy}\index[Spells]{Protection from Minor Energy}\\
\textbf{School}: Abjuration\\
\textbf{Level}: 1, Rare\\
\textbf{Cast Time}: 1 Reaction\\
\textbf{Range}: Contact\\
\textbf{Components}: V, S\\
\textbf{Duration}: 1 minute\\
You cast the spell on contact with a willing creature. For the spell's duration, the target has Damage Reduction from the chosen energy equal to 5. You can sacrifice the duration of the spell, ending it, to reduce the damage taken from an energy source by 20 (as if you had Damage Resistance). Damage 20 from that energy source).\\
\textbf{For each Magical Critical Success obtained} in the Magic Test you can influence another person or double the duration.

\medskip\textbf{Protection from Poisons}\index[Spells]{Protection from Poisons}\\
\textbf{School}: Abjuration\\
\textbf{Level}: 2, Uncommon\\
\textbf{Launch Time}: 2 Shares\\
\textbf{Range}: Contact\\
\textbf{Components}: V, S\\
\textbf{Duration}: 1 hour\\
For the spell's duration, the target has +1d6 on saving throws against being poisoned, and has resistance to poison damage.\\
\textbf{In case of two Magical Critical Successes obtained} in the Magic Test you can cancel a poison circulating on the target.

\medskip\textbf{Marking Smite}\index[Spells]{Marking Smite}\\
\textbf{School}: Invocation\\
\textbf{Level}: 2, Municipality\\
\textbf{Cast Time}: 1 Immediate Action\\
\textbf{Range}: Personal\\
\textbf{Components}: V\\
\textbf{Duration}: 1 minute\\
The next time you hit a creature with a melee weapon attack during the spell's duration, the weapon glows with a magical glow as you strike. The attack deals an additional 1d6 Light damage to the target, which becomes visible if it is invisible and emits dim light in a 3-foot radius. Additionally, the target cannot become invisible until the spell ends. \\
\textbf{For each Magical Critical Success obtained} in the Magic Test the additional damage increases by 1d6.

\medskip\textbf{Purify Food and Drink}\index[Spells]{Purify Food and Drink}\\
\textbf{School}: Animals and Plants\\
\textbf{Level}: 1, Municipality\\
\textbf{Launch Time}: 2 Shares\\
\textbf{Range}: 3 metres\\
\textbf{Components}: V, S\\
\textbf{Duration}: Instant\\
All nonmagical food and drink in a 3-foot-radius sphere, centered at a point within range of your choice, is cleansed and freed of poisons and diseases. A rotting food is cleaned and made edible.

\medskip\textbf{Frost Ray}\index[Spells]{Cantrip - Frost Ray}\\
\textbf{School}: Water\\
\textbf{Level}: 0, Municipality\\
\textbf{Cast Time}: 1 Action\\
\textbf{Range}: 18 metres\\
\textbf{Components}: V, S\\
\textbf{Duration}: Instant\\
A frozen beam of blue light strikes a creature within range. Make a ranged spell attack against the target. If you hit, he takes 1d8 cold damage, and his speed is reduced by 10 feet until the start of your next round. \\
The damage of the spell increases by 1d8 when you reach CM 5, CM 11 and CM 17, but it costs 2 Actions to cast it enhanced and 2 Magic Points, it is also necessary to have taken Adept of Magic in this Magic List a number of times equal to the enhancements that you want to apply.\\
\textbf{For every two Magic Critical Success obtained} in the Magic Test you create an additional ice cream bundle.

\medskip\textbf{Radius of Fatigue}\index[Spells]{Radius of Fatigue}\\
\textbf{School}: Necromancy\\
\textbf{Level}: 2, Municipality\\
\textbf{Launch Time}: 2 Shares\\
\textbf{Range}: 18 metres\\
\textbf{Components}: V, S\\
\textbf{Duration}: 1 minute\\
A black beam of debilitating energy shoots from your finger towards a creature within range. Make a ranged spell attack against the target. If you hit, the target will deal half damage with weapon attacks that use Force until the spell ends.\\
\textbf{For every two Magic Critical Success obtained} in the Magic Test you increase the target's Fatigue level by 1.

\medskip\textbf{Hot Ray}\index[Spells]{Hot Ray}\\
\textbf{School}: Fire\\
\textbf{Level}: 2, Municipality\\
\textbf{Launch Time}: 2 Shares\\
\textbf{Range}: 36 metres\\
\textbf{Components}: V, S\\
\textbf{Duration}: Instant\\
You create three beams of fire and project them at three targets within range. You can project them at the same target or different targets.\\
Make a ranged spell attack for each ray. On a hit, the target takes 2d6 fire damage.\\
\textbf{For each Magical Critical Success obtained} in the Magic Test you create an additional ray.

\medskip\textbf{Spider Web}\index[Spells]{Spider Web}\\
\textbf{School}: Animals and Plants\\
\textbf{Level}: 2, Municipality\\
\textbf{Launch Time}: 2 Shares\\
\textbf{Range}: 18 metres\\
\textbf{Components}: V, S, M (a piece of spider web)\\
\textbf{Duration}: 1 hour\\
You summon a thick mass of dense, sticky webbing to a point within range, chosen by you. For the duration, the web fills a 20-foot cube from that point. Spiderweb is difficult terrain and makes that area darkened slightly.\\
If the web is not anchored between two solid masses (such as walls or trees) or stretched along a floor, wall, or ceiling, the summoned web collapses in on itself, and the spell ends at the start of your next round. The canvases stretched on a flat surface have a depth of 1 meter.\\
Each creature that begins its round in the web or enters it during its round must make a Reflex saving throw. On a failed save, the creature is entangled for as long as it remains in the web or until it breaks free.\\
A creature entangled in the webs can use 2 actions to make a new saving throw. If she passes it, she is no longer in the way.
The web is flammable and if exposed to flames it immediately catches fire and burns for 2 rounds, dealing 2d4 points of fire damage to each creature within its area.

\medskip\textbf{Enchanted Club}\index[Spells]{Cantrip - Enchanted Club}\\
\textbf{School}: Animals and Plants\\
\textbf{Level}: 0, Municipality\\
\textbf{Cast Time}: 1 Immediate Action\\
\textbf{Range}: Contact\\
\textbf{Components}: V, S, M (mistletoe, a four-leaf clover leaf, and a club or fighting stick)\\
\textbf{Duration}: 1 minute\\
The wood of a club or fighting staff you are holding is infused with the power of nature. For the duration of the spell, when using that weapon you can use your spellcasting ability in place of Strength for attack rolls and melee damage, and the weapon's damage die becomes a d8. The weapon also becomes magical, if it isn't already. The spell ends if you cast it again or drop your weapon.\\
\textbf{For each Magical Critical Success obtained} in the Magic Test the duration doubles or you get +1 to damage.

\medskip\textbf{Wonderful Palace}\index[Spells]{Wonderful Palace}\\
\textbf{School}: Summon\\
\textbf{Level}: 7, Legendary\\
\textbf{Launch Time}: 1 minute\\
\textbf{Range}: 90 metres\\
\textbf{Components}: V, S, M (a miniature portal carved from ivory, a small piece of polished marble, and a tiny silver spoon; each of these items must be worth at least 5 gp)\ \
\textbf{Duration}: 24 hours\\
Within range, you summon an extradimensional dwelling that remains for the spell's duration. Choose where its front door is located. The entrance door emits a gentle light and is 1 meter wide by 3 meters high. You and all the creatures you indicated when you cast the spell can enter the extradimensional home, as long as the door remains open. You can open or close the door if you are within 30 feet of it. While closed, the door is invisible.\\
Beyond the door there is a magnificent entrance, beyond which numerous rooms unfold. The atmosphere is clean, fresh and welcoming. You can create as many floors as you like, but the space cannot exceed 50 cubes each with a 10-foot edge. The place is furnished and decorated as you like. Contains enough food to satisfy a 9-course banquet for 100 people. A staff of 100 almost transparent servants is at the service of anyone who enters. It's up to you to decide the visual appearance of these minions and their clothing. They absolutely obey your orders. Each minion can perform any task a normal human minion can perform, but they cannot attack or perform any action that would directly harm another creature. Servants can then collect items, clean, repair, fold clothes, light fires, serve food, pour wine, and so on. Servants can go anywhere in the mansion, but cannot leave it. Furniture and other objects created by this spell turn to smoke when taken out of the home. When the spell ends, any creatures within the extradimensional space are expelled into the open space closest to the exit.\\
\textbf{Note}: the spell cast every day in the same place for a year becomes permanent.\\
\textbf{For each Magical Critical Success obtained} in the Magic Test the duration doubles or remove a month from the count to make it permanent.

\medskip\textbf{Mental Regression}\index[Spells]{Mental Regression}\\
\textbf{School}: Enchantment\\
\textbf{Level}: 8, Rare\\
\textbf{Launch Time}: 2 Shares\\
\textbf{Range}: 45 metres\\
\textbf{Components}: V, S, M (a handful of clay, crystal, glass or mineral spheres)\\
\textbf{Duration}: Instant\\
You assault the mind of a creature within range and that you can see, attempting to fragment its intellect and personality. The target takes 4d6 damage and must make a Will save. If the saving throw fails, the creature's Intelligence and Charisma scores drop to -4. The creature cannot cast spells, activate magical items, understand languages, or communicate in any intelligible way. The creature can, however, identify its friends, track them, and even protect them. After 30 days, the creature can repeat the saving throw against the spell. If it succeeds, the spell ends. If it fails, the effect is permanent.\\
the spell can be ended within 30 days by greater restoration, healing, or wish.

\medskip\textbf{Reincarnation}\index[Spells]{Reincarnation}\\
\textbf{School}: Animals and Plants\\
\textbf{Level}: 5, Rare\\
\textbf{Launch Time}: 1 hour\\
\textbf{Range}: Contact\\
\textbf{Components}: V, S, M (rare oils and ointments worth at least 1000 gp, which the spell consumes)\\
\textbf{Duration}: Instant\\
You come into contact with a dead humanoid or a dead humanoid fragment. As long as the creature has not been dead for more than 10 days, the spell forms a new adult body for it and then calls its soul to enter the body. If the target's soul is not free or willing to do so, the spell fails.\\
The magic fashions a new body, which will likely cause the creature to change its race. The Storyteller rolls a d10 and consults the following table to determine what form the creature takes when brought back to life, or the Storyteller chooses the form.\\

\medskip
\begin{tabular}{ll}
\textbf{d10} &\textbf{Race}\\
\toprule
0 & Wolf/Eagle/Fox/Lynx (roll 1d4)\\
1&Nano\\
2&Elf\\
3&Half-Elf\\
4&Half-Orc\\
5&Boar/Badger/Dog/Rat (roll 1d4)\\
6&Nibali\\
7&Bear/Owl/Raccoon/Cat (roll 1d4)\\
8&Human\\
9&Same previous breed\\
\end{tabular}

The reincarnated creature remembers its past life and experiences (same AC and CM scores, skills and proficiencies). He retains the abilities he had in his original form if he is able to apply them.\\
\textbf{This spell is not available except to Devotees and Followers of Shayalia or Efrem}\\
\emph{Note}: a Devotee or Follower of Shayalia or Ephrem will always reincarnate the creature as an animal, but can choose the type.\\
It is not possible to be reincarnated as a gnome if you were not a gnome before.

\medskip\textbf{Resistance}\index[Spells]{Cantrip - Resistance}\\
\textbf{School}: Abjuration\\
\textbf{Level}: 0, Municipality\\
\textbf{Cast Time}: 1 Reaction\\
\textbf{Range}: Contact\\
\textbf{Components}: V, S, M (a miniature cloak)\\
\textbf{Duration}: Instant\\
You cast the spell on contact with a willing creature. Once before the spell ends, the target can roll a d4 and add the result to a saving throw of its choice. He can roll the die before or after making the saving throw. Then the spell ends.\\
\textbf{For each Magical Critical Success obtained} in the Magic Test you can make another creature benefit from the bonus.

\medskip\textbf{Breathe Under Water}\index[Spells]{Breathe Under Water}\\
\textbf{School}: Water, Air\\
\textbf{Level}: 3, Municipality\\
\textbf{Launch Time}: 2 Actions\\
\textbf{Range}: 9 metres\\
\textbf{Components}: V, S, M (a straw or a straw)\\
\textbf{Duration}: 24 hours\\
This spell allows up to ten willing creatures within range that you can see to breathe underwater until the spell ends. Subjected creatures also retain their normal method of breathing.\\
\textbf{For each Magical Critical Success obtained} in the Magic Test you can choose an additional creature.

\medskip\textbf{Raise Dead}\index[Spells]{Raise Dead}\\
\textbf{School}: Necromancy\\
\textbf{Level}: 5, Legendary\\
\textbf{Launch Time}: 1 hour\\
\textbf{Range}: Contact\\
\textbf{Components}: V, S, M (a diamond worth at least 500 gp, which the spell consumes)\\
\textbf{Duration}: Instant\\
You bring a dead creature back to life, as long as it hasn't been dead for more than 10 days. If the creature's soul is both willing and free to rejoin the body, the creature returns to life with 1 hit point.\\
This spell also neutralizes any poison and cures nonmagical illnesses that afflicted the creature at the time of death. This spell, however, does not remove magical diseases, curses, or similar effects; if these are not removed before casting the spell, they will resume manifesting when the creature returns to life. The spell cannot bring an undead creature back to life.\\
This spell closes all mortal wounds, but does not restore missing body parts. If the creature is missing body parts or organs critical to survival (the head, for example) the spell automatically fails.\\
Coming back from the dead is an ordeal. The target takes a -4 penalty on all attack rolls, saving throws, and ability checks. Each time the target finishes a night's rest the penalty is reduced by 1 until it disappears.\\
\textbf{This spell should not be available. Only a Patron can bring back to life.}

\medskip\textbf{Regeneration}\index[Spells]{Regeneration}\\
\textbf{School}: Transmutation\\
\textbf{Level}: 7, Legendary\\
\textbf{Launch Time}: 1 minute\\
\textbf{Range}: Contact\\
\textbf{Components}: V, S, M (a rosary and Holy Water)\\
\textbf{Duration}: 1 hour\\
You cast the spell on contact with a creature to stimulate its natural healing ability. The target regains 4d8 + 15 hit points. For the spell's duration, the target regains 1 hit point at the start of each of its rounds (6 hit points per minute). The target's severed body limbs (fingers, legs, tails, etc.), if it has any, are restored in 2 minutes. If you have the severed part and hold it against the stump, the spell causes the limb to mend itself with the stump in 3 rounds.\\
\textbf{For each Magical Critical Success obtained} in the Magic Test you double the Hit Points recovered per round.

\medskip\textbf{Remove Disease}\index[Spells]{Remove Disease}\label{rimuovimalattie}\hypertarget{remove disease}{} \\
\textbf{School}: Care\\
\textbf{Level}: 4, Municipality\\
\textbf{Cast Time}: 1 turn\\
\textbf{Range}: Contact\\
\textbf{Components}: V, S\\
\textbf{Duration}: Instant\\
You can put an end to a natural disease. In the case of magical diseases your spell DC must be higher than the disease DC.

\textbf{For each Magical Critical Success obtained} in the Magic Test you can heal one more person or consider a +4 to overcome the DC of the disease.

\medskip\textbf{Remove Curse}\index[Spells]{Remove Curse}\\
\textbf{School}: Abjuration\\
\textbf{Level}: 3, Municipality\\
\textbf{Launch Time}: 2 Shares\\
\textbf{Range}: Contact\\
\textbf{Components}: V, S\\
\textbf{Duration}: Instant\\
If the object or person has been cursed by the cast curse spell, or otherwise the Storyteller decides that the object has a particular curse, then the DC of the person casting remove curse must be higher than that of the curse.

\textbf{For each Magical Critical Success obtained} in the Magic Test you can heal one more person or consider a +4 to overcome the DC of the curse.

Whether it was sufficient to cast the spell or it was cast with a Magic Test, the curse remains, but the spell allows you to remove the object and throw it away.

\medskip\textbf{Remove Poison}\index[Spells]{Remove Poison}\label{incrimuoviveleno}\hypertarget{incrimepoison}\\
\textbf{School}: Water, Care\\
\textbf{Level}: 3, Municipality\\
\textbf{Cast Time}: 1 round\\
\textbf{Range}: Contact\\
\textbf{Components}: V, S\\
\textbf{Duration}: Instant\\
The target affected by the spell is no longer poisoned.

\textbf{For each Magical Critical Success obtained} in the Magic Test add +4 to your DC to understand if it has exceeded that of the poison.

\medskip\textbf{Rebirth}\index[Spells]{Rebirth}\\
\textbf{School}: Healing, Necromancy\\
\textbf{Level}: 3, Very Rare\\
\textbf{Launch Time}: 10 Minutes\\
\textbf{Range}: Contact\\
\textbf{Components}: V, S, M (diamond worth 300 gp, which the spell consumes)\\
\textbf{Duration}: Instant\\
A creature that died in the last minute and that you are in contact with returns to life with 1 hit point. This spell cannot bring back to life people who have died of old age, nor can it restore missing body parts.\\
The creature brought back to life must make a Fortitude save at DC 15 or it will not return to life due to the trauma it has suffered; if it does return to life it is fatigued 3.\\
\textbf{Note}: at the Storyteller's discretion this could be the only spell allowed to bring a creature back to life, otherwise the rule applies that only a Patron can bring it back to life.

\medskip\textbf{Repair}\index[Spells]{Cantrip - Repair}\\
\textbf{School}: Earth\\
\textbf{Level}: 0, Municipality\\
\textbf{Launch Time}: 1 minute\\
\textbf{Range}: Contact\\
\textbf{Components}: V, S, M (two magnets)\\
\textbf{Duration}: Instant\\
This spell repairs a single break or split in an object you touch, such as a broken chain, two halves of a broken key, a tattered cloak, or a leaky waterskin. As long as the break or split is no larger than 30 centimeters in any dimension, you are able to repair it, leaving no trace of the damage suffered. This spell can physically repair a magical item or construct, but cannot restore the magical functions of these items.

\medskip\textbf{Inviolate Rest}\index[Spells]{Inviolate Rest}\\
\textbf{School}: Necromancy\\
\textbf{Level}: 2, Uncommon\\
\textbf{Launch Time}: 2 Actions\\
\textbf{Range}: Contact\\
\textbf{Components}: V, S, M (a pinch of salt and a piece of copper placed on each eye of the corpse, which must remain there for the duration)\\
\textbf{Duration}: 10 days\\
You come into contact with a corpse or other remains. For the duration, the target is protected from rot and cannot become undead. \\
\textbf{For each Magical Critical Success obtained} in the Magic Test you double the duration up to a maximum of one year.

\medskip\textbf{Uncontainable Laughter}\index[Spells]{Uncontainable Laughter}\\
\textbf{School}: Enchantment\\
\textbf{Level}: 1, Uncommon\\
\textbf{Launch Time}: 2 Actions\\
\textbf{Range}: 9 metres\\
\textbf{Components}: V, S, M (small cakes and a feather that is waved in the air)\\
\textbf{Duration}: 1 minute
A creature that you choose within range and that you can see perceives everything as tremendously hilarious and hilarious, roaring with laughter as long as it is under this spell. The target must succeed on a Will save or fall prone, being incapacitated and unable to get up for the duration. Creatures with an Intelligence score of -2 or less ignore the effect.\\
At the end of each of its rounds and whenever it takes damage, the target can make another Will saving throw. The target has +1d6 to the saving throw if it took damage in the round. If he succeeds, the spell ends.

\medskip\textbf{Heat Metal}\index[Spells]{Heat Metal}\\
\textbf{School}: Fire\\
\textbf{Level}: 2, Uncommon\\
\textbf{Launching Time}: 2 Actions\\
\textbf{Range}: 18 metres\\
\textbf{Components}: V, S, M (a piece of iron and a flame)\\
\textbf{Duration}: 1 minute\\
Choose a metal artifact, such as a metal weapon or medium or heavy metal armor, that is within range and that you can see. Make the object glow red from the heat. Any creature in physical contact with the object takes 1d8 fire damage when you cast this spell. Until the spell ends, you can use 2 Actions to deal this damage again in your subsequent rounds.\\
If a creature is holding or wearing the item and takes damage from it, the creature must succeed at a Fortitude save or throw the item away if it is able. If he does not throw the object, he has -1d6 on attack rolls and ability checks until the start of his next round. As long as the creature is within 60 feet of the caster, the spell does not end but the object stops being hot.\\
\textbf{For each Magic Critical Success obtained} in the Magic Test the damage increases by 1d8.

\medskip\textbf{Lower Restoration}\index[Spells]{Lower Restoration}\\
\textbf{School}: Care\\
\textbf{Level}: 2, Municipality\\
\textbf{Launch Time}: 2 Actions\\
\textbf{Range}: Contact\\
\textbf{Components}: V, S\\
\textbf{Duration}: Instant\\
You can end a nonmagical disease or condition that afflicts a creature you are in contact with. The condition can be blinded, deafened or paralyzed. Can reduce your Fatigue level by one degree. You recover 2d6 maximum hit points lost, but do not increase your current hit points. You can regain 1 lost Characteristic point non-permanently.

At the Storyteller's discretion, if the condition was caused by magic, the Magical Expertise value of the caster must exceed the Magical Expertise of the person causing the effect.
If necessary, you can make a Magic Test and for each critical success you add 6 to your Magical Expertise to see if you are able to remove the effect.

\medskip\textbf{Superior Restoration}\index[Spells]{Superior Restoration}\\
\textbf{School}: Care\\
\textbf{Level}: 5, Uncommon\\
\textbf{Launch Time}: 2 Actions\\
\textbf{Range}: Contact\\
\textbf{Components}: V, S, M (diamond dust worth at least 100 gp, which the spell consumes)\\
\textbf{Duration}: Instant\\
Imbue a touching creature with positive healing energy to negate a debilitating effect.
At the Storyteller's discretion, if the condition was caused by magic, the Magical Expertise value of the caster must exceed the Magical Expertise of the person causing the effect.
If necessary, you can make a Magic Test and for each critical success you add 6 to your Magical Expertise to see if you are able to remove the effect.


- An effect that Charmed the target.\\
- Make the target recover 2 points to a stat. Recover 1 point if the loss was permanent.\\
- Maximum hit points return to normal, but do not increase current hit points.\\
- You are able to alleviate Fatigue conditions by two ranks.

\medskip\textbf{Awakening}\index[Spells]{Awakening}\\
\textbf{School}: Animals and Plants\\
\textbf{Level}: 5, Rare\\
\textbf{Launch Time}: 8 hours\\
\textbf{Range}: Contact\\
\textbf{Components}: V, S, M (an agate worth at least 1000 gp, which the spell consumes)\\
\textbf{Duration}: Instant\\
After spending your casting time drawing magical paths with a precious gem, you come into contact with a Huge or smaller beast or plant. The target must have no Intelligence score or have Intelligence -3 or less. The target gains Intelligence 0. The target also gains the ability to speak a language you know. If the target is a plant, it gains the ability to move its limbs, roots, vines, vines, and so on, and gains human-like senses. The Storyteller will choose statistics appropriate to the type of plant awakened, such as the statistics for awakened bush or awakened tree.\\
The awakened beast or plant is Charmed by you for 30 days or until you or your companions cause damage to it. When the Charmed condition ends, the awakened creature chooses whether to remain friendly to you, based on how you treated it while it was charmed.\\
\textbf{For each Magical Critical Success obtained} in the Magic Test you double the duration of the fascination up to a maximum of 1 year.

\medskip\textbf{Quick Retreat}\index[Spells]{Quick Retreat}\\
\textbf{School}: Transmutation\\
\textbf{Level}: 1, Uncommon\\
\textbf{Cast Time}: 1 Immediate Action\\
\textbf{Range}: Personal\\
\textbf{Components}: V, S\\
\textbf{Duration}: Concentration, 1 minute\\
This spell allows you to move at an incredible pace. When you cast this spell you gain a bonus Move Action.\\
\textbf{For each Magical Critical Success obtained} in the Magic Test the duration increases by 1 round.

\medskip\textbf{Skip}\index[Spells]{Skip}\\
\textbf{School}: Air\\
\textbf{Level}: 1, Municipality\\
\textbf{Launch Time}: 2 Shares\\
\textbf{Range}: Contact\\
\textbf{Components}: V, S, M (the hind leg of a grasshopper)\\
\textbf{Duration}: 1 minute\\
The jumping distance of the creature you are in contact with at the time of casting is tripled until the spell ends.

\medskip\textbf{Sanctify}\index[Spells]{Sanctify}\\
\textbf{School}: Invocation\\
\textbf{Level}: 5, Rare\\
\textbf{Launch Time}: 24 hours\\
\textbf{Range}: Contact\\
\textbf{Components}: V, S, M (herbs, oils and incense worth at least 1000 gp, which the spell consumes)\\
\textbf{Duration}: Until dissolved\\
Imbue the area surrounding a point you are in contact with with your Patron's power. The area can have a maximum radius of 60 feet, and the spell fails if it includes an area already under the effect of a sanctify spell. The area subject to the spell generates the following effects.
\emph{First things first}, celestials, elementals, fey, demons, and undead cannot enter the area, nor can such a creature charm, frighten, or possess others within it. Any creature charmed, frightened, or possessed by such a creature is no longer charmed, frightened, or possessed from the moment it enters this area. You can exclude one or more of these creature types from this effect.\\
\emph{Second thing}, you can bind an additional effect to the area. Choose the effect from the list below, or choose one presented to you by the Storyteller. Some of these effects apply to creatures in the area; you can decide whether the effects apply to all creatures, Devoted or Follower creatures of a specific Patron, or creatures of a specific type, such as orcs or trolls. When an affected creature enters this area for the first time in a round or begins its round here, it must make a Will saving throw. On a success, the creature ignores the additional effect until it leaves the area.\\

- \emph{Courage}. Subjected creatures cannot be frightened while in this area. Extradimensional interference. Subject creatures cannot move or travel using teleportation or other extradimensional or interplanar means.\\
\emph{Languages}. Subject creatures can communicate with any other creature in the area, even if they don't share a common language.\\
- \emph{Daylight}. Bright light fills the area. Magical darkness created by spells of a lower level than that used to cast this spell cannot extinguish the light. The duration in this case is one week.\\
- \emph{Darkness}. Darkness fills the area. Normal light, and even magical light created by spells of a lower level than that used to cast this spell, cannot illuminate the area.\\
- \emph{Fear}. Subject creatures are frightened while remaining in this area.\\
- \emph{Energy Protection}. Affected creatures gain resistance to one damage type of your choice (except bludgeoning, piercing, or slashing damage) for as long as they remain in the area.\\
- \emph{Inviolate Rest}. Dead bodies buried in the area cannot be transformed into undead.\\
- \emph{Silence}. No sound can emanate from within the area, and no sound can enter it.\\
- \emph{Energy Vulnerability}. Affected creatures gain vulnerability to a damage type of your choice (except bludgeoning, piercing, or slashing damage) for as long as they remain in the area.

\medskip\textbf{Shrine}\index[Spells]{Shrine}\\
\textbf{School}: Abjuration\\
\textbf{Level}: 1, Municipality\\
\textbf{Cast Time}: 1 Immediate Action\\
\textbf{Range}: 9 metres\\
\textbf{Components}: V, S, M (a small silver mirror)\\
\textbf{Duration}: 1 minute\\
Protect a creature within range from attacks. Until the spell ends, any creature that targets the protected creature with a harmful attack or spell must first make a Will saving throw. If the saving throw fails, the attacker must choose a new target or lose the attack or spell. This spell does not protect the protected creature from area effects, such as a fireball blast. If the protected creature makes an attack or casts a spell that affects enemy creatures, the spell ends.

\medskip\textbf{Private Shrine}\index[Spells]{Private Shrine}\\
\textbf{School}: Abjuration\\
\textbf{Level}: 4, Very Rare\\
\textbf{Launch Time}: 10 minutes\\
\textbf{Range}: 36 metres\\
\textbf{Components}: V, S, M (a thin sheet of lead, a piece of opaque glass, a cotton ball or fabric, and powdered chrysolite)\\
\textbf{Duration}: 24 hours \\
Protect an area with magic. The area is a cube that can be as small as 1 meter of edge or as large as 30 meters of edge. The spell lasts until the duration ends or you use an action to end it.\\
When you cast the spell, you decide what type of protection it provides, choosing one or more of the following properties:\\

- Sound cannot cross the perimeter of the protected area.\\
- The perimeter of the protected area appears dark and foggy, making it impossible to see through it (even with darkvision).\\
- Sensors created by divination spells cannot appear within the protected area or pass through its perimeter barrier.\\
- Creatures in the area cannot be targeted by divination spells.\\
- Nothing can teleport into or out of the protected area.\\
- Within the protected area, planar travel is prohibited.\\

Casting this spell on the same spot every day for a year makes the effect permanent.\\
\textbf{For each Magical Critical Success obtained} in the Magic Test you can increase the size of the cube by 10 meters of edge or increase the duration by 12 hours.

\medskip\textbf{Cast Curse}\index[Spells]{Cast Curse}\\
\textbf{School}: Necromancy\\
\textbf{Level}: 3, Uncommon\\
\textbf{Launch Time}: 2 Actions\\
\textbf{Range}: Contact\\
\textbf{Components}: V, S\\
\textbf{Duration}: 1 minute\\
A creature you touch must succeed on a Will save or be cursed for the spell's duration. When you cast this spell, choose the nature of the curse from the following options:\\

- Choose an ability score. While cursed, the target has -1d6 on ability checks and saving throws based on that ability score if applicable.\\
- While cursed, the target has -1d6 on attack rolls against you.\\
- While cursed, the target must make a Will saving throw at the start of each of its rounds. If he fails, he wastes the actions of that round doing nothing.\\
- While the target is cursed, your attacks and spells deal an additional 1d8 Void damage against it.\\

The remove curse spell (see description) ends this effect. At the Storyteller's discretion, you may choose a curse with a different effect, but it should not be more powerful than those described above. The Storyteller holds the final judgment on a curse's effect.\\
\textbf{If you get a critical} the duration of the curse is one day. If you get 3 criticals the duration is permanent.

\medskip\textbf{Cast Lesser Curse}\index[Spells]{Cast Lesser Curse}\\
\textbf{School}: Universal\\
\textbf{Level}: 1, Municipality\\
\textbf{Launch Time}: 2 Shares\\
\textbf{Range}: Contact\\
\textbf{Components}: V, S\\
\textbf{Duration}: 1 minute\\
A creature you touch must succeed on a Will save or be cursed for the spell's duration. When you cast this spell, choose the nature of the curse from the following options:\\

- Choose an ability score. While cursed, the target has -2 on ability checks and saving throws based on that ability score if applicable.\\
- While cursed, the target has -2 on attack rolls against you.\\
- While cursed, the target must make a Will saving throw at the start of each of its rounds. If he fails, he wastes 2 actions of his round without doing anything.

The remove curse spell (see description) ends this effect. At the Storyteller's discretion, you may choose a curse with a different effect, but it should not be more powerful than those described above. The Storyteller holds final judgment on a curse's effect.


\medskip\textbf{Lock-lock}\index[Spells]{Lock-lock}\\
\textbf{School}: Transmutation\\
\textbf{Level}: 2, Municipality\\
\textbf{Launch Time}: 2 Shares\\
\textbf{Range}: 18 metres\\
\textbf{Components}: V\\
\textbf{Duration}: Instant\\
Choose an object that is within range and that you can see. The object may be a door, box, handcuffs, lock, or other object that has a common or magical method of preventing access.\\
A target that is closed by a common lock or that is locked or barred is opened, unlocked, or freed. If the object has multiple locks, only one of them is opened.\\
If you choose a target that is kept closed with Magic Lock that spell is suppressed for 10 minutes, during which time the target can be opened as normal. When you cast this spell, a loud knock, audible up to 300 feet away, emanates from the target object.\\
\textbf{For each Magical Critical Success obtained} in the Magic Test you can open another padlock/lock within range.

\medskip\textbf{Meteor Swarm}\index[Spells]{Meteor Swarm}\hypertarget{meteor shower}{}\\
\textbf{School}: Fire, Earth\\
\textbf{Level}: 9, Legendary\\
\textbf{Launch Time}: 2 Shares\\
\textbf{Range}: 1.5 kilometres\\
\textbf{Components}: V, S\\
\textbf{Duration}: Instant\\
Glowing balls of fire crash to the ground at four different points within range and that you can see. Each creature in a 2-meter radius sphere centered on the point you choose must make a Reflex saving throw. The sphere propagates around the corners. A creature takes 20d6 fire damage and 20d6 bludgeoning damage on a failed save, or half as much.
these damages if he exceeds it. A creature in the area of ​​more than one fireburst is affected only once.\\
\textbf{Saving Throw Success/Critical Failure}: On a critical failure the damage is doubled, on a critical success the damage is further halved\\
\textbf{For every 3 critical points obtained} in the Magic Test choose another point of impact.

\medskip\textbf{Stone Carving}\index[Spells]{Stone Carving}\\
\textbf{School}: Earth\\
\textbf{Level}: 4, Municipality\\
\textbf{Launch Time}: 2 Actions\\
\textbf{Range}: Contact\\
\textbf{Components}: V, S, M (malleable clay, which must be worked to obtain a vague shape of the stone object)\\
\textbf{Duration}: Instant\\
Carve into any shape that suits your purposes a stone object of Medium or smaller size, or a section of stone no larger than 3 feet thick in any direction, with which you are in contact.\\
So, for example, you could carve a large stone into a weapon, idol, or coffin, or create a small passage through the wall, as long as the wall is less than 3 feet thick. You could also fashion a stone door or its frame to seal the door. The object you create can have up to two hinges and a latch, but it is impossible to create more complex mechanisms.

\medskip\textbf{Discover the Path}\index[Spells]{Discover the Path}\\
\textbf{School}: Divination\\
\textbf{Level}: 6, Uncommon\\
\textbf{Launch Time}: 1 minute\\
\textbf{Range}: Personal\\
\textbf{Components}: V, S, M (some divination tools - ivory sticks, bones, cards, teeth, or engraved runes - worth at least 100 gp, and an item from the location you wish to find)\\
\textbf{Duration}: 1 day\\
This spell allows you to find the shortest and most direct physical route to a specific fixed location that you are familiar with and is on the same plane of existence. If you indicate a destination on another plane of existence, a destination that moves (such as a moving fortress), or a nonspecific destination (such as \emph{a green dragon's lair}), the spell fails.\\
For the duration of the spell, as long as you are on the same plane of existence as the destination, you will know how far away it is and in what direction it is. As you travel to it, whenever you are presented with a choice of routes, you will automatically determine which is the shortest and most direct (but not necessarily the safest) route to your destination.\\
\textbf{For each critical} obtained in the Magic Test the spell lasts 8 hours longer.

\medskip\textbf{Discover Traps}\index[Spells]{Discover Traps}\\
\textbf{School}: Divination\\
\textbf{Level}: 2, Municipality\\
\textbf{Launch Time}: 2 Shares\\
\textbf{Range}: 36 metres\\
\textbf{Components}: V, S\\
\textbf{Duration}: 10 minutes\\
For the duration of the spell you sense the presence of any trap within your line of sight. A trap, for the purposes of this spell, includes anything that is capable of inflicting a sudden or unexpected effect that you may consider harmful or undesirable, and that was expressly intended as such by its creator. As a result, the spell would sense an area under the alarm spell, a glyph of ward, or a mechanical trap door, but would not reveal a natural weakness in the floor, an unstable ceiling, or a hidden hole.\\
The trap is highlighted in your sight with a purple signal.

\medskip\textbf{Secret Chest}\index[Spells]{Secret Chest}\\
\textbf{School}: Summon\\
\textbf{Level}: 4, Rare\\
\textbf{Launch Time}: 2 Shares\\
\textbf{Range}: Contact\\
\textbf{Components}: V, S, M (a crafted chest, 1 meter x 50 cm x 50 cm, constructed of rare materials worth at least 5000 gp, and a Tiny replica of it made of the same materials and worth at least 50) \\
\textbf{Duration}: Instant\\
Hide a chest and all its contents on the Ethereal Plane. When you cast this spell you must be in contact with the chest and the miniature replica that serves as the material component. The chest can hold up to 0.25 cubic meters of non-living material (1 x meter x 50 centimeters x 50 centimeters). While the chest remains on the Ethereal Plane, you can use an action to contact the replica and recall the chest. It will respawn in an unoccupied space on the ground within 1 meter of you. You can send the chest back to the Ethereal Plane, using an action and making contact with both the chest and the replica.\\
After 60 days, there is a cumulative rate of 5% per day for the spell's effect to end. \\
The effect ends if the spell is cast again, if the replica chest is destroyed, or if you decide to end the spell with an action. If the spell ends and the chest is on the Ethereal Plane, it is irretrievably lost.

\medskip\textbf{Illusory Writing}\index[Spells]{Illusory Writing}\\
\textbf{School}: Illusion\\
\textbf{Level}: 1, Municipality\\
\textbf{Launch Time}: 1 minute\\
\textbf{Range}: Contact\\
\textbf{Components}: S, M (a lead-based ink worth at least 10 gp, which the spell consumes)\\
\textbf{Duration}: 10 days\\
You write on a scroll, a piece of paper, or some other writing material and imbue it with a powerful illusion that lasts for the duration of the spell.\\
To you and any creatures you point to when you cast the spell, the writing appears normal, in your handwriting, and conveys whatever meaning you intended to convey when you wrote the text. For everyone else, the writing appears as if it were written in an unknown or magical script, which is incomprehensible. Alternatively, you can make the writing appear to be a totally different message, in a different handwriting and language, although it must be a language you are familiar with.\\
If the spell is dispelled, both the original writing and the illusion vanish. A creature with true seeing can read the hidden message.

\medskip\textbf{Scry}\index[Spells]{Scry}\\
\textbf{School}: Divination\\
\textbf{Level}: 5, Rare\\
\textbf{Launch Time}: 10 minutes\\
\textbf{Range}: Personal\\
\textbf{Components}: V, S, M (a focus worth at least 1000 gp, such as a crystal ball, a silver mirror, or a spring filled with Holy Water)\\
\textbf{Duration}: Concentration, maximum 10 minutes\\
You can see and hear a particular creature of your choice that is on the same plane of existence as you. The target must make a Will save, modified by how well you know the target and your physical connection to it. If the target knows you are casting the spell, it can voluntarily fail the saving throw, should it wish to be observed by
you.

\medskip

\begin{tabular}{ll}
\toprule
\textbf{Knowledge} & \textbf{Mod. to TS}\\
Have you heard about it &+5\\
You met the target &+0\\
You know the target well &-5\\
\end{tabular}

\begin{tabular}{ll}
\toprule
\textbf{Connection} & \textbf{Mod. TS}\\
Description or image &-2\\
Property or garment & -4\\
Part of the body (hair...)&-10\\
\end{tabular}

\medskip

On a successful save, the target ignores the spell's effects, and you cannot use this spell against it again for 24 hours.\\
If the saving throw fails, the spell creates an invisible sensor within 10 feet of the target. Through the sensor you can hear and see as if you were on site. The sensor moves with the target, remaining within 10 feet of it for the spell's duration. A creature that can see invisible objects sees the sensor as a glowing sphere about the size of a fist.\\
Instead of targeting a creature, you can target a location you've seen before. When you choose this option, the sensor appears in that location but does not move.

\medskip\textbf{Shield}\index[Spells]{Shield}\\
\textbf{School}: Abjuration\\
\textbf{Level}: 1, Municipality\\
\textbf{Casting Time}: 1 Reaction, which you take when you are hit by an attack or target of the Arcane Bolt spell\\
\textbf{Range}: Personal\\
\textbf{Components}: V, S\\
\textbf{Duration}: 1 round\\
A barrier of invisible magical force appears to protect you. Until the start of your next round you have a +2 bonus to Defense including the trigger attack, and you take no damage from Arcane Bolt and Occult Bolt.\\
\textbf{For each Magical Critical Success obtained} in the Magic Test you increase the duration by 1 round.

\medskip\textbf{Shield of Faith}\index[Spells]{Shield of Faith}\\
\textbf{School}: Abjuration\\
\textbf{Level}: 1, Municipality\\
\textbf{Cast Time}: 1 Immediate Action\\
\textbf{Range}: 18 metres\\
\textbf{Components}: V, S, M (a small parchment with a fragment of sacred text written on it)\\
\textbf{Duration}: 10 minutes\\
A shimmering field appears surrounding a creature you choose within range, granting it a +2 bonus to Defense for the spell's duration.
\textbf{For each Magical Critical Success obtained} in the Magic Test you influence another creature.

\medskip\textbf{Fire Shield}\index[Spells]{Fire Shield}\\
\textbf{School}: Fire, Water\\
\textbf{Level}: 4, Uncommon\\
\textbf{Launch Time}: 2 Shares\\
\textbf{Range}: Personal\\
\textbf{Components}: V, S, M (a bit of phosphorus or a firefly) \\
\textbf{Duration}: 10 minutes\\
Thin, vaporous flames envelop your body for the duration of the spell, emitting bright light in a 10-foot radius and dim light for an additional 10 feet. You can end the spell early, using an action to end it.\\
Flames give you a hot shield or a cold shield, your choice. The hot shield grants you resistance to cold damage, while the cold shield grants you resistance to heat damage.\\
Additionally, whenever a creature within 3 feet of you hits you with a melee attack, the shield erupts in flame. The attacker takes 2d8 fire damage from a hot shield, or 2d8 cold damage from a cold shield.

\medskip\textbf{Darkvision}\index[Spells]{Darkvision}\\
\textbf{School}: Transmutation\\
\textbf{Level}: 2, Municipality\\
\textbf{Launching Time}: 2 Actions\\
\textbf{Range}: Contact\\
\textbf{Components}: V, S, M (or a pinch of carrot or a dried blueberry)\\
\textbf{Duration}: 1 hour of real game time\\
A willing creature you are in contact with gains the ability to see in the dark. For the duration of the spell, that creature has darkvision up to a range of 30 feet.\\
\textbf{For each Magical Critical Success obtained} in the Magic Test you double the duration.


\medskip\textbf{Loyal Hound}\index[Spells]{Loyal Hound}\\
\textbf{School}: Summon\\
\textbf{Level}: 4, Rare\\
\textbf{Launching Time}: 2 Actions\\
\textbf{Range}: 9 metres\\
\textbf{Components}: V, S, M (a tiny silver whistle, and a piece of bone and a thread)\\
\textbf{Duration}: 8 hours\\
You can summon a phantom guard dog to an unoccupied space within range of which you can see, where it will remain for the duration of the spell, until it is dismissed as an action, or until it moves more than 100 feet away from you.\\
The hound is invisible to all creatures except you and cannot be harmed. When a Small or larger creature approaches within 30 feet of it without first speaking the password you specified when you cast the spell, the hound begins barking loudly. The hound sees invisible creatures and can see into the Ethereal Plane. It ignores illusions. At the start of each of your rounds, the hound attempts to bite a creature within 3 feet of it that is hostile to you. The hound's attack bonus is equal to your spell ability modifier + CM. On a hit, it deals 2d8 piercing damage.

\medskip\textbf{Seem}\index[Spells]{Seem}\\
\textbf{School}: Illusion\\
\textbf{Level}: 5, Uncommon\\
\textbf{Launch Time}: 2 Shares\\
\textbf{Range}: 9 metres\\
\textbf{Components}: V, S\\
\textbf{Duration}: 8 hours\\
This spell allows you to change the appearance of any number of creatures within range and that you can see. Give each target a new illusory appearance. An unwilling creature can make a Will saving throw and ignores the spell on a successful one.\\
The spell disguises one's physical appearance as well as clothing, armor, weapons, and equipment. You can make each creature look a foot shorter or taller, look thin, fat, or somewhere in between. You cannot change the shape of the target's body, and so you must choose a shape that has the same basic distribution of limbs. \\
For everything else, the illusion is limited only by your imagination. The spell lasts for its duration, unless you use an action to end it first. The changes brought about by this spell are not capable of withstanding physical inspection. For example, if you use this spell to add a hat to a creature's clothing, objects pass through the hat, and anyone who touches it would feel nothing and would end up touching the creature's head and hair.
If you use this spell to appear thinner than you are, the hand of a person who tries to touch you will bounce off you, while at first sight it appears to stop in mid-air. A creature can use 2 Actions to inspect a target and make an Awareness check against the spell's saving throw DC, if it takes 3 Actions it has a +1d6 bonus. If he succeeds, he understands that the target is disguised.

\medskip\textbf{Demiplane}\index[Spells]{Demiplane}\\
\textbf{School}: Summon\\
\textbf{Level}: 8, Rare\\
\textbf{Launch Time}: 2 Shares\\
\textbf{Range}: 18 metres\\
\textbf{Components}: Y\\
\textbf{Duration}: 1 hour\\
You create a shadow door on a flat surface within range and that you can see. The door is large enough for a Medium creature to pass through easily. When opened, the door leads to a demiplane that appears as an empty 30-foot room in each dimension, made of wood and stone. When the spell ends, the door disappears, and any creatures or objects within the demiplane become trapped there, while the door disappears on the other side as well.\\
Each time you cast this spell, you create a new demiplane, or you allow the shadow door to connect to a demiplane created by a previous casting of the spell, or you increase a known demiplane you previously created by another 30 feet in each dimension. \\
Additionally, if you know the nature and contents of a demiplane created by another creature's casting of this spell, you can cause the shadow door to connect to that demiplane instead.

\medskip\textbf{Magic Lock}\index[Spells]{Magic Lock}\\
\textbf{School}: Abjuration\\
\textbf{Level}: 2, Municipality\\
\textbf{Launch Time}: 2 Shares\\
\textbf{Range}: Contact\\
\textbf{Components}: V, S, M (gold dust worth at least 25 gp, which is consumed by the spell) \\
\textbf{Duration}: Until dissolved\\
You cast the spell upon contact with a closed door, window, portal, chest, or other entrance, and it becomes locked for the duration. You and the creatures you indicated when you cast this spell can open the item normally. You can also set a password that, when spoken within 3 feet of the item, suppresses the spell for 1 minute. Otherwise the opening is impassable until it is destroyed or the spell is dispelled or suppressed. Casting lockpick on the item suppresses Magic Lock for 10 minutes.\\
While subject to this spell, the object is more difficult to destroy or force open; the DC to break it or pick a lock on it increases by 10.\\
\textbf{For each Magical Critical Success obtained} in the Magic Test you can influence another closure.

\medskip\textbf{Invisible Servant}\index[Spells]{Invisible Servant}\\
\textbf{School}: Summon\\
\textbf{Level}: 1, Municipality\\
\textbf{Launch Time}: 2 Shares\\
\textbf{Range}: 18 metres\\
\textbf{Components}: V, S, M (a piece of rope and a piece of wood)\\
\textbf{Duration}: 1 hour\\
This spell creates an almost invisible force only bordered by a light aura (of the color of your choice) that performs simple tasks at your command, until the spell ends. The minion forms in an unoccupied space on the field within range. He has Defense 10, 1 hit point, Strength 0 and cannot attack. If he drops to 0 hit points, the spell ends.
As an immediate action, during each of your rounds, you can mentally command the minion to move up to 10 feet and interact with an object. The servant can perform simple tasks like a human servant, such as gathering things, cleaning, repairing, folding clothes, lighting fires, serving food, and pouring wine. Once you give the command, the minion will perform the task to the best of its ability until it is completed, and then wait for your next command. \\
If you command the servant to perform a task that will cause it to move more than 60 feet away from you, the spell ends.

\medskip\textbf{Freezing Sphere}\index[Spells]{Freezing Sphere}\\
\textbf{School}: Water\\
\textbf{Level}: 6, Rare\\
\textbf{Launch Time}: 2 Shares\\
\textbf{Range}: 90 metres\\
\textbf{Components}: V, S, M (a small crystal ball)\\
\textbf{Duration}: Instant\\
An icy orb of cold energy shoots from your fingertips to a point of your choosing within range, where it explodes into a sphere 60 feet in radius. Each creature in the area must make a Fortitude saving throw. On a failed save, a creature takes 10d6 cold damage. If he succeeds, he takes half this damage.\\
If the orb strikes a body of water or a liquid composed primarily of water (but not including water-based creatures), it freezes the liquid to a depth of 6 inches in a 30-foot square area. Ice lasts 1 minute. Creatures that were swimming on the surface of the frozen water become trapped in the ice. A trapped creature can use two actions to make a new saving throw to free itself.\\
If you wish, after completing the spell, you can refrain from shooting the orb. A small orb, about the size of a slingstone, cold to the touch, appears in your hand. At any time, you, or a creature you give the orb to, can throw the orb (up to a range of 40 feet). This will shatter on impact, with the same effect as normal spell casting. You can also place the globe on the ground without it shattering. After 1 minute, if the orb has not already been shattered, it will explode.\\
\textbf{For each Magic Critical Success obtained} in the Magic Test the damage increases by 1d6.\\
\textbf{Saving Throw Success/Critical Failure}: In case of a critical failure the damage is doubled, in case of a critical success the damage is further halved

\medskip\textbf{Elastic Sphere}\index[Spells]{Elastic Sphere}\\
\textbf{School}: Invocation\\
\textbf{Level}: 4, Rare\\
\textbf{Launch Time}: 2 Shares\\
\textbf{Range}: 90 metres\\
\textbf{Components}: V, S, M (a hemispherical piece of transparent crystal and a corresponding hemispherical piece of gum arabic)\\
\textbf{Duration}: Concentration, maximum 1 minute\\
A sphere of glowing energy envelops a creature or object of Large or smaller range. An unwilling creature must make a Reflex saving throw. On a failed save, the creature is enchanted for the duration.\\
Nothing (not physical objects, not energy, not other spell effects) can pass through this barrier, in or out, although a creature within the sphere can breathe without difficulty. The sphere is immune to all damage, and a creature inside it cannot be harmed by attacks or effects originating from outside it, nor can a creature inside the sphere damage anything outside it. The sphere is weightless and just large enough to hold the creature or object inside. An enveloped creature can use 1 Action to push against the walls of the sphere and then roll it up to half the creature's speed. Likewise, the orb can be picked up and moved by other creatures.\\
A disintegrate spell that targets the orb destroys it without damaging anything inside.

\medskip\textbf{Fiery Sphere}\index[Spells]{Fiery Sphere}\\
\textbf{School}: Fire\\
\textbf{Level}: 2, Municipality\\
\textbf{Launch Time}: 2 Shares\\
\textbf{Range}: 18 metres\\
\textbf{Components}: V, S, M (a little tallow, a pinch of sulfur and a handful of powdered iron)\\
\textbf{Duration}: 1 minute\\
For the duration of the spell, a sphere 3 feet in diameter appears in a space within range, chosen by you. Any creature that ends its round within 3 feet of the sphere must make a Reflex saving throw. The creature takes 2d6 fire damage on a failed save, or half as much damage on a successful one.\\
As an action you can move the sphere 30 feet. If you crash the sphere into a creature, the creature must make a saving throw against the sphere's damage, and the sphere will stop moving for that round.
When you move the sphere, you can move it over barriers up to 1 meter high, and make it jump spaces up to 3 meters wide. The sphere ignites flammable objects not being worn or carried, and radiates bright light in a 10-foot radius and dim light for an additional 10 feet.\\
While you have this spell active you are distracted when casting other spells.\\
\textbf{For each Magic Critical Success obtained} in the Magic Test the damage increases by 1d6.

\medskip\textbf{Blur}\index[Spells]{Blur}\\
\textbf{School}: Illusion\\
\textbf{Level}: 2, Municipality\\
\textbf{Launch Time}: 2 Shares\\
\textbf{Range}: Personal\\
\textbf{Components}: V\\
\textbf{Duration}: 1 minute \\
Your body becomes blurry, indistinct and shaky to anyone who sees you. For the duration of the spell, all creatures have -1d6 on attack rolls against you. Attackers who do not rely on sight are immune to this effect, for example if they have blindsight or are able to distinguish illusions, such as true seeing.

\medskip\textbf{Piercing Gaze}\index[Spells]{Piercing Gaze}\\
\textbf{School}: Necromancy\\
\textbf{Level}: 6, Very Rare\\
\textbf{Launch Time}: 2 Shares\\
\textbf{Range}: Personal\\
\textbf{Components}: V, S\\
\textbf{Duration}: Concentration, maximum 1 minute\\
For the duration of the spell, your eyes transform into a black void infused with terrible power. A creature of your choice within 60 feet of you that you can see must succeed on a Will save or suffer one of the following effects of your choice for the duration. During each of your rounds, until the spell ends, you can use two Actions to target another creature, but you cannot again target a creature that succeeds at a saving throw against this piercing gaze cast.\\

- \emph{Asleep}. The target falls unconscious. It awakens if it takes any amount of damage or if another creature uses 2 Actions to rouse it from sleep.\\
- \emph{Sick}. The target has -1d6 on attack rolls and ability checks. At the end of each of his rounds, he can make another Will saving throw. If it succeeds, the effect ends.\\
- \emph{Panicked}. The target is scared of you. During each of its rounds, the frightened creature must use two Move Actions and move away from you by the shortest and safest route possible, unless it has no room to move. If the target moves to a location at least 60 feet away from you, where it cannot see you, this effect ends.

\medskip\textbf{Silence}\index[Spells]{Silence}\\
\textbf{School}: Illusion\\
\textbf{Level}: 2, Municipality\\
\textbf{Launch Time}: 2 Shares\\
\textbf{Range}: 36 metres\\
\textbf{Components}: V, S\\
\textbf{Duration}: 10 minutes\\
For the duration of the spell, no sound can be created within or through a 20-foot-radius sphere centered on a point you choose within range. Any creature or object that is completely inside the sphere is immune to sonic damage, and creatures that are completely inside it are deafened. It is impossible to cast a spell that includes a verbal component while inside it.\\
\textbf{For each Magical Critical Success obtained} in the Magic Test the duration doubles.

\medskip\textbf{Symbol}\index[Spells]{Symbol}
\textbf{School}: Abjuration\\
\textbf{Level}: 7, Uncommon\\
\textbf{Launch Time}: 2 Shares\\
\textbf{Range}: Contact\\
\textbf{Components}: V, S, M (mercury, phosphorus, and powdered diamond and opal with a total value of at least 1000 gp, which the spell consumes)\\
\textbf{Duration}: Until dispelled or activated\\
When you cast this spell, you inscribe a harmful glyph on a surface (such as a section of floor, wall, or table) or inside an object that can be closed to hide the glyph (such as a book, scroll, or chest ). If you choose a surface, the glyph can cover a surface area no larger than 10 feet in diameter. If you choose an object, that object must stay in place; if the object is moved more than 10 feet from where the spell was cast, the glyph is broken, and the spell ends without being activated.\\
The glyph is nearly invisible and can be found with a Survival check against your spells' saving throw DC.\\
You decide what activates the glyph when the spell is cast.\\
For glyphs inscribed on a surface, typical activation includes touching or standing over the glyph, removing another object covering the glyph, moving within a certain distance of the glyph, or manipulating the object on which the glyph is inscribed. glyph.\\
For glyphs inscribed on an object, typical activation includes opening the object, moving within a certain distance of the object, or seeing or reading the glyph.\\
You can refine the activation so that the spell activates only under certain circumstances or according to certain physical peculiarities (such as height or weight) or species of creature (for example, the protection could act against hags or shapeshifters). You can also set conditions to prevent the glyph from being triggered, such as saying a password.\\
When inscribing the glyph choose one of the following options as its effect. Once activated, the glyph glows, filling a 60-foot sphere of dim light for 10 minutes, after which the spell ends. Any creature in the sphere when the glyph activates becomes the target of its effect, as does a creature that enters the sphere for the first time during a round or ends its round there.\\

- \emph{Dementia}. Each target must make a Will saving throw. If the saving throw fails, the target becomes demented for 1 minute. A demented creature cannot perform actions, does not understand what others say to it, cannot read, and speaks only in a slur. The Narrator controls their movements, which are erratic.\\
- \emph{Discord}. Each target must make a Fortitude saving throw. On a failed save, the target begins bickering and arguing with another creature for 1 minute. During this time, he is unable to make any meaningful communications and has -1d6 on attack rolls and ability checks. Ache. Each target must make a Fortitude saving throw. On a failed save, the target is incapacitated by searing pain.\\
- \emph{Death}. Each target must make a Fortitude saving throw, taking 10d10 void damage on a failed save, or half as much damage on a successful one.\\
- \emph{Fear}. Each target must make a Will save, and be frightened for 1 minute on a failed save. While frightened, the target drops whatever it is holding and must move at least 30 feet away from the glyph on each of its rounds, if able.\\
- \emph{No Confidence}. Each target must make a Will saving throw. On a failed save, the target is overcome by despair for 1 minute. During this time, you can't attack or target any creatures with harmful abilities, spells, or other magical effects.\\
- \emph{Sleep}. Each target must make a Will saving throw, and fall unconscious for 10 minutes on a failed save. A creature awakens if it takes damage or if someone uses an action to awaken it.\\
- \emph{Stunning}. Each target must make a Will saving throw, and be stunned for 1 minute on a failed save.

\medskip\textbf{Simulacrum}\index[Spells]{Simulacrum}\\
\textbf{School}: Illusion\\
\textbf{Level}: 7, Rare\\
\textbf{Launch Time}: 12 hours\\
\textbf{Range}: Contact\\
\textbf{Components}: V, S, M (plenty of snow or ice to create a life-size copy of the duplicated creature; some hair, nails or other piece of that creature's body to place in the snow or to the ice; and a powdered ruby ​​worth 1,500 gp, scattered over the duplicate and consumed by the spell)\\
\textbf{Duration}: Until dissolved\\
You create an illusory duplicate of a beast or humanoid that remains within range for the spell's entire casting time. The duplicate is a creature, partly real and made of ice or snow, that can take actions and interact like a normal creature. It appears to be identical to the original, but has half the maximum hit points of that creature, half the Magical Proficiency and Weapon Proficiency scores, and comes without equipment. Otherwise, the illusion uses all the statistics of the creature it duplicates.\\
The simulacrum is friendly towards you and the creatures you indicate. It obeys commands you speak, moving and acting according to your wishes and acting during your round in combat. The simulacrum lacks the ability to learn or become more powerful, and therefore never increases in level or characteristics, nor can it regain spent spell slots.\\
If the simulacrum is damaged, you can repair it in an alchemical laboratory, using rare herbs and minerals worth 100 gp per hit point recovered. The simulacrum remains until it drops to 0 hit points, at which point it transforms back into snow and instantly melts. If you cast this spell again, any duplicates you create with this spell currently active are immediately destroyed.

\medskip\textbf{Dream}\index[Spells]{Dream}\\
\textbf{School}: Illusion\\
\textbf{Level}: 5, Uncommon\\
\textbf{Launch Time}: 2 Shares\\
\textbf{Range}: Special\\
\textbf{Components}: V, S, M (a handful of sand, a tip of ink, and a writing pen taken from a sleeping bird)\\
\textbf{Duration}: 8 hours\\
This spell shapes a creature's dreams. Choose a creature known to you as the target of the spell. The target must be on the same plane of existence as you. Creatures that don't sleep can't be affected by this spell. You or a willing creature you are in contact with enters a trance state, acting as a messenger. While in trance, the messenger is aware of his surroundings, but cannot take actions or move.\\
For the duration of the spell, if the target is asleep, the messenger appears in the target's dreams and can converse with the target as long as the target remains asleep. The messenger can also shape the dream environment, creating terrain, objects, and other images. The messenger can emerge from the trance at any time, ending the spell's effect early. Upon awakening, the target remembers his dream perfectly. If the target is awake when you cast the spell, the messenger becomes aware of this and can end the trance (and the spell) or wait for the target to fall asleep. The messenger may then appear in the target's dreams.\\
You can make the messenger appear monstrous and terrifying to the target. If you do so, the messenger can deliver a message of up to ten words and then the target must make a Will saving throw. On a failed save, the echoes of the frightening monstrosity generate a nightmare for the duration of the target's sleep, preventing the target from gaining any benefit from that rest. Additionally, when the target awakens, it takes 3d6 damage.\\
If you have a lock of hair, clipped nails, or similar portion of the target's body, he makes his saving throw with -1d6.

\medskip\textbf{Nap}\index[Spells]{Nap}\\
\textbf{School}: Alteration\\
\textbf{Level}: 2, Legendary\\
\textbf{Cast Time}: 1 round\\
\textbf{Range}: 6 metres\\
\textbf{Components}: V, S, M (a feather, a piece of white cotton)\\
\textbf{Duration}: 1 minute\\
This spell allows the caster to put up to 1 creature per Magical Expertise/4 to rest for 1 hour. The creature must be consenting.

This hour of rest is equivalent to 8 hours of rest when it comes to recovering Magic Points and Hit Points. You cannot benefit from the spell more than once in 36 hours.\\
\textbf{For each Magical Critical Success obtained} in the Magic Test you influence 1 more creature.

\medskip\textbf{Sleep}\index[Spells]{Sleep}\\
\textbf{School}: Enchantment\\
\textbf{Level}: 1, Municipality\\
\textbf{Launch Time}: 2 Shares\\
\textbf{Range}: 27 metres\\
\textbf{Components}: V, S, M (a pinch of sand, rose petals or a cricket)\\
\textbf{Duration}: 1 minute\\
This spell places creatures in a magical torpor. Roll 5d8 the total is the number of hit points of creatures on which the spell can act. Creatures within 20 feet of your chosen point within range are affected in ascending order of Hit Points (ignoring unconscious creatures).\\
Starting with the creature with the lowest current hit points, each creature affected by this spell falls unconscious until the spell ends, the sleeper takes damage, or someone uses an action to shake or slap the sleeper . Subtract each creature's hit points from the total before considering the creature with the next lowest hit point value. A creature's hit points must be equal to or less than the remaining total for the effect to affect it. Undead and creatures that cannot be charmed are not affected by this spell.\\
\textbf{For each Magical Critical Success achieved} in the Magic Test you affect 2d8 additional Hit Points.

\medskip\textbf{Arcane Sword}\index[Spells]{Arcane Sword}\\
\textbf{School}: Invocation\\
\textbf{Level}: 7, Rare\\
\textbf{Launch Time}: 2 Shares\\
\textbf{Range}: 18 metres\\
\textbf{Components}: V, S, M (a miniature platinum sword with a copper and zinc hilt and pommel, worth 250 gp)\\
\textbf{Duration}: Concentration, maximum 1 minute \\
For the spell's duration, you create a floating sword-shaped plane of force within range. When the sword appears, you make a melee attack with CM modifier + spell modifier against a target you choose within 1 meter of the sword. On a hit, the target takes 3d10 force damage. Until the spell ends, you can use an action each of your rounds to move the sword 20 feet to a point you can see and repeat this attack against the same or different target.

\medskip\textbf{Colored Spray}\index[Spells]{Colored Spray}\\
\textbf{School}: Illusion\\
\textbf{Level}: 1, Municipality\\
\textbf{Launch Time}: 2 Shares\\
\textbf{Range}: Personal (3 meter cone)\\
\textbf{Components}: V, S, M (a pinch of dust or sand that is colored red, yellow and blue)\\
\textbf{Duration}: 1 round\\
A burst of dazzling, colorful lights emits from your hand. Roll 6d10; the total is the amount of hit points of creatures this spell affects. Creatures in a 10-foot cone originating from you are affected in ascending order of their current Hit Points (ignoring unconscious creatures and creatures that can't see).\\
Starting with the creature that has the fewest current hit points, each creature subject to this spell is blinded until the spell ends. Subtract each creature's hit points from the total before moving on to the creature with the next lowest hit point total. A creature's hit points must be equal to or less than the remaining total for the spell to affect it. \\
\textbf{For each Magical Critical Success obtained} in the Magic Test roll 1d10 additional Hit Points.

\medskip\textbf{Prismatic Spray}\index[Spells]{Prismatic Spray}\\
\textbf{School}: Invocation\\
\textbf{Level}: 7, Rare\\
\textbf{Launch Time}: 2 Shares\\
\textbf{Range}: Personal (18 meter cone)\\
\textbf{Components}: V, S\\
\textbf{Duration}: Instant\\
Eight rays of multicolored light shoot out from your hand. Each ray is a different color and has a different power and purpose. Each creature in a 60-foot cone must make a Reflex saving throw. For each target, roll a d8 to determine the color of the beam that hit it.\\

- \emph{1. Red}. The target takes 10d6 fire damage on a failed save, or half as much damage on a successful one.\\
- \emph{2. Orange}. The target takes 10d6 acid damage on a failed save, or half as much damage on a successful one.\\
- \emph{3. Yellow}. The target takes 10d6 lightning damage on a failed save, or half as much damage on a successful one.\\
- \emph{4. Green}. The target takes 10d6 poison damage on a failed save, or half as much damage on a successful one.\\
- \emph{5. Blue}. The target takes 10d6 cold damage on a failed save, or half as much damage on a successful one.\\
- \emph{6. Indigo}. If the saving throw fails, the target is restrained. He must then make a Fortitude saving throw at the start of each of his rounds. If you succeed at the saving throw three times, the spell ends. If you fail your save three times, you are permanently turned to stone and become victim to the petrified condition. Successes and failures do not have to be consecutive; keep track of both until the target has obtained three of the same type.\\
- \emph{7. Violet}. If the saving throw fails, the target is blinded. It must then make a Will saving throw at the start of your next round. On a successful save, the blindness ends. If the saving throw fails, the creature is transported to another plane of existence of the Storyteller's choice and is no longer blinded (usually, a creature that is not on its home plane is exiled to it, while other creatures they are usually brought to the Astral or Ethereal planes).\\
- \emph{8. Special}. The target is hit by two beams. Roll twice more, rerolling 8s.

\medskip\textbf{Poison Spray}\index[Spells]{Cantrip - Poison Spray}\\
\textbf{School}: Animals and Plants\\
\textbf{Level}: 0, Uncommon\\
\textbf{Cast Time}: 1 Action\\
\textbf{Range}: 3 metres\\
\textbf{Components}: V, S\\
\textbf{Duration}: Instant\\
You extend your hand toward a creature within range and that you can see, and project a cloud of poisonous gas from your palm. The creature must succeed on a Fortitude save or take 1d12 poison damage. \\
The damage of the spell increases by 1d8 when you reach CM 5, CM 11 and CM 17, but it costs 2 Actions to cast it enhanced and 2 Magic Points, it is also necessary to have taken Adept of Magic in this Magic List a number of times equal to the enhancements that you want to apply.\\
\textbf{For every two Magical Critical Successes obtained} in the Magic Test you affect another creature within range.

\medskip\textbf{Dazzling Grip}\index[Spells]{Cantrip - Lightning Grip}\\
\textbf{School}: Air\\
\textbf{Level}: 0, Municipality\\
\textbf{Cast Time}: 1 Action\\
\textbf{Range}: Contact\\
\textbf{Components}: V, S\\
\textbf{Duration}: Instant\\
Lightning flashes from your hands, shocking a creature you try to make contact with. Make a melee spell attack against the target. You have +1d6 on the attack roll if the target is wearing armor made of metal. On a hit, the target takes 1d8 lightning damage, and can't take reactions until the start of its next round.\\
The damage of the spell increases by 1d8 when you reach CM 5, CM 11 and CM 17, but it costs 2 Actions to cast it enhanced and 2 Magic Points, it is also necessary to have taken Adept of Magic in this Magic List a number of times equal to the enhancements that you want to apply.\\
\textbf{For two each Magical Critical Success obtained} in the Magic Test the damage increases by 1d8

\medskip\textbf{Kyrin Currant Juice Concentrate}\index[Spells]{Kyrin Currant Juice Concentrate}\\
\textbf{School}: Animals and Plants, Earth\\
\textbf{Level}: 2, Uncommon\\
\textbf{Launch Time}: 2 Shares\\
\textbf{Range}: 9 metres\\
\textbf{Components}: V, M (12 currants that the spell consumes)\\
\textbf{Duration}: 1 minute \\
You extract the acidic sap from the currants and project a line of acid spray 30 feet long and 3 feet wide in a direction of your choice. Each creature in the line must succeed on a Reflex saving throw or be covered in acid for the duration of the spell or until a creature uses two actions to wash the acid off itself or another creature. A creature covered in acid takes 2d4 acid damage at the start of each of its rounds.\\
\textbf{For each Magical Critical Success obtained} in the Magic Test the damage increases by 2d4

\medskip\textbf{Hint}\index[Spells]{Hint}\\
\textbf{School}: Enchantment\\
\textbf{Level}: 2, Municipality\\
\textbf{Launch Time}: 2 Shares\\
\textbf{Range}: 9 metres\\
\textbf{Components}: V, M (a snake's tongue and a piece of honeycomb or a drop of sweet oil)\\
\textbf{Duration}: 8 hours \\
You suggest a course of activity (limited to a sentence or two) and magically influence a creature of your choice that you can see and hear and understand you. Creatures that can't be charmed are immune to this effect. The suggestion must be spoken so that the course of action sounds reasonable. Asking a creature to stab itself, throw itself on a spear, set itself on fire, or do some other obviously harmful act automatically negates the spell's effects.\\
The target must make a Will saving throw. If it fails the saving throw, it follows the course of action you describe to the best of its ability. The suggested course of action may continue for the duration of the spell. If the suggested activity can be completed in a shorter time, the spell ends when the subject finishes doing what was asked.\\
You can also specify conditions that will trigger a special activity for the spell's duration. For example, you might suggest that a knight give up his warhorse to the first beggar he comes across. If the condition is not met before the spell ends, the activity will not be performed. If you or any of your companions damage the target, the spell ends.

\medskip\textbf{Mass Suggestion}\index[Spells]{Mass Suggestion}\\
\textbf{School}: Enchantment\\
\textbf{Level}: 6, Uncommon\\
\textbf{Launch Time}: 2 Shares\\
\textbf{Range}: 18 metres\\
\textbf{Components}: V, M (a snake's tongue and a piece of honeycomb or a drop of sweet oil)\\
\textbf{Duration}: 24 hours\\
You suggest a course of activity (limited to one or two sentences) and magically influence up to twelve creatures within range that you can see and hear and understand you, chosen by you. Creatures that can't be charmed are immune to this effect. The suggestion must be spoken so that the course of action sounds reasonable. Asking a creature to stab itself, throw itself on a spear, set itself on fire, or do some other obviously harmful act automatically negates the spell's effects.\\
Each target must make a Will saving throw. If it fails the saving throw, it follows the course of action you describe to the best of its ability. The suggested course of action may continue for the duration of the spell. If the suggested activity can be completed in a shorter time, the spell ends when the subject finishes doing what was asked.\\
You can also specify conditions that will trigger a special activity for the spell's duration. For example, you might suggest that a group of soldiers give all their money to the first beggar they come across. If the condition is not met before the spell ends, the activity will not be performed. If you or any of your companions harm a creature affected by this spell, the spell ends for that creature.\\
\textbf{For each Magical Critical Success obtained} in the Magic Test add one day to the duration.

\medskip\textbf{Thaumaturgy}\index[Spells]{Cantrip - Thaumaturgy}\\
\textbf{School}: Universal\\
\textbf{Level}: 0, Uncommon\\
\textbf{Launch Time}: 2 Shares\\
\textbf{Range}: 9 metres\\
\textbf{Components}: V\\
\textbf{Duration}: Maximum 1 minute\\
You manifest within range a minor trick, a sign of supernatural power. You create one of the following magical effects within range:\\

- Your voice sounds three times louder than normal for 1 minute.\\
- Cause the flames to flicker, grow stronger, dimmer, or change color for 1 minute.\\
- Causes harmless tremors on the ground for 1 minute.\\
- You create an instantaneous noise, such as a clap of thunder, the call of a crow, or an eerie whisper, originating from a point within range of your choosing.\\
- Cause an unlocked door or window to swing open or slam shut.\\
- Change the appearance of your eyes for 1 minute.


If you cast this spell multiple times, you can keep up to three one-minute effects active at a time, and you can end these effects with an action.\\
\textbf{For each Magical Critical Success obtained} in the Magic Test you can manifest an additional magical effect.

\medskip\textbf{Telekinesis}\index[Spells]{Telekinesis}\\
\textbf{School}: Transmutation\\
\textbf{Level}: 5, Uncommon\\
\textbf{Launch Time}: 2 Shares\\
\textbf{Range}: 18 metres\\
\textbf{Components}: V, S\\
\textbf{Duration}: Concentration, maximum 10 minutes \\
You gain the ability to move or manipulate creatures or objects by thought. When you cast this spell, and as 2 actions during each round, you can exert your will on one creature or object within range and that you can see, causing the appropriate effect among the following. You can always act on the same target round after round, or choose a new one each time. If you change targets, the previous target is no longer subject to the spell.
\emph{Creature}. You can attempt to move a Huge or smaller creature. Make an opposing saving throw between your Willpower with your spellcasting ability modifier versus a Fortitude saving throw. If you win the contest, you move the creature 30 feet in any direction, including upward, but not exceeding the spell's range. Until the end of your next round, the creature is restrained by your telekinetic grasp. A creature lifted high remains suspended in mid-air.\\
In subsequent rounds, you can use 2 Actions to attempt to maintain your telekinetic hold on the creature by repeating the contest.
\emph{Item}. You can attempt to move an object weighing up to 500 pounds. If the item is not being worn or carried, you automatically move it 30 feet in any direction, but not beyond the spell's range.\\
If the object is being worn or carried by a creature, you must make a saving throw pitting your Willpower with your spellcasting ability modifier against a Fortitude save modified by the Strength of the creature holding it. If you win the contest, you drag the object away from that creature and move it 30 feet in any direction, but not beyond the spell's range.\\
You can exert precise control over objects with your telekinetic grasp, allowing you to manipulate a simple tool, open a door or container, insert or retrieve an object from an open container, or pour material into a vial.

\medskip\textbf{Teleport}\index[Spells]{Teleport}\\
\textbf{School}: Summon\\
\textbf{Level}: 7, Municipality\\
\textbf{Launch Time}: 2 Shares\\
\textbf{Range}: 3 metres\\
\textbf{Components}: V\\
\textbf{Duration}: Instant\\
This spell instantly teleports you and eight other willing creatures (or a single object) within range and that you can see, chosen by you, to a destination of your choosing. If the target is an object, it must fit within a 10-foot cube, and cannot be held or carried by an unwilling creature.\\
The destination you choose must be known to you, and must be on the same plane of existence as you. Your familiarity with the destination determines whether you get there.\\
The DM rolls a d100 and consults the table.
\end{multicols}
\medskip
\begin{tabular}{llllll}
\toprule
d100 &Error&Similar Area&Off Target&On Target\\
Permanent circle&-&-&-&01-100\\
Associated Object&-&-&-&01-100\\
Very Familiar&01-05&06-13&14-24&25-100\\
Seen by chance&01-33&34-43&44-53&54-100\\
Seen once&01-43&44-53&54-73&74-100\\
Description&01-43&44-53&54-73&74-100\\
False Destination&01-50&51-100&-&-\\
\end{tabular}
\medskip
\begin{multicols}{2}

\emph{Permanent Circle} indicates a permanent teleportation circle for which you know the seal sequence.\\
\emph{Associated item} indicates that you possess an item taken within the last six months from the desired destination, such as a book from a wizard's library, linens from the royal suite, or a piece of marble from a lich's secret tomb.\\
\emph{Very familiar} is a place you have been very often, a place you have studied carefully, or a place you can see when you cast the spell.\\
\emph{Seen by chance} is a place you have seen more than once but are not very familiar with. \\
\emph{Seen once} is a place you have seen only once, perhaps through magic.\\ \emph{Description} is a place whose location and appearance you know only through someone else's description, perhaps a map.\\
\emph{False destination} is a place that doesn't exist. Maybe you tried to peer into an enemy's hideout but instead saw an illusion, or you're trying to teleport to a familiar place that no longer exists. \\
\emph{On Target}. You and your party (or the target item) appear wherever you wish.\\
\emph{Out of Target}. You and your party (or the target object) appear a random distance from the destination in a random direction. The distance off target is 1d10 x 1d10 percent of the distance travelled. For example, if you tried to travel 110 miles, land off target, and roll a 5 and a 3 on two d10s, then you would be 15\% off target, or 17 miles. The Storyteller determines the off-target direction randomly, rolling a d8 and indicating the 1 as north, the 2 as northeast, the 3 as east, and so on following the compass directions. If you're teleporting to a coastal city and end up 17 miles out at sea, you could be in trouble!\\
\emph{Similar Area}. You and your party (or the target object) end up in a different area that is visually or thematically similar to the target area. For example, if you are headed to your personal laboratory, you may end up in another spellcaster's laboratory or in an alchemical item shop that has many of the tools and instruments in your laboratory. Typically, you appear in the closest similar location, but since the spell has no range limit, you could end up almost anywhere on the same plane.\\
\emph{Error}. The unpredictable magic of the spell causes a difficult journey. Each teleported creature (or the target object) takes 3d10 force damage, and the Storyteller rerolls on the table to see where they end up (multiple errors can occur, dealing damage every
time).\\
\textbf{NOTE}: Teleporting from Curyan to Tiya and vice versa has only a 5% success rate.

\medskip\textbf{Firestorm}\index[Spells]{Firestorm}\\
\textbf{School}: Fire\\
\textbf{Level}: 7, Rare\\
\textbf{Launch Time}: 2 Shares\\
\textbf{Range}: 45 metres\\
\textbf{Components}: V, S\\
\textbf{Duration}: Instant\\
A storm composed of roaring flames appears at a point within range, chosen by you. The storm area consists of up to ten 10-foot cubes, which you can arrange however you like. Each cube must have at least one face adjacent to that of another cube. Each creature in the area must make a Reflex saving throw. She takes 7d10 fire damage on a failed save, or half as much damage on a successful one. The fire damages objects in the area and ignites flammable objects that are not being worn or carried. If desired, plant life in the area remains unharmed by the effects of this spell. \\
\textbf{For each Magical Critical Success obtained} in the Magic Test you increase the area of ​​a cube with an edge of 3 metres. \\
\textbf{Saving Throw Success/Critical Failure}: In case of a critical failure the damage is doubled, in case of a critical success the damage is further halved

\medskip\textbf{Ice Storm}\index[Spells]{Ice Storm}\\
\textbf{School}: Water, Air\\
\textbf{Level}: 4, Uncommon\\
\textbf{Launch Time}: 2 Shares\\
\textbf{Range}: 90 metres\\
\textbf{Components}: V, S, M (a pinch of powder and a few drops of water)\\
\textbf{Duration}: Instant\\
A hailstorm of ice hits the ground in a cylinder 6 meters in radius and 12 meters high centered on a point within range. Each creature in the cylinder must make a Reflex saving throw. The creature takes 2d8 bludgeoning damage and 4d6 cold damage on a failed save, or half as much on a successful one. Hail turns the storm's area of ​​effect into difficult terrain until the end of your next round.\\
\textbf{For each Magical Critical Success obtained} in the Magic Test the damage increases by 2d8.\\
\textbf{Saving Throw Success/Critical Failure}: On a critical failure the damage is doubled, on a critical success the damage is further halved

\medskip\textbf{Sleet Storm}\index[Spells]{Sleet Storm}\\
\textbf{School}: Water\\
\textbf{Level}: 3, Very Rare\\
\textbf{Launch Time}: 2 Shares\\
\textbf{Range}: 45 metres\\
\textbf{Components}: V, S, M (a pinch of powder and a few drops of water)\\
\textbf{Duration}: 1 minute\\
Until the spell ends, freezing rain and sleet falls in a 20-foot-tall, 40-foot-radius cylinder centered at a point you choose within range. The area is in dim light, while the exposed flames are extinguished. The ground in the area is covered in slippery ice, making it difficult terrain. When a creature enters the spell's area for the first time during a round or begins its round there, it must make a Reflex saving throw. If she fails, she falls prone. If a creature in the spell's area is concentrating, it must succeed at a Fortitude save against the spell's save DC or lose concentration.

\medskip\textbf{Black Tentacles}\index[Spells]{Black Tentacles}\\
\textbf{School}: Summon\\
\textbf{Level}: 4, Uncommon\\
\textbf{Launch Time}: 2 Shares\\
\textbf{Range}: 27 metres\\
\textbf{Components}: V, S, M (a piece of giant octopus or giant squid tentacle)\\
\textbf{Duration}: 1 minute\\
Slimy ebony tentacles fill a 20-foot square on the ground, within range and visible to you. For the duration of the spell, these tendrils transform the area into difficult terrain.\\
When a creature enters the affected area for the first time in a round or begins its round here, it must succeed on a Reflex saving throw or take 3d6 bludgeoning damage and be \hyperlink{entangled}{entangled} by the tentacles until the end of the spell. A creature that begins its round in the area and is already entangled by the tentacles takes 3d6 bludgeoning damage. A creature entangled by the tentacles can use 2 actions to make a new saving throw to be free that round.

\medskip\textbf{Earthquake}\index[Spells]{Earthquake}\\
\textbf{School}: Earth\\
\textbf{Level}: 8, Very Rare\\
\textbf{Launch Time}: 2 Shares\\
\textbf{Range}: 150 metres\\
\textbf{Components}: V, S, M (a pinch of soil, a piece of stone and a lump of clay)\\
\textbf{Duration}: Concentration, maximum 1 minute\\
You cause a seismic disturbance at a point on the terrain within range and that you can see. For the duration, an intense tremor shakes the ground in a 100-foot radius circle centered on that point and shakes creatures and structures in that area that are in contact with the ground. The terrain in the area becomes difficult terrain. Each creature on the ground that is concentrating must make a Fortitude saving throw. If he fails, his concentration is broken.\\
When you cast this spell and at the end of each round you spend concentrating on it, each creature in the area that is on the ground must make a Reflex saving throw. On a failed save, the creature falls prone.\\
This spell has additional effects depending on the type of terrain in the area, at the Storyteller's discretion. Fissures. At the beginning of the round following the one in which you cast the spell, rifts open throughout the spell's area. A total of 1d6 fissures open at locations chosen by the Storyteller. Each is 1d10 x 10 feet deep, 10 feet wide, and extends from one side of the spell's area to the other. A creature standing where a fissure opens must succeed on a Reflex save or fall into it. A creature that succeeds on its saving throw moves to the edge of the fissure as it opens.\\
A fissure opening under a structure causes it to immediately collapse (see below). Structures. The tremor deals 50 bludgeoning damage to any structures in contact with the ground in the area when you cast the spell and at the end of each of your rounds until the spell ends. If a structure drops to 0 hit points, it collapses and may harm nearby creatures. A creature that is half the height of the structure or less away from the structure must make a Reflex saving throw. On a failed save, the creature takes 5d6 bludgeoning damage, falls prone, and is submerged in rubble. He will then have to take 2 actions and succeed at a DC 20 Dexterity (Athletics) check to free himself. The Storyteller can adjust the DC up or down, depending on the nature of the rubble. On a successful save, the creature takes only half damage and does not fall or become buried.

\medskip\textbf{Illusory Ground}\index[Spells]{Illusory Ground}\\
\textbf{School}: Illusion\\
\textbf{Level}: 4, Uncommon\\
\textbf{Launch Time}: 10 minutes\\
\textbf{Range}: 90 metres\\
\textbf{Components}: V, S, M (a stone, a twig and a piece of green plant)\\
\textbf{Duration}: 24 hours \\
Make a piece of natural terrain within range, in a 500-foot cube, look, sound, and smell like some other type of natural terrain. As a result, open fields or a road may be transformed into a swamp, hills, a crevasse, or some other type of difficult or impassable terrain. A pond can be transformed into a grassy clearing, a precipice into a gentle slope, a rock-strewn ravine into a wide, smooth road. Built structures, equipment, and creatures within the area do not change in appearance.\\
The tactile features of the terrain are unchanged, so creatures that enter the area are likely to reveal the illusion. If the difference isn't obvious upon contact, a creature warily examining the illusion can attempt an Awareness check against your spells' saving throw DC to disbelieve it. A creature that recognizes the illusion for what it is perceives it as a vague image superimposed on the ground.

\medskip\textbf{Icy Touch}\index[Spells]{Cantrip - Icy Touch}\\
\textbf{School}: Necromancy\\
\textbf{Level}: 0, Municipality\\
\textbf{Cast Time}: 1 Action\\
\textbf{Range}: 36 metres\\
\textbf{Components}: V, S\\
\textbf{Duration}: 1 round\\
You create a skeletal ghostly hand in the space of a creature within range. Make a ranged spell attack against the creature, attacking it with the chill of death. On a hit, the target takes 1d8 Void damage, and cannot regain Hit Points until the start of your next round. Until then, the hand will remain locked on the target. If you hit an undead target, it also has -1d6 on attack rolls against you until the end of its next round.\\
The damage of the spell increases by 1d8 when you reach CM 5, CM 11 and CM 17, but it costs 2 Actions to cast it enhanced and 2 Magic Points, it is also necessary to have taken Adept of Magic in this Magic List a number of times equal to the enhancements that you want to apply.\\
\textbf{For every two Magical Critical Successes obtained} in the Magic Test you create an additional skeletal hand that must attack a different creature within range.

\medskip\textbf{Vampiric Touch}\index[Spells]{Vampiric Touch}\\
\textbf{School}: Necromancy\\
\textbf{Level}: 3, Municipality\\
\textbf{Launch Time}: 2 Shares\\
\textbf{Range}: Personal\\
\textbf{Components}: V, S\\
\textbf{Duration}: 1 minute \\
Contact with your shadow-shrouded hand can drain the life force of others to heal your wounds. Make a melee spell attack against a creature within range. If you hit, the target takes 3d6 Void damage, and you regain a number of Hit Points equal to half the Void damage you dealt. Until the spell ends, you can make this attack again each round as your attack action.\\
While you have this spell active you are considered Distracted for casting other spells.\\
\textbf{For each Magical Critical Success obtained} in the Magic Test the damage increases by 1d12.

\medskip\textbf{Hypnotic Plot}\index[Spells]{Hypnotic Plot}\\
\textbf{School}: Illusion\\
\textbf{Level}: 3, Municipality\\
\textbf{Launch Time}: 2 Shares\\
\textbf{Range}: 36 metres\\
\textbf{Components}: S, M (a glowing stick of incense or a crystal vial filled with phosphorescent material)\\
\textbf{Duration}: 1 minute\\
You create a twisting pattern of colors at range that moves through the air inside a 30-foot cube. The plot appears for a moment and then vanishes. Each creature in the area that sees the pattern must make a Will saving throw. On a failed save, a creature is charmed for the duration. While charmed by this spell, the creature is incapacitated and has speed 0. The spell ends for the affected creature if it takes damage or if someone uses an action to shake it from its dazed state.

\medskip\textbf{Transformation}\index[Spells]{Transformation}\\
\textbf{School}: Transmutation\\
\textbf{Level}: 9, Rare\\
\textbf{Launch Time}: 2 Shares\\
\textbf{Range}: Personal\\
\textbf{Components}: V, S, M (a jade circlet worth at least 1,500 gp, which you must place on your head before casting the spell)\\
\textbf{Duration}: 1 hour\\
You take the form of a different creature for the duration. The new form can be that of any creature whose challenge rating is equal to or lower than your CM. The creature cannot be a construct or undead, and you must have seen it at least once. You transform into an average example of that creature, one with no specific Abilities. You can remain in your assumed form until the spell ends. You automatically retransform if you fall unconscious, drop to 0 hit points, or die. Your game statistics are replaced by the statistics of the chosen creature, except your Traits, and your Intelligence, Wisdom, and Charisma scores. You retain all your skill proficiencies and saving throws, as well as gaining those of the creature. If the creature has the same skills as you and the bonus listed in its statistics is higher than yours, use the creature's bonus instead of yours. You cannot use any additional actions or lair actions of the new form.\\
When you transform, you take on the creature's Hit Points and Hit Dice. When you return to your normal form, you return to the number of Hit Points you had before you transformed. However, if you retransform because you have been reduced to 0 Hit Points, all excess damage is returned to your original form. Unless the excess damage reduces your normal form to 0 Hit Points, you will not fall unconscious. \\
You retain all the benefits of any Skill you possess, race, or other source and can use them if the new form is physically capable of using them. However, you cannot use any of your special senses, such as darkvision, unless the new form also possesses the same sense. You can speak only if the creature can normally speak.\\
When you transform, you choose whether your equipment falls to the ground in your space, merges with your new form, or is worn by it. Worn gear functions as normal, but it is up to the Storyteller to decide whether it is comfortable for the new form to wear such a piece of gear, based on the creature's size and dimensions. Your equipment does not change size or adapt to the new form, and any equipment that the new form cannot wear must be dropped or fused with the new form. Equipment that fuses is ineffective.\\
During the spell's duration, you can use two actions to assume a different form following the same restrictions and rules as your original form, with one exception: if your new form has more Hit Points than your current form, your Hit Points remain at their current level .\\
\textbf{NOTE}: You must be a follower of Ephrem or Shayalia to learn this spell

\medskip\textbf{Restser's Furious Transformation}\index[Spells]{Restser's Furious Transformation}\\
\textbf{School}: Transmutation\\
\textbf{Level}: 6, Very Rare\\
\textbf{Launching Time}: 2 Actions\\
\textbf{Range}: Personal\\
\textbf{Components}: V, S, M (20cc of alcoholic drink that is consumed by casting the spell, a magical weapon)\\
\textbf{Duration}: 1 round for Magical Proficiency\\
This spell allows a spellcaster to channel his magical energies into transforming himself into a powerful fighter.

Until the end of the spell's duration, your weapon proficiency becomes equal to your magical proficiency.

Based on the magical weapon held in your hand at the time of the spell, you become competent in the Weapons List to which that weapon belongs. If the weapon is present in multiple lists, the spellcaster will choose the list. The caster gains the abilities of that Weapon List as if he had chosen it a number of times equal to half his points in Magical Expertise.

The caster gains 4 temporary hit points per point of magical proficiency possessed.
The unchanged scores of the physical characteristics (Strength, Dexterity and Constitution) become 2 if lower than 2.

For the duration of the spell the caster is no longer able to cast spells.

\medskip\textbf{Arboreal Translation}\index[Spells]{Arboreal Translation}\\
\textbf{School}: Animals and Plants\\
\textbf{Level}: 5, Rare\\
\textbf{Launch Time}: 2 Shares\\
\textbf{Range}: Personal\\
\textbf{Components}: V, S\\
\textbf{Duration}: maximum 1 minute\\
You gain the ability to enter a tree and move from inside it to another tree of the same species within 500 feet. Both trees must be alive and at least the same size as you. You must use 1 meter of movement to enter the tree. You instantly learn the location of all other trees of the same species within 150 meters, and as part of the movement taken to enter the tree, you can pass into one of the other trees or exit the tree you entered. You respawn at a point of your choice within 1 meter of the destination tree, using 1 more Move Action. If you have no movement left to use, you respawn within 1 meter of the tree you entered.\\
You can use this transportation ability once per round for the spell's duration. You must end each round outside of a tree.

\medskip\textbf{Plant Transport}\index[Spells]{Plant Transport}\\
\textbf{School}: Animals and Plants\\
\textbf{Level}: 6, Very Rare\\
\textbf{Launch Time}: 2 Shares\\
\textbf{Range}: 3 metres\\
\textbf{Components}: V, S\\
\textbf{Duration}: 1 round\\
This spell creates a magical bond between an inanimate plant of Large or greater size within range and another plant, at any distance, on the same plane of existence. You must have seen or come into contact with the target vegetable at least once. For the duration of the spell, any creature can enter the target plant and exit the target plant using 1 move action.

\medskip\textbf{Rope Trick}\index[Spells]{Rope Trick}\\
\textbf{School}: Transmutation\\
\textbf{Level}: 2, Municipality\\
\textbf{Launch Time}: 1 minute\\
\textbf{Range}: Contact\\
\textbf{Components}: V, S, M (powdered wheat extract and a string of parchment)\\
\textbf{Duration}: 1 hour\\
You come into contact with a piece of rope up to 18 meters long. One end of the string rises into the air until the string hangs perpendicular to the ground. At the opposite end of the rope, an invisible entrance opens into an extradimensional space that remains until the spell ends \\
Extradimensional space can be reached by climbing to the top of the rope (Climb check DC 15). The space can hold up to 2 Medium or smaller creatures. The rope can be dragged through space, causing it to disappear from the sight of those outside it.\\
Attacks and spells cannot pass through the gateway into or out of extradimensional space, but those inside can see out as if they were seeing through a 3-by-3-foot window centered on the string. The Detect Magic spell allows you to see the opening. Anything in extradimensional space falls out when the spell ends.\\
\textbf{For each Magical Critical Success obtained} in the Magic Test the duration doubles or another medium or smaller creature can fit there.

\medskip\textbf{One with the stone}\index[Spells]{One with the stone}\\
\textbf{School}: Earth\\
\textbf{Level}: 3, Municipality\\
\textbf{Launch Time}: 2 Shares\\
\textbf{Range}: Contact\\
\textbf{Components}: V, S\\
\textbf{Duration}: 8 hours\\
You enter a stone object or surface large enough to contain your entire body, fusing with the stone along with any equipment you carry for the duration. Using your movement, you enter the stone at a point you are in contact with. Nothing remains of your presence that remains visible or otherwise detectable by nonmagical senses. While you are fused with the stone, you cannot see what is happening outside it, and any Awareness check you make to hear the sounds made outside it is made with -1d6. You remain aware of the passage of time and can cast spells on yourself while fused with the stone. You can use your movement to leave the stone and reappear where you entered it, thus ending the spell. Otherwise you can't move.\\
Minor damage to the stone doesn't harm you, but partially destroying it or changing its shape (so that you can no longer fit inside it) ejects you from it and deals 6d6 bludgeoning damage. Completely destroying the stone (or transmuting it into another substance) causes you to be expelled and deals 50 bludgeoning damage. If you are ejected, you fall prone in an unoccupied space, closest to where you entered the stone.\\
\textbf{For each Magical Critical Success obtained} in the Magic Test the maximum duration increases by 1 hour.

\medskip\textbf{Slimy}\index[Spells]{Slimy}\\
\textbf{School}: Animals and Plants\\
\textbf{Level}: 1, Municipality\\
\textbf{Launch Time}: 2 Shares\\
\textbf{Range}: 18 metres\\
\textbf{Components}: V, S, M (a piece of pork rind or butter or greasy meat)\\
\textbf{Duration}: 1 minute\\
Slippery grease covers the ground in a 10-foot square centered on a point within range, making it difficult terrain for the duration of the spell\\
When the blubber appears, each target standing in the area must succeed on a Reflex save or fall prone. A creature that enters the area or ends its round there must succeed at a Reflex saving throw or fall prone.

\medskip\textbf{See Invisibility}\index[Spells]{See Invisibility}\\
\textbf{School}: Divination\\
\textbf{Level}: 2, Municipality\\
\textbf{Launch Time}: 2 Shares\\
\textbf{Range}: Personal\\
\textbf{Components}: V, S, M (a pinch of talc and a handful of silver dust)\\
\textbf{Duration}: 1 hour\\
For the duration of the spell, you see invisible creatures and objects as if they were visible, and you can also see into the Ethereal Plane. Ethereal creatures and objects appear ghostly and transparent to you.

\medskip\textbf{Speed}\index[Spells]{Speed}\\
\textbf{School}: Transmutation\\
\textbf{Level}: 3, Uncommon\\
\textbf{Launch Time}: 2 Shares\\
\textbf{Range}: 9 metres\\
\textbf{Components}: V, S, M (a grated liquorice root)\\
\textbf{Duration}: 1 minute\\
You modify the passage of time by speeding it up around a maximum of 1d4 creatures in a 20-foot cube within range. Until the spell ends, targets can perform an additional Attack or Move Action.
This spell counters and is countered by \hyperlink{slow}{Slow}.\\
When the spell ends, targets cannot move or take Actions until their next round, while they become suddenly drowsy.\\
\textbf{For each Magical Critical Success obtained} in the Magic Test you can influence one additional creature.\\
\textbf{For every three Magical Critical Success obtained} in the Magic Test you can increase the Actions per round by an additional 1.

\medskip\textbf{Surveillance and Interdiction}\index[Spells]{Surveillance and Interdiction}\\
\textbf{School}: Abjuration\\
\textbf{Level}: 6, Uncommon\\
\textbf{Launch Time}: 10 minutes\\
\textbf{Range}: Contact\\
\textbf{Components}: V, S, M (burnt incense, a small measure of sulfur and oil, a tied thong, a small amount of earth giant's blood, and a small silver rod worth at least 10 mo)\\
\textbf{Duration}: 24 hours\\
You create a ward that protects up to 225 square meters of floor (a square area of ​​15 meters on a side, or one hundred squares of 1 meter on a side or twenty-five squares of 3 meters on a side). The restricted area can be up to 6 meters high, and shaped however you like. You can ward off multiple floors of a stronghold by dividing the area between them, as long as you can walk continuously in each adjacent area while casting the spell\\
When you cast this spell, you can specify individuals who ignore any or all of this spell's effects. You can also specify a password that, when spoken aloud, makes the speaker immune to these effects.\\
Surveillance and interdiction creates the following effects within the prohibited area.\\
\emph{Corridors}. Fog fills all the forbidden corridors, making them heavily darkened. Additionally, at any intersection or fork in the passage that offers a choice of direction, there is a 50\% chance that a creature, excluding you, will believe it is going in the direction opposite to the one it chose.\\
\emph{Doors}. All doors in the forbidden area are magically locked, as if sealed by the Magic Lock spell. Additionally, you can cover up to ten doors with an illusion (equivalent to the illusory object function of the minor illusion spell) to make them appear to be simple sections of wall.\\
\emph{Scale}. Cobwebs cover all the stairs in the forbidden area from top to bottom, as per the cobweb spell. These threads regrow in 10 minutes if burned or torn while vigilance and interdiction remains active.\\
Other Spells in Effect. You can place one of the following magical effects of your choice within the building's restricted area\\

- Place dancing lights in four corridors. You can indicate a simple program that the lights will repeat for the duration of the vigilance and interdiction.\\
- Place magic mouth in two places.\\
- Nauseating Fog Square in two places. The vapors appear in the place you indicated; they return within 10 minutes if dispersed by the wind while surveillance and interdiction is still active.\\
- Place a constant gust of wind in a hallway or room.\\
- Place a suggestion in a location. Select a 3-foot square area, and any creature that enters or passes through that area mentally receives the suggestion.\\

The entire forbidden area radiates magic. A dispel magic spell cast against a specific effect, if successful, removes only that effect. You can create a perpetually guarded and warded structure by casting this spell on it every day for one year.\\
\textbf{If you score three criticals} the duration is permanent.

\medskip\textbf{Vigour}\index[Spells]{Vigour}\\
\textbf{School}: Care\\
\textbf{Level}: 4, Rare\\
\textbf{Launch Time}: 2 Shares\\
\textbf{Range}: Meters contact\\
\textbf{Components}: V, S, M (water, salt, sugar)\\
\textbf{Duration}: 1 round for Magical Proficiency\\
The creature affected by this spell recovers one level of Fatigue, gaining 3d6 Temporary Hit Points. He can concentrate his energy to take an Attack Action without multiattack penalty or take an additional Move Action.


\medskip\textbf{Binding of Interdiction}\index[Spells]{Binding of Interdiction}\\
\textbf{School}: Abjuration\\
\textbf{Level}: 2, Municipality\\
\textbf{Launch Time}: 2 Shares\\
\textbf{Range}: Contact\\
\textbf{Components}: V, S, M (a pair of platinum rings worth 50 gp each, which you and the target must wear for the duration)\\
\textbf{Duration}: 1 hour\\
You cast the spell on contact with a creature you want to protect. You create a mystical connection between you and the target until the spell ends. As long as the target is within 60 feet of you, it gains a +1 bonus on Defense and saving throws and has resistance to all damage. Additionally, whenever the target takes damage, you take the same amount. The spell ends if you drop to 0 hit points or you and the target move more than 60 feet away. It ends even if you cast it again on the same creature it's already affecting. You can end the spell with an action.

\medskip\textbf{True Vision}\index[Spells]{True Vision}\\
\textbf{School}: Divination\\
\textbf{Level}: 6, Rare\\
\textbf{Launch Time}: 2 Shares\\
\textbf{Range}: Contact\\
\textbf{Components}: V, S, M (an eye ointment that costs 25 gp; made of powdered mushrooms, saffron, and fat; is consumed by the spell)\\
\textbf{Duration}: 1 hour\\
You cast the spell on contact with a willing creature. The target receives the ability to see things as they really are. For the duration of the spell, the creature has true vision, notices secret doors hidden by magic, and can see into the Ethereal Plane, up to a range of 120 feet.

\medskip\textbf{Displayed Life}\index[Spells]{Displayed Life}\\
\textbf{School}: Necromancy\\
\textbf{Level}: 1, Municipality\\
\textbf{Launch Time}: 2 Shares\\
\textbf{Range}: Personal\\
\textbf{Components}: V, S, M (a small amount of alcohol or distilled spirit)\\
\textbf{Duration}: 1 hour\\
Empowering yourself with a necromantic semblance of vitality, you gain 1d4 + 4 temporary hit points for the duration.\\
\textbf{For each Magical Critical Success obtained} in the Magic Test you gain 5 temporary Hit Points.

\medskip\textbf{Flying}\index[Spells]{Flying}\\
\textbf{School}: Air\\
\textbf{Level}: 3, Municipality\\
\textbf{Launch Time}: 2 Shares\\
\textbf{Range}: Contact\\
\textbf{Components}: V, S, M (a feather from the wing of any bird)\\
\textbf{Duration}: 10 minutes \\
You cast the spell on contact with a willing creature. For the duration of the spell, the target gains a flying speed of 60 feet. When the spell ends, if it is still in the air, the target falls unless it manages to stop its descent.\\
Casting a spell while flying is more complex, you are Distracted if you fail a DC 11 Fly check.\\
\textbf{For each Magical Critical Success obtained} in the Magic Test you can target an additional creature or double the duration.

\medskip\textbf{Mental Shield}\index[Spells]{Mental Shield}\\
\textbf{School}: Abjuration\\
\textbf{Level}: 8, Uncommon\\
\textbf{Launch Time}: 2 Shares\\
\textbf{Range}: Contact\\
\textbf{Components}: V, S\\
\textbf{Duration}: 24 hours\\
Until the spell ends, a willing creature you are in contact with during the casting is immune to any effects that would sense its emotions or read its thoughts, divination spells, and the Charmed condition. the spell also negates wish spells and other spells or effects of similar power employed for
influence the target's mind or gain information about it.\\
\textbf{For each Magical Critical Success obtained} in the Magic Test the duration doubles. If you get three criticals the duration is permanent.

\medskip\textbf{Zone of Truth}\index[Spells]{Zone of Truth}\\
\textbf{School}: Enchantment\\
\textbf{Level}: 2, Uncommon\\
\textbf{Launch Time}: 2 Shares\\
\textbf{Range}: 18 metres\\
\textbf{Components}: V, S\\
\textbf{Duration}: 10 minutes\\
You create a magical zone that protects against deception in a 10-foot-radius sphere centered on a point of your choice within range. Until the spell ends, a creature that enters the spell's area for the first time in a round, or begins its round within it, must make a Will saving throw. On a failed save, the creature cannot deliberately utter lies while within range of the spell. You know whether a creature succeeds or fails its saving throw. A creature under the spell is aware of this and can therefore avoid answering questions that it would normally answer with a lie. This creature can give elusive answers as long as it stays within the bounds of truth.

\end{multicols}

%\vspace{2cm}
%\begin{center}
% \includegraphics[width=0.4\linewidth]{immagini/Bocca_della_Verita.png}
% \medskip
% \emph{La Bocca della Verita', Church of Santa Maria in Cosmedin, Rome}
%\end{center}


\pagebreak

\subsection{Ancient and lost spells}

The spells present here have been lost to history and only legends point to their existence.\\
These spells not only have Legendary Rarity but only the most erudite have heard of them. Very often these are spells that were contrary to the will of some Patron who took steps to eliminate them from history and knowledge.

\begin{multicols}{2}

\medskip\textbf{Planar Ally}\index[Spells]{Planar Ally}\\
\textbf{School}: Summon\\
\textbf{Level}: 6, Legendary\\
\textbf{Launch Time}: 10 minutes\\
\textbf{Range}: 18 metres\\
\textbf{Components}: V, S\\
\textbf{Duration}: Instant\\
You beg an otherworldly entity for help. The being must be known to you: a god, a primordial, a demon prince, or some other creature of great power. That entity sends a celestial, elemental, or demon loyal to it to aid you, causing the creature to appear in an unoccupied space within range. If you know the name of a specific creature, you can speak its name when you cast this spell to request that creature's aid, though you may still receive another one (at the Storyteller's discretion).\\
When the creature appears, it is under no obligation to act in any particular way. You can ask the creature to perform a service in exchange for a reward, but it is not obligated to satisfy you. The task required could be easy (€9538{fly us over the edge} or €9539{help us fight this battle}) or complex (€9540{spy on our enemies} or €9541{protect us during the our exploration of the underground}). You must be able to communicate with the creature to bargain for its services. The reward can take many forms. A celestial might ask for a sizable donation of gold or magical items to an allied temple, while a demon might ask for a human sacrifice or the gift of treasure. Some creatures may exchange their services for a quest that you must undertake on their behalf. As a general rule, a task that can be measured in minutes requires a reward of 100 gp per minute. A task measured in hours requires 1000 gp per hour. A task measured in days (maximum 10 days) requires 10,000 gp per day. The Storyteller can modify these rewards based on the circumstances under which the spell was cast. If the task aligns with the creature's morals, the request for payment may be halved or even canceled. Non-dangerous tasks usually ask for only half of the suggested payment, while very dangerous tasks may require higher donations. It is rare for these creatures to accept tasks that seem suicidal.\\
After the creature completes the task, or when the agreed-upon period of service has ended, the creature will return to its home plane after reporting to you, if appropriate to the task performed and if possible. If you are unable to agree on a price for the creature's services, the creature will immediately return to its home plane. A creature conscripted to join your party is considered a member of your party, and receives a full share of the experience point rewards.

\medskip\textbf{Moon Flare}\index[Spells]{Moon Flare}\\
\textbf{School}: Invocation\\
\textbf{Level}: 2, Legendary\\
\textbf{Launch Time}: 2 Shares\\
\textbf{Range}: 36 metres\\
\textbf{Components}: V, S, M (several night beauty seeds and a piece of opalescent plush)\\
\textbf{Duration}: Concentration, maximum 1 minute\\
A silvery beam of pale light shines in a 3-foot-radius, 40-foot-tall cylinder centered at a point within range. Until the spell ends, a dim light fills the cylinder. \\
When a creature enters the spell's area for the first time during a round or begins its round here, it is engulfed in ghostly flames that cause terrible pain, and must make a Fortitude saving throw. It takes 2d10 Light damage on a failed save, or half as much damage on a successful one. A shapeshifter saves -1d6. If he fails, he immediately returns to his original form and cannot assume a different form until he leaves the spell's light.\\
During each of your rounds after casting the spell, you can use an action to move the
18 meter beam in any direction. \\
\textbf{For each Magical Critical Success obtained} in the Magic Test the damage increases by 1d10.

\medskip\textbf{Contact Other Planes}\index[Spells]{Contact Other Planes}\\
\textbf{School}: Divination\\
\textbf{Level}: 5, Legendary\\
\textbf{Launch Time}: 1 minute\\
\textbf{Range}: Personal\\
\textbf{Components}: V\\
\textbf{Duration}: 1 minute\\
You mentally contact a demigod, the spirit of a long-deceased sage, or some other mysterious entity from another plane. Contacting extraplanar Intelligence can strain or even break your mind. When you cast this spell, make a DC 15 Will save. If you fail, you take 6d6 points of damage and are left demented until dawn the next day. While demented, you cannot perform actions, you cannot understand what other creatures are saying, you cannot read, and you speak only in ramblings. The greater restoration spell can end this effect. If you succeed on the saving throw, you can ask the entity up to five questions. You must ask the questions before the spell ends. The Narrator will answer each question with one word: \emph{yes}, \emph{no}, \emph{maybe}, \emph{never}, \emph{irrelevant} or \emph{confused } (if the entity doesn't know the answer to the question). If a one-word answer would be misleading, the Storyteller might give a short sentence instead.

\medskip\textbf{Summon Celestials}\index[Spells]{Summon Celestials}\\
\textbf{School}: Summon\\
\textbf{Level}: 7, Legendary\\
\textbf{Launch Time}: 1 minute\\
\textbf{Range}: 27 metres\\
\textbf{Components}: V, S\\
\textbf{Duration}: 10 minutes\\
You summon a celestial of challenge rating 4 or lower that appears in an unoccupied space within range and that you can see. The celestial disappears when he drops to 0 hit points or the spell ends. The celestial is friendly toward you and your companions for the duration of the spell. He rolls initiative for the celestial, who acts during his own round. He obeys any verbal command given to him (without requiring you to take any action), as long as he does not violate his Traits. If you do not give commands to the celestial, it will defend itself from hostile creatures, but will not perform other actions.
\textbf{For each Magical Critical Success obtained} in the Magic Test you increase the CR of the summoned creature by one.

\medskip\textbf{Summon Woodland Creatures}\index[Spells]{Summon Woodland Creatures}\\
\textbf{School}: Summon\\
\textbf{Level}: 4, Legendary\\
\textbf{Launch Time}: 2 Shares\\
\textbf{Range}: 18 metres\\
\textbf{Components}: V, S, M (a summoned creature holly berry)\\
\textbf{Duration}: 1 hour \\
You summon fey spirits that appear in unoccupied spaces within range that you can see. Choose one of the following options to determine what appears:\\

- A fey of challenge rating 2 or lower

- Two fey of challenge rating 1 or lower

- Four fey of challenge rating 1/2 or lower

- Eight fey of challenge rating 1/4 or lower

\medskip
A summoned creature disappears when it drops to 0 hit points or when the spell ends. Summoned creatures are friendly towards you and your companions.\\
\textbf{For each Magical Critical Success obtained} two additional creatures of a lower rank or one of a higher rank will appear in the Magic Test.


\medskip\textbf{Summon Kobold}\index[Spells]{Summon Kobold}\\
\textbf{School}: Summon\\
\textbf{Level}: 6, Legendary\\
\textbf{Launch Time}: 1 minute\\
\textbf{Range}: 27 metres\\
\textbf{Components}: V, S\\
\textbf{Duration}: 1 hour \\
You summon a fey spirit of challenge rating 6 or lower, or a fey spirit that takes the form of a beast of challenge rating 6 or lower. It appears in an unoccupied space within range that you can see. The fairy creature disappears when it drops to 0 hit points or when the spell ends.\\
The fairy creature is friendly towards you and your companions.\\
\textbf{For each Magical Critical Success obtained} in the Magic Test you increase the CR of the summoned creature by 1.

\medskip\textbf{Guardian of Faith}\index[Spells]{Guardian of Faith}\\
\textbf{School}: Summon\\
\textbf{Level}: 4, Legendary\\
\textbf{Launch Time}: 2 Shares\\
\textbf{Range}: 9 metres\\
\textbf{Components}: V\\
\textbf{Duration}: 8 hours\\
A Large ghostly guardian appears for the duration and floats in an unoccupied space within range and that you can see, chosen by you. The guardian occupies that space and is indistinguishable except for a glowing sword and a shield bearing your Patron's symbol.\\
Any creature hostile to you that enters a space within 10 feet of the guardian for the first time in a round must make a Reflex saving throw. The creature takes 20 Light/Void damage on a failed save, or half as much damage on a successful one. The guardian vanishes after dealing a total of 60 damage.

\medskip\textbf{Spiritual Guardians}\index[Spells]{Spiritual Guardians}\\
\textbf{School}: Summon\\
\textbf{Level}: 3, Legendary\\
\textbf{Launch Time}: 2 Shares\\
\textbf{Range}: Personal (3 meter radius)\\
\textbf{Components}: V, S, M (a sacred symbol)\\
\textbf{Duration}: Concentration, maximum 10 minutes\\
Call for spirits to protect you. For the duration of the spell, they will float around you at a distance of 10 feet. You determine what your Spiritual Guardians look like. You can designate any number of creatures that are immune to it. An affected creature's speed is halved within the area, and when a creature enters the area for the first time in a round or begins its round there, it must make a Will saving throw. He takes 3d8 Light or Void damage on a failed save, or half as much damage on a successful one.\\
\textbf{For each Magical Critical Success obtained} in the Magic Test the damage increases by 1d8

\medskip\textbf{Planar Bond}\index[Spells]{Planar Bond}\\
\textbf{School}: Abjuration\\
\textbf{Level}: 5, Legendary\\
\textbf{Launch Time}: 1 hour\\
\textbf{Range}: 18 metres\\
\textbf{Components}: V, S, M (a jewel worth at least 1000 gp, which the spell consumes)\\
\textbf{Duration}: 24 hours\\
With this spell, you attempt to bind a celestial, elemental, fey, or demon to your service. The creature must remain within range for the entire casting of the spell. (Usually, the creature is first summoned to the center of an inverted magic circle to keep it trapped while this spell is cast.) Upon completion of the cast, the target must make a Will saving throw. If he fails his save, he is bound to your service for the duration. If the creature was summoned or created by another spell, that spell's duration is extended to match the duration of this spell. A bound creature must follow your instructions to the best of its ability. You might command the creature to accompany you on an adventure, protect a location, or deliver a message. The creature obeys your instructions to the letter, but if it is hostile to you, it will try to twist your words to its own ends. If the creature fully complies with your instructions before the spell ends, it will return to you to tell you what has happened if you are on the same plane of existence. If you are on different planes of existence, she will return to the place where you bound her and remain there until the spell ends.
\textbf{For each Magical Critical Success obtained} in the Magic Test you double the creature's permanence.

\medskip\textbf{Hunter's Mark}\index[Spells]{Hunter's Mark}\\
\textbf{School}: Divination\\
\textbf{Level}: 1, Legendary\\
\textbf{Launch Time}: 2 Shares\\
\textbf{Range}: 27 metres\\
\textbf{Components}: V \\
\textbf{Duration}: Concentration, maximum 1 hour\\
Choose a creature within range that you can see. The creature is mystically marked as your prey. Until the spell ends, you deal an additional 1d6 points of damage to the target whenever you hit it with a weapon attack, and you have +1d6 on Awareness or Survival checks to find it.\\
If the target drops to 0 hit points before the spell ends, you can use an immediate action during your next round to mark a new creature.\\
\textbf{For each Magical Critical Success obtained} in the Magic Test you can maintain concentration on the spell for another hour.

\medskip\textbf{Portal}\index[Spells]{Portal}\\
\textbf{School}: Summon\\
\textbf{Level}: 9, Legendary\\
\textbf{Launch Time}: 2 Shares\\
\textbf{Range}: 18 metres\\
\textbf{Components}: V, S, M (a diamond worth at least 5000 gp)\\
\textbf{Duration}: Concentration, maximum 1 minute\\
You summon into an unoccupied space within range that you can see a portal connected to a specific place on a different plane of existence. The portal is a circular opening you create, 3 to 20 feet in diameter. You can orient the portal in any direction you want. The portal remains for the duration.\\
The portal has a front and a back on both planes on which it appears. Travel through the portal is only possible by moving from the front. Anything that does so is instantly transported to the other plane, appearing in the unoccupied space closest to the portal.\\
Deities and other planar rulers can prevent spell-created portals from opening in their presence or anywhere in their domains. When you cast this spell, you can speak the name of a specific creature (the alias, title, or nickname does not work). If that creature is on a different plane than you, the portal opens near the named creature and pulls the creature through it to the nearest unoccupied space on your side of the portal. You hold no special power over the creature, and it is free to act as the Storyteller sees fit. He might leave, attack you, or help you.

\medskip\textbf{Resurrection}\index[Spells]{Resurrection}\\
\textbf{School}: Necromancy\\
\textbf{Level}: 7, Legendary\\
\textbf{Launch Time}: 1 hour\\
\textbf{Range}: Contact\\
\textbf{Components}: V, S, M (a diamond worth at least 1000 gp, which the spell consumes)\\
\textbf{Duration}: Instant\\
You cast the spell upon contact with a dead creature, who is not an Elf, who has been dead for no more than a century, who has not died of old age, and who is not undead. If her soul is free and willing, the target will return to life with all of her Hit Points.
This spell neutralizes all poisons and cures normal diseases that afflicted the creature when it died. However, it does not remove magical diseases, curses, and the like; if these effects are not removed before the spell is cast, they will afflict the target upon its return to life.\\
This spell closes all mortal wounds and restores any missing body parts. Coming back from the dead is an ordeal. The target takes a -4 penalty on all attack rolls, saving throws, and ability checks. Each time the target finishes a night's rest the penalty is reduced by 1 until it disappears.\\
Casting this spell to bring back a creature that has been dead for a year or more exhausts you. Until you finish a night's rest, you will no longer be able to cast spells and you will have -1d6 on all attack rolls, ability checks, and saving throws.\\
The creature brought back to life must make a Fortitude save at DC 13 or not return to life due to the trauma suffered.\\
\textbf{This spell should not be available. Only a Patron can bring back to life.}

\medskip\textbf{Pure Resurrection}\index[Spells]{Pure Resurrection}\\
\textbf{School}: Necromancy\\
\textbf{Level}: 9, Legendary\\
\textbf{Launch Time}: 1 hour\\
\textbf{Range}: Contact\\
\textbf{Components}: V, S, M (some Holy Water and diamonds worth 25,000 gp, which the spell consumes)\\
\textbf{Duration}: Instant\\
You cast the spell on contact with a creature, who is not an Elf, who has been dead for no more than 200 years and who died of any reason other than old age. If its soul is free and willing, the creature will return to life with all its hit points. \\
This spell closes all wounds, neutralizes any poison, cures all diseases, and removes any curse that afflicted the creature when it died. The spell replaces damaged organs and limbs.\\
The spell can also provide a new body if the original no longer exists, in which case you must speak the creature's name. The creature will then respawn in an unoccupied space of your choice, within 10 feet of you. \\
\textbf{This spell should not be available. Only a Patron can bring back to life.}

\medskip\textbf{Save the Dying}\index[Spells]{Cantrip - Save the Dying}\\
\textbf{School}: Animals and Plants\\
\textbf{Level}: 0, Legendary\\
\textbf{Cast Time}: 1 round\\
\textbf{Range}: Contact\\
\textbf{Components}: V, S, M (an offering to your Patron of at least 5 gp, which the spell consumes)\\
\textbf{Duration}: Instant\\
A creature with 0 hit points that you are in contact with returns to 1 hit point. The spell has no effect on undead or constructs.\\
\textbf{For each Magical Critical Success obtained} in the Magic Test you heal the creature by 1d4 Hit Points.

\medskip\textbf{Planar Shift}\index[Spells]{Planar Shift}\\
\textbf{School}: Summon\\
\textbf{Level}: 7, Legendary\\
\textbf{Launch Time}: 2 Shares\\
\textbf{Range}: Contact\\
\textbf{Components}: V, S, M (a forked metal rod worth at least 250 gp, attuned to a specific plane of existence)\\
\textbf{Duration}: Instant\\
You and up to eight other consenting creatures, who join hands to form a circle, are transported to a different plane of existence. You can specify a target destination in general terms, and you will respawn in or near that destination, at the Storyteller's discretion.\\
Alternatively, if you know the sigil sequence of a teleportation circle to another plane of existence, the spell can lead you to that circle. If the teleportation circle is too small to fit all the creatures you carry with you, they will respawn in the unoccupied space closest to the circle.\\
You can use this spell to banish an unwilling creature to another plane. Choose a creature within reach and make a melee spell attack against it. If you hit, the creature must make a Will saving throw. If the creature fails its saving throw, it is transported to a random location on the plane of existence you specify. A creature thus transported will have to find its own way back to your current plane of existence.

\medskip\textbf{Storm of Vengeance}\index[Spells]{Storm of Vengeance}\\
\textbf{School}: Air, Water\\
\textbf{Level}: 9, Legendary\\
\textbf{Launch Time}: 2 Shares\\
\textbf{Range}: Sight\\
\textbf{Components}: V, S\\
\textbf{Duration}: Concentration, maximum 1 minute\\
A simmering storm cloud forms, centered somewhere you can see and spreading out to a 110 meter radius. The area is illuminated by lightning, thunder echoes and strong winds sweep through it. When the cloud appears, each creature beneath it (that is, no more than 5,000 feet below the cloud) must make a Fortitude saving throw. On a failed save, the creature takes 2d6 sonic damage and is deafened for 5 minutes.\\
Each round you maintain concentration on this spell, the storm produces additional effects during your round.\\
\emph{Round 2}. Acid rain falls from the cloud. Each creature and object under the cloud takes 1d6 acid damage.\\
\emph{Round 3}. You call down six bolts of lightning from the cloud to strike six creatures or objects of your choice beneath the cloud. A specific creature or object cannot be struck by more than one bolt of lightning. An affected creature must make a Reflex saving throw. The creature takes 10d6 lightning damage on a failed save, or half as much damage on a successful one. \\
\emph{Round 4}. The cloud produces a heavy hailstorm. Each creature under the cloud takes 2d6 bludgeoning damage.\\
\emph{Round 5-10}. Gusts of wind and freezing rain hit the area beneath the cloud. The area becomes difficult terrain and is in dim light. Each creature in the area takes 1d6 cold damage. In the area it becomes impossible to carry out attacks with ranged weapons. Wind and rain are considered a serious distraction for the purposes of maintaining concentration on spells.\\ Finally, gusts of strong wind (ranging from 30 to 75 kilometers per hour) automatically disperse fog, mist and similar phenomena in the area , whether natural or magical.

\medskip\textbf{Find Familiar}\index[Spells]{Find Familiar}\\
\textbf{School}: Animals and Plants\\
\textbf{Level}: 1, Legendary\\
\textbf{Launch Time}: 1 hour\\
\textbf{Range}: 3 metres\\
\textbf{Components}: V, S, M (10 gp of charcoal, incense, and herbs to be consumed by fire in a brass brazier)\\
\textbf{Duration}: Instant\\
Gain the service of a familiar, a spirit that takes on an animal form of your choosing: seahorse, raven, weasel, hawk, cat, crab, owl, lizard, fish (whim), octopus, bat, spider, frog (toad) , poisonous rat or snake. Appearing in an unoccupied space within range, the familiar has the statistics of the chosen form, though it is a celestial, fey, or demon type (your choice) instead of a beast. Your familiar acts independently of you, but always obeys your commands. In combat, he rolls initiative and acts during his round. A familiar cannot attack, but can perform other actions as normal.
You cannot have more than one pet at a time. \\
\textbf{Check Familiar Skill} For pet abilities, you must have the Familiar Skill.


\end{multicols}

\vfill

\begin{center}
\includegraphics[keepaspectratio,width=0.60\textwidth]{immagini/Goetic_circle_from_The_Lesser_Key_of_Solomon.png}

\emph{The Circle of Solomon and Triangle of Solomon from The Goetia: The Lesser Key of Solomon the King, The Book of Evil Spirits by L. W. De Laurence}
\end{center}

\pagebreak



\subsection{Spell list by List, Rarity, Level}\hypertarget{school list}{}

Next to the title of each Magic List, the Characteristic linked to establishing the maximum castable level is indicated, the Rarity and level of the magic is indicated next to each spell.

\begin{multicols}{3}


\flushleft{\textbf{Water List - Dexterity}}

Frost Ray, Common, 0\\
Energy Weapon, Very Rare, 1\\
Create or Destroy Water, Common, 1\\
Cure Light Wounds, Common, 1\\
Fog Cloud, Common, 1\\
Acid Arrow, Common, 2\\
Walking on Water, Municipality, 3\\
Nauseating Fog, Uncommon, 3\\
Breathing Under Water, Common, 3\\
Remove Poison, Common, 3\\
Sleet Storm, Very Rare, 3\\
Check Water, Municipality, (4)\\
Summon Lesser Elementals, Uncommon, 4\\
Fire Shield, Uncommon, 4\\
Ice Storm, Uncommon, 4\\
Cone of Cold, Municipality, 5\\
Summon Elemental, Rare, 5\\
Death Mist, Rare, 5\\
Ice Wall, Municipality, (6)\\
Freezing Sphere, Rare, 6\\
Check Weather, Rare, 8\\


\flushleft{\medskip\textbf{List of the Air - Charisma}}

Stretta Folgorante, Municipality, 0\\
Energy Weapon, Very Rare, 1
Fall Feather, Common, 1\\
Fog Cloud, Common, 1\\
Thundering Wave, Municipality, 1\\
Skip, Common, 1\\
Gust of Wind, Municipality, 2\\
Levitation, Common, 2\\
Glittering Dust, Uncommon, 2\\
Lightning, Municipality, 3\\
Call the Lightning, Common, 3\\
Wall of Wind, Uncommon, 3\\
Nauseating Fog, Uncommon, 3\\
Breathing Under Water, Common, 3\\
Flying, Common, 3\\
Life Bubble, Uncommon, 4\\
Summon Lesser Elementals, Uncommon, 4\\
Ice Storm, Uncommon, 4\\
Summon Elemental, Rare, 5\\
Death Mist, Rare, 5\\
Walking on the Wind, Uncommon, 6\\
Chained Lightning, Rare, 6\\
Check Weather, Very Rare, 8\\
Prismatic Wall, Rare, (9)\\


\flushleft{\medskip\textbf{Fire List - Strength}}

Produce Flame, Common, 0\\
Energy Weapon, Very Rare, 1\\
Fire Bolt, Common, 1\\
Hot Wave, Common, 1\\
Burning Blade, Common, 2\\
Flamethrower, Rare, 2\\
Pyromaster, Uncommon, 2\\
Glittering Dust, Uncommon, 2\\
Searing Ray, Common, 2\\
Heating Metal, Uncommon, 2\\
Fire Orb, Common, 2\\
Kyrin's Flaming Acorn Hail, Rare, 3\\
Cattalm's Blessing, Very Rare, 3\\
Fireball, Common, 3\\
Summon Lesser Elementals, Uncommon, 4\\
Wall of Fire, Uncommon, 4\\
Fire Shield, Uncommon, 4\\
Flaming Strike, Common, 5\\
Summon Elemental, Rare, 5\\
Delayed Fireball, Rare, 7\\
Firestorm, Rare, 7\\
Incendiary Cloud, Rare, 8\\
Meteor Swarm, Legendary, 9\\


\flushleft{\medskip\textbf{List of the Land - Constitution}}

Repair, Municipality, 0\\
Energy Weapon, Very Rare, 1\\
Eithne's Mudball. Uncommon, 1\\
Kyrin's Land Reading, Uncommon, 2\\
Pass Without Traces, Common, 2\\
One with the stone, Common, 3\\
Acid Arrow, Common, 2\\
Kyrin Lemon Gragnola, Very Rare, 2\\
Kyrin Currant Juice Concentrate, Uncommon, 2\\
Summon Lesser Elementals, Uncommon, 4\\
Stoneskin, Uncommon, 4\\
Stone Carving, Common, 4\\
Summon Elemental, Rare, 5\\
Stone Wall, Municipality, 5\\
Pass Door, Uncommon, 5\\
Stone in Mud - Mud in Stone, Uncommon - Very Rare, 5\\
Flesh in Stone - Stone in Flesh, Uncommon - Rare, 6\\
Move Ground, Uncommon, 6\\
Earthquake, Very Rare, 8\\
Meteor Swarm, Legendary, 9\\


\flushleft{\medskip\textbf{Abjuration - Intelligence}}

Resistance, Common, 0\\
Magic Armor, Uncommon, 1\\
Protection from Good and Evil, Common, 1\\
Minor Energy Protection, Rare, 1\\
Sanctuary, Municipality, 1\\
Shield, Common, 1\\
Shield of Faith, Common, 1\\
Alarm, Municipality, 1\\
Protection from Poisons, Uncommon, 2\\
Magic Lock, Common, 2\\
Interdiction bond, Municipality, 2\\
Anti-Detection, Uncommon, 3\\
Magic Circle, Common, 3\\
Counterspell, Common, 3\\
Dispel Magic, Common, 3\\
Blessing of Life, Rare, 3\\
Glyph of Warding, Common, 3\\
Energy Protection, Municipality, 3\\
Remove Curse, Common, 3\\
Life Bubble, Uncommon, 4\\
Exile, Municipality, 4\\
Death Ward, Uncommon, 4\\
Freedom of Movement, Municipality, 4\\
Private Sanctuary, Very Rare, 4\\
Dispel Good and Evil, Rare, 5\\
Dispel Magic Advanced, Rare, 5\\
Orb of Invulnerability, Common, 6\\
Forbidden, Uncommon, 6\\
Surveillance and Interdiction, Uncommon, 6\\
Symbol, Uncommon, 7\\
Sacred Aura, Municipality, 8\\
Anti-Magic Field, Rare, 8\\
Mind Shield, Uncommon, 8\\
Imprison, Rare, 9\\

\flushleft{\medskip\textbf{Animals and Plants - Wisdom}}

Enchanted Club, Common, 0\\
Poisonous Spray, Uncommon, 0\\
Hinder, Common, 1\\
Talking to Animals, Common, 1\\
Purify Food and Drink, Municipality, 1\\
Find Familiar, Legendary, 1\\
Slimy, Common, 1\\
Friendship with Animals, Uncommon, 1\\
Messenger Animal, Common, 2\\
Beneficial Berries, Common, 2\\
Growth of Spuntoni, Municipality, 2\\
Summon Mount, Common, 2\\
Kyrin's Acorn Crab, Uncommon, 2\\
Kyrin Lemon Gragnola, Very Rare, 2\\
Locate Animals and Plants, Uncommon, 2\\
Spider Moves, Uncommon, 2\\
Pass Without Traces, Common, 2\\
Barkhide, Common, 2\\
Spiderweb, Common, 2\\
Kyrin Currant Juice Concentrate, Uncommon, 2\\
Wands in Snakes, Uncommon, 3\\
Plant Growth, Uncommon, 3\\
Summon Animals, Uncommon, 3\\
Kyrin's Flaming Acorn Hail, Rare, 3\\
Talking to Plants, Rare, 3\\
Dominate Beasts, Common, 4\\
Giant Insect, Uncommon, 4\\
Locate Creature, Common, 4\\
Metamorphosis, Municipality, 4\\
Kyrin's Chestnut Crab, Very Rare, 5\\
Anti-Life Shell, Uncommon, 5\\
Insect Plague, Rare, 5\\
Reincarnation, Rare, 5\\
Awakening, Rare, 5\\
Arboreal Translation, Rare, 5\\
Wall of Thorns, Uncommon, 6\\
Vegetable Transport, Very Rare, 6\\
Animal Shapes, Rare, 8\\
Pure Metamorphosis, Rare, 9\\


\flushleft{\medskip\textbf{Enchantment - Charisma}}

Cruel Mockery, Municipality, 0\\
Finger, Rare, 0\\
Charming People, Municipality, 1\\
Command, Municipality, 1\\
Heroism, Uncommon, 1\\
Uncontainable Laughter, Uncommon, 1\\
Sleep, Common, 1\\
Anathema, Common, 1\\
Block Person, Municipality, 2\\
Calming Emotions, Municipality, 2\\
Entrance, Municipality, 2\\
Nap, Legendary, 2\\
Suggestion, Municipality, 2\\
Zone of Truth, Uncommon, 2\\
Cattalm's Blessing, Very Rare, 3\\
Block Person Advanced, Uncommon, 4\\
Compulsion, Uncommon, 4\\
Confusion, Municipality, 4\\
Dominate Beasts, Very Rare, 4\\
Duress, Rare, 5\\
Dominate People, Uncommon, 5\\
Change Memory, Very Rare, 5\\
Mass Suggestion, Uncommon, 6\\
Dislike/Like, Rare, 8\\
Contagious Confusion, Very Rare, 8\\
Irresistible Dance, Legendary, 8\\
Dominate Monster, Uncommon, 8\\
Word of Power Stun, Uncommon, 8\\
Mental Regression, Rare, 8\\
Word of Power Kill, Rare, 9\\

\flushleft{\medskip\textbf{Healing - Wisdom}}

Cure Light Wounds, Common, 1\\
Healing Word, Uncommon, 1\\
Healing Prayer, Municipality, 2\\
Ristorare Inferiore, Municipality, 2\\
Help, Uncommon, 2\\
Cure Serious Wounds, Uncommon, 3\\
Destroy Undead, Uncommon, 3\\
Mass Healing Word, Rare, 3\\
Remove Poison, Common, 3\\
Rebirth, Very Rare, 3\\
Remove Disease, Municipality, 4\\
Vigor, Rare, 4\\
Cure Critical Wounds, Uncommon, 5\\
Superior Restaurant, Uncommon, 5\\
Healing, Rare, 6\\
Regeneration, Legendary, 7\\
Mass Healing, Legendary, 9\\
Cure Mass Wounds, Uncommon, (variable)\\


\flushleft{\medskip\textbf{Divination - Wisdom}}

Accurate Hit, Common, 0\\
Language Comprehension, Common, 1\\
Guide, Municipality, 1\\
Identification of Good and Evil, Common, 1\\
Reading Comprehension, Uncommon, 2\\
Identification of Thoughts, Rare, 2\\
Disease and Poison Detection, Uncommon, 2\\
Locate Object, Municipality, 2\\
Omen, Common, 2\\
Discover Traps, Municipality, 2\\
See Invisibility, Common, 2\\
Languages, Municipality, 3\\
Clairvoyance, Common, 3\\
Arcane Eye, Common, 4\\
Communion, Rare, 5\\
Communion with Nature, Very Rare, 5\\
Knowledge of Legends, Common, 5\\
Telepathic Bond, Rare, 5\\
Scry, Rare, 5\\
Divination, Common, 6\\
Discover the Path, Uncommon, 6\\
Seeing the True, Rare, 6\\
Prediction, Uncommon, 9\\


\flushleft{\medskip\textbf{Conjuration - Intelligence}}

Craft Beer, Rare, 0\\
Acid Gush, Common, 0\\
Magic Hand, Common, 0\\
Invisible Cook, Common, 1\\
Floating Disk, Common, 1\\
Cattalm's Slap, Uncommon, 1\\
Invisible Servant, Common, 1\\
Veiled Step, Uncommon, 2\\
Create Food and Water, Common, 3\\
Summon Woodland Creatures, Legendary, 4\\
Dimension Door, Common, 4\\
Secret Chest, Rare, 4\\
Loyal Hound, Rare, 4\\
Black Tentacles, Uncommon, 4\\
Teleportation Circle, Uncommon, 5\\
Instant Summons, Rare, 6\\
Word of Retreat, Rare, 6\\
Planar Ally, Legendary, 6\\
Summon Celestials, Legendary, 7\\
Wonderful Palace, Legendary, 7\\
Planar Shift, Legendary, 7\\
Teleport, Common, 7\\
Labyrinth, Rare, 8\\
Demiplane, Rare, 8\\
Desire, Uncommon, 9\\
Portal, Legendary, 9\\

\flushleft{\medskip\textbf{Illusion - Intelligence}}

Disguise Self, Common, 1\\
Silent Image, Common, 1\\
Illusory Writing, Common, 1\\
Color Spray, Common, 1\\
Arcanist's Magical Aura, Uncommon, 2\\
Bocca Magica, Municipality, 2\\
Mirror Image, Common, 2\\
Invisibility, Common, 2\\
Laydel's Tear, Very Rare, 2\\
Blur, Common, 2\\
Silence, Municipality, 2\\
Phantom Steed, Common, 3\\
Image Major, Common, 3\\
Fear, Uncommon, 3\\
Hypnotic Plot, Common, 3\\
Greater Invisibility, Uncommon, 4\\
Illusory Terrain, Uncommon, 4\\
Deadly Hallucination, Uncommon, 4\\
Crafting, Rare, 5\\
Mislead, Uncommon, 5\\
Seem, Uncommon, 5\\
Dream, Uncommon, 5\\
Programmed Illusion, Uncommon, 6\\
Projected Image, Uncommon, 7\\
Arcane Mirage, Rare, 7\\
Simulacrum, Rare, 7\\
Fatal, Rare, 9\\

\flushleft{\medskip\textbf{Invocation - Constitution}}

Dancing Lights, Uncommon, 0\\
Flaming Strike, Rare, 1\\
Tracer Bolt, Uncommon, 1\\
Hidden Bolt, Common, 1\\
Divine Favor, Uncommon, 1\\
Luminescence, Uncommon, 1\\
Darkness, Invocation, 1\\
Greater Blessing, Uncommon, 2\\
Crushing, Municipality, 2\\
Marking Punishment, Common, 2\\
Spiritual Weapon, Common, 2\\
Glittering Strike, Uncommon, 2\\
Supreme Blessing, Rare, 3\\
Hut, Uncommon, 3\\
Blinding Strike, 3\\
Send, Municipality, 3\\
Daylight, Common, 3\\
Elastic Sphere, Rare, 4\\
Arcane Hand, Uncommon, 5\\
Wall of Strength, Common, 5\\
Hallow, Rare, 5\\
Solar Flare, Uncommon, 6\\
Heroes' Feast, Uncommon, 6\\
Blade Barrier, Municipality, 6\\
Circle of Death, Very Rare, 6\\
Contingency, Municipality, 6\\
Divine Word, Very Rare, 7\\
Arcane Sword, Rare, 7\\
Prismatic Spray, Rare, 7\\
Solar Blast, Rare, 8\\
Force Cage, Rare, 8\\

\flushleft{\medskip\textbf{Necromancy - Constitution}}

Frosty Touch, Common, 0\\
False Life, Common, 1\\
Cry of pain, Rare, 1\\
Blindness/Deafness, Common, 2\\
Inflict Wounds, Common, 2\\
Fatigue Radius, Common, 2\\
Inviolate Rest, Uncommon, 2\\
Help, Uncommon, 2\\
Animate Dead, Common, 3\\
Advanced Blindness/Deafness, Uncommon, 3\\
Speak with the Dead, Rare, 3\\
Rebirth, Very Rare, 3\\
Cast Curse, Uncommon, 3\\
Vampiric Touch, Common, 3\\
Wither, Uncommon, 4\\
Contagion, Uncommon, 5\\
Raise Dead, Legendary, 5\\
Create Undead, Uncommon, 6\\
Finger of Death, Common, 6\\
Wound, Uncommon, 6\\
Magic Jar, Very Rare, 6\\
Piercing Gaze, Very Rare, 6\\
Resurrection, Legendary, 7\\
Clone, Uncommon, 8\\
Astral Projection, Very Rare, 9\\
Pure Resurrection, Legendary, 9\\

\flushleft{\medskip\textbf{Transmutation - Dexterity}}

Message, Common, 0\\
Fast Pass, Very Rare, 1\\
Rapid Retreat, Uncommon, 1\\
Alter Self, Common, 1\\
Magic Weapon, Common, 2\\
Enhanced Feature, Common, 2\\
Enlarge/Reduce, Common, 2\\
Burglary, Municipality, 2\\
Darkvision, Common, 2\\
Rope Trick, Common, 2\\
Gaseous Form, Uncommon, 3\\
Intermittent, Uncommon, 3\\
Slow, Uncommon, 3\\
Speed, Uncommon, 3\\
Fabricate, Municipality, 4\\
Animate Objects, Common, 5\\
Telekinesis, Uncommon, 5\\
Disintegration, Uncommon, 6\\
Restser's Furious Transformation, Very Rare, 6\\
Conceal, Rare, 7\\
Ethereal Form, Rare, 7\\
Gravity Reversal, Rare, 7\\
Talkativeness, Rare, 8\\
Stopping Time, Very Rare, 9\\
Transformation, Rare, 9\\

\flushleft{\medskip\textbf{Universal - any}}

Sacred Flame, Common, 0\\
Magic Mark, Common, 0\\
Prestidigitation, Common, 0\\
Thaumaturgy, Uncommon, 0\\
Druidic Artifice, Non Municipality, 0\\
Blessing, Municipality, 1\\
Arcane Bolt, Common, 1\\
Identify, Municipality, 1\\
Minor Illusion, Common, 1\\
Detect Magic, Common, 1\\
Magic Reading, Common, 1\\
Light, Common, 1\\
Cast Lesser Curse, Common, 1\\
Bless Water, Municipality, 2\\
Everflame, Legendary, 2\\

\end{multicols}

%\vfill

%\begin{center}
%\includegraphics[width=0.3\linewidth]{immagini/the-discovery-of-witchcraft.png}
%\emph{"The Discoverie of Witchcraft' by Reginald Scot, 16th century }
%\end{center}

you can
\pagebreak



\section{Advantages}\index{Advantages}\hypertarget{advantages}{}\label{vantaggiinizio}

\begin{changemargin}{0.3cm}{0.3cm}\begin{enfasi}{I love being a superhero! The working hours are terrible, the pay is non-existent... but at least I don't run the risk of getting fired! (PK)
}\end{enfasi}\end{changemargin}\medskip

\begin{multicols}{2}

\lettrine[lines=2, lhang=0.33, loversize=0.25, findent=1.5em]{Each} character can have, and is not required to have, Advantages. These must be interesting, enjoyable, fun and above all playable.

Each Perk has a cost, to be paid at each level. It shouldn't be mandatory to take an Advantage, nor should you take advantages just because they make you strong. The purpose of a Vantage is to amaze and have fun.

Having an Advantage means being different, being a freak, having that detail that makes you special and unique, but not always the strongest, most powerful or invincible. An advantage isn't just an ability, it's a role-playing opportunity. The player is invited to be creative in choosing advantages and also in creating new ones, the cost is then decided with the Narrator. And it is always the Narrator who has the final say on the chosen Advantages.

Various advantages do not have a concrete and immediate practical effect but rather enrich the character's background and history. When you choose the advantages, and consequently the disadvantages, it is not like going shopping for super powers and extraordinary abilities, but for peculiarities, foibles, specialties that the character possesses and which once again make him different, unique, only yours .

Therefore advantages and disadvantages must also and above all be played and interpreted.

The Narrator could also insert thematic advantages and disadvantages to the adventure but also peculiar to the characterization of the character such as immunity to diseases, healing touches, extrasensory abilities, abilities that modify the relationship with a familiar... Always be scrupulous in the analysis and evaluation of benefits, remembering that there must also be an adequate value of disadvantages.


\begin{itemize}

\item
Advantages with {*} and all those with a cost of 15 or higher are at the Storyteller's discretion in being eligible for selection.

\item
Advantages are chosen at the first level, any advantage taken at subsequent levels must be agreed with the Storyteller.

\item
The cost points of an Advantage are paid with the points taken from Disadvantages.

\item
Bonuses given to skills are test-specific when indicated in parentheses.

\item Unless otherwise indicated, it costs two Actions to activate an Advantage (if the effect is not permanent).

\end{itemize}


\begin{changemargin}{0.3cm}{0.3cm}\begin{enfasi}{
From a great advantage comes a great disadvantage! (cit. \emph{With great power comes great responsibility}, Amazing Fantasy 15, Stan Lee)
}\end{enfasi}\end{changemargin}\medskip


\subsection{List of Benefits}\index{List of Benefits}

\textbf{Wings of Providence} \index{Wings of Providence}20: you have wings, the choice of shape and color is up to you, they are usually on your shoulder blades and make you fly. Unless otherwise agreed, movement in flight is equal to racial movement on the ground.

\textbf{Ambidextrous}\index{Ambidextrous} 10: you can use both hands indifferently. Penalties on checks where two hands are used decrease by 2

\textbf{Animal Friend}\index{Animal Friend} 5: +2 on checks to handle animals (even wild ones)

\textbf{Amphibian}\index{Amphibian} 20: you can breathe both underwater and air

\textbf{Rainbow}\index{Rainbow} 5: you are an artist. Your fingers spontaneously produce color

\textbf{Aura of courage}\index{Aura of courage} 15: around you, in the distance within 3 meters you instill courage. +2 Saving Throw vs. natural or magical fear effects.

\textbf{Claws}\index{Claws} 5: every now and then remember to trim the claws. 1d4 damage per attack. Natural attacks with the second hand take the damage bonus given by Strength.
10: Claws deal 1d6 damage.

%\begin{center}
%\includegraphics[width=0.3\linewidth]{immagini/claw.png}
%\end{center}


\textbf{Drinking is good for you}\index{Drinking is good for you} 5: Prerequisite: The liver does not count. Your body metabolizes alcohol very effectively. A liter of beer recovers 1d4 Hit Points, a bottle of liquor 1d8 Hit Points. Not if it's bad quality.. You can still get drunk.

10: 1 liter of beer recovers 2d4 Hit Points, a bottle of good liquor 2d8 Hit Points. If of poor quality, no. You can't get drunk on natural liquids.

\textbf{Cat fall}\index{Cat fall} 5: You ignore the first 3 meters of the fall. +2 to Stealth.

\textbf{Chameleon}\index{Chameleon} 10-20: Your skin can change color. Time needed 1 minute/1 round.

\textbf{Shapechanger}\index{Shapechanger} 40: As an Alter Self spell. It can be used every 10 minutes.

\textbf{Walking on air} \index{Walking on air}30: not too controlled. Anything other than walking requires a Dexterity check or falling prone (but not on the ground).

\textbf{Walking on water} \index{Walking on water} 30: but don't put on airs..

\textbf{Magnetic} \index{Magnetic}5-10: release light whenever you want. luckily not literally. $\pm1/2$ for Charisma based tests.

\textbf{Reduced consumption} \index{Reduced consumption}5: you drink and eat half as much as a normal man. You are under weight.

\textbf{Metabolism Checker} \index{Metabolism Checker} 10: The name alone is amazing! Each round you reduce your Bleed damage by 2.

You regain hit points as if you had double your Constitution score.

\textbf{Effective Heals} \index{Effective Heals}10: +1d6 Hit Points healed whenever you use a Heal spell on yourself or others.

\textbf{Daredevil} \index{Daredevil}10: you like to throw yourself into the fray, especially if there is danger. +2 Attack/Defense Rolls while surrounded by three or more opponents.

\textbf{Teeth} \index{Teeth}5: Your bite hurts, 1d6. Brush your teeth every now and then..

\textbf{Universal digestion} \index{Universal digestion}5: as long as it doesn't hurt you eat, +2 Fortitude saving throw vs. poisons. Immune to natural stomach disorders.

\textbf{Absolute Direction} \index{Absolute Direction}5: You always know where magnetic north is. You have a +1d6 on orientation checks.

\textbf{Hard to subdue} \index{Hard to subdue}10: +2 Will save on spells from the Enchantment List.

\textbf{Hard to kill} \index{Hard to kill}5: You do not faint at 0 hit points, but at -LV/2 in hit points. Die at 15 + Constitutionx3 Hit Points.

\textbf{Empathy with plants} \index{Empathy with plants}10: I understand the suffering of crushed grass.

\textbf{Empathy} 5: +2 on Sense Emotion checks.

\textbf{Animal Empathy} \index{Animal Empathy}10: +1d6 on checks to handle animals (even wild ones).

\textbf{Spiritual Empathy} \index{Spiritual Empathy}5: You don't talk to spirits, but you feel their emotions.

\textbf{Hermaphrodite} \index{Hermaphrodite}0: lgbtE!.

\textbf{Forged in Steel} \index{Forged in Steel}5: Through painful operations your skin has been covered with metal plates. Your base Defense is 13.

\textbf{Shadow Form} \index{Shadow Form}30: Consider being able to transform into a shadow for 1 hour per level. You can only move in shaded areas.

\textbf{Lucky} \index{Lucky}5: 2 times a day you can roll a 1 on the die to 6, to be declared before the die roll.

\textbf{Very Lucky} \index{Very Lucky} 10: 2 times a day you can roll a 1 or a 2 on the 6 die, to be declared even after the die roll.

\textbf{Accelerated healing}\index{Accelerated healing}: 5 every morning you recover double the Hit Points you would normally recover. It stacks with Metabolism Control. \index{Metabolism Control}.

\textbf{Healer}\index{Healer} 5: You know where to put your hands. +1d6 on First Aid checks

\textbf{The liver cannot be counted} \index{The liver cannot be counted}10: you can drink a lot and you don't get drunk

\textbf{Enlightened} \index{Enlightened}10-20: shed light.. literally. Emit light in a 10 to 20 foot radius, 1 hour per level. You can control (20) the output or not (10).

\textbf{Immune}\index{Immune} 5-20: to what?

\textbf{Invisible} \index{Invisible}40: Your body is invisible. Always. And it's not magic...

\textbf{Anger} \index{Anger}5: You are capable of becoming enraged. +2 to melee damage and -1 attack and defense rolls. Every other 5 points +2 damage -1 Attack and Defense Roll, max 20 points. Duration 4 (even non-consecutive) rounds every 5 points. It is activated with an Action.

\textbf{My shadow is my friend} \index{My shadow is my friend}10: You can place your shadow wherever you want within 3 meters. Your shadow can manipulate objects like an invisible minion. You are considered to be able to cast touch spells via your shadow (which must be present) within 10 feet. The conditions must exist for there to be a shadow.

\begin{center}
\includegraphics[width=0.35\linewidth]{immagini/shadow.png}
\end{center}

\textbf{Bounds of Fury} \begin{document}9855{Bounds of Fury} : You can summon ethereal bonds that threaten your enemies. For 3 times a day with the cost of 2 Actions all opponents within 9 meters around you are affected by the Encumber spell until the next round, DC 10 + 1/2 LV + Charisma.

\textbf{Slow and Steady} 5: \index{Slow and Steady}You are exceptionally stable on your feet. You can't be moved or lifted except by a creature 2 sizes larger.

\textbf{Universal language} \index{Universal language}10. Your language skills are impressive. After two days in contact with a new language you are able to speak it correctly. After 7 days of being away from the environment you forget the language. You gain a +2 on language-based checks.

\textbf{Explosive Magic} \index{Explosive Magic}10: Your Invocation spells that cause damage have an extra die of damage (when a die needs to be rolled..).

\textbf{Fairy Hands} \index{Fairy Hands}10: +1d6 Fairy Hands and Escape Artist checks involving hands. You can get a 14 as if you would get a 10 on the relevant tests.

\textbf{Webbed Hand Foot} \index{Webbed Hand Foot}5: +1d6 on swim checks.

\textbf{Early bird} \index{Early bird}5-10-15: you just need to sleep 6/5/4 hours a night to be completely rested.

\textbf{Medium} \index{Medium}10-20: sometimes you want it, other times they look for you.

\textbf{Photographic memory}\index{Photographic memory} 20-50: fortunately it is not permanent (50). +2d6 on recall checks (Knowledge and Awareness).

\textbf{Hairy nose} \index{Hairy nose}5: Your nostrils filter toxins in the air you breathe. +2 on related checks. Your nose is size... not small.

\textbf{You don't sleep}\index{You don't sleep} 20{*}: and I don't know how you do it..

\textbf{You don't age}\index{You don't age} 20{*}: You don't age (but they can kill you anyway).

\textbf{You don't eat you drink} \index{You don't eat you drink}20: and I don't know how you do it..

\textbf{You're not breathing} \index{You're not breathing}20: and I don't know how you do it..

\textbf{The Smell of Blood} \index{The Smell of Blood}10: The smell of blood is a powerful drug
Prerequisites: You cannot have \emph{Liver does not count}. You gain +1 to attack rolls and +1 to damage rolls for each enemy you kill with your weapon in the round. This bonus cannot exceed +4/+4. The bonus remains active until the round following the last kill made. Creatures with less than 3 LV of you do not count.

\textbf{Oracle} \index{Oracle}20: for some it is a curse. The use must always be agreed with the Narrator and the Patron.

\textbf{Excellent view} \index{Excellent view}5: you have an excellent view (12/10). +2 on related Awareness checks that use sight.

\textbf{Excellent sense of smell and taste} \index{Excellent sense of smell and taste}5: You have excellent taste and sense of smell. +2 on Awareness checks that use smell or taste.

You gain +1d6 on checks to recognize a potion or natural poison.

\textbf{Excellent touch} \index{Excellent touch}5: You have excellent touch. you know how to read with your fingers. You are able to find a hidden door by touching the wall.

\begin{center}
\includegraphics[width=0.9\linewidth]{immagini/braille2.png}
\end{center}


\textbf{Excellent hearing}\index{Excellent hearing} 5: You have excellent hearing. +2 on Awareness checks involving hearing.

\textbf{Talking with animals}\index{Talking with animals} 20: choose a family (sheep, marsupials, cavies..).

\textbf{Talking with plants} \index{Talking with plants}25: I've always wanted to talk to courgettes..

\textbf{Blindsight}(blindsight):\index{Blindsense} \index{Blindsight}30: You can perceive anything with your senses within 60 feet, from smell to heat. You can \emph{see} through and up 18 meters, 5 cm of stone, 10 cm of wood, 0.2 cm of metal.

\textbf{Perfect balance} 5:\index{Perfect balance} +2 on related Acrobatics checks.

\textbf{Fast Feet}\index{Fast Feet} 10/20: Your movement increases by 1/2 metres.

\textbf{Green Thumb}\index{Green Thumb} 5: +1d6 on Profession checks (Herbalist, Gardener..).

\textbf{Iron Lungs}\index{Iron Lungs} 5: You can hold your breath 2*Constitution minutes (minimum 2 minutes).

\textbf{Precognition} 30{*}\index{Precognition}: Same as the Foresight spell.

\textbf{Recovery}\index{Recovery} 10: Your body spontaneously produces caffeine. It takes you half the time to recover from the fatigued condition.

\textbf{Resistance}\index{Resistance} 5-10: +1/+2 Reflex or Fortitude or Will saving throw.

\textbf{Damage Resistance}\index{Damage Resistance} 10: -1 damage. -1 additional damage for every 5 additional points. Establishes the type of resistance (cutting, blunt, piercing, fire...).

\textbf{Magic Resistance}\index{Magic Resistance} 20: You have Magic Resistance 2.

\textbf{Reconstruction}\index{Reconstruction} 30: losing a hand has never been a problem..

\textbf{Regeneration}\index{Regeneration} 30: +1 Hit Points per Turn (you do not regenerate limbs).

\textbf{Fast regeneration} 40: +1 Hit Points per round (you do not regenerate limbs). You die if they destroy your body (or all that's left is ash).

\textbf{Shortcut} 30: you can decrease up to two sizes. Lasts up to 8 hours.

\hypertarget{Rhinoceros}{}\textbf{Rhinoceros} 10 : Your charge is destructive. It is considered that nothing under the strength of iron bars (hardness 15) can stop your charge. You leave a trail of destruction behind you. +2 on charging attack rolls and +1d6 damage.

\textbf{Mental Shield}\index{Mental Shield} 5: +2 Saving Throw on mental controls and influences.

\textbf{Senses protected}\index{Senses protected} 5: +2 Saving Throw against sounds/lights/vapors or spells that act on and through your senses.

\textbf{Common sense}\index{Common sense} 5: if you are about to make a bad impression a little bell warns you.

\textbf{Fashion sense}\index{Fashion sense} 5: you always know how to dress well, even with just a rag.

\textbf{Sense of vibrations} \index{Sense of vibrations} \index{Sense Telluric} (Sense Telluric) 30: everything makes the earth tremble a little, or almost, within an 18 meter radius around you.

\textbf{Sense of time} \index{Sense of time}5: you always know what time it is, day or night.

\textbf{Spider Sense}\index{Spider Sense} 15: No, a radioactive man didn't bite you, but you are extremely sensitive to danger. +2 initiative, you can't be surprised.

\textbf{Fearless} \index{Fearless}10: You are immune to fear, magical or otherwise.

\textbf{Silent} \index{Silent}5: +1d6 on Stealth checks.

\textbf{Spine} \index{Spine}5: and you're ugly too. 1d4 damage.

\textbf{Super Dog Tags} \index{Super Dog Tags}5: Reduce Bleed damage by 1 at the end of each round.

\textbf{Language Talent}\index{Language Talent} 5: You learn two languages ​​by investing 1 point in Linguistic Knowledge.

\textbf{Wild Talent}: \index{Wild Talent}Let's talk about it.

\textbf{Icy Touch} \index{Icy Touch}10: By touching a dead person (within 1 day per level) you can see and feel what happened in their last round of life.

\textbf{Troll} \index{Troll}60: You regenerate 5 hit points per round even if the hit points are negative. You also regenerate limbs. You can only be \emph{killed} by fire or acid. A condition could still keep you at negative hit points (e.g. submerged underwater).

\textbf{Subsonic hearing}\index{Subsonic hearing} 10: Hear frequencies inaudible to humans (like a dog)

\textbf{Seeing the invisible} \index{Seeing the invisible}15: X-ray vision is better.. drool..

\textbf{Understanding the truth}\index{Understanding the truth} 5: The truth has a sound of its own. +1d6 on Sense Emotion checks.

\textbf{Demonic Sight} \index{Demonic Sight}15: See in total darkness, even magical, up to 18 meters.

\textbf{Perimeter View} \index{Perimeter View}5: Sole ? +2 on Awareness checks from side.

\textbf{Telescopic Vision}\index{Telescopic Vision} 10: +1d6 on Awareness checks based on vision but from afar.

\textbf{Soothing Voice} \index{Soothing Voice}5: +2 to Charisma checks that use the voice,

\textbf{Subsonic Voice}\index{Subsonic Voice} 10: Emit sounds inaudible to humans. Dogs hate you.

\end{multicols}

\pagebreak

\section{Disadvantages}\index{Disadvantages}

\begin{changemargin}{0.3cm}{0.3cm}\begin{enfasi}{If you have to be a cripple, better be a rich cripple. (Tyrion Lannister)}\end{enfasi}\end{changemargin}\medskip

\begin{multicols}{2}

\lettrine[lines=2, lhang=0.33, loversize=0.25, findent=1.5em]{U}{no} disadvantage characterizes the character, defines his limits and fears. Each character must have at least 1 role disadvantage and this does not give them bonus points.

The points taken with the psychological/physical Disadvantages serve to cover the points spent with the Advantages. Obviously the Evil Narrator also likes more disadvantages...

€9982 € {Each player must play his disadvantages otherwise he will not gain experience points and will be denied the use of Advantages.}

A disadvantage can be \emph{cancelled} during the character's story and there must be an adventure that justifies everything. As always the Storyteller has the final say on any choice of advantages and disadvantages.

\bigskip

Suggestions
\begin{itemize}
\item
Get cons that are fun to play, even if they will get you into trouble.
\item
Get cons that are interesting to play with other players even if they get them into trouble.
\item
Get disadvantages that fit the character.
\item
Get some cons you won't regret.
\end{itemize}

\textbf{Be careful}:

\begin{itemize}
\item
Avoid cons that are difficult to play either because they are completely out of touch with the system or totally useless or severely harmful to others. If you want to be an extreme pacifist, evaluate the character and the group carefully.
\item
Don't take any disadvantages that might make you feel ashamed to act.
\item
Don't take disadvantages that have nothing to do with the character (in perfect contradiction to what has already been said...).
\item
Don't take silly disadvantages (like the fear of turning right, of elevators...).
\item
If you take a severe disadvantage, recite it well, the Storyteller will be able to reward you.
\end{itemize}

\subsection{Role Disadvantages and Psychological/Physical Disadvantages}\hypertarget{disadvantages}{}\label{svantaggidiruolo}

The disadvantages are divided into two categories, \textbf{Role Disadvantages} and \textbf{Psycho/Physical Disadvantages}.

The \textbf{Role Disadvantages} are small defects, tics, large and small problems that serve to give a more \emph{human} depth to the character. They have a deliberately ambiguous and playful description, choose them carefully and discuss with the Narrator how you intend to interpret this disadvantage.

The player is invited to create new role disadvantages. These disadvantages do not grant a bonus or penalty or give points for taking advantages. \emph{But they are fun!}

\bigskip

The \textbf{Psycho/physical disadvantages} are instead more impactful in the game, in everyday life, giving concrete disadvantages. These disadvantages provide the points with which \textbf{pay} for the advantages. At the bottom you will find a list of Phobias.

\end{multicols}

\pagebreak

\subsubsection{Role Disadvantages}\index{Role Disadvantages}


\begin{multicols}{2}


\textbf{Alcoholism}:\index{Alcoholism} you like to drink, and a lot.. but when will you stop?

\textbf{Fashionable}\index{Fashionable}: probably yours, even with new clothes you never dress well. The color combination is always an eyesore.

\textbf{Animal Friendly}:\index{Animal Friendly} meaning fleas, ticks, lice, bedbugs... flies. You have a zoo on you.

\textbf{Attracts animals}: \index{Attracts animals}you don't know why but you are always surrounded by cats, dogs, bunnies, cockatrices..

\textbf{He invites trouble}\index{He attracts trouble}: it's not my fault if the dragon deviated to come and poop here..

\textbf{Banana}: \index{Banana}the one you try to get in your hair, but can't. Your hair doesn't agree with you.

\textbf{Low pain threshold}: \index{Low pain threshold} he scratched me, help! I am dying!!!

\textbf{Pimples}: \index{Pimples}full, your face is pockmarked and these disgusting yellow pimples keep forming.

\textbf{Ciuccione}: \index{Ciuccione}you don't do it often, but in the moments when you are most nervous you take out the old wooden pacifier... (or failing that, your thumb is always fine).

\textbf{Coward}:\index{Coward} it's better to run away, sorry, let's gather all the information first before attacking.

\textbf{Cogito ergo sum}: \index{Cogito ergo sum}you have a tendency to talk to yourself, but out loud even if there are people around and even if they are not friendly.

\textbf{Advisers}\index{Advisors}: they never understand but you do it for them. They never understand how generous you are with your valuable advice.

\textbf{Gullible}: \index{Gullible}come on? Really ? and at what height did the donkey fly?

\textbf{Hamster}: \index{Hamster}intended as memory. You can't associate names with faces.

\textbf{Rotten teeth}: \index{Rotten teeth}probably the toothbrush you use doesn't have real boar bristles...

\textbf{Nose picking}:\index{Nose picking} I hope they are at least good.

\textbf{Diva}: \index{Diva}or at least you think you are. You don't miss an opportunity to show off your non-existent singing, comic and aesthetic skills... with everyone's laughter.

\textbf{Common face}: \index{Common face}What's your name? I think I've seen you before...

\textbf{Galant}: \index{Galant}bordering on the maniacal, in every gesture you make you are formal, appropriate and friendly.

\textbf{Killer}:\index{Killer} no, you are not a murderer. But your hands and feet are always cold.

\textbf{Frighten animals}: \index{Frighten animals}can also be convenient, if it weren't for the horses that run away and the bears that attack...

\textbf{Unable to have fun}: \index{Unable to have fun}so? It's your problem, not mine.

\textbf{English}: \index{English}intended as humour. Nobody ever gets your jokes.

\textbf{Mangione}: \index{Mangione}CIOMP!. Never skimp, it could be your last meal!

\textbf{Meteora}:\index{Meteora} you suffer from compulsive and noisy bloating, not to mention the unpleasant smell.

\textbf{Megalomaniac}:\index{Megalomaniac} let's involve the armies of the seven kingdoms and penetrate the dungeon!

\textbf{Mint}: \index{Mint}if you only ate garlic and onion your breath would be less smelly

\textbf{Musician}: \index{Musician}with the mouth. You whistle constantly, whenever you are lost in thought or very tense. you start whistling.

\textbf{Not empathetic}: \index{Not empathetic}why is the child whose teddy bear I just set on fire crying?

\textbf{Obsession}:\index{Obsession} again, again, again. Another tube of skin cream!!

\textbf{Package}: \index{Package}yours. You always have a hand down there. Maybe the pants are tight? and no, I won't shake your hand.

\textbf{Bad character}: \index{Bad character}It's okay to be grumpy.. but do you always have to make it obvious?

\textbf{Pezzata}: \index{Pezzata}no, not the cow or your mare but your armpit. You sweat profusely, whether it's hot or cold... or nervous.

\textbf{Mental rigidity}:\index{Mental rigidity} no, I don't understand, the map says to go right. I don't care if there isn't a right.

\textbf{Knowing}\index{Knowing}: the right answer is only yours. There's no doubt about it... for you.

\textbf{Nosebleeds}: \index{Nosebleeds} happen, and always as soon as you see a woman/man (depending on your tastes) that you like.

\textbf{Scarf}: \index{Scarf}you must always have a certain type of garment on you and visible, otherwise you won't leave the cave.

\textbf{Secret}: \index{Secret}I have a secret, so much of a secret that I don't know if I even know it...

\textbf{Following Chaos}: \index{Following Chaos}is stronger than you, you can never obey any law or authority in charge.

\textbf{Follow the Law}: \index{Follow the Law}is stronger than you, no matter what the law is, you don't break it.

\textbf{Tattooed:} \index{Tattooed}tattooing is the way of life. You already have at least 30\% of your body tattooed and you don't miss opportunities to get new tattoos.

\textbf{Topi}:\index{Topi} you are a TOPI!

\textbf{Umarel}\index{Umarel}: Whenever there is a construction job he is stronger than you, you have to stop and comment on the terrible skill of the workers or designers.

\textbf{Nails}:\index{Nails} you are a compulsive nail eater, the tips of your fingers bleed sometimes.

\textbf{Last word}\index{Last word}: he is stronger than you, you must have the last word in every speech.

\textbf{Old inside}\index{Old inside}: \emph{Ehhh in my time!}. It's not a question of age. You always have to complain about everything, your vitality is that of an eighty year old.

\end{multicols}

\pagebreak

\subsection{Psycho/physical disadvantages}\index{Psycho/physical disadvantages}


\begin{multicols}{2}

\textbf{Albino}\index{Albino}

You are White, almost like milk. You don't tan and you can't stand the light, your skin is delicate.

\textbf{13}: In addition to being extremely recognizable you have the following disadvantages: Myopia and Photosensitivity and Sensitive Skin.

\textbf{Allergy}\index{Allergy}

You have some type of allergy. I hope not serious. Make sure you always have a poison removal potion with you.

\textbf{5:} In the presence of a specific allergen the character sneezes loudly until the allergen is removed, -1 on all checks (e.g. Allergic to Beer).

\textbf{10}: The character suffers from coughing attacks, hypertearing, dizziness, -2 to all checks. Fortitude save DC 10 to avoid suffocation. The roll must be repeated every 20 rounds until you have moved away from the allergen.

\textbf{15:} The character suffers from violent coughing attacks, nausea, cold sweats, palpitation. -1d6 on all checks, requires a DC 15 Fortitude save or fall unconscious. The rolls must be repeated every 5 rounds until the allergen is removed.

\textbf{20}: The character falls prey to a respiratory crisis, and is unable to perform any action other than vomiting, gasping for air and trying to survive. Failing a DC 25 Fortitude save causes the character to die in inhuman spasms. The roll must be repeated every round until the allergen is removed.

Note: allergens that are too rare are not valid.

\textbf{Hallucinations}\index{Hallucinations}

there is something wrong in your head, every now and then a spark is triggered.

\textbf{10}: The character sees and hears things that aren't there. Each day you roll 1d6.
If 1 or 2 come up, nothing happens.
On a 3,4 or 5 one or two hallucinatory episodes will occur with methods and times at the discretion of the Narrator.
On a 6 the character will be the victim of horrendous and disgusting visions (or the opposite) lasting 1d4 hours.

\textbf{Amnesia}\index{Amnesia}

\textbf{10}: You have forgotten your past and with it the memory of friends, enemies, goals. There is no way to recover lost memories.

\textbf{Ascetic}\index{Ascetic}

10, the rule says so. You will not bring more than 10 items with you.

\textbf{20}: You cannot own more than 10 magic or normal items or coins or weapons. Luckily, clothes don't matter.

\textbf{Stutterer}\index{Stutterer}

You know how to speak, but badly.

\textbf{5:} You have an annoying tendency to stutter just when you have something important to say. In these critical situations only sketchy sounds come out of your lips. You have -2 on checks based on word use.

\textbf{Bad Character}\index{Bad Character}

Good manners are always optional.

\textbf{5}: You have never learned the art of diplomacy and hate being contradicted or insulted. This does not mean that you take action, but that when faced with an insult or a frank criticism you tend to silence your interlocutor with very unpleasant expressions. You have a -2 to Charisma-based checks

\textbf{Spendthrift}\index{Spendthrift}

\textbf{10}: You must spend half your mission earnings on trivial pleasures.

\textbf{15}: You must spend all your mission earnings on trivial pleasures.

You have to eat expensive foods, drink fine wine and spirits, buy luxurious clothes, rest in famous inns. You cannot spend on weapons or magical items, you can purchase non-magical equipment.

\textbf{Charitable}\index{Charitable}

\textbf{10}: You must donate half of your mission earnings to charity.

\textbf{15}: Cannot hold more than 10 gp in cash.

\textbf{Blindness}\index{Blindness}

\textbf{10}: You are blind, impaired lateral vision, problems understanding the distance of things.
Skills such as Awareness and attack rolls to attack with thrown weapons have a -4. Defense worsens by 2.

\textbf{20}: you are blind. You do not see. all enemies are Invisible.

\textbf{Kleptomania}\index{Kleptomania}

\textbf{5}: You feel the irresistible urge to grab \emph{interesting} objects from time to time. If you haven't stolen at least one item in a day, you won't be able to use Fate Points the following day.

\textbf{Code of Ethics/Vote}\index{Code of Ethics}\index{Vote}

You have made a vow, a promise, an oath that conditions your action.

5-10: establish the rules well, in black and white, and be clear with the Narrator.

\textbf{Compulsive}\index{Compulsive}

There are certain behaviors, necessary for you, which you absolutely cannot do without (e.g. walking avoiding stains on the ground or passing only over them, removing the weapon only in a certain way, etc.).
These behaviors must be declared and explained when choosing the disadvantage.

\textbf{5-10}: when you are prey to compulsive behavior you have a -2 on Awareness checks / you are always the last to act regardless of the initiative rolled or the marching order.

\textbf{Color blindness}\index{Color blindness}

You are color blind, a sunset will be something sad seen in grey.

\textbf{5}: you have no awareness of colors (achromatopsia). See everything in grayscale.

\textbf{Deformity}\index{Deformity}

Not everyone is born beautiful or straight. There are also those who are born crooked and ugly.

\textbf{5}: Minor malformation, affects your choice of Strength or Dexterity or Constitution. Subtract 1 point from this stat.

\textbf{10}: Two characteristics of your choice cannot exceed 2 points except magically. You have half movement.

\textbf{20}: Serious malformation. Three characteristics of your choice cannot exceed 1 point except magically. You have half movement.

\textbf{Depression}\index{Depression}

Every day is a bad day and nothing will make it better.

\textbf{8}: You love the Blues but unfortunately you have lost the joy of living, the enthusiasm, the hope.

Nothing seems to matter, you just drag yourself wearily from one day to the next. -2 on each Basic Proficiency check.

\textbf{Addiction}\index{Addiction}

\textbf{10}: You have an addiction, be it alcohol, drugs, cheese... If you don't consume a suitable dose every day (the Narrator will be able to tell you how much is enough) you get a -2 to all saving throws. After 3 days of abstinence you also become Depressed

\textbf{Dyslexia}\index{Dyslexia}

jk j0j zo mdbbdfd

\textbf{10}: You are unable to read and write. You are unable to correctly associate sounds with letters and shapes with sounds.

\textbf{Compulsive Dishonesty}\index{Compulsive Dishonesty}

Lie, he's stronger than you.

\textbf{5}: The character is driven by his own insecurity to always lie. Whenever the character is forced to admit his responsibilities or in any case to speak against his own interests, or in any situation in which he feels \emph{examined}, he will make up rather imaginative stories even putting friends and relatives in danger.

\textbf{Chronic Pain}\index{Chronic Pain}

oh how bad. Enchanter, are you using a cure on me today too?

\textbf{10}: You do not recover Hit Points except magically.

\textbf{Haemophilia}\index{Haemophilia}

you tend to bleed all the time, even at the most inopportune moments.

\textbf{8}: PATCH!!! (each attack you suffer automatically accumulates Bleeding +1)

\textbf{Epilepsy}\index{Epilepsy}

always and only at the most inopportune moments.

\textbf{15}: Whenever you fail a saving throw or attack roll, you fall to the ground for 1d6 rounds convulsing, the attack roll or saving throw is considered to have failed. You are considered helpless while convulsing.

\textbf{Fetishism}\index{Fetishism}

If you don't smell a woman's foot you become depressed.

€ 10182 € {5}: The character is irresistibly attracted to an object, body, category ... Every day that he is away from his source of pleasure, he considers himself to have fallen into Depression.

\textbf{Memories}\index{Memories}

Hey are you there? why did you become paralyzed? And when did you learn these things?

\textbf{5}: on each proficiency check roll a d4. On a 1-2 you make the normal check, on a 3 you make the check with a -2, on a 4 you make the check with a +2.

\textbf{Phobias}\index{Phobias}

\textbf{Various, 2-10}: The character is terrified by an object, by a category of people or living beings, by a situation. In the presence of the triggering cause, the character falls prey to a panic attack: his only desire is to escape as far as possible from the source of his terror, by any means; anyone who blocks his path is to be considered an enemy. If the character finds himself unable to escape, he falls into a catatonic state until the trigger is eliminated. See table of possible phobias at the bottom.

\textbf{Photosensitivity}\index{Photosensitivity}

The light, even if it's light, bothers you.

\textbf{5}: The character has a -1 on every Basic Proficiency check in which the brightness is at least daylight.

\textbf{10}: The character has a -2 on all basic proficiencies and attack rolls where the brightness is at least that of a lantern or light spell.

\textbf{20}: The character has a -3 on all basic skills and attack rolls where the brightness is at least that of a torch. The character is so sensitive that it is impossible for him to move freely in directly or less lit places, he will prefer to move and travel at night.

\textbf{Dormouse}\index{Dormouse}

you like to sleep and a lot. Snore.

\textbf{5}: +2 for every 2 hours over 8, otherwise you are tired..

\textbf{Awkwardness}\index{Awkwardness}

\textbf{10}: Your Dexterity score cannot exceed 2. You have a -2 to all checks that require Dexterity (deactivate devices, empty pockets, climb, initiative....).

\textbf{Hygienist}\index{Hygienist}

I ran out of soap. I'M OUT OF SOAP! .. I don't touch that sword, even if it shines with holy light and flies in mid-air until it is disinfected!

\textbf{5}: You have the urge to constantly clean yourself and clean everything you touch.

\textbf{Unconsciousness}\index{Unconsciousness}

\textbf{5}: If you have to do something the most direct and immediate plan is the best choice. You can't think of plans that last more than a minute. You get a +1 to Initiative and a -1 to Attack Rolls.

\textbf{Undecided}\index{Undecided}

Let's not do it, let's wait until tomorrow...maybe it's better!

\textbf{10}: You never act first. -1d6 on initiative checks.

\textbf{Recurring Nightmares}\index{Recurring Nightmares}

\textbf{10}: The character cannot sleep well. Every night he rolls a d4. With 1 the character sleeps normally, 2 or 3 the character sleeps restlessly and wakes up tired, with 4 you wake up in the middle of the night screaming, in the morning you are tired.

\textbf{Open Book}\index{Open Book}

yes, I know, I can keep quiet, you've already understood everything.

\textbf{5}: it's not that you can't lie, it's that you have a -1d6 on Deceive checks.

\textbf{Migraine}\index{Migraine}

It's never a good day. You suffer from constant and ferocious headaches.

\textbf{15}: The character suffers from violent headaches. Every day the character rolls a d4: on a 1 the character suffers no effects, on a 2 or 3 he suffers a -1 penalty on all checks, on a 4 the penalty becomes -2.

\textbf{Cursed}\index{Cursed}

You are Cursed. A dark fate has stained your soul.

\textbf{5-10}: You carry a curse. Discuss this with the Narrator.

\textbf{Myopia}\index{Myopia}

He hopes to find some glasses.

\textbf{5}: You don't see much. You have a -2 on attack rolls with ranged weapons and Awareness checks beyond 40 feet.

\textbf{15}: You see very little. You have -1d6 to attack rolls with ranged weapons and Awareness checks beyond 30 feet. You have -2 in melee combat.

\textbf{Mute}\index{Mute}

You can't talk and what's worse you can't even slander the guy who's stepping on your toe.

\textbf{10}: You are unable to make sounds. You don't speak or rather no one hears you. You take a -1d6 on word-based checks.

\textbf{Dyscalculia}\index{Dyscalculia}

1+1= ?

\textbf{10}: the character has a disorder that prevents him from mastering the concept of numbering. Not only is he unable to perform the simplest operations, he is also unable to understand the concepts of greater/lesser, or quantitative information of any kind.
Be careful what change they give you...

\textbf{Obesity}\index{Obesity}

You're definitely out of shape, and by a lot.

\textbf{10}: Dexterity cannot be above 2. You have a -2 on Dexterity checks and reflex saving throws. You gain a +2 on Fortitude saving throws.

\textbf{Defective smell/taste}\index{Defective smell/taste}

Nose, palate, burnt tongue, abuse of chilli pepper or wasabi... there can be many causes.

\textbf{5}: -2 two on tests that use taste or smell. You don't feel any flavors or smells unless they are extreme.

\textbf{Compulsive Honesty}\index{Compulsive Honesty}

\textbf{10}: You don't know how to lie, the very idea of ​​telling a lie makes you nervous.

\textbf{Crystal Bones}\index{Crystal Bones}

It would be called osteogenesis imperfecta but for you it's just constant pain.

\textbf{5}: The character has fragile bones. Each damage caused by a blunt weapon causes 2 additional hit points of damage.

\textbf{10}: The character has very fragile bones. Each damage caused by a blunt weapon causes 5 additional hit points of damage.

\textbf{Monco}\index{Monco}

You're maimed, it's up to you which hand to choose.

\textbf{7}: You are missing your off hand.

\textbf{13}: You are missing your primary hand. -2 to all rolls involving the use of the hand.

\textbf{Paranoioso}\index{Paranoioso}

You're paranoid and boring.

\textbf{5}: You always behave furtively, even without there being any actual need, thus arousing suspicion in the people around you.

Any Awareness check used to understand your opponent's actions and thoughts has an additional -5 difficulty, and a botch convinces you that the target has something vital to hide.

\textbf{Sensitive Skin}\index{Sensitive Skin}

You don't love the Sun, or at least your skin doesn't love it.

\textbf{5}: Your character burns easily, prolonged exposure without adequate protection leads to painful and unsightly burns and discomfort. Each Fire or Light damage deals 2 added damage.

\textbf{10}: You are extremely sensitive to ultraviolet. Each Fire or Light damage deals 5 added damage.

\textbf{Lazy}\index{Lazy}

you are slow and listless.

\textbf{5}: -2 to initiative.

\textbf{Noisy}\index{Noisy}

You don't do it on purpose, but there is always some noise around you. A rattling sword, a yawn, a burp, a noisy shoe...

\textbf{5}: You have a -2 on Stealth checks.

\textbf{Weak Blood}\index{Weak Blood}

\textbf{10}: The character's immune system is definitely pitiful. -2 to Fortitude saving throws.

\textbf{Carelessness}\index{Carelessness}

Oops...I didn't realize that!

\textbf{10}: You tend not to pay attention to what is going on around you, unless you have very good reasons to be alert, or are actively looking for something. You take -1d6 to Awareness.

\textbf{Schizophrenia}\index{Schizophrenia}

It wasn't me, but the other one!

\textbf{4}: You have more personality, or perhaps the other person thinks so.

The character has at least one second personality (max 6).
Each additional Personality to manage, beyond the first, grants a +1 to the cost.
So having 3 personalities brings the disadvantage to 6 points.

Every day 1d6 is rolled, with 1-2 it is personality 1, with 3-4 it is personality 2, with 5-6 it is the third personality that comes to light.

\textbf{Unlucky}\index{Unlucky}

things don't just happen, you also have to know how to look for them.

\textbf{5}: You ignore the first critical you score (TC or TS) of the day.

\textbf{7}: You ignore the first three criticals you make (TC or TS) in the day.

\textbf{Manic Depressive Syndrome}\index{Manic Depressive Syndrome}\index{Depression}

Today is Friday !!! It's Friday!!!

\textbf{7}: The character alternates states of euphoria with moments of dark desperation. 1d4 is rolled each day. On a 1 the character has a \emph{normal} mood. On a 2 or 3 he is considered to be in Depression, on a 4 he is in a state of joyful exaltation (see Unconsciousness) and bravado.

\textbf{Awe}\index{Awe}

I apologize.

\textbf{10}: The character is very insecure and tends to blindly trust others, especially if they are charismatic. You get a -2 on Intimidate and Perform checks
You get a -2 on enchantment saving throws.

\textbf{Light Sleep}\index{Light Sleep}\index{fatigued}
.
Every noise disturbs you, you can never sleep well

\textbf{5}: If you sleep in an area with natural/human noises (forest/city) you cannot rest well. In the morning you are tired. You can avoid this problem by using earplugs, which impose a -1d6 on Awareness checks against hearing to wake you up.

\textbf{Deafness}\index{Deafness}

Silence has a sound of its own says those who hear us, for you it is just a heartbreaking silent scream.

\textbf{10}: You can't hear us. You can't make Awareness checks that require hearing. You can't listen to people talking. But you can read lips if you know how.

\textbf{Dizziness}\index{Dizziness}

The discomforts appear when the character is aware of his height. Just for walking in an elevated position he has no penalty

\textbf{5}: At heights above 20 meters you tend to freeze. You get a -2 on all basic proficiency checks, attack rolls, and saving throws.

\textbf{7}: At heights above 10 meters you tend to freeze. You get a -4 on all basic proficiency checks, attack rolls, and saving throws.

\textbf{10}: At heights above 6 meters you tend to get stuck. You take a -1d6 on all basic proficiency checks, attack rolls, and saving throws.

\textbf{Reduced night vision}\index{Reduced night vision}

Your eyes don't work well with reduced brightness.

\textbf{5}: When the brightness is dim the character has an additional -2 to attack rolls.

\textbf{Shyness}\index{Shyness}

\textbf{5}: You are shy and reserved.

You have a -1 to Charisma-based checks

\textbf{Lame}\index{Lame}

you're limping.

\textbf{5}: your movement per Action is reduced by 2 meters (from 9 to 7, from 6 to 4).

\textbf{7}: your movement per Action is halved (from 9 to 4, from 6 to 3).

\textbf{10}: You are significantly crippled. -2 on checks requiring Dexterity, your movement is halved.

\end{multicols}

\bigskip

\textbf{Phobia Table (5-15 points)}\index{Phobias}\index[Tables]{Phobia Table}

\begin{tabular}{ll}
\textbf{Phobia Name} & \textbf{Description}\\
\toprule
Blennophobia & Fear of Slimy Things\\
Keraunophobia & Fear of Thunder\\
Hypochondria & Fear of Disease\\
Claustrophobia & Fear of Closed Places\\
Coimetrophobia & Fear of the Cemetery\\
Hedonophobia & Fear of Being able to Experience Physical Pleasure\\
Eisoptrophobia & Fear of Mirrors\\
Glossophobia & Fear of Public Speaking\\
Monophobia & Fear of Being Alone\\
Necrophobia & Fear of Dead Bodies\\
Nyctophobia & Fear of the Dark\\
Acrophobia & Fear of Heights\\
Agoraphobia & Fear of Open Spaces\\
Rupophobia & Fear of Dirty and Unhygienic. Feel the need to clean\\
Haphephobia & Fear of Contact and Being Touched\\
Asymmetrophobia & Fear of Non-Symmetrical Things\\
Gymnophobia & Fear of Nudity\\
Hemophobic & Fear of Blood\\
Traumatophobia & Fear of Injury\\
Sciophobia & Fear of Shadows\\
\end{tabular}


\pagebreak

\section{Optional - Iconic Skills}\index{Optional - Iconic Skills}\hypertarget{Iconic Skills}{}\label{abilitaiconiche}


\begin{changemargin}{0.3cm}{0.3cm}\begin{enfasi}{
Life has become immeasurably better since I was forced to stop taking it seriously. (Daniel Day Lewis)
}\end{enfasi}\end{changemargin}\medskip


\begin{multicols}{2}

%{\small

These skills represent the pinnacle of a character, not intended as the final skills of the 20th level, but as skills linked to the way of roleplaying, to the type of character that has been created and grown. These skills should only be given to characters who have been raised from first level to at least 15th level, it is a recognition of the player.

They are optional skills because they are strong, peculiar, unique and \emph{broken}. The Narrator should give them at the end of a long and unique campaign when the characters are now legends. Each character can have only one iconic ability, an ability that distinguishes heroes, capable of actions at the limit and beyond the human. Players are encouraged to create new Iconic Abilities based on character development.\\

{\small

{\large \textbf{A Light against the darkness}}\index{A Light against the darkness}

\textbf{Suggested requirements: Patron Ljust, Sumkjr}

Once a day you emit sacred light around you for 60 minutes which has the effects of the Protection from Good and Evil spell against Devotees and Followers not of your Patron. You can channel the light once per day and all creatures who are Followers or Devotees of other Patrons within a 10 meter radius of you must make a Fortitude save at DC 10 + sum of Traits in common with the Patron + Wisdom or be stunned for 2d6 rounds. \\

{\large \textbf{The Blacksmith}}\index{The Blacksmith}

\textbf{Suggested requirements: metalworking skills}

Your skills in working with weapons and armor are legendary.
Any armor you make encumbers and weighs one category lower, weapons deal damage one die category higher.\\

{\large \textbf{The Oracle of War}}\index{The Oracle of War}

\textbf{Suggested requirements: master melee fighter}

Every weapon in your hands is lethal. The weapon die doubles as the damage dealt by Strength doubles. E.g. a long sword does 2d8 damage and if you have Strength +3 the total damage becomes 2d8+6\\

{\large \textbf{The Hero without fear}}\index{The Hero without fear}

\textbf{Suggested requirements: courageous and resolute}

Once per opponent you can ignore (for 1d4 rounds) the conditions that afflict you as a Reaction.\\

{\large \textbf{Mindmaster}}\index{Mindmaster}

\textbf{Suggested requirements: an adventurous life managed with intelligence and coolness}

You can use a mental Ability score (Intelligence, Wisdom, or Charisma) in place of a physical one (Strength, Dexterity, Constitution) on all checks.\\

{\large \textbf{On a pale horse}}\index{On a pale horse}

\textbf{Suggested requirements: not fearing death, having killed many opponents}

You are the closest thing to death your enemies will ever see.
When you kill an enemy, all opponents (who may have seen the scene) within a 10m radius must make a Will save DC 10 + Weapon Proficiency + Charisma, costing a reaction, or be affected as by the Fear spell. The ability can be used 3 times per day.\\

{\large \textbf{The Magical Fury}}\index{The Magical Fury}

\textbf{Suggested requirements: a life dedicated to explosive magic}

You are capable of raising hell with magic. The difficulty (DC) of each of your spells increases by 2, when you make a Magic Test you roll 3d6 more and ignore 2 die rolls.\\

{\large \textbf{The Shadow}}\index{The Shadow}

\textbf{Suggested requirements: a life dedicated to hiding and surprising enemies}

Three times per day you can teleport to the shadow of another creature within 100 feet. Cost 1 Reaction.\\

{\large \textbf{The Mother}}\index{The Mother}

\textbf{Suggested requirements: spent more time in animal form than in one's own form}

It has the innate ability to leave the tracks of any animal, compatible with your size, even if you are not transformed. You can speak to any animal as if you were always under the effect of the Speak with Animals spell.\\

{\large \textbf{The Dead Man}}\index{The Dead Man}

\textbf{Suggested requirements: a life always on the brink of death}

Three times per day when your Hit Points drop below 1, you regain 3d12 Hit Points with a Reaction Action. This ability can also be used when your Hit Points are negative or you should be directly dead.\\

{\large \textbf{The Hunter}}\index{The Hunter}

\textbf{Suggested requirements: a life dedicated to hunting and tracking}

Your Survival checks have a +2d6 bonus. The first hit against an opponent automatically scores 2 critical points.
}

\end{multicols}



\pagebreak

\section{Cosmology}\index{Cosmology}\hypertarget{cosmology}{}\label{cosmologia}

\begin{changemargin}{0.3cm}{0.3cm}\begin{enfasi}{
It is easier to dominate over those who believe in nothing (The Neverending Story, Kmorf)

\medskip

Do you believe there is only one God? You are right; even the demons believe it and tremble! (James the Just 2, 19. Editor's note Referring to his own Patron)}\end{enfasi}\end{changemargin}\medskip

\begin{changemargin}{0.3cm}{0.3cm}\begin{narratore}
In OBSS the deities are different from traditional RPG gods.

The deities, the Patrons, love to get their hands dirty, participate in the affairs of the creatures who adore them, for them it is a continuous challenge to have more believers, followers and people more similar, in terms of traits, to them.

The Patrons were created as \href{https://www.treccani.it/vocabolario/parossismo/}{paroxysm} of the human soul, where everything is an excess. As spirits freed from Pandora's box they have the sole purpose of bringing their Traits to dominion making them the most common and present among creatures, especially among the most powerful.
\end{narratore}\end{changemargin}

\begin{multicols}{2}

\bigskip

\lettrine[lines=2, lhang=0.33, loversize=0.25, findent=1.5em]{I}{n} principle was the nothingness which in itself contained everything.

The Energy deriving from the most primordial impulses exploded in all its power without any control.

Love, hate, fear, pain, joy, serenity... everything was tangled in a thick and infinite skein whose core was forming.

These energies, emotions and impulses have begun to create three entities: Atmos, the one who is responsible for controlling the flow of time and space, the one who assists and documents; Ljust the positive energy, heat, light, life and syntropy; Calicante, the negative energy, cold hatred, destruction, death and entropy.

While Atmos does not have a definable form, Ljust and Calicante manifested themselves as two tongues of flame of a single progenitor energy.

\textbf{Ljust} \index{Ljust}is the representation of what light and life always carry with them. It represents the purity of the feeling of love, the protection of life, respect for others, curiosity for the new, the desire to always improve oneself, the strength to fight with courage and value for the common good. It is the vital push for change, the chaos that evolves but does not destroy.

\textbf{Calicante}\index{Calicante} is the representation of darkness, hatred, anger and violence. Calicante is revenge and cold destruction, there is no interest in any form of life rather he uses them, exploits them and only in certain cases suffers their presence. He sadistically loves suffering. It is entropy that annihilates and annihilates and finds pleasure in doing so.

\textbf{Atmos} \index{Atmos}is the witness, the one who marks the passage of time and transcribes every event of Yeru and the Patrons of Genesis. An entity born from creation to prevent absolute destruction. He watches over and transcribes what the Patrons of Genesis, the divinities who generated creation, do.

Together the two Patrons of Genesis gave life to everything we know. Calicante created Tiya\index{Tiya} and Ljust created Curyan\index{Curyan}, the two kingdoms that make up our world, Yeru.

They played with forms and energies creating two mirror-image kingdoms but contrasting and distinct. Tiya and Curyan, like Calicante and Ljust, are part of a whole but, exactly like the Patrons of Genesis, they are also profoundly different and magically divided. In fact, there is both a physical barrier, formed by deadly perennial sea storms, and a magical one, which delimit its borders and keep them clearly divided.

But just like their two creators who cannot be completely divided and distant, cannot exist, so Tiya and Curyan are divided but also in contact with each other through the Portals. Portals that generate autonomously, without any control and prediction, due to the magical energy that presses, pushes and feeds itself in the \emph{non-place} on the border of the two kingdoms and which is generated by the continuous emotions of the Patrons of Genesis.

It is these magical routes that allow you to move between Tiya and Curyan and travel in \emph{non-place} or what is outside of Yeru.

Ljust and Calicante decided, strangely by mutual agreement, to generate a Patron who would oversee these fractures, who was capable of sensing, opening and blocking these Portals. Thus, \textbf{Lynx}, the Guardian of Portals, was created.

Many try to move from Tiya to Curyan to seek peace and serenity. Others try to cross the reverse border in search of adventure and power, some try through normal routes, others through the Portals, many are lost forever in the \emph{non-place}.

Lynx \index{Lynx}supervises the cosmic void, access to the Planes, to the portals which, with the alternation of chaos and order, of good and evil, of light and darkness, are increasingly creating fractures on the border existing between the two reigns. Lynx perceives them, hears them, knows where they are being generated or shut down, with the passage of time, in fact, some of these Portals have become stable and definitive, while others continue to be generated randomly and always in a totally unknown way remain active or run out. Continuously traveling in the non-place Lynx closes the largest portals but for one who closes another one opens. Lynx stripped the Magic Lists of many of the spells that affect the planes, to protect Yeru from outside creatures.

Precisely while carrying out this important role, Lynx clashed with a strange, reptilian, gigantic, winged, powerful, strong, intelligent and magical creature. A red Dragon,\index{Tàhil} named Tàhil.
The latter moved in \emph{non-place} with maximum freedom, without any difficulty and approached Lynx. The Atmos diaries tell of how Lynx tried to stop him and speak to us, of how he was ferociously attacked, of the screams of the Guardian Patron that were heard echoing in both kingdoms, of the sound almost similar to a guttural roar that pierced the silence in the kingdoms of Tiya and Curyan. Of the intervention of Ljust and Calicante. The first to save Lynx and the second to discover and learn about this fascinating new \emph{weapon}.

Lynx was saved. Ljust infused him with her healing magic and helped him regenerate. However, he left himself scarred in memory of the meeting.

Tàhil led many Dragons on Yeru and these, moved by their thirst for destruction and power, also spread through the Portals present on both Tiya and Curyan. For this reason the dragons are called \emph{Dragons of Tàhil}.

Hordes of dragons of all colors have darkened the skies for decades. No nation was saved. Looting, raids and violence were perpetrated indifferently in the two kingdoms. They were highly intelligent, cunning and violent, powerful beyond belief, and extremely evil. They had an extraordinary robustness. But above all, they did not fear the Patrons. They did not submit to them.

Atmos, concerned for the balance of creation, channeled the primordial and divine energies of the Patrons of Genesis, creating divinities that could rival the dragons and could defend Yeru.

The first created by Atmos, with the help of Ljust, and the intervention of Calicante, was \textbf{Gradh}\index{Gradh}, Patron of Humanity (and of all sentient races), the one who would defended creation from dragons and other Patrons. Gradh embodies the dualism of the two Patrons of Genesis, Ljust's innate instinct for protection, defense and care and Calicante's instinct for revenge, violence and fury.

He courageously throws himself into battles, attacks the enemy without fear, protects the weakest, defends life but is not afraid to follow the path of the most destructive revenge towards those who exploit and destroy lives for no reason. Gradh loves to immerse himself among people and live with them, like them. He does not feel totally at ease in the pantheon with the other Patrons nor among common people, he is Human among Patrons and Patron among Humans. Passionate and kind, he is the Patron who cares most about the fate of Yeru and his creatures.

Gradh immediately perceived that the Dragons represented an element of further chaos, further suffering and war. As Patron of Yeru and his creatures he felt the Dragons as alien creatures, not original, not part of the Genesis plan.

Distrustful by nature, Gradh decided to propose to the Patrons of Genesis to make a pact with the Dragons.

Here, just over 300 years ago, on the 15th Prineva of 65 of the sixth cycle, on the unreachable island of Alantia that divides Tiya and Curyan, on one side there were the flames of Ljust and Calicante, Lynx and Gradh while Tàhil, the red dragon evil and immortal and Elysan\index{Elysan}, the wise and good silver dragon on the other. Atmos, everywhere, kept track of events.

Gradh tried to force the expulsion of the Dragons and Lynx the permanent closure of the Portals. Ljust tried to mediate by understanding that not all Dragons were evil and that they could bring knowledge and a new evolutionary boost to Yeru.

Calicante pretended to agree with Ljust with the sole purpose of bringing more chaos and destruction through the Dragons.

Having understood that the outcome of the meeting had already been decided, Gradh and Lynx abandoned the Plain of Solitude, leaving the Dragons and the Patrons of Genesis to formalize the division of Yeru. It had been a resounding defeat for both of them, Gradh was from then on even more wary, if not prejudiced, towards all dragons.

Tàhil became the First General of Calicante and the Dark Hatred created for him a secret and unattainable kingdom, a land for him and the Dragons where they could rest and then bring destruction on Yeru. Elysan swore loyalty and trust to Ljust, and together they promised to rule Curyan to the best of their ability.

Lynx, now an external spectator, did not remain without doing anything, fearing the worst he created a Portal that led to a new Yeru, a different planet outside the influence of Dragons and Patrons. His research takes him to an almost idyllic world full of nature, animal life and without Patrons, as he liked. He named it Ker €10393 € {Ker} in memory of an old beloved. There is very little information about this world, only a few high Lynx Devotees know about it and even fewer have visited it.

The fact remains that the junction portal exists and is stable, where exactly it is is not known but he has already given the gift to Yeru, the Gnomes €10394 € {Gnomes}.

The tongues of divine energies were too intense, chaotic, and pure for Atmos to govern to shape further Patrons alone. Using the raw power of the Patrons of Genesis he created other Patrons, each influenced differently by Calicante or Ljust. These Patrons turned out to be less perfect and divine than his intentions, more imperfect and {human} as they originated from the uncontrollable and pure emotions of the Patrons of Genesis. These new Patrons shape wills, found kingdoms, command from the shadows the pawns who dare to ask for their favors.

In this apparent calm the Patrons perpetuate their interests, to become the strongest, the most important, those who have the greatest followers. The aim is only one: to have as many people as possible who follow their Traits.

If a Patron acts personally or indiscriminately he knows that he will trigger the reaction of Gradh or the intervention of Atmos which will prevent him from uncontrolled and massive use of his powers directly on Yeru. This rarely stops them and nature itself and all creatures are often influenced by the will of the Patrons.

In Tiya, but sometimes also in Curyan, aberrations arise more and more often, ever new diseases, cursed lands where nothing can grow, not to mention madness that often involves those who should instead protect the citizens. It's a hard life for the common man who continually has to face drought or floods, animal deaths and irregular if not absurd weather, hordes of creatures who come from nowhere who just want to exterminate everyone. At every step he must look around because he can never know who has sold his soul to a Patron to live one more day.

In Curyan we see the harmony and almost perfect coexistence between nature and different races develop. There is pain, there is illness and death but all as a natural cycle of life as part of the same that is protected, guided, helped. A rich and generous land and for this reason an increasingly victim of the machinations of the Patrons if they follow Calicante.

Everywhere the strongest enemies are the Dragons who make incursions to bring destruction and death and sow fear and horror.

It's not always all idyllic, vast regions of Curyan are becoming incubators of dark and evil races, legions of the undead led by powerful necromancers are massing on the borders, Dragons are training their corrupt followers and dark spiers in the sky promise storm.

\subsection{Patrons}\index{Patrons}\hypertarget{patrons}{}\label{patroni}

\begin{changemargin}{0.3cm}{0.3cm}\begin{enfasi}{
Conan: Which gods do you pray to?

Subotai: I pray from the rooftops and you?

Conan: I pray to Chrom, but only rarely... he doesn't listen. (Conan the Barbarian, 1982 film)

\medskip

For just as the body without the spirit is dead, so faith without works is also dead. (James the Just 2, 26. Editor's note: Referring to the scores of the Traits connected to the Patron...)
}
\end{enfasi}\end{changemargin}\medskip


All creatures, even those who do not use magic, can feel the influence of these Powers, of these Patrons.

If a character, due to his way of being (playing) and behaving, has at least one Trait in common with a Patron and indeed matures and strengthens these beliefs, even if he has not sworn loyalty to a Patron he could still feel the influence of the Patron and receive of gifts from it.

A Patron is happy if someone follows his dictates, Tratti, and gives small powers to those who do so as recognition for the loyalty reserved for him, intentionally or otherwise. The powers indicated under \emph{Common Traits} are cumulative. Unless otherwise indicated, powers can be used 1 time per day and cost 2 Actions.
When a spell is indicated, it is manifested without magic checks or armor penalties.

Every \textbf{Patron prefers one or more energy forms}, if you are a Follower you can use that energy in your magic, if you are a Devotee you must, in the same way the preferred Magic Lists are indicated, i.e. lists in which the Devotee It has some advantages of use.

The forms of Energy are distinguished between positive, neutral and negative sources, they also serve you to better understand your Master, sorry, the Patron you serve.

Do the sum of the elements, if positive the Patron can be considered good, if with a zero value the Patron is neutral, if with a negative value the Patron is evil, for convenience in the list of energy forms it is indicated if the Patron is \textbf {B}good, \textbf{N}neutral or \textbf{M}bad.

In the description of the Patron you will also find his manifestation, that is, what happens when a character acts in a way that is particularly and significantly suited to the Traits followed by the Patron. The effect is purely scenic and circumstantial but always leaves an impression on anyone who observes it, usually guaranteeing an advancement point in some Trait connected to the Patron.

There is also an indication of the patron's favorite weapon. There are no advantages in using it, unless you have the specific \hyperlink{thepatronandmyweapon}{Feat} (p. \pageref{ilpatronoelamiaarma}), the choice is purely personal and left to the devotion of the character.

Below the indication of the preferred weapon there is an indication of the Rule \index{Rule} or the behavior that the Devotee must try to respect.

A spellcaster who relies on a Patron, at least 3 Traits in common, becomes a Devotee. If he has at least 2 Traits in common and relies on a Patron then he is said to be a Follower. The \textbf{Advantage} indicated is for the Devotee only.

He may not even follow any Patron despite having multiple Traits in common.

\begin{changemargin}{0.3cm}{0.3cm}\begin{narratore}
The Storyteller can still grant Follower or Devotee even if the Traits don't match perfectly. At the player's request and at his discretion, he can evaluate the similarity of some character traits to those of the Patron and evaluate them as suitable for being a Follower or Devotee. In these situations it is necessary to understand how the player frames the character and understand not only if the Traits but also the feelings of the character are similar to the chosen Patron.
\end{narratore}\end{changemargin}

The abilities you gain related to Traits in common are independent of being a Devotee, Follower, or simply an atheist. \index{Advantages}

\bigskip

\textbf{Table of Elements - Energy}\index[Tables]{Table of Elements}

\medskip

\begin{tabular}{llll}
\textbf{Positive} (+1) & \textbf{Neutral} (0) & \textbf{Negative} (-1)\\
\toprule
En. Positive & Fire & En. Negative\\
Light & Cold & Emptiness\\
& Sound & \\
& Electricity & \\
\end{tabular}

\begin{changemargin}{0.3cm}{0.3cm}\begin{tcolorbox}[title = Devotees and Followers]

Being Devotees or Followers is your choice, no one forces it on you. You must feel it as an opportunity for role playing, as an enrichment of the character and not a constraint. Being Devotees or Followers does not mean being inclined to the will of the Patron, on the contrary, it means being even more convinced of one's own Traits, of one's own personality. \textbf{A Patron does not ask for prayers, but asks to be yourself}.

\end{tcolorbox}\end{changemargin}

\subsubsection{Miracles, Interventions and Wonders}\index{Miracles, Interventions and Wonders}

In a world where deities are so capricious, fickle and thirsty for devotees, it makes sense for them to be generous to those who can then spread their Traits.

A favor asked of a Patron always has a price that is neither obvious nor obvious. The Storyteller must carefully evaluate the character's plea and judge whether the request is relevant to the Patron's Traits. If so, roll 1d100 and score less than half the Trait points in common with the Patron. Or decide independently according to the course of the adventure.

\subsubsection{Ljust}\label{ljust}

\index{Ljust}

The Lady of Light, the one who radiates warmth and love. Generator of the impulses of love, protection, kindness, joy and forgiveness. She embodies the protective aspect of a mother, the strength and audacity of a fighter, the passion of a young lover and the cheerfulness, the search for the new, the imagination of a little girl. Ljust embodies the beauty of life and every creature that contemplates her sees what for her is the maximum harmony of her and falls prone to her charm.

Ljust can only be chosen by a character with 4 Traits in common with her, basically you are born to be a Devotee of Ljust. Over the course of the eras Ljust decided to select, choose and reward the women who most innately and profoundly showed love for life, curiosity for the new, unshakable strength, dedication, trust, respect and care for others by giving them the powers and opportunity to study and grow as Pupils of the Light. These Students must follow the 8 Steps rule.\\

\noindent- \textbf{Symbol}: A star surrounded by sun rays\\
- \textbf{Characteristic}(Devoted): Wisdom or Charisma\\
- \textbf{Traits}: Courageous, Generous, Empathetic, Protective, Instinctive, Nonconformist, Sensitive, Extroverted, Fair, Compassionate, Altruistic. The Devotee of Ljust has 4 Traits in common with the Patron.\\
- \textbf{Manifestation}: Golden light floods the caster.\\
- \textbf{Sum of Traits in common at 5 points}: you can cast the spell Light as a Reaction, 3 times per day\\
- \textbf{\textbf{Sum of Traits in common at 10 points}}: you gain a +2 on Fortitude saving throws\\
- \textbf{Sum of traits in common at 15 points}: armor of light protects you, you gain +2 to all saving throws and defense, the effect is permanent.\\
- \textbf{Sum of Traits in common at 20 points}: you can cast the Solar Flare spell. 1 time a day.\\
- \textbf{Elements/B} (Follower/Devotee): Positive Energy, Light\\
- \textbf{Advantage} (Devoted): Effective cures\\
- \textbf{Privileged Magic Lists}(Follower/Devotee): Cure, Abjuration\\
- \textbf{Favorite Weapon}: Bastard Sword\\
- \textbf{Rule}: Accept the invitation to a dance

\medskip

\textbf{The 8 Steps of the Students}\index{8 Steps of the Students}\index{Students}

The Pupils of the Light are a secret group of Devotees who, out of total affinity with Ljust, have undertaken the hard path of good and love. It is among the oldest groups founded in Yeru. The Students, 99 as a maximum number, but unfortunately often less numerous, are Devotees of Ljust and must follow the 8 Steps of the Light\\

\noindent 1. Love and protect those around you with all of yourself, with total and sincere dedication.\\
2. Don't let your inaction generate suffering.\\
3. Yes, a point of comparison. Let your Light elevate the people around you so that they can see hope, serenity, calm, protection and security in you.\\
4. Use intelligence, cunning and wit. You are far-sighted and resolute in action.\\
5. Your work is for the common good. May your Light always be high and intense.\\
6. Seek no other Light than yours and that of your sisters.\\
7. Be bright but don't blind those around you.\\
8. Be the difference between despair and hope.\\

\medskip

The students constructed a harmonious dance by transforming the steps of their Rule into dance.

There are also Pupils of other kinds, rare but historically proven.

\subsubsection{Calicante}\index{Calicante}\label{calicante}

\begin{changemargin}{0.3cm}{0.3cm}\begin{enfasi}{
Superstition is the religion of weak spirits. (Edmund Burke)
}\end{enfasi}\end{changemargin}

It's dark, cold and angry. It contains within itself hatred, violence, destruction, revenge and perennial dissatisfaction. It collects the capricious and discontented personality of a child, the violent and sadistic boredom of a young man, the destructive force of a hurricane and the anger of a fighter who has nothing left to lose. Calicante just by his presence makes you uncomfortable, makes you feel in danger, fascinates but with the weapons of fear and inconstancy.

Calicante can only be chosen by characters who have 4 Traits in common with him. His devotees are the best assassins, his closest profession. Those who show the greatest disregard for danger and the lives of others. His favorites are those who are feared, hated, those who are violent and cruel but deadly efficient and decisive in every combat situation.

\noindent- \textbf{Symbol}: A black whirlwind\\
- \textbf{Characteristic}: Strength or Dexterity\\
- \textbf{Traits}:Selfish, Vengeful, Superb, Wrathful, Passionate, Cynical, Competitive, Creative, Dishonorable, Anarchic, Brutal. The Devotee of Calicante has 4 Traits in common with the Patron.\\
- \textbf{Manifestation}: sword dripping with black blood\\
- \textbf{Sum of Traits in common at 5 points} points: You can cast the Darkness spell. Once a day\\
- \textbf{Sum of Traits in common at 10 points}: Your weapon is cloaked in shadow. You gain +2 to attack rolls and +1d4 void damage for 2d6 rounds, once per day.\\
- \textbf{Sum of Traits in common at 15 points}: Create 4 bolts of Void, each bolt does 2d6 damage, automatically hits within 18 meters. Once a day.\\
- \textbf{Sum of Traits in common at 20 points}: You create a zone of protective energy around you within a 3 meter radius, you halve all damage you receive, it is not possible to recover Hit Points in the area. Duration 10 consecutive minutes, once a day.\\
- \textbf{Elements/M}: Negative Energy, Void\\
- \textbf{Advantages}: Mental Shield\\
- \textbf{Privileged Magic Lists}: Fire, Necromancy\\
- \textbf{Favorite Weapon}: Machete\\
- \textbf{Rule}: Never leave a direct offense unpunished

\subsubsection{Atmos}\index{Atmos}\label{atmos}

\begin{changemargin}{0.3cm}{0.3cm}\begin{narratore}
So what is time? If no one asks me, I know; if I want to explain it to anyone who asks me, I don't know anymore. (Augustine of Hippo)
\end{narratore}\end{changemargin}

The keeper of Time and the Clock Tower, as he initiated time and the creation of new Patrons, will stop the challenge between them and the surviving Patrons will be judged, their works evaluated and Ljust or Calicante will benefit from it. Like a challenge from a single copper coin new Patrons, new ideals will be created and we, little creatures, will see new civilizations and flourishing kingdoms born. The story is little known, only the few Devotees of Atmos, scribes and scholars of the library of Time, know the secret and the flow of time and the race, the others, ignorant, will live their time with a master certainly guided by a Patron .

Atmos, the Patron of Time is the guardian of history, he is the one who keeps track of the thousand and more worlds that have been created.

Atmos has the unique power reserved only for him to be able to banish a Patron from creation if he becomes too strong and threatens Calicante and Ljust. Atmos has used this power before. Atmos, both for its totally neutral nature and for its role, has never taken sides.

All Patrons fear Atmos for his power, the most terrible for them, that is, their alienation, oblivion, forgetfulness, being distracted by time and challenge.

To be a Devotee of Atmos at the time of the ritual it is necessary that the future Devotee possess at least four Traits in common with him, love history and knowledge.

Dressed in a soft brown habit and leather shoes he moves among the infinite shelves of the Library of Knowledge with a strange time meter always hanging from his waist.

\noindent- \textbf{Symbol}: A white book with a pocket watch placed on top\\
- \textbf{Characteristic}: Intelligence or Wisdom\\
- \textbf{Traits}: \textbf{Atmos}: Observant, Detached, Prudent, Reflective, Upright, Anxious, Paranoid, Complaining, Distrustful, Foresighted, Apathetic. The Devotee of Atmos has 4 Traits in common with the Patron.\\
- \textbf{Manifestation}: the spell develops as if in slow motion, it is only an illusory effect\\
- \textbf{Sum of Traits in common at 5 points}: Always know the exact date and time.\\
- \textbf{Sum of Traits in common at 10 points}: You have an innate intuition for knowledge. You have +1d6 on Knowledge checks\\
- \textbf{Sum of Traits in common at 15 points}: You can cast the spell Globe of Invulnerability, 1 time per day.\\
- \textbf{Sum of Traits in common at 20 points}: Whenever you have to make an Arcana check you can take 18 as if you would take 10\\
- \textbf{Elements/N}: Sound, Cold\\
- \textbf{Advantage}: Sense of time\\
- \textbf{Privileged Magic Lists}: Divination, Abjuration\\
- \textbf{Favorite Weapon}: Light Mace\\
- \textbf{Rule}: Do not accept or give compensation if it is not deserved

\subsubsection{Lynx}\index{Lynx}\label{lynx}

\begin{changemargin}{0.3cm}{0.3cm}\begin{narratore}
People don't take trips, trips are made by people. (John Steinbeck)
\end{narratore}\end{changemargin}

Patron of Portals, he can only be chosen by characters who have at least 3 Traits in common. He is the first Patron generated by Ljust and Calicante, created to protect Yeru from external attacks.

Serious, with cold eyes of a very light blue, he is the Guardian of the Portals and of what is Beyond. Lethal guardian for those who try to pass them without permission, attentive guide for those who ask for his help and permission. He uses his scars as a shield to keep everyone away. He is the lone controller of the world.

His Devotees are the travelers par excellence, those who guard and protect Yeru from what is alien, from what could disturb creation.

\noindent- \textbf{Symbol}: A portal into the darkness\\
- \textbf{Characteristic}: Dexterity or Intelligence\\
- \textbf{Traits}: Lonely, Serious, Rigid, Controlled, Courageous, Insensitive, Stubborn, Determined, Intolerant, Introverted, Rational\\
- \textbf{Demonstration}: as if the panorama no longer had a horizon\\
- \textbf{Sum of Traits in common at 5 points} points: Once a day you can perform an additional movement Action\\
- \textbf{Sum of Traits in common at 10 points}: You can cast Dimension Door once per day\\
- \textbf{Sum of Traits in common at 15 points}: You can cast the Banishment spell, 1 time per day, DC 30.\\
- \textbf{Sum of Traits in common at 20 points}: You can teleport 500km per day (even more teleports or teleported as long as the total sum does not exceed 500km)\\
- \textbf{Elements/N}: Fire, Electricity\\
- \textbf{Advantage}: Slow and Still\\
- \textbf{Privileged Magic Lists}: Evocation, Water\\
- \textbf{Favorite Weapon}: Short Sword\\
- \textbf{Rule}: Don't leave an environment unexplored

\subsubsection{Gradh}\index{Gradh}\label{gradh}

\begin{changemargin}{0.3cm}{0.3cm}\begin{narratore}
The man who has ceased to be afraid has ceased to worry. (Francis Herbert Bradley)
\end{narratore}\end{changemargin}


The first Patron created by Atmos under the guidance of Ljust and the influence of Calicante.

Gradh embodies Ljust's innate instinct for protection, defense and care. Gradh is the most similar and deeply linked to Ljust that has been created. He is balance, rationality and empathy.
Where there is defense, care and protection there is Gradh.

But Calicante could not allow the creation of a Patron totally devoted to Ljust and so he infused Gradh with the coldness of revenge and the fury of anger. So Gradh, in the act of defending humanity, often first has to protect it from himself.

Gradh does not like to openly challenge Cattalm because he knows that he would play right into his hands, so he cunningly tries to lure him into his playground, where no life will be in danger and there he shows off his strategic and combat superiority.

Passionate and cold, he is perhaps the most human Patron of the current pantheon. His warm and charismatic gaze, which when he loves and protects is a reassuring chocolate colour, can become cold and sharp with the shades of frozen earth when he is prey to the fury of battle or revenge. Gradh loves to study the world around him and go unnoticed. He often hides among people and € 10496 € {lives} his human life, but doesn't really let anyone get close to him. Gradh draws towards him with the same ease with which he pushes away from himself.

The Devotee of Gradh is proud and proud, indomitable and protective, and saddened, because no matter how hard he tries to bring balance and peace, evil always continues to thrive.

\noindent- \textbf{Symbol}: A shield with two intertwined spirals engraved on it.\\
- \textbf{Feature}: Strength\\
- \textbf{Traits}: Indomitable, Protective, Vengeful, Courageous, Cold, Distrustful, Impetuous, Presumptuous, Dark, Reserved, Melancholic, Competitive \\
- \textbf{Manifestation}: two coils, one black like a shadow and one bright like a spark, surround your weapon, intertwining\\
- \textbf{Sum of Traits in common at 5 points} points: You can cast the Cure Serious Wounds spell, but it causes 1d6 damage. 1 time a day\\
- \textbf{Sum of Traits in common at 10 points}: For 10 consecutive minutes you have a bonus of +1d6 Saving Throw on Reflexes and Fortitude. Once a day.\\
- \textbf{Sum of Traits in common at 15 points}: You exude an aura that grants all your companions within a 3 meter radius a +2 Saving Throw. Once a day, for 30 consecutive minutes\\
- \textbf{Sum of Traits in common at 20 points}: You cast the Fireball spell. The spell causes 60 negative energy damage. DC 25 Reflexes to halve. 2 times a day\\
- \textbf{Elements/N}: Positive Energy - Negative Energy\\
- \textbf{Advantage}: Senses protected\\
- \textbf{Privileged Magic Lists}: Abjuration, Invocation\\
- \textbf{Favorite Weapon}: Heavy Mace\\
- \textbf{Rule}: Do not allow a fiend to walk on Yeru

\subsubsection{Atherim}\index{Atherim}\label{atherim}

\begin{changemargin}{0.3cm}{0.3cm}\begin{enfasi}{
The one to whom you confide your secret becomes the master of your freedom. (François de La Rochefoucauld)\\

There is nothing hidden that will not be revealed, nor secret that will not be known. (Luke, 12, 1-7)
}\end{enfasi}\end{changemargin}

The Guardian Patron. Many see Atherim's generous breasts as a sign of voluptuousness and passion. They are enchanted by her busty beauty and do not see the crystal eyes that instill fear in anyone who dares to even think of approaching her.

Atherim is the guardian of dreams and hopes, the one to whom you can entrust your desires, like a mother. He is the Patron Saint of Children, Secrets and Midwives.

With a cheerful smile and a good soul she will always be ready to help you make your dreams come true. And like a mother Atherim protects and guards secrets and passions. Atherim is silent. She is the one she keeps forever, within her soul the secrets of Yeru.

The Devotee of Atherim takes to heart those who have made promises, punishes those who break them and those who reveal secrets. Many Atherim Devotees are diplomats, notaries and midwives.\\

\noindent- \textbf{Symbol}: A gloved woman's hand holding an ampoule rich in flows\\
- \textbf{Feature}: Wisdom\\
- \textbf{Traits}: Cheerful, Calm, Industrious, Good, Silent, Clement, Patient, Shy, Emotional, Meek, Gullible\\
- \textbf{Event}: a serene and calming silence falls around the enchanter\\
- \textbf{Sum of Traits in common at 5 points}: You can add 1d6 to a saving throw after rolling it but before knowing whether it was successful or not. Once a day, as a Reaction.\\
- \textbf{Sum of Traits in common at 10 points}: Gain 30 temporary Hit Points. Duration 1 hour, once a day, as an immediate action.\\
- \textbf{Sum of Traits in common at 15 points}: You can cast the Zone of Truth spell 3 times per day, without a saving throw.\\
- \textbf{Sum of Traits in common at 20 points}: Each potion you drink has double the duration or effect if immediate.\\
- \textbf{Elements/B}: Positive Energy, Electricity\\
- \textbf{Advantage}: Metabolism control\\
- \textbf{Privileged Magic Lists}: Enchantment\\
- \textbf{Favorite Weapon}: Dagger\\
- \textbf{Rule}: Do not reveal a confided secret

\subsubsection{Belevon}\index{Belevon}\label{belevon}

\begin{changemargin}{0.3cm}{0.3cm}\begin{enfasi}{
Where there is a man, there is also a lie. (Robert Louis Stevenson)\\

No one has such a good memory that they can be a perfect liar. (Abraham Lincoln)
}\end{enfasi}\end{changemargin}

He is the Patron who best embodies lies and pretense for the sake of his own gain. He only loves himself. He is a narcissist who only surrounds himself with people who pander and flatter him. He abhors loneliness but at the same time he hates being touched by anyone.

He is always looking for new things, for wonderful objects that he exchanges and reciprocates with other objects. He likes to argue and haggle, argue and push the sale to the limit.

From the appearance of a young boy he perfectly embodies a dangerous scoundrel.

The Devotee of Belevon is well described by the rich and curious merchant who never misses an opportunity to deal in new goods. He is not driven by greed or accumulation but by the art of commerce and exchange.

\noindent- \textbf{Symbol}: A golden cage\\
- \textbf{Feature}: Intelligence\\
- \textbf{Traits}: Confused, Narcissistic, Chaste, Liar, Curious, Double Agent, Inconstant, Clumsy, Imprudent, Insolent, Envious\\
- \textbf{Manifestation}: as if the golden bars of a cage were intertwined around the caster\\
- \textbf{Sum of Traits in common at 5 points} points: You can cast the Prestidigitation spell, 3 times per day.\\
- \textbf{Sum of Traits in common at 10 points}: You gain the ability to cast the Greater Image spell once per day.\\
- \textbf{Sum of common traits at 15 points}: You can cast the spell Deadly Hallucination. 1 time a day\\
- \textbf{Sum of Traits in common at 20 points}: By touching an object you gain a general understanding of the history of whoever created it. Once a day. Costs 3 Actions.\\
- \textbf{Elements/N}: Fire, Sound\\
- \textbf{Advantage}: Lucky\\
- \textbf{Privileged Magic Lists}: Illusion\\
- \textbf{Favorite Weapon}: Light Pike\\
- \textbf{Rule}: Negotiate on the price whether buying or selling

\subsubsection{Cattalm}\index{Cattalm}\label{cattalm}

\begin{changemargin}{0.3cm}{0.3cm}\begin{enfasi}{
It's not being angry that matters, it's being angry about the right things. I told her: look at it from the Darwinian perspective. Anger is there to make you efficient. This is its survival function. That's why it was given to you. If it makes you ineffective, drop it like a hot potato. (Philip Roth)
}\end{enfasi}\end{changemargin}

Generated directly by Calicante, as a response to Ljust's creation of Gradh, it is pure destruction, chaos and entropy. Cattalm has the sole aim of destroying, bringing chaos and disease, earthquakes and floods.

Cattalm is among the few Patrons who dares to openly challenge Gradh and does so with joy because he knows that their battle will only bring further destruction. Cattalm accepts and invites every creature capable of hatred, capable of destroying and wounding, to be its Devotee. Many of his devotees are monstrous creatures or aberrations.

Cattalm, on the other hand, is among the most wonderful Patrons, with shiny white skin, soft feather wings and light silver armor. As much as his delicate features make him a beautiful being, as much as he aspires to destruction.

Cattalm loves chaos which manifests itself in the most violent ways with earthquakes, floods, tsunamis, diseases if not directly fiery rains. He almost never acts directly but lets chaos and destruction work for him.

Ljust could not help but intervene in the creation of such an explicitly evil Patron and, secretly from Calicante, instilled in Cattalm love and protection for children. Cattalm destroys, poisons, weakens but not the children, not even indirectly, rather he himself takes action to cancel the evils caused by his nature.

It has already happened that entire villages were flooded and all the little ones were found on wooden roofs like barges.

Whenever a calamity occurs it is customary to say that \emph{Cattalm stumped his foot}.\\

\noindent- \textbf{Symbol}: A giant wave towering over the coast\\
- \textbf{Feature}: Strength\\
- \textbf{Traits}: Destructive, Anarchic, Meticulous, Sadistic, Provocative, Brutal, Fatalist, Impassive, Belligerent, Calculating, Petty\\
- \textbf{Event}: the roar of thunder\\
- \textbf{Sum of Traits in common at 5 points} points: Through your weapons you weaken the designated opponent. After a critical hit you can increase your fatigue by one level. Once a day as a Reaction.\\
- \textbf{Sum of Traits in common at 10 points}: Your touch rots food (up to 50kg/Encumbrance 10) and water (a cube with an edge of 10m). Once a day\\
- \textbf{Sum of Traits in common at 15 points}: Your gaze blinds with anger. You cast the Confusion spell, but the only possible outcome is that the targets attack random subjects. DC 25. 1 time a day\\
- \textbf{Sum of Traits in common at 20 points}: You cast the spell Cone of Cold 60 damage, but the damage is Void. DC 25. Once a day\\
- \textbf{Elements/M}: Negative Energy - Void\\
- \textbf{Advantage}: Hard to kill\\
- \textbf{Privileged Magic Lists}: Fire\\
- \textbf{Favorite Weapon}: Battle Ax \\
- \textbf{Rule}: Don't do good deeds without a profit

\subsubsection{Ephrem}\index{Ephrem}\label{efrem}\hypertarget{ephrem}{}

\begin{changemargin}{0.3cm}{0.3cm}\begin{enfasi}{
Not deviating from nature and training ourselves on its laws and examples is wisdom. (Lucius Anneus Seneca)
}\end{enfasi}\end{changemargin}

He is the Patron Saint of those who make nature their home. He embodies within himself the purest aspects of nature itself, aggressive as only the most lethal felines can be; but also wild like the most hidden clearings and rigorous like only nature can be.

Efrem aims to defend Nature from the contamination of man, from this infesting species that destroys everything it encounters.

The Devotees of Ephrem. also called druids\index{Druid}, they are more closely linked to the natural element. They manipulate mainly elemental magic and also defend or attack using animals and natural creatures. The most powerful forcing even the Dragons to obey.

Ephrem Devotees have the supreme goal of protecting animals and plants, places and everything that is natural and not artificial. Usually solitary and grumpy, he cannot understand the reason for the hatred that, from his point of view, the man unloads on Yeru.

An Ephrem Devotee respects life as well as death, in the natural process that is evolution and the life cycle. Sometimes he decides to settle in a certain environment and elects it as his territory and protects it as if it were his home. Other times he decides to be a wanderer and intervene throughout Yeru to protect his beloved plants and animals.

\noindent- \textbf{Symbol}: A bracket with a vine wrapped around it\\
- \textbf{Feature}: Constitution\\
- \textbf{Traits}: Indifferent, Loyal, Ironic, Pragmatic, Measured, Sober, Austere, Grumpy, Respectful, Solitary, Sincere\\
- \textbf{Demonstration}: spirals of leaves envelop the sword\\
- \textbf{Sum of Traits in common at 5 points} points: Your touch makes non-magical animals docile. Will save 20 to resist. 3 times a day. Cost 2 Actions.\\
- \textbf{Sum of Traits in common at 10 points}: You gain +1d6 to all Survival checks that take place in a natural environment.\\
- \textbf{Sum of Traits in common at 15 points}: You can cast the Beneficial Berries spell 1 time per day. Each berry heals 1d6 hit points and removes nonmagical diseases or poisons.\\
- \textbf{Sum of Traits in common at 20 points}: Your touch is that of the master. You can tame even magical creatures such as Aberrations or Dragons that you touch. Will save DC 30. Once per day. Cost 2 Actions\\
- \textbf{Elements/N}: Electricity, Sound\\
- \textbf{Advantage}: Empathy with plants\\
- \textbf{Privileged Magic Lists}: Animals and Plants and an Elemental Magic List.\\
- \textbf{Favorite Weapon}: Staff\\
- \textbf{Rule}: Nature is always your first choice

\subsubsection{Erondil}\index{Erondil}\label{erondil}\hypertarget{erondil}{}

\begin{changemargin}{0.3cm}{0.3cm}\begin{enfasi}{
Good reasoning is stronger than two strong hands. (Sophocles)
}\end{enfasi}\end{changemargin}

Patron of Earth and Air, Erondil is the Lord of the most concrete and rational elements. He who, endowed with infinite power and rationality, gives his Devotees the power of manipulating the earth, the gift of creating gigantic constructions of millenary strength from simple mud. He concludes his works with attention and precision.
Even with difficulty because if the final result doesn't satisfy him he unleashes his lightning to destroy it instantly. Perfectionist and insatiable, something is rarely exactly as he imagined it.

Orderly and exuberant he is the lord of storms, thunder and lightning, earthquakes and destruction. He loves to surround himself with the roar of thunder, the roar of the crumbling earth. He can be destructive towards those who do not respect Yeru.
He has arms and chest covered in almost silvery tattoos that tell the legends of Earth and Air.

The Devotees of Erondil are the engineers of the impossible, every time matter, gravity and reason itself must be challenged, a Devotee of Erondil will find his match, he will find the right challenge for a Builder of the Impossible.\\

\noindent- \textbf{Symbol}: a sandcastle with lightning on it\\
- \textbf{Feature}: Wisdom\\
- \textbf{Traits}: Insatiable, Perfectionist, Dreamer, Exuberant, Jealous, Destructive, Orderly, Superficial, Naive, Pragmatic, Rational\\
- \textbf{Demonstration}: sound of storm and roar of landslide\\
- \textbf{Sum of common traits 5 points}: You no longer fear falls. You can cast the Fall Feather spell 3 times per day, only on yourself.\\
- \textbf{Sum of Traits in common at 10 points}: Your touch shapes the stone. You can cast the Pass Door spell 1 time per day.\\
- \textbf{Sum of Traits in common at 15 points}: You can cast the Lightning spell from your hands. Reflex Saving Throw DC 30 to halve. Cost 2 Actions.\\
- \textbf{Sum of Traits in common at 20 points}: You are able to create a very deep pit (1km) under your opponent (size up to large). Reflex Saving Throw 30 or fall 100 meters each round. Once a day. After 1 minute the pit closes with whoever is inside. Cost 2 Actions.\\
- \textbf{Elements/N}: Sound, Electricity\\
- \textbf{Advantage}: Universal digestion\\
- \textbf{Privileged Magic Lists}: Air, Earth\\
- \textbf{Favorite Weapon}: Warhammer\\
- \textbf{Rule}: You must not allow the destruction of architectural monuments

\subsubsection{Gaya}\index{Gaya}\label{gaya}\hypertarget{gaya}{}

\begin{changemargin}{0.3cm}{0.3cm}\begin{enfasi}{
{\small Splendor of the finished day, which lifts me up and fills me,

prophetic hour, now that you bring back the past!

And you swell my throat, you, most egalitarian divine,

you, earth and life until the last ray shines, I sing. (Song at Sunset, Walt Whitman)}
}\end{enfasi}\end{changemargin}


Patron of Water and Fire, in the depths of the earth, where water and lava meet, Gaya enjoys painting. She loves to surround herself with flows of fire and water as if to create a dance between them. She loves the sounds of nature, the crashing of waves on the rocks, the falling of raindrops on the cobblestones, the hum of a crackling fire.

She paints by mixing hot and cold. The crystalline and impetuous water with the intriguing and burning fire. Jealous of beauty and the arts, she keeps all her works safe in an almost maniacal and protected order. As a true artist, she uses the elements to make the wonders of nature shine. Gaya is the painter of sunsets and storms.

Gaya Devotees are fickle and over-the-top artists. They are those who recreate the magic of dawn or sunset or the stormy sea in their works, they are those who put poetry and madness into normality.\\

\noindent- \textbf{Symbol}: a brush on the sky\\
- \textbf{Feature}: Intelligence\\
- \textbf{Traits}: Chassis, Instinctive, Impetuous, Emotional, Touchy, Lunatic, Dreamer, Jealous, Fickle, Enthusiastic, Narcissist\\
- \textbf{Manifestation}: coils of fire and water surround the caster\\
- \textbf{Sum of Traits in common at 5 points} points: You can create up to 5 liters of water or 1 liter of good quality liquor. Once a day. Cost 2 Actions.\\
- \textbf{Sum of Traits in common at 10 points}: Your metabolism is not afraid of the cold. You resist magical cold damage and are immune to natural cold damage.\\
- \textbf{Sum of Traits in common at 15 points}: You can breathe underwater like you breathe air. Resist nonmagical fire damage\\
- \textbf{Sum of Traits in common at 20 points}: Generate a rain of fire. You cast the flaming strike spell, DC 25 once per day. Resist magical fire damage.\\
- \textbf{Elements/N}: Cold, Fire\\
- \textbf{Advantage}: Rainbow\\
- \textbf{Privileged Magic Lists}: Water, Fire\\
- \textbf{Favorite Weapon}: Trident\\
- \textbf{Rule}: Don't stop yourself from listening to good music

\bigskip


\begin{changemargin}{0.3cm}{0.3cm}\begin{narratore}
\textbf{Gaia} and \textbf{Erondil} are like the two sides of the same coin and supervise the elements, Gaia water and fire and Erondil Air and Earth; they act as a direct expression of the major Patrons, they are small manifestations of their immense power.
\end{narratore}\end{changemargin}

\subsubsection{Krondal}\index{Krondal}\label{krondal}


\begin{changemargin}{0.3cm}{0.3cm}\begin{enfasi}{
Freedom, Sancho, is one of the most precious gifts the heavens have ever given to men; neither the treasures that the earth contains nor that the sea covers are to be compared with it; for freedom, as for honor, one can and must put one's life at risk. (Miguel de Cervantes)
}\end{enfasi}\end{changemargin}

He is a powerful but shy and reserved Patron. He keeps himself on the sidelines, out of the picture until he feels the deprivation of freedom.

He cannot see into the future, he cannot know people but his formidable instincts make him the most fearsome fighter you will ever meet. Brave to the point of recklessness, he acts fearlessly in battle. With a good spirit, Krondal enters the field in the most important moments, when it is not a situation that is decided but the future of life, of one's personal freedom.

Krondal, the blind fury, has a deep respect for freedom and is profoundly against any extremism, racism or dictatorship.

A Devotee of Krondal is typically a bodyguard, a protector, the sheriff who knows and must decide for the good of his country, whatever the cost.
A Devotee of Krondal does not judge people or facts based on their context but rather sticks to his ethics of protection and freedom.

Under modest and threadbare clothes, but always clean, he hides a fighter's physique.\\

\noindent- \textbf{Symbol}: A sword held vertically in front of you\\
- \textbf{Characteristic}: Charisma\\
- \textbf{Traits}: Attentive, Devoted, Correct, Liberal, Conformist, Insatiable, Bold, Reserved, Self-effacing, Instinctive, Courageous, Reckless\\
- \textbf{Manifestation}: the Devotee's cloak or robe becomes clean and shiny\\
- \textbf{Sum of Traits in common at 5 points} points: Curse your opponent. You cast the cast curse spell once per day. DC 20 to resist.\\
- \textbf{Sum of Common Traits at 10 points}: You don't want to be tied up or handcuffed. Twice per day you can only cast Freedom of Movement on yourself.\\
- \textbf{Sum of Traits in common at 15 points}: Your presence takes away the sight of your opponents. Designate up to 6 creatures within 30 feet, they must make a Fortitude save at DC 30 or be blind to you only for 1d4 rounds.\\
- \textbf{Sum of Traits in common at 20 points}: Your weapon is more effective against enemies. Each creature hit must make a Will save DC 20 or be paralyzed for 3 rounds. Once the creature succeeds on its saving throw, it cannot be affected again for the next 24 hours. Once a day, activating the ability costs 1 Action and lasts 1 minute.\\
- \textbf{Elements/B}: Positive Energy, Fire\\
- \textbf{Advantage}: Magnetic\\
- \textbf{Privileged Magic Lists}: Abjuration\\
- \textbf{Favorite Weapon}: Long Sword\\
- \textbf{Rule}: Do not allow abuse

\subsubsection{Ledyal}\index{Ledyal}\index{Laydel}\label{ledyal}\label{laydel}\hypertarget{ledyal}{} \hypertarget{laydel}{}

\begin{changemargin}{0.3cm}{0.3cm}\begin{enfasi}{
The soul of a man is immortal and incorruptible. (Plato)\\

It is not the body that defines me (anonymous creature)
}\end{enfasi}\end{changemargin}

He is the Patron Saint without a precise face, without a voice except a song. Changeable in body and without a clear definition of his being. He appears with a long fiery red cloak made of fabric made of a thousand butterflies. His touch is life and peace, he protects those who need his favors regardless of whether he asks for them or not. He desires a world without suffering, with only happiness and harmony. Suspicious and deeply introverted, he does not believe those who agree with him. He has a heart full of life and goodness but he doesn't have a body to love with.

Ledyal also has a twin sister, or maybe another personality, or maybe they are the same Patron, no one has ever seen them together. The \emph{twin} \textbf{Laydel}\index{Laydel} does not tolerate suffering, despises those who cause pain, kills without fear any creature who has sinned against an innocent, anyone who has caused suffering.\\

\noindent- \textbf{Symbol}: A butterfly dripping blood as it flies\\
- \textbf{Feature}: Wisdom (Ledyal) - Strength (Laydel)\\
- \textbf{Ledyal Traits}: Introverted, Suspicious, Charitable, Clement, Shy, Loyal, Generous, Understanding, Passionate, Patient, Spontaneous \\
- \textbf{Laydel Traits}: Upright, Suspicious, Relentless, Susceptible, Passionate, Angry, Anarchic, Destructive, Touchy, Cynical, Ruthless\\
- \textbf{Manifestation}: as if a cloak of butterflies enveloped the Devotee\\
- \textbf{Sum of Traits in common at 5 points} points: Your touch is life/attack. 3 times per day you can touch a living creature and heal it/cause 1d6 hit points. Cost 2 Actions (also includes the Touch Action)\\
- \textbf{Sum of Traits in common at 10 points}: Your touch is peace. You can cast the Sanctuary spell twice per day.\\
- \textbf{Sum of Traits in common at 15 points}: Your aura protects your companions. Within 6 meters your companions have a +4 to Defense and a +2 to Saving Throws. Duration 10 consecutive minutes, once a day. Cost 2 Actions.\\
- \textbf{Sum of Traits in common at 20 points}: You radiate a healing sphere around you. Each creature within 20 feet is healed for 60 hit points. Once a day. In Laydel's case the effect is the opposite. Cost 2 Actions\\
- \textbf{Elements/B}: Positive Energy, Electricity\\
- \textbf{Advantage}: Healer (Ledyal) or Fearless (Laydel)\\
- \textbf{Privileged Magic Lists}: Cure or Invocation\\
- \textbf{Favorite Weapon}: Truncheon/Spiked Chain\\
- \textbf{Rule}: Do not allow violence against a creature's gender

\subsubsection{Nethergal}\index{Nethergal}\label{nethergal}

\begin{changemargin}{0.3cm}{0.3cm}\begin{enfasi}{
Dreams are answers to questions that we are not yet able to formulate. (X-Files)\newline

An uninterpreted dream is like an unread letter. (Talmud)
}\end{enfasi}\end{changemargin}


The Patron Messenger. Nethergal's letter flies on a goose's feather. Rapid, impetuous, direct, Nethergal is the messenger, the one to whom one can entrust thoughts and writings. Sarcastic and talkative she will inquire about your purposes, she will ask you for information on the writings entrusted to her with explicit frankness and will always have something to say about the message to be conveyed but she will also be equally direct and precise in delivering it.

Nethergal is not just chatter and gossip, whatever text is written she knows it, there is no written code or secret that she doesn't know.

The Devotee of Nethergal is a fine linguist, an expert in riddles and rebuses, a Devotee who, unlike Atmos, does not limit himself to guarding the writings but spreads their knowledge.

A Nethergal Devotee is a teacher, a professor of languages ​​in a college, a learned expert on a thousand topics.

Nethergal also has another role, she is the Patron Saint of dreams and visions, she shares this task with Sixiser who instead dominates nightmares.

\noindent- \textbf{Symbol}: an iridescent white feather\\
- \textbf{Characteristic}: Dexterity\\
- \textbf{Traits}: Sarcastic, Impetuous, Immature, Logorrhea, Competitive, Reckless, Sociable, Impatient, Clumsy, Outspoken, Curious\\
- \textbf{Event}: cascade of feathers, a goose in flight\\
- \textbf{Sum of Common Traits at 5 points} points: You can send a message of up to 144 characters to a subject that you can see within 50 meters without being heard/seen. Once an hour. Cost 1 Action. The subject must understand the language used.\\
- \textbf{Sum of Traits in common at 10 points}: By placing your hand on a book you learn its contents as if you had read it. One book a week. You lose the knowledge thus gained after a week. Time 1 round. The written language of the tome must be known.\\
- \textbf{Sum of Traits in common at 15 points}: You can fly, like the spell of the same name, 1 hour per day. Cost 1 Reaction.\\
- \textbf{Sum of Common Traits at 20 points}: Understand any writing that is not magical or coded.\\
- \textbf{Elements/N}: Electricity, Sound\\
- \textbf{Advantage}: Absolute Direction\\
- \textbf{Privileged Magic Lists}: Transmutation, Air\\
- \textbf{Favorite Weapon}: Light crossbow\\
- \textbf{Rule}: Do not destroy a book or letter

\subsubsection{Nedraf}\index{Nedraf}\label{nedraf}


\begin{changemargin}{0.3cm}{0.3cm}\begin{enfasi}{
Does he really think he's fighting for anything other than his survival? (Matrix Revolutions, Film)\newline

No matter how narrow the passage,\\
How full of punishments is life.\\
I am the master of my own destiny:\\
I am the captain of my soul (Invictus, William Ernest Henley)
}\end{enfasi}\end{changemargin}


The Surviving Patron, the never-tired old wolf who has gone through and fought a thousand battles. His flesh is bruised, his body covered in battle scars and bruises but nothing will break him. Tenacity, passion, experience and a lot of anger make Nedraf not only an excellent fighter on any occasion but a connoisseur of the environment around him. Thanks to his impeccable training he knows how to make the most of the resources available. He knows how to passionately spur the men under his orders.
Nedraf represents the one you would always want next to you in every battle.

Many captains of fortune and commanding officers are Devotees of Nedraf. The Devotee of Nedraf does not give up, does not give up, does not abandon his companions but this does not mean he is reckless or irrational in his choices.\\

\noindent- \textbf{Symbol}: a strong hand, wrapped in a blood-stained bandage brandishing a sword\\
- \textbf{Feature}: Constitution\\
- \textbf{Traits}: Disciplined, Combative, Tenacious, Aggressive, Planner, Mischievous, Honorable, Competitive, Angry, Rational, Determined\\
- \textbf{Demonstration}: the smell of blood and metal spreads in the air\\
- \textbf{Sum of Traits in common at 5 points} points: You can wear light armor without penalty on the Magic Test\\
- \textbf{Sum of Traits in common at 10 points}: Acquire a bonus point on a Weapon List. It may or may not be known\\
- \textbf{Sum of Traits in common at 15 points}: You can wear medium armor without penalty on the Magic and Dexterity Test\\
- \textbf{Sum of Traits in common at 20 points}: Acquire a bonus point on a Weapon List. It may or may not be known\\
- \textbf{Elements/B}: Positive energy, Sound\\
- \textbf{Advantage}: Accelerated healing\\
- \textbf{Privileged Magic Lists}: Enchantment, Earth\\
- \textbf{Favorite Weapon}: Two-handed Greatsword\\
- \textbf{Rule}: Don't abandon your companions

\subsubsection{Nihar}\index{Nihar}\label{nihar}

\begin{changemargin}{0.3cm}{0.3cm}\begin{enfasi}{
What is a hero? He is an individual with great talent and extraordinary courage, who knows how to choose good over evil, who sacrifices himself to save others, but above all... who acts when he has everything to lose and nothing to gain. (They Called Him Jeeg Robot, movie)
}\end{enfasi}\end{changemargin}

He is the Patron Saint of accidental heroes. Thoughtful and calm, he loves good wine and carousing. He is the one you would never choose as a comrade in arms because of his appearance and his jovial attitude. But then when it's time to be there, to fight, to make the difference, he amazes everyone and solves the game.

He has the appearance of a small man, with sumptuous and refined clothes and a guarded and cheerful expression. He always protects himself no matter what, showing the world exactly what the world wants to see. He carefully monitors the reality around him and even if it is always easier to see him with a glass in his hand, if you don't let yourself be fooled by appearances you will notice how his eyes never lose sight of the danger, the problem. He is careful, he doesn't trust anything or anyone. He made his weaknesses into his strengths.

\noindent- \textbf{Symbol}: A dagger resting near a glass of wine\\
- \textbf{Feature}: Intelligence\\
- \textbf{Traits}: Altruistic, Determined, Courteous, Attentive, Distrustful, Chaotic, Joker, Foresight, Combative, Ironic\\
- \textbf{Event}: the sound of a toast or the uncorking of a bottle\\
- \textbf{Sum of Traits in common at 5 points} points: You can turn water into wine. One liter a day. Cost 2 Actions. 2 times a day.\\
- \textbf{Sum of Traits in common at 10 points}: A Reaction Action, you gain a bonus of +2d6 on a Proficiency check in that round. 1 time a day.\\
- \textbf{Sum of Traits in common at 15 points}: Your light weapon always deals critical damage when you hit. The bonus is always active.\\
- \textbf{Sum of Traits in common at 20 points}: The dishes you prepare are delicious. Anyone who fills themselves with a dish prepared by you recovers 2d6 Hit Points and is cured of poisons, including magical ones. Max 6 people per day. 0.5 hours of preparation per person.\\
- \textbf{Elements/B}: Positive Energy, Fire\\
- \textbf{Advantage}: Universal language\\
- \textbf{Privileged Magic Lists}: Enchantment, Divination\\
- \textbf{Favorite Weapon}: Short Sword\\
- \textbf{Rule}: Don't refuse a good glass of wine

\subsubsection{Orudjs}\index{Orudjs}\label{orudjs}

\begin{changemargin}{0.3cm}{0.3cm}\begin{enfasi}{
Nothing is easier than deluding yourself. Because man believes what he desires is true. (Demosthenes)
}\end{enfasi}\end{changemargin}


That is, the Patron Saint of illusion and fiction. He who only with the gift of speech, the gesticulation of his hands, his charismatic voice and his intriguing gaze manages to sell his every word as absolute truth. He loves theater for what it is for him, the representation of human falsehood, of being many people and in reality no one. He loves politics and its intrigues. He pretends to listen to those around him but in reality he is not interested in other people's stories because his are always the best.

He is a coward without limits and the few truths he says, and they are truly rare, are said by him only to save himself.

With a rather ordinary and almost obvious appearance, as soon as he opens his mouth and begins his stories he manages to attract the attention of the entire room. In fact, he has a warm and persuasive voice which, accompanied by the excellent dialectic he possesses, enchants every listening ear.

His Devotees are skilled actors and entertainers, undercover spies, diplomats or politicians.

\noindent- \textbf{Symbol}: A white theatrical mask with only the mouth open and eyes closed\\
- \textbf{Characteristic}: Charisma\\
- \textbf{Traits}: Ironic, Cowardly, Know-it-all, Sociable, Inconstant, Creative, Lunatic, Dishonest, Snobbish, Liar, Insolent\\
- \textbf{Event}: the sound of a deep and contagious laugh\\
- \textbf{Sum of Traits in common at 5 points} points: Your elocution is already legendary. +2 on Perform checks.\\
- \textbf{Sum of Traits in common at 10 points}: You can cast Silent Image 3 times per day.\\
- \textbf{Sum of Common Traits at 15 points}: Your elocution is already legendary. Additional +1d6 on Perform checks. You can cast Greater Image 1 time per day.\\
- \textbf{Sum of Traits in common at 20 points}: Your voice is persuasive. A creature you detect that listens to you for at least one minute must make a Will save DC 30 or be under the influence of dominate person/monster. Once a day\\
- \textbf{Elements/N}: Electricity, Fire\\
- \textbf{Advantage}: Persuasive voice\\
- \textbf{Privileged Magic Lists}: Enchantment, Illusion\\
- \textbf{Favorite Weapon}: Rapier\\
- \textbf{Rule}: You must always have the last word

\subsubsection{Orlaith}\index{Orlaith}\label{orlaith}

\begin{changemargin}{0.3cm}{0.3cm}\begin{enfasi}{
I was made to fight crime, not to govern it. The time has not yet come when honest men can serve their country with impunity. The defenders of freedom will always be outcasts as long as the mob of scoundrels dominates. (Maximilien de Robespierre)
}\end{enfasi}\end{changemargin}

That is, the Patron Saint of Justice and Vengeance. He follows the laws slavishly and expects his subordinates to carry out the orders given without any discussion. He is moved by a kind and good spirit which however he keeps well hidden behind his direct and incisive, shameless and deadly actions. Orlaith is revenge made law. He acts out of a sense of justice with his methods. What attracts him is his bearing and his proud gaze.

Orlaith's Devotees are often judges and executioners, people who have decided to bring justice everywhere, because Orlaith cannot stand still, there is always someone to judge and punish.\\

\noindent- \textbf{Symbol}: A hand stretched out over a closed book\\
- \textbf{Feature}: Strength\\
- \textbf{Traits}: Impartial, Fair, Vengeful, Valorous, Outspoken, Outgoing, Spontaneous, Enterprising, Obnoxious, Conformist, Traditionalist\\
- \textbf{Demonstration}: the image of a steelyard, unbalanced.\\
- \textbf{Sum of Traits in common at 5 points} points: You call to you 1 mastiff (normal) who obeys your commands. Duration 1 minute. Once a day. Cost 2 Actions.\\
- \textbf{Sum of Traits in common at 10 points}: A pair of handcuffs manifests around the wrists of the creature (maximum large size) within 27 meters. Reflex save DC 25 to cancel. Cost 2 Actions. Once a day. Force/Escape Artist DC 20 to break free.\\
- \textbf{Sum of Traits in common at 15 points}: You create a ray of Light 27 meters long and a few centimeters wide. Each creature passed through takes 5d6 damage, DC 25 Reflexes for half. Once a day. Cost 2 Actions.\\
- \textbf{Sum of Traits in common at 20 points}: Your hearing is only for the truth. Around you for 3 meters, including yourself, the Zone of Truth is always active.\\
- \textbf{Elements/B}: Light, Sound\\
- \textbf{Advantage}: Common sense\\
- \textbf{Privileged Magic Lists}: Illusion, Fire\\
- \textbf{Favorite Weapon}: Infantryman's Lance\\
- \textbf{Rule}: Do not refuse an order from a legitimate authority

\subsubsection{Rezh}\index{Rezh}\label{rezh}


\begin{changemargin}{0.3cm}{0.3cm}\begin{enfasi}{
Greed, I can't find a better word, is valid, greed is right, greed works, greed clarifies, penetrates and captures the essence of the evolutionary spirit. Greed in all its forms: greed for life, for love, for knowledge, for money, has shaped the forward momentum of all humanity. And greed, listen to me, will not only save the Teldar Charter, but also the other dysfunctional society called America. (Gordon Gekko from the film Wall Street, 1987)
}\end{enfasi}\end{changemargin}

The Patron who despises everything. Rezh loves, wants, touches, admires only his shiny and shiny coins. They are never enough, no wealth is ever enough for her. Rezh, the miser keeps everything to herself, knows no compassion, knows no charity, knows no sharing. Her hunger for money and riches makes her prone to any meanness. She despises everything and everyone and judges everything and everyone following only her personal yardstick. In every coin there is a little Rezh. Rezh's imprint can be seen in the oxidation of each coin.

If money buys freedom Rezh must accumulate more and more if it will ever be enough.

Rezh's Devotees are usually chosen by her from the ranks of the greediest and richest. Their aim is to accumulate wealth, more and more.
Often the Devotees of Rezh become explorers, grave robbers, people always looking for treasure and an extra coin.\\

\noindent- \textbf{Symbol}: a pile of coins with a rat nearby\\
- \textbf{Feature}: Intelligence\\
- \textbf{Traits}: Greedy, Arrogant, Mean, Cold, Jealous, Habitual, Uncertain, Irritable, Attentive, Disloyal, Intolerant\\
- \textbf{Manifestation}: the sound of falling coins envelops the caster\\
- \textbf{Sum of Traits in common at 5 points} points: You are an expert in coins and gems, no counterfeiter can deceive you. +1d6 on related Awareness and Knowledge checks.\\
- \textbf{Sum of Traits in common at 10 points}: You use gems as receptacles. You can download a spell of 3rd level or lower into a gem, which must have a minimum value of 10th x spell level. The gem retains the enchantment for 6 hours. To activate the gem you use 2 actions and the spell it contains is performed by consuming the gem.\\
- \textbf{Sum of Traits in common at 15 points}: You can take 1 gold coin out of your pocket whenever you want. Max 10 gp per day. Cost 1 Action.\\
- \textbf{Sum of Traits in common at 20 points}: Your armor is covered in golden glitter and gems. You gain +4 Defense and +2d6 Fortitude save for 1 hour. Cost 1 Reaction, once per day.\\
- \textbf{Elements/M}: Vacuum, Electricity\\
- \textbf{Advantage}: Fairy Hands\\
- \textbf{Privileged Magic Lists}: Abjuration\\
- \textbf{Favorite Weapon}: Sickle\\
- \textbf{Rule}: Don't leave treasure unattended

\subsubsection{Sumkjr}\index{Sumkjr}\label{sumkjr}

\begin{changemargin}{0.3cm}{0.3cm}\begin{enfasi}{
Everything that is not donated is lost. (Dominique Lapierre)
}\end{enfasi}\end{changemargin}

Patron of the Arcanum of Light. Sumkjr is goodness, fairness, loyalty, justice, protection.

Sumkjr is the knight who protects the innocent, he is the sword of Ljust in the final battle. Defends the weak and soothes wounds.

Sumkjr brings the Light of Ljust everywhere, no danger can ever stop Sumkjr from his continuous, infinite, search for good.

A Devotee of Sumkjr acts loyally and with honor, always pursuing the ultimate good, his being cannot be bent to evil, injustice, dishonor.

With courage and determination the Devotee faces every challenge but not only out of a sense of duty, but because he is deeply devoted to his destiny. Sumkjr knows that few people stand up to this standard because unlike the Devotees of the Patroness of Genesis, his Devotees are not born to be such, but become so thanks to their profound and determined willpower.

For this reason Ljust intervenes on their behalf with the elaborate Rite of Renewal, thanks to which every year the deserving Devotee is made to add a Trait point.

Sumkjr is a valiant soldier, the best friend of the righteous.

Calicante, filled with horror at the sight of such a Patron, deprived him of the ability to love and experience true feelings of affection. Bringing good to a Sumkjr Devotee is something normal, just as it is normal to not be able to be empathetic with those who suffer. The Devotee knows what he must do and why, but he is unable to be moved or love when faced with the suffering or caresses of a woman/man.\\

\noindent- \textbf{Symbol}: three drops of blood falling one after the other\\
- \textbf{Characteristic}: Charisma\\
- \textbf{Traits}: Just, Curious, Good, Valorous, Candid, Disorderly, Idealist, Martyr, Protective, Humble, Stubborn\\
- \textbf{Manifestation}: the Devotee is wrapped in a golden brocade cloak\\
- \textbf{Sum of Traits in common at 5 points} points: The touch of your sword is life. A creature touched by your weapon regains 3d6 hit points. Once a day. Cost 2 Actions.\\
- \textbf{Sum of Traits in common at 10 points}: Your Will is stronger than metal. You gain a +2 on Will saving throws\\
- \textbf{Sum of Traits in common at 15 points}: You can cast the spell Cone of Cold, 40 Electricity damage. DC 25 to halve. Once a day. Cost 2 Actions.\\
- \textbf{Sum of Traits in common at 20 points}: You sacrifice your life to bring a creature dead for no more than 1 week to life. Once. Cost 3 Actions.\index{Resurrect}\\
- \textbf{Elements/B}: Positive Energy, Electricity\\
- \textbf{Advantage}: Aura of Courage\\
- \textbf{Privileged Magic Lists}: Cure\\
- \textbf{Favorite Weapon}: Bastard Sword\\
- \textbf{Rule}: Do not perform sexual acts\\

\textbf{The 7 Bright Rules}\index{7 Bright Rules}\\

The Seven Luminous Rules are a set of rules and behaviors held, in various capacities, by Devotees who want to follow the path of the Light of Just.

The Devotees of Sumkjr must follow all of them and 7 other Devotees of other Patrons, always positive or at least neutral, follow only some of these dictates, as a rule to avoid falling into the arms of Calicante\\

\noindent 1. Protect the weak and those who cannot defend themselves from abuse\\
2. Love life and protect it.\\
3. Fight against injustices and those who bring suffering and pain\\
4. Soothe wounds and pains. Calm souls and promote peace and harmony\\
5. Honesty and Loyalty are your foundation\\
6. You are a teacher of virtue. Let others be inspired by your deeds\\
7. Don't let your inaction generate suffering.

\subsubsection{Shayalia}\index{Shayalia}\label{shayalia}

\begin{changemargin}{0.3cm}{0.3cm}\begin{enfasi}{
Whoever plants a garden sows happiness (Chinese proverb)\\

Nothing great in the world has been accomplished without passion. (Georg Wilhelm Friedrich Hegel)

}\end{enfasi}\end{changemargin}

Patron of the Arcana of Darkness. Shayalia is the dark soul of perdition, of betrayal, of the most sordid and sinful lust. She loves brothels. She likes the acrid smell of sweat, the skin shiny with oils and perfumes. The passions, the revenge that are consumed there, the physical and moral destruction that is perpetrated in those places is her life.

Shayalia is a woman, often alone, who enjoys and experiences the most physical pleasures of her own. She lives on long and carefully planned revenges. Vengeful and amoral, she does not judge with human standards, her enjoyment of her is not even remotely understandable. Shayalia is the closest thing to Calicante that has been created. It is the passions, the impulses, the humoral liquids that make it inebriate.

Shayalia is the concubine who bewitches and destroys you, drop by drop. Her poisons are her weapons, her human weaknesses are her field.

The Devotees of Shayalia are spies, bastard children, lovers of powerful lords who act in the shadows.

Just disgusted by the vision of such a Patron, she instilled in Shayalia a love and passion for plants and animals. And so many of the most famous botanists, herbalists and zoologists are Devotees of Shayalia, perhaps the only things that she Shayalia can truly love.

\begin{changemargin}{0.3cm}{0.3cm}\begin{narratore}
While Ephrem is the patron of the wild, Shayalia embodies devotion and love for nature. The first oversees uncontaminated nature from above, the second descends and becomes one with it by building magnificent gardens.
\end{narratore}\end{changemargin}

\medskip

\noindent- \textbf{Symbol}: a crumpled pillow stained with blood\\
- \textbf{Characteristic}: Charisma\\
- \textbf{Traits}: Lustful, Detached, Cold, Vengeful, Arrogant, False, Mischievous, Rebellious, Dishonest, Dreamer, Affable\\
- \textbf{Manifestation}: the Devotee is wrapped in a black velvet cloak\\
- \textbf{Sum of Traits in common at 5 points} points: The time to prepare a potion is halved. Healing spells also affect animals and plants.\\
- \textbf{Sum of Common Traits at 10 points}: Your touch is life for nature. Your healing spells work on natural animals and plants in a maximized manner.\\
- \textbf{Sum of Traits in common at 15 points}: From your palm you secrete poison. Your touch, or melee weapon carries the poison. Saving Throw Fortitude DC 25 or -2 to Wisdom and Dexterity for 10 minutes, a poisoned subject cannot be poisoned again for 24 hours. Cost 1 Action.\\
- \textbf{Sum of Traits in common at 20 points}: Your touch is life for nature. You can heal magical animals and plants. You are immune to natural poisons. +1d6 Nature Knowledge.
- \textbf{Elements/M}: Vacuum, Electricity\\
- \textbf{Advantage}: Animal Empathy\\
- \textbf{Privileged Magic Lists}: Illusion or Animals and Plants and an Elemental Magic List\\
- \textbf{Favorite Weapon}: Whip\\
- \textbf{Rule}: Don't give up on humiliating

\bigskip

\textbf{Sumkjr} and \textbf{Shayalia} are complementary in holding the elusive ranks of creatures. They act as a direct expression of the Patrons of Genesis.

\subsubsection{Sixiser}\index{Sixiser}\label{sixiser}

\begin{changemargin}{0.3cm}{0.3cm}\begin{enfasi}{
The strength that opposes destiny is actually a weakness. (Franz Kafka)\\

Now be quiet, child, don't cry,

Mommy will make all your nightmares come true. (Mother, 1979 The Wall, Pink Floyd)

}\end{enfasi}\end{changemargin}


The Patron who is indifferent to the present as he is totally, compulsively obsessed with the future and his destiny. In the most remote corners of the known worlds it is said that Sixiser accumulates everything, indifferent to everything and everyone.

Terrified by the future he sees, by a hypothetical end of himself and altogether he lives a life of retreat, spiritual and physical. He voluntarily deprives himself of everything necessary. But at the same time he accumulates whatever object crosses his path in the hope of a return.

He is paranoid and doesn't trust anyone. He uses his divination powers to know and scrutinize everyone.

Sixiser is master of nightmares, dreams more frightening than visions of death. He often uses nightmares as a means of communication with his followers.

Sixiser Devotees are often necromancers surrounded by undead and other silent, obedient creatures. Those who take refuge in search of solitude and study, those who aim to expand and govern entire cities and nations in order to feel more secure, are Devoted to Sixiser.\\

\noindent- \textbf{Symbol}: A chest overflowing with everything that cannot be closed\\
- \textbf{Feature}: Wisdom\\
- \textbf{Traits}: Reserved, Indifferent, Hoarder, Paranoid, Liar, Insensitive, Fearful, Petty, Ambitious, Austere, Know-it-all\\
- \textbf{Manifestation}: two hands surrounding, as if to hide, the enchanter's head\\
- \textbf{Sum of Traits in common at 5 points} points: acquire twilight vision up to 9 meters, or 18 meters if already present.\\
- \textbf{Sum of Traits in common at 10 points}: see in the dark even magical within 9 meters. You automatically detect nonmagical traps within 10 feet of you.\\
- \textbf{Sum of Traits in common at 15 points}: By touching an object you are able to understand all its magical and non-magical properties, even if it is cursed. 3 times a day.\\
- \textbf{Sum of Traits in common at 20 points}: You are able to animate a creature that has been dead for no more than a day as an undead from 1 level of Challenge (type zombie/skeleton depending on the status). Once a day. Cost 2 Actions.\\
- \textbf{Elements/M}: Electricity, Negative Energy\\
- \textbf{Advantage}: Reduced consumption\\
- \textbf{Privileged Magic Lists}: Necromancy\\
- \textbf{Favorite Weapon}: Glaive\\
- \textbf{Rule}: Don't trust it

\subsubsection{Tazher}\index{Tazher}\label{tazher}

\begin{changemargin}{0.3cm}{0.3cm}\begin{enfasi}{
A person often ends up looking like his shadow. (Rudyard Kipling)
}\end{enfasi}\end{changemargin}

The Patron of Shadows; he who is silent, kills you. You'll never know why. You will never know what he looks like, but if you suddenly get a cold feeling, Tazher is behind you ready to take your life.

Double agent with a bad soul, ask for his help only if you are willing to pay the price that he and he alone will decide.

He lives at night, he lives at night. Shadows are his friends and darkness is his cloak. Profoundly individualistic with a grumpy and touchy character, he has no friends and does not maintain relationships of any kind.

Ljust, horrified by so much hatred and nihilism, instilled respect for the dead in the Patron. A Deve si Tazher will not attack a deceased person or violate his corpse. Many undead hunters are devotees of Tazher.

The Devotee of Tazher is the thief, the murderer, the bandit, anyone who lives for darkness and his own gain. A Tazher Devotee is extremely dangerous in combat.\\

\noindent- \textbf{Symbol}: The glint of the blade in the dark\\
- \textbf{Characteristic}: Dexterity\\
- \textbf{Traits}: Grumpy, Calculating, Perfectionist, Mean, Insensitive, Grumpy, Corrupt, shy, Disloyal, Imaginative, Hypocrite\\
- \textbf{Manifestation}: the Devotee's shadow comes to life by moving the weapon\\
- \textbf{Sum of Traits in common at 5 points} points: Gains +2 on Stealth checks.\\
- \textbf{Sum of Traits in common at 10 points}: Once a day you make an additional attack (without penalty). An Immediate Action.\\
- \textbf{Sum of Traits in common at 15 points}: as long as you walk over shadows or in the dark (dim light or darkness) you are invisible. You can still be detected with light or divination spells.\\
- \textbf{Sum of Traits in common at 20 points}: Every successful melee attack in the round generates a critical. Cost 1 Reaction Action to be declared even after the attack roll but before knowing if the rolls were successful. Usable 3 times a day.\\
- \textbf{Elements/M}: Void, Ice\\
- \textbf{Advantage}: My shadow is my friend\\
- \textbf{Privileged Magic Lists}: Transmutation\\
- \textbf{Favorite Weapon}: Glaive up for auction\\
- \textbf{Rule}: 5 Seconds. Time to steal from a dead man, no more.

\subsubsection{Thaft}\index{Thaft}\label{thaft}


\begin{changemargin}{0.3cm}{0.3cm}\begin{enfasi}{
Death, therefore the most atrocious of all evils, does not exist for us. When we live death is not there, when it is there we are not there. It is nothing to either the living or the dead. For the living there is no more, the dead are no more. (Epicurus)
}\end{enfasi}\end{changemargin}

The Patron who accompanies in birth and death. Silent, he stands aside and observes the passing of men's lives. Almost humble in its simplicity, Thaft is everywhere. Silent witness to human life; the moment a life slips away, Thaft witnesses, the moment a life is born, Thaft is present.

Thaft also knows that you can't always just be an observer. Through his sacred and magical notebook he can decide and judge the lives of men, because if a sword wounds, it is only Thaft who decides their death.

The Devotees of Thaft are the priests of the final journey, those who protect and watch over the souls and bodies of the dead. Deeply opposed to the use of the undead, they pursue their destruction.

A Devotee of Thaft respects life as well as death and is not afraid of causing destruction for greater balance.

Thaft was shaped by Atmos.\\

\noindent- \textbf{Symbol}: An open book with a skull on it\\
- \textbf{Feature}: Wisdom\\
- \textbf{Traits}: Simple, Silent, Mild, Confident, Disciplined, Optimistic, Modest, Worldly, Respectful, Outspoken, Indulgent\\
- \textbf{Demonstration}: you hear the cry of a newborn baby or the sigh of death\\
- \textbf{Sum of Traits in common at 5 points} points: Your touch is lethal to the undead. Your touch deals 2d6 damage to an undead. Cost 2 Actions including touch. Up to 3 times a day.\\
- \textbf{Sum of Traits in common at 10 points}: Your touch soothes. Once per day you can remove Blindness or Deafness. Cost 2 Actions.\\
- \textbf{Sum of Traits in common at 15 points}: An undead, with CR lower than the sum of your Traits in common, must make a DC 30 Fortitude saving throw or be destroyed if touched by your hand. Cost 2 Actions.\\
- \textbf{Sum of Traits in common at 20 points}: Kill the touched creature. Will save DC 30 or death. Once a week. Cost 2 Actions.\\
- \textbf{Elements/N}: Sound, Electricity\\
- \textbf{Advantage}: Frosty touch\\
- \textbf{Privileged Magic Lists}: Necromancy, Animals and plants\\
- \textbf{Favorite Weapon}: Bow\\
- \textbf{Rule}: Do not create an undead

\subsubsection{Torbiorn}\index{Torbiorn}\label{torbion}

\begin{changemargin}{0.3cm}{0.3cm}\begin{enfasi}{
Regarding arrogance, the violent suffer from it, but the wise laugh at it. (Titus Livy, attributed to Astimede)
}\end{enfasi}\end{changemargin}


The Patron who best embodies the concept \emph{it is never enough}. Tall, beautiful like a classical statue but, just like the latter, without warmth and life, Torbiorn borders on maniacal perfection in his dressing and attitude.

Nothing is ever enough for him. No one is ever up to him. And here he is, with arrogance and irony, changing everything that can be changed in order to appease this profound dissatisfaction. If the final result achieved does not satisfy him, and it happens very often, his cynicism takes over and destroys everything without caring about the suffering he is causing to those around him.

The Devotee of Torbiorn is the typical rich and lazy aristocrat, the one who always seeks the easiest and least risky path.

Regardless of others, he enjoys exploiting other people's work and benefiting from it.

\noindent- \textbf{Symbol}: An opaque mirror\\
- \textbf{Characteristic}: Charisma\\
- \textbf{Traits}: Haughty, Anxious, Vain, Touchy, Prudent, Irate, Undisciplined, Licentious, Conceited, Weak, Deceitful\\
- \textbf{Manifestation}: shards of broken mirror all around the Devotee like a whirlwind\\
- \textbf{Sum of Traits in common at 5 points} points: With one gesture you can refresh your clothes and yourself, making them clean and fragrant. Cost 1 Action. 3 times a day.\\
- \textbf{Sum of Traits in common at 10 points}: Your spit is poisonous. If the touch attack roll hits -2 Strength, not stackable. Duration 1 minute. Three times a day. Cost 1 Action.\\
- \textbf{Sum of Traits in common at 15 points}: By staring the target in the eyes you force him to stop. The subject cannot perform Move Actions. Will save DC 30. Once per day. Cost 2 Actions.\\
- \textbf{Sum of Traits in common at 20 points}: Tendrils grow from your fingers and sting up to 10 opponents. Each tendril, up to 60 feet long, causes 2d6 damage, Reflex save DC 25 to halve. Cost 2 Actions.\\
- \textbf{Elements/N}: Fire, Sound\\
- \textbf{Advantage}: Hard to subjugate\\
- \textbf{Privileged Magic Lists}: Transmutation\\
- \textbf{Favorite Weapon}: One-Handed Axe\\
- \textbf{Rule}: Don't be sloppy, poorly dressed or untidy.

\medskip

\begin{changemargin}{0.3cm}{0.3cm}\begin{narratore}

In agreement with the Narrator, and adequately motivated, it is possible to change Advantage and Privileged Magic Lists.

\end{narratore}\end{changemargin}

\end{multicols}


\vfill

\begin{center}
\includegraphics[keepaspectratio,width=0.6\textwidth]{immagini/archetipijung.png}

\emph{Jung's 12 Archetypes}
\end{center}

\pagebreak

\subsubsection{List of Patrons - Tract}\index{List of Patrons - Tract}\hypertarget{tablelinkpatrontract}{}\label{tabellacollegamentopatronotratto}

\textbf{Atherim}: Cheerful, Good, Calm, Clement, Gullible, Emotional, Industrious, Mild, Patient, Silent, Shy\\
\textbf{Atmos}: Anxious, Apathetic, Distrustful, Detached, Upright, Complaining, Observant, Paranoid, Foresighted, Prudent, Reflective\\
\textbf{Belevon}: Liar, Chaste, Confused, Curious, Double Agent, Clumsy, Imprudent, Inconstant, Insolent, Envious, Narcissistic\\
\textbf{Calicante}: Anarchic, Brutal, Cynical, Competitive, Creative, Dishonorable, Selfish, Wrathful, Passionate, Superb, Vengeful \\
\textbf{Cattalm}: Anarchic, Belligerent, Brutal, Calculating, Destructive, Fatalist, Impassive, Petty, Meticulous, Provocative, Sadistic\\ 
\textbf{Efrem}: Austere, Indifferent, Ironic, Loyal, Measured, Pragmatic, Respectful, Grumpy, Sincere, Sober, Solitary\\
\textbf{Erondil}: Destructive, Exuberant, Jealous, Unpleasant, Naive, Orderly, Perfectionist, Pragmatic, Rational, Dreamer, Superficial\\
\textbf{Gaya}: Chassis, Emotional, Enthusiastic, Jealous, Impetuous, Instinctive, Lunatic, Narcissistic, Touchy, Dreamer, Fickle\\
\textbf{Gradh}: Competitive, Courageous, Dark, Distrustful, Cold, Impetuous, Indomitable, Melancholic, Conceited, Protective, Reserved, Vengeful\\
\textbf{Krondal}: Attentive, Bold, Reckless, Conformist, Courageous, Correct, Devoted, Unpleasable, Instinctive, Liberal, Reserved, Selfish\\
\textbf{Laydel}: Anarchic, Cynical, Destructive, Relentless, Upright, Angry, Passionate, Touchy, Suspicious, Ruthless, Susceptible\\
\textbf{Ledyal}: Charitable, Clement, Understanding, Generous, Introverted, Loyal, Passionate, Patient, Suspicious, Spontaneous, Shy\\
\textbf{Ljust}: Altruistic, Nonconformist, Compassionate, Courageous, Correct, Empathic, Extroverted, Generous, Instinctive, Protective, Sensitive\\
\textbf{Lynx}: Stubborn, Controlled, Courageous, Determined, Insensitive, Intolerant, Introverted, Rational, Rigid, Serious, Solitary\\ 
\textbf{Nedraf}: Aggressive, Combative, Competitive, Determined, Disciplined, Mischievous, Honorable, Planning, Angry, Rational, Tenacious\\ 
\textbf{Nethergal}: Reckless, Competitive, Curious, Immature, Impatient, Impetuous, Talkative, Clumsy, Sarcastic, Outspoken, Sociable\\
\textbf{Nihar}: Altruistic, Attentive, Joker, Chaotic, Combative, Courteous, Determined, Distrustful, Ironic, Foresight\\
\textbf{Orlaith}: Obnoxious, Conformist, Outgoing, Fair, Impartial, Enterprising, Outspoken, Spontaneous, Traditionalist, Valorous, Vengeful \\
\textbf{Orudjs}: Liar, Cowardly, Creative, Dishonest, Inconstant, Insolent, Ironic, Lunatic, Know-it-all, Snobbish, Sociable\\ 
\textbf{Rezh}: Habitual, Arrogant, Attentive, Greedy, Mean, Cold, Jealous, Uncertain, Intolerant, Irritable, Disloyal\\ 
\textbf{Shayalia}: Affable, Arrogant, Dishonest, Detached, False, Cold, Lustful, Mischievous, Rebellious, Dreamer, Vengeful\\
\textbf{Sixiser}: Hoarder, Ambitious, Austere, Liar, Indifferent, Insensitive, Petty, Paranoid, Fearful, Reserved, Know-it-all\\
\textbf{Sumkjr}: Good, Candid, Curious, Disorderly, Just, Idealist, Martyr, Stubborn, Protective, Humble, Valorous\\
\textbf{Tazher}: Gruff, Calculating, Evil, Corrupt, Imaginative, Insensitive, Hypocritical, Perfectionist, shy, Grumpy, Disloyal\\
\textbf{Thaft}: Disciplined, Indulgent, Mild, Modest, Worldly, Optimistic, Respectful, Outspoken, Simple, Confident, Silent\\
\textbf{Torbiorn}: Haughty, Anxious, Conceited, Weak, Undisciplined, Deceitful, Irate, Licentious, Touchy, Prudent, Vain

\pagebreak

\section{Equipment}\hypertarget{equipment}{}\label{equipaggiamento}

\subsection{Wealth and Money}\index{Wealth and Money}
\begin{changemargin}{0.3cm}{0.3cm}\begin{enfasi}{I'm ready to go. I have a backpack! (Morgan Grimes, Chuck, TV Series)
}\end{enfasi}\end{changemargin}

\begin{multicols}{2}

\label{ricchezza-e-denaro}

\lettrine[lines=2, lhang=0.33, loversize=0.25, findent=1.5em]{L}{e} common coins come in different denominations based on the relative value of the metal they are made of. The three most common types of coins are the gold coin (mo), the silver coin (ma), and the copper coin (mr).

A capable (but not exceptional) craftsman can earn one gold piece per day. Gold coin is the standard measure of wealth, although money itself is not widely used. When merchants discuss deals involving goods or services worth hundreds or thousands of gold coins, the transactions do not normally involve any exchange of cash. Gold money, on the other hand, is a measure of value and the exchange is carried out in gold bars, letters of credit or valuable goods.

One gold coin is worth ten silver coins, the most used type of coin among the population. A silver coin can cover a laborer's daily wage and buy a flask of lantern oil or a bed for a night in a cheap inn.

One silver coin is worth ten copper coins, which are normally used by workers and beggars. A single copper coin can buy a candle or a piece of chalk.

Sometimes, however, unusual coins made of other precious metals appear among the treasures. The electrum coin (me) and the platinum coin (mp) come from forgotten empires and lost kingdoms; when used in transactions they sometimes arouse suspicion and skepticism. An electrum coin is worth five silver coins while a platinum coin is worth ten gold coins.

A common coin weighs about ten grams, so that fifty coins weigh half a kilo.

A character who starts playing generally has enough gold coins to purchase the basic elements: some weapons, second-hand armor (the least expensive one) and some miscellaneous equipment. As the character embarks on adventures and accumulates loot, he can afford better equipment and magical items. At first level, characters have coins and equipment totaling about 100 gp.

\subsubsection{Coins}\index{Coins}

The most common currency is the gold coin (mo). One gold coin is worth 10 silver coins (but). Each silver coin is worth 10 copper coins (mr). In addition to copper, silver and gold coins there are also platinum (mp) coins, which are each worth 10 gold coins and electrum (me) which are worth 5 silver coins.

\medskip

\textbf{Table: Coin Equivalence}\index[Tables]{Coin Equivalence Table}

\medskip

\begin{tabular}{llllll}

\textbf{Money} & \textbf{MR}&\textbf{MA}&\textbf{ME}&\textbf{MO}&\textbf{MP}\\
\toprule
Copper & 1 & 1/10 & 1/50 & 1/100 & 1/1000\\
Silver & 10 & 1 & 1/5 & 1/10 & 1/100\\
Electrum & 50 & 5 & 1 & 1/2 & 1/10\\
Gold & 100 & 10 & 2 & 1 & 1/10\\
Platinum & 1000 & 100 & 20 & 10 & 1\\
\end{tabular}

\medskip

Usually payments over 100 gold coins take place in 1, 2, 5 kilogram gold bars, equivalent to 100, 200 and 500 gold coins or better yet in gems. In the case of even larger sums, it is possible that a letter of credit from some banking institution will be issued (but valid in very few important cities).

\subsubsection{Wealth at first level}\index{Wealth at first level}\index{First level money}

Typically, a first-level character starts with 100 gp that he can spend on basic equipment.

\subsubsection{Other Riches - Trade Goods}\index{Other Riches}

Merchants usually trade goods even without the use of coins.
To get an idea of ​​trade transactions, some trade goods are described in the table.

\medskip

\textbf{Table: Examples of other riches}\index[Tables]{Table Examples of other riches}

\medskip


\begin{tabular}{ll}
\textbf{Cost} & \textbf{Item}\\
\toprule
1 mr & Wheat (0.5 kg)\\
2 mr & Flour (0.5 kg) or chicken (1)\\
1 sp & Iron (0.5 kg)\\
5 ma & Tobacco or copper (0.5 kg)\\
1 gp & Cinnamon (0.5 kg) or goat \\
2 gp & Ginger or pepper (0.5 kg) or sheep (1)\\
3 gp & Pig (1) \\
4 gp & Linen (1 m\textsuperscript{2}\\
5 gp & Salt or silver (0.5 kg) \\
10 gp& Silk (1 m) or cow (1)\\
15 gp& Saffron(0.5 kg)/ox (1)\\
30 gp&Cloves (1kg)\\
\end{tabular}

\medskip

Also consult the chapter on Encumbrance in Movement and Transport.

\end{multicols}

\pagebreak

\section{Equipment - Weapons}\index{Equipment}\index{Weapons}\label{equipaggiamentoarmi}
\hypertarget{equipment.weapons}{}

\label{equipaggiamento---armi}
\begin{changemargin}{0.3cm}{0.3cm}\begin{enfasi}{
This is my rifle. There are many like him, but this is mine. My rifle is my best friend, it's my life. I must dominate it as I dominate my life. Without me my rifle is nothing; without my rifle I am nothing. I must know how to hit the target, I must shoot better than my enemy who is trying to kill me, I must shoot before he shoots me and I will do it. In the presence of God I swear by this creed: my rifle and myself are the defenders of the homeland, we are the dominators of our enemies, we are the saviors of our lives and so be it, until there is no more enemy but only peace, amen. (Full Metal Jacket, Film 1987)

\medskip

The truly good sword is the one that stays in its sheath. (Sanjuro)}\end{enfasi}\end{changemargin}

\medskip

Using a weapon without the proper proficiency imposes a -1d6 to hit

The table presents the name of the weapon, its cost in gold coins, the damage and the type of damage (whether slashing, bludgeoning or piercing), the range, the Weapon List it belongs to and the special characteristics it can have. See also \hyperref[sec:loading-and-transport-encumbrance capacity]{Loading and Transport Capacity.}

\medskip

\textbf{Table: List of Weapons}\index[Tables]{Table List of Weapons}

%\begin{tabularx}{llllll}
\begin{xltabular}{0.99\textwidth}{lllX}
\textbf{Weapon}&\textbf{Cost}&\textbf{Size/Damage} & \textbf{Range, List, Special}\\
\toprule
Halberd& 10 & G/1d10 P/T& \textbf{Spears}, \textbf{Staffs}, Countercharge, Long weapon, ED9 \\
Short Composite Bow& notes*& M/Arrows& 20 meters, \textbf{Bows}\\
Short Bow& 30 & M/1d6 P& 15 meters, \textbf{Bows}\\
Long Composite Bow& notes*& G/Arrows& 36 metres, \textbf{Bows}\\
Long Bow& 75 & G/Arrows& 20 metres, \textbf{Bows}\\
Hammer Axe& 16 & M/1d6 T/C& \textbf{Axes}\\
One-handed axe& 6 & M/1d6 T& 6 metres, \textbf{Axes}, \textbf{Throwing Weapons}, Versatile\\
Battle Axe& 10 & M/1d10 T&\textbf{Axes}\\
One-handed crossbow& 100& M/Bolts& 6 meters, \textbf{Crossbows}\\
Light Crossbow & 35 & P/Bolts & 15 metres, \textbf{Simple Weapons}, \textbf{Crossbows}\\
Heavy crossbow & 50 & G/Bolts & 30 meters \textbf{Crossbows}\\
Staff& 3& M/1d6 C& \textbf{Simple Weapons}, Long Weapon, Versatile\\
Brandistocco& 10 & M/2d4 P/T& \textbf{Spears}, Countercharge, Long Weapon\\
Spiked Chain& 25 & G/2d4 P& 3 meters, \textbf{Rotating Balls}, Long Weapon\\
Estoc& 25& G/1d8 P& \textbf{Auctions}, Long weapon\\
Scythe& 18 & G/2d4 P/T& \textbf{Weapons of Death}, Long Weapon\\
Sickle& 6& P/1d6 T& \textbf{Weapons of Death}\\
Pole Glaive& 12 & G/1d10 P/T& \textbf{Spears}, Countercharge, Long Weapon, ED9\\
Glaive& 75 & M/2d4 T& \textbf{Graceful Weapons}, ED7\\
Slingshot& -& P/1d4 B& 10 meters, \textbf{Bows}\\
Double Flail& 90 & M/1d10 C& \textbf{Spinning Balls}, \textbf{Double Weapons}\\
Heavy Flail& 15 & M/1d10 C& \textbf{Spinning Balls}\\
Scourge& 8& M/1d8 C& \textbf{Rotating Balls}, \textbf{Skull Breaker}\\
Whip& 1& M/1d3 T& \textbf{Spinning Balls}, Long Weapon\\
Javelin& 1& P/1d6P& 12 metres, \textbf{Simple Weapons}, \textbf{Auctions}, \textbf{Throwing Weapons}\\
Great Double Axe& 25 & G/1d12 T& \textbf{Axes}, \textbf{Double Weapons}, Long Weapon\\
Big Club& 2& M/1d8 C&\textbf{Skull Breaker}\\
Spiked Gauntlet& 5& P/1d4 P&\textbf{Stun Weapons}\\
Katana& 300& M/1d10 T& \textbf{Swords}, \textbf{Lethal weapons}, ED9\\
Infantryman's spear& 2& M/1d8 P&3 meters, \textbf{Spears}, Long weapon, Countercharge\\
Spear& 10 & G/1d8 P&\textbf{Spears}, Long Weapon, Countercharge\\
Machete& 10 & M/1d6 T&\textbf{Lethal Weapons}\\
War fist& 7& G/1d10 C& \textbf{Skull Breaker}\\
Truncheon& 1& P/1d6 C& \textbf{Stun weapons}, non-lethal\\
Warhammer& 5& M/1d8 C/P& 6 meters, \textbf{Skull Breaker}\\
Light Mace& 3& P/1d6 C/T& \textbf{Simple Weapons}, \textbf{Light Weapons}, \textbf{Skull Breaker} \\
Heavy Mace& 5& M/1d8 C/T& \textbf{Skull Breaker}\\
Spiked mace& 6& M 1d8 C/P& \textbf{Simple Weapons}, \textbf{Skull Breaker}\\
Naginata& 8& G/1d10 T&\textbf{Spears}, Long Weapon, ED9\\
Light Pike& 4& M/1d4 P&\textbf{Weapons of Death}\\
Heavy Pike& 8& G/1d6 P&\textbf{Weapons of Death}, Long Weapon\\
Dagger& 2& P/1d4 P& 6 metres, \textbf{Simple Weapons}, \textbf{Light weapons}, \textbf{Throwing Weapons}\\
Fist/Naked Kick& notes*& P/1d4 C&Versatile\\
Club& 1& P/1d6 C& \textbf{Simple Weapons}, \textbf{Skull Breaker}\\
Scimitar& 15 & M/1d6 T&\textbf{Light Weapons}, \textbf{Graceful Weapons}, Versatile\\
Short Sword& 10 & P/1d6 P&\textbf{Light Weapons}, \textbf{Swords}, Versatile\\
Long Sword& 15 & M/1d8 T&\textbf{Swords}\\
Double-bladed sword& 100& G/1d8 T& \textbf{Double weapons}, \textbf{Swords}\\
Bastard Sword& 35 & M/1d8 T&\textbf{Swords}, 1d8 1-handed, 2d6 2-handed\\
Broadsword& 12 & M/2d4 T&\textbf{Swords}, 2d4 one-handed, 1d10 two-handed\\
Two-handed greatsword& 50 & G/2d6 T&\textbf{Swords}\\
Rapier& 20 & P/1d6 P& \textbf{Light Weapons}, \textbf{Graceful Weapons}, Versatile\\
Trident& 15 & M/1d6 P/T& 3 meters, \textbf{Auctions}, \textbf{Thrownable Weapons}, Long Weapon, Countercharge\\
Urgrosh& 18 & M/1d6 T/P& \textbf{Spears}, \textbf{Double weapons}\\
\end{xltabular}

\medskip

A \textbf{Weapon} Small has \textbf{Encumbrance} 1, a Medium Weapon has Encumbrance 2, a Large Weapon has Encumbrance 4, a Huge Weapon has Encumbrance 8.\index{Weapon Encumbrance}\index{Encumbrance Weapons}

\medskip

\textbf{Table: List of projectiles - Bows - Crossbows - Slingshots}\index[Tables]{Table List of projectiles - Bows - Crossbows - Slingshots}

\begin{tabular}{lcc}
\textbf{Projectile Name}& \textbf{Number of shots/Cost (gp)} & \textbf{Damage / Type}\\
\toprule
Crossbow bolts (one handed, light) & 10/1 gp & 1d6 P\\
Crossbow bolts (heavy) & 3/1 gp & 1d10 P\\
Hunting Arrows (Short Bow, Long Bow) & 20/1 gp & 1d6 P\\
War Arrows (Longbow) & 10/1 gp & 1d8 P\\
Marble Marbles (slingshots) & 15/1 gp & 1d4 B\\
Stone (slingshots)& -& 1d2 B\\
\end{tabular}

\medskip

A \textbf{Quiver} (full or empty) of Projectiles (Arrows or Darts) has \textbf{Encumbrance} 2.\index{Projectile Encumbrance}\\

A \textbf{heavy bolt} for Crossbow penetrates metal armor more easily causing +2 additional damage.\index{Quel from Heavy Crossbow}\index{Heavy Crossbow}

\begin{multicols}{2}

A +1 Weapon costs 1500 gp, +2 5000 gp. It is not possible to purchase weapons with enchantments higher than +2, they must be found.

A magic arrow/bolt/stone with a +1 bonus costs 25 gp, if +2 it costs 100 gp. Bullets with a magic bonus greater than +2 are almost impossible to find.

\textbf{A projectile does not gain magical properties because its caster is magical.}

\medskip

\textbf{Empty Fist}: \hyperlink{Empty Fist}{see List of Weapons}

\bigskip

\textbf{Composite Bow}\index{Composite Bow}
A composite bow is a particularly sturdy and rigid bow that requires a certain minimum of Strength to be used effectively.
A \textbf{composite bow} long has a fixed modifier, from +1 to +5, the bonus applies only to damage and not to attack rolls. A composite bow applies a damage bonus equal to \textbf{minimum value between Strength and its bonus}.

A +3 composite bow used by a character with Strength 2 cannot be fully shot and so the arrow that fires will have a damage modifier of +2.
A +1 composite bow used by a character with Strength 4 can be fully shot and thus the arrow that fires will have a damage modifier of +1.

The cost of a composite bow depends on its modifier.
A composite bow with a +1 modifier costs 75 gp, +2 150 gp, +3 300 gp, +4 600 gp, +5 1500 gp. It is not possible to purchase composite bows with bonuses greater than +3, they must be \emph{found}.

A composite shortbow has a maximum Strength modifier of +3.

\textbf{Crossbow}\index{Crossbows}\index{Crossbow refill}
A heavy crossbow requires two Actions to reload. A light or one-handed crossbow requires 1 action to reload.

\textbf{Range}\index{Range}\index{Throw away}
The distance indicated is that at full attack roll. Each ranged weapon can hit within three times the listed distance.

If the target is within the indicated distance there is no hit penalty, if the target is between the first and second increments the hit penalty is -1d6. If the target is between the second and third increments the hit penalty is -2d6.

A javelin thrown within 12 meters has no penalty, but thrown within 24 meters has a -1d6 to hit, at a distance between 24 and 36 meters a -2d6 to hit, beyond that it cannot be thrown.

\medskip

\begin{center}
\includegraphics[width=0.7\linewidth]{immagini/bow2.png}
\end{center}

\medskip

A \textbf{Projectile that hits is considered destroyed}, if it misses it has a 50\% (4-5-6 on a d6) chance that it is still intact.

A magic projectile adds its bonuses to those of the caster to determine the attack roll and damage.

\medskip

The \textbf{Size of the Weapon}\label{dimensionediunarma}\hypertarget{sizeofaweapon}{} is indicated as P (small), M (medium), G (large) and refers to a medium creature. \hyperref[weapon too large]{See section Weapon too large}

A \textbf{higher dimensional weapon} \index{Higher dimensional weapon} such as a Longsword forged for an Ogre increases its damage die by one category.

The Weapons indicated a \textbf{Damage type}\index{Damage type}, i.e. T/C/P.

These letters indicate whether the damage is of the Cutting, Blunt or Puncture type. This characteristic can be important because certain creatures can be immune to or take less damage from a particular type of wound (e.g. a skeleton against a piercing weapon or a gelatinous cube against a cutting weapon...).

A weapon can be used to cause a different type of damage (from slashing to piercing or bludgeoning) by reducing the damage die by one category (e.g. Long Sword to deal bludgeoning damage causes 1d6).

\medskip


\textbf{Masterwork Weapons}\index{Masterwork Weapons}

A perfect weapon is a weapon created by a very skilled gunsmith which, although not magical, thanks to its perfect balance and sharpening, has a +1 to attack roll.

To create a perfect weapon, a gunsmith must exceed the DC set for the creation of the weapon with a critical success (DC = 15 + $\sqrt{Costo Arma} $)\index{Building weapons}

A masterwork weapon costs twice as much as a normal weapon.

\textbf{Improvised Weapons}\index{Improvised Weapons}

Sometimes items that were not designed to be weapons can have some combat effectiveness. Since these are not objects designed for this use, the creature that attacks with one of them suffers a -1d6 penalty on its attack roll. A small (bottle) improvised weapon deals 1d3 damage, medium (a chair leg) 1d6, large (table leg) deals 1d8 damage.

An improvised thrown weapon has a range of 10 feet.

\medskip

\textbf{Throwing weapons}\index{Throwing weapons}

A sword or in any case a weapon not made to be thrown can still be thrown at the opponent. The attack roll takes -1d6 and the weapon does a lower category of damage (the long sword does 1d6, a short sword 1d4..). The range is 3 meters.

\medskip

\textbf{Using a weapon without the proper proficiency if it is not a simple weapon} Imposes a -1d6 on the attack roll.

\textbf{Example}: A small creature using a halberd in close combat has -1d6 because the weapon is large, -1d6 because it is not proficient, -1d6 because it uses the weapon in melee.

In this case, since the penalties are greater than 3d6, the character does not roll dice but only uses his Weapon Expertise and Strength as the value to hit.


\end{multicols}

\vfill

\begin{center}
\includegraphics[width=0.6\linewidth]{immagini/armiriempitivo3.png}
\end{center}


\pagebreak

\section{Equipment - Armor and Shields} \index{Armours}\index{Shields}\hypertarget{equipment.armor.shields}{}\label{equipaggiamentoarmature}

\label{equipaggiamento---armature-e-scudi}

\begin{changemargin}{0.3cm}{0.3cm}\begin{enfasi}{
Armor (s.f.). Suit worn if your tailor is a blacksmith. (Ambrose Bierce)

\medskip

Fantozzi armor: 4-wind vane acting as a plume, scary Viking helmet with zero visibility, bronze jockstrap taken from the statue of Pepin the Short and, on his feet, molten lead charcoal irons. Overall weight of Fantozzi armor: 4 quintals, 32 kilos and 7 and a half ounces. (Superfantozzi, Film)} \end{enfasi}\end{changemargin}\medskip

\lettrine[lines=2, lhang=0.33, loversize=0.25, findent=1.5em]{L}{e} armor helps to be unaffected (raises Defense) and penalizes Magic Checks and competence checks.

The Proficiency Penalty is the penalty that applies to proficiency checks affected by the weight and bulk of the armor. Different armor, specific or magical have different scores, this table serves as a guideline for the Storyteller.\index{Armor Penalty}

\subsubsection{Armor Table}\index[Tables]{Armor Table}

\label{tabella-armature}
\begin{tabular}{llllllll}
%\begin{xltabular}{0.95\textwidth}{lXXXXXXX}
\textbf{Armor} & \textbf{Cost (gp)} & \textbf{Defense} & \textbf{Comp. Penalty} & \textbf{Type} & \textbf{Movement} & €11094 €{Trial Magic}&\textbf{Encumbrance}\\
\toprule
Padded & 5 & 1 & 0 & L & 0 & NO &2\\
Leather & 10 & 2 & 0 & L & 0 & YES &2\\
Reinforced leather& 25 & 3 & 0 & L & 0 & YES &2\\
Mail Jacket & 15 & 4 & -1 & M & 0 &+1d6&4\\
Scales & 50 & 5 & -1 & M & 0 &+1d6&4\\
Rings & 150 & 6 & -1 & M & 0 &+1d6&4\\
Breastplate & 200 & 6 & -2 & M & 0 &+1d6&4\\
Bands & 250 & 7 & -2 & P & 0 &+2d6&8\\
Half Armor & 1200 & 8 & -2 & P & 1 &+2d6&8\\
from Field & 1400 & 9 & -3 & P & 2 &+2d6&8\\
Complete & 1500 & 10 & -4 & P & 3 &+2d6&8\\
\end{tabular}

\begin{multicols}{2}

\textbf{Cost}: it is for a medium size armor.

\textbf{Defense}: it is the bonus given to Defense

\textbf{Comp. Penalty}: is the penalty given to Proficiency checks given by the weight and bulk of the armor.

\textbf{Type}: indicates whether the armor is \textbf{L}light, \textbf{M}medium or \textbf{P}heavy. A \textbf{Armour} Light has €11105{Encumbrance} 2, a Medium Armor has Encumbrance 4, a Heavy Armor has Encumbrance 8.\index{Armor Encumbrance}

\textbf{Mov. (movement)}: is the reduction in meters of movement to be applied per Movement Action.

\textbf{Magic Test}: indicates whether the test should be performed (YES) or not (NO). If dice are indicated (+1d6,+2d6..) it means that the Magic Test must be made with the added dice marked. See Armor and Shields and Magic.

\textbf{Costs}: The cost of a +1 armor or shield is 2250gp, +2 10000gp. It is practically not possible to purchase armor or shields or weapons with enchantments higher than +2, they must be \emph{found}.

\subsubsection{Armor, Shields and Magic}\index{Armor and Magic}\index{Shields and Magic}\hypertarget{armor and magic}{}\label{armatureemagie}

All Armor, with the exception of Padded Armor, forces the spell caster to pass a Magic Test without considering any critical successes.\index{Armor and Magic Test}. Light Armor and Shields force you to make the Magic Test with no dice added, Medium ones with +1d6, Heavy ones with +2d6. 

In practice, the heavier the armor or shield, the more dice you roll, the more chances you have of failing.

Even if it is the player who requests a Magic Test, only one test will be performed with the sum of the dice due to armor and/or shields. \textbf{Wearing armor or shield will negate any critical magical successes gained but will not prevent critical magical failures.}

\medskip

\begin{changemargin}{0.3cm}{0.3cm}\begin{narratore} When you count the encumbrance given by the armor and shield \textbf{worn} you must divide it by two.

The Encumbrance of armor and shields is to be understood when it is \emph{loaded in the backpack}, i.e. transported but not worn.\end{narratore}\end{changemargin}

\subsubsection{Armor Description}

\textbf{Light Armour}

Made of lightweight, flexible materials, light armor benefits agile adventurers by offering protection without sacrificing mobility.

\emph{Padded}. Padded armor consists of layers of fabric and padding sewn together.

\emph{Leather}. The chest and shoulder pads of this armor are made of leather that has been hardened after being boiled in oil. The rest of the armor is made up of
softer and more flexible materials.

\emph{Reinforced Leather}. Made of tough but flexible leather, reinforced leather armor is accented with rivets or spikes.

\medskip

\textbf{Medium Armor}

Medium armor offers more protection than light armor, but limits movement.

\emph{Knit Jacket}. Made of interlocking metal rings, a mail jacket is worn over layers of clothing or leather. This type of armor offers modest protection to the upper body, while the noise of the rings rubbing together is muffled by the other layers.

\emph{flakes}. This armor consists of a mail and greaves (sometimes a separate skirt) of leather covered by overlapping pieces of metal, similar to the scales of a fish. The armor comes complete with gloves.

\emph{Rings}. This armor is leather armor with heavy rings sewn onto it. The rings serve to strengthen the armor against sword and ax blows. The armor comes complete with gloves.

\emph{Bib number}. This armor consists of a metal chest worn over a layer of leather. Although it leaves the arms and legs relatively exposed, the armor provides good protection to the character's vital organs, without causing great bulk.

\begin{center}
\includegraphics[width=0.7\linewidth]{immagini/donnacavalierecavallo.png}
\end{center}

\textbf{Heavy Armour}

\emph{Bands}. This armor is made of metal strips sewn to a sturdy back of leather and chainmail. The size of the metal plates, interconnected to the metal bands, and the layers of armor underneath make it one of the most protective of armor.

\emph{Half Armour}. Half plate armor consists of shaped metal plates that cover most of the character's body. It does not include leg protection other than simple greaves tied with leather laces.

\emph{from Campo}. Very similar to full body armor but lighter in construction sacrificing a little protection for greater flexibility and mobility.

\emph{Complete}. This armor consists of interlocking shaped metal plates that cover the entire body. Plate armor includes gauntlets, heavy leather boots, a helmet with a visor, and a thick layer of padding beneath the armor. Buckles and laces distribute the weight of the armor across the entire body.


\subsubsection{Basic rules for using armor}

\textbf{Using Armor without the appropriate proficiency} prevents you from using the Dexterity bonus and decreases the Defense bonus provided by 1.

\textbf{Using a Shield without the appropriate proficiency} worsens the attack roll by 1 and decreases the Defense Bonus granted by 1.

\textbf{Sleeping in Armor}: If you sleep in medium or heavy armor, you are automatically \hyperlink{Fatigued}{Fatigued} the following day.

Sleeping in light armor does not cause Fatigue.

The character's \textbf{movement capacity} will remain the same until banded armour, then it will progressively decrease. The value indicated in the Mov. column they are the fewer meters that the character makes per Movement Action.

For example, a human in full armor has movement 6 meters, a dwarf 3 meters.

\textbf{Weight}: the weight indicated refers to the version for Medium-sized characters. Armor adapted for Small characters weighs half as much, while for Large characters it weighs twice as much.

\textbf{Masterwork Armor}\index{Masterwork Armor}

Perfect armor is armor created by a very skilled blacksmith which, although not magical, thanks to its perfect balance, has a +1 to Defense.

To create perfect armor, a blacksmith must exceed the DC set for the creation of the armor with a critical success.

Masterwork armor costs twice as much as normal armor.

\subsubsection{Magic armor}\index{Magic armor}\index{Magic shields}

A magical armor or magical shield not only protects better but is also lighter and similar to magic.

+1 armor lowers the Proficiency penalty by 1 and the Movement penalty by 1 meter.
A +2 armor or shield also removes 1 die from the Magic Test if added. +3 armor further removes 1 from the Proficiency penalty, 1m from Movement and removes 1 die from the Magic Test.

\subsubsection{Shields}

The \textbf{Shields} \index{Shields}allow you to increase your Defense, the more imposing and heavier the shield is, the more it protects, the more the penalties on magical competence checks increase and the less easy it makes fighting (penalty on attack rolls) .

Shields can be Light, Medium or Heavy.

\end{multicols}

\subsubsection{Shield Table}\index[Tables]{Shield Table}

\label{tabella-scudi}

\begin{tabular}{lccccc}
\textbf{Shields} & \textbf{Cost} & \textbf{Defense Bonus} & \textbf{TC penalty} & €11157{Magic Test} & \textbf{Type}\\
\toprule
Light wooden shield & 3 gp & 1 & 0& SI & L\\
Light metal shield & 9 gp & 1 & 0& SI & L\\
Medium wooden shield & 5 gp & 2 & 0& +1d6 & M\\
Medium metal shield & 12 gp & 2 & 0& +1d6 & M\\
Heavy wooden shield & 9 gp & 3 & 1& +2d6 & P\\
Heavy metal shield & 20 gp & 3 & 1& +2d6 & P\\
\end{tabular}

\begin{multicols}{2}

\textbf{The bonus it grants to Defense is counted only in the round from when the shield is raised}.\index{Shield raised}

\textbf{Defense Bonus}: this is the bonus that applies to Defense when the shield is worn.

\textbf{TC Penalty}: it is the penalty to the attack roll that occurs when the shield is worn.

\textbf{Type}: indicates the size of the shield. \textbf{L}light, \textbf{M}dium, \textbf{P}heavy.

A Light €11168{Shield} has €11169{Encumbrance} 1, a Medium Shield has Encumbrance 2, a Heavy Shield has Encumbrance 4.€11170{Shield Encumbrance}

The penalty to the \textbf{Magic Check} is added to any penalty due to the armor and is applied when the shield is worn.\index{Magic Penalty for Shield and Armor}

A shield can be used as \textbf{improvised weapon}. The attack roll is penalized by -1d6 and a small shield does 1d4 damage (B/T), a medium shield does 1d6 damage (B/T), a heavy shield does 1d8 damage (B/T).

Using the shield as an improvised weapon does not apply its Defense bonus unless you take a Reaction to reset it to Defense after attacking.

Holding a shield takes up your hand and arm.

\subsubsection{Putting on and taking off armour}\index{Putting on and taking off armour}

Putting on and taking off armor is an operation that requires time and attention, doing it quickly does not help and actually tends to worsen the protection given by the armour.

\end{multicols}

\textbf{Table: Times for putting on and taking off armour}\index[Tables]{Table Times for putting on and taking off armour}

\begin{tabular}{llll}
\textbf{Armor Type}& \textbf{Put on} & \textbf{Put on quickly} & \textbf{Remove}\\
\toprule
Shield& 1 action & - & 1 action\\
Padded, Leather, Reinforced Leather & 1 minute & 3 rounds & - \\
Giaco di Maglia & 1 minute & 5 rounds & 5 rounds\\
Scales, Rings, Breastplate, Bands & 4 minutes & 1 minute{*} & 1 minute\\
Half Armour, Field, Complete & 4 minutes{*}{*}& 4 minutes{*}& 1d4+1 minutes\\
\end{tabular}

\bigskip

\begin{multicols}{2}

{*} If someone helps, the time is halved. A single character who is doing nothing else can help one or two characters adjacent to him. Two characters cannot help each other put on armor at the same time.

{*}{*} You need help to put on this armor. Without help it can only be put on quickly.

\textbf{Putting on armor in a hurry} carries a -1 penalty to Defense and an additional +1 penalty on Proficiency checks.\\


\end{multicols}


\vfill

\begin{center}
\includegraphics[width=0.4\linewidth]{immagini/buckler.png}

\emph{Buckler, front and back}

\end{center}


%\begin{center}
%\includegraphics[width=0.3\linewidth]{immagini/armaturacorpetto.png}
%\end{center}

\pagebreak


\section{Goods and Services}\index{Goods}\index{Services}


\subsection{Wealth, Money and Equipment}\index{Wealth, Money and Equipment}


\begin{changemargin}{0.3cm}{0.3cm}\begin{enfasi}{
- Doc... we just need a little bit of plutonium.

\medskip

- Ah, I'm sure that in '85 plutonium could be bought in the local grocery store, but in '55 the matter is much more complicated! (Back to the Future, Film 1985)}
\end{enfasi}\end{changemargin}\medskip

\begin{multicols}{2}

\subsubsection{Selling Treasures}

\lettrine[lines=2, lhang=0.33, loversize=0.25, findent=1.5em]{N}{in the} dungeons you explore you will have ample opportunities to find treasure, equipment, weapons, armor and more. Usually, you will be able to sell treasures and trinkets when you reach a town or other settlement, as long as you can find buyers and merchants interested in your loot.

\medskip

\textbf{Weapons, Armor and Other Equipment }

As a general rule, weapons, armor and other undamaged equipment when sold are paid half the original value. The weapons and armor used by monsters are unlikely to be in optimal condition for sale.

\medskip

\textbf{Magic Items}

Selling magic items is a problem. Finding someone willing to buy a potion or scroll isn't too difficult, but most items are beyond the reach of anyone except the wealthiest nobles. Furthermore, aside from a few common magic items, it is difficult to find magic items or spells for sale. The value of magic staggers vile coin and should always be treated with consideration.

\medskip

\textbf{Gems, Jewels and Art Objects}

These items are easier to trade and you can decide to exchange them for money or use them as currency in transactions. In the case of exceptionally valuable treasures, the Storyteller may require that you first find a buyer in a large town or even a larger community.

\medskip

\textbf{Goods}

On the borderlands, most transactions occur through barter. Many practical commodities such as iron ingots, bags of salt, livestock, and so on can be exchanged as current money at their full value.

\textbf{Perfect Equipment}

A perfect tool, in addition to costing 10 times as much as the normal version, grants a +1 to the check in which it is used.

\medskip

%\end{multicols}

%\vfill
%\begin{center}
%\includegraphics[width=0.7\linewidth]{immagini/jewelry-box-2931784_1280.png}
%\end{center}
%\pagebreak

%\begin{multicols}{2}

\subsubsection{Adventure Equipment}\index{Adventure Equipment}\index{Things to buy}

This is a short, non-exhaustive list of equipment your characters might be interested in purchasing. The list is certainly not exhaustive or complete but it can provide you with pricing guidelines.

As a Narrator always use common sense in requests, carefully evaluate the type of request, the need for the object, the place where it is bought and how it is bought.

Depending on the type of companion, additional items such as firearms or alchemicals may be available.

\medskip

{\small
\begin{tabularx}{0.42\textwidth}{lll}
\textbf{Item} & \textbf{Cost} & \textbf{Eng.}\\
Abaco&2 mo&L\\
Monk's Robe & 5 gp & 1\\
Craftsman's robe& 1 gp& 1\\
Peasant dress& 1 ma& 1\\
Courtier's dress & 30 gp & 1\\
Explorer's Outfit & 10 gp& 1\\
Entertainer's Outfit & 3 gp& 1\\
Noble's dress & 75 gp & 2\\
Scholar's Robe & 5 gp& 1\\
Traveler's Outfit & 1 gp& 2\\
Winter dress & 8 mo& 2\\
Royal Robe & 200 gp & 3\\
Steel and flint & 1 gp&\\
Intense Acid (ampule) & 10 gp & L \\
Holy water (ampoule) & 25 gp& L\\
Sewing needle & 5 sts &- \\
Holly and mistletoe & - & -\\
Fish hook & 1 ma& - \\
Ampoule (empty)& 3 mr& L \\
Signet ring & 5 gp& - \\
Poison ring & +20 gp&-\\
Antitoxin (bottle) & 50 gp & L\\
Portable Ram & 10 gp& 3 \\
Craftsman's tools& 5 gp& 2\\
Thieves' tools & 30 gp& 1\\
Rod (3 m) & 5 mr& 2\\
Climber's Tools & 80 gp& 1\\
Banquet (per person) & 10 gp & -\\
Bandolier & 3 gp & L\\
Rowboat& 50 gp & 12\\
Barge & 3000 gp & -\\
Barrel (empty)& 2 gp& 4\\
Staff & 2 gp& 1\\
Merchant's Scales & 2 gp& 1\\
Beer Mug& 5 mr& L\\
Beer Carafe& 2 ma& L\\
Ceramic mug & 2 mr& L\\
Bottle of ink or potion & 1 gp& L \\
Bag& 5 ma&L\\
Belt pouch (empty) & 1 mo& L\\
Components Bag & 25 gp& L\\
Healer's Bag& 50 gp & 1\\
\end{tabularx}

\begin{tabularx}{0.42\textwidth}{llll}
\textbf{Object} & \textbf{Cost} & \textbf{Encumbrance}\\
Glass bottle & 2 mo& L \\
Ceramic jug (5lt) & 2 mr& L\\
Campanella & 1 gp& - \\
Candle & 1 mr& -\\
Fishing rod & 1 gp&1\\
Spyglass & 1000 gp & 1 \\
Ceramic jug & 2 mr& L\\
Meat (1 piece) & 3 but& L\\
Cart & 15 gp & 10\\
Wagon & 35 gp& 20\\
Carriage & 300 gp & -\\
Pulley and tackle & 20 gp& 2 \\
Paper (sheet)& 4 ma& -\\
Chest (empty) & 2 gp& 3 \\
Chain (10 ft.) & 30 gp & 1\\
Sealing wax& 1 gp& -\\
Waxed&5 ma&1\\
Basket (empty) & 4 sp& 1 \\
Rock climber's bolt& 1 sp&L\\
Hourglass& 25 gp & -\\
Winter blanket & 5 ma& 1 \\
Hemp rope (15 m)& 1 gp& 1\\
Coarse hemp rope (15 m)& 2 mo& 2 \\
Spider silk rope (15 m)& 10 gp & L\\
Sharpening whetstone & 2 mr& L \\
Case for Darts or Arrows & 1 gp& 1 \\
Case for maps or scrolls & 1 gp& 1 \\
Whistle & 8 ma& - \\
Cheese (1 piece)& 1 but& \\
Chest & 5 gp&4\\
Alchemist's Fire (flask)& 20 gp& L\\
Galley & 30k gp & -\\
Metal hook & 1 mo& L\\
Chalk, (1 piece) & 1 mr& - \\
Bed& 1 but& 1 \\
Ink (30 g bottle)& 8 gp& - \\
Alchemist's Laboratory & 200 gp & 5\\
Common Lantern& 1 gp& 2 \\
Bulging Lens Lantern & 12 gp & 1 \\
Shieldable lantern& 7 gp& 1 \\
Firewood (per day)& 1 mr& 4 \\
Magnifying glass & 100 gp & -\\
Locanda Buona (sleep) & 2 mo& -\\
Locanda Modesta (sleep)& 5 but& -\\
Poor Inn (sleep) & 2 but& -\\
Mallet& 1 gp& 2 \\
Handcuffs & 15 gp & L \\
Hammer& 5 sp& 1 \\
Bit and bridle & 2 gp&1\\
Sailing ship & 10k gp & -\\
Warship & 25k gp & -\\
Longship & 10k gp & -\\
Lantern oil& 1 sp& 1 \\
Water Clock & 1000 gp & -\\
Wineskin & 1 gp& 2 \\
Shovel or shovel & 2 gp& 1 \\
Bread (per loaf) & 2 mr& -\\
Meals (per day) Good & 5 ma&-\\
Meals (per day) Modesto& 3 Tues&-\\
Meals (per day) Poor & 6 mr&-\\
Plush & 2 ma& - \\
Nib & 1 ma& - \\
Iron pot & 8 sp & 1 \\
Parchment (Sheet) & 2 sp& - \\
Miner's pickaxe & 3 gp& 2 \\
Crowbar& 2 gp& 1 \\
Healing Potion & 50 gp & L\\
Enhanced Healing Potion & 125 gp & L\\
\end{tabularx}


\begin{tabularx}{0.42\textwidth}{lll}
\textbf{Object} & \textbf{Cost} & \textbf{Encumbrance}\\
Perfume & 5 gp & L\\
Grapple & 1 gp& 1 \\
Travel rations (per day) & 5 ma& L \\
Oar & 2 mo& 2\\
Fishing net (2.25 m)& 4 gp& 1 \\
Saddlebags & 4 gp& 2\\
Bag (empty) & 1 but& L \\
Sleeping bag & 3 mo& 2 \\ \index{Sleeping bag}\index{Bed}
Soap (per 0.5 kg) & 5 ma& - \\
Ladder (3 m) & 2 ma& 3 \\
Bucket (empty)& 5 ma& L\\
Gallop Saddle & 30 gp& 2\\
Military Saddle & 50 gp & 3\\
Cargo Saddle & 15 gp & 2\\
Exotic saddle& 40 mo& 3\\
Lock/Padlock Good & 80 gp & -\\
Lock/padlock Medium & 40 mo& \\
Simple lock/padlock & 20 gp & -\\
Upper Lock/Padlock & 150 gp & - \\
Metal Spheres (100) & 3 gp & 1\\
Silver Holy Symbol & 25 gp& L\\
Wooden holy symbol & 1 gp& L\\
Sled& 20 gp & 3 \\
Small metal mirror & 10 gp & L\\
Stabling (per day) & 5 ma& -\\
Common musical instrument& 5 gp& 2\\
Tagliola& 5mo&3\\
Canvas (per m2)& 1 ma& L \\
Tent & 10 gp & 3 \\
Torch& 1 but& 1\\
Tribolo (20) & 1 ma& L \\
Camouflage Tricks & 50 gp& L\\
Spade or Shovel & 1 gp&1\\
Devoted Robe & 5 gp& 1\\
Good Wine (bottle) & 10 gp& 1\\
House wine (carafe) & 2 ma& 1\\
Backpack & 2 gp & 1 \\
\end{tabularx}}


\begin{changemargin}{0.3cm}{0.3cm}\begin{enfasi}{
Any sufficiently advanced technology is indistinguishable from magic. (Arthur C. Clarke, from Profiles of the Future)
}\end{enfasi}\end{changemargin}

\textbf{Intense Acid}. As an action, you can splash the contents of this vial onto a creature within 3 feet of you or throw the vial up to 20 feet, shattering it on impact. In either case, make a ranged attack roll against the creature or object, treating the acid as an improvised weapon (-1d6 attack roll). On a hit, the target takes 2d6 acid damage.

\textbf{Holy water}. As an action, you can splash the contents of this flask onto a creature within 3 feet of you or throw the flask up to 20 feet, shattering it on impact. In either case, make a ranged attack roll against the creature or object, treating the holy water as an improvised weapon. If you hit, and the target is a fiend or undead, it takes 2d4 points of positive energy damage.

\textbf{Ampulla (empty)}: small glass or ceramic amphora with a thin neck.

\textbf{Signet Ring}: metal circlet, generally valuable, with an engraving suitable for imprinting seals on sealing wax.

\textbf{Poison Ring}: +20 gp, compared to the ring cost, this ring has a small compartment under the gem, usually used to contain poison. Opening and closing it requires an action; doing so without being noticed requires a DC 20 Fairy Hands check.

\textbf{Antitoxin}. A creature that drinks from this vial of liquid gains +1d6 on saving throws against poison for 1 hour. It does not grant any bonuses to undead and constructs.

\textbf{Portable Aries}. You can use a portable battering ram to knock down doors. When doing so, you gain a +1d6 bonus on Strength checks. Another character can help you with the use of the ram, giving you +2 on the check.

\textbf{Fishing Equipment}. This kit includes a wooden rod, silk thread, wooden cutter, steel hooks, lead weight, velvet bait and a landing net.

\textbf{Bandolier}. This specialized belt for holding small objects such as potions or scrolls is worn around the neck. Extracting and drinking an object from it costs 1 Action, as if from a belt.\index{Bandolier}

\textbf{Metal Marbles}. As an action, you can scatter a single bag of these tiny metal marbles to cover a flat, 10-foot square area. A creature that passes through the covered area must succeed on a DC 12 Reflex save or fall prone. A creature that passes through the area at half speed does not have to make the saving throw.

\textbf{Merchant's Scale}. A merchant's scale includes a small barbell, a plate, and an assortment of weights up to 1 pound. With it, you can measure the exact weight of small objects, such as precious metals or commodities, to help you determine their value.

%\begin{center}
%\includegraphics[height=0.6\linewidth]{immagini/stadera.png}
%\end{center}

\textbf{Components Bag}. A component pouch is a small waterproof leather pouch with compartments containing all the material components and other special items you need to cast your spells, except those components that have a specific cost or are uncommon materials ( as indicated in the spell description).

\textbf{Purse}. A fabric or leather pouch can contain, among other things, up to 20 slingshot bullets or 50 blowgun needles. A compartmentalized pouch for holding spell components is called a component pouch.

\textbf{Candle}. For 1 hour of real game time, a candle casts dim light in a 1 meter radius.

\textbf{Waxed coat}. It is a coat treated to be water-repellent, allowing you to stay dry even in the rain.

\textbf{Telescope}. Objects viewed through a telescope are magnified to twice their size.

\textbf{Pulley and Hoist}. A series of levers connected by a cable and a hook to attach to objects, pulley and tackle allow you to pull up up to four times
weight you can normally lift.

\textbf{Chain}. A chain has 15 hit points and a hardness of 6. It can be broken with a successful DC 24 Strength check.

\textbf{Rock climber's pitons}. You must use 1 at least every 6 meters to fix the rope to the wall.

\textbf{Necrophagous Cockroach Colony}: 3 gp, this glass jar contains carnivorous necrophagous cockroaches. Cockroaches must be fed at least 125 grams of meat per day or they die. When released onto a dead organism, they devour its flesh in 1d4 days, leaving only the bones. Scavenger roaches only eat dead flesh and cannot harm living creatures. Once released, roaches cannot be returned to the jar.

\textbf{Rope}. A rope, whether made of hemp or silk, has 2 hit points and can be broken with a successful DC 19 Strength check. The thick version has 4 hit points, DC 22.

\textbf{Silk Cord} (15 m): 10 gp, this spider silk cord has 4 hit points and can be broken with a DC 23 Strength check

\textbf{Quiver}. A quiver can hold up to 12 arrows\index{Quiver}.

%\begin{center}
%\includegraphics[width=0.5\linewidth]{immagini/forziere.png}
%\end{center}


\textbf{Alchemical Fire}. This sticky fluid ignites when it comes into contact with air. With two actions, you can throw this flask up to 20 feet, shattering it on impact. Make a ranged attack roll against the creature or object, treating alchemical fire as an improvised weapon. On a hit, the target takes 1d6 fire damage at the start of each of its rounds. A creature can end this damage by spending two Actions and succeeding on a DC 12 Dexterity check. If the check succeeds, the flames are extinguished.

\textbf{Healer's Bag}. This kit is a leather bag containing bandages, ointments and splints. The kit can be used ten times. Grants a +2 to first aid checks.

\textbf{Lunch Kit}. 4 gp. This small tin contains a bowl and some simple cutlery. The two parts of the box can be detached, and one side used as a cooking pot and the other as a plate or container

\textbf{Climber's Kit}. 8 mo. A climbing kit includes special pitons, boot spikes, gloves and a harness. You can anchor yourself using the climber's kit with one action; when you do so, you cannot fall more than 7 meters from where you anchored, and you cannot climb more than 7 meters from where you anchored without first undoing the anchor.

\textbf{Lantern}. A lantern projects bright light in a 10-foot radius and dim light for an additional 20 feet. Once lit, it burns for 3 hours of real game time with one ampoule (0.5 liters) of oil.

%\begin{center}
%\includegraphics[width=0.6\linewidth]{immagini/lanterna.png}
%\end{center}

\textbf{Protruding Lens Lantern}. A projecting lens lantern projects light in a 3 meter cone and dim light for a further 9 metres. Once lit, it burns for 3 hours of real game time with one ampoule (0.5 liters) of oil.

\textbf{Shudderable Lantern}. A shieldable lantern projects light in a 6 meter radius and dim light for a further 6 metres. Once lit, it burns for 1 hour of real game time with one ampoule (0.5 liters) of oil. As an action, you can lower the shielding, dimming the light to a 3-foot radius.

\textbf{Magnifying Glass}. This lens allows you to take a closer look at small objects. It is also a useful substitute for flint and steel when starting a fire. Starting a fire with the magnifying glass requires at least sunlight as bright as sunlight, wood to light, and about 5 minutes for the wood to catch fire. A magnifying glass provides aid (+1d6) on any check made to evaluate or analyze a small or highly detailed object.

\textbf{Hunter's Lens}: 100 gp, this complex lens is placed over one eye and occupies the eye slot when in use. When used with a ranged attack, you reduce ranged attack penalties by 1d6. Objects within 30 feet become difficult to see, and you take a –1d6 penalty on sight-based Awareness checks and attack rolls.

\textbf{Handcuffs}. These metal tools can imprison a Small or Medium creature. To free yourself from the handcuffs you must pass a DC 24 Dexterity check. To break them you must pass a DC 24 Strength check. Each set of handcuffs comes with a key. Without the key, a creature can use Escape Artist or Disable Device to open the lock with a successful DC 18 check. The handcuffs have 15 hit points and hardness 2

\textbf{Oil}. It is usually bought in a clay flask containing 0.5 litres. As an action, you can splash the oil in this flask onto a creature within 3 feet of you or throw it up to 20 feet, shattering it on impact. In either case, make a ranged attack roll against the creature or object, treating the oil as an improvised weapon. If you hit, the target is covered in oil. If the target takes any amount of fire damage before the oil dries (after 1 minute), the target takes an additional 1d6 fire damage from the burning oil per round. If ignited, the oil burns for 2 rounds and deals 1d6 fire damage to any creature that enters the area or ends its round within it. A creature can take this damage only once per round. You can also pour a vial of oil on the floor to cover a 1 meter square, as long as the surface is flat.

\textbf{crowbar}. Using a crowbar gives +1d6 on Strength checks whenever the crowbar's leverage can be applied.

\textbf{Healing Potion}. This generic healing potion allows you to recover 1d8+1 hit points.

\textbf{Enhanced Healing Potion}. This generic healing potion allows you to recover 3d8+3 hit points.

\begin{changemargin}{0.3cm}{0.3cm}\begin{narratore} As much as I am personally against characters purchasing magic items, healing potions must be available.
\end{narratore}\end{changemargin}

\textbf{Rations}. Rations consist of dry food suitable for long journeys, and include dried meat, dried fruit, biscuits and nuts.

\textbf{Box with Bait}. This small container contains stone, steel and tinder (usually a dry rag soaked in oil) used to start a fire. Using it to light a torch (or any other easily ignitable object) requires two actions. Lighting any other fire takes 1 minute.

\textbf{Box for Maps or Scrolls}. This cylindrical leather box can contain, rolled up, up to ten pieces of paper or five sheets of parchment.

\textbf{Quiver for Crossbow Bolts}. This wooden box contains up to 12 crossbow bolts.

\textbf{Lock}. A key is supplied with the lock. Without the key, a creature can pick this lock with a successful DC 17 Disable Device check. The Storyteller may decide that better quality locks are available for higher prices.

%\begin{center}
%\includegraphics[width=0.6\linewidth]{immagini/serratura.png}
%\end{center}

\textbf{Sacred Symbol}. A sacred symbol is the depiction of a Patron. It could be an amulet depicting the symbol of a Patron, the same symbol carefully engraved or woven onto an emblem or shield, or a tiny box containing a sacred relic.

\textbf{Earplugs} 3 cp, made of cotton or waxed cork, earplugs grant a +2 bonus on saving throws against effects that require hearing but inflict a –4 penalty on Awareness checks that rely on hearing .

\textbf{Tent}. A simple portable canvas shelter, a tent can hold two people. It takes about 20 minutes to pitch a tent.

\textbf{Tome of Magic}. A 10-page Tome of Magic, meaning it can contain 10 spell levels, costs 100gp. Usually if the caster comes from a Magic Academy or a coven of Devotees he can purchase it at half the price.\index{Tome of Magic, buy}

\textbf{Torch}. A torch burns for \textbf{1 hour of real game time}, providing light in a 3 meter radius and dim light for a further 6 metres. If you make an attack roll with a lit torch, improvised weapon, and hit, you deal 14d damage plus 1 additional fire damage. \index{Torch}

\textbf{Hunting Trap}. 12 gp, 2. You use two actions to set this trap, made of a serrated steel ring, which springs when a creature steps on the metal plate in its center. The trap is attached by a heavy chain to an immovable object, such as a tree or a spike stuck in the ground. A creature that steps on the plate must succeed on a DC 15 Reflex save or take 1d4 piercing damage and stop moving. A creature can use 2 actions to succeed on a DC 15 Strength check, and if it succeeds, it frees itself or frees another creature within reach. Each failed attempt deals 1 piercing damage to the trapped creature.

\begin{center}
\includegraphics[width=0.4\linewidth]{immagini/tribolo.png}
\end{center}


\textbf{Tribolo}. As an action, you can scatter a single bag of these tiny caltrops to cover a 3-foot square area. A creature that passes through the covered area must succeed on a DC 15 Reflex save or take 1 piercing damage. Until the creature regains at least 1 hit point, its walking speed is decreased by 10 feet. A creature that passes through the area at half speed doesn't have to make the roll

\textbf{Basic Poison}. You can use the poison in this vial to cover a slashing or piercing weapon or up to three pieces of ammunition. Applying the poison requires an action. A creature struck by a poisoned weapon or ammunition must succeed on a DC 12 Fortitude save or take 1d4 points of poison damage.
Once applied, the poison remains effective for 1 minute before drying.

\subsubsection{Basic equipment}
If the character chooses to purchase her starting equipment, she can purchase an outfit at the indicated price, which is generally cheaper than purchasing the individual items separately.

\textbf{Adventurer's Gear (18 gp)}. Includes a backpack, a crowbar, a hammer, 10 pitons, 10 torches, a steel and flint, 10 daily rations and a waterskin. The package also includes 15 meters of hemp rope tied to the backpack.

\textbf{Hunter's Equipment (24 gp)}: contains steel and flint, a belt pouch, an 18m rope, a bed, an oilskin, a waterskin, an iron pot, travel rations (5 days), flashlights (10) and a backpack.

\textbf{Diplomat's Gear (57 gp)}. Includes a chest, 2 cases for maps and scrolls, a fine robe, a bottle of ink, a nib, a lantern, 2 flasks of oil, 5 sheets of paper, a vial of perfume, sealing wax and soap.


\textbf{Devotee's Equipment (30 gp)}: contains steel and flint, a belt pouch, a Spell Component Pouch, candles (10), 60' rope, a bedroll, an iron pot, a waterskin, travel rations (for 5 days), soap, a wooden holy symbol, a cheap holy text, torches (10), and a backpack.

\textbf{Explorer's Gear (15 gp)}. Includes a backpack, a bed, a mess tin, a steel and flint, 10 torches, 10 daily rations and a waterskin. The package also includes 15 meters of hemp rope tied to the backpack.

\textbf{Cave Explorer Gear (24 gp)}: Contains a set of basic tools for exploring ruins and abandoned cities includes 2 candles, a chalk, a hammer and 4 Rock Pitons, 60 feet of rope, a lantern shieldable with 5 ampoules of oil, 2 bags, 2 torches, travel rations (for 3 days)

\textbf{Entertainer's Equipment (60 gp}). Includes a backpack, a bed, 2 costumes, 5 candles, 5 daily rations, a waterskin and camouflage tricks.

\textbf{Lock Locker's Equipment (24 gp)}. Includes a backpack, a bag with 1000 metallic spheres, 3 meters of string, a bell, 5 candles, a crowbar, a hammer, 10 rock climbing pitons, a shieldable lantern, 2 oil bottles, 5 daily rations, a steel and flint and a wineskin. The package also includes 15 meters of hemp rope tied to the backpack.

\textbf{Scholar's Endowment (60 gp)}. Includes a backpack, a study book, an ink bottle, a nib, 10 sheets of parchment, a sandbag and a small knife.


\end{multicols}

\subsubsection{Container Capacity}

\begin{tabularx}{0.95\textwidth}{lXl|lXl}
\textbf{Object}&\textbf{Capacity}&\textbf{CoC}&\textbf{Object}&\textbf{Capacity}&\textbf{CoC}\\
\toprule
Bag&1 cube with 30 cm edge/3 kg of equipment&1&Barrel&160 liters of liquid, 4 cubes with 30 cm edge&35\\
Mug&0.5 liter&L&Bottle&1 liter of liquid&L\\
Bucket&12 liters of liquid, 1 cube with 25 cm edge&3&Basket&2 cubes with 30 cm edge/20 kg of equipment&5\\
Bag&1 cube with 30 cm edge/15 kg of equipment&3&Chest&12 cubes with 30 cm edge/150 kg of equipment&35\\
Vial&120 ml of liquids&L&Skin&2 liters of liquids&1\\
Carafe&4 liters of liquid&2&Backpack&2 cubes with 30 cm edge/30 kg of equipment&6\\
\end{tabularx}


\medskip

There is also the version of the Perfect Backpack (+100 gp) which grants a +1 to the transportable bulk value.

\begin{multicols}{2}

\subsubsection{Tools}

The list of tools presented helps the characters perform tests related to their professions.

Profession-related checks are usually related to Wisdom.

For example, a check worth \emph{Calligraphy} is resolved with a Wisdom check, if the character has the appropriate tools available (\emph{Calligrapher's Supplies}) he gets a bonus of +2 on the check.

If the character has to make a test on her profession, this will be done with a bonus equal to half the character's level, if she also has the tools available, the test will receive a further bonus of +2.

\end{multicols}

\medskip

\begin{tabularx}{0.95\textwidth}{llX|llX}
\textbf{Object}&\textbf{Cost}&\textbf{Eng.}&\textbf{Object}&\textbf{Cost}&\textbf{Eng.}\\
\toprule
Burglary/Forger's Tools&25 gp&1&Herbalist's Bag&5 gp&1\\
Dice&1 sp&-&Deck of Cards&5 sp&-\\
Dragon Chess&1 gp&1&Three Dragons in the Dark&1 gp&-\\
Poisoner's Substances&50 gp&1&Alchemist's Supplies&50 gp&2\\
Calligrapher's Supplies&10 gp&1&Pourer's Supplies&20 gp&2\\
Shoemaker's Tools&5 gp&2&Cartographer's Tools&15 gp&2\\
Leatherworker's Tools&5 gp&2&Builder's Tools&10 gp&2\\
Blacksmith's Tools&20 gp&3&Carpenter's Tools&8 gp&2\\
Jeweler's Tools&25 gp&1&Carver's Tools&1 gp&2\\
Inventor's Tools&50 gp&2&Painter's Tools&10 gp&1\\
Blower's Tools&30 gp&2&Weaver's Tools&1 gp&2\\
Potter's Tools&10 gp&2&Cooking Utensils&1 gp&2\\
Navigator Tools&25 gp&2&Ciaramella&2 gp&1\\
Bagpipes&30 gp&1&Horn&3 gp&L\\
Dulcimer&25 mo&2&Flute&2 mo&0L\\
Pan flute&12 mo&L&Lyre&30 mo&L\\
Lute&35 gp&1&Drum&6 gp&1\\
Purple&30 gp&1&Camouflage Tricks&25 gp&1\\
\end{tabularx}

\begin{multicols}{2}

\subsubsection{Mounts and Vehicles}

A good mount can allow a character to quickly traverse wilderness, but its primary purpose is to carry equipment that would otherwise slow its master.


The table \emph{Mounts and Other Animals} indicates the costs of transport animals, for information on the movement and transport capacity of each animal see information in the chapter \hyperlink{table-mounts-and-vehicles}{Table Mounts and Vehicles} (page \pageref{table-mounts-and-vehicles}).

In fantasy worlds there are other mounts besides those listed in this section, but these are rare mounts that are not normally available for purchase, such as certain flying mounts (pegasi, griffins, hippogriffs and other similar animals) or even some mounts aquatic (such as giant seahorses).

To obtain such a mount you often need to steal an egg and raise the creature yourself, make a pact with a powerful entity, or negotiate with the mount itself.

\textbf{Harding}. A harness is armor designed to protect an animal's head, neck, chest, and body. Each type of armor listed in the Armor table in this chapter can be purchased as barding. It costs four times as much as the equivalent armor made for humanoids, and weighs twice as much.


\textbf{Saddle}. A rider can attach to a military saddle to remain in place of him on an active mount during battle. A military saddle grants +1d6 on checks the character makes to remain in the saddle. An exotic saddle is required to ride an aquatic or flying creature.


\begin{center}
\includegraphics[width=0.7\linewidth]{immagini/bardatura.png}

\emph{Full harness}
\end{center}

\begin{center}
\includegraphics[height=0.7\linewidth]{immagini/sella2.png}
\end{center}

\textbf{Rowing Boats}. Barges and rowboats are usually used on lakes and rivers. If a vessel follows the current, the speed of the current (usually 4.5 km per hour) is added to its speed. Generally it is not possible to row against the current if the current has a significant intensity, but it is possible to bring these boats up a watercourse by bringing them to the shore and having them towed by one or more beasts of burden. A rowboat weighs 100 pounds (Encumbrance 10) if adventurers must transport it by land.

\medskip

\textbf{Table: Mounts and Other Animals}\index[Tables]{Table Mounts and Other Animals}

\begin{tabularx}{0.42\textwidth}{ll}
\toprule
\textbf{Mount}&\textbf{Cost}\\
Donkey or Mule&8 gp\\
Camel&50 gp\\
Elephant&200 gp\\
Mastino&25 mo\\
Galloping Horse&75 gp\\
War Horse&400 gp\\
Draft Horse&50 gp\\ 
Pony&30 mo\\ 
\end{tabularx}

\medskip

\textbf{Harness and Shooting Vehicles}\\
\begin{tabularx}{0.45\textwidth}{llX}
\toprule
\textbf{Item}&\textbf{Cost}&\textbf{Weight}\\
Harness&x4&x2\\
Chariot&250 gp&50 kg\\
Saddlebags&4 gp&4 kg\\
Cart&15 gp&100 kg\\
Wagon&35 gp&200 kg\\
Carriage&100 gp&300 kg\\
Bit and Bridle&2 gp&0.5 kg\\
Nutrition (per day)&5 mr&15 kg\\
\end{tabularx}

\bigskip

\textbf{Saddle}\\
\begin{tabularx}{0.45\textwidth}{llX}
\toprule
\textbf{Item}&\textbf{Cost}&\textbf{Weight}\\
From Cargo&5 gp&7.5 kg\\
From Gallop&10 gp&12.5 kg\\
Exotic&60 gp&20 kg\\
Military&20 gp&15 kg\\
Sled&20 gp&150 kg\\
Stabling (per day)&1 Tue&\\
\end{tabularx}

\bigskip

\textbf{Boats}\\
\begin{tabularx}{0.45\textwidth}{llX}
\toprule
\textbf{Item}&\textbf{Cost}&\textbf{Speed}\\
Rowing boat&50 mo&2.25 km per hour\\
Barge&3000 mo&1.5 km per hour\\
Galley&30000 gp&6 km per hour\\
Sailing ship&10000 mo&3 km per hour\\
Warship&25,000 gp&3.75 km per hour\\
Long Ship&10000 gp&4.5 km per hour\\
\end{tabularx}

\subsubsection{Services}


Adventurers can pay non-player characters to help them or act on their behalf in the most diverse circumstances. Most of these cohorts have more than ordinary skills, while others have mastered an art or craft, and some have specialized in some adventuring skill.

Other common cohorts include the many inhabitants of a typical town or city whom adventurers can hire to carry out a specific task. For example, a spellcaster might pay a carpenter to build a fine chest (and its miniature replica) to use for a spell.
A warrior might commission a blacksmith to forge a special sword.

\medskip

\textbf{Services}

\bigskip

\begin{tabularx}{0.45\textwidth}{ll}
\textbf{Service}&\textbf{Cost}\\
\toprule
Carriage inside a city&5 mr/1 km\\
Carriage between two towns&1 m/1 km\\
Able Gregorian&2 gp per day\\
Inexperienced Wingman&5 but per day\\
Messenger&5 mr/1.5 km\\
Ship passage&1 m/1.5 km\\
Entrance toll&5 mr/5 ma\\
\end{tabularx}


\subsubsection{Magic Services}

\textbf{Spell Level x Spell Level ×100 gp}

This is the cost of having a spellcaster manipulate magic, plus any spell components. This cost assumes that you can go to the caster and ask him to perform a certain magic as you wish (usually it takes him at least 8 hours to prepare). If you want to take the caster somewhere to use magic you need to negotiate with him, and the basic answer is \emph{no}.

If the spell has dangerous consequences, the caster must receive certain proof that the character has the ability to pay and that he will not fail to do so if these consequences occur (provided he agrees to cast the required spell, which is not the case). not at all safe). When it comes to spells that transport the character and the caster across a distance, you must pay for the spell twice even if the character does not wish to travel back with the caster.

\begin{center}
\includegraphics[width=0.8\linewidth]{immagini/riempitivocavalieriapranzo.png}
\end{center}

Not every village and town has a spellcaster capable enough to manipulate magic. As a general rule, you need to move to at least a small town to be fairly sure of finding a charmer. In a small town you might find a spellcaster capable of casting spells at level 2, in a large town those at level 3, in a small town for those at level 5, in a large city for those at level 6, in a metropolis for those of level 8. Even in a metropolis you are not sure of finding a spellcaster capable of casting spells with level 9 or more.


\subsubsection{Special Objects and Substances}\index{Special Substances}

\textbf{Antiemetic} 25 gp, this sweet and savory green liquid creates a sense of warmth and comfort. The syrup protects the stomach and makes it more resistant. For 1 hour after drinking it you gain a +4 bonus on saving throws to resist effects that make you nauseated or against ingestion poisons.

\textbf{Antibiotic} (vial) 50 gp, drinking a vial of this foul-tasting milky white liquid gives you a +4 bonus on saving throws against Diseases made in the next hour. If already infected, you can make two saving throws to resist the disease on that particular day (without the +4 bonus) and keep the best result. Single dose.

\textbf{Antitoxin} (bottle) 50 gp, if you drink the antitoxin, you gain a +4 bonus on all Fortitude saving throws against poisons for 1 hour. Single dose.

\textbf{Smoke Stick} 20, this alchemically treated wooden stick instantly creates thick opaque smoke when ignited. The smoke fills a 10-foot cube, except the smoke is dissipated in 1 round by a moderate or stronger wind. The staff is consumed in 1 round and the smoke then dissipates naturally. All creatures in the affected area have full cover.

\textbf{Alchemist's Coffee} 1 gp, much loved by young people, it is a brown crystalline powder. Mixed with water it creates a bitter drink that cures the effects of a hangover. Single dose. Work DC 15

\textbf{Bag of Impediment} 50 gp, This round leather bag is filled with molasses, resin, or other sticky substance. When you hurl the bag at a creature (as a ranged touch attack with a range of 10 feet), the bag opens and the substance inside entangles and entangles the victim, becoming tough and elastic with exposure to air.

The substance does not affect creatures of Huge size or larger. A flying creature is not stuck to the ground, but must make a DC 15 Reflex save or lose the ability to fly (provided it uses its wings to do so), falling to the ground. The impediment bag does not work underwater.

\textbf{Blood Stopper} 25 gp, this pink, sticky substance helps heal wounds. Using a dose grants a +4 bonus on First Aid checks. 6 Uses.

\textbf{Alkaline Flask} 15 gp, this flask of caustic liquids reacts with the natural acids of the oozes. It is possible to throw an alkaline flask as a splash weapon with a range of 3 meters. Against non-ooze creatures, an alkaline flask functions like an acid flask. Against oozes and other acidic creatures, the alkaline flask deals double the damage indicated by Flask of Acid.

\textbf{Smoke Generator} 25 gp, this small clay sphere contains two alchemical substances separated by a thin barrier. When the sphere breaks, the substances come together and fill an area of ​​melee with a cloud of harmless, blackish smoke. The smoke bomb functions like a smoke staff, but the smoke remains for 1 round before dissipating. It is possible to throw a smoke bomb as a touch attack with a range of 3 meters.

\textbf{Alchemist's Fire} 20 gp, you can throw a flask of alchemist's fire as a splash weapon. Treat the attack as a ranged touch attack, with a range of 10 feet.

The direct hit deals 1d6 fire damage. All creatures within melee range of where the flask fell take 1 fire damage as a result of the splash. In the round following the direct hit, the victim takes an additional 1d6 fire damage. The victim can use 1 Action to try to put out the flames before suffering this additional damage. You must succeed at a DC 15 Reflex save to put out the flames. Using 2 Actions gives the character a +2 bonus on the saving throw. Diving into water or dousing the flames by magical means automatically extinguishes the flames.

\textbf{Plaster for Casts}: 5 but, this dry white powder, mixed with water, thickens within an hour to create a solid material. It can be used to create a cast of a footprint or bas-relief, fill holes or cracks in walls or (if applied to a cloth covering) to fix a broken bone. Hardened chalk has a hardness of 1 and 5 hit points for every 1 inch of thickness. A 5 pound pot of cast can cover a melee radius 1 inch deep, create five casts for the forearm or calf of a Medium-sized creature, or two full casts per arm or leg. Single dose.

\textbf{Liquid Ice} (vial) 40 gp, also called \emph{alchemist's ice}, this crystalline blue fluid begins to evaporate as soon as it is removed from the container. Over the next 1d6 rounds you can use it to freeze a liquid or cover an object with a thin layer of ice. It is also possible to throw liquid ice as a splash weapon. A direct hit deals 1d6 points of cold damage, while creatures within melee range take 1 point of cold damage from the splash. The package contains 3 doses.

\textbf{Alchemical Fat} 5 gp, each pot of this blackish substance can cover one Medium or two Small creatures. Covering yourself in alchemical fat gives you a +4 bonus on grapple checks and when escaping grapples. The effect lasts 4 hours or until the fat is washed away.

\textbf{Detect Light} 1 gp, this hand-sized metal plate is covered in a transparent, light-sensitive cream. When exposed to light, the cream darkens and becomes opaque depending on how much light is present. Bright light causes it to darken in 1 round, normal light in 3 rounds, and dim light in 10 rounds.
The plate is sold wrapped in heavy cloth to avoid accidental exposure.

\textbf{Advanced Spot Light} 50 gp, this metal plate similar to the Spot Light plate is approximately 50cm*50cm in size. If exposed to light, it imprints on it the image of the surrounding environment within 3 metres.

\textbf{Thunderstone} 30 gp, you can hurl this stone with a ranged attack with a range of 20 feet. When it hits a hard surface (or is hit with force), it creates a deafening noise that is equivalent to a sonic attack. Creatures within 10 feet must make a DC 15 Fortitude save or be deafened for 1 hour. Single use.

\textbf{Lightning Powder} 50 gp, this silvery powder burns and explodes almost instantly if exposed to fire, rubbing it or throwing it forcefully against a surface (1 Action). Creatures within 10 feet are blinded for 1 round (Fortitude DC 13 negates). The package contains 3 doses.

\textbf{Blade Protector} 40 gp, this transparent resin protects a weapon from attacks from Slimes, Rust Eaters and effects that corrode or melt weapons, making it immune to such attacks for 24 hours. One jar can cover one two-handed weapon, two one-handed or light weapons, or 50 rounds of ammunition. Applying it requires 2 Actions. The package contains 3 doses.

\textbf{Universal Solvent} (vial) 20 gp, this bubbling purple jelly devours stickers. Each vial can cover one melee range. It destroys normal adhesives (such as pitch, resin, or glue) in 1 round, but requires 1d4+1 rounds to dispel more powerful adhesives (bags of impediment, cobwebs, etc.). It has no effect on magic stickers.

\textbf{Fiery Brand} 1 gp, the alchemical substance at the tip of this small wooden staff ignites when rubbed against a rough surface. Creating a flame with a firebrand is much quicker than creating one with steel, flint (or magnifying glass), and tinder. Lighting a torch with a firebrand costs 1 Action (rather than 2 Actions), and lighting any other fire requires at least 3 Actions.

\subsubsection{Alchemical Equipment}

\textbf{Reagent Paper} 1 gp, this piece of paper can help identify liquids. Its color changes depending on traits such as acidity, salinity and magic. Consuming a sheet grants a +2 bonus on Job (alchemy) or Arcana checks to identify potions or other liquids.

\textbf{Explosive Ink} (Vial) 40 gp, This alchemically infused ink helps ensure that a secret message is destroyed after being read. If light hits the ink after it dries, the chemicals cause it to spontaneously combust within 1 minute
This combustion is small: it is not significant enough to set fire to anything other than paper. Ink used on other materials such as stone or wood simply fades away, leaving no trace of the writing
One vial of this ink contains enough to write 10 short messages of no more than 50 words each.

\textbf{Luthiers' Oil} 50 gp, this golden oil smells of ancient wood. When applied to the case of a wooden musical instrument it improves the quality of the sound. For 1 hour, anyone who plays the instrument gains a +2 bonus on the appropriate Perform check.

\textbf{Nightingale Lozenge} 50 gp, this honey-coated candy is made of calming reagents. If eaten, it takes 1 round to begin to take effect, after which it grants a +2 bonus on Perform (singing) checks for 1 hour.

\textbf{Waystones} 50 gp, these small white pebbles are alchemically treated so that they give off a soft light when activated by rubbing them against each other. The luminescence is dim, just enough to illuminate the stone. The duration is 8 hours.

\textbf{Tracking Powder} 30 gp, when scattered on the ground, this very fine light blue powder reveals the traces of any creature or individual that has passed through the area in the last 48 hours.
The dust also provides a +8 bonus on Survival checks to locate tracks. A single application can cover an area of ​​3 meters. The tracer powder is sold in small leather bags that contain 10 applications each.

\subsubsection{Alchemical Remedies}\index{Alchemical Remedies}

\label{rimedi-alchemici}

\textbf{Carbonated Help} 25 gp, this package is full of spiny-edged leaves and has a pungent odor almost strong enough to make your eyes water. While chewing the leaves, you ignore the effects of being fatigued. The leaves last for 10 rounds, after which only a pile of mush remains.
When the effect of the carbonated aid wears off, you increase your fatigue level by 1 rank. One package is enough for 1 time only.

\textbf{Anti-Poison Balm} 15 gp, this herbal balm can be applied directly to the skin to prevent the effects of Poisons on contact. If a creature touches poison by touch, but applies the balm to itself within 1 round of contact, it makes the saving throw twice and takes the higher result. Single use.

\textbf{Coagulating Balm} 5 but, applying this herbal balm to a wound cures 1 damage, it is not possible to use more than two doses per day on the same patient. The package is for 3 uses.

\textbf{Fortifying Bitter} 20 gp, this alcoholic liquid generates a pleasant sensation of heat when ingested. For the next hour, you gain a +2 bonus on saving throws against fear. Using multiple doses within the same 24 hours makes you nauseated for 1 hour. The package is for 3 uses.

\begin{center}
\includegraphics[width=0.7\linewidth]{immagini/zaino.png}
\end{center}

\subsubsection{The Standard Backpack\texorpdfstring{\huge{\textregistered}}{\textregistered}} \index{Standard Backpack}

The Standard Backpack €11404 €11405 € is a list of objects that I have marked over time, adding everything that I needed during my adventures.
Take this as a starting point to understand what objects to have behind you, don't write them all down otherwise the Narrator will start seriously looking at the Encumbrance rules!

This is the contents of the adventurer's backpack: belt, 3 candles, 6 torches, tinder and steel, 7 dry rations, water flask, rolled mattress, tarpaulin, tent, 18 meters of rope, net, metal mirror, crowbar , compass, 3 lantern oil, ink, chalk, charcoal, hook, spade, fish hook, rags, 2m metal cable, whistle, 6 empty potion vials, marble marbles, brass bell, 1kg of flour in a bag , 3 wedges, 12 meter metal chain, 2 handcuffs, 8 rock climbing pitons, hammer, pulley, grappling hook.

\begin{center}
\includegraphics[width=0.8\linewidth]{immagini/mercante.png}
\end{center}

\subsection{Expenses and Lifestyle}\index{Expenses and Lifestyle}

When not descending into the bowels of the earth, exploring ruins for lost treasure, or waging war against the forces of looming darkness, adventurers must also think of the most common needs. Even in a fantastic way, people must satisfy basic needs such as food, shelter and clothing. All this has a cost, even if certain lifestyles cost more than others.\\

Lifestyle expenses are an easy way to account for the costs of living in a fantasy world. They cover a character's accommodation, food, drink, and all other essential needs. These expenses also cover the cost of maintaining your character's equipment, so he or she will be ready when the next call to adventure comes. At the start of each week or month (player's choice), each character chooses a lifestyle from the Lifestyle Expenses table and pays the price required to maintain that lifestyle. The prices listed are daily, so those who wish to calculate their cost of subsistence for a period of thirty days will have to multiply the price indicated by 30. A character can change his lifestyle from one period to another, based on the funds available his disposition, or he can maintain the same lifestyle throughout his career.

Your lifestyle choice can have consequences. A character who maintains a wealthy lifestyle can more easily make contacts with the rich and powerful, but runs the risk of attracting a few thieves. Similarly, a poor lifestyle may help him avoid criminals, but it is unlikely to allow him to make important contacts.\\


\textbf{Lifestyle Expenses}

\medskip

\begin{tabular}{ll}
\textbf{Lifestyle}&\textbf{Price per day}\\
\toprule
Miserable&-\\
Shabby&1 but\\
Poor&2 but\\
Modesto&1 mo\\
Well-off&2 mo\\
Rich&4 mo\\
Aristocrat&Minimum 10 gp\\
\end{tabular}\\


\textbf{Miserable}. The character lives in inhumane conditions. He has no place to call home and takes refuge where he can, sneaking into a barn, huddling in an old box or relying on the good hearts of those luckier than him. A miserable lifestyle presents dangers aplenty. Violence, disease and hunger follow the character wherever he goes. The other wretches might set their sights on his armor, weapons, and adventuring gear, which are a fortune by their standards. Most people don't take the character into consideration at all.

\textbf{Shabby}. The character lives in a drafty stable, a mud-floored hut just outside the village, or in a flea-filled hostel in the worst part of town. He benefits from minimal shelter from the elements, but lives in a desperate and often violent environment, in places plagued by disease, hunger and misfortune. Most people don't take it into consideration at all and the law protects it little or nothing. Most people who lead this lifestyle are marked by some terrible misfortune: branded as exiles, suffering from a mental disorder or illness of some kind.

\textbf{Poor}. A poor lifestyle means having to get by without the amenities available in a stable community. Basic food and shelter, poor quality clothing and unpredictable living conditions result in a lifestyle that is perhaps sufficient to survive, but certainly not very pleasant. The character sleeps in a hostel or in a common room on the first floor of a tavern. He benefits from a minimum of legal protection, but he still has to deal with acts of violence, crime and disease. Unskilled labourers, junk dealers, beggars, thieves, mercenaries and other unsavory figures tend to adopt this lifestyle.

%\begin{center}
%\includegraphics[width=0.7\linewidth]{immagini/mendicante.png}
%\emph{Beggar - Francesco Londonio}
%\end{center}

\textbf{Modest}. A modest lifestyle keeps a character out of the slums and allows him to take care of his equipment. The character lives in an old part of the city, has a rented room in a boarding house, an inn or a temple. He does not suffer from hunger or thirst and lives in a clean, even if Spartan, environment. Ordinary individuals who lead a modest lifestyle include soldiers with families, laborers, students, priests, amateur spellcasters, and so on.

\textbf{Wealthy}. A character capable of adopting a wealthy lifestyle can afford quality clothing and take care of his equipment without difficulty. He lives in a house on a well-known block or has a private room at a quality inn. He associates with merchants, skilled craftsmen and military officers.


\begin{center}
\includegraphics[width=0.6\linewidth]{immagini/lucullo.png}

\emph{Lucius Licinius Lucullus. Rome, 117 BC, Naples 56 BC). Roman military and politician}
\end{center}

\textbf{Rich}. A character who adopts a wealthy lifestyle lives in luxury, even though he may not have achieved the social prestige associated with the old values ​​of nobility and royal blood. He leads a lifestyle comparable to that of a highly successful merchant, an esteemed servant of a royal house, or the owner of a few small businesses. He stays in a respectable home, usually a spacious house in a respectable part of town or a comfortable apartment at a well-known inn. He is probably assisted by a small group of servants.

\textbf{Aristocrat}. The character lives comfortably and abundantly and frequents environments populated by the most powerful figures in the community. He has an excellent home, perhaps a house in the most elegant neighborhood of the city or perhaps a series of rooms in the most renowned inn. He dines at the best restaurants, serves himself at the most skilled and fashionable tailors and can count on various servants who take care of his every need. He receives invitations to the social events of the rich and powerful and spends his evenings in the company of politicians, guild leaders, high priests and nobles. He must also contend with the deceptions and betrayals perpetrated at the highest levels. The greater his wealth, the greater the chance that he will be drawn into some political intrigue, sometimes as a pawn, sometimes as an active participant.

\end{multicols}

%\vfill

%\begin{center}
%\includegraphics[width=0.6\linewidth]{immagini/carrozza.png}
%\end{center}


\pagebreak

\subsection{Special Materials}\index{Special Materials}

\begin{changemargin}{0.3cm}{0.3cm}\begin{enfasi}{
For this purpose, Captain De Medici had all the armor burnished, to surprise the enemy even in the dark. (The profession of weapons, Ermanno Olmi, film 2001)}\end{enfasi}\end{changemargin}\medskip

\begin{multicols}{2}

Armor and weapons can be constructed from materials that possess innate special qualities. If you craft armor or a weapon with more than one special material, you receive the benefits of only the prevailing material. However, you can build a double weapon with each head made of a different special material.

\subsubsection{Living Steel}\index{Living Steel}\index[Tables]{Living Steel Cost Table}

\label{acciaio-vivente}

\begin{tabularx}{0.45\textwidth}{Xl}
\textbf{Living Steel item type} & \textbf{Cost modifier}\\
\toprule
Ammo & +40 gp per ammo\\
Weapon & +1000 gp\\
Light armor & +3000 gp\\
Medium armor & +8000 gp\\
Heavy armor & +12000 gp\\
Shield & +600 gp\\
Other items & 3000 gp/kg\\
\end{tabularx}

\medskip
A living steel tree is characterized by particularly hard wood like steel. The origin of these trees remains a mystery to almost everyone. A living steel tree is an ordinary tree planted by a Devotee of Ephrem or Shayalia and given a blessing.

Livingsteel armor and shields are formally made of wood but have the same characteristics as adamantium. This particular wood is the favorite of those who fight and live for nature. It is not easy to identify a living steel tree for a non-expert and for this reason it is extremely rare to find it raw, at most it is possible to find already made weapons or armor.

Living steel has 35 hit points per inch of thickness and hardness 15.

\subsubsection{Adamantium}\index{Adamantium}\index[Tables]{Adamantium Cost Table}

\label{adamantio}

\begin{tabularx}{0.45\textwidth}{Xl}
\textbf{Adamantium item type} & \textbf{Cost modifier}\\
\toprule
Ammo & +60 gp per ammo\\
Weapon & +1500 gp\\
Light Armor & +5000 gp\\
Medium armor & +10000 gp\\
Heavy armor & +15000 gp\\
Shield & +1000 gp\\
Other items & 5000 gp/kg\\
\end{tabularx}

\medskip
This extremely hard metal is found only in meteorites and contributes to the quality of a weapon or armor.

Therefore adamantium weapons and ammunition have a +1 bonus on attack rolls, and the penalty given by the armor (Skill Penalties and Magic Checks) is reduced by 1 (or one die) compared to normal armor of its size. same kind. Objects without metallic parts cannot be constructed with adamantium. An arrow may be made of adamantium, but a quarterstaff is not.

Weapons and armor normally made of steel and constructed of adamantium have one-third more hit points than normal. Adamantium has 40 Hit Points per 1 inch of thickness and Hardness 20.

\subsubsection{Alchemical Silver}\index{Alchemical Silver}\index[Tables]{Silver Weapon Cost Table}

\label{argento-alchemico}

\begin{tabularx}{0.45\textwidth}{Xl}
\textbf{Alchemical Silver Item Type} & \textbf{Cost Modifier}\\
\toprule
Ammo & +2 gp per ammo\\
Light weapon & +20 gp\\
Medium weapon & +90 gp\\
Heavy weapon & +180 gp\\
Shield & +100 gp\\
\end{tabularx}

The alchemical silvering process can only be applied to metallic weapons and does not work on special metals such as adamantium, cold iron and mithral.

A complex process involving metallurgy and alchemy can bind silver to a weapon made of steel so that it bypasses the damage reduction of creatures such as werewolves.

An alchemical silver weapon retains the hardness and hit points of the original weapon.

\subsubsection{Cold Iron}\index{Cold Iron}

\label{ferro-freddo}

This iron is mined deep underground and is known for its effectiveness against demons and goblins. It is forged at a lower temperature to retain its delicate properties. Crafting weapons made of cold iron costs twice as much as their normal counterparts. Additionally, any magical enhancements cost an additional 2,000 gp. This boost is applied the first time the item is buffed, not once per added quality.

Items without metal parts cannot be made of cold iron. An arrow might be made of cold iron but a club is not (unless it is all metal). A dual weapon that is only half cold iron increases its cost by 50\%.

The cold iron has 30 hit points per 1 inch of thickness and hardness 10.


\subsubsection{Mithral}\index{Mithral}\index[Tables]{Mithral Weapon Cost Table}

\label{mithral}

\begin{tabularx}{0.45\textwidth}{Xl}
\textbf{Item type in Mithral} & \textbf{Cost modifier}\\
\toprule
Light armor & +1000 gp\\
Medium armor & +4000 gp\\
Heavy armor & +9000 gp\\
Shield & +1000 gp\\
Other items & +1000 gp/kg\\
\end{tabularx}

\bigskip

\begin{center}
\includegraphics[width=0.9\linewidth]{immagini/mithral.png}
\end{center}


Mithral is a very rare, shiny, silver-like metal that is lighter than iron but just as hard. When processed like steel, it makes a wonderful material from which to create armor, and is occasionally used for other items as well. Most mithral armor is one category lighter than normal, and is easier on movement and other limitations. Heavy armor is treated as medium armor, and medium armor is treated as light, but light armor remains light.

This decrease does not apply to the proficiency required to wear the armor in question (to wear heavy mithral armor requires Weapon Proficiency 3, although this is considered average due to other factors). You must be proficient in the appropriate type of armor, otherwise you incur the relevant penalties as normal.

The Magic Tests for mithral armor and shields decrease by 2 dice (while the test is still necessary) and the penalty on proficiency checks decrease by 2 (to a minimum of 0), the movement penalties decrease by 1 meter.

The mithral has 30 hit points for every 1 inch of thickness and hardness 15.

\subsubsection{Dragon Skin}\index{Dragon Skin}

\label{pelle-di-drago}

Armorsmiths can craft dragon hides to make armor or shields.
A dragon provides enough hide for a single hide armor for a creature one size smaller than the dragon. By selecting only the finest scales and hides, an armorsmith can produce band mail for a creature two sizes smaller, half armor for a creature three sizes smaller, and plate or full armor. for a creature four sizes too small.

\begin{center}
\includegraphics[width=0.9\linewidth]{immagini/dragonhide.png}
\end{center}


Dragon skin armor or shield cannot be bought, it is always necessary to bring the raw materials to the craftsman who will take care of building the armor.

However, there is always enough hide to produce a light or heavy shield in addition to the armor, as long as the dragon is Large or larger.
If the dragon skin comes from a Dragon that has immunity to an energy type, the armor is also immune to that energy type, although it does not grant the wearer any protection. If the shield or armor is later granted the ability to protect the wearer from a specific type of energy, the cost of this upgrade is reduced by 25\%.

A Dragonhide Armor forces you to make a Magic Test without additional dice when you cast a spell, the Skill penalty decreases by 1 (to a minimum of 0), movement penalties decrease by 1 meter.

Dragonhide armor costs 10 times as much armor of that type, but does not take longer to craft. Magical or medium or heavy type armor must be found.

Dragon hide has 10 hit points per 1 inch of thickness and hardness 10. Dragon hide is typically 0.5 to 1 inch thick.

\end{multicols}

\pagebreak

\section{Break Through and Enter}\index{Break Through}\index{Enter}

\begin{changemargin}{0.3cm}{0.3cm}\begin{enfasi}{
Sooner or later a day comes in a man's life when, to go where he has to go, if there are no doors or windows, he has to break through the wall. (Bernard Malamud)\\

The crime of theft will be punished by branding the thieves in the chest. In case of repetition of the crime, first the ears will be cut off and then two fingers of the hands. (Twoslad, Citizen Rights and Duties)

}\end{enfasi}\end{changemargin}\medskip

\begin{multicols}{2}

\label{sfondare-ed-entrare}

\lettrine[lines=2, lhang=0.33, loversize=0.25, findent=1.5em]{Q}{when} you try to break an object there are two choices: hit it with an object (weapon?) or break it with brute force.

\smallskip

\subsection{Size matters...}

Depending on the size of the object this can be more or less easy to hit.\\

\textbf{Table: Size and Defense of Objects - Hitting an Object}\index[Tables]{Table Size and Defense of Objects - Hitting an Object}

\medskip

\begin{tabularx}{0.43\textwidth}{llll}
\textbf{Size} & \textbf{Mod. Defense} & \textbf{Dimensions}\\
\toprule
Colossal & -8 &18m+\\
Mammoth & -6 &9-18m\\
Huge & -4 &4-9m\\
Large & -2 &2.4-4m\\
Average & +0 &1.2-2.4m\\
Small & +2 &60-120cm\\
Lowercase & +4 &30-60cm\\
Petite & +6 &15-30cm\\
Very small & +8 &5-20cm\\
\end{tabularx}

\medskip
\textbf{Defense Modifier}

Objects are easier to hit than creatures since they usually don't move, but many are tough enough to ignore damage with each hit. An object's Defense is equal to 10 + its Size modifier (see Table: Hitting an Object) + its Dexterity modifier (if it has one).

If you use 3 Actions to aim, you automatically hit with a melee weapon.

\subsection{Hardness}

The following table shows materials and objects with relative Hardness, Hit Points and DC to break or break through\index{Break through}\index{Break through}

When you try to break or break through something with brute force rather than by inflicting damage you must make a Strength check (Fortitude + Strength) to see if you succeed.
Since Hardness does not affect the DC to break the item, this value depends more on how the item is constructed than on the material. The DC indicated is for common objects, a glass 20 cm thick will not have DC 6 to break.

See also \hyperlink{tabledoors}{Table: Doors}, page. \pageref{door table}

\end{multicols}


\textbf{Table: Hardness and Hit Points of objects}\index[Tables]{Table of Hardness and Hit Points of objects}

\begin{tabularx}{0.95\textwidth}{llllll}
\toprule{}
\textbf{Material} & \textbf{Hardness}& \textbf{PF} & \textbf{DC} & \textbf{Example Objects}\\
Paper, Glass, Cloth & 0 & 1 & 3 & Sheets of paper, window glass, light fabric\\
Heavy Cloth & 1 & 4 & 12 & Cloth Armor, Heavy Jacket, Sack, Tent\\
Glass & 1 & 4 & 6 & Glass block, glass table, heavy vase\\
Rope, Leather & 2 & 4 & 19 & Hemp Rope\\
Thin Wood & 3 & 12 & 14 & Chair\\
Leather Armor & 4 & 16 & 22 & Leather Armor, saddle, thick hemp rope\\
Thin stone & 4 & 16 & 20 & Slate, slate tiles, stone cladding\\
Steel or thin iron &5 & 20 & 23 & Silk rope, steel shield, short sword\\
Wood & 5 & 20 & 18 & Chest, table\\
Stone & 7 & 28 & 35 & Paving stone, statue\\
Steel or Iron & 9 & 36 & 26 & Chain, Steel Armor, Iron, Long Sword\\
Wooden frame & 10 & 40 & 20 & Wooden wall\\
Stone Structure & 14 & 56 & 35 & Stone Wall\\
Steel or iron structure&18&90&45&Iron plate wall\\
\end{tabularx}

\begin{multicols}{2}

\subsection{Damaging objects}

\textbf{Energy Attacks}: almost all objects have Damage Resistance towards energy attacks (fire, electricity..), divide the damage by 2 before applying Hardness. Certain types of energy may be particularly effective against certain objects at the Storyteller's discretion.

For example, fire might deal double damage to scrolls, cloth, and other items that burn easily. Crystal or ceramic objects and creatures may take double damage (vulnerability) against a sonic attack.

Negative or Positive Energy does not damage objects, only living or non-living creatures.

\textbf{Ineffective Weapons}: Certain weapons simply cannot deal damage to certain objects. For example, a blunt weapon cannot cut a rope.
Likewise, it is decidedly difficult to break down a door or stone wall with most melee weapons unless they are specifically designed to do so, such as pickaxes and hammers.

\textbf{Immunity}: Inanimate objects are immune to nonlethal damage and critical hits (but not burst damage). Animated objects, if not treated as creatures, also have these immunities.\index{Object critical immunity}

\textbf{Damaged Objects}: A damaged object remains fully functional until its Hit Points reach 0, at which point it is considered destroyed. Damaged objects (but not destroyed ones) can be repaired by a Craftsman Profession and some Spells.

\textbf{Saving Throw}: Unattended non-magical objects never save. They are considered to have failed their saving throws if available.

An object held by a character (whether held, touched, or worn) succeeds on the saving throw if the character succeeds.\index{Object saving throw}

\textbf{Magic Items always have Saving Throws}. The bonus on Fortitude, Reflex, or Will saving throws of a Magic Item is equal to 2 + level x2 of the most powerful spell it contains. If the item does not have an enchantment, it is considered a bonus of +4 for every +1 bonus possessed. Kept (worn) Magic Items make a saving throw only if their owner fails his own. If an effect specifically affects the magical item and not the wearer then only the magical item makes the saving throw.

An \textbf{enchanted object}\index{Damaging enchanted object} such as a weapon or armor has its Hardness, Hit Points and DC to break increased by half compared to the non-magical equivalent.

\textbf{Animated objects}: Animated objects count as creatures for determining their Defense and Hit Points (they are not considered inanimate objects).

\medskip

\begin{center}
\includegraphics[width=0.6\linewidth]{immagini/portarinforzata2.png}

\emph{Reinforced door}
\end{center}

\subsection{Size matters to Break Through...}\index{Break Through}\index{Size matters to Break Through...}

\label{sfondare}

Creatures larger or smaller than Medium have a size bonus or penalty on their Strength check to break down a door:

\medskip

\textbf{Table: Strength test modifiers based on your size}\index[Tables]{Table of Strength test modifiers for Breaking down door}

\medskip

\begin{tabular}{ll|ll}
\textbf{Size} & \textbf{Mod.}&\textbf{Size} & \textbf{Mod.}\\
\toprule
Very Small & -16& Large & +4\\
Tiny & -12 &Huge & +8\\
Tiny & -8& Gargantuan & +12\\
Small & -4 & Colossal & +16\\
Normal & +0&&\\
\end{tabular}

\medskip

A \textbf{crowbar}\index{crowbar} or a \textbf{portable battering ram}\index{ram} increase the character's chance of breaking down a door by +1d6.

\end{multicols}

\vfill


\begin{center}
\includegraphics[width=0.6\linewidth]{immagini/kitladro.png}

\emph{Villain's Kit}
\end{center}

\pagebreak

\section{Environment}\index{Environment}

\label{ambiente}
\begin{changemargin}{0.3cm}{0.3cm}\begin{enfasi}{
Nature is not cruel, it is just ruthlessly indifferent. This is one of the hardest lessons a human being has to learn. (Richard Dawkins)

\medskip

The main antidote to a bad environment is, of course, to replace it with a good one. (Robert Baden-Powell)}\end{enfasi}\end{changemargin}\medskip

\begin{multicols}{2}

\lettrine[lines=2, lhang=0.33, loversize=0.25, findent=1.5em]{D}{ai} lifeless deserts to dungeons full of traps, the environment helps to define the world, make it alive, dynamic and rich. Allows you to create an exciting and immersive gaming experience.

\subsection{Environmental Rules}

\label{regole-ambientali}

\subsubsection{Vision and Light}\index{Vision}\index{Light}

\label{sec:visione-e-luce}

In a natural environment, lighting can take on different shades and these shades help to understand how far a creature can see.

The light gradations can be:
\begin{itemize}
\item
\textbf{Darkness}': pitch dark, can be natural or magical
\item
\textbf{Dim/Poor light/Dim light}: little lighting, allows you to recognize silhouettes, slightly darkened\index{Slightly darkened}
\item
\textbf{Light}: intense light, bright, covering, sunny light
\end{itemize}

The lighting sources, or their absence, will determine how much light there is and up to what distance. The Light Sources Table indicates for the most common light sources the fully illuminated beam, the less illuminated one (Dim Light) and the duration.

Many spells and objects use \emph{real game time} as their duration, i.e. the rounds or turns are not counted to establish the duration but rather the time the torch, lantern or spell is turned on is marked on the card. Another method can be to set a timer on your smartphone. In this way, management will be easier and greater attention will be paid to consumable resources.


\begin{changemargin}{0.3cm}{0.3cm}\begin{narratore}The different functioning of the light sources aims to make exploration darker, darker and more difficult, especially of caves and areas without light sources. No more groups casting Light every minute or shouting \emph{Darkvision!}. Darkness helps the imagination and raises the level of tension. Emphasize the crackling of the torch flame, the swaying and sometimes almost dying out due to sudden currents. Make what's around the characters mysterious!.
\end{narratore}\end{changemargin}

\medskip

\textbf{Table: Light sources}\index[Tables]{Table of light sources}

\medskip

\index{Dim Light}

\begin{tabular}{l|cc|c}
\textbf{Source of} &\multicolumn{2}{c}{\textbf{Radius in metres}}& \textbf{Duration} \\
\textbf{Light}& \textbf{Light} & \textbf{Dim Light} &\\
\toprule
Candle & - & 1 meter & 1 hour\\
Flashlight & 3 meters & 6 meters & 1 hour\\
Lantern & 6 meters & 12 meters & 3 hours \\
\multicolumn{4}{c}{\textbf{Spells}}\\
Light & 3 meters & 6 meters &30 min. \\
Daylight & 6 meters & 12 meters & 1 hour \\
\end{tabular}

\smallskip

The indicated duration is expressed as the duration of real playing time.\index{Light Duration}


\medskip

\begin{center}
\includegraphics[width=0.8\linewidth]{immagini/oscurita.png}

\emph{Henry Justice Ford}
\end{center}

\medskip

\textbf{Dim light}\index{Dim light} is the light beyond a light source. It's passing through a 3 meter corridor if it's lit only by light candles, it's a full moon night, it's a slightly darkened area.
Generally speaking, a light source creates dim light in a radius twice the normal light radius. A creature in Dim Light has a -2 on Awareness checks and a -1 on attack rolls.

\medskip

\textbf{Darkness}\index{Darkness}: it is complete darkness without any source of light. To creatures with normal vision, darkness is what lies beyond the dim Light.
The \textbf{blind character}\index{Blind} or who fights in the dark (and cannot see in the dark) has -1d6 to Awareness and all opponents are invisible.

\medskip

\textbf{Light}\index{Light} is the light outdoors under the sun, but also if you hold a torch in your hand or in a corridor lit by lanterns. If the light sources do not follow one another, areas of dim light or darkness are created.

\subsubsection{Types of Vision and Lighting}

\begin{itemize}
\item
A creature with \textbf{Normal Vision} \index{Normal Vision} sees up to the distance, as a circular ray around the light source, indicated in Light. Beyond is Dim Light and beyond is Darkness.

\item
A creature with \textbf{Light Vision} \index{Light Vision} sees without difficulty up to the distance, as a circular ray around the light source, indicated in Dim light, or indicated by the race if smaller, beyond that it is darkness.

\item
A creature with \textbf{Darkvision} \index{Darkvision} sees in darkness as if there were Dim Light up to the distance indicated by its darkvision ability.
Darkvision is black and white vision.
\end{itemize}

\begin{changemargin}{0.3cm}{0.3cm}\begin{tcolorbox}[title = Note on light sources]
You will have noticed, or will soon do so, that magical light sources work differently, very often they last much less or generate little light. This is due to the will of a Patron and as such only a Patron can nullify its effects (or the Narrator!).
\end{tcolorbox}\end{changemargin}

\subsubsection{Dark}\index{Dark}

\label{buio}

Torches and lanterns can be suddenly extinguished by a gust of wind, magical light sources can be dispelled or countered, and some magical traps can create areas of impenetrable darkness.

In certain cases, some characters or monsters may be able to see while others are blinded. For purposes of the rules that follow, a Blinded creature is simply a creature that cannot see its surroundings.

\subsubsection{Blinded}\index{Blinded}\index{Invisible}

\label{accecato}

Blinded creatures lose their ability to deal extra damage caused by, for example, the backstab ability (but not damage burst or critical on hit).

Blinded creatures treat terrain as difficult \index{Moving in the dark}. They must make a DC 12 Acrobatics check per move action to move at normal speed. If the check fails they fall prone. Blinded creatures can't charge.

A creature that is blinded, or fighting an invisible creature,\index{Invisible} can make an Awareness check at difficulty 20 (or 10+the opponent's Stealth if the opponent does not want to be found) to locate the creature as long as it is within 6 meters from the character.

A blinded \index{Blinded}creature takes a -2 penalty on Strength and Dexterity-based Proficiency checks and automatically fails any Awareness checks that depend on sight.

Additionally, a blinded creature cannot use spells that involve gaze and is immune to spells that involve gaze.

See attack modifier details in \hyperlink{invisibility}{Invisibility} (page \pageref{invisibility}).

\subsubsection{Falls}\index{Falls}\index{Falls}\hypertarget{Falls}{}

\label{cadute}

Creatures that fall get hurt. Divide the height of the fall (in meters) by 3, round down, the resulting number is the d6 of damage suffered. Eg 16 meters of fall is 16/3=5d6 damage. For practicality, it is suggested to apply 1 damage for every 1 meter of fall.

Creatures that take damage from a fall land prone.

A successful DC 15 Acrobatics check allows the character to reduce damage by 3 when falling from less than 20 feet.

Falls onto soft surfaces (soft ground, mud, etc.) reduce damage by 3. 

A character can end his move Action with a fall, but only if he has not done any damage can he continue with the same Action, otherwise he must first get up from prone.


\begin{center}
\includegraphics[width=0.8\linewidth]{immagini/oggetticadenti.png}

\emph{Henry Justice Ford}
\end{center}

In a round of free fall you fall 150 meters (50d6 or 150 damage), in the first segment you fall 20 meters, then 80m then 150m. A character cannot cast spells while falling unless the fall is 100 meters or more. You are Distracted while trying to cast a spell while falling.\index{Casting spells while falling}

\medskip

\noindent \textbf{Fall into the water}\index{Fall into the water}

Falls into water are handled a little differently. As long as the water has a depth of at least 3 meters and the dive is from a height of within 12 metres, no damage is suffered.

To determine the damage from falling into water, subtract 15 meters from the height of the fall, add 1d6 damage for every 3 meters remaining ($((H-15)/3)*1d6)$).

Characters who willingly dive into the water take no damage if they succeed on a DC 15 Swim check and if the water is at least 20 feet deep. The DC of the test increases by 5 every 5 meters above 15 meters in height.

\subsubsection{Effects of Acid}\index{Acid}

\label{effetti-dellacido}

Corrosive acids deal 1d6 points of damage per round of exposure, except in the case of total immersion (such as in a bath of acid) which deals 10d6 points of damage per round. An acid attack, such as from a thrown flask or a monster's saliva/breath, should be treated as an exposure round.

The vapors produced by most acids are equivalent to inhaled poisons. Those who approach a large blob of acid must make a Fortitude save DC 13 or take 1 temporary point of Constitution damage per round of exposure. This poison has no frequency, so a creature is safe if it moves away from the acid.

Creatures immune to the caustic properties of acid may still drown if completely submerged in it (see Drowning).

\subsubsection{Effects of Smoking}\index{Smoking}

\label{effetti-del-fumo}

A character forced to breathe thick smoke must succeed at a Fortitude save each round (DC 15, +1 for each previous check) or spend the round coughing and choking. A character who continues to suffocate for 2 consecutive rounds takes 1d6 nonlethal damage per additional round of exposure. The smoke obscures vision, providing light cover (+2 Defense) to characters within it.

\subsubsection{Hunger and Thirst}\index{Hunger}\index{Thirst}

\label{fame-e-sete}

The characters may find themselves without water or food and without the means to obtain them. In normal climates, Medium characters need at least 2 liters of fluid and 0.5 kg of decent food per day to avoid hunger, Small characters need half that. In very hot climates, characters may need two or three times that amount of water to avoid dehydration.

Every day without food you must make a Fortitude save at difficulty 11 +1 per day without food, if you have no drink the difficulty increases by +3.

If you fail your saving throw you take 1d4 damage and become increasingly fatigued. Fatigue penalties remain until you eat and drink enough.

\subsubsection{Falling objects}\index{Falling objects}\index{Falling objects}

\label{oggetti-cadenti}

Just as characters take damage from falling more than 10 feet, they also take damage from falling objects.

Objects that fall on characters deal damage depending on their weight and the distance they fell.

The \textbf{Table: Damage from Falling Objects} determines the amount of damage dealt by an object based on its size. The object is assumed to be made of a dense, heavy material, such as stone.
Objects made of lighter materials may deal half or less of the listed damage, at the Storyteller's discretion. For example, a Huge boulder that hits a character deals 6d6 damage, while a wooden wagon might only deal 3d6.

Additionally, if the object falls from closer than 10 feet away, it deals half the listed damage. If an object falls from a distance greater than 20 meters, it deals double damage. The falling object takes the same amount of damage it deals.

\bigskip

\textbf{Table: Damage from Falling Objects}\index[Tables]{Damage from Falling Objects Table}

\medskip

\begin{tabular}{ll}
\textbf{Item Size} & \textbf{Damage}\\
\toprule
Tiny or Smaller & 1d6\\
Small & 2d6\\
Average & 3d6\\
Large & 4d6\\
Huge & 6d6\\
Gargantuan & 8d6\\
Colossal & 10d6\\
\end{tabular}

\bigskip

Dropping an object on a creature requires a ranged touch attack (see €11625{touchattack}{Touch Attack}, page €11626{touchattack}). These attacks usually have a range of 10 feet. If an object falls on a creature, the creature must make a DC 15 Reflex saving throw if hit for half damage if it is aware of the falling object. Falling objects that are part of a trap use the trap rules instead of the ones described here.

\subsubsection{Water Hazards}\index{Water Hazards}\index{Water}\hypertarget{water-hazards}{}\label{pericoli-dellacqua}

Any character can cross relatively calm water that has no depth greater than his height, without needing to check. \index{Swim}

A creature with a swim speed can move through water at its listed speed without making Swim checks. You have a +2d6 bonus on any Swim check to perform a particular action or avoid a hazard.
The creature can always choose to take a 10 on a Swim check, even if distracted or in danger while swimming. She can't take the 10 only in case of stormy waters. Such a creature can use the run action while swimming, as long as it swims in a straight line.

A spellcaster is considered distracted if she casts a spell while swimming.

If you do not have the Swim \textbf{moving in water} movement type, it is considered \textbf{\emph{difficult terrain}, and therefore you move at half the speed indicated by movement. 

If the creature can swim, no checks are needed to move normally in calm water, unless it wants to run (DC 13) or the waters are rough (DC 15) or stormy (DC 20).

If the creature cannot swim then it must make a Fortitude saving throw of DC 13 every round it wants to move, if the water is choppy the DC is 19 and if it is stormy the DC is 24, if it wants to \emph{run } the DC increases by 4.
In case of failure you don't move and you get a -1 on the next test, in case of a critical failure the next test gets -4. When the cumulative penalties are 9 or more you begin to sink and drown (see below).

If the swim check fails, the creature takes 1d6 lethal damage if the water flows over rocks and depressions.

Very deep water is not only pitch black but deals even worse damage due to pressure on the order of 1d6 points of damage per minute every 100 feet separating the character from the surface. A successful Fortitude save (DC 15, +1 for each previous check) means that the submerged character takes no damage for that minute. The water, beyond 150 meters deep, is very cold and inflicts 1d6 Nonlethal Damage per minute of exposure due to hypothermia.

\medskip
\begin{center}
\includegraphics[width=0.7\linewidth]{immagini/affogare.png}

\emph{Henry Justice Ford}\end{center}
\medskip

\textbf{Drowning}\index{Drowning}\index{Drowning}\index{Suffocating}€11644{holding your breath}{}\label{trattenereilfiato}\index{Holding your breath}

Any character can hold their breath for a number of rounds equal to 6 rounds of their Constitution score, with a minimum of 3 rounds. For each Action performed the remaining duration decreases by 1 round, casting a spell with Verbal components consumes 2 more rounds of air. After this period of time, the character must make a DC 12 Fortitude save each round to continue holding his breath. Each round, the DC increases by 2.

A caster who casts spells underwater is considered distracted.

If the saving throw fails, the character immediately drops to 0 hit points and faints. From the next round he begins to lose 1 hit point per round until he dies (or is revived!)

You can drown in substances other than water, such as sand, quicksand, very fine dust or a silo full of spelled, or simply by holding your breath.

\begin{center}
\includegraphics[height=0.7\linewidth]{immagini/desert.png}
\end{center}

\subsubsection{Dangers of the Heat}\index{Hot}

\label{pericoli-del-caldo}

A creature subjected to very high temperatures (above 100F) must succeed at a Fortitude save every hour (DC 15, +1 for each previous check) or take 1d4 nonlethal damage. If she wears heavy clothing or any type of armor, she takes a –1d6 penalty on these saving throws. A character adds his Survival proficiency value and can give a bonus to companions equal to half the value for the same saving throw. Unconscious characters begin taking lethal damage (1d4 points per hour).

A character who takes nonlethal damage due to exposure to heat is subject to heatstroke and is fatigued. These penalties end when the character recovers Nonlethal Damage suffered due to the heat.

Infernal heat (air temperature above 60 C, fire, boiling water, lava) deals lethal damage. Breathing air at these temperatures deals 1d6 fire damage per minute (no saving throw).

Boiling water deals 1d6 points of scalding damage, unless you are completely immersed in it, in which case you would take 10d6 points of damage per round of exposure.

\subsubsection{Catch Fire}\index{Catch Fire}\index{Fire}

\label{prendere-fuoco}

Characters exposed to boiling oil, campfires, or non-instantaneous magical fires may see their clothing, hair, or equipment catch fire. The spells specify whether they can start fires.

Characters in danger of catching fire can make a DC 15 Reflex save to avoid this fate. If a character's clothes or hair catches fire, he immediately takes 1d6 points of damage. For each subsequent round the burning character must make another Reflex saving throw. Failure means he takes an additional 1d6 points of damage that round. Success indicates that the fire is extinguished (i.e., once it succeeds on its saving throw, it is no longer burning).

A character on fire can automatically extinguish the flames by jumping into enough water to put them out. If there are no large quantities of water available, rolling on the ground or dampening the flame with cloaks or similar can grant the character +1d6 on the saving throw.

Those unfortunate enough to see their equipment or clothing catch fire must succeed at a Reflex save (DC 15) for each item. Flammable objects that fail the roll take the same amount of damage as the character.


\begin{center}
\includegraphics[width=0.7\linewidth]{immagini/fuocopericolo.png}
\end{center}

\medskip

\textbf{Lava Effects}\index{Lava}

Lava or magma deals 2d6 points of damage per round of exposure, except in cases of total submersion (such as when a character falls into the crater of an active volcano), which deals 20d6 points of damage per round (plus any falling damage and perhaps finding a ring ..).

Damage from the magma continues for 1d3 rounds after the exposure ends, but this additional damage is only half that dealt during the last round of actual contact (10/20/5). An Immunity or Resistance to fire also serves as resistance to lava or magma. However, Immune or Fire-Resistant creatures may drown if immersed in lava (see Drowning).


\subsubsection{Dangers of the Cold}\index{Cold}

\label{pericoli-del-freddo}

Poorly dressed characters in cold climates (below 5 C) must succeed at a Fortitude save every hour (DC 15, +1 for each previous check) or take 1d6 points of nonlethal damage.
In conditions of extreme cold or exposure below -17 C, an inadequately dressed character must make a Fortitude saving throw every 10 minutes (DC 15, +1 for each previous check), taking 1d6 Lethal Damage for each saving throw failed. Characters wearing winter clothing need to test for cold and exposure only once per hour.

A character adds his Survival proficiency value to saving throws and can give a bonus to companions equal to half the value for the same saving throw.

A character who takes nonlethal damage from cold or exposure is subject to chilblains or hypothermia (treat him as fatigued). These penalties end when the character recovers from nonlethal damage suffered due to cold and exposure.

Intolerably cold or exposed conditions (below -28 C) deal characters 1d6 lethal damage per minute (with no saving throw) unless specifically protected.

\begin{center}
\includegraphics[height=0.6\linewidth]{immagini/snowfall.png}
\end{center}

\subsubsection{Effects of Ice}\index{Ice}

Characters walking on ice are as if they are walking on difficult terrain. Movement is halved, any Acrobatics checks have a +5 difficulty increase. Characters who are in contact with ice for a long time may suffer damage from extreme cold.

\subsubsection{Slow Choking}\index{Choking}

A Medium-sized character can breathe peacefully for approximately 6 hours in a sealed chamber measuring 10 feet on a side. After this time, she takes 1d6 nonlethal damage every 15 minutes. Each additional Medium-sized character or each significant fire (a torch, for example) proportionally reduces the duration of breathable air. Once knocked unconscious by accumulating Nonlethal Damage, characters begin taking Lethal Damage at the same rate. Small characters consume half the air of Medium characters.

\subsection{Weather - Weather}\index{Weather}

\label{tempo-atmosferico---meteo}

Sometimes the weather can play an important role in the course of an adventure. Table: Random Weather is a generic table that can be used to establish local weather conditions. Table terms are defined below:

\end{multicols}

\medskip

\textbf{Table: Random Weather}\index[Tables]{Random Weather Table}

\medskip

\begin{tabularx}{0.95\textwidth}{llXXX}
\textbf{d\%} & \textbf{Weather} & \textbf{Cold Climate}& \textbf{Temperate Climate {*}} & \textbf{Desert}\\
\toprule
01-70 & Normal& Cold, calm & Normal for the season {*}{*} & Hot, calm\\
71-80 & Abnormal & Heat Wave (01-30) / Cold Wave (31-100) & Heat Wave (01-50) - Cold Wave (51-100) & Torrid, ventilated \\
81-90 & Inclement & Precipitation (snow)& Precipitation normal for the season& Hot, breezy \\
91-99 & Storm & Snowstorm& Lightning Storm / Snowstorm& Dust Storm \\
100& Violent storm& Blizzard & Blizzard, blizzard, hurricane, tornado & Downpour\\
\end{tabularx}

\medskip

* Temperate includes forests, hills, swamps, mountains, plains, and warm marine areas.

**Winter is cold, summer is hot, autumn and spring are moderate. The marshes are always slightly warmer in winter.

\begin{multicols}{2}

\textbf{Downpour}: Treat this as rain (see Precipitation below), but offers coverage like fog. It can cause flooding and usually lasts 2d4 hours.

\textbf{Hot}: The temperature is between 15 and 30 C during the day, and between 6 and 11 degrees less at night.

\textbf{Calm}: Light wind (between 0 and 15 km/h).

\textbf{Cold}: Temperature between -17 and 5 C during the day, and between 6 and 11 degrees less at night.

\textbf{Moderate}: Temperature between 5 and 15 C during the day, and between 6 and 11 degrees less at night.

\textbf{Hot Wave}: Increases the temperature by 6 C.

\textbf{Cold Wave}: Lowers the temperature by 6 C.

\textbf{Precipitation}: Roll a d100 to determine whether the precipitation is fog (01-30), rain/snow (31-90), or sleet/hail (91-00). Snow and sleet only occur when the temperature is 0C or lower. Most precipitation lasts 2d4 hours. Hail, on the other hand, lasts only 3d6 minutes but is usually accompanied by 1d4 hours of rain.

\textbf{Storm} (Lightning/Snow/Dust): The wind is very strong (45 to 75 km/h) and visibility is reduced by three quarters. Storms last 2d4-1 hours. See Storms, below, for further details.

\textbf{Storm} (Blizzard/Blizzard/Hurricane/Tornado): The wind speed is greater than 75 km/h (see Table: Wind Effects). Additionally, blizzards are accompanied by heavy snowfall (1d3 \texttimes{} 30 cm), and hurricanes are accompanied by downpours. Storms last 1d6 hours, blizzards 1d3 days. Hurricanes can last up to a week, but the greatest impact to characters will occur over a period of 24 to 48 hours as the center of the storm moves through their area. Tornadoes last very little (1d6 \texttimes{} 10 minutes), and usually form as part of a lightning storm.

\textbf{Torrid}: Temperature between 30 and 43 C during the day and between 6 and 11 degrees less at night.

\textbf{Ventilated}: The wind speed is moderate to strong (15 to 45 km/h); see Table: Wind Effects.

\textbf{Rain, Snow, Sleet and Hail}: Bad weather frequently slows down or blocks land transport and makes navigation practically impossible. torrential downpours and blizzards obscure the view as much as a dense fog would.


\begin{center}
\includegraphics[width=0.9\linewidth]{immagini/Paesaggio-pioggia-Auvers.png}

\emph{Vincent van Gogh, Landscape in the rain in Auvers, 1890, oil on canvas, 50 x 100 cm}
\end{center}


Most precipitation manifests as rain, but in cold climates it can also manifest as snow, sleet, or hail. Precipitation of any kind, followed by a drop in temperature from above to below 0C can produce ice.


\textbf{Heavy rain}\index{Heavy rain}: Rain halves visibility, and imposes a -1d6 penalty on Awareness checks. It has the same effect as very strong wind on flames, ranged weapon attacks, and Awareness checks as very strong wind.

\textbf{Snow}\index{Snow}: While falling, snow has the same effects as rain on visibility, ranged weapon attacks and Awareness checks and the terrain is considered difficult. A one-day snowfall leaves 3d6*2.5 centimeters of snow on the ground.

\textbf{Heavy Snow}: A heavy snowfall has the same effects as a normal snowfall, but obscures visibility like fog (see Fog). A day of heavy snow leaves 2d4 x 30 centimeters of snow on the ground and the terrain is considered doubly difficult (movement/4). Heavy snow accompanied by strong or very strong winds can result in snowdrifts 1d4 x 3 feet deep, especially on and around objects large enough to deflect the wind (a cabin or large tent, for example).
There is a 10\% chance that heavy snow will be accompanied by lightning (see Lightning Storm). Snow has the same effects as moderate wind on flames.

\textbf{Sleet}: This is basically frozen rain, which has the same effects as rain when it falls (except that the probability of extinguishing protected flames is 75\%) and those of snow once it has settled.

\textbf{Hail}: Hail does not reduce visibility, but the sound of falling hail makes hearing-based Awareness checks more difficult (-1d6 penalty). Sometimes (5% chance) the hail can be so large that it deals 1 lethal damage (per storm) to anything outside. Once deposited, hail has the same effect on movement as snow.

\subsubsection{Storms}\index{Storms}

\label{tempeste}

The combined effects of precipitation (or dust) and wind, which accompany all storms, reduce visibility by three-quarters, imposing a –8 penalty on all Awareness checks. Storms make attacks with ranged weapons impossible, except with siege weapons, which suffer a -1d6 penalty on attack rolls.
They automatically extinguish candles, torches or similar unprotected flames. Protected flames, such as those from lanterns, are shaken violently and have a 50\% chance of dying out. See Table: Wind Effects for possible consequences on creatures caught outside without shelter.

Storms are of three types.

\textbf{Dust Storm (Challenge Rank 3)}

these desert storms differ from other storms in that they have no precipitation. In contrast, dust storms carry grains of sand that obscure vision, smother unprotected flames, and can even extinguish protected ones (50% chance). Many dust storms are accompanied by very strong winds and leave behind a deposit of 1d6 \texttimes{} 2.5 centimeters of sand.
There is also a 10\% chance of encountering large dust storms with gales of wind (see Table: Wind Effects). These violent dust storms deal 1d3 nonlethal damage per round to anyone caught out in the open without shelter, and also pose the risk of suffocation (see Drowning, except that a character with a scarf or similar covering over his mouth and nose does not start to suffocate unless after a number of rounds equal to 10 \texttimes{} his Constitution score). Great dust storms settle behind (2d3-1) x 12 inches of sand.

\textbf{Snowstorm}

In addition to the winds and precipitation common to other storms, snowstorms deposit 1d6 \texttimes{} 2.5 centimeters of snow on the ground.

\textbf{Lightning Storm}

in addition to winds and precipitation (usually rain, but sometimes also hail), lightning storms are accompanied by electrical discharges that pose a danger to characters who find themselves outdoors without shelter (especially if they are wearing metal armor). As a general rule, one lightning strike per minute can be considered for a period of one hour in the heart of the storm. Each bolt deals between 4d8 and 10d8 electricity damage. One in ten lightning storms is accompanied by a tornado.

\textbf{Violent Storms}

Very strong winds and torrential rainfall reduce visibility to zero, making it impossible to make Awareness checks and make attacks with ranged weapons. Unprotected flames are automatically extinguished, and there is a 75\% chance that this will happen for protected ones as well. Creatures caught in these areas must make a Fortitude save or face effects depending on their size (see Table: Wind Effects). Severe storms are divided into the following four types.

\begin{center}
\includegraphics[width=0.95\linewidth]{immagini/Vincent_van_Gogh_tempesta.png}

\emph{Vincent van Gogh, Wheat field under a stormy sky (Auvers-sur-Oise, July 1890)}

\end{center}


\textbf{Blizzard}

Although they have little or no precipitation, storms can cause extensive damage due to the force of the wind.

\textbf{Blizzard}

The combination of strong winds, thick snow (usually 1d3 \texttimes{} 30 cm) and intense cold makes the blizzards lethal to anyone who is not prepared for them.

\textbf{Hurricane}

In addition to very strong winds and heavy rain, hurricanes are followed by floods. Many activities in an adventure are impossible under these conditions.

\textbf{Tornado}

In addition to very strong winds, tornadoes can seriously injure and kill those caught inside.

\subsubsection{Fog}\index{Fog}

\label{nebbia}

Whether in the form of a low-altitude cloud or a mist rising from the ground, fog obstructs vision beyond a distance of 10 feet. Creatures farther than 10 feet gain Light Cover (+2 Defense).

The fog makes the terrain difficult.

The fog could also be very thick in which case creatures further away than 3 meters enjoy medium coverage (+4 to Defense) and those within 1 meter still have light coverage.

\subsubsection{Winds}\index{Winds}

\label{venti}

Winds can create swirls of sand or dust, fuel large fires, capsize small boats and disperse gases or vapors. If they are strong enough they can even knock characters to the ground (see Table: Wind Effects), interfere with ranged attacks, or impose penalties on some Skill Checks.

\medskip

\textbf{Table: Wind Effects Wind Force}\index[Tables]{Wind Effects Table Wind Force}

\medskip

\begin{tabularx}{0.45\textwidth}{llX}
\textbf{Intensity} & \textbf{Speed} & \textbf{Range Attacks} \\
\toprule
Lightweight & 0-15km &\\
Moderate & 16.5-30 km/h& \\
Strong & 31.5-45 & -2 \\
Very strong & 45.5-75km/h & -4 \\
Storm & 76.5-111km/h & impossible \\
Hurricane & 12-261km/h & impossible \\
Tornado & 262-450km/h & impossible\\
\end{tabularx}

\bigskip

\textbf{Light Wind}

A gentle breeze, which has no practical effect on the game.

\textbf{Moderate Wind}

A strong wind, which has a 50\% chance of extinguishing any small unprotected flame, such as that of a candle.

\textbf{Strong wind}: Gusts that automatically extinguish unprotected flames (candles, torches and similar). These gusts impose a -2 penalty on ranged attack rolls and Awareness checks.

\textbf{Very Strong Wind}

In addition to automatically extinguishing unprotected flames, winds of this intensity violently agitate protected flames (such as those in a lantern) and have a 50\% chance of extinguishing them. Ranged weapon attacks and Awareness checks take a -1d6 penalty.

\textbf{Storm}\index{Storm}

Strong enough to knock down branches or even entire trees, blizzards automatically extinguish unprotected flames and have a 75\% chance of extinguishing protected flames, such as lantern flames. Ranged weapon attacks are impossible, and even siege weapons take a -1d6 penalty on attack rolls. Awareness checks that rely on hearing take a –2d6 penalty to the howling wind.

\textbf{Hurricane}\index{Hurricane}

Extinguishes all flames. Ranged attacks are impossible (except with siege weapons which take a -2d6 penalty on attack rolls). Even hearing-based Awareness checks are impossible, and all the characters can hear is the howling of the wind. Hurricanes are often capable of knocking down trees.

\textbf{Tornado (Challenge Rank 10)}\index{Tornado}

Extinguishes all flames. All ranged attacks are impossible (including those with siege weapons), as are hearing-based Awareness checks. Instead of being blown away (see Table: Wind Effects), characters who are in the immediate vicinity of a tornado and who fail a Fortitude save (DC 20+) are sucked into the tornado.

Those who come into contact with the tornado are lifted off the ground and tossed about for 1d10 rounds, taking 6d6 points of damage per round, before being violently ejected. The creature is ejected from a height of 1d6 feet per round it remains in the tornado.

Although the rotational speed of a tornado can reach 450 km/h, the cone itself moves forward at an average of 45 km/h (about 75 meters per round). A tornado is capable of uprooting trees, destroying buildings, and causing other forms of similar devastation.

\end{multicols}

%\vfill
%
%\begin{center}
%\includegraphics[keepaspectratio,width=0.6\textwidth]{immagini/blizzard.png}
%
%\end{center}

\pagebreak

\section{Water Adventures}\index{Water Adventures}

\label{avventure-in-acqua}
\begin{changemargin}{0.3cm}{0.3cm}\begin{enfasi}{
He looked at the sea and understood how alone he was now. (The Old Man and the Sea, Ernest Hemingway)}\end{enfasi}\end{changemargin}\medskip

\begin{multicols}{2}

\lettrine[lines=2, lhang=0.33, loversize=0.25, findent=1.5em]{L}{'water} allows societies to exist, but it can also destroy them. Life could not exist without it. Trade and travel are facilitated by his presence. Yet water can also kill, both by drowning people and by generating large-scale floods and tsunamis. Terrestrial life is dependent on water but at the same time fears it.

\textbf{Aquatic Adventures}

An aquatic adventure can take place anywhere water is the main element of the land: such as swamps, rivers, lakes, ponds, oceans, the Plane of Water, and the like. Aquatic adventures, however, do not require characters to have the ability to breathe underwater; The introduction of Aquatic challenges for low-level adventurers brings a lot of tension and a sense of danger to an adventure.

\textbf{Adapting to Aquatic Environments}

The rules for underwater combat apply to creatures that are not native to this dangerous environment, like most PCs. For prolonged Aquatic adventures and particularly in-depth explorations, characters will require the use of magic to continue their adventures. Transformation or Abjuration spells are of obvious use.

Pressure damage can be completely avoided with spells that offer resistance.

\subsection{Underwater fighting}\label{combatteresottacqua}\index{Underwater fighting}\hypertarget{underwater fighting}{}
Creatures that live on land have considerable difficulty fighting underwater. Water affects a creature's defense, attack rolls, damage, and movement.

\begin{itemize}
\item
A creature underwater loses its Dexterity bonus to Defense.
\item
A creature that is not under the \emph{Freedom of Movement} spell makes attack rolls with -1d6 and the opponent is considered to have Resistance to slashing and bludgeoning damage.

Weapons such as Trident, Short Spear, Short Sword, Javelin have no melee hitting penalty.
\item
Moving or swimming in water is considered \textbf{\emph{difficult} terrain}.
\end{itemize}
\medskip

These penalties are only valid if you do not have a move such as Swim.

\subsubsection{Underwater ranged attacks}\index{Underwater attacks}
Thrown weapons are ineffective underwater, even when thrown from land. Ranged weapon attacks take a –2 penalty on attack rolls and –1 on damage for every 8 feet of water they pass through.

\subsubsection{Attacks from the mainland}
Those characters who swim, float, or wade through surface water, or wade through water that is at least chest deep, enjoy medium coverage.

A fully submerged creature has complete cover against opponents on land.


\subsubsection{Magical effects in water}
Magical effects are unaffected, except those that require an attack roll (which are treated like all other effects) and fire effects.

\textbf{Fire}: Nonmagical fire (including alchemist's fire) does not burn underwater. Fire spells or magical effects are ineffective underwater. A partially submerged creature has fire resistance.

\begin{center}

\includegraphics[width=0.78\linewidth]{immagini/avventure_acqua_grey.png}

\emph{The Mermaid and the Boy - Henry Justice Ford}
\end{center}

\subsection{Nautical Adventures}

Water can provide the setting for a different and unique gaming experience: the nautical adventure. In such a scenario, the effects and dangers of underwater adventures are replaced by surface challenges, as the characters and their opponents use ships and boats to navigate the environment. Usually, nautical adventures resolve normally, with a combat aboard a ship similar to a land one. If combat occurs during a storm or in rough seas, treat the ship's deck as difficult terrain. Remember to consider the effects on Concentration checks for weather or roll.

\subsubsection{Rapid Sea Combat}


\begin{changemargin}{0.3cm}{0.3cm}\begin{enfasi}{
Everyone knows how to be a helmsman on a calm sea (Lucio Anneo Seneca)
}
\end{enfasi}\end{changemargin}\medskip

When ships do the fighting, things change a bit. The following rules are not intended to accurately simulate all aspects of naval combat, but only to provide you with quick and simple rules for disentangling such situations when they turn into a nautical adventure, whether a battle between two ships or between a ship and a sea monster.

\textbf{Preparation}: Determine which types of ships are involved in combat (see Table: Ship Statistics). Use a large, empty battle grid to represent the waters where the battle takes place. A single square corresponds to 9 meters of distance. Depict each ship by placing tokens occupying the appropriate number of squares (toy ships make excellent tokens and can be found in model shops).

\textbf{Starting Combat}: When combat begins, let characters (and important allied NPCs) roll Initiative normally; the ship moves and attacks based on the captain's initiative result. If one of the ships in battle uses sails to move, randomly determine which direction the wind is blowing by rolling 1d8 and following the guidelines for Missing Weapons.

\textbf{Move}: Based on the captain's Initiative score, the ship can move at its base speed in a single round as if the Action corresponded to that of the captain himself (or double his speed as the captain's only action). round), as long as it has its minimum full crew. The ship can increase or decrease its speed by 30 feet per round, until it reaches its maximum speed. Alternatively, the captain can change direction (maximum one side of a square at a time) (2 Actions). A ship can only change direction at the start of the round.

\textbf{Attacks}: Members in excess of a ship's minimum crew requirement may be placed to operate Siege Engines. Siege Engines attack based on the captain's Initiative score.


\begin{center}
\includegraphics[width=0.9\linewidth]{immagini/acquapericoli.png}
\end{center}

A ship can also attempt to ram a target if it hosts the minimum crew. To ram a target, the ship must move at least 30 feet and end up with its bow in a square adjacent to it.
Then, the ship's captain makes a Profession (sailor) check: if the result is equal to or greater than the target's Defense, the ship hits its target, inflicting damage as indicated on Table: Ship Statistics and at the same time suffering the minimal damage. A ship equipped with a ram deals an additional 3d6 points of damage to the target (the attacking ship takes no additional damage).

\textbf{Sinking}\index{Sinking}

A ship gains the sinking condition when its hit points drop to 0 or less. A sinking ship cannot move or attack and is considered sunk after 10 rounds. Every 25 points of damage taken by a sinking ship reduces the sinking by 1 round. The Craft spell allows you to repair a sinking ship if the ship's Hit Points are above 0, in which case the ship loses the sinking condition. Typically, nonmagical repairs take too long to save a ship from sinking once it begins to sink.

\textbf{Ship Statistics}

In the real world, there is a wide variety of boats and ships, from small rafts to massive galleons. To represent this, Table: Ship Statistics classifies seven standard ship sizes and their respective statistics. Just as real-world cultures have created and adapted different types of vessels, races in fantasy worlds might create their own bizarre ships.
Storytellers could use or modify these statistics to suit the needs of their creations and, in any case, describe such means of transport as they wish. All ships have the following traits.


\begin{center}
\includegraphics[width=0.8\linewidth]{immagini/navenotte.png}
\end{center}

\textbf{Type}: This is a general category that lists the basic type of ship.

\textbf{Defense}: The ship's defense. To calculate a ship's effective Defense, add the captain's Profession (sailor) score to the ship's base Defense. Touch attacks against a ship ignore the captain's modifier. A ship is never surprised.

\textbf{Basic Save}: A ship's basic saving throw modifier (Fortitude, Reflex, and Wisdom) have the same value. To determine a ship's actual saving throw modifiers, add the captain's Profession (sailor) modifier to this value.

\textbf{Maximum Speed}: The maximum speed of a ship in combat. An asterisk indicates that the ship has sails and can move at double speed if it moves in the same direction as the wind. A ship that only has sails can move only in the presence of wind.

\textbf{Armaments}: The number of Siege Engines that can be equipped on the ship. A ram uses one of these slots, and a ship can only be equipped with a ram.

\textbf{Ramming}: The amount of damage a ship deals with a successful ramming attack (without a ram).

\textbf{Squares}: The number of squares the ship occupies on the combat grid.

\textbf{Crew}: The first number indicates the minimum crew the ship needs to function normally, excluding weapons officers. The second indicates the maximum number of crew plus additional soldiers or passengers. A ship without its minimum crew can only move, change speed, change direction, or ram if its captain succeeds on a DC 20 Profession (sailor) check.
A crew that exceeds the minimum number does not affect movement, but its members can replace fallen members or operate additional weapons.

\bigskip

\end{multicols}

\textbf{Table: Ship Statistics}\index[Tables]{Ship Statistics Table}

\medskip

\begin{tabular}{llllllllll}
\textbf{Type} & \textbf{Defense} & \textbf{PF} & \textbf{Basic ST} & \textbf{Speed. (m/s)} & \textbf{Weapon} & \textbf{Spur} & \textbf{Quad}. & \textbf{Crew}\\
\toprule
Raft & 9 & 10& +0& 4.5 & 0 & 1d6 & 2 & 1/4\\
Rowboat & 9& 20& +2& 9 & 0 & 2d6+6 & 6 & 1/3\\
Boat & 8& 60& +4& 9 & 1 & 2d6+6 & 12 & 4/15+100\\
Longship& 6& 75& +5& 18 & 1 & 4d6+18 & 40 & 50/75+100\\
Sailboat & 6& 125 & +6& 18 & 2 & 3d6+12 & 20 & 20/50+120\\
Warship & 2& 175 & +7& 18 & 3 & 3d6+12 & 35 & 60/80+160\\
Galley& 2& 200 & +8& 27 & +4 & 6d6+24 & 60 & 200/250+200\\
\end{tabular}

\vfill

\begin{center}
\includegraphics[width=0.65\linewidth]{immagini/Galley_running_before_the_wind.png}

\smallskip

\emph{Thin gelea, length 45m, width 5m, draft approximately 200cm} 

\end{center}

\pagebreak

\section{City Adventures}\index{City}

\label{avventure-in-citta}
\begin{changemargin}{0.3cm}{0.3cm}\begin{enfasi}{
God created the countryside, and man created the city. (William Cowper)}\end{enfasi}\end{changemargin}\medskip

\begin{multicols}{2}

\lettrine[lines=2, lhang=0.33, loversize=0.25, findent=1.5em]{At first} sight, a city is very similar to a dungeon, as it is made up of walls, doors, rooms and corridors . Adventures set in cities differ from those set in dungeons for two main reasons: access to a greater number of resources and they must take into account the presence of law enforcement.

\textbf{Access to Resources}: Unlike dungeons and wildernesses, characters can buy and sell Equipment very quickly in town. A large city or metropolis likely has high-level NPCs and experts specializing in the more obscure fields of knowledge who can offer help and interpret clues. When characters are injured and bruised, they can always return to the comforts of their rooms in the inn.

The freedom to make a retreat and access to market goods means players have more control over the pace of gameplay in a city adventure.

%\begin{center}
% \includegraphics[width=0.8\linewidth]{immagini/cavalieri.png}
%\end{center}

\textbf{Law Enforcement}: The other element of distinction between going on an adventure in a city and exploring a dungeon lies in the fact that the dungeon is, almost by definition, a place without rules where the only law is that of the jungle: kill or be killed.

A city, on the other hand, is supported by a code of laws, many of which are designed explicitly to prevent the kind of behavior in which adventurers often indulge: killing and pillaging. However, city laws recognize the serious threat that monsters pose to city stability, and it is very rare that the prohibition on killing also applies to monsters such as Aberrations or Fiends.

Most evil humanoids, however, usually enjoy the same protection as all other citizens. Having a set of Evil Traits is not a crime (except perhaps in those cities where there is a strict theocracy, with the magical power necessary to enforce the law); only evil acts are considered a violation of the law.


Even when the adventurers encounter an evildoer committed to committing the most horrible crimes against the city's population, the law still looks down on those who take justice into their own hands by killing the evildoer or otherwise preventing him from being brought before a tribunal to be punished. tried.

\textbf{Weapon and Spell Limitations}

Each city has its own laws regarding what weapons you can carry in public and limitations on spells.

City laws may not affect all characters equally. A man of faith who moves with a weapon in tow is not hindered in any way by the law of binding weapons, but a spellcaster suffers a considerable reduction in his power if his Tome is confiscated from the city ​​gates.

\textbf{Urban Elements}

Walls, doors, dim lighting and uneven terrain: in many ways a city is similar to a dungeon. New elements suitable for a city setting are described below.

\textbf{Walls and Gates}

Many cities are defended by a circle of walls. Normal city walls are made of reinforced stone, 1 meter thick and 6 meters high. Such a wall is quite smooth, and a DC 30 Climb check is required to climb it. The walls have small battlements on one side to provide a parapet for the guards at the top, the space to walk on the walls is barely enough for one guard.


\medskip

\begin{center}
\includegraphics[width=0.85\linewidth]{immagini/muraparigi.png}

\emph{Chronicles, Jean Froissart. Queen Isabella of France arrives in Paris, 15th century}
\end{center}

\medskip

\textbf{The walls}

Unlike smaller cities, metropolises often also have internal walls, sometimes the old walls erected when the city was smaller, or walls that separate the various neighborhoods from each other. Sometimes these walls are as high and wide as the external ones, but much more often they are smaller in size.

\textbf{Watchtowers}: Some city walls have watchtowers that appear at regular intervals. Few cities have enough guards to place on each watchtower, unless the city expects an attack from outside. The towers offer an elevated view of the surrounding countryside as well as a bulwark of defense against enemy invaders.

Watchtowers are usually 3 meters taller than the wall they are part of, and their diameter is 5 times the thickness of the wall. Loopholes for archers open on the upper floors of the tower, and the top is crenelated in the same way as the surrounding walls. In the smaller towers (about 7 meters in diameter, along a 1 meter thick wall) a simple ladder connects the inside of the tower to the roof. In the larger towers there are actual stairs.

Access to the tower is protected by heavy wooden doors, with iron reinforcements and good locks (Disable Devices DC 25). Normally it is the captain of the guard who keeps the access key to the tower, and a second copy is kept in the internal fortress or in the city barracks.

\textbf{Gates}: A typical access gate to the city is composed of a guardhouse with two shutters and slits in the space between them. In small towns and cities, the main entrance is protected by double iron doors set into the city walls.

The gates are usually left open during the day and locked or barred at night. Generally, only one gate lets travelers in after dark, and it is guarded by guards who will only open the doors for someone who looks honest, presents the appropriate documents, or bribes them with a sufficient amount (depending on the type of city and of guards).


\textbf{Guards and Soldiers}

A city usually has full-time military personnel equal to 1\% of its adult population, patrol or conscript soldiers can be equal to 5\% of the population. Full-time soldiers are city guards responsible for maintaining order in the city, with a role similar to that of modern police, and (to a much lesser extent) defending the city from external assaults. Forced conscription soldiers are called to arms in the event of an attack in the city.

A typical deployment of city guards is distributed in three shifts of eight hours each, with 30% of its forces on duty during the day (from 8am to 4pm), 35% on duty in the evening (from 4pm to midnight). and 35\% of duty on the night shift (from midnight to 8 am). At any given time, 80\% of the guards on duty are patrolling the streets, while the remaining 20\% are assigned to various posts around the city, ready to react to any alarms. A similar guard post is present in at least every city neighborhood (a neighborhood is made up of several neighborhoods).

The majority of the city guards are made up of fighters, almost all of them 1st level. Officers are higher level fighters and maybe even some spellcasters.

\textbf{Siege Engines}\index{Siege Engines}

Siege engines are large weapons, temporary structures, or mechanisms traditionally used to lay siege to a castle or fortress.

\textbf{Heavy Catapult}: \index{Catapult}A heavy catapult is a gigantic siege engine capable of hurling boulders or other heavy objects with great force. Since the catapult's launch arc is very high, the contraption is able to hit areas outside of its line of sight. To fire a heavy catapult, the machine operator's leader makes a DC 15 special check using only his Attack Proficiency value with his Intelligence modifier, the range penalty, and the modifier from the lower section of the table. : Siege Engines.

If the check is successful, the catapult's boulder hits the melee area the catapult was aiming for, dealing the indicated damage to any objects or characters in the area. Characters who succeed on a DC 15 Reflex save take half damage. Once the boulder hits the area, subsequent shots will hit the same area, unless the catapult is redirected or the wind changes direction or speed.

If a catapult's boulder misses its target, use the weapons table at €11816{scatter}{scatter} (p. €11817{attackmebyscatter}). The distance covered is equal to 1d4x10 meters.


\medskip

\begin{center}
\includegraphics[width=0.85\linewidth]{immagini/armidaassedio.png}
\end{center}

To load a catapult requires a series of actions that take up the entire round. A DC 15 Strength check is required to lower the catapult arm; Most catapults have wheels that allow up to two operators to use the Help Another action to assist the main pulley operator.

A Profession (siege engineer) check with DC 15 allows you to lock the arm into place, and then another Profession (siege engineer) check with DC 15 will be used to load the projectile onto the catapult. It takes four rounds to reload a heavy catapult (multiple catapult operators can perform these actions in the same round, so four people can reload a catapult in a single round). A heavy catapult occupies a space of 4 meters.

\textbf{Light Catapult}: This is a smaller and lighter version of the heavy catapult. It essentially functions like a heavy catapult, except that a DC 10 Strength check is required to lock the arm into place, and only 2 rounds to redirect the catapult. A light catapult occupies a space of 10 feet.

\textbf{Balista}: \index{Balista}A ballista is basically a huge fixed heavy crossbow. Its size makes it difficult for most creatures to use. Thus, a medium creature takes a -1d6 penalty on attack rolls when using a ballista, and a small creature takes a -6 penalty. For a creature smaller than large, it takes 2 rounds to reload the ballista after firing.

A ballista occupies a space of 2 meters.

\textbf{Aries}:\index{Aries} This massive trunk is sometimes tied and suspended from a movable truss that allows its handlers to swing it with ever-increasing force against a target. As the round's only action, the character closest to the ram's tip makes an attack roll against the building's Defense, applying a -1d6 penalty for lack of proficiency (it is not possible to have proficiency in the use of this machine). In addition to the damage listed on Table: Siege Engines, up to nine other characters can push the ram and add their Strength modifiers to the ram's damage if they reserve an attack action to do so. At least one Huge or larger creature, 2 Large creatures, 4 Medium creatures, or 8 Small creatures are needed to operate a ram (Tiny or smaller creatures can't use a ram).

A ram is usually 30 feet long. In a battle, creatures wielding a ram must stand in two adjacent ranks of equal length with the ram supported between the two ranks.

\textbf{Siege Tower}\index{Siege Tower}: This machine is a huge wooden tower mounted on wheels or cylinders that can be pushed against a wall to allow besiegers to climb the tower and thus reach top of the walls benefiting from Coverage. The wooden walls of the tower are usually about 30 cm thick.

A typical siege tower occupies a space of 4 meters. Creatures inside it push it at a speed of 10 feet (a siege tower can't run). The eight creatures pushing the tower on the ground floor enjoy full cover, those on the upper floors enjoy medium cover and can shoot through the archer slits.

\end{multicols}

\medskip

\textbf{Table: Catapult Attack Modifiers}\index[Tables]{Catapult Attack Modifiers Table}

\medskip

\begin{tabular}{p{0.52\textwidth}p{0.40\textwidth}}
\textbf{Circumstance} & \textbf{Modifier}\\
\toprule
- Line of sight does not reach the target area & -6\\
- Consecutive shot (operators can see where the most recent missed shots fell) & cumulative +2 for previous missed shots (max +10)\\
- Consecutive shot (operators cannot see where the most recent missed shots fell but an observer provides indications) & +1 cumulative for previous missed shots (max +5))\\
\end{tabular}

\medskip

\textbf{Table: Siege Engines}\index[Tables]{Siege Engines Table}

\medskip

%\begin{tabular}{llllll}
%\textbf{Machine} & \textbf{Cost (gp)} & \textbf{Damage} & %\textbf{Range} & \textbf{Soldiers}\\
%\toprule
%Heavy catapult & 800 & 6d6 & 60m & 4\\
%Light catapult & 550 & 4d6 & 45m & 2\\
%Ballista& 500 & 3d8 & 36m & 1\\
%Aries & 1000 & 3d6 & - & 10\\
%Siege Tower & 2000 & - & - & 20\\
%\end{tabular}

\begin{tabularx}{0.95\textwidth}{lXlll|lXlll}
\textbf{Machine} & \textbf{Cost (gp)} & \textbf{Damage} & \textbf{Range} & \textbf{Soldiers}&\textbf{Machine} & \textbf{Cost (gp)} & \textbf{Damage} & \textbf{Range} & \textbf{Soldiers}\\
\toprule
Catapult &&&&&Catapult&&&&\\
heavy & 800 & 6d6 & 60m & 4& light & 550 & 4d6 & 45m & 2\\
\hline
Ballista& 500 & 3d8 & 36m & 1&Aries & 1000 & 3d6 & - & 10\\
\hline
\multicolumn{5}{l}{Siege Towers}& & 2000 & - & - & 20\\


\end{tabularx}

\begin{multicols}{2}

\medskip

\textbf{City Streets}\index{City Streets}

Typical city streets are narrow and winding. Most city streets are 3 to 6 meters wide, while alleys range from 3 meters wide to just 1 meter. If the paved floor is in good condition it is possible to move normally, while roads in bad condition and severely damaged are considered difficult terrain equivalent.

Some cities do not have large avenues, especially those that grew gradually from small settlements. Cities that have been planned out, or perhaps consumed by a major fire that allowed authorities to build new roads across what were once settled areas, may have some larger roads running through them. These main roads are 8 meters wide, allowing carts to pass next to each other, with 1 meter pavements on both sides.

\textbf{Crowd}: The city streets are crowded with people coming and going, busy with various daily chores. In most cases it is not necessary to include every 1st level commoner on the map when you come to a fight on the main avenue of the city. Instead, it is sufficient to indicate which areas on the map are occupied by the crowd. If the crowd sees something dangerous, it will move away at a rate of 30 feet per round on Initiative count 10. To make contact with the crowd you must have melee range. The crowd provides Complete Coverage.

\textbf{Direct the Crowd}: A DC 15 Diplomacy or DC 20 Intimidate check is required to persuade a crowd to move in a certain direction, and the crowd must be able to hear or see the character doing so the attempt. It takes a whole round to make the Diplomacy check, while only one Action is needed to make the Intimidate check.

If two or more characters attempt to push the crowd in two different directions, they make opposing Diplomacy or Intimidate checks to determine who the crowd will listen to. The crowd will ignore both if both test results are lower than the above DCs.

\textbf{Roofs}: Climbing onto a roof usually requires scaling a wall, unless a character can reach a roof by jumping down from a window, balcony, or higher bridge. Flat roofs are common only in warm climates (accumulating snow can cause a flat roof to collapse) and are easy to run over. Moving to the top of a roof requires a DC 20 Acrobatics check. Moving horizontally on a sloped roof (moving parallel to its peak, essentially) requires a DC 15 Acrobatics check. Moving up and down a sloped roof requires an Acrobatics check with DC 10.

Sooner or later a character will reach the end of the roof, and will have to make a long jump to get to the next roof or to get down to the ground. The distance separating one roof from the next is usually 3 meters, but the roof on the other side may be 1 meter higher or lower, or at the same height. Use the directions given for Acrobatics to determine whether the character is capable of making a jump.

\textbf{Sewers}: To enter the sewers, characters usually must open a grate (1 round) and jump down 10 feet. Sewers are built exactly like dungeons, except the floor is slippery or covered in water. Sewers are also similar to dungeons in terms of the creatures you can encounter inside them. Some cities were built on the ruins of older civilizations, so the sewers could also lead to treasures and dangers from a bygone era.

\textbf{City Buildings}
Most city buildings are divided into three categories. Many buildings in a city are two to five stories tall and are built next to each other to form long rows, interrupted only by main or secondary streets. These terraced buildings usually house a shop on the ground floor, with offices or apartments on the upper floors. Inns, wealthier businesses, and larger warehouses (as well as any mills, tanneries, and other space-intensive businesses) are typically large, free-standing buildings up to five stories tall.
Finally, smaller homes, shops, warehouses and warehouses are simple one-story wooden buildings, especially in the poorer neighborhoods.

\textbf{City Lighting}
If a city has large access avenues, these will be illuminated by lanterns hung at a height of approximately 2 meters on the sides of the buildings. These lanterns are placed at a distance of 9 meters from each other, so the lighting in these streets is practically continuous. The secondary streets and alleys are not lit; it is customary for citizens to pay a lanternman to accompany them if they have to go out at night. Alleys can be dark places even during the day, thanks to the shadows of the surrounding taller buildings. A dark alley during the day is not dark enough to provide complete but light coverage.

\subsubsection{Working in the city}\index{Working in the city}\index{Downtime}

During breaks between one adventure and another or because a certain amount of time must pass for a certain thing to happen, the characters can try to make use of their Skills to earn some coins.

The characters make one check per day in Crafts or Herbalism or Entertaining (depending on the activity they carry out), depending on their success they will earn or not.

Subtract 15 from the test performed, if the test is 16 or more, from the character and square the difference, the result is the silver coins earned ($(15-Prova)^2$).

\end{multicols}

\vfill

\begin{center}
\includegraphics[width=0.58\linewidth]{immagini/fognelondra.png}

\emph{Map of London Sewers, 1880}
\end{center}


\pagebreak

\section{Adventures and Disasters}\index{Adventures}\index{Disasters}

\label{avventure-e-disastri}
\begin{changemargin}{0.3cm}{0.3cm}\begin{enfasi}{
For one thing, no one gets left behind. (anonymous)}\end{enfasi}\end{changemargin}\medskip

\begin{multicols}{2}


\lettrine[lines=2, lhang=0.33, loversize=0.25, findent=1.5em]{Natural disasters} are terrifying environmental hazards that bring death and devastation. Supernatural ones can be even more destructive, as they can disfigure a world forever. A disaster is more like an adventure than an encounter, and does not have a specific Challenge Rating. Rather, each part of the disaster should be treated as a separate encounter designed with a Challenge rating appropriate for the PCs.

Below are rules for dealing with the effects of three different types of disasters, both natural and supernatural. Some disasters occur quickly, such as earthquakes and tsunamis, while others proceed through numerous phases, such as forest fires, volcanoes, and undead uprisings. Adjust the adventure outline to suit the disaster, to allow events to unfold over the course of a few minutes or many days as you need.

\textbf{Volcanoes}\index{Volcanoes}

When the earth's crust breaks and ejects its molten heart, one of the most dramatic natural disasters occurs: the eruption of a volcano. Volcanic eruptions offer a wide range of options to the Storyteller, including lava, lava bombs, poisonous gases, and pyroclastic flows. Storytellers might also consider foreshadowing a dramatic volcanic eruption (or volcanic dragons) with pre-existing dangers, such as avalanches and minor earthquakes.

\textbf{Lava}\index{Lava}

Lava flows are generally associated with non-explosive eruptions and can be a permanent feature of active volcanoes. Lava flows are mostly slow and move at 5 meters per round, but the hottest ones are fast and reach 12 meters per round. Channeled lava, as in a lava tube, is very dangerous, moving at a rate of 120 feet per round (a Challenge rating of 6). Creatures reached by a lava flow must succeed on a DC 20 Reflex save or be submerged in the lava. Success indicates that they are in contact with the Lava but not Immersed.

\textbf{Lava Bombs} (Challenge level 2 or 8)\index{Lava Bombs}

Chunks of molten stone can be hurled many kilometers from an erupting volcano, cooling into solid stone before reaching the ground. A typical lava bomb hits a point designated by the Storyteller and explodes in a 20-foot radius. All creatures in the area must succeed on a DC 15 Reflex save or take 4d6 points of damage. Creatures that have Cover or are able to cover themselves (such as with a shield) gain a +2 bonus on this roll. Sometimes very large lava bombs form, dealing 12d6 points of damage. Normal lava bombs have a Challenge rating of 2, large lava bombs have a Challenge rating of 5.

\textbf{Poisonous Gases} (Challenge level 5)\index{Poisonous Gases}

One of a volcano's most insidious threats is toxic gas, often unnoticed amid the fire and destruction. Different types of poisonous vapors arise from a volcanic eruption, some visible, some not. Poisonous gases deal 1d3 Constitution damage per round if inhaled (Fortitude DC 15 negates, DC increases by 1 for each round of exposure), visible ones also function as Thick Smoke. The poisonous gas clouds move downwards and generally reach a height of 6 meters. Strong winds can deflect gas clouds, as can tall barriers provided the gas has somewhere else to go.

\textbf{Pyroclastic Flows} (Challenge level 10)

Some volcanic eruptions create a devastating wave of burning ash, hot gases and volcanic debris called a pyroclastic flow that can travel for miles. A pyroclastic flow is treated as an Avalanche traveling 500 feet per round, combined with the poison gas effects noted above. Contact with the hot debris from the flow deals 2d6 points of fire damage per round, while any creature buried by the flow takes 10d6 points of fire damage per round.

\textbf{Tsunami}\index{Tsunami}

Tsunamis, sometimes attributed to tidal waves, are tremendous surges of water caused by underwater earthquakes, volcanic explosions, landslides or asteroid impacts. Tsunamis cannot be detected until they reach shallow water, when the mass of water forms a large wave. Depending on the size of the tsunami and the slope of the coast, the wave can cover any distance, from a hundred meters to over a kilometer on land, leaving a trail of destruction in its wake. The water then retreats, carrying away all sorts of debris and creatures out to sea.

The exact devastation caused is subject to the Storyteller's discretion, but a typical tsunami topples or uproots all temporary or poorly constructed structures in its path, destroys approximately 25\% of well-constructed buildings (causing significant damage to those that remain), and leaves solid fortifications slightly damaged. At least 1/4 of the population living in the area (including animals and monsters) dies in the disaster, washed out to sea, drowned on the beach, or buried under rubble.

A creature can avoid being carried out to sea with a DC 25 Swim check; otherwise it is dragged 6d6 x 10 feet from shore. The waters after a tsunami are always considered rough or stormy, barring magical influences. A creature involved in the collapse of a building suffers 6d6 points of damage (Reflex save DC 15 halves), or half as much if the structure is particularly small. there is a 50\% chance that the creature will be buried (as with a Collapse), or that the tsunami will destroy the building, freeing the creature from the rubble.

\textbf{Rise of the Undead}\index{Undead}

The result of an ancient curse or the will of a Patron, one of the most terrifying supernatural disasters is the rising of the Undead: the dead who emerge from the grave to claim the living. This disaster can affect any area where the dead have been buried, not just towns and cities. Many a battlefield has seen the rise of a legion of withered Undead fighters. Undead uprisings occur in waves, with the timing varying depending on the major forces at play. Events can unfold over the course of a few days, with the devastation of a city, or extend for weeks with the terrified population huddling behind bolted doors and struggling to survive. During the day, life often returns to some semblance of normality, as daylight temporarily suppresses the power of undeath.

\textbf{The Restless Dead}

In the first few nights of an Undead uprising, the recently dead reanimate as zombies. Those buried in consecrated ground are not revived, but bodies left unburied or in mass graves stagger out onto the streets, wreaking havoc. Initially, only a few corpses are able to free themselves from their coffins and graves, but each evening, the number of living corpses increases. When dawn comes, the dead seek safety in their graves or other hidden places. Anyone caught in the daylight frets Confused until he is destroyed or reaches shelter. At the Storyteller's discretion, nonhumanoid corpses may rise as undead on subsequent nights.

\textbf{The Awakening of the Skeletons}

As the uprising progresses, older and older corpses join the ranks of the Undead. Skeletons bearing traces of long-rotted grave robes claw their way out of cemeteries and crypts, acting with a malevolence and organization rarely found among their kind. The Undead remain intelligenceless, but the magical power behind the raid gives them the efficiency and tactical acumen of a living army. The Skeletons find weapons and armor with which to equip themselves for battle. The elite Skeleton champions lead the troops, using Magic Items stolen from abandoned tombs. Finally, Ghouls and Wights also prowl the streets during the dark, along with other lesser Undead with free will

\textbf{Lost Souls}

While the uprising gathers forces, the restless souls of corpses long reduced to dust also awaken. Ghosts, Shadows, Wraiths, and even Specters arise to hunt the living. Some Ghosts may free themselves from the uprising's malevolent influence, and enterprising characters may glean valuable information from these restless spirits.

The infusion of negative energy strengthens the Undead within the raid area, granting the benefits of a Blessing. Once hallowed areas are now treated as normal terrain and can serve as new sources of corpses for Undead armies; the sanctified ground remains inviolate.

As the Undead grow stronger, the rising tide of negative energy brings the Shadow Plane closer, dulling or graying colors except during the brightest hours of the day. Even the most light-vulnerable Undead can move with impunity from late afternoon to midmorning.

\textbf{Necropolis}

The flow of negative energy is irreversible, darkness eventually claims the area, covering it with perpetual shadow. The hallowed ground remains a rare sanctuary, but only until it is destroyed by malevolent forces outside.

Heroes who died in the battles return as fearsome Undead generals. The few living survivors are enslaved as slaves. The area becomes a city of death or construction begins if no city existed or survived. Undead with free will gather in this new sanctuary, and only the greatest heroes manage to return from this withered area to the world of the living.

\end{multicols}

\vfill

\begin{center}
\includegraphics[width=0.45\linewidth]{immagini/anubis.png}

\emph{Representation of \href{https://it.wikipedia.org/wiki/Anubis}{Anubis}}
\end{center}



\pagebreak

-\section{Dungeon Adventures}\index{Dungeon}

\begin{changemargin}{0.3cm}{0.3cm}\begin{enfasi}{

\st{Linux} Dungeon is user friendly. It's just very picky about who his friends are. (anonymous)

\medskip

The dungeon is tilted. The Creatures Are Enraged Because They Can't Play Marbles (Dungeon Keeper 2, Video Game, 1999)

}\end{enfasi}\end{changemargin}\medskip

\label{avventure-nei-dungeon}
\begin{multicols}{2}

\lettrine[lines=2, lhang=0.33, loversize=0.25, findent=1.5em]{D}{i} all the strange places an adventurer can explore, none are deadlier than a dungeon. These mazes, filled with deadly traps, hungry monsters, and wondrous treasures, test the characters' every skill and ability. These rules can apply to any type of dungeon, from a shipwreck to a vast underground cave complex.

\begin{changemargin}{0.3cm}{0.3cm}\begin{narratore}
The dungeon, cave, catacomb, cave, underground ecosystem... call it what you like, it is a cornerstone of the adventure!

A dungeon is a recipe made of humidity, stench, stale air, dirt, mud, remains of creatures, traps, slimes, traps (plenty...), monsters, enemies, monsters (plenty!), darkness, sinister noises, mushrooms , creaks, yelps, screams, moans.. but also of fear, tension, shiver, terror \& horror, enfasi, anger, pain, disappointment and treasures!!!

Your dungeon is never just a cave. NEVER!
\end{narratore}
\end{changemargin}

Whether they are caves, caves, caves, dens, dens, \emph{Dungeons} often represent the focal center of adventure, exploration and survival.

The characters will spend a lot of time in these environments and the Storyteller must be prepared and ready for the environment they will encounter.

When preparing a cave it is necessary to think intelligently about the type of cave and the creatures you will encounter, each cave is a complex ecosystem.
Putting a group of lizardfolk in without thinking about what they eat, where they sleep, what kind of organization they have is dangerous, not to mention inserting a chimera.
Will he have atrophied wings because the cave is 3 meters high and 3 meters wide and he finds it difficult to move? What did she eat during this period? Rather better to use a gorgon that feeds on minerals...

If designed with attention and care, a cave can become an excellent experience of encounters, situations and adventure.

\subsection{The underground}\index{The underground}

The natural conditions of the subsoil depend on various factors but there are certainly points common to all.

- No lights to illuminate the spaces. There may be sporadic fluorescent fungi, which radiate dim light nearby, but nothing that can illuminate the entire environment

- Humid environment

- Usually cool ambient temperature, there are rarely caves with extreme temperatures in both hot and cold.

\subsubsection{Lighting}\index{Lighting}

In a cave there are no artificial or natural sources of light other than those introduced by sentient creatures. There may be groups of fungi, lichens, dimly illuminating the ground where they grow, within 1 meter, but nothing else around.
Furthermore, if torn from the ground they lose their bioluminescence after 2d4 turns.

Creatures that live in caves become accustomed to the darkness by developing some form of alternate vision, such as darkvision, tremor sense, or blindsight.

Even the flashlight itself can provide limited relief as its light beam is 3 meters plus 3 meters of dim light and lasts for an hour before going out.

Check the sections on \hyperlink{hedging}{Hedging} and \hyperlink{invisibility}{Invisibility} (page \pageref{invisibility}) for more information.

\subsubsection{Movement}\index{Movement}

If you do not have the means to see the terrain it is considered difficult and potholes, precipices and various obstacles can be very dangerous.\index{Difficult terrain in darkness}

In case of total darkness and in a natural environment, a DC 12 Dexterity check must be made every 30 meters or stumble and suffer 1 temporary damage.

\subsection{Types of caves}

Different types of caves can be identified:

\begin{itemize}


\item \textbf{created by the flow of water}. In this case the tunnel can be quite chaotic in its unraveling due to the type of rocks that the water encountered. There may still be underground rivers and lakes.

\item \textbf{created by erosion}. In this case the water is probably no longer there except in a minimal part, the resulting caves can also be very large with rooms tens if not hundreds of meters wide.

\item may have been \textbf{created by a volcano} with the flow of lava. In this case the tunnel dug out of the rock is often linear and somewhat smooth, the lava once congealed and then crumbled over the millennia.

\item may be \textbf{arctic caves}, carved out of ice by water. In this case, carefully evaluate the surrounding environment and the freezing temperature.

\item can be \textbf{artificial caves}, built by creatures of different types.

\end{itemize}

\subsubsection{The four types of Dungeons}\index{The four types of Dungeons}

The four basic types of dungeons are defined by their current state. Many dungeons are variations of these basic types or combinations of multiple types. Occasionally, ancient dungeons are used by new inhabitants for different purposes.

\textbf{Ruined Structure}: Once inhabited, this place is now abandoned (fully or partially) by its original creators and is occupied by other creatures. Many subterranean creatures seek out underground and abandoned buildings in which to establish their lairs. Any traps that may have existed have probably already been removed or activated, wandering beasts may be found.

\medskip
\begin{center}
\includegraphics[width=0.7\linewidth]{immagini/avventure_dungeon.png}

\emph{The Red Romance Book, Henry Justice Ford}
\end{center}
\medskip

\textbf{Occupied Structure}: This dungeon is still in use. Creatures (usually intelligent) still inhabit it, although they may not be the creators of the dungeon. An occupied structure could be a house, a fortress, a temple, an active mine, a prison, a headquarters...

This type of dungeon is less likely to have traps or wandering beasts, and more likely to have organized guards, both standing guard and patrolling. The traps and wandering beasts you may encounter are often under the control of the occupants. Occupied structures have furnishings suitable for the inhabitants, as well as decorations, food supplies, and the ability for the inhabitants to move around.

The inhabitants may also have access to a communication system, and almost always control at least one access to the outside.

Some dungeons are partially occupied and partially empty or in ruins. In these cases, the occupants are usually not the original builders of the place, but rather a group of intelligent creatures who have established their base, lair, or fortification within the abandoned dungeon.

€11936 € {Safe Haven}: When someone wants to protect something, he often buries it underground. Whether the object he wishes to protect is a fabulous treasure, a forbidden artifact, or the corpse of an important man, these valuable objects are placed inside a dungeon and surrounded by barriers, traps, and guardians.

The safe haven type dungeon is the one that will have more traps and fewer wandering beasts. It is normally built for functionality rather than appearance, although it is sometimes decorated with statues and painted walls, especially for the tombs of important people.

\begin{center}
\includegraphics[width=0.65\linewidth]{immagini/dungeon.png}
\end{center}

Sometimes, however, a treasure room or crypt is constructed to house living guardians. The problem with this strategy is that you need to keep the creatures alive between one intrusion attempt and another. Magic is usually your best bet for supplying these creatures with food and water. Tomb and tomb builders typically place undead and constructs, which need no sustenance or rest, to protect their dungeons. Magic traps can attack intruders by summoning monsters into the dungeon that disappear when they finish their task.

\textbf{Natural Cave Complex}: The underground caves provide shelter for all types of creatures of the deep. Created naturally and connected by a system of labyrinthine passages, these caverns lack any semblance of order, logic, or decoration. Without any intelligent power building it, this type of dungeon is the least likely to feature traps or doors.

Multiple varieties of mushrooms live in caves, sometimes growing to form enormous forests of mushrooms and mushrooms, where subterranean predators prowl, hunting those who feed on these plants. Some varieties of mushrooms produce a phosphorescent glow that can provide the natural cave complex with its own limited source of illumination. In other areas, the use of Daylight spells can grant sufficient light for green plants to grow.

Often, a natural cave complex is connected to other types of dungeons, having been discovered when the man-made dungeon was built. A cave complex can connect two independent dungeons, sometimes producing a strange mixed environment. A natural cave complex joined to another dungeon often provides a path that subterranean creatures can use to reach and populate an artificial dungeon.

\subsection{Exploring}\index{Exploring}\index{Move carefully}

Moving inside a dungeon requires attention and cold blood. Uneven floors, sinister noises, trap doors and traps, lights that appear and disappear make it difficult to venture safely into these dangerous environments.

The characters will have to be careful, actively look for traps, observe into the distance and maintain a cautious attitude. All this means that the movement is halved if the characters \emph{take precautions} to avoid problems, i.e. have a minimum bonus to Awareness checks.

Describing what the character does to look for traps, passages... \emph{problems} or requesting a check (Survival or Awareness) at DC 13 can give general indications on the \emph{sensation} that something is wrong.

\subsection{Dungeon Ground}\index{Dungeon Ground}

The following rules cover the basic terrain that can be found in a dungeon.

\subsubsection{Walls}\index{Walls}\label{pareti}\hypertarget{walls}{}

Sometimes brick walls (stones stacked on top held together with lime) divide the dungeons into corridors and rooms. Dungeon walls may also be carved from bare rock, resulting in a chiseled appearance, or they may be composed of smooth, plain stone as found in natural caves. Dungeon walls are difficult to damage or breach, but are usually easily scaled.

\end{multicols}
\textbf{Table: Walls}\index[Tables]{Walls Table}
\medskip

\begin{tabularx}{0.95\textwidth}{XllllX}
\textbf{Wall Type} & \textbf{Thickness} & \textbf{Breakthrough} & \textbf{Hardness} & \textbf{Hit Points} & \textbf{Climb DC}\\
\toprule
Stone Bricks & 30cm & 35 & 8 & 90 & 20\\
Top stone bricks & 30cm & 35 & 8 & 120 & 25\\
Reinforced Stone Bricks & 30 & 45 & 8 & 180 & 20\\
Carved Stone & 90 & 50 & 8 & 540 & 25\\
Rough stone & 150 cm & 65 & 8 & 900 & 25\\
Iron & 7.5cm & 30 & 10& 90& 25\\
Card & variable & 1 & --& 1 & 30\\
Wood & 15cm& 20 & 5 & 60& 21\\
\end{tabularx}

\medskip

\textbf{Table: Digging a tunnel}\index[Tables]{Table Digging a tunnel}

\medskip

\begin{tabular}{llll}
\textbf{Miner}&\multicolumn{3}{c}{\textbf{Material to Excavate - 1 minute}}\\
&\textbf{Soil}&\textbf{Stone} €11969{soft}&\textbf{Hard stone}\\
\toprule
Human&30cm&15cm&7cm\\
Gnome &45 cm&30 cm&15 cm\\
Dwarf/Orc & 55cm&45cm&20cm\\
Stone Giant& 3 m& 1.5 m& 75 cm\\
Xorn &6 m&6 m& 6m\\
Earth Elemental & 9 m&9 m&9 m\\
\end{tabular}

\medskip

The excavated distances indicated are assumed to be obtained with suitable tools such as spades or pickaxes, otherwise reduce to a third.

\begin{multicols}{2}

\textbf{Stone Brick Walls}: The most common type of wall for a dungeon, stone walls are usually at least a foot thick. Often these ancient walls have holes and cracks, inside which slimes and small creatures can nest, waiting there for their prey. The stone brick walls are able to block all but the loudest noises. A DC 20 Climb check is required to move along a brick wall.

\textbf{Higher Quality Stone Brick Walls}: Sometimes stone brick walls are better constructed (smoother, with better fitted stones and less damage) and occasionally these higher quality walls are covered by mortar or stucco . These walls are often embellished with paintings, bas-reliefs or other decorations. Higher-quality brick walls are no more difficult to damage than ordinary brick walls, but they are more difficult to climb (DC 25).

\textbf{Reinforced Walls} These are brick walls with iron bars on one or both sides, or inserted into the wall itself to reinforce it. The Hardness of the reinforced wall remains the same, but the Hit Points are doubled and the DC for the Strength check to break through it is increased by 10.

\textbf{Carved Stone Walls}: These walls are generally found in rooms or passages carved into bare rock. The rough surface of a carved wall has tiny ridges on which fungi can grow and cracks within which subterranean pests, bats or snakes can live.

When such a wall has another side (the wall separates two rooms in a dungeon), the wall is at least 3 feet thick; if it were thinner it would risk causing everything to collapse because it would not be able to support the weight of the stone vault. A DC 25 Climb check is required to scale a carved stone wall.

\textbf{Rough Stone Walls}: These surfaces are irregular and rarely flat. They are smooth to the touch but full of tiny holes, hidden alcoves, and protrusions at various heights. They are usually wet or at least humid, as natural caves are typically the product of water seepage. When such a wall from another side, the wall is usually at least 150 centimeters thick.

A DC 15 Climb check is required to move along a rough stone wall.

\textbf{Iron Walls}: These walls are placed inside the dungeons around important places such as treasure rooms.

\textbf{Paper Walls}: Paper walls are the opposite of iron ones, used as screens to block the view but nothing more.

\textbf{Wooden Walls}: Wooden walls are often found as recent additions to older dungeons, used to create animal enclosures, storage, or even just to divide a larger series of smaller rooms.

\textbf{Magically Treated Walls}: These walls are stronger than average, with a higher Hardness, with more Hit Points and to break through them you need to overcome a higher DC. Magic can usually double the hardness and hit points of the wall and add up to +20 to its DC to break through it. A magically treated wall also gains a saving throw against spells that might affect it, with the saving throw bonus equal to 2 + half the caster level of the spell reinforcing the wall. Creating a magical wall requires the Craft Wondrous Item feat and the expenditure of 1,500 gp for each 10-by-10-foot section.

\textbf{Louver Walls}: Louvre walls can be constructed of any durable material, but are usually made of brick, carved stone, or wood. They allow defenders to fire arrows or crossbow bolts at intruders while remaining behind the relative protection of a wall. Archers behind the slits enjoy superior cover that gives them a +8 bonus to Defense, a +1d6 bonus on Reflex saving throws.\index{Slits and arrows}

\subsubsection{Floors}\index{Floors}

As with walls, there are many types of dungeon floors.

\textbf{Paved}: Like brick walls, floors can be made of stones fitted together. They are usually full of cracks and usually barely level. Sludge and mold grow inside these cracks. In some cases the water flows in small drains through the stones or forms stagnant pools. Flagstone is the most common type of floor in dungeons.

\textbf{Uneven Pavement}: Over time, some floors may become so uneven that they require a DC 10 Acrobatics check to run or charge across their surface. Those who fail the check cannot move during that round. Such dangerous floors should actually be the exception and not the rule.

%\begin{center}
% \includegraphics[width=0.9\linewidth]{immagini/pavimento_grey.png}
%\end{center}

\textbf{Carved Stone Floor}: Rough and uneven, carved stone floors are usually covered in loose stones, gravel, dust and other debris. A DC 10 Acrobatics check is required to run or charge across such a floor. A failure means the character can still act, but cannot run or charge that round.

\textbf{Poor crushed stone}: Small and sparse debris is present on the ground. A floor with little gravel adds 2 to the DC of Acrobatics checks.

\textbf{Thick gravel}: The ground is covered with debris of all sizes. Crushed stone is considered difficult terrain. A densely packed floor adds 5 to the DC of Acrobatics checks, and adds 2 to the DC of Awareness vs. Stealth checks.

\textbf{Smooth Stone Floor}: Smooth, perfect and sometimes even polished floors are only found in dungeons created by capable and careful builders.

\textbf{Natural Stone Floor}: The floor of a natural cave is as uneven as the walls. It is unlikely that these caves have large flat surfaces; their floors are more likely to be arranged on multiple levels.

Some surfaces might vary in elevation by as little as 30 centimetres, so that moving from one point to another is no more difficult than climbing a step on a ladder, but in some places the floor might go up or down more than 1.5 metres, forcing the character a Climb check (page \pageref{Climb}) to move from one surface to another.

Unless there is a path carved out by time or well-trodden, the terrain is considered difficult and therefore the movement is halved. For practical reasons, steps under 50cm are considered difficult terrain and those within 1.5m are doubly difficult terrain.\index{Steps} Charging and running in these environments are impossible, except on the courses in question.

\textbf{Slippery}: Water, ice, slime or blood can make any floor described in this section more treacherous. Slippery floors increase the DC of Acrobatics checks by 5.

\textbf{Grate}: A grate often covers a pit or area beneath the main floor. Grates are usually made of iron, but larger ones may also be made of reinforced tree trunks. Many grates have hinges that allow access to the area below (these grates can be locked like a door), while others are fixed and designed so that they cannot be moved. A typical 1-inch-thick iron grate has 25 hit points, hardness 10, and DC 27 to break through or dislodge.

\textbf{Ledges}: Ledges allow creatures to walk over an area below. They are often arranged around pits, along underground rivers, as balconies that surround a large room, or provide a position from which archers can stand to attack enemies from above.

Narrow ledges (less than 30 centimeters wide) require Acrobatics checks (DC 15) from those moving on them 3 Movement Actions. A failure means the character who was moving falls off the ledge.

Sometimes the ledges have a railing. In these cases the characters gain a +1d6 bonus on Acrobatics checks to move along the ledge. A character near the railing has a +2 bonus on his opposed Strength check to avoid being pushed off the ledge.

\textbf{Transparent Floors}: Transparent floors, made of strengthened glass or magical materials allow you to observe a dangerous environment from above. Transparent floors are usually placed above lava pools, arenas, monster lairs, and torture chambers. They can be used by defenders to guard an area.

\textbf{Sliding Floors}: A sliding floor is a type of trap door, created to be moved and reveal something underneath. A sliding floor typically moves so slowly that anyone standing on it can avoid falling into the opening, as long as they have room to move. If such a floor slides so fast that there is a chance that a character will fall into whatever is beneath it (sharp spears, a vat of boiling oil, or a shark-infested pool, acid...) then yes treat it like a trap.

\textbf{Trap Floors}: These floors were designed to suddenly become dangerous. With the application of just the right amount of weight or the pull of a nearby lever, spikes emerge from the floor, flames or puffs of steam shoot from hidden holes, or the entire floor moves. These strange floors are usually found inside arenas, designed to make fights more exciting and deadly. This type of floor is handled like a trap.

\subsection{The doors}\index{Doors}

\index{Doors}Doors within dungeons are more than simple entrances or exits. Often they can be real meetings. Dungeon doors come in three basic types: wooden, stone, and iron.

\end{multicols}

\textbf{Table: Doors}\index[Tables]{Door Table}\index{Blocking a door}\label{tabellaporte}\hypertarget{door table}{}

\medskip

\begin{tabular}{llllll}
\textbf{Door type} & \textbf{Typical thickness (cm)} & \textbf{Hardness} & \textbf{Hit Points} & \multicolumn{2}{c}{\textbf{DC to open}} \\

&& & & \textbf{Locked} & \textbf{Locked}\\
\toprule
Plain wood & 2.5 & 5 & 10& 15 & 18\\
Good wood & 3.75 & 5 & 15& 18 & 21\\
Sturdy wood& 5 & 5 & 20& 25 & 28\\
Stone & 10 & 8 & 60 & 31 & 34\\
Iron & 5 & 10& 60& 30 & 33\\
Wooden gate valve & 7.5 & 5 & 30& 27& 30\\
Iron gate valve & 5 & 10& 60& 28& 31\\
Lock & - & 15& 30& - & -\\
Hinges & - & 10& 30& - & -\\
\end{tabular}

\medskip

\begin{flushleft}
\textbf{Blocked / Stuck}: Breakthrough DC (Strength check, +1d6 with a crowbar)\\
\textbf{Locked}: DC to Lock (Disable Devices check)\\

Breaking down a door with your shoulder/kick costs 1 Action. Forcing it with a crowbar costs 2 Actions.\index{Breaking Doors Action}\index{Forcing Doors Action}

\textbf{critical failure} in a Strength check means having hurt yourself in the breakthrough maneuver. Until at least 10 minutes have passed, it is no longer possible to break down a door.\index{Fail strength test to break down doors}\index{Break down doors}
\end{flushleft}

\begin{multicols}{2}

\medskip

\textbf{Wooden Doors}\index{Wooden Doors}: Built with thick nailed boards, sometimes reinforced with iron bars (also placed to prevent deformations produced by the humidity of the dungeons), the wooden ones are the most common type of door. Wooden doors vary in hardness: they can be plain, good or sturdy. Simple doors (DC 15 to break through) are not designed to keep out motivated attackers.

Good doors (DC 18 to break through), while strong and durable, are not designed to take a great deal of damage. The sturdy doors (DC 25 to break through) are clad in iron and are fairly strong barriers against those who try to pass through them. Iron hinges support the door and usually a circular ring placed in the center is used to open it. Sometimes, instead of a ring, a door has an iron bar on one or both sides that functions as a handle.

In inhabited dungeons these doors are usually well maintained (not locked) and not locked, although important areas will probably be locked.

\textbf{Stone Doors}\index{Stone Doors}: Constructed from blocks of solid stone, these heavy and unwieldy doors are often designed to rotate on themselves when opened, although dwarves and others Skilled craftsmen are able to construct hinges strong enough to support the weight of a stone door.

Secret doors hidden along a stone wall are usually made of stone. Otherwise, doors of this type are designed to become strong barriers that protect whatever lies beyond them. As a result they are often found locked or barred.

\textbf{Iron Doors}\index{Iron Doors}: Rusty but sturdy, iron doors in a dungeon are hinged like wooden ones. These doors are the strongest doors of the non-magical type. They are usually locked or barred.

\begin{center}
\includegraphics[width=0.8\linewidth]{immagini/porta_grey.png}
\end{center}

\textbf{Breaking Down}\index{Breaking Down Doors}: Dungeon doors can be locked, trapped, reinforced, barred, magically sealed, or sometimes simply blocked.

All but the weakest characters will be able to break down a door with a heavy tool such as a mallet, and numerous spells and magical items can offer characters an easy way to get past a locked door.

\textbf{DC 13 or less}: A door that anyone can break down.

\textbf{DC 13 --15}: A door that a strong person would have to break down in a single attempt, and that a person of average strength might have some hope of breaking down in a single blow.

\textbf{DC 16--20}: A door that practically anyone could break down, given enough time.

\textbf{DC 21--25}: A door that only a strong or very strong person has any hope of breaking down, and probably not on the first try.

\textbf{DC 26 or higher}: A door that only a person of exceptional strength can have any hope of breaking down.

\textbf{Locks}: Dungeon doors are often locked and so the Disable Devices perk comes in handy. The locks are recessed on the edge opposite the hinges or straight into the center of the door. Locks usually control an iron or wooden bar that extends from the door into the wall that supports it.

The locks secure between two rings, one on the door and one on the wall. More complex locks, such as combination or enigma locks, are usually built into the door itself.

The DC to pick a lock with a Disable Device check often falls between 15 and 30, although there are locks with higher or lower DCs. A door can have more than one lock, each of which must be opened separately.\index{Blocking a door}. Picking a lock without burglary tools carries a penalty of -1d6 on the check.\index{Breaking without tools}

A critical failure to open a door or padlock will cause breakage of burglary tools.\index{Breakage of burglary tools}\index{Failure to open a lock}

Locks are often equipped with traps, usually poisoned needles that pop out to prick the thief's fingers.

\subsubsection{Breaking a lock}\index{Breaking a lock}

A special door might have a keyless lock, but it requires you to guess the right combination of nearby levers or press symbols on a panel in the correct order to open it.

\textbf{Blocked Doors}: Dungeons are often damp places, and in some cases doors get stuck, particularly if they are made of wood. It is usually assumed that approximately 10% of wooden doors and 5% of other doors are blocked. These values ​​can be doubled (to 20\% and 10\% respectively) in the case of long abandoned or neglected dungeons.

\textbf{Barred Doors}: When a character tries to break through a barred door, it is the quality of the bar that makes the difference, not the material of the door itself. Breaking down a door closed by a wooden bar requires a Strength check of DC 25, and the DC increases to 30 in the case of a metal bar.

The characters can attack the door and destroy it, leaving the bar hanging in the clear passage. Using a crowbar to force a jammed/blocked door grants a +1d6 on the check.\index{Crowbar on door}

\textbf{Magic Seals}: Spells placed on a door can make crossing a door difficult.

A door on which a magic block has been cast is considered closed even if it does not physically have a lock. A lockpick or destroy magic spell or a successful Strength check is required to pass a locked door in this way.

\textbf{Hinges}: Most doors have hinges. Obviously sliding doors are not (these rather have grooves in the floor, which allow them to slide sideways with ease).

\textbf{Standard Hinges}: These hinges are made of metal and hold the door together with its support or the wall. Remember that the door opens towards the side where the hinges are located (so if the hinges are on the PCs' side, the door will open towards them; otherwise it will open towards the other direction).

Adventurers can remove the hinges one at a time by succeeding at various Disable Device checks (only if, of course, they are facing the side of the door on which the hinges are located). Such an action has a DC of 20, as many of the hinges are rusted or jammed.

Breaking a hinge is difficult. Most have Hardness 10 and 30 Hit Points. The DC to break a hinge is the same as that needed to break down the door


\begin{center}
\includegraphics[width=0.8\linewidth]{immagini/cardini.png}
\end{center}


\textbf{Insertion Hinges}: These hinges are much more complex and are only found in areas of excellent construction. These hinges are built into the wall and allow the door to open in both directions. Characters cannot reach the hinges to remove them unless they break through the door support or wall. Push-in hinges are usually found on stone doors, but are also sometimes seen on wooden or iron doors.

\textbf{Pins}: The pins are not true hinges, but simple pegs that protrude from the top and bottom of the door and fit into holes in its support, allowing it to turn. The advantages of pins are that they cannot be removed like hinges and that they are easy to make. The disadvantage is that since the door pivots on its center of gravity (usually in the middle), nothing larger than half the width of the door can fit through it.

Doors equipped with hinges are usually made of stone and often also wide enough to overcome the disadvantage. Another solution is to place the pin towards one end and make the door thicker on that side and thinner on the other, so that it opens more or less like a normal door.

Secret doors within walls often rotate, as the lack of hinges makes it easier to conceal the door's presence. Pins also allow objects such as a bookcase to be used as secret doors.

\textbf{Secret Doors}: Disguised as a common portion of a wall (or floor or ceiling), as a bookcase, as a hearth, as a fountain, a secret door leads to a secret passage or to a room.

Someone searching the area can find a secret door (if one exists) with a successful Awareness check (with DC 20 for a common secret door and DC 30 for a very well hidden door).

Many secret doors require a special method to open, such as a hidden button or pressure plate. Secret doors can open like common doors, pivot, slide, collapse, lift, or even lower like a drawbridge.

A builder might place a secret door very low near the floor or very high on a wall, making it more difficult to both find and use the door.

\textbf{Magic Doors} Enchanted by the original builder, a door can warn explorers not to continue. It may be protected from damage, with increased Hardness or more Hit Points, as well as an improved saving throw bonus. A magical door may not lead to the space behind it, but may actually be a portal to a far distant place or even to another plane of existence. Other magical doors may require a password or special keys to open. 
Magic doors can only be opened by specific command or by canceling the magic that pervades them, very few have a lock.
In this case the Storyteller might decide to increase the Disable Device check by 10, bringing it to 30 or more.

\textbf{Shutters}: These special doors are made of rods of iron or thick reinforced wood that drop from a recess in the upper part of an arch. Sometimes a portcullis has horizontal bars to form a grid, other times it does not. Usually raised with a winch or similar machinery, the shutters can be quickly lowered, and the bars end in spikes to discourage anyone from passing under them (or from attempting to run through them as they lower). Once down, a portcullis closes, unless it is so large that no normal person would be able to lift it. However, lifting a typical portcullis requires a DC 25 Strength check.

\textbf{Walls, Doors and Detection actions}

Stone walls, iron walls, and iron doors are generally thick enough to block most Divinations. Wooden walls, wooden and stone doors are generally not thick enough to do the same. However, a secret stone door built into a wall and as thick as the wall itself (at least 12 inches) will block most of these Actions.

\textbf{Stairs} The most traditional method of connecting different levels of a dungeon is through stairs. A character can climb or descend a ladder as part of her movement without penalty but cannot run. Increase the DC of any Acrobatics checks made on a scale of 4. Some stairs, which are particularly steep, are treated as difficult terrain.

\subsection{Dangers in Dungeons}\index{Dangers in Dungeons}

In dungeons and caves, in addition to monsters, there are also other dangers including collapses, mold, fungi and more.

\subsubsection{Collapses and Failures (Challenge level 8)}\index{Collapses and Failures}

Collapses and failures in tunnels are extremely dangerous. Not only do dungeon explorers run the risk of being crushed by tons of stone, but also, should they survive, being stuck under a pile of debris or unable to reach an exit.

A collapse buries anyone in the middle of the buried area, and then the debris that rolls away will inflict damage on anyone in the periphery of the buried area. A typical collapse corridor might have a 3 meter radius buried zone and a 1 meter radius debris flow zone at the end of that buried zone.

An unsafe ceiling can be identified with an Engineering Knowledge check of DC 20 or a Bricklaying Profession check of DC 20.

An unsafe ceiling can collapse under the impact of a large force. A character can cause a collapse by destroying half of the pillars holding up the ceiling.

Characters in the buried area take 8d6 points of damage, or half damage if they succeed at a DC 15 Reflex save. They are then buried. Characters in the sliding area take 3d6 points of damage, or no damage if they succeed at a DC 15 Reflex save. Characters in the sliding area are also buried if they fail the saving throw.

Buried characters take 1d6 nonlethal damage for each minute they remain under the rubble. If a character in this condition falls unconscious, he must make a Fortitude saving throw DC 15. If the character fails the check, he begins taking 1d6 lethal damage per minute until he is freed or dies.

Characters who are not buried can dig their companions out from under the rubble. In 1 minute, using only their hands, a character can move an amount of rock and debris equal to five times their heavy load limit. The amount of loose rock that fills an area of ​​1 cubic meter weighs approximately 1 ton (1000 kg). Equipped with the right tools, such as a pickaxe, crowbar, or shovel, a digger can take half the time it would take doing it by hand. You could also allow a buried character to free themselves by succeeding at a DC 30 Strength check.


\begin{center}
\includegraphics[width=0.5\linewidth]{immagini/arcoserpenti.png}

\emph{Henry Justice Ford}
\end{center}


\subsubsection{Slides, Molds and Mushrooms}\index{Sludges, Molds and Mushrooms}

In the dank, dark recesses of dungeons, mold and mildew thrive—fear the columns of mold! As for spells and other special effects, all slimes, molds, and fungi are considered vegetables. Like traps, dangerous slimes and molds have a Challenge rating and characters earn Experience Points for encountering them.

A shiny organic slime coats anything that remains immersed in the dark, damp dungeons for too long. This type of slime, although it can be repellent, is not dangerous. Mold and fungi abound in dark, cold, damp places. While some are as harmless as regular dungeon sludge, others are quite dangerous. Edible mushrooms, vesicles, yeasts, molds, and other types of fibrous, bulbous fungi, or entire swathes of fungal spores can be found in most dungeons. They are usually harmless and are often even edible (although most are unappealing or have a strange taste).

\textbf{Screeching Boletus}\index{Screeping Boletus}: These human-sized purple mushrooms emit a piercing sound that lasts 1d3 rounds whenever there is movement or a source of light within 10 feet. This shout makes it impossible to hear other sounds or noises within melee range. The sound attracts nearby creatures who are willing to investigate. Some creatures that live near screeches have learned that noise most often means food.

\textbf{Green Slime}\index{Green Slime} (Challenge Rank 4): This dungeon hazard is a treacherous variety of normal slime. The green slime devours flesh and organic materials that come into contact with it and is even capable of dissolving metals. Bright green, wet and sticky, it is distributed in patches on walls, floors and ceilings and reproduces by consuming organic material. It drops from walls and ceilings when it detects movement (and possible food) beneath it.

The green slime deals 1 point of Constitution damage each round it devours flesh. On the first round of contact, the slime can be removed from a creature (probably destroying the object used to remove it), but after the first round it must be frozen, burned, or cut (also dealing damage to its victim) to be removed . Anything that deals fire or cold damage, sunlight, or a remove disease spell destroys a patch of green slime. In the case of wood or metal, the green slime deals 2d6 points of damage per round, ignoring the hardness of the metal but not that of the wood. Does not damage the stone. Defense 10, Hit Points 30, Saving Throws T 3, R 0, V 1.

\textbf{Phosphorescent Mushroom}\index{Phosphorescent Mushroom}: This strange subterranean mushroom exudes a faint purple luminescence that illuminates caves and underground passages like a candle. Rare spots of this mushroom illuminate like a torch. Torn from its environment it goes out in 1d4 turns.

\begin{center}
\includegraphics[width=0.9\linewidth]{immagini/funghi.png}

\emph{They are glowing in the dark, believe me! and fried they are even better!}
\end{center}

\textbf{Yellow Mold} \index{Yellow Mold}(Challenge level 6): If disturbed within a 3 meter radius, it releases a cloud of poisonous spores. Everyone within 10 feet of the mold must succeed on a DC 15 Fortitude save or take 1d3 points of Constitution damage. Another DC 15 Fortitude save is required once per round for the next 5 rounds or to avoid taking another 1d3 points of Constitution damage. A successful saving throw blocks this effect. Fire destroys yellow mold, while sunlight renders it inert. Defense 10, Hit Points 25, Saving Throws T 3, R 0, V 1, Vulnerability to Fire.

\textbf{Brown Mold} \index{Brown Mold}(Challenge rank 2): Brown mold feeds on heat, extracting it from everything around it. It usually occurs in patches with a diameter of 1 meter and the temperature around the mold is always cold within a radius of 3 metres. Living creatures within 3 feet of it take 3d6 nonlethal cold damage. If a fire source is brought within 1 meter of the mold it immediately doubles in size. Cold damage, such as that dealt by a cone of cold, destroys it instantly. Defense 10, Hit Points 12, Saving Throws T 3, R 0, V 1, Vulnerability Cold, converts fire damage taken into Hit Points.

\subsubsection{Example of Dungeon Traps}\index{Example of Dungeon Traps}

The name of the trap, the DC for the Survival test to find the trap and the indications for its use are indicated.\\

%\textbf{Flooded room, DC 17}: if the characters do not notice the pressure plate on the floor it will cause the entrance door to seal and the room will begin to fill with water.
%The room fills with water in 10 rounds. A DC 15 Survival check, combined with a DC 12 Swim check, causes the plate to be detected, which triggers the water release.

\textbf{Crushing Room, DC 15}: if the characters do not notice the pressure plate on the floor it will seal the entrance door and very loud screeching and gearing noises will fill the room. The walls will begin to move closer to each other like the ceiling to the floor. If the characters do not find the hidden tile (DC 17) they will suffer 10d6 crushing damage. The trap is easier to detect than others because the walls are thicker making the room smaller.

\textbf{Crushing Ceiling, DC 18}: if the characters do not notice the activation system (pressure plate, cable, broken light beam...) a 3m x 3m section of ceiling will fall on the characters with damage of 3d6.

\textbf{Cobweb Tunnel, DC 12}: This tunnel is evidently full of thick, dense, sturdy cobwebs. If the characters enter they are considered Hindered. After 1d4 rounds of permanence, an activator will generate a spark, setting the webs on fire for 1d4 rounds. Each round you take 2d4 fire damage inside the tunnel.

\textbf{Pit, DC 15}: The careless character will cause a 3m x 3m section of floor to collapse onto a pit. This can be a simple pit (1d6 fall damage), with spikes (1d6+2d4), with acid (1d6 per round), with undead...

\textbf{Garrotte, DC 14}: This trap can be very tricky. A magically sharpened thread is 1 meter from the ground, between one wall and the opposite one, and flows towards the players.
A DC 13 Athletics check is required or suffer 2d6 slashing damage.

\textbf{Crushing door, DC 16}: this door, as soon as it is touched, rotates on central hinges and as it rotates it hits the character (or characters if it is a large door). It deals 1d6 bludgeoning damage and continues spinning for 1d6 rounds.

\textbf{Finger Cutter, DC 14}: This trap is very sneaky. It has a hole approximately 1 cm in diameter and 7 cm deep. Anything that touches the bottom will trigger the trap, causing 2d4 damage to the inserted finger/object. The blade may also be poisoned.

\end{multicols}

\pagebreak

\section{Adventure Dangers}\index{Adventure Dangers}


\begin{changemargin}{0.3cm}{0.3cm}\begin{enfasi}{
An adventure is a reasonable outcome. Two are better, three deserve to be passed down, and four... no one will ever dispute four adventures. (John Steinbeck)

\medskip


He who, even if he is safe, stays on guard is in less danger. (Publilius Syrus)} \end{enfasi}\end{changemargin}\medskip

\label{pericoli-in-avventura}

\begin{multicols}{2}

\lettrine[lines=2, lhang=0.33, loversize=0.25, findent=1.5em]{The}{t} world is full of dangers as well as ravenous dragons and fiends. Hazards are threats that exist in the environment and have much in common with traps, but are usually part of the location rather than built. Hazards fall into three main categories: environmental, living, and magical.

Environmental hazards include landslides, fires and the like. Living hazards include creatures that, while not considered monsters, pose a threat to \emph{unwary} adventurers, such as slimes, fungi, and mosses. Magical dangers are the most unpredictable and can be remnants of arcane experiments, strange subterranean radiations, or failed ancient spells.

\medskip

\begin{center}
\includegraphics[width=0.8\linewidth]{immagini/boscopericoli.png}
\end{center}

\textbf{Antimagic (challenge level 6)}\index{Antimagic}

A zone of magic-destroying magical entropy, Antimagicians form at the sites of great magical duels, through the destruction of powerful artifacts or from vortices of mystical energy at the edges of antimagic zones. They range in size from small bubbles of just a few meters to large areas the size of a city.

A successful DC 20 Arcana check reveals the proximity of an Antimagic with a tingling in the air. An active spell brought into an Antimagic may be dispelled, and any spell cast within it is subject to an immediate counterspell. If you roll a critical on the Magic Test, it passes the counterspell but generates no further effects.

If the spell fails, the release of magical energy deals 2d6 force damage in an explosion in a 10-foot radius centered on the caster; a Reflex save at DC 15 allows you to halve this damage.

A spell manifested by an object always fails.

If multiple overlapping blasts hit the same target, only the most damaging one applies. A spell that has resisted an attempted dissolution is not affected again unless it exits and re-enters Antimagic.

The most powerful Antimagic are even more destructive. Each +1 increase in Challenge rating increases the damage by 1d6 and the save DC by 1.

\medskip
\textbf{Steal Air (Challenge level 1 or 4)}\index{Steal Air}

Gas pockets pose a risk to miners, spelunkers, and adventurers investigating caves. Nonflammable gases have a Challenge rating of 1 and require a DC 25 Survival check to be noticed. Creatures that breathe that air must succeed at a Fortitude save (DC 15 +1 for each previous roll) every hour or become fatigued. Creatures that hold their breath can avoid these effects.

Flammable vapors are much more dangerous (Challenge level 4). This gas replaces the breathable air in the lungs, causing fatigue. Additionally, any open flame or spark causes an explosion that deals 6d6 points of damage (DC 15 Reflex save for half) to anyone in the cave or within 10 feet of an entrance. Fire burns the oxygen in the air, making it unbreathable for 2d4 minutes. After an explosion, flammable gas generally takes many days to return to dangerous levels.

\medskip
\textbf{Parasites}\index{Parasites}

Parasites such as earworms or necrophagous larvae cause parasitosis, a type of Affliction similar to Diseases. Parasites can only be cured through specific treatments; regardless of how many saving throws you make, the parasite continues to afflict the target. Although a Remove Disease (or similar effect) immediately kills a parasite, immunity to Disease offers no protection, since it is caused by parasites.

\medskip
\textbf{Ear Finder (Challenge level 5)}\index{Ear Finder}

Earseekers are tiny white worms that live in rotten wood or other organic debris. They can be noticed with an Awareness check (DC 15). Otherwise, a living creature that searches their lair inadvertently acquires one or more ear-seekers, which then seek out a warm area on the creature's body, preferring the ear canal, and lay 2d8 eggs there before dying.

The eggs hatch 4d6 hours later and the larvae devour the flesh around them. When their host dies, the little worms crawl out and look for a new one.

Remove Disease kills all earworms or unhatched eggs on a host. Some earseekers prefer to live in corrupted wood, often hiding in dungeon doors. The small holes left by this variant are very difficult to notice (Awareness DC 20).

\medskip
\textbf{Ear finder}

Type: Parasitosis

TS: Fortitude DC 15

Onset: 4d6 hours

Saving Throw Frequency: 1 per hour

Effects: 1d3 to Constitution if the saving throw fails

\medskip
\textbf{Mnemonic Crystals (Challenge Rank 3)}\index{Mnemonic Crystals}

Memory crystals are large (10-40 feet tall) clusters of purple quartz crystals that radiate a strong aura of alteration. Identifying them requires an Arcana check of DC 25.

Memory crystals store magical energy to grow and defend themselves. A memory crystal absorbs spells cast within 10 feet of it. The caster can make a DC 22 Will save to avoid the effect.

Damaging or breaking the crystals causes absorbed spells to be expelled in a burst of magical energy that deals 1d4 points of damage per Magic Points absorbed (usually 10d4) to all those within a 20-foot radius.

Memory crystals are very fragile (Hardness 0, 1 Hit Point).
In areas rich in crystals, creatures that pass through them must succeed at a DC 10 Acrobatics check to avoid walking over them or grazing them and breaking them.

\medskip
\textbf{Necrophagous Larvae (Challenge level 4)}\index{Necrophagous Larvae}

Once they occupy a living body, the larvae burrow toward the host's heart, brain and other key internal organs, ultimately causing its death.

In the first round of parasitosis, applying fire to the entry hole can kill the larvae and save the host, but the host takes 1d6 fire damage.

Pulling them out works too, but the longer the larvae remain in the host, the more damage this method causes. To extract the larvae requires a sharp weapon and a First Aid check with DC 20, inflicting 1d6 damage for each round that the host has been afflicted by parasites. If the First Aid check succeeds, a larva is removed. Remove Disease kills all necrophagous larvae present in a host.

\medskip
\textbf{Necrophagous Larvae}

Type: Parasitosis

TS: Fortitude DC 17

Onset: immediate

Frequency: 1/round

Effects: 1 Constitution damage per larva

\medskip
\textbf{Magnetized Ore (Challenge Rank 2)}\index{Magnetized Ore}

The strange energies of the underworld can charge stones and mineral veins with powerful magnetic fields, creating a danger for those who carry or wear ferrous metals. All iron or steel things brought within 10 feet of the ore are dragged toward it.

\begin{center}
\includegraphics[width=0.8\linewidth]{immagini/neodimio.png}

\emph{Neodymium}
\end{center}

Any creature with more than 4 metal encumbrance is inexorably drawn towards the magnetic ore. A Fortitude save with a Strength modifier of DC 25 is allowed to avoid approaching or being able to detach yourself from the large magnet.

\medskip
\textbf{Cursed Well (Challenge Rank 3)}\index{Cursed Well}

The lingering effects of ancient curses or the noxious energy that ripples from a submerged cursed magical item can transform a simple well of water into a risky magical hazard. A cursed well lures passers-by into its depths through the illusion (DC 16 Will save to disbelieve it) of a wondrous treasure at the bottom only 10 feet deep. Any creature that reaches the treasure activates the curse.

A creature within the well must succeed on a DC 16 Will save or be affected by the curse, which distorts its perception of the well. The water seems to thicken in a viscous sapropelite (Author's note: also sapropel or fetid slime, used in geology to indicate a blackish, pasty and more or less compacted slime, originating from the deposit of organism remains in stagnant or slightly moving waters mixed with calcareous or siliceous shells of microorganisms and clayey substances) which pushes the creature towards the bottom at 12 metres.

A Swim check is required at DC 16 each round, failure means you begin to drown.

A cursed well radiates strong magic, and can be destroyed by Spell Destruction or Remove Curse.

\medskip
\textbf{Poison Oak (Challenge Rank 1 or 3)}\index{Poison Oak}

Contact with poison oak (Challenge Rank 1) causes a painful, itchy rash that leaves the victim fatigued until the damage heals. Full body contact or inhalation of smoke from burning poison oak could be fatal (Challenge rating 3), 2 additional ranks of Fatigued. A DC 15 Nature (or Herbalism) check reveals the dangers inherent in the seemingly harmless plant. This danger can also be used for similar harmful plants (poison ivy, poison sumac or stinging nettles, but the latter are not dangerous when burning).

\textbf{Poison Oak}

Type: Poison, contact

TS: Fortitude DC 13

Onset: 1 hour

Effects: 1d4 Dexterity damage, creature is fatigued until damage heals

Heal: 1 ST


\subsection{Preparing for rest}\index{Preparing for rest}\index{Sleeping}\index{On-call shifts}

Every adventurer must rest every now and then, he must do it carefully and being careful not to run into nasty and dangerous surprises.

Whenever a character ends a 24-hour period without at least 8 hours of sleep, he must succeed at a DC 12 Fortitude save or become fatigued.

Each further missed rest will make him even more fatigued by accumulating the relevant penalties. If the character stays awake for multiple days, fighting sleep becomes more difficult. After the first 24 hours, DC increases by 4 for each consecutive 24-hour period spent without 8 hours of sleep. The DC returns to 12 when the character completes a rest of at least 8 hours.

Sleeping in medium or heavy armor makes you fatigued, unless you have the \hyperlink{second skin}{Second skin} skill.

You are unable to sleep 8 hours at intervals less than 16 hours.\index{Sleep several times a day}

A demanding activity such as fighting, casting spells, riding, if continued for more than 10 minutes undermines the benefits of the rest taken.

\subsubsection{Organize Guard Shifts}

If the group is large, the guard shifts to monitor and control the environment become shorter.

\medskip{}

\textbf{Table: Duration of guard shifts}\index[Tables]{Table Duration of guard shifts}

This table indicates the duration of the guard shifts and the total rest time of the group, assuming that they rest for at least 8 hours.

\medskip{}

\begin{tabularx}{0.45\textwidth}{XXX}
\textbf{Members} &\textbf{Duration}&\textbf{Duration}\\
\textbf{of the group}&\textbf{of the Shift}&\textbf{Total}\\
\textbf{2} & 8 h & 16 h\\
\textbf{3} & 4 h & 12 h\\
\textbf{4} & 2 hours and 30 minutes. & 10 hours and 30 minutes\\
\textbf{5} & 2 h & 10 h\\
\textbf{6} & 1 h 30 min. & 9 hours and 30 minutes\\
\end{tabularx}

\medskip{}

An abrupt noise grants an Awareness check of DC 15, or equal to the opponent's +8 Stealth check, to wake up.\index{Waking up by noise}\index{Noise in the night}

\end{multicols}

\vfill

\begin{center}
\includegraphics[width=0.45\linewidth]{immagini/mappaparigi.png}

\emph{Ancient map of Paris}
\end{center}

\pagebreak

\subsection{Adventures and Traps}\index{Traps}\label{trappole}


\begin{changemargin}{0.3cm}{0.3cm}\begin{enfasi}{
Anyone who always places the trap in the same place will not catch any iguanas. (African Proverb)}\end{enfasi}\end{changemargin}\medskip


\begin{multicols}{2}

Almost everywhere you can encounter a trap. Traps can be magical or mechanical in nature. Mechanical traps include pits, arrows, falling rocks, water-filled rooms, spinning blades, and anything else that depends on a mechanism to operate. Magical traps are magical trap devices or trap spells. Trap magic devices generate the effects of a spell when activated. Trap spells are spells such as glyph of ward and symbol that function as traps.

\textbf{Traps in the Game}
When adventurers encounter a trap, you should know how the trap activates and what it does, as well as have an idea of ​​how the characters can detect the trap and be able to disarm or avoid it.

\subsubsection{Trigger a Trap}
Most traps are triggered when a creature comes to a place or touches something that the trap's creator wanted to protect. Normal activation systems are pressure pads or false sections of floor, pulling a cable, turning a handle and using the wrong key in the lock. Magical traps often trigger when a creature enters an area or touches an object. Some magical traps (such as the glyph of ward spell) have more complex activation conditions, including the use of passwords to prevent the trap from being activated.

\subsubsection{Detect and Disable a Trap}
Usually, some elements of a trap are clearly visible upon careful inspection.

The trap's description specifies the checks and DCs needed to detect it, disable it, or both. A character actively seeking a trap can attempt a \textbf{Survival} check against the trap's DC.

The Storyteller can also compare the DC to detect the trap against the characters' Survival scores (roll 8) to determine whether a party member notices the trap. If the adventurers notice the trap before activating it, they may attempt to disarm it, either permanently or long enough to allow them passage.

The Storyteller may require a Disable Device check. If you don't have burglary tools\index{Burglary tools} or adequate ones, you do the test with a -1d6 penalty. \index{Deactivate devices without tools}The Survival skill can also be used albeit with a -2d6 to deactivate a trap, padlock..., in this case the duration of the operation is equal to 1 Action per DC of the trap.

If you want to temporarily disable a trap, add 6 to the difficulty. This will disable the trap for 2d4 minutes.

A magical trap can be disabled with a Disable Device check as long as the Arcana value is at least 1/4 of the trap's DC in addition to any other checks listed in the trap's description. The dispel magic spell has a chance to nullify most magical traps.\index{Disable magic traps}

If the check to deactivate or disable the trap fails\index{Failure to disable trap} and you roll a critical failure (two 1s or 2 twos and a 1 on the check) the trap is triggered.

In most cases, the description of the trap is clear enough that the Storyteller can judge whether a character's actions locate or foil the trap.

Use common sense, drawing on the description of the trap to determine what happens. No trap design could ever be able to anticipate every possible action the characters might attempt to take.

The Storyteller should allow a character to discover a trap without having to make proficiency checks if his action or description of what he does would clearly reveal the presence of the trap.

Foiling traps can be a little trickier. Let's take the case of a chest defended by a trap. If the chest is opened without pulling on the two handles on the sides, a mechanism located inside fires a barrage of poisoned needles towards anyone in front of it.

After inspecting the chest and making some tests, the characters are still not sure if he is equipped with traps. Rather than opening the chest, they point a shield in front of it and open it remotely using an iron rod. In this case, the trap is activated, but the barrage of needles is fired at the shield without harming anyone.

Traps are often designed with mechanisms that allow them to be disarmed or bypassed.

\subsubsection{Trap Effects}
The effects of traps can range from simple inconveniences to lethal. A trap's description specifies what happens when it is triggered.
A trap's attack bonus, the saving throw DC to resist its effects, and the damage it deals can vary based on the trap's threat.

Use the Trap Saving Throw and Attack Bonus DC table and the Damage Severity by Level table as suggestions for the three levels of trap severity.

\medskip

\begin{center}
\includegraphics[width=0.9\linewidth]{immagini/medusa.png}
\end{center}


\textbf{Table: DC of Saving Throws and Trap Attack Bonus}\index[Tables]{Table DC of Saving Throws and Trap Attack Bonus}

\medskip

\begin{tabularx}{0.45\textwidth}{XXX}
Trap Hazard & Saving Throw DC & Attack Bonus\\
\toprule
Setback&13-14&+4 to +6\\
Dangerous&16-20&+8 to +10\\
Deadly&21-26&+12 to +15\\
\end{tabularx}

\medskip

\textbf{Table: Damage Severity by Level}\index[Tables]{Damage Severity Table by Level}

\medskip

\begin{tabularx}{0.45\textwidth}{XlXX}
Character Level&Mishap&Dangerous&Deadly\\
\toprule
1st-4th&1d10&2d10&4d10\\
5-10&2d10&4d10&10d10\\
11-16&4d10&10d10&18d10\\
17-20&10d10&18d10&24d10\\
\end{tabularx}

\medskip

\subsubsection{Complex Traps}
Complex traps function like normal traps, except that once activated they perform a series of actions each round.

A complex trap turns the process of dealing with a trap into something more like a combat encounter. When a complex trap activates, roll its initiative.

The trap description includes an initiative bonus. During its round, the trap activates again, often performing an action, be it an attack, an effect that changes over time, or a dynamic challenge. Otherwise, the complex trap can be located and disabled in the usual ways.

\subsubsection{Example Traps}
\textbf{Poisoned Needle}

Mechanical trap

A poisoned needle is hidden inside the lock of a chest, or other openable object. Opening the chest without the proper key will trigger the needle, which dispenses a dose of poison.

When the trap is activated, the needle extends 7.5 centimeters from the lock. A creature within range takes 1 piercing damage and 11 (2d10) poison damage, and must succeed on a DC 20 Fortitude save or take -1d6 on attack rolls and -1d6 on Expertise checks for 1 hour.

The character who succeeds on a DC 22 Survival check can deduce the presence of the trap from the modifications made to the lock to accommodate the needle. A successful Disable Device check using burglary tools disarms the trap by removing the needle from the lock. \textbf{A failed check to pick the lock triggers the trap}\index{Trap, fail check}. Declaring to stick a stick in the lock is equally effective in disabling the trap.

\medskip

\textbf{Poisoned Darts}

Mechanical trap

When a creature steps on a hidden pressure plate, poison darts are fired from a spring-loaded mechanism or pressurized tubes cunningly hidden within the surrounding walls. An area might feature multiple pressure plates, each connected to its own set of darts.

The tiny holes in the walls are hidden by dust and cobwebs, or cleverly hidden among the bas-reliefs, murals or frescoes that adorn the room. The DC of the check to notice them (Survival) is 18.

The character who passes a Survival check with DC 18 can deduce the presence of the hidden pressure plate from the differences in the flooring it is made of compared to the rest of the floor.

Wedging a metal spike or other object under the pressure plate prevents the trap from activating. Filling the holes with fabric or wax prevents the darts contained inside from escaping.

The trap is activated when more than 10 kilos of weight is placed on the pressure plate, thus causing four darts to fire. Each bolt makes a ranged attack with a +10 attack bonus against a random target within 10 feet of the pressure plate (line of sight has no impact on this attack roll).

If there are no targets in the area, the bolt hits nothing. A target hit takes 2 (1d4) piercing damage and must make a DC 18 Fortitude save and take 11 (2d10) poison damage on a failed save, or half as much damage on a successful one.


\medskip

\textbf{Fosse}

Mechanical trap

We present below four basic types of pits.

\medskip

\emph{Simple Pit}

The simple grave is a hole dug in the ground. The hole is covered by a large fabric anchored to the edges of the pit and camouflaged with earth and debris.
The DC to notice the pit is 12. Anyone who steps on the fabric falls into the pit and pulls the fabric along with them, taking damage based on the depth of the pit (usually 10 feet, but some pits are deeper).

\medskip

\emph{Hidden Pit}

This pit has a cover made of material identical to that of the surrounding floor.
Passing an Awareness check with DC 18 you notice the absence of traces in the section of floor that forms the cover of the pit.

You must succeed on a DC 18 Survival check to confirm that that section of floor actually covers a pit.

When a creature steps on the cover, it swings open like a trapdoor, sending the intruder tumbling into the pit below. The pit is usually between 3 and 6 meters deep, but it can be even more.

Once the pit has been located, an iron spike or similar object can be driven between the pit cover and the surrounding ground to prevent the cover from opening, making passage safe. The cover can also be magically kept closed using the Magic Lock spell or similar spells.

\medskip
\emph{Snap Pit}

This pit is identical to the Hidden Pit Trap, with one key exception: the trapdoor covering the pit hides a spring-loaded mechanism. After a creature falls into the pit, the cover snaps shut to trap the victim inside.

You must succeed on a DC 20 Strength check to force the cover open. The cover can also be destroyed. A character inside the pit can also attempt to disable the spring mechanism from within by succeeding at a Disable Device check on DC 18 and using thieves' tools, as long as he can reach and see the mechanism in question. In some cases, another mechanism causes the pit to reopen.

\medskip

\emph{Pit with Spikes}

The pit is a simple, hidden or click pit, at the bottom of which there are wooden spikes or iron spikes. A creature that falls into the pit takes 11 (2d10) piercing damage from the spikes, in addition to the falling damage.

More cruel versions of this trap have poison sprinkled on the spikes located at the bottom of the pit. In that case, anyone who takes piercing damage from the spikes must also make a DC 16 Fortitude save and take 22 (4d10) poison damage on a failed save, or half as much damage on a successful one.


\medskip

\textbf{Falling Net}

Mechanical trap

This trap uses a wire to release a net hanging from the ceiling.

The cable is placed 7 centimeters from the ground and extends between two columns or trees. The net is hidden by cobwebs or foliage. The DC (Survival) to notice the cable and the net is 15. A successful Disable Device check with DC 20 using thieves' tools disables the cable.

A character without thieves' tools can still attempt the check with -1d6 using a sharp weapon or tool. If the check fails, the trap is activated.

When the trap is activated, the net is released covering a 3 meter square area. All creatures in the area are trapped by the net and are entangled, while those that fail a Fortitude save, with a DC 13 Strength modifier, also fall prone.

A creature can use 2 Actions to make a DC 13 Strength check, freeing itself or another creature within reach if it succeeds.

The network has Defense 10 and 20 Hit Points. Dealing 5 slashing damage to the net destroys a 3-foot square section of it, freeing any creatures trapped in that section.

\medskip

\textbf{Rolling Sphere}

Mechanical trap

When 10 or more pounds are placed on the trap's pressure plate, a hidden trap door in the ceiling opens, releasing a 10-foot-diameter sphere made entirely of stone.

By passing a DC 20 Survival check a character can notice the trapdoor and pressure plate. If an examination of the floor is accompanied by a successful DC 20 Survival check, it will reveal the presence of the pressure plate through the difference in the structure of the flooring that accommodates it. The same test carried out while checking the ceiling will reveal the presence of a trap door. Wedging an iron spike or other object under the pressure plate will prevent the trap from activating.

Activating the sphere causes all creatures present to roll initiative. The sphere rolls initiative with a +8 bonus.

During its round, the sphere moves 60 feet in a straight line. The sphere can move through a creature's space, and creatures can move through the space it occupies, treating it as difficult terrain.

Whenever the sphere enters a creature's space or a creature enters its space while the sphere is rolling, the creature must succeed on a DC 15 Reflex save or take 55 (10d10) bludgeoning damage and be knocked prone.

The ball stops when it hits a wall or similar barrier. It cannot turn corners, but skilled dungeon builders incorporate slight bends and curvilinear turns into nearby passages that allow the sphere to continue moving.

With 2 Actions, a creature within 3 feet of the sphere can attempt to slow it with a successful DC 20 Strength check. If the check is successful, the sphere's speed is reduced by 10 feet. If the sphere's speed drops to 0, it stops moving and is no longer a threat.

\medskip

\textbf{Collapsing Ceiling}

Mechanical trap

This trap uses a wire to collapse the supports holding up an unstable section of ceiling.

The cable is placed 7 centimeters from the ground and extends between the two supports. The DC (Survival) to notice the cable is 13. A successful Disable Device check with DC 20 using thieves' tools disables the cable.

A character without thieves' tools can still attempt the check with -1d6 using a sharp weapon or tool. If the check fails, the trap is activated.

Anyone inspecting the supports can easily deduce that they are only supported. As an action, the character can drop a support and activate the trap.

The ceiling above the cave is in poor condition, and anyone who sees it can see that it is in danger of collapsing. When the trap is activated, the unstable ceiling collapses. All creatures in the area beneath the unstable section must make a DC 20 Reflex saving throw, taking 22 (4d10) bludgeoning damage on a failed save or half as much damage on a successful one. Once the trap is activated, the floor of the area is filled with rubble and becomes difficult terrain.


\medskip

\textbf{Blowing Fire Statue}

Magic trap

This trap is activated when an intruder steps on a hidden pressure plate, releasing a blast of magical flame from a nearby statue.

The DC (Survival) to notice the pressure plate or burn marks on the floor and walls is 20. A spell or other effect that can sense the presence of magic, such as detect magic, reveals a magical aura of invocation around the statue.

The trap is activated when more than 10 kilos of weight is placed on the pressure plate, causing a 30-foot cone of fire to erupt from the statue. All creatures in the cone must make a DC 17 Reflex saving throw, taking 22 (4d10) fire damage on a failed save or half as much damage on a successful one.

Inserting an iron spike or other object under the pressure plate prevents the trap from activating. A Disable Device check at DC 20 (and you must have 3 in Arcana) disables the trap. A dispel magic (DC 17) cast on the statue destroys the trap.


\medskip

\textbf{Enchantment Traps and Dispel Magic}\index{Enchantment Traps and Dispel Magic}

The above traps may come with a spell that activates with the trap.
The saving throws to resist the spell are the same as for the spell cast by object or as indicated in the trap description.

A Dispel Magic cancels the enchantment on the trap if it has a Challenge Rating of 2 or less and disables its magical effect for 10 minutes if it has a Challenge Rating of 3.
An Advanced Dispel Magic cancels the enchantment on the trap if it is CR 4 or lower and disables its magical effect for 10 minutes if it is CR 5. In case of a Magical Critical in the casting of the spell, a greater degree of trap is acted upon.


\bigskip

\subsubsection{Other examples of traps}

More traps are here presented for your delight.


\medskip

\textbf{Small legend}:

Challenge Rating: indicates the challenge rating of the trap

Type: Whether the trap is mechanical or magical

DC Survival: What is the trial and difficulty to reveal the trap

DC Deactivate Devices: What is the trial and difficulty of deactivating the trap.

Activator: whether it activates by contact or distance

Reset: Whether the trap can be reset once triggered

Effect: What is the effect of the trap


\medskip


\textbf{Poisoned Dart}

Challenge level: 1

Type: Mechanical

DC Survival: 20

DC Disable Devices: 20

Activator: contact

Recovery: none

Effect: Ranged attack 40 feet +10 (1d3 damage plus Lucos' Fermented Slime, pg \pageref{bavadilucos})


\textbf{Arrow}

Challenge level: 1

Type: Mechanical

DC Survival: 20

DC Disable Devices: 20

Activator: contact

Recovery: none

Effect: Ranged attack 40 feet +15 (1d8+1/×3)


\textbf{Fossa}

Challenge level: 1

Type: Mechanical

DC Survival: 20

DC Disable Devices: 20

Trigger: location

Reset: manual

Effect: 10-foot-deep pit (2d6 falling damage)

Save: Reflexes DC 20 avoid

Target: Multiple targets (all targets 3 meter radius)


\textbf{Scything Blade}

Challenge level: 1

Type: Mechanical

DC Survival: 20

DC Disable Devices: 20

Trigger: location

Reset: manual

Effect: Melee attack +10 (1d8+1/×3)

Target: Multiple targets (all targets in a line within 3 meters)


\textbf{Pit with Spikes}

Challenge level: 2

Type: Mechanical

DC Survival: 20

DC Disable Devices: 20

Trigger: location

Reset: manual

Effect: 10-foot-deep pit (2d6 fall damage) + spikes (Melee attack +10, 1d4 spikes per target for 1d4+2 damage each)

Save: Reflexes DC 20 avoid

Target: multiple targets (all targets in a 3 meter square)


\textbf{Hot wave}

Challenge level: 2

Type: magical

DC Survival: 26

DC Disable Devices/Arcana: 6/26

Activator: proximity (Alarm)

Recovery: none

Effect: 2d4 fire damage

TS: Reflexes DC 11 half

Target: Multiple targets (all targets in a 6 meter long and 3 meter final cone)


\textbf{Javelin}

Challenge level: 2

Type: Mechanical

DC Survival: 20

DC Disable Devices: 20

Trigger: location

Recovery: none

Effect: Ranged attack 40 feet +15 (1d6+6), range 20 feet


\textbf{Acid Arrow}

Challenge level: 3

Type: magical

DC Survival: 27

DC Disable Devices/Arcana: 7/27

Activator: proximity (Alarm)

Recovery: none

Effect: Ranged attack 50 feet (2d4 acid damage for 4 rounds)


\textbf{Hidden Pit}

Challenge level: 3

Type: Mechanical

DC Survival: 25

DC Disable Devices: 20
Trigger: location

Reset: manual

Effect: Medium deep pit (3d6 falling damage)

Save: Reflexes DC 20 avoid

Target: multiple targets (all targets in a 3 meter square)


\textbf{Electric Arc}

Challenge rating: 4

Type: magical

DC Survival: 25

DC Disable Devices/Arcana: 20/5

Activator: contact

Recovery: none

Effect: Arc flash, 4d6 electricity damage

TS: Reflexes DC 20 halves

Target: Multiple targets (all targets in a line 6 meters away)


\textbf{Wall Sickle}

Challenge rating: 4

Type: Mechanical

DC Survival: 20

DC Disable Devices: 20

Trigger: location

Recovery: automatic

Effect: Melee attack +20 (2d4+6)


\textbf{Falling Block}

Challenge rating: 5

Type: Mechanical

DC Survival: 20

DC Disable Devices: 20

Trigger: location

Reset: manual

Effect: Melee attack +15 (6d6)

Target: multiple targets (all targets in a 3 meter square)

\textbf{Fiery Strike}

Challenge rating: 6

Type: magical

DC Survival: 30

DC Disable Devices/Arcana: 30/8

Activator: proximity (Alarm)

Recovery: none

Effect: 8d6 fire damage, range 10 feet

TS: Reflexes DC 17 half

Target: Multiple targets (all targets in a 3 meter radius cylinder)

\textbf{Poisoned Arrow}

Challenge rating: 6

Type: Mechanical

DC Survival: 20

DC Disable Devices: 20

Trigger: location

Recovery: none

Effect: Ranged attack 60 feet +15 (1d6 plus poison ×3)


\textbf{Frozen Fangs}

Challenge rating: 7

Type: Mechanical

DC Survival: 25

DC Disable Devices: 20

Trigger: location

Duration: 3 rounds

Recovery: none

Effect: Range 10 feet (frozen water spray, 3d6 cold damage)

TS: Reflexes DC 20 halves

Target: Multiple targets (all targets in a 3x3x3 meter room)


\textbf{Trap Gas}

Challenge rating: 8

Type: Mechanical

DC Survival: 25

DC Disable Devices: 20

Trigger: location

Recovery: Repairable

Effect: Poisonous gas

Target: Multiple targets (all targets in a 3x3x3 meter room)


\textbf{Arrow Flurry}

Challenge rating: 9

Type: Mechanical

DC Survival: 25

DC Disable Devices: 25

Activator: visual (Arcane Eye)

Recovery: Repairable

Effect: Ranged attack +20 (6d6)

Target: Multiple targets (all targets in a 6 meter line)


\textbf{Concealed Pit with Spikes}

Challenge rating: 8

Type: Mechanical

DC Survival: 25

DC Disable Devices: 20

Trigger: location

Reset: manual

Effect: 50 ft. deep pit (5d6 falling damage) + spikes (Melee attack +15, 1d4 spikes per target for 1d6+5 damage each)

Save: Reflexes DC 20 avoid

Target: multiple targets (all targets in a cube with sides 3x3x3 meters)


\textbf{Dazzling Floor}

Challenge rating: 9

Type: magical

DC Survival: 26

DC Disable Devices/Arcana: 26/7

Activator: proximity (Alarm)

Duration: 1d6 rounds

Recovery: none

Effect: Melee touch attack +9, 4d6 Electricity damage

Target: Multiple targets (all targets in a 6x6x3 meter room)

\textbf{Energy Drain}

Challenge rating: 10

Type: magical

DC Survival: 34

DC Disable Devices/Arcana: 34/9

Activator: visual (True Vision)

Recovery: none

Effect: Ranged touch attack 60 feet +10, max hit points drop by 10d4 + fatigued.

Save: Fortitude DC 23 negates after 24 hours


\textbf{Blade Room}

Challenge rating: 10

Type: Mechanical

DC Survival: 25

DC Disable Devices: 20

Trigger: location

Duration: 1d4 rounds

Recovery: Repairable

Effect: Melee attack +20 (3d8+3)

Target: Multiple targets (all targets in a 3x3x3 meter room)


\textbf{Ice Shard Cone}

Challenge rating: 11

Type: magical

DC Survival: 30

DC Disable Devices/Arcana: 30/8

Activator: proximity (Alarm)

Recovery: none

Effect: Cone of ice spears, 15d6 cold damage

TS: Reflexes DC 17 half

Target: Multiple targets (all targets in a cone 18 meters long and 6 meters trailing)


\textbf{Deadly Spear}

Challenge rating: 18

Type: Mechanical

DC Survival: 30

DC Disable Devices: 30

Trigger: visual

Reset: manual

Effect: Ranged attack 120 feet +20 (1d8+6 plus poison)


\textbf{Hellfire}

Challenge rating: 13

Type: magical

DC Survival: 31

DC Disable Devices/Arcana: 8/31

Activator: proximity (Alarm)

Recovery: none

Effect: 60 Fire damage

TS: Reflexes DC 14 half

Target: Multiple targets (all targets in a 20-foot radius burst)


\textbf{Crushing Boulder}

Challenge rating: 15

Type: Mechanical

DC Survival: 30

DC Disable Devices: 20

Trigger: location

Reset: manual

Effect: Melee attack +15 (16d6)

Target: multiple targets (all targets in a 3 meter square)


\textbf{Empowered Attack}

Challenge rating: 16

Type: magical

DC Survival: 33

DC Disable Devices: 33

Activator: visual (True Vision)

Recovery: none

Effect: +9 ranged touch 60 feet, 30d6 damage, ST: Fortitude DC 19 reduces to 5d6 damage


\textbf{Lightning Gallery}

Challenge rating: 17

Type: magical

DC Survival: 29

DC Disable Devices: 29

Activator: proximity (Alarm)

Duration: 1d6 rounds

Recovery: none

Effect: 8d6 Electricity damage)

TS: Reflexes DC 16 half

Target: all targets in a 12x3x3 meter corridor


\textbf{Poisoned Pit}

Challenge rating: 12

Type: Mechanical

DC Survival: 25

DC Disable Devices: 20

Trigger: location

Reset: manual

Effect: 50 ft. deep pit (5d6 falling damage) + spikes (melee attack +15, 1d4 spikes per target for 1d6+5 damage each plus poison)

Save: Reflexes DC 25 avoid

Target: Multiple targets (all targets in a 3x3 meter square)


\textbf{Meteor Shower}

Challenge rating: 19

Type: magical

DC Survival: 34

DC Disable Devices: 34

Trigger: visual

Recovery: none

Effect: 4 meteors to separate targets, +9 touch ranged 90 feet, 2d6 impact plus 6d6 fire damage

Save: Reflex DC 23 halves fire damage

Target: Multiple targets (four targets, two of which cannot be more than 12m apart)


\textbf{Destruction}

Challenge rating: 20

Magical guy

DC Survival: 34

DC Disable Devices: 34

Activator: proximity (Alarm)

Recovery: none

Effect: Death saving throw

Save: Fortitude DC 23 reduces damage to 5d12 otherwise 10d12

\end{multicols}

\medskip

\begin{changemargin}{0.3cm}{0.3cm}\begin{tcolorbox}[title = Tups and the trap]{\small
In this example I bring you the old school approach when they were supposed to be there of traps. Nothing prevents the Storyteller from allowing Survival checks or Deactivating Devices. I can only say that this approach is more engaging though.

\medskip

\emph{Narrator}: A 3 meter wide corridor leads north, into darkness.

\emph{Tups}: We move forward feeling the floor with our 3 meter pole.

\emph{Narrator}: The pole was left stuck in the collision with the stone idol.
[\emph{If he had used the pole the trap would have been discovered easily}.]
Continue down the corridor?

\emph{Tups}: No, I'm suspicious. Can I see some cracks in the floor, perhaps square in shape?

\emph{Narrator}: No, there are millions of cracks, you can't see a pit that clearly [\emph{Narrator estimates that the pit is well camouflaged and Tups has little lighting to see well}]

\emph{Tups}: Ok, I'll take my water flask from my backpack. I'm going to pour some water on the floor. Does it seem to dig into the floor at any point or reveal some form of texture?

\emph{Narrator}: Yes, the water seems to flow around a square shape, slightly raised on the floor.

\emph{Tups}: Does it look like a covered grave?

\emph{Narrator}: It could be

\emph{Tups}: can I deactivate it?

\emph{Narrator}: how? [\emph{The Narrator deliberately does not make a test, but involves the player}]

\emph{Tups}: I stick the crowbar in it so that the mechanism doesn't open the trapdoor [\emph{Tups doesn't ask you to roll a die to understand how to disarm it or disarm it directly, he explains to the Narrator how he does it and that's it }]

\emph{Narrator}: You cross the area now safely and see that it opens onto a small room with two reinforced wooden doors... }

\medskip

Freely inspired by \href{https://friendorfoe.com/d/Old%20School%20Primer.pdf}{ \textbf{Quick Primer for Old School Gaming}}

\end{tcolorbox}\end{changemargin}

\begin{changemargin}{0.3cm}{0.3cm}\begin{narratore}
A visible/obvious trap forces players to interact with it, to make an effort to understand how it works and to do their best to avoid or deactivate it. When you can, avoid resolutions based only on the dice roll (Look for traps/Disable traps), rather reward the player's simple but creative ingenuity to avoid the danger... and maybe sooner or later they will remember to recover the crowbar. ..!
\end{narratore}\end{changemargin}

\pagebreak

\subsection{Optional - Reputation and Fame}\index{Reputation}\index{Fame}\index{Optional - Reputation and Fame}


\begin{changemargin}{0.3cm}{0.3cm}\begin{enfasi}{
Fame and honor sometimes come more easily to those who do not seek them. (Titus Livy)}\end{enfasi}\end{changemargin}\medskip

\begin{multicols}{2}

While some heroes are content with the rewards of their exploits or hide behind a veneer of humility, others seek to live forever in the sagas and songs of their epic deeds. History measures a hero's success with tales of triumph and daring, repeated for generations.

A hero who cannot recall his story to anyone soon falls into oblivion, along with his untold efforts. The story of the brave deeds becomes the yardstick by which a hero is measured, and sculpts both his identity and his reputation.

Reputation represents how the general public perceives the character positively or negatively. This perception precedes him, speaks for him in his absence and determines how he will be treated by those who have heard of him. Reputation means different things to different character types, based on the social and cultural values ​​of different regions. A character who embodies the qualities of a hero in one region might be considered depraved or immoral in another. An icon widely revered and respected in her homeland could slide from fame to oblivion if she travels to a neighboring kingdom.

When using these reputation rules, the Storyteller must determine what reputation means to the players and NPCs in the campaign. For example, a Viking-themed campaign might base reputation on looting.

If you manage to acquire a strong or notable reputation, you may be praised for your actions and rewarded with resources beyond those obtainable from lesser-known individuals. Likewise, reputation can be used to influence people socially, politically or economically.

Fame rises and falls based on your actions. Current Fame determines your overall reputation, the Sphere of Notoriety defines the places where reputation benefits can be applied.

\end{multicols}

\textbf{Table: how to acquire Fame points}\index[Tables]{Table how to acquire Fame points}

\medskip

\begin{tabularx}{0.95\textwidth}{lX}
\textbf{Events}&\textbf{Mod. Fame}\\
\toprule
\textbf{Positive Events}&\\
Acquire notable treasure from a worthy opponent&+1\\
Consecrate a temple to your Patron&+1\\
Creating a Powerful Magic Item&+12\\
Raise a Level&+1\\
Detect and disarm three or more traps with adequate CR in a row&+1\\
Make a noteworthy historical, scientific, or magical discovery&+1\\
Possess a legendary item or artifact&+14\\
Receiving a medal or similar honor from a public figure&+1\\
Returning a significant magical item or relic to its owner&+1\\
Sacking the stronghold of a powerful noble(enemy)&+1\\
Defeat in single combat an enemy with a CR higher than your level &+15\\
As a group win a combat match with an APL +3 plus&+1\\
Defeat a public slanderer in combat&+2\\
Succeed at a Profession check with DC 30 or more to create a work or object &+2\\
Succeed at a public Intimidate check with DC 30 or more (must be witnesses)&+2\\
Pass a public Perform check with DC 30 or more (must be witnesses)&+2\\
Complete an adventure with a difficulty appropriate for your level&+3\\
Obtain a formal title (lady, lord, knight, etc.)&+3\\
Defeat a key (campaign) rival in combat&+5\\
\textbf{Negative Events}&\\
Being convicted of a minor crime&-1\\
Accompanying yourself with an unbecoming person&-18\\
Being convicted of a serious nonviolent crime&-2\\
Publicly fleeing an encounter with a weaker opponent&-3\\
Attacking innocent people&-5\\
Being convicted of a serious violent crime&-5\\
Publicly losing a match to a weaker opponent&-5\\
Being convicted of murder&-8\\
Being convicted of treason&-10\\
\end{tabularx}

\bigskip

\begin{multicols}{2}

\subsubsection{Fame}

You begin the game with Fame equal to your character level + your Charisma modifier. Fame ranges from -100 to 100, with 0 representing lack of notoriety.

Throughout the campaign, words and deeds help build a reputation. While an adventurer accomplishes many feats, not all are significant enough to warrant a change in Fame. If possible, the Storyteller should stick to those feats that directly affect the story or campaign, and not award points for secondary victories.

The significance of a specific feat should be up to the Storyteller's discretion, but Table: Events of Fame provides some examples. If Fame drops below 0, see Disrepute and Infamy below.

\subsubsection{Sphere of Notoriety}

The reputation of a character travels hand in hand with the story of his exploits. Although he is a great hero in his homeland, when he travels elsewhere he will soon find that his reputation diminishes and that sooner or later he will arrive in regions where he is completely unknown. The higher the reputation, the larger the Influenced Area.

Fame determines the maximum radius of the Sphere of Notoriety. The Sphere of Notoriety has a radius of 150 kilometers, and typically increases by another 150 kilometers when Fame reaches 10, 20, 30, 40, and 55.

Increasing the Sphere of Notoriety is not always automatic, you can express an opinion on where your reputation is concentrated. For example, you might request that your sphere extend further south toward a large city and ignore the barbarian tribes to the east, or that it extend inland toward another country instead of outward to the ocean.

Although reputation can spread by chance, it usually does so deliberately, because wandering storytellers embellish stories of a character's exploits to make them more entertaining, his allies amplify the most common exploits, his enemies repeat gossip about him to hire others and fight it, or the character himself tells his story to pleased listeners.

Where these stories are told determines where you will be known and creates a Sphere of Notoriety: a heroic sorceress might hire Bards to boast of her magic in a nearby kingdom she plans to visit, while an antagonistic Barbarian might push south the wounded survivors of his raids, to spread fear among his next victims.

The following actions and conditions affect your Charisma, Diplomacy, or Intimidate check modifier for the purpose of expanding your Sphere of Notoriety.

\medskip

\textbf{Table: Notoriety Sphere Modifiers}\index[Tables]{Notoriety Sphere Modifiers Table}

\end{multicols}

\medskip

\begin{tabularx}{0.95\textwidth}{Xl}
\textbf{Action}&\textbf{Test Modifier}\\
\toprule
Allies or minions spread stories of the PC's exploits before his arrival & + 5 \\
A Bard spreads stories or songs of the PC's exploits before the PC arrives & + 1/2 Bard's Entertainment Score \\
You have contact with the settlement's NPCs&+1\\
You have enemies in the settlement&+1\\
Distance from own Sphere of Notoriety&-1 for 15 kilometers\\
The primary language of the settlement is different from your own&-5\\
\end{tabularx}

\begin{multicols}{2}

\subsubsection{The Fame Level}

\begin{itemize}


\item The Fame score is what makes the character popular.

\item A fame score within 10 points will make him a local, small-town hero.

\item A fame score between 10 and 20 points will make him a public figure, known to everyone in a small town or a neighborhood hero in a big city.

€12300 € A score between 20 and 30 points makes the character known to everyone even in a big city, his deeds are also known in the region perhaps not in all the details.

€12301 € A score between 30 and 40 is a real celebrity in his city, known by name even in neighboring cities and respected throughout the region.

\item A fame between 40 and 50 points makes the character a true eminence respected in the state.

\item A score over 55 points makes the character a legend whose deeds are passed down and magnified for centuries to come.

\end{itemize}

\subsubsection{Discredit and Infamy}


If your Fame drops below 0, your reputation is based on infamy rather than Fame. Treat Fame as a positive number rather than a negative number for all rules related to Fame, Sphere of Notoriety, and Prestige Points (for example, an antagonist Fame of -20 equals a hero Fame of 20 for his admirers.


If an event can increase Fame, you can choose to increase Fame (bringing it closer to 0) or decrease it (making it a larger negative number). For example, if a character's Fame is 20 and you publicly roll a 30 on a Profession check to create a sword (which is typically +2), you can increase the Fame to 18 or decrease it to 22.

Negative events that decrease Fame always count as negative (an antagonist who attacks innocent people does not inspire sympathy in the audience).

If you have negative Fame, non-evil NPCs will often have hostile or hostile reactions (see Table: Reactions to Negative Fame). Note that if you have a reputation as a powerful and dangerous person, NPCs may avoid you rather than confront you.


\end{multicols}

\textbf{Table: Reactions to Negative Fame}\index[Tables]{Table Reactions to Negative Fame}

\medskip

\begin{tabularx}{0.95\textwidth}{lX}
\textbf{Fame}&\textbf{Reaction}\\
\toprule
-5&Merchants, mercenaries, and innkeepers charge the PC an additional 10\% to discourage doing business in their community.\\
-8&Merchants, mercenaries and innkeepers refuse to do business. The PC who enters a store is immediately asked to leave. If he refuses, the owner calls the authorities or fellow citizens to throw him out.\\
-10&When the PC approaches, the shops close their windows and bar their doors. Most citizens refuse to talk to him. Others urge him to leave immediately. If he stays more than 24 hours or acts blatantly against the citizens, his Fame decreases by 5 and the citizens rally to drive the PC away.\\
-15&Inflamed by the PC's shameless audacity to show up in the community, an angry mob gathers. If the PC does not leave within a few minutes, the crowd begins to bombard him with rotten fruit, branches, and rocks.\\
-20&An angry crowd forms immediately after the PC enters the city. Not wanting to wait for a potentially corrupt trial, they try to capture and execute him for his crimes.
-25&An authority figure has issued an edict of arrest against the PC, including a reward for anyone who captures him. This is well known and many want to collect it.\\
-30&An authority figure has placed a bounty on the PC's head. This is well known and many want to collect it.\\

\end{tabularx}

\vfill

\begin{center}
\includegraphics[keepaspectratio,width=0.55\textwidth]{immagini/Eastern_Story_Teller_1878.png}

\emph{Legends are told. Travelers in the Middle East Archive, Wilhelm Gentz}
\end{center}

\pagebreak

\section{Poisons, Potions and Diseases}\index{Poisons}\index{Potions}\index{Diseases}

\label{veleni-e-pozioni}


\begin{changemargin}{0.3cm}{0.3cm}\begin{enfasi}{
One day, a man was hit by a poisoned arrow. Anxious friends and relatives called a doctor. When they approached him to take the arrow, the man said to them: "Before I do so, I would like to know who pierced me with this arrow... Was he a slave, a king, or a Brahmin? Was he big? Small? Of what color was his skin? Where did he live? And how was the arrow made?

While he was asking himself all these questions... the poison took effect and the wounded man ended up dying. (Buddha)}\end{enfasi}\end{changemargin}\medskip

\begin{multicols}{2}

\subsection{Type of Poison and Potion}\label{tipidiveleno}

\lettrine[lines=2, lhang=0.33, loversize=0.25, findent=1.5em]{Poisons} and potions can be distinguished based on how you come into contact with them.
Not all poisons are toxic if ingested or inhaled.

To identify a natural potion you need a Herbalism check at DC 12 + the rarity of the plant or in the case of Poisons the difficulty is equal to the saving throw of the same. It costs 1 Action every 10 DC or, with Herbalism at 6 or more, it costs 1 Action every 15 DC, with 12 points it costs 1 Action every 20 DC. Unless otherwise described, potions must be drunk (ingested).

\textbf{Contact}: they are contracted when someone touches the poison with their bare skin. Contact poisons usually have an onset time of 1 round. A contact poison can be an ointment, balm, liquid of any density or even powder if specific for contact and not inhalation.

\textbf{Ingestion}: They activate when a creature eats or drinks them. Ingested poisons usually have an onset time of 10 minutes.

\textbf{Injury}: they are transferred mainly with the attacks of some creatures and through weapons sprinkled with poison. Wounding poisons usually have an instantaneous onset time.

\textbf{Inhalation (R)}: are activated when a creature enters an area containing such poisons. Many inhaled poisons fill a volume equal to a cube with a side of 3x3x3 meters per dose. Creatures may attempt to hold their breath while within the area to avoid inhaling the toxin.
A creature can hold its breath for 6 rounds for its Constitution score, with a minimum of 3 rounds and each Action decreases the remaining time by 1 round.
Once the time has elapsed, they must make a Fortitude saving throw at difficulty 12 each round to avoid inhaling the gas. Each round you hold your breath increases the difficulty check by 2.
See also the rules for holding your breath and suffocating in \hyperlink{hold your breath}{Environment}


\subsection{Onset and Effect}\index{Poison Onset}\index{Poison Activation Time}\label{insorgenzaveleno}

Onset means how long it takes for the poison or potion to take effect. If the onset time is 1 Turn it means that due to the effects of the poison/potion and the saving throw it is made after 10 minutes. If onset is not specified in the poison/potion table, it means that the effect is immediate after coming into contact with the poison.

The effect of a poison/potion is immediate upon onset. Check the description of the poison to understand its effect. If the Fortitude save is successful, the poison has had no effect and can be considered neutralized.

There are some cases in which the Frequency item is present, on these occasions the saving throw must be repeated every time the indicated frequency passes, in case of failure of the saving throw the indicated damage is reapplied.

%\begin{center}
%\includegraphics[height=0.5\linewidth]{immagini/potion.png}
%\end{center}

\begin{changemargin}{0.3cm}{0.3cm}\begin{narratore}
The poisons proposed here are some of the many present and possible. Use them as guidelines. If due to your ethics and style you don't like poisons, especially the nastier ones, I suggest using the Generic Potions found at the end of the chapter. They are milder and less personal poisons, probably more easily used by players too.
\end{narratore}\end{changemargin}

\subsubsection{Poisoned}\index{Poisoned}\label{avvelenato}

\textbf{First Dose}: When you are exposed to a poison for the first time (either through your own action or that of someone else), you must make a saving throw with the onset to avoid being poisoned.

\textbf{Success}: Poison is resisted. You suffer no negative effects and no further saving throws are required.

\textbf{Failure}: You have been poisoned and immediately suffer the listed effect.

\textbf{Multiple doses}: If you are exposed to multiple doses of the same poison in the same round the difficulty of the saving throw increases by 1 per additional dose.\index{Poison multiple doses}

\textbf{At different times}: if you are exposed to the poison at different times, each time there will be a new saving throw and you will suffer any effects at the expected onsets.

If you are exposed to different poisons, you must make a saving throw for each type of poison you take.


\begin{changemargin}{0.3cm}{0.3cm}\begin{tcolorbox}[title = Poison ?]
{Poison is a double-edged sword. As long as you use it it's fine but if they use it against you, maybe anyway, it becomes a problem. There are also ethical aspects to using poisons, consider whether your Traits allow you to use poisons and what types.}\end{tcolorbox}\end{changemargin}

\subsection{Apply Poison}\index{Apply Poison}\label{applicareveleno}

Applying poison to a weapon or ammunition requires 3 Actions.

Whenever a character applies or prepares a poison for use he must roll 3d6+Intelligence and if he fails the check he has come into contact with the poison and must make a saving throw against the poison as normal. This does not consume the dose of poison.

Whenever a character attacks with a poisoned weapon, if she fails her attack roll, she exposes herself to the poison's effects. This consumes the poison on the weapon.\index{Critical failure with poisoned weapon}
One potion of poison is enough to cover a medium weapon or 3 arrows in poison. The poison is thus consumed and remains active on the weapon until it hits.

A creature under the effects of a poison, or even if it has not been unleashed, has the Poisoned condition.

\subsection{Removing poison}

The spell \hyperlink{incrimovivenono}{Remove Poison} (page \pageref{incrimovivenono}) removes poisons, and therefore the poisoned condition, which have not yet taken effect as long as the DC of the Poison is lower than the DC of the Remove Poison spell. If the DC of the poison is not expressed then it is considered that simply casting the spell is enough to nullify its effects.

Each Magic Critical obtained with the Test of magic in the casting of the spell is equivalent to +4 in the calculation of the DC see (\hyperlink{spell saving throw}{Saving Throws - Resist the spell}, page €12345{spell saving throw}) for surpass that of poison.

A First Aid check\index{First Aid and Poisons}\index{Poisons and First Aid}, 3 Actions, which is at least half the DC of the poison within the onset time, allows you to make a new saving throw . Once the First Aid test has been taken it is no longer possible to do it again until after the onset.
Continuous First Aid treatment for 8 hours allows you to make a new saving throw with a +1d6 bonus after the poison is activated.

\medskip

\subsection{Create Natural Poisons}\index{Create Natural Poisons}\label{crearevelenonaturale}

Natural poisons can be made using Herbalism. The DC to prepare a poison is equal to the DC of the saving throw which requires -5. If you buy the ingredients, the cost to prepare the potion is half the indicated sales cost, if you look for them in nature the cost for production drops to a quarter. The time to brew these potions/drugs is equal to the DC/2 in hours.

Getting a critical failure on the Herbalism check exposes you to poison during its preparation. If the DC Herbalist's check is successful, 1d2+1 doses are prepared.

The following examples represent just some of the possible poisons. All costs are expressed in Gold Coins.

Poisons are presented, especially in the Monstrorium with this heading: Poison Name, Use (I/R/F/C), Onset Time, Saving Throw DC, Effect.

\begin{center}
\includegraphics[height=0.3\linewidth]{immagini/poison.png}
\end{center}

\begin{changemargin}{0.3cm}{0.3cm}\begin{narratore}
Poisons are part of the long tradition of trouble and adversity in role-playing games. When you want to use a poison, first think about why it is there, who it was supposed to be used for, for what purpose. Not all poisons have to kill, a skilled thief could also use poisons that stun or weaken the will of his target just enough to have the safe opened.
\end{narratore}\end{changemargin}

\end{multicols}

\vfill

\begin{center}
\includegraphics[width=0.4\linewidth]{immagini/funeralebarca.png}
\end{center}



\textbf{Table: Poisons}\index[Tables]{Poisons Table}\label{tabellaveleni}

\medskip

\begin{tabularx}{1\textwidth}{m{4.5cm}lllm{6.5cm}l} %{XlllXl}
\toprule
\textbf{Poison Name} & \textbf{Use} & \textbf{TS} & \textbf{Ins.} & \textbf{Effect (damage)} & \textbf{MO}\\
\toprule
Barsar's Purple Berry\index{Barsar's Purple Berry} & I & 18 & 1 round & Incapable of violence for 3d8 hours & 40 \\
\toprule
Ditch Blue Berries \index{Ditch Blue Berries} & I & 21 & 1 Turn & -1d3 Intelligence and Wisdom for 6 hours& 55\\
\toprule
Fermented Lucos Slime \index{Fermented Lucos Slime}\label{bavadilucos}& F & 15 & - & 1d8 Hit Points & 25\\
\toprule
Yellow Bark Ash \index{Yellow Bark Ash} & F & 15 & 6 rounds & Unconscious for 1d3 hours & 25\\
\toprule
Purple Concentrate \index{Purple Concentrate} & F & 15 & & 2d6 Hit Points & 15\\
\toprule
Daraka Fingers\index{Daraka Fingers} & F & 17 & - & -1d6 Strength, for 1 hour & 35\\
\toprule
Pink Spiky Grass \index{Pink Spiky Grass} & I & 22 & 1 round & -1d6 Dexterity, for 1 hour & 60\\
\toprule
Purple Shrew Liver \index{Purple Shrew Liver} & I & 25 & 1 hour & 2d6 damage to Wisdom and Intelligence. Permanent & 75 \\
\toprule
Mucot's White Ribbon \index{Mucot's White Ribbon} & C & 20 & - & Sleeps for 2d12 hours & 20\\
\toprule
Curna Fumes\index{Curna Fumes} & R & 18 & - & -1d3 Wisdom & 40\\
\toprule
Blue Frost \index{Blue Frost} & F & 18 & & 3d6 Cold Hit Points& 25\\
\toprule
Purple Shrew Fat \index{Purple Shrew Fat} & C & 13 & 1 round & 2d12 Hit Points & 15\\
\toprule
Kreex's Tongue \index{Kreex's Tongue} & F & 20 & - & The wound is bleeding. +1 bleed damage. 1 use in 24 hours. & 50 \\
\toprule
Red Mixture \index{Red Mixture} & F & 13 & - & -1d6 TC/TS for 10 minutes & 10\\
\toprule
Yellow Moss \index{Yellow Moss}& I & 20 & 1 round & the creature gains a bounty. -2 Int and Wis. Duration 10 minutes & 50\\
\toprule
Dennar Kernel \index{Dennar Kernel} & I & 13 & 1 round & -1d2 Strength, for 3 days & 15\\
\toprule
Nabar Oil \index{Nabar Oil} & R-F& 20 & - & Confused for 2d6 rounds & 50\\
\toprule
Blue Toad Hide \index{Blue Toad Hide} & C & 22 & 1 minute & Paralyzed for 1d6 turns & 60\\
\toprule
Omro Rose Pollen\index{Omro Rose Pollen} & I & 15 & - & -1d3 Constitution and Dexterity, for 1 hour & 25\\
\toprule
Ragmor's Scent \index{Ragmor's Scent} & R & 16 & - & -1d3 Charisma, for 1 day & 30\\
\toprule
Thrun's Blood \index{Thrun's Blood} & C & 26 & - & -1d3 Constitution & 80\\
\toprule
Ythis Juice\index{Ythis Juice} & I & 14 & 1 Turn & -1d2 Intelligence, for 1g & 20\\
\toprule
Ottalm Poison\index{Ottalm Poison} & F & 20 & - & Death or -1d2 Permanent Constitution & 50\\
\toprule
Blood Serpent Poison \index{Blood Serpent Poison} & F & 25 & - & Paralysis for 1d6 hours -1d4 Strength points for 7 days & 75 \\
\end{tabularx}

\medskip

\textbf{Application}: \textbf{I}(ingestion), \textbf{F}(exhalation), \textbf{C}(contact), \textbf{R}(exhalation).

The saving throw is always Fortitude unless otherwise specified

Lost ability points are regained at a rate of 1 per day unless permanent or otherwise indicated.

\vfill

\begin{center}
\includegraphics[width=0.25\linewidth]{immagini/mandragola2.png}

\emph{Mandrake Plant}
\end{center}

\subsection{Natural potions}\index{Potions}\label{pozioninaturali}

\begin{changemargin}{0.3cm}{0.3cm}\begin{enfasi}{
I believe that a leaf of grass is no less than a day's work done by the stars. (Walt Whitman)
}\end{enfasi}\end{changemargin}

\begin{multicols}{2}

The time to \textbf{prepare} these potions/drugs is equal to DC/2 in hours, while the difficulty of the Herbalism check is equal to DC -5. If you buy the ingredients, the cost to prepare the potion is half the indicated sales cost, if you look for them in nature the cost for production drops to a quarter.

If the DC Herbalism check is successful, 1d2+1 potions are prepared (from 1 dose).

You cannot benefit from more than one dose of natural potions (of each type) per day, unlike magical ones.

\end{multicols}

\medskip
{\small
\begin{xltabular}{0.95\textwidth}{llllXlc}
\textbf{Name} & \textbf{Use} & \textbf{Ins.} & \textbf{DC} & \textbf{Effect}& \textbf{Loc.} & \textbf{Cost} \\
\toprule
Arduuar\index{Arduuar} & I & 1 round & 25& Removes Poisons & SZ7 & 75 \\
\toprule
Arkasun\index{Arkasun} & C & 1 Turn & 25& Heals 1d6 Hit Points per Turn for 3 turns& TM7 & 75 \\
\toprule
Arlan\index{Arlan} & C & 5 rounds & 15& Heal 1d6+3 Hit Points & TT5 & 50 \\
\toprule
Arlandas\index{Arlandas} & R & 1 hour& 24& Repair fractures & CF5 & 200 \\
\toprule
Attarna\index{Attarna} & I & 1 Turn & 20& Grants a new Disease Saving Throw with a +1d6 & TF7 & 50 \\
\toprule
Ljust Berries \index{Ljust Berries} & I & 1 round & 16& Taken in the evening you recover double the Hit Points minimum 4) & AZ6 & 10 \\
\toprule
Ljust's Kiss\index{Ljust's Kiss} & C & 1 round & 35& Heal 100 Hit Points & HO8 & 500 \\
\toprule
Barannie\index{Barannie} & I & 1 minute & 15& Removes nausea & MD6 & 3 \\
\toprule
Burthelas \index{Burthelas} & I & 1 Turn & 32& Regenerate hands& HD7 & 410 \\
\toprule
Dagmathir Bark\index{Dagmathir Bark Powder} & R & 1 round & 25& Removes one level of Fatigue & SS5 & 15 \\
\toprule
Aklent Bark\index{Aklent Bark} & I & 1 Turn & 10& Bark chewed for at least 10 rounds grants a +1 Saving Throw vs. Poison for the next 24 hours & TM6 & 1 \\
\toprule
Culcoa\index{Culcoa}& C & 1 round & 16& You recover 2d6 from fire damage & Save7 & 15 \\
Darsirion\index{Darsirion} & C & 1 round & 25& Heal 1d4 Hit Points & CM4 & 5 \\
\toprule
Delrean Plus\index{Delrean Plus} & I & 1 round & 18& Keeps insects away for 3 days & CC6 & 5 \\
\toprule
Delrean\index{Delrean} & C & 1 round & 15& Repel insects for 1 day & CC6 & 2 \\
\toprule
Draaf \index{Draaf} & C & 1 round & 20& Heal 1d8 Hit Points & SO6 & 50 \\
\toprule
Eldrin'tail\index{Eldrin'tail}& I & 1 round & 15& Grants a new Poison saving throw & FH7 & 18 \\
\toprule
Illa Berry Extract\index{Burnt Illa Berry Extract}& I & 1 round & 15& +2 Initiative, +2 Dexterity, -1d6 Will save, for 10 minutes & MS6 & 5 \\
\toprule
Gisenosa root extract\index{Gisenosa root extract} & I & 3 shifts & 15& Cure coughs and colds & MT6 & 3 \\
\toprule
Febfendi \index{Febfendi}& C & 1 Turn & 25& Regenerate ears & CF7 & 75 \\
\toprule
Garioe\index{Garioe}& I & 1 round & 25& Heal 2d6 Hit Points& AZ7 & 95 \\
\toprule
Geffnull \index{Geffnull}& I & 5 rounds & 28& Heal 3d8+3 Hit Points & EV8 & 150 \\
\toprule
Gusterbloon \index{Gusterbloon} & C & 1 round & 20& Skin becomes darker granting +1d6 on Stealth checks & CM5 & 8 \\
\toprule
Gylvert\index{Gylvert} & I & 1 minute & 25& Allows breathing underwater for 4 hours & MO7 & 3 \\
\toprule
Harfy \index{Harfy} & C & - I & 12& -1 to bleed & SS6 & 3 \\
\toprule
Harfindar\index{Harfindar} & I & 1 Turn & 15& Aborts& SS7 & 3 \\
\toprule
Jojopo\index{Jojopo}& C & 1 round & 15& You recover 2d6 from cold damage & FM6 & 18 \\
\toprule
Kelventare\index{Kelventare} & I & 1d4 rounds & 28& Recover 2d6 Hit Points & TT7 & 100 \\
\toprule
Klagul\index{Klagul}& C & 1 Turn & 20& Cleans teeth & SS4 & 2 \\
\toprule
Klandor\index{Klandor} & I & I & 15& Removes paralysis. Increases fatigue level by 1& HB6 & 18 \\
\toprule
Klynkyx\index{Klynkyx} & C & 6 Turn & 15& Makes all hair fall out for 1d6+4 days & MO6 & 4 \\
\toprule
White Musk Yeast \index{White Musk Yeast} & I & 1 minute & 12& Baked products using this yeast cause uncontrollable and incredibly smelly bloating for 12 hours & CA3 & 1 \\
\toprule
Red Tongue of Xabax\index{Red Tongue of
\toprule
Melandrir\index{Melandrir} & I & 1 round & 15& Grants a new Disease saving throw with +5 & CF7 & 100 \\
\toprule
Mirenna\index{Mirenna} & I & 1 round & 20& Heal 5 Hit Points & CM6 & 30 \\
\toprule
Mixture 31\index{Mixture 31}& I & 1 Turn & 20&The mount is extremely resistant. +6 hours of canter per day & SM6 & 15 \\
\toprule
Silver Musk\index{Silver Musk}& I & I & 25& Removes Magical Diseases & MU8 & 250 \\
\toprule
Musekiss\index{Musekiss} & C & 1 hour & 30& Regenerate lower limbs & TH9 & 550 \\
\toprule
Nazamuse \index{Nazamuse}& I & I & 30& Removes Poisons and Natural Diseases & EW9 & 175 \\
\toprule
Nelthalion \index{Nelthalion} & I & I & 15& Makes you vomit& SR3 & 1 \\
\toprule
Lisbeth's Petals \index{Lisbeth's Petals} & I & 1 Turn & 15&+2 Intelligence, -2 Dexterity for 10 minutes & MC6 & 20 \\
\toprule
Green Rose Pollen\index{Green Rose Pollen}& R & 3 turns & 25& You recover 2d4 damage Intelligence and Wisdom & FA8 & 35 \\
\toprule
Dried Kathaus Root\index{Dried Kathaus Root} & R & 1 round & 20& +2 Strength & Dexterity for 1 hour & FW6 & 50 \\
\toprule
Rewky\index{Rewky} & I & 1 Turn & 25& Heal 2d8 Hit Points & TD6 & 20\\
\toprule
Siranmuse\index{Siranmuse} & I & 1 day & 30& Regenerate internal organs & SS8 & 850 \\
\toprule
Ucsaboo \index{Ucsaboo} & C & 1 Turn & 30& Regenerate Eyes & MO8 & 400 \\
\toprule
Urk Egg\index{Urk Egg}& I & 1 Turn & 12& 1 Day of Food& FH7 & 1 \\
\toprule
Uscaboo \index{Uscaboo} & R & 1 Turn & 25& Removes blindness & MO7 & 125 \\
\toprule
Wickalim\index{Wickalim} & I & 1 hour & 15& Heal 2 Hit Points & TD3 & 5 \\
\toprule
Yaveth\index{Yaveth}& I & 1 Turn & 20& Heal 2d8 Hit Points& MO5 & 100 \\
\end{xltabular}}


\subsubsection{Notes on Poisons and Potions}

\textbf{Purple Shrew Liver}: poisoning recognizable by the typical bloodshot eyes

\textbf{Fermented Lucos slime}: Lucos is a herbivorous and peaceful lizard. The collected slime must be fermented in the dark for 1 week before being usable.

\textbf{Purple Shrew}: according to many, the Shrew is Cattalm's favorite pet. Aggressive, violent, dangerous in every fiber.

\textbf{Daraka Fingers}: Daraka Fingers are the fruit of the Daraka tree. The elongated, black pod resembles the fingers of the ancient goddess of darkness.

\textbf{Nabar Oil}: the small Nabar berries are exclusively eaten by Shrews, immune to their evil effects. Boiled for a long time it becomes an excellent ointment for the skin.

\textbf{Urk's Egg}: Urk is a large beetle, the egg is slightly larger than a hazelnut. It is usually first smoked with beech wood, eaten raw the flavor is musty and earthy.

\textbf{Mixture 31}: a studied set of drugs for horses. Once the effect ends, the creature must make a Fortitude save DC 23 or fall unconscious for 12 hours.

\textbf{Barsar Purple Berry}: curiosity, the purple shrew is disgusted by these berries.

\textbf{Ottalm Poison}: The Ottalm is a variant of Purple Shrew with a poisonous sting.

\textbf{Yellow Bark Ash}: the bark must first be macerated and beaten in water and salt. The resulting mush must be dried and then heated without burning it directly.

\textbf{Red Tongue of Xabax}: it is the long petal of the Xabax. Of the 7 petals only the long one has the substances necessary to prepare the ointment.

\textbf{Dry Kathaus root}: small black tuber, extremely hard and woody. It is usually left to dry in the sun before grinding it.

\textbf{Lisbeth petals}: extremely fragrant, they resemble those of a rose.

\textbf{Curna fumes}: Curna is the inflorescence of the common thistle.

\textbf{Aklent Bark}: also called \emph{Skunk Bush} for its pungent and characteristic smell.

\textbf{Gisenosa root extract}: artichoke-type plant, extremely thorny. It tends to grow surrounded by \emph{Tribulus terrestris} or \emph{footkisser}.

\textbf{Silver Musk}: very similar, for a non-expert, to White Musk. Berries are harvested.

\subsection{Where to find plants}

Ex: Gusterbloon FT5. The first Letter indicates CLIMATE, the Second indicates ENVIRONMENT, the Third indicates RARITY. Rarity indicates the chance, on a d10, of finding the sought after herb/plant. Roll 1d10 and do more than the indicated number, clearly if the climate and environment match.

\medskip

\textbf{Table: Potions - Places correspondence}\index[Tables]{Potions - Places correspondence table}

\medskip

\begin{tabular}{ll|ll|ll}
\textbf{1' letter} & \textbf{Climate} & \textbf{2' letter} & \textbf{Environment} & \textbf{2' letter} & \textbf{Environment} \\
\toprule
A & Arid & A & Alpine & B & Gorges\\
C & Cold & C & Coniferous Forest & D & Deciduous Forest\\
E & Perennial ice & F & River and stream banks & G & Frozen fields\\
F & Severe Cold & H & Dry Fields &J & Jungle, Rainy Forests\\
H & Humid and hot & M & Mountain & N & Ocean, salt flats\\
M & Temperate & S & Short grass & T & Long grass\\
S & Semi arid & U & Caves & Underground & V & Volcanic\\
T & Cool Temperate & W & Landfills / Waste & Z & Desert\\
X & Unknown & X & Unknown&&\\
\end{tabular}

\subsection{Generic potions}\index{Generic potions}\index{Potions}

The Narrator is free to use all the potions and poisons indicated above or use ready-to-use generic potions, which can be purchased in almost any herbalist or potions shop.

The costs and effects of these generic potions are indicated in the table. The onset is always immediate, the duration for the cures is immediate, for the others it is 1 hour (so the Remove Poison potion immunizes you for 1 hour against a poison). For potions that cause damage the saving throw is to negate their effects.

\textbf{Table: Generic Potions}\index[Tables]{Table of Generic Potions}\label{pozionigeneriche}

\medskip

\begin{tabularx}{0.95\textwidth}{lXcc}
\textbf{Potion Name}& \textbf{Effect}& \textbf{Cost (gp)}& \textbf{Application}\\
\toprule
Heal & recover 1d8+1 Hit Points & 50 & Ingestion\\
Enhanced healing & recovers 3d8+3 Hit Points & 125 & Ingestion\\
Weakening& -1d6 CT. TS DC 15 Fortitude & 34 & Ingestion\\
Enhanced Weakening & -1d6 CT. TS DC 18 Stamina & 50 & Wounding \\
Poison& take 2d6+2 damage. TS DC 15 Fortitude & 30 & Ingestion \\
Poison enhanced & take 2d8+2 damage. Save DC 18 Stamina & 25 & Wound \\
Remove Poison & negates the onset of a poison if taken within the activation, or grants a new saving throw with +1d6 & 75 & Ingestion\\
\end{tabularx}

These generic potions like natural potions only take effect the first time they are taken within 24 hours. If the character dedicates 1 round to drinking a Healing Potion it will have maximized effect.


\subsection{Optional - Drugs}\index{Drugs}\index{Optional - Drugs}\hypertarget{Drugs}{}\label{droghe}

\textbf{Table: Drug List}\index[Tables]{Drug List Table}

\medskip
{\small

\begin{tabularx}{0.99\textwidth}{XlllXrr}
\textbf{Name} & \textbf{Use} & \textbf{Ins.} & \textbf{DC} & \textbf{Effect}& \textbf{Loc.} & \textbf{Cost} \\
\toprule
Fermented Luside Leaves\index{Fermented Luside Leaves} & I & 1st Turn & 17& Sensory hallucinations for 2d4 hours. +2 Charisma & Intelligence & SF7 & 5 \\
\toprule
Ferpillon \index{Ferpillon}& I & 1 round & 20& Sleeps for 24 hours& SC5 & 50 \\
\toprule
Gray Anointed \index{Grey Anointed} & I & 1 round & 24& Removes mental conditioning caused by spells lower than 5th level& AH9 & 80 \\
\toprule
Arpasur's Ash \index{Arpasur's Ash} & R & 1 round & 20& Removes 2 levels of fatigued & FT6 & 10 \\
\toprule
%Purple Shrew Dried Meat \index{Purple Spider Dried Meat} & I & 1 round & 24& +4 Strength -4 Intelligence (minimum -3) for 1 turn& SH7 & 30 \\
%\toprule
Melzaa Alcoholic Extract\index{Melzaa Alcoholic Extract} & I & 1 round & 20& +1d4 Strength, +1d4 Dexterity. -1d6 Will save. For 3 hours & AF6 & 25 \\
\toprule
Inut scented essence\index{Inut scented essence} & R & I & 15& +2 Intelligence, for 1d8 hours& HB6 & 15 \\
\toprule
Julnnaus Pollen\index{Julnnaus Pollen} & R & I & 20& +3 Constitution for 2 hours & ST6 & 25 \\
\toprule
Erain Flower Pollen \index{Erain Flower Pollen} & R & 1 round & 20& +2 Strength and Intelligence and Dexterity. +3d6 temporary hit points, for 1 hour & FT7 & 75 \\
\end{tabularx}
}
\begin{multicols}{2}

\medskip

\textbf{The use of drugs is completely optional, the Narrator decides their presence and availability also based on the sensitivity of the players}.

Drugs are addictive. Once the effect wears off within 24 hours, make a Will save at difficulty 15 or take another dose, the next saving throw will have difficulty +1 and so on.

Each time you take a new dose within 2 weeks of the first, the saving throw to avoid becoming addicted increases by 1. Not taking a dose increases your Fatigue level by one.

It requires 7 successful saving throws in a row to end the addictive effect.

\subsection{Optional - Drinking too much}\index{Drinking too much}\index{Optional - Drinking too much}\hypertarget{alcoholism}{}\label{beretroppo}\index{Drunks}

A creature can drink a number of mugs of ale equal to its Constitution score. Any new mug forces the creature to make a Fortitude saving throw at DC 11, each subsequent mug increases the saving throw by +2. When the saving throw fails, the creature is intoxicated and is considered to be under the effect of the confusion spell for a number of turns equal to the failed save score.

Higher strength beers or spirits require a more difficult saving throw.

The Storyteller can decide to manage a \emph{tipsy} creature only through \emph{roleplaying}.

\subsection{Diseases}\index{Diseases}\hypertarget{Diseases}{}\label{malattie}

In principle, diseases are managed like poisons, you make a saving throw to check if you are infected and other saving throws to heal.
Usually the trigger time of a disease is not as immediate as a poison and yet the magical ones can be disruptive and act in a few minutes.

Each disease must have indicated the time of onset, the initial saving throw, how often the saving throw must be redone, how many successes on the saving throw are necessary to heal, the effects suffered.

E.g. Minor Demonic Fever: 1 minute, Fortitude save DC 18, 6 hours, 3 successes, -1 Constitution and Wisdom

Lesser demon fever forces a Fortitude save at DC 18 after just one minute of having it. Subsequently every 6 hours the saving throw must be redone and the disease remains until at least 3 consecutive successes have been made on the saving throw. Every 6 hours the sick person loses 1 point of Constitution and Wisdom.

To recover from a non-natural disease, such as those afflicted by monsters, it is necessary to pass the required saving throws or have a spell of \hyperlink{removediseases}{Remove Diseases} available (page \pageref{removediseases}). It is not necessary to succeed the saving throws consecutively, but you must obviously reach the number before dying.

A test of \textbf{First Aid}\index{First Aid and Illnesses}\index{Illnesses and first aid}, with a DC equal to at least half the DC of the disease (or 15 if not indicated), carried out between Saving throw and subsequent, allows you to have a +1 on the saving throw to resist the effects of the disease. An entire night's treatment grants +1d6 on the next saving throw.

The Remove Disease spell grants healing from the affliction as long as the spell's casting DC is higher than the disease's DC.

Each Magic Critical obtained with the Test of magic in the casting of the spell is equivalent to +4 in the calculation of the DC see (\hyperlink{spell saving throw}{Saving Throws - Resist the spell}, page \pageref{spell saving throw}) for overcome that of the disease.

Being affected several times by the same disease does not increase the difficulty of recovery nor does it change its times and effects.

Examples of Diseases:

\textbf{Demonic Influence}: 1 minute, save Fortitude DC 16, 12 hours, 2 successes, -1 Constitution\index{Demonic Influence}

\textbf{Rezh's Corruption}: 1 day, Will save DC 18, 1 hour, 2 successes, -1d6 Maximum Hit Points\index{Rezh's Corruption}

\textbf{Fungal Plague}\index{Fungal Plague}: 8 hours, Fortitude save DC 24, 12 hours, 2 successes, -1 point to Dexterity and Intelligence

\textbf{Violent Torpor}\index{Violent Torpor}: 24 hours, Will save DC 12, 12 hours, 1 success, +1 Melee Weapon Damage and -1 Wisdom

\textbf{Minor Demonic Fever}\index{Minor Demonic Fever}: 1 minute, Fortitude save DC 18, 6 hours, 3 successes, -1 Constitution and Wisdom

\textbf{Black Blood}\index{Black Blood}: 10 minutes, Fortitude save 28, 12 hours, 1 success, loss of half hit points remaining

\textbf{Plague T}\index{Plague T}: 1 minute, Fortitude save 30, 2 hours, 3 successes, achieve 3 consecutive successes otherwise you are transformed into a zombie. It is transmitted through wounding.

\end{multicols}

\vfill

\begin{center}
\includegraphics[width=0.18\linewidth]{immagini/plaguedoctor.png}

\emph{Engraving of the Plague Doctor, Paul Furst, 1656}
\end{center}


\pagebreak

\section{Movement and Transport}\index{Transport}\index{Movement}

\label{movimento-e-trasporto}

\begin{changemargin}{0.3cm}{0.3cm}\begin{enfasi}{
My left foot works great, but I still wouldn't be able to walk if it weren't for my right foot! (Madagascar 3 - Europe's Most Wanted, Film)

\medskip

When you can no longer run, walk fast; when you can no longer walk fast, walk; when you can no longer walk, use a cane; but never hold back. (Mother Teresa of Calcutta)}
\end{enfasi}\end{changemargin}\medskip


\begin{multicols}{2}

\lettrine[lines=2, lhang=0.33, loversize=0.25, findent=1.5em]{I}{l} movement can be distinguished based on which situation it applies.

\medskip

\begin{itemize}
\item Tactical, when fighting, use precise distances, map and squares of 1 meter on each side
\item Local, to explore an area, measured in meters per minute.
\item By land, to move from one place to another, measured in km per hour or day.
\end{itemize}

\subsection{Movement Types}\label{tipodimovimento}

When moving in different movement situations (Tactical, Local Overland), creatures generally walk or run.

\textbf{Walking}:\index{Walking} Walking represents an unhurried but decisive movement of approximately 4 km per hour for an unencumbered human. Per Move Action the creature travels the distance indicated in Move.

\textbf{Running}\index{Running}: It means moving about 12 km per hour for a human.

Running as a move action doubles your movement speed.
The running character has a penalty of 1d6 on attack rolls and 4 on defense in the round in which he runs.
Only in non-combat situations does running triple movement, i.e. when using Local or Overland Movement.

\subsection{Table: Movement and Distance and Speed: on foot}\index{Movement on foot}\index[Tables]{Table Movement and Distance and Speed: on foot}

This table shows basic ground movement values ​​in non-combat situations.

\medskip

\begin{tabularx}{0.43\textwidth}{lccc}
\multirow{2}*{Type of movement} &
\multicolumn{3}{c}{Movement} \\
\cmidrule(lr){2-4} & 6m & 9m & 12m \\
\midrule
\multicolumn{4}{c}{\textbf{Movement (Tactical)}}\\
Walking & 6m & 9m & 12m \\
Run (x2) & 12m & 18m & 24m \\
\multicolumn{4}{c}{\textbf{One minute (Local)}} \\
Walking & 36m & 54m & 72m \\
Running (x3) & 108m & 162m & 216m \\
\multicolumn{4}{c}{\textbf{One hour (By Land)}} \\
Walking & 3km & 4km & 6km \\
Running (x3) & 9km & 12km & 18km \\
\multicolumn{4}{c}{\textbf{One day (By Land)}} \\
Walking & 24km & 32km & 54km \\
\end{tabularx}


\subsection{Tactical Movement}\index{Tactical Movement}\label{movimentotattico}

Tactical Movement is used during combat.
Distances are measured in one meter squares, movement is managed through Movement Actions.

A character can use 1 (Move) Action to move up to its full range of movement. He can perform the Move Action several times in the round, up to 3 times, thus moving triple his movement.

He can also perform a Dash Action € 12680 € {Dash} or € 12681 € {Run} and therefore move double his Movement in a single Action. However, he thus incurs penalties for running (-1d6 to attack rolls, -4 Defense).

A character can perform up to 3 Sprint Actions, i.e. he runs for the entire round thus completing his movement * 6.

\subsubsection{Movement Obstructed}\index{Difficult Terrain}\label{terrenodifficile}

Difficult terrain, obstacles or poor visibility can impede movement. When movement is hindered you move at half speed, 2 Actions are needed to cover your distance of 9 meters (if you are human without encumbrance..), or with a move Action you only cover 4 meters.

If more than one particular condition exists, add any applicable additional costs together, i.e. if the terrain is difficult and you are moving on all fours it means moving a quarter of your movement.

In some situations movement is so hindered\index{Movement almost impossible} that the distance that can be covered per Action is minimal, in which case all 3 Actions can be used to move only 1 meter in any direction.

Do not apply this rule to cross impassable terrain or to move when it is not possible to do so in any way.

You cannot \textbf{Sprint} (Run) or \textbf{Charge} \index{Charge on Difficult Terrain} \index{Sprint on Difficult Terrain} easily across a \textbf{path that hinders movement} , or difficult terrain. The player can attempt a DC 20 Acrobatics check to be able to charge or run, but only cover half the distance. The Acrobatics check is not necessary, even if you complete half the movement, if you have the Advantage \hyperlink{rhinoceros}{Rhinoceros}.

Moving prone\index{Moving prone}\index{Moving on all fours}, Swimming or Crawling\index{Crawling} is considered difficult terrain, Climbing is doubly difficult.

Terrain where the bodies of creatures are present is considered difficult\index{Moving on bodies}.

\subsubsection{Through enemies}\index{Through enemies}\index{Cross occupied squares}\label{attraversonemici}\index{Juggle}

\index{Crossing through enemies}\index{Movement through}A character can \textbf{crossing} but not stopping in \textbf{an area occupied} by a companion without being \hyperlink{restricted}{\textbf {restricted}}\index{Restricted}. 

To cross terrain where there is a hostile creature, you must make an Opposed Dexterity or Strength check (your choice) with the creature you want to \textbf{cross} the terrain. \textbf{Crossing the space occupied by an enemy costs 1 Action, in addition to the movement made}. The terrain occupied by the hostile creature is considered difficult.

If you fail, you remain in the immediately previous square, with the risk of being restricted. Both the move action and the pass action are considered completed.
If the enemy has the Feat \hyperlink{opportunist}{Opportunist}, in addition to obstructing the passage, he can perform an attack (1 Reaction).

A medium-sized or smaller creature can share the same square with a small-sized creature.

\begin{changemargin}{0.3cm}{0.3cm}\begin{tcolorbox}[title = Tups in the tunnel] %player box
Tups is with his companions in a narrow tunnel in single file. He's in fourth place.

Suddenly an enemy appears in front of him and Tups is the fastest to react, using a Movement Action \emph{\textbf{crosses}} the 3 companions in front of him, remaining \textbf{restricted} with the first in the line.

He could decide to (among various possibilities):

\smallskip

- delay the Action and stay still, let the companions in front of him flow first. 

\smallskip

- stay narrow and attack. 2 Shares remain.

\smallskip

- push the partner (1 Action) into the previous square, making him squeeze with another partner. 1 Action remains.

\smallskip

- push your partner (1 Action) into the next square! making him cross the enemy's square. 1 Action remains.

\smallskip

- go back (1 Action) to its initial square. 1 Action remains.

\smallskip

-try to cross the opponent (1 Action), but if he fails he would be restricted with his partner, damaging both and he would only have 1 more Action left

\end{tcolorbox}\end{changemargin}

\subsubsection{Being restricted with someone}\index{Being restricted with someone}
Two restricted creatures, i.e. those that share the same square (and are not small in size) suffer a -1d6 to attack roll and a -4 to Defense while restricted.\index{Restricted}

\subsubsection{Exchange places}\index{Exchange places}
A character in contact with another creature can use \textbf{an Action} to \textbf{swap places} with it. If the creature is hostile, an Opposed Strength Test is required to succeed in the exchange. For each size difference, whoever has the larger one gets +1d6 bonus on the check. Costs 1 Action.

%\begin{changemargin}{0.3cm}{0.3cm}\begin{narratore} %box narrator
%If you want raw realism then it is difficult terrain to cross even areas where there are friendly creatures. \end{narratore}\end{changemargin}

\subsubsection{Pass through bottlenecks or restrictions}\index{Pass through bottlenecks or restrictions}\index{Bottlenecks}\index{Restrictions}

Passing through a gap one size smaller is equivalent to moving through difficult terrain. For example, a medium creature, occupying 1 square, that must pass through a half-square (half-meter) narrow passage treats that path as difficult terrain.

You can't pass through restrictions narrower than half the creature's size.

\subsection{Local Movement}\index{Local Movement}\label{movimentolocale}

Characters exploring an area use local movement, measured in meters per minute.

In these situations it is not essential to measure the distance precisely but as soon as the situation becomes \emph{problematic} or requires attention the map converts into tactical movement, gridded and measured.

\medskip

\begin{itemize}
\item
Walking: A character can safely walk in Local Movement for 8 hours per day.
\item
Running: A character can Run for a number of minutes equal to three times their local Movement Constitution score without needing to rest (minimum 1 round).
\end{itemize}


\subsection{Land Movement}\index{Land Movement}\label{movimentoviaterra}

Characters traveling long distances use Overland movement. Overland movement is measured in hours or days. One day represents 8 hours of real travel time. For rowing boats, one day means rowing for 10 hours. For sailing ships it represents 24 hours of movement.

Walking for longer can be exhausting (see Forced March, below).

\textbf{Going Fast}\index{Going Fast}\label{andareveloci}

You can go fast (movement*2) for 1 hour without any problems. Speeding for a second hour between two sleep cycles deals 1 nonlethal damage, and each additional hour deals double the damage taken in the previous hour. A character who takes nonlethal damage from fast pacing is considered fatigued for that day.

A Fatigued character cannot Run or Charge.

\textbf{Running}\index{Running}\label{correre}

You can't Run for a long time. Attempts to Run and Rest in cycles work like Going Fast.

\textbf{Forced March}\index{Forced March}\label{marciaforzata}

On a normal walking day, you can walk for 8 hours. The rest of the day is used to set up and break camp, rest and eat.

If you walk for more than 8 hours you must make a Fortitude save at difficulty 11 +1 for each consecutive day of forced walking or you become fatigued. The saving throw is made every 2 hours beyond 8 hours of walking otherwise the fatigued level increases.

The forced march can be held for a number of days equal to your Constitution value +1 before incurring Fatigue regardless of the outcome of the saving throw.

\textbf{Land}\index{Land}\label{terreno}

The terrain you travel over affects how much distance you travel in an hour or a day. Depending on the environment, climate, quality of the road, the Narrator can evaluate that the movement can be normal, reduced by a third, reduced by half or so impervious and difficult as to reduce it to a quarter of the total possible movement.

\textbf{Movement in the saddle}\index{Movement in the saddle}\label{movimentoacavallo}

A mount carrying a rider can move at a fast pace. However, the damage it takes is normal damage rather than nonlethal. She can also be forced into a forced march, but her Constitution checks automatically fail and the damage she takes is normal damage. Mounts are also considered fatigued when they take damage from fast pacing or forced marching.

\textbf{Mount Bardings}\index{Mount Bardings}\index{Horse Armour}

A mount can be barded with armor. Light armor will grant a bonus to Defense of +2, Medium armor will grant a bonus of +4 to Defense by reducing movement by 25\%, Heavy armor will give a +6 to Defense by lowering movement by 33\%.

\end{multicols}

%\medskip
%\begin{center}
%\includegraphics[height=0.3\linewidth]{immagini/carretto.png}
%\end{center}

\subsection{Table: Mounts and Vehicles}\index{Mounts}\index{Vehicles}\index[Tables]{Mounts and Vehicles Table}\index{Horse movement}\index{Movement per day on horse }

\medskip

\label{tabella-cavalcature-e-veicoli}\index{Dog}\index{Pony}\index{Cart}\index{Raft}\index{Boat}\index{Ship}\hypertarget{table-mounts-and-vehicles }{}

\begin{tabularx}{0.95\textwidth}{llXX}
\multirow{2}*{\textbf{Mount or Vehicle}} & \textbf{Carried space} & \textbf{Movement} & \textbf{Movement}\\
&\textbf{(CdC)}&\textbf{Per hour} & \textbf{Per day}\\
\toprule
Gallop Dog & 30 & 6km & 36km \\
Light Horse & 60 & 8km & 48km \\
Heavy Horse & 80 & 7km & 42km \\
Pony & 30 & 5km & 30km \\
Donkey or Mule & 55 & 6km & 48km \\
Camel & 50 & 8km & 48km \\
Elephant & 160 & 6km & 36km \\
\toprule
\textbf{Boat} & & & \\
\toprule
Raft or Barge (pole or trailer) & 225 & 0.75km & 7.5km \\
Rowing Boat** & 425 & 1.5km & 15km \\
Rowing Boat** & 200 & 2.25km & 22.5km \\
Sailing Ship & 800 & 3km & 72km \\
Warship (sails and oars) & 2200 & 3.5km & 90km \\
Longship (sails and oars) & 600 & 5km & 108km \\
Galley (oars and sails) & 3300 & 6km & 144km \\
\end{tabularx}

\begin{multicols}{2}

\bigskip

A mount can carry a creature on its back only if it is smaller in size than itself. The movement per day is intended for 6 hours of riding, beyond these hours the mount becomes exhausted requiring a whole day of rest.\index{Hours of riding per day}

**Rafts, barges and barges are used on lakes and rivers. If they follow the current, add the speed of the current (usually 4.5 km/h) to the speed of the boat. In addition to being pushed with poles or oars for 10 hours, the boat can also be carried by the current for another 14 hours, if someone is able to steer it, and therefore another 100 km is added in the 24 hours. These boats cannot be rowed against a very strong current, but can be pulled against the current by pack animals on the shore.

The rafts and barges equipped for transport are small inns that allow a frugal meal of the daily catch and some fruit and vegetables brought from shore. There are no rooms to sleep. For those who request, for a small fee, mats are spread out and comfortable mattresses are rolled out and if the climate makes it necessary, blankets are also provided.

The driving of the Raft or Barge takes place in 8-hour shifts per day, to also allow continuous navigation. When it is night navigation stops or continues with the sole force of the current if not impetuous and there are no known dangers. For a surcharge it is also possible to browse 24 hours a day.

If the journey lasts several days it becomes an opportunity for the characters to get to know each other when in the long evenings they gather together with the other guests and sailors to eat a meal and tell stories.

\subsection{Escape and Pursuit}\index{Escape}\index{Pursuit}\label{fugainseguimento}

In round-to-round movement it is impossible for a slow character to escape a fast character without some kind of help. Likewise it is not a problem for a fast character to escape a slower one.

When the chase takes place in the city or in any case in an environment that allows one to hide or lose track, if the speed of the two characters involved is the same, the pursuer and the pursued must make 3 consecutive saving throws on opposing Reflexes. Whoever wins the challenge manages to disappear or catch the fugitive.

If the chase takes place in the open where there is no way to hide or lose track, perform 3 opposing Fortitude saving throws to determine which of the two parties can maintain the pace longer. Whoever wins the challenge manages to lose the pursuer or catch the fugitive.

\subsection{Loading and Transport Capacity: Dimensions}\index{Loading Capacity}\index{Dimensions}

\label{sec:capacita-di-carico-e-trasporto-ingombro}

\subsubsection{Weight and Dimensions}\index{Dimensions}\index{Weight}

Carrying treasure, dragon pieces, full armor not to mention disproportionate weapons or battering rams, pulleys and tackle, make movement difficult.

When evaluating the weight transported, also think about the size!
Carrying a 12 meter x 6 meter roll of silk is not a demanding physical activity, it will be a few kilos, but the size is such that it cannot allow for any further load.

There may be light but extremely bulky objects (hollow trunks, silk carpets...) or small but very heavy objects (mercury spheres, gold-woven clothes), for all these objects the weight value must also be considered based on the 'encumbrance.

Each object has its own Encumbrance value, in general \textbf{every 3 kg there is 1 as an Encumbrance factor}. This value can also become 5Kg if the object is easily transportable. The Encumbrance values ​​of the objects are added together to give the total load carried which compares with the Load Capacity of the creature.\index{Kili and Encumbrance}

Objects with little weight and volume have a footprint of \textbf{Lightweight} (L). These items count as 1 Encumbrance for every 10 items. For every 500 coins you have 1 Encumbrance.\index{Coin Encumbrance}

\subsubsection{Loading Capacity}\label{capacitadicarico}\index{Loading Capacity}

A creature's Carrying Capacity is the sum of its Size, Strength and Constitution.

The Size of a creature grants a bonus to \textbf{CdC} (Carrying Capacity) equal to 9 if Small, 16 if Medium, 25 if Large. The Encumbrance of a creature if dragged\index{Drag a body} of weight is equal to half its Carrying Capacity plus its encumbrance.\index{Encumbrance Creatures transported}

When the total CoC is exceeded then moving and making Dexterity-based proficiency checks becomes problematic. You become weighed down, your movement capacity drops by half, and Dexterity-based Proficiency checks have a -3 penalty.

If the CdC is doubled then it is no longer possible to move due to the encumbrance of the weights carried.

\emph{Remember that the armor and shield when worn have half the size of when carried.}

Ex. Tups is wearing Ringed Armor (encumbrance 2 being worn), a long sword (medium weapon, encumbrance 2), a spiked mace (eng. 2), 18 light objects (eng. 1), a backpack (eng. 1), a tent (ing. 2), a lantern (eng. 1). Total Encumbrance = 11.

Tups is a Medium creature with Strength -1 and Constitution -1 (he is a bit puny and weak..) this gives him a Carrying Capacity of 16-1-1 = 14.

Tups' CdC is greater than her size but he has to be careful, maybe it's better if he leaves the tent on the horse...

If the load is placed on a cart you can push it at full movement if within your CdC, at half the movement if within double the CdC and at a quarter of the movement if within quadruple the CdC.

If multiple creatures push or pull a cart, consider the highest one as the CoC and add half of the other creatures. A chariot can be pushed by 1 creature +1 per chariot size above medium.

\subsubsection{Larger and Smaller Creatures}

The \textbf{Table: CdC transported based on size}\index{Encumbrance transported based on size} shows the Loading Capacity based on size. The Strength and Constitution values ​​must be added to the value given by the size. 

\medskip

\begin{tabularx}{0.45\textwidth}{ll|ll}
\textbf{Size} & \textbf{Eng.}&\textbf{Size} & \textbf{Eng.}\\
\toprule
Very small &1/4& Large & 25\\
Petite & 1 & Huge & 36\\
Tiny & 4& Mammoth&49\\
Small & 9 & Colossal&64\\
Average & 16&&\\
\end{tabularx}

\medskip

Creatures with 4 legs or more can carry larger loads. 


%\begin{center}
%\includegraphics[height=0.5\linewidth]{immagini/cavallo.png}
%\end{center}

\textbf{Table: transportation modifiers for multi-legged creatures}\index[Tables]{Table of transportation modifiers for multi-legged creatures}

\medskip

\begin{tabularx}{0.45\textwidth}{ll}
\textbf{Creature Paws} & \textbf{CoC}\\
\toprule
4 legs & x2\\
6 legs & x2.5\\
8 legs & x3\\
12 legs & x4\\
every other 2 legs & +0.5\\
\end{tabularx}

\medskip

These Tables are to be used for unusual animals not listed or similar to those in Table: Mounts and Vehicles.


\subsection{Other Types of Movement}

\label{altri-tipi-di-movimento}

\subsubsection{Swim}\index{Swim}\label{nuotare}

See Capito Ambiente for \hyperlink{dangers-of-water}{swimming tests} (page \pageref{dangers-of-water}) and \hyperlink{battresottacqua}{fighting under water} (page \pageref{combattoresottacqua} ).

\subsubsection{Scaler}\index{Scaler}\label{scalare}

A creature with a climb speed has a +2d6 bonus on all climb checks, when necessary. If the creature must make a Climb check to climb any wall or slope, it can always choose to take a 10, even if rushed or threatened during the climb.

If a creature with a Climb speed attempts a rapid climb (see above), it is as if it took a Dash action and makes a single Climb check at DC 13. If the creature does not have a Climb score listed, the value is considered equal to its CR + Movement in meters to Climb.

A creature has no penalties to Defense while climbing and has no penalties to attack rolls while attacking.

If you do not have the type of \textbf{Scalar movement} it is considered as \textbf{doubly difficult terrain}, and therefore you move at a quarter of the Movement.

\subsubsection{Dig}\index{Dig}\label{scavare}

A creature with a Burrow speed can tunnel through earth, but not through rock unless the descriptive text says otherwise. Creatures can't charge or run while digging.

Most burrowing creatures leave no tunnels for other creatures to use (either because the material they burrow through fills the tunnel behind them or because they don't actually displace material when they burrow), see the individual creature's description for details.

\subsubsection{Walking - Speed ​​On Ground}

Land speed is the normal speed for characters who do not climb, swim, or fly.

\subsubsection{Flying}\index{Flying}\label{volare}

Flying for a creature with this ability is like walking for a land creature. A creature with flight uses its actions to move but is unlikely to be affected by difficult terrain.

A flying creature that takes damage in a single hit by half its maximum hit points must make a Fortitude save at DC 17 or fall to the ground.

\end{multicols}

\vfill

\begin{center}
\includegraphics[width=0.47\linewidth]{immagini/grifonicastello.png}
\end{center}




\pagebreak

\section{Burn}\index{Burn}\index{Narrator}

\label{masterizzare}

\begin{changemargin}{0.3cm}{0.3cm}\begin{enfasi}{
The one who controls the story is not the voice: it is the ear. (Italo Calvino)

\medskip

To do what you want you have to be born a king or a fool. (\emph{o Narrator}, Editor's note) (Lucio Anneo Seneca)

\medskip

The Dungeons \& Dragons RPG (\emph{and also OBSS}) is about storytelling in worlds of swords and sorcery. It shares elements with childhood pretend play. Like those games, D\&D is driven by imagination. It's about imagining the imposing castle under the stormy night sky and imagining how a fantasy adventurer might react to the challenges the scene presents. (DnD 5e Basic Rules)\\

It's not the job of the DM (\emph{Narrator}) alone to entertain the players and make sure they have fun. Each person playing is responsible for everyone's enjoyment of the game. They all speed up the game, heighten the drama, help establish how comfortable the group feels role-playing, and bring the game world to life with their imagination. Everyone should also treat each other with respect and consideration: personal arguments and arguments between characters get in the way of fun.

Different people have different ideas about what's fun in D\&D. Remember that the \emph{right way} to play D\&D is the way you and your players agree and enjoy it. If everyone comes to the table ready to contribute to the game, everyone will have fun." (Dungeon Master Guide, 4ed)

}\end{enfasi}\end{changemargin}\medskip

\begin{multicols}{2}

\subsection{The Narrator}

\label{il-narratore}

\lettrine[lines=2, lhang=0.33, loversize=0.25, findent=1.5em]{M}{entre} the player plays a character in an adventure, the Narrator is the one who manages it. He certainly has a lot more work to do, but creating an entire world for your friends to explore can be very satisfying.

The role of the Narrator is not easy but grants enormous privileges. Seeing your friends play, have fun, go crazy over doubts, riddles and situations you create gives a lot of fun and moments of true conviviality.

Your role is that of the great orchestrator, planner or even landscaper if you prefer, with a few simple brushstrokes you outline the structure and the players will then add details and situations.

\begin{changemargin}{0.3cm}{0.3cm}\begin{narratore}
OBSS wants to help you and other players have fun. Always use common sense when applying a rule. Your goal is not to kill characters but to create worlds and campaigns that evolve around the characters and the world you create, their actions and decisions. Incorporate things that interest players, keep them involved, make them understand that the world is alive and they are part of it. If you are good at your adventures, the situations will echo in other sessions and off the table.
\end{narratore}\end{changemargin}

Your \emph{work and fun} is fundamental and very important, the quality of the gaming session also depends on you. Your aim is above all to have fun, be creative, improvise, act, create ingenious situations. As long as you're having fun, chances are the players are having fun too!

\textbf{Remember that you are not the protagonist in the adventure, but the characters}, don't steal the scene but like a great dance be the conductor of the orchestra where the instruments are the possibilities offered by OBSS, the music is the adventure and the dancers the characters.


\subsection{Experience Points}\index{Experience Points}\index{PX}

\label{punti-esperienza}

In OBSS the Experience Points that characters gain are used to determine their level and therefore the skills and abilities available to them.

The characters will gain Experience Points based on the monsters defeated but also on other factors such as objectives, ideas, particular actions, difficulties overcome... but also treasures recovered!

The main suggestion is to reward the characters who have worked most hard for the group, those who have contributed most to the success of the adventure and the session. Experience Points measure not only success but also participation in the game.
It is therefore possible to have characters with different Experience Points and potentially even different levels.

The Experience Points awarded by defeating a monster are indicated in the Monsterarium e.g. Challenge 13 (10000 XP). These Experience Points must be divided among all the characters who participated in the battle in any way.

The Experience Points by Level Table indicates the Experience Points needed to move from one level to the next.

Never exaggerate when assigning Experience Points otherwise you risk unbalancing the game and having to significantly modify the adventure. Also be clear with the players at the beginning of the campaign, in Session Zero, how Experience Points will be calculated, distributed and what can be done to get more.

\medskip

\textbf{Table: Experience Points by Level}\index[Tables]{Table of Experience Points by Level}\label{tabellapuntiesperienza}

\begin{tabularx}{0.45\textwidth}{lX|lX}
\textbf{Level} & \textbf{Experience Points}&\textbf{Level} & \textbf{Experience Points}\\
\toprule
1&0 &11&300000\\
2&2000 &12&390000\\
3&8000 &13&490000\\
4&15000 &14&600000\\
5&35000 &15&740000\\
6&60000 &16&890000\\
7&90000 &17&1050000\\
8&120000 &18&1250000\\
9&170000 &19&1470000\\
10&220000 &20&1730000\\
&+prev*0.2&&\\
\end{tabularx}

%\begin{tabularx}{0.45\textwidth}{lX|lX}
% \textbf{Level} & \textbf{Experience Points}&\textbf{Level} & \textbf{Experience Points}\\
% \toprule
% 1&0 &11&122970\\
% 2&1610 &12&198960\\
% 3&2620 &13&321920\\
% 4&4240 &14&600000\\
% 5&6850 &15&520860\\
% 6&11090 &16&842750\\
% 7&17940 &17&1363570\\
% 8&29030 &18&2206260\\
% 9&46970 &19&3569730\\
% 10&76000 &20&5775820\\
% &+prev*1.680&&\\
%\end{tabularx}

\medskip

However, you must not only take into account the Experience Points granted by your opponents but you must evaluate the characters and group during the session.

Whenever the character or group:\index{Bonus Experience Points}\label{puntiesperienzabonus}\index{Bonus Experience Points}
\begin{itemize}
\item \textbf{Reach the set objectives} (prize for the group or character);
\item \textbf{Make full use of and indeed be alternative in the use of your skills and abilities} (character reward);
\item \textbf{Solve problems in a creative, imaginative and functional way} (character award);
\item \textbf{Propose plans and actions that work and are alternatives to what is expected} (prize for the character);
\item \textbf{Discovers or initiates adventure clues and creates new plots} (character reward);
\item \textbf{You use a skill or object in an intelligent and cunning way} (character reward);
\item \textbf{You use a spell in an ingenious (and alternative) way} (character reward);
\item \textbf{Perform an action that puts your life at risk for the group} (character reward);
\item \textbf{Carry out actions following the beliefs of your Patron (for Devotees). These should give you Trait points} (character reward);
\item \textbf{Convert an NPC, of ​​equivalent level, to his Patron (for Devotees only)} (character reward);
\item \textbf{Collect at least 500*Level in gold coins (or equivalent treasure)} (group reward, once per session maximum);
\end{itemize}

I also suggest evaluating these actions to reward the player's effort
\begin{itemize}
\item \textbf{Be collaborative with other players} (reward for the group or character);
\item \textbf{Help a player in difficulty} (reward for the character or group);
\end{itemize}

Grant 200 Experience Points * Character Level.

%Award in Experience Points 1\% of the Experience Points of the next level (write down on a piece of paper the times the reward is earned and then add the Experience Points only once).\index{Experience Points Reward}

A further approach, but to be considered only in the most close-knit and mature groups, is at the end of the session to ask the players to choose who among them played best, in a combination of role, inspiration, incisiveness and collaboration. Reward his character with 200 XP per Character Level.

These Experience Points can be assigned to the group and therefore to all the characters or to the single character.
There is no need to give Experience Points at the end of the game session, keep track of them and inform the players when there is a moment of pause, to reflect on what has happened and done.
In this system, approximately 8/12 sessions are needed to level up, potentially even much less if the players prove to be good and deal with situations brilliantly.

Construct the session so that all the characters can participate and no one feels excluded.

\begin{center}
\includegraphics[width=0.7\linewidth]{immagini/deathbeowulf.png}

\emph{Henry Justice Ford}
\end{center}

When I say \emph{encounter} don't think of the simple clash with monsters, an encounter means any role-playing event that challenges and tests the characters. This challenge can be a witty discussion with the noble who doesn't want to pay them at the end of a mission, to the challenge of a riddle, rebus, or well-placed traps. Experience Points are earned based on the difficulty of the challenges.

A monster does not necessarily have to be killed to gain Experience Points, it is sufficient to defeat it, capture it, win in a different way. In case of retreat by the characters or enemy, grant half of the Experience Points foreseen for the battle if there has at least been an attempted challenge.

As far as possible, each session should include a role part, an exploration part, three combat parts (even many more than three), a rest part.


\begin{changemargin}{0.3cm}{0.3cm}\begin{narratore}
It may seem anachronistic when the sixth edition of the most famous role-playing game is already in development to go back to rewarding characters based on the gold taken from monsters.

However, I can guarantee you that if your group is particularly \emph{poor} in role-playing or simply prefers a more combative style, knowing that the gold collected equates to experience can make going on an adventure much more dynamic and exciting.

OBSS is based on the principles of OSR and as such the exploration and combat phase has its own important and vital weight.
\end{narratore}\end{changemargin}

Let it be clear that nothing prevents you from preparing a level up based on fixed points (milestones) during the adventure. Your table, your rules!

\subsection{Meetings}\index{Meetings}

\label{incontri}

\begin{changemargin}{0.3cm}{0.3cm}\begin{enfasi}{What is life without hope? A throw of the dice in the darkness, in the delirium. (Ambrogio Bazzero)}
\end{enfasi}\end{changemargin}


An encounter is a moment of tension and hope, fear and challenge. It is an opportunity to show and demonstrate one's abilities and to work as a group.

An encounter is not an opportunity to show off your absolute power, either as a Storyteller or as a Player. The Narrator will be able to \st{punish} educate the player who wants to be beyond the group and not part of it.

On the following pages you will find instructions for creating easy, medium, high, extraordinary, deadly and epic challenges.

Through the tools provided by the manual and your experience with the group you will know what level the challenge offers and you will evaluate both its impact in terms of experience points and rewards.

An encounter is an event that confronts the characters with a specific problem that they must solve. Many are fights with monsters or hostile NPCs, but there are other types: a corridor full of traps, a political interaction with a suspicious king, a dangerous passage over a rickety rope bridge, an uncomfortable argument with a friendly NPC who he believes a character has betrayed him, or anything that adds a little drama to the game.

Puzzles, interpretation challenges and tests of skill are the classic methods for solving encounters. The most complex encounters to build and balance will be the combat encounters. Trust your instincts and the suggestions provided in OBSS.

A clash can also start out clearly unbalanced, it will be up to the foresight of the players to understand when to run away!

When planning a combat encounter, first decide what level of challenge you want the PCs to face, then follow the steps outlined below.

\textbf{Determine APL}: \index{APL}Determine the average level of the characters: this is the Average Party Level (APL for short, Average Party Level). You should round this value to the nearest whole number (this is one of the few exceptions to the rounding down rule).

Note that this encounter creation reference guide assumes a party of four or five characters. If your group has six or more players, add one to their average level. If your party contains three or fewer players, subtract one from their average level. For example, if your party consists of six players, two 5th level and four 7th level, the APL is 7th (38 levels total, divided by six players, rounding to the nearest integer, and adding one to the final result).


\textbf{Determining the Challenge Rating}: The Challenge Rating (CR) is a convenience number used to indicate the relative risks presented by a monster, trap, hazard, or other encounter: the higher the Challenge Rating is higher, the more dangerous the encounter. Refer to Table: Determining Encounters to determine the Degree of Challenge your party should face, based on the difficulty of the challenge you want and the APL.

\medskip

\textbf{Table: Determining Encounters}\index[Tables]{Determining Encounters Table}

\medskip

\begin{tabular}{ll}
\textbf{difficulty} & \textbf{Challenge Rating}\\
\toprule
Easy & APL\\
Average & APL +2\\
High & APL +3\\
Extraordinary & APL +4\\
Deadly & APL +6\\
Epic & APL +8\\
\end{tabular}

\subsubsection{How many fights to face}\index{How many fights to face}\label{quantiincontri}

There is no single answer. It's your choice, the system finds its balance between 3 and 5 battles per day. Of course they don't all have to be on High difficulty!. Clashes are ultimately a management of resources to use against an enemy. These resources are hit points, spells, potions, scrolls and consumable items possessed. If you place an Extraordinary challenge as the first encounter it is likely that the players will then decide to rest to recover their energy, otherwise you could opt to tire them out slowly with medium encounters and then try them with a higher difficulty. Finally, remember that a \emph{clash} does not necessarily have to be physical, but also traps, puzzles/riddles, alternative challenges... anything that makes you consume resources and reason.

Always evaluate where they are moving and what is around, it will be natural to find the right number and types of clashes and enemies.

\begin{center}
\includegraphics[width=0.7\linewidth]{immagini/impegnativa.png}

\emph{Henry Justice Ford}
\end{center}

\begin{changemargin}{0.3cm}{0.3cm}\begin{enfasi}{
The essence of the world is play... we play the serious, we play the authentic, we play reality, work and struggle, we play love and death and we even play the game. (Eugen Fink)
}\end{enfasi}\end{changemargin}


\subsubsection{Building the Encounter}\index{Building the Encounter}\label{costruireincontro}

To build an encounter, first calculate the value of the APL (the average level of your group).

To develop your encounter, add creatures, traps, and hazards until you reach your planned APL.

Start by calculating the challenges with the highest Challenge rating of the match, completing the rest with smaller challenges.

For example, you want your party of six 7th-level characters to have a Medium challenge and face some Gargoyles (challenge rating 2 each), some Xorns (challenge rating 5), and their leader, a Stone Giant (challenge rating of Challenge 7). Characters have APL 8 and Table: Determining Encounters states that a Medium challenge for an APL 8 is an encounter rated Challenge 10 (Medium Difficulty = APL+2).

Starting from an established Challenge level (10), follow this table to establish how many monsters to include in the battle.

\medskip

\textbf{Table: Challenge level weight for APL calculation}\index[Tables]{Challenge level weight table for APL calculation}

\medskip

\begin{tabularx}{0.45\textwidth}{XX}
\textbf{GS vs challenge objective} & \textbf{\% per opponent}\\
\toprule
-7/8 & 5\\
-6 & 10\\
-5 & 15\\
-4 & 20\\
-3 & 30\\
-2 & 50\\
-1 & 65\\
0 & 80\\
+1 & 90\\
+2 & 100\\
\end{tabularx}


\textbf{To reach the goal we must add \emph{the percentages} of each individual opponent until we reach 100, or 100\% of the challenge.}

In our example, a Stone Giant has a Challenge rating of 7, i.e. a Challenge rating of -3 compared to our difficulty target, a Challenge rating of 10, the they have a Challenge level of 2 or -8 compared to a Challenge level of 8.

An enemy with a Challenge rating of -3 has a weight of 30, a Challenge rating of -5 has a weight of 15, and a Challenge rating of -8 has a weight of 5.

To reach the goal of a Challenge Rank 10 I will put 1 Challenge Rank -3 (i.e. a stone giant), 3 Challenge Rank -5 (i.e. three Xorns) and 5 Challenge Rank -8 (i.e. five gargoyles). The Total will be 30 (one Stone Giant) + 3*15 (three Xorn) + 5*5 (gargoyle) = 30+45+25 = 100. Goal achieved!

The total Experience Points will be: 2900+3*1800+5*450 = 10550 Experience Points / 6 Characters = 1750 Experience Points per character!

Opponents with a Challenge rating lower than 8 compared to the APL are counted and weighed only if they are higher than 20 as a unit.

\begin{changemargin}{0.3cm}{0.3cm}\begin{narratore}
But how many clashes can you manage per session?

There is no exact answer (2/4 per session/day???), a lot depends on the characters that make up the group and what game they play.
Stay focused on the adventure, don't think that it's better not to tire the characters otherwise they won't be able to handle the next battle. You have to think based on the surrounding environment and who is around, the characters will have to worry about the challenges.

In any case, common sense always helps. If you wear out the characters, make sure it isn't a \emph{Total party kill} every session.
\end{narratore}\end{changemargin}

\subsubsection{Clashes too fast}

One problem you might run into is that the fight resolves too quickly. There can be several reasons and as many solutions.

If players expect few encounters it is likely that they will use their best resources and options immediately at the start of the fight and thus quickly defeat the enemies, in this case take them by surprise with successive waves of enemies.

It is possible that there are too few enemies and therefore even if they are \emph{strong} by channeling all attacks on them they are easy prey for the characters, in this case the gregarious or preventing them from resting and therefore recovering Magic Points and Hit Points will help.

It is obviously possible that the fight is not calibrated well and you have actually balanced the match for it to be too easy, this is the easiest case to resolve, experience will teach you how to better construct matches by adding or replacing opponents.

\subsubsection{The Boss Battle}\index{The Boss Battle}\index{BBEG}

When you prepare a battle with the boss, or with what you can define as a significant enemy who has a certain weight in the development of the campaign, you must worry about making the challenge interesting!

If the battle is to be memorable, it's not enough to place the bad guy, organize everything so that all the events are engaging and exciting.

Organize your enemies so that:

\begin{itemize}

\item arrive in multiple waves so that there is a sense of false success
\item that the enemies arrive from multiple sides so as not to concentrate forces only on one side
\item that more or less difficult enemies are interspersed so that there is a sense of false security
\item that the environment is significant and plays an important role in combat
\item divide the characters on several fronts
\item make the attack not look like an attack
\item play with wit and don't let yourself be demoralized.
\end{itemize}

And in any case always remember: it's not a clash between the Narrator and the Players! The goal is to create memorable moments!!!

\subsubsection{Add NPCs}\index{Add NPCs}

A creature that has levels, skills, competencies, which could be a character but is managed by the Storyteller is considered an NPC. These creatures can play a very important role and should not be used as simple monsters. Give it depth and you will create unforgettable figures.

\subsubsection{Ad Hoc Challenge Rating Changes}\index{Ad Hoc Challenge Rating Changes}

While you can change the specific Challenge rating of the monster by advancing it, applying modifications or levels, you can also adjust the difficulty of the encounter by applying ad hoc modifications to the encounter or the creature itself.

Described here are three additional ways you can alter the difficulty of the encounter.\\

\textbf{Terrain Favorable to PCs}\index{Terrain Favorable to PCs}

An encounter with a monster that is not in its preferred element (such as a Yeti encountered in a lava-filled cave, or a huge Dragon encountered in a very small room) gives the characters an advantage. Treat the encounter as having a lower Challenge rating than its actual Challenge rating.\\

\textbf{Terrain Unfavorable to PCs}\index{Terrain Unfavorable to PCs}

Monsters are designed with the assumption that they are encountered in their preferred terrain: encountering an Aboleth underwater does not increase the Challenge rating of the encounter, even if no character is capable of breathing underwater.

If, on the other hand, the terrain has a more significant impact on the encounter (such as an encounter against a creature with blindsight in an area that suppresses all sources of light), you can increase the challenge rating of the encounter. of a higher degree.\\

\textbf{NPC Equipment Changes}\index{NPC Equipment Changes}

You can increase or decrease the difficulty given by NPCs by modifying their Equipment. An NPC encountered without equipment should have a Challenge rating reduced by 1 (provided that the loss of equipment is truly counterproductive for the NPC), while an NPC who has equipment equivalent to that of a character (as indicated on Table: Wealth of Characters per Level) has a Challenge rating 1 higher than its actual Challenge rating.

Care must be taken when assigning NPCs this additional equipment, especially at higher levels, where you can consume the entire treasure of your adventure in one fell swoop!

\subsubsection{Assign PXs}\index{Assign PXs}\label{assegnarepuntiesperienza}

The characters advance in level by defeating monsters, overcoming challenges, having fun, completing the adventure and grabbing treasures: in doing so they earn Experience Points (XP for short). You can award Experience Points as soon as a challenge is overcome, but this may interrupt the flow of the game. It is easier to award experience points at the end of a game session (or multiple sessions) that allows the characters to reflect on what happened. The player can use the time available between game sessions to update the card.

\subsubsection{Arranging Treasures}\index{Arranging Treasures}\label{disporretesori}

As characters level up, the amount of treasure they carry and use also increases. In OBSS, all characters of the same level are assumed to have more or less the same amount of treasure and magic items. Since a character's primary income comes from treasure and loot from adventures, it's important to moderate the wealth and treasure in your adventures.

To help you arrange treasures, the amount of magical items and loot characters receive for their adventures is linked to the Challenge rating of the encounters they face: the higher the Challenge rating of the encounter, the greater the treasure awarded .

\begin{center}
\includegraphics[width=0.7\linewidth]{immagini/tesoro2.png}
\end{center}

\textbf{Table: Treasure Values ​​per Encounter} lists the amount of treasure each encounter should award based on the average character level and the campaign's XP progression rate. If magic is rare in the game, halve these values. If the game is more epic, double these values.

\begin{changemargin}{0.3cm}{0.3cm}\begin{narratore}
How to distribute treasure is an important matter. Treasures should not be thrown in your face, much less hidden so that it is impossible to find them.

A tip is to make sure that the treasures (and coins) found in the dungeons are distributed according to this criterion:

\smallskip

- monsters will have a third on them

- a third will be hidden behind secret passages or traps

- a third will be scattered around

\smallskip

This will encourage players to continue exploration, face monsters, and actively search the dungeon.
\end{narratore}\end{changemargin}

\medskip

\textbf{Table: Treasure Values ​​per Encounter}\index{Treasure}\index[Tables]{Table Treasure Values ​​per Encounter}\label{valoretesoroincontro}\hypertarget{treasure valueencounter}{}

\begin{tabularx}{0.45\textwidth}{XX}

\textbf{Challenge Rating} & \textbf{per Encounter (mo)} \\
\toprule
\textbf{0 (Easy) } & LV*LV*3\\
\textbf{+2 (Average)} & LV*LV*5\\
\textbf{+3 (High)} & LV*LV*7\\
\textbf{+4 (Extraordinary)} & LV*LV*10\\
\textbf{+6 (Deadly)} & LV*LV*15\\
\textbf{+8 (Epic)} & LV*LV*20\\
\end{tabularx}
\bigskip{}

Use this table to get an indication of the value of the match's Treasure.\index{Challenge Treasure}

A \textbf{Double treasure}\index{Double Treasure} means that a party facing an encounter with a Challenge Rating relative to APL of +2 will find the equivalent of Level * Level * 5 * 2 in gold coins equivalent.

A normal Treasury, e.g. APL 6lv faces an encounter at High Difficulty (APL+3), it will be equivalent to Level*Level*7=6*6*7 gold coins from the encounter.

A treasure \textbf{accidental} indicates that there may be treasures, objects, coins only \emph{accidentally} obtained, perhaps by killing other creatures. Said differently, the creature does not collect treasures, but could still have accumulated something.

I suggest valuing an accidental treasure a quarter of a normal treasure. An accidental treasure lends itself very well to providing particular objects that can give directions or provide new adventure plots (what was this adamantium flask for?)...

Encounters against NPCs usually reward three times as much treasure as an encounter with a monster, thanks to the NPC's Equipment. To compensate, make sure your characters go through a couple of additional encounters that award little in the way of treasure.

Animals, Plants, Constructs, Unintelligent Undead, Oozes and Traps are excellent \emph{encounters with little treasure}. Alternatively, if the characters face a number of creatures with little or no treasure, they should have the opportunity to obtain a number of items of more significant value in the immediate future to offset the imbalance. As a general rule, characters should not possess any magical items worth more than half the character's total wealth, so check carefully before rewarding characters with very expensive items.

\subsection{Character Wealth per Level}\index{Wealth per Level}

The \textbf{Table: Character Wealth by Level} by Level indicates the amount of gold equivalent in treasures and items that each character should have at a specific level. Please note that this table is based on a standard game model.

Adventures with rare magic might award only half this value, while more epic adventures might award double it. It is assumed that some treasure is consumed during an adventure (such as potions and scrolls), and that some of the lesser-used items are sold for half their value to purchase more useful equipment.

Table: Character Wealth by Level can also be used to determine equipment for characters starting after 1st level, such as a new character created to replace a dead one. Characters should spend no more than a third of their total wealth on a single item.

For a balanced method, characters who are created after 1st level should spend 25\% of their wealth on weapons, 25\% on armor and protective items, 25\% on other magic items, 15\% on \% for expendable items such as wands, scrolls and potions and 10\% for normal equipment and coins. Different character types may spend their wealth differently than suggested; for example, arcane spellcasters may spend more on magical and expendable items than on weapons.\\

\textbf{Table: Character Wealth by Level}\index[Tables]{Character Wealth Table by Level}


\begin{tabularx}{0.45\textwidth}{XX|XX}
\textbf{Level} & \textbf{Wealth (gp)} & \textbf{Level} & \textbf{Wealth (gp)}\\
\toprule
1 & 100 & 11 & 13900\\
2 & 160 & 12 & 19900\\
3 & 220 & 13 & 25900\\
4 & 340 & 14 & 37900\\
5 & ​​530 & 15 & 49800\\
6 & 2030 & 16 & 67700\\
7 & 3660 & 17 & 85700\\
8 & 5780 & 18 & 142000\\
9 & 8100 & 19 & 253000\\
10 & 11000 & 20 & 365000\\
\end{tabularx}

\subsection{Magic Treasures}\index{Magic Treasures}\index{Finding Magical Treasures}\label{tesorimagici}

Magical Treasures must be inserted sparingly and thoughtfully, try to resist the temptation to be generous with the characters because they will easily get used to it and you will hardly be able to recover the situation.

Generally speaking, every magical item found by the characters is placed by the Storyteller with a reason, purpose, and reasoning.

Nothing stops you from playing on the \hyperlink{generazionetesorimagici}{Random Generation Tables for Magical Items} (page \pageref{generazionetesorimagici}) I just invite you to be careful.

Set a Standard Treasure monster to have 1\% of having a magic item, a Double Treasure to have 3\%, and a Triple Treasure to have 5\%. Obviously evaluate case by case depending on the type of monster and if for example it has a lair, goes hunting for adventurers...\index{Probability of magic objects in treasure}

If you prefer an established and balanced distribution, follow the instructions below.\index{Magic items per level}

As regards \textbf{permanent} magic objects such as weapons, armor, other objects without charges or with daily charges, you can distribute the objects according to this cumulative scheme:

\begin{itemize}
\item levels 1-4: an uncommon item
\item levels 5-7: a second uncommon item
\item levels 8-10: a rare item
\item levels 11-13: a second rare item
\item levels 14-16: a very rare item
\item after the 17th: very rare or legendary item.
\end{itemize}

As for the \textbf{consumables} such as potions, scrolls or objects with a scalar use of charges, you can distribute the objects according to this cumulative scheme:

\begin{itemize}
\item levels 1-5: a common consumable
\item levels 6-10: an uncommon consumable
\item levels 11-15: a rare consumable
\item levels 16-19: a very rare consumable
\item beyond the 20th: a legendary consumable
\end{itemize}


All of this clearly depends on the level of magic you want to give to the adventure.


\subsubsection{Building a Loot}\index{Building a Loot}\label{costruireunbottino}

It's often enough to tell your players that they found 5,000 gp in gems and 10,000 gp in jewels, but it's more interesting to provide details. Giving a treasure a personality can not only help the game's verisimilitude, but can sometimes spark new adventures.

The information on the following pages can help you randomly determine treasure types: many of the items have values ​​given, but you can assign them as you see fit. It is easier to place the most expensive items first: if you want, you can also determine the magic items randomly using the tables in Magic Items to determine which items are present in the treasure.

Once you have consumed a considerable portion of the value of the treasure, the rest can simply be made up of scattered coins and non-magical items with values ​​defined according to your needs.

\textbf{Coins}: Coins in a treasure can be made of copper, silver, gold and platinum: silver and gold are the most common, but you can decide differently. For coins and their exchange value go to Equipment.

\textbf{Gems}: Although you can assign any value to a gem, some may be worth more than others. Use the value categories below (and associated gemstones) as a reference guide when assigning values ​​to gemstones.

\textbf{Low Quality Gems} (10 gp): agate; azurite; blue quartz; hematite; lapis lazuli; malachite; obsidian; rhodochrosite; eye of the Tiger; turquoise; river pearl (irregular).

\textbf{Semi Precious Gems} (50 gp): heliotrope, carnelian; chalcedony; chrysoprase; citrine; jasper; lunaria; onyx; chrysolite; rock crystal (clear quartz); sardonic; sardonyx; rose, smoked or star rose quartz; zircon.

\textbf{Medium Quality Gemstones} (100 gp): amber; amethyst; chrysoberyl; coral; red or green-brown garnet; jade; jet; white, golden, pink or silver pearl; red, red-brown or dark green spinel; tourmaline.

\textbf{High Quality Gemstones} (500 gp): alexandrite; aquamarine; purple garnet; black Pearl; dark blue spinel; golden yellow topaz.

\textbf{Jewels} (1000 gp): emerald; white, black, or fire opal; blue sapphire; fiery yellow or vermilion corundum; blue or black star sapphire.

\textbf{Exceptional Jewels} (5000 gp or more): Crystalline brilliant green emerald, diamond, hyacinth, ruby, crystalline honey mouse.

\textbf{Non-Magical Treasures} This category includes jewelry, fine clothing, goods, alchemical items, masterwork items, and others.

Unlike gems, many of these items have set values, but you can always increase the value of the item by decorating it with precious stones or particularly artistic workmanship.

This cost increase grants no additional abilities: a gem-embellished cold iron scimitar worth 40,000 gp functions as a normal 330 gp cold iron scimitar. Below you will find numerous examples of non-magical treasures, with typical values.

\textbf{Fine Works of Art} (100 gp or more): While some works of art are composed of precious materials, the value of most paintings, sculptures, literary works, fine clothing, and the like lies in workmanship with which they are made and in the skill of those who made them. Art objects are often bulky or difficult to move, and fragile, making retrieval and transportation an adventure in itself.

\textbf{Minor Jewelry} (50 gp): This category includes jewelry made with materials such as brass, bronze, copper, ivory, or exotic woods, sometimes embellished with very small or defective low-quality gems. Minor jewelry includes rings, bracelets and earrings.

\textbf{Normal Jewelry} (100-500 gp): Most jewelry is made of silver, gold, jade, or coral, and often decorated with semi-precious gems or medium-quality precious stones. Normal jewelry includes all types of minor jewelry plus bracelets, necklaces and brooches.

\textbf{Precious Jewelry} (500 gp or more): Precious jewelry is made of gold, mithral, ​​platinum, or similar rare metals. Such items include the normal types of jewelry plus scepters, pendants, and other large items.

\textbf{Excellently made tools} (100-300 gp): This category includes tools for Professions or Skills: see Equipment for details and costs of these items.

\textbf{Common Items} (up to 1000 gp): There are many valuable items of an alchemical or common nature that can be used as treasure. Most alchemical items are portable and estimable items, but others such as locks, sacred symbols, telescopes, fine wines, or fine clothing can also make interesting parts of a treasure. Trade goods can also serve as treasure: 5 kg of saffron, for example, is worth 150 gp.

\begin{changemargin}{0.3cm}{0.3cm}\begin{narratore}
Never overdo it with treasures, especially magical ones. A treasure must not become a habit, even more so if it is something special and particular. One thing can be coins, gems and consumables, another thing is the real treasures, the magical, particular, unique ones.

Respecting the Law of the Prize does not mean lining the pockets of the characters, otherwise they will become bored with risking their lives for new treasures and objects. When you find a magical object, always think in perspective. It's true that it can be nice to see players happy with what they've found but then you'll be forced to give something even more powerful in the next adventure.

\end{narratore}\end{changemargin}

\medskip


\textbf{Treasure Maps and Information Items} (variables): Items such as treasure maps, legal documents for ships and houses, lists of informants or watch lists, passwords, and the like can be fun objects to find in a treasure: you can establish the value of these objects as you wish and they can be of double use as they can generate ideas for new adventures.

\textbf{Magic Items}

Of course, the discovery of a Magical Item is the real prize for any adventurer. Be careful when placing Magic Items in a treasure: it is much more satisfying for many players to find a magic item than to buy it.

While you should generally place items with careful thought about their likely effects on your campaign, it can be fun to spawn magic items in a random treasure. Be careful, however! It's easy, with a little luck (or bad luck) on the dice to inflate your game with too much treasure or deprive it of it. The placement of random magic items should always be tempered by the Storyteller's common sense.

Even spells are real treasures and rewards like magical objects. Consider carefully which ones can be found. Remember that a magical ability is not a copyable spell, only those present in tomes, scrolls and anything else specifically created to be a repository of spells are suitable for copying.

\subsection{Acting}\index{Acting}\label{ruolare}

\label{recitare}

A role-playing game is not a simple roll of dice, it is a meeting of thoughts, opinions, challenges, struggles. It is a cathartic, liberating, evolutionary and instructive game.

It is right that there is combat, struggle, blood, fear and action, in the same way there must be the possibility of playing your characters with their disadvantages, advantages, powers and stories and also personal dramas.

The player must always impersonate the character, empathize and actively participate.

There may be side situations, managed quickly, which are done in the third person, yet every time it is necessary to play then it must be real, done by the player fully immersing himself in the character.

\medskip

\textbf{When a player interprets well and describes the action he is going to carry out in a \textbf{participatory}, \textbf{engaging}, \textbf{inspired} way, give him a reward, grant a bonus of + 1 to the action he is performing}

\medskip

Point this out to the player who, thanks to his interpretation, has that bonus.

At the same time there may be situations that prove unpleasant for some players to manage and play. Be very careful in this case, going against the sensibilities of a player, of a friend, is not the same as going against the ethics or morals of a character. If you feel a sense of discomfort or embarrassment, stop the game immediately and clarify the situation with the players and resume only when you have agreed on how to change the situation to prevent it from happening again.\index{Veils and Prohibitions}\index{ Play and don't scare}


\begin{changemargin}{0.3cm}{0.3cm}\begin{enfasi}
{Focus on the people, not the rules. Push for a group style of play; the interpretation is fun but should not hinder the plan; support your teammates. (Frank Mentzer)}\end{enfasi}\end{changemargin}


\begin{center}
\includegraphics[width=0.9\linewidth]{immagini/Hoxne_Hoard_1.png}

\emph{Reproduction of the Hoxne Hoard}
\end{center}


\subsection{About OBSS and the dice rolls}\index{About OBSS and the dice rolls}\label{obssedadi}

OBSS uses a unique dice roll system by mixing a 3d6 distribution with the potential of exploding 6s. This system manages to guarantee a good variance and, although it concentrates the results around the central values ​​of the distribution, it leaves the upper limit open to particularly lucky shots.

If you want to have fun studying the corresponding curve I recommend the site \href{https://anydice.com/}{Anydice}. This is the pseudo code to insert:

{\small function: explode ROLLEDVALUE:n \{

if ROLLEDVALUE = 6 \{ result: 6 + [explode d6] \}

if ROLLEDVALUE = 1 \{ result: 0 \}

result: ROLLEDVALUE\}

output 3d[explode d6]}

or click \href{https://anydice.com/program/2610e}{here} for the code already entered.

\subsubsection{Optional - Critical Rolls Variant}\index{Optional - Critical Rolls Variant}\label{variantetiricritici}\hypertarget{critical rolls variant}{}

In OBSS the rule applies that if a success is achieved by obtaining at least two 6s on the dice it is defined as a critical success, in the same way if I fail the test by obtaining two 1s in the dice roll then it is defined as a critical failure.

These critical rolls, which are very dependent on the randomness of the dice, may not appeal to all players and as per the \hyperlink{variant critical roll}{Optional Variant Critical Roll} rule (page \pageref{variant critical roll}) this optional rule establishes that in case of Competence Check, Attack Roll and Saving Throw:

- treat a failed check by 6 or more as a critical failure.

- treat a successful check of 6 or more as a critical success.

- for each multiple of 6 for which the test is successful or failed, an additional critical success or failure is accumulated respectively. 

The same principle applies to the Magic Test which is performed by rolling 3d6 + Magical Expertise + 2*Adept of Magic in that magic list and the result must exceed a difficulty equal to 10 + Level of Manifested Magic*3. \subsection{Adventures in OBSS} \hypertarget{OSR}{} \index{OSR}\index{Adventure in OBSS}\label{avventureinobss}

I suggest reading the article in full: \href{https://lithyscaphe.blogspot.com/p/principia-apocrypha.html} {Principia Apocrypha}

https://lithyscaphe.blogspot.com/p/principia-apocrypha.html the following is an adapted and modified summary of the guidelines I follow when burning OBSS.

OBSS follows the principles of \href{https://it.wikipedia.org/wiki/Old_School_Renaissance}{OSR} (wikipedia). Adventures in OBSS aim to be lethal, have a freely explorable world, a sketchy plot, focus on problem-solving and have a reward system focused on exploration, treasures and group participation. OBSS doesn't care too much about encounter balance and appreciates the resourcefulness of the players and captures their ideas and puts them into the adventure.

For me, OSR is not tables of random encounters and chaotic randomization or a specific rulebook, it is rather the spirit of adventure, wonder, fear, glory, amazement and challenge that develops in adventures. Don't be too linear, too predictable, add the right mix to your adventures that always makes them unique.

If you don't like the method, use the one you like best, personally over the decades I have learned to appreciate and appreciate the spontaneity and naturalness that the cornerstones of OSR bring to the game.

\bigskip

\textbf{These are basic rules for the Storyteller that I suggest for running adventures.}\index{Guidelines for Storytellers}\index{OSR Principles}

\medskip

\begin{itemize}

\item
You are the Narrator, your Rules, your World.

Don't be limited by the adventure, the system, the monster list, always feel free to modify and adapt according to the needs of the adventure and the group

\item
Remember to be fair and correct. Improvise, adapt as much as you want but be consistent. If you establish a rule (or a change to a rule) follow it to the end.

At the same time, if you need a rule and can't find it, use common sense, it is definitely the right choice at that moment.

Respect the dice and the results obtained, as will happen to the players, special results will happen to you too. Rightly so.

\item
You don't have to save the characters' asses. You are neither their friend nor their enemy. Your role is to tell stories that arise from the characters' stories, their actions and inactions.

\item
Outline the story, write the central parts or to read to the players but don't be dominated or bound by what you expect. More often than not the players will surprise you, it is better to know where they are moving and what is around them in order to always react on time.

The players give direction to the adventure and you unravel it.

\item
Appreciate chance and create different situations where players can choose different paths or weave new ones. It's your luck to have creative players who know how to surprise you.

\item
Don't force anyone to do something, let the players make mistakes, let them pay for their choices. You don't have to hinder them or give them direction. It requires considerable imagination and ability to adapt on your part, but the adventure will certainly benefit from it.

\item
The characters are explorers, by definition. Focus on exploration, the more you explore the more situations you create, the more connections you create in the adventure, the more you get to know other NPCs the more areas there are to explore.

It makes you understand that treasures are experience, in a literal and practical sense. You will never have to push them into a dungeon but their lust for experience and treasure.

\item
Have the players solve problems, not the characters. Let the scenes play out, they're always better than a roll of the dice. Encourage players to interact and ask for a trial only as a last chance. Propose problems that do not necessarily have to be solved with a roll of the dice but rather through multiple actions, even complex ones.

It rewards creative actions and courageous choices, above all intelligence and the desire to find alternative and creative situations.

\item
Make players ask you for information, engage with the environment and with each other. Encourage interaction with the outside world and only allow a die roll as a last resort.

\item
Great challenges and risks always yield great rewards. Don't disappoint the players (except for adventure purposes) by denying them the right treasure or experience, the deeper they delve the more lethal the dangers the greater the reward (Law of Reward).

\item
There must be no habits or customs. Don't create a standard.
Always try to surprise players with out-of-place (but meaningful) monsters, anomalous traps, alternative environments. Different situations will stimulate players to solve each problem differently.

Prepare different solutions and accept different solutions. Bring into the adventure problems and situations that together allow for a solution, each room must not be an aseptic environment but contain clues and solutions for other problems even without a direct solution.

\item
Accept death. A fight if such is always lethal, don't be afraid to hurt or kill the characters. Make them think, study the enemy, understand what the best approach is; and finally amaze them. The characters must first outwit their enemies and plan if they want to survive.
If you protect the characters the game will lack tension and the players will solve all problems with brute force.
Dungeons don't have to be environments that need to be emptied of monsters. The purpose of monsters is to limit and direct actions, consume options.

If the players are always looking for a head-on collision then give it to them, as they demand.

\item
Keep your attention high. It makes sure that the passage of time has consequences, if players fear the passage of time they will make bolder or perhaps wrong choices. Maintain tension between the desire to explore and loot and the fear of staying still for too long.

\item
You are the source of the information, the players process it, the characters use it.

Don't hide information that the characters need to know or already know, you won't have to be a professor but at the same time make sure they are aware of what's around them.
At the same time you don't have to reveal everything right away, let them investigate, snoop around. Like an onion, the information they obtain will be hidden under layers of other information, perhaps of less importance.

\item
Clues create situations. Let your specific and curious clues capture the players' attention. Like bait on a hook it lures players into situations of doubt, where they can investigate and understand what's happening.

Don't fill the adventure with useless details, leave room for the players' creativity and imagination, but the details you provide must not only make sense but be necessary for the adventure.

\item
If players tend to forget useful information given, try using an NPC who has memory or have them take notes, there's no harm in being prepared.

\item
The adventure is never static, much less the world in which the characters move.
The world is as important if not more than the adventure itself. Player actions can trigger global events. Always think about the consequences of gestures.

\item
If you use NPCs (non-player characters), don't make them mere characters, make sure that the characters can become attached to you and consider the NPC one of the group on a par with all the others.

\item
Monsters don't have to be stupid. Make them talk, think, escape... they want to live too!

\item
Remember the Law of Reward. Reward the bold, reward those who go deeper into the caves. Reward those who survive.

\end{itemize}

\subsection{Session Zero}\index{Session Zero}\label{sessionezero}\hypertarget{sessionzero}{}

Session Zero, the first game session, has a particular value and importance. It can be the session in which you get to know each other for the first time, often it is the session in which the characters you will play are created, it is always the session in which common rules and expectations are established.

Session Zero serves to establish what and how you will play, what will be the main characteristics of the campaign and the group that will be created.

To get off to a good start as a group of players it is important to know each other personally and have trust and respect in others. You don't have to say everything about yourself but at least your passions, interests, curiosities, what at least at the beginning serves to create a friendship.

I suggest Storytellers establish clear rules for good play. Unfortunately, experience teaches us that we are all different people with different styles, perspectives and expectations and often, rather than coinciding, they are actually opposites in our way of playing and being.
Knowing yourself also serves this purpose, to understand if your character can fit in with the group of characters and to understand if your person and personality is in some way similar to other people or not.

€13088 € {Before starting, the Narrator should clarify what the essential rules are at his table}. An example of rules could be:

- Know the limits of others. Each person has a different sensitivity to certain topics (rape, slavery, racism, violence...) it is essential that we clarify together what the limits should never be exceeded.

- Respect every person you play with. This includes being on time and not canceling without an important reason. Also respect the boundaries that others have communicated.

- Players must create a cohesive group made up of collaborating individuals.

- Players must know or at least try to learn the rules of the game.

- As a Narrator you must understand the players' level of knowledge of the system and, if necessary, when it is necessary to stop and explain the rules.\\

\textbf{The basic information of the adventure must be shared and established.}

- Introduce in general the campaign or adventures that will take place. Indicates the type (heroic, dark, gothic, horror, politics, infinite caves, exploration, survival..) and degree of difficulty.

- Introduce necessary information related to the setting or provide handouts and manuals on the topic. Indicate if there are any Skills that may be less useful than others.

- If you use optional rules, explain them well and make sure they are valid for both players and opponents.

- Indicates the list or type of Traits accepted and if there are limits in the choice of Patrons.\\

\textbf{Other useful information concerns}:

- What is allowed to bring and use at the table and what is not (drinks, eating, cell phones, alcohol, smoking..). Know if there are animals in the house.

- Minimum number of players to play the session. Establish the game day and times.\\


Ultimately, Session Zero is critical to establishing a solid foundation for good roleplaying. It helps create a collaborative environment where everyone feels included, and helps avoid potential problems and disagreements during the course of the campaign.

Even in the best already close-knit group, it is always good to remember and share these suggestions at the start of a campaign.


\end{multicols}

%\vfill

%\begin{center}
%\includegraphics[keepaspectratio,width=0.4\textwidth]{immagini/dungeonsample.png}

%\emph{Detail of a dungeon}
%\end{center}

\pagebreak

\section{Create Magic Items}\index{Create Magic Items}

\begin{changemargin}{0.3cm}{0.3cm}\begin{enfasi}{
To create is to live twice. (Albert Camus)}\end{enfasi}\end{changemargin}\medskip

\begin{multicols}{2}

\label{creare-oggetti-magici}

\lettrine[lines=2, lhang=0.33, loversize=0.25, findent=1.5em]{P}{er} Creating Magic Items requires the Creation of Magic Item Feats.

The costs listed here are those of production, the revenue can be around 20% of the production price. The DC to create an item is 15+2*Level of the spell contained, or +4 for each +1 of the item.\index{DC to create magic item}

Knowing the enchantment (or having it available via Scroll) that applies to the item is a requirement of every magical item you create.

\begin{changemargin}{0.3cm}{0.3cm}\begin{narratore}
Creating magic items can break the balance of the game. A character with abundant resources and time can create objects that break the balance of the adventure. I suggest that the NPCs, the non-player characters managed by the Narrator, create the most wonderful objects. At the same time the sale of valuables above 2000mo should be as limited as possible.
\end{narratore}\end{changemargin}

\subsubsection{Create Magic Rings}\index{Magic Rings}\label{creareanellimagici}

To create a magic ring, a character needs a heat source. You also need a supply of materials, of which the most obvious is a ring or ring pieces to assemble. 

The production cost of the ring is equal to level*level*4000, a Ring with Invisibility costs 2*2*4000=16000 gp

\smallskip

\begin{center}
\includegraphics[width=0.5\linewidth]{immagini/onering2.png}

\emph{It goes without saying that the One Ring is...}
\end{center}

\smallskip

A ring allows you to fix a spell to make the effect always active.
The ring must have an intrinsic value equal to at least 500mo*sum of the levels of the spell it must host.

A ring can accommodate a level 9 spell or if multiple spells the maximum level is 7.

It is also possible to insert an activation spell, in this case consult the costs of the Rods.

Forging a ring takes 1 day for every 1000 gp of the base price. In the case of multiple spells, the costs and times add up.

\medskip

\textbf{Item Creation Feat required}: Craft Greater Magic Items.

\subsection{Create Magical Armor and Shields}\index{Create Magical Armor}\label{crearearmaturemagiche}

To create magical armor or shield, a character needs a heat source and some tools for working with iron, wood, or leather. You also need a supply of materials, of which the most obvious is the armor/shield itself or pieces of armor to assemble. An armor/shield that is to be enchanted must be of quality.

\smallskip

\begin{center}
\includegraphics[width=0.6\linewidth]{immagini/Rustning_Gustav_Vasa.png}

\emph{Armour for Gustav I of Sweden by Kunz Lochner, c. 1540 (Livrustkammaren)}
\end{center}

\smallskip

If the prerequisites for creating armor include spells, the caster must know those spells.

The production cost of +1 magical armor costs 2050 gp, +2 7500 gp, +3 12000 gp, +4 25000 gp, +5 45000 gp plus the price of the armor itself.

Infusing an enchantment into armor costs the same as creating a ring with that enchantment.

Crafting magical armor/shields takes one day for every 1000 gp of the base price value.

\medskip

\textbf{Item Creation Feat required}: Craft Magic Items.

\subsection{Create Magical Weapons}\index{Create Magical Weapons}\label{crearearmimagiche}

To create a magical weapon, a character needs a heat source and some tools to work the iron or material from which the weapon is made. He also needs a supply of materials, of which the most obvious is the weapon itself or the weapon parts to be assembled. Only a quality weapon can be enchanted to become a magical weapon, and its cost must be added to the total enchantment cost to determine the final market value.

A magical weapon must have at least a +1 bonus to have any special ability or spell.

\medskip

\begin{center}
\includegraphics[width=0.6\linewidth]{immagini/exacaliburfuori.png}

\emph{The drawing of the sword from the stone, Henrietta Elizabeth Marshall's Our Island Story (1906)}
\end{center}

\medskip

If the prerequisites for crafting the weapon include spells, the caster must know those spells.

At the moment of creation, the caster must decide whether the weapon emanates light or not, as a secondary effect of the magic infused into the weapon. This decision does not affect the price or creation time, but once the item is completed, the decision is final.

Crafting dual weapons is treated as equivalent to crafting dual weapons with regards to cost, time, and Special Abilities.

The production cost of a +1 Weapon is 1200 gp, +2 4000 gp, +3 11000 gp, +4 25000 gp, +5 45000 gp plus the price of the weapon (important only if it is made of some rare or precious material ).

The production cost of a +1 Arrow is 20 gp, +2 75 gp, +3 325 gp. More powerful enchantments are extremely rare.

Infusing an enchantment into a weapon has a cost as if you created a ring with that enchantment, if continuous, otherwise if single use like a potion.

Crafting a magical weapon takes one day for every 1,000 gp of the base price value.

\medskip

\textbf{Item Creation Feat required}: Craft Magic Items.

\subsection{Create Wands}\index{Create Wands}\label{crearebacchette}

The production cost of the Wand is equal to level*level*800, a Wand with Invisibility costs 2*2*800=4200 gp

A wand is a magical object that holds a previously charged spell.

To recharge a wand a spellcaster must imbue the same spell and have the Create Magic Item Feat. The wand recovers a charge but the caster, in addition to having used Magic Points, spends the equivalent of 100*level gold coins on components.

A wand can contain a maximum spell level of 5.

To create a wand, a character needs a supply of materials, the most obvious of which is a wand or wand pieces to assemble. Wands are always fully charged (20 charges) upon creation.

The caster must know the spell he places in the wand.

\begin{center}
\includegraphics[width=0.6\linewidth]{immagini/wand.png}
\end{center}

Crafting a wand takes 1 day for every 1000 gp of the base price value.

\medskip

\textbf{Item Creation Feat required}: Craft Magic Items.

\subsection{Create Sticks}\index{Create Sticks}\label{crearebastoni}

\textbf{Basic Club Costs}

\bigskip

The production cost of the Staff is equal to level*level*1200, a Staff with Invisibility costs 2*2*1200=4800 gp

\bigskip

A Staff is a magical object into which one or more spells are charged.

When a staff is activated you can use one spell at a time.

To craft a staff, a character needs a supply of materials, the most obvious of which is a staff or pieces of a staff to assemble.

The staves are always fully charged, 10 charges, at creation.

A Staff can contain a maximum spell level of 8, or in the case of several spells the maximum level is 6.

\begin{center}
\includegraphics[width=0.3\linewidth]{immagini/staff2.png}
\end{center}

Crafting a staff takes 1 day for every 1000 gp of the base price.

\medskip

\textbf{Item Creation Feat required}: Craft Greater Magic Items.

\subsection{Create Scrolls}\index{Create Scrolls}\index{Scrolls}\index{Isy Scroll}
\index{Easy scrolls}\index{Buy spells}\label{crearepergamene}
\medskip

There are two types of magical scrolls, those that can be performed by everyone (called ISY SCROLL, or Easy) and those that require the magical ability to cast spells, i.e. Magical Expertise greater than or equal to 1.

Easy scrolls cost to craft equal to spell level*level*rarity*160 gp.

Normal, non-easy scrolls cost to produce equal to spell level*level*rarity*80 gp

An enchantment of Common rarity multiplies the cost *1, Uncommon *2, Rare *4, Very Rare*8, Legendary*16.

\begin{center}
\includegraphics[width=0.45\linewidth]{immagini/scroll3.png}
\end{center}

If a scroll includes multiple spells, the cost is equal to the sum of the various spells. There can be no normal scroll spells on an ISY SCROLL scroll, and vice versa.

The spellcaster must know the spells he places on the scroll. Preparing a scroll requires 30 minutes of work per spell level present.

An ISY scroll can contain a maximum spell level of 3, while a normal scroll can contain a maximum spell level of 9, in the case of multiple spells the maximum level is 8.

\medskip

To read a scroll you need:\\

\textbf{in case of ISY SCROLL scrolls}:

- to understand the content, an Intelligence check (or Arcana if known) at difficulty DC 10 is sufficient

- to be able to read and cast the scroll's spell, an Intelligence check (or Arcana if known) is required at difficulty 12.

\medskip

\textbf{in case of normal scrolls}:

- to understand its content, an Arcana test at difficulty 15 is required

- to be able to read and cast the scroll's spell you need an Arcana check at difficulty 11 + Spell Level and have access to the Magic List of the spell contained and that this has a level equal to the maximum castable +1.

\medskip

The \textbf{casting time} of a spell from a scroll is equal to the casting time of the current spell.

\textbf{Item Creation Feat required}: Craft Magic Items

Proficiency used in creation: Arcana or Profession (scribe).

When a scroll is used or copied, it is destroyed.

\textbf{Note}: A Tome of Magic is equivalent to a set of normal scrolls. A character in desperate straits can read the spell page from the Tome of Magic and manifest the magic as if from a scroll. The pages containing the spell will become pulverized and the caster will have to find a source from which to copy the spell back to the Tome. He cannot copy the spell onto the Tome anyway because he has learned it. \index{Tome of Magic as parchment}

\subsection{Create Potions}\index{Create Potions}\index{Potions}\label{crearepozioni}

A potion contains the infusion of only one spell, each potion is therefore disposable.

\medskip

The production cost of the Potion is equal to level*level*80, a Potion with Invisibility costs 2*2*80=320 gp

\medskip

To create a potion, a character needs a horizontal work surface and some containers to brew the liquids, as well as a heat source to boil the brew.

A Potion can contain a maximum spell level of 3.

All ingredients and materials for brewing a potion must be fresh and unused.

The caster must know the spell he places in the potion. The brewing time of a potion is equal to double the spell level contained in hours.

\textbf{Item Creation Feat required}: Distill Potions.

\medskip

\begin{center}
\includegraphics[width=0.7\linewidth]{immagini/potion2.png}

\emph{A witch, raising her arm above a flaming cauldron, recites a spell; a young woman kneels in front of the cauldron. Mezzotint by J. Dixon after J.H. Mortimer, 1773}
\end{center}


\subsection{Create Rods}\index{Create Rods}\index{Rods}\label{creareverghe}

A rod is a special wand that is capable of regenerating its charges. They are precious and very expensive objects.

To craft a staff, a character needs a supply of materials, the most obvious of which is a staff or pieces of a staff to assemble.

\medskip

The production cost of the Rod is equal to level*level*3200, a Rod with Invisibility costs 2*2*3200=12800 gp

\bigskip

A rod is able to cast its spell once per day.

Multiply the cost by 4 if he is able to cast it 2 times, multiply by 8 if he is able to cast it 3 times per day.

You can also cast the spell contained in the rod once more per day, after which the rod is destroyed.

A rod can contain a maximum spell level of 3.

The caster must know the spell he places on the Rod.

Crafting a rod takes 1 day for every 1000 gp of the base price.

\textbf{Item Creation Feat required}: Craft Greater Magic Items.

\subsection{Add New Capacities}\index{Add New Capacities}\label{aggiungerecapacitamagiche}

Sometimes, a lack of funds or time makes it impossible to craft the desired magic item, but fortunately, it is possible to enhance or modify a created magic item. Only time, gold and the various prerequisites required by the new ability that you want to add to the magical object place restrictions on the type of additional powers that one can imbue.

The cost to add additional abilities to an item is the same as if the item were nonmagical, minus the value of the original item. Thus, a +1 longsword can become a +2 vorpal longsword, and the cost of creation is equal to that of a +2 vorpal longsword minus the cost of a +1 longsword.

There are many factors to consider when determining the price of an invented magical item. The easiest way to decide on price is to compare the new item to an item that already has a price, and use that price as a guide.

\end{multicols}

\vfill

\begin{center}
\includegraphics[width=0.15\linewidth]{immagini/Rod_of_asclepius.png}

\emph{The staff of Asclepius is an ancient Greek symbol associated with medicine. It consists of a snake coiled around a rod.}
\end{center}


\pagebreak

\section{Magic Item Rules}\index{Magic Item Rules}\hypertarget{identifyom}{}\label{regoleoggettimagici}

\begin{multicols}{2}

\lettrine[lines=2, lhang=0.33, loversize=0.25, findent=1.5em]{Q}{these} are the indications on the use of magic objects.

\label{oggetti-magici}
\begin{itemize}
\item
A character can \textbf{carry numerous (up to 10) magic objects} on him but to determine the bonus to \textbf{Defense} no more than 2 objects can be added (e.g. 1 magic ring and a bracelet) . Armor and Shield are not considered in this count.
\item
The same principle applies to the bonus to \textbf{Saving Throws}, you can only add bonuses from two items.
\item
If you have multiple magic items that grant bonuses to \textbf{Characteristics} then only the one with the highest bonus is counted.
\item
A character \textbf{cannot wear more than two magic rings} otherwise they resonate causing 1d6 damage (not reducible or curable) per round for each ring beyond the second.
\item
\textbf{recognize a magic item} and its abilities require an Arcana check at DC 30. \textbf{10 minutes}. With Arcana score 6 it costs 5 minutes, with 12 it costs 1 minute, with Arcana 18 it costs 1 Round.
\item
A \textbf{magic object that manifests spells} does not perform any Magic Tests. The \textbf{Saving Throw} it imposes, if not specified, is equal to 10 + level*2 of the spell it manifests.\index{Saving Throw for object spells}
\item
\textbf{Activate magical abilities}: unless otherwise indicated, activating a magical ability of an object costs 2 Actions.
\item
A magical object that provides a \textbf{static bonus (or penalty)} applies its value even if the object has not been identified, the Storyteller will silently apply this bonus to Defense, Attack Rolls, Saving Throws... .informing the player who perceives how the object interacts with the situation.
\item 
A magical item that has daily uses recharges at dawn the day after it is used.


\end{itemize}

\subsubsection{Weapons}

\textbf{Weapons}: A weapon with a special ability must have at least a +1 bonus. Weapons cannot have the same special ability more than once.

The magic bonus of a \textbf{weapon can be identified} by rolling two critical on an attack roll or by dedicating 1 hour of training, any magical talents or abilities remain hidden.

\subsubsection{Armor and Shields}

\textbf{Weapons}: Armor or shield with a special ability must have at least a +1 bonus. Armor and Shields cannot have the same special ability more than once.

\textbf{Armor}: every +2 magic lowers the penalty to Feats by 1 and the penalty to Magic Tests by one die.

\textbf{Shields}: every +2 magic removes one die from the Magic Test

\textbf{The cost of Weapons and Armor}: larger than Medium in size is at least double (or quadruple depending on the size). Small armor or Small weapons, although requiring less material, cost the same amount as medium weapons and armor.

\subsection{Bounty and Magic Items}\label{tagliaoggettimagici}

\label{taglia-e-oggetti-magici}

When a magical piece of clothing or jewelry is discovered, size is most often not an issue—many magical clothing is easy for everyone to wear or magically fits the wearer. As a rule, size should not prevent characters of various body types from using a magic item.

There may be some rare exceptions, especially with items made for a specific race.

Randomly found weapons and armor have a 30% chance of being Small (01--30), a 60% chance of being Medium (31--90), and a 10% chance of being another size . Armor does not scale to the owner's size unless otherwise indicated.

\subsection{Magic Items on the Body}\index{Magic Items on the Body}

\label{oggetti-magici-sul-corpo}

Many magical items must be worn by a character who wishes to use them or benefit from their abilities. A humanoid creature can wear up to 10 magical items at a time. Each of these items must be worn over a specific part of the body called \textbf{slot}.

A humanoid-shaped body can wear magical equipment on these body parts:

\textbf{Fingers}: rings (two maximum).

\textbf{Clothes}: cuirasses, armor, tunics and robes

\textbf{Belt}: belts.

\textbf{Necklace}: amulets, necklaces, medallions, scarabs, brooches, talismans and scarves

\textbf{Hands}: Weapons, gauntlets and gauntlets.

\textbf{Eyes}: eyes, glasses and lenses.

\textbf{Feet}: shoes, boots and slippers.

\textbf{Wrist}: bracelets and bracelets.

\textbf{Arms}: shields.

\textbf{Shoulders}: capes and cloaks.

\textbf{Head}: hats, tiaras, helmets, masks, crowns, bands and phylacteries.

\textbf{Chest}: shirts, jackets, sweaters and cloaks.

\subsection{Saving Throws Against the Powers of Magical Items}\index{Saving Throws}

\label{tiri-salvezza-contro-i-poteri-degli-oggetti-magici}

Magic items normally cast spells or other magical effects. For a saving throw against magic or a magical effect generated by a magical item, the DC is 10 + level of the manifested spell x2 unless otherwise specified.

\subsection{Damaging Magic Items}\index{Damaging Magic Items}

\label{danneggiare-gli-oggetti-magici}

A magical item does not need to make a saving throw unless it is unattended, is the specific target of the effect, or its owner rolls a natural 0 (three times 1) on his saving throw.

Magical objects are always entitled to a saving throw against something that could harm them, even when a normal object of the same type would have no chance of saving. Magical items always use the same bonus on saving throws, regardless of type (Fortitude, Reflex, or Will). A magical item's saving throw bonus is equal to 2 + 2x the level of the most powerful spell it houses (or +4 for every +1 it has). The only exceptions to this rule are intelligent magical objects, which make Will saving throws based on their Wisdom score.

\subsection{Repairing Magic Items}\index{Repairing Magical Items}
\label{riparare-gli-oggetti-magici}

Repairing a magical item requires materials and time, equal to half the time and cost of creating it.

\subsection{Refills, Doses and Multiple Uses}\index{Refills}\index{Doses}\index{Multiple Uses}

\label{cariche-dosi-e-usi-multipli}

Many objects, particularly wands and staves, have power limited to the number of charges they contain. Normally, objects with charges never exceed the maximum of 20 charges (10 for staves). If similar items are found as a random part of a treasure, roll 5d6 and divide by 2 to determine the number of charges remaining (rounding down, minimum 1). If an object has a maximum number of charges other than 20, roll randomly to see how many charges are left.

The prices indicated refer to items at their maximum charges (when an item is created, it always has its maximum charges). The value of an object depends on the number of residual charges, in the case of objects that can be used even with few or without charges, the value remains higher.

\end{multicols}

\subsection{Acquiring Magic Items}\index{Acquiring Magic Items}\index[Tables]{Acquiring Magic Items Table}

\label{acquisire-oggetti-magici}

\bigskip

\begin{tabular}{llllll}
\textbf{Community Size} & \textbf{Base Value} & \textbf{Common} & \textbf{Uncommon} & \textbf{Rare}\\
\toprule
Settlement & 50th & 1d2 items && \\
Village & 200mo& 1d4 objects && \\
Village & 500gp & 1d6 items & 1d2 items & \\
Small town & 1000mo & 1d4 items & 1d2 items & \\
Large town & 2000th & 1d6 items & 1d4 items & 1d2 items\\
Small town & 4000gp & 2d4 items & 1d6 items & 1d4 items\\
Large city & 8000gm & 3d4 items & 2d4 items & 1d6 items\\
Metropolis & 16000mo & {*} & 3d4 objects & 2d4 objects\\
\end{tabular}

{*} Almost all minor magical items are found in a metropolis.

\begin{multicols}{2}

\bigskip

Magic items are precious, and most large cities have at least one or two magic item vendors, from a simple potion seller to a blacksmith who specializes in forging magical swords. Of course, not every item in this manual is available in every city.

The following guidelines help Storytellers determine what items are available in a specific community. They assume a campaign with an average level of magic. Some cities may deviate greatly from this baseline at the Storyteller's discretion. The Storyteller should keep a list of items available from each merchant and should occasionally replenish supplies with new acquisitions.

The number and types of magic items available in a community depend on its size. Each community has a base value linked to it (see Table: Available Magic Items).

there is a 75\% chance that any item of that value or less can be found for sale easily in that community. Additionally, the community has a number of other items for sale. These objects are determined randomly and are divided into categories (minor, medium or major).

After determining the number of items available in each category, consult the Random Generation of Magic Items chapter to determine the type of each item (potion, scroll, ring, weapon, etc.) before moving on to the specific tables to determine the item Exactly. Reroll whenever the items do not meet the community's base value.

If the use of magic in the campaign you play is rare, you should halve the base value and number of items in each community. In campaigns with extremely rare or no magic, there may be no magic items for sale at all. Storytellers running these types of campaigns should make adjustments to the challenges faced by characters due to the lack of magic items.

Campaigns with abundant magic items may have communities with double the established base value and random magic items available. Alternatively, all communities could be made to be one size category larger for the purpose of determining available magic items. In a campaign with very common magic, all magic items can be purchased in a metropolis.

Nonmagical items and gear are typically available in a community of any size unless the item is very expensive, such as full armor, or made of an unusual material, such as an adamantium longsword. These items should follow the base value guideline to determine their availability, at the Storyteller's discretion.

\end{multicols}

\vfill

\begin{center}
\includegraphics[keepaspectratio,width=0.90\textwidth]{immagini/Alchemical_laboratory_Wellcome_M0005193.png}

\emph{Alchemical laboratory}
\end{center}

\pagebreak


\section{Random generation of Magic Items}\index{Random generation of Magic Items}\label{generazionetesorimagici}\hypertarget{generation of magic treasures}{}

\begin{changemargin}{0.3cm}{0.3cm}\begin{enfasi}
{Like all unrequited love, even that for things comes at a price in the long run. (Adolfo Bioy Casares)}
\end{enfasi}\end{changemargin}\medskip

\begin{multicols}{2}

In preparing the adventure, the Narrator can place the magical objects he prefers, whenever necessary, or in pure OSR style rely on random generation.

This approach is not always recommended, the results could change the adventure if not the entire campaign! Usually if an enemy has a magic item there is a reason and this item will have a purpose. The fact remains that every now and then, rolling dice on the tables for the random generation of magical treasures is very satisfying and fun!

\begin{changemargin}{0.3cm}{0.3cm}\begin{narratore} %box narrator
Yeru is a world with a low magical profile, magical objects exist but they are rare and even more powerful ones. While natural potions and small trinkets can be found everywhere what has prevented the creation of so many items is the cost to create them. Maybe they were once more accessible but now the creation of the most powerful objects, and I mean swords +3 not wonderful or almost unique objects, require resources that almost no one has or is interested in spending.

This means that if you want to have any hope of obtaining some magical object it is necessary to explore the oldest, least known areas... and the deeper you go the more likely you are to find something.
\end{narratore}\end{changemargin}

First of all, it is necessary to establish what type of object will be generated.

\medskip

\textbf{Table: Type of Magic Item}\index[Tables]{Table Type of Magic Item}

\medskip

\begin{tabular}{lc}
\textbf{Type of magic item}&\textbf{3d6}\\
Amulets, Necklaces, Jewels&3-4\\
Belts, Helmets, Boots and Gloves&5-6\\
Armor and Shields&7-8\\
Magical Weapons&9-10\\
Potions, Filters and Oils&11-13\\
Wands, Sticks and Rods&14\\
Rings&15\\
Hats, Cloaks, Glasses, Tunics&16\\
Manuals and Tomes&17\\
Various Magical Items&18\\
\end{tabular}

\subsubsection{Weapons}

\textbf{Table: Weapon Generation}\index[Tables]{Weapon Generation Table}

\medskip

\begin{tabularx}{0.45\textwidth}{lX}
\textbf{1d100} & \textbf{Magic Bonus}\\
1-50 & +1\\
51-65 & -1 Cursed\\
66-72 & +2\\
73-76 & +3\\
77-79 & +4\\
80 & +5\\
81-87 & retreat + Special Ability Type 3 Weapons\\
88-91 & reroll + Special Ability Type 2 Weapons\\
92-94 & retreat + Special Ability Type 1 Weapons\\
95-100 &-2 Cursed\\
\end{tabularx}

\medskip

When in the \emph{Magic Bonus} it says \emph{reroll + Special Ability Type Weapons...} it means that you must reroll the 1d100, ignoring other results above 80 and keep the magic bonus obtained, then you can roll on the \emph{Special Ability Table Weapon Type...} resulting.

\medskip

\textbf{Table: Special Ability Type 1 Weapons}\index[Tables]{Table Special Ability Type 1 Weapons}

\medskip

\begin{tabular}{ll}
\textbf{1d100} & \textbf{Special Ability Type 1 Weapons}\\
1-8 &Accumulate Spells\\
9-16 &Anathema\\
17-21& Dancing\\
22-27& Defensive\\
28-34& Destroyer of Giants\\
35-41& Destruction\\
42-47& Light Energy\\
48-54& Gloriosa\\
55-60& Guardian\\
61-63& Lucky\\
65-70& Thief of Nine Lives\\
71-73& Sacred\\
74-80& Ghost Touch\\
81-86& Vampire\\
87-92& Speed\\
93-99& Cursed Weapon\\
100 &Vorpal\\
\end{tabular}

%\medskip
%
%\begin{center}
%\includegraphics[width=0.55\linewidth]{immagini/armatura-med.png}
%\end{center}
%
%\medskip

\textbf{Table: Special Ability Type 2 Weapons}\index[Tables]{Table Special Ability Type 2 Weapons}

\medskip

\begin{tabular}{ll}
\textbf{1d100} & \textbf{Special Ability Type 2 Weapons}\\
1-8& Conductive\\
9-16&Brave\\
17-23&Cruel\\
24-30&Duel\\
31-36&Innate Fury\\
37-43&Vital Impulse\\
44-58&Immoral\\
59-60&Lethal\\
61-65&Perfidious\\
66-69&Pietosa\\
70-74&Punitive\\
75-79&Cursed\\
80-85&Dismissive\\
87-95&Terror\\
95-100 &Titanica\\
\end{tabular}

\bigskip

\begin{center}
\includegraphics[width=0.7\linewidth]{immagini/shield1.png}
\end{center}

\textbf{Table: Special Ability Type 3 Weapons}\index[Tables]{Table Special Ability Type 3 Weapons}

\medskip

\begin{tabular}{ll}
\textbf{1d100} & \textbf{Special Ability Type 3 Weapons}\\
1-4& Adaptive\\
5-8 &Sharp\\
9-12& Dragon Slayer\\
13-16& Giant Slayer\\
17-20 &Hunter\\
21-24 &Corrosive\\
25-28& Designator\\
29-32& Distance\\
33-36& Extinguish Fire\\
37-40 &Fanatizing\\
41-44 &Injury\\
45-48 &Dazzling\\
49-52 &Frozen\\
53-56 &Fiery\\
57-60 &Marina\\
61-65 &Masking\\
65-69 &Ghost Ammo\\
70-72 &Infinite Ammo\\
73-76 &Planar\\
77-80 &Pprehensile\\
81-82 &Researcher\\
83-84 &Returning\\
85-88 &Thundering\\
89-91 &Transforming\\
92-95& Findthings\\
96-100 & Cursed Weapon\\
\end{tabular}

\medskip

\subsubsection{Armor and Shields}

\textbf{Table: Armor/Shield Generation}\index[Tables]{Armor/Shield Generation Table}

\medskip

\begin{tabularx}{0.45\textwidth}{lX}
\textbf{1d100} & \textbf{Magic Bonus}\\
1-50 & +1\\
51-65 & -1 Cursed\\
66-72 & +2\\
73-76 & +3\\
77-79 & +4\\
80 & +5\\
81-85 & reroll + Special Ability Armor / Shield Type 2\\
86-90 & reroll + Special Ability Armor / Shield Type 1\\
91-100 &-2 Cursed\\
\end{tabularx}

\medskip

When in the \emph{Magic Bonus} it says \emph{reroll + Special Ability Armor/Shield Type...} it means that you must reroll the 1d100, ignoring other results above 80 and keep the magic bonus obtained, then you can roll on the resulting \emph{Armor/Shield Type Special Ability Table...}.

\textbf{Table: Special Armor/Shield Ability Type 1}\index[Tables]{Table Special Armor/Shield Ability Type 1}

\begin{tabularx}{0.45\textwidth}{lX}
\textbf{1d100} & \textbf{Special Ability Armor/Shield Type 1}\\
1-5 & Aries\\
6-10 &Balanced\\
11-15& Archer's Bracers\\
16-20& Defense Bracers\\
21-25& Major Defense Bracers\\
26-30& Bright\\
31-35& Determination\\
36-40& Spell Defense\\
41-45& Elegant\\
46-50& Hospitable\\
51-55& Poison Resistance\\
56-60& Energy Resistance\\
61-65& Superior Energy Resistance\\
66-70& Wild\\
71-75& Dragon Scales\\
76-80& Animated Shield\\
81-85& Projectile Attraction Shield\\
86-90& Dragon's Breath\\
91-95& Ghost Touch\\
95-100 & Armor/Cursed Shield\\
\end{tabularx}

\medskip

\textbf{Table: Special Armor/Shield Ability Type 2}\index[Tables]{Table Special Armor/Shield Ability Type 2}

\begin{tabularx}{0.45\textwidth}{lX}
\textbf{1d100} & \textbf{Special Ability Armor/Shield Type 2}\\
1-5 &Blinding\\
6-10 &Adamantium\\
11-15& Amorphous\\
16-20& Antihaemorrhagic\\
21-25& Wrangler\\
26-30& Load\\
31-35& Demon Armor\\
36-40& Denial\\
41-45& Sweatshirt\\
46-50& Ethereal Form\\
51-55& Invulnerability\\
56-60& Untraceable\\
61-65 &Masking\\
66-70 &Mithral\\
71-75 &Shadow\\
76-80 &Perceptual\\
81-85 &Titanica\\
86-90 &Vulnerability\\
91-100& Armor/Cursed Shield\\
\end{tabularx}

\medskip

When the Cursed special ability is indicated, you must reroll and reverse the weapon's magical bonuses, so +2 armor or shield becomes -2 armor or shield.

\subsubsection{Amulets, Necklaces and Jewels}\index[Tables]{Amulets, Necklaces and Jewels Generation Table}


\begin{tabular}{ll}
\textbf{Object Type}&\textbf{1d8}\\
Amulets, Necklaces and Jewels Type 1&1-6\\
Amulets, Necklaces and Jewels Type 2&7-8\\
\end{tabular}

\medskip

\begin{tabularx}{0.45\textwidth}{lX}
\textbf{1d100} & \textbf{Amulets, Necklaces and Jewels Type 1}\\
1-8 & Anti-Poison Amulet\\
8-12 & Amulet of Gangrene\\
12-18 & Healing Amulet\\
19-26 & Amulet Against Possession\\
27-34 & Amulet of Inevitable Location\\
35 & Amulet of the Planes\\
36-42 & Amulet of Protection from Detection and Location\\
42-46 & Amulet of Physical Resistance\\
47-53 & Explosion Circlet\\
53-60 & Adaptation Necklace\\
61-70 & Necklace of Strangulation\\
71-77 & Fireball Necklace\\
78-83 & Rosary Necklace\\
84-90 & Death Scarab\\
91-100 & Protection Beetle\\
\end{tabularx}

\medskip

\begin{tabularx}{0.45\textwidth}{lX}
\textbf{1d100} & \textbf{Amulets, Necklaces and Jewels Type 1}\\
1-7 & Elemental Gem\\
8-13& Gem of Luminosity\\
9-16& Gem of Sight\\
17-26& Monster Attractor Jewel\\
27-33& Medallion of Thoughts\\
34-41& Medallion of Feather Fall\\
42-49& Pearl of Wisdom\\
50-57& Defense Pin\\
58-60& Talisman of Pure Good\\
61-62& Talisman of Extreme Evil\\
63-70& Talisman of Protection from Poison\\
71-78& Talisman of Health\\
79-85& Sphere Talisman\\
86-100& Worthless jewel
\end{tabularx}


\begin{center}
\includegraphics[width=0.8\linewidth]{immagini/gauntlet.png}\\
\end{center}


\subsubsection{Belts, Helmets, Boots and Gloves}\index[Tables]{Belts, Helmets, Boots and Gloves Generation Table}

\begin{tabularx}{0.45\textwidth}{lX}
\textbf{1d100} & \textbf{Belts, Helmets, Boots and Gloves}\\
1-3 &Giant's Belt\\
3-6 &Dwarf Belt\\
6-11 &Helm of Speech Understanding\\
12 &Helm of Luster\\
13-17 &Helm of the Underwater Movement\\
18-22 &Helm of Telepathy\\
23-26 &Helm of Teleportation\\
27-31 &Bullet Catching Gloves\\
31-35 &Gloves of Orcish Power\\
36-41 &Swimming and Climbing Gloves\\
41-46 &Gloves of Dexterity\\
47-52 &Clumsy Gloves\\
53-58 &Spider Slippers\\
59-63 &Winged Boots\\
64-66 &Running and Jumping Boots\\
67-77 &Elven Boots\\
78-83 &Winter Boots\\
84-90 &Boots of Levitation\\
91-95 &Boots of Speed\\
96-100 &Dancing Boots\\
\end{tabularx}

\subsubsection{Wands, Sticks and Rods}\index[Tables]{Wands, Sticks and Rods Generation Table}

Roll 1d8 to determine if a wand or staff or rod is found.

\medskip

\begin{tabular}{ll}\\
\textbf{Object Type}&\textbf{1d8}\\
Chopsticks&1-4\\
Sticks&5-7\\
Rods&8\\
\end{tabular}

\medskip

\textbf{Table: Wand Generation}\index[Tables]{Wand Generation Table}

\medskip

\begin{tabularx}{0.45\textwidth}{lX}
\textbf{1d100} & \textbf{Wand}\\
1-5& Metal Detector Wand\\
6-10 &Wand of Arcane Bolts\\
11-15 &Wand of Conveniences\\
16-20 &Wand of Lightning\\
21-25& Fire Wand\\
26-30& Ice Wand\\
31-35& Individual Wand. of the Magic\\
36-38& Individual Wand. of Enemies\\
39-44& Wand of Illusions\\
45-48& Wand of Detection of Secret Doors\\
46-50& Wand of Light\\
51 &War Mage's Wand\\
52 &Wand of Metamorphosis\\
53 &Wand of Wonder\\
54 &Wand of Negation\\
55-60& Wand of Fireballs\\
61-65 &Wand of Paralysis\\
66-70& Wand of Fear\\
71-75 &Wand Discover Traps\\
76-80& Wand of Secrets\\
81-85& Web Wand\\
86-90& Wand of Binding\\
91-95& Wand of Assisted Escape\\
96-100&Cursed Wand\\
\end{tabularx}

\medskip

\textbf{Table: Stick Generation}\index[Tables]{Stick Generation Table}

\medskip

\begin{tabularx}{0.45\textwidth}{lX}
\textbf{1d100} & \textbf{Staff}\\
62 &Archmage's Staff\\
63-65& Withering Stick\\
66-67& Staff of the Woods\\
68-70& Charming Staff\\
71-72& Striking Stick\\
73-74& Fire Staff\\
75-76& Frost Staff\\
77-78& Healing Stick\\
79-80& Staff of Swarming Insects\\
81-82& Python Staff\\
83 &Staff of Power\\
84-86& Staff of Thunder and Lightning\\
87 &Staff of Witchcraft\\

\end{tabularx}

\medskip

\textbf{Table: Rod Generation}\index[Tables]{Rod Generation Table}

\medskip

\begin{tabularx}{0.45\textwidth}{lX}
\textbf{1d100} & \textbf{Rod}\\
1-10&Enchantment Rod\\
11-20&Rod of Absorption\\
21-30&Immovable Rod\\
31-41&Rod of Mighty Strike\\
42-50&Rod of Sovereign Strength\\
51-60&Rod of Readiness\\
61-70&Rod of Security\\
71-80&Rod of Sovereignty\\
81-90& Tentacle Rod\\
91-100& Cursed Rod\\
\end{tabularx}

\begin{center}
\includegraphics[width=0.8\linewidth]{immagini/cupdrinking.png}\\

\emph{Drinking cup depicting scenes from the Odyssey, Athens 550–525 B.C.}
\end{center}

\subsubsection{Potions, Filters and Oils}\index[Tables]{Potions, Filters and Oils Generation Table}

\begin{tabular}{ll}
\textbf{Potion}&\textbf{1d8}\\
Potion Type 1 &1-4\\
Potion Type 2 &5-7\\
Potion Type 3 &8\\
\end{tabular}

\medskip

\begin{tabular}{ll}
\textbf{1d100} & \textbf{Potion Type 1}\\
1-8 &Climbing Potion\\
9-15 &Growth Potion\\
16-23 &Potion of Heroism\\
24-29 &Potion of Gaseous Form\\
30-35 &Giant Strength Potion\\
36-46 &Healing Potion\\
47-53 &Potion of Deception\\
54-64 &Invisibility Potion\\
65-74 &Potion of Levitation\\
77-78 &Potion of Endurance\\
79-84 &Potion of Breathing Underwater\\
84-90 &Shrinking Potion\\
91-95 &Potion of Speed\\
96-100 &Flight Potion\\
\end{tabular}

\medskip

\begin{tabular}{ll}
\textbf{1d100} & \textbf{Potion Type 2}\\
1-10 &Potion of Animal Clairaudience\\
11-20 &Potion of Animal Clairvoyance\\
21-28 &Animal Control Potion\\
29-33 &Dragon Control Potion\\
34-38 &Potion of Undead Control\\
39-49 &Potion of People Control\\
50-55 &Potion of Plant Control\\
56-66 &Potion of Invulnerability\\
67-77 &Mindreading Potion\\
78-85 &Poison Potion\\
86-95 &Potion of Greater Healing\\
96-100 &Potion of Greater Poison\\
\end{tabular}

\medskip

\begin{tabular}{ll}
\textbf{1d100} & \textbf{Potion Type 3}\\
1-13 & Love Potion\\
14-27 & Treasure Finder Filter\\
28-40 & Sharpness Oil\\
41-53 & Ethereal Shape Oil\\
54-66 & Slippery Oil\\
67-79 & Animal Friendship Potion\\
80-85 & Potion of Longevity\\
86-95 & Potion of Metamorphosis\\
96-100& Greater Poison Potion\\
\end{tabular}

\subsubsection{Rings}\index[Tables]{Ring Generation Table}

\begin{tabular}{ll}
\textbf{Ring}&\textbf{3d6}\\
Ring Type 1 &3-16\\
Ring Type 2 &17-18\\
\end{tabular}

\medskip

\begin{tabular}{ll}
\textbf{1d100} & \textbf{Rings Type 1}\\
1-5 & Spell Storage Ring\\
6-13 & Ring of Aries\\
14-21 & Feather Drop Ring\\
22-28 & Ring of Walking on Water\\
29-35 & Heat Ring\\
36-41 & Ring of Weakness\\
42-47 & Ring of Evasion\\
48-50 & Ring of Animal Influence\\
51-55 & Ring of Deception\\
56-61 & Ring of Freedom of Action\\
61-67 & Swimming Ring\\
68-77 & Protection Ring\\
76-84 & Resistance Ring\\
85-93 & Jump Ring\\
93-100 & Telekinesis Ring\\
\end{tabular}

\medskip

\begin{center}
\includegraphics[width=0.8\linewidth]{immagini/romanring.png}
\end{center}

\begin{tabular}{ll}
\textbf{1d100} & \textbf{Rings Type 2}\\
1-8 & Ring of People Control\\
9-17& Ring of Plant Control\\
18-23& Ring of Water Elementals\\
24-29& Ring of the Air Elementals\\
31-36& Ring of the Fire Elementals\\
37-42& Ring of Earth Elementals\\
43-48& Djinni Summoning Ring\\
49-56& Spell Repelling Ring\\
57-65& Ring of Invisibility\\
66-75& Regeneration Ring\\
76-83& Mind Shield Ring\\
84-90& Ring of Shooting Stars\\
91-92& Ring of Three Wishes\\
92-96& Ring of Three Wishes sold out\\
97-100 & Ring of X-ray Vision\\

\end{tabular}


\subsubsection{Hats, Cloaks, Glasses, Tunics}\index[Tables]{Generation Table Hats, Cloaks, Glasses, Tunics}

\begin{tabularx}{0.45\textwidth}{lX}
\textbf{1d100} & \textbf{Hats, Cloaks, Glasses, Tunics}\\

1-3 &Bandana of Intelligence\\
4-10 &Camouflage Hat\\
11-17& Arachnid Cloak\\
18-23& Charlatan's Cloak\\
24-29& Cloak of Distortion\\
30-40& Elven Cloak\\
41-45& Manta Ray\\
46-50& Bat Cloak\\
51-57& Cloak of Protection\\
58-62& Cloak of Spell Resistance\\
63-68& Cloak of Poisoning\\
69-72& Eyes of Petrification\\
73-75& Charming Eyes\\
76-77& Eyes of the Eagle\\
78-80& Eyes of Detailed Sight\\
80-82& Night Glasses\\
83-86 &Tunic of Mimicry\\
87 &Archmage's Tunic\\
88 &Tunic of Shimmering Colors\\
89-91& Robe of Weakening\\
92-94 &Eye Tunic\\
95-99 &Tunic of Useful Items\\
100 &Robe of the Stars\\
\end{tabularx}


\subsubsection{Manuals and Tomes}\index[Tables]{Manuals and Tomes Generation Table}

\begin{tabularx}{0.45\textwidth}{lX}
\textbf{1d100} & \textbf{Manuals and Tomes}\\
1-9 & Golem Manual\\
10-24 & Good Health Manual\\
25-40 &Action Speed ​​Manual\\
40-54 &Physical Exercise Manual\\
55-69 &Tome of Authority and Influence\\
70-84 &Tome of Understanding\\
85-100& Tome of Clear Thought\\
\end{tabularx}

\subsubsection{Various Magical Items}\index[Tables]{Miscellaneous Magical Item Generation Table}

Roll 1d10 to determine if a rare or legendary magic item is found or from the lists of miscellaneous magic items

\medskip

\begin{tabular}{ll}
\textbf{Object Type} & \textbf{1d12}\\
Various Magical Items 1&1-3\\
Various Magical Items 2&4-5\\
Various Magical Items 3&6-7\\
Various Magical Items 4&8-9\\
Various Magical Items 5&10-12\\
Rare and Legendary&10\\
\end{tabular}

\medskip


\subsubsection{Rares and Legendaries}\index[Tables]{Rares and Legendaries Generation Table}

\medskip

\begin{tabularx}{0.45\textwidth}{lX}
\textbf{1d100} & \textbf{Magic Item}\\
1-3 &Wings of Flight\\
4-6 &Iron Vial\\
7-10 &Elemental Water Amphora\\
11-12& Crab Apparatus\\
13-15& Folding Boat\\
17-20& Type III Preservative Bag\\
21-24& Preservative Bag Type IV\\
25-28& Bean Bag\\
29-30& Efreeti Bottle\\
31 & Potions Jug\\
32-33& Invocation Candle\\
34-35 &Horn of Valhalla\\
36-39 &Phylactery of Youth\\
40-42 &Instant Fortress\\
43-45 &Deck of Wonders\\
46-49 &Miniature with Wonderful Power\\
50-53 &Killing Ammo\\
54-58 &Crystal Ball\\
59-62 &Scroll against elementals\\
63-65 &Scroll Against the Undead\\
66-70 &Sewer Pipe\\
71-75 &Pigments of Wonder\\
76-83 &Cubic Portal\\
84-85 &Well of Many Worlds\\
86-87 &Mirror of Mental Ability\\
88-89 &Mirror Traps Life\\
90-91 &Sphere of Annihilation\\
92-94 &Elemental Air Censer\\
95-96 &Laptop Compartment\\
97-98 &Hooves of Speed\\
99-100 &Hooves of the Zephyr\\
\end{tabularx}

\medskip

\subsubsection{Miscellaneous magic items 1}\index[Tables]{Miscellaneous magic items generation table 1}

\begin{tabularx}{0.45\textwidth}{lX}
\textbf{1d100} & \textbf{Miscellaneous magic items 1}\\
1-8& Purifying water\\
9-17&Opening Battalion\\
18-27&Preservative Bag Type I\\
28-34& Climbing Rope\\
35-43&Efficient Quiver\\
44-48&Locating arrow\\
49-52&Crossbow of Arcane Bolts\\
53-60&Lantern of Revelation\\
61-70&Pearl of Power\\
71-80&Stone of Good Luck\\
81-83&Universal Solvent\\
84-94&Restorative Ointment\\
95-100&Practical Backpack\\
\end{tabularx}

\subsubsection{Miscellaneous magic items 2}\index[Tables]{Miscellaneous magic items generation table 2}


\begin{tabularx}{0.45\textwidth}{lX}
\textbf{1d100} & \textbf{Miscellaneous magic items 2}\\
1-8 &Brazier of Fire Elementals\\
9-17 &Cursed Brazier of Sleep\\
18-27& Cold protection cube\\
28-34& Air Elemental Censer\\
35-43& Hindering Net\\
44-52& Entanglement Net\\
53-60& Animated Attack Broom\\
61-70& Flying Broom\\
71-80& Cursed Flight Broom\\
81-88& Mirror of Duplication\\
89-90& Flying Carpet\\
99-100& Titan Hoe\\
\end{tabularx}

\medskip
\subsubsection{Miscellaneous magic items 3}\index[Tables]{Miscellaneous magic items generation table 3}

\begin{tabularx}{0.45\textwidth}{lX}
\textbf{1d100} & \textbf{Miscellaneous magic items 3}\\
1-8 &Bag of Cancellation\\
9-18 &Jug of Infinite Water\\
19-26& Dimensional Strains\\
27-35& Supreme Glue\\
36-40& Meditation incense\\
41-51& Scroll of protection against magic\\
52-60& Disclosing Powder\\
61-70& Vanishing Dust\\
71-82& Sneeze Powder\\
83-90& Arcane Stone\\
91-96& Weight Stone\\
97-100& Arcane Fan\\
\end{tabularx}

\begin{center}
\includegraphics[width=0.8\linewidth]{immagini/ancientdrum.png}
\end{center}


\subsubsection{Miscellaneous magic items 4}\index[Tables]{Miscellaneous magic items generation table 4}

\begin{tabularx}{0.45\textwidth}{lX}
\textbf{1d100} & \textbf{Miscellaneous magic items 4}\\
1-5 &Vial of Curses\\
6-10 &Battle of Cannibalism\\
11-16& Preservative Bag Type II\\
17-20& Devouring Bag\\
21-25& Steaming Bottle\\
26-31& Portable Hole\\
32-37& Healthy Air Series\\
38-43& Rope of Entanglement\\
44-48& Choke Rope\\
49-50& Horn of Destruction\\
51-52& Cube Force\\
53-58& Iron Binding Bands\\
59-64& Phylactery against non-motions\\
65-69& Incense of Obsession\\
70-71& Deck of Illusions\\
72-76& Hypnotic Crystal Ball\\
77-82& Scroll against werewolves\\
83-84& Earth Elemental Stone\\
85-89& Scary Fife\\
90-92& Arcane Feather\\
93-94& Aridity Dust\\
95-96& Panic Drums\\
97-98& Drums of Stun\\
99-100& Thurible of Cursed Summoning\\
\end{tabularx}


\begin{center}
\includegraphics[width=0.6\linewidth]{immagini/ancientbraziers2.png}

\emph{Teotihuacano Old God vessels: Top - stone brazier in Natural History Museum of Los Angeles County}
\end{center}




\pagebreak

\section{Description of Magic Items}\index{Description of Magic Items}


Magic items are presented alphabetically by grouping categories. The description of a magic item provides the item's name, its category, rarity, and its magical properties.

Although the costs are reported, it is always good to grant magical objects as prizes, treasure, following a mission.

Generally speaking, a Common object, the only one that could easily be found in a large city, can cost from 50 to 100 gp, an Uncommon one between 150 and 500 gp, a Rare one between 500 and 5000 gp, a Very Rare up to 30,000 gp and beyond there is only legend...

Items with a bonus over +2, or Legendary, are never bought, it must be an epic adventure to find them.

\medskip

Spells are also magical objects and as such, if the Narrator allows, they can be purchased (horror! there is nothing more beautiful than finding a new spell among the treasures of an adventure).

A spell costs in gold coins level*level*level*80\index{Buying a spell}

\subsection{Special Abilities of Magical Weapons}

\lettrine[lines=2, lhang=0.33, loversize=0.25, findent=1.5em]{U}{n'} weapon with a special ability must have at least +1 magic bonus.

Listed here are the magical abilities that an armor, shield or weapon can have in addition to the generic magical bonus (+1,+2....). Use this list as guidelines and examples, same thing for prices, use them as an indication of rarity.

\index[Magic Items]{Magic Weapons!Magic Weapons}\smallskip* \textbf{Magic Weapon}

\emph{Weapon (any)} +1 1800 gp, +2 6000 gp, +3 17000 gp, +4 45000 gp, +5 80000 gp

You have a bonus on attack rolls and damage rolls made with this weapon. The bonus is determined by the rarity of the weapon. Some magical weapons have additional properties, such as emitting light.

\smallskip* \textbf{Accumulate Spells}\index[Magic Items]{Magical Weapons!Accumulate Spells}

A Spell Storage weapon allows a spellcaster to store a targeted spell up to level 3 in the weapon. The spell must have a standard casting time of 2 Actions. Whenever the weapon hits a creature and the latter takes damage, the wielder of the weapon can release the spell with an immediate action.

Once the spell is cast, a spellcaster can store any other targeted spells within it, again up to level 3.

The weapon magically reveals to its wielder the name of the spell currently contained. A randomly created Spell Storage weapon has a 50\% chance to already have an enchantment contained within it. This special ability can only be added to melee weapons.

A Spell Storage weapon emits a strong aura of the Invocation school, plus the aura of the spell it contains.

\textbf{Details}: Aura Invocation strong and variable; Requirements to Craft Greater Magic Items, Cost +3000 gp.

\smallskip* \textbf{Adaptive}\index[Magic Items]{Magical Weapons!Adaptive}

This ability can only be added to composite bows. An Adaptive bow reacts to the strength of its wielder, acting as a bow with a Strength bonus equal to that of its wielder. The wielder can fire with a lower Strength bonus (and cause less damage) if desired.

\textbf{Details}: Weak Transmutation Aura; Requirements for Creating Magic Items, List of Animals and Plants; Cost +1500 gp.

\smallskip* \textbf{Sharpened}\index[Magical Items]{Magical Weapons!Sharpened}

This critical ability allows you to count the number of 6s rolled by increasing it by 1. Only slashing or piercing melee weapons can be sharpened.

\textbf{Details}: Moderate Transmutation Aura; Requirements to Craft Greater Magic Items, Land List; Cost +5000 gp.

\index[Magical Items]{Magic Weapons!Dragon Slayer}\smallskip* \textbf{Dragon Slayer}

When you hit a dragon with this weapon, the dragon takes an additional 3d6 damage of the weapon's type. For the purposes of this weapon, dragon is any creature of the dragon type.

\textbf{Details}: Moderate Invocation Aura; Requirements to Create Greater Magic Items; Cost +8000 gp.

\index[Magic Items]{Magic Weapons! Giant Slayers}\smallskip* \textbf{Giant Slayers}

When you hit a giant with this weapon, the giant takes an additional 2d6 points of damage of the weapon's type and must succeed on a DC 18 Fortitude save or fall prone. For the purposes of this weapon, giant any creature of the giant type.

\textbf{Details}: Moderate Invocation Aura; Requirements to Create Greater Magic Items; Cost +8000 gp.

\smallskip* \textbf{Destroyer of Giants}\index[Magical Items]{Magical Weapons!Destroyer of Giants}

You must wear the \emph{giant belt} (any variety) and the \emph{gloves of orcish power} to be able to use this weapon.

While using the hammer your Strength score increases by 2 (to a maximum of 7).

When you roll a critical attack roll with this weapon against a giant, the giant must succeed on a DC 21 Fortitude save or die.

You can expend 1 charge and make a ranged weapon attack by throwing it as if it had a range of 20 feet. If the attack hits, the hammer produces a thunderclap that can be heard up to 300 feet away. The target and all creatures within 30 feet of it must succeed on a DC 21 Fortitude save or be stunned until the end of your next round.

The hammer has 5 charges, and regains 1 spent charge each day at dawn.

\smallskip* \textbf{Anathema}\index[Magic Items]{Magical Weapons!Anathema}

A Curse weapon excels at attacking certain creatures. Against the favored enemy, his effective bonus becomes +2. The weapon also deals an additional +2d6 damage against that enemy. The following table is used to randomly determine the weapon's chosen enemy:

\medskip

\begin{tabular}{ll}
d\% &Favored Enemy\\
01-05 &Aberrations\\
06-09 &Beasts\\
10-16 &Constructs\\
17-22 &Dragons\\
23-27 &Faeries\\
28-60 &Humanoids (choose subtype)\\
61-70 &Magical Creatures\\
71-72 &Oozes\\
73-88 &Fiends\\
89-90 &Plants\\
91-98 &Undead\\
99-100 &Insects\\
\end{tabular}

\medskip

\textbf{Details}: Moderate Summoning Aura; Greater Magic Item Crafting Requirements, Abjuration List; Cost +3000 gp.

\smallskip* \textbf{Hunter}\index[Magic Items]{Magical Weapons!Hunter}

A Hunter's weapon helps its wielder locate and capture prey. While the weapon is held in the hand, the wielder gains the weapon's bonus on Survival checks made to track any creatures the weapon has damaged over the previous day. Deals +1d6 damage to creatures whose tracks have been tracked with Survival by the wielder in the previous day.

\textbf{Details}: Moderate Divination Aura; Requirements to Create Greater Magic Items, Locate Animals and Plants; Cost +3000 gp.

\smallskip* \textbf{Conductive}\index[Magic Items]{Magic Weapons!Conductive}

A Conductive weapon can channel the energy of a magical ability that requires a melee or ranged touch attack to hit its target.

When the wielder successfully makes an attack of the appropriate type, he can choose to expend two uses of his spell-like ability to channel it through the weapon to strike the opponent, who suffers the effects of the weapon's attack and those of the special ability (such as channeling energy, laying on of hands...).

This weapon's special ability can only be used once per round (even if it has multiple Conductive weapons).

\textbf{Details}: Moderate Necromancy Aura; Requirements to Create Greater Magic Items, Magic Hand; Cost +3000 gp.

\smallskip* \textbf{Brave}\index[Magical Items]{Magical Weapons!Brave}

This special ability can only be added to a melee weapon. A Brave weapon strengthens its wearer's courage and morale in battle. The wielder gains a bonus on saving throws against fear equal to the weapon's bonus.

\textbf{Details}: Weak Enchantment Aura; Requirements to Create Magic Items, Heroism, Fear; Cost +3000 gp.

\smallskip* \textbf{Corrosive}\index[Magic Items]{Magical Weapons!Corrosive}

On command, a corrosive weapon coats itself in a layer of acid that deals an additional 1d6 acid damage when it hits the target. The acid does not harm the wielder. The effect lasts until a new command is given.

\textbf{Details}: Moderate Invocation Aura; Craft Magic Item Requirements, Acid Arrow; Cost +3000 gp.

\smallskip* \textbf{Cruel}\index[Magical Items]{Magical Weapons!Cruel}

A Cruel weapon feeds on fear and suffering. When the wielder strikes a frightened creature with a cruel weapon, the creature becomes sickened for 1 round. When the wielder uses the weapon to render unconscious or kill a creature, she gains 5 temporary hit points that last for 10 minutes.

\textbf{Details}: Weak Necromancy Aura; Craft Magic Item Requirements, Fear, Cost +3000 gp.

\smallskip* \textbf{Dancing}\index[Magic Items]{Magic Weapons!Dancing}

As a standard action, a Dancer weapon can be released to fight on its own. The weapon fights for 4 rounds using the Defense of the one who released it and then falls to the ground.

It always remains close to the person who freed it, even if they move by physical or magical means. If the one who vacated it has a free hand, he can regain the attacking weapon on its own, as an immediate action, but once recovered, the sword will no longer be able to dance (attack on its own) for 4 rounds.

This ability can only be added to melee weapons.

\textbf{Details}: Strong Transmutation Aura; Requirements to Create Greater Magic Items, Animate Objects; Cost +25000 gp.

\smallskip* \textbf{Designer}\index[Magical Items]{Magical Weapons!Designer}

This special ability can only be added to ranged weapons or ammunition. Whenever a ranged weapon or ammo with this ability hits a creature, it magically designates the target. All allies gain a +2 bonus on attack rolls for 1 round. Multiple hits on the same target do not increase the bonuses or their duration.

\textbf{Details}: Moderate Enchantment Aura; Requirements to Craft Greater Magic Items, Light; Cost +6000 gp.

\smallskip* \textbf{Defensive}\index[Magic Items]{Magic Weapons!Defensive}

A Defensive weapon allows the wielder to transfer part or all of the weapon's bonus to their Defense as a bonus that can be combined with any other bonuses. As an immediate action, the wielder can choose how to use the weapon's bonus at the start of the round, before using it, and the Defense bonus lasts until the next round.

\textbf{Details}: Moderate Abjuration Aura; Greater Magic Item Crafting Requirements, Shield; Cost +3000 gp.

\smallskip* \textbf{Distance}\index[Magic Items]{Magic Weapons!Distance}

This special ability can only be added to projectiles. A Ranged projectile has double the range given by the weapon it launches.

\textbf{Details}: Moderate Divination Aura; Craft Magic Item Requirements, Clairvoyance; Cost +3000 gp.

\smallskip* \textbf{Destruction}\index[Magic Items]{Magical Weapons!Destruction}

A weapon of Destruction is the bane of all Undead. Each undead creature hit in combat must succeed on a DC 14 Will save or be destroyed or take an additional 2d8 Light damage. A weapon of Destruction must be a bludgeoning melee weapon.

\textbf{Details}: Strong Summon Aura; Crafting Greater Magic Item Requirements, Healing; Cost +6000 gp.

\smallskip* \textbf{Duel}\index[Magic Items]{Magical Weapons!Duel}

This ability can only be granted to a melee weapon. A dueling weapon (which must be a weapon that can be used with the Weapon Finesse feat) grants the wielder a +1d6 bonus on Initiative checks, as long as the weapon has been drawn and wielded when the dueling check is made. Initiative.

\textbf{Details}: Weak Transmutation Aura; Requirements for Creating Magic Items, List of Animals and Plants; Cost +7000 gp.

\smallskip* \textbf{Light Energy}\index[Magical Items]{Magical Weapons!Light Energy}

This object looks like the hilt of a long sword, but without the blade. When you grasp the hilt, you can use two actions to cause a blade of pure luminescence to form, or cause the blade embedded in the hilt to disappear.

As long as the sword exists, this magical longsword has the Versatile property. If you are proficient with short swords or long swords, you are also proficient with the sun blade.

You gain a +2 bonus on attack and damage rolls made with this weapon, which deals Light damage instead of slashing damage. When you hit an undead creature with it, the target takes an additional 1d8 Light damage.

The sword's glowing blade emits bright light in a 10-foot radius and dim light for an additional 10 feet. Light is sunlight. While the blade is active, you can use two actions to expand or contract the range of the bright and dim light by 3 feet each, to a maximum of 30 feet or a minimum of 10 feet each.

\textbf{Details}: Strong Transmutation Aura; Craft Wondrous Magic Item Requirements, Everflame, Solar Blast; Cost +45000 gp.

\smallskip* \textbf{Extinguish Fire}\index[Magic Items]{Magical Weapons!Extinguish Fire}

This special ability can only be added to melee weapons. An Extinguish Fire weapon can extinguish Medium or smaller nonmagical fire. When used against a Fire creature, it deals an additional 1d6 points of damage. The wielder of a fire-extinguishing weapon gains a +2 competence bonus on saving throws against fire-based effects, and the weapon itself is immune to fire damage.

\textbf{Details}: Weak Transmutation Aura; Craft Magic Item Requirements, Water List; Cost +3000 gp.

\smallskip* \textbf{Fanatizing}\index[Magic Items]{Magical Weapons!Fanatizing}

Cursed Weapon. This ability grants a +2 bonus to attacks, however, at the start of battle, it causes the bearer to be seized with uncontrollable rage. The character will attack the closest creature, whether enemy or friendly, until none remain alive within 60 feet.

\smallskip* \textbf{Wounding}\index[Magic Items]{Magical Weapons!Wounding}

This ability can only be added to melee weapons. A wounding weapon deals 1 bleed damage when it hits a creature. Multiple damage from this weapon increases Bleed damage by up to 10.
Bleeding creatures take bleed damage at the start of their round.

Creatures immune to critical hits are immune to bleed damage dealt by this weapon.

\textbf{Details}: Moderate Necromantic Aura; Greater Magical Item Crafting Requirements, Contagion; Cost +6000 gp.

\smallskip* \textbf{Dazzling}\index[Magic Items]{Magic Weapons! Dazzling}

On command, a shocking weapon is engulfed in crackling electricity that deals an additional 1d6 points of electricity damage with each successful hit. This electricity does not harm the wielder of the weapon. The effect always remains active as long as the weapon is drawn.

\textbf{Details}: Moderate Invocation Aura; Requirements to Craft Greater Magic Items, Lightning; Cost +3000 gp.

\smallskip* \textbf{Innate Fury}\index[Magical Items]{Magical Weapons!Innate Fury}

This special ability can only be added to melee weapons. A Fury Fury weapon draws power from the anger and frustration its wielder feels when fighting enemies who refuse to die. Each time the wielder deals damage to an opponent with the weapon, his bonus increases by +1 when making attacks against that enemy (to a maximum total bonus of +5). This additional bonus disappears if the opponent dies, or if the wielder uses it to attack a different creature, misses on an attack roll, or 1 hour passes.

\textbf{Details}: Aura Moderate enchantment; Greater Magical Item Crafting Requirements, Heroism; Cost +4000 gp.

\smallskip* \textbf{Lucky}\index[Magic Items]{Magic Weapons!Lucky}

While you have the sword on you, you also receive a +1 bonus on saving throws.

- \emph{Luck}. If you have the sword on you, you can rely on its luck (it requires no actions) to re-roll an attack roll, proficiency check, or saving throw whose result you are not happy with. You are forced to use the second die result. This property cannot be used again until the next dawn.

- \emph{Desire}. While holding it, you can use two actions to expend 1 charge and cast the wish spell through it. This property cannot be used again until the next dawn. The sword has 1d4-1 charges, and loses this property if it runs out of charges.

\textbf{Details}: Very strong Invocation Aura; Craft Mythic Magic Item Requirements, Wish; Cost +30000 gp.

\smallskip* \textbf{Frozen}\index[Magic Items]{Magical Weapons!Frozen}

On command, a Frost weapon is enveloped in a terrible chill that deals 1d6 points of cold damage per successful hit. This cold does not harm the wielder of the weapon. The effect always remains active as long as the weapon is drawn.

\textbf{Details}: Moderate Invocation Aura; Greater Magic Item Crafting Requirements, Water List; Cost +3000 gp.

\smallskip* \textbf{Glorious}\index[Magic Items]{Magical Weapons!Glorious}

A Glorious weapon glows with a dazzling light equal to that of a Daylight spell when drawn. The wielder cannot suppress this light, although it can be temporarily suppressed by any effect that can suppress daylight.

When a Glorious weapon scores a Critical Hit, the target is Blinded until the start of the wielder's next round (Fortitude DC 14 negates). Only one melee weapon can have the Glorious ability.

\textbf{Details}: Moderate Invocation Aura; Craft Magic Item Requirements, Blindness/Deafness, Daylight; Cost +6000 gp.

\smallskip* \textbf{Guardian}\index[Magic Items]{Magical Weapons!Guardian}

This ability can only be added to melee weapons. A guardian weapon allows the wielder to transfer some or all of the weapon's bonus to their saving throws as a bonus that stacks with all others. As an immediate action, the wielder of the weapon chooses how to distribute the weapon's bonus at the start of his round before using the weapon. The bonus on all saving throws lasts until your next round. Only the weapon's own bonus can be sacrificed, no other bonuses deriving from other effects can be used.

If a weapon has both the Defensive and Guardian abilities, sacrificing a single point of the bonus improves either Defense or Saving Throws, but not both.

\textbf{Details}: Moderate Abjuration Aura; Greater Magic Item Crafting Requirements, Resistance; Cost +3000 gp.

\smallskip* \textbf{Immoral}\index[Magical Items]{Magical Weapons!Immoral}

This ability can only be added to melee weapons. When an Immoral weapon strikes an opponent, it produces a flash of Void that reverberates between the wielder and its target. The energy deals an additional 2d6 points of damage to the opponent and 1d6 points of damage to the wielder.

\textbf{Details}: Moderate Invocation Aura; Greater Magic Item Crafting Requirements, Debilitation; Cost +3000 gp.

\smallskip* \textbf{Life Impulse}\index[Magic Items]{Magical Weapons!Life Impulse}

This special ability can only be added to melee weapons. A Life Surge weapon increases and sustains the wielder's life energy while in the midst of combat. The wielder gains a bonus on saving throws against necromancy effects (including ability damage, ability drain, and maximum hit point reductions from undead powers) equal to the weapon's bonus. Additionally, whenever the wielder gains temporary hit points from any source, he adds the weapon's bonus to it.

\textbf{Details}: Moderate Summoning Aura; Greater Craft Magic Item, Cure Serious Wounds, Greater Restoration Requirements; Cost +6000 gp.

\smallskip* \textbf{Fiery}\index[Magical Items]{Magical Weapons!Fiery}

On command, a flaming weapon is engulfed in flames that deal 1d6 fire damage per successful hit. This fire does not harm the wielder of the weapon. The effect remains active until you turn it off with another command.

\textbf{Details}: Moderate Invocation Aura; Greater Magic Item Crafting Requirements, Fireball; Cost +3000 gp.

\index[Magical Items]{Magical Weapons! Thief of Nine Lives}\smallskip* \textbf{Thief of Nine Lives}

You gain a +2 bonus on attack and damage rolls made with this magical weapon. If you score a critical hit against a creature that has fewer than 100 hit points, it must succeed on a DC 17 Fortitude save or be immediately slain, as the sword drains the life force from its body (constructs and undead are immune to this property).

The sword has 1d8 + 1 charges, and loses 1 charge when a creature is slain. When the sword no longer has charges, it loses this property.

\textbf{Details}: Strong Necromantic Aura; Greater Magic Item Crafting Requirements, Fireball; Cost +25000 gp.

\smallskip* \textbf{Lethal}\index[Magical Items]{Magical Weapons!Lethal}

This special ability can only be added to melee weapons that normally deal nonlethal, stun damage. All damage from a lethal weapon is normal (lethal). On command, immediate action, the weapon suppresses this ability until the wielder orders it to reactivate.

\textbf{Details}: Weak Necromancy Aura; Greater Craft Magic Item Requirements, Cure Light Wounds (reversed); Cost +3000 gp.

\smallskip* \textbf{Navy}\index[Magic Items]{Magical Weapons!Navy}

This special ability can only be added to melee weapons. A Marine weapon works safely in aquatic environments. With the weapon in hand, the wielder gains a bonus on Swim checks equal to double the weapon's bonus.

Additionally, the wielder does not suffer the normal penalties to attack rolls and damage from being underwater, as if he were subject to a freedom of movement spell.

\textbf{Details}: Moderate Necromancy Aura; Requirements to Create Greater Magic Items, Freedom of Movement, Cost +3000 gp.

\smallskip* \textbf{Masking}\index[Magic Items]{Magical Weapons!Masking}

A cloak weapon can be commanded to change its shape and appear as another object of similar size. The weapon retains all its properties (including weight) even when disguised, but does not radiate magic. Only True Seeing or other similar magic reveals the true nature of the transformed weapon. After a Disguise weapon is used to attack, this special ability is suppressed for 1 minute.

\textbf{Details}: Moderate Illusion Aura; Craft Greater Magic Item, Magic Weapon, Disguise Self Requirements; Cost +2000 gp.

\smallskip* \textbf{Ghost Ammo}\index[Magic Items]{Magic Weapons!Ghost Ammo}

This ability can only be granted to ammunition. An ammo with this special weapon ability dissolves 1 round after being thrown. Additionally, if the Bullet hits a target, the wound caused closes as the ammunition disintegrates. The Projectile deals damage normally, but leaves no visible trace of violence.

The price refers to 50 Ghost Ammo.

\textbf{Details}: Moderate Transmutation Aura; Craft Greater Magic Item, Disintegration, Repair Requirements; Cost +1000.

\smallskip* \textbf{Infinite Ammo}\index[Magic Items]{Magic Weapons!Infinite Ammo}

Only bows and crossbows can be weaponized by Infinite Ammo. Whenever an Infinite Ammo weapon is nocked, a single nonmagical arrow or bolt is created spontaneously by its magic, so the wielder never needs to load the weapon with ammunition.

If the wielder attempts to load the weapon with more ammunition, the arrow or bolt created immediately vanishes and the weapon can be loaded as normal. This ability does not reduce the amount of time it takes to charge or fire the weapon. The created arrow or bolt vanishes if removed from the weapon; it persists only if cast. Unlike regular bow or crossbow ammunition, these arrows and bolts are always destroyed when fired.

\textbf{Details}: Moderate Summoning Aura; Requirements to Craft Greater Magic Items, Creation; Cost +6000 gp.

\index[Magic Items]{Magic Weapons! Wicked}\smallskip* \textbf{Wicked}

When you roll a 17 or 18 on an attack roll with this magical weapon, the target takes an additional 7 damage of the weapon's type.

\textbf{Details}: Weak Summoning Aura; Craft Magic Item Requirements, Causes Light Wounds; Cost +3000 gp.

\smallskip* \textbf{Pitiful}\index[Magic Items]{Magic Weapons!Pitiful}

All damage dealt by the weapon is temporary.

On command, the weapon suppresses this ability until commanded to reactivate it (allowing it to deal lethal damage).

\textbf{Details}: Weak Summoning Aura; Craft Magic Item Requirements, Cure Light Wounds; Cost +3000 gp.

\smallskip* \textbf{Planar}\index[Magic Items]{Magical Weapons!Planar}

A Planar weapon is effective against all types of extradimensional beings, being able to overcome their resistance to physical damage. When used to attack Outsiders, a Planar weapon ignores 5 points of their Damage Reduction or Resistances.

\textbf{Details}: Moderate Summoning Aura; Greater Magic Item Crafting Requirements, Planar Shift; Cost +3000 gp.

\smallskip* \textbf{Pprehensile}\index[Magic Items]{Magical Weapons!Pprehensile}

This ability can only be granted to whips. A prehensile whip can, as a move action, grapple onto an object as if it were a grappling hook. The Whip can then be used to scale surfaces or swing across a room or any outdoor area.

\textbf{Details}: Moderate Enchantment Aura; Crafting Greater Magic Item Requirements, Rope Trick; Cost +2,500.

\index[Magic Items]{Magic Weapons!Mace of Punishment}\smallskip* \textbf{Mace of Punishment}

You gain an additional +3 to hit and damage when using this weapon to attack a construct.

When you roll a critical attack roll with this weapon, the target takes an additional 7 bludgeoning damage, or 14 additional bludgeoning damage if it is a construct. If a construct has 25 hit points or fewer remaining after taking this damage, it is destroyed.

\textbf{Details}: Aura Strong Invocation; Requirements to Create Greater Magic Items; Cost +7000 gp.

\smallskip* \textbf{Researcher}\index[Magical Items]{Magical Weapons!Researcher}

This ability can only be added to ranged weapons. A Seeker weapon veers toward its target, negating any miss chance that might apply, such as that from Concealment. The wielder must still aim the weapon in the right square. Arrows accidentally shot into an empty space, for example, do not veer to hit Invisible opponents if any are nearby.

\textbf{Details}: Strong Divination Aura; Requirements to Create Greater Magic Items, True Seeing; Cost +3000 gp.

\smallskip* \textbf{Returning}\index[Magic Items]{Magic Weapons!Returning}

A returning weapon can teleport to its wielder's hand as an immediate action, even if it is in the possession of another creature. This ability has a maximum range of 100 feet, and effects that block teleportation prevent a Returning weapon from returning. A returning weapon must be in a creature's possession for at least 24 hours for this ability to work.

\textbf{Details}: Moderate Summoning Aura; Crafting Greater Magic Item Requirements, Teleportation; Cost +3000 gp.

\smallskip* \textbf{Holy}\index[Magical Items]{Magical Weapons!Holy}

You gain a +3 bonus on attack and damage rolls made with this magical weapon. When you hit a fiend or undead with it, that creature takes an additional 2d10 Light damage.

While you hold the unsheathed sword, it creates a 10-foot radius aura around you. You and all creatures friendly to you within the aura gain +1d6 on saving throws against spells and other magical effects cast by Followers or Devotees of other Patrons. If you have Traits in common with the Patron 13 or more, the aura's radius increases to 30 feet.

\textbf{Details}: Moderate Invocation Aura; Common Traits 12; Requirements to Create Greater Magic Items; Cost +6000 gp.



\smallskip* \textbf{Contemptuous}\index[Magical Items]{Magical Weapons!Contemptuous}

This special ability can only be added to melee weapons. A Contemptuous weapon helps its wielder survive desperate conditions. It remains in the wielder's hands even if the wielder is Frightened, Stunned, or Unconscious. The wielder adds his bonus as a bonus on First Aid checks when unconscious or dying and also adds the same to saving throws against spells that cause instant death.

\textbf{Details}: Strong Abjuration Aura; Craft Greater Magic Item Requirements, Stabilize; Cost +6000 gp.

\index[Magic Items]{Magic Weapons!Terror}\smallskip* \textbf{Terror}

While holding it, you can use two actions and expend 1 charge to unleash a wave of terror.
Each creature of your choice within a 30-foot radius of you must succeed on a DC 17 Will save or be frightened of you for 1 minute. While frightened in this way, a creature must spend its rounds trying to move as far away from you as possible, and can't knowingly move into a space that's within 30 feet of you. It also cannot perform reactions. As its action, it can only use the Move action to Disengage. If it can't move anywhere, the creature can use the Total Defense Action.

At the end of each of its rounds, the creature can repeat the saving throw, ending the effect on itself if it succeeds. This magical weapon has 3 charges, and regains 1d3 charges each day at dawn.

\textbf{Details}: Moderate Enchantment Aura; Greater Magical Item Crafting Requirements, Fear; Cost +8000 gp.

\smallskip* \textbf{Titanica}\index[Magic Items]{Magic Weapons!Titanica}

This weapon is 3 m long and weighs almost 40 kg (8 Encumbrance), it can only be used by a giant (or an enlarged character). If used as a weapon it has a +2 bonus to hit and inflicts 1d4x10 wounds. It can also be used to quickly drive poles as large as tree trunks and to tear down doors and gates with just a few blows.

\textbf{Details}: Moderate Transmutation Aura; Craft Greater Magic Item Requirements, Enlarge/Reduce; Cost +3000 gp.

\smallskip* \textbf{Ghost Touch}\index[Magical Items]{Magical Weapons!Ghost Touch}

A Phantom Touch weapon deals critical damage when it hits ethereal creatures and ignores magical weapon resistances. As long as you have a Phantom Touch weapon, you can see ethereal creatures.

\textbf{Details}: Moderate Summoning Aura; Greater Magic Item Crafting Requirements, Planar Shift; Cost +3000 gp.

\smallskip* \textbf{Thundering}\index[Magical Items]{Magical Weapons!Thundering}

A Thunderous weapon creates a tremendous crash similar to thunder when it scores a Critical Hit. The sonic energy does not harm the person holding the weapon and deals an additional 1d8 points of sonic damage for each successful critical hit. Anyone subjected to a critical hit from a thunderous weapon must make a Fortitude saving throw of DC 14 or remains deaf permanently.

\textbf{Details}: Weak Necromancy Aura; Craft Magic Item Requirements, Blindness/Deafness; Cost +3000 gp.

\smallskip* \textbf{Transforming}\index[Magic Items]{Magic Weapons!Transforming}

This ability can only be added to melee weapons. A transforming weapon alters its form at its wielder's command, becoming any other melee weapon of similar size. For example, a transforming longsword can take the form of any other Medium one-handed melee weapon, such as a Scimitar, light flail, or trident, but not a medium light or two-handed melee weapon (such as a medium shortsword or two-handed greatsword).

The weapon retains all of its abilities, including bonuses and special weapon abilities, except those prohibited by its current new form. If left unattended, the weapon returns to its original form.

\textbf{Details}: Moderate Transmutation Aura; Requirements to Craft Greater Magic Items, Greater Creation; Cost +5000 gp.

\smallskip* \textbf{Things Finder}\index[Magic Items]{Magic Weapons!Things Finder}
This ability allows the wielder of this weapon to cast the locate object spell once per day

\textbf{Details}: Light Divination Aura; Craft Magic Item Requirements, Locate Item; Cost +1000 gp.

\index[Magical Items]{Magical Weapons!Vampire}\smallskip* \textbf{Vampire}

When you attack a creature with this magical weapon and roll a critical attack roll, the target, other than constructs and undead, takes an additional 10 void damage and you gain 10 temporary hit points.

\textbf{Details}: Moderate Necromantic Aura; Greater Magic Item Crafting Requirements, Vampire Touch; Cost +8000 gp.

\smallskip* \textbf{Speed}\index[Magic Items]{Magic Weapons!Speed}

When making multiple attacks (2 Actions), the wielder of a Speed ​​weapon can make an additional attack with the weapon. The additional attack does not have the penalties of multiple attacks. This ability does not stack with similar spells or effects.

\textbf{Details}: Moderate Transmutation Aura; Greater Magic Item Crafting Requirements, Speed; Cost +15000 gp.

\index[Magic Items]{Magic Weapons!Vorpal}\smallskip* \textbf{Vorpal}

Despite being a +1 magic weapon, it is considered a +5 magic weapon to evaluate immunity and bonuses to attack rolls and damage. Additionally, the weapon ignores slashing damage resistance. When you attack a creature that has at least one head with this weapon and roll at least 3 6s on the attack roll, you cut off one of the creature's heads. The creature dies if it cannot survive without losing its head.

A creature is immune to this effect if it is immune to slashing damage, does not have or need a head, or the Storyteller decides that the creature is too large for its head to be severed by this weapon.

Such a creature instead takes an additional 6d8 slashing damage from the hit.

\textbf{Details}: Very strong Invocation Aura; Requirements to Craft Mythical Magic Items; Cost +150000 gp, legendary.

\subsection{Special Abilities of Armor and Magic Shields}

Most magical armor and shields have bonuses only, but some have some of the special abilities described below. Armor or shield with special abilities must have at least a +1 bonus.

\index{Armor / Magic Shield}\smallskip* \textbf{Armor / Magic Shield}

\emph{Armor (any)} +1 2500 gp, +2 10000 gp, +3 18000 gp, +4 35000 gp, +5 80000 gp

\emph{Shields (small, medium, heavy)}: +1 1500 gp, +2 4000 gp, +3 9000 gp, +4 20000 gp, +5 35000 gp

While holding this shield/armor, you have a bonus to Defense determined by the shield/armor's magic bonus. This bonus is in addition to the normal Defense bonus provided by the shield/armor.

\smallskip* \textbf{Blinding}\index[Magical Items]{Armor and Shields!Blinding}

A shield equipped with this enchantment casts a blinding light up to twice per day at the wielder's command. Everyone within 20 feet of the shield, except the wielder, must succeed at a DC 14 Reflex save or be blinded for 1d4 rounds.

\textbf{Details}: Moderate Invocation Aura; Crafting Requirements Craft Greater Magic Items, Daylight; Cost +3000 gp.

\index[Magic Items]{Armor and Shields!Adamantium}\smallskip* \textbf{Adamantium}

Armor (medium or heavy, but not leather), uncommon +700 gp above the base price of the armor. While wearing it, any critical hit you take becomes a normal hit (but does not protect against burst damage).


\smallskip* \textbf{Amorphous}\index[Magic Items]{Armor and Shields!Amorphous}

Once per day on command, the wearer of the armor (along with any equipment he or she is wearing) can take the form of a viscous liquid that is capable of passing through any space through which thick mud could reasonably flow. While you use this ability, your speed is reduced to 10 feet and you can only take move actions. You can assume this form for 1 minute or until you spend a move action to return to your natural form. Amorphous armor must be made primarily of leather, cloth, or other flexible, organic material.

\textbf{Details}: Moderate Transmutation Aura; Construction Requirements Craft Greater Magic Items, Metamorphosis, Cost +2250 gp.


\smallskip* \textbf{Anti-hemorrhagic}\index[Magic Items]{Armor and Shields! Anti-hemorrhagic}

Antihemorrhagic armor helps stop blood loss from the wearer's wounds, automatically tightening like a tourniquet at the appropriate points while also magically reducing the extent of the wound.

Antihemorrhagic armor reduces hit point damage by 1 per hit and you cannot take bleed damage.

\textbf{Details}: Moderate Transmutation Aura; Construction Requirements Craft Greater Magic Items, Cure Critical Wounds, Lesser Restoration, or Stabilize; Cost +3000 gp.

\smallskip* \textbf{Aries}\index[Magic Items]{Armor and Shields!Aries}

These shields are very solid and often bear the emblem of a ram or a bull. When the wearer of a ram shield makes a shield attack as part of a charge, the shield's Defense bonus applies to attack and damage rolls. This does not stack with any other upgrades the shield has. This ability is not applicable to light-type shields.

\textbf{Details}: Weak Invocation Aura; Construction Requirements Craft Greater Magic Items, Cost +3000 gp.


\smallskip* \textbf{Brawler}\index[Magical Items]{Armor and Shields!Brawler}

Those wearing brawler armor gain a +2 bonus on attack rolls and damage rolls for unarmed attacks. His unarmed strikes count as magical weapons for the purposes of overcoming damage reduction. The Brawler ability can only be applied to light armor.

\textbf{Details}: Weak Transmutation Aura; Construction Requirements Craft Magic Items, Strength of the Bull; Cost +15000 gp

\smallskip* \textbf{Balanced}\index[Magic Items]{Armor and Shields!Balanced}

This armor repels anything that threatens to knock the wearer to the ground. The bearer gains a +1d6 bonus against anyone who tries to push or knock him to the ground.

Falling to the ground while wearing Balanced armor is a move action instead of an immediate action. The Balanced ability can be applied to light or medium armor, but not heavy armor or shields.

\textbf{Details}: Weak Transmutation Aura; Construction Requirements Craft Magic Items, Cost +3000 gp.

\index[Magical Items]{Armor and Shields!Archer's Bracers}\smallskip* \textbf{Archer's Bracers}

While wearing these bracers, you have proficiency with the longbow and shortbow, and you gain a +2 bonus to damage rolls for ranged attacks made with these weapons.

\textbf{Details}: Weak Transmutation Aura; Construction Requirements Craft Magic Items, Cost +3000 gp.

\index[Magic Items]{Armor and Shields! Defense Bracers}\smallskip* \textbf{Defense Bracers}
\emph{Wonderful object, rare}

While wearing these bracers, you have a +1, +2, +3, +4+, +5 bonus to your Defense if you are not wearing any armor and not using a shield.

\textbf{Details}: Aura Abjuration; Crafting Requirements Craft Greater Magic Items, Cost +6000 gp, 15000 gp, 30000 gp, 45000 gp, 60000 gp.

\index[Magical Items]{Armor and Shields!Bracelets of Greater Defense}\smallskip* \textbf{Bracelets of Greater Defense}
\emph{Wonderful item, legendary}

These bracelets function like armor even though they are not armor. You are enveloped in an invisible magical shield that grants you Defense 15, 17, 19, 21, 23. Defense can be increased with magical items that improve Defense, except armor and shields.

\textbf{Details}: Aura Abjuration; Crafting Requirements Craft Greater Magic Items, Cost +12,000 gp, 24,000 gp, 36,000 gp, 50,000 gp, 75,000 gp


\smallskip* \textbf{Brilliant}\index[Magical Items]{Armor and Shields!Brilliant}

Armor and shields with the Brilliant special ability radiate light like a torch when worn, which can be suppressed or reactivated on command. The appearance of the object is usually characterized by bright colors and a brilliant luster even when not illuminated. Once per day, the bearer can command his armor or shield to glow with the intensity of a daylight spell for 10 minutes or until commanded to dim it.

This armor must be cleaned at least once a week or it will lose its powers for a week.

\textbf{Details}: Moderate Invocation Aura; Building Requirements Craft Magic Items, Daylight; Cost +3750 gp.

\smallskip* \textbf{Load}\index[Magic Items]{Armor and Shields!Load}

Cargo armor distributes the weight carried by the wearer more effectively, allowing the wearer to carry more without suffering the effects of Encumbrance. The wearer's storage capacity is increased by 50\%.

\textbf{Details}: Weak Transmutation Aura; Crafting Requirements Craft Magic Items, Passive Armor; Cost +2000 gp.

\index[Magical Items]{Armor and Shields!Demonic Armor}\smallskip* \textbf{Demonic Armor}

While wearing armor you can understand and speak the Abyssal. Additionally, the armor's clawed mittens transform unarmed strikes made with your hands into magical weapons that deal slashing damage, with a +1 bonus to attack rolls and damage rolls and a d8 as a damage die.

\textbf{Details}: Strong Summon Aura; Crafting Requirements Craft Greater Magic Items; Cost +5000 gp.

\smallskip* \textbf{Denegation}\index[Magic Items]{Armor and Shields!Denegation}

When the wearer is the target of a Critical Hit or Burst damage made with a melee weapon, he can automatically negate this Critical Hit and make it a normal attack. This ability can only be applied to heavy armor. The ability can be used a number of times per day equal to the weapon's magical bonus.

\textbf{Details}: Strong Abjuration Aura; Crafting Requirements Craft Greater Magic Items; Cost +25000 gp.

\smallskip* \textbf{Determination}\index[Magic Items]{Armor and Shields!Determination}

A shield or armor grants the ability to fight in seemingly impossible circumstances. Once per day, when the wielder reaches 0 or fewer hit points, the item automatically activates the cure serious wounds spell.

\textbf{Details}: Moderate Summoning Aura; Crafting Requirements Craft Greater Magic Items, Cure Serious Wounds; Cost +15000 gp.

\index[Magic Items]{Armor and Shields! Defense against Spells}\smallskip* \textbf{Defense against Spells}

You have +1d6 on saving throws against spells and other magical effects.

\textbf{Details}: Strong Abjuration Aura; Crafting Requirements Craft Greater Magic Items; Cost +5000 gp.

\index[Magical Items]{Armor and Shields!Elegant}\smallskip* \textbf{Elegant}

You can use two actions to speak the command word to make the armor take on the appearance of an ordinary suit or some other type of armor. You decide the appearance, including color, style and accessories, but the armor/shield retains its normal bulk and weight. The illusory appearance lasts until you use this property again or remove your armor.

\textbf{Details}: Moderate Illusion Aura; Crafting Requirements Craft Greater Magic Items; Cost +3000 gp.


\smallskip* \textbf{Sweatshirt}\index[Magic Items]{Armor and Shields!Sweatshirt}

Armor with the Fleece ability counts towards the penalties of wearing armor over light armor. The character can move almost without difficulty with this armor.

\textbf{Details}: Strong Transmutation Aura; Crafting Requirements Craft Greater Magic Items; Cost +6000 gp.

\smallskip* \textbf{Ethereal Form}\index[Magical Items]{Armor and Shields!Ethereal Form}

On command, this property allows the wearer to become ethereal (as the ethereal form spell) once per day. The character can remain Ethereal for as long as he wishes but, once returned to normal, he can no longer become Ethereal for that day.

\textbf{Details}: Strong Transmutation Aura; Crafting Requirements Craft Wondrous Magic Item, Ethereal Form; Cost +24500 gp.

\smallskip* \textbf{Invulnerability}\index[Magic Items]{Armor and Shields!Invulnerability}

This armor grants the wearer damage reduction of 5/magic. Armor with Invulnerability emits a strong Abjuration aura.

\textbf{Details}: Strong Abjuration Aura; Crafting Requirements Craft Mythical Magic Items, Wish; Cost +15000 gp.

\smallskip* \textbf{Untraceable}\index[Magic Items]{Armor and Shields! Untraceable}

Untraceable armor lightens the wearer's steps and camouflages their appearance. Survival checks to track the wearer take a –5 penalty, and the wearer gains a +5 bonus on Stealth checks. Only leather or leather armor can be Untraceable.

\textbf{Details}: Weak Transmutation Aura; Construction Requirements Craft Magic Items, Pass Without Traces; Cost +3750 gp.

\smallskip* \textbf{Masking}\index[Magical Items]{Armor and Shields!Masking}

On command, such armor changes its shape and appears like a normal set of clothes. Armor retains all of its properties (including weight) even when masked. Only True Seeing or other similar magic reveals the true nature of the transformed armor.

\textbf{Details}: Moderate Illusion Aura; Construction Requirements Craft Greater Magic Items, Disguise Self; Cost +1350 gp.

\index[Magical Items]{Armor and Shields!Mithral}\smallskip* \textbf{Mithral}

Medium or heavy armor, but not leather, uncommon +800 gp above the base price of the armor. Mithral is a light, flexible metal. A mail jacket or mithral breastplate can be worn under normal clothing. Reduces the weight category by 1 to determine penalties on proficiency and magic checks.

\smallskip* \textbf{Shadow}\index[Magic Items]{Armor and Shields!Shadow}

This armor makes the wearer blurry whenever he attempts to hide, providing a +4 bonus on his Stealth checks to hide in shadows. The armor check penalty applies normally.

\textbf{Details}: Weak Illusion Aura; Construction Requirements Craft Magic Items, Invisibility, Silence; Cost +1875 gp.

\smallskip* \textbf{Hospitable}\index[Magic Items]{Armor and Shields!Hospitable}

An armor or shield with this special ability hides live animals within its iconography to keep them safe. The bearer with a command word magically stores an animal to which he is bonded, such as a Familiar or a Mount. The stored animal appears as a symbol on your armor or shield, whether it is an imitation of the animal's appearance or a more symbolic, abstract representation.

While stored, the animal sleeps and provides no benefits (such as a familiar's Feat bonus) to the wearer. The size of storable animals depends on the type of armor or shield. Light or medium armor and light or heavy shields can store an animal of up to the wearer's size. Heavy armor or a tower shield can store an animal up to one size category larger than its wearer. A second command word releases the stored animal in the hospitable armor or shield. A freed animal immediately awakens, appears in a space adjacent to the bearer, and can take actions the round it appears.

Since the stored animal is sleeping rather than being in suspended animation (or even hibernating), it ages and hungers at its normal rate while stored. A Hospitable armor or shield automatically releases a stored animal 24 hours after it is stored inside it.

\textbf{Details}: Moderate Summoning Aura; Crafting Requirements Craft Greater Magic Items, Secret Chest; Cost +3750 gp.


\smallskip* \textbf{Perceptive}\index[Magical Items]{Armor and Shields!Perceptual}

Perceptive armor comes to the rescue when the wearer has been blinded, is in total darkness (if the wearer does not have darkvision or the see in the dark ability), or is in magical darkness. When one of these conditions affects the wearer of the armor, perceptive armor immediately grants him blindsight in a 3-foot radius. As soon as the wearer can see again, the additional senses cease. The wearer of the armor cannot gain these abilities by closing his eyes.

\textbf{Details}: Strong Divination Aura; Crafting Requirements Craft Wondrous Magic Item, True Seeing; Cost +15000 gp.

\smallskip* \textbf{Poison Resistance}\index[Magical Items]{Armor and Shields!Poison Resistance}

Armor or shield with this special ability gives the wearer a +3 bonus on saving throws against poison.

\textbf{Details}: Weak Transmutation Aura; Crafting Requirements Craft Greater Magic Items, Remove Poison; Cost +1125 gp.

\smallskip* \textbf{Energy Resistance}\index[Magic Items]{Armor and Shields!Energy Resistance}

This type of armor or shield protects against a type of energy (Fire, Light, Sound, Electricity, Positive Energy, Negative Energy, Cold, Void) and is decorated with designs depicting the element it protects from. The armor or shield absorbs the first 10 points of energy damage per attack that would normally be taken by the wearer.

\textbf{Details}: Weak Abjuration Aura; Construction Requirements Craft Magic Items, Protection from Energy; Cost +9000 gp.

\smallskip* \textbf{Greater Energy Resistance}\index[Magical Items]{Armor and Shields!Greater Energy Resistance}

This type of armor or shield protects against a type of energy (Fire, Light, Sound, Electricity, Positive Energy, Negative Energy, Cold, Void) and is decorated with designs depicting the element it protects from. The armor or shield grants Resistance to the indicated energy.

\textbf{Details}: Moderate Abjuration Aura; Crafting Requirements Craft Greater Magic Items, Energy Protection; Cost +21000 gp.

\smallskip* \textbf{Selvatica}\index[Magical Items]{Armor and Shields!Selvatica}

Armor with this special ability generally appears to be made of magically hardened animal hide. Anyone wearing armor or a shield with this ability retains Defense even while transformed into an animal (either by Spell or Feat).

Armor and shields with this ability usually bear leaf motifs. While the wearer is in Wild Form, the armor is not visible.

\textbf{Details}: Moderate Transmutation Aura; Crafting Requirements Craft Greater Magic Items, Metamorphosis; Cost +15000 gp.

\index[Magic Items]{Armor and Shields!Dragon Scales}\smallskip* \textbf{Dragon Scales}

This armor or shield is made from the scales of some kind of dragon.

While wearing it, you have +1d6 on saving throws against Frightening Presence and dragon breath weapons, and you have resistance to a type of damage determined by the species of dragon that provided the scales.

Additionally, with two actions you can focus your senses to magically determine the distance and direction of the nearest dragon within 28 miles that is of the same species as the armor. This special action cannot be used again until the next dawn.

\textbf{Details}: Moderate Abjuration Aura; Crafting Requirements Craft Greater Magic Items; Cost +8000 gp.

\smallskip* \textbf{Animated Shield}\index[Magic Items]{Armor and Shields!Animated Shield}

While holding this shield, with two actions you can speak a command word and make it animate. The shield will float in the air within your space to protect you as if you were holding it, leaving your hand free.

The shield remains animated for 1 minute, until you use two actions to end its effect, are incapacitated, or die, at which point the shield will fall to the ground or return to your hand if you have a free hand.

\textbf{Details}: Strong Transmutation Aura; Crafting Requirements Craft Greater Magic Items, Animate Objects; Cost +6000 gp.

\index[Magic Items]{Armor and Shields!Shield of Projectile Attraction}\smallskip* \textbf{Shield of Projectile Attraction}

While holding this shield you apparently have resistance to damage from ranged weapon attacks.

\emph{Cursed version}.

Taking off the shield does not end the curse. Whenever a ranged weapon attack is made against a target within 10 feet of you, the curse causes you to become the target of the attack.

\textbf{Details}: Strong Transmutation Aura; Crafting Requirements Craft Greater Magic Items, Animate Objects; Cost +2000 gp.


\smallskip* \textbf{Dragon's Breath}\index[Magic Items]{Armor and Shields!Dragon's Breath}

A shield with this special ability is usually made with a dragon's jaws gaping on the front. A shield with the Dragon's Breath special ability is tied to one type of energy (poison, electricity, cold, or fire). The shield regains 1d4 charges each dawn and can hold up to 10 charges.

On command, 2 Actions, the wearer can consume 1 to 5 charges of the shield to cause it to emit a breath weapon in a 10-foot cone that deals 1d4 points of energy damage per charge expended (Reflex DC 11 halved). This damage is the same energy type tied to the shield. A shield cannot have more than one Dragon Breath ability.

\textbf{Details}: Weak Invocation Aura; Construction Requirements Craft Magic Items, Searing Wave; Cost +2500 gp.

\smallskip* \textbf{Titanica}\index[Magic Items]{Armor and Shields!Titanica}

Armor with the Titanic property is almost comically oversized, even if the effect is only external and the interior accommodates a creature as normal, without requiring modification. A creature wearing Titanic armor is considered one size category higher, including for the purposes of using items and weapons or being affected by special attacks that depend on size, such as Swallow and Trample.

\textbf{Details}: Moderate Transmutation Aura; Crafting Requirements Craft Greater Magic Items, Enlarge; Cost +15000 gp.

\smallskip* \textbf{Ghost Touch}€13799[Magical Items]{Armor and Shields!Ghost Touch}

This armor or shield appears almost transparent. The Defense value given by the armor is counted against attacks from corporeal and Incorporeal creatures. The armor or shield can be picked up, moved, and worn at any time by both corporeal and incorporeal creatures. Incorporeal creatures gain the object's bonus against corporeal and incorporeal attacks, and still retain the ability to pass through solid objects.

\textbf{Details}: Strong Transmutation Aura; Crafting Requirements Craft Wondrous Magic Item, Ethereal Form; Cost +15000 gp.

\index[Magic Items]{Armor and Shields!Vulnerabilities}\smallskip* \textbf{Vulnerabilities}

While wearing it, you have resistance to one of the following damage types: bludgeoning, piercing, or slashing. The Storyteller chooses the type. The armor is cursed; while you are cursed, you have vulnerability to two of the three damage types associated with the armor (other than the one you have resistance to).

\textbf{Details}: Moderate Necromantic Aura; Construction Requirements Craft Magic Items, Cast Curse; Cost +3000 gp.


\subsection{Amulets, Necklaces and Jewels}

\index[Magic Items]{Magic Items! Anti-Poison Amulet}\smallskip* \textbf{Anti-Poison Amulet}
3,000 gp, uncommon, this gem hanging on a silver chain is black and shiny. The wearer has a +1d6 save vs. poison.

\index[Magic Items]{Magic Items! Amulet of Gangrene}\smallskip* \textbf{Amulet of Gangrene}
this engraved gem hanging on a chain appears to be of little value. If a character keeps it with him for more than 1 day, he is struck by terrible gangrene which causes him to permanently lose 1 point of Dexterity, Constitution and Charisma per week. The gem (and gangrene) can only be neutralized by Remove Curse and cure disease, followed by healing or wishing. Gangrene can also be defeated by grinding a health amulet and sprinkling its powder on the afflicted character

\index[Magic Items]{Healing Amulet}\smallskip* \textbf{Healing Amulet}
25,000 gp, very rare, this gem hanging on a gold chain is red and shiny. The wearer recovers hit points twice as quickly as normal (also maximum hit points). The amulet prevents you from taking Bleed damage.

\index[Magic Items]{Amulet Against Possession}\smallskip* \textbf{Amulet Against Possession}
32,000 gp, very rare, the owner of this copper amulet becomes immune to possession and domination spells.

\index[Magic Items]{Amulet of Inevitable Location}\smallskip* \textbf{Amulet of Inevitable Location}
This cursed amulet has the appearance of an amulet of unobtainability. On the contrary, it makes the owner vulnerable to this type of magic. The chance of observing the wielder and the duration of spells used for this purpose double.

\index[Magical Items]{Amulet of the Planes}\smallskip* \textbf{Amulet of the Planes}
160,000 gp, legendary, While wearing this amulet, you can use two actions to name a location that you are familiar with that is on another plane of existence. Make an Intelligence check of DC 18. If you succeed, you cast the planar shift spell via the amulet. If the check fails, you and each creature and object within 5 feet of you are transported to a random destination. Roll a 1d8. From 1 to 4, reach a random destination on the floor you named. On 5 to 8, you reach a randomly determined plane of existence.

\index[Magic Items]{Amulet of Protection from Detection and Location}\smallskip* \textbf{Amulet of Protection from Detection and Location}
20,000 gp, rare, while wearing this amulet you are hidden from divination magic. You cannot be targeted by these spells or sensed by magical scrying sensors.

\index[Magical Items]{Amulet of Physical Resistance}\smallskip* \textbf{Amulet of Physical Resistance}
8000 gp, rare, not while wearing this amulet you have a +2 on Fortitude saving throws.

\index[Magic Items]{Explosion Circlet}\smallskip* \textbf{Explosion Circlet}
1500 gp, uncommon, While wearing this circlet, you can use two actions to cast the searing ray spell through it. The circlet cannot be used in this way again until the next dawn.


\index[Magical Items]{Necklace of Adaptation}\smallskip* \textbf{Necklace of Adaptation}
1500 gp, uncommon, while wearing this necklace, you can breathe normally in any environment that has air, and you have +1d6 on saving throws made against noxious gases and vapors.

\index[Magical Items]{Necklace of Strangulation}\smallskip* \textbf{Necklace of Strangulation}
This necklace looks like a jewel of great value. As soon as it is worn, it quickly tightens around the neck, dealing 6 damage per round. It cannot be removed by any means except with a wish or Remove Curse, remaining clasped around its victim's neck even after death. The necklace will only come loose when the victim has become a skeleton, ready to be picked up by an unsuspecting treasure hunter.

\index[Magic Items]{Fireball Necklace}\smallskip* \textbf{Fireball Necklace}
depending on the spheres present: 500 gp, 1000 gp, 1600 gp, 2300 gp, 3100 gp, 4000 gp, 4500 gp, 5000 gp, 5500 gp, 6000 gp, uncommon/rare/very rare: 1d6 + hang from this necklace 3 spheres. You can use two actions to detach a sphere and throw it up to 60 feet away. When it reaches the end of its trajectory, the sphere detonates as a fireball spell (DC 18).

\index[Magical Items]{Rosary Necklace}\smallskip* \textbf{Rosary Necklace}
3000 gp + variable, rare, this necklace has 1d4 + 2 magic orbs made of aquamarine, black pearl, or topaz. He also has several non-magical orbs. If a magical orb were removed from the necklace, that orb would lose its magic.

There are six types of magic spheres. The Storyteller decides the type of each sphere in the necklace. A necklace can have more than one sphere of the same type. To use it, you must wear the necklace. Each sphere contains a spell that you can cast with two actions, with the Spell DC equal to 10+2xLevel on a saving throw. Once a magic orb's spell has been cast, you cannot use that orb again until the next dawn.

\medskip

\begin{tabularx}{0.45\textwidth}{llX}
\textbf{3d6} &\textbf{Sphere of...} &\textbf{Spell}\\
\hline
3-5 &Blessing &blessing\\
6-11& Heal &Heal Serious Wounds or Lower Restoration\\
12-14 &Favour& Superior catering\\
15-16& Punish &Brain Punishment\\
17 &Wind& Walking in the wind\\
18 &Summon &Blade Barrier\\
\end{tabularx}


\index[Magic Items]{Elemental Gem}\smallskip* \textbf{Elemental Gem}
1200 gp, uncommon, this gem contains a spark of elemental energy. When you use two actions to shatter the gem, it summons an elemental as if you had cast the summon elementals spell, and the gem's magic fades. The type of gem determines the elemental summoned by the spell.

\medskip

\begin{tabular}{ll}
\textbf{Gem} &\textbf{Summoned Elemental}\\
\hline
Red Corundum & Fire Elemental\\
Yellow Diamond & Earth Elemental\\
Emerald &Water Elemental\\
Blue Sapphire&Air Elemental\\
\end{tabular}

\medskip

\index[Magic Items]{Gem of Luminosity}\smallskip* \textbf{Gem of Luminosity}
5000 gp, rare, this prism has 50 charges. While holding it, you can use two actions to speak one of three command words to cause one of the following effects:

\begin{itemize}
\item
The first command word causes the gem to produce bright light in a 30-foot radius and dim light for an additional 30 feet. The effect consumes 1 charge. Lasts until you use two actions to repeat the command word or use another gem function, or 6 hours have passed.

\item
The second command word expends 1 charge and causes the gem to project a bright beam of light at a visible creature within 60 feet of you. The creature must succeed on a DC 17 Fortitude save or be blinded for 1 minute.

\item
The third command word expends 5 charges and causes the gem to radiate blinding light in a 30-foot cone originating from you. Each creature within the cone must make a saving throw as if hit by the beam created by the second command word.

\end{itemize}

\medskip

When all of the gem's charges have been spent, the gem becomes an ordinary jewel worth 50 gp.

\index[Magic Items]{Gem of Sight}\smallskip* \textbf{Gem of Sight}
32,000 gp, very rare, with two actions, you can speak the gem's command word and spend 1 charge. For the next 10 minutes, when you look through the gem you have true vision up to 120 feet away. The gem has 3 charges, and recovers 1 spent charge every day at dawn.

\index[Magic Items]{Monster Attracting Jewel}\smallskip* \textbf{Monster Attracting Jewel}
this magical jewel is cursed, the owner attracts wandering monsters with double the probability. Monsters will also chase him twice as likely if he flees. The jewel cannot be abandoned and will immediately reappear on the owner's person whenever he tries to get rid of it. Only Remove Curse will allow the owner to leave the jewel behind.

\index[Magical Items]{Medallion of Featherfall}\smallskip* \textbf{Medallion of Featherfall}
400 gp, uncommon, this medallion automatically activates the Feather Fall spell when the bearer falls from a height of 6 feet or more.

\index[Magical Items]{Locket of Thoughts}\smallskip* \textbf{Locket of Thoughts}
3,000 gp, uncommon, while wearing this medallion, you can use two actions and expend 1 charge to cast the detect thoughts spell through it (save DC 15). The medallion has 3 charges, and recovers 1 spent charge every day at dawn.

\index[Magical Items]{Pearl of Wisdom}\smallskip* \textbf{Pearl of Wisdom}
20,000 gp, rare, this magical pearl gives an extra point of Wisdom that keeps it with you for 4 weeks. After this time the pearl must always be worn so as not to lose its benefits. There is a 5\% chance that a pearl is cursed and has the opposite effect. In this case, after 4 weeks, the negative effect is permanent and can only be canceled by desire.

\index[Magic Items]{Death Scarab}\smallskip* \textbf{Death Scarab}
this beetle brooch looks like a simple good luck charm. However, if held for 1 round or carried for 1 turn, it transforms into a hideous carnivorous insect. Equipped with powerful mandibles, the ravenous creature penetrates through leather and fabric, sinking into the flesh and reaching the heart in 1 round. After killing its victim, the creature resumes pin form. Only the heat that comes from contact with a living being can animate the monstrous insect, so placing the pin in a box or display case is a sufficient precaution to avoid any danger.

\index[Magic Items]{Protection Scarab}\smallskip* \textbf{Protection Scarab}
36,000 gp, legendary, if you hold this scarab-shaped medallion in your hand for 1 round, an inscription appears on it revealing its magical nature. While on you, it provides two benefits

- You have +2 on saving throws against spells.

- The scarab has 12 charges. If you fail a saving throw against a necromancy spell or harmful effect originating from an undead creature, you can use a reaction action to expend 1 charge and turn the failed save into a success. The scarab turns to dust and is destroyed when its last charge is spent.

\index[Magic Items]{Brooch of Defense}\smallskip* €13889{Brooch of Defense}

7500 gp, uncommon, the brooch can absorb 101 damage from Force spells, then loses its magical properties.

\index[Magic Items]{Talisman of Pure Good}\smallskip* \textbf{Talisman of Pure Good}
50,000 gp, legendary, a Devotee of Gradh or Sumkjr in possession of this item can cause a chasm of flame to appear at the feet of a Devotee of Calicante or Shayalia within 30 yards. The victim is engulfed in fire and falls screaming towards the center of the Earth. A talisman of pure good has 6 charges and cannot be recharged. If a Devotee of Calicante or Shayalia touches him he suffers 6d6 wounds. Any other Devotee or Follower is not affected. The Talisman pulses with light within 120 feet of a Devotee or Follower of Calicante or Shayalia.

\index[Magic Items]{Talisman of Extreme Evil}\smallskip* \textbf{Talisman of Extreme Evil}
50,000 gp, legendary, this talisman works exactly like the pure good talisman but with the Patrons reversed.

\index[Magic Items]{Talisman of Protection from Poison}\smallskip* \textbf{Talisman of Protection from Poison}
5000 gp, rare, poisons have no effect on you while you wear this pendant. You are immune to the poisoned condition and have immunity to poison damage.

\index[Magic Items]{Health Talisman}\smallskip* \textbf{Health Talisman}
5000 gp, rare, while wearing this pendant you are immune to the possibility of contracting any disease. If you are already infected with a disease, its effects are suspended as long as you wear this pendant.

\index[Magical Items]{Talisman of the Sphere}\smallskip* \textbf{Talisman of the Sphere}
75,000 gp, legendary, when you make an Arcana check to control a sphere of annihilation while holding this talisman you have a bonus of 5. Additionally, when you begin the round with control of a sphere of annihilation, you can use two actions to levitate it 10 feet plus a number of additional feet equal to 3 x your Intelligence rating.

\subsection{Belts, Helmets, Boots and Gloves}

\index[Magical Items]{Giant's Belt}\smallskip* \textbf{Giant's Belt}

10000/15000/20000/30000/45000 gp, rarity varies, while wearing this belt, your score reaches the score awarded by the belt. If your Strength score is already equal to or higher than the belt's score, the item has no effect on you.

There are four variations of this belt, each corresponding to a species of true giants. The Stone Giant's Girdle and the Frost Giant's Girdle appear different, but have the same effect.

\medskip

\begin{tabular}{lll}
\textbf{Giant Type}& \textbf{Strength} &\textbf{Rarity}\\
\hline
\textbf{Hill} &5& Rare\\
\textbf{Frost/Stone}& 6 &Very Rare\\
\textbf{Fire} &7& Very Rare\\
\textbf{Clouds} &8& Legendary\\
\textbf{Storms}& 9& Legendary\\
\end{tabular}

\index[Magical Items]{Dwarf Belt}\smallskip* \textbf{Dwarf Belt}
86,000 gp, rare, while wearing this belt, you gain the following benefits:

- your Constitution score increases by 1, up to a maximum of 5.

- you have +2 on Charisma checks made to interact with dwarves.

Furthermore, while you are wearing the belt you have a 50% chance every day at dawn of seeing a thick beard grow, if it can grow, or of seeing yours even thicker, if you already have it.

If you are not a dwarf, you gain the following additional benefits when wearing this belt:

- You have +2 on saving throws vs. poison and have resistance to poison damage. You have darkvision with a range of 60 feet. You can speak, read and write Dwarvish.

\index[Magic Items]{Helm of Understanding Languages}\smallskip* \textbf{Helm of Understanding Languages}
600 gp, common, while wearing this helmet, you can use two actions to cast the understand languages ​​spell through it at will.

\index[Magic Items]{Helm of Luster}\smallskip* \textbf{Helm of Luster}
75,000 gp, legendary, this luminous helm is set with 1d10 diamonds, 2d10 rubies, 3d10 fire opals, and 4d10 opals. Any gems removed from the helmet are reduced to dust. When all gems are removed or destroyed, the helmet loses its magic. While wearing it you get the following benefits:

\medskip

\begin{itemize}
\item
You can use two actions to cast one of the following spells, using one of the helmet's gems of the specified type as a component: daylight (opal), wall of fire (ruby), fireball (fire opal), or prismatic spray (diamond). ). When the spell is cast the gem is destroyed and disappears from the helmet.

\item
As long as it has at least one diamond, the helmet glows in a 30-foot radius when at least one undead is within this area. Any undead that begins its round within the area takes 1d6 Light damage.

\item
As long as the helmet has at least one ruby, you have resistance to fire damage.
\end{itemize}

\medskip

As long as the helmet has at least one fire opal, you can use two actions and speak a command word to cause a weapon you are holding to be engulfed in flames. The flames emit light in a 10-foot radius and dim light for an additional 10 feet. The flames are harmless to you and the weapon. When you hit with an attack made with the flaming weapon, the target takes an additional 1d6 fire damage. The flames last until you use two actions to speak the command word again or until you drop or sheath your weapon.

If you are wearing the helmet and take fire damage after a failed saving throw against a spell, the helmet emits a beam of light through its remaining gems. Each creature within 60 feet of the helm, other than you, must succeed on a DC 21 Reflex save or be struck by the beam, taking Light damage equal to the number of gems in the helm x 5. Then, the gems and l 'helmet are destroyed.

\index[Magic Items]{Helm of Underwater Movement}\smallskip* \textbf{Helm of Underwater Movement}
4000 gp, rare, this helmet, usually made of fish skin, grants the ability to breathe underwater, movement swim 60 feet, echolocation 60 feet. The power is usable for 6 hours a day and recharges at dawn.

\index[Magic Items]{Helm of Telepathy}\smallskip* \textbf{Helm of Telepathy}
12,000 gp, rare, while wearing this helmet, you can use two actions to cast the detect thoughts spell through it (Save DC 13). As long as you maintain your concentration on the spell, you can use two actions to send a telepathic message to the creature you are focused on. It can reply (using two actions to do so) as long as you continue to focus on it.

While you focus on a creature with detect thoughts, you can use two actions to cast the suggestion spell (Save DC 13) on that creature from the helm. Once used, the suggestion property cannot be used again until the next dawn.

\index[Magic Items]{Helm of Teleportation}\smallskip* \textbf{Helm of Teleportation}
64,000 gp, Rare, While wearing this helmet, you can use two actions and expend 1 charge to cast the teleportation spell through it. The helmet has 3 charges, and recovers 1 each morning at dawn.

\index[Magic Items]{Projectile Catching Gloves}\smallskip* \textbf{Projectile Catching Gloves}
3000 gp, uncommon, these quanta almost seem to blend into your skin when you wear them. When a ranged weapon attack hits you while you are wearing them, you can use a reaction action to reduce the damage by 1d10 + Dexterity, as long as you have a free hand. If you reduce the damage to 0 and the projectile is small enough to hold in your hand, you can grab it.

\index[Magical Items]{Gloves of Orcish Power}\smallskip* \textbf{Gloves of Orcish Power}
9000 gp, rare, while wearing these mittens your Strength is 4. The gauntlets have no effect if your Strength is already 4 or more.

\index[Magical Items]{Swimming and Climbing Gloves}\smallskip* \textbf{Swimming and Climbing Gloves}
2000 gp, uncommon, while wearing both of these gauntlets, climbing and swimming cost you no additional movement. Additionally, you have a +1d6 bonus on Constitution and Wisdom checks made while climbing or swimming.

\index[Magic Items]{Gloves of Dexterity}\smallskip* \textbf{Gloves of Dexterity}
12,000 gp, rare, these gloves give the owner a minimum Dexterity of +2 and if he already has a score of +2 this increases by 1 (up to a maximum of +4). Furthermore, the owner gains +1d6 in Fairy Hands Proficiency

\index[Magical Items]{Clumsy Gloves}\smallskip* \textbf{Clumsy Gloves}
these gloves can be made of soft leather or heavy protective material suitable for use with armor. In the first case they appear to be gloves of dexterity. In the second case they appear to be gauntlets of orcish power. With each trial the gloves appear to serve the above functions until the wearer is under attack or in a life or death situation. At that moment the curse activates. The character becomes clumsy, with a 50% chance each round of dropping an object he is holding. Gloves reduce Dexterity by 2 points. Once the curse is active, the gauntlets can only be removed with a Remove Curse spell or a wish.

\index[Magical Items]{Spider Slippers}\smallskip* \textbf{Spider Slippers}
5,000 gp, uncommon, While wearing these lightweight shoes, you can move up, down, and along vertical surfaces and upside down on the ceiling, leaving your hands free. You have a climb speed equal to your movement speed. However, slippers do not allow you to move in this way on difficult terrain, such as walls covered in ice, oil, rubble...

\index[Magic Items]{Winged Boots}\smallskip* \textbf{Winged Boots}
15,000 gp, Rare, While wearing these boots, you have a flying speed equal to your movement speed. You can use these boots to fly for up to 4 hours, all together or divided into short flights, each taking a minimum of 1 minute in duration. If the duration ends while you are flying, you descend at a rate of 30 feet per round until you land. The boots recover 2 hours of flying ability each dawn.

\index[Magical Items]{Running and Jumping Boots}\smallskip* \textbf{Running and Jumping Boots}
5,000 gp, uncommon, while wearing these boots, your movement speed becomes 30 feet unless it is faster, and your speed is not reduced if you are encumbered or wearing heavy armor. Additionally, you jump three times the normal distance, up to a maximum of 30 feet.

\index[Magical Items]{Elven Boots}\smallskip* \textbf{Elven Boots}
3,000 gp, uncommon, while wearing these boots, your footsteps make no sound, no matter what surface you are crossing. Moving silently doesn't force you to treat the terrain as difficult (but it still might be).

\index[Magical Items]{Boots of Winter}\smallskip* \textbf{Boots of Winter}
10,000 gp, rare, while wearing these boots you have resistance to cold damage, you ignore difficult terrain produced by snow or ice. You can tolerate temperatures down to -45C without needing additional protection. If you wear warm clothing, you can tolerate temperatures as low as -75C.

\index[Magical Items]{Boots of Levitation}\smallskip* \textbf{Boots of Levitation}
5000 gp, rare, while wearing these boots, you can use two actions at will to cast the levitate spell on yourself.

\index[Magical Items]{Boots of Speed}\smallskip* €13977{Boots of Speed}
5000 gp, rare, while wearing these boots, you can use a bonus Action to use only to move. You can end the effect whenever you want. The effect lasts until finished, for a maximum of 10 minutes per day. The capacity recharges at dawn.

\index[Magic Items]{Dancing Boots}\smallskip* \textbf{Dancing Boots}
these cursed boots work like other magical boots. However, when the character enters combat or attempts to flee from potential combat, he is affected by an irresistible dance spell, with no saving throw chance. It is possible to remove the Dancing Boots with the Remove Curse or Wish spell.

\subsection{Wands, Rods and Sticks}

\index[Magic Items]{Metal Searching Wand}\smallskip* \textbf{Metal Searching Wand}
500 gp, uncommon, when a charge is expended, the wand points in the direction of any metallic mass of at least 100 kg within 20 feet. The person holding the wand has an intuitive perception of the type of metal identified.

\index[Magic Items]{Wand of Arcane Bolts}\smallskip* \textbf{Wand of Arcane Bolts}
8000 gp, rare, while holding this wand, you can use two actions to expend 1 or more of its charges to cast an arcane bolt through it, like the spell of the same name. Each charge generates 1 bolt. The wand has 7 charges. The wand recovers 1d3+1 expended charges at dawn each day. However, if you expend the last charge of the wand, roll 1d6 and if you roll a 1 the wand turns to dust and is destroyed.

\index[Magic Items]{Wand of Conveniences}\smallskip* \textbf{Wand of Conveniences}
300 gp, common, The bearer can spend 1 charge to cast the invisible servant or invisible cook or floating disk spells. The wand has 7 charges which are regained at dawn.

\index[Magic Items]{Lightning Wand}\smallskip* \textbf{Lightning Wand}
32,000 gp, rare, while holding this wand, you can use two actions to expend 1 charge to cast the lightning spell through it (Save DC 18).
This wand has 7 charges. The wand regains 1d3 + 1 expended charges at dawn each day. However, if you expend the last charge of the wand, roll 1d6 and if you roll a 1 the wand turns to dust and is destroyed.

\index[Magic Items]{Fire Wand}\smallskip* \textbf{Fire Wand}
18,000 gp, very rare, a fire wand produces several spells and consumes 1 charge + manifested spell level. The spells that can be manifested are: Searing Wave, Pyroexpert, Fireball, Wall of Fire. While the wand is held in the hand, every 1 on the dice for fire damage it inflicts is treated as a 2. The wand has 7 charges and recovers 1 at dawn.

\index[Magic Items]{Ice Wand}\smallskip* \textbf{Ice Wand}
15,000 gp, very rare, a fire wand produces several spells and consumes 1 charge + manifested spell level. Manifestable spells are: ray of frost, sleet storm, ice storm, cone of cold. While the wand is held in the hand, every 1 on the dice for cold damage it inflicts is treated as a 2. The wand has 7 charges and recovers 1 at dawn.

\index[Magic Items]{Wand of Detect Magic}\smallskip* \textbf{Wand of Detect Magic}
1500 gp, uncommon, while holding this wand, with two actions you can spend 1 charge to cast the detect magic spell through it. This wand has 7 charges, and regains 1d3 spent charges each morning at dawn.

\index[Magic Items]{Wand of Detect Enemies}\smallskip* \textbf{Wand of Detect Enemies}
4000 gp, rare, while holding this wand, you can use two actions and spend 1 charge to speak its command word. For the next minute, you know the direction of the nearest hostile creature within 60 feet of you, but not the distance between you. The wand can sense the presence of hostile creatures that are ethereal, invisible, disguised, or hidden, as well as those in plain sight. The effect ends if you stop holding the wand. This wand has 7 charges. The wand regains 2 expended charges at dawn each day. However, if you expend the last charge of the wand, roll 1d6 and if you roll a 1 the wand turns to dust and is destroyed.

\index[Magic Items]{Wand of Illusions}\smallskip* \textbf{Wand of Illusions}
3000 gp, rare, the wielder of this wand can cast Greater Image (3), Silent Image (1), Mirror Image (2). Each spell costs a number of charges equal to level +1. While concentrating on the effect, the character can only move at half speed. If he is hit he must succeed on a Magic Test or the illusion vanishes immediately.

\index[Magic Items]{Wand of Detecting Secret Doors}\smallskip* \textbf{Wand of Detecting Secret Doors}
300 gp, uncommon, this wand points to the nearest secret passage within 20 feet. The effect consumes one charge of the 7 available, every day at dawn all charges are recovered.

\index[Magic Items]{Wand of Light}\smallskip* \textbf{Wand of Light}
3500 gp, rare, a wand of light manifests several spells and consumes 1 charge + level of the spell manifested. Manifestable spells are: dancing lights, light, everlasting flame, daylight. Finally, by spending 5 charges, the wielder can create a beam of intense sunlight. The intense golden-yellow light has a range of 36 m and forms a sphere of light with a diameter of 12 m. Anyone within the area of ​​effect is blinded and stunned for 1 round if they fail a DC 17 Fortitude save. The golden orb has a devastating effect on undead, inflicting 6d6 Light wounds with no save chance. This wand has 7 charges. The wand regains 2 expended charges at dawn each day. However, if you expend the last charge of the wand, roll 1d6 and if you roll a 1 the wand turns to dust and is destroyed.

\index[Magic Items]{War Wizard's Wand}\smallskip* \textbf{War Wizard's Wand}
1500/5500/25000 gp, uncommon (+1), rare (+2), or very rare (+3), while you wield this wand, you gain a bonus on spell attack rolls determined by the wand's rarity. Additionally, you ignore light cover when making a spell attack.

\index[Magic Items]{Wand of Metamorphosis}\smallskip* \textbf{Wand of Metamorphosis}
32,000 gp, very rare, while holding this wand, you can use two actions to expend 1 charge to cast the polymorph spell through it (DC 18, Will save). This wand has 3 charges. The wand regains 1 spent charge at dawn each day. However, if you expend the last charge of the wand, roll 1d6 on a 1, the wand turns to dust and is destroyed.

\index[Magic Items]{Wand of Wonders}\smallskip* \textbf{Wand of Wonders}
25,000 gp, very rare, while holding this wand, you can expend 1 charge with two actions and choose a target within 120 feet of you. The target can be a creature, an object, or a point in space. The Storyteller randomly decides or determines what will happen when you use the wand. Spells cast via the wand have a save DC of 18. If the spell normally has a range in meters, the range becomes 120 feet if it isn't already. If an effect covers an area, you must center the spell on the target and include it there. If an effect affects as many subjects as possible, the Storyteller randomly determines who is affected.

This wand has 7 charges. The wand recovers 1 charge each day at dawn. If you expend the last charge of the wand, roll 1d6 if you roll a 1 the wand turns to dust and is destroyed.

Each time you use the wand of wonders, roll a d100 and consult this table.

\end{multicols}

\begin{center}
\includegraphics[width=0.3\linewidth]{immagini/bacchette.png}

\end{center}

\medskip

\begin{tabularx}{0.95\textwidth}{lX}
\textbf{d100}& \textbf{Contents}\\
\hline
01-05 &Slow casting.\\
06-10 &Cast fairy fire.\\
11-15 &You are stunned until the start of your next round, and you think something amazing has happened.\\
16-20 &Throw gust of wind.\\
21-25 &You cast detect thoughts on your chosen target. If your target is not a creature, you take 1d6 damage instead.\\
26-30 &Cast Nauseating Mist.\\
31-33 &Heavy rain falls in a 60-foot radius centered on the target. The area becomes slightly darkened. The rain continues to fall until the start of your next round.\\
34-36 &An animal appears in the unoccupied space closest to the target. The animal is not under your control and is acting as normal. Roll a d100 to determine what species of animal appears 01-25, a rhinoceros; 26-50, an elephant; 51-100, a rat.\\
37-46 &Lightning throws.\\
47-49 &A cloud of 600 enormous butterflies fills a 30-foot radius around the target. The area becomes heavily darkened. The butterflies remain for 10 minutes.\\
50-53 &Enlarge the target as if you had cast the enlarge/reduce spell. If the target can't be affected by the spell, or if it isn't a creature, you become the target.\\
54-58 &You cast darkness.\\
59-62 &Thick grass sprouts in a 60-foot radius around the target. If grass is already there, it grows tenfold and stays that way for 1 minute.\\
63-65 &An object of the Storyteller's choice disappears on the Ethereal Plane. The object must not be worn or carried, must be within 120 feet of the target, and no larger than 10 feet in any dimension.\\
66-69 &You shrink as if you had the enlarge/shrink spell cast on you.\\
70-79 &Throw fireball.\\
80-84 &You cast invisibility on yourself.\\
85-87 &Leaves grow on the target. If you have chosen a point in space as your target, leaves will appear on the creature closest to that point. Unless plucked, the leaves will turn brown and fall off after 24 hours.\\
88-90& A stream of 1d4 x 10 gems worth 1 gp each flows from the tip of the wand in a line 30 feet long and 3 feet wide. Each gem deals 1 bludgeoning damage, and their total damage is divided equally among all creatures on the line.\\
91-95 &A barrage of twinkling, colorful lights extends from you in a 30-foot radius. You and all creatures in the area must succeed on a DC 15 Fortitude save or be blinded for 1 minute. A creature can repeat the saving throw at the end of each of its rounds, ending the effect on itself on a success.\\
96-97 &The target's skin turns deep blue for 1d10 days. If you chose a point in space, the subject will be the creature closest to that point.\\
98-00 &If the target is a creature, it must make a DC 18 Fortitude saving throw. If the target is not a creature, the target becomes you and you make the saving throw. If the saving throw fails by 5 or more, the target is petrified. If the save fails less than that, the target is restrained and begins to turn to stone. While restrained in this way, the target must repeat the saving throw at the end of each of its rounds, becoming petrified on a failure or ending the effect on a success. The target remains petrified until freed by the stone to flesh spell or similar magic.\\
\end{tabularx}

\medskip

\begin{multicols}{2}

\index[Magic Items]{Wand of Negation}\smallskip* \textbf{Wand of Negation}
35,000 gp, very rare, this wand negates spells or similar effects produced by magical items. The wielder points the wand at an object within 120 feet, and it sends out a light gray beam that strikes the target. The ray automatically negates manifestations of spells or similar effects of level 3 or lower. Each use of the wand costs 1 charge and it can only be used once per round. This wand has 3 charges. The wand recovers 1 charge each day at dawn. If you expend the last charge of the wand, roll 1d6 if you roll a 1 the wand turns to dust and is destroyed.

\index[Magic Items]{Wand of Fireballs}\smallskip* \textbf{Wand of Fireballs}
32,000 gp, rare, while holding this wand, you can use two actions to expend 1 charge to cast the fireball spell through it (Save DC 18). This wand has 7 charges. The wand regains 1 spent charge at dawn each day. However, if you expend the last charge of the wand, roll 1d6 and if you roll a 1 the wand turns to dust and is destroyed.

\index[Magic Items]{Wand of Paralysis}\smallskip* \textbf{Wand of Paralysis}
16,000 gp, rare, while holding this wand, you can use two actions to expend 1 charge to cause a thin beam to shoot from its tip at a visible creature within 60 feet of you. The target must succeed on a DC 17 Fortitude save or be paralyzed for 1 minute. At the end of each of the target's rounds, it can make a DC 15 Fortitude save, ending the effect on itself on a success. This wand has 7 charges. The wand recovers 1 spent charge at dawn each day. However, if you expend the last charge of the wand, roll 1d6 and if you roll a 1 the wand turns to dust and is destroyed.

\index[Magic Items]{Wand of Fear}\smallskip* \textbf{Wand of Fear}
13,000 gp, rare, this wand has 7 charges for the following properties. The wand recovers 1 spent charge at dawn each day. However, if you spend the last charge of the wand, roll 1 and if you roll a 1 the wand turns to dust and is destroyed.

\textbf{Command} While holding this wand, you can use two actions to expend 1 charge and command another creature to flee or crawl, as per the command spell (Saving Throw DC 18)

\textbf{Cone of Fear} While holding this wand, you can use two actions to expend 2 charges, causing the tip of the wand to emit light in a 60-foot cone. Each creature in the cone must succeed on a DC 18 Will save or be frightened of you for 1 minute. While frightened in this way, a creature must spend its rounds trying to move as far away from you as possible, and it can't voluntarily move within 30 feet of you.

It also cannot perform reactions. As its action, the creature can only use the Dash action or attempt to free itself from an effect that prevents it from moving. If it can't move anywhere, the creature can use the Total Defense action. At the end of each of its rounds, the creature can repeat the saving throw, ending the effect on itself if it succeeds. This wand has 7 charges. The wand recovers 1 spent charge at dawn each day. However, if you expend the last charge of the wand, roll 1d6 and if you roll a 1 the wand turns to dust and is destroyed.

\index[Magic Items]{Trap Detector Wand}\smallskip* \textbf{Trap Detector Wand}
400 gp, uncommon, this wand targets the nearest trap within 20 feet. The effect consumes a charge. This wand has 7 charges. The wand regains all expended charges at dawn each day.

\index[Magic Items]{Wand of Secrets}\smallskip* \textbf{Wand of Secrets}
500 gp, uncommon, while holding this wand, you can use two actions to expend 1 charge and detect if a secret door or trap is within 30 feet of you, the wand pulses and points to the one closest to you. The wand has 3 charges. The wand regains all expended charges at dawn each day.


\index[Magic Items]{Cobweb Wand}\smallskip* \textbf{Cobweb Wand}
8000 gp, uncommon, while holding it, you can use two actions to expend 1 charge to cast the web spell through it (Save DC 18). This wand has 7 charges. The wand regains 1 spent charge at dawn each day. However, if you expend the last charge of the wand, roll 1d6 and if you roll a 1 the wand turns to dust and is destroyed.

\index[Magic Items]{Wand of Bond}\smallskip* \textbf{Wand of Bond}
10,000 gp, rare, this wand has 7 charges for the following properties. The wand recovers 1 spent charge at dawn each day. However, if you expend the last charge of the wand, roll 1d6 and if you roll a 1 the wand turns to dust and is destroyed. While holding this wand, you can use two actions and expend some of its charges to cast one of the following spells (Save DC 21):

\textbf{block monsters} (5 charges) or \textbf{block people} (2 charges).

\index[Magic Items]{Wand of Assisted Escape}\smallskip* \textbf{Wand of Assisted Escape}
2000 gp, rare, while holding this wand, you can use your reaction action and spend 1 charge to gain +1d6 on saving throws you make to avoid being paralyzed or entangled, or you can spend 1 charge to gain +1d6 on any checks you make to escape a grasping attempt.

\index[Magical Items]{Staff of the Archmage}\smallskip* \textbf{Staff of the Archmage}
125,000 gp, legendary, the archmage's staff is a very powerful version of the sorcery staff. It makes various spells available to the owner. The staff can be used to manifest spells: Magic Lock, detect magic, enlarge/shrink, and light. These abilities do not require the consumption of charges. In addition, the staff has the following abilities that cost 1 charge per use: dispel magic, lightning, invisibility, wall of fire, fireball, door pass, pyroexpert, web, lockpick, and ice storm. The following powerful abilities cost 2 charges per use: summon elemental, planar shift, telekinesis. The wielder of the staff receives a +2 bonus on saving throws vs. spells. The staff can be recharged, but only by absorbing the magical energies launched at the owner, who can absorb them in quantities equal to 1 charge per spell level. This operation is the only action possible in a round, and the staff cannot be used for other effects in the same round in which it absorbs energy. Each staff has a maximum number of possible charges, and it will absorb charges only up to its limit without incurring deleterious effects. The wielder has no way of knowing this limit, or how many charges have been used, unless he uses some magical method. If the staff absorbs excess energy, it explodes as in the case of a final blow, described below. An archmage's staff can be used for a final strike, which requires it to be broken by its wielder. The breakage must not be accidental and must be declared. All charges stored in the staff are released instantly within a 30-foot radius. All creatures within 10 feet suffer wounds equal to 10 times the number of charges in the staff; between 3 m and 6 m the wounds are 6 times the number of charges; and between 6 m and 9 m the wounds are 4 times the number of charges. A DC 25 Fortitude save reduces the damage by half. The character who breaks the staff has a 50% chance of going to another plane of existence, otherwise the explosive release of magical energy destroys him. When all charges have been expended, the staff becomes a +2 staff. If the charges are exhausted it cannot be used for a final blow.

\index[Magical Items]{Staff of Withering}\smallskip* €14066{Staff of Withering}
3000 gp, rare, the staff can be wielded as a magical combat staff. If you hit, it deals damage like a normal battle staff, and you can expend 1 charge to deal an additional 2d10 Void damage to the target. Additionally, the target must succeed on a DC 18 Fortitude save or have -1d6 for 1 hour on any Expertise check or saving throw that requires Constitution. This staff has 3 charges and regains 1d3 expended charges at midnight.

\index[Magic Items]{Staff of the Woods}\smallskip* €14069{Staff of the Woods}
44,000 gp, rare, the staff can be wielded as a magical combat staff that grants a +2 bonus to attack and damage rolls made with it. While wielding it, you also have a +2 bonus on spell attack rolls.
This staff has 10 charges for the following properties. Recovers 2 spent charges every day at dawn. If you expend the staff's last charge, roll 1d6. If you roll a 1, the staff blackens, turns to ash, and is destroyed.

- \emph{Spells}. You can use two actions to expend 1 or more charges of the staff to cast one of the following spells through it, using your spell save DC: animal friendship (1 charge), locate animals and plants (1 charge), wall of thorns (6 charges), talking to animals (3 charges), leathery skin (2 charges), or awakening (5 charges). You can also use two actions to cast the pass without clues spell using your staff
spend charges.

- \emph{Tree Shape}. You can use two actions to plant one end of the staff in fertile soil and spend 1 charge to transform the staff into a vigorous fruit tree. The tree is 18 meters high, with a trunk 1 meter in diameter; at the top its branches extend for 6 meters. The tree looks like a normal tree but radiates a faint aura of transmutation magic if it is targeted by the detect magic spell. While in contact with the tree and using another action to speak its command word, you return the staff to its normal form. Any creature on the tree falls when it turns back into a stick.

\index[Magic Items]{Staff of Charm}\smallskip* \textbf{Staff of Charm}
12,000 gp, rare, while holding this staff, you can use two actions to expend 1 charge to cast charm person, command, or understand language through it, using your spell saving throw DC. The staff can be used as a magical fighting staff.

If you are holding the staff and fail a saving throw against an enchantment spell that targets only you and not an area, you can make the failed save a success. You will no longer be able to use this property of the staff until dawn the next day.

If you succeed on a saving throw against an enchantment spell that targets only you, with or without the staff's intervention, you can use a reaction action to expend 3 charges from the staff and turn the spell against the caster, as if the spell had been cast by you.

The staff has 7 charges, and regains 1 spent charge each day at dawn. If you expend the last charge, roll 1d6 if you roll a 1 the staff becomes a normal fighting staff.

\index[Magic Items]{Staff of Striking}\smallskip* \textbf{Staff of Striking}
25,000 gp, very rare, this staff can be wielded as a magical combat staff that grants a +3 bonus to attack and damage rolls made with it. When you hit with a melee attack using the staff, you can expend up to 3 of its charges. For each charge expended, the target takes an additional 1d6 force damage. The staff has 10 charges, and regains 2 spent charges each day at dawn. If you expend the last charge, roll 1d6 if you roll a 1 the staff becomes a normal fighting staff.

\index[Magic Items]{Fire Staff}\smallskip* \textbf{Fire Staff}
16,000 gp, very rare, while you wield this staff, you have resistance to fire damage.
Additionally, you can use two actions to expend 1 or more of its charges to cast one of the following spells through it: Searing Wave (1 charge, DC 13), Wall of Fire (4 charges, DC 19), or Fireball (3 charges , DC 17).

The staff has 10 charges, and regains 2 spent charges each day at dawn. If you expend the staff's last charge, roll 1d6. If you roll a 1, the staff blackens, turns to ash, and is destroyed.

\index[Magical Items]{Staff of Frost}\smallskip* \textbf{Staff of Frost}
26,000 gp, very rare, while you wield this staff, you have resistance to cold damage.
Additionally, you can use two actions to expend 1 or more of its charges to cast one of the following spells through it.

- \emph{Spells}: cone of cold (5 charges, DC 21), wall of ice (4 charges, DC 19), cloud of fog (1 charge, DC 13) or ice storm (4 charges, DC 19 ).

The staff has 10 charges, and regains 2 spent charges each day at dawn. If you spend the last charge of the staff, roll 1d6 if you roll a 1 the staff turns to water and is destroyed.

\index[Magical Items]{Staff of Healing}\smallskip* \textbf{Staff of Healing}
13,000 gp, rare, while holding it, you can use two actions to expend 1 or more of its charges to cast one of the following spells through it: cure light wounds (1 charge), restore lesser (2 charges), remove disease (3 charges ). This staff has 10 charges, and regains 1 spent charge each day at dawn. If you expend the last charge of the staff, roll 1d6 if you roll a 1 the staff vanishes in a flash of light, lost forever.

\index[Magical Items]{Staff of Swarming Insects}\smallskip* \textbf{Staff of Swarming Insects}
160,000 gp, rare, this staff has 10 charges that you can expend to use the properties described below and regains 1 charge each day at dawn. If you expend the last charge of the staff, roll 1d6 on a 1 a swarm of insects consumes and destroys the staff, and then disperses.

- \emph{Spells}. While holding this staff, you can use two actions to expend its charges and cast one of the following spells: giant insect (4 charges, DC 19) or insect plague (5 charges, DC 21).

- \emph{Insect Cloud}. While wielding this staff, you can use two actions and expend 1 charge to cause a swarm of harmless insects to spread in a 30-foot radius around you. The bugs remain for 10 minutes, making the area heavily darkened for everyone except you. The swarm moves with you, remaining centered on you. A wind of at least 15 kilometers per hour disperses the swarm and ends the effect.

\index[Magic Items]{Python Staff}\smallskip* \textbf{Python Staff}
2,000 gp, uncommon, you can use two actions to speak the staff's command word and hurl it to the ground up to 10 feet away. The staff becomes a giant constrictor snake under your control and acts on its own initiative count. Using two actions to say the command word again, you return the staff to its normal shape in the space previously occupied by the snake.

During your round you can give mental commands to the snake as long as it is within 60 feet of you and you are not incapacitated. You decide what actions the snake will take and where it will move during its next round, or you can give it a generic command, such as attacking your enemies or defending a location. If the serpent is reduced to 0 hit points, it dies and reverts to its staff form. Then, the staff shatters and is destroyed. If the serpent transforms back into staff form before losing all its Hit Points, it regains all of its lost Hit Points.

\index[Magical Items]{Staff of Power}\smallskip* \textbf{Staff of Power}
150,000 gp, legendary, this staff can be wielded as a magical combat staff that grants a +2 bonus to attack and damage rolls made with it. While wielding it, you receive a +2 bonus on defense, saving throws, and spell attack rolls. This staff has 20 charges for the following properties. Recovers 1d8 + 1 spent charges each day at dawn. If you expend the staff's last charge, roll 1d6. If you roll a 1 or less, the staff retains its +2 bonus on attack and damage rolls but loses all other properties.

- \emph{Power Strike}. When you hit with a melee attack using this staff, you can expend 1 charge to deal an additional 1d6 force damage to the target.

- \emph{Spells}. While holding this staff, you can use two actions to expend 1 or more of its charges to cast one of the following spells through it: hold monster (5 charges, DC 21), cone of cold (5 charges, DC 21), orb of invulnerability ( 6 charges, DC 22), levitation (2 charges DC 15), wall of force (5 charges, DC 21), fireball (3 charges DC 17), Arcane bolt (1 charge), ray of weakening (1 charge DC 11) or lightning (3 DC 17 charges).

- \emph{Revenge Strike}. You can use two actions to break the staff on your knee or against a solid surface, performing a revenge strike. The staff is destroyed and releases its remaining magic in an explosion that expands to fill a 30-foot radius sphere centered on it.

You have a 50\% chance to instantly travel to a random plane of existence, thus avoiding the explosion. If you fail to avoid the effect, you take force damage equal to 16 x the number of charges in the staff. Every other creature in the area must make a Reflex save DC 27. If the save fails, the creature takes an amount of damage based on the distance from the point of origin of the explosion, as shown on the table below.

On a successful save, the creature takes half as much damage.

\medskip

\begin{tabularx}{0.45\textwidth}{Xl}
\hline
\textbf{Distance from origin} &\textbf{Damage}\\
3 meters or less &8 x charges in the stick\\
Up to 6 meters & 6 x charges in the stick\\
Up to 9 meters & 4 x charges in the stick\\
\end{tabularx}

\medskip

Note: the Staff of the Archimagus and the Power are similar, this is because they were prepared by two bitter enemies who wanted to create the most powerful Staff.

\index[Magic Items]{Staff of Thunder and Lightning}\smallskip* \textbf{Staff of Thunder and Lightning}
10,000 gp, very rare, the staff can be wielded as a magical combat staff that grants a +2 bonus to attack and damage rolls made with it. It also has the following properties. Once one of these properties is used, it cannot be used again until the next dawn.

- \emph{Thunderbolt}. When you hit with a melee attack using the staff, you can cause the target to take an additional 2d6 lightning damage.

- \emph{Thunder}. When you hit with a melee attack using the staff, you can cause the staff to make the sound of thunder, audible up to 300 feet away. The affected target must succeed on a DC 21 Fortitude save or be stunned until the end of your next round.

- \emph{Lightning Strike}. You can use two actions to cause lightning to leap from the tip of the staff in a line 3 feet wide and 120 feet long. Each creature in the line must make a DC 21 Reflex saving throw, taking 9d6 lightning damage on a failed save, or half as much damage on a successful one.

- \emph{Thunderclap}. You can use two actions to cause the staff to produce a deafening roar of thunder, audible up to 600 feet away. Each creature within 60 feet of you (excluding you) must make a Fortitude save of DC 21. On a failed save, the creature takes 2d6 points of sonic damage and is deafened for 1 minute. On a successful save, it takes half damage and is not deafened.

- \emph{Thunder and Lightning}. You can use two Actions to use the Lightning Strike and Thunderclap properties together. Doing so does not consume the daily use of those properties, only the use of this one.

\index[Magical Items]{Staff of Sorcery}\smallskip* \textbf{Staff of Sorcery}
85,000 gp, very rare, in combat, this staff functions as a +1 staff. Can be used to cast summon elemental, invisibility, wallpass, and web. The staff can be used as a wand of paralysis. Each of these powers requires a charge. It is possible to break the staff to produce a "final blow", the effect of which depends on the number of remaining charges. The staff explodes in a large sphere of flame, striking all creatures within 30 feet (including the owner of the staff) and inflicting 8 wounds per remaining charge, Fortitude save DC 27 for half.

\index[Magical Items]{Rod of Enchantment}\smallskip* \textbf{Rod of Enchantment}
28,000 gp, rare, by spending 1 charge, the owner can cast dominate beasts, with 2 charges dominate people, and with 3 charges dominate monsters.

\index[Magic Items]{Rod of Absorption}\smallskip* \textbf{Rod of Absorption}
50,000 gp, very rare, while wielding this rod, you can use an action to absorb a spell that targets only you and has no area of ​​effect. The effect of the absorbed spell is canceled, and the spell's energy (not the spell itself) is absorbed by the rod. Over the course of its existence the rod can absorb and contain up to a sum of 31 levels of spells. Once the rod has absorbed 8 spells (max level 4), it can no longer absorb any more. If you are the target of a spell that the rod cannot contain, the rod has no effect on the spell. When you pick up the staff, you know how many spells the staff has absorbed so far. If you are a spellcaster and wield the rod, you can convert all the energy contained in it for 10 more Magic Points.

\index[Magic Items]{Immovable Rod}\smallskip* \textbf{Immovable Rod}
5000 gp, uncommon, this flat iron rod has a button at one end. You can use two actions to press the button, which causes the rod to magically stay in place. Until you or another creature uses two actions to press the button again, the rod will not move, even if it defies gravity. The rod can support up to 4000 kilos of weight. More weight causes the rod to deactivate and fall. A creature can use two actions to make a DC 30 Strength check, moving the rod 10 feet on a success.

\index[Magic Items]{Rod of Mighty Strike}\smallskip* \textbf{Rod of Mighty Strike}
30,000 gp, very rare, a rod of mighty strike deals 1d8+3 wounds, and functions as a +3 magical light mace. When the rod is used against golems, it consumes 1 charge per hit dealt, and inflicts 2d8+6 wounds. Note that when the rod is used as a weapon against a golem, a critical attack roll instantly annihilates it. In addition, this rod inflicts additional wounds on fiends and undead. When attacking these monsters, a critical attack roll causes 1 charge to be expended, and the rod inflicts triple wounds.

\index[Magical Items]{Rod of Sovereign Strength}\smallskip* \textbf{Rod of Sovereign Strength}
50,000 gp, legendary, this rod has a flanged head, and functions as a magical mace that grants a +3 bonus to attack and damage rolls made with it. The rod has properties associated with the six different buttons that are arranged along the handle. It also has three other properties described below.

\textbf{Six Buttons}. You can press one of the rod's six buttons with two actions. The button's effect lasts until you press a different button or press the same button again, causing the rod to return to its normal form.

- If you press the \emph{button 1}, the rod becomes a flame weapon, and a fiery blade comes out from the end opposite the flanged head.

- If you press the \emph{button 2}, the flanged head of the rod folds and two crescent-shaped blades emerge, transforming the rod into a magical battle ax that grants a +3 bonus on attack and damage rolls carried out with it.

- If you press the \emph{button 3}, the flanged head of the rod folds back, and a spearhead pops out from the end of the rod, while the handle extends up to 6 feet, turning the rod into a magical spear that grants a +3 bonus to attack and damage rolls made with it.

- If you press the \emph{button 4}, the rod transforms into a climbing pole up to 15 meters long, as requested by you. On hard surfaces like granite, one spike at the bottom and three at the top hold the rod in place. Horizontal bars 7.5 centimeters long unravel along the sides of the rod, 30 centimeters apart, to form a ladder. The rod can support 2000 kilos. Higher weight or lack of solid anchoring causes the rod to return to its normal shape.

- If you press the \emph{button 5}, the rod transforms into a battering ram and gives the user a +10 bonus on Strength checks made to break down doors, barricades or other barriers.

- If you press the \emph{button 6}, the rod assumes or remains in its normal shape and indicates magnetic north (nothing happens if this function of the rod is used in areas without magnetic north). The rod also gives you an approximate knowledge of the depth underground and your height above sea level.

\emph{Suck Life}. When you hit a creature with a melee attack using the rod, you can force the target to make a DC 21 Fortitude save. On a failed save, the target takes an additional 4d6 Void damage and is deducted from the hit point maximum, and you recover a number of Hit Points equal to half the Void damage dealt. Once used, this property cannot be used again until dawn the next day.

\textbf{Paralyze}. When you hit a creature with a melee attack using the rod, you can force the target to make a DC 21 Fortitude save. On a failed save, the target is paralyzed for 1 minute. The target can repeat the saving throw at the end of each of its rounds, ending the effect on itself on a success. Once used, this property cannot be used again until dawn the next day.

\emph{Terrify}. While you hold this rod, you can force each creature you see within 30 feet of you to make a DC 21 Will saving throw. On a failed save, the target is frightened of you for 1 minute. The frightened target can repeat the saving throw at the end of each of its rounds, ending the effect on itself if it succeeds. Once used, this property cannot be used again until dawn the next day.

This rod cannot be reloaded. When the charges run out, one remains

\index[Magical Items]{Rod of Wits}\smallskip* \textbf{Rod of Wits}
25,000 gp, very rare, this flanged-headed rod has the following properties.

\emph{Readiness}. While holding this rod, you have +2 on Wisdom checks and initiative rolls.

\emph{Spells}. While holding this rod, you can use two actions to cast one of the following spells through it: detect good and evil, detect magic, detect poison and disease, or see invisibility.

\emph{Protective Aura}. With two actions, you can drive the pointed end of the rod into the ground. At that point the head of the rod will radiate bright light in a radius of 18 meters and dim light for a further 18 meters. Within this bright light, you and any creatures friendly to you gain a +1 bonus to Defense and saving throws, and can sense the location of any hostile invisible creatures that are also within the bright light. The rod's head stops glowing and ends its effect after 10 minutes, or when a creature uses two actions to pull the rod from the ground. This property cannot be used again until dawn the next day.

\index[Magic Items]{Rod of Security}\smallskip* \textbf{Rod of Security}
90,000 gp, very rare, While holding this rod, you can use two actions to activate it. As a result, the rod transports you and up to 199 other visible consenting creatures to a paradise located in extraplanar space. You will choose the shape of this paradise. It could be a peaceful garden, a pleasant clearing, a cheerful tavern, an immense palace, a tropical island, or a fantastic fair, or anything else you can imagine. Whatever its nature, paradise contains enough food and drink to nourish its visitors. Everything that can be interacted with in extraplanar space can only exist within it.

For every hour spent in this paradise, a visitor regains Hit Points as if she had had a night's rest. Furthermore, as long as the creatures remain in paradise they do not age, although time passes normally. Visitors can stay in the paradise for a maximum of 200 days divided by the number of creatures present (round down).

When the time runs out or you use two actions to end it, all visitors respawn in the location they occupied when you activated the rod, or in the unoccupied space closest to it. The rod cannot be used again until ten days have passed.

\index[Magical Items]{Rod of Sovereignty}\smallskip* \textbf{Rod of Sovereignty}
16,000 gp, rare, you can use two actions and present the rod and demand obedience from each visible creature within 120 feet of you of your choice. Each target must succeed on a DC 17 Will save or be charmed by you for 8 hours. While charmed in this way, the creature views you as a trusted leader. If she is harmed by you or your companions, or ordered to do something contrary to her nature, the target will stop being charmed in this way. The rod cannot be used again before the next dawn.

\index[Magic Items]{Tentacle Rod}\smallskip* \textbf{Tentacle Rod}
5000 gp, rare, this rod is a magical weapon that ends in three leather tentacles. While holding the rod, you can use two actions to direct each tentacle to attack a visible creature within 10 feet of you. Each tentacle makes a melee attack roll with a +9 bonus. On a hit, the tentacle deals 1d6 bludgeoning damage. If you hit a target with all three tentacles, it must make a DC 15 Fortitude save. On a failed save, the creature's speed is halved, it has -1d6 on Reflex saving throws, and for 1 minute it cannot use his reactions. Furthermore, during each of his rounds, he can take two actions or two actions but not both. The target can repeat the saving throw at the end of each of its rounds, ending the effect on itself on a success.


\subsection{Potions - Oils}

\index[Magic Items]{Potion of Friendship with Animals}\smallskip* \textbf{Potion of Friendship with Animals}

uncommon, 200 gp, when you drink this potion, for 1 hour you can cast the Friendship with Animals spell at will (Save DC 15).

\index[Magic Items]{Climbing Potion}\smallskip* \textbf{Climbing Potion}

common, 250 gp, when you drink this potion, for 1 hour you gain climb speed equal to your movement speed. During this period you have +1d6 on Endurance checks you make to make a climb.

\smallskip* €14167{Potion of Animal Clairaudience}€14168[Magical Items]{Potion of Animal Clairaudience}

uncommon, 500 gp, this potion gives the drinker the ability to perceive sounds through the ears of an animal within a 60-foot radius. A lead barrier blocks this effect.

\smallskip* \textbf{Potion of Animal Clairvoyance}\index[Magical Items]{Potion of Animal Clairvoyance}
uncommon, 500 gp, this potion gives the drinker the ability to see through the eyes of an animal within a 60-foot radius. A lead barrier blocks this effect.

\smallskip* \textbf{Animal Control Potion}\index[Magic Items]{Animal Control Potion}
rare 1500 gp, anyone who drinks this stance is as if they had cast dominate beast


\smallskip* €14176{Potion of Dragon Control}€14177[Magic Items]{Potion of Dragon Control}
legendary, 5000 gp, this potion grants power equivalent to the dominate monster spell on a single type of dragon. You can control a dragon within 60 feet for 5d4 rounds.

\smallskip* €14179{Potion of Undead Control}€14180[Magic Items]{Potion of Undead Control}
2500 gp, rare, although undead are normally immune to this type of effect, this potion allows the drinker to influence up to 4 undead with CR 3 or less (intelligent or not) as if using the charm spell . The duration of the effect is 5d4 rounds.

\smallskip* \textbf{Potion of People Control}\index[Magic Items]{Potion of People Control}
500 gp, uncommon, once ingested, this potion grants the drinker a power analogous to the charm spell.

\smallskip* €14185{Potion of Plant Control}€14186[Magic Items]{Potion of Plant Control}
1500 gp, rare, whoever drinks this potion is able to control all plants and plant creatures (including fungi) in a 20x20 foot square area and within a distance of 90 feet. The effect lasts 5d4 rounds. Plants obey as best they can (for example, vines can become twisted and thickened, causing slowness or impediment to vision). You can command sentient plant creatures, but they are entitled to a DC 19 Will save. As with other types of enchantment, you cannot command a controlled creature to harm itself.

\index[Magic Items]{Growth Potion}\smallskip* \textbf{Growth Potion}
300 gp, uncommon, when you drink this potion, for 1d4 hours you gain the enlarge effect of the enlarge/reduce spell (requires no concentration).

\index[Magical Items]{Potion of Heroism}\smallskip* \textbf{Potion of Heroism}
200 gp, rare, when you drink this potion, you gain 10 temporary hit points that last 1 hour. For the same duration you are under the effect of the blessing spell (requires no concentration).

\index[Magic Items]{Gased Form Potion}\smallskip* \textbf{Gased Form Potion}

1500 gp, rare, when you drink this potion, for 1 hour or until you end the effect with two actions you gain the effect of the gaseous form spell (requires no concentration).

\index[Magic Items]{Potion of Giant Strength}\smallskip* €14198{Potion of Giant Strength}
varies rarity, varies cost, when you drink this potion, for 1 hour your Strength score changes. The type of giant determines the score (see table below). The potion has no effect if your Strength score is equal to or higher than the new score. The Potion of Frost Giant Strength and the Potion of Stone Giant Strength have the same effect.

- of the hills, Strength 5, Uncommon 500 gp

- stone or frost, Strength 6, Rare 1000 gp

- of fire, Strength 7, Rare 2000 gp

- of the clouds, Strength 8, Very rare 5000 gp

- of storms, Strength 9, Legendary 10,000 gp

\index[Magic Items]{Healing Potion}\smallskip* \textbf{Healing Potion}
varies rarity, varies cost, when you drink from this potion, you recover a number of Hit Points that varies depending on the rarity of the healing potion.

- Common, Hit Points 2d4 + 2, 75 gp

- Major, Hit Points 4d4 + 4, 150 gp

\index[Magic Items]{Potion of Healing}\smallskip* \textbf{Potion of Greater Healing}
varies rarity, varies cost, when you drink from this potion, you recover a number of Hit Points that varies depending on the rarity of the healing potion.

- Superior, Hit Points 8d4 + 8, 350 gp

- Ultimate, Hit Points 10d4 + 20, 1500 gp

\smallskip* \textbf{Potion of Deception}\index[Magic Items]{Potion of Deception}

500 gp, rare, this potion has a very appropriate name, as it convinces whoever drinks it that he has ingested a potion of another type. For example, a fake “clairaudience potion” could make the person who drank it hear sounds that don't actually exist. If multiple people taste this potion, there is a 90\% chance that they will agree that it is the same type.

\index[Magic Items]{Invisibility Potion}\smallskip* ​​\textbf{Invisibility Potion}
200 gp, very rare, when you drink this potion, you become invisible for 1 hour. While you are invisible, everything you carry or wear also remains invisible with you. The effect ends if you attack or cast a spell.

\smallskip* \textbf{Potion of invulnerability}\index[Magic Items]{Potion of invulnerability}
800 gp, rare, a potion of invulnerability gives the drinker a +2 bonus on saving throws and a 2-point improvement to Defense.

\index[Magic Items]{Mind Reading Potion}\smallskip* \textbf{Mind Reading Potion}
200 gp, rare, when you drink this potion, you gain the effect of the detect thoughts spell (Save DC 15).

\smallskip* \textbf{Potion of Levitation}\index[Magic Items]{Potion of Levitation}
200 gp, uncommon, this potion has the same effect as the levitation spell.

\smallskip* \textbf{Potion of Longevity}\index[Magic Items]{Potion of Longevity}
15,000 gp, legendary, this potion makes you younger by 1d12 years. Regained youth not only reverses natural aging, but also aging caused by magical effects or creatures. There is a danger in using this potion, as each time you drink a potion of longevity, there is a cumulative 1\% chance that all benefits previously gained with potions of this type are nullified. It is not possible to consume a partial dose of this potion.

\smallskip* \textbf{Potion of Metamorphosis}\index[Magic Items]{Potion of Metamorphosis}
2500 gp, rare, this potion grants a power similar to the polymorph spell.

\index[Magic Items]{Potion of Resistance}\smallskip* \textbf{Potion of Resistance}
300 gp, uncommon, when you drink this potion, you gain resistance to one damage type for 1 hour. The Storyteller chooses the type of damage or determines it randomly (Acid, Cold, Fire, Force, Lightning, Void, Poison, Light, Sound)

\index[Magical Items]{Potion of Underwater Breathing}\smallskip* \textbf{Potion of Underwater Breathing} \emph{Potion, uncommon} 200 gp

After drinking this potion, you can breathe underwater for 1 hour.

\index[Magic Items]{Shrinking Potion}\smallskip* \textbf{Shrinking Potion}
300 gp, rare, when you drink this potion, for 1d4 hours you gain the reduce effect of the enlarge/reduce spell (requires no concentration).

\index[Magic Items]{Poison Potion}\smallskip* \textbf{Poison Potion}
250 gp, uncommon, this brew looks, smells, and tastes like a potion of healing or other beneficial potion. However, it is actually a poison disguised as illusion magic. The identify spell reveals its true nature.

If you drink it, you take 3d6 points of poison damage, and must succeed on a DC 13 Fortitude save or be poisoned an additional round and take 1d6 points of damage at the start of the next round.

\index[Magic Items]{Poison Potion}\smallskip* \textbf{Greater Poison Potion}
450 gp, uncommon, this brew looks, smells, and tastes like a potion of healing or other beneficial potion. However, it is actually a poison disguised as illusion magic. If identified, the true nature is understood.

If you drink it, you take 5d6 poison damage, and must succeed on a DC 18 Fortitude save or be poisoned. At the start of each of your rounds, while you are poisoned in this way, you take 2d6 poison damage. You can repeat the saving throw at the end of each of your rounds. If the saving throw succeeds, the poison damage taken in subsequent rounds decreases by 1d6. The poison ceases its effects when the damage drops to 0d6.

\index[Magic Items]{Potion of Speed}\smallskip* \textbf{Potion of Speed}
400 gp, very rare, when you drink this potion, you gain the effect of the haste spell for 1 minute (requires no concentration).

\index[Magic Items]{Flight Potion}\smallskip* \textbf{Flight Potion}
500 gp, very rare, when you drink this potion, for 1 hour you gain flying speed equal to your normal movement speed and can float. If the potion expires while you are flying, you fall unless you have some other method of staying in the air.

\index[Magic Items]{Love Potion}\smallskip* \textbf{Love Potion}
120 gp, uncommon, you will be Charmed for 1 hour by the first creature you see within 10 minutes of drinking this philtre. If the creature is of a species or genus to which you are normally attracted, as long as you are Charmed you will consider it your one great love.

\smallskip* \textbf{Treasure Finder Filter}\index[Magic Objects]{Treasure Finder Filter}
500 gp, rare, whoever drinks this potion can perceive treasures containing precious metals or gems within 72 meters, as long as they have a value of at least 50 gold pieces. You can sense the direction of the treasure, but not its exact distance. No nonmagical barrier can prevent you from perceiving the treasures, except a sheet of lead.

\index[Magic Items]{Oil of Sharpness}\smallskip* \textbf{Oil of Sharpness}
3200 gp, very rare, this oil can coat one slashing or piercing weapon or up to 5 slashing or piercing ammunition. Applying the oil takes 1 minute. For 1 hour, the oil-covered weapon is magical and has a +3 bonus on attack rolls and damage rolls.

\index[Magical Items]{Ethereal Shape Oil}\smallskip* \textbf{Ethereal Shape Oil}
2000 gp, rare, one dose of oil is enough to cover a creature of Medium or smaller, and the equipment it wears and carries (an additional vial is required for each size category above Medium). Applying the oil takes 10 minutes. The creature then gains the effect of the ethereal shape spell for 1 hour.

\index[Magic Items]{Slippery Oil}\smallskip* \textbf{Slippery Oil}
500 gp, uncommon, the oil can cover a Medium or smaller creature, along with any equipment it wears or carries (an additional vial is required for each size category above Medium). Applying the oil takes 10 minutes. The creature then gains the benefit of the freedom of movement spell for 8 hours. Alternatively, with two actions you can pour the oil onto the ground, duplicating the effect of the anointed spell on that area for 8 hours.


\subsection{Rings}

\index[Magic Items]{Spell Accumulation Ring}\smallskip* \textbf{Spell Accumulation Ring}
24,000 gp, rare, this ring stores spells cast on it, storing them until the wearer uses them. The ring can accumulate up to 3 Spells for a total of 17 Magic Points with a maximum of 6 single Magic Points.

Any creature can cast a stacked spell of level 1 to 5 on the ring by touching it. The spell has a DC equal to 10 + 2 x Spell Level, any attack roll is made by the caster of the spell.

The cast caster must aim for the ring to have it absorbed. If the ring cannot contain the spell, the spell manifests normally. A spell cast through this ring is no longer contained within it, freeing up space for other spells.

\index[Magical Items]{Ring of Aries}\smallskip* \textbf{Ring of Aries}
5000 gp, rare, while wearing this ring, you can use two actions to expend 1 to 3 charges to attack a visible creature within 60 feet of you.

The ring produces a ghostly ram's head and makes its attack roll with a bonus of +7. On a hit, for each charge expended, the target takes 2d10 force damage and is pushed 3 feet away from you.

Alternatively, you can expend 1 to 3 ring charges with two actions to attempt to break a visible object within 60 feet of you that is not being worn or carried. The ring makes a Strength +5 check for each charge spent.

This ring has 3 charges, and regains 1d3 spent charges each morning at dawn.

\index[Magical Items]{Feather Drop Ring}\smallskip* \textbf{Feather Drop Ring}
2000 gp, rare, if you fall more than 3 feet while wearing this ring, the Feather Fall spell is activated

\index[Magical Items]{Ring of Walking on Water}\smallskip* €14275{Ring of Walking on Water}
1500 gp, uncommon, While wearing this ring, you can stand or move on any liquid surface as if it were solid ground.

\index[Magic Items]{Ring of Heat}\smallskip* \textbf{Ring of Heat}
5,000 gp, uncommon, while you wear this ring, you have resistance to cold damage. Additionally, you and everything you wear and carry are immune to the effects of temperatures as low as -45C.

\index[Magical Items]{Ring of Water Elementals}\smallskip* \textbf{Ring of Water Elementals}
250,000 gp, legendary. this ring is connected to the Elemental Plane of Water. While wearing it, you have +1d6 on attack rolls against elementals of the Elemental Plane of Water, and they have -1d6 on attack rolls made against you.

You can spend 2 ring charges to cast dominate monster on a water elemental. Additionally, you can stand and walk on liquid surfaces as if they were solid ground. You can speak and understand Aquan.

If you help kill a water elemental while wearing the ring, you gain access to the following additional properties:

\smallskip- You can breathe underwater and have speed again equal to your movement speed.

\smallskip- You can cast the following spells through the ring, spending the required number of charges: create or destroy water (1 charge), control weather (3 charges), wall of ice (3 charges) or ice storm ( 2 charges).

\medskip
The ring has 5 charges. Recovers 1d4 + 1 charges each day at dawn. Spells cast through the ring have a save DC of 21.

\index[Magical Items]{Ring of the Air Elementals}\smallskip* \textbf{Ring of the Air Elementals}
250,000 gp, legendary, this ring is connected to the Elemental Plane of Air. While wearing it, you have +1d6 on attack rolls against elementals of the Elemental Plane of Air, and they have -1d6 on attack rolls made against you.

You can spend 2 ring charges to cast dominate monster on an air elemental. Additionally, when you fall, you fall 60 feet per round and take no damage from the fall. You can talk and understand Ictun.

If you help kill an air elemental while wearing the ring, you gain access to the following additional properties:

\smallskip- You have resistance to lightning damage.

\smallskip- You have a flying speed equal to your movement speed and can hover.

\smallskip- You can cast the following spells through the ring, spending the required number of charges: Chained Lightning (3 charges), gust of wind (2 charges) or wall of wind (1 charge).

\medskip

The ring has 5 charges. Recovers 1d4 + 1 charges each day at dawn.

Spells cast through the ring have a save DC of 21.


\index[Magical Items]{Ring of the Fire Elementals}\smallskip* \textbf{Ring of the Fire Elementals}
250,000 gp, legendary, this ring is linked to the Elemental Plane of Fire. While wearing it, you have +1d6 on attack rolls against elementals of the Elemental Plane of Fire, and they have -1d6 on attack rolls made against you.

You can spend 2 ring charges to cast dominate monster on a fire elemental. Additionally, you have resistance to fire damage. You can speak and understand Ignan.

If you help kill a fire elemental while wearing the ring, you gain access to the following additional properties:

\smallskip- You have immunity to fire damage.

\smallskip- You can cast the following spells through the ring, spending the required number of charges: Searing Wave (1 charge), Wall of Fire (3 charges) or Fireball (2 charges).

\medskip

The ring has 5 charges. Recovers 1d4 + 1 charges each day at dawn.

Spells cast through the ring have a save DC of 21.

\index[Magic Items]{Ring of Earth Elementals}\smallskip* \textbf{Ring of Earth Elementals}
250,000 gp, legendary, this ring is connected to the Elemental Plane of Earth. While wearing it, you have +1d6 on attack rolls against elementals of the Elemental Plane of Earth, and they have -1d6 on attack rolls made against you.

You can spend 2 ring charges to cast dominate monster on an earth elemental. Additionally, you can move through difficult terrain composed of rubble, stone, or dirt as if it were normal terrain. You can speak and understand Tremun.

If you help slay an earth elemental while wearing the ring, you gain access to the following additional properties:

\smallskip- You have resistance to acid damage.

\smallskip- You can move through solid dirt or rock as if they were difficult terrain. If you end your round there, you are thrown out to the nearest unoccupied space you last occupied.

\smallskip- You can cast the following spells through the ring, spending the required number of charges: carve stone (2 charges), stone wall (3 charges) or stone skin (1 charge).

\medskip

The ring has 5 charges. Recovers 1d4 + 1 charges each day at dawn.

Spells cast through the ring have a save DC of 21.

\smallskip* \textbf{Ring of People Control}\index[Magic Items]{Ring of People Control}
2500 gp, rare, this ring grants the wearer the ability to use the charm spell once per day. The effect lasts until the controller ends it, 1 hour passes, or dispel magic is used.

\smallskip* \textbf{Ring of Plant Control}\index[Magic Items]{Ring of Plant Control}
5000 gp, very rare, the wearer of this ring can control plants and plant creatures in a 10x10 foot square area within a distance of 60 feet. Even if a plant is immobile, it can move while under the effect of this ring. The control lasts as long as the person exercising it maintains total concentration, which prevents any other action.

\smallskip* \textbf{Ring of Weakness}\index[Magic Items]{Ring of Weakness}
rare, once worn, this ring can only be removed by removing curse. Over the course of 6 rounds, the wearer's strength is reduced to -3.

\index[Magic Items]{Ring of Three Wishes}\smallskip* \textbf{Ring of Three Wishes}
75,000 gp, legendary, while you wear this ring, you can use two actions to expend 1 of its 1d3 charges to cast the wish spell through it. The ring loses its magic when you use the last charge.

\index[Magic Items]{Ring of Evasion}\smallskip* \textbf{Ring of Evasion}
5000 gp, rare, while wearing this ring and failing a Reflex save, you can use your reaction action to expend 1 charge to succeed on the saving throw you just failed. This ring has 3 charges, and regains 1d3 spent charges each morning at dawn.

\index[Magical Items]{Djinni Summoning Ring}\smallskip* \textbf{Djinni Summoning Ring}
35,000 gp, legendary, while wearing this ring, you can speak its command word with two actions to summon a specific djinni of the Elemental Plane of Air. The djinni appears in an unoccupied space of your choice, within 120 feet of you. It remains as long as you remain focused (as if concentrating on a spell), for up to 1 hour, or until it drops to 0 hit points. He then returns to his home plane.

As long as it remains summoned, the djinni is friendly towards you and your companions. It obeys any command you give it, no matter the language used. If you do not give him commands, the djinni will defend himself from attacks but will not take any other actions.

After the djinni leaves, it cannot be summoned again for 24 hours, and if the djinni dies the ring loses its magic.

\index[Magical Items]{Ring of Animal Influence}\smallskip* €14325{Ring of Animal Influence} \emph{Ring, rare} 4000 gp

While wearing this ring, you can use two actions to expend 1 of its charges to cast one of the following spells through it: friendship with animals (save DC 15), speak with animals, fear (save DC 15, takes target only beasts that have Intelligence -2 or less).

This ring has 3 charges, and regains 1d3 spent charges each day at dawn.

\smallskip* \textbf{Ring of Deception}\index[Magical Items]{Ring of Deception}
rare, the wearer of this cursed ring is convinced that he has a power chosen by the Narrator or determined randomly.

\index[Magic Items]{Ring of Invisibility}\smallskip* \textbf{Ring of Invisibility}
10,000 gp, very rare, while wearing this ring, you can make yourself invisible with two actions. Everything you wear or carry becomes invisible along with you. You remain invisible until the ring is removed, you attack or cast a spell, or until you use two actions to become visible again.

\index[Magical Items]{Ring of Freedom of Action}\smallskip* \textbf{Ring of Freedom of Action}
20,000 gp, Rare, While you wear this ring, difficult terrain costs you no additional movement. Additionally, magic can neither reduce your speed nor make you paralyzed or hindered.

\index[Magical Items]{Swimming Ring}\smallskip* \textbf{Swimming Ring}
3,000 gp, uncommon, while wearing this ring, you have a swim speed of 40 feet.

\index[Magic Items]{Ring of Protection}\smallskip* \textbf{Ring of Protection}
cost varies, rarity varies, while wearing this ring, you have a +1 (5000 gp, rare), +2 (7500 gp, rare), +3 (12000 gp, very rare) bonus to Defense and saving throws.

\index[Magic Items]{Spell Repelling Ring}\smallskip* \textbf{Spell Repelling Ring}
35,000 gp, legendary, while wearing this ring, you have +1d6 on saving throws against any spell that targets only you and not an area of ​​effect. Additionally, if you make a Save Critical Success and the spell is level 6 or lower, the spell has no effect on you and instead targets the caster who cast the spell

\index[Magic Items]{Ring of Regeneration}\smallskip* \textbf{Ring of Regeneration}
12,000 gp, very rare, while wearing this ring, you regain 1d6 hit points every 10 minutes, as long as you have at least 1 hit point remaining. If you lose a body part, the ring causes the missing part to regrow and return to full functionality in 1d6 + 1 days, as long as you always have at least 1 hit point left for the entire period.

\index[Magic Items]{Ring of Resistance}\smallskip* \textbf{Ring of Resistance}
6000 gp, rare, while wearing this ring, you have resistance to one type of damage. The gem set in the ring indicates the type of damage, which is chosen or determined randomly by the Storyteller.

\medskip

\begin{tabular}{lll}
\textbf{d10} & \textbf{Damage Type} & \textbf{Gem}\\

\hline
1 &Acid &Pearl\\
2& Strength &Sapphire\\
3& Cold &Tourmaline\\
4& Lightning &Citrine\\
5& Fire &Garnet\\
6& Empty& Jet\\
7& Positive Energy &Jade\\
8& Light &Topaz\\
9& Sound &Spinello\\
10& Negative Energy &Amethyst\\
\end{tabular}

\medskip

\index[Magic Items]{Jump Ring}\smallskip* \textbf{Jump Ring}
2500 gp, uncommon, while wearing this ring, with two actions you can cast the leap spell through it at will, but the target can only be you.

\index[Magical Items]{Ring of the Mind Shield}\smallskip* \textbf{Ring of the Mental Shield}
16,000 gp, uncommon, While you wear this ring, you are immune to magic that allows other creatures to read your thoughts, determine whether you are lying, learn your Traits, or learn what type of creature you are. Creatures can only communicate with you telepathically if you allow them to.

You can use two actions to make the ring invisible until another action makes it visible again, until you remove it or die. If you die while wearing this ring, your soul is captured in it, unless it already hosts another soul. You can decide to stay in the ring or reach the afterlife. As long as your soul remains in the ring, you can communicate telepathically with any creature wearing it. The wearer cannot prevent this form of telepathic communication.

\index[Magic Items]{Ring of Shooting Stars}\smallskip* \textbf{Ring of Shooting Stars}
14,000 gp, very rare, while wearing this ring in dim light or darkness, you can cast dancing lights and light at will through it. Casting either spell through the ring requires two actions. The ring has 6 charges for the following other properties.

The ring recovers 1d6 spent charges each day at dawn.

\emph{Luminescence}. Spend 1 charge with two actions to cast the glow spell through the ring.

\emph{Lightning Ball}. You can expend 2 charges with two actions to create one to four spheres of lightning 3 feet in diameter. The more orbs you create, the less powerful each orb will be individually.
Each orb appears in an unoccupied space you can see within 120 feet of you. The sphere lasts as long as you concentrate on it (as if concentrating on a spell), up to a maximum of 1 minute. Each sphere radiates dim light in a 30-foot radius. With two actions you can move each sphere up to 30 feet, but no more than 120 feet away from you. When a creature other than you is within 3 feet of an orb, the orb discharges lightning at that creature and then disappears. That creature must make a DC 18 Reflex saving throw. On a failed save, the creature takes lightning damage based on the number of orbs you created (4 orbs, 2d4 damage; 3 orbs, 2d6 damage; 2 orbs, 5d4 damage; 1 sphere, 4d12 damage).

\emph{Shooting Stars}. You can spend 1 to 3 charges with two actions. For each charge spent, you fire a spark of light from the ring at a visible point within 60 feet of you. Each creature in a 10-foot cube originating from that point is covered in sparks and must make a DC 15 Dexterity saving throw, taking 5d4 fire damage on a failed save, or half as much damage on a successful one.

\index[Magic Items]{Telekinesis Ring}\smallskip* \textbf{Telekinesis Ring}
80,000 gp, very rare, while wearing this ring, you can cast the telekinesis spell at will, but you can only target objects that are not being worn or carried.

\index[Magic Items]{Ring of X-Ray Sight}\smallskip* \textbf{Ring of X-Ray Sight}
6000 gp, Rare, While wearing this ring, you can use two actions to speak its command word. When you do this, you can see through solid matter for 1 minute. This view has a radius of 30 feet. To you, solid objects within the beam appear transparent and do not block light from passing through them.

This sight can penetrate 30 centimeters of stone, 2.5 centimeters of common metal, or up to 90 centimeters of wood or earth. Denser substances block vision, as does a thin sheet of lead. Whenever you use the ring again before finishing a night's rest you must succeed at a DC 18 Fortitude save or become fatigued.

\subsection{Hats, Cloaks, Glasses, Tunics}


\index[Magical Items]{Bandana of Intelligence}\smallskip* \textbf{Bandana of Intelligence}
8000 gp, rare, while wearing this bandana your Intelligence is +4. The band has no effect if your Intelligence is already +4 or higher.

\index[Magical Items]{Camouflage Hat}\smallskip* \textbf{Camouflage Hat}
5000 gp, uncommon, while wearing this hat, you can use two actions to cast the disguise self spell at will. The spell ends when the hat is removed.

\index[Magical Items]{Arachnid Cloak}\smallskip* \textbf{Arachnid Cloak}
8000 gp, very rare, while wearing this elegant robe of black silk woven with silver threads, you gain the following benefits:

\medskip

\begin{itemize}
\item
You have resistance to poison damage.
\item
You have a climb speed equal to your movement speed.
\item
You can move up, down and along vertical surfaces and upside down on ceilings, keeping your hands free.
\item
You can't get caught in any sort of web, and you move through webs as if they were difficult terrain.
\item
You can use two actions to cast the web spell (Save DC 15). The web created by the spell fills double its normal area. Once used, this cloak property cannot be used again until the next dawn.
\end{itemize}

\index[Magical Items]{Charlatan's Cloak}\smallskip* \textbf{Charlatan's Cloak}
8000 gp, rare, while wearing this cloak that smells faintly of sulfur, you can use it to cast the dimension door spell with two actions. This cloak's property cannot be used again until dawn. When you disappear, you leave behind a cloud of smoke, and reappear at your destination within a similar cloud of smoke. This smoke slightly obscures the space you left and the space you respawn in, and dissipates at the end of your next round. A light or stronger wind disperses the smoke.

\index[Magic Items]{Distortion Cloak}\smallskip* \textbf{Distortion Cloak}
60,000 gp, rare, while you wear this cloak, it casts an illusion that makes you appear as if you are standing somewhere near your actual location, causing all creatures to have -1d6 on attack rolls against you. If you take damage, the property ceases to function until the start of your next round. This property is suppressed while you are incapacitated, restrained, or otherwise unable to move.

\index[Magical Items]{Elven Cloak}\smallskip* \textbf{Elven Cloak}
5000 gp, uncommon, while you wear this cloak with the hood up, Awareness checks made to notice you are -1d6, and you have +1d6 on Stealth checks made to hide. Pulling the hood up or down requires two actions.

\index[Magical Items]{Manta Ray Cloak}\smallskip* \textbf{Manta Ray Cloak}
6,000 gp, uncommon, while wearing this cloak with the hood up, you can breathe underwater and have a swim speed of 60 feet. Pulling the hood up or down takes 1 action.

\index[Magical Items]{Bat Cloak}\smallskip* \textbf{Bat Cloak}
6000 gp, rare, while wearing this cloak, you have +1d6 on Dexterity checks. In areas of dim light or darkness, you can grip the edges of the Cloak with both hands and use it to move at flying speeds of 40 feet. If you were to stop holding the edges of the Cloak while flying this way, you lose your flying speed. While wearing the Cloak in an area of ​​dim light or darkness, you can use your action to cast morph on yourself, transforming into a bat. While in bat form, you maintain your Intelligence, Wisdom, and Charisma scores. The Cloak cannot be used in this way again until the next dawn.

\index[Magic Items]{Cloak of Protection}\smallskip* \textbf{Cloak of Protection}
rarity varies, cost varies, while wearing this Cloak, you gain a +1 (uncommon, 3500 gp), +2 (rare, 6000 gp), +3 (very rare, 15000 gp) bonus to Defense and saving throws.

\index[Magic Items]{Cloak of Spell Resistance}\smallskip* \textbf{Cloak of Spell Resistance}
uncommon, 3000 gp, while wearing this cloak, you have +2 on saving throws against spells.


\smallskip* \textbf{Cloak of Poisoning}\index[Magic Items]{Cloak of Poisoning}
rare, 4000 gp, this cloak is usually made of wool, although it can also be made of leather. The garment can be handled without danger, but as soon as it is worn it causes 5d6 poison damage. Each subsequent round a DC 21 Fortitude save can be made to reduce damage by 1d6 to a minimum of 1d6 damage remaining. The cloak can only be removed with a Remove Curse or wish spell.

\smallskip* \textbf{Eyes of Petrification}\index[Magical Items]{Eyes of Petrification}
these two magical crystal lenses overlap the pupils of the eyes. When a creature places these lenses, it is immediately petrified without a saving throw. About a quarter of these objects (25% chance) instead allow the person who places them to petrify with his gaze, but in this case the victims are entitled to a saving throw. Two types of magic lenses cannot be combined.

\index[Magical Items]{Charming Eyes}\smallskip* \textbf{Charming Eyes}
3000 gp, uncommon, while wearing these crystal lenses before your eyes, you can expend 1 charge with two actions to cast the charm person spell (save DC 15) on a humanoid within 30 feet of you, as long as you and you can see the target. The lenses have 3 charges and recover 1 charge of those spent each day at dawn.

\index[Magical Items]{Eagle Eyes}\smallskip* \textbf{Eagle Eyes}
4,500 gp, uncommon, while wearing these crystal lenses before your eyes, you have +1d6 on sight-based Awareness checks. In clear visibility conditions, you can make out details of even very distant creatures and objects as small as 50 centimeters.

\index[Magical Items]{Eyes of Detailed Sight}\smallskip* \textbf{Eyes of Detailed Sight}
2500 gp, uncommon, while wearing these crystal lenses before your eyes, you can see much better than normal up to a distance of 30 centimeters. You have +1d6 on sight-based Awareness checks while searching an area or studying an object at close range.

\index[Magic Items]{Night Glasses}\smallskip* \textbf{Night Glasses}
1500 gp, uncommon, while wearing these dark lenses, you have darkvision, with a range of 60 feet. If you already have darkvision, wearing these glasses increases your range by 60 feet.

\smallskip* \textbf{Tunic of Mimicry}\index[Magical Items]{Tunic of Mimicry}
1500 gp, rare, when wearing this robe, a character immediately understands its power. A camouflage tunic allows the character to blend in with the surrounding environment, whatever it may be, and to hide. He has +1d6 on Stealth checks to hide in shadows. The wielder can take on the appearance of another humanoid at will, as with the Alter Self (Change Appearance) spell. In this case, the owner's friends and those who know him very well are instinctively aware of his true identity.

\smallskip* \textbf{Archmage's Robe}\index[Magical Items]{Archmage's Robe}
8000 gp, legendary, this seemingly normal robe can be yellow (01-45 on 1d100), gray (46-75), or black (76-00). It can only be worn by a spellcaster with Magical Proficiency 2 or higher. It grants the following bonuses:

- Defense 15

- +2 on saving throws against spells and magical items

\index[Magical Items]{Tunic of Shimmering Colours}\smallskip* \textbf{Tunic of Shimmering Colours}
6000 gp, very rare, this robe has 3 charges, and regains 1d3 spent charges each day at dawn. When you wear it, you can use two actions and expend 1 charge to cause the garment to produce a shifting pattern of dazzling colors until the end of your next round. During this time, the robe casts bright light in a 30-foot radius and dim light for an additional 30 feet. Creatures that see you have -1d6 on attack rolls against you. Additionally, any creature under the bright light who sees you when the robe's power is activated must succeed on a DC 17 Will save or be stunned until the effect ends.

\smallskip* \textbf{Tunic of Weakening}\index[Magic Items]{Tunic of Weakening}
5,000 gp, rare, a robe of enfeeblement seems like a magical robe of another kind. As soon as a character puts it on, his Strength and Intelligence drop to -3 and he loses the ability to cast spells. The robe can be removed easily, but to restore the attributes requires Remove Curse followed by healing.

\index[Magical Items]{Robe of the Eyes}\smallskip* \textbf{Robe of the Eyes}
30,000 gp, rare, this robe is adorned with a design of eyes. While wearing it, you get the following benefits:

- The robe allows you to see in all directions and you have +1d6 on sight-based Awareness checks.

- You have darkvision with a range of 120 feet.

- You can see invisible creatures and objects, as well as in the Ethereal Plane, up to a range of 36 meters.

The eyes of the robe cannot be closed or averted, and while wearing this robe you are never considered to have your eyes closed or averted.

A light spell cast on the robe or a daylight spell cast within 3 feet of the robe causes you to be blinded for 1 minute. At the end of each of your rounds, you can make a Fortitude save (DC 13 for light or DC 17 for daylight), ending the blinded condition on a successful save.

\index[Magic Items]{Tunic of Useful Items}\smallskip* \textbf{Tunic of Useful Items}
300 gp, uncommon, while wearing this robe covered in patches of various shapes and colors, you can use two actions to peel off one of the patches, causing it to become the object or creature it represents. When the last patch is removed, the dressing gown becomes a regular garment. The robe has two of each of the following patches:

10-foot pole, Hemp rope (50 feet, coiled), Lens lantern (full and lit), Dagger, Sack, Steel mirror.

Additionally, the robe has 4d4 other patches. The Narrator chooses the patches or determines them at random, choosing from properties that are totally different from those already present.

He rolls a d100 on the following table to discover the properties of the other 4d4 patches of the useful items robe.

\end{multicols}

\medskip
\begin{tabularx}{0.95\textwidth}{lX}
\textbf{d100} & \textbf{Effect}\\
\hline
01-08 & Purse with 100 mo.\\
09-15& Silver chest (30 cm long, 15 cm wide and deep) worth 500 gp.\\
16-22& Iron door (maximum 10 feet wide and high, barred on the side of your choice), which you can place on any opening within range; it adapts to fit into the opening, fixing itself and creating hinges.\\
23-30 &10 gems worth 100 gp each.\\
31-44 &A wooden ladder (8 metres).\\
45-51 &A racehorse with saddlebags 52-59 Pit (a 10-foot cube), which you can place on the ground within 10 feet of you.\\
60-68 &4 healing potions. \\
69-75 &Rowing boat (4 meters long).\\
76-83& Spell scroll containing a 1st to 3rd level spell.\\
84-90& Two mastiffs.\\
91-96 &Window (60 x 120 cm, maximum depth of 60 cm), which you can place on any vertical surface within reach.\\
97-100 &Portable Aries.\\
\end{tabularx}

\begin{multicols}{2}

\medskip

\index[Magical Items]{Robe of the Stars}\smallskip* \textbf{Tunic of the Stars}
60,000 gp, rare, while wearing this robe, you gain a +1 bonus on saving throws. Six stars, located on the upper front of the robe, are larger than the others. While wearing this robe, you can use two actions to draw out one of the stars and use it to cast Arcane Bolt. Every day at sunset, the removed star reappears on the robe. While wearing the robe, you can use two actions to enter the Astral Plane along with anything you are wearing or carrying. You will stay there until you use two actions to return to the floor you were on before. You respawn on the space you last occupied, or if that space is occupied, on the nearest unoccupied space.

\subsection{Manuals, Tomes, Books}


\index[Magic Items]{Golem Manual}\smallskip* \textbf{Golem Manual}
10,000 gp, very rare, this tome contains the information and enchantments necessary to build a particular type of golem. The Storyteller chooses the type of golem that can be built or determines it randomly. To decipher and use the manual you must have at least Magical Expertise 10. A creature that cannot use the golem manual and tries to read it takes 6d6 force damage.

To create a golem, you must spend the time indicated above, working without interruption with the manual at hand and resting for no more than 8 hours a day. You also have to pay the specified cost to purchase the necessary materials.

Once you finish creating the golem, the book is consumed by arcane flames. The golem comes to life when the ashes of the manual are scattered on it. It will be under your control and understands and obeys orders spoken by you.

\medskip

\begin{tabular}{llll}
3d6 &Golem &Time &Cost\\
\hline
3-4 &Clay &30 days &65000 gp\\
5-16 &Meat &60 days& 50,000 gp\\
17 &Iron &120 days &100,000 gp\\
18 &Stone& 90 days &80,000 gp\\
\end{tabular}

\smallskip* \textbf{Manual of Good Health}\index[Magic Items]{Manual of Good Health}
15,000 gp, very rare, this tome contains instructions for strengthening the body and health. To read the book you need 24 hours in a minimum of 3 days. His instructions must be followed for 4 weeks, at the end of which the reader will permanently gain a Constitution point. Once read, the book loses its magical power and the reader will never be able to use a similar one again.

\smallskip* \textbf{Action Speed ​​Manual}\index[Magic Items]{Action Speed ​​Manual}
15,000 gp, very rare, this tome contains exercises for balance and coordination. It works like a good health manual, but gives you a point of Dexterity.

\smallskip* \textbf{Physical Exercise Manual}\index[Magic Objects]{Physical Exercise Manual}
15,000 gp, very rare, this tome works exactly like the health manual, but gives the reader a point of Strength.

\index[Magical Items]{Tome of Authority and Influence}\smallskip* \textbf{Tome of Authority and Influence}
15,000 gp, very rare, this book contains guidance on how to influence and charm others, and its words are suffused with magic. If you spend 48 hours in a period of 6 days or less studying the book's contents and practicing its directions, your Charisma score increases by 1. Then the manual loses its magic, only to regain it after a century.

\index[Magical Items]{Tome of Understanding}\smallskip* \textbf{Tome of Understanding}
15,000 gp, very rare, this book contains exercises in intuition and discernment, and its words are suffused with magic. If you spend 48 hours in a period of 6 days or less studying the book's contents and practicing its directions, your Wisdom score increases by 1, and so does your maximum score for that trait. Then the manual loses its magic, only to recover it after a century.

\index[Magic Items]{Tome of Clear Thought}\smallskip* \textbf{Tome of Clear Thought}
15,000 gp, very rare, this book contains exercises in memory and logic, and its words are suffused with magic. If you spend 48 hours in a period of 6 days or less studying the book's contents and practicing its directions, your Intelligence score increases by 1. Then the manual loses its magic, only to regain it after a century.

\subsection{Various Magical Items}

\smallskip* \textbf{Purifying water}\index[Magic Items]{Purifying water}
500 gp, rare, this sweet liquid can be used to purify water (even to desalinate sea water) and to transform poisons, acids, and other harmful liquids into a drinkable beverage. Furthermore, purifying water neutralizes the effectiveness of every other potion. This potion can transform up to 1000 cubic meters of almost any water-based liquid, but only 10 cubic meters of acid. The effects are permanent, and a purified liquid cannot be spoiled or contaminated again for a period of 5d4 rounds.

\index[Magical Items]{Wings of Flight}\smallskip* \textbf{Wings of Flight}
54,000 gp, legendary, while wearing this cloak, you can use two actions to speak its command word, transforming it into a pair of bat- or bird-like wings that sprout from your back for 1 hour or until you repeat the command word with an action. The wings provide you with a flight speed of 18 meters. When they disappear, you won't be able to use them again until dawn the next day.

\index[Magic Items]{Iron Vial}\smallskip* \textbf{Iron Vial}
35,000 gp, legendary, this iron bottle has a brass cap. You can use two actions to speak the flask's command word, targeting a visible creature within 60 feet of you. If the target is native to a plane of existence other than the one you are on, it must succeed on a DC 21 Will save or be trapped in the flask. If the target has already been trapped in the flask, it receives +1d6 on its saving throw. Once trapped, the creature will remain in the flask until freed. The flask can only contain one creature at a time. A creature trapped in the flask does not need to breathe, eat, or sleep, and does not age. You can use two actions to remove the vial's cap and free the creature inside. The creature will be friendly towards you and your companions for 1 hour and will obey your commands for that duration. If you do not give her commands or give her one that would result in her death, she will defend herself but take no further actions. At the end of the duration, the creature will act according to its normal behavior

The identify spell reveals that a creature is inside the flask, but the only way to determine what sort of creature it is is to open the flask. A newly discovered iron flask may already contain a creature chosen by the Storyteller or determined randomly.

\medskip

\begin{tabular}{ll}
\hline
d100 &Contains\\
1-50 &Empty\\
51-66 &Demon \\
67 &Angelo Deva\\
68-69 &Devil (upper)\\
70-73 &Devil (lower)\\
74-75 &Genie Djinni\\
76-77 &Genius Efreeti\\
78-83 &Elemental (any)\\
84-86 &Invisible Persecutor\\
87-90 &Night hag\\
91 &Angelo Planetar\\
92-95 &Salamander\\
96 &Angelo Solar\\
97-99 &Succubus/Nightmare\\
100 &Xorn\\
\end{tabular}
\medskip


\smallskip* \textbf{Elemental Water Amphora}\index[Magical Items]{Elemental Water Amphora}
2500 gp, rare, this amphora can be used to summon and control a water elemental similarly to the summon elemental spell. You must prepare the magic item and conduct a ritual for one round before the actual summoning, which takes one round. After the elemental has been summoned, you must maintain concentration to be able to give it orders. The amphora can be used once a day.

\index[Magic Items]{Crab Apparatus}\smallskip* \textbf{Crab Apparatus}
15,000 gp, legendary, this object appears as a sealed iron barrel of Large size and weighing 500 pounds. The barrel hides a latch, which can be found by succeeding on a DC 25 Intelligence check. Removing the latch opens a compartment at one end of the apparatus, allowing two creatures of Medium or smaller size to fit inside. At the opposite end there are ten levers, each in a neutral position, capable of moving up or down. When certain levers are employed, the apparatus transforms and resembles a giant lobster.

The apparatus is a Large object with the following statistics.

Defense: 20, Hit Points: 200, Speed: 30 ft., swim 30 ft. (or 0 ft. both if legs and tail are not extended)

Damage Immunity: Poison

To be used as a vehicle, the device requires a pilot. When the appliance door is closed, the compartment is watertight, and does not allow air or water to seep. The compartments hold enough air for 10 hours, divided by the number of creatures inside. The apparatus floats in water and can even go underwater to a depth of 270 meters. Below this threshold, the apparatus takes 2d6 bludgeoning damage per minute from the pressure. A creature inside the compartment can use two actions to move up or down up to two levers. After each use, the lever returns to its neutral position. Each lever, from left to right, functions as shown in the table below.

1: Extends legs and tail, allowing the apparatus to walk and swim. Retracts legs and tail, reducing the apparatus's speed to 0 and making it unable to benefit from speed bonuses.

2: Opens the front door. Closes the front door.

3: Opens the side portholes (two on each side). Closes the side portholes (two on each side).

4: Extends two claws from the front side of the apparatus. Retracts the claws.

5: Make a melee weapon attack with each claw extended: +8 to attack roll, reach 3 ft., one target. Hit: 7 (2d6) bludgeoning damage. Make a melee weapon attack with each claw extended: +8 to attack roll, reach 3 ft., one target. Hit: The target is grappled (DC 18 to escape).

6: The apparatus walks or swims forward. The apparatus walks or swims backwards.

7: The apparatus turns 90 degrees to the left. The apparatus turns 90 degrees to the right.

8: Slits in the front emit bright light in a 30-foot radius and dim light for additional meters. Turns off the lights.

9: The apparatus sinks 6 meters into liquids. The apparatus rises 6 meters from the liquids.

10: Unlocks and opens the tailgate. Closes and seals the tailgate.


\smallskip* \textbf{Vial of Curses}\index[Magic Items]{Vial of Curses}
800 gp, rare, this object has the appearance of a flask, bottle, decanter, container, flask, or jug. May contain liquid or give off smoke. When the flask is first uncorked, all creatures within 30 feet are cursed.

\index[Magic Items]{Folding Boat}\smallskip* \textbf{Folding Boat}
12,000 gp, rare, this object appears to be a wooden box measuring 30 centimeters long, 15 centimeters wide, and 15 centimeters deep. It weighs 2 kilos and floats. It can be opened to place objects inside. This item has three command words, each of which requires two actions to utter. A command word causes the box to unfold into a boat 3 meters long, 1.5 meters wide and 50 centimeters deep. The boat has a pair of oars, an anchor, a mast and a sail. The boat can hold up to four Medium-sized creatures.

The second command word causes the box to unfold into a ship 7.2 meters long, 2.5 meters wide, and 2 meters deep. The ship has a deck, rowing rows, five sets of oars, a rudder, an anchor, a cabin and a square-rigged mast. The ship can hold fifteen Medium-sized creatures.

The third command word causes the folding boat to fold back into the box, as long as no creatures are on board. Any items on board that cannot fit into the box remain outside the box while it folds. Any object on board that can fit into the box, fits in.

\smallskip* \textbf{Drowning Basin}: This cursed basin has the appearance of a water elemental amphora. However, instead of summoning an elemental, it releases a globe of water that envelops the character's head. He drowns in 2d4 rounds unless he succeeds on a saving throw vs. spells. Water is “sticky” and can only be removed with magic (dispel magic or destroy water).

\index[Magic Items]{Battle of the Opening}\smallskip* \textbf{Battle of the Opening}
1500 gp, rare, this hollow metal tube measures approximately 30 centimeters in length and weighs 0.5 kilos. You can beat it with two actions, pointing it at an object within 120 feet that can be opened, such as a door or lock. The clapper makes a clear sound, and a lock or latch on the item opens unless the sound is prevented from reaching the item. If there are no locks or laces left to open, the item opens itself.

The clapper can be used ten times. After tithing, it cracks and becomes unusable.

\smallskip* \textbf{Crossbow of Arcane Bolts}\index[Magic Items]{Crossbow of Arcane Bolts} This small one-handed crossbow has the ability to manifest a magical bolt.
By spending 2 Actions it is possible to fire a magic bolt as if it were a single Arcane Bolt.

\smallskip* \textbf{Battle of Cannibalism}\index[Magic Items]{Battle of Cannibalism}
This item looks like an Opening Clapper. It functions as such for the first round of use (and has 1d4x10 charges for this purpose). However, on the second jingle all creatures within 60 feet must succeed on a DC 21 Will save or fall prey to ravenous hunger, attacking the nearest humanoid to kill and devour it. Every other round a new saving throw is allowed. If no humanoids are present, the affected creatures will attack other creatures present.

\index[Magic Items]{Bag of Preservation}\smallskip* \textbf{Bag of Preservation}

There are different types of storage bags and they all have in common the ability to contain much more than they should given their size.

Preservative Bags are divided into 4 types (Type I, II, III; IV) depending on their conservation capacity.

If the bag is overloaded, punctured, or torn, the bag breaks and is destroyed and its contents scattered throughout the Astral Plane. If the bag is turned inside out, its contents are ejected, unharmed, but the bag must be put right side up before it can be used again. Breathing creatures placed in the bag can survive there for a number of minutes equal to 10 divided by the number of creatures (minimum 1 minute), after which they will begin to suffocate.

Placing a bag of holding within the extradimensional space generated by a utility backpack, portable hole, or similar object destroys both objects and opens a portal to the Astral Plane. The portal originates at the point where one object has been placed inside the other. Any creature within 10 feet of the portal is sucked into it and reappears in a random place on the Astral Plane, then the portal closes again. The portal is one-way and cannot be reopened.


Some spellcasters prefer to create Chests of Holding, which function in the same way as bags of holding.

\index[Magic Items]{Type I Preservative Bag}\smallskip* \textbf{Type I Preservative Bag}
500 gp, uncommon, this is the smallest model of the holding bags. It appears to be a small bag 20cm in diameter with a mouth about as wide.
It is not possible to bring in objects that are more than 20cm wide and more than 50cm long.
The maximum capacity is 20 kg/Dimensions 7.

\index[Magic Items]{Type II Preservative Bag}\smallskip* \textbf{Type II Preservative Bag}
1000 gp, uncommon, this is the average model of holding bags. Apparently it is a bag 40 cm in diameter with a mouth about as wide.
It is not possible to bring in objects that are more than 40cm wide and more than 100cm long.
The maximum capacity is 100 kg/Dimensions 25.

\index[Magic Items]{Type III Preservative Bag}\smallskip* \textbf{Type III Preservative Bag}
1500 gp, rare, apparently a bag 80 cm in diameter with a mouth about as wide.
It is not possible to bring in objects that are more than 80cm wide and more than 150cm long. The maximum capacity is 200 kg/Dimensions 50.

\index[Magic Items]{Type IV Preservative Bag}\smallskip* \textbf{Type IV Preservative Bag}
5000 gp, very rare, apparently a large bag 120 cm in diameter with a mouth approximately as wide.
It is not possible to bring in objects that are more than 120cm wide and more than 200cm long. The maximum capacity is 300 kg/Dimensions 75.

\index[Magical Items]{Devouring Bag}\smallskip* \textbf{Devouring Bag}
2000 gp, rare, bag appears as a bag of holding. If the bag is turned inside out its properties stop working. An extradimensional creature attached to the bag can sense anything placed inside. The animal or vegetable matter placed entirely inside the bag is devoured and is lost forever. When a part of a living creature is placed in the bag, there is a 50\% chance that the creature will be pulled into the bag. A creature inside the bag can use two actions to attempt to escape with a successful DC 18 Strength check.

Another creature can use two actions to grab the creature inside the bag and pull it out, succeeding on a DC 20 Strength check (and as long as it isn't dragged into the bag itself). Any creature that begins its round inside the bag is devoured, its body destroyed.

Inanimate objects, up to 27 dm3 of material, can be placed inside the bag. However, once a day, the bag swallows any object placed inside it and spits it back out into another plane of existence. The Storyteller determines the time and the plan. If the bag were torn to pieces or torn, it is destroyed, and whatever it contains would be transported to a random location in the Astral Plane.

\index[Magical Items]{Efreeti Bottle}\smallskip* \textbf{Efreeti Bottle}
15000 gp, very rare, this painted brass bottle weighs 500 grams. When you use two actions to remove the cap, a cloud of thick smoke billows from the bottle. At the end of your round, the smoke dissipates in a harmless burst of fire, and an efreeti appears in an unoccupied space within 30 feet of you. The first time the bottle is opened, the Storyteller randomly determines what happens.

\medskip

\begin{tabularx}{0.45\textwidth}{lX}
\textbf{3d6} &\textbf{Effect}\\
\hline
3-5 & The efreeti attacks you. After fighting for 5 rounds, the efreeti disappears and the bottle loses its magic.\\
6-16 &The efreeti obeys you for 1 hour, acting on your commands. Then he goes back into the bottle, and a new cap can contain him. The cap cannot be removed before 24 hours have passed. The next two times the bottle is opened, the same effect occurs again. If the bottle is opened a fourth time, the efreeti runs away and disappears, and the bottle loses its magic.\\
17-18 & The efreeti can cast the wish spell on your behalf three times. It disappears when it grants the final wish or after 1 hour, when the bottle loses its magic.
\end{tabularx}


\index[Magic Items]{Bean Bag}\smallskip* \textbf{Bean Bag}
5000 gp, rare, inside this bag are 3d4 dried beans. The bag weighs 250 grams plus 125 grams for each bean it contains.

If you dump the contents of the bag onto the ground, the beans explode in a 10-foot radius. Each creature in the area, including you, must make a DC 18 Reflex saving throw, taking 5d4 fire damage on a failed save, or half as much damage on a successful one.

The fire ignites flammable objects in the area that are not being worn or carried. If you remove the bean from the bag, plant it in soil or sand, and water it, the bean will produce an effect 1 minute later, starting from the spot in the soil where it was planted. The Storyteller chooses the effect or determines it randomly.

\end{multicols}

\begin{center}
\includegraphics[width=0.35\linewidth]{immagini/borsetta.png}

\emph{Preservation bag, classic model, Type II}
\end{center}


\medskip

\begin{tabularx}{0.95\textwidth}{lX}
\textbf{d100} & \textbf{Effect}\\
\hline
01 &5d4 mushrooms appear. If a creature eats a mushroom, roll a dice. If the roll is odd, he must succeed on a DC 15 Fortitude save or take 5d6 poison damage and remain poisoned for 1 hour. If the result is a tie, he gains 5d6 temporary hit points for 1 hour.\\
02-10 &Erupts a geyser that spews water, beer, juice, tea, vinegar, wine, or oil (at the Storyteller's discretion) 30 feet into the air for 1d12 rounds.\\
11-20 &A tree man appears. There is a 50\% chance that the treeman is chaotic evil and will attack you.\\
21-30 &An animated stone statue bearing your likeness rises from the ground. She will start to threaten you verbally. If you were to leave and other people came to the location, the statue would describe you as the most dangerous of criminals, and urge them to seek you out and attack you. If you are on the same plane of existence as the statue, it will always know where you are. After 24 hours the statue will become inanimate.\\
31-40 &A campfire producing blue flames rises from the ground and burns for 24 hours (or until extinguished).\\
41-50 &They spit 1d6 + 6 howler mushrooms.\\
51-60 &1d4 + 8 fuchsia toads appear. Whenever a toad is touched, it transforms into a Large or smaller monster of the Storyteller's choice. The monster remains for 1 minute and then disappears in a puff of fuchsia smoke. 61-70 A bulette comes out of the ground and attacks.\\
71-80 &A fruit tree grows. It has 1d10 + 20 fruits. 1d8 of these function as a randomly determined magic potion, while one of them functions as an ingested poison of the type determined by the Storyteller. The tree vanishes after 1 hour. The collected fruits, however, remain and maintain their magic for 30 days. \\
81-90 &A nest appears with 1d4 + 3 eggs. Any creature that eats an egg must make a Fortitude save DC 28. If the save succeeds, the
creature permanently increases its lowest ability score by 1, choosing randomly in the event of a tie. If the saving throw fails, the creature takes 10d6 force damage from a magical blast within it.\\
91-99 &A pyramid with a square base of 18 meters emerges from the ground. Inside is a sarcophagus containing a sovereign mummy. The pyramid is regarded as the lair of the mummy ruler, and its sarcophagus contains treasure of the Storyteller's choice.\\
100 &An enormous beanstalk grows on site, to a height of the Narrator's choosing. The peak leads wherever the Storyteller wants, be it a cloud giant's castle or another plane of existence.
\end{tabularx}

\begin{multicols}{2}

\medskip

\index[Magic Items]{Smoking Bottle}\smallskip* \textbf{Smoking Bottle}
1200 gp, uncommon, smoke continually escapes from the mouth of this brass bottle, held in place by its lead stopper. The bottle weighs 500 grams. When you use two actions to remove the cork, a cloud of thick smoke spreads in a 60-foot radius around the bottle. The cloud area is heavily darkened. For each minute the bottle remains open and within the cloud, the radius increases by 10 feet until it reaches its maximum radius of 120 feet.

The cloud persists as long as the bottle remains open. Closing the bottle requires you to speak its command word with two actions. Once the bottle is closed, the cloud disperses after 10 minutes. A moderate wind (15 to 30 km/h) can disperse the smoke in 1 minute, and a strong wind (more than 30 km/h) can disperse it in 1 round.

\smallskip* \textbf{Bag of Cancellation}\index[Magic Items]{Bag of Cancellation}
9000 gp, rare, this magical bag functions as a bag of holding for 1d6 days. After this period, all the material inside it or new material added is subject to a transformation depending on its nature. Precious stones become useless stones, and precious metals are transformed into less valuable metals such as lead. Magical items lose their power without a saving throw, and transform into mundane items of their type. Only extremely powerful magical items are possibly immune to this effect.

\index[Magical Items]{Brazier of Fire Elementals}\smallskip* €14561{Brazier of Command of Fire Elementals}
8000 gp, rare, While the fire burns within this brass brazier, you can use two actions to speak the brazier's command word and summon a fire elemental, as if you had cast the summon elementals spell. The brazier cannot be used in this way again until the next dawn.

The brazier weighs 2.5 kilos.

\smallskip* \textbf{Cursed Brazier of Sleep}\index[Magical Items]{Cursed Brazier of Sleep}
This brazier looks like, and functions like, a fire element command brazier. However, when activated, smoke thickens in a 10-foot radius around the brazier, putting anyone in the area into a cursed sleep unless they succeed on a DC 21 Will save. A fire elemental appears. normally, but is hostile and attacks all creatures present. Creatures subject to cursed sleep sleep indefinitely until killed, unless Remove Curse is used.

\index[Magical Items]{Jug of Infinite Water}\smallskip* \textbf{Jug of Infinite Water} 12,000 gp 12,000 gp, uncommon, this stoppered flask emits a liquid sound when moved, as if contained water. The jug weighs 1 kilo. You can use two actions to remove the cap and speak one of three command words, at which point an amount of fresh water or salt water (your choice) will pour out of the flask, until the start of your next round. Choose one of the following options:

\medskip

\begin{itemize}
\item
\emph{Stream} produces 4 liters of water.
\item
\emph{Fountain} produces 20 liters of water.
\item
\emph{Geyser} produces 150 liters of water which are projected from a geyser 9 meters long and 30 centimeters wide. With two actions, while holding the jug, you can target a visible creature within 30 feet of you as the geyser's target.

The target must succeed on a DC 15 Fortitude save or take 1d4 bludgeoning damage and be knocked prone. Instead of a creature, you can target an object that is not worn or carried and weighs no more than 100 pounds. The object is knocked over or pushed 10 feet away from you.
\end{itemize}

\smallskip* \textbf{Potions Jug}\index[Magic Items]{Potions Jug}
18,000 gp, legendary, this blue ceramic jug has a solid gold stopper. The jug contains 1d4+1 magical potions, each of which can be poured every 2 days. Specific potions are determined at random, remain the same over time, and must always be poured in the same order. Not all of them are necessarily beneficial.

\index[Magic Items]{Portable Hole}\smallskip* \textbf{Portable Hole}
10,000 gp, rare, this elegant black fabric, soft as silk, folds to the size of a handkerchief. It unfolds in a circular layer 1 meter in diameter. You can use 1 round to deploy a portable hole and place it on or against a solid surface, in which the portable hole creates a hole 10 feet deep. Any creature small enough can use the Portable Hole to pass through the wall or surface it is resting on as long as it is less than 10 feet deep.

You can use 1 round to close a Portable Hole by taking the edges of the fabric and folding it over. Folding the fabric closes the hole, and any creature or object inside is ejected with a 50% chance of exiting one way or the other.

Placing a portable hole within the extradimensional space created by a bag of storage, laptop compartment, utility backpack, or similar object instantly destroys both objects and opens a portal to the Astral Plane. The portal originates from the point where one object was placed inside the other. Any creature within 10 feet of the portal is sucked into it and deposited in a random location in the Astral Plane. Then the portal closes. The portal is one-way and cannot be reopened.

\index[Magic Items]{Invocation Candle}\smallskip* \textbf{Invocation Candle}
8000 gp, very rare, this long, thin candle is dedicated to a Patron and shares his Traits. The Candle Traits can be identified through a 1 hour ritual of following the candle.

The Storyteller chooses the Patron and the Traits associated with it or determines it randomly.

Candle magic is activated when the candle is lit with two actions. After burning for 4 hours, the candle is destroyed. You can decide to turn it off early to use it again later. Deduct the time the candle has left before it burns out in 1-minute increments to determine how long the candle burned.

When lit, the candle radiates dim light in a 30-foot radius. Any creature within the candle's Devoted or Follower light makes attack rolls, saving throws, and ability checks with +1d6.

Alternatively, when you light the candle for the first time, you can cast the portal spell. Doing so destroys the candle.

\index[Magic Items]{Dimensional Strains}\smallskip* \textbf{Dimensional Strains}
4000 gp, rare, you can use 2 Actions to place these manacles on an incapacitated creature. The handcuffs fit any creature from Small to Large. In addition to serving as common shackles, shackles prevent a creature bound by them from using any method of extradimensional movement, including teleportation or travel to different planes of existence. However, they do not prevent a creature from passing through an interdimensional portal.

You and any creature you point to when using the shackles can use two actions to remove them. Once every 30 days, the bound creature can make a DC 40 Strength check. On a success, the creature breaks free and destroys the shackles.

\index[Magic Items]{Supreme Glue}\smallskip* \textbf{Supreme Glue}
400 gp, uncommon, this milky white, viscous substance can form a permanent adhesive bond between any two objects. It must be contained in a jar or ampoule that has been coated inside with slippery oil. When found, its container holds 1d6 + 1 per 30 grams. 30 grams of glue can cover a square surface of 30 centimeters on each side. The glue takes 1 minute to set. Once the glue is set, the bond created can only be broken by the universal solvent or oil of the ethereal form, or by the wish spell.

\smallskip* \textbf{Healthy Air Necklace}\index[Magic Items]{Healthy Air Necklace}
2500 gp, uncommon, this necklace is a chain with a platinum locket. The necklace's magic surrounds the wearer with a bubble of pure air, making them immune to the effects of vapors and gases. The bubble allows you to survive in an airless environment for a week.

\index[Magic Items]{Climbing Rope}\smallskip* \textbf{Climbing Rope}
2000 gp, uncommon, this silk rope is 18 meters long, weighs 1.5 kilos and can support up to 1,500 kilos. If you hold one end of the rope and use two actions to say the command word, the rope comes to life. With two actions you can command the other end to move to a destination of your choice. That end moves 10 feet during your round when it receives your first command, and 10 feet during each subsequent round until it reaches its destination, up to its maximum length, or until you tell it to stop. You can also tell the rope to tighten or release from an object, knot or untwist, or rewind for carrying. If you tell the rope to tie a knot, large knots will appear at 30cm intervals along the rope. While knotted, the rope decreases to a length of 50 feet and grants +1d6 on checks made to climb it.

The rope has Defense 20, Hardness 3 and 20 Hit Points. He regains 1 hit point every 5 minutes as long as he has at least 1 hit point. If the rope drops to 0 hit points, it is destroyed.

\index[Magical Items]{Rope of Entanglement}\smallskip* \textbf{Rope of Entanglement}
4000 gp, rare, this rope is 9 meters long and weighs 1.5 kilos. If you hold one end of the rope and use two actions to speak its command word, the other end will spring forward to entangle a visible creature within 20 feet of you. The target must succeed on a DC 18 Reflex save or become entangled. You can release the creature by using two actions to speak a second command word. A target entangled by the rope can use two actions to make a DC 18 Strength or Escape Artist check (target's choice). On a success, the creature is no longer entangled by the rope.

The rope has Defense 20 and 20 Hit Points. He regains 1 hit point every 5 minutes as long as he has at least 1 hit point. if the rope drops to 0 hit points, it is destroyed.

\smallskip* \textbf{Choke Rope}\index[Magic Items]{Choke Rope}
rare, this magical rope, although normal in appearance, can come alive and attack those who try to use it, tightening around the neck and trying to strangle its victim. The choke cord is long enough to strangle up to 1d4 creatures within a 10-foot radius, dealing 2d6 wounds per round to each. You must succeed at a DC 19 Reflex save to avoid being caught. The rope has Defense 22 and 25 Hit Points, but only those who are not strangled can attack it. Victims cannot free themselves in any way, nor can they cast spells.

\index[Magic Items]{Horn of Destruction}\smallskip* \textbf{Horn of Destruction}
750 gp, rare, you can use two actions to speak the horn's command word and then blow it, sending out a thunderous blast in a 30-foot cone and audible up to 600 feet away. Each creature within the cone must make a DC 18 Fortitude save. On a failed save, the creature takes 5d6 sonic damage and is deafened for 1 minute. On a successful save, the creature takes half damage and is not deafened. Creatures and objects made of glass or crystal have -1d6 on their saving throws and take 10d6 sonic damage instead of 5d6.

Each use of the horn's magic has a 20\% chance to cause it to explode. The blast deals 10d6 fire damage to the blower and destroys the horn.

\index[Magic Items]{Horn of Valhalla}\smallskip* \textbf{Horn of Valhalla}
6000 gp, rare, you can use two Actions to blow this horn. In response, the warrior spirits of Asgard appear within 60 feet of you. These spirits use berserker statistics. They return to Asgard after 1 hour or when they drop to 0 hit points. Once used, the horn cannot be used again until 7 days have passed.


\index[Magic Items]{Cube of Force}\smallskip* \textbf{Cube of Force}
16,000 gp, rare, this cube has 2.5 centimeters of edge. Each face has a unique mark that can be pressed. The cube starts with 36 charges, and regains 3d6 expended charges each day at dawn. You can use two Actions to press one of the cube's faces, spending a number of charges based on the cube's face.

Each face has a different effect. If there are no more charges left in the cube, nothing happens. Otherwise, an invisible barrier of force rises, forming a 10-foot cube. The barrier is centered on you, moves with you, and lasts for 1 minute, until you use two actions to press the sixth face of the cube, or the cube runs out of charges. You can change the effect of the barrier by pressing a different face of the cube and spending the required number of charges, resetting its duration.

If your movement causes the barrier to come into contact with a solid object that cannot pass through the cube, you will not be able to approach the object as long as the barrier remains.

\medskip

The cube loses charges when the barrier is targeted by certain spells or comes into contact with certain spells or magic item effects, as indicated on the following table.

\medskip

\begin{tabular}{ll}
\textbf{Spell or Item} &\textbf{Lost Charges}\\
\hline
Arcane Bolt (5 hits) &1\\
Disintegration &1d12\\
Wall of Fire &1d4\\
Pass Door & 1d6\\
Prismatic Spray &3d6\\
\end{tabular}

\medskip

\begin{tabularx}{0.45\textwidth}{llX}
\textbf{Face} & \textbf{Charges}& \textbf{Effect}\\
\hline
1& 1& Gas, wind and fog cannot penetrate the barrier\\
2& 2 &Nonliving matter cannot cross the barrier. Walls, floors and ceilings can pass through it at your discretion.\\
3 &3 &Living matter cannot cross the barrier.\\
4 &4 &The spell's effects cannot pass through the barrier.\\
5 &5 &Nothing can cross the barrier. Walls, floors and ceilings can pass through it at your discretion.\\
6 &0& The barrier is deactivated. \\
\end{tabularx}


\smallskip* \textbf{Cold protection cube}\index[Magic Objects]{Cold protection cube}
2500 gp, rare, this cubic pendant is activated and deactivated by pressing one face (immediate action). When activated, it emanates a cubic protective field with an edge of 3 m (similar to that of a force cube but with a different effect). The temperature inside the protective field is maintained at 21 C. The field absorbs all cold attacks, negating them completely. If it negates more than 50 cold damage in a round (whether from a single attack or multiple attacks), however, the magical field collapses and cannot be reactivated for an hour. If the field negates more than 100 cold wounds in a round, the cube is destroyed.

\index[Magical Items]{Iron Binding Bands}\smallskip* \textbf{Iron Binding Bands}
5000 gp, rare, this rusty iron sphere measures 7.5 centimeters in diameter and weighs 500 grams. You can use two actions to speak a command word and hurl the orb at a visible Huge or smaller creature within 60 feet of you. The sphere moves in the air, opening into a network of metal bands. Make a ranged attack roll, if you hit, the target is restrained until you take two actions to speak a command word and free it. Doing so, or failing to attack, causes the bands to contract and return to a sphere.

A creature, including the entangled one, can use two actions to make a DC 25 Strength check to break the iron bands. On a success, the object is destroyed, and the entangled creature is free. If the check fails, any further attempts the creature makes automatically fail until 24 hours have passed. Once the bands have been used they cannot be used again until the next dawn.

\index[Magic Items]{Efficient Quiver}\smallskip* \textbf{Efficient Quiver}
2500 gp, rare, each of the quiver's three compartments is connected to an extradimensional space that allows it to carry numerous objects while never weighing more than 1 pound.

The smaller compartment can hold up to 60 arrows, bolts or similar items. The middle compartment can hold up to 18 javelins or similar items. The longest compartment can hold up to 6 long items, such as bows, fighting sticks or spears. You can draw any item contained in the quiver as if you were taking it from a normal quiver or scabbard.

\smallskip* \textbf{Phylactery against the undead}\index[Magical Items]{Phylactery against the undead}
1000 gp, rare, this sacred object allows you to use the Turn Undead Feat with a +2 bonus to the sum of Traits in common with the Patron.

\smallskip* \textbf{Phylactery of Youth}\index[Magic Items]{Phylactery of Youth}
10,000 gp, legendary, the strip of parchment from this phylactery is usually enclosed in a small metal tube worn around the neck. When a character wears it, his natural aging rate drops to 75\%, while any magical aging is reduced by half.

\index[Magic Items]{Instant Fortress}\smallskip* \textbf{Instant Fortress}
75,000 gp, very rare, you can use two actions to place this 1-inch-edged metal cube on the ground and speak its command word. The cube quickly grows into a fortress that will remain until you use two actions to say the dismissal command word, which only works when the fortress is empty.

The fortress is a square tower, 6 meters per side and 9 meters high, with slits on all sides and battlements at the top. Its interior is divided into two floors, with a staircase running along a wall to join them. The staircase ends with a trap door that opens onto the roof. When activated, the tower has a small door on the side facing you. The door opens only at your command, which you can give with two actions. It is immune to the lockpick spell and similar spells, such as that of the door clapper.

Each creature in the area where the fortress appears must make a DC 17 Reflex saving throw, taking 10d10 bludgeoning damage on a failed save, or half as much damage on a successful one. In both cases, the creature is pushed into a space outside the fortress but close to it. Objects in the area that are not being worn or carried take the same damage and are automatically pushed.

The tower is made of adamantium, and its magic prevents it from being toppled. The roof, door, and walls each have 100 hit points, immunity to damage from nonmagical weapons except siege weapons, and resistance to all other damage.

Only the wish spell can repair the fortress. Each wish cast causes the roof, door, or one of the walls to recover 50 hit points.


\smallskip* \textbf{Locating arrow}\index[Magic Items]{Locating arrow}
400 gp, uncommon, this arrow can be used up to 8 times per day. It is thrown into the air, and when it lands it indicates a desired direction or place. Possible indications include the nearest exit or entrance, stairs, passages, caves, and similar areas.

\index[Magical Items]{Censer of the Command of the Air Elementals}\smallskip* €14645{Censer of the Command of the Air Elementals}
8000 gp, rare, While the incense burns within this censer, you can use two actions to speak the brazier's command word and summon an air elemental, as if you had cast the summon elementals spell. The censer cannot be used in this way again until the next dawn. This 15 centimeter wide and 30 centimeter high censer resembles a chalice with a decorated cover. It weighs 0.5 kilos.

\smallskip* \textbf{Incense of meditation}\index[Magical Items]{Incense of meditation}
5,000 gp, rare, this sweet-smelling block of incense is indistinguishable from regular incense until lit. When it burns, its particular fragrance and pearly smoke are recognizable with an Arcana check at DC 13. After a spellcaster has spent 8 hours reviewing the Tome and meditating near a lit block, he will acquire the ability to cast his spells with the maximum possible effect and duration, spells that require a saving throw will impose an additional -1 penalty. Each incense block burns for 8 hours and the effect persists for another 8 hours. Usually 2d4 blocks of incense are found in the same case.

\index[Magical Items]{Lantern of Revelation}\smallskip* \textbf{Lantern of Revelation}
5,000 gp, uncommon, While lit, this lantern burns for 6 hours on 1 flask of oil, radiating bright light in a 30-foot radius and dim light for an additional 30 feet. Invisible creatures and objects are made visible while under the bright light of the lantern.

He can use two actions to lower his cover, dimming the light with a radius of 3 feet.

\smallskip* \textbf{Incense of Obsession}\index[Magical Items]{Incense of Obsession}
rare, very similar to the incense of meditation, this incense also gives the user the impression of its effect, but will be under Confusion for 24 hours if he fails a DC 23 Will save.

\index[Magic Items]{Deck of Illusions}\smallskip* \textbf{Deck of Illusions}
6500 gp, uncommon, this box contains a set of parchment cards. A complete deck contains 34 cards, each depicting a different creature. The creatures represented are left to the discretion of the Narrator. Usually the decks found around are missing 3d6-3 cards.

The magic of the deck only works if the cards are drawn at random (you can use a deck of normal playing cards modified to simulate the illusion deck). You can use two actions to draw a card from the deck and throw it to a point on the ground 30 feet away from you.

The illusion of one or more creatures forms on top of the cast card and remains until it is dispelled. The illusory creature appears real, of the appropriate size, and behaves as if it were a real creature, except that it cannot deal damage. As long as you are within 120 feet of the illusory creature and can see it, you can use two actions to magically move it anywhere within 30 feet of the card. Any physical interaction with the illusory creature reveals it as an illusion, as objects pass through it. Someone who uses two actions to visually inspect the creature identifies it as illusory with a successful DC 17 Intelligence check. The creature then appears transparent to her.
The illusion lasts until the card is moved or the illusion dispelled. When the illusion ends, the image on the card disappears, and that card can no longer be used.

\end{multicols}

%\medskip

%\begin{center}
%\includegraphics[width=0.55\linewidth]{immagini/Incenso.png}

%\end{center}

\begin{tabular}{ll|ll}
\textbf{Playing Card}& \textbf{Illusion}&\textbf{Playing Card}& \textbf{Illusion}\\
\hline
Ace of Hearts & Red Dragon & Ace of Diamonds & Beholder\\
King of Hearts & Knight and Four Guards & King of Diamonds & Archmage and Apprentice Wizard\\
Queen of Hearts &Succubus or Nightmare&Queen of Diamonds &Night Hag\\
Jack of Hearts & Druid & Jack of Diamonds & Assassin\\
Ten of Hearts &Cloud Giant&Ten of Diamonds &Fire Giant\\
Nine of Hearts &Ettin&Nine of Diamonds &Oni\\
Eight of Hearts & Bugbear & Eight of Diamonds & Gnoll\\
Two of Hearts &Goblin&Two of Diamonds &Kobold\\
Ace of Spades & Lich & Ace of Clubs & Iron Golem\\
King of Spades &Priest and two acolytes&King of Clubs &Bandit Captain and three bandits\\
Queen of Spades & Medusa & Queen of Clubs & Erinyes\\
Jack of Spades & Veteran & Jack of Clubs & Berserker\\
Ten of Spades &Frost Giant&Ten of Clubs &Hill Giant\\
Nine of Spades &Trolls&Nine of Clubs &Ogre\\
Eight of Spades & Hobgoblin & Eight of Clubs & Ogre\\
Two of Spades &Goblin&Two of Clubs &Kobold\\
Joker (2) &You (the owner of the deck)&&\\
\end{tabular}

\begin{multicols}{2}

\medskip

\index[Magic Items]{Deck of Wonders}\smallskip* \textbf{Deck of Wonders}
100,000 gp, legendary, usually found in a pouch or box, which contains cards made of ivory or fleece. Most of these decks (75\%) have only thirteen cards, while the remaining decks have twenty-two.

Before drawing a card, you must declare how many cards you intend to draw and then draw them randomly (you can use a modified deck of playing cards to simulate the deck). Any cards drawn in excess of this number have no effect. Otherwise, as soon as you draw a card from the deck, its spell takes effect.

You must draw each card within 1 hour of the previous draw. If you do not draw the chosen number of cards, the remaining number of cards will come out of the deck spontaneously and take effect at the same time. Once a card is drawn, it will fade from existence. Unless the card is the Fool or the Jester, the card reappears in the deck, making it possible to draw the same card twice.

\medskip

\end{multicols}

\begin{tabularx}{0.95\textwidth}{lX|lX}
\textbf{Playing Card}& \textbf{Card}&\textbf{Playing Card}& \textbf{Card}\\
\hline
Ace of Diamonds & Vizier* & Ace of Hearts & Fate*\\
King of Diamonds &Sun&King of Hearts &Throne\\
Queen of Diamonds & Moon & Queen of Hearts & Key\\
Jack of Diamonds & Star & Jack of Hearts & Knight\\
Two of diamonds &Comet*&Two of hearts &Gem*\\
Ace of Clubs &Spurs*&Ace of Spades& Dungeon*\\
King of Clubs & The Void & King of Spades & Ruin \\
Queen of Clubs & Flames & Queen of Spades & Euryale\\
Jack of Clubs & Skull & Jack of Hearts & Knave\\
Two of Clubs &Idiot&Two of Spades &Hanging Man*\\
Jolly &Matto*&Jolly &Buffone\\
\end{tabularx}

\begin{multicols}{2}

\medskip

* Only in 22 card deck

\emph{Hanging} (only in pack of 22). Your mind is blown, and you change 2 Traits

\emph{Jester}. You gain 35 XP or you can draw two additional cards in addition to your declared draws.

\emph{Knight}. You gain the services of a 4th-level AC fighter who appears in a space of your choice within 30 feet of you. The warrior is of the same race as you and will serve you loyally until death, believing that it was fate that brought him into your service. The character is controlled by you.

\emph{Key}. A rare, very rare, or legendary magical weapon that you are proficient with appears in your hand. The Storyteller determines what type of weapon it is.

\emph{Comet} (only in deck of 22). If you single-handedly defeat the next monster or hostile group you encounter, you will gain enough experience points to gain a level. Otherwise, this card will have no effect.

\emph{Euryale}. You are cursed by the card and take a -2 penalty on all saving throws as long as you remain cursed in this way. Only a Patron or Fate card magic can end this curse.

\emph{Fate} (only in deck of 22). The structure of reality dissolves and reforms, allowing you to avoid or erase an event as if it never happened. You can use this card's magic as soon as you draw it or wait any other time until you die.

\emph{Flames}. A powerful devil becomes your enemy. The devil will try to ruin and infest your existence, savoring your suffering until the moment he tries to kill you. This enmity will last until you or the devil dies.

\emph{Villain}. A non-player character of the Storyteller's choice becomes hostile towards you. The identity of the new enemy is unknown until the NPC or someone else reveals it. Nothing short of a wish or divine intervention will end the NPC's hostility towards you.

\emph{Gem} (only in deck of 22). Twenty-five jewels worth 2,000 gp each or fifty gems worth 1,000 gp each appear before your feet.

\emph{Idiot} (only in deck of 22). Permanently reduce your Intelligence score by 2 (to a minimum score of 3). You may draw an additional card before your other declared draws.

\emph{Moon}. You gain the ability to cast the wish spell 1d3 times.

\emph{Matto} (only in deck of 22). You lose 10000 XP, discard this card, and draw from the deck again, counting both draws as just one of your draws. If losing that number of XP would cause you to lose a level, you will instead be left with just enough XP to maintain your level.

\emph{Ruin}. You lose all riches you have with you, apart from other magic items. Assets, buildings, and lands you own are lost in the least reality-altering way. Any documents that prove you are the owner of something you lost due to this card disappear.

\emph{Sun}. You gain 35 XP, and a wondrous item (determined by the Storyteller) appears in your hands.

\emph{Underground} (only in deck of 22). You disappear and are buried in a state of suspended animation within an extradimensional sphere. Everything you were wearing or carrying remains in the space you occupied when you disappeared. You will remain imprisoned until you are found and removed from the sphere. You cannot be located by any divination magic, but the wish spell can reveal the location of your prison. No further cards are drawn.

\emph{Spurs} (only in pack of 22). Any magical item you are wearing or carrying is disintegrated. Artifacts in your possession are not disintegrated, but vanish.

\emph{Star}. Increase your ability score by 1. The score can exceed 5, but cannot exceed 7.
\emph{Skull}. You summon an avatara of death (a ghostly humanoid skeleton clad in a tattered black robe, holding a ghostly scythe). It appears in a space of the Storyteller's choice within 10 feet of you and attacks you, warning everyone else that you must win the battle alone. The avatar fights until you die or until it drops to 0 hit points, at which point it vanishes. If someone tries to help you, they will summon their avatar of death. A creature slain by a death avatara cannot be brought back to life.

\emph{Throne}. You gain +1d6 Diplomacy. Additionally, you gain ownership rights to a small keep somewhere in the world. However, the keep is currently occupied by monsters, which you will have to hunt before you can claim it as your own.

\emph{Vizier} (only in deck of 22). At any time of your choosing, within one year of drawing this card, you can thoughtfully ask for an answer to your question and receive a truthful answer to it. Aside from providing information, the answer can help you solve a complex problem or dilemma. In other words, knowledge is provided along with wisdom on how to employ it.

\emph{Empty}. This black card indicates disaster. Your soul is kidnapped from your body and imprisoned inside an object in a location of the Storyteller's choosing. One or more powerful creatures protect this place. As long as your soul is thus trapped, your body is incapacitated. The wish spell cannot restore your soul, but it can reveal the whereabouts of the object containing it. No more cards are drawn.

\emph{Avatar of Death}

Medium undead, neutral evil

\textbf{Strength} +3

\textbf{Dexterity}' +3

\textbf{Intelligence} +3

\textbf{Wisdom} +3

\textbf{Charisma} +3

\textbf{Defense} 20

\textbf{Hit Points} half of its summoner's Hit Points

\textbf{Movement}: Speed ​​18 m, flight 18 m (floats)

\textbf{Damage Immunity}: Void, poison

\textbf{Condition Immunity}: Charmed, Poisoned, Paralyzed, Petrified, Frightened, Fainted

\textbf{Senses}: darkvision 18 m, true vision 18 m

\textbf{Languages}: all languages ​​known by its summoner

\textbf{Challenge} (0 PX)

\textbf{Incorporeal Movement}. The avatar can pass through creatures and objects as if they were difficult terrain. You take 5 (1d10) force damage if you end your round inside an object.

\textbf{Immunity to Turn}. The avatar is immune to effects that turn undead.

\textbf{Shares}

\textbf{Reaper Scythe}. The avatara thrusts its spectral scythe into a creature within 3 feet of it, dealing 7 (1d8 + 3) piercing damage plus 4 (1d8) negative energy damage.

\index[Magical Items]{Miniature of Wonderful Power}\smallskip* €14718{Miniature of Wonderful Power}
variable rarity, variable cost, a miniature of marvelous power is a figurine of a beast, small enough to fit in your pocket. If you use two actions to speak a command word and throw the figure to a point on the ground within 60 feet of you, the figure becomes a living creature. If the space in which the creature appears is occupied by another creature or object, or if there is not enough space for the creature, the miniature does not transform.

The creature is friendly towards you and your companions. She understands your languages ​​and obeys the orders you give her. If you give it no commands, the creature defends itself but takes no other actions. See the Bestiary for the creature's other statistics.

The creature remains for the duration specified for each miniature. At the end of the duration, the creature reverts to its miniature form. It transforms early if it drops to 0 hit points or if you use two actions to speak the command word again while touching it. After the creature returns to a miniature, its properties can no longer be used until a certain amount of time has passed, as specified in the miniature's description.

\emph{Onyx Dog} (Rare, 500 gp). This onyx figurine depicts a dog. He can become a mastiff for up to 6 hours. The mastiff has Intelligence -2 and can speak Common. He also has darkvision with a range of 60 feet and can see invisible creatures and objects within that range. Once used, it cannot be used again until 7 days have passed.

\emph{Ivory Goat (Rare. 1000 gp)}. These ivory goat figurines are always created in sets of three. Each goat has a unique appearance and functions differently from others. Their properties are as follows:

The Terror Goat can become a Giant Goat for up to 3 hours. The billy goat cannot attack, but you can remove its horns and use them as weapons. One horn becomes a knight's lance +1 while the other becomes a longsword +2.

Removing a horn requires two actions, and the weapons disappear and the horns reappear when the goat reverts to its miniature form. Additionally, the billy goat radiates a 30-foot radius aura of terror as long as you ride it. Any creature hostile to you that begins its round within the aura must succeed on a DC 17 Will save or remain
frightened by the billy goat for 1 minute, or until the billy goat returns to miniature form. The frightened creature can repeat the saving throw at the end of each of its rounds, ending the effect on a success. Once a creature succeeds on its saving throw against this effect, it is immune to the goat's aura for the next 24 hours. Once used, the miniature cannot be used again until 15 days have passed.

The labor goat can become a giant goat for up to 3 hours. Once used, it cannot be used again until 30 days have passed.
The Traveling Goat can become a Big Goat with the same stats as a Racehorse. It has 24 charges, and each hour or portion of it spent in beast form costs 1 charge. As long as it has charges, you can use it as much as you like. Once the charges run out, it returns to being a miniature and cannot be used again before 7 days have passed, when it will have recovered all its charges.

\emph{Silver Raven} (Uncommon, 300 gp). This silver figurine depicts a raven. Can become a raven for up to 6 hours. Once used, it cannot be used again until 2 days have passed. While in raven form, the miniature allows you to cast the animal messenger spell on it at will.

\emph{Obsidian Steed} (Very Rare, 1000 gp). This smooth obsidian figurine becomes a nightmare for up to 24 hours. The nightmare fights only to defend itself. Once used, it cannot be used again until 5 days have passed.

\emph{Marble Elephant} (Rare, 1500 gp). This marble figurine is approximately 10 centimeters wide and high. It can become an elephant for up to 24 hours. Once used, it cannot be used again until 7 days have passed.

\emph{Bronze Griffin} (Rare, 1250 gp). This bronze figurine depicts a rampant griffin. Can become a griffin for up to 6 hours. Once used, it cannot be used again until 5 days have passed.

\emph{Snake Owl} (Rare, 400 gp). This serpentine owl figurine can become a giant owl for up to 8 hours. Once used, it cannot be used again until 2 days have passed. If you are on the same plane of existence, the owl can communicate telepathically with you at any distance.

\emph{Golden Lions} (Rare, 800 gp). These gold lion figurines are always created in pairs. You can use one or both thumbnails at the same time. Each can become a lion for up to 1 hour. Once one of the lions has been used, it cannot be used again until 7 days have passed.

\index[Magic Items]{Killing Ammo}\smallskip* \textbf{Killing Ammo}
700 gp, very rare, if a creature of the type, race, or group that the kill arrow is associated with takes damage from the arrow, the creature must make a DC 21 Fortitude save, taking an additional 6d10 piercing damage if it fails, or half as much damage on a success.

Once the kill arrow deals additional damage to the creature, it becomes a nonmagical arrow.

\index[Magic Items]{Crystal Ball}\smallskip* \textbf{Crystal Ball}
50,000 gp, very rare or legendary, a typical crystal ball is about 15 centimeters in diameter. While touching it, you can cast the scrying spell through it (save DC 21). The following variant crystal balls are legendary items and have additional properties.

\emph{Crystal Mind Reading Ball}. This crystal ball is approximately 12cm in diameter. While touching it, you can cast the scrying spell through it (Save DC 21). You can use two actions to cast the detect thoughts spell (save DC 21) while scrying through this crystal ball, targeting creatures you can see that are within 30 feet of the spell's sensor. You don't have to focus on this pinpointing of thoughts to maintain it for its duration, which ends when scrying ends.

\emph{Crystal Ball of Telepathy}. This crystal ball is approximately 12cm in diameter. While touching it, you can cast the scrying spell through it (Save DC 21). As you scry through this crystal ball, you can communicate telepathically with creatures you can see that are within 30 feet of the spell's sensor. You can also use two actions to cast the suggestion spell (Save DC 21) on one of these creatures via the sensor. You do not have to concentrate on this suggestion to maintain it for its duration, which ends if scrying ends. Once used, the crystal ball's suggestion power cannot be used again until the next dawn.

\emph{Crystal Ball of True Seeing}. This crystal ball is approximately 12cm in diameter. While touching it, you can cast the scrying spell through it (Save DC 21). As you scry with this crystal ball, you have true vision with a 120-foot radius centered on the spell's sensor.

\smallskip* \textbf{Hypnotic Crystal Ball}\index[Magic Items]{Hypnotic Crystal Ball}
Rare, this cursed object is indistinguishable from a normal Crystal Ball. However, anyone attempting to use the device is charmed for 1d6 turns, and a telepathic suggestion is implanted in their mind if they fail a DC 27 Will save. The user of the device believes they have seen the desired creature or scene, but in reality is under the influence of a powerful spellcaster, or even a power or being from another plane of existence. With each further use the user falls more and more under the influence of the controller, as a servo or as an instrument. The user is always unaware that he is being subjugated.

\index[Magic Items]{Scroll of Spells}\smallskip* \textbf{Scroll of Spells}
variable rarity, see scroll creation costs, a spell scroll contains the words of a single spell, written in a mystical code.

To read a scroll you need:

\textbf{in case of ISY SCROLL scrolls}:

To understand the content, an Arcana check at difficulty DC 10 is sufficient

to be able to read and cast the scroll's spell requires an Intelligence check (or Arcana if known) at difficulty 12.

\textbf{in case of normal scrolls}:

To understand its content, an Arcana test at difficulty 15 is required

An Arcana check at difficulty 20 is required to be able to read and cast the scroll's spell.

Casting the spell by reading it from a scroll takes the normal spell casting time. Once the spell is cast, the words on the scroll vanish, and the scroll is reduced to dust. If the cast is interrupted, the scroll does not dissolve.

\smallskip* \textbf{Scroll of protection against elementals}\index[Magic Items]{Scroll of protection against elementals}
800 gp, rare, protects against all elementals for 20 rounds, granting +4 Defense and saving throws against attacks or effects produced by elementals.

\smallskip* \textbf{Scroll against werewolves}\index[Magic Items]{Protective scroll against werewolves}
700 gp, uncommon, protects against all werewolves for 20 rounds, granting +4 Defense and saving throws against attacks or effects produced by werewolves.

\smallskip* \textbf{Scroll against the undead}\index[Magic Items]{Scroll of protection against the undead}
900 gp, uncommon, protects against all undead for 20 rounds, granting +4 Defense and saving throws against attacks or effects produced by undead.

\smallskip* \textbf{Scroll against magic}\index[Magic Items]{Protective scroll against magic}
1500 gp, rare, scroll casts an Anti-Magic Field spell.

\index[Magical Items]{Pearl of Power}\smallskip* \textbf{Pearl of Power}
6000 gp, uncommon, while you have the pearl with you, you can use two actions to regain 2d4 Magic Points. Once used, the pearl cannot be used again until the next dawn. There are more powerful variants that recover more points.

\smallskip* \textbf{Stone of Weight}\index[Magic Items]{Stone of Weight}
this object looks like a smooth, shiny black stone. When the bearer is involved in combat or flight, he suddenly suffers the effects of the slow spell. Once taken, the stone cannot be thrown away normally, as after a short time it magically reappears on the possessor's person. To permanently get rid of the stone you need the Remove Curse spell.

\index[Magic Items]{Arcane Stone}\smallskip* \textbf{Arcane Stone}\index{Ioun Stone}
variable cost, variable rarity, there are different types of arcane stone, each type a specific combination of shapes and colors.

When you use two actions to throw one of these stones into the air, the stone begins to orbit your head at a distance of 1d3 x 12 inches and grants you a benefit.
Then, another creature must use two actions to grab or snag the stone and separate it from you, succeeding on an attack roll vs. Defense 24 or succeeding on a DC 31 Dexterity check. You can use two actions to grab and set the stone aside, ending its effect.

A stone has Defense 24, 10 Hit Points, and resistance to all damage. While it orbits your head it is considered a worn object.

\emph{Dexterity} (very rare, 3000 gp). While orbiting your head your Dexterity score increases by 1, to a maximum of 5.

\emph{Absorption} (very rare, 6000 gp). While it orbits your head, you can use your Action to cancel a spell of level 4 or lower cast by a visible creature that targets only you. Once the stone has cleared 5 Spells, it runs out and turns dull grey, losing its magic.

\emph{Authority} (very rare, 3000 gp). While orbiting your head your Charisma score increases by 1, to a maximum of 5.

\emph{Awareness} (rare, 12000 gp). As it orbits your head you cannot be surprised.

\emph{Strength} (very rare, 3000 gp). While orbiting your head your Strength score increases by 1, to a maximum of 5.

\emph{Intelligence} (very rare, 3000 gp). While it orbits your head, your Intelligence score increases by 1, to a maximum of 5.

\emph{Intuition} (very rare, 3000 gp). While it orbits your head, your Wisdom score increases by 2, to a maximum of 5.

\emph{Protection} (rare, 10000 gp). While orbiting your head you gain a +1 bonus to Defense.

\emph{Sustenance} (rare, 3500 gp). While it orbits your head you do not need to eat or drink.

\index[Magical Items]{Stone of Good Luck}\smallskip* \textbf{Stone of Good Luck}
4,500 gp, uncommon, while the stone is with you, you gain a +1 bonus on ability checks and saving throws.

\index[Magical Items]{Stone of Control of Earth Elementals}\smallskip* €14780{Stone of Control of Earth Elementals}
8000 gp, rare, if the stone touches the ground, you can use two actions to speak the command word and summon an earth elemental, as if you had cast the summon elementals spell. The stone cannot be used in this way again until the next dawn. The stone weighs 2.5 kilos.

\index[Magic Items]{Sewer Pipe}\smallskip* \textbf{Sewer Pipe}
2000 gp, uncommon, you must be proficient with wind instruments to use this fife. While you use this fife, normal rats and giant rats are indifferent towards you and will not attack you unless you threaten or harm them. If you blow the pipe with two actions, you can use two actions to spend 1 to 3 charges, summoning a swarm of rats for each charge spent, as long as there are enough rats within 750 meters of you to summon in this manner (at the Storyteller's discretion ). If there aren't enough rats to form a swarm, the charge is wasted. The summoned swarms move towards the music via the shortest possible route, but are not otherwise under your control. The fife has 3 charges and regains 1d3 spent charges each day at dawn.

Whenever a swarm of rats that is not under the control of another creature comes within 30 feet of you while you are playing the pipe, you can make a Charisma check contested by the swarm's Wisdom check. If you lose the contest, the swarm behaves as normal and cannot be distracted by the fife's music again for the next 24 hours. If you win the contest, the swarm is attracted by the fife's music and becomes friendly towards you and your companions as long as you continue to play the fife with two actions each round. A friendly swarm obeys your commands. If you do not give orders to a friendly swarm, it will defend itself but take no further actions.

If a friendly swarm cannot hear the fife's music at the start of the round, your control of that swarm ends, and the swarm behaves as it normally would and cannot be lured by the fife's music again for the next 24 hours.

\index[Magic Items]{Pife of Fright}\smallskip* \textbf{Fife of Fright}
6000 gp, uncommon, you must be proficient with wind instruments to use this fife. You can use two actions to play it and spend 1 charge to create an enchanting, ghostly sound. Each creature within 30 feet of you that hears you play must succeed on a DC 17 Will save or be frightened of you for 1 minute. If you wish, all creatures in the area that are not hostile to you can automatically succeed on their saving throw. A creature that fails the saving throw can repeat it at the end of its round, ending the effect on itself on a successful one. A creature that succeeds on its saving throw is immune to this pipe's effect for 24 hours. The fife has 3 charges and regains 1d3 spent charges each day at dawn.

\index[Magic Items]{Wonder Pigments}\smallskip* \textbf{Wonder Pigments}
400 gp, very rare, usually found in 1d4 jars inside elegant wooden boxes together with a brush (with a total weight of 500 grams), these pigments allow you to create three-dimensional objects, painting them in two dimensions. The paint flows from the brush to form the desired object as you focus on the image

Each pot of paint is enough to cover 90 square meters of surface area, allowing you to create inanimate objects and terrain features (doors, pits, flowers, trees, cells, rooms or weapons) that take up a total of 270 cubic metres. It takes 10 minutes to cover 90 paintings.

When you complete the painting, the painted object or terrain feature becomes a real object, not a magical one. So, painting a door on a wall creates a real door that can be opened to access what lies beyond it. Painting a pit on the floor creates an actual pit, the depth of which is counted towards the total area of ​​items you can create.

Nothing created from pigments can have a value greater than 25 gp. If you paint an object of higher value (a diamond or a pile of gold), the object will appear authentic, but careful examination will reveal that it is made of plaster, bone, or some other worthless material.

If you paint a form of energy, such as fire or lightning, the energy appears but dissipates as you complete the painting, causing no damage to anything.

\index[Magical Items]{Arcane Feather}\smallskip* \textbf{Arcane Feather}
variable cost, variable rarity, this tiny object resembles a feather. There are several types of arcane feathers, each with a single one-time effect. The Storyteller chooses the type of arcane feather.

\emph{Tree}. You must be outdoors to use this arcane feather. You can use two actions to place it on an unoccupied space on the ground. The feather vanishes, and a nonmagical oak tree appears in its place. The tree is 18 meters tall and has a trunk 1 meter in diameter. At the top, its branches extend up to 6 meters. 50 gp

\emph{Again}. You can use two actions to place the arcane feather on a boat or ship. For the next 24 hours, the vessel cannot be moved in any way. Touching the vessel with the arcane feather again ends this effect. When the effect ends, the feather vanishes. 50 gp

\emph{Whip}. You can use two actions to throw the arcane feather to a point within 10 feet of you. The feather vanishes and a floating whip appears in its place. You can then use two actions to make a melee spell attack against a creature within 10 feet of the whip, with a +9 attack bonus. On a hit, the target takes 1d6 + 5 force damage. During your round, with two actions you can direct the whip to fly up to 20 feet and repeat the attack against a creature within 10 feet of it. The whip wears off after 1 hour, when you use two actions to dismiss it, or when you are incapacitated or die. 250 gp

\emph{Swan Ship}. You can use two actions to place the arcane feather on a body of water at least 60 feet in diameter. The feather vanishes and in its place appears a boat 15 meters long and 6 meters wide in the shape of a swan. The boat moves on its own and moves through the water at a speed of 9 kilometers per hour. You can use two actions while on board to command it to move or turn 90 degrees. The boat can carry up to thirty-two creatures of Medium or smaller size. A Large creature counts as four Medium creatures, while a Huge creature counts as nine Medium creatures. The boat vanishes after 24 hours. You can dismiss the boat with two actions. 3000 gp

\emph{Bird}. You can use two actions to throw the arcane feather 3 feet into the air. The feather vanishes and a huge multicolored bird takes its place. The bird has the statistics of a Roc, but obeys simple commands and cannot attack. It can carry up to 250 kilos while flying at its maximum speed (24 kilometers per hour for a maximum of 216 kilometers per day, with one hour of rest every 3 hours of flight), or 500 kilos of weight at half speed. The bird vanishes after flying the maximum possible distance in a day or if it drops to 0 hit points. You can dismiss the bird with two actions. 3000 gp

\emph{Fan}. If you are on a boat or ship, you can use two actions to throw the arcane feather up to 10 feet into the air. The feather vanishes and a giant fan appears in its place. The fan floats and creates a wind strong enough to fill the ship's sails, increasing its speed by 7.5 kilometers per hour for 8 hours. You can dismiss the fan with two actions. 250 gp

\index[Magical Items]{Dust of Aridity}\smallskip* \textbf{Dust of Aridity}
120 gp, rare, this small package contains 1d6 + 4 pinches of dust. You can use two actions to sprinkle a pinch of dust on the water. The dust turns a 10-foot cube of water into a ball of dust the size of a marble, which floats or settles where the dust was thrown. dust. The weight of the ball is negligible.

Anyone can use two actions to smash the ball against a hard surface, causing the ball to break and release the water absorbed by the dust. Doing so exhausts the ball's magic.

An elemental composed primarily of water and exposed to a pinch of this dust must make a DC 15 Fortitude save, taking 10d6 void damage on a failed save, or half as much damage on a successful one.

\smallskip* \textbf{Revealing Dust}\index[Magic Items]{Revealing Dust}
500 gp, uncommon, this fine powder appears to be a very light metallic dust. A handful of this substance sprayed into the air covers all objects within a 10-foot radius, making everything visible. When sprayed through a blowpipe, the powder fills a cone 6 meters long and 1 meter wide at the end. The dust nullifies the effects of illusory power, the warp cloak, the elven cloak, and the special abilities of creatures such as unstable molossians and warp panthers; the effect lasts 2d10 turns. The revealing powder is usually stored in small silk pouches or hollow tubes made of bone; normally there are 5d10 doses of powder.

\index[Magic Items]{Powder of Vanishing}\smallskip* \textbf{Powder of Vanishing}
700 gp, rare, found in small bags, this powder looks like very fine sand. There is enough in one bag for one use. When you use two actions to toss dust into the air, you and each creature and object within 10 feet of you become invisible for 2d4 minutes. The duration is the same for all subjects, and when the magic takes effect the dust is consumed. If a creature under the effects of the dust attacks or casts a spell, invisibility ends only for that creature.

\index[Magic Items]{Sneezing and Choking Powder}\smallskip* \textbf{Sneezing and Choking Powder}
480 gp, uncommon, found in small containers, this powder appears to be fine sand. It appears similar to vanishing dust, and the identify spell reveals it as such. There is enough for one use. When you use two actions to throw a handful of dust into the air, you and all creatures within 30 feet of you that need to breathe must succeed on a DC 17 Fortitude save or stop breathing, and begin sneezing. uncontrollable manner. A creature afflicted in this way is incapacitated and suffocates. As long as it is conscious, the creature can repeat the saving throw at the end of each of its rounds, ending the effect on a success. The lesser restoration spell can also end the effect afflicting the creature.

\index[Magic Items]{Cubic Portal}\smallskip* \textbf{Cubic Portal}
40,000 gp, legendary, this 3-inch cube radiates palpable magical energy. The six faces of the cube are each connected to a different plane of existence, one of which is the Material Plane. The other faces are connected to planes determined by the Storyteller.

You can use two actions to press a face of the cube to cast the portal spell through it, opening a passage to the plane connected to that face. Alternatively, if you use two actions to press a face twice, you can cast the planar shift spell (Save DC 17) through the cube and transport its targets to the plane connected to that face. The cube has 3 charges. Each use of the cube expends 1 charge. The cube recovers 1 spent charge each day at dawn.

\index[Magical Items]{Well of Many Worlds}\smallskip* \textbf{Well of Many Worlds}
75,000 gp, legendary, this elegant black fabric, soft as silk, is rolled up to the size of a handkerchief. It unfolds into a circular sheet 1.8 meters in diameter. You can use two actions to unfold and place the Well of Many Worlds on a solid surface, on which it creates a two-way portal to another world or plane of existence. Each time the object opens a portal, the Storyteller decides where it leads. You can use two actions to close an open portal by grabbing the edges of the cloth and folding them over. Once a Well of Many Worlds has opened a portal, it cannot do so again until 1d8 hours have passed.

\smallskip* \textbf{Hindering Net}\index[Magical Items]{Hanging Net}
800 gp, rare, this 10-foot square net can be thrown at an opponent to entangle him. The net is very resistant and requires the strength of a giant (Str 5) to tear it with bare hands. The net also resists cuts, and must be hit with extreme precision (Defense 25, Hit Points 30) for it to give way. The net can also be hung or placed on the ground as a trap, which will magically activate upon the owner's command.

\smallskip* \textbf{Entanglement Net}\index[Magic Items]{Entanglement Net}
900 gp, rare, this net can only be used underwater, but functions exactly like a entanglement net on the surface, floating up to 30 feet away if needed to entangle an opponent.

\smallskip* \textbf{Animated Attack Broom}\index[Magic Items]{Animated Attack Broom}
This object is indistinguishable in appearance from a normal broom. In all tests it is identical to a flying broom, up to 6 meters high. When this happens the broom performs a pirouette and causes its pilot to fall on his head from a height of 1d4+5 x 30 cm (no fall damage is dealt since the distance is less than 3 m). The broom then attacks the victim, hitting him in the face with the brush and beating him with the handle. The broom makes two attacks per round with each end (two attacks with the brush and two with the handle for a total of four attacks). The brush blinds the victim for 1 round when it hits. The handle inflicts 1d3 wounds. The broom has Defense 13, 18 Hit Points, and has +4 to attack rolls.

\index[Magic Objects]{Flying Broom}\smallskip* \textbf{Flying Broom}
8,000 gp, uncommon, this wooden broom, weighing about 3 pounds, functions like a normal broom until you sit on it and speak the command word. It then begins to float beneath you and can be ridden in the air. It has a flight speed of 15 meters. It can carry up to 200 kilos, but its flight speed becomes 9 meters if it were to carry more than 100 kilos. When you land, the broom stops floating.

By saying the command word, naming the place and if you are familiar with it, you can send the broom alone to a place up to 1.5 kilometers from you. The broom will return to you when you say another command word, as long as it is still within 1.5 kilometers of you.

\smallskip* \textbf{Broom of Cursed Flight}\index[Magic Items]{Broom of Cursed Flight}
This magic broom looks like a flying broom. However, when activated, it flies up to 50 feet in the air or to the ceiling (whichever is lower) and then stops functioning, causing its rider to plummet. Then the broom falls to the ground and loses its magical power.

\index[Magical Items]{Sphere of Annihilation}\smallskip* \textbf{Sphere of Annihilation}
250,000 gp, legendary, this black sphere 50 centimeters in diameter is actually a hole in the structure of the multiverse, floating in space and stabilized by the magical field that surrounds it.

The sphere annihilates all the matter it passes through and all the matter that passes through it. The only exception is artifacts. Unless the artifact is susceptible to damage from the sphere of annihilation, it can pass through the sphere without problem. Anything else that touches the sphere and is not completely engulfed and annihilated by it takes 4d10 force damage per round.

The sphere remains immobile until someone controls it. If you are within 60 feet of an uncontrolled sphere, you can take two actions to make an Arcana check with DC 30. If you succeed, the sphere levitates in a direction of your choice, a number of meters equal to 1 x Intelligence ( minimum 1 meter). If you fail, the sphere moves 10 feet towards you. A creature whose space the sphere enters must succeed on a DC 15 Reflex saving throw or be touched by it, taking 4d10 force damage.

If you attempt to control a sphere that is under the control of another creature, you make a contested arcana check against the other creature's arcana. The winner of the contest gains control of the sphere and can levitate it as normal.

If the sphere comes into contact with a planar portal, such as one created by the portal spell, or extradimensional space, such as one within a portable hole, the Storyteller randomly determines what happens, using the following table.

\medskip

\begin{tabularx}{0.45\textwidth}{lX}
\textbf{3d6}& \textbf{Result}\\
\hline
3-10 &The sphere is destroyed\\
12-16& The sphere moves through the portal or into extradimensional space.\\
17-18 &A space rift sends every creature and object within 54 meters of the sphere, including the sphere, into a random plane of existence.\\
\end{tabularx}

\medskip

\index[Magic Items]{Universal Solvent}\smallskip* \textbf{Universal Solvent}
300 gp, legendary, this tube contains a white liquid with a strong odor of alcohol. You can use two actions to pour its contents onto a surface within reach. The liquid instantly dissolves 1000cm x cm of adhesive it comes into contact with, including supreme glue.

\smallskip* \textbf{Mirror of Mental Ability}\index[Magic Items]{Mirror of Mental Ability}
15000 gp, very rare, this object looks like an ordinary mirror one and a half meters high and 60 cm wide. On command, the wielder can use it in the following ways:

- I would read a person's thoughts reflected on their surface with telepathy (without needing to understand an unknown language).

- Seeing other places as with a crystal ball, with the possibility of seeing in other planes, as long as they are sufficiently familiar to the observer.

- Create a portal to visit other places. The owner must first visualize the place, then physically enter the mirror, alone or with the companions he wishes. The mirror will create an invisible portal on the other side, through which the owner, or anyone who can locate it, can pass through.

- Once a week, the mirror can accurately answer a question about a person reflected on its surface (an effect similar to the lore knowledge spell.

\smallskip* \textbf{Mirror of Duplication}\index[Magic Items]{Mirror of Duplication}
legendary, this mirror is a little more than a meter high and a little less wide. When a creature reflects itself on the mirror's surface, its reflected image (an identical duplicate in every way) comes out to attack the original. The duplicate has all the equipment and powers of the original, including magic. The duplicate disappears immediately, along with all its items, when it or the original dies.

\index[Magic Items]{Life Trapping Mirror}\smallskip* \textbf{Life Trapping Mirror}
18,000 gp, rare, when this 120 centimeter tall mirror is looked at indirectly, its surface shows a vague image of the creature. The mirror weighs 25 kilos, has Defense 11, 10 Hit Points and vulnerability to bludgeoning damage. It shatters and is destroyed when reduced to 0 hit points.

If the mirror is hanging from a vertical surface and you are within 3 feet of it, you can use two actions to speak its command word and activate it. It will remain active until you say the command word again.

Any creature other than you who sees its reflection in the activated mirror while within 30 feet of it must succeed on a DC 17 Will save or become trapped, along with anything it is wearing or carrying, in one of twelve extradimensional mirror cells. This saving throw receives +1d6 if the creature knows the nature of the mirror and the constructs automatically succeed on the saving throw.

An extradimensional cell is an infinite space filled with a dense haze that reduces visibility to 10 feet. Creatures trapped in mirror cells do not age, and do not need to eat, drink, or sleep. A creature trapped inside a cell can escape by using magic that allows planeswalking. Otherwise, the creature is confined to the cell until freed.

If the mirror traps a creature but its twelve extradimensional cells are already occupied, the mirror releases one of the trapped creatures at random to house the new prisoner. The freed creature appears in an unoccupied space within sight of the mirror but facing away from it. If the mirror is shattered, all creatures it contains are freed and reappear in an unoccupied space near it.

While you are within 3 feet of the mirror, you can use two actions to speak the name of one of the creatures trapped inside or call out a particular cell number. The creature named or contained in the named cell appears as an image on the mirror surface. After that, you and the named creature can communicate normally.

In a similar way, you can use two actions to speak a second command word and free one of the creatures trapped in the mirror. The freed creature appears, along with all its possessions, in the unoccupied space closest to the mirror and facing away from it.

\smallskip* \textbf{Panic Drums}\index[Magic Items]{Panic Drums}
1500 gp, uncommon, these drums are similar to timpani (small, easily transportable percussion instruments). They are found in pairs and have an inconspicuous appearance. If both are played, all creatures within 250 feet (except those within a 10-foot circle centered on the drums) are struck by Fear and flee at full speed for 30 rounds. A DC 21 Will save is allowed to save from the effects.

\smallskip* \textbf{Drums of Stun}\index[Magical Items]{Drums of Stun}
rare, these two paired drums resemble panic drums; when both are played, all creatures within 10 feet must succeed on a DC 21 Fortitude save and be stunned for 2d4 rounds. All creatures within 70 feet are immediately deafened. Spells of greater restoration, healing, regeneration, or similar effects can cure deafness.

\index[Magic Items]{Flying Carpet}\smallskip* \textbf{Flying Carpet}
15,000 gp, very rare, you can speak the carpet's command word with two actions to make the carpet float and fly. It moves according to the directions you tell it by voice, as long as you are within 30 feet of it.

There are four sizes of flying carpet. The Storyteller chooses the size of the carpet or determines it randomly.

\medskip

\begin{tabular}{llll}
d100 &Size (cm)&Capacity &Speed. of Flight\\
01-20& 90 x 150 &100 kg/25&23 metres\\
21-55& 120 x 180 &200 kg/50&18 metres\\
56-80& 150 x 210 &300 kg/75&12 metres\\
81-100& 180 x 270 & 400 kg/100& 9 metres\\
\end{tabular}

\medskip
The Capacity value indicates both the weight transported and the overall dimensions. The carpet can carry up to double the load indicated on the table, but flies at half speed if it carries more.

\smallskip* \textbf{Elemental Air Censer}\index[Magic Items]{Elemental Air Censer}
1500 gp, rare, this censer can be used to summon and control an air elemental similarly to the summon elemental spell. You must prepare the magic item and conduct a ritual for one round before the actual summoning, which takes one round. After the elemental has been summoned, you must maintain concentration to be able to give it orders.

\smallskip* \textbf{Cursed Conjuring Censer}\index[Magic Items]{Cursed Conjuring Censer}
Rare, this censer looks like, and appears to function like, an elemental censer of air. However, once lit it is impossible to turn it off for 1d4 rounds. Each round an air elemental emerges and attacks all nearby creatures.

\index[Magical Items]{Restorative Ointment}\smallskip* \textbf{Restorative Ointment}
5,000 gp, uncommon, this glass jar, 3 inches in diameter, contains 1d4 + 1 doses of a thick mixture. The jar and its contents weigh 250 grams. With two actions, a dose of ointment can be swallowed or applied to the skin. The creature that receives it recovers 2d8 + 2 hit points, stops being poisoned, and is cured of any disease.

\index[Magic Items]{Laptop Compartment}\smallskip* \textbf{Laptop Compartment}
10,000 gp, rare, this elegant black fabric, soft as silk, folds to the size of a handkerchief and unfolds to a circle 2 meters in diameter. You can use two actions to deploy a Portable Pod and place it on or against a solid surface, in which the Portable Pod creates an extradimensional hole 10 feet deep. The cylindrical space inside the hole is on a different plane, and therefore cannot be used to open passages. Any creature inside an open Portable Compartment can climb out of it.

You can use two actions to close a Portable Compartment by taking the edges of the fabric and folding it over. Folding the fabric closes the Well, and any creatures or objects inside remain in extradimensional space. No matter what it contains, the Compartment weighs nothing.

If the compartment is folded, a creature within the compartment's dimensional space can use two actions to make a DC 10 Strength check. If the check succeeds, the creature breaks free and reappears within 3 feet of the portable compartment or the compartment. creature that carries it. A breathing creature can survive inside a closed portable hole for up to 10 minutes, after which it begins to suffocate.

Placing a Portable Compartment within the extradimensional space created by a bag of storage, utility backpack, or similar object instantly destroys both objects and opens a portal to the Astral Plane. Any creature within 10 feet of the portal is sucked into it and deposited in a random location in the Astral Plane. Then the portal disappears.

\index[Magic Items]{Arcane Fan}\smallskip* \textbf{Arcane Fan}
1500 gp, uncommon, while holding this fan, you can use two actions to cast the gust of wind spell through it (save DC 15). Once used, the fan
it should not be used again until the next dawn. Each time it is used before then, there is a cumulative 20\% chance that it will fail and break into useless magicless shreds.


\index[Magic Objects]{Practical Backpack}\smallskip* \textbf{Practical Backpack}
7000 gp, rare, this backpack has a central and two side pouches, each of which is actually an extradimensional space. Each side bag can contain 10 kilos of material, which does not exceed a volume of 60 dm3

The large central bag can contain up to 240 dm3 or 40 kilos of material. The backpack always weighs 2.5 kilos, whatever its contents.

Placing an object inside the backpack follows the normal rules of interaction with objects. Retrieving an item from your backpack requires the use of two actions. When you look for an object in your backpack, it will magically always be at the top of the pile of objects it contains.

The backpack has some limitations. If overloaded, or cut or torn by a sharp object, the backpack will split and be destroyed. If the backpack is destroyed, what it contained is lost forever, although an artifact will always reappear somewhere in the multiverse. If the backpack is turned inside out, what it contains is expelled, without causing any damage, and the backpack must be put back the right way up before it can be used again. If a breathing creature is placed inside the backpack, it can survive there for a maximum of 10 minutes before it begins to suffocate.

Placing the backpack within the extradimensional space created by a holding bag, portable hole, or similar object immediately destroys both objects and opens a portal to the Astral Plane. The portal originates from the point where the objects were placed inside each other. Any creature within 10 feet of the portal is sucked through it and dragged to a random location in the Astral Plane. Then the portal closes. The portal is one-way and cannot be reopened.

\smallskip* \textbf{Titan Hoe}\index[Magic Items]{Titan Hoe}
2000 gp, uncommon, this oversized tool is 10 feet long and weighs 250 pounds (30 Encumbrance), and can only be used by a giant (or an enlarged character) to move large amounts of dirt and build earthworks (a cube of 3 m per Turn). The hoe can also be used to split stone very quickly. If used as a weapon it has a +3 bonus to hit and inflicts 5d6 wounds.

\index[Magic Items]{Hooves of Speed}\smallskip* €14887{Hooves of Speed}
5000 gp, rare, these iron clogs come in sets of four. When all four hooves are attached to a horse or similar creature, they increase that creature's movement speed by 30 feet.

\index[Magical Items]{Hooves of the Zephyr}\smallskip* \textbf{Hooves of the Zephyr}
1500 gp, very rare, these iron clogs come in sets of four. When all four hooves are attached to a horse or similar creature, they allow that creature to move normally, while hovering about four inches above the ground. This effect means that the creature can pass through or pass over non-solid or unstable surfaces, such as water or lava. The creature leaves no tracks and ignores difficult terrain. Additionally, the creature can move at its normal speed for up to 12 hours a day without suffering fatigue from the forced march.

\end{multicols}

\pagebreak

\section{Cursed Items}\index{Cursed Items}

\begin{changemargin}{0.3cm}{0.3cm}\begin{enfasi}{When a wicked man curses his opponent, he curses himself. (Sirach)

\medskip

If you curse a person there will be two pits. (Japanese proverb)}
\end{enfasi}\end{changemargin}\medskip

\begin{multicols}{2}

\label{oggetti-maledetti}

\lettrine[lines=2, lhang=0.33, loversize=0.25, findent=1.5em]{G}{Cursed objects are magical objects with a potentially negative influence on the character.

Cursed objects are almost never made intentionally, rather they are the result of botched work, of craftsmen with little experience or of the lack of adequate components or agreements not respected with some Patron.

The Storyteller can ask for an Arcana check with a DC equal to 10+ days taken to build the magical object in the case of particularly complex objects or if there were problematic situations in the creation and if the check fails or a critical failure is rolled, roll on the table to determine the type of curse the item has.

A curse can also manifest itself as a result of negative or extreme emotional influences involving an object.

\medskip

\textbf{Common Item Curses}

\medskip

\begin{tabular}{ll}
\textbf{\%} & \textbf{Curse}\\
\toprule
01-15 & Deception\\
16-40 & Opposite Effect or Target\\
41-50 & Discontinuous Operation\\
51-65 & Requirement\\
66-90 & Inconvenience\\
91-100 & Completely different effect\\
\end{tabular}

\medskip

Cursed items are \hypertarget{cursed itemsid}{identified} like any other magical item with one exception: unless the Arcana check to identify the item exceeds 35 or the Identify spell is cast with an Arcana check. Magic and get a magic critical (2 times 6) the curse is not detected. If the check is below 35 or without magic critical all that is revealed is the original purpose of the magic item.

If the item is known to be cursed, the nature of the curse can be determined by using DC \hyperlink{identifyom}{standard} to identify the item.

\begin{center}
\includegraphics[width=0.75\linewidth]{immagini/vasobasano.png}

\emph{Basano vase. This vase was made in the second half of the 15th century and is made of silver. DC 35}
\end{center}


\begin{changemargin}{0.3cm}{0.3cm}\begin{narratore}
A curse is always a particular \emph{inconvenience}, which is not used randomly. Think carefully about the cursed objects that you will let the characters find because they will ask you for a lot of information and you will have to be ready.

There is no need for the curse to be excessive and limiting, it can very well be ridiculous or particular, make sure it is characterizing. The character must not feel (except if you want) condemned forever, use the opportunity to build new adventures and team spirit.
\end{narratore}\end{changemargin}


\subsection{Removing Cursed Items}\index{Removing Cursed Items}

While some cursed objects can simply be put down, others exert a strong compulsion on the owner to keep them with them, no matter the cost. Others reappear even if abandoned or impossible to throw away.

These items can only be removed after the Remove Curse spell is cast on the character or item.

If the object has been cursed via the Cast Curse spell, or in any case the Storyteller decides that the object has a particular curse, then the DC of the person casting Remove Curse must be compared (see \hyperlink{spellsave throw}{Saving Throws - Resist the spell}, page \pageref{salvation spell}) with the DC of the item's curse. If the caster of the Remove Curse spell has a higher DC than the item then the item can be removed.

A spellcaster can cast remove curse with a magic check, and for each magical critical success he adds +4 to his DC to compare it to that of the object.

If the caster of Remove Curse is successful then the item can be removed in the next round and the curse remains and strikes again if the item is used/worn again.

Each cursed object has its own method of being destroyed, from being thrown into an active volcano, to being struck by the hammer of the God of Thunder (or Patron...) or devoured by a Colossal Sand Worm if not struck by the breath of a red dragon and a white dragon at the same time...

For many cursed objects, casting the Remove Curses spell is sufficient, without specifying the DC to beat.

\subsection{Common Effects of Cursed Items}

The most common effects of cursed objects are as follows; the Storyteller can invent new effects for specific cursed objects.

\subsubsection{Deception}

Those who use the object continue to believe that it is what it seems at first glance, but in reality it has no power, apart from that of deceiving. The user is mentally tricked into believing that it works and cannot be convinced otherwise except with the use of Remove Curse

\begin{center}
\includegraphics[width=0.70\linewidth]{immagini/mirror.png}

\emph{The mirror in The Myrtles Plantation. DC 28}
\end{center}

\subsubsection{Effect or Opposite Target}

These cursed objects tend to have functional defects which in some cases generate effects diametrically opposed to those desired by their creator, while in other cases they tend to affect the user instead of someone else.

The category of magical items with opposite effects also includes weapons that inflict penalties on attack rolls and damage, rather than bonuses.

The most interesting thing is that these items may not even be a disadvantage for those who own them.

Since a character shouldn't immediately know what a magic item's bonus or effect is, he shouldn't learn the nature of his curse either. Once he finds out, the Remove Curse Spell will be needed to free himself from the object.

\subsection{Discontinuous Operation}

Discontinuous objects work exactly as they should, when they work. Determine whether the object is Unreliable, Conditional, or Uncontrollable.

\medskip
\subsubsection{Unreliable}

Each time the item is activated, there is a 5\% chance that it will not function.

\subsubsection{Conditional}

This item only works in certain situations. To determine what they are, choose an activation condition or consult the table below.

\subsubsection{Uncontrollable}

An uncontrollable object tends to activate randomly. Roll a d\% each day. On a roll of 01--05 the item activates spontaneously at a certain time of day.

\medskip

\begin{tabularx}{0.45\textwidth}{lX}
\textbf{\%} & \textbf{Situation}\\
\toprule
01-03 & Temperature below zero\\
04-05 & Temperature above zero\\
06-10 & During the day\\
11am-3pm & Overnight\\
16-20 & Exposed to sunlight\\
21-25 & In the absence of sunlight\\
26-34 & Underwater\\
35-37 & Out of water\\
38-45 & Underground\\
46-55 & On the surface\\
56-60 & Within 10 feet of a creature type\\
61-64 & Within 10 feet of a race or creature type\\
65-72 & Within 10 feet of a spellcaster\\
73-80 & Within 10 feet of a Follower or Devotee of a specific Patron\\
81-85 & In the hands of a non-spellcaster character\\
86-90 & In the hands of a spellcaster character\\
91-95 & In the hands of a creature with a particular Trait\\
96 & In the hands of a creature of a particular kind\\
97-99 & On non-sacred days or during particular astronomical anniversaries\\
100 & More than 150 km from a certain location\\
\end{tabularx}

\subsection{Requirement}

Some items have much more difficult requirements to meet for them to work. For the item in question to work, you may need to meet one of the following conditions:

\begin{itemize}
\item The character must eat twice as much as normal.
\item The character must sleep twice as much as normal.
\item The character must complete at least one specific mission.
\item The character must sacrifice (destroy) 100 gp worth of precious objects or materials per day.
€14931 € The character must swear loyalty to a particular noble or to his family.
\item The character must abandon all other magical items.
\item The character must be a Follower or Devotee of a specific Patron
\item The character must have a minimum number of ranks in a particular skill.
€14935 € The character must sacrifice part of his life energy (1 point of permanent Constitution) the first time he uses the item.
\item The object must be purified with the Holy Water of a specific Patron every day.
\item The object must be bathed in at least half a liter of blood (animal or humanoid) per day.
\item The item must be used to kill one living creature per day.
\item The item must be used at least once per day, or it will stop working for its current owner.
\item When wielded, the item must draw blood (weapons only). It cannot be set aside or exchanged for another item until it has scored a hit.
\end{itemize}

\medskip

\begin{center}
\includegraphics[width=0.8\linewidth]{immagini/donnalemb.png}

\emph{Woman of Lemb or Statue of the Goddess of Death, 3500 BC. DC 40}
\end{center}

\medskip

The requirements depend on the affordability of the item which should never be determined randomly. An intelligent object with a requirement often imposes its requirement due to its personality.

If the requirement is not met, the object stops working. If it is met, the item usually works for a full day before having to meet the requirement again (although some requirements must be met only once, others once a month, and still others continuously).

\subsection{Inconvenience}

Items that have drawbacks usually have positive effects on those who use them, but they also have negative aspects. While problems sometimes only come to light when items are used (or held in the hand, in the case of items like weapons), they usually remain present until the character gets rid of the item in question.

Unless otherwise noted, ailments remain active for as long as the item remains in the character's possession. The DC of the saving throw to avoid these effects is equal to 10 + DC of the curse (if you have not established the difficulty, set the saving throw, usually on Will, to DC 25)

\end{multicols}

\medskip

\begin{changemargin}{0.3cm}{0.3cm}\begin{narratore}The list is an example to be able to randomly generate effects on the owner of the object. Take inspiration and be creative! However, don't let a curse make it impossible to play the character, rather it must be seen as an opportunity to try, do, something different. Never throw a cursed object at random into the treasure pile, always think about what could happen and what consequences it will generate. A cursed object always requires a high level of attention and planning on the part of the Storyteller\end{narratore}\end{changemargin}

\bigskip

\textbf{Table: Effects of cursed magic items}\index[Tables]{Table Effects of cursed magic items}

\medskip

{\small
\begin{tabularx}{0.95\textwidth}{lX}
\textbf{\%} & \textbf{Inconvenience}\\
\toprule
01-02& Character's hair grows 1 inch per hour.\\
02-04& The character's nails grow 1 cm every 8 hours\\
05-06 & The character's height decreases by 5d10 cm \\
07-09 & The character's height increases by 5d10 cm \\
10-11 & The temperature around the object is 5 C colder than normal.\\
12-13 & The temperature around the object is 20 C colder than normal.\\
14-15 & The temperature around the object is 5 C warmer than normal.\\
16-17 & The temperature around the object is 20 C warmer than normal.\\
18-20 & The character's hair color changes.\\
21-23 & The character's skin color changes.\\
24& The character's hair color changes every hour\\
25& The character's skin color changes every hour\\
26 & Horns like a ram grow on the character's head\\
27 & An antler like an elk grows on the character's head\\
28-29 & The character now bears a distinctive sign (a tattoo, a strange glow, etc.).\\
30-32 & The character's gender changes every day at dawn.\\
33-34 & The Character's race or species changes.\\
35 & The PC is struck by a randomly determined Disease, which cannot be cured.\\
36-39 & The object constantly makes unpleasant sounds (moans, curses, insults...).\\
40 & The object looks ridiculous (bright colors, shape, glows with a pink glow, etc.).\\
41 & A small blue unicorn, visible only with magic, always flies around the Character giving useless advice and making stupid jokes.\\
42& Every day you have a sudden desire and ability to crochet for at least 1 hour.\\
43-45 & The character becomes extremely possessive of the object.\\
46-49 & The character has an uncontrollable fear of losing the item or of it being damaged.\\
50 & A Trait is replaced\\
51& The character's metabolism changes and becomes exclusively carnivorous\\
52& The character's metabolism changes and becomes exclusively vegetarian\\
53-54 & The character must attack the creature closest to him (5\% chance every day).\\
55-57 & The character is Stunned for 1d4 rounds each time the object has served its purpose\\
58-60 & Character becomes deaf\\
61-64 & Maximum Hit Points drop by 10 permanently (remaining at a minimum of 1).\\
65 & Maximum Hit Points drop by 20 permanently (remaining at a minimum of 1).\\
66-68 & The PC gains a Phobia at random.\\
69-71 & TS on Volontà every day at dawn with mod. Intelligence or take 1 permanent Intelligence damage.\\
72-74 & Will save each day at dawn or take 1 permanent Wisdom damage.\\
75-77 & TS on Volontà every day at dawn with mod. Charisma or take 1 permanent Charisma damage.\\
78-80 & TS on Tempra every day at dawn with mod. Strength or take 1 permanent Strength damage.\\
81-83 & TS on Tempra every day at dawn with mod. Dexterity or take 1 permanent Dexterity damage.\\
84-85 & Fortitude save every day at dawn or take 1 permanent Constitution damage.\\
86-89& The PC begins to talk about himself in the third person.\\
90-92& Domestic horses, dogs, and cats become hostile.\\
93& A Patron will do anything to kill you.\\
94 & The PC is teleported 10d100 km away every day at dawn.\\
95 & The character is transformed into a random creature of a specific species (5\% chance every day).\\
96 & The character is transformed into a specific creature (5\% chance every day).\\
97 & The character can no longer use magic items or spells with levels above 5\\
98 & The character can no longer use magic items or spells with levels above 3\\
99 & The character can no longer use Spells\\
100 & Roll twice\\
\end{tabularx}}


\pagebreak

\section{Yeru}\index{Yeru}\index{Atilantis}

\begin{changemargin}{0.3cm}{0.3cm}\begin{enfasi}{
So the Earth is really round. But I didn't imagine it was blue. Why do the men who live on such a beautiful planet do nothing but fight among themselves? (Nadia - The mystery of the blue stone)

\medskip

The planet does not belong to us, we belong to it. We are passing through, he remains. (Pierre Rabhi)}\end{enfasi}\end{changemargin}\medskip

\begin{multicols}{2}

\label{yeru}

\lettrine[lines=2, lhang=0.33,loversize=0.25, findent=1.5em]{Y}{eru} is the reference planet of OBSS. A planet fractured both physically and magically.

Two stars Sparka and Andhakara rotate around Yeru.\index{Sparka}\index{Andhakara}

Sparka is of a warm golden color and is the one who brings heat and light, around her Yeru makes a complete revolution in 336 days of 24 hours each.

Sparka only ever illuminates the northern hemisphere of Yeru, called Curyan \index{Curyan}.

Andhakara always and only illuminates the southern hemisphere of Yeru, Tiya, and is instead a blue and cold star, devoid of life, it is the one that brings energetic storms and strange natural events. It brings a cold twilight.

If the 2pm (06-20) daylight hours see Sparka and Andhakara protagonists in their dance in the sky; the 10 hours of night see the two moons of Yeru named Idam and Kenatu as total protagonists.

The inhabitants of Yeru call them their moons even if in reality they are not just moons but actual inhabited planets.

The two moons are large and imposing on the night sky, Idam of a reddish gray color and Kenata of a warm mother-of-pearl gray control the tides and influence navigation with their presence.

Yeru has a peculiar and unique distribution of lands, the result of the whim of the Patrons of Genesis (Ljust and Calicante), you can imagine it as a mirror system on the equator.

The lands do not join at the equator, leaving about 200 km of open sea.

The emerged lands that make up the northern hemisphere and the southern hemisphere are almost symmetrical and symbiotic with each other. The shape and division of the lands are very similar to each other. But from a climatic point of view there are profound differences.

The open sea border area is wild and inscrutable. The deepest and most powerful storms continually discharge their energies and even magic cannot penetrate. In the eye of this perennial and gigantic maelstrom is the civilized and very powerful Alantia, considered by many to be a legendary island and cradle of civilization.

Many areas of Yeru are still unmapped and unexplored, primal chaos rules these areas and anything becomes possible.

There are few cities with more than 50,000 inhabitants. Each state has a capital which, due to the mocking fate of Yeru, is very often destroyed or disappears. The law is often absent and only that of the strongest prevails.

Wide lands unravel where ancient remains of disappeared civilizations are a refuge for new inhabitants. There are layers upon layers of historical civilizations beneath one's feet with treasures, secrets, caves and protectors.

Curyan is governed by the force of life, this region experiences a sort of perennial hot season with temperature gradations and atmospheric phenomena that vary depending on the latitude.

Areas with a torrid and humid climate intersect with others with dry heat and no precipitation; there are phenomena such as sandstorms in desert areas and strong and devastating tropical storms in the lush central bays.
There are pleasantly warm areas and others cooled by fresh breezes coming from the northern glaciers.

Tiya, on the other hand, is a semi-withered hemisphere, the light that comes in is barely enough to allow agriculture and the farmed animals have a pale and emaciated appearance.

The richest area is the one closest to the equator where the cold light of Andhakara only slightly attenuates, making room for some Sparka rays.
In this narrow band agriculture is more flourishing and there are fewer devastating meteorological phenomena.

It is the hemisphere where the law of the strongest applies, where people fight to live and there are few states that have an effective protection system.

The sea that embraces the equatorial is strong and tumultuous, very few boats venture from one continent to another, this means that exchanges between Tiya and Curyan are almost nil by sea, only very few captains, and secretly, dare to cross the maelstrom.

\subsection{Adventures in Yeru}\index{Adventures in Yeru}\label{avventureinyeru}

The \emph{problem} for adventurers and explorers is the extreme diversification and changeability not only of the environment but also of the cultures and civilizations that can be encountered.
Since every thousand years something very important changes it is possible that entire islands disappear or appear with once forgotten civilizations, entire cities are suddenly submerged by water or vegetation, or worse yet entire empires become belligerent with the undead (yes, it has happened this too, along with a couple of zombie apocalypses that lasted several centuries).

You are never sure what you can find in Yeru!.
Furthermore, the worst events are those that affect the rich and vital Curyan while more positive events take Tiya.

Yeru can never be said to be explored, the same area can change from one day to the next because a Patron decided so. Curious, ineffable, fickle, they are capable of building the adventure of life in a clap of hands just to enjoy the show.

Whether you are from Curyan or Tiya your life will not be easy, nothing will be given to you. Young people on both continents flee from poverty, abuse and violence to embark on a new life even richer in poverty, abuse and violence but which at least is only theirs, the fruit of their own choices.

When you decided to undertake this new career, or were torn from the previous one, you knew that it would not be easy, that Yeru himself, through his Patrons, would do everything to defeat and humiliate you, but the Law of Reward is superior even to Patrons and you would have had your prize even if what remained an unknown.

\medskip

\begin{changemargin}{0.3cm}{0.3cm}\begin{narratore}
Use the setting you prefer! Yeru is an example of a chaotic and slightly anarchic world dominated by the ever-changing moods of capricious deities.
Personally I prefer less high fantasy settings, but any setting from the Forgotten Realms to Golarion to Mystara will be fine. You are the Storyteller, you are the world, you are the ones projecting light and darkness!.

The first suggestion I give you is to know the setting well, the greater your knowledge, the more easily you will be able to adapt to the situations that happen to you.
\end{narratore}\end{changemargin}



\subsection{Notable places of Yeru}\label{luoghidiyeru}

\subsubsection{Kranguran Desert}

In this immense desert hide gigantic monsters. Some hidden under the sand like enormous dinosaurs use telluric sense to hunt their prey.

Every creature in this desert is gigantic, monstrous and disproportionate in appearance, as if born from someone's nightmare.

The vegetation itself in the few oases present is enormous and hypertrophic.

\subsubsection{Knandir City}

This rich, prosperous and populous ancient city was destroyed overnight by a gigantic cataclysm.

It is said that Cattalm's will was so pervasive that all buildings were destroyed or severely damaged. Not satisfied with the work, I condemn the city to be out of phase with reality, making it disappear in the eyes of everyone else.

The few surviving inhabitants perished in atrocious suffering, condemned to not being able to go out, not having anything to eat or drink.

The city was cursed and in the few days of the year in which it is possible to reach it, every person who sets foot there to plunder the immense treasures it contains seems condemned to never leave again, victim of the curse or of the numerous ghosts, spirits and undead of the previous inhabitants .

Furthermore, the city never appears in the same place but moves following a pattern that is not well understood. The lost Scroll of Knandir is said to explain his whereabouts.

\subsubsection{The Silent Sea}

There is a particular area of ​​sea, between three major islands and containing several smaller islands, where any sound is silenced. A sound that is generated in those waters, and not on dry land, is silenced. 2 distinct floating cities dedicated to ancient psionic lore have arisen.

\subsubsection{The tower of the blue gorillas}

The origin of this ancient and magical building is now forgotten, it is said that it was created to challenge a Patron, probably Gradh. The tower, with a square base measuring 20 meters on each side, is apparently 7 floors high. On each floor, whose map seems to be constantly changing, blue gorillas appear, absolutely brutal and with the intention of killing everyone in the tower. Once the last gorilla on the floor is defeated, the door leading to the stairs to the next floor opens and the characters can go up. With each floor the gorillas become stronger, more resistant and more intelligent. It is known that already on the 4th floor they also acquire magical powers. Entering characters can leave whenever they want, if they die inside the tower they will automatically be teleported out, but alive with 1 hit point and extremely tired, without the most precious object they had on them at the time of death. There are no magical objects inside the tower, at least on the known floors, the only thing the characters gain is experience for fighting. The current record was reaching the 7th floor. Will new heroes be able to reach the end (???) of the tower, and what rewards will there be for those who survive?


\pagebreak

\subsection{The Portals}\index{Portals}

\begin{changemargin}{0.3cm}{0.3cm}\begin{enfasi}{
Never open doors to those who open them even without your permission. (Stanislaw Jerzy Lec)}\end{enfasi}\end{changemargin}\medskip

\label{i-portali}

In a world where sea transfers only work between islands in the same hemisphere as does the Teleportation spell, the ability to use portals to transfer goods and people has gained significant traction.

This proliferation of small, large, long-lasting or instantaneous tunnels has caused a tear in the dimensional fabric of Yeru, in turn generating a proliferation of more or less large and long-lasting spontaneous tunnels.

And these Portals are the cause of many problems in both Tiya and Curyan as they not only bind the two hemispheres but connect all of Yeru to other worlds (or so we think since few have returned to report it...).

There are known and stable portals, so far, connecting Tiya to Curyan, almost all under the control, not to mention within the castle, of royalty or power.

There are areas where portals open more frequently but the destination is not always certain.

Then there are the dragon portals. The dragons are not native to Yeru but have been drawn to these magical gates, causing havoc and terror for Tiya and Curyan.

The dragons have well understood the nature of Yeru and with their fine intelligence and innate ability to shape magic they have built their portals by bringing hundreds of dragons. All evil.

Yes, there are no good dragons on Yeru with a few exceptions.

There has always been an attempt to destroy dragon portals, with sacrifice and blood. Many were destroyed, others were generated. It is an endless war, the only one that can unite the peoples and destinies of the two hemispheres.

\end{multicols}

\vfill

\begin{center}
\includegraphics[width=0.7\linewidth]{immagini/ancientwell.png}\\

\medskip

\emph{Ancient well and portal}
\end{center}


%Tahil red
%Elysan silver
%Curyan life
%Tiya dark

\pagebreak



\subsection{The Calendar}\index{Calendar}

\begin{changemargin}{0.3cm}{0.3cm}\begin{enfasi}{
I often ended up on a calendar. But never for a specific date. (Marilyn Monroe)\\

It all began on the thirteenth hour of the thirteenth day of the thirteenth month... We were there to discuss the printing errors in the calendars purchased by the school. (The Simpsons)} \end{enfasi}\end{changemargin}\medskip


\begin{multicols}{2}

\label{il-calendario}

Based on the Kenatu cycle it has 12 months of 28 days.

These are the names of the months starting from what is defined as the beginning of the year
\bigskip

1st) Ianas (season: spring)

2nd) Prineva (season: spring)

3rd) Marc (season: summer)

4th) Epral (season: summer)

5th) Meea (season: summer)

6th) Vernam (season: autumn)

7th) Ilai (season: autumn)

8th) Arkast (season: autumn)

9th) Cester (season: winter)

10th) Koper (season: winter)

11th) Narava (season: winter)

12th) Kartan (season: spring)

\bigskip
A seven-day cycle, week, is made up of named days

1st) Kalint (or Sparka's day, usually a holiday)

2nd) Iratam

3rd) Munrat

4th) Arai

5th) Venran

6th) Kittam

7th) Viltar

The day is divided into 24 hours

\subsection{Beyond death}

Yeruites have a fairly pessimistic view of what happens after death. For most, after death there is nothing except the dissolution of the body.

Devotees and Followers believe that their spirit will reunite with the Patron, making it stronger.

Others still believe that each spirit incarnates 4 times to then be judged by the Patrons of Genesis and sent to the plane assigned to him.

It is not known what the truth is.

\end{multicols}

\subsection{Millennial Cycles}\index{Millennial Cycles}

\begin{changemargin}{0.3cm}{0.3cm}\begin{enfasi}{
Then I saw an angel coming down from heaven with the key to the Abyss and a large chain in his hand.

He seized the dragon, the ancient serpent - that is, the devil, Satan - and chained him for a thousand years;

he threw him into the Abyss, locked him there and sealed the door over him, so that he would no longer deceive the nations, until the thousand years were completed. After these he will have to be untied for some time. (Revelation 20,1-3, apostle John)
}\end{enfasi}\end{changemargin}\medskip

\begin{multicols}{2}

The myth says that every thousand years the Yeru dies to be reborn again, more beautiful than before.

It's not exactly like that but it's very close.

It is known to a few scholars of Atmos that every thousand years the recognized Patrons and from which many draw their powers disappear and give way, after exactly 1 year, to new Patrons.

Suddenly spells stop working, only magical objects that can absorb and store magic work (such as a Potion, Armor or Weapon if not a Ring or a Staff that has charges, but not objects that recharge automatically like the Rods), even Devotees or Followers no longer have access to any Magic Lists.

With some exceptions. The Patrons of Genesis, Atmos and Lynx and the Victor are the only ones who remain constant and do not change and only their Devotees and Followers can continue to use the known Magic Lists.

Starting from the sixth month the old Followers and Devotees begin to hear voices, to dream of new faces and new Patrons.

Each new Patron, based on the Traits he commands, approaches a believer and tries to convince him to accept him as the new Patron.

This Follower/Devotee must have at least two Traits in common with the new Patron to be his Follower and at least 3 to remain Devoted (as always).

Spellcasters will only be able to use spells at the end of the year, regardless of whether they follow a Patron or not.

It is an extremely turbulent and agitated period where wars and revenge break out taking advantage of the absence of magic. For many it is a period of pure hatred and violence where the lowest instincts are vented knowing then that they will not be judged by any deity.

The truth is that every thousand years the Patrons of Genesis judge their creatures, the Patrons, evaluating who did better and who did worse. It is a challenge between Calicante and Ljust to see who has, through the Patrons, obtained the most Followers and Devotees.

The Patron who more than anyone has proven capable of maintaining and conquering people will also remain in the following millennium, this will be the Victor and believers will sing of his glory and power for another thousand years.

Intoxicated by the victory, the Patron of Genesis will express a wish that the other will have to try to respect as much as possible. Obviously he/she could show it themselves but the satisfaction of making the other person do something that he/she hates is superior to anything.

And this is why something happens every thousand years, in addition to the birth of new Patrons. It can be a continent, sea.. moon, animals... something massive changes for all Yeruites. It is a time of global upheaval.

Only the supreme Devotees of Atmos know this truth as they know that the Patrons of Genesis after the victory lie together for six months generating the new Patrons.

\bigskip

\begin{changemargin}{0.3cm}{0.3cm}\begin{narratore}
Evaluate carefully when to start your campaigns, based on the duration and year you could encounter these events.
Use the change in the Patrons to your advantage and benefit your adventure, let players play a bit of rest from magic, help players with more magical characters to recover.
\end{narratore}\end{changemargin}


\subsection{The Ancient Gods and the 7 Cycles}\index{The Ancient Gods and the 7 Cycles}\index{Ancient Gods}\index{7 Cycles}

Legend has it, practically unknown even to the most erudite, that 7 cycles ago the Patrons did not exist. They weren't always there to corrupt you, drag you into perdition, or unleash hordes of enemies upon you.

Once, the first time, the Patrons of Genesis entrusted to Atmos the \textbf{Book of the Gods}\index{Book of the Gods}, a tome of extraordinary power, capable of creating divinities.

Atmos wrote within them their Names, abilities, ethics and purpose. Thus 15 deities were born, some good, others indifferent, a few purely evil, with the aim of educating, instructing and improving the creatures of Yeru.

What exactly happened then is not known, some sages suppose that these gods also failed in the noble mission of raising Yeru from barbarism and incivility, others say that their contribution was not direct enough, in any case they were banished and replaced with the Patrons.

Certainly more direct, ruthless, interested in life on Yeru. They were not intended to improve life perhaps, but they undoubtedly gave the Patrons of Genesis much more enjoyment.

Legends say that the Book of the Gods was then destroyed, dispersed and only one page of that magnificent book was saved and whoever finds it and writes its name will be able to become a divinity, or perhaps a Patron slave to his impulses?


\end{multicols}


\vfill

\begin{center}
\includegraphics[width=0.4\linewidth]{immagini/Aztec_calendar.png}\\

\medskip

\emph{Ancient Aztec calendar}
\end{center}



\pagebreak

\section{The plans}\index{The plans}\label{ipiani}

\begin{multicols}{2}

\lettrine[lines=2, lhang=0.33, loversize=0.25, findent=1.5em]{A}{Although} endless adventures await you in Yeru, there are other worlds beyond this one, other continents, other planets, other galaxies . However, even beyond the existence of countless planets there are other worlds, dimensions completely different from reality, known as planes of existence. Traveling between one Plane and another is complex and each has its own rules.

Although the number of planes is limited only by the imagination, they can all be reduced to five general types: the Material Plane, the Transitional Planes, the Energy Planes, the Outer Planes, and the countless demiplanes.

\begin{changemargin}{0.3cm}{0.3cm}\begin{narratore} %box narrator
Consistent with the setting of OBSS the Plans should not be reachable. Lynx doesn't allow it. Yeru was born as a closed and isolated planet although this did not stop the Dragons and other fiends from arriving. The Storyteller decides Yeru's level of isolation.
\end{narratore}\end{changemargin}

\subsection{What is a Plan?}

You can imagine the Planes as gigantic spheres of indeterminate diameter that fluctuate as if they were planets in a cosmic \emph{empty} that is the Astral Plane.

For practicality, and only to help the limited human mind, Yeru (which corresponds to the Material Plane) is imagined as at the center, arranged in a star around it are the Energy Planes (the Elemental Planes and the Energy Planes), more distant External Planes. Among the various Plans there are those of Transition.

\emph{Material Plane}\index{Material Plane}: The Material Plane tends to be the most similar to Yeru and functions using the same natural rules. Its size depends on the campaign: it can conform only to the actual game world, or encompass an entire universe of planets, moons, stars and galaxies. The Material Plane is the basic plane for the game.

\emph{Transition Planes}\index{Transition Planes}: These planes have an important element in common: they all coexist with the other Planes and serve to travel between overlapping realities. These planes are strongly interconnected with the Material Plane, and can be accessed using numerous spells. They also have native inhabitants. Some Transition Plans are described below.

\begin{itemize}
\item
\emph{Astral Plane}\index{Astral Plane}: It is the void between the planes, a silver Plane that connects the Material Plane to the Energy Planes and the External Planes, the Astral Plane is the means through which the souls of deceased reach the afterlife. A traveler in the Astral Plane sees the plane as an infinite void periodically punctuated by tiny glimpses of physical reality detached from the countless overlapping planes. Powerful spellcasters use the Astral Plane for a brief fraction of a second when teleporting, or can use it to travel between planes with spells such as Astral Projection.

\item
\emph{Ethereal Plane}\index{Ethereal Plane}: The Ethereal Plane is a nebulous and hidden dimension overlapping with the Material Plane and the Plane of Shadows. Travelers traversing the Ethereal Plane experience the real world as if it were insubstantial and can move between solid objects without being seen in the real world. Bizarre creatures inhabit the Ethereal Plane, as well as ghosts and dreams, many of which can sometimes extend their influence into the real world in mysterious and terrifying ways. Powerful spellcasters use the Ethereal Plane with spells such as Ethereal Form, Intermittent.

\item
\emph{Plane of Shadows}\index{Plane of Shadows}: The mysterious and deadly Plane of Shadows is a grey, colorless version of the Material Plane. It overlaps with the Material Plane and is in many ways a distorted and perverse reflection of the Material Plane, infused with negative energy (see Energy Planes) and inhabited by terrible monsters such as shadows or even worse creatures. Powerful spellcasters use the Plane of Shadow to quickly travel immense distances on the Material Plane.
\end{itemize}

\medskip

\emph{Energy Planes}\index{Energy Planes}: These planes are the embodiments of the basic elements that build the universe. They are composed of a single type of energy or element. The inhabitants of a specific Inner Plane themselves are composed of the plane's element. Among the Energy Plans there are:

\medskip

\begin{itemize}
\item
Elemental Planes\index{Elemental Planes}: The four classic Inner Planes are the Plane of Water, the Plane of Air, the Plane of Fire and the Plane of Earth. From these planes come the creatures known as elementals, but they are also inhabited by other bizarre creatures, such as genies, xorns, mephits, and invisible stalkers.

\item
Planes of Energy\index{Planes of Energy}: There are two planes of energy, The Plane of Positive Energy (from which the vital sparks come) and the Plane of Negative Energy (from which the corruption of undeath comes). The energy of both planes is infused into reality, and the flow of this energy flows through all creatures from birth to death. Devotees use the power of these two planes when Channeling Positive or Negative Energy. As easy as it is to think that Ljust is from the Plane of Positive Energy and Calicante from the Plane of Negative Energy, this is not the case since both are a single source of energy that transcends the Planes.

\end{itemize}

%\medskip
%\end{multicols}
%\pagebreak

%\vfill
%\begin{center}
%\includegraphics[width=0.96\linewidth]{immagini/mappaplanare3.png}\\
%\medskip
%\emph{Planar Map. Licensed by the author.\\ https://www.reddit.com/r/ImaginaryGolarion/comments/97rog0/pathfinder\_map\_to\_the\_planes}
%\end{center}
%\pagebreak

%\begin{multicols}{2}

\emph{External Planes}\index{External Planes}: Vast beyond imagination, it is to them that the souls of the dead reach and it is here that the Patrons dwell. Each of them has its own set of Traits, which represents a particular moral or ethical aspect, and their inhabitants tend to behave following these Traits.

The Outer Planes are also the final resting place of spirits from the Material Plane, whether destined for a peaceful continuation or eternal damnation. The inhabitants of the Outer Planes form the mythologies of civilizations, including angels and demons, titans and devils, and countless other incarnations of the possible. Each game world should have different Outer Planes that conform to specific themes and needs, but the classic Outer Planes include Heaven (lawful and good traits), the Abyss (anarchic and evil traits), Hell (lawful and evil) and the Elysée (freedom and goodness). Powerful spellcasters can contact the Outer Planes for guidance and counsel with spells such as Commune and Contact Other Planes, or they can summon allies with Conjuration spells.

\emph{Half-Planes}\index{Half-Planes}\index{Planes Semi-Planes}: This category is used to collect all the other extradimensional spaces that function like planes but which have measurable and limited access and dimensions. Other types of planes are theoretically infinite in size, but a half-plane could be only a few hundred meters long. There are countless demiplanes adrift in the astral plane, and while many are connected to the Astral Plane and the Ethereal Plane, others are cut off entirely from the Transitional Planes and are only reachable through well-hidden portals or dark magic. A demiplane usually combines the aspects and characteristics of multiple Planes.

\subsection{Journey between the Planes}
Two planes that are separate from each other do not overlap or connect directly to each other. They are like planets in different orbits. The only way to move from one plane to another is to pass through a third plane, such as a Transition Plane.

\emph{Adjacent Planes}: Those floors that connect to each other at specific points are considered adjacent. Where they touch, there is a connection through which travelers can exit one reality and enter the other. Usually demiplanes can serve as a connecting portal.

\subsection{Planar Features}\index{Planar Features}
Each plane of existence has its own peculiarities; the natural laws of his universe. Planar features are divided into general areas. All plans have the following features.

\emph{Physical Characteristics}: Determine the physical and natural laws of the plane, including how gravity and time work.

\emph{Elemental and Energetic Characteristics}: The influence of elemental and energetic forces is determined by these characteristics.

\emph{Traits}: Just as characters can have Traits, so many planes are tied to a particular moral or ethic.

\emph{Magical Features}: Magic works differently from plane to plane; magical characteristics demarcate the boundary between what magic can and cannot do on each plane.

\emph{Physical Characteristics}

The two most important natural laws determined by physical traits concern the function of gravity and time. Other physical characteristics concern the size and shape of a plane and the way in which its nature can be altered.\\

\textbf{Gravity}\\

The direction of gravitational pull may be unusual, and may even change directions within the same plane.\\

\textbf{Time}\\

The pace at which time passes can vary across different planes, although it remains constant within any specific plane. Time is always subjective to the viewer. The same subjectivity applies to the various plans. Travelers may find that they are gaining or losing time by moving between floors, but from their perspective, time passes naturally.


\emph{Normal Time}: Defines the passage of time on the Material Plane. One hour on a plane characterized by normal time is equivalent to 1 hour on the Material Plane. Unless otherwise specified in the description of a plane, it is assumed that it features normal weather.
\emph{Irregular Time}: Some planes feature slowing and accelerating time, so an individual may lose or gain time while moving between planes like this and others. For the inhabitants of such a plane, time passes naturally and movement goes unnoticed.


\emph{Time Flowing}: On some planes, the flow of time is considerably faster or slower. Someone could travel to another plane, spend a year there, and then return to the Material Plane to find that only 6 seconds have passed. Anything on the floor you returned to lived just a few seconds longer. For the traveler and the objects, spells, and effects at work on him, that year of distance was completely real. When designing how time works on planes with flowing time, think about the time flow of the Material Plane first, then the flow of time on the other plane.


\emph{Absence of Time}: On planes with this characteristic, time passes but its effects are limited. How timelessness affects certain activities and conditions such as hunger, thirst, aging, poison effects, and healing varies from plane to plane. The danger of a timeless plane is that when an individual leaves that plane to come to another where time flows normally, conditions such as starvation and aging occur retroactively. If a plane has timelessness related to magic, any spell cast with a non-instantaneous duration becomes permanent until dispelled.\\


\textbf{Elemental and Energy Characteristics}\\

Four basic elements and two types of energy combine to shape everything; the elements are water, air, fire and earth; the types of energy are positive and negative. The Material Plane reflects a balance of these elements and energies: it is possible to find them all. Each of the Inner Planes is dominated by an element or type of energy. Many planes do not have any elemental or energetic characteristics; these characteristics are specified in the description of a plan only if present.

\emph{Dominant Water}: Planes with this characteristic are mostly liquid. Visitors who cannot breathe underwater or who cannot reach an air pocket are likely to drown. Fire creatures are extremely uncomfortable on water-dominant planes. These creatures, made of fire, take 1d10 damage each round.

\emph{Dominant Air}: Consisting essentially of open space, floors with this characteristic host just a few pieces of floating stone or other solid material. They usually have a breathable atmosphere, although such a plane may have clouds of acidic or toxic gas. Creatures of the Earth subtype are uncomfortable on planes with dominant air given the small amount or absence of natural earth with which to come into contact. However, they do not suffer any actual damage.

\emph{Dominant Land}: Planes with this characteristic are mostly solid. Travelers who reach it are at risk of suffocation unless they reach a cave or other crevice. Worse yet, individuals without the Burrow ability become trapped underground and must dig their own way out (3 feet per round).
Creatures of the Air subtype are uncomfortable on earth-dominant planes as they consider them cramped and claustrophobic, but aside from having difficulty moving, they do not encounter any other inconveniences.

\emph{Dominant Fire}: Planes with this characteristic are made up of flames that burn continuously without exhausting their fuel source. Planes with dominant fire are extremely hostile to creatures of the Material Plane, and those without Resistance or Immunity to fire are quickly incinerated. Wood, paper, unprotected fabric and other flammable materials catch fire almost instantly, as do those wearing unprotected and flammable clothing. In addition, individuals take 3d10 fire damage each round they remain on a fire-dominant plane. Creatures of the Water subtype are extremely uncomfortable on fire-dominant planes. These creatures, made of water, take double damage each round.

\emph{Dominant Negative Energy}: Planes with this characteristic are vast, empty recesses that suck the life essence of travelers who pass through them. They tend to be desert-like, tormented planes, stripped of color and filled with winds carrying the faint moans of those who perished within them. There are two types of traits based on dominant negative energy: lesser and greater dominant negative energy. In the former, living creatures take 1d6 points of damage per round. At 0 hit points or less, these are reduced to ash.

The latter are even more dangerous. Each round, those inside must make a Fortitude save with DC 25, the maximum hit points are reduced by 6. If they die in this way they become a Wraith. The Death Ward spell protects the traveler from the damage and energy drain of a plane with dominant negative energy.

\emph{Dominant Positive Energy}: The abundance of life characterizes the planes that present this characteristic. As with the planes with dominant negative energy, the planes with dominant positive energy can also be lesser and greater.
A plane with minor dominant positive energy is a tumultuous explosion of life in all its forms. The colors are brighter, the fires hotter, the noises louder and the sensations more intense thanks to the positive energy spread across the plane. All individuals on a plane with dominant positive energy regenerate 2 hit points per round.\\

\textbf{Traits}\\

Some planes have a predisposition towards a specific set of Traits. The inhabitants of these planes mostly share this set of Traits or parts of them. A plane's set of Traits influences its social interactions. Characters who have Traits different from those of most inhabitants may have difficulty when dealing with the plane's natives and situations. Traits have multiple components. First of all there are the moral and ethical components. Secondly, there may be a specific indication of whether this set of Traits manifests itself in a moderate or more marked way. Many planes have no Traits; the latter are specified in the description of a plan only if present

In general, the elemental, astral and ethereal planes do not have any Traits.\\

\textbf{Magic Features}\\

The magical characteristics of a plane defines the magic on that plane relative to the Material Plane. Particular locations on a plane (such as those under the direct control of deities) may have a different magical feature apply.


\emph{Normal Magic}: This magical characteristic means that all spells and supernatural abilities work as described. Unless otherwise described, a plane is assumed to have the normal magic trait.

\emph{Dead Magic}: Marks planes where magic does not exist at all. A plane with the dead magic characteristic functions in all respects as an Antimagic Field. Divination spells cannot detect someone who is on a plane of dead magic, nor can a spellcaster use the Teleport spell to move in and out of one. The only exception to the no-magic rule is permanent planar portals, which still function normally.

\emph{Enhanced Magic}: On planes with this magical feature, particular spells and spell-like abilities are easier to use or produce more powerful effects than they operate on the Material Plane. Natives of a plane are aware of which spells and spell-like abilities are enhanced, but planar travelers may discover this on their own. If a spell is empowered, it functions as if it had rolled a magic critical on the Magic Test.

\emph{Hindered Magic}: Particular spells and spell-like abilities are more difficult to use on planes with this characteristic, often because the nature of the plane hinders them. To cast a thwarted spell he must roll a critical on the Magic Test. If the check fails, the spell has no effect but is still wasted. If the check succeeds, the spell takes effect normally.

\emph{Limited Magic}: Planes with this feature only allow the use of spells and spell-like abilities that meet particular requirements. Magic can be limited in its effects by certain schools or subschools, by effects with certain descriptors, or by effects of a given level (or by any combination of these aspects). Spells and spell-like abilities that don't meet the requirements simply have no effect.

\emph{Wild Magic}: On a plane with the wild magic characteristic, spells and spell-like abilities work in a totally different and sometimes dangerous way. There is a chance that any spell or spell-like ability used on a wild magic plane will have no effect. When the caster casts a spell he must make two checks, if even one fails it means something unusual happens; roll a d100 and consult the

\medskip

\end{multicols}

\textbf{Table: Effects of Wild Magic}\index[Tables]{Table Effects of Wild Magic}

\medskip

\begin{xltabular}{0.95\textwidth}{lX}
d100&Effect\\
\toprule
01-19&The spell bounces back to the caster with normal effect. If the spell cannot affect the caster, it has no effect.\\
20-23&A circular pit 3 meters in diameter opens beneath the caster's feet; its depth is 10 feet per caster's Magical Expertise.\\
24-27&The spell has no effect, but the target or targets of the spell are hit by a shower of small objects (anything from flowers to rancid fruit), which disappear as soon as they hit. The attack continues for 1 round. During this time, targets are blinded and must make a magic check to cast spells.\\
28-31&The spell affects a random target or area. Randomly choose a different target from among those within the spell's range or center the spell in a random location within that range. To randomly generate direction, roll 1d8 and count clockwise, starting from the south. To randomly generate the range, roll 3d6. Multiply the result by 1 meter for short-range spells, 6 meters for medium-range spells, and 24 meters for long-range spells.\\
32-35&The spell functions normally, but any material components are not consumed and no Magic Points are used. Likewise, an item doesn't lose charges, and the effect doesn't affect the use limit of an item or spell-like ability.\\
36-39&The spell has no effect. Instead, someone (friend or foe) within 30 feet of you receives the effect of a Heal spell.\\
40-43&The spell has no effect. Instead, Deep Darkness and Silence effects cover a 30-foot radius around you for 2d4 rounds.\\
44-47&The spell has no effect. Instead, a Gravity Reversal effect covers a 30-foot radius around you for 1 round.\\
48-51&The spell takes effect, but shimmering colors swirl around the caster for 1d4 rounds. Treat this area as a Glitter Dust effect with a saving throw DC 10 + the level of the spell that caused the result.\\
52-59&Nothing happens. The spell has no effect. Any material component is used. The spell or spell slot is used, an item loses charges, and the effect affects the usage limit of an item or spell-like ability.\\
60-71&Nothing happens. The spell has no effect. Any material component is not used. The spell does not disappear from the caster's mind (a spell slot or prepared spell can still be used). An item does not lose charges, and the effect does not affect the use limit of an item or spell-like ability.\\
72-98&The spell takes effect normally.\\
99-100&The spell has enhanced effect. The Magic Test automatically generates a critical\\
\end{xltabular}

%\addvspace{2cm}

\begin{multicols}{2}


\subsection{The Plans}

\subsubsection{Material Plan}\index{Material Plan}\label{pianomateriale}
The Material Plane is the core of most cosmologies and defines what is considered normal. This is the plane on which most campaigns focus.\\
The Material Plane has the following features:\\
\emph{Normal Gravity}\\
\emph{Normal Time}\\
\emph{No Elemental or Energetic Traits}: However, specific locations may exhibit such traits.\\
\emph{Moderately Neutral}: Although in some places it may have high concentrations of evil or good, law or chaos Traits.\\
Normal Magic\\

\subsubsection{Plane of Shadows}\index{Plane of Shadows}\label{pianoombre}
The Shadow Plane is a dimly lit dimension that simultaneously coincides and coexists with the Material Plane. It overlaps the Material Plane as much as the Ethereal Plane, so the planar traveler can exploit the Plane of Shadow to cover great distances quickly. The Shadow Plane is a black and white world: the environment is devoid of color. If it weren't for this, it would resemble the Material Plane. Despite the absence of light sources, some plants, animals, and humanoids call the Shadow Plane their home.

The Shadow Plane has the following characteristics:

\emph{Imperfect Geography}: Parts of the Shadow Plane continually flow into other planes. Therefore, despite the presence of reference points, creating a precise map of the plane is almost impossible.

\emph{Traits}: Undisciplined, Free, Superficial, Vengeful, Pessimistic

\emph{Enhanced Magic}: Spells that work with shadow are enhanced on the Shadow Plane. Despite the dark nature of the Plane of Shadow, spells that generate, use, or manipulate darkness are unaffected by the plane.

\emph{Magic Hindered}: Light spells or spells that use or generate light or fire are hindered on the Shadow Plane. Spells that generate light are less effective in general, since all light sources on this plane have half their range.


\subsubsection{Negative Energy Plan}\index{Negative Energy Plan}\label{pianoenergianegativa}
For an observer there is very little to see on the Negative Energy Plane. It is a dark and empty place, an infinite tomb into which the traveler could fall until the plane has erased light and life. The Plane of Negative Energy is the most hostile of the Internal Planes, the most indifferent and intolerant towards life. Only creatures immune to its drain effects can survive here.

The Negative Energy Plan has the following characteristics:

\emph{Greater Dominant Negative Energy}: Creatures other than undead take 10 Hit Points of Void damage per round. Upon death you become a Wraith.

In zones of Lesser Dominant Negative Energy, creatures other than undead take 2 hit points of void damage per round.

\emph{Enhanced Magic}: Spells and magical abilities that use negative energy are enhanced. Feats that harness negative energy, such as channel negative energy, gain a +4 bonus to the saving throw DC to resist the ability.

\emph{Magic Hindered}: Spells and spell-like abilities that use positive energy (including healing spells) are hindered. Characters on this plane must pass a Magic Critical Test to cast spells that cure or remove negative effects.


\subsubsection{Positive Energy Plan}\index{Positive Energy Plan}\label{pianoenergiapositiva}
The Plane of Positive Energy has no surface and is similar to the Plane of Air with its totally open space. However, every corner of this plane is brightly lit by an innate power. This power is dangerous for mortal forms, not predisposed to suffer it. Despite its beneficial effects, it is one of the most hostile Inner Planes. A character without defenses will overflow with power as soon as positive energy is channeled into him. But since his mortal form is unable to contain such power, he will be incinerated, like a speck of dust caught on the edge of a supernova. Visits to the Positive Energy Plane are short-lived, and even then travelers must be adequately protected.
The Positive Energy Plan has the following characteristics:

\emph{Greater Dominant Positive Energy}: Every 10 rounds you are affected by the Greater Restoration spell. 10 Hit Points are regenerated per round, once the maximum Hit Points are reached, 10 temporary Hit Points are gained per round, when the temporary Hit Points reach double the maximum Hit Points the creature explodes in colored energy.

In minor dominant positive energy zones, you are affected by the lesser restoration spell every 10 rounds. 2 Hit Points are regenerated per round, once at most you take 2 temporary Hit Points per round, when the temporary Hit Points reach double the maximum Hit Points the creature explodes in colored energy.

\emph{Enhanced Magic}: Spells and magical abilities that use positive energy are enhanced. Feats that harness positive energy, such as Channel Positive Energy, gain a +4 bonus to DC to resist the ability.

\emph{Magic Hindered}: Spells and spell-like abilities that use negative energy (including spells you inflict) are hampered.

\subsubsection{Elemental Plane of Water}\index{Elemental Plane of Water}\label{pianoacqua}
The Water Plane is a sea without a bottom or surface, a liquid environment illuminated by diffused light. It is one of the most hospitable Inner Planes, once the traveler overcomes the problem of breathing underwater. The infinite oceans of this plane range between freezing cold and scorching heat, and between saltwater and freshwater. The plane's permanent settlements spawn around pieces of shipwreck suspended in this endless fluid, drifting with the tides.

The Water Plan has the following characteristics:

\emph{Dominant Water}

\emph{Enhanced Magic}: Spells and spell-like abilities that use spells or effects from the Water Elemental List or Water Outsiders are enhanced.

\emph{Magic Hindered}: Spells and spell-like abilities that use spells or effects from the Fire Elemental List or Outsiders of the Fire subtype are hampered.


\subsubsection{Elemental Plane of Air}\index{Elemental Plane of Air}\label{pianoaria}
The Plane of Air is an empty plane, consisting of sky in every direction. It is the most comfortable and livable of the inner planes and is home to all kinds of creatures of the air. In fact, flying creatures gain great advantage on this plane. While travelers can survive well here even without the ability to fly, they are still at a disadvantage.
The Air Plan has the following characteristics:

\emph{Dominant Air}

\emph{Enhanced Magic}: Spells and spell-like abilities that use spells or effects from the Air Elemental List or Air Outsiders are enhanced.

\emph{Magic Hindered}: Spells and spell-like abilities that use spells or effects from the Air Elemental List or Outsiders of the Air subtype are hampered.


\subsubsection{Elemental Plane of Fire}\index{Elemental Plane of Fire}\label{pianofuoco}
On the Plane of Fire everything is illuminated. The ground is nothing more than vast, shifting layers of compressed fire. The air is moved by the heat of the continuous rains of fire and the most common liquid is magma. The oceans are composed of liquid flame and the mountains flow molten lava. Here the fire persists without power or air, but the flammable elements introduced onto the surface are quickly consumed.

The Fire Plane has the following characteristics:

\emph{Dominant Fire}

\emph{Greater Fire Dominance}: Each round you take 10 hit points of unresistible fire damage if you are not immune to fire.

\emph{Lesser Fire Dominant}: Each round you take 2 Hit Points of fire damage.

\emph{Enhanced Magic}: Spells and spell-like abilities that use spells or effects from the Fire Elemental List or Fire Outsiders are enhanced.

\emph{Magic Hindered}: Spells and spell-like abilities that use spells or effects from the Water Elemental List or Outsiders of the Water subtype are hampered.

\subsubsection{Elemental Plane of Earth}\index{Elemental Plane of Earth}\label{pianoterra}
The Earth Plane is a solid place composed of earth and stone. An imprudent traveler could find himself buried by this vast solid mass: his pulverized remains will remain a warning to those who dare to follow him. Despite its solid and rigid nature, the Plane of Earth has variable consistency, ranging from soft soil to veins of harder, more precious metal.

The Plane of the Earth has the following characteristics:

\emph{Dominant Land}

\emph{Enhanced Magic}: Spells and spell-like abilities that use spells or effects from the Earth Elemental List or Earth Outsiders are enhanced.

\emph{Magic Hindered}: Spells and spell-like abilities that use spells or effects from the Air Elemental List or Outsiders of the Air subtype are hampered.

\subsubsection{Ethereal Plan}\index{Ethereal Plan}\label{pianoetereo}
The Ethereal Plane coexists with the Material Plane and often with other planes as well. The Material Plane itself is visible from the Ethereal Plane, but appears silent and indistinct; the colors blend together and the boundaries are blurred. While it is possible to see the Material Plane from the Ethereal Plane, the Ethereal Plane is usually invisible to those on the Material Plane. Normally, creatures from the Ethereal Plane cannot attack those from the Material Plane, and vice versa. A traveler on the Ethereal Plane is invisible, incorporeal, and totally silent to someone on the Material Plane.

The Ethereal Plane has the following characteristics:

\emph{No Gravity}

\emph{Normal Magic}: Spells function normally on the Ethereal Plane, even if they do not cross the Material Plane. The only exceptions are spells and spell-like abilities that affect ethereal entities.

No magical attacks pass from the Ethereal Plane to the Material Plane, including force attacks.


\subsubsection{Astral Plane}\index{Astral Plane}\label{pianoastrale}
The Astral Plane is the space between all planes. When a character passes through a portal or projects his spirit to another plane of existence, he travels through the Astral Plane. Even spells that allow instantaneous movement across a plane affect the Astral Plane slightly. The latter is a great endless expanse of clear silver sky, both above and below. Occasional bits of solid matter can be found here, but most of the Astral Plane is open, boundless space.

The Astral Plane has the following characteristics:

\emph{Absence of Time}: Age, hunger, thirst, suffering (such as Diseases, Curses and Poisons) and natural healing have no effects in the Astral Plane, although they resume their functioning when the traveler leaves the floor.

\emph{Enhanced Magic}: All spells and spell-like abilities used in the Astral Plane have a speed of 1 action. Spells and spell-like abilities that are already hasted are unaffected, as are the enchantments of magic items. Spells sped up in this way are still prepared and cast at their original level.

\subsubsection{Vacuus}\index{Vacuus Plan}\label{pianovuoto}
The plane of the Void. Desolate wastes beneath a putrid sky, Vacuus is shrouded in a sickening black fog, and the oppressive twilight of an endless solar eclipse. The mortal Styx is born in Vacuus, before entering like a twisted serpent into the other planes. Vacuus is one of the most hostile Outer Planes: kingdom of the Caridion, fiends of pure evil indifferent to the conflict between law and chaos, who represent oblivion and destruction. The Caridions, ruled by four Archcharidions with patron-like powers, are feared as devourers of souls.

Vacuus has the following characteristics:

\emph{Traits}: Destructive, Relentless, Insatiable, Irrational, Wrathful, Sadistic

\emph{Enhanced Magic}: Evil spells and magical abilities are enhanced.

\emph{Magic Hindered}: Good spells and benevolent spell-like abilities are hampered.

\subsubsection{Abyss}\index{Plane of the Abyss}\label{pianoabisso}
The Abyss, a multi-layered plane is made up of gigantic canyons and gorges that gape in the fabric of the Outer Planes and is bordered by the nefarious waters of the River Styx. The infinite layers of the Abyss, bordering all the Outer Planes, are connected to each other by constantly shifting paths. In the Abyss there are no rules, laws, order or hope. The Abyss represents the corruption of freedom, a nightmarish realm of absolute horror where desire and suffering take demonic form, a land of proliferation of countless races of Demons, among the most ancient beings in all creation. It is said that if the being at the deepest layer ever decided to wake up, all the planes would cease to exist.

The Abyss has the following characteristics:

\emph{Traits}: Anarchist, Vengeful, Touchy, Arrogant, Double Agent

\emph{Strongly Chaotic and Strongly Evil}

\emph{Enhanced Magic}: Chaotic or evil spells and spell-like abilities are enhanced.

\emph{Magic Impaired}: Spells and lawful or good spell-like abilities are hindered.


\subsubsection{Elysée}\index{Elysée Floor}\label{pianoeliseo}
A vast land of pristine wilderness, Elisha is the plane of benevolent chaos, freedom and independence, personified in the native Yazata. In the Elysée, selfless cooperation and fierce competition clash violently, but such conflicts never overshadow the noble concepts of courage, creativity and good unhindered by rules or laws.

Eliseo has the following characteristics:

\emph{Traits}: Good, Charitable, Anarchic, Innovative, Competitive

\emph{Enhanced Magic}: Chaotic or good spells and spell-like abilities are enhanced.

\emph{Magic Hindered}: Lawful or evil spells and spell-like abilities are hindered.


\subsubsection{Hell}\index{Plane of Hell}\label{pianoinferno}
The nine layers of Hell form a structured labyrinth of premeditated evil where torment goes hand in hand with purification. Plane of iron cities, burning wastes, frozen glaciers and endless volcanic peaks, Hell is divided into nine concentric layers, each under the cruel rule of an archdevil. Torture, anguish, and suffering are inevitable in Hell, but they are meted out methodically, not out of spite or whim, and support a planned design under the watchful eyes of Hell's disciplined ranks of lesser devils. The nine layers of Hell are, from top to bottom, Avernus, Dis, Erebus, Phlegethon, Stygia, Malebolge, Cocytus, Caina and Nessus.

Hell has the following characteristics:

\emph{Traits}: Evil, Disciplined, Wrathful, Sadistic, Arrogant

\emph{Strongly Lawful and Strongly Evil}

\emph{Enhanced Magic}: Lawful or evil spells and spell-like abilities are enhanced.

\emph{Magic Hindered}: Chaotic or good spells and spell-like abilities are hindered.

\subsubsection{Nirvana}\index{Plan of Nirvana}\label{pianonirvana}
Nirvana is an impartial paradise existing between the two extremes of Elisha and Paradise. Its wondrous mountains, hills and dense forests meet the traveler's expectations of a pastoral paradise, but Nirvana also contains mysteries that lead to enlightenment. Nirvana is a sanctuary and resting place for all who seek redemption or enlightenment. The Agathos natives of Nirvana have willingly set aside their transcendence to guard the enigmas of the plane, while the celestials battle the forces of evil present between the planes.

Nirvana has the following characteristics:

\emph{Traits}: Good, Kind, Calm, Simple, Confident

\emph{Enhanced Magic}: Spells and good magical abilities are enhanced.

\emph{Magic Hindered}: Spells and evil spell-like abilities are hampered.


\subsubsection{Paradise}\index{Piano del Paradiso}\label{pianoparadiso}
The Plane of Heaven is an orderly realm of honor and compassion divided into seven layers. The slopes of Paradise are full of orderly and well-structured cities and clean and well-kept gardens and orchards. Although they began their lives as mortals, the native Archons of Heaven see law and good as two inseparable halves of the same supreme concept and stand against the cosmic corruptions of chaos and evil.

Paradise has the following characteristics:

\emph{Traits}: Good, Rigid, Combative, Practical. Sincere, Valorous

\emph{Enhanced Magic}: Lawful or good spells and spell-like abilities are enhanced.

\emph{Magic Hindered}: Chaotic or evil spells and spell-like abilities are hampered.


\subsubsection{Purgatory}\index{Plan of Purgatory}\label{pianopurgatorio}
Each soul passes through Purgatory to be judged before being sent to its ultimate destination. Vast cemeteries and wastelands fill its dark expanses, along with dusty, echoing courts of justice for the dead. Purgatory is the home of the Aeons, a race that embodies the dualistic nature of existence and whose members are constantly at war and at peace with each other and with themselves.

Purgatory has the following characteristics:

\emph{Absence of Time}: Age, hunger, thirst, suffering (such as Diseases, Curses and Poisons) and natural healing have no effect in Purgatory, although they resume their functioning when the traveler leaves the plane .

\emph{Enhanced Magic}: Spells and spell-like abilities that affect death or rest are enhanced.


\subsubsection{Utopia}\index{Plan of Utopia}\label{pianoutopia}
Utopia is a stronghold of order pitted against anarchy and the endless demonic hordes of the Abyss. It is a great city of eternal perfection, whose streets and buildings are models of architecture and aesthetics: everything is in order and nothing happens by chance. Although Utopia is not ruled by any race, the Ordinax make it their home, constantly seeking to expand their perfect city.

Utopia has the following characteristics:

\emph{Traits}: Rigid, Disciplined, Serious, Direct, Cold

\emph{Enhanced Magic}: Lawful spells and spell-like abilities are enhanced.

\emph{Magic Hindered}: Spells and chaotic spell-like abilities are hampered.


\subsubsection{Genesis Plan}\index{Genesis Plan}\label{pianogenesi}

Tradition has it that the Patrons of Genesis are on a plane at the borders of everything and within everything. How this \emph{place} is is not known to any mortal. Legends, purely fantastic stories, tell of an environment of pure divine energy capable of materializing every thought.

No one can get to the Plane of Genesis, if it ever exists, without an explicit invitation from Calicante, Ljust, or Atmos. The plan is also forbidden to Patrons except Lynx.


\end{multicols}

\pagebreak

\section{OBSS monstruary}\index{Monstruario}

\begin{changemargin}{0.3cm}{0.3cm}\begin{enfasi}{Those who fight with monsters must be careful not to become a monster by doing so. And if you gaze long into an abyss, the abyss will also gaze into you. (Friedrich Nietzsche)

\medskip

Monsters can only be defeated by their own kind. (Claymore)

\medskip

The tragedy of monsters is that they are too big and powerful to be accepted by mankind. (Ishiro Honda)}\end{enfasi}\end{changemargin}\medskip

\begin{multicols}{2}

\lettrine[lines=2, lhang=0.33, loversize=0.25, findent=1.5em]{B}{welcome} to a universe full of adversaries, often evil, other times violent, also devious, also intelligent, perhaps petty and almost always gigantic.. and whatever else you want. Monsters are the cornerstone of any fantasy role-playing game.

Monsters are explained and presented here, certainly not all of them, much less exhaustive, use them to populate the adventures of your companions with nightmares.

\medskip

\begin{center}

\includegraphics[width=0.8\linewidth]{immagini/sangiorgioedrago.png}

\emph{Saint George and the Dragon (circa 1460) by Paolo Uccello. National Gallery London}
\end{center}

\subsection{Introduction}

An adventure is not just a set of adversaries but of situations, places, surprises, in short everything that can fascinate, involve, amaze and engage the players. But monsters are also useful. Hitting has a cathartic, liberating aspect.

Insert difficult and dangerous monsters into the adventure where needed but every now and then, rarely, make the players feel powerful, make them face monsters that they can solve in just a few rounds. Describe combat by emphasizing hits, critical hits, pain, and monster blood. Make people understand how powerful the characters can be.

Other times make the monsters inspire fear because they are big, hungry, magical and evil, it is necessary that the players are afraid for their characters, that they never take victory for granted.

The opponent's strength lies in the confidence in describing the situation, in a few words, in staring the players in the eyes. Get the players involved and once they have your attention the characters will be more attentive too. Try to put monsters that are consistent with the environment, the adventure, the situation. Don't randomly roll on tables, a well organized clash is much more satisfying than random shows that \emph{spawn}.

Don't reduce everything to an MMORG where the goal is just to kill everything and everyone, there can always be plenty of choices if you put in a little effort.

\begin{changemargin}{0.3cm}{0.3cm}\begin{tcolorbox}[title = Facing Monsters]
{
Let this old man give you a couple of tips young adventurer!

- Not all enemies can be defeated with a sword, many times a club is also needed!

- Sometimes weapons and brute force are not enough. If you don't have companions who can cast spells make sure you always have the ability to start a fire.

- Escape. It's always a valid option if you have the opportunity and you see that the situation doesn't bode well.

- Organized! Don't enter the dungeon with your head down and never stop except when you're dead! Rest, explore, check out your environment and when you're safe and feeling better, move on! even your enemies organize themselves and rest in the meantime, be careful!

- Sometimes you can also talk to enemies, they also don't always want to die.

- If you have to kill, do it with malice and speed. Don't waste time and optimize your shots, save your energy and immediately prepare for another fight.

}\end{tcolorbox}\end{changemargin}

\subsection{Editing Creatures}

Despite the colorful collection of encounters in this book, you may still find yourself in a quandary when it comes to finding the perfect creature for your adventure. Feel free to modify existing creatures and transform them into something that is more useful to you, perhaps borrowing a feature or two from a different monster.

Keep in mind that modifying an opponent may change their challenge rating.

\subsection{Size and Hit Dice}

A monster can be Tiny, Small, Medium, Large, Huge, or Gargantuan and Colossal in size. The Size Categories table shows a creature's average size, how much space it takes up on the grid, and what hit points you use to determine its hit points.

If not indicated, the reach of a creature depends on the size and the weapon used (think of a gigantic greatsword wielded by a titan..)

\end{multicols}

\textbf{Table: Size Categories, Occupied Squares and Range}\index[Tables]{Table Size Categories, Occupied Squares and Range}\index{Range for Creatures}\index{Squares for Creatures}€15191 €{Size and squares}\index{Creatures per square}\label{tagliaedimensioni}\hypertarget{size and dimensions}{} 

\medskip

\begin{tabularx}{0.95\textwidth}{llllll}
\toprule
\textbf{Size}& \textbf{Dimension} & \textbf{Example}&\textbf{Squares}&\textbf{Hit Dice}&\textbf{Range}\\
Tiny & 25 x 25 cm&Cat, sprite& 1/4&d4&0m\\
Small & 0.5 x 0.5 m & Goblin, dog, Gnome&1/2&d6&1m\\
Medium & 1 x 1m & Orc, Human, Elf, Dwarf, Nibali &1&d8&1m\\
Large & 2 x 2 m& Ogre&2x2&d10&1m\\
Huge & 3 x 3 m & Giant, Ent&3x3&d12&2m\\
Gargantuan & 4 x 4 m&Kraken, Dragon&4x4&3d6&2m\\
Colossal & 12 x 12 m&Elder Dragon, Tarrasque&6x6&2d12&6m\\
\end{tabularx}


\medskip

The more experienced will have noticed that the dimensions of the creatures are smaller than usual, this is because the miniatures on the market are made for a scale of 1 square = 1.5 metres, while in OBSS 1 square = 1 metre.

\begin{multicols}{2}

\subsection{Type}

A monster's type refers to its basic nature. Certain spells, magic items, abilities, and other game effects interact in special ways with creatures of a specific type. For example, a \emph{dragon slaying arrow} deals extra damage not only to dragons but also to all other dragon-type creatures, such as turtle dragons and wyverns.

The game includes the following types of monsters:

\smallskip\textbf{Aberrations}, totally alien creatures. Many possess innate magical abilities that tap into the creature's alien mind rather than the mystical forces of the world. Classic examples of aberrations are abolets, watchers, mind flayers, and chaos batrachians.

\smallskip\textbf{Beasts}, non-humanoid creatures that are a natural component of a fantasy world. Some possess magical powers, but most lack intelligence and have no form of society or language. Classic examples of beasts are all common animal species, dinosaurs, and giant versions of animals.

\smallskip\textbf{Celestials}, creatures native to the Upper Planes. Many of them are servants of the deities, employed as messengers or agents in the mortal world and for the planes.\\
Celestials are good in nature, classic examples of celestials are angels, couatl and pegasi.

\smallskip\textbf{Constructs}, are created and not born. Some are programmed by their creators to follow a simple set of instructions, while others are sentient and capable of thinking on their own. Golems are the most representative constructs.

\smallskip\textbf{Dragons}, are large reptilian creatures of ancient origin and enormous power. True dragons, including the good dragons of Ljust and the evil dragons of Tàhil, are highly intelligent and possess innate magical gifts. This category also includes creatures distantly related to true dragons, but less powerful, less intelligent and less magical, such as wyverns and pseudo-dragons.

\smallskip\textbf{Elementals}, are creatures native to the elemental planes. Some creatures of this type are little more than animated masses of their respective element, and include creatures simply called elementals. Other creatures possess biological forms infused with elemental energy. Genie races, including djinn and efreet, form the major civilizations of the elemental planes. Other elemental creatures include azers, invisible stalkers, and water oddities.

\smallskip\textbf{Fairies}, are magical creatures closely linked to the forces of nature. They live in hidden clearings and foggy forests. Examples of fey are dryads, pixies, fairies and satyrs and La Topi.

\smallskip\textbf{Giants}, tower over humans and their peers. They are human in shape, although some have multiple heads (ettins) or deformities (fomorians). The six variants of true giants are hill giant, stone giant, frost giant, fire giant, cloud giant, and storm giant. Beyond these, ogres and trolls are also giants.

\smallskip\textbf{Fiends, Demons, Devils}, perverse creatures native to the Lower Planes. Some serve deities, but many more operate under archdevils and demon princes. Sometimes evil priests and spellcasters summon fiends into the material world to do their bidding. If an evil celestial is a rarity, a good fiend is practically inconceivable. Fiends include demons, devils, hellhounds, and rakshasas.

\smallskip\textbf{Oozes}, are gelatinous creatures that rarely have a fixed shape. They live mainly underground, settling in caves and dungeons, feeding on waste, carcasses or creatures unfortunate enough to stumble upon them. Black slimes and gelatinous cubes are among the most recognizable slimes.

\smallskip\textbf{Monstrosities}, are monsters in the strictest sense of the term fearful creatures that are not common, nor truly natural, and almost never benign. Some are the result of magical experiments gone wrong, while others are the product of terrible curses (among which we remember the minotaur). They defy categorization, and in some ways serve as an all-encompassing category for those creatures that don't correspond to any other type of monster.

\smallskip\textbf{Undead}, are once-living creatures brought to a horrible state of undeath through the practice of necromantic magic or some blasphemous curse. The undead include walking corpses, such as vampires and zombies, or incorporeal spirits, such as ghosts and specters. Some more intelligent undead speak Exspiran, a language made of dark whispers.

\smallskip\textbf{Plants}, in this context these are plant creatures, not normal flora. Most of them are mobile, and some are carnivorous. The most classic example of plants are walking mounds and ent. Fungioid creatures such as gas spores and myconids also fall into this category.

\begin{center}
\includegraphics[width=0.60\linewidth]{immagini/sanmichelesatana.png}\\
\emph{Saint Michael defeats Satan. Raphael and his assistants (1518). Louvre Museum}
\end{center}

\smallskip\textbf{Humanoids}, are the main population of the game worlds, civilized and savage, including humans and a wide range of other species. They have a language and culture, little or no innate magical ability (though many humanoids can learn spells), and a bipedal form. The most common races of humanoid are those best suited as player characters: humans, dwarves, elves, and nibals, various. Almost as numerous, but more brutal and savage, and almost all evil, are the goblinoid races (goblins, hobgoblins and bugbears), orcs, gnolls, lizardfolk and kobolds.\\

\medskip

These categories can in turn be grouped into types of Creatures:
\smallskip
\begin{itemize}
\item
The \textbf{Natural Creatures}: they are Insects, Reptiles, Beasts, Humanoids, Plants, Aquatic Creatures, Monstrosities, Slimes
\item
The \textbf{Magical Creatures} are: Fiends, Demons, Devils, Fairies, Spirits, Undead, Giants, Celestials, Constructs, Aberrations (everything that is alien or unnatural) and Dragons.

If a Natural Creature has magical powers then it is also considered a Magical Creature.
\end{itemize}


\medskip\textbf{Labels}

A monster may have one or more tags indicated in parentheses, following its type. For example, an orc has the type \emph{humanoid (orc)}. The labels in parentheses provide additional categorizations for certain creatures. Labels do not have their own specific rules, but some game elements, such as magic items, can refer to them. For example, a spear that is particularly effective against demons would work against any monster that has the demon tag.

\subsection{Trades}

The monsters do not have a detailed list of Traits, you will only find indications on the axes of Chaos, Law, Good and Evil. Remember that they are indications, exceptions can occur especially in the more intelligent species.
Certain creatures are \textbf{misaligned}, that is, they do not have moral or ethical conduct.

\subsection{Defense}

A monster wearing armor or carrying a shield has a Defense that takes into account armor, shield, and Dexterity. Otherwise, a monster's Defense is based on its Dexterity value and natural armor if it has it (the "\emph{hide}"). If a monster has natural armor, wears armor, or carries a shield, it is indicated in parentheses after its Defense value.

If the monster is \textbf{caught by surprise} subtract -4 from Defense.

\subsection{Hit Points}

Usually when a monster drops to 0 hit points, it dies or is considered dead.

A monster's hit points are presented both as a pool of dice and as an average value. For example, a monster with 2d8 hit points has an average of 9 hit points (2 x 4.5).

It may happen that players ask you \textbf{\emph{how the monster is doing}}, I suggest you never go into detail by saying how many hit points it has in total or how many it has lost, but rather stay in these grades: Not wounded ( Full Hit Points), Wounded (30\% Hit Points suffered), Seriously wounded (at least 50\% Hit Points suffered), or give a generic description of the state. \index{How is the monster}\index{Ask the monster for hit points}

A monster's Constitution value also influences the number of Hit Points it has. His Constitution value is multiplied by the number of Hit Dice he has and the result is added to his Hit Points. For example, a monster that has Constitution 1 and 2d8 Hit Dice, and will therefore have 2d8+2 Hit Points (average 11).

\subsubsection{Angry}\index{Angry}\index{Bloodied}\label{mostroarrabbiato}

At the Storyteller's discretion, a creature that has lost at least 50\% of its total Hit Points accesses a \emph{anger reserve} which allows it to perform particular actions.
Monsters with a Challenge Rating of 5 or higher can have a \textbf{Angry} card. The Angry Feat can be used once per fight.

Particularly ferocious and powerful creatures could have more notes than Arrabbiato and both, respecting any conditions marked, can be activated.

\subsubsection{Optional - Everyone Angry}\index{Optional - Everyone Angry}
For greater aggression you can make the monster get +1d6 to attack rolls or to damage or saving throws depending on the type of creature when it drops below half hit points.\index{Bloody}\index{Bloodied} \index{Angry}\\
You can also decide that the creature negates a condition it has on itself.


\begin{center}
\includegraphics[width=0.65\linewidth]{immagini/roc.png}\\
\emph{Henry Justice Ford}
\end{center}

\subsection{Movement}

A monster's Movement tells you how much it can move during its round per Move Action

All creatures have a walking movement, simply called monster movement. Creatures that do not possess any form of terrain movement have a movement speed of 0 feet.

Some creatures have one or more of the following additional movement modes.

\smallskip\textbf{Swimming}

A monster that has a swim speed does not have to spend extra movement to swim (it is not difficult terrain)

\smallskip\textbf{Climb}

A monster that has a climb speed can use all or part of its movement to move up vertical surfaces. The monster does not have to spend extra movement (x4) to climb.

\smallskip\textbf{Excavation}

A monster that has a burrow speed can use its speed to pass through sand, dirt, mud, etc. A monster cannot burrow through solid rock unless it has a special trait that allows it to do so.

\smallskip\textbf{Flight}

A monster that has a flying speed can use all or part of its movement to fly. Some monsters have the \textbf{floating} ability, which makes them difficult to take down. The monster stops floating when it dies.

\subsection{Ability Scores}

Each monster has six ability scores (Strength, Dexterity, Constitution, Intelligence, Wisdom, Charisma)

\subsection{Skills}\index{Monster Weapons Proficiency}

The Skills item is reserved for those monsters that are capable of one or more skills. For example, a monster that is very observant and stealthy might have bonuses on Awareness and Dexterity checks.

Other modifiers may also apply, for example, a monster might have a larger bonus than expected to account for its great expertise.

If not indicated, but necessary for tests (not for attack rolls, where the value already marked is used), a monster's Weapon Proficiency is equal to its Challenge Rating.

\subsection{Vulnerabilities, Resistances and Immunities}\index{Weapon equivalencies}\index{Magic Fists}\label{vulnerabilitaresistenze}
Some creatures have vulnerabilities, resistances, or immunities to certain types of damage. Particular creatures are even resistant or immune to nonmagical attacks (a magical attack is an attack made via a spell, a magical item or weapon, or another source of magic).

It is also possible that a specific minimum magical bonus is indicated in order to damage the creature. In the case of creatures immune to critical attacks, this applies to both spells and weapons, the damage explosion remains effective. \index{Monster Critic}

Additionally, certain creatures are immune to certain conditions. If a monster is immune to a game effect that isn't considered damage or a condition, it instead has a special trait.

The table below indicates which magical weapon enchantment is necessary to overcome the indicated immunity. The minimum Weapons Proficiency score is also indicated in case you hit with kicks and punches.

In the case of a character with a Weapons List \textbf{Empty Fist}, you check how many times the list has been taken.

\medskip

\textbf{Table: Equivalence of Magical Weapons}\index[Tables]{Table of Equivalence of Magical Weapons}\label{equivalenzaarmimagiche}\hypertarget{equivalence of magical weapons}{}

\medskip

\begin{tabular}{lp{0.055\textwidth}p{0.06\textwidth}p{0.07\textwidth}}
\toprule
\textbf{Immunity} & \textbf{Magic Weapon} & \textbf{Nat. Attack. (CA)}& \textbf{Empty Fist}\\
+1 & +1 & 3& 2\\
+2 & +2 & 6& 4\\
Cold Iron & +1 & 4& 2\\
Silver & +1 & 4& 2\\
Adamantium & +2 & 6& 4\\
+3 & +3 & 12& 8\\
+4 & +4 & 16& 12\\
+5 & +5 & - & 16\\
\end{tabular}

\subsection{Sensi}

The Senses entry lists any special senses the monster has. The special senses are described below. If the Senses entry is not present, the creature has standard senses (vision, smell, taste, touch...).

If not specified, a monster's Awareness is equal to its Challenge Rating/2 + Wisdom.

\subsubsection{Telluric Perception}

A monster with tremor sensing can detect and find the origins of vibrations within a specific radius, as long as the monster and the source of the vibration are in contact with the same terrain or substance. Earthsense cannot be used to detect flying or incorporeal creatures. Many burrowing creatures, such as ankhegs and earth behemoths, possess this special sense.

\subsubsection{Light Vision or Darkvision}

A creature with low-light vision can see in the dimmest of lights, but not in complete darkness unlike those with darkvision. Many creatures that live underground possess this special sense. See chapter \hyperlink{visioneluce}{Special Features}.

\subsubsection{Vision of the Truth}

A monster with true seeing can, up to a specified range, see through normal and magical darkness, see invisible creatures and objects, automatically detect illusions and succeed on saving throws against them, sense the original form of a shapeshifter or a creature transformed by magic. Additionally, the creature can see into the Ethereal Plane up to the same range.

\subsubsection{Blind View}

A creature with blindsight can perceive its surroundings, without relying on sight, up to a specific range.

Eyeless creatures such as grimlocks and oozes, and creatures with echolocation or enhanced senses, such as bats and dragons, possess this sense.

If a monster is naturally blind, this is noted in parentheses, in which case the range of its blindsight also defines the maximum range of its perception.

\begin{center}
\includegraphics[width=0.65\linewidth]{immagini/ciclope.png}

\emph{Henry Justice Ford}
\end{center}

\subsection{Languages}

The languages ​​a monster can speak are listed in alphabetical order. If a monster can understand a language but cannot speak it, this is indicated in this entry. If a monster does not have the Languages ​​note, it means that it does not know languages ​​other than its own language (if applicable).

\subsection{Telepathy}

Telepathy is an ability that allows a monster to mentally communicate with another creature within specified range. The contacted creature does not need to speak the same language as the monster to communicate in this way. A creature without telepathy can receive and respond to telepathic messages but cannot start or end a telepathic conversation.

A telepathic monster does not need to see the creature being contacted and can end telepathic contact at any time. The contact is broken as soon as the two creatures are no longer within range or if the telepathic monster contacts another creature within range. A telepathic monster can begin or end a telepathic conversation without having to use an action, but while the monster is incapacitated it cannot initiate telepathic contact, and any ongoing contact is terminated. To initiate telepathic communication the target must at least have been identified.

A creature in the area of ​​a \emph{anti-magic field} or anywhere else where magic doesn't work can send or receive telepathic messages.

\subsection{Challenge}

A monster's \textbf{challenge rating} (CR) tells you how great a threat it poses. A properly equipped and rested party of four adventurers must be able to defeat a monster of a challenge rating equal to their average level without suffering casualties. For example, a party of four 3rd-level characters would find a challenge rating 3 monster a normal, non-dangerous challenge.

Monsters that are significantly weaker than 1st-level characters have a challenge rating of less than 1. Monsters with a challenge rating of 0 present no problems except in large numbers; those without real attacks are not worth experience points.

Some monsters present a challenge beyond what even a 20th-level party can handle. These monsters have a challenge rating of 21 or higher and are designed to test the abilities of the strongest characters.

\subsection{Recognize Monsters}\label{riconoscereimostri}\hypertarget{Recognize Monsters}{} \index{Recognize Monsters}

Knowing how to recognize a monster can be extremely useful and is something that should never be underestimated.

For \textbf{recognize a monster} you make a Knowledge check. (\textbf{1 Share}) on:\\

\emph{Arcana}: Giants, Constructs, Spirits, Monstrosities, Aberrations, Dragons

\emph{Planes}: Elementals

\emph{Occult}: Fiends (Devils and Demons), Spirits, Undead

\emph{Religion}: Spirits, Undead, Celestials

\emph{Dungeon}: Aberrations, Monstrosities, Oozes, and Subterranean Creatures

\emph{Nature}: Beasts, Plants, Fairies\\

The DC of the check is equal to 10 + the creature's challenge rating + rarity/notoriety factor (common (0), uncommon (+1), rare (+2), very rare (+4), legendary +(8) ).

The information obtainable depends on the degree of success achieved. \\

- \emph{within 2}: name, type, main feature\\
- \emph{up to 7}: which is the best saving throw, one resistance/immunity to Conditions, one vulnerability to Conditions, typical attack\\
- \emph{up to 12}: which is the worst Saving Throw, one resistance/immunity to Conditions, one immunity to Damage, one vulnerability to Conditions, one vulnerability to Damage type\\
- \emph{up to 15}: two Condition immunities, one Damage immunity, one Condition vulnerability, one Damage type vulnerability\\
- \emph{up to 17}: relative level of challenge or whether it is an easy, medium, high, extraordinary, deadly or epic battle \\
- \emph{up to 20}: attack and special defenses\\

The information obtained is cumulative, meaning if the check succeeds by 15 you get the information within 2, 7 and 12.

\subsection{Special Offers}

Special traits (which appear after a monster's challenge rating but before any actions or reactions) are quirks that will likely play a role in a combat encounter and that require explanation.


\begin{center}
\includegraphics[width=0.7\linewidth]{immagini/lich2.png}

\emph{Lich - Battle of Wesnoth}

\end{center}


\subsection{Spells}

A monster with the Spells ability is able to cast Spells.

The \textbf{DC is 10 + spell level x2 + Intelligence or Wisdom depending on the best or indicated characteristic}. A monster does not need to make Magic Checks unless it has a Magical Expertise value (e.g. Lich, Mummy, Naga...). If Weapon Proficiency needs to be calculated for using spells and it is not specified, then it is equal to half the Magical Proficiency score.

\subsection{Innate Spells}

A monster with the innate ability to cast spells has the Spellcasting special trait.
A monster's innate spells cannot be exchanged for other spells.

\subsection{Shares}

Monsters also act according to the scheme of 3 Actions available per round. Skills and abilities can be marked that allow him to perform a higher number of Actions.

When a monster carries out its actions, it can choose from the options in the Actions section of its stat block or use one of the Actions available to all creatures, such as Dash or Hide, unless otherwise indicated use an Action (not part of Multiattack) costs 2 Actions.

\subsubsection{Melee and Ranged Attacks}

The most common action a monster will take in combat will be a melee or ranged attack. These can be spell attacks or weapon attacks, where the weapon can be an artifact or a natural weapon, such as claws or a spiked tail.

\emph{\textbf{Creature vs. Target}.} The target of a melee or ranged attack is usually a creature or target.

\textbf{Range}: the indicated range is the distance \textbf{within} how many meters the creature can hit the opponent. Even if the range is greater than that of the opponent, advantages such as Long Weapon (+2 to TC) are not considered with natural attacks. A creature with 0 reach must be on you to hit you, extremely small creatures usually have 0 reach.

\emph{\textbf{Hits.}} Any damage dealt or other effect that occurs as a result of an attack that hits the target is described in the notation \emph{Hits}. You can choose to take the average damage or roll the dice; for this reason both the average damage and a dice formula are presented.

I suggest that the \textbf{Critical Shot} is still applicable for enemies while the €15340{Damage Explosion} is to be used if you want a more lethal campaign.\index{Monster Critical Damage}

\textbf{\emph{Missing}.} If an attack has a miss effect, that information is provided by the notation ``\emph{Missing}''.

\emph{\textbf{Damage.}} If a monster wields crafted weapons, it deals appropriate damage to the weapon. Larger monsters usually wield larger weapons that deal extra damage when they hit. If they use this type of weapon, the damage is already marked, otherwise if they pick up or use an unexpected weapon, double the weapon dice if the creature is Large, triple them if Huge and quadruple them if Mammoth if they use weapons of their size.

A creature has -1d6 on attack rolls with a weapon made one size larger than itself. The Storyteller may decide that weapons two or more sizes larger than the attacker's are completely impossible to use.

\begin{changemargin}{0.3cm}{0.3cm}\begin{narratore}
A creature that has at least Challenge Rating 6 at the Storyteller's discretion can launch an attack of Opportunity at the cost of 1 Reaction (see \hyperlink{opportunist}{Opportunist} page €15348{opportunist}).\\

To enhance the monsters and make them more effective you can decide that each monster has a \hyperlink{damage reduction}{Damage Reduction} equal to half its Challenge Rating ($\frac{GS}{2}$/-)

\end{narratore}\end{changemargin}

\subsubsection{Multi-attack}

The Multiattack Action consumes 2 Actions even if it brings more than 2 attacks. Each attack follows the rules of \hyperlink{multiplemelee attacks}{multiattack} (page \pageref{multiplemelee attacks}).

Other Attack Actions listed under Multiattack but not part of those listed in the Multiattack description cost 1 Action.\index{Multiattack}

\subsubsection{Grapple Rules for Monsters}

Many monsters have a special attack that allows them to quickly grab prey. When a monster hits with such an attack, it does not need to make an additional check to determine whether the grapple succeeds, unless the attack says otherwise.

A creature grabbed by the monster can use two Actions to attempt to escape it. To do so, she must succeed on an opposed Strength check (Fortitude save with Strength bonus) against the escape DC in the monster's stat block. If no escape DC is given, assume the DC equals 10 + Fortitude save + monster's Strength +1d6 per Size difference.

\subsubsection{Ammunition}

A monster carries enough ammunition with it to make its ranged attacks. You can assume that a monster has 2d4 projectiles for a thrown weapon attack (javelins, boulders...), and 2d10 projectiles for a projectile weapon such as a bow or crossbow.

\subsubsection{Reactions}

If a monster can do something special with its reactions, it is listed here. If a creature has no special reactions, this section is absent.


\subsubsection{Limited Use}

Some special abilities have restrictions on the number of times they can be used.

\textbf{\emph{X/Day}.} The notation ``X/Day'' indicates a special ability that can be used X times before dawn to recover expended uses. For example, ``1/Day'' indicates a special ability that can be used once before the monster must wait for the new dawn.

\emph{\textbf{Recharge X-Y.}} The notation ``Recharge of combat. At the start of each monster's round, roll a d6. If the result is one of the cooldown numbers, the monster regains use of the special ability. The ability also recharges at the dawn of a new day.

%\begin{center}
%\includegraphics[width=0.6\linewidth]{immagini/cupido.png}
%
% € 15362 € {Eros with his bow. Capitoline Museums}
%\end{center}

For example, \emph{Recharge 5-6} indicates that a monster can use its special ability once. Then, at the start of the monster's round, it regains use of the ability if it rolls 5 or 6 on a d6.

\begin{center}
\includegraphics[width=0.55\linewidth]{immagini/polpo.png}

\emph{Alphonse de Neuville - Hetzel edition of 20000 Lieues Sous les Mers}
\end{center}


\subsection{Equipment}

The stat block refers to equipment, beyond the weapons or armor used by the monster. A creature that normally wears clothing, such as a humanoid, is assumed to be appropriately dressed.

You can equip monsters with additional equipment as you like, using the chapter \hyperlink{equipment}{Equipment} as a source of inspiration, you decide how much of the monster's equipment is recoverable after the creature is slain or if any part of your equipment is still usable. For example, dented armor made for one monster is unlikely to be usable by anyone else. If a spellcasting monster requires material components to cast its spells, assume that it has the material components to cast spells in its stat block.

\subsection{Additional Shares}

Certain creatures can perform special actions outside of their own round, and some can extend their power to the environment, causing extraordinary magical effects to manifest in their vicinity.

A creature with additional actions can take a certain number of special actions, called \emph{additional actions}, outside of its round. Only one additional action can be used at a time, and only at the end of another creature's round. It costs no Actions or Reactions to use an Additional Action. A creature with additional actions regains the additional actions it used at the start of its round. She is not forced to use her additional actions and cannot use additional actions while incapacitated or otherwise unable to perform Reactions. If surprised, she cannot use them until after her first round of combat.

If a creature takes the form of a creature with additional actions, perhaps through a spell, it doesn't gain the additional actions or lair actions.

\subsubsection{A Creature's Lair}\index{A Creature's Lair}\index{Lair Actions}

A creature with additional Actions may have a section describing its lair and the special effects it can create there while it is there, either by its own volition or simply by its presence. This section applies only to legendary creatures that spend a lot of time in their lairs where they are highly likely to be encountered.

If a creature with additional actions has a \textbf{Lair Action}, it can use it to harness the ambient magic of its lair. On initiative count 10, losing ties, the creature can use one of its lair action options. You may not do so while incapacitated or otherwise unable to perform actions. If surprised, she cannot use it until after her first round of combat.

\subsection{Types of Treasure}

Each type of creature can prefer a different type of treasure (intended as objects, coins, gems...). These are just suggestions on how to build the monster's treasure.

See also \hyperlink{valoretesorointerno}{Table: Treasury Values ​​per Incontro} (page \pageref{valoretesorointerno}).

\medskip

\begin{itemize}

\item \textbf{Aberration}
Many aberrations have little regard for treasure, possessing only what they take from the remains of their previous victims. Others are cunning adversaries who use various magical items and treasures to enhance their abilities.

\item \textbf{Animal}
The animals do not care for treasures at all, instead leaving coins and objects with the remains of their meals. For those with treasure, it is typically found in their lairs, scattered among bones and other scraps.

\item \textbf{Magical Beast}
Caring little for values, most magical beasts are solely in search of their next meal. The hideouts of these creatures are often littered with precious trinkets and magical items.

\item \textbf{Constructed}
The only treasure carried by constructs is generally part of the construct itself, such as a weapon or magic item. Constructs, however, are typically used to guard treasures or more valuable magical items.

\item \textbf{Dragon}
Known for their precious treasures, dragons often mull over piles of coins, gems, magic items, and other expensive items.

\item \textbf{External}
Outsiders are among the most varied types of creatures and as a result could have really any kind of treasure on them or hidden in their shelters. The Storyteller should consider the individual creature when determining the type of treasure that best suits that exterior.

\item \textbf{Folletto}
Above all else, fey value beautiful and magical objects. They have little regard for the instruments of exchange and commerce used by more civilized races, such as coins and securities.

\item \textbf{Slime}
Oozes have no concept of such things as treasure and leave behind everything they find in their search for the next meal. Whatever treasure they may carry is completely accidental.

\item \textbf{Undead}
The treasures carried by the undead vary depending on whether or not it is an intelligent creature. Mindless undead typically have only the meager valuables they carried with them in life, rarely truly usable as treasure, while intelligent ones exploit a vast array of magical items to destroy the living.

\item \textbf{Parasite}
Like other mindless creatures, parasites do not covet treasure, although these creatures are sometimes found haunting areas where valuables are stored.

\item \textbf{Humanoid}
Creatures of this type are very varied, but even the most primitive humanoids use magical equipment and items to some extent. In larger groups, such as communities, humanoids often have large amounts of treasure that they guard collectively.

\item \textbf{Vegetable}
Like animals, plant creatures place no value on treasure, and anything that might be found where they grow simply represents the undigested remains of a previous victim.

\end{itemize}

\subsection{Experience Points for GS}

Each monster if \emph{defeated} grants a certain amount of Experience Points to be divided among all the participants in the battle.

This table indicates the relative Experience Points for GS.

\medskip


\textbf{Table: Level of Challenge and Experience Points}\index[Tables]{Table of Experience Points by Level of Challenge}

\medskip

\begin{tabularx}{0.42\textwidth}{ll|ll|ll}

\textbf{GS} & \textbf{PX} &\textbf{GS} & \textbf{PX} &\textbf{GS} & \textbf{PX}\\
0 & 10 & 9 & 5000 & 21&33000\\
1/8 & 25 & 10 & 5900 & 22&41000\\
1/4 & 50 & 11 & 7200 & 23&50000\\
1/2 & 100 & 12 & 8400 & 24&62000\\
1 & 200 & 13 & 10000 & 25&75000\\
2 & 450& 14 & 11500 & 26&90000\\
3 & 700& 15 & 13000 & 27&105000\\
4 & 1100& 16 & 15000 & 28&120000\\
5 & ​​1800& 17 & 18000 & 29&135000\\
6 & 2300& 18 & 20000 & 30&155000\\
7 & 2900& 19 & 22000 & &\\
8 & 3900& 20 & 25000 & &\\
\end{tabularx}

\subsubsection{Optional - Experience per Challenge}\index{Optional - Experience per Challenge}

With this system, Experience Points are given based on the relative difficulty of the Challenge given the level of the characters. A fight with 5 Trolls will not give (1800 x 5) Experience Points, but depending on the relative challenge it will grant a different amount.

The group of Trolls (Challenge 5, 1800 XP) does not always give 1800 XP to a defeated troll; if faced by a low level group, i.e. for an Extraordinary difficulty challenge, it will give more while faced by a high level group, where 5 trolls are a High challenge, it will give less.

With this system, every 1000 Experience Points you level up. All the considerations in the Mastering chapter to prepare for battles apply.

\medskip

\textbf{Table: Experience Points by Level of Challenge}\index[Tables]{Table of Experience Points by Level of Challenge}

\begin{tabular}{ll|ll}
\textbf{Challenge Rating} & \textbf{PX}&\textbf{Challenge Rating} & \textbf{PX}\\
\toprule
Easy & 20 & Medium & 30\\
High & 50 & Extraordinary & 80\\
Deadly & 120 & Epic & 160\\
\end{tabular}

\medskip

This system is also used to calculate the XP earned for traps or challenges that have been overcome. The Experience Points reward for each personal or group goal achieved are 10.


\end{multicols}

\vfill


\begin{changemargin}{0.3cm}{0.3cm}\begin{enfasi}{
I am the monster that breathing men would long to slay. I am Dracula. (Dracula, Bram Stoker)}\end{enfasi}\end{changemargin}


\pagebreak

\subsection{The Monsters}

\begin{changemargin}{0.3cm}{0.3cm}\begin{narratore}
The creatures presented here are intended to be a full-bodied example of the adversaries your characters might encounter. Be careful, it doesn't mean that they are all enemies or that they necessarily have negative intentions.

More civilized creatures will have their own individual ethical and moral conduct, even within the same group of adversaries there are those who could be more hostile or simply indifferent.

Take advantage of the peculiarities and uniqueness of the creatures to create unexpected and tactically challenging encounters. Don't be obvious but not absurd in your choices, there must always be consistency in choosing creatures.
\end{narratore}\end{changemargin}


\bigskip

\begin{changemargin}{0.3cm}{0.3cm}\begin{enfasi}{

Amon Goth: Control is power. This is power.

Oskar Schindler: Is this why they fear us?

Amon Goth: We have the power to kill. This is why they fear us.

Oskar Schindler: They fear us because we have the power to kill arbitrarily. A man commits a crime, he should have thought about it, we get him killed and we feel at peace. Or we kill it ourselves and feel even better. This is not power though! This is justice, it is something different from power. Power is when we have every justification to kill and we don't.

Amon Goth: Is this the power?

Oskar Schindler: The emperors had this. A man steals something, is brought before the emperor and drops to the ground trembling, begs for mercy. He is aware that he is about to leave. And the emperor forgives him instead. That man, undeserving, leaves him free.

(Schindler's list, Film, 1993)
}\end{enfasi}\end{changemargin}\medskip


\medskip

\begin{multicols}{2}



\medskip\index[Monstery]{Aboleth}\textbf{Aboleth}

\emph{Great Aberration, Lawful Evil}

\textbf{STRENGTH} +5

\textbf{DEXTERITY} -1

\textbf{CONSTITUTION} +2

\textbf{INTELLIGENCE} +4

\textbf{WISDOM} +2

\textbf{CHARRISMA} +4

\textbf{Initiative} +4 -- \textbf{Defense} 22

\textbf{Hit Points} 135 (18d10 + 36)

\textbf{Movement} 3m, swim 12m

\textbf{Saving Throws} Fortitude +8, Reflexes +5, Will +11

\textbf{Skills} Awareness +10, History +12

\textbf{Senses} darkvision 36 m

\textbf{Languages} Language of the Depths, telepathy 36 m

\textbf{Challenge} 10 (5900 PX)

\emph{\textbf{Amphibian.}} The aboleth can breathe air and water.

€15452 € {€15453 € {Mucus Cloud.}} While he is underwater, the aboleth is enveloped in mutant mucus. A creature that comes into contact with the aboleth, or hits it with a melee attack while within 3 feet of it, must make a DC 14 Fortitude saving throw. On a failed save, the creature is sickened for 1d4 hours . The diseased creature can only breathe underwater.

\emph{\textbf{Telepathic Probe.}} If a creature communicates telepathically with the aboleth, and the aboleth can see it, the aboleth learns its greatest desires.

\textbf{Shares}

\emph{\textbf{Multiattack.}} The aboleth makes three tentacle attacks

\emph{\textbf{Tentacle.} Melee weapon attack}: +9 to hit, reach 10 ft., one target.

\emph{Hit}: 12 (2d6 + 5) bludgeoning damage. If the target is a creature, it must succeed on a DC 14 Fortitude save or become sickened. The disease has no effect for 1 minute and can be removed by any magic that cures diseases. After 1 minute, the diseased creature's skin becomes translucent and slimy, the creature cannot regain Hit Points unless it is underwater, and the disease can only be removed by \emph{heal} or another cure disease spell level 3 or higher. When the creature is outside a body of water, it takes 6 (1d12) acid damage every 10 minutes unless its skin is bathed before those 10 minutes have passed.

\emph{\textbf{Tail.} Melee weapon attack}: +9 to hit, reach 10 ft., one target.

\emph{Hits:} 15 (3d6 + 5) bludgeoning damage.

\emph{\textbf{Enslave (3/Day).}} The aboleth targets a creature it can see within 30 feet of it. The target must succeed on a DC 14 Will save or be magically charmed by the aboleth until the aboleth dies or the two are on different planes of existence. The charmed target is under the aboleth's control and cannot take reactions. The aboleth and the target can communicate telepathically with each other over any distance.

Whenever the charmed target takes damage, it can repeat the saving throw. On a success, the effect ends. No more than once every 24 hours, he can repeat the saving throw when he is at least 1 mile away from the aboleth.

\textbf{Additional Shares}

The aboleth can perform 3 additional Actions, chosen from the options below. It can use only one Additional option at a time, and only at the end of another creature's round. The aboleth regains any additional Actions spent at the start of its round.

\textbf{Spot.} The aboleth makes a Wisdom (Awareness) check.

\textbf{Psychic Drain (Costs 2 Actions).} A creature charmed by the aboleth takes 10 (3d6) damage and the aboleth recovers a number of Hit Points equal to the damage suffered by the creature.

\textbf{Tail Sweep.} The aboleth makes a tail attack.

\textbf{Ecology}\\
Environment: Any Aquatic\\
Organization: Solitary, pair, brood (3-6) or pack (7-19)\\
\textbf{Treasure}: Double\\
\textbf{Description}\\
As their primitive appearance suggests, hermaphroditic aboleths are among the oldest life forms in the world. Already ancient when the gods began to take an interest in the Material Plane, the aboleths have always lived far from other mortals: they are alien, cold and always busy weaving plans. They once ruled the world in a vast empire, and today they see other life forms as food or slaves... sometimes both at once. They despise the gods, as they believe they are the true masters of creation. An aboleth is 7 meters long and weighs approximately 3.2 tons. In the darkest depths of the sea, the aboleths still live in their grotesque cities, cyclopean and nauseating. They are served by countless slaves taken from every nation, both terrestrial and marine, and the terrestrial ones are doubly slaves to their masters and their mucus, which allows them to breathe underwater, the aboleths encountered alone are usually explorers from these hidden cities, looking for new slaves.



\subsection{Angels}



\medskip\index[Monstruary]{Angelo Deva}\textbf{Angelo Deva}

\emph{Medium celestial, legal good}

\textbf{STRENGTH} +4

\textbf{DEXTERITY} +4

\textbf{CONSTITUTION} +4

\textbf{INTELLIGENCE} +3

\textbf{WISDOM} +5

\textbf{CHARISMA} +5

\textbf{Initiative} +4 -- \textbf{Defense} 22

\textbf{Hit Points} 136 (16d8 + 64)

\textbf{Movement} 9 m, flight 27 m

\textbf{Saving Throws} Fortitude +16, Reflexes +13, Will +11

\textbf{Skills} Sense Emotions +9, Awareness +9

\textbf{Damage Resistances} from Light; from a non-magical weapon

\textbf{Condition Immunity} fascinated, fatigued, scared

\textbf{Senses} darkvision 36 m

\textbf{Languages} all, telepathy 36 m

\textbf{Challenge} 10 (5900 PX)

\emph{\textbf{Angelic Weapons.}} The deva's weapon attacks are magical. When the deva hits with any weapon, the weapon deals an additional 4d8 Light damage (already included in the attack).

\emph{\textbf{Innate Spells.}} The deva's innate spellcasting ability is Charisma (DC 17 for spell saving throws). The deva can innately cast the following spells, using only verbal components:

At will: \emph{identification of good and evil}

1/day: \emph{communion, reviving the dead}

\emph{\textbf{Magic Resistance.}} The deva has +1d6 on saving throws against spells and other magical effects.

\textbf{Shares}

\emph{\textbf{Multiattack.}} The deva makes two melee attacks.

\emph{\textbf{Mace.} Melee weapon attack}: +19 to hit, reach 1 m, one target.

\emph{Hits:} 7 (1d6 + 4) bludgeoning damage plus 18 (4d8) Light damage.

\emph{\textbf{Healing Touch (3/Day).}} The deva comes into contact with another creature. The target magically regains 20 (4d8 + 2) hit points and is free from any blindness, disease, curse, deafness, or poison.

\emph{\textbf{Shapeshifting.}} The deva can magically transform into a humanoid or beast whose challenge rating is equal to or lower than its own, or return to its true form. Upon death he returns to his true form. Whatever equipment it is wearing or carrying is absorbed or transported into the new form (the deva's choice).

In the new form, the deva retains his game statistics and ability to speak, but his Defense, movement methods, Strength, Dexterity, and special senses are replaced by those of the new form, and he gains any statistics or abilities (Additional Actions and lair actions) possessed by its new form and not its original.

\textbf{Ecology}
Environment: Any level with Good Traits\\
Organization: Solo, couple, or squadron (3-6)\\
\textbf{Treasure}: Double (Fiery Greatsword +1, more treasure)\\
\textbf{Description}\\
Movanic devas make up the infantry ranks of the celestial armies, though they spend most of their time patrolling the Positive, Negative, and Material Planes. On the Positive Plane they guard the wandering good souls. On the Negative Plane they fight the undead and other strange beings that hunt in the ravenous void. Their rare visits to the Material Plane are usually for the purpose of bringing aid to powerful mortals, when great danger threatens to cause an entire kingdom to fall into the hands of evil.

\medskip\index[Monster]{Angel, Planetar}\textbf{Angel Planetar}

\emph{Large celestial, legal good}

\textbf{STRENGTH} +7

\textbf{DEXTERITY} +5

\textbf{CONSTITUTION} +7

\textbf{INTELLIGENCE} +4

\textbf{WISDOM} +6

\textbf{CHARISMA} +7

\textbf{Initiative} +5 -- \textbf{Defense} 27

\textbf{Hit Points} 200 (16d10 + 112)

\textbf{Movement} 12 m, flight 36 m

\textbf{Saving Throws} Fortitude +23, Reflexes +21, Will +22

\textbf{Skills} Awareness +11

\textbf{Damage Resistances} from Light;

\textbf{Condition Immunity} charmed, fatigued, frightened, weapons +1

\textbf{Senses} true vision 36 m

\textbf{Languages} all, telepathy 36 m

\textbf{Challenge} 16 (15000 PX)

\emph{\textbf{Angelic Weapons.}} The planetar's weapon attacks are magical. When you hit with any weapon, the weapon deals an additional 5d8 Light damage (already listed on the attack).

\emph{\textbf{Divine Awareness.}} The planetar immediately recognizes lies.

\emph{\textbf{Innate Spells.}} The orrery's innate spellcasting characteristic is Charisma (DC 20 for spell saving throws). The orrery can innately cast the following spells, without the need for material components:

At will: \emph{detection of good and evil}, \emph{invisibility} (personal only)

3/day: \emph{blade barrier, Fiery Strike, dispel good and evil} \emph{evil, raise dead}

1/day: \emph{communion, check the weather, plague of insects}

\emph{\textbf{Magic Resistance.}} The planetar has +1d6 on saving throws against spells and other magical effects.

\textbf{Shares}

\emph{\textbf{Multiattack.}} The planetar makes two melee attacks.

\emph{\textbf{Broadsword.} Melee weapon attack}: +26 to hit, reach 2 m, one target.

\emph{Hits:} 21 (4d6 + 7) slashing damage plus 22 (5d8) Light damage.

\emph{\textbf{Healing Touch (4/Day).}} The planetar comes into contact with another creature. The target magically regains 30 (6d8 + 3) hit points and is free from any blindness, disease, curse, deafness, or poison.

\emph{\textbf{Angry:}}

-the Planetar summons angelic powers to aid him. Using 3 Actions summon 1d4 Deva Angels.

- The Planetar deals critical damage (2d6) every time it hits until the end of the fight. Costs 1 Action.


\textbf{Ecology}
Environment: Any floor with Good Traits\\
Organization: Solo or couple\\
\textbf{Treasure}: Double (Sacred Greatsword +3)\\
\textbf{Description}\\
Planetars are the generals of celestial armies aimed at the destruction of evil. A planetar is typically 2.7 meters tall and weighs about 250 kg. They are excellent diplomats, but against fiends they prefer war rather than negotiating a peace.

\medskip\index[Monstruary]{Angelo, Solar}\textbf{Angelo Solar}

\emph{Large celestial, legal good}

\textbf{STRENGTH} +8

\textbf{DEXTERITY} +6

\textbf{CONSTITUTION} +8

\textbf{INTELLIGENCE} +7

\textbf{WISDOM} +7

\textbf{CHARISMA} +10

\textbf{Initiative} +7 -- \textbf{Defense} 31

\textbf{Hit Points} 243 (18d10 + 144)

\textbf{Movement} 15 m, flight 45 m

\textbf{Saving Throws} Fortitude +29, Reflexes +27, Will +28

\textbf{Skills} Awareness +14

\textbf{Damage Resistances} from Light;

\textbf{Damage Immunity} Void, Poison, weapons +2

\textbf{Condition Immunity} charmed, poisoned, fatigued, frightened, weapon +2

\textbf{Senses} true vision 36 m

\textbf{Languages} all, telepathy 36 m

\textbf{Challenge} 21 (33000 PX)

\emph{\textbf{Angelic Weapons.}} The solar's weapon attacks are magical. When you hit with any weapon, the weapon deals an additional 6d8 Light damage (already included in the attack).

\emph{\textbf{Divine Awareness.}} Solar recognizes lies immediately.

\emph{\textbf{Innate Spells.}} The solar's innate spellcasting ability is Charisma (DC 25 for spell saving throws). The solar can innately cast the following spells, without the need for material components:

At will: \emph{detection of good and evil}, \emph{invisibility} (personal only)

3/day: \emph{blade barrier, Fiery Strike, dispel good and evil, resurrection}

1/day: \emph{communion, check weather}

\emph{\textbf{Magic Resistance.}} The solar has +1d6 on saving throws against spells and other magical effects.

\textbf{Shares}

\emph{\textbf{Multiattack.}} The solar makes two attacks with its greatsword.

\emph{\textbf{Broadsword.} Melee weapon attack}: +30 to hit, reach 1 m, one target.

\emph{Hits:} 22 (4d6 + 8) slashing damage plus 27 (6d8) Light damage.

\emph{\textbf{Longbow of Slaying.} Ranged weapon attack}: +30 to hit, range 45m, one target.

\emph{Hits:} 15 (2d8 + 6) piercing damage plus 27 (6d8) Light damage. If the target is a creature with 100 hit points or less, it must succeed on a DC 15 Fortitude saving throw or die.

\emph{\textbf{Flying Sword.}} The solar frees his greatsword to magically float into an unoccupied space within 1 meter of him. If the solar can see the sword, as a bonus action he can mentally command it to fly up to 50 feet and make an attack against a target or return to the solar's hand. If the floating sword is the target of an effect, it is treated as if it were wielded by the solar. If the solar dies, the floating sword falls to the ground.

\emph{\textbf{Healing Touch (4/Day).}} The solar comes into contact with another creature. The target magically regains 40 (8d8 + 4) hit points and is free from any blindness, disease, curse, deafness, or poison.

The solar can perform 3 additional actions, chosen from the following options. He can only use one Additional Action at a time and only at the end of another creature's round. The solar recovers the additional actions spent at the start of its round.

\textbf{Incandescent Explosion (Costs 2 Actions).} The solar emits divine magical energy. Each creature of your choice within a 10-foot radius must make a DC 30 Reflex saving throw, taking 14 (4d6) fire damage plus 14 (4d6) Light damage on a failed save, or half as much on a failed save. succeeds.

\textbf{Dazzling Gaze (Costs 3 Actions).} The solar targets a creature within 30 feet that it can see. If the target can see the solar, the target must succeed on a DC 18 Fortitude save or be blinded until a spell such as \emph{lesser restoration} removes the blindness.

€15615 € {Teleportation.} The solar magically teleports up to 120 feet away, along with any equipment he is wearing or carrying, to an unoccupied space that he can see.

\textbf{Ecology}\\
Environment: Any floor with Good Traits\\
Organization: Solo or couple\\
\textbf{Treasure}: Double (Full Armor +5, Dancing Greatsword +5, Composite Longbow +5)\\
\textbf{Description}\\
Solars are the most powerful of angels, usually right-hand men of a deity or champions of causes that benefit an entire world or plane. A solar is usually almost human in appearance, although some of them resemble other humanoid races and some even have more unusual shapes. A solar is about 2.7 meters tall, weighs about 250 kg and has a deep, commanding voice that is impossible to ignore. Most of them have silver or golden skin.

Blessed with an array of more powerful magical abilities, solars are fearsome adversaries capable of single-handedly slaying the most powerful evil creatures. Among the celestials they are considered the most excellent trackers, and the best among them, it is said, are able to follow days-old tracks left by a Pit Devil across the Astral Plane. Some of them take up the mantle of monster slayers and hunt down powerful fiends and undead such as devourers, night hags, nightshades, and pit fiends, even making forays into evil planes and the Negative Energy Plane to destroy these creatures at their source, before they can harm mortals. Some of the oldest solars have completed their mission, and have a reputation for killing creatures that are now extinct.

Solars accept the role of guardians, usually of supernatural concepts or objects or creatures of great importance. On one world, a group of solars protect the sun's energy conduits against attempts by evil races such as the Elves to turn it off and bring eternal darkness. On another, seven solars watch over seven mystical chains that keep the gods of evil imprisoned in a demiplane. On yet another, a solar with a flaming sword protects the Earthly Paradise, preventing all creatures from entering.

In worlds where gods can take physical form, solars are sent to become prophets and gurus (often in the guise of mortals), thus laying the foundations of cults that will become great religions. In worlds oppressed by evil, solars are the clandestine priests who bring hope to the oppressed or who allow themselves to be martyred so that their essence can explode into the surrounding regions and grow in the hearts of future heroes.

While not deities, solars' power approaches that of demigods, and they often act as advisors to younger or weaker deities. In some polytheistic faiths, mortals venerate one or more solars as aspects or equal servants of true deities (though never without the approval of the deity in question) or consider more famous solars as children, consorts, or lovers of true deities (which which, depending on the deity, could correspond to the truth).

Unlike other angels, most solars are created as direct servants of the gods, blending good souls and pure divine energy, but increasingly these powerful angels are created through the promotion of lesser angels such as devas and planetars. It rarely happens that particularly powerful and pure souls directly ascend to solar status. The oldest among them predate the creation of mortals, and are among the earliest creations of the gods. These solars are champions among their kind, and have little to no interaction with mortals, focusing instead on abstract concepts such as gravity, entropy, dark matter, and primal evil.

Solars who spend much time on the Material Plane, especially those who take the form of mortals, are sometimes the source of aasimar or half-celestial bloodlines in human families, sometimes due to romance, sometimes simply due to the proximity of mortals to their celestial emanations. It is rare that there are direct descendants of them, and when this happens it is always a mortal mother who carries the child: although solars can appear of any sex, the gods have not granted them the possibility of giving birth to a child. This is why solars tend to seek a mortal lover. Other solars have little regard for one of their peers who gives a child to a mortal, so solar fathers tend to avoid contact with their offspring, to avoid bringing shame upon themselves. Solars, however, tend to monitor their children from afar and, in times of difficulty, help them, albeit in mysterious and discreet ways.

All angels respect the power and wisdom of solars, and while they tend to work alone, they sometimes command armies led by planetars and act as generals for great raids against the legions of Hell or the hordes of the Abyss.

\medskip\index[Monstery]{Ankheg}\textbf{Ankheg}

\emph{Large monstrosity, misaligned}

\textbf{STRENGTH} +3

\textbf{DEXTERITY} +0

\textbf{CONSTITUTION} +1

\textbf{INTELLIGENCE} -5

\textbf{WISDOM} +1

\textbf{CHARISMA} -2

\textbf{Initiative} +0 -- \textbf{Defense} 15, 12 while prone

\textbf{Hit Points} 39 (6d10 + 6)

\textbf{Movement} 9 m, excavation 3 m

\textbf{Saving Throws} Fortitude +3, Reflex +2, Will +3

\textbf{Senses} Darkvision 60 ft., tremor perception 60 ft.,

\textbf{Languages} -

\textbf{Challenge} 2 (450 PX)

\textbf{Shares}

\emph{\textbf{Bite.} Melee weapon attack}: +5 to hit, reach 1 m, one target.

\emph{Hits:} 10 (2d6 + 3) slashing damage plus 3 (1d6) acid damage. If the target is a Large or smaller creature, she is grappled (DC 13 to escape). Until the grapple ends, the ankheg can only bite the grabbed creature and has +1d6 to attack rolls against it.

\emph{\textbf{Acid Spray (Recharge 6).}} The ankheg spits acid in a line 30 feet long and 3 feet wide, as long as it isn't grabbing any creatures. Each creature on that line must make a DC 13 Reflex saving throw, and take 10 (3d6) acid damage on a failed save, or half as much damage on a successful one.

\textbf{Ecology}\\
Environment: Temperate or warm plains\\
Organization: Solitary, couple or nest (3-6)\\
\textbf{Honey}: Accidental\\
\textbf{Description}\\
Ankhegs are an all-too-common plague for rural areas. This horse-sized burrowing monster generally avoids heavily populated areas, but its predilection for the flesh of livestock and humans keeps it away from uninhabited areas. Their favorite habitat is the rural countryside, given that the loose soil makes it very easy for them to move by digging. There are tales of larger ankhegs living in remote deserts, feeding on scorpions and camels and rarely coming into contact with civilization (a desert ankheg is a Huge advanced ankheg).

In combat, ankhegs prefer to attack with their bite. Against multiple opponents, an ankheg grabs one of its targets and attempts to retreat underground. A creature dragged underground can breathe, albeit with difficulty (even the ankheg must, so the tunnels are quite porous), but it is often eaten alive before its companions can save it.

Ankhegs dig with their legs and jaws, moving very quickly through dirt, sand, and gravel (not rock). A burrowing ankheg often stops to build tunnels, coating the walls with thick oral secretion. If an ankheg wants to build a tunnel while digging, it must move at half its digging speed. A typical ankheg tunnel is 10 feet high and wide, vaguely circular in shape, and 60 to 150 feet long ([1d10+5]×10). Groups of ankheg share territory and create complex networks of tunnels beneath the countryside, sometimes creating chasms where too many of them dig at the same time.

Even though ankhegs resemble immense insects, they are more intelligent and, with a little time and a good trainer, can become pets or cargo. The fact that even domesticated ankhegs tend to spit acid when frightened or surprised makes them unsafe in more civilized regions, but among wild races, such as Hobgoblins, Troglodytes and especially Orcs they are popular as guardians or even pet pets. An ankheg can reach a length of 3 meters and weigh around 400 kg.

\medskip\index[Monster]{Harpy}\textbf{Harpy}

\emph{Medium monstrosity, chaotic evil}

\textbf{STRENGTH} +1

\textbf{DEXTERITY} +1

\textbf{CONSTITUTION} +1

\textbf{INTELLIGENCE} -2

\textbf{WISDOM} +0

\textbf{CHARISMA} +1

\textbf{Initiative} +1 -- \textbf{Defense} 12

\textbf{Hit Points} 38 (7d8 + 7)

\textbf{Movement} 6 m, flight 12 m

\textbf{Saving Throws} Fortitude +2, Reflexes +2, Will +1

\textbf{Languages} Municipality

\textbf{Challenge} 1 (200 PX)

\textbf{Shares}

\emph{\textbf{Multiattack.}} The armor makes two attacks: one with its claws and one with its club.

\emph{\textbf{Claws.} Melee weapon attack}: +3 to hit, reach 1 m, one target.

\emph{Hits:} 5 (2d4 + 1) slashing damage, 1 bleed damage.

\emph{\textbf{Club.} Melee weapon attack}: +3 to hit, reach 1 m, one target.

\emph{Hits:} 3 (1d4 + 1) bludgeoning damage.

\emph{\textbf{Enchantment Song.}} The harpy sings a magical melody. Each humanoid and giant within 300 feet of the harpy who can hear the song must succeed on a DC 11 Will save or be charmed until the song ends. The harpy must take a bonus action on its next round to continue singing. He can stop singing at any time. The song ends if the harpy is incapacitated.

While charmed by the harpy, a target is incapacitated and ignores the songs of other harpies. If the charmed target is more than 3 feet from the harpy, the target must move during its round to move towards the harpy using the most direct route. Before moving into dangerous terrain, such as lava or a pit, and before taking damage from any source other than the harpy, the target can repeat the saving throw. A creature can repeat the saving throw at the end of each of its rounds. If the saving throw is successful, the effect ends for that target.

A target that succeeds on the saving throw is immune to that harpy's song for the next 24 hours.

\textbf{Ecology}\\
Environment: Temperate Swamps\\
Organization: Solitary, pair or flock (3-12)\\
\textbf{Treasure}: Standard (Leather Armor, Spiked Mace and other treasure)\\
\textbf{Description}\\
Often seen as evil and corrupt creatures, harpies know how others think and act. This perceptive ability gives them an advantage in finding their favorite meals. While wild creatures easily fall victim to enchantment song, these evil bird-women prefer meals peppered with complex sentient thoughts. Easy prey makes for a boring meal.

Though ultimately savage and without any remorse for their actions, several harpies live among humanoid societies and enjoy exploiting creatures they deem potential meals.

Harpies tend to wear trinkets and trinkets stolen from their victims, because they love to delight in the brilliant adornments of men. Up close, these creatures reek of the stench of their devoured victims, and rarely let creatures not yet charmed get too close so that they cannot smell the blood and decay on their feathers. For this reason, many harpies cover themselves with perfumes and aromatic oils.

Harpies are markedly different depending on the region in which they live. Some resemble a mix of vultures and women, while others bear the regal features of hawks and falcons on their feathers. Rare broods of harpies, in isolated, tropical places around the world, also have feathers as colorful as parrots.

\medskip\index[Monstery]{Azer}\textbf{Azer}

\emph{Elemental average, lawful neutral}

\textbf{STRENGTH} +3

\textbf{DEXTERITY} +1

\textbf{CONSTITUTION} +2

\textbf{INTELLIGENCE} +1

\textbf{WISDOM} +1

\textbf{CHARISMA} +0

\textbf{Initiative} +1 -- \textbf{Defense} 18 (natural armor, shield)

\textbf{Hit Points} 39 (6d8 + 12)

\textbf{Movement} 9 m

\textbf{Saving Throws} Fortitude +2, Reflex +1, Will +1

\textbf{Damage Immunity} Fire, Poison

\textbf{Condition Immunity} poisoned

\textbf{Languages} Ignan

\textbf{Challenge} 2 (450 PX)

€15696 € {€15697 € {Heated Weapons.}} When the azer hits with a metal melee weapon, he deals an additional 3 (1d6) fire damage (already included in the attack).

\emph{\textbf{Heated Body.}} A creature that comes into contact with the azer or hits him with a melee attack while within 3 feet of him takes 5 (1d10) fire damage.

\emph{\textbf{Living Fire.}} An azer does not need food, drink, or sleep.

\emph{\textbf{Lighting.}} The azer radiates bright light in a 3 meter radius and dim light for an additional 3 meters.

\textbf{Shares}

\emph{\textbf{Warhammer.} Melee weapon attack}: +6 to hit, reach 1 m, one target.

\emph{Hits:} 7 (1d8 + 3) bludgeoning damage, or 8 (1d10 + 3) bludgeoning damage if used two-handed to make a melee attack, plus 3 (1d6) fire damage.

\textbf{Ecology}\\
Environment: Any terrain (Plane of Fire)\\
Organization: Solo, duo, group (3-6), squad (11-20 plus 2 3rd level sergeants and 1 3rd-6th level leader) or clan (30-100 plus 50% non-combatants plus 1 3rd level sergeant for every 20 adults, 5 5th level lieutenants and 3 7th level captains)\\
\textbf{Treasure}: Standard (Masterwork Scale Armor, Masterwork Warhammer, Light Hammer, other treasure)\\
\textbf{Description}\\
A proud and industrious race from the Plane of Fire, the Azers toil in their fortresses of bronze and brass, ever ready to fight their long, simmering war against the Efreet. The Azers live in a society where every member knows his place. Born with specific duties, usually related to the activities of their father or mother, Azers dedicate themselves to these occupations throughout their lives. A caste system further keeps Azer society in line. Nobles, who rule without being accountable to anyone, wear kilts of ornate brass as a symbol of their caste, while those of merchants and shop owners are made of durable bronze. Copper kilts are worn by the working caste, made up of servants, artisans and labourers.

Capable of channeling heat through metal weapons and tools, the Azer almost never use non-metallic weapons, and prefer melee to ranged attacks. They usually take prisoners, taking them back to their fortresses and forcing them to work for them for a year and a day.

More than half a million Azer live in the legendary City of Brass. Most of these unfortunate Azers live lives of Slavery under the Efreet. The Azers subjugated to this Slavery continue to perform their duties without question, preferring to wait for their contracts to be concluded or hoping that their masters will die or be defeated. Dedication to order burns bright in this Race, so much so that some of the Azer Slaves serve as overseers over their own people. Outside of the City of Brass, Azers are free to live their lives, often in other Planar metropolises, crafting items, selling goods, and running taverns.

To the untrained eye, the Azers look strikingly similar to each other. They are 1.2 meters tall but weigh 100 kg.

\medskip\index[Monstery]{Banshee}\textbf{Banshee}

\emph{Medium undead, chaotic evil}

\textbf{STRENGTH} -5

\textbf{DEXTERITY} +5

\textbf{CONSTITUTION} +0

\textbf{INTELLIGENCE} +1

\textbf{WISDOM} +1

\textbf{CHARISMA} +4

\textbf{Initiative} +5 -- \textbf{Defense} 15

\textbf{Hit Points} 58 (13d8)

\textbf{Movement} 0 m, fly 18 m (floats)

\textbf{Saving Throws}: Fortitude +4, Reflexes +9, Will +5

\textbf{Damage Resistances} acid, lightning, fire, sound; magic weapon +1

\textbf{Damage Immunity} from Void, Poison, Cold, from non-magical weapons

\textbf{Condition Immunity} charmed, grabbed, poisoned, entangled, paralyzed, petrified, prone, fatigued, bleed

\textbf{Senses} Darkvision 18 m

\textbf{Languages} Elvish, Common, Exspiran

\textbf{Challenge} 4 (1100 PX)

\emph{\textbf{Detection of Life}}. The banshee senses the presence of creatures other than undead and constructs within a 5 kilometer radius. He knows the general direction they are in, but not their precise location.

\emph{\textbf{Incorporeal Movement}}. The Banshee can move through other creatures and objects as if they were difficult terrain. She takes 5 (1d10) force damage if she ends her round inside an object.

\emph{\textbf{Undead Nature.}} The Banshee has no need for air, food, drink, or sleep.

\emph{\textbf{Light Sensitivity}}. While she is in sunlight, the banshee has -1d6 on attack rolls, as well as on sight-based Wisdom (Awareness) checks.

\textbf{Shares}

\emph{\textbf{Corrupting Touch}}. Touch Attack: +6 to hit, reach 3 ft., one target.

\emph{Hit}: 12 (3d6 +2) void damage.

\emph{\textbf{Terrifying Face}}. Any non-undead creature within 60 feet of the banshee that can see it must succeed on a DC 15 Will save with a Charisma modifier or be frightened for 1 minute. A frightened target can repeat the saving throw at the end of each of its rounds, suffering -1d6 if the Banshee is within line of sight; if she succeeds, the effect ends for him. If a target succeeds on its saving throw or the effect ends for it, that target is immune to the Banshee's Dreadface for the next 24 hours.

\emph{\textbf{Lament (1/Day)}}. The Banshee emits a mournful wail, as long as she is not exposed to sunlight. This wail has no effect on constructs and undead. Any other creature within 30 feet of her that can hear her must make a Fortitude save DC 15; if he fails, he drops to 0 Hit Points, while if he succeeds, he suffers 35 (10d6) psychic damage.

\textbf{Ecology}\\
Environment: Any\\
Organization: Solitaire\\
\textbf{Treasure}: None\\
\textbf{Description}\\
The Banshee is the enraged spirit of an elf who has betrayed her loved ones or has been betrayed herself. Driven mad by pain, the Banshee wreaks her vengeance on every living creature (innocent or guilty) with her fearsome touch and deadly screams.

\medskip\index[Monstery]{Basilisk}\textbf{Basilisk}

\emph{Medium monstrosity, misaligned}

\textbf{STRENGTH} +3

\textbf{DEXTERITY} -1

\textbf{CONSTITUTION} +2

\textbf{INTELLIGENCE} -4

\textbf{WISDOM} -1

\textbf{CHARISMA} -2

\textbf{Initiative} -1 -- \textbf{Defense} 17

\textbf{Hit Points} 52 (8d8 + 16)

\textbf{Movement} 6 m

\textbf{Saving Throws}: Fortitude +5, Reflexes +2, Will +2

\textbf{Senses} Darkvision 18 m

\textbf{Languages} -

\textbf{Challenge} 3 (700 PX)

\emph{\textbf{Petrifying Gaze.}} If a creature begins its round within 30 feet of the basilisk and the two can see each other, if not incapacitated the basilisk can force the creature to make a Fortitude saving throw DC 12 If the creature fails its saving throw, it magically begins to turn to stone and is entangled. The creature must repeat the saving throw at the end of its next round. If she succeeds, the effect ends. On a failed save, the creature is petrified until freed by the \emph{restoration} \emph{greater} spell or other magic.

A creature that is not surprised can look away to avoid the saving throw at the start of its round. In that case, he won't be able to see the basilisk until the start of his next round, when he can look away again. If she were to look at the basilisk in the meantime, she should immediately make the saving throw.

If the basilisk is within 30 feet of its bright light reflection and sees it, it mistakes it for a rival and becomes the target of its gaze.

\textbf{Shares}

\emph{\textbf{Bite.} Melee weapon attack}: +7 to hit, reach 1 m, one target.

\emph{Hits:} 10 (2d6 + 3) piercing damage plus Basilisk Poison.

\emph{Poison:} Basilisk Poison, F, instant, 14, Slow 1/3r.

\textbf{Ecology}\\
Environment: Any\\
Organization: Solitary, couple or colony (3-6)\\
\textbf{Honey}: Accidental\\
\textbf{Description}\\
The basilisk, often called \emph{King of Snakes} is an aggressive eight-legged reptile that has the ability to turn creatures into stone with its gaze. Legend has it that, like the Cockatrice, the first basilisks were born from eggs laid by snakes and hatched by roosters, but very little in the physiology of the basilisk leaves room for this theory.

Basilisks live in almost all dry environments, from forest to desert, and their skin tends to reflect their surroundings: a desert basilisk may be bronze or brown, while one that lives in forests may be green. switched on. They tend to use caves, dens or other sheltered areas as shelter. These shelters are often marked by statues depicting people and animals in natural poses, which are nothing more than the petrified remains of the unfortunate people who came across a basilisk.

Basilisks have the ability to consume petrified creatures; the acid produced by their stomach dissolves and extracts nutrients from the stone, although the process is slow and inefficient, making them sluggish and inert. As a result, basilisks rarely attack or hunt prey that avoids their gaze, relying on the element of surprise in order not to run out of food. When not waiting for the small mammals, birds or reptiles that are part of their diet, basilisks spend their time sleeping in burrows. Those brave enough to capture basilisks or hide treasure near them find that these beings can act as guardians or watchdogs.

An adult basilisk is almost 3 meters long, half of which is taken up by the long tail, and weighs 135 kilos. Some breeds have small curved horns on their noses or small crown-like crests of bony stingers on top of their heads. Although they are generally solitary creatures that come together only to mate and lay eggs, in particularly dangerous areas they can gather in small groups to protect themselves and attack intruders en masse.

For unknown reasons, weasels, ferrets and mice are immune to the basilisk's gaze, and sometimes sneak into the burrows while the adult is hunting to feed on its young. Some legends say that the blood of a basilisk can transform ordinary stones into another material, but these are probably witnesses who have misinterpreted the magical restoration of previously petrified creatures or body parts.

\medskip\index[Monstery]{Behir}\textbf{Behir}

\emph{Huge monstrosity, neutral evil}

\textbf{STRENGTH} +6

\textbf{DEXTERITY} +3

\textbf{CONSTITUTION} +4

\textbf{INTELLIGENCE} -2

\textbf{WISDOM} +2

\textbf{CHARISMA} +1

\textbf{Initiative} +3 -- \textbf{Defense} 23

\textbf{Hit Points} 168 (16d12 + 64)

\textbf{Movement} 15m, climb 12m

\textbf{Saving Throws}: Fortitude +15, Reflexes +14, Will +13

\textbf{Skills} Stealth +7, Awareness +6

\textbf{Damage Immunity} electricity

\textbf{Senses} darkvision 27 m

\textbf{Languages} Draconic

\textbf{Challenge} 11 (7200 PX)

\textbf{Shares}

\emph{\textbf{Multiattack.}} The behir makes two attacks: one with its bite and one with its crush.

\emph{\textbf{Bite.} Melee weapon attack}: +16 to hit, reach 10 ft., one target.

\emph{Hits:} 22 (3d10 + 6) piercing damage.

\emph{\textbf{Crush.} Melee weapon attack}: +16 to hit, reach 1 ft., one Large or smaller creature.

\emph{Hits:} 17 (2d10 + 6) bludgeoning damage plus 17 (2d10 + 6) slashing damage. The target is grappled (DC 16 to escape) If the behir is not already crushing another creature, the target is grappled and entangled until the grapple ends.

\emph{\textbf{Swallow.}} The behir makes a bite attack against a Medium or smaller target it is grabbing. If the attack hits, the target is engulfed, and the grapple ends. The engulfed target is blinded and restrained, has full cover against attacks and other effects outside the behir, and takes 21 (6d6) acid damage at the start of each round of the behir. The behir can swallow only one creature at a time.

If the behir takes 30 or more damage in a single round from a creature it has swallowed, it must succeed on a DC 14 Fortitude saving throw at the end of that round or vomit the creature, which falls prone in a space within 10 feet of the behir. If the behir dies, an engulfed creature is no longer restrained by it and can exit the corpse using 15 feet of movement, exiting prone.

\emph{\textbf{Lightning Breath (Recharge 5-6).}} The behir exhales lightning in a line 6 meters long and 1 meter wide. Each creature on that line must make a DC 16 Reflex saving throw and take 66 (12d10) lightning damage on a failed save, or half as much damage on a successful one.

\emph{\textbf{Angry:}} Behir recharges his lightning breath. It costs 2 Actions.

\textbf{Ecology}\\
Environment: Hills and Hot Deserts\\
Organization: Solo or couple\\
\textbf{Treasure}: Double\\
\textbf{Description}\\
Instinctive and greedy, the behir spends much of its time crawling across the sandy hills and desert rocks that form its territory, hunting any creatures that dare enter its territory. Its six pairs of robust and clawed legs remain folded at its sides for much of the time, and are only extended in combat to grab enemies, to run at a gallop or to climb the slopes of sheer cliffs, their dens. creatures.

The average behir is 12 meters long and weighs about 1800 kg. In addition to the two prominent horns on the head, many have decorative spines at regular intervals along the spine.

While territorial and bestial in its fury, the behir is neither stupid nor necessarily evil although, due to its self-centeredness and tendency to claim everything in existence as its own, it often comes into conflict with other races. As such, a behir can be bribed or persuaded by intrepid negotiators willing to approach him. In these cases, a behir's tendency to attack first and reason later (or not at all) means that anyone trying to reach an agreement must have good reasons and quickly impress the behir with an attractive offer.

It is often said that behirs are somehow linked to blue dragons, but the true nature of this bond remains a mystery. Many dragons deny any Bond and frown upon the behir due to their lack of intelligence: an affront that infuriates the already impulsive behir. Precisely for this reason, many behir bear a grudge against dragons and are ready to attack any dragon that enters their territory.

\medskip\index[Monster]{Exploding Cockroach}\textbf{Exploding Cockroach}

\emph{Small Elemental, neutral}

\textbf{STRENGTH} +1

\textbf{DEXTERITY} +2

\textbf{CONSTITUTION} +1

\textbf{INTELLIGENCE} -5

\textbf{WISDOM} -1

\textbf{CHARISMA} -2

\textbf{Initiative} +2 -- \textbf{Defense} 14

\textbf{Hit Points} 45 (8d8 + 9)

\textbf{Move} 4 m, jump 9 m, dig 2 m

\textbf{Saving Throws} Fortitude +5, Reflexes +6, Will +3

\textbf{Damage Resistances} bludgeoning

\textbf{Damage Immunity} from fire

\textbf{Condition Immunity} fatigued, scared

\textbf{Senses} blind sight 5 m

\textbf{Languages} -

\textbf{Challenge} 2 (450 PX)

\emph{fire detection}: the Explosive Cockroach can detect fires within 100 meters of distance, as long as it is equal to or greater than a torch

\emph{Dig}: The explosive cockroach can dig into solid ground midway through its movement.

\textbf{Shares}

\emph{\textbf{Multiattack.}} the Exploding Cockroach can make 1 charge attack or emit a fiery goo.

\emph{\textbf{Charge.}} Melee attack: +6 to hit, reach 1 meter, one target.

\emph{Hits:} 12 (3d6 + 3) bludgeoning damage. The creature must make a Fortitude save at DC 11 or fall prone.

\emph{\textbf{Fire Mash}} Ranged attack: +7 to hit, range 3 metres. The Explosive Cockroach regurgitates a sticky, flammable liquid into the air. Refill 1/3-6.

\emph{Hits:} 18 (4d6 + 6) fire damage. Reflex save DC 13 to halve.

\emph{\textbf{Death:}} When the Explosive Cockroach dies the jelly inside in contact with the air explodes all around, within a radius of 1 meter around the cockroach the flames cause 12 (4d6) damage, Reflex save DC 15 to halve.

\textbf{Ecology}\\
Environment: Hot caves\\
Organization: Solitary, nest (8-64)\\
\textbf{Treasure}: Diamond 1d4x1d50mo\\
\textbf{Description}\\
Blast Cockroaches are creatures native between the elemental planes of fire and earth. Usually attracted to environments rich in flames, stone or at least heat and earth.
With a shape proportionate to that of a common cockroach if not about 40 cm long and weighing about 4 kg, it is a creature completely devoid of intellect, acting only by pure instinct.
They are now common in caves near volcanoes or red dragon lairs having become accustomed to living on Yeru.

In the nest where they live there is at least one queen who commands the cockroaches, who is extremely larger and stronger. Explosive Cockroaches feed on coal, burnt wood, burnt carcasses. They are extremely greedy for diamonds which, once burned, are real delicacies.

\medskip\index[Monster]{B.O.C.}\textbf{B.O.C.}

\emph{large monstrosity, lawful evil}

\textbf{STRENGTH} +4

\textbf{DEXTERITY} +3

\textbf{CONSTITUTION} +2

\textbf{INTELLIGENCE} -2

\textbf{WISDOM} +1

\textbf{CHARRISMA} -1

\textbf{Initiative} +2 -- \textbf{Defense} 17

\textbf{Hit Points} 42 (8d8 + 10)

\textbf{Movement} 13 m

\textbf{Saving Throws} Fortitude +6, Reflexes +7, Will +5

\textbf{Skills} Stealth +8, Awareness +6

\textbf{Resistance} +4 on saving throws to spells from the Illusion List

\textbf{Senses} darkvision 20 m, low-light vision 18 m

\textbf{Languages} common, can only understand it

\textbf{Challenge} 4 (1100 PX)

\textbf{Shares}

\emph{\textbf{Multiattack.}} The B.O.C makes two claw attacks and one bite attack, or it makes two tentacle attacks

\emph{\textbf{Claws.} Melee weapon attack}: +6 to hit, reach 10 ft., one target, 1 bleed damage.

\emph{Hits:} 7 (1d6 + 4) slashing damage.

\emph{\textbf{Bite.} Melee weapon attack}: for each claw that hit the B.O.C gets +2 to hit with the bite. +8 to hit, reach 10 ft., one target.

\emph{Hits:} 10 (1d8 + 6) slashing damage.

\emph{\textbf{Tentacles.} Melee weapon attack}: Each tentacle can hit up to 6 meters away and each can hit a different target, +6 to hit.

\emph{Hits:} 6 (1d4 + 4) bludgeoning damage

\emph{\textbf{Deflect the light.}} The B.O.C. is constantly affected by an effect that alters its position, each attack roll has -1d6. This penalty is eliminated if the B.O.C. can be attacked. without using your sight to locate it.

The B.O.C. it constantly bends the light around itself, appearing almost a meter away from its real position. This ability is not affected by normal vision, only true vision, blindsight, or telluric sense can correctly perceive the B.O.C.

\textbf{Ecology}\\
Environment: Hills and forests\\
Organization: Solitary, pair or pack (2d4)\\
\textbf{Honey}: Accidental\\
\textbf{Description}\\
The Black Ops Cat better known as B.O.C. it is a large predatory feline, obviously black in color. Ferocious, insatiable, kills for the sake of hunting. He usually acts in a pack and is extremely loyal to the group.

\medskip\index[Monster]{Bugbear}\textbf{Bugbear}

\emph{Medium humanoid (goblinoid), chaotic evil}

\textbf{STRENGTH} +2

\textbf{DEXTERITY} +2

\textbf{CONSTITUTION} +1

\textbf{INTELLIGENCE} -1

\textbf{WISDOM} +0

\textbf{CHARRISMA} -1

\textbf{Initiative} +2 -- \textbf{Defense} 17

\textbf{Hit Points} 27 (5d8 + 5)

\textbf{Movement} 9 m

\textbf{Saving Throws} Fortitude +2, Reflexes +3, Will +1

\textbf{Skills} Stealth +6, Survival +2

\textbf{Senses} Darkvision 18 m

\textbf{Languages} Common, Goblin

\textbf{Challenge} 1 (200 PX)

\emph{\textbf{Surprise Attack.}} If the bugbear surprises a creature and hits it with an attack during the first round of combat, the target takes an additional 7 (2d6) damage
from the attack.

\emph{\textbf{Brute.}} A melee weapon deals an additional die of damage when the bugbear hits with it (already included in the attack).

\textbf{Shares}

\emph{\textbf{Spiked Mace.} Melee weapon attack}: +4 to hit, reach 1 m, one target.

\emph{Hits:} 11 (2d8 + 2) piercing damage.

\emph{\textbf{Javelin.} Melee or Ranged Weapon Attack}: +4 to hit, reach 1m or range 12m, one target.

\emph{Hits:} 9 (2d6 + 2) piercing damage in melee or 5 (1d6 + 2) piercing damage in range.

\textbf{Ecology}\\
Environment: Temperate mountains\\
Organization: Solo, duo, group (3-6) or warband (7-12 plus 2 1st level fighters and 1 3rd-5th level captain)\\
\textbf{Treasure}: NPC Equipment (Leather Armor, Light Wooden Shield, Spiked Mace, 3 Javelins, more treasure)\\
\textbf{Description}\\
The bugbear is the largest of the Goblinoid race, a lumbering brute that is at least a head taller than most Humans. They are loners who prefer to live and kill alone rather than in tribes, although it is not unusual to find a small band of Bugbears collaborating or living with a tribe of Goblins or Hobgoblins acting as elite guards or executioners.

Bugbears do not form large settlements like goblins or nations like hobgoblins; they prefer something smaller and chaotic that leaves them free to do what they like (kill and torture) on a more personal level. Humans are bugbears' favorite prey, and most bugbears count human flesh as a staple of their diet. Ghoulish trophies such as ears and fingers are common decorations among bugbears.

Bugbears, when they turn to religion, favor deities of murder and violence, with various demon lords among the favorites. A typical bugbear is 7 feet tall and weighs 200 pounds.

\medskip\index[Monstery]{Bulette}\textbf{Bulette}

\emph{Big beast, misaligned}

\textbf{STRENGTH} +4

\textbf{DEXTERITY} +0

\textbf{CONSTITUTION} +5

\textbf{INTELLIGENCE} -4

\textbf{WISDOM} +0

\textbf{CHARRISMA} -3

\textbf{Initiative} +0 -- \textbf{Defense} 20

\textbf{Hit Points} 94 (9d10 + 45)

\textbf{Movement} 12 m, excavation 12 m

\textbf{Saving Throws} Fortitude +10, Reflexes +5, Will +5

\textbf{Skills} Awareness +6

\textbf{Senses} Darkvision 60 ft., tremor perception 60 ft.

\textbf{Languages} -

\textbf{Challenge} 5 (1800 PX)

\emph{\textbf{Standing Jump.}} A bulette can jump up to 30 feet long and up to 16 feet high with or without a running start.

\textbf{Shares}

\emph{\textbf{Bite.} Melee weapon attack}: +11 to hit, reach 1 m, one target.

\emph{Hits:} 30 (4d12 + 4) piercing damage.

\emph{\textbf{Lethal Leap.}} If the bulette can jump at least 10 feet as part of its movement, it can then use this action to land on its feet in a space containing one or more creatures. Each of these creatures must succeed on a DC 16 Fortitude or Reflex save (target's choice) or be knocked prone and take 14 (3d6 + 4) bludgeoning damage plus 14 (3d6 + 4) slashing damage. On a successful save, the creature takes only half damage, is not knocked prone, and is pushed 3 feet out of the bulette's space to an unoccupied space of the creature's choice. If there are no unoccupied spaces within range, the creature falls prone in the bulette's space.

\emph{\textbf{Angry:}} The bulette recharges its last energy, recovering three times its CR in hit points. Costs 1 Action.

\textbf{Ecology}\\
Environment: Temperate Hills\\
Organization: Solo or couple\\
\textbf{Treasure}: None\\
\textbf{Description}\\
The creation of an unknown wizard of the past, the bulette has now become a ferocious hill predator. Digging rapidly beneath the ground, it cuts through the surface with its dorsal fin, leaving behind a distinctive trail. The bulette leaps out, freeing itself from stones and dirt, to tear its prey to pieces without remorse, thus giving rise to its nickname of \emph{land shark}.

Bulettes are known for their bad tempers, attacking creatures much larger than themselves without fear. Solitary beasts except for the occasional breeding pair, they spend most of their time patrolling their territories, which can exceed 4 km2, hunting and punishing intruders with a fury capable of shaking hillsides.

Bulettes are perfect machines for devouring and destroying bones, armor, and even magical items with their mighty jaws and churning stomach acid. For lack of anything else, a bulette might munch on common objects, but for some reason it does not willingly eat elven flesh, perhaps a sign of the involvement of elven magic in their creation, or of dwarves, although it can slaughter members of both Stingray. Halflings, however, are among these beasts' favorite foods, and no sensible halfling would venture into bulette territory lightly.

The bulette is a cunning fighter, surprising its foes with impressive agility. One of its favorite tactics is to charge forward and pounce on its prey, attacking with its razor-sharp claws. It is said that the flesh behind the dorsal crest of the beast is particularly tender, and that those who are willing or able to wait for the fin to be raised in the excitement of fighting or mating can attempt to deliver a fatal blow there, even if the Most people who have faced a land shark agree that the best way to win a fight with a bulette is to avoid it altogether.

\medskip\index[Monster]{Black Knight}\textbf{Black Knight}

\emph{Medium undead, chaotic evil}

\textbf{STRENGTH} +5

\textbf{DEXTERITY} +1

\textbf{CONSTITUTION} +5

\textbf{INTELLIGENCE} +1

\textbf{WISDOM} +2

\textbf{CHARISMA} +3

\textbf{Initiative} +3 -- \textbf{Defense} 28

\textbf{Hit Points} 171 (18d8+90)

\textbf{Movement} 9 metres

\textbf{Saving Throws}: Fortitude +23, Reflexes +19, Will +20

\textbf{Skills} Intimidate +12, Religion +8, Knowledge of the Planes +8, Arcane Knowledge +5

\textbf{Damage Resistances} cold, lightning

\textbf{Damage Immunity} Void, Poison; weapons +1

\textbf{Condition Immunity} charmed, poisoned, paralyzed, fatigued, frightened, bleeding

\textbf{Senses} Darkvision 36 m

\textbf{Languages} Common, Abyssal, Exspiran

\textbf{Challenge} 18 (20000 PX)

\emph{\textbf{Spells.}} The Black Knight has CM 7. His spellcasting characteristic is Charisma. The Black Knight knows the following spells:

level 1 (4 slots): \emph{Command, Arcane Bolt, Searing Wave, Shield}

level 2 (3 slots): \emph{block person, magic weapon}

level 3 (3 slots): \emph{counterspell, dispel magic, fireball}

level 4 (3 slots): \emph{exile, Marking Smite (with 1 automatic magic critical, Void damage)}

\emph{\textbf{Undead Nature.}} The Black Knight has no need for air, food, drink, or sleep.

€15,990 € {€15,991 € {Legendary Resistance (1 / Day).}} If the Black Knight fails a saving throw, he may choose to succeed instead.

\emph{\textbf{Turn Resistance.}} The Black Knight has +1d6 on saving throws against effects that turn undead.

\textbf{Shares}

\emph{\textbf{Multiattack.} 3 longsword attacks +3}: +27 to hit, reach 1 m, up to three different creatures, or 1 sword strike with Corruption

\emph{Hits:} 13 (1d10+5+3) slashing damage + Flaming Strike (Void damage)

\emph{Corruption:} 15 (1d10+10) slashing damage. The target must make a DC 18 Will save or lose 1/10 of a Trait point tied to a good Patron if present.

\textbf{Ecology}\\
Environment: Any\\
Organization: Solitaire\\
\textbf{Treasure}: longsword +3 or full armor +3, the rest of the equipment disappears with the death of the Black Knight.\\

\textbf{Description}
Damned to the depths of his soul, the Black Knight is the antithesis of the knight of Sumkjr, indeed often born from the corruption of a knight of Sumkjr. A fearsome, cunning, tactical opponent, he loves to behave and reason, maliciously, like a person still alive. His tactic is to launch the Fireball as soon as possible and then consume the victim with Marking Smite.

\medskip\index[Monster]{Centaur}\textbf{Centaur}

\emph{Large monstrosity, neutral good}

\textbf{STRENGTH} +4

\textbf{DEXTERITY} +2

\textbf{CONSTITUTION} +2

\textbf{INTELLIGENCE} -1

\textbf{WISDOM} +1

\textbf{CHARISMA} +0

\textbf{Initiative} +2 -- \textbf{Defense} 13

\textbf{Hit Points} 45 (6d10 + 12)

\textbf{Movement} 15 m

\textbf{Saving Throws} Fortitude +4, Reflexes +4, Will +3

\textbf{Skills} Acrobatics +6, Awareness +3, Survival +3

\textbf{Languages} Elvish, Sylvan

\textbf{Challenge} 2 (450 PX)

\emph{\textbf{Charge.}} If the centaur moves at least 30 feet towards the target and hits with a pike attack during the same round, the target takes an additional 10 (3d6) piercing damage.

\textbf{Shares}

\emph{\textbf{Multiattack.}} The centaur makes two attacks: one with the pike and one with the hooves or two with the longbow.

\emph{\textbf{Pike.} Melee weapon attack}: +6 to hit, reach 10 ft., one target.

\emph{Hits:} 9 (1d10 + 4) piercing damage.

\emph{\textbf{Hooves.} Melee weapon attack}: +6 to hit, reach 1 m, one target.

\emph{Hits:} 11 (2d6 + 4) bludgeoning damage.

\emph{\textbf{Longbow.} Ranged weapon attack}: +4 to hit, range 45m, one target.

\emph{Hits:} 6 (1d8 + 2) piercing damage.

\textbf{Ecology}\\
Environment: Temperate plains and forests\\
Organization: Solo, duo, warband (3-10), tribe (11-30 plus 3 3rd level hunters and 1 6th level leader)\\
\textbf{Treasure}: Standard (Plate Armor, Heavy Metal Shield, Long Sword, Spear, other treasure)\\
\textbf{Description}\\
Legendary hunters and skilled warriors, centaurs are part man, part horse. Generally located on the fringes of civilization, this stoic population varies enormously in appearance: the skin color is usually very tanned but similar to that of humans from neighboring regions, while the lower part of the body has the tones of local equines. They have dark colored hair and eyes and rather marked facial features, while their total size depends on the size of the horse whose lower part of their body they have. So, although the average centaur stands 7 feet tall and weighs more than 2300 pounds, multiple regional variations exist, from slender plains runners to massive mountain hunters.

Centaurs live on average about 60 years. Distant from other races and in conflict with others of their kind, centaurs are an ancient race slowly beginning to accept the modern world. While the majority of centaurs still live in tribes roaming vast plains or at the edges of mystical forests, some have abandoned the isolationist ways of their ancestors to settle in cosmopolitan cities. Often these free spirits are considered outcasts and despised by their tribes, and therefore the decision to abandon them is a heavy one. In some cases, however, entire tribes led by progressive leaders have begun to trade or form alliances with other communities of humanoids, especially Elves, sometimes Gnomes, and more rarely Humans or Dwarves. Many races remain wary of centaurs, however, mostly due to legends that portray them as territorial and ferocious creatures and the periodic violent clashes they have with stubborn settlers and expanding lands.

\medskip\index[Monstruary]{Chimera}\textbf{Chimera}

\emph{Great monstrosity, chaotic evil}

\textbf{STRENGTH} +4

\textbf{DEXTERITY} +0

\textbf{CONSTITUTION} +4

\textbf{INTELLIGENCE} -4

\textbf{WISDOM} +2

\textbf{CHARISMA} +0

\textbf{Initiative} +0 -- \textbf{Defense} 17

\textbf{Hit Points} 114 (12d10 + 48)

\textbf{Movement} 9 m, flight 18 m

\textbf{Saving Throws} Fortitude +10, Reflexes +6, Will +8

\textbf{Skills} Awareness +8

\textbf{Senses} Darkvision 18 m

\textbf{Languages} understands Draconic but cannot speak

\textbf{Challenge} 6 (2300 XP)

\textbf{Shares}

\emph{\textbf{Multiattack.}} The chimera makes three attacks: one with its bite, one with its horns and one with its claws. When fire breath is available, he can use breath weapon in place of his bite or horns.

\emph{\textbf{Claws.} Melee weapon attack}: +10 to hit, reach 1 m, one target.

\emph{Hits:} 11 (2d6 + 4) slashing damage, 1 bleed damage.

\emph{\textbf{Horns.} Melee weapon attack}: +10 to hit, reach 1 m, one target.

\emph{Hits:} 10 (1d12 + 4) bludgeoning damage.

\emph{\textbf{Bite.} Melee weapon attack}: +10 to hit, reach 1 m, one target.

\emph{Hits:} 11 (2d6 + 4) piercing damage.

\emph{\textbf{Fiery Breath (Recharge 5-6).}} The dragon head exhales fire in a 5 meter cone. Each creature in that area must make a DC 15 Reflex saving throw and take 31 (7d8) fire damage on a failed save, or half as much damage on a successful one.

\emph{\textbf{Angry:}} the Chimera glows with energy. Refills your Fire Breath. Costs 1 Action.

\textbf{Ecology}\\
Environment: Temperate Hills\\
Organization: Solitary, pair, pack (3-6) or flock (7-12)\\
\textbf{Treasure}: Standard\\
\textbf{Description}\\
Chimeras are monstrous creatures born from primal evil. Hateful and ravenous, they hunt both on the ground and in the air. A chimera's dragon head can be any type of evil dragon, with the corresponding breath weapon and wings generally having the same scales as the head. Chimeras speak in three overlapping voices, but they do so rarely, typically only to flatter a more powerful creature. A chimera is 1 meter tall at the withers, reaching a length of 4 meters and a weight of 350 kg.\\
Chimeras prefer meat, but can survive on vegetables if necessary (although when forced to do so their mood worsens further). The fact that they fly means they can choose their prey carefully, and they generally hunt in large areas looking for the easy ones. They are too stupid and belligerent to acquire followers, although a tribe of kobolds may sometimes make offers to them. On the contrary, they are intelligent and stubborn enough to make mediocre pets, and only a creature much more powerful than them can subdue them. They can form equal partnerships with respectful humanoids or similar creatures, and also agree to be used as mounts. A pride of chimaeras has a hierarchy similar to that of lions, with a dominant male commanding the group and most of the hunting done by the females. A solitary chimera can be a solitary young male or a female with cubs nearby.


\medskip\index[Monstery]{Chuul}\textbf{Chuul}

\emph{Great aberration, chaotic evil}

\textbf{STRENGTH} +4

\textbf{DEXTERITY} +0

\textbf{CONSTITUTION} +3

\textbf{INTELLIGENCE} -3

\textbf{WISDOM} +0

\textbf{CHARRISMA} -3

\textbf{Initiative} +0 -- \textbf{Defense} 18

\textbf{Hit Points} 93 (11d10 + 33)

\textbf{Movement} 9m, swim 9m

\textbf{Saving Throws} Fortitude +7, Reflexes +4, Will +4

\textbf{Skills} Awareness +4

\textbf{Damage Immunity} Poison

\textbf{Condition Immunity} poisoned

\textbf{Senses} Darkvision 18 m

\textbf{Languages} understands the Language of the Depths but cannot speak

\textbf{Challenge} 4 (1100 PX)

\emph{\textbf{Amphibious.}} The chuul can breathe air and water.

\emph{\textbf{Sense of Magic.}} The chuul senses magic within 120 feet of him. This trait works like the \emph{detect} \emph{magical} spell but is not magical in itself.

\textbf{Shares}

\emph{\textbf{Multiattack.}} The chuul makes two claw attacks. If the chuul is grabbing a creature, he can also use its tentacles once.

\emph{\textbf{Claws.} Melee weapon attack}: +10 to hit, reach 10 ft., one target.

\emph{Hits:} 11 (2d6 + 4) bludgeoning damage. A target is grappled (DC 14 to escape) if it is Large or smaller and the chuul isn't already grappling two other creatures.

\emph{\textbf{Tentacles.}} A creature grabbed by the chuul must succeed on a DC 13 Fortitude saving throw or be poisoned for 1 minute. Until the poison ends, the target is paralyzed. The target can repeat the saving throw at the end of each of its rounds, ending the effect on itself on a success.

\textbf{Ecology}\\
Environment: Temperate Swamps\\
Organization: Solitary, pair or pack (3-6)\\
\textbf{Treasury}: Standard\\
\textbf{Description}\\
Chuuls are armored crustacean-like predators, always lurking beneath the surface of shallow ponds and sloughs, emerging from hiding to grab their prey with their claws and then paralyze them with their mouth tentacles before eating them alive.

Chuuls are excellent swimmers, but prefer to attack creatures that are land-based or accustomed to shallow water. Once they have grabbed their victims, the chuul often drag them into deep water. Lizardfolk are the chuul's favorite prey, although the pale, underground-dwelling chuul species prefer morlocks, duergar, unwary elves, and other unfortunates who get too close to their subterranean waterways, with the exception of troglodytes whose taste i chuul find it particularly disgusting.

Chuuls are surprisingly intelligent, and many engage in idle speculation about their origins and motivations. They speak a chirping, gurgling dialect of the Common, but few of them are inclined to chat with those who are not of their kind, and if there is a Chuul society outside of the frenetic mating season, no one has yet discovered it. Instead, chuul minds seem devoted only to finding the perfect ambush location to attack other intelligent creatures and how to decorate their elaborate lairs with trophies of their victims. Although chuuls appear uninterested in using tools, they have a compulsive need to collect those of their victims. A typical chuul is 2.3 meters tall and weighs 325 kg.

\medskip\index[Monster]{Kobold}\textbf{Kobold}

\emph{Small humanoid (kobold), lawful evil}

\textbf{STRENGTH} -2

\textbf{DEXTERITY} +2

\textbf{CONSTITUTION} -1

\textbf{INTELLIGENCE} -1

\textbf{WISDOM} -2

\textbf{CHARRISMA} -1

\textbf{Initiative} +2 -- \textbf{Defense} 13

\textbf{Hit Points} 5 (2d6 - 2)

\textbf{Movement} 9 m

\textbf{Saving Throws} Fortitude +0, Reflexes +1, Will -2

\textbf{Senses} Darkvision 18 m

\textbf{Languages} Common, Draconic

\textbf{Challenge} 1/8 (25 PX)

\emph{\textbf{Light Sensitivity}}. While in sunlight, the kobold has -1d6 on attack rolls, as well as on sight-based Wisdom (Awareness) checks.

\emph{\textbf{Pack Tactics.}} The kobold has +1d6 on attack rolls against a creature if at least one of the kobold's allies is within 3 feet of the creature and that ally is not incapacitated.

\textbf{Shares}

\emph{\textbf{Dagger.} Melee weapon attack}: +4 to hit, reach 1 m, one target.

\emph{Hits:} 4 (1d4 + 2) piercing damage.

\emph{\textbf{Slingshot.} Ranged weapon attack}: +4 to hit, range 9m, one target.

\emph{Hits:} 4 (1d4 + 2) bludgeoning damage.

\textbf{Ecology}\\
Environment: Temperate forests or underground\\
Organization: solitary, group (2-4), nest (5-30 plus an equal number of non-combatants, 1 3rd level sergeant for every 20 adults and 1 4th-6th level leader) or tribe (31-300 more than 35\% non-combatants, 1 3rd level sergeant per 20 adults, 2 4th level lieutenants, 1 6th-8th level leader, and 5-16 Dire Rats)\\
\textbf{Treasure}: NPC equipment (Leather Armor, Spear, Sling, other treasure), 2d6 silver coins\\
\textbf{Description}\\
Kobolds are creatures of the dark, most likely to be found in vast underground warrens or in dark corners of forests where the sun never shines. Because of their physical resemblance, kobolds loudly proclaim themselves heirs of the draconic lineage and destined to rule the land under the wings of their great divine cousins, but most dragons regard them as little more than nuisance insects. But even as they proclaim divine ancestry and the evidence of their destiny, kobolds are aware of their weakness. Cowardly and scheming, they never fight openly if they can help it, instead setting ambushes and traps, burrowing into their warrens behind a blanket of crude but ingenious traps, or descending upon the enemy in vast, howling hordes.

The shade of the kobolds also varies between the brothers of the same brood, ranging between the colors of the dragons of Tàhil, with a predominance of red and purple, and more rarely white, green, blue and black.

Kobolds have a weakness for silver but, being terrible miners, they prefer to prey on adventurers' silver coins and eat them as if they were butter biscuits. Kobolds can digest silver quite quickly and the more they eat the brighter their scales become and the kobolds appear healthy.


\medskip\index[Monster]{Cockatrice}\textbf{Cockatrice}

\emph{Small monstrosity, misaligned}

\textbf{STRENGTH} -2

\textbf{DEXTERITY} +1

\textbf{CONSTITUTION} +1

\textbf{INTELLIGENCE} -4

\textbf{WISDOM} +1

\textbf{CHARRISMA} -3

\textbf{Initiative} +1 -- \textbf{Defense} 12

\textbf{Hit Points} 27 (6d6 + 6)

\textbf{Movement} 6 m, flight 12 m

\textbf{Saving Throws} Fortitude +0, Reflexes +1, Will +1

\textbf{Senses} Darkvision 18 m

\textbf{Languages} -

\textbf{Challenge} 1/2 (100 PX)

\textbf{Shares}

\emph{\textbf{Bite.} Melee weapon attack}: +3 to hit, reach 3 ft., one creature.

\emph{Hits:} 3 (1d4 + 1) piercing damage, and the target must succeed on a DC 11 Fortitude save or be Slowed 1/1r due to progressive petrification. If subsequent bites cause the creature to have no more Actions, the creature is petrified for 24 hours.

\textbf{Ecology}\\
Environment: Temperate plains\\
Organization: Solitary, couple, squadron (3-5) or flock (6-12)\\
\textbf{Treasure}: None\\
\textbf{Description}\\
Stupid, malevolent, and repulsive, cockatrices are shunned by other creatures for their ability to turn flesh to stone. Legends state that the first cockatrice emerged from an egg laid by a rooster and hatched by a toad. Whether this story is true or not, today's cockatrices breed with each other in terrifying, dirty burrows haphazardly dug by at least a dozen clucking creatures. Males far outnumber females in these flocks, and are distinguished only by their barbels and crests. A typical cockatrice stands just over 60 centimeters tall and weighs 2.5 kg.

Although their diet consists mainly of seeds and petrified insects (which act both as gastroliths and nourishment in the creature's stomach), cockatrices fiercely defend their territory from anything they deem a threat, and the wanderings of wandering males in search of new places to build dens sometimes lead them into involuntary contact with humans, with devastating results.

The cockatrice's strange ability to turn other creatures to stone is her best defense, and her lair is invariably filled with the remains of petrified enemies. Ironically, however, weasels and ferrets, the creatures most likely to end up in cockatrice nests to eat their eggs, appear completely immune to this effect. For unknown reasons, cockatrices are both terrified and furious with common roosters, and are equally likely to flee or attack when a confrontation occurs.


\medskip\index[Monstery]{Couatl}\textbf{Couatl}

\emph{Heavenly average, legal good}

\textbf{STRENGTH} +3

\textbf{DEXTERITY} +5

\textbf{CONSTITUTION} +3

\textbf{INTELLIGENCE} +4

\textbf{WISDOM} +5

\textbf{CHARISMA} +4

\textbf{Initiative} +5 -- \textbf{Defense} 21

\textbf{Hit Points} 97 (13d8 + 39)

\textbf{Movement} 9 m, flight 9 m

\textbf{Saving Throws} Fortitude +9, Reflexes +13, Will +14

\textbf{Damage Resistances} from Light

\textbf{Damage Immunity} from non-magical weapons

\textbf{Senses} true vision 36 m

\textbf{Languages} all, telepathy 36 m

\textbf{Challenge} 4 (1100 PX)

\emph{\textbf{Magical Weapons.}} The couatl's weapon attacks are magical.

\emph{\textbf{Innate Spells.}} The couatl's innate spellcasting ability is Charisma. The couatl can cast these spells innately, using only verbal components:

At will: \emph{detection of good and evil, detection of magic, detection of thoughts}

3/day each: \emph{blessing, create food and water, cure wounds,} \emph{protection from poisons, lesser restoration, shrine, shield} 1/day each: €16197{greater restoration, scrying, dream }

€16,198 € {€16,199 € {Warded Mind.}} The couatl is immune to scrying and any effect that senses his emotions, reads his thoughts, or detects his location.

\textbf{Shares}

\emph{\textbf{Bite.} Melee weapon attack}: +8 to hit, reach 3 ft., one creature.

\emph{Hits:} 8 (1d6 + 5) piercing damage, and the target must succeed on a DC 13 Fortitude save or be poisoned for 24 hours. Until the poison ends, the target is unconscious. Another creature can take an action to awaken the target.

\emph{\textbf{Crush.} Melee weapon attack}: +6 to hit, reach 10 ft., one Medium or smaller creature.

\emph{Hit:} 10 (2d6 + 3) bludgeoning damage, and the target is grappled (DC 15 to escape). Until the grapple ends, the target is entangled, and the couatl cannot constrict another target.

\emph{\textbf{Shapeshifting.}} The couatl can magically transform into a humanoid or beast whose challenge rating is equal to or lower than its own, or revert to its true form. Upon death he returns to his true form. Whatever equipment he is wearing or carrying is absorbed or transported into the new form (the couatl's choice).

In the new form, the couatl retains its game statistics and ability to speak, but its Defense, movement methods, Strength, Dexterity, and other actions are replaced by those of the new form, and it gains any statistics or abilities (Additional Actions and lair actions) possessed by its new form and not its original. If the new form has a bite attack, the couatl can use its bite in the new form.

\textbf{Ecology}\\
Environment: warm forests\\
Organization: Solitary, pair or flock (3-6)\\
\textbf{Treasury}: Standard\\
\textbf{Description}\\
Couatl are servants of lawful good deities, though some operate independently of any higher entities. Respected and admired for their wisdom and beauty, they seek to lead mortals to the right path and use their powers to fight evil, especially those known to planeswalk. Some couatls are seen as benevolent deities by isolated societies, and although couatls shudder at the mere thought of pretending to be a deity, they allow these misconceptions to perpetuate as they allow them to lead these societies onto paths of peace and cooperation with their neighbors . A couatl is about 3.6 meters long, with a wingspan of about 5 meters and weighs 900 kg.

As native outsiders, couatls must eat. They prefer the same foods as true snakes, such as mammals and birds, although they have been known to devour evil humanoids. Because they prefer to spend their time pursuing their goals rather than hunting, they appreciate offerings of food, especially small wild boars and birds. A couatl sometimes shows its appreciation to an adventurer or party who has done it a service by giving them 1d4 of its brilliantly colored feathers. These freely obtained feathers, when used as an additional material component, allow a spellcaster who casts Planar Ally to summon that specific couatl without paying the normal cost in gold or other values, provided that the couatl approves the service requested by the caster.

\medskip\index[Monster]{Creeping Cumulus}\textbf{Creeping Cumulus}

\emph{Large plant, misaligned}

\textbf{STRENGTH} +4

\textbf{DEXTERITY} -1

\textbf{CONSTITUTION} +3

\textbf{INTELLIGENCE} -3

\textbf{WISDOM} +0

\textbf{CHARISMA} -3

\textbf{Initiative} -1 -- \textbf{Defense} 18

\textbf{Hit Points} 136 (16d10 + 48)

\textbf{Movement} 6m, swim 6m

\textbf{Saving Throws} Fortitude +8, Reflexes +4, Will +5

\textbf{Skills} Stealth +2

\textbf{Damage Resistances} cold, fire

\textbf{Damage Immunity} Electricity

\textbf{Condition Immunity} blinded, deafened, fatigue

\textbf{Senses} blindsight 18 m (blind beyond this range)

\textbf{Languages} -

\textbf{Challenge} 5 (1800 PX)

\emph{\textbf{Lightning Absorption.}} Whenever the creeping mound takes lightning damage, it takes no damage and recovers a number of Hit Points equal to the lightning damage dealt.

\textbf{Shares}

\emph{\textbf{Multiattack.}} The crawling pile makes two slam attacks. If both attacks hit a Medium or smaller creature, the target is grappled (DC 14 to escape) and the creeping mound uses Wrap on it.

\emph{\textbf{Slam.} Melee weapon attack}: +11 to hit, reach 1 m, one target.

\emph{Hits:} 13 (2d8 + 4) bludgeoning damage.

\emph{\textbf{Wrap.}} The creeping mound envelops a Medium or smaller creature it has grabbed. The shrouded target is blinded, restrained, and unable to breathe, and must succeed on a DC 14 Fortitude saving throw at the start of each round of the barrow or take 13 (2d8 + 4) bludgeoning damage. If the pile moves, the enveloped target moves with it. The mound can only envelop one creature at a time.

\emph{\textbf{Angry:}} The Crawling Cum releases a wave of electricity. All creatures within 10 feet take 3d6 electricity damage. It costs 2 Actions.

\textbf{Ecology}\\
Environment: Temperate Forests or Swamps\\
Organization: Solitaire\\
\textbf{Treasure}: Standard\\
\textbf{Description}\\
Creeping mounds, also called just creeping piles, look like decaying plant masses. They are intelligent carnivorous plants, with a penchant for elven flesh. The brain and sensory organs are located in the upper body. Creeping mounds typically have a circumference of 2.3 meters and are 1.8 to 2.7 meters high. They weigh approximately 1,900 kg.

Crawling mounds are strange creatures, more like a tangle of parasitic vines than a single rooted plant. They are omnivores, capable of drawing sustenance from anything, clinging to trees to suck their sap, inserting roots into the soil to absorb simple nutrients, or consuming the flesh and bones of prey.

Crawling mounds are incredibly stealthy in their natural environment. They blend in with the surrounding terrain and can wait motionless for days for potential prey to arrive. They can be practically anywhere and attack at any time without warning and regardless of whether there are survivors or not, as long as they have food.

Crawling mounds usually lead a nomadic, solitary existence in deep forests and fetid swamps but can also be found underground in thickets of mushrooms. Worrying rumors speak of groups of creeping mounds gathering around large mounds deep in jungles and swamps, often during violent lightning storms. The reason for this behavior is unknown, and many sages wonder if there is a dark and alien purpose behind it.




\subsection{Demons}



\medskip\index[Monster]{Balor}\textbf{Balor}

\emph{Huge fiend (demon), chaotic evil}

\textbf{STRENGTH} +8

\textbf{DEXTERITY} +2

\textbf{CONSTITUTION} +6

\textbf{INTELLIGENCE} +5

\textbf{WISDOM} +3

\textbf{CHARISMA} +6

\textbf{Initiative} +5 -- \textbf{Defense} 29

\textbf{Hit Points} 262 (21d12 + 126)

\textbf{Movement} 12 m, flight 24 m

\textbf{Saving Throws} Fortitude +24, Reflexes +21, Will +22

\textbf{Damage Resistances} cold, lightning;

\textbf{Damage Immunity} Fire, poison, weapons +1

\textbf{Condition Immunity} poisoned

\textbf{Damage Vulnerability} cold iron

\textbf{Senses} true vision 36 m

\textbf{Languages} Abyssal, telepathy 36 m

\textbf{Challenge} 19 (22000 PX)

\emph{\textbf{Magical Weapons.}} The demon's weapon attacks are magical.

\emph{\textbf{Fire Aura.}} At the start of each of the demon's rounds, each creature within 3 feet of him takes 10 (3d6) fire damage, and flammable objects in the aura and that are not worn or carried catch fire. A creature that comes into contact with the demon or hits it with a melee attack while within 3 feet of it takes 10 (3d6) fire damage.

\emph{\textbf{Magic Resistance.}} The demon has +1d6 on saving throws against spells and other magical effects.

\emph{\textbf{Death Spasm.}} When the demon dies, it explodes; each creature within 30 feet of it must make a DC 20 Reflex saving throw, taking 70 (20d6) fire damage on a failed save, or half as much damage on a successful one. The explosion sets fire to flammable objects that are not being worn or carried, and destroys the demon's weapons.

\textbf{Shares}

\emph{\textbf{Multiattack.}} The demon makes two attacks: one with the longsword and one with the whip.

\emph{\textbf{Whip.} Melee weapon attack}: +30 to hit, reach 30 ft., one target.

\emph{Hits:} 15 (2d6 + 8) slashing damage plus 10 (3d6) fire damage, and the target must succeed on a DC 20 Fortitude saving throw or be dragged 7 meters towards the demon.

\emph{\textbf{Long Sword.} Melee weapon attack}: +30 to hit, reach 10 ft., one target.

\emph{Hits:} 21 (3d8 + 8) slashing damage plus 13 (3d8) lightning damage. If the demon scores a critical hit, it rolls damage three times, rather than twice.

\emph{\textbf{Teleport.}} The demon magically teleports, along with any equipment it is wearing or carrying, to an unoccupied space that it can see within 120 feet.

\textbf{Ecology}\\
Environment: Any (Abyss)\\
Organization: Solitary or warband (1 Balor and 2-5 Glabrezu)\\
\textbf{Treasure}: Standard (Unholy Longsword+1, Flame Whip+1, other treasure)\\
\textbf{Description}\\
When people whisper terrifying tales of demonic creatures, they mostly imagine a towering figure of fire and flesh, a horned nightmare armed with flaming whip and sword, flying through the night in search of its prey. The demon these people fear is Balor, and this fear is fully justified, since few demons can match the mighty Balor in strength or brutality.

In the Abyss, Balor are mostly in the service of demon lords, as generals or captains (when they are not extremely powerful balors, known as balor lords). A balor typically commands vast legions of demons, and while he often allows these eager, drooling minions to fight his battles, he is anything but a coward. If the opportunity presents itself to join a fight, few balors choose to hold back.

A Balor stands 4.2 meters tall and weighs 2,250 kg. Only the cruelest mortal souls can fuel the creation of a balor: unlike other demons, it often takes numerous souls of powerful villains to birth a new balor.

\medskip\index[Monster]{Demogorgon}\textbf{Demogorgon}

\emph{huge fiend (demon prince), chaotic evil}

\textbf{STRENGTH} +9

\textbf{DEXTERITY} +2

\textbf{CONSTITUTION} +8

\textbf{INTELLIGENCE} +5

\textbf{WISDOM} +3

\textbf{CHARISMA} +7

\textbf{Initiative} +5 -- \textbf{Defense} 35

\textbf{Hit Points} 468 (26d10+208)

\textbf{Move} 15 metres, swim 9m

\textbf{Saving Throws}: Fortitude +34, Reflexes +28, Will +29

\textbf{Skills} all +15

\textbf{Damage Resistances} cold, lightning, fire

\textbf{Damage Immunity} Void, Poison; weapons +2

\textbf{Condition Immunity} charmed, poisoned, paralyzed, fatigued, frightened

\textbf{Senses} True vision 40 m

\textbf{Languages} all, telepathy 45 m

\textbf{Challenge} 26 (90000 PX)

\emph{\textbf{Spells.}} The Demogorgon has CM 20. Its spellcasting characteristic is Charisma. The Demogorgon knows the following spells:

At will: Detect Magic, Major Image

level 3 (4 slots): \emph{dispel magic, fear, telekinesis}

level 4 (1 slot): \emph{projected image, mental regression}

\emph{\textbf{Demonic Nature.}} The Demogorgon has no need for air, food, drink, or sleep.

€16322 € {€16323 € {Legendary Resistance (3 / Day).}} If the Demogorgon fails a saving throw, it can choose to succeed instead.

\emph{\textbf{Turn Resistance.}} The Demogorgon has +1d6 on saving throws against effects that turn undead.

\emph{\textbf{Two heads.}} Demogorgon has +1d6 on saving throws against being blind, deaf, unconscious

\textbf{Shares}

\emph{\textbf{Multiattack.} 2 tentacle attacks}: +30, reach 10 feet, one creature. All Demogorgon attacks are treated as +2 magical.

\emph{Hits:} 35 (4d12 +9) bludgeoning damage. The affected creature must make a Fortitude save at DC 36 or its maximum hit points drop by the same amount.

\emph{\textbf{Stance}} Demogorgon stares at a creature it can see within 40 meters. The target must make a Will save at DC 23.

\emph{Gaze Effect:} Demogorgon chooses one of these effects or randomly:

1. Powerful Look. The target is unconscious until the next round or until the Demogorgon is out of line of sight

2. Hypnotic Gaze. The target is dominated by the Demogorgon who determines every action. This look requires the use of both heads of the Demogorgon.

3. Gaze of Madness. The target is under the influence of the Confusion spell which lasts, without further saving throw, as long as the Demogorgon is within sight. The Demogorgon does not have to remain concentrated for the effect to last.

\textbf{Additional Shares}

The Demogorgon can take 3 additional actions, chosen from those below and one per round only at the end of another creature's round.

\textbf{Tail.} The Demogorgon attacks with its tail. +30 to hit, range 5 meters, one target. If it hits, it deals 31 hit points of bludgeoning damage plus 4d6 void damage

\textbf{Gaze of Madness.} Demogorgon uses either the Powerful Gaze or the Gaze of Madness

\textbf{Ecology}\\
Environment: Abyss\\
Organization: Unique\\
\textbf{Treasure}: Triple\\

\textbf{Description}
Demogorgon is an enormous demon, prince of the abyss and madness, about 5 meters tall. It appears as a bipedal reptilian with two baboon heads, its necks are long and serpentine like its tentacled arms. The two heads of Demogorgon have distinct personalities that detest each other. They often attempt to dominate each other, and many of the stories involving the Demogorgon deal with how one or the other head tries to dominate the whole. There is a strong rivalry between the Demogorgon and Orcus.


\medskip\index[Monstery]{Dretch}\textbf{Dretch}

\emph{Little fiend (demon), chaotic evil}

\textbf{STRENGTH} +0

\textbf{DEXTERITY} +0

\textbf{CONSTITUTION} +1

\textbf{INTELLIGENCE} -3

\textbf{WISDOM} -1

\textbf{CHARISMA} -4

\textbf{Initiative} +0 -- \textbf{Defense} 12

\textbf{Hit Points} 18 (4d6 + 4)

\textbf{Movement} 6 m

\textbf{Saving Throws} Fortitude +1, Reflexes +0, Will -1

\textbf{Damage Resistances} cold, lightning, fire

\textbf{Damage Immunity} Poison

\textbf{Condition Immunity} poisoned

\textbf{Damage Vulnerability} cold iron

\textbf{Senses} Darkvision 18 m

\textbf{Languages} Abyssal, telepathy 60 ft. (works only with creatures that understand Abyssal)

\textbf{Challenge} 1/4 (50 PX)

\textbf{Shares}

\emph{\textbf{Multiattack.}} The demon makes two attacks: one with its bite and one with its claws.

\emph{\textbf{Claws.} Melee weapon attack}: +2 to hit, reach 1 m, one target.

\emph{Hits:} 5 (2d4) slashing damage.

\emph{\textbf{Bite.} Melee weapon attack}: +2 to hit, reach 1 m, one target.

\emph{Hits:} 3 (1d6) piercing damage.

\emph{\textbf{Fetid Cloud (1/Day).}} A foul green gas extends within a 10-foot radius of the demon. The gas travels around the corners, and its area is slightly darkened. Remains for 1 minute or until blown away by a strong wind. Any creature that begins its round in that area must succeed on a DC 11 Fortitude saving throw or become poisoned until the start of its next round. While poisoned in this way, the target can take only one action or reaction during its round.

\textbf{Ecology}\\
Environment: Any (Abyss)\\
Organization: Solo, couple, gang (3-5), group (6-12) or crowd (13+)\\
\textbf{Treasure}: None\\
\textbf{Description}\\
Even the lowest demon of the Abyss is dangerous and possesses the compelling need to spread ruin and dismay. The wretched dretch is as hideous and fetid as he is cruel, though he lacks the strength and power to satisfy his urge to brutalize others in his home realm. The dretch's purpose for existence is to serve more powerful demons as expendable victims, and only a lucky few manage to survive long enough to evolve.

Dretches are a favorite target for amateurs in abyssal summons. Relatively weak and easy to intimidate, dretches can often be forced into long periods of servitude using vague promises of opportunities to vent their frustrations and anger against weaker adversaries. Yet the potential dretch summoner would do well to remember that these demons are just as cowardly and treacherous as other demons. A dretch faced with a more powerful enemy will be only too happy to trade any information he has for his miserable life.

Unlike most demons, the dretch's sloppy personality and disdain for prolonged physical labor rarely yield results. Advanced dretches are rare, but those who can find the strength within themselves to become more than they were at the time of their creation become the poor rulers of the Abyss, cruel and embittered, ruling over parasites, broken souls, mindless dead and other dretches. Their empires are limited to abandoned stretches of sewers beneath forgotten cities, unstable swampy wastes shunned by most sensible minds, and other unwelcome corners of the Abyss that even demons find inconvenient or loathsome. Yet to the dretch lords these kingdoms are their empires, and they defend them with merciful tenacity.

A dretch is 1.2 meters tall and weighs 90 kg. Dretches are usually formed from the souls of evil and indolent mortals: only a small fragment of soul is enough to give rise to such a horrifying birth. A single soul can often cause a small army of dretches to appear, and the sight of a horde of newly hatched dretches breaking free from the pulsating protomatter of the Abyss is both sickening and terrifying.

\medskip\index[Monstruary]{Glabrezu}\textbf{Glabrezu}

\emph{Great fiend (demon), chaotic evil}

\textbf{STRENGTH} +5

\textbf{DEXTERITY} +2

\textbf{CONSTITUTION} +5

\textbf{INTELLIGENCE} +4

\textbf{WISDOM} +3

\textbf{CHARISMA} +3

\textbf{Initiative} +4 -- \textbf{Defense} 22

\textbf{Hit Points} 157 (15d10 + 75)

\textbf{Movement} 12 m

\textbf{Saving Throws} Fortitude +18, Reflexes +4, Will +11

\textbf{Damage Resistances} cold, lightning, fire; from a non-magical weapon

\textbf{Damage Immunity} Poison

\textbf{Condition Immunity} poisoned

\textbf{Damage Vulnerability} cold iron

\textbf{Senses} true vision 36 m

\textbf{Languages} Abyssal, telepathy 36 m

\textbf{Challenge} 9 (5000 PX)

\emph{\textbf{Innate Spells.}} The demon's spellcasting ability is Intelligence. The demon can cast these spells innately, without the need for material components:

At will: \emph{dispel magic, detect magic, darkness}

1/day each: \emph{confusion, power word stun, fly}

\emph{\textbf{Magic Resistance.}} The demon has +1d6 on saving throws against spells and other magical effects.

\textbf{Shares}

\emph{\textbf{Multiattack.}} The demon makes four attacks: two with its claws and two with its fists. Alternatively, he can make two claw attacks and cast a spell.

\emph{\textbf{Chela.} Melee weapon attack}: +14 to hit, reach 10 ft., one target.

\emph{Hits:} 16 (2d10 + 5) bludgeoning damage. If the target is a Medium or smaller creature, it is grappled (DC 15 to escape). The glabrezu has two claws, each of which can grasp a target.

\emph{\textbf{Fist.} Melee weapon attack}: +14 to hit, reach 1 m, one target.

\emph{Hits:} 7 (2d4 + 2) bludgeoning damage.

\emph{\textbf{Angry:}} the glabrezu creates a duplicate of itself from the shadow plane. This duplicate has the same characteristics as the glabrezu but does not attack. When attacking the glabrezu you have a 50% chance of attacking the shadow duplicate.

\textbf{Ecology}\\
Environment: Any (Abyss)\\
Organization: Solo or squad (1 glabrezu, 1 Succubus and 2-5 Vrock)
\textbf{Treasury}: Standard\\
\textbf{Description}\\
While the succubus is a demon who lures its prey by exploiting their carnal desires and needs, the glabrezu is a tempter of another kind. Ferocious and bestial in form, the glabrezu is actually a master of deception and lies. With his ability to hide his true form behind pleasing illusions, he uses his magic to grant the wishes of mortal humanoids, as a form of reward for those who succumb to his tricks and deceptions. A wish granted by a glabrezu satisfies the expresser's need in the most disastrous way possible, although these consequences may not immediately prove so. A blacksmith struggling to establish himself may desire fame and skill in his chosen profession, only to discover that his best patron is a cruel, sadistic murderer who uses weapons to further his own destructive desires. A lonely man who expresses the desire to have a partner could see his wish come true with an old flame returning to life in the form of a vampire, and other examples of this type. The glabrezu is very creative in fulfilling a mortal's desires.

A glabrezu is 5.3 meters tall and weighs just over 3000 kg. These evil demons originate from the souls of traitors, deceivers and subversives: souls of mortals who, in life, swore falsely or used betrayal and deception to ruin the lives of others.

\medskip\index[Monstery]{Hezrou}\textbf{Hezrou}

\emph{Great fiend (demon), chaotic evil}

\textbf{STRENGTH} +4

\textbf{DEXTERITY} +3

\textbf{CONSTITUTION} +5

\textbf{INTELLIGENCE} 5 (-2)

\textbf{WISDOM} +1

\textbf{CHARISMA} +1

\textbf{Initiative} +3 -- \textbf{Defense} 20

\textbf{Hit Points} 136 (13d10 + 65)

\textbf{Movement} 9 m

\textbf{Saving Throws} Fortitude +16, Reflexes +3, Will +9

\textbf{Damage Resistances} cold, lightning, fire; from a non-magical weapon

\textbf{Damage Immunity} Poison

\textbf{Condition Immunity} poisoned

\textbf{Damage Vulnerability} cold iron

\textbf{Senses} darkvision 36 m

\textbf{Languages} Abyssal, telepathy 36 m

\textbf{Challenge} 8 (3900 PX)

\emph{\textbf{Stench.}} Any creature that begins its round within 10 feet of the demon must succeed on a DC 14 Fortitude saving throw or be poisoned, -1 Strength and Dexterity, until the start of its round. On a successful save, the creature is immune to the croaking demon's stench for 24 hours.

\emph{\textbf{Magic Resistance.}} The demon has +1d6 on saving throws against spells and other magical effects.

\textbf{Shares}

\emph{\textbf{Multiattack.}} The demon makes three attacks: one with its bite and two with its claws.

\emph{\textbf{Claw.} Melee weapon attack}: +11 to hit, reach 1 m, one target.

\emph{Hits:} 11 (2d6 + 4) slashing damage, 2 bleed damage.

\emph{\textbf{Bite.} Melee weapon attack}: +11 to hit, reach 1 m, one target.

\emph{Hits:} 15 (2d10 + 4) piercing damage and minor Demonic Fever disease.

\emph{Minor Demonic Fever}: 1 minute, Fortitude save DC 18, 6 hours, 3 successes, -1 Constitution and Wisdom.

\emph{\textbf{Angry:}} Hezrou releases an incendiary cloud of stench. All creatures around him within 10 feet must make a DC 18 Reflex saving throw to halve the 4d10 fire damage. It costs 2 Actions.

\textbf{Ecology}\\
Environment: Any (Abyss)\\
Organization: Solo or gang (2-4)\\
\textbf{Treasury}: Standard\\
\textbf{Description}\\
the hezrou lives in the vast swamps, marshes, and waterways of the Abyss, equally at home in water and on land. The presence of a hezrou has a detrimental effect on flora, causing knots and mutations, and surrounding waters, making them smelly and salty-tasting, peculiarities more easily identifiable on the Material Plane than in the Abyss. Prolonged exposure to this corruption causes horrific transformations and deformities. Often entire isolated communities of deformed mutants owe their twisted appearance less to their depraved customs than to the proximity of a hezrou.

While quite intelligent, a hezrou can honestly be said to waste his intellect. These beings prefer the simplest pleasures: sleep, the taste of torture, the bliss of feeding on living flesh, or the joy of feeling something beautiful break and crumble in the grip of their fists. They do not often seek to build empires or lead cults, although few hezrou would turn down potential followers who come to them of their own free will.

These monstrous, bestial creatures are born from the souls of evil mortals who have poisoned themselves, their relatives, or their environment, for example, drug addicts, murderers, and alchemists who did not care how their experiments poisoned the natural world.

\medskip\index[Monstery]{Marilith}\textbf{Marilith}

\emph{Great fiend (demon), chaotic evil}

\textbf{STRENGTH} +4

\textbf{DEXTERITY} +5

\textbf{CONSTITUTION} +5

\textbf{INTELLIGENCE} +4

\textbf{WISDOM} +3

\textbf{CHARISMA} +5

\textbf{Initiative} +5 -- \textbf{Defense} 26

\textbf{Hit Points} 189 (18d10 + 90)

\textbf{Movement} 12 m

\textbf{Saving Throws} Fortitude +21, Reflexes +21, Will +19

\textbf{Damage Resistances} cold, lightning, fire

\textbf{Damage Immunity} Poison, weapons +1

\textbf{Condition Immunity} poisoned

\textbf{Damage Vulnerability} cold iron

\textbf{Senses} true vision 36 m

\textbf{Languages} Abyssal, telepathy 36 m

\textbf{Challenge} 16 (15000 PX)

\emph{\textbf{Magical Weapons.}} The demon's weapon attacks are magical.

\emph{\textbf{Reactive.}} the marlith can make a Parry Reaction during each round.

\emph{\textbf{Magic Resistance.}} The demon has +1d6 on saving throws against spells and other magical effects.

\textbf{Shares}

\emph{\textbf{Multiattack.}} The demon makes seven attacks: six with its longswords and one with its tail.

\emph{\textbf{Tail.} Melee weapon attack}: +18 to hit, reach 10 ft., one creature.

\emph{Hits:} 15 (2d10 + 4) bludgeoning damage. If the target is Medium or smaller, it is grappled (DC 19 to escape). Until the grapple ends, the target is restrained, and the demon can automatically strike the target with its tail, but cannot make tail attacks against other targets.

\emph{\textbf{Long Sword.} Melee weapon attack}: +18 to hit, reach 1 m, one target.

\emph{Hits:} 13 (2d8 + 4) slashing damage.

\textbf{Reactions}

\emph{\textbf{Parry.}} The demon adds 5 to its Defense against a melee attack that would hit it. To do so, the demon must be able to see its attacker and wield a melee weapon.

\emph{\textbf{Angry:}}

- the marilith sharpens her swords together, each longsword attack gains Bleeding 1/20. Costs 2 Actions, lasts until the end of the fight.

- the marilith condemns the opponent to the abyss. Cost 2 Actions. The opponent must make a DC 29 Will save or be transported into the abyss.

\textbf{Ecology}\\
Environment: Any (Abyss)\\
Organization: Solo, pair or platoon (1 marilith, 1-3 Glabrezu and 3-14 Babau)\\
\textbf{Treasure}: Double (6 Long Swords, more treasure)\\
\textbf{Description}\\
Rulers of demonic hordes and queens of abyssal nations, fearsome mariliths serve demon lords as rulers, advisors, and even lovers, yet their supremacy as strategists makes them especially sought after as generals and army commanders. The most powerful mariliths serve no one and instead command ravenous demonic legions.

A marilith is 1.8 to 2.7 meters tall, 6 meters long from head to tip of tail, and weighs 2000 kg. Only the most arrogant and proud of evil souls, usually those of cruel rulers, sadistic generals, and particularly violent warlords, can cause the birth of a marilith.

\medskip\index[Monstery]{Nalfeshnee}\textbf{Nalfeshnee}

\emph{Great fiend (demon), chaotic evil}

\textbf{STRENGTH} +5

\textbf{DEXTERITY} +0

\textbf{CONSTITUTION} +6

\textbf{INTELLIGENCE} +4

\textbf{WISDOM} +1

\textbf{CHARISMA} +2

\textbf{Initiative} +4 -- \textbf{Defense} 25

\textbf{Hit Points} 184 (16d10 + 96)

\textbf{Movement} 6 m, flight 9 m

\textbf{Saving Throws} Fortitude +22, Reflexes +9, Will +21

\textbf{Damage Resistances} cold, lightning, fire; from a non-magical weapon

\textbf{Damage Immunity} Poison

\textbf{Condition Immunity} poisoned

\textbf{Damage Vulnerability} cold iron

\textbf{Senses} darkvision 36 m

\textbf{Languages} Abyssal, telepathy 36 m

\textbf{Challenge} 13 (10000 PX)

\emph{\textbf{Magic Resistance.}} The demon has +1d6 on saving throws against spells and other magical effects.

\textbf{Shares}

\emph{\textbf{Multiattack.}} The demon uses, if possible, Halo of Horror. It then makes three attacks: one with its bite and two with its claws.

\emph{\textbf{Claw.} Melee weapon attack}: +18 to hit, reach 10 ft., one target.

\emph{Hits:} 15 (3d6 + 5) slashing damage, 2 bleed damage.

\emph{\textbf{Bite.} Melee weapon attack}: +18 to hit, reach 1 m, one target.

\emph{Hits:} 32 (5d10 + 5) piercing damage and Demonic Fever.

\emph{Demonic Fever}: 1 minute, Fortitude save DC 23, 4 hours, 3 successes, -1 Constitution and Wisdom/4 hours.

\emph{\textbf{Halo of Horror (Recharge 5-6).}} The demon emits a multi-coloured, shimmering magical light. Each creature within 5 feet of the demon that can see the light must succeed on a DC 15 Will save or be frightened for 1 minute. A creature can repeat the saving throw at the end of each of its rounds, ending the effect for itself on a success. If the creature's saving throw succeeds or the effect ends for it, the creature is immune to the Halo of
Horror of the demon groaned for the next 24 hours.

\emph{\textbf{Teleport.}} The demon teleports, along with any equipment it is wearing or carrying, to an unoccupied space that it can see up to 120 feet away. It is a Move Action.

€16,543 € {€16,544 € {Angry:}} The nalfeshnee mimes arcane words and gestures seen within 3 previous rounds and casts a spell he witnesses. He costs 3 Actions.

\textbf{Ecology}
Environment: Any (Abyss)\\
Organization: Solo or warband (1 nalfeshnee, 1 Hezrou and 2-5 Vrock)\\
\textbf{Treasury}: Standard\\
\textbf{Description}\\
Few demons understand the internal mechanics that govern the Abyss like the nalfeshnee, and it is not uncommon for these demons to serve the Abyss itself rather than a demon lord. Some oversee the organic realms that spawn new demons, while others guard places of particular importance in the hidden recesses of the plane. Often the kingdom of a nalfeshnee in the Abyss is superior in strength and size to the largest of mortal kingdoms, as these demons have a natural predisposition to govern and impose a sort of order on the chaos of the Abyss. Mortal summoners often call upon them for their crazy but unparalleled intellect, carefully examining the agreements made with these demons to avoid any hidden consequences and unintended consequences, as a nalfeshnee rarely accepts something that, in some twisted way, does not allow it to satisfy the needs and desires of the Abyss.

Nalfeshnee are 6 meters tall and weigh 4000 kg. They are created from the souls of evil greedy or greedy mortals, especially those who have reigned over empires of slavery, theft, brigandage, and other even more violent vices.

\medskip\index[Monstery]{Orcus}\textbf{Orcus}

\emph{huge fiend (demon prince), chaotic evil}

\textbf{STRENGTH} +8

\textbf{DEXTERITY} +2

\textbf{CONSTITUTION} +7

\textbf{INTELLIGENCE} +5

\textbf{WISDOM} +5

\textbf{CHARISMA} +7

\textbf{Initiative} +5 -- \textbf{Defense} 30

\textbf{Hit Points} 390 (26d8+182)

\textbf{Move} 15 meters, fly 15 meters

\textbf{Saving Throws}: Fortitude +33, Reflexes +28, Will +31

\textbf{Skills} all +13

\textbf{Damage Resistances} cold, lightning, fire

\textbf{Damage Immunity} Void, Poison; weapons +2

\textbf{Condition Immunity} charmed, poisoned, paralyzed, fatigued, frightened

\textbf{Senses} True vision 40 m

\textbf{Languages} all, telepathy 45 m

\textbf{Challenge} 26 (90000 PX)

\emph{\textbf{Spells.}} Orcus has CM 17. His spellcasting ability is Charisma. Orcus knows the following spells:

At will: Detect Magic, Chilling Touch

level 3 (3 slots): \emph{Dispel spells}

level 6 (3 slots): \emph{Create undead}

level 9 (1 slot): \emph{Stop time}

\emph{\textbf{Demonic Nature.}} Orcus has no need for air, food, drink, or sleep.

€16577 € {€16578 € {Legendary Resistance (3 / Day).}} If the Orcus fails a saving throw, he can choose to succeed instead.

€16579 € {€16580 € {Lord of the undead.}} Orcus can always decide the type of undead he creates and this remains under his control indefinitely, furthermore he can cast the spell in any condition you find.

\textbf{Shares}

\emph{\textbf{Multiattack.} 2 wand attacks}: +30, reach 10 feet, one creature. All of Orcus' attacks are treated as +3 magical.

\emph{Hits:} 21 (3d8 + 8) bludgeoning damage + 13 (2d12) void damage

\emph{\textbf{Tail}} Orcus strikes with his tail. +30, reach 10 feet, one creature

\emph{Hits:} 21 (3d8 + 8) bludgeoning damage + 18 (4d8) Poison damage

\textbf{Additional Shares}

The Orcus can take 3 additional actions, chosen from those below and one per round only at the end of another creature's round.

\textbf{Tail.} The Orcus attacks with its tail. +30 to hit, range 5 meters, one target. If hits 21 (3d8 + 8) bludgeoning damage + 18 (4d8) Poison damage

\textbf{Taste of Death.} Orcus casts the spell Fiery Strike, blasphemously, with Void damage

\textbf{Ecology}\\
Environment: Abyss\\
Organization: Unique\\
\textbf{Treasure}: Triple\\

\textbf{Description}
Orcus is the Demon Prince of the undead. He favors the company and service of the undead. He wishes to see all life disappear and it all transform into a gigantic necropolis of the undead under his command. Orcus has the head and legs of a goat, ram-like horns, a bloated body, bat-like wings, and a long tail.


\medskip\index[Monstery]{Quasit}\textbf{Quasit}

\emph{Low fiend (demon, shapeshifter), chaotic evil}

\textbf{STRENGTH} -3

\textbf{DEXTERITY} +3

\textbf{CONSTITUTION} +0

\textbf{INTELLIGENCE} -2

\textbf{WISDOM} +0

\textbf{CHARRISMA} +0

\textbf{Initiative} +3 -- \textbf{Defense} 14

\textbf{Hit Points} 7 (3d4)

\textbf{Move} 12 m (3 m, fly 12 m in bat form; 12 m, climb 12 m in centipede form; 12 m, swim 12 m in toad form)

\textbf{Saving Throws} Fortitude +1, Reflexes +3, Will +0

\textbf{Skills} Stealth +5

\textbf{Damage Resistances} cold, lightning, fire; from a non-magical weapon

\textbf{Damage Immunity} Poison

\textbf{Condition Immunity} poisoned

\textbf{Senses} darkvision 36 m

\textbf{Languages} Abyssal, Municipality

\textbf{Challenge} 1 (200 PX)

€16616 € {€16617 € {Shapeshifter.}} The demon can use its action to transform into a bestial bat, centipede, or toad form, or to return to its true form. His stats are the same in all forms, although attacks may vary for some of them. Whatever equipment he is wearing or carrying is not transformed. Upon death he returns to his true form.

\emph{\textbf{Magic Resistance.}} The demon has +1d6 on saving throws against spells and other magical effects.

\textbf{Shares}

\emph{\textbf{Claws (Beast Form Bite).} Melee weapon attack}: +4 to hit, reach 1 m, one target. \emph{Hits:} 5 (1d4 + 3) piercing damage. If the target is a creature, it must succeed on a DC 10 Fortitude saving throw or take 5 (2d4) poison damage and be poisoned, -1 Strength and Dexterity, for 1 minute. The creature can repeat the saving throw at the end of each of its rounds, ending the effect on a success.

\emph{\textbf{Invisibility.}} The demon remains invisible until it attacks or ends its concentration. Whatever the demon is carrying or wearing remains invisible as long as it remains in contact with the demon.

\emph{\textbf{Fright (1/Day).}} A creature chosen by the demon that is within 20 feet of it must succeed on a DC 10 Will saving throw or be frightened for 1 minute. The target can repeat the saving throw at the end of each of its rounds, with -1d6 if the demon is in line of sight, ending the effect prematurely if the saving throw succeeds.

\textbf{Ecology}\\
Environment: Any (Abyss)\\
Organization: Solitary or flock (2-12)\\
\textbf{Treasury}: Standard\\
\textbf{Description}\\
The quasit may be the least powerful demon, but it is not among the least respected: even quasits believe themselves superior to the Dretch hordes, and true to their nature, the Dretch lack the courage or drive to prove them wrong. A quasit's primary role in life is that of a familiar in service to a spellcaster, but those quasits who escape this humiliating servitude gain a will of their own and are far more dangerous. A typical quasit is 45 centimeters tall and weighs only 4 kg.

Unique among the demonic hordes, quasits are not born from the souls of evil deceased mortals, but from living souls: when a spellcaster tries to call upon a quasit as a familiar, his soul touches the Abyss and it reacts, creating from its matter a quasit connected to the caster's soul and generating a powerful bond between the two.

Newly created quasits are born directly into the Material Plane, where they become familiars and, as long as they are subject to their master's will, they hate and despise him, since they can feel the pulse of his soul and know that they could aspire to something more . A quasit serves, yet watches and watches for mistakes that could cost its master his life, or rather, that allow him to turn against his master. When a quasit's owner dies, it can attempt to follow its soul into the Great Beyond, succeeding at a DC 15 Will save. This effect functions as planar shift but only affects the quasit and transports it into the Abyss, causing it to become the master's soul is his, in larva form, rather than using it to create new demonic life forms. In this way, a quasit can use its newly captured soul to bargain with more powerful denizens of the lower planes, and perhaps achieve a vile promotion that transforms it into a more powerful lifeform.

Rarely does a quasit choose to ignore its master's death and remain on the Material Plane in search of other ways to entertain itself—usually settling in an urban area where there are many individuals to torment.

\medskip\index[Monster]{Vrock}\textbf{Vrock}

\emph{Great fiend (demon), chaotic evil}

\textbf{STRENGTH} +3

\textbf{DEXTERITY} +2

\textbf{CONSTITUTION} +4

\textbf{INTELLIGENCE} -1

\textbf{WISDOM} +1

\textbf{CHARRISMA} -1

\textbf{Initiative} +2 -- \textbf{Defense} 18

\textbf{Hit Points} 104 (11d10 + 44)

\textbf{Movement} 12 m, flight 18 m

\textbf{Saving Throws} Fortitude +13, Reflexes +10, Will +6

\textbf{Damage Resistances} cold, lightning, fire; from a non-magical weapon

\textbf{Damage Immunity} Poison

\textbf{Condition Immunity} poisoned

\textbf{Senses} darkvision 36 m

\textbf{Languages} Abyssal, telepathy 36 m

\textbf{Challenge} 6 (2300 PX)

\emph{\textbf{Magic Resistance.}} The demon has +1d6 on saving throws against spells and other magical effects.

\textbf{Shares}

\emph{\textbf{Multiattack.}} The demon makes two attacks: one with its beak and one with its spurs.

\emph{\textbf{Beak.} Melee weapon attack}: +12 to hit, reach 1 m, one target.

\emph{Hits:} 10 (2d6 + 3) piercing damage.

\emph{\textbf{Spurs.} Melee weapon attack}: +12 to hit, reach 1 m, one target.

\emph{Hits:} 14 (2d10 + 3) slashing damage.

\emph{\textbf{Spores (Recharge 6).}} A cloud of toxic spores spreads in a 5 meter radius around the demon. Spores spread around corners. Each creature in that area must succeed on a DC 14 Fortitude saving throw or become poisoned. While poisoned in this way, a target takes 5 (1d10) poison damage at the start of each of its rounds. The target can repeat the saving throw at the end of each of its rounds, ending the effect on a success. Emptying a vial of Holy Water onto the target also ends the effect.

\emph{\textbf{Stunning Screech (1/Day).}} The demon lets out a horrifying screech. Each creature within 20 feet of it that can hear it, and is not a demon, must succeed on a DC 14 Fortitude saving throw or be stunned until the end of the demon's next round.

\emph{\textbf{Angry:}} The Vrock scrapes its beak with its spurs making them even sharper. Until the end of the combat, the damage caused by Beaks and Spurs causes 1 Bleeding damage up to a maximum of 10 damage.

\textbf{Ecology}\\
Environment: Any (Abyss)\\
Organization: Solo, couple or gang (3-10)\\
\textbf{Treasury}: Standard\\
\textbf{Description}\\
Profane champions of the Abyss, vrocks embody all the anger, hatred, and violence of this realm. As voracious and grotesquely opportunistic as the scavenger they resemble, vrocks delight in the shedding of blood, enjoying the sound and sensation of ripping the still-throbbing intestines from a living creature.\\
A typical vrock is 2.3 meters tall and weighs 200 kg. These creatures usually originate from the souls of evil mortals filled with hatred and anger, particularly those who were professional criminals, mercenaries, or assassins.



\medskip\index[Monster]{Nightmare Steed}\textbf{Nightmare Steed}

\emph{Great fiend, neutral evil}

\textbf{STRENGTH} +4

\textbf{DEXTERITY} +2

\textbf{CONSTITUTION} +3

\textbf{INTELLIGENCE} +0

\textbf{WISDOM} +1

\textbf{CHARISMA} +2

\textbf{Initiative} +2 -- \textbf{Defense} 15

\textbf{Hit Points} 68 (8d10 + 24)

\textbf{Movement} 18 m, flight 24 m

\textbf{Saving Throws} Fortitude +6, Reflexes +5, Will +4

\textbf{Damage Immunity} Fire

\textbf{Languages} includes Abyssal, Common and Infernal but cannot speak

\textbf{Challenge} 3 (700 PX)

\emph{\textbf{Grant Fire Resistance.}} The nightmare steed can grant resistance to fire damage to anyone who rides it.

\emph{\textbf{Lighting.}} The nightmarish steed radiates bright light in a 10-foot radius and dim light for an additional 10 feet.

\textbf{Shares}

\emph{\textbf{Hooves.} Melee weapon attack}: +6 to hit, reach 10 ft., one target.

\emph{Hits:} 13 (2d8 + 4) bludgeoning damage plus 7 (2d6) fire damage.

\emph{\textbf{Ethereal Pass.}} The nightmarish steed and up to three willing creatures within 10 feet of it can magically enter the Ethereal Plane from the Material Plane and vice versa.

\textbf{Ecology}\\
Environment: Any\\
Organization: Solitaire\\
\textbf{Treasure}: None\\
\textbf{Description}\\
Nightmares are flaming harbingers of death. They allow only the most evil creatures to ride them, and are never just mounts, but collaborate in the destruction caused by their riders.


\subsection{Devils}

\begin{changemargin}{0.3cm}{0.3cm}\begin{enfasi}{Hell is empty, all the devils are here. (William Shakespeare, The Tempest)}\end{enfasi}\end{changemargin}\medskip

\medskip\index[Monstery]{Bearded Devil}\textbf{Bearded Devil}

\emph{Medium fiendish (devil), lawful evil}

\textbf{STRENGTH} +3

\textbf{DEXTERITY} +2

\textbf{CONSTITUTION} +2

\textbf{INTELLIGENCE} -1

\textbf{WISDOM} +0

\textbf{CHARISMA} +0

\textbf{Initiative} +2 -- \textbf{Defense} 15

\textbf{Hit Points} 52 (8d8 + 16)

\textbf{Movement} 9 m

\textbf{Saving Throws} Fortitude +9, Reflexes +7, Will +3

\textbf{Damage Resistances} cold; from a non-magical or non-silver weapon

\textbf{Damage Immunity} Fire, poison

\textbf{Condition Immunity} poisoned

\textbf{Senses} darkvision 36 m

\textbf{Languages} Infernal, telepathy 36 m

\textbf{Challenge} 3 (700 PX)

\emph{\textbf{Magic Resistance.}} The devil has +1d6 on saving throws against spells and other magical effects.

\emph{\textbf{Resolved.}} The devil cannot be frightened as long as he can see an allied creature within 30 feet of him.

\emph{\textbf{Devil's Sight.}} The devil's darkvision is not limited by magical darkness.

\textbf{Shares}

\emph{\textbf{Multiattack.}} The devil makes two attacks: one with his beard and one with his glaive.

\emph{\textbf{Beard.} Melee weapon attack}: +7 to hit, reach 3 ft., one creature.

\emph{Hits:} 6 (1d8 + 2) piercing damage, and the target must succeed on a DC 12 Fortitude saving throw or be poisoned for 1 minute. While poisoned in this way, the target cannot regain Hit Points. The target can repeat the saving throw at the end of each of its rounds, ending the effect on a successful save.

\emph{\textbf{Glaffer.} Melee weapon attack}: +7 to hit, reach 3, one target.

\emph{Hits:} 8 (1d10 + 3) slashing damage. If the target is a creature, excluding constructs and undead, it must succeed on a Fortitude save 12 or lose 5 (1d10) hit points at the start of each of its rounds due to the infernal wound. Each time the devil hits the wounded target with this attack, the damage dealt by the wound increases by 5 (1d10). Any creature can take an action to block the wound with a successful DC 12 Wisdom (First Aid) check. The wound also closes if the target receives healing magic.

\textbf{Ecology}\\
Environment: Any (Hell)\\
Organization: Solo, couple, team (3-10) or troop (10-40)\\
\textbf{Treasure}: Standard (Glaive, other treasure)\\
\textbf{Description}\\
Elite warriors of the infernal legions, the bearded devils, or barbazu, fight savagely in the name of their infernal masters and command brutal hordes of the damned in battle. They gather and train with their glaives forged in the underworld, among the vaults of the third circle of Hell, Erebus, but inevitably return to the first circle, Avernus, to serve alongside the fearsome lord Barbatos.

The barbazu love to make charging attacks with their glaives and try to maintain a distance of 3 meters between themselves and their opponents, so that they can use their characteristic polearms to maximum effectiveness. Against an opponent who has superior reach (or is able to avoid the devil's favorite tactic), they drop their glaives and rely on their claws and hideous beards. When standing, bearded devils stand more than 6 feet tall (although their crouched position in battle often makes them appear shorter) and weigh more than 200 pounds.


\medskip\index[Monstrorium]{Chain Devil}\textbf{Chain Devil}

\emph{Medium fiendish (devil), lawful evil}

\textbf{STRENGTH} +4

\textbf{DEXTERITY} +2

\textbf{CONSTITUTION} +4

\textbf{INTELLIGENCE} +0

\textbf{WISDOM} +1

\textbf{CHARISMA} +2

\textbf{Initiative} +2 -- \textbf{Defense} 20

\textbf{Hit Points} 85 (10d8 + 40)

\textbf{Movement} 9 m

\textbf{Saving Throws} Fortitude +9, Reflexes +4, Will +3

\textbf{Damage Resistances} cold; from a non-magical or non-silver weapon

\textbf{Damage Immunity} Fire, poison

\textbf{Condition Immunity} poisoned

\textbf{Senses} darkvision 36 m

\textbf{Languages} Infernal, telepathy 36 m

\textbf{Challenge} 8 (3900 PX)

\emph{\textbf{Magic Resistance.}} The devil has +1d6 on saving throws against spells and other magical effects.

\emph{\textbf{Devil's Sight.}} The devil's darkvision is not limited by magical darkness.

\textbf{Shares}

\emph{\textbf{Multiattack.}} The devil makes two chain attacks.

\emph{\textbf{Chain.} Melee weapon attack}: +16 to hit, reach 10 ft., one target.

\emph{Hits:} 11 (2d6 + 4) slashing damage. The target is grappled (DC 14 to escape) if the devil isn't already grappling another creature. Until the grapple ends, the target is restrained and takes 7 (2d6) piercing damage at the start of each of its rounds.

\emph{\textbf{Animate Chains (Recharge after 1 hour).}} Up to four chains that the devil can see and are within 60 feet of him produce sharp edges and animate under the devil's control, as long as those chains are neither worn nor carried by anyone else.

Each animated chain is an object with Defense 20, 20 Hit Points, resistance to piercing damage, and immunity to sonic damage. When the devil uses Multiattack during his round, he can use each animated chain to make an additional chain attack. An animated chain can grab a creature on its own but can't make attacks while grabbing. An animated chain reverts to its inanimate state if it is reduced to 0 hit points or if the devil is incapacitated or dies.

\textbf{Reactions}

\emph{\textbf{Unnerving Mask.}} When a creature the devil can see begins its round within 30 feet of the devil, the devil can create an illusion to resemble that lost love or a bitter rival of that one. creature. If the creature can see the devil, it must succeed on a DC 14 Will save or remain frightened until the end of its round.

\emph{\textbf{Angry:}} the Chain Devil waves his chains in front of him. Until the end of the fight the Defense is 23. It costs 1 Action.

\textbf{Ecology}\\
Environment: Any\\
Organization: Solitaire, pair, ring (3-6) or chain (7-20)\\
\textbf{Treasury}: Standard\\
\textbf{Description}
Often classified by laymen among the ranks of infernal devils, sadomasochists are not true devils. While some are known to live in Hell, they exist outside the hierarchies established by the gods of Hell and its archdevils and can sometimes be found on other planes, particularly the Plane of Shadow. Many suggest that they are natives of the Hell that existed before the advent of the diabolical race, although others speculate that they were brought to the plane by some sadistic power. Regardless of their origins, they roam the planes indulging in their desire to cause and receive suffering, seeking pain through violent abductions and sadistic depravities.


\medskip\index[Monstery]{Horned Devil}\textbf{Horned Devil}

\emph{Great fiend (devil), lawful evil}

\textbf{STRENGTH} +6

\textbf{DEXTERITY} +3

\textbf{CONSTITUTION} +5

\textbf{INTELLIGENCE} +1

\textbf{WISDOM} +3

\textbf{CHARISMA} +3

\textbf{Initiative} +3 -- \textbf{Defense} 23

\textbf{Hit Points} 178 (17d10 + 85)

\textbf{Movement} 6 m, flight 18 m

\textbf{Saving Throws} Fortitude +18, Reflexes +17, Will +13

\textbf{Damage Resistances} cold;

\textbf{Damage Immunity} Fire, poison, weapons +1

\textbf{Condition Immunity} poisoned

\textbf{Senses} darkvision 36 m

\textbf{Languages} Infernal, telepathy 36 m

\textbf{Challenge} 11 (7200 PX)

\emph{\textbf{Magic Resistance.}} The devil has +1d6 on saving throws against spells and other magical effects.

\emph{\textbf{Devil's Sight.}} The devil's darkvision is not limited by magical darkness.

\textbf{Shares}

\emph{\textbf{Multiattack.}} The devil makes three melee attacks: two with its pitchfork and one with its tail. He can use Throw Flame in place of any melee attacks.

\emph{\textbf{Tail.} Melee weapon attack}: +18 to hit, reach 10 ft., one target.

\emph{Hits:} 10 (1d8 + 6) piercing damage. If the target is a creature, excluding constructs and undead, it must succeed on a Fortitude 17 or bleed 10 save (3d6). Whenever the devil wounds the target with this attack, the damage dealt by the bleed increases by 10 (3d6).

\emph{\textbf{Pitchfork.} Melee weapon attack}: +18 to hit, reach 10 ft., one target.

\emph{Hits:} 15 (2d8 + 6) piercing damage.

\emph{\textbf{Sting.} Melee weapon attack}: +18 to hit, reach 10 ft., one target.

\emph{Hits:} 13 (2d8 + 4) piercing damage plus 17 (5d6) poison damage, and the target must succeed on a DC 14 Fortitude save, or be poisoned, -1 Strength and Dexterity, for 1 minute . The target can repeat the saving throw at the end of each of its rounds, ending the effect on a success.

\emph{\textbf{Throw Flame.} Ranged spell attack}: +7 to hit, range 45 m, one target.

\emph{Hits:} 14 (4d6) fire damage. If the target is a flammable object that is not being worn or carried, it catches fire.

\emph{\textbf{Angry:}} the Horned Devil drains the life that enemies are losing. Until the end of the next round he recovers all the Hit Points lost by Bleeding due to wounds caused by him.

\textbf{Ecology}\\
Environment: Any (Hell)\\
Organization: Solitary, pair or flock (3-10)\\
\textbf{Treasure}: Standard (Unholy Spiked Chain+1, more treasure)\\
\textbf{Description}\\
Among the deadliest warriors of the archdevils and skilled commanders of the lesser devils, horned devils spread the rules of Hell wherever they pass. These greater devils are trained, forged, and reforged to be among the most relentless and obedient warriors in the multiverse. The horned devils of the infernal armies are known as cornugons, while the largest among them are called malebranche.

A typical horned devil reaches a remarkable height of 2.7 meters, has wings with a span of 4.2 meters, and weighs 350 kg.


\medskip\index[Monstruary]{Pit Devil}\textbf{Pit Devil}\hypertarget{Pit Devil}{}

\emph{Great fiend (devil), lawful evil}

\textbf{STRENGTH} +8

\textbf{DEXTERITY} +2

\textbf{CONSTITUTION} +7

\textbf{INTELLIGENCE} +6

\textbf{WISDOM} +4

\textbf{CHARISMA} +7

\textbf{Initiative} +6 -- \textbf{Defense} 29

\textbf{Hit Points} 300 (24d10 + 168)

\textbf{Movement} 9 m, flight 18 m

\textbf{Saving Throws} Fortitude +27, Reflexes +22, Will +24

\textbf{Damage Resistances} cold;

\textbf{Damage Immunity} Fire, poison, weapons +2

\textbf{Condition Immunity} poisoned

\textbf{Senses} true vision 36 m

\textbf{Languages} Infernal, telepathy 36 m

\textbf{Challenge} 20 (25000 PX)

\emph{\textbf{Magic Weapon.}} The pit devil's weapon attacks are magical.

\emph{\textbf{Aura of Fear.}} Any creature hostile to the devil that begins its round within 20 feet of it must make a DC 21 Will save, unless the devil is incapacitated. If the saving throw fails, the creature is frightened until the start of its next round. If the creature's saving throw succeeds, the creature is immune to the devil's fear aura for the next 24 hours.

\emph{\textbf{Innate Spells.}} The pit devil spellcasting ability is Charisma. The pit fiend can cast these spells innately, without the need for material components:

At will: \emph{detect magic, fireball}

3/day each: \emph{block monsters, wall of fire}

\emph{\textbf{Magic Resistance.}} The devil has +1d6 on saving throws against spells and other magical effects.

\textbf{Shares}

\emph{\textbf{Multiattack.}} The devil makes four attacks: one with its bite, one with its claw, one with its club and one with its tail.

\emph{\textbf{Claw.} Melee weapon attack}: +30 to hit, reach 10 ft., one target.

\emph{Hits:} 17 (2d8 + 8) slashing damage, 3/20 bleed damage.

\emph{\textbf{Tail.} Melee weapon attack}: +30 to hit, reach 10 ft., one target.

\emph{Hits:} 24 (3d10 + 8) bludgeoning damage.

\emph{\textbf{Mace.} Melee weapon attack}: +30 to hit, reach 10 ft., one target.

\emph{Hits:} 15 (2d6 + 8) bludgeoning damage plus 21 (6d6) fire damage.

\emph{\textbf{Bite.} Melee weapon attack}: +30 to hit, reach 1 m, one target.

\emph{Hits:} 22 (4d6 + 8) piercing damage. The target must succeed on a DC 21 Fortitude save or be poisoned. While poisoned in this way, the target cannot regain Hit Points, and takes 21 (6d6) poison damage at the start of each of its rounds. The poisoned target can repeat the saving throw at the end of each of its rounds, ending the effect on itself.

\textbf{Ecology}\\
Environment: Any (Hell)\\
Organization: Solitary, couple or council (3-9)\\
\textbf{Treasure}: Double\\
\textbf{Description}
Rulers of infernal realms, generals of Hell's armies, and advisors to archdevils, pit fiends are the embodiment of the terrible and frightening pinnacle of the diabolical race.

Massive, with indomitable physiques and gifted with ingenious evil intellects, these diabolical tyrants possess great autonomy both in the service of the archdevils and in their sovereignty over the hellish wastes of slaves or when they are busy subjugating the mortal worlds. Solid muscles stretch across their gigantic bodies, armored with thick, cutting plates capable of blocking almost any attack. Their dagger-sized fanged jaws and bestial faces hide some of Hell's most insidious minds.

Born in the depths of Nessus, the ninth and deepest circle of Hell, pit fiends are created from the ranks of the cornugon and gelugon by archdevils and their dukes alone. Although many travel to the upper circles and beyond Hell, in command of the infernal legions, most remain in the Nessus, serving the courts of the mighty of Hell or in dark covens of unnameable purposes.

Pit devils are always more than 4.2 meters tall, with a wingspan of more than 6 meters and weigh more than 500 kg.

Pit devils are lords of fire and prefer territories touched by flames. In Hell, this predisposition means that Averno, Dis, Malebolge, Nessus, and Phlegethon are the groups that most easily host their temple-citadels enveloped in flames. Fanatics obsessed with diabolical superiority and the strictest obedience, the pit devils, if left to act undisturbed, gather immense armies, combing the pits of Hell in search of the most depraved lemures to transform them into true devils. Once they are certain that they have created the perfect legions, they turn their attention to the demiplanes and the most vulnerable mortal worlds, anticipating their conquest.

Servants of archdevils or other unique infernal warlords, pit fiends devote themselves to their cause, obeying the will of the nobles chosen by some dark Patron in the hope that, one day, they will gain the favor of the Prince of Darkness or of Hell itself. While obedient to the hierarchies of their race, they are also strict in enforcing the rules and, if a pit devil found himself serving an unworthy master, he would feel obliged to depose him. Thus, whether lords or servants, pit devils embody the will of the relentless laws of hell and ensure that only the most powerful devils can (or dare) thrive.

Only the most powerful of mortal spellcasters can or dare summon a pit fiend. The reactions of these types of devils to the summoning are swift and premeditated, usually characterized by an uncontainable fury at the idea that such an insignificant being could waste their immortal time. Anyone who is unable to face his burning anger is killed and his soul damned to Hell and placed at the service of the summoned devil. Whoever manages to control these greater devils also manages to intrigue them.

A pit devil may dutifully serve a mortal lord for centuries, but his goal always remains the same: to increasingly corrupt his soul, ensure his complete damnation, and, when he finally dies, claim his soul and begin the process to make him a totally corrupt Lemur servant.

Pit devils are aware that they are immortal and are intelligent enough to have incredibly disciplined patience. Therefore the oldest pit devils see in their legions the faces of the countless madmen who once claimed to be their masters.


\medskip\index[Monstery]{Ice Devil}\textbf{Ice Devil}

\emph{Great fiend (devil), lawful evil}

\textbf{STRENGTH} +5

\textbf{DEXTERITY} +2

\textbf{CONSTITUTION} +4

\textbf{INTELLIGENCE} +4

\textbf{WISDOM} +2

\textbf{CHARISMA} +4

\textbf{Initiative} +4 -- \textbf{Defense} 25

\textbf{Hit Points} 180 (19d10 + 76)

\textbf{Movement} 12 m

\textbf{Saving Throws} Fortitude +18, Reflexes +16, Will +16

\textbf{Damage Immunity} cold, fire, poison, weapons +1

\textbf{Condition Immunity} poisoned

\textbf{Senses} blindsight 60 ft., darkvision 120 ft

\textbf{Languages} Infernal, telepathy 36 m

\textbf{Challenge} 14 (11,500 PX)

\emph{\textbf{Magic Resistance.}} The devil has +1d6 on saving throws against spells and other magical effects.

\emph{\textbf{Devil's Sight.}} The devil's darkvision is not limited by magical darkness.

\textbf{Shares}

\emph{\textbf{Multiattack.}} The devil makes three attacks: one with its bite, one with its claws and one with its tail. Alternatively he makes two attacks: one with his tail and one with a spear.

\emph{\textbf{Claws.} Melee weapon attack}: +21 to hit, reach 1 m, one target.

\emph{Hits:} 10 (2d4 + 5) slashing damage plus 10 (3d6) cold damage, 1 bleed damage.

\emph{\textbf{Tail.} Melee weapon attack}: +21 to hit, reach 10 ft., one target.

\emph{Hits:} 12 (2d6 + 5) bludgeoning damage plus 10 (3d6) cold damage.

\emph{\textbf{Ice Spear.} Melee weapon attack}: +21 to hit, reach 10 ft., one target.

\emph{Hits:} 14 (2d8 + 5) piercing damage plus 10 (3d6) cold damage. If the target is a creature, it must succeed on a DC 15 Fortitude save, or have its speed reduced by 10 feet for 1 minute; during each of his rounds he can only take one action or one bonus action, but not both; he cannot react. The target can repeat the saving throw at the end of each of its rounds, ending the effect on itself on a success.

\emph{\textbf{Bite.} Melee weapon attack}: +10 to hit, reach 1 m, one target.

\emph{Hits:} 12 (2d6 + 5) piercing damage plus 10 (3d6) cold damage. Save on Fortitude DC 18 or Slowed 1/1r.

\emph{\textbf{Wall of Ice (Recharge 6).}} The devil magically forms a wall of opaque ice on a solid surface that he can see within 60 feet of him. The wall is 30 centimeters thick and up to 9 meters wide and up to 3 meters high, or a hemispherical dome up to 6 meters in diameter. When the wall appears, any creature in its space is pushed out of it by the shortest route. The creature chooses which side of the wall to end up on, unless the creature is incapacitated. The creature then makes a DC 17 Reflex saving throw, taking 35 (10d6) cold damage on a failed save, or half as much damage on a successful one.

The wall remains for 1 minute or until the devil is incapacitated or dies. The wall can be damaged and punctured; each 10-foot section has Defense 5, 30 Hit Points, vulnerability to fire damage, and immunity to acid, cold, void, and poison damage. If a section is destroyed, it leaves a patina of freezing air in the space the wall previously occupied. Whenever a creature ends up moving through this freezing air during a round, willing or otherwise, it must make a DC 17 Fortitude saving throw, taking 17 (5d6) cold damage on a failed save, or half as much damage on a failed save. he succeeds. The frigid air dissipates as the rest of the wall vanishes.

\emph{\textbf{Angry:}} the Ice Devil aims at the enemy's heart and tries to tear it out. The creature within 3 feet must make a Fortitude save DC 18 or have its heart torn out.


\textbf{Ecology}\\
Environment: Any (Hell)\\
Organization: Solo, squad (2-3), council (4-10), or contingent (1-3 ice devils, 2-6 horned devils, and 1-4 bone devils\\
\textbf{Treasure}: Standard (Frozen Spear+1, more treasure)\\
\textbf{Description}\\

Enlightened strategists of Hell's armies, the insectoid ice devils are among the most ingenious and cruel minds in hell. An ice devil hides in his chest a frozen heart stolen from a mortal, allowing him to make decisions free of emotion. Born in the frozen circle of Cocytus, the seventh circle of hell, most ice devils migrate to Caina, the eighth circle, where they plot to damn the world. Although they have the most alien and monstrous appearance of all devils, few others are accorded greater respect.

In combat he sends his subordinates forward, so that he can evaluate the tactics, strengths and weaknesses of the opponent in the rear, and provide them with support with his magical abilities, avoiding catching them in the area of ​​effect of his spells: attitude not due to a sense of camaraderie, but rather to the cold and logical truth that his allies can survive longer in a clash if they are not exposed to friendly fire.

Ice Devils stand 3.6 meters tall and weigh approximately 350 kg.


\medskip\index[Monstrorium]{Bone Devil}\textbf{Bone Devil}

\emph{Great fiend (devil), lawful evil}

\textbf{STRENGTH} +4

\textbf{DEXTERITY} +3

\textbf{CONSTITUTION} +4

\textbf{INTELLIGENCE} +1

\textbf{WISDOM} +2

\textbf{CHARISMA} +3

\textbf{Initiative} +3 -- \textbf{Defense} 24

\textbf{Hit Points} 142 (15d10 + 60)

\textbf{Movement} 12 m, flight 12 m

\textbf{Saving Throws} Fortitude +12, Reflexes +12, Will +7

\textbf{Skills} Deceive +7, Sense Emotions +6

\textbf{Damage Resistances} cold; from a non-magical or non-silver weapon

\textbf{Damage Immunity} Fire, poison

\textbf{Condition Immunity} poisoned

\textbf{Senses} darkvision 36 m

\textbf{Languages} Infernal, telepathy 36 m

\textbf{Challenge} 9 (5000 PX)

\emph{\textbf{Magic Resistance.}} The devil has +1d6 on saving throws against spells and other magical effects.

\emph{\textbf{Devil's Sight.}} The devil's darkvision is not limited by magical darkness.

\textbf{Shares}

\emph{\textbf{Multiattack.}} The devil makes three attacks: two with its claws and one with its stinger, or one with its barbed polearm and one with its stinger.

\emph{\textbf{Barbed Pole Weapon.} Melee weapon attack}: +12 to hit, reach 10 ft., one target.

\emph{Hits:} 17 (2d12 + 4) piercing damage. If the target is a creature of Huge size or smaller, it is grappled (DC 14 to escape). Until the grapple ends, the devil cannot use his mounted weapon on another target.

\emph{\textbf{Claw.} Melee weapon attack}: +12 to hit, reach 10 ft., one target.

\emph{Hits:} 8 (1d8 + 4) slashing damage, 1 bleed damage.

\emph{\textbf{Sting.} Melee weapon attack}: +12 to hit, reach 10 ft., one target.

\emph{Hits:} 13 (2d8 + 4) piercing damage plus 17 (5d6) poison damage, and the target must succeed on a DC 14 Fortitude save, or be poisoned, -1 Strength and Dexterity, for 1 minute . The target can repeat the saving throw at the end of each of its rounds, ending the effect on a success.

\emph{\textbf{Angry:}} the Bone Devil attacks all creatures around him with his holstered weapon. All creatures within 10 feet suffer a barbed weapon attack without being grabbed. Cost 2 Actions.

\textbf{Ecology}\\
Environment: Any (Hell)\\
Organization: Solo, squad (2-3), council (4-10), or contingent (1-3 ice devils, 2-6 horned devils, and 1-4 bone devils\\
\textbf{Treasure}: Standard (Frozen Spear+1, more treasure)\\
\textbf{Description}\\
Enlightened strategists of Hell's armies, the insectoid ice devils are among the most ingenious and cruel minds in Hell's legions. Known among devils as a gelugon, an ice devil hides within its chest a frozen heart stolen from a mortal, allowing it to make decisions free of emotion. Born in the frozen circle of Cocytus, the seventh circle of hell, most ice devils migrate to Caina, the eighth circle, where they plot to damn the world from courts of frozen steel. Although they have the most alien and monstrous appearance of all devils, few others are accorded greater respect.

In combat, a gelugon sends his subordinates forward, so that he can evaluate the tactics, strengths and weaknesses of the opponent in the rear, and provide them with support with his magical abilities, avoiding catching them in the area of ​​effect of his spells: attitude not due to a sense of camaraderie, but rather to the cold and logical truth that his allies can survive longer in a fight if they are not exposed to friendly fire. Gelugons stand 3.6 meters tall and weigh approximately 350 kg.


\medskip\index[Monstery]{Thorny Devil}\textbf{Thorny Devil}

\emph{Little fiend (devil), lawful evil}

\textbf{STRENGTH} +0

\textbf{DEXTERITY} +2

\textbf{CONSTITUTION} +1

\textbf{INTELLIGENCE} +0

\textbf{WISDOM} +2

\textbf{CHARISMA} -1

\textbf{Initiative} +2 -- \textbf{Defense} 14

\textbf{Hit Points} 22 (5d6 + 5)

\textbf{Movement} 6 m, flight 12 m

\textbf{Damage Resistances} cold; from a non-magical or non-silver weapon

\textbf{Damage Immunity} Fire, poison

\textbf{Condition Immunity} poisoned

\textbf{Senses} darkvision 36 m

\textbf{Languages} Infernal, telepathy 36 m

\textbf{Challenge} 2 (450 PX)

\emph{\textbf{Magic Resistance.}} The devil has +1d6 on saving throws against spells and other magical effects.

€16,991 € {€16,992 € {Fly.}} The devil does not provoke attacks of opportunity when he flies out of an enemy's reach.

\emph{\textbf{Limited Thorns.}} The devil has twelve caudal spines. Used thorns grow back at midnight.

\emph{\textbf{Devil's Sight.}} The devil's darkvision is not limited by magical darkness.

\textbf{Shares}

\emph{\textbf{Multiattack.}} The devil makes two attacks: one with its bite and one with its pitchfork or two with its tail spines.

\emph{\textbf{Pitchfork.} Melee weapon attack}: +2 to hit, reach 1 m, one target.

\emph{Hits:} 3 (1d6) piercing damage.

\emph{\textbf{Bite.} Melee weapon attack}: +2 to hit, reach 1 m, one target.

\emph{Hits:} 5 (2d4) slashing damage.

\emph{\textbf{Caul Spine.} Ranged Weapon Attack}: +4 to hit, range 6m, one target.

\emph{Hits:} 4 (1d4 + 2) piercing damage plus 3 (1d6) fire damage.

\textbf{Ecology}\\
Environment: Any (Hell)\\
Organization: Solo, couple, group (3-5) or platoon (6-11)\\
\textbf{Treasury}: Standard\\
\textbf{Description}\\
Sentinels of the vaults of Hell, jailers of the blackest souls and living weapons of the infernal forges, the hooked devils, known to diabolists as hamatula, impose their shackles on the damned and guard the nefarious work of the greater devils. A hamatula loves to feel warm blood on its spines and prefers to jump into battle when given the opportunity to fight.

Hamatula are collectors and organizers, and are the favored allies of eager summoners, as they often bring with them tempting treasures from the vaults of Hell or know the path to deadly riches. If left to act freely, in the hideouts of these devils they often display the pierced trophies of old victims, hanging like perverse collections of insects on bloody walls.

Most hook devils stand 7 feet and up and weigh 300 pounds, though their lean, muscular bodies appear larger due to the ever-growing, blade-sharp spikes that protrude from their bodies.

\medskip\index[Monstery]{Erinhes}\textbf{Erinhes}

\emph{Medium fiendish (devil), lawful evil}

\textbf{STRENGTH} +4

\textbf{DEXTERITY} +3

\textbf{CONSTITUTION} +4

\textbf{INTELLIGENCE} +2

\textbf{WISDOM} +2

\textbf{CHARISMA} +4

\textbf{Initiative} +3 -- \textbf{Defense} 24 (plate armour)

\textbf{Hit Points} 153 (18d8 + 72)

\textbf{Movement} 9 m, flight 18 m

\textbf{Saving Throws} Fortitude +16, Reflexes +15, Will +14

\textbf{Damage Resistances} cold; from a non-magical or non-silver weapon

\textbf{Damage Immunity} Fire, poison

\textbf{Condition Immunity} poisoned

\textbf{Senses} true vision 36 m

\textbf{Languages} Infernal, telepathy 36 m

\textbf{Challenge} 12 (8400 PX)

\emph{\textbf{Devil Weapons.}} The Erinyes' weapon attacks are magical and deal an additional 13 (3d8) poison damage when they hit (already included in attacks).

\emph{\textbf{Magic Resistance.}} The Erinyes has +1d6 on saving throws against spells and other magical effects.

\textbf{Shares}

\emph{\textbf{Multiattack.}} The Erinyes makes three attacks.

\emph{\textbf{Long Sword.} Melee weapon attack}: +17 to hit, reach 1 m, one target.

\emph{Hits:} 8 (1d8 + 4) slashing damage, or 9 (1d10 + 4) slashing damage if used with two hands, plus 13 (3d8) poison damage.

\emph{\textbf{Longbow.} Ranged weapon attack}: +17 to hit, range 45m, one target.

\emph{Hits:} 7 (1d8 + 4) piercing damage plus 13 (3d8) poison damage, and the target must succeed on a DC 14 Fortitude save or be poisoned, -1 Strength and Dexterity. The poison remains until removed by a spell \emph{lesser restoration} or similar.

\textbf{Reactions}

\emph{\textbf{Parry.}} The Erinyes adds 4 to its Defense against a melee attack that would hit it. To do so, the Erinyes must be able to see its attacker and wield a melee weapon.

\emph{\textbf{Angry:}} The Erinyes channels its magical energy into an attack. The target of the attack is hit by an infernal flame that causes 12d6 void damage. Saving Throw DC 18 Reflexes for half. It costs 2 Actions.

\textbf{Ecology}\\
Environment: Any (Hell)\\
Organization: Solo or trio\\
\textbf{Treasure}: Triple (Fiery Composite Longbow+1 [Strength +5], rope, Longsword+1)\\
\textbf{Description}\\
Known by many names, the Fallen, the Ashenwings, and the Furies, the devils known as Erinyes insult their angelic form with their lust for vengeance and bloody justice. Executioners, not judges, the Erinyes hover above the blade-sharp ledges of Dis, the second cosmopolitan circle of Hell, always careful to seize every opportunity for battle, be it in defense of Hell, at the whim of their diabolical lords or for the passionate call of capricious mortal summoners. All erinyes weave deadly living cords from their own hair, which they use in battle to entangle and lift their enemies into the air, taunting and condemning them for their transgressions before dropping them from great heights.

Erinyes are beautiful, dark angels who deliberately enhance their sensuality with scars and bruises. Yet, despite their beauty, Erinyes are not seductresses: they lack the subtlety and patience required for this refined emotional art, preferring to solve their problems with acts of swift and atrocious violence. Often an Erinyes will hold off on her killing blow while she attempts to kill an enemy, only to prolong their suffering. Death is generally the only way to escape the attentions of an Erinyes, and the more powerful ones are very skilled at keeping their enemies alive but helpless, thus prolonging their torment, even going so far as to keep them alive with magic. The most powerful Erinyes torturers are said to be gifted with abilities that allow the suffering they inflict to persist even after the subject's death. Most Erinyes stand just under 6 feet tall and weigh around 150 pounds, and their feathery black wings span more than 10 feet.


\medskip\index[Monstery]{Imp}\textbf{Imp}

\emph{Low fiend (devil, shapeshifter), lawful evil}

\textbf{STRENGTH} -2

\textbf{DEXTERITY} +3

\textbf{CONSTITUTION} +1

\textbf{INTELLIGENCE} +0

\textbf{WISDOM} +1

\textbf{CHARISMA} +2

\textbf{Initiative} +3 -- \textbf{Defense} 14

\textbf{Hit Points} 10 (3d4 + 3)

\textbf{Move} 6 m, fly 12 m (6 m in rat form; 6 m, fly 18 m in crow form; 6 m, climb 6 m in spider form)

\textbf{Saving Throws} Fortitude +1, Reflexes +6, Will +4

\textbf{Skills} Stealth +5, Deception +4, Sense Emotions +3

\textbf{Damage Resistances} cold; from a non-magical or non-silver weapon

\textbf{Damage Immunity} Fire, poison

\textbf{Condition Immunity} poisoned

\textbf{Senses} darkvision 36 m

\textbf{Languages} Infernal, Common

\textbf{Challenge} 1 (200 PX)

€17,077 € {€17,078 € {Shapeshifter.}} The devil can use his action to transform into a bestial form of a rat, crow, or spider, or to return to his true form. His stats are the same in all forms, although attacks may vary for some of them. Whatever equipment he is wearing or carrying is not transformed. Upon death he returns to his true form.

\emph{\textbf{Magic Resistance.}} The devil has +1d6 on saving throws against spells and other magical effects.

\emph{\textbf{Devil's Sight.}} The devil's darkvision is not limited by magical darkness.

\textbf{Shares}

\emph{\textbf{Sting (Bite in Beast Form).} Melee weapon attack}: +5 to hit, reach 3 ft., one creature.

\emph{Hit:} 5 (1d4 + 3) piercing damage, and the target must make a DC 11 Fortitude saving throw, taking 10 (3d6) poison damage on a failed save, or half as much damage on a successful one .

€17087 € {€17088 € {Invisibility.}} The devil remains invisible until he attacks or ends his concentration. Whatever the devil is carrying or wearing remains invisible as long as he remains in contact with the devil.

\textbf{Ecology}\\
Environment: Any (Hell)\\
Organization: Solitary, pair or flock (3-10)\\
\textbf{Treasury}: Standard\\
\textbf{Description}\\
Born straight from the pits of Hell, imps are the least powerful of the devils, though these cruel and invasive creatures play an important role in the corruption of mortal souls. Free from the hierarchies and duties of infernal armies, imps revel in every opportunity to travel to the Material Plane and subtly tempt mortals into ever more depraved acts.

Voluntarily serving spellcasters in the role of familiars, imps play the part of faithful servants, often offering their masters astute advice and infernal insights. In reality, imps work to send souls to Hell, ensuring that their master's soul, along with many others, is doomed to damnation after death.

Imps vary greatly in appearance, across a broad spectrum of bestial and grotesque features, though many take the form of a reddish-skinned, winged humanoid with bulbous features. The typical imp is only 60 centimeters tall, has a wingspan of 90 centimeters and weighs 5 kg.

One imp in a thousand has the ability to communicate telepathically with creatures within 50 feet and the power to change its form to that of a Small or Tiny animal, as if by a Feral Shape II spell. These consular imps are highly prized by powerful devils, who send them as servants to their favored followers or to corrupt mortal heroes. A consular imp can be summoned with the Improved Familiar feat, but only by a spellcaster of 8th level or higher. Diabolists tell of other races of imp with similarly specialized abilities, but if these creatures actually exist they are extremely rare.

Unlike other devils, imps often find themselves free and alone on the Material Plane, particularly after they have been summoned to serve as familiars and their masters have died (often, indirectly, due to the imp's own machinations). Without any means of returning home, these imps, free from any ties to arcane masters, can become dangerous nuisances or even lead small tribes of bloodthirsty humanoids, such as Goblins or Kobolds.


\medskip\index[Monster]{Lemur}\textbf{Lemur}

\emph{Medium fiendish (devil), lawful evil}

\textbf{STRENGTH} +0

\textbf{DEXTERITY} -3

\textbf{CONSTITUTION} +0

\textbf{INTELLIGENCE} -5

\textbf{WISDOM} +0

\textbf{CHARRISMA} -4

\textbf{Initiative} -3 -- \textbf{Defense} 8

\textbf{Hit Points} 13 (3d8)

\textbf{Movement} 5 metres

\textbf{Saving Throws} Fortitude +4, Reflexes +3, Will +0

\textbf{Damage Resistances} cold

\textbf{Damage Immunity} Fire, poison

\textbf{Condition Immunity} fascinated, poisoned, frightened

\textbf{Senses} darkvision 36 m

\textbf{Languages} understands the Infernal but cannot speak

\textbf{Challenge} 0 (10 PX)

\emph{\textbf{Diabolical Invigoration.}} A lemur that dies in the Nine Hells returns to life with all its hit points in 1d10 days unless it is killed by a creature with good traits on which the 'spell \emph{bless} or his remains be sprinkled with Holy Water.

\emph{\textbf{Devil's Sight.}} The devil's darkvision is not limited by magical darkness.

\textbf{Shares}

\emph{\textbf{Fist.} Melee weapon attack}: +3 to hit, reach 1 m, one target.

\emph{Hits:} 2 (1d4) bludgeoning damage.

\textbf{Ecology}\\
Environment: Any (Hell)\\
Organization: Solitary, pair, group (3-5), swarm (6-17) or host (10-40 or more)\\
\textbf{Treasure}: None\\
\textbf{Description}\\
The lowest of devils, lemures originate from the ranks of souls condemned to hell, shapeless masses of quivering flesh. The spark of instinct or memory that survives in their sleeping consciousness usually shapes their features, which mimic those of their torturers or the tortured souls around them. Grotesque and useless, a lemur's features reveal nothing of what it once was. Many sport various hideous faces or are nothing more than seething columns of cancerous flesh. Only their lumpy limbs, which they constantly flail about, appear to function properly, and they are used only to destroy any non-infernal life that gets too close.

Lemurs on the move consolidate into forms more than 4 feet tall and weighing more than 200 pounds, though these disgusting devils, when resting, often have the indistinct appearance of misshapen-featured masses of melted flesh.

Although they are among the most revolting creatures in existence, lemurs play a vital role in the twisted ecology of Hell. When, at the end of its mortal existence, a soul is damned, either because it worships diabolical forces or because it lacks faith in other divinities, it joins the masses of suffering souls that fill the plains of Avernus, the first circle of 'Hell. Here the torments begin, as lesser devils push them along with other spirits, preparing them for the arduous journey to one of the deepest circles of hell, usually one suitable for the appropriate punishment for the crimes committed by the soul, or simply towards dominion of a devil who needs new slaves. Once in the realm of their damnation, souls face countless centuries of torment at the hands of devils, other evil beings, and the deadly machinations of Hell itself. As the mortal essence slowly goes mad, these creatures forget their lives, becoming first savage and eventually little more than automatons driven by hatred and fear. After eons of this existence, the cruel process of Hell either totally destroys the soul or, in the case of the most profane spirits, rededicates these forgotten beings in the form of lemurs, the most basic life form of devils, senseless hordes of rotting flesh and diabolical. These loathsome beings gather in great masses, revolting waves of thousands upon thousands of these creatures.

The greater devils are able to recognize the most corrupt among them and, through mysterious tortures or thanks to the powers of Hell itself, reshape them into true devils, newly reborn and ready to obediently serve in the legions of the damned.


\subsection{Dinosaurs}

\medskip\index[Monstery]{Plesiosaur}\textbf{Plesiosaur}

\emph{Great beast, misaligned}

\textbf{STRENGTH} +4

\textbf{DEXTERITY} +2

\textbf{CONSTITUTION} +3

\textbf{INTELLIGENCE} -4

\textbf{WISDOM} +1

\textbf{CHARRISMA} -3

\textbf{Initiative} +2 -- \textbf{Defense} 14

\textbf{Hit Points} 68 (8d10 + 24)

\textbf{Movement} 6m, swim 12m

\textbf{Saving Throws} Fortitude +18, Reflexes +11, Will +9

\textbf{Skills} Stealth +4, Awareness +3

\textbf{Languages} -

\textbf{Challenge} 2 (450 PX)

\emph{\textbf{Hold Your Breath.}} The plesiosaur can hold its breath for 1 hour.

\textbf{Shares}

\emph{\textbf{Bite.} Melee weapon attack}: +6 to hit, reach 10 ft., one target.

\emph{Hits:} 14 (3d6 + 4) piercing damage.

\textbf{Ecology}\\
Environment: Warm Aquatic\\
Organization: Solitary, pair or pack (3-6)\\
\textbf{Treasure}: None\\
\textbf{Description}\\
The plesiosaur is a long-necked aquatic reptile. Although technically not a dinosaur, this creature and its peers are often found hunting in lakes and oceans where dinosaurs are likely to be found.


\medskip\index[Monster]{Tyrannosaurus}\textbf{Tyrannosaurus}

\emph{Huge beast, misaligned}

\textbf{STRENGTH} +7

\textbf{DEXTERITY} +0

\textbf{CONSTITUTION} +4

\textbf{INTELLIGENCE} -4

\textbf{WISDOM} +1

\textbf{CHARISMA} -1

\textbf{Initiative} +0 -- \textbf{Defense} 17

\textbf{Hit Points} 136 (13d12 + 52)

\textbf{Movement} 15 m

\textbf{Saving Throws} Fortitude +15, Reflex +12, Will +10

\textbf{Skills} Awareness +4

\textbf{Languages} -

\textbf{Challenge} 8 (3900 PX)

\textbf{Shares}

\emph{\textbf{Multiattack.}} The Tyrannosaurus makes two attacks: one with its bite and one with its tail. Cannot make both attacks against the same target.

\emph{\textbf{Tail.} Melee weapon attack}: +14 to hit, reach 10 ft., one target.

\emph{Hits:} 20 (3d8 + 7) bludgeoning damage.

\emph{\textbf{Bite.} Melee weapon attack}: +14 to hit, reach 10 ft., one target.

\emph{Hits:} 33 (4d12 + 7) piercing damage. If the target is a Medium or smaller creature, it is grappled (DC 17 to escape). Until the grapple ends, the target is entangled, and the tyrannosaur cannot use its bite against another target.

\emph{\textbf{Angry:}} the Tyrannosaurus is filled with murderous fury. He attacks any friendly or enemy creature. The attack roll gains +1d6 and the bite causes Bleeding 3/15.

\textbf{Ecology}\\
Environment: Forests and Warm Plains\\
Organization: Solitary, pair or pack (3-6)\\
\textbf{Treasure}: None\\
\textbf{Description}\\
The Tyrannosaurus is a primary predator that measures 12 meters in length and weighs 7000 kg.


\medskip\index[Monster]{Triceratops}\textbf{Triceratops}

\emph{Huge beast, misaligned}

\textbf{STRENGTH} +6

\textbf{DEXTERITY} -1

\textbf{CONSTITUTION} +3

\textbf{INTELLIGENCE} -4

\textbf{WISDOM} +0

\textbf{CHARISMA} -3

\textbf{Initiative} -1 -- \textbf{Defense} 16

\textbf{Hit Points} 95 (10d12 + 30)

\textbf{Movement} 15 m

\textbf{Saving Throws} Fortitude +15, Reflexes +8, Will +5

\textbf{Languages} -

\textbf{Challenge} 5 (1800 PX)

\emph{\textbf{Sweeping Charge.}} If the triceratops moves at least 20 feet towards a creature and hits it with a gore attack during the same round, the target must succeed on a DC 15 Fortitude saving throw or fall prone. If the target is prone, the triceratops can make a stomp attack against it as a bonus action.

\textbf{Shares}

\emph{\textbf{Gored.} Melee weapon attack}: +13 to hit, reach 1 m, one target.

\emph{Hits:} 24 (3d10 + 6) piercing damage.

\emph{\textbf{Stomp.} Melee weapon attack}: +13 to hit, reach 1 ft., one prone creature.

\emph{Hits:} 22 (3d10 + 6) bludgeoning damage.

\textbf{Ecology}\\
Environment: Warm Plains\\
Organization: Solitary, pair or pack (5-8)\\
\textbf{Treasure}: None\\
\textbf{Description}\\
Triceratops is an irascible and stubborn herbivore. A typical triceratops is 9 meters long and weighs 10,000 kg.

\medskip\index[Monster]{Devours Brains}\textbf{Devours Brains}

\emph{Small aberration, chaotic evil}

\textbf{STRENGTH} +1

\textbf{DEXTERITY} +6

\textbf{CONSTITUTION} +5

\textbf{INTELLIGENCE} +3

\textbf{WISDOM} +0

\textbf{CHARISMA} +3

\textbf{Initiative} +10 -- \textbf{Defense} 22

\textbf{Hit Points} 84 (8d8 + 48)

\textbf{Movement} 12 m

\textbf{Saving Throws} Fortitude +14, Reflexes +14, Will +9

\textbf{Damage Resistance} non-magical weapons, cold, electricity

\textbf{Damage Immunity} Fire

\textbf{Condition Immunity} spells from the Illusion and Charm magic lists

\textbf{Senses} Blind Sight 18 m

\textbf{Languages} telepathy 50 m

\textbf{Challenge} 9 (3900 PX)

\emph{\textbf{Eyes of Magic.}} The Brain Eater has Detect Magic always active.

\emph{\textbf{Innate Spells.}} The Brain Eater's spellcasting characteristic is Charisma. The Brain Eater can innately cast the following spells, without the need for material components:

At will: \emph{Confusion (single target), Inflict Serious Wounds, Invisibility}

3/day: \emph{Cure Moderate Wounds, Globe of Invulnerability}

\textbf{Shares}

\emph{\textbf{Multiattack.}} The Brain Devourer can make 4 attacks, one per claw

\emph{\textbf{Claw.} Melee weapon attack}: +9 to hit, reach 3 ft., one creature.

\emph{Hits:} 3 slashing damage (1d4+1), 1 bleed damage.

\textbf{Special abilities}

\emph{\textbf{Body theft}}

By spending 3 Actions a Brain Eater can become tiny and crawl into the mouth/nose/ears of a defenseless or dead creature and reach the brain to feed on it. This is an action that kills the creature. The Brain Eater takes control of the body and can use it at will, as if controlling the victim with a dominate monster spell. The brain eater has full access to all of the host's defensive and offensive abilities except for spell-like abilities and spells (although the brain eater can still use its spell-like abilities). A host body must not have been dead for more than 1 day for this ability to work, and even after being successfully occupied the bodies decay and become unusable in 7 days (unless this period is extended with the Inviolate Rest spell). As long as the Brain Eater occupies the body, he knows (and can speak) the languages ​​known by the victim and information about her identity and personality, but cannot possess her specific memories and knowledge. Damage dealt to the host body, which has double the original Hit Points, does not harm the Brain Eater and if the host body is destroyed the Brain Eater exits and is Stunned for 1 round.

\textbf{Ecology}\\
Environment: Any dungeon\\
Organization: Solitary, brood (2-6) or tribe (7-16)\\
\textbf{Treasure}: Double\\
\textbf{Description}\\
A Brain Eater is nothing more than a brain of about 50 cm equipped with 4 powerful clawed legs.

Believed by some to be invaders from another dimension or planet, the sinister Brain Eaters are certainly one of the cruelest races in the world. Unable to feel emotion or wallow in the sins of their own physical pleasure, brain eaters are forced to steal bodies to satisfy their gluttony, lust, and cruelty. There are stories that tell of entire underground cities of these creatures wearing bodies as if they were dressed to have frightening orgies and macabre feasts. Devour Brains Lonely often live in ruins or caves on the fringes of civilized regions so they can make periodic forays into cities to “acquire” a tempting new body.

Shayalia's garden is said to be full of Brain Eaters.

A Brain Eater is 90cm long and weighs around 30kg.

\medskip\textbf{Dobi}\\\index[Monstery]{Dobi}
\emph{Fairy lowercase}\\
\textbf{Strength}: -3\\
\textbf{Dexterity}: -1\\
\textbf{Constitution}: +2\\
\textbf{Intelligence}: -2\\
\textbf{Wisdom}: +1\\
\textbf{Charisma}: +3\\
\textbf{Defense}: 12 -- \textbf{Initiative}: +0\\
\textbf{Hit Points}: 6 (1d8 + 2)\\
\textbf{Move}: 3 m, Swim 9 m\\
\textbf{Saving Throws}: Fortitude +2, Reflexes +0, Will +1 \\
\textbf{Senses}: low-light vision 18 m \\
\textbf{Languages}: understands the Municipality, but does not speak it \\
\textbf{Challenge} 0 (10 PX)\\
\textbf{Immunity}: to damage from non-blunt non-magical weapons\\
\textbf{Resistance}: cutting damage, perforation\\
\emph{\textbf{Dobi}} The Dobi sticks, to move it you need to be kind and ask it.\\
\emph{\textbf{Dobi Dobi Dobi}} When the Dobi takes more than 3 hit points of damage from a non-blunt weapon it splits into two smaller Dobis each with the same amount of Hit Points remaining as the previous Dobi.\ \
\smallskip\textbf{Shares}\\
\emph{\textbf{Dobi Dobi}} the Dobi projects an aura of Calming Emotions like the spell of the same name but the saving throw is not allowed. The Dobi can affect only one creature at a time with its power.
\textbf{Ecology}\\
Environment: Swamps\\
Organization: group\\
\textbf{Honey}: Accidental\\
\textbf{Description}\\
{\small "...I moved the leaves of the marsh and saw a strange ball of fur on the ground, about ten centimeters in diameter, light in color. Intrigued, I picked it up, stroking its soft fur and examined it carefully. It appeared to have no limbs or signs of having a snout with eyes, ears, or mouth, but as soon as I petted it the ball vibrated, making a squeaking noise.

Finally I saw two lively little black eyes open in all that fur and then two little round ears emerge, then two short but robust legs, suitable for jumping, resting on the ground and two others, still short but with five toes each, at half height .

- Dobi! - replied the little animal, expressing a sort of joy and enthusiasm. - Dobi dobi! -.

- So cute! - I exclaimed, caressing him. It was the cutest little animal I had ever seen. "But now I'm putting you back down."

- Dobi - replied the furball.
I brought my hand to the ground, but the animal didn't move. I tried to remove it from my hand, but it remained stuck to the other. I grabbed it with two fingers, pulling hard and quickly placed it on the ground, but it immediately jumped on my foot and remained stuck there. I had to cross the marsh with the dobi attached to my foot, not counting the other four that I found clinging to the armor."}

From \emph{Journey to the first world, book by Federica Angeli}

\medskip\index[Monstery]{Doppelganger}\textbf{Doppelganger}

\emph{Medium monstrosity (shapeshifter), neutral}

\textbf{STRENGTH} +0

\textbf{DEXTERITY} +4

\textbf{CONSTITUTION} +2

\textbf{INTELLIGENCE} +0

\textbf{WISDOM} +1

\textbf{CHARISMA} +2

\textbf{Initiative} +4 -- \textbf{Defense} 16

\textbf{Hit Points} 52 (8d8 + 16)

\textbf{Movement} 9 m

\textbf{Saving Throws} Fortitude +4, Reflexes +5, Will +6

\textbf{Skills} Deceive +6, Sense Emotions +3

\textbf{Condition Immunity} fascinated

\textbf{Senses} Darkvision 18 m

\textbf{Languages} Municipality

\textbf{Challenge} 3 (700 PX)

€17306 € {€17307 € {Shapeshifter.}} The doppelganger can use his action to change his form to that of a Small or Medium humanoid he has seen, or to return to his true form. His stats, other than size, are the same in all forms. Whatever equipment he is wearing or carrying is not transformed. Upon death he returns to his true form.

\emph{\textbf{Lurking.}} In the first round of combat, the doppelganger has +1d6 to attack rolls against any creature it takes by surprise.

\emph{\textbf{Surprise Attack.}} If the doppelganger surprises a creature and hits it with an attack during the first round of combat, the target takes an additional 10 (3d6) damage from the attack.

\textbf{Shares}

\emph{\textbf{Multiattack.}} The doppelganger makes two melee attacks.

\emph{\textbf{Slam.} Melee weapon attack}: +6 to hit, reach 1 m, one target.

\emph{Hits:} 7 (1d6 + 4) bludgeoning damage.

\emph{\textbf{Read Thoughts.}} The doppelganger magically reads the surface thoughts of a creature within 60 feet of him. The effect can penetrate barriers, but 3 feet of wood or earth, 20 inches of stone, 2 inches of metal, or a thin sheet of lead blocks it. While the target is within range, the doppelganger can continue to read the target's thoughts, as long as the doppelganger's concentration is not broken (like the concentration of a spell). While reading a target's mind, the doppelganger has +1d6 on Wisdom and Charisma checks against the target.

\textbf{Ecology}\\
Environment: Any\\
Organization: Solo, couple or group (3-6)\\
\textbf{Treasure}: NPC Equipment\\
\textbf{Description}\\
Doppelgangers are strange beings who can take the form of those they encounter. In its natural form, the creature more or less resembles a humanoid, but slender and frail, with thin limbs and incompletely formed facial features. His complexion is pale, he is hairless and his eyes are white and vacant.

Doppelgangers prefer to infiltrate societies where they can amass wealth and power, and see little prospect of founding cities with their peers. Younger doppelgangers test their abilities on small tribes of orcs or goblins, then move on to more complex societies such as dwarven, elven, and human communities. Rather than become targets by occupying positions of leadership, they prefer to maintain power from behind the throne, or use multiple identities to manipulate influential citizens or entire guilds.

Doppelgangers make excellent use of their natural mimicry to set ambushes, bait traps, and infiltrate humanoid societies. While they are not usually evil, they are only interested in themselves and view everyone else as toys to be manipulated and deceived. They take great pleasure in invading human societies to satisfy their desires; some enjoy complex political games while others continually try to change their race, gender, and love partners. While not the norm, those doppelgangers who use their gifts for cruel and sadistic purposes are very notorious, and these shapeshifters are largely responsible for their race's sinister reputation. Certainly, a creature capable of shape-shifting has an advantage when trying to avoid being captured for its crimes, and some particularly malevolent doppelgangers enjoy ending romantic relationships by staging betrayals.

Insistent rumors speak of even more powerful doppelgangers capable not only of changing their appearance, but also of taking on extraordinary and supernatural abilities, memories and even abilities of the creatures they choose to impersonate.


\subsection{Dragons}

\begin{changemargin}{0.3cm}{0.3cm}\begin{enfasi}{
Oh damn may Lynx close all your portals\\
Oh murderers may Sumkjir exterminate you\\
O devastators, may Nedraf break your bones!\\
(expletives against Dragons)}\end{enfasi}\end{changemargin}\medskip

Dragons are fearsome, dangerous, ancient creatures; they represent power itself.

Each Dragon has full access to all spells from a specific magic list depending on their color.

This access is granted by Tàhil or Ljust depending on whether they are dragons loyal to one or the other.

And it is from this distinction that dragons are divided between Tàhil and Ljust Dragons. The former represent in various capacities and degrees chaos, destruction, violence and death, while the Dragons of Ljust are the emblem of the good, just, correct, protective. While the dragons of Tàhil are usually also defined as chromatic, those of Ljust are defined as metallic.

The Dragons of Ljust present on Yeru are transport errors, perhaps because the Tàhil portal opened while an evil dragon fought a good dragon.

\textbf{Failing the saving throw} against a dragon's breath on a critical count doubles the damage taken while succeeding on a critical save does not further halve the damage received.\index{Dragon's Breath}

\begin{itemize}
\item Each Dragon can cast spells up to a maximum level equal to a quarter of his Challenge Rank, with a minimum access to the first level.
\item Each Dragon has a number of Magic Points equal to 5 times its Challenge Rating
\item Each Dragon has a Magical Expertise score equal to half its Challenge Rating
\end{itemize}

\medskip

\textbf{Table: Access List of Magic for Dragons}\index[Tables]{Table access for List of Magic for Dragons}

\medskip

\begin{tabular}{ll}
\hline
\textbf{Dragon Color}& \textbf{Magic List Name} \\
White & Water\\
Blue&Air\\
Yellow & Fire, Evocation\\
Black&Water, Necromancy\\
Purple&Earth\\
Fire red\\
Green&Animals and Plants\\
Silver&Transmutation, Illusion\\
Bronze&Abjuration\\
Gold&Healing, Summon\\
Brass&Divination\\
Copper&Invocation\\
\end{tabular}

\medskip

All Dragons have access to the Universal magic list and favor certain spells that are noted in their description.

In the \textbf{Description of each Ancient Dragon} you will find a brief description of the type of dragon.

\emph{Note}: unfortunately it is not true that the dragons have only been here 300 years, in reality the last victory of the millennium has made us forget that these creatures have been on Yeru for many millennia and have been sowing destruction, chaos and death for just as long.


\subsubsection{Tàhil Dragons}

\medskip\index[Monstery]{Ancient White Dragon}\textbf{Ancient White Dragon}

\emph{Huge dragon, chaotic evil}

\textbf{STRENGTH} +8

\textbf{DEXTERITY} +0

\textbf{CONSTITUTION} +8

\textbf{INTELLIGENCE} +0

\textbf{WISDOM} +1

\textbf{CHARISMA} +2

\textbf{Initiative} +0 -- \textbf{Defense} 30

\textbf{Hit Points} 333 (18x3d6 + 144)

\textbf{Move} 12 m, swim 12 m, dig 12 m, fly 24 m

\textbf{Saving Throws} Fortitude +28, Reflexes +20, Will +21

\textbf{Skills} Stealth +6, Awareness +13

\textbf{Damage Immunity} cold, weapons +1

\textbf{Senses} darkvision 120 ft., blindsight 60 ft

\textbf{Languages} Common, Draconic

\textbf{Challenge} 20 (25000 PX)

\emph{\textbf{Ice Walking.}} The dragon can move and climb on icy surfaces without needing to make ability checks. Additionally, difficult terrain consisting of ice or snow costs him no additional movement.

€17363 €{€17364 €{Legendary Resistance (3 / Day).}} If the dragon fails a saving throw, it can choose to succeed instead.

\emph{\textbf{Magic Resistance:}} 3lv

\textbf{Shares}

\emph{\textbf{Multiattack.}} The dragon can use its Frightening Presence. Then make three attacks: one with the bite and two with the claws.

\emph{\textbf{Claw.} Melee weapon attack}: +30 to hit, reach 10 ft., one target.

\emph{Hits:} 15 (2d6 + 8) slashing damage, 3/20 bleed damage.

\emph{\textbf{Tail.} Melee weapon attack}: +30 to hit, reach 20 ft., one target.

\emph{Hits:} 17 (2d8 + 8) bludgeoning damage.

\emph{\textbf{Bite.} Melee weapon attack}: +30 to hit, reach 5 meters, one target.

\emph{Hits:} 19 (2d10 + 8) piercing damage plus 9 (2d8) cold damage.

\emph{\textbf{Frightening Presence.}} Each creature chosen by the dragon, within 120 feet of it and aware of its presence, must succeed on a DC 16 Will save or be frightened for 1 minute. A creature can repeat the saving throw at the end of each of its rounds, ending the effect on a success. If the creature's saving throw succeeds or the effect ends for it, the creature is immune to the dragon's Frightening Presence for the next 24 hours.

\emph{\textbf{Icy Breath (Recharge 5-6).}} The dragon exhales a blast of ice in a 27 meter cone. Each creature in that area must make a DC 22 Fortitude saving throw and take 72 (16d8) cold damage on a failed save, or half as much damage on a successful one.

\textbf{Additional Shares}

The dragon can perform 3 additional Actions, chosen from the following options. It can use only one Additional option at a time, and only at the end of another creature's round. The dragon recovers additional Actions spent at the start of its round.

\textbf{Wing Attack (Costs 2 Actions).} The dragon flaps its wings. Each creature within 5 feet of the dragon must succeed on a DC 22 Reflex save or take 15 (2d6 + 8) bludgeoning damage and be knocked prone. The dragon can then fly up to half its flight movement.

\textbf{Tail Attack.} The dragon makes a tail attack.

\textbf{Spot.} The dragon makes a Wisdom (Awareness) check.

\textbf{Ecology}\\
Environment: Cold Mountains\\
Organization: Solitaire\\
\textbf{Treasure}: Triple\\
\textbf{Description}\\
White Dragons are among the wildest and most animalistic of all dragons.
They love cold, icy places, finding refuge in the coldest valleys such as icy mountain peaks and frozen steppes.

White Dragons have a wild appearance and almost always show their teeth and claws are drawn to move nimbly on the frozen ground.
They have no movement penalties on these terrains.

They exploit their natural camouflage to attack and capture prey, they are excellent hunters, very intelligent in exploiting the environment.

Not very inclined towards magic, however, they know how to blow shards of ice much more frequently than other dragons. It is immune to cold and ice based attacks.

Their lairs are frozen caves in the mountains or carved into the most massive glaciers.

White Dragons have +1d6 on magic checks and can ignore a die rolled on the Water List check and is immune to cold.\\

\textbf{Spells}\index{White Dragon Spells}\\
This Dragon's favorite spells are:\\
- Fire Shield\\
- Ice storm\\
- Sleet Storm


\medskip\index[Monster]{Adult White Dragon}\textbf{Adult White Dragon}

\emph{Huge dragon, chaotic evil}

\textbf{STRENGTH} +6

\textbf{DEXTERITY} +0

\textbf{CONSTITUTION} +6

\textbf{INTELLIGENCE} -1

\textbf{WISDOM} +1

\textbf{CHARISMA} +1

\textbf{Initiative} +0 -- \textbf{Defense} 25

\textbf{Hit Points} 200 (16d12 + 96)

\textbf{Move} 12 m, swim 12 m, dig 9 m, fly 24 m

\textbf{Saving Throws} Fortitude +19, Reflexes +13, Will +14

\textbf{Skills} Stealth +5, Awareness +11

\textbf{Damage Immunity} cold

\textbf{Senses} darkvision 120 ft., blindsight 60 ft

\textbf{Languages} Common, Draconic

\textbf{Challenge} 13 (10000 PX)

\emph{\textbf{Ice Walking.}} The dragon can move and climb on icy surfaces without needing to make ability checks. Additionally, difficult terrain consisting of ice or snow costs him no additional movement.

€17414 €{€17415 €{Legendary Resistance (3 / Day).}} If the dragon fails a saving throw, it can choose to succeed instead.

\textbf{Shares}

\emph{\textbf{Multiattack.}} The dragon can use its Frightening Presence and then make three attacks: one with its bite and two with its claws.

\emph{\textbf{Claw.} Melee weapon attack}: +21 to hit, reach 1 m, one target, 1 bleed damage.

\emph{Hits:} 13 (2d6 + 6) slashing damage.

\emph{\textbf{Tail.} Melee weapon attack}: +21 to hit, reach 5 metres, one target.

\emph{Hits:} 15 (2d8 + 6) bludgeoning damage.

\emph{\textbf{Bite.} Melee weapon attack}: +21 to hit, reach 10 ft., one target.

\emph{Hits:} 17 (2d10 + 6) piercing damage plus 4 (1d8) cold damage.

\emph{\textbf{Frightening Presence.}} Each creature chosen by the dragon, within 120 feet of it and aware of its presence, must succeed on a DC 14 Will save or be frightened for 1 minute. A creature can repeat the saving throw at the end of each of its rounds, ending the effect on a success. If the creature's saving throw succeeds or the effect ends for it, the creature is immune to the dragon's Frightening Presence for the next 24 hours.

\emph{\textbf{Icy Breath (Recharge 5-6).}} The dragon exhales a blast of ice in a 60-foot cone. Each creature in that area must make a DC 19 Fortitude saving throw and take 54 (12d8) cold damage on a failed save, or half as much damage on a successful one.

\textbf{Additional Shares}

The dragon can perform 3 additional Actions, chosen from the following options. He can only use one Additional option at a time, and only at the end of another creature's round. The dragon recovers additional Actions spent at the start of its round.

\textbf{Wing Attack (Costs 2 Actions).} The dragon flaps its wings. Each creature within 10 feet of the dragon must succeed on a DC 19 Reflex save or take 13 (2d6 + 6) bludgeoning damage and be knocked prone. The dragon can then fly up to half its flight movement. \textbf{Tail Attack.} The dragon makes a tail attack
.
\textbf{Spot.} The dragon makes a Wisdom (Awareness) check.

\emph{\textbf{Angry:}} The Adult White Dragon can perform these actions for 2 Actions.

\emph{Focus}: The creature interrupts an ongoing mental effect on itself

\emph{Brutality}: the creature attacks with unprecedented ferocity. +1d6 to attack roll, an extra 6 is added to the critical count until the end of the fight.

\textbf{Ecology}\\
Environment: Cold Mountains\\
Organization: Solitaire\\
\textbf{Treasure}: Triple\\
\textbf{Description}\\
See Ancient White Dragon description.\\
\textbf{Spells}\index{White Dragon Spells}\\
This Dragon's favorite spells are:\\
- Fire Shield\\
- Ice storm\\
- Sleet Storm


\medskip\index[Monster]{Young White Dragon}\textbf{Young White Dragon}

\emph{Great dragon, chaotic evil}

\textbf{STRENGTH} +4

\textbf{DEXTERITY} +0

\textbf{CONSTITUTION} +4

\textbf{INTELLIGENCE} -2

\textbf{WISDOM} +0

\textbf{CHARRISMA} +1

\textbf{Initiative} +0 -- \textbf{Defense} 20

\textbf{Hit Points} 133 (14d10 + 56)

\textbf{Move} 12 m, swim 12 m, dig 6 m, fly 24 m

\textbf{Saving Throws} Fortitude +8, Reflexes +7, Will +5

\textbf{Skills} Stealth +3, Awareness +6

\textbf{Damage Immunity} cold

\textbf{Senses} darkvision 120 ft., blindsight 30 ft

\textbf{Languages} Common, Draconic

\textbf{Challenge} 6 (2300 PX)

\emph{\textbf{Ice Walking.}} The dragon can move and climb on icy surfaces without needing to make ability checks. Additionally, difficult terrain consisting of ice or snow costs him no additional movement.

\textbf{Shares}

\emph{\textbf{Multiattack.}} The dragon can use its Frightening Presence. Then make three attacks: one with the bite and two with the claws.

\emph{\textbf{Claw.} Melee weapon attack}: +6 to hit, reach 1 m, one target.

\emph{Hits:} 11 (2d6 + 4) slashing damage, 1 bleed damage.

\emph{\textbf{Bite.} Melee weapon attack}: +6 to hit, reach 10 ft., one target.

\emph{Hits:} 15 (2d10 + 4) piercing damage plus 4 (1d8) cold damage.

\emph{\textbf{Icy Breath (Recharge 5-6).}} The dragon exhales a blast of ice in a 30-foot cone. Each creature in that area must make a DC 15 Fortitude saving throw and take 45 (10d8) cold damage on a failed save, or half as much damage on a successful one.

\textbf{Ecology}\\
Environment: Cold Mountains\\
Organization: Solitaire\\
\textbf{Treasure}: Triple\\
\textbf{Description}\\
See Ancient White Dragon description.\\
\textbf{Spells}\index{White Dragon Spells}\\
This Dragon's favorite spells are:\\
- Fire Shield\\
- Ice storm\\
- Sleet Storm


\medskip\index[Monstery]{White Baby Dragon}\textbf{White Baby Dragon}

\emph{Medium dragon, chaotic evil}

\textbf{STRENGTH} +2

\textbf{DEXTERITY} +0

\textbf{CONSTITUTION} +2

\textbf{INTELLIGENCE} -3

\textbf{WISDOM} +0

\textbf{CHARISMA} +0

\textbf{Initiative} +0 -- \textbf{Defense} 17

\textbf{Hit Points} 32 (5d8 + 10)

\textbf{Move} 9m, swim 9m, dig 5m, fly 18m

\textbf{Saving Throws} Fortitude +2, Reflex +1, Will +1

\textbf{Skills} Stealth +2, Awareness +4

\textbf{Damage Immunity} cold

\textbf{Senses} Darkvision 60 ft., blindsight 10 ft

\textbf{Languages} Draconic

\textbf{Challenge} 2 (450 PX)

\textbf{Shares}

\emph{\textbf{Bite.} Melee weapon attack}: +5 to hit, reach 10 ft., one target.

\emph{Hits:} 15 (2d10 + 4) piercing damage plus 4 (1d8) cold damage.

\emph{\textbf{Icy Breath (Recharge 5-6).}} The dragon exhales a blast of ice in a 5 meter cone. Each creature in that area must make a DC 12 Fortitude saving throw and take 22 (5d8) cold damage on a failed save, or half as much damage on a successful one.

\textbf{Ecology}\\
Environment: Cold Mountains\\
Organization: Solitaire\\
\textbf{Treasure}: Triple\\
\textbf{Description}\\
See Ancient White Dragon description.\\


\medskip\index[Monstery]{Ancient Blue Dragon}\textbf{Ancient Blue Dragon}

\emph{Ghostly dragon, lawful evil}

\textbf{STRENGTH} +9

\textbf{DEXTERITY} +0

\textbf{CONSTITUTION} +8

\textbf{INTELLIGENCE} +4

\textbf{WISDOM} +3

\textbf{CHARISMA} +5

\textbf{Initiative} +4 -- \textbf{Defense} 34

\textbf{Hit Points} 481 (26x3d6 + 208)

\textbf{Movement} 12 m, dig 12 m, fly 24 m

\textbf{Saving Throws} Fortitude +31, Reflexes +23, Will +27

\textbf{Skills} Stealth +7, Awareness +17

\textbf{Damage Immunity} Electricity, weapons +1

\textbf{Senses} darkvision 120 ft., blindsight 60 ft

\textbf{Languages} Common, Draconic

\textbf{Challenge} 23 (50000 PX)

€17532 €{€17533 €{Legendary Resistance (3 / Day).}} If the dragon fails a saving throw, it can choose to succeed instead.

\emph{\textbf{Magic Resistance:}} 3lv

\textbf{Shares}

\emph{\textbf{Multiattack.}} The dragon can use its Frightening Presence. Then make three attacks: one with the bite and two with the claws.

\emph{\textbf{Claw.} Melee weapon attack}: +16 to hit,
range 3 m, one target.

\emph{Hits:} 16 (2d6 + 9) slashing damage, 3/20 bleed damage.

\emph{\textbf{Tail.} Melee weapon attack}: +30 to hit, reach 20 ft., one target.

\emph{Hits:} 18 (2d8 + 9) bludgeoning damage.

\emph{\textbf{Bite.} Melee weapon attack}: +30 to hit, reach 5 meters, one target.

\emph{Hits:} 20 (2d10 + 9) piercing damage plus 11 (2d10) lightning damage.

€17548 € {€17549 € {Frightening Presence.}} Each creature chosen by the dragon that is within 120 feet of it and aware of its presence must succeed on a DC 20 Will save or be frightened for 1 minute. A creature can repeat the saving throw at the end of each of its rounds, ending the effect on a success. If the creature's saving throw succeeds or the effect ends for it, the creature is immune to the dragon's Frightening Presence for the next 24 hours.

\emph{\textbf{Lightning Breath (Recharge 5-6).}} The dragon exhales lightning in a line 36 meters long and 3 meters wide. Each creature on that line must make a DC 23 Reflex saving throw and take 88 (16d10) lightning damage on a failed save, or half as much damage on a successful one.

\textbf{Additional Shares}

The dragon can perform 3 additional Actions, chosen from the following options. It can use only one Additional option at a time, and only at the end of another creature's round. The dragon recovers additional Actions spent at the start of its round.

\textbf{Wing Attack (Costs 2 Actions).} The dragon flaps its wings. Each creature within 5 feet of the dragon must succeed on a DC 24 Reflex save or take 16 (2d6 + 9) bludgeoning damage and be knocked prone. The dragon can then fly up to half its flight movement.

\textbf{Tail Attack.} The dragon makes a tail attack.

\textbf{Spot.} The dragon makes a Wisdom (Awareness) check.

\textbf{Spot.} The dragon makes a Wisdom (Awareness) check.\\
\textbf{Ecology}\\
Environment: Mountain peaks\\
Organization: Solitaire\\
\textbf{Treasure}: Triple\\
\textbf{Description}\\
Blue Dragons live in the clouds, flying (and levitating) through storms.

Blue Dragons have a serpentine, elongated, binding appearance, with long, swept-back horns.

A Blue Dragon's face is less wrinkled and remains smooth.
They are the only dragons to have no wings yet fly better than any other dragon.

Their magical but natural ability to fly combined with the fact that they feed on electricity makes them purely flying creatures that almost never come down to the ground (and never touch the ground considering it impure and dirty!), they prefer to remain in the clouds, especially among the darkest and full of energy to feed themselves

The Blue Dragon's lair is usually among the highest peaks of mountains possibly high enough to reach the clouds. This is never covered and often resembles gigantic nests.

Blue Dragons can assimilate meat but not vegetables, they do not derive nutrients from what they eat as they have a purely electrical metabolism.

They are social dragons, who love to be with their own kind and are very protective of their offspring.
Usually you never find a single nest, but entire plateaus dominated by dozens of dragons.

They don't get along with the purple dragons who they despise for having given up flight to live underground.

Blue Dragons have +1d6 on magic checks and can ignore a die rolled on the Air List check and is immune to electricity.\\
\textbf{Spells}\index{Blue Dragon Spells}\\
This Dragon's favorite spells are:\\
- Deadly Fog\\
- Invoke the Lightning\\
- Ice storm

\medskip\index[Monster]{Adult Blue Dragon}\textbf{Adult Blue Dragon}

\emph{Huge dragon, lawful evil}

\textbf{STRENGTH} +7

\textbf{DEXTERITY} +0

\textbf{CONSTITUTION} +6

\textbf{INTELLIGENCE} +3

\textbf{WISDOM} +2

\textbf{CHARISMA} +4

\textbf{Initiative} +3 -- \textbf{Defense} 27

\textbf{Hit Points} 225 (18d12 + 108)

\textbf{Movement} 12 m, dig 12 m, fly 24 m

\textbf{Saving Throws} Fortitude +21, Reflexes +16, Will +18

\textbf{Skills} Stealth +5, Awareness +12

\textbf{Damage Immunity} Electricity

\textbf{Senses} darkvision 120 ft., blindsight 60 ft

\textbf{Languages} Common, Draconic

\textbf{Challenge} 16 (15000 PX)

€17582 € {€17583 € {Legendary Resistance (3 / Day).}} If the dragon fails a saving throw, it can choose to succeed instead.

\textbf{Shares}

\emph{\textbf{Multiattack.}} The dragon can use its Frightening Presence. Then make three attacks: one with the bite and two with the claws.

\emph{\textbf{Claw.} Melee weapon attack}: +26 to hit, reach 1 m, one target.

\emph{Hits:} 14 (2d6 + 7) slashing damage, 1 bleed damage.

\emph{\textbf{Tail.} Melee weapon attack}: +26 to hit, reach 5 meters, one target.

\emph{Hits:} 16 (2d8 + 7) bludgeoning damage.

\emph{\textbf{Bite.} Melee weapon attack}: +26 to hit, reach 10 ft., one target.

\emph{Hits:} 18 (2d10 + 7) piercing damage plus 5 (1d10) lightning damage.

€17596 € {€17597 € {Frightening Presence.}} Each creature chosen by the dragon that is within 120 feet of it and aware of its presence must succeed on a DC 17 Will save or be frightened for 1 minute. A creature can repeat the saving throw at the end of each of its rounds, ending the effect on a success. If the creature's saving throw succeeds or the effect ends for it, the creature is immune to the dragon's Frightening Presence for the next 24 hours.

\emph{\textbf{Lightning Breath (Recharge 5-6).}} The dragon exhales lightning in a line 27 meters long and 1 meter wide. Each creature on that line must make a DC 19 Reflex saving throw and take 66 (12d10) lightning damage on a failed save, or half as much damage on a successful one.

\textbf{Additional Shares}

The dragon can perform 3 additional Actions, chosen from the following options. It can use only one Additional option at a time, and only at the end of another creature's round. The dragon recovers additional Actions spent at the start of its round.

\textbf{Wing Attack (Costs 2 Actions).} The dragon flaps its wings. Each creature within 10 feet of the dragon must succeed on a DC 20 Reflex save or take 14 (2d6 + 7) bludgeoning damage and be knocked prone. The dragon can then fly up to half of its flight movement.

\textbf{Tail Attack.} The dragon makes a tail attack.

\textbf{Spot.} The dragon makes a Wisdom (Awareness) check.

\emph{\textbf{Angry:}} The Adult Blue Dragon can perform these actions for 2 Actions.

\emph{Focus}: The creature interrupts an ongoing mental effect on itself

\emph{Brutality}: the creature attacks with unprecedented ferocity. +1d6 to attack roll, an extra 6 is added to the critical count until the end of the fight.

\textbf{Ecology}\\
Environment: Mountain peaks\\
Organization: Solitaire\\
\textbf{Treasure}: Triple\\
\textbf{Description}\\
See Ancient Blue Dragon Description.\\
\textbf{Spells}\index{Blue Dragon Spells}\\
This Dragon's favorite spells are:\\
- Deadly Fog\\
- Invoke the Lightning\\
- Ice storm



\medskip\index[Monster]{Young Blue Dragon}\textbf{Young Blue Dragon}

\emph{Huge dragon, lawful evil}

\textbf{STRENGTH} +5

\textbf{DEXTERITY} +0

\textbf{CONSTITUTION} +4

\textbf{INTELLIGENCE} +2

\textbf{WISDOM} +1

\textbf{CHARISMA} +3

\textbf{Initiative} +2 -- \textbf{Defense} 23

\textbf{Hit Points} 152 (16d10 + 64)

\textbf{Movement} 12 m, dig 12 m, fly 24 m

\textbf{Saving Throws} Fortitude +10, Reflexes +8, Will +8

\textbf{Skills} Stealth +4, Awareness +9

\textbf{Damage Immunity} Electricity

\textbf{Senses} darkvision 120 ft., blindsight 30 ft

\textbf{Languages} Common, Draconic

\textbf{Challenge} 9 (5000 PX)

\textbf{Shares}

\emph{\textbf{Multiattack.}} The dragon can make three attacks: one with its bite and two with its claws.

\emph{\textbf{Claw.} Melee weapon attack}: +13 to hit, reach 1 m, one target.

\emph{Hits:} 12 (2d6 + 5) slashing damage, 1 bleed damage.

\emph{\textbf{Bite.} Melee weapon attack}: +13 to hit, reach 10 ft., one target.

\emph{Hits:} 16 (2d10 + 5) piercing damage plus 5 (1d10) lightning damage.

\emph{\textbf{Lightning Breath (Recharge 5-6).}} The dragon exhales lightning in a line 18 meters long and 1 meter wide. Each creature on that line must make a DC 16 Reflex saving throw and take 55 (10d10) lightning damage on a failed save, or half as much damage on a successful one.

\emph{\textbf{Angry:}} the Young Blue Dragon recharges his lightning breath.

\textbf{Ecology}\\
Environment: Mountain peaks\\
Organization: Solitaire\\
\textbf{Treasure}: Triple\\
\textbf{Description}\\
See Ancient Blue Dragon Description.\\
\textbf{Spells}\index{Blue Dragon Spells}\\
This Dragon's favorite spells are:\\
- Deadly Fog\\
- Invoke the Lightning\\
- Ice storm



\medskip\index[Monster]{Blue Baby Dragon}\textbf{Blue Baby Dragon}

\emph{Huge dragon, lawful evil}

\textbf{STRENGTH} +3

\textbf{DEXTERITY} +0

\textbf{CONSTITUTION} +2

\textbf{INTELLIGENCE} +1

\textbf{WISDOM} +0

\textbf{CHARISMA} +2

\textbf{Initiative} +1 -- \textbf{Defense} 19

\textbf{Hit Points} 52 (8d8 + 16)

\textbf{Movement} 9 m, dig 5 meters, fly 18 m

\textbf{Saving Throws} Fortitude +4, Reflexes +1, Will +1

\textbf{Skills} Stealth +2, Awareness +4

\textbf{Damage Immunity} Electricity

\textbf{Senses} Darkvision 60 ft., blindsight 10 ft

\textbf{Languages} Draconic

\textbf{Challenge} 3 (700 PX)

\textbf{Shares}

\emph{\textbf{Bite.} Melee weapon attack}: +5 to hit, reach 1 m, one target.

\emph{Hits:} 8 (1d10 + 3) piercing damage plus 3 (1d6) lightning damage.

\emph{\textbf{Lightning Breath (Recharge 5-6).}} The dragon exhales lightning in a line 30 feet long and 3 feet wide. Each creature on that line must make a DC 12 Reflex saving throw and take 22 (4d10) lightning damage on a failed save, or half as much damage on a successful one.

\textbf{Ecology}\\
Environment: Mountain peaks\\
Organization: Solitaire\\
\textbf{Treasure}: Triple\\
\textbf{Description}\\
See Ancient Blue Dragon Description.

\medskip\textbf{Ancient Yellow Dragon}\index[Monstery]{Ancient Yellow Dragon}\\
\emph{Ghostly dragon, neutral evil}\\
\textbf{Strength}: +10\\
\textbf{Dexterity}: +1\\
\textbf{Constitution}: +8\\
\textbf{Intelligence}: +3\\
\textbf{Wisdom}: +2\\
\textbf{Charisma}: +4\\
\textbf{Defense}: 27 (natural armour) - \textbf{Initiative}: +4\\
\textbf{Hit Points}: 481 (26x3d6 + 208)\\
\textbf{Move}: 12 m, dig 24 m, climb 24, fly 12 m\\
\textbf{Saving Throws}: Fortitude +21, Reflexes +13, Will +19\\
\textbf{Skills}: Crime +7, Awareness +17\\
\textbf{Damage Immunity}: lightning\\
\textbf{Senses}: Darkvision 120 ft., blindsight 60 ft.\\
\textbf{Languages} Common, Draconic\\
\textbf{Challenge}: 23 (50000 PX)\smallskip\\
€17701 €{€17702 €{Legendary Resistance (3 / Day).}} If the dragon fails a saving throw, it can choose to succeed instead. \\
\emph{\textbf{Magic Resistance:}} 3lv\\
\smallskip\textbf{Shares}\\
\emph{\textbf{Multiattack.}} The dragon can use its Frightening Presence. Then make three attacks: one with your bite and two with your claws.\\
\emph{\textbf{Claw.} Melee weapon attack}: +30 to hit, reach 10 ft., one target.\\
\emph{Hits:} 16 (2d6 + 9) slashing damage, 3/20 bleed damage.\\
\emph{\textbf{Tail.} Melee weapon attack}: +30 to hit, reach 20 ft., one target.\\
\emph{Hits:} 18 (2d8 + 9) bludgeoning damage.\\
\emph{\textbf{Bite.} Melee weapon attack}: +30 to hit, reach 5 yards, one target.\\
\emph{Hits:} 20 (2d10 + 9) piercing damage plus 11 (2d10) lightning damage.\\
\emph{\textbf{Frightening Presence.}} Each creature chosen by the dragon that is within 120 feet of it and aware of its presence must succeed on a DC 25 Will save or be frightened for 1 minute. A creature can repeat the saving throw at the end of each of its rounds, ending the effect on a success. If the creature's saving throw succeeds or the effect ends for it, the creature is immune to the dragon's Frightful Presence for the next 24 hours.\\
\emph{\textbf{Incendiary Breath (Recharge 5-6).}} The dragon exhales scorching air in a line 36 meters long and 3 meters wide. Each creature on that line must make a DC 30 Reflex saving throw and take 88 (16d10) fire damage on a failed save, or half as much damage on a successful one.\\
\textbf{Additional Shares}\\
The dragon can take 3 additional actions, chosen from the options below. It can only use one Additional Action at a time, and only at the end of another creature's round. The dragon recovers the Additional Actions spent at the start of its round.\\
\textbf{Wing Attack (Costs 2 Actions).} The dragon flaps its wings. Each creature within 15 feet of the dragon must succeed on a DC 31 Reflex save or take 16 (2d6 + 9) bludgeoning damage and be knocked prone. The dragon can then fly up to half its flight speed.\\
\textbf{Tail Attack.} The dragon makes a tail attack.\\
\textbf{Spot.} The dragon makes a Wisdom (Awareness) check.\\
\textbf{Ecology}\\
Environment: Hot Deserts\\
Organization: Solitaire\\
\textbf{Treasure}: Triple\\
\textbf{Description}\\
Yellow Dragons have scales of various shades of yellow which, as they grow, begin to resemble more and more the color of the sands where they live, from light yellow to brick ocher.

They are very intelligent but being solitary by nature they have no interest in communicating with other breeds.

They live in deserts where they often ambush their prey by hiding at the bottom of large holes dug in the sand.
As soon as they sense movement above them they come out and devour any creature.
They have a passion for dwarven meat which they find tasty even if dry.

The Yellow Dragon, although intelligent, is a killing machine and is unlikely to come to terms unless it finds itself in serious danger.

A Yellow Dragon has +1d6 on magic checks and can ignore a die rolled on the Fire List check and is immune to fire.
\\
\textbf{Spells}\index{Yellow Dragon Spells}\\
This Dragon's favorite spells are:\\
- Create food and water\\
- Wall of Fire\\
- Fire shield

\medskip\index[Monster]{Ancient Black Dragon}\textbf{Ancient Black Dragon}

\emph{Huge dragon, chaotic evil}

\textbf{STRENGTH} +8

\textbf{DEXTERITY} +2

\textbf{CONSTITUTION} +7

\textbf{INTELLIGENCE} +3

\textbf{WISDOM} +2

\textbf{CHARRISMA} +4

\textbf{Initiative} +3 -- \textbf{Defense} 33

\textbf{Hit Points} 367 (21x3d6 + 147)

\textbf{Movement} 12 m, climb 12 m, fly 24 m

\textbf{Saving Throws} Fortitude +28, Reflexes +23, Will +23

\textbf{Skills} Stealth +9, Awareness +16

\textbf{Damage Immunity} acid, weapons +1

\textbf{Senses} darkvision 120 ft., blindsight 60 ft

\textbf{Languages} Common, Draconic

\textbf{Challenge} 21 (33000 XP)

\emph{\textbf{Amphibian.}} The dragon can breathe air and water.

€17753 €{€17754 €{Legendary Resistance (3 / Day).}} If the dragon fails a saving throw, it can choose to succeed instead.

\emph{\textbf{Magic Resistance:}} 3lv

\textbf{Shares}

\emph{\textbf{Multiattack.}} The dragon can use its Frightening Presence. Then make three attacks: one with the bite and two with the claws.

\emph{\textbf{Claw.} Melee weapon attack}: +30 to hit, reach 10 ft., one target.

\emph{Hits:} 15 (2d6 + 8) slashing damage, 3/20 bleed damage.

\emph{\textbf{Tail.} Melee weapon attack}: +30 to hit, reach 20 ft., one target.

\emph{Hits:} 17 (2d8 + 8) bludgeoning damage.

\emph{\textbf{Bite.} Melee weapon attack} : +30 to hit, reach 5 meters, one target.

\emph{Hits:} 19 (2d10 + 8) piercing damage plus 9 (4d6) acid damage.

\emph{\textbf{Frightening Presence.}} Each creature chosen by the dragon, within 120 feet of it and aware of its presence, must succeed on a DC 19 Will save or be frightened for 1 minute. A creature can repeat the saving throw at the end of each of its rounds, ending the effect on a success. If the creature's saving throw succeeds or the effect ends for it, the creature is immune to the dragon's Frightening Presence for the next 24 hours.

\emph{\textbf{Acid Breath (Recharge 5-6).}} The dragon exhales acid in a 27 meter line 3 meters wide. Each creature in that area must make a DC 22 Reflex saving throw and take 67 (15d8) acid damage on a failed save, or half as much damage on a successful one.

\textbf{Additional Shares}

The dragon can perform 3 additional Actions, chosen from the following options. It can use only one Additional option at a time, and only at the end of another creature's round. The dragon recovers additional Actions spent at the start of its round.

\textbf{Wing Attack (Costs 2 Actions).} The dragon flaps its wings. Each creature within 5 feet of the dragon must succeed on a DC 23 Reflex save or take 15 (2d6 + 8) bludgeoning damage and be knocked prone. The dragon can then fly up to half its flight movement.

\textbf{Tail Attack.} The dragon makes a tail attack.

\textbf{Spot.} The dragon makes a Wisdom (Awareness) check.

\textbf{Ecology}\\
Environment: Warm Swamps\\
Organization: Solitaire\\
\textbf{Treasure}: Triple\\
\textbf{Description}\\
Black Dragons are violent and aggressive, living in swamps and swamps and generally ruling as undisputed masters.

Black Dragons are menacing creatures that have large, forward-curving horns.
The head connects to a relatively short neck and a large, muscular lizard-like body.

They have very small wings that are located on the sides, but they can still fly thanks to magic.
They have webbed feet to allow them to swim more easily in the marshy areas where they live.

Black Dragons tend to make their lairs in the center of the swamp or swamp.
They consider that territory theirs and no one can get wet without suffering their wrath.

A black dragon's lair can be a gigantic pile of logs but also an underground cave submerged in water, if not the bottom of a lake.
Being able to breathe underwater they do not worry about where to build their home.

Their home is always protected by traps and their evil followers who bring them food, possibly alive.

The environment where a black dragon lives suffers its effects, acid vapors, destruction, corruption are immediately perceptible.

The Black Dragon represents the Traits of selfishness and violence by hating all life, including black dragons themselves.

Black Dragons have +1d6 on magic checks and can ignore a die rolled on the Necromancy List check and are immune to acid.\\
\textbf{Spells}\index{Black Dragon Spells}\\
This Dragon's favorite spells are:\\
- Reanimate Dead\\
- Create Undead\\
- Cast Curse

Well yes, the Black Dragon is the only creature on Yeru that can bring a dead person to life despite all the constraints imposed by the Patrons.


\medskip\index[Monster]{Adult Black Dragon}\textbf{Adult Black Dragon}

\emph{Huge dragon, chaotic evil}

\textbf{STRENGTH} +6

\textbf{DEXTERITY} +2

\textbf{CONSTITUTION} +5

\textbf{INTELLIGENCE} +2

\textbf{WISDOM} +1

\textbf{CHARISMA} +3

\textbf{Initiative} +2 -- \textbf{Defense} 28

\textbf{Hit Points} 195 (17d12 + 85)

\textbf{Movement} 12 m, climb 12 m, fly 24 m

\textbf{Saving Throws} Fortitude +22, Reflexes +19, Will +18

\textbf{Skills} Stealth +7, Awareness +11

\textbf{Damage Immunity} acid

\textbf{Senses} darkvision 120 ft., blindsight 60 ft

\textbf{Languages} Common, Draconic

\textbf{Challenge} 17 (18000 PX)

\emph{\textbf{Amphibian.}} The dragon can breathe air and water.

€17804 €{€17805 €{Legendary Resistance (3 / Day).}} If the dragon fails a saving throw, it can choose to succeed instead.

\textbf{Shares}

\emph{\textbf{Multiattack.}} The dragon can use its Frightening Presence. Then make three attacks: one with the bite and two with the claws.

\emph{\textbf{Claw.} Melee weapon attack}: +26 to hit, reach 1 m, one target.

\emph{Hits:} 13 (2d6 + 6) slashing damage, 1 bleed damage.

\emph{\textbf{Tail.} Melee weapon attack}: +26 to hit, reach 5 meters, one target.

\emph{Hits:} 15 (2d8 + 6) bludgeoning damage.

\emph{\textbf{Bite.} Melee weapon attack}: +26 to hit, reach 10 ft., one target.

\emph{Hits:} 17 (2d10 + 6) piercing damage plus 4 (1d8) acid damage.

\emph{\textbf{Frightening Presence.}} Each creature chosen by the dragon, within 120 feet of it and aware of its presence, must succeed on a DC 16 Will save or be frightened for 1 minute. A creature can repeat the saving throw at the end of each of its rounds, ending the effect on a success. If the creature's saving throw succeeds or the effect ends for it, the creature is immune to the dragon's Frightening Presence for the next 24 hours.

\emph{\textbf{Acid Breath (Recharge 5-6).}} The dragon exhales acid in a 18 meter line 1 meter wide. Each creature in that area must make a DC 18 Reflex saving throw and take 54 (12d8) acid damage on a failed save, or half as much damage on a failed save.
succeeds.

\textbf{Additional Shares}

The dragon can perform 3 additional Actions, chosen from the following options. He can only use one Additional option at a time, and only at the end of another creature's round. The dragon recovers additional Actions spent at the start of its round.

\textbf{Wing Attack (Costs 2 Actions).} The dragon flaps its wings. Each creature within 10 feet of the dragon must succeed on a DC 19 Reflex save or take 13 (2d6 + 6) bludgeoning damage and be knocked prone. The dragon can then fly up to half of its flight movement.

\textbf{Tail Attack.} The dragon makes a tail attack.

\textbf{Spot.} The dragon makes a Wisdom (Awareness) check.

\emph{\textbf{Angry:}} The Adult Black Dragon can perform these actions for 2 Actions.

\emph{Focus}: The creature interrupts an ongoing mental effect on itself

\emph{Brutality}: the creature attacks with unprecedented ferocity. +1d6 to attack roll, an extra 6 is added to the critical count until the end of the fight.


\textbf{Ecology}\\
Environment: Warm Swamps\\
Organization: Solitaire\\
\textbf{Treasure}: Triple\\
\textbf{Description}\\
See Ancient Black Dragon Description.\\
\textbf{Spells}\index{Black Dragon Spells}\\
This Dragon's favorite spells are:\\
- Reanimate Dead\\
- Create Undead\\
- Cast Curse


\medskip\index[Monster]{Young Black Dragon}\textbf{Young Black Dragon}

\emph{Great dragon, chaotic evil}

\textbf{STRENGTH} +4

\textbf{DEXTERITY} +2

\textbf{CONSTITUTION} +3

\textbf{INTELLIGENCE} +1

\textbf{WISDOM} +0

\textbf{CHARISMA} +2

\textbf{Initiative} +2 -- \textbf{Defense} 22

\textbf{Hit Points} 127 (15d10 + 45)

\textbf{Movement} 12 m, climb 12 m, fly 24 m

\textbf{Saving Throws} Fortitude +9, Reflexes +8, Will +7

\textbf{Skills} Stealth +5, Awareness +6

\textbf{Damage Immunity} acid

\textbf{Senses} darkvision 120 ft., blindsight 30 ft

\textbf{Languages} Common, Draconic

\textbf{Challenge} 7 (2900 PX)

\emph{\textbf{Amphibian.}} The dragon can breathe air and water.

\textbf{Shares}

\emph{\textbf{Multiattack.}} The dragon can make three attacks: one with its bite and two with its claws.

\emph{\textbf{Claw.} Melee weapon attack}: +9 to hit, reach 1 m, one target.

\emph{Hits:} 11 (2d6 + 4) slashing damage, 1 bleed damage.

\emph{\textbf{Bite.} Melee weapon attack}: +9 to hit, reach 10 ft., one target.

\emph{Hits:} 11 (2d10 + 4) piercing damage plus 4 (1d8) acid damage.

\emph{\textbf{Acid Breath (Recharge 5-6).}} The dragon exhales acid in a 30-foot line 3 feet wide. Each creature in that area must make a DC 14 Reflex saving throw and take 49 (11d8) acid damage on a failed save, or half as much damage on a successful one.

\emph{\textbf{Angry:}} The Young Black Dragon recharges his acid breath. Costs 1 Action.

\textbf{Ecology}\\
Environment: Warm Swamps\\
Organization: Solitaire\\
\textbf{Treasure}: Triple\\
\textbf{Description}\\
See Ancient Black Dragon Description.\\
\textbf{Spells}\index{Black Dragon Spells}\\
This Dragon's favorite spells are:\\
- Reanimate Dead\\
- Create Undead\\
- Cast Curse

\medskip\index[Monstery]{Black Baby Dragon}\textbf{Black Baby Dragon}

\emph{Medium dragon, chaotic evil}

\textbf{STRENGTH} +2

\textbf{DEXTERITY} +2

\textbf{CONSTITUTION} +1

\textbf{INTELLIGENCE} +0

\textbf{WISDOM} +0

\textbf{CHARISMA} +1

\textbf{Initiative} +2 -- \textbf{Defense} 18

\textbf{Hit Points} 33 (6d8 + 6)

\textbf{Movement} 9 m, climb 9 m, fly 18 m

\textbf{Saving Throws} Fortitude +2, Reflexes +2, Will +0

\textbf{Skills} Stealth +4, Awareness +4

\textbf{Damage Immunity} acid

\textbf{Senses} Darkvision 60 ft., blindsight 10 ft

\textbf{Languages} Draconic

\textbf{Challenge} 2 (450 PX)

\emph{\textbf{Amphibian.}} The dragon can breathe air and water.

\textbf{Shares}

\emph{\textbf{Bite.} Melee weapon attack}: +4 to hit, reach 1 m, one target.

\emph{Hits:} 7 (1d10 + 2) piercing damage plus 2 (1d4) acid damage.

\emph{\textbf{Acid Breath (Recharge 5-6).}} The dragon exhales acid in a 5 meter line 1 meter wide. Each creature in that area must make a DC 11 Reflex saving throw and take 22 (5d8) acid damage on a failed save, or half as much damage on a successful one.

\textbf{Ecology}\\
Environment: Warm Swamps\\
Organization: Solitaire\\
\textbf{Treasure}: Triple\\
\textbf{Description}\\
See Ancient Black Dragon Description.

\medskip\index[Monstery]{Ancient Purple Dragon}\textbf{Ancient Purple Dragon}

\emph{Ghostly dragon, lawful evil}

\textbf{STRENGTH} +8

\textbf{DEXTERITY} +3

\textbf{CONSTITUTION} +4

\textbf{INTELLIGENCE} +4

\textbf{WISDOM} +4

\textbf{CHARISMA} +4

\textbf{Initiative} +5 -- \textbf{Defense} 32

\textbf{Hit Points} 385 (22x3d6 + 154)

\textbf{Move} 12 m, dig 24 m

\textbf{Saving Throws} Fortitude +26, Reflexes +25, Will +26

\textbf{Skills} Dungeon Knowledge +8, Intimidation +11, Sense Emotions +10, Awareness +15

\textbf{Damage Immunity} sound, weapons +1

\textbf{Senses} darkvision 36 m, Telluric Sense 72 m

\textbf{Languages} Common, Draconic

\textbf{Challenge} 22 (41000 PX)

\emph{\textbf{Terrestrial.}} The dragon cannot breathe or eat as long as it is underground.

\emph{\textbf{Legendary Resistance (3/Day).}} If the dragon fails a saving throw, it can choose to succeed instead.

\emph{\textbf{Magic Resistance:}} 3lv

\textbf{Shares}

\emph{\textbf{Multiattack.}} The dragon can use its Frightening Presence. Then make three attacks: one with the bite and two with the claws.

\emph{\textbf{Claw.} Melee weapon attack}: +30 to hit, reach 10 ft., one target.

\emph{Hits:} 15 (2d6 + 8) slashing damage, 3/20 bleed damage.

\emph{\textbf{Tail.} Melee weapon attack}: +30 to hit, reach 20 ft., one target.

\emph{Hits:} 17 (2d8 + 8) bludgeoning damage.

\emph{\textbf{Bite.} Melee weapon attack}: +30 to hit, reach 5 metres, one target.

\emph{Hits:} 19 (2d10 + 8) piercing damage plus 10 (3d6) poison damage.

\emph{\textbf{Frightening Presence.}} Each creature chosen by the dragon, within 120 feet of it and aware of its presence, must succeed on a DC 19 Will save or be frightened for 1 minute. A creature can repeat the saving throw at the end of each of its rounds, ending the effect on a success. If the creature's saving throw succeeds or the effect ends for it, the creature is immune to the dragon's Frightening Presence for the next 24 hours.

\emph{\textbf{Sonic Breath (Recharge 5-6).}} The dragon emits a 27 meter cone. Each creature in that area must make a DC 22 Fortitude saving throw and take 77 (22d6) sonic damage on a failed save, or half as much damage on a successful one.

\textbf{Additional Shares}

The dragon can perform 3 additional Actions, chosen from the following options. He can only use one Additional option at a time, and only at the end of another creature's round. The dragon recovers additional Actions spent at the start of its round.

\textbf{Slam (Costs 2 Actions).} The dragon jumps on the spot. Each creature within 5 feet of the dragon must succeed on a DC 23 Reflex save or take 15 (2d6 + 8) bludgeoning damage and be knocked prone. The dragon can then move up to half its move.

\textbf{Tail Attack.} The dragon makes a tail attack.

\textbf{Spot.} The dragon makes a Wisdom (Awareness) check.

\textbf{Ecology}\\
Environment: Caverns\\
Organization: Solitaire, underground creatures\\
\textbf{Treasure}: Triple\\
\textbf{Description}\\
Purple Dragons live underground and have perfectly adapted to underground life.
Capable of seeing in the dark as if it were broad daylight, equipped with Telluric Sense, they have lost the ability to fly but acquired the ability to dig with the same speed as if they were running.

A Purple Dragon is very territorial and will establish a perimeter (about 5 km in radius) where it creates, if not already present, an intricate series of tunnels and caves for its minions.

A Purple Dragon is very protective of his creatures, of those who bring him food and offer him treasures.

They have a stocky appearance and have long, fine teeth and enormous claws that continually grow. They have a very powerful sonic attack that often creates collapses in caves, collapses that are completely indifferent to him.

He is strong and courageous, arrogant but not brazen. He is not afraid to fight if he thinks he will win. He always takes the battle underground where he can create pits to make enemies fall or escape if necessary.

A Purple Dragon has +1d6 on magic checks and can ignore a die roll on the Earth List check and is immune to damage and sonic effects.\\
\textbf{Spells}\index{Purple Dragon Spells}\\
This Dragon's favorite spells are:\\
- Acid Arrow\\
- Pass Without Traces\\
- Sculpting Stone


\medskip\index[Monstery]{Ancient Red Dragon}\textbf{Ancient Red Dragon}

\emph{Huge dragon, chaotic evil}

\textbf{STRENGTH} +10

\textbf{DEXTERITY} +0

\textbf{CONSTITUTION} +9

\textbf{INTELLIGENCE} +4

\textbf{WISDOM} +2

\textbf{CHARISMA} +6

\textbf{Initiative} +4 -- \textbf{Defense} 34

\textbf{Hit Points} 546 (28x3d6 + 252)

\textbf{Movement} 12 m, climb 12 m, fly 24 m

\textbf{Saving Throws} Fortitude +33, Reflexes +24, Will +26

\textbf{Skills} Stealth +7, Awareness +16

\textbf{Damage Immunity} Fire, weapons +1

\textbf{Senses} darkvision 120 ft., blindsight 60 ft

\textbf{Languages} Common, Draconic

\textbf{Challenge} 24 (62000 PX)

€17,977 € {€17,978 € {Legendary Resistance (3 / Day).}} If the dragon fails a saving throw, it can choose to succeed instead.

\emph{\textbf{Magic Resistance:}} 3lv

\textbf{Shares}

\emph{\textbf{Multiattack.}} The dragon can use its Frightening Presence and then make three attacks: one with its bite and two with its claws.

\emph{\textbf{Claw.} Melee weapon attack}: +30 to hit, reach 10 ft., one target.

\emph{Hits:} 17 (2d6 + 10) slashing damage, 3/20 bleed damage.

\emph{\textbf{Tail.} Melee weapon attack}: +30 to hit, reach 20 ft., one target.

\emph{Hits:} 19 (2d8 + 10) bludgeoning damage.

\emph{\textbf{Bite.} Melee weapon attack}: +30 to hit, reach 5 metres, one target.

\emph{Hits:} 21 (2d10 + 10) piercing damage plus 14 (4d6) fire damage.

\emph{\textbf{Frightening Presence.}} Each creature chosen by the dragon, within 120 feet of it and aware of its presence, must succeed on a DC 21 Will save or be frightened for 1 minute. A creature can repeat the saving throw at the end of each of its rounds, ending the effect on a success. If the creature's saving throw succeeds or the effect ends for it, the creature is immune to the dragon's Frightening Presence for the next 24 hours.

\emph{\textbf{Fiery Breath (Recharge 5-6).}} The dragon exhales fire in a 27 meter cone. Each creature in that area must make a DC 24 Reflex saving throw and take 91 (26d6) fire damage on a failed save, or half as much damage on a successful one.

\textbf{Additional Shares}

The dragon can perform 3 additional Actions, chosen from the following options. He can only use one Additional option at a time, and only at the end of another creature's round. The dragon recovers additional Actions spent at the start of its round.

\textbf{Wing Attack (Costs 2 Actions).} The dragon flaps its wings. Each creature within 5 feet of the dragon must succeed on a DC 25 Reflex save or take 17 (2d6 + 10) bludgeoning damage and be knocked prone. The dragon can then fly up to half its flight movement.

\textbf{Tail Attack.} The dragon performs a tail attack.

\textbf{Spot.} The dragon makes a Wisdom (Awareness) check.

\emph{\textbf{Angry}}: The red dragon shakes and roars. Once per day, the first time he is Angry, he ends all negative conditions on himself and all abilities recharge. The breath weapon recharges on 3-6.

\textbf{Ecology}\\
Environment: Warm mountains\\
Organization: Solitaire\\
\textbf{Treasure}: Triple\\
\textbf{Description}\\
The Red Dragon believes himself to be the King of Dragons due to his physical power and his breath capable of melting stone.

Red Dragons are the largest dragons in both body size and wingspan.
Often the scales, of a dark red almost like blood, have sharp and elongated edges.

Red Dragons prefer warm mountains and if possible directly inside a volcano.

They fight using their size, wings, bite, claws... in short, everything they are and have at their disposal. A Red Dragon always fights to the death, it does not retreat or run away or give up a challenge, the pride with which they are proud does not allow them to appear weak.

A Red Dragon has +1d6 on magic checks and can ignore a die rolled on the Fire List check and is immune to fire.\\
\textbf{Spells}\index{Red Dragon Spells}\\
This Dragon's favorite spells are:\\
- Fire ball\\
- Incendiary Cloud\\
- Wall of fire


\medskip\index[Monstery]{Adult Red Dragon}\textbf{Adult Red Dragon}

\emph{Huge dragon, chaotic evil}

\textbf{STRENGTH} +8

\textbf{DEXTERITY} +0

\textbf{CONSTITUTION} +7

\textbf{INTELLIGENCE} +3

\textbf{WISDOM} +1

\textbf{CHARISMA} +5

\textbf{Initiative} +3 -- \textbf{Defense} 28

\textbf{Hit Points} 256 (19d12 + 133)

\textbf{Movement} 12 m, climb 12 m, fly 24 m

\textbf{Saving Throws} Fortitude +23, Reflexes +17, Will +18

\textbf{Skills} Stealth +6, Awareness +13

\textbf{Damage Immunity} Fire

\textbf{Senses} darkvision 120 ft., blindsight 60 ft

\textbf{Languages} Common, Draconic

\textbf{Challenge} 17 (18000 PX)

€18028 € {€18029 € {Legendary Resistance (3 / Day).}} If the dragon fails a saving throw, it can choose to succeed instead.

\textbf{Shares}

\emph{\textbf{Multiattack.}} The dragon can use its Frightening Presence and then make three attacks: one with its bite and two with its claws.

\emph{\textbf{Claw.} Melee weapon attack}: +28 to hit, reach 1 m, one target.

\emph{Hits:} 15 (2d6 + 8) slashing damage, 1 bleed damage.

\emph{\textbf{Tail.} Melee weapon attack}: +28 to hit, reach 5 meters, one target.

\emph{Hits:} 17 (2d8 + 8) bludgeoning damage.

\emph{\textbf{Bite.} Melee weapon attack}: +28 to hit, reach 10 ft., one target.

\emph{Hits:} 19 (2d10 + 8) piercing damage plus 7 (2d6) piercing damage
fire.

\emph{\textbf{Frightening Presence.}} Each creature chosen by the dragon, within 120 feet of it and aware of its presence, must succeed on a DC 19 Will save or be frightened for 1 minute. A creature can repeat the saving throw at the end of each of its rounds, ending the effect on a success. If the creature's saving throw succeeds or the effect ends for it, the creature is immune to the dragon's Frightening Presence for the next 24 hours.

\emph{\textbf{Fiery Breath (Recharge 5-6).}} The dragon exhales fire in a 60-foot cone. Each creature in that area must make a DC 21 Reflex saving throw and take 63 (18d6) fire damage on a failed save, or half as much damage on a successful one.

\textbf{Additional Shares}

The dragon can perform 3 additional Actions, chosen from the following options. He can only use one Additional option at a time, and only at the end of another creature's round. The dragon recovers additional Actions spent at the start of its round.

\textbf{Wing Attack (Costs 2 Actions).} The dragon flaps its wings. Each creature within 10 feet of the dragon must succeed on a DC 22 Reflex save or take 15 (2d6 + 8) bludgeoning damage and be knocked prone. The dragon can then fly up to half its flight movement.

\textbf{Tail Attack.} The dragon makes a tail attack.

\textbf{Spot.} The dragon makes a Wisdom (Awareness) check.

\emph{\textbf{Angry:}} The Adult Red Dragon can perform these actions for 2 Actions.

\emph{Focus}: The creature interrupts an ongoing mental effect on itself

\emph{Brutality}: the creature attacks with unprecedented ferocity. +1d6 to attack roll, an extra 6 is added to the critical count until the end of the fight.


\textbf{Ecology}\\
Environment: Warm mountains\\
Organization: Solitaire\\
\textbf{Treasure}: Triple\\
\textbf{Description}\\
See Ancient Red Dragon Description.\\
\textbf{Spells}\index{Red Dragon Spells}\\
This Dragon's favorite spells are:\\
- Fire ball\\
- Incendiary Cloud\\
- Wall of fire


\medskip\index[Monstery]{Young Red Dragon}\textbf{Young Red Dragon}

\emph{Great dragon, chaotic evil}

\textbf{STRENGTH} +6

\textbf{DEXTERITY} +0

\textbf{CONSTITUTION} +5

\textbf{INTELLIGENCE} +2

\textbf{WISDOM} +0

\textbf{CHARISMA} +4

\textbf{Initiative} +2 -- \textbf{Defense} 23

\textbf{Hit Points} 178 (17d10 + 85)

\textbf{Movement} 12 m, climb 12 m, fly 24 m

\textbf{Saving Throws} Fortitude +11, Reflexes +8, Will +10

\textbf{Skills} Stealth +4, Awareness +8

\textbf{Damage Immunity} Fire

\textbf{Senses} darkvision 120 ft., blindsight 30 ft

\textbf{Languages} Common, Draconic

\textbf{Challenge} 10 (5900 PX)

\textbf{Shares}

\emph{\textbf{Multiattack.}} The dragon can make three attacks: one with its bite and two with its claws.

\emph{\textbf{Claw.} Melee weapon attack}: +16 to hit, reach 1 m, one target.

\emph{Hits:} 13 (2d6 + 6) slashing damage, 1 bleed damage.

\emph{\textbf{Bite.} Melee weapon attack}: +16 to hit, reach 10 ft., one target.

\emph{Hits:} 17 (2d10 + 6) piercing damage plus 3 (1d6) fire damage.

\emph{\textbf{Fiery Breath (Recharge 5-6).}} The dragon exhales fire in a 30-foot cone. Each creature in that area must make a DC 17 Reflex saving throw and take 56 (16d6) fire damage on a failed save, or half as much damage on a successful one.

\emph{\textbf{Angry:}} the young red dragon recharges his fiery breath. Costs 1 Action.

\textbf{Ecology}\\
Environment: Warm mountains\\
Organization: Solitaire\\
\textbf{Treasure}: Triple\\
See Ancient Red Dragon Description.\\
\textbf{Spells}\index{Red Dragon Spells}\\
This Dragon's favorite spells are:\\
- Fire ball\\
- Incendiary Cloud\\
- Wall of fire


\medskip\index[Monstery]{Red Baby Dragon}\textbf{Red Baby Dragon}

\emph{Medium dragon, chaotic evil}

\textbf{STRENGTH} +4

\textbf{DEXTERITY} +0

\textbf{CONSTITUTION} +3

\textbf{INTELLIGENCE} +1

\textbf{WISDOM} +0

\textbf{CHARISMA} +2

\textbf{Initiative} +1 -- \textbf{Defense} 19

\textbf{Hit Points} 75 (10d8 + 30)

\textbf{Movement} 9 m, climb 9 m, fly 18 m

\textbf{Saving Throws} Fortitude +4, Reflexes +3, Will +1

\textbf{Skills} Stealth +2, Awareness +4

\textbf{Damage Immunity} Fire

\textbf{Senses} Darkvision 60 ft., blindsight 10 ft

\textbf{Languages} Draconic

\textbf{Challenge} 4 (1100 PX)

\textbf{Shares}

\emph{\textbf{Bite.} Melee weapon attack}: +8 to hit, reach 1 m, one target.

\emph{Hits:} 9 (1d10 + 4) piercing damage plus 3 (1d6) fire damage.

\emph{\textbf{Fiery Breath (Recharge 5-6).}} The dragon exhales fire in a 5 meter cone. Each creature in that area must make a DC 13 Reflex saving throw and take 24 (7d6) fire damage on a failed save, or half as much damage on a successful one.

\textbf{Ecology}\\
Environment: Warm mountains\\
Organization: Solitaire\\
\textbf{Treasure}: Triple\\
See Ancient Red Dragon Description.


\medskip\index[Monstery]{Ancient Green Dragon}\textbf{Ancient Green Dragon}

\emph{Ghostly dragon, lawful evil}

\textbf{STRENGTH} +8

\textbf{DEXTERITY} +1

\textbf{CONSTITUTION} +7

\textbf{INTELLIGENCE} +5

\textbf{WISDOM} +3

\textbf{CHARISMA} +4

\textbf{Initiative} +5 -- \textbf{Defense} 32

\textbf{Hit Points} 385 (22x3d6 + 154)

\textbf{Move} 12 m, swim 12 m, fly 24 m

\textbf{Saving Throws} Fortitude +29, Reflexes +23, Will +25

\textbf{Skills} Stealth +8, Deception +11, Sense Emotions +10, Awareness + 15

\textbf{Damage Immunity} Poison, weapons +1

\textbf{Condition Immunity} poisoned

\textbf{Senses} darkvision 120 ft., blindsight 60 ft

\textbf{Languages} Common, Draconic

\textbf{Challenge} 22 (41000 PX)

\emph{\textbf{Amphibian.}} The dragon can breathe air and water.

€18,147 € {€18,148 € {Legendary Resistance (3 / Day).}} If the dragon fails a saving throw, it can choose to succeed instead.

\emph{\textbf{Magic Resistance:}} 3lv

\textbf{Shares}

\emph{\textbf{Multiattack.}} The dragon can use its Frightening Presence. Then make three attacks: one with the bite and two with the claws.

\emph{\textbf{Claw.} Melee weapon attack}: +30 to hit, reach 10 ft., one target.

\emph{Hits:} 15 (2d6 + 8) slashing damage, 3/20 bleed damage.

\emph{\textbf{Tail.} Melee weapon attack}: +30 to hit, reach 20 ft., one target.

\emph{Hits:} 17 (2d8 + 8) bludgeoning damage.

\emph{\textbf{Bite.} Melee weapon attack}: +30 to hit, reach 5 metres, one target.

\emph{Hits:} 19 (2d10 + 8) piercing damage plus 10 (3d6) poison damage.

€18,163 € {€18,164 € {Frightening Presence.}} Each creature chosen by the dragon that is within 120 feet of it and aware of its presence must succeed on a DC 19 Will save or be frightened for 1 minute. A creature can repeat the saving throw at the end of each of its rounds, ending the effect on a success. If the creature's saving throw succeeds or the effect ends for it, the creature is immune to the dragon's Frightening Presence for the next 24 hours.

\emph{\textbf{Poisonous Breath (Recharge 5-6).}} The dragon exhales poisonous gases in a 27 meter cone. Each creature in that area must make a DC 22 Fortitude saving throw and take 77 (22d6) poison damage on a failed save, or half as much damage on a successful one.

\textbf{Additional Shares}

The dragon can perform 3 additional Actions, chosen from the following options. It can use only one Additional option at a time, and only at the end of another creature's round. The dragon recovers additional Actions spent at the start of its round.

\textbf{Wing Attack (Costs 2 Actions).} The dragon flaps its wings. Each creature within 5 feet of the dragon must succeed on a DC 23 Reflex save or take 15 (2d6 + 8) bludgeoning damage and be knocked prone. The dragon can then fly up to half its flight movement.

\textbf{Tail Attack.} The dragon makes a tail attack.

\textbf{Spot.} The dragon makes a Wisdom (Awareness) check.

\textbf{Ecology}\\
Environment: Temperate Forests\\
Organization: Solitaire\\
\textbf{Treasure}: Triple\\
\textbf{Description}\\
Green Dragons love forests and uncontaminated nature where they consider themselves the undisputed masters and kings.

The powerful green dragons have rounded heads and pronounced ears set back, the horns are short and not pointed.
The claws and jaws are devastating, powerful and capable of cutting through anything.
The nose is wide and the nostrils open as if it were going to blow at any moment.

Green dragon breath is poison, so it can kill living creatures but not plants.

A green dragon's lair is always near a water source, possibly in the most lush and uncontaminated part of the forest.

A Green Dragon does not like to fly and prefers to jump, crushing with its weight and tearing with its claws.

Among the many dragons, the green one is perhaps the one that will make the adventurers talk if they show respect and fear of his royalty.

Green Dragons have +1d6 on magic checks and can ignore a die rolled on the List of Animals and Plants check and are immune to both magical and natural poisons.\\
\textbf{Spells}\index{Green Dragon Spells}\\
This Dragon's favorite spells are:\\
- Anti-life shell\\
- Locate animals and plants\\
- Remove poison


\medskip\index[Monster]{Adult Green Dragon}\textbf{Adult Green Dragon}

\emph{Huge dragon, lawful evil}

\textbf{STRENGTH} +6

\textbf{DEXTERITY} +1

\textbf{CONSTITUTION} +5

\textbf{INTELLIGENCE} +4

\textbf{WISDOM} +2

\textbf{CHARISMA} +3

\textbf{Initiative} +4 -- \textbf{Defense} 27

\textbf{Hit Points} 207 (18d12 + 90)

\textbf{Move} 12 m, swim 12 m, fly 24 m

\textbf{Saving Throws} Fortitude +20, Reflexes +16, Will +17

\textbf{Skills} Stealth +6, Deception +8, Sense Emotions +7, Awareness +12

\textbf{Damage Immunity} Poison

\textbf{Condition Immunity} poisoned

\textbf{Senses} darkvision 120 ft., blindsight 60 ft

\textbf{Languages} Common, Draconic

\textbf{Challenge} 15 (13000 PX)

\emph{\textbf{Amphibian.}} The dragon can breathe air and water.

€18,199 € {€18,200 € {Legendary Resistance (3 / Day).}} If the dragon fails a saving throw, it can choose to succeed instead.

\textbf{Shares}

\emph{\textbf{Multiattack.}} The dragon can use its Frightening Presence. Then make three attacks: one with the bite and two with the claws.

\emph{\textbf{Claw.} Melee weapon attack}: +23 to hit, reach 1 m, one target.

\emph{Hits:} 13 (2d6 + 6) slashing damage, 1 bleed damage.

\emph{\textbf{Tail.} Melee weapon attack}: +23 to hit, reach 5 meters, one target.

\emph{Hits:} 15 (2d8 + 6) bludgeoning damage.

\emph{\textbf{Bite.} Melee weapon attack}: +23 to hit, reach 10 ft., one target.

\emph{Hits:} 17 (2d10 + 6) piercing damage plus 7 (2d6) poison damage.

€18213 € {€18214 € {Frightening Presence.}} Each creature chosen by the dragon that is within 120 feet of it and aware of its presence must succeed on a DC 16 Will save or be frightened for 1 minute. A creature can repeat the saving throw at the end of each of its rounds, ending the effect on a success. If the creature's saving throw succeeds or the effect ends for it, the creature is immune to the dragon's Frightening Presence for the next 24 hours.

\emph{\textbf{Poisonous Breath (Recharge 5-6).}} The dragon exhales poisonous gases in a 60-foot cone. Each creature in that area must make a DC 18 Fortitude saving throw and take 56 (16d6) poison damage on a failed save, or half as much damage on a successful one.

\textbf{Additional Shares}

The dragon can perform 3 additional Actions, chosen from the following options. It can use only one Additional option at a time, and only at the end of another creature's round. The dragon recovers additional Actions spent at the start of its round.

\textbf{Wing Attack (Costs 2 Actions).} The dragon flaps its wings. Each creature within 10 feet of the dragon must succeed on a DC 19 Reflex save or take 13 (2d6 + 6) bludgeoning damage and be knocked prone. The dragon can then fly up to half its flight movement.

\textbf{Tail Attack.} The dragon makes a tail attack.

\textbf{Spot.} The dragon makes a Wisdom (Awareness) check.

\emph{\textbf{Angry:}} The Adult Green Dragon can perform these actions for 2 Actions.

\emph{Focus}: The creature interrupts an ongoing mental effect on itself

\emph{Brutality}: the creature attacks with unprecedented ferocity. +1d6 to attack roll, an extra 6 is added to the critical count until the end of the fight.


\textbf{Ecology}\\
Environment: Temperate Forests\\
Organization: Solitaire\\
\textbf{Treasure}: Triple\\
\textbf{Description}\\
See Ancient Green Dragon Description.\\
\textbf{Spells}\index{Green Dragon Spells}\\
This Dragon's favorite spells are:\\
- Anti-life shell\\
- Locate animals and plants\\
- Remove poison



\medskip\index[Monstery]{Young Green Dragon}\textbf{Young Green Dragon}

\emph{Great Dragon, Lawful Evil}

\textbf{STRENGTH} +4

\textbf{DEXTERITY} +1

\textbf{CONSTITUTION} +3

\textbf{INTELLIGENCE} +3

\textbf{WISDOM} +1

\textbf{CHARISMA} +2

\textbf{Initiative} +3 -- \textbf{Defense} 22

\textbf{Hit Points} 136 (16d10 + 48)

\textbf{Move} 12 m, swim 12 m, fly 24 m

\textbf{Saving Throws} Fortitude +9, Reflexes +7, Will +9

\textbf{Skills} Stealth +4, Deception +5, Awareness +7

\textbf{Damage Immunity} Poison

\textbf{Condition Immunity} poisoned

\textbf{Senses} darkvision 120 ft., blindsight 30 ft
\textbf{Languages} Common, Draconic

\textbf{Challenge} 8 (3900 PX)

\emph{\textbf{Amphibian.}} The dragon can breathe air and water.

\textbf{Shares}

\emph{\textbf{Multiattack.}} The dragon can make three attacks: one with its bite and two with its claws.

\emph{\textbf{Claw.} Melee weapon attack}: +11 to hit, reach 1 m, one target.

\emph{Hits:} 11 (2d6 + 4) slashing damage, 1 bleed damage.

\emph{\textbf{Bite.} Melee weapon attack}: +11 to hit, reach 10 ft., one target.

\emph{Hits:} 15 (2d10 + 4) piercing damage plus 7 (2d6) poison damage.

\emph{\textbf{Poisonous Breath (Recharge 5-6).}} The dragon exhales poisonous gases in a 30-foot cone. Each creature in that area must make a DC 14 Fortitude saving throw and take 42 (12d6) poison damage on a failed save, or half as much damage on a successful one.

\emph{\textbf{Angry:}} the Young Green Dragon recharges his Poison Breath.

\textbf{Ecology}\\
Environment: Temperate Forests\\
Organization: Solitaire\\
\textbf{Treasure}: Triple\\
\textbf{Description}\\
See Ancient Green Dragon Description.\\
\textbf{Spells}\index{Green Dragon Spells}\\
This Dragon's favorite spells are:\\
- Anti-life shell\\
- Locate animals and plants\\
- Remove poison


\medskip\index[Monster]{Green Baby Dragon}\textbf{Green Baby Dragon}

\emph{Medium dragon, lawful evil}

\textbf{STRENGTH} +2

\textbf{DEXTERITY} +1

\textbf{CONSTITUTION} +1

\textbf{INTELLIGENCE} +2

\textbf{WISDOM} +0

\textbf{CHARRISMA} +1

\textbf{Initiative} +2 -- \textbf{Defense} 18

\textbf{Hit Points} 38 (7d8 + 7)

\textbf{Move} 9 m, swim 9 m, fly 18 m

\textbf{Saving Throws} Fortitude +3, Reflexes +1, Will +0

\textbf{Skills} Stealth +3, Awareness +4

\textbf{Damage Immunity} Poison

\textbf{Condition Immunity} poisoned

\textbf{Senses} Darkvision 60 ft., blindsight 10 ft

\textbf{Languages} Draconic

\textbf{Challenge} 2 (450 PX)

\emph{\textbf{Amphibian.}} The dragon can breathe air and water.

\textbf{Shares}

\emph{\textbf{Bite.} Melee weapon attack}: +4 to hit, reach 1 m, one target.

\emph{Hits:} 7 (1d10 + 2) piercing damage plus 3 (1d6) poison damage.

\emph{\textbf{Poisonous Breath (Recharge 5-6).}} The dragon exhales poisonous gases in a 5 meter cone. Each creature in that area must make a DC 11 Fortitude saving throw and take 21 (6d6) poison damage on a failed save, or half as much damage on a successful one.

\textbf{Ecology}\\
Environment: Temperate Forests\\
Organization: Solitaire\\
\textbf{Treasure}: Triple\\
\textbf{Description}\\
See Ancient Green Dragon Description.


\subsubsection{Dragons of Ljust}

Very few good dragons, or Ljust as they are called, are present in Yeru.
Elysan is probably the best known and most powerful, an ancient silver dragon.


\medskip\index[Monstery]{Ancient Silver Dragon}\textbf{Ancient Silver Dragon}

\emph{Mammoth dragon, legal good}

\textbf{STRENGTH} +10

\textbf{DEXTERITY} +0

\textbf{CONSTITUTION} +9

\textbf{INTELLIGENCE} +4

\textbf{WISDOM} +2

\textbf{CHARISMA} +6

\textbf{Initiative} +4 -- \textbf{Defense} 34

\textbf{Hit Points} 487 (25x3d6 + 225)

\textbf{Movement} 12 m, flight 24 m

\textbf{Saving Throws} Fortitude +32, Reflexes +23, Will +25

\textbf{Skills} Arcana +11, Stealth +7, Awareness +16, History +11

\textbf{Damage Immunity} cold, weapons +1

\textbf{Senses} darkvision 120 ft., blindsight 60 ft

\textbf{Languages} Common, Draconic

\textbf{Challenge} 23 (50000 PX)

€18324 €{€18325 €{Legendary Resistance (3 / Day).}} If the dragon fails a saving throw, it can choose to succeed instead.

\emph{\textbf{Magic Resistance:}} 3lv

\textbf{Shares}

\emph{\textbf{Multiattack.}} The dragon can use its Frightening Presence. Then make three attacks: one with the bite and two with the wings
claws.

\emph{\textbf{Claw.} Melee weapon attack}: +30 to hit, reach 10 ft., one target.

\emph{Hits:} 17 (2d6 + 10) slashing damage, 3/20 bleed damage.

\emph{\textbf{Tail.} Melee weapon attack}: +30 to hit, reach 20 ft., one target.

\emph{Hits:} 19 (2d8 + 10) bludgeoning damage.

\emph{\textbf{Bite.} Melee weapon attack}: +30 to hit, reach 5 meters, one target.

\emph{Hits:} 21 (2d10 + 10) piercing damage.

\emph{\textbf{Frightening Presence.}} Each creature chosen by the dragon, within 120 feet of it and aware of its presence, must succeed on a DC 21 Will save or be frightened for 1 minute. A creature can repeat the saving throw at the end of each of its rounds, ending the effect on a success. If the creature's saving throw succeeds or the effect ends for it, the creature is immune to the dragon's Frightening Presence for the next 24 hours.

\emph{\textbf{Breath Weapon (Recharge 5-6).}} The dragon uses one of the following breath weapons:

\emph{Icy Breath.} The dragon exhales an icy blast in a 27 meter cone. Each creature in the area must make a Fortitude save DC 24, taking 67 (15d8) cold damage on a failed save, or half as much damage on a successful one.

\emph{Paralyzing Breath.} The dragon exhales paralyzing gas in a 23 meter cone. Each creature in the area must succeed on a Fortitude save of 24 or be paralyzed for 1 minute. A creature can repeat the saving throw at the end of each of its rounds, ending the effect on itself on a success.

\emph{\textbf{Shapeshifting.}} The dragon can magically transform into a humanoid or beast whose challenge rating is equal to or lower than its own, or return to its true form. Upon death he returns to his true form.

Whatever equipment it is wearing or carrying is absorbed or transported into the new form (the dragon's choice).

In its new form, the dragon retains its Traits, Hit Points, Hit Dice, speech, proficiencies, Legendary Stamina, lair actions, and Intelligence, Wisdom, and Charisma scores, in addition to this action. His statistics and abilities are otherwise replaced by those of the new form, except the new form's additional actions.

\textbf{Additional Shares}

The dragon can perform 3 additional Actions, chosen from the following options. He can only use one Additional option at a time, and only at the end of another creature's round. The dragon recovers additional Actions spent at the start of its round.

\textbf{Wing Attack (Costs 2 Actions).} The dragon flaps its wings. Each creature within 5 feet of the dragon must succeed on a DC 25 Reflex save or take 17 (2d6 + 10) bludgeoning damage and be knocked prone. The dragon can then fly at up to half its flight speed.

\textbf{Tail Attack.} The dragon makes a tail attack.

\textbf{Spot.} The dragon makes a Wisdom (Awareness) check.

\textbf{Ecology}\\
Environment: Temperate Mountains\\
Organization: Solitaire\\
\textbf{Treasure}: Triple\\
\textbf{Description}\\
Of all dragons, silver dragons are the bravest, adhering to a code of chivalry that requires them to help the weak, defeat evil, and behave honorably.\\
\textbf{Spells}\index{Silver Dragon Spells}\\
This Dragon's favorite spells are:\\
- Slowness\\
- To manufacture\\
- Dream

\medskip\index[Monster]{Adult Silver Dragon}\textbf{Adult Silver Dragon}

\emph{Huge dragon, legal good}

\textbf{STRENGTH} +8

\textbf{DEXTERITY} +0

\textbf{CONSTITUTION} +7

\textbf{INTELLIGENCE} +3

\textbf{WISDOM} +1

\textbf{CHARISMA} +5

\textbf{Initiative} +3 -- \textbf{Defense} 27

\textbf{Hit Points} 243 (18d12 + 126)

\textbf{Movement} 12 m, flight 24 m

\textbf{Saving Throws} Fortitude +22, Reflexes +16, Will +17

\textbf{Skills} Arcane +8, Stealth +5, Awareness +11, History +8

\textbf{Damage Immunity} cold

\textbf{Senses} darkvision 120 ft., blindsight 60 ft

\textbf{Languages} Common, Draconic

\textbf{Challenge} 16 (1500 PX)

€18377 €{€18378 €{Legendary Resistance (3 / Day).}} If the dragon fails a saving throw, it can choose to succeed instead.

\textbf{Shares}

\emph{\textbf{Multiattack.}} The dragon can use its Frightening Presence. Then make three attacks: one with the bite and two with the claws.

\emph{\textbf{Claw.} Melee weapon attack}: +27 to hit, reach 1 m, one target.

\emph{Hits:} 15 (2d6 + 8) slashing damage, 1 bleed damage.

\emph{\textbf{Tail.} Melee weapon attack}: +27 to hit, reach 5 meters, one target.

\emph{Hits:} 17 (2d8 + 8) bludgeoning damage.

\emph{\textbf{Bite.} Melee weapon attack}: +27 to hit, reach 10 ft., one target.

\emph{Hits:} 19 (2d10 + 8) piercing damage.

\emph{\textbf{Frightening Presence.}} Each creature chosen by the dragon, within 120 feet of it and aware of its presence, must succeed on a DC 18 Will save or be frightened for 1 minute. A creature can repeat the saving throw at the end of each of its rounds, ending the effect on a success. If the creature's saving throw succeeds or the effect ends for it, the creature is immune to the dragon's Frightening Presence for the next 24 hours.

\emph{\textbf{Breath Weapon (Recharge 5-6).}} The dragon uses one of the following breath weapons:

\emph{Icy Breath.} The dragon exhales an icy blast in a 60-foot cone. Each creature in the area must make a Fortitude save DC 20, taking 58 (13d8) cold damage on a failed save, or half as much damage on a successful one.

\emph{Paralyzing Breath.} The dragon exhales paralyzing gas in a 60-foot cone. Each creature in the area must succeed on a Fortitude save of 20 or be paralyzed for 1 minute. A creature can repeat the saving throw at the end of each of its rounds, ending the effect for itself on a success.

\emph{\textbf{Shapeshifting.}} The dragon can magically transform into a humanoid or beast whose challenge rating is equal to or lower than its own, or return to its true form. Upon death he returns to his true form. Whatever equipment it is wearing or carrying is absorbed or transported into the new form (the dragon's choice).

In its new form, the dragon retains its Traits, Hit Points, Hit Dice, speech, proficiencies, Legendary Stamina, lair actions, and Intelligence, Wisdom, and Charisma scores, in addition to this action. His statistics and abilities are otherwise replaced by those of the new form, except the new form's additional actions.

\textbf{Additional Shares}

The dragon can perform 3 additional Actions, chosen from the following options. He can only use one Additional option at a time, and only at the end of another creature's round. The dragon recovers additional Actions spent at the start of its round.

\textbf{Wing Attack (Costs 2 Actions).} The dragon flaps its wings. Each creature within 10 feet of the dragon must succeed on a DC 21 Reflex save or take 15 (2d6 + 8) bludgeoning damage and be knocked prone. The dragon can then fly up to half its flight movement.

\textbf{Tail Attack.} The dragon makes a tail attack.

\textbf{Spot.} The dragon makes a Wisdom (Awareness) check.



\emph{\textbf{Angry:}} The Adult Silver Dragon can perform these actions for 2 Actions.

\emph{Focus}: The creature interrupts an ongoing mental effect on itself

\emph{Brutality}: the creature attacks with unprecedented ferocity. +1d6 to attack roll, an extra 6 is added to the critical count until the end of the fight.


\textbf{Ecology}\\
Environment: Temperate Mountains\\
Organization: Solitaire\\
\textbf{Treasure}: Triple\\
\textbf{Description}\\
Of all dragons, silver dragons are the bravest, adhering to a code of chivalry that requires them to help the weak, defeat evil, and behave honorably.\\
\textbf{Spells}\index{Silver Dragon Spells}\\
This Dragon's favorite spells are:\\
- Slowness\\
- To manufacture\\
- Dream

\medskip\index[Monster]{Young Silver Dragon}\textbf{Young Silver Dragon}

\emph{Large dragon, legal good}

\textbf{STRENGTH} +6

\textbf{DEXTERITY} +0

\textbf{CONSTITUTION} +5

\textbf{INTELLIGENCE} +2

\textbf{WISDOM} +0

\textbf{CHARRISMA} +4

\textbf{Initiative} +2 -- \textbf{Defense} 23

\textbf{Hit Points} 168 (16d10 + 80)

\textbf{Movement} 12 m, flight 24 m

\textbf{Saving Throws} Fortitude +10, Reflexes +8, Will +12

\textbf{Skills} Arcana +6, Stealth +4, Awareness +8, History +6

\textbf{Damage Immunity} cold

\textbf{Senses} darkvision 120 ft., blindsight 30 ft

\textbf{Languages} Common, Draconic

\textbf{Challenge} 9 (5000 PX)

\textbf{Shares}

\emph{\textbf{Multiattack.}} The dragon can make three attacks: one with its bite and two with its claws.

\emph{\textbf{Claw.} Melee weapon attack}: +15 to hit, reach 1 m, one target.

\emph{Hits:} 13 (2d6 + 6) slashing damage, 1 bleed damage.

\emph{\textbf{Bite.} Melee weapon attack}: +15 to hit, reach 10 ft., one target.

\emph{Hits:} 17 (2d10 + 6) piercing damage.

\emph{\textbf{Breath Weapon (Recharge 5-6).}} The dragon uses one of the following breath weapons:

\emph{Icy Breath.} The dragon exhales an icy blast in a 30-foot cone. Each creature in the area must make a Fortitude save DC 17, taking 54 (12d8) cold damage on a failed save, or half as much damage on a successful one.

\emph{Paralyzing Breath.} The dragon exhales paralyzing gas in a 30-foot cone. Each creature in the area must succeed on a Fortitude save 17 or be paralyzed for 1 minute. A creature can repeat the saving throw at the end of each of its rounds, ending the effect for itself on a success.

€18445 € {€18446 € {Angry:}} the young silver dragon recharges one of her breath weapons.

\textbf{Ecology}\\
Environment: Temperate Mountains\\
Organization: Solitaire\\
\textbf{Treasure}: Triple\\
\textbf{Description}\\
Of all dragons, silver dragons are the bravest, adhering to a code of chivalry that requires them to help the weak, defeat evil, and behave honorably.\\
\textbf{Spells}\index{Silver Dragon Spells}\\
This Dragon's favorite spells are:\\
- Slowness\\
- To manufacture\\
- Dream

\medskip\index[Monster]{Silver Hatchling Dragon}\textbf{Silver Hatchling Dragon}

\emph{Average dragon, legal good}

\textbf{STRENGTH} +4

\textbf{DEXTERITY} +0

\textbf{CONSTITUTION} +3

\textbf{INTELLIGENCE} +1

\textbf{WISDOM} +0

\textbf{CHARISMA} +2

\textbf{Initiative} +1 -- \textbf{Defense} 18

\textbf{Hit Points} 45 (6d8 + 18)

\textbf{Movement} 9 m, flight 18 m

\textbf{Saving Throws} Fortitude +3, Reflexes +3, Will +2

\textbf{Skills} Stealth +2, Awareness +4

\textbf{Damage Immunity} cold

\textbf{Senses} Darkvision 60 ft., blindsight 10 ft

\textbf{Languages} Draconic

\textbf{Challenge} 2 (450 PX)

\textbf{Shares}

\emph{\textbf{Bite.} Melee weapon attack}: +6 to hit, reach 1 m, one target.

\emph{Hits:} 9 (1d10 + 4) piercing damage.

\emph{\textbf{Breath Weapon (Recharge 5-6).}} The dragon uses one of the following breath weapons:

\emph{Icy Breath.} The dragon exhales an icy blast in a 5 meter cone. Each creature in the area must make a Fortitude saving throw 13, taking 18 (4d8) cold damage on a failed save, or half as much damage on a successful one.

\emph{Paralyzing Breath.} The dragon exhales paralyzing gas in a 5 meter cone. Each creature in the area must succeed on a Fortitude save 13 or be paralyzed for 1 minute. A creature can repeat the saving throw at the end of each of its rounds, ending the effect for itself on a success.

\textbf{Ecology}\\
Environment: Temperate Mountains\\
Organization: Solitaire\\
\textbf{Treasure}: Triple\\
\textbf{Description}\\
Of all dragons, silver ones are the bravest, adhering to a code of chivalry that requires them to help the weak, defeat evil, and behave honorably.


\medskip\index[Monstery]{Ancient Bronze Dragon}\textbf{Ancient Bronze Dragon}

\emph{Gorgous dragon, chaotic good}

\textbf{STRENGTH} +9

\textbf{DEXTERITY} +0

\textbf{CONSTITUTION} +8

\textbf{INTELLIGENCE} +4

\textbf{WISDOM} +3

\textbf{CHARISMA} +5

\textbf{Initiative} +4 -- \textbf{Defense} 33

\textbf{Hit Points} 444 (24x3d6 + 192)

\textbf{Move} 12 m, swim 12 m, fly 24 m

\textbf{Saving Throws} Fortitude +30, Reflexes +22, Will +24

\textbf{Skills} Stealth +7, Sense Emotions +10, Awareness +17

\textbf{Damage Immunity} Electricity, weapons +1

\textbf{Senses} darkvision 36 m, blindsight 18 m

\textbf{Languages} Common, Draconic

\textbf{Challenge} 22 (41000 PX)

\emph{\textbf{Amphibian.}} The dragon can breathe air and water.

€18505 € {€18506 €{Legendary Resistance (3 / Day).}} If the dragon fails a saving throw, it can choose to succeed instead.

\emph{\textbf{Magic Resistance:}} 3lv

\textbf{Shares}

\emph{\textbf{Multiattack.}} The dragon can use its Frightening Presence. Then make three attacks: one with the bite and two with the claws.

\emph{\textbf{Claw.} Melee weapon attack}: +30 to hit, reach 10 ft., one target.

\emph{Hits:} 16 (2d6 + 9) slashing damage, 3/20 bleed damage.

\emph{\textbf{Tail.} Melee weapon attack}: +30 to hit, reach 20 ft., one target.

\emph{Hits:} 18 (2d8 + 9) bludgeoning damage.

\emph{\textbf{Bite.} Melee weapon attack}: +30 to hit, reach 5 meters, one target.

\emph{Hits:} 20 (2d10 + 9) piercing damage.

\emph{\textbf{Frightening Presence.}} Each creature chosen by the dragon, within 120 feet of it and aware of its presence, must succeed on a DC 20 Will save or be frightened for 1 minute. A creature can repeat the saving throw at the end of each of its rounds, ending the effect on a success. If the creature's saving throw succeeds or the effect ends for it, the creature is immune to the dragon's Frightening Presence for the next 24 hours.

\emph{\textbf{Breath Weapon (Recharge 5-6).}} The dragon uses one of the following breath weapons:

\emph{Lightning Breath.} The dragon exhales lightning in a line 36 meters long and 3 meters wide. Each creature in the line must make a DC 23 Reflex save, taking 88 (16d10) lightning damage on a failed save, or half as much damage on a successful one. \emph{Repulsive Breath.} The dragon exhales repulsive energy in a 30-foot cone. Each creature in that area must succeed on a DC 23 Fortitude save or be moved 60 feet away from them.
dragon.

\emph{\textbf{Shapeshifting.}} The dragon can magically transform into a humanoid or beast whose challenge rating is equal to or lower than its own, or return to its true form. Upon death he returns to his true form. Whatever equipment it is wearing or carrying is absorbed or transported into the new form (the dragon's choice).

In its new form, the dragon retains its Traits, Hit Points, Hit Dice, speech, proficiencies, Legendary Stamina, lair actions, and Intelligence, Wisdom, and Charisma scores, in addition to this action. His statistics and abilities are otherwise replaced by those of the new form, except the new form's additional actions.

\textbf{Additional Shares}

The dragon can perform 3 additional Actions, chosen from the following options. He can only use one Additional option at a time, and only at the end of another creature's round. The dragon recovers additional Actions spent at the start of its round.

\textbf{Wing Attack (Costs 2 Actions).} The dragon flaps its wings. Each creature within 5 feet of the dragon must succeed on a DC 24 Reflex save or take 16 (2d6 + 9) bludgeoning damage and be knocked prone. The dragon can then fly up to half its flight movement.

\textbf{Tail Attack.} The dragon makes a tail attack.

\textbf{Spot.} The dragon makes a Wisdom (Awareness) check.

\textbf{Ecology}\\
Environment: Temperate Coastal Areas\\
Organization: Solitaire\\
\textbf{Treasure}: Triple\\
\textbf{Description}\\
Bronze dragons are known to ally themselves with travelers and adventurers if the cause and reward are right and proper\\
\textbf{Spells}\index{Bronze Dragon Spells}\\
This Dragon's favorite spells are:\\
- Orb of Invulnerability\\
- Freedom of movement


\medskip\index[Monstery]{Adult Bronze Dragon}\textbf{Adult Bronze Dragon}

\emph{Huge dragon, chaotic good}

\textbf{STRENGTH} +7

\textbf{DEXTERITY} +0

\textbf{CONSTITUTION} +6

\textbf{INTELLIGENCE} +3

\textbf{WISDOM} +2

\textbf{CHARISMA} +4

\textbf{Initiative} +3 -- \textbf{Defense} 27

\textbf{Hit Points} 212 (17d12 + 102)

\textbf{Move} 12 m, swim 12 m, fly 24 m

\textbf{Saving Throws} Fortitude +21, Reflexes +14, Will +16

\textbf{Skills} Stealth +5, Sense Emotions +7, Awareness +12

\textbf{Damage Immunity} Electricity

\textbf{Senses} darkvision 120 ft., blindsight 60 ft

\textbf{Languages} Common, Draconic

\textbf{Challenge} 15 (13000 PX)

\emph{\textbf{Amphibian.}} The dragon can breathe air and water.

€18560 € {€18561 € {Legendary Resistance (3 / Day).}} If the dragon fails a saving throw, it can choose to succeed instead.

\textbf{Shares}

\emph{\textbf{Multiattack.}} The dragon can use its Frightening Presence and then make three attacks: one with its bite and two with its claws.

\emph{\textbf{Claw.} Melee weapon attack}: +24 to hit, reach 1 m, one target.

\emph{Hits:} 14 (2d6 + 7) slashing damage, 1 bleed damage.

\emph{\textbf{Tail.} Melee weapon attack}: +24 to hit, reach 5 meters, one target.

\emph{Hits:} 16 (2d8 + 7) bludgeoning damage.

\emph{\textbf{Bite.} Melee weapon attack}: +24 to hit, reach 10 ft., one target.

\emph{Hits:} 18 (2d10 + 7) piercing damage.

\emph{\textbf{Frightening Presence.}} Each creature chosen by the dragon, within 120 feet of it and aware of its presence, must succeed on a DC 17 Will save or be frightened for 1 minute. A creature can repeat the saving throw at the end of each of its rounds, ending the effect on a success. If the creature's saving throw succeeds or the effect ends for it, the creature is immune to the dragon's Frightening Presence for the next 24 hours.

\emph{\textbf{Breath Weapon (Recharge 5-6).}} The dragon uses one of the following breath weapons:

\emph{Lightning Breath.} The dragon exhales lightning in a line 27 meters long and 1 meter wide. Each creature in the line must make a DC 19 Reflex saving throw, taking 66 (12d10) lightning damage on a failed save, or half as much damage on a successful one. \emph{Repulsive Breath.} The dragon exhales repulsive energy in a 30-foot cone. Each creature in that area must succeed on a DC 19 Fortitude save or be moved 60 feet away from the dragon.

\emph{\textbf{Shapeshifting.}} The dragon can magically transform into a humanoid or beast whose challenge rating is equal to or lower than its own, or return to its true form. Upon death he returns to his true form. Whatever equipment it is wearing or carrying is absorbed or transported into the new form (the dragon's choice).

In its new form, the dragon retains its Traits, Hit Points, Hit Dice, speech, proficiencies, Legendary Stamina, lair actions, and Intelligence, Wisdom, and Charisma scores, in addition to this action. His statistics and abilities are otherwise replaced by those of the new form, except the new form's additional actions.

\textbf{Additional Shares}

The dragon can perform 3 additional Actions, chosen from the following options. He can only use one Additional option at a time, and only at the end of another creature's round. The dragon recovers additional Actions spent at the start of its round.

\textbf{Wing Attack (Costs 2 Actions).} The dragon flaps its wings. Each creature within 10 feet of the dragon must succeed on a DC 20 Reflex save or take 14 (2d6 + 7) bludgeoning damage and be knocked prone. The dragon can then fly up to half its flight movement.

\textbf{Tail Attack.} The dragon makes a tail attack.

\textbf{Spot.} The dragon makes a Wisdom (Awareness) check.

\emph{\textbf{Angry:}} The Adult Bronze Dragon can perform these actions for 2 Actions.

\emph{Focus}: The creature interrupts an ongoing mental effect on itself

\emph{Brutality}: the creature attacks with unprecedented ferocity. +1d6 to attack roll, an extra 6 is added to the critical count until the end of the fight.


\textbf{Ecology}\\
Environment: Temperate Coastal Areas\\
Organization: Solitaire\\
\textbf{Treasure}: Triple\\
\textbf{Description}\\
Bronze dragons are known to ally themselves with travelers and adventurers if the cause and reward are right and proper\\
\textbf{Spells}\index{Bronze Dragon Spells}\\
This Dragon's favorite spells are:\\
- Orb of Invulnerability\\
- Freedom of movement

\medskip\index[Monstery]{Young Bronze Dragon}\textbf{Young Bronze Dragon}

\emph{Great Dragon, Chaotic Good}

\textbf{STRENGTH} +5

\textbf{DEXTERITY} +0

\textbf{CONSTITUTION} +4

\textbf{INTELLIGENCE} +2

\textbf{WISDOM} +1

\textbf{CHARISMA} +3

\textbf{Initiative} +2 -- \textbf{Defense} 22

\textbf{Hit Points} 142 (15d10 + 60)

\textbf{Move} 12 m, swim 12 m, fly 24 m

\textbf{Saving Throws} Fortitude +10, Reflexes +8, Will +10

\textbf{Skills} Stealth +3, Sense Emotions +4, Awareness +7

\textbf{Damage Immunity} Electricity

\textbf{Senses} darkvision 120 ft., blindsight 30 ft

\textbf{Languages} Common, Draconic

\textbf{Challenge} 8 (3900 PX)

\emph{\textbf{Amphibian.}} The dragon can breathe air and water.

\textbf{Shares}

\emph{\textbf{Multiattack.}} The dragon can use three attacks: one with its bite and two with its claws.

\emph{\textbf{Claw.} Melee weapon attack}: +12 to hit, reach 1 m, one target.

\emph{Hits:} 12 (2d6 + 5) slashing damage, 1 bleed damage.

\emph{\textbf{Bite.} Melee weapon attack}: +12 to hit, reach 10 ft., one target.

\emph{Hits:} 16 (2d10 + 5) piercing damage.

\emph{\textbf{Breath Weapon (Recharge 5-6).}} The dragon uses one of the following breath weapons:

\emph{Lightning Breath.} The dragon exhales lightning in a line 18 meters long and 1 meter wide. Each creature in the line must make a DC 15 Reflex saving throw, taking 55 (10d10) lightning damage on a failed save, or half as much damage on a successful one.

\emph{Repulsive Breath.} The dragon exhales repulsive energy in a 30-foot cone. Each creature in that area must succeed on a DC 15 Fortitude save or be moved 40 feet away from the dragon.

€18630 ​​€ {€18631 € {Angry:}} the Young Bronze Dragon recharges one of his breath weapons.

\textbf{Ecology}\\
Environment: Temperate Coastal Areas\\
Organization: Solitaire\\
\textbf{Treasure}: Triple\\
\textbf{Description}\\
Bronze dragons are known to ally themselves with travelers and adventurers if the cause and reward are right and proper\\
\textbf{Spells}\index{Bronze Dragon Spells}\\
This Dragon's favorite spells are:\\
- Orb of Invulnerability\\
- Freedom of movement


\medskip\index[Monstery]{Bronze Hatchling Dragon}\textbf{Bronze Hatchling Dragon}

\emph{Average dragon, chaotic good}

\textbf{STRENGTH} +3

\textbf{DEXTERITY} +0

\textbf{CONSTITUTION} +2

\textbf{INTELLIGENCE} +1

\textbf{WISDOM} +0

\textbf{CHARISMA} +2

\textbf{Initiative} +1 -- \textbf{Defense} 18

\textbf{Hit Points} 32 (5d8 + 10)

\textbf{Move} 9 m, swim 9 m, fly 18 m

\textbf{Saving Throws} Fortitude +2, Reflexes +1, Will +1

\textbf{Skills} Stealth +2, Awareness +4

\textbf{Damage Immunity} Electricity

\textbf{Senses} Darkvision 60 ft., blindsight 10 ft

\textbf{Languages} Draconic

\textbf{Challenge} 2 (450 PX)

\emph{\textbf{Amphibian.}} The dragon can breathe air and water.

\textbf{Shares}

\emph{\textbf{Bite.} Melee weapon attack}: +5 to hit,
range 1 m, one target.

\emph{Hits:} 8 (1d10 + 3) piercing damage.

\emph{\textbf{Breath Weapon (Recharge 5-6).}} The dragon uses one of the following breath weapons:

\emph{Lightning Breath.} The dragon exhales lightning in a line 12 meters long and 1 meter wide. Each creature in the line must make a DC 12 Reflex saving throw, taking 16 (3d10) lightning damage on a failed save, or half as much damage on a successful one.

\emph{Repulsive Breath.} The dragon exhales repulsive energy in a 30-foot cone. Each creature in that area must succeed on a DC 12 Fortitude save or be moved 30 feet away from the dragon.

\textbf{Ecology}\\
Environment: Temperate Coastal Areas\\
Organization: Solitaire\\
\textbf{Treasure}: Triple\\
\textbf{Description}\\
Bronze dragons are known to ally themselves with travelers and adventurers if the cause and reward are just and appropriate.

\medskip\index[Monstery]{Ancient Gold Dragon}\textbf{Ancient Gold Dragon}

\emph{Mammoth dragon, legal good}

\textbf{STRENGTH} +10

\textbf{DEXTERITY} +2

\textbf{CONSTITUTION} +9

\textbf{INTELLIGENCE} +4

\textbf{WISDOM} +3

\textbf{CHARISMA} +9

\textbf{Initiative} +4 -- \textbf{Defense} 34

\textbf{Hit Points} 546 (28x3d6 + 252)

\textbf{Move} 12 m, swim 12 m, fly 24 m

\textbf{Saving Throws} Fortitude +33, Reflexes +26, Will +27

\textbf{Skills} Stealth +9, Sense Emotions +10, Awareness +17, Deception +16

\textbf{Damage Immunity} Fire, weapons +1

\textbf{Senses} darkvision 120 ft., blindsight 60 ft

\textbf{Languages} Common, Draconic

\textbf{Challenge} 24 (62000 PX)

\emph{\textbf{Amphibian.}} The dragon can breathe air and water.

€18692 €{€18693 €{Legendary Resistance (3 / Day).}} If the dragon fails a saving throw, it can choose to succeed instead.

\emph{\textbf{Magic Resistance:}} 3lv

\textbf{Shares}

\emph{\textbf{Multiattack.}} The dragon can use its Frightening Presence. Then make three attacks: one with the bite and two with the claws.

\emph{\textbf{Claw.} Melee weapon attack}: +30 to hit, reach 10 ft., one target.

\emph{Hits:} 17 (2d6 + 10) slashing damage, 3/20 bleed damage.

\emph{\textbf{Tail.} Melee weapon attack}: +30 to hit, reach 20 ft., one target.

\emph{Hits:} 19 (2d8 + 10) bludgeoning damage.

\emph{\textbf{Bite.} Melee weapon attack}: +30 to hit, reach 5 metres, one target.

\emph{Hits:} 21 (2d10 + 10) piercing damage.

\emph{\textbf{Frightening Presence.}} Each creature chosen by the dragon, within 120 feet of it and aware of its presence, must succeed on a DC 24 Will save or be frightened for 1 minute. A creature can repeat the saving throw at the end of each of its rounds, ending the effect on a success. If the creature's saving throw succeeds or the effect ends for it, the creature is immune to the dragon's Frightening Presence for the next 24 hours.

\emph{\textbf{Breath Weapon (Recharge 5-6).}} The dragon uses one of the following breath weapons:

\emph{Fiery Breath.} The dragon exhales fire in a 27 meter cone. Each creature in the area must make a DC 24 Reflex saving throw, taking 71 (13d10) fire damage on a failed save, or half as much damage on a successful one.

\emph{Weakening Breath.} The dragon exhales gas in a 27 meter cone. Each creature in that area must succeed on a DC 24 Fortitude save or have -1d6 on Strength-based attack rolls, Strength checks, and Fortitude saves for 1 minute. A creature can repeat the saving throw at the end of each of its rounds, ending the effect on itself on a success.

\emph{\textbf{Shapeshifting.}} The dragon can magically transform into a humanoid or beast whose challenge rating is equal to or lower than its own, or return to its true form. Upon death he returns to his true form. Whatever equipment it is wearing or carrying is absorbed or transported into the new form (the dragon's choice).

In its new form, the dragon retains its Traits, Hit Points, Hit Dice, speech, proficiencies, Legendary Stamina, lair actions, and Intelligence, Wisdom, and Charisma scores, in addition to this action. His statistics and abilities are otherwise replaced by those of the new form, except the new form's additional actions.

\textbf{Additional Shares}

The dragon can perform 3 additional Actions, chosen from the following options. He can only use one Additional option at a time, and only at the end of another creature's round. The dragon recovers additional Actions spent at the start of its round.

\textbf{Wing Attack (Costs 2 Actions).} The dragon flaps its wings. Each creature within 5 feet of the dragon must succeed on a DC 25 Reflex save or take 17 (2d6 + 10) bludgeoning damage and be knocked prone. The dragon can then fly up to half its flight movement.

\textbf{Tail Attack.} The dragon makes a tail attack.

\textbf{Spot.} The dragon makes a Wisdom (Awareness) check.

\textbf{Ecology}\\
Environment: Warm plains\\
Organization: Solitaire\\
\textbf{Treasure}: Triple\\
\textbf{Description}\\
Golden dragons are the emblem of virtue. The other dragons of Ljust revere them as agents of divine powers and exemplary members of the draconic race, and often seek them for advice or aid.\\
\textbf{Spells}\index{Golden Dragon Spells}\\
This Dragon's favorite spells are:\\
- Healing\\
- Superior Restaurant\\
- Black Tentacles


\medskip\index[Monstery]{Adult Golden Dragon}\textbf{Adult Golden Dragon}

\emph{Huge dragon, legal good}

\textbf{STRENGTH} +8

\textbf{DEXTERITY} +2

\textbf{CONSTITUTION} +7

\textbf{INTELLIGENCE} +3

\textbf{WISDOM} +2

\textbf{CHARRISMA} +7

\textbf{Initiative} +3 -- \textbf{Defense} 28

\textbf{Hit Points} 256 (19d12 + 133)

\textbf{Move} 12 m, swim 12 m, fly 24 m

\textbf{Saving Throws} Fortitude +24, Reflexes +19, Will +19

\textbf{Skills} Stealth +8, Sense Emotions +8, Awareness +14, Deception +13

\textbf{Damage Immunity} Fire

\textbf{Senses} darkvision 120 ft., blindsight 60 ft

\textbf{Languages} Common, Draconic

\textbf{Challenge} 17 (18000 PX)

\emph{\textbf{Amphibian.}} The dragon can breathe air and water.

€18747 €{€18748 €{Legendary Resistance (3 / Day).}} If the dragon fails a saving throw, it can choose to succeed instead.

\textbf{Shares}

\emph{\textbf{Multiattack.}} The dragon can use its Frightening Presence. Then make three attacks: one with the bite and two with the claws.

\emph{\textbf{Claw.} Melee weapon attack}: +28 to hit, reach 1 m, one target.

\emph{Hits:} 15 (2d6 + 8) slashing damage, 1 bleed damage.

\emph{\textbf{Tail.} Melee weapon attack}: +28 to hit, reach 5 meters, one target.

\emph{Hits:} 17 (2d8 + 8) bludgeoning damage.

\emph{\textbf{Bite.} Melee weapon attack}: +28 to hit, reach 10 ft., one target.

\emph{Hits:} 19 (2d10 + 8) piercing damage.

\emph{\textbf{Frightening Presence.}} Each creature chosen by the dragon, within 120 feet of it and aware of its presence, must succeed on a DC 21 Will save or be frightened for 1 minute. A creature can repeat the saving throw at the end of each of its rounds, ending the effect on a success. If the creature's saving throw succeeds or the effect ends for it, the creature is immune to the dragon's Frightening Presence for the next 24 hours.

\emph{\textbf{Breath Weapon (Recharge 5-6).}} The dragon uses one of the following breath weapons:

\emph{Fiery Breath.} The dragon exhales fire in a 60-foot cone. Each creature in the area must make a DC 21 Reflex saving throw, taking 66 (12d10) fire damage on a failed save, or half as much damage on a successful one.

\emph{Weakening Breath.} The dragon exhales gas in a 60-foot cone. Each creature in that area must succeed on a DC 21 Fortitude save or have -1d6 on Strength-based attack rolls, Strength checks, and Fortitude saves for 1 minute. A creature can repeat the saving throw at the end of each of its rounds, ending the effect on itself on a success.

\emph{\textbf{Shapeshifting.}} The dragon can magically transform into a humanoid or beast whose challenge rating is equal to or lower than its own, or return to its true form. Upon death he returns to his true form. Whatever equipment it is wearing or carrying is absorbed or transported into the new form (the dragon's choice).

In its new form, the dragon retains its Traits, Hit Points, Hit Dice, speech, proficiencies, Legendary Stamina, lair actions, and Intelligence, Wisdom, and Charisma scores, in addition to this action. His statistics and abilities are otherwise replaced by those of the new form, except the new form's additional actions.

\textbf{Additional Shares}

The dragon can perform 3 additional Actions, chosen from the following options. He can only use one Additional option at a time, and only at the end of another creature's round. The dragon recovers additional Actions spent at the start of its round.

\textbf{Wing Attack (Costs 2 Actions).} The dragon flaps its wings. Each creature within 10 feet of the dragon must succeed on a DC 22 Reflex save or take 15 (2d6 + 8) bludgeoning damage and be knocked prone. The dragon can then fly up to half its flight movement.

\textbf{Tail Attack.} The dragon makes a tail attack.

\textbf{Spot.} The dragon makes a Wisdom (Awareness) check.

\emph{\textbf{Angry:}} The Adult Gold Dragon can perform these actions for 2 Actions.

\emph{Focus}: The creature interrupts an ongoing mental effect on itself

\emph{Brutality}: the creature attacks with unprecedented ferocity. +1d6 to attack roll, an extra 6 is added to the critical count until the end of the fight.


\textbf{Ecology}\\
Environment: Warm plains\\
Organization: Solitaire\\
\textbf{Treasure}: Triple\\
\textbf{Description}\\
Golden dragons are the emblem of virtue. The other dragons of Ljust revere them as agents of divine powers and exemplary members of the draconic race, and often seek them for advice or aid.\\
\textbf{Spells}\index{Golden Dragon Spells}\\
This Dragon's favorite spells are:\\
- Healing\\
- Superior Restaurant\\
- Black Tentacles


\medskip\index[Monster]{Young Golden Dragon}\textbf{Young Golden Dragon}

\emph{Large dragon, legal good}

\textbf{STRENGTH} +6

\textbf{DEXTERITY} +2

\textbf{CONSTITUTION} +5

\textbf{INTELLIGENCE} +3

\textbf{WISDOM} +1

\textbf{CHARISMA} +5

\textbf{Initiative} +3 -- \textbf{Defense} 23

\textbf{Hit Points} 178 (17d10 + 85)

\textbf{Move} 12 m, swim 12 m, fly 24 m

\textbf{Saving Throws} Fortitude +12, Reflexes +9, Will +13

\textbf{Skills} Stealth +6, Sense Emotions +5, Awareness +9, Deception +9

\textbf{Damage Immunity} Fire

\textbf{Senses} darkvision 120 ft., blindsight 30 ft

\textbf{Languages} Common, Draconic

\textbf{Challenge} 10 (5900 PX)

\emph{\textbf{Amphibian.}} The dragon can breathe air and water.

\textbf{Shares}

\emph{\textbf{Multiattack.}} The dragon can make three attacks: one with its bite and two with its claws.

\emph{\textbf{Claw.} Melee weapon attack}: +16 to hit, reach 1 m, one target.

\emph{Hits:} 13 (2d6 + 6) slashing damage, 1 bleed damage.

\emph{\textbf{Bite.} Melee weapon attack}: +16 to hit, reach 10 ft., one target.

\emph{Hits:} 17 (2d10 + 6) piercing damage.

\emph{\textbf{Breath Weapon (Recharge 5-6).}} The dragon uses one of the following breath weapons:

\emph{Fiery Breath.} The dragon exhales fire in a 30-foot cone. Each creature in the area must make a DC 17 Reflex saving throw, taking 55 (10d10) fire damage on a failed save, or half as much damage on a successful one.

\emph{Weakening Breath.} The dragon exhales gas in a 30-foot cone. Each creature in that area must succeed on a DC 17 Fortitude save or have -1d6 on Strength-based attack rolls, Strength checks, and Fortitude saves for 1 minute. A creature can repeat the saving throw at the end of each of its rounds, ending the effect on itself on a success.

€18817 € {€18818 € {Angry:}} the young golden dragon recharges one of his breath weapons. Costs 1 Action.

\textbf{Ecology}\\
Environment: Warm plains\\
Organization: Solitaire\\
\textbf{Treasure}: Triple\\
\textbf{Description}\\
Golden dragons are the emblem of virtue. The other dragons of Ljust revere them as agents of divine powers and exemplary members of the draconic race, and often seek them for advice or aid.\\
\textbf{Spells}\index{Golden Dragon Spells}\\
This Dragon's favorite spells are:\\
- Healing\\
- Superior Restaurant\\
- Black Tentacles


\medskip\index[Monstery]{Golden Baby Dragon}\textbf{Golden Baby Dragon}

\emph{Average dragon, legal good}

\textbf{STRENGTH} +4

\textbf{DEXTERITY} +2

\textbf{CONSTITUTION} +3

\textbf{INTELLIGENCE} +2

\textbf{WISDOM} +0

\textbf{CHARISMA} +3

\textbf{Initiative} +2 -- \textbf{Defense} 19

\textbf{Hit Points} 60 (8d8 + 24)

\textbf{Move} 9 m, swim 9 m, fly 18 m

\textbf{Saving Throws} Fortitude +3, Reflexes +2, Will +1

\textbf{Skills} Stealth +4, Awareness +4

\textbf{Damage Immunity} Fire

\textbf{Senses} Darkvision 60 ft., blindsight 10 ft

\textbf{Languages} Draconic

\textbf{Challenge} 3 (700 PX)

\emph{\textbf{Amphibian.}} The dragon can breathe air and water.

\textbf{Shares}

\emph{\textbf{Bite.} Melee weapon attack}: +6 to hit, reach 1 m, one target.

\emph{Hits:} 9 (1d10 + 4) piercing damage.

\emph{\textbf{Breath Weapon (Recharge 5-6).}} The dragon uses one of the following breath weapons:

\emph{Fiery Breath.} The dragon exhales fire in a 5 meter cone. Each creature in the area must make a DC 13 Reflex saving throw, taking 22 (4d10) fire damage on a failed save, or half as much damage on a successful one.

\emph{Weakening Breath.} The dragon exhales gas in a 5 meter cone. Each creature in that area must succeed on a DC 13 Fortitude save or have -1d6 on Strength-based attack rolls, Strength checks, and Fortitude saves for 1 minute. A creature can repeat the saving throw at the end of each of its rounds, ending the effect on itself on a success.

\textbf{Ecology}\\
Environment: Warm plains\\
Organization: Solitaire\\
\textbf{Treasure}: Triple\\
\textbf{Description}\\
Golden dragons are the emblem of virtue. The other dragons of Ljust revere them as agents of divine powers and exemplary members of the draconic race, and often seek them for advice or aid.

\medskip\index[Monstery]{Ancient Brass Dragon}\textbf{Ancient Brass Dragon}

\emph{Large dragon, chaotic good}

\textbf{STRENGTH} +8

\textbf{DEXTERITY} +0

\textbf{CONSTITUTION} +7

\textbf{INTELLIGENCE} +3

\textbf{WISDOM} +2

\textbf{CHARRISMA} +4

\textbf{Initiative} +3 -- \textbf{Defense} 30

\textbf{Hit Points} 297 (17x3d6 + 119)

\textbf{Movement} 12 m, dig 12 m, fly 24 m

\textbf{Saving Throws} Fortitude +27, Reflexes +20, Will +22

\textbf{Skills} Stealth +6, Awareness +14, Deception +10, History +9

\textbf{Damage Immunity} Fire, weapons +1

\textbf{Senses} darkvision 120 ft., blindsight 60 ft

\textbf{Languages} Common, Draconic

\textbf{Challenge} 20 (25000 PX)

€18877 €{€18878 €{Legendary Resistance (3 / Day).}} If the dragon fails a saving throw, it can choose to succeed instead.

\emph{\textbf{Magic Resistance:}} 3lv

\textbf{Shares}

\emph{\textbf{Multiattack.}} The dragon can use its Frightening Presence. Then make three attacks: one with the bite and two with the claws.

\emph{\textbf{Claw.} Melee weapon attack}: +30 to hit, reach 10 ft., one target.

\emph{Hits:} 15 (2d6 + 8) slashing damage, 3/20 bleed damage.

\emph{\textbf{Tail.} Melee weapon attack}: +30 to hit, reach 20 ft., one target.

\emph{Hits:} 17 (2d8 + 8) bludgeoning damage.

\emph{\textbf{Bite.} Melee weapon attack}: +30 to hit, reach 5 metres, one target.

\emph{Hits:} 19 (2d10 + 8) piercing damage.

\emph{\textbf{Frightening Presence.}} Each creature chosen by the dragon, within 120 feet of it and aware of its presence, must succeed on a DC 18 Will save or be frightened for 1 minute. A creature can repeat the saving throw at the end of each of its rounds, ending the effect on a success. If the creature's saving throw succeeds or the effect ends for it, the creature is immune to the dragon's Frightening Presence for the next 24 hours.

\emph{\textbf{Breath Weapon (Recharge 5-6).}} The dragon uses one of the following breath weapons:

\emph{Fiery Breath.} The dragon exhales fire in a line 27 meters long and 3 meters wide. Each creature in the line must make a DC 21 Reflex saving throw, taking 56 (16d6) fire damage on a failed save, or half as much damage on a successful one.

\emph{Sleep Breath.} The dragon exhales sleep gas in a 27 meter cone. Each creature in that area must succeed on a Fortitude save of 21 or fall unconscious for 10 minutes. This effect ends if the unconscious creature takes damage or someone takes an action to awaken it.

\emph{\textbf{Shapeshifting.}} The dragon can magically transform into a humanoid or beast whose challenge rating is equal to or lower than its own, or return to its true form. Upon death he returns to his true form. Whatever equipment it is wearing or carrying is absorbed or transported into the new form (the dragon's choice).

In its new form, the dragon retains its Traits, Hit Points, Hit Dice, speech, proficiencies, Legendary Stamina, lair actions, and Intelligence, Wisdom, and Charisma scores, in addition to this action. His statistics and abilities are otherwise replaced by those of the new form, except the new form's additional actions.

\textbf{Additional Shares}

The dragon can perform 3 additional Actions, chosen from the following options. He can only use one Additional option at a time, and only at the end of another creature's round. The dragon recovers additional Actions spent at the start of its round.

\textbf{Wing Attack (Costs 2 Actions).} The dragon flaps its wings. Each creature within 5 feet of the dragon must succeed on a DC 22 Reflex save or take 15 (2d6 + 8) bludgeoning damage and be knocked prone. The dragon can then fly up to half its flight movement.

\textbf{Tail Attack.} The dragon makes a tail attack.

\textbf{Spot.} The dragon makes a Wisdom (Awareness) check.

\textbf{Ecology}\\
Environment: Hot Deserts\\
Organization: Solitaire\\
\textbf{Treasure}: Triple\\
\textbf{Description}\\
Excellent conversationalists, brass dragons prefer to talk rather than fight. Brass dragons lair near humanoid settlements, where they can hear the latest news and gossip.\\
\textbf{Spells}\index{Brass Dragon Spells}\\
This Dragon's favorite spells are:\\
- Vision of the truth\\
- Knowledge of Legends\\
- Scrutinize


\medskip\index[Monstery]{Adult Brass Dragon}\textbf{Adult Brass Dragon}

\emph{Huge dragon, chaotic good}

\textbf{STRENGTH} +6

\textbf{DEXTERITY} +0

\textbf{CONSTITUTION} +5

\textbf{INTELLIGENCE} +2

\textbf{WISDOM} +1

\textbf{CHARISMA} +3

\textbf{Initiative} +2 -- \textbf{Defense} 25

\textbf{Hit Points} 172 (15d12 + 75)

\textbf{Movement} 12 m, dig 9 m, fly 24 m

\textbf{Saving Throws} Fortitude +19, Reflexes +13, Will +14

\textbf{Skills} Stealth +5, Awareness +11, Deception +8, History +7

\textbf{Damage Immunity} Fire

\textbf{Senses} darkvision 120 ft., blindsight 60 ft

\textbf{Languages} Common, Draconic

\textbf{Challenge} 13 (10000 PX)

€18930 € {€18931 € {Legendary Resistance (3 / Day).}} If the dragon fails a saving throw, it can choose to succeed instead.

\textbf{Shares}

\emph{\textbf{Multiattack.}} The dragon can use its Frightful Presence. Then make three attacks: one with the bite and two with the claws.

\emph{\textbf{Claw.} Melee weapon attack}: +20 to hit, reach 1 m, one target.

\emph{Hits:} 13 (2d6 + 6) slashing damage, 1 bleed damage.

\emph{\textbf{Tail.} Melee weapon attack}: +20 to hit, reach 5 meters, one target.

\emph{Hits:} 15 (2d8 + 6) bludgeoning damage.

\emph{\textbf{Bite.} Melee weapon attack}: +20 to hit, reach 10 ft., one target.

\emph{Hits:} 17 (2d10 + 6) piercing damage.

\emph{\textbf{Frightening Presence.}} Each creature chosen by the dragon, within 120 feet of it and aware of its presence, must succeed on a DC 16 Will save or be frightened for 1 minute. A creature can repeat the saving throw at the end of each of its rounds, ending the effect on a success. If the creature's saving throw succeeds or the effect ends for it, the creature is immune to the dragon's Frightening Presence for the next 24 hours.

\emph{\textbf{Breath Weapon (Recharge 5-6).}} The dragon uses one of the following breath weapons:

\emph{Fiery Breath.} The dragon exhales fire in a line 18 meters long and 1 meter wide. Each creature in the line must make a DC 18 Reflex saving throw, taking 45 (13d6) fire damage on a failed save, or half as much damage on a successful one.

\emph{Sleep Breath.} The dragon exhales sleep gas in a 60-foot cone. Each creature in that area must succeed on a Fortitude saving throw 18 or fall unconscious for 10 minutes. This effect ends if the unconscious creature takes damage or someone takes an action to awaken it.

\textbf{Additional Shares}

The dragon can perform 3 additional Actions, chosen from the following options. He can only use one Additional option at a time, and only at the end of another creature's round. The dragon recovers additional Actions spent at the start of its round.

\textbf{Wing Attack (Costs 2 Actions).} The dragon flaps its wings. Each creature within 10 feet of the dragon must succeed on a DC 19 Reflex save or take 13 (2d6 + 6) bludgeoning damage and be knocked prone. The dragon can then fly up to half its flight movement.

\textbf{Tail Attack.} The dragon makes a tail attack.

\textbf{Spot.} The dragon makes a Wisdom (Awareness) check.

\emph{\textbf{Angry:}} The Adult Brass Dragon can perform these actions for 2 Actions.

\emph{Focus}: The creature interrupts an ongoing mental effect on itself

\emph{Brutality}: the creature attacks with unprecedented ferocity. +1d6 to attack roll, an extra 6 is added to the critical count until the end of the fight.


\textbf{Ecology}\\
Environment: Hot Deserts\\
Organization: Solitaire\\
\textbf{Treasure}: Triple\\
\textbf{Description}\\
Excellent conversationalists, brass dragons prefer to talk rather than fight. Brass dragons lair near humanoid settlements, where they can hear the latest news and gossip.\\
\textbf{Spells}\index{Brass Dragon Spells}\\
This Dragon's favorite spells are:\\
- Vision of the truth\\
- Knowledge of Legends\\
- Scrutinize

\medskip\index[Monstery]{Young Brass Dragon}\textbf{Young Brass Dragon}

\emph{Great Dragon, Chaotic Good}

\textbf{STRENGTH} +4

\textbf{DEXTERITY} +0

\textbf{CONSTITUTION} +3

\textbf{INTELLIGENCE} +1

\textbf{WISDOM} +0

\textbf{CHARISMA} +2

\textbf{Initiative} +1 -- \textbf{Defense} 20

\textbf{Hit Points} 110 (13d10 + 39)

\textbf{Movement} 12 m, dig 6 m, fly 24 m

\textbf{Saving Throws} Fortitude +9, Reflexes +8, Will +7

\textbf{Skills} Stealth +3, Awareness +6, Deception +5

\textbf{Damage Immunity} Fire

\textbf{Senses} darkvision 120 ft., blindsight 30 ft

\textbf{Languages} Common, Draconic

\textbf{Challenge} 6 (2300 PX)

\textbf{Shares}

\emph{\textbf{Multiattack.}} The dragon can make three attacks: one with its bite and two with its claws.

\emph{\textbf{Claw.} Melee weapon attack}: +7 to hit, reach 1 m, one target.

\emph{Hits:} 11 (2d6 + 4) slashing damage, 1 bleed damage.

\emph{\textbf{Bite.} Melee weapon attack}: +7 to hit, reach 10 ft., one target.

\emph{Hits:} 15 (2d10 + 4) piercing damage.

\emph{\textbf{Breath Weapon (Recharge 5-6).}} The dragon uses one of the following breath weapons:

\emph{Fiery Breath.} The dragon exhales fire in a line 12 meters long and 1 meter wide. Each creature in the line must make a DC 14 Reflex saving throw, taking 42 (12d6) fire damage on a failed save, or half as much damage on a successful one. \emph{Sleep Breath.} The dragon exhales sleep gas in a 30-foot cone. Each creature in that area must succeed on a Fortitude saving throw 14 or fall unconscious for 5 minutes. This effect ends if the unconscious creature takes damage or someone takes an action to awaken it.

\textbf{Ecology}\\
Environment: Hot Deserts\\
Organization: Solitaire\\
\textbf{Treasure}: Triple\\
\textbf{Description}\\
Excellent conversationalists, brass dragons prefer to talk rather than fight. Brass dragons lair near humanoid settlements, where they can hear the latest news and gossip.\\
\textbf{Spells}\index{Brass Dragon Spells}\\
This Dragon's favorite spells are:\\
- Vision of the truth\\
- Knowledge of Legends\\
- Scrutinize

\medskip\index[Monstery]{Baby Brass Dragon}\textbf{Baby Brass Dragon}

\emph{Average dragon, chaotic good}

\textbf{STRENGTH} +2

\textbf{DEXTERITY} +0

\textbf{CONSTITUTION} +1

\textbf{INTELLIGENCE} +0

\textbf{WISDOM} +0

\textbf{CHARISMA} +1

\textbf{Initiative} +0 -- \textbf{Defense} 17

\textbf{Hit Points} 16 (3d8 + 3)

\textbf{Movement} 9 m, dig 5 meters, fly 18 m

\textbf{Saving Throws} Fortitude +2, Reflexes +0, Will +1

\textbf{Skills} Stealth +2, Awareness +4

\textbf{Damage Immunity} Fire

\textbf{Senses} Darkvision 60 ft., blindsight 10 ft

\textbf{Languages} Draconic

\textbf{Challenge} 1 (200 PX)

\textbf{Shares}

\emph{\textbf{Bite.} Melee weapon attack}: +4 to hit, reach 1 m, one target.

\emph{Hits:} 7 (1d10 + 2) piercing damage.

\emph{\textbf{Breath Weapon (Recharge 5-6).}} The dragon uses one of the following breath weapons:

\emph{Fiery Breath.} The dragon exhales fire in a line 6 meters long and 1 meter wide. Each creature in the line must make a DC 11 Reflex save, taking 14 (4d6) fire damage on a failed save, or half as much damage on a successful one.

\emph{Sleep Breath.} The dragon exhales sleep gas in a 5 meter cone. Each creature in that area must succeed on a Fortitude save 11 or fall unconscious for 1 minute. This effect ends if the unconscious creature takes damage or someone takes an action to awaken it.

\textbf{Ecology}\\
Environment: Hot Deserts\\
Organization: Solitaire\\
\textbf{Treasure}: Triple\\
\textbf{Description}\\
Excellent conversationalists, brass dragons prefer to talk rather than fight. Brass dragons lair near humanoid settlements, where they can hear the latest news and gossip.


\medskip\index[Monstery]{Ancient Copper Dragon}\textbf{Ancient Copper Dragon}

\emph{Large dragon, chaotic good}

\textbf{STRENGTH} +8

\textbf{DEXTERITY} +1

\textbf{CONSTITUTION} +7

\textbf{INTELLIGENCE} +5

\textbf{WISDOM} +3

\textbf{CHARRISMA} +4

\textbf{Initiative} +5 -- \textbf{Defense} 33

\textbf{Hit Points} 350 (20x3d6 + 140)

\textbf{Movement} 12 m, climb 12 m, fly 24 m

\textbf{Saving Throws} Fortitude +28, Reflexes +24, Will +24

\textbf{Skills} Stealth +8, Deception +11, Awareness +17

\textbf{Damage Immunity} acid, weapons +1

\textbf{Senses} darkvision 120 ft., blindsight 60 ft

\textbf{Languages} Common, Draconic

\textbf{Challenge} 21 (33000 PX)

€19052 € {€19053 € {Legendary Resistance (3 / Day).}} If the dragon fails a saving throw, it can choose to succeed instead.

\emph{\textbf{Magic Resistance:}} 3lv

\textbf{Shares}

\emph{\textbf{Multiattack.}} The dragon can use its Frightening Presence. Then make three attacks: one with the bite and two with the claws.

\emph{\textbf{Claw.} Melee weapon attack}: +30 to hit, reach 10 ft., one target.

\emph{Hits:} 15 (2d6 + 8) slashing damage, 3/20 bleed damage.

\emph{\textbf{Tail.} Melee weapon attack}: +30 to hit, reach 6 ft., one target.

\emph{Hits:} 17 (2d8 + 8) bludgeoning damage.

\emph{\textbf{Bite.} Melee weapon attack}: +30 to hit, reach 5 metres, one target.

\emph{Hits:} 19 (2d10 + 8) piercing damage.

\emph{\textbf{Frightening Presence.}} Each creature chosen by the dragon, within 120 feet of it and aware of its presence, must succeed on a DC 19 Will save or be frightened for 1 minute. A creature can repeat the saving throw at the end of each of its rounds, ending the effect on a success. If the creature's saving throw succeeds or the effect ends for it, the creature is immune to the dragon's Frightening Presence for the next 24 hours.

\emph{\textbf{Breath Weapon (Recharge 5-6).}} The dragon uses one of the following breath weapons:

\emph{Acid Breath.} The dragon exhales acid in a line 27 meters long and 3 meters wide. Each creature in the line must make a DC 22 Reflex saving throw, taking 63 (14d8) acid damage on a failed save, or half as much damage on a successful one.

\emph{Slowing Breath.} The dragon exhales gas in a 27 meter cone. Each creature in that area must succeed on a DC 22 Fortitude save. If the save fails, the creature has one less action per round and has its speed halved. These effects last 1 minute. The creature can repeat the saving throw at the end of each of its rounds, ending the effect on itself on a success.

\emph{\textbf{Shapeshifting.}} The dragon can magically transform into a humanoid or beast whose challenge rating is equal to or lower than its own, or return to its true form. Upon death he returns to his true form. Whatever equipment it is wearing or carrying is absorbed or transported into the new form (the dragon's choice).

In its new form, the dragon retains its Traits, Hit Points, Hit Dice, speech, proficiencies, Legendary Stamina, lair actions, and Intelligence, Wisdom, and Charisma scores, in addition to this action. His stats and abilities

they are otherwise replaced by those of the new form, except Additional Actions of the new form.

\textbf{Additional Shares}

The dragon can perform 3 additional Actions, chosen from the following options. He can only use one Additional option at a time, and only at the end of another creature's round. The dragon recovers additional Actions spent at the start of its round.

\textbf{Wing Attack (Costs 2 Actions).} The dragon flaps its wings. Each creature within 5 feet of the dragon must succeed on a DC 23 Reflex save or take 15 (2d6 + 8) bludgeoning damage and be knocked prone. The dragon can then fly up to half its flight movement.

\textbf{Tail Attack.} The dragon makes a tail attack.

\textbf{Spot.} The dragon makes a Wisdom (Awareness) check.

\textbf{Ecology}\\
Environment: Warm Hills\\
Organization: Solitaire\\
\textbf{Treasure}: Triple\\
\textbf{Description}\\
This temperamental dragon tries to hinder and frustrate its enemies during combat.\\
\textbf{Spells}\index{Copper Dragon Spells}\\
This Dragon's favorite spells are:\\
- Blade Barrier\\
- Wall of Force\\
- Elastic sphere


\medskip\index[Monstery]{Adult Copper Dragon}\textbf{Adult Copper Dragon}

\emph{Huge dragon, chaotic good}

\textbf{STRENGTH} +6

\textbf{DEXTERITY} +1

\textbf{CONSTITUTION} +5

\textbf{INTELLIGENCE} +4

\textbf{WISDOM} +2

\textbf{CHARISMA} +3

\textbf{Initiative} +4 -- \textbf{Defense} 25

\textbf{Hit Points} 184 (16d12 + 80)

\textbf{Movement} 12 m, climb 12 m, fly 24 m

\textbf{Saving Throws} Fortitude +19, Reflexes +15, Will +15

\textbf{Skills} Stealth +6, Deception +8, Awareness +12

\textbf{Damage Immunity} acid

\textbf{Senses} darkvision 120 ft., blindsight 60 ft

\textbf{Languages} Common, Draconic

\textbf{Challenge} 14 (11,500 PX)

€19105 €{€19106 €{Legendary Resistance (3 / Day).}} If the dragon fails a saving throw, it can choose to succeed instead.

\textbf{Shares}

\emph{\textbf{Multiattack.}} The dragon can use its Frightening Presence. Then make three attacks: one with the bite and two with the claws.

\emph{\textbf{Claw.} Melee weapon attack}: +22 to hit, reach 1 m, one target.

\emph{Hits:} 13 (2d6 + 6) slashing damage, 1 bleed damage.

\emph{\textbf{Tail.} Melee weapon attack}: +22 to hit, reach 5 meters, one target.

\emph{Hits:} 15 (2d8 + 6) bludgeoning damage.

\emph{\textbf{Bite.} Melee weapon attack}: +22 to hit, reach 10 ft., one target.

\emph{Hits:} 17 (2d10 + 6) piercing damage.

\emph{\textbf{Frightening Presence.}} Each creature chosen by the dragon, within 120 feet of it and aware of its presence, must succeed on a DC 16 Will save or be frightened for 1 minute. A creature can repeat the saving throw at the end of each of its rounds, ending the effect on a success. If the creature's saving throw succeeds or the effect ends for it, the creature is immune to the dragon's Frightening Presence for the next 24 hours.

\emph{\textbf{Breath Weapon (Recharge 5-6).}} The dragon uses one of the following breath weapons:

\emph{Acid Breath.} The dragon exhales acid in a line 18 meters long and 1 meter wide. Each creature in the line must make a DC 18 Reflex saving throw, taking 54 (12d8) acid damage on a failed save, or half as much damage on a successful one.

\emph{Slowing Breath.} The dragon exhales gas in a 27 meter cone. Each creature in that area must succeed on a DC 18 Fortitude save. If the save fails, the creature has one less action per round and has its speed halved. These effects last 1 minute. The creature can repeat the saving throw at the end of each of its rounds, ending the effect on itself on a success.

\textbf{Additional Shares}

The dragon can perform 3 additional Actions, chosen from the following options. He can only use one Additional option at a time, and only at the end of another creature's round. The dragon recovers additional Actions spent at the start of its round.

\textbf{Wing Attack (Costs 2 Actions).} The dragon flaps its wings. Each creature within 10 feet of the dragon must succeed on a DC 19 Reflex save or take 13 (2d6 + 6) bludgeoning damage and be knocked prone. The dragon can then fly up to half its flight movement.

\textbf{Tail Attack.} The dragon makes a tail attack.

\textbf{Spot.} The dragon makes a Wisdom (Awareness) check.

\emph{\textbf{Angry:}} The Adult Copper Dragon can perform these actions for 2 Actions.

\emph{Focus}: The creature interrupts an ongoing mental effect on itself

\emph{Brutality}: the creature attacks with unprecedented ferocity. +1d6 to attack roll, an extra 6 is added to the critical count until the end of the fight.


\textbf{Ecology}\\
Environment: Warm Hills\\
Organization: Solitaire\\
\textbf{Treasure}: Triple\\
\textbf{Description}\\
This temperamental dragon tries to hinder and frustrate its enemies during combat.\\
\textbf{Spells}\index{Copper Dragon Spells}\\
This Dragon's favorite spells are:\\
- Blade Barrier\\
- Wall of Force\\
- Elastic sphere



\medskip\index[Monstery]{Young Copper Dragon}\textbf{Young Copper Dragon}

\emph{Great Dragon, Chaotic Good}

\textbf{STRENGTH} +4

\textbf{DEXTERITY} +1

\textbf{CONSTITUTION} +3

\textbf{INTELLIGENCE} +3

\textbf{WISDOM} +1

\textbf{CHARISMA} +2

\textbf{Initiative} +3 -- \textbf{Defense} 21

\textbf{Hit Points} 119 (14d10 + 42)

\textbf{Movement} 12 m, climb 12 m, fly 24 m

\textbf{Saving Throws} Fortitude +9, Reflex +8, Will +8

\textbf{Skills} Stealth +4, Deception +5, Awareness +7

\textbf{Damage Immunity} acid

\textbf{Senses} darkvision 120 ft., blindsight 30 ft

\textbf{Languages} Common, Draconic

\textbf{Challenge} 7 (2900 PX)

\textbf{Shares}

\emph{\textbf{Multiattack.}} The dragon can make three attacks: one with its bite and two with its claws.

\emph{\textbf{Claw.} Melee weapon attack}: +10 to hit, reach 1 m, one target.

\emph{Hits:} 11 (2d6 + 4) slashing damage, 1 bleed damage.

\emph{\textbf{Bite.} Melee weapon attack}: +10 to hit, reach 10 ft., one target.

\emph{Hits:} 15 (2d10 + 4) piercing damage.

\emph{\textbf{Breath Weapon (Recharge 5-6).}} The dragon uses one of the following breath weapons:

\emph{Acid Breath.} The dragon exhales acid in a line 12 meters long and 1 meter wide. Each creature in the line must make a DC 14 Reflex saving throw, taking 40 (9d8) acid damage on a failed save, or half as much damage on a successful one.

\emph{Slowing Breath.} The dragon exhales gas in a 27 meter cone. Each creature in that area must succeed on a DC 14 Fortitude save. If the save fails, the creature has one less action per round and has its speed halved. These effects last 1 minute. The creature can repeat the saving throw at the end of each of its rounds, ending the effect on itself on a success.

\emph{\textbf{Angry:}} the Young Copper Dragon recharges one of its two breath weapons. Cost 1 Action.

\textbf{Ecology}\\
Environment: Warm Hills\\
Organization: Solitaire\\
\textbf{Treasure}: Triple\\
\textbf{Description}\\
This temperamental dragon tries to hinder and frustrate its enemies during combat.\\
\textbf{Spells}\index{Copper Dragon Spells}\\
This Dragon's favorite spells are:\\
- Blade Barrier\\
- Wall of Force\\
- Elastic sphere

\medskip\textbf{Copper Dragon Hatchling}

\emph{Average dragon, chaotic good}

\textbf{STRENGTH} +2

\textbf{DEXTERITY} +1

\textbf{CONSTITUTION} +1

\textbf{INTELLIGENCE} +2

\textbf{WISDOM} +0

\textbf{CHARISMA} +1

\textbf{Initiative} +2 -- \textbf{Defense} 17

\textbf{Hit Points} 22 (4d8 + 4)

\textbf{Movement} 9m, climb 9m, fly 18m

\textbf{Saving Throws} Fortitude +2, Reflex +2, Will +0

\textbf{Skills} Stealth +3, Awareness +4

\textbf{Damage Immunity} acid

\textbf{Senses} Darkvision 60 ft., blindsight 10 ft

\textbf{Languages} Draconic

\textbf{Challenge} 1 (200 PX)

\textbf{Shares}

\emph{\textbf{Bite.} Melee weapon attack}: +4 to hit, reach 1 m, one target.

\emph{Hits:} 7 (1d10 + 2) piercing damage.

\emph{\textbf{Breath Weapon (Recharge 5-6).}} The dragon uses one of the following breath weapons:

\emph{Acid Breath.} The dragon exhales acid in a line 6 meters long and 1 meter wide. Each creature in the line must make a DC 11 Reflex save, taking 18 (4d8) acid damage on a failed save, or half as much damage on a successful one.

\emph{Slowing Breath.} The dragon exhales gas in a 27 meter cone. Each creature in that area must succeed on a DC 11 Fortitude save. If the save fails, the creature has one less action per round and has its speed halved. These effects last 1 minute. The creature can repeat the saving throw at the end of each of its rounds, ending the effect on itself on a success.

\textbf{Ecology}\\
Environment: Warm Hills\\
Organization: Solitaire\\
\textbf{Treasure}: Triple\\
\textbf{Description}\\
This temperamental dragon tries to hinder and frustrate its enemies during combat.

\medskip\index[Monstruary]{Tàhil}\textbf{Tàhil}

\emph{Colossal dragon}

\textbf{STRENGTH} +10

\textbf{DEXTERITY} +0

\textbf{CONSTITUTION} +10

\textbf{INTELLIGENCE} +8

\textbf{WISDOM} +8

\textbf{CHARISMA} +9

\textbf{Initiative} +8 -- \textbf{Defense} 40

\textbf{Hit Points} 615 (30x3d6+300)

\textbf{Move} 20 meters, fly 20 meters

\textbf{Saving Throws}: Fortitude +40, Reflexes +30, Will +38

\textbf{Skills} all +18

\textbf{Damage Immunity} cold, lightning, fire, acid, poison, sound, weapons +3

\textbf{Condition Immunity} charmed, poisoned, paralyzed, fatigued, frightened

\textbf{Senses} Darkvision 60 m, True seeing 40 m

\textbf{Languages} all

\textbf{Challenge} 30 (155,000 PX)

\textbf{Immortal on Yeru.} When Tàhil's body is killed on Yeru it reforms in 3d6 days in the lair made by Calicante.

\emph{\textbf{Spells.}} Tàhil has CM 20. His spellcasting ability is Charisma. Tàhil knows the following spells:

At will: Divine Word

\emph{\textbf{Divine Nature.}} Tàhil has no need for air, food, drink or sleep. Spells of 5th level or lower have no effect on Tàhil unless he wishes them to.

\emph{\textbf{Master of the Dragons.}} Every non-Ljust Dragon on Yeru is faithful and obedient to Tàhil's will.

\emph{\textbf{Voice of the Master.}} Tàhil can converse with every Tàhil dragon in Yeru, regardless of distance.

\emph{\textbf{Call of the Master.}} Tàhil opens a portal and 1d2+1 Tàhil dragons of random age and color come out. The power is usable 1 time per day.

€19240 € {€19241 € {Legendary Resistance (5 / Day).}} If the Tàhil fails a saving throw, he can choose to succeed instead.

\emph{\textbf{More heads.}} Tàhil has +1d6 on saving throws against being blind, deaf, or unconscious. Tàhil can perform up to 6 Reactions per round.

\emph{\textbf{Regeneration.}} Tàhil regenerates 30 Hit Points at the start of his round

\textbf{Shares}

\emph{\textbf{Multiattack.}} Tàhil can use his Frightening Presence or make 3 attacks (2 with claws and one with his tail) or just one with his bite. Claw +30, reach 5 meters. Tail +30 range 8 meters. Bite +30, reach 6 meters. All of Tàhil's attacks are considered +5 magic.

\emph{Hits:} Claw, 24 (4d6 +10, 5/40 bleed damage) slashing. Tail, 28 (4d8 +10) bludgeoning. Bite 48 (8d6 +10) on a critical hit, the bite cuts off the body in half if the Fortitude save is failed at DC 30.

\emph{\textbf{Frightening Presence}} Any creature that can see Tàhil and is within 80 meters must make a Will save at DC 26 or be frightened for 1 minute. Each round the creature can make the saving throw, if it succeeds it is immune to Tàhil's Frightening Presence for the next 24 hours.

\textbf{Additional Shares}

The Tàhil can take 3 additional actions, chosen from those below and one per round only at the end of another creature's round. Tàhil can change the color of his head to access the powers of other dragon types. The actions depend on the chosen head.

\textbf{Claw attack.}: +19, reach 6 meters, one target. If it hits 32 (4d10 + 10, 3 from Bleed) slashing damage plus 14 (4d6) acid damage (Black head) or Electricity (Blue head) or Poison (Green head) or Fire (Red head) or Cold damage (White head) or by Fire (Yellow head) or by Sound (Purple head)

\textbf{Black Head.}: Costs 2 Additional actions, Tàhil blows Acid into a 40 meter cone. Reflex save DC 27 or take 68 (15d8) acid damage or halve.

\textbf{Blue Head.}: Costs 2 Additional actions, Tàhil blows Electricity into a 40 meter cone. Reflex save DC 27 or take 88 (16d10) Electricity damage or halve.

\textbf{Green Head.}: Costs 2 Additional actions, Tàhil breathes Poison into a 30 meter cone. Reflex save DC 27 or take 77 (22d6) poison damage or halve.

\textbf{Red Head.}: Costs 2 Additional actions, Tàhil breathes Fire into a 30 meter cone. Reflex save DC 27 or take 91 (26d6) Fire damage or halve.

\textbf{White Head.}: Costs 2 Additional actions, Tàhil blows Ice into a 30 meter cone. Reflex save DC 27 or take 72 (16d8) ice damage or halve.

\textbf{Purple Head.}: Costs 2 Additional actions, Tàhil blows Sound into a 30 meter cone. Reflex save DC 27 or take 90 (18d8) Sound damage or halve.

\textbf{Yellow Head.}: Costs 2 Additional actions, Tàhil blows hot sand into a 60 meter cone. Reflex save DC 27 or take 72 (16d8) Fire damage or halve.


\textbf{Ecology}\\
Environment: Unknown\\
Organization: Unique\\
\textbf{Treasure}: Special\\

\textbf{Description}
Tàhil is the Patron of Dragons incarnate. Nothing resists his fury, madness, anger and destruction. Tàhil is a mammoth creature with 7 dragon heads, each colored differently, each representing a color of a Dragon. See chapter on Cosmology for details of his story.


\medskip\index[Monstery]{Drider}\textbf{Drider}

\emph{Great monstrosity, chaotic evil}

\textbf{STRENGTH} +3

\textbf{DEXTERITY} +3

\textbf{CONSTITUTION} +4

\textbf{INTELLIGENCE} +1

\textbf{WISDOM} +2

\textbf{CHARISMA} +1

\textbf{Initiative} +3 -- \textbf{Defense} 22

\textbf{Hit Points} 123 (13d10 + 52)

\textbf{Movement} 9m, climb 9m

\textbf{Saving Throws} Fortitude +11, Reflexes +9, Will +9

\textbf{Skills} Stealth +9, Awareness +5

\textbf{Senses} darkvision 36 m

\textbf{Languages} Elvish, Language of the Depths

\textbf{Challenge} 6 (2300 PX)

\emph{\textbf{Walking on the Web.}} The drider ignores movement restrictions caused by the webs.

\emph{\textbf{Faerie Lineage.}} The drider has +1d6 on saving throws to avoid being charmed, and magic cannot put a drider to sleep.

\emph{\textbf{Innate Spells.}} The drider's innate spellcasting ability is Wisdom. The drider can innately cast the following spells, without the need for material components:

All you want: \emph{dancing lights}

1/Day: \emph{luminescence, darkness}

\emph{\textbf{Climb as a Spider.}} The drider can climb difficult surfaces, including standing upside down on the ceiling, without needing to make an ability check.

\textbf{Shares}

\emph{\textbf{Multiattack.}} The drider makes three attacks with its longsword or longbow. He can replace one of these attacks with a bite attack.

\emph{\textbf{Bite.} Melee weapon attack}: +11 to hit, reach 3 ft., one creature.

\emph{Hits:} 2 (1d4) piercing damage plus 9 (2d8) poison damage.

\emph{\textbf{Long Sword.} Melee weapon attack}: +1 to hit, reach 1 m, one target.

\emph{Hits:} 7 (1d8 + 3) slashing damage, or 8 (1d8 + 3) slashing damage if used with two hands.

\emph{\textbf{Longbow.} Ranged weapon attack}: +11 to hit, range 45m, one target.

\emph{Hits:} 7 (1d8 + 3) piercing damage plus 4 (1d8) poison damage.

\emph{\textbf{Angry:}} the Drider collects poisonous saliva and spits it onto his weapons. Until the end of the fight, Longsword's attack causes 1d8 poison damage. Costs 1 Action.

\textbf{Ecology}\\
Environment: Any dungeon\\
Organization: Solo, couple or group (3-8)\\
\textbf{Treasure}: double (Masterwork Heavy Mace, Masterwork Composite Longbow [Strength +2] with 20 Arrows, more treasure)\\
\textbf{Description}\\
Created from the body of an elf, altered and mutated through special poisons and elixirs to take on the characteristics of a giant spider, the drider is a dangerous creature.\\
Driders are sexually dimorphic. The spidery lower body of a female drider is sleek and graceful, often resembling the body of a black widow, while the elfin upper torso retains her alluring curves and beautiful face (with the exception of the venomous, razor-sharp fangs). . A male drider's lower body is as stocky as a tarantula, while his upper body has a lean physique and supports a hideous, more spider-like than elf-like face, complete with fanged mandibles.


\medskip\index[Monsterary]{Dryad}\textbf{Dryad}

\emph{Medium fairy, neutral}

\textbf{STRENGTH} +0

\textbf{DEXTERITY} +1

\textbf{CONSTITUTION} +0

\textbf{INTELLIGENCE} +2

\textbf{WISDOM} +2

\textbf{CHARRISMA} +4

\textbf{Initiative} +2 -- \textbf{Defense} 12 (17 with \emph{bark skin})

\textbf{Hit Points} 22 (5d8)

\textbf{Damage Vulnerability} cold iron

\textbf{Movement} 9 m

\textbf{Saving Throws} Fortitude +5, Reflexes +9, Will +7

\textbf{Skills} Stealth +5, Awareness +4

\textbf{Senses} Darkvision 18 m

\textbf{Languages} Elvish, Sylvan

\textbf{Challenge} 1 (200 PX)

\emph{\textbf{Arboreal Walk.}} Once during her round, the dryad can use 10 feet of movement to magically enter a living tree within her reach and emerge from another living tree within 60 feet of the first tree, reappearing in an unoccupied space within 1 meter of the second tree. Both trees must be Large or larger.

\emph{\textbf{Innate Spells.}} The dryad's innate spellcasting ability is Charisma (DC 14 for spell saving throws). The dryad can innately cast the following spells, without requiring material components. At will:

\emph{druidic artifice}

3/day each: \emph{beneficial berries}, \emph{hinder} 1/day:
\emph{pass without traces, leathery skin, club} \emph{enchanted}

\emph{\textbf{Speak with Animals and Plants.}} The dryad can communicate with beasts and plants as if they spoke the same language.

\emph{\textbf{Magic Resistance.}} The dryad has +1d6 on saving throws against spells and other magical effects.

\textbf{Shares}

\emph{\textbf{Club.} Melee weapon attack}: +2 to hit (+6 to hit with €19347{staff}), reach 1 m, one target.

\emph{Hit:} 2 (1d4) bludgeoning damage, or 8 (1d8 + 4) bludgeoning damage with \emph{staff}

\emph{\textbf{Faerie Charm.}} The dryad can target one humanoid or beast within 30 feet of her that she can see. If the target can see the dryad, he must succeed on a DC 14 Will save or become charmed by the magic. Fascinated creatures consider the dryad a trusted friend to listen to and protect. Although the target is not under the dryad's control, she will interpret the dryad's requests or actions as favorably as possible.

Each time the dryad or her allies deal damage to the target, it can repeat the saving throw, ending the effect on a success. Otherwise, the effect lasts 24 hours or until the dryad dies, is on a different plane of existence than the target, or ends the effect with a bonus action. If the target's saving throw succeeds, the target is immune to the dryad's fey charm for the next 24 hours.

The dryad can charm no more than one humanoid or three beasts at a time.

\textbf{Ecology}\\
Environment: Temperate Forests\\
Organization: Solitary, pair or grove (3-8)\\
\textbf{Treasure}: standard (Masterwork Longbow with 20 Arrows, Dagger, other treasure)\\
\textbf{Description}\\
Dryads are tree sprites who love secluded woods far from humanoids in need of lumber. Dryads' primary concern is their own survival and that of their beloved forests, and they have been known to magically compel travelers to help them with tasks they cannot perform. They are friendly with non-evil druids and rangers, as they recognize their empathy or respect for nature.\\
Dryads are benevolent guardians of trees, and while not violent by nature, they can block and thwart threats to their homes or turn enemies into allies. Some keep one or more charmed humanoids in their territory to defend it or to divert attackers. Disabled enemies are typically dragged to the edge of the forest by the dryads' allies and driven away, but evil or hostile ones are slain once the fight is over.

\medskip\index[Monstery]{Duergar}\textbf{Duergar}

\emph{Medium humanoid (dwarf), lawful evil}

\textbf{STRENGTH} +2

\textbf{DEXTERITY} +0

\textbf{CONSTITUTION} +2

\textbf{INTELLIGENCE} +0

\textbf{WISDOM} +0

\textbf{CHARRISMA} -1

\textbf{Initiative} +2 -- \textbf{Defense} 17 (scale armour, shield)

\textbf{Hit Points} 26 (4d8 + 8)

\textbf{Movement} 8 m

\textbf{Saving Throws} Fortitude +4, Reflexes +0, Will +1

\textbf{Damage Resistance} poison

\textbf{Senses} darkvision 36 m

\textbf{Languages} Dwarven, Language of the Depths

\textbf{Challenge} 1 (200 PX)

\emph{\textbf{Duerga Resilience.}} The duergar has +1d6 on saving throws against poisons, spells, and illusions, as well as resisting being charmed or paralyzed.

\emph{\textbf{Light Sensitivity}}. While in sunlight, the duergar has -1d6 on attack rolls, as well as on sight-based Wisdom (Awareness) checks.

\textbf{Shares}

€19379 € {€19380 € {Enlarge (Recharge after 1 hour).}} For 1 minute, the duergar magically increases in size, along with anything he is carrying or wearing. While enlarged, the duergar is Large in size, doubles the damage dice of Strength-based weapon attacks (already included in attacks), and has +1d6 on Strength checks and Strength saving throws. If the duergar does not have enough space to become Large, it gains the maximum size allowed by the available space.

\emph{\textbf{War Pickaxe.} Melee Weapon Attack}: +4 to hit, reach 1 m, one target.

\emph{Hits:} 6 (1d8 + 2) piercing damage, or 11 (2d8 + 2) piercing damage when enlarged.

\emph{\textbf{Javelin.} Melee or ranged weapon attack}: +4 to hit, reach 1m or range 12m, one target. \emph{Hit:} 5 (1d6 + 2) piercing damage or 9 (2d6 + 2) damage
perforating when magnified.

\emph{\textbf{Invisibility (Recharge after 1 hour).}} The duergar becomes magically invisible for up to 1 hour (as if maintaining concentration for a spell) or until it attacks, casts a spell, use Enlarge or his concentration is broken. All equipment the duergar is wearing or carrying becomes invisible along with him.

\textbf{Ecology}\\
Environment: Any dungeon\\
Organization: solitary, group (2-5), squad (6-12 plus 3 3rd level sergeants and 1 3rd-8th level leader), or clan (13-80 plus 25\% non-combat children plus 1 3rd level sergeant for every 5 adults, 3-6 3rd-6th level lieutenants, and 1-4 9th level captains)\\
\textbf{Treasure}: NPC equipment (Chainmail, Heavy Metal Shield, Warhammer, Light Crossbow with 20 Bolts, 3d6 gp, more treasure)\\
\textbf{Description}\\
Distant relatives of the Dwarves, darker and more deformed, the Duergar are foul-tempered creatures who hate intruders in their subterranean realms, but never more than the Dwarves. They live in communities deep underground. They have dull gray skin, as if it were dirty with dust or ash, but this natural shade allows them to camouflage better in underground areas. They are a race of slavers, but while they force non-Dwarf captives to backbreaking labor, they kill captured Dwarves without hesitation. In combat, Duergar shoot crossbows, and then switch to the Warhammer a few rounds later. If outnumbered, or if there is adequate danger (and space), a Duergar will use its Enlarge ability and attack.



\subsection{Elementals}
The generic and somewhat mathematical formulas of the elementals are presented first, followed by examples.


\medskip\index[Monstery]{Generic Water Elemental}\textbf{Generic Water Elemental}\\
\emph{CR/3 (Small, Medium, Large, Huge, Gargantuan, Colossal)}\\
\textbf{STRENGTH} +2+CR/3\\
\textbf{DEXTERITY} 0+CR/4\\
\textbf{CONSTITUTION} +2+CR/3\\
\textbf{INTELLIGENCE} -2+CR/6\\
\textbf{WISDOM} +0+CR/5\\
\textbf{CHARISMA} +0+CR/7\\
\textbf{Initiative} =DEX -- \textbf{Defense} 10+CR+DEX\\
\textbf{Hit Points} CR*CR*4\\
\textbf{Movement} 9 m, swimming CR*4 m\\
\textbf{Saving Throws} Fortitude +CR+CON, Reflexes +CR+DEX, Will +CR+WIS\\
\textbf{Damage Resistances} acid; from a non-magical weapon\\
\textbf{Damage Immunity} Poison\\
\textbf{Condition Immunity} grabbed, poisoned, entangled, paralyzed, petrified, unconscious, prone, fatigue\\
\textbf{Senses} Darkvision 18 m\\
\textbf{Languages} Aquan\\
\textbf{Challenge} CR\\
€19414 € {€19415 € {Freezing.}} If the elemental takes cold damage, it partially freezes; his movement is reduced by 20 feet until the end of his next round.
\emph{\textbf{Water Form.}} The elemental can enter a hostile creature's space and stop there. He can move through a space as narrow as 3 centimeters without having to squeeze himself.
\emph{\textbf{Elemental Nature.}} An elemental has no need for air, food, drink, or sleep.\\
\textbf{Shares}\\
\emph{\textbf{Multiattack.}} The elemental makes two slam attacks.\\
\emph{\textbf{Slam.} Melee weapon attack}: +CR+STR to hit, CR/10 feet, one target.\\
\emph{Hits:} CR*1d8 bludgeoning damage.\\
\emph{\textbf{Submerge (Recharge 4-6).}} Each creature in the elemental's space must make a Fortitude saving throw DC CR+CR/2. On a failed save, the target takes (1d8+1)*CR/2 bludgeoning damage. If it is size CR/3 >=4, the target is also grappled (DC CR*2 to escape). Until the grapple ends, the target is restrained and cannot breathe unless it can breathe water. If the saving throw succeeds, the target is pushed out of the elemental's space.\\
The elemental can grab a creature of size CR/3 or 2 of CR/2 or. At the start of each elemental's round, each grappled target takes (1d6)*CR/2 bludgeoning damage. A creature within 10 feet of the elemental can pull a creature or object away from it, spending an action to attempt to succeed on a DC 2+CR*2 Strength check.\\


\medskip\index[Monstery]{Generic Air Elemental}\textbf{Generic Air Elemental}

\textbf{STRENGTH} +0+CR/4\\
\textbf{DEXTERITY} +3+CR/4\\
\textbf{CONSTITUTION} +0+CR/5\\
\textbf{INTELLIGENCE} -2+CR/6\\
\textbf{WISDOM} -1+CR/5\\
\textbf{CHARISMA} +0+CR/5\\
\textbf{Initiative} =DEX -- \textbf{Defense} 10+CR+DEX\\
\textbf{Hit Points} CR*CR*2\\
\textbf{Move} 0 m, fly CR*4 m\\
\textbf{Saving Throws} Fortitude +CR+CON, Reflexes +CR+DEX, Will +CR+WIS\\
\textbf{Damage Resistances} lightning, sound; from a non-magical weapon\\
\textbf{Damage Immunity} Poison\\
\textbf{Condition Immunity} grabbed, poisoned, entangled, paralyzed, petrified, unconscious, prone, fatigue\\
\textbf{Senses} Darkvision 18 m\\
\textbf{Languages} Ictun\\
\textbf{Challenge} CR\\
\emph{\textbf{Air Form.}} The elemental can enter a hostile creature's space and stop there. He can move through a space as narrow as 3 centimeters without having to squeeze himself.
\emph{\textbf{Elemental Nature.}} An elemental has no need for air, food, drink, or sleep.\\
\textbf{Shares}\\
\emph{\textbf{Multiattack.}} The elemental makes two slam attacks.\\
\emph{\textbf{Slam.} Melee weapon attack}: +CR+STR to hit, CR/10 feet, one target.\\
\emph{Hits:} 1d6*CR/3 bludgeoning damage.\\
\emph{\textbf{Whirlwind (Recharge 4-6).}} Each creature in the elemental's space must make a Fortitude saving throw DC CR*1.5. On a failed save, the target takes 1d8*CR/3 bludgeoning damage and is knocked CR meters away from the elemental in a random direction and falls prone. If a thrown target hits an object, such as a wall or the floor, it takes 3 (1d6) bludgeoning damage for every 10 feet it was thrown. If the target is thrown at another creature, that creature must succeed on a DC 13 Reflex saving throw or take the same damage and be knocked prone.
On a successful save, the target takes half bludgeoning damage and is not knocked away or knocked prone.


\medskip\index[Monstrury]{Generic Fire Elemental}\textbf{Generic Fire Elemental}

\textbf{STRENGTH} +0+CR/4\\
\textbf{DEXTERITY} +2+CR/4\\
\textbf{CONSTITUTION} +1+CR/4\\
\textbf{INTELLIGENCE} -2+CR/6\\
\textbf{WISDOM} -1+CR/5\\
\textbf{CHARISMA} -2+CR/5\\
\textbf{Initiative} =DEX -- \textbf{Defense} 10+CR+DEX\\
\textbf{Hit Points} CR*CR*3\\
\textbf{Movement} 15 m\\
\textbf{Saving Throws} Fortitude +CR+CON, Reflexes +CR+DEX, Will +CR+WIS\\
\textbf{Damage Resistances} from non-magical weapons\\
\textbf{Damage Immunity} Fire, Poison\\
\textbf{Condition Immunity} grabbed, poisoned, entangled, paralyzed, petrified, unconscious, prone, fatigue\\
\textbf{Senses} Darkvision 18 m\\
\textbf{Languages} Ignan\\
\textbf{Challenge} CR\\
\emph{\textbf{Fire Form.}} The elemental can move through a space up to 3 centimeters wide without squeezing. A creature that contacts or hits the elemental with a melee attack while within 3 feet of it takes 5 (1d10) fire damage. Additionally, the elemental can enter a hostile creature's space and stop there. The first time it enters a creature's space in a round, the creature takes CR fire damage and catches fire; until someone takes an action to put out the flames, the creature will take CR fire damage at the start of each of its rounds.\\
\emph{\textbf{Lighting.}} The elemental emits bright light in a radius of CR*2 meters and dim light for additional CR*2 meters.\\
\emph{\textbf{Elemental Nature.}} An elemental has no need for air, food, drink, or sleep.\\
\emph{\textbf{Susceptibility to Water.}} The elemental takes 1 cold damage for every 1 meter it moves in water or for every 4 liters of water splashed on it.
\textbf{Shares}\\
\emph{\textbf{Multiattack.}} The elemental makes two touch attacks.\\
\emph{\textbf{Slam.} Melee weapon attack}: +CR+STR to hit, CR/10 feet, one target.\\
\emph{Hits:} CR*2 fire damage. If the target is a flammable creature or object, it catches fire. Until a creature takes an action to extinguish the flames, the creature takes CR fire damage at the start of each of its rounds.


\medskip\index[Monstrury]{Generic Earth Elemental}\textbf{Generic Earth Elemental}

\textbf{STRENGTH} +CR\\
\textbf{DEXTERITY} -2+CR/5\\
\textbf{CONSTITUTION} +2+CR/3\\
\textbf{INTELLIGENCE} -3+CR/6\\
\textbf{WISDOM} -1+CR/5\\
\textbf{CHARISMA} -3+CR/6\\
\textbf{Initiative} =DEX -- \textbf{Defense} 10+CR+DEX\\
\textbf{Hit Points} CR*CR*6\\
\textbf{Movement} 9 m, excavation 9 m\\
\textbf{Saving Throws} Fortitude +CR+CON, Reflexes +CR+DEX, Will +CR+WIS\\
\textbf{Damage Resistances} from non-magical weapons\\
\textbf{Damage Immunity} Poison\\
\textbf{Condition Immunity} grabbed, poisoned, paralyzed, petrified, unconscious, prone, fatigue\\
\textbf{Damage Vulnerability} sound\\
\textbf{Senses} tremorsense 60 ft., darkvision 60 ft.\\
\textbf{Languages} Tremun\\
\textbf{Challenge} CR\\
\emph{\textbf{Siege Monster.}} The elemental deals double damage to objects and structures.\\
\emph{\textbf{Elemental Nature.}} An elemental has no need for air, food, drink, or sleep.\\
\emph{\textbf{Earth Glide.}} The elemental can burrow through nonmagical, unworked earth and stone. When he does this, the elemental does not disturb the material he moves.
\textbf{Shares}\\
\emph{\textbf{Multiattack.}} The elemental makes two slam attacks.\\
\emph{\textbf{Slam.} Melee weapon attack}: +CR*2+STR to hit, CR/10 feet, one target.\\
\emph{Hits:} CR*3 bludgeoning damage.


\medskip\index[Monstery]{Greater Water Elemental}\textbf{Greater Water Elemental}

\emph{Huge Elemental, Neutral}

\textbf{STRENGTH} +6

\textbf{DEXTERITY} +3

\textbf{CONSTITUTION} +5

\textbf{INTELLIGENCE} -2

\textbf{WISDOM} +1

\textbf{CHARRISMA} +0

\textbf{Initiative} +4 -- \textbf{Defense} 21

\textbf{Hit Points} 158

\textbf{Movement} 9m, swim 33m

\textbf{Saving Throws} Fortitude +12, Reflex +12, Will +6

\textbf{Damage Resistances} acid; from a non-magical weapon

\textbf{Damage Immunity} Poison

\textbf{Condition Immunity} grabbed, poisoned, entangled, paralyzed, petrified, unconscious, prone, fatigued

\textbf{Senses} Darkvision 18 m

\textbf{Languages} Aquan

\textbf{Challenge} 9

€19548 € {€19549 € {Freezing.}} If the elemental takes cold damage, it partially freezes; his movement is reduced by 20 feet until the end of his next round.

\emph{\textbf{Water Form.}} The elemental can enter a hostile creature's space and stop there. He can move through a space as narrow as 3 centimeters without having to squeeze himself.

\emph{\textbf{Elemental Nature.}} An elemental has no need for air, food, drink, or sleep.

\textbf{Shares}

\emph{\textbf{Multiattack.}} The elemental makes two slam attacks.

\emph{\textbf{Slam.} Melee weapon attack}: +19 to hit, reach 10 ft., one target.

\emph{Hits:} 26 (4d8 + 10) bludgeoning damage.

\emph{\textbf{Submerge (Recharge 4-6).}} Each creature in the elemental's space must make a DC 21 Fortitude saving throw. On a failed save, the target takes 25 (5d8 + 5) bludgeoning damage . If it is Huge or smaller, the target is also grappled (DC 19 to escape). Until the grapple ends, the target is restrained and cannot breathe unless it can breathe water. If the saving throw succeeds, the target is pushed out of the elemental's space.

The elemental can grapple one Huge creature or up to two Large or smaller ones at a time. At the start of each elemental's round, each grappled target takes 25 (4d8 + 5) bludgeoning damage. A creature within 10 feet of the elemental can pull a creature or object out of it, spending an action to attempt a successful DC 19 Strength check.

\textbf{Ecology}\\
Environment: Any (Water Plane)\\
Organization: Solo, couple or group (3-8)\\
\textbf{Treasure}: None\\
\textbf{Description}\\
Water elementals are patient, unyielding creatures composed of living water, fresh or salty. They prefer to cover their opponents with water or drag them in to gain an advantage.\\
Like other elementals, all water elementals have a unique appearance and shape. Many are wave-like creatures with vaguely humanoid faces and smaller waves at the sides that act as arms. Another common form is that of some aquatic creature, such as a shark or octopus, but made entirely of water.\\
A large water elemental stands 30 feet tall and weighs 22,000 pounds.

\medskip\index[Monstery]{Water Elemental}\textbf{Water Elemental}

\emph{Large elemental, neutral}

\textbf{STRENGTH} +4

\textbf{DEXTERITY} +2

\textbf{CONSTITUTION} +4

\textbf{INTELLIGENCE} -3

\textbf{WISDOM} +0

\textbf{CHARISMA} -1

\textbf{Initiative} +4 -- \textbf{Defense} 17

\textbf{Hit Points} 114 (12d10 + 48)

\textbf{Movement} 9m, swim 27m

\textbf{Saving Throws} Fortitude +9, Reflexes +8, Will +2

\textbf{Damage Resistances} acid; from a non-magical weapon

\textbf{Damage Immunity} Poison

\textbf{Condition Immunity} grabbed, poisoned, entangled, paralyzed, petrified, unconscious, prone, fatigued

\textbf{Senses} Darkvision 18 m

\textbf{Languages} Aquan

\textbf{Challenge} 5 (1800 PX)

€19586 € {€19587 € {Freezing.}} If the elemental takes cold damage, it partially freezes; his movement is reduced by 20 feet until the end of his next round.

\emph{\textbf{Water Form.}} The elemental can enter a hostile creature's space and stop there. He can move through a space as narrow as 3 centimeters without having to squeeze himself.

\emph{\textbf{Elemental Nature.}} An elemental does not need air,
food, drink or sleep.

\textbf{Shares}

\emph{\textbf{Multiattack.}} The elemental makes two slam attacks.

\emph{\textbf{Slam.} Melee weapon attack}: +11 to hit, reach 1 m, one target.

\emph{Hits:} 13 (2d8 + 4) bludgeoning damage.

\emph{\textbf{Submerge (Recharge 4-6).}} Each creature in the elemental's space must make a DC 15 Fortitude saving throw. On a failed save, the target takes 13 (2d8 + 4) bludgeoning damage . If it is Large or smaller, the target is also grappled (DC 14 to escape). Until the grapple ends, the target is restrained and cannot breathe unless it can breathe water. If the saving throw succeeds, the target is pushed out of space
of the elemental.

The elemental can grapple one Large or up to two Medium or smaller creatures at a time. At the start of each elemental's round, each grappled target takes 13 (2d8 + 4) bludgeoning damage. A creature within 3 feet of the elemental can pull a creature or object out of it, spending an action to attempt a DC 14 Strength check.

\textbf{Ecology}\\
Environment: Any (Water Plane)\\
Organization: Solo, couple or group (3-8)\\
\textbf{Treasure}: None\\
\textbf{Description}\\
Water elementals are patient, unyielding creatures composed of living water, fresh or salty. They prefer to cover their opponents with water or drag them in to gain an advantage.\\
Like other elementals, all water elementals have a unique appearance and shape. Many are wave-like creatures with vaguely humanoid faces and smaller waves at the sides that act as arms. Another common form is that of some aquatic creature, such as a shark or octopus, but made entirely of water.\\
A large water elemental stands 15 feet tall and weighs 2500 pounds.


\medskip\index[Monstery]{Minor Water Elemental}\textbf{Minor Water Elemental}

\emph{Medium elemental, neutral}

\textbf{STRENGTH} +2

\textbf{DEXTERITY} +1

\textbf{CONSTITUTION} +2

\textbf{INTELLIGENCE} -3

\textbf{WISDOM} +0

\textbf{CHARISMA} -1

\textbf{Initiative} +4 -- \textbf{Defense} 15

\textbf{Hit Points} 16 (2d8 + 4)

\textbf{Movement} 9m, swim 27m

\textbf{Saving Throws} Fortitude +3, Reflexes +4, Will +0

\textbf{Damage Resistances} acid; from a non-magical weapon

\textbf{Damage Immunity} Poison

\textbf{Condition Immunity} grabbed, poisoned, entangled, paralyzed, petrified, unconscious, prone, fatigued

\textbf{Senses} Darkvision 18 m

\textbf{Languages} Aquan

\textbf{Challenge} 2

€19624 € {€19625 € {Freezing.}} If the elemental takes cold damage, it partially freezes; his movement is reduced by 20 feet until the end of his next round.

\emph{\textbf{Water Form.}} The elemental can enter a hostile creature's space and stop there. He can move through a space as narrow as 3 centimeters without having to squeeze himself.

\emph{\textbf{Elemental Nature.}} An elemental has no need for air, food, drink, or sleep.

\textbf{Shares}

\emph{\textbf{Multiattack.}} The elemental makes two slam attacks.

\emph{\textbf{Slam.} Melee weapon attack}: +5 to hit, reach 1 m, one target.

\emph{Hits:} 6 (1d6 + 2) bludgeoning damage.

\emph{\textbf{Submerge (Recharge 4-6).}} Each creature in the elemental's space must make a DC 13 Fortitude saving throw. On a failed save, the target takes 8 (2d4 + 4) bludgeoning damage . If it is Medium or smaller, the target is also grappled (DC 12 to escape). Until the grapple ends, the target is restrained and cannot breathe unless it can breathe water. If the saving throw succeeds, the target is pushed out of the elemental's space.

The elemental can grab one Medium or up to two Small creatures at a time. At the start of each elemental's round, each grappled target takes 8 (2d4 + 4) bludgeoning damage. A creature within 3 feet of the elemental can pull a creature or object out of it, spending an action to attempt a successful DC 12 Strength check.

\textbf{Ecology}\\
Environment: Any (Water Plane)\\
Organization: Solo, couple or group (3-8)\\
\textbf{Treasure}: None\\
\textbf{Description}\\
Water elementals are patient, unyielding creatures composed of living water, fresh or salty. They prefer to cover their opponents with water or drag them in to gain an advantage.\\
Like other elementals, all water elementals have a unique appearance and shape. Many are wave-like creatures with vaguely humanoid faces and smaller waves at the sides that act as arms. Another common form is that of some aquatic creature, such as a shark or octopus, but made entirely of water.\\
A large water elemental stands 6 feet tall and weighs 180 pounds.

\medskip\index[Monstery]{Air Elemental}\textbf{Air Elemental}

\emph{Large elemental, neutral}

\textbf{STRENGTH} +2

\textbf{DEXTERITY} +5

\textbf{CONSTITUTION} +2

\textbf{INTELLIGENCE} -2

\textbf{WISDOM} +0

\textbf{CHARRISMA} -2

\textbf{Initiative} +5 -- \textbf{Defense} 18

\textbf{Hit Points} 90 (12d10 + 24)

\textbf{Movement} 0 m, fly 27 m (floats)

\textbf{Saving Throws} Fortitude +9, Reflexes +13, Will +2

\textbf{Damage Resistances} lightning, sound; from a non-magical weapon

\textbf{Damage Immunity} Poison

\textbf{Condition Immunity} grabbed, poisoned, entangled, paralyzed, petrified, unconscious, prone, fatigue

\textbf{Senses} Darkvision 18 m

\textbf{Languages} Ictun

\textbf{Challenge} 5 (1800 PX)

\emph{\textbf{Air Form.}} The elemental can enter a hostile creature's space and stop there. He can move through a space as narrow as 3 centimeters without having to squeeze himself.

\emph{\textbf{Elemental Nature.}} An elemental has no need for air, food, drink, or sleep.

\textbf{Shares}

\emph{\textbf{Multiattack.}} The elemental makes two slam attacks.

\emph{\textbf{Slam.} Melee weapon attack}: +8 to hit, reach 1 m, one target.

\emph{Hits:} 14 (2d8 + 5) bludgeoning damage.

\emph{\textbf{Whirlwind (Recharge 4-6).}} Each creature in the elemental's space must make a DC 13 Fortitude saving throw. On a failed save, the target takes 15 (3d8 + 2) bludgeoning damage and is thrown 20 feet away from the elemental in a random direction and falls prone. If a thrown target hits an object, such as a wall or the floor, it takes 3 (1d6) bludgeoning damage for every 10 feet it was thrown. If the target is thrown at another creature, that creature must succeed on a DC 13 Reflex saving throw or take the same damage and be knocked prone.

On a successful save, the target takes half bludgeoning damage and is not knocked away or knocked prone.

\textbf{Ecology}\\
Environment: Plane of Air\\
Organization: Solo, couple or group (3-8)\\
\textbf{Treasure}: None\\
\textbf{Description}\\
Air elementals are swift, flying creatures made of air. Primitive and territorial, they do not like to be summoned or controlled by mortals, and prefer to spend their time on the Plane of Air, flying across the infinite sky.\\
While all air elementals of the same size have the same stats, the exact appearance of each varies greatly between individuals: one may appear as an animated swirl of wind and smoke, while another as a creature of smoke similar to a bird with sparkling eyes and windy wings.\\
An air elemental prefers to attack flying creatures, not only because of the advantages it has with its mastery of the air, but also because it hates touching the ground. An air elemental can move underwater, and while it is in no danger of drowning, it has no ranks in Swim and loses much of its mobility and speed underwater.\\
A Great Air elemental stands 15 feet tall and weighs 5 pounds.

\medskip\index[Monstery]{Fire Elemental}\textbf{Fire Elemental}

\emph{Large elemental, neutral}

\textbf{STRENGTH} +0

\textbf{DEXTERITY} +3

\textbf{CONSTITUTION} +3

\textbf{INTELLIGENCE} -2

\textbf{WISDOM} +0

\textbf{CHARRISMA} -2

\textbf{Initiative} +3 -- \textbf{Defense} 16

\textbf{Hit Points} 102 (12d10 + 36)

\textbf{Movement} 15 m

\textbf{Saving Throws} Fortitude +8, Reflexes +11, Will +4

\textbf{Damage Resistances} from non-magical weapons

\textbf{Damage Immunity} Fire, poison

\textbf{Condition Immunity} grabbed, poisoned, entangled, paralyzed, petrified, prone, unconscious, fatigue

\textbf{Senses} Darkvision 18 m

\textbf{Languages} Ignan

\textbf{Challenge} 5 (1800 PX)

\emph{\textbf{Fire Form.}} The elemental can move through a space up to 3 centimeters wide without squeezing. A creature that contacts or hits the elemental with a melee attack while within 3 feet of it takes 5 (1d10) fire damage. Additionally, the elemental can enter a hostile creature's space and stop there. The first time it enters a creature's space in a round, the creature takes 5 (1d10) fire damage and catches fire; until someone takes an action to put out the flames, the creature takes 5 (1d10) fire damage at the start of each of its rounds.

\emph{\textbf{Lighting.}} The elemental emits bright light in a 30-foot radius and dim light for an additional 30 feet.

\emph{\textbf{Elemental Nature.}} An elemental has no need for air, food, drink, or sleep.

\emph{\textbf{Susceptibility to Water.}} The elemental takes 1 cold damage for every 1 meter it moves in water or for every 4 liters of water splashed on it.

\textbf{Shares}

\emph{\textbf{Multiattack.}} The elemental makes two touch attacks.

\emph{\textbf{Contact.} Melee weapon attack}: +7 to hit, reach 1 m, one target.

\emph{Hits:} 10 (2d8 + 5) fire damage. If the target is a flammable creature or object, it catches fire. Until a creature takes an action to extinguish the flames, the creature takes 5 (1d10) fire damage at the start of each of its rounds.

\textbf{Ecology}
Environment: Any (Plane of Fire)\\
Organization: Solo, couple or group (3-8)\\
\textbf{Treasure}: None\\
\textbf{Description}\\
Fire elementals are fast, cruel creatures made of living flames. They enjoy frightening those weaker than themselves, and terrorize any creature they can set alight. A fire elemental cannot enter water or any nonflammable liquid. A body of water is an impenetrable barrier unless the elemental can step over or jump over it, or it is covered with flammable material (such as a film of oil).\\
Fire elementals vary in appearance; They typically manifest in the form of serpentine coils of smoke and flame, but some fire elementals take on appearances more similar to those of humans, demons, or other monsters to add to the terror when they suddenly appear. A fire elemental's body appears to be made of semi-stable flames or puffs of sparks, smoke, or ash.\\

A large fire elemental stands 15 feet tall.

\medskip\index[Monstery]{Earth Elemental}\textbf{Earth Elemental}

\emph{Large Elemental, Neutral}

\textbf{STRENGTH} +5

\textbf{DEXTERITY} -1

\textbf{CONSTITUTION} +5

\textbf{INTELLIGENCE} -3

\textbf{WISDOM} +0

\textbf{CHARRISMA} -3

\textbf{Initiative} -1 -- \textbf{Defense} 20

\textbf{Hit Points} 126 (12d10 + 60)

\textbf{Movement} 9 m, excavation 9 m

\textbf{Saving Throws} Fortitude +9, Reflex +1, Will +6

\textbf{Damage Vulnerability} sound

\textbf{Damage Resistances} from non-magical weapons

\textbf{Damage Immunity} Poison

\textbf{Condition Immunity} poisoned, paralyzed, petrified, prone, unconscious, fatigue,

\textbf{Senses} tremor perception 60 ft., darkvision 60 ft

\textbf{Languages} Tremun

\textbf{Challenge} 5 (1800 PX)

\emph{\textbf{Siege Monster.}} The elemental deals double damage to objects and structures.

\emph{\textbf{Elemental Nature.}} An elemental has no need for air, food, drink, or sleep.

\emph{\textbf{Earth Glide.}} The elemental can burrow through nonmagical, unworked earth and stone. When doing so, the elemental does not disturb the material it moves.
\textbf{Shares}

\emph{\textbf{Multiattack.}} The elemental makes two slam attacks.

\emph{\textbf{Slam.} Melee weapon attack}: +12 to hit, reach 10 ft., one target.

\emph{Hits:} 14 (2d8 + 5) bludgeoning damage.

\textbf{Ecology}
Environment: Any (Plane of Earth)\\
Organization: Solo, couple or group (3-8)\\
\textbf{Treasure}: None\\
\textbf{Description}\\
Earth elementals are slow, stubborn creatures made of stone or earth. When they stand completely still they are indistinguishable from a pile of rocks or a small hill.\\

When an earth elemental moves heavily, its outward appearance may change, even if its statistics remain identical to those of its peers of the same size. Earth elementals are mostly made of rock, earth, or crystal, with glittering gems for eyes. The larger ones have the appearance of stone humanoids. Clumps of vegetation often grow on the ground that forms part of an earth elemental's body.\\

A large earth elemental stands 15 feet tall and weighs 10,000 pounds.

\medskip\index[Monstery]{Ettercap}\textbf{Ettercap}

\emph{Medium Monstrosity, Neutral Evil}

\textbf{STRENGTH} +2

\textbf{DEXTERITY} +2

\textbf{CONSTITUTION} +1

\textbf{INTELLIGENCE} -2

\textbf{WISDOM} +1

\textbf{CHARISMA} 8 (-2)

\textbf{Initiative} +2 -- \textbf{Defense} 14

\textbf{Hit Points} 44 (8d8 + 8)

\textbf{Movement} 9m, climb 9m

\textbf{Saving Throws} Fortitude +6, Reflexes +4, Will +6

\textbf{Skills} Stealth +4, Awareness +3, Survival +3

\textbf{Senses} Darkvision 18 m

\textbf{Languages} -

\textbf{Challenge} 2 (450 PX)

\emph{\textbf{Walking on the Web.}} The ettercap ignores movement restrictions caused by webs.

\emph{\textbf{Climb as a Spider.}} The ettercap can climb difficult surfaces, including standing upside down on the ceiling, without needing to make a check.

\emph{\textbf{Web Sense.}} While in contact with a web, the ettercap knows the exact location of any other creature in contact with the same web.

\textbf{Shares}

\emph{\textbf{Multiattack.}} The ettercap makes two attacks: one with its bite and one with its claws

\emph{\textbf{Claws.} Melee weapon attack}: +4 to hit, reach 1 m, one target.

\emph{Hits:} 7 (2d4 + 2) slashing damage, 1 bleed damage.

\emph{\textbf{Bite.} Melee weapon attack}: +4 to hit, reach 1 m, one target.

\emph{Hits:} 6 (1d8 + 2) piercing damage plus 4 (1d8) poison damage. The target must succeed on a DC 11 Fortitude save or be poisoned, -1 Strength and Dexterity, for 1 minute. The creature can repeat the saving throw at the end of each of its rounds, ending the effect if it succeeds.

\emph{\textbf{Web (Recharge 5-6).} Ranged Weapon Attack}: +4 to hit, range 30ft, one Large or smaller creature. \emph{Hits:} The creature is entangled by the web. As an action, the entangled creature can make a DC 11 Strength check, freeing itself from the web on a success. The effect ends if the canvas is destroyed. The web has Defense 10, 5 Hit Points, vulnerability to fire damage, and immunity to bludgeoning and poison damage.

\textbf{Ecology}\\
Environment: Temperate Forests\\
Organization: solitary, pair or nest (3-6 plus 2-8 giant spiders)\\
\textbf{Treasury}: Standard\\
\textbf{Description}\\
Ettercaps are usually 1.8 meters tall and weigh around 100 kg. They are solitary and rarely join others of their kind except for mating. When they group together, they tend to attract various species of spiders, forming a strange mix of ettercaps and arachnids.\\
Ettercaps are known for constructing cunning traps made of spiderwebs and other natural materials, which they use to capture prey. They build spider web shelters in the highest branches of trees away from other terrestrial predators, and use monstrous spiders as lookouts and guardians.\\
Ettercaps are not brave, but their traps often prevent the enemy from drawing their weapons. An ettercap attacks with claws and venomous bites. It generally avoids melee with opponents who can still move and flees if they break free.


\medskip\index[Monstery]{Ettin}\textbf{Ettin}

\emph{Great giant, chaotic evil}

\textbf{STRENGTH} +5

\textbf{DEXTERITY} -1

\textbf{CONSTITUTION} +3

\textbf{INTELLIGENCE} -2

\textbf{WISDOM} +0

\textbf{CHARRISMA} -1

\textbf{Initiative} -1 -- \textbf{Defense} 14

\textbf{Hit Points} 85 (10d10 + 30)

\textbf{Movement} 12 m

\textbf{Saving Throws} Fortitude +9, Reflexes +2, Will +5

\textbf{Skills} Awareness +4

\textbf{Languages} Giant, Goblinoid

\textbf{Challenge} 4 (1100 PX)

\emph{\textbf{Two Heads.}} The ettin has +1d6 on Wisdom (Awareness) checks and on saving throws against blinded, charmed, deafened, unconscious, frightened, and stunned conditions.

\emph{\textbf{Wake.}} When one of the ettin's heads is asleep, the other is awake.

\textbf{Shares}

\emph{\textbf{Multiattack.}} The ettin makes two attacks: one with the battle ax and one with the spiked mace.

\emph{\textbf{Battle Axe.} Melee weapon attack}: +11 to hit, reach 1 m, one target.

\emph{Hits:} 14 (2d8 + 5) slashing damage.

\emph{\textbf{Spiked Mace.} Melee weapon attack}: +11 to hit, reach 1 m, one target.

\emph{Hits:} 14 (2d8 + 5) piercing damage.

\textbf{Ecology}\\
Environment: Cold hills\\
Organization: Solitary, pair, group (3-6), troop (1-2 plus 1-2 Brown Bears, band (3-6 plus 1-2 Brown Bears) or colony (3-6 plus 1-2 Brown Bears and 7-12 Orcs, or 9-16 Goblins)\\
\textbf{Treasure}: Standard (Leather Armor, 2 Light Flails, 4 Javelins, more treasure)\\
\textbf{Description}\\
Ettin, or two-headed giants, are malevolent and unpredictable nocturnal hunters. The two heads grant them unparalleled powers of perception, making them excellent guardians.\\
Ettin look like Hill Giants and Stone Giants, but their fanged faces betray orcish ancestry. They have pinkish brown skin and never bathe unless forced to, which makes them so dirty and filthy that their skin appears thick and grey.\\
Adults are 3.9 meters tall and weigh 2,600 kg. Ettin live about 75 years.\\
The ettin have no language of their own but speak a mixed jargon of Giant, Goblin, and Orc. Creatures that speak any of these languages ​​can communicate with an ettin by making a DC 15 Intelligence check. The check is made once for each piece of information; if the other creature speaks two of these languages ​​the DC is 10, while for someone who speaks all three it is 5.\\
Although the ettin are not very intelligent, they are cunning warriors. They prefer to ambush their victims rather than engage them in combat, but once battle has begun, an ettin fights furiously until the enemy is dead.\\
Ettin are solitary creatures, taking up residence in the safety of rock caves and cavities, often surrounded by holes and ditches, and sometimes keeping cave bears as pets or guardians.\\
A particularly powerful ettin can attract a group of followers, especially Goblins or Orcs. However, these gatherings are mostly exceptions, and rarely last long, with the individualistic Ettin going their own way as soon as opportunities for plunder and plunder diminish or the leader is killed.\\
Typically they form breeding pairs to raise their offspring only for short periods before each going their own way. Young ettins mature rapidly, reaching adult size within a year, allowing them to fend for themselves.

\medskip\index[Monster]{Ghost}\textbf{Ghost}

\emph{Average undead, any trait}

\textbf{STRENGTH} -2

\textbf{DEXTERITY} +1

\textbf{CONSTITUTION} +0

\textbf{INTELLIGENCE} +0

\textbf{WISDOM} +1

\textbf{CHARISMA} +3

\textbf{Initiative} +1 -- \textbf{Defense} 13

\textbf{Hit Points} 45 (10d8)

\textbf{Movement} 0 m, fly 12 m (floats)

\textbf{Saving Throws} Fortitude +7, Reflexes +6, Will +7

\textbf{Damage Resistances} acid, lightning, fire, sound; from nonmagical weapons

\textbf{Damage Immunity} Cold, Void, Poison

\textbf{Condition Immunity} charmed, grabbed, poisoned, entangled, paralyzed, petrified, prone, fatigued, frightened, bleeding

\textbf{Senses} Darkvision 18 m

\textbf{Languages} any language known in life, Exspiran

\textbf{Challenge} 4 (1100 PX)

\emph{\textbf{Incorporeal Movement.}} The ghost can pass through other creatures and objects as if they were difficult terrain. He takes 5 (1d10) force damage if he ends his round inside an object.

\emph{\textbf{Undead Nature.}} The ghost does not need air, food, drink or sleep.

\emph{\textbf{Ethereal Sight.}} The ghost can see 60 feet into the Ethereal Plane when on the Material Plane, and vice versa.

\textbf{Shares}

\emph{\textbf{Withering Touch.} Melee weapon attack}: +6 to hit, reach 1 m, one target.

\emph{Hits:} 17 (4d6 + 3) Void damage. The target must make a Fortitude save at DC 15 or become fatigued.

\emph{\textbf{Ethereality.}} The ghost enters the Ethereal Plane from the Material Plane, or vice versa. She is visible on the Material Plane while in the Ethereal Plane, and vice versa, but cannot interact with anything on the other plane.

\emph{\textbf{Possession (Recharge 6).}} A humanoid, within 3 feet and visible to the ghost, must succeed on a DC 13 Will save or become possessed by the ghost; the ghost then disappears, and the target is incapacitated and loses control of its body. The ghost now controls the body but does not deprive the target of awareness of her. The ghost cannot be the target of attacks, spells, or other effects, except those that turn undead, and retains its Traits, Intelligence, Wisdom, Charisma, and immunity to being charmed and frightened. Otherwise it uses the statistics of the possessed target, but does not access the target's knowledge and skills.

The possession lasts until the body drops to 0 Hit Points, the ghost ends it with a bonus action, or the ghost is turned or expelled by an effect such as the \emph{dispel good and evil} spell. When the possession ends, the ghost reappears in an unoccupied space within 3 feet of the body. The target is immune to this ghost's possession for 24 hours after a successful saving throw or after the possession ends.

\emph{\textbf{Horrific Face.}} Each creature that is not undead within 60 feet of the ghost and can see it must succeed on a DC 13 Will save or be frightened for 1 minute. If the saving throw fails by 5 or more, the target also ages 1d4 x 10 years. A frightened target can repeat the saving throw at the end of each of its rounds, ending the effect for itself if it succeeds. If the target's saving throw succeeds and the effect ends for him, the target is immune to the ghost's Horrifying Visage for the next 24 hours. The aging effect can be reversed with the spell \emph{greater restoration}, but only if cast within 24 hours of the aging effect.

\textbf{Ecology}
Environment: any\\
Organization: solitaire\\
\textbf{Treasure}: NPC equipment\\
\textbf{Description}\\
When a soul is not granted rest due to some grave injustice, real or perceived, it sometimes returns as a ghost. These beings are eternally anguished, devoid of substance and incapable of making things right. While ghosts can have any Trait, many cling to the world of the living with a strong sense of hatred and anger, and become evil as a result; even a good creature can become a hateful and cruel ghost after death.\\

More than other monsters, the ghost must have a well-defined background. Why did this character become a ghost? What legends surround him? An encounter with a ghost should never happen accidentally - there are plenty of other incorporeal undead, such as Wraiths and Wraiths, for that. A proper encounter with a ghost should occur in a scene at the culmination of a long period of tension built with minor servants or manifestations of undead spirits. The ghost example above represents a human princess murdered by an unfaithful lover; after a confrontation, he tied her with chains and threw her into the castle well, where she drowned. The ghost's abilities were selected based on the background, showing how a powerful antagonist can be created. By applying the archetype to creatures with their own levels and therefore skills or with significant racial abilities, much more powerful ghosts can be created.\\

When a ghost is created, it obtains copies of the objects that it valued in life (provided that the originals are not in the possession of other creatures). Equipment works normally for the ghost but passes through material objects or creatures. A +1 or higher-enhanced weapon, however, can harm material creatures. A ghost can use shields and armor only if they have the Ghost Touch ability.\\

The original items are left behind, just like the ghost's physical remains. If another creature holds the original, the incorporeal copy vanishes. This loss inevitably infuriates the ghost, who stops at nothing to return the object to where it originally lay (and regain its use).


\medskip\index[Monstery]{Gurgling Maws}\textbf{Gurgling Maws}

\emph{Average aberration, neutral}

\textbf{STRENGTH} +0

\textbf{DEXTERITY} -1

\textbf{CONSTITUTION} +3

\textbf{INTELLIGENCE} -4

\textbf{WISDOM} +0

\textbf{CHARISMA} -2

\textbf{Initiative} -1 -- \textbf{Defense} 10

\textbf{Hit Points} 67 (9d8 + 27)

\textbf{Movement} 3m, swim 3m

\textbf{Saving Throws} Fortitude +8, Reflexes +4, Will +5

\textbf{Condition Immunity} prone

\textbf{Senses} Darkvision 18 m

\textbf{Languages} -

\textbf{Challenge} 2 (450 PX)

€19887 € {€19888 € {Gurgling.}} As long as the maw can see a creature and is not incapacitated, it speaks incoherent sentences. Any creature that begins its round within 20 feet of the maw and can hear its gurgling must make a DC 10 Will save. On a failed save, the creature can take no reactions until the start of its next round and rolls a d8 to determine what he will do during his round. On a 1 to 4, the creature does nothing. On a 5 or 6, the creature takes no action or bonus action and uses all of its movement to move in a randomly determined direction. On a 7 or 8, the creature makes a melee attack against a randomly determined creature within its reach or does nothing if it is unable to make such an attack.

\emph{\textbf{Aberrant Terrain.}} Terrain within a 10-foot radius around the maw is considered difficult terrain. Each creature that begins its round in that area must succeed on a DC 10 Fortitude saving throw or have its movement reduced to 0 until the start of its next round.

\textbf{Shares}

\emph{\textbf{Multiattack.}} The gurgling maw makes a bite attack and, if it can, a Blinding Spit.

\emph{\textbf{Bite.} Melee weapon attack}: +3 to hit, reach 3 ft., one creature.

\emph{Hits:} 17 (5d6) piercing damage. If the target is Medium or smaller, it must succeed on a DC 10 Fortitude save or be knocked prone. If the target is killed by this damage, it is absorbed by the maw.

\emph{\textbf{Blinding Spit (Recharge 5-6).}} The maw spits a chemical orb at a visible point within 5 meters of it. The orb explodes on impact in a blinding flash of light. Each creature within 3 feet of the flash must succeed on a DC 13 Reflex saving throw or be blinded until the end of the maw's next round.

\textbf{Ecology}\\
Environment: Any Dungeon\\
Organization: Solitaire\\
\textbf{Treasury}: Standard\\
\textbf{Description}\\
Disgusting, nauseating and hungry: these are the only words that aptly describe the gurgling maw. Foul beasts that lurk in caves, sewers, and nightmares, the Maw has no social, ecological, or religious meaning other than its ability to drive those who hear it mad. Some scholars believe that the gurgling maw is a smaller variant of the much more dangerous shoggoth, while others theorize that it is a punishment from some powerful entity or deity inflicted on those who have wronged it.

\medskip\index[Monstery]{Phoenix}\textbf{Phoenix}

\emph{Heavenly Gargantuan, Brave, Protective, Good}

\textbf{STRENGTH} +8

\textbf{DEXTERITY} +6

\textbf{CONSTITUTION} +5

\textbf{INTELLIGENCE} +5

\textbf{WISDOM} +6

\textbf{CHARISMA} +6

\textbf{Initiative} +11 -- \textbf{Defense} 28

\textbf{Hit Points} 210 (20d10 + 100)

\textbf{Damage Vulnerability} magical cold

\textbf{Movement} 9 m, fly 27 m (good)

\textbf{Saving Throws} Fortitude +19, Reflexes +21, Will +21

\textbf{Damage Immunity} Fire, Light, poison, weapons +1

\textbf{Condition Immunity} grabbed, poisoned, entangled, paralyzed, petrified, prone, unconscious, fatigued, bleeding

\textbf{Regeneration} A Phoenix regenerates 10 Hit Points at the start of each of its rounds

\textbf{Senses} Darkvision 60 ft., Low-light vision 60 ft.

\textbf{Languages} Ictun, Celestial, Common, Ignan

\textbf{Challenge} 15 (13000 PX)

\emph{\textbf{Awareness of Light.}} The Phoenix always has the following spells active \emph{Detect Magic, Detect Disease and Poison, See Invisibility}

\emph{\textbf{Innate Spells.}} The Phoenix's spellcasting ability is Charisma. The Phoenix can innately cast the following spells, without the need for material components:

At will: \emph{Cure Critical Wounds, Dispel Magic, Eternal Flame, Remove Curse, Metamorphosis (humanoids only)}

3/day: \emph{Mass Cure Critical Wounds, Heal, Wall of Fire, Greater Restoration, Firestorm}

1 time: €19931 € {Resurrection} the Phoenix, by sacrificing its life permanently, can bring a creature back to life.

\textbf{Shares}

\emph{\textbf{Multiattack.}} The Phoenix can attack with two claws and its bite

\emph{\textbf{Bite.} Melee weapon attack}: +23 to hit, reach 20 ft., one creature.

\emph{Hits:} 19 piercing damage (2d8+8 + 1d6 Light)

\emph{\textbf{Claw.} Melee weapon attack}: +23 to hit, reach 20 ft., one creature.

\emph{Hits:} 17 slashing damage (2d6+8 + 1d6 Light)

\textbf{Special abilities}

\emph{\textbf{Rebirth}}

A slain Phoenix is ​​reduced to a 3 cubic meter bonfire where a phoenix egg lies in the center. After 1d4+4 rounds this egg hatches into a perfectly healthy Phoenix. The only way to avoid rebirth is to remove the egg from the bonfire (20d6 Light damage) or use a Disintegrate spell on the egg.
A Phoenix can resurrect in this way once a year, if she dies before this time has passed, the death is final. Killing a Phoenix unleashes the wrath of the Pupils of Light and the knights of Sumkjr.

\emph{\textbf{Wings of Flame}}

The Phoenix can turn its feathers into flame without using Actions. These feathers deal 1d6 Fire damage + 1d6 Light damage to all creatures within 20 feet at the start of its round.

\emph{\textbf{Angry:}} only legends tell of an angry Phoenix and it is said that a Patron intervened directly.

\textbf{Ecology}\\
Environment: Deserts and hot hills\\
Organization: Solitaire\\
\textbf{Treasury}: Standard\\
\textbf{Description}\\
Legend has it that the Phoenixes are Ljust's pet birds, they are certainly majestic and beautiful creatures and emanate a light similar to that of the Patroness of Genesis. The movement of their wings produces no noise while their voice is singing. The phoenix is ​​a legendary bird of fire and light that usually lives in deserts. They are very intelligent and wise creatures and sometimes using their ability to metamorphose they travel to cities where they help those who fight against the darkness.

\medskip\index[Monstery]{Bone Bloom}\textbf{Bone Bloom}

\emph{Large undead, unaligned}

\textbf{STRENGTH} +3

\textbf{DEXTERITY} +2

\textbf{CONSTITUTION} +4

\textbf{INTELLIGENCE} -2

\textbf{WISDOM} -2

\textbf{CHARISMA} -3

\textbf{Initiative} +2 -- \textbf{Defense} 18

\textbf{Hit Points} 105 (6d10 + 64)

\textbf{Movement} 12 m

\textbf{Saving Throws}: Fortitude +10, Reflexes +8, Will +4

\textbf{Damage Vulnerability} from Void

\textbf{Damage Immunity} Poison

\textbf{Damage Resistances} piercing, cutting, from Light

\textbf{Condition Immunity} poisoned, fatigue, bleeding, slowed, slow

\textbf{Senses} Blindsight 18 m

\textbf{Languages} includes common, druidic, sylvan but cannot speak

\textbf{Challenge} 6 (2300 PX)

\emph{\textbf{One foot in Nature.}} As long as Bone Bloom is in contact with the earth he regenerates 6 Hit Points at the start of his round.

\emph{\textbf{One in Nature.}} As long as Bone Bloom is in a natural environment and does not move, it attacks by surprise if not noticed. An Awareness 21 check is required to notice this.

\emph{\textbf{Undead Nature.}} Bone Bloom requires no air, food, drink, or sleep.

\textbf{Shares}

\emph{\textbf{Multiattack}} Bone Bloom can attack with the Great Club 3 times or Breathe Spores and perform a Great Club attack

\emph{\textbf{Great Club.} Melee weapon attack}: +9 to hit, reach 2 ft., one target.

\emph{Hits:} 17 (2d10 + 6) bludgeoning damage

\emph{\textbf{Spore Breath}}: 6 meter radius. Bone Bloom releases spores and pollen all around it. Any creature breathing within 20 feet of the Bone Bloom must make a Fortitude save at DC 17. If the save fails, the creature takes 3d8 poison damage and is under the spell's influence. {Slowness}. If the save is successful, it takes half damage and is slowed until the end of the next round.

\emph{\textbf{Angry:}} the Bone Bloom gathers the energies of nature around itself, withering it. Recovers 50 hit points. It costs 2 Actions.

\textbf{Ecology}\\
Environment: Any forest\\
Organization: Solitaire, groups (2d12)\\
\textbf{Honey}: Accidental\\
\textbf{Description}\\
The Bone Blooms are creatures that died in the depths of the forest for the most disparate reasons. Nature, not wanting to waste anything, animates the creature to make it its defender. At first glance, a Bone Bloom is no different from a collection of colorful lichens, small mushrooms and grass, so much is it covered in nature.


\subsection{Mushrooms}

\medskip\index[Monster]{Screeching Mushroom}\textbf{Screeping Mushroom}

\emph{Average layout, misaligned}

\textbf{STRENGTH} -5

\textbf{DEXTERITY} -5

\textbf{CONSTITUTION} +0

\textbf{INTELLIGENCE} -5

\textbf{WISDOM} -4

\textbf{CHARRISMA} -5

\textbf{Initiative} -5 -- \textbf{Defense} 6

\textbf{Hit Points} 13 (3d8)

\textbf{Movement} 0 m

\textbf{Saving Throws}: Fortitude -3, Reflexes +3, Will -4

\textbf{Condition Immunity} blinded, deafened, scared

\textbf{Senses} blindsight 9 m (blind beyond this range)

\textbf{Languages} -

\textbf{Challenge} 0 (10 PX)

\emph{\textbf{False Appearance.}} While the screeching mushroom remains immobile, it is indistinguishable from a normal mushroom.

\textbf{Shares}

\emph{\textbf{Screech.}} When a bright light or creature is within 30 feet of the screech mushroom, it emits a screech that can be heard up to 300 feet away. The screeching mushroom continues to screech until the source of the disturbance has moved out of range and for another 1d4 rounds thereafter, or until its hat deflates.

\textbf{Ecology}\\
Environment: Any dungeon\\
Organization: Solitary, pair or scrub (3-12)\\
\textbf{Honey}: Accidental\\
\textbf{Description}\\
A screeching mushroom is about 50 cm tall, with a large brown cap. Once the scream is emitted the hat deflates.

There are stories of Duergar cooks specializing in cooking these mushrooms into exquisite dishes. The best ones even manage not to deflate their hat.

\medskip\index[Monster]{Violet Mushroom}\textbf{Violet Mushroom}

\emph{Average layout, misaligned}

\textbf{STRENGTH} -4

\textbf{DEXTERITY} -5

\textbf{CONSTITUTION} +0

\textbf{INTELLIGENCE} -5

\textbf{WISDOM} -4

\textbf{CHARISMA} -5

\textbf{Initiative} -5 -- \textbf{Defense} 6

\textbf{Hit Points} 18 (4d8)

\textbf{Movement} 2 m

\textbf{Saving Throws}: Fortitude -3, Reflexes -3, Will -3

\textbf{Condition Immunity} blinded, deafened, scared

\textbf{Senses} blindsight 9 m (blind beyond this range)

\textbf{Languages} -

\textbf{Challenge} 1/4 (50 XP)

\emph{\textbf{False Appearance.}} While the violet mushroom remains immobile, it is indistinguishable from a normal mushroom.

\textbf{Shares}

\emph{\textbf{Multiattack.}} The mushroom makes 1d4 Putrid Touch attacks.

\emph{\textbf{Putrid Contact.} Melee weapon attack}: +2 to hit, reach 10 ft., one target.

\emph{Hits:} 4 (1d8) Void damage.

\textbf{Ecology}\\
Environment: Any dungeon\\
Organization: Solitary, pair or scrub (3-12)\\
\textbf{Treasure}: Accidental\\
\textbf{Description}\\
Purple mushrooms are one of the most well-known and feared cave dangers. A traveler may often notice the marks left by the purple mushroom on those who live or hunt in places where these carnivorous mushrooms lurk. These deep and horrible scars look like furrows gouged into the flesh: the signs of a close encounter with a purple fungus.

A purple mushroom feeds on decaying organic matter, but unlike most mushrooms, it is not a passive consumer. The tendrils of a purple fungus can strike with unexpected speed, and are coated in a virulent poison that causes flesh to rot with sickening speed. This powerful poison, if left untreated, can quickly rot an entire arm or leg, leaving behind only bones that will soon corrode as well.

Although purple mushrooms can move, they only do so to attack or hunt prey. A purple mushroom with a steady flow of rot to feed on is content to stay in one place. Many underground dwellers, particularly Troglodytes and Vegepygmies, use this behavior to their advantage and place multiple purple mushrooms at key junctions and entrances of their caves as guardians, making sure to provide them with enough corpses to prevent them from entering the shelter in search of food.

Some species of Shrieking Boleto are quite similar in appearance to purple mushrooms, although they lack tentacled branches. It's not unusual to find shriekers and purple mushrooms in the same clump, especially in areas where other creatures grow these mushrooms as guardians.

A purple mushroom is 1.2 meters tall and weighs 25 kg.


\medskip\index[Monstery]{Wisp}\textbf{Wisp}

\emph{Tiny undead, chaotic evil}

\textbf{STRENGTH} -5

\textbf{DEXTERITY} +9

\textbf{CONSTITUTION} +0

\textbf{INTELLIGENCE} +1

\textbf{WISDOM} +2

\textbf{CHARISMA} +0

\textbf{Initiative} +9 -- \textbf{Defense} 20

\textbf{Hit Points} 22 (9d4)

\textbf{Move} 0 m, fly 15 m (float)

\textbf{Saving Throws}: Fortitude +3, Reflexes +12, Will +9

\textbf{Damage Immunity} Lightning, Poison

\textbf{Damage Resistances} acid, cold, fire, void, sound; weapons that are not magical

\textbf{Condition Immunity} grappled, poisoned, entangled, paralyzed, unconscious, prone, fatigue, bleeding

\textbf{Senses} darkvision 36 m

€20070 € {Languages} the languages ​​he knew in life

\textbf{Challenge} 2 (450 PX)

\emph{\textbf{Consume Life.}} As a bonus action, the wisp can target a creature it can see within 3 feet of it that has 0 hit points and is still alive. The target must succeed on a DC 10 Fortitude saving throw against this spell or die. If the target dies, the wisp regains 10 (3d6) hit points.

\emph{\textbf{Ephemeral.}} The will-o'-the-wisp cannot wear or carry anything.

\emph{\textbf{Variable Lighting.}} The will-o'-the-wisp casts bright light in a radius of 1 to 6 meters and dim light for an additional number of meters equal to the chosen radius. The wisp can modify this radius as a bonus action.

\emph{\textbf{Incorporeal Movement.}} The wisp can move through other creatures and objects as if they were difficult terrain. He takes 5 (1d10) force damage if he ends his round inside an object.

\emph{\textbf{Undead Nature.}} The will-o'-the-wisp has no need for air, food, or drink.

\textbf{Shares}

\emph{\textbf{Shock.} Melee spell attack}: +9 to hit, reach 3 ft., one creature.

\emph{Hits:} 9 (2d8) lightning damage.

\emph{\textbf{Invisibility.}} The wisp and its light become magically invisible until it attacks or uses Consume Life, or until its concentration ends (as if it were concentrating on a spell).

\textbf{Ecology}
Environment: Any Swamp\\
Organization: Solo, pair or sequence (3-4)\\
\textbf{Honey}: Accidental\\
\textbf{Description}\\
Every hunter and farmer who lives near a bog or swamp has given a name to these balls of dim light: jack lanterns, candles of the dead, walking fires, pine lights, ghost lights, rush lights; but everyone knows that they are dangerous predators and false guides in the darkness.

Vicious creatures that feed on the strong psychic emanations of terrified creatures, will-o'-the-wisps take pleasure in placing gullible travelers in dangerous situations. In the wilds, where they are very common, will-o'-the-wisps prefer simple tactics such as positioning themselves on rocks or quicksand where they can easily be mistaken for lanterns (especially if they can set the trap near actual warning lanterns), so as to attract travelers towards danger. On rare occasions, will-o'-the-wisps in search of an easy life move into a city and settle near the gallows or follow an army invisibly, so as to feed on the fear of dying men; why the vast majority choose to remain in the swamps, where victims are scarce, remains a mystery.

Wisps can only rely on their electrical shock in dangerous situations, so they prefer to let other creatures or hazards deal with their victims while they float nearby and feast.

Wisps can glow any color they wish, but are most often yellow, white, green, or blue. They can also vary their brightness to create a pattern: many will-o'-the-wisps like to create shapes that vaguely resemble skulls in their luminescence to heighten the terror in their victims. Their true bodies are barely visible globes of translucent spongy material about 30 centimeters long that weigh 1.5 kg and can emit light across their entire surface. Wisps' light shines roughly like a torch, and while they don't appear to use sounds to communicate, they hear perfectly and can vibrate their bodies so rapidly that they mimic speech.

Despite being vilified by most sentient creatures, will-o'-the-wisps are actually quite intelligent, although they think in a completely alien way. Sometimes they organize themselves into groups called sequences; their society and purposes remain completely unknown, as do their origins, although they are sometimes known to make pacts with those who provide them with large quantities of suitably terrified victims.

Wisps have no age and are effectively immortal, unless they die a violent death; Older will-o'-the-wisps can be excellent repositories of knowledge from the past, although convincing one of these cruel creatures to cooperate can be quite tricky.


\medskip\index[Monster]{Flogger}\textbf{Flogger}

\emph{Large monstrosity, neutral evil}

\textbf{STRENGTH} +4

\textbf{DEXTERITY} -1

\textbf{CONSTITUTION} +3

\textbf{INTELLIGENCE} +3

\textbf{WISDOM} +3

\textbf{CHARISMA} -2

\textbf{Initiative} +3 -- \textbf{Defense} 23

\textbf{Hit Points} 93 (11d10 + 33)

\textbf{Movement} 3m, climb 3m

\textbf{Saving Throws}: Fortitude +13, Reflexes +5, Will +13

\textbf{Skills} Stealth +5, Awareness +6

\textbf{Senses} Darkvision 18 m

\textbf{Languages} -

\textbf{Challenge} 5 (1800 PX)

\emph{\textbf{False Appearance.}} When the roper remains motionless, it is indistinguishable from a normal rock formation, such as a stalagmite.

\emph{\textbf{Climb as a Spider.}} The roper can climb difficult surfaces, including standing upside down on the ceiling, without needing to make an ability check.

\emph{\textbf{Grasping Tendrils.}} The roper can have up to six tendrils at a time. Each tendril can be attacked (Defense 20; 10 Hit Points; immunity to poison damage). Destroying a vine deals no damage to the roper, who can produce a replacement vine on his next round. A tendril can also be broken if a creature takes an action and succeeds on a DC 15 Strength check against it.

\textbf{Shares}

\emph{\textbf{Multiattack.}} The roper can make four attacks with its tendrils, use wrap, and make one bite attack.

\emph{\textbf{Bite.} Melee weapon attack}: +7 to hit, reach 2 m, one target.

\emph{Hits:} 22 (4d8 + 4) piercing damage and Purulent Necrosis Disease

\emph{Purulent Necrosis:} 1 day, ST Fortitude DC 15, 12 hours, 1 success, -1 Constitution.

\emph{\textbf{Tensile.} Melee weapon attack}: +7 to hit, reach 50 ft., one creature.

\emph{Hits:} The target is grabbed (DC 15 to escape). Until the grapple ends, the target is restrained and has -1d6 on Strength checks and Fortitude saving throws, while the roper cannot use the same tendril against another target.

\emph{\textbf{Wrap.}} The roper drags creatures it grabs 25 feet toward it.

\emph{\textbf{Angry:}} the roper emits a nauseating cacophonous wave. All creatures within 20 feet must make a Fortitude save DC 18 or be nauseated until the end of the next round. It costs 2 Actions.

\textbf{Ecology}
Environment: Any Dungeon\\
Organization: Solo, couple or group (3-6)\\
\textbf{Treasury}: Standard\\
\textbf{Description}\\
The roper is an ambush hunter. Able to change the coloration and shape of its body, a hidden roper resembles a stalagmite of stone or ice (or in low-ceilinged locations, a column of stone or ice). In areas without these hiding traits, a roper can compress its body until it resembles a boulder. The lashes that he can extrude are not made of flesh but of a thick semi-liquid material similar to partially melted wax but with the resistance of an iron chain and the ability to numb the flesh and weaken the strength. The roper can use these lashes with great skill and fly them up to 15 meters to steal objects that catch his attention.

Despite its alien and monstrous form, the roper is one of the most intelligent inhabitants of the underground. They do not form large societies (although they are often found living together with other underground creatures such as the Brain Eaters, with whom they sometimes ally), but they often congregate in small groups. Particularly interested in the philosophy of life and death, and the more subtle aspects of the world's most sinister and cruel religions, a roper can talk or argue for hours with those he initially simply sought to eat. Some stories tell of particularly gifted orators and philosophers who were kept for days or even years as pets or conversation partners by gangs of whipping boys; Ultimately, though, if they can't escape, the ropers' appetites end up getting the better of their curious intelligence, especially in cases where these pets consistently surpass the wits and patience of their keepers.
A roper is 2.7 meters tall and weighs 1,100 kg.


\medskip\index[Monstery]{Gablin}\textbf{Gablin}

\emph{Little fiend (goblinoid), chaotic evil}

\textbf{STRENGTH} +2

\textbf{DEXTERITY} +1

\textbf{CONSTITUTION} +1

\textbf{INTELLIGENCE} -2

\textbf{WISDOM} -1

\textbf{CHARISMA} -2

\textbf{Initiative} +1 -- \textbf{Defense} 14

\textbf{Hit Points} 6 (2d4 + 2)

\textbf{Movement} 9 m

\textbf{Saving Throws}: Fortitude +4, Reflexes +2, Will +0

\textbf{Damage Resistance}: Void

\textbf{Senses} Darkvision 18 m

\textbf{Languages} understand the Municipality but do not speak it, Abyssal

\textbf{Challenge} 1/2 (100 PX)

\emph{\textbf{Light Sensitivity}}. While in sunlight, the gablin has -1d6 on attack rolls, as well as on sight-based Wisdom (Awareness) checks.

\textbf{Shares}

\emph{\textbf{Short Sword.} Melee weapon attack}: +2 to hit, reach 1 m, one target.

\emph{Hits:} 5 (1d6 + 2) slashing damage.

\emph{\textbf{Bite.} Melee weapon attack}: +3 to hit, touch, one target.

\emph{Hits:} 2 (1d1 + 1) piercing damage.

\textbf{Ecology}\\
Environment: Anywhere\\
Organization: Group (8-12), warband (10-24) or tribe (50+, 1 sergeant 3rd level per 20 adults, 1 or 2 lieutenants 4th or 5th level, 1 leader 6th -8th level, 6-12 wild wolves and 1-4 Ogres or 1-2 Gablin Champion)\\
\textbf{Treasure}: Occasional\\
\textbf{Description}\\
The Gablin are the scum of the scum, it is said that a Gablin is born with every bad thought and there are certainly a lot of them.
The Gablin are small, dark-skinned humanoids with green streaks initially generated by Cattalm's will with the sole purpose of bringing destruction, death and suffering.
Gablin can hide anywhere as long as it is close to a food source, they usually prefer sewers or abandoned structures near villages.
A Gablin's sole purpose is to kill and perpetuate the species. Gablin are all male, and their foul nature makes them capable of impregnating any humanoid female.
Usually gestation lasts only 3 weeks during which women are tortured to strengthen the 1d6+2 babies she carries in her womb. The birth usually ends with the Gablin cubs disembowelling their mother and making her their first meal.
This method of procreation combined with their voracious hunger for blood and flesh make them among the most hated and feared creatures.
Even if individually they are not particularly fearsome, Gablin always move in groups and if this exceeds two dozen then there is almost always a Gablin Spellcaster or even a Gablin Champion to lead them.


\medskip\index[Monster]{Champion Gablin}\textbf{Champion Gablin}

\emph{Great fiend, chaotic evil}

\textbf{STRENGTH} +4

\textbf{DEXTERITY} +2

\textbf{CONSTITUTION} +3

\textbf{INTELLIGENCE} +1

\textbf{WISDOM} +0

\textbf{CHARRISMA} -1

\textbf{Initiative} +2 -- \textbf{Defense} 18

\textbf{Hit Points} 60 (7d10 + 25)

\textbf{Movement} 12 m

\textbf{Saving Throws}: Fortitude +9, Reflexes +6, Will +3

\textbf{Damage Resistance}: Void

\textbf{Senses} Darkvision 18 m

\textbf{Languages} Common, Abyssal

\textbf{Challenge} 3 (700 XP)

\textbf{Shares}

\emph{\textbf{Heavy Club.} Melee weapon attack}: +7 to hit, reach 2 m, one target.

\emph{Hits:} 11 (2d6 + 4) bludgeoning damage.

\emph{\textbf{Summon Gablin}}: 3 Actions. The Gablin spills his blood on the ground and this causes 2d4 Gablins, losing 1 Hit Point

\textbf{Ecology}\\
Environment: Any\\
Organization: Lead a group of Gablins\\
\textbf{Treasure}: Standard (Leather Armor, Heavy Club)\\
\textbf{Description}\\
Gablin Champions are generated spontaneously when the number of Gablin present reaches 20. Vastly larger, stronger and more intelligent than a Gablin, the Champions are the leaders of the group, those who plan the battles and clashes.
They have no qualms about sending the Gablin to slaughter or killing anything that breathes. Imbued with the spirit of Cattalm their aim is always and only to destroy and kill.


\medskip\index[Monstruary]{Paladin Gablin}\textbf{Paladin Gablin}

\emph{Great fiend, chaotic evil}

\textbf{STRENGTH} +5

\textbf{DEXTERITY} +2

\textbf{CONSTITUTION} +3

\textbf{INTELLIGENCE} +2

\textbf{WISDOM} +3

\textbf{CHARISMA} +3

\textbf{Initiative} +4 -- \textbf{Defense} 21

\textbf{Hit Points} 105 (10d10 + 50)

\textbf{Movement} 12 m

\textbf{Saving Throws}: Fortitude +12, Reflexes +11, Will +12

\textbf{Damage Resistance}: Void

\textbf{Senses} Darkvision 18 m

\textbf{Languages} Common, Abyssal

\textbf{Challenge} 6 (2300 XP)

\textbf{Shares}

\emph{\textbf{Multiattack.}} Paladin Gablin attacks with 2 strikes with his bastard sword.

\emph{\textbf{Bastard Sword.} Melee weapon attack}: +13 to hit, reach 2 m, one target.

\emph{Hits:} 10 (1d10 + 5) bludgeoning damage, plus 1d6 void damage. If the affected creature is a Follower or Devotee of Gradh, the damage increases by an additional 1d6.

\emph{\textbf{Summon Gablin}}: 3 Actions. The Gablin spills his blood on the ground and 3d4 Gablin arise.

\textbf{Fiendish Aura}: The Gablin Paladin exudes a 20-foot-radius aura around him that grants +2 to attack rolls and damage rolls to all other Gablins and imposes -2 to attack rolls and saves on others creatures that are not Devotees or Followers of Cattalm.

\textbf{Ecology}\\
Environment: Any\\
Organization: Leading a Gablin army\\
\textbf{Treasure}: Standard (Field Armor, Bastard Sword)\\
\textbf{Description}\\
Gablin Paladins are among the most powerful Gablins known, the true chosen of Cattalm. Summoned by Cattalm's most powerful followers, they can alone lead hundreds of Gablin and thanks to their acumen prepare careful plans and bring havoc and destruction to entire regions.

\medskip\index[Monster]{Gargoyle}\textbf{Gargoyle}

\emph{Medium elemental, chaotic evil}

\textbf{STRENGTH} +2

\textbf{DEXTERITY} +0

\textbf{CONSTITUTION} +3

\textbf{INTELLIGENCE} -2

\textbf{WISDOM} +0

\textbf{CHARRISMA} -2

\textbf{Initiative} +0 -- \textbf{Defense} 16

\textbf{Hit Points} 52 (7d8 + 21)

\textbf{Movement} 9 m, flight 18 m

\textbf{Saving Throws}: Fortitude +4, Reflexes +6, Will +4

\textbf{Damage Resistances} from non-magical weapons or those that are not made of adamantium

\textbf{Damage Immunity} Poison

\textbf{Condition Immunity} poisoned, petrified, fatigued

\textbf{Senses} Darkvision 18 m

\textbf{Languages} Tremun

\textbf{Challenge} 2 (450 PX)

\emph{\textbf{False Appearance.}} While the gargoyle remains immobile, it is indistinguishable from an inanimate statue.

\emph{\textbf{Elemental Nature.}} A gargoyle has no need for air, food, drink, or sleep.

\textbf{Shares}

\emph{\textbf{Multiattack.}} The gargoyle makes two attacks: one with its bite and one with its claws.

\emph{\textbf{Claws.} Melee weapon attack}: +5 to hit, reach 1 m, one target.

\emph{Hits:} 5 (1d6 + 2) slashing damage, 1 bleed damage.

\emph{\textbf{Bite.} Melee weapon attack}: +5 to hit, reach 1 m, one target.

\emph{Hits:} 5 (1d6 + 2) piercing damage.

\textbf{Ecology}
Environment: Any\\
Organization: Solitary, pair or flock (3-12)\\
\textbf{Treasury}: Standard\\
\textbf{Description}\\
Gargoyles often appear to be winged stone statues, as they can remain still indefinitely and then surprise enemies. Gargoyles tend towards obsessive-compulsive behavior, as diverse as their species is abundant. Books, stolen trinkets, weapons, and trophies collected from fallen enemies are just a few examples of the types of items a gargoyle can collect to decorate its lair and territory.

Gargoyles tend to have a solitary lifestyle, although they sometimes form fearsome flocks called wings for protection and entertainment. Under certain conditions, a tribe of gargoyles can even ally themselves with other creatures, but even the most stable of these alliances can fall apart for the slightest of reasons; gargoyles are just treacherous, mean, and vindictive.

Gargoyles are known to dwell in the heart of larger cities, crouching among the stone decorations of cathedrals and buildings where they hide in plain sight by day, swooping down to feed on vagrants, beggars, and other unfortunates at night.

The longer a gargoyle tribe dwells in an area of ​​buildings or ruins, the more its members begin to resemble the architectural style of the area. The changes a gargoyle's appearance undergoes are slow and subtle, but over the years they can become dramatic.

An unusual variant of the gargoyle does not live among buildings and ruins but under the waves of the sea. These creatures are known as kapoacinth; They have the same base stats as normal gargoyles, except they have the aquatic subtype and their wings grant them a swimming speed of 40 feet (but are useless for flying). Kapoacinths inhabit shallow coastal regions where they can crawl out of the surf to hunt local residents. They are more likely to form flocks, as kapoacinth prefer group life to solitary life.

\medskip\index[Monstruary]{G.E.C.}\textbf{G.E.C.}

\emph{Great Aberration, Chaotic Evil}

\textbf{STRENGTH} +6

\textbf{DEXTERITY} +1

\textbf{CONSTITUTION} +5

\textbf{INTELLIGENCE} +1

\textbf{WISDOM} +1

\textbf{CHARISMA} +1

\textbf{Initiative} +2 -- \textbf{Defense} 20 (chitin)

\textbf{Hit Points} 95 (12d8 + 50)

\textbf{Move} 9 m, dig 9 m

\textbf{Saving Throws}: Fortitude +15, Reflexes +11, Will +12

\textbf{Resistance} +4 on saving throws to spells from the Charm and Illusion List

\textbf{Skills} Awareness +10

\textbf{Senses} Darkvision 60 ft., telluric sense 60 ft.

\textbf{Languages} -

\textbf{Challenge} 10 (5900 PX)

\textbf{Shares}

\emph{\textbf{Multi-attack.}} The G.E.C. it can attack with two claws or with its bite

\textbf{Claws}: Melee natural weapon attack: +21 to hit, reach 10 ft., one target.

\emph{Hits:} 15 (3d6 + 5) slashing damage, 1 bleed damage.

\textbf{Bite}: Melee natural weapon attack: +21 to hit, reach 10 ft., one target

\emph{Hits:} 16 (3d8 + 5) slashing damage, 1 bleed damage, Blurred Vision.

\textbf{Blurred Vision:} is a poison effect, Will save DC 18 or until the end of the next round the target has -1d6 on attack rolls.

\emph{\textbf{Look.}} It is enough to look at the G.E.C. to be affected by Confusion, as a spell of the same name. To resist you must make a Will save at DC 15. Each round you can repeat the saving throw to resist the effect.

Fighting without looking at the G.E.C. imposes -1d6 on attack rolls.

\emph{\textbf{Angry:}} the G.E.C. lets out a cacophonous roar. Creatures within 10 feet of him must make a Will save at DC 20 or be affected by confusion for 2 rounds. It costs 2 Actions.

\textbf{Ecology}\\
Environment: Underground\\
Organization: solitary, group (2-4) \\
\textbf{Honey}: Accidental\\
\textbf{Description}\\
The Great Chitinous Being, or G.E.C, is an insect with a vague humanoid appearance, almost 3 meters tall, powerful and equipped with two very strong and resistant claws capable of digging and shearing any material. 4 small, central and multi-faceted eyes give off a dim, iridescent luminescence that confuses creatures that meet their gaze.

Probably the result of some transformation spell gone wrong, the G.E.C. they are masters of the underground. Creatures with real intelligence love elven meat and fight tactically and wisely.

\subsection{Geni}

\medskip\index[Monstery]{Djinni}\textbf{Djinni}

\emph{Large elemental, chaotic good}

\textbf{STRENGTH} +5

\textbf{DEXTERITY} +2

\textbf{CONSTITUTION} +6

\textbf{INTELLIGENCE} +2

\textbf{WISDOM} +3

\textbf{CHARISMA} +5

\textbf{Initiative} +2 -- \textbf{Defense} 23

\textbf{Hit Points} 161 (14d10 + 84)

\textbf{Movement} 9 m, flight 27 m

\textbf{Saving Throws} Fortitude +4, Reflexes +9, Will +7

\textbf{Damage Immunity} Electricity, sound

\textbf{Senses} darkvision 36 m

\textbf{Languages} Ictun

\textbf{Challenge} 11 (7200 PX)

€20315 € {€20316 € {Elemental Death.}} If the djinni dies, his body disintegrates in a hot breeze, leaving behind only the equipment the djinni was wearing or carrying.

€20317 € {€20318 € {Innate Spells.}} The djinni's innate spellcasting characteristic is Charisma 17. He can innately cast the following spells, without the need for material components:

At will: \emph{detect good and evil, detect magic, thunder wave}

3/day each: \emph{walk on the wind, create food and water} (can create wine instead of water), \emph{languages}

1/day each: \emph{crafting}, \emph{summon elementals} (air elemental only), \emph{gaseous form, major image}, €20325{invisibility,} €20326{displacement planar}

\textbf{Shares}

\emph{\textbf{Multiattack.}} The djinni makes three attacks of
scimitar.

\emph{\textbf{Scimitar.} Melee weapon attack}: +15 to hit, reach 1 m, one target.

\emph{Hits:} 12 (2d6 + 5) slashing damage plus 3 (1d6) lightning or sonic damage (gin's choice).

\emph{\textbf{Create Whirlwind.}} A swirling cylinder of air 3 feet in radius and 30 feet tall magically forms at a point visible to the djinni within 120 feet of it. The whirlwind remains as long as the djinni maintains concentration (as if concentrating on a spell). Any creature other than the djinni that enters the whirlwind must succeed on a DC 18 Fortitude saving throw or be restrained by it. The djinni can move the whirlwind up to 60 feet with an action, and creatures entangled by the whirlwind move with it. The whirlwind ends if the djinni loses sight of it.

A creature can use its action to free a creature entangled from the whirlwind, including itself, with a successful DC 18 Strength check. If the check succeeds, the creature is no longer entangled and moves to the closest space outside the whirlwind. turbines.

\textbf{Ecology}
Environment: Any (Plane of Air)\\
Organization: Solitary, couple, company (3-6) or band (7-10)\\
\textbf{Treasure}: Standard (Masterwork Scimitar, more treasure)\\
\textbf{Description}\\
Djinn (singular djinni) are Genies from the Elemental Plane of Air. They are said to be made of clouds and have the strength of the most powerful storms. A Djinni is about 3 meters tall and weighs about 500 kg.

Djinn disdain physical combat, preferring to use their magical powers and aerial abilities against enemies. A Djinni defeated in Combat generally takes flight and becomes a whirlwind to harass those pursuing it. When faced with no choice but to fight in melee, most Djinn prefer to wield Masterwork Two-Handed Scimitars.

Towards other Djinns, Djinn get along well with Janni and Marid. They are frequently at odds with the Shaitans, and are sworn enemies of the Efreeti, despising these ferocious genies more than any other of the genie races. The conflict between the Efreeti and the Djinn is so legendary that many spellcasters attempt (with varying degrees of success) to secure the service of a Djinni by promising aid in their cause against their hated enemies.


\medskip\index[Monstery]{Efreeti}\textbf{Efreeti}

\emph{Great elemental, lawful evil}

\textbf{STRENGTH} +6

\textbf{DEXTERITY} +1

\textbf{CONSTITUTION} +7

\textbf{INTELLIGENCE} +3

\textbf{WISDOM} +2

\textbf{CHARRISMA} +3

\textbf{Initiative} +3 -- \textbf{Defense} 23

\textbf{Hit Points} 200 (16d10 + 112)

\textbf{Movement} 12 m, flight 18 m

\textbf{Saving Throws} Fortitude +7, Reflexes +10, Will +9

\textbf{Damage Immunity} Fire

\textbf{Senses} darkvision 36 m

\textbf{Languages} Ignan

\textbf{Challenge} 11 (7200 PX)

€20357 € {€20358 € {Elemental Death.}} If the efreeti dies, its body disintegrates in a flash of fire and a puff of smoke, leaving behind only the equipment the efreeti was wearing or transporting.

\emph{\textbf{Innate Spells.}} The efreeti's innate spellcasting ability is Charisma. He can innately cast the following spells, without the need for material components:

All you want: \emph{detect the magic}

3/day each: \emph{enlarge/reduce, languages}

1/day each: \emph{summon elementals} (fire elemental only), \emph{gas form, greater image}, €20365{invisibility, wall of fire, planar shift}

\textbf{Shares}

\emph{\textbf{Multiattack.}} The efreeti makes two scimitar attacks or uses Fling Flame twice.

\emph{\textbf{Scimitar.} Melee weapon attack}: +21 to hit, reach 1 m, one target.

\emph{Hits:} 13 (2d6 + 6) slashing damage plus 7 (2d6) fire damage.

\emph{\textbf{Throw Flame.} Ranged Weapon Attack}: +16 to hit, range 36 m, one target.

\emph{Hits:} 17 (5d6) fire damage.

\textbf{Ecology}
Environment: Any (Plane of Fire)\\
Organization: Solitary, couple, company (3-6) or band (7-12)\\
\textbf{Treasure}: Standard (Perfect Glaive, more treasure)\\
\textbf{Description}\\
The Efreet (singular Efreeti) are Genies from the Plane of Fire. They are 3.6 meters tall and weigh around 1000 kg.

The Efreet have few allies among the other Genii: they hate the Djinni, and attack them on sight, they cannot stand the Marid, and see the Janni as weak and fragile. The Efreet often cooperate well with the Shaitans, yet even these alliances are temporary.


\subsection{Ghoul}

\medskip\index[Monstery]{Ghast}\textbf{Ghast}

\emph{Medium undead, chaotic evil}

\textbf{STRENGTH} +3

\textbf{DEXTERITY} +3

\textbf{CONSTITUTION} +0

\textbf{INTELLIGENCE} +0

\textbf{WISDOM} +0

\textbf{CHARRISMA} -1

\textbf{Initiative} +3 -- \textbf{Defense} 14

\textbf{Hit Points} 36 (8d8)

\textbf{Movement} 9 m

\textbf{Saving Throws}: Fortitude +2, Reflexes +2, Will +5

\textbf{Damage Resistances} from Void

\textbf{Damage Immunity} Poison

\textbf{Condition Immunity} charmed, poisoned, fatigued

\textbf{Senses} Darkvision 18 m

\textbf{Languages} Municipality, Exspiran

\textbf{Challenge} 2 (450 PX)

\emph{\textbf{Stench.}} Any creature that begins its round within 3 feet of the ghast must succeed on a DC 12 Fortitude saving throw or be nauseated until the start of its next round. On a successful save, the creature is immune to the stench of the ghast for the next 24
hours.

\emph{\textbf{Turn Rebellion.}} The ghast and all ghouls within 30 feet of it have +1d6 on saving throws against effects that turn undead.

\textbf{Shares}

\emph{\textbf{Claws.} Melee weapon attack}: +6 to hit, reach 1 m, one target.

\emph{Hits:} 10 (2d6 + 3) slashing damage. If the target is a creature, other than an undead, it must succeed on a DC 12 Fortitude save or be paralyzed for 1 minute. The target can repeat the saving throw at the end of each of its rounds, ending the effect on a successful save.

\emph{\textbf{Bite.} Melee weapon attack}: +6 to hit, reach 3 ft., one creature.

\emph{Hits:} 12 (2d8 + 3) piercing damage.

\textbf{Ecology}\\
Environment: Any terrain\\
Organization: Solitary, group (2-4) or pack (7-12)\\
\textbf{Treasury}: Standard\\
\textbf{Description}\\
Ghasts are Ghouls with a deeper connection to the void. A ghast's paralysis also affects Elves. Ghasts roam in packs or command groups of common ghouls. The stench of death and decay surrounding these creatures is overwhelming.


\medskip\index[Monster]{Ghoul}\textbf{Ghoul}

\emph{Medium undead, chaotic evil}

\textbf{STRENGTH} +1

\textbf{DEXTERITY} +2

\textbf{CONSTITUTION} +0

\textbf{INTELLIGENCE} -2

\textbf{WISDOM} +0

\textbf{CHARRISMA} -2

\textbf{Initiative} +2 -- \textbf{Defense} 13

\textbf{Hit Points} 22 (5d8)

\textbf{Movement} 9 m

\textbf{Saving Throws}: Fortitude +1, Reflexes +2, Will +4

\textbf{Damage Immunity} Poison

\textbf{Condition Immunity} charmed, poisoned, fatigued

\textbf{Senses} Darkvision 18 m

\textbf{Languages} Municipality

\textbf{Challenge} 1 (200 PX)

\textbf{Shares}

\emph{\textbf{Claws.} Melee weapon attack}: +4 to hit, reach 1 m, one target.

\emph{Hits:} 7 (2d4 + 2) slashing damage, 1 bleed damage. If the target is a creature, other than an elf or undead, it must succeed on a DC 12 Fortitude save or be paralyzed for 1 minute. The target can repeat the saving throw at the end of each of its rounds, ending the effect on a successful save.

\emph{\textbf{Bite.} Melee weapon attack}: +4 to hit, reach 3 ft., one creature.

\emph{Hits:} 9 (2d6 + 2) piercing damage.

\textbf{Ecology}
Environment: Any terrain\\
Organization: Solitary, group (2-4) or pack (7-12)\\
\textbf{Treasury}: Standard\\
\textbf{Description}\\
Ghouls are undead who frequent graveyards and eat corpses. Legends hold that the first ghouls were cannibalistic humans brought back from the dead by an unnatural hunger, or humans who fed on the decaying remains of their peers in life and who died (and then were reborn) from a horrendous disease; the true origin of these undead scavengers is uncertain.

Ghouls lurk on the fringes of civilization (in or near cemeteries or city sewers) where they can find ample supplies of their favorite food. Although they prefer rotting bodies and often bury their victims to improve their flavor, they will eat the dead fresh if they are hungry enough.


While many surface ghouls live primitively, rumors speak of ghoul cities deep underground commanded by priests who worship cruel ancient gods or strange lords of hunger demons. These \emph{civilized} ghouls are no less hideous in their eating habits, and indeed their concept of a well-laid banquet table is perhaps even more hideous than the idea of ​​a fresh meal taken from a coffin.

\medskip\index[Monster]{Ghoul, Black}\textbf{Ghoul, Black}

\emph{Medium undead, chaotic evil}

\textbf{STRENGTH} +4

\textbf{DEXTERITY} +2

\textbf{CONSTITUTION} +2

\textbf{INTELLIGENCE} +0

\textbf{WISDOM} +1

\textbf{CHARRISMA} -2

\textbf{Initiative} +2 -- \textbf{Defense} 19

\textbf{Hit Points} 105 (15d8+30)

\textbf{Movement} 12 m

\textbf{Saving Throws}: Fortitude +11, Reflexes +11, Will +8

\textbf{Damage Immunity} Poison, void, critical damage, bleeding, nonmagical or silver weapons

\textbf{Condition Immunity} charmed, poisoned, fatigue,

\textbf{Senses} Darkvision 18 m

\textbf{Languages} Municipality, Exspiran

\textbf{Challenge} 6 (2300 XP)

\textbf{\emph{Nefarious Aura}}: The Black Ghoul constantly emanates an aura around him that weakens the defenses of everyone except other ghouls. Every two rounds of remaining in the aura of 12 meters radius around the Black Ghoul adds a -1 to all saving throws, when you move away from the Black Ghoul you recover 1 point per round.

\textbf{Shares}

\emph{\textbf{Claws.} Melee weapon attack}: +12 to hit, reach 1 m, one target.

\emph{Hits:} 15 (2d10 + 4) slashing damage, 2 bleed damage. If the target is a creature, other than an elf or undead, it must succeed on a DC 16 Fortitude save or be paralyzed for 1 minute. The target can repeat the saving throw at the end of each of its rounds, ending the effect on a successful save.

\emph{\textbf{Bite.} Melee weapon attack}: +13 to hit, reach 3 ft., one creature.

\emph{Hits:} 18 (3d8 + 6) piercing damage, 1 from bleed, Ghoul Disease

\emph{Ghoul Disease:} 3 days, Fortitude save DC 18, 6 hours, 3 successes, you transform into a Ghoul

\textbf{Ecology}
Environment: Any terrain\\
Organization: Group (4-8) or pack (14-24)\\
\textbf{Treasure}: Standard\\
\textbf{Description}\\
The Black Ghoul represents one of the evolutionary elite of the Ghouls. Usually at the head of a group at least one rotting ghoul to around 18 ghouls.

\medskip\index[Monster]{Ghoul, Mother}\textbf{Ghoul, Mother}

\emph{Medium undead, chaotic evil}

\textbf{STRENGTH} +0

\textbf{DEXTERITY} +3

\textbf{CONSTITUTION} +2

\textbf{INTELLIGENCE} +2

\textbf{WISDOM} +1

\textbf{CHARISMA} +2

\textbf{Initiative} +3 -- \textbf{Defense} 21

\textbf{Hit Points} 90 (10d10+45)

\textbf{Movement} 9 m

\textbf{Saving Throws}: Fortitude +9, Reflexes +11, Will +9

\textbf{Damage Immunity} Poison, void, critical damage, bleeding, weapons +1

\textbf{Condition Immunity} charmed, poisoned, fatigued

\textbf{Senses} Darkvision 18 m

\textbf{Languages} Municipality, Exspiran

\textbf{Challenge} 5 (1800 PX)

\textbf{Shares}

\emph{\textbf{Claws.} Melee weapon attack}: +5 to hit, reach 1 m, one target.

\emph{Hits:} 12 (2d6 + 6) slashing damage, 2 bleed damage. If the target is a creature other than undead, she must succeed on a DC 15 Fortitude save or be paralyzed for 1 minute. The target can repeat the saving throw at the end of each of its rounds, ending the effect on a successful save. If the creature fails the save then it is the victim of the Ghoul's curse. Within 1d3+1 days he will transform into a Ghoul. A DC 19 Remove Curse is required within the transformation to avoid the transformation.

\emph{\textbf{Bite.} Melee weapon attack}: +6 to hit, reach 3 ft., one creature.

\emph{Hits:} 8 (2d6 + 2) piercing damage.

\textbf{Ecology}
Environment: Any terrain\\
Organization: Clan (7-12+)\\
\textbf{Treasury}: Standard\\
\textbf{Description}\\
The Mother Ghoul is usually the head of a clan of ghouls that can reach up to several dozen members. She is respected and feared, she is usually among the most intelligent evolved ghouls and highly appreciated for her ability to transform living people into ghouls. Their tactic involves wounding and not killing several people so that when they return home and then transformed they can attack and kill the entire village.


\medskip\index[Monstery]{Ghoul, rotting}\textbf{Ghoul, rotting}

\emph{Great undead, chaotic evil}

\textbf{STRENGTH} +1

\textbf{DEXTERITY} +2

\textbf{CONSTITUTION} +3

\textbf{INTELLIGENCE} -1

\textbf{WISDOM} +0

\textbf{CHARRISMA} -2

\textbf{Initiative} +2 -- \textbf{Defense} 15

\textbf{Hit Points} 82 (12d10+12)

\textbf{Movement} 6 m

\textbf{Saving Throws}: Fortitude +7, Reflexes +5, Will +4

\textbf{Damage Immunity} Poison, bleeding, critical, Void, non-magical or silver weapons

\textbf{Condition Immunity} charmed, poisoned, fatigued

\textbf{Senses} darkvision 36 m

\textbf{Languages} Municipality, Exspiran

\textbf{Challenge} 4 (110 PX)

\textbf{\emph{Regeneration}}. The rotting ghoul regenerates 5 hit points per round unless it is in full sunlight or took light damage in the previous round. If the Rotting Ghoul is in a graveyard it regains 10 hit points per round.

\textbf{Shares}

\emph{\textbf{Claws.} Melee weapon attack}: +5 to hit, reach 2 m, one target.

\emph{Hits:} 12 (2d10 + 2) slashing damage, 1 bleed damage. If the target is a creature, other than an undead, it must succeed on a DC 14 Fortitude save or be paralyzed for 1 minute.

\emph{\textbf{Bite.} Melee weapon attack}: +6 to hit, reach 3 ft., one creature.

\emph{Hits:} 10 (2d8 + 2) piercing damage.

\emph{\textbf{Aura of Suffering.}}: The Putrescent Ghoul exudes a 20-foot aura around him in which any successful attack automatically deals critical damage. Activating this aura costs 1 Action and lasts until the start of the next round.

\textbf{Ecology}
Environment: Any terrain\\
Organization: Group (4-8) or pack (10-18)\\
\textbf{Treasury}: Standard\\
\textbf{Description}\\
Rotting Ghouls are one of the many evolutions of Ghouls. Continuous contact with negative energy and feeding on corpses of all kinds for centuries have made him bigger, stronger and capable of inflicting and causing the most dangerous wounds to be inflicted.


\subsection{Giants}

\medskip\index[Monstery]{Hill Giant}\textbf{Hill Giant}

\emph{Huge giant, chaotic evil}

\textbf{STRENGTH} +5

\textbf{DEXTERITY} -1

\textbf{CONSTITUTION} +4

\textbf{INTELLIGENCE} -3

\textbf{WISDOM} -1

\textbf{CHARRISMA} -2

\textbf{Initiative} -1 -- \textbf{Defense} 16

\textbf{Hit Points} 105 (10d12 + 40)

\textbf{Movement} 12 m

\textbf{Saving Throws}: Fortitude +11, Reflexes +2, Will +3

\textbf{Skills} Awareness +2

\textbf{Languages} Giant

\textbf{Challenge} 5 (1800 PX)

\textbf{Shares}

\emph{\textbf{Multiattack.}} The giant makes two attacks with the heavy club.

\emph{\textbf{Heavy Club.} Melee weapon attack}: +12 to hit, reach 10 ft., one target.

\emph{Hits:} 18 (3d8 + 5) bludgeoning damage.

\emph{\textbf{Rock.} Ranged weapon attack}: +6 to hit, range 18m, one target.

\emph{Hits:} 21 (3d10 + 5) bludgeoning damage.

\textbf{Ecology}\\
Environment: Temperate Hills\\
Organization: Lone, group (2-5), warband (6-8), raiding party (9-12 plus 1d4 Dire Wolves) or tribe (13-30 plus 35% non-combatants plus 1 4th-level fighter leader 6th level, 11-16 dire wolves, 1-4 ogres and 13-20 orc slaves)\\
\textbf{Treasure}: Standard (Leather Armor, Heavy Club, other treasure)\\
\textbf{Description}\\
Hill giants have light brown to reddish skin, brown or black hair, and brown or black eyes. They wear layers of crudely tanned hides with fur still on them. They rarely wash or repair their clothing, and prefer to simply add new layers as the old ones wear out. Adults are about 3 meters tall and weigh more or less 550 kg. Hill giants can live to be 200 years old, although they rarely reach this age.

Hill giants prefer to fight from atop ledges and cliffs, where they can pel their opponents with rocks and boulders, thus limiting personal risk. They like to make overrun attacks against smaller creatures early in combat, and only then take a stance and begin swinging their massive clubs.

Hill giants are nomadic by nature, preferring to travel from place to place to raid and plunder. Although they like temperate climates more, they do not disdain traveling far from their favored environment if the raid is abundant and prosperous. They are, on the whole, very selfish creatures, who rarely face battles that they are not sure they will win. Hill giants are known for pushing each other when faced with formidable opponents, and are quick to sacrifice a mate to save their own skin. Roaming bands of hill giants are widespread across the temperate hills, and their constant aggression makes them one of the most feared dangers in this environment.


Solitary, non-evil hill giants are very rare, but they can sometimes be found in other humanoid societies, though they are almost never accepted into major cities or population centers. They are comfortable as workers and soldiers in remote frontier towns, and often serve as rudimentary diplomats to negotiate with raiding bands of Hill giants. Unfortunately, Hill Giants who abandon their racial way of life for civilization are mocked and often killed on sight by their nomadic brethren. However, these Hill Giants \emph{civilized} can find their place in society and many have managed to live a peaceful and tranquil existence.


\medskip\index[Monstery]{Fire Giant}\textbf{Fire Giant}

\emph{Huge giant, lawful evil}

\textbf{STRENGTH} +7

\textbf{DEXTERITY} -1

\textbf{CONSTITUTION} +6

\textbf{INTELLIGENCE} +0

\textbf{WISDOM} +2

\textbf{CHARRISMA} +1

\textbf{Initiative} +0 -- \textbf{Defense} 22 (plate armour)

\textbf{Hit Points} 162 (13d12 + 78)

\textbf{Movement} 9 m

\textbf{Saving Throws}: Fortitude +14, Reflexes +8, Will +10

\textbf{Skills} Acrobatics +11, Awareness +6

\textbf{Damage Immunity} Fire

\textbf{Languages} Giant

\textbf{Challenge} 9 (5000 XP)

\textbf{Shares}

\emph{\textbf{Multiattack.}} The giant makes two attacks with his greatsword.

\emph{\textbf{Broadsword.} Melee weapon attack}: +20 to hit, reach 10 ft., one target.

\emph{Hits:} 28 (6d6 + 7) slashing damage.

\emph{\textbf{Rock.} Ranged weapon attack}: +12 to hit, range 18m, one target.

\emph{Hits:} 29 (4d10 + 7) bludgeoning damage.

€20602 € {€20603 € {Angry:}} the fire giant channels his energy onto the weapon, which causes +2d6 fire damage.

\textbf{Ecology}
Environment: Warm mountains\\
Organization: Solo, group (2-5), warband (6-12 plus 35\% non-combatants and 1 1st-2nd level adept or devotee), raiding party (6-12 plus 1 adept or wizard of 3rd-5th level, 2-5 Hellhounds and 2-3 Trolls or Ettin) or tribe (20-30 plus 1 adept, wizard or devotee of 6th-7th level; 1 warrior king or ranger of 8th- 9th level; and 17-38 Hellhounds, 12-22 Trolls, 7-12 Ettins, and 1-2 Young Red Dragons)\\
\textbf{Treasure}: Standard (Half Armor, Greatsword, other treasure)\\
\textbf{Description}\\
Fire giants are the most rigid and martial of giants, always ready for war and to brutally treat anyone they encounter. Their rigid command structure requires soldiers, officers, and even generals, and that all obey their king's orders without question.

Fire giants have bright orange hair that glows and sparkles as if on fire. An adult male is between 3.6 and 4.8 meters tall, with a ribcage of about 2.7 meters, and weighs about 3,500 kg. Females are slightly shorter and slimmer. Fire giants can live up to 350 years.

Fire giants wear robes of sturdy fabric or leather that are orange, yellow, black, or red. The warriors wear helmets and half-armour of burnished steel and wield greatswords that swing across the battlefield. In large groups, fire giants fight with brutal and efficient group tactics, and do not hesitate to sacrifice a few comrades to ambush the enemy.

Fire giants prefer warm places: the warmer the better. They can be found in deserts, volcanoes, hot springs and deep in the earth near lava vents. They live in castles, fortified settlements, or large caves, and the architecture of these places reflects their rigid, militaristic lifestyle, with officers living in better quarters than their subordinates.



\medskip\index[Monster]{Frost Giant}\textbf{Frost Giant}

\emph{Huge giant, neutral evil}

\textbf{STRENGTH} +6

\textbf{DEXTERITY} -1

\textbf{CONSTITUTION} +5

\textbf{INTELLIGENCE} -1

\textbf{WISDOM} +0

\textbf{CHARRISMA} +1

\textbf{Initiative} -1 -- \textbf{Defense} 19 (composite armour)

\textbf{Hit Points} 138 (12d12 + 60)

\textbf{Movement} 12 m

\textbf{Saving Throws} Fortitude +14, Reflexes +3, Will +6

\textbf{Skills} Acrobatics +9, Awareness +3

\textbf{Damage Immunity} Cold

\textbf{Languages} Giant

\textbf{Challenge} 8 (3900 XP)

\textbf{Shares}

\emph{\textbf{Multiattack.}} The giant makes two attacks with its double axe.

\emph{\textbf{Double Axe.} Melee weapon attack}: +18 to hit, reach 10 ft., one target.

\emph{Hits:} 25 (3d12 + 6) slashing damage.

\emph{\textbf{Rock.} Ranged weapon attack}: +11 to hit, range 18m, one target.

\emph{Hits:} 28 (4d10 + 6) bludgeoning damage.

€20635 € {€20636 € {Angry:}} the Frost Giant channels his energies through the weapon. The weapon deals an additional 2d6 cold damage.

\textbf{Ecology}\\
Environment: Cold mountains\\
Organization: Solo, warband (3-5), party (6-12 plus 35\% non-combatants and 1 1st-2nd level wizard or Devotee), raiding party (6-12 plus 35\% non-combatants, 1 Devotee or wizard of 3rd-5th level, 1-4 Winter Wolves and 2-3 Ogres) or tribe (21-30 plus 1 adept, wizard or Devotee of 6th-7th level; 1 jarl Barbarian or ranger 7 -9th level; and 15-36 Winter Wolves, 13-22 Ogres and 1-2 Young White Dragons)\\
\textbf{Treasure}: Standard (Mail Jacket, Double Axe, other treasure)\\
\textbf{Description}\\
A frost giant has light blue or dirty yellow hair, and eyes that are typically the same color. They dress in skins and furs, adorning themselves with whatever jewels they possess. Frost giant fighters also wear mail jackets and metal helmets decorated with horns and feathers. An adult male is 5 meters tall and weighs around 1,400 kg. Females are slightly shorter and slender, but otherwise identical to males. Frost giants can live up to 250 years.

Frost giants are greatly feared, as their lust for destruction and war and their contemptuous behavior drive them to ever greater displays of brutality. Frost giants begin by attacking from a distance, hurling rocks until they run out of ammunition or their opponent approaches, then attack them with their enormous axes. A favorite tactic is to set up an ambush by hiding under the snow above an icy or snowy slope, where opponents will have difficulty reaching them, and then start by causing an avalanche before charging into battle. Frost giants can hide very well in snowy environments and are masters of stealth in their domain.

Frost giants survive by hunting and raiding alone, as they live in cold, desolate environments. Frost giant groups are almost evenly divided between those who live in makeshift settlements or abandoned castles and those who roam the frozen north as nomads in search of loot and supplies. Frost giant leaders are called jarls and demand absolute obedience from their followers. At any time a jarl may be challenged in combat for leadership of the tribe. These challenges typically end in the death of one of the contestants. A single jarl can often count on a dozen or more smaller tribes of frost giants as an extension of him. In these cases, the leaders of the smaller tribes are known as captains or warlords.

Frost giants love to take captives and use them as both slaves and raw material. Usually each group of frost giants keeps 1-2 humanoid slaves chained to a slave trainer: the meanest and cruelest of the group after the jarl. They also have a fondness for monstrous pets: White Dragons and Winter Wolves are popular choices, but Remorhaz and Yetis can also be found in a frost giant's lair.

\medskip\index[Monstery]{Cloud Giant}\textbf{Cloud Giant}

\emph{Huge giant, neutral good (50\%) or neutral evil (50\%)}

\textbf{STRENGTH} +8

\textbf{DEXTERITY} +0

\textbf{CONSTITUTION} +6

\textbf{INTELLIGENCE} +1

\textbf{WISDOM} +3

\textbf{CHARISMA} +3

\textbf{Initiative} +1 -- \textbf{Defense} 19

\textbf{Hit Points} 200 (16d12 + 96)

\textbf{Movement} 12 m

\textbf{Saving Throws} Fortitude +16, Reflexes +6, Will +10

\textbf{Skills} Sense Emotions +7, Awareness +7

\textbf{Languages} Common, Giant

\textbf{Challenge} 9 (5000 PX)

\emph{\textbf{Innate Spells.}} The giant's spellcasting ability is Charisma. The giant can cast these spells innately, without the need for material components:

At will: \emph{detect magic, light, fog cloud}

3/day each: \emph{Feather Drop, Veil Step, Telekinesis}

1/day each: \emph{check weather, gaseous form}

\emph{\textbf{Sense of Smell.}} The giant has +1d6 on Wisdom (Awareness) checks that rely on smell.

\textbf{Shares}

\emph{\textbf{Multiattack.}} The giant makes two attacks with the Spiked Mace.

\emph{\textbf{Spiked mace.} Melee weapon attack}: +22 to hit, reach 10 ft., one target.

\emph{Hits:} 21 (3d8 + 8) piercing damage.

\emph{\textbf{Rock.} Ranged weapon attack}: +14 to hit, range 18m, one target.

\emph{Hits:} 30 (4d10 + 8) bludgeoning damage.

\emph{\textbf{Angry:}} the Cloud Giant waves his weapon above his head, summoning storm clouds and casting the Call Lightning spell. He costs 2 Actions.

\textbf{Ecology}\\
Environment: Temperate Mountains\\
Organization: Solo, group (2-5), family (2-5 plus 35\% non-combatants plus 1 4th-7th level wizard or Devoted and 2-5 Griffins) or tribe (6-20 plus 1 oracle wizard or Devotee of 7th-12th level and 2-5 Griffins)\\
\textbf{Treasure}: Standard (Mail Jacket, Spiked Mace, more treasure)\\
\textbf{Description}\\
The skin color of cloud giants varies from milky white to dusty blue. Adult males are approximately 5.3 meters tall and weigh approximately 2,500 kg. Females are slightly shorter and slimmer. Cloud giants can live up to 400 years, they dress in precious clothes and jewels. For many, appearance indicates status. The better the clothes and the finer the jewelry, the more important the wearer. They also enjoy music, and most play one or more instruments (the harp is a favorite).

Cloud giants can have unusually varied Traits; about half are good and half evil. Good cloud giants build roads connecting their settlements with human roads to promote trade. It is not unusual to see a good cloud giant walking among humans, for example, in a human city near a high mountain range. Evil cloud giants tend not to create permanent settlements and instead prefer to live in crude shelters on high peaks, from which they descend only to rob villages of what they might need. These two philosophies often lead to the outbreak of violent and long-lasting wars between neighboring tribes.

There are many legends that speak of magical cities of cloud giants located among the clouds themselves, floating on the winds and circumnavigating the world. While cloud giants acknowledge that these are mostly fantasies, some claim to have seen them and have dedicated their entire existence to finding them.


\medskip\index[Monstery]{Stone Giant}\textbf{Stone Giant}

\emph{Huge giant, neutral}

\textbf{STRENGTH} +6

\textbf{DEXTERITY} +2

\textbf{CONSTITUTION} +5

\textbf{INTELLIGENCE} +0

\textbf{WISDOM} +1

\textbf{CHARRISMA} -1

\textbf{Initiative} +2 -- \textbf{Defense} 21

\textbf{Hit Points} 126 (11d12 + 55)

\textbf{Movement} 12 m

\textbf{Saving Throws} Fortitude +12, Reflexes +6, Will +7

\textbf{Skills} Acrobatics +12, Awareness +4

\textbf{Senses} Darkvision 18 m

\textbf{Languages} Giant

\textbf{Challenge} 7 (2900 XP)

\emph{\textbf{Stone Mimicry.}} The giant has +1d6 on Stealth (Hide) checks made to hide in rocky terrain.

\textbf{Shares}

\emph{\textbf{Multiattack.}} The giant makes two attacks with the heavy club.

\emph{\textbf{Heavy Club.} Melee weapon attack}: +19 to hit, reach 5 metres, one target.

\emph{Hits:} 19 (3d8 + 6) bludgeoning damage.

\emph{\textbf{Rock.} Ranged weapon attack}: +15 to hit, range 18m, one target.

\emph{Hits:} 28 (4d10 + 6) bludgeoning damage. If the target is a creature, it must succeed on a DC 17 Fortitude save or fall prone.

\textbf{Reactions}

\emph{\textbf{Grab Rock.}} If a rock or similar object is thrown at the giant, the giant can, with a successful DC 10 Reflex save, catch the projectile and take no bludgeoning damage from it.

\emph{\textbf{Angry:}} the Stone Giant concentrates his energies making his skin hard as stone. Until the end of the next round he gains a Damage Reduction of 13. It costs 2 Actions

\textbf{Ecology}
Environment: Temperate mountains\\
Organization: Lone, group (2-5), warband (4-8), hunting party (9-12 plus 1 Elder) or tribe (13-30 plus 35\% non-combatants, 1-3 Elders and 4-6 Cruel Bears)\\
\textbf{Treasure}: Standard (Heavy Club, more treasure)\\
\textbf{Description}\\
Stone giants prefer thick leather clothing, dyed shades of brown and gray to blend in with the stone around them. Adults are about 3.6 meters tall, weigh about 750 kg and can live up to 800 years.

Stone giants fight from a distance if possible, but if they cannot avoid melee they use giant stone clubs. One of the Stone giants' favorite tactics is to stand still, blending in with the landscape, and then advance by hurling rocks and surprise their enemies.

Stone giants prefer to live in huge caves on rocky peaks. They rarely live more than a few days' travel from other bands of Stone giants and raise shared herds of goats and other livestock.

Older Stone giants tend to wander away from the tribe for a long time, either living alone somewhere or trying to fit into other humanoid civilizations. After decades of self-imposed exile, those who return are known as Elder Rock Giants.


\medskip\index[Monstery]{Storm Giant}\textbf{Storm Giant}

\emph{Huge giant, chaotic good}

\textbf{STRENGTH} +9

\textbf{DEXTERITY} +2

\textbf{CONSTITUTION} +5

\textbf{INTELLIGENCE} +3

\textbf{WISDOM} +4

\textbf{CHARISMA} +4

\textbf{Initiative} +3 -- \textbf{Defense} 23 (scale armour)

\textbf{Hit Points} 230 (20d12 + 100)

\textbf{Movement} 15 m, swim 15 m

\textbf{Saving Throws} Fortitude +18, Reflexes +15, Will +17

\textbf{Skills} Arcana +8, Acrobatics +14, Awareness +9, History +8

\textbf{Damage Resistances} cold

\textbf{Damage Immunity} Electricity, sound

\textbf{Languages} Common, Giant

\textbf{Challenge} 13 (10000 PX)

\emph{\textbf{Amphibian.}} The giant can breathe air and water.

\emph{\textbf{Innate Spells.}} The giant's spellcasting ability is Charisma. The giant can cast these spells innately, without the need for material components:

At will: \emph{controlled fall, detect magic,} \emph{levitation, light}

3/day each: \emph{check weather, breathe} \emph{underwater}

\textbf{Shares}

\emph{\textbf{Multiattack.}} The giant makes two attacks with its greatsword.

\emph{\textbf{Broadsword.} Melee weapon attack}: +29 to hit, reach 10 ft., one target.

\emph{Hits:} 30 (6d6 + 9) slashing damage.

\emph{\textbf{Rock.} Ranged weapon attack}: +22 to hit, range 18m, one target.

\emph{Hits:} 35 (4d12 + 9) bludgeoning damage.

\emph{\textbf{Lightning Strike (Recharge 5-6).}} The giant hurls a magical bolt of lightning at a visible point within 150 meters of himself. Each creature within 10 feet of that point must make a DC 17 Reflex saving throw, taking 54 (12d8) lightning damage on a failed save, or half as much on a successful one.

\emph{\textbf{Angry:}} the storm giant charges the entire area around him with electricity until the end of the fight. A creature that ends the round within 20 feet of the giant takes 13 (3d8) electricity damage. Costs 1 Action.

\textbf{Ecology}\\
Environment: Any warm\\
Organization: Solo or family (2-5 plus 1 7th-10th level wizard or Devotee, 1-2 Rocs, 2-6 Griffins and 2-8 Sharks)\\
\textbf{Treasure}: Standard (Masterwork Plate Mail, Masterwork Composite Longbow [Strength +9] with 20 Arrows, Masterwork Greatsword, other treasure)\\
\textbf{Description}\\
Storm giants tend to have tan complexions, although rare specimens have purple skin, purple or dark blue hair, and silver-gray or purple eyes. The color purple is considered auspicious among storm giants, and those who possess it tend to become leaders among their people. Adults are normally 6.3 meters tall and weigh 6000 kg. Storm giants can live up to 600 years.

When at rest, they prefer to wear short tunics and wide belts at the hips, sandals or bare feet and a hair band. They wear a few simple but excellently made jewels, the most common being anklets (preferred by barefoot giants), rings or tiaras. But when they equip themselves for battle, they wear gleaming plate armor and wield enormous greatswords and bows.

Storm giants are generally solitary, preferring to inhabit remote coasts or tropical islands. As their name suggests, they are prone to violent mood swings. Storm giants are quick to anger in the face of evil and can be brutal and dangerous enemies when insulted. In battle, they prefer to hurl a hail of arrows at their enemies, drawing their greatswords only after their opponents get close.

Storm giants live in beautiful towers, castles, or walled settlements and love to cultivate the land. They have huge, well-tended gardens and manage hundreds of acres of crops per group. They often employ other humanoids, such as Elves or Humans, as support to run their immense farms. An enclave of storm giants often takes responsibility for the safety of an entire island or coastline.

\medskip\index[Monster]{Gnoll}\textbf{Gnoll}

\emph{Medium humanoid (gnoll), chaotic evil}

\textbf{STRENGTH} +2

\textbf{DEXTERITY} +1

\textbf{CONSTITUTION} +0

\textbf{INTELLIGENCE} -2

\textbf{WISDOM} +0

\textbf{CHARRISMA} -2

\textbf{Initiative} +1 -- \textbf{Defense} 16 (leather armour, shield)

\textbf{Hit Points} 22 (5d8)

\textbf{Movement} 9 m

\textbf{Saving Throws}: Fortitude +4, Reflexes +0, Will +0

\textbf{Senses} Darkvision 18 m

\textbf{Languages} Gnoll

\textbf{Challenge} 1/2 (100 PX)

€20,779 € {€20,780 € {Rage.}} When the gnoll reduces a creature to 0 hit points with a melee attack during its round, it can take a bonus action to move up to half its move and make a bite attack.

\textbf{Shares}

\emph{\textbf{Bite.} Melee weapon attack}: +4 to hit, reach 3 ft., one creature.

\emph{Hits:} 4 (1d4 + 2) piercing damage, Gnoll Rage Disease

\emph{Gnoll Rage:} 1 day, Fortitude save DC 13, 12 hour, 1 success, -2 Wisdom

\emph{\textbf{Spear.} Melee or Ranged Weapon Attack}: +4 to hit, reach 1 m or range 6 m, one target.

\emph{Hits:} 5 (1d6 + 2) piercing damage or 6 (1d8 + 2) piercing damage if used with two hands to make a melee attack.

\emph{\textbf{Longbow.} Ranged weapon attack}: +3 to hit, range 45m, one target.

\emph{Hits:} 5 (1d8 + 1) piercing damage.

\textbf{Ecology}\\
Environment: Hot plains, deserts\\
Organization: Solitary, couple, hunting group (2-5 and 1-2 Hyenas), band (10-100 adults plus 50\% small non-combatants, 1 3rd level sergeant for every 20 adults, 1 4th-level leader 6th level and 5-8 Hyenas) or tribes (20-200 plus 1 3rd level sergeant for every 20 adults, 1 or 2 4th or 5th level lieutenants, 1 6th-8th level leader, 7- 12 hyenas and 4-7 hyaenodonts)\\
\textbf{Treasure}: NPC equipment (Leather Armor, Heavy Wooden Shield, Spear, other treasure)\\
\textbf{Description}\\
Gnolls are a race of large, bulky humanoids who resemble hyenas in more than just appearance; they show an obvious affinity with these scavengers, so much so that they are kept as pets, and reflect many of the behaviors of these animals.

Gnolls are skilled hunters, but would much rather scavenge or steal a carcass than hunt prey. This laziness drives them to obtain slaves of any available species to force them to dig burrows, gather supplies and water, and even hunt for their gnoll masters.

Creatures other than hyenas or gnolls become food or slaves, depending on the tribe's temperament. Even a dead or fallen companion becomes a fresh meal for a gnoll, who can honor a famous member of the tribe with a short prayer or entirely cook one who has died of a devastating disease: otherwise, gnolls do not see a dead fellow member much differently than any other creature. More civilized gnolls do not eat their captives: instead, they keep them as slaves, to defend or improve their lair or to exchange them with other tribes or slave gangs.

Gnolls enjoy fighting, but only when they have superior numbers. In other situations, they prefer to avoid combat except as a means of obtaining a carcass from another hunter, or as an ingenious ambush to bring down a large meal. These hyena men see no value in courage or heroism and instead prefer to flee once it is clear that victory is unattainable, arguing that it is better to run away with your tail between your legs than lose it altogether.

During combat, gnolls use an odd combination of pack tactics and individual strategies. If a gnoll is confident of winning, he tries to overthrow the weaker opponent rather than help his companions. If the gnolls are in trouble, they gang up on a powerful opponent and attempt to eliminate him, hoping to force his allies to flee.

Gnoll leaders have forest ranger skills but it is not impossible to find gnolls devoted to some ravenous Patron. They rarely master magic effectively.


\medskip\index[Monstery]{Depth Gnome}\textbf{Depth Gnome}

\emph{Small humanoid (gnome), neutral good}

\textbf{STRENGTH} +2

\textbf{DEXTERITY} +2

\textbf{CONSTITUTION} +2

\textbf{INTELLIGENCE} +1

\textbf{WISDOM} +0

\textbf{CHARRISMA} -1

\textbf{Initiative} +2 -- \textbf{Defense} 16 (shirt jacket)

\textbf{Hit Points} 16 (3d6 + 6)

\textbf{Movement} 6 m

\textbf{Saving Throws}: Fortitude +6, Reflexes +6, Will +2

\textbf{Skills} Stealth +4, Awareness +2

\textbf{Senses} darkvision 36 m

\textbf{Languages} Gnomic, Depth Language, Tremun

\textbf{Challenge} 1/2 (100 PX)

\emph{\textbf{Gnome Cunning.}} The gnome has +1d6 on saving throws against magic.

\emph{\textbf{Stone Camouflage.}} The gnome has +1d6 on Stealth (Hide) checks made to hide in rocky terrain.

\emph{\textbf{Innate Spells.}} The gnome's innate spellcasting ability is Intelligence. The gnome can cast these spells innately, without the need for components:

At will: \emph{anti-detection} (personal)

1/day each: \emph{disguise self, blindness/deafness, blurriness}

\textbf{Shares}

\emph{\textbf{War Pickaxe.} Melee Weapon Attack}: +4 to hit, reach 1 m, one target.

\emph{Hits:} 6 (1d8 + 2) piercing damage.

\emph{\textbf{Poison Dart.} Ranged Weapon Attack}: +4 to hit, range 9m, one target.

\emph{Hits:} 4 (1d4 + 2) piercing damage, and the target must succeed on a DC 12 Fortitude save or be poisoned, -1 Strength and Dexterity, for 1 minute. The target can repeat the saving throw at the end of each of its rounds, ending the effect on itself on a success.

\textbf{Ecology}
Environment: Any dungeon\\
Organization: Solo, company (2-4), squad (5-20 plus 1 3rd-6th leader and two 3rd level sergeants), or band (30-50 plus 1 3rd level sergeant for every 20 adults, 5 5th level lieutenants, 3 7th level captains, and 2-5 Medium Earth Elementals)\\
\textbf{Treasure}: NPC equipment (Heavy Pickaxe, Light Crossbow with 10 Bolts, other treasure)\\
\textbf{Description}\\
Deep gnomes are a branch of the gnome race. They dwell underground, in hidden cities, safe from dark elves and other subterranean races. Their skin is the color of rock, usually gray or brown. Males are bald and females have sparse gray hair.

\medskip\index[Monstery]{Globule}\textbf{Globule}

\emph{Small aberration, evil}

\textbf{STRENGTH} -2

\textbf{DEXTERITY} +2

\textbf{CONSTITUTION} +0

\textbf{INTELLIGENCE} +3

\textbf{WISDOM} +1

\textbf{CHARISMA} +3

\textbf{Initiative} +3 -- \textbf{Defense} 15

\textbf{Hit Points} 30 (5d10 + 5)

\textbf{Movement} fly 18 m

\textbf{Saving Throws}: Fortitude +4, Reflexes +6, Will +5

\textbf{Skills} -

\textbf{Senses} darkvision 36 m

\textbf{Languages} understands the common language but does not speak it

\textbf{Challenge} 1 (200 PX)

\textbf{Vulnerability} Fire

\textbf{Immunity} Empty, Cold

\textbf{Condition Immunity} poison, prone

\textbf{I hate birds} the Globule has +1d6 to attack rolls against birds. Attack birds and flying creatures first

\textbf{Unusual nature} The Globule does not breathe

\textbf{I hate water} the Globule hates getting wet and for every 5 liters of water splashed on it it suffers 1d4 damage

\textbf{Shares}

\emph{\textbf{Tentacle}}. Melee attack, +5 to hit, reach 10 feet, one target

\emph{\textbf{Hits}} 5 (1d6+2) Void damage. The target must make a Fortitude save at DC 11 or increase its fatigue rating by 1.

\textbf{\emph{Shine}} Once per day the Orb becomes extremely bright, creatures within a 20-foot radius of it must make a Fortitude save at DC 13 or become blind for 3 rounds.

\textbf{Ecology}
Environment: Any, desert, nocturnal\\
Organization: Solitaire, 2d4 groups\\
\textbf{Treasure}: None\\
\textbf{Description}\\
The Globules are magical aberrations coming from some open portal to the Beyond. Creatures of cold and emptiness seem like small stars that only yearn to suck the life out of the creatures they encounter.
Intelligent and cunning, they prefer to attack by remaining in the air and weakening the opponent until he is mortally tired. Once a Globule has been killed, all that remains is a small star-shaped creature with a large central eye, completely white.


\medskip\index[Monstery]{Goblin}\textbf{Goblin}

\emph{Small humanoid (goblinoid), chaotic evil}

\textbf{STRENGTH} +0

\textbf{DEXTERITY} +0

\textbf{CONSTITUTION} +1

\textbf{INTELLIGENCE} -1

\textbf{WISDOM} -2

\textbf{CHARRISMA} -1

\textbf{Initiative} +0 -- \textbf{Defense} 13

\textbf{Hit Points} 7 (2d6 + 1)

\textbf{Movement} 9 m

\textbf{Saving Throws}: Fortitude +1, Reflexes +1, Will -1

\textbf{Senses} Darkvision 18 m

\textbf{Languages} Common, Goblin

\textbf{Challenge} 1/4 (50 PX)

\textbf{Shares}

\emph{\textbf{Short Sword.} Melee weapon attack}: +1 to hit, reach 1 m, one target.

\emph{Hits:} 4 (1d6 + 1) slashing damage

\emph{\textbf{Short Bow.} Ranged Weapon Attack}: +1 to hit, range 15m, one target.

\emph{Hits:} 3 (1d6) piercing damage.

\textbf{Ecology}\\
Environment: Any Temperate\\
Organization: Group (4-9), warband (10-24), or tribe (50+ plus 50\% non-combatants\\
\textbf{Treasure}: 1d4 silver coins\\
\textbf{Description}\\
Goblins are wild, unpredictable, noisy.
Goblins prefer to live in caves, deep forests, and abandoned ancient structures when available. Goblins do not like to build but rather to destroy and then complain that there is nothing useful.

Goblins are very superstitious, viewing magic with a mixture of awe and fear. Everything they don't understand is magic to them and this leads them to be extremely suspicious of everything and to destroy everything, since what they don't understand must be destroyed.

Goblins are ravenous and can eat enormous quantities of food. a goblin doesn't give up eating anything except maybe salad..


\subsection{Golem}

\medskip\index[Monstery]{Clay Golem}\textbf{Clay Golem}

\emph{Large construct, misaligned}

\textbf{STRENGTH} +5

\textbf{DEXTERITY} -1

\textbf{CONSTITUTION} +4

\textbf{INTELLIGENCE} -4

\textbf{WISDOM} -1

\textbf{CHARISMA} -5

\textbf{Initiative} -1 -- \textbf{Defense} 19

\textbf{Hit Points} 133 (14d10 + 56)

\textbf{Movement} 6 m

\textbf{Saving Throws}: Fortitude +4, Reflexes +3, Will +4

\textbf{Damage Immunity} acid, poison; from a non-magical weapon or that is not made of adamantium

\textbf{Condition Immunity} charmed, poisoned, paralyzed, petrified, fatigued, frightened

\textbf{Senses} Darkvision 18 m

\textbf{Languages} understands the languages ​​of its creator but cannot speak

\textbf{Challenge} 9 (5000 PX)

€20916 € {€20917 € {Berserk.}} Whenever the golem begins its round with 60 hit points or less, it rolls a d6. If you roll a 6, the golem goes berserk. During each of its rounds while berserk, the golem attacks the closest creature it can see. If there is no creature close enough to move and attack, the golem attacks an object, preferring objects smaller than itself. Once the golem has gone berserk, it will continue to be berserk until it is destroyed or all of its hit points are regained.

\emph{\textbf{Magical Weapons.}} The golem's weapon attacks are magical.

€20920 € {€20921 € {Acid Absorption.}} Whenever the golem suffers acid damage, it takes no damage but instead regains an equal number of Hit Points.

€20922 € {€20923 € {Unchangeable Form.}} The golem is immune to any spell or effect that would alter its form.

\emph{\textbf{Construct Nature.}} A golem has no need for air, food, drink, or sleep.

\emph{\textbf{Magic Resistance.}} The golem has +1d6 on saving throws against spells and other magical effects.

\textbf{Shares}

\emph{\textbf{Multiattack.}} The golem makes two slam attacks or a single cursed fist attack

\emph{\textbf{Slam.} Melee weapon attack}: +18 to hit, reach 1 m, one target.

\emph{Hits:} 16 (2d10 + 5) bludgeoning damage.

\emph{\textbf{Cursed Fist.}: Natural weapon attack}: +20 to hit, reach 1 m, one target

\emph{Hits:} 16 (2d6 + 5) bludgeoning damage. Cursed Fist wounds heal at the rate of 1 Hit Point per day. Magical cures, spells or potions, heal 1 Hit Point per cure die + any fixed amount.

\emph{\textbf{Speed ​​(Recharge 5-6).}} Until the end of its next round, the golem gains a +2 magical bonus to Defense, has +1d6 on Reflex saving throws, and can use slam attacks as a bonus action.

\textbf{Ecology}\\
Environment: Any\\
Organization: Solo or group (2-4)\\
\textbf{Treasure}: None\\
\textbf{Description}\\
Clay golems wear no clothing, except for a garment of treated leather or metal around the hips. On average they are more than 2.3 meters tall and weigh 300 kilos.

\textbf{Construction}
A clay golem can be sculpted from a single block of clay weighing at least 500 pounds, treated with rare powders and oils worth 1,500 gp.


\medskip\index[Monstery]{Flesh Golem}\textbf{Flesh Golem}

\emph{Average construct, neutral}

\textbf{STRENGTH} +4

\textbf{DEXTERITY} -1

\textbf{CONSTITUTION} +4

\textbf{INTELLIGENCE} -2

\textbf{WISDOM} +0

\textbf{CHARISMA} -3

\textbf{Initiative} -1 -- \textbf{Defense} 12

\textbf{Hit Points} 93 (11d8 + 44)

\textbf{Movement} 9 m

\textbf{Saving Throws}: Fortitude +8, Reflexes +2, Will +3

\textbf{Damage Immunity} Electricity, poison; from a non-magical weapon or that is not made of adamantium

\textbf{Condition Immunity} charmed, poisoned, paralyzed, petrified, fatigued, frightened

\textbf{Senses} Darkvision 18 m

\textbf{Languages} understands its creator's languages ​​but cannot
speak

\textbf{Challenge} 5 (1800 PX)

\emph{\textbf{Berserk.}} Whenever the golem begins its round with 40 hit points or less, roll a d6. If you roll a 6, the golem goes berserk. During each of its rounds while berserk, the golem attacks the closest creature it can see. If there is no creature close enough to move and attack, the golem attacks an object, preferring objects smaller than itself. Once the golem has gone berserk, it will continue to be berserk until it is destroyed or regains all of its Hit Points.

\emph{\textbf{Magical Weapons.}} The golem's weapon attacks are magical.

€20967 € {€20968 € {Lightning Absorption.}} Whenever the golem suffers lightning damage, it takes no damage but instead regains an equal number of Hit Points.

€20969 € {€20970 € {Fire Aversion.}} If the golem takes fire damage, he has -1d6 on attack rolls and proficiency checks until the end of his next round.

€20971 € {€20972 € {Unchangeable Form.}} The golem is immune to any spell or effect that would alter its form.

\emph{\textbf{Construct Nature.}} A golem has no need for air, food, drink, or sleep.

\emph{\textbf{Magic Resistance.}} The golem has +1d6 on saving throws against spells and other magical effects.

\textbf{Shares}

\emph{\textbf{Multiattack.}} The golem makes two slam attacks.

\emph{\textbf{Slam.} Melee weapon attack}: +11 to hit, reach 1 m, one target.

\emph{Hits:} 13 (2d8 + 4) bludgeoning damage. The affected creature must make a Fortitude save at DC 15 or become ill. Each time the saver fails, he takes one less action the following round. If it loses 3 actions, i.e. fails its saving throw 3 times in a row, the creature dies. As soon as the saving throw is successful, the disease is eradicated.

\emph{\textbf{Angry:}} the flesh golem overloads. For 2d4 rounds he can perform one additional Move or Attack Action. Costs 1 Action.

\textbf{Ecology}\\
Environment: Any\\
Organization: Solo or group (2-4)\\
\textbf{Treasure}: None\\
\textbf{Description}\\
A flesh golem is a monstrous collection of stolen humanoid anatomical parts sewn together. Its cadaverous flesh has a pale green or yellowish tone. A flesh golem wears any type of clothing its creator desires, normally just a worn pair of pants. He has no equipment or weapons. A flesh golem stands more than 8 feet tall and weighs 500 pounds.

A flesh golem does not speak, although it can make a hoarse growl. He walks and moves with a jerky gait, as if he doesn't have full control of his body.

While many flesh golems are devoid of reason, there are tales of exceptional golems that have somehow retained memories of their previous lives. The head (and therefore the brain) of these flesh golems must be the right combination of freshness and (in the previous life) decisiveness, but luck and chance also seem to be of absolute importance so that the intellect is preserved during their creation . Of course, those who build flesh golems prefer to have mindless slaves rather than ones with a will of their own, so intelligent flesh golems are rare.


\medskip\index[Monster]{Iron Golem}\textbf{Iron Golem}

\emph{Large construct, misaligned}

\textbf{STRENGTH} +7

\textbf{DEXTERITY} -1

\textbf{CONSTITUTION} +5

\textbf{INTELLIGENCE} -4

\textbf{WISDOM} +0

\textbf{CHARISMA} -5

\textbf{Initiative} -1 -- \textbf{Defense} 28

\textbf{Hit Points} 210 (20d10 + 100)

\textbf{Movement} 9 m

\textbf{Saving Throws}: Fortitude +21, Reflexes +15, Will +16

\textbf{Damage Immunity} Fire, poison; from a non-magical weapon or that is not made of adamantium

\textbf{Condition Immunity} charmed, poisoned, paralyzed, petrified, fatigued, frightened

\textbf{Senses} darkvision 36 m

\textbf{Languages} understands the languages ​​of its creator but cannot speak

\textbf{Challenge} 16 (15000 PX)

\emph{\textbf{Magical Weapons.}} The golem's weapon attacks are magical.

€21010 € {€21011 € {Fire Absorption.}} Whenever the golem takes fire damage, it takes no damage but instead regains an equal number of Hit Points.

\emph{\textbf{Unchangeable Form.}} The golem is immune to any spell or effect that would alter its form.

\emph{\textbf{Construct Nature.}} A golem has no need for air, food, drink, or sleep.

\emph{\textbf{Magic Resistance.}} The golem has +1d6 on saving throws against spells and other magical effects.

\textbf{Shares}

\emph{\textbf{Multiattack.}} The golem makes two melee attacks.

\emph{\textbf{Slam.} Melee weapon attack}: +30 to hit, reach 1 m, one target.

\emph{Hits:} 20 (3d8 + 7) bludgeoning damage.

\emph{\textbf{Sword.} Melee weapon attack}: +30 to hit, reach 10 ft., one target.

\emph{Hits:} 23 (3d10 + 7) slashing damage.

\emph{\textbf{Poisonous Breath (Recharge 6).}} The golem exhales poisonous gas in a 5 meter cone. Each creature in that area must make a DC 19 Fortitude saving throw, taking 45 (10d8) poison damage on a failed save, or half as much damage on a successful one.

\textbf{Ecology}\\
Environment: Any\\
Organization: Solo or group (2-4)\\
\textbf{Treasure}: None\\
\textbf{Description}\\
An iron golem has a humanoid-shaped body made of iron. The creator can give it any form he wishes, but it almost always features armor of some kind, be it ceremonial and precious or simple and utilitarian. Compared to a stone golem it has a much more defined appearance. Iron golems sometimes carry a weapon with them, though more often than not they tend to favor their slam attacks with it.

An iron golem is 3.6m tall and weighs around 2,500 kilos. An iron golem cannot speak or utter a voice. Furthermore, it does not emit any recognizable odor.

Although the practice of building iron golems gradually fell into disuse, venerable members of some great civilizations of the past considered the ability to forge iron golems of staggering strength and size to be a point of pride. These golems (of size greater than or equal to Huge), in some remote corners of the world, still exist, and still mechanically carry out orders given to them by long-gone empires.

\textbf{Construction}
To build an iron golem you need 5,000 pounds of iron, smelted with rare dyes worth at least 10,000 gp.


\medskip\index[Monstery]{Stone Golem}\textbf{Stone Golem}

\emph{Large construct, misaligned}

\textbf{STRENGTH} +6

\textbf{DEXTERITY} -1

\textbf{CONSTITUTION} +5

\textbf{INTELLIGENCE} -4

\textbf{WISDOM} +0

\textbf{CHARISMA} -5

\textbf{Initiative} -1 -- \textbf{Defense} 22

\textbf{Hit Points} 178 (17d10 + 85)

\textbf{Movement} 9 m

\textbf{Saving Throws}: Fortitude +4, Reflexes +3, Will +4

\textbf{Damage Immunity} Poison; from a non-magical weapon or that is not made of adamantium

\textbf{Condition Immunity} charmed, poisoned, paralyzed, petrified, fatigued, frightened

\textbf{Senses} darkvision 36 m

\textbf{Languages} understands the languages ​​of its creator but cannot speak

\textbf{Challenge} 10 (5900 PX)

\emph{\textbf{Magical Weapons.}} The golem's weapon attacks are magical.

\emph{\textbf{Unchangeable Form.}} The golem is immune to any spell or effect that would alter its form.

\emph{\textbf{Construct Nature.}} A golem has no need for air, food, drink, or sleep.

\emph{\textbf{Magic Resistance.}} The golem has +1d6 on saving throws against spells and other magical effects.

\textbf{Shares}

\emph{\textbf{Multiattack.}} The golem makes two slam attacks.

\emph{\textbf{Slam.} Melee weapon attack}: +21 to hit, reach 1 m, one target.

\emph{Hits:} 19 (3d8 + 6) bludgeoning damage.

\emph{\textbf{Slow (Recharge 5-6).}} The golem targets one or more creatures within 10 feet of it that it can see. Each target must make a DC 17 Will save against this spell. If the saving throw fails, the target can't use reactions, has its speed halved, and can't make more than one attack during its round. Additionally, during its round the target can take one action or a bonus action, but not both. These effects last for 1 minute. The target can repeat the saving throw at the end of each of its rounds, ending the effect for itself on a success.

\textbf{Ecology}\\
Environment: Any\\
Organization: Solo or group (2-4)\\
\textbf{Treasure}: None\\
\textbf{Description}\\
A stone golem has a humanoid body made of stone, often stylized to suit its creator. For example, he may be sculpted to wear armor, with particular symbols carved into his breastplate, or have designs inlaid into the stone of his limbs. The head is often carved to look like a helmet or the head of some beast. While he can be sculpted with a shield or a stone weapon such as a sword, these cosmetic choices do not affect his combat capabilities.

As with most golems, a stone golem cannot speak and makes no sound other than stone scraping against stone as it moves. A stone golem is 2.7 meters tall and weighs approximately 1000 kg.

There are numerous variations of Stone Golems, depending on the materials they are made of but also as expressions of elemental spirits, that is, it is possible for an elemental spirit to inhabit a rock (or gem) and define its appearance and animate it as its own body .

\textbf{Construction}
A stone golem's body is carved from a single block of hard stone, such as granite, weighing at least 3,000 pounds. The stone must be of exceptional quality, and cost 5,000 gp.


\medskip\index[Monster]{Gorgon}\textbf{Gorgon}

\emph{Large monstrosity, misaligned}

\textbf{STRENGTH} +5

\textbf{DEXTERITY} +0

\textbf{CONSTITUTION} +4

\textbf{INTELLIGENCE} -4

\textbf{WISDOM} +1

\textbf{CHARRISMA} -2

\textbf{Initiative} +0 -- \textbf{Defense} 22

\textbf{Hit Points} 114 (12d10 + 48)

\textbf{Movement} 12 m

\textbf{Saving Throws}: Fortitude +13, Reflexes +6, Will +7

\textbf{Skills} Awareness +4

\textbf{Condition Immunity} Petrified

\textbf{Senses} Darkvision 18 m

\textbf{Languages} -

\textbf{Challenge} 5 (1800 PX)

\emph{\textbf{Sweeping Charge.}} If the gorgon moves at least 20 feet in a straight line towards the target and hits it with a gore attack during the same round, the target must succeed on a Fortitude saving throw DC 16 or fall prone. If the target is prone, the gorgon can make a hoof attack against it as a bonus action.

\textbf{Shares}

\emph{\textbf{Gored.} Melee weapon attack}: +12 to hit, reach 1 m, one target.

\emph{Hits:} 18 (2d12 + 5) piercing damage.

\emph{\textbf{Hooves.} Melee weapon attack}: +12 to hit, reach 1 m, one target.

\emph{Hits:} 16 (2d10 + 5) bludgeoning damage.

\emph{\textbf{Petrifying Breath (Recharge 5-6).}} The gorgon exhales petrifying gas in a 30-foot cone. Each creature in that area must succeed on a DC 13 Fortitude save. If the save fails, the target begins to turn to stone and is restrained. The entangled target must repeat the saving throw at the end of its next round. On a success, the effect on the target ends. On a failed save, the target is petrified until freed by the \emph{greater restoration} spell or similar magic.

\emph{\textbf{Angry:}} the Gorgon concentrates all its petrifying venom in a single breath. She costs 2 shares. A creature within melee range must make a Fortitude save at DC 15 or turn to stone for 24 hours.

\textbf{Ecology}\\
Environment: Temperate Plains, Rocky Hills and Underground\\
Organization: Solitary, pair, pack (3-4) or herd (5-12)\\
\textbf{Treasure}: None\\
\textbf{Description}\\
Gorgons are magical and irascible creatures: although they may appear to be constructs at first glance, beneath their artificial-looking metal plates they are made of flesh and bone. Like aggressive bulls, they challenge any unknown creature they encounter, often running over their opponent's corpse or shattering its petrified remains until the creature is no longer recognizable. Females are just as dangerous as males, and both sexes look identical. A typical gorgon is 1.8 meters tall and 2.3 meters long. It weighs approximately 2000 kg.

Gorgons obtain their nourishment by consuming minerals, particularly the stone of their petrified victims, and any statue they create is completely devoured. They cannot digest metal or gems, so their dung (which resembles a sharp-smelling gray dust) often contains small rough crystals and gold nuggets. Their aggression towards all other creatures means that there are few, if any, predators and prey in their pastures. Each herd is led by a dominant bull; lone gorgons are generally adolescent bulls removed from the dominant bull's herd.

Their flesh is tough and muscular (once the armor is removed), and to those who taste it is quite nutritious. Many stone giant tribes believe that eating gorgon flesh increases their natural armor. Pulverized gorgon horns are worth 250 gp as an alternative material component for magic items and spells that affect Force or Stone.


\medskip\index[Monster]{Grick}\textbf{Grick}

\emph{Medium monstrosity, neutral}

\textbf{STRENGTH} +2

\textbf{DEXTERITY} +2

\textbf{CONSTITUTION} +0

\textbf{INTELLIGENCE} -4

\textbf{WISDOM} +2

\textbf{CHARISMA} -3

\textbf{Initiative} +2 -- \textbf{Defense} 15

\textbf{Hit Points} 27 (6d8)

\textbf{Movement} 9m, climb 9m

\textbf{Saving Throws}: Fortitude +3, Reflexes +3, Will +2

\textbf{Damage Resistance} from non-magical weapons

\textbf{Senses} Darkvision 18 m

\textbf{Languages} -

\textbf{Challenge} 2 (450 PX)

\emph{\textbf{Stone Camouflage.}} The grick has +1d6 on Stealth (Hide) checks made to hide in rocky terrain.

\textbf{Shares}

€21132 € {€21133 € {Multiattack.}} The grick makes an attack with his tentacles. If the attack hits, the grick can make a beak attack against the same target.

\emph{\textbf{Tentacles.} Melee weapon attack}: +4 to hit, reach 1 m, one target.

\emph{Hits:} 9 (2d6 + 2) slashing damage.

\emph{\textbf{Beak.} Melee weapon attack}: +4 to hit, reach 1 m, one target.

\emph{Hits:} 5 (1d6 + 2) piercing damage.

\textbf{Ecology}: \\
Environment: Any Dungeon\\
Organization: Solitary or cluster (2-5)\\
\textbf{Honey}: Accidental\\

\textbf{Description}
The worm-like grick is the terror of the caves and tunnels it inhabits, waiting in ambush near busy tunnels or underground cities, to leap out of the darkness and capture its prey. It is rare for such prey to be consumed on site. The grick prefers to bring fresh food to its den, a narrow tunnel or to the high ledge of a cave, where it can consume it in small bites, in peace.

The origins of the grick are unknown. And although it has rudimentary intelligence, it has no society to speak of, and most of the time single specimens are encountered. On the occasions when the unlucky travelers encounter more than one, groups of grick do not appear to communicate or work with each other: instead, each attacks individual targets and retreats with its loot as soon as it manages to take down an opponent. Capable predators, gricks also have strange, weapon-resistant skin that makes them particularly dangerous. Many inexperienced adventurers have perished under a grick's attack simply because they were unable to harm the creature with their nonmagical weapons. Those familiar with gricks (especially Dwarves, Morlocks, and Troglodytes) know that the best strategy for dealing with them is to retreat and wait for more powerful or magical reinforcements.

Gricks rely on their dark color and ability to scale walls to stay out of sight until they are ready to attack. On several occasions when food is scarce in a given region, gricks head to the surface and wander the desert in search of prey, but these sojourns are almost always out of necessity, and eventually gricks soon find entrances to new underground lairs . They prefer darkness and the comfort of a roof over their heads, avoiding the open sky and doing much to stay covered by trees, low clouds, or buildings.


\medskip\index[Monster]{Griffin}\textbf{Griffin}

\emph{Large monstrosity, misaligned}

\textbf{STRENGTH} +4

\textbf{DEXTERITY} +2

\textbf{CONSTITUTION} +3

\textbf{INTELLIGENCE} -4

\textbf{WISDOM} +1

\textbf{CHARRISMA} -1

\textbf{Initiative} +2 -- \textbf{Defense} 13

\textbf{Hit Points} 59 (7d10 + 21)

\textbf{Movement} 9 m, flight 24 m

\textbf{Saving Throws}: Fortitude +7, Reflexes +6, Will +4

\textbf{Skills} Awareness +5

\textbf{Senses} Darkvision 18 m

\textbf{Languages} -

\textbf{Challenge} 2 (450 PX)

\emph{\textbf{Honed Sight.}} The griffin has +1d6 on Wisdom (Awareness) checks based on sight.

\textbf{Shares}

\emph{\textbf{Multiattack.}} The griffin makes two attacks: one with its beak and one with its claws.

\emph{\textbf{Claws.} Melee weapon attack}: +7 to hit, reach 1 m, one target.

\emph{Hits:} 11 (2d6 + 4) slashing damage, 1 bleed damage.

\emph{\textbf{Beak.} Melee weapon attack}: +7 to hit, reach 1 m, one target.

\emph{Hits:} 8 (1d8 + 4) piercing damage.

\textbf{Ecology}\\
Environment: Temperate Hills\\
Organization: Solitary, pair or pack (6-10)\\
\textbf{Honey}: Accidental\\
\textbf{Description}\\
Griffon vultures are powerful aerial predators, swooping down from their lofty nests to seize their prey with their beaks and claws. Aggressive and territorial, they are no mere beasts, but cunning fighters and loyal companions to those who earn their respect, fighting to the death to protect their friends and kin.

Weighing over 250 kg and 2.3 meters long, from sharp beak to crested tail, the griffin possesses an imposing profile that has long been used in heraldry and other iconography as a symbol of power, authority and justice. In reality, the griffon is less interested in abstract concepts and more in hunting for food and defense. While they can sometimes be trained or befriended to serve as mounts, griffins do not possess an innate affinity with humanoids, and often engage in bloody conflicts with civilized races in an attempt to procure their favorite food: horse flesh. City folk may marvel at a trained griffon's majestic style and 25-foot wingspan, but those farmers forced to share territory with his kind know better to hurry home and secure their flocks when they hear the hunting cries of the beasts.

Griffins mate for life, and often for years seek revenge for the killing of their mate or child. It was precisely this innate stubbornness and fierce loyalty that brought them into domestic use as mounts and guardians of treasures. Despite the inherent danger, the trade in captured griffins and stolen eggs is brisk, with eggs worth up to 2,000 gp each and live youngsters worth up to 3,000. Characters who desire a griffin as a mount, however, should know that buying or violently taming intelligent creatures such as griffins is considered slavery by most good deities, and earning a griffin's willing loyalty is no easy task. Reaching a mutual agreement (or even friendship) is a much more elegant and safer way to secure a griffon as a mount.

Before it can be ridden into combat, a griffon must practice carrying its rider's weight. To be well trained, a griffon must first have a friendly attitude toward its handler (with a Handle Animal, Diplomacy, or Intimidate check). After that, 6 weeks of practice and a successful DC 20 Handle Animal check are enough for the beast to be comfortable with the load and, due to their intelligence, trained griffins can be assumed to know all the tricks listed in the description of the game. Animal Trainer skills, and it is also possible for them to learn new commands by giving simple requests in Common.

Griffins can carry up to 150 kg as a light load, 300 kg as a medium load and 450 kg as a heavy load. To ride a griffon you need an exotic saddle.


\medskip\index[Monstery]{Grimlock}\textbf{Grimlock}

\emph{Medium humanoid (grimlock), neutral evil}

\textbf{STRENGTH} +3

\textbf{DEXTERITY} +1

\textbf{CONSTITUTION} +1

\textbf{INTELLIGENCE} -1

\textbf{WISDOM} -1

\textbf{CHARRISMA} -2

\textbf{Initiative} +1 -- \textbf{Defense} 12

\textbf{Hit Points} 11 (2d8 + 2)

\textbf{Movement} 9 m

\textbf{Saving Throws}: Fortitude +3, Reflexes +1, Will +0

\textbf{Skills} Acrobatics +5, Stealth +3, Awareness +3

\textbf{Condition Immunity} blinded

\textbf{Senses} blindsight 30 feet or 10 feet if deafened (blind beyond this range)

\textbf{Languages} Language of the Depths

\textbf{Challenge} 1/4 (50 XP)

\emph{\textbf{Stone Camouflage.}} The grimlock has +1d6 on Stealth (Hide) checks made to hide in rocky terrain.

\emph{\textbf{Blind Senses.}} The grimlock cannot use blindsight while deafened and no longer scent.

\emph{\textbf{Sense of Smell and Hearing.}} The grimlock has +1d6 on Wisdom (Awareness) checks that rely on hearing or smell.

\textbf{Shares}

\emph{\textbf{Sharpened Bone Club.} Melee weapon attack}: +5 to hit, reach 1 m, one target.

\emph{Hits:} 5 (1d4 + 3) bludgeoning damage plus 2 (1d4) piercing damage.

\emph{\textbf{Longbow.} Ranged weapon attack}: +3 to hit, range 45m, one target.

\emph{Hits:} 5 (1d8 + 1) piercing damage.

\textbf{Ecology}\\
Grimlocks inhabit the abandoned settlements of other Races and are often found as Slaves to other, more organized creatures, such as Duergar and Elves. They are believed to be an even more degenerate offshoot of the Morlocks, who travel from Sekamina to hunt the Grimlocks for food and consider their flesh a delicacy.\\
\textbf{Description}\\
The Grimlocks are blind and savage human creatures who live in the realm of the dark lands of the depths, where they organize themselves into small tribal groups.

\medskip\index[Monstery]{Protector Guardian}\textbf{Protector Guardian}

\emph{Large construct, misaligned}

\textbf{STRENGTH} +4

\textbf{DEXTERITY} -1

\textbf{CONSTITUTION} +4

\textbf{INTELLIGENCE} -2

\textbf{WISDOM} +0

\textbf{CHARISMA} -4

\textbf{Initiative} -1 -- \textbf{Defense} 21

\textbf{Hit Points} 142 (15d10 + 60)

\textbf{Movement} 9 m

\textbf{Saving Throws}: Fortitude +6, Reflexes +1, Will +2

\textbf{Damage Immunity} Poison

\textbf{Condition Immunity} charmed, poisoned, paralyzed, fatigued, frightened

\textbf{Senses} Darkvision 60 ft., blindsight 10 ft

\textbf{Languages} understands commands given in any language but cannot speak

\textbf{Challenge} 7 (2900 XP)

\emph{\textbf{Accumulate Spells.}} A spellcaster wearing the guardian guardian's amulet can cause the guardian to accumulate a spell of level 4 or lower. To do so, the caster must cast the spell on the guardian. The spell has no effect but is stored within the guardian. When commanded to do so by the wearer of the amulet or a situation predetermined by the caster arises, the guardian casts the accumulated spell with all the parameters set by the original caster, without the need for components. When the spell is cast or any new spell is stacked, all previously stacked spells are lost.

\emph{\textbf{Construct Nature.}} The guardian has no need for air, food, drink, or sleep.

\emph{\textbf{Regeneration.}} The protective guardian recovers 10 Hit Points at the start of its round if it still has at least 1.

\emph{\textbf{Bound.}} The protective guardian is magically bound to an amulet. As long as the guardian and the amulet are on the same plane of existence, the wearer of the amulet can telepathically call upon the guardian to join him, and the guardian will know the distance and direction in which the amulet is located. If the guardian is within 60 feet of the wearer of the amulet, half the damage taken by the wearer (rounded down) is transferred to the guardian. If the amulet is destroyed, the guardian is incapacitated until a replacement amulet is created. The guardian's amulet can be subjected to direct attack if it is not worn or carried by anyone. It has Defense 10, 10 Hit Points and immunity to poison damage. Building an amulet takes 1 week and costs 10,000 gp in components.

\textbf{Shares}

\emph{\textbf{Multiattack.}} The golem makes two punch attacks.

\emph{\textbf{Fist.} Melee weapon attack}: +15 to hit, reach 1 m, one target.

\emph{Hits:} 11 (2d6 + 4) bludgeoning damage.

\textbf{Reactions}

\emph{\textbf{Shield.}} When a creature attacks the wearer of the guardian's amulet, the guardian grants a +2 bonus to its Defense if within 1 meter of its controller.

\medskip\index[Monstery]{Hobgoblin}\textbf{Hobgoblin}

\emph{Medium humanoid (goblinoid), lawful evil}

\textbf{STRENGTH} +1

\textbf{DEXTERITY} +1

\textbf{CONSTITUTION} +1

\textbf{INTELLIGENCE} +0

\textbf{WISDOM} +0

\textbf{CHARRISMA} -1

\textbf{Initiative} +1 -- \textbf{Defense} 19 (mail armour, shield)

\textbf{Hit Points} 11 (2d8 + 2)

\textbf{Movement} 9 m

\textbf{Saving Throws}: Fortitude +5, Reflexes +2, Will +1

\textbf{Senses} Darkvision 18 m

\textbf{Languages} Common, Goblin

\textbf{Challenge} 1/2 (100 PX)

\emph{\textbf{+1d6 Martial.}} Once per round, the hobgoblin can deal an additional 7 (2d6) damage to a creature it hits with a weapon attack if that creature is within 3 feet of an ally of the hobgoblin who is not incapacitated.

\textbf{Shares}

\emph{\textbf{Long Sword.} Melee weapon attack}: +3 to hit, reach 1 m, one target.

\emph{Hits:} 5 (1d8 + 1) slashing damage or 6 (1d10 + 1) slashing damage if used with two hands.

\emph{\textbf{Longbow.} Ranged weapon attack}: +3 to hit, range 45m, one target.

\emph{Hits:} 5 (1d8 + 1) piercing damage.

\textbf{Ecology}\\
Environment: Temperate Hills\\
Organization: Group (4-9), warband (10-24) or tribe (25+ plus 50\% non-combatants, 1 3rd level sergeant per 20 adults, 1 or 2 4th or 5th level lieutenants , 1 6th-8th level boss, 6-12 Leopards and 1-4 Ogres or 1-2 Trolls)\\
\textbf{Treasure}: NPC Equipment (Studded Leather Armor, Light Metal Shield, Long Sword, Longbow with 20 Arrows, other treasure)\\
\textbf{Description}\\
Hobgoblins are militaristic and prolific, a combination that makes them very dangerous in some regions. They procreate rapidly, replacing fallen members with new soldiers, keeping their numbers constant regardless of the fate of the war. Generally it doesn't take much for them to declare war, but in most cases the reason is to capture new Slaves: life as a Slave in a Hobgoblin lair is brutal and short, and new Slaves are always needed to replace those who die or are eaten .

Of all the Goblinoid Races, the Hobgoblin is by far the most civilized.
They see the larger, more reclusive Bugbears as tools to be hired and used where necessary, usually for specific missions requiring murder and theft, and they view the smaller Goblin species with a mixture of shame and frustration. Hobgoblins admire the tenacity of goblins, though their diminutive relatives' unpredictable nature and passion for fire makes them unwelcome additions to hobbgoblin tribes or settlements. However, most Hobgoblin tribes include a small group of Goblins, who normally hide in the worst corners of the settlement.

Many Hobgoblin tribes combine a love of war with a keen intellect. The science of siege engines, alchemy, and complex feats of engineering fascinate most hobgoblins, and those who are particularly gifted are treated like heroes and always achieve high-ranking positions in the tribe. Slaves with refined minds are valued, so raids on Dwarven cities are commonplace.

Hobgoblins are known to distrust and despise magic. Their Shamans are regarded with a mixture of fear and respect, and are normally forced to live alone on the fringes of the tribe's lair. Hobgoblins have never been heard of practicing Magic, or, as the Hobgoblins say, \emph{Elven Magic}. This is the cause of their hatred of Magic: Hobgoblins hate Elves.

A Hobgoblin stands 1 meter tall and weighs 80 kg.


\medskip\index[Monster]{Hydra}\textbf{Hydra}

\emph{Huge monstrosity, misaligned}

\textbf{STRENGTH} +5

\textbf{DEXTERITY} +1

\textbf{CONSTITUTION} +5

\textbf{INTELLIGENCE} -4

\textbf{WISDOM} +0

\textbf{CHARISMA} -2

\textbf{Initiative} +1 -- \textbf{Defense} 19

\textbf{Hit Points} 172 (15d12 + 75)

\textbf{Movement} 9m, swim 9m

\textbf{Saving Throws}: Fortitude +8, Reflexes +7, Will +3

\textbf{Skills} Awareness +6

\textbf{Senses} Darkvision 18 m

\textbf{Languages} -

\textbf{Challenge} 8 (3900 XP)

\emph{\textbf{Multiple Heads.}} The hydra has five heads. As long as it has more than one head, the hydra has +1d6 on saving throws against the blinded, charmed, deafened, frightened, stunned, or unconscious conditions.

Whenever the hydra takes 25 or more damage in a single round, one of its heads dies. If all the heads die, the hydra dies too.

At the end of its round, the hydra regrows two heads for each of its heads killed since its last round, unless it took fire damage from its last round. The hydra regains 10 hit points for each head regrown in this way.

\emph{\textbf{Reactive Heads.}} For each head possessed beyond the first, the hydra receives an extra Reaction Action which can only be used to make Awareness checks.

\emph{\textbf{Hold Your Breath.}} The hydra can hold its breath for 1 hour.

\emph{\textbf{Wake.}} While the hydra sleeps, at least one of its heads remains awake.

\textbf{Shares}

\emph{\textbf{Multiattack.}} The hydra makes as many bite attacks as it has heads.

\emph{\textbf{Bite.} Melee weapon attack}: +13 to hit, reach 10 ft., one target.

\emph{Hits:} 10 (1d10 + 5) piercing damage.

\textbf{Ecology}\\
Environment: Temperate Swamps\\
Organization: Solitaire\\
\textbf{Treasure}: Standard\\
\textbf{Description}\\
The hydra is a multi-headed, but stupid dragon.


\medskip\index[Monstery]{Hippogriff}\textbf{Hippogriff}

\emph{Great beast, misaligned}

\textbf{STRENGTH} +3

\textbf{DEXTERITY} +1

\textbf{CONSTITUTION} +1

\textbf{INTELLIGENCE} -4

\textbf{WISDOM} +1

\textbf{CHARRISMA} -1

\textbf{Initiative} +1 -- \textbf{Defense} 12

\textbf{Hit Points} 19 (3d10 + 3)

\textbf{Movement} 12 m, flight 18 m

\textbf{Saving Throws}: Fortitude +5, Reflexes +5, Will +2

\textbf{Skills} Awareness +5

\textbf{Languages} -

\textbf{Challenge} 1 (200 PX)

\emph{\textbf{Refined Sight.}} The hippogriff has +1d6 on Wisdom (Awareness) checks that rely on sight.

\textbf{Shares}

\emph{\textbf{Multiattack.}} The hippogriff makes two attacks: one with its beak and one with its claws.

\emph{\textbf{Claws.} Melee weapon attack}: +5 to hit, reach 1 m, one target.

\emph{Hits:} 10 (2d6 + 3) slashing damage.

\emph{\textbf{Beak.} Melee weapon attack}: +5 to hit, reach 1 m, one target.

\emph{Hits:} 8 (1d10 + 3) piercing damage.

\textbf{Ecology}\\
Environment: Temperate Hills or Plains\\
Organization: Solitary, pair or flock (7-12)\\
\textbf{Treasure}: None\\
\textbf{Description}\\
The hippogriff has the wings, front legs and head of a large bird of prey and the tail and body of a magnificent horse. Since horses are a favorite food of griffins, scholars say that a wizard with a sense of humor long ago created this unfortunate fusion between a horse and a falcon as a joke.

The hippogriff's feathers are similar in color to those of a hawk or an eagle; however, some breeders have managed to produce specimens with completely white or charcoal-colored feathers. A hippogriff's torso and hind limbs are most often bay, hazel, or gray in color, with some coats showing piebald or even palomino coloring. A hippogriff is 3.3 meters long and weighs up to 680 kg.

The territorial hippogriffs fiercely protect their domain. Hippogriffs must also guard the skies for other predators, as they are a favorite food of griffins, wyverns, and young dragons. Hippogriffs nest in vast grassy prairies, rugged hills, and flowing prairies. Exceptionally hardy hippogriffs make their homes within niches or canyon walls, from which they scour rocky deserts for coyotes, deer, and sometimes humanoids. Hippogriffs prefer mammals, however they will graze on grass after any meat meal to aid digestion. These dietary habits can be dangerous for both farmers and their herds, so ranching communities often place rewards on them. The victims of these hunting parties are often stuffed, and stuffed hippogriffs frequently decorate frontier taverns and remote outposts.

Far easier to train than griffins and as intelligent as horses, hippogriffs are trained as stud animals by select companies of mounted soldiers, who patrol the skies and swoop down on unwitting enemies. Although they are magical beasts, hippogriffs can be trained as if they were young if captured as animals. An adult hippogriff is much more difficult to train, and to do so one must follow the normal rules for training magical beasts using such an ability. A hippogriff saddle must be made in such a way as not to hinder the movements of the creature's wings; these saddles are always exotic saddles.

Hippogriffs are oviparous: as a general rule, a hippogriff's nest contains only one egg at a time. The hippogriff egg is worth 200 gp, but a healthy young hippogriff is worth 500 gp. A fully trained hippogriff as a mount can see its value rise to 5,000 gp or more. A hippogriff can carry 90 kg as a light load, 180 kg as a medium load, and 270 kg as a heavy load.


\medskip\index[Monstery]{Kraken}\textbf{Kraken}

\emph{Mammoth monstrosity (titan), chaotic evil}

\textbf{STRENGTH} +10

\textbf{DEXTERITY} +0

\textbf{CONSTITUTION} +7

\textbf{INTELLIGENCE} +6

\textbf{WISDOM} +4

\textbf{CHARISMA} +5

\textbf{Initiative} +6 -- \textbf{Defense} 30

\textbf{Hit Points} 472 (27x3d6 + 189)

\textbf{Movement} 6m, swim 18m

\textbf{Saving Throws}: Fortitude +30, Reflexes +23, Will +27

\textbf{Damage Immunity} Electricity, weapons +1

\textbf{Condition Immunity} paralyzed, scared

\textbf{Senses} true vision 36 m

\textbf{Languages} includes Abyssal, Celestial, Infernal and Druidic but cannot speak, telepathy 36 m

\textbf{Challenge} 23 (50000 PX)

\emph{\textbf{Amphibious.}} The kraken can breathe air and water.

\emph{\textbf{Freedom of Movement.}} The kraken ignores difficult terrain, and magical effects cannot reduce its speed or cause it to become entangled. It can expend 1 meter of movement to free itself from nonmagical restraints or from being grabbed.

\emph{\textbf{Siege Monster.}} The kraken deals double damage to objects and structures.

\textbf{Shares}

\emph{\textbf{Multiattack.}} The kraken makes three tentacle attacks, each of which can be replaced by a use of Slingshot.

\emph{\textbf{Bite.} Melee weapon attack}: +30 to hit, reach 20 ft., one target.

\emph{Hits:} 23 (3d8 + 10) piercing damage. If the target is a Large or smaller creature grabbed by the kraken, that creature is swallowed, and the grapple ends. While engulfed, the creature is blinded and restrained, has full cover against attacks and other effects from outside the kraken, and takes 42 (12d6) acid damage at the start of each of the kraken's rounds.

If the kraken takes 50 or more damage in a single round from a creature within it, the kraken must succeed on a DC 25 Fortitude saving throw or vomit all swallowed creatures, which fall prone in a space within 10 feet of the kraken. If the kraken dies, an engulfed creature is no longer restrained by it and can escape the corpse using 15 feet of movement, exiting prone.

\emph{\textbf{Tentacle.} Melee weapon attack}: +30 to hit, reach 30 ft., one target.

\emph{Hit:} 20 (3d6 + 10) bludgeoning damage, and the target is grappled (DC 18 to escape). Until the grapple ends, the target is entangled. The kraken has ten tentacles, each of which can grab a target.

\emph{\textbf{Sling.}} A held object or creature grabbed by the kraken, Large or smaller, is thrown 60 feet in a random direction and knocked prone. If the thrown target hits a solid surface, it takes 3 (1d6) bludgeoning damage for every 10 feet it travels. If the target is thrown at another creature, that creature must succeed on a DC 18 Reflex saving throw or take the same damage and be knocked prone.

\emph{\textbf{Lightning Storm.}} The kraken magically creates three bolts of energy, each of which can strike a target within 120 feet and that the kraken can see. The target must make a DC 23 Reflex saving throw, and take 22 (4d10) lightning damage on a failed save, or half as much on a successful one.

\textbf{Additional Shares}

The kraken can perform 3 additional Actions, chosen from the options below. It can use only one Additional option at a time, and only at the end of another creature's round. The kraken regains any additional Actions spent at the start of its round.

\textbf{Tentacle Attack or Slingshot.} The kraken makes a tentacle attack or uses Slingshot.

\textbf{Ink Cloud (Costs 3 Actions).} While underwater, the kraken expels a cloud of ink with a radius of 60 feet. The cloud spreads around the corners, and that area is heavily obscured for all creatures except the kraken. Each creature other than the kraken that ends its round in the area must succeed on a Fortitude save of 23, taking 16 (3d10) poison damage on a failed save, or half as much on a successful one. A strong current disperses the cloud, which otherwise vanishes at the end of the kraken's next round. \textbf{Lightning Storm (Costs 2 Actions).} The kraken uses Lightning Storm.

\textbf{Ecology}\\
Environment: Any Ocean\\
Organization: Solitaire\\
\textbf{Treasure}: Triple\\
\textbf{Description}\\
The legendary kraken is one of sailors' greatest fears, for it is a creature the size of a whale, can strike depths without being seen, can command the winds and weather conditions necessary for the ship to move, and possesses the cruel intellect of most of the most ruthless and creative criminals in the world. Some believe that krakens are divine punishment, while others believe them to be the true lords of the deep, who consider air-breathing races to be nothing more than livestock.

Many legends have arisen that he understands Druidic language.

A kraken is almost 30 meters long and weighs 2000 kg.


\medskip\index[Monster]{Lamia}\textbf{Lamia}

\emph{Great monstrosity, chaotic evil}

\textbf{STRENGTH} +3

\textbf{DEXTERITY} +1

\textbf{CONSTITUTION} +2

\textbf{INTELLIGENCE} +2

\textbf{WISDOM} +2

\textbf{CHARISMA} +3

\textbf{Initiative} +2 -- \textbf{Defense} 15

\textbf{Hit Points} 97 (13d10 + 26)

\textbf{Movement} 9 m

\textbf{Saving Throws}: Fortitude +6, Reflexes +9, Will +11

\textbf{Skills} Stealth +3, Deception +7, Sense Emotions +4,

\textbf{Senses} Darkvision 18 m

\textbf{Languages} Abyssal, Municipality

\textbf{Challenge} 4 (1100 PX)

\emph{\textbf{Innate Spells.}} The lamia's innate spellcasting characteristic is Charisma. The lamia can innately cast the following spells, without the need for material components:

At will: \emph{disguise self} (any humanoid form)\emph{,} \emph{larger image}

3/Day each: \emph{charm on people, mirror image,}

\emph{scrutinize, suggestion}

1/Day: \emph{restriction}

\textbf{Shares}

\emph{\textbf{Multiattack.}} The lamia makes two attacks: one with its claws and one with its dagger or Intoxicating Touch.

\emph{\textbf{Claws.} Melee weapon attack}: +9 to hit, reach 1 m, one target.

\emph{Hits:} 14 (2d10 + 3) slashing damage, 1 bleed damage.

\emph{\textbf{Dagger.} Melee weapon attack}: +9 to hit, reach 1 m, one target.

\emph{Hits:} 5 (1d4 + 3) piercing damage.

\emph{\textbf{Intoxicating Touch.} Melee spell attack}: +5 to hit, reach 3 ft., one creature.

\emph{Hits:} The target is cursed for 1 hour by this spell. Until the curse ends, the target has -1d6 on Will saving throws and all proficiency checks.

\textbf{Ecology}\\
Environment: Temperate Deserts\\
Organization: Solitary, couple or sect (3-12)\\
\textbf{Treasure}: Double (Dagger+1, more treasure)\\

\textbf{Description}\\
Hateful heirs of an ancient curse, lamias have the appearance of slender, attractive women from the waist up, while underneath they have the body of a mighty lion. Even their humanoid features bear distinctive feline traits, their eyes are narrow and feral and their teeth resemble the fangs of predators. A typical standing lamia stands 1.8 meters tall, is 2.3 meters long and weighs more than 325 kg.

Lamias are attracted to ruined towers, abandoned cities and forgotten monuments that satisfy the crude aesthetic standards of these lethal huntresses; especially those in arid or sterile areas. However, lamias favor decrepit temples. They take joy in seeing the temples of good deities in ruins and deviate from their path to endanger these thriving sacred places.

Lamias view the older females of their group as leaders, mothers, and shamans, clinging to them with fanatical reverence. While lamias shun most religions, seeing them as the source of the curse that has forced them into these bestial forms, elder lamias claim to hear the whispers of the wind that lashes the desert and to know the cold whims of the stars, and they rely on these mystical sources to guide their people.

The lamias presented here are only the most common and least powerful members of this cursed race; others have serpentine, flying and even more perverse shapes.


\medskip\index[Monstery]{Lich}\textbf{Lich}

\emph{Medium undead, evil traits}

\textbf{STRENGTH} +0

\textbf{DEXTERITY} +3

\textbf{CONSTITUTION} +3

\textbf{INTELLIGENCE} +5

\textbf{WISDOM} +2

\textbf{CHARISMA} +3

\textbf{Initiative} +5 -- \textbf{Defense} 28

\textbf{Hit Points} 135 (18d8 + 54)

\textbf{Movement} 9 m

\textbf{Saving Throws}: Fortitude +24, Reflexes +24, Will +23

\textbf{Damage Resistances} cold, lightning, void

\textbf{Damage Immunity} Poison; from a non-magical weapon

\textbf{Condition Immunity} charmed, poisoned, paralyzed, fatigued, frightened, bleeding

\textbf{Senses} true vision 36 m

\textbf{Languages} Common plus five other languages, Exspiran

\textbf{Challenge} 21 (33000 XP)

\emph{\textbf{Spells.}} The lich has CM 18. His spellcasting ability is Intelligence. The lich knows the following spells:

Cantrips (at will): €21458{magic hand, prestidigitation, ray} €21459{frost}

level 1 (4 slots): \emph{Arcane bolt, detect magic,} \emph{thunder wave, shield}

level 2 (3 slots): \emph{acid arrow, mirror image,} \emph{detect thoughts, invisibility}

level 3 (3 slots): \emph{animate dead, counterspell, dispel} \emph{spells, fireball}

level 4 (3 slots): \emph{wither, dimension door}

level 5 (3 slots): \emph{Deadly Mist, scry}

level 6 (1 slot): \emph{disintegration, orb of invulnerability}

level 7 (1 slot): \emph{finger of death, planar shift}

level 8 (1 slot): \emph{dominate monsters, power word stun}

level 9 (1 slot): \emph{kill word}

\emph{\textbf{Undead Nature.}} The lich has no need for air, food, drink, or sleep.

\emph{\textbf{Legendary Resistance (3/Day).}} If the lich fails a saving throw, he can choose to succeed instead.

\emph{\textbf{Turn Resistance.}} The lich has +1d6 on saving throws against effects that turn undead.

\emph{\textbf{Reinvigoration.}} If he possesses a phylactery, the destroyed lich gains a new body in 1d10 days, recovering all his Hit Points and returning to activity. The new body appears within 1 meter of the phylactery.

\emph{\textbf{Soul Sacrifices.}} A lich must periodically feed souls to his phylactery to sustain the magic that maintains his body and consciousness. To do this, use the \emph{imprison} spell. Instead of choosing one of the spell's normal options, the lich uses it to magically trap the target's body and soul within the phylactery. The phylactery must be on the same plane as the lich for this spell to work. A lich's phylactery can contain only one creature at a time, and \emph{dispel magic} cast as a 9th-level spell on the phylactery releases any creature imprisoned within it. A creature imprisoned in the phylactery for 24 hours is consumed and destroyed, after which nothing short of divine intervention can bring it back to life.

A lich that forgets or fails to maintain its body with sacrificed souls begins to fall apart, and may eventually transform into a demilich.

\textbf{Shares}

\emph{\textbf{Paralyzing Touch.} Melee spell attack}: +18 to hit, reach 3 ft., one creature.

\emph{Hits:} 10 (3d6) cold damage. The target must succeed on a DC 18 Fortitude save or be paralyzed for 1 minute. The target can repeat the saving throw at the end of each of its rounds, ending the effect on itself on a success.

\textbf{Additional Shares}

The lich can perform 3 additional Actions, chosen from the following options. It can use only one Additional option at a time, and only at the end of another creature's round. The lich regains any additional Actions spent at the start of its round.

\emph{\textbf{Destroy Life (Costs 3 Actions).}} Each creature except undead within 20 feet of the lich must make a DC 18 Fortitude saving throw against this spell, taking 21 (6d6) damage from Void on a failed save, or half as much damage on a successful one. Creatures become Fatigued.

\emph{\textbf{Frightening Gaze (Costs 2 Actions).}} The lich fixes its gaze on a visible creature within 10 feet of it. The target must succeed on a DC 18 Will save against this spell or be frightened for 1 minute. The frightened target can repeat the saving throw at the end of each of its rounds, ending the effect on itself on a success. If the target's saving throw succeeds or the effect ends for it, the target is immune to the lich's gaze for the next 24 hours.

\emph{\textbf{Paralyzing Touch (Costs 2 Actions).}} The lich uses his Paralyzing Touch.

\emph{\textbf{Cantrip.}} The lich casts a cantrip.

\textbf{Ecology}\\
Environment: Any\\
Organization: Solitaire\\
\textbf{Treasure}: NPC Equipment (Ring of Protection +2, Sash of Lore +2 (Awareness), Boots of Levitation, scroll of Dominate Person, scroll of Teleportation, potion of Invisibility)\\

\textbf{Description}
Few creatures are more feared than liches. The pinnacle of the Necromantic arts, the lich is a spellcaster who has chosen to give up life and cheat death by becoming undead. While many who reach such heights of power would do anything to achieve immortality, the idea of ​​becoming a lich is abhorred by many creatures. The process involves extracting the caster's life force and imprisoning it in a specially prepared phylactery; the caster surrenders his life, but remains trapped between life and death, and as long as his phylactery remains intact he can continue his research and work without fearing the passage of time.



\medskip\index[Monster]{Lizardfolk}\textbf{Lizardfolk}

\emph{Medium humanoid (lizardfolk), neutral}

\textbf{STRENGTH} +2

\textbf{DEXTERITY} +0

\textbf{CONSTITUTION} +1

\textbf{INTELLIGENCE} -2

\textbf{WISDOM} +1

\textbf{CHARISMA} -2

\textbf{Initiative} +0 -- \textbf{Defense} 16 (natural armor, shield)

\textbf{Hit Points} 22 (4d8 + 4)

\textbf{Movement} 9m, swim 9m

\textbf{Saving Throws}: Fortitude +4, Reflexes +0, Will +0

\textbf{Skills} Stealth +4, Awareness +3, Survival +5

\textbf{Languages} Draconic

\textbf{Challenge} 1/2 (100 PX)

\emph{\textbf{Hold Your Breath.}} The lizardfolk can hold its breath for 15 minutes.

\textbf{Shares}

\emph{\textbf{Multiattack.}} The lizardfolk makes two melee attacks, each with a different weapon.

\emph{\textbf{Javelin.} Melee or Ranged Weapon Attack}: +4 to hit, reach 1m or range 12m, one target. \emph{Hits:} 5 (1d6 + 2) piercing damage.

\emph{\textbf{Bite.} Melee weapon attack}: +4 to hit, reach 1 m, one target.

\emph{Hits:} 5 (1d6 + 2) piercing damage.

\emph{\textbf{Heavy Club.} Melee weapon attack}: +4 to hit, reach 1 m, one target.

\emph{Hits:} 5 (1d6 + 2) bludgeoning damage.

\emph{\textbf{Spiked Shield.} Melee weapon attack}: +4 to hit, reach 1 m, one target.

\emph{Hits:} 5 (1d6 + 2) piercing damage.

\textbf{Ecology}\\
Environment: temperate swamps\\
Organization: solitaire, couple, gang (3-12) or tribe (13-60)\\
\textbf{Treasure}: NPC Equipment (Heavy Wooden Shield, Spiked Mace, 3 Javelins)\\

\textbf{Description}\\
Lizardfolk are proud and powerful predatory reptiles that make their communal homes in sparse villages in the recesses of swamps and marshes. Uninterested in colonizing the arid lands and content with the simple weapons and rituals that have served them well for millennia, lizardfolk are seen by many of the other races as backward savages, but within their isolated communities they are in reality a vital people rich in tradition and with an oral history that dates back to before man walked upright.

Most lizardfolk stand 6 to 7 feet tall and weigh 250 to 300 pounds, and have powerful muscles covered in gray, green or brown scales. Some rays have small dorsal crests or brightly colored collars, and all swim well by moving with rapid movements of their powerful 1.2 meter long tail. Even though they are fully at home in the water, they hold their breath and return to their homes on artificial hills to breed and sleep. Because their reptilian blood makes them slow in the cold, many lizardfolk hunt and work during the day and retreat to their homes at night to huddle with others in their tribe to share the warmth of great peat fires.

While they are generally neutral, lizardfolk's aloof behavior, staunch rejection of the gifts of civilization, and legendary ferocity in battle make them misjudged by most humanoids. These traits come for good reason, however, as their low reproductive rate is unmatched among warm-blooded humanoids, and if the tribes did not defend their swampy territories to their last breath they would soon find themselves overwhelmed by hordes of mammals. As for their propensity to eat the bodies of the dead both friend and foe, the practical lizardfolk are quick to point out that life is hard in the swamp, and nothing should go to waste.

The lizardfolk presented here live in marshy environments. Lizardfolk tribes can live just as well in other environments, but they gain Climb 15 feet instead of Swim for speed.


\medskip\index[Monster]{Cursed Immortal}\textbf{Cursed Immortal}

\emph{Medium aberration (human), basically crazy}

\textbf{STRENGTH} +3

\textbf{DEXTERITY} +1

\textbf{CONSTITUTION} +2

\textbf{INTELLIGENCE} -1

\textbf{WISDOM} +1

\textbf{CHARISMA} -2

\textbf{Initiative} +3 -- \textbf{Defense} 15

\textbf{Hit Points} 75 (12d8 + 21)

\textbf{Movement} 9 m

\textbf{Saving Throws} Fortitude +6, Reflexes +5, Will +5

\textbf{Magic Resistance} The Cursed Immortal has +1d6 on all spell saving throws

€21555 € {Skills} Awareness +3, profession he had in life

\textbf{Damage Immunity} cold, fire, void

\textbf{Condition Immunity} charmed, poisoned, petrified, frightened

\textbf{Unconscious} The Cursed One has no sense of taste or smell

\textbf{Languages} Common, Dwarven, Elven

\textbf{Challenge} 4 (1100 PX)

\emph{\textbf{Immortal}} The Cursed Immortal regenerates 1 hit point per round, allowing him to regenerate limbs and return to life. The only way to kill him is by dissolving him in magical acid. Remove Curse at DC 30 kills him instantly.

\emph{\textbf{Different nature}} The Accursed Immortal does not eat, drink, sleep or age. He is not undead

\textbf{Shares}

\emph{\textbf{Multiattack.}} The Cursed Immortal makes three longsword attacks.

\emph{\textbf{Sword.} Melee weapon attack}: +6 to hit, reach 1 m, one target.

\emph{Hits:} 12 (1d10 + 7) slashing damage.

\textbf{Ecology}\\
Environment: Any\\
Organization: Solitaire\\
\textbf{Treasure}: NPC Equipment (Studded Leather Armor, 2 Daggers, Sword, more Soro)\\
\textbf{Description}\\
The Cursed Immortal is a person often cursed by a Patron or a powerful spellcaster with the curse of crazy immortal life. The curse breaks the person's balance and he finds himself wandering without a goal or objective. Every now and then they remember who they were and then continue searching for whoever cursed them.
With the aim of getting himself killed once and for all, he throws himself into every fight hoping that his opponent will be able to kill him once and for all.


\subsection{Werewolf}

\medskip\index[Monster]{Wereboar}\textbf{Wereboar}

\emph{Medium humanoid (human, shapeshifter), neutral evil}

\textbf{STRENGTH} +3

\textbf{DEXTERITY} +0

\textbf{CONSTITUTION} +2

\textbf{INTELLIGENCE} +0

\textbf{WISDOM} +0

\textbf{CHARRISMA} -1

\textbf{Initiative} +0 -- \textbf{Defense} 12 in humanoid form, 13 in boar or hybrid form

\textbf{Hit Points} 78 (12d8 + 24)

\textbf{Movement} 9 m (12 m in boar form)

\textbf{Saving Throws}: Fortitude +7, Reflexes +1, Will +4

\textbf{Skills} Awareness +2

\textbf{Damage Immunity} from non-magical or non-silvered weapons

\textbf{Languages} Common (cannot speak in boar form)

\textbf{Challenge} 4 (1100 PX)

\emph{\textbf{Charge (Boar or Hybrid Form Only).}} If the wereboar moves in a straight line at least 5 meters towards a target and then strikes it with its tusks during the same round, the target takes an additional 7 (2d6) slashing damage. If the target is a creature, it must succeed on a DC 13 Fortitude save or fall prone.

\emph{\textbf{Relentless (Recharge after 1 hour).}} If the wereboar takes 14 or fewer points of damage that would reduce it to 0 hit points, it instead drops to 1 hit point.

\emph{\textbf{Shapeshifting.}} The wereboar can use its action to transform into a boar-humanoid hybrid or a boar, or to return to its true form, which is humanoid. His stats, aside from Defense, are the same across all forms. Whatever equipment he is wearing or carrying is not transformed. Upon death he returns to his true form.

\textbf{Shares}

\emph{\textbf{Multiattack (Humanoid or Hybrid Form Only).}} The wereboar makes two attacks, of which only one can be with its tusks.

\emph{\textbf{Maul (Humanoid or Hybrid Form Only).} Melee Weapon Attack}: +9 to hit, reach 1 m, one target. \emph{Hits:} 10 (2d6 + 3) bludgeoning damage.

\emph{\textbf{Tusks (Boar or Hybrid Form Only).} Melee Weapon Attack}: +9 to hit, reach 1 m, one target. \emph{Hits:} 10 (2d6 + 3) slashing damage. If the target is a humanoid, it must succeed on a DC 12 Fortitude save or be cursed by the wereboar's lycanthropy.

\textbf{Ecology}\\
Environment: Any Forest or Plains\\
Organization: Solo, couple, family (3-8) or troop (3-8 plus 1-4 Boars)\\
\textbf{Treasure}: NPC Equipment (Studded Leather Armor, 2 Daggers, other treasure)\\
\textbf{Description}\\
In their humanoid form, wereboars tend to be stocky, with upturned noses, shaggy fur, and prominent incisors. They have red, brown or black hair but some are also blond, white or bald. They typically have hair on their upper lip, and males usually cannot grow beards. Because they are stubborn and aggressive, they have small communities of their own and do not mix with non-werewolves: they usually live on small, perfectly normal-looking farms. They tend to have large families and many children.


\medskip\index[Monster]{Werewolf}\textbf{Werewolf}

\emph{Medium humanoid (human, shapeshifter), chaotic evil}

\textbf{STRENGTH} +2

\textbf{DEXTERITY} +1

\textbf{CONSTITUTION} +2

\textbf{INTELLIGENCE} +0

\textbf{WISDOM} +0

\textbf{CHARISMA} +0

\textbf{Initiative} +1 -- \textbf{Defense} 13 in humanoid form, 14 in wolf or hybrid form

\textbf{Hit Points} 58 (9d8 + 18)

\textbf{Movement} 9 m (12 m in wolf form)

\textbf{Saving Throws}: Fortitude +5, Reflexes +1, Will +2

\textbf{Skills} Stealth +3, Awareness +4

\textbf{Damage Immunity} from non-magical or non-silvered weapons

\textbf{Languages} Common (cannot speak in wolf form)

\textbf{Challenge} 3 (700 PX)

\emph{\textbf{Shapeshifting.}} The werewolf can use its action to transform into a wolf-humanoid hybrid or a wolf, or to return to its true form, which is humanoid. His stats, aside from Defense, are the same across all forms. Whatever equipment he is wearing or carrying is not transformed. Upon death he returns to his true form.

\emph{\textbf{Refined Hearing and Smell.}} The werewolf has +1d6 on Wisdom (Awareness) checks that rely on hearing or smell.

\textbf{Shares}

\emph{\textbf{Multiattack (Humanoid or Hybrid Form Only).}} The werewolf makes two attacks: one with its bite and one with its claws or spear.

\emph{\textbf{Claws (Hybrid Form only).} Melee weapon attack}: +6 to hit, reach 3 ft., one creature. \emph{Hits:} 7 (2d4 + 2) slashing damage.

\emph{\textbf{Spear (Humanoid Form Only).} Melee or Ranged Weapon Attack}: +4 to hit, reach 3 ft. or range 20 ft., one creature.

\emph{Hits:} 5 (1d6 + 2) piercing damage or 6 (1d8 + 2) piercing damage if used with two hands in a melee attack.

\emph{\textbf{Bite (Wolf or Hybrid Form Only).} Melee Weapon Attack}: +6 to hit, reach 1 m, one target.

\emph{Hits:} 6 (1d8 + 2) piercing damage. If the target is a humanoid, it must succeed on a DC 12 Fortitude save or be cursed by the werewolf's lycanthropy.

\textbf{Ecology}\\
Environment: Any Terrain\\
Organization: Solitary, pair or pack (3-6)\\
\textbf{Treasure}: NPC Equipment (Chain Mail, Long Sword, Light Crossbow with 20 Bolts, other treasure)\\
\textbf{Description}\\
In human form, werewolves resemble normal people, although some tend to have a feral appearance and unruly hair. Eyebrows that grow together, an index finger that is longer than the middle finger, and strange birthmarks on the palm of the hand are all commonly accepted signs that a person is actually a werewolf. Of course, these telltale signs aren't always accurate, because these physical traits exist in normal people too, but in areas where werewolves are a common problem, these traits can be considered damning regardless.

\medskip\index[Monster]{Werebear}\textbf{Werebear}

\emph{Medium humanoid (human, shapeshifter), neutral good}

\textbf{STRENGTH} +4

\textbf{DEXTERITY} +0

\textbf{CONSTITUTION} +3

\textbf{INTELLIGENCE} +0

\textbf{WISDOM} +1

\textbf{CHARISMA} +1

\textbf{Initiative} +0 -- \textbf{Defense} 13 in humanoid form, 14

in bear or hybrid form

\textbf{Hit Points} 135 (18d8 + 54)

\textbf{Move} 9m (12m, climb 9m in bear form or hybrid form)

\textbf{Saving Throws}: Fortitude +5, Reflexes +6, Will +2

\textbf{Skills} Awareness +7

\textbf{Damage Immunity} from non-magical or non-silvered weapons

\textbf{Languages} Common (cannot speak in bear form)

\textbf{Challenge} 5 (1800 PX)

\emph{\textbf{Shapeshifting.}} The werebear can use its action to transform into a bear-humanoid hybrid or a bear, or to return to its true form, which is humanoid. His stats, aside from Defense, are the same across all forms. Whatever equipment he is wearing or carrying is not transformed. Upon death he returns to his true form.

\emph{\textbf{Sense of Smell.}} The werebear has +1d6 on Wisdom (Awareness) checks that rely on smell.

\textbf{Shares}

\emph{\textbf{Multiattack.}} In bear form, the werebear makes two claw attacks. In humanoid form, it makes two double ax attacks. In hybrid form, he can attack as a bear or a humanoid.

\emph{\textbf{Claw (Bear or Hybrid Form Only).} Melee Weapon Attack}: +11 to hit, reach 1 m, one target. \emph{Hits:} 13 (2d8 + 2) slashing damage.

\emph{\textbf{Double-headed Ax (Humanoid or Hybrid Form Only).} Melee Weapon Attack}: +11 to hit, reach 1 m, one target. \emph{Hits:} 10 (1d12 + 4) slashing damage.

\emph{\textbf{Bite (Bear or Hybrid Form Only).} Melee Weapon Attack}: +11 to hit, reach 1 m, one target.

\emph{Hits:} 15 (2d10 + 4) piercing damage. If the target is a humanoid, it must succeed on a DC 14 Fortitude save or be cursed by the werebear's lycanthropy.


\textbf{Ecology}\\
Environment: Any Forest\\
Organization: Solo, couple, family (3-6) or troop (3-6 plus 1-4 black or gray bears)\\
\textbf{Treasure}: NPC Equipment (Mail Jacket, Masterwork Battle Axe, 2 Masterwork Throwing Axes, more treasure)\\
\textbf{Description}\\
In their humanoid forms, werebears tend to be muscular and have broad shoulders, rugged features, and dark eyes. They have red, brown or black hair and seem accustomed to a life of hard work. Although the most benign of werewolves, they are shunned by most normal people, who fear their animalistic transformation. For the most part they live in isolated forested areas or in small family units of their own species. They avoid confronting outsiders, but do not hesitate if they must drive evil humanoids from their territories.

\medskip\index[Monster]{Wererat}\textbf{Wererat}

\emph{Medium humanoid (human, shapeshifter), lawful evil}

\textbf{STRENGTH} +0

\textbf{DEXTERITY} +2

\textbf{CONSTITUTION} +1

\textbf{INTELLIGENCE} +0

\textbf{WISDOM} +0

\textbf{CHARISMA} -1

\textbf{Initiative} +2 -- \textbf{Defense} 13

\textbf{Hit Points} 33 (6d8 + 6)

\textbf{Movement} 9 m

\textbf{Saving Throws}: Fortitude +2, Reflexes +5, Will +3

\textbf{Skills} Stealth +4, Awareness +2

\textbf{Damage Immunity} from non-magical or non-silvered weapons

\textbf{Senses} Darkvision 60 ft. (in rat form only)

\textbf{Languages} Common (cannot speak in rat form)

\textbf{Challenge} 2 (450 PX)

\emph{\textbf{Shapeshifting.}} The wererat can use its action to transform into a rat-humanoid hybrid or a rat, or to return to its true form, which is humanoid. His stats, aside from Defense, are the same across all forms. Whatever equipment he is wearing or carrying is not transformed. Upon death he returns to his true form.

\emph{\textbf{Sense of Smell.}} The wererat has +1d6 on Wisdom (Awareness) checks that rely on smell.

\textbf{Shares}

\emph{\textbf{Multiattack (Humanoid or Hybrid Form Only).}} The wererat makes two attacks, of which only one can be a bite attack.

\emph{\textbf{Short Sword (Humanoid or Hybrid Form Only).} Melee Weapon Attack}: +4 to hit, reach 1 m, one target. \emph{Hits:} 5 (1d6 + 2) piercing damage.

\emph{\textbf{Hand Crossbow (Humanoid or Hybrid Form Only).} Ranged Weapon Attack}: +4 to hit, range 9m, one target.

\emph{Hits:} 5 (1d6 + 2) piercing damage.

\emph{\textbf{Bite (Rat or Hybrid Form Only).} Melee Weapon Attack}: +4 to hit, reach 1 m, one target.

\emph{Hits:} 4 (1d4 + 2) piercing damage. If the target is a humanoid, it must succeed on a DC 11 Fortitude save or be cursed by the wererat's lycanthropy.

\textbf{Ecology}\\
Environment: Any Urban\\
Organization: Solo, pair, pack (5-10) or guild (11-30 plus 5-12 Cruel Rats)\\
\textbf{Treasure}: NPC Equipment (Masterwork Studded Leather Armor, Short Sword, Light Crossbow with 20 Bolts, more treasure)\\
\textbf{Description}\\
Natural wererats are short, lean, and muscular, with alert, alert eyes, and have nervous movements. Males often have thin, stunted whiskers.

\medskip\index[Monster]{Weretiger}\textbf{Weretiger}

\emph{Medium humanoid (human, shapeshifter), neutral}

\textbf{STRENGTH} +3

\textbf{DEXTERITY} +2

\textbf{CONSTITUTION} +3

\textbf{INTELLIGENCE} +0

\textbf{WISDOM} +1

\textbf{CHARISMA} +0

\textbf{Initiative} +2 -- \textbf{Defense} 14

\textbf{Hit Points} 120 (16d8 + 48)

\textbf{Movement} 9 m (12 m in tiger form)

\textbf{Saving Throws}: Fortitude +2, Reflexes +7, Will +4

\textbf{Skills} Stealth +4, Awareness +5

\textbf{Damage Immunity} from non-magical weapons that are not silvered

\textbf{Senses} Darkvision 18 m

\textbf{Languages} Common (cannot speak in tiger form)

\textbf{Challenge} 4 (1.1100 PX)

\emph{\textbf{Leap.}} If the weretiger moves at least 5 feet in a straight line towards a creature and hits it with a claw attack during the same round, the target must succeed on a Fortitude saving throw DC 14 or fall prone. If the target is prone, the weretiger can make a bite attack against it as a bonus action.

€21749 € {€21750 € {Shapeshifting.}} The weretiger can use her action to transform into a tiger-humanoid hybrid or a tiger, or to return to its true form, which is humanoid. His stats, aside from Defense, are the same across all forms. Whatever equipment he is wearing or carrying is not transformed. Upon death he returns to his true form.

\emph{\textbf{Smell and Hearing.}} The weretiger has +1d6 on Wisdom (Awareness) checks based on smell and hearing.

\textbf{Shares}

\emph{\textbf{Multiattack (Humanoid or Hybrid Form Only).}} In humanoid form, the weretiger makes two scimitar attacks or two longbow attacks. In hybrid form, it can attack as a humanoid or make two claw attacks.

\emph{\textbf{Claw (Tiger or Hybrid Form Only).} Melee Weapon Attack}: +9 to hit, reach 1 m, one target. \emph{Hits:} 7 (1d8 + 3) slashing damage, 1 bleed damage.

\emph{\textbf{Bite (Tiger or Hybrid Form Only).} Melee Weapon Attack}: +9 to hit, reach 1 m, one target.

\emph{Hits:} 8 (1d10 + 3) piercing damage. If the target is a humanoid, it must succeed on a DC 13 Fortitude save or be cursed by the weretiger's lycanthropy.

\emph{\textbf{Scimitar (Humanoid or Hybrid Form Only).} Melee Weapon Attack}: +9 to hit, reach 1 m, one target. \emph{Hits:} 6 (1d6 + 3) slashing damage.

\emph{\textbf{Longbow (Humanoid or Hybrid Form Only).} Ranged Weapon Attack}: +8 to hit, range 45m, one target.

\emph{Hits:} 6 (1d8 + 2) piercing damage.

\textbf{Ecology}
Environment: Any Plains or Swamp\\
Organization: Solo or couple\\
\textbf{Treasure}: NPC Equipment (Studded Leather Armor, Short Sword, 2 Daggers, more treasure)\\
\textbf{Description}\\
Weretigers in humanoid form have large eyes, elongated noses, prominent cheekbones, and brown or red, or white, black, or blue-gray hair. Their movements are careful and graceful, and those who look at them might mistake them for an excellent cutpurse, a graceful dancer, or a skilled courtesan.


\medskip\index[Monster]{Manticore}\textbf{Manticore}

\emph{Large monstrosity, lawful evil}

\textbf{STRENGTH} +3

\textbf{DEXTERITY} +3

\textbf{CONSTITUTION} +3

\textbf{INTELLIGENCE} -2

\textbf{WISDOM} +1

\textbf{CHARISMA} -1

\textbf{Initiative} +3 -- \textbf{Defense} 16

\textbf{Hit Points} 68 (8d10 + 24)

\textbf{Movement} 9 m, flight 15 m

\textbf{Saving Throws}: Fortitude +9, Reflexes +7, Will +3

\textbf{Senses} Darkvision 18 m

\textbf{Languages} Municipality

\textbf{Challenge} 3 (700 PX)

\emph{\textbf{Grow Tail Spines.}} The manticore has twenty-four spines in its tail. Used thorns grow back at dawn.

\textbf{Shares}

\emph{\textbf{Multiattack.}} The manticore makes three attacks: one with its bite and two with its claws or three with its tail spines.

\emph{\textbf{Claw.} Melee weapon attack}: +7 to hit, reach 1 m, one target.

\emph{Hits:} 6 (1d6 + 3) slashing damage, 1 bleed damage.

\emph{\textbf{Bite.} Melee weapon attack}: +7 to hit, reach 1 m, one target.

\emph{Hits:} 7 (1d8 + 3) piercing damage.

\emph{\textbf{Tail Thorns.} Ranged Weapon Attack}: +7 to hit, range 30m, one target.

\emph{Hits:} 7 (1d8 + 3) piercing damage.

\textbf{Ecology}
Environment: Hills and Warm Swamps\\
Organization: Solitary, pair or pack (3-6)\\
\textbf{Treasure}: Standard\\

\textbf{Description}\\
Manticores are ferocious predators that survey large areas in search of fresh meat. A typical manticore is about 3 meters long and weighs about 500 kg. Some have heads similar to those of humans, usually bearded. Males and females are very similar.

Manticores eat any type of meat, even carrion, but prefer human meat and rarely miss an opportunity to enjoy this delight. They are crafty and social enough to make deals with evil humanoids to form alliances or force them to offer tribute, and many powerful creatures task them with guarding or controlling a place or area. They prefer to make their dens in high places, such as hilltops and caves in cliffs.

Although manticores are similar to magical creations, they have long been counted among the natural species. Curiously, manticores appear strangely fertile and can interbreed with numerous other similarly shaped species, including Lions, Tigers, Lamias, Sphinxes, and Chimeras.

\medskip\index[Monster]{Assassin's Cloak}\textbf{Assassin's Cloak}

\emph{Large aberration, chaotic neutral}

\textbf{STRENGTH} +3

\textbf{DEXTERITY} +2

\textbf{CONSTITUTION} +1

\textbf{INTELLIGENCE} +1

\textbf{WISDOM} +1

\textbf{CHARISMA} +2

\textbf{Initiative} +2 -- \textbf{Defense} 18

\textbf{Hit Points} 78 (12d10 + 12)

\textbf{Movement} 3 m, flight 12 m

\textbf{Saving Throws}: Fortitude +6, Reflexes +5, Will +7

\textbf{Skills} Stealth +5

\textbf{Senses} Darkvision 18 m

\textbf{Languages} Language of the Depths

\textbf{Challenge} 8 (3900 XP)

\emph{\textbf{False Appearance.}} While the killer cloak remains motionless without exposing the lower body, it is indistinguishable from a black leather cloak.

\emph{\textbf{Light Sensitivity}}. While in sunlight, the murder cloak has -1d6 on attack rolls, as well as on sight-based Wisdom (Awareness) checks.

\emph{\textbf{Damage Transfer.}} While attached to a creature, the murder cloak takes only half the damage dealt to it (round down), and the creature victim of the murder cloak takes the The other half.

\textbf{Shares}

\emph{\textbf{Multiattack.}} The murder cloak makes two attacks:
one with the bite and one with the tail.

\emph{\textbf{Bite.} Melee weapon attack}: +12 to hit, reach 3 ft., one creature.

\emph{Hits:} 10 (2d6 + 3) piercing damage, and if the target is Large or smaller, the murder cloak sticks to it.
As long as the assassin's cloak is attached he has a +1d6 to attack rolls. When he makes an attack roll with the bonus of +1d6 and hits, the target is blinded and unable to breathe. The murder cloak can come off by spending 1 Move Action. A creature, including the target, can take its action to peel off the murderous cloak by making a successful Fortitude check with a Strength modifier DC 16.

\emph{\textbf{Tail.} Melee weapon attack}: +12 to hit, reach 10 ft., one creature.

\emph{Hits:} 7 (1d8 + 3) slashing damage.

\emph{\textbf{Apparitions (Recharge after 1 hour).}} If not under bright light, the murder cloak creates three illusory duplicates of itself, which move together with it and imitate its actions, exchanging positions to make it impossible to understand which is the real murderer's mantle. If the original is placed in an area of ​​bright light, the duplicates fade away.

Whenever a creature targets the murder cloak with an attack or harmful spell while duplicates are still present, that creature randomly determines whether it targets the murder cloak or one of the duplicates. A creature that cannot see or that relies on senses other than sight ignores this magical effect.

A duplicate has Defense and uses the cloak's saving throws. If an attack hits a duplicate, or if a duplicate fails a saving throw against an effect that deals damage, it vanishes.

\emph{\textbf{Moan.}} Each creature within 60 feet of the killing cloak that can hear its moan and that is not an aberration must succeed on a DC 13 Will save or be frightened until it ends of the next round of the killing cloak. If the creature's saving throw
succeeds, the creature is immune to the deathcloak's wail for the next 24 hours.

\emph{\textbf{Angry:}} the Killer Cloak recharges the Apparitions ability. Costs 1 Action.

\textbf{Ecology}
Environment: Underground\\
Organization: Solitary, pair, group (3-6) or flock (7-12)\\
\textbf{Treasure}: Standard\\
\textbf{Description}\\
Resembling horribly evil flying manta rays, murder cloaks are mysterious and paranoid creatures. A typical specimen has a wingspan of 2.3 meters and weighs 50 kg.

Their motivations are mysterious and confusing, and they distrust even their own kind. The strange shape allows them to be mistaken for cloaks, tapestries, or other common objects, and some stories tell of murderous cloaks teaming up with other creatures, riding on their backs and helping to protect their allies for unfathomable reasons. Some specimens are priests of ancient deities, commanding cults of murderous cloaks and Skum intent on performing horrible rites with sinister purposes.


\medskip\index[Monstrorium]{Darkmantle}\textbf{Darkmantle}

\emph{Small monstrosity, misaligned}

\textbf{STRENGTH} +3

\textbf{DEXTERITY} +1

\textbf{CONSTITUTION} +1

\textbf{INTELLIGENCE} -4

\textbf{WISDOM} +0

\textbf{CHARISMA} -3

\textbf{Initiative} +1 -- \textbf{Defense} 12

\textbf{Hit Points} 22 (5d6 + 5)

\textbf{Movement} 3 m, flight 9 m

\textbf{Saving Throws}: Fortitude +5, Reflexes +3, Will +0

\textbf{Skills} Stealth +3

\textbf{Sensi} blind sight 18 m

\textbf{Languages} -

\textbf{Challenge} 1/2 (100 PX)

\emph{\textbf{Echolocation.}} The darkmantle cannot use blindsight if deafened.

\emph{\textbf{False Appearance.}} While the darkmantle remains immobile, it is indistinguishable from a rock formation such as a stalactite or stalagmite.

\textbf{Shares}

\emph{\textbf{Smash.} Melee weapon attack}: +5 to hit, reach 3 ft., one creature.

\emph{Hits:} 6 (1d6 + 3) bludgeoning damage and the darkmantle sticks to the creature. If the target is Medium or smaller, the darkmantle has +1d6 to attack rolls and attaches itself to the target's head, making the target blind and unable to breathe as long as the darkmantle remains attached in this way.

While attached to the target, the darkmantle can't attack any other creature except the target, but it has +1d6 to its attack rolls. The darkmantle's speed becomes 0 and he cannot benefit from any speed bonuses when moving with the target.

A creature can detach the darkmantle with an action and succeeding on a DC 13 Strength check. During its round, the darkmantle can detach itself from the target by using 1 Move Action.

\emph{\textbf{Aura of Darkness (1/Day).}} A magical darkness with a 5 meter radius extends from the darkmantle, moving with it, and spreading around corners. The darkness lasts as long as the darkcloak maintains concentration, a maximum of 10 minutes (as if concentrating on a spell). Darkvision cannot penetrate this darkness, nor can it be illuminated by any natural light. If any part of the darkness overlaps an area of ​​light created by a spell of 2nd or lower level, the spell that is creating the light is dispelled.

\textbf{Ecology}
Environment: Any (Underground)\\
Organization: Solitary, pair or brood (3-12)\\
\textbf{Treasure}: None\\
\textbf{Description}\\
the tentacular opening of a mantoscuro has a width of just under 1 m; when it is hanging from the vault of a cave, disguised as a stalactite, its length varies between 60 and 90 cm. A typical specimen of darkmantle weighs 20 kg. The creature's head and body are usually the color of basalt or dark granite, but its membranous tentacles can change color to match its surroundings.

Darkmantles are not particularly skilled climbers, but they are able to hang from the vault of a cave like bats, hooked by the hooks at the bottom of their tentacles, so that their dangling body is almost indistinguishable from a stalactite. From this hidden position the creature waits for the prey to pass beneath it and, at this point, it breaks away, launching itself towards it, slamming into the target and attempting to wrap its membranous tentacles around it. If the darkmantle misses its prey, it climbs back up and lunges at the prey again, until the latter is defeated or the darkmantle is seriously injured (in which case it flutters to the ceiling to hide, hoping that its prey will let it go). . This creature's innate ability to cloak its surroundings with magical darkness gives it an added advantage against opponents who require light to see.

Darkmantles prefer to live and hunt in caves and burrows closer to the surface, since these offer more frequent passage of prey for these monsters to hunt. However, they are not limited to these dark caves and can sometimes be encountered in abandoned fortresses or even in the sewers of crowded cities. Any place where food is plentiful and there is a ceiling to hang from is a possible lair for a darkmantle.

The life cycle of a darkmantle is rapid: the young become adults within a few months and most die of old age after a few years. As a result, generations of darkmantles follow each other rapidly and over the years the evolution of these creatures is equally rapid. For this reason, a cave's ecosystem can have important effects on a darkmantle's appearance, capabilities, and tactics. In aquatic caves, darkcloaks capable of swimming can develop, while creatures that inhabit places subject to volcanism could develop a specific resistance to fire. Other variants of darkmantle could have more resistant skins and instead of falling to crush the prey they could simply throw themselves trying to pierce it similarly to real stalactites. It is rumored that the darkest and deepest caves hide darkcloaks of incredible size, capable of suffocating several human-sized targets simultaneously in their enveloping embrace.


\medskip\index[Monstery]{Medusa}\textbf{Medusa}

\emph{Medium monstrosity, lawful evil}

\textbf{STRENGTH} +0

\textbf{DEXTERITY} +2

\textbf{CONSTITUTION} +3

\textbf{INTELLIGENCE} +1

\textbf{WISDOM} +1

\textbf{CHARISMA} +2

\textbf{Initiative} +2 -- \textbf{Defense} 18

\textbf{Hit Points} 127 (17d8 + 51)

\textbf{Movement} 9 m

\textbf{Saving Throws}: Fortitude +6, Reflexes +8, Will +7

\textbf{Skills} Stealth +5, Deception +5, Sense Emotions +4, Awareness +4

\textbf{Senses} Darkvision 18 m

\textbf{Languages} Municipality

\textbf{Challenge} 6 (2300 PX)

\emph{\textbf{Petrifying Gaze.}} If a creature begins its round within 30 feet of a jellyfish whose eyes it can see, the jellyfish, if it is not incapacitated and can also see the creature, can force it to make a DC 14 Fortitude save. If the creature critically fails the save, it is instantly petrified, otherwise it magically begins to turn to stone and is restrained. The entangled creature must repeat the saving throw at the end of its next round. If she succeeds, the effect ends. On a failed save, the creature is petrified until freed by a \emph{greater restoration} spell or other magic.

A creature that is not surprised can look away to avoid the saving throw at the start of its round. In that case, he won't be able to see the jellyfish until the start of his next round, when he can look away again. If he were to look at the jellyfish in the meantime, he should immediately make the saving throw.

If the jellyfish sees its reflection on a reflective surface within 30 feet of it in an area of ​​bright light, its curse causes it to be affected by the same gaze as her.

\textbf{Shares}

\emph{\textbf{Multiattack.}} The medusa makes three attacks, one with her serpentine hair and two with her short sword or two ranged attacks with her longbow.

\emph{\textbf{Serpentine hair.} Melee weapon attack}: +9 to hit, reach 1 m, one target.

\emph{Hits:} 4 (1d4 + 2) piercing damage plus 14 (4d6) poison damage.

\emph{\textbf{Short Sword.} Melee weapon attack}: 9 to hit, reach 1 m, one target.

\emph{Hits:} 5 (1d6 + 2) piercing damage.

\emph{\textbf{Longbow.} Ranged weapon attack}: +9 to hit, range 45m, one target.

\emph{Hits:} 6 (1d8 + 2) piercing damage plus 7 (2d6) poison damage.

\textbf{Ecology}\\
Environment: Temperate swamps and underground\\
Organization: Solitaire\\
\textbf{Treasure}: Double (Dagger, Masterwork Longbow with 20 Arrows, more treasure)\\
\textbf{Description}\\
Jellyfish are human-like creatures with snakes for hair. From a distance of 30 feet or more, a jellyfish can easily pass for a beautiful woman if she wears something that covers her serpentine hair; when she wears clothing that conceals her head and face she can be mistaken for a human even at close range. Jellyfish use lies and disguises to hide their faces until opponents are close enough to use their petrifying gaze, although they like to toy with their prey and can use flaming arrows to trap enemies from a distance. Some enjoy creating intricate decorations with their victims, using petrification to add flair to their swampy hideouts, but many jellyfish take care to hide the evidence of their previous encounters so that their new enemies are unaware of their dangerousness. presence.

Accustomed to hiding, city jellyfish are generally thieves, while those in the wilderness often end up as forest rangers. The jellyfish of the best-known legends, however, are the ones that take spellcaster levels. Charismatic and intelligent, urban jellyfish are often involved in thieves' guilds and other aspects of the criminal underworld. Jellyfish can form alliances with blind creatures or intelligent undead, both of whom are immune to their petrifying gaze. Jellyfish enchantresses often serve as oracles or prophetesses, generally living in remote areas of legendary power or inauspicious history. These oracle jellyfish take great delight in their roles, and if presented with the right gifts and flattery, the secrets they offer can be truly useful. Naturally, the hideouts of these powerful creatures are decorated with statues of those who have offended them, as a warning to use due caution during encounters.

All jellyfish are female. Rarely, a medusa decides to take a humanoid male as a mate, usually with the help of a Love Elixir or some similar magic, and they always take care not to petrify their captive, unless they have grown bored of the company of she.


\subsection{Mefiti}

\medskip\index[Monster]{Ice Mephito}\textbf{Ice Mephito}

\emph{Small elemental, neutral evil}

\textbf{STRENGTH} -2

\textbf{DEXTERITY} +1

\textbf{CONSTITUTION} +0

\textbf{INTELLIGENCE} -1

\textbf{WISDOM} +0

\textbf{CHARISMA} +1

\textbf{Initiative} +1 -- \textbf{Defense} 12

\textbf{Hit Points} 21 (6d6)

\textbf{Movement} 9 m, flight 9 m

\textbf{Saving Throws}: Fortitude +2, Reflexes +5, Will +3

\textbf{Skills} Stealth +3, Awareness +2

\textbf{Damage Vulnerability} bludgeoning, fire

\textbf{Damage Immunity} Cold, Poison

\textbf{Condition Immunity} poisoned

\textbf{Senses} Darkvision 18 m

\textbf{Languages} Aquan, Ictun

\textbf{Challenge} 1/2 (100 PX)

\emph{\textbf{False Appearance.}} While the mephito remains immobile, it is indistinguishable from an ordinary shard of ice.

\emph{\textbf{Innate Spells (1/Day).}} The mephito can innately cast \emph{fog cloud}, without the need for material components. His innate spellcaster ability is Charisma.

\emph{\textbf{Elemental Nature.}} A mephite has no need for food, drink, or sleep.

\emph{\textbf{Deadly Blast.}} When the mephit dies, it explodes in a burst of ice shards. Each creature within 3 feet of it must make a DC 10 Reflex saving throw or take 4 (1d8) slashing damage on a failed save, or half as much damage on a failed save.
successfull.

\textbf{Shares}

\emph{\textbf{Claws.} Melee weapon attack}: +3 to hit, reach 3 ft., one creature.

\emph{Hits:} 3 (1d4 + 1) slashing damage plus 2 (1d4) cold damage.

\emph{\textbf{Icy Breath (Recharge 6).}} The mephitus exhales a 5 meter cone of cold air. Each creature in the area must make a DC 10 Reflex saving throw, taking 5 (2d4) cold damage on a failed save, or half as much damage on a successful one.

\textbf{Ecology}\\
Environment: Any (elemental plane of air)\\
Organization: Solitary, pair, group (3-6) or flock (7-12)\\
\textbf{Treasure}: Standard\\
\textbf{Description}\\
Mephits are the servants of powerful elemental creatures. Key sites and locations on the elemental planes are filled with mephits scrambling to perform an important duty or errand.

Ice mephits are commonly found on the Plane of Air. These mephits are distant and cruel.


\medskip\index[Monster]{Magma Mephito}\textbf{Magma Mephito}

\emph{Small elemental, neutral evil}

\textbf{STRENGTH} -1

\textbf{DEXTERITY} +1

\textbf{CONSTITUTION} +1

\textbf{INTELLIGENCE} -2

\textbf{WISDOM} +0

\textbf{CHARISMA} +0

\textbf{Initiative} +1 -- \textbf{Defense} 12

\textbf{Hit Points} 22 (5d6 + 5)

\textbf{Movement} 9 m, flight 9 m

\textbf{Saving Throws}: Fortitude +2, Reflexes +5, Will +3

\textbf{Skills} Stealth +3

\textbf{Vulnerability to Damage} cold

\textbf{Damage Immunity} Fire, Poison

\textbf{Condition Immunity} poisoned

\textbf{Senses} Darkvision 18 m

\textbf{Languages} Ignan, Tremun

\textbf{Challenge} 1/2 (100 PX)

\emph{\textbf{False Appearance.}} While the mephito remains immobile, it is indistinguishable from an ordinary pool of magma.

\emph{\textbf{Innate Spells (1/Day).}} The mephito can innately cast \emph{heat metal} (spell save DC 10), without the need for material components. His innate spellcaster ability is Charisma.

\emph{\textbf{Elemental Nature.}} A mephite has no need for food, drink, or sleep.

\emph{\textbf{Deadly Blast.}} When the mephite dies, it explodes in a burst of lava. Each creature within 3 feet of it must make a DC 11 Reflex saving throw or take 7 (2d6) fire damage on a failed save, or half as much damage on a successful one.

\textbf{Shares}

\emph{\textbf{Claws.} Melee weapon attack}: +3 to hit, reach 3 ft., one creature.

\emph{Hits:} 3 (1d4 + 1) slashing damage plus 2 (1d4) fire damage.

\emph{\textbf{Fiery Breath (Recharge 6).}} The mephito exhales a 5 meter cone of fire. Each creature in the area must make a DC 11 Reflex saving throw, taking 7 (2d6) fire damage on a failed save, or half as much damage on a successful one.

\textbf{Ecology}\\
Environment: Any (elemental plane of fire)\\
Organization: Solitary, pair, group (3-6) or flock (7-12)\\
\textbf{Treasury}: Standard\\
\textbf{Description}\\
Mephits are the servants of powerful elemental creatures. Key sites and locations on the elemental planes are filled with mephits scrambling to perform an important duty or errand.

Magma mephits are commonly found on the Plane of Fire. These mephits are stupid brutes.


\medskip\index[Monstrorium]{Powder Mephite}\textbf{Powder Mephite}

\emph{Small elemental, neutral evil}

\textbf{STRENGTH} -3

\textbf{DEXTERITY} +2

\textbf{CONSTITUTION} +0

\textbf{INTELLIGENCE} -1

\textbf{WISDOM} +0

\textbf{CHARISMA} +0

\textbf{Initiative} +2 -- \textbf{Defense} 13

\textbf{Hit Points} 17 (5d6)

\textbf{Movement} 9 m, flight 9 m

\textbf{Saving Throws}: Fortitude +2, Reflexes +5, Will +3

\textbf{Skills} Stealth +4, Awareness +2

\textbf{Damage Vulnerability} fire

\textbf{Damage Immunity} Poison

\textbf{Condition Immunity} poisoned

\textbf{Senses} Darkvision 18 m

\textbf{Languages} Ictun, Tremun

\textbf{Challenge} 1/2 (100 PX)

\emph{\textbf{Innate Spells (1/Day).}} The mephito can innately cast \emph{sleep} (spell save DC 10), without the need for material components. His innate spellcasting ability is Charisma.

\emph{\textbf{Elemental Nature.}} A mephite has no need for food, drink, or sleep.

\emph{\textbf{Deadly Blast.}} When the mephite dies, it explodes in a blast of dust. Each creature within 3 feet of it must succeed on a DC 10 Fortitude saving throw or be blinded for 1 minute. A blinded creature can repeat the saving throw during each of its rounds, ending the effect on itself on a success.

\textbf{Shares}

\emph{\textbf{Claws.} Melee weapon attack}: +4 to hit, reach 3 ft., one creature.

\emph{Hits:} 4 (1d4 + 2) slashing damage.

\emph{\textbf{Blinding Breath (Recharge 6).}} The mephitus exhales a 5 meter cone of blinding dust. Each creature in the area must succeed on a DC 10 Reflex saving throw or be blinded for 1 minute. A blinded creature can repeat the saving throw during each of its rounds, ending the effect on itself on a success.

\textbf{Ecology}\\
Environment: Any (elemental plane of air)\\
Organization: Solitary, pair, group (3-6) or flock (7-12)\\
\textbf{Treasury}: Standard\\
\textbf{Description}\\
Mephits are the servants of powerful elemental creatures. Key sites and locations on the elemental planes are filled with mephits scrambling to perform an important duty or errand.

Dust mephits are commonly found on the Plane of Air. These mephits are irritating and persistent.

\medskip\index[Monster]{Steam Mephith}\textbf{Steam Mephith}

\emph{Small elemental, neutral evil}

\textbf{STRENGTH} -3

\textbf{DEXTERITY} +0

\textbf{CONSTITUTION} +0

\textbf{INTELLIGENCE} +0

\textbf{WISDOM} +0

\textbf{CHARISMA} +1

\textbf{Initiative} +0 -- \textbf{Defense} 11

\textbf{Hit Points} 21 (6d6)

\textbf{Movement} 9 m, flight 9 m

\textbf{Saving Throws}: Fortitude +2, Reflexes +5, Will +3

\textbf{Damage Immunity} Fire, Poison

\textbf{Condition Immunity} poisoned

\textbf{Senses} Darkvision 18 m

\textbf{Languages} Aquan, Ignan

\textbf{Challenge} 1/4 (50 XP)

\emph{\textbf{Innate Spells (1/Day).}} The mephito can innately cast \emph{blur}, without the need for material components. His innate spellcasting ability is Charisma.

\emph{\textbf{Elemental Nature.}} A mephite has no need for food, drink, or sleep.

\emph{\textbf{Deadly Blast.}} When the mephite dies, it explodes into a cloud of steam. Each creature within 3 feet of it must succeed on a DC 10 Reflex saving throw or take 4 (1d8) fire damage.

\textbf{Shares}

\emph{\textbf{Claws.} Melee weapon attack}: +2 to hit, reach 3 ft., one creature.

\emph{Hits:} 2 (1d4) slashing damage plus 2 (1d4) fire damage.

\emph{\textbf{Steamy Breath (Recharge 6).}} The mephitus exhales a 5 meter cone of hot steam. Each creature in the area must make a DC 10 Reflex saving throw, taking 4 (1d8) fire damage on a failed save, or half as much damage on a successful one.

\textbf{Ecology}\\
Environment: Any (elemental plane of fire)\\
Organization: Solitary, pair, group (3-6) or flock (7-12)\\
\textbf{Treasury}: Standard\\
\textbf{Description}\\
Mephits are the servants of powerful elemental creatures. Key sites and locations on the elemental planes are filled with mephits scrambling to perform an important duty or errand.

Steam mephits are commonly found on the Plane of Fire. These mephits are insolent and contemptuous.



\subsection{Hags}

\medskip\index[Monster]{Marine Hag}\textbf{Marine Hag}

\emph{Medium fey, chaotic evil}

\textbf{STRENGTH} +3

\textbf{DEXTERITY} +1

\textbf{CONSTITUTION} +3

\textbf{INTELLIGENCE} +1

\textbf{WISDOM} +1

\textbf{CHARISMA} +1

\textbf{Initiative} +1 -- \textbf{Defense} 15

\textbf{Hit Points} 52 (7d8 + 21)

\textbf{Damage Vulnerability} cold iron

\textbf{Movement} 9m, swim 12m

\textbf{Saving Throws}: Fortitude +5, Reflexes +7, Will +5

\textbf{Senses} Darkvision 18 m

\textbf{Languages} Aquan, Common, Giant

\textbf{Challenge} 2 (450 PX)

\emph{\textbf{Amphibian.}} The hag can breathe air and water.

\emph{\textbf{Horrific Appearance.}} Any humanoid that begins its round within 30 feet of the hag and can see its true form must make a DC 11 Will save. If the save fails, the creature stay scared for 1 minute. A creature can repeat the saving throw at the end of each of its rounds, with -1d6 if the hag is in line of sight, and ending the effect on a successful save. If the creature's saving throw succeeds or the effect ends on it, the creature is immune to the Horrific Aspect for the next 24 hours.

Unless the target is surprised or the revelation of the hag's true form is sudden, the target can look away and avoid making the initial saving throw. Until the start of its next round, a creature that looks away
has -1d6 to attack rolls against the hag.

\textbf{Shares}

\emph{\textbf{Claws.} Melee weapon attack}: +5 to hit, reach 1 m, one target.

\emph{Hits:} 10 (2d6 + 3) slashing damage, 1 bleed damage.

€22,101 € {€22,102 € {Illusory Appearance.}} The hag covers herself and everything she is wearing or carrying in a magical illusion that gives her the appearance of a loathsome creature of approximately the same size and shape humanoid. The illusion ends if the hag takes a bonus action to end it or if she dies.

The changes brought about by this effect are not able to pass physical inspection. For example, the hag might appear as a creature without claws, but a person touching her hands would feel them. Otherwise, a creature must take an action to visually inspect the illusion and succeed on a DC 16 Intelligence check to realize that the hag is disguised.

\emph{\textbf{Death Stare.}} The hag targets one frightened creature visible within 30 feet of her. If the target can see the hag, it must succeed on a DC 11 Will save against this spell or drop to 0 hit points.

\textbf{Ecology}\\
Environment: any aquatic\\
Organization: solitary or coven (3 hags of any species)\\
\textbf{Treasury}: standard\\
\textbf{Description}\\
These wicked and monstrous hags possess terrifying traits that few dare to look at, they take pleasure in the discord and death of sailors, and torment seafarers with inevitable misfortunes. Sea hags always look terrible, and despite their ravenous nature, they are generally emaciated creatures who look as if they are about to starve. They are 1.8 meters tall and weigh 75 kg.

Sea hags prefer to live close to shore where fishing boats and merchant ships are more common, and away from urban areas so that their actions do not attract too much attention from possible enemies, although it is not unusual for a brave or greedy settles in a port city or at the mouth of a deep river.

Sea hags form covens similar to other hags, but their aquatic nature generally causes them to refrain from forming mixed covens. If a Green Hag lives along the coast (often in a salt marsh or coastal marsh), a coven is made up of two sea hags who respect the Green Hag as mother and leader. Most commonly, a sea hag coven consists of a group of sea hags who are particularly friendly and close.


\medskip\index[Monster]{Night Hag}\textbf{Night Hag}

\emph{Medium fiendish, neutral evil}

\textbf{STRENGTH} +4

\textbf{DEXTERITY} +2

\textbf{CONSTITUTION} +3

\textbf{INTELLIGENCE} +3

\textbf{WISDOM} +2

\textbf{CHARISMA} +3

\textbf{Initiative} +3 -- \textbf{Defense} 20

\textbf{Hit Points} 112 (15d8 + 45)

\textbf{Movement} 9 m

\textbf{Saving Throws}: Fortitude +14, Reflexes +8, Will +11

\textbf{Skills} Stealth +6, Deception +7, Sense Emotions +6, Awareness +6,

\textbf{Damage Resistances} cold, fire; from a non-magical weapon or are not silvered

\textbf{Senses} darkvision 36 m

\textbf{Languages} Abyssal, Common, Infernal, Druidic

\textbf{Challenge} 5 (1800 PX)

\emph{\textbf{Innate Spells.}} The hag's innate spellcasting ability is Charisma (DC 14 for spell saving throws. The hag can innately cast the following spells, without needing
material components.

At will: \emph{Arcane bolt, detect magic} 2/day each: €22131{enfeeblement ray, sleep, shift} €22132{planar} (personal)

\emph{\textbf{Magic Resistance.}} The hag has +1d6 on saving throws against spells and other magical effects.

\textbf{Shares}

\emph{\textbf{Claws (Hag Form Only).} Melee Weapon Attack}: +10 to hit, reach 1 m, one target.

\emph{Hits:} 13 (2d8 + 4) slashing damage, 1 bleed damage.

\emph{\textbf{Ethereal Form.}} The hag magically enters the Ethereal Plane from the Material Plane, and vice versa. To do this you must have €22,141 {heart of stone}.

€22142 € {€22143 € {Haunt Nightmares (1 / Day).}} While she is on the Ethereal Plane, the hag magically comes into contact with a sleeping humanoid on the Material Plane. The \emph{protection from good and evil} spell cast on the target prevents this contact, as does \emph{magic circle}. As long as contact persists, the target suffers from horrific visions. If these visions last for at least 1 hour, the target gains no benefit from her rest, and its maximum hit points are reduced by 5 (1d10). If this effect reduces the target's maximum hit points to 0, the target dies, and if the target was evil, its soul becomes trapped in the hag's €22146 €22147 €{bag of souls}. The reduction from the target's maximum hit points remains until she is removed by the €22148 €{restoration} €22149 €{greater} spell or similar magic.

€22,150 € {€22,151 € {Shapeshifting.}} The hag can magically transform into a Small or Medium-sized humanoid female, or return to her true form. Her stats are the same in any form. Any equipment she was carrying or wearing is not transformed. Upon death, he returns to his true form.



\medskip\index[Monster]{Green Hag}\textbf{Green Hag}

\emph{Medium fairy, neutral evil}

\textbf{STRENGTH} +4

\textbf{DEXTERITY} +1

\textbf{CONSTITUTION} +3

\textbf{INTELLIGENCE} +1

\textbf{WISDOM} +2

\textbf{CHARISMA} +2

\textbf{Initiative} +1 -- \textbf{Defense} 19

\textbf{Hit Points} 82 (11d8 + 33)

\textbf{Damage Vulnerability} cold iron

\textbf{Movement} 9 m

\textbf{Saving Throws}: Fortitude +6, Reflexes +7, Will +7

\textbf{Skills} Arcana +3, Stealth +3, Deception +4, Awareness +4

\textbf{Senses} Darkvision 18 m

\textbf{Languages} Common, Draconic, Sylvan

\textbf{Challenge} 3 (700 PX)

\emph{\textbf{Amphibian.}} The hag can breathe air and water.

\emph{\textbf{Imitation.}} The hag can imitate animal sounds and humanoid voices. A creature that hears these noises can determine that it is an imitation by succeeding on a DC 14 Wisdom check.

\emph{\textbf{Innate Spells.}} The hag's innate spellcasting ability is Charisma (DC 12 for spell saving throws). The hag can innately cast the following spells, without requiring material components.

All you want: \emph{minor illusion, dancing lights, evil mockery}

\textbf{Shares}

\emph{\textbf{Claws.} Melee weapon attack}: +6 to hit, reach 1 m, one target.

\emph{Hits:} 13 (2d8 + 4) slashing damage, 1 bleed damage.

€22183 € {€22184 € {Illusory Appearance.}} The hag covers herself and everything she is wearing or carrying in a magical illusion that gives her the appearance of another creature of approximately the same size and humanoid shape. The illusion ends if the hag takes a bonus action to end it or if she dies.

The changes brought about by this effect are not able to pass physical inspection. For example, the hag might appear as a smooth-skinned creature, but contact would reveal her rough skin. Otherwise, a creature must take an action to visually inspect the illusion and succeed on a DC 20 Intelligence check to realize that it is a hag in disguise.

€22,185 € {€22,186 € {Invisible Passage.}} The hag can make herself invisible until she attacks or casts a spell, or until she ends her concentration (as if she were concentrating on a spell). While she is invisible, she leaves no physical trace of her passing, so her trail can only be followed by magic. All equipment she is carrying or wearing becomes invisible along with her.

\textbf{Ecology}
Environment: Temperate swamps\\
Organization: Solitary or coven (3 hags of any type)\\
\textbf{Treasure}: Standard\\
\textbf{Description}\\
Terrifying wrinkled old women who frequent foul swamps and tangled forests, green hags harbor an intense hatred for all that is beautiful and pure. Making use of their various illusory abilities, these crones delight in killing the innocent, shocking noble souls, and debasing pure hearts. They love to use Disguise Himself to take the form of attractive young girls so as to seduce and seduce young men from their affections and relatives, and to corrupt noble and honest citizens with all manner of depravity and scandal. Some green hags prefer to reveal their true nature to their beloved in a moment carefully crafted to drive man mad with horror and shame. Others prolong their flirtation and do everything they can to completely ruin the lives of the men they seduce before showing them the truth. Finally, the luckiest of these unfortunates end up being devoured by their green hag lover: for the unfortunate ones, the final fate can be much worse, given that the green hag's cruel imagination is immense. A typical green hag stands between 1.5 and 1.8 meters tall and weighs just under 80kg.


\subsection{Slimes}

\medskip\index[Monstruary]{Straw Amoeba}\textbf{Straw Amoeba}

\emph{Large slime, misaligned}

\textbf{STRENGTH} +2

\textbf{DEXTERITY} -2

\textbf{CONSTITUTION} +2

\textbf{INTELLIGENCE} -4

\textbf{WISDOM} -2

\textbf{CHARRISMA} -5

\textbf{Initiative} +2 -- \textbf{Defense} 9

\textbf{Hit Points} 45 (6d10 + 12)

\textbf{Movement} 3m, climb 3m

\textbf{Saving Throws}: Fortitude +8, Reflexes -3, Will -3

\textbf{Damage Resistances} acid

\textbf{Damage Immunity} Electricity, cutting

\textbf{Condition Immunity} blinded, fascinated, deafened, prone, fatigued, frightened

\textbf{Senses} blindsight 18m (blind beyond this range)

\textbf{Languages} -

\textbf{Challenge} 2 (450 PX)

\emph{\textbf{Amorphous.}} The amoeba can move through a space up to 3 centimeters wide without having to squeeze.

\emph{\textbf{Nature of Slime.}} The amoeba does not need to sleep.

\emph{\textbf{Climb as a Spider.}} The amoeba can climb difficult surfaces, including standing upside down on the ceiling, without needing to make an ability check.

\textbf{Shares}

\emph{\textbf{Pseudopod.} Melee weapon attack}: +4 to hit, reach 1 m, one target.

\emph{Hits:} 9 (2d6 + 2) bludgeoning damage plus 3 (1d6) acid damage.

\textbf{Reactions}

€22223 € {€22224 € {Split.}} When a Medium or larger amoeba takes lightning or slashing damage, it splits into two new amoebas that have at least 10 Hit Points. Each new amoeba has a number of hit points equal to half the original amoeba's, rounded down. The new amoebae are one size smaller than the original one.

\textbf{Ecology}
Environment: Underground or Temperate Swamps\\
Organization: Solitaire\\
\textbf{Treasure}: None\\
\textbf{Description}\\
The Straw Amoeba are animated masses of protoplasm of a color similar to a repellent amalgam of yellow, orange and brown. When at rest, their flat, pulsating body is about 15 centimeters tall and extends all the way around; in movement, they gather in a vaguely spherical shape and almost seem to move by rolling. Their malleable bodies allow them to pass through cracks and holes much smaller than the space they occupy. Creatures that live underground often seal all openings to defend themselves from Straw Amoeba.

The highly specialized acid of the Straw Amoeba only dissolves the flesh. This discovery led many master poisoners and alchemists to look for specimens to study. These experiments resulted in several specific weapons designed to destroy bodies. It is said of the existence of a slow-acting poison that destroys the cells of living creatures one by one, the secret of which is well kept by its creator.

Some notes in a forgotten tome speak of a funeral ritual used in distant places. Instead of burning the body, it was sealed in a stone sarcophagus with a Straw Amoeba, which dissolved the body. Afterwards, the gravediggers inserted the jelly into an urn complete with a bronze plaque with the name of the deceased. This practice protects the objects buried with the dead person (who is quickly reduced to a shiny skeleton) and the essence of the creature, which was believed to still live inside the jelly.

The Straw Amoeba are approximately 15 centimeters tall, have a diameter of up to 3 meters and weigh approximately 1,300 kilos. In combat, they curl up and produce long wet pseudopods to hit and grab anything that moves.

Although the typical Straw Amoeba has the statistics presented here, deep beneath the earth these predators can reach monstrous dimensions. There is also talk of Straw Amoeba which have developed other ways of capturing prey. For example, jellies that poison by touch and expel clouds of toxic gas that burn the eyes and mouth, leaving one defenseless but conscious as this protoplasmic beast slithers over bodies and feeds on them.


\medskip\index[Monster]{Jelly Cube}\textbf{Jelly Cube}

\emph{Large slime, misaligned}

\textbf{STRENGTH} +2

\textbf{DEXTERITY} -4

\textbf{CONSTITUTION} +5

\textbf{INTELLIGENCE} -5

\textbf{WISDOM} -2

\textbf{CHARISMA} -5

\textbf{Initiative} -4 -- \textbf{Defense} 7

\textbf{Hit Points} 84 (8d10 + 40)

\textbf{Movement} 5 metres

\textbf{Saving Throws}: Fortitude +9, Reflexes -4, Will -4

\textbf{Damage Immunity} non-magical edged weapons, critical

\textbf{Condition Immunity} blinded, fascinated, deafened, prone, fatigued, frightened

\textbf{Senses} blindsight 18 m (blind beyond this range)

\textbf{Languages} -

\textbf{Challenge} 2 (450 PX)

\emph{\textbf{Slime Cube.}} The cube takes up its entire space. Other creatures can enter the space, but are subject to the cube's Submergence and have -1d6 on their saving throw.

Creatures inside the cube are visible but have complete coverage.

A creature within 3 feet of the cube can take an action to pull a creature or object out of the cube. Doing so requires a successful DC 12 Strength check, and the creature making the attempt takes 10 (3d6) acid damage.

The cube can contain only one Large creature or a maximum of four Medium or smaller creatures at a time.

\emph{\textbf{Slime Nature.}} The cube does not need to sleep.

\emph{\textbf{Transparent.}} Even when in plain sight, you must succeed on a DC 15 Wisdom (Awareness) check to notice a cube that has not moved or attacked. A creature that tries to enter the cube's space while unaware of its presence is surprised by the cube.

\textbf{Shares}

\emph{\textbf{Pseudopod.} Melee weapon attack}: +4 to hit, reach 1 m, one target.

\emph{Hits:} 10 (3d6) acid damage.

\emph{\textbf{Submerge.}} The cube moves up to its maximum movement. As he does so, he can enter the space of a Large or smaller creature. Each time the cube enters a creature's space, the creature must make a DC 12 Reflex saving throw.

On a successful save, the creature can choose to be pushed back or to the side 3 feet. A creature that decides not to be pushed suffers the consequences of a failed saving throw.

If the saving throw fails, the cube enters the creature's space, and the creature takes 10 (3d6) acid damage and is submerged. The submerged creature cannot breathe, is entangled, and takes 21 (6d6) acid damage at the start of the cube's round. When the cube moves, the submerged creature moves with it.

A submerged creature can attempt to escape by taking an action to make a DC 12 Strength check. If successful, the creature escapes and enters the space of its choice within 3 feet of the cube.

\textbf{Ecology}
Environment: Any dungeon\\
Organization: Solitaire\\
\textbf{Honey}: Accidental\\
\textbf{Description}\\
Among the most unusual and peculiar predators of dungeons, gelatinous cubes spend their existence wandering aimlessly through underground tunnels and dark caves, incorporating organic materials such as plants, waste, carrion and even living creatures. Matter that the cube cannot digest, such as metals and stone, fills the creature's volume with debris, and the creature can sometimes expel some of it from its body. Often the treasure and belongings of past victims remain inside the gelatinous cube: a ghostly image of their material remains.

Some sages believe that these creatures evolved from Gray Slimes. Some beings use the gelatinous cubes as guardians of dungeons and subterranean fortifications, trapping these immense creatures in crates of solid metal and transporting them by power or magic to their final guard post. They are particularly effective waste disposal mechanisms; a tribe can trap a gelatinous cube in a pit or other area it cannot climb by using it as a dunghill or even a death trap, depending on the ingenuity of the creatures that captured it.

The gelatinous cubes typically have an edge of 3 meters and weigh more than 7,500 kg, although some underground explorers claim that larger specimens exist underground. In areas where food is abundant, gelatinous cubes can live for hundreds, if not thousands, of years. However, if organic matter is lacking for more than 6 months, a gelatinous cube begins to decay, and its walls begin to leak, rapidly breaking down into liquid mucus until the entire body collapses and disappears completely.


\medskip\index[Monster]{Grey Slime}\textbf{Grey Slime}

\emph{Medium slime, misaligned}

\textbf{STRENGTH} +1

\textbf{DEXTERITY} -2

\textbf{CONSTITUTION} +3

\textbf{INTELLIGENCE} -5

\textbf{WISDOM} -2

\textbf{CHARISMA} -4

\textbf{Initiative} -2 -- \textbf{Defense} 9

\textbf{Hit Points} 22 (3d8 + 9)

\textbf{Movement} 3m, climb 3m

\textbf{Saving Throws}: Fortitude +9, Reflexes -4, Will -4

\textbf{Damage Resistances} acid, cold, fire

\textbf{Condition Immunity} blinded, fascinated, deafened, prone, fatigued, frightened

\textbf{Senses} blindsight 18 m (blind beyond this range)

\textbf{Languages} -

\textbf{Challenge} 1/2 (100 PX)

\emph{\textbf{Amorphous.}} The slime can move through a space up to centimeters wide without having to squeeze.

\emph{\textbf{Corrode Metal.}} Any nonmagical weapon made of metal that hits the slime becomes corroded. After dealing damage, the weapon takes a permanent, cumulative -1 penalty on damage rolls. If the penalty reaches -5, the weapon is destroyed. Nonmagical ammunition made of metal that hits the ooze is destroyed after dealing damage.

The ooze can devour nonmagical metal up to 2 inches thick in 1 round.

\emph{\textbf{False Appearance.}} When the slime remains motionless, it is indistinguishable from a puddle of oil or a wet stone.

\emph{\textbf{Nature of Slime.}} Slime does not need to sleep.

\textbf{Shares}

\emph{\textbf{Pseudopod.} Melee weapon attack}: +3 to hit, reach 1 m, one target.

\emph{Hits:} 4 (1d6 + 1) bludgeoning damage plus 7 (2d6) acid damage, and if the target is wearing metal armor, it is partially dispelled and takes a permanent, cumulative -1 penalty to the Defense it offers. The armor is destroyed if the penalty reduces its Defense to 10.

\textbf{Ecology}\\
Environment: Cold swamps and underground\\
Organization: Solitaire\\
\textbf{Treasure}: None\\
\textbf{Description}\\
Crawling through cold swamps and misty swamps or, sometimes in dungeons and caves, the gray oozes consume every organic substance they encounter. Although devoid of intelligence, the gray slime is one of the creatures that causes many problems due to its transparency. While she cannot easily climb walls or swim, her habit of hiding in the thick mud along marshy banks or remaining motionless in harmless-looking puddles on the gray floor of a dungeon makes her very difficult to notice and avoid.

Some sages believe that the gray oozes are the result of a failed alchemical experiment, while others theorize that the first gray oozes arose spontaneously from a pit of magical detritus. Of course, these theories that do not consider them to be living organisms, but rather the result of an unfortunate mixture of caustic fluids and magical residue, are derided by those who live in areas infested by these creatures, which have no history of magical pollution.


\medskip\index[Monster]{Black Protoplasm}\textbf{Black Protoplasm}

\emph{Large slime, misaligned}

\textbf{STRENGTH} +3

\textbf{DEXTERITY} -3

\textbf{CONSTITUTION} +3

\textbf{INTELLIGENCE} -5

\textbf{WISDOM} -2

\textbf{CHARRISMA} -5

\textbf{Initiative} -3 -- \textbf{Defense} 9

\textbf{Hit Points} 85 (10d10 + 30)

\textbf{Movement} 6m, climb 6m

\textbf{Saving Throws}: Fortitude +9, Reflexes -2, Will -2

\textbf{Damage Immunity} acid, cold, lightning, slashing, critical

\textbf{Condition Immunity} blinded, fascinated, deafened, prone, fatigued, frightened

\textbf{Senses} blindsight 18 m (blind beyond this range)

\textbf{Languages} -

\textbf{Challenge} 4 (1100 PX)

\emph{\textbf{Amorphous.}} The black pudding can move through a space up to 3 centimeters wide without having to squeeze.

\emph{\textbf{Corrosive Form.}} A creature that comes into contact with the black pudding or hits it with a melee attack while within 3 feet of it takes 4 (1d8) acid damage. Any nonmagical weapon made of metal or wood that hits black pudding corrodes. After dealing damage, the weapon takes a permanent, cumulative -1 penalty on damage rolls. If the penalty reaches -5, the weapon is destroyed. Nonmagical ammunition made of metal or wood that hits black pudding is destroyed after dealing damage.

The black pudding can devour 2-inch-thick wood or nonmagical metal in 1 round.

\emph{\textbf{Slime Nature.}} Black pudding does not need to sleep.

\emph{\textbf{Climb as Spider.}} The black pudding can climb difficult surfaces, including standing upside down on the ceiling, without needing to make an ability check.

\textbf{Shares}

\emph{\textbf{Pseudopod.} Melee weapon attack}: +7 to hit, reach 1 m, one target.

\emph{Hits:} 6 (1d6 + 3) bludgeoning damage plus 18 (4d8) acid damage. Additionally, nonmagical armor worn by the target is partially dispelled and takes a permanent, cumulative -1 penalty to the Defense it offers. The armor is destroyed if the penalty reduces its Defense to 10.

\textbf{Reactions}

\emph{\textbf{Split.}} When a Medium or larger black pudding takes lightning or slashing damage, it splits into two new black puddings of at least 10 Hit Points each. Each new black pudding has a number of hit points equal to half the original black pudding's, rounded down. The new black puddings are one size smaller than the original.

\textbf{Ecology}\\
Environment: Any dungeon\\
Organization: Solitaire\\
\textbf{Treasure}: None\\
\textbf{Description}\\
Black protoplasms are the scavengers of the underworld, constantly searching for food. They can sense organic or metallic bodies within 60 feet, and instinctively attack such objects or beings until they dissolve, or until the slime is killed. A black pudding reproduces by breaking off a piece of its body and forming a new, smaller pudding that reaches adulthood within a month. Some of the most intelligent creatures in the underworld use black pudding to naturally dispose of garbage, creating stone quarries to house the pudding, then throwing organic waste or enemies into it.
The largest specimens of black protoplasms have been sighted in the deepest regions of the world: Mammoth individuals possessing up to 240 HP. It is said that there are also colored protoplasms: some white that live in the arctic areas, brown in the swamps and reddish in color that populate the desert.


\medskip\index[Monster]{Mimic}\textbf{Mimic}

\emph{Medium monstrosity (shapeshifter), neutral}

\textbf{STRENGTH} +3

\textbf{DEXTERITY} +1

\textbf{CONSTITUTION} +2

\textbf{INTELLIGENCE} -3

\textbf{WISDOM} +1

\textbf{CHARRISMA} -1

\textbf{Initiative} +1 -- \textbf{Defense} 13

\textbf{Hit Points} 58 (9d8 + 18)

\textbf{Movement} 5 metres

\textbf{Saving Throws}: Fortitude +5, Reflexes +5, Will +6

\textbf{Skills} Stealth +5

\textbf{Damage Immunity} acid

\textbf{Condition Immunity} prone

\textbf{Senses} Darkvision 18 m

\textbf{Languages} -

\textbf{Challenge} 2 (450 PX)

\emph{\textbf{Clingy (Object Shape Only).}} The mimic sticks to anything it comes into contact with. A Huge or smaller creature that the mimic attaches to is considered grappled by it (escape DC 13). Make ability checks to escape from
this grab have -1d6.

\emph{\textbf{Grabber.}} The mimic has +1d6 on attack rolls against a creature it grabs.

€22361 € {€22362 € {False Appearance (Object Shape Only).}} While the mimic remains immobile, he is indistinguishable from an ordinary object.

€22363 € {€22364 € {Shapeshifter.}} The mimic can use its action to transform into an object, or to return to its true amorphous form. His stats are the same in any form. Whatever equipment he is wearing or carrying does not transform. Upon death he returns to his true appearance.

\textbf{Shares}

\emph{\textbf{Bite.} Melee weapon attack}: +6 to hit, reach 1 m, one target.

\emph{Hits:} 7 (1d8 + 3) piercing damage plus 4 (1d8) acid damage.

\emph{\textbf{Pseudopod.} Melee weapon attack}: +6 to hit, reach 1 m, one target.

\emph{Hits:} 7 (1d8 + 3) bludgeoning damage. If the mimic is in object form, the target becomes a victim of the Cling trait.

\textbf{Ecology}
Environment: Any\\
Organization: Solitaire\\
\textbf{Honey}: Accidental\\
\textbf{Description}\\
Mimics are believed to be the result of an alchemist's attempt to bring an inanimate object to life through the application of a mystical reagent, the formula of which has been lost. Over the years, these strange but intelligent creatures have learned the ability to transform themselves into simulacra of manufactured objects, particularly in places rarely frequented by a small number of creatures, where they increase their chances of success with an attack on their victims.

While mimics are not inherently evil, some scholars suggest that they attack humans and other intelligent creatures more for entertainment than for food. The desire to deceive others is part of their being, and their surprise attacks represent the culmination of this desire.

A typical mimic has a volume of 2 cubic meters (1 m by 1 m by 2 m) and weighs approximately 450 kg. Legends and stories speak of mimics of larger sizes, with the ability to take the form of houses, ships or entire underground complexes which they garnish with treasures (both real and fake) to lure their unsuspecting food inside.


\medskip\index[Monstery]{Minotaur}\textbf{Minotaur}

\emph{Great monstrosity, chaotic evil}

\textbf{STRENGTH} +4

\textbf{DEXTERITY} +0

\textbf{CONSTITUTION} +3

\textbf{INTELLIGENCE} -2

\textbf{WISDOM} +3

\textbf{CHARRISMA} -1

\textbf{Initiative} +0 -- \textbf{Defense} 16

\textbf{Hit Points} 76 (9d10 + 27)

\textbf{Movement} 12 m

\textbf{Saving Throws}: Fortitude +6, Reflexes +5, Will +5

\textbf{Skills} Awareness +7

\textbf{Senses} Darkvision 18 m

\textbf{Languages} Abysmal

\textbf{Challenge} 3 (700 XP)

\emph{\textbf{Charge.}} If the minotaur moves at least 10 feet towards a target and hits it with a gore attack during the same round, the target takes an additional 9 (2d8) piercing damage. If the target is a creature, it must succeed on a DC 14 Fortitude save or be pushed up to 10 feet away and fall prone.

\emph{\textbf{Unwary.}} At the start of its round, the minotaur can gain +1d6 on all melee weapon attack rolls made during that round, but attack rolls against it have + 1d6 until the start of his next round.

€22398 € {€22399 € {Remembering the Labyrinth.}} The minotaur can perfectly remember any route he has taken.

\textbf{Shares}

\emph{\textbf{Double Axe.} Melee weapon attack}: +8 to hit, reach 1 m, one target.

\emph{Hits:} 17 (2d12 + 4) slashing damage.

\emph{\textbf{Gored.} Melee weapon attack}: +8 to hit, reach 1 m, one target.

\emph{Hits:} 13 (2d8 + 4) piercing damage.

\textbf{Ecology}\\
Environment: Temperate Ruins and Dungeons\\
Organization: Solo, couple or group (3-4)\\
\textbf{Treasure}: Standard (Double Axe, other treasure)\\
\textbf{Description}\\
No one holds a grudge like a minotaur. Scorned by civilized races and born centuries ago from a divine curse, minotaurs have hunted, killed, and devoured lesser humanoids to punish real or perceived slights for longer than they can remember. Most cultures have legends about how minotaurs were created by vengeful or wronged deities who punished humans by distorting their appearance, taking away their beauty and intelligence, and giving them bull heads. Yet most modern minotaurs despise these legends and do not believe themselves to be the pranks of some deity, but models of divine perfection created by the cruel and powerful demon lord Baphomet.

The traditional hiding places of minotaurs are labyrinths, both mazes built to confuse and disconcerte, and natural ones created by a tangle of caves or other underground passages. Thanks to their natural cunning, minotaurs use their labyrinthine hideouts to deter unwary enemies who try to track them down or who simply stumble into their hideouts and get lost, slowly hunting down intruders who try in vain to find a way out. Only when desperation has clearly taken over does the minotaur strike his lost victims. When dealing with a group, minotaurs often let one creature escape, to spread its terrible tale and lure others, who hope to slay the beasts, into their labyrinths. Of course, to minotaurs, these would-be heroes make for delicious fare.

Minotaurs can also be found in the service of a more powerful monster or evil creature, and serve it as long as they can hunt and eat to their heart's content. Generally this means guarding some powerful object or valuable location, but it can also mean working as a mercenary, hunting down the master's enemies.

Minotaurs are relatively straightforward fighters, using their horns to gruesomely gore nearby living creatures when they begin to fight.


\subsection{Mummies}

\medskip\index[Monster]{Mummy}\textbf{Mummy}

\emph{Medium undead, lawful evil}

\textbf{STRENGTH} +3

\textbf{DEXTERITY} -1

\textbf{CONSTITUTION} +2

\textbf{INTELLIGENCE} -2

\textbf{WISDOM} +0

\textbf{CHARRISMA} +1

\textbf{Initiative} -1 -- \textbf{Defense} 13

\textbf{Hit Points} 58 (9d8 + 18)

\textbf{Movement} 6 m

\textbf{Saving Throws}: Fortitude +4, Reflexes +2, Will +8

\textbf{Damage Vulnerability} fire

\textbf{Damage Resistances} from non-magical weapons

\textbf{Damage Immunity} from Void, poison

\textbf{Condition Immunity} charmed, poisoned, paralyzed, fatigued, frightened, bleeding

\textbf{Senses} Darkvision 18 m

€22431 € {Languages} the languages ​​he knew in life

\textbf{Challenge} 3 (700 XP)

\emph{\textbf{Undead Nature.}} A mummy has no need for air, food, drink, or sleep.

\textbf{Shares}

\emph{\textbf{Multiattack.}} The mummy can use its Fearful Glare and make a rotting fist attack.

\emph{\textbf{Rotting Fist.} Melee weapon attack}: +7 to hit, reach 1 m, one target.

\emph{Hits:} 10 (2d6 + 3) bludgeoning damage plus 10 (3d6) void damage. If the target is a creature it must succeed on a Fortitude save of 13 or be cursed by mummy rot. The cursed target cannot regain Hit Points, and its maximum Hit Points decrease by 10 (3d6) every 24 hours the curse lasts. If the curse reduces the target's maximum hit points to 0, the target dies, and her body turns to dust. The curse lasts until removed by spell \emph{remove curse} or other magic.

\emph{\textbf{Frightening Glare.}} The mummy targets a creature that it can see and is within 60 feet of it. If the target can see the mummy, it must succeed on a DC 12 Will save against this spell or remain frightened until the end of the mummy's next round. If the target fails its saving throw by 5 or more, it is also paralyzed for the same duration. A target that succeeds on the saving throw is immune to the Dread Glare of all mummies (but not sovereign mummies) for the next 24 hours.

\medskip\index[Monstery]{Sovereign Mummy}\textbf{Sovereign Mummy}

\emph{Medium undead, lawful evil}

\textbf{STRENGTH} +4

\textbf{DEXTERITY} +0

\textbf{CONSTITUTION} +3

\textbf{INTELLIGENCE} +0

\textbf{WISDOM} +4

\textbf{CHARISMA} +3

\textbf{Initiative} +0 -- \textbf{Defense} 25

\textbf{Hit Points} 97 (13d8 + 39)

\textbf{Movement} 6 m

\textbf{Saving Throws}: Fortitude +18, Reflexes +15, Will +19

\textbf{Skills} Religion +5, History +5

\textbf{Damage Vulnerability} fire

\textbf{Damage Immunity} from Void, poison; weapons +1

\textbf{Condition Immunity} charmed, poisoned, paralyzed, fatigued, frightened

\textbf{Senses} Darkvision 18 m

€22464 € {Languages} the languages ​​he knew in life

\textbf{Challenge} 15 (13000 PX)

\emph{\textbf{Heart of the Sovereign Mummy.}} As part of the ritual that creates a sovereign mummy, the heart and viscera of the creature are removed from the corpse and placed inside sealed containers. These containers are usually made of stone or ceramic, engraved or painted with religious hieroglyphs.

As long as its withered heart remains intact, the mummy ruler cannot be permanently destroyed. When it drops to 0 Hit Points, the mummy ruler turns to dust and reforms at full strength 24 hours later, emerging from the dust near the sealed jar that contains her heart. To prevent a sovereign mummy from reforming and destroy it once and for all, its heart must be reduced to ash. For this reason, the mummy ruler usually keeps the heart and viscera hidden within a hidden tomb.

The heart of the sovereign mummy has Defense 5, 25 Hit Points, and immunity to all damage except fire.

€22468 € {€22469 € {Spells.}} The mummy has CM 10. Her spellcasting characteristic is Wisdom. The mummy has the following spells prepared: Cantrips (at will): \emph{holy flame, thaumaturgy}

level 1 (4 slots): \emph{command, tracking bolt, shield of faith}

level 2 (3 slots): \emph{spiritual weapon, block people, silence}

level 3 (3 slots): \emph{animate dead, dispel magic}

level 4 (3 slots): \emph{divination, guardian of faith}

level 5 (2 slots): \emph{contagion, insect plague}

level 6 (1 slot): \emph{wound}

\emph{\textbf{Undead Nature.}} A mummy has no need for air, food, drink, or sleep.

\emph{\textbf{Magic Resistance.}} The mummy ruler has +1d6 on saving throws against spells or other magical effects.

€22481 € {€22482 € {Invigoration.}} A sovereign mummy forms a new body within 24 hours if her heart remains intact, recovering all Hit Points and being able to act again. The new body appears within 3 feet of the sovereign mummy's heart.

\textbf{Shares}

\emph{\textbf{Multiattack.}} The mummy can use its Fearful Glare and make a rotting fist attack.

\emph{\textbf{Rotting Fist.} Melee weapon attack}: +22 to hit, reach 1 m, one target.

\emph{Hits:} 14 (3d6 + 4) bludgeoning damage plus 21 (6d6) void damage. If the target is a creature it must succeed on a Fortitude save of 25 or be cursed by mummy rot. The cursed target cannot regain Hit Points, and its maximum Hit Points decrease by 10 (3d6) for every 24 hours the curse lasts. If the curse reduces the target's maximum hit points to 0, the target dies, and her body turns to dust. The curse lasts until removed by spell \emph{remove curse} or other magic.

\emph{\textbf{Frightening Glare.}} The mummy targets a creature that it can see and is within 60 feet of it. If the target can see the mummy, it must succeed on a DC 18 Will save against this spell or remain frightened until the end of the mummy's next round. If the target fails the saving throw by 5 or more, she is also paralyzed for the same duration. A target that succeeds on the saving throw is immune to the Dread Glare of all mummies (but not sovereign mummies) for the next 24 hours.

\textbf{Additional Shares}

The mummy ruler can perform 3 additional Actions, chosen from the options below. It can use only one Additional option at a time, and only at the end of another creature's round. The mummy ruler regains any additional Actions spent at the start of his round.

\emph{\textbf{Attack.}} The mummy ruler makes an attack with its rotting fist or uses its Fearful Glare.

\emph{\textbf{Channel Negative Energy (Costs 2 Actions).}} The mummy ruler can magically unleash negative energy. Creatures within 60 feet of the mummy ruler, including those behind barriers or corners, cannot regain hit points until the end of the mummy ruler's next round.

\emph{\textbf{Blasphemous Word (Costs 2 Actions).}} The sovereign mummy utters a blasphemous word. Each creature, excluding undead, within 10 feet of the mummy ruler that can hear this magical phrase must succeed on a DC 16 Fortitude saving throw or be stunned until the end of the mummy ruler's next round.

\emph{\textbf{Blinding Dust.}} Blinding dust and sand swirl magically around the mummy ruler. Each creature within 3 feet of the mummy ruler must succeed on a DC 16 Fortitude saving throw or be blinded until the end of the creature's next round.

\emph{\textbf{Sand Whirlwind (Costs 2 Actions).}} The sovereign mummy can magically transform into a sand whirlwind, moving up to 18 meters, and then returning to its normal form. While in whirlwind form, the mummy ruler is immune to all damage, and cannot be grabbed, petrified, knocked prone, entangled, or stunned. Equipment worn or carried by the mummy ruler remains in her possession.

\emph{\textbf{Angry:}} The sovereign Mummy is hungry for life. She channels the energy of death and destruction in a 40-foot radius around her. Each creature must succeed at a DC 22 Fortitude save to take half or 22 damage. The Mummy regains all hit points lost by other creatures.

\textbf{Ecology}
Environment: Any\\
Organization: Lonely, Cort (3-6) or Flock (7-12)\\
\textbf{Treasure}: Double\\
\textbf{Description}\\
Many cultures practice the sacred art of mummification, though the sinister magical techniques used to imbue corpses with undead vitality are far less common. In some ancient lands, such blasphemous techniques were honed through centuries of ceremony and countless deaths, resulting in mummies of terrible power. On rare occasions, if the deceased was of high status and excessive wickedness, he might undergo such elaborate rituals, rising from the grave as a fearsome mummy lord. Likewise, a ruler known for his mischief or who died in a moment of great anger might spontaneously present himself as a vengeful despot. Regardless of the exact circumstances of her resurrection, a mummy ruler retains the abilities she had in life, becoming a creature consumed by the desire to restore her dominion and rule both the living and the dead.


\subsection{Naga}

\medskip\index[Monstery]{Naga Guardian}\textbf{Naga Guardian}

\emph{Large monstrosity, legal good}

\textbf{STRENGTH} +4

\textbf{DEXTERITY} +4

\textbf{CONSTITUTION} +3

\textbf{INTELLIGENCE} +3

\textbf{WISDOM} +4

\textbf{CHARISMA} +4

\textbf{Initiative} +4 -- \textbf{Defense} 23

\textbf{Hit Points} 127 (15d10 + 45)

\textbf{Movement} 12 m

\textbf{Saving Throws}: Fortitude +9, Reflexes +12, Will +12

\textbf{Damage Immunity} Poison

\textbf{Condition Immunity} charmed, poisoned

\textbf{Senses} Darkvision 18 m

\textbf{Languages} Celestial, Common

\textbf{Challenge} 10 (5900 PX)

\emph{\textbf{Spells.}} The naga has CM 11. His spellcasting characteristic is Wisdom (+8 to hit with spell attacks), and he needs only the verbal components to cast his spells. The naga prepares the following spells:

Cantrips (at will): \emph{holy flame, repair, thaumaturgy}

level 1 (4 slots): \emph{command, cure wounds, shield of faith}

level 2 (3 slots): \emph{block people, calm emotions}

level 3 (3 slots): \emph{clairvoyance, cast curse}

level 4 (3 slots): \emph{exile, freedom of movement}

level 5 (2 slots): \emph{Fiery Strike, constraint}

level 6 (1 slot): \emph{true vision}

€22538 € {€22539 € {Reinvigoration.}} If he dies, the naga returns to life in 1d6 days and recovers all his Hit Points. Only the \emph{wish} spell can prevent this trait from working.

\textbf{Shares}

\emph{\textbf{Bite.} Melee weapon attack}: +14 to hit, reach 10 ft., one creature.

\emph{Hit:} 8 (1d8 + 4) piercing damage, and the target must make a DC 15 Fortitude saving throw, taking 45 (10d8) poison damage on a failed save, or half as much damage on a he succeeds.

\emph{\textbf{Spit Poison.} Ranged Weapon Attack}: +14 to hit, range 5m, one creature.

\emph{Hits:} The target must make a Fortitude save DC 15, taking 45 (10d8) poison damage on a failed save, or half as much damage on a successful one.

\textbf{Ecology}\\
Environment: Temperate Plains\\
Organization: Solitary, couple or nest (3-6)\\
\textbf{Treasury}: Standard\\
\textbf{Description}\\
Though ferocious in appearance, with brilliant scales, cobra-like hoods, and powerful serpentine bodies, naga guardians serve as conscientious protectors of places of exceptional power and sacredness. Their scales often sport elaborate designs similar to those of exotic jungle snakes. A typical guardian naga reaches a length of 4.2 meters and an approximate weight of 175 kg.

While some naga guardians adhere to the exotic practices of ancient or forgotten deities, others are simply drawn to sites of marked natural beauty, such as temples on mighty waterfalls, natural pinnacles, and mountaintops, guarding them with the utmost reverence and sense of duty. Often these naga join faiths that are still active, serving as protectors of shrines or ancient treasures. A pair of naga may settle near a site they deem worthy of protection, hatching a brood and raising their offspring there. When young people reach adulthood, they can choose to leave to find their own home or stay to protect the area watched by their parents. Sometimes, a naga guardian guarding a ruin or temple is only the latest in a succession of sentinels that have come and gone over the centuries. These sentinels often take the same name as their predecessors, appearing to be a single, exceptionally long-lived individual.


\medskip\index[Monstery]{Spiritual Naga}\textbf{Spiritual Naga}

\emph{Great monstrosity, chaotic evil}

\textbf{STRENGTH} +4

\textbf{DEXTERITY} +3

\textbf{CONSTITUTION} +2

\textbf{INTELLIGENCE} +3

\textbf{WISDOM} +2

\textbf{CHARISMA} +3

\textbf{Initiative} +3 -- \textbf{Defense} 19

\textbf{Hit Points} 75 (10d10 + 20)

\textbf{Movement} 12 m

\textbf{Saving Throws}: Fortitude +8, Reflexes +10, Will +10

\textbf{Damage Immunity} Poison

\textbf{Condition Immunity} charmed, poisoned

\textbf{Senses} Darkvision 18 m

\textbf{Languages} Abyssal, Municipality

\textbf{Challenge} 8 (3900 XP)

€22571 € {€22572 € {Spells.}} The naga has CM 10. His spellcasting ability is Intelligence (+6 to hit with spell attacks), and he needs only the verbal components to perform his spells spells. The naga prepares the following spells:

Cantrips (at will): \emph{minor illusion, magic hand, ray of} \emph{frost}

level 1 (4 slots): \emph{charm people, detect magic,} \emph{sleep}

level 2 (3 slots): \emph{block people, detect thoughts}

level 3 (3 slots): \emph{lightning, breathing underwater}

level 4 (3 slots): \emph{wither, dimension door}

level 5 (2 slots): \emph{dominate people}

€22581 € {€22582 € {Reinvigoration.}} If he dies, the naga returns to life in 1d6 days and recovers all his Hit Points. Only the \emph{wish} spell can prevent this trait from working.

\textbf{Shares}

\emph{\textbf{Bite.} Melee weapon attack}: +12 to hit, reach 10 ft., one creature.

\emph{Hit:} 7 (1d8 + 4) piercing damage, and the target must make a DC 13 Fortitude saving throw, taking 31 (7d8) poison damage on a failed save, or half as much damage on a he succeeds.

\subsection{Animated Objects}

\medskip\index[Monstery]{Animated Armor}\textbf{Animated Armor}

\emph{Average construct, misaligned}

\textbf{STRENGTH} +2

\textbf{DEXTERITY} +0

\textbf{CONSTITUTION} +1

\textbf{INTELLIGENCE} -5

\textbf{WISDOM} -4

\textbf{CHARISMA} -5

\textbf{Initiative} +0 -- \textbf{Defense} 19

\textbf{Hit Points} 33 (6d8 + 6)

\textbf{Movement} 7 m

\textbf{Saving Throws}: Fortitude +2, Reflexes +0, Will -4

\textbf{Damage Immunity} Poison

\textbf{Condition Immunity} blinded, fascinated, deafened, poisoned, paralyzed, petrified, fatigued, frightened

\textbf{Senses} blindsight 18 m (blind beyond this range)

\textbf{Languages} -

\textbf{Challenge} 1 (200 PX)

\emph{\textbf{False Appearance.}} While the armor remains immobile, it is indistinguishable from normal armor.

\emph{\textbf{Susceptibility to Anti-Magic.}} The armor is incapacitated if it is in the area of ​​a \emph{anti-magic field}. If targeted by \emph{dispel} \emph{spells}, the armor must succeed on a Fortitude saving throw against the spell's save DC or be knocked unconscious for 1 minute.

\textbf{Shares}

\emph{\textbf{Multiattack.}} The armor makes two melee attacks.

\emph{\textbf{Slam.} Melee weapon attack}: +4 to hit, reach 1 m, one target.

\emph{Hits:} 5 (1d6 + 2) bludgeoning damage.

\medskip\index[Monster]{Flying Sword}\textbf{Flying Sword}

\emph{Small construct, misaligned}

\textbf{STRENGTH} +1

\textbf{DEXTERITY} +2

\textbf{CONSTITUTION} +0

\textbf{INTELLIGENCE} -5

\textbf{WISDOM} -3

\textbf{CHARISMA} -5

\textbf{Initiative} +2 -- \textbf{Defense} 18

\textbf{Hit Points} 17 (5d6)

\textbf{Movement} 0 m, fly 15 m (floats)

\textbf{Saving Throws} Fortitude +1, Reflexes +3, Will -4

\textbf{Damage Immunity} Poison

\textbf{Condition Immunity} blinded, charmed, deafened, poisoned, paralyzed, petrified, frightened

\textbf{Senses} blindsight 18 m (blind beyond this range)

\textbf{Languages} -

\textbf{Challenge} 1/4 (50 PX)

\emph{\textbf{False Appearance.}} While the weapon remains immobile and is not flying, it is indistinguishable from a normal sword.

\emph{\textbf{Susceptibility to Anti-Magic.}} The sword is incapacitated if it is in the area of ​​a \emph{anti-magic field}. If targeted by \emph{dispel} \emph{spells}, the sword must succeed on a Fortitude saving throw against the spell's save DC or be knocked unconscious for 1 minute.

\textbf{Shares}

\emph{\textbf{Long Sword.} Melee weapon attack}: +3 to hit, reach 1 m, one target.

\emph{Hits:} 5 (1d8 + 1) slashing damage.


\medskip\index[Monstery]{Carpet of Suffocation}€22655{Carpet of Suffocation}

\emph{Large construct, misaligned}

\textbf{STRENGTH} +3

\textbf{DEXTERITY} +2

\textbf{CONSTITUTION} +0

\textbf{INTELLIGENCE} -5

\textbf{WISDOM} -4

\textbf{CHARRISMA} -5

\textbf{Initiative} +2 -- \textbf{Defense} 13

\textbf{Hit Points} 33 (6d10)

\textbf{Movement} 3 m

\textbf{Saving Throws}: Fortitude +4, Reflexes +2, Will -4

\textbf{Damage Immunity} Poison

\textbf{Condition Immunity} blinded, charmed, deafened, poisoned, paralyzed, petrified, frightened

\textbf{Senses} blindsight 18 m (blind beyond this range)

\textbf{Languages} -

\textbf{Challenge} 2 (450 PX)

\emph{\textbf{False Appearance.}} While the carpet remains immobile, it is indistinguishable from a normal carpet.

\emph{\textbf{Susceptibility to Anti-Magic.}} The carpet is incapacitated while in the area of ​​a \emph{anti-magic field}. If it is the target of \emph{dispel} \emph{spells}, the carpet must succeed on a Fortitude saving throw against the caster's saving throw DC or fall unconscious for 1 minute.

\emph{\textbf{Damage Transfer.}} While grabbing a creature, the carpet takes only half the damage dealt to it, and the creature grabbed by the carpet takes the other half.

\textbf{Shares}

\emph{\textbf{Choke.} Melee weapon attack}: +5 to hit, reach 1 ft., one Medium or smaller creature.

\emph{Hits:} The creature is grappled (DC 13 to escape). Until the grapple ends, the target is entangled, blinded, and at risk of suffocation, but the carpet cannot suffocate another target. Additionally, at the start of each of the target's rounds, the target takes 10 (2d6 + 3) bludgeoning damage.

\medskip\index[Monstery]{Ogre}\textbf{Ogre}

\emph{Great giant, chaotic evil}

\textbf{STRENGTH} +4

\textbf{DEXTERITY} -1

\textbf{CONSTITUTION} +3

\textbf{INTELLIGENCE} -3

\textbf{WISDOM} -2

\textbf{CHARRISMA} -2

\textbf{Initiative} -1 -- \textbf{Defense} 12 (leather armour)

\textbf{Hit Points} 59 (7d10 + 21)

\textbf{Movement} 12 m

\textbf{Saving Throws}: Fortitude +6, Reflexes +0, Will +1

\textbf{Senses} Darkvision 18 m

\textbf{Languages} Common, Giant

\textbf{Challenge} 2 (450 PX)

\textbf{Shares}

\emph{\textbf{Heavy Club.} Melee weapon attack}: +6 to hit, reach 1 m, one target.

\emph{Hits:} 13 (2d8 + 4) bludgeoning damage.

\emph{\textbf{Javelin.} Melee or Ranged Weapon Attack}: +6 to hit, reach 1m or range 12m, one target.

\emph{Hits:} 11 (2d6 + 4) piercing damage.

\textbf{Ecology}\\
Environment: Cold or temperate hills\\
Organization: Solo, couple, group (3-4) or family (5-16)\\
\textbf{Treasure}: Standard (Leather Armor, Heavy Club, 4 Javelins, other)\\
\textbf{Description}\\
There are horrendous elements in stories about ogres: brutality and savagery, cannibalism and torture. Then rapes, dismemberments, necrophilia, incest, mutilations and other examples of cruelty. Those who have never encountered ogres consider these stories a warning. Anyone who has survived such an encounter knows that stories are nothing compared to reality.

Ogres enjoy the suffering of others. If they don't have the smaller breeds at their disposal to crush between their fat hands or to violate in violent embraces, they have fun with each other. For ogres there is no taboo. You might think that, left to their own devices, a tribe of ogres would tear themselves apart and that only the strongest would survive: if there's one thing ogres respect, though, it's family.

Ogre tribes are known as families, and many of their deformities are caused by the common practice of incest. The chief of the tribe is often the father, but in some cases a female ogre is able to claim the title of mother. The ogre tribes argue among themselves, which keeps them busy and prevents them from tormenting their neighbors. Every now and then, however, a particularly violent or feared patriarch emerges, capable of uniting multiple families under his command.

The regions inhabited by ogres are sad and degraded places, as these giants live in squalor and feel no need to be in harmony with their surroundings. The border between the civilized lands and those of the ogres is a place of desperation inhabited by outcasts, where the Ogremans live, deformed offspring born from the raids that the ogres carry out in human lands.

Ogre games are violent and cruel: victims used as toys are lucky to die on the first day. The ogres' cruel sense of humor is the only time they show any creativity: the ogre torture methods and tools seem straight out of nightmares.

Their great strength and lack of imagination make them particularly suited to heavy work, in mines, as blacksmiths or in logging. The most powerful giants (especially those of the Hills and Rocks) often subjugate ogre families to become their servants.

An adult ogre is about 3 meters tall and weighs about 325 kg.


\medskip\index[Monster]{Shadow}\textbf{Shadow}

\emph{Medium undead, chaotic evil}

\textbf{STRENGTH} -2

\textbf{DEXTERITY} +2

\textbf{CONSTITUTION} +1

\textbf{INTELLIGENCE} -2

\textbf{WISDOM} +0

\textbf{CHARISMA} -1

\textbf{Initiative} +2 -- \textbf{Defense} 13

\textbf{Hit Points} 16 (3d8 + 3)

\textbf{Movement} 12 m

\textbf{Saving Throws}: Fortitude +3, Reflexes +3, Will +4

\textbf{Skills} Stealth +4 (+6 to dim light or darkness)

\textbf{Damage Vulnerability} from Light

\textbf{Damage Resistances} acid, cold, lightning, fire, sound; from a non-magical weapon

\textbf{Damage Immunity} from Void, poison

\textbf{Condition Immunity} grabbed, poisoned, entangled, paralyzed, petrified, prone, fatigued, frightened, bleeding

\textbf{Senses} Darkvision 18 m

\textbf{Languages} -

\textbf{Challenge} 1/2 (100 PX)

\emph{\textbf{Amorphous.}} The shadow can move through a space as narrow as 3 centimeters without constricting.

\emph{\textbf{Sunlight Weakness.}} While in sunlight, the shadow has -1d6 on attack rolls, proficiency checks, and saving throws.

\emph{\textbf{Shadow Spirit.}} While in an area of ​​dim light the Shadow regenerates 5 Hit Points at the start of its round, if it is in an area of ​​darkness it regenerates 10 Hit Points at the start of his round and can become invisible using 1 Action. Shadow Spirit increases the Shadow's Challenge Rating by 1.

\emph{\textbf{Shadow Stealth.}} When in dim light or darkness, the shadow can take the hide action as a bonus action.

\emph{\textbf{Undead Nature.}} A shadow requires no air, food, drink, or sleep.

\textbf{Shares}

\emph{\textbf{Force Drain.} Melee weapon attack}: +4 to hit, reach 3 ft., one creature.

\emph{Hits:} 9 (2d6 + 2) Void damage, and the target's Strength score is reduced by 1. The target dies if this reduces its Strength to -5. Otherwise, the reduction remains until the target rests 8 hours.

If a non-evil humanoid dies from this attack, a new shadow will animate its corpse within 1d4 hours.

\textbf{Ecology}
Environment: Any\\
Organization: Solitary, pair, group (3–6) or swarm (7–12)\\
\textbf{Treasury}: Standard\\

\textbf{Description}\\
The evil shadow moves along the border between the darkness of darkness and the harsh truth of light. The shadow prefers to haunt the ruins that civilization leaves behind, where it hunts living creatures foolish enough to stumble into its territory. The shadow is a hideous undead, and as such has no apparent purpose or motivation beyond draining life force and vitality from living beings.


\medskip\index[Monstery]{Homunculus}\textbf{Homunculus}

\emph{Lowercase construct, neutral}

\textbf{STRENGTH} -3

\textbf{DEXTERITY} +2

\textbf{CONSTITUTION} +0

\textbf{INTELLIGENCE} +0

\textbf{WISDOM} +0

\textbf{CHARRISMA} -2

\textbf{Initiative} +2 -- \textbf{Defense} 14

\textbf{Hit Points} 5 (2d4)

\textbf{Movement} 6 m, flight 12 m

\textbf{Saving Throws}: Fortitude +0, Reflexes +4, Will +1

\textbf{Damage Immunity} Poison

\textbf{Condition Immunity} charmed, poisoned

\textbf{Senses} Darkvision 60 ft., blindsight 10 ft

\textbf{Languages} understands the languages ​​of its creator but cannot speak

\textbf{Challenge} 0 (10 PX)

€22774 € {€22775 € {Telepathic Bond.}} While the homunculus is on the same plane of existence as her master, she can magically communicate to her master what he perceives, and the two can communicate telepathically.

\textbf{Shares}

\emph{\textbf{Bite.} Melee weapon attack}: +4 to hit, reach 3 ft., one creature.

\emph{Hits:} 1 piercing damage, and the target must succeed on a DC 10 Fortitude save or be poisoned, -1 Strength and Dexterity, for 1 minute. If the saving throw fails by 5 or more, the target is instead poisoned for 5 (1d10) minutes and is also unconscious while poisoned in this way.

\medskip\index[Monstery]{Oni}\textbf{Oni}

\emph{Large Giant, Lawful Evil}

\textbf{STRENGTH} +4

\textbf{DEXTERITY} +0

\textbf{CONSTITUTION} +3

\textbf{INTELLIGENCE} +2

\textbf{WISDOM} +1

\textbf{CHARISMA} +2

\textbf{Initiative} +2 -- \textbf{Defense} 20 (chainmail)

\textbf{Hit Points} 110 (13d10 + 39)

\textbf{Movement} 9 m, flight 9 m

\textbf{Saving Throws}: Fortitude +7, Reflexes +4, Will +6

\textbf{Skills} Arcane +5, Deception +8, Awareness +4

\textbf{Senses} Darkvision 18 m

\textbf{Languages} Common, Giant

\textbf{Challenge} 7 (2900 XP)

\emph{\textbf{Magical Weapons.}} The oni's weapon attacks are magical.

\emph{\textbf{Innate Spells.}} The oni's spellcasting ability is Charisma. The oni can cast these spells innately, without the need for material components:

At will: \emph{invisibility, darkness}

1/day: \emph{charm on people, cone of cold, gaseous form, sleep}

\emph{\textbf{Regeneration.}} If it has at least 1 hit point, the oni regains 10 hit points at the start of its round.

\textbf{Shares}

\emph{\textbf{Multiattack.}} The oni makes two attacks, with its claws or its glaive.

\emph{\textbf{Claw (Oni Form only).} Melee weapon attack}: +11 to hit, reach 1 m, one target. \emph{Hits:} 8 (1d8 + 4) slashing damage.

\emph{\textbf{Glaffer.} Melee weapon attack}: +7 to hit, reach 10 ft., one target.

\emph{Hits:} 15 (2d10 + 4) slashing damage, or 9 (1d10 + 4) slashing damage in Small or Medium form.

\emph{\textbf{Shapeshifting.}} The oni can magically transform into a Small or Medium humanoid, a Large giant, or return to its true form. Aside from his size, his stats are the same in each form. The only equipment that is transformed is the glaive, which shrinks so that it can also be wielded in humanoid form. If the oni dies, it reverts to its true form, and the glaive returns to its original size.

€22818 € {€22819 € {Angry:}} the Oni is filled with a murderous fury, until the end of the fight his Claw attacks cause Bleeding 2/10.


\medskip\index[Monster]{Orc}\textbf{Orc}

\emph{Medium humanoid (orc), chaotic neutral}

\textbf{STRENGTH} +2

\textbf{DEXTERITY} +1

\textbf{CONSTITUTION} +2

\textbf{INTELLIGENCE} +0

\textbf{WISDOM} +0

\textbf{CHARISMA} +0

\textbf{Initiative} +2 -- \textbf{Defense} 14 (leather armour)

\textbf{Hit Points} 12 (2d6 + 6)

\textbf{Movement} 9 m

\textbf{Saving Throws}: Fortitude +2, Reflexes +2, Will +1

\textbf{Skills} Intimidate +1

\textbf{Senses} Darkvision 18 m

\textbf{Languages} Common, Goblinoid

\textbf{Challenge} 1/2 (100 PX)

\textbf{Shares}

\emph{\textbf{Sword.} Melee weapon attack}: +4 to hit, reach 1 m, one target.

\emph{Hits:} 8 (1d12 + 2) slashing damage.

\emph{\textbf{Javelin.} Melee or Ranged Weapon Attack}: +5 to hit, reach 1m or range 12m, one target. \emph{Hits:} 6 (1d6 + 3) piercing damage.

\textbf{Ecology}\\
Environment: Temperate or underground hills and mountains\\
Organization: solo, group (2-4), squad (11-20 plus 2 3rd level sergeants and 1 3rd-6th level leader) or gang \\
\textbf{Treasure}: NPC Equipment (Studded Leather Armor, Sword, 4 Javelins, other treasure)\\
\textbf{Description}\\
Orcs are a race created by Cattalm as an experiment to see if a creature more intelligent but equally ferocious than orcs could be dominant.
The experiment was quite successful with the orcs founding kingdoms and conquering several regions. The chaotic push with the passage of time, acculturation, becoming sedentary and the evolution of society has brought the orcs further and further outside the confines of Cattalm, even if it does not take away the fact that many barbaric aspects have remained in traditional culture.
An adult male orc is 1.6 meters tall and weighs approximately 60 kg. A peculiar characteristic is the face, and nose in particular, like a pig. Orcs and humans can mate.

\medskip\index[Monster]{Orc}\textbf{Orc}

\emph{Medium humanoid (orc), chaotic evil}

\textbf{STRENGTH} +3

\textbf{DEXTERITY} +1

\textbf{CONSTITUTION} +3

\textbf{INTELLIGENCE} -2

\textbf{WISDOM} +0

\textbf{CHARISMA} +0

\textbf{Initiative} +1 -- \textbf{Defense} 14 (leather armour)

\textbf{Hit Points} 18 (3d8 + 6)

\textbf{Movement} 9 m

\textbf{Saving Throws}: Fortitude +3, Reflexes +1, Will +2

\textbf{Skills} Intimidate +2

\textbf{Senses} Darkvision 18 m

\textbf{Languages} Common, Goblinoid

\textbf{Challenge} 1 (100 PX)

€22868 € {€22869 € {Aggressive.}} As a bonus action, the orc can move up to half its movement toward a hostile creature that it can see.

\textbf{Shares}

\emph{\textbf{Double Axe.} Melee weapon attack}: +5 to hit, reach 1 m, one target.

\emph{Hits:} 9 (1d12 + 3) slashing damage.

\emph{\textbf{Javelin.} Melee or Ranged Weapon Attack}: +5 to hit, reach 1m or range 12m, one target. \emph{Hits:} 6 (1d6 + 3) piercing damage.

\textbf{Ecology}\\
Environment: Temperate or underground hills and mountains\\
Organization: solo, group (2-4), squad (11-20 plus 2 3rd level sergeants and 1 3rd-6th level leader) or gang \\
\textbf{Treasure}: NPC Equipment (Studded Leather Armor, Glaive, 4 Javelins, other treasure)\\
\textbf{Description}\\
The main difference between orcs and civilized humanoids, besides their brute strength and inferior intelligence, is their character. As a culture, orcs are violent and aggressive, and the strong dominate the weak through fear and brutality. They take what they want by force and have no qualms about taking entire villages as slaves if they have the chance. They care nothing for comfort, and their villages and camps tend to be dirty, precarious places, full of drunken brawls, fighting arenas, and other sadistic entertainment. Lacking the patience necessary to cultivate and capable of raising only the most robust and self-sufficient animals, orcs find it easier to take the fruit of their labor from others. They are arrogant and quick to fly into a rage when challenged, but they care about honor only as long as doing so benefits them.

An adult male orc is 2 meters tall and weighs approximately 115 kg. Orcs and humans can mate, although this usually occurs during raids, and not as a consensual union. Many orc tribes breed half-orcs on purpose, as they make excellent strategists and chieftains.

Although popular belief says that orcs were created by Cattalm to destroy and bring chaos, it is also true that very often they are victims of prejudices and summary judgments. Not all orcs are the same and not only physically, individual orcs if not entire tribes live their existence in a normal, civilized manner and yet in no state of Yeru are there penalties for those who kill an orc.

\medskip\index[Monstery]{Wall Climbing Horror}\textbf{Wall Climbing Horror}

\emph{Large monstrosity, misaligned}

\textbf{STRENGTH} +4

\textbf{DEXTERITY} +0

\textbf{CONSTITUTION} +2

\textbf{INTELLIGENCE} -2

\textbf{WISDOM} +1

\textbf{CHARISMA} -2

\textbf{Initiative} +1 -- \textbf{Defense} 15

\textbf{Hit Points} 75 (10d10 + 25)

\textbf{Movement} 9 m, climb 9 m

\textbf{Saving Throws}: Fortitude +5, Reflexes +3, Will +4

\textbf{Senses} darkvision 10 ft., blindsight 60 ft

\textbf{Languages} Wall-Climbing Horror

\textbf{Challenge} 3 (700 PX)

\emph{\textbf{Radar Sense.}} the Wall-Climbing Horror cannot use blindsight if he is deafened.

\textbf{Shares}

\emph{\textbf{Multiattack.}} The Wall-Climbing Horror makes two attacks with its barbed claws.

\emph{\textbf{Claws.} Melee weapon attack}: +7 to hit, reach 1 m, one target.

\emph{Hits:} 10 (2d6 + 4) piercing damage, 1 bleed damage.

\textbf{Ecology}\\
\textbf{Environment: Underground}
Organization: Solitary, pair or pack (3-8)\\
\textbf{Honey}: Accidental\\
\textbf{Description}\\
The Wallcrawler Horror is a ferocious underground predator, aggressively defending its hunting grounds. The subterranean caverns where these creatures reside echo with the thumps and swishes of their hooks as these creatures climb rocky cliffs or cave walls.

A Wall-Climbing Horror is a monstrous creature with a head like that of a vulture and the thorax of an enormous beetle, protected by an exoskeleton studded with sharp bony protrusions. It takes its name not only from its hideous appearance but also from the fact that, using its long, muscular limbs which end in deadly curved hooked claws, it climbs the walls.

\emph{Echoes in the Darkness}. Horror Wallcrawlers communicate by striking their exoskeleton or surrounding rock surfaces with their hooks. What appears to others as a random noise is actually an elaborate language that only the Horror Wall Climbers understand and whose echo spreads for kilometers and kilometers underground.

\emph{Pard of Predators}. Wall Climbing Horrors are omnivorous creatures: they feed on mushrooms, lichens, plants and any creature they can catch. Thanks to their hooked limbs, horrors have excellent grip on rocky surfaces and use their climbing skills to ambush prey from above. They hunt in packs and work together to face the biggest and most dangerous adversaries. If a battle goes badly, a Wall-Climbing Horror quickly climbs up a cave wall to escape.

\emph{Solidarity Clans}. Hook horrors live in large family groups or clans. Each clan is led by the eldest female, who usually places her mate in charge of the clan's hunters. Wall-Climbing Horrors lay their eggs in a central, well-defended area of ​​caves used as their lair.

\medskip\index[Monstery]{Owlbear}\textbf{Owlbear}

\emph{Great beast, misaligned}

\textbf{STRENGTH} +5

\textbf{DEXTERITY} +1

\textbf{CONSTITUTION} +3

\textbf{INTELLIGENCE} -4

\textbf{WISDOM} +1

\textbf{CHARISMA} -2

\textbf{Initiative} +1 -- \textbf{Defense} 15

\textbf{Hit Points} 59 (7d10 + 21)

\textbf{Movement} 12 m

\textbf{Saving Throws}: Fortitude +10, Reflexes +5, Will +2

\textbf{Skills} Awareness +3

\textbf{Senses} Darkvision 18 m

\textbf{Languages} -

\textbf{Challenge} 3 (700 PX)

\emph{\textbf{Hot sense of smell and sight.}} The owlbear has +1d6 on Wisdom (Awareness) checks that rely on smell or sight.

\textbf{Shares}

\emph{\textbf{Multiattack.}} The Owlbear makes two attacks: one with its beak and one with its claws.

\emph{\textbf{Claws.} Melee weapon attack}: +9 to hit, reach 1 m, one target.

\emph{Hits:} 14 (2d8 + 5) slashing damage.

\emph{\textbf{Beak.} Melee weapon attack}: +9 to hit, reach 3 ft., one creature.

\emph{Hits:} 10 (1d10 + 5) piercing damage.

\textbf{Ecology}\\
\textbf{Environment: Temperate Forests}
Organization: Solitary, pair or pack (3-8)\\
\textbf{Honey}: Accidental\\
\textbf{Description}\\
The origins of the Owlbear are a matter of debate among scholars of monstrous creatures. Most of them agree that it was a Wizard, in the past, who created the first example by combining a bear with a giant owl; perhaps as an experiment into some crazy concept of the nature of life, but more likely because of its utter insanity. Whatever the original purpose of such a crazy creation as the Owlbear, the creature has begun to reproduce, and has become one of the woodland's best-known predators.

Owlbears are savage predators, known for their foul tempers, aggression, and ferocity. They tend to attack anything that moves in front of them, even if this shows no belligerent intentions. Many scholars who have encountered these creatures in the wild have noted that they always have bloodshot eyes that spin all around just before an attack. This is generally seen as a sign of madness, suggesting that all Owlbears are born with a pathological need to fight and kill, but more realistic researchers believe it is due to the structure of their sharp eyes.

Owlbears inhabit the innermost and hidden areas of the woods, and prepare their lairs within tangled forests or dark, deep caves. They can hunt both day and night, depending on the habits of the prey that populate the territories surrounding their lair.

Adult Owlbears live in pairs and hunt prey in packs, leaving their cubs in their dens. In a den you can usually find 1d6 hatchlings, which can be worth up to 750 gp in city markets.

Although it is almost impossible to tame them due to their wild nature, Owlbears can be exploited as guardians of a specific territory, provided they are left free to move within it to hunt. Professional trainers charge up to 2,000 gp to train an Owlbear to become a guardian that obeys simple commands (DC 23 for a baby Owlbear, DC 30 for an adult Owlbear).\\

\emph{\textbf{Variant}}: \textbf{Polar Owlbear}\index{Polar Owlbear}\\
This Owlbear is present in arctic or snowy mountain regions. Unlike the normal Owlbear he is more robust and strong. It has 85 Hit Points, +10 to hit, 21 claw damage +1 from bleed, 15 beak damage. CR 4

\medskip\index[Monster]{Wise Owlbear}\textbf{Wise Owlbear}

\emph{Large monstrosity, neutral}

\textbf{STRENGTH} +3

\textbf{DEXTERITY} +1

\textbf{CONSTITUTION} +2

\textbf{INTELLIGENCE} +3

\textbf{WISDOM} +3

\textbf{CHARISMA} +1

\textbf{Initiative} +3 -- \textbf{Defense} 15

\textbf{Hit Points} 45 (7d10 + 10)

\textbf{Movement} 12 m

\textbf{Saving Throws}: Fortitude +10, Reflexes +5, Will +4

\textbf{Skills} Awareness +9

\textbf{Senses} Darkvision 18 m

\textbf{Languages} understands and reads the following languages: Common, Druidic, Celestial, Infernal, Dwarven, Elven, Orc, Giant, Exspiran, Elemental languages

\textbf{Challenge} 3 (700 PX)

\emph{\textbf{Hot sense of smell and sight.}} The wise owlbear has +1d6 on Wisdom (Awareness) checks that rely on smell or sight.

\emph{\textbf{Innate Spells.}} The wise owlbear's spellcasting ability is Intelligence. The wise Owlbear can innately cast the following spells, without requiring material components:

At will: \emph{Magic hand}

\textbf{Shares}

\emph{\textbf{Multiattack.}} The Wise Owlbear makes two attacks: one with its beak and one with its claws.

\emph{\textbf{Claws.} Melee weapon attack}: +7 to hit, reach 1 m, one target.

\emph{Hits:} 14 (2d8 + 5) slashing damage.

\emph{\textbf{Beak.} Melee weapon attack}: +7 to hit, reach 3 ft., one creature.

\emph{Hits:} 10 (1d10 + 5) piercing damage.

\textbf{Ecology}\\
\textbf{Environment: Temperate Forests}
Organization: Solitary, pair or pack (3-8)\\
\textbf{Treasure}: Standard + 10\% Manuals and Tomes\\
\textbf{Description}\\
The origins of the wise Owlbear are as mysterious as those of its unwise relative but enthusiasts of these creatures trace them to descend directly from Nethergal as a variant of the original Owlbear.
Usually the wise Owlbear loves to surround himself with books and adores the company of other wise men but does not disdain the tales of adventurers and the captivating ballads of storytellers. The wise Owlbear has a real talent for languages ​​and despite not being able to speak in a manner understandable to a man he is able to understand many spoken and written languages ​​and in a few days he is able to learn new ones (such as Vantaggio Lingua Universale) both spoken and written. The wise Owlbear can read any language or code if he has the opportunity to study it for 3 days.
Usually weaker and more fragile than their close relative, they are nevertheless fearsome beings in combat.
Preferably, a wise Owlbear does not attack except in defense and seeks an approach that is as tactical and useful as possible. A characteristic feature of wise Owlbears is a scarf worn around the absent neck. Killing a wise Owlbear is an affront to the Devotees and Followers of Nethergal, it has also happened that the Patron himself took away the ability to communicate from those who have committed cruel acts with his favorite creatures.

Training a wise Owlbear is much easier than an Owlbear but the creature's high intelligence will push it to be an equal or a familiar rather.

The Magic Hand spell is usually used to leaf through more delicate tomes and to write, albeit extremely slowly.

\medskip\index[Monstery]{Otyugh}\textbf{Otyugh}

\emph{Large aberration, neutral}

\textbf{STRENGTH} +3

\textbf{DEXTERITY} +0

\textbf{CONSTITUTION} +4

\textbf{INTELLIGENCE} +2

\textbf{WISDOM} +1

\textbf{CHARISMA} -2

\textbf{Initiative} +0 -- \textbf{Defense} 17

\textbf{Hit Points} 114 (12d10 + 48)

\textbf{Movement} 9 m

\textbf{Saving Throws}: Fortitude +3, Reflexes +2, Will +6

\textbf{Senses} darkvision 36 m

\textbf{Languages} Otyugh

\textbf{Challenge} 5 (1800 PX)

\emph{\textbf{Limited Telepathy.}} The otyugh can magically transmit simple messages and images to any creature within 120 feet of it that can understand a language. This form of telepathy does not allow the receiving creature to respond telepathically.

\textbf{Shares}

\emph{\textbf{Multiattack.}} The otyugh makes three attacks: one with its bite and two with its tentacles.

\emph{\textbf{Bite.} Melee weapon attack}: +9 to hit, reach 1 m, one target.

\emph{Hits:} 12 (2d8 + 3) piercing damage. If the target is a creature, it must succeed on a DC 15 Fortitude saving throw vs. disease or remain ill until the disease is cured. Every 24 hours thereafter, the target must repeat the saving throw, reducing its hit point maximum by 5 (1d10) on a failed save. If the saving throw succeeds, the disease is over. The target dies if the disease reduces its maximum hit points to 0.

This reduction in the character's maximum hit points lasts until the disease is cured.

\emph{\textbf{Tentacle.} Melee weapon attack}: +9 to hit, reach 10 ft., one target.

\emph{Hits:} 7 (1d8 + 3) bludgeoning damage plus 4 (1d8) piercing damage. If the target is Medium or smaller, it is grappled (DC 13 to escape) and entangled until the grapple ends. The otyugh has two tentacles, each of which can grasp a different target.

\emph{\textbf{Tentacle Smash.}} The otyugh smashes creatures grasped by its tentacles, into each other or onto the floor. Each creature must succeed on a DC 14 Fortitude saving throw or take 10 (2d6 + 3) bludgeoning damage and be stunned until the end of the otyugh's next round. On a successful save, the target takes half bludgeoning damage and is not stunned.

\emph{\textbf{Angry:}} The otyugh emits a scent that clouds the senses. All creatures within a 20-foot radius must make a DC 18 Will save or act randomly, as a \hyperlink{inconfusion}{Confusion} spell (p. \pageref{inconfusion}), until the end of the next round. It costs 2 Actions.

\textbf{Ecology}\\
Environment: Any Dungeon\\
Organization: Solo, couple or group (3-4)\\
\textbf{Treasury}: Standard\\
\textbf{Description}\\
Otyughs are particularly filthy and hideous creatures that live in places that sane people tend to avoid. Their lairs are found in sewers, cesspools, landfills and the most mephitic swamps: the dirtier a place, the more it attracts otyughs. They love the role of scavenger, wandering underground caverns looking for new tidbits among the waste. Once found, they gorge themselves and take back to their den what they cannot consume in one go. Otyughs spend a lot of time in their filthy lairs, which they fill with carrion and dung, which releases noxious odors.

Intelligent creatures that live underground near otyughs sometimes form alliances of convenience with them. They provide their waste and raw meat to the otyugh, making them a veritable means of disposal. In return, otyughs leave their benefactors alone, do not attack them, and can even act as guardians.

What most races find terrifying about otyughs is not their diet or the smell of their lairs, but the fact that creatures with their tastes are not just mindless scavengers. The otyughs are in fact surprisingly intelligent, and love to form alliances with those who supply them with foods more refined than manure and dirt. Most otyughs realize that other creatures find them revolting, but few truly care.

\medskip\index[Monstery]{Panoptikhan}\textbf{Panoptikhan}

\emph{Great aberration, evil}

\textbf{STRENGTH} +0

\textbf{DEXTERITY} +1

\textbf{CONSTITUTION} +2

\textbf{INTELLIGENCE} +3

\textbf{WISDOM} +2

\textbf{CHARISMA} +2

\textbf{Initiative} +5 -- \textbf{Defense} 26

\textbf{Hit Points} 82 (11d8 + 38)

\textbf{Movement} 1 m, flight 10 meters (good)

\textbf{Saving Throws}: Fortitude +14, Reflexes +13, Will +16

\textbf{Resistance}: acid, electricity

\textbf{Senses} darkvision 36m, true vision 18m

\textbf{Languages} telepathy 50 m

\textbf{Challenge} 12 (8400 PX)

\textbf{Shares}

\emph{\textbf{Multiattack.}} The Panoptikhan can attack with two short tentacles.

\emph{\textbf{Tentacle.} Melee weapon attack}: +12 to hit, reach 1 m, one target.

\emph{Hits:} 6 (1d6 + 3) piercing slash damage.

\emph{\textbf{He who sees everything}}. The Panoptikhan can activate one of its eyed tentacles (2 Actions). The Panoptikhan has CM 14.

\emph{The One That Freezes}: The eye focuses on a target within 18 meters, a ray of frost is activated on it. 8d8 cold damage, Reflex save DC 23 to completely avoid the blow.

\emph{The one that melts}: the eye focuses on a target within 9 meters, a beam that has acid effects is activated on this. 4d8 acid damage, Reflex save DC 23 to halve damage.

\emph{The one that burns}: the eye focuses on a target within 18 meters, a fiery ray is activated on it. 8d8 fire damage, Reflex save DC 23 to completely avoid the hit.

\emph{The one that paralyzes}: the eye focuses on a target within 9 meters, a ray is activated on this target that paralyzes the creature. Will save DC 23 to completely avoid the effects.

\emph{The one that slows down}: the eye points in a 9 meter cone. A slowing ray is projected onto affected creatures. Will save DC 23 to completely avoid the effects. Duration 1 minute.

\emph{What is confusing}: the eye points in an 18 meter cone. A ray is projected onto affected creatures, causing confusion. Will save DC 23 to completely avoid the effects. Duration 1 minute, each round you can make a new saving throw to recover from the effects.

\emph{The one that puts you to sleep}: the eye focuses on a target within 36 meters, a ray is activated on this which puts the creature to sleep. Will save DC 23 to completely avoid the effects.

\emph{What moves}; this eye can manifest the spell Magic Hand or Telekinesis.

\emph{\textbf{One look.}} The Panoptikhan activates the central eye. The central eye can be used as a Reaction Action to cast Counterspell on a spell it sees being cast.

\emph{\textbf{Angry:}} the Panoptikhan in the grip of his blindest fury activates 1d6 random eyes on random targets. It costs 3 Actions.

\textbf{Ecology}\\
Environment: Any Dungeon\\
Organization: Solitary, couple\\
\textbf{Treasure}: Triple\\
\textbf{Description}\\
The Panoptikhan are xenophobic aberrations, balls of hard flying flesh equipped with a large central eye, a large mouth and 7 tentacles about 1 meter long, each with an eye (about 10 cm in diameter) of a different color.

Little is known about the origin of the Panoptikhan, they are thought to be an evolutionary experiment by Calicante, in an attempt to create a sentient, dominant race.

Unfortunately, arrogance, pride and the desire to be the center of attention caused these attempts at society to founder and the Panoptikhan disappeared underground.

The Panoptikhans have a very long lifespan, in the order of a thousand years, but they are also creatures that have more than doubled this limit. Panoptikhans increase in size with age and so do the number of eyes. The statistics reported here refer to an adult specimen of approximately 300 years of age.


\medskip\index[Monster]{Pegasus}\textbf{Pegasus}

\emph{Great celestial, chaotic good}

\textbf{STRENGTH} +4

\textbf{DEXTERITY} +2

\textbf{CONSTITUTION} +3

\textbf{INTELLIGENCE} +0

\textbf{WISDOM} +2

\textbf{CHARISMA} +1

\textbf{Initiative} +2 -- \textbf{Defense} 13

\textbf{Hit Points} 59 (7d10 + 21)

\textbf{Movement} 18 m, flight 27 m

\textbf{Saving Throws} Fortitude +7, Reflexes +6, Will +4

\textbf{Skills} Awareness +6

\textbf{Languages} includes Celestial, Common, Elvish and Sylvan but cannot speak

\textbf{Challenge} 2 (450 PX)

\textbf{Shares}

\emph{\textbf{Hooves.} Melee weapon attack}: +6 to hit, reach 1 m, one target.

\emph{Hits:} 11 (2d6 + 4) bludgeoning damage.

\textbf{Ecology}
Environment: Temperate and Warm Plains\\
Organization: Solitary, pair or pack (6-10)\\
\textbf{Treasure}: None\\
\textbf{Description}\\
Pegasus is a magnificent winged horse that sometimes serves the cause of good. While highly prized as flying mounts, pegasi are shy creatures who rarely make friends. A typical pegasus stands 1.8 meters tall at the withers, weighs 750 kg and has a wingspan of 6 meters. Most pegasi are white, but sometimes some pegasi have different colors.

Pegasus, despite appearances, is as intelligent as a human. Those who try to train one to be a mount will find that the pegasus is recalcitrant and even violent. A pegasus cannot speak, but understands Common and prefers the company of good creatures. The correct way to convince a Pegasus to be your mount is to befriend him with Diplomacy, favors and good deeds. A pegasus normally has an indifferent attitude towards good creatures, ill-disposed towards neutral ones and hostile towards evil ones. Before it can serve as a mount, a pegasus must be made friendly via a Diplomacy check or otherwise. Riding a pegasus requires an exotic saddle or bareback riding, as a normal saddle interferes with its wings. A pegasus can fight while carrying a knight, but the knight cannot attack back unless he succeeds at a Ride check. Trained pegasi do not fear combat, and the rider does not need to make a Ride check to control them.

Pegasi lay eggs that are worth 1000 gp each on the market, while hatchlings fetch 2000 gp each. Being intelligent and good creatures, selling eggs and young is essentially slavery: in good societies those who do so are despised or punished by law.

Pegasi mature like horses. Professional trainers charge 1000 to train a pegasus, who will serve a good or neutral knight faithfully for life.

A light load for a pegasus is up to 150 kg; an average load is 150.5-300 kg; a heavy load is 300.5-450 kg.

In some pegasi the blood of an ancestor who was a heroic stallion is still strong. These champions have the lifespan of a human, perfect maneuverability, fire resistance 10, a +4 racial bonus on saving throws against poison and immunity to petrification, an additional +4 on attack rolls, +4 on defense, +25 HP, +4 to all saving throws and deal +1d6 additional damage. Some can say a few words in Celestial or Common. They realize their superiority to other horses and pegasi, they do not have to be trained to fly with a rider, but only allow the greatest heroes to ride them.

\medskip\index[Monster]{Invisible Persecutor}\textbf{Invisible Persecutor}

\emph{Medium elemental, neutral}

\textbf{STRENGTH} +3

\textbf{DEXTERITY} +4

\textbf{CONSTITUTION} +2

\textbf{INTELLIGENCE} +0

\textbf{WISDOM} +2

\textbf{CHARISMA} +0

\textbf{Initiative} +4 -- \textbf{Defense} 17

\textbf{Hit Points} 104 (16d8 + 32)

\textbf{Move} 15 m, fly 15 m (float)

\textbf{Saving Throws}: Fortitude +13, Reflexes +11, Will +4

\textbf{Skills} Stealth +10, Awareness +8

\textbf{Damage Resistances} from non-magical weapons

\textbf{Damage Immunity} Poison

\textbf{Condition Immunity} grabbed, poisoned, entangled, paralyzed, petrified, unconscious, prone, fatigued

\textbf{Senses} Darkvision 18 m

\textbf{Languages} Ictun, understands the Municipality but does not speak it

\textbf{Challenge} 6 (2300 XP)

\emph{\textbf{Infallible Hunter.}} The summoner assigns prey to the stalker. The stalker knows the direction and distance of the prey as long as both are on the same plane of existence. The stalker also knows the location of his summoner.

\emph{\textbf{Invisibility.}} The persecutor is invisible.

\emph{\textbf{Elemental Nature.}} An invisible stalker has no need for air, food, drink, or sleep.

\textbf{Shares}

\emph{\textbf{Multiattack.}} The stalker makes two slam attacks.

\emph{\textbf{Slam.} Melee weapon attack}: +12 to hit, reach 1 m, one target.

\emph{Hits:} 10 (2d6 + 3) bludgeoning damage.

\emph{\textbf{Angry:}} the Invisible Persecutor breaks the pact and returns to the elemental plane of air.

\textbf{Ecology}
Environment: Any\\
Organization: Solitaire\\
\textbf{Treasure}: None\\
\textbf{Description}\\

Native to the Plane of Air, these creatures move through the world following errands for those who summon them. Invisible Hunters usually act as guardians and assassins. Their natural invisibility and stealth allow them to follow their prey without being seen and benefit them when they decide to eliminate a target.

Many invisible hunters, however, view these tasks as mere demands on mortals. If they are given a particularly complex or unpleasant task, an invisible hunter will try to find a loophole if the instruction is poorly worded. For example, Wizards who summon an invisible hunter with the instruction “protect me from danger” might be escorted to a distant hidden location, or even taken to the Plane of Air.

Due to the constant summoning, many invisible hunters oppose the inhabitants of the Material Plane. Those newly summoned to the mortal world know only the stories of their peers and maintain an open attitude towards those who call them. Over time, or if they serve an evil master, they begin to form a negative opinion of these deadly creatures, which leads them to pervert instructions and harm their masters. For older and more experienced invisible hunters, the only thing that protects their summoner is the magic that binds them. These creatures always attempt to use inconsistencies in the wording of their tasks and literal distortions in intentions to find a way to annoy, hurt, or even kill those who brought them to this plane.


\medskip\index[Monstery]{Pseudodragon}\textbf{Pseudodragon}

\emph{Tiny dragon, neutral good}

\textbf{STRENGTH} -2

\textbf{DEXTERITY} +2

\textbf{CONSTITUTION} +1

\textbf{INTELLIGENCE} +0

\textbf{WISDOM} +1

\textbf{CHARISMA} +0

\textbf{Initiative} +2 -- \textbf{Defense} 14

\textbf{Hit Points} 7 (2d4 + 2)

\textbf{Movement} 5 metres, flight 18 m

\textbf{Saving Throws}: Fortitude +4, Reflexes +5, Will +4

\textbf{Skills} Stealth +4, Awareness +3

\textbf{Senses} Darkvision 60 ft., blindsight 10 ft

\textbf{Languages} understands Common and Draconic but does not speak

\textbf{Challenge} 1/4 (50 XP)

\emph{\textbf{Magic Resistance.}} The pseudodragon has +1d6 on saving throws against spells and other magical effects.

\emph{\textbf{Honed Senses.}} The pseudodragon has +1d6 on Wisdom (Awareness) checks that rely on sight, hearing, and smell.

\emph{\textbf{Limited Telepathy.}} The pseudodragon can communicate simple ideas, emotions, and images telepathically with any creature within 100 feet of it that can understand a language.

\textbf{Shares}

\emph{\textbf{Bite.} Melee weapon attack}: +4 to hit, reach 1 m, one target.

\emph{Hits:} 4 (1d4 + 2) piercing damage.

\emph{\textbf{Sting.} Melee weapon attack}: +4 to hit, reach 3 ft., one creature.

\emph{Hits:} 4 (1d4 + 2) piercing damage, and the target must succeed on a DC 11 Fortitude save or be poisoned, -1 Strength and Dexterity, for 1 hour. If the saving throw result is 6 or less, the target falls unconscious for the same duration, or until it takes damage or another creature uses an action to awaken it.

\textbf{Ecology}\\
Environment: Temperate forests\\
Organization: Solitary, couple or nest (3-5)\\
\textbf{Treasury}: Standard\\
\textbf{Description}\\
Pseudodragons are small relatives of true dragons, playful and shy. They speak by chirping, hissing, growling, and purring, but can communicate telepathically with any intelligent creature. If approached peacefully with offers of food, they are willing to share information about what is found in their territory, but threats and violence make them flee.

Pseudodragons are carnivores and eat insects, rodents, birds and snakes, although they eat eggs and love butter, cheese and fish. Sometimes they hunt on the ground like lizards or by flying like predatory birds. Intelligent like most humanoids, they do not like to be treated like pets, and prefer to be considered friends. They distrust evil creatures, can join spellcasters and Devotees as Familiars, and some have befriended Druids and rangers or collaborate with good dragons as sentinels. Pseudo-dragons become Familiars only if they appreciate the caster's personality (and if the caster has the Familiar Skill and Charisma at least 1), but they can also bond with people whose company they appreciate. A pseudodragon might follow a character in this way for days, weeks, years, or even a lifetime, provided they are well fed and treated with affection.

Upon reaching adulthood, a pseudodragon's body is 30 centimeters long with a 60 centimeter tail, and weighs approximately 3.5 kg. Pseudodragon eggs are as large as chicken eggs, but leathery in texture and spotted brown, and females lay them in batches of 2-5 each spring. A nest of pseudodragons (which are a family group, and are not hatched from the same batch of eggs) usually consists of a pair of adults and several near-adult hatchlings.


\medskip\index[Monstery]{Rakshasa}\textbf{Rakshasa}

\emph{Medium fiendish, lawful evil}

\textbf{STRENGTH} +2

\textbf{DEXTERITY} +3

\textbf{CONSTITUTION} +4

\textbf{INTELLIGENCE} +1

\textbf{WISDOM} +3

\textbf{CHARISMA} +5

\textbf{Initiative} +3 -- \textbf{Defense} 23

\textbf{Hit Points} 110 (13d8 + 52)

\textbf{Movement} 12 m

\textbf{Saving Throws}: Fortitude +17, Reflexes +16, Will +16

\textbf{Skills} Deceive +10, Sense Emotions +8

\textbf{Damage Vulnerability} piercing magical weapons wielded by
good creatures

\textbf{Damage Immunity} bludgeoning, weapons +1

\textbf{Senses} Darkvision 18 m

\textbf{Languages} Common, Infernal

\textbf{Challenge} 13 (10000 PX)

€23,188 € {€23,189 € {Magic Immunity Limited.}} The rakshasa is immune to affect or detection by spells of 6th or lower level unless he wishes to be subject to them. He has +1d6 on saving throws against all other spells and magical effects.

\emph{\textbf{Innate Spells.}} The rakshasa's spellcasting characteristic is Charisma (+10 to hit with spell attacks). The rakshasa can innately cast the following spells, without requiring material components:

At will: \emph{disguise self, minor illusion, detect thoughts, magic hand}

3/Day each: \emph{charm people, major image,} \emph{detect magic, invisibility, suggestion} 1/Day: \emph{dominate people, planar movement, seeing the truth, flying}

\textbf{Shares}

\emph{\textbf{Multiattack.}} The rakshasa can make two claw attacks.

\emph{\textbf{Claw.} Melee weapon attack}: +13 to hit, reach 1 m, one target.

\emph{Hits:} 9 (2d6 + 2) slashing damage, and if the target is a creature it remains cursed. The magical curse takes effect whenever the target rests, filling the target's thoughts with horrific images and dreams. The cursed target receives no benefit from finishing a rest. The curse lasts until removed by a \emph{remove curse} spell or similar magic.

\textbf{Ecology}
Environment: Any\\
Organization: Solitary, couple or cult (3-12)\\
\textbf{Treasure}: Double (Dagger+1, more treasure)\\
\textbf{Description}\\
The rakshasa is an evil spirit that disguises itself as a humanoid creature so it can stalk its prey incognito. An embodiment of the taboos of most societies and capable of taking on the appearance of those it seeks to corrupt, a rakshasa performs many horrific deeds. If they were human, their blasphemy, cannibalism, and even worse acts would mark them as criminals deserving of the cruelest of hells.

When in no other guise, the rakshasa appears as a humanoid with the head of an animal. It often has the head of a large cat (such as tigers or panthers) or snake (such as cobras or vipers) and, although it is rarer, it can have the head of a gorilla, jackal, vulture, elephant, mantis, lizard, rhinoceros, wild boar and many still others. In many cases, the type of head a rakshasa possesses says something about its personality: a tiger-headed rakshasa is stealthy and ravenous, while one with a boar's head can be gluttonous and cruel. These differences rarely affect the rakshasa's base statistics, although more powerful variants than the standard exist with multiple heads, more powerful magical powers, and additional strange and deadly special abilities.

Rakshasas despise religions; they recognize the power of the gods, but see themselves as the only beings worthy of veneration by the mortal races. Rakshasa Devotees are therefore quite rare. While rakshasas are outsiders, they are also creatures of the Material Plane, and some believe that the early rakshasas chose this exile over some other role offered to them by a long-forgotten god. While they are generally solitary, it is not uncommon to find large families of rakshasas working together to bring about the downfall of a mortal civilization from within, over many generations.

A rakshasa is 1.8 meters tall and weighs 90 kg.

\medskip\index[Monstrorium]{Deadreavers}\textbf{Deadreavers}

\emph{Large construct, undead, unaligned}

\textbf{STRENGTH} +5

\textbf{DEXTERITY} +0

\textbf{CONSTITUTION} +4

\textbf{INTELLIGENCE} -4

\textbf{WISDOM} -2

\textbf{CHARISMA} -5

\textbf{Initiative} +2 -- \textbf{Defense} 21

\textbf{Hit Points} 105 (10d10 + 50)

\textbf{Movement} 9 m

\textbf{Saving Throws} Fortitude +15, Reflexes +9, Will +7

\textbf{Awareness} +4

\textbf{Damage Immunity} Poison

\textbf{Condition Immunity} poisoned, charmed, fatigued, paralyzed, petrified, bleeding

\textbf{Senses} darkvision 30 m

\textbf{Languages} understands all the creatures' languages ​​but cannot speak

\textbf{Challenge} 6 (2300 XP)

\emph{\textbf{Undead Nature.}} The Death Reaper has no need for air, food, drink, or sleep.

€23229 € {€23230 € {Immutable Form.}} As a construct he cannot be affected by spells or effects that change his form.

\emph{\textbf{Container.}} The Morti Reaper has an opening compartment with a door on the metal back that can contain up to 100kg of objects, large to small.

\emph{\textbf{Air Resistance.}} The Death Reaper has innate resistance to spells from the Air Magic List.

\emph{\textbf{Sensitive to Fire.}} The Death Reaper takes one less action the following round if it takes fire damage.

\textbf{Shares}

\emph{\textbf{Multiattack.}} The Death Reaper attacks with two claws or attacks with one claw and uses Paralyzing Eye.

\textbf{\emph{Chela.}} +9 to hit, reach 1 meter

\emph{Hit}: 16 (2d10 + 5) bludgeoning damage

\emph{\textbf{Paralysing Eye}}: The affected creature, within 60 feet, must make a Fortitude save at DC 16 or be paralyzed for 2d4 rounds.

\textbf{Ecology}\\
Environment: Any, caves\\
Organization: 1-2 Deathreavers, 1d4+1 Guardians\\
\textbf{Treasure}: How much collected (Triple)\\
\textbf{Description}\\
The Deadreavers are particular undead built from pieces of various corpses and pieces of iron to resemble some kind of large armored crabs.
The back, completely metallic, acts as a container for the treasures that the Razziamorti finds, the claws, in a variable number between 6 and 8, are just over a meter long and have the characteristic of each leaving a different imprint as they are assembled from pieces of metal and different bodies.

The large central eye, perhaps once belonging to a humanoid, allows the controller and builder of the Death Reaper to see and command it. The purpose of a Deathreaper is to explore, usually a cave system or path, searching for the remains of past raiders and adventurers for magical items and treasures.

Usually a Deadreaper is always accompanied by several guardians (other creatures at the command of the controller) who help him in \emph{fixing} any \emph{resistances} still active.


\medskip\index[Monstery]{Remorhaz}\textbf{Remorhaz}

\emph{Huge monstrosity, misaligned}

\textbf{STRENGTH} +7

\textbf{DEXTERITY} +1

\textbf{CONSTITUTION} +5

\textbf{INTELLIGENCE} -3

\textbf{WISDOM} +0

\textbf{CHARISMA} -3

\textbf{Initiative} +1 -- \textbf{Defense} 23

\textbf{Hit Points} 195 (17d12 + 85)

\textbf{Movement} 9 m, excavation 6 m

\textbf{Saving Throws}: Fortitude +11, Reflexes +7, Will +4

\textbf{Damage Immunity} Cold, Fire

\textbf{Senses} Darkvision 60 ft., telluric sense 60 ft.

\textbf{Languages} -

\textbf{Challenge} 11 (7200 PX)

\emph{\textbf{Heated Body.}} A creature that comes into contact with the remorhaz or hits it with a melee attack while within 3 feet of it takes 10 (3d6) fire damage.

\textbf{Shares}

\emph{\textbf{Bite.} Melee weapon attack}: +18 to hit, reach 10 ft., one target.

\emph{Hits:} 40 (6d10 + 7) piercing damage plus 10 (3d6) fire damage. If the target is a creature, it is grappled (DC 17 to escape). Until the grapple ends, the target is entangled, and the remorhaz cannot bite another target.

\emph{\textbf{Swallow.}} The remorhaz makes a bite attack against a Medium or smaller target it is grappling. If the attack hits, the creature takes bite damage and is swallowed, and the grapple ends. The engulfed target is blinded and restrained, has full cover against attacks and other effects outside the remorhaz, and takes 21 (6d6) acid damage at the start of each round of the remorhaz.

If the remorhaz takes 30 or more damage in a single round from a creature within it, the remorhaz must succeed on a DC 15 Fortitude save at the end of that round or vomit all swallowed creatures, which fall prone in a space within 3 meters from the remorhaz. If the remorhaz dies, an engulfed creature is no longer restrained by it and can exit the corpse using 15 feet of movement, exiting prone.

\emph{\textbf{Angry:}} the Remorhaz heats its body even more until the end of the fight, bringing the fire damage to 18 (6d6) for those within 1 meter.

\textbf{Ecology}\\
Environment: Cold Deserts and Glaciers\\
Organization: Solitaire\\
\textbf{Treasure}: None\\
\textbf{Description}\\
In a world of ice and snow, remorhaz are particularly feared for the terrible fire that burns within their bodies. This internal fire causes the plates along its back to become red-hot when the creature is particularly angry, excited, or panicked. Creatures that have adapted to arctic regions are often particularly vulnerable to fire, making the remorhaz's primary defense incredibly powerful and securing its role as a dangerous predator of icy areas. Remorhaz live in vast labyrinths carved into the heart of glaciers. These beasts use their heat to carve tunnels through the ice, tunnels whose smooth glass walls quickly refreeze in their wake, creating numerous incredibly stable warrens.

Although the remorhaz has much in common with smaller surface parasites, this beast is surprisingly intelligent. Although unable to speak, the typical remorhaz understands the Giant well, and giant tribes often take advantage of this to form alliances with these beasts. Frost Giants are particularly obsessed with them; these giants face the cruel and deadly burns that a remorhaz can inflict to become friends with the worm while obtaining a powerful weapon to use against their enemies: fire. Other giants use these beasts as living forges, as their backs are hot enough to melt metal.

A remorhaz is 7 meters long and weighs 5000 kg.



\medskip\index[Monster]{Rust Eater}\textbf{Rust Eater}

\emph{Medium Monstrosity, misaligned}

\textbf{STRENGTH} +1

\textbf{DEXTERITY} +1

\textbf{CONSTITUTION} +1

\textbf{INTELLIGENCE} -4

\textbf{WISDOM} +1

\textbf{CHARISMA} -2

\textbf{Initiative} +1 -- \textbf{Defense} 15

\textbf{Hit Points} 27 (5d8 + 5)

\textbf{Movement} 12 m

\textbf{Saving Throws}: Fortitude +2, Reflexes +4, Will +5

\textbf{Senses} Darkvision 18 m

\textbf{Languages} -

\textbf{Challenge} 1/2 (100 PX)

\emph{\textbf{Smell for Iron.}} The rust monster can detect, by smell, the exact location of ferrous metals within 36 meters.

\emph{\textbf{Rust Metal.}} Any nonmagical weapon made of metal that hits the rust monster corrodes after applying damage. Nonmagical ammunition made of metal that hits the rust monster is considered destroyed after dealing damage.

\textbf{Shares}

\emph{\textbf{Bite.} Melee weapon attack}: +3 to hit, reach 1 m, one target.

\emph{Hits:} 5 (1d8 + 1) piercing damage.

\emph{\textbf{Antennas.}} The rust monster corrodes nonmagical ferrous metal objects that it can see that are within 1 meter. If the object is not worn or carried, contact with the rust monster destroys a 30-centimeter cube of it. If the item is worn or carried by a creature, the creature can make a DC 11 Reflex save to avoid contact with the rust monster.

If the object it comes into contact with is worn or carried metal armor or shield, it takes a permanent, cumulative -2 penalty to the Defense it provides. Armor reduced to Defense 0 or shields that drop to a bonus of +0 are destroyed. If the object it comes into contact with is a metal weapon held by someone, it rusts it as described in the Rust Metal trait.

\textbf{Ecology}
Environment: Any Dungeon\\
Organization: Solitary, couple or nest (3-10)\\
\textbf{Treasure}: Accidental (no metal treasure)\\
\textbf{Description}\\
Of all the terrifying beasts an explorer might encounter underground, only the rust monster targets what the average adventurer values ​​most: his treasure.

Typically 1 meter long and weighing at least 100 kg, the rustophage resembles a crustacean and would be scary enough even without the alien nutritional process from which it takes its name. Rust monsters eat metal objects, preferring those made of iron and ferrous alloys such as steel, but they also devour mithral, ​​adamantium, and enchanted metals with equal ease. Any metal touched by the rust monster's delicate antennae or armored skin corrodes and turns to dust within seconds, making it one of the most feared beasts among subterranean adventurers and Dwarf miners who must defend their forges and compete with them for gold. gold.

While rust monsters have no natural tendency for violence, their insatiable hunger drives them to charge at anything that comes near them with enough metal on them, and any resistance is met with unexpected ferocity. It is not uncommon for rust mongers in metal-poor areas to track fleeing victims for days using their metal-sniffing ability, as long as the victims still have intact metal objects.\\
Fortunately, it is often possible to escape the attentions of a rust monster by throwing a dense metal object, such as a shield, at it and running in the opposite direction. Those who frequent rust-eating areas quickly learn to keep wooden or stone weapons handy.


\medskip\index[Monstery]{Sahuagin}\textbf{Sahuagin}

\emph{Medium humanoid (sahuagin), lawful evil}

\textbf{STRENGTH} +1

\textbf{DEXTERITY} +0

\textbf{CONSTITUTION} +1

\textbf{INTELLIGENCE} +1

\textbf{WISDOM} +1

\textbf{CHARRISMA} -1

\textbf{Initiative} +1 -- \textbf{Defense} 13

\textbf{Hit Points} 22 (4d8 + 4)

\textbf{Movement} 9m, swim 12m

\textbf{Saving Throws}: Fortitude +4, Reflexes +4, Will +4

\textbf{Skills} Awareness +5

\textbf{Senses} darkvision 36 m

\textbf{Languages} Sahuagin

\textbf{Challenge} 1/2 (100 PX)

\emph{\textbf{Limited Amphibian.}} The sahuagin can breathe air and water, but must remain submerged at least once every 4 hours to avoid suffocation.

\emph{\textbf{Blood Frenzy.}} The sahuagin has +1d6 on melee attack rolls against any creature that is not at full hit points.

\emph{\textbf{Telepathy with Sharks}}. The sahuagin can magically command any shark within 120 feet of him, using a limited form of telepathy.

\textbf{Shares}

\emph{\textbf{Multiattack.}} The sahuagin can make two melee attacks: one with its bite and one with its claws or spear.

\emph{\textbf{Claws.} Melee weapon attack}: +3 to hit, reach 1 m, one target.

\emph{Hits:} 3 (1d4 + 1) slashing damage.

\emph{\textbf{Spear.} Melee or Ranged Weapon Attack}: +3 to hit, reach 1m or range 6m, one target.

\emph{Hits:} 4 (1d6 + 1) piercing damage, or 5 (1d8 + 1) piercing damage if used with two hands to make a melee attack.

\emph{\textbf{Bite.} Melee weapon attack}: +3 to hit, reach 1 m, one target.

\emph{Hits:} 3 (1d4 + 1) piercing damage.

\textbf{Ecology}\\
Environment: Temperate or Warm Oceans\\
Organization: Solo, duo, squad (5-8), patrol (11-20 plus 1 3rd level lieutenant and 1-2 Sharks), gang (20-80 plus 100\% non-combatants, 1 3rd level lieutenant and 1 4th level captain for every 20 adults, and 1-2 Sharks) or tribes (70-160 plus 100\% non-combatants, 1 3rd level lieutenant for every 20 adults, 1 4th level captain for every 40 adults, 9 guards of 4th level, 1-4 novices of 3rd-6th level, 1 priestess of 7th level, 1 baron of 6th-8th level, and 5-8 Sharks)
\textbf{Treasure}: NPC equipment (Trident, Heavy Crossbow with 10 Bolts, other treasure)\\
\textbf{Description}\\
Ravenous and cruel, sahuagin are, unfortunately, among the most prosperous of oceanic races. Great cities have been built by this race in the dark depths of the ocean trenches, and some fortresses stand near the coasts from where they launch continuous assaults against the air-breathing enemies who live near the shore. Proud and warlike, sahuagin rarely ally with others, and view other aquatic races, such as aboleths, merfolk, and the like as competitors. The only creatures they seem to respect other than their own kind are sharks; in these relentless predators, in fact, the sahuagin see much of themselves. A sahuagin stands 2.1 meters tall and weighs approximately 125 kg.

Sahuagin are subject to genetic mutations, and when a mutant is born he almost always rises to the noble or command ranks in society. The most common sahuagin mutation consists of an extra pair of arms (granting two additional claw attacks or the ability to wield multiple weapons). Some speak of the rare malenti, sahuagin who do not appear to be sharkmen but aquatic elves, although they share the bloodlust and cruel nature of their kin. Malenti often serve as spies or assassins for Sahuagin rulers, but there are tales of entire tribes of malenti in remote areas of the sea.


\medskip\index[Monster]{Salamander}\textbf{Salamander}

\emph{Large elemental, neutral evil}

\textbf{STRENGTH} +4

\textbf{DEXTERITY} +2

\textbf{CONSTITUTION} +2

\textbf{INTELLIGENCE} +0

\textbf{WISDOM} +0

\textbf{CHARISMA} +1

\textbf{Initiative} +2 -- \textbf{Defense} 18

\textbf{Hit Points} 90 (12d10 + 24)

\textbf{Movement} 9 m

\textbf{Saving Throws}: Fortitude +10, Reflexes +7, Will +6

\textbf{Damage Vulnerability} cold

\textbf{Damage Resistances} from non-magical weapons

\textbf{Damage Immunity} Fire

\textbf{Senses} Darkvision 18 m

\textbf{Languages} Ignan

\textbf{Challenge} 5 (1800 PX)

\emph{\textbf{Heated Weapons.}} Any metallic melee weapon the salamander wields deals an additional 3 (1d6) fire damage per hit (already included in the attack).

\emph{\textbf{Heated Body.}} A creature that comes into contact with the salamander or hits it with a melee attack while within 3 feet of it takes 7 (2d6) fire damage.

\textbf{Shares}

\emph{\textbf{Multiattack.}} The salamander makes two attacks: one with its spear and one with its tail.

\emph{\textbf{Tail.} Melee weapon attack}: +10 to hit, reach 10 ft., one target.

\emph{Hits:} 11 (2d6 + 4) bludgeoning damage plus 7 (2d6) fire damage, and the target is grappled (DC 14 to escape). Until the grapple ends, the target is entangled, the salamander can automatically strike the target with its tail, and the salamander cannot make tail attacks against other targets.

\emph{\textbf{Spear.} Melee or ranged weapon attack}: +9 to hit, reach 1m, range 6m, one target.

\emph{Hits:} 11 (2d6 + 4) piercing damage, or 13 (2d8 +4) piercing damage if used with two hands to make a melee attack, plus 3 (1d6) fire damage.

\emph{\textbf{Angry:}} the Salamander focuses its flames into a ranged attack. A creature within 30 feet must make a DC 18 Reflex saving throw for half damage. The creature is hit by an orb of fire that deals 4d6 fire damage. It costs 2 Actions.

\textbf{Ecology}
Environment: Any (Plane of Fire)\\
Organization: Solo, couple or group (3-5)\\
\textbf{Treasure}: Standard (Spear, other non-flammable treasure)\\
\textbf{Description}\\
The Salamanders are native to the Plane of Fire, where their legions of fierce fighters are greatly feared by the other inhabitants of the Plane. Since many of the stronger Fire Elemental Races enslave the Salamanders for their metalworking prowess and fighting ability, the Salamanders hate the Efreet and others with a fervor.

Although their hiding places exceed 250 degrees C in temperature, Salamanders can tolerate lower temperatures. They generally do so if forced, and are also more gruff and irascible than normal in these environments. Although hailing from the Plane of Fire, the Salamander Race identifies more with the Abyss, and has a great respect for Demons (particularly those associated with fire, such as the Balor and certain flame-bound Demon Lords). This is why it is not unusual to encounter a large group of Salamanders in the Abyss.

Salamanders are often summoned into the Material Plane to serve as guardians or, more commonly, as makers of armor, weapons, and other metallurgical items, as their skill in this field is legendary. Salamanders also infest those areas of the Material Plane where the boundary between this world and the Plane of Fire has become blurred, such as near and within Volcanoes.

Inhabiting such extreme areas, Salamanders only possess treasures that resist high temperatures, such as Swords, Armor, jewelry, Rods, and other items that have a high melting point. Salamander society is cruel and based on power and the ability to subjugate those inferior to them. Beings less than Salamanders who cause trouble face a slow and painful death.



\medskip\index[Monster]{Satyr}\textbf{Satyr}

\emph{Medium fairy, chaotic neutral}

\textbf{STRENGTH} +1

\textbf{DEXTERITY} +3

\textbf{CONSTITUTION} +0

\textbf{INTELLIGENCE} +1

\textbf{WISDOM} +0

\textbf{CHARISMA} +2

\textbf{Initiative} +3 -- \textbf{Defense} 15 (leather armour)

\textbf{Hit Points} 31 (7d8)

\textbf{Damage Vulnerability} cold iron

\textbf{Movement} 12 m

\textbf{Saving Throws}: Fortitude +4, Reflexes +8, Will +8

\textbf{Skills} Stealth +5, Perform +6, Awareness +2

\textbf{Languages} Common, Elvish, Sylvan

\textbf{Challenge} 1/2 (100 PX)

\emph{\textbf{Magic Resistance.}} The satyr has +1d6 on saving throws against spells and other magical effects.

\textbf{Shares}

\emph{\textbf{Gored.} Melee weapon attack}: +3 to hit, reach 1 m, one target.

\emph{Hits:} 6 (2d4 + 1) bludgeoning damage.

\emph{\textbf{Short Sword.} Melee weapon attack}: +5 to hit, reach 1 m, one target.

\emph{Hits:} 6 (1d6 + 3) piercing damage.

\emph{\textbf{Short Bow.} Ranged Weapon Attack}: +5 to hit, range 24m, one target.

\emph{Hits:} 6 (1d6 + 3) piercing damage.

\textbf{Ecology}\\
Environment: Temperate Forests\\
Organization: Solo, couple, gang (3-6) or party (7-11)\\
\textbf{Treasure}: Standard (Dagger, Short Bow plus 20 Arrows, perfect pan flute, more treasure)\\
\textbf{Description}\\
Satyrs, known in many regions as fauns, are debauched and hedonistic creatures of the deepest, most primal parts of the forests. They adore wine, music and the pleasures of the flesh, they are renowned as libertines and dacoits who court naive girls and shepherd boys and leave behind a trail of embarrassing explanations and unwanted pregnancies.

Although their bodies are almost always those of attractive, well-proportioned men, satyrs' seductive abilities lie in their musical talent. With the aid of his flute, a satyr is capable of weaving a vast array of melodic spells designed to charm others into complying with his capricious desires.

In addition to their constant flirting, satyrs often serve as guardians of their forests, and those who manage to turn the faun's lust into anger will likely find themselves up against the most dangerous of the animals surrounding the faun. Furthermore, although satyrs tend to place their own entertainment above the rights of others, they harbor no resentment against those they seduce.

The children born from these encounters are always pure-blooded satyrs and are usually taken away by their wanton fathers soon after birth.


\medskip\index[Monster]{Skeleton}\textbf{Skeleton}

\emph{Medium undead, lawful evil}

\textbf{STRENGTH} +0

\textbf{DEXTERITY} +2

\textbf{CONSTITUTION} +2

\textbf{INTELLIGENCE} -2

\textbf{WISDOM} -1

\textbf{CHARRISMA} -3

\textbf{Initiative} +2 -- \textbf{Defense} 14 (armor pieces)

\textbf{Hit Points} 13 (2d8 + 4)

\textbf{Movement} 9 m

\textbf{Saving Throws}: Fortitude +0, Reflexes +2, Will +2

\textbf{Damage Vulnerability} bludgeoning

\textbf{Damage Resistances} piercing, cutting

\textbf{Damage Immunity} Poison

\textbf{Condition Immunity} poisoned, fatigue, bleeding

\textbf{Senses} Darkvision 18 m

€23446 € {Languages} includes all the languages ​​he spoke in life but cannot speak

\textbf{Challenge} 1/4 (50 XP)

\emph{\textbf{Undead Nature.}} The skeleton does not require air, food, drink or sleep.

\textbf{Shares}

\emph{\textbf{Short Sword.} Melee weapon attack}: +4 to hit, reach 1 m, one target.

\emph{Hits:} 5 (1d6 + 2) piercing damage.

\emph{\textbf{Short Bow.} Ranged Weapon Attack}: +4 to hit, range 24m, one target.

\emph{Hits:} 5 (1d6 + 2) piercing damage.

\textbf{Ecology}\\
Environment: Any\\
Organization: Any\\
\textbf{Treasure}: None (Broken Mail Jacket, Broken Scimitar)\\
\textbf{Description}\\
Skeletons are the bones of the animated dead, brought to unlife by unholy magic. For the most part, skeletons are willless automatons, but they possess an evil cunning granted to them by the force that animates them: a cunning that allows them to bear weapons and wear armor.

\medskip\index[Monstery]{War Horse Skeleton}\textbf{War Horse Skeleton}

\emph{Great undead, lawful evil}

\textbf{STRENGTH} +4

\textbf{DEXTERITY} +1

\textbf{CONSTITUTION} +2

\textbf{INTELLIGENCE} -4

\textbf{WISDOM} -1

\textbf{CHARRISMA} -3

\textbf{Initiative} +1 -- \textbf{Defense} 14 (barding pieces)

\textbf{Hit Points} 22 (3d10 + 6)

\textbf{Movement} 18 m

\textbf{Saving Throws}: Fortitude +4, Reflexes +3, Will +1

\textbf{Damage Vulnerability} bludgeoning

\textbf{Damage Resistances} piercing, cutting

\textbf{Damage Immunity} Poison

\textbf{Condition Immunity} poisoned, fatigue, bleeding

\textbf{Senses} Darkvision 18 m

\textbf{Languages} -

\textbf{Challenge} 1/2 (100 PX)

\emph{\textbf{Undead Nature.}} The skeleton requires no air, food, drink, or sleep.

\textbf{Shares}

\emph{\textbf{Hooves.} Melee weapon attack}: +6 to hit, reach 1 m, one target.

\emph{Hits:} 11 (2d6 + 4) bludgeoning damage.

\medskip\index[Monstery]{Minotaur Skeleton}\textbf{Minotaur Skeleton}

\emph{Great undead, lawful evil}

\textbf{STRENGTH} +4

\textbf{DEXTERITY} +0

\textbf{CONSTITUTION} +2

\textbf{INTELLIGENCE} -2

\textbf{WISDOM} -1

\textbf{CHARRISMA} -3

\textbf{Initiative} +0 -- \textbf{Defense} 13

\textbf{Hit Points} 67 (9d10 + 18)

\textbf{Movement} 12 m

\textbf{Saving Throws}: Fortitude +6, Reflexes +3, Will +2

\textbf{Damage Vulnerability} bludgeoning

\textbf{Damage Immunity} Poison

\textbf{Damage Resistances} piercing, cutting

\textbf{Condition Immunity} poisoned, fatigue, bleeding

\textbf{Senses} Darkvision 18 m

\textbf{Languages} understands the Abyssal but cannot speak

\textbf{Challenge} 2 (450 PX)

\emph{\textbf{Charge.}} If the minotaur skeleton moves at least 10 feet in a straight line towards the target and then hits it with a gore attack during the same round, the target suffers 9 (2d8) additional piercing damage. If the target is a creature, it must succeed on a DC 14 Fortitude save or be pushed 10 feet back and fall prone.

\emph{\textbf{Undead Nature.}} The skeleton does not require air, food, drink or sleep.

\textbf{Shares}

\emph{\textbf{Double Axe.} Melee weapon attack}: +6 to hit, reach 1 m, one target.

\emph{Hits:} 17 (2d12 + 4) slashing damage.

\emph{\textbf{Gored.} Melee weapon attack}: +6 to hit, reach 1 m, one target.

\emph{Hits:} 13 (2d8 + 4) piercing damage.

\medskip\index[Monster]{Hell Hound}\textbf{Hell Hound}

\emph{Medium fiendish, lawful evil}

\textbf{STRENGTH} +3

\textbf{DEXTERITY} +1

\textbf{CONSTITUTION} +2

\textbf{INTELLIGENCE} -2

\textbf{WISDOM} +1

\textbf{CHARRISMA} -2

\textbf{Initiative} +1 -- \textbf{Defense} 17

\textbf{Hit Points} 45 (7d8 + 14)

\textbf{Movement} 15 m

\textbf{Saving Throws}: Fortitude +6, Reflexes +5, Will +1

\textbf{Skills} Awareness +5

\textbf{Damage Immunity} Fire

\textbf{Senses} Darkvision 18 m

\textbf{Languages} understands the Infernal but cannot speak

\textbf{Challenge} 3 (700 PX)

\emph{\textbf{Refined Hearing and Smell.}} The hound has +1d6 on Wisdom (Awareness) checks that rely on hearing or smell.

\emph{\textbf{Pack Tactics.}} The hound has +1d6 on attack rolls against a creature if at least one of the hound's allies is within 3 feet of the creature and that ally is not incapacitated.

\textbf{Shares}

\emph{\textbf{Bite.} Melee weapon attack}: +7 to hit, reach 1 m, one target.

\emph{Hits:} 7 (1d6 + 3) piercing damage plus 7 (2d6) fire damage.

\emph{\textbf{Fiery Breath (Recharge 5-6).}} The hound exhales fire in a 5 meter cone. Each creature in that area must make a DC 12 Reflex saving throw, and take 21 (6d6) fire damage on a failed save, or half as much damage on a successful one.



\subsection{Sphinxes}

\medskip\index[Monstruary]{Androsphinx}\textbf{Androsphinx}

\emph{Large monstrosity, lawful neutral}

\textbf{STRENGTH} +6

\textbf{DEXTERITY} +0

\textbf{CONSTITUTION} +5

\textbf{INTELLIGENCE} +3

\textbf{WISDOM} +4

\textbf{CHARISMA} +6

\textbf{Initiative} +3 -- \textbf{Defense} 26

\textbf{Hit Points} 199 (19d10 + 95)

\textbf{Movement} 12 m, flight 18 m

\textbf{Saving Throws}: Fortitude +22, Reflexes +17, Will +21

\textbf{Skills} Arcana +9, Awareness +10, Religion +15

\textbf{Damage Immunity} from non-magical weapons

\textbf{Condition Immunity} fascinated, scared

\textbf{Senses} true vision 36 m

\textbf{Languages} Common, Sphinx

\textbf{Challenge} 17 (18000 PX)

\emph{\textbf{Magical Weapons.}} The sphinx's weapon attacks are magical.

\emph{\textbf{Inscrutable.}} The sphinx is immune to any effects that sense its emotions or read its thoughts, as well as any divination spells that you reject. Wisdom (Sense Emotion) checks to discern the sphinx's intentions or sincerity are -1d6.

\emph{\textbf{Spells.}} The sphinx has CM 12.
His spellcasting ability is Wisdom (spell save DC 18, +10 to hit with spell attacks). He does not need material components to cast his spells. The sphinx has the following spells prepared:

Cantrips (at will): \emph{sacred flame, saving the dying,} \emph{thaumaturgy}

level 1 (4 slots): \emph{command, detect magic,} \emph{detect evil and good}

level 2 (3 slots): \emph{lower refreshment, zone of truth}

level 3 (3 slots): \emph{dispel magic, languages}

level 4 (3 slots): \emph{exile, freedom of movement}

level 5 (2 slots): \emph{Fiery Strike, greater restoration}

level 6 (1 slot): \emph{heroes' banquet}

\textbf{Shares}

\emph{\textbf{Multiattack.}} The sphinx can make two claw attacks.

\emph{\textbf{Claw.} Melee weapon attack}: +17 to hit, reach 1 m, one target.

\emph{Hits:} 17 (2d6 + 10) slashing damage, 1 bleed damage.

\emph{\textbf{Roar (3/Day).}} The sphinx lets out a magical roar. Each time it roars before a new dawn the roar is louder and the effect is different, as detailed below. Each creature within 500 feet of the sphinx and capable of hearing its roar must make a saving throw.

\textbf{First Roar.} Any creature that fails a DC 18 Will save is frightened for 1 minute. A frightened creature can repeat the saving throw at the end of each of its rounds, ending the effect on itself if it succeeds.

\textbf{Second Roar.} Any creature that fails a DC 18 Will save is deafened and frightened for 1 minute. A frightened creature is paralyzed and can repeat the saving throw at the end of each of its rounds, ending the effect on itself if it succeeds.

\textbf{Third Roar.} Each creature makes a Fortitude save DC 18. Anyone who fails the save takes 44 (8d10) sonic damage and is knocked prone. On a successful save, the creature takes half as much damage and is not knocked prone.

\textbf{Additional Shares}

The sphinx can perform 3 additional Actions, chosen from the following options. It can use only one Additional option at a time, and only at the end of another creature's round. The sphinx recovers additional Actions spent at the start of its round.

\textbf{Claw Attack.} The sphinx makes a claw attack.

\textbf{Perform a Spell (Costs 3 Actions).} The sphinx casts a spell from the list of prepared spells, using a spell slot as normal.

\textbf{Teleport (Costs 2 Actions).} The sphinx magically teleports, along with any equipment it is wearing or carrying, to an unoccupied space it can see, up to 120 feet away.

\emph{\textbf{Angry:}} the Sphinx asks a riddle. The creature must respond within 6 rounds, if it fails or does not respond it must make a DC 31 Will save or be paralyzed. Each round he can attempt the saving throw again in an attempt to give an answer. Costs 1 Action.


\textbf{Ecology}\\
Environment: Hills or Hot Deserts\\
Organization: Solitaire\\
\textbf{Treasury}: Standard\\
\textbf{Description}\\
The androsphinxes, the most powerful of the common sphinxes, believe that they represent all that is worthy and noble in their species and act as if the weight of the entire world rests on their good example. They look at the Cryosphinxes with paternalistic disdain, the Hieracosphinxes with poorly concealed disgust and the Gynosphinxes as the only other sphinxes worthy of their time.

Androsphinxes display a grumpy and spiteful façade towards foreigners. They make no effort to hide their annoyance when irritated. They also tend to be jealous of their territory, although less so than other sphinxes. They almost inevitably issue warnings and bombastic proclamations before attacking, and almost always respect a call to negotiate. Androsphinxes trade information and conversation, not treasure, for safe passage.

Androsphinxes are 3.6 meters tall and weigh 500 kg.


\medskip\index[Monstruary]{Ginosphinx}\textbf{Ginosphinx}

\emph{Great monstrosity, lawful neutral}

\textbf{STRENGTH} +4

\textbf{DEXTERITY} +2

\textbf{CONSTITUTION} +3

\textbf{INTELLIGENCE} +4

\textbf{WISDOM} +4

\textbf{CHARISMA} +4

\textbf{Initiative} +4 -- \textbf{Defense} 23

\textbf{Hit Points} 136 (16d10 + 48)

\textbf{Movement} 12 m, flight 18 m

\textbf{Saving Throws}: Fortitude +11, Reflexes +9, Will +10

\textbf{Skills} Arcana +14, Awareness +9, Religion +9, History +14

\textbf{Damage Resistances} from non-magical weapons

\textbf{Condition Immunity} fascinated, scared

\textbf{Senses} true vision 36 m

\textbf{Languages} Common, Sphinx

\textbf{Challenge} 11 (7200 PX)

\emph{\textbf{Magical Weapons.}} The sphinx's weapon attacks are magical.

\emph{\textbf{Inscrutable.}} The sphinx is immune to any effects that sense its emotions or read its thoughts, as well as any divination spells that you reject. Wisdom (Sense Deception) checks to discern the sphinx's intentions or sincerity are -1d6.

\emph{\textbf{Spells.}} The sphinx has CM 9. Its spellcasting ability is Intelligence (spell save DC 17. It needs no material components to cast its spells. The sphinx has the following spells prepared: Cantrips (at will): \emph{minor illusion, magic hand,} \emph{prestidigitation}

level 1 (4 slots): \emph{identify, detect magic, shield}

level 2 (3 slots): \emph{locate object, darkness, suggestion}

level 3 (3 slots): \emph{dispel magic, languages, remove curse}

level 4 (3 slots): \emph{exile, greater invisibility}

level 5 (2 slots): \emph{legend knowledge}

\textbf{Shares}

\emph{\textbf{Multiattack.}} The sphinx can make two claw attacks.

\emph{\textbf{Claw.} Melee weapon attack}: +11 to hit, reach 1 m, one target.

\emph{Hits:} 13 (2d8 + 4) slashing damage, 1 bleed damage.

\textbf{Additional Shares}

The sphinx can perform 3 additional Actions, chosen from the following options. She can only use one Additional option at a time, and only at the end of another creature's round. The sphinx recovers additional Actions spent at the start of its round.

\textbf{Claw Attack.} The sphinx makes a claw attack.

\textbf{Perform a Spell (Costs 3 Actions).} The sphinx casts a spell from the list of prepared spells, using a spell slot as normal.

€23651 €{Teleport (Costs 2 Actions).} The sphinx magically teleports, along with any equipment she is wearing or carrying, to an unoccupied space she can see, up to 120 feet away.

\textbf{Ecology}
Environment: Hot deserts and hills\\
Organization: Solitary, couple or cult (3-6)\\
\textbf{Treasure}: Double\\
\textbf{Description}\\
While there are several types of sphinx, the one that scholars refer to as Gynosphinx (a name that many sphinxes find offensive) is a wise and majestic yet terrifying creature when angry. Less moralistic than their male counterparts (the Androsphinxes, totally different creatures from the one presented here), sphinxes are prudent and methodical when making decisions, and pride themselves on their cold logic and impartiality. They have little patience with lesser variants of sphinxes, considering them little more than animals. Sphinxes love complicated riddles and riddles, and treasure unusual facts and arcane dilemmas more than gold or gems.

While not great scholars in the traditional sense, sphinxes' great appreciation for enigmas leads them to research a wide variety of subjects, often making them a valuable source of information, especially when making use of their magical abilities. They are usually happy to have contact with other races, and regularly offer material goods in exchange for new and interesting information or riddles. They make excellent guardians of temples, tombs, and other important places, as long as they are entertained properly. Sphinxes place great importance on kindness, but can be capricious: they may altruistically decide to share their latest puzzles with travelers but do not think twice about devouring them if they do not pay enough attention or do not provide any useful clues to their resolution.

A typical sphinx is 3 meters long and weighs around 400 kg. Although their wings can keep them in the air for long periods of time, they are poor fliers, preferring to land before engaging in combat, attacking with their powerful claws. Despite being extremely territorial, sphinxes tend to warn intruders several times before attacking.

\medskip\index[Monster]{Hissing}\textbf{Hissing}

\emph{Large monstrosity, chaotic}

\textbf{STRENGTH} +2

\textbf{DEXTERITY} +1

\textbf{CONSTITUTION} +1

\textbf{INTELLIGENCE} -3

\textbf{WISDOM} +0

\textbf{CHARRISMA} -2

\textbf{Initiative} +1 -- \textbf{Defense} 14

\textbf{Hit Points} 32 (5d10+5)

\textbf{Movement} 6 m, climb 6 m

\textbf{Saving Throws}: Fortitude +3, Reflexes +3, Will +2

\textbf{Skills} Stealth +4, Awareness +3

\textbf{Senses} Darkvision 18 m

\textbf{Fine Senses}: The Hisser has +1d6 on Awareness checks based on hearing or smell

\textbf{Languages}: -

\textbf{Challenge} 2 (450 PX)

\textbf{Shares}

\emph{\textbf{Multiattack.}} The Hissing One can make two attacks with its claws or one strike with its tail.

\emph{\textbf{Claw.} Melee weapon attack}: +4 to hit, reach 1 m, one target.

\emph{Hits:} 6 (1d8+2) slashing damage.

\emph{\textbf{Tail Whip}}: The Hissing Whisper swings its long tail and strikes a target.

\emph{Hits:} 11 (2d8+2) bludgeoning and 7 (2d6) slashing damage, reach 10 feet. In the event of a critical hit, any armor or shield is damaged, lowering the opponent's Defense by 1. Armor damage is not considered permanent.

\textbf{Reactions}

\emph{\textbf{Knock Down}}: when the Hissing is attacked by a creature within its tail's reach, it is lashed, forcing the attacker, after resolving its attack, to make a Fortitude/Reflex saving throw at DC 12 or take 7 (2d6) bludgeoning damage and fall prone. If the saving throw succeeds I only take half damage and I'm not prone.

\textbf{Ecology}\\
Environment: Caverns\\
Organization: Solitary, couple or nest (2-4)\\
\textbf{Treasure}: random\\

\textbf{Description}

The Sibilants, so called because of the noise their tail makes when wagging, are a very particular creature. At first glance it resembles a crocodile, about 5 meters long, 4 of which is a tail, but has 8 legs and a short, flattened snout. The extremely robust tail ends with a kind of hook that the Hissing uses to hit, kill and grab enemies as if it were an additional paw.

Dark grey, brown in colour, they prefer to hide in the darkness and attack when hungry or to defend their territory. They try to keep their distance in combat and if seriously wounded they escape by climbing the walls.


\medskip\index[Monster]{Spiritello}\textbf{Spiritello}

\emph{Lowercase fairy, neutral good}

\textbf{STRENGTH} -4

\textbf{DEXTERITY} +4

\textbf{CONSTITUTION} +0

\textbf{INTELLIGENCE} +2

\textbf{WISDOM} +1

\textbf{CHARISMA} +0

\textbf{Initiative} +4 -- \textbf{Defense} 16 (leather armour)

\textbf{Hit Points} 2 (1d4)

\textbf{Damage Vulnerability} cold iron

\textbf{Movement} 3 m, flight 12 m

\textbf{Saving Throws}: Fortitude +0, Reflexes +5, Will +2

\textbf{Skills} Stealth +8 (the check is made with -1d6 if the sprite is flying), Awareness +3

\textbf{Languages} Common, Elvish, Sylvan

\textbf{Challenge} 1/4 (50 XP)

\textbf{Shares}

\emph{\textbf{Long Sword.} Melee weapon attack}: +2 to hit,

range 1 m, one target.

\emph{Hits:} 1 slashing damage.

\emph{\textbf{Short Bow.} Ranged Weapon Attack}: +6 to hit, range 12 m, one target.

\emph{Hits:} 1 piercing damage. If the target is a creature, it must succeed on a DC 10 Fortitude save or be poisoned, -1 Strength and Dexterity, for 1 minute. If the result of this saving throw is 5 or less, the target falls unconscious for the same duration, or until it takes damage or another creature uses an action to awaken it.

\emph{\textbf{Invisibility.}} The sprite remains invisible until it attacks or ends its concentration. Anything the imp is carrying or wearing remains invisible as long as it is in contact with the imp.

\emph{\textbf{Heart Sight.}} The sprite comes into contact with a creature and learns its current emotional state. If the target fails a DC 10 Fortitude save, the sprite also learns the creature's Traits. Celestials, fiends, and undead automatically fail this saving throw.

\textbf{Description}\\
Sprites gather in groups deep in forested regions, united in the cause to protect nature. Entire tribes of sprites have declared themselves protectors of a particular person, place or creature of particular importance in their lands, even if the being does not want or need any protection.

A sprite's body is naturally luminous, although the creature can vary the color and intensity of the light emitted from its body at will. Immediately after his death, a sprite's body dissolves into a shimmering mist. The sprites are the smallest of the elves, just over 22 centimeters tall and weighing rarely more than 1 kg.

In many ways sprites are more primitive than most sprites. They enjoy the company of their own kind, but tend to distrust other fey and assume that any humanoid or creature they have not specifically chosen to protect means them harm. Even animals are usually considered dangerous by them. The reason for this distrust is largely due to the tiny size of these creatures, which makes them easy prey for predators. Therefore, a sprite's initial reaction to danger is to flee: it typically uses its magical abilities to slow down or distract its pursuers, and then relies on its flying speed and size to help it escape.

Although sprites themselves have an uncultivated and wild nature, they have a healthy curiosity for all things with innate magic. They are particularly attracted to places of great latent magical power, such as the ruins of ancient temples. This curiosity also makes them unusually suited to the role of familiars. A 5th-level neutral chaotic spellcaster can gain a faerie as a familiar if he has the Familiar skill.


\medskip\index[Monstery]{Striga (Stygian Bird)}\textbf{Striga (Stygian Bird)}

\emph{Tiny beast, misaligned}

\textbf{STRENGTH} -3

\textbf{DEXTERITY} +3

\textbf{CONSTITUTION} +0

\textbf{INTELLIGENCE} -4

\textbf{WISDOM} -1

\textbf{CHARRISMA} -2

\textbf{Initiative} +3 -- \textbf{Defense} 15

\textbf{Hit Points} 2 (1d4)

\textbf{Movement} 3 m, flight 12 m

\textbf{Saving Throws}: Fortitude +2, Reflexes +6, Will +1

\textbf{Senses} Darkvision 18 m

\textbf{Languages} -

\textbf{Challenge} 1/8 (25 PX)

\textbf{Shares}

\emph{\textbf{Blood Drain.} Melee Weapon Attack}: +5 to hit, reach 3 ft., one creature.

\emph{Hits:} 5 (1d4 + 3) piercing damage and the striga attaches to the target. While attacked, the striga does not attack. Instead, at the start of each striga round, the target loses 5 (1d4 + 3) hit points due to blood loss.

The striga can detach itself by spending 1 meter of movement. He does this automatically after draining 10 Hit Points from the target or upon the target's death. A creature, including the target, can use its action to detach the striga.

\textbf{Ecology}
Environment: Temperate and warm swamps\\
Organization: Solitary, colony (2-4), flock (5-8), cloud (9-14) or swarm (15-40)\\
\textbf{Treasure}: None\\
\textbf{Description}\\
Strigae are dangerous bloodsuckers that infest swamps and prey on wild animals, livestock and unsuspecting travelers. While individually weak, swarms of these creatures are capable of draining a man dry in minutes, leaving only a desiccated corpse in their wake.

More like mammals than insects, strigae soar on their four fleshy wings, seeking warm-blooded prey. They often hide near pools of drinkable water waiting for travelers to lower their guard before attacking them and drinking their fill, sticking their trunks into exposed veins. After feeding, they fly away to hide in the mud and reeds to lay their eggs and rest until hunger drives them to hunt again.

Strigae are usually about 30 centimeters long, with a wingspan of about double that, and weigh less than 0.5 kg. They are rusty red or reddish brown, and have dirty yellow bellies, but those that have not fed adequately are pale pink.

\medskip\index[Monster]{Succubus}\textbf{Succubus}

\emph{Medium fiend (shapeshifter), neutral evil}

\textbf{STRENGTH} -1

\textbf{DEXTERITY} +3

\textbf{CONSTITUTION} +1

\textbf{INTELLIGENCE} +2

\textbf{WISDOM} +1

\textbf{CHARISMA} +5

\textbf{Initiative} +3 -- \textbf{Defense} 17

\textbf{Hit Points} 66 (12d8 + 12)

\textbf{Movement} 9 m, flight 18 m

\textbf{Saving Throws}: Fortitude +7, Reflexes +9, Will +10

\textbf{Skills} Stealth 5, Sense Emotions +5, Awareness +5, Deception +9

\textbf{Damage Resistances} cold, lightning, fire, poison; from a non-magical weapon

\textbf{Senses} Darkvision 18 m

\textbf{Languages} Abyssal, Common, Infernal, telepathy 18 m

\textbf{Challenge} 4 (1100 PX)

€23,766 € {€23,767 € {Telepathic Bond.}} The fiend ignores the range restrictions of his telepathy when communicating with a creature he has charmed. The two are not even forced to be on the same plane of existence.

€23768 € {€23769 € {Shapeshifter.}} The fiend can use his action to transform into a Small or Medium humanoid, or to return to his true form. Without wings, the fiend loses its flying speed. Aside from size and speed, his stats are the same across all forms. Whatever equipment he is wearing or carrying is not transformed. Upon death he returns to his true form.

\textbf{Shares}

\emph{\textbf{Claw (Fiendish Form only).} Melee weapon attack}: +6 to hit, reach 1 m, one target.

\emph{Hits:} 6 (1d6 + 3) slashing damage.

\emph{\textbf{Charm.}} A humanoid visible to the fiend within 30 feet of it must succeed on a DC 15 Will save or be magically charmed for 1 day. The charmed target obeys the fiend's verbal or telepathic commands. If the target takes damage or receives a suicide command, it can repeat the saving throw, ending the effect on a success. If the target succeeds on its saving throw against the effect, or if the effect ends, the target is immune to the fiend's charm for the next 24 hours.

The fiend can charm only one target at a time. If you charm another, the effect on the previous target ends.

\emph{\textbf{Sucking Kiss.}} The fiend kisses a fascinated creature or a willing creature. The target must make a DC 14 Fortitude saving throw against this spell, taking 32 (5d10 + 5) damage on a failed save, or half as much damage on a successful one. The target's maximum hit points are reduced by an amount equal to the damage taken. This reduction lasts until dawn breaks. The target dies if this effect reduces its maximum hit points to 0.

\emph{\textbf{Ethereal Form.}} The fiend magically enters the Ethereal Plane from the Material Plane, and vice versa.

\textbf{Ecology}\\
Environment: Any (Abyss)\\
Organization: Solitary, couple or harem (3-12)\\
\textbf{Treasure}: double\\
\textbf{Description}\\
Among the demonic hordes a succubus can often reach very high levels of power, using her manipulations and sensual charms, and many demonic wars rage due to the devious machinations of these creatures. A succubus originates from the souls of particularly lecherous and greedy evil mortals.


\medskip\index[Monstruary]{Tarrasque}\textbf{Tarrasque}

\emph{Colossal monstrosity (titan), misaligned}

\textbf{STRENGTH} +10

\textbf{DEXTERITY} +0

\textbf{CONSTITUTION} +10

\textbf{INTELLIGENCE} -2

\textbf{WISDOM} +0

\textbf{CHARISMA} +0

\textbf{Initiative} +0 -- \textbf{Defense} 35

\textbf{Hit Points} 676 (33x3d6 + 330)

\textbf{Movement} 24 m

\textbf{Saving Throws}: Fortitude +40, Reflexes +30, Will +30

\textbf{Damage Immunity} Fire, poison, electricity; weapons +2

\textbf{Condition Immunity} charmed, poisoned, paralyzed, frightened, fatigued

\textbf{Sensi} blind sight 36 m

\textbf{Languages} -

\textbf{Challenge} 30 (155000 PX)

\emph{\textbf{Reflective Carapace.}} Whenever the Tarrasque is the target of a \emph{Arcane Bolt} spell, a line spell, or a spell that requires a ranged attack roll, roll a d6. On 1 to 5, the Tarrasque ignores it. On a 6, the Tarrasque ignores it, and the effect is reflected back at the caster as if it originated from the Tarrasque, turning the caster into the target.

\emph{\textbf{Siege Monster.}} The Tarrasque deals double damage to objects and structures.

€23808 € {€23809 € {Legendary Resistance (3 / Day).}} If the Tarrasque fails a saving throw, he can choose to succeed instead.

\emph{\textbf{Magic Resistance.}} The Tarrasque has +1d6 on saving throws against spells or other magical effects.

\emph{\textbf{Regeneration.}} The Tarrasque regenerates 10 Hit Points at the start of its round if it took no acid damage in the previous round.

\textbf{Shares}

\emph{\textbf{Multiattack.}} The Tarrasque can use its Frightful Presence. Then he makes five attacks: one with his bite, two with his claws, one with his horns, and one with his tail. Instead of biting he can use Swallow. The Tarrasque's attacks are treated as +4 magical.

\emph{\textbf{Claw.} Melee weapon attack}: +30 to hit, reach 5 meters, one target.

\emph{Hits:} 28 (4d8 + 10) slashing damage, 3 bleed damage.

\emph{\textbf{Tail.} Melee weapon attack}: +30 to hit, reach 20 ft., one target.

\emph{Hits:} 24 (4d6 + 10) bludgeoning damage. If the target is a creature, it must succeed on a DC 20 Fortitude saving throw or fall prone.

\emph{\textbf{Horns.} Melee weapon attack}: +30 to hit, reach 10 ft., one target.

\emph{Hits:} 32 (4d10 + 10) piercing damage.

\emph{\textbf{Bite.} Melee weapon attack}: +30 to hit, reach 10 ft., one target.

\emph{Hits:} 36 (4d12 + 10) piercing damage. If the target is a creature, it is grappled (DC 20 to escape). Until the grapple ends, the target is entangled, and the Tarrasque cannot use its bite against another target.

\emph{\textbf{Swallow.}} The Tarrasque makes a bite attack against a Large or smaller target it is grappling. If the attack hits, the target is engulfed, and the grapple ends. The engulfed target is blinded and restrained, has full cover against attacks and other effects outside the Tarrasque, and takes 56 (16d6) acid damage at the start of each round of the Tarrasque.

If the Tarrasque takes 60 or more damage in a single round from a creature within it, the Tarrasque must succeed on a DC 30 Fortitude saving throw at the end of that round or vomit all swallowed creatures, which fall prone in a space within 3 meters from Tarrasque. If the Tarrasque dies, an engulfed creature is no longer restrained by it and can exit the corpse using 30 feet of movement, exiting prone.

\emph{\textbf{Frightening Presence.}} Each creature chosen by the Tarrasque that is within 120 feet of it and aware of its presence must succeed on a DC 17 Will save or be frightened for 1 minute. A creature can repeat the saving throw at the end of each of its rounds, with -1d6 if the Tarrasque is in line of sight, ending the effect for itself on a success. If the creature's saving throw succeeds or the effect ends for it, the creature is immune to the Tarrasque's Frightening Presence for the next 24 hours.

\textbf{Additional Shares}

The Tarrasque can perform 3 additional Actions, chosen from the options below. It can use only one Additional option at a time, and only at the end of another creature's round. The tarrasque regains any additional actions spent at the start of its round.

\textbf{Attack.} The Tarrasque makes a claw or tail attack. \textbf{Chew (Costs 2 Actions).} The Tarrasque makes a bite attack or uses Swallow.

\textbf{Move.} The Tarrasque moves up to half its move.

\textbf{Ecology}\\
Environment: Any\\
Organization: Solitaire\\
\textbf{Treasure}: None\\
\textbf{Description}\\
The legendary Tarrasque is among the most destructive monsters in the world. Fortunately, he spends most of his time in a kind of deep hibernation in an unknown cave in a remote corner of the world. When he awakens, however, entire kingdoms die.

While not particularly intelligent, the Tarrasque is intelligent enough to understand some words in the language of the Depths (despite not being able to speak). Likewise, the fury is not uncontrolled: it focuses on the creature that harmed it the most and is difficult to distract by trickery.

Legend has it that the Tarrasque is Cattalm's pet.

\medskip\index[Monstery]{Flaming Skull}\textbf{Flaming Skull}

\emph{Small Undead, Evil Traits}

\textbf{STRENGTH} +0

\textbf{DEXTERITY} +1

\textbf{CONSTITUTION} +1

\textbf{INTELLIGENCE} +1

\textbf{WISDOM} +0

\textbf{CHARISMA} +0

\textbf{Initiative} +1 -- \textbf{Defense} 13

\textbf{Hit Points} 7 (1d8 + 3)

\textbf{Movement} flight 10 m

\textbf{Saving Throws}: Fortitude +1, Reflexes +2, Will +1

\textbf{Damage Resistances} from Void

\textbf{Damage Immunity} Fire, poison, from non-magical weapon

\textbf{Condition Immunity} charmed, poisoned, paralyzed, fatigued, frightened, bleeding

\textbf{Senses} Darkvision 18 m

\textbf{Challenge} 2 (200 PX)

\emph{\textbf{Spells.}} A Flaming Skull can cast the following spells innately.

at will: \emph{Produce Flame}

1 time per day: \emph{Kyrin's Flaming Acorn Crab}

\emph{\textbf{Undead Nature.}} The Flaming Skull has no need for air, food, drink, or sleep.

\textbf{Ecology}\\
Environment: Any\\
Organization: Solitary, pair, patrol (2d4)\\
\textbf{Treasure}: none\\

\textbf{Description}

Flaming Skulls are created from the corpses of spellcasters specializing in the Fire Magic List, via a variant of the Raise Dead spell.

Used as guardians and torches they often represent a first line of defense in dungeons.

\medskip\index[Monster]{Dragon Tortoise}\textbf{Dragon Tortoise}

\emph{Ghostly Dragon, Neutral}

\textbf{STRENGTH} +7

\textbf{DEXTERITY} +0

\textbf{CONSTITUTION} +5

\textbf{INTELLIGENCE} +0

\textbf{WISDOM} +1

\textbf{CHARISMA} +1

\textbf{Initiative} +0 -- \textbf{Defense} 29

\textbf{Hit Points} 341 (22x3d6 + 110)

\textbf{Movement} 6m, swim 12m

\textbf{Saving Throws} Fortitude +22, Reflexes +17, Will +18

\textbf{Senses} Darkvision 18 m

\textbf{Languages} Aquan, Draconic

\textbf{Challenge} 17 (18000 PX)

\emph{\textbf{Amphibian.}} The dragon tortoise can breathe air and water.

\textbf{Shares}

\emph{\textbf{Multiattack.}} The dragon can make three attacks: one with its bite and two with its claws. He can make one tail attack instead of two claw attacks.

\emph{\textbf{Claw.} Melee weapon attack}: +26 to hit, reach 10 ft., one target.

\emph{Hits:} 16 (2d8 + 7) slashing damage.

\emph{\textbf{Tail.} Melee weapon attack}: +26 to hit, reach 5 meters, one target.

\emph{Hits:} 26 (3d12 + 7) bludgeoning damage. If the target is a creature, it must succeed on a DC 20 Fortitude save or be pushed 10 feet away from the dragon tortoise and fall prone.

\emph{\textbf{Bite.} Melee weapon attack}: +26 to hit, reach 5 meters, one target.

\emph{Hits:} 26 (3d12 + 7) piercing damage.

\emph{\textbf{Steam Breath (Recharge 5-6).}} The dragon turtle exhales hot steam in a 60-foot cone. Each creature in that area must make a DC 18 Fortitude saving throw and take 52 (15d6) fire damage on a failed save, or half as much damage on a successful one. Being underwater provides no resistance against this type of damage.

\textbf{Ecology}
Environment: Temperate aquatic\\
Organization: Solitaire\\
\textbf{Treasure}: Double\\
\textbf{Description}\\
Dragon turtles inhabit fresh and salt waters, where they pose some of the greatest dangers to sailors and those who travel by ship across the world's sea lanes. Experienced sailors know what the dragon turtles in the area want and frequently make offerings of gold and magic to ensure safe passage or avoid the area entirely. For its part, a dragon turtle appreciates and even expects such tolls and gifts, and a dragon turtle that expects gifts but is ignored is a dangerous enemy indeed.

The color of a dragon tortoise's shell varies from individual to individual. Some have opaque brown and rusty-red shells, while others have deep blue-green carapaces with silvery highlights on the rocky tips. The coloration of the head, tail and legs is slightly paler than the shell and includes golden streaks along the crest and spines.

Dragon turtles claim huge offshore territories, encompassing regions often exceeding 75 square km. Here, these dangerous beasts capsize ships that do not respect their territories, adding submerged wrecks and their precious cargoes to their hiding places. Dragon tortoises generally make their lairs in deep caves accessible only through water, and often decorate them not only with riches stolen from ships they have sunk, but also with the wrecks of these unfortunate vessels. Their territorial nature and their predilection for this type of burrow puts them in direct conflict with other underwater races such as Marinids and Sahuagin.

Large fish, such as tuna, sturgeon and even sharks, are among the dragon turtles' favorite foods, but being omnivorous, they sometimes also feed on large underwater fields of seaweed. They certainly do not disdain supplementing their diet with passengers from sinking ships, even if this practice is not due to evil or cruelty. Dragon tortoises have shells 5 meters in diameter, with limbs extending a few meters beyond, and measure 7 meters from the tip of their nose to the end of their powerful tail.

\medskip\textbf{Mice, The}\\\index[Monstery]{Mice, The}
\emph{Tiny fairy, Patron}\\
\textbf{Strength}: -1\\
\textbf{Dexterity}: +4\\
\textbf{Constitution}: +0\\
\textbf{Intelligence}: +6\\
\textbf{Wisdom}: +2\\
\textbf{Charisma}: +6\\
\textbf{Defense}: 17 -- \textbf{Initiative}: +15\\
\textbf{Hit Points}: 4 (1d10 - 1)\\
\textbf{Movement}: 6 m\\
\textbf{Saving Throws}: Fortitude +40, Reflexes +40, Will +40 \\
\textbf{Senses}: Telluric Sense 30, Darkvision 30 m, True Vision 30 m\\
\textbf{Languages}: all\\
\textbf{Challenge} 0 (10 PX)\\
\textbf{Immunity}: to damage from weapons with a magic bonus lower than +6\\
\textbf{Immunity}: any effect that doesn't please Topi\\
\textbf{Immunity}: Topi does not want to be influenced by any magic\\
\textbf{Immunity}: to suffer any type of critical roll\\
€23928 € {€23929 € {It's La Topi}} La Topi has +3d6 (or +18) every time she has to roll dice or count a value.
Any attacks made by the Topi are considered magic +6 and are not resistable.\\
\textbf{Shares}\\
\emph{\textbf{Musetto}} every creature of Mice's choice, within 30 meters, suffers a Musetto. The creature is moved 2d6 feet away and takes 3d6 damage\\
\emph{\textbf{Mouse Bite} Melee Weapon Attack}: +26 to hit, reach 1 m, one target.\\
\emph{Hits:} 6 piercing damage.\\
\emph{\textbf{Scratch} up to 8 Melee Weapon Attacks}: hits automatically, reach 1 m, up to 4 targets.\\
\emph{Hits:} 1 piercing damage.\\
€23939 € {€23940 € {Angry:}} Mouse does what she wants. Cost 1 Reaction.\\
\textbf{Ecology}\\
Environment: Anywhere\\
Organization: Solitaire\\
\textbf{Treasure}: Special\\
\textbf{Description}\\
She might be mistaken for a little white mouse, but La Topi is much more. Smart, intelligent, beautiful, she loves going to the market and buying handbags.


\medskip\textbf{Darktorch}\\\index[Monster]{Darktorch}
\emph{Average, undead, evil}\\
\textbf{Strength}: +3\\
\textbf{Dexterity}: +1\\
\textbf{Constitution}: +2\\
\textbf{Intelligence}: +0\\
\textbf{Wisdom}: -1\\
\textbf{Charisma}: -2\\
\textbf{Defense}: 17 -- \textbf{Initiative}: +2\\
\textbf{Hit Points}: 75 (12d10 +20)\\
\textbf{Movement}: 6 m\\
\textbf{Saving Throws}: Fortitude +9, Reflexes +8, Will +7 \\
\textbf{Senses}: Darkvision, sees in magical darkness\\
\textbf{Damage Resistances} from Void; from a non-magical weapon or one that is not silver\\
\textbf{Damage Immunity} Poison\\
\textbf{Condition Immunity} poisoned, fatigue, bleeding\\
\textbf{Vulnerability} Light\\
€23964 € {Invisible in the dark} a Darktorch is completely invisible as long as she is in the dark \\
€ 23965 € {Languages}: she understands the common language, but does not speak \\
\emph{\textbf{Undead Nature.}} Darktorch has no need for air, food, drink or sleep.\\
\textbf{Challenge} 4 (1100 PX)\\
\emph{\textbf{Light Sensitivity}}. While in sunlight, Darktorch has -1d6 to attack rolls\\
\textbf{Multi-attack}\\
\emph{\textbf{Attack}} Darkflash attacks twice with his torch or performs Cone of Sadness\\
\emph{\textbf{Torch}} Melee attack, +8 to hit\\
\emph{\textbf{Hits}} 7 (1d6+3) bludgeoning damage, casts Darkness spell on affected target, lasts until Darktorch is destroyed\\
\emph{\textbf{Cone of Sadness}} 6 meter cone. Affected creatures must make a DC 14 Will save or fall into a grim desperation that grants -1d6 to attack rolls, -2 to melee damage.\\
\textbf{Ecology}\\
Environment: Dungeon\\
Organization: Solitaire, group 2d4\\
\textbf{Treasure}: Special\\
\textbf{Description}\\
A Darktorch was an adventurer, like you, who died in terror after the last torch went out. A Darktorch is an undead, usually humanoid, with a vaguely nondescript appearance, who wields a torch that emanates pure darkness. Its purpose is to kill new adventurers by enveloping them in eternal darkness.

Usually the Darktorch hides in the darkness waiting to touch the opponent and envelop him in his curse. A creature killed by a Darktorch returns to life as a Darktorch after 1d3 days.

A Darktorch leaves its torch on the ground when destroyed. This torch, of pure darkness can cast the Darkness touch spell three times per day, out of the hands of a Darktorch if exposed to sunlight it is destroyed in 2d4 rounds.


\medskip\index[Monster]{Troll}\textbf{Troll}

\emph{Great giant, chaotic evil}

\textbf{STRENGTH} +5

\textbf{DEXTERITY} +1

\textbf{CONSTITUTION} +5

\textbf{INTELLIGENCE} -2

\textbf{WISDOM} -1

\textbf{CHARISMA} -2

\textbf{Initiative} +1 -- \textbf{Defense} 18

\textbf{Hit Points} 84 (8d10 + 40)

\textbf{Movement} 9 m

\textbf{Saving Throws}: Fortitude +11, Reflexes +4, Will +3

\textbf{Skills} Awareness +2

\textbf{Senses} Darkvision 18 m

\textbf{Languages} Giant

\textbf{Challenge} 5 (1800 PX)

\emph{\textbf{Sense of Smell.}} The troll has +1d6 on Wisdom (Awareness) checks that rely on smell.

\emph{\textbf{Regeneration.}} The troll recovers 10 Hit Points at the start of its round. If the troll takes acid or fire damage, this trait does not function at the start of the troll's next round. The troll dies only if it starts its round at -5 hit points and cannot regenerate.

\textbf{Shares}

\emph{\textbf{Multiattack.}} The troll can make three attacks: one with its bite and two with its claws.

\emph{\textbf{Claw.} Melee weapon attack}: +11 to hit, reach 1 m, one target.

\emph{Hits:} 12 (2d6 + 5) slashing damage, 1 bleed damage.

\emph{\textbf{Bite.} Melee weapon attack}: +11 to hit, reach 1 m, one target.

\emph{Hits:} 8 (1d6 + 8) piercing damage.

\textbf{Ecology}\\
Environment: Cold Mountains\\
Organization: Solo or gang (2-4)\\
\textbf{Treasury}: Standard\\
\textbf{Description}\\
Trolls possess sharp claws and incredible regenerative abilities that allow them to heal almost any wound. They are hunchbacked, ugly but very strong: combined with their claws, their strength allows them to tear flesh with their bare hands. Trolls are about 10 feet tall, but their posture makes them appear shorter. An adult troll weighs approximately 500 kg.

A troll's appetite and regenerative abilities make it a fearless fighter, charging headlong into the nearest living creature and attacking with all its fury. Only fire makes a troll hesitate, but even what is mortal danger to him does not stop him from advancing. Those who face trolls know to locate and burn any part of themselves after a fight, because even from the smallest shred of their body, a complete troll can be reborn over time. Fortunately, only the larger parts of a troll, such as limbs, regrow this way.

Despite their ferocity, trolls are extraordinarily tender and gentle towards their young. Female trolls work in groups, spending a lot of time teaching the pups how to hunt and defend themselves before sending them off to find their own territory. A male troll lives a solitary existence, briefly meeting females only to mate. All trolls spend their time foraging for food, as they must consume enormous quantities of it every day or starve to death. For this reason, most trolls create their own hunting territory which is often defended by fighting with rivals. Such battles are usually non-lethal, but trolls know their weaknesses well, exploiting them to kill their opponent in lean times.

It is universally known that trolls can naturally change, acquiring for short periods the most peculiar characteristics of the creatures they feed on. You have no idea how funny a Pegasutroll can be...


\medskip\index[Monster]{Aquatic Man}\textbf{Aquatic Man}

\emph{Medium humanoid (aquatic man), neutral}

\textbf{STRENGTH} +0

\textbf{DEXTERITY} +1

\textbf{CONSTITUTION} +1

\textbf{INTELLIGENCE} +0

\textbf{WISDOM} +0

\textbf{CHARISMA} +1

\textbf{Initiative} +1 -- \textbf{Defense} 12

\textbf{Hit Points} 11 (2d8 + 2)

\textbf{Movement} 3m, swim 12m

\textbf{Saving Throws}: Fortitude +3, Reflexes +1, Will -1; +2 vs. Enchantment

\textbf{Skills} Awareness +2

\textbf{Languages} Aquan, Municipality

\textbf{Challenge} 1/8 (25 PX)

\emph{\textbf{Amphibian.}} The aquatic man can breathe air and water.

\textbf{Shares}

\emph{\textbf{Spear.} Melee or Ranged Weapon Attack}: +2 to hit, reach 1m or range 6m, one target.

\emph{Hits:} 3 (1d6) piercing damage, or 4 (1d8) piercing damage if used with two hands to make a melee attack.

\textbf{Ecology}\\
Environment: Temperate Oceans\\
Organization: Solo, patrol (2-6), band (6-10 plus a 3rd level lieutenant, company (11-60 plus 3 3rd level lieutenants, 2 5th level commanders, 1 7th level commodore and 3-12 Calamari\\
\textbf{Treasure}: NPC equipment (Trident, Light Crossbow with 10 Bolts, other treasure)\\
\textbf{Description}\\
Physically, Fishmen resemble their ancestors, with expressive foreheads, pale skin, dark hair, and purple eyes. They have three thin gills on their necks, but can pass for Human for short periods if they wish.

\medskip\index[Monster]{Tree Man (Arborom)}\textbf{Tree Man (Arborom)}

\emph{Huge plant, chaotic good}

\textbf{STRENGTH} +6

\textbf{DEXTERITY} -1

\textbf{CONSTITUTION} +5

\textbf{INTELLIGENCE} +1

\textbf{WISDOM} +3

\textbf{CHARRISMA} +1

\textbf{Initiative} +1 -- \textbf{Defense} 21

\textbf{Hit Points} 138 (12d12 + 60)

\textbf{Movement} 9 m

\textbf{Saving Throws}: Fortitude +13, Reflexes +3, Will +9

\textbf{Damage Resistances} bludgeoning, piercing

\textbf{Damage Vulnerability} fire

\textbf{Languages} Common, Druidic, Elvish, Sylvan

\textbf{Challenge} 9 (5000 XP)

€24064 € {€24065 € {False Appearance.}} While the tree man remains motionless, he is indistinguishable from a normal tree.

\emph{\textbf{Siege Monster.}} The Treeman deals double damage to objects and structures.

\textbf{Shares}

\emph{\textbf{Multiattack.}} The Treeman makes two slam attacks.

\emph{\textbf{Slam.} Melee weapon attack}: +16 to hit, reach 2 m, one target.

\emph{Hits:} 16 (3d6 + 6) bludgeoning damage.

\emph{\textbf{Rock.} Ranged weapon attack}: +16 to hit, range 18m, one target.

\emph{Hits:} 28 (4d10 + 6) bludgeoning damage.

\emph{\textbf{Animate Trees (1/Day).}} The Treeman magically animates one or two visible trees within 60 feet of him. These trees have the same stats as the Arborom, except they have Intelligence and Charisma scores -3, cannot speak, and only have the Slam attack option. An animated tree acts as the Treeman's ally. The tree remains for 1 day or until it dies; until the treeman dies or is more than 120 feet away from the tree, or until the treeman takes a bonus action to transform him back into an inanimate tree. Then the tree will take root, if possible.

\textbf{Ecology}\\
Environment: Any forest\\
Organization: Solitary or scrub (2-7)\\
\textbf{Treasury}: Standard\\
\textbf{Description}\\
Arborom are guardians of the forests and ambassadors of the trees. As old as the forests themselves, they see themselves as parents and shepherds rather than gardeners: they are slow and methodical, but terrifying when forced to fight to defend their flock. While they rarely seek the company of short-lived races and have an innate distrust of change, they show tolerance towards those who wish to learn from their long, slow monologues, especially those in whose eyes they see a desire to protect the wilderness. Against those who threaten their forests, especially lumberjacks gathering wood or those who would clear a forest to build a road or fort, the Arborom's anger is unleashed swift and devastating. They are capable of demolishing what others build: a trait that helps them during their excesses of fury.

Arborom are primarily solitary creatures, and a single individual is often responsible for an entire forest, but they sometimes gather in groups called groves to exchange news and reproduce.

In times of grave danger, all the groves in a region come together for a months-long meeting called a council, but such events are very rare, and even millennia pass between councils.

A typical Arborom is 9 meters tall, with a trunk diameter of 60 centimeters, and weighs approximately 2,250 kg. Arboroms resemble the most common trees in the territories where they live. 

The Arborom are said to be created at the behest of Ephrem.

\medskip\index[Monster]{Magma Man}\textbf{Magma Man}

\emph{Small elemental, chaotic neutral}

\textbf{STRENGTH} -2

\textbf{DEXTERITY} +2

\textbf{CONSTITUTION} +1

\textbf{INTELLIGENCE} -1

\textbf{WISDOM} +0

\textbf{CHARISMA} +0

\textbf{Initiative} +2 -- \textbf{Defense} 15

\textbf{Hit Points} 9 (2d6 + 2)

\textbf{Movement} 9 m

\textbf{Saving Throws}: Fortitude +6, Reflexes +4, Will +3

\textbf{Damage Resistances} from non-magical weapons

\textbf{Damage Immunity} Fire

\textbf{Senses} Darkvision 18 m

\textbf{Languages} Ignan

\textbf{Challenge} 1/2 (100 PX)

€24102 € {€24103 € {Incendiary Lighting.}} As a bonus action, the magma man can light or extinguish his flames. While the flame is lit, the Magma Man radiates bright light in a 10-foot radius and dim light for an additional 10 feet.

\emph{\textbf{Deadly Blast.}} When the Magma Man dies, he explodes in a blast of fire and magma. Each creature within 10 feet of it must make a DC 11 Reflex saving throw, taking 7 (2d6) fire damage on a failed save, or half as much damage on a successful one. Flammable objects that are not worn or carried and that are in the area catch fire.

\textbf{Shares}

\emph{\textbf{Touch.} Melee weapon attack}: +4 to hit, reach 1 m, one target.

\emph{Hits:} 7 (2d6) fire damage. If the target is a flammable creature or object, it catches fire. As long as a creature takes an action to extinguish the flame, the creature takes 3 (1d6) fire damage at the end of each of its rounds.

\textbf{Ecology}\\
Environment: Any terrain (Plane of Fire)\\
Organization: Solo or gang (2-8)\\
\textbf{Treasure}: Standard\\
\textbf{Description}\\
Although lava men, or as they call themselves Ignim, populate the Plane of Fire, they sometimes slip through elemental cracks in the Material Plane. These cracks usually form in places of strong heat, such as volcanoes or underground rivers of magma, or in places of strong and unpredictable magic. The latter scenario usually ends in more complex events, as Ignim tend to unintentionally set fire to nearby flammable objects.

While not courageous, these small outsiders are nevertheless fearsome enemies of creatures without resistance to their intense heat. Their touch incinerates clothing, and creatures that strike their bodies with steel run the risk of turning their weapons to slag. The Ignim's best defense at home on the Plane of Fire is their numbers. The settlements, dotted with magma lakes and spraying geysers of molten rock, are teeming with incredible numbers of these creatures.

The Ignim are paranoid and distrustful. Always fearful of the larger denizens of the Plane of Fire, the Ignim pepper intruders with thousands of questions, asking where they are going, where they are coming from, and what they are doing near their precious magma lakes. If the travelers' answers are unsatisfactory, the Ignim attempt to rid themselves of the creatures as quickly as possible. Anyone who refuses to leave risks being thrown into a lake of liquid rock.

The Ignim take great pride in how they tend their magma lakes. Each lake has a different purpose: for swimming, for cooking meals or for relaxing. The Ignim place minerals and salts into these lakes to suit their purpose. Cooking lakes (sometimes called {killer lakes} by foreigners) are warmer than others, and recreational ones are usually darker than swimming ones.

At maturity, Ignim are 1.2 meters tall, their dense composition making them weigh 150 kg.

\medskip\index[Monster]{Unicorn}\textbf{Unicorn}

\emph{Large celestial, legal good}

\textbf{STRENGTH} +4

\textbf{DEXTERITY} +2

\textbf{CONSTITUTION} +2

\textbf{INTELLIGENCE} +0

\textbf{WISDOM} +3

\textbf{CHARISMA} +3

\textbf{Initiative} +2 -- \textbf{Defense} 15

\textbf{Hit Points} 67 (9d10 + 18)

\textbf{Movement} 15 m

\textbf{Saving Throws}: Fortitude +7, Reflexes +7, Will +6; +2 resistance against Void, Negative Energy

\textbf{Damage Immunity} Poison

\textbf{Condition Immunity} charmed, poisoned, paralyzed

\textbf{Senses} Darkvision 18 m

\textbf{Languages} Celestial, Elvish, Sylvan, telepathy 18 m

\textbf{Challenge} 5 (1800 PX)

\emph{\textbf{Magical Weapons.}} The unicorn's weapon attacks are magical.

\emph{\textbf{Charge.}} If the unicorn moves at least 20 feet in a straight line towards the target and hits it with a horn attack during the same round, the target takes 9 (2d8) piercing damage additional. If the target is a creature, it must succeed on a DC 15 Fortitude save or fall prone.

\emph{\textbf{Innate Spells.}} The unicorn's innate spellcasting ability is Charisma (DC 14 for spell saving throws). The unicorn can innately cast the following spells, requiring no components:

At will: \emph{druidic artifice, identification of good and evil,} \emph{passing without traces}

1/day each: \emph{calm emotions, dissolve good and evil,} \emph{hinder}

\emph{\textbf{Magic Resistance.}} The unicorn has +1d6 on saving throws against spells and other magical effects.

\textbf{Shares}

\emph{\textbf{Multiattack.}} The unicorn makes two attacks: one with its hooves and one with its horn.

\emph{\textbf{Horn.} Melee weapon attack}: +10 to hit, reach 1 m, one target.

\emph{Hits:} 8 (1d8 + 4) piercing damage.

\emph{\textbf{Hooves.} Melee weapon attack}: +10 to hit, reach 1 m, one target.

\emph{Hits:} 11 (2d6 + 4) bludgeoning damage.

\emph{\textbf{Teleportation (1/Day).}} The unicorn can magically teleport itself and up to three other willing visible creatures within 1 meter of it, along with any equipment they are wearing or carrying , in a place familiar to the unicorn, which is a maximum of 1.5 kilometers away.

\emph{\textbf{Healing Touch (3/Day).}} The unicorn comes into contact with another creature via its horn. The target magically regains 11 (2d8 + 2) hit points. Additionally, contact removes all diseases and neutralizes all poisons afflicting the target.

\textbf{Additional Shares}

The unicorn can perform 3 additional Actions, chosen from the options below. It can use only one Additional option at a time, and only at the end of another creature's round. The unicorn recovers the additional actions spent at the start of its round.

\textbf{Self-healing (Costs 3 Actions).} The unicorn magically recovers 11 (2d8 + 2) Hit Points.

\textbf{Shimmering Shield (Costs 2 Actions).} The unicorn creates a shimmering magical field that surrounds it or another creature visible to it within 60 feet. The target gains a +2 bonus to Defense until the end of the unicorn's next round.

\textbf{Hooves.} The unicorn makes an attack with its hooves.

\textbf{Ecology}\\
Environment: Temperate Forests\\
Organization: Solitaire, couple or blessing (3-6)\\
\textbf{Treasure}: None\\
\textbf{Description}\\
Unicorns are fierce, intelligent sylvan creatures who prefer to remain isolated, appearing only to defend their homes from evil. They avoid all creatures except good fey, good humanoid women, and animals native to their forest, but may join other good creatures against common enemies. A typical unicorn is 2.3 meters long, 1.3 meters tall at the withers and weighs 600 kg.

Unicorn pairs stay together for life and dwell in particular clearings or within the forests they defend. They allow good and neutral creatures to pass through them, hunt or live there, but evil creatures or those who would like to disturb the ecosystem, for example by hunting for fun or cutting down trees to sell the wood, are quickly removed or killed. On some rare occasions, unicorns whose partners have been killed take young women of rare virtue as surrogates, allowing them to ride them and become their lifelong guardians. If the woman bonds with someone else, such as a child or a lover, the bond with the unicorn is lovingly dissolved, generating the legend that unicorns only befriend virgins.

A unicorn's horn is the source of its powers, and to use its magical abilities on other creatures it must touch them with it. Evil creatures place great value on unicorn horns as reagents for potions of healing and dark rites: a powdered unicorn horn is worth 800 gp when used to create a magical healing item.

\subsection{Vampires}

\medskip\index[Monster]{Vampire}\textbf{Vampire}

\emph{Medium undead (shapeshifter), lawful evil}

\textbf{STRENGTH} +4

\textbf{DEXTERITY} +4

\textbf{CONSTITUTION} +4

\textbf{INTELLIGENCE} +3

\textbf{WISDOM} +2

\textbf{CHARISMA} +4

\textbf{Initiative} +4 -- \textbf{Defense} 23

\textbf{Hit Points} 144 (17d8 + 68)

\textbf{Movement} 9 m

\textbf{Saving Throws}: Fortitude +17, Reflexes +17, Will +14

\textbf{Skills} Stealth +9, Awareness +17

\textbf{Damage Immunity} from Void; from a non-magical weapon

\textbf{Condition Immunity} charmed, deafened, bleeding

\textbf{Senses} darkvision 36 m

€24186 € {Languages} the languages ​​he knew in life, Exspiran

\textbf{Challenge} 13 (10000 PX)

\emph{\textbf{Shapeshifting.}} If the vampire is not under sunlight or immersed in running water, he can use an action to transform into a Tiny bat, a cloud of Medium mist, or return to his true form.

While in bat form, the vampire cannot speak, his movement speed is 3 feet, and he has a flight speed of 30 feet. His stats, other than size and speed, are unchanged. Whatever equipment he is wearing transforms with it, but whatever he was carrying is knocked to the ground. Upon death he returns to his true form.

While in mist form, the vampire cannot perform actions, speak, or manipulate objects. He is weightless, has a flight speed of 20 feet, can hover, and can enter a hostile creature's space and stop there. Furthermore, if air passes through a space, the mist can do the same without constricting, but it cannot pass through water. He has +1d6 on Fortitude and Reflex saves and is immune to all nonmagical damage, except damage taken by the light of the
Sun.

\emph{\textbf{Weaknesses of the Vampire.}} The vampire has the following defects:

€24192 € {Damaged by Running Water.} The vampire takes 20 acid damage if he ends his round in running water.

\emph{Hypersensitivity to Light.} The vampire takes 20 Light damage when he begins his round in sunlight. While in sunlight, he has -1d6 on attack rolls and proficiency checks.

\emph{Stake in the Heart.} If a piercing weapon made of wood is driven into the vampire's heart while the vampire is incapacitated in his resting place, the vampire is paralyzed until the stake is removed.

\emph{Prohibition.} The vampire cannot enter a house without invitation from its occupants.

\emph{\textbf{Escape into the Mist.}} When reduced to 0 Hit Points outside of its resting place, the vampire transforms into a cloud of mist (as per the Shapeshifting trait) instead of falling without senses, as long as it is not exposed to sunlight or running water. If he cannot transform, he is destroyed.

While at 0 Hit Points in this form, he cannot revert to his vampire form, and must reach his resting place within 2 hours or be destroyed. Once he reaches his resting place, he reverts to his vampire form. He will then remain paralyzed until he has recovered at least 1 hit point. After spending at least 1 hour in its resting place at 0 hit points, the vampire will regain 1 hit point.

\emph{\textbf{Undead Nature.}} The vampire does not need air.

€24200 €{€24201 €{Legendary Resistance (3 / Day).}} If the vampire fails a saving throw, he can choose to succeed instead.

€24202 €{€24203 €{Regeneration.}} The vampire regains 20 hit points at the start of his round if he has at least 1 hit point and is not exposed to sunlight or running water. If the vampire takes Light damage or Holy Water damage, this trait does not function at the start of the vampire's next round.

\emph{\textbf{Climb as a Spider.}} The vampire can climb difficult surfaces, including standing upside down on the ceiling, without needing to make an ability check.

\textbf{Shares}

\emph{\textbf{Multiattack.}} The vampire can make two attacks, but only one of them can be a bite attack.

\emph{\textbf{Unarmed Strike (Vampire Form Only).} Melee Weapon Attack}: +18 to hit, reach 3 ft., one creature.

\emph{Hits:} 8 (1d8 + 4) bludgeoning damage. Instead of dealing damage, the vampire can grapple the target (escape DC 18).

\emph{\textbf{Bite (Bat or Vampire Form Only).} Melee Weapon Attack}: +18 to hit, reach 1 yd., a willing creature or a creature the vampire grabs, incapacitates, or entangles.

\emph{Hits:} 7 (1d6 + 4) piercing damage plus 10 (3d6) void damage. The target's maximum hit points are reduced by an amount equal to the void damage taken, and the vampire regains a number of hit points equal to that amount, Fortitude save DC 23 to resist the loss of maximum hit points. The target becomes fatigued. The target dies if this effect reduces its hit point maximum to 0. A humanoid killed in this way and then buried in the ground reanimates the following night as vampiric spawn under the vampire's control.

€24215 € {€24216 € {Charm.}} The vampire targets a humanoid within 30 feet that he can see. If the target can see the vampire, it must make a DC 17 Will save against this spell or be charmed by it. The fascinated target considers the vampire a trusted friend to be listened to and protected. Although the target is not under the vampire's control, it takes the vampire's requests and actions as favorably as possible, and is a willing target of the vampire's bite attack.

Whenever the vampire or the vampire's companions do something harmful to the target, the target can repeat the saving throw, ending the effect on itself on a success. Otherwise, the effect persists for 24 hours or until the vampire is destroyed, is on a different plane of existence than the target, or takes a bonus action to end the effect.

\emph{\textbf{Children of the Night (1/Day).}} The vampire magically summons 2d4 swarms of bats or rats, as long as the sun has not risen. While he is outside, the vampire can summon 3d6 wolves instead. The summoned creatures arrive in 1d4 rounds, acting as the vampire's allies and obeying his commands. The beasts remain for 1 hour, until the vampire dies, or until he dismisses them with a bonus action.

\textbf{Additional Shares}

The vampire can perform 3 additional Actions, chosen from the following options. He can only use one Additional option at a time, and only at the end of another creature's round. At the beginning of his round, the vampire recovers the additional Actions he has spent.

\textbf{Unarmed Strike.} The vampire makes an unarmed strike.

\textbf{Bite (Costs 2 Actions).} The vampire makes a bite attack.

\textbf{Move.} The vampire moves with its movement without provoking attacks of opportunity.

\textbf{Ecology}
Environment: Any\\
Organization: Solo or family (vampire plus 2-8 Spawn)\\
\textbf{Treasure}: NPC Equipment (Ring of Protection +2, Sash of Seduction +4, Cloak of Resistance +3)\\
\textbf{Description}\\
Vampires are undead humanoid creatures that feed on the blood of the living. They look much like they did in life, often becoming more attractive, though some appear tough and feral instead.


\medskip\index[Monster]{Vampiric Spawn}\textbf{Vampiric Spawn}

\emph{Medium undead, neutral evil}

\textbf{Initiative} +0 -- \textbf{Defense} 18

\textbf{Hit Points} 82 (11d8 + 33)

\textbf{Movement} 9 m

\textbf{Saving Throws} Fortitude +3, Reflex +2, Will +5

\textbf{STRENGTH} +3

\textbf{DEXTERITY} +3

\textbf{CONSTITUTION} +3

\textbf{INTELLIGENCE} +0

\textbf{WISDOM} +0

\textbf{CHARISMA} +1

\textbf{Skills} Stealth +6, Awareness +3

\textbf{Damage Resistances} from Void; from a non-magical weapon

\textbf{Senses} Darkvision 18 m

€24244 € {Languages} the languages ​​he knew in life

\textbf{Challenge} 6 (1800 PX)

\emph{\textbf{Weaknesses of the Vampiric Spawn.}} The Vampiric Spawn has the following flaws:

\emph{Damaged by Running Water.} The Vampiric Spawn takes 20 acid damage if it ends its round in running water.

\emph{Hypersensitivity to Light.} The Vampiric Spawn takes 20 Light damage when it begins its round in sunlight. While in sunlight, he has -1d6 on attack rolls and proficiency checks.

\emph{Stake through the Heart.} The Vampiric Spawn is destroyed if a wooden piercing weapon is driven through its heart while it is incapacitated within its resting place.

\emph{Prohibition.} Vampiric Spawn cannot enter a home without invitation from its occupants.

\emph{\textbf{Undead Nature.}} Vampiric Spawn have no need for air.

\emph{\textbf{Regeneration.}} The Vampiric Spawn regains 10 hit points at the start of its round if it has at least 1 hit point and is not exposed to sunlight or running water. If the Vampiric Spawn takes Light damage or Holy Water damage, this trait does not function at the start of the vampire's next round.

\emph{\textbf{Climb as a Spider.}} The Vampiric Spawn can climb difficult surfaces, including standing upside down on the ceiling, without needing to make an ability check.

\textbf{Shares}

\emph{\textbf{Multiattack.}} The vampiric spawn can make two attacks, but only one of them can be a bite attack.

\emph{\textbf{Claws.} Melee weapon attack}: +9 to hit, reach 3 ft., one creature.

\emph{Hits:} 8 (2d4 + 3) slashing damage. Instead of dealing damage, the vampire can grapple the target (escape DC 13).

\emph{\textbf{Bite.} Melee weapon attack}: +9 to hit, reach 1 m, a creature grabbed, incapacitated, or entangled by the vampire.

\emph{Hits:} 6 (1d6 + 3) piercing damage plus 7 (2d6) void damage. The target's maximum hit points are reduced by an amount equal to the void damage taken, and the vampire regains a number of hit points equal to that amount, Fortitude save DC 16 to resist the loss of maximum hit points. The target dies if this effect reduces its maximum hit points to 0. The creature becomes fatigued.

\textbf{Ecology}\\
Environment: Any\\
Organization: Solo, couple, group (3-6) or group (7-12)\\
\textbf{Treasury}: Standard\\
\textbf{Description}\\
A vampire can choose to create vampiric offspring from a victim instead of making him a full vampire only when he uses his create offspring ability on a humanoid creature. This decision must be made as soon as a vampire kills an appropriate creature using its bite.

\medskip\index[Monster]{Flesh worms}\textbf{Flesh worms}

\emph{tiny monstrosity, misaligned}

\textbf{STRENGTH} -4

\textbf{DEXTERITY} +0

\textbf{CONSTITUTION} -2

\textbf{INTELLIGENCE} -4

\textbf{WISDOM} 0

\textbf{CHARISMA} -4

\textbf{Initiative} +0 -- \textbf{Defense} 11

\textbf{Hit Points} 1 (1d6 -2)

\textbf{Movement} 1 m

\textbf{Saving Throws}: Fortitude -1, Reflexes +0, Will -4

\textbf{Sensi} telluric view 3 m

\textbf{Languages} -

\textbf{Challenge} 1 (200 PX)

\textbf{Shares}

\emph{\textbf{Infest flesh.}} These tiny creatures penetrate exposed flesh without making an attack roll as long as the flesh is exposed to them.

\emph{\textbf{Hits.}} Within 2d4 rounds the flesh worms (3d6 creatures) burrow into the tissue heading towards the heart. The worm infestation causes 1 hit point of damage per round as they burrow. Once you get to the heart each round the character must make a Fortitude saving throw DC 14, with a cumulative penalty of -1 per round. Once the saving throw fails the character dies.

\emph{\textbf{Eradicate Flesh Worms.}} The only way is to use a living flame (a torch causes 1d6 damage per application or a spell like Searing Wave) on the part where the worms are burrowing. Each application of fire can eliminate 3d6 worms. A DC 15 First Aid check removes 1d4 parasites but causes 1d4 damage upon extraction. After 2d4 rounds the worms are too deep and it is useless to apply fire, only a Cure Disease, or Healing, spell can completely eradicate the infestation.

\textbf{Ecology}\\
Environment: Rotten trees, rotting flesh\\
Organization: 3d6 groups\\
\textbf{Treasure}: None\\
\textbf{Description}\\

Flesh worms are among the most feared pests among adventurers. They are found in damp heaps of rotten leaves or trunks, in rotting corpses, and in murky waters. Pale, slimy, equipped with very sharp teeth, just over 4 millimeters long, they penetrate exposed flesh very easily and sense the heartbeat where they are going. As they burrow into the flesh they can be felt and even seen crawling under the skin.


\medskip\index[Monster]{Purple Worm}\textbf{Purple Worm}

\emph{Mammoth monstrosity, misaligned}

\textbf{STRENGTH} +9

\textbf{DEXTERITY} -2

\textbf{CONSTITUTION} +6

\textbf{INTELLIGENCE} -5

\textbf{WISDOM} -1

\textbf{CHARRISMA} -3

\textbf{Initiative} -2 -- \textbf{Defense} 26

\textbf{Hit Points} 247 (15x3d6 + 90)

\textbf{Movement} 15 m, excavation 9 m

\textbf{Saving Throws}: Fortitude +21, Reflexes +13, Will +14

\textbf{Senses} blind sight 9 m, telluric sense 18 m

\textbf{Languages} -

\textbf{Challenge} 15 (13000 PX)

\emph{\textbf{Tunnel Digger.}} The worm can burrow through solid rock at half its burrowing speed and leaves a tunnel 10 feet in diameter behind it.

\textbf{Shares}

\emph{\textbf{Multiattack.}} The worm makes two attacks: one with its bite and one with its stinger.

\emph{\textbf{Bite.} Melee weapon attack}: +30 to hit, reach 10 ft., one target.

\emph{Hits:} 22 (3d8 + 9) piercing damage. If the target is a Large creature, it must succeed on a DC 19 Reflex saving throw or be swallowed by the worm. While engulfed, the creature is blinded and restrained, has full cover against attacks and other effects from outside the worm, and takes 21 (6d6) acid damage at the start of each of the worm's rounds.

If the worm takes 30 or more damage in a single round from a creature within it, the worm must succeed on a DC 21 Fortitude saving throw at the end of its round or vomit all swallowed creatures, which fall prone in a space within 3 meters from the worm. If the worm dies, an engulfed creature is no longer hindered by it and can escape the corpse using 20 feet of movement, exiting prone.

\emph{\textbf{Sting.} Melee weapon attack}: +9 to hit, reach 10 ft., one creature.

\emph{Hit:} 19 (3d6 + 9) piercing damage, and the target must make a DC 19 Fortitude saving throw, taking 42 (12d6) poison damage on a failed save, or half as much damage on a he succeeds.

\textbf{Ecology}\\
Environment: Any dungeon\\
Organization: Solitaire\\
\textbf{Honey}: Accidental\\
\textbf{Description}\\
Purple worms are gigantic scavengers that inhabit the deepest regions of the world, eating any organic material they encounter. They are known to swallow their prey whole. It's not uncommon to hear of an adventuring party disappearing into the ravenous maw of a purple worm, screaming in terror as its members vanished one by one.

As they seek out living creatures to devour, purple worms also swallow a huge amount of soil and minerals as they burrow underground. A purple worm's innards may contain a considerable number of gems and other objects that can resist the corrosive acid within its esophagus. In areas rich in precious minerals, such as those near dwarven mines, the natural tunnels created by the purple worm excavations are often filled with considerable numbers of raw gold nuggets.

A purple worm generally claims a large underground cavern as its lair, and although it returns there to rest and digest food, it spends most of its time on the prowl, burrowing through the endless darkness or sliding along pre-existing tunnels. constant search for food to satiate his immense hunger. Though nearly mindless, purple worms are rarely stupid. They are popular as guardians among those who can magically control them or have a room in their lair large enough to accommodate them.


\medskip\index[Monster]{Tentacled Crawling Worm}\textbf{Tentacled Crawling Worm}

\emph{Large monstrosity, misaligned}

\textbf{STRENGTH} +4

\textbf{DEXTERITY} +1

\textbf{CONSTITUTION} +3

\textbf{INTELLIGENCE} -4

\textbf{WISDOM} 1

\textbf{CHARRISMA} -3

\textbf{Initiative} +2 -- \textbf{Defense} 17

\textbf{Hit Points} 55 (7d10 + 31)

\textbf{Movement} 9 m, climb 9 m

\textbf{Saving Throws}: Fortitude +5, Reflexes +4, Will +7

\textbf{Senses} Darkvision 18 m

\textbf{Languages} -

\textbf{Challenge} 4 (1000 PX)

\emph{\textbf{Climb as a Spider.}} The Tentacled Crawler Worm can climb difficult surfaces, including standing upside down on the ceiling, without needing to make an ability check.

\textbf{Shares}

\emph{\textbf{Multiattack.}} The Tentacled Crawling Worm makes 3 attacks, one with its bite and two with its tentacles.

\emph{\textbf{Bite.} Melee weapon attack}: +8 to hit, reach 1 m, one target.

\emph{Hits:} 10 (2d8 + 6) piercing damage.

\emph{\textbf{Tentacle.} Melee weapon attack}: +7 to hit, reach 10 ft., one creature.

\emph{Hits:} 1 bludgeoning damage. The target must make a DC 18 Fortitude saving throw or be paralyzed until the end of the next round.

\textbf{Ecology}\\
Environment: Any dungeon\\
Organization: Solitary, pair, tribe (8-12 +3d6 small)\\
\textbf{Honey}: Accidental\\
\textbf{Description}\\

A typical tentacled crawling worm is an annelid almost 3 meters long and weighs around 400 kilograms. Dark in color (of various shades from blue to green to brown) it is a large worm with a powerful mouth and long, light tentacles along the entire head.

The Tentacled Creeping Worm, although equipped with short legs, does not walk but crawls, secreting a sticky mucus which allows it to climb even on surfaces in any orientation.

They are ravenous creatures that never miss an opportunity to hunt and devour or preserve corpses in which to sow their eggs. They love Nibali flesh and will feed on any living creature (often rats given the typical sewer environment).

The origins of the Tentacled Crawling Worms are rather speculative, some hypothesize that an enchanter tried, and failed critically, to transform himself into a Purple Worm, others firmly believe that the gardens of Shayalia needed more fertilization and so the Patroness transformed ordinary earthworms into these terrifying creatures to devour and digest the buried corpses.

\medskip\index[Monster]{Wyvern}\textbf{Wyvern}

\emph{Large dragon, misaligned}

\textbf{STRENGTH} +4

\textbf{DEXTERITY} +0

\textbf{CONSTITUTION} +3

\textbf{INTELLIGENCE} -3

\textbf{WISDOM} +1

\textbf{CHARISMA} -2

\textbf{Initiative} +0 -- \textbf{Defense} 16

\textbf{Hit Points} 110 (13d10 + 39)

\textbf{Movement} 6 m, flight 24 m

\textbf{Saving Throws}: Fortitude +9, Reflexes +6, Will +8

\textbf{Skills} Awareness +4

\textbf{Senses} Darkvision 18 m

\textbf{Languages} -

\textbf{Challenge} 6 (2300 XP)

\textbf{Shares}

\emph{\textbf{Multiattack.}} The wyvern can make two attacks: one with its bite and one with its stinger. While flying, she can use her claws in place of one of her other attacks.

\emph{\textbf{Claws.} Melee weapon attack}: +13 to hit, reach 1 m, one target.

\emph{Hits:} 13 (2d8 + 4) slashing damage, 1 bleed damage.

\emph{\textbf{Bite.} Melee weapon attack}: +13 to hit, reach 10 ft., one creature.

\emph{Hits:} 11 (2d6 + 4) piercing damage.

\emph{\textbf{Sting.} Melee weapon attack}: +13 to hit, reach 10 ft., one creature.

\emph{Hits:} 11 (2d6 + 4) piercing damage. The target must make a DC 15 Fortitude saving throw, and take 24 (7d6) poison damage on a failed save, or half as much damage on a successful one.

\emph{\textbf{Angry:}} the Wyvern points its tail at the enemy and generates a 3 meter cone of poison. You can make a DC 19 Reflex save to halve the 7d8 poison damage.

\textbf{Ecology}\\
Environment: Temperate or warm hills\\
Organization: Solitary, pair or flock (3-6)\\
\textbf{Treasury}: standard\\
\textbf{Description}\\
Wyverns are brutal, violent reptiles related to dragons. They are always aggressive and impatient and prefer to achieve their goals using force. For this reason, dragons look upon wyverns with superiority, considering these distant relatives as primitive savages devoid of style and intelligence.

In most cases, this generalization is spot on. While certainly not of animal intellect or capable of speech, most wyverns do not care about diplomacy, preferring to fight first and argue later only if they find themselves faced with an adversary they cannot defeat or escape from.

Wyverns are territorial creatures. While they occasionally hunt larger prey in larger groups, they are solitary creatures whose hunting territory extends from 160 to 320 square km. It is known that wyverns often fight each other to the death over a territory rich in prey.

While constantly hungry and prone to attack, a wyvern can be made friendly through a careful combination of flattery, intimidation, food, and treasure, to make it a powerful ally. They often serve Giants and Monstrous Humanoids as guardians, and some Boggard and Lizardfolk tribes use them as mounts, though such arrangements are often quite costly in terms of food and gold, as few wyverns are willing to serve creatures for long. similar as mounts.

A wyvern is about 15 feet long, and the tail alone accounts for about half that length. A wyvern weighs on average 1000 kg.


\medskip\index[Monstery]{Wight}\textbf{Wight}

\emph{Medium undead, neutral evil}

\textbf{STRENGTH} +2

\textbf{DEXTERITY} +2

\textbf{CONSTITUTION} +3

\textbf{INTELLIGENCE} +0

\textbf{WISDOM} +1

\textbf{CHARISMA} +2

\textbf{Initiative} +2 -- \textbf{Defense} 16 (studded armor)

\textbf{Hit Points} 45 (6d8 + 18)

\textbf{Movement} 9 m

\textbf{Saving Throws}: Fortitude +3, Reflexes +2, Will +5

\textbf{Skills} Stealth +4, Awareness +3

\textbf{Damage Resistances} from Void; from a non-magical weapon or one that is not silvered

\textbf{Damage Immunity} Poison

\textbf{Condition Immunity} poisoned, fatigue, bleeding

\textbf{Senses} Darkvision 18 m

\textbf{Languages} the languages ​​he knew in life, Exspiran

\textbf{Challenge} 3 (700 PX)

\emph{\textbf{Undead Nature.}} The wight has no need for air, food, drink, or sleep.

\emph{\textbf{Light Sensitivity}}. While in sunlight, the wight has -1d6 on attack rolls, as well as on sight-based Wisdom (Awareness) checks.

\textbf{Shares}

\emph{\textbf{Multiattack.}} The wight can make two longsword attacks or two longbow attacks. He can use Drain Life in place of one of his longsword attacks.

\emph{\textbf{Drain Life.} Melee weapon attack}: +5 to hit, reach 3 ft., one creature.

\emph{Hits:} 5 (1d6 + 2) Void damage. The target must succeed on a DC 13 Fortitude saving throw or have its maximum hit points reduced by an amount equal to the damage taken. The target becomes fatigued. The target dies if the effect reduces its maximum hit points to 0.

A humanoid killed by this attack revives 24 hours later as a zombie under the wight's control, unless the humanoid is first brought back to life or the body is destroyed. The wight can control no more than twelve zombies at a time.

\emph{\textbf{Long Sword.} Melee weapon attack}: +5 to hit, reach 1 m, one target.

\emph{Hits:} 6 (1d8 + 2) slashing damage or 7 (1d10 + 2) slashing damage if used with two hands.

\emph{\textbf{Longbow.} Ranged weapon attack}: +5 to hit, range 45m, one target.

\emph{Hits:} 6 (1d8 + 2) piercing damage.

\textbf{Ecology}\\
Environment: any\\
Organization: Solitary, couple, group (3-6) or pack (7-12)\\
\textbf{Treasury}: Standard\\
\textbf{Description}\\
Wights are humanoids resurrected as undead due to necromancy, a violent death, or an extremely malevolent personality. In some cases, a wight arises when an undead spirit permanently bonds to a corpse, often that of a warrior. They are barely recognizable to those who knew them in life: their flesh is corrupted by evil and undeath, their eyes burn with hatred and their teeth become those of a beast. In a sense, a wight is the link between ghouls and wraiths: a deformed corpse that drains life energy with touch.

Being undead, wights do not need to breathe, so they can sometimes be found underwater, although they are not particularly good swimmers unless they originate from swimming creatures such as aquatic elves and merfolk. Underwater, wights prefer low-ceilinged caves where their poor swimming abilities are not a limitation.

\medskip\index[Monstery]{Wraith}\textbf{Wraith}

\emph{Medium undead, neutral evil}

\textbf{STRENGTH} -2

\textbf{DEXTERITY} +3

\textbf{CONSTITUTION} +3

\textbf{INTELLIGENCE} +1

\textbf{WISDOM} +2

\textbf{CHARISMA} +2

\textbf{Initiative} +3 -- \textbf{Defense} 16

\textbf{Hit Points} 67 (9d8 + 27)

\textbf{Movement} 0 m, fly 18 m (floats)

\textbf{Saving Throws}: Fortitude +6, Reflexes +4, Will +6

\textbf{Damage Resistances} acid, cold, lightning, fire, sound; from a non-magical weapon or one that is not silvered

\textbf{Damage Immunity} from Void, poison

\textbf{Condition Immunity} charmed, grabbed, poisoned, entangled, paralyzed, petrified, prone, fatigued, bleed

\textbf{Senses} Darkvision 18 m

\textbf{Languages} the languages ​​he knew in life, Exspiran

\textbf{Challenge} 5 (1800 PX)

\emph{\textbf{Incorporeal Movement.}} The wraith can pass through creatures and objects as if they were difficult terrain. He takes 5 (1d10) force damage if he ends his round inside an object.

\emph{\textbf{Undead Nature.}} The wraith has no need for air, food, drink, or sleep.

\emph{\textbf{Light Sensitivity}}. While in sunlight, the wraith has -1d6 on attack rolls, as well as on sight-based Wisdom (Awareness) checks.

\textbf{Shares}

\emph{\textbf{Drain Life.} Melee weapon attack}: +7 to hit, reach 3 ft., one creature.

\emph{Hits:} 21 (4d8 + 3) Void damage. The target must succeed on a DC 15 Fortitude saving throw or have its maximum hit points reduced by an amount equal to the damage taken. The target becomes fatigued. The target dies if the effect reduces its maximum hit points to 0.

\emph{\textbf{Create Wraith.}} The wraith targets a humanoid within 10 feet of it that has been dead for no more than 1 minute and from violent causes. The target's spirit animates as a wraith in its corpse's space and in the nearest unoccupied space. The wraith is under the wraith's control. The wraith can have no more than seven wraiths under its control at a time.

\emph{\textbf{Angry:}} the Wraith channels his negative energies into a blast of Void around himself within a 20-foot radius. All creatures must make a Fortitude saving throw DC 15 or take 3d6 void damage, on a successful save they are slowed 1/3r.

\textbf{Ecology}\\
Environment: Any\\
Organization: Solitary, couple, group (3-6) or pack (7-12)\\
\textbf{Treasure}: None\\
\textbf{Description}\\
Wraiths are creatures born of evil and darkness. They detest light and living creatures, having lost most of their connection to their former life.


\medskip\index[Monstery]{Xorn}\textbf{Xorn}

\emph{Medium elemental, neutral}

\textbf{STRENGTH} +3

\textbf{DEXTERITY} +0

\textbf{CONSTITUTION} +6

\textbf{INTELLIGENCE} +0

\textbf{WISDOM} +0

\textbf{CHARISMA} +0

\textbf{Initiative} +0 -- \textbf{Defense} 22

\textbf{Hit Points} 73 (7d8 + 42)

\textbf{Movement} 6 m, excavation 6 m

\textbf{Saving Throws}: Fortitude +8, Reflexes +2, Will +5

\textbf{Skills} Stealth +3, Awareness +6

\textbf{Damage Resistances} piercing and cutting of non-magical or non-adamantium weapons

\textbf{Senses} Darkvision 60 ft., telluric sense 60 ft.

\textbf{Languages} Tremun

\textbf{Challenge} 5 (1800 PX)

\emph{\textbf{Stone Camouflage.}} The xorn has +1d6 on Stealth (Hide) checks made to hide in rocky terrain.

\emph{\textbf{Earth Scrolling.}} The xorn can burrow through nonmagical, unworked earth and stone. When it does so, the xorn does not disturb the material it moves.

\emph{\textbf{Treasure Sense.}} The xorn can precisely locate, by smell, the location of precious metals and stones, such as coins and gems, within 60 feet of it.

\textbf{Shares}

\emph{\textbf{Multiattack.}} The xorn makes three claw attacks and one bite attack.

\emph{\textbf{Claw.} Melee weapon attack}: +9 to hit, reach 1 m, one target.

\emph{Hits:} 6 (1d6 + 3) slashing damage, 1 bleed damage.

\emph{\textbf{Bite.} Melee weapon attack}: +9 to hit, reach 1 m, one target.

\emph{Hits:} 13 (3d6 + 3) piercing damage.

\emph{\textbf{Angry:}} the Xorn belches out its last eaten gems and nuggets. In a 20-foot cone, all creatures must make a DC 18 Reflex save to halve the damage, 3d8 bludgeoning, of thrown gems and minerals.

\textbf{Ecology}\\
Environment: Any (Plane of Earth)\\
Organization: Solo, couple or group (3-6)\\
\textbf{Treasure}: Standard (precious metals, gems and jewels, and magic gems only)\\
\textbf{Description}
Strange creatures as wide as they are tall, xorns have little interest in natives of the Material Plane, if not for the gems and precious metals they may have with them. Hidden beneath the surface of the ground for what might seem like a very long time to a human, a xorn can wait months, even years, for ideal prey, and then attack those who bring with them their favorite food, such as a particular gem or a certain type of silver. Adventurers who enter xorn-inhabited regions often bring with them small nuggets of minerals or low-value gems and crystals to use as tribute. Although its value is usually directly proportional to its flavor and how palatable it may be, most xorn are quite greedy, preferring quantity over quality.

The treasure that a xorn carries with it or hides in its lair consists of a snack that it has saved for the next day. Offering a particularly delicious (and expensive) jewel or precious metal to a xorn can cement a temporary alliance. Because xorns can pass through rock with ease, they make excellent guides in subterranean regions.

Xorns are not very religious, but those among them who find faith are usually devoted to Ephrem (though it is rare, if not unlikely, for xorns to have Animal Companions, as they cannot follow them into the rock, and instead choose the domain of the Earth). Bards and xorn Devotees are not unknown: Bards usually choose Perform (song), and Devotees invariably have the Elemental Bloodline (earth).


\medskip\index[Monster]{Zombies}\textbf{Zombies}

\emph{Medium undead, neutral evil}

\textbf{STRENGTH} +1

\textbf{DEXTERITY} -2

\textbf{CONSTITUTION} +3

\textbf{INTELLIGENCE} -4

\textbf{WISDOM} -2

\textbf{CHARISMA} -3

\textbf{Initiative} -2 -- \textbf{Defense} 9

\textbf{Hit Points} 22 (3d8 + 9)

\textbf{Movement} 6 m

\textbf{Saving Throws} Fortitude +0, Reflexes +0, Will +3

\textbf{Damage Immunity} Poison

\textbf{Condition Immunity} poisoned, bleeding

\textbf{Senses} Darkvision 18 m

€24535 € {Languages} includes all the languages ​​he spoke in life but cannot speak

\textbf{Challenge} 1/4 (50 XP)

\emph{\textbf{Undead Nature.}} The zombie has no need for air, food, drink, or sleep.

\emph{\textbf{Fortitude of the Undead.}} If the damage reduces the zombie to 0 hit points, the zombie must make a Fortitude saving throw DC 5 + the damage taken, unless the damage is Light or a critical hit. If successful, the zombie instead drops to 1 hit point.

\textbf{Shares}

\emph{\textbf{Slam.} Melee weapon attack}: +3 to hit, reach 1 m, one target.

\emph{Hits:} 4 (1d6 + 1) bludgeoning damage.

\textbf{Ecology}\\
Environment: Any\\
Organization: Any\\
\textbf{Treasure}: None\\
\textbf{Description}\\
Zombies are the animated corpses of dead creatures, forced to move by necromantic spells such as Animate Dead. While zombies encountered are typically slow and sturdy, others possess different traits, allowing them to spread a disease or move more quickly.

Zombies are mindless automatons and can do nothing but follow orders. If left to their own devices, they wait immobile or move around in search of living creatures to slaughter and devour. Zombies attack to the point of destruction, regardless of their safety.

Although capable of following orders, zombies are often set free with orders to kill all living creatures. They are often encountered in packs that infest the lands frequented by the living, in search of prey. Most zombies are created through Animate Dead. Such zombies are always standard, unless the creator also casts Haste or Remove Paralysis to create Swift Zombies or Contagion to create Plague Zombies.


\medskip\index[Monster]{Zombie Ogre}\textbf{Zombie Ogre}

\emph{Great undead, neutral evil}

\textbf{STRENGTH} +4

\textbf{DEXTERITY} -2

\textbf{CONSTITUTION} +4

\textbf{INTELLIGENCE} -4

\textbf{WISDOM} -2

\textbf{CHARRISMA} -3

\textbf{Initiative} -2 -- \textbf{Defense} 9

\textbf{Hit Points} 85 (9d10 + 36)

\textbf{Movement} 9 m

\textbf{Saving Throws}: Fortitude +6, Reflexes +0, Will +3

\textbf{Damage Immunity} Poison

\textbf{Condition Immunity} poisoned, bleeding

\textbf{Senses} Darkvision 18 m

\textbf{Languages} includes Common and Giant but cannot speak

\textbf{Challenge} 2 (450 PX)

\emph{\textbf{Undead Nature.}} The zombie has no need for air, food, drink, or sleep.

\emph{\textbf{Fortitude of the Undead.}} If the damage reduces the zombie to 0 hit points, the zombie must make a Fortitude saving throw DC 5 + the damage taken, unless the damage is Light or a critical hit. If successful, the zombie instead drops to 1 hit point.

\textbf{Shares}

\emph{\textbf{Spiked Mace.} Melee weapon attack}: +6 to hit, reach 1 m, one target.

\emph{Hits:} 13 (2d8 + 4) bludgeoning damage.





\subsection{Appendix A: Various Creatures}

This appendix contains statistics of various animals, parasites and
other creatures. Statistics are organized alphabetically.

\medskip\textbf{Awakened Tree}\index[Monstery]{Awakened Tree}

The awakened tree is a normal tree provided by ability magic
sentience and mobility.

\emph{Huge plant, misaligned}

\textbf{STRENGTH} +4

\textbf{DEXTERITY} -2

\textbf{CONSTITUTION} +2

\textbf{INTELLIGENCE} +0

\textbf{WISDOM} +0

\textbf{CHARRISMA} -2

\textbf{Initiative} +0 -- \textbf{Defense} 14

\textbf{Hit Points} 59 (7d12 + 14)

\textbf{Movement} 6 m

\textbf{Saving Throws}: Fortitude +6, Reflexes -1, Will +1

\textbf{Damage Vulnerability} fire

\textbf{Damage Resistances} bludgeoning, piercing

\textbf{Languages} a language known by its creator

\textbf{Challenge} 2 (450 PX)

\emph{\textbf{False Appearance.}} While the tree remains immobile, it is indistinguishable from a normal tree.

\textbf{Shares}

\emph{\textbf{Slam.} Melee Weapon Attack}: +6 to hit, reach 10 ft., one target.

\emph{Hits:} 14 (3d6 + 4) bludgeoning damage.

\medskip\textbf{Moose}\index[Monster]{Moose}

\emph{Big beast, misaligned}

\textbf{STRENGTH} +3

\textbf{DEXTERITY} +0

\textbf{CONSTITUTION} +1

\textbf{INTELLIGENCE} -4

\textbf{WISDOM} +0

\textbf{CHARRISMA} -2

\textbf{Initiative} +0 -- \textbf{Defense} 11

\textbf{Hit Points} 13 (2d10 + 2)

\textbf{Movement} 15 m

\textbf{Saving Throws}: Fortitude +4, Reflexes +1, Will +0

\textbf{Languages} -

\textbf{Challenge} 1/4 (50 XP)

\emph{\textbf{Charge.}} If the moose moves at least 20 feet towards the target and hits it with a beak attack during the same round, the target takes an additional 7 (2d6) bludgeoning damage. If the target is a creature, it must succeed on a Fortitude saving throw
DC 13 or fall prone.

\textbf{Shares}

\emph{\textbf{Rostro.} Melee Weapon Attack}: +5 to hit, reach 1 m, one target.

\emph{Hits:} 6 (1d6 + 3) bludgeoning damage.

\emph{\textbf{Hooves.} Melee Weapon Attack}: +5 to hit, reach 1 ft., one prone creature.

\emph{Hits:} 8 (2d4 + 3) bludgeoning damage.

\medskip\textbf{Giant Moose}\index[Monster]{Giant Moose}

\emph{Huge beast, misaligned}

\textbf{STRENGTH} +4

\textbf{DEXTERITY} +3

\textbf{CONSTITUTION} +2

\textbf{INTELLIGENCE} -2

\textbf{WISDOM} +2

\textbf{CHARISMA} +0

\textbf{Initiative} +3 -- \textbf{Defense} 15

\textbf{Hit Points} 42 (5d12 + 10)

\textbf{Movement} 18 m

\textbf{Saving Throws}: Fortitude +8, Reflexes +7, Will +2

\textbf{Skills} Awareness +4

\textbf{Languages} Giant Elk, includes Common, Elvish and

Silvano but he cannot speak to them

\textbf{Challenge} 2 (450 PX)

\emph{\textbf{Charge.}} If the moose moves at least 20 feet towards the target and hits it with a beak attack during the same round, the target takes an additional 7 (2d6) bludgeoning damage. If the target is a creature, it must succeed on a DC 14 Fortitude save or fall prone.

\textbf{Shares}

\emph{\textbf{Rostro.} Melee Weapon Attack}: +6 to hit, reach 10 ft., one target.

\emph{Hits:} 11 (2d6 + 4) piercing damage.

\emph{\textbf{Hooves.} Melee Weapon Attack}: +6 to hit, reach 1 ft., one prone creature.

\emph{Hits:} 22 (4d4 + 4) bludgeoning damage.

\medskip\textbf{Eagle}\index[Monstery]{Eagle}

\emph{Small beast, misaligned}

\textbf{STRENGTH} -2

\textbf{DEXTERITY} +2

\textbf{CONSTITUTION} +0

\textbf{INTELLIGENCE} -4

\textbf{WISDOM} +2

\textbf{CHARRISMA} -2

\textbf{Initiative} +2 -- \textbf{Defense} 13

\textbf{Hit Points} 3 (1d6)

\textbf{Movement} 3 m, flight 18 m

\textbf{Saving Throws}: Fortitude +3, Reflexes +4, Will +2

\textbf{Skills} Awareness +4

\textbf{Languages} -

\textbf{Challenge} 0 (10 PX)

\emph{\textbf{Honed Sight.}} The eagle has +1d6 on Wisdom (Awareness) checks based on sight.

\textbf{Shares}

\emph{\textbf{Spurs.} Melee Weapon Attack}: +4 to hit, reach 1 m, one target.

\emph{Hits:} 4 (1d4 + 2) slashing damage.

\medskip\textbf{Giant Eagle}\index[Monstery]{Giant Eagle}

The giant eagle is a noble creature that speaks its own language and understands that of other races.

\emph{Great beast, neutral good}

\textbf{STRENGTH} +3

\textbf{DEXTERITY} +3

\textbf{CONSTITUTION} +1

\textbf{INTELLIGENCE} -1

\textbf{WISDOM} +2

\textbf{CHARISMA} +0

\textbf{Initiative} +3 -- \textbf{Defense} 14

\textbf{Hit Points} 26 (4d10 + 4)

\textbf{Movement} 3 m, flight 24 m

\textbf{Saving Throws}: Fortitude +5, Reflexes +7, Will +3

\textbf{Skills} Awareness +4

\textbf{Languages} Giant Eagle, understands Municipality and Ictun but cannot speak them

\textbf{Challenge} 1 (200 PX)

\emph{\textbf{Honed Sight.}} The eagle has +1d6 on Wisdom (Awareness) checks based on sight.

\textbf{Shares}

\emph{\textbf{Multiattack.}} The eagle makes two attacks: one with its beak and one with its spurs.

\emph{\textbf{Beak.} Melee Weapon Attack}: +5 to hit, reach 1 m, one target.

\emph{Hits:} 6 (1d6 + 3) piercing damage.

\emph{\textbf{Spurs.} Melee Weapon Attack}: +5 to hit, reach 1 m, one target.

\emph{Hits:} 10 (2d6 + 3) slashing damage.

\medskip\textbf{Vulture}\index[Monstery]{Vulture}

\emph{Medium beast, misaligned}

\textbf{STRENGTH} -2

\textbf{DEXTERITY} +0

\textbf{CONSTITUTION} +1

\textbf{INTELLIGENCE} -4

\textbf{WISDOM} +1

\textbf{CHARISMA} -3

\textbf{Initiative} +0 -- \textbf{Defense} 11

\textbf{Hit Points} 5 (1d8 + 1)

\textbf{Movement} 3 m, flight 15 m

\textbf{Saving Throws}: Fortitude +6, Reflexes +3, Will +1; +4 against diseases

\textbf{Skills} Awareness +3

\textbf{Languages} -

\textbf{Challenge} 0 (10 PX)

\emph{\textbf{Hot sense of smell and sight.}} The vulture has +1d6 on Wisdom (Awareness) checks that rely on smell or sight.

\emph{\textbf{Pack Tactics.}} The vulture has +1d6 to attack rolls against a creature if at least one of the vulture's allies is within 3 feet of the creature and that ally is not incapacitated.

\textbf{Shares}

\emph{\textbf{Beak.} Melee Weapon Attack}: +2 to hit, reach 1 m, one target.

\emph{Hits:} 2 (1d4) piercing damage.

\medskip\textbf{Giant Vulture}\index[Monstery]{Giant Vulture}

The giant vulture possesses superior intelligence and a malicious attitude.

\emph{Great beast, neutral evil}

\textbf{STRENGTH} +2

\textbf{DEXTERITY} +0

\textbf{CONSTITUTION} +2

\textbf{INTELLIGENCE} -2

\textbf{WISDOM} +1

\textbf{CHARRISMA} -2

\textbf{Initiative} +0 -- \textbf{Defense} 11

\textbf{Hit Points} 22 (3d10 + 6)

\textbf{Movement} 3 m, flight 18 m

\textbf{Saving Throws}: Fortitude +10, Reflexes +6, Will +3; +4 against diseases

\textbf{Skills} Awareness +3

\textbf{Languages} understands the Municipality but cannot speak

\textbf{Challenge} 1 (200 PX)

\emph{\textbf{Hot sense of smell and sight.}} The vulture has +1d6 on Wisdom (Awareness) checks that rely on smell or sight.

\emph{\textbf{Pack Tactics.}} The vulture has +1d6 to attack rolls against a creature if at least one of the vulture's allies is within 3 feet of the creature and that ally is not incapacitated.

\textbf{Shares}

\emph{\textbf{Multiattack.}} The vulture makes two attacks: one with its beak and one with its spurs.

\emph{\textbf{Beak.} Melee Weapon Attack}: +4 to hit, reach 1 m, one target.

\emph{Hits:} 7 (2d4 + 2) piercing damage.

\emph{\textbf{Spurs.} Melee Weapon Attack}: +4 to hit, reach 1 m, one target.

\emph{Hits:} 9 (2d6 + 2) slashing damage.

\medskip\textbf{Baboon}\index[Monster]{Baboon}

\emph{Small beast, misaligned}

\textbf{STRENGTH} -1

\textbf{DEXTERITY} +2

\textbf{CONSTITUTION} +0

\textbf{INTELLIGENCE} -3

\textbf{WISDOM} +1

\textbf{CHARRISMA} -2

\textbf{Initiative} +2 -- \textbf{Defense} 13

\textbf{Hit Points} 3 (1d6)

\textbf{Movement} 9m, climb 9m

\textbf{Saving Throws}: Fortitude +3, Reflexes +4, Will +1

\textbf{Languages} -

\textbf{Challenge} 0 (10 PX)

\emph{\textbf{Pack Tactics.}} The baboon has +1d6 to attack rolls against a creature if at least one of the baboon's allies is within 3 feet of the creature and that ally is not incapacitated.

\textbf{Shares}

\emph{\textbf{Bite.} Melee Weapon Attack}: +1 to hit, reach 1 m, one target.

\emph{Hits:} 1 (1d4 - 1) piercing damage.


\medskip\textbf{Killer Whale (Orca)}\index[Monster]{Orca}

\emph{Huge beast, misaligned}

\textbf{STRENGTH} +4

\textbf{DEXTERITY} +0

\textbf{CONSTITUTION} +1

\textbf{INTELLIGENCE} -4

\textbf{WISDOM} +1

\textbf{CHARISMA} -2

\textbf{Initiative} +0 -- \textbf{Defense} 14

\textbf{Hit Points} 90 (12d12 + 12)

\textbf{Movement} 0 m, swim 18 m

\textbf{Saving Throws}: Fortitude +9, Reflexes +8, Will +5

\textbf{Skills} Awareness +3

\textbf{Senses} blind sight 36 m

\textbf{Languages} -

\textbf{Challenge} 3 (700 PX)

\emph{\textbf{Echolocation.}} The whale cannot use blindsight if deafened.

\emph{\textbf{Hold your breath.}} The whale can hold its breath for 30 minutes

\emph{\textbf{Refined Hearing.}} The whale has +1d6 on Wisdom (Awareness) checks that rely on hearing.

\textbf{Shares}

\emph{\textbf{Bite.} Melee Weapon Attack}: +6 to hit, reach 1 m, one target.

\emph{Hits:} 21 (5d6 + 4) piercing damage.

\medskip\textbf{Axe Beak}\index[Monster]{Axe Beak}

The ax-bill is a large, slender bird without wings but with powerful legs, a wedge-shaped beak, and a very bad temper.

\emph{Great beast, misaligned}

\textbf{STRENGTH} +2

\textbf{DEXTERITY} +1

\textbf{CONSTITUTION} +1

\textbf{INTELLIGENCE} -4

\textbf{WISDOM} +0

\textbf{CHARRISMA} -3

\textbf{Initiative} +1 -- \textbf{Defense} 12

\textbf{Hit Points} 19 (3d10 + 3)

\textbf{Movement} 15 m

\textbf{Saving Throws}: Fortitude +3, Reflexes +1, Will +1

\textbf{Languages} -

\textbf{Challenge} 1/4 (50 PX)

\textbf{Shares}

\emph{\textbf{Beak.} Melee Weapon Attack}: +4 to hit, reach 1 m, one target.

\emph{Hits:} 6 (1d8 + 2) slashing damage.

\medskip\textbf{Death Dog}\index[Monster]{Death Dog}

The Death Dog is a hideous, two-headed hound that prowls plains, deserts, and dungeons.

\emph{Medium monstrosity, neutral evil}

\textbf{STRENGTH} +2

\textbf{DEXTERITY} +2

\textbf{CONSTITUTION} +2

\textbf{INTELLIGENCE} -4

\textbf{WISDOM} +1

\textbf{CHARISMA} -2

\textbf{Initiative} +2 -- \textbf{Defense} 13

\textbf{Hit Points} 39 (6d8 + 12)

\textbf{Movement} 12 m

\textbf{Saving Throws}: Fortitude +4, Reflexes +5, Will +2

\textbf{Skills} Stealth +4, Awareness +5

\textbf{Senses} vision in the dark 36 m

\textbf{Languages} -

\textbf{Challenge} 1 (200 PX)

\emph{\textbf{Double-headed.}} The dog has +1d6 on Awareness checks and on saving throws against blinded, charmed, deafened, frightened, stunned, or unconscious conditions.

\textbf{Shares}

\emph{\textbf{Multiattack.}} The dog makes two bite attacks.

\emph{\textbf{Bite.} Melee Weapon Attack}: +4 to hit, reach 1 m, one target.

\emph{Hits:} 5 (1d6 + 2) piercing damage. If the target is a creature, it must succeed on a DC 12 Fortitude saving throw against the disease or remain sick until the disease is cured. After every 24 hours, the creature must repeat the saving throw, reducing its maximum hit points by 5 (1d10) on a failure. This reduction lasts until the disease is treated. The creature dies if the disease reduces its maximum hit points to 0.

\medskip\textbf{Intermittent Dog}\index[Monster]{Intermittent Dog}

The blink dog derives its name from its ability to weave in and out of reality, a talent it uses to attack and avoid being attacked.

\emph{Medium fairy, legal good}

\textbf{STRENGTH} +1

\textbf{DEXTERITY} +3

\textbf{CONSTITUTION} +1

\textbf{INTELLIGENCE} +0

\textbf{WISDOM} +1

\textbf{CHARISMA} +0

\textbf{Initiative} +3 -- \textbf{Defense} 14

\textbf{Hit Points} 22 (4d8 + 4)

\textbf{Damage Vulnerability} cold iron

\textbf{Movement} 12 m

\textbf{Saving Throws}: Fortitude +5, Reflexes +5, Will +4

\textbf{Skills} Stealth +5, Awareness +3

\textbf{Languages} Intermittent dog, understands Sylvan but cannot speak it

\textbf{Challenge} 1/4 (50 XP)

\emph{\textbf{Refined Hearing and Smell.}} The dog has +1d6 on Wisdom (Awareness) checks that rely on hearing or smell.

\textbf{Shares}

\emph{\textbf{Bite.} Melee Weapon Attack}: +3 to hit, reach 1 m, one target.

\emph{Hits:} 4 (1d6 + 1) piercing damage.

€24890 € {€24891 € {Teleport (Recharge 4-6).}} The dog magically teleports, along with whatever it is wearing or carrying, up to 40 feet to an unoccupied space it can see. Before or after teleporting, the dog can make a bite attack.

\medskip\textbf{Caprone}\index[Monstruario]{Caprone}

\emph{Medium beast, misaligned}

\textbf{STRENGTH} +1

\textbf{DEXTERITY} +0

\textbf{CONSTITUTION} +0

\textbf{INTELLIGENCE} -4

\textbf{WISDOM} +0

\textbf{CHARRISMA} -3

\textbf{Initiative} +0 -- \textbf{Defense} 11

\textbf{Hit Points} 4 (1d8)

\textbf{Movement} 12 m

\textbf{Saving Throws}: Fortitude +1, Reflexes +1, Will +0

\textbf{Languages} -

\textbf{Challenge} 0 (10 PX)

\emph{\textbf{Charge.}} If the goat moves at least 20 feet towards the target and hits with a bill attack during the same round, the target takes an additional 2 (1d4) bludgeoning damage. If the target is a creature, it must succeed on a DC 10 Fortitude saving throw
or fall prone.

\emph{\textbf{Steady Feet.}} The goat has +1d6 on Fortitude and Reflex saving throws made against effects that would cause it to fall prone.

\textbf{Shares}

\emph{\textbf{Rostro.} Melee Weapon Attack}: +3 to hit, reach 1 m, one target.

\emph{Hits:} 3 (1d4 + 1) bludgeoning damage.

\medskip\textbf{Giant Goat}\index[Monster]{Giant Goat}

\emph{Big beast, misaligned}

\textbf{STRENGTH} +3

\textbf{DEXTERITY} +0

\textbf{CONSTITUTION} +1

\textbf{INTELLIGENCE} -4

\textbf{WISDOM} +1

\textbf{CHARRISMA} -2

\textbf{Initiative} +0 -- \textbf{Defense} 12

\textbf{Hit Points} 19 (3d10 + 3)

\textbf{Movement} 12 m

\textbf{Saving Throws}: Fortitude +4, Reflexes +1, Will +1

\textbf{Languages} -

\textbf{Challenge} 1/2 (100 PX)

\emph{\textbf{Charge.}} If the goat moves at least 20 feet towards the target and hits with a bill attack during the same round, the target takes an additional 5 (2d4) bludgeoning damage. If the target is a creature, it must succeed on a DC 13 Fortitude saving throw or fall prone.

\emph{\textbf{Steady Feet.}} The goat has +1d6 on Fortitude and Reflex saving throws made against effects that would cause it to fall prone.

\textbf{Shares}

\emph{\textbf{Rostro.} Melee Weapon Attack}: +5 to hit, reach 1 m, one target.

\emph{Hits:} 8 (2d4 + 3) bludgeoning damage.

\medskip\textbf{Race Horse}\index[Monster]{Race Horse}

\emph{Big beast, misaligned}

\textbf{STRENGTH} +3

\textbf{DEXTERITY} +0

\textbf{CONSTITUTION} +1

\textbf{INTELLIGENCE} -4

\textbf{WISDOM} +0

\textbf{CHARISMA} -2

\textbf{Initiative} +0 -- \textbf{Defense} 11

\textbf{Hit Points} 13 (2d10 + 2)

\textbf{Movement} 18 m

\textbf{Saving Throws}: Fortitude +3, Reflexes +1, Will +1

\textbf{Languages} -

\textbf{Challenge} 1/4 (50 PX)

\textbf{Shares}

\emph{\textbf{Hooves.} Melee Weapon Attack}: +5 to hit, reach 1 m, one target.

\emph{Hits:} 8 (2d4 + 3) bludgeoning damage.

\medskip\textbf{War Horse}\index[Monster]{War Horse}

\emph{Big beast, misaligned}

\textbf{STRENGTH} +4

\textbf{DEXTERITY} +1

\textbf{CONSTITUTION} +1

\textbf{INTELLIGENCE} -4

\textbf{WISDOM} +1

\textbf{CHARISMA} -2

\textbf{Initiative} +1 -- \textbf{Defense} 12 (plus possible barding)

\textbf{Hit Points} 19 (3d10 + 3)

\textbf{Movement} 18 m

\textbf{Saving Throws}: Fortitude +4, Reflexes +2, Will +1

\textbf{Languages} -

\textbf{Challenge} 1/2 (100 PX)

\emph{\textbf{Sweeping Charge.}} If the horse moves at least 20 feet towards the target and hits it with a hoof attack during the same round, the target must succeed on a Fortitude save DC 14 or fall prone. If the target is prone, the horse can make another hoof attack against it as a bonus action.

\textbf{Shares}

\emph{\textbf{Hooves.} Melee Weapon Attack}: +6 to hit, reach 1 m, one target.

\emph{Hits:} 11 (2d6 + 4) bludgeoning damage.

\medskip\textbf{Draft Horse}\index[Monster]{Draft Horse}

\emph{Big beast, misaligned}

\textbf{STRENGTH} +4

\textbf{DEXTERITY} +0

\textbf{CONSTITUTION} +1

\textbf{INTELLIGENCE} -4

\textbf{WISDOM} +0

\textbf{CHARISMA} -2

\textbf{Initiative} +0 -- \textbf{Defense} 11

\textbf{Hit Points} 19 (3d10 + 3)

\textbf{Movement} 12 m

\textbf{Saving Throws}: Fortitude +5, Reflexes +1, Will +2

\textbf{Languages} -

\textbf{Challenge} 1/4 (50 PX)

\textbf{Shares}

\emph{\textbf{Hooves.} Melee Weapon Attack}: +6 to hit, reach 1 m, one target.

\emph{Hits:} 9 (2d4 + 4) bludgeoning damage.

\medskip\textbf{Giant Sea Horse}\index[Monster]{Giant Sea Horse}

The giant seahorse is often used as a mount by aquatic humanoids.

\emph{Great beast, misaligned}

\textbf{STRENGTH} +1

\textbf{DEXTERITY} +2

\textbf{CONSTITUTION} +0

\textbf{INTELLIGENCE} -4

\textbf{WISDOM} +1

\textbf{CHARRISMA} -3

\textbf{Initiative} +2 -- \textbf{Defense} 14

\textbf{Hit Points} 16 (3d10)

\textbf{Movement} 0 m, swim 12 m

\textbf{Saving Throws}: Fortitude +2, Reflexes +3, Will +1

\textbf{Languages} -

\textbf{Challenge} 1/2 (100 PX)

\emph{\textbf{Charge.}} If the seahorse moves at least 20 feet towards the target and hits with a bill attack during the same round, the target takes an additional 7 (2d6) bludgeoning damage. If the target is a creature, it must succeed on a DC 11 Fortitude save or be knocked prone.

\emph{\textbf{Breathe Water.}} The seahorse can only breathe underwater.

\textbf{Shares}

\emph{\textbf{Rostro.} Melee Weapon Attack}: +3 to hit, reach 1 m, one target.

\emph{Hits:} 4 (1d6 + 1) bludgeoning damage.

\medskip\textbf{Giant Centipede}\index{Giant Centipede}

\emph{Small beast, misaligned}

\textbf{STRENGTH} -3

\textbf{DEXTERITY} +2

\textbf{CONSTITUTION} +1

\textbf{INTELLIGENCE} -5

\textbf{WISDOM} -2

\textbf{CHARISMA} -4

\textbf{Initiative} +2 -- \textbf{Defense} 14

\textbf{Hit Points} 4 (1d6 + 1)

\textbf{Movement} 9m, climb 9m

\textbf{Saving Throws}: Fortitude -2, Reflexes +3, Will -2

\textbf{Senses} blind sight 9 m

\textbf{Languages} -

\textbf{Challenge} 1/4 (50 PX)

\textbf{Shares}

\emph{\textbf{Bite.} Melee Weapon Attack}: +4 to hit, reach 3 ft., one creature.

\emph{Hits:} 4 (1d4 + 2) piercing damage and the target must succeed on a DC 11 Fortitude saving throw or take 10 (3d6) poison damage. If poison damage reduces the target to 0 Hit Points, the target is stable but remains poisoned for 1 hour, even after regaining Hit Points, and is paralyzed while poisoned in this way.

\medskip\textbf{Deer}\index[Monster]{Deer}

\emph{Medium beast, misaligned}

\textbf{STRENGTH} +0

\textbf{DEXTERITY} +3

\textbf{CONSTITUTION} +0

\textbf{INTELLIGENCE} -4

\textbf{WISDOM} +2

\textbf{CHARRISMA} -3

\textbf{Initiative} +3 -- \textbf{Defense} 14

\textbf{Hit Points} 4 (1d8)

\textbf{Movement} 15 m

\textbf{Saving Throws}: Fortitude +2, Reflexes +3, Will +2

\textbf{Languages} -

\textbf{Challenge} 0 (10 PX)

\textbf{Shares}

\emph{\textbf{Bite.} Melee Weapon Attack}: +2 to hit, reach 1 m, one target.

\emph{Hits:} 2 (1d4) piercing damage.

\medskip\textbf{Boar}\index[Monster]{Boar}

\emph{Medium beast, misaligned}

\textbf{STRENGTH} +1

\textbf{DEXTERITY} +0

\textbf{CONSTITUTION} +1

\textbf{INTELLIGENCE} -4

\textbf{WISDOM} -1

\textbf{CHARRISMA} -3

\textbf{Initiative} +0 -- \textbf{Defense} 12

\textbf{Hit Points} 11 (2d8 + 2)

\textbf{Movement} 12 m

\textbf{Saving Throws}: Fortitude +2, Reflexes +1, Will -1

\textbf{Languages} -

\textbf{Challenge} 1/4 (50 PX)

\emph{\textbf{Charge.}} If the boar moves at least 20 feet towards the target and hits with a tusk attack during the same round, the target takes an additional 3 (1d6) slashing damage. If the target is a creature, it must succeed on a Fortitude saving throw
DC 11 or fall prone.

\emph{\textbf{Relentless (Recharge after 1 hour).}} If the boar takes 7 or fewer points of damage that would reduce it to 0 hit points, it instead drops to 1 hit point.

\textbf{Shares}

\emph{\textbf{Fang.} Melee Weapon Attack}: +3 to hit, reach 1 m, one target.

\emph{Hits:} 4 (1d6 + 1) slashing damage.

\medskip\textbf{Giant Boar}\index[Monster]{Giant Boar}

\emph{Great beast, misaligned}

\textbf{STRENGTH} +3

\textbf{DEXTERITY} +0

\textbf{CONSTITUTION} +3

\textbf{INTELLIGENCE} -4

\textbf{WISDOM} -2

\textbf{CHARRISMA} -3

\textbf{Initiative} +0 -- \textbf{Defense} 13

\textbf{Hit Points} 42 (5d10 + 15)

\textbf{Movement} 12 m

\textbf{Saving Throws}: Fortitude +4, Reflexes +2, Will +0

\textbf{Languages} -

\textbf{Challenge} 2 (450 PX)

\emph{\textbf{Charge.}} If the boar moves at least 20 feet towards the target and hits with a tusk attack during the same round, the target takes an additional 7 (2d6) slashing damage. If the target is a creature, it must succeed on a DC 13 Fortitude save or fall prone.

\emph{\textbf{Relentless (Recharge after 1 hour).}} If the boar takes 10 or fewer points of damage that would reduce it to 0 hit points, it instead drops to 1 hit point.

\textbf{Shares}

\emph{\textbf{Fang.} Melee Weapon Attack}: +5 to hit, reach 1 m, one target.

\emph{Hits:} 10 (2d6 + 3) slashing damage.

\medskip\textbf{Crocodile}\index[Monster]{Crocodile}

\emph{Great beast, misaligned}

\textbf{STRENGTH} +2

\textbf{DEXTERITY} +0

\textbf{CONSTITUTION} +1

\textbf{INTELLIGENCE} -4

\textbf{WISDOM} +0

\textbf{CHARISMA} -3

\textbf{Initiative} +0 -- \textbf{Defense} 13

\textbf{Hit Points} 19 (3d10 + 3)

\textbf{Movement} 6m, swim 9m

\textbf{Saving Throws}: Fortitude +6, Reflexes +4, Will +2

\textbf{Skills} Stealth +2

\textbf{Languages} -

\textbf{Challenge} 1/2 (100 PX)

\emph{\textbf{Hold Your Breath.}} The crocodile can hold its breath for 15 minutes.

\textbf{Shares}

\emph{\textbf{Bite.} Melee Weapon Attack}: +4 to hit, reach 3 ft., one creature.

\emph{Hits:} 7 (1d10 + 2) piercing damage, and the target is grappled (DC 12 to escape). Until the grapple ends, the target is entangled, and the crocodile cannot use its bite against another target.

\medskip\textbf{Giant Crocodile}\index[Monster]{Giant Crocodile}

\emph{Huge beast, misaligned}

\textbf{STRENGTH} +5

\textbf{DEXTERITY} -1

\textbf{CONSTITUTION} +3

\textbf{INTELLIGENCE} -4

\textbf{WISDOM} +0

\textbf{CHARRISMA} -2

\textbf{Initiative} -1 -- \textbf{Defense} 15

\textbf{Hit Points} 85 (9d12 + 27)

\textbf{Move} 9 m, swim 15 m

\textbf{Saving Throws}: Fortitude +15, Reflexes +8, Will +8

\textbf{Skills} Stealth +5

\textbf{Languages} -

\textbf{Challenge} 5 (1800 PX)

\emph{\textbf{Hold Your Breath.}} The crocodile can hold its breath for 30 minutes.

\textbf{Shares}

\emph{\textbf{Multiattack.}} The crocodile makes two attacks: one with its bite and one with its tail.

\emph{\textbf{Tail.} Melee Weapon Attack}: +8 to hit, reach 3 m, a target not grabbed by the crocodile.

\emph{Hits:} 14 (2d8 + 5) bludgeoning damage. If the target is a creature, it must succeed on a DC 16 Fortitude save or fall prone.

\emph{\textbf{Bite.} Melee Weapon Attack}: +8 to hit, reach 1 m, one target.

\emph{Hits:} 21 (3d10 + 5) piercing damage, and the target is grappled (DC 16 to escape). Until the grapple ends, the target is entangled, and the crocodile cannot use its bite against another target.

\medskip\textbf{Raven}\index[Monstery]{Raven}

\emph{Tiny beast, misaligned}

\textbf{STRENGTH} -4

\textbf{DEXTERITY} +2

\textbf{CONSTITUTION} -1

\textbf{INTELLIGENCE} -4

\textbf{WISDOM} +1

\textbf{CHARISMA} -2

\textbf{Initiative} +2 -- \textbf{Defense} 13

\textbf{Hit Points} 1 (1d4 - 1)

\textbf{Movement} 3 m, flight 15 m

\textbf{Saving Throws}: Fortitude +1, Reflexes +4, Will +2

\textbf{Skills} Awareness +3

\textbf{Languages} -

\textbf{Challenge} 0 (10 PX)

\emph{\textbf{Imitation.}} The crow can imitate simple sounds it has heard, such as a person's whisper, a child's cry or an animal's cry. A creature that hears the sound can identify it as an imitation with a successful DC 10 Wisdom (Survival) check.

\textbf{Shares}

\emph{\textbf{Beak.} Melee Weapon Attack}: +4 to hit, reach 1 m, one target.

\emph{Hits:} 1 piercing damage.

%\medskip\textbf{Weasel}\index[Monster]{Weasel}

%\emph{Tiny beast, misaligned}

%\textbf{STRENGTH} -4

%\textbf{DEXTERITY} +3

%\textbf{CONSTITUTION} -1

%\textbf{INTELLIGENCE} -4

%\textbf{WISDOM} +1

%\textbf{CHARRISMA} -4

%\textbf{Initiative} +3 -- \textbf{Defense} 14

%\textbf{Hit Points} 1 (1d4 - 1)

%\textbf{Movement} 9 m

%\textbf{Saving Throws}: Fortitude +2, Reflexes +4, Will +1

%\textbf{Skills} Stealth +5, Awareness +3

%\textbf{Languages} -

%\textbf{Challenge} 0 (10 PX)

%\emph{\textbf{Hearing and Smell Sharp.}} The weasel has +1d6 on Wisdom (Awareness) checks that rely on hearing or smell.

%\textbf{Shares}

%\emph{\textbf{Bite.} Melee Weapon Attack}: +5 to hit, reach 1 m, one target.

%\emph{Hits:} 1 piercing damage.

\medskip\textbf{Giant Weasel}\index[Monster]{Giant Weasel}

\emph{Medium beast, misaligned}

\textbf{STRENGTH} +0

\textbf{DEXTERITY} +3

\textbf{CONSTITUTION} +0

\textbf{INTELLIGENCE} -3

\textbf{WISDOM} +1

\textbf{CHARRISMA} -3

\textbf{Initiative} +3 -- \textbf{Defense} 14

\textbf{Hit Points} 9 (2d8)

\textbf{Movement} 12 m

\textbf{Saving Throws}: Fortitude +6, Reflexes +7, Will +2

\textbf{Skills} Stealth +5, Awareness +3

\textbf{Senses} vision in the dark 18 m

\textbf{Languages} -

\textbf{Challenge} 1/8 (25 PX)

\emph{\textbf{Hearing and Smell Sharp.}} The weasel has +1d6 on Wisdom (Awareness) checks that rely on hearing or smell.

\textbf{Shares}

\emph{\textbf{Bite.} Melee Weapon Attack}: +5 to hit, reach 1 m, one target.

\emph{Hits:} 5 (1d4 + 3) piercing damage.

\medskip\textbf{Elephant}\index[Monstery]{Elephant}

\emph{Huge beast, misaligned}

\textbf{STRENGTH} +6

\textbf{DEXTERITY} -1

\textbf{CONSTITUTION} +3

\textbf{INTELLIGENCE} -4

\textbf{WISDOM} +0

\textbf{CHARISMA} -2

\textbf{Initiative} -1 -- \textbf{Defense} 14

\textbf{Hit Points} 76 (8d12 + 24)

\textbf{Movement} 12 m

\textbf{Saving Throws}: Fortitude +13, Reflexes +7, Will +6

\textbf{Languages} -

\textbf{Challenge} 4 (1000 PX)

\emph{\textbf{Sweeping Charge.}} If the elephant moves at least 20 feet towards a creature and hits it with a gore attack during the same round, the target must succeed on a Fortitude saving throw DC 12 or fall prone. If the target is prone, the elephant can make a stomp attack against it as a bonus action.

\textbf{Shares}

\emph{\textbf{Gored.} Melee Weapon Attack}: +8 to hit, reach 1 m, one target.

\emph{Hits:} 19 (3d8 + 6) piercing damage.

\emph{\textbf{Stomp.} Melee Weapon Attack}: +8 to hit, reach 1 m, one prone target.

\emph{Hits:} 22 (3d10 + 6) bludgeoning damage.

\medskip\textbf{Hawk}\index[Monster]{Hawk}

\emph{Tiny beast, misaligned}

\textbf{STRENGTH} -3

\textbf{DEXTERITY} +3

\textbf{CONSTITUTION} -1

\textbf{INTELLIGENCE} -4

\textbf{WISDOM} +2

\textbf{CHARRISMA} -2

\textbf{Initiative} +3 -- \textbf{Defense} 14

\textbf{Hit Points} 1 (1d4 - 1)

\textbf{Movement} 3 m, flight 18 m

\textbf{Saving Throws}: Fortitude +2, Reflexes +5, Will +2

\textbf{Skills} Awareness +4

\textbf{Languages} -

\textbf{Challenge} 0 (10 PX)

\emph{\textbf{Honed Sight.}} The hawk has +1d6 on Wisdom (Awareness) checks that rely on sight.

\textbf{Shares}

\emph{\textbf{Spurs.} Melee Weapon Attack}: +5 to hit, reach 1 m, one target.

\emph{Hits:} 1 slashing damage.

\medskip\textbf{Blood Hawk}\index[Monstery]{Blood Hawk}

Named after its crimson feathers and aggressive nature, the blood hawk fearlessly attacks using its sharp beak.

\emph{Small beast, misaligned}

\textbf{STRENGTH} -2

\textbf{DEXTERITY} +2

\textbf{CONSTITUTION} +0

\textbf{INTELLIGENCE} -4

\textbf{WISDOM} +2

\textbf{CHARRISMA} -3

\textbf{Initiative} +2 -- \textbf{Defense} 13

\textbf{Hit Points} 7 (2d6)

\textbf{Movement} 3 m, flight 18 m

\textbf{Saving Throws}: Fortitude +3, Reflexes +6, Will +3

\textbf{Skills} Awareness +4

\textbf{Languages} -

\textbf{Challenge} 1/8 (25 PX)

\emph{\textbf{Pack Tactics.}} The hawk has +1d6 to attack rolls against a creature if at least one of the hawk's allies is within 3 feet of the creature and that ally is not incapacitated.

\emph{\textbf{Honed Sight.}} The hawk has +1d6 on Wisdom (Awareness) checks that rely on sight.

\textbf{Shares}

\emph{\textbf{Beak.} Melee Weapon Attack}: +4 to hit, reach 1 m, one target.

\emph{Hits:} 4 (1d4 + 2) piercing damage.

\medskip\textbf{Pirana}\index[Monstery]{Pirana}

The pirana is a carnivorous fish with sharp teeth.

\emph{Tiny beast, misaligned}

\textbf{STRENGTH} -4

\textbf{DEXTERITY} +3

\textbf{CONSTITUTION} -1

\textbf{INTELLIGENCE} -5

\textbf{WISDOM} -2

\textbf{CHARRISMA} -4

\textbf{Initiative} +3 -- \textbf{Defense} 14

\textbf{Hit Points} 1 (1d4 - 1)

\textbf{Movement} 0 m, swim 12 m

\textbf{Saving Throws}: Fortitude -4, Reflexes +3, Will -2

\textbf{Senses} vision in the dark 18 m

\textbf{Languages} -

\textbf{Challenge} 0 (10 PX)

\emph{\textbf{Blood Frenzy.}} The pirana has +1d6 on melee attack rolls against any creature that is not at full hit points.

\emph{\textbf{Water Breathing.}} The pirana can only breathe underwater.

\textbf{Shares}

\emph{\textbf{Bite.} Melee Weapon Attack}: +5 to hit, reach 1 m, one target.

\emph{Hits:} 1 piercing damage.

\medskip\textbf{Cat}\index[Monster]{Cat}

\emph{Tiny beast, misaligned}

\textbf{STRENGTH} -4

\textbf{DEXTERITY} +2

\textbf{CONSTITUTION} +0

\textbf{INTELLIGENCE} -4

\textbf{WISDOM} +1

\textbf{CHARRISMA} -2

\textbf{Initiative} +2 -- \textbf{Defense} 13

\textbf{Hit Points} 2 (1d4)

\textbf{Movement} 12m, climb 9m

\textbf{Saving Throws}: Fortitude +1, Reflexes +4, Will +1

\textbf{Skills} Stealth +4, Awareness +3

\textbf{Languages} -

\textbf{Challenge} 0 (10 PX)

\emph{\textbf{Sense of Smell.}} The cat has +1d6 on Wisdom (Awareness) checks that rely on smell.

\textbf{Shares}

\emph{\textbf{Claws.} Melee Weapon Attack}: +0 to hit, reach 1 m, one target.

\emph{Hits:} 1 slashing damage.

\medskip\textbf{Giant Crab}\index[Monster]{Giant Crab}

\emph{Medium beast, misaligned}

\textbf{STRENGTH} +1

\textbf{DEXTERITY} +2

\textbf{CONSTITUTION} +0

\textbf{INTELLIGENCE} -5

\textbf{WISDOM} -1

\textbf{CHARRISMA} -4

\textbf{Initiative} +2 -- \textbf{Defense} 16

\textbf{Hit Points} 13 (3d8)

\textbf{Movement} 9m, swim 9m

\textbf{Saving Throws}: Fortitude +5, Reflexes +2, Will +1

\textbf{Skills} Stealth +4

\textbf{Senses} blind sight 9 m

\textbf{Languages} -

\textbf{Challenge} 1/8 (25 PX)

\emph{\textbf{Amphibian.}} The crab can breathe air and water.

\textbf{Shares}

\emph{\textbf{Claw (Claw).} Melee Weapon Attack}: +3 to hit, reach 1 m, one target.

\emph{Hits:} 4 (1d6 + 1) bludgeoning damage and the target is grappled (DC 11 to escape). The crab has two claws, each of which can grasp only one target.

\medskip\textbf{Owl}\index[Monster]{Owl}

\emph{Tiny beast, misaligned}

\textbf{STRENGTH} -4

\textbf{DEXTERITY} +1

\textbf{CONSTITUTION} -1

\textbf{INTELLIGENCE} -4

\textbf{WISDOM} +1

\textbf{CHARRISMA} -2

\textbf{Initiative} +1 -- \textbf{Defense} 12

\textbf{Hit Points} 1 (1d4 - 1)

\textbf{Movement} 1 m, flight 18 m

\textbf{Saving Throws}: Fortitude +2, Reflexes +5, Will +2

\textbf{Skills} Stealth +3, Awareness +3

\textbf{Senses} vision in the dark 36 m

\textbf{Languages} -

\textbf{Challenge} 0 (10 PX)

\emph{\textbf{Flying.}} The owl does not provoke attacks of opportunity when it flies out of an enemy's reach.

\emph{\textbf{Sharpened Hearing and Sight.}} The owl has +1d6 on Wisdom (Awareness) checks that rely on hearing or sight.

\textbf{Shares}

\emph{\textbf{Spurs.} Melee Weapon Attack}: +3 to hit, reach 1 m, one target.

\emph{Hits:} 1 slashing damage.

\medskip\textbf{Giant Owl}\index[Monster]{Giant Owl}

Giant owls are intelligent creatures that protect the sylvan kingdoms.

\emph{Great beast, neutral}

\textbf{STRENGTH} +1

\textbf{DEXTERITY} +2

\textbf{CONSTITUTION} +1

\textbf{INTELLIGENCE} -1

\textbf{WISDOM} +1

\textbf{CHARISMA} +0

\textbf{Initiative} +2 -- \textbf{Defense} 13

\textbf{Hit Points} 19 (3d10 + 3)

\textbf{Movement} 1 m, flight 18 m

\textbf{Saving Throws}: Fortitude +1, Reflexes +4, Will +1

\textbf{Skills} Stealth +4, Awareness +5

\textbf{Senses} vision in the dark 36 m

\textbf{Languages} Giant Owl, includes Common, Elvish and Sylvan but cannot speak them

\textbf{Challenge} 1/4 (50 XP)

\emph{\textbf{Flying.}} The owl does not provoke attacks of opportunity when it flies out of an enemy's reach.

\emph{\textbf{Sharpened Hearing and Sight.}} The owl has +1d6 on Wisdom (Awareness) checks that rely on hearing or sight.

\textbf{Shares}

\emph{\textbf{Spurs.} Melee Weapon Attack}: +3 to hit, reach 1 m, one target.

\emph{Hits:} 8 (2d6 + 1) piercing damage.

\medskip\textbf{Hyena}\index[Monster]{Hyena}

\emph{Medium beast, misaligned}

\textbf{STRENGTH} +0

\textbf{DEXTERITY} +1

\textbf{CONSTITUTION} +1

\textbf{INTELLIGENCE} -4

\textbf{WISDOM} +1

\textbf{CHARRISMA} -3

\textbf{Initiative} +1 -- \textbf{Defense} 12

\textbf{Hit Points} 5 (1d8 + 1)

\textbf{Movement} 15 m

\textbf{Saving Throws}: Fortitude +5, Reflexes +5, Will +1

\textbf{Skills} Awareness +3

\textbf{Languages} -

\textbf{Challenge} 0 (10 PX)

\emph{\textbf{Pack Tactics.}} The hyena has +1d6 to attack rolls against a creature if at least one of the hyena's allies is within 3 feet of the creature and that ally is not incapacitated.

\textbf{Shares}

\emph{\textbf{Bite.} Melee Weapon Attack}: +2 to hit, reach 1 m, one target.

\emph{Hits:} 3 (1d6) piercing damage.

\medskip\textbf{Giant Hyena}\index[Monster]{Giant Hyena}

\emph{Great beast, misaligned}

\textbf{STRENGTH} +3

\textbf{DEXTERITY} +2

\textbf{CONSTITUTION} +2

\textbf{INTELLIGENCE} -4

\textbf{WISDOM} +1

\textbf{CHARRISMA} -2

\textbf{Initiative} +2 -- \textbf{Defense} 13

\textbf{Hit Points} 45 (6d10 + 12)

\textbf{Movement} 15 m

\textbf{Saving Throws}: Fortitude +6, Reflexes +6, Will +2

\textbf{Skills} Awareness +3

\textbf{Languages} -

\textbf{Challenge} 1 (200 PX)

\emph{€25499{Rage.}} When the hyena reduces a creature to 0 hit points with a melee attack during its round, the hyena can take a bonus action to move up to half its move and take a bite attack.

\textbf{Shares}

\emph{\textbf{Bite.} Melee Weapon Attack}: +5 to hit, reach 1 m, one target.

\emph{Hits:} 10 (2d6 + 3) piercing damage.

\medskip\textbf{Lion}\index[Monster]{Lion}

\emph{Great beast, misaligned}

\textbf{STRENGTH} +3

\textbf{DEXTERITY} +2

\textbf{CONSTITUTION} +1

\textbf{INTELLIGENCE} -4

\textbf{WISDOM} +1

\textbf{CHARRISMA} -1

\textbf{Initiative} +2 -- \textbf{Defense} 13

\textbf{Hit Points} 26 (4d10 + 4)

\textbf{Movement} 15 m

\textbf{Saving Throws}: Fortitude +6, Reflexes +7, Will +2

\textbf{Skills} Stealth +6, Awareness +3

\textbf{Languages} -

\textbf{Challenge} 1 (200 PX)

\emph{\textbf{Leap.}} If the lion moves at least 20 feet towards a creature and hits it with a claw attack during the same round, the target must succeed on a DC 13 Fortitude saving throw or fall prone. If the target is prone, the lion can make a
bite attack as a bonus action.

\emph{\textbf{Sense of Smell.}} The lion has +1d6 on Wisdom (Awareness) checks that rely on smell.

\emph{\textbf{Jump with Run-Up.}} With 3 meters of run-up, the lion can long jump up to 7 meters.

\emph{\textbf{Pride Tactics.}} The lion has +1d6 to attack rolls against a creature if at least one of the lion's allies is within 3 feet of the creature and that ally is not incapacitated.

\textbf{Shares}

\emph{\textbf{Claw.} Melee Weapon Attack}: +5 to hit, reach 1 m, one target.

\emph{Hits:} 6 (1d6 + 3) slashing damage, 1 bleed damage.

\emph{\textbf{Bite.} Melee Weapon Attack}: +5 to hit, reach 1 m, one target.

\emph{Hits:} 7 (1d8 + 3) piercing damage.

\medskip\textbf{Lizard}\index[Monster]{Lizard}

\emph{Tiny beast, misaligned}

\textbf{STRENGTH} -4

\textbf{DEXTERITY} +0

\textbf{CONSTITUTION} +0

\textbf{INTELLIGENCE} -5

\textbf{WISDOM} -1

\textbf{CHARRISMA} -4

\textbf{Initiative} +0 -- \textbf{Defense} 11

\textbf{Hit Points} 2 (1d4)

\textbf{Movement} 6m, climb 6m

\textbf{Saving Throws}: Fortitude +1, Reflexes +4, Will +1

\textbf{Senses} vision in the dark 9 m

\textbf{Languages} -

\textbf{Challenge} 0 (10 PX)

\emph{\textbf{Climb as a Spider.}} The lizard can climb difficult surfaces, including standing upside down on the ceiling, without needing to make an ability check.

\textbf{Shares}

\emph{\textbf{Bite.} Melee Weapon Attack}: +0 to hit, reach 1 m, one target.

\emph{Hits:} 1 piercing damage.

\medskip\textbf{Giant Lizard}\index[Monster]{Giant Lizard}

Giant lizards are fearsome predators and are often used as mounts or draft animals by reptilian humanoids and underground residents.

\emph{Great beast, misaligned}

\textbf{STRENGTH} +2

\textbf{DEXTERITY} +1

\textbf{CONSTITUTION} +1

\textbf{INTELLIGENCE} -4

\textbf{WISDOM} +0

\textbf{CHARRISMA} -3

\textbf{Initiative} +1 -- \textbf{Defense} 13

\textbf{Hit Points} 19 (3d10 + 3)

\textbf{Movement} 9m, climb 9m

\textbf{Saving Throws}: Fortitude +11, Reflexes +8, Will +4

\textbf{Senses} vision in the dark 9 m

\textbf{Languages} -

\textbf{Challenge} 1/4 (50 PX)

\textbf{Shares}

\emph{\textbf{Bite.} Melee Weapon Attack}: +4 to hit, reach 1 m, one target.

\emph{Hits:} 6 (1d8 + 2) piercing damage.

\textbf{VARIANT}

Some giant lizards possess one or both of the following traits.

\emph{\textbf{Climb as a Spider.}} The lizard can climb difficult surfaces, including standing upside down on the ceiling, without needing to make an ability check.

\emph{\textbf{Hold Your Breath.}} The lizard can hold its breath for 15 minutes. (A lizard with this trait also has a swimming speed of 30 feet.)

\medskip\textbf{Wolf}\index[Monster]{Wolf}

\emph{Medium beast, misaligned}

\textbf{STRENGTH} +1

\textbf{DEXTERITY} +2

\textbf{CONSTITUTION} +1

\textbf{INTELLIGENCE} -4

\textbf{WISDOM} +1

\textbf{CHARISMA} -2

\textbf{Initiative} +2 -- \textbf{Defense} 14

\textbf{Hit Points} 11 (2d8 + 2)

\textbf{Movement} 12 m

\textbf{Saving Throws}: Fortitude +5, Reflexes +5, Will +1

\textbf{Skills} Stealth +4, Awareness +3

\textbf{Languages} -

\textbf{Challenge} 1/4 (50 PX)

\emph{\textbf{Refined Hearing and Smell.}} The wolf has +1d6 on Wisdom (Awareness) checks that rely on hearing or smell.

\emph{\textbf{Pack Tactics.}} The wolf has +1d6 to attack rolls against a creature if at least one of the wolf's allies is within 3 feet of the creature and that ally is not incapacitated.

\textbf{Shares}

\emph{\textbf{Bite.} Melee Weapon Attack}: +4 to hit, reach 1 m, one target.

\emph{Hits:} 7 (2d4 + 2) piercing damage. If the target is a creature, it must succeed on a DC 11 Fortitude save or fall prone.

\medskip\textbf{Dinowolf (Direwolf)}\index[Monster]{Dinowolf (Direwolf}

\emph{Great beast, misaligned}

\textbf{STRENGTH} +3

\textbf{DEXTERITY} +2

\textbf{CONSTITUTION} +2

\textbf{INTELLIGENCE} -2

\textbf{WISDOM} +1

\textbf{CHARRISMA} -2

\textbf{Initiative} +2 -- \textbf{Defense} 15

\textbf{Hit Points} 37 (5d10 + 10)

\textbf{Movement} 15 m

\textbf{Saving Throws}: Fortitude +7, Reflexes +6, Will +2

\textbf{Skills} Stealth +4, Awareness +3

\textbf{Languages} -

\textbf{Challenge} 1 (200 PX)

\emph{\textbf{Refined Hearing and Smell.}} The wolf has +1d6 on Wisdom (Awareness) checks that rely on hearing or smell.

\emph{\textbf{Pack Tactics.}} The wolf has +1d6 to attack rolls against a creature if at least one of the wolf's allies is within 3 feet of the creature and that ally is not incapacitated.

\textbf{Shares}

\emph{\textbf{Bite.} Melee Weapon Attack}: +5 to hit, reach 1 m, one target.

\emph{Hits:} 10 (2d6 + 3) piercing damage. If the target is a creature, it must succeed on a DC 13 Fortitude save or fall prone.

\medskip\textbf{Winter Wolf}\index[Monster]{Winter Wolf}

Winter wolves live in the arctic regions and are evil, intelligent creatures with snow-white fur and ice-colored eyes.

\emph{Large monstrosity, neutral evil}

\textbf{STRENGTH} +4

\textbf{DEXTERITY} +1

\textbf{CONSTITUTION} +2

\textbf{INTELLIGENCE} -2

\textbf{WISDOM} +1

\textbf{CHARRISMA} -1

\textbf{Initiative} +1 -- \textbf{Defense} 15

\textbf{Hit Points} 75 (10d10 + 20)

\textbf{Movement} 15 m

\textbf{Saving Throws}: Fortitude +9, Reflexes +6, Will +3

\textbf{Skills} Stealth +3, Awareness +5

\textbf{Damage Immunity} cold

\textbf{Languages} Common, Giant, Winter Wolf

\textbf{Challenge} 3 (700 PX)

\emph{\textbf{Snow Camouflage.}} The wolf has +1d6 on Stealth (Hide) checks made to hide on snowy ground.

\emph{\textbf{Refined Hearing and Smell.}} The wolf has +1d6 on Wisdom (Awareness) checks that rely on hearing or smell.

\emph{\textbf{Pack Tactics.}} The wolf has +1d6 to attack rolls against a creature if at least one of the wolf's allies is within 3 feet of the creature and that ally is not incapacitated.

\textbf{Shares}

\emph{\textbf{Bite.} Melee Weapon Attack}: +6 to hit, reach 1 m, one target.

\emph{Hits:} 11 (2d6 + 4) piercing damage. If the target is a creature, it must succeed on a DC 14 Fortitude save or fall prone.

\emph{\textbf{Icy Breath (Recharge 5-6).}} The wolf exhales a blast of icy wind in a 5 meter cone. Each creature in that area must make a DC 12 Reflex saving throw, and take 18 (4d8) cold damage on a failed save, or half as much damage on a successful one.

\medskip\textbf{Mammoth}\index[Monstery]{Mammoth}

The mammoth is an elephant-like creature with thick fur and long tusks.

\emph{Huge beast, misaligned}

\textbf{STRENGTH} +7

\textbf{DEXTERITY} -1

\textbf{CONSTITUTION} +5

\textbf{INTELLIGENCE} -4

\textbf{WISDOM} +0

\textbf{CHARISMA} -2

\textbf{Initiative} -1 -- \textbf{Defense} 16

\textbf{Hit Points} 126 (11d12 + 55)

\textbf{Movement} 12 m

\textbf{Saving Throws}: Fortitude +14, Reflexes +10, Will +7

\textbf{Languages} -

\textbf{Challenge} 6 (2300 XP)

\emph{\textbf{Sweeping Charge.}} If the mammoth moves at least 20 feet towards a creature and hits it with a gore attack during the same round, the target must succeed on a DC 18 Fortitude saving throw or fall prone. If the target is prone, the mammoth can make a stomp attack against it as a bonus action.

\textbf{Shares}

\emph{\textbf{Gored.} Melee Weapon Attack}: +10 to hit, reach 10 ft., one target.

\emph{Hits:} 25 (4d8 + 7) piercing damage.

\emph{\textbf{Stomp.} Melee Weapon Attack}: +10 to hit, reach 1 m, one prone creature.

\emph{Hits:} 29 (4d10 + 7) bludgeoning damage.

\medskip\textbf{Mastiff}\index[Monster]{Mastiff}

\textbf{The} mastiffs are impressive hounds prized by humanoids for their reality and sharpened senses.

\emph{Medium beast, misaligned}

\textbf{STRENGTH} +1

\textbf{DEXTERITY} +2

\textbf{CONSTITUTION} +1

\textbf{INTELLIGENCE} -4

\textbf{WISDOM} +1

\textbf{CHARISMA} -2

\textbf{Initiative} +2 -- \textbf{Defense} 13

\textbf{Hit Points} 5 (1d8 + 1)

\textbf{Movement} 12 m

\textbf{Saving Throws}: Fortitude +3, Reflexes +3, Will +1

\textbf{Skills} Awareness +3, Tracking +3

\textbf{Languages} -

\textbf{Challenge} 1/8 (25 PX)

\emph{\textbf{Refined Hearing and Smell.}} The mastiff has +1d6 on Wisdom (Awareness) checks that rely on hearing or smell.

\textbf{Shares}

\emph{\textbf{Bite.} Melee Weapon Attack}: +3 to hit, reach 1 m, one target.

\emph{Hits:} 4 (1d6 + 1) piercing damage. If the target is a creature, it must succeed on a DC 11 Fortitude save or fall prone.

\medskip\textbf{Mule}\index[Monster]{Mule}

\emph{Medium beast, misaligned}

\textbf{STRENGTH} +2

\textbf{DEXTERITY} +0

\textbf{CONSTITUTION} +1

\textbf{INTELLIGENCE} -4

\textbf{WISDOM} +0

\textbf{CHARISMA} -3

\textbf{Initiative} +0 -- \textbf{Defense} 11

\textbf{Hit Points} 11 (2d8 + 2)

\textbf{Movement} 12 m

\textbf{Saving Throws}: Fortitude +3, Reflexes +1, Will +1

\textbf{Languages} -

\textbf{Challenge} 1/8 (25 PX)

\emph{\textbf{Beast of Burden.}} The mule is considered a Large animal for the purposes of determining its carrying capacity.

\emph{\textbf{Steady Feet.}} The mule has +1d6 on Fortitude and Reflex saving throws made against effects that would cause it to fall prone.

\textbf{Shares}

\emph{\textbf{Hooves.} Melee Weapon Attack}: +2 to hit, reach 1 m, one target.

\emph{Hits:} 4 (1d4 + 2) bludgeoning damage.

\medskip\textbf{Brown Bear}\index[Monster]{Brown Bear}

\emph{Big beast, misaligned}

\textbf{STRENGTH} +4

\textbf{DEXTERITY} +0

\textbf{CONSTITUTION} +3

\textbf{INTELLIGENCE} -4

\textbf{WISDOM} +1

\textbf{CHARISMA} -2

\textbf{Initiative} +0 -- \textbf{Defense} 12

\textbf{Hit Points} 34 (4d10 + 12)

\textbf{Movement} 12m, climb 9m

\textbf{Saving Throws}: Fortitude +6, Reflexes +2, Will +3

\textbf{Skills} Awareness +3

\textbf{Languages} -

\textbf{Challenge} 1 (200 PX)

\emph{\textbf{Kind Smell.}} The bear has +1d6 on Wisdom (Awareness) checks that rely on smell.

\textbf{Shares}

\emph{\textbf{Multiattack.}} The bear makes two attacks: one with its bite and one with its claws.

\emph{\textbf{Claws.} Melee Weapon Attack}: +5 to hit, reach 1 m, one target.

\emph{Hits:} 11 (2d6 + 4) slashing damage.

\emph{\textbf{Bite.} Melee Weapon Attack}: +5 to hit, reach 1 m, one target.

\emph{Hits:} 8 (1d8 + 4) piercing damage.

\medskip\textbf{Black Bear}\index[Monster]{Black Bear}

\emph{Medium beast, misaligned}

\textbf{STRENGTH} +2

\textbf{DEXTERITY} +0

\textbf{CONSTITUTION} +2

\textbf{INTELLIGENCE} -4

\textbf{WISDOM} +1

\textbf{CHARISMA} -2

\textbf{Initiative} +0 -- \textbf{Defense} 12

\textbf{Hit Points} 19 (3d8 + 6)

\textbf{Movement} 12m, climb 9m

\textbf{Saving Throws}: Fortitude +4, Reflexes +1, Will +1

\textbf{Skills} Awareness +3

\textbf{Languages} -

\textbf{Challenge} 1/2 (100 PX)

\emph{\textbf{Kind Smell.}} The bear has +1d6 on Wisdom (Awareness) checks that rely on smell.

\textbf{Shares}

\emph{\textbf{Multiattack.}} The black bear makes two attacks: one with its bite and one with its claws.

\emph{\textbf{Claws.} Melee Weapon Attack}: +3 to hit, reach 1 m, one target.

\emph{Hits:} 7 (2d4 + 2) slashing damage, 1 bleed damage.

\emph{\textbf{Bite.} Melee Weapon Attack}: +3 to hit, reach 1 m, one target.

\emph{Hits:} 5 (1d6 + 2) piercing damage.

\medskip\textbf{Polar Bear}\index[Monster]{Polar Bear}

\emph{Great beast, misaligned}

\textbf{STRENGTH} +5

\textbf{DEXTERITY} +0

\textbf{CONSTITUTION} +3

\textbf{INTELLIGENCE} -4

\textbf{WISDOM} +1

\textbf{CHARISMA} -2

\textbf{Initiative} +0 -- \textbf{Defense} 13

\textbf{Hit Points} 42 (5d10 + 15)

\textbf{Movement} 12m, swim 9m

\textbf{Saving Throws}: Fortitude +10, Reflexes +7, Will +4

\textbf{Skills} Awareness +3

\textbf{Languages} -

\textbf{Challenge} 2 (450 PX)

\emph{\textbf{Kind Smell.}} The bear has +1d6 on Wisdom (Awareness) checks that rely on smell.

\textbf{Shares}

\emph{\textbf{Multiattack.}} The bear makes two attacks: one with its bite and one with its claws.

\emph{\textbf{Claws.} Melee Weapon Attack}: +7 to hit, reach 1 m, one target.

\emph{Hits:} 12 (2d6 + 5) slashing damage.

\emph{\textbf{Bite.} Melee Weapon Attack}: +7 to hit, reach 1 m, one target.

\emph{Hits:} 9 (1d8 + 5) piercing damage.

\textbf{VARIANT: CAVE BEAR}\index[Monster]{Cave Bear}

Some bears have adapted to life underground. They have the same statistics as polar bears, but with 18 m vision in the dark.

\medskip\textbf{Panther}\index[Monster]{Panther}

\emph{Medium beast, misaligned}

\textbf{STRENGTH} +2

\textbf{DEXTERITY} +2

\textbf{CONSTITUTION} +0

\textbf{INTELLIGENCE} -4

\textbf{WISDOM} +2

\textbf{CHARISMA} -2

\textbf{Initiative} +2 -- \textbf{Defense} 13

\textbf{Hit Points} 13 (3d8)

\textbf{Movement} 15 m, climb 12 m

\textbf{Saving Throws}: Fortitude +3, Reflexes +5, Will +3

\textbf{Skills} Stealth +6, Awareness +4

\textbf{Languages} -

\textbf{Challenge} 1/4 (50 PX)

\emph{\textbf{Leap.}} If the panther moves at least 20 feet towards a creature and hits it with a claw attack during the same round, the target must succeed on a DC 12 Fortitude saving throw or fall prone. If the target is prone, the panther can make a bite attack against it as a bonus action.

\emph{\textbf{Kind Sense of Smell.}} The panther has +1d6 on Wisdom (Awareness) checks that rely on smell.

\textbf{Shares}

\emph{\textbf{Claw.} Melee Weapon Attack}: +4 to hit, reach 1 m, one target.

\emph{Hits:} 4 (1d4 + 2) slashing damage, 1 bleed damage.

\emph{\textbf{Bite.} Melee Weapon Attack}: +4 to hit, reach 1 m, one target.

\emph{Hits:} 5 (1d6 + 2) piercing damage.


\medskip\textbf{Pony}\index[Monster]{Pony}

\emph{Medium beast, misaligned}

\textbf{STRENGTH} +2

\textbf{DEXTERITY} +0

\textbf{CONSTITUTION} +1

\textbf{INTELLIGENCE} -4

\textbf{WISDOM} +0

\textbf{CHARRISMA} -2

\textbf{Initiative} +0 -- \textbf{Defense} 11

\textbf{Hit Points} 11 (2d8 + 2)

\textbf{Movement} 12 m

\textbf{Saving Throws}: Fortitude +5, Reflexes +4, Will +0

\textbf{Languages} -

\textbf{Challenge} 1/8 (25 PX)

\textbf{Shares}

\emph{\textbf{Hooves.} Melee Weapon Attack}: +4 to hit, reach 1 m, one target.

\emph{Hits:} 7 (2d4 + 2) bludgeoning damage.

\medskip\textbf{Spider}\index[Monster]{Spider}

\emph{Tiny beast, misaligned}

\textbf{STRENGTH} 2 (-5)

\textbf{DEXTERITY} +2

\textbf{CONSTITUTION} -1

\textbf{INTELLIGENCE} -5

\textbf{WISDOM} +0

\textbf{CHARISMA} -4

\textbf{Initiative} +2 -- \textbf{Defense} 13

\textbf{Hit Points} 1 (1d4 - 1)

\textbf{Movement} 6m, climb 6m

\textbf{Saving Throws}: Fortitude -4, Reflexes +2, Will -4

\textbf{Skills} Stealth +4

\textbf{Senses} vision in the dark 9 m

\textbf{Languages} -

\textbf{Challenge} 0 (10 PX)

\emph{\textbf{Walking the Web.}} The spider ignores movement restrictions caused by the webs.

\emph{\textbf{Climb as Spider.}} The spider can climb difficult surfaces, including standing upside down on the ceiling, without needing to make an ability check.

\emph{\textbf{Web Sense.}} While in contact with a web, the spider knows the exact location of any other creature in contact with the same web.

\textbf{Shares}

\emph{\textbf{Bite.} Melee Weapon Attack}: +4 to hit, reach 1 ft., one creature.

\emph{Hits:} 1 piercing damage and the target must succeed on a Fortitude save 9 or take 2 (1d4) poison damage.

\medskip\textbf{Phase Spider}\index[Monstery]{Phase Spider}

The phase spider possesses the magical ability to enter and exit the Ethereal Plane. It seems to appear out of nowhere and quickly disappears after attacking.

\emph{Large monstrosity, misaligned}

\textbf{STRENGTH} +2

\textbf{DEXTERITY} +2

\textbf{CONSTITUTION} +1

\textbf{INTELLIGENCE} -2

\textbf{WISDOM} +0

\textbf{CHARISMA} -2

\textbf{Initiative} +2 -- \textbf{Defense} 15

\textbf{Hit Points} 32 (5d10 + 5)

\textbf{Movement} 9m, climb 9m

\textbf{Saving Throws}: Fortitude +8, Reflexes +8, Will +3

\textbf{Skills} Stealth +6

\textbf{Senses} vision in the dark 18 m

\textbf{Languages} -

\textbf{Challenge} 3 (700 PX)

\emph{\textbf{Walking on the Web.}} The spider ignores movement restrictions caused by the webs.

\emph{\textbf{Ethereal Step.}} As a bonus action, the spider can magically move from the Material Plane to the Ethereal Plane, or vice versa.

\emph{\textbf{Climb as Spider.}} The spider can climb difficult surfaces, including standing upside down on the ceiling, without needing to make an ability check.

\textbf{Shares}

\emph{\textbf{Bite.} Melee Weapon Attack}: +4 to hit, reach 1 ft., one creature.

\emph{Hits:} 7 (1d10 + 2) piercing damage and the target must make a DC 11 Fortitude saving throw, and take 18 (4d8) poison damage on a failed save, or half as much damage on a he succeeds. If poison damage reduces the target to 0 Hit Points, the target is stable but poisoned for 1 hour, even after regaining Hit Points, and remains paralyzed while poisoned in this way.

\medskip\textbf{Giant Spider}\index[Monster]{Giant Spider}

\emph{Great beast, misaligned}

\textbf{STRENGTH} +2

\textbf{DEXTERITY} +3

\textbf{CONSTITUTION} +1

\textbf{INTELLIGENCE} -4

\textbf{WISDOM} +0

\textbf{CHARISMA} -3

\textbf{Initiative} +2 -- \textbf{Defense} 15

\textbf{Hit Points} 26 (4d10 + 4)

\textbf{Movement} 9m, climb 9m

\textbf{Saving Throws}: Fortitude +4, Reflexes +4, Will +1

\textbf{Skills} Stealth +7

\textbf{Senses} blind sight 3 m, vision in the dark 18 m

\textbf{Languages} -

\textbf{Challenge} 1 (200 PX)

\emph{\textbf{Walking the Web.}} The spider ignores movement restrictions caused by the webs.

\emph{\textbf{Climb as Spider.}} The spider can climb difficult surfaces, including standing upside down on the ceiling, without needing to make an ability check.

\emph{\textbf{Web Sense.}} While in contact with a web, the spider knows the exact location of any other creature in contact with the same web.

\textbf{Shares}

\emph{\textbf{Bite.} Melee Weapon Attack}: +5 to hit, reach 1 ft., one creature.

\emph{Hits:} 7 (1d8 + 3) piercing damage and the target must make a Fortitude save DC 11, and suffer 9

(2d8) poison damage on a failed save, or half as much damage on a successful one. If poison damage reduces the target to 0 Hit Points, the target is stable but poisoned for 1 hour, even after regaining Hit Points, and remains paralyzed while poisoned in this way.

\emph{\textbf{Webweb (Recharge 5-6).} Ranged Weapon Attack}: +5 to hit, range 30ft, one creature.

\emph{Hits:} The target is entangled by the web. As an action, the entangled target can make a DC 12 Strength check and, on a success, break the web. The web can also be attacked and destroyed (AC 10; Hit Points 5; vulnerability to fire damage; immunity to bludgeoning and poison damage).

\medskip\textbf{Giant Wolf Spider}\index[Monster]{Giant Wolf Spider}

A giant wolf spider hunts prey on open ground or hides in burrows or crevices in the ground to set up ambushes.

\emph{Medium beast, misaligned}

\textbf{STRENGTH} +1

\textbf{DEXTERITY} +3

\textbf{CONSTITUTION} +1

\textbf{INTELLIGENCE} -4

\textbf{WISDOM} +1

\textbf{CHARISMA} -3

\textbf{Initiative} +3 -- \textbf{Defense} 14

\textbf{Hit Points} 11 (2d8 + 2)

\textbf{Movement} 12m, climb 12m

\textbf{Saving Throws}: Fortitude +2, Reflexes +4, Will +1

\textbf{Skills} Stealth +7, Awareness +3

\textbf{Senses} blind sight 3 m, vision in the dark 18 m

\textbf{Languages} -

\textbf{Challenge} 1/4 (50 PX)

\emph{\textbf{Walking the Web.}} The spider ignores movement restrictions caused by the webs.

\emph{\textbf{Climb as Spider.}} The spider can climb difficult surfaces, including standing upside down on the ceiling, without needing to make an ability check.

\emph{\textbf{Web Sense.}} While in contact with a web, the spider knows the exact location of any other creature in contact with the same web.

\textbf{Shares}

\emph{\textbf{Bite.} Melee Weapon Attack}: +3 to hit, reach 1 ft., one creature.

\emph{Hits:} 4 (1d6 + 1) piercing damage and the target must make a DC 11 Fortitude saving throw, and take 7 (2d6) poison damage on a failed save, or half as much damage on a he succeeds. If poison damage reduces the target to 0 Hit Points, the target is stable but poisoned for 1 hour, even after regaining Hit Points, and remains paralyzed while poisoned in this way.

\medskip\textbf{Frog}\index[Monster]{Frog}

\emph{Tiny beast, misaligned}

\textbf{STRENGTH} -5

\textbf{DEXTERITY} +1

\textbf{CONSTITUTION} -1

\textbf{INTELLIGENCE} -5

\textbf{WISDOM} -1

\textbf{CHARISMA} -4

\textbf{Initiative} +1 -- \textbf{Defense} 12

\textbf{Hit Points} 1 (1d4 - 1)

\textbf{Movement} 6 m, swim 6 m

\textbf{Saving Throws}: Fortitude -4, Reflexes +1, Will -2

\textbf{Skills} Stealth +3, Awareness +1

\textbf{Senses} vision in the dark 9 m

\textbf{Languages} -

\textbf{Challenge} 0 (0 PX)

\emph{\textbf{Amphibian.}} The frog can breathe air and water.

\emph{\textbf{Standing Jump.}} A frog can jump up to 3 meters long and up to 1 meter high, with or without a run-up.

A \textbf{frog} has no attachments. It feeds on small insects and usually lives near marshes, inside trees or underground.

\medskip\textbf{Giant Frog}\index[Monster]{Giant Frog}

\emph{Medium beast, misaligned}

\textbf{STRENGTH} +1

\textbf{DEXTERITY} +1

\textbf{CONSTITUTION} +0

\textbf{INTELLIGENCE} -4

\textbf{WISDOM} +0

\textbf{CHARISMA} -4

\textbf{Initiative} +1 -- \textbf{Defense} 12

\textbf{Hit Points} 18 (4d8)

\textbf{Movement} 9m, swim 9m

\textbf{Saving Throws}: Fortitude +2, Reflexes +2, Will +0

\textbf{Skills} Stealth +3, Awareness +2

\textbf{Senses} vision in the dark 9 m

\textbf{Languages} -

\textbf{Challenge} 1/4 (50 XP)

\emph{\textbf{Amphibian.}} The frog can breathe air and water.

\emph{\textbf{Standing Jump.}} A frog can jump up to 6 meters long and up to 3 meters high, with or without a run-up.

\textbf{Shares}

\emph{\textbf{Bite.} Melee Weapon Attack}: +3 to hit, reach 1 m, one target.

\emph{Hits:} 4 (1d6 + 1) piercing damage and the target is grappled (DC 11 to escape). Until the grapple ends, the target is entangled, and the frog cannot use its bite against another target.

\emph{\textbf{Swallow.}} The frog makes a bite attack against a Small or smaller target it is grabbing. If the attack hits, the target is engulfed, and the grapple ends. The swallowed target is blinded and restrained, has full cover against attacks and other effects outside the frog, and takes 5 (2d4) acid damage at the start of each of the frog's rounds. The frog can only swallow one target at a time. If the frog dies, an engulfed creature is no longer restrained by it and can exit the corpse using 3 feet of movement, exiting prone.

\medskip\textbf{Rat}\index[Monstery]{Rat}

\emph{Tiny beast, misaligned}

\textbf{STRENGTH} -4

\textbf{DEXTERITY} +0

\textbf{CONSTITUTION} -1

\textbf{INTELLIGENCE} -4

\textbf{WISDOM} +0

\textbf{CHARRISMA} -3

\textbf{Initiative} +0 -- \textbf{Defense} 11

\textbf{Hit Points} 1 (1d4 - 1)

\textbf{Movement} 6 m

\textbf{Saving Throws}: Fortitude -4, Reflexes +0, Will +0

\textbf{Senses} vision in the dark 9 m

\textbf{Languages} -

\textbf{Challenge} 0 (10 PX)

\emph{\textbf{Kind Smell.}} The rat has +1d6 on Wisdom (Awareness) checks that rely on smell.

\textbf{Shares}

\emph{\textbf{Bite.} Melee Weapon Attack}: +0 to hit, reach 1 m, one target.

\emph{Hits:} 1 piercing damage.

\medskip\textbf{Giant Rat}\index[Monster]{Giant Rat}

\emph{Small beast, misaligned}

\textbf{STRENGTH} -2

\textbf{DEXTERITY} +2

\textbf{CONSTITUTION} +0

\textbf{INTELLIGENCE} -4

\textbf{WISDOM} +0

\textbf{CHARRISMA} -3

\textbf{Initiative} +2 -- \textbf{Defense} 13

\textbf{Hit Points} 7 (2d6)

\textbf{Movement} 9 m

\textbf{Saving Throws}: Fortitude +3, Reflexes +5, Will +1

\textbf{Senses} vision in the dark 18 m

\textbf{Languages} -

\textbf{Challenge} 1/8 (25 PX)

\emph{\textbf{Kind Smell.}} The rat has +1d6 on Wisdom (Awareness) checks that rely on smell.

\emph{\textbf{Pack Tactics.}} The rat has +1d6 to attack rolls against a creature if at least one of the rat's allies is within 3 feet of the creature and that ally is not incapacitated.

\textbf{Shares}

\emph{\textbf{Bite.} Melee Weapon Attack}: +4 to hit, reach 1 m, one target.

\emph{Hits:} 4 (1d4 + 2) piercing damage.

\textbf{VARIANT: SICK GIANT RAT}\index[Monster]{Sick Giant Rat}

Some giant rats carry a terrible disease that they spread by biting. A diseased giant rat has challenge rating 1/8 (25 XP) and the following action instead of its normal bite attack.

\emph{\textbf{Bite.} Melee Weapon Attack}: +4 to hit, reach 1 m, one target.

\emph{Hits:} 4 (1d4 + 2) piercing damage. If the target is a creature, it must succeed on a DC 10 Fortitude saving throw or contract a disease. Until the disease is cured, the target cannot regain Hit Points except through magical methods, and the target's maximum Hit Points decrease by 3 (1d6) every 24 hours. If the target's maximum hit points drop to 0 as a result of the disease, the target dies.

\medskip\textbf{Rhinoceros}\index[Monster]{Rhinoceros}

\emph{Great beast, misaligned}

\textbf{STRENGTH} +5

\textbf{DEXTERITY} -1

\textbf{CONSTITUTION} +2

\textbf{INTELLIGENCE} -4

\textbf{WISDOM} +1

\textbf{CHARRISMA} -2

\textbf{Initiative} -1 -- \textbf{Defense} 12

\textbf{Hit Points} 45 (6d10 + 12)

\textbf{Movement} 12 m

\textbf{Saving Throws}: Fortitude +10, Reflexes +4, Will +2

\textbf{Languages} -

\textbf{Challenge} 2 (450 PX)

\emph{\textbf{Charge.}} If the rhino moves at least 20 feet towards a target and hits it with a gore attack during the same round, the target takes an additional 9 (2d8) bludgeoning damage. If the target is a creature, it must succeed on a DC 15 Fortitude save or fall prone.

\textbf{Shares}

\emph{\textbf{Gored.} Melee Weapon Attack}: +7 to hit, reach 1 m, one target.

\emph{Hits:} 14 (2d8 + 5) bludgeoning damage.

\medskip\textbf{Giant Toad}\index[Monster]{Giant Toad}

\emph{Great beast, misaligned}

\textbf{STRENGTH} +2

\textbf{DEXTERITY} +1

\textbf{CONSTITUTION} +1

\textbf{INTELLIGENCE} -4

\textbf{WISDOM} +0

\textbf{CHARISMA} -4

\textbf{Initiative} +1 -- \textbf{Defense} 12

\textbf{Hit Points} 39 (6d10 + 6)

\textbf{Movement} 6m, swim 12m

\textbf{Saving Throws}: Fortitude +6, Reflexes +6, Will +0

\textbf{Senses} vision in the dark 9 m

\textbf{Languages} -

\textbf{Challenge} 1 (200 PX)

\emph{\textbf{Amphibian.}} The toad can breathe air and water.

\emph{\textbf{Standing Jump.}} A toad can jump up to 20 feet long and up to 10 feet high, with or without a running start.

\textbf{Shares}

\emph{\textbf{Bite.} Melee Weapon Attack}: +4 to hit, reach 1 m, one target.

\emph{Hits:} 7 (1d10 + 2) piercing damage plus 5 (1d10) poison damage, and the target is grappled (DC 13 to escape). Until the grapple ends, the target is entangled, and the toad cannot use its bite against another target.

\emph{\textbf{Swallow.}} The toad makes a bite attack against a Medium or smaller target it is grabbing. If the attack hits, the target is engulfed, and the grapple ends. The swallowed target is blinded and restrained, has full cover against attacks and other effects outside the frog, and takes 10 (3d6) acid damage at the start of each round of the toad. The toad can only swallow one target at a time.

If the toad dies, a swallowed creature is no longer restrained by it and can exit the corpse using 3 feet of movement, exiting prone.

\medskip\textbf{Giant Fire Beetle}\index[Monstery]{Giant Fire Beetle}

A giant fire beetle is a nocturnal creature that possesses a pair of glowing glands capable of emitting light for 1d6 days after the beetle's death.

\emph{Small beast, misaligned}

\textbf{STRENGTH} -1

\textbf{DEXTERITY} +0

\textbf{CONSTITUTION} +1

\textbf{INTELLIGENCE} -5

\textbf{WISDOM} -2

\textbf{CHARRISMA} -4

\textbf{Initiative} +0 -- \textbf{Defense} 14

\textbf{Hit Points} 4 (1d6 + 1)

\textbf{Movement} 9 m

\textbf{Saving Throws}: Fortitude +2, Reflexes +0, Will +0

\textbf{Senses} blind sight 9 m

\textbf{Languages} -

\textbf{Challenge} 0 (10 PX)

\emph{\textbf{Lighting.}} The scarab radiates bright light in a 10-foot radius and dim light for an additional 10 feet.

\textbf{Shares}

\emph{\textbf{Bite.} Melee Weapon Attack}: +1 to hit, reach 1 m, one target.

\emph{Hits:} 2 (1d6 - 1) slashing damage.

\medskip\textbf{Jackal}\index[Monster]{Jackal}

\emph{Small beast, misaligned}

\textbf{STRENGTH} -1

\textbf{DEXTERITY} +2

\textbf{CONSTITUTION} +0

\textbf{INTELLIGENCE} -4

\textbf{WISDOM} +1

\textbf{CHARISMA} -2

\textbf{Initiative} +2 -- \textbf{Defense} 13

\textbf{Hit Points} 3 (1d6)

\textbf{Movement} 12 m

\textbf{Saving Throws}: Fortitude -1, Reflexes +3, Will +1

\textbf{Skills} Awareness +3

\textbf{Languages} -

\textbf{Challenge} 0 (10 PX)

\emph{\textbf{Pack Tactics.}} The jackal has +1d6 on attack rolls against a creature if at least one of the jackal's allies is within 3 feet of the creature and that ally is not incapacitated.

\emph{\textbf{Refined Hearing and Smell.}} The jackal has +1d6 on Wisdom (Awareness) checks that rely on hearing or smell.

\textbf{Shares}

\emph{\textbf{Bite.} Melee Weapon Attack}: +1 to hit, reach 1 m, one target.

\emph{Hits:} 1 (1d4 - 1) piercing damage.

\medskip\textbf{Swarms}\index[Monstery]{Swarms}

The swarms presented below are not normal or benign gatherings of small creatures. Instead, they form as a result of an external, often malignant, influence. Even druids are unable to charm these swarms, and their aggression is almost unnatural.

\textbf{Swarm of Centipedes}\index[Monstery]{Swarm of Centipedes}

\emph{Medium swarm of Tiny Beasts, misaligned}

\textbf{STRENGTH} -4

\textbf{DEXTERITY} +1

\textbf{CONSTITUTION} +0

\textbf{INTELLIGENCE} -5

\textbf{WISDOM} -2

\textbf{CHARRISMA} -5

\textbf{Initiative} +1 -- \textbf{Defense} 13

\textbf{Hit Points} 22 (5d8)

\textbf{Movement} 6m, climb 6m

\textbf{Saving Throws}: Fortitude -1, Reflexes +3, Will +1

\textbf{Damage Resistances} bludgeoning, piercing, cutting

\textbf{Condition Immunity} charmed, grabbed, entangled, paralyzed, petrified, prone, frightened, stunned

\textbf{Senses} blind sight 3 m

\textbf{Languages} -

\textbf{Challenge} 1/2 (100 PX)

\emph{\textbf{Swarm.}} The swarm can occupy another creature's space and vice versa, and the swarm can move through any opening large enough for a Tiny Insect. The swarm cannot regain Hit Points or gain temporary Hit Points.

\textbf{Shares}

\emph{\textbf{Bites.} Melee Weapon Attack}: +3 to hit, reach 0 m, one target in the swarm's space.

\emph{Hits:} 10 (4d4) piercing damage, or 5 (2d4) piercing damage if the swarm has half or less of its Hit Points. A creature reduced to 0 hit points by a centipede swarm and stable remains poisoned for 1 hour, even after regaining hit points, and remains paralyzed by the poison during this time.

\medskip\textbf{Swarm of Crows}\index[Monstery]{Swarm of Crows}

\emph{Medium swarm of Tiny Beasts, misaligned}

\textbf{STRENGTH} -2

\textbf{DEXTERITY} +2

\textbf{CONSTITUTION} -1

\textbf{INTELLIGENCE} -4

\textbf{WISDOM} +1

\textbf{CHARISMA} -2

\textbf{Initiative} +2 -- \textbf{Defense} 13

\textbf{Hit Points} 24 (7d8 -- 7)

\textbf{Movement} 3 m, flight 15 m

\textbf{Saving Throws}: Fortitude -1, Reflexes +3, Will +2

\textbf{Skills} Awareness +5

\textbf{Damage Resistances} bludgeoning, piercing, cutting

\textbf{Condition Immunity} charmed, grabbed, entangled, paralyzed, petrified, prone, frightened, stunned

\textbf{Languages} -

\textbf{Challenge} 1/4 (50 PX)

\emph{\textbf{Swarm.}} The swarm can occupy another creature's space and vice versa, and the swarm can move through any opening large enough for a Tiny Crow. The swarm cannot regain Hit Points or gain temporary Hit Points.

\textbf{Shares}

\emph{\textbf{Beaks.} Melee Weapon Attack}: +4 to hit, reach 1 m, one target in the swarm's space.

\emph{Hits:} 7 (2d6) piercing damage, or 3 (1d6) piercing damage if the swarm has half or less of its Hit Points.

\medskip\textbf{Pirana Swarm}\index[Monstery]{Pirana Swarm}

\emph{Medium swarm of Tiny Beasts, misaligned}

\textbf{STRENGTH} +1

\textbf{DEXTERITY} +3

\textbf{CONSTITUTION} -1

\textbf{INTELLIGENCE} -5

\textbf{WISDOM} -2

\textbf{CHARISMA} -4

\textbf{Initiative} +3 -- \textbf{Defense} 14

\textbf{Hit Points} 28 (8d8 -- 8)

\textbf{Movement} 0 m, swim 12 m

\textbf{Saving Throws}: Fortitude -3, Reflexes +4, Will -1

\textbf{Damage Resistances} bludgeoning, piercing, cutting

\textbf{Condition Immunity} charmed, grabbed, entangled, paralyzed, petrified, prone, frightened, stunned

\textbf{Senses} vision in the dark 18 m

\textbf{Languages} -

\textbf{Challenge} 1 (200 PX)

\emph{\textbf{Blood Frenzy.}} The swarm has +1d6 on melee attack rolls against any creature that is not at maximum Hit Points.

\emph{\textbf{Water Breathing.}} The swarm can only breathe underwater.

\emph{\textbf{Swarm.}} The swarm can occupy another creature's space and vice versa, and the swarm can move through any opening large enough for a Tiny Pirana. The swarm cannot regain Hit Points or gain temporary Hit Points.

\textbf{Shares}

\emph{\textbf{Bites.} Melee Weapon Attack}: +5 to hit, reach 0 m, one creature in the swarm's space.

\emph{Hits:} 14 (4d6) piercing damage, or 7 (2d6) piercing damage if the swarm has half or less of its Hit Points.

\medskip\textbf{Swarm of Insects}\index[Monstery]{Swarm of Insects}

\emph{Medium swarm of Tiny Beasts, misaligned}

\textbf{STRENGTH} -4

\textbf{DEXTERITY} +1

\textbf{CONSTITUTION} +0

\textbf{INTELLIGENCE} -5

\textbf{WISDOM} -2

\textbf{CHARRISMA} -5

\textbf{Initiative} +1 -- \textbf{Defense} 13

\textbf{Hit Points} 22 (5d8)

\textbf{Movement} 6m, climb 6m

\textbf{Saving Throws}: Fortitude -3, Reflexes +2, Will -1

\textbf{Damage Resistances} bludgeoning, piercing, cutting

\textbf{Condition Immunity} charmed, grabbed, entangled, paralyzed, petrified, prone, frightened, stunned

\textbf{Senses} blind sight 3 m

\textbf{Languages} -

\textbf{Challenge} 1/2 (100 PX)

\emph{\textbf{Swarm.}} The swarm can occupy another creature's space and vice versa, and the swarm can move through any opening large enough for a Tiny Insect. The swarm cannot regain Hit Points or gain temporary Hit Points.

\textbf{Shares}

\emph{\textbf{Bites.} Melee Weapon Attack}: +3 to hit, reach 0 m, one target in the swarm's space.

\emph{Hits:} 10 (4d4) piercing damage, or 5 (2d4) piercing damage if the swarm has half or less of its Hit Points.

\medskip\textbf{Swarm of Bats}\index[Monstery]{Swarm of Bats}

\emph{Medium swarm of Tiny Beasts, misaligned}

\textbf{STRENGTH} -3

\textbf{DEXTERITY} +2

\textbf{CONSTITUTION} +0

\textbf{INTELLIGENCE} -4

\textbf{WISDOM} +1

\textbf{CHARISMA} -3

\textbf{Initiative} +2 -- \textbf{Defense} 13

\textbf{Hit Points} 22 (5d8)

\textbf{Movement} 0 m, flight 9 m

\textbf{Saving Throws}: Fortitude -2, Reflexes +4, Will +2

\textbf{Damage Resistances} bludgeoning, piercing, cutting

\textbf{Condition Immunity} charmed, grabbed, entangled, paralyzed, petrified, prone, frightened, stunned

\textbf{Sensi} blind sight 18 m

\textbf{Languages} -

\textbf{Challenge} 1/4 (50 XP)

\emph{\textbf{Echolocation.}} The swarm cannot use blindsight if deafened.

\emph{\textbf{Swarm.}} The swarm can occupy another creature's space and vice versa, and the swarm can move through any opening large enough for a Tiny Bat. The swarm cannot regain Hit Points or gain temporary Hit Points.

\emph{\textbf{Refined Hearing.}} The swarm has +1d6 on Wisdom (Awareness) checks that rely on hearing.

\textbf{Shares}

\emph{\textbf{Bites.} Melee Weapon Attack}: +4 to hit, reach 0 m, one creature in the swarm's space.

\emph{Hits:} 5 (2d4) piercing damage, or 2 (1d4) piercing damage if the swarm has half or less of its Hit Points.

\medskip\textbf{Swarm of Spiders}\index[Monstery]{Swarm of Spiders}

\emph{Medium swarm of Tiny Beasts, misaligned}

\textbf{STRENGTH} -4

\textbf{DEXTERITY} +1

\textbf{CONSTITUTION} +0

\textbf{INTELLIGENCE} -5

\textbf{WISDOM} -2

\textbf{CHARRISMA} -5

\textbf{Initiative} +1 -- \textbf{Defense} 13

\textbf{Hit Points} 22 (5d8)

\textbf{Movement} 6m, climb 6m

\textbf{Saving Throws}: Fortitude -3, Reflexes +2, Will -1

\textbf{Damage Resistances} bludgeoning, piercing, cutting

\textbf{Condition Immunity} charmed, grabbed, entangled, paralyzed, petrified, prone, frightened, stunned

\textbf{Senses} blind sight 3 m

\textbf{Languages} -

\textbf{Challenge} 1/2 (100 PX)

\emph{\textbf{Walking the Web.}} The swarm ignores movement restrictions caused by the webs.

\emph{\textbf{Climb as a Spider.}} The swarm can climb difficult surfaces, including standing upside down on the ceiling, without needing to make an ability check.

\emph{\textbf{Web Sense.}} While in contact with a web, the swarm knows the exact location of any other creature in contact with the same web.

\emph{\textbf{Swarm.}} The swarm can occupy another creature's space and vice versa, and the swarm can move through any opening large enough for a Tiny Insect. The swarm cannot regain Hit Points or gain temporary Hit Points.

\textbf{Shares}

\emph{\textbf{Bites.} Melee Weapon Attack}: +3 to hit, reach 0 m, one target in the swarm's space.

\emph{Hits:} 10 (4d4) piercing damage, or 5 (2d4) piercing damage if the swarm has half or less of its Hit Points.

\medskip\textbf{Swarm of Rats}\index[Monstery]{Swarm of Rats}

\emph{Medium swarm of Tiny Beasts, misaligned}

\textbf{STRENGTH} -1

\textbf{DEXTERITY} +0

\textbf{CONSTITUTION} -1

\textbf{INTELLIGENCE} -4

\textbf{WISDOM} +0

\textbf{CHARRISMA} -4

\textbf{Initiative} +0 -- \textbf{Defense} 11

\textbf{Hit Points} 24 (7d8 - 7)

\textbf{Movement} 9 m

\textbf{Saving Throws}: Fortitude +0, Reflexes +1, Will +1

\textbf{Damage Resistances} bludgeoning, piercing, cutting

\textbf{Condition Immunity} charmed, grabbed, entangled, paralyzed, petrified, prone, frightened, stunned

\textbf{Senses} vision in the dark 9 m

\textbf{Languages} -

\textbf{Challenge} 1/4 (50 XP)

\emph{\textbf{Sense of Smell.}} The swarm has +1d6 on Wisdom (Awareness) checks that rely on smell.

\emph{\textbf{Swarm.}} The swarm can occupy another creature's space and vice versa, and the swarm can move through any opening large enough for a Tiny Rat. The swarm cannot regain Hit Points or gain temporary Hit Points.

\textbf{Shares}

\emph{\textbf{Bites.} Melee Weapon Attack}: +2 to hit, reach 0 m, one target in the swarm's space.

\emph{Hits:} 7 (2d6) piercing damage, or 3 (1d6) piercing damage if the swarm has half or less of its Hit Points.

\medskip\textbf{Swarm of Scarabs}\index{Swarm of Scarabs}

\emph{Medium swarm of Tiny Beasts, misaligned}

\textbf{STRENGTH} -4

\textbf{DEXTERITY} +1

\textbf{CONSTITUTION} +0

\textbf{INTELLIGENCE} -5

\textbf{WISDOM} -2

\textbf{CHARISMA} -5

\textbf{Initiative} +1 -- \textbf{Defense} 13

\textbf{Hit Points} 22 (5d8)

\textbf{Move} 6m, climb 6m, dig 6m

\textbf{Saving Throws}: Fortitude -3, Reflexes +2, Will -1

\textbf{Damage Resistances} bludgeoning, piercing, cutting

\textbf{Condition Immunity} charmed, grabbed, entangled, paralyzed, petrified, prone, frightened, stunned

\textbf{Senses} blind sight 3 m

\textbf{Languages} -

\textbf{Challenge} 1/2 (100 PX)

\emph{\textbf{Swarm.}} The swarm can occupy another creature's space and vice versa, and the swarm can move through any opening large enough for a Tiny Insect. The swarm cannot regain Hit Points or gain temporary Hit Points.

\textbf{Shares}

\emph{\textbf{Bites.} Melee Weapon Attack}: +3 to hit, reach 0 m, one target in the swarm's space.

\emph{Hits:} 10 (4d4) piercing damage, or 5 (2d4) piercing damage if the swarm has half or less of its Hit Points.

\medskip\textbf{Swarm of Poisonous Snakes}\index{Swarm of Poisonous Snakes}

\emph{Medium swarm of Tiny Beasts, misaligned}

\textbf{STRENGTH} -1

\textbf{DEXTERITY} +4

\textbf{CONSTITUTION} +0

\textbf{INTELLIGENCE} -5

\textbf{WISDOM} +0

\textbf{CHARISMA} -4

\textbf{Initiative} +4 -- \textbf{Defense} 15

\textbf{Hit Points} 36 (8d8)

\textbf{Movement} 9m, swim 9m

\textbf{Saving Throws}: Fortitude +0, Reflexes +5, Will +1

\textbf{Damage Resistances} bludgeoning, piercing, cutting

\textbf{Condition Immunity} charmed, grabbed, entangled, paralyzed, petrified, prone, frightened, stunned

\textbf{Senses} blind sight 3 m

\textbf{Languages} -

\textbf{Challenge} 2 (450 PX)

\emph{\textbf{Swarm.}} The swarm can occupy another creature's space and vice versa, and the swarm can move through any opening large enough for a Tiny Snake. The swarm cannot regain Hit Points or gain temporary Hit Points.

\textbf{Shares}

\emph{\textbf{Bites.} Melee Weapon Attack}: +6 to hit, reach 0 m, one creature in the swarm's space.

\emph{Hits:} 7 (2d6) piercing damage, or 3 (1d6) piercing damage if the swarm has half or less of its hit points, and the target must make a Fortitude saving throw DC 10, and suffer 14 (4d6) poison damage on a failed save, or half as much damage on a successful one.

\medskip\textbf{Swarm of Wasps}\index{Swarm of Poisonous Snakes}

\emph{Medium swarm of Tiny Beasts, misaligned}

\textbf{STRENGTH} -4

\textbf{DEXTERITY} +1

\textbf{CONSTITUTION} +0

\textbf{INTELLIGENCE} -5

\textbf{WISDOM} -2

\textbf{CHARISMA} -5

\textbf{Initiative} +1 -- \textbf{Defense} 13

\textbf{Hit Points} 22 (5d8)

\textbf{Movement} 1 m, flight 9 m

\textbf{Saving Throws}: Fortitude -3, Reflexes +2, Will -1

\textbf{Damage Resistances} bludgeoning, piercing, cutting

\textbf{Condition Immunity} charmed, grabbed, entangled, paralyzed, petrified, prone, frightened, stunned

\textbf{Senses} blind sight 3 m

\textbf{Languages} -

\textbf{Challenge} 1/2 (100 PX)

\emph{\textbf{Swarm.}} The swarm can occupy another creature's space and vice versa, and the swarm can move through any opening large enough for a Tiny Insect. The swarm cannot regain Hit Points or gain temporary Hit Points.

\textbf{Shares}

\emph{\textbf{Bites.} Melee Weapon Attack}: +3 to hit, reach 0 m, one target in the swarm's space.

\emph{Hits:} 10 (4d4) piercing damage, or 5 (2d4) piercing damage if the swarm has half or less of its Hit Points.


\medskip\textbf{Monkey}\index[Monster]{Monkey}

\emph{Small beast, misaligned}

\textbf{STRENGTH} -3

\textbf{DEXTERITY} +2

\textbf{CONSTITUTION} +0

\textbf{INTELLIGENCE} -3

\textbf{WISDOM} +1

\textbf{CHARRISMA} -2

\textbf{Initiative} +2 -- \textbf{Defense} 13

\textbf{Hit Points} 3 (1d6)

\textbf{Movement} 9m, climb 9m

\textbf{Saving Throws}: Fortitude +0, Reflexes +3, Will +1

\textbf{Skills} Acrobatics +6, Awareness +3

\textbf{Languages} -

\textbf{Challenge} 1/4 (10 PX)

\textbf{Shares}

\emph{\textbf{Scratch.} Melee weapon attack}: +1 to hit, reach 1 m, one target.

\emph{Hits:} 1 (1d4 - 1) slashing damage.

\emph{\textbf{Bite.} Melee Weapon Attack}: +1 to hit, reach 1 meter, one target.

\emph{Hits:} 2 (1d4) piercing damage.

\medskip\textbf{Monkey}\index[Monster]{Monkey}

\emph{Medium beast, misaligned}

\textbf{STRENGTH} +3

\textbf{DEXTERITY} +2

\textbf{CONSTITUTION} +2

\textbf{INTELLIGENCE} -2

\textbf{WISDOM} +1

\textbf{CHARRISMA} -2

\textbf{Initiative} +2 -- \textbf{Defense} 13

\textbf{Hit Points} 19 (3d8 + 6)

\textbf{Movement} 9m, climb 9m

\textbf{Saving Throws}: Fortitude +3, Reflexes +3, Will +2

\textbf{Skills} Acrobatics +5, Awareness +3

\textbf{Languages} -

\textbf{Challenge} 1/2 (100 PX)

\textbf{Shares}

\emph{\textbf{Multiattack.}} The ape makes two punch attacks.

\emph{\textbf{Fist.} Melee Weapon Attack}: +5 to hit, reach 1 m, one target.

\emph{Hits:} 6 (1d6 + 3) bludgeoning damage.

\emph{\textbf{Rock.} Ranged Weapon Attack}: +5 to hit, range 8m, one target.

\emph{Hits:} 6 (1d6 + 3) bludgeoning damage.

\medskip\textbf{Giant Ape}\index[Monster]{Giant Ape}

\emph{Huge beast, misaligned}

\textbf{STRENGTH} +6

\textbf{DEXTERITY} +2

\textbf{CONSTITUTION} +4

\textbf{INTELLIGENCE} -2

\textbf{WISDOM} +1

\textbf{CHARRISMA} -2

\textbf{Initiative} +2 -- \textbf{Defense} 16

\textbf{Hit Points} 157 (15d12 + 60)

\textbf{Movement} 12m, climb 12m

\textbf{Saving Throws}: Fortitude +7, Reflexes +6, Will +4

\textbf{Skills} Acrobatics +9, Awareness +4

\textbf{Languages} -

\textbf{Challenge} 7 (2900 XP)

\textbf{Shares}

\emph{\textbf{Multiattack.}} The ape makes two punch attacks.

\emph{\textbf{Fist.} Melee Weapon Attack}: +9 to hit, reach 10 ft., one target.

\emph{Hits:} 22 (3d10 + 6) bludgeoning damage.

\emph{\textbf{Rock.} Ranged Weapon Attack}: +9 to hit, range 15m, one target.

\emph{Hits:} 30 (7d6 + 6) bludgeoning damage.

\medskip\textbf{Scorpion}\index[Monstery]{Scorpion}

\emph{Tiny beast, misaligned}

\textbf{STRENGTH} -4

\textbf{DEXTERITY} +0

\textbf{CONSTITUTION} -1

\textbf{INTELLIGENCE} -5

\textbf{WISDOM} -1

\textbf{CHARRISMA} -4

\textbf{Initiative} +0 -- \textbf{Defense} 12

\textbf{Hit Points} 1 (1d4 - 1)

\textbf{Movement} 3 m

\textbf{Saving Throws}: Fortitude -3, Reflexes +2, Will -1

\textbf{Senses} blind sight 3 m

\textbf{Languages} -

\textbf{Challenge} 0 (10 PX)

\textbf{Shares}

\emph{\textbf{Sting.} Melee Weapon Attack}: +2 to hit, reach 1 ft., one creature.

\emph{Hits:} 1 piercing damage and the target must make a DC 9 Fortitude saving throw, and take 4 (1d8) poison damage on a failed save, or half as much damage on a successful one.

\medskip\textbf{Giant Scorpion}\index[Monster]{Giant Scorpion}

\emph{Big beast, misaligned}

\textbf{STRENGTH} +2

\textbf{DEXTERITY} +1

\textbf{CONSTITUTION} +2

\textbf{INTELLIGENCE} -5

\textbf{WISDOM} -1

\textbf{CHARISMA} -4

\textbf{Initiative} +1 -- \textbf{Defense} 17

\textbf{Hit Points} 52 (7d10 + 14)

\textbf{Movement} 12 m

\textbf{Saving Throws}: Fortitude +7, Reflexes +1, Will +1

\textbf{Sensi} blind sight 18 m

\textbf{Languages} -

\textbf{Challenge} 3 (700 PX)

\textbf{Shares}

\emph{\textbf{Multiattack.}} The scorpion makes three attacks: two with its claws and one with its stinger.

\emph{\textbf{Claw.} Melee Weapon Attack}: +4 to hit, reach 1 m, one target.

\emph{Hits:} 6 (1d8 + 2) bludgeoning damage and the target is grappled (DC 12 to escape). The scorpion has two claws, each of which can grasp only one target.

\emph{\textbf{Sting.} Melee Weapon Attack}: +4 to hit, reach 1 ft., one creature.

\emph{Hits:} 7 (1d10 + 2) piercing damage, and the target must make a DC 12 Fortitude saving throw, and take 22 (4d10) poison damage on a failed save, or half as much damage on a he succeeds.

\medskip\textbf{Constrictor Serpent}\index[Monster]{Constrictor Serpent}

\emph{Great beast, misaligned}

\textbf{STRENGTH} +2

\textbf{DEXTERITY} +2

\textbf{CONSTITUTION} +1

\textbf{INTELLIGENCE} -5

\textbf{WISDOM} +0

\textbf{CHARISMA} -4

\textbf{Initiative} +2 -- \textbf{Defense} 13

\textbf{Hit Points} 13 (2d10 + 2)

\textbf{Movement} 9m, swim 9m

\textbf{Saving Throws}: Fortitude +3, Reflexes +2, Will +0

\textbf{Senses} blind sight 3 m

\textbf{Languages} -

\textbf{Challenge} 1/4 (50 XP)

\textbf{Shares}

\emph{\textbf{Bite.} Melee Weapon Attack}: +4 to hit, reach 3 ft., one creature.

\emph{Hits:} 5 (1d6 + 2) piercing damage.

\emph{\textbf{Crush.} Melee Weapon Attack}: +4 to hit, reach 3 ft., one creature.

\emph{Hit:} 6 (1d8 + 2) bludgeoning damage, and the target is grappled (DC 14 to escape). Until the grapple ends, the creature is entangled, and the serpent cannot constrict another target.

\medskip\textbf{Giant Constrictor Snake}\index[Monster]{Giant Constrictor Snake}

\emph{Huge beast, misaligned}

\textbf{STRENGTH} +4

\textbf{DEXTERITY} +2

\textbf{CONSTITUTION} +1

\textbf{INTELLIGENCE} -5

\textbf{WISDOM} +0

\textbf{CHARRISMA} -4

\textbf{Initiative} +2 -- \textbf{Defense} 13

\textbf{Hit Points} 60 (8d12 + 8)

\textbf{Movement} 9m, swim 9m

\textbf{Saving Throws}: Fortitude +3, Reflexes +2, Will +0

\textbf{Skills} Awareness +2

\textbf{Senses} blind sight 3 m

\textbf{Languages} -

\textbf{Challenge} 2 (450 PX)

\textbf{Shares}

\emph{\textbf{Bite.} Melee Weapon Attack}: +6 to hit, reach 10 ft., one creature.

\emph{Hits:} 11 (2d6 + 4) piercing damage.

\emph{\textbf{Crush.} Melee Weapon Attack}: +6 to hit, reach 3 ft., one creature.

\emph{Hits:} 13 (2d8 + 4) bludgeoning damage, and the target is grappled (DC 16 to escape). Until the grapple ends, the creature is entangled, and the serpent cannot constrict another target.

\medskip\textbf{Poisonous Snake}\index[Monster]{Poisonous Snake}

\emph{Tiny beast, misaligned}

\textbf{STRENGTH} -4

\textbf{DEXTERITY} +3

\textbf{CONSTITUTION} +0

\textbf{INTELLIGENCE} -5

\textbf{WISDOM} +0

\textbf{CHARRISMA} -4

\textbf{Initiative} +3 -- \textbf{Defense} 14

\textbf{Hit Points} 2 (1d4)

\textbf{Movement} 9m, swim 9m

\textbf{Saving Throws}: Fortitude +1, Reflexes +4, Will +1

\textbf{Senses} blind sight 3 m

\textbf{Languages} -

\textbf{Challenge} 1/8 (25 PX)

\textbf{Shares}

\emph{\textbf{Bite.} Melee Weapon Attack}: +5 to hit, reach 1 m, one target.

\emph{Hits:} 1 piercing damage and the target must make a Fortitude save DC 10, and take 5 (2d4) poison damage on a failed save, or half as much damage on a successful one.

\medskip\textbf{Giant Poisonous Snake}\index[Monster]{Giant Poisonous Snake}

\emph{Medium beast, misaligned}

\textbf{STRENGTH} +0

\textbf{DEXTERITY} +4

\textbf{CONSTITUTION} +1

\textbf{INTELLIGENCE} -4

\textbf{WISDOM} +0

\textbf{CHARISMA} -4

\textbf{Initiative} +4 -- \textbf{Defense} 15

\textbf{Hit Points} 11 (2d8 + 2)

\textbf{Movement} 9m, swim 9m

\textbf{Saving Throws}: Fortitude +1, Reflexes +5, Will +2

\textbf{Skills} Awareness +2

\textbf{Senses} blind sight 3 m

\textbf{Languages} -

\textbf{Challenge} 1/4 (50 XP)

\textbf{Shares}

\emph{\textbf{Bite.} Melee Weapon Attack}: +6 to hit, reach 10 ft., one target.

\emph{Hit:} 6 (1d4 + 4) piercing damage and the target must make a DC 11 Fortitude saving throw, and take 10 (3d6) poison damage on a failed save, or half as much damage on a he succeeds.

\medskip\textbf{Flying Serpent}\index[Monster]{Flying Serpent}

A flying serpent is a brightly colored, winged serpent found in remote jungles.

\emph{Tiny beast, misaligned}

\textbf{STRENGTH} -3

\textbf{DEXTERITY} +4

\textbf{CONSTITUTION} +0

\textbf{INTELLIGENCE} -4

\textbf{WISDOM} +1

\textbf{CHARRISMA} -3

\textbf{Initiative} +4 -- \textbf{Defense} 15

\textbf{Hit Points} 5 (2d4)

\textbf{Move} 9 m, swim 9 m, fly 18 m

\textbf{Saving Throws}: Fortitude -2, Reflexes +5, Will +1

\textbf{Senses} blind sight 3 m

\textbf{Languages} -

\textbf{Challenge} 1/8 (25 PX)

\emph{\textbf{Flying.}} The snake does not provoke attacks of opportunity when it flies out of an enemy's reach.

\textbf{Shares}

\emph{\textbf{Bite.} Melee Weapon Attack}: +6 to hit, reach 1 m, one target.

\emph{Hits:} 1 piercing damage plus 7 (3d4) poison damage.

\medskip\textbf{Hunter Shark}\index[Monstery]{Hunter Shark}

A hunter shark is 4 to 6 meters long and usually hunts solitarily in deeper waters.

\emph{Great beast, misaligned}

\textbf{STRENGTH} +4

\textbf{DEXTERITY} +1

\textbf{CONSTITUTION} +2

\textbf{INTELLIGENCE} -5

\textbf{WISDOM} +0

\textbf{CHARRISMA} -3

\textbf{Initiative} +1 -- \textbf{Defense} 13

\textbf{Hit Points} 45 (6d10 + 12)

\textbf{Movement} 0 m, swim 12 m

\textbf{Saving Throws}: Fortitude +4, Reflexes +2, Will +0

\textbf{Skills} Awareness +2

\textbf{Senses} blind sight 9 m

\textbf{Languages} -

\textbf{Challenge} 2 (450 PX)

\emph{\textbf{Blood Frenzy.}} The shark has +1d6 on melee attack rolls against any creature that is not at maximum hit points.

\emph{\textbf{Breathe Water.}} The shark can only breathe underwater.

\textbf{Shares}

\emph{\textbf{Bite.} Melee Weapon Attack}: +6 to hit, reach 1 m, one target.

\emph{Hits:} 13 (2d8 + 4) piercing damage.

\medskip\textbf{Coral Shark}\index[Monster]{Coral Shark}

Coral sharks are 2 to 3 meters long and live in shallower waters and along coral reefs.

\emph{Medium beast, misaligned}

\textbf{STRENGTH} +2

\textbf{DEXTERITY} +1

\textbf{CONSTITUTION} +1

\textbf{INTELLIGENCE} -5

\textbf{WISDOM} +0

\textbf{CHARISMA} -3

\textbf{Initiative} +1 -- \textbf{Defense} 13

\textbf{Hit Points} 22 (4d8 + 4)

\textbf{Movement} 0 m, swim 12 m

\textbf{Saving Throws}: Fortitude +2, Reflexes +2, Will +1

\textbf{Skills} Awareness +2

\textbf{Senses} blind sight 9 m

\textbf{Languages} -

\textbf{Challenge} 1/2 (100 PX)

\emph{\textbf{Breathe Water.}} The shark can only breathe underwater.

\emph{\textbf{Shoal Tactics.}} The shark has +1d6 to attack rolls against a creature if at least one of the shark's allies is within 3 feet of the creature and that ally is not incapacitated.

\textbf{Shares}

\emph{\textbf{Bite.} Melee Weapon Attack}: +4 to hit, reach 1 m, one target.

\emph{Hits:} 6 (1d8 + 2) piercing damage.

\medskip\textbf{Giant Shark}\index[Monster]{Giant Shark}

The giant shark is 9 meters long and yes meet

normally only in the deeper oceans.

\emph{Huge beast, misaligned}

\textbf{STRENGTH} +6

\textbf{DEXTERITY} +0

\textbf{CONSTITUTION} +5

\textbf{INTELLIGENCE} -5

\textbf{WISDOM} +0

\textbf{CHARISMA} -3

\textbf{Initiative} +0 -- \textbf{Defense} 16

\textbf{Hit Points} 126 (11d12 + 55)

\textbf{Movement} 0 m, swim 15 m

\textbf{Saving Throws}: Fortitude +7, Reflexes +2, Will +1

\textbf{Skills} Awareness +3

\textbf{Sensi} blind sight 18 m

\textbf{Languages} -

\textbf{Challenge} 5 (1800 PX)

\emph{\textbf{Blood Frenzy.}} The shark has +1d6 to attack rolls

in melee against any creature that isn't at full hit points.

\emph{\textbf{Breathing Water.}} The shark can only breathe underwater.

\textbf{Shares}

\emph{\textbf{Bite.} Melee Weapon Attack}: +9 to hit, reach 1 m, one target.

\emph{Hits:} 22 (3d10 + 6) piercing damage.

\medskip\textbf{Striga}\index[Monster]{Striga}

This hideous monster looks like a cross between a large bat and an oversized mosquito. Its legs end in long pincers, and its long, needle-like proboscis slices through the air as it seeks to feed on the blood of living creatures.

\emph{Tiny beast, misaligned}

\textbf{STRENGTH} -3

\textbf{DEXTERITY} +3

\textbf{CONSTITUTION} +0

\textbf{INTELLIGENCE} -4

\textbf{WISDOM} -1

\textbf{CHARRISMA} -2

\textbf{Initiative} +3 -- \textbf{Defense} 15

\textbf{Hit Points} 2 (1d4)

\textbf{Movement} 3 m, flight 12 m

\textbf{Saving Throws}: Fortitude -3, Reflexes +4, Will -1

\textbf{Senses} vision in the dark 18 m

\textbf{Languages} -

\textbf{Challenge} 1/8 (25 PX)

\textbf{Shares}

\emph{\textbf{Blood Drain.} Melee Weapon Attack}: +5 to hit, reach 3 ft., one creature.

\emph{Hits:} 5 (1d4 + 3) piercing damage and the striga attaches to the target. While he is attacked, the striga does not attack. Instead, at the start of each striga round, the target loses 5 (1d4 + 3) hit points due to blood loss.

The striga can detach itself by spending 1 meter of movement. He does this automatically after draining 10 Hit Points from the target or upon the target's death. A creature, including the target, can use its action to detach the striga.

\medskip\textbf{Rate}\index[Monster]{Rate}

\emph{Tiny beast, misaligned}

\textbf{STRENGTH} -3

\textbf{DEXTERITY} +0

\textbf{CONSTITUTION} +1

\textbf{INTELLIGENCE} -4

\textbf{WISDOM} +1

\textbf{CHARRISMA} -3

\textbf{Initiative} +0 -- \textbf{Defense} 11

\textbf{Hit Points} 3 (1d4 + 1)

\textbf{Movement} 6 m, excavation 1 m

\textbf{Saving Throws}: Fortitude -3, Reflexes +1, Will +1

\textbf{Senses} vision in the dark 9 m

\textbf{Languages} -

\textbf{Challenge} 0 (10 PX)

\emph{\textbf{Kind Smell.}} The badger has +1d6 on Wisdom (Awareness) checks that rely on smell.

\textbf{Shares}

\emph{\textbf{Bite.} Melee Weapon Attack}: +2 to hit, reach 1 m, one target.

\emph{Hits:} 1 piercing damage.

\medskip\textbf{Giant Badger}\index[Monster]{Giant Badger}

\emph{Medium beast, misaligned}

\textbf{STRENGTH} +1

\textbf{DEXTERITY} +0

\textbf{CONSTITUTION} +2

\textbf{INTELLIGENCE} -4

\textbf{WISDOM} +1

\textbf{CHARISMA} -3

\textbf{Initiative} +0 -- \textbf{Defense} 11

\textbf{Hit Points} 13 (2d8 + 4)

\textbf{Movement} 9 m, excavation 3 m

\textbf{Saving Throws}: Fortitude +2, Reflexes +1, Will +2

\textbf{Senses} vision in the dark 9 m

\textbf{Languages} -

\textbf{Challenge} 1/4 (50 PX)

\emph{\textbf{Kind Smell.}} The badger has +1d6 on Wisdom (Awareness) checks that rely on smell.

\textbf{Shares}

\emph{\textbf{Multiattack.}} The badger makes two attacks: one with its bite and one with its claws.

\emph{\textbf{Claws.} Melee Weapon Attack}: +3 to hit, reach 1 m, one target.

\emph{Hits:} 6 (2d4 + 1) slashing damage.

\emph{\textbf{Bite.} Melee Weapon Attack}: +3 to hit, reach 1 m, one target.

\emph{Hits:} 4 (1d6 + 1) piercing damage.

\medskip\textbf{Tiger}\index[Monster]{Tiger}

\emph{Big beast, misaligned}

\textbf{STRENGTH} +3

\textbf{DEXTERITY} +2

\textbf{CONSTITUTION} +2

\textbf{INTELLIGENCE} -4

\textbf{WISDOM} +1

\textbf{CHARISMA} -1

\textbf{Initiative} +2 -- \textbf{Defense} 13

\textbf{Hit Points} 37 (5d10 + 10)

\textbf{Movement} 12 m

\textbf{Saving Throws}: Fortitude +4, Reflexes +4, Will +2

\textbf{Skills} Stealth +6, Awareness +3

\textbf{Senses} vision in the dark 18 m

\textbf{Languages} -

\textbf{Challenge} 1 (200 PX)

\emph{\textbf{Leap.}} If the tiger moves at least 20 feet towards a creature and hits it with a claw attack during the same round, the target must succeed on a Fortitude saving throw DC 13 or fall prone. If the target is prone, the tiger can make a bite attack against it as a bonus action.

\emph{\textbf{Sense of Smell.}} The tiger has +1d6 on Wisdom (Awareness) checks that rely on smell.

\textbf{Shares}

\emph{\textbf{Claw.} Melee Weapon Attack}: +5 to hit, reach 1 m, one target.

\emph{Hits:} 7 (1d8 + 3) slashing damage, 1 bleed damage.

\emph{\textbf{Bite.} Melee Weapon Attack}: +5 to hit, reach 1 m, one target.

\emph{Hits:} 8 (1d10 + 3) piercing damage.

\medskip\textbf{Saber-Toothed Tiger}\index[Monstery]{Saber-Toothed Tiger}

\emph{Great beast, misaligned}

\textbf{STRENGTH} +4

\textbf{DEXTERITY} +2

\textbf{CONSTITUTION} +2

\textbf{INTELLIGENCE} -4

\textbf{WISDOM} +1

\textbf{CHARISMA} -1

\textbf{Initiative} +2 -- \textbf{Defense} 13

\textbf{Hit Points} 52 (7d10 + 14)

\textbf{Movement} 12 m

\textbf{Saving Throws}: Fortitude +5, Reflexes +3, Will +2

\textbf{Skills} Stealth +6, Awareness +3

\textbf{Languages} -

\textbf{Challenge} 2 (450 PX)

\emph{\textbf{Leap.}} If the tiger moves at least 20 feet towards a creature and hits it with a claw attack during the same round, the target must succeed on a DC 14 Fortitude saving throw or fall prone. If the target is prone, the tiger can make a bite attack against it as a bonus action.

\emph{\textbf{Sense of Smell.}} The tiger has +1d6 on Wisdom (Awareness) checks that rely on smell.

\textbf{Shares}

\emph{\textbf{Claw.} Melee Weapon Attack}: +6 to hit, reach 1 m, one target.

\emph{Hits:} 12 (2d6 + 5) slashing damage, 1 bleed damage.

\emph{\textbf{Bite.} Melee Weapon Attack}: +6 to hit, reach 1 m, one target.

\emph{Hits:} 10 (1d10 + 5) piercing damage.

\medskip\textbf{Giant Vespa}\index[Monstruario]{Giant Vespa}

\emph{Medium beast, misaligned}

\textbf{STRENGTH} +0

\textbf{DEXTERITY} +2

\textbf{CONSTITUTION} +0

\textbf{INTELLIGENCE} -5

\textbf{WISDOM} +0

\textbf{CHARISMA} -4

\textbf{Initiative} +2 -- \textbf{Defense} 13

\textbf{Hit Points} 13 (3d8)

\textbf{Movement} 3 m, flight 15 m

\textbf{Saving Throws}: Fortitude +1, Reflexes +3, Will +0

\textbf{Languages} -

\textbf{Challenge} 1/2 (100 PX)

\textbf{Shares}

\emph{\textbf{Sting.} Melee Weapon Attack}: +4 to hit, reach 1 ft., one creature.

\emph{Hits:} 5 (1d6 + 2) piercing damage, and the target must make a DC 11 Fortitude saving throw, and take 10 (3d6) poison damage on a failed save, or half as much damage if he succeeds. If poison damage reduces the target to 0 Hit Points, the target is stable but poisoned for 1 hour, even after regaining Hit Points, and remains paralyzed while poisoned in this way.

\medskip\textbf{Worg}\index[Monstery]{Worg}

Worgs are monstrous, wolf-like predators that love to hunt and devour creatures weaker than themselves.

\emph{Large monstrosity, neutral evil}

\textbf{STRENGTH} +3

\textbf{DEXTERITY} +1

\textbf{CONSTITUTION} +1

\textbf{INTELLIGENCE} -2

\textbf{WISDOM} +0

\textbf{CHARISMA} -1

\textbf{Initiative} +1 -- \textbf{Defense} 14

\textbf{Hit Points} 26 (4d10 + 4)

\textbf{Movement} 15 m

\textbf{Saving Throws}: Fortitude +3, Reflexes +2, Will +2

\textbf{Skills} Awareness +4

\textbf{Senses} vision in the dark 18 m

\textbf{Languages} Goblin, Worg

\textbf{Challenge} 1/2 (100 PX)

\emph{\textbf{Hearing and Smell.}} The worg has +1d6 on Wisdom (Awareness) checks that rely on hearing or smell.

\textbf{Shares}

\emph{\textbf{Bite.} Melee Weapon Attack}: +5 to hit, reach 1 m, one target.

\emph{Hits:} 10 (2d6 + 3) piercing damage. If the target is a creature, it must succeed on a DC 13 Fortitude saving throw or fall prone.

\subsection{Appendix B: Non-Player Characters}\index[Monster]{Non-Player Characters}

This appendix contains statistics for various humanoid non-player characters (NPCs) that adventurers may encounter during a campaign, from lowly commoners to powerful archmages. These statistics can be used to represent human and non-human NPCs.

Customize NPCs

There are many simple ways to customize the NPCs in this appendix for use in your home campaign.

\emph{\textbf{Changing Spells.}} One way to customize an NPC spellcaster is to replace one or more of his spells. You can replace any spell on the list
NPC spells with a different spell of the same level. Changing spells in this way does not change the NPC's challenge rating.

\textbf{\emph{Changing Weapons and Armor}.} You can improve or worsen the NPC's armor or add or change weapons. Changes to Defense and damage can change the NPC's challenge rating.

\emph{\textbf{Magic Items}}. The more powerful an NPC is, the more likely they are to possess one or more magical items. A wizard, for example, might have a wand or staff, as well as one or more potions and scrolls. Providing an NPC with a powerful magical item capable of dealing damage may change their challenge rating.

Some example magic items are described later in this document.

\textbf{Fighters}

Fighters are individuals who earn their living by putting their sword at the service of an individual or an ideal.

\medskip\textbf{Guard}

Guards include members of the city watch, sentries of a citadel or fortified city, and the bodyguards of nobles and merchants.

\emph{Medium humanoid (any race), any Trait}

\textbf{STRENGTH} +1

\textbf{DEXTERITY} +1

\textbf{CONSTITUTION} +1

\textbf{INTELLIGENCE} +0

\textbf{WISDOM} +0

\textbf{CHARRISMA} +0

\textbf{Initiative} +1 -- \textbf{Defense} 17 (mail jacket, shield)

\textbf{Hit Points} 11 (2d8 + 2)

\textbf{Movement} 9 m

\textbf{Saving Throws}: Fortitude +3, Reflexes +1, Will +1

\textbf{Skills} Awareness +2

\textbf{Languages} any language (usually the Municipality)

\textbf{Challenge} 1/8 (25 PX)

\textbf{Shares}

\emph{\textbf{Spear.} Melee or Ranged Weapon Attack}: +3 to hit, reach 1m or range 6m, one target.

\emph{Hits:} 4 (1d6 + 1) piercing damage or 5 (1d8 + 1) piercing damage if used with two hands to make a melee attack.

\medskip\textbf{Veteran}

Warriors survived for a long time, earning a great reputation as expert and skilled fighters.

\emph{Medium humanoid (any race), any Trait}

\textbf{STRENGTH} +3

\textbf{DEXTERITY} +1

\textbf{CONSTITUTION} +2

\textbf{INTELLIGENCE} +0

\textbf{WISDOM} +0

\textbf{CHARRISMA} +0

\textbf{Initiative} +1 -- \textbf{Defense} 19 (strip armour)

\textbf{Hit Points} 58 (9d8 + 18)

\textbf{Movement} 9 m

\textbf{Saving Throws}: Fortitude +4, Reflexes +2, Will +3

\textbf{Skills} Acrobatics +5, Awareness +2

\textbf{Languages} any language (usually the Municipality)

\textbf{Challenge} 3 (700 PX)

\textbf{Shares}

\emph{\textbf{Multiattack.}} The veteran makes two longsword attacks. If he has drawn a short sword, he can also make a short sword attack.

\emph{\textbf{Long Sword.} Melee Weapon Attack}: +5 to hit, reach 1 m, one target.

\emph{Hits:} 7 (1d8 + 3) slashing damage, or 8 (1d10 + 3) slashing damage if used with two hands.

\emph{\textbf{Short Sword.} Melee Weapon Attack}: +5 to hit, reach 1 m, one target.

\emph{Hits:} 6 (1d6 + 3) piercing damage.

\emph{\textbf{Heavy Crossbow.} Ranged Weapon Attack}: +3 to hit, range 30m, one target. \emph{Hits:} 6 (1d10 + 1) piercing damage.

\medskip\textbf{Knight}

Knights are fighters who swear allegiance to rulers, religious orders, and noble causes. The knight's Traits determine the extent to which he is willing to honor his oath.

\emph{Medium humanoid (any race), any Trait}

\textbf{STRENGTH} +3

\textbf{DEXTERITY} +0

\textbf{CONSTITUTION} +2

\textbf{INTELLIGENCE} +0

\textbf{WISDOM} +0

\textbf{CHARISMA} +2

\textbf{Initiative} +0 -- \textbf{Defense} 20 (plate armour)

\textbf{Hit Points} 52 (8d8 + 16)

\textbf{Movement} 9 m

\textbf{Saving Throws}: Fortitude +4, Reflexes +1, Will +3

\textbf{Languages} any language (usually the Municipality)

\textbf{Challenge} 3 (700 PX)

\emph{\textbf{Brave.}} The knight has +1d6 on saving throws against being frightened.

\textbf{Shares}

\emph{\textbf{Multiattack.}} The knight makes two melee attacks.

\emph{\textbf{Big Sword.} Melee Weapon Attack}: +5 to hit, reach 1 m, one target.

\emph{Hits:} 10 (2d6 + 3) slashing damage.

\emph{\textbf{Heavy Crossbow.} Ranged Weapon Attack}: +2 to hit, range 30m, one target.

\emph{Hits:} 5 (1d10) piercing.

\emph{\textbf{Authority (Recharge after 1 hour)}}. For 1 minute, the knight can speak a special command or warning whenever a nonhostile creature within 30 feet of him that he can see makes an attack roll or saving throw. The creature can add a d4 to his roll as long as it can hear and understand the knight. A creature can benefit from only one Leadership die at a time. This effect ends if the knight is incapacitated.

\textbf{Reactions}

\emph{\textbf{Parry.}} The knight can add 2 to his Defense against a melee attack that would hit him. To do so, the knight must see the attacker and be holding a melee weapon.


\medskip\textbf{Citizens}

This category includes those individuals who are responsible for running the world, carrying out the tasks necessary to ensure that fields are cultivated, cities administered, food grown and
new territories explored.

\medskip\textbf{Noble}

The nobles rule over the population, by virtue of a birthright or by accumulated wealth. Among these are also the courtiers who crowd the courts of the rich and powerful.

\emph{Medium humanoid (any race), any Trait}

\textbf{STRENGTH} +0

\textbf{DEXTERITY} +1

\textbf{CONSTITUTION} +0

\textbf{INTELLIGENCE} +1

\textbf{WISDOM} +2

\textbf{CHARISMA} +3

\textbf{Initiative} +1 -- \textbf{Defense} 16 (bib)

\textbf{Hit Points} 9 (2d8)

\textbf{Movement} 9 m

\textbf{Saving Throws}: Fortitude +1, Reflexes +1, Will +2

\textbf{Skills} Sense Emotions +4, Deceive +5

\textbf{Languages} any two languages

\textbf{Challenge} 1/8 (25 PX)

\textbf{Shares}

\emph{\textbf{Rapier.} Melee Weapon Attack}: +3 to hit, reach 1 m, one target.

\emph{Hits:} 5 (1d8 + 1) piercing damage.

\textbf{Reactions}

€27116 € {€27117 € {Parry.}} The noble adds 2 to his Defense against a melee attack that would hit him. To do this, the noble must see

the attacker and wield a melee weapon.

\medskip\textbf{Popular}

The commoners include farmers, serfs, slaves, servants, pilgrims, merchants, artisans and hermits.

\emph{Medium humanoid (any race), any Trait}

\textbf{STRENGTH} +0

\textbf{DEXTERITY} +0

\textbf{CONSTITUTION} +0

\textbf{INTELLIGENCE} +0

\textbf{WISDOM} +0

\textbf{CHARISMA} +0

\textbf{Initiative} +0 -- \textbf{Defense} 11

\textbf{Hit Points} 4 (1d8)

\textbf{Movement} 9 m

\textbf{Saving Throws}: Fortitude +0, Reflexes +0, Will +0

\textbf{Languages} any language (usually the Municipality)

\textbf{Challenge} 0 (10 PX)

\textbf{Shares}

\emph{\textbf{Club.} Melee Weapon Attack}: +2 to hit, reach 1 m, one target.

\emph{Hits:} 2 (1d4) bludgeoning damage.

\medskip\textbf{Criminals}

Criminals are individuals who live on the margins of the law, earning their bread by carrying out activities often considered illicit and immoral.

\medskip\textbf{Bandit/Pirate}

Whether they are men of the street or seafarers (pirates), they earn their living by plundering others.

\emph{Medium humanoid (any race), any non-lawful Trait}

\textbf{STRENGTH} +0

\textbf{DEXTERITY} +1

\textbf{CONSTITUTION} +1

\textbf{INTELLIGENCE} +0

\textbf{WISDOM} +0

\textbf{CHARISMA} +0

\textbf{Initiative} +1 -- \textbf{Defense} 13 (leather armour)

\textbf{Hit Points} 11 (2d8 + 2)

\textbf{Movement} 9 m

\textbf{Saving Throws}: Fortitude +1, Reflexes +2, Will +1

\textbf{Languages} any language (usually the Municipality)

\textbf{Challenge} 1/8 (25 PX)

\textbf{Shares}

\emph{\textbf{Scimitar.} Melee Weapon Attack}: +3 to hit, reach 1 m, one target.

\emph{Hits:} 4 (1d6 + 1) slashing damage.

\emph{\textbf{Light Crossbow.} Ranged Weapon Attack}: +3 to hit, range 24m, one target. \emph{Hits:} 5 (1d8 + 1) slashing damage.

\medskip\textbf{Spy}

A spy is an individual trained in obtaining secrets for someone, or sometimes to resell them to the highest bidder.

\emph{Medium humanoid (any race), any Trait}

\textbf{STRENGTH} +0

\textbf{DEXTERITY} +2

\textbf{CONSTITUTION} +0

\textbf{INTELLIGENCE} +1

\textbf{WISDOM} +2

\textbf{CHARISMA} +3

\textbf{Initiative} +2 -- \textbf{Defense} 13

\textbf{Hit Points} 27 (6d8)

\textbf{Movement} 9 m

\textbf{Saving Throws}: Fortitude +2, Reflexes +3, Will +3

\textbf{Skills} Stealth +4, Sense Emotions +4, Investigation +5, Awareness +6, Deception +5, Fairy Hands +4

\textbf{Languages} any two languages

\textbf{Challenge} 1 (200 PX)

\emph{\textbf{Sneak Attack (1/Turn).}} The spy deals an additional 7 (2d6) damage when he hits a target with a weapon attack and has +1d6 on his attack roll, or when the target is within 3 feet of an assassin's ally who is not incapacitated and the assassin does not have -1d6 on his attack roll.

\emph{\textbf{Cunning Action.}} During each of his rounds, the spy can use a bonus action to take the Retreat, Hide, or Dash action.

\textbf{Shares}

\emph{\textbf{Multiattack.}} The spy makes two melee attacks.

\emph{\textbf{Short Sword.} Melee Weapon Attack}: +4 to hit, reach 1 m, one target.

\emph{Hits:} 5 (1d6 + 2) piercing damage.

\emph{\textbf{Crossbow.} Ranged Weapon Attack}: +4 to hit, range 9m, one target. \emph{Hits:} 5 (1d6 + 2) piercing damage.


\medskip\textbf{Bandit Captain/Pirate}

Whether he lives on land or at sea, he is an individual with a great personality who manages to keep the rabble who respond to his orders in line.

\emph{Medium humanoid (any race), any non-lawful Trait}

\textbf{STRENGTH} +2

\textbf{DEXTERITY} +3

\textbf{CONSTITUTION} +2

\textbf{INTELLIGENCE} +2

\textbf{WISDOM} +0

\textbf{CHARISMA} +2

\textbf{Initiative} +2 -- \textbf{Defense} 16 (studded leather armour)

\textbf{Hit Points} 65 (10d8 + 8)

\textbf{Movement} 9 m

\textbf{Saving Throws}: Fortitude +5, Reflexes +5, Will +3

\textbf{Skills} Acrobatics +4, Deception +4

\textbf{Languages} any two languages

\textbf{Challenge} 2 (450 PX)

\textbf{Shares}

\emph{\textbf{Multiattack.}} The captain makes three melee attacks: two with the scimitar and one with the dagger. Or the captain makes two ranged dagger attacks.

\emph{\textbf{Scimitar.} Melee Weapon Attack}: +5 to hit, reach 1 m, one target.

\emph{Hits:} 6 (1d6 + 3) slashing damage.

\emph{\textbf{Dagger.} Melee or Ranged Weapon Attack}: +5 to hit, reach 1m or range 6m, one target. \emph{Hits:} 5 (1d4 + 3) piercing damage.

\textbf{Reactions}

\emph{\textbf{Save.}} The captain adds 2 to his Defense against a melee attack that would hit him. To do so, the captain must see the attacker and wield a melee weapon.

\medskip\textbf{Assassin}

Whether loners or members of a guild, assassins are paid to eliminate, often quietly and discreetly, rivals and enemies of their employers.

\emph{Medium humanoid (any race), any Bad Trait}

\textbf{STRENGTH} +0

\textbf{DEXTERITY} +3

\textbf{CONSTITUTION} +2

\textbf{INTELLIGENCE} +1

\textbf{WISDOM} +0

\textbf{CHARRISMA} +0

\textbf{Initiative} +3 -- \textbf{Defense} 19 (studded leather armour)

\textbf{Hit Points} 78 (12d8 + 24)

\textbf{Movement} 9 m

\textbf{Saving Throws}: Fortitude +10, Reflexes +11, Will +8

\textbf{Skills} Acrobatics +6, Stealth +9, Awareness +3, Deception +3


\textbf{Languages} Thieves' Slang plus two other languages

\textbf{Challenge} 8 (3900 XP)

\emph{\textbf{Assassinate.}} During his first round, the assassin has +1d6 to attack rolls against creatures that have not yet taken a round. Any hit the assassin lands against a surprised creature is a critical hit.

\emph{\textbf{Sneak Attack (1/Turn).}} The assassin deals an additional 14 (4d6) damage when he hits a target with a weapon attack and has +1d6 on his attack roll, or when the target is within 3 feet of an assassin's ally who is not incapacitated, and the assassin does not have -1d6 on his attack roll.

\emph{\textbf{Evasion.}} If the assassin is the victim of an effect that allows a Reflex saving throw to take half damage, the assassin takes no damage on a successful save, and only the half if he fails.

\textbf{Shares}

\emph{\textbf{Multiattack.}} The assassin makes two attacks with short swords.

\emph{\textbf{Short Sword.} Melee Weapon Attack}: +6 to hit, reach 1 m, one target.

\emph{Hits:} 6 (1d6 + 3) piercing damage, and the target must make a DC 15 Fortitude saving throw, taking 24 (7d6) poison damage on a failed save, or half as much damage on a he succeeds.

\emph{\textbf{Light Crossbow.} Ranged Weapon Attack}: +6 to hit, range 24m, one target.

\emph{Hit:} 7 (1d8 + 3) piercing damage, and the target must make a DC 15 Fortitude saving throw, taking 24 (7d6) poison damage on a failed save, or half as much damage on a he succeeds.

\medskip\textbf{Wizard}

The magician spends his life in the study and practice of magic.

\textbf{VARIANT: FAMILIES}

Any spellcaster who can cast the spell €27257 €{find} €27258 €{familiar} is likely to have a familiar. The familiar can be one of the creatures described in the spell (see the €27259{Basic Rules}) or some other Tiny monster, such as a slithering claw, an imp, a pseudodragon, or an imp.

\medskip\textbf{Adventuring Wizard}

A novice wizard, who has successfully navigated his first adventures and has begun to establish a reputation as a noble or infamous adventurer.

\emph{Medium humanoid (any race), any evil}

\textbf{STRENGTH} -1

\textbf{DEXTERITY} +2

\textbf{CONSTITUTION} +0

\textbf{INTELLIGENCE} +3

\textbf{WISDOM} +1

\textbf{CHARRISMA} +0

\textbf{Initiative} +3 -- \textbf{Defense} 13

\textbf{Hit Points} 22 (5d8)

\textbf{Movement} 9 m

\textbf{Saving Throws}: Fortitude +0, Reflexes +3, Will +2

\textbf{Skills} Arcana +5, History +5

\textbf{Languages} any four languages

\textbf{Challenge} 1 (200 PX)

\emph{\textbf{Spells.}} The wizard has CM 4. His spellcasting ability is Intelligence (+5 to hit with spell attacks). The Wizard has prepared the following spells: Cantrips (at will):

\emph{light, magical hand, dazzling grasp}

level 1 (4 slots): \emph{charm people, Arcane Bolt}

level 2 (3 slots): \emph{Block person, veiled step}

\textbf{Shares}

\emph{\textbf{Staff.} Melee Weapon Attack}: +1 to hit, reach 1 m, one target.

\emph{Hits:} 3 (1d8 - 1) bludgeoning damage.

\medskip\textbf{Grand Wizard}

A magician who has established a good reputation in the area and who attracts students from everywhere.

\emph{Medium humanoid (any race), any Trait}

\textbf{STRENGTH} -1

\textbf{DEXTERITY} +2

\textbf{CONSTITUTION} +0

\textbf{INTELLIGENCE} +3

\textbf{WISDOM} +1

\textbf{CHARISMA} +0

\textbf{Initiative} +3 -- \textbf{Defense} 15 (18 with €27297{Magician's armour})

\textbf{Hit Points} 40 (9d8)

\textbf{Movement} 9 m

\textbf{Saving Throws}: Fortitude +1, Reflexes +4, Will +3

\textbf{Skills} Arcana +6, History +6

\textbf{Languages} any four languages

\textbf{Challenge} 6 (2300 XP)

\emph{\textbf{Spells.}} The wizard has CM 9. His spellcasting skill is Intelligence (+6 to hit with spell attacks). The Wizard has prepared the following spells:

Cantrips (at will): \emph{fire bolt, light, magic hand,}
\emph{prestidigitation}

level 1 (4 slots): \emph{Wizard's Armor, Arcane Bolt,}
\emph{detect magic, shield}

level 2 (3 slots): \emph{veiled step, suggestion}

level 3 (3 slots): \emph{counterspell, fireball, flying}

level 4 (3 slots): \emph{greater invisibility, ice storm}

level 5 (1 slot): \emph{cone of cold}

\textbf{Shares}

\emph{\textbf{Dagger.} Melee or Ranged Weapon Attack}: +5 to hit, reach 1m or range 6m, one target. \emph{Hits:} 4 (1d4 + 2) piercing damage.

\medskip\textbf{Archmage}

A very powerful (and also very old) wizard who studies the secrets of the multiverse.

\emph{Medium humanoid (any race), any Trait}

\textbf{STRENGTH} +0

\textbf{DEXTERITY} +2

\textbf{CONSTITUTION} +1

\textbf{INTELLIGENCE} +5

\textbf{WISDOM} +2

\textbf{CHARRISMA} +3

\textbf{Initiative} +5 -- \textbf{Defense} 18 (21 with €27329{Magician's armour})

\textbf{Hit Points} 99 (18d8 + 18)

\textbf{Movement} 9 m

\textbf{Saving Throws}: Fortitude +13, Reflexes +14, Will +14

\textbf{Skills} Arcana +13, History +13

\textbf{Damage Resistances} spell damage; bludgeoning, piercing, and non-magical slashing (from \emph{stone skin})

\textbf{Languages} any six languages

\textbf{Challenge} 12 (8400 PX)

\emph{\textbf{Spells.}} The wizard has CM 18. His spellcasting skill is Intelligence (+9 to hit with spell attacks).

The archmage can cast \emph{disguise self} and \emph{invisibility} at will and has prepared the following spells: Cantrips (at will): \emph{fire bolt, light, magic hand,}
\emph{prestidigitation, dazzling grip}

level 1 (4 slots): \emph{magical armor*, Arcane Bolt, identify, detect magic}

level 2 (3 slots): \emph{mirror image, detect thoughts, veiled step}

level 3 (3 slots): \emph{counterspell, lightning}

level 4 (3 slots): \emph{exile, stone skin*, fire shield}

level 5 (3 slots): \emph{cone of cold, wall of force, scry}

level 6 (1 slot): \emph{globe of invulnerability}

level 7 (1 slot): \emph{teleport}

level 8 (1 slot): \emph{Mental Shield*}

level 9 (1 slot): \emph{stop time}

The archmage casts these {*} spells on himself before combat.

\textbf{Shares}

\emph{\textbf{Dagger.} Melee or Ranged Weapon Attack}: +6 to hit, reach 1m or range 6m, one target. \emph{Hits:} 4 (1d4 + 2) piercing damage.


\medskip\textbf{Priests}

Priests are devotees of a deity or faith who take it upon themselves to impart divine teachings to their flock.

\medskip\textbf{Cultist}

Cultists pledge allegiance to dark powers, and often show signs of madness in their beliefs and practices.

\emph{Medium humanoid (any race), any Bad Trait}

\textbf{STRENGTH} +0

\textbf{DEXTERITY} +1

\textbf{CONSTITUTION} +0

\textbf{INTELLIGENCE} +0

\textbf{WISDOM} +0

\textbf{CHARISMA} +0

\textbf{Initiative} +0- \textbf{Defense} 13 (leather armour)

\textbf{Hit Points} 9 (2d8)

\textbf{Movement} 9 m

\textbf{Saving Throws}: Fortitude +1, Reflexes +1, Will +2

\textbf{Skills} Deception +2, Religion +2

\textbf{Languages} any language (usually the Municipality)

\textbf{Challenge} 1/8 (25 PX)

\emph{\textbf{Dark Devotion.}} The cultist has +1d6 on saving throws against being charmed or frightened.

\textbf{Shares}

\emph{\textbf{Scimitar.} Melee Weapon Attack}: +3 to hit, reach 3 ft., one creature.

\emph{Hits:} 4 (1d6 + 1) slashing damage.

\medskip\textbf{Acolyte}

Acolytes are junior members of the clergy, and usually answer to a priest of higher rank. They perform various functions in a temple and are granted the ability to cast minor spells by their deity.

\emph{Medium humanoid (any race), any Trait}

\textbf{STRENGTH} +0

\textbf{DEXTERITY} +0

\textbf{CONSTITUTION} +0

\textbf{INTELLIGENCE} +0

\textbf{WISDOM} +2

\textbf{CHARISMA} +0

\textbf{Initiative} +0 -- \textbf{Defense} 11

\textbf{Hit Points} 9 (2d8)

\textbf{Movement} 9 m

\textbf{Saving Throws}: Fortitude +0, Reflexes +0, Will +3

\textbf{Skills} First Aid +4, Religion +2

\textbf{Languages} any language (usually the Municipality)

\textbf{Challenge} 1/4 (50 XP)

\emph{\textbf{Spells.}} The acolyte has CM 1. His spellcasting ability is Wisdom (+4 to hit with spell attacks). The acolyte has prepared the following spells: Cantrips (at will): \emph{holy flame, light, thaumaturgy} level 1 (3 slots): €27402{blessing}, €27403{cure wounds, sanctuary}

\medskip\textbf{Shares}

\emph{\textbf{Club.} Melee Weapon Attack}: +2 to hit, reach 1 m, one target.

\emph{Hits:} 2 (1d4) bludgeoning damage.

\textbf{Cult Fanatic}

They are the leaders of a cult, who use their charisma and dogmas to influence the weak-willed.

\emph{Medium humanoid (any race), any Bad Trait}

\textbf{STRENGTH} +0

\textbf{DEXTERITY} +2

\textbf{CONSTITUTION} +1

\textbf{INTELLIGENCE} +0

\textbf{WISDOM} +1

\textbf{CHARISMA} +2

\textbf{Initiative} +2 -- \textbf{Defense} 14 (leather armour)

\textbf{Hit Points} 33 (6d8 + 6)

\textbf{Movement} 9 m

\textbf{Saving Throws}: Fortitude +2, Reflexes +2, Will +3

\textbf{Skills} Deception +4, Deception +4, Religion +2

\textbf{Languages} any language (usually the Municipality)

\textbf{Challenge} 2 (450 PX)

\emph{\textbf{Spells.}} The priest has CM 4. His spellcasting ability is Wisdom (+3 to hit with spell attacks). The priest has prepared the following spells: Cantrips (at will): \emph{holy flame, light, thaumaturgy}

level 1 (4 slots): \emph{command, inflict wounds, shield of faith}

level 2 (3 slots): \emph{spiritual weapon, block person}

\emph{\textbf{Dark Devotion.}} The cultist has +1d6 on saving throws against being charmed or frightened.

\textbf{Shares}

\emph{\textbf{Multiattack.}} The zealot makes two melee attacks.

\emph{\textbf{Dagger.} Melee or Ranged Weapon Attack}: +4 to hit, reach 3 ft. or range 20 ft., one creature. \emph{Hits:} 4 (1d4 + 2) piercing damage.

\medskip\textbf{High Priest}

They are individuals in command of a temple or other sacred site and who have several acolytes at their disposal.

\emph{Medium humanoid (any race), any Trait}

\textbf{STRENGTH} +0

\textbf{DEXTERITY} +0

\textbf{CONSTITUTION} +1

\textbf{INTELLIGENCE} +1

\textbf{WISDOM} +3

\textbf{CHARISMA} +1

\textbf{Initiative} +1 -- \textbf{Defense} 14 (shirt jacket)

\textbf{Hit Points} 27 (5d8 + 5)

\textbf{Movement} 7 m

\textbf{Saving Throws}: Fortitude +1, Reflexes +1, Will +4

\textbf{Skills} First Aid +7, Deception +3, Religion +4

\textbf{Languages} any two languages

\textbf{Challenge} 2 (450 PX)

\emph{\textbf{Divine Eminence.}} As a bonus action, the priest can expend a spell slot to cause his melee weapon attack to deal an additional 10 (3d6) Light damage. The benefit lasts until the end of the round.

\emph{\textbf{Spells.}} The priest has CM 5. His spellcasting ability is Wisdom (+5 to hit with spell attacks). The priest has prepared the following spells: Cantrips (at will): \emph{holy flame, light, thaumaturgy}

level 1 (4 slots): \emph{cure wounds, tracer bolt, sanctuary}

level 2 (3 slots): \emph{spiritual weapon, lesser restoration}

level 3 (2 slots): \emph{dispel magic}, €27463{spiritual guardians}

\textbf{Shares}

\emph{\textbf{Mace.} Melee Weapon Attack}: +2 to hit, reach 1 m, one target.

\emph{Hits:} 3 (1d6) bludgeoning damage.


\medskip\textbf{Savages}

These individuals live on the fringes of civilization, sometimes rarely coming into contact with it. Uncomfortable within walls and in civilized lands, they are in their environment when they can move through the wilderness.

\medskip\textbf{Berserker}

Hailing from the wilds, the unpredictable berserkers gather in warbands and are always looking for conflicts in which to fight.

\emph{Medium humanoid (any race), any Chaotic Trait}

\textbf{STRENGTH} +3

\textbf{DEXTERITY} +1

\textbf{CONSTITUTION} +3

\textbf{INTELLIGENCE} -1

\textbf{WISDOM} +0

\textbf{CHARRISMA} -1

\textbf{Initiative} +1 -- \textbf{Defense} 14 (leather armour)

\textbf{Hit Points} 67 (9d8 + 27)

\textbf{Movement} 9 m

\textbf{Saving Throws}: Fortitude +4, Reflexes +3, Will +2

\textbf{Languages} any language (usually the Municipality)

\textbf{Challenge} 2 (450 PX)

\emph{\textbf{Unwary.}} At the start of its round, the berserker can gain +1d6 on all melee weapon attack rolls made during that round, but attack rolls against it have + 1d6 until the start of his next round.

\textbf{Shares}

\emph{\textbf{Big Axe.} Melee Weapon Attack}: +5 to hit, reach 1 m, one target.

\emph{Hits:} 9 (1d12 + 3) slashing damage.\\

\textbf{Tribal Fighter}

They are the defenders of the tribes living on the fringes of civilization.

\emph{Medium humanoid (any race), any Trait}

\textbf{STRENGTH} +1

\textbf{DEXTERITY} +0

\textbf{CONSTITUTION} +1

\textbf{INTELLIGENCE} -1

\textbf{WISDOM} +0

\textbf{CHARISMA} -1

\textbf{Initiative} +0 -- \textbf{Defense} 13 (leather armour)

\textbf{Hit Points} 11 (2d8 + 2)

\textbf{Movement} 9 m

\textbf{Saving Throws}: Fortitude +2, Reflexes +1, Will +1

\textbf{Languages} any language

\textbf{Challenge} 1/8 (25 PX)

\emph{\textbf{Pack Tactics.}} The tribal fighter has +1d6 to attack rolls against a creature if at least one of the bruiser's allies is within 3 feet of the creature and that ally is not incapacitated.

\textbf{Shares}

\emph{\textbf{Spear.} Melee or Ranged Weapon Attack}: +3 to hit, reach 1m or range 6m, one target.

\emph{Hits:} 4 (1d6 + 1) piercing damage, or 5 (1d8 + 1) piercing damage if used with two hands to make a melee attack.

\medskip\textbf{Druid}

Druids protect the natural world from monsters and the advance of civilization. Some are tribal shamans who heal the sick, pray to animal spirits, and provide spiritual advice. They are usually devotees of Ephrem or Shayalia.

\emph{Medium humanoid (any race), any Trait}

\textbf{STRENGTH} +0

\textbf{DEXTERITY} +1

\textbf{CONSTITUTION} +1

\textbf{INTELLIGENCE} +1

\textbf{WISDOM} +2

\textbf{CHARISMA} +0

\textbf{Initiative} +1 -- \textbf{Defense} 12 (17 with \emph{bark skin}*)

\textbf{Hit Points} 27 (5d8 + 5)

\textbf{Movement} 9 m

\textbf{Saving Throws}: Fortitude +1, Reflexes +2, Will +3 \\

\textbf{Skills} First Aid +4, Nature +3, Awareness +4

\textbf{Languages} Druidic plus two other languages

\textbf{Challenge} 2 (450 PX)

\emph{\textbf{Spells.}} The priest has CM 4. His spellcasting ability is Wisdom (+4 to hit with spell attacks). The priest has prepared the following spells: Cantrips (at will): \emph{Druidic Artifice, staff, produce flame}

level 1 (4 slots): \emph{hinder, thunder wave, talk to gods}
\emph{animals, fast pace}

level 2 (3 slots): \emph{messenger animal, bark skin}

\textbf{Shares}

\emph{\textbf{Combat Staff.} Melee Weapon Attack}: +2 to hit (+4 to hit with \emph{staff*}), reach 1m or range 6m, one target.

\emph{Hits:} 3 (1d6) bludgeoning damage, or 6 (1d8 + 2) bludgeoning damage with \emph{staff} or if held with two hands.

\medskip\textbf{Explorer}

Skilled hunters and trail trackers.

\emph{Medium humanoid (any race), any Trait}

\textbf{STRENGTH} +0

\textbf{DEXTERITY} +2

\textbf{CONSTITUTION} +1

\textbf{INTELLIGENCE} +0

\textbf{WISDOM} +1

\textbf{CHARRISMA} +0

\textbf{Initiative} +2 -- \textbf{Defense} 14 (leather armour)

\textbf{Hit Points} 16 (3d8 + 3)

\textbf{Movement} 9 m

\textbf{Saving Throws}: Fortitude +1, Reflexes +2, Will +3

\textbf{Skills} Stealth +6, Nature +4, Awareness +5, Survival +5

\textbf{Languages} any language (usually Common)

\textbf{Challenge} 1/2 (100 PX)

\emph{\textbf{Hot Sense and Sense.}} The explorer has +1d6 on Wisdom (Awareness) checks that rely on smell or sight.

\textbf{Shares}

\emph{\textbf{Multiattack.}} The ranger makes two melee attacks or two ranged attacks.

\emph{\textbf{Short Sword.} Melee Weapon Attack}: +4 to hit, reach 1 m, one target.

\emph{Hits:} 5 (1d6 + 2) piercing damage.

\emph{\textbf{Longbow.} Melee Weapon Attack}: +4 to hit, range 45m, one target.

\emph{Hits:} 6 (1d8 + 2) piercing damage.


\end{multicols}

%{\scriptsize
%\printindex}
%\end{document}

\pagebreak

\section{List of Monsters by Challenge Level}

\begin{multicols}{3}
{\footnotesize
\noindent Eagle, CR 0 (10 PX)\\
Vulture, CR 0 (10 XP)\\
Baboon, CR 0 (10 XP)\\
Goat, CR 0 (10 XP)\\
Deer, CR 0 (10 XP)\\
Raven, CR 0 (10 XP)\\
Weasel, CR 0 (10 XP)\\
Hawk, CR 0 (10 XP)\\
Shrieking Mushroom, CR 0 (10 XP)\\
Cat, CR 0 (10 XP)\\
Owl, CR 0 (10 XP)\\
Hyena, CR 0 (10 XP)\\
Lemur, CR 0 (10 XP)\\
Lizard, CR 0 (10 XP)\\
Homunculus, CR 0 (10 XP)\\
Pirana, CR 0 (10 XP)\\
Commoner, CR 0 (10 XP)\\
Spider, CR 0 (10 XP)\\
Frog, CR 0 (10 XP)\\
Rat, CR 0 (10 XP)\\
Giant Fire Beetle, CR 0 (10 PX)\\
Jackal, CR 0 (10 XP)\\
Scorpion, CR 0 (10 XP)\\
Badger, CR 0 (10 PX)\\Megalopede
Mice, Challenge: 0 (10 PX)\\
Bandit/Pirate, CR 1/8 (25 XP)\\
Kobold, CR 1/8 (25 XP)\\
Cultist, CR 1/8 (25 XP)\\
Giant Weasel, CR 1/8 (25 PX)\\
Blood Hawk, CR 1/8 (25 XP)\\
Giant Crab, CR 1/8 (25 PX)\\
Guard, CR 1/8 (25 XP)\\
Mastiff, CR 1/8 (25 XP)\\
Mule, CR 1/8 (25 PX)\\
Noble, CR 1/8 (25 XP)\\
Pony, CR 1/8 (25 XP)\\
Giant Rat, CR 1/8 (25 XP)\\
Poisonous Serpent, CR 1/8 (25 XP)\\
Flying Serpent, CR 1/8 (25 XP)\\
Striga, CR 1/8 (25 XP)\\
Striga (Stygian Bird), CR 1/8 (25 XP)\\
Waterman, CR 1/8 (25 XP)\\
Acolyte, CR 1/4 (50 XP)\\
Elk, CR 1/4 (50 XP)\\
Ax Beak, CR 1/4 (50 XP)\\
Intermittent Dog, CR 1/4 (50 PX)\\
Racehorse, CR 1/4 (50 PX)\\
Draft Horse, CR 1/4 (50 XP)\\
Giant Centipede, CR 1/4 (50 PX)\\
Boar, CR 1/4 (50 PX)\\
Dretch, CR 1/4 (50 XP)\\
Violet Mushroom, CR 1/4 (50 XP)\\
Gablin, CR 1/4 (50 XP)\\
Grimlock, CR 1/4 (50 XP)\\
Giant Owl, CR 1/4 (50 PX)\\
Giant Lizard, CR 1/4 (50 PX)\\
Wolf, CR 1/4 (50 XP)\\
Steam Mephite, CR 1/4 (50 XP)\\
Panther, CR 1/4 (50 PX)\\
Pseudodragon, CR 1/4 (50 XP)\\
Giant Wolf Spider, CR 1/4 (50 XP)\\
Giant Frog, CR 1/4 (50 PX)\\
Skeleton, CR 1/4 (50 PX)\\
Crow Swarm, CR 1/4 (50 XP)\\
Swarm of Bats, CR 1/4 (50 PX)\\
Rat Swarm, CR 1/4 (50 XP)\\
Constrictor Serpent, CR 1/4 (50 XP)\\
Giant Poisonous Serpent, CR 1/4 (50 XP)\\
Flying Sword, CR 1/4 (50 XP)\\
Sprite, CR 1/4 (50 XP)\\
Giant Badger, CR 1/4 (50 PX)\\
Zombies, CR 1/4 (50 XP)\\
Giant Goat, CR 1/2 (100 XP)\\
War Horse, CR 1/2 (100 XP)\\
Giant Sea Horse, CR 1/2 (100 PX)\\
Crocodile, CR 1/2 (100 PX)\\
Cockatrice, CR 1/2 (100 XP)\\
Explorer, CR 1/2 (100 XP)\\
Gnoll, CR 1/2 (100 XP)\\
Deep Gnome, CR 1/2 (100 PX)\\
Hobgoblin, CR 1/2 (100 XP)\\
Lizardfolk, CR 1/2 (100 PX)\\
Darkcloak, CR 1/2 (100 PX)\\
Ice Mephitus, CR 1/2 (100 XP)\\
Magma Mephitus, CR 1/2 (100 XP)\\
Dust Mephite, CR 1/2 (100 XP)\\
Gray Slime, CR 1/2 (100 XP)\\
Shadow, CR 1/2 (100 PX)\\
Orc, CR 1/2 (100 XP)\\
Black Bear, CR 1/2 (100 PX)\\
Rustmage, CR 1/2 (100 XP)\\
Sahuagin, CR 1/2 (100 XP)\\
Satyr, CR 1/2 (100 XP)\\
Warhorse Skeleton, CR 1/2 (100 XP)\\
Insect Swarm, CR 1/2 (100 XP)\\
Swarm of Spiders, CR 1/2 (100 XP)\\
Swarm of Scarabs, CR 1/2 (100 XP)\\
Swarm of Wasps, CR 1/2 (100 XP)\\
Swarms, CR 1/2 (100 XP)\\
Ape, CR 1/2 (100 XP)\\
Coral Shark, CR 1/2 (100 PX)\\
Magma Man (Ignim), CR 1/2 (100 PX)\\
Giant Wasp, CR 1/2 (100 PX)\\
Worg, CR 1/2 (100 XP)\\
Giant Eagle, CR 1 (200 XP)\\
Animated Armor, CR 1 (200 XP)\\
Harpy, CR 1 (200 XP)\\
Giant Vulture, CR 1 (200 PX)\\
Bugbear, CR 1 (200 XP)\\
Death Dog, CR 1 (200 XP)\\
Dinowolf (Direwolf), CR 1 (200 XP)\\
Brass Dragon Hatchling, CR 1 (200 XP)\\
Copper Dragon Hatchling, CR 1 (200 XP)\\
Dryad, CR 1 (200 XP)\\
Duergar, CR 1 (200 XP)\\
Ghoul, CR 1 (200 XP)\\
Globule, CR 1 (200 XP)\\
Giant Hyena, CR 1 (200 XP)\\
Imp, CR 1 (200 PX)\\
Hippogriff, CR 1 (200 HP)\\
Lion, CR 1 (200 XP)\\
Adventurer Wizard, CR 1 (200 XP)\\
Orc, CR 1 (100 XP)\\
Brown Bear, CR 1 (200 XP)\\
Quasit, CR 1 (200 PX)\\
Giant Spider, CR 1 (200 XP)\\
Giant Toad, CR 1 (200 XP)\\
Pirana Swarm, CR 1 (200 XP)\\
Spy, CR 1 (200 PX)\\
Tiger, CR 1 (200 XP)\\
Awakened Tree, CR 2 (450 PX)\\
Giant Elk, CR 2 (450 PX)\\
Straw Amoeba, CR 2 (450 PX)\\
Ankheg, CR 2 (450 XP)\\
Azer, CR 2 (450 PX)\\
Berserker, CR 2 (450 XP)\\
Explosive Cockroach, CR 2 (450 PX)\\
Bandit/Pirate Captain, CR 2 (450 XP)\\
Centaur, CR 2 (450 XP)\\
Giant Boar, CR 2 (450 PX)\\
Gelatinous Cube, CR 2 (450 PX)\\
Thorny Devil, CR 2 (450 PX)\\
White Dragon Hatchling, CR 2 (450 PX)\\
Silver Dragon Hatchling, CR 2 (450 HP)\\
Bronze Dragon Hatchling, CR 2 (450 XP)\\
Black Dragon Hatchling, CR 2 (450 PX)\\
Green Hatchling Dragon, CR 2 (450 HP)\\
Druid, CR 2 (450 XP)\\
Lesser Water Elemental, CR 2 (450 XP)\\
Ettercap, CR 2 (450 PX)\\
Bubbling Maw, CR 2 (450 XP)\\
Wisp, CR 2 (450 XP)\\
Gargoyle, CR 2 (450 XP)\\
Ghast, CR 2 (450 XP)\\
Grick, CR 2 (450 PX)\\
Griffon, CR 2 (450 PX)\\
Sea Hag, CR 2 (450 PX)\\
Mimic, CR 2 (450 PX)\\
Ogre, CR 2 (450 XP)\\
Polar Bear, CR 2 (450 PX)\\
Pegasus, CR 2 (450 PX)\\
Plesiosaur, CR 2 (450 XP)\\
Wererat, CR 2 (450 XP)\\
Rhino, CR 2 (450 PX)\\
Priest, CR 2 (450 XP)\\
Minotaur Skeleton, CR 2 (450 XP)\\
Poisonous Snake Swarm, CR 2 (450 XP)\\
Giant Constrictor Serpent, CR 2 (450 XP)\\
Hissing, CR 2 (450 PX)\\
Hunter Shark, CR 2 (450 XP)\\
Carpet of Suffocation, CR 2 (450 PX)\\
Flaming Skull, CR 2 (200 XP)\\
Saber-toothed Tiger, CR 2 (450 XP)\\
Zombie Ogre, CR 2 (450 XP)\\
Killer Whale (Orca), CR 3 (700 HP)\\
Basilisk, CR 3 (700 XP)\\
Gablin Champion, CR 3 (700)\\
Knight, CR 3 (700 XP)\\
Nightmare Steed, CR 3 (700 XP)\\
Bearded Devil, CR 3 (700 PX)\\
Doppelganger, CR 3 (700 XP)\\
Blue Dragon Hatchling, CR 3 (700 PX)\\
Golden Dragon Hatchling, CR 3 (700 XP)\\
Winter Wolf, CR 3 (700 XP)\\
Werewolf, CR 3 (700 XP)\\
Manticore, CR 3 (700 XP)\\
Green Hag, CR 3 (700 XP)\\
Minotaur, CR 3 (700 XP)\\
Mummy, CR 3 (700 XP)\\
Wall Climbing Horror, CR 3 (700 XP)\\
Owlbear, CR 3 (700 XP)\\
Wise Owlbear, CR 3 (700 XP)\\
Phase Spider, CR 3 (700 XP)\\
Giant Scorpion, CR 3 (700 XP)\\
Hellhound, CR 3 (700 XP)\\
Veteran, CR 3 (700 XP)\\
Wight, CR 3 (700 PX)\\
B.O.C., CR 4 (1100 PX)\\
Banshee, CR 4 (1100 XP)\\
Chuul, CR 4 (1100 XP)\\
Wereboar, CR 4 (1100 PX)\\
Couatl, CR 4 (1100 PX)\\
Red Dragon Hatchling, CR 4 (1100 XP)\\
Elephant, CR 4 (1100 PX)\\
Ettin, CR 4 (1100 PX)\\
Ghost, CR 4 (1100 PX)\\
Rotting Ghoul, CR 4 (1100 XP)\\
Lamia, CR 4 (1100 PX)\\
Cursed Immortal, CR 4 (1100 XP)\\
Black Protoplasm, CR 4 (1100 XP)\\
Succubus, CR 4 (1100 XP)\\
Weretiger, CR 4 (1100 XP)\\
Darktorch, CR 4 (1100 PX)\\
Tentacled Crawler Worm, CR 4 (1100 XP)\\
Bulette, CR 5 (1800 PX)\\
Giant Crocodile, CR 5 (1800 PX)\\
Creeping Cumulus, CR 5 (1800 PX)\\
Fire Elemental, CR 5 (1800 XP)\\
Water Elemental, CR 5 (1800 XP)\\
Air Elemental, CR 5 (1800 XP)\\
Earth Elemental, CR 5 (1800 XP)\\
Roper, CR 5 (1800 PX)\\
Ghoul, Mother, CR 5 (1800 XP)\\
Hill Giant, CR 5 (1800 XP)\\
Flesh Golem, CR 5 (1800 XP)\\
Gorgon, CR 5 (1800 XP)\\
Night Hag, CR 5 (1800 HP)\\
Werebear, CR 5 (1800 PX)\\
Otyugh, CR 5 (1800 XP)\\
Salamander, CR 5 (1800 PX)\\
Giant Shark, CR 5 (1800 PX)\\
Triceratops, CR 5 (1800 XP)\\
Troll, CR 5 (1800 XP)\\
Unicorn, CR 5 (1800 PX)\\
Wraith, CR 5 (1800 XP)\\
Xorn, CR 5 (1800 XP)\\
Chimera, CR 6 (2300 XP)\\
Young White Dragon, CR 6 (2300 HP)\\
Young Brass Dragon, CR 6 (2300 HP)\\
Drider, CR 6 (2300 PX)\\
Ghoul, Black, CR 6 (2300 PX)\\
Great Mage, CR 6 (2300 XP)\\
Mammoth, CR 6 (2300 PX)\\
Medusa, CR 6 (2300 XP)\\
Paladin Gablin, CR 6 (2300 XP)\\
Invisible Stalker, CR 6 (2300 XP)\\
Vampiric Spawn, CR 6 (1800 XP)\\
Wyvern, CR 6 (2300 XP)\\
Vrock, CR 6 (2300 PX)\\
Young Copper Dragon, CR 7 (2900 XP)\\
Young Black Dragon, CR 7 (2900 HP)\\
Stone Giant, CR 7 (2900 XP)\\
Protective Guardian, CR 7 (2900 HP)\\
Oni, CR 7 (2900 XP)\\
Giant Ape, CR 7 (2900 XP)\\
Assassin, CR 8 (3900 XP)\\
Chain Devil, CR 8 (3900 XP)\\
Young Bronze Dragon, CR 8 (3900 XP)\\
Young Green Dragon, CR 8 (3900 HP)\\
Frost Giant, CR 8 (3900 XP)\\
Hezrou, GS 8 (3900 PX)\\
Hydra, CR 8 (3900 XP)\\
Killer Cloak, CR 8 (3900 XP)\\
Spiritual Naga, CR 8 (3900 XP)\\
Tyrannosaurus, CR 8 (3900 XP)\\
Bone Devil, CR 9 (5000 XP)\\
Devour Brains, CR 9 (5000 XP)\\
Young Blue Dragon, CR 9 (5000 XP)\\
Young Silver Dragon, CR 9 (5000 XP)\\
Greater Water Elemental, CR 9 (5000 XP)\\
Fire Giant, CR 9 (5000 XP)\\
Cloud Giant, CR 9 (5000 XP)\\
Glabrezu, CR 9 (5000 XP)\\
Clay Golem, CR 9 (5000 XP)\\
Treeman (Treant), CR 9 (5000 XP)\\
Aboleth, CR 10 (5900 XP)\\
Angelo Deva, CR 10 (5900 PX)\\
Young Golden Dragon, CR 10 (5900 XP)\\
Young Red Dragon, CR 10 (5900 HP)\\
G.E.C., CR 10 (5900 PX)\\
Stone Golem, CR 10 (5900 XP)\\
Naga Guardian, CR 10 (5900 XP)\\
Behir, CR 11 (7200 PX)\\
Horned Devil, CR 11 (7200 PX)\\
Djinni, CR 11 (7200 PX)\\
Efreeti, CR 11 (7200 PX)\\
Ginosphinx, CR 11 (7200 PX)\\
Remorhaz, CR 11 (7200 XP)\\
Archmage, CR 12 (8400 XP)\\
Erinyes, CR 12 (8400 PX)\\
Panoptikhan, CR 12 (8400 PX)\\
Adult White Dragon, CR 13 (10000 HP)\\
Adult Brass Dragon, CR 13 (10000 HP)\\
Storm Giant, CR 13 (10000 XP)\\
Nalfeshnee, CR 13 (10000 XP)\\
Rakshasa, CR 13 (10000 XP)\\
Vampire, CR 13 (10000 XP)\\
Ice Devil, CR 14 (11,500 XP)\\
Adult Copper Dragon, CR 14 (11,500 XP)\\
Adult Bronze Dragon, CR 15 (13000 HP)\\
Adult Green Dragon, CR 15 (13000 HP)\\
Phoenix, CR 15 (13000 XP)\\
Sovereign Mummy, CR 15 (13000 XP)\\
Purple Worm, CR 15 (13000 XP)\\
Planetar Angel, CR 16 (15000 PX)\\
Adult Blue Dragon, CR 16 (15000 HP)\\
Adult Silver Dragon, CR 16 (15000 HP)\\
Iron Golem, CR 16 (15000 XP)\\
Marilith, CR 16 (15000 XP)\\
Androsphinx, CR 17 (18000 XP)\\
Adult Golden Dragon, CR 17 (18000 HP)\\
Adult Black Dragon, CR 17 (18000 HP)\\
Adult Red Dragon, CR 17 (18000 HP)\\
Dragon Tortoise, CR 17 (18000 XP)\\
Black Knight, CR 18 (20000 XP)\\
Balor, CR 19 (22000 XP)\\
Pit Devil, CR 20 (25000 XP)\\
Ancient White Dragon, CR 20 (25000 XP)\\
Ancient Brass Dragon, CR 20 (25000 XP)\\
Angelo Solar, GS 21 (33000 PX)\\
Ancient Copper Dragon, CR 21 (33000 XP)\\
Ancient Black Dragon, CR 21 (33000 XP)\\
Lich, CR 21 (33000 XP)\\
Ancient Bronze Dragon, CR 22 (41000 XP)\\
Ancient Green Dragon, CR 22 (41000 HP)\\
Ancient Blue Dragon, CR 23 (50000 XP)\\
Ancient Silver Dragon, CR 23 (50000 XP)\\
Ancient Yellow Dragon, Challenge: 23 (50000 XP)\\
Kraken, CR 23 (50000 XP)\\
Ancient Gold Dragon, CR 24 (62000 XP)\\
Ancient Red Dragon, CR 24 (62000 HP)\\
Demogorgon, CR 26 (90000 HP)\\
Orcus, CR 26 (90000 XP)\\
Tàhil, CR 30 (155000 PX)\\
Tarrasque, CR 30 (155000 PX)\\
}

\end{multicols}

\pagebreak

\subsection{Monster Conversion}\index{Monster Conversion}

\bigskip

To add more monsters to OBSS I invite you to convert them from Pathfinder or the 5ed of the famous role-playing game. OBSS is basically a d20 system heavily modified in the dynamics but not in the foundations of the numerical values.

\medskip

\textbf{Conversion from Pathfinder}

Let's take the Common Ogre for example

https://www.d20pfsrd.com/bestiary/monster-listings/humanoids/orcs/orc/

let's leave out the descriptive parts and focus on the numbers and values.

\bigskip{}

\textbf{Orc (Challenge rating 1/3)} this value remains the same in OBSS

\textbf{XP 135} this value must be traced back to the experience points table for CR (100 px in this case)

\textbf{Orc warrior 1} we don't care about the class, only the race at most.

\textbf{CE Medium humanoid} indicates to us that the creature is medium sized, humanoid and evil, for the purposes of the Traits the creature is not of such a level as to have attracted the attention of a Patron.

\textbf{Init +0} is initiative, take the value from Dexterity or Intelligence if higher.

\textbf{Senses} darkvision 60 ft.; Perception -1, remains the same, the same values ​​and abilities remain. In this case 60 feet means the distance is 20 meters

\textbf{Weakness} light sensitivity we look for the equivalent in OBSS where possible, in this case light photophobia or the indicated penalties are applied directly.

\textbf{AC} 13, touch 10, flat-footed 13 (+3 armor) this is Defense.

\textbf{Weapon Proficiency}: equal to the indicated BAB

\textbf{Magical Proficiency}: is basically half the Challenge rating. Only useful if the creature has magical powers.

\textbf{hp} 6 (1d10+1) remains the same

\textbf{Strength} +3, Ref +0, Willpower -1 translate into Stamina, Reflexes and Willpower. The score remains the same

\textbf{Speed} 30 ft. It's movement, in this case it's 30 feet per move action.

\textbf{Melee} Glaive +5 (2d4+4/18--20) is my attack roll and damage. It stays the same

\textbf{Ranged} javelin +1 (1d6+3) is the attack roll and damage. It stays the same

\textbf{Str} 17, Dex 11, Con 12, Int 7, Wis 8, Cha 6. You only have to take the bonus part.

\textbf{Base Atk} +1; CMB +4; CMD 14 the first value determines the AC. I suggest using the hit bonuses indicated in the melee directly.

\textbf{Feats} Weapon Focus (Glaive) Weapon Focus. The skill bonus is already calculated in the Melee values

\textbf{Skills} Intimidate +2 remains the same.

\textbf{Ferocity} (Ex): An orc remains conscious and can continue fighting even if its hit point total is below 0. It is still staggered and loses 1 hit point each round. A creature with ferocity still dies when its total hit point reaches a negative amount equal to its Constitution score. You take the skill as it is.


\bigskip

\textbf{Conversion from 5e} of the famous Role Playing Game:

- Increase Defense and attack rolls by 1/2 point per creature's Challenge Rating, rounded up.

- For saving throws, take the creature's Challenge Rating as a basis and then apply the ability modifier relating to Dexterity (Reflexes), Constitution (Fortitude), Willpower (Wisdom).

\pagebreak

\section{Conditions}\index{Conditions}

\begin{changemargin}{0.3cm}{0.3cm}\begin{enfasi}{
The greatness of man lies in the decision to be stronger than his condition. (Albert Camus)}
\end{enfasi}\end{changemargin}\medskip

%{\small
\begin{multicols}{2}

\label{condizioni}

\textbf{Damage Category Increase}\index{Dice Increase}\index{Die Size}: when the rule tells you to increase the size or category of a die follow this scheme.

1d4 > 1d6 > 1d8 > 1d10 > 2d6 > 2d8 > 2d10 > 3d6 > 3d8 > 3d10.

\textbf{Blinded}:\index{Blinded} The character can't see anything. takes a -2 penalty to most Strength and Dexterity-based skills.

All vision-based checks or activities (such as reading, or any vision-based Awareness checks) automatically fail. All opponents are considered to have invisibility towards the blinded character.

Blinded characters always treat the terrain as difficult and must make a DC 12 Acrobatics check to move faster than their half speed. Creatures that fail this check fall prone. Characters who remain blinded for a long time may become accustomed to some of these penalties and begin to overcome some of them, at the Storyteller's discretion.

Whoever attacks a creature that is invisible to it has a -1d6 to attack roll, the invisible creature that attacks a creature that cannot see it has +1d6 to attack roll

\textbf{Charmed}:\index{Charmed}: A charmed creature can't attack or target its charmer with special abilities or harmful magical effects.

Any potential threat, such as an approaching hostile creature, allows the charmed creature a new saving throw against the charm's effect. Any overt threat, such as someone drawing a weapon, casting a spell, or pointing a ranged weapon at the charmed creature, automatically ends the effect.

An ally of the charmed creature can shake it to allow it a new saving throw by spending 2 actions.

The charmer has +1d6 on any Proficiency check to interact socially with the creature.

\textbf{Fatigued}\index{Fatigued}\hypertarget{Fatigued}{}\index{Exhausted}€2656: A fatigued character cannot run or charge and suffers a -2 penalty to Defense and Attack Rolls and saving throws. If he does anything normally tiring, his Fatigued rating increases and he also takes penalties on movement and proficiency checks.

If a character does not sleep at least 6 (Fortitude save DC 17) hours or sleeps with medium or heavy armor in the morning he is tired.

A Fatigued 2 character moves more slowly and takes a –4 penalty on attack rolls, defense, and saving throws. After 1 hour of complete rest (or Lesser Restoration), a Fatigued 2 character becomes Fatigued. 

\medskip

\textbf{Table: Fatigue Levels}\index[Tables]{Fatigue Levels Table}

\medskip

\begin{tabularx}{0.45\textwidth}{lcl}
\textbf{Conditions}& \textbf{Pen./Mov/Comp.}&\textbf{Rec.}\\
\hline
Fatigued &2/-/-&1h\\
Fatigued 2&4/2m/-4&1h\\
Fatigued 3&6/3m/-6&8h\\
Fatigued 4&8/6m/-8&12h\\
Fatigued 5&Faint&12h\\
Fatigued 6&Death&--\\
\end{tabularx}


After 8 hours of rest a creature goes from Fatigued 3 to Fatigued 2 and after another hour it goes to Fatigued, as long as it suffers no further fatigue


\medskip

\textbf{Grabbed}\index{Grabbed}: A grabbed character cannot move but can try to Push. She must use two Actions to free herself (Fortitude save opposed with Strength bonus + 1d6 per Size difference). 

He can attack with melee weapons if they are small (he will hardly be able to use a greatsword, halberd... a dagger or short sword is more likely). He has -2 Defense and is Distracted.

The conditions apply to those who are grabbed and those who grab.

\textbf{Friendly}:\index{Friendly} A friendly creature will not attack the character unless explicitly threatened.

\textbf{Drown/Hold your breath}: \index{Drown}\index{Hold your breath}\index{Suffocate} Any character can hold their breath for a number of rounds equal to 6 rounds for their character score Constitution, with a minimum of 3 rounds. For each Action performed the remaining duration decreases by 1 round. After this period of time, the character must make a DC 12 Fortitude save each round to continue holding his breath. Each round, the DC increases by 2. See page. \pageref{hold your breath}

\textbf{Deafened}:\index{Deafened}\index{Deaf} A deafened character cannot hear. He automatically fails all sound-based Awareness checks and is considered distracted when casting spells with at least verbal components.

Characters who remain deafened for long periods of time can become accustomed to these penalties and overcome some of them, at the Storyteller's discretion.

\textbf{Poisoned}\index{Poisoned}: any subject under the influence of a poison or potion is considered poisoned, regardless of whether it is already producing the effects or has yet to produce them given the time of onset.

\textbf{Blocked}\index{Blocked}\hypertarget{Blocked}{}: A blocked creature has its arms blocked. She can move by trying to Push, she must use two Actions to free herself (see Grabbed). It takes -4 to Defense and Reflex saving throws. A blocked spellcaster is Distracted and must make a Magic Check with a Critical Magic Success or be unable to cast spells. -1d6 to attack roll. Anyone who has Blocked a creature is still considered Grabbed.

\textbf{Coup de Grace}:\index{Coup de Grace} As its only action in the round, a creature can use a melee weapon to deal a final blow to an incapacitated or defenseless character. He can also use a bow or crossbow, the main thing is that it is adjacent to the target.

The attacker automatically hits and deals three critical hits. Creatures immune to critical hits cannot suffer a Coup de Grace.

\textbf{Confused}: \index{Confused}\index{Confusion}A confused creature is mentally darkened and cannot act normally. A confused creature cannot distinguish an ally from an enemy and treats everyone as an enemy.

If a confused creature is attacked, it always attacks the creature that last attacked it, until that creature dies or leaves its line of sight.

Roll a die on the table below at the start of each round for the confused creature to see what it does that round.

\textbf{d100 Behavior:}

01-25 Acts normally

26-50 Does nothing but stammer incoherently

51-75 You inflict 1d8 + Strength modifier with the weapon in your hand

76-100 Attacks the nearest creature (for this purpose, a Familiar counts as part of the subject itself)

A confused creature that is unable to perform the indicated action will do nothing but babble incoherently. Attackers have no special advantage when attacking a confused creature. Any confused creature that is attacked automatically attacks its attacker back on its next round.

\textbf{Distracted}\index{Distracted}: If the caster is severely distracted, impeded, disturbed, bleeding, grabbed, is under attack while trying to cast a spell he must make a Magic Test.

€27658 €{Dominated}: €27659 €{Dominated} If you have a common language, you can generally force the subject to carry out commands within the limits of his capabilities. If you don't share any language, you can only give basic commands like \emph{come here}, \emph{go there}, \emph{fight} or \emph{stand still}. You are aware of what the subject is feeling but you do not receive direct sensory perceptions from him, nor can you communicate with him telepathically.

Once an order is given to the dominated creature, it continues to attempt to carry out the order to the exclusion of all other activities except those necessary for daily survival (such as eating, sleeping, and so on). Because of this limited range of activity, an Awareness check with DC 15 (rather than DC 25) can determine whether the subject's behavior has been affected by an enchantment effect.

By concentrating completely on the spell (2 Actions), you can receive sensory perceptions as interpreted by the subject's mind, even if the subject cannot communicate them anyway. You can't actually see through the subject's eyes, so it's not like you're there, but you can realize what's happening.

Clearly self-destructive orders are obviously not carried out. Once control has been established, the range within which it can be maintained is unlimited as long as both subjects remain on the same level. You don't need to see the subject to check it. If you don't spend at least 1 minute concentrating on the spell each day, the subject receives a new saving throw to free yourself from the control.

\textbf{Sleep}\index{Sleep}: Whenever a character ends a 24-hour period without at least 8 hours of sleep, he or she must succeed at a DC 12 Fortitude save or become fatigued. Each further missed rest will make him even more fatigued by accumulating the relevant penalties.
If the character stays awake for multiple days, fighting sleep becomes more difficult. After the first 24 hours, DC increases by 4 for each consecutive 24-hour period spent without 8 hours of sleep. The DC returns to 12 when the character completes a rest of at least 8 hours.

\textbf{Flanking}\index{Flanking}\index{Flanking}: a creature is flanked if it has two non-flanking opponents around it and a hypothetical line joining the opponents crosses the creature's square completely. The two creatures get +2 to attack or defense rolls.

\textbf{Unprepared / Surprised}\index{Unprepared}\index{Surprised}:
A surprised/unprepared creature has a -4 penalty to Defense and Reflex saving throws. He will not be able to react, he will not use Actions or Reactions unless explicitly permitted; from the next round you will be able to declare the initiative and act normally.

\textbf{Incapacitated}\index{Incapacitated}: An incapacitated creature can't take actions or reactions. She is flat-footed (-4 to Defense and Reflex Saving Throws)

\textbf{Grappling}:\index{Fighting} A grappling creature is restrained by a creature, trap, or effect. See \textbf{Grabbed}.

\textbf{Helpless}:\index{Helpless}\index{Asleep}\index{Unconscious}\index{Dying}\hypertarget{helpless}{}\hypertarget{dying}{} A a character who is asleep, Unconscious, Dying, or for some other reason completely at the mercy of his opponents, is considered Helpless.

A Helpless creature cannot take Actions or Reactions or speak, melee attacks against it have a +2d6 bonus. She is not aware of what is happening around her. The creature drops whatever it is holding and falls prone.

The creature automatically fails Fortitude and Reflex saving throws.

The creature loses its Dexterity bonus to Defense.

\textbf{Hanged}:\index{Hanged}\hypertarget{Hanged}{} A entangled character has difficulty moving, but can still attempt to move unless the restraining bonds are anchored to an object immobile or challenged by an opposing force.

An entangled creature treats terrain as Hard, cannot Run or Charge, and takes a –2 penalty on Defense and attack rolls.

An entangled character who tries to cast a spell is considered distracted.

\textbf{Invisible}:\index{Invisible} Invisible creatures are not perceivable by sight.
Whoever attacks a creature that is invisible to it has a -1d6 to attack roll, the invisible creature that attacks a creature that cannot see it has +1d6 to attack roll.

\textbf{Dying} \index{Dying}: A dying character has -1 Hit Points. He's defenseless. Each round he loses 1 hit point until he dies or is healed. See \textbf{Helpless}

\textbf{Nauseated}\index{Nauseated}: If the penalty is not already expressed, a nauseated creature has -1d6 on attack rolls, saving throws and skill checks.

€27694 €{Dead}: €27695 €{Dead} €27696 {Dead} {} The character's soul permanently abandons his body. Dead characters cannot benefit from normal or magical healing, and cannot be brought back to life by a spell. Only a Patron has enough power to return the soul to the body and bring the creature back to life. The School of Necromancy has spells to reanimate a body as undead.

\textbf{Paralysed}: \index{Paralysed} A paralyzed character cannot perform Actions or Reactions or speak, melee attacks against her have a +2d6 bonus. The creature is aware of its surroundings and does not drop objects. The creature automatically fails Reflex saving throws. The creature loses its Dexterity bonus to Defense.

A winged creature in flight, the moment it is paralyzed, can no longer flap its wings and falls. A paralyzed swimmer can no longer swim and may drown.

Terrain occupied by a paralyzed (or dead) creature is considered difficult terrain.

\textbf{Loss of Characteristic points}\index{Loss of Characteristic points}\index{Loss of Characteristic points}: when Characteristic scores decrease, remember to remove any Hit Points 1 per Constitution point lost per level, lower Rolls Save (Dexterity, Constitution, Wisdom), Attack Rolls (Strength and Dexterity), Defense (Defense). If not indicated as permanent, a total of 1 Ability point is recovered per day of rest.

\textbf{Petrified}: \index{Petrified}A petrified character has been turned to stone and is unconscious and \textbf{Helpless}.

If a petrified character cracks or breaks, but the broken pieces are joined to the body when it returns to flesh, the character is not hurt or damaged. If the character's petrified body is incomplete when transformed back into flesh, the body remains incomplete and may have some permanent loss of hit points and/or other impairments.

The creature has resistance to all damage. The creature is immune to poisons and diseases, but any poisons and diseases already in its system are only suspended, not neutralized. The creature has no perception of the environment or cognitive abilities.

\textbf{Fear, Frightened}:\index{Fear}\index{Frightened}\hypertarget{fear condition}{} Spells, magic items, and certain creatures can affect characters with fear. In many cases, the character must make a Will saving throw to resist the effects, and a failed save indicates that the character is frightened.

A frightened creature has -1d6 on attack rolls, saving throws, and proficiency checks as long as the source of its fear is visible. A frightened creature cannot willingly approach the source of its fear.

\textbf{Unconscious}\index{Unconscious}: He is considered to be \textbf{Helpless}.

\textbf{Prone}\index{Prone}: those who are prone have a -4 to attack and a -4 to Defense. Getting up from prone costs 2 Actions. You can't go prone if you're flying.

The player can make an Acrobatics check if he rolls 13 or more and it costs 1 Action to get up. If you fail the check you cannot take any further actions that round and remain prone.

When your Acrobatics score reaches 6, getting up from prone costs 1 Action. With Acrobatics 8 it costs an Immediate Action.

When you are prone you can crawl\index{Crawling}€27715{On all fours} or move on all fours. The terrain is considered difficult and you are still considered prone until you get up.

The terrain is considered difficult and you are still considered prone until you get up.

\textbf{Maximum Hit Points}\index{Maximum Hit Points}: A creature that suffers an attack that lowers its maximum hit points must first reduce its current maximum hit points and then reduce its current hit points by the same amount if not already removed. If the maximum hit points reach 0, the creature is dead. Maximum Hit Points are recovered at the rate of 1 per Constitution value per 8 hours of rest.

\textbf{Damage Resistance}\index{Damage Resistance}: A creature that has Damage Resistance is considered to automatically halve damage from the specified source, e.g. Damage Resistance: Sound. Damage Resistance can also be indicated with a numerical value, e.g. Damage Resistance: Fire 10. In this case the protection works on the first 10 damage suffered, in case of an effect that grants a saving throw to halve, first the amount of protection is removed from the total, then the saving throw is performed to halve the residual damage.

\textbf{Stunned/Unconscious}:\index{Stunned}\index{Unconscious} is considered to be \textbf{Helpless}. He can speak with difficulty.

\textbf{Hold your breath}: read \textbf{Drowning/Hold your breath}

\textbf{Slowed}\index{Slowed}: A slowed creature is unable to perform all 3 possible Actions in the round. Slowed is always indicated with two values, the first indicates how many fewer Actions are performed per round, the second the duration of the effect, if marked with a - then it does not have an indicated end. E.g. Slow motion 1/3r, Slow motion 2/-.

\textbf{Restricted}\index{Restricted}\hypertarget{restricted}{} : Two medium or small creatures sharing the same map square are considered restricted. Both creatures take -1d6 to attack rolls and defense rolls (-4) as long as they share the space. A creature can share the square with a creature at least three times its size without penalty.

\textbf{Broken}\index{Broken}: The broken condition has the following effects, depending on the item:

- If the item is a weapon, all attacks made with the item take a -2 penalty on attack rolls and damage rolls. Such weapons only score a Critical Hit on a 3x6 and deal a maximum of 1x additional damage.

- If the item is armor or a shield, the bonus it grants to Defense is halved, rounding up. Broken armor doubles the armor penalty on the Skill Check.

- If the item is a tool required for a Skill, all Skill checks made with it suffer a -2 penalty.

- If the item is a Wand or Staff, use double the necessary charges each time it is used.

- If the item does not fit into any of the previous categories, the broken condition has no effect on its use. Items with a broken condition, regardless of type, are worth 25\% of their normal cost. If the item is magical, it can only be repaired with the Craft spell used by a spellcaster of a level equal to or higher than the one who created the item.

\textbf{Bleeding}\index{Bleeding}\index{Bleeding}\hypertarget{bleeding}{}: A creature taking bleed damage takes the indicated amount of damage at the start of its round. Bleeding can be reduced by one point by passing a First Aid check with DC 12, 2 Actions.
For each Bleeding value above 1 the difficulty increases by 2. Cost \textbf{2 Actions}.

A 1 minute treatment grants 1 success, no check. Each critical success reduces bleeding by one additional point. Some bleed effects cause ability damage or even ability drain.

Bleeding is reduced by 1 for each Potion or Spell healing die. If the subject's hit points are brought back to the maximum value, the bleeding stops. \index{Bleeding and treatment}

Unless otherwise indicated, bleed damage stacks with a maximum of 10 hit points per round. Bleeding damage is referred to as Bleeding Value/Max Value, where Value is the bleed rating caused by the attack and Max Value is the maximum bleed rating that can be achieved.

\textbf{Vulnerability}\index{Vulnerability}: works the opposite of Resistance. The damage is doubled before any saving throw.

\end{multicols}


\pagebreak

\section{Author and creator}\index{Author}

\bigskip

\begin{center}
\textbf{Andres Zanzani} - azanzani@gmail.com
\end{center}

\bigskip

\begin{center}{\large \textbf{Contributions}}\end{center}

Federica Angeli

\bigskip

\section{Play materials}\index{Card}\index{Manual}\index{Screen}\index{Character information}

\label{scheda-e-manuale}
%{\normalsize


You are invited to download \textbf{Old Bell School System} from GitHub freely and without constraints other than those expressed in the license.
The main site is \href{https://github.com/buzzqw/TUS}{https://github.com/buzzqw/TUS}

\medskip

* \textbf{OBSS manual}:
\href{https://github.com/buzzqw/TUS/blob/master/OBSS/OBSS.pdf}{OBSS.pdf} https://github.com/buzzqw/TUS/blob/master/OBSS/OBSS .pdf

\medskip

* \textbf{Card}:
\href{https://github.com/buzzqw/TUS/blob/master/OBSS/OBSS-scheda.pdf}{OBSS-Scheda.pdf} https://github.com/buzzqw/TUS/blob/master /OBSS/OBSS-card.pdf

\smallskip

3 page version 
\href{https://github.com/buzzqw/TUS/blob/master/OBSS/OBSS-scheda-v3.pdf}{OBSS-scheda-v3.pdf} https://github.com/buzzqw/TUS /blob/master/OBSS/OBSS-scheda-v3.pdf

\medskip

* \textbf{Narrator's Screen}: 
\href{https://github.com/buzzqw/TUS/blob/master/OBSS/screen.pdf}{screen.pdf} https://github.com/buzzqw/TUS/blob/master/OBSS/screen .pdf

\medskip

* \textbf{character info}:
\href{https://github.com/buzzqw/TUS/blob/master/OBSS/OBSS-schema-narratore-personaggi.pdf}{OBSS-schema-narratore-personaggi.pdf}

https://github.com/buzzqw/TUS/blob/master/OBSS/OBSS-schema-narratore-personaggi.pdf

\medskip

* \textbf{changelog} \href{https://github.com/buzzqw/TUS/blob/master/OBSS/changelog.md}{changelog.md} url {https://github.com/buzzqw/ TUS/blob/master/OBSS/changelog.md}


\medskip

\begin{center}\index{Maps}
\textbf{Maps}:
\end{center}

All maps (beautifully prepared by Lorenzo Caputo) can be downloaded from the folder \href{https://github.com/buzzqw/TUS/tree/master/OBSS/immagini}{Images}

\medskip

\small{

\href{https://github.com/buzzqw/TUS/blob/master/OBSS/immagini/Curyan.jpg}{Curyan} https://github.com/buzzqw/TUS/blob/master/OBSS/immagini /Curyan.jpg

\href{https://github.com/buzzqw/TUS/blob/master/OBSS/immagini/Tiya.jpg}{Tiya} https://github.com/buzzqw/TUS/blob/master/OBSS/immagini /Tiya.jpg

\href{https://github.com/buzzqw/TUS/blob/master/OBSS/immagini/Mappacomplete.jpg}{Complete map} https://github.com/buzzqw/TUS/blob/master/OBSS/ images/Mappacomplete.jpg

\href{https://github.com/buzzqw/TUS/blob/master/OBSS/images/Curyan-zona0.jpg}{Partial map of Curyan} https://github.com/buzzqw/TUS/blob/ master/OBSS/images/Curyan-zona0.jpg

\href{https://github.com/buzzqw/TUS/blob/master/OBSS/immagini/Tiya-zona0.jpg}{Partial map of Tiya} https://github.com/buzzqw/TUS/blob/ master/OBSS/images/Tiya-zona0.jpg}

\medskip

Partial maps are included in the manual but of lower quality.
\bigskip

\section{Thanks}\index{Thanks}\index{EditoriFolli}

A huge thank you to \href{http://www.editorifolli.it/gdr/dnd5/srd5/}{EditoriFolli} (http://www.editorifolli.it/gdr/dnd5/srd5/) and his collaborators for the 5e SRD translation work. Without their work this manual would not have been possible.

\bigskip

Lorenzo Caputo, for his precious collaboration and patience in preparing the maps of Curyan and Tiya.

You can contact him on his Instagram page €27793 € {https://www.instagram.com/galiosjourney/} {galiosjourney}

https://www.instagram.com/galiosjourney/ or by email Frank.thegamer@outlook.com

\bigskip

Thanks to \href{https://github.com/ThomasJockin/readexpro}{Readex} (https://github.com/ThomasJockin/readexpro) for the font used. Readex is a highly readable font even for people with reading difficulties.


The font \emph{Italic} and \textbf{\emph{Italic Bold}} are taken from \href{https://dejavu-fonts.github.io/}{Dejavu} ( https://dejavu -fonts.github.io/ )

\bigskip

For any report or advice, open an issue on GitHub, or send me an email azanzani@gmail.com or contact me on the OBSS group on Telegram \href{https://t.me/obssgdr}{https://t.me/ obssgdr}

bigskip

\begin{center}
Powered by \Large\LaTeX\ {\normalsize {\&}} \Large\textbf{GitHub}
\end{center}


\vfill

\begin{changemargin}{0.3cm}{0.3cm}\begin{enfasi}{
And enjoy the game. (Players' Guide to Immortals. Frank Mentzer)
}\end{enfasi}\end{changemargin}\medskip

\thispagestyle{plain}
\begin{center}
\includepdf[pages={1,2},addtotoc={1,section,0,Character Sheet,incl:first},scale=0.85]{OBSS-sheet.pdf}
%\includepdf[pages={1,2},scale=0.85]{OBSS-scheda.pdf}
\end{center}

%\thispagestyle{plain}
%\begin{center}
%\begin{tikzpicture}[remember picture,overlay]
% \node[anchor=south west,inner sep=0pt] at ($(current page.south west)+(1cm,1cm)$) {
% \includegraphics[scale=0.93]{OBSS-scheda-0.png}
% };
%\end{tikzpicture}
%\end{center}
%\pagebreak
%\thispagestyle{plain}
%\begin{center}
%\begin{tikzpicture}[remember picture,overlay]
% \node[anchor=south west,inner sep=0pt] at ($(current page.south west)+(1cm,1cm)$) {
% \includegraphics[scale=0.93]{OBSS-scheda-1.png}
% };
%\end{tikzpicture}
%\end{center}

\pagebreak


\section{My Options}\index{My Options}

\normalsize

I am also a Storyteller and as much as I have built OBSS based on my preferences and preferences there are some Options that I like to make available to the characters, especially if they already have experience.

At my gaming table I usually choose these Options:

\begin{itemize}

\item
\hyperlink{partial success}{Partial success} pag. \pageref{partial success}

\item
\hyperlink{initiative variant}{Initiative variant}, if I have expert players. Page \pageref{initiative variant}

\item
\hyperlink{variant criticalshot}{Variant Critical Shot} and \hyperlink{variantcriticalshot}{Variant Critical Shot} are up to the player to choose whether to use them or not. To be decided in Session Zero. Page \pageref{tirocriticovariant} and \pageref{tirocriticivariante}

\item \hyperlink{OptionalCritical Shot Actions}{Critical Shot Actions} suggested for expert players, player's choice. Page \pageref{OptionalCritical Shooting Actions}

\item
\hyperlink{lunicaregola}{The Only Rule} I use it if I am the Narrator of a group of beginners. Page \pageref{the only rule}

\item
\hyperlink{optional suprememagic}{Supreme Magic} if I want to facilitate high-level spellcasters. Page \pageref{optionalsuprememagic}

\item
\hyperlink{Iconic Abilities}{Iconic Abilities} for long campaigns. Page \pageref{iconic abilities}

\item
\hyperlink{drugs}{Drugs} NO. Only in the case of groups composed of mature and adult-minded people. Page \pageref{drugs}

\end{itemize}


\vfill
{\small

\begin{multicols}{2}

\subsubsection*{Notes}

For me, OBSS should be played in a frank manner, without too many thoughts and bizarre plans. OBSS is not made to kill characters but in the same way it does not facilitate their survival, everything is up to the Narrator to decide how to play. The key to the game is in the Narrator, in the style of the players and the interest of the group, OBSS wants to offer the framework, the tools, to play the adventure.

Try to emphasize the scenes, also be theatrical in your descriptions, remove the veneer from the clean and politically correct game. It's always your world, your table and your game, try to give that immersion that is often a little lost in more modern systems.
When there is a fight, let it be one! You must hear the clang of weapons, the clash of armor, the ozone in the air caused by lightning, the crackling burns of fireballs. Let players appreciate the possibilities offered by the system and have fun trying to find the best way to perform the test.

You choose whether the characters are scoundrels just trying to survive and accumulate treasures or give a more classic or epic edge to the adventure. OBSS combines both choices, especially using one Option over another.

Create the group, and I don't just mean as a set of characters, but also as a set of players. A group where people respect and trust each other (possibly...). Build adventures that involve everyone, where everyone can make their contribution. There may be more \emph{stitched} adventures around a character but this does not \textbf{must} exclude others from participating, in the broadest term of the word, do not make the session a monologue between you and the individual player.
Take advantage of each adventure to introduce the characters to each other, nothing unites us more than the fear of dying!

Once the group is formed, and it may take time, then exploit the personal stories, clues and hypotheses created by the players to shape situations and events. Like a heavy flywheel that rotates this will continue to create situations, adventures and new plots to follow.

There may be difficulties in creating the group, unfortunately it happens. Try to talk to the player who is causing problems. Try to understand if it is his character who does not \emph{work} with the group or it is the player who has not well understood the mechanics of the group.

For this reason I always suggest you do the so-called \hyperlink{sessionzero}{Session Zero} (page \pageref{sessionzero}), where as the Narrator you will outline in broad terms what the cornerstones of the adventure are, what you expect from the characters, what are the basic moral rules to follow. There is nothing worse than a group of disjointed characters where everyone wants to do something different and doesn't care about the €27855{common goal}.

It is very important to understand what players like, each person and group wants a certain style of play and it is correct to try to please them. If the group wants political adventures, romantic dramas, try to make them find satisfaction in the adventure. If, however, they prefer fighting more, then don't skimp on battles as long as they are consistent with the adventure itself.

Make it clear that you must function as a group of players and characters in order for everyone to play better and have fun and have a better chance of survival. No player should be above the others, only the Storyteller has the final say.

Finally, always be correct, for better or for worse. There will be more unlucky sessions and others where the dice will find the right path, where the brilliant idea will save the group. Don't be the Narrator who saves \textbf{always and in any case} the characters, help can be needed every now and then especially in the most unlucky session, but respect the choices of the characters and the results of the dice. Remember that players have Fate Points to use unlike the poor monsters!. Finally, remember that the Narrator must not steal the show from the characters, but rather build the environment, drama and passions around them. There is nothing more annoying than a Narrator with delusions of protagonism, either in the management of the NPCs or in imposing choices and scenes.

And finally a truism: \textbf{have fun}, make an effort \textbf{everyone} so that the session has that mix of tension, fun and satisfaction. You are people who want to play, have fun and be together, never forget that.
\end{multicols}}

\pagebreak

\section{License}\index{License}

\medskip

\begin{center}
\textbf{Old Bell School System (OBSS)} is a product of Andres Zanzani. OBSS is licensed under the license \href{https://creativecommons.org/licenses/by-sa/4.0/legalcode}{Attribution-ShareAlike 4.0 International} - https://creativecommons.org/licenses/by-sa/ 4.0/legalcode
\end{center}

\medskip

In summary (but read the full license carefully!) you are free to:\\

\textbf{Share} — reproduce, distribute, communicate to the public, display in public, represent, perform and perform this material in any medium and format\\

\textbf{Edit} — remix, transform the material and base your works on it for any purpose, including commercial. This license is acceptable for Free Cultural Works.\\

The \textbf{licensor} cannot revoke these rights as long as you comply with the terms of the license. Under the following conditions:\\

\textbf{Attribution} — You must give appropriate credit, provide a link to the license, and indicate if changes were made. You may do so in any reasonable manner possible, but not in any way that suggests the licensor endorses you or your use of the material.\\

\textbf{Same License} — If you remix, transform, or build upon the material, you must distribute your contributions under the same license as the original material.\\

\textbf{No additional restrictions} — You may not apply legal terms or technological measures that legally restrict others from doing what the license allows them to do.\\

\textbf{Notes}:
You do not have to comply with the terms of the license for those components of the material that are in the public domain or where your use is permitted by an exception or limitation provided by law.
No guarantees are provided. The license may not give you all permissions necessary for your intended use. For example, third-party rights such as publicity rights, privacy rights and moral rights may restrict your use of the material.\\

\textbf{Author's Notes}:
Any use of OBSS or its parts or ideas please communicate in advance.
The images included in the manual, excluding the Curyan and Tiya maps, are in the public domain or are unlicensed. In case of inclusion of copyrighted images by mistake, I invite you to report them for removal.\\

\pagebreak

\invisiblesection{Maps}\index{Maps}


\begin{center}
\includegraphics[width=\linewidth]{immagini/Curyan-zona0-reduced.png}\\

\medskip

\emph{Curyan detail}
\end{center}

\begin{center}
\includegraphics[width=\linewidth]{immagini/Tiya-zona0-reduced.png}\\

\medskip

\emph{Tiya's detail}
\end{center}



\pagebreak

{\scriptsize\printindex}

%\immediate\write18{./contaobss.sh > contaobssidx.txt}
%\immediate\openin\myscriptresult=./contaobssidx.txt
%\read\myscriptresult to \ScriptResult
%\immediate\closein\myscriptresult

\vfill

%Total elements in the index \ScriptResult

\TotalBox{OBSS}

{\scriptsize\printindex[Tables]}

%\immediate\write18{./contaspell.sh > contaspell.txt}
%\immediate\openin\myscriptresult=./countspell.txt
%\read\myscriptresult to \ScriptResult
%\immediate\closein\myscriptresult

\vfill

%Total items in this index \ScriptResult

\TotalBox{Tables}



{\scriptsize\printindex[Spells]}

%\immediate\write18{./countspell.sh > countspell.txt}
%\immediate\openin\myscriptresult=./countspell.txt
%\read\myscriptresult to \ScriptResult
%\immediate\closein\myscriptresult

\vfill

%Total items in this index \ScriptResult

\TotalBox{Spells}

{\scriptsize\printindex[Magic Items]}


%\immediate\write18{./contaomagici.sh > contaomagici.txt}
%\immediate\openin\myscriptresult=./contaomagici.txt
%\read\myscriptresult to \ScriptResult
%\immediate\closein\myscriptresult

\vfill

%Total items in this index \ScriptResult

\TotalBox{Magic Items}

{\scriptsize\printindex[Monstery]}

\vfill

\TotalBox{Monstrorium}

%Total items in this index \ScriptResult

%\immediate\write18{./contamonsters.sh > contamonsters.txt}
%\immediate\openin\myscriptresult=./monstercount.txt
%\read\myscriptresult to \ScriptResult
%\immediate\closein\myscriptresult

\end{document}
