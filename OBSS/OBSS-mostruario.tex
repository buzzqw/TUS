\section{Mostruario di OBSS}\index{Mostruario}

\textbf{Arrivano i Mostri...}

\begin{changemargin}{0cm}{0.5cm}\begin{enfasi}{Chi lotta con i mostri deve guardarsi di non diventare, così facendo, un mostro. E se tu scruterai a lungo in un abisso, anche l'abisso scruterà dentro di te. (Friedrich Nietzsche)

\medskip

I mostri possono essere sconfitti soltanto dai loro simili. (Claymore)

\medskip

La tragedia dei mostri è di essere troppo grandi e potenti per essere accettati dal genere umano. (Ishiro Honda)}\end{enfasi}\end{changemargin}\medskip

\begin{multicols}{2}

\lettrine[lines=2, lhang=0.33, loversize=0.25, findent=1.5em]{B}{envenuti} in un universo ricco di nemici cattivi violenti subdoli intelligenti meschini giganteschi.. e quant'altro tu vorrai. I mostri sono il caposaldo di qualsiasi gioco di ruolo fantasy.

Vengono qui spiegati e presentati dei mostri, non certo tutti ne tanto meno esaustivi, usateli per popolare di incubi le avventure dei vostri compagni.

\medskip

\begin{center}

\includegraphics[width=0.8\linewidth]{immagini/sangiorgioedrago.png}

\textit{San Giorgio e il drago (1460 circa) di Paolo Uccello. National Gallery di Londra}
\end{center}

\subsection{Introduzione}

Un avventura non è solo un insieme di mostri ma di situazioni, di luoghi, di sorprese, insomma di tutto ciò che può affascinare, coinvolgere stupire, impegnare i giocatori. Ma anche i mostri servono. Picchiare ha un aspetto catartico, liberatorio.

Inserite nell'avventura mostri difficili e pericolosi dove serve ma ogni tanto, raramente fate sentire i giocatori potenti, fagli affrontare mostri che in pochissimi round possono risolvere. Descrivete il combattimento enfatizzando i colpi, i critici, il dolore ed il sangue dei mostri. Fate capire quanto possano essere potenti i personaggi.

Altre volte fate che i mostri incutano timore perché' sono grossi, affamati, magici e cattivi, è necessario che i giocatori abbiano paura per i loro personaggi, che non diano mai per scontato la vittoria.

La sicurezza nel descrivere la situazione, poche battute, il fissare negli occhi i giocatori. Coinvolgete i giocatori ed una volta che avranno la vostra attenzione anche i personaggi saranno più attenti. Cercate di mettere mostri coerenti all'ambiente, all'avventura, alla situazione. Non tirate a caso su tabelle, uno scontro ben organizzato da molta più soddisfazione che mostri a caso che \textit{spawnano}.

Non riducete tutto a un MMORG dove l'obiettivo è solo uccidere tutto e tutti, ci possono essere sempre tante scelte se ti impegni un pò.

\begin{changemargin}{0.3cm}{0.3cm}\begin{tcolorbox}[title = Affrontare i mostri]
{
Lascia che questo vecchio ti dia un paio di consigli giovane avventuriero!

- Non tutti i nemici si sconfiggono con la spada, molte volte serve anche una mazza!

- A volte le armi e la forza bruta non bastano. Se non hai compagni che possono lanciare incantesimi assicurati di avere sempre la possibilità di appiccare un fuoco.

- Scappa. E' sempre una opzione valida se hai modo e vedi che la situazione non promette niente di buono.

- Organizzati! non entrare nel dungeon a testa bassa senza mai fermarti tranne quando sei morto! Riposati, esplora, controlla l'ambiente e quando sei sicuro e stai meglio prosegui! anche i tuoi nemici si organizzano e si riposano intanto, stai attento!

- A volte si può anche parlare con i nemici, anche loro non vogliono morire sempre.

- Se devi uccidere fallo con cattiveria e velocità. Non perdere tempo e ottimizza i colpi, risparmia le energie e preparati immediatamente ad un altro scontro.

}\end{tcolorbox}\end{changemargin}

\subsection{Modificare le Creature}

Nonostante la variopinta collezione di mostri in questo manuale, potresti comunque trovarti in imbarazzo quando si tratta di trovare la creatura perfetta per una tua avventura. Sentiti libro di modificare le creature esistenti e trasformarle in qualcosa che ti sia più utile, magari prendendo in prestito uno o due caratteristiche da un mostro diverso.

Tieni a mente che modificare un mostro potrebbe cambiarne il grado di sfida.

\subsection{Taglia e Dadi Vita}

Un mostro può essere di taglia Minuscola, Piccola, Media, Grande, Enorme o Mastodontica e Colossale. La tabella Categorie di Taglia mostra la grandezza media di una creatura, quanto spazio occupi sulla griglia e che dado vita usi per determinare i suoi Punti Ferita.

Se non indicata la portata di una creatura dipende dalla taglia e dall'arma usata (pensate ad un gigantesco spadone brandito da un titano..)

\end{multicols}

\textbf{Tabella: Categorie di Taglia, Quadretti occupati e Portata}\index{Tabella Categorie di Taglia, Quadretti occupati e Portata}\index{Portata per creature}\index{Quadretti per creature}\index{Taglia e quadretti}\index{Creature per quadretto}

\medskip

\begin{tabularx}{0.95\textwidth}{llllll}
\toprule
\textbf{Taglia}& \textbf{Spazio} & \textbf{Esempio}&\textbf{Quadretti}&\textbf{Dado Vita}&\textbf{Portata}\\
Minuscola & 25 x 25 cm&Gatto, spiritello& 1/4&d4&0m\\
Piccola & 0,5 x 0,5 m & Goblin, cane, Gnomo&1/2&d6&0m\\
Media & 1 x 1 m & Orco, Umano, Elfo, Nano, Nibali &1&d8&1m\\
Grande & 3 x 3 m& Ogre&2x2&d10&2m\\
Enorme & 5 x 5 m & Gigante, Ent&3x3&d12&3m\\
Mastodontico & 6 x 6 m&Kraken, Drago&4x4&3d6&4m\\
Colossale & 12 x 12 m&Drago anziano, Tarrasque&6x6&2d12&6m\\
\end{tabularx}

\medskip

Le dimensioni occupate sulla griglia sono ridotte rispetto alla grandezza effettiva in quanto le miniature in commercio sono predisposte con la scala 1 quadretto=1.5metri. \textit{Quando OBSS diventerà il gioco di ruolo dominante al mondo allora avrà le sue miniature in scala!}

\begin{multicols}{2}

\subsection{Tipo}

Il tipo di un mostro si riferisce alla sua natura basilare. Certi incantesimi, oggetti magici, Abilità e altri effetti del gioco interagiscono in modi speciali con le creature di un tipo specifico. Ad esempio, una \textit{freccia ammazza draghi} infligge danni extra non solo ai draghi ma anche a tutte le altre creature del tipo drago, come i draghi tartaruga e le viverne.

Il gioco comprende i seguenti tipi di mostri:

\smallskip\textbf{Aberrazioni}, creature totalmente aliene. Molte di esse possiedono innate abilità magiche che attingono alla mente aliena della creatura anziché dalle forze mistiche del mondo. Esempi classici di aberrazioni sono aboleti, osservatori, scortica mente e i batraci del caos.

\smallskip\textbf{Bestie}, creature non umanoidi che sono una componente naturale di un mondo fantasy. Alcune possiedono poteri magici, ma la maggior parte è priva di Intelligenza e non ha alcuna forma di società o linguaggio. Esempi classici di bestie sono tutte le specie di animali comuni, i dinosauri e le versioni giganti degli animali.

\smallskip\textbf{Celestiali}, creature native dei Piani Superiori. Molti di loro sono servitori delle divinità, impiegati come messaggeri o agenti nel mondo dei mortali e per i piani.\\
I celestiali sono di natura buona, esempi classici di celestiali sono angeli, couatl e pegasi.

\smallskip\textbf{Costrutti}, sono creati e non partoriti. Alcuni sono programmati dai loro creatori per seguire una semplice serie di istruzioni, mentre altri sono senzienti e capaci di pensare per proprio conto. I golem sono i costrutti più rappresentativi.

\smallskip\textbf{Draghi}, sono grandi creature rettili di antica origine ed enorme potere. I veri draghi, compresi i buoni draghi metallici e i malvagi draghi cromatici, sono molto intelligenti e possiedono doti magiche innate. In questa categoria si collocano anche creature lontanamente imparentate con i veri draghi, ma meno potenti, meno intelligenti e meno magiche, come le viverne e gli pseudodraghi.

\smallskip\textbf{Elementali}, sono creature native dei piani elementali. Alcune creature di questo tipo sono poco più che masse animate del rispettivo elemento, e includono le creature chiamate semplicemente elementali. Altre creature possiedono forme biologiche infuse di energia elementale. Le razze dei geni, compresi djinn ed efreet, formano le civiltà più importanti dei piani elementali. Altre creature elementali sono gli azer, i persecutori invisibili e le bizzarrie d'acqua.

\smallskip\textbf{Fatati}, sono creature magiche strettamente legate alle forze della natura. Vivono in radure nascoste e foreste nebbiose. Esempi di fatati sono driadi, pixie, fate e satiri e La Topi.

\smallskip\textbf{Giganti}, troneggiano sugli umani e i loro simili. Sono di forma umana, sebbene alcuni abbiano più teste (ettin) o deformità (fomori). Le sei varianti dei veri giganti sono gigante di collina, gigante di pietra, gigante del gelo, gigante del fuoco, gigante delle nuvole, gigante delle tempeste. Oltre questi, anche ogri e troll sono giganti.

\smallskip\textbf{Immondi}, creature perverse native dei Piani Inferiori. Alcune sono al servizio di divinità, ma molte di più operano agli ordini di arcidiavoli e principi demoni. A volte sacerdoti e incantatori malvagi evocano gli immondi nel mondo materiale perché eseguano le loro volontà. Se un celestiale malvagio è una rarità, un immondo buono è praticamente inconcepibile. Gli immondi includono demoni, diavoli, segugi infernali e rakshasa.

\smallskip\textbf{Melme}, sono creature gelatinose che difficilmente hanno una forma fissa. Vivono principalmente sottoterra, stabilendosi in grotte e sotterranei, nutrendosi di rifiuti, carcasse o creature tanto sfortunate da incapparvi. I protoplasmi neri e i cubi gelatinosi sono tra le melme più riconoscibili.

\smallskip\textbf{Mostruosità}, sono mostri nel senso più stretto del termine creature spaventose che non sono comuni, né davvero naturali, e quasi mai benigne. Alcune sono il risultato di esperimenti magici andati male, mentre altri sono il prodotto di terribili maledizioni (tra cui ricordiamo il minotauro). Sfuggono a qualsiasi categorizzazione, e in qualche modo servono da categoria onnicomprensiva per quelle creature che non corrispondono a nessun altro tipo di mostro.

\smallskip\textbf{Non Morti}, sono creature un tempo vive condotte ad un orribile stato di non morte tramite la pratica della magia negromantica o qualche blasfema maledizione. Tra i non morti si annoverano cadaveri ambulanti, come vampiri e zombi, oppure spiriti incorporei, come fantasmi e spettri.

\smallskip\textbf{Piante}, in questo contesto si tratta di creature vegetali, non della normale flora. La maggior parte di esse sono mobili, e alcune sono carnivore. L'esempio più classico di piante sono i tumuli ambulanti e gli ent. Anche le creature fungoidi come le spore gassose e i miconidi rientrano in questa categoria.

\begin{center}
\includegraphics[width=0.60\linewidth]{immagini/sanmichelesatana.png}\\
\textit{San Michele sconfigge Satana. Raffaello ed aiuti (1518). Museo del Louvre}
\end{center}

\smallskip\textbf{Umanoidi}, sono la popolazione principale dei mondi di gioco, civilizzati e selvaggi, comprendono gli umani e un'ampia gamma di altre specie. Possiedono una lingua e una cultura, poche o nessuna abilità magica innata (sebbene molti umanoidi possano apprendere gli incantesimi), ed una forma bipede. Le razze più comuni di umanoide sono quelle più adatte come personaggi del giocatore: umani, nani, elfi e nibali, diversi. Quasi altrettanto numerose, ma più brutali e selvagge, e quasi tutte malvagie, sono le razze goblinoidi (goblin, hobgoblin e bugbear), orchi, gnoll, lucertoloidi e coboldi.\\

\medskip

Queste categorie possono essere a loro volta raggruppate in tipologie di Creature:
\smallskip
\begin{itemize}
\item
Le \textbf{Creature Naturali}: sono Insetti, Rettili, Bestie, Umanoidi, Piante, Creature acquatiche, Monstrusità, Melme
\item
Le \textbf{Creature Magiche} sono: Immondi, Fatati, Spiriti, Non morti, Giganti, Celestiali, Costrutti, Aberrazioni (tutto ciò che è alieno o innaturale) e Draghi.

Se una Creatura Naturale ha poteri magici allora si considera anche come Creatura Magica.
\end{itemize}


\medskip\textbf{Etichette}

Un mostro può presentare una o più etichette indicate tra parentesi, a seguire il suo tipo. Ad esempio un orco ha il tipo \textit{umanoide (orco)}. Le etichette tra parentesi forniscono ulteriori categorizzazioni per determinate creature. Le etichette non hanno delle proprie regole specifiche, ma alcuni elementi del gioco, come gli oggetti magici, vi possono fare riferimento. Ad esempio, una lancia particolarmente efficace contro i demoni, funzionerebbe contro qualsiasi mostro che abbia l'etichetta demone.

\subsection{Tratti}

I mostri non presentano l'elenco dettagliato dei Tratti, troverete solo l'indicazione sugli assi del Chaos, Legge, Bene e Male. Ricordatevi che sono indicazioni, le eccezioni possono capitare specialmente nelle specie più intelligenti.
Determinate creature sono \textbf{disallineate}, ovvero non hanno una condotta morale o etica.

\subsection{Difesa}

Un mostro che indossa un'armatura o trasporta uno scudo ha una Difesa che tiene conto dell'armatura, dello scudo e della Destrezza. Altrimenti, la Difesa di un mostro è basata sul suo valore di Destrezza e l'armatura naturale se la possiede. Se un mostro possiede un'armatura naturale, indossa armature o trasporta uno scudo, viene indicato tra parentesi dopo il valore della sua Difesa.

Qualora il mostro fosse \textbf{colto di sorpresa} sottraete alla Difesa il valore di Destrezza e Scudo se presente.

Se il mostro viene colpito con un \textbf{effetto a tocco} (Difesa a tocco) sottraete alla Difesa il valore dell'armatura e scudo.

\subsection{Punti Ferita}

Di solito quando scende a 0 Punti Ferita, un mostro muore o viene distrutto.

I Punti Ferita di un mostro sono presentati sia come un insieme di dadi che come valore medio. Ad esempio, un mostro con 2d8 Punti Ferita ha di media 9 Punti Ferita (2 x 4,5).

Capiterà che i giocatori vi chiedano \textbf{\textit{come sta il mostro}}, vi suggerisco di non scendere mai nei dettagli dicendo quanti PF ha in tutto o ne ha persi, bensì rimanere in questi gradi:  Non ferito (PF pieni), Ferito (30\% PF subiti), Gravemente ferito (almeno 50\% PF subiti), ovvero dare una descrizione generica dello stato. \index{Come sta il mostro}\index{Chiedere PF del Mostro}

Anche il valore di Costituzione di un mostro influenza il numero di Punti Ferita che possiede. Il suo valore di Costituzione viene moltiplicato per il numero di Dadi Vita che possiede e il risultato viene sommato ai suoi Punti Ferita. Ad esempio, un mostro che ha Costituzione 1 e 2d8 Dadi Vita, e avrà quindi 2d8+2 Punti Ferita (media 10).

\begin{center}
	\includegraphics[width=0.65\linewidth]{immagini/roc.png}\\
	\textit{Henry Justice Ford}
\end{center}

\subsection{Movimento}

Il Movimento di un mostro ti dice di quanto si possa muovere durante il suo round per Azione di Movimento

Tutte le creature possiedono un movimento di passeggio, detto semplicemente movimento del mostro. Le creature che non possiedono alcuna forma di spostamento terreno hanno velocità di passeggio 0 metri.

Alcune creature possiedono uno o più dei seguenti modi di movimento aggiuntivi.

\smallskip\textbf{Nuoto}

Un mostro che possiede una velocità di nuoto non deve spendere movimento extra per nuotare (non è terreno difficile)

\smallskip\textbf{Scalata}

Un mostro che possiede una velocità di scalata può usare tutto o solo parte del suo movimento per muoversi su superfici verticali. Il mostro non deve spendere movimento extra (x4) per scalare.

\smallskip\textbf{Scavo}

Un mostro che possiede una velocità di scavo può usare la sua velocità per attraversare sabbia, terra, fango, ecc. Un mostro non può scavare attraverso la roccia solida a meno che non possieda un tratto speciale che glielo permetta.

\smallskip\textbf{Volo}

Un mostro che possiede una velocità di volo può usare tutto o solo parte del suo movimento per volare. Alcuni mostri hanno l'abilità di \textbf{fluttuare}, che li rende difficili da abbattere. Il mostro smette di fluttuare quando muore.

\subsection{Punteggi di Caratteristica}

Ogni mostro possiede sei punteggi di caratteristica (Forza, Destrezza, Costituzione, Intelligenza, Saggezza, Carisma)

\subsection{Competenze}

La voce Competenze è riservata a quei mostri che sono capaci in una o più competenze. Ad esempio, un mostro che è molto attento e furtivo potrebbe avere bonus alle prove di Saggezza (Consapevolezza) e Destrezza.

Si possono applicare anche altri modificatori, ad esempio, un mostro potrebbe avere un bonus più grande del previsto per tenere conto della sua grande perizia.


\subsection{Vulnerabilità, Resistenze e Immunità}\index{Equivalenze armi}\index{Pugni magici}
Alcune creature possiedono vulnerabilità, resistenze o immunità ad un certo tipo di danno. Creature particolari sono addirittura resistenti o immuni agli attacchi non magici (un attacco magico è un attacco sferrato tramite un incantesimo, un oggetto magico, o un'altra fonte di magia).

E' anche possibile che sia indicato uno specifico bonus magico minimo per poter danneggiare la creatura.

Inoltre, certe creature sono immuni a determinate condizioni. Se un mostro è immune ad un effetto di gioco che non viene considerato danno o condizione, possiede invece un tratto speciale.

Nella tabella sottostante viene indicato quale incantamento magico dell'arma è necessario per superare l'immunità indicata. E' anche indicato il livello minimo di attacco naturale  (Competenza Armi) nel caso si colpisca con calci e pugni.

In caso di personaggio con Lista d'Armi \textbf{Pugno Vuoto} si controlla quanto volte si è presa la lista.

\medskip

\textbf{Tabella: Equivalenza Armi Magiche}\index{Tabella Equivalenza Armi Magiche}\hypertarget{equivalenzemagiche}{}\label{equivalenzaarmimagiche}

\medskip

\begin{tabular}{lp{0.055\textwidth}p{0.06\textwidth}p{0.07\textwidth}}
\toprule
\textbf{Immunità} & \textbf{Magia Arma} & \textbf{Attacco Nat.}& \textbf{Pugno Vuoto}\\
+1         & +1     &  3&  2\\
+2         & +2     &  6&  4\\
Ferro Freddo & +1     &  4&  2\\
Argento  & +1     &  4&  2\\
Adamantio       & +2     &  6&  4\\
+3      & +3     &  12& 6\\
+4         & +4     &  16& 8\\
+5         & +5     & - &  8\\
\end{tabular}

\subsection{Sensi}

La voce Sensi elenca qualsiasi senso speciale di cui il mostro sia in possesso. I sensi speciali sono descritti di seguito. Se non è presente la voce Sensi, la creatura ha dei sensi standard (visione, olfatto, gusto, tatto...)

\subsubsection{Percezione Tellurica}

Un mostro con percezione tellurica può individuare e trovare le origini delle vibrazioni entro uno specifico raggio, purché il mostro e la fonte della vibrazione siano in contatto con lo stesso terreno o sostanza. La percezione tellurica non può essere impiegata per individuare creature volanti o incorporee. Molte creature scavatrici, come gli ankheg e i colossi di terra, possiedono questo senso speciale.

\subsubsection{Visione Crepuscolare o Scurovisione}

Una creatura con Visione Crepuscolare può vedere nella più tenue delle luci, ma non nell'oscurità completa a differenza di quelle che possiedono scurovisione. Molte creature che vivono sottoterra possiedono questo senso speciale.  Vedi capitolo \hyperlink{visioneeluce}{Caratteristiche Speciali}.

\subsubsection{Visione del Vero}

Un mostro con la visione del vero può, fino ad una specifica gittata, vedere attraverso l'oscurità normale e magica, vedere creature e oggetti invisibili, automaticamente individuare le illusioni e riuscire i Tiri Salvezza contro di loro, eercepire la forma originale di un mutaforma o di una creatura trasformata dalla magia. Inoltre, la creatura può vedere nel Piano Etereo fino alla stessa gittata.

\subsubsection{Vista Cieca}

Una creatura con vista cieca può percepire l'ambiente circostante, senza fare affidamento alla vista, fino ad una specifica gittata.

Le creature senza occhi come i grimlock e le melme e le creature con ecolocazione o sensi potenziati, come i pipistrelli ed i draghi, possiedono questo senso.

Se un mostro è cieco di natura, la cosa viene annotata tra parentesi, ad indicare che la portata della sua vista cieca definisce anche la portata massima della sua percezione.

\begin{center}
\includegraphics[width=0.65\linewidth]{immagini/ciclope.png}

\textit{Henry Justice Ford}
\end{center}

\subsection{Linguaggi}

Le lingue che un mostro può parlare sono riportate in ordine alfabetico. A volte un mostro può capire una lingua ma non parlarla, e la cosa viene indicata in questa voce. Se un mostro non ha la nota Linguaggi significa che non conosce linguaggi diversi dalla propria lingua (se applicabile).

\subsection{Telepatia}

La telepatia è un'abilità che permette ad un mostro di comunicare mentalmente con un'altra creatura nel raggio di azione specificato. La creatura contattata non è necessario che parli la stessa lingua del mostro per comunicare in questo modo. Una creatura senza telepatia può ricevere e rispondere a messaggi telepatici ma non può iniziare o terminare una conversazione telepatica.

Un mostro telepatico non ha bisogno di vedere la creatura contattata e può terminare il contatto telepatico in qualsiasi momento. Il contatto è infranto non appena le due creature non si trovano più entro il raggio di azione o se il mostro telepatico contatta un'altra creatura a gittata. Un mostro telepatico può iniziare o terminare una conversazione telepatica senza dover usare un'azione, ma mentre il mostro è inabile, non può dare inizio ad un contatto telepatico, e qualsiasi contatto in corso viene terminato. Per avviare una comunicazione telepatica l'obiettivo deve essere stato almeno individuato.

Una creatura nell'area di un \textit{campo anti-magia} o in qualsiasi altro posto in cui la magia non funziona può inviare o ricevere messaggi telepatici.

\subsection{Sfida}

Il \textbf{grado di sfida} (GS) di un mostro vi dice quanto sia grande la minaccia che pone. Una compagnia di quattro avventurieri equipaggiata in maniera appropriata e riposata deve essere in grado di sconfiggere un mostro dal grado di sfida pari al proprio livello medio senza subire perdite. Ad esempio, una compagnia di quattro personaggi di 3° livello dovrebbe ritenere un mostro di grado di sfida 3 una degna sfida, ma non letale.

I mostri che sono significativamente più deboli dei personaggi di 1° livello hanno un grado di sfida inferiore ad 1. I mostri con un grado di sfida 0 non presentano problemi eccetto in grandi numeri; quelli privi di reali attacchi non valgono punti esperienza.

Alcuni mostri presentano una sfida superiore a quelle che anche una compagnia di 20° livello sia in grado di gestire. Questi mostri hanno grado di sfida 21 o superiore e sono progettati proprio per mettere alla prova le capacità dei personaggi.

\subsection{Tratti Speciali}

I tratti speciali (che compaiono dopo il grado di sfida di un mostro ma prima di qualsiasi azione o reazione) sono peculiarità che avranno probabilmente un ruolo in un incontro di combattimento e che richiedono delle spiegazioni.

\begin{center}
	\includegraphics[width=0.7\linewidth]{immagini/lich2.png}

	\textit{Lich - Battle of Wesnoth}

\end{center}

\subsection{Incantesimi}

Un mostro con il privilegio Incantesimi è in grado di lanciare Incantesimi.

Un mostro può lanciare un incantesimo dal suo elenco di incantesimi senza effettuare la Prova di Magia e senza la possibilità di effettuare tiri critici o meno. La DC è 10 + incantesimo x2 + Intelligenza o Saggezza a seconda della caratteristica migliore oppure indicata. Un mostro con incantesimi non può convertire i Punti Magia di incantesimo di livello superiore in incantesimi di livello inferiore, tranne se ha un valore di Competenza Magica (es. Lich, Mummia, Naga...).

\subsection{Incantesimi Innati}

Un mostro con l'abilità innata di lanciare incantesimi ha il tratto speciale Incantesimi.
Gli incantesimi innati di un mostro non possono essere scambiati con altri incantesimi.

Un mostro non esegue mai volontariamente la Prova di Magia tranne se ha Intelligenza maggiore di 1.


\subsection{Azioni}

Anche i mostri agiscono secondo lo schema delle 3 Azioni disponibile per round. Possono essere segnate abilità e capacità che gli permettono di eseguire un numero più elevato di Azioni.

Quando un mostro svolge le sue azioni, può scegliere tra le opzioni della sezione Azioni del suo blocco statistiche o impiegare una delle azioni disponibili a tutte le creature, come Scattare o Nascondersi.

\subsubsection{Attacchi da Mischia e a Distanza}

L'azione più comune che un mostro effettuerà in combattimento, sarà un attacco da mischia o a distanza. Possono essere attacchi con incantesimi o attacchi con armi, dove l'arma può essere un manufatto o un'arma naturale, come gli artigli o la coda chiodata.

\textit{\textbf{Creatura contro Bersaglio}.} Il bersaglio di un attacco da mischia o a distanza è di solito una creatura o un bersaglio.

\textbf{Portata}: la portata indicata è la distanza \textbf{entro} quanti metri la creatura può colpire l'avversario. Anche se la portata è superiore a quella dell'avversario non si considerano vantaggi tipo Arma Lunga (+2 al TC) con gli attacchi naturali. Una creatura con portata 0 deve esserti addosso per colpirti, solitamente hanno portata 0 le creature estremamente piccole.

\textit{\textbf{Colpisce.}} Qualsiasi danno inflitto o altro effetto che avviene come risultato di un attacco che colpisce il bersaglio viene descritto nell'annotazione ``\textit{Colpisce}''. Puoi scegliere se prendere il danno medio o tirare i dadi; per questo motivo vengono presentati sia il danno medio che una formula di dadi.

Suggerisco che per i nemici sia comunque applicabile il Tiro Critico mentre l'Esplosione del Danno è da usarsi se si vuole una campagna più letale.\index{Danno Critico Mostri}

\textbf{\textit{Manca}.} Se un attacco ha un effetto prodotto da un colpo a vuoto, quell'informazione viene fornita dall'annotazione ``\textit{Manca}''.

\textit{\textbf{Danni.}} Se un mostro impugna armi manufatte, infligge danni appropriati all'arma. I mostri più grossi di solito impugnano armi di dimensioni superiori che infliggono danni extra quando colpiscono. Se usano questo tipo di armi il danno è già segnato, altrimenti se raccolgono o usano un arma non prevista raddoppiare i dadi dell'arma se la creatura è Grande, triplicarli se Enorme e quadruplicarli se Mastodontica qualora usino armi della loro taglia.

Una creatura ha -1d6 ai tiri per colpire con un'arma costruita per una taglia superiore alla sua. Il Narratore può decidere che le armi di due o più taglie più grandi di quella dell'attaccante sono del tutto impossibili da usare.

\begin{changemargin}{0.3cm}{0.3cm}\begin{narratore}
Una creatura che abbia almeno Grado di Sfida 6 a discrezione del Narratore può sferrare un attacco di Opportunità (vedi \hyperlink{opportunista}{Opportunista})\\

Per valorizzare i mostri e renderli più incisivi potete decidere che ogni mostro abbia una  \hyperlink{riduzionedeldanno}{Riduzione del Danno} pari a metà del suo Grado di Sfida ( $\frac{GS}{2}$/- )

\end{narratore}\end{changemargin}

\subsubsection{Multiattacco}

Una creatura che può effettuare più attacchi durante il suo round ha l'abilità Multiattacco. L'Azione Multiattacco consuma 2 Azioni anche se porta più di 2 attacchi.

\subsubsection{Regole dell'Afferrare per i Mostri}

Molti mostri possiedono un attacco speciale che gli permette di afferrare rapidamente la preda. Quando un mostro colpisce con un simile attacco, non deve effettuare un'ulteriore prova di caratteristica per determinare se l'afferrare riesce, a meno che l'attacco non dica altrimenti.

Una creatura afferrata dal mostro può usare un azione per tentare di sfuggirgli. Per farlo, deve riuscire una prova di Forza contrapposta (TS Tempra con bonus Forza) contro la DC di fuga nel blocco statistiche del mostro. Se non viene fornita una DC di fuga, assumere che la DC sia uguale a al bonus del Tiro Salvezza su Tempra + Forza del mostro.

\subsubsection{Munizioni}

Un mostro porta con sé munizioni sufficienti per effettuare i suoi attacchi a distanza. Puoi presumere che un mostro abbia 2d4 proiettili per un attacco con armi da lancio (giavellotti, macigni..), e 2d10 proiettili per un'arma a proiettili come un arco o una balestra.

\subsubsection{Reazioni}

Se un mostro può compiere qualcosa di speciale con le sue reazioni, è riportato qui. Se una creatura non ha reazioni speciali, questa sezione è assente.


\subsubsection{Uso Limitato}

Alcune abilità speciali hanno restrizioni sul numero di volte che possono essere usate.

\textbf{\textit{X/Giorno}.} L'annotazione ``X/Giorno'' indica un'abilità speciale che può essere usata X volte prima che il sorga l'alba per recuperare gli usi consumati. Ad esempio, ``1/Giorno'' indica un'abilità speciale che può essere usata una volta prima che il mostro debba aspettare la nuova alba.

\textit{\textbf{Ricarica X-Y.}} L'annotazione ``Ricarica X-Y'' indica che il mostro può usare un'abilità speciale una volta e che l'abilità ha una probabilità casuale di ricaricarsi ogni round seguente di combattimento. All'inizio di ciascun round del mostro, tira un d6. Se il risultato è uno dei numeri dell'annotazione di ricarica, il mostro recupera l'uso dell'abilità speciale. L'abilità si ricarica anche all'alba di un nuovo giorno.

%\begin{center}
%\includegraphics[width=0.6\linewidth]{immagini/cupido.png}
%
%\textit{Eros con il suo arco. Musei Capitolini}
%\end{center}

Ad esempio, "Ricarica 5-6" indica che un mostro può usare la sua abilità speciale una volta. Poi, all'inizio del round del mostro, recupera l'uso dell'abilità se tira 5 o 6 su di un d6.

\begin{center}
\includegraphics[width=0.55\linewidth]{immagini/polpo.png}

\textit{Alphonse de Neuville - Hetzel edition of 20000 Lieues Sous les Mers}
\end{center}


\subsection{Equipaggiamento}

Il blocco statistiche si riferisce all'equipaggiamento, oltre le armi o le armature utilizzate dal mostro. Una creatura che normalmente indossa abiti, come un umanoide, si assume sia abbigliato in maniera appropriata.

Puoi equipaggiare i mostri con ulteriore equipaggiamento come preferisci, utilizzando il capitolo \hyperlink{equipaggiamento}{Equipaggiamento} come fonte di ispirazione, e sei tu a decidere quanto dell'equipaggiamento del mostro è recuperabile dopo che la creatura è stata uccisa o se qualsiasi parte del suo equipaggiamento sia ancora utilizzabile. Ad esempio, un'armatura ammaccata fatta per un mostro difficilmente sarà utilizzabile da qualcun altro. Se un mostro incantatore necessita di componenti materiali per lanciare i suoi incantesimi, dai per scontato che abbia le componenti materiali per lanciare gli incantesimi nel suo blocco statistiche.

\subsection{Azioni Aggiuntive}

Certe creature possono possono eseguire azioni speciali al di fuori del proprio round, ed alcune possono estendere il proprio potere all'ambiente, provocando l'avvenimento di effetti magici straordinari nelle loro vicinanze.

Una creatura con azioni aggiuntive può effettuare un certo numero di azioni speciali -- dette azioni aggiuntive -- al di fuori del suo round. Solo un'azione aggiuntiva può essere usata alla volta e solo al termine del round di un'altra creatura. Una creatura con azioni aggiuntive recupera all'inizio del suo round le azioni aggiuntive che ha usato. Non è obbligata ad usare le sue azioni aggiuntive e non può usare le azioni aggiuntive mentre è inabile o altrimenti incapace di effettuare azioni. Se sorpresa, non può usarle fin dopo il suo primo round di combattimento.

Se una creatura assume la forma di una creatura con azioni aggiuntive, magari tramite un incantesimo, non ne ottiene però le azioni aggiuntive, le azioni da tana.

\subsubsection{La Tana di una Creatura}

Una creatura con azioni aggiuntive può presentare una sezione che ne descrive la tana e gli effetti speciali che vi può creare mentre si trova lì, o per propria volontà o semplicemente grazie alla sua presenza. Questa sezione si applica solo alle creature leggendarie che trascorrono molto tempo nelle loro tane ed è altamente probabile che li vi vengano incontrate.

\subsubsection{Azioni da Tana}

Se una creatura con azioni aggiuntive ha un'azione da tana, può usarla per imbrigliare la magia ambientale della sua tana. Al conteggio di iniziativa 10, perdendo i pareggi, la creatura può usare una delle sue opzioni di azioni da tana. Non può farlo mentre è inabile o altrimenti incapace di effettuare azioni. Se sorpresa, non può farne uso fino a dopo il suo primo round di combattimento.

\subsubsection{Tipologie di Tesoro}

Ogni tipologia di creatura può preferire un tipo di tesoro (inteso come oggetti, monete, gemme...) diverso. Questi sono solo suggerimenti su come costruire il tesoro del mostro.

\medskip

\begin{itemize}

\item \textbf{Aberrazione}
Molte aberrazioni hanno scarsa considerazione per i tesori, possedendo solo quel che prendono dai resti delle loro vittime precedenti. Altre sono avversari astuti che usano vari oggetti magici e tesori per potenziare le loro capacità.

\item \textbf{Animale}
Gli animali non si curano affatto dei tesori, lasciando invece monete e oggetti con i resti dei loro pasti. Per quelli con un tesoro, quest'ultimo in genere si trova nelle loro tane, sparso tra le ossa e gli altri scarti.

\item \textbf{Bestia Magica}
Curandosi poco dei valori, la maggior parte delle bestie magiche è unicamente in cerca del suo prossimo pasto. I nascondigli di queste creature sono spesso disseminati di ninnoli preziosi e oggetti magici.

\item \textbf{Costrutto}
Il solo tesoro portato dai costrutti generalmente è parte del costrutto stesso, come un'arma o un oggetto magico. I costrutti, però, vengono tipicamente usati per sorvegliare tesori o oggetti magici di maggior valore.

\item \textbf{Drago}
Noti per i loro preziosi tesori, i draghi spesso rimuginano su pile di monete, gemme, oggetti magici e altri oggetti costosi.

\item \textbf{Esterno}
Gli esterni sono tra i tipi di creature più vari e di conseguenza potrebbero avere davvero qualsiasi tipo di tesoro su di loro o nascosto nei loro rifugi. Il Narratore dovrebbe considerare la singola creatura quando determina il tipo di tesoro che più si adatta a quell'esterno.

\item \textbf{Folletto}
Sopra ogni altra cosa, i folletti danno valore agli oggetti belli e magici. Hanno scarsa considerazione per gli strumenti di scambio e commercio usati dalle razze più civilizzate, come monete e valori.

\item \textbf{Melma}
Le melme non concepiscono cose come i tesori e lasciano tutto quel che trovano nella loro ricerca del prossimo pasto. Qualsiasi tesoro possano trasportare è completamente accidentale.

\item \textbf{Non Morto}
I tesori trasportati dai non morti variano a seconda che si tratti o meno di una creatura intelligente. I non morti privi di intelletto tipicamente hanno solo i miseri valori che portavano con sé in vita, raramente davvero utilizzabili come tesori, mentre quelli intelligenti sfruttano una vasta gamma di oggetti magici per distruggere i viventi.

\item \textbf{Parassita}
Come le altre creature senza mente, i parassiti non bramano tesori, sebbene queste creature talvolta si trovino a infestare aree dove sono custoditi dei valori.

\item \textbf{Umanoide}
Le creature di questo tipo sono molto variegate, ma persino gli umanoidi più primitivi usano equipaggiamento e oggetti magici in qualche misura. In gruppi più grandi, come le comunità, gli umanoidi spesso possiedono una grande quantità di tesori che custodiscono collettivamente.

\item \textbf{Vegetale}
Come gli animali, le creature vegetali non danno peso ai tesori, e tutto ciò che si potrebbe trovare dove crescono rappresenta semplicemente i resti non digeriti di una vittima precedente.

\end{itemize}

\end{multicols}

\pagebreak

\subsection{I Mostri}


\begin{changemargin}{0.3cm}{0.3cm}\begin{narratore}
Le creature qui presentate vogliono essere un esempio, corposo, dei nemici che i tuoi amici potrebbero incontrare. Attenzione, non è detto che siano tutti nemici o per forza che abbiano intenzione negative.

Creature più civilizzate avranno una loro condotta etica e morale individuale, anche all'interno di uno stesso gruppo di "nemici" c'è chi potrebbe essere più "nemico" o semplicemente indifferente.

Sfrutta le peculiarità e unicità delle creature per creare incontri non scontati e sfidanti dal punto di vista tattico. Non essere scontato ma neanche assurdo nelle scelte, deve sempre esserci della coerenza nel scegliere le creature.
\end{narratore}\end{changemargin}


\bigskip

\begin{changemargin}{0cm}{0.5cm}\begin{enfasi}{

Amon Goth: Il controllo è potere. Questo è il potere.

Oskar Schindler: è per questo che ci temono?

Amon Goth: Abbiamo il potere di uccidere. Per questo ci temono.

Oskar Schindler: Ci temono perché abbiamo il potere di uccidere arbitrariamente. Un uomo commette un reato, doveva pensarci, lo facciamo uccidere e ci sentiamo in pace. O lo uccidiamo noi stessi e ci sentiamo ancora meglio. Questo non è il potere però! Questa è giustizia, è una cosa diversa dal potere. Il potere è quando abbiamo ogni giustificazione per uccidere e non lo facciamo.

Amon Goth: è questo il potere?

Oskar Schindler: L'avevano gli imperatori questo. Un uomo ruba qualcosa, viene portato davanti all'imperatore e si lascia cadere per terra tremante, implora per avere pietà. E' conscio che sta per andarsene. E l'imperatore lo perdona, invece. Quell'uomo, immeritevole, lo lascia libero.

(Schindler's list - La lista di Schindler, Film, 1993)
}\end{enfasi}\end{changemargin}\medskip


\begin{changemargin}{0cm}{0.5cm}\begin{enfasi}{
Io sono il mostro che gli uomini che respirano bramerebbero uccidere. Io sono Dracula. (Dracula di Bram Stoker)}\end{enfasi}\end{changemargin}\medskip


\bigskip

\begin{multicols}{2}

\medskip\index[Mostruario]{Aboleth}\textbf{Aboleth}

\textit{Grande aberrazione, legale malvagio}

\textbf{FORZA} +5

\textbf{DESTREZZA} -1

\textbf{COSTITUZIONE} +2

\textbf{INTELLIGENZA} +4

\textbf{SAGGEZZA} +2

\textbf{CARISMA} +4

\textbf{Iniziativa} +4 -- \textbf{Difesa} 22

\textbf{Punti Ferita} 135 (18d10 + 36)

\textbf{Movimento} 3 m, nuoto 12 m

\textbf{Tiri Salvezza} Tempra +8, Riflessi +5, Volontà +11

\textbf{Competenze} Consapevolezza +10, Storia +12

\textbf{Sensi} scurovisione 36 m

\textbf{Linguaggi} Linguaggio delle Profondità, telepatia 36 m

\textbf{Sfida} 10 (5.900 PX)

\textit{\textbf{Anfibio.}} L'aboleth può respirare aria e acqua.

\textit{\textbf{Nube di Muco.}} Mentre è sott'acqua, l'aboleth è avvolto da muco mutante. Una creatura che entri a contatto con l'aboleth, o che lo colpisca con un attacco da mischia mentre si trova entro 1 metro da esso, deve effettuare un Tiro Salvezza di Tempra DC 14. Se lo fallisce, la creatura resta ammalata per 1d4 ore. La creatura ammalata può respirare solo sott'acqua.

\textit{\textbf{Sonda Telepatica.}} Se una creatura comunica telepaticamente con l'aboleth, e l'aboleth può vederla, l'aboleth ne apprende i più grandi desideri.

\textbf{Azioni}

\textit{\textbf{Multiattacco.}} L'aboleth effettua tre attacchi con i tentacoli

\textit{\textbf{Tentacolo.} Attacco con arma da mischia}: +9 a colpire, portata 3 m, un bersaglio.

\textit{Colpisce:} 12 (2d6 + 5) danni da botta. Se il bersaglio è una creatura, deve riuscire un Tiro Salvezza di Tempra DC 14 o divenire ammalato. La malattia non produce alcun effetto per 1 minuto e può essere rimossa da qualsiasi magia che curi le malattie. Dopo 1 minuto, la pelle della creatura ammalata diventa trasparente e viscida, la creatura non può recuperare Punti Ferita a meno che non sia sott'acqua, e la malattia può essere rimossa solo da \textit{guarire} o un altro incantesimo cura malattie di livello 3 o più. Quando la creatura si trova al di fuori di un corpo d'acqua, subisce 6 (1d12) danni da acido ogni 10 minuti a meno che la sua pelle non venga bagnata prima che siano passati questi 10 minuti.

\textit{\textbf{Coda.} Attacco con arma da mischia}: +9 a colpire, portata 3 m, un bersaglio.

\textit{Colpisce:} 15 (3d6 + 5) danni da botta.

\textit{\textbf{Schiavizzare (3/Giorno).}} L'aboleth prende a bersaglio una creatura che può vedere entro 9 metri da esso. Il bersaglio deve riuscire un Tiro Salvezza di Volontà DC 14 o restare affascinato magicamente dall'aboleth finché l'aboleth muore o i due si trovano su piani di esistenza differenti. Il bersaglio affascinato è sotto il controllo dell'aboleth e non può effettuare reazioni. L'aboleth e il bersaglio possono comunicare telepaticamente tra di loro a qualsiasi distanza.

Ogniqualvolta il bersaglio affascinato subisce danni, può ripetere il Tiro Salvezza. Se lo riesce, l'effetto termina. Non più di una volta ogni 24 ore, può ripetere il Tiro Salvezza quando si trova almeno a 1,5 chilometri di distanza dall'aboleth.

\textbf{Azioni Aggiuntive}

L'aboleth può effettuare 3 Azioni aggiuntive, scelte tra le opzioni seguenti. Può usare solo un'opzione leggendaria alla volta e solo al termine del turno di un'altra creatura. L'aboleth recupera le Azioni aggiuntive spese all'inizio del proprio round.

\textbf{Individuare.} L'aboleth effettua una prova di Saggezza (Consapevolezza).

\textbf{Risucchio Psichico (Costa 2 Azioni).} Una creatura affascinata dall'aboleth subisce 10 (3d6) danni e l'aboleth recupera un numero di Punti Ferita pari al danno subito dalla creatura.

\textbf{Spazzata di Coda.} L'aboleth effettua un attacco di coda.

\textbf{Ecologia}\\
Ambiente: Qualsiasi Acquatico\\
Organizzazione: Solitario, coppia, nidiata (3-6) o branco (7-19)\\
\textbf{Tesoro}: Doppio\\
\textbf{Descrizione}\\
Come suggerisce il loro aspetto primitivo, gli ermafroditi aboleth sono fra le più antiche forme di vita al mondo. Già antichi quando gli dei cominciarono ad interessarsi del Piano Materiale, gli aboleth hanno sempre vissuto lontani dagli altri mortali: sono alieni, freddi e sempre occupati ad intessere piani. Un tempo governavano il mondo in un vasto impero, ed oggi vedono le altre forme di vita come cibo o schiavi... a volte entrambe le cose assieme. Disprezzano gli dei, poiché ritengono di essere loro i veri signori del creato, un aboleth è lungo 7 metri e pesa circa 3,2 tonnellate. Nelle più oscure profondità del mare, gli aboleth abitano ancora nelle loro grottesche città, ciclopiche e nauseabonde. Sono serviti da innumerevoli schiavi presi da ogni nazione, sia terrestre che marina, e quelli terrestri sono doppiamente schiavi dei loro padroni e del loro muco, che permette loro di respirare sott'acqua, gli aboleth incontrati da soli sono di solito esploratori provenienti da queste città nascoste, in cerca di nuovi schiavi.

\subsection{Angeli}

\medskip\index[Mostruario]{Angelo Deva}\textbf{Angelo Deva}

\textit{Medio celestiale, legale buono}

\textbf{FORZA} +4

\textbf{DESTREZZA} +4

\textbf{COSTITUZIONE} +4

\textbf{INTELLIGENZA} +3

\textbf{SAGGEZZA} +5

\textbf{CARISMA} +5

\textbf{Iniziativa} +4 -- \textbf{Difesa} 22

\textbf{Punti Ferita} 136 (16d8 + 64)

\textbf{Movimento} 9 m, volo 27 m

\textbf{Tiri Salvezza} Tempra +16, Riflessi +13, Volontà +11

\textbf{Competenze} Percepire Emozioni +9, Consapevolezza +9

\textbf{Resistenze ai Danni} da Luce; da arma non magica

\textbf{Immunità alle Condizioni} affascinato, affaticamento, spaventato

\textbf{Sensi} scurovisione 36 m

\textbf{Linguaggi} tutte, telepatia 36 m

\textbf{Sfida} 10 (5.900 PX)

\textit{\textbf{Armi Angeliche.}} Gli attacchi con arma del deva sono magici. Quando il deva colpisce con qualsiasi arma, l'arma infligge 4d8 danni da Luce aggiuntivi (già compresi nell'attacco).

\textit{\textbf{Incantesimi Innati.}} La caratteristica da incantatore innato del deva è il Carisma (DC 17 per i Tiri Salvezza degli incantesimi). Il deva può lanciare in maniera innata i seguenti incantesimi, con l'uso delle sole componenti verbali:

A volontà: \textit{individuazione del bene e del male}

1/giorno: \textit{comunione, rianimare morti}

\textit{\textbf{Resistenza alla Magia.}} Il deva ha +1d6 ai Tiri Salvezza contro incantesimi e altri effetti magici.

\textbf{Azioni}

\textit{\textbf{Multiattacco.}} Il deva effettua due attacchi da mischia.

\textit{\textbf{Mazza.} Attacco con arma da mischia}: +19 a colpire, portata 1 m, un bersaglio.

\textit{Colpisce:} 7 (1d6 + 4) danni da botta più 18 (4d8) danni da Luce.

\textit{\textbf{Tocco Guaritore (3/Giorno).}} Il deva entra a contatto con un'altra creatura. Il bersaglio recupera magicamente 20 (4d8 + 2) Punti Ferita ed è libero da qualsiasi cecità, malattia, maledizione, sordità o veleno.

\textit{\textbf{Mutare Forma.}} Il deva può trasformarsi magicamente in un umanoide o bestia il cui grado di sfida sia pari o inferiore al proprio, o tornare alla sua vera forma. Alla morte ritorna alla sua vera forma. Qualsiasi equipaggiamento stia indossando o trasportando viene assorbito o trasportato nella nuova forma (a scelta del deva).

Nella nuova forma, il deva mantiene le sue statistiche di gioco e la facoltà di parlare, ma la sua Difesa, metodi di movimento, Forza, Destrezza e sensi speciali vengono rimpiazzati da quelli della nuova forma, e ottiene qualsiasi statistica o capacità (Azioni aggiuntive e azioni da tana) possedute dalla sua nuova forma e non dalla sua originale.

\textbf{Ecologia}
Ambiente: Qualsiasi piano di con Tratti buono\\
Organizzazione: Solitario, coppia, o squadriglia (3-6)\\
\textbf{Tesoro}: Doppio (Spadone Infuocato +1, altro tesoro)\\
\textbf{Descrizione}\\
I deva movanici compongono i ranghi della fanteria delle armate celesti, sebbene trascorrano la maggior parte del loro tempo pattugliando il Piano Positivo, quello Negativo e quello Materiale. Sul Piano Positivo sorvegliano le anime buone erranti, e questa volta li mette in conflitto con gli Jyoti. Sul Piano Negativo combattono i non morti, gli Sceaduinar e altri strani esseri che cacciano nel famelico vuoto. Le loro rare volte sul Piano Materiale hanno solitamente lo scopo di portare aiuto a potenti mortali, quando un grande pericolo minaccia di far cadere nelle mani del male un intero regno.

\medskip\index[Mostruario]{Angelo Planetar}\textbf{Angelo Planetar}

\textit{Grande celestiale, legale buono}

\textbf{FORZA} +7

\textbf{DESTREZZA} +5

\textbf{COSTITUZIONE} +7

\textbf{INTELLIGENZA} +4

\textbf{SAGGEZZA} +6

\textbf{CARISMA} +7

\textbf{Iniziativa} +5 -- \textbf{Difesa} 27

\textbf{Punti Ferita} 200 (16d10 + 112)

\textbf{Movimento} 12 m, volo 36 m

\textbf{Tiri Salvezza} Tempra +19, Riflessi +11, Volontà +19

\textbf{Competenze} Consapevolezza +11

\textbf{Resistenze ai Danni} da Luce;

\textbf{Immunità alle Condizioni} affascinato, affaticamento, spaventato, armi +1

\textbf{Sensi} visione del vero 36 m

\textbf{Linguaggi} tutte, telepatia 36 m

\textbf{Sfida} 16 (15000 PX)

\textit{\textbf{Armi Angeliche.}} Gli attacchi con arma del planetar sono magici. Quando colpisce con qualsiasi arma, l'arma infligge 5d8 danni da Luce aggiuntivi (già compresi nell'attacco).

\textit{\textbf{Consapevolezza Divina.}} Il planetar riconosce immediatamente le bugie.

\textit{\textbf{Incantesimi Innati.}} La caratteristica da incantatore innato del planetario è il Carisma (DC 20 per i Tiri Salvezza degli incantesimi). Il planetario può lanciare in maniera innata i seguenti incantesimi, senza bisogno di componenti materiali:

A volontà: \textit{individuazione del bene e del male}, \textit{invisibilità} (solo personale)

3/giorno: \textit{barriera di lame, colpo infuocato, dissolvi il bene e il} \textit{male, rianimare morti}

1/giorno: \textit{comunione, controllare tempo atmosferico, piaga degli insetti}

\textit{\textbf{Resistenza alla Magia.}} Il planetar ha +1d6 ai Tiri Salvezza contro incantesimi e altri effetti magici.

\textbf{Azioni}

\textit{\textbf{Multiattacco.}} Il planetar effettua due attacchi da mischia.

\textit{\textbf{Spadone.} Attacco con arma da mischia}: +26 a colpire, portata 1 m, un bersaglio.

\textit{Colpisce:} 21 (4d6 + 7) danni taglienti più 22 (5d8) danni da Luce.

\textit{\textbf{Tocco Guaritore (4/Giorno).}} Il planetar entra a contatto con un'altra creatura. Il bersaglio recupera magicamente 30 (6d8 + 3) Punti Ferita ed è libero da qualsiasi cecità, malattia, maledizione, sordità o veleno.

\textbf{Ecologia}
Ambiente: Qualsiasi piano con Tratti buono\\
Organizzazione: Solitario o coppia\\
\textbf{Tesoro}: Doppio (Spadone Sacro +3)\\
\textbf{Descrizione}\\
I planetar sono i generali delle armate celestiali volti alla distruzione del male. Un planetar è di norma alto 2,7 metri e pesa circa 250 kg. Sono ottimi diplomatici, ma contro gli immondi preferiscono una guerra piuttosto che negoziare una pace.


\medskip\index[Mostruario]{Angelo Solar}\textbf{Angelo Solar}

\textit{Grande celestiale, legale buono}

\textbf{FORZA} +8

\textbf{DESTREZZA} +6

\textbf{COSTITUZIONE} +8

\textbf{INTELLIGENZA} +7

\textbf{SAGGEZZA} +7

\textbf{CARISMA} +10

\textbf{Iniziativa} +7 -- \textbf{Difesa} 31

\textbf{Punti Ferita} 243 (18d10 + 144)

\textbf{Movimento} 15 m, volo 45 m

\textbf{Tiri Salvezza} Tempra +25, Riflessi +14, Volontà +23

\textbf{Competenze} Consapevolezza +14

\textbf{Resistenze ai Danni} da Luce;

\textbf{Immunità ai Danni} da Vuoto, veleno, armi +2

\textbf{Immunità alle Condizioni} affascinato, avvelenato, affaticamento, spaventato, arma +2

\textbf{Sensi} visione del vero 36 m

\textbf{Linguaggi} tutte, telepatia 36 m

\textbf{Sfida} 21 (33000 PX)

\textit{\textbf{Armi Angeliche.}} Gli attacchi con arma del solar sono magici. Quando colpisce con qualsiasi arma, l'arma infligge 6d8 danni da Luce aggiuntivi (già compresi nell'attacco).

\textit{\textbf{Consapevolezza Divina.}} Il solar riconosce immediatamente le bugie.

\textit{\textbf{Incantesimi Innati.}} La caratteristica da incantatore innato del solar è il Carisma (DC 25 per i Tiri Salvezza degli incantesimi). Il solar può lanciare in maniera innata i seguenti incantesimi, senza bisogno di componenti materiali:

A volontà: \textit{individuazione del bene e del male}, \textit{invisibilità} (solo personale)

3/giorno: \textit{barriera di lame, colpo infuocato, dissolvi il bene e il male, resurrezione}

1/giorno: \textit{comunione, controllare tempo atmosferico}

\textit{\textbf{Resistenza alla Magia.}} Il solar ha +1d6 ai Tiri Salvezza contro incantesimi e altri effetti magici.

\textbf{Azioni}

\textit{\textbf{Multiattacco.}} Il solar effettua due attacchi con lo spadone.

\textit{\textbf{Spadone.} Attacco con arma da mischia}: +30 a colpire, portata 1 m, un bersaglio.

\textit{Colpisce:} 22 (4d6 + 8) danni taglienti più 27 (6d8) danni da Luce.

\textit{\textbf{Arco Lungo dell'Uccisione.} Attacco con arma a distanza}: +30 a colpire, gittata 45m, un bersaglio.

\textit{Colpisce:} 15 (2d8 + 6) danni perforanti più 27 (6d8) danni da Luce. Se il bersaglio è una creatura con 100 Punti Ferita o meno, deve riuscire un Tiro Salvezza di Tempra DC 15 o morire.

\textit{\textbf{Spada Volante.}} Il solare libera il suo spadone perché fluttui magicamente in uno spazio non occupato entro 1 metro da lui. Se il solare può vedere la spada, con un'azione bonus le può ordinare mentalmente di volare per un massimo di 15 metri ed effettuare un attacco contro un bersaglio o ritornare nella mano del solare. Se la spada fluttuante è bersaglio di un effetto, si considera come se fosse impugnata dal solare. Se il solare muore, la spada fluttuante cade a terra.

\textit{\textbf{Tocco Guaritore (4/Giorno).}} Il solare entra a contatto con un'altra creatura. Il bersaglio recupera magicamente 40 (8d8 + 4) Punti Ferita ed è libero da qualsiasi cecità, malattia, maledizione, sordità o veleno.

Il solare può effettuare 3 azioni aggiuntive, scelte tra le opzioni seguenti. Può usare solo un'Azione Aggiuntiva alla volta e solo al termine del round di un'altra creatura. Il solare recupera le azioni aggiuntive spese all'inizio del proprio round.

\textbf{Esplosione Incandescente (Costa 2 Azioni).} Il solare emette energia magica divina. Ogni creatura di sua scelta, in un raggio di 3 metri, deve effettuare un Tiro Salvezza su Riflessi DC 30, subendo 14 (4d6) danni da fuoco più 14 (4d6) danni da Luce se fallisce il Tiro Salvezza, o la metà se lo riesce.

\textbf{Sguardo Accecante (Costa 3 Azioni).} Il solare prende a bersaglio una creatura entro 9 metri e che possa vedere. Se il bersaglio può vedere il solare, il bersaglio deve riuscire un Tiro Salvezza su Tempra DC 18 o restare accecato finché un incantesimo come \textit{ristorare inferiore} non rimuoverà la cecità.

\textbf{Teletrasporto.} Il solare si teletrasporta magicamente fino a 36 metri di distanza, insieme a tutto l'equipaggiamento che sta indossando o trasportando, in uno spazio non occupato e che può vedere.

\textbf{Ecologia}\\
Ambiente: Qualsiasi piano con Tratti buono\\
Organizzazione: Solitario o coppia\\
\textbf{Tesoro}: Doppio (Armatura Completa +5, Spadone Danzante +5, Arco Lungo Composito +5)\\
\textbf{Descrizione}\\
I solar sono i più potenti fra gli angeli, solitamente braccio destro di una divinità o campioni di cause che portano beneficio ad un intero mondo o piano. Un solar ha di solito aspetto quasi umano, anche se alcuni di essi somigliano ad altre razze umanoidi ed alcuni hanno anche forme più inusuali. Un solar è alto circa 2,7 metri, pesa circa 250 kg ed ha una voce profonda ed imperiosa, impossibile da ignorare. La maggior parte di essi ha pelle argentata o dorata.

Benedetti con una serie di capacità magiche più potenti, i solar sono avversari terribili in grado di uccidere da soli le più potenti creature malvagie. Fra i celestiali sono considerati i più eccellenti cercatori di tracce, ed i migliori fra loro, si dice, sono in grado di seguire tracce vecchie di giorni lasciate da un Diavolo della Fossa attraverso il Piano Astrale. Alcuni di essi prendono il manto di uccisori di mostri e danno la caccia a potenti immondi e non morti come i divoratori, le megere notturne, le Ombre Notturne ed i Diavoli della Fossa, compiendo perfino incursioni nei piani malvagi e nel Piano dell'Energia Negativa per distruggere queste creature alla fonte, prima che possano fare del male ai mortali. Alcuni fra i solar più antichi hanno portato a termine la loro missione, ed hanno fama di uccisori di creature oggi estinte.

I solar accettano il ruolo di guardiani, di solito di concetti soprannaturali o di oggetti o creature di grande importanza. Su un mondo, un gruppo di solar protegge i condotti di energia del sole contro i tentativi di spegnerlo e di portare tenebre eterne operati da razze malvagie come gli Elfi. Su un altro, sette solar vegliano su sette catene mistiche che tengono gli dèi del male imprigionati in un semipiano. Su un altro ancora, un solar con una spada fiammeggiante protegge il Paradiso Terrestre, impedendo a tutte le creature di entrare.

Nei mondi in cui gli dèi possono prendere forma fisica, i solar vengono inviati per diventare profeti e guru (spesso in guisa di mortali), gettando così le fondamenta di culti che diverranno grandi religioni. Nei mondi oppressi dal male, i solar sono i sacerdoti clandestini che portano speranza agli oppressi o che si lasciano martirizzare così che la loro essenza possa esplodere nelle regioni circostanti e crescere nel cuore dei futuri eroi.

Pur non essendo divinità, il potere dei solar si avvicina a quello dei semidei, e spesso fanno da consiglieri per le divinità più giovani o deboli. In alcune fedi politeiste, i mortali venerano uno o più solar come aspetti o servitori alla pari delle vere divinità (comunque mai senza l'approvazione della divinità in questione) o considerano i solar più famosi come figli, consorti o amanti delle vere divinità (cosa che, a seconda della divinità, potrebbe corrispondere al vero).

A differenza degli altri angeli, la maggior parte dei solar viene creata come diretta servitrice degli dèi, amalgamando anime buone ed energia divina pura, ma sempre più spesso questi potenti angeli vengono creati attraverso la "promozione" di angeli minori come deva e planetar. Raramente accade che anime particolarmente potenti e pure ascendano direttamente allo status di solar. I più antichi fra loro sono anteriori alla creazione dei mortali, e sono fra le prime creazioni degli dèi. Questi solar sono campioni fra i loro simili, ed hanno poca o nessuna interazione con i mortali, concentrandosi invece su concetti astratti quali la gravità, l'entropia, la materia oscura ed il male primevo.

I solar che passano molto tempo sul Piano Materiale, specialmente quelli che prendono la forma di mortali, sono a volta fonte di linee di sangue aasimar o mezzo-celestiali nelle famiglie umane, a volte a causa di una storia d'amore, a volte semplicemente per la vicinanza dei mortali alle loro emanazioni celestiali. È raro che vi siano loro discendenti diretti, e quando ciò accade è sempre una madre mortale a portare in grembo il figlio: anche se i solar possono apparire di qualsiasi sesso, gli dèi non hanno concesso loro la possibilità di dare alla luce un figlio. È per questo che i solar tendono a cercare un'amante mortale. Gli altri solar hanno poca considerazione di un loro simile che dà un figlio ad una mortale, quindi i padri solar tendono ad evitare i contatti con la loro progenie, per evitare di attirare vergogna su di sé. I solar, però, tendono a controllare da lontano i loro figli e, in tempo di difficoltà, ad aiutarli, anche se in modi misteriosi e discreti.

Tutti gli angeli rispettano il potere e la saggezza dei solar, e sebbene essi tendano a lavorare da soli, a volte comandano armate guidate da planetar e fanno da generali per le grandi incursioni contro le legioni dell'Inferno o le orde dell'Abisso.

\medskip\index[Mostruario]{Ankheg}\textbf{Ankheg}

\textit{Grande mostruosità, disallineato}

\textbf{FORZA} +3

\textbf{DESTREZZA} +0

\textbf{COSTITUZIONE} +1

\textbf{INTELLIGENZA} -5

\textbf{SAGGEZZA} +1

\textbf{CARISMA} -2

\textbf{Iniziativa} +0 -- \textbf{Difesa} 15, 12 mentre è prono

\textbf{Punti Ferita} 39 (6d10 + 6)

\textbf{Movimento} 9 m, scavo 3 m

\textbf{Sensi} scurovisione 18 m, percezione tellurica 18 m,

\textbf{Linguaggi} -

\textbf{Sfida} 2 (450 PX)

\textbf{Azioni}

\textit{\textbf{Morso.} Attacco con arma da mischia}: +5 a colpire, portata 1 m, un bersaglio.

\textit{Colpisce:} 10 (2d6 + 3) danni taglienti più 3 (1d6) danni da acido. Se il bersaglio è una creatura di taglia Grande o inferiore, è afferrata (DC 13 per fuggire). Fino al termine dell'afferrare, l'ankheg può mordere solo la creatura afferrata e ha +1d6 ai tiri di attacco contro di essa.

\textit{\textbf{Spruzzo Acido (Ricarica 6).}} L'ankheg sputa acido in una linea lunga 9 metri e larga 1 metro, purché non stia afferrando nessuna creatura. Ogni creatura su quella linea deve effettuare un Tiro Salvezza di Riflessi DC 13, e subire 10 (3d6) danni da acido se fallisce il Tiro Salvezza, o la metà di questi danni se lo riesce.

\textbf{Ecologia}\\
Ambiente: Pianure temperate o calde\\
Organizzazione: Solitario, coppia o nido (3-6)\\
\textbf{Tesoro}: Accidentale\\
\textbf{Descrizione}\\
Gli ankheg sono una piaga fin troppo comune per le zone rurali. Questo mostro scavatore grande come un cavallo in genere evita le zone abitate più popolose, ma la sua predilezione per la carne del bestiame e degli esseri umani li tiene lontano dalle zone disabitate. Il loro habitat favorito è rappresentato delle campagne rurali, dato che il terriccio smosso rende loro molto facile muoversi scavando. Si narra di ankheg più grandi che vivono nei deserti remoti, che si cibano di Scorpioni e Cammelli e vengono raramente in contatto con la civiltà (un ankheg del deserto è un ankheg avanzato Enorme).

In combattimento, gli ankheg preferiscono attaccare con il morso. Contro più avversari, un ankheg Afferra uno dei bersagli e tenta di ritirarsi sottoterra. Una creatura trascinata sottoterra può respirare, anche se con difficoltà (anche l'ankheg deve farlo, quindi i tunnel sono abbastanza porosi), ma spesso viene mangiata viva prima che i suoi compagni possano salvarla.

Gli ankheg scavano con le loro gambe e le mandibole, muovendosi velocissimi attraverso il terriccio, la sabbia e la ghiaia (non la roccia). Un ankheg che scava si ferma spesso a costruire tunnel, cospargendo le pareti con una densa secrezione orale. Se un ankheg vuole costruire un tunnel mentre scava deve muoversi a metà della propria velocità di scavare. Un tipico tunnel di ankheg è alto e largo 3 metri, di forma vagamente circolare e lungo da 18 a 45 metri ([1d10+5]×10). I gruppi di ankheg condividono lo stesso territorio e creano complesse reti di tunnel sotto le campagne, a volte creando voragini nei punti in cui troppi di essi scavano allo stesso tempo.

Anche se gli ankheg somigliano ad immensi insetti sono più intelligenti e, con un pò di tempo e un buon addestratore, possono diventare animali domestici o da carico. Il fatto che anche "addomesticati" gli ankheg tendano a sputare acido quando spaventati o sorpresi li rende poco sicuri nelle regioni più civilizzate, ma fra le razze selvagge, come gli Hobgoblin, i Trogloditi e soprattutto gli Orchi sono popolari come guardiani o perfino animali da salotto. Un ankheg può raggiungere una lunghezza di 3 metri e pesare circa 400 kg.

\medskip\index[Mostruario]{Arpia}\textbf{Arpia}

\textit{Media mostruosità, caotico malvagio}

\textbf{FORZA} +1

\textbf{DESTREZZA} +1

\textbf{COSTITUZIONE} +1

\textbf{INTELLIGENZA} -2

\textbf{SAGGEZZA} +0

\textbf{CARISMA} +1

\textbf{Iniziativa} +1 -- \textbf{Difesa} 12

\textbf{Punti Ferita} 38 (7d8 + 7)

\textbf{Movimento} 6 m, volo 12 m

\textbf{Linguaggi} Comune

\textbf{Sfida} 1 (200 PX)

\textbf{Azioni}

\textit{\textbf{Multiattacco.}} L'armatura effettua due attacchi: uno con gli artigli e uno con il randello.

\textit{\textbf{Artigli.} Attacco con arma da mischia}: +3 a colpire, portata 1 m, un bersaglio.

\textit{Colpisce:} 5 (2d4 + 1) danni taglienti, 1 danno da Sanguinamento.

\textit{\textbf{Randello.} Attacco con arma da mischia}: +3 a colpire, portata 1 m, un bersaglio.

\textit{Colpisce:} 3 (1d4 + 1) danni da botta.

\textit{\textbf{Canto Ammaliatore.}} L'arpia canta una melodia magica. Ogni umanoide e gigante entro 90 metri dall'arpia e che possa udire la canzone deve riuscire un Tiro Salvezza di Volontà DC 11 o restare affascinato fino al termine della canzone. L'arpia deve effettuare un'azione bonus durante il suo prossimo round per continuare a cantare. Può smettere di cantare in qualsiasi momento. Il canto ha termine se l'arpia è inabile.

Mentre è affascinato dall'arpia, un bersaglio è inabile e ignora le canzoni di altre arpie. Se il bersaglio affascinato si trova a più di 1 metro dall'arpia, il bersaglio deve muoversi durante il proprio round per dirigersi verso l'arpia usando la via più diretta. Prima di muoversi in un terreno pericoloso, come lava o un pozzo, e prima di subire danno da qualsiasi fonte che non sia l'arpia, il bersaglio potrà ripetere il Tiro Salvezza. Una creatura può ripetere il Tiro Salvezza al termine di ciascun proprio round. Se il Tiro Salvezza ha successo, l'effetto ha termine per quel bersaglio.

Un bersaglio che riesce il Tiro Salvezza è immune al canto di quell'arpia per le successive 24 ore.

\textbf{Ecologia}\\
Ambiente: Paludi Temperate\\
Organizzazione: Solitario, coppia o stormo (3-12)\\
\textbf{Tesoro}: Standard (Armatura di Cuoio, Mazza chiodata e altro tesoro)\\
\textbf{Descrizione}\\
Spesso viste come creature malvagie e corrotte, le arpie sanno come gli altri pensano e agiscono. Questa capacità percettiva offre loro un vantaggio nel trovare i loro pasti preferiti. Sebbene le creature selvatiche cadano facilmente vittime del canto ammaliatore, queste malvagie donne-uccello preferiscono pasti conditi con complessi pensieri senzienti. Le facili prede rendono il pasto noioso.

Anche se in definitiva selvagge e senza alcun rimorso per le loro azioni, diverse arpie vivono presso le società umanoidi e si divertono a sfruttare le creature che reputano potenziali pasti.

Le arpie tendono ad indossare ninnoli e ciondoli rubati alle loro vittime, perché amano compiacersi dei brillanti ornamenti degli uomini. Da vicino queste creature trasudano del puzzo delle loro vittime divorate e raramente lasciano che le creature non ancora ammaliate si avvicinino troppo, cosicché non sentano l'odore del sangue e della putrefazione sulle loro penne. Per questo motivo, molte arpie si cospargono di profumi e oli aromatici.

Le arpie sono marcatamente differenti a seconda della regione in cui vivono. Alcune assomigliano ad una mescolanza di avvoltoi e donne, mentre altri portano sulle penne i tratti regali di falchi e falconi. Rare nidiate di arpie, in luoghi isolati e tropicali del mondo, hanno anche piume colorate come i pappagalli.

\medskip\index[Mostruario]{Azer}\textbf{Azer}

\textit{Media elementale, legale neutrale}

\textbf{FORZA} +3

\textbf{DESTREZZA} +1

\textbf{COSTITUZIONE} +2

\textbf{INTELLIGENZA} +1

\textbf{SAGGEZZA} +1

\textbf{CARISMA} +0

\textbf{Iniziativa} +1 -- \textbf{Difesa} 18 (armatura naturale, scudo)

\textbf{Punti Ferita} 39 (6d8 + 12)

\textbf{Movimento} 9 m

\textbf{Tiri Salvezza} Tempra +2, Riflessi +1, Volontà +1

\textbf{Immunità ai Danni} fuoco, veleno

\textbf{Immunità alle Condizioni} avvelenato

\textbf{Linguaggi} Ignan

\textbf{Sfida} 2 (450 PX)

\textit{\textbf{Armi Riscaldate.}} Quando l'azer colpisce con un'arma da mischia in metallo, infligge 3 (1d6) danni da fuoco aggiuntivi (già inclusi nell'attacco).

\textit{\textbf{Corpo Riscaldato.}} Una creatura che entri a contatto con l'azer o lo colpisca con un attacco da mischia mentre si trova entro 1 metro da lui subisce 5 (1d10) danni da fuoco.

\textit{\textbf{Fuoco Vivente.}} Un azer non ha bisogno di cibo, bevande o di dormire.

\textit{\textbf{Illuminazione.}} L'azer irradia luce intensa in un raggio di 3 metri e luce fioca per ulteriori 3 metri.

\textbf{Azioni}

\textit{\textbf{Martello da Guerra.} Attacco con arma da mischia}: +6 a colpire, portata 1 m, un bersaglio.

\textit{Colpisce:} 7 (1d8 + 3) danni da botta, o 8 (1d10 + 3) danni da botta se usato a due mani per effettuare un attacco da mischia, più 3 (1d6) danni da fuoco.

\textbf{Ecologia}\\
Ambiente: Qualsiasi terreno (Piano del Fuoco)\\
Organizzazione: Solitario, coppia, gruppo (3-6), squadra (11-20 più 2 sergenti di 3° livello e 1 capo di 3°-6° livello) o clan (30-100 più 50\% di non combattenti più 1 sergente di 3° livello ogni 20 adulti, 5 tenenti di 5° livello e 3 capitani di 7° livello)\\
\textbf{Tesoro}: Standard (Corazza a Scaglie Perfetta, Martello da Guerra Perfetto, Martello Leggero, altro tesoro)\\
\textbf{Descrizione}\\
Una Razza orgogliosa e industriosa proveniente dal Piano del Fuoco, gli Azer lavorano nelle loro fortezze di bronzo e d'ottone, sempre pronti a combattere la loro lunga e ribollente guerra contro gli Efreet. Gli Azer vivono in una società in cui ogni membro sa qual è il suo posto. Nati con specifici doveri, solitamente legati alle attività del padre o della madre, gli Azer si dedicano a queste occupazioni per tutta la vita. Un sistema di caste provvede a tenere ulteriormente in riga la società Azer. I nobili, che regnano senza dover rendere conto a nessuno, indossano kilt di ottone decorato come simbolo della loro casta, mentre quelli dei mercanti e dei proprietari di negozi sono in resistente bronzo. I kilt di rame sono indossati dalla casta lavoratrice, composta da servitori, artigiani e braccianti.

Capaci di incanalare calore tramite le Armi e gli attrezzi in metallo, gli Azer non utilizzano quasi mai Armi non metalliche, e prediligono il corpo a corpo agli attacchi a distanza. Sono soliti fare prigionieri, riportandoli alle loro fortezze e obbligandoli a lavorare per loro per un anno e un giorno.

Nella leggendaria Città d'Ottone abitano più di mezzo milione di Azer. La maggior parte di questi sfortunati Azer vive una vita di Schiavitù sotto gli Efreet. Gli Azer soggiogati a questa Schiavitù continuano a eseguire i loro doveri senza porre domande, preferendo aspettare la conclusione dei loro contratti o sperando che i loro padroni muoiano o vengano sconfitti. La dedizione all'ordine arde intensa in questa Razza, al punto che alcuni degli Schiavi Azer fungono da supervisori sulla loro stessa gente. Al di fuori della Città d'Ottone, gli Azer sono liberi di vivere le loro vite, spesso in altre metropoli Planari, creando oggetti, vendendo merci e gestendo taverne.

A un occhio non allenato gli Azer si somigliano tra loro in modo impressionante. Sono alti 1,2 metri ma pesano 100 kg.

\medskip\index[Mostruario]{Banshee}\textbf{Banshee}

\textit{Media non morto, caotico malvagio}

\textbf{FORZA} -5

\textbf{DESTREZZA} +5

\textbf{COSTITUZIONE} +0

\textbf{INTELLIGENZA} +1

\textbf{SAGGEZZA} +1

\textbf{CARISMA} +4

\textbf{Iniziativa} +5 -- \textbf{Difesa} 15

\textbf{Punti Ferita} 58 (13d8)

\textbf{Movimento} 0 m, volo 18 m (fluttua)

\textbf{Tiri Salvezza}: Tempra +4, Riflessi +9, Volontà +5

\textbf{Resistenze al Danno} acido, fulmine, fuoco, suono; da arma magica +1

\textbf{Immunità al Danno} da Vuoto, Veleno, Freddo

\textbf{Immunità alle Condizioni} affascinato, afferrato, avvelenato, intralciato, paralizzato, pietrificato, prono, affaticamento

\textbf{Sensi} scurovisione 18 m

\textbf{Linguaggi} elfico, comune

\textbf{Sfida} 4 (1.100 PX)

\textit{\textbf{Individuazione della Vita}}. La Banshee percepisce la presenza di creature che non siano non morti e costrutti entro un raggio di 5 chilometri. Conosce la direzione generale in cui si trovano, ma non la loro precisa ubicazione.

\textit{\textbf{Movimento Incorporeo}}. La Banshee può muoversi attraverso altre creature e oggetti come se fossero terreno difficile. Subisce 5 (1d10) danni da forza se termina il suo turno all'interno di un oggetto.

\textit{\textbf{Natura Non Morta.}} La Banshee non ha bisogno di aria, cibo, bevande o sonno.

\textit{\textbf{Sensibilità alla Luce}}. Mentre è alla luce del sole, la Banshee ha -1d6 ai tiri di attacco, oltre che alle prove di Saggezza (Consapevolezza) basate sulla vista.

\textbf{Azioni}

\textit{\textbf{Tocco Corruttore}}. Attacco contro Difesa a Tocco in Mischia: +6 al tiro per colpire, portata 1 m, un bersaglio.

\textit{Colpito}: 12 (3d6 +2) danni da vuoto.

\textit{\textbf{Volto Terrificante}}. Ogni creatura che non sia un non morto situata entro 18 metri dalla Banshee e che sia in grado di vederla deve superare un Tiro Salvezza su Volontà con modificatore Carisma con DC 15, altrimenti è spaventata per 1 minuto. Un bersaglio spaventato può ripetere il Tiro Salvezza alla fine di ogni suo turno, subendo -1d6 se la Banshee si trova entro linea di vista; se supera il tiro, l'effetto per lui termina. Se un bersaglio supera il Tiro Salvezza o l'effetto per lui termina, quel bersaglio è immune al Volto Terrificante della Banshee per le 24 ore successive.

\textit{\textbf{Lamento (1/Giorno)}}. La Banshee emette un lamento funesto, purché non sia esposta alla luce del sole. Questo lamento non ha alcun effetto sui costrutti e sui non morti. Ogni altra creatura situata entro 9 metri da lei e in grado di udirla deve effettuare un Tiro Salvezza su Tempra DC 15; se lo fallisce, scende a O Punti Ferita, mentre se lo supera, subisce 35 (10d6) danni psichici.

\textbf{Ecologia}\\
Ambiente: Qualsiasi\\
Organizzazione: Solitario\\
\textbf{Tesoro}: Nessuno\\
\textbf{Descrizione}\\
La Banshee è lo spirito infuriato un elfa che ha tradito i propri cari o è stata a sua volta tradita. Impazzita per il dolore, la Banshee riversa la propria vendetta su ogni creatura vivente (innocente o colpevole) con il suo temibile tocco e le sue grida mortali.

\medskip\index[Mostruario]{Basilisco}\textbf{Basilisco}

\textit{Media mostruosità, disallineato}

\textbf{FORZA} +3

\textbf{DESTREZZA} -1

\textbf{COSTITUZIONE} +2

\textbf{INTELLIGENZA} -4

\textbf{SAGGEZZA} -1

\textbf{CARISMA} -2

\textbf{Iniziativa} -1 -- \textbf{Difesa} 17

\textbf{Punti Ferita} 52 (8d8 + 16)

\textbf{Movimento} 6 m

\textbf{Sensi} scurovisione 18 m

\textbf{Linguaggi} -

\textbf{Sfida} 3 (700 PX)

\textit{\textbf{Sguardo Pietrificante.}} Se una creatura comincia il suo round entro 9 metri dal basilisco e i due si possono vedere vicendevolmente, se non inabile il basilisco può obbligare la creatura ad effettuare un Tiro Salvezza di Tempra DC 12. Se la creatura fallisce il Tiro Salvezza, inizia magicamente a trasformarsi in pietra ed è intralciata. La creatura deve ripetere il Tiro Salvezza al termine del suo prossimo round. Se lo riesce, l'effetto termina. Se lo fallisce, la creatura è pietrificata finché non viene liberata dall'incantesimo \textit{ristorare} \textit{superiore} o altra magia.

Una creatura che non sia sorpresa, può distogliere lo sguardo per evitare il Tiro Salvezza all'inizio del suo round. In quel caso, non potrà vedere il basilisco fino all'inizio del suo prossimo round, quando potrà distogliere nuovamente lo sguardo. Se nel frattempo dovesse guardare il basilisco, dovrebbe immediatamente effettuare il Tiro Salvezza.

Se il basilisco si trova entro 9 metri dal suo riflesso a luce intensa e lo vede, lo scambia per un rivale e diventa il bersaglio del proprio sguardo.

\textbf{Azioni}

\textit{\textbf{Morso.} Attacco con arma da mischia}: +7 a colpire, portata 1 m, un bersaglio.

\textit{Colpisce:} 10 (2d6 + 3) danni perforanti più 7 (2d6) danni da veleno.

\textbf{Ecologia}\\
Ambiente: Qualsiasi\\
Organizzazione: Solitario, coppia o colonia (3-6)\\
\textbf{Tesoro}: Accidentale\\
\textbf{Descrizione}\\
Il basilisco, spesso chiamato "Re dei Serpenti" è un rettile a otto zampe di indole aggressiva che ha la capacità di trasformare le creature in pietra con il suo sguardo. La leggenda narra che, come la Cockatrice, i primi basilischi nacquero da uova deposte da serpenti e covate da galli, ma ben poco nella fisiologia del basilisco lascia spazio a questa teoria.

I basilischi vivono in quasi tutti gli ambienti asciutti, dalla foresta al deserto, e la loro pelle tende a rispecchiare l'ambiente che li circonda: un basilisco del deserto può essere bronzeo o marrone, mentre uno che vive nelle foreste può essere di colore verde acceso. Tendono a usare come rifugio le grotte, le tane o altre zone riparate. Questi rifugi sono spesso segnalati da statue raffiguranti persone e animali in pose naturali, che non sono altro che i resti pietrificati degli sventurati imbattutisi in un basilisco.

I basilischi hanno la capacità di consumare le creature pietrificate; l'acido prodotto dal loro stomaco dissolve ed estrae sostanze nutrienti dalla pietra, sebbene il processo sia lento e inefficiente, il che li rende pigri e inerti. Di conseguenza, i basilischi raramente attaccano o cacciano le prede che evitano il loro sguardo, contando sulla loro Furtività e l'elemento di sorpresa al fine di non rimanere senza cibo. Quando non sono in attesa dei piccoli mammiferi, uccelli o rettili che fanno parte della loro dieta, i basilischi passano il tempo a dormire nelle tane. Coloro che sono abbastanza coraggiosi da catturare i basilischi o da nascondere un tesoro vicino a loro, scoprono che questi esseri possono fare da custodi o da cani da guardia.

Un basilisco adulto è lungo quasi 4 metri, di cui la metà occupata dalla lunga coda, e pesa 135 chili. Alcune razze presentano delle piccole corna ricurve sul naso o piccole creste di pungiglioni ossuti sopra la testa simili a una corona. Sebbene siano creature in genere solitarie che si riuniscono solo per accoppiarsi e deporre le uova, in zone particolarmente pericolose possono riunirsi in piccoli gruppi per proteggersi e attaccare gli intrusi in massa.

Per motivi ignoti, le donnole e i furetti sono immuni allo sguardo del basilisco, e a volte si intrufolano nelle tane mentre l'adulto è a caccia per cibarsi dei suoi piccoli. Alcune leggende narrano che il sangue di un basilisco può tramutare comuni pietre in un altro materiale, ma probabilmente si tratta di testimoni che hanno mal interpretato la ristorazione magica di creature o di parti del corpo pietrificate in precedenza.

\medskip\index[Mostruario]{Behir}\textbf{Behir}

\textit{Enorme mostruosità, neutrale malvagio}

\textbf{FORZA} +6

\textbf{DESTREZZA} +3

\textbf{COSTITUZIONE} +4

\textbf{INTELLIGENZA} -2

\textbf{SAGGEZZA} +2

\textbf{CARISMA} +1

\textbf{Iniziativa} +3 -- \textbf{Difesa} 23

\textbf{Punti Ferita} 168 (16d12 + 64)

\textbf{Movimento} 15 m, scalata 12 m

\textbf{Competenze} Muoversi Silenziosamente / Nascondersi +7, Consapevolezza +6

\textbf{Immunità al Danno} fulmine

\textbf{Sensi} scurovisione 27 m

\textbf{Linguaggi} Draconico

\textbf{Sfida} 11 (7.200 PX)

\textbf{Azioni}

\textit{\textbf{Multiattacco.}} Il behir effettua due attacchi: uno con il morso e uno per stritolare.

\textit{\textbf{Morso.} Attacco con arma da mischia}: +16 a colpire, portata 3 m, un bersaglio.

\textit{Colpisce:} 22 (3d10 + 6) danni perforanti.

\textit{\textbf{Stritolare.} Attacco con arma da mischia}: +16 a colpire, portata 1 m, una creatura di taglia Grande o inferiore.

\textit{Colpisce:} 17 (2d10 + 6) danni da botta più 17 (2d10 + 6) danni taglienti. Il bersaglio è afferrato (DC 16 per fuggire) Se il behir non sta già stritolando un'altra creatura, il bersaglio è afferrato e intralciato fino al termine dell'afferrare.

\textit{\textbf{Inghiottire.}} Il behir effettua una attacco di morso contro un bersaglio di taglia Media o inferiore che sta afferrando. Se l'attacco colpisce, il bersaglio è inghiottito, e l'afferrare ha termine. Il bersaglio inghiottito è accecato e intralciato, ha copertura completa contro gli attacchi e altri effetti all'esterno del behir, e subisce 21 (6d6) danni da acido all'inizio di ciascun turno del behir. Il behir può inghiottire solo una creatura alla volta.

Se il behir subisce 30 o più danni in un singolo turno da una creatura che ha inghiottito, deve riuscire un Tiro Salvezza di Tempra DC 14 al termine di quel turno o vomitare la creatura, che ricade prona in uno spazio entro 3 metri dal behir. Se il behir muore, una creatura inghiottita non è più intralciata da esso e può uscire dal cadavere utilizzando 5 metri di movimento, uscendo prona.

\textit{\textbf{Soffio di Fulmine (Ricarica 5-6).}} Il behir esala fulmini in una linea lunga 6 metri e larga 1 metro. Ogni creatura su quella linea deve effettuare un Tiro Salvezza di Riflessi DC 16 e subire 66 (12d10) danni da fulmine se fallisce il Tiro Salvezza, o la metà di questi danni se lo riesce.

\textbf{Ecologia}\\
Ambiente: Colline e Deserti Caldi\\
Organizzazione: Solitario o coppia\\
\textbf{Tesoro}: Doppio\\
\textbf{Descrizione}\\
Istintivo e bramoso, il behir trascorre gran parte del tempo a strisciare per le colline sabbiose e le rocce del deserto che formano il suo territorio, dando la caccia a tutte le creature che osano entrare nel suo territorio. Le sue sei paia di zampe robuste e dotate di artigli restano piegate ai suoi fianchi per gran parte del tempo, e si stendono solo in combattimento per afferrare i nemici, per Correre al galoppo o per Scalare i pendii delle scogliere a picco, tane di queste creature.

In media il behir è lungo 12 metri e pesa circa 1800 Kg. Oltre alle due corna prominenti sulla testa, molti hanno aculei decorativi a intervalli regolari lungo la spina dorsale.

Pur essendo territoriale e bestiale nella sua furia, il behir non è né stupido né necessariamente malvagio anche se, a causa del suo egocentrismo e della tendenza a rivendicare come sua ogni cosa esistente, entra spesso in conflitto con le altre razze. In quanto tale, un behir può essere corrotto o convinto da intrepidi negoziatori disposti ad avvicinarglisi. In questi casi, la tendenza di un behir ad attaccare prima e a ragionare poi (o non ragionare affatto) significa che chiunque cerchi di trovare un accordo deve avere dei validi motivi e far subito colpo sul behir con un'offerta allettante.

Spesso si dice che i behir siano in qualche modo legati ai draghi blu, ma la vera natura di questo Legame rimane un mistero. Molti draghi negano qualsiasi Legame e non vedono di buon occhio i behir per la loro scarsa intelligenza: un affronto che fa infuriare i behir, di per sé già impulsivi. Proprio per questo, molti behir portano rancore verso i draghi e sono pronti ad attaccare qualunque drago entri nel loro territorio.

\medskip\index[Mostruario]{Blatta Esplosiva}\textbf{Blatta Esplosiva}

\textit{Piccolo Elementale, neutrale}

\textbf{FORZA} +1

\textbf{DESTREZZA} +2

\textbf{COSTITUZIONE} +1

\textbf{INTELLIGENZA} -5

\textbf{SAGGEZZA} -1

\textbf{CARISMA} -2

\textbf{Iniziativa} +2 -- \textbf{Difesa} 14

\textbf{Punti Ferita} 45 (8d8 + 9)

\textbf{Movimento} 4 m, salto 9 m, scavare 2 m

\textbf{Tiri Salvezza} Tempra +5, Riflessi +6, Volontà +3

\textbf{Resistenze al Danno} da botta da arma non magica

\textbf{Immunità al Danno} da fuoco

\textbf{Immunità alle Condizioni} affaticamento, spaventato

\textbf{Sensi} vista cieca 5 m

\textbf{Linguaggi} -

\textbf{Sfida} 2 (450 PX)

\textit{individuazione del fuoco}: la Blatta Esplosiva può percepire fuochi entro 100 metri di distanza, purché pari o superiori ad una torcia

\textit{Scavare}: la blatta esplosiva può scavare nel terreno solido a metà del proprio movimento.

\textbf{Azioni}

\textit{\textbf{Multiattacco.}} la Blatta Esplosiva può effettuare 1 attacco di carica oppure emettere una poltiglia di fuoco.

\textit{\textbf{Carica.}} Attacco da mischia: +6 a a colpire, portata 1 metro, un bersaglio.

\textit{Colpisce:} 12 (3d6 + 3) danni da botta. La creature deve effettuare un Tiro Salvezza su Tempra a DC 11 o cadere prona.

\textit{\textbf{Poltiglia di Fuoco}} Attacco da distanza: +7 al colpire, portata 3 metri, 2 quadretti. La Blatta Esplosiva rigurgita un liquido appiccicoso e infiammabile all'aria. Ricarica 1/3-6.

\textit{Colpisce:} 18 (4d6 + 6) danni da fuoco. Tiro Salvezza su Riflessi DC 13 per dimezzare.

\textit{\textbf{Morte:}} Quando la Blatta Esplosiva muore la gelatina all'interno a contatto con l'aria esplode tutto intorno, nel raggio di 1 metro attorno alla blatta le fiamme causano 12 (4d6) di danno, Tiro Salvezza su Riflessi DC 15 per dimezzare.

\textbf{Ecologia}\\
Ambiente: caverne calde\\
Organizzazione: Solitario, nido (8-64)\\
\textbf{Tesoro}: Diamante 1d4x1d50mo\\
\textbf{Descrizione}\\
Le Blatta Esplosive sono creature native tra il piano elementale del fuoco e della terra. Solitamente attirati da ambienti ricchi di fiamme, pietra o almeno caldo e terra.
Dalla forma proporzionata a quelli di una comune blatta se non unga circa 40 cm e pensante circa 4 kg, è una creatura completamente priva di intelletto agendo solo per puro istinto.
Sono ormai comuni nella caverne prossime a vulcani o tane di drago rosso essendosi abituate a vivere su Yeru.

Nel nido dove dimorano c'è almeno una regina che comanda le blatte, estremamente più grossa e forte. Le Blatte Esplosive si nutrono di carbone, legni bruciati, carcasse bruciate. Sono estremamente golosi di diamanti che una volta bruciati sono delle vere e proprie leccornie.

\medskip\index[Mostruario]{B.O.C.}\textbf{B.O.C.}

\textit{grande mostruosità, legale malvagio}

\textbf{FORZA} +4

\textbf{DESTREZZA} +3

\textbf{COSTITUZIONE} +2

\textbf{INTELLIGENZA} -2

\textbf{SAGGEZZA} +1

\textbf{CARISMA} -1

\textbf{Iniziativa} +2 -- \textbf{Difesa} 17

\textbf{Punti Ferita} 42 (8d8 + 10)

\textbf{Movimento} 13 m

\textbf{Competenze} Muoversi Silenziosamente / Nascondersi +8, Consapevolezza +6

\textbf{Resistenza} +4 ai Tiri Salvezza agli incantesimi della Lista Illusione

\textbf{Sensi} scurovisione 20 m, visione crepuscolare 18 m

\textbf{Linguaggi} comune, può solo comprenderlo

\textbf{Sfida} 4 (1.100 PX)

\textbf{Azioni}

\textit{\textbf{Multiattacco.}} Il B.O.C effettua due attacchi con artigli ed uno con il morso, oppure effettua due attacchi con i tentacoli

\textit{\textbf{Artigli.} Attacco con arma da mischia}: +6 a colpire, portata 3 m, un bersaglio, 1 danno da Sanguinamento.

\textit{Colpisce:} 7 (1d6 + 4) danni da taglio.

\textit{\textbf{Morso.} Attacco con arma da mischia}: per ogni artiglio che ha colpito il B.O.C ottiene +2 al colpire con il morso. +8 a colpire, portata 3 m, un bersaglio.

\textit{Colpisce:} 10 (1d8 + 6) danni da taglio.

\textit{\textbf{Tentacoli.} Attacco con arma da mischia}: Ogni tentacolo può colpire fino a 6 metri di distanza ed ognuno può colpire un bersaglio diverso, +6 al colpire.

\textit{Colpisce:} 6 (1d4 + 4) danni da botta

\textit{\textbf{Deflettere la luce.}} Il B.O.C. è costantemente influenzato da un effetto che ne altera la posizione, ogni Tiro per Colpire ha -1d6. Questa penalità si elimina se si può attaccare il B.O.C. senza usare la vista per individuarlo.

Il B.O.C. piega costantemente la luce intorno a se apparendo quasi un metro spostato rispetto alla sua reale posizione. Questa abilità non è influenzata da visioni di tipo normali, solo visione del vero, vista cieca o senso tellurico possono percepire correttamente il B.O.C.

\textbf{Ecologia}\\
Ambiente: Colline e foreste\\
Organizzazione: Solitario, coppia oppure branco (2d4)\\
\textbf{Tesoro}: Accidentale\\
\textbf{Descrizione}\\
Il Black Ops Cat meglio conosciuto con B.O.C. è un grande felino predatore, ovviamente di colore nero. Feroce, insaziabile, uccide per il gusto di cacciare. Agisce solitamente in branco ed è estremamente fedele al gruppo.

\medskip\index[Mostruario]{Bugbear}\textbf{Bugbear}

\textit{Media umanoide (goblinoide), caotico malvagio}

\textbf{FORZA} +2

\textbf{DESTREZZA} +2

\textbf{COSTITUZIONE} +1

\textbf{INTELLIGENZA} -1

\textbf{SAGGEZZA} +0

\textbf{CARISMA} -1

\textbf{Iniziativa} +2 -- \textbf{Difesa} 17

\textbf{Punti Ferita} 27 (5d8 + 5)

\textbf{Movimento} 9 m

\textbf{Competenze} Muoversi Silenziosamente / Nascondersi +6, Sopravvivenza +2

\textbf{Sensi} scurovisione 18 m

\textbf{Linguaggi} Comune, Goblin

\textbf{Sfida} 1 (200 PX)

\textit{\textbf{Attacco di Sorpresa.}} Se il bugbear sorprende una creatura e la colpisce con un attacco durante il primo round di combattimento, il bersaglio subisce 7 (2d6) danni aggiuntivi
dall'attacco.

\textit{\textbf{Bruto.}} Un'arma da mischia infligge un dado aggiuntivo di danno quando il bugbear colpisce con essa (già incluso nell'attacco).

\textbf{Azioni}

\textit{\textbf{Mazza Chiodata.} Attacco con arma da mischia}: +4 a colpire, portata 1 m, un bersaglio.

\textit{Colpisce:} 11 (2d8 + 2) danni perforanti.

\textit{\textbf{Giavellotto.} Attacco con arma da mischia o a Distanza}: +4 a colpire, portata 1 m o gittata 12m, un bersaglio.

\textit{Colpisce:} 9 (2d6 + 2) danni perforanti in mischia o 5 (1d6 + 2) danni perforanti a gittata.

\textbf{Ecologia}\\
Ambiente: Montagne temperate\\
Organizzazione: Solitario, coppia, gruppo (3-6) o banda da guerra (7-12 più 2 Guerrieri di 1° livello e 1 capitano di 3°-5° livello)\\
\textbf{Tesoro}: Equipaggiamento da PNG (Armatura di Cuoio, Scudo Leggero di Legno, Mazza chiodata, 3 Giavellotti, altro tesoro)\\
\textbf{Descrizione}\\
Il bugbear è il più grande degli esponenti della razza Goblinoide, un bruto dai movimenti pesanti che supera di almeno una testa la maggior parte degli Umani. Sono solitari che preferiscono vivere ed uccidere da soli piuttosto che in tribù, sebbene non sia insolito trovare una piccola banda di Bugbear che collabora o vive con una tribù di Goblin od Hobgoblin fungendo da guardia d'élite o carnefici.

I bugbear non formano grandi insediamenti come i goblin o nazioni come gli hobgoblin; preferiscono qualcosa di più piccolo e caotico che li lasci liberi di fare quello che preferiscono (uccidere e torturare) a un livello più personale. Gli umani sono le prede preferite dei bugbear, e la maggior parte di essi annovera la carne umana come uno degli alimenti principali della propria dieta. Macabri trofei quali orecchie e dita sono decorazioni comuni tra i bugbear.

I bugbear, quando si rivolgono alla religione, prediligono le divinità dell'omicidio e della violenza, con i vari signori dei demoni tra i preferiti. Un tipico bugbear è alto 2,1 metri e pesa 200 kg.

\medskip\index[Mostruario]{Bulette}\textbf{Bulette}

\textit{Grande mostruosità, disallineato}

\textbf{FORZA} +4

\textbf{DESTREZZA} +0

\textbf{COSTITUZIONE} +5

\textbf{INTELLIGENZA} -4

\textbf{SAGGEZZA} +0

\textbf{CARISMA} -3

\textbf{Iniziativa} +0 -- \textbf{Difesa} 20

\textbf{Punti Ferita} 94 (9d10 + 45)

\textbf{Movimento} 12 m, scavo 12 m

\textbf{Competenze} Consapevolezza +6

\textbf{Sensi} scurovisione 18 m, percezione tellurica 18 m

\textbf{Linguaggi} -

\textbf{Sfida} 5 (1.800 PX)

\textit{\textbf{Salto da Fermo.}} Un bulette può saltare in lungo fino a 9 metri e in alto fino a 5 m con o senza la rincorsa.

\textbf{Azioni}

\textit{\textbf{Morso.} Attacco con arma da mischia}: +11 a colpire, portata 1 m, un bersaglio.

\textit{Colpisce:} 30 (4d12 + 4) danni perforanti.

\textit{\textbf{Salto Letale.}} Se il bulette può saltare di almeno 4 metri come parte del suo movimento, può usare poi questa azione per atterrare in piedi in uno spazio che contiene una o più creature. Ciascuna di queste creature deve riuscire un Tiro Salvezza di Tempra o Riflessi DC 16 (a scelta del bersaglio) o venire gettata prona e subire 14 (3d6 + 4) danni da botta più 14 (3d6 + 4) danni taglienti. Se il Tiro Salvezza riesce, la creatura subisce solo la metà dei danni, non è gettata prona, e viene spinta di 1 metro fuori dello spazio del bulette in uno spazio non occupato a scelta della creatura. Se non ci sono spazi non occupati a gittata, la creatura cade prona nello spazio del bulette.

\textbf{Ecologia}\\
Ambiente: Colline Temperate\\
Organizzazione: Solitario o coppia\\
\textbf{Tesoro}: Nessuno\\
\textbf{Descrizione}\\
Creazione di uno sconosciuto mago del passato, il bulette ora è diventato un feroce predatore di collina. Scavando rapidamente sotto il terreno, fende la superficie con la sua pinna dorsale lasciandosi dietro una scia caratteristica. Il bulette balza fuori, liberandosi da pietre e terriccio, per fare a pezzi la sua preda senza rimorsi, dando così origine al suo soprannome di "squalo terrestre".

I bulette sono noti per il pessimo carattere, e attaccano creature molto più grandi di loro senza alcuna paura. Bestie solitarie tranne per le occasionali coppie in fase riproduttiva, passano la maggior parte del tempo pattugliando i loro territori, che possono superare i 4 km2, cacciando e punendo gli intrusi con una furia in grado di scuotere i pendii delle colline.

I bulette sono macchine perfette per divorare e distruggere ossa, armature e anche oggetti magici con le loro possenti mascelle e l'acido ribollente del loro stomaco. In mancanza d'altro, un bulette potrebbe sgranocchiare oggetti comuni, ma per qualche ragione non mangia volontariamente carne di elfo, segno forse di un coinvolgimento della magia elfica nella loro creazione, o di nani, anche se può far strage dei membri di entrambe le razze. Gli Halfling, invece, sono tra i cibi preferiti di queste bestie, e non ci sono Halfling assennati che si avventurino nel territorio di un bulette a cuor leggero.

Il bulette è un combattente astuto, e sorprende i nemici con agilità impressionante. Una delle sue tattiche preferite è lanciarsi alla carica e balzare sulla preda attaccando con i suoi artigli affilati come rasoi. Si dice che la carne dietro la cresta dorsale della bestia sia particolarmente tenera, e che quanti vogliano o riescano ad attendere che la pinna venga sollevata nella concitazione del combattimento o dell'accoppiamento possano tentare di sferrare un colpo mortale in quel punto, anche se la maggior parte di quelli che hanno affrontato uno squalo terrestre concordano sul fatto che il miglior modo per vincere un combattimento con un bulette sia evitarlo del tutto.

\medskip\index[Mostruario]{Cavaliere Nero}\textbf{Cavaliere Nero}

\textit{Media non morto, caotico malvagio}

\textbf{FORZA} +5

\textbf{DESTREZZA} +1

\textbf{COSTITUZIONE} +5

\textbf{INTELLIGENZA} +1

\textbf{SAGGEZZA} +2

\textbf{CARISMA} +3

\textbf{Iniziativa} +3 -- \textbf{Difesa} 28

\textbf{Punti Ferita} 171 (18d8+90)

\textbf{Movimento} 9 metri

\textbf{Tiri Salvezza}: Tempra +22, Riflessi +18, Volontà +20

\textbf{Competenze} Intimidire +12, Religione +8, Conoscenza Piani +8, Conoscenza Arcana +5

\textbf{Resistenze al Danno} freddo, fulmine

\textbf{Immunità al Danno} Vuoto, Veleno; armi +1

\textbf{Immunità alle Condizioni} affascinato, avvelenato, paralizzato, affaticamento, spaventato

\textbf{Sensi} Scurovisione 36 m

\textbf{Linguaggi} Comune, Abissale

\textbf{Sfida} 18 (20000 PX)

\textit{\textbf{Incantesimi.}} Il Cavaliere Nero ha CM 7. La sua caratteristica da incantatore è il Carisma, +3 a colpire con attacchi da incantesimo. Il Cavaliere Nero conosce i seguenti incantesimi:

livello 1 (4 slot): \textit{Comando, dardo incantato, mani brucianti, scudo}

livello 2 (3 slot): \textit{blocca persona, arma magica}

livello 3 (3 slot): \textit{controincantesimo, dissolvi magie, palla di fuoco}

livello 4 (3 slot): \textit{esilio, Punizione marchiante (con 1 critico magico automatico, danno da Vuoto)}

\textit{\textbf{Natura Non Morta.}} Il Cavaliere Nero non ha bisogno di aria, cibo, bevande o sonno.

\textit{\textbf{Resistenza Leggendaria (1/Giorno).}} Se il Cavaliere Nero fallisce un Tiro Salvezza, può scegliere invece di riuscirvi.

\textit{\textbf{Resistenza allo Scacciare.}} Il Cavaliere Nero ha +1d6 ai Tiri Salvezza contro gli effetti che scacciano i non morti.

\textbf{Azioni}

\textit{\textbf{Multiattacco.} 3 attacchi con spada lunga +3}: +27 al colpire, portata 1 m, fino a tre creature differenti, oppure 1 colpo di spada con Corruzione

\textit{Colpisce:} 13 (1d10+5+3) danni da taglio + Punizione Marchiante (2d6 da vuoto)

\textit{Corruzione:} 15 (1d10+10) danni da taglio. L'obiettivo deve fare un Tiro Salvezza su Volontà DC 18 oppure perdere un 1/10 di un punto Tratto legato ad un Patrono buono se presente.

\textbf{Ecologia}\\
Ambiente: Qualsiasi\\
Organizzazione: Solitario\\
\textbf{Tesoro}: spada lunga +3 od armatura completa +3, il resto dell'equipaggiamento scompare con la morte del Cavaliere Nero.\\

\textbf{Descrizione}
Dannato fin nel profondo della sua anima il Cavaliere Nero è l'antitesi del cavaliere di Sumkjr, anzi spesso nasce dalla corruzione di un cavaliere di Sumkjr. Avversario temibile, furbo, tattico, adora comportarsi e ragionare, malignamente, come una persona ancora viva. La sua tattica è di lanciare Punizione Marchiante prima di incominciare a combattere e poi lanciare Palla di fuoco appena possibile.

\medskip\index[Mostruario]{Centauro}\textbf{Centauro}

\textit{Grande mostruosità, neutrale buono}

\textbf{FORZA} +4

\textbf{DESTREZZA} +2

\textbf{COSTITUZIONE} +2

\textbf{INTELLIGENZA} -1

\textbf{SAGGEZZA} +1

\textbf{CARISMA} +0

\textbf{Iniziativa} +2 -- \textbf{Difesa} 13

\textbf{Punti Ferita} 45 (6d10 + 12)

\textbf{Movimento} 15 m

\textbf{Competenze} Acrobatica +6, Consapevolezza +3, Sopravvivenza +3

\textbf{Linguaggi} Elfico, Silvano

\textbf{Sfida} 2 (450 PX)

\textit{\textbf{Carica.}} Se il centauro si muove di almeno 9 metri diretto verso il bersaglio e colpisce con un attacco di picca durante lo stesso turno, il bersaglio subisce 10 (3d6) danni perforanti aggiuntivi.

\textbf{Azioni}

\textit{\textbf{Multiattacco.}} Il centauro effettua due attacchi: uno con la picca e uno con gli zoccoli o due con l'arco lungo.

\textit{\textbf{Picca.} Attacco con arma da mischia}: +6 a colpire, portata 3 m, un bersaglio.

\textit{Colpisce:} 9 (1d10 + 4) danni perforanti.

\textit{\textbf{Zoccoli.} Attacco con arma da mischia}: +6 a colpire, portata 1 m, un bersaglio.

\textit{Colpisce:} 11 (2d6 + 4) danni da botta.

\textit{\textbf{Arco Lungo.} Attacco con arma a Distanza}: +4 a colpire, gittata 45m, un bersaglio.

\textit{Colpisce:} 6 (1d8 + 2) danni perforanti.

\textbf{Ecologia}\\
Ambiente: Pianure e foreste temperate\\
Organizzazione: Solitario, coppia, banda (3-10), tribù (11-30 più 3 cacciatori di 3° livello e 1 capo di 6° livello)\\
\textbf{Tesoro}: Standard (Corazza di Piastre, Scudo Pesante di Metallo, Spada Lunga, Lancia, altro tesoro)\\
\textbf{Descrizione}\\
Leggendari cacciatori e abili guerrieri, i centauri sono in parte uomini e in parte cavalli. Generalmente collocata ai margini della civilizzazione, questa stoica popolazione varia enormemente come aspetto: di solito il colore della pelle è molto abbronzato ma simile a quello degli umani delle regioni limitrofe, mentre la parte inferiore del corpo ha le tonalità degli equini locali. Hanno capelli e occhi di colore scuro e i tratti del volto piuttosto marcati, mentre la loro stazza totale dipende dalla taglia del cavallo di cui hanno la parte inferiore del corpo. Quindi, anche se un centauro medio è alto in piedi 2,1 metri e pesa più di 1000 kg, esistono molteplici varianti regionali, dagli esili corridori delle pianure ai massicci cacciatori di montagna.

I centauri vivono in media circa 60 anni. Distanti dalle altre razze e in conflitto con gli altri della loro specie, i centauri sono una razza antica che lentamente comincia ad accettare il mondo moderno. Anche se la maggioranza dei centauri vive ancora in tribù vagando per vaste pianure o ai margini di mistiche foreste, alcuni hanno abbandonato i modi isolazionisti dei loro antenati per stabilirsi in città cosmopolite. Spesso questi spiriti liberi sono considerati dei reietti e vengono disprezzati dalle loro tribù, e pertanto la decisione di abbandonarle è una scelta pesante. In alcuni casi, comunque, intere tribù guidate da capi progressisti hanno cominciato a commerciare o stringere alleanze con altre comunità di umanoidi, specie Elfi, a volte Gnomi, e più raramente Umani o Nani. Molte razze rimangono caute nei confronti dei centauri, però, per lo più a causa di leggende che li ritraggono come creature territoriali e feroci e dei periodici scontri violenti che essi hanno con i coloni testardi e i paesi in via di espansione.

\medskip\index[Mostruario]{Chimera}\textbf{Chimera}

\textit{Grande mostruosità, caotico malvagio}

\textbf{FORZA} +4

\textbf{DESTREZZA} +0

\textbf{COSTITUZIONE} +4

\textbf{INTELLIGENZA} -4

\textbf{SAGGEZZA} +2

\textbf{CARISMA} +0

\textbf{Iniziativa} +0 -- \textbf{Difesa} 17

\textbf{Punti Ferita} 114 (12d10 + 48)

\textbf{Movimento} 9 m, volo 18 m

\textbf{Competenze} Consapevolezza +8

\textbf{Sensi} scurovisione 18 m

\textbf{Linguaggi} comprende il Draconico ma non può parlare

\textbf{Sfida} 6 (2.300 PX)

\textbf{Azioni}

\textit{\textbf{Multiattacco.}} La chimera effettua tre attacchi: uno con il morso, uno con le corna e uno con gli artigli. Quando il soffio infuocato è disponibile, può usare il soffio al posto del morso o delle corna.

\textit{\textbf{Artigli.} Attacco con arma da mischia}: +10 a colpire, portata 1 m, un bersaglio.

\textit{Colpisce:} 11 (2d6 + 4) danni taglienti, 1 danno da Sanguinamento.

\textit{\textbf{Corna.} Attacco con arma da mischia}: +10 a colpire, portata 1 m, un bersaglio.

\textit{Colpisce:} 10 (1d12 + 4) danni da botta.

\textit{\textbf{Morso.} Attacco con arma da mischia}: +10 a colpire, portata 1 m, un bersaglio.

\textit{Colpisce:} 11 (2d6 + 4) danni perforanti.

\textit{\textbf{Soffio Infuocato (Ricarica 5-6).}} La testa di drago esala fuoco in un cono di 5 metri. Ogni creatura in quell'area deve effettuare un Tiro Salvezza di Riflessi DC 15 e subire 31 (7d8) danni da fuoco se fallisce il Tiro Salvezza, o la metà di questi danni se lo riesce.

\textbf{Ecologia}\\
Ambiente: Colline Temperate\\
Organizzazione: Solitario, coppia, branco (3-6) o stormo (7-12)\\
\textbf{Tesoro}: Standard\\
\textbf{Descrizione}\\
Le chimere sono mostruose creature nate dal male primordiale. Odiose e fameliche, cacciano sia a terra che in aria. La testa di drago di una chimera può essere di qualunque tipo di drago malvagio, con il soffio corrispondente e le ali generalmente dotate delle stesse scaglie della testa. Le chimere parlano con tre voci che si sovrappongono, ma lo fanno raramente, tipicamente solo per adulare una creatura più potente. Una chimera è alta al garrese 1 metro, raggiungendo la lunghezza di 3 metri e il peso di 350 kg.\\
Le chimere preferiscono la carne, ma possono sopravvivere di vegetali se necessario (anche se quando sono costrette a farlo il loro umore peggiora ulteriormente). Il fatto che volino significa che possono scegliere con attenzione le loro prede, e generalmente cacciano in vaste aree cercando quelle facili. Sono troppo stupide e belligeranti per acquisire seguaci, anche se a volte una tribù di coboldi può far loro delle offerte. Al contrario, sono abbastanza intelligenti e caparbie da essere mediocri animali domestici, e solo una creatura molto più potente di loro può riuscire a sottometterle. Possono formare collaborazioni paritarie con umanoidi rispettosi o creature simili, e acconsentono anche ad essere usate come cavalcature. Un branco di chimere ha una gerarchia simile a quella dei leoni, con un maschio dominante che comanda il gruppo e la maggior parte delle cacce svolte dalle femmine. Una chimera solitaria può essere un giovane maschio solitario o una femmina con i cuccioli nelle vicinanze.


\medskip\index[Mostruario]{Chuul}\textbf{Chuul}

\textit{Grande aberrazione, caotico malvagio}

\textbf{FORZA} +4

\textbf{DESTREZZA} +0

\textbf{COSTITUZIONE} +3

\textbf{INTELLIGENZA} -3

\textbf{SAGGEZZA} +0

\textbf{CARISMA} -3

\textbf{Iniziativa} +0 -- \textbf{Difesa} 18

\textbf{Punti Ferita} 93 (11d10 + 33)

\textbf{Movimento} 9 m, nuoto 9 m

\textbf{Competenze} Consapevolezza +4

\textbf{Immunità ai Danni} veleno

\textbf{Immunità alle Condizioni} avvelenato

\textbf{Sensi} scurovisione 18 m

\textbf{Linguaggi} comprende la Linguaggio delle Profondità ma non può parlare

\textbf{Sfida} 4 (1.100 PX)

\textit{\textbf{Anfibio.}} Il chuul può respirare aria e acqua.

\textit{\textbf{Senso della Magia.}} Il chuul percepisce la magia entro 36 metri da sé. Questo tratto funziona come l'incantesimo \textit{individuazione} \textit{del magico} ma di per sé non è magico.

\textbf{Azioni}

\textit{\textbf{Multiattacco.}} Il chuul effettua due attacchi con le chele. Se il chuul sta afferrando una creatura, può anche usare i suoi tentacoli una volta.

\textit{\textbf{Chele.} Attacco con arma da mischia}: +10 a colpire, portata 3 m, un bersaglio.

\textit{Colpisce:} 11 (2d6 + 4) danni da botta. Un bersaglio è afferrato (DC 14 per fuggire) se è di taglia Grande o inferiore e il chuul non sta già afferrando altre due creature.

\textit{\textbf{Tentacoli.}} Una creatura afferrata dal chuul deve riuscire un Tiro Salvezza di Tempra DC 13 o restare avvelenata per 1 minuto. Fino al termine dell'avvelenamento, il bersaglio è paralizzato. Il bersaglio può ripetere il Tiro Salvezza al termine di ciascun suo round, terminando l'effetto per sé in caso di successo.

\textbf{Ecologia}\\
Ambiente: Paludi Temperate\\
Organizzazione: Solitario, coppia o branco (3-6)\\
\textbf{Tesoro}: Standard\\
\textbf{Descrizione}\\
I chuul sono predatori corazzati simili ai crostacei, sempre in agguato sotto la superficie degli stagni e dei pantani poco profondi, che escono dal loro nascondiglio per afferrare le loro prede con le loro chele e poi paralizzarle con i tentacoli della bocca prima di mangiarle vive.

I chuul sono eccellenti nuotatori, ma preferiscono attaccare le creature terrestri o abituate ad acque poco profonde. Una volta afferrate le loro vittime, i chuul spesso le trascinano nell'acqua profonda. I lucertoloidi sono le prede preferite dei chuul, anche se le pallide specie di chuul che vivono nei sotterranei preferiscono morlock, duergar, incauti elfi e altri sfortunati che si avvicinano troppo ai loro corsi d'acqua sotterranei, ad eccezione dei trogloditi il cui sapore i chuul trovano particolarmente disgustoso.

I chuul sono sorprendentemente intelligenti e molti si impegnano in inutili speculazioni sulle loro origini e motivazioni. Parlano un cinguettante e gorgogliante dialetto del Comune, ma pochi di essi sono inclini a chiacchierare con quanti non siano della loro razza, e se esiste una società chuul al di fuori del frenetico periodo degli amori, nessuno lo ha ancora scoperto. Al contrario, le menti dei chuul sembrano dedite solo alla ricerca del luogo perfetto in cui tendere un'imboscata per attaccare altre creature intelligenti e a come decorare le loro elaborate tane con trofei delle loro vittime. Anche se i chuul sembrano non interessati all'utilizzo di utensili, hanno un bisogno compulsivo di collezionare quelli delle loro vittime. Un tipico chuul è alto 2,4 metri e pesa 325 kg.

\medskip\index[Mostruario]{Coboldo}\textbf{Coboldo}

\textit{Piccola umanoide (coboldo), legale malvagio}

\textbf{FORZA} -2

\textbf{DESTREZZA} +2

\textbf{COSTITUZIONE} -1

\textbf{INTELLIGENZA} -1

\textbf{SAGGEZZA} -2

\textbf{CARISMA} -1

\textbf{Iniziativa} +2 -- \textbf{Difesa} 13

\textbf{Punti Ferita} 5 (2d6 - 2)

\textbf{Movimento} 9 m

\textbf{Sensi} scurovisione 18 m

\textbf{Linguaggi} Comune, Draconico

\textbf{Sfida} 1/8 (25 PX)

\textit{\textbf{Sensibilità alla Luce}}. Mentre è alla luce del sole, il coboldo ha -1d6 ai tiri per colpire, oltre che alle prove di Saggezza (Consapevolezza) basate sulla vista.

\textit{\textbf{Tattiche di Branco.}} Il coboldo ha +1d6 ai tiri per colpire contro una creatura se almeno uno degli alleati del coboldo si trova entro 1 metro dalla creatura e quell'alleato non è inabile.

\textbf{Azioni}

\textit{\textbf{Pugnale.} Attacco con arma da mischia}: +4 a colpire, portata 1 m, un bersaglio.

\textit{Colpisce:} 4 (1d4 + 2) danni perforanti.

\textit{\textbf{Fionda.} Attacco con arma a distanza}: +4 a colpire, gittata 9m, un bersaglio.

\textit{Colpisce:} 4 (1d4 + 2) danni da botta.

\textbf{Ecologia}\\
Ambiente: Foreste temperate o sotterranei\\
Organizzazione: solitario, gruppo (2-4), nido (5-30 più un ugual numero di non combattenti, 1 sergente di 3° livello ogni 20 adulti e 1 capo di 4°-6° livello) o tribù (31-300 più di 35\% di non combattenti, 1 sergente di 3° livello ogni 20 adulti, 2 tenenti di 4° livello, 1 capo di 6°-8° livello e 5-16 Ratti Crudeli)\\
\textbf{Tesoro}: Equipaggiamento da PNG (Armatura di Cuoio, Lancia, Fionda, altro tesoro)\\
\textbf{Descrizione}\\
I coboldi sono creature dell'oscurità, che si incontrano più facilmente in enormi dedali sotterranei o negli angoli bui delle foreste dove il sole non batte mai. A causa della somiglianza fisica, i coboldi si proclamano a gran voce eredi della stirpe draconica e destinati a governare la terra sotto l'ala dei loro grandi cugini divini, ma la maggior parte dei draghi li considera poco più che insetti fastidiosi. Ma, anche se proclamano discendenze divine e l'evidenza del loro destino, i coboldi sono consapevoli della loro debolezza. Codardi ed intriganti, non lottano mai apertamente se possono evitarlo, tendendo invece imboscate e trappole, rintanandosi nei loro dedali dietro una coltre di rozzi ma ingegnosi trabocchetti, o rovesciandosi sul nemico in vaste orde ululanti.

La tonalità dei coboldi varia anche tra i fratelli della stessa covata, spaziando tra i colori dei draghi cromatici, con una predominanza del rosso, e più di rado bianco, verde, blu e nero.

\medskip\index[Mostruario]{Cockatrice}\textbf{Cockatrice}

\textit{Piccola mostruosità, disallineato}

\textbf{FORZA} -2

\textbf{DESTREZZA} +1

\textbf{COSTITUZIONE} +1

\textbf{INTELLIGENZA} -4

\textbf{SAGGEZZA} +1

\textbf{CARISMA} -3

\textbf{Iniziativa} +1 -- \textbf{Difesa} 12

\textbf{Punti Ferita} 27 (6d6 + 6)

\textbf{Movimento} 6 m, volo 12 m

\textbf{Sensi} scurovisione 18 m

\textbf{Linguaggi} -

\textbf{Sfida} 1/2 (100 PX)

\textbf{Azioni}

\textit{\textbf{Morso.} Attacco con arma da mischia}: +3 a colpire, portata 1 m, una creatura.

\textit{Colpisce:} 3 (1d4 + 1) danni perforanti, e il bersaglio deve riuscire un Tiro Salvezza di Tempra DC 11 per non essere magicamente pietrificato. Se fallisce il Tiro Salvezza, la creatura inizia a trasformarsi in pietra ed è intralciata. Al termine del turno successivo deve ripetere il Tiro Salvezza. Se lo riesce, l'effetto ha termine. Se lo fallisce, la creatura è pietrificata per 24 ore.

\textbf{Ecologia}\\
Ambiente: Pianure temperate\\
Organizzazione: Solitario, coppia, squadriglia (3-5) o stormo (6-12)\\
\textbf{Tesoro}: Nessuno\\
\textbf{Descrizione}\\
Stupide, malevole e repellenti, le cockatrici sono evitate dalle altre creature per la loro capacità di trasformare la carne in pietra. Le leggende affermano che la prima cockatrice emerse da un uovo deposto da un gallo e covato da un rospo. Che questa storia sia vera o no, le cockatrici odierne si riproducono fra loro in tane terrificanti e sporche scavate a casaccio da almeno una dozzina di creature chioccianti. I maschi sono molto più numerosi delle femmine in questi stormi, e si distinguono solo per barbigli e creste. Una tipica cockatrice è alta poco più di 60 centimetri e pesa 2,5 kg.

Anche se la loro dieta consiste principalmente di semi e insetti pietrificati (che nello stomaco della creatura fungono sia da gastroliti che da nutrimento), le cockatrici difendono ferocemente il loro territorio da tutto ciò che ritengono una minaccia, e i vagabondaggi dei maschi raminghi in cerca di nuovi luoghi dove costruire tane a volte li portano ad involontari contatti con gli umani, con risultati devastanti.

La strana capacità della cockatrice di trasformare le altre creature in pietra è la sua miglior difesa, e la sua tana è invariabilmente piena di resti dei nemici pietrificati. Per ironia della sorte, tuttavia, donnole e furetti, le creature che più probabilmente finiscono nei nidi delle cockatrici per mangiarne le uova, sembrano completamente immuni a questo effetto. Per ragioni sconosciute, le cockatrici sono sia terrorizzate che furiose con i galli comuni, e c'è la stessa probabilità che fuggano o attacchino quando avviene un confronto.


\medskip\index[Mostruario]{Couatl}\textbf{Couatl}

\textit{Media celestiale, legale buono}

\textbf{FORZA} +3

\textbf{DESTREZZA} +5

\textbf{COSTITUZIONE} +3

\textbf{INTELLIGENZA} +4

\textbf{SAGGEZZA} +5

\textbf{CARISMA} +4

\textbf{Iniziativa} +5 -- \textbf{Difesa} 21

\textbf{Punti Ferita} 97 (13d8 + 39)

\textbf{Movimento} 9 m, volo 9 m

\textbf{Tiri Salvezza} Tempra +9, Riflessi +13, Volontà +14

\textbf{Resistenze al Danno} da Luce

\textbf{Immunità al Danno} da arma non magica

\textbf{Sensi} visione del vero 36 m

\textbf{Linguaggi} tutte, telepatia 36 m

\textbf{Sfida} 4 (1.100 PX)

\textit{\textbf{Armi Magiche.}} Gli attacchi con armi del couatl sono magici.

\textit{\textbf{Incantesimi Innati.}} La caratteristica da incantatore innato del couatl è il Carisma. Il couatl può lanciare questi incantesimi in maniera innata, usando solo componenti verbali:

A volontà: \textit{individuazione del bene e del male, individuazione del magico, individuazione dei pensieri}

3/giorno ciascuno: \textit{benedizione, creare cibo e acqua, cura ferite,} \textit{protezione dai veleni, ristorare inferiore, santuario, scudo} 1/giorno ciascuno: \textit{ristorare superiore, scrutare, sogno}

\textit{\textbf{Mente Protetta.}} Il couatl è immune allo scrutare e qualsiasi effetto che percepisca le sue emozioni, legga i suoi pensieri o individui la sua posizione.

\textbf{Azioni}

\textit{\textbf{Morso.} Attacco con arma da mischia}: +8 a colpire, portata 1 m, una creatura.

\textit{Colpisce:} 8 (1d6 + 5) danni perforanti, e il bersaglio deve riuscire un Tiro Salvezza di Tempra DC 13 o restare avvelenato per 24 ore. Fino al termine dell'avvelenamento, il bersaglio è privo di sensi. Un'altra creatura può effettuare un'azione per risvegliare il bersaglio.

\textit{\textbf{Stritolare.} Attacco con arma da mischia}: +6 a colpire, portata 3 m, una creatura di taglia Media o inferiore.

\textit{Colpisce:} 10 (2d6 + 3) danni da botta, e il bersaglio è afferrato (DC 15 per fuggire). Fino al termine dell'afferrare, il bersaglio è intralciato, e il couatl non può stritolare un altro bersaglio.

\textit{\textbf{Mutare Forma.}} Il couatl può trasformarsi magicamente in un umanoide o bestia il cui grado di sfida sia pari o inferiore al proprio, o tornare alla sua vera forma. Alla morte ritorna alla sua vera forma. Qualsiasi equipaggiamento stia indossando o trasportando viene assorbito o trasportato nella nuova forma (a scelta del couatl).

Nella nuova forma, il couatl mantiene le sue statistiche di gioco e la facoltà di parlare, ma la sua Difesa, metodi di movimento, Forza, Destrezza e altre azioni vengono rimpiazzati da quelli della nuova forma, e ottiene qualsiasi statistica o capacità (Azioni aggiuntive e azioni da tana) possedute dalla sua nuova forma e non dalla sua originale. Se la nuova forma ha un attacco di morso, il couatl può usare il proprio morso nella nuova forma.

\textbf{Ecologia}\\
Ambiente: foreste calde\\
Organizzazione: Solitario, coppia o stormo (3-6)\\
\textbf{Tesoro}: Standard\\
\textbf{Descrizione}\\
I couatl sono servitori di divinità legali buone, anche se alcuni operano in maniera indipendente da qualsiasi entità superiore. Rispettati ed ammirati per la loro saggezza e bellezza, cercano di portare i mortali sulla retta via e usano i loro poteri per combattere il male, specie quelli noti per viaggiare tra i piani. Alcuni couatl sono visti come divinità benevole da società isolate e, anche se i couatl rabbrividiscono al solo pensiero di fingere di essere una divinità, consentono che si perpetuino questi malintesi poiché permettono loro di guidare queste società su sentieri di pace e cooperazione con i loro vicini. Un couatl è lungo circa 3,6 metri, con un'apertura alare di circa 5 metri e pesa 900 kg.

Come esterni nativi, i couatl devono mangiare. Preferiscono gli stessi alimenti dei veri serpenti, come mammiferi e uccelli, anche se è noto che divorano gli umanoidi malvagi. Poiché preferiscono passare il tempo a perseguire i loro intenti anziché cacciare, apprezzano le offerte di cibo, in particolare piccoli cinghiali e volatili. Un couatl talvolta mostra il suo apprezzamento a un avventuriero o a un gruppo che gli ha reso un servizio donandogli 1d4 delle sue brillanti piume colorate. Queste piume ottenute gratuitamente, se usate come componente materiale aggiuntivo, permettono ad un incantatore che lancia Alleato Planare di evocare quello specifico couatl senza pagare il normale costo in oro o altri valori, a condizione che il couatl approvi il servizio richiesto dall'incantatore.

\medskip\index[Mostruario]{Cumulo Strisciante}\textbf{Cumulo Strisciante}

\textit{Grande pianta, disallineato}

\textbf{FORZA} +4

\textbf{DESTREZZA} -1

\textbf{COSTITUZIONE} +3

\textbf{INTELLIGENZA} -3

\textbf{SAGGEZZA} +0

\textbf{CARISMA} -3

\textbf{Iniziativa} -1 -- \textbf{Difesa} 18

\textbf{Punti Ferita} 136 (16d10 + 48)

\textbf{Movimento} 6 m, nuoto 6 m

\textbf{Competenze} Muoversi Silenziosamente / Nascondersi +2

\textbf{Resistenze al Danno} freddo, fuoco

\textbf{Immunità al Danno} fulmine

\textbf{Immunità alle Condizioni} accecato, assordato, affaticamento

\textbf{Sensi} vista cieca 18 m (cieco oltre questo raggio)

\textbf{Linguaggi} -

\textbf{Sfida} 5 (1.800 PX)

\textit{\textbf{Assorbimento dei Fulmini.}} Ogni qual volta il cumulo strisciante subisce danni da fulmine, non subisce danni e recupera un numero di Punti Ferita pari al danno da fulmine inferto.

\textbf{Azioni}

\textit{\textbf{Multiattacco.}} Il cumulo strisciante effettua due attacchi di schianto. Se entrambi gli attacchi colpiscono una creatura di taglia Media o inferiore, il bersaglio è afferrato (DC 14 per fuggire) e il cumulo strisciante usa Avvolgere su di esso.

\textit{\textbf{Schianto.} Attacco con arma da mischia}: +11 a colpire, portata 1 m, un bersaglio.

\textit{Colpisce:} 13 (2d8 + 4) danni da botta.

\textit{\textbf{Avvolgere.}} Il cumulo strisciante avvolge una creatura di taglia Media o inferiore che ha afferrato. Il bersaglio avvolto è accecato, intralciato e impossibilitato a respirare, e deve riuscire un Tiro Salvezza di Tempra DC 14 all'inizio di ciascun turno del tumulo o subire 13 (2d8 + 4) danni da botta. Se il cumulo si muove, il bersaglio avvolto si muove con esso. Il cumulo può avvolgere solo una creatura alla volta.

\textbf{Ecologia}\\
Ambiente: Foreste o Paludi Temperate\\
Organizzazione: Solitario\\
\textbf{Tesoro}: Standard\\
\textbf{Descrizione}\\
I cumuli striscianti, chiamati anche soltanto striscianti, sembrano masse vegetali in decomposizione. Sono piante carnivore intelligenti, con un debole per la carne elfica. Il cervello e gli organi sensoriali si trovano nella parte superiore del corpo. Di solito i cumuli striscianti hanno una circonferenza di 2,4 metri e sono alti da 1,8 a 2,7 metri. Pesano circa 1.900 kg.

I cumuli striscianti sono strane creature, più simili a un groviglio di rampicanti parassiti che ad una singola pianta dotata di radici. Sono onnivori, capaci di trarre sostentamento da qualsiasi cosa, avvinghiandosi agli alberi per succhiarne la linfa, inserendo le radici nel terreno per assorbire nutrienti semplici o consumando la carne e le ossa dalle prede.

I cumuli striscianti sono incredibilmente furtivi nel loro ambiente naturale. Si confondono con il terreno circostante e possono attendere immobili per giorni l'arrivo di una potenziale preda. Possono essere praticamente ovunque ed attaccare in qualsiasi momento senza alcun preavviso e senza curarsi che ci siano o meno sopravvissuti, fintanto che hanno da mangiare.

Di solito i cumuli striscianti conducono un'esistenza nomade e solitaria in profonde foreste e fetide paludi ma possono essere trovati anche sottoterra, in mezzo a boschetti di funghi. Voci preoccupanti parlano di gruppi di cumuli striscianti che si radunano intorno a grandi tumuli nelle profondità di giungle e paludi, spesso durante violente tempeste di fulmini. Il motivo di questo comportamento è sconosciuto, e molti saggi si chiedono se dietro non ci sia uno scopo oscuro ed alieno.

\subsection{Demoni}

\medskip\index[Mostruario]{Balor}\textbf{Balor}

\textit{Enorme immondo (demone), caotico malvagio}

\textbf{FORZA} +8

\textbf{DESTREZZA} +2

\textbf{COSTITUZIONE} +6

\textbf{INTELLIGENZA} +5

\textbf{SAGGEZZA} +3

\textbf{CARISMA} +6

\textbf{Iniziativa} +5 -- \textbf{Difesa} 29

\textbf{Punti Ferita} 262 (21d12 + 126)

\textbf{Movimento} 12 m, volo 24 m

\textbf{Tiri Salvezza} Tempra +29, Riflessi +17, Volontà +25

\textbf{Resistenze al Danno} freddo, fulmine;

\textbf{Immunità al Danno} fuoco, veleno, armi +1

\textbf{Immunità alle Condizioni} avvelenato

\textbf{Vulnerabilità al Danno} ferro freddo

\textbf{Sensi} visione del vero 36 m

\textbf{Linguaggi} Abissale, telepatia 36 m

\textbf{Sfida} 19 (22000 PX)

\textit{\textbf{Armi Magiche.}} Gli attacchi con arma del demone sono magici.

\textit{\textbf{Aura di Fuoco.}} All'inizio di ciascun turno del demone, ciascuna creatura entro 1 metro da lui subisce 10 (3d6) danni da fuoco, e gli oggetti infiammabili che si trovano nell'aura e che non sono indossati o trasportati prendono fuoco. Una creatura che entri a contatto con il demone o lo colpisca con un attacco da mischia mentre si trova entro 1 metro da esso subisce 10 (3d6) danni da fuoco.

\textit{\textbf{Resistenza alla Magia.}} Il demone ha +1d6 ai Tiri Salvezza contro incantesimi e altri effetti magici.

\textit{\textbf{Spasmo Mortale.}} Quando il demone muore, esplode; ciascuna creatura entro 9 metri da esso deve effettuare un Tiro Salvezza di Riflessi DC 20, subendo 70 (20d6) danni da fuoco se fallisce il Tiro Salvezza, o la metà di questi danni se lo riesce. L'esplosione appicca il fuoco agli oggetti infiammabili che non sono indossati o trasportati, e distrugge le armi del demone.

\textbf{Azioni}

\textit{\textbf{Multiattacco.}} Il demone effettua due attacchi: uno con la spada lunga e uno con la frusta.

\textit{\textbf{Frusta.} Attacco con arma da mischia}: +30 a colpire, portata 9 m, un bersaglio.

\textit{Colpisce:} 15 (2d6 + 8) danni taglienti più 10 (3d6) danni da fuoco, e il bersaglio deve riuscire un Tiro Salvezza di Tempra DC 20 o venire trascinato 7 metri verso il demone.

\textit{\textbf{Spada Lunga.} Attacco con arma da mischia}: +30 a colpire, portata 3 m, un bersaglio.

\textit{Colpisce:} 21 (3d8 + 8) danni taglienti più 13 (3d8) danni da fulmine. Se il demone ottiene un colpo critico, tira il danno tre volte, invece che due.

\textit{\textbf{Teletrasporto.}} Il demone si teletrasporta magicamente, insieme a tutto l'equipaggiamento che indossa o trasporta, in uno spazio non occupato e che può vedere entro 36 metri.

\textbf{Ecologia}\\
Ambiente: Qualsiasi (Abisso)\\
Organizzazione: Solitario o banda di guerra (1 Balor e 2-5 Glabrezu)\\
\textbf{Tesoro}: Standard (Spada Lunga Sacrilega+1, Frusta Infuocata+1, altro tesoro)\\
\textbf{Descrizione}\\
Quando la gente sussurra terrificanti racconti di creature demoniache, immagina per lo più un'imponente figura di fuoco e carne, un incubo cornuto armato di frusta e spada fiammeggianti, che vola nella notte in cerca delle sue prede. Il demone che queste persone temono è il Balor, e questa paura è pienamente giustificata, dal momento che pochi demoni possono eguagliare il possente Balor in forza o in brutalità.

Nell'Abisso, i Balor sono per lo più al servizio dei signori dei demoni, in qualità di generali o capitani (quando non si tratti di balor estremamente potenti, noti come signori dei balor). Un balor solitamente comanda vaste legioni di demoni e, sebbene spesso consenta a questi servi bramosi e sbavanti di combattere le sue battaglie, è tutt'altro che un codardo. Se si presenta l'opportunità di unirsi ad uno scontro, sono pochi i balor che scelgono di trattenersi.

Un Balor è alto 4,2 metri e pesa 2.250 kg. Solo le anime mortali più crudeli possono alimentare la creazione di un balor: a differenza degli altri demoni, spesso occorrono numerose anime di potenti malvagi per far nascere un nuovo balor.

\medskip\index[Mostruario]{Demogorgone}\textbf{Demogorgone}

\textit{enorme immondo (principe demone), caotico malvagio}

\textbf{FORZA} +9

\textbf{DESTREZZA} +2

\textbf{COSTITUZIONE} +8

\textbf{INTELLIGENZA} +5

\textbf{SAGGEZZA} +3

\textbf{CARISMA} +7

\textbf{Iniziativa} +5 -- \textbf{Difesa} 35

\textbf{Punti Ferita} 468 (26d10+208)

\textbf{Movimento} 15 metri, nuotare 9m

\textbf{Tiri Salvezza}: Tempra +34, Riflessi +28, Volontà +29

\textbf{Competenze} tutte +15

\textbf{Resistenze al Danno} freddo, fulmine, fuoco

\textbf{Immunità al Danno} Vuoto, Veleno; armi +2

\textbf{Immunità alle Condizioni} affascinato, avvelenato, paralizzato, affaticamento, spaventato

\textbf{Sensi} Visione del vero 40 m

\textbf{Linguaggi} tutti, telepatia 45 m

\textbf{Sfida} 26 (90000 PX)

\textit{\textbf{Incantesimi.}} Il Demogorgone ha CM 20. La sua caratteristica da incantatore è il Carisma, +7 a colpire con attacchi da incantesimo. Il Demogorgon conosce i seguenti incantesimi:

A volontà: Individuazione del magico, Immagine maggiore

livello 3 (4 slot): \textit{dissolvi magie, paura, telecinesi}

livello 4 (1 slot): \textit{immagine proiettata, regressione mentale}

\textit{\textbf{Natura Demoniaca.}} Il Demogorgone non ha bisogno di aria, cibo, bevande o sonno.

\textit{\textbf{Resistenza Leggendaria (3/Giorno).}} Se il Demogorgone fallisce un Tiro Salvezza, può scegliere invece di riuscirvi.

\textit{\textbf{Resistenza allo Scacciare.}} Il Demogorgone ha +1d6 ai Tiri Salvezza contro gli effetti che scacciano i non morti.

\textit{\textbf{Due teste.}} Demogorgone ha +1d6 ai Tiri Salvezza contro essere cieco, sordo, svenuto

\textbf{Azioni}

\textit{\textbf{Multiattacco.} 2 attacchi con tentacolo}: +30, portata 3 metri, una creatura. Tutti gli attacchi di Demogorgone sono considerati magici +2.

\textit{Colpisce:} 35 (4d12 +9) danni da botta. La creatura colpita deve fare un Tiro Salvezza su Tempra a DC 36 od i suoi Punti Ferita massimi scendono dello stesso ammontare.

\textit{\textbf{Sguardo}} Demogorgone fissa una creatura che può vedere entro 40 metri. Il bersaglio deve fare un Tiro Salvezza su Volontà a DC 23.

\textit{Effetto Sguardo:} Demogorgone sceglie uno di questi effetti oppure è a caso:

1. Sguardo Potente. Il bersaglio è svenuto fino al prossimo round o finché il Demogorgon è fuori dalla linea di vista

2. Sguardo Ipnotico. Il bersaglio è in dominato dal Demogorgone che ne stabilisce ogni azione. Questo sguardo necessità dell'utilizzo di entrambe le teste del Demogorgon.

3. Sguardo della Follia. Il bersaglio è sotto l'influenza dell'incantesimo Confusione che permane, senza Tiro Salvezza ulteriore, finché Demogorgone è in area di vista. Il Demogorgone non deve rimanere concentrato per il perdurare dell'effetto.

\textbf{Azioni Aggiuntive}

Il Demogorgone può effettuare 3 azioni aggiuntive, scelte da quelle sottostanti ed una per round solo al termine del round di un altra creatura.

\textbf{Coda.} Il Demogorgone attacca con la coda. +30 to al colpire, portata 5 metri, un obiettivo. Se colpisce 31 Punti Ferita di danno da botta pi+ +4d6 danni da Vuoto

\textbf{Sguardo di Follia.} Demogorgone usa o lo sguardo Potente o lo Sguardo della Follia

\textbf{Ecologia}\\
Ambiente: Abisso\\
Organizzazione: Unico\\
\textbf{Tesoro}: Triplo\\

\textbf{Descrizione}
Demogorgone è un enorme demone, principe dell'abisso e della follia alto circa 5 metri. Appare come un rettiloide bipede con due teste da babbuino, i colli sono lunghi e serpentini come le braccia tentacolari. Le due teste di Demogorgone sono hanno personalità distinte che si detestano. Spesso tentano di dominarsi a vicenda e molte delle storie che riguardano il Demogorgone trattano proprio su come una o l'altra testa cechi di dominare il tutto. Tra il Demogorgone ed Orcus c'é una forte rivalità.


\medskip\index[Mostruario]{Dretch}\textbf{Dretch}

\textit{Piccola immondo (demone), caotico malvagio}

\textbf{FORZA} +0

\textbf{DESTREZZA} +0

\textbf{COSTITUZIONE} +1

\textbf{INTELLIGENZA} -3

\textbf{SAGGEZZA} -1

\textbf{CARISMA} -4

\textbf{Iniziativa} +0 -- \textbf{Difesa} 12

\textbf{Punti Ferita} 18 (4d6 + 4)

\textbf{Movimento} 6 m

\textbf{Resistenze al Danno} freddo, fulmine, fuoco

\textbf{Immunità al Danno} veleno

\textbf{Immunità alle Condizioni} avvelenato

\textbf{Vulnerabilità al Danno} ferro freddo

\textbf{Sensi} scurovisione 18 m

\textbf{Linguaggi} Abissale, telepatia 18 m (funziona solo con le creature che comprendono l'Abissale)

\textbf{Sfida} 1/4 (50 PX)

\textbf{Azioni}

\textit{\textbf{Multiattacco.}} Il demone effettua due attacchi: uno con il morso e uno con gli artigli.

\textit{\textbf{Artigli.} Attacco con arma da mischia}: +2 a colpire, portata 1 m, un bersaglio.

\textit{Colpisce:} 5 (2d4) danni taglienti.

\textit{\textbf{Morso.} Attacco con arma da mischia}: +2 a colpire, portata 1 m, un bersaglio.

\textit{Colpisce:} 3 (1d6) danni perforanti.

\textit{\textbf{Nube Fetida (1/Giorno).}} Un disgustoso gas verde si estende in un raggio di 3 metri dal demone. Il gas si propaga intorno agli angoli, e la sua area è oscurata leggermente. Rimane per 1 minuto o finché non viene disperso da un forte vento. Qualsiasi creatura che inizi il proprio round in quell'area deve riuscire un Tiro Salvezza di Tempra DC 11 o restare avvelenata fino all'inizio del suo prossimo round. Mentre è avvelenato in questo modo, il bersaglio, durante il suo round, può effettuare solo un'azione o un'azione bonus, ma non entrambe, e non può effettuare reazioni.

\textbf{Ecologia}\\
Ambiente: Qualsiasi (Abisso)\\
Organizzazione: Solitario, coppia, banda (3-5), gruppo (6-12) o folla (13+)\\
\textbf{Tesoro}: Nessuno\\
\textbf{Descrizione}\\
Anche il più infimo demone dell'Abisso è pericoloso e possiede la necessità impellente di spargere rovina e sgomento. Il miserabile dretch è tanto orripilante e fetido quanto crudele, anche se non possiede la forza ed il potere per riuscire a soddisfare la sua voglia di brutalizzare gli altri nel suo reame nativo. Lo scopo dell'esistenza dei dretch è quello di servire demoni più potenti come vittime sacrificabili, e solo pochi fortunati riescono a sopravvivere abbastanza a lungo da evolversi.

I dretch sono i bersagli preferiti dai dilettanti in evocazioni abissali. Relativamente deboli e facili da intimorire, i dretch spesso possono essere obbligati a lunghi periodi di servitù utilizzando vaghe promesse di opportunità di sfogare le loro frustrazioni e la loro rabbia contro avversari più deboli. Eppure il potenziale evocatore di dretch farebbe meglio a ricordarsi che questi demoni sono codardi ed infidi quanto gli altri demoni. Un dretch che si trova di fronte un nemico più potente sarà assai lieto di scambiare qualsiasi informazione di cui disponga in cambio della sua miserevole vita.

A differenza della maggior parte dei demoni, la sciatta personalità del dretch ed il suo disprezzo per il lavoro fisico prolungato raramente danno dei risultati. I dretch avanzati sono rari, ma quelli che riescono a trovare la forza in se stessi per diventare più di quello che erano al momento della loro creazione divengono i sovrani poveri dell'Abisso, crudeli ed amareggiati, che regnano su parassiti, anime spezzate, non morti privi di intelletto e altri dretch. I loro imperi sono limitati a tratti abbandonati di fogne sotto città dimenticate, instabili distese paludose evitate dalle menti più sensate ed altri sgraditi angoli dell'Abisso che persino i demoni considerano scomodi o ripugnanti. Eppure per i signori dei dretch questi regni sono i loro imperi, e li difendono con pietosa tenacia.

Un dretch è alto 1,2 metri e pesa 90 kg. I dretch solitamente si formano dalle anime di mortali malvagi ed indolenti: è sufficiente solo un piccolo frammento di anima per dare origine ad una nascita così orripilante. Una sola anima spesso può causare l'apparizione di una piccola armata di dretch, e la vista di un'orda di dretch appena nati che si liberano dalla protomateria pulsante dell'Abisso è al contempo nauseante e terrificante.

\medskip\index[Mostruario]{Glabrezu}\textbf{Glabrezu}

\textit{Grande immondo (demone), caotico malvagio}

\textbf{FORZA} +5

\textbf{DESTREZZA} +2

\textbf{COSTITUZIONE} +5

\textbf{INTELLIGENZA} +4

\textbf{SAGGEZZA} +3

\textbf{CARISMA} +3

\textbf{Iniziativa} +4 -- \textbf{Difesa} 22

\textbf{Punti Ferita} 157 (15d10 + 75)

\textbf{Movimento} 12 m

\textbf{Tiri Salvezza} Tempra +18, Riflessi +4, Volontà +11

\textbf{Resistenze al Danno} freddo, fulmine, fuoco; da arma non magica

\textbf{Immunità al Danno} veleno

\textbf{Immunità alle Condizioni} avvelenato

\textbf{Vulnerabilità al Danno} ferro freddo

\textbf{Sensi} visione del vero 36 m

\textbf{Linguaggi} Abissale, telepatia 36 m

\textbf{Sfida} 9 (5000 PX)

\textit{\textbf{Incantesimi Innati.}} La caratteristica da incantatore del demone è l'Intelligenza. Il demone può lanciare questi incantesimi in maniera innata, senza bisogno di componenti materiali:

A volontà: \textit{dissolvi magie, individuazione del magico, oscurità}

1/giorno ciascuno: \textit{confusione, parola del potere stordire, volare}

\textit{\textbf{Resistenza alla Magia.}} Il demone ha +1d6 ai Tiri Salvezza contro incantesimi e altri effetti magici.

\textbf{Azioni}

\textit{\textbf{Multiattacco.}} Il demone effettua quattro attacchi: due con le chele e due con i pugni. In alternativa, può effettuare due attacchi con le chele e lanciare un incantesimo.

\textit{\textbf{Chela.} Attacco con arma da mischia}: +14 a colpire, portata 3 m, un bersaglio.

\textit{Colpisce:} 16 (2d10 + 5) danni da botta. Se il bersaglio è una creatura di taglia Media o inferiore, è afferrato (DC 15 per fuggire). Il glabrezu possiede due chele, ciascuna delle quali può afferrare un bersaglio.

\textit{\textbf{Pugno.} Attacco in mischia con arma}: +14 a colpire, portata 1 m, un bersaglio.

\textit{Colpisce:} 7 (2d4 + 2) danni da botta.

\textbf{Ecologia}\\
Ambiente: Qualsiasi (Abisso)\\
Organizzazione: Solitario o drappello (1 glabrezu, 1 Succube e 2-5 Vrock)
\textbf{Tesoro}: Standard\\
\textbf{Descrizione}\\
Mentre la Succube è un demone che adesca la sua preda sfruttandone i desideri e le necessità carnali, il glabrezu è un tentatore di altro genere. Feroce e dalla forma bestiale, il glabrezu è in realtà un maestro di inganni e bugie. Con la sua abilità di nascondere la sua vera forma dietro piacenti illusioni, usa la sua magia per esaudire i desideri degli umanoidi mortali, come forma di ricompensa per coloro che soccombono ai suoi inganni e raggiri. Un desiderio esaudito da un glabrezu appaga la necessità di chi lo esprime nel modo più rovinoso possibile, sebbene queste conseguenze possano non rivelarsi immediatamente tali. Un fabbro che fatica ad affermarsi potrebbe desiderare fama ed abilità nella professione scelta, solo per scoprire che il suo miglior patrono è un crudele e sadico omicida che usa le armi per promuovere i propri distruttivi desideri. Un uomo solo che esprime il desiderio di avere una compagna, potrebbe vedere il suo desiderio avverarsi con una sua vecchia fiamma ritornata alla "vita" in forma di vampiro, ed altri esempi di questo tipo. Il glabrezu è assai creativo nel soddisfare i desideri di un mortale.

Un glabrezu è alto 5,4 metri e pesa poco più di 3000 kg. Questi perfidi demoni si originano dalle anime dei traditori, dei falsi e dei sovversivi: anime di mortali che, in vita, giurarono il falso o utilizzarono il tradimento e l'inganno per rovinare le vite altrui.

\medskip\index[Mostruario]{Hezrou}\textbf{Hezrou}

\textit{Grande immondo (demone), caotico malvagio}

\textbf{FORZA} +4

\textbf{DESTREZZA} +3

\textbf{COSTITUZIONE} +5

\textbf{INTELLIGENZA} 5 (-2)

\textbf{SAGGEZZA} +1

\textbf{CARISMA} +1

\textbf{Iniziativa} +3 -- \textbf{Difesa} 20

\textbf{Punti Ferita} 136 (13d10 + 65)

\textbf{Movimento} 9 m

\textbf{Tiri Salvezza} Tempra +16, Riflessi +3, Volontà +9

\textbf{Resistenze al Danno} freddo, fulmine, fuoco; da arma non magica

\textbf{Immunità al Danno} veleno

\textbf{Immunità alle Condizioni} avvelenato

\textbf{Vulnerabilità al Danno} ferro freddo

\textbf{Sensi} scurovisione 36 m

\textbf{Linguaggi} Abissale, telepatia 36 m

\textbf{Sfida} 8 (3.900 PX)

\textit{\textbf{Fetore.}} Qualsiasi creatura che inizi il suo round entro 3 metri dal demone, deve riuscire un Tiro Salvezza di Tempra DC 14 o restare avvelenata fino all'inizio del proprio round. Se riesce il Tiro Salvezza, la creatura è immune al fetore del demone gracidante per 24 ore.

\textit{\textbf{Resistenza alla Magia.}} Il demone ha +1d6 ai Tiri Salvezza contro incantesimi e altri effetti magici.

\textbf{Azioni}

\textit{\textbf{Multiattacco.}} Il demone effettua tre attacchi: uno con il morso e due con gli artigli.

\textit{\textbf{Artiglio.} Attacco con arma da mischia}: +11 a colpire, portata 1 m, un bersaglio.

\textit{Colpisce:} 11 (2d6 + 4) danni taglienti, 2 danni da Sanguinamento.

\textit{\textbf{Morso.} Attacco con arma da mischia}: +11 a colpire, portata 1 m, un bersaglio.

\textit{Colpisce:} 15 (2d10 + 4) danni perforanti.

\textbf{Ecologia}\\
Ambiente: Qualsiasi (Abisso)\\
Organizzazione: Solitario o banda (2-4)\\
\textbf{Tesoro}: Standard\\
\textbf{Descrizione}\\
l'hezrou vive nelle vaste paludi, acquitrini e corsi d'acqua dell'Abisso, a suo agio sia nell'acqua che sulla terraferma. La presenza di un hezrou ha un effetto dannoso su flora, causando nodosità e mutazioni, e acque circostanti, rendendole maleodoranti e dal sapore salmastro, peculiarità più facilmente individuabili nel Piano Materiale che nell'Abisso. L'esposizione prolungata a questa corruzione causa orrende trasformazioni e deformità. Spesso intere comunità isolate di mutanti deformi devono il loro aspetto contorto non tanto ai loro depravati costumi quanto alla vicinanza di un hezrou.

Sebbene sia abbastanza intelligente, si può onestamente dire che un hezrou sprechi il proprio intelletto. Questi esseri preferiscono i piaceri più semplici: dormire, il gusto della tortura, la beatitudine di cibarsi di carne vivente o la gioia di sentire qualcosa di bello rompersi e sbriciolarsi nella stretta dei loro pugni. Non cercano spesso di costruire imperi o porsi alla testa di culti, sebbene pochi hezrou rifiuterebbero potenziali seguaci che vengano ad offrirglisi di loro spontanea volontà.

Queste mostruose e bestiali creature nascono dalle anime di mortali malvagi che hanno avvelenato se stessi, i loro parenti o il loro ambiente, ad esempio, drogati, assassini ed alchimisti che non si sono preoccupati di come i loro esperimenti avvelenassero il mondo naturale.

\medskip\index[Mostruario]{Marilith}\textbf{Marilith}

\textit{Grande immondo (demone), caotico malvagio}

\textbf{FORZA} +4

\textbf{DESTREZZA} +5

\textbf{COSTITUZIONE} +5

\textbf{INTELLIGENZA} +4

\textbf{SAGGEZZA} +3

\textbf{CARISMA} +5

\textbf{Iniziativa} +5 -- \textbf{Difesa} 26

\textbf{Punti Ferita} 189 (18d10 + 90)

\textbf{Movimento} 12 m

\textbf{Tiri Salvezza} Tempra +25, Riflessi +18, Volontà +13

\textbf{Resistenze al Danno} freddo, fulmine, fuoco

\textbf{Immunità al Danno} veleno, armi +1

\textbf{Immunità alle Condizioni} avvelenato

\textbf{Vulnerabilità al Danno} ferro freddo

\textbf{Sensi} visione del vero 36 m

\textbf{Linguaggi} Abissale, telepatia 36 m

\textbf{Sfida} 16 (15000 PX)

\textit{\textbf{Armi Magiche.}} Gli attacchi con armi del demone sono magici.

\textit{\textbf{Reattivo.}} Il demone può effettuare una Azione di Reazione durante ciascun turno di combattimento.

\textit{\textbf{Resistenza alla Magia.}} Il demone ha +1d6 ai Tiri Salvezza contro incantesimi e altri effetti magici.

\textbf{Azioni}

\textit{\textbf{Multiattacco.}} Il demone effettua sette attacchi: sei con le spade lunghe e uno con la coda.

\textit{\textbf{Coda.} Attacco con arma da mischia}: +18 a colpire, portata 3 m, una creatura.

\textit{Colpisce:} 15 (2d10 + 4) danni da botta. Se il bersaglio è di taglia Media o inferiore, è afferrato (DC 19 per fuggire). Fino al termine dell'afferrare, il bersaglio è intralciato, e il demone può colpire automaticamente il bersaglio con la coda, ma non può effettuare attacchi di coda contro altri bersagli.

\textit{\textbf{Spada Lunga.} Attacco con arma da mischia}: +18 a colpire, portata 1 m, un bersaglio.

\textit{Colpisce:} 13 (2d8 + 4) danni taglienti.

\textbf{Reazioni}

\textit{\textbf{Parata.}} Il demone somma 5 alla sua Difesa contro un attacco da mischia che lo colpirebbe. Per farlo, il demone deve poter vedere il suo attaccante e impugnare un'arma da mischia.

\textbf{Ecologia}\\
Ambiente: Qualsiasi (Abisso)\\
Organizzazione: Solitario, coppia o plotone (1 marilith, 1-3 Glabrezu e 3-14 Babau)\\
\textbf{Tesoro}: Doppio (6 Spade Lunghe, altro tesoro)\\
\textbf{Descrizione}\\
Sovrane di orde demoniache e regine di nazioni abissali, le temibili marilith servono i signori dei demoni come governanti, consigliere e persino amanti, eppure la loro supremazia come strateghe le rende particolarmente richieste come generali e comandanti d'armate. Le marilith più potenti non sono al servizio di nessuno e comandano invece fameliche legioni demoniache.

Una marilith è alta da 1,8 a 2,7 metri, lunga 6 metri dalla testa alla punta della coda, e pesa 2000 kg. Solo le anime malvagie più arroganti ed orgogliose, solitamente quelle di crudeli sovrani, sadici generali e signori della guerra particolarmente violenti, possono causare la nascita di una marilith.

\medskip\index[Mostruario]{Nalfeshnee}\textbf{Nalfeshnee}

\textit{Grande immondo (demone), caotico malvagio}

\textbf{FORZA} +5

\textbf{DESTREZZA} +0

\textbf{COSTITUZIONE} +6

\textbf{INTELLIGENZA} +4

\textbf{SAGGEZZA} +1

\textbf{CARISMA} +2

\textbf{Iniziativa} +4 -- \textbf{Difesa} 25

\textbf{Punti Ferita} 184 (16d10 + 96)

\textbf{Movimento} 6 m, volo 9 m

\textbf{Tiri Salvezza} Tempra +22, Riflessi +9, Volontà +21

\textbf{Resistenze al Danno} freddo, fulmine, fuoco; da arma non magica

\textbf{Immunità al Danno} veleno

\textbf{Immunità alle Condizioni} avvelenato

\textbf{Vulnerabilità al Danno} ferro freddo

\textbf{Sensi} scurovisione 36 m

\textbf{Linguaggi} Abissale, telepatia 36 m

\textbf{Sfida} 13 (10000 PX)

\textit{\textbf{Resistenza alla Magia.}} Il demone ha +1d6 ai Tiri Salvezza contro incantesimi e altri effetti magici.

\textbf{Azioni}

\textit{\textbf{Multiattacco.}} Il demone usa, se possibile, Aureola di Orrore. Poi effettua tre attacchi: uno con il morso e due con gli artigli.

\textit{\textbf{Artiglio.} Attacco con arma da mischia}: +18 a colpire, portata 3 m, un bersaglio.

\textit{Colpisce:} 15 (3d6 + 5) danni taglienti, 2 danni da Sanguinamento.

\textit{\textbf{Morso.} Attacco con arma da mischia}: +18 a colpire, portata 1 m, un bersaglio.

\textit{Colpisce:} 32 (5d10 + 5) danni perforanti.

\textit{\textbf{Aureola di Orrore (Ricarica 5-6).}} Il demone emette una luce magica multicolore e scintillante. Ogni creatura entro 5 metri dal demone e che possa vedere la luce, deve riuscire un Tiro Salvezza su Volontà DC 15 o restare spaventata per 1 minuto. Una creatura può ripetere il Tiro Salvezza al termine di ciascun suo round, terminando l'effetto per sé se lo riesce. Se il Tiro Salvezza della creatura riesce o l'effetto ha termine per essa, la creatura è immune all'Aureola di
Orrore del demone gemente per le successive 24 ore.

\textit{\textbf{Teletrasporto.}} Il demone si teletrasporta, insieme a tutto l'equipaggiamento che sta indossando o trasportando, in uno spazio non occupato che possa vedere fino a 36 metri di distanza.

\textbf{Ecologia}
Ambiente: Qualsiasi (Abisso)\\
Organizzazione: Solitario o banda di guerra (1 nalfeshnee, 1 Hezrou e 2-5 Vrock)\\
\textbf{Tesoro}: Standard\\
\textbf{Descrizione}\\
Sono pochi i demoni che comprendono le meccaniche interne che regolano l'Abisso come i nalfeshnee, e non è raro che questi demoni servano l'Abisso stesso invece che un signore dei demoni. Alcuni sovrintendono i reami organici che generano i nuovi demoni, mentre altri custodiscono luoghi di particolare importanza nei recessi nascosti del piano. Spesso il regno di un nalfeshnee nell'Abisso è superiore per forze e dimensioni al più grande dei regni mortali, in quanto questi demoni hanno una predisposizione naturale a governare ed imporre una sorta di ordine al caos dell'Abisso. Gli evocatori mortali spesso li richiamano per il loro folle ma impareggiabile intelletto, esaminando accuratamente gli accordi presi con questi demoni onde evitare eventuali conseguenze nascoste e risvolti non voluti, in quanto un nalfeshnee raramente accetta qualcosa che, in qualche modo contorto, non gli consenta di soddisfare le necessità ed i desideri dell'Abisso.

I nalfeshnee sono alti 6 metri e pesano 4000 kg. Sono creati dalle anime di malvagi mortali avari o bramosi, in particolare di coloro che hanno regnato su imperi di schiavitù, furto, brigantaggio e altri vizi ancora più violenti.

\medskip\index[Mostruario]{Orcus}\textbf{Orcus}

\textit{enorme immondo (principe demone), caotico malvagio}

\textbf{FORZA} +8

\textbf{DESTREZZA} +2

\textbf{COSTITUZIONE} +7

\textbf{INTELLIGENZA} +5

\textbf{SAGGEZZA} +5

\textbf{CARISMA} +7

\textbf{Iniziativa} +5 -- \textbf{Difesa} 30

\textbf{Punti Ferita} 390 (26d8+182)

\textbf{Movimento} 15 metri, volare 15 metri

\textbf{Tiri Salvezza}: Tempra +33, Riflessi +28, Volontà +31

\textbf{Competenze} tutte +13

\textbf{Resistenze al Danno} freddo, fulmine, fuoco

\textbf{Immunità al Danno} Vuoto, Veleno; armi +2

\textbf{Immunità alle Condizioni} affascinato, avvelenato, paralizzato, affaticamento, spaventato

\textbf{Sensi} Visione del vero 40 m

\textbf{Linguaggi} tutti, telepatia 45 m

\textbf{Sfida} 26 (90000 PX)

\textit{\textbf{Incantesimi.}} Orcus ha CM 17. La sua caratteristica da incantatore è il Carisma, +7 a colpire con attacchi da incantesimo. Orcus conosce i seguenti incantesimi:

A volontà: Individuazione del magico, Tocco gelido

livello 3 (3 slot): \textit{Dissolvi incantesimi}

livello 6 (3 slot): \textit{Creare non morti}

livello 9 (1 slot): \textit{Fermare il tempo}

\textit{\textbf{Natura Demoniaca.}} Orcus non ha bisogno di aria, cibo, bevande o sonno.

\textit{\textbf{Resistenza Leggendaria (3/Giorno).}} Se il Orcus fallisce un Tiro Salvezza, può scegliere invece di riuscirvi.

\textit{\textbf{Signore dei non morti.}} Orcus può sempre decidere il tipo di non morto che crea e questo rimane sotto il suo controllo per tempo indefinito, oltretutto può lanciare l'incantesimo in qualsiasi condizione si trovi.

\textbf{Azioni}

\textit{\textbf{Multiattacco.} 2 attacchi con bacchetta}: +30, portata 3 metri, una creatura. Tutti gli attacchi di Orcus sono considerati magici +3.

\textit{Colpisce:} 21 (3d8 + 8) danni da botta + 13 (2d12) da Vuoto

\textit{\textbf{Coda}} Orcus colpisce con la sua coda. +30, portata 3 metri, una creatura

\textit{Colpisce:} 21 (3d8 + 8) danni da botta + 18 (4d8) da Veleno

\textbf{Azioni Aggiuntive}

Il Orcus può effettuare 3 azioni aggiuntive, scelte da quelle sottostanti ed una per round solo al termine del round di un altra creatura.

\textbf{Coda.} Il Orcus attacca con la coda. +30 to al colpire, portata 5 metri, un obiettivo. Se colpisce 21 (3d8 + 8) danni da botta + 18 (4d8) da Veleno

\textbf{Assaggio di Morte.} Orcus lancia l'incantesimo Colpo Infuocato, in maniera blasfema, con danni da Vuoto

\textbf{Ecologia}\\
Ambiente: Abisso\\
Organizzazione: Unico\\
\textbf{Tesoro}: Triplo\\

\textbf{Descrizione}
Orcus è il Principe Demone dei non morti. Predilige la compagnia e servizio dei non morti. Desidera vedere scomparire tutta la vita e questa trasformarsi tutta in una gigantesca necropoli di non morti sotto il suo comando. Orcus ha la testa e le gambe da capra, corna simili a montoni, un corpo gonfio, ali da pipistrello e una lunga coda.


\medskip\index[Mostruario]{Quasit}\textbf{Quasit}

\textit{Minuscola immondo (demone, mutaforma), caotico malvagio}

\textbf{FORZA} -3

\textbf{DESTREZZA} +3

\textbf{COSTITUZIONE} +0

\textbf{INTELLIGENZA} -2

\textbf{SAGGEZZA} +0

\textbf{CARISMA} +0

\textbf{Iniziativa} +3 -- \textbf{Difesa} 14

\textbf{Punti Ferita} 7 (3d4)

\textbf{Movimento} 12 m (3 m, volo 12 m in forma di pipistrello; 12 m, scalata 12 m in forma di centopiedi; 12 m, nuoto 12 m in forma di rospo)

\textbf{Competenze} Muoversi Silenziosamente / Nascondersi +5

\textbf{Resistenze al Danno} freddo, fulmine, fuoco; da arma non magica

\textbf{Immunità al Danno} veleno

\textbf{Immunità alle Condizioni}
avvelenato

\textbf{Sensi} scurovisione 36 m

\textbf{Linguaggi} Abissale, Comune

\textbf{Sfida} 1 (200 PX)

\textit{\textbf{Mutaforma.}} Il demone può usare la sua azione per trasformarsi in una forma bestiale da pipistrello, centopiedi o rospo, o per tornare alla sua vera forma. Le sue statistiche sono le stesse in tutte le forme, sebbene gli attacchi possano variare per alcune di esse. Qualsiasi equipaggiamento stia indossando o trasportando non viene trasformato. Alla morte ritorna alla sua vera forma.

\textit{\textbf{Resistenza alla Magia.}} Il demone ha +1d6 ai Tiri Salvezza contro incantesimi e altri effetti magici.

\textbf{Azioni}

\textit{\textbf{Artigli (Morso in Forma di Bestia).} Attacco con arma da mischia}: +4 a colpire, portata 1 m, un bersaglio. \textit{Colpisce:} 5 (1d4 + 3) danni perforanti. Se il bersaglio è una creatura, deve riuscire un Tiro Salvezza di Tempra DC 10 o subire 5 (2d4) danni da veleno e restare avvelenato per 1 minuto. La creatura può ripetere il Tiro Salvezza al termine di ciascun suo round, ponendo termine all'effetto se lo riesce.

\textit{\textbf{Invisibilità.}} Il demone resta invisibile finché non attacca o termina la sua concentrazione. Qualsiasi cosa che il demone stia trasportando o indossando resta invisibile finché rimane in contatto con il demone.

\textit{\textbf{Spavento (1/Giorno).}} Una creatura scelta dal demone che si trovi entro 6 metri da lui, deve riuscire un Tiro Salvezza su Volontà DC 10 o restare spaventata per 1 minuto. Il bersaglio può ripetere il Tiro Salvezza al termine di ciascun suo round, con -1d6 se il demone è in linea di visuale, ponendo termine all'effetto prematuramente se riesce il Tiro Salvezza.

\textbf{Ecologia}\\
Ambiente: Qualsiasi (Abisso)\\
Organizzazione: Solitario o stormo (2-12)\\
\textbf{Tesoro}: Standard\\
\textbf{Descrizione}\\
Il quasit è forse il demone meno potente, ma non è tra i meno rispettati: persino i quasit si ritengono superiori alle orde di Dretch e, fedeli alla propria natura, i Dretch mancano del coraggio o degli stimoli necessari a dimostrare loro che si sbagliano. Il ruolo primario in vita di un quasit è quello di famiglio al servizio di un incantatore, ma quei quasit che sfuggono a questa umiliante servitù acquisiscono una volontà propria e sono molto più pericolosi. Un quasit tipico è alto 45 centimetri e pesa solo 4 kg.

Unici tra le orde demoniache, i quasit non nascono dalle anime di malvagi mortali deceduti, ma da anime viventi: quando un incantatore cerca di richiamare a sé un quasit come famiglio, la sua anima sfiora l'Abisso ed esso reagisce, creando dalla sua materia un quasit collegato all'anima dell'incantatore e generando un potente legame tra i due.

I quasit appena creati vengono alla luce direttamente nel Piano Materiale, dove diventano famigli e, finché sono soggetti alla volontà del loro padrone, lo odiano e disprezzano, dal momento che possono percepire il pulsare delle sua anima e sanno che potrebbero aspirare a qualcosa di più. Un quasit serve, eppure osserva e vigila nell'attesa di errori che possano costare la vita al suo signore, o meglio, che gli consentano di rivoltarsi contro il proprio padrone. Quando il padrone di un quasit muore, questi può cercare di seguirne l'anima nel Grande Oltre, superando un Tiro Salvezza su Volontà con DC 15. Questo effetto funziona come Spostamento Planare ma influisce solo sul quasit e lo trasporta nell'Abisso, facendo diventare sua l'anima del padrone, in forma di larva, piuttosto che utilizzarla per creare nuove forme di vita demoniache. In questo modo, un quasit può usare l'anima appena catturata per contrattare con abitanti più potenti dei piani inferiori, e magari raggiungere un'abietta "promozione" che lo trasformi in una forma di vita più potente.

Raramente un quasit decide di ignorare la morte del proprio padrone e di rimanere nel Piano Materiale in cerca di altri modi per divertirsi: solitamente insediandosi in un'area urbana dove ci sono molti individui da tormentare.

\medskip\index[Mostruario]{Vrock}\textbf{Vrock}

\textit{Grande immondo (demone), caotico malvagio}

\textbf{FORZA} +3

\textbf{DESTREZZA} +2

\textbf{COSTITUZIONE} +4

\textbf{INTELLIGENZA} -1

\textbf{SAGGEZZA} +1

\textbf{CARISMA} -1

\textbf{Iniziativa} +2 -- \textbf{Difesa} 18

\textbf{Punti Ferita} 104 (11d10 + 44)

\textbf{Movimento} 12 m, volo 18 m

\textbf{Tiri Salvezza} Tempra +13, Riflessi +10, Volontà +6

\textbf{Resistenze al Danno} freddo, fulmine, fuoco; da arma non magica

\textbf{Immunità al Danno} veleno

\textbf{Immunità alle Condizioni} avvelenato

\textbf{Sensi} scurovisione 36 m

\textbf{Linguaggi} Abissale, telepatia 36 m

\textbf{Sfida} 6 (2.300 PX)

\textit{\textbf{Resistenza alla Magia.}} Il demone ha +1d6 ai Tiri Salvezza contro incantesimi e altri effetti magici.

\textbf{Azioni}

\textit{\textbf{Multiattacco.}} Il demone effettua due attacchi: uno con il becco e uno con gli speroni.
o
\textit{\textbf{Becco.} Attacco con arma da mischia}: +12 a colpire, portata 1 m, un bersaglio.

\textit{Colpisce:} 10 (2d6 + 3) danni perforanti.

\textit{\textbf{Speroni.} Attacco con arma da mischia}: +12 a colpire, portata 1 m, un bersaglio.

\textit{Colpisce:} 14 (2d10 + 3) danni taglienti.

\textit{\textbf{Spore (Ricarica 6).}} Una nube di spore tossiche si diffonde in un raggio di 5 metri intorno al demone. Le spore si propagano intorno agli angoli. Ogni creatura in quell'area deve riuscire un Tiro Salvezza di Tempra DC 14 o restare avvelenata. Mentre  avvelenato in questo modo, un bersaglio subisce 5 (1d10) danni da  veleno all'inizio di ciascun suo round. Il bersaglio può ripetere il  Tiro Salvezza al termine di ciascun suo round, ponendo termine  all'effetto se lo riesce. Anche svuotare una fiala di acqua sacra sul  bersaglio pone termine all'effetto.

\textit{\textbf{Strillo Stordente (1/Giorno).}} Il demone emette uno strillo orripilante. Ogni creatura entro 6 metri da esso e che lo possa udire, e non sia un demone, deve riuscire un Tiro Salvezza su Tempra DC 14 o restare stordita fino al termine del prossimo round del demone.

\textbf{Ecologia}\\
Ambiente: Qualsiasi (Abisso)\\
Organizzazione: Solitario, coppia o banda (3-10)\\
\textbf{Tesoro}: Standard\\
\textbf{Descrizione}\\
Profani campioni dell'Abisso, i vrock incarnano tutta la rabbia, l'odio e la violenza di questo reame. Tanto voraci e grottescamente opportunisti quanto il saprofago a cui assomigliano, i vrock si deliziano nello spargimento di sangue, godendo del suono e delle sensazioni derivanti dallo strappare gli intestini ancora pulsanti da una creatura vivente.\\
Un vrock tipico è alto 2,4 metri e pesa 200 kg. Queste creature solitamente si originano dalle anime di malvagi mortali colmi di odio e di collera, in particolare coloro che erano criminali professionisti, mercenari o assassini.



\medskip\index[Mostruario]{Destriero da Incubo}\textbf{Destriero da Incubo}

\textit{Grande immondo, neutrale malvagio}

\textbf{FORZA} +4

\textbf{DESTREZZA} +2

\textbf{COSTITUZIONE} +3

\textbf{INTELLIGENZA} +0

\textbf{SAGGEZZA} +1

\textbf{CARISMA} +2

\textbf{Iniziativa} +2 -- \textbf{Difesa} 15

\textbf{Punti Ferita} 68 (8d10 + 24)

\textbf{Movimento} 18 m, volo 24 m

\textbf{Immunità al Danno} fuoco

\textbf{Linguaggi} comprende Abissale, Comune e Infernale ma non può parlare

\textbf{Sfida} 3 (700 PX)

\textit{\textbf{Conferire Resistenza al Fuoco.}} Il destriero da incubo può conferire resistenza al danno da fuoco a chiunque lo cavalchi.

\textit{\textbf{Illuminazione.}} Il destriero da incubo irradia luce intensa in un raggio di 3 metri e luce fioca per ulteriori 3 metri.

\textbf{Azioni}

\textit{\textbf{Zoccoli.} Attacco con arma da mischia}: +6 a colpire, portata 3 m, un bersaglio.

\textit{Colpisce:} 13 (2d8 + 4) danni da botta più 7 (2d6) danni da fuoco.

\textit{\textbf{Passo Etereo.}} Il destriero da incubo e fino a tre creature consenzienti entro 3 metro da esso possono entrare magicamente nel Piano Etereo dal Piano Materiale e viceversa.

\textbf{Ecologia}\\
Ambiente: Qualsiasi (Abaddon)\\
Organizzazione: Solitario\\
\textbf{Tesoro}: Nessuno\\
\textbf{Descrizione}\\
Gli incubi sono fiammeggianti messaggeri di morte. Permettono solo alle creature più malvagie di cavalcarli, e non sono mai soltanto cavalcature, ma collaborano nella distruzione provocata dai loro cavalieri.

\subsection{Diavoli}

\begin{changemargin}{0.3cm}{0.3cm}\begin{enfasi}{L'inferno è vuoto, tutti i diavoli sono qui. (William Shakespeare, La Tempesta)}\end{enfasi}\end{changemargin}\medskip

\medskip\index[Mostruario]{Diavolo Barbuto}\textbf{Diavolo Barbuto}

\textit{Media immondo (diavolo), legale malvagio}

\textbf{FORZA} +3

\textbf{DESTREZZA} +2

\textbf{COSTITUZIONE} +2

\textbf{INTELLIGENZA} -1

\textbf{SAGGEZZA} +0

\textbf{CARISMA} +0

\textbf{Iniziativa} +2 -- \textbf{Difesa} 15

\textbf{Punti Ferita} 52 (8d8 + 16)

\textbf{Movimento} 9 m

\textbf{Tiri Salvezza} Tempra +9, Riflessi +7, Volontà +3

\textbf{Resistenze al Danno} freddo; da arma non magica o che non sia argentata

\textbf{Immunità al Danno} fuoco, veleno

\textbf{Immunità alle Condizioni} avvelenato

\textbf{Sensi} scurovisione 36 m

\textbf{Linguaggi} Infernale, telepatia 36 m

\textbf{Sfida} 3 (700 PX)

\textit{\textbf{Resistenza alla Magia.}} Il diavolo ha +1d6 ai Tiri Salvezza contro incantesimi e altri effetti magici.

\textit{\textbf{Risoluto.}} Il diavolo non può essere spaventato finché riesce a vedere una creatura alleata entro 9 metri da lui.

\textit{\textbf{Vista del Diavolo.}} La scurovisione del diavolo non è limitata dall'oscurità magica.

\textbf{Azioni}

\textit{\textbf{Multiattacco.}} Il diavolo effettua due attacchi: uno con la barba e uno con il falcione.

\textit{\textbf{Barba.} Attacco con arma da mischia}: +7 a colpire, portata 1 m, una creatura.

\textit{Colpisce:} 6 (1d8 + 2) danni perforanti, e il bersaglio deve riuscire un Tiro Salvezza di Tempra DC 12 o restare avvelenato per 1 minuto. Mentre è avvelenato in questo modo, il bersaglio non può recuperare Punti Ferita. Il bersaglio può ripetere il Tiro Salvezza al termine di ciascun suo round, terminando l'effetto se riesce il Tiro Salvezza.

\textit{\textbf{Falcione.} Attacco con arma da mischia}: +7 a colpire, portata 3 m, un bersaglio.

\textit{Colpisce:} 8 (1d10 + 3) danni taglienti. Se il bersaglio è una creatura, ad esclusione di costrutti e non morti, deve riuscire un Tiro Salvezza su Tempra 12 o perdere 5 (1d10) Punti Ferita all'inizio di ciascun suo round a causa della ferita infernale. Ogni volta che il diavolo colpisce il bersaglio ferito con questo attacco, il danno inflitto dalla ferita aumenta di 5 (1d10). Qualsiasi creatura può effettuare un'azione per bloccare la ferita con una prova riuscita di Saggezza (Pronto Soccorso) DC 12. La ferita si richiude anche nel caso in cui il bersaglio riceva della magia guaritrice.

\textbf{Ecologia}\\
Ambiente: Qualsiasi (Inferno)\\
Organizzazione: Solitario, coppia, squadra (3-10) o truppa (10-40)\\
\textbf{Tesoro}: Standard (Falcione, altro tesoro)\\
\textbf{Descrizione}\\
Guerrieri scelti delle legioni infernali, i diavoli barbuti, o barbazu, combattono selvaggiamente in nome dei loro signori infernali e in battaglia comandano orde brutali di dannati. Si radunano e si addestrano con i loro falcioni forgiati negli inferi, tra le volte del terzo girone dell'Inferno, Erebo, ma ritornano inevitabilmente nel primo girone, Averno, per servire al fianco del temibile signore Barbatos.

I barbazu amano effettuare attacchi di carica con i loro falcioni e cercano di mantenere una distanza di 3 metri tra loro ed i loro avversari, così che possono utilizzare le loro caratteristiche armi ad asta con la massima efficacia. Contro un avversario che ha una portata superiore (oppure è in grado di evitare la tattica preferita del diavolo), gettano i falcioni e si affidano ai loro artigli ed alle orribili barbe. In posizione eretta i diavoli barbuti sono alti più di 1,8 metri (sebbene la posizione accovacciata che tengono in battaglia li faccia spesso sembrare più bassi) e pesano più di 100 kg.


\medskip\index[Mostruario]{Diavolo delle Catene}\textbf{Diavolo delle Catene}

\textit{Media immondo (diavolo), legale malvagio}

\textbf{FORZA} +4

\textbf{DESTREZZA} +2

\textbf{COSTITUZIONE} +4

\textbf{INTELLIGENZA} +0

\textbf{SAGGEZZA} +1

\textbf{CARISMA} +2

\textbf{Iniziativa} +2 -- \textbf{Difesa} 20

\textbf{Punti Ferita} 85 (10d8 + 40)

\textbf{Movimento} 9 m

\textbf{Tiri Salvezza} Tempra +9, Riflessi +4, Volontà +3

\textbf{Resistenze al Danno} freddo; da arma non magica che non siano argentati

\textbf{Immunità al Danno} fuoco, veleno

\textbf{Immunità alle Condizioni} avvelenato

\textbf{Sensi} scurovisione 36 m

\textbf{Linguaggi} Infernale, telepatia 36 m

\textbf{Sfida} 8 (3.900 PX)

\textit{\textbf{Resistenza alla Magia.}} Il diavolo ha +1d6 ai Tiri Salvezza contro incantesimi e altri effetti magici.

\textit{\textbf{Vista del Diavolo.}} La scurovisione del diavolo non è limitata dall'oscurità magica.

\textbf{Azioni}

\textit{\textbf{Multiattacco.}} Il diavolo effettua due attacchi con la catena.

\textit{\textbf{Catena.} Attacco con arma da mischia}: +16 a colpire, portata 3 m, un bersaglio.

\textit{Colpisce:} 11 (2d6 + 4) danni taglienti. Il bersaglio è afferrato (DC 14 per fuggire) se il diavolo non sta già afferrando un'altra creatura. Fino al termine dell'afferrare, il bersaglio è intralciato e subisce 7 (2d6) danni perforanti all'inizio di ciascun suo round.

\textit{\textbf{Animare Catene (Ricarica dopo un 1 ora).}} Fino a quattro catene che il diavolo possa vedere e si trovano entro 18 metri da lui producono dei bordi affilati e si animano sotto il controllo del diavolo, purché quelle catene non siano né indossate né trasportate da qualcun altro.

Ogni catena animata è un oggetto con Difesa 20, 20 Punti Ferita, resistenza ai danni perforanti, e immunità ai danni da suono. Quando il diavolo usa Multiattacco durante il suo round, può usare ciascuna catena animata per effettuare un ulteriore attacco di catena. Una catena animata può afferrare una creatura per conto proprio ma non può effettuare attacchi mentre afferra. Una catena animata ritorna al suo stato inanimato se viene ridotta a 0 Punti Ferita o se il diavolo è reso inabile o muore.

\textbf{Reazioni}

\textit{\textbf{Maschera Snervante.}} Quando una creatura che il diavolo può vedere inizia il proprio round entro 9 metri dal diavolo, il diavolo può creare un'illusione per assomigliare all'amore perduto o un acerrimo rivale di quella creatura. Se la creatura può vedere il diavolo, deve riuscire un Tiro Salvezza di Volontà DC 14 o rimanere spaventata fino al termine del suo round.

\textbf{Ecologia}\\
Ambiente: Qualsiasi\\
Organizzazione: Solitario, coppia, anello (3-6) o catena (7-20)\\
\textbf{Tesoro}: Standard\\
\textbf{Descrizione}
Spesso classificati dai profani tra le fila dei diavoli infernali, i sadomasochistici non sono veri diavoli. Anche se alcuni sono noti per vivere all'Inferno, essi esistono al di fuori delle gerarchie stabilite dagli dei degli inferi e dai suoi arcidiavoli e a volte si possono trovare su altri piani, in particolare sul Piano delle Ombre. Molti suggeriscono che siano nativi dell'Inferno che esisteva prima dell'avvento della stirpe diabolica, anche se altri ipotizzano che siano stati portati sul piano da qualche sadica potenza. Indipendentemente dalle loro origini vagano per i piani assecondano il loro desiderio di causare e ricevere sofferenza, ricercando il dolore attraverso violenti rapimenti e sadiche depravazioni.


\medskip\index[Mostruario]{Diavolo Cornuto}\textbf{Diavolo Cornuto}

\textit{Grande immondo (diavolo), legale malvagio}

\textbf{FORZA} +6

\textbf{DESTREZZA} +3

\textbf{COSTITUZIONE} +5

\textbf{INTELLIGENZA} +1

\textbf{SAGGEZZA} +3

\textbf{CARISMA} +3

\textbf{Iniziativa} +3 -- \textbf{Difesa} 23

\textbf{Punti Ferita} 178 (17d10 + 85)

\textbf{Movimento} 6 m, volo 18 m

\textbf{Tiri Salvezza} Tempra +18, Riflessi +17, Volontà +13

\textbf{Resistenze al Danno} freddo; da botta, perforante e tagliente o che non siano argentati

\textbf{Immunità al Danno} fuoco, veleno, armi +1

\textbf{Immunità alle Condizioni} avvelenato

\textbf{Sensi} scurovisione 36 m

\textbf{Linguaggi} Infernale, telepatia 36 m

\textbf{Sfida} 11 (7.200 PX)

\textit{\textbf{Resistenza alla Magia.}} Il diavolo ha +1d6 ai Tiri Salvezza contro incantesimi e altri effetti magici.

\textit{\textbf{Vista del Diavolo.}} La scurovisione del diavolo non è limitata dall'oscurità magica.

\textbf{Azioni}

\textit{\textbf{Multiattacco.}} Il diavolo effettua tre attacchi da mischia: due con il forcone e uno con la coda. Può usare Scagliare Fiamma al posto di qualsiasi attacco da mischia.

\textit{\textbf{Coda.} Attacco con arma da mischia}: +18 a colpire, portata 3 m, un bersaglio.

\textit{Colpisce:} 10 (1d8 + 6) danni perforanti. Se il bersaglio è una creatura, ad esclusione di costrutti e non morti, deve riuscire un Tiro Salvezza su Tempra 17 o perdere 10 (3d6) Punti Ferita all'inizio di ciascun suo round a causa della ferita infernale. Ogni volta che il diavolo ferisce il bersaglio con questo attacco, il danno inflitto dalla ferita aumenta di 10 (3d6). Qualsiasi creatura può effettuare un'azione per bloccare la ferita riuscendo una prova di Saggezza (Pronto Soccorso) DC 12. La ferita si richiude anche nel caso in cui il bersaglio riceva magia guaritrice.

\textit{\textbf{Forcone.} Attacco con arma da mischia}: +18 colpire, portata 3 m, un bersaglio.

\textit{Colpisce:} 15 (2d8 + 6) danni perforanti.

\textit{\textbf{Pungiglione.} Attacco con arma da mischia}: +18 a colpire, portata 3 m, un bersaglio.

\textit{Colpisce:} 13 (2d8 + 4) danni perforanti più 17 (5d6) danni da veleno, e il bersaglio deve riuscire un Tiro Salvezza di Tempra DC 14, o restare avvelenato per 1 minuto. Il bersaglio può ripetere il Tiro Salvezza al termine di ciascun suo round, terminando l'effetto se lo
riesce.

\textit{\textbf{Scagliare Fiamma.} Attacco con incantesimo a Distanza}: +7 a colpire, gittata 45 m, un bersaglio.

\textit{Colpisce:} 14 (4d6) danni da fuoco. Se il bersaglio è un oggetto infiammabile che non sia indossato o trasportato, prende fuoco.

\textbf{Ecologia}\\
Ambiente: Qualsiasi (Inferno)\\
Organizzazione: Solitario, coppia o stormo (3-10)\\
\textbf{Tesoro}: Standard (Catena Chiodata Sacrilega+1, altro tesoro)\\
\textbf{Descrizione}\\
Tra i più letali guerrieri degli arcidiavoli ed abili comandanti dei diavoli minori, i diavoli cornuti divulgano le regole dell'Inferno dovunque passano. Questi diavoli maggiori sono addestrati, forgiati e riforgiati per essere tra i più implacabili ed obbedienti guerrieri del multiverso. I diavoli cornuti delle truppe degli eserciti infernali sono noti come cornugon, mentre i più grandi tra loro sono chiamati malebranche.

Un diavolo cornuto tipico raggiunge la ragguardevole altezza di 2,7 metri, è dotato di ali con un'apertura di 4,2 metri, e pesa 350 kg.


\medskip\index[Mostruario]{Diavolo della Fossa}\textbf{Diavolo della Fossa}\hypertarget{diavolodellafossa}{}

\textit{Grande immondo (diavolo), legale malvagio}

\textbf{FORZA} +8

\textbf{DESTREZZA} +2

\textbf{COSTITUZIONE} +7

\textbf{INTELLIGENZA} +6

\textbf{SAGGEZZA} +4

\textbf{CARISMA} +7

\textbf{Iniziativa} +6 -- \textbf{Difesa} 29

\textbf{Punti Ferita} 300 (24d10 + 168)

\textbf{Movimento} 9 m, volo 18 m

\textbf{Tiri Salvezza} Tempra +24, Riflessi +21, Volontà +18

\textbf{Resistenze al Danno} freddo; da botta, perforante e tagliente di armi che non siano argentati

\textbf{Immunità al Danno} fuoco, veleno, armi +2

\textbf{Immunità alle Condizioni} avvelenato

\textbf{Sensi} visione del vero 36 m

\textbf{Linguaggi} Infernale, telepatia 36 m

\textbf{Sfida} 20 (25000 PX)

\textit{\textbf{Arma Magica.}} Gli attacchi con arma del diavolo della fossa sono magici.

\textit{\textbf{Aura di Paura.}} Qualsiasi creatura ostile al diavolo che inizi il suo round entro 6 metri da esso, deve effettuare un Tiro Salvezza su Volontà DC 21, a meno che il diavolo non sia inabile. Se fallisce il Tiro Salvezza, la creatura è spaventata fino all'inizio del suo prossimo round. Se il Tiro Salvezza della creatura riesce, la creatura è immune all'Aura di Paura del diavolo per le successive 24 ore.

\textit{\textbf{Incantesimi Innati.}} La caratteristica da incantatore diavolo della fossa è il Carisma. Il diavolo della fossa può lanciare questi incantesimi in maniera innata, senza bisogno di componenti materiali:

A volontà: \textit{individuazione del magico, palla di fuoco}

3/giorno ciascuno: \textit{blocca mostri, muro di fuoco}

\textit{\textbf{Resistenza alla Magia.}} Il diavolo ha +1d6 ai Tiri Salvezza contro incantesimi e altri effetti magici.

\textbf{Azioni}

\textit{\textbf{Multiattacco.}} Il diavolo effettua quattro attacchi: uno con il morso, uno con l'artiglio, uno con la mazza e uno con la coda.

\textit{\textbf{Artiglio.} Attacco con arma da mischia}: +30 a colpire, portata 3 m, un bersaglio.

\textit{Colpisce:} 17 (2d8 + 8) danni taglienti, 3 danni da Sanguinamento (fino ad un massimo di 20).

\textit{\textbf{Coda.} Attacco con arma da mischia}: +30 a colpire, portata 3 m, un bersaglio.

\textit{Colpisce:} 24 (3d10 + 8) danni da botta.

\textit{\textbf{Mazza.} Attacco con arma da mischia}: +30 a colpire, portata 3 m, un bersaglio.

\textit{Colpisce:} 15 (2d6 + 8) danni da botta più 21 (6d6) danni da fuoco.

\textit{\textbf{Morso.} Attacco con arma da mischia}: +30 a colpire, portata 1 m, un bersaglio.

\textit{Colpisce:} 22 (4d6 + 8) danni perforanti. Il bersaglio deve riuscire un Tiro Salvezza di Tempra DC 21 o restare avvelenato. Mentre è avvelenato in questo modo, il bersaglio non può recuperare Punti Ferita, e subisce 21 (6d6) danni da veleno all'inizio di ciascun suo round. Il bersaglio avvelenato può ripetere il Tiro Salvezza al termine di ciascun suo round, terminando l'effetto su di sé.

\textbf{Ecologia}\\
Ambiente: Qualsiasi (Inferno)\\
Organizzazione: Solitario, coppia o concilio (3-9)\\
\textbf{Tesoro}: Doppio\\
\textbf{Descrizione}
Sovrani di reami infernali, generali delle armate dell'Inferno e consiglieri degli arcidiavoli, i diavoli della fossa sono la personificazione del terribile e spaventoso apice della razza diabolica.

Massicci, dal fisico indomito e dotati di ingegnosi intelletti malvagi, questi diabolici tiranni possiedono grande autonomia sia al servizio degli arcidiavoli che nella loro sovranità su distese infernali di schiavi o quando sono impegnati a soggiogare i mondi mortali. Solidi muscoli si tendono sui loro giganteschi corpi, corazzati da spesse placche taglienti capaci di bloccare quasi tutti gli attacchi. Le fauci dotate di zanne grandi come pugnali ed i loro volti bestiali nascondono alcune tra le menti più insidiose dell'Inferno.

Nati nelle profondità di Nessus, il nono e più profondo girone dell'Inferno, i diavoli della fossa vengono creati dai ranghi dei cornugon e dei gelugon solamente dagli arcidiavoli e dai loro duchi. Sebbene molti viaggino fino ai gironi superiori e oltre l'Inferno, al comando delle legioni infernali, la maggior parte rimane nel Nessus, al servizio delle corti dei potenti dell'Inferno o in oscure congreghe dagli innominabili propositi.

I diavoli della fossa sono sempre alti più di 4,2 metri, con una apertura alare di oltre 6 metri ed un peso superiore ai 500 kg.

I diavoli della fossa sono signori del fuoco e prediligono i territori lambiti dalle fiamme. All'Inferno, questa loro predisposizione fa sì che Averno, Dite, Malebolge, Nessus, e Flegetonte siano i gironi che più facilmente ospitano i loro templi-cittadelle avvolti dalle fiamme. Fanatici ossessionati dalla superiorità diabolica e dalla più ferrea obbedienza, i diavoli della fossa, se lasciati agire indisturbati, radunano immensi eserciti, rastrellando le fosse dell'Inferno alla ricerca dei lemure più depravati per trasformarli in veri diavoli. Una volta certi di aver creato le legioni perfette, volgono la loro attenzione ai semipiani ed ai mondi mortali più vulnerabili, pregustandone la conquista.

Servitori degli arcidiavoli o di altri unici signori della guerra infernali, i diavoli della fossa si votano alla loro causa, obbedendo alla volontà dei nobili scelti da qualche Patrono oscuro nella speranza che, un giorno, riescano ad ottenere il favore del Principe dell'Oscurità o dell'Inferno stesso. Pur obbedienti alle gerarchie della propria razza, sono anche severi nel farne rispettare le regole e, se un diavolo della fossa si trovasse a servire un padrone indegno, si riterrebbe in dovere di deporlo. Pertanto, siano essi signori o servitori, i diavoli della fossa incarnano la volontà delle implacabili leggi dell'inferno e si assicurano che solo i diavoli più potenti possano (o osino) prosperare.

Solo i più potenti tra gli incantatori mortali possono od osano evocare un diavolo della fossa. Le reazioni di questo tipo di diavoli all'evocazione sono rapide e premeditate, solitamente caratterizzate da una furia incontenibile all'idea che un essere così insignificante possa sprecare il loro tempo immortale. Chi non riesce a fronteggiarne la bruciante rabbia viene ucciso e la sua anima dannata all'Inferno e posta al servizio del diavolo evocato. Chi riesce a controllare questi diavoli maggiori riesce anche ad intrigarli.

Un diavolo della fossa può servire rispettosamente un signore mortale per secoli, ma il suo scopo rimane sempre lo stesso: corromperne sempre più l'anima, assicurarsi la sua completa dannazione e, quando questi alla fine muore, rivendicarne l'anima ed iniziare il processo per farne un servitore lemure totalmente corrotto.

I diavoli della fossa sono consapevoli di essere immortali e sono abbastanza intelligenti da avere una pazienza incredibilmente disciplinata. Pertanto i diavoli della fossa più antichi vedono nelle loro legioni i volti degli innumerevoli folli che un tempo hanno preteso di ritenersi loro padroni.


\medskip\index[Mostruario]{Diavolo del Ghiaccio}\textbf{Diavolo del Ghiaccio}

\textit{Grande immondo (diavolo), legale malvagio}

\textbf{FORZA} +5

\textbf{DESTREZZA} +2

\textbf{COSTITUZIONE} +4

\textbf{INTELLIGENZA} +4

\textbf{SAGGEZZA} +2

\textbf{CARISMA} +4

\textbf{Iniziativa} +4 -- \textbf{Difesa} 25

\textbf{Punti Ferita} 180 (19d10 + 76)

\textbf{Movimento} 12 m

\textbf{Tiri Salvezza} Tempra +15, Riflessi +14, Volontà +12

\textbf{Resistenze al Danno} da botta, perforante e tagliente di armi che non siano argentate

\textbf{Immunità al Danno} freddo, fuoco, veleno, armi +1

\textbf{Immunità alle Condizioni} avvelenato

\textbf{Sensi} vista cieca 18 m, scurovisione 36 m

\textbf{Linguaggi} Infernale, telepatia 36 m

\textbf{Sfida} 14 (11.500 PX)

\textit{\textbf{Resistenza alla Magia.}} Il diavolo ha +1d6 ai Tiri Salvezza contro incantesimi e altri effetti magici.

\textit{\textbf{Vista del Diavolo.}} La scurovisione del diavolo non è limitata dall'oscurità magica.

\textbf{Azioni}

\textit{\textbf{Multiattacco.}} Il diavolo effettua tre attacchi: uno con il morso, uno con gli artigli e uno con la coda. In alternativa effettua due attacchi: uno con la coda e uno con lancia.

\textit{\textbf{Artigli.} Attacco con arma da mischia}: +21 a colpire, portata 1 m, un bersaglio.

\textit{Colpisce:} 10 (2d4 + 5) danni taglienti più 10 (3d6) danni da freddo, 1 danno da Sanguinamento.

\textit{\textbf{Coda.} Attacco con arma da mischia}: +21 a colpire, portata 3 m, un bersaglio.

\textit{Colpisce:} 12 (2d6 + 5) danni da botta più 10 (3d6) danni da freddo.

\textit{\textbf{Lancia di Ghiaccio.} Attacco con arma da mischia}: +21 a colpire, portata 3 m, un bersaglio.

\textit{Colpisce:} 14 (2d8 + 5) danni perforanti più 10 (3d6) danni da freddo. Se il bersaglio è una creatura, deve riuscire un Tiro Salvezza su Tempra DC 15, o avere per 1 minuto la velocità ridotta di 3 metri; durante ciascun suo round può effettuare solo un'azione o un'azione bonus, ma non entrambe; non può effettuare reazioni. Il bersaglio può ripetere il Tiro Salvezza al termine di ciascun suo round, terminando l'effetto su di sé in caso di successo.

\textit{\textbf{Morso.} Attacco con arma da mischia}: +10 a colpire, portata 1 m, un bersaglio.

\textit{Colpisce:} 12 (2d6 + 5) danni perforanti più 10 (3d6) danni da freddo.

\textit{\textbf{Muro di Ghiaccio (Ricarica 6).}} Il diavolo forma magicamente un muro di ghiaccio opaco su di una superficie solida che possa vedere entro 18 metri da lui. Il muro è spesso 30 centimetri e largo fino a 9 metri per un massimo di 3 metri di altezza, oppure  una cupola semisferica di massimo 6 metri di diametro. Quando la  parete appare, ogni creatura nel suo spazio viene spinta fuori da esso  tramite la via più breve. La creatura sceglie su quale lato del muro  finire, a meno che la creatura non sia inabile. La creatura poi  effettua un Tiro Salvezza di Riflessi DC 17, subendo 35 (10d6) danni da freddo se lo fallisce, o la metà di questi danni se lo riesce.

Il muro rimane per 1 minuto o finché il diavolo non è reso inabile o muore. Il muro può essere danneggiato e bucato; ogni sezione di 3 metri ha Difesa 5, 30 Punti Ferita, vulnerabilità al danno da fuoco, e immunità al danno da acido, freddo, da Vuoto e da veleno. Se una sezione viene distrutta, lascia una patina di aria gelida nello spazio che occupava prima il muro. Ogni volta che una creatura finisce per muoversi attraverso quest'aria gelida durante un turno, consenziente o meno, deve effettuare un Tiro Salvezza di Tempra DC 17, subendo 17 (5d6)danni da freddo se lo fallisce, o la metà di questi danni se lo riesce. L'aria gelida si dissipa quando il resto del muro svanisce.


\medskip\index[Mostruario]{Diavolo d'Ossa}\textbf{Diavolo d'Ossa}

\textit{Grande immondo (diavolo), legale malvagio}

\textbf{FORZA} +4

\textbf{DESTREZZA} +3

\textbf{COSTITUZIONE} +4

\textbf{INTELLIGENZA} +1

\textbf{SAGGEZZA} +2

\textbf{CARISMA} +3

\textbf{Iniziativa} +3 -- \textbf{Difesa} 24

\textbf{Punti Ferita} 142 (15d10 + 60)

\textbf{Movimento} 12 m, volo 12 m

\textbf{Tiri Salvezza} Tempra +12, Riflessi +12, Volontà +7

\textbf{Competenze} Ingannare +7, Percepire Emozioni +6

\textbf{Resistenze al Danno} freddo; da arma non magica o che non sia argentata

\textbf{Immunità al Danno} fuoco, veleno

\textbf{Immunità alle Condizioni} avvelenato

\textbf{Sensi} scurovisione 36 m

\textbf{Linguaggi} Infernale, telepatia 36 m

\textbf{Sfida} 9 (5000 PX)

\textit{\textbf{Resistenza alla Magia.}} Il diavolo ha +1d6 ai Tiri Salvezza contro incantesimi e altri effetti magici.

\textit{\textbf{Vista del Diavolo.}} La scurovisione del diavolo non è limitata dall'oscurità magica.

\textbf{Azioni}

\textit{\textbf{Multiattacco.}} Il diavolo effettua tre attacchi: due con gli artigli e uno con il pungiglione oppure uno con la sua arma inastata uncinata e uno con il pungiglione.

\textit{\textbf{Arma Inastata Uncinata.} Attacco con arma da mischia}: +12 a colpire, portata 3 m, un bersaglio.

\textit{Colpisce:} 17 (2d12 + 4) danni perforanti. Se il bersaglio è una creatura di taglia Enorme o inferiore, è afferrato (DC 14 per fuggire). Fino al termine dell'afferrare, il diavolo non può usare la sua arma inastata su di un altro bersaglio.

\textit{\textbf{Artiglio.} Attacco con arma da mischia}: +12 a colpire, portata 3 m, un bersaglio.

\textit{Colpisce:} 8 (1d8 + 4) danni taglienti, 1 danno da Sanguinamento.

\textit{\textbf{Pungiglione.} Attacco con arma da mischia}: +12 a colpire, portata 3 m, un bersaglio.

\textit{Colpisce:} 13 (2d8 + 4) danni perforanti più 17 (5d6) danni da veleno, e il bersaglio deve riuscire un Tiro Salvezza di Tempra DC 14, o restare avvelenato per 1 minuto. Il bersaglio può ripetere il Tiro Salvezza al termine di ciascun suo round, terminando l'effetto se lo riesce.

\textbf{Ecologia}\\
Ambiente: Qualsiasi (Inferno)\\
Organizzazione: Solitario, squadra (2-3), concilio (4-10) o contingente (1-3 diavoli del ghiaccio, 2-6 diavoli cornuti e 1-4 diavoli d'ossa\\
\textbf{Tesoro}: Standard (Lancia Gelida+1, altro tesoro)\\
\textbf{Descrizione}\\
Strateghi illuminati delle armate dell'Inferno, gli insettoidi diavoli del ghiaccio sono tra le menti più ingegnose e crudeli nelle legioni dell'inferno. Noto come gelugon tra le fila dei diavoli, un diavolo del ghiaccio nasconde nel suo petto un cuore ghiacciato trafugato ad un mortale, che gli permette di prendere decisioni libero da emozioni. Nati nel girone ghiacciato di Cocito, il settimo girone infernale, la maggior parte dei diavoli del ghiaccio migra a Caina, l'ottavo girone, dove complotta per dannare il mondo da corti di gelido acciaio. Sebbene abbiano le sembianze più aliene e mostruose tra tutti i diavoli, a pochi altri viene accordato un maggiore rispetto.

In combattimento un gelugon manda avanti i suoi sottoposti, così da poter valutare le tattiche, i punti di forza e le debolezze dell'avversario nelle retrovie, e fornire loro supporto con le sue capacità magiche, evitando di coglierli nell'area di effetto dei suoi incantesimi: atteggiamento non dovuto ad un senso di cameratismo, bensì alla fredda e logica verità che i suoi alleati possono sopravvivere più a lungo in uno scontro se non sono esposti a fuoco amico. I gelugon sono alti 3,6 metri e pesano approssimativamente 350 kg.


\medskip\index[Mostruario]{Diavolo Spinoso}\textbf{Diavolo Spinoso}

\textit{Piccola immondo (diavolo), legale malvagio}

\textbf{FORZA} +0

\textbf{DESTREZZA} +2

\textbf{COSTITUZIONE} +1

\textbf{INTELLIGENZA} +0

\textbf{SAGGEZZA} +2

\textbf{CARISMA} -1

\textbf{Iniziativa} +2 -- \textbf{Difesa} 14

\textbf{Punti Ferita} 22 (5d6 + 5)

\textbf{Movimento} 6 m, volo 12 m

\textbf{Resistenze al Danno} freddo; da arma non magica o che non sia argentata

\textbf{Immunità al Danno} fuoco, veleno

\textbf{Immunità alle Condizioni} avvelenato

\textbf{Sensi} scurovisione 36 m

\textbf{Linguaggi} Infernale, telepatia 36 m

\textbf{Sfida} 2 (450 PX)

\textit{\textbf{Resistenza alla Magia.}} Il diavolo ha +1d6 ai Tiri Salvezza contro incantesimi e altri effetti magici.

\textit{\textbf{Sorvolare.}} Il diavolo non provoca attacchi di opportunità quando vola via dalla portata di un nemico.

\textit{\textbf{Spine Limitate.}} Il diavolo possiede dodici spine caudali. Le spine usate ricrescono a mezzanotte.

\textit{\textbf{Vista del Diavolo.}} La scurovisione del diavolo non è limitata dall'oscurità magica.

\textbf{Azioni}

\textit{\textbf{Multiattacco.}} Il diavolo effettua due attacchi: uno con il morso e uno con il suo forcone o due con le sue spine caudali.

\textit{\textbf{Forcone.} Attacco con arma da mischia}: +2 a colpire, portata 1 m, un bersaglio.

\textit{Colpisce:} 3 (1d6) danni perforanti.

\textit{\textbf{Morso.} Attacco con arma da mischia}: +2 a colpire, portata 1 m, un bersaglio.

\textit{Colpisce:} 5 (2d4) danni taglienti.

\textit{\textbf{Spina Caudale.} Attacco con arma a Distanza}: +4 a colpire, gittata 6m, un bersaglio.

\textit{Colpisce:} 4 (1d4 + 2) danni perforanti più 3 (1d6) danni da fuoco.

\textbf{Ecologia}\\
Ambiente: Qualsiasi (Inferno)\\
Organizzazione: Solitario, coppia, gruppo (3-5) o plotone (6-11)\\
\textbf{Tesoro}: Standard\\
\textbf{Descrizione}\\
Sentinelle delle volte dell'Inferno, carcerieri delle anime più nere e armi viventi delle forge infernali, i diavoli uncinati, noti ai diabolisti come hamatula, impongono ai dannati i loro ceppi e custodiscono il nefasto operato dei diavoli maggiori. Un hamatula ama sentire il sangue caldo sulle proprie spine e preferisce gettarsi nella mischia quando gli viene offerta l'opportunità di combattere.

Gli hamatula sono collezionisti ed organizzatori, e sono gli alleati favoriti di bramosi evocatori, dal momento che spesso portano con sé tesori tentatori dalle volte dell'Inferno o conoscono il sentiero per ottenere mortali ricchezze. Se lasciati agire liberamente, nei nascondigli di questi diavoli spesso fanno mostra i trofei trafitti di vecchie vittime, appesi come perverse collezioni di insetti su muri insanguinati.

La maggior parte dei diavoli uncinati è alta dai 2,1 metri in su e pesa 150 kg, sebbene i loro corpi asciutti e muscolosi sembrino più grossi per via degli spuntoni in continua crescita che fuoriescono dai loro corpi, taglienti come lame.

\medskip\index[Mostruario]{Erinni}\textbf{Erinni}

\textit{Media immondo (diavolo), legale malvagio}

\textbf{FORZA} +4

\textbf{DESTREZZA} +3

\textbf{COSTITUZIONE} +4

\textbf{INTELLIGENZA} +2

\textbf{SAGGEZZA} +2

\textbf{CARISMA} +4

\textbf{Iniziativa} +3 -- \textbf{Difesa} 24 (armatura di piastre)

\textbf{Punti Ferita} 153 (18d8 + 72)

\textbf{Movimento} 9 m, volo 18 m

\textbf{Tiri Salvezza} Tempra +11, Riflessi +12, Volontà +7

\textbf{Resistenze al Danno} freddo; da arma non magica o che non sia argentata

\textbf{Immunità al Danno} fuoco, veleno

\textbf{Immunità alle Condizioni} avvelenato

\textbf{Sensi} visione del vero 36 m

\textbf{Linguaggi} Infernale, telepatia 36 m

\textbf{Sfida} 12 (8.400 PX)

\textit{\textbf{Armi Diaboliche.}} Gli attacchi con arma dell'erinni sono magici e infliggono 13 (3d8) danni da veleno aggiuntivi quando colpiscono (già incluso negli attacchi).

\textit{\textbf{Resistenza alla Magia.}} L'erinni ha +1d6 ai Tiri Salvezza contro incantesimi e altri effetti magici.

\textbf{Azioni}

\textit{\textbf{Multiattacco.}} L'erinni effettua tre attacchi.

\textit{\textbf{Spada Lunga.} Attacco con arma da mischia}: +17 a colpire, portata 1 m, un bersaglio.

\textit{Colpisce:} 8 (1d8 + 4) danni taglienti, o 9 (1d10 + 4) danni taglienti se usata con due mani, più 13 (3d8) danni da veleno.

\textit{\textbf{Arco Lungo.} Attacco con arma a Distanza}: +17 a colpire, gittata 45m, un bersaglio.

\textit{Colpisce:} 7 (1d8 + 4) danni perforanti più 13 (3d8) danni da veleno, e il bersaglio deve riuscire un Tiro Salvezza di Tempra DC 14 o restare avvelenato. Il veleno rimane finché non viene rimosso da un incantesimo \textit{ristorazione inferiore} o simile.

\textbf{Reazioni}

\textit{\textbf{Parata.}} L'erinni somma 4 alla sua Difesa contro un attacco da mischia che lo colpirebbe. Per farlo, l'erinni deve poter vedere il suo attaccante e impugnare un'arma da mischia.

\textbf{Ecologia}\\
Ambiente: Qualsiasi (Inferno)\\
Organizzazione: Solitario o trio\\
\textbf{Tesoro}: Triplo (Arco Lungo Composito Infuocato+1 [Forza +5], corda, Spada Lunga+1)\\
\textbf{Descrizione}\\
Note con molti nomi, i Caduti, le Ali Cineree e le Furie, i diavoli conosciuti come erinni insultano la loro forma angelica con la loro brama di vendetta e sanguinosa giustizia. Carnefici, non giudici, le erinni volteggiano sopra i cornicioni affilati come lame di Dite, il secondo girone cosmopolita dell'Inferno, sempre attente a cogliere ogni occasione di battaglia, che sia a difesa dell'inferno, per il capriccio dei loro diabolici signori o per l'appassionata chiamata di capricciosi evocatori mortali. Tutte le erinni intrecciano con i loro stessi capelli letali corde viventi, che utilizzano in battaglia per intralciare e sollevare in aria i loro nemici, schernendoli e condannandoli per le loro trasgressioni prima di lasciarli precipitare da grandi altezze.

Le erinni sono angeli bellissimi ed oscuri che accrescono deliberatamente la propria sensualità con cicatrici e lividi. Eppure, nonostante la loro bellezza, le erinni non sono seduttrici: mancano loro la sottigliezza e la pazienza richieste per questa raffinata arte emotiva, poiché preferiscono risolvere i loro problemi con atti di rapida ed atroce violenza. Spesso una erinni tratterrà il suo colpo mortale mentre tenta di uccidere un nemico, solo per prolungarne le sofferenze. La morte è in genere l'unico modo per sfuggire alle attenzioni di una erinni, e quelle più potenti sono abilissime nel tenere i loro nemici in vita ma inermi, così da prolungare il loro tormento, arrivando addirittura a mantenerli vivi con la magia. Si dice che le più potenti torturatrici erinni siano dotate di capacità che permettono alle sofferenze da loro inflitte di perdurare anche dopo la morte del soggetto.  La maggior parte delle erinni è alta poco meno di 1,8 metri e pesa circa 70 kg, e le loro ali nere piumate hanno una apertura superiore ai 3 metri.


\medskip\index[Mostruario]{Imp}\textbf{Imp}

\textit{Minuscola immondo (diavolo, mutaforma), legale malvagio}

\textbf{FORZA} -2

\textbf{DESTREZZA} +3

\textbf{COSTITUZIONE} +1

\textbf{INTELLIGENZA} +0

\textbf{SAGGEZZA} +1

\textbf{CARISMA} +2

\textbf{Iniziativa} +3 -- \textbf{Difesa} 14

\textbf{Punti Ferita} 10 (3d4 + 3)

\textbf{Movimento} 6 m, volo 12 m (6 m in forma di ratto; 6 m, volo 18 m in forma di corvo; 6 m, scalata 6 m in forma di ragno)

\textbf{Tiri Salvezza} Tempra +1, Riflessi +6, Volontà +4

\textbf{Competenze} Muoversi Silenziosamente / Nascondersi +5, Ingannare +4, Percepire Emozioni +3

\textbf{Resistenze al Danno} freddo; da arma non magica o che non sia argentata

\textbf{Immunità al Danno} fuoco, veleno

\textbf{Immunità alle Condizioni} avvelenato

\textbf{Sensi} scurovisione 36 m

\textbf{Linguaggi} Infernale, Comune

\textbf{Sfida} 1 (200 PX)

\textit{\textbf{Mutaforma.}} Il diavolo può usare la sua azione per trasformarsi in una forma bestiale da ratto, corvo o ragno, o per tornare alla sua vera forma. Le sue statistiche sono le stesse in tutte le forme, sebbene gli attacchi possano variare per alcune di esse. Qualsiasi equipaggiamento stia indossando o trasportando non viene trasformato. Alla morte ritorna alla sua vera forma.

\textit{\textbf{Resistenza alla Magia.}} Il diavolo ha +1d6 ai Tiri Salvezza contro incantesimi e altri effetti magici.

\textit{\textbf{Vista del Diavolo.}} La scurovisione del diavolo non è limitata dall'oscurità magica.

\textbf{Azioni}

\textit{\textbf{Pungiglione (Morso in Forma di Bestia).} Attacco con arma da mischia}: +5 a colpire, portata 1 m, una creatura.

\textit{Colpisce:} 5 (1d4 + 3) danni perforanti, e il bersaglio deve effettuare un Tiro Salvezza di Tempra DC 11, subendo 10 (3d6) danni da veleno se lo fallisce, o la metà di questi danni se lo riesce.

\textit{\textbf{Invisibilità.}} Il diavolo resta invisibile finché non attacca o termina la sua concentrazione. Qualsiasi cosa che il diavolo stia trasportando o indossando, resta invisibile finché rimane in contatto con il diavolo.

\textbf{Ecologia}\\
Ambiente: Qualsiasi (Inferno)\\
Organizzazione: Solitario, coppia o stormo (3-10)\\
\textbf{Tesoro}: Standard\\
\textbf{Descrizione}\\
Nati direttamente dalle fosse dell'Inferno, gli imp sono i diavoli meno potenti, anche se queste crudeli ed invadenti creature svolgono un ruolo importante nella corruzione delle anime mortali. Libere dalle gerarchie e dai doveri delle armate infernali, gli imp si dilettano ad ogni opportunità di viaggiare fino al Piano Materiale e di tentare astutamente i mortali, spingendoli a compiere atti sempre più depravati.

Volontariamente al servizio di incantatori nel ruolo di famigli, gli imp recitano la parte dei fedeli servitori, offrendo spesso ai loro padroni astuti consigli ed infernali intuizioni. In realtà, gli imp operano per inviare anime all'Inferno, accertandosi che l'anima del loro padrone, insieme a molte altre, sia destinata alla dannazione dopo la morte.

Gli imp variano molto in aspetto, in un ampio spettro di tratti bestiali e grotteschi, sebbene molti di essi abbiano la forma di un umanoide alato dalla pelle rossiccia, con lineamenti bulbosi. Il tipico imp è alto solamente 60 centimetri, ha un'apertura alare di 90 centimetri e pesa 5 kg.

Un imp su mille è dotato dell'abilità di comunicare telepaticamente con creature entro 15 metri ed il potere di modificare la propria forma in quella di un animale Piccolo o Minuscolo, come per effetto di un incantesimo Forma Ferina II. Questi imp consolari sono assai apprezzati dai diavoli potenti, che li inviano come servitori ai loro seguaci preferiti o per corrompere eroi mortali. Un imp consolare può essere evocato per mezzo del talento Famiglio Migliorato, ma solo da un incantatore di 8° livello o superiore. I diabolisti narrano di altre razze di imp con capacità altrettanto specializzate, ma se queste creature esistono realmente si tratta di casi estremamente rari.

Diversamente dagli altri diavoli, gli imp si ritrovano spesso liberi e soli nel Piano Materiale, in particolare dopo che sono stati evocati per servire come famigli ed i loro padroni sono morti (spesso, indirettamente, a causa delle macchinazioni dell'imp stesso). Senza alcun mezzo per poter fare ritorno a casa questi imp, liberi da ogni legame con padroni arcani, possono diventare pericolosi seccatori o persino porsi a capo di piccole tribù di sanguinosi umanoidi, quali Goblin o Coboldi.


\medskip\index[Mostruario]{Lemure}\textbf{Lemure}

\textit{Media immondo (diavolo), legale malvagio}

\textbf{FORZA} +0

\textbf{DESTREZZA} -3

\textbf{COSTITUZIONE} +0

\textbf{INTELLIGENZA} -5

\textbf{SAGGEZZA} +0

\textbf{CARISMA} -4

\textbf{Iniziativa} -3 -- \textbf{Difesa} 8

\textbf{Punti Ferita} 13 (3d8)

\textbf{Movimento} 5 metri

\textbf{Tiri Salvezza} Tempra +4, Riflessi +3, Volontà +0

\textbf{Resistenze al Danno} freddo

\textbf{Immunità al Danno} fuoco, veleno

\textbf{Immunità alle Condizioni} affascinato, avvelenato, spaventato

\textbf{Sensi} scurovisione 36 m

\textbf{Linguaggi} comprende l'Infernale ma non può parlare

\textbf{Sfida} 0 (10 PX)

\textit{\textbf{Rinvigorimento Diabolico.}} Un lemure che muore nei Nove Inferi ritorna in vita con tutti i suoi Punti Ferita in 1d10 giorni a meno che non venga ucciso da una creatura con tratti buoni su cui sia stato eseguito l'incantesimo \textit{benedire} o i suoi resti vengano
cosparsi di acqua sacra.

\textit{\textbf{Vista del Diavolo.}} La scurovisione del diavolo non è limitata dall'oscurità magica.

\textbf{Azioni}

\textit{\textbf{Pugno.} Attacco con arma da mischia}: +3 a colpire, portata 1 m, un bersaglio.

\textit{Colpisce:} 2 (1d4) danni da botta.

\textbf{Ecologia}\\
Ambiente: Qualsiasi (Inferno)\\
Organizzazione: Solitario, coppia, gruppo (3-5), sciame (6-17) o schiera (10-40 o più)\\
\textbf{Tesoro}: Nessuno\\
\textbf{Descrizione}\\
I più infimi tra i diavoli, i lemure hanno origine dalle fila delle anime condannate all'inferno, masse informi di carne tremolante. La scintilla di istinto o di memoria che sopravvive nella loro coscienza addormentata solitamente dà forma ai loro tratti, che imitano quelli dei suoi torturatori o delle anime torturate che li circondano. Grotteschi ed inutili, i tratti di un lemure non rivelano nulla di quello che è stato un tempo. Molti sfoggiano diversi orribili volti o sono nulla più di colonne ribollenti di carne cancerosa. Solamente i loro arti bitorzoluti, che agitano in continuazione, sembrano funzionare correttamente, e vengono usati solo per distruggere qualsiasi forma di vita non infernale che si avvicini troppo.

I lemure in movimento si consolidano in forme alte più di 1,2 metri e pesanti più di 100 kg, sebbene questi disgustosi diavoli, quando stanno riposando, spesso hanno l'indistinto aspetto di masse di carne disciolta dai tratti deformi.

Sebbene siano tra le più rivoltanti creature esistenti, i lemure rivestono un ruolo vitale nella perversa ecologia dell'Inferno. Quando, al termine della sua esistenza mortale, un'anima viene dannata, sia perché adoratrice di forze diaboliche che per mancanza di fede in altre divinità, essa si unisce alle masse delle anime sofferenti che riempiono le pianure dell'Averno, il primo girone dell'Inferno. Qui iniziano i tormenti, mentre diavoli minori le sospingono assieme ad altri spiriti, preparandole all'arduo viaggio fino ad uno dei gironi dell'inferno più profondi, solitamente uno adatto alla punizione appropriata per i crimini commessi dall'anima, oppure semplicemente verso il dominio di un diavolo che necessita di nuovi schiavi. Una volta giunte nel regno della loro dannazione, le anime affrontano innumerevoli secoli di tormento per mano dei diavoli, di altri esseri malvagi e delle letali macchinazioni dell'Inferno stesso. Mentre l'essenza mortale impazzisce lentamente, queste creature dimenticano le loro vite, divenendo prima selvagge e infine poco più che automi guidati dall'odio e dalla paura. Dopo eoni di questa esistenza, il crudele procedimento dell'Inferno distrugge totalmente l'anima oppure, nel caso degli spiriti più profani, riconsacra questi esseri dimenticati sotto la forma di lemure, la forma di vita più elementare dei diavoli, insensate orde di carne putrescente e diabolica. Questi esseri ripugnanti si radunano in grandi masse, rivoltanti ondate formate da migliaia e migliaia di queste creature.

I diavoli maggiori sono in grado di riconoscere i più corrotti tra loro e, per mezzo di torture misteriose o grazie ai poteri stessi dell'Inferno, le riplasmano in veri diavoli, appena rinati e pronti a servire obbedienti nelle legioni dei dannati.


\subsection{Dinosauri}

\medskip\index[Mostruario]{Plesiosauro}\textbf{Plesiosauro}

\textit{Grande bestia, disallineato}

\textbf{FORZA} +4

\textbf{DESTREZZA} +2

\textbf{COSTITUZIONE} +3

\textbf{INTELLIGENZA} -4

\textbf{SAGGEZZA} +1

\textbf{CARISMA} -3

\textbf{Iniziativa} +2 -- \textbf{Difesa} 14

\textbf{Punti Ferita} 68 (8d10 + 24)

\textbf{Movimento} 6 m, nuoto 12 m

\textbf{Tiri Salvezza} Tempra +18, Riflessi +11, Volontà +9

\textbf{Competenze} Muoversi Silenziosamente / Nascondersi +4, Consapevolezza +3

\textbf{Linguaggi} -

\textbf{Sfida} 2 (450 PX)

\textit{\textbf{Trattenere il Fiato.}} Il plesiosauro può trattenere il fiato per 1 ora.

\textbf{Azioni}

\textit{\textbf{Morso.} Attacco con arma da mischia}: +6 a colpire, portata 3 m, un bersaglio.

\textit{Colpisce:} 14 (3d6 + 4) danni perforanti.

\textbf{Ecologia}\\
Ambiente: Acquatico Caldo\\
Organizzazione: Solitario, coppia o branco (3-6)\\
\textbf{Tesoro}: Nessuno\\
\textbf{Descrizione}\\
Il plesiosauro è un rettile acquatico dal lungo collo. Sebbene tecnicamente non sia un dinosauro, questa creatura ed i suoi simili si trovano spesso a cacciare in laghi ed oceani nei quali è facile trovare dei dinosauri.


\medskip\index[Mostruario]{Tirannosauro}\textbf{Tirannosauro}

\textit{Enorme bestia, disallineato}

\textbf{FORZA} +7

\textbf{DESTREZZA} +0

\textbf{COSTITUZIONE} +4

\textbf{INTELLIGENZA} -4

\textbf{SAGGEZZA} +1

\textbf{CARISMA} -1

\textbf{Iniziativa} +0 -- \textbf{Difesa} 17

\textbf{Punti Ferita} 136 (13d12 + 52)

\textbf{Movimento} 15 m

\textbf{Tiri Salvezza} Tempra +15, Riflessi +12, Volontà +10

\textbf{Competenze} Consapevolezza +4

\textbf{Linguaggi} -

\textbf{Sfida} 8 (3.900 PX)

\textbf{Azioni}

\textit{\textbf{Multiattacco.}} Il tirannosauro effettua due attacchi: uno con il morso e uno con la coda. Non può effettuare entrambi gli attacchi contro lo stesso bersaglio.

\textit{\textbf{Coda.} Attacco con arma da mischia}: +14 a colpire, portata 3 m, un bersaglio.

\textit{Colpisce:} 20 (3d8 + 7) danni da botta.

\textit{\textbf{Morso.} Attacco con arma da mischia}: +14 a colpire, portata 3 m, un bersaglio.

\textit{Colpisce:} 33 (4d12 + 7) danni perforanti. Se il bersaglio è una creatura di taglia Media o inferiore, è afferrato (DC 17 per fuggire). Fino al termine dell'afferrare, il bersaglio è intralciato, e il tirannosauro non può usare il morso contro un altro bersaglio.

\textbf{Ecologia}\\
Ambiente: Foreste e Pianure Calde\\
Organizzazione: Solitario, coppia o branco (3-6)\\
\textbf{Tesoro}: Nessuno\\
\textbf{Descrizione}\\
Il tirannosauro è un predatore primario che misura 12 metri di lunghezza e pesa 7000 kg.


\medskip\index[Mostruario]{Triceratopo}\textbf{Triceratopo}

\textit{Enorme bestia, disallineato}

\textbf{FORZA} +6

\textbf{DESTREZZA} -1

\textbf{COSTITUZIONE} +3

\textbf{INTELLIGENZA} -4

\textbf{SAGGEZZA} +0

\textbf{CARISMA} -3

\textbf{Iniziativa} -1 -- \textbf{Difesa} 16

\textbf{Punti Ferita} 95 (10d12 + 30)

\textbf{Movimento} 15 m

\textbf{Tiri Salvezza} Tempra +15, Riflessi +8, Volontà +5

\textbf{Linguaggi} -

\textbf{Sfida} 5 (1.800 PX)

\textit{\textbf{Carica Travolgente.}} Se il triceratopo si muove di almeno 6 metri diretto verso una creatura e la colpisce con un attacco di incornata durante lo stesso turno, il bersaglio deve riuscire un Tiro Salvezza su Tempra DC 15 o cadere prono. Se il bersaglio è prono, il triceratopo può effettuare un attacco di pestone contro di lui come azione bonus.

\textbf{Azioni}

\textit{\textbf{Incornata.} Attacco con arma da mischia}: +13 a colpire, portata 1 m, un bersaglio.

\textit{Colpisce:} 24 (3d10 + 6) danni perforanti.

\textit{\textbf{Pestone.} Attacco con arma da mischia}: +13 a colpire, portata 1 m, una creatura prona.

\textit{Colpisce:} 22 (3d10 + 6) danni da botta.

\textbf{Ecologia}\\
Ambiente: Pianure Calde\\
Organizzazione: Solitario, coppia o branco (5-8)\\
\textbf{Tesoro}: Nessuno\\
\textbf{Descrizione}\\
Il triceratopo è un erbivoro irascibile e caparbio. Un tipico triceratopo è lungo 9 metri e pesa 10000 kg.

\medskip\index[Mostruario]{Divora Cervelli}\textbf{Divora Cervelli}

\textit{Piccola aberrazione, caotico malvagio}

\textbf{FORZA} +1

\textbf{DESTREZZA} +6

\textbf{COSTITUZIONE} +5

\textbf{INTELLIGENZA} +3

\textbf{SAGGEZZA} +0

\textbf{CARISMA} +3

\textbf{Iniziativa} +10 -- \textbf{Difesa} 22

\textbf{Punti Ferita} 84 (8d8 + 48)

\textbf{Movimento} 12 m

\textbf{Tiri Salvezza} Tempra +7, Riflessi +8, Volontà +8

\textbf{Resistenza al Danno} armi non magiche, freddo, elettricità

\textbf{Immunità al Danno} fuoco

\textbf{Immunità alle Condizioni} incantesimi dalle liste di magia Illusione e Charme

\textbf{Sensi} Vista Cieca 18 m

\textbf{Linguaggi} telepatia 50 m

\textbf{Sfida} 9 (3.900 PX)

\textit{\textbf{Occhi della Magia.}} Il Divora Cervelli ha Individuazione del Magico sempre attivo.

\textit{\textbf{Incantesimi Innati.}} La caratteristica da incantatore del Divora Cervelli è il Carisma. Il Divora Cervelli può lanciare in maniera innata i seguenti incantesimi, senza bisogno di componenti materiali:

A volontà: \textit{Confusione (un unico bersaglio), Infliggi Ferite Gravi, Invisibilità}

3/giorno: \textit{Cura Ferite Moderate, Globo di Invulnerabilità}

\textbf{Azioni}

\textit{\textbf{Multiattacco.}} Il Divora Cervelli può effettuare 4 attacchi, uno per artiglio

\textit{\textbf{Artiglio.} Attacco con arma da mischia}: +9 al a colpire, portata 1 m, una creatura.

\textit{Colpisce:} 3 danni da taglio (1d4+1), 1 danno da Sanguinamento.

\textbf{Abilità speciali}

\textit{\textbf{Furto del corpo}}

Spendendo 3 Azioni un Divora Cervelli può diventare minuscolo e strisciare nella bocca/naso/orecchie di una creatura indifesa o morta ed arrivare al cervello per nutrirsene. Si tratta di una azione che uccide la creatura. Il Divora Cervelli assume il controllo del corpo e lo può usare a suo piacimento, come se controllasse la vittima con un incantesimo Dominare Mostri. Il Divora Cervelli ha pieno accesso a tutte le capacità difensive e offensive dell’ospite tranne che per le capacità magiche e gli incantesimi (anche se il Divora Cervelli può comunque usare le proprie capacità magiche). Un corpo ospite non deve essere morto da più di 1 giorno perché questa capacità funzioni, e anche dopo essere stati occupati con successo i corpi si decompongono diventando inutilizzabili in 7 giorni (a meno che questo periodo venga prolungato con l'incantesimo Riposo Inviolato). Finché il Divora Cervelli occupa il corpo, conosce (e può parlare) i linguaggi conosciuti dalla vittima e le informazioni sulla sua identità e personalità, ma non può possederne gli specifici ricordi e conoscenze. Il danno inflitto al corpo, che ha il doppio dei Punti Ferita originali, ospite non danneggia il Divora Cervelli e se il corpo ospite viene distrutto il Divora Cervelli esce ed è Stordito per 1 round.

\textbf{Ecologia}\\
Ambiente: Qualsiasi sotterraneo\\
Organizzazione: Solitario, covata (2-6) o tribù (7-16)\\
\textbf{Tesoro}: Doppio\\
\textbf{Descrizione}\\
Un Divora Cervelli altro non è che un cervello di circa 50 cm dotato di 4 potenti zampe artigliate.

Ritenuti da qualcuno invasori provenienti da un’altra dimensione o pianeta, i sinistri divora Cervelli sono certamente una tra le razze più crudeli del mondo. Incapaci di provare emozioni o di sguazzare nei peccati del proprio piacere fisico, i divora cervelli sono costretti a rubare corpi per soddisfare la loro golosità, lussuria e crudeltà. Esistono storie che narrano di intere città sotterranee di queste creature che indossano corpi come se fossero vestiti per consumare spaventose orge e macabri festini. Divora Cervelli solitari spesso vivono in rovine o caverne ai margini delle regioni civilizzate per poter fare periodiche scorrerie in città per “acquistare” un nuovo allettante corpo.

Si dice che il giardino di Shayalia sia pieno di Divora Cervelli.

Un Divora Cervelli è lungo 90 cm e pesa circa 30 kg.

\medskip\textbf{Dobi}\\\index[Mostruario]{Dobi}
\textit{Minuscola fatata}\\
\textbf{Forza}: -3\\
\textbf{Destrezza}: -1\\
\textbf{Costituzione}: +2\\
\textbf{Intelligenza}: -2\\
\textbf{Saggezza}: +1\\
\textbf{Carisma}: +3\\
\textbf{Difesa}: 12 -- \textbf{Iniziativa}: +0\\
\textbf{Punti Ferita}: 6 (1d8 + 2)\\
\textbf{Movimento}: 3 m, Nuotare 9 m\\
\textbf{Tiri Salvezza}: Tempra +2, Riflessi +0, Volontà +1 \\
\textbf{Sensi}: visione crepuscolare 18 m \\
\textbf{Lingue}: - \\
\textbf{Sfida} 0 (10 PX)\\
\textbf{Immunità}: al danno delle armi non magiche non da botta\\
\textbf{Resistenza}: danni da taglio, perforazione\\
\textit{\textbf{Dobi}} Il Dobi si appiccica, per spostarlo è necessario essere gentili e chiederglielo.\\
\textit{\textbf{Dobi Dobi Dobi}} Quando il Dobi subisce più di 3 punti ferita di danno con un arma non da botta si divide in due Dobi più piccoli ognuno con lo stesso ammontare di Punti Ferita rimasti al Dobi precedente.\\
\smallskip\textbf{Azioni}\\
\textit{\textbf{Dobi Dobi}} il Dobi proietta un aura di Calmare Emozioni come l'omonimo incantesimo ma non è concesso il Tiro Salvezza. Il Dobi può influenzare una sola creatura alla volta con il suo potere.\\
\textbf{Ecologia}\\
Ambiente: Paludi\\
Organizzazione: gruppo\\
\textbf{Tesoro}: Accidentale\\
\textbf{Descrizione}\\
{\small "...Smossi le foglie dell'acquitrino e vidi a terra una strana palla di pelo, di circa dieci centimetri di diametro, di colore chiaro. Incuriosito lo raccolsi, accarezzando il suo pelo soffice e lo scrutai con attenzione. Sembrava non avere arti o segni di possedere un muso con occhi, orecchie, bocca, ma non appena lo accarezzai la palla vibrò, emettendo uno squittio.

Finalmente scorsi due occhietti neri e vispi aprirsi in tutto quel pelo e poi due orecchiette tonde spuntare, quindi due zampette corte ma robuste, adatte al salto, appoggiate a terra e altre due, sempre corte ma dotate di ben cinque dita ognuna, a mezza altezza.

- Dobi! - rispose l’animaletto, esprimendo una sorta di gioia ed entusiasmo. - Dobi dobi! -.

- Che carino! - esclamai, accarezzandolo. Era l’animaletto più tenero che avessi mai visto. - Ora però ti rimetto giù -.

- Dobi - rispose la palla di pelo.
Portai a terra la mano, ma l'animale non si mosse. Provai a staccarmelo dalla mano, ma rimase appiccicato all'altra. Lo presi con due dita, tirando forte e lo appoggiai veloce a terra, ma subito mi saltò sul piede e vi rimase attaccato. Dovetti attraversare l'acquitrino con il dobi attaccato al piede, senza contare gli altri quattro che trovai avvinghiati all'armatura."

Dal \textit{Viaggio nel primo mondo} di \textbf{Tristan Cassandiel}}


\medskip\index[Mostruario]{Doppelganger}\textbf{Doppelganger}

\textit{Media mostruosità (mutaforma), neutrale}

\textbf{FORZA} +0

\textbf{DESTREZZA} +4

\textbf{COSTITUZIONE} +2

\textbf{INTELLIGENZA} +0

\textbf{SAGGEZZA} +1

\textbf{CARISMA} +2

\textbf{Iniziativa} +4 -- \textbf{Difesa} 16

\textbf{Punti Ferita} 52 (8d8 + 16)

\textbf{Movimento} 9 m

\textbf{Tiri Salvezza} Tempra +4, Riflessi +5, Volontà +6

\textbf{Competenze} Ingannare +6, Percepire Emozioni +3

\textbf{Immunità alle Condizioni} affascinato

\textbf{Sensi} scurovisione 18 m

\textbf{Linguaggi} Comune

\textbf{Sfida} 3 (700 PX)

\textit{\textbf{Mutaforma.}} Il doppelganger può usare la sua azione per cambiare la propria forma in quella di un umanoide Piccolo o Medio che abbia visto, o per tornare alla sua vera forma. Le sue statistiche, a parte la taglia, sono le stesse in tutte le forme. Qualsiasi equipaggiamento stia indossando o trasportando non viene trasformato. Alla morte ritorna alla sua vera forma.

\textit{\textbf{Appostato.}} Nel primo round di combattimento, il doppelganger ha +1d6 ai tiri di attacco contro qualsiasi creatura abbia preso di sorpresa.

\textit{\textbf{Attacco di Sorpresa.}} Se il doppelganger sorprende una creatura e la colpisce con un attacco durante il primo round di combattimento, il bersaglio subisce 10 (3d6) danni aggiuntivi dall'attacco.

\textbf{Azioni}

\textit{\textbf{Multiattacco.}} Il doppelganger effettua due attacchi da mischia.

\textit{\textbf{Schianto.} Attacco con arma da mischia}: +6 a colpire, portata 1 m, un bersaglio.

\textit{Colpisce:} 7 (1d6 + 4) danni da botta.

\textit{\textbf{Leggere Pensieri.}} Il doppelganger legge magicamente i pensieri di superficie di una creatura entro 18 metri da lui. L'effetto può penetrare le barriere, ma 1 metro di legno o terra, 50 centimetri di pietra, 5 centimetri di metallo, o un sottile foglio di piombo lo blocca. Mentre il bersaglio è a gittata, il doppelganger può continuare a leggerne i pensieri, purché la concentrazione del doppelganger non venga infranta (come la concentrazione di un incantesimo). Mentre legge la mente di un bersaglio, il doppelganger ha +1d6 alle prove di Saggezza e Carisma contro il bersaglio.

\textbf{Ecologia}\\
Ambiente: Qualsiasi\\
Organizzazione: Solitario, coppia o banda (3-6)\\
\textbf{Tesoro}: Equipaggiamento da PNG\\
\textbf{Descrizione}\\
I doppelganger sono strani esseri che possono assumere la forma di coloro che incontrano. Nella sua forma naturale, la creatura somiglia più o meno ad un umanoide, ma snello e fragile, con membra magre e tratti facciali non del tutto formati. La sua carnagione è pallida, è glabro ed i suoi occhi sono bianchi e vacui.

I doppelganger preferiscono infiltrarsi in società dove possono ammassare ricchezza e potere, e vedono scarse prospettive nel fondare città coi loro simili. I doppelganger più giovani sperimentano le loro abilità su piccole tribù di orchi o goblin, poi si spostano in società più complesse come comunità naniche, elfiche e umane. Piuttosto che diventare bersagli occupando posizioni di comando, preferiscono mantenere il potere da dietro il trono, o usare molteplici identità per manipolare cittadini influenti o intere gilde.

I doppelganger fanno un uso eccellente della loro mimica naturale per tendere imboscate, trappole con esca e infiltrarsi nelle società umanoidi. Anche se di solito non sono malvagi, sono interessati solo a sé stessi e considerano tutti gli altri come giocattoli da manipolare ed ingannare. Piace loro moltissimo invadere le società umane per soddisfare i loro desideri; alcuni amano i complessi giochi politici mentre altri cercano continuamente di cambiare razza, sesso e partner amorosi. Nonostante non sia la norma, quei doppelganger che usano i loro doni per scopi crudeli e sadici sono molto famosi, e questi mutaforma sono i principali responsabili della sinistra reputazione della loro razza. Certamente, una creatura capace di cambiare forma è avvantaggiata quando cerca di evitare di essere catturata per i suoi crimini, ed alcuni doppelganger particolarmente malevoli godono nel troncare relazioni amorose inscenando tradimenti.

Delle voci insistenti parlano di doppelganger ancor più potenti capaci non solo di cambiare il loro aspetto, ma anche di far proprie abilità, ricordi ed anche capacità straordinarie e soprannaturali delle creature che scelgono di impersonare.


\subsection{Draghi}

Ogni Drago ha pieno accesso a tutti gli incantesimi di una specifica lista di magia a seconda del proprio colore.

Questo accesso è garantito da Tàhil o Ljust a seconda che siano draghi fedeli ad uno o all'altro.

\begin{itemize}
\item Ogni Drago può lanciare incantesimi sino ad un livello pari ad un quarto del suo Grado si Sfida, con un minimo accesso al primo livello.
\item Ogni Drago ha un numero di Punti Magia pari 5 volte il suo Grado di Sfida
\item Ogni Drago ha un punteggio di Competenza Magica pari alla metà del suo Grado di Sfida
\end{itemize}

\medskip

\textbf{Tabella: accesso Lista di Magia per Draghi}\index{Tabella accesso Lista di Magia per Draghi}

\medskip

\begin{tabular}{ll}
\hline
\textbf{Colore del Drago}& \textbf{Nome Lista magia} \\
Bianco& Aria\\
Blu& Aria\\
Giallo& Fuoco, Evocazione\\
Nero&Acqua, Necromanzia\\
Porpora&Terra\\
Rosso&Fuoco\\
Verde&Animali e Piante\\
Argento&Trasmutazione, Illusione\\
Bronzo&Abiurazione\\
Oro&Cura, Evocazione\\
Ottone&Divinazione\\
Rame&Invocazione\\
\end{tabular}

\medskip

Tutti i Draghi hanno accesso alla lista di magia Universale e prediligono certi incantesimi che sono segnati nella loro descrizione.

\subsubsection{Draghi Cromatici}

\medskip\index[Mostruario]{Drago Bianco Antico}\textbf{Drago Bianco Antico}

\textit{Mastodontica drago, caotico malvagio}

\textbf{FORZA} +8

\textbf{DESTREZZA} +0

\textbf{COSTITUZIONE} +8

\textbf{INTELLIGENZA} +0

\textbf{SAGGEZZA} +1

\textbf{CARISMA} +2

\textbf{Iniziativa} +0 -- \textbf{Difesa} 30

\textbf{Punti Ferita} 333 (18x3d6 + 144)

\textbf{Movimento} 12 m, nuoto 12 m, scavo 12 m, volo 24 m

\textbf{Tiri Salvezza} Tempra +19, Riflessi +14, Volontà +16

\textbf{Competenze} Muoversi Silenziosamente / Nascondersi +6, Consapevolezza +13

\textbf{Immunità al Danno} freddo, armi +1

\textbf{Sensi} scurovisione 36 m, vista cieca 18 m

\textbf{Linguaggi} Comune, Draconico

\textbf{Sfida} 20 (25000 PX)

\textit{\textbf{Camminare sul Ghiaccio.}} Il drago può muoversi e arrampicarsi su superfici ghiacciate senza bisogno di effettuare prove di caratteristica. Inoltre, il terreno difficile composto di ghiaccio o neve non gli costa movimento aggiuntivo.

\textit{\textbf{Resistenza Leggendaria (3/Giorno).}} Se il drago fallisce un Tiro Salvezza, può scegliere invece di riuscire.

\textbf{Azioni}

\textit{\textbf{Multiattacco.}} Il drago può usare la sua Presenza Spaventosa. Poi effettuare tre attacchi: uno con il morso e due con gli artigli.

\textit{\textbf{Artiglio.} Attacco con arma da mischia}: +30 a colpire, portata 3 m, un bersaglio.

\textit{Colpisce:} 15 (2d6 + 8) danni taglienti, 3 danni da Sanguinamento (fino ad un massimo di 20).

\textit{\textbf{Coda.} Attacco con arma da mischia}: +30 a colpire, portata 6 m, un bersaglio.

\textit{Colpisce:} 17 (2d8 + 8) danni da botta.

\textit{\textbf{Morso.} Attacco con arma da mischia}: +30 a colpire, portata 5 metri, un bersaglio.

\textit{Colpisce:} 19 (2d10 + 8) danni perforanti più 9 (2d8) danni da freddo.

\textit{\textbf{Presenza Spaventosa.}} Ogni creatura scelta dal drago, che si trovi entro 36 metri da esso e consapevole della sua presenza, deve riuscire un Tiro Salvezza di Volontà DC 16 o restare spaventata per 1 minuto. Una creatura può ripetere il Tiro Salvezza al termine di ciascun suo round, terminando l'effetto se lo riesce. Se il Tiro Salvezza della creatura ha successo o l'effetto ha termine per essa, la creatura è immune alla Presenza Spaventosa del drago per le successive 24 ore.

\textit{\textbf{Soffio Gelido (Ricarica 5-6).}} Il drago esala un'esplosione di ghiaccio in un cono di 27 metri. Ogni creatura in quell'area deve effettuare un Tiro Salvezza di Tempra DC 22 e subire 72 (16d8) danni da freddo se fallisce il Tiro Salvezza, o la metà di questi danni se lo riesce.

\textbf{Azioni Aggiuntive}

Il drago può effettuare 3 Azioni aggiuntive, scelte tra le opzioni seguenti. Può usare solo un'opzione leggendaria alla volta e solo al termine del turno di un'altra creatura. Il drago recupera le Azioni aggiuntive spese all'inizio del proprio round.

\textbf{Attacco di Ala (Costa 2 Azioni).} Il drago batte le ali. Ogni creatura entro 5 metri dal drago deve riuscire un Tiro Salvezza su Riflessi DC 22 o subire 15 (2d6 + 8) danni da botta e venir gettato prono. Il drago può poi volare fino a metà del suo movimento di volo.

\textbf{Attacco di Coda.} Il drago effettua un attacco di coda.

\textbf{Individuare.} Il drago effettua una prova di Saggezza (Consapevolezza).

\textbf{Ecologia}\\
Ambiente: Montagne Fredde\\
Organizzazione: Solitario\\
\textbf{Tesoro}: Triplo\\
\textbf{Descrizione}\\
Anche se molti lo considerano il più debole e bestiale tra i draghi cromatici, il drago bianco supplisce alla sua mancanza di astuzia con la pura e semplice ferocia. I draghi bianchi vivono su remote e gelide cime montuose, e nei bassopiani artici, facendo la loro tana in scintillanti caverne piene di ghiaccio e neve. Preferiscono che i loro pasti siano completamente congelati.\\

\textbf{Incantesimi}\index{Incantesimi da Drago Bianco}\\
Gli incantesimi preferiti di questo Drago sono:\\
- Scudo di Fuoco\\
- Tempesta di ghiaccio\\
- Tempesta di Nevischio


\medskip\index[Mostruario]{Drago Bianco Adulto}\textbf{Drago Bianco Adulto}

\textit{Enorme drago, caotico malvagio}

\textbf{FORZA} +6

\textbf{DESTREZZA} +0

\textbf{COSTITUZIONE} +6

\textbf{INTELLIGENZA} -1

\textbf{SAGGEZZA} +1

\textbf{CARISMA} +1

\textbf{Iniziativa} +0 -- \textbf{Difesa} 25

\textbf{Punti Ferita} 200 (16d12 + 96)

\textbf{Movimento} 12 m, nuoto 12 m, scavo 9 m, volo 24 m

\textbf{Tiri Salvezza} Tempra +13, Riflessi +9, Volontà +10

\textbf{Competenze} Muoversi Silenziosamente / Nascondersi +5, Consapevolezza +11

\textbf{Immunità al Danno} freddo

\textbf{Sensi} scurovisione 36 m, vista cieca 18 m

\textbf{Linguaggi} Comune, Draconico

\textbf{Sfida} 13 (10000 PX)

\textit{\textbf{Camminare sul Ghiaccio.}} Il drago può muoversi e arrampicarsi su superfici ghiacciate senza bisogno di effettuare prove di caratteristica. Inoltre, il terreno difficile composto di ghiaccio o neve non gli costa movimento aggiuntivo.

\textit{\textbf{Resistenza Leggendaria (3/Giorno).}} Se il drago fallisce un Tiro Salvezza, può scegliere invece di riuscire.

\textbf{Azioni}

\textit{\textbf{Multiattacco.}} Il drago può usare la sua Presenza Spaventosa e poi effettuare tre attacchi: uno con il morso e due con gli artigli.

\textit{\textbf{Artiglio.} Attacco con arma da mischia}: +21 a colpire, portata 1 m, un bersaglio, 1 danno da Sanguinamento.

\textit{Colpisce:} 13 (2d6 + 6) danni taglienti.

\textit{\textbf{Coda.} Attacco con arma da mischia}: +21 a colpire, portata 5 metri, un bersaglio.

\textit{Colpisce:} 15 (2d8 + 6) danni da botta.

\textit{\textbf{Morso.} Attacco con arma da mischia}: +21 a colpire, portata 3 m, un bersaglio.

\textit{Colpisce:} 17 (2d10 + 6) danni perforanti più 4 (1d8) danni da freddo.

\textit{\textbf{Presenza Spaventosa.}} Ogni creatura scelta dal drago, che si trovi entro 36 metri da esso e consapevole della sua presenza, deve riuscire un Tiro Salvezza di Volontà DC 14 o restare spaventata per 1 minuto. Una creatura può ripetere il Tiro Salvezza al termine di ciascun suo round, terminando l'effetto se lo riesce. Se il Tiro Salvezza della creatura ha successo o l'effetto ha termine per essa, la creatura è immune alla Presenza Spaventosa del drago per le successive 24 ore.

\textit{\textbf{Soffio Gelido (Ricarica 5-6).}} Il drago esala un'esplosione di ghiaccio in un cono di 18 metri. Ogni creatura in quell'area deve effettuare un Tiro Salvezza di Tempra DC 19 e subire 54 (12d8) danni da freddo se fallisce il Tiro Salvezza, o la metà di questi danni se lo riesce.

\textbf{Azioni Aggiuntive}

Il drago può effettuare 3 Azioni aggiuntive, scelte tra le opzioni seguenti. Può usare solo un'opzione leggendaria alla volta e solo al termine del turno di un'altra creatura. Il drago recupera le Azioni aggiuntive spese all'inizio del proprio round.

\textbf{Attacco di Ala (Costa 2 Azioni).} Il drago batte le ali. Ogni creatura entro 3 metri dal drago deve riuscire un Tiro Salvezza su Riflessi DC 19 o subire 13 (2d6 + 6) danni da botta e venir gettato prono. Il drago può poi volare fino a metà del suo movimento di volo. \textbf{Attacco di Coda.} Il drago effettua un attacco di coda
.
\textbf{Individuare.} Il drago effettua una prova di Saggezza (Consapevolezza).

\textbf{Ecologia}\\
Ambiente: Montagne Fredde\\
Organizzazione: Solitario\\
\textbf{Tesoro}: Triplo\\
\textbf{Descrizione}\\
Anche se molti lo considerano il più debole e bestiale tra i draghi cromatici, il drago bianco supplisce alla sua mancanza di astuzia con la pura e semplice ferocia. I draghi bianchi vivono su remote e gelide cime montuose, e nei bassopiani artici, facendo la loro tana in scintillanti caverne piene di ghiaccio e neve. Preferiscono che i loro pasti siano completamente congelati.\\
\textbf{Incantesimi}\index{Incantesimi da Drago Bianco}\\
Gli incantesimi preferiti di questo Drago sono:\\
- Scudo di Fuoco\\
- Tempesta di ghiaccio\\
- Tempesta di Nevischio


\medskip\index[Mostruario]{Drago Bianco Giovane}\textbf{Drago Bianco Giovane}

\textit{Grande drago, caotico malvagio}

\textbf{FORZA} +4

\textbf{DESTREZZA} +0

\textbf{COSTITUZIONE} +4

\textbf{INTELLIGENZA} -2

\textbf{SAGGEZZA} +0

\textbf{CARISMA} +1

\textbf{Iniziativa} +0 -- \textbf{Difesa} 20

\textbf{Punti Ferita} 133 (14d10 + 56)

\textbf{Movimento} 12 m, nuoto 12 m, scavo 6 m, volo 24 m

\textbf{Tiri Salvezza} Tempra +8, Riflessi +7, Volontà +5

\textbf{Competenze} Muoversi Silenziosamente / Nascondersi +3, Consapevolezza +6

\textbf{Immunità al Danno} freddo

\textbf{Sensi} scurovisione 36 m, vista cieca 9 m

\textbf{Linguaggi} Comune, Draconico

\textbf{Sfida} 6 (2.300 PX)

\textit{\textbf{Camminare sul Ghiaccio.}} Il drago può muoversi e arrampicarsi su superfici ghiacciate senza bisogno di effettuare prove di caratteristica. Inoltre, il terreno difficile composto di ghiaccio o neve non gli costa movimento aggiuntivo.

\textbf{Azioni}

\textit{\textbf{Multiattacco.}} Il drago può usare la sua Presenza Spaventosa. Poi effettuare tre attacchi: uno con il morso e due con gli artigli.

\textit{\textbf{Artiglio.} Attacco con arma da mischia}: +6 a colpire, portata 1 m, un bersaglio.

\textit{Colpisce:} 11 (2d6 + 4) danni taglienti, 1 danno da Sanguinamento.

\textit{\textbf{Morso.} Attacco con arma da mischia}: +6 a colpire, portata 3 m, un bersaglio.

\textit{Colpisce:} 15 (2d10 + 4) danni perforanti più 4 (1d8) danni da freddo.

\textit{\textbf{Soffio Gelido (Ricarica 5-6).}} Il drago esala un'esplosione di ghiaccio in un cono di 9 metri. Ogni creatura in quell'area deve effettuare un Tiro Salvezza di Tempra DC 15 e subire 45 (10d8) danni da freddo se fallisce il Tiro Salvezza, o la metà di questi danni se lo riesce.

\textbf{Ecologia}\\
Ambiente: Montagne Fredde\\
Organizzazione: Solitario\\
\textbf{Tesoro}: Triplo\\
\textbf{Descrizione}\\
Anche se molti lo considerano il più debole e bestiale tra i draghi cromatici, il drago bianco supplisce alla sua mancanza di astuzia con la pura e semplice ferocia. I draghi bianchi vivono su remote e gelide cime montuose, e nei bassopiani artici, facendo la loro tana in scintillanti caverne piene di ghiaccio e neve. Preferiscono che i loro pasti siano completamente congelati.\\
\textbf{Incantesimi}\index{Incantesimi da Drago Bianco}\\
Gli incantesimi preferiti di questo Drago sono:\\
- Scudo di Fuoco\\
- Tempesta di ghiaccio\\
- Tempesta di Nevischio


\medskip\index[Mostruario]{Drago Bianco Cucciolo}\textbf{Drago Bianco Cucciolo}

\textit{Media drago, caotico malvagio}

\textbf{FORZA} +2

\textbf{DESTREZZA} +0

\textbf{COSTITUZIONE} +2

\textbf{INTELLIGENZA} -3

\textbf{SAGGEZZA} +0

\textbf{CARISMA} +0

\textbf{Iniziativa} +0 -- \textbf{Difesa} 17

\textbf{Punti Ferita} 32 (5d8 + 10)

\textbf{Movimento} 9 m, nuoto 9 m, scavo 5 metri, volo 18 m

\textbf{Tiri Salvezza} Tempra +2, Riflessi +1, Volontà +1

\textbf{Competenze} Muoversi Silenziosamente / Nascondersi +2, Consapevolezza +4

\textbf{Immunità al Danno} freddo

\textbf{Sensi} scurovisione 18 m, vista cieca 3 m

\textbf{Linguaggi} Draconico

\textbf{Sfida} 2 (450 PX)

\textbf{Azioni}

\textit{\textbf{Morso.} Attacco con arma da mischia}: +5 a colpire, portata 3 m, un bersaglio.

\textit{Colpisce:} 15 (2d10 + 4) danni perforanti più 4 (1d8) danni da freddo.

\textit{\textbf{Soffio Gelido (Ricarica 5-6).}} Il drago esala un'esplosione di ghiaccio in un cono di 5 metri. Ogni creatura in quell'area deve effettuare un Tiro Salvezza di Tempra DC 12 e subire 22 (5d8) danni da freddo se fallisce il Tiro Salvezza, o la metà di questi danni se lo riesce.

\textbf{Ecologia}\\
Ambiente: Montagne Fredde\\
Organizzazione: Solitario\\
\textbf{Tesoro}: Triplo\\
\textbf{Descrizione}\\
Anche se molti lo considerano il più debole e bestiale tra i draghi cromatici, il drago bianco supplisce alla sua mancanza di astuzia con la pura e semplice ferocia. I draghi bianchi vivono su remote e gelide cime montuose, e nei bassopiani artici, facendo la loro tana in scintillanti caverne piene di ghiaccio e neve. Preferiscono che i loro pasti siano completamente congelati.

\medskip\index[Mostruario]{Drago Blu Antico}\textbf{Drago Blu Antico}

\textit{Mastodontica drago, legale malvagio}

\textbf{FORZA} +9

\textbf{DESTREZZA} +0

\textbf{COSTITUZIONE} +8

\textbf{INTELLIGENZA} +4

\textbf{SAGGEZZA} +3

\textbf{CARISMA} +5

\textbf{Iniziativa} +4 -- \textbf{Difesa} 34

\textbf{Punti Ferita} 481 (26x3d6 + 208)

\textbf{Movimento} 12 m, scavo 12 m, volo 24 m

\textbf{Tiri Salvezza} Tempra +21, Riflessi +13, Volontà +19

\textbf{Competenze} Muoversi Silenziosamente / Nascondersi +7, Consapevolezza +17

\textbf{Immunità al Danno} fulmine, armi +1

\textbf{Sensi} scurovisione 36 m, vista cieca 18 m

\textbf{Linguaggi} Comune, Draconico

\textbf{Sfida} 23 (50000 PX)

\textit{\textbf{Resistenza Leggendaria (3/Giorno).}} Se il drago fallisce un Tiro Salvezza, può scegliere invece di riuscire.

\textbf{Azioni}

\textit{\textbf{Multiattacco.}} Il drago può usare la sua Presenza Spaventosa. Poi effettuare tre attacchi: uno con il morso e due con gli artigli.

\textit{\textbf{Artiglio.} Attacco con arma da mischia}: +16 a colpire,
portata 3 m, un bersaglio.

\textit{Colpisce:} 16 (2d6 + 9) danni taglienti, 3 danno da Sanguinamento (fino ad un massimo di 20).

\textit{\textbf{Coda.} Attacco con arma da mischia}: +30 a colpire, portata 6 m, un bersaglio.

\textit{Colpisce:} 18 (2d8 + 9) danni da botta.

\textit{\textbf{Morso.} Attacco con arma da mischia}: +30 a colpire, portata 5 metri, un bersaglio.

\textit{Colpisce:} 20 (2d10 + 9) danni perforanti più 11 (2d10) danni da fulmine.

\textit{\textbf{Presenza Spaventosa.}} Ogni creatura scelta dal drago, che si trovi entro 36 metri da esso e consapevole della sua presenza, deve riuscire un Tiro Salvezza di Volontà DC 20 o restare spaventata per 1 minuto. Una creatura può ripetere il Tiro Salvezza al termine di ciascun suo round, terminando l'effetto se lo riesce. Se il Tiro Salvezza della creatura ha successo o l'effetto ha termine per essa, la creatura è immune alla Presenza Spaventosa del drago per le successive 24 ore.

\textit{\textbf{Soffio Fulminante (Ricarica 5-6).}} Il drago esala fulmini in una linea lunga 36 metri e larga 3 metri. Ogni creatura su quella linea deve effettuare un Tiro Salvezza di Riflessi DC 23 e subire 88 (16d10) danni da fulmine se fallisce il Tiro Salvezza, o la metà di questi danni se lo riesce.

\textbf{Azioni Aggiuntive}

Il drago può effettuare 3 Azioni aggiuntive, scelte tra le opzioni seguenti. Può usare solo un'opzione leggendaria alla volta e solo al termine del turno di un'altra creatura. Il drago recupera le Azioni aggiuntive spese all'inizio del proprio round.

\textbf{Attacco di Ala (Costa 2 Azioni).} Il drago batte le ali. Ogni creatura entro 5 metri dal drago deve riuscire un Tiro Salvezza su Riflessi DC 24 o subire 16 (2d6 + 9) danni da botta e venir gettato prono. Il drago può poi volare fino a metà del suo movimento di volo.

\textbf{Attacco di Coda.} Il drago effettua un attacco di coda.

\textbf{Individuare.} Il drago effettua una prova di Saggezza (Consapevolezza).

\textbf{Individuare.} Il drago effettua una prova di Saggezza (Consapevolezza).\\
\textbf{Ecologia}\\
Ambiente: Picchi montuosi\\
Organizzazione: Solitario\\
\textbf{Tesoro}: Triplo\\
\textbf{Descrizione}\\
I draghi blu sono intriganti consumati ed ossessivamente ordinati. In combattimento, i draghi blu preferiscono prendere di sorpresa i nemici, se possibile, e non esitano a ritirarsi se le cose si mettono male. Preferiscono fare la loro tana vicino a quelli che controllano, qualche volta anche entro i confini di una città.\\
\textbf{Incantesimi}\index{Incantesimi da Drago Blu}\\
Gli incantesimi preferiti di questo Drago sono:\\
- Catena di fulmini\\
- Gabbia di Forza\\
- Teletrasporto\\
- Forma Eterea


\medskip\index[Mostruario]{Drago Blu Adulto}\textbf{Drago Blu Adulto}

\textit{Enorme drago, legale malvagio}

\textbf{FORZA} +7

\textbf{DESTREZZA} +0

\textbf{COSTITUZIONE} +6

\textbf{INTELLIGENZA} +3

\textbf{SAGGEZZA} +2

\textbf{CARISMA} +4

\textbf{Iniziativa} +3 -- \textbf{Difesa} 27

\textbf{Punti Ferita} 225 (18d12 + 108)

\textbf{Movimento} 12 m, scavo 12 m, volo 24 m

\textbf{Tiri Salvezza} Tempra +15, Riflessi +10, Volontà +13

\textbf{Competenze} Muoversi Silenziosamente / Nascondersi +5, Consapevolezza +12

\textbf{Immunità al Danno} fulmine

\textbf{Sensi} scurovisione 36 m, vista cieca 18 m

\textbf{Linguaggi} Comune, Draconico

\textbf{Sfida} 16 (15000 PX)

\textit{\textbf{Resistenza Leggendaria (3/Giorno).}} Se il drago fallisce un Tiro Salvezza, può scegliere invece di riuscire.

\textbf{Azioni}

\textit{\textbf{Multiattacco.}} Il drago può usare la sua Presenza Spaventosa. Poi effettuare tre attacchi: uno con il morso e due con gli artigli.

\textit{\textbf{Artiglio.} Attacco con arma da mischia}: +26 a colpire, portata 1 m, un bersaglio.

\textit{Colpisce:} 14 (2d6 + 7) danni taglienti, 1 danno da Sanguinamento.

\textit{\textbf{Coda.} Attacco con arma da mischia}: +26 a colpire, portata 5 metri, un bersaglio.

\textit{Colpisce:} 16 (2d8 + 7) danni da botta.

\textit{\textbf{Morso.} Attacco con arma da mischia}: +26 a colpire, portata 3 m, un bersaglio.

\textit{Colpisce:} 18 (2d10 + 7) danni perforanti più 5 (1d10) danni da fulmine.

\textit{\textbf{Presenza Spaventosa.}} Ogni creatura scelta dal drago, che si trovi entro 36 metri da esso e consapevole della sua presenza, deve riuscire un Tiro Salvezza di Volontà DC 17 o restare spaventata per 1 minuto. Una creatura può ripetere il Tiro Salvezza al termine di ciascun suo round, terminando l'effetto se lo riesce. Se il Tiro Salvezza della creatura ha successo o l'effetto ha termine per essa, la creatura è immune alla Presenza Spaventosa del drago per le successive 24 ore.

\textit{\textbf{Soffio Fulminante (Ricarica 5-6).}} Il drago esala fulmini in una linea lunga 27 metri e larga 1 metro. Ogni creatura su quella linea deve effettuare un Tiro Salvezza di Riflessi DC 19 e subire 66 (12d10) danni da fulmine se fallisce il Tiro Salvezza, o la metà di questi danni se lo riesce.

\textbf{Azioni Aggiuntive}

Il drago può effettuare 3 Azioni aggiuntive, scelte tra le opzioni seguenti. Può usare solo un'opzione leggendaria alla volta e solo al termine del turno di un'altra creatura. Il drago recupera le Azioni aggiuntive spese all'inizio del proprio round.

\textbf{Attacco di Ala (Costa 2 Azioni).} Il drago batte le ali. Ogni creatura entro 3 metri dal drago deve riuscire un Tiro Salvezza su Riflessi DC 20 o subire 14 (2d6 + 7) danni da botta e venir gettato prono. Il drago può poi volare fino a metà della del suo movimento di volo.

\textbf{Attacco di Coda.} Il drago effettua un attacco di coda.

\textbf{Individuare.} Il drago effettua una prova di Saggezza (Consapevolezza).

\textbf{Ecologia}\\
Ambiente: Picchi montuosi\\
Organizzazione: Solitario\\
\textbf{Tesoro}: Triplo\\
\textbf{Descrizione}\\
I draghi blu sono intriganti consumati ed ossessivamente ordinati. In combattimento, i draghi blu preferiscono prendere di sorpresa i nemici, se possibile, e non esitano a ritirarsi se le cose si mettono male. Preferiscono fare la loro tana vicino a quelli che controllano, qualche volta anche entro i confini di una città.\\
\textbf{Incantesimi}\index{Incantesimi da Drago Blu}\\
Gli incantesimi preferiti di questo Drago sono:\\
- Catena di fulmini\\
- Gabbia di Forza\\
- Teletrasporto\\
- Forma Eterea


\medskip\index[Mostruario]{Drago Blu Giovane}\textbf{Drago Blu Giovane}

\textit{Enorme drago, legale malvagio}

\textbf{FORZA} +5

\textbf{DESTREZZA} +0

\textbf{COSTITUZIONE} +4

\textbf{INTELLIGENZA} +2

\textbf{SAGGEZZA} +1

\textbf{CARISMA} +3

\textbf{Iniziativa} +2 -- \textbf{Difesa} 23

\textbf{Punti Ferita} 152 (16d10 + 64)

\textbf{Movimento} 12 m, scavo 12 m, volo 24 m

\textbf{Tiri Salvezza} Tempra +10, Riflessi +8, Volontà +8

\textbf{Competenze} Muoversi Silenziosamente / Nascondersi +4, Consapevolezza +9

\textbf{Immunità al Danno} fulmine

\textbf{Sensi} scurovisione 36 m, vista cieca 9 m

\textbf{Linguaggi} Comune, Draconico

\textbf{Sfida} 9 (5000 PX)

\textbf{Azioni}

\textit{\textbf{Multiattacco.}} Il drago può effettuare tre attacchi: uno con il morso e due con gli artigli.

\textit{\textbf{Artiglio.} Attacco con arma da mischia}: +13 a colpire, portata 1 m, un bersaglio.

\textit{Colpisce:} 12 (2d6 + 5) danni taglienti, 1 danno da Sanguinamento.

\textit{\textbf{Morso.} Attacco con arma da mischia}: +13 a colpire, portata 3 m, un bersaglio.

\textit{Colpisce:} 16 (2d10 + 5) danni perforanti più 5 (1d10) danni da fulmine.

\textit{\textbf{Soffio Fulminante (Ricarica 5-6).}} Il drago esala fulmini in una linea lunga 18 metri e larga 1 metro. Ogni creatura su quella linea deve effettuare un Tiro Salvezza di Riflessi DC 16 e subire 55 (10d10) danni da fulmine se fallisce il Tiro Salvezza, o la metà di questi danni se lo riesce.

\textbf{Ecologia}\\
Ambiente: Picchi montuosi\\
Organizzazione: Solitario\\
\textbf{Tesoro}: Triplo\\
\textbf{Descrizione}\\
I draghi blu sono intriganti consumati ed ossessivamente ordinati. In combattimento, i draghi blu preferiscono prendere di sorpresa i nemici, se possibile, e non esitano a ritirarsi se le cose si mettono male. Preferiscono fare la loro tana vicino a quelli che controllano, qualche volta anche entro i confini di una città.\\
\textbf{Incantesimi}\index{Incantesimi da Drago Blu}\\
Gli incantesimi preferiti di questo Drago sono:\\
- Catena di fulmini\\
- Gabbia di Forza\\
- Teletrasporto\\
- Forma Eterea


\medskip\index[Mostruario]{Drago Blu Cucciolo}\textbf{Drago Blu Cucciolo}

\textit{Enorme drago, legale malvagio}

\textbf{FORZA} +3

\textbf{DESTREZZA} +0

\textbf{COSTITUZIONE} +2

\textbf{INTELLIGENZA} +1

\textbf{SAGGEZZA} +0

\textbf{CARISMA} +2

\textbf{Iniziativa} +1 -- \textbf{Difesa} 19

\textbf{Punti Ferita} 52 (8d8 + 16)

\textbf{Movimento} 9 m, scavo 5 metri, volo 18 m

\textbf{Tiri Salvezza} Tempra +4, Riflessi +1, Volontà +1

\textbf{Competenze} Muoversi Silenziosamente / Nascondersi +2, Consapevolezza +4

\textbf{Immunità al Danno} fulmine

\textbf{Sensi} scurovisione 18 m, vista cieca 3 m

\textbf{Linguaggi} Draconico

\textbf{Sfida} 3 (700 PX)

\textbf{Azioni}

\textit{\textbf{Morso.} Attacco con arma da mischia}: +5 a colpire, portata 1 m, un bersaglio.

\textit{Colpisce:} 8 (1d10 + 3) danni perforanti più 3 (1d6) danni da fulmine.

\textit{\textbf{Soffio Fulminante (Ricarica 5-6).}} Il drago esala fulmini in una linea lunga 9 metri e larga 1 metro. Ogni creatura su quella linea deve effettuare un Tiro Salvezza di Riflessi DC 12 e subire 22 (4d10) danni da fulmine se fallisce il Tiro Salvezza, o la metà di questi danni se lo riesce.

\textbf{Ecologia}\\
Ambiente: Picchi montuosi\\
Organizzazione: Solitario\\
\textbf{Tesoro}: Triplo\\
\textbf{Descrizione}\\
I draghi blu sono intriganti consumati ed ossessivamente ordinati. In combattimento, i draghi blu preferiscono prendere di sorpresa i nemici, se possibile, e non esitano a ritirarsi se le cose si mettono male. Preferiscono fare la loro tana vicino a quelli che controllano, qualche volta anche entro i confini di una città.

\medskip\textbf{Drago Giallo Antico}\index[Mostruario]{Drago Giallo Antico}\\
\textit{Mastodontica drago, neutrale malvagio}\\
\textbf{Forza}: +10\\
\textbf{Destrezza}: +1\\
\textbf{Costituzione}: +8\\
\textbf{Intelligenza}: +3\\
\textbf{Saggezza}: +2\\
\textbf{Carisma}: +4\\
\textbf{Difesa}: 27 (armatura naturale) - \textbf{Iniziativa}: +4\\
\textbf{Punti Ferita}: 481 (26x3d6 + 208)\\
\textbf{Movimento}: 12 m, scavo 24 m, scalata 24, volo 12 m\\
\textbf{Tiri Salvezza}: Tempra +21, Riflessi +13, Volontà +19\\
\textbf{Competenze}: Criminalità +7, Consapevolezza +17\\
\textbf{Immunità al Danno}: fulmine\\
\textbf{Sensi}: Scurovisione 36 m, vista cieca 18 m\\
\textbf{Linguaggi} Comune, Draconico\\
\textbf{Sfida}: 23 (50000 PX)\smallskip\\
\textit{\textbf{Resistenza Leggendaria (3/Giorno).}} Se il drago fallisce un Tiro Salvezza, può scegliere invece di riuscire. \\
\smallskip\textbf{Azioni}\\
\textit{\textbf{Multiattacco.}} Il drago può usare la sua Presenza Spaventosa. Poi effettuare tre attacchi: uno con il morso e due con gli artigli.\\
\textit{\textbf{Artiglio.} Attacco con arma da mischia}: +30 al colpire, portata 3 m, un bersaglio.\\
\textit{Colpisce:} 16 (2d6 + 9) danni taglienti, 3 danno da Sanguinamento (fino ad un massimo di 20).\\
\textit{\textbf{Coda.} Attacco con arma da mischia}: +30 al colpire, portata 6 m, un bersaglio.\\
\textit{Colpisce:} 18 (2d8 + 9) danni da botta.\\
\textit{\textbf{Morso.} Attacco con arma da mischia}: +30 al colpire, portata 5 metri, un bersaglio.\\
\textit{Colpisce:} 20 (2d10 + 9) danni perforanti più 11 (2d10) danni da fulmine.\\
\textit{\textbf{Presenza Spaventosa.}} Ogni creatura scelta dal drago, che si trovi entro 36 metri da esso e consapevole della sua presenza, deve riuscire un Tiro Salvezza su Volontà DC 25 o restare spaventata per 1 minuto. Una creatura può ripetere il Tiro Salvezza al termine di ciascun suo round, terminando l'effetto se lo riesce. Se il Tiro Salvezza della creatura ha successo o l'effetto ha termine per essa, la creatura è immune alla Presenza Spaventosa del drago per le successive 24 ore.\\
\textit{\textbf{Soffio Incendiario (Ricarica 5-6).}} Il drago esala aria rovente in una linea lunga 36 metri e larga 3 metri. Ogni creatura su quella linea deve effettuare un Tiro Salvezza su Riflessi DC 30 e subire 88 (16d10) danni da fuoco se fallisce il Tiro Salvezza, o la metà di questi danni se lo riesce.\\
\textbf{Azioni Aggiuntive}\\
Il drago può effettuare 3 azioni aggiuntive, scelte tra le opzioni seguenti. Può usare solo un'Azione Aggiuntiva alla volta e solo al termine del round di un'altra creatura. Il drago recupera le Azioni Aggiuntive spese all'inizio del proprio round.\\
\textbf{Attacco di Ala (Costa 2 Azioni).} Il drago batte le ali. Ogni creatura entro 5 metri dal drago deve riuscire un Tiro Salvezza su Riflessi DC 31 o subire 16 (2d6 + 9) danni da botta e venir gettato prono. Il drago può poi volare fino a metà della sua velocità di volo.\\
\textbf{Attacco di Coda.} Il drago effettua un attacco di coda.\\
\textbf{Individuare.} Il drago effettua una prova di Saggezza (Consapevolezza).\\
\textbf{Ecologia}\\
Ambiente: Deserti Caldi\\
Organizzazione: Solitario\\
\textbf{Tesoro}: Triplo\\
\textbf{Descrizione}\\
I draghi gialli sono predatori famelici e combattenti indisciplinati. Amano la caccia ed uccidere, sono consumati predatori che istintivamente aggrediscono chiunque sia nel loro territorio. Il deserto è il loro terreno dove scavano trappole grezze per catturare le loro povere vittime.
Il soffio di un drago giallo è un onda di calore (danno da fuoco).
\\
\textbf{Incantesimi}\index{Incantesimi da Drago Giallo}\\
Gli incantesimi preferiti di questo Drago sono:\\
- Riscaldare Metallo\\
- Palla di Fuoco\\
- Scudo di Fuoco


\medskip\index[Mostruario]{Drago Nero Antico}\textbf{Drago Nero Antico}

\textit{Mastodontica drago, caotico malvagio}

\textbf{FORZA} +8

\textbf{DESTREZZA} +2

\textbf{COSTITUZIONE} +7

\textbf{INTELLIGENZA} +3

\textbf{SAGGEZZA} +2

\textbf{CARISMA} +4

\textbf{Iniziativa} +3 -- \textbf{Difesa} 33

\textbf{Punti Ferita} 367 (21x3d6 + 147)

\textbf{Movimento} 12 m, scalata 12 m, volo 24 m

\textbf{Tiri Salvezza} Tempra +20, Riflessi +13, Volontà +18

\textbf{Competenze} Muoversi Silenziosamente / Nascondersi +9, Consapevolezza +16

\textbf{Immunità al Danno} acido, armi +1

\textbf{Sensi} scurovisione 36 m, vista cieca 18 m

\textbf{Linguaggi} Comune, Draconico

\textbf{Sfida} 21 (33000 PX)

\textit{\textbf{Anfibio.}} Il drago può respirare aria e acqua.

\textit{\textbf{Resistenza Leggendaria (3/Giorno).}} Se il drago fallisce un Tiro Salvezza, può scegliere invece di riuscire.

\textbf{Azioni}

\textit{\textbf{Multiattacco.}} Il drago può usare la sua Presenza Spaventosa. Poi effettuare tre attacchi: uno con il morso e due con gli artigli.

\textit{\textbf{Artiglio.} Attacco con arma da mischia}: +30 a colpire, portata 3 m, un bersaglio.

\textit{Colpisce:} 15 (2d6 + 8) danni taglienti, 3 danno da Sanguinamento (fino ad un massima di 20).

\textit{\textbf{Coda.} Attacco con arma da mischia}: +30 a colpire, portata 6 m, un bersaglio.

\textit{Colpisce:} 17 (2d8 + 8) danni da botta.

\textit{\textbf{Morso.} Attacco con arma da mischia} : +30 a colpire, portata 5 metri, un bersaglio.

\textit{Colpisce:} 19 (2d10 + 8) danni perforanti più 9 (4d6) danni da acido.

\textit{\textbf{Presenza Spaventosa.}} Ogni creatura scelta dal drago, che si trovi entro 36 metri da esso e consapevole della sua presenza, deve riuscire un Tiro Salvezza di Volontà DC 19 o restare spaventata per 1 minuto. Una creatura può ripetere il Tiro Salvezza al termine di ciascun suo round, terminando l'effetto se lo riesce. Se il Tiro Salvezza della creatura ha successo o l'effetto ha termine per essa, la creatura è immune alla Presenza Spaventosa del drago per le successive 24 ore.

\textit{\textbf{Soffio Acido (Ricarica 5-6).}} Il drago esala acido in una linea di 27 metri larga 3 metri. Ogni creatura in quell'area deve effettuare un Tiro Salvezza di Riflessi DC 22 e subire 67 (15d8) danni da acido se fallisce il Tiro Salvezza, o la metà di questi danni se lo riesce.

\textbf{Azioni Aggiuntive}

Il drago può effettuare 3 Azioni aggiuntive, scelte tra le opzioni seguenti. Può usare solo un'opzione leggendaria alla volta e solo al termine del turno di un'altra creatura. Il drago recupera le Azioni aggiuntive spese all'inizio del proprio round.

\textbf{Attacco di Ala (Costa 2 Azioni).} Il drago batte le ali. Ogni creatura entro 5 metri dal drago deve riuscire un Tiro Salvezza su Riflessi DC 23 o subire 15 (2d6 + 8) danni da botta e venir gettato prono. Il drago può poi volare fino a metà del suo movimento di volo.

\textbf{Attacco di Coda.} Il drago effettua un attacco di coda.

\textbf{Individuare.} Il drago effettua una prova di Saggezza (Consapevolezza).

\textbf{Ecologia}\\
Ambiente: Paludi Calde\\
Organizzazione: Solitario\\
\textbf{Tesoro}: Triplo\\
\textbf{Descrizione}\\
Signori delle paludi e degli acquitrini più cupi, i draghi neri sono i padroni incontrastati del loro territorio, che dominano con crudeltà e infondendo terrore in chi abita nelle vicinanze. I draghi neri si stanziano nelle zone più remote delle paludi, specie in caverne sul fondo di pozze fetide e buie. Dentro, ammassano i loro luridi tesori e dormono tra radici e fango. I draghi neri amano il cibo un pò marcio e spesso lasciano un pasto a marcire in una pozza per giorni prima di consumarlo. I draghi neri preferiscono tesori che non si decompongono o degradano, accumulando tesori di monete, pietre preziose, gioielli e altri oggetti di pietra o metallo.\\
\textbf{Incantesimi}\index{Incantesimi da Drago Nero}\\
Gli incantesimi preferiti di questo Drago sono:\\
- Dito della morte\\
- Disintegrazione\\
- Blocca Mostri


\medskip\index[Mostruario]{Drago Nero Adulto}\textbf{Drago Nero Adulto}

\textit{Enorme drago, caotico malvagio}

\textbf{FORZA} +6

\textbf{DESTREZZA} +2

\textbf{COSTITUZIONE} +5

\textbf{INTELLIGENZA} +2

\textbf{SAGGEZZA} +1

\textbf{CARISMA} +3

\textbf{Iniziativa} +2 -- \textbf{Difesa} 28

\textbf{Punti Ferita} 195 (17d12 + 85)

\textbf{Movimento} 12 m, scalata 12 m, volo 24 m

\textbf{Tiri Salvezza} Tempra +14, Riflessi +10, Volontà +12

\textbf{Competenze} Muoversi Silenziosamente / Nascondersi +7, Consapevolezza +11

\textbf{Immunità al Danno} acido

\textbf{Sensi} scurovisione 36 m, vista cieca 18 m

\textbf{Linguaggi} Comune, Draconico

\textbf{Sfida} 17 (18000 PX)

\textit{\textbf{Anfibio.}} Il drago può respirare aria e acqua.

\textit{\textbf{Resistenza Leggendaria (3/Giorno).}} Se il drago fallisce un Tiro Salvezza, può scegliere invece di riuscire.

\textbf{Azioni}

\textit{\textbf{Multiattacco.}} Il drago può usare la sua Presenza Spaventosa. Poi effettuare tre attacchi: uno con il morso e due con gli artigli.

\textit{\textbf{Artiglio.} Attacco con arma da mischia}: +26 a colpire, portata 1 m, un bersaglio.

\textit{Colpisce:} 13 (2d6 + 6) danni taglienti, 1 danno da Sanguinamento.

\textit{\textbf{Coda.} Attacco con arma da mischia}: +26 a colpire, portata 5 metri, un bersaglio.

\textit{Colpisce:} 15 (2d8 + 6) danni da botta.

\textit{\textbf{Morso.} Attacco con arma da mischia}: +26 a colpire, portata 3 m, un bersaglio.

\textit{Colpisce:} 17 (2d10 + 6) danni perforanti più 4 (1d8) danni da acido.

\textit{\textbf{Presenza Spaventosa.}} Ogni creatura scelta dal drago, che si trovi entro 36 metri da esso e consapevole della sua presenza, deve riuscire un Tiro Salvezza di Volontà DC 16 o restare spaventata per 1 minuto. Una creatura può ripetere il Tiro Salvezza al termine di ciascun suo round, terminando l'effetto se lo riesce. Se il Tiro Salvezza della creatura ha successo o l'effetto ha termine per essa, la creatura è immune alla Presenza Spaventosa del drago per le successive 24 ore.

\textit{\textbf{Soffio Acido (Ricarica 5-6).}} Il drago esala acido in una linea di 18 metri larga 1 metro. Ogni creatura in quell'area deve effettuare un Tiro Salvezza di Riflessi DC 18 e subire 54 (12d8) danni da acido se fallisce il Tiro Salvezza, o la metà di questi danni se lo
riesce.

\textbf{Azioni Aggiuntive}

Il drago può effettuare 3 Azioni aggiuntive, scelte tra le opzioni seguenti. Può usare solo un'opzione leggendaria alla volta e solo al termine del turno di un'altra creatura. Il drago recupera le Azioni aggiuntive spese all'inizio del proprio round.

\textbf{Attacco di Ala (Costa 2 Azioni).} Il drago batte le ali. Ogni creatura entro 3 metri dal drago deve riuscire un Tiro Salvezza su Riflessi DC 19 o subire 13 (2d6 + 6) danni da botta e venir gettato prono. Il drago può poi volare fino a metà della del suo movimento di volo.

\textbf{Attacco di Coda.} Il drago effettua un attacco di coda.

\textbf{Individuare.} Il drago effettua una prova di Saggezza (Consapevolezza).

\textbf{Ecologia}\\
Ambiente: Paludi Calde\\
Organizzazione: Solitario\\
\textbf{Tesoro}: Triplo\\
\textbf{Descrizione}\\
Signori delle paludi e degli acquitrini più cupi, i draghi neri sono i padroni incontrastati del loro territorio, che dominano con crudeltà e infondendo terrore in chi abita nelle vicinanze. I draghi neri si stanziano nelle zone più remote delle paludi, specie in caverne sul fondo di pozze fetide e buie. Dentro, ammassano i loro luridi tesori e dormono tra radici e fango. I draghi neri amano il cibo un pò marcio e spesso lasciano un pasto a marcire in una pozza per giorni prima di consumarlo. I draghi neri preferiscono tesori che non si decompongono o degradano, accumulando tesori di monete, pietre preziose, gioielli e altri oggetti di pietra o metallo.\\
\textbf{Incantesimi}\index{Incantesimi da Drago Nero}\\
Gli incantesimi preferiti di questo Drago sono:\\
- Dito della morte\\
- Disintegrazione\\
- Blocca Mostri


\medskip\index[Mostruario]{Drago Nero Giovane}\textbf{Drago Nero Giovane}

\textit{Grande drago, caotico malvagio}

\textbf{FORZA} +4

\textbf{DESTREZZA} +2

\textbf{COSTITUZIONE} +3

\textbf{INTELLIGENZA} +1

\textbf{SAGGEZZA} +0

\textbf{CARISMA} +2

\textbf{Iniziativa} +2 -- \textbf{Difesa} 22

\textbf{Punti Ferita} 127 (15d10 + 45)

\textbf{Movimento} 12 m, scalata 12 m, volo 24 m

\textbf{Tiri Salvezza} Tempra +9, Riflessi +8, Volontà +7

\textbf{Competenze} Muoversi Silenziosamente / Nascondersi +5, Consapevolezza +6

\textbf{Immunità al Danno} acido

\textbf{Sensi} scurovisione 36 m, vista cieca 9 m

\textbf{Linguaggi} Comune, Draconico

\textbf{Sfida} 7 (2.900 PX)

\textit{\textbf{Anfibio.}} Il drago può respirare aria e acqua.

\textbf{Azioni}

\textit{\textbf{Multiattacco.}} Il drago può effettuare tre attacchi: uno con il morso e due con gli artigli.

\textit{\textbf{Artiglio.} Attacco con arma da mischia}: +9 a colpire, portata 1 m, un bersaglio.

\textit{Colpisce:} 11 (2d6 + 4) danni taglienti, 1 danno da Sanguinamento.

\textit{\textbf{Morso.} Attacco con arma da mischia}: +9 a colpire, portata 3 m, un bersaglio.

\textit{Colpisce:} 11 (2d10 + 4) danni perforanti più 4 (1d8) danni da acido.

\textit{\textbf{Soffio Acido (Ricarica 5-6).}} Il drago esala acido in una linea di 9 metri larga 1 metro. Ogni creatura in quell'area deve effettuare un Tiro Salvezza di Riflessi DC 14 e subire 49 (11d8) danni da acido se fallisce il Tiro Salvezza, o la metà di questi danni se lo riesce.

\textbf{Ecologia}\\
Ambiente: Paludi Calde\\
Organizzazione: Solitario\\
\textbf{Tesoro}: Triplo\\
\textbf{Descrizione}\\
Signori delle paludi e degli acquitrini più cupi, i draghi neri sono i padroni incontrastati del loro territorio, che dominano con crudeltà e infondendo terrore in chi abita nelle vicinanze. I draghi neri si stanziano nelle zone più remote delle paludi, specie in caverne sul fondo di pozze fetide e buie. Dentro, ammassano i loro luridi tesori e dormono tra radici e fango. I draghi neri amano il cibo un pò marcio e spesso lasciano un pasto a marcire in una pozza per giorni prima di consumarlo. I draghi neri preferiscono tesori che non si decompongono o degradano, accumulando tesori di monete, pietre preziose, gioielli e altri oggetti di pietra o metallo.\\
\textbf{Incantesimi}\index{Incantesimi da Drago Nero}\\
Gli incantesimi preferiti di questo Drago sono:\\
- Dito della morte\\
- Disintegrazione\\
- Blocca Mostri


\medskip\index[Mostruario]{Drago Nero Cucciolo}\textbf{Drago Nero Cucciolo}

\textit{Media drago, caotico malvagio}

\textbf{FORZA} +2

\textbf{DESTREZZA} +2

\textbf{COSTITUZIONE} +1

\textbf{INTELLIGENZA} +0

\textbf{SAGGEZZA} +0

\textbf{CARISMA} +1

\textbf{Iniziativa} +2 -- \textbf{Difesa} 18

\textbf{Punti Ferita} 33 (6d8 + 6)

\textbf{Movimento} 9 m, scalata 9 m, volo 18 m

\textbf{Tiri Salvezza} Tempra +2, Riflessi +2, Volontà +0

\textbf{Competenze} Muoversi Silenziosamente / Nascondersi +4, Consapevolezza +4

\textbf{Immunità al Danno} acido

\textbf{Sensi} scurovisione 18 m, vista cieca 3 m

\textbf{Linguaggi} Draconico

\textbf{Sfida} 2 (450 PX)

\textit{\textbf{Anfibio.}} Il drago può respirare aria e acqua.

\textbf{Azioni}

\textit{\textbf{Morso.} Attacco con arma da mischia}: +4 a colpire, portata 1 m, un bersaglio.

\textit{Colpisce:} 7 (1d10 + 2) danni perforanti più 2 (1d4) danni da acido.

\textit{\textbf{Soffio Acido (Ricarica 5-6).}} Il drago esala acido in una linea di 5 metri larga 1 metro. Ogni creatura in quell'area deve effettuare un Tiro Salvezza di Riflessi DC 11 e subire 22 (5d8) danni da acido se fallisce il Tiro Salvezza, o la metà di questi danni se lo riesce.

\textbf{Ecologia}\\
Ambiente: Paludi Calde\\
Organizzazione: Solitario\\
\textbf{Tesoro}: Triplo\\
\textbf{Descrizione}\\
Signori delle paludi e degli acquitrini più cupi, i draghi neri sono i padroni incontrastati del loro territorio, che dominano con crudeltà e infondendo terrore in chi abita nelle vicinanze. I draghi neri si stanziano nelle zone più remote delle paludi, specie in caverne sul fondo di pozze fetide e buie. Dentro, ammassano i loro luridi tesori e dormono tra radici e fango. I draghi neri amano il cibo un pò marcio e spesso lasciano un pasto a marcire in una pozza per giorni prima di consumarlo. I draghi neri preferiscono tesori che non si decompongono o degradano, accumulando tesori di monete, pietre preziose, gioielli e altri oggetti di pietra o metallo.

\medskip\index[Mostruario]{Drago Rosso Antico}\textbf{Drago Rosso Antico}

\textit{Mastodontica drago, caotico malvagio}

\textbf{FORZA} +10

\textbf{DESTREZZA} +0

\textbf{COSTITUZIONE} +9

\textbf{INTELLIGENZA} +4

\textbf{SAGGEZZA} +2

\textbf{CARISMA} +6

\textbf{Iniziativa} +4 -- \textbf{Difesa} 34

\textbf{Punti Ferita} 546 (28x3d6 + 252)

\textbf{Movimento} 12 m, scalata 12 m, volo 24 m

\textbf{Tiri Salvezza} Tempra +22, Riflessi +13, Volontà +21

\textbf{Competenze} Muoversi Silenziosamente / Nascondersi +7, Consapevolezza +16

\textbf{Immunità al Danno} fuoco, armi +1

\textbf{Sensi} scurovisione 36 m, vista cieca 18 m

\textbf{Linguaggi} Comune, Draconico

\textbf{Sfida} 24 (62000 PX)

\textit{\textbf{Resistenza Leggendaria (3/Giorno).}} Se il drago fallisce un Tiro Salvezza, può scegliere invece di riuscire.

\textbf{Azioni}

\textit{\textbf{Multiattacco.}} Il drago può usare la sua Presenza Spaventosa e poi effettuare tre attacchi: uno con il morso e due con gli artigli.

\textit{\textbf{Artiglio.} Attacco con arma da mischia}: +30 a colpire, portata 3 m, un bersaglio.

\textit{Colpisce:} 17 (2d6 + 10) danni taglienti, 3 danno da Sanguinamento (fino ad un massimo di 20).

\textit{\textbf{Coda.} Attacco con arma da mischia}: +30 a colpire, portata 6 m, un bersaglio.

\textit{Colpisce:} 19 (2d8 + 10) danni da botta.

\textit{\textbf{Morso.} Attacco con arma da mischia}: +30 a colpire, portata 5 metri, un bersaglio.

\textit{Colpisce:} 21 (2d10 + 10) danni perforanti più 14 (4d6) danni da fuoco.

\textit{\textbf{Presenza Spaventosa.}} Ogni creatura scelta dal drago, che si trovi entro 36 metri da esso e consapevole della sua presenza, deve riuscire un Tiro Salvezza di Volontà DC 21 o restare spaventata per 1 minuto. Una creatura può ripetere il Tiro Salvezza al termine di ciascun suo round, terminando l'effetto se lo riesce. Se il Tiro Salvezza della creatura ha successo o l'effetto ha termine per essa, la creatura è immune alla Presenza Spaventosa del drago per le successive 24 ore.

\textit{\textbf{Soffio Infuocato (Ricarica 5-6).}} Il drago esala fuoco in un cono di 27 metri. Ogni creatura in quell'area deve effettuare un Tiro Salvezza su Riflessi DC 24 e subire 91 (26d6) danni da fuoco se fallisce il Tiro Salvezza, o la metà di questi danni se lo riesce.

\textbf{Azioni Aggiuntive}

Il drago può effettuare 3 Azioni aggiuntive, scelte tra le opzioni seguenti. Può usare solo un'opzione leggendaria alla volta e solo al termine del turno di un'altra creatura. Il drago recupera le Azioni aggiuntive spese all'inizio del proprio round.

\textbf{Attacco di Ala (Costa 2 Azioni).} Il drago batte le ali. Ogni creatura entro 5 metri dal drago deve riuscire un Tiro Salvezza su Riflessi DC 25 o subire 17 (2d6 + 10) danni da botta e venir gettato prono. Il drago può poi volare fino a metà del suo movimento di volo.

\textbf{Attacco di Coda.} Il drago effettua un attacco di coda.

\textbf{Individuare.} Il drago effettua una prova di Saggezza (Consapevolezza).

\textbf{Ecologia}\\
Ambiente: Montagne calde\\
Organizzazione: Solitario\\
\textbf{Tesoro}: Triplo\\
\textbf{Descrizione}\\
Poche creature sono più crudeli e terribili del possente drago rosso. Sovrano dei cromatici, il terribile drago rosso porta rovina e morte nelle terre minacciate dalla sua presenza.\\
\textbf{Incantesimi}\index{Incantesimi da Drago Rosso}\\
Gli incantesimi preferiti di questo Drago sono:\\
- Palla di Fuoco\\
- Nube Incendiaria\\
- Muro di fuoco


\medskip\index[Mostruario]{Drago Rosso Adulto}\textbf{Drago Rosso Adulto}

\textit{Enorme drago, caotico malvagio}

\textbf{FORZA} +8

\textbf{DESTREZZA} +0

\textbf{COSTITUZIONE} +7

\textbf{INTELLIGENZA} +3

\textbf{SAGGEZZA} +1

\textbf{CARISMA} +5

\textbf{Iniziativa} +3 -- \textbf{Difesa} 28

\textbf{Punti Ferita} 256 (19d12 + 133)

\textbf{Movimento} 12 m, scalata 12 m, volo 24 m

\textbf{Tiri Salvezza} Tempra +16, Riflessi +10, Volontà +15

\textbf{Competenze} Muoversi Silenziosamente / Nascondersi +6, Consapevolezza +13

\textbf{Immunità al Danno} fuoco

\textbf{Sensi} scurovisione 36 m, vista cieca 18 m

\textbf{Linguaggi} Comune, Draconico

\textbf{Sfida} 17 (18000 PX)

\textit{\textbf{Resistenza Leggendaria (3/Giorno).}} Se il drago fallisce un Tiro Salvezza, può scegliere invece di riuscire.

\textbf{Azioni}

\textit{\textbf{Multiattacco.}} Il drago può usare la sua Presenza Spaventosa e poi effettuare tre attacchi: uno con il morso e due con gli artigli.

\textit{\textbf{Artiglio.} Attacco con arma da mischia}: +28 a colpire, portata 1 m, un bersaglio.

\textit{Colpisce:} 15 (2d6 + 8) danni taglienti, 1 danno da Sanguinamento.

\textit{\textbf{Coda.} Attacco con arma da mischia}: +28 a colpire, portata 5 metri, un bersaglio.

\textit{Colpisce:} 17 (2d8 + 8) danni da botta.

\textit{\textbf{Morso.} Attacco con arma da mischia}: +28 a colpire, portata 3 m, un bersaglio.

\textit{Colpisce:} 19 (2d10 + 8) danni perforanti più 7 (2d6) danni da
fuoco.

\textit{\textbf{Presenza Spaventosa.}} Ogni creatura scelta dal drago, che si trovi entro 36 metri da esso e consapevole della sua presenza, deve riuscire un Tiro Salvezza di Volontà DC 19 o restare spaventata per 1 minuto. Una creatura può ripetere il Tiro Salvezza al termine di ciascun suo round, terminando l'effetto se lo riesce. Se il Tiro Salvezza della creatura ha successo o l'effetto ha termine per essa, la creatura è immune alla Presenza Spaventosa del drago per le successive 24 ore.

\textit{\textbf{Soffio Infuocato (Ricarica 5-6).}} Il drago esala fuoco in un cono di 18 metri. Ogni creatura in quell'area deve effettuare un Tiro Salvezza su Riflessi DC 21 e subire 63 (18d6) danni da fuoco se fallisce il Tiro Salvezza, o la metà di questi danni se lo riesce.

\textbf{Azioni Aggiuntive}

Il drago può effettuare 3 Azioni aggiuntive, scelte tra le opzioni seguenti. Può usare solo un'opzione leggendaria alla volta e solo al termine del turno di un'altra creatura. Il drago recupera le Azioni aggiuntive spese all'inizio del proprio round.

\textbf{Attacco di Ala (Costa 2 Azioni).} Il drago batte le ali. Ogni creatura entro 3 metri dal drago deve riuscire un Tiro Salvezza su Riflessi DC 22 o subire 15 (2d6 + 8) danni da botta e venir gettato prono. Il drago può poi volare fino a metà del suo movimento di volo.

\textbf{Attacco di Coda.} Il drago effettua un attacco di coda.

\textbf{Individuare.} Il drago effettua una prova di Saggezza (Consapevolezza).

\textbf{Ecologia}\\
Ambiente: Montagne calde\\
Organizzazione: Solitario\\
\textbf{Tesoro}: Triplo\\
\textbf{Descrizione}\\
Poche creature sono più crudeli e terribili del possente drago rosso. Sovrano dei cromatici, il terribile drago rosso porta rovina e morte nelle terre minacciate dalla sua presenza.\\
\textbf{Incantesimi}\index{Incantesimi da Drago Rosso}\\
Gli incantesimi preferiti di questo Drago sono:\\
- Palla di Fuoco\\
- Nube Incendiaria\\
- Muro di fuoco


\medskip\index[Mostruario]{Drago Rosso Giovane}\textbf{Drago Rosso Giovane}

\textit{Grande drago, caotico malvagio}

\textbf{FORZA} +6

\textbf{DESTREZZA} +0

\textbf{COSTITUZIONE} +5

\textbf{INTELLIGENZA} +2

\textbf{SAGGEZZA} +0

\textbf{CARISMA} +4

\textbf{Iniziativa} +2 -- \textbf{Difesa} 23

\textbf{Punti Ferita} 178 (17d10 + 85)

\textbf{Movimento} 12 m, scalata 12 m, volo 24 m

\textbf{Tiri Salvezza} Tempra +11, Riflessi +8, Volontà +10

\textbf{Competenze} Muoversi Silenziosamente / Nascondersi +4, Consapevolezza +8

\textbf{Immunità al Danno} fuoco

\textbf{Sensi} scurovisione 36 m, vista cieca 9 m

\textbf{Linguaggi} Comune, Draconico

\textbf{Sfida} 10 (5.900 PX)

\textbf{Azioni}

\textit{\textbf{Multiattacco.}} Il drago può effettuare tre attacchi: uno con il morso e due con gli artigli.

\textit{\textbf{Artiglio.} Attacco con arma da mischia}: +16 a colpire, portata 1 m, un bersaglio.

\textit{Colpisce:} 13 (2d6 + 6) danni taglienti, 1 danno da Sanguinamento.

\textit{\textbf{Morso.} Attacco con arma da mischia}: +16 a colpire, portata 3 m, un bersaglio.

\textit{Colpisce:} 17 (2d10 + 6) danni perforanti più 3 (1d6) danni da fuoco.

\textit{\textbf{Soffio Infuocato (Ricarica 5-6).}} Il drago esala fuoco in un cono di 9 metri. Ogni creatura in quell'area deve effettuare un Tiro Salvezza su Riflessi DC 17 e subire 56 (16d6) danni da fuoco se fallisce il Tiro Salvezza, o la metà di questi danni se lo riesce.

\textbf{Ecologia}\\
Ambiente: Montagne calde\\
Organizzazione: Solitario\\
\textbf{Tesoro}: Triplo\\
\textbf{Descrizione}\\
Poche creature sono più crudeli e terribili del possente drago rosso. Sovrano dei cromatici, il terribile drago rosso porta rovina e morte nelle terre minacciate dalla sua presenza.\\
\textbf{Incantesimi}\index{Incantesimi da Drago Rosso}\\
Gli incantesimi preferiti di questo Drago sono:\\
- Palla di Fuoco\\
- Nube Incendiaria\\
- Muro di fuoco


\medskip\index[Mostruario]{Drago Rosso Cucciolo}\textbf{Drago Rosso Cucciolo}

\textit{Media drago, caotico malvagio}

\textbf{FORZA} +4

\textbf{DESTREZZA} +0

\textbf{COSTITUZIONE} +3

\textbf{INTELLIGENZA} +1

\textbf{SAGGEZZA} +0

\textbf{CARISMA} +2

\textbf{Iniziativa} +1 -- \textbf{Difesa} 19

\textbf{Punti Ferita} 75 (10d8 + 30)

\textbf{Movimento} 9 m, scalata 9 m, volo 18 m

\textbf{Tiri Salvezza} Tempra +4, Riflessi +3, Volontà +1

\textbf{Competenze} Muoversi Silenziosamente / Nascondersi +2, Consapevolezza +4

\textbf{Immunità al Danno} fuoco

\textbf{Sensi} scurovisione 18 m, vista cieca 3 m

\textbf{Linguaggi} Draconico

\textbf{Sfida} 4 (1.100 PX)

\textbf{Azioni}

\textit{\textbf{Morso.} Attacco con arma da mischia}: +8 a colpire, portata 1 m, un bersaglio.

\textit{Colpisce:} 9 (1d10 + 4) danni perforanti più 3 (1d6) danni da fuoco.

\textit{\textbf{Soffio Infuocato (Ricarica 5-6).}} Il drago esala fuoco in un cono di 5 metri. Ogni creatura in quell'area deve effettuare un Tiro Salvezza di Riflessi DC 13 e subire 24 (7d6) danni da fuoco se fallisce il Tiro Salvezza, o la metà di questi danni se lo riesce.

\textbf{Ecologia}\\
Ambiente: Montagne calde\\
Organizzazione: Solitario\\
\textbf{Tesoro}: Triplo\\
\textbf{Descrizione}\\
Poche creature sono più crudeli e terribili del possente drago rosso. Sovrano dei cromatici, il terribile drago rosso porta rovina e morte nelle terre minacciate dalla sua presenza.


\medskip\index[Mostruario]{Drago Verde Antico}\textbf{Drago Verde Antico}

\textit{Mastodontica drago, legale malvagio}

\textbf{FORZA} +8

\textbf{DESTREZZA} +1

\textbf{COSTITUZIONE} +7

\textbf{INTELLIGENZA} +5

\textbf{SAGGEZZA} +3

\textbf{CARISMA} +4

\textbf{Iniziativa} +5 -- \textbf{Difesa} 32

\textbf{Punti Ferita} 385 (22x3d6 + 154)

\textbf{Movimento} 12 m, nuoto 12 m, volo 24 m

\textbf{Tiri Salvezza} Tempra +20, Riflessi +12, Volontà +20

\textbf{Competenze} Muoversi Silenziosamente / Nascondersi +8, Ingannare +11, Percepire Emozioni +10, Consapevolezza + 15

\textbf{Immunità al Danno} veleno, armi +1

\textbf{Immunità alle Condizioni}
avvelenato

\textbf{Sensi} scurovisione 36 m, vista cieca 18 m

\textbf{Linguaggi} Comune, Draconico

\textbf{Sfida} 22 (41000 PX)

\textit{\textbf{Anfibio.}} Il drago può respirare aria e acqua.

\textit{\textbf{Resistenza Leggendaria (3/Giorno).}} Se il drago fallisce un Tiro Salvezza, può scegliere invece di riuscire.

\textbf{Azioni}

\textit{\textbf{Multiattacco.}} Il drago può usare la sua Presenza Spaventosa. Poi effettuare tre attacchi: uno con il morso e due con gli artigli.

\textit{\textbf{Artiglio.} Attacco con arma da mischia}: +30 a colpire, portata 3 m, un bersaglio.

\textit{Colpisce:} 15 (2d6 + 8) danni taglienti, 3 danno da Sanguinamento (fino ad un massimo di 20).

\textit{\textbf{Coda.} Attacco con arma da mischia}: +30 a colpire, portata 6 m, un bersaglio.

\textit{Colpisce:} 17 (2d8 + 8) danni da botta.

\textit{\textbf{Morso.} Attacco con arma da mischia}: +30 a colpire, portata 5 metri, un bersaglio.

\textit{Colpisce:} 19 (2d10 + 8) danni perforanti più 10 (3d6) danni da veleno.

\textit{\textbf{Presenza Spaventosa.}} Ogni creatura scelta dal drago, che si trovi entro 36 metri da esso e consapevole della sua presenza, deve riuscire un Tiro Salvezza di Volontà DC 19 o restare spaventata per 1 minuto. Una creatura può ripetere il Tiro Salvezza al termine di ciascun suo round, terminando l'effetto se lo riesce. Se il Tiro Salvezza della creatura ha successo o l'effetto ha termine per essa, la creatura è immune alla Presenza Spaventosa del drago per le successive 24 ore.

\textit{\textbf{Soffio Velenoso (Ricarica 5-6).}} Il drago esala gas velenosi in un cono di 27 metri. Ogni creatura in quell'area deve effettuare un Tiro Salvezza di Tempra DC 22 e subire 77 (22d6) danni da veleno se fallisce il Tiro Salvezza, o la metà di questi danni se lo riesce.

\textbf{Azioni Aggiuntive}

Il drago può effettuare 3 Azioni aggiuntive, scelte tra le opzioni seguenti. Può usare solo un'opzione leggendaria alla volta e solo al termine del turno di un'altra creatura. Il drago recupera le Azioni aggiuntive spese all'inizio del proprio round.

\textbf{Attacco di Ala (Costa 2 Azioni).} Il drago batte le ali. Ogni creatura entro 5 metri dal drago deve riuscire un Tiro Salvezza su Riflessi DC 23 o subire 15 (2d6 + 8) danni da botta e venire gettato prono. Il drago può poi volare fino a metà del suo movimento di volo.

\textbf{Attacco di Coda.} Il drago effettua un attacco di coda.

\textbf{Individuare.} Il drago effettua una prova di Saggezza (Consapevolezza).

\textbf{Ecologia}\\
Ambiente: Foreste Temperate\\
Organizzazione: Solitario\\
\textbf{Tesoro}: Triplo\\
\textbf{Descrizione}\\
I draghi verdi vivono nelle antiche foreste del mondo, vagando in cerca di preda sotto giganteschi tetti di foglie. Di tutti i draghi cromatici, i draghi verdi sono forse quelli con cui ci si può più facilmente accordare diplomaticamente.\\
\textbf{Incantesimi}\index{Incantesimi da Drago Verde}\\
Gli incantesimi preferiti di questo Drago sono:\\
- Nube Mortale\\
- Terreno illusorio\\
- Rimuovi veleno


\medskip\index[Mostruario]{Drago Verde Adulto}\textbf{Drago Verde Adulto}

\textit{Enorme drago, legale malvagio}

\textbf{FORZA} +6

\textbf{DESTREZZA} +1

\textbf{COSTITUZIONE} +5

\textbf{INTELLIGENZA} +4

\textbf{SAGGEZZA} +2

\textbf{CARISMA} +3

\textbf{Iniziativa} +4 -- \textbf{Difesa} 27

\textbf{Punti Ferita} 207 (18d12 + 90)

\textbf{Movimento} 12 m, nuoto 12 m, volo 24 m

\textbf{Tiri Salvezza} Tempra +14, Riflessi +9, Volontà +14

\textbf{Competenze} Muoversi Silenziosamente / Nascondersi +6, Ingannare +8, Percepire Emozioni +7, Consapevolezza +12

\textbf{Immunità al Danno} veleno

\textbf{Immunità alle Condizioni} avvelenato

\textbf{Sensi} scurovisione 36 m, vista cieca 18 m

\textbf{Linguaggi} Comune, Draconico

\textbf{Sfida} 15 (13000 PX)

\textit{\textbf{Anfibio.}} Il drago può respirare aria e acqua.

\textit{\textbf{Resistenza Leggendaria (3/Giorno).}} Se il drago fallisce un Tiro Salvezza, può scegliere invece di riuscire.

\textbf{Azioni}

\textit{\textbf{Multiattacco.}} Il drago può usare la sua Presenza Spaventosa. Poi effettuare tre attacchi: uno con il morso e due con gli artigli.

\textit{\textbf{Artiglio.} Attacco con arma da mischia}: +23 a colpire, portata 1 m, un bersaglio.

\textit{Colpisce:} 13 (2d6 + 6) danni taglienti, 1 danno da Sanguinamento.

\textit{\textbf{Coda.} Attacco con arma da mischia}: +23 a colpire, portata 5 metri, un bersaglio.

\textit{Colpisce:} 15 (2d8 + 6) danni da botta.

\textit{\textbf{Morso.} Attacco con arma da mischia}: +23 a colpire, portata 3 m, un bersaglio.

\textit{Colpisce:} 17 (2d10 + 6) danni perforanti più 7 (2d6) danni da veleno.

\textit{\textbf{Presenza Spaventosa.}} Ogni creatura scelta dal drago, che si trovi entro 36 metri da esso e consapevole della sua presenza, deve riuscire un Tiro Salvezza di Volontà DC 16 o restare spaventata per 1 minuto. Una creatura può ripetere il Tiro Salvezza al termine di ciascun suo round, terminando l'effetto se lo riesce. Se il Tiro Salvezza della creatura ha successo o l'effetto ha termine per essa, la creatura è immune alla Presenza Spaventosa del drago per le successive 24 ore.

\textit{\textbf{Soffio Velenoso (Ricarica 5-6).}} Il drago esala gas velenosi in un cono di 18 metri. Ogni creatura in quell'area deve effettuare un Tiro Salvezza di Tempra DC 18 e subire 56 (16d6) danni da veleno se fallisce il Tiro Salvezza, o la metà di questi danni se lo riesce.

\textbf{Azioni Aggiuntive}

Il drago può effettuare 3 Azioni aggiuntive, scelte tra le opzioni seguenti. Può usare solo un'opzione leggendaria alla volta e solo al termine del turno di un'altra creatura. Il drago recupera le Azioni aggiuntive spese all'inizio del proprio round.

\textbf{Attacco di Ala (Costa 2 Azioni).} Il drago batte le ali. Ogni creatura entro 3 metri dal drago deve riuscire un Tiro Salvezza su Riflessi DC 19 o subire 13 (2d6 + 6) danni da botta e venir gettato prono. Il drago può poi volare fino a metà del suo movimento di volo.

\textbf{Attacco di Coda.} Il drago effettua un attacco di coda.

\textbf{Individuare.} Il drago effettua una prova di Saggezza (Consapevolezza).

\textbf{Ecologia}\\
Ambiente: Foreste Temperate\\
Organizzazione: Solitario\\
\textbf{Tesoro}: Triplo\\
\textbf{Descrizione}\\
I draghi verdi vivono nelle antiche foreste del mondo, vagando in cerca di preda sotto giganteschi tetti di foglie. Di tutti i draghi cromatici, i draghi verdi sono forse quelli con cui ci si può più facilmente accordare diplomaticamente.\\
\textbf{Incantesimi}\index{Incantesimi da Drago Verde}\\
Gli incantesimi preferiti di questo Drago sono:\\
- Nube Mortale\\
- Terreno illusorio\\
- Rimuovi veleno


\medskip\index[Mostruario]{Drago Verde Giovane}\textbf{Drago Verde Giovane}

\textit{Grande drago, legale malvagio}

\textbf{FORZA} +4

\textbf{DESTREZZA} +1

\textbf{COSTITUZIONE} +3

\textbf{INTELLIGENZA} +3

\textbf{SAGGEZZA} +1

\textbf{CARISMA} +2

\textbf{Iniziativa} +3 -- \textbf{Difesa} 22

\textbf{Punti Ferita} 136 (16d10 + 48)

\textbf{Movimento} 12 m, nuoto 12 m, volo 24 m

\textbf{Tiri Salvezza} Tempra +9, Riflessi +7, Volontà +9

\textbf{Competenze} Muoversi Silenziosamente / Nascondersi +4, Ingannare +5, Consapevolezza +7

\textbf{Immunità al Danno} veleno

\textbf{Immunità alle Condizioni} avvelenato

\textbf{Sensi} scurovisione 36 m, vista cieca 9 m
\textbf{Linguaggi} Comune, Draconico

\textbf{Sfida} 8 (3.900 PX)

\textit{\textbf{Anfibio.}} Il drago può respirare aria e acqua.

\textbf{Azioni}

\textit{\textbf{Multiattacco.}} Il drago può effettuare tre attacchi: uno con il morso e due con gli artigli.

\textit{\textbf{Artiglio.} Attacco con arma da mischia}: +11 a colpire, portata 1 m, un bersaglio.

\textit{Colpisce:} 11 (2d6 + 4) danni taglienti, 1 danno da Sanguinamento.

\textit{\textbf{Morso.} Attacco con arma da mischia}: +11 a colpire, portata 3 m, un bersaglio.

\textit{Colpisce:} 15 (2d10 + 4) danni perforanti più 7 (2d6) danni da veleno.

\textit{\textbf{Soffio Velenoso (Ricarica 5-6).}} Il drago esala gas velenosi in un cono di 9 metri. Ogni creatura in quell'area deve effettuare un Tiro Salvezza di Tempra DC 14 e subire 42 (12d6) danni da veleno se fallisce il Tiro Salvezza, o la metà di questi danni se lo riesce.

\textbf{Ecologia}\\
Ambiente: Foreste Temperate\\
Organizzazione: Solitario\\
\textbf{Tesoro}: Triplo\\
\textbf{Descrizione}\\
I draghi verdi vivono nelle antiche foreste del mondo, vagando in cerca di preda sotto giganteschi tetti di foglie. Di tutti i draghi cromatici, i draghi verdi sono forse quelli con cui ci si può più facilmente accordare diplomaticamente.\\
\textbf{Incantesimi}\index{Incantesimi da Drago Verde}\\
Gli incantesimi preferiti di questo Drago sono:\\
- Nube Mortale\\
- Terreno illusorio\\
- Rimuovi veleno

\medskip\index[Mostruario]{Drago Verde Cucciolo}\textbf{Drago Verde Cucciolo}

\textit{Media drago, legale malvagio}

\textbf{FORZA} +2

\textbf{DESTREZZA} +1

\textbf{COSTITUZIONE} +1

\textbf{INTELLIGENZA} +2

\textbf{SAGGEZZA} +0

\textbf{CARISMA} +1

\textbf{Iniziativa} +2 -- \textbf{Difesa} 18

\textbf{Punti Ferita} 38 (7d8 + 7)

\textbf{Movimento} 9 m, nuoto 9 m, volo 18 m

\textbf{Tiri Salvezza} Tempra +3, Riflessi +1, Volontà +0

\textbf{Competenze} Muoversi Silenziosamente / Nascondersi +3, Consapevolezza +4

\textbf{Immunità al Danno} veleno

\textbf{Immunità alle Condizioni} avvelenato

\textbf{Sensi} scurovisione 18 m, vista cieca 3 m

\textbf{Linguaggi} Draconico

\textbf{Sfida} 2 (450 PX)

\textit{\textbf{Anfibio.}} Il drago può respirare aria e acqua.

\textbf{Azioni}

\textit{\textbf{Morso.} Attacco con arma da mischia}: +4 a colpire, portata 1 m, un bersaglio.

\textit{Colpisce:} 7 (1d10 + 2) danni perforanti più 3 (1d6) danni da veleno.

\textit{\textbf{Soffio Velenoso (Ricarica 5-6).}} Il drago esala gas velenosi in un cono di 5 metri. Ogni creatura in quell'area deve effettuare un Tiro Salvezza di Tempra DC 11 e subire 21 (6d6) danni da veleno se fallisce il Tiro Salvezza, o la metà di questi danni se lo riesce.

\textbf{Ecologia}\\
Ambiente: Foreste Temperate\\
Organizzazione: Solitario\\
\textbf{Tesoro}: Triplo\\
\textbf{Descrizione}\\
I draghi verdi vivono nelle antiche foreste del mondo, vagando in cerca di preda sotto giganteschi tetti di foglie. Di tutti i draghi cromatici, i draghi verdi sono forse quelli con cui ci si può più facilmente accordare diplomaticamente.


\subsubsection{Draghi Metallici}

\medskip\index[Mostruario]{Drago d'Argento Antico}\textbf{Drago d'Argento Antico}

\textit{Mastodontica drago, legale buono}

\textbf{FORZA} +10

\textbf{DESTREZZA} +0

\textbf{COSTITUZIONE} +9

\textbf{INTELLIGENZA} +4

\textbf{SAGGEZZA} +2

\textbf{CARISMA} +6

\textbf{Iniziativa} +4 -- \textbf{Difesa} 34

\textbf{Punti Ferita} 487 (25x3d6 + 225)

\textbf{Movimento} 12 m, volo 24 m

\textbf{Tiri Salvezza} Tempra +21, Riflessi +15, Volontà +23

\textbf{Competenze} Arcano +11, Muoversi Silenziosamente / Nascondersi +7, Consapevolezza +16, Storia +11

\textbf{Immunità al Danno} freddo, armi +1

\textbf{Sensi} scurovisione 36 m, vista cieca 18 m

\textbf{Linguaggi} Comune, Draconico

\textbf{Sfida} 23 (50000 PX)

\textit{\textbf{Resistenza Leggendaria (3/Giorno).}} Se il drago fallisce un Tiro Salvezza, può scegliere invece di riuscire.

\textbf{Azioni}

\textit{\textbf{Multiattacco.}} Il drago può usare la sua Presenza Spaventosa. Poi effettuare tre attacchi: uno con il morso e due con gli
artigli.

\textit{\textbf{Artiglio.} Attacco con arma da mischia}: +30 a colpire, portata 3 m, un bersaglio.

\textit{Colpisce:} 17 (2d6 + 10) danni taglienti, 3 danno da Sanguinamento (fino ad un massimo di 20).

\textit{\textbf{Coda.} Attacco con arma da mischia}: +30 a colpire, portata 6 m, un bersaglio.

\textit{Colpisce:} 19 (2d8 + 10) danni da botta.

\textit{\textbf{Morso.} Attacco con arma da mischia}: +30 a colpire, portata 5 metri, un bersaglio.

\textit{Colpisce:} 21 (2d10 + 10) danni perforanti.

\textit{\textbf{Presenza Spaventosa.}} Ogni creatura scelta dal drago, che si trovi entro 36 metri da esso e consapevole della sua presenza, deve riuscire un Tiro Salvezza di Volontà DC 21 o restare spaventata per 1 minuto. Una creatura può ripetere il Tiro Salvezza al termine di ciascun suo round, terminando l'effetto se lo riesce. Se il Tiro Salvezza della creatura ha successo o l'effetto ha termine per essa, la creatura è immune alla Presenza Spaventosa del drago per le successive 24 ore.

\textit{\textbf{Arma a Soffio (Ricarica 5-6).}} Il drago usa una delle seguenti armi a soffio:

\textit{Soffio Gelido.} Il drago esala un'esplosione ghiacciata in un cono di 27 metri. Ogni creatura nell'area deve effettuare un Tiro Salvezza su Tempra DC 24, subendo 67 (15d8) danni da freddo se fallisce il Tiro Salvezza, o la metà di questi danni se lo riesce.

\textit{Soffio Paralizzante.} Il drago esala un gas paralizzante in un cono di 24 metri. Ogni creatura nell'area deve riuscire un Tiro Salvezza su Tempra 24 o restare paralizzata per 1 minuto. Una creatura può ripetere il Tiro Salvezza al termine di ciascun suo round, terminando l'effetto per sé in caso di successo.

\textit{\textbf{Mutare Forma.}} Il drago può trasformarsi magicamente in un umanoide o bestia il cui grado di sfida sia pari o inferiore al proprio, o tornare alla sua vera forma. Alla morte ritorna alla sua vera forma.

Qualsiasi equipaggiamento stia indossando o trasportando viene assorbito o trasportato nella nuova forma (a scelta del drago).

Nella nuova forma, il drago mantiene i suoi Tratti, Punti Ferita, Dadi Vita, la facoltà di parlare, le competenze, la Resistenza Leggendaria, le azioni da tana, e i punteggi di Intelligenza, Saggezza e Carisma, oltre a questa azione. Le sue statistiche e capacità vengono altrimenti rimpiazzate da quelle della nuova forma, eccetto Azioni aggiuntive della nuova forma.

\textbf{Azioni Aggiuntive}

Il drago può effettuare 3 Azioni aggiuntive, scelte tra le opzioni seguenti. Può usare solo un'opzione leggendaria alla volta e solo al termine del turno di un'altra creatura. Il drago recupera le Azioni aggiuntive spese all'inizio del proprio round.

\textbf{Attacco di Ala (Costa 2 Azioni).} Il drago batte le ali. Ogni creatura entro 5 metri dal drago deve riuscire un Tiro Salvezza su Riflessi DC 25 o subire 17 (2d6 + 10) danni da botta e venir gettato prono. Il drago può poi volare fino a metà della sua velocità di volo.

\textbf{Attacco di Coda.} Il drago effettua un attacco di coda.

\textbf{Individuare.} Il drago effettua una prova di Saggezza (Consapevolezza).

\textbf{Ecologia}\\
Ambiente: Montagne Temperate\\
Organizzazione: Solitario\\
\textbf{Tesoro}: Triplo\\
\textbf{Descrizione}\\
Tra tutti i draghi, quelli d'argento sono i più coraggiosi, e si attengono ad un codice cavalleresco che impone loro di aiutare i deboli, sconfiggere il male e comportarsi in modo onorevole.\\
\textbf{Incantesimi}\index{Incantesimi da Drago Argento}\\
Gli incantesimi preferiti di questo Drago sono:\\
- Cono di freddo\\
- Tempesta di ghiaccio\\
- Creazione\\
- Capanna

\medskip\index[Mostruario]{Drago d'Argento Adulto}\textbf{Drago d'Argento Adulto}

\textit{Enorme drago, legale buono}

\textbf{FORZA} +8

\textbf{DESTREZZA} +0

\textbf{COSTITUZIONE} +7

\textbf{INTELLIGENZA} +3

\textbf{SAGGEZZA} +1

\textbf{CARISMA} +5

\textbf{Iniziativa} +3 -- \textbf{Difesa} 27

\textbf{Punti Ferita} 243 (18d12 + 126)

\textbf{Movimento} 12 m, volo 24 m

\textbf{Tiri Salvezza} Tempra +15, Riflessi +12, Volontà +17

\textbf{Competenze} Arcano +8, Muoversi Silenziosamente / Nascondersi +5, Consapevolezza +11, Storia +8

\textbf{Immunità al Danno} freddo

\textbf{Sensi} scurovisione 36 m, vista cieca 18 m

\textbf{Linguaggi} Comune, Draconico

\textbf{Sfida} 16 (1500 PX)

\textit{\textbf{Resistenza Leggendaria (3/Giorno).}} Se il drago fallisce un Tiro Salvezza, può scegliere invece di riuscire.

\textbf{Azioni}

\textit{\textbf{Multiattacco.}} Il drago può usare la sua Presenza Spaventosa. Poi effettuare tre attacchi: uno con il morso e due con gli artigli.

\textit{\textbf{Artiglio.} Attacco con arma da mischia}: +27 a colpire, portata 1 m, un bersaglio.

\textit{Colpisce:} 15 (2d6 + 8) danni taglienti, 1 danno da Sanguinamento.

\textit{\textbf{Coda.} Attacco con arma da mischia}: +27 a colpire, portata 5 metri, un bersaglio.

\textit{Colpisce:} 17 (2d8 + 8) danni da botta.

\textit{\textbf{Morso.} Attacco con arma da mischia}: +27 a colpire, portata 3 m, un bersaglio.

\textit{Colpisce:} 19 (2d10 + 8) danni perforanti.

\textit{\textbf{Presenza Spaventosa.}} Ogni creatura scelta dal drago, che si trovi entro 36 metri da esso e consapevole della sua presenza, deve riuscire un Tiro Salvezza di Volontà DC 18 o restare spaventata per 1 minuto. Una creatura può ripetere il Tiro Salvezza al termine di ciascun suo round, terminando l'effetto se lo riesce. Se il Tiro Salvezza della creatura ha successo o l'effetto ha termine per essa, la creatura è immune alla Presenza Spaventosa del drago per le successive 24 ore.

\textit{\textbf{Arma a Soffio (Ricarica 5-6).}} Il drago usa una delle seguenti armi a soffio:

\textit{Soffio Gelido.} Il drago esala un'esplosione ghiacciata in un cono di 18 metri. Ogni creatura nell'area deve effettuare un Tiro Salvezza su Tempra DC 20, subendo 58 (13d8) danni da freddo se fallisce il Tiro Salvezza, o la metà di questi danni se lo riesce.

\textit{Soffio Paralizzante.} Il drago esala un gas paralizzante in un cono di 18 metri. Ogni creatura nell'area deve riuscire un Tiro Salvezza su Tempra 20 o restare paralizzata per 1 minuto. Una creatura può ripetere il Tiro Salvezza al termine di ciascun suo round, terminando l'effetto per sé in caso di successo.

\textit{\textbf{Mutare Forma.}} Il drago può trasformarsi magicamente in un umanoide o bestia il cui grado di sfida sia pari o inferiore al proprio, o tornare alla sua vera forma. Alla morte ritorna alla sua vera forma. Qualsiasi equipaggiamento stia indossando o trasportando viene assorbito o trasportato nella nuova forma (a scelta del drago).

Nella nuova forma, il drago mantiene i suoi Tratti, Punti Ferita, Dadi Vita, la facoltà di parlare, le competenze, la Resistenza Leggendaria, le azioni da tana, e i punteggi di Intelligenza, Saggezza e Carisma, oltre a questa azione. Le sue statistiche e capacità vengono altrimenti rimpiazzate da quelle della nuova forma, eccetto Azioni aggiuntive della nuova forma.

\textbf{Azioni Aggiuntive}

Il drago può effettuare 3 Azioni aggiuntive, scelte tra le opzioni seguenti. Può usare solo un'opzione leggendaria alla volta e solo al termine del turno di un'altra creatura. Il drago recupera le Azioni aggiuntive spese all'inizio del proprio round.

\textbf{Attacco di Ala (Costa 2 Azioni).} Il drago batte le ali. Ogni creatura entro 3 metri dal drago deve riuscire un Tiro Salvezza su Riflessi DC 21 o subire 15 (2d6 + 8) danni da botta e venir gettato prono. Il drago può poi volare fino a metà del suo movimento di volo.

\textbf{Attacco di Coda.} Il drago effettua un attacco di coda.

\textbf{Individuare.} Il drago effettua una prova di Saggezza (Consapevolezza).

\textbf{Ecologia}\\
Ambiente: Montagne Temperate\\
Organizzazione: Solitario\\
\textbf{Tesoro}: Triplo\\
\textbf{Descrizione}\\
Tra tutti i draghi, quelli d'argento sono i più coraggiosi, e si attengono ad un codice cavalleresco che impone loro di aiutare i deboli, sconfiggere il male e comportarsi in modo onorevole.\\
\textbf{Incantesimi}\index{Incantesimi da Drago Argento}\\
Gli incantesimi preferiti di questo Drago sono:\\
- Cono di freddo\\
- Tempesta di ghiaccio\\
- Creazione\\
- Capanna

\medskip\index[Mostruario]{Drago d'Argento Giovane}\textbf{Drago d'Argento Giovane}

\textit{Grande drago, legale buono}

\textbf{FORZA} +6

\textbf{DESTREZZA} +0

\textbf{COSTITUZIONE} +5

\textbf{INTELLIGENZA} +2

\textbf{SAGGEZZA} +0

\textbf{CARISMA} +4

\textbf{Iniziativa} +2 -- \textbf{Difesa} 23

\textbf{Punti Ferita} 168 (16d10 + 80)

\textbf{Movimento} 12 m, volo 24 m

\textbf{Tiri Salvezza} Tempra +10, Riflessi +8, Volontà +12

\textbf{Competenze} Arcano +6, Muoversi Silenziosamente / Nascondersi +4, Consapevolezza +8, Storia +6

\textbf{Immunità al Danno} freddo

\textbf{Sensi} scurovisione 36 m, vista cieca 9 m

\textbf{Linguaggi} Comune, Draconico

\textbf{Sfida} 9 (5000 PX)

\textbf{Azioni}

\textit{\textbf{Multiattacco.}} Il drago può effettuare tre attacchi: uno con il morso e due con gli artigli.

\textit{\textbf{Artiglio.} Attacco con arma da mischia}: +15 a colpire, portata 1 m, un bersaglio.

\textit{Colpisce:} 13 (2d6 + 6) danni taglienti, 1 danno da Sanguinamento.

\textit{\textbf{Morso.} Attacco con arma da mischia}: +15 a colpire, portata 3 m, un bersaglio.

\textit{Colpisce:} 17 (2d10 + 6) danni perforanti.

\textit{\textbf{Arma a Soffio (Ricarica 5-6).}} Il drago usa una delle seguenti armi a soffio:

\textit{Soffio Gelido.} Il drago esala un'esplosione ghiacciata in un cono di 9 metri. Ogni creatura nell'area deve effettuare un Tiro Salvezza su Tempra DC 17, subendo 54 (12d8) danni da freddo se fallisce il Tiro Salvezza, o la metà di questi danni se lo riesce.

\textit{Soffio Paralizzante.} Il drago esala un gas paralizzante in un cono di 9 metri. Ogni creatura nell'area deve riuscire un Tiro Salvezza su Tempra 17 o restare paralizzata per 1 minuto. Una creatura può ripetere il Tiro Salvezza al termine di ciascun suo round, terminando l'effetto per sé in caso di successo.

\textbf{Ecologia}\\
Ambiente: Montagne Temperate\\
Organizzazione: Solitario\\
\textbf{Tesoro}: Triplo\\
\textbf{Descrizione}\\
Tra tutti i draghi, quelli d'argento sono i più coraggiosi, e si attengono ad un codice cavalleresco che impone loro di aiutare i deboli, sconfiggere il male e comportarsi in modo onorevole.\\
\textbf{Incantesimi}\index{Incantesimi da Drago Argento}\\
Gli incantesimi preferiti di questo Drago sono:\\
- Cono di freddo\\
- Tempesta di ghiaccio\\
- Creazione\\
- Capanna

\medskip\index[Mostruario]{Drago d'Argento Cucciolo}\textbf{Drago d'Argento Cucciolo}

\textit{Media drago, legale buono}

\textbf{FORZA} +4

\textbf{DESTREZZA} +0

\textbf{COSTITUZIONE} +3

\textbf{INTELLIGENZA} +1

\textbf{SAGGEZZA} +0

\textbf{CARISMA} +2

\textbf{Iniziativa} +1 -- \textbf{Difesa} 18

\textbf{Punti Ferita} 45 (6d8 + 18)

\textbf{Movimento} 9 m, volo 18 m

\textbf{Tiri Salvezza} Tempra +3, Riflessi +3, Volontà +2

\textbf{Competenze} Muoversi Silenziosamente / Nascondersi +2, Consapevolezza +4

\textbf{Immunità al Danno} freddo

\textbf{Sensi} scurovisione 18 m, vista cieca 3 m

\textbf{Linguaggi} Draconico

\textbf{Sfida} 2 (450 PX)

\textbf{Azioni}

\textit{\textbf{Morso.} Attacco con arma da mischia}: +6 a colpire, portata 1 m, un bersaglio.

\textit{Colpisce:} 9 (1d10 + 4) danni perforanti.

\textit{\textbf{Arma a Soffio (Ricarica 5-6).}} Il drago usa una delle seguenti armi a soffio:

\textit{Soffio Gelido.} Il drago esala un'esplosione ghiacciata in un cono di 5 metri. Ogni creatura nell'area deve effettuare un Tiro Salvezza su Tempra 13, subendo 18 (4d8) danni da freddo se fallisce il Tiro Salvezza, o la metà di questi danni se lo riesce.

\textit{Soffio Paralizzante.} Il drago esala un gas paralizzante in un cono di 5 metri. Ogni creatura nell'area deve riuscire un Tiro Salvezza su Tempra 13 o restare paralizzata per 1 minuto. Una creatura può ripetere il Tiro Salvezza al termine di ciascun suo round, terminando l'effetto per sé in caso di successo.

\textbf{Ecologia}\\
Ambiente: Montagne Temperate\\
Organizzazione: Solitario\\
\textbf{Tesoro}: Triplo\\
\textbf{Descrizione}\\
Tra tutti i draghi, quelli d'argento sono i più coraggiosi, e si attengono ad un codice cavalleresco che impone loro di aiutare i deboli, sconfiggere il male e comportarsi in modo onorevole.


\medskip\index[Mostruario]{Drago di Bronzo Antico}\textbf{Drago di Bronzo Antico}

\textit{Mastodontica drago, caotico buono}

\textbf{FORZA} +9

\textbf{DESTREZZA} +0

\textbf{COSTITUZIONE} +8

\textbf{INTELLIGENZA} +4

\textbf{SAGGEZZA} +3

\textbf{CARISMA} +5

\textbf{Iniziativa} +4 -- \textbf{Difesa} 33

\textbf{Punti Ferita} 444 (24x3d6 + 192)

\textbf{Movimento} 12 m, nuoto 12 m, volo 24 m

\textbf{Tiri Salvezza} Tempra +21, Riflessi +13, Volontà +21

\textbf{Competenze} Muoversi Silenziosamente / Nascondersi +7, Percepire Emozioni +10, Consapevolezza +17

\textbf{Immunità al Danno} fulmine, armi +1

\textbf{Sensi} scurovisione 36 m, vista cieca 18 m

\textbf{Linguaggi} Comune, Draconico

\textbf{Sfida} 22 (41000 PX)

\textit{\textbf{Anfibio.}} Il drago può respirare aria e acqua.

\textit{\textbf{Resistenza Leggendaria (3/Giorno).}} Se il drago fallisce un Tiro Salvezza, può scegliere invece di riuscire.

\textbf{Azioni}

\textit{\textbf{Multiattacco.}} Il drago può usare la sua Presenza Spaventosa. Poi effettuare tre attacchi: uno con il morso e due con gli artigli.

\textit{\textbf{Artiglio.} Attacco con arma da mischia}: +30 a colpire, portata 3 m, un bersaglio.

\textit{Colpisce:} 16 (2d6 + 9) danni taglienti, 3 danno da Sanguinamento (fino ad un massimo di 20).

\textit{\textbf{Coda.} Attacco con arma da mischia}: +30 a colpire, portata 6 m, un bersaglio.

\textit{Colpisce:} 18 (2d8 + 9) danni da botta.

\textit{\textbf{Morso.} Attacco con arma da mischia}: +30 a colpire, portata 5 metri, un bersaglio.

\textit{Colpisce:} 20 (2d10 + 9) danni perforanti.

\textit{\textbf{Presenza Spaventosa.}} Ogni creatura scelta dal drago, che si trovi entro 36 metri da esso e consapevole della sua presenza, deve riuscire un Tiro Salvezza di Volontà DC 20 o restare spaventata per 1 minuto. Una creatura può ripetere il Tiro Salvezza al termine di ciascun suo round, terminando l'effetto se lo riesce. Se il Tiro Salvezza della creatura ha successo o l'effetto ha termine per essa, la creatura è immune alla Presenza Spaventosa del drago per le successive 24 ore.

\textit{\textbf{Arma a Soffio (Ricarica 5-6).}} Il drago usa una delle seguenti armi a soffio:

\textit{Soffio Fulminante.} Il drago esala fulmini in una linea lunga 36 metri e larga 3 metri. Ogni creatura sulla linea deve effettuare un Tiro Salvezza su Riflessi DC 23, subendo 88 (16d10) danni da fulmine se fallisce il Tiro Salvezza, o la metà di questi danni se lo riesce. \textit{Soffio Repulsivo.} Il drago esala dell'energia repulsiva in un cono di 9 metri. Ogni creatura in quell'area deve riuscire un Tiro Salvezza su Tempra DC 23, altrimenti viene allontana di 18 metri dal
drago.

\textit{\textbf{Mutare Forma.}} Il drago può trasformarsi magicamente in un umanoide o bestia il cui grado di sfida sia pari o inferiore al proprio, o tornare alla sua vera forma. Alla morte ritorna alla sua vera forma. Qualsiasi equipaggiamento stia indossando o trasportando viene assorbito o trasportato nella nuova forma (a scelta del drago).

Nella nuova forma, il drago mantiene i suoi Tratti, Punti Ferita, Dadi Vita, la facoltà di parlare, le competenze, la Resistenza Leggendaria, le azioni da tana, e i punteggi di Intelligenza, Saggezza e Carisma, oltre a questa azione. Le sue statistiche e capacità vengono altrimenti rimpiazzate da quelle della nuova forma, eccetto Azioni aggiuntive della nuova forma.

\textbf{Azioni Aggiuntive}

Il drago può effettuare 3 Azioni aggiuntive, scelte tra le opzioni seguenti. Può usare solo un'opzione leggendaria alla volta e solo al termine del turno di un'altra creatura. Il drago recupera le Azioni aggiuntive spese all'inizio del proprio round.

\textbf{Attacco di Ala (Costa 2 Azioni).} Il drago batte le ali. Ogni creatura entro 5 metri dal drago deve riuscire un Tiro Salvezza su Riflessi DC 24 o subire 16 (2d6 + 9) danni da botta e venir gettato prono. Il drago può poi volare fino a metà del suo movimento di volo.

\textbf{Attacco di Coda.} Il drago effettua un attacco di coda.

\textbf{Individuare.} Il drago effettua una prova di Saggezza (Consapevolezza).

\textbf{Ecologia}\\
Ambiente: Zone Costiere Temperate\\
Organizzazione: Solitario\\
\textbf{Tesoro}: Triplo\\
\textbf{Descrizione}\\
I draghi di bronzo sono noti per allearsi con viaggiatori ed avventurieri se causa e ricompensa sono giuste e adeguate\\
\textbf{Incantesimi}\index{Incantesimi da Drago Bronzo}\\
Gli incantesimi preferiti di questo Drago sono:\\
- Controllare Tempo Atmosferico\\
- Invocare il Fulmine


\medskip\index[Mostruario]{Drago di Bronzo Adulto}\textbf{Drago di Bronzo Adulto}

\textit{Enorme drago, caotico buono}

\textbf{FORZA} +7

\textbf{DESTREZZA} +0

\textbf{COSTITUZIONE} +6

\textbf{INTELLIGENZA} +3

\textbf{SAGGEZZA} +2

\textbf{CARISMA} +4

\textbf{Iniziativa} +3 -- \textbf{Difesa} 27

\textbf{Punti Ferita} 212 (17d12 + 102)

\textbf{Movimento} 12 m, nuoto 12 m, volo 24 m

\textbf{Tiri Salvezza} Tempra +15, Riflessi +10, Volontà +15

\textbf{Competenze} Muoversi Silenziosamente / Nascondersi +5, Percepire Emozioni +7, Consapevolezza +12

\textbf{Immunità al Danno} fulmine

\textbf{Sensi} scurovisione 36 m, vista cieca 18 m

\textbf{Linguaggi} Comune, Draconico

\textbf{Sfida} 15 (13000 PX)

\textit{\textbf{Anfibio.}} Il drago può respirare aria e acqua.

\textit{\textbf{Resistenza Leggendaria (3/Giorno).}} Se il drago fallisce un Tiro Salvezza, può scegliere invece di riuscire.

\textbf{Azioni}

\textit{\textbf{Multiattacco.}} Il drago può usare la sua Presenza Spaventosa e poi effettuare tre attacchi: uno con il morso e due con gli artigli.

\textit{\textbf{Artiglio.} Attacco con arma da mischia}: +24 a colpire, portata 1 m, un bersaglio.

\textit{Colpisce:} 14 (2d6 + 7) danni taglienti, 1 danno da Sanguinamento.

\textit{\textbf{Coda.} Attacco con arma da mischia}: +24 a colpire, portata 5 metri, un bersaglio.

\textit{Colpisce:} 16 (2d8 + 7) danni da botta.

\textit{\textbf{Morso.} Attacco con arma da mischia}: +24 a colpire, portata 3 m, un bersaglio.

\textit{Colpisce:} 18 (2d10 + 7) danni perforanti.

\textit{\textbf{Presenza Spaventosa.}} Ogni creatura scelta dal drago, che si trovi entro 36 metri da esso e consapevole della sua presenza, deve riuscire un Tiro Salvezza di Volontà DC 17 o restare spaventata per 1 minuto. Una creatura può ripetere il Tiro Salvezza al termine di ciascun suo round, terminando l'effetto se lo riesce. Se il Tiro Salvezza della creatura ha successo o l'effetto ha termine per essa, la creatura è immune alla Presenza Spaventosa del drago per le successive 24 ore.

\textit{\textbf{Arma a Soffio (Ricarica 5-6).}} Il drago usa una delle seguenti armi a soffio:

\textit{Soffio Fulminante.} Il drago esala fulmini in una linea lunga 27 metri e larga 1 metro. Ogni creatura sulla linea deve effettuare un Tiro Salvezza di Riflessi DC 19, subendo 66 (12d10) danni da fulmine se fallisce il Tiro Salvezza, o la metà di questi danni se lo riesce. \textit{Soffio Repulsivo.} Il drago esala dell'energia repulsiva in un cono di 9 metri. Ogni creatura in quell'area deve riuscire un Tiro Salvezza su Tempra DC 19, altrimenti viene allontana di 18 metri dal drago.

\textit{\textbf{Mutare Forma.}} Il drago può trasformarsi magicamente in un umanoide o bestia il cui grado di sfida sia pari o inferiore al proprio, o tornare alla sua vera forma. Alla morte ritorna alla sua vera forma. Qualsiasi equipaggiamento stia indossando o trasportando viene assorbito o trasportato nella nuova forma (a scelta del drago).

Nella nuova forma, il drago mantiene i suoi Tratti, Punti Ferita, Dadi Vita, la facoltà di parlare, le competenze, la Resistenza Leggendaria, le azioni da tana, e i punteggi di Intelligenza, Saggezza e Carisma, oltre a questa azione. Le sue statistiche e capacità vengono altrimenti rimpiazzate da quelle della nuova forma, eccetto Azioni aggiuntive della nuova forma.

\textbf{Azioni Aggiuntive}

Il drago può effettuare 3 Azioni aggiuntive, scelte tra le opzioni seguenti. Può usare solo un'opzione leggendaria alla volta e solo al termine del turno di un'altra creatura. Il drago recupera le Azioni aggiuntive spese all'inizio del proprio round.

\textbf{Attacco di Ala (Costa 2 Azioni).} Il drago batte le ali. Ogni creatura entro 3 metri dal drago deve riuscire un Tiro Salvezza su Riflessi DC 20 o subire 14 (2d6 + 7) danni da botta e venir gettato prono. Il drago può poi volare fino a metà del suo movimento di volo.

\textbf{Attacco di Coda.} Il drago effettua un attacco di coda.

\textbf{Individuare.} Il drago effettua una prova di Saggezza (Consapevolezza).

\textbf{Ecologia}\\
Ambiente: Zone Costiere Temperate\\
Organizzazione: Solitario\\
\textbf{Tesoro}: Triplo\\
\textbf{Descrizione}\\
I draghi di bronzo sono noti per allearsi con viaggiatori ed avventurieri se causa e ricompensa sono giuste e adeguate\\
\textbf{Incantesimi}\index{Incantesimi da Drago Bronzo}\\
Gli incantesimi preferiti di questo Drago sono:\\
- Controllare Tempo Atmosferico\\
- Invocare il Fulmine


\medskip\index[Mostruario]{Drago di Bronzo Giovane}\textbf{Drago di Bronzo Giovane}

\textit{Grande drago, caotico buono}

\textbf{FORZA} +5

\textbf{DESTREZZA} +0

\textbf{COSTITUZIONE} +4

\textbf{INTELLIGENZA} +2

\textbf{SAGGEZZA} +1

\textbf{CARISMA} +3

\textbf{Iniziativa} +2 -- \textbf{Difesa} 22

\textbf{Punti Ferita} 142 (15d10 + 60)

\textbf{Movimento} 12 m, nuoto 12 m, volo 24 m

\textbf{Tiri Salvezza} Tempra +10, Riflessi +8, Volontà +10

\textbf{Competenze} Muoversi Silenziosamente / Nascondersi +3, Percepire Emozioni +4, Consapevolezza +7

\textbf{Immunità al Danno} fulmine

\textbf{Sensi} scurovisione 36 m, vista cieca 9 m

\textbf{Linguaggi} Comune, Draconico

\textbf{Sfida} 8 (3.900 PX)

\textit{\textbf{Anfibio.}} Il drago può respirare aria e acqua.

\textbf{Azioni}

\textit{\textbf{Multiattacco.}} Il drago può usare effettuare tre attacchi: uno con il morso e due con gli artigli.

\textit{\textbf{Artiglio.} Attacco con arma da mischia}: +12 a colpire, portata 1 m, un bersaglio.

\textit{Colpisce:} 12 (2d6 + 5) danni taglienti, 1 danno da Sanguinamento.

\textit{\textbf{Morso.} Attacco con arma da mischia}: +12 a colpire, portata 3 m, un bersaglio.

\textit{Colpisce:} 16 (2d10 + 5) danni perforanti.

\textit{\textbf{Arma a Soffio (Ricarica 5-6).}} Il drago usa una delle seguenti armi a soffio:

\textit{Soffio Fulminante.} Il drago esala fulmini in una linea lunga 18 metri e larga 1 metro. Ogni creatura sulla linea deve effettuare un Tiro Salvezza di Riflessi DC 15, subendo 55 (10d10) danni da fulmine se fallisce il Tiro Salvezza, o la metà di questi danni se lo riesce.

\textit{Soffio Repulsivo.} Il drago esala dell'energia repulsiva in un cono di 9 metri. Ogni creatura in quell'area deve riuscire un Tiro Salvezza su Tempra DC 15, altrimenti viene allontana di 12 metri dal drago.

\textbf{Ecologia}\\
Ambiente: Zone Costiere Temperate\\
Organizzazione: Solitario\\
\textbf{Tesoro}: Triplo\\
\textbf{Descrizione}\\
I draghi di bronzo sono noti per allearsi con viaggiatori ed avventurieri se causa e ricompensa sono giuste e adeguate\\
\textbf{Incantesimi}\index{Incantesimi da Drago Bronzo}\\
Gli incantesimi preferiti di questo Drago sono:\\
- Controllare Tempo Atmosferico\\
- Invocare il Fulmine


\medskip\index[Mostruario]{Drago di Bronzo Cucciolo}\textbf{Drago di Bronzo Cucciolo}

\textit{Media drago, caotico buono}

\textbf{FORZA} +3

\textbf{DESTREZZA} +0

\textbf{COSTITUZIONE} +2

\textbf{INTELLIGENZA} +1

\textbf{SAGGEZZA} +0

\textbf{CARISMA} +2

\textbf{Iniziativa} +1 -- \textbf{Difesa} 18

\textbf{Punti Ferita} 32 (5d8 + 10)

\textbf{Movimento} 9 m, nuoto 9 m, volo 18 m

\textbf{Tiri Salvezza} Tempra +2, Riflessi +1, Volontà +1

\textbf{Competenze} Muoversi Silenziosamente / Nascondersi +2, Consapevolezza +4

\textbf{Immunità al Danno} fulmine

\textbf{Sensi} scurovisione 18 m, vista cieca 3 m

\textbf{Linguaggi} Draconico

\textbf{Sfida} 2 (450 PX)

\textit{\textbf{Anfibio.}} Il drago può respirare aria e acqua.

\textbf{Azioni}

\textit{\textbf{Morso.} Attacco con arma da mischia}: +5 a colpire,
portata 1 m, un bersaglio.

\textit{Colpisce:} 8 (1d10 + 3) danni perforanti.

\textit{\textbf{Arma a Soffio (Ricarica 5-6).}} Il drago usa una delle seguenti armi a soffio:

\textit{Soffio Fulminante.} Il drago esala fulmini in una linea lunga 12 metri e larga 1 metro. Ogni creatura sulla linea deve effettuare un Tiro Salvezza di Riflessi DC 12, subendo 16 (3d10) danni da fulmine se fallisce il Tiro Salvezza, o la metà di questi danni se lo riesce.

\textit{Soffio Repulsivo.} Il drago esala dell'energia repulsiva in un cono di 9 metri. Ogni creatura in quell'area deve riuscire un Tiro Salvezza su Tempra DC 12, altrimenti viene allontana di 9 metri dal drago.

\textbf{Ecologia}\\
Ambiente: Zone Costiere Temperate\\
Organizzazione: Solitario\\
\textbf{Tesoro}: Triplo\\
\textbf{Descrizione}\\
I draghi di bronzo sono noti per allearsi con viaggiatori ed avventurieri se causa e ricompensa sono giuste e adeguate.

\medskip\index[Mostruario]{Drago d'Oro Antico}\textbf{Drago d'Oro Antico}

\textit{Mastodontica drago, legale buono}

\textbf{FORZA} +10

\textbf{DESTREZZA} +2

\textbf{COSTITUZIONE} +9

\textbf{INTELLIGENZA} +4

\textbf{SAGGEZZA} +3

\textbf{CARISMA} +9

\textbf{Iniziativa} +4 -- \textbf{Difesa} 34

\textbf{Punti Ferita} 546 (28x3d6 + 252)

\textbf{Movimento} 12 m, nuoto 12 m, volo 24 m

\textbf{Tiri Salvezza} Tempra +23, Riflessi +14, Volontà +24

\textbf{Competenze} Muoversi Silenziosamente / Nascondersi +9, Percepire Emozioni +10, Consapevolezza +17, Ingannare +16

\textbf{Immunità al Danno} fuoco, armi +1

\textbf{Sensi} scurovisione 36 m, vista cieca 18 m

\textbf{Linguaggi} Comune, Draconico

\textbf{Sfida} 24 (62000 PX)

\textit{\textbf{Anfibio.}} Il drago può respirare aria e acqua.

\textit{\textbf{Resistenza Leggendaria (3/Giorno).}} Se il drago fallisce un Tiro Salvezza, può scegliere invece di riuscire.

\textbf{Azioni}

\textit{\textbf{Multiattacco.}} Il drago può usare la sua Presenza Spaventosa. Poi effettuare tre attacchi: uno con il morso e due con gli artigli.

\textit{\textbf{Artiglio.} Attacco con arma da mischia}: +30 a colpire, portata 3 m, un bersaglio.

\textit{Colpisce:} 17 (2d6 + 10) danni taglienti, 3 danno da Sanguinamento (fino ad un massimo di 20).

\textit{\textbf{Coda.} Attacco con arma da mischia}: +30 a colpire, portata 6 m, un bersaglio.

\textit{Colpisce:} 19 (2d8 + 10) danni da botta.

\textit{\textbf{Morso.} Attacco con arma da mischia}: +30 a colpire, portata 5 metri, un bersaglio.

\textit{Colpisce:} 21 (2d10 + 10) danni perforanti.

\textit{\textbf{Presenza Spaventosa.}} Ogni creatura scelta dal drago, che si trovi entro 36 metri da esso e consapevole della sua presenza, deve riuscire un Tiro Salvezza di Volontà DC 24 o restare spaventata per 1 minuto. Una creatura può ripetere il Tiro Salvezza al termine di ciascun suo round, terminando l'effetto se lo riesce. Se il Tiro Salvezza della creatura ha successo o l'effetto ha termine per essa, la creatura è immune alla Presenza Spaventosa del drago per le successive 24 ore.

\textit{\textbf{Arma a Soffio (Ricarica 5-6).}} Il drago usa una delle seguenti armi a soffio:

\textit{Soffio Infuocato.} Il drago esala fuoco in un cono di 27 metri. Ogni creatura nell'area deve effettuare un Tiro Salvezza di Riflessi DC 24, subendo 71(13d10) danni da fuoco se fallisce il Tiro Salvezza, o la metà di questi danni se lo riesce.

\textit{Soffio Indebolente.} Il drago esala del gas in un cono di 27 metri. Ogni creatura in quell'area deve riuscire un Tiro Salvezza su Tempra DC 24 o avere -1d6 ai tiri di attacco basati sulla Forza, prove di Forza, e Tiri Salvezza su Tempra per 1 minuto. Una creatura può ripetere il Tiro Salvezza al termine di ciascun suo round, terminando l'effetto su di sé in caso di successo.

\textit{\textbf{Mutare Forma.}} Il drago può trasformarsi magicamente in un umanoide o bestia il cui grado di sfida sia pari o inferiore al proprio, o tornare alla sua vera forma. Alla morte ritorna alla sua vera forma. Qualsiasi equipaggiamento stia indossando o trasportando viene assorbito o trasportato nella nuova forma (a scelta del drago).

Nella nuova forma, il drago mantiene i suoi Tratti, Punti Ferita, Dadi Vita, la facoltà di parlare, le competenze, la Resistenza Leggendaria, le azioni da tana, e i punteggi di Intelligenza, Saggezza e Carisma, oltre a questa azione. Le sue statistiche e capacità vengono altrimenti rimpiazzate da quelle della nuova forma, eccetto Azioni aggiuntive della nuova forma.

\textbf{Azioni Aggiuntive}

Il drago può effettuare 3 Azioni aggiuntive, scelte tra le opzioni seguenti. Può usare solo un'opzione leggendaria alla volta e solo al termine del turno di un'altra creatura. Il drago recupera le Azioni aggiuntive spese all'inizio del proprio round.

\textbf{Attacco di Ala (Costa 2 Azioni).} Il drago batte le ali. Ogni creatura entro 5 metri dal drago deve riuscire un Tiro Salvezza su Riflessi DC 25 o subire 17 (2d6 + 10) danni da botta e venir gettato prono. Il drago può poi volare fino a metà del suo movimento di volo.

\textbf{Attacco di Coda.} Il drago effettua un attacco di coda.

\textbf{Individuare.} Il drago effettua una prova di Saggezza (Consapevolezza).

\textbf{Ecologia}\\
Ambiente: Pianure calde\\
Organizzazione: Solitario\\
\textbf{Tesoro}: Triplo\\
\textbf{Descrizione}\\
I draghi d'oro sono l'emblema della virtù. Gli altri draghi metallici li riveriscono come agenti delle potenze divine e membri esemplari della razza draconica, e spesso li cercano per consigli o aiuto.\\
\textbf{Incantesimi}\index{Incantesimi da Drago d'Oro}\\
Gli incantesimi preferiti di questo Drago sono:\\
- Blocca Mostri\\
- Muro di Fuoco\\
- Porta Dimensionale

\medskip\index[Mostruario]{Drago d'Oro Adulto}\textbf{Drago d'Oro Adulto}

\textit{Enorme drago, legale buono}

\textbf{FORZA} +8

\textbf{DESTREZZA} +2

\textbf{COSTITUZIONE} +7

\textbf{INTELLIGENZA} +3

\textbf{SAGGEZZA} +2

\textbf{CARISMA} +7

\textbf{Iniziativa} +3 -- \textbf{Difesa} 28

\textbf{Punti Ferita} 256 (19d12 + 133)

\textbf{Movimento} 12 m, nuoto 12 m, volo 24 m

\textbf{Tiri Salvezza} Tempra +17, Riflessi +11, Volontà +18

\textbf{Competenze} Muoversi Silenziosamente / Nascondersi +8, Percepire Emozioni +8, Consapevolezza +14, Ingannare +13

\textbf{Immunità al Danno} fuoco

\textbf{Sensi} scurovisione 36 m, vista cieca 18 m

\textbf{Linguaggi} Comune, Draconico

\textbf{Sfida} 17 (18000 PX)

\textit{\textbf{Anfibio.}} Il drago può respirare aria e acqua.

\textit{\textbf{Resistenza Leggendaria (3/Giorno).}} Se il drago fallisce un Tiro Salvezza, può scegliere invece di riuscire.

\textbf{Azioni}

\textit{\textbf{Multiattacco.}} Il drago può usare la sua Presenza Spaventosa. Poi effettuare tre attacchi: uno con il morso e due con gli artigli.

\textit{\textbf{Artiglio.} Attacco con arma da mischia}: +28 a colpire, portata 1 m, un bersaglio.

\textit{Colpisce:} 15 (2d6 + 8) danni taglienti, 1 danno da Sanguinamento.

\textit{\textbf{Coda.} Attacco con arma da mischia}: +28 a colpire, portata 5 metri, un bersaglio.

\textit{Colpisce:} 17 (2d8 + 8) danni da botta.

\textit{\textbf{Morso.} Attacco con arma da mischia}: +28 a colpire, portata 3 m, un bersaglio.

\textit{Colpisce:} 19 (2d10 + 8) danni perforanti.

\textit{\textbf{Presenza Spaventosa.}} Ogni creatura scelta dal drago, che si trovi entro 36 metri da esso e consapevole della sua presenza, deve riuscire un Tiro Salvezza di Volontà DC 21 o restare spaventata per 1 minuto. Una creatura può ripetere il Tiro Salvezza al termine di ciascun suo round, terminando l'effetto se lo riesce. Se il Tiro Salvezza della creatura ha successo o l'effetto ha termine per essa, la creatura è immune alla Presenza Spaventosa del drago per le successive 24 ore.

\textit{\textbf{Arma a Soffio (Ricarica 5-6).}} Il drago usa una delle seguenti armi a soffio:

\textit{Soffio Infuocato.} Il drago esala fuoco in un cono di 18 metri. Ogni creatura nell'area deve effettuare un Tiro Salvezza di Riflessi DC 21, subendo 66 (12d10) danni da fuoco se fallisce il Tiro Salvezza, o la metà di questi danni se lo riesce.

\textit{Soffio Indebolente.} Il drago esala del gas in un cono di 18 metri. Ogni creatura in quell'area deve riuscire un Tiro Salvezza su Tempra DC 21 o avere -1d6 ai tiri di attacco basati sulla Forza, prove di Forza, e Tiri Salvezza su Tempra per 1 minuto. Una creatura può ripetere il Tiro Salvezza al termine di ciascun suo round, terminando l'effetto su di sé in caso di successo.

\textit{\textbf{Mutare Forma.}} Il drago può trasformarsi magicamente in un umanoide o bestia il cui grado di sfida sia pari o inferiore al proprio, o tornare alla sua vera forma. Alla morte ritorna alla sua vera forma. Qualsiasi equipaggiamento stia indossando o trasportando viene assorbito o trasportato nella nuova forma (a scelta del drago).

Nella nuova forma, il drago mantiene i suoi Tratti, Punti Ferita, Dadi Vita, la facoltà di parlare, le competenze, la Resistenza Leggendaria, le azioni da tana, e i punteggi di Intelligenza, Saggezza e Carisma, oltre a questa azione. Le sue statistiche e capacità vengono altrimenti rimpiazzate da quelle della nuova forma, eccetto Azioni aggiuntive della nuova forma.

\textbf{Azioni Aggiuntive}

Il drago può effettuare 3 Azioni aggiuntive, scelte tra le opzioni seguenti. Può usare solo un'opzione leggendaria alla volta e solo al termine del turno di un'altra creatura. Il drago recupera le Azioni aggiuntive spese all'inizio del proprio round.

\textbf{Attacco di Ala (Costa 2 Azioni).} Il drago batte le ali. Ogni creatura entro 3 metri dal drago deve riuscire un Tiro Salvezza su Riflessi DC 22 o subire 15 (2d6 + 8) danni da botta e venir gettato prono. Il drago può poi volare fino a metà del suo movimento di volo.

\textbf{Attacco di Coda.} Il drago effettua un attacco di coda.

\textbf{Individuare.} Il drago effettua una prova di Saggezza (Consapevolezza).

\textbf{Ecologia}\\
Ambiente: Pianure calde\\
Organizzazione: Solitario\\
\textbf{Tesoro}: Triplo\\
\textbf{Descrizione}\\
I draghi d'oro sono l'emblema della virtù. Gli altri draghi metallici li riveriscono come agenti delle potenze divine e membri esemplari della razza draconica, e spesso li cercano per consigli o aiuto.\\
\textbf{Incantesimi}\index{Incantesimi da Drago d'Oro}\\
Gli incantesimi preferiti di questo Drago sono:\\
- Blocca Mostri\\
- Muro di Fuoco\\
- Porta Dimensionale

\medskip\index[Mostruario]{Drago d'Oro Giovane}\textbf{Drago d'Oro Giovane}

\textit{Grande drago, legale buono}

\textbf{FORZA} +6

\textbf{DESTREZZA} +2

\textbf{COSTITUZIONE} +5

\textbf{INTELLIGENZA} +3

\textbf{SAGGEZZA} +1

\textbf{CARISMA} +5

\textbf{Iniziativa} +3 -- \textbf{Difesa} 23

\textbf{Punti Ferita} 178 (17d10 + 85)

\textbf{Movimento} 12 m, nuoto 12 m, volo 24 m

\textbf{Tiri Salvezza} Tempra +12, Riflessi +9, Volontà +13

\textbf{Competenze} Muoversi Silenziosamente / Nascondersi +6, Percepire Emozioni +5, Consapevolezza +9, Ingannare +9

\textbf{Immunità al Danno} fuoco

\textbf{Sensi} scurovisione 36 m, vista cieca 9 m

\textbf{Linguaggi} Comune, Draconico

\textbf{Sfida} 10 (5.900 PX)

\textit{\textbf{Anfibio.}} Il drago può respirare aria e acqua.

\textbf{Azioni}

\textit{\textbf{Multiattacco.}} Il drago può effettuare tre attacchi: uno con il morso e due con gli artigli.

\textit{\textbf{Artiglio.} Attacco con arma da mischia}: +16 a colpire, portata 1 m, un bersaglio.

\textit{Colpisce:} 13 (2d6 + 6) danni taglienti, 1 danno da Sanguinamento.

\textit{\textbf{Morso.} Attacco con arma da mischia}: +16 a colpire, portata 3 m, un bersaglio.

\textit{Colpisce:} 17 (2d10 + 6) danni perforanti.

\textit{\textbf{Arma a Soffio (Ricarica 5-6).}} Il drago usa una delle seguenti armi a soffio:

\textit{Soffio Infuocato.} Il drago esala fuoco in un cono di 9 metri. Ogni creatura nell'area deve effettuare un Tiro Salvezza di Riflessi DC 17, subendo 55 (10d10) danni da fuoco se fallisce il Tiro Salvezza, o la metà di questi danni se lo riesce.

\textit{Soffio Indebolente.} Il drago esala del gas in un cono di 9 metri. Ogni creatura in quell'area deve riuscire un Tiro Salvezza di Tempra DC 17 o avere -1d6 ai tiri di attacco basati sulla Forza, prove di Forza, e Tiri Salvezza su Tempra per 1 minuto. Una creatura può ripetere il Tiro Salvezza al termine di ciascun suo round, terminando l'effetto su di sé in caso di successo.

\textbf{Ecologia}\\
Ambiente: Pianure calde\\
Organizzazione: Solitario\\
\textbf{Tesoro}: Triplo\\
\textbf{Descrizione}\\
I draghi d'oro sono l'emblema della virtù. Gli altri draghi metallici li riveriscono come agenti delle potenze divine e membri esemplari della razza draconica, e spesso li cercano per consigli o aiuto.\\
\textbf{Incantesimi}\index{Incantesimi da Drago d'Oro}\\
Gli incantesimi preferiti di questo Drago sono:\\
- Blocca Mostri\\
- Muro di Fuoco\\
- Porta Dimensionale

\medskip\index[Mostruario]{Drago d'Oro Cucciolo}\textbf{Drago d'Oro Cucciolo}

\textit{Media drago, legale buono}

\textbf{FORZA} +4

\textbf{DESTREZZA} +2

\textbf{COSTITUZIONE} +3

\textbf{INTELLIGENZA} +2

\textbf{SAGGEZZA} +0

\textbf{CARISMA} +3

\textbf{Iniziativa} +2 -- \textbf{Difesa} 19

\textbf{Punti Ferita} 60 (8d8 + 24)

\textbf{Movimento} 9 m, nuoto 9 m, volo 18 m

\textbf{Tiri Salvezza} Tempra +3, Riflessi +2, Volontà +1

\textbf{Competenze} Muoversi Silenziosamente / Nascondersi +4, Consapevolezza +4

\textbf{Immunità al Danno} fuoco

\textbf{Sensi} scurovisione 18 m, vista cieca 3 m

\textbf{Linguaggi} Draconico

\textbf{Sfida} 3 (700 PX)

\textit{\textbf{Anfibio.}} Il drago può respirare aria e acqua.

\textbf{Azioni}

\textit{\textbf{Morso.} Attacco con arma da mischia}: +6 a colpire, portata 1 m, un bersaglio.

\textit{Colpisce:} 9 (1d10 + 4) danni perforanti.

\textit{\textbf{Arma a Soffio (Ricarica 5-6).}} Il drago usa una delle seguenti armi a soffio:

\textit{Soffio Infuocato.} Il drago esala fuoco in un cono di 5 metri. Ogni creatura nell'area deve effettuare un Tiro Salvezza di Riflessi DC 13, subendo 22 (4d10) danni da fuoco se fallisce il Tiro Salvezza, o la metà di questi danni se lo riesce.

\textit{Soffio Indebolente.} Il drago esala del gas in un cono di 5 metri. Ogni creatura in quell'area deve riuscire un Tiro Salvezza su Tempra DC 13 o avere -1d6 ai tiri di attacco basati sulla Forza, prove di Forza, e Tiri Salvezza su Tempra per 1 minuto. Una creatura può ripetere il Tiro Salvezza al termine di ciascun suo round, terminando l'effetto su di sé in caso di successo.

\textbf{Ecologia}\\
Ambiente: Pianure calde\\
Organizzazione: Solitario\\
\textbf{Tesoro}: Triplo\\
\textbf{Descrizione}\\
I draghi d'oro sono l'emblema della virtù. Gli altri draghi metallici li riveriscono come agenti delle potenze divine e membri esemplari della razza draconica, e spesso li cercano per consigli o aiuto.

\medskip\index[Mostruario]{Drago d'Ottone Antico}\textbf{Drago d'Ottone Antico}

\textit{Mastodontica drago, caotico buono}

\textbf{FORZA} +8

\textbf{DESTREZZA} +0

\textbf{COSTITUZIONE} +7

\textbf{INTELLIGENZA} +3

\textbf{SAGGEZZA} +2

\textbf{CARISMA} +4

\textbf{Iniziativa} +3 -- \textbf{Difesa} 30

\textbf{Punti Ferita} 297 (17x3d6 + 119)

\textbf{Movimento} 12 m, scavo 12 m, volo 24 m

\textbf{Tiri Salvezza} Tempra +20, Riflessi +13, Volontà +18

\textbf{Competenze} Muoversi Silenziosamente / Nascondersi +6, Consapevolezza +14, Ingannare +10, Storia +9

\textbf{Immunità al Danno} fuoco, armi +1

\textbf{Sensi} scurovisione 36 m, vista cieca 18 m

\textbf{Linguaggi} Comune, Draconico

\textbf{Sfida} 20 (25000 PX)

\textit{\textbf{Resistenza Leggendaria (3/Giorno).}} Se il drago fallisce un Tiro Salvezza, può scegliere invece di riuscire.

\textbf{Azioni}

\textit{\textbf{Multiattacco.}} Il drago può usare la sua Presenza Spaventosa. Poi effettuare tre attacchi: uno con il morso e due con gli artigli.

\textit{\textbf{Artiglio.} Attacco con arma da mischia}: +30 a colpire, portata 3 m, un bersaglio.

\textit{Colpisce:} 15 (2d6 + 8) danni taglienti, 3 danno da Sanguinamento (fino ad un massimo di 20)

\textit{\textbf{Coda.} Attacco con arma da mischia}: +30 a colpire, portata 6 m, un bersaglio.

\textit{Colpisce:} 17 (2d8 + 8) danni da botta.

\textit{\textbf{Morso.} Attacco con arma da mischia}: +30 a colpire, portata 5 metri, un bersaglio.

\textit{Colpisce:} 19 (2d10 + 8) danni perforanti.

\textit{\textbf{Presenza Spaventosa.}} Ogni creatura scelta dal drago, che si trovi entro 36 metri da esso e consapevole della sua presenza, deve riuscire un Tiro Salvezza di Volontà DC 18 o restare spaventata per 1 minuto. Una creatura può ripetere il Tiro Salvezza al termine di ciascun suo round, terminando l'effetto se lo riesce. Se il Tiro Salvezza della creatura ha successo o l'effetto ha termine per essa, la creatura è immune alla Presenza Spaventosa del drago per le successive 24 ore.

\textit{\textbf{Arma a Soffio (Ricarica 5-6).}} Il drago usa una delle seguenti armi a soffio:

\textit{Soffio Infuocato.} Il drago esala fuoco in una linea lunga 27 metri e larga 3 metri. Ogni creatura sulla linea deve effettuare un Tiro Salvezza su Riflessi DC 21, subendo 56 (16d6) danni da fuoco se fallisce il Tiro Salvezza, o la metà di questi danni se lo riesce.

\textit{Soffio Soporifero.} Il drago esala del gas soporifero in un cono di 27 metri. Ogni creatura in quell'area deve riuscire un Tiro Salvezza su Tempra 21 o cadere svenuta per 10 minuti. Questo effetto
termina se la creatura svenuta subisce danni o qualcuno impiega un'azione per risvegliarla.

\textit{\textbf{Mutare Forma.}} Il drago può trasformarsi magicamente in un umanoide o bestia il cui grado di sfida sia pari o inferiore al proprio, o tornare alla sua vera forma. Alla morte ritorna alla sua vera forma. Qualsiasi equipaggiamento stia indossando o trasportando viene assorbito o trasportato nella nuova forma (a scelta del drago).

Nella nuova forma, il drago mantiene i suoi Tratti, Punti Ferita, Dadi Vita, la facoltà di parlare, le competenze, la Resistenza Leggendaria, le azioni da tana, e i punteggi di Intelligenza, Saggezza e Carisma, oltre a questa azione. Le sue statistiche e capacità vengono altrimenti rimpiazzate da quelle della nuova forma, eccetto Azioni aggiuntive della nuova forma.

\textbf{Azioni Aggiuntive}

Il drago può effettuare 3 Azioni aggiuntive, scelte tra le opzioni seguenti. Può usare solo un'opzione leggendaria alla volta e solo al termine del turno di un'altra creatura. Il drago recupera le Azioni aggiuntive spese all'inizio del proprio round.

\textbf{Attacco di Ala (Costa 2 Azioni).} Il drago batte le ali. Ogni creatura entro 5 metri dal drago deve riuscire un Tiro Salvezza su Riflessi DC 22 o subire 15 (2d6 + 8) danni da botta e venir gettato prono. Il drago può poi volare fino a metà del suo movimento di volo.

\textbf{Attacco di Coda.} Il drago effettua un attacco di coda.

\textbf{Individuare.} Il drago effettua una prova di Saggezza (Consapevolezza).

\textbf{Ecologia}\\
Ambiente: Deserti Caldi\\
Organizzazione: Solitario\\
\textbf{Tesoro}: Triplo\\
\textbf{Descrizione}\\
Ottimi conversatori, i draghi d'ottone preferiscono parlare invece che combattere. I draghi d'ottone fanno la tana vicino agli insediamenti umanoidi, dove possono udire le notizie e i pettegolezzi più recenti.\\
\textbf{Incantesimi}\index{Incantesimi da Drago d'Ottone}\\
Gli incantesimi preferiti di questo Drago sono:\\
- Inviare\\
- Trama Ipnotica\\
- Lingue


\medskip\index[Mostruario]{Drago d'Ottone Adulto}\textbf{Drago d'Ottone Adulto}

\textit{Enorme drago, caotico buono}

\textbf{FORZA} +6

\textbf{DESTREZZA} +0

\textbf{COSTITUZIONE} +5

\textbf{INTELLIGENZA} +2

\textbf{SAGGEZZA} +1

\textbf{CARISMA} +3

\textbf{Iniziativa} +2 -- \textbf{Difesa} 25

\textbf{Punti Ferita} 172 (15d12 + 75)

\textbf{Movimento} 12 m, scavo 9 m, volo 24 m

\textbf{Tiri Salvezza} Tempra +14, Riflessi +10, Volontà +12

\textbf{Competenze} Muoversi Silenziosamente / Nascondersi +5, Consapevolezza +11, Ingannare +8, Storia +7

\textbf{Immunità al Danno} fuoco

\textbf{Sensi} scurovisione 36 m, vista cieca 18 m

\textbf{Linguaggi} Comune, Draconico

\textbf{Sfida} 13 (10000 PX)

\textit{\textbf{Resistenza Leggendaria (3/Giorno).}} Se il drago fallisce un Tiro Salvezza, può scegliere invece di riuscire.

\textbf{Azioni}

\textit{\textbf{Multiattacco.}} Il drago può usare la sua Presenza Spaventosa. Poi effettuare tre attacchi: uno con il morso e due con gli artigli.

\textit{\textbf{Artiglio.} Attacco con arma da mischia}: +20 a colpire, portata 1 m, un bersaglio.

\textit{Colpisce:} 13 (2d6 + 6) danni taglienti, 1 danno da Sanguinamento.

\textit{\textbf{Coda.} Attacco con arma da mischia}: +20 a colpire, portata 5 metri, un bersaglio.

\textit{Colpisce:} 15 (2d8 + 6) danni da botta.

\textit{\textbf{Morso.} Attacco con arma da mischia}: +20 a colpire, portata 3 m, un bersaglio.

\textit{Colpisce:} 17 (2d10 + 6) danni perforanti.

\textit{\textbf{Presenza Spaventosa.}} Ogni creatura scelta dal drago, che si trovi entro 36 metri da esso e consapevole della sua presenza, deve riuscire un Tiro Salvezza di Volontà DC 16 o restare spaventata per 1 minuto. Una creatura può ripetere il Tiro Salvezza al termine di ciascun suo round, terminando l'effetto se lo riesce. Se il Tiro Salvezza della creatura ha successo o l'effetto ha termine per essa, la creatura è immune alla Presenza Spaventosa del drago per le successive 24 ore.

\textit{\textbf{Arma a Soffio (Ricarica 5-6).}} Il drago usa una delle seguenti armi a soffio:

\textit{Soffio Infuocato.} Il drago esala fuoco in una linea lunga 18 metri e larga 1 metro. Ogni creatura sulla linea deve effettuare un Tiro Salvezza di Riflessi DC 18, subendo 45 (13d6) danni da fuoco se fallisce il Tiro Salvezza, o la metà di questi danni se lo riesce.

\textit{Soffio Soporifero.} Il drago esala del gas soporifero in un cono di 18 metri. Ogni creatura in quell'area deve riuscire un Tiro Salvezza su Tempra 18 o cadere svenuta per 10 minuti. Questo effetto termina se la creatura svenuta subisce danni o qualcuno impiega un'azione per risvegliarla.

\textbf{Azioni Aggiuntive}

Il drago può effettuare 3 Azioni aggiuntive, scelte tra le opzioni seguenti. Può usare solo un'opzione leggendaria alla volta e solo al termine del turno di un'altra creatura. Il drago recupera le Azioni aggiuntive spese all'inizio del proprio round.

\textbf{Attacco di Ala (Costa 2 Azioni).} Il drago batte le ali. Ogni creatura entro 3 metri dal drago deve riuscire un Tiro Salvezza su Riflessi DC 19 o subire 13 (2d6 + 6) danni da botta e venir gettato prono. Il drago può poi volare fino a metà del suo movimento di volo.

\textbf{Attacco di Coda.} Il drago effettua un attacco di coda.

\textbf{Individuare.} Il drago effettua una prova di Saggezza (Consapevolezza).

\textbf{Ecologia}\\
Ambiente: Deserti Caldi\\
Organizzazione: Solitario\\
\textbf{Tesoro}: Triplo\\
\textbf{Descrizione}\\
Ottimi conversatori, i draghi d'ottone preferiscono parlare invece che combattere. I draghi d'ottone fanno la tana vicino agli insediamenti umanoidi, dove possono udire le notizie e i pettegolezzi più recenti.\\
\textbf{Incantesimi}\index{Incantesimi da Drago d'Ottone}\\
Gli incantesimi preferiti di questo Drago sono:\\
- Inviare\\
- Trama Ipnotica\\
- Lingue

\medskip\index[Mostruario]{Drago d'Ottone Giovane}\textbf{Drago d'Ottone Giovane}

\textit{Grande drago, caotico buono}

\textbf{FORZA} +4

\textbf{DESTREZZA} +0

\textbf{COSTITUZIONE} +3

\textbf{INTELLIGENZA} +1

\textbf{SAGGEZZA} +0

\textbf{CARISMA} +2

\textbf{Iniziativa} +1 -- \textbf{Difesa} 20

\textbf{Punti Ferita} 110 (13d10 + 39)

\textbf{Movimento} 12 m, scavo 6 m, volo 24 m

\textbf{Tiri Salvezza} Tempra +9, Riflessi +8, Volontà +7

\textbf{Competenze} Muoversi Silenziosamente / Nascondersi +3, Consapevolezza +6, Ingannare +5

\textbf{Immunità al Danno} fuoco

\textbf{Sensi} scurovisione 36 m, vista cieca 9 m

\textbf{Linguaggi} Comune, Draconico

\textbf{Sfida} 6 (2.300 PX)

\textbf{Azioni}

\textit{\textbf{Multiattacco.}} Il drago può effettuare tre attacchi: uno con il morso e due con gli artigli.

\textit{\textbf{Artiglio.} Attacco con arma da mischia}: +7 a colpire, portata 1 m, un bersaglio.

\textit{Colpisce:} 11 (2d6 + 4) danni taglienti, 1 danno da Sanguinamento.

\textit{\textbf{Morso.} Attacco con arma da mischia}: +7 a colpire, portata 3 m, un bersaglio.

\textit{Colpisce:} 15 (2d10 + 4) danni perforanti.

\textit{\textbf{Arma a Soffio (Ricarica 5-6).}} Il drago usa una delle seguenti armi a soffio:

\textit{Soffio Infuocato.} Il drago esala fuoco in una linea lunga 12 metri e larga 1 metro. Ogni creatura sulla linea deve effettuare un Tiro Salvezza di Riflessi DC 14, subendo 42 (12d6) danni da fuoco se fallisce il Tiro Salvezza, o la metà di questi danni se lo riesce. \textit{Soffio Soporifero.} Il drago esala del gas soporifero in un cono di 9 metri. Ogni creatura in quell'area deve riuscire un Tiro Salvezza su Tempra 14 o cadere svenuta per 5 minuti. Questo effetto termina se la creatura svenuta subisce danni o qualcuno impiega un'azione per risvegliarla.

\textbf{Ecologia}\\
Ambiente: Deserti Caldi\\
Organizzazione: Solitario\\
\textbf{Tesoro}: Triplo\\
\textbf{Descrizione}\\
Ottimi conversatori, i draghi d'ottone preferiscono parlare invece che combattere. I draghi d'ottone fanno la tana vicino agli insediamenti umanoidi, dove possono udire le notizie e i pettegolezzi più recenti.\\
\textbf{Incantesimi}\index{Incantesimi da Drago d'Ottone}\\
Gli incantesimi preferiti di questo Drago sono:\\
- Inviare\\
- Trama Ipnotica\\
- Lingue

\medskip\index[Mostruario]{Drago d'Ottone Cucciolo}\textbf{Drago d'Ottone Cucciolo}

\textit{Media drago, caotico buono}

\textbf{FORZA} +2

\textbf{DESTREZZA} +0

\textbf{COSTITUZIONE} +1

\textbf{INTELLIGENZA} +0

\textbf{SAGGEZZA} +0

\textbf{CARISMA} +1

\textbf{Iniziativa} +0 -- \textbf{Difesa} 17

\textbf{Punti Ferita} 16 (3d8 + 3)

\textbf{Movimento} 9 m, scavo 5 metri, volo 18 m

\textbf{Tiri Salvezza} Tempra +2, Riflessi +0, Volontà +1

\textbf{Competenze} Muoversi Silenziosamente / Nascondersi +2, Consapevolezza +4

\textbf{Immunità al Danno} fuoco

\textbf{Sensi} scurovisione 18 m, vista cieca 3 m

\textbf{Linguaggi} Draconico

\textbf{Sfida} 1 (200 PX)

\textbf{Azioni}

\textit{\textbf{Morso.} Attacco con arma da mischia}: +4 a colpire, portata 1 m, un bersaglio.

\textit{Colpisce:} 7 (1d10 + 2) danni perforanti.

\textit{\textbf{Arma a Soffio (Ricarica 5-6).}} Il drago usa una delle seguenti armi a soffio:

\textit{Soffio Infuocato.} Il drago esala fuoco in una linea lunga 6 metri e larga 1 metro. Ogni creatura sulla linea deve effettuare un Tiro Salvezza su Riflessi DC 11, subendo 14 (4d6) danni da fuoco se fallisce il Tiro Salvezza, o la metà di questi danni se lo riesce.

\textit{Soffio Soporifero.} Il drago esala del gas soporifero in un cono di 5 metri. Ogni creatura in quell'area deve riuscire un Tiro Salvezza su Tempra 11 o cadere svenuta per 1 minuto. Questo effetto termina se la creatura svenuta subisce danni o qualcuno impiega un'azione per risvegliarla.

\textbf{Ecologia}\\
Ambiente: Deserti Caldi\\
Organizzazione: Solitario\\
\textbf{Tesoro}: Triplo\\
\textbf{Descrizione}\\
Ottimi conversatori, i draghi d'ottone preferiscono parlare invece che combattere. I draghi d'ottone fanno la tana vicino agli insediamenti umanoidi, dove possono udire le notizie e i pettegolezzi più recenti.


\medskip\index[Mostruario]{Drago di Rame Antico}\textbf{Drago di Rame Antico}

\textit{Mastodontica drago, caotico buono}

\textbf{FORZA} +8

\textbf{DESTREZZA} +1

\textbf{COSTITUZIONE} +7

\textbf{INTELLIGENZA} +5

\textbf{SAGGEZZA} +3

\textbf{CARISMA} +4

\textbf{Iniziativa} +5 -- \textbf{Difesa} 33

\textbf{Punti Ferita} 350 (20x3d6 + 140)

\textbf{Movimento} 12 m, scalata 12 m, volo 24 m

\textbf{Tiri Salvezza} Tempra +20, Riflessi +13, Volontà +19

\textbf{Competenze} Muoversi Silenziosamente / Nascondersi +8, Ingannare +11, Consapevolezza +17

\textbf{Immunità al Danno} acido, armi +1

\textbf{Sensi} scurovisione 36 m, vista cieca 18 m

\textbf{Linguaggi} Comune, Draconico

\textbf{Sfida} 21 (33000 PX)

\textit{\textbf{Resistenza Leggendaria (3/Giorno).}} Se il drago fallisce un Tiro Salvezza, può scegliere invece di riuscire.

\textbf{Azioni}

\textit{\textbf{Multiattacco.}} Il drago può usare la sua Presenza Spaventosa. Poi effettuare tre attacchi: uno con il morso e due con gli artigli.

\textit{\textbf{Artiglio.} Attacco con arma da mischia}: +30 a colpire, portata 3 m, un bersaglio.

\textit{Colpisce:} 15 (2d6 + 8) danni taglienti, 3 danno da Sanguinamento (fino ad un massimo di 20).

\textit{\textbf{Coda.} Attacco con arma da mischia}: +30 a colpire, portata 6 m, un bersaglio.

\textit{Colpisce:} 17 (2d8 + 8) danni da botta.

\textit{\textbf{Morso.} Attacco con arma da mischia}: +30 a colpire, portata 5 metri, un bersaglio.

\textit{Colpisce:} 19 (2d10 + 8) danni perforanti.

\textit{\textbf{Presenza Spaventosa.}} Ogni creatura scelta dal drago, che si trovi entro 36 metri da esso e consapevole della sua presenza, deve riuscire un Tiro Salvezza di Volontà DC 19 o restare spaventata per 1 minuto. Una creatura può ripetere il Tiro Salvezza al termine di ciascun suo round, terminando l'effetto se lo riesce. Se il Tiro Salvezza della creatura ha successo o l'effetto ha termine per essa, la creatura è immune alla Presenza Spaventosa del drago per le successive 24 ore.

\textit{\textbf{Arma a Soffio (Ricarica 5-6).}} Il drago usa una delle seguenti armi a soffio:

\textit{Soffio Acido.} Il drago esala acido in una linea lunga 27 metri e larga 3 metri. Ogni creatura sulla linea deve effettuare un Tiro Salvezza su Riflessi DC 22, subendo 63 (14d8) danni da acido se fallisce il Tiro Salvezza, o la metà di questi danni se lo riesce.

\textit{Soffio Rallentante.} Il drago esala del gas in un cono di 27 metri. Ogni creatura in quell'area deve riuscire un Tiro Salvezza su Tempra DC 22. Se fallisce il Tiro Salvezza, la creatura non può usare la sua reazione, ha la velocità dimezzata, e non può effettuare più di un attacco durante il suo round. Inoltre, la creatura può usare un'azione o un'azione bonus, ma non entrambe. Questi effetti permangono 1 minuto. La creatura può ripetere il Tiro Salvezza al termine di ciascun suo round, terminando l'effetto su di sé in caso di successo.

\textit{\textbf{Mutare Forma.}} Il drago può trasformarsi magicamente in un umanoide o bestia il cui grado di sfida sia pari o inferiore al proprio, o tornare alla sua vera forma. Alla morte ritorna alla sua vera forma. Qualsiasi equipaggiamento stia indossando o trasportando viene assorbito o trasportato nella nuova forma (a scelta del drago).

Nella nuova forma, il drago mantiene i suoi Tratti, Punti Ferita, Dadi Vita, la facoltà di parlare, le competenze, la Resistenza Leggendaria, le azioni da tana, e i punteggi di Intelligenza, Saggezza e Carisma, oltre a questa azione. Le sue statistiche e capacità

vengono altrimenti rimpiazzate da quelle della nuova forma, eccetto Azioni aggiuntive della nuova forma.

\textbf{Azioni Aggiuntive}

Il drago può effettuare 3 Azioni aggiuntive, scelte tra le opzioni seguenti. Può usare solo un'opzione leggendaria alla volta e solo al termine del turno di un'altra creatura. Il drago recupera le Azioni aggiuntive spese all'inizio del proprio round.

\textbf{Attacco di Ala (Costa 2 Azioni).} Il drago batte le ali. Ogni creatura entro 5 metri dal drago deve riuscire un Tiro Salvezza su Riflessi DC 23 o subire 15 (2d6 + 8) danni da botta e venir gettato prono. Il drago può poi volare fino a metà del suo movimento di volo.

\textbf{Attacco di Coda.} Il drago effettua un attacco di coda.

\textbf{Individuare.} Il drago effettua una prova di Saggezza (Consapevolezza).

\textbf{Ecologia}\\
Ambiente: Colline Calde\\
Organizzazione: Solitario\\
\textbf{Tesoro}: Triplo\\
\textbf{Descrizione}\\
Questo drago capriccioso durante il combattimento cerca di ostacolare e frustrare i suoi nemici.\\
\textbf{Incantesimi}\index{Incantesimi da Drago di Rame}\\
Gli incantesimi preferiti di questo Drago sono:\\
- Metamorfosi\\
- Confusione\\
- Nube maleodorante


\medskip\index[Mostruario]{Drago di Rame Adulto}\textbf{Drago di Rame Adulto}

\textit{Enorme drago, caotico buono}

\textbf{FORZA} +6

\textbf{DESTREZZA} +1

\textbf{COSTITUZIONE} +5

\textbf{INTELLIGENZA} +4

\textbf{SAGGEZZA} +2

\textbf{CARISMA} +3

\textbf{Iniziativa} +4 -- \textbf{Difesa} 25

\textbf{Punti Ferita} 184 (16d12 + 80)

\textbf{Movimento} 12 m, scalata 12 m, volo 24 m

\textbf{Tiri Salvezza} Tempra +14, Riflessi +10, Volontà +13

\textbf{Competenze} Muoversi Silenziosamente / Nascondersi +6, Ingannare +8, Consapevolezza +12

\textbf{Immunità al Danno} acido

\textbf{Sensi} scurovisione 36 m, vista cieca 18 m

\textbf{Linguaggi} Comune, Draconico

\textbf{Sfida} 14 (11.500 PX)

\textit{\textbf{Resistenza Leggendaria (3/Giorno).}} Se il drago fallisce un Tiro Salvezza, può scegliere invece di riuscire.

\textbf{Azioni}

\textit{\textbf{Multiattacco.}} Il drago può usare la sua Presenza Spaventosa. Poi effettuare tre attacchi: uno con il morso e due con gli artigli.

\textit{\textbf{Artiglio.} Attacco con arma da mischia}: +22 a colpire, portata 1 m, un bersaglio.

\textit{Colpisce:} 13 (2d6 + 6) danni taglienti, 1 danno da Sanguinamento.

\textit{\textbf{Coda.} Attacco con arma da mischia}: +22 a colpire, portata 5 metri, un bersaglio.

\textit{Colpisce:} 15 (2d8 + 6) danni da botta.

\textit{\textbf{Morso.} Attacco con arma da mischia}: +22 a colpire, portata 3 m, un bersaglio.

\textit{Colpisce:} 17 (2d10 + 6) danni perforanti.

\textit{\textbf{Presenza Spaventosa.}} Ogni creatura scelta dal drago, che si trovi entro 36 metri da esso e consapevole della sua presenza, deve riuscire un Tiro Salvezza di Volontà DC 16 o restare spaventata per 1 minuto. Una creatura può ripetere il Tiro Salvezza al termine di ciascun suo round, terminando l'effetto se lo riesce. Se il Tiro Salvezza della creatura ha successo o l'effetto ha termine per essa, la creatura è immune alla Presenza Spaventosa del drago per le successive 24 ore.

\textit{\textbf{Arma a Soffio (Ricarica 5-6).}} Il drago usa una delle seguenti armi a soffio:

\textit{Soffio Acido.} Il drago esala acido in una linea lunga 18 metri e larga 1 metro. Ogni creatura sulla linea deve effettuare un Tiro Salvezza su Riflessi DC 18, subendo 54 (12d8) danni da acido se fallisce il Tiro Salvezza, o la metà di questi danni se lo riesce.

\textit{Soffio Rallentante.} Il drago esala del gas in un cono di 18 metri. Ogni creatura in quell'area deve riuscire un Tiro Salvezza su Tempra DC 18. Se fallisce il Tiro Salvezza, la creatura non può usare la sua reazione, ha la velocità dimezzata, e non può effettuare più di un attacco durante il suo round. Inoltre, la creatura può usare un'azione o un'azione bonus, ma non entrambe. Questi effetti permangono 1 minuto. La creatura può ripetere il Tiro Salvezza al termine di ciascun suo round, terminando l'effetto su di sé in caso di successo.

\textbf{Azioni Aggiuntive}

Il drago può effettuare 3 Azioni aggiuntive, scelte tra le opzioni seguenti. Può usare solo un'opzione leggendaria alla volta e solo al termine del turno di un'altra creatura. Il drago recupera le Azioni aggiuntive spese all'inizio del proprio round.

\textbf{Attacco di Ala (Costa 2 Azioni).} Il drago batte le ali. Ogni creatura entro 3 metri dal drago deve riuscire un Tiro Salvezza su Riflessi DC 19 o subire 13 (2d6 + 6) danni da botta e venir gettato prono. Il drago può poi volare fino a metà del suo movimento di volo.

\textbf{Attacco di Coda.} Il drago effettua un attacco di coda.

\textbf{Individuare.} Il drago effettua una prova di Saggezza (Consapevolezza).

\textbf{Ecologia}\\
Ambiente: Colline Calde\\
Organizzazione: Solitario\\
\textbf{Tesoro}: Triplo\\
\textbf{Descrizione}\\
Questo drago capriccioso durante il combattimento cerca di ostacolare e frustrare i suoi nemici.\\
\textbf{Incantesimi}\index{Incantesimi da Drago di Rame}\\
Gli incantesimi preferiti di questo Drago sono:\\
- Metamorfosi\\
- Confusione\\
- Nube maleodorante


\medskip\index[Mostruario]{Drago di Rame Giovane}\textbf{Drago di Rame Giovane}

\textit{Grande drago, caotico buono}

\textbf{FORZA} +4

\textbf{DESTREZZA} +1

\textbf{COSTITUZIONE} +3

\textbf{INTELLIGENZA} +3

\textbf{SAGGEZZA} +1

\textbf{CARISMA} +2

\textbf{Iniziativa} +3 -- \textbf{Difesa} 21

\textbf{Punti Ferita} 119 (14d10 + 42)

\textbf{Movimento} 12 m, scalata 12 m, volo 24 m

\textbf{Tiri Salvezza} Tempra +9, Riflessi +8, Volontà +8

\textbf{Competenze} Muoversi Silenziosamente / Nascondersi +4, Ingannare +5, Consapevolezza +7

\textbf{Immunità al Danno} acido

\textbf{Sensi} scurovisione 36 m, vista cieca 9 m

\textbf{Linguaggi} Comune, Draconico

\textbf{Sfida} 7 (2.900 PX)

\textbf{Azioni}

\textit{\textbf{Multiattacco.}} Il drago può effettuare tre attacchi: uno con il morso e due con gli artigli.

\textit{\textbf{Artiglio.} Attacco con arma da mischia}: +10 a colpire, portata 1 m, un bersaglio.

\textit{Colpisce:} 11 (2d6 + 4) danni taglienti, 1 danno da Sanguinamento.

\textit{\textbf{Morso.} Attacco con arma da mischia}: +10 a colpire, portata 3 m, un bersaglio.

\textit{Colpisce:} 15 (2d10 + 4) danni perforanti.

\textit{\textbf{Arma a Soffio (Ricarica 5-6).}} Il drago usa una delle seguenti armi a soffio:

\textit{Soffio Acido.} Il drago esala acido in una linea lunga 12 metri e larga 1 metro. Ogni creatura sulla linea deve effettuare un Tiro Salvezza su Riflessi DC 14, subendo 40 (9d8) danni da acido se fallisce il Tiro Salvezza, o la metà di questi danni se lo riesce.

\textit{Soffio Rallentante.} Il drago esala del gas in un cono di 9 metri. Ogni creatura in quell'area deve riuscire un Tiro Salvezza su Tempra DC 14. Se fallisce il Tiro Salvezza, la creatura non può usare la sua reazione, ha la velocità dimezzata, e non può effettuare più di un attacco durante il suo round. Inoltre, la creatura può usare un'azione o un'azione bonus, ma non entrambe. Questi effetti permangono 1 minuto. La creatura può ripetere il Tiro Salvezza al termine di ciascun suo round, terminando l'effetto su di sé in caso di successo.

\textbf{Ecologia}\\
Ambiente: Colline Calde\\
Organizzazione: Solitario\\
\textbf{Tesoro}: Triplo\\
\textbf{Descrizione}\\
Questo drago capriccioso durante il combattimento cerca di ostacolare e frustrare i suoi nemici.\\
\textbf{Incantesimi}\index{Incantesimi da Drago di Rame}\\
Gli incantesimi preferiti di questo Drago sono:\\
- Metamorfosi\\
- Confusione\\
- Nube maleodorante

\medskip\textbf{Drago di Rame Cucciolo}

\textit{Media drago, caotico buono}

\textbf{FORZA} +2

\textbf{DESTREZZA} +1

\textbf{COSTITUZIONE} +1

\textbf{INTELLIGENZA} +2

\textbf{SAGGEZZA} +0

\textbf{CARISMA} +1

\textbf{Iniziativa} +2 -- \textbf{Difesa} 17

\textbf{Punti Ferita} 22 (4d8 + 4)

\textbf{Movimento} 9 m, scalata 9 m, volo 18 m

\textbf{Tiri Salvezza} Tempra +2, Riflessi +2, Volontà +0

\textbf{Competenze} Muoversi Silenziosamente / Nascondersi +3, Consapevolezza +4

\textbf{Immunità al Danno} acido

\textbf{Sensi} scurovisione 18 m, vista cieca 3 m

\textbf{Linguaggi} Draconico

\textbf{Sfida} 1 (200 PX)

\textbf{Azioni}

\textit{\textbf{Morso.} Attacco con arma da mischia}: +4 a colpire, portata 1 m, un bersaglio.

\textit{Colpisce:} 7 (1d10 + 2) danni perforanti.

\textit{\textbf{Arma a Soffio (Ricarica 5-6).}} Il drago usa una delle seguenti armi a soffio:

\textit{Soffio Acido.} Il drago esala acido in una linea lunga 6 metri e larga 1 metro. Ogni creatura sulla linea deve effettuare un Tiro Salvezza su Riflessi DC 11, subendo 18 (4d8) danni da acido se fallisce il Tiro Salvezza, o la metà di questi danni se lo riesce.

\textit{Soffio Rallentante.} Il drago esala del gas in un cono di 5 metri. Ogni creatura in quell'area deve riuscire un Tiro Salvezza su Tempra DC 11. Se fallisce il Tiro Salvezza, la creatura non può usare la sua reazione, ha la velocità dimezzata, e non può effettuare più di un attacco durante il suo round. Inoltre, la creatura può usare un'azione o un'azione bonus, ma non entrambe. Questi effetti permangono 1 minuto. La creatura può ripetere il Tiro Salvezza al termine di ciascun suo round, terminando l'effetto su di sé in caso di successo.

\textbf{Ecologia}\\
Ambiente: Colline Calde\\
Organizzazione: Solitario\\
\textbf{Tesoro}: Triplo\\
\textbf{Descrizione}\\
Questo drago capriccioso durante il combattimento cerca di ostacolare e frustrare i suoi nemici.

\medskip\index[Mostruario]{Tàhil}\textbf{Tàhil}

\textit{mastodontica drago}

\textbf{FORZA} +10

\textbf{DESTREZZA} +0

\textbf{COSTITUZIONE} +10

\textbf{INTELLIGENZA} +8

\textbf{SAGGEZZA} +8

\textbf{CARISMA} +9

\textbf{Iniziativa} +8 -- \textbf{Difesa} 40

\textbf{Punti Ferita} 615 (30x3d6+300)

\textbf{Movimento} 20 metri, volare 20 metri

\textbf{Tiri Salvezza}: Tempra +40, Riflessi +30, Volontà +38

\textbf{Competenze} tutte +18

\textbf{Immunità al Danno} freddo, fulmine, fuoco, acido, veleno, suono, armi +3

\textbf{Immunità alle Condizioni} affascinato, avvelenato, paralizzato, affaticamento, spaventato

\textbf{Sensi} Scurovisione 60 m, Visione del vero 40 m

\textbf{Linguaggi} Comune, Draconico

\textbf{Sfida} 30 (155,000 PX)

\textbf{Immortale su Yeru.} Quando il corpo di Tàhil viene ucciso sul Yeru questo si riforma in 3d6 giorni nella tana fatta da Calicante.

\textit{\textbf{Incantesimi.}} Tàhil ha CM 20. La sua caratteristica da incantatore è il Carisma, +9 a colpire con attacchi da incantesimo. Tàhil conosce i seguenti incantesimi:

A volontà: Parola Divina

\textit{\textbf{Natura Divina.}} Tàhil non ha bisogno di aria, cibo, bevande o sonno. Gli incantesimi di 5 livello o inferiore non hanno effetto su Tàhil tranne se lo vuole.

\textit{\textbf{Padrone dei Draghi.}} Ogni Drago non metallico su Yeru è fedele ed ubbidiente al volere di Tàhil.

\textit{\textbf{Voce del Padrone.}} Tàhil può dialogare con ogni drago cromatico presente in Yeru, indipendentemente dalla distanza.

\textit{\textbf{Richiamo del Padrone.}} Tàhil apre un portale ed escono 1d2+1 dragi metallici di età e colore casuale. Il potere è usabile 1 volta al giorno.

\textit{\textbf{Resistenza Leggendaria (5/Giorno).}} Se il Tàhil fallisce un Tiro Salvezza, può scegliere invece di riuscirvi.

\textit{\textbf{Più teste.}} Tàhil ha +1d6 ai Tiri Salvezza contro essere cieco, sordo, svenuto. Tàhil può eseguire fino a 6 Reazioni per round.

\textit{\textbf{Rigenerazione.}} Tàhil rigenera 30 Punti Ferita all'inizio del suo round

\textbf{Azioni}

\textit{\textbf{Multiattacco.}} Tàhil può usare la sua Presenza Spaventosa oppure effettuare 3 attacchi (2 con artigli ed uno con la coda) oppure uno solo con il morso. Artiglio +30, portata 5 metri. Coda +30 portata 8 metri. Morso +30, portata 6 metri. Tutti gli attacchi di Tàhil sono considerati magici +5.

\textit{Colpisce:} Artiglio, 24 (4d6 +10, 5 danni da sanguinamento, fino ad un massimo di 40) da taglio. Coda, 28 (4d8 +10) da botta. Morso 48 (8d6 +10) in caso di critico nel colpire del morso mozza il corpo a metà della creatura se non si riesce un TS su Tempra a DC 30.

\textit{\textbf{Presenza Spaventosa}} Ogni creatura che possa vedere Tàhil e sia entro 80 metri deve fare un Tiro Salvezza su Volontà a DC 26 o essere Spaventato per 1 minuto. Ogni round la creatura può effettuare il Tiro Salvezza, se questo riesce è immune alla Presenza Spaventosa di Tàhil per le successive 24 ore.

\textbf{Azioni Aggiuntive}

Il Tàhil può effettuare 3 azioni aggiuntive, scelte da quelle sottostanti ed una per round solo al termine del round di un altra creatura. Le azioni dipendono dalla testa scelta.

\textbf{Attacco con Artiglio.}: +19, portata 6 metri, un obiettivo. Se colpisce 32 (4d10 + 10, 3 da Sanguinamento) danno da taglio più 14 (4d6) danni da acido (testa Nera) oppure Elettricità (testa Blu) oppure da Veleno (testa Verde) oppure da Fuoco (testa Rossa) oppure da Freddo (testa Bianca) oppure da Fuoco (testa Gialla) oppure da Suono (testa Viola)

\textbf{Testa Nera.}: Costa 2 azioni leggendarie, Tàhil soffia Acido in un cono di 40 metri. Tiro Salvezza su Riflessi DC 27 o prendere 68 (15d8) di danno da acido oppure dimezzare.

\textbf{Testa Blu.}: Costa 2 azioni leggendarie, Tàhil soffia Elettricità in un cono di 40 metri. Tiro Salvezza su Riflessi DC 27 o prendere 88 (16d10) di danno da Elettricità oppure dimezzare.

\textbf{Testa Verde.}: Costa 2 azioni leggendarie, Tàhil soffia Veleno in un cono di 30 metri. Tiro Salvezza su Riflessi DC 27 o prendere 77 (22d6) di danno da Veleno oppure dimezzare.

\textbf{Testa Rossa.}: Costa 2 azioni leggendarie, Tàhil soffia Fuoco in un cono di 30 metri. Tiro Salvezza su Riflessi DC 27 o prendere 91 (26d6) di danno da Fuoco oppure dimezzare.

\textbf{Testa Bianca.}: Costa 2 azioni leggendarie, Tàhil soffia Ghiaccio in un cono di 30 metri. Tiro Salvezza su Riflessi DC 27 o prendere 72 (16d8) di danno da Ghiaccio oppure dimezzare.

\textbf{Testa Viola.}: Costa 2 azioni leggendarie, Tàhil soffia Suono in un cono di 30 metri. Tiro Salvezza su Riflessi DC 27 o prendere 90 (18d8) di danno da Suono oppure dimezzare.

\textbf{Testa Gialla.}: Costa 2 azioni leggendarie, Tàhil soffia sabbia rovente in un cono di 60 metri. Tiro Salvezza su Riflessi DC 27 o prendere 72 (16d8) di danno da Fuoco oppure dimezzare.


\textbf{Ecologia}\\
Ambiente: Sconosciuto\\
Organizzazione: Unico\\
\textbf{Tesoro}: Speciale\\

\textbf{Descrizione}
Tàhil è il Patrono dei Draghi incarnato. Nulla resiste alla sua furia, follia, rabbia e distruzione. Tàhil è una mastodontica creatura con 7 teste di drago, ognuna colorata in modo diverso, ognuna a rappresentare un colore di un Drago. Vedi capitolo sulla Cosmologia per i dettagli della sua storia.


\medskip\index[Mostruario]{Drider}\textbf{Drider}

\textit{Grande mostruosità, caotico malvagio}

\textbf{FORZA} +3

\textbf{DESTREZZA} +3

\textbf{COSTITUZIONE} +4

\textbf{INTELLIGENZA} +1

\textbf{SAGGEZZA} +2

\textbf{CARISMA} +1

\textbf{Iniziativa} +3 -- \textbf{Difesa} 22

\textbf{Punti Ferita} 123 (13d10 + 52)

\textbf{Movimento} 9 m, scalata 9 m

\textbf{Tiri Salvezza} Tempra +7, Riflessi +5, Volontà +9

\textbf{Competenze} Muoversi Silenziosamente / Nascondersi +9, Consapevolezza +5

\textbf{Sensi} scurovisione 36 m

\textbf{Linguaggi} Elfico, Linguaggio delle Profondità

\textbf{Sfida} 6 (2.300 PX)

\textit{\textbf{Camminare sulla Tela.}} Il drider ignora le restrizioni al movimento provocate dalle ragnatele.

\textit{\textbf{Discendenza Fatata.}} Il drider ha +1d6 ai Tiri Salvezza per non restare affascinato, e la magia non può far addormentare un drider.

\textit{\textbf{Incantesimi Innati.}} La caratteristica da incantatore innato del drider è la Saggezza. Il drider può lanciare in maniera innata i seguenti incantesimi, senza bisogno di componenti materiali:

A volontà: \textit{luci danzanti}

1/Giorno: \textit{luminescenza, oscurità}

\textit{\textbf{Scalare come Ragno.}} Il drider può scalare superfici difficili, compreso lo stare a testa in giù sul soffitto, senza bisogno di effettuare una prova di abilità.

\textbf{Azioni}

\textit{\textbf{Multiattacco.}} Il drider effettua tre attacchi con la spada lunga o con l'arco lungo. Può rimpiazzare uno di questi attacchi con un attacco di morso.



\textit{\textbf{Morso.} Attacco con arma da mischia}: +11 a colpire, portata 1 m, una creatura.

\textit{Colpisce:} 2 (1d4) danni perforanti più 9 (2d8) danni da veleno.

\textit{\textbf{Spada Lunga.} Attacco con arma da mischia}: +1 a colpire, portata 1 m, un bersaglio.

\textit{Colpisce:} 7 (1d8 + 3) danni taglienti, o 8 (1d8 + 3) danni taglienti se usata con due mani.

\textit{\textbf{Arco Lungo.} Attacco con arma a Distanza}: +11 a colpire, gittata 45m, un bersaglio.

\textit{Colpisce:} 7 (1d8 + 3) danni perforanti più 4 (1d8) danni da veleno.

\textbf{Ecologia}\\
Ambiente: Qualsiasi sotterraneo\\
Organizzazione: Solitario, coppia o gruppo (3-8)\\
\textbf{Tesoro}: doppio (Mazza Pesante Perfetta, Arco Lungo Composito Perfetto [Forza +2] con 20 Frecce, altro tesoro)\\
\textbf{Descrizione}\\
Creato dal corpo di un elfo, alterato e mutato attraverso speciali veleni ed elisir per assumere le caratteristiche di un ragno gigante, il drider è una creatura pericolosa.\\
I drider sono sessualmente dimorfici. La parte inferiore da ragno del corpo di un drider femmina è lucente ed aggraziata, spesso simile al corpo di una vedova nera, mentre il busto superiore di elfo mantiene le sue curve allettanti e il bel viso (con l'eccezione delle venefiche zanne acuminate). La parte inferiore del corpo di un drider maschio è tozza come una tarantola, mentre quella superiore ha un fisico asciutto e supporta un'orrenda faccia più da ragno che da elfo, completa di mandibole zannute.


\medskip\index[Mostruario]{Driade}\textbf{Driade}

\textit{Media fatato, neutrale}

\textbf{FORZA} +0

\textbf{DESTREZZA} +1

\textbf{COSTITUZIONE} +0

\textbf{INTELLIGENZA} +2

\textbf{SAGGEZZA} +2

\textbf{CARISMA} +4

\textbf{Iniziativa} +2 -- \textbf{Difesa} 12 (17 con \textit{pelle di corteccia})

\textbf{Punti Ferita} 22 (5d8)

\textbf{Vulnerabilità al Danno} ferro freddo

\textbf{Movimento} 9 m

\textbf{Tiri Salvezza} Tempra +5, Riflessi +9, Volontà +7

\textbf{Competenze} Muoversi Silenziosamente / Nascondersi +5, Consapevolezza +4

\textbf{Sensi} scurovisione 18 m

\textbf{Linguaggi} Elfico, Silvano

\textbf{Sfida} 1 (200 PX)

\textit{\textbf{Camminata Arborea.}} Uno volta durante il suo round, la driade può usare 3 metri di movimento per entrare magicamente in un albero vivo a sua portata ed emergere da un altro albero vivo entro 18 metri dal primo albero, ricomparendo in uno spazio non occupato entro 1 metro dal secondo albero. Entrambi gli alberi devono essere di taglia Grande o superiore.

\textit{\textbf{Incantesimi Innati.}} La caratteristica da incantatore innato della driade è il Carisma (DC 14 per i Tiri Salvezza degli incantesimi). La driade può lanciare in maniera innata i seguenti incantesimi, senza aver bisogno di componenti materiali. A volontà:

\textit{arte del druido}

3/giorno ciascuno: \textit{bacche benefiche}, \textit{intralciare} 1/giorno:
\textit{passare senza tracce, pelle coriacea, randello} \textit{incantato}

\textit{\textbf{Parlare con Animali e Piante.}} La driade può comunicare con bestie e piante come se parlassero la stessa lingua.

\textit{\textbf{Resistenza alla Magia.}} La driade ha +1d6 ai Tiri Salvezza contro incantesimi e altri effetti magici.

\textbf{Azioni}

\textit{\textbf{Randello.} Attacco con arma da mischia}: +2 a colpire (+6 a colpire con \textit{bastone}), portata 1 m, un bersaglio.

\textit{Colpisce:} 2 (1d4) danni da botta, o 8 (1d8 + 4) danni da botta con \textit{bastone}

\textit{\textbf{Fascino Fatato.}} La driade può prendere a bersaglio un umanoide o bestia entro 9 metri da lei e che possa vedere. Se il bersaglio può vedere la driade, deve riuscire un Tiro Salvezza su Volontà DC 14 o restare affascinato dalla magia. Le creature affascinate considerano la driade un'amica fidata da ascoltare e proteggere. Sebbene il bersaglio non sia sotto il controllo della driade, interpreterà le richieste o le azioni della driade nel modo più favorevole possibile.

Ogni volta che la driade o i suoi alleati arrecano danno al bersaglio, esso può ripetere il Tiro Salvezza, terminando l'effetto in caso di successo. Altrimenti, l'effetto permane 24 ore o finché la driade muore, si trova su di un piano di esistenza diverso rispetto al bersaglio, o termina l'effetto con un'azione bonus. Se il Tiro Salvezza del bersaglio riesce, il bersaglio sarà immune al Fascino Fatato della driade per le successive 24 ore.

La driade non può tenere affascinati più di un umanoide o tre bestie alla volta.

\textbf{Ecologia}\\
Ambiente: Foreste Temperate\\
Organizzazione: Solitario, coppia o boschetto (3-8)\\
\textbf{Tesoro}: standard (Arco Lungo Perfetto con 20 Frecce, Pugnale, altro tesoro)\\
\textbf{Descrizione}\\
Le driadi sono folletti-albero che amano i boschi appartati lontani dagli umanoidi bisognosi di legname. L'interesse principale delle driadi è la propria sopravvivenza e quella delle adorate foreste, e sono note per costringere magicamente i viaggiatori ad aiutarle in quei compiti che non possono espletare. Sono amichevoli con druidi e guardiaboschi non malvagi, dato che riconoscono la loro empatia o il loro rispetto per la natura.\\
Le driadi sono benevole guardiane degli alberi, e sebbene non siano violente di natura, possono bloccare e sventare le minacce alle loro dimore o trasformare i nemici in alleati. Alcune tengono uno o più umanoidi ammaliati nel proprio territorio per difenderlo o per sviare gli assalitori. I nemici resi inabili in genere vengono trascinati ai confini della foresta dagli alleati delle driadi e scacciati, ma quelli malvagi o ostili vengono uccisi una volta finito il combattimento.

\medskip\index[Mostruario]{Duergar}\textbf{Duergar}

\textit{Media umanoide (nano), legale malvagio}

\textbf{FORZA} +2

\textbf{DESTREZZA} +0

\textbf{COSTITUZIONE} +2

\textbf{INTELLIGENZA} +0

\textbf{SAGGEZZA} +0

\textbf{CARISMA} -1

\textbf{Iniziativa} +2 -- \textbf{Difesa} 17 (armatura di scaglie, scudo)

\textbf{Punti Ferita} 26 (4d8 + 8)

\textbf{Movimento} 8 m

\textbf{Tiri Salvezza} Tempra +4, Riflessi +0, Volontà +1

\textbf{Resistenza al Danno} veleno

\textbf{Sensi} scurovisione 36 m

\textbf{Linguaggi} Nanico, Linguaggio delle Profondità

\textbf{Sfida} 1 (200 PX)

\textit{\textbf{Resilienza Duerga.}} Il duergar ha +1d6 ai Tiri Salvezza contro veleni, incantesimi e illusioni, oltre al resistere al restare affascinato o paralizzato.

\textit{\textbf{Sensibilità alla Luce}}. Mentre è alla luce del sole, il duergar ha -1d6 ai tiri di attacco, oltre che alle prove di Saggezza (Consapevolezza) basate sulla vista.

\textbf{Azioni}

\textit{\textbf{Ingrandire (Ricarica dopo un 1 ora).}} Per 1 minuto, il duergar aumenta magicamente di taglia, insieme a tutto ciò che sta trasportando o indossando. Mentre è ingrandito, il duergar è di taglia Grande, raddoppia i dadi di danno degli attacchi con armi basate sulla Forza (già incluso negli attacchi), e ha +1d6 alle prove di Forza e ai Tiri Salvezza di Forza. Se il duergar non ha sufficiente spazio per diventare Grande, ottiene la massima taglia concessa dallo spazio a disposizione.

\textit{\textbf{Piccone da Guerra.} Attacco con arma da mischia}: +4 a colpire, portata 1 m, un bersaglio.

\textit{Colpisce:} 6 (1d8 + 2) danni perforanti, o 11 (2d8 + 2) danni perforanti quando ingrandito.

\textit{\textbf{Giavellotto.} Attacco con arma da mischia o a Distanza}: +4 a colpire, portata 1 m o gittata 12m, un bersaglio. \textit{Colpisce:} 5 (1d6 + 2) danni perforanti o 9 (2d6 + 2) danni
perforanti quando ingrandito.

\textit{\textbf{Invisibilità (Ricarica dopo un 1 ora).}} Il duergar diventa magicamente invisibile al massimo per un'ora (come se stesse mantenendo la concentrazione per un incantesimo) o finché non attacca, lancia un incantesimo, usa Ingrandire o la sua concentrazione viene spezzata. Tutto l'equipaggiamento che il duergar indossa o trasporta diventa invisibile assieme a lui.

\textbf{Ecologia}\\
Ambiente: Qualsiasi sotterraneo\\
Organizzazione: solitario, gruppo (2-5), squadra (6-12 più 3 sergenti di 3° livello e 1 capo di 3°-8° livello), o clan (13-80 più 25\% di bambini non combattenti più 1 sergente di 3° livello ogni 5 adulti, 3-6 tenenti di 3°-6° livello, e 1-4 capitani di 9° livello)\\
\textbf{Tesoro}: equipaggiamento da PNG (Cotta di Maglia, Scudo Pesante di Metallo, Martello da Guerra, Balestra Leggera con 20 Quadrelli, 3d6 mo, altro tesoro)\\
\textbf{Descrizione}\\
Lontani parenti dei Nani, più cupi e deformi, i Duergar sono creature dal pessimo carattere che odiano gli intrusi nei loro reami sotterranei, ma mai più dei Nani. Vivono in comunità nelle profondità del sottosuolo. Hanno pelle grigio opaco, come fosse sporca di polvere o cenere, ma questa tonalità naturale permette di mimetizzarsi meglio nelle zone sotterranee. Sono una Razza di schiavisti, ma mentre costringono i prigionieri non Nani a lavori massacranti, uccidono senza remore i Nani catturati. In combattimento, i Duergar tirano di balestra, e poi passano al Martello da Guerra qualche round dopo. Se in inferiorità numerica, o se c'è un pericolo (e spazio) adeguato, un Duergar userà la sua capacità Ingrandire ed attaccherà.

\medskip\index[Mostruario]{Elementale dell'Acqua Maggiore}\textbf{Elementale dell'Acqua Maggiore}

\textit{Enorme elementale, neutrale}

\textbf{FORZA} +6

\textbf{DESTREZZA} +3

\textbf{COSTITUZIONE} +5

\textbf{INTELLIGENZA} -2

\textbf{SAGGEZZA} +1

\textbf{CARISMA} +0

\textbf{Iniziativa} +4 -- \textbf{Difesa} 21

\textbf{Punti Ferita} 158

\textbf{Movimento} 9 m, nuoto 33 m

\textbf{Tiri Salvezza} Tempra +12, Riflessi +12, Volontà +6

\textbf{Resistenze al Danno} acido; da arma non magica

\textbf{Immunità al Danno} veleno

\textbf{Immunità alle Condizioni} afferrato, avvelenato, intralciato, paralizzato, pietrificato, privo di sensi, prono, affaticamento

\textbf{Sensi} scurovisione 18 m

\textbf{Linguaggi} Aquan

\textbf{Sfida} 9

\textit{\textbf{Congelamento.}} Se l'elementale subisce danno da freddo, gela parzialmente; il suo movimento è ridotto di 6 metri fino al termine del suo prossimo round.

\textit{\textbf{Forma d'Acqua.}} L'elementale può entrare nello spazio di una creatura ostile e fermarsi lì. Può muoversi attraverso uno spazio stretto fino a 3 centimetri senza doversi stringere.

\textit{\textbf{Natura Elementale.}} Un elementale non ha bisogno di aria,cibo, bevande o sonno.

\textbf{Azioni}

\textit{\textbf{Multiattacco.}} L'elementale effettua due attacchi di schianto.

\textit{\textbf{Schianto.} Attacco con arma da mischia}: +19 a colpire, portata 3 m, un bersaglio.

\textit{Colpisce:} 26 (4d8 + 10) danni da botta.

\textit{\textbf{Sommergere (Ricarica 4-6).}} Ogni creatura nello spazio dell'elementale deve effettuare un Tiro Salvezza di Tempra DC 21. Se lo fallisce, il bersaglio subisce 25 (5d8 + 5) danni da botta. Se è di taglia Enorme o inferiore, il bersaglio è anche afferrato (DC 19 per fuggire). Fino al termine dell'afferrare, il bersaglio è intralciato e non può respirare a meno che non sia in grado di respirare acqua. Se il Tiro Salvezza riesce, il bersaglio viene spinto fuori dallo spazio dell'elementale.

L'elementale può afferrare una creatura Enorme o fino a due Grandi o più piccole alla volta. All'inizio di ciascun turno dell'elementale, ogni bersaglio afferrato subisce 25 (4d8 + 5) danni da botta. Una creatura entro 3 metro dall'elementale può trascinare fuori da esso una creatura o oggetto, impiegando un'azione per tentare di riuscire una prova di Forza DC 19.

\textbf{Ecologia}\\
Ambiente: Qualsiasi (Piano dell'Acqua)\\
Organizzazione: Solitario, coppia o gruppo (3-8)\\
\textbf{Tesoro}: Nessuno\\
\textbf{Descrizione}\\
Gli elementali dell'acqua sono creature pazienti e inflessibili composte di acqua vivente, dolce o salata. Preferiscono coprire d'acqua i loro avversari o trascinarveli dentro per ottenere un vantaggio.\\
Come gli altri elementali, tutti gli elementali dell'acqua hanno aspetto e forma unici. Molti sono creature dall'aspetto simile ad un'ondata con faccia vagamente umanoide e onde più piccole ai lati che fungono da braccia. Un'altra forma comune è quella di una qualche creatura acquatica, come uno squalo o un polpo, ma fatta interamente d'acqua.\\
Un elementale dell'acqua grande è alto 9 metri e pesa 10000kg.

\medskip\index[Mostruario]{Elementale dell'Acqua}\textbf{Elementale dell'Acqua}

\textit{Grande elementale, neutrale}

\textbf{FORZA} +4

\textbf{DESTREZZA} +2

\textbf{COSTITUZIONE} +4

\textbf{INTELLIGENZA} -3

\textbf{SAGGEZZA} +0

\textbf{CARISMA} -1

\textbf{Iniziativa} +4 -- \textbf{Difesa} 17

\textbf{Punti Ferita} 114 (12d10 + 48)

\textbf{Movimento} 9 m, nuoto 27 m

\textbf{Tiri Salvezza} Tempra +9, Riflessi +8, Volontà +2

\textbf{Resistenze al Danno} acido; da arma non magica

\textbf{Immunità al Danno} veleno

\textbf{Immunità alle Condizioni} afferrato, avvelenato, intralciato, paralizzato, pietrificato, privo di sensi, prono, affaticamento

\textbf{Sensi} scurovisione 18 m

\textbf{Linguaggi} Aquan

\textbf{Sfida} 5 (1.800 PX)

\textit{\textbf{Congelamento.}} Se l'elementale subisce danno da freddo, gela parzialmente; il suo movimento è ridotto di 6 metri fino al termine del suo prossimo round.

\textit{\textbf{Forma d'Acqua.}} L'elementale può entrare nello spazio di una creatura ostile e fermarsi lì. Può muoversi attraverso uno spazio stretto fino a 3 centimetri senza doversi stringere.

\textit{\textbf{Natura Elementale.}} Un elementale non ha bisogno di aria,
cibo, bevande o sonno.

\textbf{Azioni}

\textit{\textbf{Multiattacco.}} L'elementale effettua due attacchi di schianto.

\textit{\textbf{Schianto.} Attacco con arma da mischia}: +11 a colpire, portata 1 m, un bersaglio.

\textit{Colpisce:} 13 (2d8 + 4) danni da botta.

\textit{\textbf{Sommergere (Ricarica 4-6).}} Ogni creatura nello spazio dell'elementale deve effettuare un Tiro Salvezza di Tempra DC 15. Se lo fallisce, il bersaglio subisce 13 (2d8 + 4) danni da botta. Se è di taglia Grande o inferiore, il bersaglio è anche afferrato (DC 14 per fuggire). Fino al termine dell'afferrare, il bersaglio è intralciato e non può respirare a meno che non sia in grado di respirare acqua. Se il Tiro Salvezza riesce, il bersaglio viene spinto fuori dallo spazio
dell'elementale.

L'elementale può afferrare una creatura Grande o fino a due Medie o più piccole alla volta. All'inizio di ciascun turno dell'elementale, ogni bersaglio afferrato subisce 13 (2d8 + 4) danni da botta. Una creatura entro 1 metro dall'elementale può trascinare fuori da esso una creatura o oggetto, impiegando un'azione per tentare di riuscire una prova di Forza DC 14.

\textbf{Ecologia}\\
Ambiente: Qualsiasi (Piano dell'Acqua)\\
Organizzazione: Solitario, coppia o gruppo (3-8)\\
\textbf{Tesoro}: Nessuno\\
\textbf{Descrizione}\\
Gli elementali dell'acqua sono creature pazienti e inflessibili composte di acqua vivente, dolce o salata. Preferiscono coprire d'acqua i loro avversari o trascinarveli dentro per ottenere un vantaggio.\\
Come gli altri elementali, tutti gli elementali dell'acqua hanno aspetto e forma unici. Molti sono creature dall'aspetto simile ad un'ondata con faccia vagamente umanoide e onde più piccole ai lati che fungono da braccia. Un'altra forma comune è quella di una qualche creatura acquatica, come uno squalo o un polpo, ma fatta interamente d'acqua.\\
Un elementale dell'acqua grande è alto 4,8 metri e pesa 1125kg.


\medskip\index[Mostruario]{Elementale dell'Acqua Minore}\textbf{Elementale dell'Acqua Minore}

\textit{Medio elementale, neutrale}

\textbf{FORZA} +2

\textbf{DESTREZZA} +1

\textbf{COSTITUZIONE} +2

\textbf{INTELLIGENZA} -3

\textbf{SAGGEZZA} +0

\textbf{CARISMA} -1

\textbf{Iniziativa} +4 -- \textbf{Difesa} 15

\textbf{Punti Ferita} 16 (2d8 + 4)

\textbf{Movimento} 9 m, nuoto 27 m

\textbf{Tiri Salvezza} Tempra +3, Riflessi +4, Volontà +0

\textbf{Resistenze al Danno} acido; da arma non magica

\textbf{Immunità al Danno} veleno

\textbf{Immunità alle Condizioni} afferrato, avvelenato, intralciato, paralizzato, pietrificato, privo di sensi, prono, affaticamento

\textbf{Sensi} scurovisione 18 m

\textbf{Linguaggi} Aquan

\textbf{Sfida} 2

\textit{\textbf{Congelamento.}} Se l'elementale subisce danno da freddo, gela parzialmente; il suo movimento è ridotto di 6 metri fino al termine del suo prossimo round.

\textit{\textbf{Forma d'Acqua.}} L'elementale può entrare nello spazio di una creatura ostile e fermarsi lì. Può muoversi attraverso uno spazio stretto fino a 3 centimetri senza doversi stringere.

\textit{\textbf{Natura Elementale.}} Un elementale non ha bisogno di aria, cibo, bevande o sonno.

\textbf{Azioni}

\textit{\textbf{Multiattacco.}} L'elementale effettua due attacchi di schianto.

\textit{\textbf{Schianto.} Attacco con arma da mischia}: +5 a colpire, portata 1 m, un bersaglio.

\textit{Colpisce:} 6 (1d6 + 2) danni da botta.

\textit{\textbf{Sommergere (Ricarica 4-6).}} Ogni creatura nello spazio dell'elementale deve effettuare un Tiro Salvezza di Tempra DC 13. Se lo fallisce, il bersaglio subisce 8 (2d4 + 4) danni da botta. Se è di taglia Media o inferiore, il bersaglio è anche afferrato (DC 12 per fuggire). Fino al termine dell'afferrare, il bersaglio è intralciato e non può respirare a meno che non sia in grado di respirare acqua. Se il Tiro Salvezza riesce, il bersaglio viene spinto fuori dallo spazio dell'elementale.

L'elementale può afferrare una creatura Media o fino a due Piccole alla volta. All'inizio di ciascun turno dell'elementale, ogni bersaglio afferrato subisce 8 (2d4 + 4) danni da botta. Una creatura entro 1 metro dall'elementale può trascinare fuori da esso una creatura o oggetto, impiegando un'azione per tentare di riuscire una prova di Forza DC 12.

\textbf{Ecologia}\\
Ambiente: Qualsiasi (Piano dell'Acqua)\\
Organizzazione: Solitario, coppia o gruppo (3-8)\\
\textbf{Tesoro}: Nessuno\\
\textbf{Descrizione}\\
Gli elementali dell'acqua sono creature pazienti e inflessibili composte di acqua vivente, dolce o salata. Preferiscono coprire d'acqua i loro avversari o trascinarveli dentro per ottenere un vantaggio.\\
Come gli altri elementali, tutti gli elementali dell'acqua hanno aspetto e forma unici. Molti sono creature dall'aspetto simile ad un'ondata con faccia vagamente umanoide e onde più piccole ai lati che fungono da braccia. Un'altra forma comune è quella di una qualche creatura acquatica, come uno squalo o un polpo, ma fatta interamente d'acqua.\\
Un elementale dell'acqua grande è alto 1,8 metri e pesa 80kg.

\medskip\index[Mostruario]{Elementale dell'Aria}\textbf{Elementale dell'Aria}

\textit{Grande elementale, neutrale}

\textbf{FORZA} +2

\textbf{DESTREZZA} +5

\textbf{COSTITUZIONE} +2

\textbf{INTELLIGENZA} -2

\textbf{SAGGEZZA} +0

\textbf{CARISMA} -2

\textbf{Iniziativa} +5 -- \textbf{Difesa} 18

\textbf{Punti Ferita} 90 (12d10 + 24)

\textbf{Movimento} 0 m, volo 27 m (fluttua)

\textbf{Tiri Salvezza} Tempra +9, Riflessi +13, Volontà +2

\textbf{Resistenze al Danno} fulmine, suono; da arma non magica

\textbf{Immunità al Danno} veleno

\textbf{Immunità alle Condizioni} afferrato, avvelenato, intralciato, paralizzato, pietrificato, privo di sensi, prono, affaticamento

\textbf{Sensi} scurovisione 18 m

\textbf{Linguaggi} Auran

\textbf{Sfida} 5 (1.800 PX)

\textit{\textbf{Forma d'Aria.}} L'elementale può entrare nello spazio di una creatura ostile e fermarsi lì. Può muoversi attraverso uno spazio stretto fino a 3 centimetri senza doversi stringere.

\textit{\textbf{Natura Elementale.}} Un elementale non ha bisogno di aria, cibo, bevande o sonno.

\textbf{Azioni}

\textit{\textbf{Multiattacco.}} L'elementale effettua due attacchi di schianto.

\textit{\textbf{Schianto.} Attacco con arma da mischia}: +8 a colpire, portata 1 m, un bersaglio.

\textit{Colpisce:} 14 (2d8 + 5) danni da botta.

\textit{\textbf{Turbine (Ricarica 4-6).}} Ogni creatura nello spazio dell'elementale deve effettuare un Tiro Salvezza di Tempra DC 13. Se lo fallisce, il bersaglio subisce 15 (3d8 + 2) danni da botta e viene scagliato a 6 metri di distanza dall'elementale in una direzione casuale e cadere prono. Se un bersaglio lanciato colpisce un oggetto, come un muro o il pavimento, subisce 3 (1d6) danni da botta per ogni 3 metri per cui è stato lanciato. Se il bersaglio viene lanciato contro un'altra creatura, quella creatura deve riuscire un Tiro Salvezza di Riflessi DC 13 o subire lo stesso danno e cadere prona.

Se il Tiro Salvezza riesce, il bersaglio subisce la metà del danno da botta e non viene scagliato via né cade prono.

\textbf{Ecologia}\\
Ambiente: Piano dell'Aria\\
Organizzazione: Solitario, coppia o gruppo (3-8)\\
\textbf{Tesoro}: Nessuno\\
\textbf{Descrizione}\\
Gli elementali dell'aria sono rapide creature volanti fatte d'aria. Primitivi e territoriali, non amano essere evocati o controllati dai mortali, e preferiscono trascorrere il loro tempo sul Piano dell'Aria, volando attraverso il cielo infinito.\\
Sebbene tutti gli elementali dell'aria della stessa taglia abbiano le stesse statistiche, l'aspetto esatto di ognuno varia molto da individuo a individuo: uno può apparire come un vortice animato di vento e fumo, mentre un altro come una creatura di fumo simile ad un uccello con occhi scintillanti e ali di vento.\\
Un elementale dell'aria preferisce attaccare le creature che volano, non solo per i vantaggi che ha grazie alla sua padronanza dell'aria, ma anche perché detesta toccare il terreno. Un elementale dell'aria può muoversi sott'acqua, e anche se non corre alcun rischio di annegamento, non ha gradi in Nuotare e sott'acqua perde gran parte della sua mobilità e velocità.\\
Un elementale dell'Aria Grande è alto 4,8 m e pesa 2 kg.

\medskip\index[Mostruario]{Elementale del Fuoco}\textbf{Elementale del Fuoco}

\textit{Grande elementale, neutrale}

\textbf{FORZA} +0

\textbf{DESTREZZA} +3

\textbf{COSTITUZIONE} +3

\textbf{INTELLIGENZA} -2

\textbf{SAGGEZZA} +0

\textbf{CARISMA} -2

\textbf{Iniziativa} +3 -- \textbf{Difesa} 16

\textbf{Punti Ferita} 102 (12d10 + 36)

\textbf{Movimento} 15 m

\textbf{Tiri Salvezza} Tempra +8, Riflessi +11, Volontà +4

\textbf{Resistenze al Danno} da arma non magica

\textbf{Immunità al Danno} fuoco, veleno

\textbf{Immunità alle Condizioni} afferrato, avvelenato, intralciato, paralizzato, pietrificato, prono, privo di sensi, affaticamento

\textbf{Sensi} scurovisione 18 m

\textbf{Linguaggi} Ignan

\textbf{Sfida} 5 (1.800 PX)

\textit{\textbf{Forma di Fuoco.}} L'elementale può spostarsi attraverso uno spazio fino a 3 centimetri di larghezza senza stringersi. Una creatura che entri a contatto o colpisca l'elementale con un attacco da mischia mentre si trova entro 1 metro da esso subisce 5 (1d10) danni da fuoco. Inoltre, l'elementale può entrare nello spazio di una creatura ostile e fermarsi lì. La prima volta che entra nello spazio di una creatura in un turno, la creatura subisce 5 (1d10) danni da fuoco e prende fuoco; finché qualcuno non impiega un'azione per spegnere le fiamme, la creatura subirà 5 (1d10) danni da fuoco all'inizio di ciascun proprio round.

\textit{\textbf{Illuminazione.}} L'elementale emette luce intensa in un raggio di 9 metri e luce fioca per ulteriori 9 metri.

\textit{\textbf{Natura Elementale.}} Un elementale non ha bisogno di aria, cibo, bevande o sonno.

\textit{\textbf{Suscettibilità all'Acqua.}} L'elementale subisce 1 danno da freddo per ogni 1 metro che si muove in acqua o per ogni 4 litri d'acqua che gli vengono spruzzati addosso.

\textbf{Azioni}

\textit{\textbf{Multiattacco.}} L'elementale effettua due attacchi di contatto.

\textit{\textbf{Contatto.} Attacco con arma da mischia}: +7 a colpire, portata 1 m, un bersaglio.

\textit{Colpisce:} 10 (2d8 + 5) danni da fuoco. Se il bersaglio è una creatura o un oggetto infiammabile, prende fuoco. Finché una creatura non impiega un'azione per spegnere le fiamme, la creatura subirà 5 (1d10) danni da fuoco all'inizio di ciascun proprio round.

\textbf{Ecologia}
Ambiente: Qualsiasi (Piano del Fuoco)\\
Organizzazione: Solitario, coppia o gruppo (3-8)\\
\textbf{Tesoro}: Nessuno\\
\textbf{Descrizione}\\
Gli elementali del fuoco sono creature veloci e crudeli fatte di fiamme viventi. Si divertono a spaventare quelli più deboli di loro, e terrorizzano qualsiasi creatura che possano incendiare. Un elementale del fuoco non può entrare nel'acqua o in qualsiasi liquido ininfiammabile. Una massa d'acqua è una barriera impenetrabile a meno che l'elementale possa scavalcarla o saltarla, oppure venga coperta con materiale infiammabile (come uno strato d'olio).\\
Gli elementali del fuoco hanno un aspetto variabile; in genere si manifestano in forma di spire serpentine fatte di fumo e fiamme, ma alcuni elementali del fuoco prendono sembianze più simili a quelle di umani, demoni o altri mostri per aumentare il terrore quando compaiono improvvisamente. Il corpo di un elementale del fuoco sembra fatto di fiamme o sbuffi di scintille, fumo o cenere semistabili.\\

Un elementale del fuoco grande è alto 4,8 metri.

\medskip\index[Mostruario]{Elementale della Terra}\textbf{Elementale della Terra}

\textit{Grande elementale, neutrale}

\textbf{FORZA} +5

\textbf{DESTREZZA} -1

\textbf{COSTITUZIONE} +5

\textbf{INTELLIGENZA} -3

\textbf{SAGGEZZA} +0

\textbf{CARISMA} -3

\textbf{Iniziativa} -1 -- \textbf{Difesa} 20

\textbf{Punti Ferita} 126 (12d10 + 60)

\textbf{Movimento} 9 m, scavo 9 m

\textbf{Tiri Salvezza} Tempra +9, Riflessi +1, Volontà +6

\textbf{Vulnerabilità al Danno} suono

\textbf{Resistenze al Danno} da arma non magica

\textbf{Immunità al Danno} veleno

\textbf{Immunità alle Condizioni} avvelenato, paralizzato, pietrificato, prono, privo di sensi, affaticamento,

\textbf{Sensi} percezione tellurica 18 m, scurovisione 18 m

\textbf{Linguaggi} Terran

\textbf{Sfida} 5 (1.800 PX)

\textit{\textbf{Mostro d'Assedio.}} L'elementale infligge danni doppi agli oggetti e le strutture.

\textit{\textbf{Natura Elementale.}} Un elementale non ha bisogno di aria, cibo, bevande o sonno.

\textit{\textbf{Planata Terrestre.}} L'elementale può scavare attraversa la terra e la pietra non magiche e non lavorate. Quando lo fa, l'elementale non disturba il materiale che sposta.
\textbf{Azioni}

\textit{\textbf{Multiattacco.}} L'elementale effettua due attacchi di schianto.

\textit{\textbf{Schianto.} Attacco con arma da mischia}: +12 a colpire, portata 3 m, un bersaglio.

\textit{Colpisce:} 14 (2d8 + 5) danni da botta.

\textbf{Ecologia}
Ambiente: Qualsiasi (Piano della Terra)\\
Organizzazione: Solitario, coppia o gruppo (3-8)\\
\textbf{Tesoro}: Nessuno\\
\textbf{Descrizione}\\
Gli elementali della terra sono creature lente ed ostinate fatte di pietra o terra. Quando stanno completamente fermi sono indistinguibili da un mucchio di pietre o una piccola collina.\\

Quando un elementale della terra si mette pesantemente in movimento, il suo aspetto esteriore può variare, anche se le sue statistiche restano identiche a quelle dei suoi simili della stessa taglia. Gli elementali della terra sono fatti per lo più di roccia, terra o cristallo, con gemme scintillanti come occhi. Quelli più grandi hanno l'aspetto di umanoidi di pietra. Ciuffi di vegetazione spesso crescono sul suolo che costituisce parte del corpo di un elementale della terra.\\

Un elementale della terra grande è alto 4,8 metri e pesa 3000 kg.


\medskip\index[Mostruario]{Ettercap}\textbf{Ettercap}

\textit{Media mostruosità, neutrale malvagio}

\textbf{FORZA} +2

\textbf{DESTREZZA} +2

\textbf{COSTITUZIONE} +1

\textbf{INTELLIGENZA} -2

\textbf{SAGGEZZA} +1

\textbf{CARISMA} 8 (-2)

\textbf{Iniziativa} +2 -- \textbf{Difesa} 14

\textbf{Punti Ferita} 44 (8d8 + 8)

\textbf{Movimento} 9 m, scalata 9 m

\textbf{Tiri Salvezza} Tempra +6, Riflessi +4, Volontà +6

\textbf{Competenze} Muoversi Silenziosamente / Nascondersi +4, Consapevolezza +3, Sopravvivenza +3

\textbf{Sensi} scurovisione 18 m

\textbf{Linguaggi} -

\textbf{Sfida} 2 (450 PX)

\textit{\textbf{Camminare sulla Tela.}} L'ettercap ignora le restrizioni al movimento provocate dalle ragnatele.

\textit{\textbf{Scalare come Ragno.}} L'ettercap può scalare superfici difficili, compreso lo stare a testa in giù sul soffitto, senza bisogno di effettuare una prova di caratteristica.

\textit{\textbf{Senso della Tela.}} Mentre è in contatto con una ragnatela, l'ettercap sa l'esatta posizione di qualsiasi altra creatura in contatto con la stessa ragnatela.

\textbf{Azioni}

\textit{\textbf{Multiattacco.}} L'ettercap effettua due attacchi: uno con il morso e uno con gli artigli

\textit{\textbf{Artigli.} Attacco con arma da mischia}: +4 a colpire, portata 1 m, un bersaglio.

\textit{Colpisce:} 7 (2d4 + 2) danni taglienti, 1 danno da Sanguinamento.

\textit{\textbf{Morso.} Attacco con arma da mischia}: +4 a colpire, portata 1 m, un bersaglio.

\textit{Colpisce:} 6 (1d8 + 2) danni perforanti più 4 (1d8) danni da veleno. Il bersaglio deve riuscire un Tiro Salvezza di Tempra DC 11 o restare avvelenato per 1 minuto. La creatura può ripetere il Tiro Salvezza al termine di ciascun suo round, terminando l'effetto se riesce il Tiro Salvezza.

\textit{\textbf{Ragnatela (Ricarica 5-6).} Attacco con arma a Distanza}: +4 a colpire, gittata 9m, una creatura di taglia Grande o minore. \textit{Colpisce:} La creatura è intralciata dalla ragnatela. Con un'azione, la creatura intralciata può effettuare una prova di Forza DC 11, liberandosi dalla tela se la riesce. L'effetto termina se la tela è distrutta. La tela ha Difesa 10, 5 Punti Ferita, vulnerabilità ai danni da fuoco, e immunità ai danni da botta e da veleno.

\textbf{Ecologia}\\
Ambiente: Foreste Temperate\\
Organizzazione: solitario, coppia o nido (3-6 più 2-8 ragni giganti)\\
\textbf{Tesoro}: Standard\\
\textbf{Descrizione}\\
Gli ettercap sono alti di solito 1,8 metri e pesano circa 100 kg. Sono solitari e raramente si uniscono ad altri della loro razza, tranne per l'accoppiamento. Quando fanno gruppo, tendono ad attrarre varie specie di ragni, formando uno strano connubio di ettercap e aracnidi.\\
Gli ettercap sono noti per la costruzione di astute trappole fatte di ragnatele e altri materiali naturali, che usano per catturare prede. Costruiscono rifugi di ragnatela, tra i rami più alti gli alberi lontano dagli altri predatori terrestri, e usano ragni mostruosi come vedette e guardiani.\\
Gli ettercap non sono coraggiosi, ma le loro trappole spesso impediscono al nemico di estrarre le armi. Un ettercap attacca con artigli e morsi velenosi. In genere evita la mischia con gli avversari che possono ancora muoversi e fugge se si liberano.


\medskip\index[Mostruario]{Ettin}\textbf{Ettin}

\textit{Grande gigante, caotico malvagio}

\textbf{FORZA} +5

\textbf{DESTREZZA} -1

\textbf{COSTITUZIONE} +3

\textbf{INTELLIGENZA} -2

\textbf{SAGGEZZA} +0

\textbf{CARISMA} -1

\textbf{Iniziativa} -1 -- \textbf{Difesa} 14

\textbf{Punti Ferita} 85 (10d10 + 30)

\textbf{Movimento} 12 m

\textbf{Tiri Salvezza} Tempra +9, Riflessi +2, Volontà +5

\textbf{Competenze} Consapevolezza +4

\textbf{Linguaggi} Gigante, Goblinoide

\textbf{Sfida} 4 (1.100 PX)

\textit{\textbf{Due Teste.}} L'ettin ha +1d6 alle prove di Saggezza (Consapevolezza) e sui Tiri Salvezza contro le condizioni accecato, affascinato, assordato, privo di sensi, spaventato e stordito.

\textit{\textbf{Veglia.}} Quando una delle due teste dell'ettin è addormentata, l'altra è sveglia.

\textbf{Azioni}

\textit{\textbf{Multiattacco.}} L'ettin effettua due attacchi: uno con l'ascia da battaglia e uno con la mazza chiodata.

\textit{\textbf{Ascia da Battaglia.} Attacco con arma da mischia}: +11 a colpire, portata 1 m, un bersaglio.

\textit{Colpisce:} 14 (2d8 + 5) danni taglienti.

\textit{\textbf{Mazza Chiodata.} Attacco con arma da mischia}: +11 a colpire, portata 1 m, un bersaglio.

\textit{Colpisce:} 14 (2d8 + 5) danni perforanti.

\textbf{Ecologia}\\
Ambiente: Colline fredde\\
Organizzazione: Solitario, coppia, gruppo (3-6), truppa (1-2 più 1-2 Orsi Bruni, banda (3-6 più 1-2 Orsi Bruni) o colonia (3-6 più 1-2 Orsi Bruni e 7-12 Orchi, o 9-16 Goblin)\\
\textbf{Tesoro}: Standard (Armatura di Cuoio, 2 Mazzafrusti Leggeri, 4 Giavellotti, altro tesoro)\\
\textbf{Descrizione}\\
Gli ettin, o giganti a due teste, sono cacciatori notturni malevoli e imprevedibili. Le due teste gli concedono impareggiabili poteri di percezione, facendone dei guardiani eccellenti.\\
Gli ettin sembrano Giganti di Collina Giganti di Pietra, ma il volto zannuto tradisce una discendenza orchesca. Hanno pelle marrone rosata e non si lavano mai se non vi sono costretti, cosa che li rende così sporchi e sudici che la loro pelle sembra spessa e grigia.\\
Gli adulti sono alti 3,9 metri e pesano 2.600 kg. Gli ettin vivono circa 75 anni.\\
Gli ettin non hanno un loro linguaggio ma parlano un gergo misto di Gigante, Goblin e Orchesco. Le creature che parlano uno qualsiasi di questi linguaggi possono comunicare con un ettin effettuando una prova di Intelligenza con DC 15. La prova si effettua una volta per ogni frammento di informazione; se l'altra creatura parla due di questi linguaggi la DC è 10, mentre per qualcuno che li parla tutti e tre è 5.\\
Sebbene gli ettin non siano molto intelligenti, sono guerrieri astuti. Preferiscono tendere imboscate alle loro vittime anziché ingaggiarle in combattimento, ma una volta che la battaglia è cominciata, un ettin combatte furiosamente fino alla morte del nemico.\\
Gli ettin sono creature solitarie, si stabiliscono nella sicurezza di cave rocciose e cavità, spesso circondate da buche e fossi, e tengono a volte degli orsi delle caverne come animali da compagnia o guardiani.\\
Un ettin particolarmente potente può attrarre un gruppo di seguaci, specie Goblin o Orchi. Comunque, questi assembramenti sono più che altro delle eccezioni, e raramente durano a lungo, con gli individualisti ettin che vanno per la loro strada appena le opportunità di saccheggio e rapina diminuiscono o se il capo viene ucciso.\\
In genere formano delle coppie riproduttive per allevare la prole solo per brevi periodi prima di riprendere ognuno la propria strada. I giovani ettin maturano rapidamente, raggiungendo la taglia adulta in un anno, potendo così provvedere a se stessi.

\medskip\index[Mostruario]{Fantasma}\textbf{Fantasma}

\textit{Media non morto, qualsiasi tratto}

\textbf{FORZA} -2

\textbf{DESTREZZA} +1

\textbf{COSTITUZIONE} +0

\textbf{INTELLIGENZA} +0

\textbf{SAGGEZZA} +1

\textbf{CARISMA} +3

\textbf{Iniziativa} +1 -- \textbf{Difesa} 13

\textbf{Punti Ferita} 45 (10d8)

\textbf{Movimento} 0 m, volo 12 m (fluttua)

\textbf{Tiri Salvezza} Tempra +7, Riflessi +6, Volontà +7

\textbf{Resistenze al Danno} acido, fulmine, fuoco, suono; da botta, perforante, tagliente di attacchi non magici

\textbf{Immunità ai Danni} freddo, da Vuoto, veleno

\textbf{Immunità alle Condizioni} affascinato, afferrato, avvelenato, intralciato, paralizzato, pietrificato, prono, affaticamento, spaventato

\textbf{Sensi} scurovisione 18 m

\textbf{Linguaggi} qualsiasi lingua conosciuta in vita

\textbf{Sfida} 4 (1.100 PX)

\textit{\textbf{Movimento Incorporeo.}} Il fantasma può attraversare altre creature e oggetti come se fossero terreno difficile. Subisce 5 (1d10) danni da forza se termina il suo round all'interno di un oggetto.

\textit{\textbf{Natura Non Morta.}} Il fantasma non ha bisogno di aria, cibo, bevande o di dormire.

\textit{\textbf{Vista Eterea.}} Il fantasma può vedere 18 metri nel Piano Etereo quando si trova sul Piano Materiale, e vice versa.

\textbf{Azioni}

\textit{\textbf{Tocco Avvizzente.} Attacco con arma da mischia}: +6 a colpire, portata 1 m, un bersaglio.

\textit{Colpisce:} 17 (4d6 + 3) danni da Vuoto. Il bersaglio deve fare un Tiro Salvezza su Tempra a DC 15 o divenire Affaticato.

\textit{\textbf{Eterealità.}} Il fantasma entra nel Piano Etereo dal Piano Materiale, o vice versa. È visibile sul Piano Materiale mentre è nel Piano Etereo, e vice versa, ma non può interagire con nulla che si trovi sull'altro piano.

\textit{\textbf{Possessione (Ricarica 6).}} Un umanoide, entro 1 metro e visibile al fantasma, deve riuscire un Tiro Salvezza di Volontà DC 13 o venire posseduto dal fantasma; il fantasma poi scompare, e il bersaglio è inabile e perde il controllo del suo corpo. Il fantasma ora controlla il corpo ma non priva il bersaglio della sua consapevolezza. Il fantasma non può essere bersaglio di attacchi, incantesimi, o altri effetti, eccetto quelli che scacciano i non morti, e mantiene i suoi Tratti, Intelligenza, Saggezza, Carisma e immunità all'essere affascinato e spaventato. Per il resto usa altrimenti le statistiche del bersaglio posseduto, ma non accede al sapere e competenze del bersaglio.

La possessione dura finché il corpo scende a 0 Punti Ferita, il fantasma la termina con un'azione bonus, o il fantasma viene scacciato o espulso da un effetto come l'incantesimo \textit{dissolvi il bene e il male}. Quando la possessione termina, il fantasma riappare in uno spazio non occupato entro 1 metro dal corpo. Il bersaglio è immune alla Possessione di questo fantasma per 24 ore dopo aver riuscito il Tiro Salvezza o al termine della possessione.

\textit{\textbf{Viso Orripilante.}} Ogni creatura che non sia non morta, entro 18 metri dal fantasma e che lo possa vedere, deve riuscire un Tiro Salvezza di Volontà DC 13 o essere spaventata per 1 minuto. Se il Tiro Salvezza fallisce di 5 o più, il bersaglio invecchia anche di 1d4 x 10 anni. Un bersaglio spaventato può ripetere il Tiro Salvezza al termine di ciascun proprio round, terminando l'effetto per sé, qualora riuscisse il Tiro Salvezza. Se il Tiro Salvezza del bersaglio riesce e per lui l'effetto ha fine, il bersaglio è immune al Viso Orripilante del fantasma per le successive 24 ore. L'effetto di invecchiamento può essere invertito con l'incantesimo \textit{ristorare superiore}, ma solo se eseguito entro 24 dall'effetto di invecchiamento.

\textbf{Ecologia}
Ambiente: qualsiasi\\
Organizzazione: solitario\\
\textbf{Tesoro}: equipaggiamento da PNG\\
\textbf{Descrizione}\\
Quando ad un'anima non è concesso il riposo a causa di qualche grave ingiustizia, vera o presunta, a volte essa torna come fantasma. Questi esseri sono eternamente angosciati, privi di sostanza e incapaci di rimettere le cose a posto. Sebbene i fantasmi possano avere qualsiasi Tratto, molti si aggrappano al mondo dei viventi con un forte senso di odio e rabbia, e come risultato diventano malvagi; anche una creatura buona dopo morta può diventare un fantasma odioso e crudele.\\

Più di altri mostri, il fantasma deve avere un background ben delineato. Perché questo personaggio è diventato un fantasma? Quali leggende lo circondano? Un incontro con un fantasma non dovrebbe mai avvenire in modo accidentale: ci sono molti altri non morti incorporei, come Wraith e Spettri, per questo. Un incontro adeguato con un fantasma dovrebbe avvenire in una scena al culmine di un lungo periodo di tensione costruito con servitori minori o manifestazioni di spiriti non morti. L'esempio di fantasma sopra rappresenta una principessa umana assassinata da un amante infedele; dopo un confronto, lui la legò con delle catene e la gettò nel pozzo del castello, dove morì annegata. Le capacità del fantasma sono state selezionate in base al background, mostrando come si possa creare un potente antagonista. Applicando l'archetipo a creature con livelli e quindi Abilità proprie o con capacità razziali significative si possono creare fantasmi molto più potenti.\\

Quando viene creato un fantasma, questi ottiene le "copie" degli oggetti a cui in vita dava particolare valore (a condizione che gli originali non siano in possesso di altre creature). L'equipaggiamento funziona normalmente per il fantasma ma passa attraverso gli oggetti o le creature materiali. Un'arma +1 o con un potenziamento superiore, tuttavia, può danneggiare le creature materiali, ma tali attacchi infliggono la metà dei danni (50\%) a meno che non sia un'arma del tocco fantasma. Un fantasma può usare scudi e armature solo se hanno la capacità Tocco Fantasma.\\

Gli oggetti originali vengono lasciati indietro, proprio come le spoglie fisiche del fantasma. Se un'altra creatura impugna l'originale, la copia incorporea svanisce. Questa perdita fa inevitabilmente infuriare il fantasma, che non si ferma davanti a nulla per riportare l'oggetto nel posto in cui giaceva originariamente (e riguadagnarne l'utilizzo).


\medskip\index[Mostruario]{Fauci Gorgoglianti}\textbf{Fauci Gorgoglianti}

\textit{Media aberrazione, neutrale}

\textbf{FORZA} +0

\textbf{DESTREZZA} -1

\textbf{COSTITUZIONE} +3

\textbf{INTELLIGENZA} -4

\textbf{SAGGEZZA} +0

\textbf{CARISMA} -2

\textbf{Iniziativa} -1 -- \textbf{Difesa} 10

\textbf{Punti Ferita} 67 (9d8 + 27)

\textbf{Movimento} 3 m, nuoto 3 m

\textbf{Tiri Salvezza} Tempra +8, Riflessi +4, Volontà +5

\textbf{Immunità alle Condizioni} prono

\textbf{Sensi} scurovisione 18 m

\textbf{Linguaggi} -

\textbf{Sfida} 2 (450 PX)

\textit{\textbf{Gorgoglio.}} Finché la fauce è in grado di vedere una creatura e non è inabile, pronuncia frasi incoerenti. Ogni creatura che inizi il suo round entro 6 metri dalla fauce e può udire il suo gorgoglio deve effettuare un Tiro Salvezza di Volontà DC 10. Se lo fallisce, la creatura non può effettuare reazioni fino all'inizio del suo prossimo round e tira un d8 per determinare cosa farà durante il proprio round. Da 1 a 4, la creatura non fa nulla. Con 5 o 6, la creatura non svolge nessun'azione o azione bonus e usa tutto il suo movimento per muoversi in una direzione determinata casualmente. Con 7 o 8, la creatura effettua un attacco da mischia contro una creatura determinata a caso entro la sua portata o non fa nulla se non è in grado di effettuare un simile attacco.

\textit{\textbf{Terreno Aberrante.}} Il terreno in un raggio di 3 metri intorno alla fauce è considerato terreno difficile. Ogni creatura che inizi il suo round in quell'area deve riuscire un Tiro Salvezza di Tempra DC 10 o vedere il suo movimento ridotto a 0 fino all'inizio del suo round successivo.

\textbf{Azioni}

\textit{\textbf{Multiattacco.}} La fauce gorgogliante effettua un attacco di morso e, se può, uno Sputo Accecante.

\textit{\textbf{Morso.} Attacco con arma da mischia}: +3 a colpire, portata 1 m, una creatura.

\textit{Colpisce:} 17 (5d6) danni perforanti. Se il bersaglio è di taglia Media o inferiore, deve riuscire un Tiro Salvezza di Tempra DC 10 o venir gettato prono. Se il bersaglio viene ucciso da questo danno, viene assorbito dalla fauce.

\textit{\textbf{Sputo Accecante (Ricarica 5-6).}} La fauce sputa un globo chimico ad un punto visibile entro 5 metri da essa. Il globo esplode all'impatto in un lampo accecante di luce. Ogni creatura entro 1 metro dal lampo deve riuscire un Tiro Salvezza di Riflessi DC 13 o restare accecata fino al termine del prossimo round della fauce.

\textbf{Ecologia}\\
Ambiente: Qualsiasi Sotterraneo\\
Organizzazione: Solitario\\
\textbf{Tesoro}: Standard\\
\textbf{Descrizione}\\
Disgustosa, nauseante e affamata: queste sono le uniche parole che descrivono in modo appropriato la fauce gorgogliante. Bestie ripugnanti che si nascondono nelle grotte, nelle fogne e negli incubi, le fauci non hanno altro senso sociale, ecologico o religioso diverso dalla loro capacità di far impazzire coloro che le ascoltano. Alcuni studiosi credono che le fauci gorgoglianti siano una variante più piccola del molto più pericoloso shoggoth, mentre altri teorizzano che sia una punizione di qualche potente entità o divinità inflitta a coloro che l'hanno offesa.

\medskip\index[Mostruario]{Fenice}\textbf{Fenice}

\textit{Mastodontica celestiale, Coraggioso, Protettivo, Buono}

\textbf{FORZA} +8

\textbf{DESTREZZA} +6

\textbf{COSTITUZIONE} +5

\textbf{INTELLIGENZA} +5

\textbf{SAGGEZZA} +6

\textbf{CARISMA} +6

\textbf{Iniziativa} +11 -- \textbf{Difesa} 28

\textbf{Punti Ferita} 210 (20d10 + 100)

\textbf{Vulnerabilità al Danno} freddo magico

\textbf{Movimento} 9 m, volare 27 m (buono)

\textbf{Tiri Salvezza} Tempra +17, Riflessi +19, Volontà +14

\textbf{Immunità al Danno} fuoco, Luce, veleno, armi +1

\textbf{Immunità alle Condizioni} afferrato, avvelenato, intralciato, paralizzato, pietrificato, prono, privo di sensi, affaticamento, sanguinamento

\textbf{Rigenerazione} una Fenice rigenera 10 Punti Ferita all'inizio di ogni suo round

\textbf{Sensi} Scurovisione 18 m, Visione Crepuscolare 18 m

\textbf{Linguaggi} Auran, Celestiale, Comune, Ignan

\textbf{Sfida} 15 (13000 PX)

\textit{\textbf{Consapevolezza della Luce.}} La Fenice ha sempre attivi i seguenti incantesimi \textit{Individuazione del Magico, Individuazione delle Malattie e dei Veleni, Vedere Invisibilità}

\textit{\textbf{Incantesimi Innati.}} La caratteristica da incantatore della Fenice è il Carisma. La Fenice può lanciare in maniera innata i seguenti incantesimi, senza bisogno di componenti materiali:

A volontà: \textit{Cura Ferite Critiche, Dissolvi Magie, Fiamma Perenne, Rimuovi Maledizione, Metamorfosi (solo in umanoidi)}

3/giorno: \textit{Cura Ferite Critiche di Massa, Guarigione, Muro di Fuoco, Ristorare Superiore, Tempesta di Fuoco}

1 volta: \textit{Resurrezione} la Fenice sacrificando la sua vita in maniera definitiva può riportare in vita una creatura.

\textbf{Azioni}

\textit{\textbf{Multiattacco.}} La Fenice può attaccare con due artigli ed il morso

\textit{\textbf{Morso.} Attacco con arma da mischia}: +23 al a colpire, portata 6 m, una creatura.

\textit{Colpisce:} 19 danni perforanti (2d8+8 + 1d6 da Luce)

\textit{\textbf{Artiglio.} Attacco con arma da mischia}: +23 al a colpire, portata 6 m, una creatura.

\textit{Colpisce:} 17 danni da taglio (2d6+8 + 1d6 da Luce)

\textbf{Abilità speciali}

\textit{\textbf{Rinascita}}

Una Fenice uccisa si riduce ad un falò di 3 metri cubi dove giace al centro un uovo di fenice. Dopo 1d4+4 round questo uovo si schiude e diventa una Fenice perfettamente sana. L'unico modo per evitare la rinascita è togliere l'uovo dal falò (20d6 di danno da Luce) od usare un incantesimo di Disintegrazione sull'uovo.
Una Fenice può resuscitare in questo modo una volta all’anno, se muore prima che sia trascorso questo tempo, la morte è definitiva. Uccidere una Fenice scatena l'ira delle Allieve della Luce e dei cavalieri di Sumkjr.

\textit{\textbf{Ali di fiamma}}

La Fenice può trasformare le sue piume in fiamma come Azione di Reazione gratuita. Queste piume infliggono 1d6 danni da fuoco + 1d6 danni da Luce a tutte le creature entro 6 metri all’inizio del suo round.

\textbf{Ecologia}\\
Ambiente: Deserti e colline calde\\
Organizzazione: Solitario\\
\textbf{Tesoro}: Standard\\
\textbf{Descrizione}\\
La leggenda narra che le Fenici siano gli uccelli da compagnia di Ljust, sicuramente sono creature maestose e bellissime ed emanano una Luce simile a quella della Patrona della Genesi. Il movimento delle loro ali non produce rumore mentre la loro voce è canto. La fenice è un leggendario uccello di fuoco e luce che vive solitamente nei deserti. Sono creature molto intelligenti e sagge ed a volte usando la loro capacità di metamorfosi si recano nelle città dove aiutano chi combatte contro l'oscurità.

\subsection{Funghi}

\medskip\index[Mostruario]{Fungo Stridente}\textbf{Fungo Stridente}

\textit{Media pianta, disallineato}

\textbf{FORZA} -5

\textbf{DESTREZZA} -5

\textbf{COSTITUZIONE} +0

\textbf{INTELLIGENZA} -5

\textbf{SAGGEZZA} -4

\textbf{CARISMA} -5

\textbf{Iniziativa} -5 -- \textbf{Difesa} 6

\textbf{Punti Ferita} 13 (3d8)

\textbf{Movimento} 0 m

\textbf{Tiri Salvezza}: Tempra -3, Riflessi +3, Volontà -4

\textbf{Immunità alle Condizioni} accecato, assordato, spaventato

\textbf{Sensi} vista cieca 9 m (cieco oltre questo raggio)

\textbf{Linguaggi} -

\textbf{Sfida} 0 (10 PX)

\textit{\textbf{Falso Aspetto.}} Mentre il fungo stridente rimane immobile, è indistinguibile da un normale fungo.

\textbf{Azioni}

\textit{\textbf{Strillo.}} Quando una luce intensa o una creatura si trova entro 9 metri dal fungo stridente, esso emette un strillo udibile fino a 90 metri di distanza. Il fungo stridente continua a strillare finché la fonte del disturbo non si è portata fuori gittata e per altri 1d4 turni successivi, ovvero finché non si è sgonfiato il cappello.

\textbf{Ecologia}\\
Ambiente: Qualsiasi sotterraneo\\
Organizzazione: Solitario, coppia o macchia (3-12)\\
\textbf{Tesoro}: Accidentale\\
\textbf{Descrizione}\\
Un fungo stridente è alto circa 50 cm, dall'ampio cappello marrone. Una volta emesso l'urlo il cappello si sgonfia.

Si racconta di cuochi Duergar specializzati nel cuocere questi funghi in pietanze sopraffine. I più bravi riescono anche a non fare sgonfiare il cappello.

\medskip\index[Mostruario]{Fungo Violetto}\textbf{Fungo Violetto}

\textit{Media pianta, disallineato}

\textbf{FORZA} -4

\textbf{DESTREZZA} -5

\textbf{COSTITUZIONE} +0

\textbf{INTELLIGENZA} -5

\textbf{SAGGEZZA} -4

\textbf{CARISMA} -5

\textbf{Iniziativa} -5 -- \textbf{Difesa} 6

\textbf{Punti Ferita} 18 (4d8)

\textbf{Movimento} 2 m

\textbf{Tiri Salvezza}: Tempra -3, Riflessi -3, Volontà -3

\textbf{Immunità alle Condizioni} accecato, assordato, spaventato

\textbf{Sensi} vista cieca 9 m (cieco oltre questo raggio)

\textbf{Linguaggi} -

\textbf{Sfida} 1/4 (50 PX)

\textit{\textbf{Falso Aspetto.}} Mentre il fungo violetto rimane immobile, è indistinguibile da un normale fungo.

\textbf{Azioni}

\textit{\textbf{Multiattacco.}} Il fungo effettua 1d4 attacchi con Contatto Putrido.

\textit{\textbf{Contatto Putrido.} Attacco con arma da mischia}: +2 a colpire, portata 3 m, un bersaglio.

\textit{Colpisce:} 4 (1d8) danni da Vuoto.

\textbf{Ecologia}\\
Ambiente: Qualsiasi sotterraneo\\
Organizzazione: Solitario, coppia o macchia (3-12)\\
\textbf{Tesoro}: Accidentale\\
\textbf{Descrizione}\\
I funghi viola sono uno dei più noti e temuti pericoli delle caverne. Un viaggiatore può spesso notare i segni lasciati dal fungo viola su coloro che vivono o cacciano nei luoghi in cui questi funghi carnivori si appostano. Queste profonde e orribili cicatrici sembrano solchi scavati nella carne: i segni di un incontro ravvicinato con un fungo viola.

Un fungo viola si nutre della materia organica putrefatta, ma a differenza della maggioranza dei funghi non è un consumatore passivo. I viticci di un fungo viola possono colpire con inaspettata rapidità e sono ricoperti di un veleno virulento che causa la putrefazione delle carni con nauseante velocità. Questo potente veleno, se trascurato, può far marcire rapidamente un intero braccio o una gamba, lasciandosi dietro solo ossa che presto si corroderanno anch'esse.

Sebbene i funghi viola possano muoversi, lo fanno solo per attaccare o cacciare la preda. Un fungo viola con un flusso regolare di putredine di cui nutrirsi si accontenta di restare in un posto. Molti abitanti del sottosuolo, in particolare Trogloditi e Vegepigmei, sfruttano questo comportamento a loro vantaggio e posizionano molteplici funghi viola in giunzioni ed entrate chiave delle loro caverne come guardiani, assicurandosi di fornire loro cadaveri a sufficienza per evitare che si addentrino nel rifugio in cerca di cibo.

Alcune specie di Boleto Stridente hanno un aspetto piuttosto simile a quello dei funghi viola, sebbene manchino di ramificazioni tentacolari. Non è strano trovare boleti stridenti e funghi viola nello stesso groviglio, specialmente nelle aree dove altre creature coltivano questi funghi come guardiani.

Un fungo viola è alto 1,2 metri e pesa 25 kg.


\medskip\index[Mostruario]{Fuoco Fatuo}\textbf{Fuoco Fatuo}

\textit{Minuscola non morto, caotico malvagio}

\textbf{FORZA} -5

\textbf{DESTREZZA} +9

\textbf{COSTITUZIONE} +0

\textbf{INTELLIGENZA} +1

\textbf{SAGGEZZA} +2

\textbf{CARISMA} +0

\textbf{Iniziativa} +9 -- \textbf{Difesa} 20

\textbf{Punti Ferita} 22 (9d4)

\textbf{Movimento} 0 m, volo 15 m (fluttua)

\textbf{Tiri Salvezza}: Tempra +3, Riflessi +12, Volontà +9

\textbf{Immunità ai Danni} fulmine, veleno

\textbf{Resistenze al Danno} acido, freddo, fuoco, da Vuoto, suono; armi che non siano magiche

\textbf{Immunità alle Condizioni} afferrato, avvelenato, intralciato, paralizzato, privo di sensi, prono, affaticamento

\textbf{Sensi} scurovisione 36 m

\textbf{Linguaggi} le lingue che conosceva in vita

\textbf{Sfida} 2 (450 PX)

\textit{\textbf{Consumare Vita.}} Con un'azione bonus, il fuoco fatuo può prendere a bersaglio una creatura che può vedere entro 1 metro da esso e che abbia 0 Punti Ferita e sia ancora in vita. Il bersaglio deve riuscire un Tiro Salvezza di Tempra DC 10 contro questa magia o morire. Se il bersaglio muore, il fuoco fatuo recupera 10 (3d6) Punti Ferita.

\textit{\textbf{Effimero.}} Il fuoco fatuo non può indossare né trasportare nulla.

\textit{\textbf{Illuminazione Variabile.}} Il fuoco fatuo promana luce intensa in un raggio da 1 a 6 metri e luce fioca per un numero di metri aggiuntivi pari al raggio scelto. Il fuoco fatuo può modificare questo raggio con un'azione bonus.

\textit{\textbf{Movimento Incorporeo.}} Il fuoco fatuo può muoversi attraverso altre creature e oggetti come se fossero terreno difficile. Subisce 5 (1d10) danni da forza se termina il suo round all'interno di un oggetto.

\textit{\textbf{Natura Non Morta.}} Il fuoco fatuo non ha bisogno di aria, cibo o bevande.

\textbf{Azioni}

\textit{\textbf{Scossa.} Attacco con incantesimo in mischia}: +9 a colpire, portata 1 m, una creatura.

\textit{Colpisce:} 9 (2d8) danni da fulmine.

\textit{\textbf{Invisibilità.}} Il fuoco fatuo e la sua luce diventano magicamente invisibili finché non attacca o usa Consumare Vita, o finché la sua concentrazione non termina (come se si stesse concentrando su di un incantesimo).

\textbf{Ecologia}
Ambiente: Qualsiasi Palude\\
Organizzazione: Solitario, coppia o sequenza (3-4)\\
\textbf{Tesoro}: Accidentale\\
\textbf{Descrizione}\\
Ogni cacciatore e agricoltore che viva vicino ad un acquitrino o a una palude ha dato un nome a queste sfere di luce fioca: jack lanterna, candele dei defunti, fuochi che camminano, luci dei pini, luci fantasma, luci di giunco; ma tutti sanno che si tratta di pericolosi predatori e false guide nell'oscurità.

Malvagie creature che si nutrono delle forti emanazioni psichiche delle creature terrorizzate, i fuochi fatui traggono piacere nel mettere i viaggiatori creduloni in situazioni pericolose. Nelle terre selvagge, dove sono molto comuni, i fuochi fatui preferiscono tattiche semplici come posizionarsi su scogli o sabbie mobili dove possono essere scambiati facilmente per lanterne (specialmente se possono predisporre la trappola nei pressi di vere lanterne di segnalazione), così da attirare i viaggiatori verso il pericolo. In rare occasioni, i fuochi fatui in cerca di vita facile si spostano in una città e si stabiliscono vicino ai patiboli o seguono, invisibili, un'armata, così da nutrirsi della paura degli uomini morenti; perché la stragrande maggioranza scelga di rimanere nelle paludi, dove le vittime scarseggiano, rimane un mistero.

I fuochi fatui possono contare solo sulla loro scossa elettrica in situazioni pericolose, quindi preferiscono lasciare che altre creature o pericoli si occupino delle loro vittime mentre loro fluttuano nelle vicinanze e banchettano.

I fuochi fatui possono brillare di qualunque colore desiderino, ma sono più spesso gialli, bianchi, verdi o blu. Possono anche variare la loro luminosità per creare un disegno: molti fuochi fatui amano creare forme che somigliano vagamente a teschi nella loro luminescenza per aumentare il terrore nelle loro vittime. I loro veri corpi sono globi di materiale spugnoso traslucido appena visibili di circa 30 centimetri che pesano 1,5 kg e possono emettere luce su tutta la loro superficie. La luce dei fuochi fatui brilla approssimativamente come una torcia, e sebbene non sembrino utilizzare suoni per comunicare, sentono perfettamente e possono far vibrare i loro corpi così rapidamente da imitare il linguaggio.

Nonostante siano denigrati dalla maggioranza delle creature senzienti, i fuochi fatui sono in realtà alquanto intelligenti, sebbene ragionino in modo completamente alieno. A volte si organizzano in gruppi chiamati "sequenze"; la loro società e i loro scopi rimangono completamente sconosciuti, così come le loro origini, sebbene talvolta siano noti per stringere patti con chi offre loro una grande quantità di vittime adeguatamente terrorizzate.

I fuochi fatui non hanno età e sono di fatto immortali, a meno che non muoiano di morte violenta; i fuochi fatui più antichi possono essere ottimi depositari di conoscenze del passato, sebbene convincere una di queste crudeli creature a cooperare possa essere piuttosto complicato.


\medskip\index[Mostruario]{Fustigatore}\textbf{Fustigatore}

\textit{Grande mostruosità, neutrale malvagio}

\textbf{FORZA} +4

\textbf{DESTREZZA} -1

\textbf{COSTITUZIONE} +3

\textbf{INTELLIGENZA} -2

\textbf{SAGGEZZA} +3

\textbf{CARISMA} -2

\textbf{Iniziativa} -1 -- \textbf{Difesa} 23

\textbf{Punti Ferita} 93 (11d10 + 33)

\textbf{Movimento} 3 m, scalata 3 m

\textbf{Tiri Salvezza}: Tempra +13, Riflessi +5, Volontà +13

\textbf{Competenze} Muoversi Silenziosamente / Nascondersi +5, Consapevolezza +6

\textbf{Sensi} scurovisione 18 m

\textbf{Linguaggi} -

\textbf{Sfida} 5 (1.800 PX)

\textit{\textbf{Falso Aspetto.}} Quando il fustigatore rimane immobile, è indistinguibile da una normale formazione rocciosa, come una stalagmite.

\textit{\textbf{Scalare come Ragno.}} Il fustigatore può scalare superfici difficili, compreso lo stare a testa in giù sul soffitto, senza bisogno di effettuare una prova di abilità.

\textit{\textbf{Viticci Afferranti.}} Il fustigatore può avere fino a sei viticci alla volta. Ogni viticcio può essere attaccato (CA 20; 10 Punti Ferita; immunità ai danni da veleno). Distruggere un viticcio non infligge danni al fustigatore, che può produrre un viticcio di rimpiazzo nel suo prossimo round. Un viticcio può essere anche rotto se una creatura effettua un'azione e riesce una prova di Forza DC 15 contro di esso.

\textbf{Azioni}

\textit{\textbf{Multiattacco.}} Il fustigatore può effettuare quattro attacchi con i suoi viticci, usare avvolgere e effettuare un attacco con il morso.

\textit{\textbf{Morso.} Attacco con arma da mischia}: +7 a colpire, portata 1 m, un bersaglio.

\textit{Colpisce:} 22 (4d8 + 4) danni perforanti.

\textit{\textbf{Viticcio.} Attacco con arma da mischia}: +7 a colpire, portata 15 m, una creatura.

\textit{Colpisce:} Il bersaglio è afferrato (DC 15 per fuggire). Fino al termine dell'afferrare, il bersaglio è intralciato e ha -1d6 alle prove di Forza e ai Tiri Salvezza su Tempra, mentre il fustigatore non può usare lo stesso viticcio contro un altro bersaglio.

\textit{\textbf{Avvolgere.}} Il fustigatore trascina le creature afferrate da lui di 7 metri verso di lui.

\textbf{Ecologia}
Ambiente: Qualsiasi Sotterraneo\\
Organizzazione: Solitario, coppia o gruppo (3-6)\\
\textbf{Tesoro}: Standard\\
\textbf{Descrizione}\\
Il fustigatore è un cacciatore da agguato. Capace di modificare la colorazione e la forma del suo corpo, un fustigatore nascosto sembra una stalagmite di pietra o ghiaccio (o in luoghi dal soffitto basso, una colonna di pietra o ghiaccio). Nelle aree prive di questi tratti per nascondersi un fustigatore può comprimere il suo corpo fino a sembrare un masso. Le sferze che può estroflettere non sono di carne ma di uno spesso materiale semiliquido simile a cera parzialmente fusa ma con la resistenza di una catena di ferro e la capacità di intirizzire la carne e indebolire le forze. Il fustigatore può usare queste sferze con grande maestria e farle volare fino a 15 metri per rubare gli oggetti che attraggono la sua attenzione.

Nonostante la sua forma aliena e mostruosa, il fustigatore è uno degli abitanti più intelligenti del sottosuolo. Non formano vaste società (anche se spesso si trovano a vivere insieme ad altre creature del sottosuolo come i Divora Cervelli, con cui a volte si alleano), ma spesso si aggregano in piccoli gruppi. Particolarmente interessato alla filosofia della vita e della morte, e agli aspetti più sottili delle religioni più sinistre e crudeli del mondo, un fustigatore può parlare o discutere per ore con quelli che inizialmente aveva semplicemente cercato di mangiare. Alcune storie parlano di oratori e filosofi particolarmente dotati che sono stati tenuti per giorni o anche anni come animali domestici o compagni di conversazione da gruppi di fustigatori; alla fine, però, se non riescono a scappare, l'appetito dei fustigatori finisce per avere la meglio sulla loro curiosa intelligenza, specialmente nei casi in cui questi animali da compagnia superano costantemente l'arguzia e la pazienza dei loro guardiani.
Un fustigatore è alto 2,7 metri e pesa 1.100 kg.


\medskip\index[Mostruario]{Gablin}\textbf{Gablin}

\textit{Piccolo immondo (goblinoide), caotico malvagio}

\textbf{FORZA} +2

\textbf{DESTREZZA} +1

\textbf{COSTITUZIONE} +1

\textbf{INTELLIGENZA} -2

\textbf{SAGGEZZA} -1

\textbf{CARISMA} -2

\textbf{Iniziativa} +1 -- \textbf{Difesa} 14

\textbf{Punti Ferita} 6 (2d4 + 2)

\textbf{Movimento} 9 m

\textbf{Tiri Salvezza}: Tempra +4, Riflessi +2, Volontà +0

\textbf{Resistenza al Danno}: Vuoto

\textbf{Sensi} scurovisione 18 m

\textbf{Linguaggi} comprendono il Comune ma non lo parlano, Abissale

\textbf{Sfida} 1/2 (100 PX)

\textit{\textbf{Sensibilità alla Luce}}. Mentre è alla luce del sole, il gablin  ha -1d6 ai tiri per colpire, oltre che alle prove di Saggezza (Consapevolezza) basate sulla vista.

\textbf{Azioni}

\textit{\textbf{Spada Corta.} Attacco con arma da mischia}: +2 a colpire, portata 1 m, un bersaglio.

\textit{Colpisce:} 5 (1d6 + 2) danni taglienti.

\textit{\textbf{Morso.} Attacco con arma da mischia}: +3 a colpire, contatto, un bersaglio.

\textit{Colpisce:} 2 (1d1 + 1) danni perforanti.

\textbf{Ecologia}\\
Ambiente: Ovunque\\
Organizzazione: Gruppo (8-12), banda da guerra (10-24) o tribù (50+, 1 sergente di 3° livello per 20 adulti, 1 o 2 luogotenenti di 4° o 5° livello, 1 capo di 6°-8° livello, 6-12 lupi selvatici e 1-4 Ogre o 1-2 Campione Gablin)\\
\textbf{Tesoro}: Occasionale\\
\textbf{Descrizione}\\
I Gablin sono la feccia della feccia, si dice che un Gablin nasce ad ogni pensiero cattivo e sicuramente sono veramente tanti.
I Gablin sono piccoli umanoidi dalla pelle scura, con striature verdi generati inizialmente per volontà di Cattalm con l'unico scopo di portare distruzione, morte e sofferenza.
I Gablin si possono nascondere ovunque purché in prossimità di una fonte di cibo, solitamente prediligono le fogne oppure strutture abbandonate vicino ai villaggi.
Lo scopo unico di un Gablin è uccidere e perpetuare la specie. I Gablin sono tutti maschi e la loro natura immonda li rende capaci di impregnare qualsiasi donna umanoide.
Solitamente la gestazione dura solo 3 settimane durante le quali le donne vengono torturate per rafforzare gli 1d6+2 piccoli che porta in grembo. Il parto solitamente si conclude con i piccoli di Gablin che sventrano la madre e ne fanno il primo loro pasto.
Questo metodo di procreazione unita alla loro voracia famelica di sangue e carne ne fanno tra le creature più odiate e temute.
Anche se singolarmente non sono particolarmente temibili i Gablin si muovono sempre in gruppo e se questo supera le due dozzine allora c'è quasi sempre un Gablin Incantatore o addirittura un Campione Gablin a guidarli.


\medskip\index[Mostruario]{Campione Gablin}\textbf{Campione Gablin}

\textit{Grande immondo, caotico malvagio}

\textbf{FORZA} +4

\textbf{DESTREZZA} +2

\textbf{COSTITUZIONE} +3

\textbf{INTELLIGENZA} +1

\textbf{SAGGEZZA} +0

\textbf{CARISMA} -1

\textbf{Iniziativa} +2 -- \textbf{Difesa} 18

\textbf{Punti Ferita} 60 (7d10 + 25)

\textbf{Movimento} 12 m

\textbf{Tiri Salvezza}: Tempra +9, Riflessi +6, Volontà +3

\textbf{Resistenza al Danno}: Vuoto

\textbf{Sensi} scurovisione 18 m

\textbf{Linguaggi} Comune, Abissale

\textbf{Sfida} 3 (700 PX)

\textbf{Azioni}

\textit{\textbf{Randello Pesante.} Attacco con arma da mischia}: +7 a colpire, portata 2 m, un bersaglio.

\textit{Colpisce:} 11 (2d6 + 4) danni da botta.

\textit{\textbf{Evocare Gablin}}: 3 Azioni. Il Gablin spilla il suo sangue a terra e a questo sorgono 2d4 Gablin, perde 1 Punto Ferita

\textbf{Ecologia}\\
Ambiente: Qualsiasi\\
Organizzazione: a capo di un gruppo di Gablin\\
\textbf{Tesoro}: Standard (Armatura di Pelle, Randello pesante)\\
\textbf{Descrizione}\\
I Campioni Gablin vengono spontaneamente quando il numero di Gablin presente raggiunge le 20 unità. Enormemente più grossi, più forti ed intelligenti di un Gablin i Campioni sono i leader del gruppo, coloro che pianificano le battaglie e gli scontri.
Non hanno remore a mandare al massacro i Gablin o ad uccidere qualsiasi cosa che respiri. Pervasi dello spirito di Cattalm il loro scopo è sempre e solo distruggere ed uccidere.


\medskip\index[Mostruario]{Paladino Gablin}\textbf{Paladino Gablin}

\textit{Grande immondo, caotico malvagio}

\textbf{FORZA} +5

\textbf{DESTREZZA} +2

\textbf{COSTITUZIONE} +3

\textbf{INTELLIGENZA} +2

\textbf{SAGGEZZA} +3

\textbf{CARISMA} +3

\textbf{Iniziativa} +4 -- \textbf{Difesa} 21

\textbf{Punti Ferita} 105 (10d10 + 50)

\textbf{Movimento} 12 m

\textbf{Tiri Salvezza}: Tempra +12, Riflessi +11, Volontà +12

\textbf{Resistenza al Danno}: Vuoto

\textbf{Sensi} scurovisione 18 m

\textbf{Linguaggi} Comune, Abissale

\textbf{Sfida} 6 (2300 PX)

\textbf{Azioni}

\textit{\textbf{Multiattacco.}} Il Paladino Gablin attacca con 2 colpi di spada bastarda.

\textit{\textbf{Spada Bastarda.} Attacco con arma da mischia}: +13 a colpire, portata 2 m, un bersaglio.

\textit{Colpisce:} 10 (1d10 + 5) danni da botta, più 1d6 danno da Vuoto. Se la creatura colpita è un Seguace o Devoto di Gradh il danno aumenta di un ulteriore 1d6.

\textit{\textbf{Evocare Gablin}}: 3 Azioni. Il Gablin spilla il suo sangue a terra e a questo sorgono 3d4 Gablin.

\textbf{Aura immonda}: il Paladino Gablin emana un aura di 6 metri di raggio intorno a lui che conferisce +2 al Tiro per Colpire ed al Danno a tutti gli altri Gablin ed impone -2 al TC e TS alle altre creature non Devote o Seguaci di Cattalm.

\textbf{Ecologia}\\
Ambiente: Qualsiasi\\
Organizzazione: a capo di un armata di Gablin\\
\textbf{Tesoro}: Standard (Armatura da campo, Spada Bastarda)\\
\textbf{Descrizione}\\
I Paladini Gablin sono tra i più potenti gablin che si conoscano, i veri eletti di Cattalm. Evocati da più potenti seguaci di Cattalm possono da soli guidare centinaia di Gablin e grazia al loro acume preparare accurati piani e portare scompiglio e distruzione in intere regioni.

\medskip\index[Mostruario]{Gargoyle}\textbf{Gargoyle}

\textit{Media elementale, caotico malvagio}

\textbf{FORZA} +2

\textbf{DESTREZZA} +0

\textbf{COSTITUZIONE} +3

\textbf{INTELLIGENZA} -2

\textbf{SAGGEZZA} +0

\textbf{CARISMA} -2

\textbf{Iniziativa} +0 -- \textbf{Difesa} 16

\textbf{Punti Ferita} 52 (7d8 + 21)

\textbf{Movimento} 9 m, volo 18 m

\textbf{Tiri Salvezza}: Tempra +4, Riflessi +6, Volontà +4

\textbf{Resistenze al Danno} da arma non magica o che non siano di adamantio

\textbf{Immunità ai Danni} veleno

\textbf{Immunità alle Condizioni} avvelenato, pietrificato, affaticamento

\textbf{Sensi} scurovisione 18 m

\textbf{Linguaggi} Terran

\textbf{Sfida} 2 (450 PX)

\textit{\textbf{Falso Aspetto.}} Mentre la gargoyle rimane immobile, è indistinguibile da una statua inanimata.

\textit{\textbf{Natura Elementale.}} Una gargoyle non ha bisogno di aria, cibo, bevande o sonno.

\textbf{Azioni}

\textit{\textbf{Multiattacco.}} La gargoyle effettua due attacchi: uno con il morso e uno con gli artigli.

\textit{\textbf{Artigli.} Attacco con arma da mischia}: +5 a colpire, portata 1 m, un bersaglio.

\textit{Colpisce:} 5 (1d6 + 2) danni taglienti, 1 danno da Sanguinamento.

\textit{\textbf{Morso.} Attacco con arma da mischia}: +5 a colpire, portata 1 m, un bersaglio.

\textit{Colpisce:} 5 (1d6 + 2) danni perforanti.

\textbf{Ecologia}
Ambiente: Qualsiasi\\
Organizzazione: Solitario, coppia o stormo (3-12)\\
\textbf{Tesoro}: Standard\\
\textbf{Descrizione}\\
I gargoyle spesso sembrano essere statue alate di pietra, poiché possono rimanere immobili indefinitamente per poi sorprendere i nemici. I gargoyle tendono a comportamenti ossessivo-compulsivi, tanto diversi quanto abbondante è la loro specie. Libri, ninnoli rubati, armi e trofei raccolti dai nemici caduti sono solo alcuni esempi dei tipi di oggetti che un gargoyle può collezionare per decorare la sua tana e il suo territorio.

I gargoyle tendono ad avere uno stile di vita solitario, anche se a volte formano temibili stormi detti "ali" per protezione e divertimento. In certe condizioni, una tribù di gargoyle può persino allearsi con altre creature, ma anche la più stabile di queste alleanze può crollare per ragioni infime; i gargoyle sono solo traditori, meschini e vendicativi.

I gargoyle sono noti per abitare nel cuore delle città più grandi, accovacciati tra le decorazioni di pietra delle cattedrali e degli edifici dove si nascondono in bella vista di giorno piombando giù per nutrirsi di vagabondi, mendicanti e altri sfortunati la notte.

Più a lungo una tribù di gargoyle dimora in un'area di edifici o rovine, più i suoi membri cominciano ad assomigliare allo stile architettonico della zona. I cambiamenti subiti dall'aspetto di un gargoyle sono lenti e sottili, ma nel corso degli anni possono diventare radicali.

Un'insolita variante del gargoyle non abita tra edifici e rovine ma sotto le onde del mare. Queste creature sono note come kapoacinth; hanno le stesse statistiche base dei gargoyle normali, eccetto che hanno il sottotipo acquatico e le loro ali gli garantiscono una velocità di nuotare di 12 metri (ma sono inutili per volare). I kapoacinth abitano nelle regioni costiere poco profonde dove possono strisciare fuori dalla spuma per dare la caccia ai residenti della zona. È più probabile che formino stormi, poiché i kapoacinth preferiscono la vita di gruppo a quella solitaria.

\medskip\index[Mostruario]{G.E.C.}\textbf{G.E.C.}

\textit{larga aberrazione, caotico malvagio}

\textbf{FORZA} +6

\textbf{DESTREZZA} +1

\textbf{COSTITUZIONE} +5

\textbf{INTELLIGENZA} +1

\textbf{SAGGEZZA} +1

\textbf{CARISMA} +1

\textbf{Iniziativa} +2 -- \textbf{Difesa} 20 (chitina)

\textbf{Punti Ferita} 95 (12d8 + 50)

\textbf{Movimento} 9 m, scavare 9 m

\textbf{Tiri Salvezza}: Tempra +8, Riflessi +3, Volontà +6

\textbf{Resistenza} +4 ai Tiri Salvezza agli incantesimi della Lista Charme e Illusione

\textbf{Competenze} Consapevolezza +10

\textbf{Sensi} scurovisione 18 m, senso tellurico 18 m

\textbf{Linguaggi} -

\textbf{Sfida} 10 (5900 PX)

\textbf{Azioni}

\textit{\textbf{Multiattacco.}} Il G.E.C. può attaccare con due artigli oppure con il morso

\textbf{Artigli}: Attacco con arma naturale da mischia: +21 a colpire, portata 3 m, un bersaglio.

\textit{Colpisce:} 15 (3d6 + 5) danni taglienti, 1 danno da Sanguinamento.

\textbf{Morso}: Attacco con arma naturale da mischia: +21 al colpire, portata 3 m, un bersaglio

\textit{Colpisce:} 16 (3d8 + 5) danni taglienti, 1 danno da Sanguinamento.

\textit{\textbf{Sguardo.}} E' sufficiente guardare il G.E.C. per essere affetti da Confusione, come omonimo incantesimo. Per resistere è necessario effettuare un Tiro Salvezza su Volontà a DC 15. Ogni round è possibile ripetere il Tiro Salvezza per resistere all'effetto.

Combattere senza guardare il G.E.C. impone -1d6 al Tiro per Colpire.

\textbf{Ecologia}\\
Ambiente: Sotterraneo\\
Organizzazione: solitario, gruppo (2-4) \\
\textbf{Tesoro}: Accidentale\\
\textbf{Descrizione}\\
Il Grande Essere Chitinoso, o G.E.C, è un insetto dal vago aspetto umanoide di quasi 4 metri di altezza, possente e dotato di due chele fortissime e resistenti capaci di scavare e tranciare qualsiasi materiale. 4 occhi piccoli, centrali e multi faccettati emanano un fioca luminescenza cangiante che confondono le creature che incrociano il loro sguardo.

Probabilmente frutto di una qualche incantesimo di trasformazione andato a male i G.E.C. sono padroni del sottosuolo. Creature dotate di una reale intelligenza amano la carne di elfo e combattono in maniera tattica ed accorta.

\subsection{Geni}

\medskip\index[Mostruario]{Djinni}\textbf{Djinni}

\textit{Grande elementale, caotico buono}

\textbf{FORZA} +5

\textbf{DESTREZZA} +2

\textbf{COSTITUZIONE} +6

\textbf{INTELLIGENZA} +2

\textbf{SAGGEZZA} +3

\textbf{CARISMA} +5

\textbf{Iniziativa} +2 -- \textbf{Difesa} 23

\textbf{Punti Ferita} 161 (14d10 + 84)

\textbf{Movimento} 9 m, volo 27 m

\textbf{Tiri Salvezza} Tempra +4, Riflessi +9, Volontà +7

\textbf{Immunità al Danno} fulmine, suono

\textbf{Sensi} scurovisione 36 m

\textbf{Linguaggi} Auran

\textbf{Sfida} 11 (7.200 PX)

\textit{\textbf{Decesso Elementale.}} Se il djinni muore, il suo corpo si disintegra in una brezza calda, lasciando dietro di sé solo l'equipaggiamento che il djinni stava indossando o trasportando.

\textit{\textbf{Incantesimi Innati.}} La caratteristica da incantatore innato del djinni è il Carisma 17, +9 a colpire con attacchi da incantesimo). Può lanciare in maniera innata i seguenti incantesimi, senza bisogno di componenti materiali:

A volontà: \textit{individuazione del bene e del male, individuazione del magico, onda tonante}

3/giorno ciascuno: \textit{camminare nel vento, creare cibo e acqua} (può creare vino al posto dell'acqua), \textit{linguaggi}

1/giorno ciascuno: \textit{creazione}, \textit{evoca elementali} (solo elementale dell'aria), \textit{forma gassosa, immagine maggiore}, \textit{invisibilità,} \textit{spostamento planare}

\textbf{Azioni}

\textit{\textbf{Multiattacco.}} Il djinni effettua tre attacchi di
scimitarra.

\textit{\textbf{Scimitarra.} Attacco con arma da mischia}: +15 a colpire, portata 1 m, un bersaglio.

\textit{Colpisce:} 12 (2d6 + 5) danni taglienti più 3 (1d6) danni da fulmine o suono (a scelta del gin).

\textit{\textbf{Creare Turbine.}} Un cilindro d'aria turbinante di 1 metro di raggio e alto 9 metri si forma magicamente in un punto visibile al djinni entro 36 metri da esso. Il turbine resta finché il djinni mantiene la concentrazione (come se si stesse concentrando su di un incantesimo). Qualsiasi creatura salvo il djinni che entri nel turbine deve riuscire un Tiro Salvezza di Tempra DC 18 o restare intralciata da esso. Il djinni può muovere il turbine di massimo 18 metri con un'azione, e le creature intralciate dal turbine si muovono con esso. Il turbine termina se il djinni lo perde di vista.

Una creatura può usare la sua azione per liberare una creatura intralciata dal turbine, compresa se stessa, riuscendo una prova di Forza DC 18. Se la prova riesce, la creatura non è più intralciata e si sposta nello spazio più vicino all'esterno del turbine.

\textbf{Ecologia}
Ambiente: Qualsiasi (Piano dell'Aria)\\
Organizzazione: Solitario, coppia, compagnia (3-6) o banda (7-10)\\
\textbf{Tesoro}: Standard (Scimitarra Perfetta, altro tesoro)\\
\textbf{Descrizione}\\
I Djinn (singolare djinni) sono Geni provenienti dal Piano Elementale dell'Aria. Si dice che siano fatti di nuvole e abbiano la forza delle tempeste più potenti. Un Djinni è alto circa 3 metri e pesa circa 500 kg.

I Djinn disdegnano il Combattimento fisico, preferendo usare i loro poteri Magici e capacità aeree contro i nemici. Un Djinni sconfitto in Combattimento generalmente prende il volo e diventa un turbine per molestare chi lo insegue. Quando non ha altra scelta che combattere in mischia, la maggioranza dei Djinn preferisce impugnare Scimitarre a Due Mani Perfette.

Verso gli altri Geni, i Djinn vanno d'accordo con gli Janni e i Marid. Sono frequentemente in contrasto con gli Shaitan, e sono nemici giurati degli Efreeti, disprezzando questi Geni feroci più di qualsiasi altra delle Razze di Geni. Il conflitto tra gli Efreeti e i Djinn è così leggendario che molti incantatori tentano (con vari gradi di successo) di assicurarsi il servizio di un Djinni promettendogli aiuto nella causa contro gli odiati nemici.


\medskip\index[Mostruario]{Efreeti}\textbf{Efreeti}

\textit{Grande elementale, legale malvagio}

\textbf{FORZA} +6

\textbf{DESTREZZA} +1

\textbf{COSTITUZIONE} +7

\textbf{INTELLIGENZA} +3

\textbf{SAGGEZZA} +2

\textbf{CARISMA} +3

\textbf{Iniziativa} +3 -- \textbf{Difesa} 23

\textbf{Punti Ferita} 200 (16d10 + 112)

\textbf{Movimento} 12 m, volo 18 m

\textbf{Tiri Salvezza} Tempra +7, Riflessi +10, Volontà +9

\textbf{Immunità al Danno} fuoco

\textbf{Sensi} scurovisione 36 m

\textbf{Linguaggi} Ignan

\textbf{Sfida} 11 (7.200 PX)

\textit{\textbf{Decesso Elementale.}} Se l'efreeti muore, il suo corpo si disintegra in un lampo di fuoco e uno sbuffo di fumo, lasciando dietro di sé solo l'equipaggiamento che l'efreeti stava indossando o trasportando.

\textit{\textbf{Incantesimi Innati.}} La caratteristica da incantatore innato dell'efreeti è il Carisma, +7 a colpire con attacchi da incantesimo). Può lanciare in maniera innata i seguenti incantesimi, senza bisogno di componenti materiali:

A volontà: \textit{individuazione del magico}

3/giorno ciascuno: \textit{ingrandire/ridurre, linguaggi}

1/giorno ciascuno: \textit{evoca elementali} (solo elementale del fuoco), \textit{forma gassosa, immagine maggiore}, \textit{invisibilità, muro di fuoco, spostamento planare}

\textbf{Azioni}

\textit{\textbf{Multiattacco.}} L'efreeti effettua due attacchi di scimitarra o usa due volte Scagliare Fiamma.

\textit{\textbf{Scimitarra.} Attacco con arma da mischia}: +21 a colpire, portata 1 m, un bersaglio.

\textit{Colpisce:} 13 (2d6 + 6) danni taglienti più 7 (2d6) danni da fuoco.

\textit{\textbf{Scagliare Fiamma.} Attacco con arma a Distanza}: +16 a colpire, gittata 36 m, un bersaglio.

\textit{Colpisce:} 17 (5d6) danni da fuoco.

\textbf{Ecologia}
Ambiente: Qualsiasi (Piano del Fuoco)\\
Organizzazione: Solitario, coppia, compagnia (3-6) o banda (7-12)\\
\textbf{Tesoro}: Standard (Falcione Perfetto, altro tesoro)\\
\textbf{Descrizione}\\
Gli Efreet (singolare Efreeti) sono Geni provenienti dal Piano del Fuoco. Sono alti 3,6 metri e pesano circa 1000 kg.

Gli Efreet hanno pochi alleati tra gli altri Geni: odiano i Djinni, e li attaccano a vista, non sopportano i Marid, e vedono i Janni come deboli e fragili. Gli Efreet spesso cooperano bene con gli Shaitan, eppure anche queste alleanze sono temporanee.


\subsection{Ghoul}

\medskip\index[Mostruario]{Ghast}\textbf{Ghast}

\textit{Media non morto, caotico malvagio}

\textbf{FORZA} +3

\textbf{DESTREZZA} +3

\textbf{COSTITUZIONE} +0

\textbf{INTELLIGENZA} +0

\textbf{SAGGEZZA} +0

\textbf{CARISMA} -1

\textbf{Iniziativa} +3 -- \textbf{Difesa} 14

\textbf{Punti Ferita} 36 (8d8)

\textbf{Movimento} 9 m

\textbf{Tiri Salvezza}: Tempra +2, Riflessi +2, Volontà +5

\textbf{Resistenze al Danno} da Vuoto

\textbf{Immunità al Danno} veleno

\textbf{Immunità alle Condizioni} affascinato, avvelenato, affaticamento

\textbf{Sensi} scurovisione 18 m

\textbf{Linguaggi} Comune

\textbf{Sfida} 2 (450 PX)

\textit{\textbf{Fetore.}} Qualsiasi creatura che inizi il suo round entro 1 metro dal ghast deve riuscire un Tiro Salvezza di Tempra DC 12 o restare nauseata fino all'inizio del suo prossimo round. Se riesce il Tiro Salvezza, la creatura è immune al Fetore del ghast per le successive 24
ore.

\textit{\textbf{Ribellione allo Scacciare.}} Il ghast e tutti i ghoul entro 9 metri da esso hanno +1d6 ai Tiri Salvezza contro gli effetti che scacciano i non morti.

\textbf{Azioni}

\textit{\textbf{Artigli.} Attacco con arma da mischia}: +6 a colpire, portata 1 m, un bersaglio.

\textit{Colpisce:} 10 (2d6 + 3) danni taglienti. Se il bersaglio è una creatura, diversa da un non morto, deve riuscire un Tiro Salvezza su Tempra DC 12 o restare paralizzata per 1 minuto. Il bersaglio può ripetere il Tiro Salvezza al termine di ciascun suo round, terminando l'effetto se riesce il Tiro Salvezza.

\textit{\textbf{Morso.} Attacco con arma da mischia}: +6 a colpire, portata 1 m, una creatura.

\textit{Colpisce:} 12 (2d8 + 3) danni perforanti.

\textbf{Ecologia}\\
Ambiente: Qualsiasi terreno\\
Organizzazione: Solitario, gruppo (2-4) o branco (7-12)\\
\textbf{Tesoro}: Standard\\
\textbf{Descrizione}\\
I ghast sono Ghoul con un legame più profondo con il vuoto. La paralisi di un ghast ha effetto anche sugli Elfi. I ghast si aggirano in branchi o comandano gruppi di Ghoul comuni. Il fetore di morte e putrefazione che circonda queste creature è travolgente.


\medskip\index[Mostruario]{Ghoul}\textbf{Ghoul}

\textit{Media non morto, caotico malvagio}

\textbf{FORZA} +1

\textbf{DESTREZZA} +2

\textbf{COSTITUZIONE} +0

\textbf{INTELLIGENZA} -2

\textbf{SAGGEZZA} +0

\textbf{CARISMA} -2

\textbf{Iniziativa} +2 -- \textbf{Difesa} 13

\textbf{Punti Ferita} 22 (5d8)

\textbf{Movimento} 9 m

\textbf{Tiri Salvezza}: Tempra +1, Riflessi +2, Volontà +4

\textbf{Immunità al Danno} veleno

\textbf{Immunità alle Condizioni} affascinato, avvelenato, affaticamento

\textbf{Sensi} scurovisione 18 m

\textbf{Linguaggi} Comune

\textbf{Sfida} 1 (200 PX)

\textbf{Azioni}

\textit{\textbf{Artigli.} Attacco con arma da mischia}: +4 a colpire, portata 1 m, un bersaglio.

\textit{Colpisce:} 7 (2d4 + 2) danni taglienti, 1 danno da Sanguinamento. Se il bersaglio è una creatura, diversa da un elfo o un non morto, deve riuscire un Tiro Salvezza su Tempra DC 12 o restare paralizzata per 1 minuto. Il bersaglio può ripetere il Tiro Salvezza al termine di ciascun suo round, terminando l'effetto se riesce il Tiro Salvezza.

\textit{\textbf{Morso.} Attacco con arma da mischia}: +4 a colpire, portata 1 m, una creatura.

\textit{Colpisce:} 9 (2d6 + 2) danni perforanti.

\textbf{Ecologia}
Ambiente: Qualsiasi terreno\\
Organizzazione: Solitario, gruppo (2-4) o branco (7-12)\\
\textbf{Tesoro}: Standard\\
\textbf{Descrizione}\\
I ghoul sono non morti che frequentano i cimiteri e mangiano i cadaveri. Le leggende sostengono che i primi ghoul fossero umani cannibali che una fame innaturale ha riportato indietro dalla morte, oppure umani che in vita si nutrivano dei resti in decomposizione dei loro simili e che morirono (e poi rinacquero) a causa di un'orrenda malattia; la vera origine di questi non morti necrofagi è incerta.

I ghoul si appostano ai margini della civilizzazione (dentro o nei pressi dei cimiteri o nelle fogne cittadine) dove possono reperire ampie scorte del loro cibo preferito. Sebbene preferiscano i corpi in putrefazione e spesso seppelliscano le loro vittime per migliorarne il sapore, mangiano i morti freschi se hanno abbastanza fame.


Anche se molti ghoul di superficie vivono in modo primitivo, delle voci parlano di città di ghoul nelle profondità del sottosuolo comandate da sacerdoti che adorano antiche divinità crudeli o strani signori dei demoni della fame. Questi ghoul "civilizzati" non sono meno orribili nelle loro abitudini alimentari, e in effetti il loro concetto di tavola ben imbandita per banchetti è forse anche più orrendo dell'idea di un pasto fresco prelevato da una bara.

\medskip\index[Mostruario]{Ghoul, Nero}\textbf{Ghoul, Nero}

\textit{Media non morto, caotico malvagio}

\textbf{FORZA} +4

\textbf{DESTREZZA} +2

\textbf{COSTITUZIONE} +2

\textbf{INTELLIGENZA} +0

\textbf{SAGGEZZA} +1

\textbf{CARISMA} -2

\textbf{Iniziativa} +2 -- \textbf{Difesa} 19

\textbf{Punti Ferita} 105 (15d8+30)

\textbf{Movimento} 12 m

\textbf{Tiri Salvezza}: Tempra +11, Riflessi +11, Volontà +8

\textbf{Immunità al Danno} veleno, vuoto, danno critico, sanguinamento, armi non magiche o d'argento

\textbf{Immunità alle Condizioni} affascinato, avvelenato, affaticamento,

\textbf{Sensi} scurovisione 18 m

\textbf{Linguaggi} Comune

\textbf{Sfida} 6 (2300 PX)

\textbf{\textit{Aura nefasta}}: il Ghoul Nero emana costantemente un aura attorno a se che indebolisce le difese di chiunque tranne che di altri ghoul. Ogni due round di permanenza nell'aura di 12 metri di raggio attorno al Ghoul Nero si cumula un -1 a tutti i TS, quando ci si allontana dal Ghoul Nero si recupera 1 punto a round.

\textbf{Azioni}

\textit{\textbf{Artigli.} Attacco con arma da mischia}: +12 a colpire, portata 1 m, un bersaglio.

\textit{Colpisce:} 15 (2d10 + 4) danni taglienti, 2 danno da Sanguinamento. Se il bersaglio è una creatura, diversa da un elfo o un non morto, deve riuscire un Tiro Salvezza su Tempra DC 16 o restare paralizzata per 1 minuto. Il bersaglio può ripetere il Tiro Salvezza al termine di ciascun suo round, terminando l'effetto se riesce il Tiro Salvezza.

\textit{\textbf{Morso.} Attacco con arma da mischia}: +13 a colpire, portata 1 m, una creatura.

\textit{Colpisce:} 18 (3d8 + 6) danni perforanti.

\textbf{Ecologia}
Ambiente: Qualsiasi terreno\\
Organizzazione: Gruppo (4-8) o branco (14-24)\\
\textbf{Tesoro}: Standard\\
\textbf{Descrizione}\\
Il Ghoul Nero rappresenta una delle elite evolutive dei Ghoul. Solitamente a capo di un gruppo almeno un ghoul putrescente a circa 18 ghoul.

\medskip\index[Mostruario]{Ghoul, Madre}\textbf{Ghoul, Madre}

\textit{Media non morto, caotico malvagio}

\textbf{FORZA} +0

\textbf{DESTREZZA} +3

\textbf{COSTITUZIONE} +2

\textbf{INTELLIGENZA} +2

\textbf{SAGGEZZA} +1

\textbf{CARISMA} +2

\textbf{Iniziativa} +3 -- \textbf{Difesa} 21

\textbf{Punti Ferita} 90 (10d10+45)

\textbf{Movimento} 9 m

\textbf{Tiri Salvezza}: Tempra +9, Riflessi +11, Volontà +9

\textbf{Immunità al Danno} veleno, vuoto, danno critico, sanguinamento, armi +1

\textbf{Immunità alle Condizioni} affascinato, avvelenato, affaticamento

\textbf{Sensi} scurovisione 18 m

\textbf{Linguaggi} Comune

\textbf{Sfida} 5 (1800 PX)

\textbf{Azioni}

\textit{\textbf{Artigli.} Attacco con arma da mischia}: +5 a colpire, portata 1 m, un bersaglio.

\textit{Colpisce:} 12 (2d6 + 6) danni taglienti, 2 danno da Sanguinamento. Se il bersaglio è una creatura diverso da un non morto, deve riuscire un Tiro Salvezza su Tempra DC 15 o restare paralizzata per 1 minuto. Il bersaglio può ripetere il Tiro Salvezza al termine di ciascun suo round, terminando l'effetto se riesce il Tiro Salvezza. Se la creatura fallisce il TS allora è vittima della maledizione del Ghoul. Entro 1d3+1 giorni si trasformerà in un Ghoul. E' necessario un Scacciare Maledizioni con Successo Magico Critico entro la trasformazione.

\textit{\textbf{Morso.} Attacco con arma da mischia}: +6 a colpire, portata 1 m, una creatura.

\textit{Colpisce:} 8 (2d6 + 2) danni perforanti.

\textbf{Ecologia}
Ambiente: Qualsiasi terreno\\
Organizzazione: Clan (7-12+)\\
\textbf{Tesoro}: Standard\\
\textbf{Descrizione}\\
La Madre Ghoul è solitamente a capo di un clan di ghoul che può raggiungere anche diverse decine di membri. Rispettata e temuta è solitamente tra i ghoul evoluti più intelligenti e molto apprezzata per la sua capacità di poter trasformare in ghoul i viventi. La loro tattica prevede di ferire e non uccidere diverse persone così che tornata a casa e poi trasformati possano attaccare ed uccidere tutto il villaggio.


\medskip\index[Mostruario]{Ghoul, putrescente}\textbf{Ghoul, putrescente}

\textit{Grande non morto, caotico malvagio}

\textbf{FORZA} +1

\textbf{DESTREZZA} +2

\textbf{COSTITUZIONE} +3

\textbf{INTELLIGENZA} -1

\textbf{SAGGEZZA} +0

\textbf{CARISMA} -2

\textbf{Iniziativa} +2 -- \textbf{Difesa} 15

\textbf{Punti Ferita} 82 (12d10+12)

\textbf{Movimento} 6 m

\textbf{Tiri Salvezza}: Tempra +7, Riflessi +5, Volontà +4

\textbf{Immunità al Danno} veleno, sanguinamento, da critico, Vuoto, armi non magiche o d'argento

\textbf{Immunità alle Condizioni} affascinato, avvelenato, affaticamento

\textbf{Sensi} scurovisione 36 m

\textbf{Linguaggi} Comune

\textbf{Sfida} 4 (110 PX)

\textbf{\textit{Rigenerazione}}. Il Ghoul Putrescente rigenera 5 PF a round tranne se è in piena luce solare o ha subito danni da Luce nel round precedente. Se il Ghoul Putrescente è in un cimitero recupera 10 PF a round.

\textbf{Azioni}

\textit{\textbf{Artigli.} Attacco con arma da mischia}: +5 a colpire, portata 2 m, un bersaglio.

\textit{Colpisce:} 12 (2d10 + 2) danni taglienti, 1 danno da Sanguinamento. Se il bersaglio è una creatura, diversa da un non morto, deve riuscire un Tiro Salvezza su Tempra DC 14 o restare paralizzata per 1 minuto.

\textit{\textbf{Morso.} Attacco con arma da mischia}: +6 a colpire, portata 1 m, una creatura.

\textit{Colpisce:} 10 (2d8 + 2) danni perforanti.

\textit{\textbf{Aura di Sofferenza.}}: il Ghoul Putrescente emana un aura di 6 metri intorno a lui nel quale ogni attacco andato a segno causa automaticamente un danno critico. Attivare questa aura costa 1 Azione e dura fino all'inizio del round successivo.

\textbf{Ecologia}
Ambiente: Qualsiasi terreno\\
Organizzazione: Gruppo (4-8) o branco (10-18)\\
\textbf{Tesoro}: Standard\\
\textbf{Descrizione}\\
I Ghoul Putrescenti sono una delle tante l'evoluzione dei Ghoul. Il contatto continuo con l'energia negativa ed il nutrirsi per secoli di cadaveri di ogni genere lo hanno reso più grande, forte e capace di infliggere e fare infliggere le ferite più pericolose.


\subsection{Giganti}

\medskip\index[Mostruario]{Gigante di Collina}\textbf{Gigante di Collina}

\textit{Enorme gigante, caotico malvagio}

\textbf{FORZA} +5

\textbf{DESTREZZA} -1

\textbf{COSTITUZIONE} +4

\textbf{INTELLIGENZA} -3

\textbf{SAGGEZZA} -1

\textbf{CARISMA} -2

\textbf{Iniziativa} -1 -- \textbf{Difesa} 16

\textbf{Punti Ferita} 105 (10d12 + 40)

\textbf{Movimento} 12 m

\textbf{Tiri Salvezza}: Tempra +11, Riflessi +2, Volontà +3

\textbf{Competenze} Consapevolezza +2

\textbf{Linguaggi} Gigante

\textbf{Sfida} 5 (1.800 PX)

\textbf{Azioni}

\textit{\textbf{Multiattacco.}} Il gigante effettua due attacchi con il randello pesante.

\textit{\textbf{Randello Pesante.} Attacco con arma da mischia}: +12 a colpire, portata 3 m, un bersaglio.

\textit{Colpisce:} 18 (3d8 + 5) danni da botta.

\textit{\textbf{Sasso.} Attacco con arma a Distanza}: +6 a colpire, gittata 18m, un bersaglio.

\textit{Colpisce:} 21 (3d10 + 5) danni da botta.

\textbf{Ecologia}\\
Ambiente: Colline Temperate\\
Organizzazione: Solitario, gruppo (2-5), banda (6-8), gruppo di razzia (9-12 più 1d4 Lupi Crudeli) o tribù (13-30 più 35\% non combattente più 1 capo combattente di 4°-6° livello, 11-16 Lupi Crudeli, 1-4 Ogre e 13-20 schiavi orchi)\\
\textbf{Tesoro}: Standard (Armatura di Pelle, Randello Pesante, altro tesoro)\\
\textbf{Descrizione}\\
I giganti di Collina hanno pelle che varia dal marrone chiaro al rossastro, capelli castani o neri, ed occhi dello stesso colore. Indossano strati di pelli rozzamente conciate con ancora il pelo. Raramente lavano o riparano i propri indumenti, e preferiscono semplicemente aggiungere nuovi strati man mano che i vecchi si logorano. Gli adulti sono alti circa 3 metri e pesano più o meno 550 kg. I giganti di Collina possono vivere fino a 200 anni, anche se raramente raggiungono quest'età.

I giganti di Collina preferiscono combattere dall'alto di sporgenze e rupi, da dove possono colpire gli avversari con rocce e massi, limitando così il rischio personale. Amano effettuare attacchi di oltrepassare contro creature più piccole all'inizio del combattimento, e solo dopo prendono posizione e iniziano a roteare i loro massicci randelli.

I giganti di Collina sono per natura nomadi e preferiscono viaggiare da un luogo all'altro per razziare e saccheggiare. Sebbene gradiscano di più i climi temperati, non disdegnano di viaggiare lontano dal loro ambiente favorito, se la razzia è abbondante e prospera. Si tratta, nel complesso, di creature molto egoiste, che raramente affrontano battaglie che non siano sicuri di vincere. I giganti delle colline sono noti per l'abitudine di spingersi l'un l'altro se devono confrontarsi con avversari temibili e non esitano a sacrificare un compagno per salvarsi la pelle. Bande erranti di giganti di Collina sono diffuse sulle colline temperate, e la loro costante aggressività li rende uno dei pericoli più temuti in questo ambiente.


I giganti di Collina solitari e non malvagi sono molto rari, ma li si può trovare qualche volta in altre società umanoidi, anche se non sono quasi mai accettati nelle città principali o nei centri popolati. Si trovano a proprio agio come lavoratori e soldati nelle remote città di frontiera, e spesso fungono da rudimentali diplomatici per negoziare con le bande di giganti di Collina razziatori. Sfortunatamente, i giganti di Collina che abbandonano il proprio stile di vita razziale per la civiltà vengono derisi e spesso uccisi a vista dai loro fratelli nomadi. Tuttavia, questi giganti di Collina "civilizzati" possono trovare il proprio posto nella società e molti sono riusciti a vivere un'esistenza pacifica e tranquilla.


\medskip\index[Mostruario]{Gigante del Fuoco}\textbf{Gigante del Fuoco}

\textit{Enorme gigante, legale malvagio}

\textbf{FORZA} +7

\textbf{DESTREZZA} -1

\textbf{COSTITUZIONE} +6

\textbf{INTELLIGENZA} +0

\textbf{SAGGEZZA} +2

\textbf{CARISMA} +1

\textbf{Iniziativa} +0 -- \textbf{Difesa} 27 (armatura di piastre)

\textbf{Punti Ferita} 162 (13d12 + 78)

\textbf{Movimento} 9 m

\textbf{Tiri Salvezza}: Tempra +14, Riflessi +4, Volontà +9

\textbf{Competenze} Acrobatica +11, Consapevolezza +6

\textbf{Immunità ai Danni} fuoco

\textbf{Linguaggi} Gigante

\textbf{Sfida} 9 (5000 PX)

\textbf{Azioni}

\textit{\textbf{Multiattacco.}} Il gigante effettua due attacchi con lo spadone.

\textit{\textbf{Spadone.} Attacco con arma da mischia}: +20 a colpire, portata 3 m, un bersaglio.

\textit{Colpisce:} 28 (6d6 + 7) danni taglienti.

\textit{\textbf{Sasso.} Attacco con arma a Distanza}: +12 a colpire, gittata 18m, un bersaglio.

\textit{Colpisce:} 29 (4d10 + 7) danni da botta.

\textbf{Ecologia}
Ambiente: Montagne calde\\
Organizzazione: Solitario, gruppo (2-5), banda (6-12 più un 35\% non combattenti e 1 adepto o Devoto di 1°-2° livello), gruppo di razziatori (6-12 più 1 adepto o mago di 3°-5° livello, 2-5 Segugi Infernali e 2-3 Troll o Ettin) o tribù (20-30 più 1 adepto, mago o Devoto di 6°-7° livello; 1 re Guerriero o guardiaboschi di 8°-9° livello; e 17-38 Segugi Infernali, 12-22 Troll, 7-12 Ettin e 1-2 Draghi Rossi Giovani)\\
\textbf{Tesoro}: Standard (Mezza Armatura, Spadone, altro tesoro)\\
\textbf{Descrizione}\\
I giganti del fuoco sono i giganti più rigidi e marziali, sempre pronti alla guerra e a trattare brutalmente chiunque incontrino. La loro rigida struttura di comando prevede soldati, ufficiali e persino generali, e che tutti obbediscano agli ordini del loro re senza discutere.

I giganti del fuoco hanno capelli arancione brillante che splendono e scintillano come se fossero in fiamme. Un maschio adulto è alto tra i 3,6 e i 4,8 metri, con una cassa toracica di circa 2,7 metri, e pesa circa 3.500 kg. Le femmine sono leggermente più basse e snelle. I giganti del fuoco possono vivere fino a 350 anni.

I giganti del fuoco indossano abiti di tessuti robusti o di pelle di color arancione, giallo, nero o rosso. I guerrieri indossano elmi e mezze armature di acciaio brunito e impugnano grandi spadoni che mulinano per il campo di battaglia. In gruppi numerosi, i giganti del fuoco combattono con tattiche di gruppo brutali ed efficienti, e non esitano a sacrificare qualche compagno per tendere un'imboscata al nemico.

I giganti del fuoco preferiscono i luoghi caldi: più caldi sono meglio è. Si possono trovare nei deserti, nei vulcani, nelle fonti termali e nelle profondità della terra nei pressi di camini lavici. Vivono in castelli, insediamenti fortificati o grandi caverne, e l'architettura di questi luoghi riflette il loro stile di vita rigido e militaristico, con gli ufficiali che abitano in alloggi migliori di quelli dei loro sottoposti.



\medskip\index[Mostruario]{Gigante del Gelo}\textbf{Gigante del Gelo}

\textit{Enorme gigante, neutrale malvagio}

\textbf{FORZA} +6

\textbf{DESTREZZA} -1

\textbf{COSTITUZIONE} +5

\textbf{INTELLIGENZA} -1

\textbf{SAGGEZZA} +0

\textbf{CARISMA} +1

\textbf{Iniziativa} -1 -- \textbf{Difesa} 19 (armatura composita)

\textbf{Punti Ferita} 138 (12d12 + 60)

\textbf{Movimento} 12 m

\textbf{Tiri Salvezza} Tempra +14, Riflessi +3, Volontà +6

\textbf{Competenze} Acrobatica +9, Consapevolezza +3

\textbf{Immunità ai Danni} freddo

\textbf{Linguaggi} Gigante

\textbf{Sfida} 8 (3.900 PX)

\textbf{Azioni}

\textit{\textbf{Multiattacco.}} Il gigante effettua due attacchi con l'ascia bipenne.

\textit{\textbf{Ascia Bipenne.} Attacco con arma da mischia}: +18 a colpire, portata 3 m, un bersaglio.

\textit{Colpisce:} 25 (3d12 + 6) danni taglienti.

\textit{\textbf{Sasso.} Attacco con arma a Distanza}: +11 a colpire, gittata 18m, un bersaglio.

\textit{Colpisce:} 28 (4d10 + 6) danni da botta.

\textbf{Ecologia}\\
Ambiente: Montagne fredde\\
Organizzazione: Solitario, banda (3-5), gruppo (6-12 più 35\% non combattenti e 1 mago o Devoto di 1°-2° livello), gruppo di razziatori (6-12 più 35\% non combattenti, 1 Devoto o mago di 3°-5° livello, 1-4 Lupi Invernali e 2-3 Ogre) o tribù (21-30 più 1 adepto, mago o Devoto di 6°-7° livello; 1 jarl Barbaro o guardiaboschi 7°-9° livello; e 15-36 Lupi Invernali, 13-22 Ogre e 1-2 Draghi Bianchi Giovani)\\
\textbf{Tesoro}: Standard (Giaco di Maglia, Ascia Bipenne, altro tesoro)\\
\textbf{Descrizione}\\
Un gigante del gelo ha capelli azzurri o giallo sporco, e occhi in genere dello stesso colore. Si vestono con pelli e pellicce, adornandosi con qualsiasi gioiello possiedano. I giganti del gelo combattenti indossano anche giachi di maglia ed elmi di metallo decorati con corna e piume. Un maschio adulto è alto 5 metri e pesa circa 1.400 kg. Le femmine sono leggermente più basse e snelle, ma per il resto sono identiche ai maschi. I giganti del gelo possono vivere fino a 250 anni.

I giganti del gelo sono molto temuti, poiché la brama di distruzione e guerra ed il loro comportamento sprezzante li spingono a manifestazioni di brutalità sempre maggiori. I giganti del gelo iniziano attaccando a distanza, scagliando rocce finché finiscono le munizioni o l'avversario si avvicina, poi lo affrontano con le loro enormi asce. Una delle tattiche preferite è tendere un'imboscata nascondendosi sotto la neve al di sopra di un pendio ghiacciato o innevato, dove gli avversari avranno difficoltà a raggiungerli, e poi iniziano causando una valanga prima di scendere in battaglia. I giganti del gelo possono nascondersi molto bene negli ambienti nevosi e sono dei maestri nella furtività nel loro dominio.

I giganti del gelo sopravvivono cacciando e razziando da soli, dato che vivono in ambienti freddi e desolati. I gruppi di giganti del gelo sono divisi quasi equamente tra quelli che vivono in insediamenti di fortuna o castelli abbandonati e quelli che vagabondano per il gelido nord, come nomadi in cerca di bottino e provviste. I capi dei giganti del gelo si chiamano jarl e richiedono obbedienza assoluta ai loro seguaci. In ogni momento uno jarl può essere sfidato in combattimento per il comando della tribù. Queste sfide tipicamente finiscono con la morte di uno dei contendenti. Un singolo jarl può spesso contare su una dozzina o più di tribù più piccole di giganti del gelo come estensione della sua. In questi casi, i capi delle tribù minori sono noti come capitani o signori della guerra.

I giganti del gelo amano prendere prigionieri e li usano sia come schiavi che come materia prima. Di solito ogni gruppo di giganti del gelo tiene 1-2 schiavi umanoidi incatenati ad un addestratore di schiavi: il più meschino e crudele del gruppo dopo lo jarl. Hanno anche una certa passione per gli animali domestici mostruosi: Draghi Bianchi e Lupi Invernali sono scelte popolari, ma nella tana di un gigante del gelo si possono trovare anche Remorhaz e Yeti.

\medskip\index[Mostruario]{Gigante delle Nuvole}\textbf{Gigante delle Nuvole}

\textit{Enorme gigante, neutrale buono (50\%) o neutrale malvagio (50\%)}

\textbf{FORZA} +8

\textbf{DESTREZZA} +0

\textbf{COSTITUZIONE} +6

\textbf{INTELLIGENZA} +1

\textbf{SAGGEZZA} +3

\textbf{CARISMA} +3

\textbf{Iniziativa} +1 -- \textbf{Difesa} 19

\textbf{Punti Ferita} 200 (16d12 + 96)

\textbf{Movimento} 12 m

\textbf{Tiri Salvezza} Tempra +16, Riflessi +6, Volontà +10

\textbf{Competenze} Percepire Emozioni +7, Consapevolezza +7

\textbf{Linguaggi} Comune, Gigante

\textbf{Sfida} 9 (5000 PX)

\textit{\textbf{Incantesimi Innati.}} La caratteristica da incantatore del gigante è il Carisma. Il gigante può lanciare questi incantesimi in maniera innata, senza bisogno di componenti materiali:

A volontà: \textit{individuazione del magico, luce, nube di nebbia}

3/giorno ciascuno: \textit{Caduta Piuma, passo nebbioso, telecinesi}

1/giorno ciascuno: \textit{controllare tempo atmosferico, forma gassosa}

\textit{\textbf{Olfatto Affinato.}} Il gigante ha +1d6 alle prove di Saggezza (Consapevolezza) basate sull'olfatto.

\textbf{Azioni}

\textit{\textbf{Multiattacco.}} Il gigante effettua due attacchi con la Mazza chiodata.

\textit{\textbf{Mazza chiodata.} Attacco con arma da mischia}: +22 a colpire, portata 3 m, un bersaglio.

\textit{Colpisce:} 21 (3d8 + 8) danni perforanti.

\textit{\textbf{Sasso.} Attacco con arma a Distanza}: +14 a colpire, gittata 18m, un bersaglio.

\textit{Colpisce:} 30 (4d10 + 8) danni da botta.

\textbf{Ecologia}\\
Ambiente: Montagne Temperate\\
Organizzazione: Solitario, gruppo (2-5), famiglia (2-5 più 35\% non combattenti più 1 mago o Devoto di 4°-7° livello e 2-5 Grifoni) o tribù (6-20 più 1 oracolo mago o Devoto di 7°-12° livello e 2-5 Grifoni)\\
\textbf{Tesoro}: Standard (Giaco di Maglia, Mazza chiodata, altro tesoro)\\
\textbf{Descrizione}\\
Il colore pelle dei giganti delle nuvole varia dal bianco latte al blu polvere. I maschi adulti sono alti circa 5,4 metri e pesano approssimativamente 2.500 kg. Le femmine sono leggermente più basse e snelle. I giganti delle nuvole possono vivere fino a 400 anni, vestono con abiti preziosi e gioielli. Per molti l'aspetto indica lo status. Migliori sono i vestiti e più raffinati i gioielli, più importante è chi li indossa. Inoltre apprezzano la musica, e la maggioranza suona uno o più strumenti (l'arpa è uno dei preferiti).

I giganti delle nuvole possono avere Tratti insolitamente vari; circa metà sono buoni e metà malvagi. I giganti delle nuvole buoni costruiscono strade che collegano i loro insediamenti con le strade degli umani per promuovere il commercio. Non è insolito vedere un gigante delle nuvole buono camminare tra gli uomini, ad esempio, in una città umana nei pressi di un'alta catena montuosa. I giganti delle nuvole malvagi tendono a non creare insediamenti stabili e anzi preferiscono vivere in rozzi rifugi su alti picchi, da cui scendono solo per depredare i villaggi di quello di cui potrebbero aver bisogno. Queste due filosofie portano spesso allo scoppio di guerre violente e durature tra tribù vicine.

Sono molte le leggende che parlano di magiche città dei giganti delle nuvole situate tra le nuvole stesse, che fluttuano sui venti e circumnavigano il mondo. Mentre i giganti delle nuvole riconoscono che si tratta per lo più di fantasie, alcuni sostengono di averle viste e hanno dedicato la loro intera esistenza a ritrovarle.


\medskip\index[Mostruario]{Gigante di Pietra}\textbf{Gigante di Pietra}

\textit{Enorme gigante, neutrale}

\textbf{FORZA} +6

\textbf{DESTREZZA} +2

\textbf{COSTITUZIONE} +5

\textbf{INTELLIGENZA} +0

\textbf{SAGGEZZA} +1

\textbf{CARISMA} -1

\textbf{Iniziativa} +2 -- \textbf{Difesa} 21

\textbf{Punti Ferita} 126 (11d12 + 55)

\textbf{Movimento} 12 m

\textbf{Tiri Salvezza} Tempra +12, Riflessi +6, Volontà +7

\textbf{Competenze} Acrobatica +12, Consapevolezza +4

\textbf{Sensi} scurovisione 18 m

\textbf{Linguaggi} Gigante

\textbf{Sfida} 7 (2.900 PX)

\textit{\textbf{Mimetismo di Pietra.}} Il gigante ha +1d6 alle prove di Destrezza (Nascondersi) effettuate per nascondersi su terreni rocciosi.

\textbf{Azioni}

\textit{\textbf{Multiattacco.}} Il gigante effettua due attacchi con il randello pesante.

\textit{\textbf{Randello Pesante.} Attacco con arma da mischia}: +19 a colpire, portata 5 metri, un bersaglio.

\textit{Colpisce:} 19 (3d8 + 6) danni da botta.

\textit{\textbf{Sasso.} Attacco con arma a Distanza}: +15 a colpire, gittata 18m, un bersaglio.

\textit{Colpisce:} 28 (4d10 + 6) danni da botta. Se il bersaglio è una creatura, deve riuscire un Tiro Salvezza di Tempra DC 17 o cadere prona.

\textbf{Reazioni}

\textit{\textbf{Afferrare Sassi.}} Se un sasso o un simile oggetto viene scagliato al gigante, il gigante può, riuscendo un Tiro Salvezza su Riflessi DC 10, afferrare il proiettile e non subire danni da botta da esso.

\textbf{Ecologia}
Ambiente: Montagne temperate\\
Organizzazione: Solitario, gruppo (2-5), banda (4-8), gruppo di caccia (9-12 più 1 Anziano) o tribù (13-30 più 35\% non combattenti, 1-3 Anziani e 4-6 Orsi Crudeli)\\
\textbf{Tesoro}: Standard (Randello Pesante, altro tesoro)\\
\textbf{Descrizione}\\
I giganti di Pietra preferiscono spessi indumenti di cuoio, tinti con tonalità di marrone e grigio per confondersi con la pietra che li circonda. Gli adulti sono alti circa 3,6 metri, pesano circa 750 kg e possono vivere fino a 800 anni.

I giganti di Pietra, se possibile, combattono a distanza, ma se non possono evitare la mischia usano giganteschi randelli di pietra. Una delle tattiche favorite dai giganti di Pietra è di stare immobili, mimetizzandosi con il paesaggio, per poi avanzare scagliando rocce e sorprendere i nemici.

I giganti di Pietra preferiscono vivere in enormi caverne sulle cime rocciose. Raramente vivono a più di qualche giorno di viaggio da altre bande di giganti di Pietra e allevano greggi condivisi di capre e altro bestiame.

I giganti di Pietra più vecchi tendono ad allontanarsi dalla tribù per molto tempo, per vivere in solitudine da qualche parte o tentando di inserirsi in altre civiltà umanoidi. Dopo decadi di esilio auto imposto, chi fa ritorno è noto come Gigante delle Rocce Anziano.


\medskip\index[Mostruario]{Gigante delle Tempeste}\textbf{Gigante delle Tempeste}

\textit{Enorme gigante, caotico buono}

\textbf{FORZA} +9

\textbf{DESTREZZA} +2

\textbf{COSTITUZIONE} +5

\textbf{INTELLIGENZA} +3

\textbf{SAGGEZZA} +4

\textbf{CARISMA} +4

\textbf{Iniziativa} +3 -- \textbf{Difesa} 23 (armatura di scaglie)

\textbf{Punti Ferita} 230 (20d12 + 100)

\textbf{Movimento} 15 m, nuoto 15 m

\textbf{Tiri Salvezza} Tempra +17, Riflessi +8, Volontà +13

\textbf{Competenze} Arcano +8, Acrobatica +14, Consapevolezza +9, Storia +8

\textbf{Resistenze al Danno} freddo

\textbf{Immunità al Danno} fulmine, suono

\textbf{Linguaggi} Comune, Gigante

\textbf{Sfida} 13 (10000 PX)

\textit{\textbf{Anfibio.}} Il gigante può respirare aria e acqua.

\textit{\textbf{Incantesimi Innati.}} La caratteristica da incantatore del gigante è il Carisma. Il gigante può lanciare questi incantesimi in maniera innata, senza bisogno di componenti materiali:

A volontà: \textit{caduta controllata, individuazione del magico,} \textit{levitazione, luce}

3/giorno ciascuno: \textit{controllare tempo atmosferico, respirare} \textit{sott'acqua}

\textbf{Azioni}

\textit{\textbf{Multiattacco.}} Il gigante effettua due attacchi con lo spadone.

\textit{\textbf{Spadone.} Attacco con arma da mischia}: +29 a colpire, portata 3 m, un bersaglio.

\textit{Colpisce:} 30 (6d6 + 9) danni taglienti.

\textit{\textbf{Sasso.} Attacco con arma a Distanza}: +22 a colpire, gittata 18m, un bersaglio.

\textit{Colpisce:} 35 (4d12 + 9) danni da botta.

\textit{\textbf{Colpo Fulminante (Ricarica 5-6).}} Il gigante scaglia una folgore magica ad un punto visibile entro 150 metri da sé. Ogni creatura entro 3 metri da quel punto deve effettuare un Tiro Salvezza su Riflessi DC 17, subendo 54 (12d8) danni da fulmine se lo fallisce, o la metà se lo supera.

\textbf{Ecologia}\\
Ambiente: Qualsiasi caldo\\
Organizzazione: Solitario o famiglia (2-5 più 1 mago o Devoto di livello 7°-10°, 1-2 Roc, 2-6 Grifoni e 2-8 Squali)\\
\textbf{Tesoro}: Standard (Corazza di Piastre Perfetta, Arco Lungo Composito Perfetto [Forza +9] con 20 Frecce, Spadone Perfetto, altro tesoro)\\
\textbf{Descrizione}\\
I giganti delle tempeste tendono ad avere carnagione abbronzata, anche se rari esemplari hanno pelle viola, capelli viola o blu scuri e occhi grigio argento o porpora. Il colore viola è considerato di buon auspicio tra i giganti delle tempeste, e coloro che lo posseggono tendono a diventare capi tra la loro gente. Gli adulti sono normalmente alti 6,3 metri e pesano 6000 kg. I giganti delle tempeste possono vivere fino a 600 anni.

Quando sono a riposo, preferiscono indossare tuniche corte e ampie cinte ai fianchi, sandali o piedi nudi e una fascia per capelli. Indossano pochi gioielli di semplice ma ottima fattura, i più comuni sono cavigliere (preferite dai giganti a piedi scalzi), anelli o diademi. Ma quando si equipaggiano per la battaglia, indossano corazze di piastre scintillanti e impugnano enormi spadoni e archi.

I giganti delle tempeste sono tendenzialmente solitari, preferendo abitare lungo remote coste o su isole tropicali. Come suggerisce il loro nome, sono inclini a violenti sbalzi di umore. I giganti delle tempeste sono facili all'ira di fronte al male e possono essere nemici brutali e pericolosi quando vengono insultati. In battaglia, preferiscono scagliare una pioggia di frecce sui loro nemici, estraendo gli spadoni solo dopo che gli avversari si sono avvicinati.

I giganti delle tempeste vivono in belle torri, castelli o in insediamenti cinti da mura e amano coltivare la terra. Possiedono enormi giardini ben curati e gestiscono centinaia di acri di coltivazioni per gruppo. Spesso impiegano altri umanoidi, come Elfi o Umani, come supporto per condurre le loro immense fattorie. Una enclave di giganti delle tempeste spesso si assume la responsabilità della sicurezza di un'intera isola o linea di costa.

\medskip\index[Mostruario]{Gnoll}\textbf{Gnoll}

\textit{Media umanoide (gnoll), caotico malvagio}

\textbf{FORZA} +2

\textbf{DESTREZZA} +1

\textbf{COSTITUZIONE} +0

\textbf{INTELLIGENZA} -2

\textbf{SAGGEZZA} +0

\textbf{CARISMA} -2

\textbf{Iniziativa} +1 -- \textbf{Difesa} 16 (armatura di pelle, scudo)

\textbf{Punti Ferita} 22 (5d8)

\textbf{Movimento} 9 m

\textbf{Tiri Salvezza}: Tempra +4, Riflessi +0, Volontà +0

\textbf{Sensi} scurovisione 18 m

\textbf{Linguaggi} Gnoll

\textbf{Sfida} 1/2 (100 PX)

\textit{\textbf{Rabbia.}} Quando lo gnoll riduce una creatura a 0 Punti Ferita con un attacco da mischia durante il proprio round, può svolgere un'azione bonus per muoversi fino a metà del suo movimento ed effettuare un attacco di morso.

\textbf{Azioni}

\textit{\textbf{Morso.} Attacco con arma da mischia}: +4 a colpire, portata 1 m, una creatura.

\textit{Colpisce:} 4 (1d4 + 2) danni perforanti.

\textit{\textbf{Lancia.} Attacco con arma da mischia o a Distanza}: +4 a colpire, portata 1 m o gittata 6 m, un bersaglio.

\textit{Colpisce:} 5 (1d6 + 2) danni perforanti o 6 (1d8 + 2) danni perforanti se usata con due mani per effettuare un attacco da mischia.

\textit{\textbf{Arco Lungo.} Attacco con arma a Distanza}: +3 a colpire, gittata 45m, un bersaglio.

\textit{Colpisce:} 5 (1d8 + 1) danni perforanti.

\textbf{Ecologia}\\
Ambiente: Pianure calde, deserti\\
Organizzazione: Solitario, coppia, gruppo di caccia (2-5 e 1-2 Iene), banda (10-100 adulti più 50\% piccoli non combattenti, 1 sergente di 3° livello ogni 20 adulti, 1 capo di 4°-6° livello e 5-8 Iene) o tribù (20-200 più 1 sergente di 3° livello ogni 20 adulti, 1 o 2 luogotenenti di 4° o 5° livello, 1 capo di 6°-8° livello, 7-12 Iene e 4-7 ienodonti)\\
\textbf{Tesoro}: equipaggiamento da PNG (Armatura di Cuoio, Scudo Pesante di Legno, Lancia, altro tesoro)\\
\textbf{Descrizione}\\
Gli gnoll sono una razza di umanoidi grandi e grossi che assomigliano alle iene non solo per il semplice aspetto; mostrano un'evidente affinità con questi animali spazzini, tanto da tenerli come animali di compagnia, e riflettono molti dei comportamenti di tali animali.

Gli gnoll sono abili cacciatori, ma preferiscono di gran lunga ripulire o trafugare una carcassa piuttosto che cacciare una preda. Questa pigrizia li spinge a procurarsi degli schiavi di qualsiasi specie disponibile per costringerli a scavare tane, raccogliere provviste e acqua e perfino cacciare per i loro padroni gnoll.

Le altre creature che non siano iene o gnoll diventano pasto o schiavi, a seconda del temperamento della tribù. Anche un compagno morto o caduto diventa un pasto fresco per uno gnoll, che può onorare un membro famoso della tribù con una breve preghiera o cucinarne interamente uno morto di una devastante malattia: altrimenti, gli gnoll non vedono un loro simile morto molto diversamente da qualsiasi altra creatura. Gli gnoll più "civilizzati" non mangiano i loro prigionieri: li tengono, invece, come schiavi, per difendere o migliorare la loro tana o per scambiarli con altre tribù o bande schiaviste.

Gli gnoll provano gusto per il combattimento, ma solo quando sono in superiorità numerica. In altre situazioni, preferiscono evitare il combattimento tranne che come mezzo per ottenere una carcassa da un altro cacciatore, o come un'ingegnosa imboscata per abbattere un lauto pasto. Questi uomini iena non vedono alcun valore nel coraggio o nell'eroismo e preferiscono invece fuggire una volta chiaro che la vittoria non è raggiungibile, sostenendo che è meglio scappare con la coda tra le gambe piuttosto che perderla del tutto.

Durante il combattimento, gli gnoll usano una strana combinazione di tattiche da branco e strategie individuali. Se uno gnoll è sicuro di vincere, tenta di abbattere l'avversario più debole piuttosto che aiutare i suoi compagni. Se gli gnoll sono in difficoltà, si coalizzano contro un avversario potente e tentano di eliminarlo, sperando di costringere alla fuga i suoi alleati.

I capi gnoll hanno competenze da guardiaboschi ma non è impossibile trovare anche gnoll Devoti a qualche famelico Patrono. Difficilmente padroneggiano in maniera efficace la magia.


\medskip\index[Mostruario]{Gnomo delle Profondità}\textbf{Gnomo delle Profondità}

\textit{Piccola umanoide (gnomo), neutrale buono}

\textbf{FORZA} +2

\textbf{DESTREZZA} +2

\textbf{COSTITUZIONE} +2

\textbf{INTELLIGENZA} +1

\textbf{SAGGEZZA} +0

\textbf{CARISMA} -1

\textbf{Iniziativa} +2 -- \textbf{Difesa} 16 (giaco di maglia)

\textbf{Punti Ferita} 16 (3d6 + 6)

\textbf{Movimento} 6 m

\textbf{Tiri Salvezza}: Tempra +6, Riflessi +6, Volontà +2

\textbf{Competenze} Muoversi Silenziosamente / Nascondersi +4, Consapevolezza +2

\textbf{Sensi} scurovisione 36 m

\textbf{Linguaggi} Gnomica, Linguaggio delle Profondità, Terran

\textbf{Sfida} 1/2 (100 PX)

\textit{\textbf{Astuzia Gnomesca.}} Lo gnomo ha +1d6 ai Tiri Salvezza contro la magia.

\textit{\textbf{Camuffamento di Pietra.}} Lo gnomo ha +1d6 alle prove di Destrezza (Nascondersi) effettuate per nascondersi su terreni rocciosi.

\textit{\textbf{Incantesimi Innati.}} La caratteristica da incantatore innato dello gnomo è l'Intelligenza. Lo gnomo può lanciare questi incantesimi in maniera innata, senza bisogno di componenti:

A volontà: \textit{anti-individuazione} (personale)

1/giorno ciascuno: \textit{camuffare sé stesso, cecità/sordità, sfocatura}

\textbf{Azioni}

\textit{\textbf{Piccone da Guerra.} Attacco con arma da mischia}: +4 a colpire, portata 1 m, un bersaglio.

\textit{Colpisce:} 6 (1d8 + 2) danni perforanti.

\textit{\textbf{Dardo Avvelenato.} Attacco con arma a Distanza}: +4 a colpire, gittata 9m, un bersaglio.

\textit{Colpisce:} 4 (1d4 + 2) danni perforanti, e il bersaglio deve riuscire un Tiro Salvezza di Tempra DC 12 o restare avvelenato per 1 minuto. Il bersaglio può ripetere il Tiro Salvezza al termine di ciascun suo round, terminando l'effetto su di sé in caso di successo.

\textbf{Ecologia}
Ambiente: Qualsiasi sotterraneo\\
Organizzazione: Solitario, compagnia (2-4), squadra (5-20 più 1 capo 3°-6° e due sergenti di 3° livello), o banda (30-50 più 1 sergente di 3° livello ogni 20 adulti, 5 tenenti di 5° livello, 3 capitani di 7° livello, e 2-5 Elementali della Terra Medi)\\
\textbf{Tesoro}: Equipaggiamento da PNG (Piccone Pesante, Balestra Leggera con 10 Quadrelli, altro tesoro)\\
\textbf{Descrizione}\\
I gnomi delle profondità, sono una branca della razza gnomesca. Dimorano nel sottosuolo, in città nascoste, al sicuro dagli elfi scuri e da altre razze sotterranee. La loro pelle è del colore della roccia, di solito grigia o marrone. I maschi sono calvi e le femmine hanno radi capelli grigi.

\medskip\index[Mostruario]{Globulo}\textbf{Globulo}

\textit{Piccola aberrazione, malvagio}

\textbf{FORZA} -2

\textbf{DESTREZZA} +2

\textbf{COSTITUZIONE} +0

\textbf{INTELLIGENZA} +3

\textbf{SAGGEZZA} +1

\textbf{CARISMA} +3

\textbf{Iniziativa} +3 -- \textbf{Difesa} 15

\textbf{Punti Ferita} 30 (5d10 + 5)

\textbf{Movimento} volare 18 m

\textbf{Tiri Salvezza}: Tempra +4, Riflessi +6, Volontà +5

\textbf{Competenze} -

\textbf{Sensi} scurovisione 36 m

\textbf{Linguaggi} comprende il comune ma non lo parla

\textbf{Sfida} 1 (200 PX)

\textbf{Vulnerabilità} Fuoco

\textbf{Immunità} Vuoto, Freddo

\textbf{Immunità alle condizioni} veleno, prono

\textbf{Odio i volatili} il Globulo ha +1d6 al Tiro per Colpire contro gli uccelli. Attacca prima gli uccelli e creature volanti

\textbf{Natura inusuale} il Globulo non respira

\textbf{Odio l'acqua} il Globulo detesta bagnarsi e ogni 5 litri di acqua spruzzata su lui subisce 1d4 di danno

\textbf{Azioni}

\textit{\textbf{Tentacolo}}. Attacco in mischia, +5 al colpire, portata 3 metri, un obiettivo

\textit{\textbf{Colpisce}} 5 (1d6+2) di danno da Vuoto. Il bersaglio deve fare un Tiro Salvezza su Tempra a DC 11 o aumentare il grado di Affaticamento di 1.

\textbf{\textit{Brillio}} una volta al giorno il Globulo diventa estremamente luminoso, le creature nel raggio di 6 metri attorno a lui devono fare un Tiro Salvezza su Tempra a DC 13 o diventare ciechi per 3 round.

\textbf{Ecologia}
Ambiente: Qualsiasi, desertico, notturno\\
Organizzazione: Solitario, gruppi 2d4\\
\textbf{Tesoro}: Nessuno\\
\textbf{Descrizione}\\
I Globuli sono aberrazioni magiche provenienti da qualche portale aperto verso l'Oltre. Creature di freddo e vuoto sembrano delle piccole stelle che anelano solo di risucchiare la vita della creature incontrate.
Intelligenti e furbe preferiscono attaccare rimanendo in volo e fiaccando l'avversario finché questo è esausto. Una volta ucciso di un Globulo non rimane che una piccola creatura a forma di stella con un grosso occhio centrale, completamente bianco.


\medskip\index[Mostruario]{Goblin}\textbf{Goblin}

\textit{Piccola umanoide (goblinoide), caotico malvagio}

\textbf{FORZA} +0

\textbf{DESTREZZA} +0

\textbf{COSTITUZIONE} +1

\textbf{INTELLIGENZA} -1

\textbf{SAGGEZZA} -2

\textbf{CARISMA} -1

\textbf{Iniziativa} +0 -- \textbf{Difesa} 13

\textbf{Punti Ferita} 7 (2d6 + 1)

\textbf{Movimento} 9 m

\textbf{Tiri Salvezza}: Tempra +1, Riflessi +1, Volontà -1

\textbf{Sensi} scurovisione 18 m

\textbf{Linguaggi} Comune, Goblin

\textbf{Sfida} 1/4 (50 PX)

\textbf{Azioni}

\textit{\textbf{Spada Corta.} Attacco con arma da mischia}: +1 a colpire, portata 1 m, un bersaglio.

\textit{Colpisce:} 4 (1d6 + 1) danni taglienti

\textit{\textbf{Arco Corto.} Attacco con arma a Distanza}: +1 a colpire, gittata 15m, un bersaglio.

\textit{Colpisce:} 3 (1d6) danni perforanti.

\textbf{Ecologia}\\
Ambiente: Qualsiasi Temperate\\
Organizzazione: Gruppo (4-9), banda da guerra (10-24) o tribù (50+ più 50\% non combattenti\\
\textbf{Tesoro}: 1d4 monete d'argento\\
\textbf{Descrizione}\\
I goblin sono selvaggi, imprevedibili, rumorosi.
I goblin preferiscono vivere nelle caverne, nel fitto delle foreste e quando ne hanno a disposizione nelle strutture antiche abbandonate. I goblin non amano costruire quanto piuttosto distruggere per poi lamentarsi che non c'è nulla di utile.

I goblin sono molto superstiziosi, e vedono la magia con un misto di timore reverenziale e  paura. Ogni cosa che non comprendono è per loro magia e questo li porta a essere estremamente sospettosi di tutto e a distruggere tutto, visto che ciò che non capiscono va distrutto.

I goblin sono famelici e possono mangiare enormi quantità di cibo. un goblin non rinuncia a mangiare nulla tranne forse l'insalata..


\subsection{Golem}

\medskip\index[Mostruario]{Golem di Argilla}\textbf{Golem di Argilla}

\textit{Grande costrutto, disallineato}

\textbf{FORZA} +5

\textbf{DESTREZZA} -1

\textbf{COSTITUZIONE} +4

\textbf{INTELLIGENZA} -4

\textbf{SAGGEZZA} -1

\textbf{CARISMA} -5

\textbf{Iniziativa} -1 -- \textbf{Difesa} 19

\textbf{Punti Ferita} 133 (14d10 + 56)

\textbf{Movimento} 6 m

\textbf{Tiri Salvezza}: Tempra +4, Riflessi +3, Volontà +4

\textbf{Immunità al Danno} acido, veleno; da arma non magica o che non siano di adamantio

\textbf{Immunità alle Condizioni} affascinato, avvelenato, paralizzato, pietrificato, affaticamento, spaventato

\textbf{Sensi} scurovisione 18 m

\textbf{Linguaggi} comprende le lingue del suo creatore ma non può parlare

\textbf{Sfida} 9 (5000 PX)

\textit{\textbf{Berserk.}} Ogni volta che il golem inizia il suo round con 60 Punti Ferita o meno, tira un d6. Se ottieni 6, il golem va in berserk. Durante ogni suo round mentre è in berserk, il golem attacca la creatura più vicina che può vedere. Se non c'è nessuna creatura abbastanza vicina da muoversi e attaccarla, il golem attacca un oggetto, con preferenza per gli oggetti più piccoli di lui. Una volta che il golem è andato in berserk, continuerà ad esserlo finché non viene distrutto o recupera tutti i suoi Punti Ferita.

\textit{\textbf{Armi Magiche.}} Gli attacchi con armi del golem sono magici.

\textit{\textbf{Assorbimento dell'Acido.}} Ogni volta che il golem è vittima di danni da acido, non subisce danni ma invece recupera un pari numero di Punti Ferita.

\textit{\textbf{Forma Immutabile.}} Il golem è immune a qualsiasi incantesimo o effetto che altererebbe la sua forma.

\textit{\textbf{Natura di Costrutto.}} Un golem non ha bisogno di aria, cibo, bevande o sonno.

\textit{\textbf{Resistenza alla Magia.}} Il golem ha +1d6 ai Tiri Salvezza contro incantesimi e altri effetti magici.

\textbf{Azioni}

\textit{\textbf{Multiattacco.}} Il golem effettua due attacchi di schianto.

\textit{\textbf{Schianto.} Attacco con arma da mischia}: +18 a colpire, portata 1 m, un bersaglio.

\textit{Colpisce:} 16 (2d10 + 5) danni da botta.

\textit{\textbf{Velocità (Ricarica 5-6).}} Fino al termine del suo prossimo round, il golem ottiene un bonus magico di +2 alla Difesa, ha +1d6 ai Tiri Salvezza su Riflessi, e può usare gli attacchi di schianto come azione bonus.

\textbf{Ecologia}\\
Ambiente: Qualsiasi\\
Organizzazione: Solitario o gruppo (2-4)\\
\textbf{Tesoro}: Nessuno\\
\textbf{Descrizione}\\
I golem di argilla non indossano abiti, eccezion fatta per un indumento di cuoio trattato o metallo attorno ai fianchi. Mediamente sono alti più di 2,4 metri e pesano 300 chili.

\textbf{Costruzione}
Un golem d'argilla può essere scolpito a partire da un unico blocco d'argilla del peso minimo di 500 chili, trattato con polveri e oli rari per il valore di 1,500 mo.


\medskip\index[Mostruario]{Golem di Carne}\textbf{Golem di Carne}

\textit{Media costrutto, neutrale}

\textbf{FORZA} +4

\textbf{DESTREZZA} -1

\textbf{COSTITUZIONE} +4

\textbf{INTELLIGENZA} -2

\textbf{SAGGEZZA} +0

\textbf{CARISMA} -3

\textbf{Iniziativa} -1 -- \textbf{Difesa} 12

\textbf{Punti Ferita} 93 (11d8 + 44)

\textbf{Movimento} 9 m

\textbf{Tiri Salvezza}: Tempra +3, Riflessi +2, Volontà +3

\textbf{Immunità al Danno} fulmine, veleno; da arma non magica o che non siano di adamantio

\textbf{Immunità alle Condizioni} affascinato, avvelenato, paralizzato, pietrificato, affaticamento, spaventato

\textbf{Sensi} scurovisione 18 m

\textbf{Linguaggi} comprende le lingue del suo creatore ma non può
parlare

\textbf{Sfida} 5 (1.800 PX)

\textit{\textbf{Berserk.}} Ogni volta che il golem inizia il suo round con 40 Punti Ferita o meno, tira un d6. Se ottieni 6, il golem va in berserk. Durante ogni suo round mentre è in berserk, il golem attacca la creatura più vicina che possa vedere. Se non c'è nessuna creatura abbastanza vicina da muoversi e attaccarla, il golem attacca un oggetto, con preferenza per gli oggetti più piccoli di lui. Una volta che il golem è andato in berserk, continuerà ad esserlo finché non viene distrutto o recupera tutti i suoi Punti Ferita.

\textit{\textbf{Armi Magiche.}} Gli attacchi con armi del golem sono magici.

\textit{\textbf{Assorbimento dei Fulmini.}} Ogni volta che il golem sia vittima di un danno da fulmine, non subisce danni ma invece recupera un pari numero di Punti Ferita.

\textit{\textbf{Avversione al Fuoco.}} Se il golem subisce danni da fuoco, ha -1d6 ai tiri di attacco e le prove di competenza fino alla fine del suo prossimo round.

\textit{\textbf{Forma Immutabile.}} Il golem è immune a qualsiasi incantesimo o effetto che altererebbe la sua forma.

\textit{\textbf{Natura di Costrutto.}} Un golem non ha bisogno di aria, cibo, bevande o sonno.

\textit{\textbf{Resistenza alla Magia.}} Il golem ha +1d6 ai Tiri Salvezza contro incantesimi e altri effetti magici.

\textbf{Azioni}

\textit{\textbf{Multiattacco.}} Il golem effettua due attacchi di schianto.

\textit{\textbf{Schianto.} Attacco con arma da mischia}: +11 a colpire, portata 1 m, un bersaglio.

\textit{Colpisce:} 13 (2d8 + 4) danni da botta.

\textbf{Ecologia}\\
Ambiente: Qualsiasi\\
Organizzazione: Solitario o gruppo (2-4)\\
\textbf{Tesoro}: Nessuno\\
\textbf{Descrizione}\\
Un golem di carne è una mostruosa collezione di parti anatomiche umanoidi trafugate e cucite insieme. La sua carne cadaverica ha tonalità verde pallido o giallognola. Un golem di carne indossa qualsiasi tipo di vestito che il suo creatore desideri, normalmente solo un logoro paio di pantaloni. Non ha Equipaggiamento né armi. Un golem di carne è alto più di 2,4 metri e pesa 250 kg.

Un golem di carne non parla, anche se può emettere una specie di ringhio rauco. Cammina e si muove con un'andatura a scatti, come se non avesse il pieno controllo del proprio corpo.

Anche se molti golem di carne sono privi di ragione, si narra di golem eccezionali che in qualche modo hanno mantenuto i ricordi della vita precedente. La testa (e quindi il cervello) di questi golem di carne deve essere la giusta combinazione di freschezza e (nella vita precedente) decisione, ma di assoluta importanza sembrano essere anche la fortuna e il caso affinché durante la loro creazione si conservi l'intelletto. Certamente quelli che costruiscono golem di carne preferiscono avere schiavi privi di intelletto piuttosto che dotati di una propria volontà, di conseguenza i golem di carne intelligenti sono rari.


\medskip\index[Mostruario]{Golem di Ferro}\textbf{Golem di Ferro}

\textit{Grande costrutto, disallineato}

\textbf{FORZA} +7

\textbf{DESTREZZA} -1

\textbf{COSTITUZIONE} +5

\textbf{INTELLIGENZA} -4

\textbf{SAGGEZZA} +0

\textbf{CARISMA} -5

\textbf{Iniziativa} -1 -- \textbf{Difesa} 28

\textbf{Punti Ferita} 210 (20d10 + 100)

\textbf{Movimento} 9 m

\textbf{Tiri Salvezza}: Tempra +6, Riflessi +5, Volontà +6

\textbf{Immunità al Danno} fuoco, veleno; da arma non magica o che non siano di adamantio

\textbf{Immunità alle Condizioni} affascinato, avvelenato, paralizzato, pietrificato, affaticamento, spaventato

\textbf{Sensi} scurovisione 36 m

\textbf{Linguaggi} comprende le lingue del suo creatore ma non può parlare

\textbf{Sfida} 16 (15000 PX)

\textit{\textbf{Armi Magiche.}} Gli attacchi con armi del golem sono magici.

\textit{\textbf{Assorbimento del Fuoco.}} Ogni volta che il golem sia vittima di un danno da fuoco, non subisce danni ma invece recupera un pari numero di Punti Ferita.

\textit{\textbf{Forma Immutabile.}} Il golem è immune a qualsiasi incantesimo o effetto che altererebbe la sua forma.

\textit{\textbf{Natura di Costrutto.}} Un golem non ha bisogno di aria, cibo, bevande o sonno.

\textit{\textbf{Resistenza alla Magia.}} Il golem ha +1d6 ai Tiri Salvezza contro incantesimi e altri effetti magici.

\textbf{Azioni}

\textit{\textbf{Multiattacco.}} Il golem effettua due attacchi da mischia.

\textit{\textbf{Schianto.} Attacco con arma da mischia}: +30 a colpire, portata 1 m, un bersaglio.

\textit{Colpisce:} 20 (3d8 + 7) danni da botta.

\textit{\textbf{Spada.} Attacco con arma da mischia}: +30 a colpire, portata 3 m, un bersaglio.

\textit{Colpisce:} 23 (3d10 + 7) danni taglienti.

\textit{\textbf{Soffio Velenoso (Ricarica 6).}} Il golem esala un gas velenoso in un cono di 5 metri. Ogni creatura in quell'area deve effettuare un Tiro Salvezza di Tempra DC 19, subendo 45 (10d8) danni da veleno se fallisce il Tiro Salvezza, o la metà di questi danni se lo riesce.

\textbf{Ecologia}\\
Ambiente: Qualsiasi\\
Organizzazione: Solitario o gruppo (2-4)\\
\textbf{Tesoro}: Nessuno\\
\textbf{Descrizione}\\
Un golem di ferro ha un corpo di forma umanoide in ferro. Il creatore può dargli qualsiasi forma desideri, ma presenta quasi sempre un'armatura di qualche tipo, sia essa cerimoniale e preziosa o semplice e d'uso. Rispetto ad un golem di pietra ha sembianze molto più definite. I golem di ferro, talvolta, portano con sé un'arma, anche se il più delle volte tendono a preferirle i loro attacchi schianto.

Un golem di ferro è alto 3,6m e pesa circa 2.500 chili. Un golem di ferro non può parlare né emettere voce. Inoltre, non emette nessun odore riconoscibile.

Anche se la pratica della costruzione di golem di ferro è gradualmente caduta in disuso, i membri venerabili di alcune grandi civiltà del passato consideravano la capacità di forgiare golem di ferro dalla forza e dalle dimensioni sconcertanti un motivo di vanto. Questi golem (di taglia maggiore o uguale a Enorme), in alcuni angoli remoti del mondo, esistono ancora, e ancora eseguono meccanicamente ordini impartiti loro da imperi ormai scomparsi.

\textbf{Costruzione}
Per costruire un golem di ferro occorrono 2.500 kg di ferro, fuso con tinture rare del valore minimo di 10000 mo.


\medskip\index[Mostruario]{Golem di Pietra}\textbf{Golem di Pietra}

\textit{Grande costrutto, disallineato}

\textbf{FORZA} +6

\textbf{DESTREZZA} -1

\textbf{COSTITUZIONE} +5

\textbf{INTELLIGENZA} -4

\textbf{SAGGEZZA} +0

\textbf{CARISMA} -5

\textbf{Iniziativa} -1 -- \textbf{Difesa} 22

\textbf{Punti Ferita} 178 (17d10 + 85)

\textbf{Movimento} 9 m

\textbf{Tiri Salvezza}: Tempra +4, Riflessi +3, Volontà +4

\textbf{Immunità al Danno} veleno; da arma non magica o che non siano di adamantio

\textbf{Immunità alle Condizioni} affascinato, avvelenato, paralizzato, pietrificato, affaticamento, spaventato

\textbf{Sensi} scurovisione 36 m

\textbf{Linguaggi} comprende le lingue del suo creatore ma non può parlare

\textbf{Sfida} 10 (5.900 PX)

\textit{\textbf{Armi Magiche.}} Gli attacchi con armi del golem sono magici.

\textit{\textbf{Forma Immutabile.}} Il golem è immune a qualsiasi incantesimo o effetto che altererebbe la sua forma.

\textit{\textbf{Natura di Costrutto.}} Un golem non ha bisogno di aria, cibo, bevande o sonno.

\textit{\textbf{Resistenza alla Magia.}} Il golem ha +1d6 ai Tiri Salvezza contro incantesimi e altri effetti magici.

\textbf{Azioni}

\textit{\textbf{Multiattacco.}} Il golem effettua due attacchi di schianto.

\textit{\textbf{Schianto.} Attacco con arma da mischia}: +21 a colpire, portata 1 m, un bersaglio.

\textit{Colpisce:} 19 (3d8 + 6) danni da botta.

\textit{\textbf{Lentezza (Ricarica 5-6).}} Il golem prende a bersaglio una o più creature entro 3 metri da lui e che possa vedere. Ciascun bersaglio deve effettuare un Tiro Salvezza di Volontà DC 17 contro questa magia. Se fallisce il Tiro Salvezza, il bersaglio non può usare reazioni, ha la velocità dimezzata, e durante il proprio round non può effettuare più di un attacco. Inoltre, durante il proprio round il bersaglio può effettuare un'azione o un'azione bonus, ma non entrambe. Questi effetti durano per 1 minuto. Il bersaglio può ripetere il Tiro Salvezza al termine di ciascun suo round, terminando l'effetto per sé, in caso di successo.

\textbf{Ecologia}\\
Ambiente: Qualsiasi\\
Organizzazione: Solitario o gruppo (2-4)\\
\textbf{Tesoro}: Nessuno\\
\textbf{Descrizione}\\
Un golem di pietra ha un corpo umanoide fatto di pietra, spesso stilizzato per soddisfare il suo creatore. Ad esempio, può essere scolpito in modo da indossare un'armatura, con particolari simboli scolpiti sulla corazza, o avere dei disegni intarsiati nella pietra dei suoi arti. La testa è spesso scolpita per sembrare un elmo o la testa di qualche bestia. Sebbene possa essere scolpito con uno scudo o un'arma di pietra come una spada, queste scelte estetiche non influenzano le sue capacità in combattimento.

Come per la maggior parte dei golem, un golem di pietra non può parlare e non emette altro suono se non quello della pietra che sfrega sulla pietra quando si muove. Un golem di pietra è alto 2,7 metri e pesa circa 1000 kg.

Esistono numerose varianti dei Golem di Pietra, a seconda del materiali di cui sono fatti ma anche come espressioni di spiriti elementali, ovvero é possibile che uno spirito elementale abiti una "roccia" (o gemma) e ne definisca l'aspetto e lo animi come proprio corpo.

\textbf{Costruzione}
Il corpo di un golem di pietra viene scolpito da un unico blocco di pietra dura, come il granito, del peso di almeno 1.500 kg. La pietra deve essere di qualità eccezionale, e costare 5000 mo.


\medskip\index[Mostruario]{Gorgone}\textbf{Gorgone}

\textit{Grande mostruosità, disallineato}

\textbf{FORZA} +5

\textbf{DESTREZZA} +0

\textbf{COSTITUZIONE} +4

\textbf{INTELLIGENZA} -4

\textbf{SAGGEZZA} +1

\textbf{CARISMA} -2

\textbf{Iniziativa} +0 -- \textbf{Difesa} 22

\textbf{Punti Ferita} 114 (12d10 + 48)

\textbf{Movimento} 12 m

\textbf{Tiri Salvezza}: Tempra +13, Riflessi +6, Volontà +7

\textbf{Competenze} Consapevolezza +4

\textbf{Immunità alle Condizioni} Pietrificato

\textbf{Sensi} scurovisione 18 m

\textbf{Linguaggi} -

\textbf{Sfida} 5 (1.800 PX)

\textit{\textbf{Carica Travolgente.}} Se la gorgone si muove di almeno 6 metri in linea retta verso il bersaglio e lo colpisce con un attacco di incornata durante lo stesso turno, il bersaglio deve riuscire un Tiro Salvezza su Tempra DC 16 o cadere prono. Se il bersaglio è prono, la gorgone può effettuare un attacco di zoccoli contro di lui come azione bonus.

\textbf{Azioni}

\textit{\textbf{Incornata.} Attacco con arma da mischia}: +12 a colpire, portata 1 m, un bersaglio.

\textit{Colpisce:} 18 (2d12 + 5) danni perforanti.

\textit{\textbf{Zoccoli.} Attacco con arma da mischia}: +12 a colpire, portata 1 m, un bersaglio.

\textit{Colpisce:} 16 (2d10 + 5) danni da botta.

\textit{\textbf{Soffio Pietrificante (Ricarica 5-6).}} La gorgone esala un gas pietrificante in un cono di 9 metri. Ogni creatura in quell'area deve riuscire un Tiro Salvezza di Tempra DC 13. Se il Tiro Salvezza fallisce, il bersaglio inizia a trasformarsi in pietra ed è intralciato. Il bersaglio intralciato deve ripetere il Tiro Salvezza al termine del suo prossimo round. Se lo riesce, l'effetto sul bersaglio ha termine. Se lo fallisce, il bersaglio è pietrificato finché non viene liberato dall'incantesimo \textit{ripristino superiore} o simile magia.

\textbf{Ecologia}\\
Ambiente: Pianure Temperate, Colline Rocciose e Sotterranei\\
Organizzazione: Solitario, coppia, branco (3-4) o mandria (5-12)\\
\textbf{Tesoro}: Nessuno\\
\textbf{Descrizione}\\
Le gorgoni sono creature magiche e irascibili: sebbene a prima vista possano sembrare dei costrutti, sotto le piastre metalliche dall'aspetto artificiale sono fatte di carne e ossa. Come tori aggressivi, sfidano qualsiasi creatura sconosciuta che incontrano, spesso travolgendo il cadavere del loro avversario o frantumando i suoi resti pietrificati finché la creatura non è più riconoscibile. Le femmine sono pericolose quanto i maschi, e i due sessi hanno l'identico aspetto. Una tipica gorgone è alta 1,8 metri e lunga 2,4 metri. Pesa circa 2000 kg.

Le gorgoni ricavano il loro nutrimento consumando minerali, in particolare la pietra delle loro vittime pietrificate, e ogni statua da loro creata viene completamente divorata. Non possono digerire metallo o gemme, così il loro sterco (che assomiglia a polvere grigia dall'odore acre) spesso contiene piccoli cristalli grezzi e pepite d'oro. La loro aggressività verso tutte le altre creature fa sì che nei loro pascoli siano pochi, se non nessuno, i predatori e le prede. Ogni mandria è guidata da un toro dominante; le gorgoni solitarie sono generalmente tori adolescenti allontanati dalla mandria del toro dominante.

La loro carne è dura e muscolosa (una volta che viene rimossa l'armatura), e per coloro che la assaggiano è abbastanza nutriente. Molte tribù di giganti della pietra credono che mangiare la carne di gorgone aumenti la loro armatura naturale. Le corna di gorgone polverizzate valgono 250 mo come componente materiale alternativo per gli oggetti magici ed incantesimi che agiscono sulla Forza o Pietra.


\medskip\index[Mostruario]{Grick}\textbf{Grick}

\textit{Media mostruosità, neutrale}

\textbf{FORZA} +2

\textbf{DESTREZZA} +2

\textbf{COSTITUZIONE} +0

\textbf{INTELLIGENZA} -4

\textbf{SAGGEZZA} +2

\textbf{CARISMA} -3

\textbf{Iniziativa} +2 -- \textbf{Difesa} 15

\textbf{Punti Ferita} 27 (6d8)

\textbf{Movimento} 9 m, scalata 9 m

\textbf{Tiri Salvezza}: Tempra +3, Riflessi +3, Volontà +2

\textbf{Resistenza al Danno} da arma non magica

\textbf{Sensi} scurovisione 18 m

\textbf{Linguaggi} -

\textbf{Sfida} 2 (450 PX)

\textit{\textbf{Camuffamento di Pietra.}} Il grick ha +1d6 alle prove di Destrezza (Nascondersi) effettuate per nascondersi su terreno roccioso.

\textbf{Azioni}

\textit{\textbf{Multiattacco.}} Il grick effettua un attacco con i suoi tentacoli. Se l'attacco colpisce, il grick può effettuare un attacco di becco contro lo stesso bersaglio.

\textit{\textbf{Tentacoli.} Attacco con arma da mischia}: +4 a colpire, portata 1 m, un bersaglio.

\textit{Colpisce:} 9 (2d6 + 2) danni taglienti.

\textit{\textbf{Becco.} Attacco con arma da mischia}: +4 a colpire, portata 1 m, un bersaglio.

\textit{Colpisce:} 5 (1d6 + 2) danni perforanti.

\textbf{Ecologia}: \\
Ambiente: Qualsiasi Sotterraneo\\
Organizzazione: Solitario o ammasso (2-5)\\
\textbf{Tesoro}: Accidentale\\

\textbf{Descrizione}
Il vermiforme grick è il terrore delle caverne e dei cunicoli in cui abita, attendendo in agguato nei pressi di tunnel molto trafficati o città sotterranee, per balzare fuori dal buio e catturare le sue prede. È raro che tali prede vengano consumate sul posto. Il grick preferisce portare il cibo fresco nella sua tana, uno stretto cunicolo o sull'alta sporgenza di una caverna, dove può consumarlo con piccoli morsi, in tranquillità.

Le origini del grick sono ignote. E anche se ha una rudimentale intelligenza, non ha alcuna società di cui parlare, e la maggior parte delle volte si incontrano singoli esemplari. Nelle occasioni in cui i viaggiatori sfortunati ne incontrano più di uno, i gruppi di grick non sembrano comunicare o lavorare tra loro: ognuno attacca invece obiettivi individuali e si ritira col suo bottino non appena riesce ad abbattere un avversario. Predatori capaci, i grick hanno anche una strana pelle resistente alle armi che li rende particolarmente pericolosi. Molti avventurieri inesperti sono periti sotto l'attacco di un grick semplicemente perché non erano in grado di danneggiare la creatura con le loro armi non magiche. Coloro che hanno familiarità con i grick (soprattutto i Nani, i Morlock e i Trogloditi) sanno che la migliore strategia per affrontarli è quella di ritirarsi e attendere rinforzi più potenti o magici.

I grick contano sul loro colore scuro e sulla capacità di scalare i muri per tenersi fuori vista, finché non sono pronti a scattare all'attacco. In più occasioni quando il cibo scarseggia in una determinata regione, i grick si dirigono verso la superficie e vagano per il deserto in cerca di prede, ma questi soggiorni sono quasi sempre per necessità, e alla fine i grick trovano presto entrate a nuove tane sotterranee. Preferiscono le tenebre e la comodità di un "tetto" sopra la testa, evitando il cielo aperto e facendo molto per restare coperto da alberi, nuvole basse o edifici.


\medskip\index[Mostruario]{Grifone}\textbf{Grifone}

\textit{Grande mostruosità, disallineato}

\textbf{FORZA} +4

\textbf{DESTREZZA} +2

\textbf{COSTITUZIONE} +3

\textbf{INTELLIGENZA} -4

\textbf{SAGGEZZA} +1

\textbf{CARISMA} -1

\textbf{Iniziativa} +2 -- \textbf{Difesa} 13

\textbf{Punti Ferita} 59 (7d10 + 21)

\textbf{Movimento} 9 m, volo 24 m

\textbf{Tiri Salvezza}: Tempra +7, Riflessi +6, Volontà +4

\textbf{Competenze} Consapevolezza +5

\textbf{Sensi} scurovisione 18 m

\textbf{Linguaggi} -

\textbf{Sfida} 2 (450 PX)

\textit{\textbf{Vista Affinata.}} Il grifone ha +1d6 nelle prove di Saggezza (Consapevolezza) basate sulla vista.

\textbf{Azioni}

\textit{\textbf{Multiattacco.}} Il grifone effettua due attacchi: uno con il becco e uno con gli artigli.

\textit{\textbf{Artigli.} Attacco con arma da mischia}: +7 a colpire, portata 1 m, un bersaglio.

\textit{Colpisce:} 11 (2d6 + 4) danni taglienti, 1 danno da Sanguinamento.

\textit{\textbf{Becco.} Attacco con arma da mischia}: +7 a colpire, portata 1 m, un bersaglio.

\textit{Colpisce:} 8 (1d8 + 4) danni perforanti.

\textbf{Ecologia}\\
Ambiente: Colline Temperate\\
Organizzazione: Solitario, coppia o branco (6-10)\\
\textbf{Tesoro}: Accidentale\\
\textbf{Descrizione}\\
I grifoni sono potenti predatori aerei, che piombano dai loro altissimi nidi per afferrare le loro prede con il becco e gli artigli. Aggressivi e territoriali, non sono semplici bestie, bensì combattenti astuti e compagni leali verso coloro che si guadagnano il loro rispetto, combattendo fino alla morte per proteggere i loro amici e i loro simili.

Del peso di oltre 250 kg e lungo 2,4 metri, dal becco aguzzo alla coda crestata, il grifone possiede un profilo imponente che è da tempo usato in araldica e in altre iconografie come simbolo di potenza, autorità e giustizia. In realtà, il grifone è meno interessato a concetti astratti e più a cacciare per nutrirsi e difendersi. Sebbene a volte possano essere addestrati o diventino amici per servire da cavalcatura, i grifoni non possiedono un'innata affinità con gli umanoidi, e spesso entrano in sanguinosi conflitti con le razze civilizzate nel tentativo di procurarsi il loro cibo preferito: carne di cavallo. La gente di città può meravigliarsi di fronte allo stile maestoso di un grifone addestrato e alla sua apertura alare di 7 metri, ma quei contadini costretti a condividere il territorio con la sua specie sanno che conviene affrettarsi in casa e mettere al sicuro le loro greggi quando sentono le urla di caccia delle bestie.

I grifoni si accoppiano per la vita, e spesso per anni cercano vendetta per l'uccisione del compagno o di un figlio. È stata proprio per questa innata caparbietà e fiera lealtà che li ha portati nell'uso domestico come cavalcature e guardiani di tesori. Nonostante l'insito pericolo, il commercio di grifoni catturati e di uova rubate è fervido, con le uova che valgono fino a 2.000 mo l'una e i giovani vivi fino a 3000. I personaggi che desiderano un grifone come cavalcatura, però, dovrebbero sapere che comprare o addomesticare con la violenza le creature intelligenti come i grifoni è ritenuto schiavitù dalla maggior parte delle divinità buone, e guadagnarsi la spontanea lealtà di un grifone non è un compito facile. Raggiungere un mutuo accordo (o perfino l'amicizia) è una strada molto più elegante e sicura per assicurarsi un grifone come cavalcatura.

Prima che lo si possa cavalcare in combattimento, un grifone deve fare pratica nel portare il peso del suo cavaliere. Per essere ben addestrato, un grifone deve per prima cosa avere un atteggiamento amichevole verso il suo addestratore (con una prova di Addestrare Animali, Diplomazia o Intimidire). Dopodiché, 6 settimane di pratica e una prova riuscita di Addestrare Animali con DC 20 sono sufficienti perché la bestia sia a suo agio con il carico e, per la loro intelligenza, si può ritenere che i grifoni addestrati conoscano tutti i trucchi elencati nella descrizione dell'abilità Addestrare Animali, ed è anche possibile che imparino nuovi comandi, impartendo semplici richieste in Comune.

I grifoni possono portare fino a 150 kg come carico leggero, 300 kg come carico medio e 450 kg come carico pesante. Per cavalcare un grifone è necessaria una sella esotica.


\medskip\index[Mostruario]{Grimlock}\textbf{Grimlock}

\textit{Media umanoide (grimlock), neutrale malvagio}

\textbf{FORZA} +3

\textbf{DESTREZZA} +1

\textbf{COSTITUZIONE} +1

\textbf{INTELLIGENZA} -1

\textbf{SAGGEZZA} -1

\textbf{CARISMA} -2

\textbf{Iniziativa} +1 -- \textbf{Difesa} 12

\textbf{Punti Ferita} 11 (2d8 + 2)

\textbf{Movimento} 9 m

\textbf{Tiri Salvezza}: Tempra +3, Riflessi +1, Volontà +0

\textbf{Competenze} Acrobatica +5, Muoversi Silenziosamente / Nascondersi +3, Consapevolezza +3

\textbf{Immunità alle Condizioni} accecato

\textbf{Sensi} vista cieca 9 m o 3 m se assordato (cieco oltre questo raggio)

\textbf{Linguaggi} Linguaggio delle Profondità

\textbf{Sfida} 1/4 (50 PX)

\textit{\textbf{Camuffamento di Pietra.}} Il grimlock ha +1d6 alle prove di Destrezza (Nascondersi) effettuate per nascondere su terreni rocciosi.

\textit{\textbf{Sensi Ciechi.}} Il grimlock non può usare la vista cieca mentre è assordato e non più fiutare.

\textit{\textbf{Olfatto e Udito Affinati.}} Il grimlock ha +1d6 alle prove di Saggezza (Consapevolezza) basate su udito o olfatto.

\textbf{Azioni}

\textit{\textbf{Randello d'Osso Appuntito.} Attacco con arma da mischia}: +5 a colpire, portata 1 m, un bersaglio.

\textit{Colpisce:} 5 (1d4 + 3) danni da botta più 2 (1d4) danni perforanti.

\textit{\textbf{Arco Lungo.} Attacco con arma a Distanza}: +3 a colpire, gittata 45m, un bersaglio.

\textit{Colpisce:} 5 (1d8 + 1) danni perforanti.

\textbf{Ecologia}\\
I Grimlock abitano gli insediamenti abbandonati di altre Razze e sono spesso trovati come Schiavi di altre creature più organizzate, come i Duergar ed Elfi. Si ritiene che si trattino di una propaggine ancora più degenerata dei Morlock, che viaggiano da Sekamina per cacciare i Grimlock per il cibo e considerano la loro carne una delicatezza.\\
\textbf{Descrizione}\\
I Grimlock sono creature umane cieche e selvagge che abitano nel regno delle terre oscure di profondità, dove si organizzano in piccoli gruppi tribali.

\medskip\index[Mostruario]{Guardiano Protettore}\textbf{Guardiano Protettore}

\textit{Grande costrutto, disallineato}

\textbf{FORZA} +4

\textbf{DESTREZZA} -1

\textbf{COSTITUZIONE} +4

\textbf{INTELLIGENZA} -2

\textbf{SAGGEZZA} +0

\textbf{CARISMA} -4

\textbf{Iniziativa} -1 -- \textbf{Difesa} 21

\textbf{Punti Ferita} 142 (15d10 + 60)

\textbf{Movimento} 9 m

\textbf{Tiri Salvezza}: Tempra +6, Riflessi +1, Volontà +2

\textbf{Immunità al Danno} veleno

\textbf{Immunità alle Condizioni} affascinato, avvelenato, paralizzato, affaticamento, spaventato

\textbf{Sensi} scurovisione 18 m, vista cieca 3 m

\textbf{Linguaggi} comprende i comandi forniti in qualsiasi lingua ma non può parlare

\textbf{Sfida} 7 (2.900 PX)

\textit{\textbf{Accumulare Incantesimi.}} Un incantatore che indossi l'amuleto del guardiano protettore può far sì che il guardiano accumuli un incantesimo di livello 4 o più basso. Per farlo, l'incantatore deve lanciare l'incantesimo sul guardiano. L'incantesimo non ha effetto ma viene accumulato all'interno del guardiano. Quando gli viene comandato di farlo da chi indossa l'amuleto o si presenta una situazione predeterminata dall'incantatore, il guardiano lancia l'incantesimo accumulato con tutti i parametri predisposti dall'incantatore originale, senza bisogno di componenti. Quando l'incantesimo viene lanciato o qualsiasi nuovo incantesimo viene accumulato, tutti gli incantesimi precedentemente accumulati vengono persi.

\textit{\textbf{Natura di Costrutto.}} Il guardiano non ha bisogno di aria, cibo, bevande o sonno.

\textit{\textbf{Rigenerazione.}} Il guardiano protettore recupera 10 Punti Ferita all'inizio del proprio round se ne possiede ancora almeno 1.

\textit{\textbf{Vincolato.}} Il guardiano protettore è vincolato magicamente ad un amuleto. Finché il guardiano e l'amuleto sono sullo stesso piano di esistenza, chi indossa l'amuleto può richiamare telepaticamente il guardiano perché lo raggiunga, e il guardiano saprà la distanza e la direzione in cui si trova l'amuleto. Se il guardiano si trova entro 18 metri da chi indossa l'amuleto, metà dei danni subiti da chi lo indossa (arrotondati per difetto) vengono trasferiti al guardiano. Se l'amuleto viene distrutto, il guardiano è inabile finché non viene creato un amuleto di rimpiazzo. L'amuleto del guardiano può essere soggetto ad un attacco diretto qualora non sia indossato o trasportato da nessuno. Ha Difesa 10, 10 Punti Ferita e immunità ai danni da veleno. Costruire un amuleto richiede 1 settimana e costa 10000 mo in componenti.

\textbf{Azioni}

\textit{\textbf{Multiattacco.}} Il golem effettua due attacchi di pugno.

\textit{\textbf{Pugno.} Attacco con arma da mischia}: +15 a colpire, portata 1 m, un bersaglio.

\textit{Colpisce:} 11 (2d6 + 4) danni da botta.

\textbf{Reazioni}

\textit{\textbf{Scudo.}} Quando una creatura attacca chi indossa l'amuleto del guardiano, il guardiano conferisce un bonus di +2 alla sua Difesa, se entro 1 metro dal suo controllore.

\medskip\index[Mostruario]{Hobgoblin}\textbf{Hobgoblin}

\textit{Media umanoide (goblinoide), legale malvagio}

\textbf{FORZA} +1

\textbf{DESTREZZA} +1

\textbf{COSTITUZIONE} +1

\textbf{INTELLIGENZA} +0

\textbf{SAGGEZZA} +0

\textbf{CARISMA} -1

\textbf{Iniziativa} +1 -- \textbf{Difesa} 19 (armatura di maglia, scudo)

\textbf{Punti Ferita} 11 (2d8 + 2)

\textbf{Movimento} 9 m

\textbf{Tiri Salvezza}: Tempra +5, Riflessi +2, Volontà +1

\textbf{Sensi} scurovisione 18 m

\textbf{Linguaggi} Comune, Goblin

\textbf{Sfida} 1/2 (100 PX)

\textit{\textbf{+1d6 Marziale.}} Una volta per turno, l'hobgoblin può infliggere 7 (2d6) danni aggiuntivi ad una creatura che colpisce con un attacco con arma, se quella creatura si trova entro 1 metro da un alleato dell'hobgoblin che non sia inabile.

\textbf{Azioni}

\textit{\textbf{Spada Lunga.} Attacco con arma da mischia}: +3 a colpire, portata 1 m, un bersaglio.

\textit{Colpisce:} 5 (1d8 + 1) danni taglienti o 6 (1d10 + 1) danni taglienti se usata con due mani.

\textit{\textbf{Arco Lungo.} Attacco con arma a Distanza}: +3 a colpire, gittata 45m, un bersaglio.

\textit{Colpisce:} 5 (1d8 + 1) danni perforanti.

\textbf{Ecologia}\\
Ambiente: Colline Temperate\\
Organizzazione: Gruppo (4-9), banda da guerra (10-24) o tribù (25+ più 50\% non combattenti, 1 sergente di 3° livello per 20 adulti, 1 o 2 luogotenenti di 4° o 5° livello, 1 capo di 6°-8° livello, 6-12 Leopardi e 1-4 Ogre o 1-2 Troll)\\
\textbf{Tesoro}: Equipaggiamento da PNG (Corazza di Cuoio Borchiato, Scudo Leggero di Metallo, Spada Lunga, Arco Lungo con 20 Frecce, altro tesoro)\\
\textbf{Descrizione}\\
Gli Hobgoblin sono militaristi e prolifici, una combinazione che li rende molto pericolosi in alcune regioni. Procreano rapidamente, rimpiazzando i membri caduti con nuovi soldati mantenendo costante il loro numero indipendentemente dalle sorti della guerra. Generalmente basta poco perché dichiarino guerra, ma nella maggior parte dei casi il motivo è per catturare nuovi Schiavi: la vita da Schiavi in un covo di Hobgoblin è brutale e breve, e nuovi Schiavi sono sempre necessari per rimpiazzare quelli che muoiono o che vengono mangiati.

Tra tutte le Razze Goblinoidi quella degli Hobgoblin è di gran lunga la più civilizzata.
Vedono i più grandi e solitari Bugbear come strumenti da assoldare e usare dove necessario, di solito per specifiche missioni che richiedono l'omicidio e il furto, e guardano alla specie più piccola dei Goblin con un misto di vergogna e frustrazione. Gli Hobgoblin ammirano la tenacia dei Goblin, sebbene la natura imprevedibile e la passione per il fuoco dei loro minuscoli parenti li rende sgradite aggiunte a tribù o insediamenti Hobbgoblin. Tuttavia, la maggior parte delle tribù Hobgoblin include un piccolo gruppo di Goblin, che normalmente si nascondono negli angoli peggiori dell'insediamento.

Molte tribù Hobgoblin uniscono l'amore per la guerra con l'intelletto acuto. La Scienza delle macchine d'assedio, l'Alchimia e le complesse imprese di ingegneria affascinano la maggior parte degli Hobgoblin, e quelli particolarmente dotati vengono trattati da eroi e ottengono sempre delle posizioni di alto rango nella tribù. Gli Schiavi dalle menti raffinate vengono apprezzati, perciò le incursioni nelle città Naniche sono cosa ordinaria.

È risaputo che gli Hobgoblin diffidano della Magia e la disprezzano, in particolar modo quella Arcana. I loro Sciamani sono considerati con un misto di paura e rispetto, e vengono normalmente costretti a vivere da soli ai margini del covo della tribù. Non si è mai sentito di Hobgoblin che pratichino la Magia Arcana o, come dicono gli Hobgoblin, la "Magia degli Elfi". Questa è la causa del loro odio per la Magia: gli Hobgoblin odiano gli Elfi.

Un Hobgoblin è alto 1 metro e pesa 80 kg.


\medskip\index[Mostruario]{Idra}\textbf{Idra}

\textit{Enorme mostruosità, disallineato}

\textbf{FORZA} +5

\textbf{DESTREZZA} +1

\textbf{COSTITUZIONE} +5

\textbf{INTELLIGENZA} -4

\textbf{SAGGEZZA} +0

\textbf{CARISMA} -2

\textbf{Iniziativa} +1 -- \textbf{Difesa} 19

\textbf{Punti Ferita} 172 (15d12 + 75)

\textbf{Movimento} 9 m, nuoto 9 m

\textbf{Tiri Salvezza}: Tempra +8, Riflessi +7, Volontà +3

\textbf{Competenze} Consapevolezza +6

\textbf{Sensi} scurovisione 18 m

\textbf{Linguaggi} -

\textbf{Sfida} 8 (3.900 PX)

\textit{\textbf{Teste Multiple.}} L'idra ha cinque teste. Finché ha più di una testa, l'idra ha +1d6 ai Tiri Salvezza contro le condizioni accecata, affascinata, assordata, spaventata, stordita o svenuta.

Ogni volta che l'idra subisce 25 o più danni in un singolo turno, una delle sue teste muore. Se tutte le teste muoiono, anche l'idra muore.

Al termine del suo round, l'idra ricresce due teste per ciascuna delle sue teste uccise dal suo ultimo turno, a meno che non abbia subito danno da fuoco dal suo ultimo turno. L'idra recupera 10 Punti Ferita per ogni testa ricresciuta in questo modo.

\textit{\textbf{Teste Reattive.}} Per ogni testa posseduta oltre la prima, l'idra riceve una Azione di Reazione extra che può essere usata solo per compiere prove di Consapevolezza.

\textit{\textbf{Trattenere il Fiato.}} L'idra può trattenere il fiato per 1 ora.

\textit{\textbf{Veglia.}} Mentre l'idra dorme, almeno una delle sue teste resta sveglia.

\textbf{Azioni}

\textit{\textbf{Multiattacco.}} L'idra effettua tanti attacchi di morso quante sono le sue teste.

\textit{\textbf{Morso.} Attacco con arma da mischia}: +13 a colpire, portata 3 m, un bersaglio.

\textit{Colpisce:} 10 (1d10 + 5) danni perforanti.

\textbf{Ecologia}\\
Ambiente: Paludi Temperate\\
Organizzazione: Solitario\\
\textbf{Tesoro}: Standard\\
\textbf{Descrizione}\\
L'idra è un drago a più teste, ma stupido.


\medskip\index[Mostruario]{Ippogrifo}\textbf{Ippogrifo}

\textit{Grande mostruosità, disallineato}

\textbf{FORZA} +3

\textbf{DESTREZZA} +1

\textbf{COSTITUZIONE} +1

\textbf{INTELLIGENZA} -4

\textbf{SAGGEZZA} +1

\textbf{CARISMA} -1

\textbf{Iniziativa} +1 -- \textbf{Difesa} 12

\textbf{Punti Ferita} 19 (3d10 + 3)

\textbf{Movimento} 12 m, volo 18 m

\textbf{Tiri Salvezza}: Tempra +5, Riflessi +5, Volontà +2

\textbf{Competenze} Consapevolezza +5

\textbf{Linguaggi} -

\textbf{Sfida} 1 (200 PX)

\textit{\textbf{Vista Affinata.}} L'ippogrifo ha +1d6 nelle prove di Saggezza (Consapevolezza) basate sulla vista.

\textbf{Azioni}

\textit{\textbf{Multiattacco.}} L'ippogrifo effettua due attacchi: uno con il becco e uno con gli artigli.

\textit{\textbf{Artigli.} Attacco con arma da mischia}: +5 a colpire, portata 1 m, un bersaglio.

\textit{Colpisce:} 10 (2d6 + 3) danni taglienti.

\textit{\textbf{Becco.} Attacco con arma da mischia}: +5 a colpire, portata 1 m, un bersaglio.

\textit{Colpisce:} 8 (1d10 + 3) danni perforanti.

\textbf{Ecologia}\\
Ambiente: Colline Temperate o Pianure\\
Organizzazione: Solitario, coppia o stormo (7-12)\\
\textbf{Tesoro}: Nessuno\\
\textbf{Descrizione}\\
L'ippogrifo ha le ali, le zampe anteriori e la testa di un grande rapace e la coda e il corpo di un magnifico cavallo. Dato che i cavalli sono il cibo preferito dei grifoni, gli studiosi affermano che un mago con senso dell'umorismo tanto tempo fa creò come scherzo questa sfortunata fusione tra un cavallo e un falco.

Le piume dell'ippogrifo hanno una colorazione simile a quelle di un falco o di un'aquila; tuttavia, alcuni allevatori sono riusciti a produrre degli esemplari con piume completamente bianche o color carbone. Il torso di un ippogrifo e le estremità posteriori sono molto spesso di colore baio, nocciola o grigio, con alcuni manti che mostrano colorazioni pezzate o anche palomino. Un ippogrifo è lungo 3,3 metri e pesa fino a 680 kg.

I territoriali ippogrifi proteggono ferocemente il loro dominio. Gli ippogrifi devono anche sorvegliare i cieli a causa degli altri predatori, dato che sono il cibo preferito di grifoni, viverne e giovani draghi. Gli ippogrifi nidificano nelle vaste praterie erbose, aspre colline e fluenti praterie. Ippogrifi eccezionalmente resistenti stabiliscono le loro dimore all'interno di nicchie o mura di canyon, da cui setacciano i deserti rocciosi alla ricerca di coyote, cervi e a volte umanoidi. Gli ippogrifi preferiscono i mammiferi, tuttavia brucano erba dopo qualsiasi pasto di carne per aiutare la digestione. Queste loro abitudini dietetiche possono risultare pericolose sia per gli allevatori che per le loro mandrie, così spesso le comunità di allevatori mettono delle ricompense su di loro. Le vittime di queste partite di caccia vengono spesso imbalsamate, e di frequente degli ippogrifi imbalsamati decorano le taverne di frontiera e gli avamposti sperduti.

Di gran lunga più facili da addestrare rispetto ai grifoni e intelligenti quanto i cavalli, gli ippogrifi vengono addestrati come animali da monta da alcune compagnie scelte di soldati a cavallo, che pattugliano i cieli e piombano addosso ai nemici inconsapevoli. Sebbene siano bestie magiche, se catturati da giovani gli ippogrifi possono venire addestrati grazie a Addestrare Animali come fossero degli animali. Un ippogrifo adulto è molto più difficile da addestrare, e per farlo bisogna seguire le normali regole per l'addestramento delle bestie magiche utilizzando tale abilità. Una sella per ippogrifo deve venire fatta in modo tale da non intralciare i movimenti delle ali della creatura; queste selle sono sempre selle esotiche.

Gli ippogrifi sono ovipari: come regola generale, il nido di un ippogrifo contiene solo un uovo alla volta. L'uovo di ippogrifo vale 200 mo, ma un giovane ippogrifo in salute vale 500 mo. Un ippogrifo completamente addestrato come cavalcatura può veder salire il proprio valore fino a 5000 mo o anche di più. Un ippogrifo può trasportare 90 kg come carico leggero, 180 kg come carico medio e 270 kg come carico pesante.


\medskip\index[Mostruario]{Kraken}\textbf{Kraken}

\textit{Mastodontica mostruosità (titano), caotico malvagio}

\textbf{FORZA} +10

\textbf{DESTREZZA} +0

\textbf{COSTITUZIONE} +7

\textbf{INTELLIGENZA} +6

\textbf{SAGGEZZA} +4

\textbf{CARISMA} +5

\textbf{Iniziativa} +6 -- \textbf{Difesa} 30

\textbf{Punti Ferita} 472 (27x3d6 + 189)

\textbf{Movimento} 6 m, nuoto 18 m

\textbf{Tiri Salvezza}: Tempra +21, Riflessi +12, Volontà +11

\textbf{Immunità al Danno} fulmine, armi +1

\textbf{Immunità alle Condizioni} paralizzato, spaventato

\textbf{Sensi} visione del vero 36 m

\textbf{Linguaggi} comprende Abissale, Celestiale, Infernale e Druidico ma non può parlare, telepatia 36 m

\textbf{Sfida} 23 (50000 PX)

\textit{\textbf{Anfibio.}} Il kraken può respirare aria e acqua.

\textit{\textbf{Libertà di Movimento.}} Il kraken ignora i terreni difficili, e gli effetti magici non possono ridurne la velocità o far sì che diventi intralciato. Può spendere 1 metro di movimento per liberarsi dalle restrizioni non magiche o dall'essere afferrato.

\textit{\textbf{Mostro d'Assedio.}} Il kraken infligge danni doppi agli oggetti e le strutture.

\textbf{Azioni}

\textit{\textbf{Multiattacco.}} Il kraken effettua tre attacchi di tentacolo, ciascuno dei quali può essere rimpiazzato da un uso di Fiondare.

\textit{\textbf{Morso.} Attacco con arma da mischia}: +30 a colpire, portata 6 m, un bersaglio.

\textit{Colpisce:} 23 (3d8 + 10) danni perforanti. Se il bersaglio è una creatura di taglia Grande o inferiore afferrato dal kraken, quella creatura viene inghiottita, e l'afferrare ha termine. Mentre è inghiottita, la creatura è accecata e intralciata, ha copertura completa contro gli attacchi e altri effetti provenienti dall'esterno del kraken, e subisce 42 (12d6) danni da acido all'inizio di ciascun turno del kraken.

Se il kraken subisce 50 o più danni in un singolo turno da una creatura al suo interno, il kraken deve riuscire un Tiro Salvezza di Tempra DC 25 o vomitare tutte le creature inghiottite, che cadono prone in uno spazio entro 3 metri dal kraken. Se il kraken muore, una creatura inghiottita non risulta più intralciata da esso e può fuggire dal cadavere usando 5 metri di movimento, uscendo prona.

\textit{\textbf{Tentacolo.} Attacco con arma da mischia}: +30 a colpire, portata 9 m, un bersaglio.

\textit{Colpisce:} 20 (3d6 + 10) danni da botta, e il bersaglio è afferrato (DC 18 per fuggire). Fino al termine dell'afferrare, il bersaglio è intralciato. Il kraken ha dieci tentacoli, ciascuno dei quali può afferrare un bersaglio.

\textit{\textbf{Fiondare.}} Un oggetto impugnato o una creatura afferrata dal kraken, di taglia Grande o inferiore viene lanciato di 18 metri in una direzione casuale e gettata prona. Se il bersaglio lanciato colpisce una superficie solida, subisce 3 (1d6) danni da botta per ogni 3 metri percorsi. Se il bersaglio viene lanciato contro un'altra creatura, quella creatura deve riuscire un Tiro Salvezza di Riflessi DC 18 o subire lo stesso danno e cadere prona.

\textit{\textbf{Tempesta di Fulmini.}} Il kraken crea magicamente tre saette di energia, ciascuna delle quali può colpire un bersaglio entro 36 metri e che il kraken possa vedere. Il bersaglio deve effettuare un Tiro Salvezza di Riflessi DC 23, e subire 22 (4d10) danni da fulmine se fallisce il Tiro Salvezza, o la metà se lo riesce.

\textbf{Azioni Aggiuntive}

Il kraken può effettuare 3 Azioni aggiuntive, scelte tra le opzioni seguenti. Può usare solo un'opzione leggendaria alla volta e solo al termine del turno di un'altra creatura. Il kraken recupera le Azioni aggiuntive spese all'inizio del proprio round.

\textbf{Attacco di Tentacolo o Fiondare.} Il kraken effettua un attacco di tentacolo o usa Fiondare.

\textbf{Nube di Inchiostro (Costa 3 Azioni).} Mentre si trova sott'acqua, il kraken espelle una nube di inchiostro con un raggio di 18 metri. La nube si propaga intorno agli angoli, e quell'area è oscurata pesantemente per tutte le creature tranne il kraken. Ciascuna creatura a parte il kraken che termini il suo round nell'area deve riuscire un Tiro Salvezza su Tempra 23, subendo 16 (3d10) danni da veleno se fallisce il Tiro Salvezza, o la metà se lo riesce. Una forte corrente disperde la nube, che altrimenti svanisce al termine del prossimo round del kraken. \textbf{Tempesta di Fulmini (Costa 2 Azioni).} Il kraken usa Tempesta di Fulmini.

\textbf{Ecologia}\\
Ambiente: Qualsiasi Oceano\\
Organizzazione: Solitario\\
\textbf{Tesoro}: Triplo\\
\textbf{Descrizione}\\
Il leggendario kraken è una delle più grandi paure dei marinai, perché è una creatura della taglia di una balena, può colpire delle profondità senza esser visto, può comandare i venti e le condizioni meteorologiche necessarie alla nave per muoversi, e possiede il crudele intelletto della maggior parte dei più spietati e creativi criminali del mondo. Alcuni credono che i kraken siano una punizione divina, mentre altri li ritengono i veri signori delle profondità, che considerano le razze che respirano aria nient'altro che bestiame.

Molte leggende sono sorte in merito al fatto che comprenda il linguaggio druidico.

Un kraken è lungo quasi 30 metri e pesa 2000 kg.


\medskip\index[Mostruario]{Lamia}\textbf{Lamia}

\textit{Grande mostruosità, caotico malvagio}

\textbf{FORZA} +3

\textbf{DESTREZZA} +1

\textbf{COSTITUZIONE} +2

\textbf{INTELLIGENZA} +2

\textbf{SAGGEZZA} +2

\textbf{CARISMA} +3

\textbf{Iniziativa} +2 -- \textbf{Difesa} 15

\textbf{Punti Ferita} 97 (13d10 + 26)

\textbf{Movimento} 9 m

\textbf{Tiri Salvezza}: Tempra +6, Riflessi +9, Volontà +11

\textbf{Competenze} Muoversi Silenziosamente / Nascondersi +3, Ingannare +7, Percepire Emozioni +4,

\textbf{Sensi} scurovisione 18 m

\textbf{Linguaggi} Abissale, Comune

\textbf{Sfida} 4 (1.100 PX)

\textit{\textbf{Incantesimi Innati.}} La caratteristica da incantatore innato della lamia è il Carisma- La lamia può lanciare in maniera innata i seguenti incantesimi, senza bisogno di componenti materiali:

A volontà: \textit{camuffare sé stesso} (qualsiasi forma umanoide)\textit{,} \textit{immagine maggiore}

3/Giorno ciascuno: \textit{charme su persone, immagine speculare,}

\textit{scrutare, suggestione}

1/Giorno: \textit{restrizione}

\textbf{Azioni}

\textit{\textbf{Multiattacco.}} La lamia effettua due attacchi: uno con gli artigli e uno con il pugnale o il Tocco Intossicante.

\textit{\textbf{Artigli.} Attacco con arma da mischia}: +9 a colpire, portata 1 m, un bersaglio.

\textit{Colpisce:} 14 (2d10 + 3) danni taglienti, 1 danno da Sanguinamento.

\textit{\textbf{Pugnale.} Attacco con arma da mischia}: +9 a colpire, portata 1 m, un bersaglio.

\textit{Colpisce:} 5 (1d4 + 3) danni perforanti.

\textit{\textbf{Tocco Intossicante.} Attacco con incantesimo in mischia}: +5 a colpire, portata 1 m, una creatura.

\textit{Colpisce:} Il bersaglio è maledetto per 1 ora da questa magia. Fino al termine della maledizione, il bersaglio ha -1d6 ai Tiri Salvezza su Volontà e a tutte le prove di competenza.

\textbf{Ecologia}\\
Ambiente: Deserti Temperati\\
Organizzazione: Solitario, coppia o setta (3-12)\\
\textbf{Tesoro}: Doppio (Pugnale+1, altro tesoro)\\

\textbf{Descrizione}\\
Eredi piene d'odio di un'antica maledizione, le lamie hanno l'aspetto di donne snelle ed attraenti dalla cintola in su, mentre sotto hanno il corpo di un possente leone. Anche le loro fattezze umanoidi portano tratti distintivi dei felini, i loro occhi sono stretti e ferini e i loro denti somigliano alle zanne dei predatori. Una tipica lamia in piedi è alta 1,8 metri, è lunga 2,4 metri e pesa più di 325 kg.

Le lamie sono attratte da torrioni in rovina, città abbandonate e monumenti dimenticati che soddisfano i rozzi canoni estetici di queste letali cacciatrici; specie quelli in zone aride o sterili. Tuttavia, le lamie prediligono i templi decrepiti. Provano gioia nel vedere in rovina i templi di divinità buone e deviano dalla loro strada per mettere in difficoltà questi fiorenti luoghi sacri.

Le lamie vedono le femmine più anziane del loro gruppo come capi, madri e sciamane, attaccandosi a loro con fanatica reverenza. Anche se le lamie rifuggono dalla maggior parte delle religioni, vedendole come la fonte della maledizione che le ha costrette in queste forme bestiali, le lamie anziane affermano di udire i sussurri del vento che flagella il deserto e di conoscere i freddi capricci delle stelle, e fanno affidamento su queste sorgenti mistiche per guidare il loro popolo.

Le lamie presentate qui sono solo gli esponenti più comuni e meno potenti di questa razza maledetta; altre hanno forme serpentine, volanti e anche più perverse.


\medskip\index[Mostruario]{Lich}\textbf{Lich}

\textit{Media non morto, tratti malvagi}

\textbf{FORZA} +0

\textbf{DESTREZZA} +3

\textbf{COSTITUZIONE} +3

\textbf{INTELLIGENZA} +5

\textbf{SAGGEZZA} +2

\textbf{CARISMA} +3

\textbf{Iniziativa} +5 -- \textbf{Difesa} 28

\textbf{Punti Ferita} 135 (18d8 + 54)

\textbf{Movimento} 9 m

\textbf{Tiri Salvezza}: Tempra +11, Riflessi +12, Volontà +16

\textbf{Resistenze al Danno} freddo, fulmine, da Vuoto

\textbf{Immunità al Danno} veleno; da arma non magica

\textbf{Immunità alle Condizioni} affascinato, avvelenato, paralizzato, affaticamento, spaventato

\textbf{Sensi} visione del vero 36 m

\textbf{Linguaggi} Comune più altre cinque lingue

\textbf{Sfida} 21 (33000 PX)

\textit{\textbf{Incantesimi.}} Il lich ha CM 18. La sua caratteristica da incantatore è l'Intelligenza, +5 a colpire con attacchi da incantesimo). Il lich conosce i seguenti incantesimi:

Trucchetti (a volontà): \textit{mano magica, prestidigitazione, raggio} \textit{di gelo}

livello 1 (4 slot): \textit{dardo incantato, individuazione del magico,} \textit{onda tonante, scudo}

livello 2 (3 slot): \textit{freccia acida, immagine speculare,} \textit{individuazione dei pensieri, invisibilità}

livello 3 (3 slot): \textit{animare morti, controincantesimo, dissolvi} \textit{magie, palla di fuoco}

livello 4 (3 slot): \textit{inaridire, porta dimensionale}

livello 5 (3 slot): \textit{nube mortale, scrutare}

livello 6 (1 slot): \textit{disintegrazione, globo di invulnerabilità}

livello 7 (1 slot): \textit{dito della morte, spostamento planare}

livello 8 (1 slot): \textit{dominare mostri, parola del potere stordire}

livello 9 (1 slot): \textit{parola del potere uccidere}

\textit{\textbf{Natura Non Morta.}} Il lich non ha bisogno di aria, cibo, bevande o sonno.

\textit{\textbf{Resistenza Leggendaria (3/Giorno).}} Se il lich fallisce un Tiro Salvezza, può scegliere invece di riuscirvi.

\textit{\textbf{Resistenza allo Scacciare.}} Il lich ha +1d6 ai Tiri Salvezza contro gli effetti che scacciano i non morti.

\textit{\textbf{Rinvigorimento.}} Se possiede un filatterio, il lich distrutto ottiene un nuovo corpo in 1d10 giorni, recuperando tutti i suoi Punti Ferita e ritornando in attività. Il nuovo corpo compare entro 1 metro dal filatterio.

\textit{\textbf{Sacrifici di Anime.}} Un lich deve periodicamente nutrire di anime il suo filatterio per sostenere la magia che mantiene il suo corpo e la sua coscienza. Per farlo usa l'incantesimo \textit{imprigionare}. Invece di scegliere una delle normali opzioni dell'incantesimo, il lich lo impiega per intrappolare magicamente il corpo e l'anima del bersaglio all'interno del filatterio. Il filatterio deve trovarsi sullo stesso piano del lich, perché questo incantesimo funzioni. Il filatterio di un lich può contenere solo una creatura alla volta, e \textit{dissolvi magie} lanciato come incantesimo di livello 9 sul filatterio libera qualsiasi creatura imprigionata al suo interno. Una creatura imprigionata nel filatterio per 24 ore viene consumata e distrutta, dopodiché nulla salvo un intervento divino potrà riportarla in vita.

Un lich che dimentichi o non riesca a mantenere il suo corpo con le anime sacrificate inizia a cascare a pezzi, e potrebbe infine trasformarsi in un semilich.

\textbf{Azioni}

\textit{\textbf{Tocco Paralizzante.} Attacco con incantesimo in mischia}: +18 a colpire, portata 1 m, una creatura.

\textit{Colpisce:} 10 (3d6) danni da freddo. Il bersaglio deve riuscire un Tiro Salvezza di Tempra DC 18 o restare paralizzato per 1 minuto. Il bersaglio può ripetere il Tiro Salvezza al termine di ciascun suo round, terminando l'effetto su di sé in caso di successo.

\textbf{Azioni Aggiuntive}

Il lich può effettuare 3 Azioni aggiuntive, scelte tra le opzioni seguenti. Può usare solo un'opzione leggendaria alla volta e solo al termine del turno di un'altra creatura. Il lich recupera le Azioni aggiuntive spese all'inizio del proprio round.

\textit{\textbf{Distruggere Vita (Costa 3 Azioni).}} Ogni creatura ad eccezione dei non morti entro 6 metri dal lich deve effettuare un Tiro Salvezza su Tempra DC 18 contro questa magia, subendo 21 (6d6) danni da Vuoto se fallisce il Tiro Salvezza, o la metà di questi danni se lo riesce. Le creature diventano Affaticate.

\textit{\textbf{Sguardo Spaventoso (Costa 2 Azioni).}} Il lich fissa il suo sguardo su di una creatura visibile entro 3 metri da esso. Il bersaglio deve riuscire un Tiro Salvezza di Volontà DC 18 contro questa magia o restare spaventato per 1 minuto. Il bersaglio spaventato può ripetere il Tiro Salvezza al termine di ciascun suo round, terminando l'effetto su di sé in caso di successo. Se il Tiro Salvezza del bersaglio è riuscito o l'effetto per lui ha termine, il bersaglio è immune allo sguardo del lich per le successive 24 ore.

\textit{\textbf{Tocco Paralizzante (Costa 2 Azioni).}} Il lich usa il suo Tocco Paralizzante.

\textit{\textbf{Trucchetto.}} Il lich lancia un trucchetto.

\textbf{Ecologia}\\
Ambiente: Qualsiasi\\
Organizzazione: Solitario\\
\textbf{Tesoro}: Equipaggiamento da PNG (Anello di Protezione +2, Fascia della Sapienza +2 (Consapevolezza), Stivali della Levitazione, pergamena di Dominare Persone, pergamena di Teletrasporto, pozione di Invisibilità)\\

\textbf{Descrizione}
Poche creature sono più temute dei lich. Apice delle arti Necromantiche, il lich è un incantatore che ha scelto di rinunciare alla vita ed ingannare la morte diventando non morto. Anche se molti di coloro che raggiungono simili vette di potenza farebbero di tutto per raggiungere l'immortalità, l'idea di diventare un lich è aborrita da molte creature. Il processo prevede di estrarre la forza vitale dell'incantatore e imprigionarla in un filatterio preparato in modo speciale; l'incantatore cede la sua vita, ma rimane intrappolato tra la vita e la morte, e fintanto che il suo filatterio rimane intatto può continuare le sue ricerche e il suo lavoro senza temere il passare del tempo.



\medskip\index[Mostruario]{Lucertoloide}\textbf{Lucertoloide}

\textit{Media umanoide (lucertoloide), neutrale}

\textbf{FORZA} +2

\textbf{DESTREZZA} +0

\textbf{COSTITUZIONE} +1

\textbf{INTELLIGENZA} -2

\textbf{SAGGEZZA} +1

\textbf{CARISMA} -2

\textbf{Iniziativa} +0 -- \textbf{Difesa} 16 (armatura naturale, scudo)

\textbf{Punti Ferita} 22 (4d8 + 4)

\textbf{Movimento} 9 m, nuoto 9 m

\textbf{Tiri Salvezza}: Tempra +4, Riflessi +0, Volontà +0

\textbf{Competenze} Muoversi Silenziosamente / Nascondersi +4, Consapevolezza +3, Sopravvivenza +5

\textbf{Linguaggi} Draconico

\textbf{Sfida} 1/2 (100 PX)

\textit{\textbf{Trattenere il Fiato.}} Il lucertoloide può trattenere il fiato per 15 minuti.

\textbf{Azioni}

\textit{\textbf{Multiattacco.}} Il lucertoloide effettua due attacchi in mischia, ciascuno con un'arma diversa.

\textit{\textbf{Giavellotto.} Attacco con arma da mischia o a Distanza}: +4 a colpire, portata 1 m o gittata 12m, un bersaglio. \textit{Colpisce:} 5 (1d6 + 2) danni perforanti.

\textit{\textbf{Morso.} Attacco con arma da mischia}: +4 a colpire, portata 1 m, un bersaglio.

\textit{Colpisce:} 5 (1d6 + 2) danni perforanti.

\textit{\textbf{Randello Pesante.} Attacco con arma da mischia}: +4 a colpire, portata 1 m, un bersaglio.

\textit{Colpisce:} 5 (1d6 + 2) danni da botta.

\textit{\textbf{Scudo Appuntito.} Attacco con arma da mischia}: +4 a colpire, portata 1 m, un bersaglio.

\textit{Colpisce:} 5 (1d6 + 2) danni perforanti.

\textbf{Ecologia}\\
Ambiente: paludi temperate\\
Organizzazione: solitario, coppia, banda (3-12) o tribù (13-60)\\
\textbf{Tesoro}: Equipaggiamento da PNG (Scudo Pesante di Legno, Mazza chiodata, 3 Giavellotti)\\

\textbf{Descrizione}\\
I lucertoloidi sono rettili predatori orgogliosi e potenti che fanno le loro case comuni in sparuti villaggi nei recessi di paludi e acquitrini. Privi di interesse verso la colonizzazione delle terre aride e soddisfatti delle loro semplici armi e dei rituali che li hanno serviti bene per millenni, i lucertoloidi sono visti da molte delle altre razze come selvaggi retrogradi, ma all'interno delle loro isolate comunità sono in realtà un popolo vitale ricco di tradizioni e con una storia orale che risale a prima che l'uomo camminasse in posizione eretta.

La maggior parte dei lucertoloidi è alta dagli 1,8 ai 2,1 metri e pesa dai 100 ai 125 kg, ed ha i possenti muscoli coperti da scaglie grigie, verdi o marroni. Alcune razze hanno piccole creste dorsali o collari dai colori brillanti, e tutte nuotano bene spostandosi con rapidi movimenti della loro possente coda lunga 1,2 metri. Anche se sono pienamente a loro agio in acqua, trattengono il fiato e tornano alle loro abitazioni poste su colline artificiali per riprodursi e dormire. Poiché il loro sangue da rettile li rende lenti al freddo, molti lucertoloidi cacciano e lavorano durante il giorno e si ritirano nelle loro dimore di notte per rannicchiarsi con gli altri della loro tribù a condividere il calore di grandi fuochi di torba.

Anche se generalmente sono neutrali, il comportamento scostante dei lucertoloidi, il loro strenuo rifiuto dei "doni" della civilizzazione, e la leggendaria ferocia in battaglia li fa mal giudicare dalla maggioranza degli umanoidi. Questi tratti derivano da buone ragioni, tuttavia, poiché il loro basso tasso di riproduzione non ha eguali tra gli umanoidi a sangue caldo, e se le tribù non difendessero i loro territori paludosi fino all'ultimo respiro si troverebbero presto sopraffatte da orde di mammiferi. Per quanto riguarda la loro propensione a mangiare i corpi dei morti sia amici che nemici, i pratici lucertoloidi sono lesti a sottolineare che la vita è dura nella palude, e nulla deve andare sprecato.

I lucertoloidi presentati qui vivono in ambienti paludosi. Le tribù lucertoloidi possono vivere altrettanto bene in altri ambienti, ma come velocità ottengono Scalare 5 metri al posto di Nuotare.


\medskip\index[Mostruario]{Maledetti immortale}\textbf{Maledetti immortale}

\textit{Media aberrazione (umano), tendenzialmente folle}

\textbf{FORZA} +3

\textbf{DESTREZZA} +1

\textbf{COSTITUZIONE} +2

\textbf{INTELLIGENZA} -1

\textbf{SAGGEZZA} +1

\textbf{CARISMA} -2

\textbf{Iniziativa} +3 -- \textbf{Difesa} 15

\textbf{Punti Ferita} 75 (12d8 + 21)

\textbf{Movimento} 9 m

\textbf{Tiri Salvezza} Tempra +6, Riflessi +5, Volontà +5

\textbf{Resistenza al Magico} il Maledetto immortale ha +1d6 ad ogni Tiro Salvezza su incantesimi

\textbf{Competenze} Consapevolezza +3, professione che aveva in vita

\textbf{Immunità al Danno} freddo, fuoco, vuoto

\textbf{Immunità alle Condizioni} affascinato, avvelenato, pietrificato, spaventato

\textbf{Privo di sensi} il Maledetto non ha senso del gusto e olfatto

\textbf{Linguaggi} Comune, nanico, elfico

\textbf{Sfida} 4 (1.100 PX)

\textit{\textbf{Immortale}} Il Maledetto immortale rigenera 1 Punto Ferita a round, ciò gli permette di rigenerare arti e tornare in vita. L'unica possibilità di ucciderlo è sciogliendolo in acido magico. Rimuovi Maledizione a DC 30 lo uccide istantaneamente.

\textit{\textbf{Natura diversa}} Il Maledetto immortale non mangia, beve, dorme, invecchia. Non è un non morto

\textbf{Azioni}

\textit{\textbf{Multiattacco.}} Il Maledetto immortale fa tre attacchi con la spada lunga.

\textit{\textbf{Spada.} Attacco con arma da mischia}: +6 a colpire, portata 1 m, un bersaglio.

\textit{Colpisce:} 12 (1d10 + 7) danni da taglio.

\textbf{Ecologia}\\
Ambiente: Qualsiasi\\
Organizzazione: Solitario\\
\textbf{Tesoro}: Equipaggiamento da PNG (Armatura di Cuoio Borchiato, 2 Pugnali, Spada, altro soro)\\
\textbf{Descrizione}\\
Il Maledetto immortale è una persona maledetta spesso da un Patrono o da una potente incantatore con la maledizione della folle vita immortale. La maledizione rombe l'equilibro della persona e questa si ritrova a girovagare senza una meta od un obiettivo. Ogni tanto si ricordano chi erano ed allora proseguono nella ricerca di chi li ha maledetti.
Con lo scopo di farsi definitivamente uccidere si getta in ogni scontro sperando che l'avversario sia in grado di ucciderlo una volta per tutte.


\subsection{Mannari}

\medskip\index[Mostruario]{Cinghiale Mannaro}\textbf{Cinghiale Mannaro}

\textit{Media umanoide (umano, mutaforma), neutrale malvagio}

\textbf{FORZA} +3

\textbf{DESTREZZA} +0

\textbf{COSTITUZIONE} +2

\textbf{INTELLIGENZA} +0

\textbf{SAGGEZZA} +0

\textbf{CARISMA} -1

\textbf{Iniziativa} +0 -- \textbf{Difesa} 12 in forma umanoide, 13 in forma di cinghiale o ibrida

\textbf{Punti Ferita} 78 (12d8 + 24)

\textbf{Movimento} 9 m (12 m in forma di cinghiale)

\textbf{Tiri Salvezza}: Tempra +7, Riflessi +1, Volontà +4

\textbf{Competenze} Consapevolezza +2

\textbf{Immunità al Danno} da arma non magica o che non sia argentata

\textbf{Linguaggi} Comune (non può parlare in forma di cinghiale)

\textbf{Sfida} 4 (1.100 PX)

\textit{\textbf{Carica (Solo Forma di Cinghiale o Ibrida).}} Se il cinghiale mannaro si muove in linea retta di almeno 5 metri verso un bersaglio e poi lo colpisce con le zanne durante lo stesso turno, il bersaglio subisce 7 (2d6) danni taglienti aggiuntivi. Se il bersaglio è una creatura, deve riuscire un Tiro Salvezza di Tempra DC 13 o cadere prono.

\textit{\textbf{Implacabile (Ricarica dopo un 1 ora).}} Se il cinghiale mannaro subisce 14 danni o meno che lo ridurrebbero a 0 Punti Ferita, scende invece a 1 punto ferita.

\textit{\textbf{Mutaforma.}} Il cinghiale mannaro può usare la sua azione per trasformarsi in un ibrido cinghiale-umanoide o in un cinghiale, o per tornare alla sua vera forma, che è umanoide. Le sue statistiche, a parte la Difesa, sono le stesse in tutte le forme. Qualsiasi equipaggiamento stia indossando o trasportando non viene trasformato. Alla morte ritorna alla sua vera forma.

\textbf{Azioni}

\textit{\textbf{Multiattacco (Solo in Forma Umanoide o Ibrida).}} Il cinghiale mannaro effettua due attacchi, di cui solo uno può essere con le zanne.

\textit{\textbf{Maglio (Soltanto in Forma Umanoide o Ibrida).} Attacco con arma da mischia}: +9 a colpire, portata 1 m, un bersaglio. \textit{Colpisce:} 10 (2d6 + 3) danni da botta.

\textit{\textbf{Zanne (Soltanto in Forma di Cinghiale o Ibrida).} Attacco con arma da mischia}: +9 a colpire, portata 1 m, un bersaglio. \textit{Colpisce:} 10 (2d6 + 3) danni taglienti. Se il bersaglio è un umanoide, deve riuscire un Tiro Salvezza di Tempra DC 12 o venire maledetto dalla licantropia del cinghiale mannaro.

\textbf{Ecologia}\\
Ambiente: Qualsiasi Foresta o Pianura\\
Organizzazione: Solitario, coppia, famiglia (3-8) o truppa (3-8 più 1-4 Cinghiali)\\
\textbf{Tesoro}: Equipaggiamento da PNG (Armatura di Cuoio Borchiato, 2 Pugnali, altro tesoro)\\
\textbf{Descrizione}\\
Nella loro forma umanoide, i cinghiali mannari tendono a essere tozzi, con nasi all'insù, pelo ispido e incisivi prominenti. Hanno capelli rossi, castani o neri ma alcuni sono anche biondi, canuti o calvi. Hanno di norma peluria sul labbro superiore, e i maschi di solito non riescono a far crescere la barba. Poiché sono testardi e aggressivi, hanno piccole comunità di loro simili e non si mischiano ai non licantropi: di solito vivono in piccole fattorie dall'aspetto assolutamente normale. Tendono ad avere grandi famiglie e molti figli.


\medskip\index[Mostruario]{Lupo Mannaro}\textbf{Lupo Mannaro}

\textit{Media umanoide (umano, mutaforma), caotico malvagio}

\textbf{FORZA} +2

\textbf{DESTREZZA} +1

\textbf{COSTITUZIONE} +2

\textbf{INTELLIGENZA} +0

\textbf{SAGGEZZA} +0

\textbf{CARISMA} +0

\textbf{Iniziativa} +1 -- \textbf{Difesa} 13 in forma umanoide, 14 in forma di lupo o ibrida

\textbf{Punti Ferita} 58 (9d8 + 18)

\textbf{Movimento} 9 m (12 m in forma di lupo)

\textbf{Tiri Salvezza}: Tempra +5, Riflessi +1, Volontà +2

\textbf{Competenze} Muoversi Silenziosamente / Nascondersi +3, Consapevolezza +4

\textbf{Immunità al Danno} da arma non magica o che non sia argentata

\textbf{Linguaggi} Comune (non può parlare in forma di lupo)

\textbf{Sfida} 3 (700 PX)

\textit{\textbf{Mutaforma.}} Il lupo mannaro può usare la sua azione per trasformarsi in un ibrido lupo-umanoide o in un lupo, o per tornare alla sua vera forma, che è umanoide. Le sue statistiche, a parte la Difesa, sono le stesse in tutte le forme. Qualsiasi equipaggiamento stia indossando o trasportando non viene trasformato. Alla morte ritorna alla sua vera forma.

\textit{\textbf{Udito e Olfatto Affinato.}} Il lupo mannaro ha +1d6 nelle prove di Saggezza (Consapevolezza) basate su udito o olfatto.

\textbf{Azioni}

\textit{\textbf{Multiattacco (Soltanto in Forma Umanoide o Ibrida).}} Il lupo mannaro effettua due attacchi: uno con il morso e uno con gli artigli o la lancia.

\textit{\textbf{Artigli (Soltanto in Forma Ibrida).} Attacco con arma da mischia}: +6 a colpire, portata 1 m, una creatura. \textit{Colpisce:} 7 (2d4 + 2) danni taglienti.

\textit{\textbf{Lancia (Soltanto in Forma Umanoide).} Attacco con arma da mischia o a Distanza}: +4 a colpire, portata 1 m o gittata 6m, una creatura.

\textit{Colpisce:} 5 (1d6 + 2) danni perforanti o 6 (1d8 + 2) danni perforanti se usata con due mani in un attacco di mischia.

\textit{\textbf{Morso (Soltanto in Forma di Lupo o Ibrida).} Attacco con arma da mischia}: +6 a colpire, portata 1 m, un bersaglio.

\textit{Colpisce:} 6 (1d8 + 2) danni perforanti. Se il bersaglio è un umanoide, deve riuscire un Tiro Salvezza di Tempra DC 12 o venir maledetto dalla licantropia del lupo mannaro.

\textbf{Ecologia}\\
Ambiente: Qualsiasi Terreno\\
Organizzazione: Solitario, coppia o branco (3-6)\\
\textbf{Tesoro}: Equipaggiamento da PNG (Cotta di Maglia, Spada Lunga, Balestra Leggera con 20 Quadrelli, altro tesoro)\\
\textbf{Descrizione}\\
Nella forma umana i lupi mannari somigliano a persone normali, anche se alcuni tendono ad avere un aspetto ferino e capelli ribelli. Sopracciglia che crescono unendosi, dito indice più lungo del medio e strane voglie sul palmo della mano sono tutti segni comunemente accettati che una persona sia in realtà un lupo mannaro. Naturalmente, questi segni rivelatori non sono sempre accurati, perché questi tratti fisici esistono anche nelle persone normali, ma nelle zone dove i lupi mannari sono un problema comune, questi tratti possono essere considerati schiaccianti a prescindere.

\medskip\index[Mostruario]{Orso Mannaro}\textbf{Orso Mannaro}

\textit{Media umanoide (umano, mutaforma), neutrale buono}

\textbf{FORZA} +4

\textbf{DESTREZZA} +0

\textbf{COSTITUZIONE} +3

\textbf{INTELLIGENZA} +0

\textbf{SAGGEZZA} +1

\textbf{CARISMA} +1

\textbf{Iniziativa} +0 -- \textbf{Difesa} 13 in forma umanoide, 14

in forma di orso o ibrida

\textbf{Punti Ferita} 135 (18d8 + 54)

\textbf{Movimento} 9 m (12 m, scalata 9 m in forma di orso o forma ibrida)

\textbf{Tiri Salvezza}: Tempra +5, Riflessi +6, Volontà +2

\textbf{Competenze} Consapevolezza +7

\textbf{Immunità al Danno} da arma non magica o che non sia argentata

\textbf{Linguaggi} Comune (non può parlare in forma di orso)

\textbf{Sfida} 5 (1.800 PX)

\textit{\textbf{Mutaforma.}} L'orso mannaro può usare la sua azione per trasformarsi in un ibrido orso-umanoide o in un orso, o per tornare alla sua vera forma, che è umanoide. Le sue statistiche, a parte la Difesa, sono le stesse in tutte le forme. Qualsiasi equipaggiamento stia indossando o trasportando non viene trasformato. Alla morte ritorna alla sua vera forma.

\textit{\textbf{Olfatto Affinato.}} L'orso mannaro ha +1d6 nelle prove di Saggezza (Consapevolezza) basate sull'olfatto.

\textbf{Azioni}

\textit{\textbf{Multiattacco.}} In forma di orso, l'orso mannaro effettua due attacchi di artiglio. In forma umanoide, effettua due attacchi di ascia bipenne. In forma ibrida, può attaccare come un orso o un umanoide.

\textit{\textbf{Artiglio (Soltanto in Forma di Orso o Ibrida).} Attacco con arma da mischia}: +11 a colpire, portata 1 m, un bersaglio. \textit{Colpisce:} 13 (2d8 + 2) danni taglienti.

\textit{\textbf{Ascia Bipenne (Soltanto in Forma Umanoide o Ibrida).} Attacco con arma da mischia}: +11 a colpire, portata 1 m, un bersaglio. \textit{Colpisce:} 10 (1d12 + 4) danni taglienti.

\textit{\textbf{Morso (Soltanto in Forma di Orso o Ibrida).} Attacco con arma da mischia}: +11 a colpire, portata 1 m, un bersaglio.

\textit{Colpisce:} 15 (2d10 + 4) danni perforanti. Se il bersaglio è un umanoide, deve riuscire un Tiro Salvezza di Tempra DC 14 o venir maledetto dalla licantropia dell'orso mannaro.


\textbf{Ecologia}\\
Ambiente: Qualsiasi Foresta\\
Organizzazione: Solitario, coppia, famiglia (3-6) o truppa (3-6 più 1-4 orsi Neri o Grigi)\\
\textbf{Tesoro}: Equipaggiamento da PNG (Giaco di Maglia, Ascia da Battaglia Perfetta, 2 Asce da Lancio Perfette, altro tesoro)\\
\textbf{Descrizione}\\
Nelle loro forme umanoidi, gli orsi mannari tendono a essere muscolosi e con spalle larghe, tratti aspri e occhi scuri. Hanno capelli rossi, castani o neri e sembrano abituati a una vita di duro lavoro. Anche se i più benigni fra i licantropi, sono evitati dalla maggior parte delle persone normali, che temono la loro trasformazione animalesca. Per la maggior parte vivono in zone boschive isolate o in piccole unità familiari della loro stessa specie. Evitano di affrontare gli stranieri, ma non esitano se devono scacciare umanoidi malvagi dai loro territori.

\medskip\index[Mostruario]{Ratto Mannar}\textbf{Ratto Mannaro}

\textit{Media umanoide (umano, mutaforma), legale malvagio}

\textbf{FORZA} +0

\textbf{DESTREZZA} +2

\textbf{COSTITUZIONE} +1

\textbf{INTELLIGENZA} +0

\textbf{SAGGEZZA} +0

\textbf{CARISMA} -1

\textbf{Iniziativa} +2 -- \textbf{Difesa} 13

\textbf{Punti Ferita} 33 (6d8 + 6)

\textbf{Movimento} 9 m

\textbf{Tiri Salvezza}: Tempra +2, Riflessi +5, Volontà +3

\textbf{Competenze} Muoversi Silenziosamente / Nascondersi +4, Consapevolezza +2

\textbf{Immunità al Danno} da arma non magica o che non sia argentata

\textbf{Sensi} scurovisione 18 m (solo in forma di ratto)

\textbf{Linguaggi} Comune (non può parlare in forma di ratto)

\textbf{Sfida} 2 (450 PX)

\textit{\textbf{Mutaforma.}} Il ratto mannaro può usare la sua azione per trasformarsi in un ibrido ratto-umanoide o in un ratto, o per tornare alla sua vera forma, che è umanoide. Le sue statistiche, a parte la Difesa, sono le stesse in tutte le forme. Qualsiasi equipaggiamento stia indossando o trasportando non viene trasformato. Alla morte ritorna alla sua vera forma.

\textit{\textbf{Olfatto Affinato.}} Il ratto mannaro ha +1d6 nelle prove di Saggezza (Consapevolezza) basate sull'olfatto.

\textbf{Azioni}

\textit{\textbf{Multiattacco (Solo in Forma Umanoide o Ibrida).}} Il ratto mannaro effettua due attacchi, di cui solo uno può essere con il morso.

\textit{\textbf{Spada Corta (Soltanto in Forma Umanoide o Ibrida).} Attacco con arma da mischia}: +4 a colpire, portata 1 m, un bersaglio. \textit{Colpisce:} 5 (1d6 + 2) danni perforanti.

\textit{\textbf{Balestra a mano (Soltanto in Forma Umanoide o Ibrida).} Attacco con arma a Distanza}: +4 a colpire, gittata 9m, un bersaglio.

\textit{Colpisce:} 5 (1d6 + 2) danni perforanti.

\textit{\textbf{Morso (Soltanto in Forma di Ratto o Ibrida).} Attacco con arma da mischia}: +4 a colpire, portata 1 m, un bersaglio.

\textit{Colpisce:} 4 (1d4 + 2) danni perforanti. Se il bersaglio è un umanoide, deve riuscire un Tiro Salvezza di Tempra DC 11 o venir maledetto dalla licantropia del ratto mannaro.

\textbf{Ecologia}\\
Ambiente: Qualsiasi Urbano\\
Organizzazione: Solitario, coppia, branco (5-10) o gilda (11-30 più 5-12 Ratti Crudeli)\\
\textbf{Tesoro}: Equipaggiamento da PNG (Armatura di Cuoio Borchiato Perfetta, Spada Corta, Balestra Leggera con 20 Quadrelli, altro tesoro)\\
\textbf{Descrizione}\\
I ratti mannari naturali sono bassi, asciutti e muscolosi, con occhi attenti e vispi, e hanno movimenti nervosi. I maschi spesso hanno sottili baffi striminziti.

\medskip\index[Mostruario]{Tigre Mannara}\textbf{Tigre Mannara}

\textit{Media umanoide (umano, mutaforma), neutrale}

\textbf{FORZA} +3

\textbf{DESTREZZA} +2

\textbf{COSTITUZIONE} +3

\textbf{INTELLIGENZA} +0

\textbf{SAGGEZZA} +1

\textbf{CARISMA} +0

\textbf{Iniziativa} +2 -- \textbf{Difesa} 14

\textbf{Punti Ferita} 120 (16d8 + 48)

\textbf{Movimento} 9 m (12 m in forma di tigre)

\textbf{Tiri Salvezza}: Tempra +2, Riflessi +7, Volontà +4

\textbf{Competenze} Muoversi Silenziosamente / Nascondersi +4, Consapevolezza +5

\textbf{Immunità al Danno} da arma non magica che non siano argentati

\textbf{Sensi} scurovisione 18 m

\textbf{Linguaggi} Comune (non può parlare in forma di tigre)

\textbf{Sfida} 4 (1.1100 PX)

\textit{\textbf{Balzo.}} Se la tigre mannara si muove di almeno 5 metri in linea retta verso una creatura e la colpisce con un attacco di artiglio durante lo stesso turno, il bersaglio deve riuscire un Tiro Salvezza su Tempra DC 14 o cadere prono. Se il bersaglio è prono, la tigre mannara può effettuare un attacco di morso contro di esso come azione bonus.

\textit{\textbf{Mutaforma.}} La tigre mannara può usare la sua azione per trasformarsi in un ibrido tigre-umanoide o in una tigre, o per tornare alla sua vera forma, che è umanoide. Le sue statistiche, a parte la Difesa, sono le stesse in tutte le forme. Qualsiasi equipaggiamento stia indossando o trasportando non viene trasformato. Alla morte ritorna alla sua vera forma.

\textit{\textbf{Olfatto e Udito Affinato.}} La tigre mannara ha +1d6 nelle prove di Saggezza (Consapevolezza) basate su olfatto e udito.

\textbf{Azioni}

\textit{\textbf{Multiattacco (Solo in Forma Umanoide o Ibrida).}} In forma umanoide, la tigre mannara effettua due attacchi di scimitarra o due attacchi di arco lungo. In forma ibrida, può attaccare come un umanoide o effettuare due attacchi di artiglio.

\textit{\textbf{Artiglio (Soltanto in Forma di Tigre o Ibrida).} Attacco con arma da mischia}: +9 a colpire, portata 1 m, un bersaglio. \textit{Colpisce:} 7 (1d8 + 3) danni taglienti, 1 danno da Sanguinamento.

\textit{\textbf{Morso (Soltanto in Forma di Tigre o Ibrida).} Attacco con arma da mischia}: +9 a colpire, portata 1 m, un bersaglio.

\textit{Colpisce:} 8 (1d10 + 3) danni perforanti. Se il bersaglio è un umanoide, deve riuscire un Tiro Salvezza di Tempra DC 13 o venir maledetto dalla licantropia della tigre mannara.

\textit{\textbf{Scimitarra (Soltanto in Forma Umanoide o Ibrida).} Attacco con arma da mischia}: +9 a colpire, portata 1 m, un bersaglio. \textit{Colpisce:} 6 (1d6 + 3) danni taglienti.

\textit{\textbf{Arco Lungo (Soltanto in Forma Umanoide o Ibrida).} Attacco con arma a Distanza}: +8 a colpire, gittata 45m, un bersaglio.

\textit{Colpisce:} 6 (1d8 + 2) danni perforanti.

\textbf{Ecologia}
Ambiente: Qualsiasi Pianura o Palude\\
Organizzazione: Solitario o coppia\\
\textbf{Tesoro}: Equipaggiamento da PNG (Armatura di Cuoio Borchiato, Spada Corta, 2 Pugnali, altro tesoro)\\
\textbf{Descrizione}\\
Le tigri mannare in forma umanoide hanno grandi occhi, nasi allungati, zigomi sporgenti e capelli castani o rossi, oppure bianchi, neri o grigio-blu. I loro movimenti sono attenti e aggraziati, e chi li guarda potrebbe scambiarli per un ottimo tagliaborse, un danzatore aggraziato o un'abile cortigiana.


\medskip\index[Mostruario]{Manticora}\textbf{Manticora}

\textit{Grande mostruosità, legale malvagio}

\textbf{FORZA} +3

\textbf{DESTREZZA} +3

\textbf{COSTITUZIONE} +3

\textbf{INTELLIGENZA} -2

\textbf{SAGGEZZA} +1

\textbf{CARISMA} -1

\textbf{Iniziativa} +3 -- \textbf{Difesa} 16

\textbf{Punti Ferita} 68 (8d10 + 24)

\textbf{Movimento} 9 m, volo 15 m

\textbf{Tiri Salvezza}: Tempra +9, Riflessi +7, Volontà +3

\textbf{Sensi} scurovisione 18 m

\textbf{Linguaggi} Comune

\textbf{Sfida} 3 (700 PX)

\textit{\textbf{Ricrescere Spine della Coda.}} La manticora possiede ventiquattro spine nella coda. Le spine usate ricrescono all'alba.

\textbf{Azioni}

\textit{\textbf{Multiattacco.}} La manticora effettua tre attacchi: uno con il morso e due con gli artigli o tre con le spine della coda.

\textit{\textbf{Artiglio.} Attacco con arma da mischia}: +7 a colpire, portata 1 m, un bersaglio.

\textit{Colpisce:} 6 (1d6 + 3) danni taglienti, 1 danno da Sanguinamento.

\textit{\textbf{Morso.} Attacco con arma da mischia}: +7 a colpire, portata 1 m, un bersaglio.

\textit{Colpisce:} 7 (1d8 + 3) danni perforanti.

\textit{\textbf{Spine della Coda.} Attacco con arma a Distanza}: +7 a colpire, gittata 30m, un bersaglio.

\textit{Colpisce:} 7 (1d8 + 3) danni perforanti.

\textbf{Ecologia}
Ambiente: Colline e Paludi Calde\\
Organizzazione: Solitario, coppia o branco (3-6)\\
\textbf{Tesoro}: Standard\\

\textbf{Descrizione}\\
Le manticore sono feroci predatori che controllano vaste aree in cerca di carne fresca. Una tipica manticora è lunga circa 3 metri e pesa circa 500 kg. Alcune hanno la testa simile a quella di un umano, in genere barbuto. Maschi e femmine sono molto simili.

Le manticore mangiano qualsiasi tipo di carne, anche quella delle carogne, ma preferiscono quella umana e raramente si lasciano sfuggire un'occasione di gustare questa delizia. Sono abbastanza furbe e sociali da stringere patti con umanoidi malvagi per formare alleanze o da costringerli ad offre tributi, e molte creature potenti le incaricano di sorvegliare o controllare un posto o una zona. Prediligono fare le loro tane in posti alti, come le sommità delle colline e le caverne tra le rupi.

Anche se le manticore sono simili a delle creazioni magiche, sono da tempo annoverate tra le specie naturali. Curiosamente, le manticore sembrano stranamente feconde e possono incrociarsi con numerose altre specie dalla forma simile, inclusi Leoni, Tigri, Lamie, Sfingi e Chimere.

\medskip\index[Mostruario]{Manto Assassino}\textbf{Manto Assassino}

\textit{Grande aberrazione, caotico neutrale}

\textbf{FORZA} +3

\textbf{DESTREZZA} +2

\textbf{COSTITUZIONE} +1

\textbf{INTELLIGENZA} +1

\textbf{SAGGEZZA} +1

\textbf{CARISMA} +2

\textbf{Iniziativa} +2 -- \textbf{Difesa} 18

\textbf{Punti Ferita} 78 (12d10 + 12)

\textbf{Movimento} 3 m, volo 12 m

\textbf{Tiri Salvezza}: Tempra +6, Riflessi +5, Volontà +7

\textbf{Competenze} Muoversi Silenziosamente / Nascondersi +5

\textbf{Sensi} scurovisione 18 m

\textbf{Linguaggi} Linguaggio delle Profondità

\textbf{Sfida} 8 (3.900 PX)

\textit{\textbf{Falso Aspetto.}} Mentre il manto assassino resta immobile senza esporre la parte inferiore del corpo, è indistinguibile da un manto di pelle nera.

\textit{\textbf{Sensibilità alla Luce}}. Mentre è alla luce del sole, il manto assassino ha -1d6 ai tiri per colpire, oltre che alle prove di Saggezza (Consapevolezza) basate sulla vista.

\textit{\textbf{Trasferimento di Danno.}} Mentre è appiccicato ad una creatura, il manto assassino subisce solo la metà dei danni che gli sono inferti (arrotondare per difetto), e la creatura vittima del manto assassino subisce l'altra metà.

\textbf{Azioni}

\textit{\textbf{Multiattacco.}} Il manto assassino effettua due attacchi:
uno con il morso e uno con la coda.

\textit{\textbf{Morso.} Attacco con arma da mischia}: +12 a colpire, portata 1 m, una creatura.

\textit{Colpisce:} 10 (2d6 + 3) danni perforanti, e se il bersaglio è di taglia Grande o inferiore, il manto assassino vi si appiccica. Se il manto assassino ha +1d6 contro il bersaglio, si appiccica alla sua testa e il bersaglio è accecato e impossibilitato a respirare finché il manto assassino vi rimane appiccicato. Mentre appiccicato il manto assassino può effettuare questo attacco solo  contro il bersaglio e ha +1d6 al tiro per colpire. Il manto  assassino può staccarsi spendendo 1 metro di movimento. Una  creatura, compreso il bersaglio, può effettuare la sua azione per  staccare il manto assassino riuscendo una prova di Forza DC 16.


\textit{\textbf{Coda.} Attacco con arma da mischia}: +12 a colpire, portata 3 m, una creatura.

\textit{Colpisce:} 7 (1d8 + 3) danni taglienti.

\textit{\textbf{Apparizioni (Ricarica dopo un 1 ora).}} Qualora non si trovi sotto luce intensa, il manto assassino crea tre duplicati illusori di sé stesso, che si muovono assieme ad esso e ne imitano le azioni, scambiandosi di posizione per rendere impossibile capire quale sia il reale manto assassino. Se l'originale si trova in un'area di luce intensa, i duplicati svaniscono.

Ogniqualvolta una creatura prenda a bersaglio il manto assassino con un attacco o un incantesimo nocivo mentre sono ancora presenti dei duplicati, quella creatura determina casualmente se prende a bersaglio il manto assassino o uno dei duplicati. Una creatura che non possa vedere o che si affida a sensi diversi dalla vista ignora questo effetto magico.

Un duplicato possiede la Difesa e usa i Tiri Salvezza del manto assassino. Se un attacco colpisce un duplicato, o se un duplicato fallisce un Tiro Salvezza contro un effetto che infligge danni, svanisce.

\textit{\textbf{Gemito.}} Ogni creatura entro 18 metri dal manto assassino, che possa udire il suo gemito e che non sia un'aberrazione, deve riuscire un Tiro Salvezza su Volontà DC 13 o essere spaventata fino al termine del prossimo round del manto assassino. Se il Tiro Salvezza della creatura
riesce, la creatura è immune al gemito del manto assassino per le successive 24 ore.

\textbf{Ecologia}
Ambiente: Sotterranei\\
Organizzazione: Solitario, coppia, schiera (3-6) o stormo (7-12)\\
\textbf{Tesoro}: Standard\\
\textbf{Descrizione}\\
Simili a mante volanti orribilmente malvagie, i manti assassini sono creature misteriose e paranoiche. Un tipico esemplare ha un'apertura alare di 2,4 metri e pesa 50 kg.

Le loro motivazioni sono misteriose e confuse, e diffidano perfino dei loro simili. La strana forma permette loro di essere scambiati per mantelli, arazzi o altri oggetti comuni, e alcune storie narrano di manti assassini che si alleano con altre creature, facendosi trasportare sulla loro schiena e contribuendo alla protezione dei loro alleati per ragioni imperscrutabili. Alcuni esemplari sono sacerdoti di antiche divinità, al comando di culti di manti assassini e Skum intenti a celebrare orribili riti dagli scopi sinistri.


\medskip\index[Mostruario]{Mantoscuro}\textbf{Mantoscuro}

\textit{Piccola mostruosità, disallineato}

\textbf{FORZA} +3

\textbf{DESTREZZA} +1

\textbf{COSTITUZIONE} +1

\textbf{INTELLIGENZA} -4

\textbf{SAGGEZZA} +0

\textbf{CARISMA} -3

\textbf{Iniziativa} +1 -- \textbf{Difesa} 12

\textbf{Punti Ferita} 22 (5d6 + 5)

\textbf{Movimento} 3 m, volo 9 m

\textbf{Tiri Salvezza}: Tempra +5, Riflessi +3, Volontà +0

\textbf{Competenze} Muoversi Silenziosamente / Nascondersi +3

\textbf{Sensi} vista cieca 18 m

\textbf{Linguaggi} -

\textbf{Sfida} 1/2 (100 PX)

\textit{\textbf{Ecolocazione.}} Il mantoscuro non può usare la vista cieca se assordato.

\textit{\textbf{Falso Aspetto.}} Mentre il mantoscuro rimane immobile, è indistinguibile da una formazione rocciosa come una stalattite o una stalagmite.

\textbf{Azioni}

\textit{\textbf{Spaccare.} Attacco con arma da mischia}: +5 a colpire, portata 1 m, una creatura.

\textit{Colpisce:} 6 (1d6 + 3) danni da botta e il mantoscuro si appiccica alla creatura. Se il bersaglio è di taglia Media o inferiore il mantoscuro ha +1d6 al tiro per colpire, si appiccica avvolgendo la testa del bersaglio, che è accecato e impossibilitato a respirare finché il mantoscuro resta appiccicato in questo modo.

Mentre è appiccicato al bersaglio, il mantoscuro non può attaccare nessun'altra creatura salvo il bersaglio, ma ha +1d6 ai suoi tiri per colpire. La velocità del mantoscuro diventa 0 e non può trarre beneficio da nessun bonus alla velocità, muovendosi assieme al bersaglio.

Una creatura può staccare il mantoscuro con un'azione e riuscendo una prova di Forza DC 13. Durante il suo round, il mantoscuro può staccarsi dal bersaglio da solo usando 1 metro di movimento.

\textit{\textbf{Aura di Oscurità (1/Giorno).}} Un'oscurità magica con 5 metri di raggio si estende dal mantoscuro, muovendosi con esso, e propagandosi oltre gli angoli. L'oscurità permane finché il mantoscuro mantiene la concentrazione, massimo 10 minuti (come se si stesse concentrando su di un incantesimo). La scurovisione non può penetrare questa oscurità, né essa può essere rischiarata da alcuna luce naturale. Se qualsiasi parte dell'oscurità si sovrappone ad un'area di luce generata da un incantesimo di livello 2 o inferiore, l'incantesimo che sta creando la luce viene dissolto.

\textbf{Ecologia}
Ambiente: Qualsiasi (sotterraneo)\\
Organizzazione: Solitario, coppia o nidiata (3-12)\\
\textbf{Tesoro}: Nessuno\\
\textbf{Descrizione}\\
l'apertura tentacolare di un mantoscuro ha un'ampiezza di poco inferiore agli 1 m; quando è appeso alla volta di una caverna, mascherato da stalattite, la sua lunghezza varia tra i 60 ed i 90 cm. Un esemplare tipico di mantoscuro pesa 20 kg. La testa ed il corpo della creatura sono solitamente del colore del basalto o del granito scuro, ma i suoi tentacoli membranosi possono cambiare colore per adattarsi all'ambiente circostante.

I mantoscuro non sono scalatori particolarmente abili, ma sono in grado di appendersi alla volta di una caverna come i pipistrelli, agganciati per mezzo degli uncini posti in fondo ai loro tentacoli, così che il loro corpo penzolante risulti quasi indistinguibile da una stalattite. Da questa postazione nascosta la creatura attende che la preda passi sotto di lei e, a questo punto, si stacca lanciandosi verso di essa, sbattendo contro il bersaglio e tentando di avvolgervi attorno i suoi membranosi tentacoli. Se il mantoscuro manca la preda, risale e si lancia nuovamente contro la preda, fino a quando quest'ultima non viene sconfitta o il mantoscuro è gravemente ferito (nel qual caso svolazza sul soffitto per nascondersi, sperando che la sua "preda" lo lasci perdere). La capacità innata di questa creatura di celare la zona circostante per mezzo dell'oscurità magica le offre un ulteriore vantaggio contro gli avversari che necessitano della luce per vedere.

I mantoscuro preferiscono vivere e cacciare nelle caverne e nei cunicoli più vicini alla superficie, dal momento che questi offrono un più frequente passaggio di prede che questi mostri possono cacciare. Non si limitano però a queste caverne buie e talvolta possono essere incontrati in fortezze abbandonate o persino nelle fogne delle città affollate. Qualsiasi luogo dove abbondi il cibo e ci sia un soffitto a cui appendersi è un possibile covo per un mantoscuro.

Il ciclo vitale di un mantoscuro è rapido: i piccoli diventano adulti nell'arco di pochi mesi e la maggior parte muore di vecchiaia dopo pochi anni. Di conseguenza le generazioni di mantoscuro si susseguono rapidamente e nel corso degli anni l'evoluzione di queste creature è altrettanto rapida. Per questa ragione l'ecosistema di una caverna può avere effetti importanti sull'aspetto, le capacità e le tattiche di un mantoscuro. In caverne acquatiche possono svilupparsi mantoscuri in grado di nuotare, mentre creature che abitano luoghi soggetti a vulcanismo potrebbero sviluppare una specifica resistenza al fuoco. Altre varianti di mantoscuro potrebbero avere pelli più resistenti ed invece di cadere per stritolare la preda potrebbero semplicemente gettarsi cercando di trafiggerla analogamente a vere e proprie stalattiti. Si mormora che le caverne più oscure e profonde nascondano mantoscuri di dimensioni incredibili, in grado di soffocare contemporaneamente diversi bersagli di dimensioni umane nel loro abbraccio avvolgente.


\medskip\index[Mostruario]{Medusa}\textbf{Medusa}

\textit{Media mostruosità, legale malvagio}

\textbf{FORZA} +0

\textbf{DESTREZZA} +2

\textbf{COSTITUZIONE} +3

\textbf{INTELLIGENZA} +1

\textbf{SAGGEZZA} +1

\textbf{CARISMA} +2

\textbf{Iniziativa} +2 -- \textbf{Difesa} 18

\textbf{Punti Ferita} 127 (17d8 + 51)

\textbf{Movimento} 9 m

\textbf{Tiri Salvezza}: Tempra +6, Riflessi +8, Volontà +7

\textbf{Competenze} Muoversi Silenziosamente / Nascondersi +5, Ingannare +5, Percepire Emozioni +4, Consapevolezza +4

\textbf{Sensi} scurovisione 18 m

\textbf{Linguaggi} Comune

\textbf{Sfida} 6 (2.300 PX)

\textit{\textbf{Sguardo Pietrificante.}} Se una creatura comincia il suo round entro 9 metri da una medusa di cui possa vedere gli occhi, la medusa, qualora la non sia inabile e possa vedere a sua volta la creatura, può obbligarla ad effettuare un Tiro Salvezza di Tempra DC 14. Se la creatura fallisce il Tiro Salvezza di 5 o più, viene pietrificata all'istante, altrimenti inizia magicamente a trasformarsi in pietra ed è intralciata. La creatura intralciata deve ripetere il Tiro Salvezza al termine del suo prossimo round. Se lo riesce, l'effetto termina. Se lo fallisce, la creatura è pietrificata finché non viene liberata dall'incantesimo \textit{ristorare superiore} o altra magia.

Una creatura che non sia sorpresa può distogliere lo sguardo per evitare il Tiro Salvezza all'inizio del proprio round. In quel caso, non potrà vedere la medusa fino all'inizio del suo prossimo round, quando potrà distogliere nuovamente lo sguardo. Se nel frattempo dovesse guardare la medusa, dovrebbe immediatamente effettuare il Tiro Salvezza.

Se la medusa vede il suo riflesso su di una superficie riflettente entro 9 metri da lei in un'area di luce intensa, a causa della propria maledizione subirà gli effetti del suo stesso sguardo.

\textbf{Azioni}

\textit{\textbf{Multiattacco.}} La medusa effettua tre attacchi -- uno con i capelli serpentini e due con la spada corta -- o due attacchi a distanza con l'arco lungo.



\textit{\textbf{Capelli Serpentini.} Attacco con arma da mischia}: +9 a colpire, portata 1 m, un bersaglio.

\textit{Colpisce:} 4 (1d4 + 2) danni perforanti più 14 (4d6) danni da veleno.

\textit{\textbf{Spada Corta.} Attacco con arma da mischia}: 9 a colpire, portata 1 m, un bersaglio.

\textit{Colpisce:} 5 (1d6 + 2) danni perforanti.

\textit{\textbf{Arco Lungo.} Attacco con arma a Distanza}: +9 a colpire, gittata 45m, un bersaglio.

\textit{Colpisce:} 6 (1d8 + 2) danni perforanti più 7 (2d6) danni da veleno.

\textbf{Ecologia}\\
Ambiente: Paludi temperate e sotterranei\\
Organizzazione: Solitario\\
\textbf{Tesoro}: Doppio (Pugnale, Arco Lungo Perfetto con 20 Frecce, altro tesoro)\\
\textbf{Descrizione}\\
Le meduse sono creature simili agli umani con serpenti al posto dei capelli. Dalla distanza di 9 metri o più, una medusa può passare facilmente per una bella donna se indossa qualcosa che copre la sua chioma serpentina; quando indossa un abbigliamento che ne cela la testa e il volto può essere scambiata per un'umana anche a distanza ravvicinata. Le meduse usano bugie e travestimenti per celare il loro volto fino a che gli avversari non sono abbastanza vicini da usare il loro sguardo pietrificante, anche se gli piace giocare con la loro preda e possono usare delle frecce fiammeggianti per intrappolare i nemici a distanza. Alcune si divertono a creare intricate decorazioni con le loro vittime, usando la pietrificazione per dare un certo tocco ai loro nascondigli paludosi, ma molte meduse hanno cura di nascondere le prove dei loro scontri precedenti così che i loro nuovi nemici non si accorgano della loro pericolosa presenza.

Avvezze a nascondersi, le meduse cittadine generalmente sono ladre, mentre quelle delle zone selvagge spesso finiscono per essere guardiaboschi. Le meduse delle leggende più note, tuttavia, sono quelle che prendono livelli da incantatore. Carismatiche ed intelligenti, le meduse urbane sono spesso coinvolte in gilde di ladri ed altri aspetti del mondo criminale. Le meduse possono formare alleanze con creature cieche o non morti intelligenti, entrambi immuni al loro sguardo pietrificante. Le meduse incantatrici fungono spesso da oracoli o profetesse, vivendo generalmente in remote zone di leggendaria potenza o dalla storia infausta. Queste meduse oracoli traggono grande diletto dal loro ruolo, e se ci si presenta con i giusti doni e adulazioni, i segreti che offrono possono essere veramente utili. Naturalmente, i nascondigli di queste potenti creature sono decorati con le statue di coloro che le hanno offese, come monito ad usare le dovute cautele durante gli incontri.

Tutte le meduse sono femmine. Raramente, una medusa decide di prendere un maschio umanoide come compagno, generalmente grazie all'aiuto di una Elisir d'Amore o qualche magia simile, ed hanno sempre cura di non pietrificare il loro prigioniero, a meno che non si siano annoiate della sua compagnia.


\subsection{Mefiti}

\medskip\index[Mostruario]{Mefito di Ghiaccio}\textbf{Mefito di Ghiaccio}

\textit{Piccola elementale, neutrale malvagio}

\textbf{FORZA} -2

\textbf{DESTREZZA} +1

\textbf{COSTITUZIONE} +0

\textbf{INTELLIGENZA} -1

\textbf{SAGGEZZA} +0

\textbf{CARISMA} +1

\textbf{Iniziativa} +1 -- \textbf{Difesa} 12

\textbf{Punti Ferita} 21 (6d6)

\textbf{Movimento} 9 m, volo 9 m

\textbf{Tiri Salvezza}: Tempra +2, Riflessi +5, Volontà +3

\textbf{Competenze} Muoversi Silenziosamente / Nascondersi +3, Consapevolezza +2

\textbf{Vulnerabilità ai Danni} da botta, fuoco

\textbf{Immunità ai Danni} freddo, veleno

\textbf{Immunità alle Condizioni} avvelenato

\textbf{Sensi} scurovisione 18 m

\textbf{Linguaggi} Aquan, Auran

\textbf{Sfida} 1/2 (100 PX)

\textit{\textbf{Falso Aspetto.}} Mentre il mefito rimane immobile, è indistinguibile da un ordinario frammento di ghiaccio.

\textit{\textbf{Incantesimi Innati (1/Giorno).}} Il mefito può lanciare in maniera innata \textit{nube di nebbia}, senza bisogno di componenti materiali. La sua caratteristica da incantatore innato è il Carisma.

\textit{\textbf{Natura Elementale.}} Un mefito non ha bisogno di cibo, bevande o sonno.

\textit{\textbf{Scoppio Mortale.}} Quando il mefito muore, esplode in uno scoppio di frammenti di ghiaccio. Ogni creatura entro 1 metro da esso deve effettuare un Tiro Salvezza di Riflessi DC 10 o subire 4 (1d8) danni taglienti in caso di fallimento, o la metà di questi danni in caso
di successo.

\textbf{Azioni}

\textit{\textbf{Artigli.} Attacco con arma da mischia}: +3 a colpire, portata 1 m, una creatura.

\textit{Colpisce:} 3 (1d4 + 1) danni taglienti più 2 (1d4) danni da freddo.

\textit{\textbf{Soffio Gelido (Ricarica 6).}} Il mefito esala un cono di 5 metri di aria fredda. Ogni creatura nell'area deve effettuare un Tiro Salvezza di Riflessi DC 10, subendo 5 (2d4) danni da freddo in caso di fallimento, o la metà di questi danni in caso di successo.

\textbf{Ecologia}\\
Ambiente: Qualsiasi (piano elementale dell'aria)\\
Organizzazione: Solitario, coppia, gruppo (3-6) o stormo (7-12)\\
\textbf{Tesoro}: Standard\\
\textbf{Descrizione}\\
I mephit sono i servitori di potenti creature elementali. I siti e le locazioni chiave dei piani elementali sono pieni di mephit che si affannano per svolgere un importante dovere o incarico.

I mephit del ghiaccio comunemente si trovano sul Piano dell'Aria. Questi mephit sono distanti e crudeli.


\medskip\index[Mostruario]{Mefito di Magma}\textbf{Mefito di Magma}

\textit{Piccola elementale, neutrale malvagio}

\textbf{FORZA} -1

\textbf{DESTREZZA} +1

\textbf{COSTITUZIONE} +1

\textbf{INTELLIGENZA} -2

\textbf{SAGGEZZA} +0

\textbf{CARISMA} +0

\textbf{Iniziativa} +1 -- \textbf{Difesa} 12

\textbf{Punti Ferita} 22 (5d6 + 5)

\textbf{Movimento} 9 m, volo 9 m

\textbf{Tiri Salvezza}: Tempra +2, Riflessi +5, Volontà +3

\textbf{Competenze} Muoversi Silenziosamente / Nascondersi +3

\textbf{Vulnerabilità ai Danni} freddo

\textbf{Immunità ai Danni} fuoco, veleno

\textbf{Immunità alle Condizioni} avvelenato

\textbf{Sensi} scurovisione 18 m

\textbf{Linguaggi} Ignan, Terran

\textbf{Sfida} 1/2 (100 PX)

\textit{\textbf{Falso Aspetto.}} Mentre il mefito rimane immobile, è indistinguibile da un'ordinaria pozza di magma.

\textit{\textbf{Incantesimi Innati (1/Giorno).}} Il mefito può lanciare in maniera innata \textit{riscaldare metallo} (DC del Tiro Salvezza dell'incantesimo 10), senza bisogno di componenti materiali. La sua caratteristica da incantatore innato è il Carisma.

\textit{\textbf{Natura Elementale.}} Un mefito non ha bisogno di cibo, bevande o sonno.

\textit{\textbf{Scoppio Mortale.}} Quando il mefito muore, esplode in uno scoppio di lava. Ogni creatura entro 1 metro da esso deve effettuare un Tiro Salvezza di Riflessi DC 11 o subire 7 (2d6) danni da fuoco in caso di fallimento, o la metà di questi danni in caso di successo.

\textbf{Azioni}

\textit{\textbf{Artigli.} Attacco con arma da mischia}: +3 a colpire, portata 1 m, una creatura.

\textit{Colpisce:} 3 (1d4 + 1) danni taglienti più 2 (1d4) danni da fuoco.

\textit{\textbf{Soffio Infuocato (Ricarica 6).}} Il mefito esala un cono di 5 metri di fuoco. Ogni creatura nell'area deve effettuare un Tiro Salvezza su Riflessi DC 11, subendo 7 (2d6) danni da fuoco in caso di fallimento, o la metà di questi danni in caso di successo.

\textbf{Ecologia}\\
Ambiente: Qualsiasi (piano elementale del fuoco)\\
Organizzazione: Solitario, coppia, gruppo (3-6) o stormo (7-12)\\
\textbf{Tesoro}: Standard\\
\textbf{Descrizione}\\
I mephit sono i servitori di potenti creature elementali. I siti e le locazioni chiave dei piani elementali sono pieni di mephit che si affannano per svolgere un importante dovere o incarico.

I mephit del magma comunemente si trovano sul Piano del Fuoco. Questi mephit sono stupidi bruti.


\medskip\index[Mostruario]{Mefito di Polvere}\textbf{Mefito di Polvere}

\textit{Piccola elementale, neutrale malvagio}

\textbf{FORZA} -3

\textbf{DESTREZZA} +2

\textbf{COSTITUZIONE} +0

\textbf{INTELLIGENZA} -1

\textbf{SAGGEZZA} +0

\textbf{CARISMA} +0

\textbf{Iniziativa} +2 -- \textbf{Difesa} 13

\textbf{Punti Ferita} 17 (5d6)

\textbf{Movimento} 9 m, volo 9 m

\textbf{Tiri Salvezza}: Tempra +2, Riflessi +5, Volontà +3

\textbf{Competenze} Muoversi Silenziosamente / Nascondersi +4, Consapevolezza +2

\textbf{Vulnerabilità ai Danni} fuoco

\textbf{Immunità ai Danni} veleno

\textbf{Immunità alle Condizioni} avvelenato

\textbf{Sensi} scurovisione 18 m

\textbf{Linguaggi} Auran, Terran

\textbf{Sfida} 1/2 (100 PX)

\textit{\textbf{Incantesimi Innati (1/Giorno).}} Il mefito può eseguire in maniera innata \textit{sonno} (DC del Tiro Salvezza dell'incantesimo 10), senza bisogno di componenti materiali. La sua abilità da incantatore innato è il Carisma.

\textit{\textbf{Natura Elementale.}} Un mefito non ha bisogno di cibo, bevande o sonno.

\textit{\textbf{Scoppio Mortale.}} Quando il mefito muore, esplode in uno scoppio di polvere. Ogni creatura entro 1 metro da esso deve riuscire un Tiro Salvezza di Tempra DC 10 o restare accecata per 1 minuto. Una creatura accecata può ripetere il Tiro Salvezza durante ciascun suo round, terminando l'effetto su di sé in caso di successo.

\textbf{Azioni}

\textit{\textbf{Artigli.} Attacco con arma da mischia}: +4 a colpire, portata 1 m, una creatura.

\textit{Colpisce:} 4 (1d4 + 2) danni taglienti.

\textit{\textbf{Soffio Accecante (Ricarica 6).}} Il mefito esala un cono di 5 metri di polvere accecante. Ogni creatura nell'area deve riuscire un Tiro Salvezza di Riflessi DC 10 o restare accecata per 1 minuto. Una creatura accecata può ripetere il Tiro Salvezza durante ciascun suo round, terminando l'effetto su di sé in caso di successo.

\textbf{Ecologia}\\
Ambiente: Qualsiasi (piano elementale dell'aria)\\
Organizzazione: Solitario, coppia, gruppo (3-6) o stormo (7-12)\\
\textbf{Tesoro}: Standard\\
\textbf{Descrizione}\\
I mephit sono i servitori di potenti creature elementali. I siti e le locazioni chiave dei piani elementali sono pieni di mephit che si affannano per svolgere un importante dovere o incarico.

I mephit della polvere comunemente si trovano sul Piano dell'Aria. Questi mephit sono irritanti ed insistenti.

\medskip\index[Mostruario]{Mefito di Vapore}\textbf{Mefito di Vapore}

\textit{Piccola elementale, neutrale malvagio}

\textbf{FORZA} -3

\textbf{DESTREZZA} +0

\textbf{COSTITUZIONE} +0

\textbf{INTELLIGENZA} +0

\textbf{SAGGEZZA} +0

\textbf{CARISMA} +1

\textbf{Iniziativa} +0 -- \textbf{Difesa} 11

\textbf{Punti Ferita} 21 (6d6)

\textbf{Movimento} 9 m, volo 9 m

\textbf{Tiri Salvezza}: Tempra +2, Riflessi +5, Volontà +3

\textbf{Immunità ai Danni} fuoco, veleno

\textbf{Immunità alle Condizioni} avvelenato

\textbf{Sensi} scurovisione 18 m

\textbf{Linguaggi} Aquan, Ignan

\textbf{Sfida} 1/4 (50 PX)

\textit{\textbf{Incantesimi Innati (1/Giorno).}} Il mefito può eseguire in maniera innata \textit{sfocatura}, senza bisogno di componenti materiali. La sua abilità da incantatore innato è il Carisma.

\textit{\textbf{Natura Elementale.}} Un mefito non ha bisogno di cibo, bevande o sonno.

\textit{\textbf{Scoppio Mortale.}} Quando il mefito muore, esplode in nube di vapore. Ogni creatura entro 1 metro da esso deve riuscire un Tiro Salvezza su Riflessi DC 10 o subire 4 (1d8) danni da fuoco.

\textbf{Azioni}

\textit{\textbf{Artigli.} Attacco con arma da mischia}: +2 a colpire, portata 1 m, una creatura.

\textit{Colpisce:} 2 (1d4) danni taglienti più 2 (1d4) danni da fuoco.

\textit{\textbf{Soffio Vaporoso (Ricarica 6).}} Il mefito esala un cono di 5 metri di vapore caldo. Ogni creatura nell'area deve effettuare un Tiro Salvezza di Riflessi DC 10, subendo 4 (1d8) danni da fuoco in caso di fallimento, o la metà di questi danni in caso di successo.

\textbf{Ecologia}\\
Ambiente: Qualsiasi (piano elementale del fuoco)\\
Organizzazione: Solitario, coppia, gruppo (3-6) o stormo (7-12)\\
\textbf{Tesoro}: Standard\\
\textbf{Descrizione}\\
I mephit sono i servitori di potenti creature elementali. I siti e le locazioni chiave dei piani elementali sono pieni di mephit che si affannano per svolgere un importante dovere o incarico.

I mephit del vapore comunemente si trovano sul Piano del Fuoco. Questi mephit sono insolenti e sprezzanti.



\subsection{Megere}

\medskip\index[Mostruario]{Megera Marina}\textbf{Megera Marina}

\textit{Media fatato, caotico malvagio}

\textbf{FORZA} +3

\textbf{DESTREZZA} +1

\textbf{COSTITUZIONE} +3

\textbf{INTELLIGENZA} +1

\textbf{SAGGEZZA} +1

\textbf{CARISMA} +1

\textbf{Iniziativa} +1 -- \textbf{Difesa} 15

\textbf{Punti Ferita} 52 (7d8 + 21)

\textbf{Vulnerabilità al Danno} ferro freddo

\textbf{Movimento} 9 m, nuoto 12 m

\textbf{Tiri Salvezza}: Tempra +5, Riflessi +7, Volontà +5

\textbf{Sensi} scurovisione 18 m

\textbf{Linguaggi} Aquan, Comune, Gigante

\textbf{Sfida} 2 (450 PX)

\textit{\textbf{Anfibio.}} La megera può respirare aria e acqua.

\textit{\textbf{Aspetto Orripilante.}} Qualsiasi umanoide che inizi il suo round entro 9 metri dalla megera e ne può vedere la vera forma deve effettuare un Tiro Salvezza di Volontà DC 11. Se fallisce il Tiro Salvezza, la creatura resta spaventata per 1 minuto. Una creatura può ripetere il Tiro Salvezza al termine di ciascun suo round, con -1d6 se la megera è in linea di visuale, e terminando l'effetto se riesce il Tiro Salvezza. Se il Tiro Salvezza della creatura riesce o l'effetto ha termine su di essa, la creatura è immune all'Aspetto Orripilante per le successive 24 ore.

A meno che il bersaglio non sia sorpreso o la rivelazione della vera forma della megera non sia improvvisa, il bersaglio può distogliere lo sguardo e evitare di effettuare il Tiro Salvezza iniziale. Fino all'inizio del suo prossimo round, una creatura che distolga lo sguardo
ha -1d6 ai tiri di attacco contro la megera.

\textbf{Azioni}

\textit{\textbf{Artigli.} Attacco in mischia con arma}: +5 a colpire, portata 1 m, un bersaglio.

\textit{Colpisce:} 10 (2d6 + 3) danni taglienti, 1 danno da Sanguinamento.

\textit{\textbf{Aspetto Illusorio.}} La megera ricopre se stessa e tutto quello che sta indossando o trasportando in un'illusione magica che le dona l'aspetto di una creatura ripugnante all'incirca della stessa taglia e forma umanoide. L'illusione termina se la megera effettua un'azione bonus per terminarla o se muore.

I cambiamenti apportati da questo effetto non sono in grado di superare le ispezioni fisiche. Ad esempio, la megera potrebbe apparire come una creatura priva di artigli, ma una persona in contatto con le sue mani li avvertirebbe. Altrimenti, una creatura deve effettuare un'azione per ispezionare visivamente l'illusione e riuscire una prova di Intelligenza DC 16 per comprendere che la megera si è camuffata.

\textit{\textbf{Occhiata Mortale.}} La megera prende a bersaglio una creatura spaventata visibile entro 9 metri da lei. Se il bersaglio può vedere la megera, deve riuscire un Tiro Salvezza di Volontà DC 11 contro questa magia o scendere a 0 Punti Ferita.

\textbf{Ecologia}\\
Ambiente: qualsiasi acquatico\\
Organizzazione: solitario o congrega (3 megere di qualsiasi specie)\\
\textbf{Tesoro}: standard\\
\textbf{Descrizione}\\
Queste perfide e mostruose megere possiedono dei tratti terrificanti che pochi osano fissare, traggono piacere dalla discordia e dalla morte dei marinai, e tormentano la gente di mare con ineluttabili sciagure. Le megere marine hanno sempre un aspetto terribile e, malgrado la loro natura famelica, in genere sono creature emaciate che sembrano sul punto di morir di fame. Sono alte 1,8 metri e pesano 75 kg.

Le megere marine preferiscono vivere vicino alla riva dove i pescherecci e i mercantili sono più comuni, e comunque lontano dalle aree urbane di modo che le loro azioni non attraggano troppo l'attenzione di possibili nemici, anche se non è insolito che una megera marina coraggiosa o avida si stabilisca in una città portuale o alla foce di un fiume profondo.

Le megere marine formano congreghe simili a quelle delle altre megere, ma la loro natura acquatica generalmente le spinge ad astenersi dal formare congreghe miste. Nel caso in cui una Megera Verde abiti lungo la costa (spesso in una palude salmastra o in una palude costiera), una congrega è formata da due megere marine che rispettano la Megera Verde come madre e capo. Molto comunemente, una congrega di megere marine consiste in un gruppo di megere marine particolarmente amiche e vicine.


\medskip\index[Mostruario]{Megera Notturna}\textbf{Megera Notturna}

\textit{Media immondo, neutrale malvagio}

\textbf{FORZA} +4

\textbf{DESTREZZA} +2

\textbf{COSTITUZIONE} +3

\textbf{INTELLIGENZA} +3

\textbf{SAGGEZZA} +2

\textbf{CARISMA} +3

\textbf{Iniziativa} +3 -- \textbf{Difesa} 20

\textbf{Punti Ferita} 112 (15d8 + 45)

\textbf{Movimento} 9 m

\textbf{Tiri Salvezza}: Tempra +14, Riflessi +8, Volontà +11

\textbf{Competenze} Muoversi Silenziosamente / Nascondersi +6, Ingannare +7, Percepire Emozioni +6, Consapevolezza +6,

\textbf{Resistenze al Danno} freddo, fuoco; da arma non magica o non siano argentati

\textbf{Sensi} scurovisione 36 m

\textbf{Linguaggi} Abissale, Comune, Infernale, Druidico

\textbf{Sfida} 5 (1.800 PX)

\textit{\textbf{Incantesimi Innati.}} La caratteristica da incantatore innato della megera è il Carisma (DC 14 per i Tiri Salvezza degli incantesimi, +6 a colpire con attacchi da incantesimo). La megera può lanciare in maniera innata i seguenti incantesimi, senza aver bisogno di
componenti materiali.

A volontà: \textit{dardo incantato, individuazione del magico} 2/giorno ciascuno: \textit{raggio di indebolimento, sonno, spostamento} \textit{planare} (personale)

\textit{\textbf{Resistenza alla Magia.}} La megera ha +1d6 ai tiri salvezza contro incantesimi e altri effetti magici.

\textbf{Azioni}

\textit{\textbf{Artigli (Solo in Forma di Megera).} Attacco con arma da mischia}: +10 a colpire, portata 1 m, un bersaglio.

\textit{Colpisce:} 13 (2d8 + 4) danni taglienti, 1 danno da Sanguinamento.

\textit{\textbf{Forma Eterea.}} La megera entra magicamente nel Piano Etereo dal Piano Materiale, e viceversa. Per farlo deve essere in possesso di un \textit{cuore di pietra}.

\textit{\textbf{Infestare Incubi (1/Giorno).}} Mentre si trova sul Piano Etereo, la megera entra magicamente in contatto con un umanoide addormentato che si trova sul Piano Materiale. L'incantesimo \textit{protezione dal bene e dal male} lanciato sul bersaglio previene questo contatto, così come \textit{cerchio magico}. Finché il contatto persiste, il bersaglio soffre di orribili visioni. Se queste visioni durano per almeno 1 ora, il bersaglio non ottiene benefici dal suo riposo, e i suoi Punti Ferita massimi sono ridotti di 5 (1d10). Se questo effetto riduce i Punti Ferita massimi del bersaglio a 0, il bersaglio muore, e se il bersaglio era malvagio, la sua anima resta intrappolata nella \textit{borsa} \textit{delle anime} della megera. La riduzione dei Punti Ferita massimi del bersaglio rimane finché non viene rimossa dall'incantesimo \textit{ristorare} \textit{superiore} o simile magia.

\textit{\textbf{Mutare Forma.}} La megera può trasformarsi magicamente in una femmina umanoide di taglia Piccola o Media, o tornare alla sua vera forma. Le sue statistiche sono le stesse in qualsiasi forma. Tutto l'equipaggiamento che stava trasportando o indossando non viene trasformato. Alla morte, ritorna alla sua vera forma.



\medskip\index[Mostruario]{Megera Verde}\textbf{Megera Verde}

\textit{Media fatato, neutrale malvagio}

\textbf{FORZA} +4

\textbf{DESTREZZA} +1

\textbf{COSTITUZIONE} +3

\textbf{INTELLIGENZA} +1

\textbf{SAGGEZZA} +2

\textbf{CARISMA} +2

\textbf{Iniziativa} +1 -- \textbf{Difesa} 19

\textbf{Punti Ferita} 82 (11d8 + 33)

\textbf{Vulnerabilità al Danno} ferro freddo

\textbf{Movimento} 9 m

\textbf{Tiri Salvezza}: Tempra +6, Riflessi +7, Volontà +7

\textbf{Competenze} Arcano +3, Muoversi Silenziosamente / Nascondersi +3, Ingannare +4, Consapevolezza +4

\textbf{Sensi} scurovisione 18 m

\textbf{Linguaggi} Comune, Draconico, Silvano

\textbf{Sfida} 3 (700 PX)

\textit{\textbf{Anfibio.}} La megera può respirare aria e acqua.

\textit{\textbf{Imitazione.}} La megera può imitare suoni animali e voci umanoidi. Una creatura che senta questi rumori può determinare che si tratti di un'imitazione riuscendo una prova di Saggezza DC 14.

\textit{\textbf{Incantesimi Innati.}} La caratteristica da incantatore innato della megera è il Carisma (DC 12 per i Tiri Salvezza degli incantesimi). La megera può lanciare in maniera innata i seguenti incantesimi, senza aver bisogno di componenti materiali.

A volontà: \textit{illusione minore, luci danzanti, beffa maligna}

\textbf{Azioni}

\textit{\textbf{Artigli.} Attacco con arma da mischia}: +6 a colpire, portata 1 m, un bersaglio.

\textit{Colpisce:} 13 (2d8 + 4) danni taglienti, 1 danno da Sanguinamento.

\textit{\textbf{Aspetto Illusorio.}} La megera ricopre sé stessa e tutto quello che sta indossando o trasportando in un'illusione magica che le dona l'aspetto di un'altra creatura all'incirca della stessa taglia e forma umanoide. L'illusione termina se la megera effettua un'azione bonus per terminarla o se muore.

I cambiamenti apportati da questo effetto non sono in grado di superare le ispezioni fisiche. Ad esempio, la megera potrebbe apparire come una creatura dalla pelle liscia, ma il contatto rivelerebbe la sua pelle ruvida. Altrimenti, una creatura deve effettuare un'azione per ispezionare visivamente l'illusione e riuscire una prova di Intelligenza DC 20 per comprendere che si tratta di una megera camuffata.

\textit{\textbf{Passaggio Invisibile.}} La megera può rendersi invisibile finché non attacca o lancia un incantesimo, o finché non termina la concentrazione (come se si stesse concentrando su di un incantesimo). Mentre è invisibile, non lascia traccia fisica del suo passaggio, quindi le sue tracce possono essere seguite solo dalla magia. Tutto l'equipaggiamento che sta trasportando o indossando diventa invisibile assieme a lei.

\textbf{Ecologia}
Ambiente: Paludi temperate\\
Organizzazione: Solitario o congrega (3 megere di qualsiasi tipo)\\
\textbf{Tesoro}: Standard\\
\textbf{Descrizione}\\
Terrificanti vecchie rugose che frequentano ripugnanti paludi e foreste intricate, le megere verdi nutrono un odio intenso per tutto ciò che è bello e puro. Facendo uso delle loro svariate capacità illusorie, queste vegliarde si dilettano nell'uccidere gli innocenti, nello sconvolgere gli animi nobili e nell'avvilire i cuori puri. Amano utilizzare Camuffare Se Stesso per assumere le forme di giovani e attraenti ragazze così da sedurre e strappare giovani uomini ai loro affetti e parenti, e corrompere nobili e onesti cittadini con ogni sorta di depravazione e scandalo. Alcune megere verdi preferiscono rivelare la loro reale natura ai loro amati in un momento attentamente architettato per spingere l'uomo alla pazzia per l'orrore e la vergogna. Altre prolungano il loro amoreggiamento e fanno di tutto per rovinare completamente la vita degli uomini da loro sedotti prima di mostrare loro la verità. Infine, i più fortunati di questi sventurati finiscono per essere divorati dalla megera verde loro amante: per gli sfortunati, il destino finale può essere molto peggiore, dato che la crudele fantasia della megera verde è immensa. Una tipica megera verde è alta tra 1,5 e 1,8 metri e pesa poco meno di 80 kg.


\subsection{Melme}

\medskip\index[Mostruario]{Ameba Paglierina}\textbf{Ameba Paglierina}

\textit{Grande melma, disallineato}

\textbf{FORZA} +2

\textbf{DESTREZZA} -2

\textbf{COSTITUZIONE} +2

\textbf{INTELLIGENZA} -4

\textbf{SAGGEZZA} -2

\textbf{CARISMA} -5

\textbf{Iniziativa} +2 -- \textbf{Difesa} 9

\textbf{Punti Ferita} 45 (6d10 + 12)

\textbf{Movimento} 3 m, scalata 3 m

\textbf{Tiri Salvezza}: Tempra +8, Riflessi -3, Volontà -3

\textbf{Resistenze al Danno} acido

\textbf{Immunità al Danno} fulmine, tagliente

\textbf{Immunità alle Condizioni} accecato, affascinato, assordato, prono, affaticamento, spaventato

\textbf{Sensi} vista cieca 18 m (cieca oltre questo raggio)

\textbf{Linguaggi} -

\textbf{Sfida} 2 (450 PX)

\textit{\textbf{Amorfo.}} L'ameba può muoversi attraverso uno spazio fino a 3 centimetri di larghezza senza doversi stringere.

\textit{\textbf{Natura di Melma.}} L'ameba non necessita di dormire.

\textit{\textbf{Scalare come Ragno.}} L'ameba può scalare superfici difficili, compreso lo stare a testa in giù sul soffitto, senza bisogno di effettuare una prova di abilità.

\textbf{Azioni}

\textit{\textbf{Pseudopodo.} Attacco con arma da mischia}: +4 a colpire, portata 1 m, un bersaglio.

\textit{Colpisce:} 9 (2d6 + 2) danni da botta più 3 (1d6) danni da acido.

\textbf{Reazioni}

\textit{\textbf{Divisione.}} Quando un'ameba Media o più grande subisce danni da fulmine o taglienti, si divide in due nuove amebe che hanno almeno 10 Punti Ferita. Ogni nuova ameba ha un numero di Punti Ferita pari alla metà dell'ameba originale, arrotondati per difetto. Le nuove amebe sono di una taglia più piccola di quella originale.

\textbf{Ecologia}
Ambiente: Sotterranei o Paludi Temperati\\
Organizzazione: Solitario\\
\textbf{Tesoro}: Nessuno\\
\textbf{Descrizione}\\
Le Ameba Paglierina sono masse animate di protoplasma di colore simile ad un repellente amalgama di giallo, arancio e marrone. Quando a riposo, il loro corpo piatto e pulsante è alto circa 15 centimetri e si estende tutto intorno; in movimento, si raccolgono in una forma vagamente sferica e sembrano quasi spostarsi rotolando. I loro corpi malleabili permettono loro di attraversare fessure e buchi molto più piccoli dello spazio che occupano. Le creature che vivono sottoterra spesso sigillano tutte le aperture per difendersi dalle Ameba Paglierina.

L'acido altamente specializzato dell'Ameba Paglierina dissolve solo la carne. Questa scoperta ha portato molti maestri avvelenatori ed alchimisti a cercarne esemplari per studiarli. Da questi esperimenti sono nate diverse armi specifiche ideate per distruggere i corpi. Si racconta dell'esistenza di un veleno ad azione lenta che distrugge ad una ad una le cellule delle creature viventi, il cui segreto è ben conservato dal suo creatore.

Alcune note in un tomo dimenticato parlano di un rituale funebre utilizzato in luoghi lontani. Invece di bruciare il corpo, esso veniva sigillato in un sarcofago di pietra con una Ameba Paglierina, che ne dissolveva il corpo. In seguito, i becchini inserivano la gelatina in un'urna completa di targa in bronzo con il nome del deceduto. Questa pratica protegge gli oggetti interrati con il morto (che viene ridotto in poco tempo ad uno scheletro lucido) e l'essenza della creatura, che si riteneva vivere ancora all'interno della gelatina.

L'Ameba Paglierina sono alte circa 15 centimetri, hanno un diametro che può arrivare a 3 metri e pesano circa 1.300 chili. In combattimento, si raccolgono su loro stesse e producono lunghi pseudopodi umidi per colpire ed afferrare qualunque cosa si muova.

Anche se la tipica Ameba Paglierina ha le statistiche qui presentate, nelle profondità della terra questi predatori possono raggiungere dimensioni mostruose. Si parla anche di Ameba Paglierina che hanno sviluppato altri modi di catturare la preda. Ad esempio, gelatine che avvelenano con il tocco e che espellono nubi di gas tossico che fa bruciare occhi e bocca, lasciando indifesi ma coscienti mentre questa bestia protoplasmatica scivola sui corpi e se ne ciba.


\medskip\index[Mostruario]{Cubo Gelatinoso}\textbf{Cubo Gelatinoso}

\textit{Grande melma, disallineato}

\textbf{FORZA} +2

\textbf{DESTREZZA} -4

\textbf{COSTITUZIONE} +5

\textbf{INTELLIGENZA} -5

\textbf{SAGGEZZA} -2

\textbf{CARISMA} -5

\textbf{Iniziativa} -4 -- \textbf{Difesa} 7

\textbf{Punti Ferita} 84 (8d10 + 40)

\textbf{Movimento} 5 metri

\textbf{Tiri Salvezza}: Tempra +9, Riflessi -4, Volontà -4

\textbf{Immunità al Danno} armi da taglio non magiche

\textbf{Immunità alle Condizioni} accecato, affascinato, assordato, prono, affaticamento, spaventato

\textbf{Sensi} vista cieca 18 m (cieca oltre questo raggio)

\textbf{Linguaggi} -

\textbf{Sfida} 2 (450 PX)

\textit{\textbf{Cubo di Melma.}} Il cubo occupa il suo intero spazio. Le altre creature possono entrare nello spazio, ma rimangono vittima del Sommergere del cubo e hanno -1d6 al Tiro Salvezza.

Le creature all'interno del cubo sono visibili ma godono di copertura completa.

Una creatura entro 1 metro dal cubo può effettuare un'azione per tirare una creatura od oggetto fuori dal cubo. Farlo richiede la riuscita di una prova di Forza DC 12, e la creatura che effettua il tentativo subisce 10 (3d6) danni da acido.

Il cubo può contenere solo una creatura Grande o un massimo di quattro creature Medie o più piccole alla volta.

\textit{\textbf{Natura di Melma.}} Il cubo non necessita di dormire.

\textit{\textbf{Trasparente.}} Anche quando è in piena vista, è necessario riuscire una prova di Saggezza (Consapevolezza) DC 15 per notare un cubo che non si è mosso o non ha attaccato. Una creatura che cerchi di entrare nello spazio del cubo mentre è inconsapevole della sua presenza resta sorpresa dal cubo.

\textbf{Azioni}

\textit{\textbf{Pseudopodo.} Attacco con arma da mischia}: +4 a colpire, portata 1 m, un bersaglio.

\textit{Colpisce:} 10 (3d6) danni da acido.

\textit{\textbf{Sommergere.}} Il cubo si muove fino al massimo del suo movimento. Nel farlo, può entrare nello spazio di una creatura di taglia Grande o più piccola. Ogni volta che il cubo entra nello spazio di una creatura, la creatura deve effettuare un Tiro Salvezza di Riflessi DC 12.

Se il Tiro Salvezza riesce, la creatura può scegliere di essere spinta indietro o di lato di 1 metro. Una creatura che decida di non farsi spingere subisce le conseguenze di un Tiro Salvezza fallito.

Se il Tiro Salvezza fallisce, il cubo entra nello spazio della creatura, che subisce 10 (3d6) danni da acido ed è sommersa. La creatura sommersa non può respirare, è intralciata e subisce 21 (6d6) danni da acido all'inizio del turno del cubo. Quando il cubo si muove, la creatura sommersa si muove con esso.

Una creatura sommersa può tentare di fuggire effettuando un'azione per compiere una prova di Forza DC 12. Se la riesce, la creatura sfugge e ed entra nello spazio di sua scelta entro 1 metro dal cubo.

\textbf{Ecologia}
Ambiente: Qualsiasi sotterraneo\\
Organizzazione: Solitario\\
\textbf{Tesoro}: Accidentale\\
\textbf{Descrizione}\\
Tra i predatori più insoliti e peculiari dei dungeon, i cubi gelatinosi trascorrono la loro esistenza vagabondando senza meta per i cunicoli sotterranei e le oscure caverne, inglobando materiali organici come piante, rifiuti, carogne e anche creature viventi. La materia che il cubo non può digerire, come metalli e pietra, riempie di detriti il volume della creatura, e a volte questa può espellerne una parte dal suo corpo. Spesso il tesoro e gli averi delle vittime passate restano dentro il cubo gelatinoso: immagine spettrale dei loro resti materiali.

Alcuni saggi credono che queste creature si siano evolute delle Melme Grigie. Alcuni esseri usano i cubi gelatinosi come guardiani di dungeon e fortificazioni sotterranee, intrappolando queste immense creature in casse di metallo massiccio e trasportandole con poteri o magie fino al loro posto di guardia finale. Sono dei meccanismi di smaltimento rifiuti particolarmente efficaci; una tribù può intrappolare un cubo gelatinoso in una fossa o un'altra area che non possa scalare usandolo come letamaio o anche trappola mortale, a seconda dell'ingegnosità delle creature che l'hanno catturato.

I cubi gelatinosi in genere hanno uno spigolo di 3 metri e pesano più di 7.500 kg, sebbene alcuni esploratori sotterranei affermino che nel sottosuolo esistano esemplari più grandi. In zone in cui il cibo abbonda, i cubi gelatinosi possono vivere per centinaia, se non migliaia, di anni. Tuttavia, se viene a mancare la materia organica per più di 6 mesi, un cubo gelatinoso comincia a deperire, e le sue pareti iniziano a colare, disfacendosi rapidamente in muco liquido finché l'intero corpo non collassa e scompare completamente.


\medskip\index[Mostruario]{Melma Grigia}\textbf{Melma Grigia}

\textit{Media melma, disallineato}

\textbf{FORZA} +1

\textbf{DESTREZZA} -2

\textbf{COSTITUZIONE} +3

\textbf{INTELLIGENZA} -5

\textbf{SAGGEZZA} -2

\textbf{CARISMA} -4

\textbf{Iniziativa} -2 -- \textbf{Difesa} 9

\textbf{Punti Ferita} 22 (3d8 + 9)

\textbf{Movimento} 3 m, scalata 3 m

\textbf{Tiri Salvezza}: Tempra +9, Riflessi -4, Volontà -4

\textbf{Resistenze al Danno} acido, freddo, fuoco

\textbf{Immunità alle Condizioni} accecato, affascinato, assordato, prono, affaticamento, spaventato

\textbf{Sensi} vista cieca 18 m (cieca oltre questo raggio)

\textbf{Linguaggi} -

\textbf{Sfida} 1/2 (100 PX)

\textit{\textbf{Amorfo.}} La melma può muoversi attraverso uno spazio fino a centimetri di larghezza senza doversi stringere.

\textit{\textbf{Corrodere Metallo.}} Qualsiasi arma non magica fatta di metallo che colpisca la melma si corrode. Dopo aver inflitto il danno, l'arma subisce una penalità permanente e cumulativa di -1 ai tiri di danno. Se la penalità arriva a -5, l'arma è distrutta. Le munizioni non magiche fatte di metallo che colpiscano la melma, si distruggono dopo aver inflitto il danno.

La melma può divorare metallo non magico dello spessore di 5 centimetri in un 1 round.

\textit{\textbf{Falso Aspetto.}} Quando la melma rimane immobile, è indistinguibile da una pozza d'olio o una pietra bagnata.

\textit{\textbf{Natura di Melma.}} La melma non necessita di dormire.

\textbf{Azioni}

\textit{\textbf{Pseudopodo.} Attacco con arma da mischia}: +3 a colpire, portata 1 m, un bersaglio.

\textit{Colpisce:} 4 (1d6 + 1) danni da botta più 7 (2d6) danni da acido, e se il bersaglio sta indossando un'armatura di metallo, questa viene parzialmente dissolta e subisce una penalità permanente e cumulativa di -1 alla Difesa che offre. L'armatura è distrutta se la penalità riduce la sua Difesa a 10.

\textbf{Ecologia}\\
Ambiente: Paludi fredde e sotterranei\\
Organizzazione: Solitario\\
\textbf{Tesoro}: Nessuno\\
\textbf{Descrizione}\\
Strisciando attraverso le fredde paludi e gli acquitrini nebbiosi o, a volte in sotterranei e caverne, le melme grigie consumano ogni sostanza organica che incontrano. Sebbene priva di intelligenza, la melma grigia è una delle creature che dà non pochi problemi per la sua trasparenza. Anche se non può arrampicarsi facilmente sui muri o nuotare, la sua abitudine di nascondersi nel fango spesso lungo le rive paludose o di rimanere immobile in pozze dall'aspetto innocuo sul pavimento grigio di un sotterraneo, la rendono molto difficile da notare e da evitare.

Alcuni saggi credono che le melme grigie siano il risultato di un esperimento alchemico fallito, mentre altri teorizzano che le prime melme grigie siano nate spontaneamente da un pozzo di detriti magici. Naturalmente, queste teorie che non le considerano organismi viventi, bensì il risultato di una sfortunata mistura di fluidi caustici e residui magici, sono derisi da chi vive nelle zone infestate da queste creature, che non hanno una storia di inquinamento magico.


\medskip\index[Mostruario]{Protoplasma Nero}\textbf{Protoplasma Nero}

\textit{Grande melma, disallineato}

\textbf{FORZA} +3

\textbf{DESTREZZA} -3

\textbf{COSTITUZIONE} +3

\textbf{INTELLIGENZA} -5

\textbf{SAGGEZZA} -2

\textbf{CARISMA} -5

\textbf{Iniziativa} -3 -- \textbf{Difesa} 9

\textbf{Punti Ferita} 85 (10d10 + 30)

\textbf{Movimento} 6 m, scalata 6 m

\textbf{Tiri Salvezza}: Tempra +9, Riflessi -2, Volontà -2

\textbf{Immunità al Danno} acido, freddo, fulmine, tagliente

\textbf{Immunità alle Condizioni} accecato, affascinato, assordato, prono, affaticamento, spaventato

\textbf{Sensi} vista cieca 18 m (cieco oltre questo raggio)

\textbf{Linguaggi} -

\textbf{Sfida} 4 (1.100 PX)

\textit{\textbf{Amorfo.}} Il protoplasma nero può muoversi attraverso uno spazio fino a 3 centimetri di larghezza senza doversi stringere.

\textit{\textbf{Forma Corrosiva.}} Una creatura che entri a contatto col protoplasma nero o lo colpisca con un attacco da mischia mentre si trova entro 1 metro da esso subisce 4 (1d8) danni da acido. Qualsiasi arma non magica fatta di metallo o legno che colpisca il protoplasma nero si corrode. Dopo aver inflitto il danno, l'arma subisce una penalità permanente e cumulativa di -1 ai tiri di danno. Se la penalità arriva a -5, l'arma è distrutta. Le munizioni non magiche fatte di metallo o legno che colpiscano il protoplasma nero, si distruggono dopo aver inflitto il danno.

Il protoplasma nero può divorare legno o metallo non magico dello spessore di 5 centimetri in un 1 round.

\textit{\textbf{Natura di Melma.}} Il protoplasma nero non necessita di dormire.

\textit{\textbf{Scalare come Ragno.}} Il protoplasma nero può scalare superfici difficili, compreso lo stare a testa in giù sul soffitto, senza bisogno di effettuare una prova di abilità.

\textbf{Azioni}

\textit{\textbf{Pseudopodo.} Attacco con arma da mischia}: +7 a colpire, portata 1 m, un bersaglio.

\textit{Colpisce:} 6 (1d6 + 3) danni da botta più 18 (4d8) danni da acido. Inoltre, un'armatura non magica indossata dal bersaglio viene parzialmente dissolta e subisce una penalità permanente e cumulativa di -1 alla Difesa che offre. L'armatura è distrutta se la penalità riduce la sua Difesa a 10.

\textbf{Reazioni}

\textit{\textbf{Divisione.}} Quando un protoplasma nero di taglia Media o più grande subisce danni da fulmine o taglienti, si divide in due nuovi protoplasma neri di almeno 10 Punti Ferita ciascuno. Ogni nuovo protoplasma nero ha un numero di Punti Ferita pari alla metà del protoplasma nero originale, arrotondati per difetto. I nuovi protoplasmi neri sono di una taglia più piccola di quella originale.

\textbf{Ecologia}\\
Ambiente: Qualsiasi sotterraneo\\
Organizzazione: Solitario\\
\textbf{Tesoro}: Nessuno\\
\textbf{Descrizione}\\
I protoplasmi neri sono gli spazzini del mondo sotterraneo, costantemente alla ricerca di cibo. Possono percepire corpi organici o metallici nel raggio di 18 metri e attaccano in modo istintivo tali oggetti o esseri finché non li dissolvono, o finché la melma non viene uccisa. Un protoplasma nero si riproduce staccando un pezzo del proprio corpo e formando un nuovo protoplasma più piccolo che raggiunge l'età adulta nel giro di un mese. Alcune tra le creature più intelligenti nel mondo sotterraneo usano i protoplasmi neri per smaltire in modo naturale la spazzatura, creando cave di pietra atte ad ospitare il protoplasma, per poi gettarvi i rifiuti organici o i nemici.
Gli esemplari più grandi di protoplasmi neri sono stati avvistati nelle regioni più profonde del mondo: individui Mastodontici che possiedono fino a 30 DV. Si dice che esistano anche protoplasmi colorati: alcuni bianchi che vivono nelle zone artiche, marroni nelle paludi e di colore rossiccio che popolano il deserto.


\medskip\index[Mostruario]{Mimic}\textbf{Mimic}

\textit{Media mostruosità (mutaforma), neutrale}

\textbf{FORZA} +3

\textbf{DESTREZZA} +1

\textbf{COSTITUZIONE} +2

\textbf{INTELLIGENZA} -3

\textbf{SAGGEZZA} +1

\textbf{CARISMA} -1

\textbf{Iniziativa} +1 -- \textbf{Difesa} 13

\textbf{Punti Ferita} 58 (9d8 + 18)

\textbf{Movimento} 5 metri

\textbf{Tiri Salvezza}: Tempra +5, Riflessi +5, Volontà +6

\textbf{Competenze} Muoversi Silenziosamente / Nascondersi +5

\textbf{Immunità al Danno} acido

\textbf{Immunità alle Condizioni} prono

\textbf{Sensi} scurovisione 18 m

\textbf{Linguaggi} -

\textbf{Sfida} 2 (450 PX)

\textit{\textbf{Aderente (Solo Forma di Oggetto).}} Il mimic aderisce a qualsiasi cosa con cui entri in contatto. Una creatura di taglia Enorme o inferiore a cui il mimic aderisce è considerata afferrata da esso (DC 13 per fuggire). Le prove di caratteristica effettuare per fuggire da
questo afferrare hanno -1d6.

\textit{\textbf{Afferratore.}} Il mimic ha +1d6 ai tiri per colpire contro una creatura da esso afferrata.

\textit{\textbf{Falso Aspetto (Solo Forma di Oggetto).}} Mentre il mimic rimane immobile, è indistinguibile da un comune oggetto.

\textit{\textbf{Mutaforma.}} Il mimic può usare la sua azione per trasformarsi in un oggetto, o per tornare alla sua vera forma amorfa. Le sue statistiche sono le stesse in qualsiasi forma. Qualsiasi equipaggiamento stia indossando o trasportando non si trasforma. Alla morte ritorna al suo vero aspetto.

\textbf{Azioni}

\textit{\textbf{Morso.} Attacco con arma da mischia}: +6 a colpire, portata 1 m, un bersaglio.

\textit{Colpisce:} 7 (1d8 + 3) danni perforanti più 4 (1d8) danni da acido.

\textit{\textbf{Pseudopodo.} Attacco con arma da mischia}: +6 a colpire, portata 1 m, un bersaglio.

\textit{Colpisce:} 7 (1d8 + 3) danni da botta. Se il mimic è in forma di oggetto, il bersaglio è vittima del tratto Aderente.

\textbf{Ecologia}
Ambiente: Qualsiasi\\
Organizzazione: Solitario\\
\textbf{Tesoro}: Accidentale\\
\textbf{Descrizione}\\
Si ritiene che i mimic siano il risultato del tentativo di un alchimista di dar vita ad un oggetto inanimato attraverso l'applicazione di un reagente mistico, la cui formula è andata perduta. Nel corso degli anni, queste creature strane ma intelligenti hanno appreso la capacità di trasformarsi in simulacri degli oggetti manufatti, in particolare nei luoghi frequentati poco da un ristretto numero di creature, dove aumentano le loro probabilità di successo con un attacco alle loro vittime.

Anche se i mimic non sono intrinsecamente malvagi, alcuni saggi suggeriscono che attacchino gli uomini e le altre creature intelligenti più per passatempo che per sfamarsi. Il desiderio di ingannare gli altri è parte del loro essere, e i loro attacchi a sorpresa rappresentano il culmine di questo desiderio.

Un tipico mimic ha un volume di 2 metri cubi (1 m per 1 m per 2 m) e pesa circa 450 kg. Leggende e storie parlano di mimic di taglie maggiori, con la capacità di assumere la forma di case, navi o interi complessi sotterranei che guarniscono con dei tesori (sia veri che falsi) per attirare al loro interno il loro ignaro cibo.


\medskip\index[Mostruario]{Minotauro}\textbf{Minotauro}

\textit{Grande mostruosità, caotico malvagio}

\textbf{FORZA} +4

\textbf{DESTREZZA} +0

\textbf{COSTITUZIONE} +3

\textbf{INTELLIGENZA} -2

\textbf{SAGGEZZA} +3

\textbf{CARISMA} -1

\textbf{Iniziativa} +0 -- \textbf{Difesa} 16

\textbf{Punti Ferita} 76 (9d10 + 27)

\textbf{Movimento} 12 m

\textbf{Tiri Salvezza}: Tempra +6, Riflessi +5, Volontà +5

\textbf{Competenze} Consapevolezza +7

\textbf{Sensi} scurovisione 18 m

\textbf{Linguaggi} Abissale

\textbf{Sfida} 3 (700 PX)

\textit{\textbf{Carica.}} Se il minotauro si muove di almeno 3 metri diretto verso un bersaglio e lo colpisce con un attacco di incornata durante lo stesso turno, il bersaglio subisce 9 (2d8) danni perforanti aggiuntivi. Se il bersaglio è una creatura, deve riuscire un Tiro Salvezza su Tempra DC 14 o venire spinto via fino a 3 metri di distanza e cadere prono.

\textit{\textbf{Incauto.}} All'inizio del suo round, il minotauro può ottenere +1d6 su tutti i tiri per colpire con armi da mischia effettuati durante quel turno, ma i tiri per colpire contro di esso hanno +1d6 fino all'inizio del suo prossimo round.

\textit{\textbf{Ricordare Labirinto.}} Il minotauro può ricordare perfettamente qualsiasi tragitto abbia percorso.

\textbf{Azioni}

\textit{\textbf{Ascia Bipenne.} Attacco con arma da mischia}: +8 a colpire, portata 1 m, un bersaglio.

\textit{Colpisce:} 17 (2d12 + 4) danni taglienti.

\textit{\textbf{Incornata.} Attacco con arma da mischia}: +8 a colpire, portata 1 m, un bersaglio.

\textit{Colpisce:} 13 (2d8 + 4) danni perforanti.

\textbf{Ecologia}\\
Ambiente: Rovine Temperate e Sotterranei\\
Organizzazione: Solitario, coppia o gruppo (3-4)\\
\textbf{Tesoro}: Standard (Ascia Bipenne, altro tesoro)\\
\textbf{Descrizione}\\
Nessuno porta rancore come un minotauro. Disprezzati dalle razze civilizzate e nati secoli fa da una maledizione divina, i minotauri hanno cacciato, ucciso e divorato gli umanoidi inferiori per punire offese vere o presunte più a lungo di quanto riescono a ricordare. La maggior parte delle culture ha leggende su come i minotauri furono creati da divinità vendicative o offese che punirono gli umani deformando le loro sembianze, sottraendo loro bellezza e intelligenza, e dotandoli di teste di toro. Eppure la maggioranza dei minotauri moderni disprezza queste leggende e non crede di essere lo scherzo di qualche divinità, ma modelli di perfezione divina creati dal crudele e potente signore dei demoni Baphomet.

I nascondigli tradizionali dei minotauri sono i labirinti, sia i dedali costruiti per confondere e sconcertare, sia quelli naturali creati da un intrico di caverne o altri passaggi sotterranei. Grazie alla loro astuzia naturale, i minotauri usano i loro nascondigli labirintici per scoraggiare gli incauti nemici che cercano di scovarli o che semplicemente incappano nei loro nascondigli e si perdono, dando lentamente la caccia agli intrusi che cercano inutilmente di trovare una via d'uscita. Solo quando la disperazione ha nettamente preso il sopravvento il minotauro colpisce le sue perdute vittime. Quando hanno a che fare con un gruppo, spesso i minotauri lasciano scappare uno creatura, affinché diffonda il suo terribile racconto e attiri altri, che sperano di uccidere queste bestie, nei loro labirinti. Naturalmente, per i minotauri, questi aspiranti eroi rappresentano delle pietanze deliziose.

I minotauri si possono trovare anche al servizio di un mostro o una creatura malvagia più potente, e lo servono fintanto che possono cacciare e mangiare a loro piacimento. Generalmente questo significa fare la guardia a qualche potente oggetto o preziosa locazione, ma può anche significare lavorare come mercenario, dando la caccia ai nemici del padrone.

I minotauri sono combattenti relativamente diretti, usando le loro corna per incornare orribilmente le creature viventi più vicine quando cominciano a combattere.


\subsection{Mummie}

\medskip\index[Mostruario]{Mummia}\textbf{Mummia}

\textit{Media non morto, legale malvagio}

\textbf{FORZA} +3

\textbf{DESTREZZA} -1

\textbf{COSTITUZIONE} +2

\textbf{INTELLIGENZA} -2

\textbf{SAGGEZZA} +0

\textbf{CARISMA} +1

\textbf{Iniziativa} -1 -- \textbf{Difesa} 13

\textbf{Punti Ferita} 58 (9d8 + 18)

\textbf{Movimento} 6 m

\textbf{Tiri Salvezza}: Tempra +4, Riflessi +2, Volontà +8

\textbf{Vulnerabilità al Danno} fuoco

\textbf{Resistenze al Danno} da arma non magica

\textbf{Immunità al Danno} da Vuoto, veleno

\textbf{Immunità alle Condizioni} affascinato, avvelenato, paralizzato, affaticamento, spaventato

\textbf{Sensi} scurovisione 18 m

\textbf{Linguaggi} le lingue che conosceva in vita

\textbf{Sfida} 3 (700 PX)

\textit{\textbf{Natura Non Morta.}} Un mummia non ha bisogno di aria, cibo, bevande o sonno.

\textbf{Azioni}

\textit{\textbf{Multiattacco.}} La mummia può usare la sua Occhiata Temibile ed effettuare un attacco con il pugno putrefacente.

\textit{\textbf{Pugno Putrefacente.} Attacco con arma da mischia}: +7 a colpire, portata 1 m, un bersaglio.

\textit{Colpisce:} 10 (2d6 + 3) danni da botta più 10 (3d6) danni da Vuoto. Se il bersaglio è una creatura deve riuscire un Tiro Salvezza su Tempra 13 o venire maledetto dalla putrefazione della mummia. Il bersaglio maledetto non può recuperare Punti Ferita e i suoi Punti Ferita massimi diminuiscono di 10 (3d6) ogni 24 ore di durata della maledizione. Se la maledizione riduce i Punti Ferita massimi del bersaglio a 0, il bersaglio muore, e il suo corpo si tramuta in polvere. La maledizione dura finché non viene rimossa dall'incantesimo \textit{rimuovi maledizione} o altra magia.

\textit{\textbf{Occhiata Temibile.}} La mummia prende a bersaglio una creatura che possa vedere e si trovi entro 18 metri da lei. Se il bersaglio può vedere la mummia, deve riuscire un Tiro Salvezza su Volontà DC 12 contro questa magia o restare spaventato fino al termine del prossimo round della mummia. Se il bersaglio fallisce il Tiro Salvezza di 5 o più, è anche paralizzato per la stessa durata. Un bersaglio che riesca il Tiro Salvezza è immune all'Occhiata Terribile di tutte le mummie (ma non delle mummie sovrane) per le successive 24 ore.

\medskip\index[Mostruario]{Mummia Sovrana}\textbf{Mummia Sovrana}

\textit{Media non morto, legale malvagio}

\textbf{FORZA} +4

\textbf{DESTREZZA} +0

\textbf{COSTITUZIONE} +3

\textbf{INTELLIGENZA} +0

\textbf{SAGGEZZA} +4

\textbf{CARISMA} +3

\textbf{Iniziativa} +0 -- \textbf{Difesa} 25

\textbf{Punti Ferita} 97 (13d8 + 39)

\textbf{Movimento} 6 m

\textbf{Tiri Salvezza}: Tempra +12, Riflessi +6, Volontà +16

\textbf{Competenze} Religione +5, Storia +5

\textbf{Vulnerabilità al Danno} fuoco

\textbf{Immunità al Danno} da Vuoto, veleno; armi +1

\textbf{Immunità alle Condizioni} affascinato, avvelenato, paralizzato, affaticamento, spaventato

\textbf{Sensi} scurovisione 18 m

\textbf{Linguaggi} le lingue che conosceva in vita

\textbf{Sfida} 15 (13000 PX)

\textit{\textbf{Cuore della Mummia Sovrana.}} Come parte del rituale che crea una mummia sovrana, il cuore e le viscere della creatura vengono rimossi dal cadavere e piazzati all'interno di contenitori sigillati. Questi contenitori sono di solito fatti in pietra o ceramica, incisi o dipinti con geroglifici religiosi.

Finché il suo cuore avvizzito rimane intatto, la mummia sovrana non può essere permanentemente distrutta. Quando scende a 0 Punti Ferita, la mummia sovrana si riduce in polvere e si riforma a piena forza 24 ore più tardi, riemergendo dalla polvere in prossimità della giara sigillata che contiene il suo cuore. Per impedire che una mummia sovrana si riformi e distruggerla una volta per tutte, bisogna ridurne il cuore in cenere. Per questo motivo, la mummia sovrana di solito tiene il cuore e le viscere nascoste all'interno di una tomba nascosta.

Il cuore della mummia sovrana ha Difesa 5, 25 Punti Ferita e immunità a tutti i danni eccetto il fuoco.

\textit{\textbf{Incantesimi.}} La mummia ha CM 10. La sua caratteristica da incantatore è la Saggezza, +9 a colpire con attacchi da incantesimo. La mummia ha preparati i seguenti incantesimi: Trucchetti (a volontà): \textit{fiamma sacra, taumaturgia}

livello 1 (4 slot): \textit{comando, dardo tracciante, scudo della fede}

livello 2 (3 slot): \textit{arma spirituale, blocca persone, silenzio}

livello 3 (3 slot): \textit{animare morti, dissolvi magie}

livello 4 (3 slot): \textit{divinazione, guardiano della fede}

livello 5 (2 slot): \textit{contagio, piaga degli insetti}

livello 6 (1 slot): \textit{ferire}

\textit{\textbf{Natura Non Morta.}} Un mummia non ha bisogno di aria, cibo, bevande o sonno.

\textit{\textbf{Resistenza alla Magia.}} La mummia sovrana ha +1d6 ai Tiri Salvezza contro incantesimi o altri effetti magici.

\textit{\textbf{Rinvigorimento.}} Una mummia sovrana forma un nuovo corpo entro 24 ore se il suo cuore resta intatto, recuperando tutti i Punti Ferita e potendo agire nuovamente. Il nuovo corpo compare entro 1 metro dal cuore della mummia sovrana.

\textbf{Azioni}

\textit{\textbf{Multiattacco.}} La mummia può usare la sua Occhiata Temibile ed effettuare un attacco con il pugno putrefacente.

\textit{\textbf{Pugno Putrefacente.} Attacco con arma da mischia}: +22 a colpire, portata 1 m, un bersaglio.

\textit{Colpisce:} 14 (3d6 + 4) danni da botta più 21 (6d6) danni da Vuoto. Se il bersaglio è una creatura deve riuscire un Tiro Salvezza su Tempra 25 o venire maledetto dalla putrefazione della mummia. Il bersaglio maledetto non può recuperare Punti Ferita, e i suoi Punti Ferita massimi diminuiscono di 10 (3d6) ogni 24 ore di durata della maledizione. Se la maledizione riduce i Punti Ferita massimi del bersaglio a 0, il bersaglio muore, e il suo corpo si tramuta in polvere. La maledizione dura finché non viene rimossa dall'incantesimo \textit{rimuovere maledizione} o altra magia.

\textit{\textbf{Occhiata Temibile.}} La mummia prende a bersaglio una creatura che possa vedere e si trovi entro 18 metri da lei. Se il bersaglio può vedere la mummia, deve riuscire un Tiro Salvezza su Volontà DC 18 contro questa magia o restare spaventato fino al termine del prossimo round della mummia. Se il bersaglio fallisce il Tiro Salvezza di 5 o più, è anche paralizzato per la stessa durata. Un bersaglio che riesca il Tiro Salvezza è immune all'Occhiata Terribile di tutte le mummie (ma non delle mummie sovrane) per le successive 24 ore.

\textbf{Azioni Aggiuntive}

La mummia sovrana può effettuare 3 Azioni aggiuntive, scelte tra le opzioni seguenti. Può usare solo un'opzione leggendaria alla volta e solo al termine del turno di un'altra creatura. La mummia sovrana recupera le Azioni aggiuntive spese all'inizio del proprio round.

\textit{\textbf{Attaccare.}} La mummia sovrana effettua un attacco con il pugno putrefacente o usa la sua Occhiata Temibile.

\textit{\textbf{Incanalare Energia Negativa (Costa 2 Azioni).}} La mummia sovrana può scatenare magicamente l'energia negativa. Le creature entro 18 metri dalla mummia sovrana, comprese quelle dietro barriere o angoli, non possono recuperare Punti Ferita fino al termine del prossimo round della mummia sovrana.

\textit{\textbf{Parola Blasfema (Costa 2 Azioni).}} La mummia sovrana pronuncia una parola blasfema. Ciascuna creatura, esclusi i non morti, entro 3 metri dalla mummia sovrana e che possa udire questa frase magica deve riuscire un Tiro Salvezza di Tempra DC 16 o restare stordita fino al termine del prossimo round della mummia sovrana.

\textit{\textbf{Polvere Accecante.}} Polvere e sabbia accecanti turbinano magicamente intorno alla mummia sovrana. Ogni creatura entro 1 metro dalla mummia sovrana deve riuscire un Tiro Salvezza di Tempra DC 16 o restare accecata fino al termine del prossimo round della creatura.

\textit{\textbf{Turbine di Sabbia (Costa 2 Azioni).}} La mummia sovrana può trasformarsi magicamente in un turbine di sabbia, muovendosi di massimo 18 metri, e tornando poi alla sua forma normale. Mentre è in forma di turbine, la mummia sovrana è immune a tutti i danni, e non può essere afferrata, pietrificata, gettata prona, intralciata o stordita. L'equipaggiamento indossato o trasportato dalla mummia sovrana rimane in suo possesso.

\subsection{Naga}

\medskip\index[Mostruario]{Naga Guardiano}\textbf{Naga Guardiano}

\textit{Grande mostruosità, legale buono}

\textbf{FORZA} +4

\textbf{DESTREZZA} +4

\textbf{COSTITUZIONE} +3

\textbf{INTELLIGENZA} +3

\textbf{SAGGEZZA} +4

\textbf{CARISMA} +4

\textbf{Iniziativa} +4 -- \textbf{Difesa} 23

\textbf{Punti Ferita} 127 (15d10 + 45)

\textbf{Movimento} 12 m

\textbf{Tiri Salvezza}: Tempra +9, Riflessi +12, Volontà +12

\textbf{Immunità ai Danni} veleno

\textbf{Immunità alle Condizioni} affascinato, avvelenato

\textbf{Sensi} scurovisione 18 m

\textbf{Linguaggi} Celestiale, Comune

\textbf{Sfida} 10 (5.900 PX)

\textit{\textbf{Incantesimi.}} Il naga ha CM 11. La sua caratteristica da incantatore è la Saggezza (+8 a colpire con attacchi con incantesimo), e ha bisogno solo delle componenti verbali per lanciare i suoi incantesimi. Il naga prepara i seguenti incantesimi:

Trucchetti (a volontà): \textit{fiamma sacra, riparare, taumaturgia}

livello 1 (4 slot): \textit{comando, cura ferite, scudo della fede}

livello 2 (3 slot): \textit{bloccare persone, calmare emozioni}

livello 3 (3 slot): \textit{chiaroveggenza, scagliare maledizione}

livello 4 (3 slot): \textit{esilio, libertà di movimento}

livello 5 (2 slot): \textit{colpo infuocato, costrizione}

livello 6 (1 slot): \textit{visione del vero}

\textit{\textbf{Rinvigorimento.}} Se muore, il naga ritorna in vita in 1d6 giorni e recupera tutti i suoi Punti Ferita. Solo l'incantesimo \textit{desiderio} può impedire a questo tratto di funzionare.

\textbf{Azioni}

\textit{\textbf{Morso.} Attacco con arma da mischia}: +14 a colpire, portata 3 m, una creatura.

\textit{Colpisce:} 8 (1d8 + 4) danni perforanti, e il bersaglio deve effettuare un Tiro Salvezza di Tempra DC 15, subendo 45 (10d8) danni da veleno se fallisce il Tiro Salvezza, o la metà di questi danni se lo riesce.

\textit{\textbf{Sputare Veleno.} Attacco con arma a Distanza}: +14 a colpire, gittata 5m, una creatura.

\textit{Colpisce:} Il bersaglio deve effettuare un Tiro Salvezza su Tempra DC 15, subendo 45 (10d8) danni da veleno se fallisce il Tiro Salvezza, o la metà di questi danni se lo riesce.

\textbf{Ecologia}\\
Ambiente: Pianure Temperate\\
Organizzazione: Solitario, coppia o nido (3-6)\\
\textbf{Tesoro}: Standard\\
\textbf{Descrizione}\\
Sebbene abbiano un aspetto feroce, con scaglie brillanti, cappucci simili a quelli dei cobra e potenti corpi serpentini, i naga guardiani fungono da coscienziosi protettori di luoghi di eccezionale potere e sacralità. Spesso le loro scaglie sfoggiano disegni elaborati simili a quelli degli esotici serpenti della giungla. Un tipico naga guardiano raggiunge la lunghezza di 4,2 metri e un peso approssimativo di 175 kg.

Mentre alcuni naga guardiani aderiscono a pratiche esotiche di divinità antiche o dimenticate, altri sono semplicemente attratti da siti dalla spiccata bellezza naturale, quali templi su imponenti cascate, pinnacoli naturali e cime di montagne, custodendoli con il massimo della reverenza e del senso del dovere. Spesso questi naga si uniscono a fedi ancora attive, servendo come protettori di santuari o antichi tesori. Una coppia di naga può stabilirsi nei pressi di un sito che ritengono meritevole di protezione, covandovi una nidiata e crescendovi la prole. Quando i giovani raggiungono l'età adulta, possono scegliere di partire per cercare la propria casa o rimanere a proteggere la zona sorvegliata dai loro genitori. A volte, un naga guardiano che custodisce delle rovine od un tempio è solo l'ultimo di una successione di sentinelle che si sono avvicendate nel corso dei secoli. Queste sentinelle spesso prendono lo stesso nome dei loro predecessori sembrando un unico individuo eccezionalmente longevo.


\medskip\index[Mostruario]{Naga Spirituale}\textbf{Naga Spirituale}

\textit{Grande mostruosità, caotico malvagio}

\textbf{FORZA} +4

\textbf{DESTREZZA} +3

\textbf{COSTITUZIONE} +2

\textbf{INTELLIGENZA} +3

\textbf{SAGGEZZA} +2

\textbf{CARISMA} +3

\textbf{Iniziativa} +3 -- \textbf{Difesa} 19

\textbf{Punti Ferita} 75 (10d10 + 20)

\textbf{Movimento} 12 m

\textbf{Tiri Salvezza}: Tempra +8, Riflessi +10, Volontà +10

\textbf{Immunità al Danno} veleno

\textbf{Immunità alle Condizioni} affascinato, avvelenato

\textbf{Sensi} scurovisione 18 m

\textbf{Linguaggi} Abissale, Comune

\textbf{Sfida} 8 (3.900 PX)

\textit{\textbf{Incantesimi.}} Il naga ha CM 10. La sua abilità da incantatore è l'Intelligenza (+6 a colpire con attacchi con incantesimo), e ha bisogno solo delle componenti verbali per eseguire i suoi incantesimi. Il naga prepara i seguenti incantesimi:

Trucchetti (a volontà): \textit{illusione minore, mano magica, raggio di} \textit{gelo}

livello 1 (4 slot): \textit{charme su persone, individuazione del magico,} \textit{sonno}

livello 2 (3 slot): \textit{blocca persone, individuazione dei pensieri}

livello 3 (3 slot): \textit{fulmine, respirare sott'acqua}

livello 4 (3 slot): \textit{inaridire, porta dimensionale}

livello 5 (2 slot): \textit{dominare persone}

\textit{\textbf{Rinvigorimento.}} Se muore, il naga ritorna in vita in 1d6 giorni e recupera tutti i suoi Punti Ferita. Solo l'incantesimo \textit{desiderio} può impedire a questo tratto di funzionare.

\textbf{Azioni}

\textit{\textbf{Morso.} Attacco con arma da mischia}: +12 a colpire, portata 3 m, una creatura.

\textit{Colpisce:} 7 (1d8 + 4) danni perforanti, e il bersaglio deve effettuare un Tiro Salvezza di Tempra DC 13, subendo 31 (7d8) danni da veleno se fallisce il Tiro Salvezza, o la metà di questi danni se lo riesce.

\subsection{Oggetti Animati}

\medskip\index[Mostruario]{Armatura Animata}\textbf{Armatura Animata}

\textit{Media costrutto, disallineato}

\textbf{FORZA} +2

\textbf{DESTREZZA} +0

\textbf{COSTITUZIONE} +1

\textbf{INTELLIGENZA} -5

\textbf{SAGGEZZA} -4

\textbf{CARISMA} -5

\textbf{Iniziativa} +0 -- \textbf{Difesa} 19

\textbf{Punti Ferita} 33 (6d8 + 6)

\textbf{Movimento} 7 m

\textbf{Tiri Salvezza}: Tempra +2, Riflessi +0, Volontà -4

\textbf{Immunità al Danno} veleno

\textbf{Immunità alle Condizioni} accecato, affascinato, assordato, avvelenato, paralizzato, pietrificato, affaticamento, spaventato

\textbf{Sensi} vista cieca 18 m (cieco oltre questo raggio)

\textbf{Linguaggi} -

\textbf{Sfida} 1 (200 PX)

\textit{\textbf{Falso Aspetto.}} Mentre l'armatura rimane immobile, è indistinguibile da una normale armatura.

\textit{\textbf{Suscettibilità all'Anti Magia.}} L'armatura è inabile se si trova nell'area di un \textit{campo anti-magia}. Se è bersaglio di \textit{dissolvi} \textit{magie}, l'armatura deve riuscire un Tiro Salvezza su Tempra contro la DC del Tiro Salvezza dell'incantesimo o restare svenuta per 1 minuto.

\textbf{Azioni}

\textit{\textbf{Multiattacco.}} L'armatura effettua due attacchi da mischia.

\textit{\textbf{Schianto.} Attacco con arma da mischia}: +4 a colpire, portata 1 m, un bersaglio.

\textit{Colpisce:} 5 (1d6 + 2) danni da botta.

\medskip\index[Mostruario]{Spada Volante}\textbf{Spada Volante}

\textit{Piccola costrutto, disallineato}

\textbf{FORZA} +1

\textbf{DESTREZZA} +2

\textbf{COSTITUZIONE} +0

\textbf{INTELLIGENZA} -5

\textbf{SAGGEZZA} -3

\textbf{CARISMA} -5

\textbf{Iniziativa} +2 -- \textbf{Difesa} 18

\textbf{Punti Ferita} 17 (5d6)

\textbf{Movimento} 0 m, volo 15 m (fluttua)

\textbf{Tiri Salvezza} Tempra +1, Riflessi +3, Volontà -4

\textbf{Immunità al Danno} veleno

\textbf{Immunità alle Condizioni} accecato, affascinato, assordato, avvelenato, paralizzato, pietrificato, spaventato

\textbf{Sensi} vista cieca 18 m (cieco oltre questo raggio)

\textbf{Linguaggi} -

\textbf{Sfida} 1/4 (50 PX)

\textit{\textbf{Falso Aspetto.}} Mentre l'arma rimane immobile e non sta volando, è indistinguibile da una normale spada.

\textit{\textbf{Suscettibilità all'Anti Magia.}} La spada è inabile se si trova nell'area di un \textit{campo anti-magia}. Se è bersaglio di \textit{dissolvi} \textit{magie}, la spada deve riuscire un Tiro Salvezza su Tempra contro la DC del Tiro Salvezza dell'incantesimo o restare svenuta per 1 minuto.

\textbf{Azioni}

\textit{\textbf{Spada Lunga.} Attacco con arma da mischia}: +3 a colpire, portata 1 m, un bersaglio.

\textit{Colpisce:} 5 (1d8 + 1) danni taglienti.


\medskip\index[Mostruario]{Tappeto del Soffocamento}\textbf{Tappeto del Soffocamento}

\textit{Grande costrutto, disallineato}

\textbf{FORZA} +3

\textbf{DESTREZZA} +2

\textbf{COSTITUZIONE} +0

\textbf{INTELLIGENZA} -5

\textbf{SAGGEZZA} -4

\textbf{CARISMA} -5

\textbf{Iniziativa} +2 -- \textbf{Difesa} 13

\textbf{Punti Ferita} 33 (6d10)

\textbf{Movimento} 3 m

\textbf{Tiri Salvezza}: Tempra +4, Riflessi +2, Volontà -4

\textbf{Immunità al Danno} veleno

\textbf{Immunità alle Condizioni} accecato, affascinato, assordato, avvelenato, paralizzato, pietrificato, spaventato

\textbf{Sensi} vista cieca 18 m (cieco oltre questo raggio)

\textbf{Linguaggi} -

\textbf{Sfida} 2 (450 PX)

\textit{\textbf{Falso Aspetto.}} Mentre il tappeto resta immobile, è indistinguibile da un normale tappeto.

\textit{\textbf{Suscettibilità all'Anti Magia.}} Il tappeto è inabile mentre si trova nell'area di un \textit{campo anti-magia}. Se è il bersaglio di \textit{dissolvi} \textit{magie}, il tappeto deve riuscire un Tiro Salvezza di Tempra contro la DC del Tiro Salvezza dell'incantatore o cadere privo di sensi per 1 minuto.

\textit{\textbf{Trasferimento di Danno.}} Mentre afferra una creatura, il tappeto subisce solo la metà dei danni che gli sono inferti, e la creatura afferrata dal tappeto subisce l'altra metà.

\textbf{Azioni}

\textit{\textbf{Soffocare.} Attacco con arma da mischia}: +5 a colpire, portata 1 m, una creatura di taglia Media o inferiore.

\textit{Colpisce:} La creatura è afferrata (DC 13 per fuggire). Fino al termine dell'afferrare, il bersaglio è intralciato, accecato e rischia di soffocare, ma il tappeto non può soffocare un altro bersaglio. Inoltre, all'inizio di ciascun turno del bersaglio, il bersaglio subisce 10 (2d6 + 3) danni da botta.

\medskip\index[Mostruario]{Ogre}\textbf{Ogre}

\textit{Grande gigante, caotico malvagio}

\textbf{FORZA} +4

\textbf{DESTREZZA} -1

\textbf{COSTITUZIONE} +3

\textbf{INTELLIGENZA} -3

\textbf{SAGGEZZA} -2

\textbf{CARISMA} -2

\textbf{Iniziativa} -1 -- \textbf{Difesa} 12 (armatura di pelle)

\textbf{Punti Ferita} 59 (7d10 + 21)

\textbf{Movimento} 12 m

\textbf{Tiri Salvezza}: Tempra +6, Riflessi +0, Volontà +1

\textbf{Sensi} scurovisione 18 m

\textbf{Linguaggi} Comune, Gigante

\textbf{Sfida} 2 (450 PX)

\textbf{Azioni}

\textit{\textbf{Randello Pesante.} Attacco con arma da mischia}: +6 a colpire, portata 1 m, un bersaglio.

\textit{Colpisce:} 13 (2d8 + 4) danni da botta.

\textit{\textbf{Giavellotto.} Attacco con arma da mischia o a Distanza}: +6 a colpire, portata 1 m o gittata 12m, un bersaglio.

\textit{Colpisce:} 11 (2d6 + 4) danni perforanti.

\textbf{Ecologia}\\
Ambiente: Colline fredde o temperate\\
Organizzazione: Solitario, coppia, gruppo (3-4) o famiglia (5-16)\\
\textbf{Tesoro}: Standard (Armatura di Pelle, Randello Pesante, 4 Giavellotti, altro)\\
\textbf{Descrizione}\\
Nelle storie riguardanti gli ogre ci sono elementi orrendi: brutalità e ferocia, cannibalismo e tortura. Poi stupri, smembramenti, necrofilia, incesto, mutilazioni e altri esempi di crudeltà. Coloro che non hanno mai incontrato gli ogre ritengono queste storie un avvertimento. Chi è sopravvissuto ad un simile incontro sa che le storie sono niente in confronto alla realtà.

Gli ogre godono della sofferenza altrui. Se non hanno a disposizione le razze più piccole da schiacciare fra le loro grasse mani o da violare in amplessi violenti, si divertono fra loro. Per gli ogre non esiste tabù. Si potrebbe pensare che, lasciata a sé stessa, una tribù di ogre si farebbe a pezzi da sola e che soltanto i più forti sopravvivrebbero: se c'è una cosa che gli ogre rispettano, però, è la famiglia.

Le tribù ogre sono conosciute come famiglie, e molte delle loro deformità sono causate dalla pratica comune dell'incesto. Il capo della tribù è spesso il padre, ma in alcuni casi un'ogre femmina è in grado di reclamare il titolo di madre. Le tribù ogre litigano fra loro, cosa che li tiene impegnati ed impedisce loro di tormentare i loro vicini. Di quando in quando, però, emerge un patriarca particolarmente violento o temuto, capace di unire più famiglie sotto il suo comando.

Le regioni abitate degli ogre sono luoghi tristi e degradati, dato che questi giganti vivono nello squallore e non sentono il bisogno di essere in armonia con quanto li circonda. Il confine fra le terre civilizzate e quelle degli ogre è un luogo di disperazione abitato da reietti, dove vivono gli Ogremanni, progenie deformi che nascono dalle razzie che gli ogre effettuano nelle terre degli umani.

I giochi degli ogre sono violenti e crudeli: le vittime utilizzate come giocattolo sono fortunate a morire il primo giorno. Il crudele senso dell'umorismo degli ogre è il solo caso in cui mostrano di possedere creatività: i metodi e gli strumenti di tortura ogre sembrano usciti dagli incubi.

La grande forza e la mancanza di immaginazione li rendono particolarmente adatti ai lavori pesanti, nelle miniere, come fabbri o nel disboscamento. I giganti più potenti (soprattutto quelli delle Colline e delle Rocce) spesso soggiogano le famiglie ogre perché diventino loro servitori.

Un ogre adulto è alto sui 3 metri e pesa circa 325 kg.


\medskip\index[Mostruario]{Ombra}\textbf{Ombra}

\textit{Media non morto, caotico malvagio}

\textbf{FORZA} -2

\textbf{DESTREZZA} +2

\textbf{COSTITUZIONE} +1

\textbf{INTELLIGENZA} -2

\textbf{SAGGEZZA} +0

\textbf{CARISMA} -1

\textbf{Iniziativa} +2 -- \textbf{Difesa} 13

\textbf{Punti Ferita} 16 (3d8 + 3)

\textbf{Movimento} 12 m

\textbf{Tiri Salvezza}: Tempra +3, Riflessi +3, Volontà +4

\textbf{Competenze} Muoversi Silenziosamente / Nascondersi +4 (+6 a luce fioca o oscurità)

\textbf{Vulnerabilità al Danno} da Luce

\textbf{Resistenze al Danno} acido, freddo, fulmine, fuoco, suono; da arma non magica

\textbf{Immunità al Danno} da Vuoto, veleno

\textbf{Immunità alle Condizioni} afferrato, avvelenato, intralciato, paralizzato, pietrificato, prono, affaticamento, spaventato

\textbf{Sensi} scurovisione 18 m

\textbf{Linguaggi} -

\textbf{Sfida} 1/2 (100 PX)

\textit{\textbf{Amorfo.}} L'ombra può muoversi attraverso uno spazio stretto fino a 3 centimetri senza stringersi.

\textit{\textbf{Debolezza alla Luce del Sole.}} Mentre si trova alla luce del sole, l'ombra ha -1d6 ai tiri per colpire, le prove di competenza e i Tiri Salvezza.

\textit{\textbf{Spirito dell'Ombra.}} Mentre si trova in una zona di luce fioca l'Ombra rigenera 5 Punti Ferita all'inizio del suo round, se si trova in una zona di oscurità rigenera 10 Punti Ferita all'inizio del suo round e può diventare invisibile usando 1 Azione. Spirito dell'Ombra aumenta il Grado di Sfida dell'Ombra di 1.

\textit{\textbf{Furtività d'Ombra.}} Quando si trova a luce fioca o all'oscurità, l'ombra può effettuare l'azione Nascondersi come azione bonus.

\textit{\textbf{Natura Non Morta.}} Un'ombra non necessita aria, cibo, bevande o sonno.

\textbf{Azioni}

\textit{\textbf{Risucchio di Forza.} Attacco con arma da mischia}: +4 a colpire, portata 1 m, una creatura.

\textit{Colpisce:} 9 (2d6 + 2) danni da Vuoto, e il punteggio di Forza del bersaglio viene ridotto di 1. Il bersaglio muore se ciò riduce la sua Forza a -5. Altrimenti, la riduzione resta finché il bersaglio non riposa 8 ore.

Se un umanoide non malvagio muore a causa di questo attacco, entro 1d4 ore dal suo cadavere si animerà una nuova ombra.

\textbf{Ecologia}
Ambiente: Qualsiasi\\
Organizzazione: Solitario, coppia, gruppo (3–6) o sciame (7–12)\\
\textbf{Tesoro}: Standard\\

\textbf{Descrizione}\\
La malvagia ombra si muove lungo il confine tra il buio delle tenebre e la dura verità della luce. L’ombra preferisce infestare le rovine che la civiltà si lascia alle spalle, dove dà la caccia alle creature viventi tanto sciocche da incappare nel suo territorio. L’ombra è un orribile non morto, e come tale non ha scopi o motivazioni apparenti oltre a risucchiare forza vitale e vitalità dagli esseri viventi.


\medskip\index[Mostruario]{Omuncolo}\textbf{Omuncolo}

\textit{Minuscola costrutto, neutrale}

\textbf{FORZA} -3

\textbf{DESTREZZA} +2

\textbf{COSTITUZIONE} +0

\textbf{INTELLIGENZA} +0

\textbf{SAGGEZZA} +0

\textbf{CARISMA} -2

\textbf{Iniziativa} +2 -- \textbf{Difesa} 14

\textbf{Punti Ferita} 5 (2d4)

\textbf{Movimento} 6 m, volo 12 m

\textbf{Tiri Salvezza}: Tempra +0, Riflessi +4, Volontà +1

\textbf{Immunità al Danno} veleno

\textbf{Immunità alle Condizioni} affascinato, avvelenato

\textbf{Sensi} scurovisione 18 m, vista cieca 3 m

\textbf{Linguaggi} comprende le lingue del suo creatore ma non può parlare

\textbf{Sfida} 0 (10 PX)

\textit{\textbf{Legame Telepatico.}} Mentre l'omuncolo si trova sullo stesso piano di esistenza del suo padrone, può comunicare magicamente al suo padrone quello che percepisce, e i due possono comunicare telepaticamente.

\textbf{Azioni}

\textit{\textbf{Morso.} Attacco con arma da mischia}: +4 a colpire, portata 1 m, una creatura.

\textit{Colpisce:} 1 danno perforante, e il bersaglio deve riuscire un Tiro Salvezza di Tempra DC 10 o restare avvelenato per 1 minuto. Se il Tiro Salvezza viene fallito di 5 o più, il bersaglio resta invece avvelenato per 5 (1d10) minuti e mentre è avvelenato in questo modo è anche privo di sensi.

\medskip\index[Mostruario]{Oni}\textbf{Oni}

\textit{Grande gigante, legale malvagio}

\textbf{FORZA} +4

\textbf{DESTREZZA} +0

\textbf{COSTITUZIONE} +3

\textbf{INTELLIGENZA} +2

\textbf{SAGGEZZA} +1

\textbf{CARISMA} +2

\textbf{Iniziativa} +2 -- \textbf{Difesa} 20 (cotta di maglia)

\textbf{Punti Ferita} 110 (13d10 + 39)

\textbf{Movimento} 9 m, volo 9 m

\textbf{Tiri Salvezza}: Tempra +7, Riflessi +4, Volontà +6

\textbf{Competenze} Arcano +5, Ingannare +8, Consapevolezza +4

\textbf{Sensi} scurovisione 18 m

\textbf{Linguaggi} Comune, Gigante

\textbf{Sfida} 7 (2.900 PX)

\textit{\textbf{Armi Magiche.}} Gli attacchi con armi dell'oni sono magici.

\textit{\textbf{Incantesimi Innati.}} La caratteristica da incantatore dell'oni è il Carisma. L'oni può lanciare questi incantesimi in maniera innata, senza bisogno di componenti materiali:

A volontà: \textit{invisibilità, oscurità}

1/giorno: \textit{charme su persone, cono di freddo, forma gassosa,}
\textit{sonno}

\textit{\textbf{Rigenerazione.}} Se ha almeno 1 punto ferita, l'oni recupera 10 Punti Ferita all'inizio del suo round.

\textbf{Azioni}

\textit{\textbf{Multiattacco.}} L'oni effettua due attacchi, con gli artigli o con il falcione.

\textit{\textbf{Artiglio (Solo Forma di Oni).} Attacco con arma da mischia}: +11 a colpire, portata 1 m, un bersaglio. \textit{Colpisce:} 8 (1d8 + 4) danni taglienti.

\textit{\textbf{Falcione.} Attacco con arma da mischia}: +7 a colpire, portata 3 m, un bersaglio.

\textit{Colpisce:} 15 (2d10 + 4) danni taglienti, o 9 (1d10 + 4) danni taglienti in forma Piccola o Media.

\textit{\textbf{Mutare Forma.}} L'oni può trasformarsi magicamente in un umanoide Piccolo o Medio, in un gigante Grande, o tornare alla sua vera forma. A parte la taglia, le sue statistiche sono le stesse in ciascuna forma. L'unico equipaggiamento che viene trasformato è il falcione, che rimpicciolisce in modo da essere impugnato anche in forma umanoide. Se l'oni muore, ritorna alla sua vera forma, e il falcione ritorna alla sua taglia originale.

\medskip\index[Mostruario]{Orchetto}\textbf{Orchetto}

\textit{Media umanoide (orco), caotico neutrale}

\textbf{FORZA} +2

\textbf{DESTREZZA} +1

\textbf{COSTITUZIONE} +2

\textbf{INTELLIGENZA} +0

\textbf{SAGGEZZA} +0

\textbf{CARISMA} +0

\textbf{Iniziativa} +2 -- \textbf{Difesa} 14 (armatura di pelle)

\textbf{Punti Ferita} 12 (2d6 + 6)

\textbf{Movimento} 9 m

\textbf{Tiri Salvezza}: Tempra +2, Riflessi +2, Volontà +1

\textbf{Competenze} Intimidire +1

\textbf{Sensi} scurovisione 18 m

\textbf{Linguaggi} Comune, Goblinoide

\textbf{Sfida} 1/2 (100 PX)

\textbf{Azioni}

\textit{\textbf{Spada.} Attacco con arma da mischia}: +4 a colpire, portata 1 m, un bersaglio.

\textit{Colpisce:} 8 (1d12 + 2) danni taglienti.

\textit{\textbf{Giavellotto.} Attacco con arma da mischia o a Distanza}: +5 a colpire, portata 1 m o gittata 12m, un bersaglio. \textit{Colpisce:} 6 (1d6 + 3) danni perforanti.

\textbf{Ecologia}\\
Ambiente: Colline e montagne temperate o sotterranei\\
Organizzazione: solitario, gruppo (2-4), squadra (11-20 più 2 sergenti di 3° livello e 1 capo di 3°-6° livello) o banda \\
\textbf{Tesoro}: Equipaggiamento da PNG (Armatura di Cuoio Borchiato, Spada, 4 Giavellotti, altro tesoro)\\
\textbf{Descrizione}\\
Gli orchetti sono una razza creata da Cattalm come esperimento con lo scopo di verificare se una creatura più intelligente ma altrettanto feroce degli orchi avrebbe potuto essere dominante.
L'esperimento è stato un discreto successo con gli orchetti che hanno fondato regni e conquistato diverse regioni. La spinta caotica con il passare del tempo, l'acculturamento, il diventare stanziali e l'evoluzione della società ha portato gli orchetti sempre più fuori dalle spire di Cattalm, anche se non toglie che molti aspetti "barbari" sono rimasti nella cultura tradizionale.
Un orco maschio adulto è alto 1,6 metri e pesa circa 60 kg. Gli orchetti e gli umani possono accoppiarsi, anche se di solito ciò avviene durante le razzie, e non come unione consensuale.

\medskip\index[Mostruario]{Orco}\textbf{Orco}

\textit{Media umanoide (orco), caotico malvagio}

\textbf{FORZA} +3

\textbf{DESTREZZA} +1

\textbf{COSTITUZIONE} +3

\textbf{INTELLIGENZA} -2

\textbf{SAGGEZZA} +0

\textbf{CARISMA} +0

\textbf{Iniziativa} +1 -- \textbf{Difesa} 14 (armatura di pelle)

\textbf{Punti Ferita} 18 (3d8 + 6)

\textbf{Movimento} 9 m

\textbf{Tiri Salvezza}: Tempra +3, Riflessi +1, Volontà +2

\textbf{Competenze} Intimidire +2

\textbf{Sensi} scurovisione 18 m

\textbf{Linguaggi} Comune, Goblinoide

\textbf{Sfida} 1 (100 PX)

\textit{\textbf{Aggressivo.}} Come azione bonus, l'orco può muoversi fino a metà del suo movimento verso una creatura ostile che possa vedere.

\textbf{Azioni}

\textit{\textbf{Ascia Bipenne.} Attacco con arma da mischia}: +5 a colpire, portata 1 m, un bersaglio.

\textit{Colpisce:} 9 (1d12 + 3) danni taglienti.

\textit{\textbf{Giavellotto.} Attacco con arma da mischia o a Distanza}: +5 a colpire, portata 1 m o gittata 12m, un bersaglio. \textit{Colpisce:} 6 (1d6 + 3) danni perforanti.

\textbf{Ecologia}\\
Ambiente: Colline e montagne temperate o sotterranei\\
Organizzazione: solitario, gruppo (2-4), squadra (11-20 più 2 sergenti di 3° livello e 1 capo di 3°-6° livello) o banda \\
\textbf{Tesoro}: Equipaggiamento da PNG (Armatura di Cuoio Borchiato, Falcione, 4 Giavellotti, altro tesoro)\\
\textbf{Descrizione}\\
La differenza principale fra gli orchi e gli umanoidi civilizzati, oltre alla loro forza bruta ed all'intelligenza inferiore, è il loro carattere. Come cultura, gli orchi sono violenti ed aggressivi, ed il forte domina il debole attraverso paura e brutalità. Prendono ciò che vogliono con la forza e non si fanno scrupoli a prendere interi villaggi come schiavi se ne hanno la possibilità. Non si curano delle comodità, ed i loro villaggi e campi tendono ad essere luoghi sporchi e precari, pieni di risse fra ubriachi, arene per i combattimenti ed altri divertimenti sadici. Privi della pazienza necessaria a coltivare e capaci di allevare solo gli animali più robusti ed autosufficienti, gli orchi ritengono più semplice prendere agli altri il frutto del loro lavoro. Sono arroganti e lesti ad infuriarsi quando sfidati, ma si preoccupano dell'onore solo finché farlo porta loro beneficio.

Un orco maschio adulto è alto 2 metri e pesa circa 115 kg. Gli orchi e gli umani possono accoppiarsi, anche se di solito ciò avviene durante le razzie, e non come unione consensuale. Molte tribù orchesche allevano i mezzorchi di proposito, dato che sono ottimi strateghi e capitribù.

Per quanto la vulgata dica che gli orchi siano stati creati da Cattalm per distruggere e portare caos è anche vero che molto spesso sono vittima di pregiudizi e giudizi sommari. Non tutti gli orchi sono uguali e non solo fisicamente, singoli orchi se non intere tribù vivono in maniera "normale, civilizzata" la loro esistenza eppure in nessun stato di Yeru sono previste pene per chi uccide un orco.

\medskip\index[Mostruario]{Orrore Arrampicamuri}\textbf{Orrore Arrampicamuri}

\textit{Grande mostruosità, disallineato}

\textbf{FORZA} +4

\textbf{DESTREZZA} +0

\textbf{COSTITUZIONE} +2

\textbf{INTELLIGENZA} -2

\textbf{SAGGEZZA} +1

\textbf{CARISMA} -2

\textbf{Iniziativa} +1 -- \textbf{Difesa} 15

\textbf{Punti Ferita} 75 (10d10 + 25)

\textbf{Movimento} 9 m, scalare 9 m

\textbf{Tiri Salvezza}: Tempra +5, Riflessi +3, Volontà +4

\textbf{Sensi} scurovisione 3 m, vista cieca 18 m

\textbf{Linguaggi} Orrore Arrampicamuri

\textbf{Sfida} 3 (700 PX)

\textit{\textbf{Senso Radar.}} l'Orrore Arrampicamuri non può usare vista cieca se è assordato.

\textbf{Azioni}

\textit{\textbf{Multiattacco.}} L'Orrore Arrampicamuri effettua due attacchi con gli artigli uncinati.

\textit{\textbf{Artigli.} Attacco con arma da mischia}: +7 a colpire, portata 1 m, un bersaglio.

\textit{Colpisce:} 10 (2d6 + 4) danni perforanti, 1 danno da Sanguinamento.

\textbf{Ecologia}\\
\textbf{Ambiente: Sottosuolo}
Organizzazione: Solitario, coppia o branco (3-8)\\
\textbf{Tesoro}: Accidentale\\
\textbf{Descrizione}\\
L'Orrore Arrampicamuri è un feroce predatore del sottosuolo, difende aggressivamente i suoi territori di caccia. Le caverne sotterranee in cui queste creature risiedono rimbombano dei colpi e dei fruscii dei loro uncini quando queste creature si arrampicano sulle rupi rocciose o sulle pareti delle caverne.

Un Orrore Arrampicamuri è una creatura mostruosa dalla testa simile a quella di un avvoltoio e dal torace di un enorme scarabeo, protetto da un esoscheletro tempestato di protuberanze ossee aguzze. Trae il suo nome oltre che dall'orrendo aspetto dal fatto che usando gli arti lunghi e muscolosi che terminano con dei micidiali artigli uncinati ricurvi stia arrampicato sulle pareti.

\textit{Echi nell'Oscurità}. Gli Orrore Arrampicamuri comunicano colpendo il loro esoscheletro o le superfici rocciose circostanti con i loro uncini. Ciò che agli altri appare come un frastuono casuale è in realtà un elaborato linguaggio che soltanto gli Orrore Arrampicamuri capiscono e il cui eco si diffonde per chilometri e chilometri nell'sottosuolo.

\textit{Branco di Predatori}. Gli Orrore Arrampicamuri sono creature onnivore: si nutrono di funghi, licheni, vegetali e di qualsiasi creatura riescano a catturare. Grazie agli arti uncinati, gli orrori beneficiano di un'ottima presa sulle superfici rocciose e usano le loro abilità da scalatori per tendere imboscate alle prede dall'alto. Vanno a caccia in branco e collaborano per affrontare gli avversari più grossi e pericolosi. Se una battaglia va male, un Orrore Arrampicamuri si arrampica rapidamente lungo la parete di una caverna per fuggire.

\textit{Clan Solidali}. Gli orrori uncinati vivono in vasti gruppi familiari o clan. Ogni clan è retto dalla femmina più anziana, che solitamente pone il suo compagno a capo dei cacciatori del clan. Gli Orrore Arrampicamuri depongono le uova in un'area centrale e ben difesa delle caverne usate come tana.

\medskip\index[Mostruario]{Orsogufo}\textbf{Orsogufo}

\textit{Grande bestia, disallineato}

\textbf{FORZA} +5

\textbf{DESTREZZA} +1

\textbf{COSTITUZIONE} +3

\textbf{INTELLIGENZA} -4

\textbf{SAGGEZZA} +1

\textbf{CARISMA} -2

\textbf{Iniziativa} +1 -- \textbf{Difesa} 15

\textbf{Punti Ferita} 59 (7d10 + 21)

\textbf{Movimento} 12 m

\textbf{Tiri Salvezza}: Tempra +10, Riflessi +5, Volontà +2

\textbf{Competenze} Consapevolezza +3

\textbf{Sensi} scurovisione 18 m

\textbf{Linguaggi} -

\textbf{Sfida} 3 (700 PX)

\textit{\textbf{Olfatto e Vista Affinati.}} L'orsogufo ha +1d6 nelle prove di Saggezza (Consapevolezza) basate su olfatto o vista.

\textbf{Azioni}

\textit{\textbf{Multiattacco.}} L'orsogufo effettua due attacchi: uno con il becco e uno con gli artigli.

\textit{\textbf{Artigli.} Attacco con arma da mischia}: +9 a colpire, portata 1 m, un bersaglio.

\textit{Colpisce:} 14 (2d8 + 5) danni taglienti.

\textit{\textbf{Becco.} Attacco con arma da mischia}: +9 a colpire, portata 1 m, una creatura.

\textit{Colpisce:} 10 (1d10 + 5) danni perforanti.

\textbf{Ecologia}\\
\textbf{Ambiente: Foreste Temperate}
Organizzazione: Solitario, coppia o branco (3-8)\\
\textbf{Tesoro}: Accidentale\\
\textbf{Descrizione}\\
Le origini dell'orsogufo sono oggetto di dibattito fra gli studiosi delle creature mostruose. La maggior parte di essi concorda che fu un Mago, in passato, a crearne il primo esemplare unendo un orso con un gufo gigante; forse come esperimento su qualche folle concetto della natura della vita, ma più probabilmente a causa della sua totale pazzia. Quale che fosse lo scopo originale di una creazione tanto folle come l'orsogufo, la creatura ha iniziato a riprodursi, ed è divenuta uno dei predatori più conosciuto delle zone boschive.

Gli orsigufo sono selvaggi predatori, noti per il loro pessimo temperamento, la loro aggressività e la loro ferocia. Tendono ad attaccare tutto ciò che si muove loro davanti, anche se questo non mostra intenzioni bellicose. Molti studiosi che hanno incontrato queste creature nelle terre selvagge hanno notato che hanno sempre occhi iniettati di sangue che ruotano tutto attorno poco prima di un attacco. Questo è generalmente visto come segno di follia, che suggerisce che tutti gli orsigufo nascano con un bisogno patologico di combattere ed uccidere, ma i ricercatori più realisti ritengono sia dovuto alla struttura dei loro occhi acuti.

Gli orsigufo abitano le zone più interne e nascoste dei boschi, e preparano le loro tane all'interno di foreste intricate o di buie e profonde caverne. Possono cacciare sia di giorno che di notte, a seconda delle abitudini delle prede che popolano i territori circostanti alla loro tana.

Gli orsigufo adulti vivono in coppia e cacciano le prede in branco, lasciando i cuccioli nelle tane. In una tana si possono trovare di solito 1d6 cuccioli, che possono valere fino a 750 mo nei mercati cittadini.

Anche se è pressoché impossibile addomesticarli a causa della loro natura selvaggia, gli orsigufo possono essere sfruttati come guardiani di un territorio specifico, sempre che vengano lasciati liberi di spostarsi al suo interno per cacciare. Gli addestratori professionisti chiedono fino a 2000 mo per addestrare un orsogufo perché diventi un guardiano che obbedisca a comandi semplici (DC 23 per un cucciolo di orsogufo, DC 30 per un orsogufo adulto).\\

\textit{\textbf{Variante}}: \textbf{Orsogufo Polare}\index{Orsogufo Polare}\\
Questo orsogufo è presente nelle regioni artiche o montane innevate. A differenza del normale orsogufo è più robusto e forte. Ha 85 PF, +10 al colpire, 21 di danno ad artiglio +1 da Sanguinamento, 15 di danno con becco. GS 4

\medskip\index[Mostruario]{Orsogufo Saggio}\textbf{Orsogufo Saggio}

\textit{Grande mostruosità, neutrale}

\textbf{FORZA} +3

\textbf{DESTREZZA} +1

\textbf{COSTITUZIONE} +2

\textbf{INTELLIGENZA} +3

\textbf{SAGGEZZA} +3

\textbf{CARISMA} +1

\textbf{Iniziativa} +3 -- \textbf{Difesa} 15

\textbf{Punti Ferita} 45 (7d10 + 10)

\textbf{Movimento} 12 m

\textbf{Tiri Salvezza}: Tempra +10, Riflessi +5, Volontà +4

\textbf{Competenze} Consapevolezza +3

\textbf{Sensi} scurovisione 18 m

\textbf{Linguaggi} comprende e legge i seguenti linguaggi: Comune, Celestiale, Infernale, Nanico, Elfico, Orchesco, Gigante

\textbf{Sfida} 3 (700 PX)

\textit{\textbf{Olfatto e Vista Affinati.}} L'orsogufo saggio ha +1d6 nelle prove di Saggezza (Consapevolezza) basate su olfatto o vista.

\textit{\textbf{Incantesimi Innati.}} La caratteristica da incantatore dell'Orsogufo saggio è l'Intelligenza. L'Orsogufo saggio può lanciare in maniera innata i seguenti incantesimi, senza aver bisogno di componenti materiali:

A volontà: \textit{Mano magica}

\textbf{Azioni}

\textit{\textbf{Multiattacco.}} L'orsogufo saggio effettua due attacchi: uno con il becco e uno con gli artigli.

\textit{\textbf{Artigli.} Attacco con arma da mischia}: +7 a colpire, portata 1 m, un bersaglio.

\textit{Colpisce:} 14 (2d8 + 5) danni taglienti.

\textit{\textbf{Becco.} Attacco con arma da mischia}: +7 a colpire, portata 1 m, una creatura.

\textit{Colpisce:} 10 (1d10 + 5) danni perforanti.

\textbf{Ecologia}\\
\textbf{Ambiente: Foreste Temperate}
Organizzazione: Solitario, coppia o branco (3-8)\\
\textbf{Tesoro}: Standard + 10\% Manuali e Tomi\\
\textbf{Descrizione}\\
Le origini dell'orsogufo saggio sono misteriose quanto quelli del suo parente non saggio ma gli appassionati di queste creature li fanno discendere direttamente da Nethergal come variante dell'orsogufo originale.
Solitamente l'Orsogufo saggio ama circondarsi di libri ed adora la compagnia di altri saggi ma non disdegna i racconti di avventurieri e le avvincenti ballate dei cantastorie. L'Orsogufo saggio ha un vero talento per le lingue e pur non potendo parlare in maniera comprensibile ad un uomo riesce a comprendere tantissime lingue parlate e scritte ed in pochi giorni è in grado di apprenderne di nuove (come Vantaggio Lingua universale) sia parlate che scritte. L'Orsogufo saggio è in grado di leggere qualsiasi lingua o codice se ha modo di studiarlo per 3 giorni.
Solitamente più deboli e fragili del parente stretto sono comunque esseri temibili in combattimento.
Di preferenza un Orsogufo saggio non attacca se non per difesa e cerca un approccio il più tattico e utile possibile. Un tratto caratteristico degli Orsigufi saggi è una sciarpa portata intorno all'assente collo. Uccidere un Orsogufo saggio è un affronto ai Devoti e Seguaci di Nethergal, è anche capitato che il Patrono stesso togliesse la capacità di comunicare a coloro si sono macchiati di efferatezze con le sue creature preferite.

Addestrare un Orsogufo saggio è molto più facile di un Orsogufo ma l'alta intelligenza della creatura lo spingerà ad avere un rapporto alla pari o come famiglio piuttosto.

L'incantesimo Mano magica è solitamente usato per sfogliare i tomi più delicati e per scrivere, anche se con estrema lentezza.

\medskip\index[Mostruario]{Otyugh}\textbf{Otyugh}

\textit{Grande aberrazione, neutrale}

\textbf{FORZA} +3

\textbf{DESTREZZA} +0

\textbf{COSTITUZIONE} +4

\textbf{INTELLIGENZA} +2

\textbf{SAGGEZZA} +1

\textbf{CARISMA} -2

\textbf{Iniziativa} +0 -- \textbf{Difesa} 17

\textbf{Punti Ferita} 114 (12d10 + 48)

\textbf{Movimento} 9 m

\textbf{Tiri Salvezza}: Tempra +3, Riflessi +2, Volontà +6

\textbf{Sensi} scurovisione 36 m

\textbf{Linguaggi} Otyugh

\textbf{Sfida} 5 (1.800 PX)

\textit{\textbf{Telepatia Limitata.}} L'otyugh può trasmettere magicamente dei semplici messaggi e immagini a qualsiasi creatura entro 36 metri da esso e che possa comprendere una lingua. Questa forma di telepatia non permette alla creatura ricevente di rispondere telepaticamente.

\textbf{Azioni}

\textit{\textbf{Multiattacco.}} L'otyugh effettua tre attacchi: uno con il morso e due con i tentacoli.

\textit{\textbf{Morso.} Attacco con arma da mischia}: +9 a colpire, portata 1 m, un bersaglio.

\textit{Colpisce:} 12 (2d8 + 3) danni perforanti. Se il bersaglio è una creatura, deve riuscire un Tiro Salvezza di Tempra DC 15 contro malattia o restare avvelenato finché la malattia non viene curata. Ogni 24 ore successive, il bersaglio deve ripetere il Tiro Salvezza, riducendo il suo massimo di Punti Ferita di 5 (1d10) se lo fallisce. Se il Tiro Salvezza riesce, la malattia è passata. Il bersaglio muore se la malattia riduce i suoi Punti Ferita massimi a 0.

Questa riduzione dei Punti Ferita massimi del personaggio, perdura finché la malattia non viene curata.

\textit{\textbf{Tentacolo.} Attacco con arma da mischia}: +9 a colpire, portata 3 m, un bersaglio.

\textit{Colpisce:} 7 (1d8 + 3) danni da botta più 4 (1d8) danni perforanti. Se il bersaglio è di taglia Media o inferiore, è afferrato (DC 13 per fuggire) e intralciato fino al termine dell'afferrare. L'otyugh ha due tentacoli, ciascun dei quali può afferrare un bersaglio diverso.

\textit{\textbf{Schianto di Tentacolo.}} L'otyugh schianta le creature afferrate dai suoi tentacoli, l'una contro l'altra o sul pavimento. Ogni creatura deve riuscire un Tiro Salvezza di Tempra DC 14 o subire 10 (2d6 + 3) danni da botta e restare stordita fino al termine del prossimo round dell'otyugh. Se il Tiro Salvezza riesce, il bersaglio subisce la metà dei danni da botta e non è stordito.

\textbf{Ecologia}\\
Ambiente: Qualsiasi Sotterraneo\\
Organizzazione: Solitario, coppia o gruppo (3-4)\\
\textbf{Tesoro}: Standard\\
\textbf{Descrizione}\\
Gli otyugh sono creature particolarmente luride ed orride che vivono in luoghi che le persone sane di mente tendono ad evitare. Le loro tane si trovano nelle fogne, nei pozzi neri, nelle discariche e nelle paludi più mefitiche: più un luogo è sporco, più attira gli otyugh. Amano il ruolo dello spazzino, e vagano per le caverne sotterranee in cerca di nuovi bocconcini in mezzo ai rifiuti. Una volta trovati, si ingozzano e riportano alla loro tana quello che non riescono a consumare in una volta sola. Gli otyugh passano parecchio tempo nelle loro luride tane, che riempiono di carogne e letame, che rilasciano effluvi mefitici.

Le creature intelligenti che vivono nelle zone sotterranee vicino agli otyugh a volte formano alleanze di convenienza con essi. Forniscono loro rifiuti e carne cruda agli otyugh, rendendoli un vero e proprio mezzo di smaltimento. In cambio, gli otyugh lasciano in pace i loro benefattori, non li attaccano e possono anche fare da guardiani.

La cosa che la maggior parte delle razze trova terrificante degli otyugh non è la loro dieta o l'odore delle loro tane, ma il fatto che creature con i loro gusti non siano solo spazzini senza cervello. Gli otyugh si mostrano infatti sorprendentemente intelligenti, ed amano formare alleanze con coloro che li riforniscono di cibi più raffinati di letame e sporcizia. La maggior parte degli otyugh si rende conto che le altre creature li trovano rivoltanti, ma sono pochi quelli a cui importa davvero.

\medskip\index[Mostruario]{Panopticon}\textbf{Panopticon}

\textit{Grande aberrazione, malvagia}

\textbf{FORZA} +0

\textbf{DESTREZZA} +1

\textbf{COSTITUZIONE} +2

\textbf{INTELLIGENZA} +3

\textbf{SAGGEZZA} +2

\textbf{CARISMA} +2

\textbf{Iniziativa} +5 -- \textbf{Difesa} 26

\textbf{Punti Ferita} 82 (11d8 + 38)

\textbf{Movimento} 1 m, volo 10 metri (buono)

\textbf{Tiri Salvezza}: Tempra +6, Riflessi +7, Volontà +10

\textbf{Resistenza}: acido, elettricità

\textbf{Sensi} scurovisione 36 m, visione del vero 18m

\textbf{Linguaggi} telepatia 50 m

\textbf{Sfida} 12 (8.400 PX)

\textbf{Azioni}

\textit{\textbf{Multiattacco.}} Il Panopticon può attaccare con due corti tentacoli.

\textit{\textbf{Tentacolo.} Attacco con arma da mischia}: +12 a colpire, portata 1 m, un bersaglio.

\textit{Colpisce:} 6 (1d6 + 3) danni da taglio perforanti.

\textit{\textbf{Colui che tutto vede}}. Il Panopticon può attivare uno dei suoi tentacoli occhiuti (2 Azioni).

\textit{Quello che gela}: l'occhio punta un bersaglio entro 18 metri, su questo viene attivato un raggio di gelo. 8d8 di danno da freddo, Tiro Salvezza Riflessi DC 23 per evitare completamente il colpo.

\textit{Quello che scioglie}: l'occhio punta un bersaglio entro 9 metri, su questo viene attivato un raggio che ha effetti di acido. 4d8 di danno da acido, Tiro Salvezza Riflessi DC 23 per dimezzare il danno.

\textit{Quello che brucia}: l'occhio punta un bersaglio entro 18 metri, su questo viene attivato un raggio infuocato. 8d8 di danno da fuoco, Tiro Salvezza Riflessi DC 23 per evitare completamente il colpo.

\textit{Quello che paralizza}: l'occhio punta un bersaglio entro 9 metri, su questo viene attivato un raggio che paralizza la creatura. Tiro Salvezza su Volontà DC 23 per evitare completamente gli effetti.

\textit{Quello che rallenta}: l'occhio punta in un cono di 9 metri. Sulle creature interessate viene proiettato un raggio che rallenta. Tiro Salvezza su Volontà DC 23 per evitare completamente gli effetti. Durata 1 minuto.

\textit{Quello che confonde}: l'occhio punta in un cono di 18 metri. Sulle creature interessate viene proiettato un raggio che causa confusione. Tiro Salvezza su Volontà DC 23 per evitare completamente gli effetti. Durata 1 minuto, ogni round è possibile effettuare un nuovo Tiro Salvezza per riprendersi dagli effetti.

\textit{Quello che addormenta}: l'occhio punta un bersaglio entro 36 metri, su questo viene attivato un raggio che addormenta la creatura. Tiro Salvezza su Volontà DC 23 per evitare completamente gli effetti.

\textit{Quello che muove}; questo occhio può manifestare l'incantesimo Mano Magica oppure Telecinesi.

\textit{\textbf{Un solo sguardo.}} Il Panopticon attiva l'occhio centrare. L'occhio centrale può essere usato come Azione di Reazione per lanciare Controincantesimo su un incantesimo che ha visto lanciare.

\textbf{Ecologia}\\
Ambiente: Qualsiasi Sotterraneo\\
Organizzazione: Solitario, coppia\\
\textbf{Tesoro}: Triplo\\
\textbf{Descrizione}\\
I Panopticon sono aberrazioni xenofobe, palle di dura carne volante dotate di un grosso occhio centrale, una grande bocca e 7 tentacoli lunghi circa 1 metro ognuno dotato di un occhio (di circa 10 cm di diametro) di colore diverso.

Poco si sa dell'origine dei panopticon, si pensa che siano un esperimento evoluzionario di Calicante, nel tentativo di creare una razza senziente e dominante.

Purtroppo l'arroganza, la superbia, il desiderio di essere al centro dell'attenzione hanno fatto naufragare questi tentativi di società ed i panopticon si sono dispersi spesso nel sottosuolo.

I Panopticon hanno una lunghissima vita, nell'ordine dei mille anni ma risultano anche creature che hanno più che raddoppiato questo limite. I Panopticon aumentano di taglia con l'età e così il numero di occhi. Le statistiche qui riportate sono riferite ad un esemplare di età adulta di circa 300 anni.


\medskip\index[Mostruario]{Pegaso}\textbf{Pegaso}

\textit{Grande celestiale, caotico buono}

\textbf{FORZA} +4

\textbf{DESTREZZA} +2

\textbf{COSTITUZIONE} +3

\textbf{INTELLIGENZA} +0

\textbf{SAGGEZZA} +2

\textbf{CARISMA} +1

\textbf{Iniziativa} +2 -- \textbf{Difesa} 13

\textbf{Punti Ferita} 59 (7d10 + 21)

\textbf{Movimento} 18 m, volo 27 m

\textbf{Tiri Salvezza} Tempra +7, Riflessi +6, Volontà +4

\textbf{Competenze} Consapevolezza +6

\textbf{Linguaggi} comprende Celestiale, Comune, Elfico e Silvano ma non può parlare

\textbf{Sfida} 2 (450 PX)

\textbf{Azioni}

\textit{\textbf{Zoccoli.} Attacco con arma da mischia}: +6 a colpire, portata 1 m, un bersaglio.

\textit{Colpisce:} 11 (2d6 + 4) danni da botta.

\textbf{Ecologia}
Ambiente: Pianure Temperate e Calde\\
Organizzazione: Solitario, coppia o branco (6-10)\\
\textbf{Tesoro}: Nessuno\\
\textbf{Descrizione}\\
Il pegaso è un magnifico cavallo alato che a volte serve la causa del bene. Seppur molto apprezzati come cavalcature volanti, i pegasi sono creature timide che difficilmente stringono amicizie. Un tipico pegaso è alto 1,8 metri al garrese, pesa 750 kg ed ha un'apertura alare di 6 metri. La maggior parte dei pegasi è bianca, ma a volte alcuni esemplari hanno colori diversi.

Il pegaso, nonostante le apparenze, è intelligente quanto un umano. Chi cerca di addestrarne uno a fare da cavalcatura, scoprirà che il pegaso è ricalcitrante e perfino violento. Un pegaso non può parlare, ma capisce il Comune e preferisce la compagnia di creature buone. Il metodo corretto per convincere un pegaso a fare da cavalcatura è farselo amico con Diplomazia, favori e buone azioni. Un pegaso ha di norma atteggiamento indifferente verso le creature buone, maldisposto verso quelle neutrali ed ostile verso quelle malvagie. Prima che possa servire come cavalcatura, un pegaso deve essere reso amichevole tramite una prova di Diplomazia o in altro modo. Cavalcare un pegaso richiede una sella esotica o Cavalcare a pelo, dato che una sella normale interferisce con le sue ali. Un pegaso può combattere portando un cavaliere, ma il cavaliere non può attaccare a sua volta se non supera una prova di Cavalcare. I pegasi addestrati non temono il combattimento ed il cavaliere non deve effettuare una prova di Cavalcare per controllarlo.

I pegasi depongono uova che sul mercato valgono 1000 mo l'una, mentre i piccoli arrivano alle 2000 mo a testa. Essendo creature intelligenti e buone, vendere uova e piccoli è essenzialmente schiavismo: nelle società buone chi lo fa è disprezzato o punito dalla legge.

I pegasi maturano come i cavalli. Gli addestratori professionisti chiedono 1000 per addestrare un pegaso, che servirà un cavaliere buono o neutrale fedelmente per tutta la vita.

Un carico leggero per un pegaso è fino a 150 kg; un carico medio è 150,5-300 kg; un carico pesante è 300,5-450 kg.

In alcuni pegasi il sangue di un antenato che era un eroico stallone è ancora forte. Questi campioni hanno la durata della vita di un umano, l'archetipo avanzato, manovrabilità perfetta, resistenza al fuoco 10, un bonus razziale di +4 ai Tiri Salvezza contro i Veleni e immunità alla Pietrificazione. Alcuni riescono a dire poche parole in Celestiale o Comune. Si rendono conto della loro superiorità agli altri cavalli ed ai pegasi, e non devono essere addestrati a volare con un cavaliere, ma permettono solo ai più grandi eroi di cavalcarli.


\medskip\index[Mostruario]{Persecutore Invisibile}\textbf{Persecutore Invisibile}

\textit{Media elementale, neutrale}

\textbf{FORZA} +3

\textbf{DESTREZZA} +4

\textbf{COSTITUZIONE} +2

\textbf{INTELLIGENZA} +0

\textbf{SAGGEZZA} +2

\textbf{CARISMA} +0

\textbf{Iniziativa} +4 -- \textbf{Difesa} 17

\textbf{Punti Ferita} 104 (16d8 + 32)

\textbf{Movimento} 15 m, volo 15 m (fluttua)

\textbf{Tiri Salvezza}: Tempra +13, Riflessi +11, Volontà +4

\textbf{Competenze} Muoversi Silenziosamente / Nascondersi +10, Consapevolezza +8

\textbf{Resistenze al Danno} da arma non magica

\textbf{Immunità ai Danni} veleno

\textbf{Immunità alle Condizioni} afferrato, avvelenato, intralciato, paralizzato, pietrificato, privo di sensi, prono, affaticamento

\textbf{Sensi} scurovisione 18 m

\textbf{Linguaggi} Auran, comprende il Comune ma non lo parla

\textbf{Sfida} 6 (2.300 PX)

\textit{\textbf{Cacciatore Infallibile.}} Il convocatore assegna una preda al persecutore. Il persecutore sa la direzione e la distanza a cui si trova la preda finché entrambi si trovano sullo stesso piano di esistenza. Il persecutore conosce anche la posizione del suo convocatore.

\textit{\textbf{Invisibilità.}} Il persecutore è invisibile.

\textit{\textbf{Natura Elementale.}} Un persecutore invisibile non ha bisogno di aria, cibo, bevande o sonno.

\textbf{Azioni}

\textit{\textbf{Multiattacco.}} La persecutore effettua due attacchi di schianto.



\textit{\textbf{Schianto.} Attacco con arma da mischia}: +12 a colpire, portata 1 m, un bersaglio.

\textit{Colpisce:} 10 (2d6 + 3) danni da botta.

\medskip\index[Mostruario]{Pseudodrago}\textbf{Pseudodrago}

\textit{Minuscola drago, neutrale buono}

\textbf{FORZA} -2

\textbf{DESTREZZA} +2

\textbf{COSTITUZIONE} +1

\textbf{INTELLIGENZA} +0

\textbf{SAGGEZZA} +1

\textbf{CARISMA} +0

\textbf{Iniziativa} +2 -- \textbf{Difesa} 14

\textbf{Punti Ferita} 7 (2d4 + 2)

\textbf{Movimento} 5 metri, volo 18 m

\textbf{Tiri Salvezza}: Tempra +4, Riflessi +5, Volontà +4

\textbf{Competenze} Muoversi Silenziosamente / Nascondersi +4, Consapevolezza +3

\textbf{Sensi} scurovisione 18 m, vista cieca 3 m

\textbf{Linguaggi} comprende il Comune e il Draconico ma non parla

\textbf{Sfida} 1/4 (50 PX)

\textit{\textbf{Resistenza alla Magia.}} Lo pseudodrago ha +1d6 ai Tiri Salvezza contro incantesimi e altri effetti magici.

\textit{\textbf{Sensi Affinati.}} Lo pseudodrago ha +1d6 alle prove di Saggezza (Consapevolezza) basate su vista, udito e olfatto.

\textit{\textbf{Telepatia Limitata.}} Lo pseudodrago può comunicare semplici idee, emozioni e immagini telepaticamente con qualsiasi creatura entro 30 metri da esso che può comprendere una lingua.

\textbf{Azioni}

\textit{\textbf{Morso.} Attacco con arma da mischia}: +4 a colpire, portata 1 m, un bersaglio.

\textit{Colpisce:} 4 (1d4 + 2) danni perforanti.

\textit{\textbf{Pungiglione.} Attacco con arma da mischia}: +4 a colpire, portata 1 m, una creatura.

\textit{Colpisce:} 4 (1d4 + 2) danni perforanti, e il bersaglio deve riuscire un Tiro Salvezza di Tempra DC 11 o restare avvelenato per 1 ora. Se il risultato del Tiro Salvezza è 6 o meno, il bersaglio cade privo di sensi per la stessa durata, o finché non subisce danni o un'altra creatura usa un'azione per risvegliarlo.

\textbf{Ecologia}\\
Ambiente: Foreste temperate\\
Organizzazione: Solitario, coppia o nido (3-5)\\
\textbf{Tesoro}: Standard\\
\textbf{Descrizione}\\
Gli pseudodraghi sono piccoli parenti dei veri draghi, giocosi e timidi. Parlano cinguettando, sibilando, ringhiando e facendo le fusa, ma possono comunicare telepaticamente con qualsiasi creatura intelligente. Se avvicinati pacificamente con offerte di cibo, sono disposti a condividere informazioni su quanto si trova nel loro territorio, ma minacce e violenza li fanno fuggire.

Gli pseudodraghi sono carnivori e mangiano insetti, roditori, uccellini e serpenti, anche se mangiano uova ed amano burro, formaggio e pesce. A volte cacciano a terra come le lucertole o volando come gli uccelli predatori. Intelligenti come la maggior parte degli umanoidi, non amano essere trattati come animali domestici, e preferiscono essere considerati amici. Diffidano delle creature malvagie, possono unirsi a incantatori e Devoti come Famigli e alcuni hanno stretto amicizia con Druidi e guardiaboschi o collaborano con i draghi buoni come sentinelle. Gli pseudodraghi diventano Famigli solo se apprezzano la personalità dell'incantatore (e se questi ha l'Abilità Famiglio e Carisma almeno 1), ma possono anche legarsi a persone delle quali apprezzano la compagnia. Uno pseudodrago potrebbe seguire in questo modo un personaggio per giorni, settimane, anni o perfino per tutta la vita, posto che siano ben nutriti e trattati con affetto.

Raggiunta l'età adulta, il corpo di uno pseudodrago è lungo 30 centimetri con una coda di 60 centimetri, e pesa circa 3,5 kg. Le uova di uno pseudodrago sono grandi come quelle di gallina, ma di consistenza simile al cuoio e macchiate di marrone, e le femmine le depongono in gruppi di 2-5 ogni primavera. Un nido di pseudodraghi (che costituiscono un gruppo familiare, e non sono nati dallo stesso gruppo di uova) di solito consiste di una coppia di adulti e diversi cuccioli quasi adulti.


\medskip\index[Mostruario]{Rakshasa}\textbf{Rakshasa}

\textit{Media immondo, legale malvagio}

\textbf{FORZA} +2

\textbf{DESTREZZA} +3

\textbf{COSTITUZIONE} +4

\textbf{INTELLIGENZA} +1

\textbf{SAGGEZZA} +3

\textbf{CARISMA} +5

\textbf{Iniziativa} +3 -- \textbf{Difesa} 23

\textbf{Punti Ferita} 110 (13d8 + 52)

\textbf{Movimento} 12 m

\textbf{Tiri Salvezza}: Tempra +9, Riflessi +12, Volontà +8

\textbf{Competenze} Ingannare +10, Percepire Emozioni +8

\textbf{Vulnerabilità al Danno} perforante di armi magiche impugnate da
creatura buone

\textbf{Immunità al Danno} da botta, armi +1

\textbf{Sensi} scurovisione 18 m

\textbf{Linguaggi} Comune, Infernale

\textbf{Sfida} 13 (10000 PX)

\textit{\textbf{Immunità alla Magia Limitata.}} Il rakshasa è immune agli affetti o all'individuazione tramite incantesimi di livello 6 o più basso a meno che non desideri esserne soggetto. Ha +1d6 ai Tiri Salvezza contro tutti gli altri incantesimi ed effetti magici.

\textit{\textbf{Incantesimi Innati.}} La caratteristica da incantatore del rakshasa il Carisma (+10 a colpire  con attacchi con incantesimi). Il rakshasa può lanciare in maniera innata i seguenti incantesimi, senza aver bisogno di componenti materiali:

A volontà: \textit{camuffare sé stesso, illusione minore, individuazione} \textit{dei pensieri, mano magica}

3/Giorno ciascuno: \textit{charme su persone, immagine maggiore,} \textit{individuazione del magico, invisibilità, suggestione} 1/Giorno: \textit{dominare persone, spostamento planare, visione del} \textit{vero, volare}

\textbf{Azioni}

\textit{\textbf{Multiattacco.}} Il rakshasa può effettuare due attacchi di artiglio.

\textit{\textbf{Artiglio.} Attacco con arma da mischia}: +13 a colpire, portata 1 m, un bersaglio.

\textit{Colpisce:} 9 (2d6 + 2) danni taglienti, e se il bersaglio è una creatura rimane maledetto. La maledizione magica ha effetto ogni qualvolta il bersaglio riposa, riempiendo i pensieri del bersaglio di immagini e sogni orribili. Il bersaglio maledetto non riceve beneficio dall'aver terminato un riposo. La maledizione perdura finché non viene rimossa dall'incantesimo \textit{rimuovi maledizione} o simile magia.

\textbf{Ecologia}
Ambiente: Qualsiasi\\
Organizzazione: Solitario, coppia o culto (3-12)\\
\textbf{Tesoro}: Doppio (Pugnale+1, altro tesoro)\\
\textbf{Descrizione}\\
Il rakshasa è uno spirito maligno che si traveste da creatura umanoide così da poter seguire la sua preda in incognito. Personificazione dei tabù della maggioranza delle società e capace di assumere l'aspetto di quelli che cerca di corrompere, un rakshasa compie moltissime azioni orribili. Se fossero umani, la loro blasfemia, il cannibalismo e gli atti ancora peggiori che compiono li marchierebbero come criminali meritevoli del più crudele degli inferni.

Quando non ha un altro aspetto, il rakshasa appare come un umanoide con la testa di un animale. Spesso ha il capo di un grosso felino (come tigri o pantere) o serpente (quali cobra o vipere) e, seppur sia più raro, può avere testa di gorilla, sciacallo, avvoltoio, elefante, mantide, lucertola, rinoceronte, cinghiale e molte altre ancora. In molti casi, il tipo di testa posseduta da un rakshasa dice qualcosa della sua personalità: un rakshasa dalla testa di tigre è furtivo e famelico, mentre uno con la testa di cinghiale può essere ghiotto e crudele. Queste differenze raramente incidono sulle statistiche base del rakshasa, anche se esistono varianti più potenti della standard con molteplici teste, poteri magici più potenti, e strane e letali capacità speciali aggiuntive.

I rakshasa disprezzano le religioni; riconoscono il potere degli dei, ma si vedono come i soli esseri degni di venerazione da parte delle razze mortali. I Devoti rakshasa sono quindi piuttosto rari. Sebbene i rakshasa siano esterni, sono anche creature del Piano Materiale, e alcuni credono che i primi rakshasa scelsero questo esilio al posto di qualche altro ruolo offertogli da un dio da tempo dimenticato. Anche se in genere sono solitari, non è raro trovare grandi famiglie di rakshasa che lavorano insieme per provocare la caduta di una civiltà mortale dall'interno, attraverso il succedersi di molte generazioni.

Un rakshasa è alto 1,8 metri e pesa 90 kg.

\medskip\index[Mostruario]{Razziamorti}\textbf{Razziamorti}

\textit{Larga costrutto, non morto, non allineato}

\textbf{FORZA} +5

\textbf{DESTREZZA} +0

\textbf{COSTITUZIONE} +4

\textbf{INTELLIGENZA} -4

\textbf{SAGGEZZA} -2

\textbf{CARISMA} -5

\textbf{Iniziativa} +2 -- \textbf{Difesa} 21

\textbf{Punti Ferita} 105 (10d10 + 50)

\textbf{Movimento} 9 m

\textbf{Tiri Salvezza} Tempra +15, Riflessi +9, Volontà +7

\textbf{Consapevolezza} +4

\textbf{Immunità al Danno} veleno

\textbf{Immunità alle Condizioni} avvelenato, affascinato, affaticato, paralizzato, pietrificato

\textbf{Sensi} scurovisione 30 m

\textbf{Linguaggi} comprende tutte le lingue del creature ma non può parlare

\textbf{Sfida} 6 (2300 PX)

\textit{\textbf{Natura Non Morta.}} Il Razziamorti non ha bisogno di aria, cibo, bevande o sonno.

\textit{\textbf{Forma Immutabile.}} Come costrutto non può essere influenzato da magie od effetti che ne cambino la forma.

\textit{\textbf{Contenitore.}} Il Razziamorti ha un comparto apribile con uno sportello sul dorso metallico che può contenere fino a 100kg di oggetti, grandi fino a taglia piccola.

\textit{\textbf{Resistenza all'Aria.}} Il Razziamorti ha una resistenza innata agli incantesimi della Lista di Magia Aria.

\textit{\textbf{Sensibile al Fuoco.}} Il Razziamorti se subisce danni da fuoco esegue una azione in meno il round dopo.

\textbf{Azioni}

\textit{\textbf{Multiattacco.}} Il Razziamorti attacca con due chele o attacca con una chela e usa l'Occhio Paralizzante.

\textbf{\textit{Chela.}} +9 al colpire, portata 1 metro

\textit{Colpire}: 16 (2d10 + 5) di danno da botta

\textit{\textbf{Occhio Paralizzante}}: la creatura interessata, entro 18 metri, deve fare un Tiro Salvezza su Tempra a DC 16 o rimanere paralizzato per 2d4 round.

\textbf{Ecologia}\\
Ambiente: Qualsiasi, caverne\\
Organizzazione: 1-2 Razziamorti, 1d4+1 guardiani\\
\textbf{Tesoro}: Quanto raccolto\\
\textbf{Descrizione}\\
I Razziamorti sono dei particolari non morti costruiti da pezzi di vario cadavere e pezzi di ferro perché assomiglino a delle specie di grossi granchi corazzati.
Il dorso, completamente metallico, funge da contenitore per i tesori che il Razziamorti trova, le chele, in numero variabile tra le 6 ed 8 sono lunghe poco più di un metro ed hanno la caratteristica di lasciare ognuna una impronta diversa essendo assemblate da pezzi di metallo e corpi di versi.

Il grosso occhio centrale, forse una volta appartenuto ad un umanoide permette al controllore e costruttore del Razziamorti di vedere e comandarlo. Lo scopo di un Razziamorti é esplorare, solitamente un sistema di caverne o percorsi, alla ricerca dei resti di passati razziatori e avventurieri per carpirne gli oggetti magici e tesori.

Solitamente un Razziamorto è sempre accompagnato da diversi guardiani (altre creature al comando del controllore) che lo aiutano nel "sistemare" eventuali "resistenze" ancora attive.


\medskip\index[Mostruario]{Remorhaz}\textbf{Remorhaz}

\textit{Enorme mostruosità, disallineato}

\textbf{FORZA} +7

\textbf{DESTREZZA} +1

\textbf{COSTITUZIONE} +5

\textbf{INTELLIGENZA} -3

\textbf{SAGGEZZA} +0

\textbf{CARISMA} -3

\textbf{Iniziativa} +1 -- \textbf{Difesa} 23

\textbf{Punti Ferita} 195 (17d12 + 85)

\textbf{Movimento} 9 m, scavo 6 m

\textbf{Tiri Salvezza}: Tempra +11, Riflessi +7, Volontà +4

\textbf{Immunità ai Danni} freddo, fuoco

\textbf{Sensi} scurovisione 18 m, senso tellurico 18 m

\textbf{Linguaggi} -

\textbf{Sfida} 11 (7.200 PX)

\textit{\textbf{Corpo Riscaldato.}} Una creatura che entri a contatto con il remorhaz o lo colpisca con un attacco da mischia mentre si trova entro 1 metro da esso, subisce 10 (3d6) danni da fuoco.

\textbf{Azioni}

\textit{\textbf{Morso.} Attacco in mischia con arma}: +18 a colpire, portata 3 m, un bersaglio.

\textit{Colpisce:} 40 (6d10 + 7) danni perforanti più 10 (3d6) danni da fuoco. Se il bersaglio è una creatura, è afferrato (DC 17 per fuggire). Fino al termine dell'afferrare, il bersaglio è intralciato, e il remorhaz non può attaccare con il morso un altro bersaglio.

\textit{\textbf{Inghiottire.}} Il remorhaz effettua una attacco di morso contro un bersaglio di taglia Media o inferiore che sta afferrando. Se l'attacco colpisce, la creatura subisce il danno da morso ed è inghiottita, e l'afferrare ha termine. Il bersaglio inghiottito è accecato e intralciato, ha copertura completa contro gli attacchi e altri effetti all'esterno del remorhaz, e subisce 21 (6d6) danni da acido all'inizio di ciascun turno del remorhaz.

Se il remorhaz subisce 30 o più danni in un singolo turno da una creatura al suo interno, il remorhaz deve riuscire un Tiro Salvezza su Tempra DC 15 al termine di quel turno o vomitare tutte le creature inghiottite, che cadono prone in uno spazio entro 3 metri dal remorhaz. Se il remorhaz muore, una creatura inghiottita non  più intralciata da esso e può uscire dal cadavere utilizzando 5 metri di movimento, uscendo prona.

\textbf{Ecologia}\\
Ambiente: Deserti Freddi e Ghiacciai\\
Organizzazione: Solitario\\
\textbf{Tesoro}: Nessuno\\
\textbf{Descrizione}\\
In un mondo di ghiaccio e neve, i remorhaz sono particolarmente temuti per il terribile fuoco che brucia dentro i loro corpi. Questo fuoco interiore fa sì che le piastre lungo il suo dorso divengano roventi quando la creatura è particolarmente arrabbiata, eccitata o nel panico. Le creature che si sono adattate alle regioni artiche spesso sono particolarmente vulnerabili al fuoco, il che rende la principale difesa del remorhaz incredibilmente potente e gli assicura il ruolo di pericoloso predatore delle zone ghiacciate. I remorhaz vivono in estesi labirinti scavati nel cuore dei ghiacciai. Queste bestie usano il loro calore per scavare tunnel nel ghiaccio, tunnel le cui lisce pareti vitree si ricongelano rapidamente lungo la loro scia creando numerosi dedali incredibilmente stabili.

Anche se il remorhaz ha molto in comune con i più piccoli parassiti di superficie, questa bestia è sorprendentemente intelligente. Sebbene incapace di palare, il tipico remorhaz capisce bene il Gigante, e spesso le tribù di giganti ne approfittano per stipulare alleanze con questi bestioni. I Giganti del Gelo sono particolarmente ossessionati da essi; questi giganti affrontano le crudeli e letali bruciature che un remorhaz può infliggere per diventare "amici del verme" ottenendo una potente arma da usare contro i loro nemici, un assassino capace di scavare attraverso il pavimento delle fortificazioni glaciali per colpire direttamente con la maggiore debolezza di un gigante del gelo: il fuoco. Altri giganti usano queste bestie come forge viventi, poiché il loro dorso è abbastanza caldo da sciogliere il metallo.

Un remorhaz è lungo 7 metri e pesa 5000 kg.



\medskip\index[Mostruario]{Rugginofago}\textbf{Rugginofago}

\textit{Media Mostruosità, disallineato}

\textbf{FORZA} +1

\textbf{DESTREZZA} +1

\textbf{COSTITUZIONE} +1

\textbf{INTELLIGENZA} -4

\textbf{SAGGEZZA} +1

\textbf{CARISMA} -2

\textbf{Iniziativa} +1 -- \textbf{Difesa} 15

\textbf{Punti Ferita} 27 (5d8 + 5)

\textbf{Movimento} 12 m

\textbf{Tiri Salvezza}: Tempra +2, Riflessi +4, Volontà +5

\textbf{Sensi} scurovisione 18 m

\textbf{Linguaggi} -

\textbf{Sfida} 1/2 (100 PX)

\textit{\textbf{Fiuto del Ferro.}} Il rugginofago può individuare, con l'olfatto, l'esatta posizione di metalli ferrosi entro 9 metri.

\textit{\textbf{Arrugginire Metallo.}} Qualsiasi arma non magica fatta di metallo che colpisca il rugginofago si corrode. Le munizioni non magiche fatte di metallo e che colpiscono il rugginofago, sono considerate distrutte dopo aver inflitto il danno.

\textbf{Azioni}

\textit{\textbf{Morso.} Attacco con arma da mischia}: +3 a colpire, portata 1 m, un bersaglio.

\textit{Colpisce:} 5 (1d8 + 1) danni perforanti.

\textit{\textbf{Antenne.}} Il rugginofago corrode gli oggetti di metallo ferroso non magici che può vedere e si trovano entro 1 metro. Se l'oggetto non è indossato o trasportato, il contatto col rugginofago ne distrugge un cubo di 30 centimetri di spigolo. Se l'oggetto è indossato o trasportato da una creatura, la creatura può effettuare un Tiro Salvezza su Riflessi DC 11 per evitare il contatto con il rugginofago.

Se l'oggetto con cui entra in contatto è un'armatura o scudo di metallo indossati o trasportati, questi subiscono una penalità permanente e cumulativa di -2 alla Difesa che forniscono. Le armature ridotte a Difesa 0 o gli scudi che scendono ad un bonus di +0 sono distrutti. Se l'oggetto con cui entra in contatto è un'arma di metallo impugnata da qualcuno, la arrugginisce come descritto nel tratto Arrugginire Metallo.

\textbf{Ecologia}
Ambiente: Qualsiasi Sotterraneo\\
Organizzazione: Solitario, coppia o nido (3-10)\\
\textbf{Tesoro}: Accidentale (nessun tesoro di metallo)\\
\textbf{Descrizione}\\
Di tutte le bestie terrificanti che un esploratore può incontrare nel sottosuolo, solo il rugginofago ha come obbiettivo quello cui l'avventuriero medio da più valore: il suo tesoro.

Lungo in genere 1 metro e dal peso di almeno 100 kg, il rugginofago somiglia ad un crostaceo e sarebbe già abbastanza spaventoso anche senza l'alieno processo nutritivo da cui prende il nome. I rugginofagi mangiano gli oggetti di metallo, preferendo quelli di ferro e di leghe ferrose come l'acciaio ma divorano anche mithral, Adamantio e metalli incantati con uguale facilità. Qualsiasi metallo toccato dalle delicate antenne del rugginofago o dalla sua pelle corazzata si corrode e si riduce in polvere in pochi secondi, facendone una delle bestie più temute dagli avventurieri sotterranei e dai Nani minatori che devono difendere le loro forge e competere con loro per l'oro.

Anche se i rugginofagi non hanno una tendenza innata per la violenza, la loro fame insaziabile li spinge a caricare qualunque cosa gli si avvicini con addosso abbastanza metallo, e qualsiasi resistenza viene accolta con inaspettata ferocia. Non è insolito che i rugginofagi in zone povere di metallo seguano le vittime in fuga per giorni usando la loro capacità di fiutare metalli, purché queste abbiano ancora oggetti di metallo intatti.\\
Fortunatamente, spesso è possibile sfuggire alle attenzioni di un rugginofago lanciandogli un oggetto di metallo denso, come uno scudo, e correndo nella direzione opposta. Quanti frequentano le aree infestate dai rugginofagi imparano velocemente a tenere a portata di mano armi di legno o pietra.


\medskip\index[Mostruario]{Sahuagin}\textbf{Sahuagin}

\textit{Media umanoide (sahuagin), legale malvagio}

\textbf{FORZA} +1

\textbf{DESTREZZA} +0

\textbf{COSTITUZIONE} +1

\textbf{INTELLIGENZA} +1

\textbf{SAGGEZZA} +1

\textbf{CARISMA} -1

\textbf{Iniziativa} +1 -- \textbf{Difesa} 13

\textbf{Punti Ferita} 22 (4d8 + 4)

\textbf{Movimento} 9 m, nuoto 12 m

\textbf{Tiri Salvezza}: Tempra +4, Riflessi +4, Volontà +4

\textbf{Competenze} Consapevolezza +5

\textbf{Sensi} scurovisione 36 m

\textbf{Linguaggi} Sahuagin

\textbf{Sfida} 1/2 (100 PX)

\textit{\textbf{Anfibio Limitato.}} Il sahuagin può respirare aria e acqua, ma deve restare sommerso almeno una volta ogni 4 ore per evitare di soffocare.

\textit{\textbf{Frenesia Sanguinaria.}} Il sahuagin ha +1d6 ai tiri per colpire in mischia contro qualsiasi creatura che non sia al massimo dei suoi Punti Ferita.

\textit{\textbf{Telepatia con gli Squali}}. Il sahuagin può comandare magicamente qualsiasi squalo entro 36 metri da sé, usando una forma limitata di telepatia.

\textbf{Azioni}

\textit{\textbf{Multiattacco.}} Il sahuagin può effettuare due attacchi da mischia: uno con il morso e uno con gli artigli o la lancia.

\textit{\textbf{Artigli.} Attacco con arma da mischia}: +3 a colpire, portata 1 m, un bersaglio.

\textit{Colpisce:} 3 (1d4 + 1) danni taglienti.

\textit{\textbf{Lancia.} Attacco con arma da mischia o a Distanza}: +3 a colpire, portata 1 m o gittata 6m, un bersaglio.

\textit{Colpisce:} 4 (1d6 + 1) danni perforanti, o 5 (1d8 + 1) danni perforanti se usata con due mani per effettuare un attacco da mischia.

\textit{\textbf{Morso.} Attacco con arma da mischia}: +3 a colpire, portata 1 m, un bersaglio.

\textit{Colpisce:} 3 (1d4 + 1) danni perforanti.

\textbf{Ecologia}\\
Ambiente: Oceani Temperati o Caldi\\
Organizzazione: Solitario, coppia, squadra (5-8), pattuglia (11-20 più 1 tenente di 3° livello e 1-2 Squali), banda (20-80 più 100\% non combattenti, 1 tenente di 3° livello e 1 capitano di 4° livello ogni 20 adulti, e 1-2 Squali) o tribù (70-160 più 100\% non combattenti, 1 tenente di 3° livello ogni 20 adulti, 1 capitano di 4° livello ogni 40 adulti, 9 guardie di 4° livello, 1-4 novizie di 3°-6° livello, 1 sacerdotessa di 7° livello, 1 barone di 6°-8° livello, e 5-8 Squali)
\textbf{Tesoro}: Equipaggiamento da PNG (Tridente, Balestra Pesante con 10 Quadrelli, altro tesoro)\\
\textbf{Descrizione}\\
Famelici e crudeli, i sahuagin sono, sfortunatamente, tra le razze oceaniche più prosperose. Grandi città sono state costruite da questa razza nelle buie profondità delle fosse oceaniche, e alcune fortezze sorgono nei pressi delle coste da dove lanciano assalti continui contro i nemici che respirano aria che vivono vicino alla riva. Orgogliosi e bellicosi, i sahuagin si alleano raramente con altri, e vedono le altre razze acquatiche, come aboleth, marinidi e simili come concorrenti. Le sole creature che sembrano rispettare oltre ai loro simili sono gli squali; in questi implacabili predatori, infatti, i sahuagin rivedono molto di loro stessi. Un sahuagin è alto 2,1 metri e pesa circa 125 kg.

I sahuagin sono soggetti a mutazioni genetiche, e quando nasce un mutante assurge quasi sempre ai ranghi nobiliari o di comando nella società. La mutazione sahuagin più comune consiste in un paio di braccia extra (che concedono due attacchi addizionali con gli artigli o la possibilità di maneggiare più armi). Alcuni parlano dei rari malenti, sahuagin che non sembrano uomini squalo ma elfi acquatici, malgrado condividano la sete di sangue e la natura crudele dei loro simili. I malenti spesso servono come spie o assassini i governanti sahuagin, ma si narra di intere tribù composte di malenti in remote zone del mare.


\medskip\index[Mostruario]{Salamandra}\textbf{Salamandra}

\textit{Grande elementale, neutrale malvagio}

\textbf{FORZA} +4

\textbf{DESTREZZA} +2

\textbf{COSTITUZIONE} +2

\textbf{INTELLIGENZA} +0

\textbf{SAGGEZZA} +0

\textbf{CARISMA} +1

\textbf{Iniziativa} +2 -- \textbf{Difesa} 18

\textbf{Punti Ferita} 90 (12d10 + 24)

\textbf{Movimento} 9 m

\textbf{Tiri Salvezza}: Tempra +10, Riflessi +7, Volontà +6

\textbf{Vulnerabilità al Danno} freddo

\textbf{Resistenze al Danno} da arma non magica

\textbf{Immunità ai Danni} fuoco

\textbf{Sensi} scurovisione 18 m

\textbf{Linguaggi} Ignan

\textbf{Sfida} 5 (1.800 PX)

\textit{\textbf{Armi Riscaldate.}} Qualsiasi arma da mischia metallica che la salamandra impugni infligge 3 (1d6) danni da fuoco aggiuntivi per colpo (già incluso nell'attacco).

\textit{\textbf{Corpo Riscaldato.}} Una creatura che entri a contatto con la salamandra o la colpisce con un attacco da mischia mentre si trova entro 1 metro da essa subisce 7 (2d6) danni da fuoco.

\textbf{Azioni}

\textit{\textbf{Multiattacco.}} La salamandra effettua due attacchi: uno con la lancia e uno con la coda.

\textit{\textbf{Coda.} Attacco con arma da mischia}: +10 a colpire, portata 3 m, un bersaglio.

\textit{Colpisce:} 11 (2d6 + 4) danni da botta più 7 (2d6) danni da fuoco, e il bersaglio è afferrato (DC 14 per fuggire). Fino al termine dell'afferrare, il bersaglio è intralciato, la salamandra può colpire automaticamente il bersaglio con la coda, e la salamandra non può effettuare attacchi di coda contro altri bersagli.

\textit{\textbf{Lancia.} Attacco con arma da mischia o a Distanza}: +9 a colpire, portata 1 m, gittata 6m, un bersaglio.

\textit{Colpisce:} 11 (2d6 + 4) danni perforanti, o 13 (2d8 +4) danni perforanti se usata con due mani per effettuare un attacco da mischia, più 3 (1d6) danni da fuoco.

\textbf{Ecologia}
Ambiente: Qualsiasi (Piano del Fuoco)\\
Organizzazione: Solitario, coppia o gruppo (3-5)\\
\textbf{Tesoro}: Standard (Lancia, altro tesoro ininfiammabile)\\
\textbf{Descrizione}\\
Le Salamandre sono native del Piano del Fuoco, dove le loro legioni di fieri combattenti sono molto temute dagli altri abitanti del Piano. Poiché molte delle più forti Razze Elementali del Fuoco Schiavizzano le Salamandre per la loro Abilità nella metallurgia e capacità combattiva, le Salamandre odiano gli Efreet e gli altri con fervore.

Anche se i loro nascondigli superano i 250 gradi C di temperatura, le Salamandre possono tollerare temperature più basse. Generalmente lo fanno se costrette, e sono anche più burbere e irascibili del normale in questi ambienti. Sebbene provenga dal Piano del Fuoco, la Razza delle Salamandre si identifica di più con l'Abisso, e ha un grande rispetto per i Demoni (in particolare quelli associati col fuoco, come i Balor e certi Signori dei Demoni legati alle fiamme). Per questo non è insolito incontrare un grosso gruppo di Salamandre nell'Abisso.

Le Salamandre sono spesso evocate nel Piano Materiale per servire come guardiani o, più comunemente, come fabbricanti di Armature, Armi e altri oggetti metallurgici, dato che la loro Abilità in questo campo è leggendaria. Le Salamandre infestano anche quelle aree del Piano Materiale dove il confine tra questo mondo e il Piano del Fuoco si è fatto labile, come vicino e dentro i Vulcani.

Abitando zone così estreme, le Salamandre posseggono solo tesori che resistono alle alte temperature, come Spade, Armature, gioielli, Verghe e altri oggetti che hanno un alto punto di fusione. La società delle Salamandre è crudele e basata sul potere e sulla capacità di soggiogare chi è inferiore a loro. Gli esseri inferiori alle Salamandre che causano problemi affrontano una morte lenta e dolorosa.



\medskip\index[Mostruario]{Satiro}\textbf{Satiro}

\textit{Media fatato, caotico neutrale}

\textbf{FORZA} +1

\textbf{DESTREZZA} +3

\textbf{COSTITUZIONE} +0

\textbf{INTELLIGENZA} +1

\textbf{SAGGEZZA} +0

\textbf{CARISMA} +2

\textbf{Iniziativa} +3 -- \textbf{Difesa} 15 (armatura di cuoio)

\textbf{Punti Ferita} 31 (7d8)

\textbf{Vulnerabilità al Danno} ferro freddo

\textbf{Movimento} 12 m

\textbf{Tiri Salvezza}: Tempra +4, Riflessi +8, Volontà +8

\textbf{Competenze} Muoversi Silenziosamente / Nascondersi +5, Intrattenere +6, Consapevolezza +2

\textbf{Linguaggi} Comune, Elfico, Silvano

\textbf{Sfida} 1/2 (100 PX)

\textit{\textbf{Resistenza alla Magia.}} Il satiro ha +1d6 ai Tiri Salvezza contro incantesimi e altri effetti magici.

\textbf{Azioni}

\textit{\textbf{Incornata.} Attacco con arma da mischia}: +3 a colpire, portata 1 m, un bersaglio.

\textit{Colpisce:} 6 (2d4 + 1) danni da botta.

\textit{\textbf{Spada Corta.} Attacco con arma da mischia}: +5 a colpire, portata 1 m, un bersaglio.

\textit{Colpisce:} 6 (1d6 + 3) danni perforanti.

\textit{\textbf{Arco Corto.} Attacco con arma a Distanza}: +5 a colpire, gittata 24m, un bersaglio.

\textit{Colpisce:} 6 (1d6 + 3) danni perforanti.

\textbf{Ecologia}\\
Ambiente: Foreste Temperate\\
Organizzazione: Solitario, coppia, banda (3-6) o festino (7-11)\\
\textbf{Tesoro}: Standard (Pugnale, Arco Corto più 20 Frecce, flauto di pan perfetto, altro tesoro)\\
\textbf{Descrizione}\\
I satiri, conosciuti in molte regioni come fauni, sono creature debosciate ed edoniste delle parti più profonde e primordiali delle foreste. Adorano il vino, la musica e i piaceri della carne, sono rinomati come libertini e bellimbusti che corteggiano le fanciulle sprovvedute e i pastorelli e si lasciano dietro una scia di spiegazioni imbarazzanti e gravidanze indesiderate.

Anche se i loro corpi sono quasi sempre quelli di uomini attraenti e ben proporzionati, le capacità seduttive dei satiri risiedono nel loro talento musicale. Con l'aiuto del suo flauto, un satiro è capace di tessere una vasta gamma di incantesimi melodici ideati per affascinare gli altri e farli accondiscendere ai suoi capricciosi desideri.

Oltre ad amoreggiare costantemente, i satiri spesso fungono da guardiani delle loro foreste, e quanti riescono a trasformare la lussuria del fauno in ira probabilmente si troveranno di fronte i più pericolosi tra gli animali che circondano il fauno. Inoltre, anche se i satiri tendono a mettere il loro divertimento al di sopra dei diritti altrui, non covano alcun risentimento contro quelli che seducono.

I bambini nati da questi incontri sono sempre satiri di sangue puro e vengono generalmente portati via dai loro sfrenati padri subito dopo la nascita.


\medskip\index[Mostruario]{Scheletro}\textbf{Scheletro}

\textit{Media non morto, legale malvagio}

\textbf{FORZA} +0

\textbf{DESTREZZA} +2

\textbf{COSTITUZIONE} +2

\textbf{INTELLIGENZA} -2

\textbf{SAGGEZZA} -1

\textbf{CARISMA} -3

\textbf{Iniziativa} +2 -- \textbf{Difesa} 14 (pezzi di armatura)

\textbf{Punti Ferita} 13 (2d8 + 4)

\textbf{Movimento} 9 m

\textbf{Tiri Salvezza}: Tempra +0, Riflessi +2, Volontà +2

\textbf{Vulnerabilità al Danno} da botta

\textbf{Resistenze al Danno} perforante e tagliente

\textbf{Immunità al Danno} veleno

\textbf{Immunità alle Condizioni} avvelenato, affaticamento

\textbf{Sensi} scurovisione 18 m

\textbf{Linguaggi} comprende tutte le lingue che parlava in vita ma non può parlare

\textbf{Sfida} 1/4 (50 PX)

\textit{\textbf{Natura Non Morta.}} Lo scheletro non necessita aria, cibo, bevande o sonno.

\textbf{Azioni}

\textit{\textbf{Spada Corta.} Attacco con arma da mischia}: +4 a colpire, portata 1 m, un bersaglio.

\textit{Colpisce:} 5 (1d6 + 2) danni perforanti.

\textit{\textbf{Arco Corto.} Attacco con arma a Distanza}: +4 a colpire, gittata 24m, un bersaglio.

\textit{Colpisce:} 5 (1d6 + 2) danni perforanti.

\textbf{Ecologia}\\
Ambiente: Qualsiasi\\
Organizzazione: Qualsiasi\\
\textbf{Tesoro}: Nessuno (Giaco di Maglia Rotto, Scimitarra Rotta)\\
\textbf{Descrizione}\\
Gli scheletri sono ossa di morti animate, portate alla non vita da magie sacrileghe. Per la maggior parte, gli scheletri sono automi privi di volontà, ma possiedono un'astuzia malvagia concessa loro dalla forza che li anima: un'astuzia che permette loro di portare armi ed indossare armature.

\medskip\index[Mostruario]{Scheletro di Cavallo da Guerra}\textbf{Scheletro di Cavallo da Guerra}

\textit{Grande non morto, legale malvagio}

\textbf{FORZA} +4

\textbf{DESTREZZA} +1

\textbf{COSTITUZIONE} +2

\textbf{INTELLIGENZA} -4

\textbf{SAGGEZZA} -1

\textbf{CARISMA} -3

\textbf{Iniziativa} +1 -- \textbf{Difesa} 14 (pezzi di bardatura)

\textbf{Punti Ferita} 22 (3d10 + 6)

\textbf{Movimento} 18 m

\textbf{Tiri Salvezza}: Tempra +4, Riflessi +3, Volontà +1

\textbf{Vulnerabilità al Danno} da botta

\textbf{Resistenze al Danno} perforante e tagliente

\textbf{Immunità al Danno} veleno

\textbf{Immunità alle Condizioni} avvelenato, affaticamento

\textbf{Sensi} scurovisione 18 m

\textbf{Linguaggi} -

\textbf{Sfida} 1/2 (100 PX)

\textit{\textbf{Natura Non Morta.}} Lo scheletro non necessita aria, cibo, bevande o sonno.

\textbf{Azioni}

\textit{\textbf{Zoccoli.} Attacco con arma da mischia}: +6 a colpire, portata 1 m, un bersaglio.

\textit{Colpisce:} 11 (2d6 + 4) danni da botta.

\medskip\index[Mostruario]{Scheletro di Minotauro}\textbf{Scheletro di Minotauro}

\textit{Grande non morto, legale malvagio}

\textbf{FORZA} +4

\textbf{DESTREZZA} +0

\textbf{COSTITUZIONE} +2

\textbf{INTELLIGENZA} -2

\textbf{SAGGEZZA} -1

\textbf{CARISMA} -3

\textbf{Iniziativa} +0 -- \textbf{Difesa} 13

\textbf{Punti Ferita} 67 (9d10 + 18)

\textbf{Movimento} 12 m

\textbf{Tiri Salvezza}: Tempra +6, Riflessi +3, Volontà +2

\textbf{Vulnerabilità al Danno} da botta

\textbf{Immunità al Danno} veleno

\textbf{Resistenze al Danno} perforante e tagliente

\textbf{Immunità alle Condizioni} avvelenato, affaticamento

\textbf{Sensi} scurovisione 18 m

\textbf{Linguaggi} comprende l'Abissale ma non può parlare

\textbf{Sfida} 2 (450 PX)

\textit{\textbf{Carica.}} Se lo scheletro di minotauro si muove di almeno 3 metri in linea retta verso il bersaglio e poi lo colpisce con un attacco di incornata durante lo stesso turno, il bersaglio subisce 9 (2d8) danni perforanti aggiuntivi. Se il bersaglio è una creatura, deve riuscire un Tiro Salvezza di Tempra DC 14 o venire spinto di 3 metri indietro e cadere prono.

\textit{\textbf{Natura Non Morta.}} Lo scheletro non necessita aria, cibo, bevande o sonno.

\textbf{Azioni}

\textit{\textbf{Ascia Bipenne.} Attacco con arma da mischia}: +6 a colpire, portata 1 m, un bersaglio.

\textit{Colpisce:} 17 (2d12 + 4) danni taglienti.

\textit{\textbf{Incornata.} Attacco con arma da mischia}: +6 a colpire, portata 1 m, un bersaglio.

\textit{Colpisce:} 13 (2d8 + 4) danni perforanti.

\medskip\index[Mostruario]{Segugio Infernale}\textbf{Segugio Infernale}

\textit{Media immondo, legale malvagio}

\textbf{FORZA} +3

\textbf{DESTREZZA} +1

\textbf{COSTITUZIONE} +2

\textbf{INTELLIGENZA} -2

\textbf{SAGGEZZA} +1

\textbf{CARISMA} -2

\textbf{Iniziativa} +1 -- \textbf{Difesa} 17

\textbf{Punti Ferita} 45 (7d8 + 14)

\textbf{Movimento} 15 m

\textbf{Tiri Salvezza}: Tempra +6, Riflessi +5, Volontà +1

\textbf{Competenze} Consapevolezza +5

\textbf{Immunità al Danno} fuoco

\textbf{Sensi} scurovisione 18 m

\textbf{Linguaggi} comprende l'Infernale ma non può parlare

\textbf{Sfida} 3 (700 PX)

\textit{\textbf{Udito e Olfatto Affinato.}} Il segugio ha +1d6 nelle prove di Saggezza (Consapevolezza) basate su udito od olfatto.

\textit{\textbf{Tattiche di Branco.}} Il segugio ha +1d6 ai tiri per colpire contro una creatura se almeno uno degli alleati del segugio si trova entro 1 metro dalla creatura e quell'alleato non è inabile.

\textbf{Azioni}

\textit{\textbf{Morso.} Attacco con arma da mischia}: +7 a colpire, portata 1 m, un bersaglio.

\textit{Colpisce:} 7 (1d6 + 3) danni perforanti più 7 (2d6) danni da fuoco.

\textit{\textbf{Soffio Infuocato (Ricarica 5-6).}} Il segugio esala fuoco in un cono di 5 metri. Ogni creatura in quell'area deve effettuare un Tiro Salvezza di Riflessi DC 12, e subire 21 (6d6) danni da fuoco se fallisce il Tiro Salvezza, o la metà di questi danni se lo riesce.



\subsection{Sfingi}

\medskip\index[Mostruario]{Androsfinge}\textbf{Androsfinge}

\textit{Grande mostruosità, legale neutrale}

\textbf{FORZA} +6

\textbf{DESTREZZA} +0

\textbf{COSTITUZIONE} +5

\textbf{INTELLIGENZA} +3

\textbf{SAGGEZZA} +4

\textbf{CARISMA} +6

\textbf{Iniziativa} +3 -- \textbf{Difesa} 26

\textbf{Punti Ferita} 199 (19d10 + 95)

\textbf{Movimento} 12 m, volo 18 m

\textbf{Tiri Salvezza}: Tempra +12, Riflessi +8, Volontà +7

\textbf{Competenze} Arcano +9, Consapevolezza +10, Religione +15

\textbf{Immunità al Danno} da arma non magica

\textbf{Immunità alle Condizioni} affascinato, spaventato

\textbf{Sensi} visione del vero 36 m

\textbf{Linguaggi} Comune, Sfinge

\textbf{Sfida} 17 (18000 PX)

\textit{\textbf{Armi Magiche.}} Gli attacchi con armi della sfinge sono magici.

\textit{\textbf{Imperscrutabile.}} La sfinge è immune a qualsiasi effetto in grado di percepirne le emozioni o leggerne i pensieri, oltre che a qualsiasi incantesimo di divinazione che rifiuti. Le prove di Saggezza (Percepire Emozioni) per discernere le intenzioni o la sincerità della sfinge hanno -1d6.

\textit{\textbf{Incantesimi.}} La sfinge ha CM 12.
La sua caratteristica da incantatore è la Saggezza (DC del Tiro Salvezza degli incantesimi 18, +10 a colpire con attacchi con incantesimo). Non ha bisogno di componenti materiali per lanciare i suoi incantesimi. La sfinge tiene preparati i seguenti incantesimi:

Trucchetti (a volontà): \textit{fiamma sacra, salvare i morenti,} \textit{taumaturgia}

livello 1 (4 slot): \textit{comando, individuazione del magico,} \textit{individuare male e bene}

livello 2 (3 slot): \textit{ristorare inferiore, zona di verità}

livello 3 (3 slot): \textit{dissolvi magie, linguaggi}

livello 4 (3 slot): \textit{esilio, libertà di movimento}

livello 5 (2 slot): \textit{colpo infuocato, ristorare superiore}

livello 6 (1 slot): \textit{banchetto degli eroi}

\textbf{Azioni}

\textit{\textbf{Multiattacco.}} La sfinge può effettuare due attacchi di artiglio.

\textit{\textbf{Artiglio.} Attacco con arma da mischia}: +17 a colpire, portata 1 m, un bersaglio.

\textit{Colpisce:} 17 (2d6 + 10) danni taglienti, 1 danno da Sanguinamento.

\textit{\textbf{Ruggito (3/Giorno).}} La sfinge emette un ruggito magico. Ogni volta che ruggisce prima di una nuova alba, il ruggito più forte e l'effetto è diverso, come dettagliato di seguito. Ogni creatura entro 150 metri dalla sfinge e capace di udirne il ruggito deve effettuare un Tiro Salvezza.

\textbf{Primo Ruggito.} Ogni creatura che fallisce un Tiro Salvezza su Volontà DC 18 resta spaventata per 1 minuto. Una creatura spaventata può ripetere il Tiro Salvezza al termine di ciascun suo round, terminandone l'effetto per sé, se lo riesce.

\textbf{Secondo Ruggito.} Ogni creatura che fallisce un Tiro Salvezza su Volontà DC 18 resta assordata e spaventata per 1 minuto. Una creatura spaventata è paralizzata e può ripetere il Tiro Salvezza al termine di ciascun suo round, terminandone l'effetto per sé, se lo riesce.

\textbf{Terzo Ruggito.} Ogni creatura effettua un Tiro Salvezza su Tempra DC 18. Chi fallisce il Tiro Salvezza subisce 44 (8d10) danni da suono ed è gettato prono. Se il Tiro Salvezza riesce, la creatura subisce la metà di questi danni e non viene gettata prona.

\textbf{Azioni Aggiuntive}

La sfinge può effettuare 3 Azioni aggiuntive, scelte tra le opzioni seguenti. Può usare solo un'opzione leggendaria alla volta e solo al termine del turno di un'altra creatura. La sfinge recupera le Azioni aggiuntive spese all'inizio del proprio round.

\textbf{Attacco di Artiglio.} La sfinge effettua un attacco di artiglio.

\textbf{Eseguire un Incantesimo (Costa 3 Azioni).} La sfinge lancia un incantesimo dalla lista degli incantesimi preparati, utilizzando uno slot incantesimo come di norma.

\textbf{Teletrasporto (Costa 2 Azioni).} La sfinge si teletrasporta magicamente, insieme a tutto l'equipaggiamento che sta indossando o trasportando, in uno spazio non occupato che possa vedere, fino a 36 metri di distanza.


\textbf{Ecologia}\\
Ambiente: Colline o Deserti Caldi\\
Organizzazione: Solitario\\
\textbf{Tesoro}: Standard\\
\textbf{Descrizione}\\
Le androsfingi, le più potenti tra le sfingi comuni, ritengono di rappresentare tutto ciò che c'è di degno e nobile nella loro specie e si atteggiano come se il peso del mondo intero poggiasse sul loro buon esempio. Guardano le Criosfingi con sufficienza paternalistica, le Ieracosfingi con malcelato disgusto e le Ginosfingi come le uniche altre sfingi degne del loro tempo.

Le androsfingi ostentano una facciata scorbutica e astiosa nei confronti degli stranieri. Non si sforzano in alcun modo di celare il loro fastidio quando sono irritate. Tendono inoltre a essere gelose del loro territorio, anche se meno delle altre sfingi. Quasi inevitabilmente lanciano avvertimenti e proclami roboanti prima di attaccare, e rispettano quasi sempre un appello a trattare. Le androsfingi barattano informazioni e conversazioni, e non tesori, in cambio di un passaggio sicuro.

Le androsfingi sono alte 3,6 metri e pesano 500 kg.


\medskip\index[Mostruario]{Ginosfinge}\textbf{Ginosfinge}

\textit{Grande mostruosità, legale neutrale}

\textbf{FORZA} +4

\textbf{DESTREZZA} +2

\textbf{COSTITUZIONE} +3

\textbf{INTELLIGENZA} +4

\textbf{SAGGEZZA} +4

\textbf{CARISMA} +4

\textbf{Iniziativa} +4 -- \textbf{Difesa} 23

\textbf{Punti Ferita} 136 (16d10 + 48)

\textbf{Movimento} 12 m, volo 18 m

\textbf{Tiri Salvezza}: Tempra +11, Riflessi +9, Volontà +10

\textbf{Competenze} Arcano +14, Consapevolezza +9, Religione +9, Storia +14

\textbf{Resistenze al Danno} da arma non magica

\textbf{Immunità alle Condizioni} affascinato, spaventato

\textbf{Sensi} visione del vero 36 m

\textbf{Linguaggi} Comune, Sfinge

\textbf{Sfida} 11 (7.200 PX)

\textit{\textbf{Armi Magiche.}} Gli attacchi con armi della sfinge sono magici.

\textit{\textbf{Imperscrutabile.}} La sfinge è immune a qualsiasi effetto in grado di percepirne le emozioni o leggerne i pensieri, oltre che a qualsiasi incantesimo di divinazione che rifiuti. Le prove di Saggezza (Percepire Inganni) per discernere le intenzioni o la sincerità della sfinge hanno -1d6.

\textit{\textbf{Incantesimi.}} La sfinge ha CM 9. La sua abilità da incantatore è l'Intelligenza (DC del Tiro Salvezza degli incantesimi 17, +9 a colpire con attacchi da incantesimo). Non ha bisogno di componenti materiali per eseguire i suoi incantesimi. La sfinge tiene preparati i seguenti incantesimi: Trucchetti (a volontà): \textit{illusione minore, mano magica,} \textit{prestidigitazione}

livello 1 (4 slot): \textit{identificare, individuazione del magico, scudo}

livello 2 (3 slot): \textit{localizza oggetto, oscurità, suggestione}

livello 3 (3 slot): \textit{dissolvi magie, linguaggi, rimuovi maledizione}

livello 4 (3 slot): \textit{esilio, invisibilità superiore}

livello 5 (2 slot): \textit{conoscenza delle leggende}

\textbf{Azioni}

\textit{\textbf{Multiattacco.}} La sfinge può effettuare due attacchi di artiglio.

\textit{\textbf{Artiglio.} Attacco con arma da mischia}: +11 a colpire, portata 1 m, un bersaglio.

\textit{Colpisce:} 13 (2d8 + 4) danni taglienti, 1 danno da Sanguinamento.

\textbf{Azioni Aggiuntive}

La sfinge può effettuare 3 Azioni aggiuntive, scelte tra le opzioni seguenti. Può usare solo un'opzione leggendaria alla volta e solo al termine del turno di un'altra creatura. La sfinge recupera le Azioni aggiuntive spese all'inizio del proprio round.

\textbf{Attacco di Artiglio.} La sfinge effettua un attacco di artiglio.

\textbf{Eseguire un Incantesimo (Costa 3 Azioni).} La sfinge esegue un incantesimo dalla lista degli incantesimi preparati, utilizzando uno slot incantesimo come di norma.

\textbf{Teletrasporto (Costa 2 Azioni).} La sfinge si teletrasporta magicamente, insieme a tutto l'equipaggiamento che sta indossando o trasportando, in uno spazio non occupato che possa vedere, fino a 36 metri di distanza.

\textbf{Ecologia}
Ambiente: Deserti e colline caldi\\
Organizzazione: Solitario, coppia o culto (3-6)\\
\textbf{Tesoro}: Doppio\\
\textbf{Descrizione}\\
Anche se esistono diversi tipi di sfinge, quella alla quale gli studiosi si riferiscono come Ginosfinge (un nome che molte sfingi trovano offensivo) è una creatura saggia e maestosa ma al contempo terrificante se arrabbiata. Meno moraliste delle loro controparti maschili (le Androsfingi, creature totalmente differenti da quella presentata qui), le sfingi sono prudenti e metodiche quando prendono delle decisioni, e sono orgogliose della loro fredda logica e della loro imparzialità. Hanno poca pazienza con le varianti inferiori di sfingi, considerandole poco più che animali. Le sfingi amano gli enigmi e gli indovinelli complicati, e fanno tesoro di fatti insoliti e dilemmi arcani molto più che di oro o gemme.

Pur non essendo grandi studiose in senso tradizionale, il grande apprezzamento delle sfingi per gli enigmi le porta a compiere ricerche in una grande varietà di materie, rendendole spesso una preziosa fonte di informazioni, specialmente quando fanno uso delle loro capacità magiche. Di solito sono felici di avere contatti con altre razze, ed offrono regolarmente beni materiali in cambio di informazioni o di indovinelli nuovi ed interessanti. Sono eccellenti guardiane di templi, tombe ed altri luoghi importanti, fintanto che vengono intrattenute in maniera adeguata. Le sfingi danno grande importanza alla gentilezza, ma possono essere capricciose: possono decidere altruisticamente di dividere i loro ultimi enigmi con dei viaggiatori ma non ci pensano due volte a divorarli se non vi prestano abbastanza attenzione o non forniscono alcun indizio utile alla loro risoluzione.

Una tipica sfinge è lunga 3 metri e pesa circa 400 kg. Anche se le loro ali possono tenerle in aria per lunghi periodi di tempo, sono delle volatrici scarse, e preferiscono atterrare prima di iniziare a combattere, attaccando con i loro poderosi artigli. Nonostante siano estremamente territoriali, le sfingi tendono ad avvisare gli intrusi varie volte prima di attaccare.

\medskip\index[Mostruario]{Sibilante}\textbf{Sibilante}

\textit{Larga mostruosità, caotico}

\textbf{FORZA} +2

\textbf{DESTREZZA} +1

\textbf{COSTITUZIONE} +1

\textbf{INTELLIGENZA} -3

\textbf{SAGGEZZA} +0

\textbf{CARISMA} -2

\textbf{Iniziativa} +1 -- \textbf{Difesa} 14

\textbf{Punti Ferita} 32 (5d10+5)

\textbf{Movimento} 6 m, arrampicarsi 6 m

\textbf{Tiri Salvezza}: Tempra +3, Riflessi +3, Volontà +2

\textbf{Competenze} Muoversi Silenziosamente +4, Consapevolezza +3

\textbf{Sensi} scurovisione 18 m

\textbf{Sensi Fini}: il Sibilante ha +1d6 alle prove di Consapevolezza basati su udito od olfatto

\textbf{Linguaggi}: -

\textbf{Sfida} 2 (450 PX)

\textbf{Azioni}

\textit{\textbf{Multiattacco.}} Il Sibilante può eseguire due attacchi con gli artigli oppure un colpo con la coda.

\textit{\textbf{Artiglio.} Attacco con arma da mischia}: +4 a colpire, portata 1 m, un bersaglio.

\textit{Colpisce:} 6 (1d8+2) danno tagliente.

\textit{\textbf{Frustata di Coda}}: il Sibilante agita la lunga coda e colpisce un bersaglio.

\textit{Colpisce:} 11 (2d8+2) danno da botta e 7 (2d6) da taglio, portata 4 metri. In caso di critico l'eventuale armatura o scudo viene danneggiata abbassando di 1 la Difesa dell'avversario. Il danno all'armatura non si considera permanente.

\textbf{Reazioni}

\textit{\textbf{Atterrare}}: quando il Sibilante è attaccato da una creatura nella a portata della sua coda questa viene sferzata obbligando l'attaccante, dopo la risoluzione del suo attacco, ad effettuare un Tiro Salvezza su Tempra/Riflessi a DC 12 o subire 7 (2d6) di danno da botta e cadere prono. Se il Tiro Salvezza riesce subisco solo metà danno e non è prona.

\textbf{Ecologia}\\
Ambiente: Caverne\\
Organizzazione: Solitario, coppia o nido (2-4)\\
\textbf{Tesoro}: casuale\\

\textbf{Descrizione}

I Sibilanti, chiamati così per via del rumore che fa la loro coda agitandosi è una creatura molto particolare. Assomiglia a prima vista ad un coccodrillo, lungo circa 5 metri di cui 4 di coda ma ha 8 zampe ed il muso corto e appiattito. La coda estremamente robusta finisce con una specie di uncino che il Sibilante usa per colpire, uccidere ed afferrare i nemici quasi fosse una zampa aggiuntiva.

Di colore grigio scuro, marrone, preferiscono nascondersi nell'oscurità ed attaccare quando affamati o per difendere il loro territorio. Cercano di tenere le distanze in combattimento e se gravemente feriti scappano arrampicandosi sulle pareti.


\medskip\index[Mostruario]{Spiritello}\textbf{Spiritello}

\textit{Minuscola fatato, neutrale buono}

\textbf{FORZA} -4

\textbf{DESTREZZA} +4

\textbf{COSTITUZIONE} +0

\textbf{INTELLIGENZA} +2

\textbf{SAGGEZZA} +1

\textbf{CARISMA} +0

\textbf{Iniziativa} +4 -- \textbf{Difesa} 16 (armatura di cuoio)

\textbf{Punti Ferita} 2 (1d4)

\textbf{Vulnerabilità al Danno} ferro freddo

\textbf{Movimento} 3 m, volo 12 m

\textbf{Tiri Salvezza}: Tempra +0, Riflessi +5, Volontà +2

\textbf{Competenze} Muoversi Silenziosamente / Nascondersi +8 (la prova è fatta con -1d6 se lo spiritello sta volando), Consapevolezza +3

\textbf{Linguaggi} Comune, Elfico, Silvano

\textbf{Sfida} 1/4 (50 PX)

\textbf{Azioni}

\textit{\textbf{Spada Lunga.} Attacco con arma da mischia}: +2 a colpire,

portata 1 m, un bersaglio.

\textit{Colpisce:} 1 danno tagliente.

\textit{\textbf{Arco Corto.} Attacco con arma a Distanza}: +6 a colpire, gittata 12 m, un bersaglio.

\textit{Colpisce:} 1 danno perforante. Se il bersaglio è una creatura, deve riuscire un Tiro Salvezza di Tempra DC 10 o restare avvelenata per 1 minuto. Se il risultato di questo Tiro Salvezza è 5 o meno, il bersaglio cade privo di sensi per la stessa durata, o finché subisce danni o un'altra creatura usa un'azione per risvegliarlo.

\textit{\textbf{Invisibilità.}} Lo spiritello resta invisibile finché non attacca o termina la sua concentrazione. Qualsiasi cosa che lo spiritello stia trasportando o indossando resta invisibile finché rimane in contatto con lo spiritello.

\textit{\textbf{Vista del Cuore.}} Lo spiritello entra in contatto con una creatura e ne apprende l'attuale stato emotivo. Se il bersaglio fallisce un Tiro Salvezza di Tempra DC 10, lo spiritello apprende anche i Tratti della creatura. Celestiali, immondi e non morti falliscono automaticamente questo Tiro Salvezza.

\textbf{Descrizione}\\
Gli spiritelli si riuniscono in gruppi nelle profondità di regioni boschive, uniti nella causa per proteggere la natura. Intere tribù di spiritelli si sono dichiarate protettrici di una determinata persona, di un luogo o di una creatura di particolare rilievo nelle loro terre, anche nel caso in cui l'essere non desideri o non necessiti di alcuna protezione.

Il corpo di uno spiritello è luminoso per natura, sebbene la creatura possa variare il colore e l'intensità della luce emessa dal suo corpo a piacimento. Subito dopo la sua morte, il corpo di uno spiritello si dissolve in una nebbia luccicante. Gli spiritelli sono i più piccoli tra i folletti, alti poco più di 22 centimetri e di un peso che raramente supera 1 kg.

Sotto molti aspetti gli spiritelli sono più primitivi della maggior parte dei folletti. Apprezzano la compagnia dei propri simili, ma tendono a diffidare degli altri folletti e presumono che qualsiasi umanoide o creatura che non hanno espressamente scelto di proteggere voglia far loro del male. Persino gli animali vengono da loro solitamente considerati pericolosi. La ragione di questa diffidenza è per buona parte dovuta alla taglia minuscola di queste creature, che le rende facili prede per i predatori. Pertanto la reazione iniziale di uno spiritello di fronte a un pericolo è darsi alla fuga: in genere utilizza le sue capacità magiche per rallentare o distrarre gli inseguitori, e in seguito si affida alla sua velocità di volare e alla sua taglia per riuscire a fuggire.

Sebbene gli spiritelli di per sé abbiano una natura incolta e selvaggia, hanno una sana curiosità per tutte le cose dotate di magia innata. Sono particolarmente attratti dai luoghi di grande potere magico latente, quali le rovine di antichi templi. Questa curiosità li rende anche insolitamente adatti al ruolo di famigli. Un incantatore caotico neutrale di 5° livello può ottenere uno spiritello come famiglio se ha l'Abilità Famiglio.


\medskip\index[Mostruario]{Strige (Uccello Stigeo)}\textbf{Strige (Uccello Stigeo)}

\textit{Minuscola bestia, disallineato}

\textbf{FORZA} -3

\textbf{DESTREZZA} +3

\textbf{COSTITUZIONE} +0

\textbf{INTELLIGENZA} -4

\textbf{SAGGEZZA} -1

\textbf{CARISMA} -2

\textbf{Iniziativa} +3 -- \textbf{Difesa} 15

\textbf{Punti Ferita} 2 (1d4)

\textbf{Movimento} 3 m, volo 12 m

\textbf{Tiri Salvezza}: Tempra +2, Riflessi +6, Volontà +1

\textbf{Sensi} scurovisione 18 m

\textbf{Linguaggi} -

\textbf{Sfida} 1/8 (25 PX)

\textbf{Azioni}

\textit{\textbf{Risucchio di Sangue.} Attacco con arma da mischia}: +5 a colpire, portata 1 m, una creatura.

\textit{Colpisce:} 5 (1d4 + 3) danni perforanti e lo strige si attacca al bersaglio. Mentre è attaccato, lo strige non attacca. Invece, all'inizio di ciascun turno dello strige, il bersaglio perde 5 (1d4 + 3) Punti Ferita a causa della perdita di sangue.

Lo strige può staccarsi spendendo 1 metro di movimento. Lo fa automaticamente dopo aver risucchiato 10 Punti Ferita dal bersaglio o alla morte del bersaglio. Una creatura, compreso il bersaglio, può usare la sua azione per staccare lo strige.

\textbf{Ecologia}
Ambiente: Paludi temperate e calde\\
Organizzazione: Solitario, colonia (2-4), stormo (5-8), nugolo (9-14) o sciame (15-40)\\
\textbf{Tesoro}: Nessuno\\
\textbf{Descrizione}\\
Gli strige sono pericolosi succhiasangue che infestano le paludi e predano animali selvatici, bestiame ed ignari viaggiatori. Pur essendo deboli individualmente, sciami di queste creature sono capaci di prosciugare un uomo in pochi minuti, lasciando dietro a loro solo un cadavere essiccato.

Più simili ai mammiferi che agli insetti, gli strige si alzano in volo con le loro quattro ali di carne, cercando prede a sangue caldo. Spesso si nascondono vicino a pozze di acqua bevibile aspettando che i viaggiatori abbassino la guardia per poi attaccarli e bere a sazietà, conficcando le loro proboscidi nelle vene scoperte. Dopo essersi nutriti, volano via a nascondersi tra la fanghiglia e tra i canneti per deporre le loro uova e riposare finché la fame non li spinge a cacciare di nuovo.

Di solito gli strige sono lunghi circa 30 centimetri, con un'apertura alare di circa il doppio, e pesano meno di 0,5 kg. Sono color rosso ruggine o marrone rossastro, ed hanno il ventre color giallo sporco, ma quelli che non si sono nutriti adeguatamente sono di colore rosa pallido.

\medskip\index[Mostruario]{Succube}\textbf{Succube}

\textit{Media immondo (mutaforma), neutrale malvagio}

\textbf{FORZA} -1

\textbf{DESTREZZA} +3

\textbf{COSTITUZIONE} +1

\textbf{INTELLIGENZA} +2

\textbf{SAGGEZZA} +1

\textbf{CARISMA} +5

\textbf{Iniziativa} +3 -- \textbf{Difesa} 17

\textbf{Punti Ferita} 66 (12d8 + 12)

\textbf{Movimento} 9 m, volo 18 m

\textbf{Tiri Salvezza}: Tempra +7, Riflessi +9, Volontà +10

\textbf{Competenze} Furtività 5, Percepire Emozioni +5, Consapevolezza +5, Ingannare +9

\textbf{Resistenze al Danno} freddo, fulmine, fuoco, veleno; da arma non magica

\textbf{Sensi} scurovisione 18 m

\textbf{Linguaggi} Abissale, Comune, Infernale, telepatia 18 m

\textbf{Sfida} 4 (1.100 PX)

\textit{\textbf{Legame Telepatico.}} L'immondo ignora le restrizioni di raggio di azione della sua telepatia quando comunica con una creatura che ha affascinato. I due non sono neppure costretti a trovarsi sullo stesso piano di esistenza.

\textit{\textbf{Mutaforma.}} L'immondo può usare la sua azione per trasformarsi in un umanoide di taglia Piccola o Media, o per tornare alla sua vera forma. Senza le ali, l'immondo perde la velocità di volo. A parte la taglia e la velocità, le sue statistiche sono le stesse in tutte le forme. Qualsiasi equipaggiamento stia indossando o trasportando non viene trasformato. Alla morte ritorna alla sua vera forma.

\textbf{Azioni}

\textit{\textbf{Artiglio (Solo Forma Immonda).} Attacco con arma da mischia}: +6 a colpire, portata 1 m, un bersaglio.

\textit{Colpisce:} 6 (1d6 + 3) danni taglienti.

\textit{\textbf{Affascinare.}} Un umanoide visibile all'immondo entro 9 metri da esso deve riuscire un Tiro Salvezza di Volontà DC 15 o restare magicamente affascinato per 1 giorno. Il bersaglio affascinato obbedisce ai comandi verbali o telepatici dell'immondo. Se il bersaglio subisce danni o riceve un comando suicida, può ripetere il Tiro Salvezza, terminando l'effetto se lo riesce. Se il bersaglio riesce il Tiro Salvezza contro l'effetto, o se l'effetto termina, il bersaglio è immune all'Affascinare dell'immondo per le successive 24 ore.

L'immondo può tenere affascinato solo un bersaglio alla volta. Se ne affascina un altro, l'effetto sul bersaglio precedente termina.

\textit{\textbf{Bacio Risucchiante.}} L'immondo bacia una creatura affascinata o una creatura consenziente. Il bersaglio deve effettuare un Tiro Salvezza di Tempra DC 14 contro questa magia, subendo 32 (5d10 + 5) danni se lo fallisce, o la metà di questi danni se lo riesce. I Punti Ferita massimi del bersaglio vengono ridotti di un ammontare pari ai danni subiti. Questa riduzione perdura finché non sorge l'alba. Il bersaglio muore se questo effetto riduce i suoi Punti Ferita massimi a 0.

\textit{\textbf{Forma Eterea.}} L'immondo entra magicamente nel Piano Etereo dal Piano Materiale, e viceversa.

\textbf{Ecologia}\\
Ambiente: Qualsiasi (Abisso)\\
Organizzazione: Solitario, coppia o harem (3-12)\\
\textbf{Tesoro}: doppio\\
\textbf{Descrizione}\\
Tra le orde demoniache una succube spesso può raggiungere altissimi livelli di potere, utilizzando le sue manipolazioni ed il suo fascino sensuale, e molte guerre demoniache imperversano a causa delle subdole macchinazioni di queste creature. Una succube si origina dalle anime di malvagi mortali particolarmente libidinosi ed avidi.


\medskip\index[Mostruario]{Tarrasque}\textbf{Tarrasque}

\textit{Colossale mostruosità (titano), disallineato}

\textbf{FORZA} +10

\textbf{DESTREZZA} +0

\textbf{COSTITUZIONE} +10

\textbf{INTELLIGENZA} -2

\textbf{SAGGEZZA} +0

\textbf{CARISMA} +0

\textbf{Iniziativa} +0 -- \textbf{Difesa} 35

\textbf{Punti Ferita} 676 (33x3d6 + 330)

\textbf{Movimento} 12 m

\textbf{Tiri Salvezza}: Tempra +31, Riflessi +22, Volontà +12

\textbf{Immunità al Danno} fuoco, veleno; armi +2

\textbf{Immunità alle Condizioni} affascinato, avvelenato, paralizzato, spaventato

\textbf{Sensi} vista cieca 36 m

\textbf{Linguaggi} -

\textbf{Sfida} 30 (155000 PX)

\textit{\textbf{Carapace Riflettente.}} Ogni volta che il Tarrasque è il bersaglio di un incantesimo \textit{dardo incantato}, un incantesimo a linea, o un incantesimo che richiede un tiro di attacco a gittata, tira un d6. Da 1 a 5, il Tarrasque lo ignora. Con 6, il Tarrasque lo ignora, e l'effetto viene riflesso contro l'incantatore come se fosse originato dal Tarrasque, trasformando l'incantatore nel bersaglio.

\textit{\textbf{Mostro d'Assedio.}} Il Tarrasque infligge danni doppi agli oggetti e le strutture.

\textit{\textbf{Resistenza Leggendaria (3/Giorno).}} Se il Tarrasque fallisce un Tiro Salvezza, può scegliere invece di riuscire.

\textit{\textbf{Resistenza alla Magia.}} Il Tarrasque ha +1d6 ai Tiri Salvezza contro incantesimi o altri effetti magici.

\textbf{Azioni}

\textit{\textbf{Multiattacco.}} Il Tarrasque può usare la sua Presenza Spaventosa. Poi effettua cinque attacchi: uno con il morso, due con gli artigli, uno con le corna, e uno con la coda. Al posto del morso può usare Inghiottire.

\textit{\textbf{Artiglio.} Attacco con arma da mischia}: +30 a colpire, portata 5 metri, un bersaglio.

\textit{Colpisce:} 28 (4d8 + 10) danni taglienti, 3 danno da Sanguinamento.

\textit{\textbf{Coda.} Attacco con arma da mischia}: +30 a colpire, portata 6 m, un bersaglio.

\textit{Colpisce:} 24 (4d6 + 10) danni da botta. Se il bersaglio è una creatura, deve riuscire un Tiro Salvezza di Tempra DC 20 o cadere prona.

\textit{\textbf{Corna.} Attacco con arma da mischia}: +30 a colpire, portata 3 m, un bersaglio.

\textit{Colpisce:} 32 (4d10 + 10) danni perforanti.

\textit{\textbf{Morso.} Attacco con arma da mischia}: +30 a colpire, portata 3 m, un bersaglio.

\textit{Colpisce:} 36 (4d12 + 10) danni perforanti. Se il bersaglio è una creatura, è afferrata (DC 20 per fuggire). Fino al termine dell'afferrare, il bersaglio è intralciato, e il Tarrasque non può usare il morso contro un altro bersaglio.

\textit{\textbf{Inghiottire.}} Il Tarrasque effettua una attacco di morso contro un bersaglio di taglia Grande o inferiore che sta afferrando. Se l'attacco colpisce, il bersaglio è inghiottito, e l'afferrare ha termine. Il bersaglio inghiottito è accecato e intralciato, ha copertura completa contro gli attacchi e altri effetti all'esterno del Tarrasque, e subisce 56 (16d6) danni da acido all'inizio di ciascun turno del Tarrasque.

Se il Tarrasque subisce 60 o più danni in un singolo turno da una creatura al suo interno, il Tarrasque deve riuscire un Tiro Salvezza su Tempra DC 30 al termine di quel turno o vomitare tutte le creature inghiottite, che cadono prone in uno spazio entro 3 metri dal Tarrasque. Se il Tarrasque muore, una creatura inghiottita non più intralciata da esso e può uscire dal cadavere utilizzando 9 metri di movimento, uscendo prona.

\textit{\textbf{Presenza Spaventosa.}} Ogni creatura scelta dal Tarrasque, che si trovi entro 36 metri da esso e consapevole della sua presenza, deve riuscire un Tiro Salvezza di Volontà DC 17 o restare spaventata per 1 minuto. Una creatura può ripetere il Tiro Salvezza al termine di ciascun suo round, con -1d6 se il Tarrasque è in linea di visuale, terminando l'effetto per sé, se lo riesce. Se il Tiro Salvezza della creatura ha successo o l'effetto ha termine per essa, la creatura è immune alla Presenza Spaventosa del Tarrasque per le successive 24 ore.

\textbf{Azioni Aggiuntive}

Il Tarrasque può effettuare 3 Azioni aggiuntive, scelte tra le opzioni seguenti. Può usare solo un'opzione leggendaria alla volta e solo al termine del turno di un'altra creatura. Il tarrasque recupera le azioni aggiuntive spese all'inizio del proprio round.

\textbf{Attacco.} Il Tarrasque effettua un attacco di artiglio o di coda. \textbf{Masticare (Costa 2 Azioni).} Il Tarrasque effettua un attacco di morso o usa Inghiottire.

\textbf{Muoversi.} Il Tarrasque si muove fino a metà del suo movimento.

\textbf{Ecologia}\\
Ambiente: Qualsiasi\\
Organizzazione: Solitario\\
\textbf{Tesoro}: Nessuno\\
\textbf{Descrizione}\\
Il leggendario Tarrasque è fra i mostri più distruttivi del mondo. Fortunatamente, passa la maggior parte del suo tempo in una specie di profondo letargo in una sconosciuta caverna in un remoto angolo del mondo. Quando si risveglia, però, muoiono interi regni.

Pur non particolarmente intelligente, il Tarrasque è abbastanza intelligente da capire alcune parole nel linguaggio delle Profondità (pur non potendo parlare). Allo stesso modo, la furia non è incontrollata: si concentra sulla creatura che l'ha danneggiato maggiormente ed è difficile distrarlo con l'inganno.

\medskip\index[Mostruario]{Teschio Fiammeggiante}\textbf{Teschio Fiammeggiante}

\textit{Piccolo non morto, Tratti malvagi}

\textbf{FORZA} +0

\textbf{DESTREZZA} +1

\textbf{COSTITUZIONE} +1

\textbf{INTELLIGENZA} +1

\textbf{SAGGEZZA} +0

\textbf{CARISMA} +0

\textbf{Iniziativa} +1 -- \textbf{Difesa} 13

\textbf{Punti Ferita} 7 (1d8 + 3)

\textbf{Movimento} volo 10 m

\textbf{Tiri Salvezza}: Tempra +1, Riflessi +2, Volontà +1

\textbf{Resistenze al Danno} da Vuoto

\textbf{Immunità al Danno} fuoco, veleno, da arma non magica

\textbf{Immunità alle Condizioni} affascinato, avvelenato, paralizzato, affaticamento, spaventato

\textbf{Sensi} scurovisione 18 m

\textbf{Sfida} 2 (200 PX)

\textit{\textbf{Incantesimi.}} Un Teschio Fiammeggiante può eseguire i seguenti incantesimi in maniera innata.

a Volontà: \textit{Produrre Fiamma}

1 volta al giorno: \textit{Gragnola di Ghiande Infuocate di Kyrin}

\textit{\textbf{Natura Non Morta.}} Il Teschio Fiammeggiante non ha bisogno di aria, cibo, bevande o sonno.

\textbf{Ecologia}\\
Ambiente: Qualsiasi\\
Organizzazione: Solitario, paio, pattuglia (2d4)\\
\textbf{Tesoro}: nessuno\\

\textbf{Descrizione}

I Teschi Fiammeggianti sono creati dai cadaveri degli incantatori specializzati nella Lista di magia del Fuoco, tramite una variante dell'incantesimo Rianimare Morti.

Usati come custodi e torce rappresentano spesso una prima linea di difesa nei dungeon.

\medskip\index[Mostruario]{Testuggine Dragona}\textbf{Testuggine Dragona}

\textit{Mastodontica drago, neutrale}

\textbf{FORZA} +7

\textbf{DESTREZZA} +0

\textbf{COSTITUZIONE} +5

\textbf{INTELLIGENZA} +0

\textbf{SAGGEZZA} +1

\textbf{CARISMA} +1

\textbf{Iniziativa} +0 -- \textbf{Difesa} 29

\textbf{Punti Ferita} 341 (22x3d6 + 110)

\textbf{Movimento} 6 m, nuoto 12 m

\textbf{Tiri Salvezza} Tempra +12, Riflessi +8, Volontà +9

\textbf{Sensi} scurovisione 18 m

\textbf{Linguaggi} Aquan, Draconico

\textbf{Sfida} 17 (18000 PX)

\textit{\textbf{Anfibio.}} La testuggine dragona può respirare aria e acqua.

\textbf{Azioni}

\textit{\textbf{Multiattacco.}} Il drago può effettuare tre attacchi: uno con il morso e due con gli artigli. Può effettuare un attacco di coda al posto di due attacchi di artiglio.

\textit{\textbf{Artiglio.} Attacco con arma da mischia}: +26 a colpire, portata 3 m, un bersaglio.

\textit{Colpisce:} 16 (2d8 + 7) danni taglienti.

\textit{\textbf{Coda.} Attacco con arma da mischia}: +26 a colpire, portata 5 metri, un bersaglio.

\textit{Colpisce:} 26 (3d12 + 7) danni da botta. Se il bersaglio è una creatura, deve riuscire un Tiro Salvezza di Tempra DC 20 o venire spinta di 3 metri lontano dalla testuggine dragona e cadere prona.

\textit{\textbf{Morso.} Attacco con arma da mischia}: +26 a colpire, portata 5 metri, un bersaglio.

\textit{Colpisce:} 26 (3d12 + 7) danni perforanti.

\textit{\textbf{Soffio di Vapore (Ricarica 5-6).}} La testuggine dragona esala un vapore caldo in un cono di 18 metri. Ogni creatura in quell'area deve effettuare un Tiro Salvezza di Tempra DC 18 e subire 52 (15d6) danni da fuoco se fallisce il Tiro Salvezza, o la metà di questi danni se lo riesce. Trovarsi sott'acqua non dà resistenza contro questo tipo di danno.

\textbf{Ecologia}
Ambiente: Acquatico temperato\\
Organizzazione: Solitario\\
\textbf{Tesoro}: Doppio\\
\textbf{Descrizione}\\
Le testuggini dragone abitano nelle acque dolci e salate, dove si attestano tra i più grandi pericoli per i marinai e coloro che viaggiano per nave attraverso le rotte marine del mondo. Gli esperti marinai sanno quello che vogliono le testuggini dragone della zona e frequentemente fanno offerte in oro e magia per garantirsi un passaggio sicuro o evitano completamente l'area. Da parte sua, una testuggine dragona apprezza ed anche si aspetta tali pedaggi e regalie, e una testuggine dragona che si aspetta regali ma viene ignorata è davvero un nemico pericoloso.

Il colore del guscio di una testuggine dragona varia da individuo a individuo. Alcuni hanno gusci opachi marrone e rosso ruggine, mentre altri hanno carapaci di un intenso color verde-blu con riflessi argentei sulle punte rocciose. La colorazione di testa, coda e zampe è lievemente più pallida del guscio e comprende striature dorate lungo la cresta e le spine.

Le testuggini dragone reclamano enormi territori in mare aperto, che comprendono regioni che spesso superano i 75 km quadrati. Qui, queste bestie pericolose capovolgono le navi che non rispettano i loro territori, aggiungendo relitti sommersi ed i loro preziosi carichi ai loro nascondigli. Le testuggini dragone generalmente fanno le loro tane in profonde caverne accessibili solo attraverso l'acqua, e spesso non solo le decorano con le ricchezze trafugate dalle navi che hanno affondato, ma anche coi relitti di queste sfortunate imbarcazioni. La loro natura territoriale e la loro predilezione per questo tipo di tane le mettono in conflitto diretto con le altre razze sottomarine come Marinidi e Sahuagin.

I grandi pesci, come tonni, storioni ed anche squali sono compresi tra i cibi preferiti dalle testuggini dragone, ma essendo onnivore, qualche volta si alimentano anche di grandi campi subacquei di alghe marine. Certamente non disdegnano di integrare la loro dieta coi passeggeri delle navi che affondano, anche se tale pratica non è dovuta né a malvagità né a crudeltà. Le testuggini dragone hanno gusci del diametro di 5 metri, con gli arti che si estendono pochi metri più in là, e misurano 7 metri dalla punta del naso all'estremità della loro possente coda.

\medskip\textbf{Topi, La}\\\index[Mostruario]{Topi, La}
\textit{Minuscola fatata}\\
\textbf{Forza}: -1\\
\textbf{Destrezza}: +4\\
\textbf{Costituzione}: +0\\
\textbf{Intelligenza}: +6\\
\textbf{Saggezza}: +2\\
\textbf{Carisma}: +6\\
\textbf{Difesa}: 17 -- \textbf{Iniziativa}: +15\\
\textbf{Punti Ferita}: 4 (1d10 - 1)\\
\textbf{Movimento}: 6 m\\
\textbf{Tiri Salvezza}: Tempra +20, Riflessi +30, Volontà +20 \\
\textbf{Sensi}: Senso tellurico 30, Scurovisione 30 m, Visione del Vero 30 m\\
\textbf{Lingue}: tutte\\
\textbf{Sfida} 0 (10 PX)\\
\textbf{Immunità}: al danno delle armi con bonus magico inferiore a +6\\
\textbf{Immunità}: a qualsiasi effetto non faccia piacere alla Topi\\
\textbf{Immunità}: a qualsiasi magia la Topi non voglia essere influenzata\\
\textbf{Immunità}: a subire a qualsiasi tipo di tiro critico\\
\textit{\textbf{E' La Topi}} La Topi ha +3d6 (oppure +18) ogni volta che deve tirare dei dadi o contare un valore.
Qualsiasi attacco effettuato dalla Topi è considerato magico +6 e non è resistibile.\\
\textbf{Azioni}\\
\textit{\textbf{Musetto}} ogni creatura a scelta di Topi, entro 30 metri, subisce un Musetto. La creatura viene allontanata di 2d6 metri e subisce 3d6 danni\\
\textit{\textbf{Morso topetto} Attacco con Arma da Mischia}: +26 al colpire, portata 1 m, un bersaglio.\\
\textit{Colpisce:} 6 danno perforante.\\
\textit{\textbf{Graffiotto} fino a 8 Attacchi con Arma da Mischia}: colpisce automaticamente, portata 1 m, fino a 4 bersagli.\\
\textit{Colpisce:} 1 danno perforante.\\
\textbf{Ecologia}\\
Ambiente: Ovunque\\
Organizzazione: Solitario\\
\textbf{Tesoro}: Speciale\\
\textbf{Descrizione}\\
Potrebbe essere scambiata per una piccola topina bianca, ma La Topi è molto di più. Furba, intelligente, bellissima adora andare al mercato e comprare borsette.


\medskip\textbf{Torciascura}\\\index[Mostruario]{Torciascura}
\textit{Media, non morto, malvagio}\\
\textbf{Forza}: +3\\
\textbf{Destrezza}: +1\\
\textbf{Costituzione}: +2\\
\textbf{Intelligenza}: +0\\
\textbf{Saggezza}: -1\\
\textbf{Carisma}: -2\\
\textbf{Difesa}: 17 -- \textbf{Iniziativa}: +2\\
\textbf{Punti Ferita}: 75 (12d10 +20)\\
\textbf{Movimento}: 6 m\\
\textbf{Tiri Salvezza}: Tempra +9, Riflessi +8, Volontà +7 \\
\textbf{Sensi}: Scurovisione, vede nell'oscurità magica\\
\textbf{Resistenze al Danno} da Vuoto; da arma non magica o che non sia argentata\\
\textbf{Immunità al Danno} veleno\\
\textbf{Immunità alle Condizioni} avvelenato, affaticamento\\
\textbf{Vulnerabilità} Luce\\
\textbf{Invisibile al buio} un Torciascura è completamente invisibile finché è nell'oscurità\\
\textbf{Lingue}: Comprende il Comume, ma non parla\\
\textit{\textbf{Natura Non Morta.}} Torciascura non ha bisogno di aria, cibo, bevande o sonno.\\
\textbf{Sfida} 4 (1100 PX)\\
\textit{\textbf{Sensibilità alla Luce}}. Mentre è alla luce del sole, Torciascura ha -1d6 ai tiri di attacco\\
\textbf{Multiattacco}\\
\textit{\textbf{Attacco}} Torciascura attacca due volte con la sua torcia oppure esegue Cono di Tristezza\\
\textit{\textbf{Torcia}} Attacco di mischia, +8 al colpire\\
\textit{\textbf{Colpisce}} 7 (1d6+3) di danno da botta, lancia l'incantesimo Oscurità sull'obiettivo colpito, durata fino alla distruzione del Torciascura\\
\textit{\textbf{Cono di Tristezza}} cono di 6 metri. Le creature colpite devono effettuare un Tiro Salvezza su Volontà DC 14 o cadere in una triste disperazione che conferisce -1d6 al Tiro per Colpire, -2 al danno da mischia.\\
\textbf{Ecologia}\\
Ambiente: Dungeon\\
Organizzazione: Solitario, gruppo 2d4\\
\textbf{Tesoro}: Speciale\\
\textbf{Descrizione}\\
Un Torciascura era un avventuriero, come voi, morto in preda al terrore dopo che l'ultima torcia si spense. Un Torciascura è un non morto, solitamente umanoide, dall'aspetto vagamente indefinito, che brandisce una torcia che emana pura oscurità. Il suo scopo è uccidere nuovi avventurieri avvolgendoli nell'oscurità eterna.

Solitamente il Torciascura si nasconde nell'oscurità aspettando di toccare l'avversario ed avvolgerlo nella sua maledizione. Una creatura uccisa da un Torciascura torna in vita come Torciascura dopo 1d3 giorni.

Un Torciascura quando viene distrutto lascia a terra la sua torcia. Questa torcia, di pura oscurità può lanciare l'incantesimo Oscurità a tocco tre volte al giorno, fuori dalle mani di un Torciascura se esposta alla luce del sole si distrugge in 2d4 round.


\medskip\index[Mostruario]{Troll}\textbf{Troll}

\textit{Grande gigante, caotico malvagio}

\textbf{FORZA} +4

\textbf{DESTREZZA} +1

\textbf{COSTITUZIONE} +5

\textbf{INTELLIGENZA} -2

\textbf{SAGGEZZA} -1

\textbf{CARISMA} -2

\textbf{Iniziativa} +1 -- \textbf{Difesa} 18

\textbf{Punti Ferita} 84 (8d10 + 40)

\textbf{Movimento} 9 m

\textbf{Tiri Salvezza}: Tempra +11, Riflessi +4, Volontà +3

\textbf{Competenze} Consapevolezza +2

\textbf{Sensi} scurovisione 18 m

\textbf{Linguaggi} Gigante

\textbf{Sfida} 5 (1.800 PX)

\textit{\textbf{Olfatto Affinato.}} Il troll ha +1d6 alle prove di Saggezza (Consapevolezza) basate sull'olfatto.

\textit{\textbf{Rigenerazione.}} Il troll recupera 10 Punti Ferita all'inizio del suo round. Se il troll subisce danno da acido o da fuoco, questo tratto non funziona all'inizio del prossimo round del troll. Il troll muore solo se inizia il suo round a -5 Punti Ferita e non può rigenerarsi.

\textbf{Azioni}

\textit{\textbf{Multiattacco.}} Il troll può effettuare tre attacchi: uno con il morso e due con gli artigli.

\textit{\textbf{Artiglio.} Attacco con arma da mischia}: +11 a colpire, portata 1 m, un bersaglio.

\textit{Colpisce:} 11 (2d6 + 4) danni taglienti, 1 danno da Sanguinamento.

\textit{\textbf{Morso.} Attacco con arma da mischia}: +11 a colpire, portata 1 m, un bersaglio.

\textit{Colpisce:} 7 (1d6 + 4) danni perforanti.

\textbf{Ecologia}\\
Ambiente: Montagne Fredde\\
Organizzazione: Solitario o banda (2-4)\\
\textbf{Tesoro}: Standard\\
\textbf{Descrizione}\\
I troll possiedono artigli affilati ed incredibili capacità rigenerative che permettono loro di guarire quasi tutte le ferite. Sono gobbi, brutti ma fortissimi: combinata con i loro artigli, la loro forza gli permette di lacerare la carne a mani nude. I troll sono alti circa 4 metri, ma la loro postura li fa apparire più bassi. Un troll adulto pesa circa 500 kg.

L'appetito di un troll e le sue capacità rigenerative lo rendono un combattente indomito, che carica a testa bassa la creatura vivente più vicina ed attacca con tutta la sua furia. Solo il fuoco fa esitare un troll, ma perfino quello che per lui è un pericolo mortale non ferma la sua avanzata. Chi affronta i troll sa di dover localizzare e bruciare qualsiasi sua parte dopo un combattimento, perché perfino dal brandello più piccolo del suo corpo, con il tempo può rinascere un troll completo. Fortunatamente, solo le parti più grandi di un troll, come gli arti, ricrescono in questo modo.

Nonostante la loro ferocia, i troll sono straordinariamente teneri e gentili verso i loro piccoli. I troll femmina lavorano in gruppo, passando molto tempo ad insegnare ai cuccioli come cacciare e difendersi prima di mandarli a cercare un proprio territorio. Un troll maschio vive un'esistenza solitaria, incontrando brevemente le femmine solo per accoppiarsi. Tutti i troll trascorrono il loro tempo a cercare cibo, dato che devono consumarne enormi quantità ogni giorno o muoiono di fame. Per questo, la maggior parte dei troll si crea un proprio territorio di caccia che viene spesso difeso combattendo con i rivali. Simili scontri sono di solito non letali, ma i troll conoscono bene le proprie debolezze, sfruttandole per uccidere l'avversario nei periodi di magra.

\medskip\index[Mostruario]{Uomo Acquatico}\textbf{Uomo Acquatico}

\textit{Media umanoide (uomo acquatico), neutrale}

\textbf{FORZA} +0

\textbf{DESTREZZA} +1

\textbf{COSTITUZIONE} +1

\textbf{INTELLIGENZA} +0

\textbf{SAGGEZZA} +0

\textbf{CARISMA} +1

\textbf{Iniziativa} +1 -- \textbf{Difesa} 12

\textbf{Punti Ferita} 11 (2d8 + 2)

\textbf{Movimento} 3 m, nuoto 12 m

\textbf{Tiri Salvezza}: Tempra +3, Riflessi +1, Volontà -1; +2 contro Ammaliamento

\textbf{Competenze} Consapevolezza +2

\textbf{Linguaggi} Aquan, Comune

\textbf{Sfida} 1/8 (25 PX)

\textit{\textbf{Anfibio.}} L'uomo acquatico può respirare aria e acqua.

\textbf{Azioni}

\textit{\textbf{Lancia.} Attacco con arma da mischia o a Distanza}: +2 a colpire, portata 1 m o gittata 6m, un bersaglio.

\textit{Colpisce:} 3 (1d6) danni perforanti, o 4 (1d8) danni perforanti se usata con due mani per effettuare un attacco da mischia.

\textbf{Ecologia}\\
Ambiente: Oceani temperati\\
Organizzazione: Solitario, pattuglia (2-6), banda (6-10 più un tenete di 3° livello, compagnia (11-60 più 3 tenenti di 3° livello, 2 comandanti di 5° livello, 1 commodoro di 7° livello e 3-12 Calamari\\
\textbf{Tesoro}: Equipaggiamento da PNG (Tridente, Balestra Leggera con 10 Quadrelli, altro tesoro)\\
\textbf{Descrizione}\\
Fisicamente, gli Uomini Pesce somigliano ai loro antenati, con fronti espressive, pelle pallida, capelli scuri e occhi porpora. Hanno tre sottili branchie sul collo, ma possono passare per Umani per brevi periodi, se lo desiderano.

\medskip\index[Mostruario]{Uomo Albero (Treant)}\textbf{Uomo Albero (Treant)}

\textit{Enorme pianta, caotico buono}

\textbf{FORZA} +6

\textbf{DESTREZZA} -1

\textbf{COSTITUZIONE} +5

\textbf{INTELLIGENZA} +1

\textbf{SAGGEZZA} +3

\textbf{CARISMA} +1

\textbf{Iniziativa} +1 -- \textbf{Difesa} 21

\textbf{Punti Ferita} 138 (12d12 + 60)

\textbf{Movimento} 9 m

\textbf{Tiri Salvezza}: Tempra +13, Riflessi +3, Volontà +9

\textbf{Resistenze al Danno} da botta, perforante

\textbf{Vulnerabilità al Danno} fuoco

\textbf{Linguaggi} Comune, Druidico, Elfico, Silvano

\textbf{Sfida} 9 (5000 PX)

\textit{\textbf{Falso Aspetto.}} Mentre l'uomo albero rimane immobile, è indistinguibile da un normale albero.

\textit{\textbf{Mostro d'Assedio.}} L'uomo albero infligge danni doppi agli oggetti e le strutture.

\textbf{Azioni}

\textit{\textbf{Multiattacco.}} L'uomo albero effettua due attacchi di schianto.

\textit{\textbf{Schianto.} Attacco con arma da mischia}: +16 a colpire, portata 1 m, un bersaglio.

\textit{Colpisce:} 16 (3d6 + 6) danni da botta.

\textit{\textbf{Sasso.} Attacco con arma a Distanza}: +16 a colpire, gittata 18m, un bersaglio.

\textit{Colpisce:} 28 (4d10 + 6) danni da botta.

\textit{\textbf{Animare Alberi (1/Giorno).}} L'uomo albero anima magicamente uno o due alberi visibili entro 18 metri da lui. Questi alberi hanno le stesse statistiche dell'ent, eccetto che hanno punteggio di Intelligenza e Carisma -3, non possono parlare, e hanno solo l'opzione di attacco Schianto. Un albero animato agisce come alleato dell'uomo albero. L'albero resta per 1 giorno o finché muore; finché l'uomo albero muore o si trova più di 36 metri lontano dall'albero, o finché l'uomo albero non effettua un'azione bonus per ritrasformarlo in un albero inanimato. Poi l'albero prenderà radici, se possibile.

\textbf{Ecologia}\\
Ambiente: Qualsiasi foresta\\
Organizzazione: Solitario o macchia (2-7)\\
\textbf{Tesoro}: Standard\\
\textbf{Descrizione}\\
I treant sono guardiani delle foreste ed ambasciatori degli alberi. Antichi quanto le foreste stesse, si vedono come genitori e pastori piuttosto che giardinieri: sono lenti e metodici, ma terrificanti quando costretti a combattere per difendere il loro gregge. Anche se raramente cercano la compagnia delle razze dalla vita breve ed hanno un'innata sfiducia verso i cambiamenti, mostrano tolleranza verso chi desidera imparare dai loro lunghi, lenti monologhi, specialmente coloro nei cui occhi leggono il desiderio di proteggere le regioni selvagge. Contro coloro che minacciano le loro foreste, specialmente i boscaioli che raccolgono legna o coloro che vorrebbero disboscare una foresta per costruire una strada o un forte, la rabbia dei treant si scatena rapida e devastante. Sono in grado di demolire ciò che gli altri costruiscono: un tratto che li aiuta durante i loro eccessi di furia.

I treant sono principalmente creature solitarie, ed un singolo individuo è spesso responsabile di un'intera foresta, ma a volte si raccolgono in gruppi detti boschetti per scambiarsi le ultime notizie e riprodursi.

In tempi di grave pericolo, tutti i boschetti di una regione si uniscono per una riunione della durata di mesi detta concilio, ma simili eventi sono molto rari, e fra i concili passano anche millenni.

Un tipico treant è alto 9 metri, con un tronco del diametro di 60 centimetri, e pesa circa 2.250 kg. I treant somigliano agli alberi più comuni dei territori dove vivono.

\medskip\index[Mostruario]{Uomo Magma (Magmin)}\textbf{Uomo Magma (Magmin)}

\textit{Piccola elementale, caotico neutrale}

\textbf{FORZA} -2

\textbf{DESTREZZA} +2

\textbf{COSTITUZIONE} +1

\textbf{INTELLIGENZA} -1

\textbf{SAGGEZZA} +0

\textbf{CARISMA} +0

\textbf{Iniziativa} +2 -- \textbf{Difesa} 15

\textbf{Punti Ferita} 9 (2d6 + 2)

\textbf{Movimento} 9 m

\textbf{Tiri Salvezza}: Tempra +6, Riflessi +4, Volontà +3

\textbf{Resistenze al Danno} da arma non magica

\textbf{Immunità ai Danni} fuoco

\textbf{Sensi} scurovisione 18 m

\textbf{Linguaggi} Ignan

\textbf{Sfida} 1/2 (100 PX)

\textit{\textbf{Illuminazione Incendiaria.}} Come azione bonus, l'uomo magma può accendere o spegnere le sue fiamme. Mentre la fiamma è accesa, l'uomo magma irradia luce intensa in un raggio di 3 metri e luce fioca per ulteriori 3 metri.

\textit{\textbf{Scoppio Mortale.}} Quando l'uomo magma muore, esplode in uno scoppio di fuoco e magma. Ogni creatura entro 3 metri da esso deve effettuare un Tiro Salvezza di Riflessi DC 11, subendo 7 (2d6) danni da fuoco se fallisce il Tiro Salvezza, o la metà di questi danni se lo riesce. Gli oggetti infiammabili che non siano indossati o trasportati e che si trovino nell'area, prendono fuoco.

\textbf{Azioni}

\textit{\textbf{Tocco.} Attacco con arma da mischia}: +4 a colpire, portata 1 m, un bersaglio.

\textit{Colpisce:} 7 (2d6) danni da fuoco. Se il bersaglio è una creatura o un oggetto infiammabile, questi prende fuoco. Fino a che una creatura effettua un'azione per estinguere la fiamma, la creatura subisce 3 (1d6) danni da fuoco al termine di ciascun suo round.

\textbf{Ecologia}\\
Ambiente: Qualsiasi terreno (Piano del Fuoco)\\
Organizzazione: Solitario o banda (2-8)\\
\textbf{Tesoro}: Standard\\
\textbf{Descrizione}\\
Benché i magmin popolino il Piano del Fuoco, a volte scivolano nelle crepe elementali nel Piano Materiale. Queste crepe di solito si formano in luoghi di forte calore, come vulcani o fiumi sotterranei di magma, o in luoghi di forte e imprevedibile magia. Quest'ultimo scenario termina di solito in eventi più complessi, dato che i magmin tendono ad appiccare involontariamente fuoco agli oggetti infiammabili vicini.

Anche se non sono coraggiosi, questi piccoli esterni sono tuttavia temibili nemici delle creature senza resistenza al loro intenso calore. Il loro tocco incenerisce gli abiti, e le creature che colpiscono i loro corpi con l'acciaio corrono il rischio di ridurre in scorie le loro armi. La miglior difesa dei magmin in patria sul Piano del Fuoco è il loro numero. Gli insediamenti, costellati di laghi di magma e spruzzanti geyser di roccia fusa, brulicano di incredibili quantità di queste creature.

I magmin sono paranoici e diffidenti. Sempre spaventati dagli abitanti più grandi del Piano del Fuoco, i magmin sommergono gli intrusi con migliaia di domande, chiedendo dove vanno, da dove vengono, e che cosa fanno vicino ai loro preziosi laghi di magma. Se le risposte dei viaggiatori non sono soddisfacenti, i magmin tentano di sbarazzarsi il più rapidamente possibile delle creature. Chi si rifiuta di andarsene rischia di essere gettato in un lago di roccia liquida.

I magmin sono molto orgogliosi di come curano i loro laghi di magma. Ogni lago ha un diverso scopo: per farsi il bagno, per cucinare i pasti o per rilassarsi. I magmin inseriscono minerali e sali in questi laghi per adeguarli al loro scopo. I laghi per cucinare (a volte chiamati dagli stranieri "laghi assassini") sono più caldi degli altri, e quelli per lo svago sono di solito più scuri di quelli da bagno.

Alla maturità, i magmin sono alti 1,2 metri, la loro densa composizione li fa pesare 150 kg.

\medskip\index[Mostruario]{Unicorno}\textbf{Unicorno}

\textit{Grande celestiale, legale buono}

\textbf{FORZA} +4

\textbf{DESTREZZA} +2

\textbf{COSTITUZIONE} +2

\textbf{INTELLIGENZA} +0

\textbf{SAGGEZZA} +3

\textbf{CARISMA} +3

\textbf{Iniziativa} +2 -- \textbf{Difesa} 15

\textbf{Punti Ferita} 67 (9d10 + 18)

\textbf{Movimento} 15 m

\textbf{Tiri Salvezza}: Tempra +7, Riflessi +7, Volontà +6; +2 resistenza contro il Vuoto, Energia Negativa

\textbf{Immunità al Danno} veleno

\textbf{Immunità alle Condizioni} affascinato, avvelenato, paralizzato

\textbf{Sensi} scurovisione 18 m

\textbf{Linguaggi} Celestiale, Elfico, Silvano, telepatia 18 m

\textbf{Sfida} 5 (1.800 PX)

\textit{\textbf{Armi Magiche.}} Gli attacchi con armi dell'unicorno sono magici.

\textit{\textbf{Carica.}} Se l'unicorno si muove di almeno 6 metri in linea retta verso il bersaglio e lo colpisce con un attacco di corno durante lo stesso turno, il bersaglio subisce 9 (2d8) danni perforanti aggiuntivi. Se il bersaglio è una creatura, deve riuscire un Tiro Salvezza su Tempra DC 15 o cadere prono.

\textit{\textbf{Incantesimi Innati.}} La caratteristica da incantatore innato dell'unicorno è il Carisma (DC 14 per i Tiri Salvezza degli incantesimi). L'unicorno può lanciare in maniera innata i seguenti incantesimi, senza bisogno di componenti:

A volontà: \textit{arte del druido, individuazione del bene e male,} \textit{passare senza tracce}

1/giorno ciascuno: \textit{calmare emozioni, dissolvi il bene e il male,} \textit{intralciare}

\textit{\textbf{Resistenza alla Magia.}} L'unicorno ha +1d6 ai Tiri Salvezza contro incantesimi e altri effetti magici.

\textbf{Azioni}

\textit{\textbf{Multiattacco.}} L'unicorno effettua due attacchi: uno con gli zoccoli e uno con il corno.

\textit{\textbf{Corno.} Attacco con arma da mischia}: +10 a colpire, portata 1 m, un bersaglio.

\textit{Colpisce:} 8 (1d8 + 4) danni perforanti.

\textit{\textbf{Zoccoli.} Attacco con arma da mischia}: +10 a colpire, portata 1 m, un bersaglio.

\textit{Colpisce:} 11 (2d6 + 4) danni da botta.

\textit{\textbf{Teletrasporto (1/Giorno).}} L'unicorno può teletrasportare magicamente sé stesso e fino a tre altre creature consenzienti visibili entro 1 metro da esso, insieme a tutto l'equipaggiamento che stanno indossando o trasportando, in un luogo familiare all'unicorno, che si trova ad un massimo di 1,5 chilometri di distanza.

\textit{\textbf{Tocco Guaritore (3/Giorno).}} L'unicorno entra a contatto tramite il corno con un'altra creatura. Il bersaglio recupera magicamente 11 (2d8 + 2) Punti Ferita. Inoltre, il contatto rimuove tutte le malattie e neutralizza tutti i veleni che affliggono il bersaglio.

\textbf{Azioni Aggiuntive}

L'unicorno può effettuare 3 Azioni aggiuntive, scelte tra le opzioni seguenti. Può usare solo un'opzione leggendaria alla volta e solo al termine del turno di un'altra creatura. L'unicorno recupera le azioni aggiuntive spese all'inizio del proprio round.

\textbf{Autoguarigione (Costa 3 Azioni).} L'unicorno recupera magicamente 11 (2d8 + 2) Punti Ferita.

\textbf{Scudo Scintillante (Costa 2 Azioni).} L'unicorno crea un campo magico scintillante che circonda lui o un'altra creatura visibile a lui entro 18 metri. Il bersaglio ottiene un bonus di +2 alla Difesa fino al termine del prossimo round dell'unicorno.

\textbf{Zoccoli.} L'unicorno effettua un attacco con gli zoccoli.

\textbf{Ecologia}\\
Ambiente: Foreste Temperate\\
Organizzazione: Solitario, coppia o benedizione (3-6)\\
\textbf{Tesoro}: Nessuno\\
\textbf{Descrizione}\\
Gli unicorni sono fiere, intelligenti creature silvane che preferiscono rimanere isolate, apparendo solo per difendere le loro dimore dal male. Evitano tutte le creature tranne i folletti buoni, le donne umanoidi buone e gli animali nativi della loro foresta, ma potrebbero unirsi ad altre creature buone contro nemici comuni. Un tipico unicorno è lungo 2,4 metri, alto 1 metro al garrese e pesa 600 kg.

Le coppie di unicorni rimangono insieme per tutta la vita e dimorano in radure particolari o all'interno delle foreste che difendono. Permettono alle creature buone e neutrali di attraversarle, cacciare o di abitarvi, ma le creature malvagie o quelle che vorrebbero turbarne l'ecosistema, ad esempio cacciando per divertimento o abbattendone gli alberi per venderne il legname, vengono in fretta allontanati o uccisi. In alcune rare occasioni, gli unicorni il cui partner è stato ucciso prendono giovani donne di rara virtù come surrogati, permettendo loro di cavalcarli divenendo loro guardiani per tutta la vita. Se la donna si lega a qualcun altro, come un figlio o un amante, il legame con l'unicorno si scioglie amorevolmente, generando la leggenda che gli unicorni diventano amici solo delle vergini.

Il corno di un unicorno è la fonte dei suoi poteri, e per utilizzare le proprie capacità magiche su altre creature questi deve toccarle con esso. Le creature malvagie danno grande valore ai corni di unicorno come reagenti per pozioni di guarigione e per riti oscuri: un corno di unicorno in polvere vale 800 mo quando utilizzato per creare un oggetto magico di guarigione.

\subsection{Vampiri}

\medskip\index[Mostruario]{Vampiro}\textbf{Vampiro}

\textit{Media non morto (mutaforma), legale malvagio}

\textbf{FORZA} +4

\textbf{DESTREZZA} +4

\textbf{COSTITUZIONE} +4

\textbf{INTELLIGENZA} +3

\textbf{SAGGEZZA} +2

\textbf{CARISMA} +4

\textbf{Iniziativa} +4 -- \textbf{Difesa} 23

\textbf{Punti Ferita} 144 (17d8 + 68)

\textbf{Movimento} 9 m

\textbf{Tiri Salvezza} : Tempra +13, Riflessi +11, Volontà +12

\textbf{Competenze} Muoversi Silenziosamente / Nascondersi +9, Consapevolezza +17

\textbf{Immunità al Danno} da Vuoto; da arma non magica

\textbf{Immunità alle Condizioni} affascinato, assordato

\textbf{Sensi} scurovisione 36 m

\textbf{Linguaggi} le lingue che conosceva in vita

\textbf{Sfida} 13 (10000 PX)

\textit{\textbf{Mutaforma.}} Se il vampiro non è sotto la luce del sole o immerso in acqua corrente, può usare la sua azione per trasformarsi in un Minuscolo pipistrello, una nube di foschia Media, o per tornare alla sua vera forma.

Mentre è in forma di pipistrello, il vampiro non può parlare, la sua velocità di passeggio è 1 metro e ha velocità di volo 9 metri. Le sue statistiche, a parte la taglia e la velocità, sono immutate. Qualsiasi equipaggiamento stia indossando si trasforma con esso, ma quello che stava trasportando viene fatto cadere a terra. Alla morte ritorna alla sua vera forma.

Mentre è in forma di foschia, il vampiro non può effettuare azioni, parlare o manipolare oggetti. È privo di peso, ha velocità di volo 6 metri, può fluttuare, e può entrare nello spazio di una creatura ostile e fermarsi lì. Inoltre, se in uno spazio vi passa dell'aria, la foschia può fare altrettanto senza stringersi, ma non può attraversare l'acqua. Ha +1d6 ai Tiri Salvezza su Tempra e Riflessi ed è immune a tutti i danni non magici, eccetto i danni subiti dalla luce del
sole.

\textit{\textbf{Debolezze del Vampiro.}} Il vampiro ha i seguenti difetti:

\textit{Danneggiato dall'Acqua Corrente.} Il vampiro subisce 20 danni da acido se termina il suo round all'interno dell'acqua corrente.

\textit{Ipersensibilità alla Luce.} Il vampiro subisce 20 danni da Luce quando inizia il suo round alla luce del sole. Mentre è alla luce del sole, ha -1d6 ai tiri di attacco e le prove di competenza.

\textit{Paletto nel Cuore.} Se un'arma perforante fatta di legno viene conficcata nel cuore del vampiro mentre il vampiro è inabile nel suo luogo di riposo, il vampiro resta paralizzato finché il paletto non viene rimosso.

\textit{Proibizione.} Il vampiro non può entrare in un'abitazione senza invito da parte dei suoi occupanti.

\textit{\textbf{Fuga nella Foschia.}} Quando scende a 0 Punti Ferita al di fuori del suo luogo di riposo, il vampiro si trasforma in una nube di foschia (come per il tratto Mutaforma) invece di cadere privo di sensi, purché non sia esposto alla luce del sole o all'acqua corrente. Se non può trasformarsi, viene distrutto.

Mentre si trova a 0 Punti Ferita in questa forma, non può tornare alla sua forma di vampiro, e deve raggiungere il suo luogo di riposo entro 2 ore o venire distrutto. Una volta raggiunto il suo luogo di riposo, ritorna alla sua forma di vampiro. Resterà quindi paralizzato finché non avrà recuperato almeno 1 punto ferita. Dopo aver trascorso almeno 1 ora nel suo luogo di riposo a 0 Punti Ferita, il vampiro recupererà 1 punto ferita.

\textit{\textbf{Natura Non Morta.}} Il vampiro non ha bisogno di aria.

\textit{\textbf{Resistenza Leggendaria (3/Giorno).}} Se il vampiro fallisce un Tiro Salvezza, può scegliere invece di riuscire.

\textit{\textbf{Rigenerazione.}} Il vampiro recupera 20 Punti Ferita all'inizio del suo round se possiede almeno 1 punto ferita e non è esposto alla luce del sole o l'acqua corrente. Se il vampiro subisce danno da Luce o danno dall'acqua sacra, questo tratto non funziona all'inizio del prossimo round del vampiro.

\textit{\textbf{Scalare come Ragno.}} Il vampiro può scalare superfici difficili, compreso lo stare a testa in giù sul soffitto, senza bisogno di effettuare una prova di abilità.

\textbf{Azioni}

\textit{\textbf{Multiattacco.}} Il vampiro può effettuare due attacchi, ma solo uno di essi può essere un attacco con morso.

\textit{\textbf{Colpo Disarmato (Solo in Forma di Vampiro).} Attacco con arma da mischia}: +18 a colpire, portata 1 m, una creatura.

\textit{Colpisce:} 8 (1d8 + 4) danni da botta. Invece di infliggere danno, il vampiro può afferrare il bersaglio (DC per fuggire 18).

\textit{\textbf{Morso (Solo in Forma di Pipistrello o Vampiro).} Attacco con arma da mischia}: +18 a colpire, portata 1 m, una creatura consenziente o una creatura afferrata dal vampiro, inabile o intralciata.

\textit{Colpisce:} 7 (1d6 + 4) danni perforanti più 10 (3d6) danni da Vuoto. I Punti Ferita massimi del bersaglio sono ridotti di un ammontare pari al danno da Vuoto subito, e il vampiro recupera un numero di Punti Ferita pari a quell'ammontare, TS Tempra DC 23 per resistere alla perdita di Punti Ferita Massimi. Il bersaglio diviene Affaticato. Questa riduzione permane fino alla nuova alba. Il bersaglio muore se questo effetto riduce i suoi Punti Ferita massimi a 0. Un umanoide ucciso in questo modo e poi sepolto nel terreno si rianima la notte seguente come progenie vampirica sotto il controllo del vampiro.

\textit{\textbf{Affascinare.}} Il vampiro prende a bersaglio un umanoide entro 9 metri che può vedere. Se il bersaglio può vedere il vampiro, deve effettuare un Tiro Salvezza di Volontà DC 17 contro questa magia o esserne affascinato. Il bersaglio affascinato considera il vampiro un amico fidato da ascoltare e proteggere. Sebbene il bersaglio non sia sotto il controllo del vampiro, prende le richieste e le azioni del vampiro nel modo più favorevole possibile, ed è un bersaglio consenziente dell'attacco con morso del vampiro.

Ogni volta che il vampiro o i compagni del vampiro fanno qualcosa di nocivo al bersaglio, questi può ripetere il Tiro Salvezza, terminando l'effetto su di sé in caso di successo. Altrimenti, l'effetto persiste 24 ore o finché il vampiro non viene distrutto, si trova su di un piano di esistenza diverso dal bersaglio, o effettua un'azione bonus per terminare l'effetto.

\textit{\textbf{Figli della Notte (1/Giorno).}} Il vampiro richiama magicamente 2d4 sciami di pipistrelli o ratti, purché il sole non sia sorto. Mentre è all'esterno, il vampiro può richiamare invece 3d6 lupi. Le creature richiamate arrivano in 1d4 round, agendo da alleati del vampiro e obbedendo ai suoi comandi. Le bestie restano per 1 ora, finché il vampiro non muore, o finché non le congeda con un'azione bonus.

\textbf{Azioni Aggiuntive}

Il vampiro può effettuare 3 Azioni aggiuntive, scelte tra le opzioni seguenti. Può usare solo un'opzione leggendaria alla volta e solo al termine del turno di un'altra creatura. Il vampiro recupera all'inizio del proprio round le Azioni aggiuntive che ha speso.

\textbf{Colpo Disarmato.} Il vampiro effettua un colpo disarmato.

\textbf{Morso (Costa 2 Azioni).} Il vampiro effettua un attacco con morso.

\textbf{Muoversi.} Il vampiro si muove del suo movimento senza provocare attacchi di opportunità.

\textbf{Ecologia}
Ambiente: Qualsiasi\\
Organizzazione: Solitario o famiglia (vampiro più 2-8 Progenie)\\
\textbf{Tesoro}: Equipaggiamento da PNG (Anello della Protezione +2, Fascia della Seduzione +4, Mantello della Resistenza +3)\\
\textbf{Descrizione}\\
I vampiri sono creature umanoidi non morte che si nutrono del sangue dei viventi. Hanno un aspetto molto simile a quando erano in vita, diventando spesso più attraenti, sebbene alcuni appaiano invece duri e ferini.


\medskip\index[Mostruario]{Progenie Vampirica}\textbf{Progenie Vampirica}

\textit{Media non morto, neutrale malvagio}

\textbf{Iniziativa} +0 -- \textbf{Difesa} 18

\textbf{Punti Ferita} 82 (11d8 + 33)

\textbf{Movimento} 9 m

\textbf{Tiri Salvezza} Tempra +3, Riflessi +2, Volontà +5

\textbf{FORZA} +3

\textbf{DESTREZZA} +3

\textbf{COSTITUZIONE} +3

\textbf{INTELLIGENZA} +0

\textbf{SAGGEZZA} +0

\textbf{CARISMA} +1

\textbf{Competenze} Muoversi Silenziosamente / Nascondersi +6, Consapevolezza +3

\textbf{Resistenze ai Danni} da Vuoto; da arma non magica

\textbf{Sensi} scurovisione 18 m

\textbf{Linguaggi} le lingue che conosceva in vita

\textbf{Sfida} 6 (1.800 PX)

\textit{\textbf{Debolezze della Progenie Vampirica.}} La Progenie Vampirica ha i seguenti difetti:

\textit{Danneggiato dall'Acqua Corrente.} La Progenie Vampirica subisce 20 danni da acido se termina il suo round all'interno dell'acqua corrente.

\textit{Ipersensibilità alla Luce.} La Progenie Vampirica subisce 20 danni da Luce quando inizia il suo round alla luce del sole. Mentre è alla luce del sole, ha -1d6 ai tiri di attacco e le prove di competenza.

\textit{Paletto nel Cuore.} La Progenie Vampirica è distrutto se un'arma perforante di legno gli viene conficcata nel cuore mentre è inabile all'interno del suo luogo di riposo.

\textit{Proibizione.} La Progenie Vampirica non può entrare in un'abitazione senza invito da parte dei suoi occupanti.

\textit{\textbf{Natura Non Morta.}} La Progenie Vampirica non ha bisogno di aria.

\textit{\textbf{Rigenerazione.}} La Progenie Vampirica recupera 10 Punti Ferita all'inizio del suo round se possiede almeno 1 punto ferita e non è esposto alla luce del sole o l'acqua corrente. Se la Progenie Vampirica subisce danno da Luce o danno dall'acqua sacra, questo tratto non funziona all'inizio del prossimo round del vampiro.

\textit{\textbf{Scalare come Ragno.}} La Progenie Vampirica può scalare superfici difficili, compreso lo stare a testa in giù sul soffitto, senza bisogno di effettuare una prova di abilità.

\textbf{Azioni}

\textit{\textbf{Multiattacco.}} La progenie vampirica può effettuare due attacchi, ma solo uno di essi può essere un attacco con morso.

\textit{\textbf{Artigli.} Attacco con arma da mischia}: +9 a colpire, portata 1 m, una creatura.

\textit{Colpisce:} 8 (2d4 + 3) danni taglienti. Invece di infliggere danno, il vampiro può afferrare il bersaglio (DC per fuggire 13).

\textit{\textbf{Morso.} Attacco con arma da mischia}: +9 a colpire, portata 1 m, una creatura afferrata dal vampiro, inabile o intralciata.

\textit{Colpisce:} 6 (1d6 + 3) danni perforanti più 7 (2d6) danni da Vuoto. I Punti Ferita massimi del bersaglio sono ridotti di un ammontare pari al danno da Vuoto subito, e il vampiro recupera un numero di Punti Ferita pari a quell'ammontare, TS Tempra DC 16 per resistere alla perdita di Punti Ferita massimi. Questa riduzione permane fino alla nuova alba. Il bersaglio muore se questo effetto riduce i suoi Punti Ferita massimi a 0. La creatura diventa Affaticata.

\textbf{Ecologia}\\
Ambiente: Qualsiasi\\
Organizzazione: Solitario, coppia, gruppo (3-6) o turba (7-12)\\
\textbf{Tesoro}: Standard\\
\textbf{Descrizione}\\
Un Vampiro può decidere di creare da una vittima una progenie vampirica anziché farne un vampiro completo solo quando usa la sua capacità creare progenie su una creatura umanoide. Questa decisione deve essere presa come azione gratuita appena un vampiro uccide una creatura appropriata usando il morso.

\medskip\index[Mostruario]{Vermi delle carne}\textbf{Vermi delle carne}

\textit{minuscola mostruosità, disallineato}

\textbf{FORZA} -4

\textbf{DESTREZZA} +0

\textbf{COSTITUZIONE} -2

\textbf{INTELLIGENZA} -4

\textbf{SAGGEZZA} 0

\textbf{CARISMA} -4

\textbf{Iniziativa} +0 -- \textbf{Difesa} 11

\textbf{Punti Ferita} 1 (1d6 -2)

\textbf{Movimento} 1 m

\textbf{Tiri Salvezza}: Tempra -1, Riflessi +0, Volontà -4

\textbf{Sensi} vista tellurica 3 m

\textbf{Linguaggi} -

\textbf{Sfida} 1 (200 PX)

\textbf{Azioni}

\textit{\textbf{Infestare la carne.}} Queste minuscole creature penetrano nella carne esposta senza effettuare Tiro per Colpire purché la carne sia esposta a contatto con loro.

\textit{\textbf{Colpisce.}} Entro 2d4 round i vermi (3d6 creature) della carne scavano nel tessuto dirigendosi verso il cuore. L'infestazione dei vermi causa 1 Punto Ferita di danno a round mentre scavano. Una volta arrivati al cuore ogni round il personaggio deve fare un Tiro Salvezza su Tempra DC 14, con un malus cumulativo di -1 per round. Una volta che il Tiro Salvezza fallisce il personaggio muore.

\textit{\textbf{Debellare i Vermi della carne.}} L'unico modo è usare una fiamma viva (una torcia causa 1d6 di danno ad applicazione od un incantesimo tipo mani brucianti) sulla parte dove i vermi stanno scavando. Ogni applicazione di fuoco può eliminare 3d6 vermi. Passati i 2d4 round i vermi sono troppo in profondità ed è inutile applicare il fuoco, solo un incantesimo di Cura Malattie può debellare completamente l'infestazione.

\textbf{Ecologia}\\
Ambiente: alberi marci, carne putrefatta\\
Organizzazione: gruppi 3d6\\
\textbf{Tesoro}: Nessuno\\
\textbf{Descrizione}\\

I vermi della carne sono tra i più temuti parassiti dagli avventurieri. Si trovano nei cumuli umidi di foglie o tronchi marci, nei cadaveri in putrefazione, nell acque torbide. Pallidi, viscidi, dotati di affilatissimi denti, lunghi poco più di 4 millimetri penetrano nella carne esposta in maniera facilissima e percepiscono il battito del cuore dove si dirigono. Mentre scavano nelle carni si possono percepire ed anche vedere strisciare sottopelle.


\medskip\index[Mostruario]{Verme Purpureo}\textbf{Verme Purpureo}

\textit{Mastodontica mostruosità, disallineato}

\textbf{FORZA} +9

\textbf{DESTREZZA} -2

\textbf{COSTITUZIONE} +6

\textbf{INTELLIGENZA} -5

\textbf{SAGGEZZA} -1

\textbf{CARISMA} -3

\textbf{Iniziativa} -2 -- \textbf{Difesa} 26

\textbf{Punti Ferita} 247 (15x3d6 + 90)

\textbf{Movimento} 15 m, scavo 9 m

\textbf{Tiri Salvezza}: Tempra +17, Riflessi +8, Volontà +4

\textbf{Sensi} vista cieca 9 m, senso tellurico 18 m

\textbf{Linguaggi} -

\textbf{Sfida} 15 (13000 PX)

\textit{\textbf{Scavatore di Tunnel.}} Il verme può scavare attraverso la roccia solida a metà della velocità di scavare e lascia un tunnel di 3 metri di diametro dietro di sè.

\textbf{Azioni}

\textit{\textbf{Multiattacco.}} Il verme effettua due attacchi: uno con il morso e uno con il pungiglione.

\textit{\textbf{Morso.} Attacco con arma da mischia}: +30 a colpire, portata 3 m, un bersaglio.

\textit{Colpisce:} 22 (3d8 + 9) danni perforanti. Se il bersaglio è una creatura di taglia Grande, deve riuscire un Tiro Salvezza di Riflessi DC 19 o venire inghiottita dal verme. Mentre è inghiottita, la creatura è accecata e intralciata, ha copertura completa contro gli attacchi e altri effetti provenienti dall'esterno del verme, e subisce 21 (6d6) danni da acido all'inizio di ciascun turno del verme.

Se il verme subisce 30 o più danni in un singolo turno da una creatura al suo interno, il verme deve riuscire un Tiro Salvezza di Tempra DC 21 al termine del suo round o vomitare tutte le creature inghiottite, che cadono prone in uno spazio entro 3 metri dal verme. Se il verme muore, una creatura inghiottita non risulta più intralciata da esso e può fuggire dal cadavere usando 6 metri di movimento, uscendo prona.

\textit{\textbf{Pungiglione.} Attacco con arma da mischia}: +9 a colpire, portata 3 m, una creatura.

\textit{Colpisce:} 19 (3d6 + 9) danni perforanti, e il bersaglio deve effettuare un Tiro Salvezza di Tempra DC 19, subendo 42 (12d6) danni da veleno se fallisce il Tiro Salvezza, o la metà di questi danni se lo riesce.

\textbf{Ecologia}\\
Ambiente: Qualsiasi sotterraneo\\
Organizzazione: Solitario\\
\textbf{Tesoro}: Accidentale\\
\textbf{Descrizione}\\
I vermi purpurei sono giganteschi necrofagi che abitano nelle regioni più profonde del mondo, mangiando qualsiasi materiale organico incontrino. Sono noti per inghiottire le loro prede intere. Non è insolito sentire di un gruppo di avventurieri scomparso all'interno delle fameliche fauci di un verme purpureo, gridando di terrore mentre i suoi membri sparivano uno alla volta.

Mentre vanno in cerca di creature viventi per divorarle, i vermi purpurei ingoiano anche un'enorme quantità di terra e minerali scavando nel sottosuolo. Le interiora di un verme purpureo possono contenere un considerevole numero di gemme e altri oggetti in grado di resistere all'acido corrosivo all'interno del suo esofago. In zone ricche di minerali preziosi, come quelle vicine alle miniere naniche, i tunnel naturali creati dagli scavi dei vermi purpurei sono spesso pieni di un notevole numero di pepite d'oro grezzo.

Un verme purpureo generalmente reclama una grande caverna sotterranea come sua tana, e anche se vi torna per riposare e digerire il cibo, passa la maggior parte del suo tempo in cerca di preda, scavando attraverso l'oscurità senza fine o scivolando lungo tunnel preesistenti alla costante ricerca di cibo per saziare la sua immensa fame. Sebbene quasi privi di intelletto, i vermi purpurei raramente sono stupidi. Sono diffusi come guardiani fra chi riesce a controllarli magicamente o hanno nel loro covo una stanza abbastanza grande da ospitarli.


\medskip\index[Mostruario]{Verme Strisciante Tentacolato}\textbf{Verme Strisciante Tentacolato}

\textit{Larga mostruosità, disallineato}

\textbf{FORZA} +4

\textbf{DESTREZZA} +1

\textbf{COSTITUZIONE} +3

\textbf{INTELLIGENZA} -4

\textbf{SAGGEZZA} 1

\textbf{CARISMA} -3

\textbf{Iniziativa} +2 -- \textbf{Difesa} 17

\textbf{Punti Ferita} 55 (7d10 + 31)

\textbf{Movimento} 9 m, scalare 9 m

\textbf{Tiri Salvezza}: Tempra +5, Riflessi +4, Volontà +7

\textbf{Sensi} scurovisione 18 m

\textbf{Linguaggi} -

\textbf{Sfida} 4 (1000 PX)

\textit{\textbf{Scalare come Ragno.}} Il Verme Strisciante Tentacolato può scalare superfici difficili, compreso lo stare a testa in giù sul soffitto, senza bisogno di effettuare una prova di abilità.

\textbf{Azioni}

\textit{\textbf{Multiattacco.}} Il Verme Strisciante Tentacolato effettua 3 attacchi, uno con il morso e due con i tentacoli.

\textit{\textbf{Morso.} Attacco con arma da mischia}: +8 a colpire, portata 1 m, un bersaglio.

\textit{Colpisce:} 10 (2d8 + 6) danni perforanti.

\textit{\textbf{Tentacolo.} Attacco con arma da mischia}: +7 a colpire, portata 3 m, una creatura.

\textit{Colpisce:} 1 danno da botta. Il bersaglio deve effettuare un Tiro Salvezza di Tempra DC 18 o rimanere paralizzato fino alla fine del round successivo.

\textbf{Ecologia}\\
Ambiente: Qualsiasi sotterraneo\\
Organizzazione: Solitario, paio, tribù (8-12 +3d6 piccoli)\\
\textbf{Tesoro}: Accidentale\\
\textbf{Descrizione}\\

Un tipico Verme Strisciante Tentacolato è un anellide lungo quasi 4 metri e pesa sui 400 kilogrammi. Di colore scuro (di varie gradazioni dal blu al verde al marrone) è un largo verme dotato di una possente bocca e lunghi e leggeri tentacoli lungo tutta la testa.

Il Verme Strisciante Tentacolato pur se dotato di corte "zampe" non cammina ma striscia secernendo un muco appiccicoso che gli permette di arrampicarsi anche su superfici in qualsisia orientamento.

Sono creature fameliche che non perdono occasione per cacciare e divorare o conservare i cadaveri dove seminare le loro uova. Amano la carne di Nibali e si nutrono di qualsiasi creatura vivente (spesso ratti dato il tipico ambiente delle fogne).

Le origini dei Vermi Striscianti Tentacolato sono piuttosto speculative, alcuni ipotizzano che un incantatore abbia provato, fallendo criticamente, a trasformarsi in un Verme Purpureo, altri credono fermamente che i giardini di Shayalia avessero bisogni di maggiore concimazione e così la Patrona trasformò dei normali lombrichi in queste terrificanti creature perché divorassero e digerissero i cadaveri seppelliti.

\medskip\index[Mostruario]{Viverna}\textbf{Viverna}

\textit{Grande drago, disallineato}

\textbf{FORZA} +4

\textbf{DESTREZZA} +0

\textbf{COSTITUZIONE} +3

\textbf{INTELLIGENZA} -3

\textbf{SAGGEZZA} +1

\textbf{CARISMA} -2

\textbf{Iniziativa} +0 -- \textbf{Difesa} 16

\textbf{Punti Ferita} 110 (13d10 + 39)

\textbf{Movimento} 6 m, volo 24 m

\textbf{Tiri Salvezza}: Tempra +9, Riflessi +6, Volontà +8

\textbf{Competenze} Consapevolezza +4

\textbf{Sensi} scurovisione 18 m

\textbf{Linguaggi} -

\textbf{Sfida} 6 (2.300 PX)

\textbf{Azioni}

\textit{\textbf{Multiattacco.}} La viverna può effettuare due attacchi: uno con il morso e uno con il pungiglione. Mentre vola, può usare i suoi artigli al posto di uno degli altri attacchi.

\textit{\textbf{Artigli.} Attacco con arma da mischia}: +13 a colpire, portata 1 m, un bersaglio.

\textit{Colpisce:} 13 (2d8 + 4) danni taglienti, 1 danno da Sanguinamento.

\textit{\textbf{Morso.} Attacco con arma da mischia}: +13 a colpire, portata 3 m, una creatura.

\textit{Colpisce:} 11 (2d6 + 4) danni perforanti.

\textit{\textbf{Pungiglione.} Attacco con arma da mischia}: +13 a colpire, portata 3 m, una creatura.

\textit{Colpisce:} 11 (2d6 + 4) danni perforanti. Il bersaglio deve effettuare un Tiro Salvezza di Tempra DC 15, e subire 24 (7d6) danni da veleno se lo fallisce, o la metà di questi danni se lo riesce.

\textbf{Ecologia}\\
Ambiente: Colline temperate o calde\\
Organizzazione: Solitario, coppia o stormo (3-6)\\
\textbf{Tesoro}: standard\\
\textbf{Descrizione}\\
Le viverne sono rettili brutali e violenti imparentati con i draghi. Sono sempre aggressive ed impazienti e preferiscono raggiungere i loro scopi utilizzando la forza. Per questa ragione, i draghi guardano alle viverne con superiorità, considerando questi loro lontani parenti come selvaggi primitivi privi di stile ed intelligenza.

Nella maggior parte dei casi, questa generalizzazione è azzeccata. Anche se non certo di intelletto animale e capace di parola, la maggior parte delle viverne non si cura della diplomazia, preferendo combattere prima e discutere poi, solo se si trovano davanti ad un avversario che non possono sconfiggere o da cui non possono fuggire.

Le viverne sono creature territoriali. Pur cacciando occasionalmente prede più grandi in gruppi più estesi, sono creature solitarie il cui territorio di caccia si estende dai 160 ai 320 km quadrati. È noto che le viverne combattono spesso fra loro fino alla morte per le contese su un territorio ricco di prede.

Seppur costantemente affamate ed inclini ad attaccare, una viverna può essere resa amichevole attraverso un'attenta combinazione di lusinghe, intimidazione, cibo e tesoro, per farne un potente alleato. Spesso servono Giganti e Umanoidi Mostruosi come guardiani come guardiani, ed alcune tribù di Boggard e Lucertoloidi le usano come cavalcature, anche se tali accordi spesso risultano parecchio costosi in termini di cibo ed oro, poiché sono poche le viverne che accettano di servire a lungo creature simili come cavalcature.

Una viverna è lunga circa 4,8 metri e la coda rappresenta da sola circa metà della lunghezza. Una viverna pesa in media 1000 kg.


\medskip\index[Mostruario]{Wight}\textbf{Wight}

\textit{Media non morto, neutrale malvagio}

\textbf{FORZA} +2

\textbf{DESTREZZA} +2

\textbf{COSTITUZIONE} +3

\textbf{INTELLIGENZA} +0

\textbf{SAGGEZZA} +1

\textbf{CARISMA} +2

\textbf{Iniziativa} +2 -- \textbf{Difesa} 16 (armatura borchiata)

\textbf{Punti Ferita} 45 (6d8 + 18)

\textbf{Movimento} 9 m

\textbf{Tiri Salvezza}: Tempra +3, Riflessi +2, Volontà +5

\textbf{Competenze} Muoversi Silenziosamente / Nascondersi +4, Consapevolezza +3

\textbf{Resistenze al Danno} da Vuoto; da arma non magica o che non sia argentata

\textbf{Immunità al Danno} veleno

\textbf{Immunità alle Condizioni} avvelenato, affaticamento

\textbf{Sensi} scurovisione 18 m

\textbf{Linguaggi} le lingue che conosceva in vita

\textbf{Sfida} 3 (700 PX)

\textit{\textbf{Natura Non Morta.}} Il wight non ha bisogno di aria, cibo, bevande o sonno.

\textit{\textbf{Sensibilità alla Luce}}. Mentre è alla luce del sole, il wight ha -1d6 ai tiri di attacco, oltre che alle prove di Saggezza (Consapevolezza) basate sulla vista.

\textbf{Azioni}

\textit{\textbf{Multiattacco.}} Il wight può effettuare due attacchi con la spada lunga o due attacchi con l'arco lungo. Può usare Risucchiare Vita al posto di uno dei suoi attacchi con la spada lunga.

\textit{\textbf{Risucchiare Vita.} Attacco con arma da mischia}: +5 a colpire, portata 1 m, una creatura.

\textit{Colpisce:} 5 (1d6 + 2) danni da Vuoto. Il bersaglio deve riuscire un Tiro Salvezza di Tempra DC 13 o vedere i suoi Punti Ferita massimi ridotti di un ammontare pari al danno subito. Il bersaglio diviene Affaticato. Questa riduzione perdura fino al sorgere della nuova alba. Il bersaglio muore se l'effetto riduce i suoi Punti Ferita massimi a 0.

Un umanoide ucciso da questo attacco si rianima 24 ore più tardi come zombi sotto il controllo del wight, a meno che l'umanoide non venga prima riportato in vita o il corpo sia distrutto. Il wight non può controllare più di dodici zombi alla volta.

\textit{\textbf{Spada Lunga.} Attacco con arma da mischia}: +5 a colpire, portata 1 m, un bersaglio.

\textit{Colpisce:} 6 (1d8 + 2) danni taglienti o 7 (1d10 + 2) danni taglienti se usata con due mani.

\textit{\textbf{Arco Lungo.} Attacco con arma a Distanza}: +5 a colpire, gittata 45m, un bersaglio.

\textit{Colpisce:} 6 (1d8 + 2) danni perforanti.

\textbf{Ecologia}\\
Ambiente: qualsiasi\\
Organizzazione: Solitario, coppia, gruppo (3-6) o branco (7-12)\\
\textbf{Tesoro}: Standard\\
\textbf{Descrizione}\\
I wight sono umanoidi risorti come non morti a causa della necromanzia, di una morte violenta o di una personalità estremamente malevola. In alcuni casi, un wight sorge quando uno spirito non morto si lega permanentemente ad un cadavere, spesso quello di un guerriero. Sono appena riconoscibili da chi li conosceva in vita: le loro carni sono corrotte dalla malvagità e dalla non morte, gli occhi ardono d'odio ed i denti divengono quelli di una bestia. In un certo senso, un wight è l'anello di congiunzione tra ghoul e spettri: un cadavere deforme che risucchia energia vitale col tocco.

Essendo non morti, i wight non hanno bisogno di respirare, così a volte si possono trovare sott'acqua, sebbene non siano nuotatori particolarmente abili a meno che non siano originati da creature nuotatrici quali elfi acquatici e marinidi. Sott'acqua i wight preferiscono le caverne dal soffitto basso dove le loro scarse capacità di nuoto non sono una limitazione.

\medskip\index[Mostruario]{Wraith}\textbf{Wraith}

\textit{Media non morto, neutrale malvagio}

\textbf{FORZA} -2

\textbf{DESTREZZA} +3

\textbf{COSTITUZIONE} +3

\textbf{INTELLIGENZA} +1

\textbf{SAGGEZZA} +2

\textbf{CARISMA} +2

\textbf{Iniziativa} +3 -- \textbf{Difesa} 16

\textbf{Punti Ferita} 67 (9d8 + 27)

\textbf{Movimento} 0 m, volo 18 m (fluttua)

\textbf{Tiri Salvezza}: Tempra +6, Riflessi +4, Volontà +6

\textbf{Resistenze al Danno} acido, freddo, fulmine, fuoco, suono; da arma non magica o che non sia argentata

\textbf{Immunità al Danno} da Vuoto, veleno

\textbf{Immunità alle Condizioni} affascinato, afferrato, avvelenato, intralciato, paralizzato, pietrificato, prono, affaticamento

\textbf{Sensi} scurovisione 18 m

\textbf{Linguaggi} le lingue che conosceva in vita

\textbf{Sfida} 5 (1.800 PX)

\textit{\textbf{Movimento Incorporeo.}} Il wraith può attraversare creature e oggetti come fossero terreno difficile. Subisce 5 (1d10) danni da forza se termina il proprio round all'interno di un oggetto.

\textit{\textbf{Natura Non Morta.}} Il wraith non ha bisogno di aria, cibo, bevande o sonno.

\textit{\textbf{Sensibilità alla Luce}}. Mentre è alla luce del sole, il wraith ha -1d6 ai tiri di attacco, oltre che alle prove di Saggezza (Consapevolezza) basate sulla vista.

\textbf{Azioni}

\textit{\textbf{Risucchiare Vita.} Attacco con arma da mischia}: +7 a colpire, portata 1 m, una creatura.

\textit{Colpisce:} 21 (4d8 + 3) danni da Vuoto. Il bersaglio deve riuscire un Tiro Salvezza di Tempra DC 15 o vedere i suoi Punti Ferita massimi ridotti di un ammontare pari al danno subito. Il bersaglio diviene Affaticato. Questa riduzione perdura fino al sorgere della nuova alba. Il bersaglio muore se l'effetto riduce i suoi Punti Ferita massimi a 0.

\textit{\textbf{Creare Spettro.}} Il wraith prende a bersaglio un umanoide entro 3 metri da esso e che sia morto da non più di 1 minuto e per cause violente. Lo spirito del bersaglio si anima come spettro nello spazio del suo cadavere e nello spazio più vicino non occupato. Lo spettro è sotto il controllo del wraith. Il wraith non può tenere più di sette spettri alla volta sotto il suo controllo.

\textbf{Ecologia}\\
Ambiente: Qualsiasi\\
Organizzazione: Solitario, coppia, gruppo (3-6) o branco (7-12)\\
\textbf{Tesoro}: Nessuno\\
\textbf{Descrizione}\\
I wraith sono creature nate dal male e dall'oscurità. Detestano la luce e le creature viventi, avendo perduto la maggior parte del legame con la loro vita precedente.


\medskip\index[Mostruario]{Xorn}\textbf{Xorn}

\textit{Media elementale, neutrale}

\textbf{FORZA} +3

\textbf{DESTREZZA} +0

\textbf{COSTITUZIONE} +6

\textbf{INTELLIGENZA} +0

\textbf{SAGGEZZA} +0

\textbf{CARISMA} +0

\textbf{Iniziativa} +0 -- \textbf{Difesa} 22

\textbf{Punti Ferita} 73 (7d8 + 42)

\textbf{Movimento} 6 m, scavo 6 m

\textbf{Tiri Salvezza}: Tempra +8, Riflessi +2, Volontà +5

\textbf{Competenze} Muoversi Silenziosamente / Nascondersi +3, Consapevolezza +6

\textbf{Resistenze al Danno} perforante e tagliente di armi non magiche o che non siano di adamantio

\textbf{Sensi} scurovisione 18 m, senso tellurico 18 m

\textbf{Linguaggi} Terran

\textbf{Sfida} 5 (1.800 PX)

\textit{\textbf{Mimetismo di Pietra.}} Lo xorn ha +1d6 alle prove di Destrezza (Nascondersi) effettuate per nascondersi su terreno roccioso.

\textit{\textbf{Scorrere sulla Terra.}} Lo xorn può scavare attraversa la terra e la pietra non magiche e non lavorate. Quando lo fa, lo xorn non disturba il materiale che sposta.

\textit{\textbf{Senso del Tesoro.}} Lo xorn può individuare precisamente, con l'olfatto, la posizione di metalli e pietre preziose, come monete e gemme, entro 18 metri da esso.

\textbf{Azioni}

\textit{\textbf{Multiattacco.}} Lo xorn effettua tre attacchi di artiglio e un attacco di morso.

\textit{\textbf{Artiglio.} Attacco con arma da mischia}: +9 a colpire, portata 1 m, un bersaglio.

\textit{Colpisce:} 6 (1d6 + 3) danni taglienti, 1 danno da Sanguinamento.

\textit{\textbf{Morso.} Attacco con arma da mischia}: +9 a colpire, portata 1 m, un bersaglio.

\textit{Colpisce:} 13 (3d6 + 3) danni perforanti.

\textbf{Ecologia}\\
Ambiente: Qualsiasi (Piano della Terra)\\
Organizzazione: Solitario, coppia o gruppo (3-6)\\
\textbf{Tesoro}: Standard (solo metalli preziosi, gemme e gioielli e gemme magiche)\\
\textbf{Descrizione}
Strane creature larghe quanto alte, gli xorn hanno poco interesse verso i nativi del Piano Materiale, non fosse per le gemme ed i metalli preziosi che potrebbero avere con sé. Nascosti sotto la superficie del terreno per un tempo che ad un umano potrebbe sembrare lunghissimo, uno xorn può attendere mesi, perfino anni, per la preda ideale, per poi assalire chi porta con sé il suo cibo preferito, come una gemma particolare o un determinato tipo di argento. Gli avventurieri che si addentrano nelle regioni abitate dagli xorn portano spesso con sé piccole pepite di minerali o gemme e cristalli di scarso valore da utilizzare come tributo. Anche se il suo valore è solitamente direttamente proporzionale al suo sapore e all'appetibilità che esso può avere, la maggior parte degli xorn è piuttosto ingorda, e preferisce la quantità alla qualità.

Il tesoro che uno xorn porta con sé o nasconde nella sua tana consiste in uno spuntino che ha conservato per il giorno successivo. Offrire un gioiello o un metallo preziosi particolarmente deliziosi (e costosi) ad uno xorn può cementare un'alleanza temporanea. Dato che gli xorn possono attraversare la roccia con facilità sono ottime guide nelle regioni sotterranee.

Gli xorn non sono molto religiosi, ma quelli fra loro che trovano la fede sono solitamente Druidi (anche se è raro, se non improbabile, che gli xorn abbiano Compagni Animali, dato che non possono seguirli nella roccia, e scelgono invece il dominio della Terra). Bardi e Devoti xorn non sono sconosciuti: i Bardi scelgono di solito Intrattenere (canto), e gli Devoti hanno invariabilmente la Stirpe Elementale (terra).


\medskip\index[Mostruario]{Zombi}\textbf{Zombi}

\textit{Media non morto, neutrale malvagio}

\textbf{FORZA} +1

\textbf{DESTREZZA} -2

\textbf{COSTITUZIONE} +3

\textbf{INTELLIGENZA} -4

\textbf{SAGGEZZA} -2

\textbf{CARISMA} -3

\textbf{Iniziativa} -2 -- \textbf{Difesa} 9

\textbf{Punti Ferita} 22 (3d8 + 9)

\textbf{Movimento} 6 m

\textbf{Tiri Salvezza} Tempra +0, Riflessi +0, Volontà +3

\textbf{Immunità al Danno} veleno

\textbf{Immunità alle Condizioni} avvelenato

\textbf{Sensi} scurovisione 18 m

\textbf{Linguaggi} comprende tutte le lingue che parlava in vita ma non può parlare

\textbf{Sfida} 1/4 (50 PX)

\textit{\textbf{Natura Non Morta.}} Lo zombi non ha bisogno di aria, cibo, bevande o sonno.

\textit{\textbf{Tempra dei Non Morti.}} Se il danno riduce lo zombi a 0 Punti Ferita, lo zombi deve effettuare un Tiro Salvezza di Tempra DC 5 + il danno subito, a meno che il danno non sia da Luce o un colpo critico. Se riesce, lo zombi scende invece a 1 punto ferita.

\textbf{Azioni}

\textit{\textbf{Schianto.} Attacco con arma da mischia}: +3 a colpire, portata 1 m, un bersaglio.

\textit{Colpisce:} 4 (1d6 + 1) danni da botta.

\textbf{Ecologia}\\
Ambiente: Qualsiasi\\
Organizzazione: Qualsiasi\\
\textbf{Tesoro}: Nessuno\\
\textbf{Descrizione}\\
Gli zombi sono i cadaveri animati di creature morte, costretti a muoversi da magie necromantiche come Animare Morti. Anche se gli zombi incontrati di norma sono lenti e robusti, altri possiedono tratti differenti, che permettono loro di diffondere una malattia o di muoversi più rapidi.

Gli zombi sono automi senza mente e non possono fare altro che seguire gli ordini. Se lasciati a loro stessi, attendono immobili o si spostano alla ricerca di creature viventi da massacrare e divorare. Gli zombi attaccano fino alla distruzione, senza curarsi della loro sicurezza.

Sebbene siano in grado di seguire gli ordini, gli zombi vengono spesso lasciati liberi con l'ordine di uccidere tutte le creature viventi. Spesso vengono incontrati in branchi che infestano le terre frequentate dai viventi, in cerca di preda. La maggior parte degli zombi viene creata attraverso Animare Morti. Simili zombi sono sempre standard, a meno che il creatore lanci anche Velocità o Rimuovi Paralisi per creare Zombi Rapidi o Contagio per creare Zombi Infetti.


\medskip\index[Mostruario]{Zombi Ogre}\textbf{Zombi Ogre}

\textit{Grande non morto, neutrale malvagio}

\textbf{FORZA} +4

\textbf{DESTREZZA} -2

\textbf{COSTITUZIONE} +4

\textbf{INTELLIGENZA} -4

\textbf{SAGGEZZA} -2

\textbf{CARISMA} -3

\textbf{Iniziativa} -2 -- \textbf{Difesa} 9

\textbf{Punti Ferita} 85 (9d10 + 36)

\textbf{Movimento} 9 m

\textbf{Tiri Salvezza}: Tempra +6, Riflessi +0, Volontà +3

\textbf{Immunità al Danno} veleno

\textbf{Immunità alle Condizioni} avvelenato

\textbf{Sensi} scurovisione 18 m

\textbf{Linguaggi} comprende Comune e Gigante ma non può parlare

\textbf{Sfida} 2 (450 PX)

\textit{\textbf{Natura Non Morta.}} Lo zombi non ha bisogno di aria, cibo, bevande o sonno.

\textit{\textbf{Tempra dei Non Morti.}} Se il danno riduce lo zombi a 0 Punti Ferita, lo zombi deve effettuare un Tiro Salvezza di Tempra DC 5 + il danno subito, a meno che il danno non sia da Luce o un colpo critico. Se riesce, lo zombi scende invece a 1 punto ferita.

\textbf{Azioni}

\textit{\textbf{Mazza Chiodata.} Attacco con arma da mischia}: +6 a colpire, portata 1 m, un bersaglio.

\textit{Colpisce:} 13 (2d8 + 4) danni da botta.


\subsection{Appendice A: Creature Varie}

Questa appendice contiene le statistiche di vari animali, parassiti e
altre creature. Le statistiche sono organizzate in ordine alfabetico.

\medskip\textbf{Albero Risvegliato}\index[Mostruario]{Albero Risvegliato}

L'albero risvegliato è un normale albero fornito dalla magia di capacità
senziente e mobilità.

\textit{Enorme pianta, disallineato}

\textbf{FORZA} +4

\textbf{DESTREZZA} -2

\textbf{COSTITUZIONE} +2

\textbf{INTELLIGENZA} +0

\textbf{SAGGEZZA} +0

\textbf{CARISMA} -2

\textbf{Iniziativa} +0 -- \textbf{Difesa} 14

\textbf{Punti Ferita} 59 (7d12 + 14)

\textbf{Movimento} 6 m

\textbf{Tiri Salvezza}: Tempra +6, Riflessi -1, Volontà +1

\textbf{Vulnerabilità al Danno} fuoco

\textbf{Resistenze al Danno} da botta, perforante

\textbf{Lingue} una lingua conosciuta dal suo creatore

\textbf{Sfida} 2 (450 PX)

\textit{\textbf{Falso Aspetto.}} Mentre l'albero rimane immobile, è indistinguibile da un normale albero.

\textbf{Azioni}

\textit{\textbf{Schianto.} Attacco con Arma da Mischia}: +6 a colpire, portata 3 m, un bersaglio.

\textit{Colpisce:} 14 (3d6 + 4) danni da botta.

\medskip\textbf{Alce}\index[Mostruario]{Alce}

\textit{Grande bestia, disallineato}

\textbf{FORZA} +3

\textbf{DESTREZZA} +0

\textbf{COSTITUZIONE} +1

\textbf{INTELLIGENZA} -4

\textbf{SAGGEZZA} +0

\textbf{CARISMA} -2

\textbf{Iniziativa} +0 -- \textbf{Difesa} 11

\textbf{Punti Ferita} 13 (2d10 + 2)

\textbf{Movimento} 15 m

\textbf{Tiri Salvezza}: Tempra +4, Riflessi +1, Volontà +0

\textbf{Lingue} -

\textbf{Sfida} 1/4 (50 PX)

\textit{\textbf{Carica.}} Se l'alce si muove di almeno 6 metri diretto verso il bersaglio e lo colpisce con un attacco di rostro durante lo stesso turno, il bersaglio subisce 7 (2d6) danni da botta aggiuntivi. Se il bersaglio è una creatura, deve riuscire un Tiro Salvezza di Tempra
DC 13 o cadere prono.

\textbf{Azioni}

\textit{\textbf{Rostro.} Attacco con Arma da Mischia}: +5 a colpire, portata 1 m, un bersaglio.

\textit{Colpisce:} 6 (1d6 + 3) danni da botta.

\textit{\textbf{Zoccoli.} Attacco con Arma da Mischia}: +5 a colpire, portata 1 m, una creatura prona.

\textit{Colpisce:} 8 (2d4 + 3) danni da botta.

\medskip\textbf{Alce Gigante}\index[Mostruario]{Alce Gigante}

\textit{Enorme bestia, disallineato}

\textbf{FORZA} +4

\textbf{DESTREZZA} +3

\textbf{COSTITUZIONE} +2

\textbf{INTELLIGENZA} -2

\textbf{SAGGEZZA} +2

\textbf{CARISMA} +0

\textbf{Iniziativa} +3 -- \textbf{Difesa} 15

\textbf{Punti Ferita} 42 (5d12 + 10)

\textbf{Movimento} 18 m

\textbf{Tiri Salvezza}: Tempra +8, Riflessi +7, Volontà +2

\textbf{Competenze} Consapevolezza +4

\textbf{Lingue} Alce Gigante, comprende il Comune, l'Elfico e il

Silvano ma non può parlarli

\textbf{Sfida} 2 (450 PX)

\textit{\textbf{Carica.}} Se l'alce si muove di almeno 6 metri diretto verso il bersaglio e lo colpisce con un attacco di rostro durante lo stesso turno, il bersaglio subisce 7 (2d6) danni da botta aggiuntivi. Se il bersaglio è una creatura, deve riuscire un Tiro Salvezza di Tempra DC 14 o cadere prono.

\textbf{Azioni}

\textit{\textbf{Rostro.} Attacco con Arma da Mischia}: +6 a colpire, portata 3 m, un bersaglio.

\textit{Colpisce:} 11 (2d6 + 4) danni perforanti.

\textit{\textbf{Zoccoli.} Attacco con Arma da Mischia}: +6 a colpire, portata 1 m, una creatura prona.

\textit{Colpisce:} 22 (4d4 + 4) danni da botta.

\medskip\textbf{Aquila}\index[Mostruario]{Aquila}

\textit{Piccola bestia, disallineato}

\textbf{FORZA} -2

\textbf{DESTREZZA} +2

\textbf{COSTITUZIONE} +0

\textbf{INTELLIGENZA} -4

\textbf{SAGGEZZA} +2

\textbf{CARISMA} -2

\textbf{Iniziativa} +2 -- \textbf{Difesa} 13

\textbf{Punti Ferita} 3 (1d6)

\textbf{Movimento} 3 m, volo 18 m

\textbf{Tiri Salvezza}: Tempra +3, Riflessi +4, Volontà +2

\textbf{Competenze} Consapevolezza +4

\textbf{Lingue} -

\textbf{Sfida} 0 (10 PX)

\textit{\textbf{Vista Affinata.}} L'aquila ha +1d6 nelle prove di Saggezza (Consapevolezza) basate sulla vista.

\textbf{Azioni}

\textit{\textbf{Speroni.} Attacco con Arma da Mischia}: +4 a colpire, portata 1 m, un bersaglio.

\textit{Colpisce:} 4 (1d4 + 2) danni taglienti.

\medskip\textbf{Aquila Gigante}\index[Mostruario]{Aquila Gigante}

L'aquila gigante è una nobile creatura che parla la propria lingua e comprende quella di altre razze.

\textit{Grande bestia, neutrale buono}

\textbf{FORZA} +3

\textbf{DESTREZZA} +3

\textbf{COSTITUZIONE} +1

\textbf{INTELLIGENZA} -1

\textbf{SAGGEZZA} +2

\textbf{CARISMA} +0

\textbf{Iniziativa} +3 -- \textbf{Difesa} 14

\textbf{Punti Ferita} 26 (4d10 + 4)

\textbf{Movimento} 3 m, volo 24 m

\textbf{Tiri Salvezza}: Tempra +5, Riflessi +7, Volontà +3

\textbf{Competenze} Consapevolezza +4

\textbf{Lingue} Aquila Gigante, comprende il Comune e l'Auran ma non può parlarli

\textbf{Sfida} 1 (200 PX)

\textit{\textbf{Vista Affinata.}} L'aquila ha +1d6 nelle prove di Saggezza (Consapevolezza) basate sulla vista.

\textbf{Azioni}

\textit{\textbf{Multiattacco.}} L'aquila effettua due attacchi: uno con il becco e uno con gli speroni.

\textit{\textbf{Becco.} Attacco con Arma da Mischia}: +5 a colpire, portata 1 m, un bersaglio.

\textit{Colpisce:} 6 (1d6 + 3) danni perforanti.

\textit{\textbf{Speroni.} Attacco con Arma da Mischia}: +5 a colpire, portata 1 m, un bersaglio.

\textit{Colpisce:} 10 (2d6 + 3) danni taglienti.

\medskip\textbf{Avvoltoio}\index[Mostruario]{Avvoltoio}

\textit{Media bestia, disallineato}

\textbf{FORZA} -2

\textbf{DESTREZZA} +0

\textbf{COSTITUZIONE} +1

\textbf{INTELLIGENZA} -4

\textbf{SAGGEZZA} +1

\textbf{CARISMA} -3

\textbf{Iniziativa} +0 -- \textbf{Difesa} 11

\textbf{Punti Ferita} 5 (1d8 + 1)

\textbf{Movimento} 3 m, volo 15 m

\textbf{Tiri Salvezza}: Tempra +6, Riflessi +3, Volontà +1; +4 contro malattie

\textbf{Competenze} Consapevolezza +3

\textbf{Lingue} -

\textbf{Sfida} 0 (10 PX)

\textit{\textbf{Olfatto e Vista Affinati.}} L'avvoltoio ha +1d6 nelle prove di Saggezza (Consapevolezza) basate su olfatto o vista.

\textit{\textbf{Tattiche di Branco.}} L'avvoltoio ha +1d6 al tiro di attacco contro una creatura se almeno uno degli alleati dell'avvoltoio si trova entro 1 metro dalla creatura e quell'alleato non è inabile.

\textbf{Azioni}

\textit{\textbf{Becco.} Attacco con Arma da Mischia}: +2 a colpire, portata 1 m, un bersaglio.

\textit{Colpisce:} 2 (1d4) danni perforanti.

\medskip\textbf{Avvoltoio Gigante}\index[Mostruario]{Avvoltoio Gigante}

L'avvoltoio gigante possiede un'intelligenza superiore e un'attitudine maligna.

\textit{Grande bestia, neutrale malvagio}

\textbf{FORZA} +2

\textbf{DESTREZZA} +0

\textbf{COSTITUZIONE} +2

\textbf{INTELLIGENZA} -2

\textbf{SAGGEZZA} +1

\textbf{CARISMA} -2

\textbf{Iniziativa} +0 -- \textbf{Difesa} 11

\textbf{Punti Ferita} 22 (3d10 + 6)

\textbf{Movimento} 3 m, volo 18 m

\textbf{Tiri Salvezza}: Tempra +10, Riflessi +6, Volontà +3; +4 contro malattie

\textbf{Competenze} Consapevolezza +3

\textbf{Lingue} comprende il Comune ma non può parlare

\textbf{Sfida} 1 (200 PX)

\textit{\textbf{Olfatto e Vista Affinati.}} L'avvoltoio ha +1d6 nelle prove di Saggezza (Consapevolezza) basate su olfatto o vista.

\textit{\textbf{Tattiche di Branco.}} L'avvoltoio ha +1d6 al tiro di attacco contro una creatura se almeno uno degli alleati dell'avvoltoio si trova entro 1 metro dalla creatura e quell'alleato non è inabile.

\textbf{Azioni}

\textit{\textbf{Multiattacco.}} L'avvoltoio effettua due attacchi: uno con il becco e uno con gli speroni.

\textit{\textbf{Becco.} Attacco con Arma da Mischia}: +4 a colpire, portata 1 m, un bersaglio.

\textit{Colpisce:} 7 (2d4 + 2) danni perforanti.

\textit{\textbf{Speroni.} Attacco con Arma da Mischia}: +4 a colpire, portata 1 m, un bersaglio.

\textit{Colpisce:} 9 (2d6 + 2) danni taglienti.

\medskip\textbf{Babbuino}\index[Mostruario]{Babbuino}

\textit{Piccola bestia, disallineato}

\textbf{FORZA} -1

\textbf{DESTREZZA} +2

\textbf{COSTITUZIONE} +0

\textbf{INTELLIGENZA} -3

\textbf{SAGGEZZA} +1

\textbf{CARISMA} -2

\textbf{Iniziativa} +2 -- \textbf{Difesa} 13

\textbf{Punti Ferita} 3 (1d6)

\textbf{Movimento} 9 m, scalata 9 m

\textbf{Tiri Salvezza}: Tempra +3, Riflessi +4, Volontà +1

\textbf{Lingue} -

\textbf{Sfida} 0 (10 PX)

\textit{\textbf{Tattiche di Branco.}} Il babbuino ha +1d6 al tiro di attacco contro una creatura se almeno uno degli alleati del babbuino si trova entro 1 metro dalla creatura e quell'alleato non è inabile.

\textbf{Azioni}

\textit{\textbf{Morso.} Attacco con Arma da Mischia}: +1 a colpire, portata 1 m, un bersaglio.

\textit{Colpisce:} 1 (1d4 - 1) danni perforanti.


\medskip\textbf{Balena Assassina (Orca)}\index[Mostruario]{Orca}

\textit{Enorme bestia, disallineato}

\textbf{FORZA} +4

\textbf{DESTREZZA} +0

\textbf{COSTITUZIONE} +1

\textbf{INTELLIGENZA} -4

\textbf{SAGGEZZA} +1

\textbf{CARISMA} -2

\textbf{Iniziativa} +0 -- \textbf{Difesa} 14

\textbf{Punti Ferita} 90 (12d12 + 12)

\textbf{Movimento} 0 m, nuoto 18 m

\textbf{Tiri Salvezza}: Tempra +9, Riflessi +8, Volontà +5

\textbf{Competenze} Consapevolezza +3

\textbf{Sensi} vista cieca 36 m

\textbf{Lingue} -

\textbf{Sfida} 3 (700 PX)

\textit{\textbf{Ecolocazione.}} La balena non può usare la vista cieca se assordata.

\textit{\textbf{Trattenere il Fiato.}} La balena può trattenere il fiato per 30 minuti

\textit{\textbf{Udito Affinato.}} La balena ha +1d6 alle prove di Saggezza (Consapevolezza) basate sull'udito.

\textbf{Azioni}

\textit{\textbf{Morso.} Attacco con Arma da Mischia}: +6 a colpire, portata 1 m, un bersaglio.

\textit{Colpisce:} 21 (5d6 + 4) danni perforanti.

\medskip\textbf{Becco d'Ascia}\index[Mostruario]{Becco d'Ascia}

Il becco d'ascia è un grosso e slanciato volatile privo di ali ma con potenti gambe, un becco a cuneo, e un pessimo carattere.

\textit{Grande bestia, disallineato}

\textbf{FORZA} +2

\textbf{DESTREZZA} +1

\textbf{COSTITUZIONE} +1

\textbf{INTELLIGENZA} -4

\textbf{SAGGEZZA} +0

\textbf{CARISMA} -3

\textbf{Iniziativa} +1 -- \textbf{Difesa} 12

\textbf{Punti Ferita} 19 (3d10 + 3)

\textbf{Movimento} 15 m

\textbf{Tiri Salvezza}: Tempra +3, Riflessi +1, Volontà +1

\textbf{Lingue} -

\textbf{Sfida} 1/4 (50 PX)

\textbf{Azioni}

\textit{\textbf{Becco.} Attacco con Arma da Mischia}: +4 a colpire, portata 1 m, un bersaglio.

\textit{Colpisce:} 6 (1d8 + 2) danni taglienti.

\medskip\textbf{Cammello}\index[Mostruario]{Cammello}

\textit{Grande bestia, disallineato}

\textbf{FORZA} +3

\textbf{DESTREZZA} -1

\textbf{COSTITUZIONE} +2

\textbf{INTELLIGENZA} -4

\textbf{SAGGEZZA} -1

\textbf{CARISMA} -3

\textbf{Iniziativa} -1 -- \textbf{Difesa} 10

\textbf{Punti Ferita} 15 (2d10 + 4)

\textbf{Movimento} 15 m

\textbf{Tiri Salvezza}: Tempra +5, Riflessi +6, Volontà +0

\textbf{Lingue} -

\textbf{Sfida} 1/8 (25 PX)

\textbf{Azioni}

\textit{\textbf{Morso.} Attacco con Arma da Mischia}: +5 a colpire, portata 1 m, un bersaglio.

\textit{Colpisce:} 2 (1d4) danni da botta.

\medskip\textbf{Cane della Morte}\index[Mostruario]{Cane della Morte}

Il cane della morte è un orribile segugio a due teste che si aggira per pianure, deserti e sotterranei.

\textit{Media mostruosità, neutrale malvagio}

\textbf{FORZA} +2

\textbf{DESTREZZA} +2

\textbf{COSTITUZIONE} +2

\textbf{INTELLIGENZA} -4

\textbf{SAGGEZZA} +1

\textbf{CARISMA} -2

\textbf{Iniziativa} +2 -- \textbf{Difesa} 13

\textbf{Punti Ferita} 39 (6d8 + 12)

\textbf{Movimento} 12 m

\textbf{Tiri Salvezza}: Tempra +4, Riflessi +5, Volontà +2

\textbf{Competenze} Muoversi Silenziosamente / Nascondersi +4, Consapevolezza +5

\textbf{Sensi} visione al buio 36 m

\textbf{Lingue} -

\textbf{Sfida} 1 (200 PX)

\textit{\textbf{Bicefalo.}} Il cane ha +1d6 nelle prove di Saggezza (Consapevolezza) e nei Tiri Salvezza contro le condizioni accecato, affascinato, assordato, spaventato, stordito o svenuto.

\textbf{Azioni}

\textit{\textbf{Multiattacco.}} Il cane effettua due attacchi di morso.

\textit{\textbf{Morso.} Attacco con Arma da Mischia}: +4 a colpire, portata 1 m, un bersaglio.

\textit{Colpisce:} 5 (1d6 + 2) danni perforanti. Se il bersaglio è una creatura, deve riuscire un Tiro Salvezza di Tempra DC 12 contro la malattia o restare avvelenato finché la malattia non viene curata. Dopo ogni 24 ore, la creatura deve ripetere il Tiro Salvezza, riducendo i suoi Punti Ferita massimi di 5 (1d10) in caso di fallimento. Questa riduzione perdura finché la malattia non viene curata. La creatura muore se la malattia riduce i suoi Punti Ferita massimi a 0.

\medskip\textbf{Cane Intermittente}\index[Mostruario]{Cane Intermittente}

Il cane intermittente deriva il nome dalla sua abilità di entrare e uscire dalla realtà, un talento che usa per attaccare ed evitare di essere attaccato.

\textit{Media fatato, legale buono}

\textbf{FORZA} +1

\textbf{DESTREZZA} +3

\textbf{COSTITUZIONE} +1

\textbf{INTELLIGENZA} +0

\textbf{SAGGEZZA} +1

\textbf{CARISMA} +0

\textbf{Iniziativa} +3 -- \textbf{Difesa} 14

\textbf{Punti Ferita} 22 (4d8 + 4)

\textbf{Vulnerabilità al Danno} ferro freddo

\textbf{Movimento} 12 m

\textbf{Tiri Salvezza}: Tempra +5, Riflessi +5, Volontà +4

\textbf{Competenze} Muoversi Silenziosamente / Nascondersi +5, Consapevolezza +3

\textbf{Lingue} Cane Intermittente, comprende il Silvano ma non può parlarlo

\textbf{Sfida} 1/4 (50 PX)

\textit{\textbf{Udito e Olfatto Affinato.}} Il cane ha +1d6 nelle prove di Saggezza (Consapevolezza) basate su udito o olfatto.

\textbf{Azioni}

\textit{\textbf{Morso.} Attacco con Arma da Mischia}: +3 a colpire, portata 1 m, un bersaglio.

\textit{Colpisce:} 4 (1d6 + 1) danni perforanti.

\textit{\textbf{Teletrasporto (Ricarica 4-6).}} Il cane si teletrasporta magicamente, insieme a qualsiasi cosa stia indossando o trasportando, fino a 12 metri in uno spazio non occupato che possa vedere. Prima o dopo il teletrasporto, il cane può effettuare un attacco di morso.

\medskip\textbf{Caprone}\index[Mostruario]{Caprone}

\textit{Media bestia, disallineato}

\textbf{FORZA} +1

\textbf{DESTREZZA} +0

\textbf{COSTITUZIONE} +0

\textbf{INTELLIGENZA} -4

\textbf{SAGGEZZA} +0

\textbf{CARISMA} -3

\textbf{Iniziativa} +0 -- \textbf{Difesa} 11

\textbf{Punti Ferita} 4 (1d8)

\textbf{Movimento} 12 m

\textbf{Tiri Salvezza}: Tempra +1, Riflessi +1, Volontà +0

\textbf{Lingue} -

\textbf{Sfida} 0 (10 PX)

\textit{\textbf{Carica.}} Se il caprone si muove di almeno 6 metri diretto verso il bersaglio e colpisce con un attacco di rostro durante lo stesso turno, il bersaglio subisce 2 (1d4) danni da botta aggiuntivi. Se il bersaglio è una creatura, deve riuscire un Tiro Salvezza di Tempra DC 10
o cadere prona.

\textit{\textbf{Piedi Saldi.}} Il caprone ha +1d6 ai Tiri Salvezza su Tempra e Riflessi effettuati contro effetti che lo farebbero cadere prono.

\textbf{Azioni}

\textit{\textbf{Rostro.} Attacco con Arma da Mischia}: +3 a colpire, portata 1 m, un bersaglio.

\textit{Colpisce:} 3 (1d4 + 1) danni da botta.

\medskip\textbf{Caprone Gigante}\index[Mostruario]{Caprone Gigante}

\textit{Grande bestia, disallineato}

\textbf{FORZA} +3

\textbf{DESTREZZA} +0

\textbf{COSTITUZIONE} +1

\textbf{INTELLIGENZA} -4

\textbf{SAGGEZZA} +1

\textbf{CARISMA} -2

\textbf{Iniziativa} +0 -- \textbf{Difesa} 12

\textbf{Punti Ferita} 19 (3d10 + 3)

\textbf{Movimento} 12 m

\textbf{Tiri Salvezza}: Tempra +4, Riflessi +1, Volontà +1

\textbf{Lingue} -

\textbf{Sfida} 1/2 (100 PX)

\textit{\textbf{Carica.}} Se il caprone si muove di almeno 6 metri diretto verso il bersaglio e colpisce con un attacco di rostro durante lo stesso turno, il bersaglio subisce 5 (2d4) danni da botta aggiuntivi. Se il bersaglio è una creatura, deve riuscire un Tiro Salvezza di Tempra DC 13 o cadere prona.

\textit{\textbf{Piedi Saldi.}} Il caprone ha +1d6 ai Tiri Salvezza su Tempra e Riflessi effettuati contro effetti che lo farebbero cadere prono.

\textbf{Azioni}

\textit{\textbf{Rostro.} Attacco con Arma da Mischia}: +5 a colpire, portata 1 m, un bersaglio.

\textit{Colpisce:} 8 (2d4 + 3) danni da botta.

\medskip\textbf{Cavallo da Corsa}\index[Mostruario]{Cavallo da Corsa}

\textit{Grande bestia, disallineato}

\textbf{FORZA} +3

\textbf{DESTREZZA} +0

\textbf{COSTITUZIONE} +1

\textbf{INTELLIGENZA} -4

\textbf{SAGGEZZA} +0

\textbf{CARISMA} -2

\textbf{Iniziativa} +0 -- \textbf{Difesa} 11

\textbf{Punti Ferita} 13 (2d10 + 2)

\textbf{Movimento} 18 m

\textbf{Tiri Salvezza}: Tempra +3, Riflessi +1, Volontà +1

\textbf{Lingue} -

\textbf{Sfida} 1/4 (50 PX)

\textbf{Azioni}

\textit{\textbf{Zoccoli.} Attacco con Arma da Mischia}: +5 a colpire, portata 1 m, un bersaglio.

\textit{Colpisce:} 8 (2d4 + 3) danni da botta.

\medskip\textbf{Cavallo da Guerra}\index[Mostruario]{Cavallo da Guerra}

\textit{Grande bestia, disallineato}

\textbf{FORZA} +4

\textbf{DESTREZZA} +1

\textbf{COSTITUZIONE} +1

\textbf{INTELLIGENZA} -4

\textbf{SAGGEZZA} +1

\textbf{CARISMA} -2

\textbf{Iniziativa} +1 -- \textbf{Difesa} 12 (più possibile bardatura)

\textbf{Punti Ferita} 19 (3d10 + 3)

\textbf{Movimento} 18 m

\textbf{Tiri Salvezza}: Tempra +4, Riflessi +2, Volontà +1

\textbf{Lingue} -

\textbf{Sfida} 1/2 (100 PX)

\textit{\textbf{Carica Travolgente.}} Se il cavallo si muove di almeno 6 metri diretto verso il bersaglio e lo colpisce con un attacco di zoccoli durante lo stesso turno, il bersaglio deve riuscire un Tiro Salvezza su Tempra DC 14 o cadere prono. Se il bersaglio è prono, il cavallo può effettuare un altro attacco di zoccoli contro di lui come azione bonus.

\textbf{Azioni}

\textit{\textbf{Zoccoli.} Attacco con Arma da Mischia}: +6 a colpire, portata 1 m, un bersaglio.

\textit{Colpisce:} 11 (2d6 + 4) danni da botta.

\medskip\textbf{Cavallo da Tiro}\index[Mostruario]{Cavallo da Tiro}

\textit{Grande bestia, disallineato}

\textbf{FORZA} +4

\textbf{DESTREZZA} +0

\textbf{COSTITUZIONE} +1

\textbf{INTELLIGENZA} -4

\textbf{SAGGEZZA} +0

\textbf{CARISMA} -2

\textbf{Iniziativa} +0 -- \textbf{Difesa} 11

\textbf{Punti Ferita} 19 (3d10 + 3)

\textbf{Movimento} 12 m

\textbf{Tiri Salvezza}: Tempra +5, Riflessi +1, Volontà +2

\textbf{Lingue} -

\textbf{Sfida} 1/4 (50 PX)

\textbf{Azioni}

\textit{\textbf{Zoccoli.} Attacco con Arma da Mischia}: +6 a colpire, portata 1 m, un bersaglio.

\textit{Colpisce:} 9 (2d4 + 4) danni da botta.

\medskip\textbf{Cavallo Marino Gigante}\index[Mostruario]{Cavallo Marino Gigante}

Il cavallo marino gigante viene spesso impiegato come cavalcatura dagli umanoidi acquatici.

\textit{Grande bestia, disallineato}

\textbf{FORZA} +1

\textbf{DESTREZZA} +2

\textbf{COSTITUZIONE} +0

\textbf{INTELLIGENZA} -4

\textbf{SAGGEZZA} +1

\textbf{CARISMA} -3

\textbf{Iniziativa} +2 -- \textbf{Difesa} 14

\textbf{Punti Ferita} 16 (3d10)

\textbf{Movimento} 0 m, nuoto 12 m

\textbf{Tiri Salvezza}: Tempra +2, Riflessi +3, Volontà +1

\textbf{Lingue} -

\textbf{Sfida} 1/2 (100 PX)

\textit{\textbf{Carica.}} Se il cavallo marino si muove di almeno 6 metri diretto verso il bersaglio e colpisce con un attacco di rostro durante lo stesso turno, il bersaglio subisce 7 (2d6) danni da botta aggiuntivi. Se il bersaglio è una creatura, deve riuscire un Tiro Salvezza su Tempra DC 11 o cadere prona.

\textit{\textbf{Respirare Acqua.}} Il cavallo marino può respirare solo sott'acqua.

\textbf{Azioni}

\textit{\textbf{Rostro.} Attacco con Arma da Mischia}: +3 a colpire, portata 1 m, un bersaglio.

\textit{Colpisce:} 4 (1d6 + 1) danni da botta.

\medskip\textbf{Centopiedi Gigante}\index{Centopiedi Gigante}

\textit{Piccola bestia, disallineato}

\textbf{FORZA} -3

\textbf{DESTREZZA} +2

\textbf{COSTITUZIONE} +1

\textbf{INTELLIGENZA} -5

\textbf{SAGGEZZA} -2

\textbf{CARISMA} -4

\textbf{Iniziativa} +2 -- \textbf{Difesa} 14

\textbf{Punti Ferita} 4 (1d6 + 1)

\textbf{Movimento} 9 m, scalata 9 m

\textbf{Tiri Salvezza}: Tempra -2, Riflessi +3, Volontà -2

\textbf{Sensi} vista cieca 9 m

\textbf{Lingue} -

\textbf{Sfida} 1/4 (50 PX)

\textbf{Azioni}

\textit{\textbf{Morso.} Attacco con Arma da Mischia}: +4 a colpire, portata 1 m, una creatura.

\textit{Colpisce:} 4 (1d4 + 2) danni perforanti e il bersaglio deve riuscire un Tiro Salvezza di Tempra DC 11 o subire 10 (3d6) danni da veleno. Se il danno da veleno riduce il bersaglio a 0 Punti Ferita, il bersaglio è stabile ma resta avvelenato per 1 ora, anche dopo aver recuperato i Punti Ferita, e mentre è avvelenato in questo modo resta paralizzato.

\medskip\textbf{Cervo}\index[Mostruario]{Cervo}

\textit{Media bestia, disallineato}

\textbf{FORZA} +0

\textbf{DESTREZZA} +3

\textbf{COSTITUZIONE} +0

\textbf{INTELLIGENZA} -4

\textbf{SAGGEZZA} +2

\textbf{CARISMA} -3

\textbf{Iniziativa} +3 -- \textbf{Difesa} 14

\textbf{Punti Ferita} 4 (1d8)

\textbf{Movimento} 15 m

\textbf{Tiri Salvezza}: Tempra +2, Riflessi +3, Volontà +2

\textbf{Lingue} -

\textbf{Sfida} 0 (10 PX)

\textbf{Azioni}

\textit{\textbf{Morso.} Attacco con Arma da Mischia}: +2 a colpire, portata 1 m, un bersaglio.

\textit{Colpisce:} 2 (1d4) danni perforanti.

\medskip\textbf{Cinghiale}\index[Mostruario]{Cinghiale}

\textit{Media bestia, disallineato}

\textbf{FORZA} +1

\textbf{DESTREZZA} +0

\textbf{COSTITUZIONE} +1

\textbf{INTELLIGENZA} -4

\textbf{SAGGEZZA} -1

\textbf{CARISMA} -3

\textbf{Iniziativa} +0 -- \textbf{Difesa} 12

\textbf{Punti Ferita} 11 (2d8 + 2)

\textbf{Movimento} 12 m

\textbf{Tiri Salvezza}: Tempra +2, Riflessi +1, Volontà -1

\textbf{Lingue} -

\textbf{Sfida} 1/4 (50 PX)

\textit{\textbf{Carica.}} Se il cinghiale si muove di almeno 6 metri diretto verso il bersaglio e colpisce con un attacco di zanna durante lo stesso turno, il bersaglio subisce 3 (1d6) danni taglienti aggiuntivi. Se il bersaglio è una creatura, deve riuscire un Tiro Salvezza di Tempra
DC 11 o cadere prono.

\textit{\textbf{Implacabile (Ricarica dopo un 1 ora).}} Se il cinghiale subisce 7 danni o meno che lo ridurrebbero a 0 Punti Ferita, scende invece a 1 punto ferita.

\textbf{Azioni}

\textit{\textbf{Zanna.} Attacco con Arma da Mischia}: +3 a colpire, portata 1 m, un bersaglio.

\textit{Colpisce:} 4 (1d6 + 1) danni taglienti.

\medskip\textbf{Cinghiale Gigante}\index[Mostruario]{Cinghiale Gigante}

\textit{Grande bestia, disallineato}

\textbf{FORZA} +3

\textbf{DESTREZZA} +0

\textbf{COSTITUZIONE} +3

\textbf{INTELLIGENZA} -4

\textbf{SAGGEZZA} -2

\textbf{CARISMA} -3

\textbf{Iniziativa} +0 -- \textbf{Difesa} 13

\textbf{Punti Ferita} 42 (5d10 + 15)

\textbf{Movimento} 12 m

\textbf{Tiri Salvezza}: Tempra +4, Riflessi +2, Volontà +0

\textbf{Lingue} -

\textbf{Sfida} 2 (450 PX)

\textit{\textbf{Carica.}} Se il cinghiale si muove di almeno 6 metri diretto verso il bersaglio e colpisce con un attacco di zanna durante lo stesso turno, il bersaglio subisce 7 (2d6) danni taglienti aggiuntivi. Se il bersaglio è una creatura, deve riuscire un Tiro Salvezza di Tempra DC 13 o cadere prono.

\textit{\textbf{Implacabile (Ricarica dopo un 1 ora).}} Se il cinghiale subisce 10 danni o meno che lo ridurrebbero a 0 Punti Ferita, scende invece a 1 punto ferita.

\textbf{Azioni}

\textit{\textbf{Zanna.} Attacco con Arma da Mischia}: +5 a colpire, portata 1 m, un bersaglio.

\textit{Colpisce:} 10 (2d6 + 3) danni taglienti.

\medskip\textbf{Coccodrillo}\index[Mostruario]{Coccodrillo}

\textit{Grande bestia, disallineato}

\textbf{FORZA} +2

\textbf{DESTREZZA} +0

\textbf{COSTITUZIONE} +1

\textbf{INTELLIGENZA} -4

\textbf{SAGGEZZA} +0

\textbf{CARISMA} -3

\textbf{Iniziativa} +0 -- \textbf{Difesa} 13

\textbf{Punti Ferita} 19 (3d10 + 3)

\textbf{Movimento} 6 m, nuoto 9 m

\textbf{Tiri Salvezza}: Tempra +6, Riflessi +4, Volontà +2

\textbf{Competenze} Muoversi Silenziosamente / Nascondersi +2

\textbf{Lingue} -

\textbf{Sfida} 1/2 (100 PX)

\textit{\textbf{Trattenere il Fiato.}} Il coccodrillo può trattenere il fiato per 15 minuti.

\textbf{Azioni}

\textit{\textbf{Morso.} Attacco con Arma da Mischia}: +4 a colpire, portata 1 m, una creatura.

\textit{Colpisce:} 7 (1d10 + 2) danni perforanti, e il bersaglio è afferrato (DC 12 per fuggire). Fino al termine dell'afferrare, il bersaglio è intralciato, e il coccodrillo non può usare il morso contro un altro bersaglio.

\medskip\textbf{Coccodrillo Gigante}\index[Mostruario]{Coccodrillo Gigante}

\textit{Enorme bestia, disallineato}

\textbf{FORZA} +5

\textbf{DESTREZZA} -1

\textbf{COSTITUZIONE} +3

\textbf{INTELLIGENZA} -4

\textbf{SAGGEZZA} +0

\textbf{CARISMA} -2

\textbf{Iniziativa} -1 -- \textbf{Difesa} 15

\textbf{Punti Ferita} 85 (9d12 + 27)

\textbf{Movimento} 9 m, nuoto 15 m

\textbf{Tiri Salvezza}: Tempra +15, Riflessi +8, Volontà +8

\textbf{Competenze} Muoversi Silenziosamente / Nascondersi +5

\textbf{Lingue} -

\textbf{Sfida} 5 (1.800 PX)

\textit{\textbf{Trattenere il Fiato.}} Il coccodrillo può trattenere il fiato per 30 minuti.

\textbf{Azioni}

\textit{\textbf{Multiattacco.}} Il coccodrillo effettua due attacchi: uno con il morso e uno con la coda.

\textit{\textbf{Coda.} Attacco con Arma da Mischia}: +8 a colpire, portata 3 m, un bersaglio non afferrato dal coccodrillo.

\textit{Colpisce:} 14 (2d8 + 5) danni da botta. Se il bersaglio è una creatura, deve riuscire un Tiro Salvezza di Tempra DC 16 o cadere prono.

\textit{\textbf{Morso.} Attacco con Arma da Mischia}: +8 a colpire, portata 1 m, un bersaglio.

\textit{Colpisce:} 21 (3d10 + 5) danni perforanti, e il bersaglio è afferrato (DC 16 per fuggire). Fino al termine dell'afferrare, il bersaglio è intralciato, e il coccodrillo non può usare il morso contro un altro bersaglio.

\medskip\textbf{Corvo}\index[Mostruario]{Corvo}

\textit{Minuscola bestia, disallineato}

\textbf{FORZA} -4

\textbf{DESTREZZA} +2

\textbf{COSTITUZIONE} -1

\textbf{INTELLIGENZA} -4

\textbf{SAGGEZZA} +1

\textbf{CARISMA} -2

\textbf{Iniziativa} +2 -- \textbf{Difesa} 13

\textbf{Punti Ferita} 1 (1d4 - 1)

\textbf{Movimento} 3 m, volo 15 m

\textbf{Tiri Salvezza}: Tempra +1, Riflessi +4, Volontà +2

\textbf{Competenze} Consapevolezza +3

\textbf{Lingue} -

\textbf{Sfida} 0 (10 PX)

\textit{\textbf{Imitazione.}} Il corvo può imitare dei semplici suoni che ha udito, come il sussurro di una persona, il pianto di un bambino o il verso di un animale. Una creatura che ode il suono può identificarlo come imitazione riuscendo una prova di Saggezza (Sopravvivenza) DC 10.

\textbf{Azioni}

\textit{\textbf{Becco.} Attacco con Arma da Mischia}: +4 a colpire, portata 1 m, un bersaglio.

\textit{Colpisce:} 1 danno perforante.

%\medskip\textbf{Donnola}\index[Mostruario]{Donnola}

%\textit{Minuscola bestia, disallineato}

%\textbf{FORZA} -4

%\textbf{DESTREZZA} +3

%\textbf{COSTITUZIONE} -1

%\textbf{INTELLIGENZA} -4

%\textbf{SAGGEZZA} +1

%\textbf{CARISMA} -4

%\textbf{Iniziativa} +3 -- \textbf{Difesa} 14

%\textbf{Punti Ferita} 1 (1d4 - 1)

%\textbf{Movimento} 9 m

%\textbf{Tiri Salvezza}: Tempra +2, Riflessi +4, Volontà +1

%\textbf{Competenze} Muoversi Silenziosamente / Nascondersi +5, Consapevolezza +3

%\textbf{Lingue} -

%\textbf{Sfida} 0 (10 PX)

%\textit{\textbf{Udito e Olfatto Affinati.}} La donnola ha +1d6 nelle prove di Saggezza (Consapevolezza) basate su udito o olfatto.

%\textbf{Azioni}

%\textit{\textbf{Morso.} Attacco con Arma da Mischia}: +5 a colpire, portata 1 m, un bersaglio.

%\textit{Colpisce:} 1 danno perforante.

\medskip\textbf{Donnola Gigante}\index[Mostruario]{Donnola Gigante}

\textit{Media bestia, disallineato}

\textbf{FORZA} +0

\textbf{DESTREZZA} +3

\textbf{COSTITUZIONE} +0

\textbf{INTELLIGENZA} -3

\textbf{SAGGEZZA} +1

\textbf{CARISMA} -3

\textbf{Iniziativa} +3 -- \textbf{Difesa} 14

\textbf{Punti Ferita} 9 (2d8)

\textbf{Movimento} 12 m

\textbf{Tiri Salvezza}: Tempra +6, Riflessi +7, Volontà +2

\textbf{Competenze} Muoversi Silenziosamente / Nascondersi +5, Consapevolezza +3

\textbf{Sensi} visione al buio 18 m

\textbf{Lingue} -

\textbf{Sfida} 1/8 (25 PX)

\textit{\textbf{Udito e Olfatto Affinati.}} La donnola ha +1d6 nelle prove di Saggezza (Consapevolezza) basate su udito o olfatto.

\textbf{Azioni}

\textit{\textbf{Morso.} Attacco con Arma da Mischia}: +5 a colpire, portata 1 m, un bersaglio.

\textit{Colpisce:} 5 (1d4 + 3) danni perforanti.

\medskip\textbf{Elefante}\index[Mostruario]{Elefante}

\textit{Enorme bestia, disallineato}

\textbf{FORZA} +6

\textbf{DESTREZZA} -1

\textbf{COSTITUZIONE} +3

\textbf{INTELLIGENZA} -4

\textbf{SAGGEZZA} +0

\textbf{CARISMA} -2

\textbf{Iniziativa} -1 -- \textbf{Difesa} 14

\textbf{Punti Ferita} 76 (8d12 + 24)

\textbf{Movimento} 12 m

\textbf{Tiri Salvezza}: Tempra +13, Riflessi +7, Volontà +6

\textbf{Lingue} -

\textbf{Sfida} 4 (1000 PX)

\textit{\textbf{Carica Travolgente.}} Se l'elefante si muove di almeno 6 metri diretto verso una creatura e la colpisce con un attacco di incornata durante lo stesso turno, il bersaglio deve riuscire un Tiro Salvezza su Tempra DC 12 o cadere prono. Se il bersaglio è prono, l'elefante può effettuare un attacco di pestone contro di lui come azione bonus.

\textbf{Azioni}

\textit{\textbf{Incornata.} Attacco con Arma da Mischia}: +8 a colpire, portata 1 m, un bersaglio.

\textit{Colpisce:} 19 (3d8 + 6) danni perforanti.

\textit{\textbf{Pestone.} Attacco con Arma da Mischia}: +8 a colpire, portata 1 m, un bersaglio prono.

\textit{Colpisce:} 22 (3d10 + 6) danni da botta.

\medskip\textbf{Falco}\index[Mostruario]{Falco}

\textit{Minuscola bestia, disallineato}

\textbf{FORZA} -3

\textbf{DESTREZZA} +3

\textbf{COSTITUZIONE} -1

\textbf{INTELLIGENZA} -4

\textbf{SAGGEZZA} +2

\textbf{CARISMA} -2

\textbf{Iniziativa} +3 -- \textbf{Difesa} 14

\textbf{Punti Ferita} 1 (1d4 - 1)

\textbf{Movimento} 3 m, volo 18 m

\textbf{Tiri Salvezza}: Tempra +2, Riflessi +5, Volontà +2

\textbf{Competenze} Consapevolezza +4

\textbf{Lingue} -

\textbf{Sfida} 0 (10 PX)

\textit{\textbf{Vista Affinata.}} Il falco ha +1d6 alle prove di Saggezza (Consapevolezza) basate sulla vista.

\textbf{Azioni}

\textit{\textbf{Speroni.} Attacco con Arma da Mischia}: +5 a colpire, portata 1 m, un bersaglio.

\textit{Colpisce:} 1 danno tagliente.

\medskip\textbf{Falco di Sangue}\index[Mostruario]{Falco di Sangue}

Dovendo il suo nome alle sue piume cremisi e alla sua natura aggressiva, il falco di sangue attacca senza timore usando il suo becco appuntito.

\textit{Piccola bestia, disallineato}

\textbf{FORZA} -2

\textbf{DESTREZZA} +2

\textbf{COSTITUZIONE} +0

\textbf{INTELLIGENZA} -4

\textbf{SAGGEZZA} +2

\textbf{CARISMA} -3

\textbf{Iniziativa} +2 -- \textbf{Difesa} 13

\textbf{Punti Ferita} 7 (2d6)

\textbf{Movimento} 3 m, volo 18 m

\textbf{Tiri Salvezza}: Tempra +3, Riflessi +6, Volontà +3

\textbf{Competenze} Consapevolezza +4

\textbf{Lingue} -

\textbf{Sfida} 1/8 (25 PX)

\textit{\textbf{Tattiche di Branco.}} Il falco ha +1d6 ai tiri di attacco contro una creatura se almeno uno degli alleati del falco si trova entro 1 metro dalla creatura e quell'alleato non è inabile.

\textit{\textbf{Vista Affinata.}} Il falco ha +1d6 alle prove di Saggezza (Consapevolezza) basate sulla vista.

\textbf{Azioni}

\textit{\textbf{Becco.} Attacco con Arma da Mischia}: +4 a colpire, portata 1 m, un bersaglio.

\textit{Colpisce:} 4 (1d4 + 2) danni perforanti.

\medskip\textbf{Pirana}\index[Mostruario]{Pirana}

Il pirana è un pesce carnivoro dai denti affilati.

\textit{Minuscola bestia, disallineato}

\textbf{FORZA} -4

\textbf{DESTREZZA} +3

\textbf{COSTITUZIONE} -1

\textbf{INTELLIGENZA} -5

\textbf{SAGGEZZA} -2

\textbf{CARISMA} -4

\textbf{Iniziativa} +3 -- \textbf{Difesa} 14

\textbf{Punti Ferita} 1 (1d4 - 1)

\textbf{Movimento} 0 m, nuoto 12 m

\textbf{Tiri Salvezza}: Tempra -4, Riflessi +3, Volontà -2

\textbf{Sensi} visione al buio 18 m

\textbf{Lingue} -

\textbf{Sfida} 0 (10 PX)

\textit{\textbf{Frenesia Sanguinaria.}} Il pirana ha +1d6 ai tiri di attacco in mischia contro qualsiasi creatura che non sia al massimo dei Punti Ferita.

\textit{\textbf{Respirare Acqua.}} Il pirana può respirare solo sott'acqua.

\textbf{Azioni}

\textit{\textbf{Morso.} Attacco con Arma da Mischia}: +5 a colpire, portata 1 m, un bersaglio.

\textit{Colpisce:} 1 danno perforante.

\medskip\textbf{Gatto}\index[Mostruario]{Gatto}

\textit{Minuscola bestia, disallineato}

\textbf{FORZA} -4

\textbf{DESTREZZA} +2

\textbf{COSTITUZIONE} +0

\textbf{INTELLIGENZA} -4

\textbf{SAGGEZZA} +1

\textbf{CARISMA} -2

\textbf{Iniziativa} +2 -- \textbf{Difesa} 13

\textbf{Punti Ferita} 2 (1d4)

\textbf{Movimento} 12 m, scalata 9 m

\textbf{Tiri Salvezza}: Tempra +1, Riflessi +4, Volontà +1

\textbf{Competenze} Muoversi Silenziosamente / Nascondersi +4, Consapevolezza +3

\textbf{Lingue} -

\textbf{Sfida} 0 (10 PX)

\textit{\textbf{Olfatto Affinato.}} Il gatto ha +1d6 alle prove di Saggezza (Consapevolezza) basate sull'olfatto.

\textbf{Azioni}

\textit{\textbf{Artigli.} Attacco con Arma da Mischia}: +0 a colpire, portata 1 m, un bersaglio.

\textit{Colpisce:} 1 danno tagliente.

\medskip\textbf{Granchio Gigante}\index[Mostruario]{Granchio Gigante}

\textit{Media bestia, disallineato}

\textbf{FORZA} +1

\textbf{DESTREZZA} +2

\textbf{COSTITUZIONE} +0

\textbf{INTELLIGENZA} -5

\textbf{SAGGEZZA} -1

\textbf{CARISMA} -4

\textbf{Iniziativa} +2 -- \textbf{Difesa} 16

\textbf{Punti Ferita} 13 (3d8)

\textbf{Movimento} 9 m, nuoto 9 m

\textbf{Tiri Salvezza}: Tempra +5, Riflessi +2, Volontà +1

\textbf{Competenze} Muoversi Silenziosamente / Nascondersi +4

\textbf{Sensi} vista cieca 9 m

\textbf{Lingue} -

\textbf{Sfida} 1/8 (25 PX)

\textit{\textbf{Anfibio.}} Il granchio può respirare aria e acqua.

\textbf{Azioni}

\textit{\textbf{Artiglio (Chela).} Attacco con Arma da Mischia}: +3 a colpire, portata 1 m, un bersaglio.

\textit{Colpisce:} 4 (1d6 + 1) danni da botta e il bersaglio è afferrato (DC 11 per fuggire). Il granchio ha due chele, ciascuna delle quali può afferrare un solo bersaglio.

\medskip\textbf{Gufo}\index[Mostruario]{Gufo}

\textit{Minuscola bestia, disallineato}

\textbf{FORZA} -4

\textbf{DESTREZZA} +1

\textbf{COSTITUZIONE} -1

\textbf{INTELLIGENZA} -4

\textbf{SAGGEZZA} +1

\textbf{CARISMA} -2

\textbf{Iniziativa} +1 -- \textbf{Difesa} 12

\textbf{Punti Ferita} 1 (1d4 - 1)

\textbf{Movimento} 1 m, volo 18 m

\textbf{Tiri Salvezza}: Tempra +2, Riflessi +5, Volontà +2

\textbf{Competenze} Muoversi Silenziosamente / Nascondersi +3, Consapevolezza +3

\textbf{Sensi} visione al buio 36 m

\textbf{Lingue} -

\textbf{Sfida} 0 (10 PX)

\textit{\textbf{Sorvolare.}} Il gufo non provoca attacchi di opportunità quando vola via dalla portata di un nemico.

\textit{\textbf{Udito e Vista Affinati.}} Il gufo ha +1d6 nelle prove di Saggezza (Consapevolezza) basate su udito o vista.

\textbf{Azioni}

\textit{\textbf{Speroni.} Attacco con Arma da Mischia}: +3 a colpire, portata 1 m, un bersaglio.

\textit{Colpisce:} 1 danno tagliente.

\medskip\textbf{Gufo Gigante}\index[Mostruario]{Gufo Gigante}

I gufi giganti sono creature intelligenti che proteggono i regni silvani.

\textit{Grande bestia, neutrale}

\textbf{FORZA} +1

\textbf{DESTREZZA} +2

\textbf{COSTITUZIONE} +1

\textbf{INTELLIGENZA} -1

\textbf{SAGGEZZA} +1

\textbf{CARISMA} +0

\textbf{Iniziativa} +2 -- \textbf{Difesa} 13

\textbf{Punti Ferita} 19 (3d10 + 3)

\textbf{Movimento} 1 m, volo 18 m

\textbf{Tiri Salvezza}: Tempra +1, Riflessi +4, Volontà +1

\textbf{Competenze} Muoversi Silenziosamente / Nascondersi +4, Consapevolezza +5

\textbf{Sensi} visione al buio 36 m

\textbf{Lingue} Gufo Gigante, comprende Comune, Elfico e Silvano ma non può parlarli

\textbf{Sfida} 1/4 (50 PX)

\textit{\textbf{Sorvolare.}} Il gufo non provoca attacchi di opportunità quando vola via dalla portata di un nemico.

\textit{\textbf{Udito e Vista Affinati.}} Il gufo ha +1d6 nelle prove di Saggezza (Consapevolezza) basate su udito o vista.

\textbf{Azioni}

\textit{\textbf{Speroni.} Attacco con Arma da Mischia}: +3 a colpire, portata 1 m, un bersaglio.

\textit{Colpisce:} 8 (2d6 + 1) danni perforanti.

\medskip\textbf{Iena}\index[Mostruario]{Iena}

\textit{Media bestia, disallineato}

\textbf{FORZA} +0

\textbf{DESTREZZA} +1

\textbf{COSTITUZIONE} +1

\textbf{INTELLIGENZA} -4

\textbf{SAGGEZZA} +1

\textbf{CARISMA} -3

\textbf{Iniziativa} +1 -- \textbf{Difesa} 12

\textbf{Punti Ferita} 5 (1d8 + 1)

\textbf{Movimento} 15 m

\textbf{Tiri Salvezza}: Tempra +5, Riflessi +5, Volontà +1

\textbf{Competenze} Consapevolezza +3

\textbf{Lingue} -

\textbf{Sfida} 0 (10 PX)

\textit{\textbf{Tattiche di Branco.}} La iena ha +1d6 ai tiri di attacco contro una creatura se almeno uno degli alleati della iena si trova entro 1 metro dalla creatura e quell'alleato non è inabile.

\textbf{Azioni}

\textit{\textbf{Morso.} Attacco con Arma da Mischia}: +2 a colpire, portata 1 m, un bersaglio.

\textit{Colpisce:} 3 (1d6) danni perforanti.

\medskip\textbf{Iena Gigante}\index[Mostruario]{Iena Gigante}

\textit{Grande bestia, disallineato}

\textbf{FORZA} +3

\textbf{DESTREZZA} +2

\textbf{COSTITUZIONE} +2

\textbf{INTELLIGENZA} -4

\textbf{SAGGEZZA} +1

\textbf{CARISMA} -2

\textbf{Iniziativa} +2 -- \textbf{Difesa} 13

\textbf{Punti Ferita} 45 (6d10 + 12)

\textbf{Movimento} 15 m

\textbf{Tiri Salvezza}: Tempra +6, Riflessi +6, Volontà +2

\textbf{Competenze} Consapevolezza +3

\textbf{Lingue} -

\textbf{Sfida} 1 (200 PX)

\textit{\textbf{Rabbia.}} Quando la iena riduce una creatura a 0 Punti Ferita con un attacco di mischia durante il proprio round, la iena può svolgere un'azione bonus per muoversi fino a metà del suo movimento ed effettuare un attacco di morso.

\textbf{Azioni}

\textit{\textbf{Morso.} Attacco con Arma da Mischia}: +5 a colpire, portata 1 m, un bersaglio.

\textit{Colpisce:} 10 (2d6 + 3) danni perforanti.

\medskip\textbf{Leone}\index[Mostruario]{Leone}

\textit{Grande bestia, disallineato}

\textbf{FORZA} +3

\textbf{DESTREZZA} +2

\textbf{COSTITUZIONE} +1

\textbf{INTELLIGENZA} -4

\textbf{SAGGEZZA} +1

\textbf{CARISMA} -1

\textbf{Iniziativa} +2 -- \textbf{Difesa} 13

\textbf{Punti Ferita} 26 (4d10 + 4)

\textbf{Movimento} 15 m

\textbf{Tiri Salvezza}: Tempra +6, Riflessi +7, Volontà +2

\textbf{Competenze} Muoversi Silenziosamente / Nascondersi +6, Consapevolezza +3

\textbf{Lingue} -

\textbf{Sfida} 1 (200 PX)

\textit{\textbf{Balzo.}} Se il leone si muove di almeno 6 metri diretto verso una creatura e la colpisce con un attacco di artiglio durante lo stesso turno, il bersaglio deve riuscire un Tiro Salvezza di Tempra DC 13 o cadere prono. Se il bersaglio è prono, il leone può effettuare un
attacco di morso come azione bonus.

\textit{\textbf{Olfatto Affinato.}} Il leone ha +1d6 alle prove di Saggezza (Consapevolezza) basate sull'olfatto.

\textit{\textbf{Salto con Rincorsa.}} Con 3 metri di rincorsa, il leone può saltare in lungo fino a 7 metri.

\textit{\textbf{Tattiche di Branco.}} Il leone ha +1d6 ai tiri di attacco contro una creatura se almeno uno degli alleati del leone si trova entro 1 metro dalla creatura e quell'alleato non è inabile.

\textbf{Azioni}

\textit{\textbf{Artiglio.} Attacco con Arma da Mischia}: +5 a colpire, portata 1 m, un bersaglio.

\textit{Colpisce:} 6 (1d6 + 3) danni taglienti, 1 danno da Sanguinamento.

\textit{\textbf{Morso.} Attacco con Arma da Mischia}: +5 a colpire, portata 1 m, un bersaglio.

\textit{Colpisce:} 7 (1d8 + 3) danni perforanti.

\medskip\textbf{Lucertola}\index[Mostruario]{Lucertola}

\textit{Minuscola bestia, disallineato}

\textbf{FORZA} -4

\textbf{DESTREZZA} +0

\textbf{COSTITUZIONE} +0

\textbf{INTELLIGENZA} -5

\textbf{SAGGEZZA} -1

\textbf{CARISMA} -4

\textbf{Iniziativa} +0 -- \textbf{Difesa} 11

\textbf{Punti Ferita} 2 (1d4)

\textbf{Movimento} 6 m, scalata 6 m

\textbf{Tiri Salvezza}: Tempra +1, Riflessi +4, Volontà +1

\textbf{Sensi} visione al buio 9 m

\textbf{Lingue} -

\textbf{Sfida} 0 (10 PX)

\textit{\textbf{Scalare come Ragno.}} La lucertola può scalare superfici difficili, compreso lo stare a testa in giù sul soffitto, senza bisogno di effettuare una prova di abilità.

\textbf{Azioni}

\textit{\textbf{Morso.} Attacco con Arma da Mischia}: +0 a colpire, portata 1 m, un bersaglio.

\textit{Colpisce:} 1 danno perforante.

\medskip\textbf{Lucertola Gigante}\index[Mostruario]{Lucertola Gigante}

Le lucertole giganti sono temibili predatori e spesso vengono impiegate come cavalcature o animali da tiro da umanoidi rettiloidi e residenti del sottosuolo.

\textit{Grande bestia, disallineato}

\textbf{FORZA} +2

\textbf{DESTREZZA} +1

\textbf{COSTITUZIONE} +1

\textbf{INTELLIGENZA} -4

\textbf{SAGGEZZA} +0

\textbf{CARISMA} -3

\textbf{Iniziativa} +1 -- \textbf{Difesa} 13

\textbf{Punti Ferita} 19 (3d10 + 3)

\textbf{Movimento} 9 m, scalata 9 m

\textbf{Tiri Salvezza}: Tempra +11, Riflessi +8, Volontà +4

\textbf{Sensi} visione al buio 9 m

\textbf{Lingue} -

\textbf{Sfida} 1/4 (50 PX)

\textbf{Azioni}

\textit{\textbf{Morso.} Attacco con Arma da Mischia}: +4 a colpire, portata 1 m, un bersaglio.

\textit{Colpisce:} 6 (1d8 + 2) danni perforanti.

\textbf{VARIANTE}

Alcune lucertole giganti possiedono uno o entrambi i seguenti tratti.

\textit{\textbf{Scalare come Ragno.}} La lucertola può scalare superfici difficili, compreso lo stare a testa in giù sul soffitto, senza bisogno di effettuare una prova di abilità.

\textit{\textbf{Trattenere il Fiato.}} La lucertola può trattenere il fiato per 15 minuti. (Una lucertola con questo tratto possiede anche velocità di nuoto 9 metri).

\medskip\textbf{Lupo}\index[Mostruario]{Lupo}

\textit{Media bestia, disallineato}

\textbf{FORZA} +1

\textbf{DESTREZZA} +2

\textbf{COSTITUZIONE} +1

\textbf{INTELLIGENZA} -4

\textbf{SAGGEZZA} +1

\textbf{CARISMA} -2

\textbf{Iniziativa} +2 -- \textbf{Difesa} 14

\textbf{Punti Ferita} 11 (2d8 + 2)

\textbf{Movimento} 12 m

\textbf{Tiri Salvezza}: Tempra +5, Riflessi +5, Volontà +1

\textbf{Competenze} Muoversi Silenziosamente / Nascondersi +4, Consapevolezza +3

\textbf{Lingue} -

\textbf{Sfida} 1/4 (50 PX)

\textit{\textbf{Udito e Olfatto Affinato.}} Il lupo ha +1d6 nelle prove di Saggezza (Consapevolezza) basate su udito o olfatto.

\textit{\textbf{Tattiche di Branco.}} Il lupo ha +1d6 ai tiri di attacco contro una creatura se almeno uno degli alleati del lupo si trova entro 1 metro dalla creatura e quell'alleato non è inabile.

\textbf{Azioni}

\textit{\textbf{Morso.} Attacco con Arma da Mischia}: +4 a colpire, portata 1 m, un bersaglio.

\textit{Colpisce:} 7 (2d4 + 2) danni perforanti. Se il bersaglio è una creatura, deve riuscire un Tiro Salvezza di Tempra DC 11 o cadere prona.

\medskip\textbf{Dinolupo (Metalupo)}\index[Mostruario]{Dinolupo (Metalupo}

\textit{Grande bestia, disallineato}

\textbf{FORZA} +3

\textbf{DESTREZZA} +2

\textbf{COSTITUZIONE} +2

\textbf{INTELLIGENZA} -2

\textbf{SAGGEZZA} +1

\textbf{CARISMA} -2

\textbf{Iniziativa} +2 -- \textbf{Difesa} 15

\textbf{Punti Ferita} 37 (5d10 + 10)

\textbf{Movimento} 15 m

\textbf{Tiri Salvezza}: Tempra +7, Riflessi +6, Volontà +2

\textbf{Competenze} Muoversi Silenziosamente / Nascondersi +4, Consapevolezza +3

\textbf{Lingue} -

\textbf{Sfida} 1 (200 PX)

\textit{\textbf{Udito e Olfatto Affinato.}} Il lupo ha +1d6 nelle prove di Saggezza (Consapevolezza) basate su udito o olfatto.

\textit{\textbf{Tattiche di Branco.}} Il lupo ha +1d6 ai tiri di attacco contro una creatura se almeno uno degli alleati del lupo si trova entro 1 metro dalla creatura e quell'alleato non è inabile.

\textbf{Azioni}

\textit{\textbf{Morso.} Attacco con Arma da Mischia}: +5 a colpire, portata 1 m, un bersaglio.

\textit{Colpisce:} 10 (2d6 + 3) danni perforanti. Se il bersaglio è una creatura, deve riuscire un Tiro Salvezza di Tempra DC 13 o cadere prona.

\medskip\textbf{Lupo Invernale}\index[Mostruario]{Lupo Invernale}

I lupi invernali abitano nelle regioni artiche e sono creature malvagie e intelligenti dal manto bianco come la neve e gli occhi color del ghiaccio.

\textit{Grande mostruosità, neutrale malvagio}

\textbf{FORZA} +4

\textbf{DESTREZZA} +1

\textbf{COSTITUZIONE} +2

\textbf{INTELLIGENZA} -2

\textbf{SAGGEZZA} +1

\textbf{CARISMA} -1

\textbf{Iniziativa} +1 -- \textbf{Difesa} 15

\textbf{Punti Ferita} 75 (10d10 + 20)

\textbf{Movimento} 15 m

\textbf{Tiri Salvezza}: Tempra +9, Riflessi +6, Volontà +3

\textbf{Competenze} Muoversi Silenziosamente / Nascondersi +3, Consapevolezza +5

\textbf{Immunità al Danno} freddo

\textbf{Lingue} Comune, Gigante, Lupo Invernale

\textbf{Sfida} 3 (700 PX)

\textit{\textbf{Camuffamento di Neve.}} Il lupo ha +1d6 alle prove di Destrezza (Nascondersi) effettuate per nascondersi su terreno innevato.

\textit{\textbf{Udito e Olfatto Affinato.}} Il lupo ha +1d6 nelle prove di Saggezza (Consapevolezza) basate su udito o olfatto.

\textit{\textbf{Tattiche di Branco.}} Il lupo ha +1d6 ai tiri di attacco contro una creatura se almeno uno degli alleati del lupo si trova entro 1 metro dalla creatura e quell'alleato non è inabile.

\textbf{Azioni}

\textit{\textbf{Morso.} Attacco con Arma da Mischia}: +6 a colpire, portata 1 m, un bersaglio.

\textit{Colpisce:} 11 (2d6 + 4) danni perforanti. Se il bersaglio è una creatura, deve riuscire un Tiro Salvezza di Tempra DC 14 o cadere prona.

\textit{\textbf{Soffio Gelido (Ricarica 5-6).}} Il lupo esala un'esplosione di vento gelido in un cono di 5 metri. Ogni creatura in quell'area deve effettuare un Tiro Salvezza di Riflessi DC 12, e subire 18 (4d8) danni da freddo se fallisce il Tiro Salvezza, o la metà di questi danni se lo riesce.

\medskip\textbf{Mammut}\index[Mostruario]{Mammut}

Il mammut è una creatura simile all'elefante dalla folta pelliccia e lunghe zanne.

\textit{Enorme bestia, disallineato}

\textbf{FORZA} +7

\textbf{DESTREZZA} -1

\textbf{COSTITUZIONE} +5

\textbf{INTELLIGENZA} -4

\textbf{SAGGEZZA} +0

\textbf{CARISMA} -2

\textbf{Iniziativa} -1 -- \textbf{Difesa} 16

\textbf{Punti Ferita} 126 (11d12 + 55)

\textbf{Movimento} 12 m

\textbf{Tiri Salvezza}: Tempra +14, Riflessi +10, Volontà +7

\textbf{Lingue} -

\textbf{Sfida} 6 (2.300 PX)

\textit{\textbf{Carica Travolgente.}} Se il mammut si muove di almeno 6 metri diretto verso una creatura e la colpisce con un attacco di incornata durante lo stesso turno, il bersaglio deve riuscire un Tiro Salvezza su Tempra DC 18 o cadere prono. Se il bersaglio è prono, il mammut può effettuare un attacco di pestone contro di lui come azione bonus.

\textbf{Azioni}

\textit{\textbf{Incornata.} Attacco con Arma da Mischia}: +10 a colpire, portata 3 m, un bersaglio.

\textit{Colpisce:} 25 (4d8 + 7) danni perforanti.

\textit{\textbf{Pestone.} Attacco con Arma da Mischia}: +10 a colpire, portata 1 m, una creatura prona.

\textit{Colpisce:} 29 (4d10 + 7) danni da botta.

\medskip\textbf{Mastino}\index[Mostruario]{Mastino}

\textbf{I} mastini sono impressionanti segugi apprezzati dagli umanoidi per la loro realtà e sensi affinati.

\textit{Media bestia, disallineato}

\textbf{FORZA} +1

\textbf{DESTREZZA} +2

\textbf{COSTITUZIONE} +1

\textbf{INTELLIGENZA} -4

\textbf{SAGGEZZA} +1

\textbf{CARISMA} -2

\textbf{Iniziativa} +2 -- \textbf{Difesa} 13

\textbf{Punti Ferita} 5 (1d8 + 1)

\textbf{Movimento} 12 m

\textbf{Tiri Salvezza}: Tempra +3, Riflessi +3, Volontà +1

\textbf{Competenze} Consapevolezza +3, Sopravvivenza (Seguire Tracce) +3

\textbf{Lingue} -

\textbf{Sfida} 1/8 (25 PX)

\textit{\textbf{Udito e Olfatto Affinato.}} Il mastino ha +1d6 nelle prove di Saggezza (Consapevolezza) basate su udito o olfatto.

\textbf{Azioni}

\textit{\textbf{Morso.} Attacco con Arma da Mischia}: +3 a colpire, portata 1 m, un bersaglio.

\textit{Colpisce:} 4 (1d6 + 1) danni perforanti. Se il bersaglio è una creatura, deve riuscire un Tiro Salvezza di Tempra DC 11 o cadere prono.

\medskip\textbf{Mulo}\index[Mostruario]{Mulo}

\textit{Media bestia, disallineato}

\textbf{FORZA} +2

\textbf{DESTREZZA} +0

\textbf{COSTITUZIONE} +1

\textbf{INTELLIGENZA} -4

\textbf{SAGGEZZA} +0

\textbf{CARISMA} -3

\textbf{Iniziativa} +0 -- \textbf{Difesa} 11

\textbf{Punti Ferita} 11 (2d8 + 2)

\textbf{Movimento} 12 m

\textbf{Tiri Salvezza}: Tempra +3, Riflessi +1, Volontà +1

\textbf{Lingue} -

\textbf{Sfida} 1/8 (25 PX)

\textit{\textbf{Bestia da Soma.}} Il mulo è considerato un animale Grande al fine di determinare la sua capacità di carico.

\textit{\textbf{Piedi Saldi.}} Il mulo ha +1d6 ai Tiri Salvezza su Tempra e Riflessi effettuati contro effetti che lo farebbero cadere prono.

\textbf{Azioni}

\textit{\textbf{Zoccoli.} Attacco con Arma da Mischia}: +2 a colpire, portata 1 m, un bersaglio.

\textit{Colpisce:} 4 (1d4 + 2) danni da botta.

\medskip\textbf{Orso Bruno}\index[Mostruario]{Orso Bruno}

\textit{Grande bestia, disallineato}

\textbf{FORZA} +4

\textbf{DESTREZZA} +0

\textbf{COSTITUZIONE} +3

\textbf{INTELLIGENZA} -4

\textbf{SAGGEZZA} +1

\textbf{CARISMA} -2

\textbf{Iniziativa} +0 -- \textbf{Difesa} 12

\textbf{Punti Ferita} 34 (4d10 + 12)

\textbf{Movimento} 12 m, scalata 9 m

\textbf{Tiri Salvezza}: Tempra +6, Riflessi +2, Volontà +3

\textbf{Competenze} Consapevolezza +3

\textbf{Lingue} -

\textbf{Sfida} 1 (200 PX)

\textit{\textbf{Olfatto Affinato.}} L'orso ha +1d6 alle prove di Saggezza (Consapevolezza) basate sull'olfatto.

\textbf{Azioni}

\textit{\textbf{Multiattacco.}} L'orso effettua due attacchi: uno con il morso e uno con gli artigli.

\textit{\textbf{Artigli.} Attacco con Arma da Mischia}: +5 a colpire, portata 1 m, un bersaglio.

\textit{Colpisce:} 11 (2d6 + 4) danni taglienti.

\textit{\textbf{Morso.} Attacco con Arma da Mischia}: +5 a colpire, portata 1 m, un bersaglio.

\textit{Colpisce:} 8 (1d8 + 4) danni perforanti.

\medskip\textbf{Orso Nero}\index[Mostruario]{Orso Nero}

\textit{Media bestia, disallineato}

\textbf{FORZA} +2

\textbf{DESTREZZA} +0

\textbf{COSTITUZIONE} +2

\textbf{INTELLIGENZA} -4

\textbf{SAGGEZZA} +1

\textbf{CARISMA} -2

\textbf{Iniziativa} +0 -- \textbf{Difesa} 12

\textbf{Punti Ferita} 19 (3d8 + 6)

\textbf{Movimento} 12 m, scalata 9 m

\textbf{Tiri Salvezza}: Tempra +4, Riflessi +1, Volontà +1

\textbf{Competenze} Consapevolezza +3

\textbf{Lingue} -

\textbf{Sfida} 1/2 (100 PX)

\textit{\textbf{Olfatto Affinato.}} L'orso ha +1d6 alle prove di Saggezza (Consapevolezza) basate sull'olfatto.

\textbf{Azioni}

\textit{\textbf{Multiattacco.}} L'orso nero effettua due attacchi: uno con il morso e uno con gli artigli.

\textit{\textbf{Artigli.} Attacco con Arma da Mischia}: +3 a colpire, portata 1 m, un bersaglio.

\textit{Colpisce:} 7 (2d4 + 2) danni taglienti, 1 danno da Sanguinamento.

\textit{\textbf{Morso.} Attacco con Arma da Mischia}: +3 a colpire, portata 1 m, un bersaglio.

\textit{Colpisce:} 5 (1d6 + 2) danni perforanti.

\medskip\textbf{Orso Polare}\index[Mostruario]{Orso Polare}

\textit{Grande bestia, disallineato}

\textbf{FORZA} +5

\textbf{DESTREZZA} +0

\textbf{COSTITUZIONE} +3

\textbf{INTELLIGENZA} -4

\textbf{SAGGEZZA} +1

\textbf{CARISMA} -2

\textbf{Iniziativa} +0 -- \textbf{Difesa} 13

\textbf{Punti Ferita} 42 (5d10 + 15)

\textbf{Movimento} 12 m, nuoto 9 m

\textbf{Tiri Salvezza}: Tempra +10, Riflessi +7, Volontà +4

\textbf{Competenze} Consapevolezza +3

\textbf{Lingue} -

\textbf{Sfida} 2 (450 PX)

\textit{\textbf{Olfatto Affinato.}} L'orso ha +1d6 alle prove di Saggezza (Consapevolezza) basate sull'olfatto.

\textbf{Azioni}

\textit{\textbf{Multiattacco.}} L'orso effettua due attacchi: uno con il morso e uno con gli artigli.

\textit{\textbf{Artigli.} Attacco con Arma da Mischia}: +7 a colpire, portata 1 m, un bersaglio.

\textit{Colpisce:} 12 (2d6 + 5) danni taglienti.

\textit{\textbf{Morso.} Attacco con Arma da Mischia}: +7 a colpire, portata 1 m, un bersaglio.

\textit{Colpisce:} 9 (1d8 + 5) danni perforanti.

\textbf{VARIANTE: ORSO DELLE CAVERNE}\index[Mostruario]{Orso delle Caverne}

Alcuni orsi si sono adattati alla vita sottoterra. Costoro hanno le stesse statistiche degli orsi polari, ma con visione al buio 18 m.

\medskip\textbf{Pantera}\index[Mostruario]{Pantera}

\textit{Media bestia, disallineato}

\textbf{FORZA} +2

\textbf{DESTREZZA} +2

\textbf{COSTITUZIONE} +0

\textbf{INTELLIGENZA} -4

\textbf{SAGGEZZA} +2

\textbf{CARISMA} -2

\textbf{Iniziativa} +2 -- \textbf{Difesa} 13

\textbf{Punti Ferita} 13 (3d8)

\textbf{Movimento} 15 m, scalata 12 m

\textbf{Tiri Salvezza}: Tempra +3, Riflessi +5, Volontà +3

\textbf{Competenze} Muoversi Silenziosamente / Nascondersi +6, Consapevolezza +4

\textbf{Lingue} -

\textbf{Sfida} 1/4 (50 PX)

\textit{\textbf{Balzo.}} Se la pantera si muove di almeno 6 metri diretta verso una creatura e la colpisce con un attacco di artiglio durante lo stesso turno, il bersaglio deve riuscire un Tiro Salvezza di Tempra DC 12 o cadere prono. Se il bersaglio è prono, la pantera può effettuare un attacco di morso contro di esso come azione bonus.

\textit{\textbf{Olfatto Affinato.}} La pantera ha +1d6 alle prove di Saggezza (Consapevolezza) basate sull'olfatto.

\textbf{Azioni}

\textit{\textbf{Artiglio.} Attacco con Arma da Mischia}: +4 a colpire, portata 1 m, un bersaglio.

\textit{Colpisce:} 4 (1d4 + 2) danni taglienti, 1 danno da Sanguinamento.

\textit{\textbf{Morso.} Attacco con Arma da Mischia}: +4 a colpire, portata 1 m, un bersaglio.

\textit{Colpisce:} 5 (1d6 + 2) danni perforanti.


\medskip\textbf{Pony}\index[Mostruario]{Pony}

\textit{Media bestia, disallineato}

\textbf{FORZA} +2

\textbf{DESTREZZA} +0

\textbf{COSTITUZIONE} +1

\textbf{INTELLIGENZA} -4

\textbf{SAGGEZZA} +0

\textbf{CARISMA} -2

\textbf{Iniziativa} +0 -- \textbf{Difesa} 11

\textbf{Punti Ferita} 11 (2d8 + 2)

\textbf{Movimento} 12 m

\textbf{Tiri Salvezza}: Tempra +5, Riflessi +4, Volontà +0

\textbf{Lingue} -

\textbf{Sfida} 1/8 (25 PX)

\textbf{Azioni}

\textit{\textbf{Zoccoli.} Attacco con Arma da Mischia}: +4 a colpire, portata 1 m, un bersaglio.

\textit{Colpisce:} 7 (2d4 + 2) danni da botta.

\medskip\textbf{Ragno}\index[Mostruario]{Ragno}

\textit{Minuscola bestia, disallineato}

\textbf{FORZA} 2 (-5)

\textbf{DESTREZZA} +2

\textbf{COSTITUZIONE} -1

\textbf{INTELLIGENZA} -5

\textbf{SAGGEZZA} +0

\textbf{CARISMA} -4

\textbf{Iniziativa} +2 -- \textbf{Difesa} 13

\textbf{Punti Ferita} 1 (1d4 - 1)

\textbf{Movimento} 6 m, scalata 6 m

\textbf{Tiri Salvezza}: Tempra -4, Riflessi +2, Volontà -4

\textbf{Competenze} Muoversi Silenziosamente / Nascondersi +4

\textbf{Sensi} visione al buio 9 m

\textbf{Lingue} -

\textbf{Sfida} 0 (10 PX)

\textit{\textbf{Camminare sulla Tela.}} Il ragno ignora le restrizioni al movimento provocate dalle ragnatele.

\textit{\textbf{Scalare come Ragno.}} Il ragno può scalare superfici difficili, compreso lo stare a testa in giù sul soffitto, senza bisogno di effettuare una prova di abilità.

\textit{\textbf{Senso della Tela.}} Mentre è in contatto con una ragnatela, il ragno sa l'esatta posizione di qualsiasi altra creatura in contatto con la stessa ragnatela.

\textbf{Azioni}

\textit{\textbf{Morso.} Attacco con Arma da Mischia}: +4 a colpire, portata 1 m, una creatura.

\textit{Colpisce:} 1 danno perforante e il bersaglio deve riuscire un Tiro Salvezza su Tempra 9 o subire 2 (1d4) danni da veleno.

\medskip\textbf{Ragno Fase}\index[Mostruario]{Ragno Fase}

Il ragno fase possiede l'abilità magica di entrare ed uscire dal Piano Etereo. Sembra apparire dal nulla e scompare rapidamente dopo aver attaccato.

\textit{Grande mostruosità, disallineato}

\textbf{FORZA} +2

\textbf{DESTREZZA} +2

\textbf{COSTITUZIONE} +1

\textbf{INTELLIGENZA} -2

\textbf{SAGGEZZA} +0

\textbf{CARISMA} -2

\textbf{Iniziativa} +2 -- \textbf{Difesa} 15

\textbf{Punti Ferita} 32 (5d10 + 5)

\textbf{Movimento} 9 m, scalata 9 m

\textbf{Tiri Salvezza}: Tempra +8, Riflessi +8, Volontà +3

\textbf{Competenze} Muoversi Silenziosamente / Nascondersi +6

\textbf{Sensi} visione al buio 18 m

\textbf{Lingue} -

\textbf{Sfida} 3 (700 PX)

\textit{\textbf{Camminare sulla Tela.}} Il ragno ignora le restrizioni al movimento provocate dalle ragnatele.

\textit{\textbf{Gita Eterea.}} Come azione bonus, il ragno può magicamente spostarsi dal Piano Materiale al Piano Etereo, o viceversa.

\textit{\textbf{Scalare come Ragno.}} Il ragno può scalare superfici difficili, compreso lo stare a testa in giù sul soffitto, senza bisogno di effettuare una prova di abilità.

\textbf{Azioni}

\textit{\textbf{Morso.} Attacco con Arma da Mischia}: +4 a colpire, portata 1 m, una creatura.

\textit{Colpisce:} 7 (1d10 + 2) danni perforanti e il bersaglio deve effettuare un Tiro Salvezza di Tempra DC 11, e subire 18 (4d8) danni da veleno se fallisce il Tiro Salvezza, o la metà di questo danno se lo riesce. Se il danno da veleno riduce il bersaglio a 0 Punti Ferita, il bersaglio è stabile ma avvelenato per 1 ora, anche dopo aver recuperato i Punti Ferita, e mentre è avvelenato in questo modo resta paralizzato.

\medskip\textbf{Ragno Gigante}\index[Mostruario]{Ragno Gigante}

\textit{Grande bestia, disallineato}

\textbf{FORZA} +2

\textbf{DESTREZZA} +3

\textbf{COSTITUZIONE} +1

\textbf{INTELLIGENZA} -4

\textbf{SAGGEZZA} +0

\textbf{CARISMA} -3

\textbf{Iniziativa} +2 -- \textbf{Difesa} 15

\textbf{Punti Ferita} 26 (4d10 + 4)

\textbf{Movimento} 9 m, scalata 9 m

\textbf{Tiri Salvezza}: Tempra +4, Riflessi +4, Volontà +1

\textbf{Competenze} Muoversi Silenziosamente / Nascondersi +7

\textbf{Sensi} vista cieca 3 m, visione al buio 18 m

\textbf{Lingue} -

\textbf{Sfida} 1 (200 PX)

\textit{\textbf{Camminare sulla Tela.}} Il ragno ignora le restrizioni al movimento provocate dalle ragnatele.

\textit{\textbf{Scalare come Ragno.}} Il ragno può scalare superfici difficili, compreso lo stare a testa in giù sul soffitto, senza bisogno di effettuare una prova di abilità.

\textit{\textbf{Senso della Tela.}} Mentre è in contatto con una ragnatela, il ragno sa l'esatta posizione di qualsiasi altra creatura in contatto con la stessa ragnatela.

\textbf{Azioni}

\textit{\textbf{Morso.} Attacco con Arma da Mischia}: +5 a colpire, portata 1 m, una creatura.

\textit{Colpisce:} 7 (1d8 + 3) danni perforanti e il bersaglio deve effettuare un Tiro Salvezza di Tempra DC 11, e subire 9

(2d8) danni da veleno se fallisce il Tiro Salvezza, o la metà di questi danni se lo riesce. Se il danno da veleno riduce il bersaglio a 0 Punti Ferita, il bersaglio è stabile ma avvelenato per 1 ora, anche dopo aver recuperato i Punti Ferita, e mentre è avvelenato in questo modo resta paralizzato.

\textit{\textbf{Ragnatela (Ricarica 5-6).} Attacco con Arma a Gittata}: +5 a colpire, gittata 9m, una creatura.

\textit{Colpisce:} Il bersaglio è intralciato dalla ragnatela. Con un'azione, il bersaglio intralciato può effettuare una prova di Forza DC 12 e, in caso di successo, spezzare la tela. La ragnatela può essere anche attaccata e distrutta (CA 10; PF 5; vulnerabilità al danno da fuoco; immunità ai danni da botta e da veleno).

\medskip\textbf{Ragno Lupo Gigante}\index[Mostruario]{Ragno Lupo Gigante}

Un ragno lupo gigante caccia le prede su terreno aperto o si nasconde in tane o fessure del terreno per tendere imboscate.

\textit{Media bestia, disallineato}

\textbf{FORZA} +1

\textbf{DESTREZZA} +3

\textbf{COSTITUZIONE} +1

\textbf{INTELLIGENZA} -4

\textbf{SAGGEZZA} +1

\textbf{CARISMA} -3

\textbf{Iniziativa} +3 -- \textbf{Difesa} 14

\textbf{Punti Ferita} 11 (2d8 + 2)

\textbf{Movimento} 12 m, scalata 12 m

\textbf{Tiri Salvezza}: Tempra +2, Riflessi +4, Volontà +1

\textbf{Competenze} Muoversi Silenziosamente / Nascondersi +7, Consapevolezza +3

\textbf{Sensi} vista cieca 3 m, visione al buio 18 m

\textbf{Lingue} -

\textbf{Sfida} 1/4 (50 PX)

\textit{\textbf{Camminare sulla Tela.}} Il ragno ignora le restrizioni al movimento provocate dalle ragnatele.

\textit{\textbf{Scalare come Ragno.}} Il ragno può scalare superfici difficili, compreso lo stare a testa in giù sul soffitto, senza bisogno di effettuare una prova di abilità.

\textit{\textbf{Senso della Tela.}} Mentre è in contatto con una ragnatela, il ragno sa l'esatta posizione di qualsiasi altra creatura in contatto con la stessa ragnatela.

\textbf{Azioni}

\textit{\textbf{Morso.} Attacco con Arma da Mischia}: +3 a colpire, portata 1 m, una creatura.

\textit{Colpisce:} 4 (1d6 + 1) danni perforanti e il bersaglio deve effettuare un Tiro Salvezza di Tempra DC 11, e subire 7 (2d6) danni da veleno se fallisce il Tiro Salvezza, o la metà di questi danni se lo riesce. Se il danno da veleno riduce il bersaglio a 0 Punti Ferita, il bersaglio è stabile ma avvelenato per 1 ora, anche dopo aver recuperato i Punti Ferita, e mentre è avvelenato in questo modo resta paralizzato.

\medskip\textbf{Rana}\index[Mostruario]{Rana}

\textit{Minuscola bestia, disallineato}

\textbf{FORZA} -5

\textbf{DESTREZZA} +1

\textbf{COSTITUZIONE} -1

\textbf{INTELLIGENZA} -5

\textbf{SAGGEZZA} -1

\textbf{CARISMA} -4

\textbf{Iniziativa} +1 -- \textbf{Difesa} 12

\textbf{Punti Ferita} 1 (1d4 - 1)

\textbf{Movimento} 6 m, nuoto 6 m

\textbf{Tiri Salvezza}: Tempra -4, Riflessi +1, Volontà -2

\textbf{Competenze} Muoversi Silenziosamente / Nascondersi +3, Consapevolezza +1

\textbf{Sensi} visione al buio 9 m

\textbf{Lingue} -

\textbf{Sfida} 0 (0 PX)

\textit{\textbf{Anfibio.}} La rana può respirare aria e acqua.

\textit{\textbf{Salto da Fermo.}} Una rana può saltare in lungo fino a 3 metri e in alto fino a 1 metro, con o senza la rincorsa.

Una \textbf{rana} è sprovvista di attacchi. Si nutre di piccoli insetti e di solito vive in prossimità di acquitrini, dentro gli alberi o sottoterra.

\medskip\textbf{Rana Gigante}\index[Mostruario]{Rana Gigante}

\textit{Media bestia, disallineato}

\textbf{FORZA} +1

\textbf{DESTREZZA} +1

\textbf{COSTITUZIONE} +0

\textbf{INTELLIGENZA} -4

\textbf{SAGGEZZA} +0

\textbf{CARISMA} -4

\textbf{Iniziativa} +1 -- \textbf{Difesa} 12

\textbf{Punti Ferita} 18 (4d8)

\textbf{Movimento} 9 m, nuoto 9 m

\textbf{Tiri Salvezza}: Tempra +2, Riflessi +2, Volontà +0

\textbf{Competenze} Muoversi Silenziosamente / Nascondersi +3, Consapevolezza +2

\textbf{Sensi} visione al buio 9 m

\textbf{Lingue} -

\textbf{Sfida} 1/4 (50 PX)

\textit{\textbf{Anfibio.}} La rana può respirare aria e acqua.

\textit{\textbf{Salto da Fermo.}} Una rana può saltare in lungo fino a 6 metri e in alto fino a 3 metri, con o senza la rincorsa.

\textbf{Azioni}

\textit{\textbf{Morso.} Attacco con Arma da Mischia}: +3 a colpire, portata 1 m, un bersaglio.

\textit{Colpisce:} 4 (1d6 + 1) danni perforanti e il bersaglio è afferrato (DC 11 per fuggire). Fino al termine dell'afferrare, il bersaglio è intralciato, e la rana non può usare il morso contro un altro bersaglio.

\textit{\textbf{Inghiottire.}} La rana effettua una attacco di morso contro un bersaglio di taglia Piccola o inferiore che sta afferrando. Se l'attacco colpisce, il bersaglio è inghiottito, e l'afferrare ha termine. Il bersaglio inghiottito è accecato e intralciato, ha copertura completa contro gli attacchi e altri effetti all'esterno della rana, e subisce 5 (2d4) danni da acido all'inizio di ciascun turno della rana. La rana può inghiottire solo un bersaglio alla volta. Se la rana muore, una creatura inghiottita non è più intralciata da essa e può uscire dal cadavere utilizzando 1 metro di movimento, uscendo prona.

\medskip\textbf{Ratto}\index[Mostruario]{Ratto}

\textit{Minuscola bestia, disallineato}

\textbf{FORZA} -4

\textbf{DESTREZZA} +0

\textbf{COSTITUZIONE} -1

\textbf{INTELLIGENZA} -4

\textbf{SAGGEZZA} +0

\textbf{CARISMA} -3

\textbf{Iniziativa} +0 -- \textbf{Difesa} 11

\textbf{Punti Ferita} 1 (1d4 - 1)

\textbf{Movimento} 6 m

\textbf{Tiri Salvezza}: Tempra -4, Riflessi +0, Volontà +0

\textbf{Sensi} visione al buio 9 m

\textbf{Lingue} -

\textbf{Sfida} 0 (10 PX)

\textit{\textbf{Olfatto Affinato.}} Il ratto ha +1d6 alle prove di Saggezza (Consapevolezza) basate sull'olfatto.

\textbf{Azioni}

\textit{\textbf{Morso.} Attacco con Arma da Mischia}: +0 a colpire, portata 1 m, un bersaglio.

\textit{Colpisce:} 1 danno perforante.

\medskip\textbf{Ratto Gigante}\index[Mostruario]{Ratto Gigante}

\textit{Piccola bestia, disallineato}

\textbf{FORZA} -2

\textbf{DESTREZZA} +2

\textbf{COSTITUZIONE} +0

\textbf{INTELLIGENZA} -4

\textbf{SAGGEZZA} +0

\textbf{CARISMA} -3

\textbf{Iniziativa} +2 -- \textbf{Difesa} 13

\textbf{Punti Ferita} 7 (2d6)

\textbf{Movimento} 9 m

\textbf{Tiri Salvezza}: Tempra +3, Riflessi +5, Volontà +1

\textbf{Sensi} visione al buio 18 m

\textbf{Lingue} -

\textbf{Sfida} 1/8 (25 PX)

\textit{\textbf{Olfatto Affinato.}} Il ratto ha +1d6 alle prove di Saggezza (Consapevolezza) basate sull'olfatto.

\textit{\textbf{Tattiche di Branco.}} Il ratto ha +1d6 al tiro di attacco contro una creatura se almeno uno degli alleati del ratto si trova entro 1 metro dalla creatura e quell'alleato non è inabile.

\textbf{Azioni}

\textit{\textbf{Morso.} Attacco con Arma da Mischia}: +4 a colpire, portata 1 m, un bersaglio.

\textit{Colpisce:} 4 (1d4 + 2) danni perforanti.

\textbf{VARIANTE: RATTO GIGANTE AMMALATO}\index[Mostruario]{Ratto Gigante ammalato}

Alcuni ratti giganti recano una terribile malattia che diffondono tramite il morso. Un ratto gigante ammalato ha grado di sfida 1/8 (25 PX) e la seguente azione invece del suo normale attacco di morso.

\textit{\textbf{Morso.} Attacco con Arma da Mischia}: +4 a colpire, portata 1 m, un bersaglio.

\textit{Colpisce:} 4 (1d4 + 2) danni perforanti. Se il bersaglio è una creatura, deve riuscire un Tiro Salvezza di Tempra DC 10 o contrarre una malattia. Fino a che la malattia non viene curata, il bersaglio non può recuperare Punti Ferita eccetto tramite metodi magici, e i Punti Ferita massimi del bersaglio diminuiscono di 3 (1d6) ogni 24 ore. Se i Punti Ferita massimi del bersaglio scendono a 0 come risultato della malattia, il bersaglio muore.

\medskip\textbf{Rinoceronte}\index[Mostruario]{Rinoceronte}

\textit{Grande bestia, disallineato}

\textbf{FORZA} +5

\textbf{DESTREZZA} -1

\textbf{COSTITUZIONE} +2

\textbf{INTELLIGENZA} -4

\textbf{SAGGEZZA} +1

\textbf{CARISMA} -2

\textbf{Iniziativa} -1 -- \textbf{Difesa} 12

\textbf{Punti Ferita} 45 (6d10 + 12)

\textbf{Movimento} 12 m

\textbf{Tiri Salvezza}: Tempra +10, Riflessi +4, Volontà +2

\textbf{Lingue} -

\textbf{Sfida} 2 (450 PX)

\textit{\textbf{Carica.}} Se il rinoceronte si muove di almeno 6 metri diretto verso un bersaglio e lo colpisce con un attacco di incornata durante lo stesso turno, il bersaglio subisce 9 (2d8) danni da botta aggiuntivi. Se il bersaglio è una creatura, deve riuscire un Tiro Salvezza su Tempra DC 15 o cadere prono.

\textbf{Azioni}

\textit{\textbf{Incornata.} Attacco con Arma da Mischia}: +7 a colpire, portata 1 m, un bersaglio.

\textit{Colpisce:} 14 (2d8 + 5) danni da botta.

\medskip\textbf{Rospo Gigante}\index[Mostruario]{Rospo Gigante}

\textit{Grande bestia, disallineato}

\textbf{FORZA} +2

\textbf{DESTREZZA} +1

\textbf{COSTITUZIONE} +1

\textbf{INTELLIGENZA} -4

\textbf{SAGGEZZA} +0

\textbf{CARISMA} -4

\textbf{Iniziativa} +1 -- \textbf{Difesa} 12

\textbf{Punti Ferita} 39 (6d10 + 6)

\textbf{Movimento} 6 m, nuoto 12 m

\textbf{Tiri Salvezza}: Tempra +6, Riflessi +6, Volontà +0

\textbf{Sensi} visione al buio 9 m

\textbf{Lingue} -

\textbf{Sfida} 1 (200 PX)

\textit{\textbf{Anfibio.}} Il rospo può respirare aria e acqua.

\textit{\textbf{Salto da Fermo.}} Un rospo può saltare in lungo fino a 6 metri e in alto fino a 3 metri, con o senza la rincorsa.

\textbf{Azioni}

\textit{\textbf{Morso.} Attacco con Arma da Mischia}: +4 a colpire, portata 1 m, un bersaglio.

\textit{Colpisce:} 7 (1d10 + 2) danni perforanti più 5 (1d10) danni da veleno, e il bersaglio è afferrato (DC 13 per fuggire). Fino al termine dell'afferrare, il bersaglio è intralciato, e il rospo non può usare il morso contro un altro bersaglio.

\textit{\textbf{Inghiottire.}} Il rospo effettua una attacco di morso contro un bersaglio di taglia Media o inferiore che sta afferrando. Se l'attacco colpisce, il bersaglio è inghiottito, e l'afferrare ha termine. Il bersaglio inghiottito è accecato e intralciato, ha copertura completa contro gli attacchi e altri effetti all'esterno della rana, e subisce 10 (3d6) danni da acido all'inizio di ciascun turno del rospo. Il rospo può inghiottire solo un bersaglio alla volta.

Se il rospo muore, una creatura inghiottita non è più intralciata da esso e può uscire dal cadavere utilizzando 1 metro di movimento, uscendo prono.

\medskip\textbf{Scarabeo di Fuoco Gigante}\index[Mostruario]{Scarabeo di Fuoco Gigante}

Uno scarabeo di fuoco gigante è una creatura notturna che possiede una coppia di ghiandole luminose capaci di emettere luce per 1d6 giorni dopo la morte dello scarabeo.

\textit{Piccola bestia, disallineato}

\textbf{FORZA} -1

\textbf{DESTREZZA} +0

\textbf{COSTITUZIONE} +1

\textbf{INTELLIGENZA} -5

\textbf{SAGGEZZA} -2

\textbf{CARISMA} -4

\textbf{Iniziativa} +0 -- \textbf{Difesa} 14

\textbf{Punti Ferita} 4 (1d6 + 1)

\textbf{Movimento} 9 m

\textbf{Tiri Salvezza}: Tempra +2, Riflessi +0, Volontà +0

\textbf{Sensi} vista cieca 9 m

\textbf{Lingue} -

\textbf{Sfida} 0 (10 PX)

\textit{\textbf{Illuminazione.}} Lo scarabeo irradia luce intensa in un raggio di 3 metri e luce fioca per ulteriori 3 metri.

\textbf{Azioni}

\textit{\textbf{Morso.} Attacco con Arma da Mischia}: +1 a colpire, portata 1 m, un bersaglio.

\textit{Colpisce:} 2 (1d6 - 1) danni taglienti.

\medskip\textbf{Sciacallo}\index[Mostruario]{Sciacallo}

\textit{Piccola bestia, disallineato}

\textbf{FORZA} -1

\textbf{DESTREZZA} +2

\textbf{COSTITUZIONE} +0

\textbf{INTELLIGENZA} -4

\textbf{SAGGEZZA} +1

\textbf{CARISMA} -2

\textbf{Iniziativa} +2 -- \textbf{Difesa} 13

\textbf{Punti Ferita} 3 (1d6)

\textbf{Movimento} 12 m

\textbf{Tiri Salvezza}: Tempra -1, Riflessi +3, Volontà +1

\textbf{Competenze} Consapevolezza +3

\textbf{Lingue} -

\textbf{Sfida} 0 (10 PX)

\textit{\textbf{Tattiche di Branco.}} Lo sciacallo ha +1d6 ai tiri di attacco contro una creatura se almeno uno degli alleati dello sciacallo si trova entro 1 metro dalla creatura e quell'alleato non è inabile.

\textit{\textbf{Udito e Olfatto Affinato.}} Lo sciacallo ha +1d6 nelle prove di Saggezza (Consapevolezza) basate su udito o olfatto.

\textbf{Azioni}

\textit{\textbf{Morso.} Attacco con Arma da Mischia}: +1 a colpire, portata 1 m, un bersaglio.

\textit{Colpisce:} 1 (1d4 - 1) danni perforanti.

\medskip\textbf{Sciami}\index[Mostruario]{Sciami}

Gli sciami presentati qui di seguito non sono dei normali o benigni raduni di piccole creature. Si formano invece come risultato di un'influenza esterna, spesso maligna. Anche i druidi non sono in grado di affascinare questi sciami, e la loro aggressività è quasi innaturale.

\textbf{Sciame di Centopiedi}\index[Mostruario]{Sciame di Centopiedi}

\textit{Medio sciame di Minuscole bestie, disallineato}

\textbf{FORZA} -4

\textbf{DESTREZZA} +1

\textbf{COSTITUZIONE} +0

\textbf{INTELLIGENZA} -5

\textbf{SAGGEZZA} -2

\textbf{CARISMA} -5

\textbf{Iniziativa} +1 -- \textbf{Difesa} 13

\textbf{Punti Ferita} 22 (5d8)

\textbf{Movimento} 6 m, scalata 6 m

\textbf{Tiri Salvezza}: Tempra -1, Riflessi +3, Volontà +1

\textbf{Resistenze al Danno} da botta, perforante, tagliente

\textbf{Immunità alle Condizioni} affascinato, afferrato, intralciato, paralizzato, pietrificato, prono, spaventato, stordito

\textbf{Sensi} vista cieca 3 m

\textbf{Lingue} -

\textbf{Sfida} 1/2 (100 PX)

\textit{\textbf{Sciame.}} Lo sciame può occupare lo spazio di un'altra creatura e viceversa, e lo sciame può muoversi attraverso qualsiasi apertura grande abbastanza per un Minuscolo insetto. Lo sciame non può recuperare Punti Ferita né ottenere Punti Ferita temporanei.

\textbf{Azioni}

\textit{\textbf{Morsi.} Attacco con Arma da Mischia}: +3 a colpire, portata 0 m, un bersaglio nello spazio dello sciame.

\textit{Colpisce:} 10 (4d4) danni perforanti, o 5 (2d4) danni perforanti se lo sciame è ha metà o meno dei suoi Punti Ferita. Una creatura ridotta a 0 Punti Ferita da uno sciame di centopiedi e stabile resta avvelenata per 1 ora, anche dopo aver recuperato i Punti Ferita, e rimane paralizzata dal veleno durante questo periodo.

\medskip\textbf{Sciame di Corvi}\index[Mostruario]{Sciame di Corvi}

\textit{Medio sciame di Minuscole bestie, disallineato}

\textbf{FORZA} -2

\textbf{DESTREZZA} +2

\textbf{COSTITUZIONE} -1

\textbf{INTELLIGENZA} -4

\textbf{SAGGEZZA} +1

\textbf{CARISMA} -2

\textbf{Iniziativa} +2 -- \textbf{Difesa} 13

\textbf{Punti Ferita} 24 (7d8 -- 7)

\textbf{Movimento} 3 m, volo 15 m

\textbf{Tiri Salvezza}: Tempra -1, Riflessi +3, Volontà +2

\textbf{Competenze} Consapevolezza +5

\textbf{Resistenze al Danno} da botta, perforante, tagliente

\textbf{Immunità alle Condizioni} affascinato, afferrato, intralciato, paralizzato, pietrificato, prono, spaventato, stordito

\textbf{Lingue} -

\textbf{Sfida} 1/4 (50 PX)

\textit{\textbf{Sciame.}} Lo sciame può occupare lo spazio di un'altra creatura e viceversa, e lo sciame può muoversi attraverso qualsiasi apertura grande abbastanza per un Minuscolo corvo. Lo sciame non può recuperare Punti Ferita né ottenere Punti Ferita temporanei.

\textbf{Azioni}

\textit{\textbf{Becchi.} Attacco con Arma da Mischia}: +4 a colpire, portata 1 m, un bersaglio nello spazio dello sciame.

\textit{Colpisce:} 7 (2d6) danni perforanti, o 3 (1d6) danni perforanti se lo sciame è ha metà o meno dei suoi Punti Ferita.

\medskip\textbf{Sciame di Pirana}\index[Mostruario]{Sciame di Pirana}

\textit{Medio sciame di Minuscole bestie, disallineato}

\textbf{FORZA} +1

\textbf{DESTREZZA} +3

\textbf{COSTITUZIONE} -1

\textbf{INTELLIGENZA} -5

\textbf{SAGGEZZA} -2

\textbf{CARISMA} -4

\textbf{Iniziativa} +3 -- \textbf{Difesa} 14

\textbf{Punti Ferita} 28 (8d8 -- 8)

\textbf{Movimento} 0 m, nuoto 12 m

\textbf{Tiri Salvezza}: Tempra -3, Riflessi +4, Volontà -1

\textbf{Resistenze al Danno} da botta, perforante, tagliente

\textbf{Immunità alle Condizioni} affascinato, afferrato, intralciato, paralizzato, pietrificato, prono, spaventato, stordito

\textbf{Sensi} visione al buio 18 m

\textbf{Lingue} -

\textbf{Sfida} 1 (200 PX)

\textit{\textbf{Frenesia Sanguinaria.}} Lo sciame ha +1d6 ai tiri di attacco in mischia contro qualsiasi creatura che non sia al massimo dei Punti Ferita.

\textit{\textbf{Respirare Acqua.}} Lo sciame può respirare solo sott'acqua.

\textit{\textbf{Sciame.}} Lo sciame può occupare lo spazio di un'altra creatura e viceversa, e lo sciame può muoversi attraverso qualsiasi apertura grande abbastanza per un Minuscolo pirana. Lo sciame non può recuperare Punti Ferita né ottenere Punti Ferita temporanei.

\textbf{Azioni}

\textit{\textbf{Morsi.} Attacco con Arma da Mischia}: +5 a colpire, portata 0 m, una creatura nello spazio dello sciame.

\textit{Colpisce:} 14 (4d6) danni perforanti, o 7 (2d6) danni perforanti se lo sciame è ha metà o meno dei suoi Punti Ferita.

\medskip\textbf{Sciame di Insetti}\index[Mostruario]{Sciame di Insetti}

\textit{Medio sciame di Minuscole bestie, disallineato}

\textbf{FORZA} -4

\textbf{DESTREZZA} +1

\textbf{COSTITUZIONE} +0

\textbf{INTELLIGENZA} -5

\textbf{SAGGEZZA} -2

\textbf{CARISMA} -5

\textbf{Iniziativa} +1 -- \textbf{Difesa} 13

\textbf{Punti Ferita} 22 (5d8)

\textbf{Movimento} 6 m, scalata 6 m

\textbf{Tiri Salvezza}: Tempra -3, Riflessi +2, Volontà -1

\textbf{Resistenze al Danno} da botta, perforante, tagliente

\textbf{Immunità alle Condizioni} affascinato, afferrato, intralciato, paralizzato, pietrificato, prono, spaventato, stordito

\textbf{Sensi} vista cieca 3 m

\textbf{Lingue} -

\textbf{Sfida} 1/2 (100 PX)

\textit{\textbf{Sciame.}} Lo sciame può occupare lo spazio di un'altra creatura e viceversa, e lo sciame può muoversi attraverso qualsiasi apertura grande abbastanza per un Minuscolo insetto. Lo sciame non può recuperare Punti Ferita né ottenere Punti Ferita temporanei.

\textbf{Azioni}

\textit{\textbf{Morsi.} Attacco con Arma da Mischia}: +3 a colpire, portata 0 m, un bersaglio nello spazio dello sciame.

\textit{Colpisce:} 10 (4d4) danni perforanti, o 5 (2d4) danni perforanti se lo sciame è ha metà o meno dei suoi Punti Ferita.

\medskip\textbf{Sciame di Pipistrelli}\index[Mostruario]{Sciame di Pipistrelli}

\textit{Medio sciame di Minuscole bestie, disallineato}

\textbf{FORZA} -3

\textbf{DESTREZZA} +2

\textbf{COSTITUZIONE} +0

\textbf{INTELLIGENZA} -4

\textbf{SAGGEZZA} +1

\textbf{CARISMA} -3

\textbf{Iniziativa} +2 -- \textbf{Difesa} 13

\textbf{Punti Ferita} 22 (5d8)

\textbf{Movimento} 0 m, volo 9 m

\textbf{Tiri Salvezza}: Tempra -2, Riflessi +4, Volontà +2

\textbf{Resistenze al Danno} da botta, perforante, tagliente

\textbf{Immunità alle Condizioni} affascinato, afferrato, intralciato, paralizzato, pietrificato, prono, spaventato, stordito

\textbf{Sensi} vista cieca 18 m

\textbf{Lingue} -

\textbf{Sfida} 1/4 (50 PX)

\textit{\textbf{Ecolocazione.}} Lo sciame non può usare la vista cieca se assordato.

\textit{\textbf{Sciame.}} Lo sciame può occupare lo spazio di un'altra creatura e viceversa, e lo sciame può muoversi attraverso qualsiasi apertura grande abbastanza per un Minuscolo pipistrello. Lo sciame non può recuperare Punti Ferita né ottenere Punti Ferita temporanei.

\textit{\textbf{Udito Affinato.}} Lo sciame ha +1d6 alle prove di Saggezza (Consapevolezza) basate sull'udito.

\textbf{Azioni}

\textit{\textbf{Morsi.} Attacco con Arma da Mischia}: +4 a colpire, portata 0 m, una creatura nello spazio dello sciame.

\textit{Colpisce:} 5 (2d4) danni perforanti, o 2 (1d4) danni perforanti se lo sciame è ha metà o meno dei suoi Punti Ferita.

\medskip\textbf{Sciame di Ragni}\index[Mostruario]{Sciame di Ragni}

\textit{Medio sciame di Minuscole bestie, disallineato}

\textbf{FORZA} -4

\textbf{DESTREZZA} +1

\textbf{COSTITUZIONE} +0

\textbf{INTELLIGENZA} -5

\textbf{SAGGEZZA} -2

\textbf{CARISMA} -5

\textbf{Iniziativa} +1 -- \textbf{Difesa} 13

\textbf{Punti Ferita} 22 (5d8)

\textbf{Movimento} 6 m, scalata 6 m

\textbf{Tiri Salvezza}: Tempra -3, Riflessi +2, Volontà -1

\textbf{Resistenze al Danno} da botta, perforante, tagliente

\textbf{Immunità alle Condizioni} affascinato, afferrato, intralciato, paralizzato, pietrificato, prono, spaventato, stordito

\textbf{Sensi} vista cieca 3 m

\textbf{Lingue} -

\textbf{Sfida} 1/2 (100 PX)

\textit{\textbf{Camminare sulla Tela.}} Lo sciame ignora le restrizioni al movimento provocate dalle ragnatele.

\textit{\textbf{Scalare come Ragno.}} Lo sciame può scalare superfici difficili, compreso lo stare a testa in giù sul soffitto, senza bisogno di effettuare una prova di abilità.

\textit{\textbf{Senso della Tela.}} Mentre è in contatto con una ragnatela, lo sciame sa l'esatta posizione di qualsiasi altra creatura in contatto con la stessa ragnatela.

\textit{\textbf{Sciame.}} Lo sciame può occupare lo spazio di un'altra creatura e viceversa, e lo sciame può muoversi attraverso qualsiasi apertura grande abbastanza per un Minuscolo insetto. Lo sciame non può recuperare Punti Ferita né ottenere Punti Ferita temporanei.

\textbf{Azioni}

\textit{\textbf{Morsi.} Attacco con Arma da Mischia}: +3 a colpire, portata 0 m, un bersaglio nello spazio dello sciame.

\textit{Colpisce:} 10 (4d4) danni perforanti, o 5 (2d4) danni perforanti se lo sciame è ha metà o meno dei suoi Punti Ferita.

\medskip\textbf{Sciame di Ratti}\index[Mostruario]{Sciame di Ratti}

\textit{Medio sciame di Minuscole bestie, disallineato}

\textbf{FORZA} -1

\textbf{DESTREZZA} +0

\textbf{COSTITUZIONE} -1

\textbf{INTELLIGENZA} -4

\textbf{SAGGEZZA} +0

\textbf{CARISMA} -4

\textbf{Iniziativa} +0 -- \textbf{Difesa} 11

\textbf{Punti Ferita} 24 (7d8 - 7)

\textbf{Movimento} 9 m

\textbf{Tiri Salvezza}: Tempra +0, Riflessi +1, Volontà +1

\textbf{Resistenze al Danno} da botta, perforante, tagliente

\textbf{Immunità alle Condizioni} affascinato, afferrato, intralciato, paralizzato, pietrificato, prono, spaventato, stordito

\textbf{Sensi} visione al buio 9 m

\textbf{Lingue} -

\textbf{Sfida} 1/4 (50 PX)

\textit{\textbf{Olfatto Affinato.}} Lo sciame ha +1d6 alle prove di Saggezza (Consapevolezza) basate sull'olfatto.

\textit{\textbf{Sciame.}} Lo sciame può occupare lo spazio di un'altra creatura e viceversa, e lo sciame può muoversi attraverso qualsiasi apertura grande abbastanza per un Minuscolo ratto. Lo sciame non può recuperare Punti Ferita né ottenere Punti Ferita temporanei.

\textbf{Azioni}

\textit{\textbf{Morsi.} Attacco con Arma da Mischia}: +2 a colpire, portata 0 m, un bersaglio nello spazio dello sciame.

\textit{Colpisce:} 7 (2d6) danni perforanti, o 3 (1d6) danni perforanti se lo sciame è ha metà o meno dei suoi Punti Ferita.

\medskip\textbf{Sciame di Scarabei}\index{Sciame di Scarabei}

\textit{Medio sciame di Minuscole bestie, disallineato}

\textbf{FORZA} -4

\textbf{DESTREZZA} +1

\textbf{COSTITUZIONE} +0

\textbf{INTELLIGENZA} -5

\textbf{SAGGEZZA} -2

\textbf{CARISMA} -5

\textbf{Iniziativa} +1 -- \textbf{Difesa} 13

\textbf{Punti Ferita} 22 (5d8)

\textbf{Movimento} 6 m, scalata 6 m, scavo 6 m

\textbf{Tiri Salvezza}: Tempra -3, Riflessi +2, Volontà -1

\textbf{Resistenze al Danno} da botta, perforante, tagliente

\textbf{Immunità alle Condizioni} affascinato, afferrato, intralciato, paralizzato, pietrificato, prono, spaventato, stordito

\textbf{Sensi} vista cieca 3 m

\textbf{Lingue} -

\textbf{Sfida} 1/2 (100 PX)

\textit{\textbf{Sciame.}} Lo sciame può occupare lo spazio di un'altra creatura e viceversa, e lo sciame può muoversi attraverso qualsiasi apertura grande abbastanza per un Minuscolo insetto. Lo sciame non può recuperare Punti Ferita né ottenere Punti Ferita temporanei.

\textbf{Azioni}

\textit{\textbf{Morsi.} Attacco con Arma da Mischia}: +3 a colpire, portata 0 m, un bersaglio nello spazio dello sciame.

\textit{Colpisce:} 10 (4d4) danni perforanti, o 5 (2d4) danni perforanti se lo sciame è ha metà o meno dei suoi Punti Ferita.

\medskip\textbf{Sciame di Serpenti Velenosi}\index{Sciame di Serpenti Velenosi}

\textit{Medio sciame di Minuscole bestie, disallineato}

\textbf{FORZA} -1

\textbf{DESTREZZA} +4

\textbf{COSTITUZIONE} +0

\textbf{INTELLIGENZA} -5

\textbf{SAGGEZZA} +0

\textbf{CARISMA} -4

\textbf{Iniziativa} +4 -- \textbf{Difesa} 15

\textbf{Punti Ferita} 36 (8d8)

\textbf{Movimento} 9 m, nuoto 9 m

\textbf{Tiri Salvezza}: Tempra +0, Riflessi +5, Volontà +1

\textbf{Resistenze al Danno} da botta, perforante, tagliente

\textbf{Immunità alle Condizioni} affascinato, afferrato, intralciato, paralizzato, pietrificato, prono, spaventato, stordito

\textbf{Sensi} vista cieca 3 m

\textbf{Lingue} -

\textbf{Sfida} 2 (450 PX)

\textit{\textbf{Sciame.}} Lo sciame può occupare lo spazio di un'altra creatura e viceversa, e lo sciame può muoversi attraverso qualsiasi apertura grande abbastanza per un Minuscolo serpente. Lo sciame non può recuperare Punti Ferita né ottenere Punti Ferita temporanei.

\textbf{Azioni}

\textit{\textbf{Morsi.} Attacco con Arma da Mischia}: +6 a colpire, portata 0 m, una creatura nello spazio dello sciame.

\textit{Colpisce:} 7 (2d6) danni perforanti, o 3 (1d6) danni perforanti se lo sciame è ha metà o meno dei suoi Punti Ferita, e il bersaglio deve effettuare un Tiro Salvezza di Tempra DC 10, e subire 14 (4d6) danni da veleno se fallisce il Tiro Salvezza, o la metà di questi danni se lo riesce.

\medskip\textbf{Sciame di Vespe}\index{Sciame di Serpenti Velenosi}

\textit{Medio sciame di Minuscole bestie, disallineato}

\textbf{FORZA} -4

\textbf{DESTREZZA} +1

\textbf{COSTITUZIONE} +0

\textbf{INTELLIGENZA} -5

\textbf{SAGGEZZA} -2

\textbf{CARISMA} -5

\textbf{Iniziativa} +1 -- \textbf{Difesa} 13

\textbf{Punti Ferita} 22 (5d8)

\textbf{Movimento} 1 m, volo 9 m

\textbf{Tiri Salvezza}: Tempra -3, Riflessi +2, Volontà -1

\textbf{Resistenze al Danno} da botta, perforante, tagliente

\textbf{Immunità alle Condizioni} affascinato, afferrato, intralciato, paralizzato, pietrificato, prono, spaventato, stordito

\textbf{Sensi} vista cieca 3 m

\textbf{Lingue} -

\textbf{Sfida} 1/2 (100 PX)

\textit{\textbf{Sciame.}} Lo sciame può occupare lo spazio di un'altra creatura e viceversa, e lo sciame può muoversi attraverso qualsiasi apertura grande abbastanza per un Minuscolo insetto. Lo sciame non può recuperare Punti Ferita né ottenere Punti Ferita temporanei.

\textbf{Azioni}

\textit{\textbf{Morsi.} Attacco con Arma da Mischia}: +3 a colpire, portata 0 m, un bersaglio nello spazio dello sciame.

\textit{Colpisce:} 10 (4d4) danni perforanti, o 5 (2d4) danni perforanti se lo sciame è ha metà o meno dei suoi Punti Ferita.

\medskip\textbf{Scimmione}\index[Mostruario]{Scimmione}

\textit{Media bestia, disallineato}

\textbf{FORZA} +3

\textbf{DESTREZZA} +2

\textbf{COSTITUZIONE} +2

\textbf{INTELLIGENZA} -2

\textbf{SAGGEZZA} +1

\textbf{CARISMA} -2

\textbf{Iniziativa} +2 -- \textbf{Difesa} 13

\textbf{Punti Ferita} 19 (3d8 + 6)

\textbf{Movimento} 9 m, scalata 9 m

\textbf{Tiri Salvezza}: Tempra +3, Riflessi +3, Volontà +2

\textbf{Competenze} Acrobatica +5, Consapevolezza +3

\textbf{Lingue} -

\textbf{Sfida} 1/2 (100 PX)

\textbf{Azioni}

\textit{\textbf{Multiattacco.}} Lo scimmione effettua due attacchi di pugno.

\textit{\textbf{Pugno.} Attacco con Arma da Mischia}: +5 a colpire, portata 1 m, un bersaglio.

\textit{Colpisce:} 6 (1d6 + 3) danni da botta.

\textit{\textbf{Sasso.} Attacco con Arma a Gittata}: +5 a colpire, gittata 8m, un bersaglio.

\textit{Colpisce:} 6 (1d6 + 3) danni da botta.

\medskip\textbf{Scimmione Gigante}\index[Mostruario]{Scimmione Gigante}

\textit{Enorme bestia, disallineato}

\textbf{FORZA} +6

\textbf{DESTREZZA} +2

\textbf{COSTITUZIONE} +4

\textbf{INTELLIGENZA} -2

\textbf{SAGGEZZA} +1

\textbf{CARISMA} -2

\textbf{Iniziativa} +2 -- \textbf{Difesa} 16

\textbf{Punti Ferita} 157 (15d12 + 60)

\textbf{Movimento} 12 m, scalata 12 m

\textbf{Tiri Salvezza}: Tempra +7, Riflessi +6, Volontà +4

\textbf{Competenze} Acrobatica +9, Consapevolezza +4

\textbf{Lingue} -

\textbf{Sfida} 7 (2.900 PX)

\textbf{Azioni}

\textit{\textbf{Multiattacco.}} Lo scimmione effettua due attacchi di pugno.

\textit{\textbf{Pugno.} Attacco con Arma da Mischia}: +9 a colpire, portata 3 m, un bersaglio.

\textit{Colpisce:} 22 (3d10 + 6) danni da botta.

\textit{\textbf{Sasso.} Attacco con Arma a Gittata}: +9 a colpire, gittata 15m, un bersaglio.

\textit{Colpisce:} 30 (7d6 + 6) danni da botta.

\medskip\textbf{Scorpione}\index[Mostruario]{Scorpione}

\textit{Minuscola bestia, disallineato}

\textbf{FORZA} -4

\textbf{DESTREZZA} +0

\textbf{COSTITUZIONE} -1

\textbf{INTELLIGENZA} -5

\textbf{SAGGEZZA} -1

\textbf{CARISMA} -4

\textbf{Iniziativa} +0 -- \textbf{Difesa} 12

\textbf{Punti Ferita} 1 (1d4 - 1)

\textbf{Movimento} 3 m

\textbf{Tiri Salvezza}: Tempra -3, Riflessi +2, Volontà -1

\textbf{Sensi} vista cieca 3 m

\textbf{Lingue} -

\textbf{Sfida} 0 (10 PX)

\textbf{Azioni}

\textit{\textbf{Pungiglione.} Attacco con Arma da Mischia}: +2 a colpire, portata 1 m, una creatura.

\textit{Colpisce:} 1 danno perforante e il bersaglio deve effettuare un Tiro Salvezza di Tempra DC 9, e subire 4 (1d8) danni da veleno se fallisce il Tiro Salvezza, o la metà di questi danni se lo riesce.

\medskip\textbf{Scorpione Gigante}\index[Mostruario]{Scorpione Gigante}

\textit{Grande bestia, disallineato}

\textbf{FORZA} +2

\textbf{DESTREZZA} +1

\textbf{COSTITUZIONE} +2

\textbf{INTELLIGENZA} -5

\textbf{SAGGEZZA} -1

\textbf{CARISMA} -4

\textbf{Iniziativa} +1 -- \textbf{Difesa} 17

\textbf{Punti Ferita} 52 (7d10 + 14)

\textbf{Movimento} 12 m

\textbf{Tiri Salvezza}: Tempra +7, Riflessi +1, Volontà +1

\textbf{Sensi} vista cieca 18 m

\textbf{Lingue} -

\textbf{Sfida} 3 (700 PX)

\textbf{Azioni}

\textit{\textbf{Multiattacco.}} Lo scorpione effettua tre attacchi: due con gli artigli e uno con il pungiglione.

\textit{\textbf{Artiglio.} Attacco con Arma da Mischia}: +4 a colpire, portata 1 m, un bersaglio.

\textit{Colpisce:} 6 (1d8 + 2) danni da botta e il bersaglio è afferrato (DC 12 per fuggire). Lo scorpione ha due artigli, ciascuno dei quali può afferrare solo un bersaglio.

\textit{\textbf{Pungiglione.} Attacco con Arma da Mischia}: +4 a colpire, portata 1 m, una creatura.

\textit{Colpisce:} 7 (1d10 + 2) danni perforanti e il bersaglio deve effettuare un Tiro Salvezza di Tempra DC 12, e subire 22 (4d10) danni da veleno se fallisce il Tiro Salvezza, o la metà di questi danni se lo riesce.

\medskip\textbf{Serpente Costrittore}\index[Mostruario]{Serpente Costrittore}

\textit{Grande bestia, disallineato}

\textbf{FORZA} +2

\textbf{DESTREZZA} +2

\textbf{COSTITUZIONE} +1

\textbf{INTELLIGENZA} -5

\textbf{SAGGEZZA} +0

\textbf{CARISMA} -4

\textbf{Iniziativa} +2 -- \textbf{Difesa} 13

\textbf{Punti Ferita} 13 (2d10 + 2)

\textbf{Movimento} 9 m, nuoto 9 m

\textbf{Tiri Salvezza}: Tempra +3, Riflessi +2, Volontà +0

\textbf{Sensi} vista cieca 3 m

\textbf{Lingue} -

\textbf{Sfida} 1/4 (50 PX)

\textbf{Azioni}

\textit{\textbf{Morso.} Attacco con Arma da Mischia}: +4 a colpire, portata 1 m, una creatura.

\textit{Colpisce:} 5 (1d6 + 2) danni perforanti.

\textit{\textbf{Stritolare.} Attacco con Arma da Mischia}: +4 a colpire, portata 1 m, una creatura.

\textit{Colpisce:} 6 (1d8 + 2) danni da botta, e il bersaglio è afferrato (DC 14 per fuggire). Fino al termine dell'afferrare, la creatura è intralciata, e il serpente non può stritolare un altro bersaglio.

\medskip\textbf{Serpente Costrittore Gigante}\index[Mostruario]{Serpente Costrittore Gigante}

\textit{Enorme bestia, disallineato}

\textbf{FORZA} +4

\textbf{DESTREZZA} +2

\textbf{COSTITUZIONE} +1

\textbf{INTELLIGENZA} -5

\textbf{SAGGEZZA} +0

\textbf{CARISMA} -4

\textbf{Iniziativa} +2 -- \textbf{Difesa} 13

\textbf{Punti Ferita} 60 (8d12 + 8)

\textbf{Movimento} 9 m, nuoto 9 m

\textbf{Tiri Salvezza}: Tempra +3, Riflessi +2, Volontà +0

\textbf{Competenze} Consapevolezza +2

\textbf{Sensi} vista cieca 3 m

\textbf{Lingue} -

\textbf{Sfida} 2 (450 PX)

\textbf{Azioni}

\textit{\textbf{Morso.} Attacco con Arma da Mischia}: +6 a colpire, portata 3 m, una creatura.

\textit{Colpisce:} 11 (2d6 + 4) danni perforanti.

\textit{\textbf{Stritolare.} Attacco con Arma da Mischia}: +6 a colpire, portata 1 m, una creatura.

\textit{Colpisce:} 13 (2d8 + 4) danni da botta, e il bersaglio è afferrato (DC 16 per fuggire). Fino al termine dell'afferrare, la creatura è intralciata, e il serpente non può stritolare un altro bersaglio.

\medskip\textbf{Serpente Velenoso}\index[Mostruario]{Serpente Velenoso}

\textit{Minuscola bestia, disallineato}

\textbf{FORZA} -4

\textbf{DESTREZZA} +3

\textbf{COSTITUZIONE} +0

\textbf{INTELLIGENZA} -5

\textbf{SAGGEZZA} +0

\textbf{CARISMA} -4

\textbf{Iniziativa} +3 -- \textbf{Difesa} 14

\textbf{Punti Ferita} 2 (1d4)

\textbf{Movimento} 9 m, nuoto 9 m

\textbf{Tiri Salvezza}: Tempra +1, Riflessi +4, Volontà +1

\textbf{Sensi} vista cieca 3 m

\textbf{Lingue} -

\textbf{Sfida} 1/8 (25 PX)

\textbf{Azioni}

\textit{\textbf{Morso.} Attacco con Arma da Mischia}: +5 a colpire, portata 1 m, un bersaglio.

\textit{Colpisce:} 1 danno perforante e il bersaglio deve effettuare un Tiro Salvezza di Tempra DC 10, e subire 5 (2d4) danni da veleno se fallisce il Tiro Salvezza, o la metà di questi danni se lo riesce.

\medskip\textbf{Serpente Velenoso Gigante}\index[Mostruario]{Serpente Velenoso Gigante}

\textit{Media bestia, disallineato}

\textbf{FORZA} +0

\textbf{DESTREZZA} +4

\textbf{COSTITUZIONE} +1

\textbf{INTELLIGENZA} -4

\textbf{SAGGEZZA} +0

\textbf{CARISMA} -4

\textbf{Iniziativa} +4 -- \textbf{Difesa} 15

\textbf{Punti Ferita} 11 (2d8 + 2)

\textbf{Movimento} 9 m, nuoto 9 m

\textbf{Tiri Salvezza}: Tempra +1, Riflessi +5, Volontà +2

\textbf{Competenze} Consapevolezza +2

\textbf{Sensi} vista cieca 3 m

\textbf{Lingue} -

\textbf{Sfida} 1/4 (50 PX)

\textbf{Azioni}

\textit{\textbf{Morso.} Attacco con Arma da Mischia}: +6 a colpire, portata 3 m, un bersaglio.

\textit{Colpisce:} 6 (1d4 + 4) danni perforanti e il bersaglio deve effettuare un Tiro Salvezza di Tempra DC 11, e subire 10 (3d6) danni da veleno se fallisce il Tiro Salvezza, o la metà di questi danni se lo riesce.

\medskip\textbf{Serpente Volante}\index[Mostruario]{Serpente Volante}

Un serpente volante è una serpe alata, dai colori intensi, rinvenuta in giungle remote.

\textit{Minuscola bestia, disallineato}

\textbf{FORZA} -3

\textbf{DESTREZZA} +4

\textbf{COSTITUZIONE} +0

\textbf{INTELLIGENZA} -4

\textbf{SAGGEZZA} +1

\textbf{CARISMA} -3

\textbf{Iniziativa} +4 -- \textbf{Difesa} 15

\textbf{Punti Ferita} 5 (2d4)

\textbf{Movimento} 9 m, nuoto 9 m, volo 18 m

\textbf{Tiri Salvezza}: Tempra -2, Riflessi +5, Volontà +1

\textbf{Sensi} vista cieca 3 m

\textbf{Lingue} -

\textbf{Sfida} 1/8 (25 PX)

\textit{\textbf{Sorvolare.}} Il serpente non provoca attacchi di opportunità quando vola via dalla portata di un nemico.

\textbf{Azioni}

\textit{\textbf{Morso.} Attacco con Arma da Mischia}: +6 a colpire, portata 1 m, un bersaglio.

\textit{Colpisce:} 1 danno perforante più 7 (3d4) danni da veleno.

\medskip\textbf{Squalo Cacciatore}\index[Mostruario]{Squalo Cacciatore}

Uno squalo cacciatore è lungo da 4 a 6 metri e di solito caccia in solitario nelle acque più profonde.

\textit{Grande bestia, disallineato}

\textbf{FORZA} +4

\textbf{DESTREZZA} +1

\textbf{COSTITUZIONE} +2

\textbf{INTELLIGENZA} -5

\textbf{SAGGEZZA} +0

\textbf{CARISMA} -3

\textbf{Iniziativa} +1 -- \textbf{Difesa} 13

\textbf{Punti Ferita} 45 (6d10 + 12)

\textbf{Movimento} 0 m, nuoto 12 m

\textbf{Tiri Salvezza}: Tempra +4, Riflessi +2, Volontà +0

\textbf{Competenze} Consapevolezza +2

\textbf{Sensi} vista cieca 9 m

\textbf{Lingue} -

\textbf{Sfida} 2 (450 PX)

\textit{\textbf{Frenesia Sanguinaria.}} Lo squalo ha +1d6 ai tiri di attacco in mischia contro qualsiasi creatura che non sia al massimo dei Punti Ferita.

\textit{\textbf{Respirare Acqua.}} Lo squalo può respirare solo sott'acqua.

\textbf{Azioni}

\textit{\textbf{Morso.} Attacco con Arma da Mischia}: +6 a colpire, portata 1 m, un bersaglio.

\textit{Colpisce:} 13 (2d8 + 4) danni perforanti.

\medskip\textbf{Squalo Corallino}\index[Mostruario]{Squalo Corallino}

Gli squali corallini sono lunghi da 2 a 3 metri e vivono nelle acque meno profonde e lungo le barriere coralline.

\textit{Media bestia, disallineato}

\textbf{FORZA} +2

\textbf{DESTREZZA} +1

\textbf{COSTITUZIONE} +1

\textbf{INTELLIGENZA} -5

\textbf{SAGGEZZA} +0

\textbf{CARISMA} -3

\textbf{Iniziativa} +1 -- \textbf{Difesa} 13

\textbf{Punti Ferita} 22 (4d8 + 4)

\textbf{Movimento} 0 m, nuoto 12 m

\textbf{Tiri Salvezza}: Tempra +2, Riflessi +2, Volontà +1

\textbf{Competenze} Consapevolezza +2

\textbf{Sensi} vista cieca 9 m

\textbf{Lingue} -

\textbf{Sfida} 1/2 (100 PX)

\textit{\textbf{Respirare Acqua.}} Lo squalo può respirare solo sott'acqua.

\textit{\textbf{Tattiche di Branco.}} Lo squalo ha +1d6 al tiro di attacco contro una creatura se almeno uno degli alleati dello squalo si trova entro 1 metro dalla creatura e quell'alleato non è inabile.

\textbf{Azioni}

\textit{\textbf{Morso.} Attacco con Arma da Mischia}: +4 a colpire, portata 1 m, un bersaglio.

\textit{Colpisce:} 6 (1d8 + 2) danni perforanti.

\medskip\textbf{Squalo Gigante}\index[Mostruario]{Squalo Gigante}

Lo squalo gigante è lungo 9 metri e lo si incontra

normalmente solo negli oceani più profondi.

\textit{Enorme bestia, disallineato}

\textbf{FORZA} +6

\textbf{DESTREZZA} +0

\textbf{COSTITUZIONE} +5

\textbf{INTELLIGENZA} -5

\textbf{SAGGEZZA} +0

\textbf{CARISMA} -3

\textbf{Iniziativa} +0 -- \textbf{Difesa} 16

\textbf{Punti Ferita} 126 (11d12 + 55)

\textbf{Movimento} 0 m, nuoto 15 m

\textbf{Tiri Salvezza}: Tempra +7, Riflessi +2, Volontà +1

\textbf{Competenze} Consapevolezza +3

\textbf{Sensi} vista cieca 18 m

\textbf{Lingue} -

\textbf{Sfida} 5 (1.800 PX)

\textit{\textbf{Frenesia Sanguinaria.}} Lo squalo ha +1d6 ai tiri di attacco

in mischia contro qualsiasi creatura che non sia al massimo dei Punti Ferita.

\textit{\textbf{Respirare Acqua.}} Lo squalo può respirare solo sott'acqua.

\textbf{Azioni}

\textit{\textbf{Morso.} Attacco con Arma da Mischia}: +9 a colpire, portata 1 m, un bersaglio.

\textit{Colpisce:} 22 (3d10 + 6) danni perforanti.

\medskip\textbf{Strige}\index[Mostruario]{Strige}

Questo orrendo mostro sembra un incrocio tra un grosso pipistrello e una zanzara sovradimensionata. Le sue zampe terminano in lunghe pinze, e la sua lunga proboscide, simile ad un ago, fende l'aria mentre cerca di nutrirsi del sangue delle creature viventi.

\textit{Minuscola bestia, disallineato}

\textbf{FORZA} -3

\textbf{DESTREZZA} +3

\textbf{COSTITUZIONE} +0

\textbf{INTELLIGENZA} -4

\textbf{SAGGEZZA} -1

\textbf{CARISMA} -2

\textbf{Iniziativa} +3 -- \textbf{Difesa} 15

\textbf{Punti Ferita} 2 (1d4)

\textbf{Movimento} 3 m, volo 12 m

\textbf{Tiri Salvezza}: Tempra -3, Riflessi +4, Volontà -1

\textbf{Sensi} visione al buio 18 m

\textbf{Lingue} -

\textbf{Sfida} 1/8 (25 PX)

\textbf{Azioni}

\textit{\textbf{Risucchio di Sangue.} Attacco con Arma da Mischia}: +5 a colpire, portata 1 m, una creatura.

\textit{Colpisce:} 5 (1d4 + 3) danni perforanti e lo strige si attacca al bersaglio. Mentre è attaccato, lo strige non attacca. Invece, all'inizio di ciascun turno dello strige, il bersaglio perde 5 (1d4 + 3) Punti Ferita a causa della perdita di sangue.

Lo strige può staccarsi spendendo 1 metro di movimento. Lo fa automaticamente dopo aver risucchiato 10 Punti Ferita dal bersaglio o alla morte del bersaglio. Una creatura, compreso il bersaglio, può usare la sua azione per staccare lo strige.

\medskip\textbf{Tasso}\index[Mostruario]{Tasso}

\textit{Minuscola bestia, disallineato}

\textbf{FORZA} -3

\textbf{DESTREZZA} +0

\textbf{COSTITUZIONE} +1

\textbf{INTELLIGENZA} -4

\textbf{SAGGEZZA} +1

\textbf{CARISMA} -3

\textbf{Iniziativa} +0 -- \textbf{Difesa} 11

\textbf{Punti Ferita} 3 (1d4 + 1)

\textbf{Movimento} 6 m, scavo 1 m

\textbf{Tiri Salvezza}: Tempra -3, Riflessi +1, Volontà +1

\textbf{Sensi} visione al buio 9 m

\textbf{Lingue} -

\textbf{Sfida} 0 (10 PX)

\textit{\textbf{Olfatto Affinato.}} Il tasso ha +1d6 alle prove di Saggezza (Consapevolezza) basate sull'olfatto.

\textbf{Azioni}

\textit{\textbf{Morso.} Attacco con Arma da Mischia}: +2 a colpire, portata 1 m, un bersaglio.

\textit{Colpisce:} 1 danno perforante.

\medskip\textbf{Tasso Gigante}\index[Mostruario]{Tasso Gigante}

\textit{Media bestia, disallineato}

\textbf{FORZA} +1

\textbf{DESTREZZA} +0

\textbf{COSTITUZIONE} +2

\textbf{INTELLIGENZA} -4

\textbf{SAGGEZZA} +1

\textbf{CARISMA} -3

\textbf{Iniziativa} +0 -- \textbf{Difesa} 11

\textbf{Punti Ferita} 13 (2d8 + 4)

\textbf{Movimento} 9 m, scavo 3 m

\textbf{Tiri Salvezza}: Tempra +2, Riflessi +1, Volontà +2

\textbf{Sensi} visione al buio 9 m

\textbf{Lingue} -

\textbf{Sfida} 1/4 (50 PX)

\textit{\textbf{Olfatto Affinato.}} Il tasso ha +1d6 alle prove di Saggezza (Consapevolezza) basate sull'olfatto.

\textbf{Azioni}

\textit{\textbf{Multiattacco.}} Il tasso effettua due attacchi: uno con il morso e uno con gli artigli.

\textit{\textbf{Artigli.} Attacco con Arma da Mischia}: +3 a colpire, portata 1 m, un bersaglio.

\textit{Colpisce:} 6 (2d4 + 1) danni taglienti.

\textit{\textbf{Morso.} Attacco con Arma da Mischia}: +3 a colpire, portata 1 m, un bersaglio.

\textit{Colpisce:} 4 (1d6 + 1) danni perforanti.

\medskip\textbf{Tigre}\index[Mostruario]{Tigre}

\textit{Grande bestia, disallineato}

\textbf{FORZA} +3

\textbf{DESTREZZA} +2

\textbf{COSTITUZIONE} +2

\textbf{INTELLIGENZA} -4

\textbf{SAGGEZZA} +1

\textbf{CARISMA} -1

\textbf{Iniziativa} +2 -- \textbf{Difesa} 13

\textbf{Punti Ferita} 37 (5d10 + 10)

\textbf{Movimento} 12 m

\textbf{Tiri Salvezza}: Tempra +4, Riflessi +4, Volontà +2

\textbf{Competenze} Muoversi Silenziosamente / Nascondersi +6, Consapevolezza +3

\textbf{Sensi} visione al buio 18 m

\textbf{Lingue} -

\textbf{Sfida} 1 (200 PX)

\textit{\textbf{Balzo.}} Se la tigre si muove di almeno 6 metri diretta verso una creatura e la colpisce con un attacco di artiglio durante lo stesso turno, il bersaglio deve riuscire un Tiro Salvezza di Tempra DC 13 o cadere prono. Se il bersaglio è prono, la tigre può effettuare un attacco di morso contro di esso come azione bonus.

\textit{\textbf{Olfatto Affinato.}} La tigre ha +1d6 alle prove di Saggezza (Consapevolezza) basate sull'olfatto.

\textbf{Azioni}

\textit{\textbf{Artiglio.} Attacco con Arma da Mischia}: +5 a colpire, portata 1 m, un bersaglio.

\textit{Colpisce:} 7 (1d8 + 3) danni taglienti, 1 danno da Sanguinamento.

\textit{\textbf{Morso.} Attacco con Arma da Mischia}: +5 a colpire, portata 1 m, un bersaglio.

\textit{Colpisce:} 8 (1d10 + 3) danni perforanti.

\medskip\textbf{Tigre dai Denti a Sciabola}\index[Mostruario]{Tigre dai Denti a Sciabola}

\textit{Grande bestia, disallineato}

\textbf{FORZA} +4

\textbf{DESTREZZA} +2

\textbf{COSTITUZIONE} +2

\textbf{INTELLIGENZA} -4

\textbf{SAGGEZZA} +1

\textbf{CARISMA} -1

\textbf{Iniziativa} +2 -- \textbf{Difesa} 13

\textbf{Punti Ferita} 52 (7d10 + 14)

\textbf{Movimento} 12 m

\textbf{Tiri Salvezza}: Tempra +5, Riflessi +3, Volontà +2

\textbf{Competenze} Muoversi Silenziosamente / Nascondersi +6, Consapevolezza +3

\textbf{Lingue} -

\textbf{Sfida} 2 (450 PX)

\textit{\textbf{Balzo.}} Se la tigre si muove di almeno 6 metri diretta verso una creatura e la colpisce con un attacco di artiglio durante lo stesso turno, il bersaglio deve riuscire un Tiro Salvezza di Tempra DC 14 o cadere prono. Se il bersaglio è prono, la tigre può effettuare un attacco di morso contro di esso come azione bonus.

\textit{\textbf{Olfatto Affinato.}} La tigre ha +1d6 alle prove di Saggezza (Consapevolezza) basate sull'olfatto.

\textbf{Azioni}

\textit{\textbf{Artiglio.} Attacco con Arma da Mischia}: +6 a colpire, portata 1 m, un bersaglio.

\textit{Colpisce:} 12 (2d6 + 5) danni taglienti, 1 danno da Sanguinamento.

\textit{\textbf{Morso.} Attacco con Arma da Mischia}: +6 a colpire, portata 1 m, un bersaglio.

\textit{Colpisce:} 10 (1d10 + 5) danni perforanti.

\medskip\textbf{Vespa Gigante}\index[Mostruario]{Vespa Gigante}

\textit{Media bestia, disallineato}

\textbf{FORZA} +0

\textbf{DESTREZZA} +2

\textbf{COSTITUZIONE} +0

\textbf{INTELLIGENZA} -5

\textbf{SAGGEZZA} +0

\textbf{CARISMA} -4

\textbf{Iniziativa} +2 -- \textbf{Difesa} 13

\textbf{Punti Ferita} 13 (3d8)

\textbf{Movimento} 3 m, volo 15 m

\textbf{Tiri Salvezza}: Tempra +1, Riflessi +3, Volontà +0

\textbf{Lingue} -

\textbf{Sfida} 1/2 (100 PX)

\textbf{Azioni}

\textit{\textbf{Pungiglione.} Attacco con Arma da Mischia}: +4 a colpire, portata 1 m, una creatura.

\textit{Colpisce:} 5 (1d6 + 2) danni perforanti e il bersaglio deve effettuare un Tiro Salvezza di Tempra DC 11, e subire 10 (3d6) danni da veleno se fallisce il Tiro Salvezza, o la metà di questi danni se lo riesce. Se il danno da veleno riduce il bersaglio a 0 Punti Ferita, il bersaglio è stabile ma avvelenato per 1 ora, anche dopo aver recuperato i Punti Ferita, e mentre è avvelenato in questo modo resta paralizzato.

\medskip\textbf{Worg}\index[Mostruario]{Worg}

I worg sono mostruosi predatori dall'aspetto simile ad un lupo che amano cacciare e divorare le creature più deboli di loro.

\textit{Grande mostruosità, neutrale malvagio}

\textbf{FORZA} +3

\textbf{DESTREZZA} +1

\textbf{COSTITUZIONE} +1

\textbf{INTELLIGENZA} -2

\textbf{SAGGEZZA} +0

\textbf{CARISMA} -1

\textbf{Iniziativa} +1 -- \textbf{Difesa} 14

\textbf{Punti Ferita} 26 (4d10 + 4)

\textbf{Movimento} 15 m

\textbf{Tiri Salvezza}: Tempra +3, Riflessi +2, Volontà +2

\textbf{Competenze} Consapevolezza +4

\textbf{Sensi} visione al buio 18 m

\textbf{Lingue} Goblin, Worg

\textbf{Sfida} 1/2 (100 PX)

\textit{\textbf{Udito e Olfatto Affinato.}} Il worg ha +1d6 nelle prove di Saggezza (Consapevolezza) basate su udito o olfatto.

\textbf{Azioni}

\textit{\textbf{Morso.} Attacco con Arma da Mischia}: +5 a colpire, portata 1 m, un bersaglio.

\textit{Colpisce:} 10 (2d6 + 3) danni perforanti. Se il bersaglio è una creatura, deve riuscire un Tiro Salvezza di Tempra DC 13 o cadere prona.

\subsection{Appendice B: Personaggi Non Giocanti}\index[Mostruario]{Personaggi Non Giocanti}

Questa appendice contiene le statistiche di vari personaggi non giocanti (PNG) umanoidi che gli avventurieri possono incontrare nel corso di una campagna, da infimi popolani a potenti arcimaghi. Queste statistiche possono essere utilizzate per rappresentare PNG umani e non.

Personalizzare i PNG

Esistono molti semplici modi di personalizzare i PNG di questa appendice per l'uso nella tua campagna casalinga.

\textit{\textbf{Cambiare Incantesimi.}} Un modo per personalizzare un PNG incantatore è quello di rimpiazzare uno o più dei suoi incantesimi. Puoi sostituire qualsiasi incantesimo della lista di
incantesimi del PNG con un diverso incantesimo dello stesso livello. Cambiare incantesimi in questo modo non modifica il grado di sfida del PNG.

\textbf{\textit{Cambiare Armi e Armatura}.} Puoi migliorare o peggiorare l'armatura del PNG o aggiungere o cambiare armi. Le modifiche alla Difesa e ai danni possono modificare il grado di sfida del PNG.

\textit{\textbf{Oggetti Magici}}. Più potente è un PNG, maggiori le probabilità che possieda uno o più oggetti magici. Un mago, ad esempio, potrebbe avere una bacchetta o un bastone magico, oltre ad una o più pozioni e pergamene. Fornire un PNG di un potente oggetto magico capace di infliggere danni potrebbe modificarne il grado di sfida.

Alcuni oggetti magici di esempio sono descritti più avanti in questo documento.

\textbf{Combattenti}

I combattenti sono individui che si guadagnano da vivere mettendo la loro spada al servizio di un individuo o un ideale.

\medskip\textbf{Guardia}

Le guardie comprendono membri della ronda cittadina, sentinelle di una cittadella o città fortificata e le guardie del corpo di nobili e mercanti.

\textit{Media umanoide (qualsiasi razza), qualsiasi Tratto}

\textbf{FORZA} +1

\textbf{DESTREZZA} +1

\textbf{COSTITUZIONE} +1

\textbf{INTELLIGENZA} +0

\textbf{SAGGEZZA} +0

\textbf{CARISMA} +0

\textbf{Iniziativa} +1 -- \textbf{Difesa} 17 (giaco di maglia, scudo)

\textbf{Punti Ferita} 11 (2d8 + 2)

\textbf{Movimento} 9 m

\textbf{Tiri Salvezza}: Tempra +3, Riflessi +1, Volontà +1

\textbf{Competenze} Consapevolezza +2

\textbf{Lingue} una qualsiasi lingua (di solito il Comune)

\textbf{Sfida} 1/8 (25 PX)

\textbf{Azioni}

\textit{\textbf{Lancia.} Attacco con Arma da Mischia o a Gittata}: +3 a colpire, portata 1 m o gittata 6m, un bersaglio.

\textit{Colpisce:} 4 (1d6 + 1) danni perforanti o 5 (1d8 + 1) danni perforanti se impiegata con due mani per effettuare un attacco da mischia.

\medskip\textbf{Veterano}

Guerrieri sopravvissuti a lungo, guadagnandosi una grande fama di esperti e abili combattenti.

\textit{Media umanoide (qualsiasi razza), qualsiasi Tratto}

\textbf{FORZA} +3

\textbf{DESTREZZA} +1

\textbf{COSTITUZIONE} +2

\textbf{INTELLIGENZA} +0

\textbf{SAGGEZZA} +0

\textbf{CARISMA} +0

\textbf{Iniziativa} +1 -- \textbf{Difesa} 19 (armatura di strisce)

\textbf{Punti Ferita} 58 (9d8 + 18)

\textbf{Movimento} 9 m

\textbf{Tiri Salvezza}: Tempra +4, Riflessi +2, Volontà +3

\textbf{Competenze} Acrobatica +5, Consapevolezza +2

\textbf{Lingue} una lingua qualsiasi (di solito il Comune)

\textbf{Sfida} 3 (700 PX)

\textbf{Azioni}

\textit{\textbf{Multiattacco.}} Il veterano effettua due attacchi con la spada lunga. Se ha estratto una spada corta, può effettuare anche un attacco con la spada corta.

\textit{\textbf{Spada Lunga.} Attacco con Arma da Mischia}: +5 a colpire, portata 1 m, un bersaglio.

\textit{Colpisce:} 7 (1d8 + 3) danni taglienti, o 8 (1d10 + 3) danni taglienti se usata con due mani.

\textit{\textbf{Spada Corta.} Attacco con Arma da Mischia}: +5 a colpire, portata 1 m, un bersaglio.

\textit{Colpisce:} 6 (1d6 + 3) danni perforanti.

\textit{\textbf{Balestra Pesante.} Attacco con Arma a Gittata}: +3 a colpire, gittata 30m, un bersaglio. \textit{Colpisce:} 6 (1d10 + 1) danni perforanti.

\medskip\textbf{Cavaliere}

I cavalieri sono combattenti che giurano fedeltà a sovrani, ordini religiosi, e nobili cause. I Tratti del cavaliere determinano fino a che punto è disposto ad onorare il suo giuramento.

\textit{Media umanoide (qualsiasi razza), qualsiasi Tratto}

\textbf{FORZA} +3

\textbf{DESTREZZA} +0

\textbf{COSTITUZIONE} +2

\textbf{INTELLIGENZA} +0

\textbf{SAGGEZZA} +0

\textbf{CARISMA} +2

\textbf{Iniziativa} +0 -- \textbf{Difesa} 20 (armatura di piastre)

\textbf{Punti Ferita} 52 (8d8 + 16)

\textbf{Movimento} 9 m

\textbf{Tiri Salvezza}: Tempra +4, Riflessi +1, Volontà +3

\textbf{Lingue} una qualsiasi lingua (di solito il Comune)

\textbf{Sfida} 3 (700 PX)

\textit{\textbf{Coraggioso.}} Il cavaliere ha +1d6 ai Tiri Salvezza contro l'essere spaventato.

\textbf{Azioni}

\textit{\textbf{Multiattacco.}} Il cavaliere effettua due attacchi da mischia.

\textit{\textbf{Spada Grossa.} Attacco con Arma da Mischia}: +5 a colpire, portata 1 m, un bersaglio.

\textit{Colpisce:} 10 (2d6 + 3) danni taglienti.

\textit{\textbf{Balestra Pesante.} Attacco con Arma a Gittata}: +2 a colpire, gittata 30m, un bersaglio.

\textit{Colpisce:} 5 (1d10) perforanti.

\textit{\textbf{Autorità (Ricarica dopo un 1 ora)}}. Per 1 minuto, il cavaliere può pronunciare un comando speciale o avvertimento ogni qualvolta una creatura non ostile entro 9 metri da lui, e che possa vedere, effettua un tiro di attacco o Tiro Salvezza. La creatura può sommare un d4 al suo tiro purché possa udire e comprendere il cavaliere. Una creatura può beneficiare di un solo dado Autorità alla volta. Questo effetto termina se il cavaliere è inabile.

\textbf{Reazioni}

\textit{\textbf{Parata.}} Il cavaliere può aggiungere 2 alla sua Difesa contro un attacco da mischia che lo colpirebbe. Per farlo, il cavaliere deve vedere l'attaccante e star impugnando un'arma da mischia.

\medskip\textbf{Gladiatore}

Addestrati per intrattenere le folle, sono tra i combattenti più pericolosi in circolazione.

\textit{Media umanoide (qualsiasi razza), qualsiasi Tratto}

\textbf{FORZA} +4

\textbf{DESTREZZA} +2

\textbf{COSTITUZIONE} +3

\textbf{INTELLIGENZA} +0

\textbf{SAGGEZZA} +1

\textbf{CARISMA} +2

\textbf{Iniziativa} +2 -- \textbf{Difesa} 19 (armatura di cuoio borchiato, scudo)

\textbf{Punti Ferita} 112 (15d8 + 45)

\textbf{Movimento} 9 m

\textbf{Tiri Salvezza}: Tempra +5, Riflessi +5, Volontà +3

\textbf{Competenze} Acrobatica +10, Intimidazione +5

\textbf{Lingue} una lingua qualsiasi (di solito il Comune)

\textbf{Sfida} 5 (1.800 PX)

\textit{\textbf{Bruto.}} Un'arma da mischia infligge un dado aggiuntivo di danno

quando un gladiatore colpisce con essa (già incluso nell'attacco).

\textit{\textbf{Coraggioso.}} Il gladiatore ha +1d6 ai Tiri Salvezza contro l'essere spaventato.

\textbf{Azioni}

\textit{\textbf{Multiattacco.}} Il gladiatore effettua tre attacchi da mischia o due attacchi a gittata.

\textit{\textbf{Lancia.} Attacco con Arma da Mischia o a Gittata}: +7 a colpire, portata 1 m o gittata 6m, un bersaglio.

\textit{Colpisce:} 11 (2d6 + 4) danni perforanti, o 13 (2d8 + 4) danni taglienti se usata con due mani.

\textit{\textbf{Botta di Scudo.} Attacco con Arma da Mischia}: +7 a colpire, portata 1 m, un bersaglio.

\textit{Colpisce:} 9 (2d4 + 4) danni da botta. Se il bersaglio è una creatura di taglia Media o inferiore, deve riuscire un Tiro Salvezza su Tempra DC 15 o cadere prono.

\textbf{Reazioni}

\textit{\textbf{Parata.}} Il gladiatore somma 3 alla sua Difesa contro un attacco da mischia che lo colpirebbe. Per farlo, il gladiatore deve vedere l'attaccante e impugnare un'arma da mischia.

\medskip\textbf{Cittadini}

In questa categoria rientrano quegli individui che si occupano di mandare avanti il mondo, svolgendo le mansioni necessarie affinché i campi vengano coltivati, le città amministrate, il cibo coltivato e
nuovi territori esplorati.

\medskip\textbf{Nobile}

I nobili comandano sulla popolazione, in virtù di un diritto di nascita o per le ricchezze accumulate. Tra costoro si annoverano anche i cortigiani che affollano le corti dei ricchi e dei potenti.

\textit{Media umanoide (qualsiasi razza), qualsiasi Tratto}

\textbf{FORZA} +0

\textbf{DESTREZZA} +1

\textbf{COSTITUZIONE} +0

\textbf{INTELLIGENZA} +1

\textbf{SAGGEZZA} +2

\textbf{CARISMA} +3

\textbf{Iniziativa} +1 -- \textbf{Difesa} 16 (pettorale)

\textbf{Punti Ferita} 9 (2d8)

\textbf{Movimento} 9 m

\textbf{Tiri Salvezza}: Tempra +1, Riflessi +1, Volontà +2

\textbf{Competenze} Percepire Emozioni +4, Ingannare +5

\textbf{Lingue} due lingue qualsiasi

\textbf{Sfida} 1/8 (25 PX)

\textbf{Azioni}

\textit{\textbf{Stocco.} Attacco con Arma da Mischia}: +3 a colpire, portata 1 m, un bersaglio.

\textit{Colpisce:} 5 (1d8 + 1) danni perforanti.

\textbf{Reazioni}

\textit{\textbf{Parata.}} Il nobile somma 2 alla sua Difesa contro un attacco da mischia che lo colpirebbe. Per farlo, il nobile deve vedere

l'attaccante e impugnare un'arma da mischia.

\medskip\textbf{Popolano}

I popolani comprendono contadini, servi, schiavi, servitori, pellegrini, mercanti, artigiani ed eremiti.

\textit{Media umanoide (qualsiasi razza), qualsiasi Tratto}

\textbf{FORZA} +0

\textbf{DESTREZZA} +0

\textbf{COSTITUZIONE} +0

\textbf{INTELLIGENZA} +0

\textbf{SAGGEZZA} +0

\textbf{CARISMA} +0

\textbf{Iniziativa} +0 -- \textbf{Difesa} 11

\textbf{Punti Ferita} 4 (1d8)

\textbf{Movimento} 9 m

\textbf{Tiri Salvezza}: Tempra +0, Riflessi +0, Volontà +0

\textbf{Lingue} una qualsiasi lingua (di solito il Comune)

\textbf{Sfida} 0 (10 PX)

\textbf{Azioni}

\textit{\textbf{Randello.} Attacco con Arma da Mischia}: +2 a colpire, portata 1 m, un bersaglio.

\textit{Colpisce:} 2 (1d4) danni da botta.

\medskip\textbf{Criminali}

I criminali sono individui che vivono al margine della legalità, procurandosi il pane svolgendo attività spesso considerate illecite e immorali.

\medskip\textbf{Picchiatore}

I picchiatori sono criminali spietati abili nell'intimidire e perpetrare atti di violenza. Lavorano per soldi e si fanno pochi scrupoli.

\textit{Media umanoide (qualsiasi razza), qualsiasi Tratto}

\textbf{FORZA} +2

\textbf{DESTREZZA} +0

\textbf{COSTITUZIONE} +2

\textbf{INTELLIGENZA} +0

\textbf{SAGGEZZA} +0

\textbf{CARISMA} +0

\textbf{Iniziativa} +0 -- \textbf{Difesa} 12 (armatura di cuoio)

\textbf{Punti Ferita} 32 (5d8 + 10)

\textbf{Movimento} 9 m

\textbf{Tiri Salvezza}: Tempra +3, Riflessi +1, Volontà +0

\textbf{Competenze} Intimidazione +2

\textbf{Lingue} una lingua qualsiasi (di solito il Comune)

\textbf{Sfida} 1/2 (100 PX)

\textit{\textbf{Tattiche di Branco.}} Il picchiatore ha +1d6 ai tiri di attacco contro una creatura se almeno uno degli alleati del picchiatore si trova entro 1 metro dalla creatura e quell'alleato non
è inabile.

\textbf{Azioni}

\textit{\textbf{Multiattacco.}} Il picchiatore effettua due attacchi da mischia.

\textit{\textbf{Mazza.} Attacco con Arma da Mischia}: +4 a colpire, portata 1 m, una creatura.

\textit{Colpisce:} 5 (1d6 + 2) danni da botta.

\textit{\textbf{Balestra Pesante.} Attacco con Arma a Gittata}: +2 a colpire, gittata 30m, un bersaglio. \textit{Colpisce:} 5 (1d10) danni perforanti.

\medskip\textbf{Bandito/Pirata}

Che siano uomini di strada o di mare (pirati) costoro guadagnano da vivere depredando il prossimo.

\textit{Media umanoide (qualsiasi razza), qualsiasi Tratto non legale}

\textbf{FORZA} +0

\textbf{DESTREZZA} +1

\textbf{COSTITUZIONE} +1

\textbf{INTELLIGENZA} +0

\textbf{SAGGEZZA} +0

\textbf{CARISMA} +0

\textbf{Iniziativa} +1 -- \textbf{Difesa} 13 (armatura di cuoio)

\textbf{Punti Ferita} 11 (2d8 + 2)

\textbf{Movimento} 9 m

\textbf{Tiri Salvezza}: Tempra +1, Riflessi +2, Volontà +1

\textbf{Lingue} una qualsiasi lingua (di solito il Comune)

\textbf{Sfida} 1/8 (25 PX)

\textbf{Azioni}

\textit{\textbf{Scimitarra.} Attacco con Arma da Mischia}: +3 a colpire, portata 1 m, un bersaglio.

\textit{Colpisce:} 4 (1d6 + 1) danni taglienti.

\textit{\textbf{Balestra Leggera.} Attacco con Arma a Gittata}: +3 a colpire, gittata 24m, un bersaglio. \textit{Colpisce:} 5 (1d8 + 1) danni taglienti.

\medskip\textbf{Spia}

Una spia è un individuo addestramento nel reperire segreti per conto di qualcuno, o a volte per rivenderli al miglior offerente.

\textit{Media umanoide (qualsiasi razza), qualsiasi Tratto}

\textbf{FORZA} +0

\textbf{DESTREZZA} +2

\textbf{COSTITUZIONE} +0

\textbf{INTELLIGENZA} +1

\textbf{SAGGEZZA} +2

\textbf{CARISMA} +3

\textbf{Iniziativa} +2 -- \textbf{Difesa} 13

\textbf{Punti Ferita} 27 (6d8)

\textbf{Movimento} 9 m

\textbf{Tiri Salvezza}: Tempra +2, Riflessi +3, Volontà +3

\textbf{Competenze} Muoversi Silenziosamente / Nascondersi +4, Percepire Emozioni +4, Investigazione +5, Consapevolezza +6, Ingannare +5, Mani di fata +4

\textbf{Lingue} due lingue qualsiasi

\textbf{Sfida} 1 (200 PX)

\textit{\textbf{Attacco Furtivo (1/Turno).}} La spia infligge 7 (2d6) danni aggiuntivi quando colpisce un bersaglio con un attacco con arma e ha +1d6 al tiro di attacco, o quando il bersaglio è entro 1 metro da un alleato dell'assassino che non è inabile e l'assassino non ha -1d6 al tiro di attacco.

\textit{\textbf{Azione Astuta.}} Durante ciascun suo round, la spia può usare un'azione bonus per effettuare l'azione Ritirarsi, Nascondersi o Scattare.

\textbf{Azioni}

\textit{\textbf{Multiattacco.}} La spia effettua due attacchi da mischia.

\textit{\textbf{Spada Corta.} Attacco con Arma da Mischia}: +4 a colpire, portata 1 m, un bersaglio.

\textit{Colpisce:} 5 (1d6 + 2) danni perforanti.

\textit{\textbf{Balestrino.} Attacco con Arma a Gittata}: +4 a colpire, gittata 9m, un bersaglio. \textit{Colpisce:} 5 (1d6 + 2) danni perforanti.


\medskip\textbf{Capitano dei Banditi/Pirata}

Che viva in terra o in mare, è un individuo munito di una grande personalità che riesce a tenere in riga la marmaglia che risponde ai suoi ordini.

\textit{Media umanoide (qualsiasi razza), qualsiasi Tratto non legale}

\textbf{FORZA} +2

\textbf{DESTREZZA} +3

\textbf{COSTITUZIONE} +2

\textbf{INTELLIGENZA} +2

\textbf{SAGGEZZA} +0

\textbf{CARISMA} +2

\textbf{Iniziativa} +2 -- \textbf{Difesa} 16 (armatura di cuoio borchiato)

\textbf{Punti Ferita} 65 (10d8 + 8)

\textbf{Movimento} 9 m

\textbf{Tiri Salvezza}: Tempra +5, Riflessi +5, Volontà +3

\textbf{Competenze} Acrobatica +4, Raggiro +4

\textbf{Lingue} due lingue qualsiasi

\textbf{Sfida} 2 (450 PX)

\textbf{Azioni}

\textit{\textbf{Multiattacco.}} Il capitano effettua tre attacchi da mischia: due con la scimitarra e uno con il pugnale. Oppure il capitano effettua due attacchi a gittata con i pugnali.

\textit{\textbf{Scimitarra.} Attacco con Arma da Mischia}: +5 a colpire, portata 1 m, un bersaglio.

\textit{Colpisce:} 6 (1d6 + 3) danni taglienti.

\textit{\textbf{Pugnale.} Attacco con Arma da Mischia o a Gittata}: +5 a colpire, portata 1 m o gittata 6m, un bersaglio. \textit{Colpisce:} 5 (1d4 + 3) danni perforanti.

\textbf{Reazioni}

\textit{\textbf{Parata.}} Il capitano somma 2 alla sua Difesa contro un attacco da mischia che lo colpirebbe. Per farlo, il capitano deve vedere l'attaccante e impugnare un'arma da mischia.

\medskip\textbf{Assassino}

Solitari o membri di una gilda, gli assassini sono pagati per eliminare, spesso in modo silenzioso e discreto, rivali e nemici dei loro datori di lavoro.

\textit{Media umanoide (qualsiasi razza), qualsiasi Tratto non buono}

\textbf{FORZA} +0

\textbf{DESTREZZA} +3

\textbf{COSTITUZIONE} +2

\textbf{INTELLIGENZA} +1

\textbf{SAGGEZZA} +0

\textbf{CARISMA} +0

\textbf{Iniziativa} +3 -- \textbf{Difesa} 19 (armatura di cuoio borchiato)

\textbf{Punti Ferita} 78 (12d8 + 24)

\textbf{Movimento} 9 m

\textbf{Tiri Salvezza}: Tempra +4, Riflessi +6, Volontà +3

\textbf{Competenze} Acrobazia +6, Muoversi Silenziosamente / Nascondersi +9, Consapevolezza +3, Raggiro +3


\textbf{Lingue} Gergo dei Ladri più due altre lingue

\textbf{Sfida} 8 (3.900 PX)

\textit{\textbf{Assassinare.}} Durante il suo primo turno, l'assassino ha +1d6 ai tiri di attacco contro le creature che non hanno ancora svolto nessun turno. Qualsiasi colpo che l'assassino mandi a segno contro una creatura sorpresa, è un colpo critico.

\textit{\textbf{Attacco Furtivo (1/Turno).}} L'assassino infligge 14 (4d6) danni aggiuntivi quando colpisce un bersaglio con un attacco con arma e ha +1d6 al tiro di attacco, o quando il bersaglio è entro 1 metro da un alleato dell'assassino che non è inabile e l'assassino non ha -1d6 al tiro di attacco.

\textit{\textbf{Evasione.}} Se l'assassino è vittima di un effetto che permette di effettuare un Tiro Salvezza di Riflessi per dimezzare i danni, l'assassino non prende danni se riesce il Tiro Salvezza, e solo la metà se lo fallisce.

\textbf{Azioni}

\textit{\textbf{Multiattacco.}} L'assassino effettua due attacchi con le spade corte.

\textit{\textbf{Spada Corta.} Attacco con Arma da Mischia}: +6 a colpire, portata 1 m, un bersaglio.

\textit{Colpisce:} 6 (1d6 + 3) danni perforanti, e il bersaglio deve effettuare un Tiro Salvezza di Tempra DC 15, subendo 24 (7d6) danni da veleno se fallisce il Tiro Salvezza, o la metà di questi danni se lo riesce.

\textit{\textbf{Balestra Leggera.} Attacco con Arma a Gittata}: +6 a colpire, gittata 24m, un bersaglio.

\textit{Colpisce:} 7 (1d8 + 3) danni perforanti, e il bersaglio deve effettuare un Tiro Salvezza di Tempra DC 15, subendo 24 (7d6) danni da veleno se fallisce il Tiro Salvezza, o la metà di questi danni se lo riesce.

\medskip\textbf{Mago}

Il mago trascorre la vita nello studio e la pratica della magia.

\textbf{VARIANTE: FAMIGLI}

Qualsiasi incantatore che possa eseguire l'incantesimo \textit{trovare} \textit{famiglio} è probabile che abbia un famiglio. Il famiglio può essere una delle creature descritte nell'incantesimo (vedi le \textit{Regole Base}) o qualche altro mostro Minuscolo, come un artiglio strisciante, un diavoletto, uno pseudodrago o un demonietto.

\medskip\textbf{Mago Avventuriero}

Un Mago novizio, che ha superato con successo le sue prime avventure e ha iniziato a stabilire una reputazione come nobile o famigerato avventuriero.

\textit{Media umanoide (qualsiasi razza), qualsiasi malvagio}

\textbf{FORZA} -1

\textbf{DESTREZZA} +2

\textbf{COSTITUZIONE} +0

\textbf{INTELLIGENZA} +3

\textbf{SAGGEZZA} +1

\textbf{CARISMA} +0

\textbf{Iniziativa} +3 -- \textbf{Difesa} 13

\textbf{Punti Ferita} 22 (5d8)

\textbf{Movimento} 9 m

\textbf{Tiri Salvezza}: Tempra +0, Riflessi +3, Volontà +2

\textbf{Competenze} Arcano +5, Storia +5

\textbf{Lingue} quattro lingue qualsiasi

\textbf{Sfida} 1 (200 PX)

\textit{\textbf{Incantesimi.}} Il mago ha CM 4. La sua abilità da incantatore è l'Intelligenza (+5 al colpire con attacchi con incantesimo). Il Mago ha preparato i seguenti incantesimi: Trucchetti (a volontà):

\textit{luce, mano magica, stretta folgorante}

livello 1 (4 slot): \textit{charme su persone, dardo incantato}

livello 2 (3 slot): \textit{bloccare persona, passo velato}

\textbf{Azioni}

\textit{\textbf{Bastone.} Attacco con Arma da Mischia}: +1 a colpire, portata 1 m, un bersaglio.

\textit{Colpisce:} 3 (1d8 - 1) danni da botta.

\medskip\textbf{Grande Mago}

Un Mago che ha stabilito una discreta fama nel territorio e che attira intorno a sé studenti da ogni dove.

\textit{Media umanoide (qualsiasi razza), qualsiasi Tratto}

\textbf{FORZA} -1

\textbf{DESTREZZA} +2

\textbf{COSTITUZIONE} +0

\textbf{INTELLIGENZA} +3

\textbf{SAGGEZZA} +1

\textbf{CARISMA} +0

\textbf{Iniziativa} +3 -- \textbf{Difesa} 15 (18 con \textit{armatura del Mago})

\textbf{Punti Ferita} 40 (9d8)

\textbf{Movimento} 9 m

\textbf{Tiri Salvezza}: Tempra +1, Riflessi +4, Volontà +3

\textbf{Competenze} Arcano +6, Storia +6

\textbf{Lingue} quattro lingue qualsiasi

\textbf{Sfida} 6 (2.300 PX)

\textit{\textbf{Incantesimi.}} Il mago ha CM 9. La sua abilità da incantatore è l'Intelligenza (+6 al colpire con attacchi con incantesimo). Il Mago ha preparato i seguenti incantesimi:

Trucchetti (a volontà): \textit{dardo infuocato, luce, mano magica,}
\textit{prestidigitazione}

livello 1 (4 slot): \textit{armatura del Mago, dardo incantato,}
\textit{individuare magia, scudo}

livello 2 (3 slot): \textit{passo velato, suggestione}

livello 3 (3 slot): \textit{controincantesimo, palla di fuoco, volare}

livello 4 (3 slot): \textit{invisibilità superiore, tempesta di ghiaccio}

livello 5 (1 slot): \textit{cono di freddo}

\textbf{Azioni}

\textit{\textbf{Pugnale.} Attacco con Arma da Mischia o a Gittata}: +5 a colpire, portata 1 m o gittata 6m, un bersaglio. \textit{Colpisce:} 4 (1d4 + 2) danni perforanti.

\medskip\textbf{Arcimago}

Un mago molto potente (e anche molto anziano) che studia i segreti del multiverso.

\textit{Media umanoide (qualsiasi razza), qualsiasi Tratto}

\textbf{FORZA} +0

\textbf{DESTREZZA} +2

\textbf{COSTITUZIONE} +1

\textbf{INTELLIGENZA} +5

\textbf{SAGGEZZA} +2

\textbf{CARISMA} +3

\textbf{Iniziativa} +5 -- \textbf{Difesa} 18 (21 con \textit{armatura del Mago})

\textbf{Punti Ferita} 99 (18d8 + 18)

\textbf{Movimento} 9 m

\textbf{Tiri Salvezza}: Tempra +8, Riflessi +10, Volontà +12

\textbf{Competenze} Arcano +13, Storia +13

\textbf{Resistenze al Danno} danno degli incantesimi; da botta, perforante e tagliente non magico (da \textit{pelle di pietra})

\textbf{Lingue} sei lingue qualsiasi

\textbf{Sfida} 12 (8.400 PX)

\textit{\textbf{Incantesimi.}} Il mago ha CM 18. La sua abilità da incantatore è l'Intelligenza (+9 al colpire con attacchi con incantesimo).

L'arcimago può eseguire \textit{camuffare sé stesso} e \textit{invisibilità} a volontà e ha preparato i seguenti incantesimi: Trucchetti (a volontà): \textit{dardo infuocato, luce, mano magica,}
\textit{prestidigitazione, stretta folgorante}

livello 1 (4 slot): \textit{armatura magica*, dardo incantato, identificare, individuare magia}

livello 2 (3 slot): \textit{immagine speculare, individuazione dei pensieri, passo velato}

livello 3 (3 slot): \textit{controincantesimo, fulmine}

livello 4 (3 slot): \textit{esilio, pelle di pietra*, scudo di fuoco}

livello 5 (3 slot): \textit{cono di freddo, muro di forza, scrutare}

livello 6 (1 slot): \textit{globo di invulnerabilità}

livello 7 (1 slot): \textit{teletrasporto}

livello 8 (1 slot): \textit{vuoto mentale*}

livello 9 (1 slot): \textit{fermare il tempo}

L'arcimago esegue questi {*} incantesimi su di sé prima del combattimento.

\textbf{Azioni}

\textit{\textbf{Pugnale.} Attacco con Arma da Mischia o a Gittata}: +6 a colpire, portata 1 m o gittata 6m, un bersaglio. \textit{Colpisce:} 4 (1d4 + 2) danni perforanti.


\medskip\textbf{Sacerdoti}

I sacerdoti sono devoti di una divinità o una fede che si prendono cura di impartire gli insegnamenti divini al loro gregge.

\medskip\textbf{Cultista}

I cultisti giurano fedeltà ai poteri oscuri, e nelle loro credenze e pratiche mostrano spesso segni di follia.

\textit{Media umanoide (qualsiasi razza), qualsiasi Tratto non buono}

\textbf{FORZA} +0

\textbf{DESTREZZA} +1

\textbf{COSTITUZIONE} +0

\textbf{INTELLIGENZA} +0

\textbf{SAGGEZZA} +0

\textbf{CARISMA} +0

\textbf{Iniziativa} +0- \textbf{Difesa} 13 (armatura di cuoio)

\textbf{Punti Ferita} 9 (2d8)

\textbf{Movimento} 9 m

\textbf{Tiri Salvezza}: Tempra +1, Riflessi +1, Volontà +2

\textbf{Competenze} Raggiro +2, Religione +2

\textbf{Lingue} una qualsiasi lingua (di solito il Comune)

\textbf{Sfida} 1/8 (25 PX)

\textit{\textbf{Oscura Devozione.}} Il cultista ha +1d6 sui Tiri Salvezza contro l'essere affascinato o spaventato.

\textbf{Azioni}

\textit{\textbf{Scimitarra.} Attacco con Arma da Mischia}: +3 a colpire, portata 1 m, una creatura.

\textit{Colpisce:} 4 (1d6 + 1) danni taglienti.

\medskip\textbf{Accolito}

Gli accoliti sono membri di grado minore del clero, e di solito rispondono ad un sacerdote di rango superiore. Svolgono diverse funzioni in un tempio e gli viene conferita dalla loro divinità l'abilità di eseguire incantesimi minori.

\textit{Media umanoide (qualsiasi razza), qualsiasi Tratto}

\textbf{FORZA} +0

\textbf{DESTREZZA} +0

\textbf{COSTITUZIONE} +0

\textbf{INTELLIGENZA} +0

\textbf{SAGGEZZA} +2

\textbf{CARISMA} +0

\textbf{Iniziativa} +0 -- \textbf{Difesa} 11

\textbf{Punti Ferita} 9 (2d8)

\textbf{Movimento} 9 m

\textbf{Tiri Salvezza}: Tempra +0, Riflessi +0, Volontà +3

\textbf{Competenze} Pronto Soccorso +4, Religione +2

\textbf{Lingue} una qualsiasi lingua (di solito il Comune)

\textbf{Sfida} 1/4 (50 PX)

\textit{\textbf{Incantesimi.}} L'accolito ha CM 1. La sua abilità da incantatore è la Saggezza (+4 al colpire con attacchi con incantesimo). L'accolito ha preparato i seguenti incantesimi: Trucchetti (a volontà): \textit{fiamma sacra, luce, taumaturgia} livello 1 (3 slot): \textit{benedizione}, \textit{cura ferite, santuario}

\medskip\textbf{Azioni}

\textit{\textbf{Randello.} Attacco con Arma da Mischia}: +2 a colpire, portata 1 m, un bersaglio.

\textit{Colpisce:} 2 (1d4) danni da botta.

\textbf{Fanatico del Culto}

Sono i capi di un culto, che usano il proprio carisma e i propri dogmi per influenzare i deboli di volontà.

\textit{Media umanoide (qualsiasi razza), qualsiasi Tratto non buono}

\textbf{FORZA} +0

\textbf{DESTREZZA} +2

\textbf{COSTITUZIONE} +1

\textbf{INTELLIGENZA} +0

\textbf{SAGGEZZA} +1

\textbf{CARISMA} +2

\textbf{Iniziativa} +2 -- \textbf{Difesa} 14 (armatura di cuoio)

\textbf{Punti Ferita} 33 (6d8 + 6)

\textbf{Movimento} 9 m

\textbf{Tiri Salvezza}: Tempra +2, Riflessi +2, Volontà +3

\textbf{Competenze} Ingannare +4, Raggiro +4, Religione +2

\textbf{Lingue} una qualsiasi lingua (di solito il Comune)

\textbf{Sfida} 2 (450 PX)

\textit{\textbf{Incantesimi.}} Il sacerdote ha CM 4. La sua abilità da incantatore è la Saggezza (+3 al colpire con attacchi con incantesimo). Il sacerdote ha preparato i seguenti incantesimi: Trucchetti (a volontà): \textit{fiamma sacra, luce, taumaturgia}

livello 1 (4 slot): \textit{comando, infliggi ferite, scudo della fede}

livello 2 (3 slot): \textit{arma spirituale, blocca persona}

\textit{\textbf{Oscura Devozione.}} Il cultista ha +1d6 sui Tiri Salvezza contro l'essere affascinato o spaventato.

\textbf{Azioni}

\textit{\textbf{Multiattacco.}} Il fanatico effettua due attacchi da mischia.

\textit{\textbf{Pugnale.} Attacco con Arma da Mischia o a Gittata}: +4 a colpire, portata 1 m o gittata 6m, una creatura. \textit{Colpisce:} 4 (1d4 + 2) danni perforanti.

\medskip\textbf{Gran Sacerdote}

Sono individui al comando di un tempio o altro luogo sacro e che hanno a loro disposizione diversi accoliti.

\textit{Media umanoide (qualsiasi razza), qualsiasi Tratto}

\textbf{FORZA} +0

\textbf{DESTREZZA} +0

\textbf{COSTITUZIONE} +1

\textbf{INTELLIGENZA} +1

\textbf{SAGGEZZA} +3

\textbf{CARISMA} +1

\textbf{Iniziativa} +1 -- \textbf{Difesa} 14 (giaco di maglia)

\textbf{Punti Ferita} 27 (5d8 + 5)

\textbf{Movimento} 7 m

\textbf{Tiri Salvezza}: Tempra +1, Riflessi +1, Volontà +4

\textbf{Competenze} Pronto Soccorso +7, Ingannare +3, Religione +4

\textbf{Lingue} due lingue qualsiasi

\textbf{Sfida} 2 (450 PX)

\textit{\textbf{Eminenza Divina.}} Come azione bonus, il sacerdote può spendere uno slot incantesimo per far sì che il suo attacco con arma da mischia infligge 10 (3d6) danni da Luce aggiuntivi. Il beneficio dura fino al termine del turno.

\textit{\textbf{Incantesimi.}} Il sacerdote ha CM 5. La sua abilità da incantatore è la Saggezza (+5 al colpire con attacchi con incantesimo). Il sacerdote ha preparato i seguenti incantesimi: Trucchetti (a volontà): \textit{fiamma sacra, luce, taumaturgia}

livello 1 (4 slot): \textit{cura ferite, dardo tracciante, santuario}

livello 2 (3 slot): \textit{arma spirituale, ristorare inferiore}

livello 3 (2 slot): \textit{dissolvi magie}, \textit{guardiani spirituali}

\textbf{Azioni}

\textit{\textbf{Mazza.} Attacco con Arma da Mischia}: +2 a colpire, portata 1 m, un bersaglio.

\textit{Colpisce:} 3 (1d6) danni da botta.


\medskip\textbf{Selvaggi}

Questi individui vivono ai margini della civiltà, a volte entrandovi raramente in contatto. A disagio tra le mura e nelle terre civilizzate, si trovano nel loro ambiente quando possono muoversi tra le terre selvagge.

\medskip\textbf{Berserker}

Provenienti da terre selvagge, gli imprevedibili berserker si radunano in compagnie di guerra e sono sempre alla ricerca di conflitti in cui combattere.

\textit{Media umanoide (qualsiasi razza), qualsiasi Tratto caotico}

\textbf{FORZA} +3

\textbf{DESTREZZA} +1

\textbf{COSTITUZIONE} +3

\textbf{INTELLIGENZA} -1

\textbf{SAGGEZZA} +0

\textbf{CARISMA} -1

\textbf{Iniziativa} +1 -- \textbf{Difesa} 14 (armatura di pelle)

\textbf{Punti Ferita} 67 (9d8 + 27)

\textbf{Movimento} 9 m

\textbf{Tiri Salvezza}: Tempra +4, Riflessi +3, Volontà +2

\textbf{Lingue} una qualsiasi lingua (di solito il Comune)

\textbf{Sfida} 2 (450 PX)

\textit{\textbf{Incauto.}} All'inizio del suo round, il berserker può ottenere +1d6 su tutti i tiri di attacco con armi da mischia effettuati durante quel turno, ma i tiri di attacco contro di esso hanno +1d6 fino all'inizio del suo prossimo round.

\textbf{Azioni}

\textit{\textbf{Ascia Grossa.} Attacco con Arma da Mischia}: +5 a colpire, portata 1 m, un bersaglio.

\textit{Colpisce:} 9 (1d12 + 3) danni taglienti.

\textbf{Combattente Tribale}

Sono i difensori delle tribù che vivono ai margini della civiltà.

\textit{Media umanoide (qualsiasi razza), qualsiasi Tratto}

\textbf{FORZA} +1

\textbf{DESTREZZA} +0

\textbf{COSTITUZIONE} +1

\textbf{INTELLIGENZA} -1

\textbf{SAGGEZZA} +0

\textbf{CARISMA} -1

\textbf{Iniziativa} +0 -- \textbf{Difesa} 13 (armatura di pelle)

\textbf{Punti Ferita} 11 (2d8 + 2)

\textbf{Movimento} 9 m

\textbf{Tiri Salvezza}: Tempra +2, Riflessi +1, Volontà +1

\textbf{Lingue} una qualsiasi lingua

\textbf{Sfida} 1/8 (25 PX)

\textit{\textbf{Tattiche di Branco.}} Il combattente tribale ha +1d6 ai tiri di attacco contro una creatura se almeno uno degli alleati del picchiatore si trova entro 1 metro dalla creatura e quell'alleato non è inabile.

\textbf{Azioni}

\textit{\textbf{Lancia.} Attacco con Arma da Mischia o a Gittata}: +3 a colpire, portata 1 m o gittata 6m, un bersaglio.

\textit{Colpisce:} 4 (1d6 + 1) danni perforanti, o 5 (1d8 + 1) danni perforanti se usata con due mani per effettuare un attacco da mischia.

\medskip\textbf{Druido}

I druidi proteggono il mondo naturale dai mostri e dall'avanzare della civiltà. Alcuni sono sciamani tribali che curano i malati, pregano agli spiriti animali e forniscono consigli spirituali.

\textit{Media umanoide (qualsiasi razza), qualsiasi Tratto}

\textbf{FORZA} +0

\textbf{DESTREZZA} +1

\textbf{COSTITUZIONE} +1

\textbf{INTELLIGENZA} +1

\textbf{SAGGEZZA} +2

\textbf{CARISMA} +0

\textbf{Iniziativa} +1 -- \textbf{Difesa} 12 (17 con \textit{pelle di corteccia}*)

\textbf{Punti Ferita} 27 (5d8 + 5)

\textbf{Movimento} 9 m

\textbf{Tiri Salvezza}: Tempra +1, Riflessi +2, Volontà +3 \\

\textbf{Competenze} Pronto Soccorso +4, Natura +3, Consapevolezza +4

\textbf{Lingue} Druidico più due altre lingue

\textbf{Sfida} 2 (450 PX)

\textit{\textbf{Incantesimi.}} Il sacerdote ha CM 4. La sua abilità da incantatore è la Saggezza (+4 al colpire con attacchi con incantesimo). Il sacerdote ha preparato i seguenti incantesimi: Trucchetti (a volontà): \textit{arte druidica, bastone, produrre fiamma}

livello 1 (4 slot): \textit{intralciare, onda tonante, parlare con gli}
\textit{animali, passo veloce}

livello 2 (3 slot): \textit{animale messaggero, pelle di corteccia}

\textbf{Azioni}

\textit{\textbf{Bastone da Combattimento.} Attacco con Arma da Mischia}: +2 a colpire (+4 a colpire con \textit{bastone*}), portata 1 m o gittata 6m, un bersaglio.

\textit{Colpisce:} 3 (1d6) danni da botta, o 6 (1d8 + 2) danni da botta con \textit{bastone} o se impugnato con due mani.

\medskip\textbf{Esploratore}

Abili cacciatori e battitori di piste.

\textit{Media umanoide (qualsiasi razza), qualsiasi Tratto}

\textbf{FORZA} +0

\textbf{DESTREZZA} +2

\textbf{COSTITUZIONE} +1

\textbf{INTELLIGENZA} +0

\textbf{SAGGEZZA} +1

\textbf{CARISMA} +0

\textbf{Iniziativa} +2 -- \textbf{Difesa} 14 (armatura di cuoio)

\textbf{Punti Ferita} 16 (3d8 + 3)

\textbf{Movimento} 9 m

\textbf{Tiri Salvezza}: Tempra +1, Riflessi +2, Volontà +3

\textbf{Competenze} Muoversi Silenziosamente / Nascondersi +6, Natura +4, Consapevolezza +5, Sopravvivenza +5

\textbf{Lingue} una qualsiasi lingua (di solito Comune)

\textbf{Sfida} 1/2 (100 PX)

\textit{\textbf{Olfatto e Vista Affinati.}} L'esploratore ha +1d6 nelle prove di Saggezza (Consapevolezza) basate su olfatto o vista.

\textbf{Azioni}

\textit{\textbf{Multiattacco.}} L'esploratore effettua due attacchi da mischia o due attacchi a gittata.

\textit{\textbf{Spada Corta.} Attacco con Arma da Mischia}: +4 a colpire, portata 1 m, un bersaglio.

\textit{Colpisce:} 5 (1d6 + 2) danni perforanti.

\textit{\textbf{Arco Lungo.} Attacco con Arma da Mischia}: +4 a colpire, gittata 45m, un bersaglio.

\textit{Colpisce:} 6 (1d8 + 2) danni perforanti.


\end{multicols}

%{\scriptsize
%\printindex}
%\end{document}

\pagebreak

\subsection{Lista Mostri per Grado di Sfida}



\begin{multicols}{3}
{%\small
\flushleft{Aquila, Sfida 0 (10 PX)\\
Avvoltoio, Sfida 0 (10 PX)\\
Babbuino, Sfida 0 (10 PX)\\
Caprone, Sfida 0 (10 PX)\\
Cervo, Sfida 0 (10 PX)\\
Corvo, Sfida 0 (10 PX)\\
Donnola, Sfida 0 (10 PX)\\
Falco, Sfida 0 (10 PX)\\
Fungo Stridente, Sfida 0 (10 PX)\\
Gatto, Sfida 0 (10 PX)\\
Gufo, Sfida 0 (10 PX)\\
Iena, Sfida 0 (10 PX)\\
Lemure, Sfida 0 (10 PX)\\
Lucertola, Sfida 0 (10 PX)\\
Omuncolo, Sfida 0 (10 PX)\\
Pirana, Sfida 0 (10 PX)\\
Popolano, Sfida 0 (10 PX)\\
Ragno, Sfida 0 (10 PX)\\
Rana, Sfida 0 (10 PX)}\\
Ratto, Sfida 0 (10 PX)\\
Scarabeo di Fuoco Gigante, Sfida 0 (10 PX)\\
Sciacallo, Sfida 0 (10 PX)\\
Scorpione, Sfida 0 (10 PX)\\
Tasso, Sfida 0 (10 PX)\\
Topi, Sfida: 0 (10 PX)\\
Bandito/Pirata, Sfida 1/8 (25 PX)\\
Cammello, Sfida 1/8 (25 PX)\\
Coboldo, Sfida 1/8 (25 PX)\\
Cultista, Sfida 1/8 (25 PX)\\
Donnola Gigante, Sfida 1/8 (25 PX)\\
Falco di Sangue, Sfida 1/8 (25 PX)\\
Granchio Gigante, Sfida 1/8 (25 PX)\\
Guardia, Sfida 1/8 (25 PX)\\
Mastino, Sfida 1/8 (25 PX)\\
Mulo, Sfida 1/8 (25 PX)\\
Nobile, Sfida 1/8 (25 PX)\\
Pony, Sfida 1/8 (25 PX)\\
Ratto Gigante, Sfida 1/8 (25 PX)\\
Serpente Velenoso, Sfida 1/8 (25 PX)\\
Serpente Volante, Sfida 1/8 (25 PX)\\
Strige, Sfida 1/8 (25 PX)\\
Strige (Uccello Stigeo), Sfida 1/8 (25 PX)\\
Uomo Acquatico, Sfida 1/8 (25 PX)\\
Accolito, Sfida 1/4 (50 PX)\\
Alce, Sfida 1/4 (50 PX)\\
Becco d'Ascia, Sfida 1/4 (50 PX)\\
Cane Intermittente, Sfida 1/4 (50 PX)\\
Cavallo da Corsa, Sfida 1/4 (50 PX)\\
Cavallo da Tiro, Sfida 1/4 (50 PX)\\
Centopiedi Gigante, Sfida 1/4 (50 PX)\\
Cinghiale, Sfida 1/4 (50 PX)\\
Dretch, Sfida 1/4 (50 PX)\\
Fungo Violetto, Sfida 1/4 (50 PX)\\
Gablin, Sfida 1/4 (50 PX)\\
Grimlock, Sfida 1/4 (50 PX)\\
Gufo Gigante, Sfida 1/4 (50 PX)\\
Lucertola Gigante, Sfida 1/4 (50 PX)\\
Lupo, Sfida 1/4 (50 PX)\\
Mefito di Vapore, Sfida 1/4 (50 PX)\\
Pantera, Sfida 1/4 (50 PX)\\
Pseudodrago, Sfida 1/4 (50 PX)\\
Ragno Lupo Gigante  , Sfida 1/4 (50 PX)\\
Rana Gigante  , Sfida 1/4 (50 PX)\\
Scheletro, Sfida 1/4 (50 PX)\\
Sciame di Corvi  , Sfida 1/4 (50 PX)\\
Sciame di Pipistrelli  , Sfida 1/4 (50 PX)\\
Sciame di Ratti, Sfida 1/4 (50 PX)\\
Serpente Costrittore, Sfida 1/4 (50 PX)\\
Serpente Velenoso Gigante, Sfida 1/4 (50 PX)\\
Spada Volante, Sfida 1/4 (50 PX)\\
Spiritello, Sfida 1/4 (50 PX)\\
Tasso Gigante, Sfida 1/4 (50 PX)\\
Zombi, Sfida 1/4 (50 PX)\\
Caprone Gigante, Sfida 1/2 (100 PX)\\
Cavallo da Guerra, Sfida 1/2 (100 PX)\\
Cavallo Marino Gigante, Sfida 1/2 (100 PX)\\
Coccodrillo, Sfida 1/2 (100 PX)\\
Cockatrice, Sfida 1/2 (100 PX)\\
Esploratore, Sfida 1/2 (100 PX)\\
Gnoll, Sfida 1/2 (100 PX)\\
Gnomo delle Profondità, Sfida 1/2 (100 PX)\\
Hobgoblin, Sfida 1/2 (100 PX)\\
Lucertoloide, Sfida 1/2 (100 PX)\\
Mantoscuro, Sfida 1/2 (100 PX)\\
Mefito di Ghiaccio, Sfida 1/2 (100 PX)\\
Mefito di Magma, Sfida 1/2 (100 PX)\\
Mefito di Polvere, Sfida 1/2 (100 PX)\\
Melma Grigia, Sfida 1/2 (100 PX)\\
Ombra, Sfida 1/2 (100 PX)\\
Orchetto, Sfida 1/2 (100 PX)\\
Orso Nero, Sfida 1/2 (100 PX)\\
Picchiatore, Sfida 1/2 (100 PX)\\
Rugginofago, Sfida 1/2 (100 PX)\\
Sahuagin, Sfida 1/2 (100 PX)\\
Satiro, Sfida 1/2 (100 PX)\\
Scheletro di Cavallo da Guerra, Sfida 1/2 (100 PX)\\
Sciame di Insetti, Sfida 1/2 (100 PX)\\
Sciame di Ragni, Sfida 1/2 (100 PX)\\
Sciame di Scarabei  , Sfida 1/2 (100 PX)\\
Sciame di Vespe, Sfida 1/2 (100 PX)\\
Sciami, Sfida 1/2 (100 PX)\\
Scimmione, Sfida 1/2 (100 PX)\\
Squalo Corallino, Sfida 1/2 (100 PX)\\
Uomo Magma (Magmin), Sfida 1/2 (100 PX)\\
Vespa Gigante, Sfida 1/2 (100 PX)\\
Worg, Sfida 1/2 (100 PX)\\
Aquila Gigante, Sfida 1 (200 PX)\\
Armatura Animata, Sfida 1 (200 PX)\\
Arpia, Sfida 1 (200 PX)\\
Avvoltoio Gigante, Sfida 1 (200 PX)\\
Bugbear, Sfida 1 (200 PX)\\
Cane della Morte, Sfida 1 (200 PX)\\
Dinolupo (Metalupo), Sfida 1 (200 PX)\\
Drago d'Ottone Cucciolo, Sfida 1 (200 PX)\\
Drago di Rame Cucciolo, Sfida 1 (200 PX)\\
Driade, Sfida 1 (200 PX)\\
Duergar, Sfida 1 (200 PX)\\
Ghoul, Sfida 1 (200 PX)\\
Globulo, Sfida 1 (200 PX)\\
Iena Gigante, Sfida 1 (200 PX)\\
Imp, Sfida 1 (200 PX)\\
Ippogrifo, Sfida 1 (200 PX)\\
Leone  , Sfida 1 (200 PX)\\
Mago Avventuriero, Sfida 1 (200 PX)\\
Orco, Sfida 1 (100 PX)\\
Orso Bruno, Sfida 1 (200 PX)\\
Quasit, Sfida 1 (200 PX)\\
Ragno Gigante, Sfida 1 (200 PX)\\
Rospo Gigante, Sfida 1 (200 PX)\\
Sciame di Pirana, Sfida 1 (200 PX)\\
Spia, Sfida 1 (200 PX)\\
Tigre, Sfida 1 (200 PX)\\
Albero Risvegliato, Sfida 2 (450 PX)\\
Alce Gigante, Sfida 2 (450 PX)\\
Ameba Paglierina, Sfida 2 (450 PX)\\
Ankheg, Sfida 2 (450 PX)\\
Azer, Sfida 2 (450 PX)\\
Berserker, Sfida 2 (450 PX)\\
Blatta Esplosiva, Sfida 2 (450 PX)\\
Capitano dei Banditi/Pirata, Sfida 2 (450 PX)\\
Centauro, Sfida 2 (450 PX)\\
Cinghiale Gigante, Sfida 2 (450 PX)\\
Cubo Gelatinoso, Sfida 2 (450 PX)\\
Diavolo Spinoso, Sfida 2 (450 PX)\\
Drago Bianco Cucciolo, Sfida 2 (450 PX)\\
Drago d'Argento Cucciolo, Sfida 2 (450 PX)\\
Drago di Bronzo Cucciolo, Sfida 2 (450 PX)\\
Drago Nero Cucciolo, Sfida 2 (450 PX)\\
Drago Verde Cucciolo, Sfida 2 (450 PX)\\
Druido, Sfida 2 (450 PX)\\
Elementale dell'Acqua Minore, Sfida 2 (450 PX)\\
Ettercap, Sfida 2 (450 PX)\\
Fauci Gorgoglianti, Sfida 2 (450 PX)\\
Fuoco Fatuo, Sfida 2 (450 PX)\\
Gargoyle, Sfida 2 (450 PX)\\
Ghast, Sfida 2 (450 PX)\\
Grick, Sfida 2 (450 PX)\\
Grifone, Sfida 2 (450 PX)\\
Megera Marina, Sfida 2 (450 PX)\\
Mimic, Sfida 2 (450 PX)\\
Ogre, Sfida 2 (450 PX)\\
Orso Polare, Sfida 2 (450 PX)\\
Pegaso, Sfida 2 (450 PX)\\
Plesiosauro, Sfida 2 (450 PX)\\
Ratto Mannaro, Sfida 2 (450 PX)\\
Rinoceronte, Sfida 2 (450 PX)\\
Sacerdote, Sfida 2 (450 PX)\\
Scheletro di Minotauro, Sfida 2 (450 PX)\\
Sciame di Serpenti Velenosi, Sfida 2 (450 PX)\\
Serpente Costrittore Gigante, Sfida 2 (450 PX)\\
Sibilante, Sfida 2 (450 PX)\\
Squalo Cacciatore, Sfida 2 (450 PX)\\
Tappeto del Soffocamento, Sfida 2 (450 PX)\\
Teschio Fiammeggiante, Sfida 2 (200 PX)\\
Tigre dai Denti a Sciabola, Sfida 2 (450 PX)\\
Zombi Ogre, Sfida 2 (450 PX)\\
Balena Assassina (Orca), Sfida 3 (700 PX)\\
Basilisco, Sfida 3 (700 PX)\\
Campione Gablin, Sfida 3 (700)\\
Cavaliere, Sfida 3 (700 PX)\\
Destriero da Incubo, Sfida 3 (700 PX)\\
Diavolo Barbuto, Sfida 3 (700 PX)\\
Doppelganger, Sfida 3 (700 PX)\\
Drago Blu Cucciolo, Sfida 3 (700 PX)\\
Drago d'Oro Cucciolo, Sfida 3 (700 PX)\\
Lupo Invernale, Sfida 3 (700 PX)\\
Lupo Mannaro, Sfida 3 (700 PX)\\
Manticora, Sfida 3 (700 PX)\\
Megera Verde, Sfida 3 (700 PX)\\
Minotauro, Sfida 3 (700 PX)\\
Mummia, Sfida 3 (700 PX)\\
Orrore Arrampicamuri, Sfida 3 (700 PX)\\
Orsogufo, Sfida 3 (700 PX)\\
Orsogufo Saggio, Sfida 3 (700 PX)\\
Ragno Fase, Sfida 3 (700 PX)\\
Scorpione Gigante, Sfida 3 (700 PX)\\
Segugio Infernale, Sfida 3 (700 PX)\\
Veterano, Sfida 3 (700 PX)\\
Wight, Sfida 3 (700 PX)\\
B.O.C., Sfida 4 (1.100 PX)\\
Banshee, Sfida 4 (1.100 PX)\\
Chuul, Sfida 4 (1.100 PX)\\
Cinghiale Mannaro, Sfida 4 (1.100 PX)\\
Couatl, Sfida 4 (1.100 PX)\\
Drago Rosso Cucciolo, Sfida 4 (1.100 PX)\\
Elefante, Sfida 4 (1.100 PX)\\
Ettin, Sfida 4 (1.100 PX)\\
Fantasma, Sfida 4 (1.100 PX)\\
Ghoul Putrescente, Sfida 4 (1.100 PX)\\
Lamia, Sfida 4 (1.100 PX)\\
Maledetti immortale, Sfida 4 (1.100 PX)\\
Protoplasma Nero, Sfida 4 (1.100 PX)\\
Succube, Sfida 4 (1.100 PX)\\
Tigre Mannara, Sfida 4 (1.100 PX)\\
Torciascura, Sfida 4 (1.100 PX)\\
Verme Strisciante Tentacolato, Sfida 4 (1.100 PX)\\
Bulette, Sfida 5 (1.800 PX)\\
Coccodrillo Gigante, Sfida 5 (1.800 PX)\\
Cumulo Strisciante, Sfida 5 (1.800 PX)\\
Elementale del Fuoco, Sfida 5 (1.800 PX)\\
Elementale dell'Acqua, Sfida 5 (1.800 PX)\\
Elementale dell'Aria, Sfida 5 (1.800 PX)\\
Elementale della Terra, Sfida 5 (1.800 PX)\\
Fustigatore, Sfida 5 (1.800 PX)\\
Ghoul, Madre, Sfida 5 (1.800 PX)\\
Gigante di Collina, Sfida 5 (1.800 PX)\\
Gladiatore, Sfida 5 (1.800 PX)\\
Golem di Carne, Sfida 5 (1.800 PX)\\
Gorgone, Sfida 5 (1.800 PX)\\
Megera Notturna, Sfida 5 (1.800 PX)\\
Orso Mannaro, Sfida 5 (1.800 PX)\\
Otyugh, Sfida 5 (1.800 PX)\\
Salamandra, Sfida 5 (1.800 PX)\\
Squalo Gigante, Sfida 5 (1.800 PX)\\
Triceratopo, Sfida 5 (1.800 PX)\\
Troll, Sfida 5 (1.800 PX)\\
Unicorno, Sfida 5 (1.800 PX)\\
Wraith, Sfida 5 (1.800 PX)\\
Xorn, Sfida 5 (1.800 PX)\\
Chimera, Sfida 6 (2.300 PX)\\
Drago Bianco Giovane, Sfida 6 (2.300 PX)\\
Drago d'Ottone Giovane, Sfida 6 (2.300 PX)\\
Drider, Sfida 6 (2.300 PX)\\
Ghoul, Nero, Sfida 6 (2.300 PX)\\
Grande Mago, Sfida 6 (2.300 PX)\\
Mammut, Sfida 6 (2.300 PX)\\
Medusa, Sfida 6 (2.300 PX)\\
Paladino Gablin, Sfida 6 (2.300 PX)\\
Persecutore Invisibile, Sfida 6 (2.300 PX)\\
Progenie Vampirica  , Sfida 6 (1.800 PX)\\
Viverna, Sfida 6 (2.300 PX)\\
Vrock, Sfida 6 (2.300 PX)\\
Drago di Rame Giovane, Sfida 7 (2.900 PX)\\
Drago Nero Giovane, Sfida 7 (2.900 PX)\\
Gigante di Pietra, Sfida 7 (2.900 PX)\\
Guardiano Protettore, Sfida 7 (2.900 PX)\\
Oni, Sfida 7 (2.900 PX)\\
Scimmione Gigante, Sfida 7 (2.900 PX)\\
Assassino, Sfida 8 (3.900 PX)\\
Diavolo delle Catene , Sfida 8 (3.900 PX)\\
Drago di Bronzo Giovane, Sfida 8 (3.900 PX)\\
Drago Verde Giovane, Sfida 8 (3.900 PX)\\
Gigante del Gelo, Sfida 8 (3.900 PX)\\
Hezrou, Sfida 8 (3.900 PX)\\
Idra, Sfida 8 (3.900 PX)\\
Manto Assassino, Sfida 8 (3.900 PX)\\
Naga Spirituale, Sfida 8 (3.900 PX)\\
Tirannosauro, Sfida 8 (3.900 PX)\\
Diavolo d'Ossa, Sfida 9 (5000 PX)\\
Divora Cervelli, Sfida 9 (5000 PX)\\
Drago Blu Giovane, Sfida 9 (5000 PX)\\
Drago d'Argento Giovane, Sfida 9 (5000 PX)\\
Elementale dell'Acqua Maggiore, Sfida 9 (5000 PX)\\
Gigante del Fuoco, Sfida 9 (5000 PX)\\
Gigante delle Nuvole, Sfida 9 (5000 PX)\\
Glabrezu, Sfida 9 (5000 PX)\\
Golem di Argilla, Sfida 9 (5000 PX)\\
Uomo Albero (Treant), Sfida 9 (5000 PX)\\
Aboleth, Sfida 10 (5.900 PX)\\
Angelo Deva, Sfida 10 (5.900 PX)\\
Drago d'Oro Giovane, Sfida 10 (5.900 PX)\\
Drago Rosso Giovane, Sfida 10 (5.900 PX)\\
G.E.C., Sfida 10 (5.900 PX)\\
Golem di Pietra, Sfida 10 (5.900 PX)\\
Naga Guardiano, Sfida 10 (5.900 PX)\\
Behir, Sfida 11 (7.200 PX)\\
Diavolo Cornuto, Sfida 11 (7.200 PX)\\
Djinni, Sfida 11 (7.200 PX)\\
Efreeti, Sfida 11 (7.200 PX)\\
Ginosfinge, Sfida 11 (7.200 PX)\\
Remorhaz, Sfida 11 (7.200 PX)\\
Arcimago, Sfida 12 (8.400 PX)\\
Erinni, Sfida 12 (8.400 PX)\\
Panopticon, Sfida 12 (8.400 PX)\\
Drago Bianco Adulto, Sfida 13 (10000 PX)\\
Drago d'Ottone Adulto, Sfida 13 (10000 PX)\\
Gigante delle Tempeste, Sfida 13 (10000 PX)\\
Nalfeshnee, Sfida 13 (10000 PX)\\
Rakshasa, Sfida 13 (10000 PX)\\
Vampiro, Sfida 13 (10000 PX)\\
Diavolo del Ghiaccio, Sfida 14 (11.500 PX)\\
Drago di Rame Adulto, Sfida 14 (11.500 PX)\\
Drago di Bronzo Adulto, Sfida 15 (13000 PX)\\
Drago Verde Adulto, Sfida 15 (13000 PX)\\
Fenice, Sfida 15 (13000 PX)\\
Mummia Sovrana, Sfida 15 (13000 PX)\\
Verme Purpureo, Sfida 15 (13000 PX)\\
Angelo Planetar, Sfida 16 (15000 PX)\\
Drago Blu Adulto, Sfida 16 (15000 PX)\\
Drago d'Argento Adulto, Sfida 16 (15000 PX)\\
Golem di Ferro, Sfida 16 (15000 PX)\\
Marilith, Sfida 16 (15000 PX)\\
Androsfinge, Sfida 17 (18000 PX)\\
Drago d'Oro Adulto, Sfida 17 (18000 PX)\\
Drago Nero Adulto, Sfida 17 (18000 PX)\\
Drago Rosso Adulto, Sfida 17 (18000 PX)\\
Testuggine Dragona, Sfida 17 (18000 PX)\\
Cavaliere Nero, Sfida 18 (20000 PX)\\
Balor, Sfida 19 (22000 PX)\\
Diavolo della Fossa, Sfida 20 (25000 PX)\\
Drago Bianco Antico, Sfida 20 (25000 PX)\\
Drago d'Ottone Antico, Sfida 20 (25000 PX)\\
Angelo Solar, Sfida 21 (33000 PX)\\
Drago di Rame Antico, Sfida 21 (33000 PX)\\
Drago Nero Antico, Sfida 21 (33000 PX)\\
Lich, Sfida 21 (33000 PX)\\
Drago di Bronzo Antico, Sfida 22 (41000 PX)\\
Drago Verde Antico, Sfida 22 (41000 PX)\\
Drago Blu Antico, Sfida 23 (50000 PX)\\
Drago d'Argento Antico, Sfida 23 (50000 PX)\\
Drago Giallo Antico, Sfida: 23 (50000 PX)\\
Kraken, Sfida 23 (50000 PX)\\
Drago d'Oro Antico, Sfida 24 (62000 PX)\\
Drago Rosso Antico, Sfida 24 (62000 PX)\\
Demogorgone, Sfida 26 (90000 PX)\\
Orcus, Sfida 26 (90000 PX)\\
Tàhil, Sfida 30 (155000 PX)\\
Tarrasque, Sfida 30 (155000 PX)\\
}

\end{multicols}