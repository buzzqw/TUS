\documentclass[a4paper,twoside,openany]{book}

% ========== CONFIGURAZIONE GEOMETRIA ==========
\usepackage[a4paper,margin=2cm]{geometry}

% ========== PACCHETTI LINGUISTICI E CODIFICA ==========
\usepackage[italian]{babel}
\usepackage[italian=guillemets]{csquotes}
\usepackage{hyphenat}
\hyphenation{
	mate-mati-ca
	re-cu-pe-ra-re
	com-pe-ten-za
	intel-li-gen-za
	cara-tte-ri-sti-che
	abi-li-tà
	in-can-te-si-mo
	at-tac-co
	di-fe-sa
	per-so-nag-gio
	av-ven-tu-ra
	espe-rien-za
	equi-pag-gia-men-to
}

% ========== PACCHETTI GRAFICI E COLORI ==========
\usepackage{xcolor}

% Definizione colori del tema OBSS
\definecolor{OBSSnavyblue}{RGB}{0,0,128}
\definecolor{OBSSprussia}{RGB}{0, 49, 83}
\definecolor{OBSSblue}{RGB}{30,84,131}
\definecolor{OBSSgold}{RGB}{255,215,0}
\definecolor{OBSSpurple}{RGB}{88,57,131}
\definecolor{lightgray}{gray}{0.95}
\definecolor{LightCyan}{RGB}{0,191,255}
\definecolor{SlateGray}{RGB}{112,128,144}
\definecolor{Lavender}{RGB}{230,230,250}
\definecolor{Coral}{RGB}{255,127,80}
\definecolor{Tan}{RGB}{210,180,140}
\definecolor{LightGreen}{RGB}{34,139,34}
\definecolor{Mauve}{RGB}{221,160,221}

% ========== PACCHETTI TIPOGRAFICI ==========
\usepackage{fontspec}
\setmainfont{AtkinsonHyperlegibleNext-Regular.ttf}[
Path=./fonts/,
BoldFont=AtkinsonHyperlegibleNext-Bold.ttf,
ItalicFont=AtkinsonHyperlegibleNext-RegularItalic.ttf,
BoldItalicFont=AtkinsonHyperlegibleNext-BoldItalic.ttf,
Ligatures=TeX
]

\usepackage{microtype}
% microtype migliora la qualità tipografica

% ========== PACCHETTI LAYOUT E SPAZIATURA ==========
\usepackage{setspace}
\usepackage{ragged2e}
\usepackage{quoting,enumitem}

% ========== PACCHETTI MATEMATICI E SCIENTIFICI ==========
\usepackage{amssymb,siunitx}

% ========== PACCHETTI TABELLE ==========
\usepackage{booktabs,multirow}
\usepackage{xltabular,tabularx}
\usepackage{colortbl}
\usepackage{hhline}
\usepackage{multicol,array}

% ========== PACCHETTI GRAFICI E FIGURE ==========
\usepackage{graphics,adjustbox,svg}
\usepackage{caption}
\usepackage{wrapfig}
\usepackage{tikz}
\usetikzlibrary{shadows,shapes.misc,calc}

% ========== PACCHETTI BOX E CORNICI ==========
\usepackage{tcolorbox}
\tcbuselibrary{skins,breakable,theorems}
\usepackage[framemethod=TikZ]{mdframed}

% ========== PACCHETTI BIBLIOGRAFIA E INDICI ==========
\usepackage[backend=bibtex]{biblatex}
\addbibresource{bibliografia.bib}
\usepackage{makeidx,imakeidx}

% ========== PACCHETTI VARI ==========
\usepackage{soul}
\usepackage{pdfpages}
\usepackage{etoolbox}
\usepackage[absolute,overlay]{textpos}
\usepackage{url}
\def\UrlBreaks{\do\/\do-\do_\do.\do:\do=\do&}
\usepackage{varioref}
\usepackage{fontawesome5}


% ========== CONFIGURAZIONI GENERALI ==========
\setcounter{secnumdepth}{-1}
\setcounter{tocdepth}{3}
\raggedbottom
\sloppy
\DeclareMathSizes{11}{11}{8}{6}

% Configurazione XeTeX per l'italiano
\XeTeXlinebreaklocale "it"
\XeTeXlinebreakskip = 0pt plus 1pt

% ========== CONFIGURAZIONE TCOLORBOX ==========
\tcbset{colback=brown!10, fonttitle=\scshape}

% ========== CONFIGURAZIONE INDICI ==========
% Macro per contare le occorrenze negli indici
\def\CountIndexOccurrences#1{%
	\expandafter\newcount\csname #1\endcsname%
	\def\indexentry##1##2{\expandafter\advance\csname #1\endcsname 1}%
	\IfFileExists{#1.idx}{\input{#1.idx}}{}%
}

% Inizializzazione contatori indici
\CountIndexOccurrences{OBSSv2}
\CountIndexOccurrences{Incantesimi}
\CountIndexOccurrences{Mostruario}
\CountIndexOccurrences{Tabelle}
\CountIndexOccurrences{OggettiMagici}
\CountIndexOccurrences{Abilita}

% Macro per visualizzare il totale delle voci
\def\TotalBox#1{\vfill%
	\fbox{Ci sono \expandafter\the\csname #1\endcsname\ voci in questo indice}\par}

% Creazione degli indici
\makeindex[columns=1, title=Indice Analitico, intoc=true]
\makeindex[columns=1, name=Tabelle, title=Elenco delle Tabelle, intoc=true]
\makeindex[columns=1, name=Incantesimi, title=Elenco degli Incantesimi, intoc=true]
\makeindex[columns=1, name=Mostruario, title=Elenco dei Mostri, intoc=true]
\makeindex[columns=1, name=OggettiMagici, title=Elenco degli Oggetti Magici, intoc=true]
\makeindex[columns=1, name=Abilita, title=Elenco delle Abilità, intoc=true]

% ========== CONFIGURAZIONE INTESTAZIONI ==========

\usepackage{fancyhdr}
\pagestyle{fancy}
\fancyhf{}
\fancyhead[LE,RO]{\leftmark}
\fancyfoot[C]{\thepage}
\renewcommand{\sectionmark}[1]{\markboth{#1}{}}

\fancypagestyle{plain}{%
	\fancyhf{}
	\fancyhead[RO]{%
		\rotatebox{90}{
			\begin{tikzpicture}[overlay,remember picture]
				\node[
				fill=lightgray,
				text=black,
				font=\footnotesize,
				inner ysep=12pt,
				inner xsep=20pt,
				rounded rectangle,
				anchor=east,
				minimum width=7cm,
				xshift=-60mm,
				yshift=-21mm,
				text height=0.4cm
				] at ($ (current page.north east) + (-1cm,-0cm) + (-4*\thesection cm,0cm) $)
				{\sffamily\itshape\small\nouppercase{\leftmark}};
			\end{tikzpicture}
		}
	}
	\fancyhead[LE]{%
		\rotatebox{90}{
			\begin{tikzpicture}[overlay,remember picture]
				\node[
				fill=lightgray,
				text=black,
				font=\footnotesize,
				inner ysep=12pt,
				inner xsep=20pt,
				rounded rectangle,
				anchor=east,
				minimum width=7cm,
				xshift=-60mm,
				yshift=-4mm,
				text height=0.4cm
				] at ($ (current page.north west) + (1cm,0cm) + (-4*\thesection cm,0cm) $)
				{\sffamily\itshape\small\nouppercase{\leftmark}};
			\end{tikzpicture}
		}
	}
	\renewcommand{\headrulewidth}{0pt}
	\renewcommand{\footrulewidth}{0pt}
	\fancyfoot[C]{\thepage}
}
\pagestyle{plain}

% ----- DEFINIZIONI PER MARGINI -----
\def\changemargin#1#2{\list{}{\rightmargin#2\leftmargin#1}\item[]}
\let\endchangemargin=\endlist

% ========== CONFIGURAZIONE TITOLI ==========
\usepackage{titlesec}

\titleformat{\chapter}[display]
{\normalfont\Huge\bfseries\sffamily\color{OBSSblue}}
{\textcolor{OBSSgold}{\chaptertitlename\ \thechapter}}
{2ex}
{\centering}

\titleformat{\section}
{\normalfont\Large\bfseries\sffamily\color{OBSSblue}}
{\thesection}
{1em}
{}
[\textcolor{OBSSgold}{\titlerule[0.8pt]}]

\titleformat{\subsection}
{\normalfont\large\bfseries\sffamily\color{OBSSblue}}
{\thesubsection}
{1em}
{}

\titlespacing{\chapter}{0pt}{0pt}{20pt}
\titlespacing{\section}{0pt}{10pt}{5pt}
\titlespacing{\subsection}{0pt}{8pt}{4pt}
\titlespacing{\subsubsection}{0pt}{6pt}{3pt}

% ========== AMBIENTI PERSONALIZZATI ==========
\newtcolorbox{narratore}[1][Narratore]{
	enhanced,
	colback=OBSSblue!8,
	colframe=OBSSblue,
	boxrule=1.2pt,
	arc=4mm,
	drop shadow={shadow xshift=1mm, shadow yshift=-1mm, opacity=0.4, shadow scale=1.05},
	title={\textbf{\textcolor{white}{\large \faBook} #1}},
	fonttitle=\sffamily\bfseries\large,
	colbacktitle=OBSSblue,
	coltitle=white,
	attach boxed title to top left={yshift=-2.5mm, xshift=3mm},
	boxed title style={boxrule=0pt, colframe=white, arc=4mm,
		drop shadow={shadow xshift=0.3mm, shadow yshift=-0.3mm, opacity=0.2}
	},
	left=3mm, right=3mm, top=4mm, bottom=2mm,
	before skip=\baselineskip, after skip=\baselineskip,
}

\newtcolorbox{giocatore}[1][Giocatore]{
	enhanced,
	breakable,
	colback=OBSSpurple!8,
	colframe=OBSSpurple,
	boxrule=1.2pt,
	arc=4mm,
	drop shadow={shadow xshift=1mm, shadow yshift=-1mm, opacity=0.4, shadow scale=1.05},
	title={\textbf{\textcolor{white}{\large $\dagger$} #1}},
	fonttitle=\sffamily\bfseries\large,
	colbacktitle=OBSSpurple,
	coltitle=white,
	attach boxed title to top left={yshift=-2.5mm, xshift=3mm},
	boxed title style={boxrule=0pt, colframe=white, arc=4mm,
		drop shadow={shadow xshift=0.3mm, shadow yshift=-0.3mm, opacity=0.2}
	},
	left=3mm, right=3mm, top=4mm, bottom=2mm,
	before skip=\baselineskip, after skip=\baselineskip,
}

\newtcolorbox{enfasi}{
	enhanced,
	colback=OBSSgold!8,
	colframe=OBSSgold,
	boxrule=1.2pt,
	arc=4mm,
	drop shadow={shadow xshift=0.3mm, shadow yshift=-0.3mm, opacity=0.2},
	left=3mm, right=3mm, top=2mm, bottom=2mm,
	before upper={\hyphenpenalty=10000\exhyphenpenalty=10000},
	before skip=\baselineskip, after skip=\baselineskip,
}

% ========== COMANDI PERSONALIZZATI ==========
\newcommand{\OBSSseparator}{%
	\begin{center}%
		\begin{tikzpicture}%
			\draw[OBSSgold, line width=1pt] (-2,0) -- (-0.5,0);%
			\node[circle, fill=OBSSgold, minimum size=4mm] at (0,0) {};%
			\draw[OBSSgold, line width=1pt] (0.5,0) -- (2,0);%
		\end{tikzpicture}%
	\end{center}%
}

\newcommand{\FatePoint}{\textcolor{OBSSgold}{$\bullet$}}
\newcommand{\FatePoints}[1]{%
	\textcolor{OBSSgold}{%
		\foreach \n in {1,...,#1}{$\bullet$\,}%
	}%
}

%\NewDocumentCommand{\feat}{m}{
%	\noindent\rule{\linewidth}{2pt}
%	\index[Abilita]{#1}\hypertarget{#1}{}\label{#1}
%	\vspace{-5.5mm}
%	\begin{tcolorbox}[
%%		colback=OBSSgold!12, colframe=OBSSgold,
%		boxrule=1.5pt, arc=3pt, boxsep=1pt,
%		left=6pt, right=6pt, top=3pt, bottom=3pt
%		]
%		\centering\textbf{\textcolor{OBSSgold!80!black}{#1}}
%	\end{tcolorbox}
%	\pdfbookmark[3]{#1}{#1}
%	\vspace{1mm}\noindent
%}

\NewDocumentCommand{\feat}{m}{
	\noindent\rule{\linewidth}{2pt}
	\index[Abilita]{#1}\hypertarget{#1}{}\label{#1}
	\vspace{-5.5mm}
	\begin{center}\textbf{\textcolor{OBSSgold!90!black}{#1}}\end{center}
	\pdfbookmark[3]{#1}{#1}
	\vspace{-1mm}\noindent
}

\NewDocumentCommand{\incantesimo}{m}{
	\noindent\rule{\linewidth}{2pt}
	\index[Incantesimi]{#1}\hypertarget{#1}{}\label{#1}
	\vspace{-5.5mm}
	\begin{center}\textbf{\textcolor{OBSSgold!90!black}{#1}}\end{center}
	\pdfbookmark[3]{#1}{#1}
	\vspace{-1mm}\noindent
}

\NewDocumentCommand{\mostro}{m}{
	\noindent\rule{\linewidth}{2pt}
	\index[Mostruario]{#1}\hypertarget{#1}{}\label{#1}
	\vspace{-5.5mm}
	\begin{center}\textbf{\textcolor{OBSSgold!90!black}{#1}}\end{center}
	\pdfbookmark[3]{#1}{#1}
	\vspace{-1mm}\noindent
}

\NewDocumentCommand{\oggettomagico}{m}{
	\noindent\rule{\linewidth}{2pt}
	\index[OggettiMagici]{#1}\hypertarget{#1}{}\label{#1}
	\vspace{-5.5mm}
	\begin{center}\textbf{\textcolor{OBSSgold!80!black}{#1}}\end{center}
	\pdfbookmark[3]{#1}{#1}
	\vspace{-1mm}\noindent
}



% ========== SISTEMA TABELLE COLORATE ==========
\newcommand{\obsscurrentcolor}{Coral}
\newcommand{\obsssetcolor}[1]{\renewcommand{\obsscurrentcolor}{#1}}

% Comandi per cambiare colore
\newcommand{\obssgrey}{\obsssetcolor{SlateGray}}
\newcommand{\obsspurple}{\obsssetcolor{Mauve}}
\newcommand{\obssblue}{\obsssetcolor{LightCyan}}
\newcommand{\obssgreen}{\obsssetcolor{LightGreen}}
\newcommand{\obsscoral}{\obsssetcolor{Coral}}


% ========== HYPERREF E BOOKMARK ==========
% Importante: hyperref deve essere caricato per ultimo (quasi)
\usepackage[
unicode,
pdfencoding=auto,
bookmarks=true,
colorlinks=true,
linkcolor=OBSSblue,
citecolor=OBSSpurple,
urlcolor=OBSSblue,
filecolor=OBSSblue,
anchorcolor=OBSSblue,
pdftitle={OBSS - Old Bell School System},
pdfsubject={Gioco di Ruolo Fantasy},
pdfauthor={Andres Zanzani},
pdfkeywords={GDR, Fantasy, Gioco di Ruolo, OBSS},
pdfcreator={XeLaTeX},
pdfproducer={TexStudio on Debian},
pdfborder={0 0 0},
breaklinks=true,
bookmarksopenlevel=1,
bookmarksnumbered=true,
bookmarksopen=true
]{hyperref}


\usepackage{bookmark}
% bookmark deve essere caricato dopo hyperref

% ========== INIZIO DOCUMENTO ==========
\begin{document}

\def \versione {1.0.4} \fontsize{11}{11.5}\selectfont  %la prima e' la grandezza del font, la seconda la spaziatura tra riga e riga  %\def \versione {0.99.36} \fontsize{10.5}{12}\selectfont


\cleardoublepage \thispagestyle{empty} \tikz[remember picture,overlay] \node[opacity=1] at (current page.center){\includegraphics[width=21cm,height=\paperheight]{immagini/copertina2-ai.png}}; \begin{textblock*}{20cm}(2cm,8cm)\Huge {\textbf{Old Bell School System}}\medskip \end{textblock*} \begin{textblock*}{22cm}(3.5cm,9cm) \Large {\textbf{(\textbf{OBSS})}}\medskip \end{textblock*} \begin{textblock*}{13cm}(9cm,15cm) \Huge{\color{black} \Huge{Fantasy Adventure Game}} \end{textblock*} \newpage~\thispagestyle{empty} \newpage~\thispagestyle{empty} %viene cercato Fantasy Adventure game


\bigskip
Dedicato all'unica Donna mai amata, colei che ogni giorno mi accompagna nei sogni

\bigskip

Mai rinunciare ai tuoi desideri, persevera fino a renderli reali.

\vfill

\begin{center}\textbf{\versione} -- \today\end{center}

\vspace{1cm}


\begin{enfasi}
Non temere l'ignoto, affrontalo con rispetto.
\end{enfasi}

\thispagestyle{empty}

\newpage~\thispagestyle{empty}%%\newpage~\thispagestyle{empty}

\pagebreak

{\Huge \begin{center} Old Bell School System \end{center}}

\bigskip

\begin{center}{\LARGE Manuale per Giocatore e Narratore}\\ \end{center}

{\large \begin{center} Guida e Regole per il Gioco di Ruolo Fantasy \end{center}}

\begin{center}di \end{center}

{\LARGE \begin{center} Andres Zanzani \end{center}}

\vspace{2cm}

\begin{center}
\includegraphics[keepaspectratio,width=0.50\textwidth]{immagini/phoenix-ai.png}
\end{center}

\vfill

\begin{mdframed}[roundcorner=10pt]

\medskip

\thispagestyle{empty}

\textbf{Playtesting}: Federica Angeli, Simona Bissi, Fabrizio Bonetti, Francesco Converso, Carlo Dall'Ara, Lucia Dolcini, Michele Faedi, Dario Galassi, Micaela Gramellini, Stefano Mannino, Samuele Mazzotti, Emanuele Pezzi, Leonardo Pezzi, Gian Luca Fava, Nicola Ricottone, Marco Valmori, Edoardo Zanzani, Isotta Zanzani, SicuramenteNonMirko,

\bigskip

\begin{flushleft}\textbf{Condizioni d'uso}: OBSS, Old Bell School System, è un marchio registrato di Andres Zanzani (azanzani@gmail.com), licenziato con Attribution-ShareAlike 4.0 International (CC BY-SA 4.0). Consultare i dettagli nel capitolo \hyperlink{Licenza}{Licenza}.
\end{flushleft}

\vspace{0.5cm}

\begin{center}
\includegraphics[keepaspectratio,width=0.4\textwidth]{immagini/CC_BY-SA_icon.svg.png}
\end{center}

\medskip

\end{mdframed}

%\cleardoublepage

\pagebreak ~

\pagenumbering{gobble}
\thispagestyle{empty}

\pagebreak ~

\setcounter{page}{1}

\pagenumbering{Roman}

\setcounter{page}{1}

\begin{multicols}{2}

{\small \tableofcontents{}}

\end{multicols}

\vfill

\begin{enfasi}
"May you make all your Saving Throws!" Frank Mentzer, Spring 1985. Master Player's Book
\end{enfasi}

\pagebreak~

\thispagestyle{empty}

\pagebreak~

%\cleardoublepage
\setcounter{page}{1}
\pagenumbering{arabic}

\begin{enfasi}
Il fatto che gli uomini non imparino molto dalla storia è la lezione più importante che la storia ci insegna. (Aldous Huxley)
\end{enfasi}

\section{La storia fino ad adesso...}

\begin{multicols}{2}

Il mondo come lo conoscevamo è un ricordo sbiadito, una tela lacerata da cataclismi e dalla furia degli dei. Leggende, miti e fantasia si sono intrecciati in un guazzabuglio cacofonico con la realtà dei fatti.

Da qualche parte nel terzo millennio del vecchio calendario, avvenne l’impensabile: ciò che mai si sarebbe potuto immaginare o desiderare. Da un giorno all’altro, la Terra si trovò coinvolta in una guerra tra entità di potenza divina, che, con la complicità delle varie nazioni, non fecero altro che distruggere il nostro povero mondo.

La \emph{Freten} era un'azienda che sviluppava sistemi energetici alternativi, basati sulla possibilità di attingere energia da altrove o, come dicevano loro, dal vuoto cosmico.
Non è mai stato chiarito quali furono le origini dei loro esperimenti; molto probabilmente avevano effettivamente trovato qualcosa (\emph{qualcuno?}), che potesse funzionare da portale per attingere a questa forma di energia pressoché illimitata.

Nel giorno dell'inaugurazione del loro primo reattore alimentato da ciò che chiamavano  \textbf{Omniessenza}, una \emph{parte} della loro \emph{invenzione}, avvenne l'impossibile.

I racconti si fanno molto confusi a questo punto, di fatto l'\emph{Omniessenza} era effettivamente qualcosa di vero e di \emph{vivo}, una parte di una energia più grande. All'attivazione del reattore questo esplose con un'energia e forza mai viste sulla Terra, buona parte di quelli che erano gli stati centrali degli USA vennero vaporizzati all'istante.

Nel punto dove una volta sorgeva la sede della Freten si aprì una breccia simile a un portale: una colossale fiamma divisa in due lingue di fuoco di colore diverso.

Da questa fiamma uscì un mastodontico \emph{drago rosso} le cui scaglie erano impenetrabili a qualsiasi arma umana. Tàhil, questo il suo nome, nelle prime 24 ore distrusse quello ciò che restava della costa orientale degli Stati Uniti.

All'alba del secondo giorno una mistica energia avvolse Tàhil e questo mutò in un drago a molte teste di colori diversi. Nel suo petto apparì un'altra breccia ed a ogni passo una moltitudine di altri draghi, per fortuna \emph{leggermente} più piccoli incominciò ad uscire.

All'alba del terzo giorno Tàhil pronunciò l'Editto della Dimenticanza, un'onda magica che fuse i componenti di ogni apparato elettrico e cancellò ogni dato. Il sapere stesso divenne un enigma, le parole dei libri si mescolarono in un caos incomprensibile.

Tàhil si stabilì in piazza Medan Merdeka a Giacarta, un posto sufficientemente ampio per permettergli di sdraiarsi e comandare i suoi eserciti di draghi.

All’alba del quarto giorno, Tàhil scatenò l’Editto della Guerra: per sette giorni, gli umani si combatterono tra loro, distruggendo ciò che restava della loro civiltà.

All’alba del dodicesimo giorno, Tàhil proclamò l’Editto dell’Anarchia, e i popoli abbatterono le forme di governo costituite. Il mondo sprofondò nel caos, senza leggi né ordine.

All’alba del diciannovesimo giorno, l’Editto del Sacrificio uccise un terzo della popolazione, in una dimostrazione spietata di potere.

All’alba del trentesimo giorno, Tàhil proclamò l’Editto della Rifondazione. Nuove brecce si aprirono, e altri esseri, altri poteri si manifestarono. Il mondo fu trasformato e nuove regole vennero scritte.

Intanto e per 1 anno intero i draghi distrussero e uccisero qualsiasi cosa, ogni persona. Nessun esercito sopravvisse, nessun governo rimase in carica, nessuna nazione si poteva ancora chiamare tale.

I Terrestri erano stati puniti per il loro affronto, solo il 10\% della popolazione era sopravvissuta.

Questi nuovi esseri facevano scomparire, sprofondare, ribaltare; distruggevano intere città, mutavano ambienti e creature, facevano comparire nuove specie. Dal nulla apparivano orde di mostri, come quelli descritti nei libri di gioco dei bambini. La realtà, per loro, era un capriccio da plasmare secondo gusti eccentrici.

Le nazioni, come le conoscevamo, non esistevano più. Anche la natura si era trasformata, assumendo forme tra le più aliene immaginabili. Molte zone erano divenute deserti nucleari, inospitali e letali per chiunque... o quasi.

Poi, tutte le entità, tranne Tàhil e i draghi, sparirono nel nulla per sei mesi.
Trascorso quel tempo, i sogni dei pochi esseri rimasti cominciarono a essere invasi da visioni di altri \emph{esseri}, altre entità.

E arrivò così la seconda ondata dei \emph{Patroni} come collettivamente si facevano chiamare. Per fortuna questi esseri si rivelarono, tutto sommato, più gentili e \emph{umani}, o almeno qualcuno lo era. Bonificarono buona parte delle zone radioattive ed insegnarono a chi accettava i loro Tratti ad attingere alla loro energia per poter formulare delle vere, reali, concrete \textbf{magie}!
Alcune entità crearono o richiamarono altre razze; vuoi per poter dominare gli umani, vuoi per poterli guidare, vuoi per aggiungere caos ed entropia al mondo.

Sono passati poco più di cento anni dalla seconda venuta eppure tanto è bastato perché La nostra Terra tornasse ad un medioevo di fantastiche origini.

Molti dei Patroni più oscuri hanno aperto portali verso regni dell'incubo se non demoniaci, altri hanno attinto dal folklore locale per divertirsi con la nostra sofferenza e morte. Come divinità i Patroni camminano sulla Terra con il solo scopo di avere più persone che le adorino, che seguano i loro insegnamenti, che abbiano i loro Tratti.

\medskip

\begin{enfasi}{Si può scoprire di più su una persona in un'ora di gioco che in un anno di conversazione. (Platone)}\end{enfasi}

\subsection{Introduzione}

Benvenuti in \textbf{OBSS}, un mondo dove l'ordinario si fonde con l'incredibile, dove la magia antica coesiste con la tecnologia avanzata e le leggende nascono dai gesti di individui comuni. In questo universo narrativo, i vostri personaggi non sono eroi predestinati, ma persone comuni, con i loro sogni, paure e ambizioni, catapultate in un vortice di eventi che cambieranno per sempre le loro vite.

Preparatevi a confrontarvi con sfide inaspettate, a stringere alleanze fragili e a lottare per la sopravvivenza in un mondo caotico e pericoloso.

Il Narratore è l’architetto di questo mondo: colui che plasma la realtà e tesse le fila della storia.
Spetta a lui presentarvi le sfide, descrivere i luoghi, dare vita ai personaggi non giocanti e interpretare le conseguenze delle vostre azioni.

In OBSS, la collaborazione tra giocatori e Narratore è fondamentale per creare un’esperienza di gioco coinvolgente e indimenticabile.
La vera regola che il Narratore deve ricordare è che qualsiasi regola sia usata finché fa divertire tutti è quella giusta!.

La sopravvivenza è la legge fondamentale di questo mondo. In OBSS, non ci sono garanzie di successo e ogni passo potrebbe essere l'ultimo. Ma è proprio in questa lotta per la sopravvivenza che si forgiano i veri eroi. Attraverso le vostre azioni, la vostra astuzia e il vostro coraggio potrete reclamare la Legge del Premio, guadagnando esperienza, ricchezze e la possibilità di influenzare il corso degli eventi.

Ogni personaggio è definito da sei Caratteristiche fondamentali che rappresentano i suoi attributi innati: Forza, Destrezza, Costituzione, Intelligenza, Saggezza e Carisma. Inoltre ogni personaggio possiede cinque Tratti, qualità uniche che lo caratterizzano a livello personale e che influenzano il suo stile di gioco e le sue interazioni con il mondo. I Tratti non sono semplici elementi di background, ma vere e proprie guide alle vostre azioni: plasmano le decisioni e determinano il legame con i Patroni, entità misteriose che offrono poteri speciali in cambio di fedeltà.

In questo manuale troverete le istruzioni per creare i vostri personaggi, per interagire con il mondo di gioco, per combattere, per usare la magia e per affrontare le sfide che incontrerete lungo il vostro cammino.

Ma prima di iniziare le vostre avventure dovrete partecipare alla Sessione Zero, un momento cruciale per gettare le basi della vostra esperienza. Durante la Sessione Zero, stabilirete le regole ed opzioni di gioco, definirete le aspettative del gruppo, creerete i vostri personaggi e darete vita alle loro storie.

In OBSS, la vostra fantasia è l'unico limite. Non abbiate paura di sperimentare, di mettere alla prova le vostre idee e di costruire personaggi che siano unici e indimenticabili. Siete liberi di plasmare il vostro destino e di lasciare il segno in questo mondo, un tiro di dado alla volta.

Le azioni sono misurate in base a delle Azioni e il successo si basa sui tiri di dado, la vostra competenza, Abilità e le vostre scelte tattiche.

Ricordate: le prove possono essere evitate con intelligenza e strategia. L'esplorazione, la capacità di risolvere enigmi e l'immaginazione sono componenti cruciali di questo gioco. Non cercate per forza la soluzione nella scheda, usate l'ingegno!

Ora che siete pronti, è tempo di abbracciare il vostro destino e di scrivere la vostra leggenda. Siate coraggiosi, siate creativi e siate pronti a tutto. Che la vostra avventura abbia inizio!

\begin{center}
Buona lettura e Buon Divertimento!
\end{center}

\begin{flushright}
Andres Zanzani
\end{flushright}

\end{multicols}

\vfill

\begin{enfasi}
D\&D ha nelle proprie origini tratti misogini e razzisti che con il tempo sono stati rimossi grazie alle tantissime persone di tutti i generi e tipi che ci hanno giocato.
OBSS vuole continuare nel solco di un gioco inclusivo e libero. Ogni gruppo è libero di approcciare argomenti controversi come meglio crede ma sempre nel rispetto di ogni giocatore e sensibilità. Non fate che OBSS sia motivo di litigio ma di unione e spirito fraterno, un gioco che unisca e mai divida. (Andres Zanzani)
\end{enfasi}

\pagebreak

\subsection{Termini Comuni}\label{Termini Comuni}

\begin{multicols}{2}

Ti elenco un pò di termini\index{Termini comuni} e concetti che troverai ripetuti più volte nel libro.

\medskip

\textbf{Arrotondamenti}: \index{Arrotondamenti}sempre per difetto se non esplicitato diversamente ma con un minimo di 1. Es. 7/2 = 3, 9/4=2, 1/2=1

\textbf{Abilità}: \index{Abilità}sono capacità particolari che il personaggio ha imparato ad usare. Spesso simili a capacità magiche, permettono azioni particolari, di sovvertire le regole e concedono dei bonus ai Tiri Salvezza che si cumulano tra loro. Si prendono ai passaggi di livello (vedi \hyperlink{abilita}{Abilità}, pag. \pageref{abilita})

\textbf{Azione}: \index{Azione} è ciò che si fa in un intervallo di tempo. Ogni cosa che viene fatta dal personaggio si misura in Azioni. Combattere, lanciare Incantesimi, scassinare, bere pozioni, lo spostarsi... in ogni round si possono fare 3 Azioni. Un'Azione dura circa 3 secondi.

\textbf{Bonus}: \index{Bonus}qualsiasi modificatore dovuto a fattori esterni, ambientali, magici, di circostanza o che decida il Narratore è un bonus o penalità da applicare al tiro di dado o difficoltà nella prova.

\textbf{Check/Prova}: \index{Check}\index{Prova}un check (o prova) è il tiro di 3d6 più il punteggio indicato dalla Caratteristica e Competenza coinvolta, potrebbero essere applicati modificatori dati da Abilità e circostanze. Se non si possiede la Competenza si tirano 2d6 + il modificatore della Caratteristica.

\textbf{Classe}: In OBSS non ci sono classi. Ogni personaggio è costruito in base a ciò che sa fare, non troverete la parola Classe nel manuale. Ogni personaggio è unico e definito dalle sue scelte.

\textbf{Colpo Critico}\index{Colpo Critico}: quando il Tiro per Colpire supera di almeno 8 la Difesa dell'avversario applichi il solo dado dell'arma in più al danno causato.

\textbf{Prova di Magia}\index{Prova di Magia}: la Prova di Magia può essere dovuta a particolari situazioni, ad esempio quando il personaggio è ferito o distratto, ma può essere richiesta anche dal giocatore.

La Prova di Magia permette al personaggio di spingersi oltre nel lancio dell'incantesimo e provare ad attingere e sfruttare più magia.

A seconda dei risultati potrebbe ottenere vantaggi o svantaggi.

\textbf{Lanciare Incantesimi sotto attacco, minaccia, distrazione..}:\index{Prova di Concentrazione}\index{Lanciare Incantesimi sotto attacco, minaccia, distrazione..}\index{Distratto} quando un incantatore vuole usare una Magia ma è disturbato, attaccato, ferito o comunque distratto durante il lancio di un incantesimo allora dovrà effettuare una Prova di Magia.

\textbf{Classe di Difficoltà (DC)}:\index{Classe di Difficoltà} \index{DC}indica quanto è difficile riuscire in una prova. Può essere usato per le competenze (nuotare..) come le conoscenze (veleni..). Negli incantesimi è la difficoltà a resistere agli incantesimi. Indica a che valore arrivare per superare e riuscire nella prova.

\textbf{Competenza} \index{Competenza}(skill)\index{Skill}: la competenza ci dice ciò che sappiamo ed il suo valore indica il grado di conoscenza della stessa. Possa essere lo studio di una lingua, l'arrampicarsi, il notare piccole cose.

\textbf{Competenza con le Armi (CA) (da mischia o distanza)} \index{Competenza con le Armi} è la tua capacità di saper colpire l'avversario con armi da mischia (spade, mazze, pugni..) o da tiro/distanza (pugnali da lancio, archi, balestre..)

\textbf{Competenza Magica (CM)}: \index{Competenza Magica}\index{CM}è la tua capacità di usare le magie, più è alto questo valore più le magie saranno efficaci, più ne avrai a disposizione, più ne potrai lanciare.

\begin{center}
\includegraphics[keepaspectratio,width=0.3\textwidth]{immagini/spiritomagia2.png}
\end{center}

\textbf{Difesa}: \index{Difesa}per Difesa si intende il valore totale ottenuto da 10 + Scudo + Armatura + Destrezza + vari ed eventuali bonus. Rappresenta la capacità di non farsi colpire e non essere ferito. Un nemico con alta Difesa potrà essere estremamente agile ed avere una \emph{pellaccia} estremamente resistente al ferimento.

\textbf{+1d6 oppure -1d6}: è un bonus o penalità ad una prova. Aggiungi o sottrai un tiro di dado a 6 alla prova. Il massimo della penalità porta i numero dei dadi lanciati a 0 ed il bonus massimo a +3d6.\index{Massimo valore d bonus}

\textbf{Distanza}:\index{Distanza} la distanza, per quando riguarda il combattimento è misurato in quadretti da 1 metro.

\textbf{Devoto}\index{Devoto}: un personaggio che si é legato ad un Patrono ed ha almeno 2 Tratti in comune.

\textbf{Seguace}: un personaggio che si è legato ad un Patrono con 1 Tratti in comune

\textbf{Esplosione del 6}:\index{Esplosione del 6} quando, esegui il Tiro per Colpire, Tiro Salvezza, Prova di Competenza, Prova di Magia, Iniziativa (leggi le specifiche nel capitolo dedicato) o comunque ogni volta che viene indicato che vale l'esplosione del 6 significa che per ogni dado tirato che ha fatto 6 va segnato e ritirato il dado. Il risultato del nuovo tiro va anche lui sommato e se si fa un 6 si continua a ritirare finché si continua a fare 6.

\textbf{Iniziativa}: \index{Iniziativa}è una prova di Destrezza oppure Intelligenza. Stabilisce l'ordine delle azioni in combattimento. Chi ha il punteggi più alto nella prova agisce per primo.

\textbf{Livello}:\index{Livello} il Livello indica la competenza e potere raggiunto dal personaggio. Può indicare quando è \emph{forte} il nemico.

\textbf{Livello dell'incantesimo}: indica la scala (da 1 a 9) della potenza magica dell'incantesimo.

\textbf{Incantatore, Mago:} \index{Incantatore}indica un qualsiasi usufruitore di magia a qualsiasi titolo.

\textbf{Mischia}: \index{Mischia}con mischia si intende il combattimento di contatto, corpo a corpo, spada a spada, ovvero quando il tuo personaggio combatte con un arma che abbia non abbia gittata (arco, balestre, fionde...) contro un avversario.
Si considera in mischia qualsiasi creatura che il personaggio possa raggiungere con la sua arma non da tiro. Una creatura di grandi dimensioni (o con un arma lunga) potrebbe essere in mischia con il personaggio ma non viceversa.

\textbf{Movimento}: \index{Movimento}il movimento rappresenta la capacità di spostarsi. Una Azione di Movimento rappresenta lo spostarsi del personaggio, più è alto il valore di Movimento più metri una creatura può muoversi.

\begin{wrapfigure}[24]{r}[.5\width+.5\columnsep]{7cm}%\itshape

\centering
\includegraphics[width=6.5cm]{immagini/merlin.png}

\emph{Merlin dictating his prophecies to his scribe. Robert de Boron's Merlin en prose (written ca 1200)}
\end{wrapfigure}

%\begin{center}
%\includegraphics[width=0.35\textwidth]{immagini/merlin.png}
%
%\emph{Merlin dictating his prophecies to his scribe. Robert de Boron's Merlin en prose (written ca 1200)}
%\end{center}

\textbf{Narratore}:\index{Narratore} è la persona che conduce l'avventura, stabilisce le regole e controlla gli elementi della storia. Il dovere di ogni Narratore è fare divertire, essere corretto ed usare il buon senso. Il Narratore ha l'ultima parola in ogni questione.

\textbf{Opzionale}:\index{Opzionale} in OBSS sono presenti diverse regole Opzionali per diversificare e personalizzare il gioco. Parlatene durante la Sessione Zero e decidete che stile dare al vostro OBSS.

\textbf{Prova di Caratteristica}\index{Prova di caratteristica}: è una prova di Competenza che usa come bonus il valore di una Caratteristica, quale Forza, Carisma..

\textbf{Patrono}:\index{Patrono} o divinità. Il Patrono è un essere superiore che può concedere poteri e garantire vantaggi.

\textbf{Penalità/Malus} \index{Penalità}: come il bonus le penalità o malus sono valori, numeri, che indicano le circostanze sfavorevoli, magie penalizzanti o quant'altro renda più difficile la prova. Purtroppo a differenza dei Bonus le penalità, se non specificato diversamente, si sommano sempre fra loro.

\textbf{PG, Personaggio}: \index{Personaggio}è la creatura che viene guidata, gestita, \emph{ruolata} dal giocatore.

\textbf{PNG}: \index{PNG}personaggio non giocante. Sono personaggi particolari, importanti o meno che il Narratore tiene per condurre l'avventura.

\textbf{Punti Esperienza/PX}: \index{Punti Esperienza} \index{PX} ogni qual volta si risolvano difficoltà, indovinelli, si affrontino mostri o si trovino dei tesori, si giochi bene il personaggio e ci si diverta si guadagna esperienza. Questi punti accumulati nel tempo stabiliscono il livello e quindi le capacità del personaggio.

\textbf{Punteggi caratteristica}: \index{Punteggi caratteristica} \index{Statistiche} abbreviati anche in caratteristica o statistiche. Ogni personaggio ha 6 Caratteristiche: Forza (FOR), Destrezza (DES), Intelligenza (INT), Saggezza (SAG) e Carisma (CAR). più è alto il punteggio maggiore è la valenza o capacità del personaggio in quello specifico ambito.

\textbf{Punti Fato}:\index{Punti Fato} \index{Fortuna del Principiante}o Fortuna del Principiante sono dei punti a disposizione che il giocatore può trasformare in d6 da aggiungere ai Tiri Salvezza o Tiri per Colpire o Tiri Competenze. Vengono chiamati Fortuna dei Principianti perché il loro numero diminuisce all'aumentare di livello del personaggio.

\textbf{Punti ferita (Punti Ferita)}:\index{Punti Ferita} \index{Punti Ferita}indicano l'energia vitale, la resistenza, la fortuna nel resistere alle ferite della creatura. Finché la creatura ha 1 punto ferita combatterà al suo meglio, senza problemi (ma potrebbe anche decidere di scappare piuttosto che morire!).

\begin{wrapfigure}[24]{l}[.5\width+.5\columnsep]{7cm}

	\centering
\end{wrapfigure}

Ad ogni passaggio di livello si guadagna un certo numero di Punti Ferita, stabilito dalle regole. Ogni ferita si sottrae da questa cumulo di energie e quando si raggiungono gli 0 (zero) Punti Ferita si sviene, incapaci di agire.

%\medskip
%\begin{center}
%\includegraphics[width=0.35\textwidth]{immagini/Sakramentarz_tyniecki_02.png}
%
%\emph{Sakramentarz Tyniecki: Majuskuła "V".}
%\end{center}
%\medskip

Se si viene ulteriormente feriti ed i Punti Ferita scendono fino 10 + il doppio del valore della Costituzione allora si muore.

\textbf{Riduzione del Danno (DR)}: \index{Riduzione del Danno} \index{DR} alcune creature hanno una resistenza innata al danno e ferite. Questa resistenza si denota come DR. La Riduzione si applica dopo la Resistenza ed i Tiri Salvezza.

\textbf{Resistenza al Danno (RD)}, \textbf{Resistenza}: \index{Resistenza al Danno}\index{RD}: una creatura potrebbe avere una resistenza ad una tipologia di danno. In questo caso si considera che dimezzi automaticamente il danno subito prima di applicare eventuali Tiri Salvezza.

\textbf{Round}:\index{Round} il combattimento o azioni sono divise in round. Un round rappresenta una unità temporale di circa 10 secondi. Durante il round ogni creatura ha la possibilità di agire in base alla sua iniziativa ed eseguire fino a 3 Azioni.

\textbf{Successo Critico/Fallimento Critico Magico}\index{Successo Critico Magico} \index{Fallimento Critico Magico}: nel caso in cui il giocatore passi la Prova di Magia con dei critici. Il Successo Critico Magico porta a spettacolose modifiche nell'incantesimo, viceversa potrebbero succedere brutte cose all'incantatore.

\textbf{Tiro Salvezza (TS)}:\index{Tiro Salvezza} \index{TS}quando una creatura è sottoposta ad un effetto particolare spesso viene concesso un Tiro Salvezza per mitigare o annullare gli effetti. Il Tiro Salvezza è un'azione che non occupa tempo o Azioni.

I Tiri Salvezza riguardano i riflessi e lo schivare (Riflessi), resistere a veleni/malattie o cambiamenti del corpo (Tempra) oppure per resistere ad attacchi mentali ed effetti che agiscano sull'arbitrio e volontà (Volontà).

\textbf{Successo Critico/Fallimento Critico nel Tiro Salvezza}\index{Sucesso Critico Magico nel Tiro Salvezza} \index{Fallimento Critico nel Tiro Salvezza}: a seconda dell'incantesimo in caso di Successo Critico nel Tiro Salvezza si dimezzano ulteriormente gli effetti mentre in caso di Fallimento Critico si subisce ancora di più il danno.

\begin{center}
\includegraphics[width=0.5\textwidth]{immagini/Jan_Steen2.png}

\emph{Jan Havicksz. Steen}
\end{center}

\textbf{Tiro per Colpire (TC)}:\index{Tiro per Colpire} \index{TC}è una prova di Attacco (Competenza Armi + Forza/Destrezza + Abilità + capacità dati da lista armi...) contro Difesa (armatura + scudo + Abilità + magia...). Il Tiro per Colpire può essere da mischia (ovvero per le creature prossime alla tua arma, a distanza di mischia) oppure da distanza (per archi, balestre, ma anche pugnali lanciati..).. Leggi bene il capitolo del combattimento.

\textbf{Tratto}: \index{Tratto}indica una componente del carattere. Ogni personaggio sceglie 5 Tratti per comporre e costruire la sua personalità.

\textbf{Turno}: \index{Turno}sono 10 minuti, ovvero 60 round

\textbf{Uno porta male}: \index{Uno porta male}se tiri un 1 con il dato togli 1 dal risultato totale. Non per questo un 6 tirato diventa un 5, l'esplosione del 6 rimane.. solo che togli 1 al risultato finale. Detta diversamente 1 vale 0.

\begin{enfasi}
Il gioco di D\&D non ha né vinti né vincitori, ha solo giocatori che amano esercitare la propria immaginazione. I giocatori ed il DM condividono la creazione di avventure in terre fantastiche dove abbondano gli eroi e la magia funziona davvero. In un certo senso, il gioco di D\&D non ha regole, solo suggerimenti di regole. Nessuna regola è inviolata, in particolare se una regola nuova o modificata incoraggerà la creatività e l'immaginazione. L'importante è godersi l'avventura. (Tom Moldvay, 03/12/1980. E tutto quanto detto vale anche per OBSS! NdA)
\end{enfasi}

\medskip

Nel Manuale troverete diverse tipologie di box, ognuno ha un significato preciso:

\medskip

\begin{enfasi}{Esempio di box contenente una citazione o frase motivazionale}\end{enfasi}

\begin{giocatore}[Informazioni per il Giocatore]Box contenente indicazioni e chiarimenti per il Giocatore.\end{giocatore}

\begin{narratore}[Informazioni per il Narratore]Box contenente indicazioni e suggerimenti per il Narratore.\end{narratore}

\end{multicols}

%\vfill

%\begin{center}
%\includegraphics[width=0.3\textwidth]{immagini/cavaliere2.png}
%\end{center}

\vfill

\begin{center}
\includegraphics[keepaspectratio,width=0.95\linewidth]{immagini/dice.png}

\medskip

\emph{Tipico set di dadi da gioco di ruolo}
\end{center}

\pagebreak

\section{Razze}\index{Razze}

\begin{enfasi}{Il vero viaggio di scoperta non consiste nel trovare nuovi territori, ma nel possedere altri occhi, vedere l'universo attraverso gli occhi di un altro, di centinaia d'altri: di osservare il centinaio di universi che ciascuno di loro osserva, che ciascuno di loro è. (Marcel Proust)

\medskip

Non è la più intelligente delle specie quella che sopravvive; non è nemmeno la più forte; la specie che sopravvive è quella che è in grado di adattarsi e di adeguarsi meglio ai cambiamenti dell'ambiente in cui si trova. (Leon C. Megginson)}\end{enfasi}\medskip

\begin{multicols}{2}

La Terra è un mondo sfaccettato e ricco di diversità, culturali, naturali e di creature.
Sono le creature a rendere il pianeta vitale e ricco, ognuna nutre, apporta, arricchisce la conoscenza di tutte le altre.

\subsection{Umani}\index{Umani}\label{umani}

\begin{center}
\includegraphics[height=0.9\linewidth]{immagini/uomovitruviano2.png}

\emph{Uomo vitruviano - Leonardo da Vinci}
\end{center}

Gli umani con il loro desiderio di scoperte, potere, gloria e violenza e capacità riproduttiva erano la razza dominante, tutta la Terra si piegava al suo volere. Finché non ci fu la venuta.
E gli umani diventarono la specie da cacciare ed uccidere, il dettato ricevuto dai primi Patroni era chiaro e assoluto, sterminare gli umani che avevano ucciso il primogenito.

E' impossibile sapere quanti umani sono sopravvissuti nel mondo, calcoli molto approssimativi li contano sotto i 500 milioni.

\textbf{Modificatori razziali}: +1 ad una caratteristica a piacere

\textbf{Caratteristiche fisiche}: altezza 160-185 cm, 50-130 kg, aspettativa di vita 65 anni (50 + 2d10 anni)

\textbf{Dimensioni}: Medie

\textbf{Velocità}: 9m

\textbf{Linguaggi}: Comune

\textbf{Vantaggio}: +1 Abilità al primo livello. Il primo punto assegnato a CA o CM si raddoppia.

\subsection{Elfi}\index{Elfi}\label{elfi}

\begin{center}
\includegraphics[height=0.7\linewidth]{immagini/elfa3-ai.png}

\emph{(B.I.C.)}
\end{center}

Gli elfi sono la razza portata direttamente dal Patrono della Genesi Calicante perché portassero dissolutezza, terrore, dolore e perdizione sulla Terra.

Deportati a forza da mondi lontani fatti di caos, guerra e dolore, dopo che le loro Legioni del Terrore imperversarono per oltre un secolo sul nostro pianeta stremando e riducendo ancor di più i pochi umani rimasti intervenne direttamente la Dama della Luce Ljust per instillare un scintilla di carità nel loro corpo senza anima.

E così i nuovi nati, ma non tutti, non hanno questo odio viscerale e mania omicida, il loro sangue non è stato macchiato da Calicante e vorrebbero vivere una vita normale a contatto con tutte le altre creature anche se ben consci di come sono visti e trattati da tutti gli altri.

Sono i giovani elfi che vogliono costruire e vivere in una nuova Terra. Le altre creature hanno imparato a giudicare un elfo in base all'età, per quanto sia difficile dare un età questa razza; un elfo \emph{anziano} è il male e va ucciso, un elfo \emph{giovane} forse non è malvagio.
Sono metodi brutali e approssimativi che purtroppo continuano ad essere diffusi ed applicati.

Gli elfi sono generalmente più bassi, minuti e snelli degli umani. Gli occhi hanno sempre con sfumature scure, le orecchie piuttosto lunghe puntano più lateralmente che in alto.

\textbf{Modificatori razziali}: +1 Intelligenza, +1 Destrezza, -1 Carisma

\textbf{Caratteristiche fisiche}: altezza 145-165 cm, 38-110 kg, aspettativa di vita 6000 anni (4000 + 20d100 anni)

\textbf{Dimensioni}: Medie

\textbf{Velocità}: 9m

\textbf{Linguaggi}: Elfico, Comune

\textbf{Vantaggio}: Abilità \emph{Concentrato}

\subsection{Nani}\index{Nani}\label{nani}

\begin{center}
\includegraphics[height=0.7\linewidth]{immagini/nana2-ai.png}

\emph{(B.I.C.)}
\end{center}

I nani sono una razza stoica e severa abituata al comunismo più puro, senza un vero concetto di proprietà ma di pura comunanza di beni secondo l'idea che ogni nano lavora per la comunità e non per se stesso.

I nani sono compatti e piazzati, raggiungono un'altezza massima di circa 140 cm con una corporatura robusta che dà loro un aspetto massiccio. Sia i maschi che le femmine portano orgogliosamente i capelli lunghi e gli uomini decorano spesso le barbe con vari generi di fermagli e trecce intricate, altresì vero che nani pelati sono frequenti, ma non senza barba. Le donne nane non hanno barba nè peluria in eccesso. Il sesso è libero e socialista.

I nani sono guidati da onore, tradizione e comunismo. Sono spesso visti come burberi, ma hanno un forte sentimento di amicizia e giustizia, rispettano chi lavora sodo e si impegna per la comunità e per il gruppo.

I nani sono la razza invitata sulla Terra dal Patrono Erondil con l'aiuto del Patrono Atmos. Portati da un lontano mondo di guerra e macchine hanno ritrovato sulla Terra lo spirito degli avi e la capacità di costruire senza distruggere.

Sono piuttosto xenofobi ed intolleranti con chi non è in linea con i loro principi.

\textbf{Modificatori razziali}: +1 Costituzione, +1 Saggezza, -1 Destrezza

\textbf{Caratteristiche fisiche}: altezza 100-140 cm, 45-90 kg, aspettativa di vita 450 anni (400 + 1d100 anni)

\textbf{Dimensioni}: Medie

\textbf{Velocità}: 6m

\textbf{Linguaggi}: Nanico, Comune

\textbf{Vantaggio}: Abilità \emph{Muro mentale}

\subsection{Gnomi}\index{Gnomi}\label{gnomi}

\begin{center}
\includegraphics[height=0.7\linewidth]{immagini/gnoma-ai.png}

\emph{Gnoma, orecchie a punta. (B.I.C.)}
\end{center}

Gli Gnomi sono esseri dalla piccola taglia ma ricchi di energia e vita. Gli Gnomi sono la razza chiesta dai Patroni Nihar e Shayalia per rallegrare ed abbellire questo sofferente mondo.

In breve tempo, grazie alla loro innata curiosità, tenacia ed inventiva sono riusciti a creare popolose e ricche città, quasi sempre all'interno di foreste vergini.

Gli Gnomi sono profondamente legati alla natura, il loro rapporto è quasi simbiotico, uno Gnomo non rinuncerà mai alla vista di un albero ed a costruire con ciò che la natura fornisce.

Gli Gnomi hanno un profondo rispetto per la natura, l'ambiente e gli animali, le loro città perfettamente funzionali ed moderne sono costruite e ricavate nelle foresta e mai distruggendola ed anzi arricchendola.

Molti Gnomi sono inventori e costruttori capaci di prove di fantasia ed ingegno fuori dal comune. Molte delle loro invenzioni sono di aiuto e supporto a tutta la comunità e la loro vita sociale è ricca e solidale.

Gli Gnomi più curiosi spesso escono dalle loro comunità, che non sono chiuse a chiunque rispetti la natura, ed intraprendono una vita di avventura alla scoperta di meraviglie e nuove opere di ingegno da poter tramandare alla comunità.

Uno gnomo costretto a stare lontano da un ambiente naturale patisce la situazione diventando triste e apatico, la sua necessità di natura è qualcosa di fisico e connaturato.

Gli Gnomi vanno d'accordo con chiunque ami la natura e non ne abusi.

Una diatriba che interessa a dire la verità ben poco gli gnomi, è quella relativa alla forma delle loro orecchie. Secondo gli elfi gli gnomi del loro mondo hanno le orecchie a punta, secondo i nani gli gnomi che conoscevano hanno invece le orecchie piccole e tonde come loro. Fatto sta che gli gnomi nascono casualmente con orecchie a punta o tonde e hanno abbastanza buon senso per ignorare la forma. Almeno quasi tutti..

\textbf{Modificatori razziali}: +1 Intelligenza, +1 Carisma, -1 Forza

\textbf{Caratteristiche fisiche}: altezza 70-110 cm, 30-50 kg, aspettativa di vita 650 anni (600 + 1d100 anni)

\textbf{Dimensioni}: Piccolo

\textbf{Velocità}: 6m

\textbf{Linguaggi}: Gnomico, Comune

\textbf{Vantaggio}: Artificio Druidico 1 volta al giorno ogni (CM+CA)/6.

\subsection{Mezzelfo}\index{Mezzelfo}\label{mezzelfo}

\begin{center}
\includegraphics[height=0.7\linewidth]{immagini/halfelf3_grayscale-ai.png}

\emph{(B.I.C.)}
\end{center}

Per un elfo non c'è nulla di più impuro di un mezzelfo. Nessun mezzelfo nasce per volontà di un Elfo. Ogni mezzelfo è figlio di violenza. Questo è almeno quello che continuano a dire gli elfi.

Ci sono anche rari mezzelfi nati da rapporti romantici. Benché solitamente di breve durata, anche per gli standard umani, questi incontri segreti portano di solito alla nascita dei mezzelfi, una razza che discende da due culture, ma non è erede di nessuna. I mezzelfi possono riprodursi tra loro, ma persino questi mezzelfi \emph{di sangue puro} sono visti come bastardi dagli elfi.

Molti elfi vedono in un mezzelfo il tradimento della missione originale, della distruzione del creato.
Pochissimi vedono un gesto di amore e dono verso un mondo sempre più imbruttito.
Solitamente vengono visti dalle altre creature come degli assassini al pari degli elfi indipendentemente che il loro sangue sia stato toccato o meno da Calicante.

I mezzelfi sono più bassi degli umani ma più alti degli elfi. Ereditano la corporatura slanciata e i lineamenti attraenti del loro lignaggio elfico, ma il colore della loro pelle è normalmente dettato dalla loro parte umana. I loro occhi tendono ad essere simili a quelli degli umani nella forma, ma presentano un'esotica gamma di colori dall'ambra al viola fino al verde smeraldo e al blu scuro.

I mezzelfi comprendono la solitudine e sanno che il carattere spesso è più un prodotto dell'esperienza di vita che della razza di appartenenza.

\textbf{Modificatori razziali}: +1 ad una Caratteristica a propria scelta

\textbf{Caratteristiche fisiche}: altezza 150-185 cm, 50-100 kg, aspettativa di vita 210 anni (180 + 5d10 anni)

\textbf{Dimensioni}: Medie

\textbf{Velocità}: 9m

\textbf{Linguaggi}: Comune, Elfico

\textbf{Vantaggio}: Una Abilità aggiuntiva a scelta.

\index{Mezzorco}

\subsection{Mezzorco}\label{mezzorco}

\begin{center}
\includegraphics[height=0.7\linewidth]{immagini/half-orc2-ai.png}

\emph{(B.I.C.)}
\end{center}

Agli occhi delle culture civilizzate, i mezzorchi sono delle mostruosità, il risultato di perversione e violenza e raramente sono il risultato di unioni amorose, come tali solitamente sono costretti a crescere velocemente e duramente, lottando continuamente per proteggersi o farsi un nome. Alcuni mezzorchi trascorrono le loro intere vite a dimostrare agli orchi purosangue che sono feroci quanto loro.

I mezzorchi sono alti in media 1.9 metri, con fisico potente e pelle verdastra o grigia. Ai maschi i canini crescono spesso piuttosto lunghi fino a sporgere dalle loro bocche e queste \emph{zanne}, unite ad una fronte a volte ampia e le orecchie un pò a punta, danno loro quel noto aspetto \emph{bestiale}.
Le femmine hanno i tratti orcheschi molto meno marcati e sono considerate selvagge e \emph{facili} da parte dei maschi umani che le mancano del rispetto dovuto.

A dispetto di questi ovvi tratti orcheschi, i mezzorchi sono tanto variegati quanto i loro genitori umani.

Se all'interno delle tribù orchesche devono guadagnarsi continuamente il rispetto dei \emph{purosangue}, nella società umana non va meglio. Derisi, sbeffeggiati, esclusi ed abbandonati i mezzorchi spesso trovano rifugio nella criminalità.

Gli orchi sono stati creati direttamente dal Patrono Cattalm con l'aiuto di Calicante. Molto della tendenza caotica e distruttrice del loro creatore rimane nella natura dei mezzorchi.

I mezzorchi sono continue vittime di pregiudizi.

\textbf{Modificatori razziali}: +2 Forza -1 Carisma

\textbf{Caratteristiche fisiche}: altezza 160-210 cm, 60 - 140 kg, aspettativa di vita 70 anni (50 + 5d10 anni)

\textbf{Dimensioni}: Medie

\textbf{Velocità}: 9m

\textbf{Linguaggi}: Comune, Orchesco

\textbf{Vantaggio}: Una Abilità aggiuntiva a scelta.

\subsection{Nibali}\index{Nibali}\label{nibali}

\begin{center}
\includegraphics[height=0.7\linewidth]{immagini/nibali2-ai.png}

\emph{(B.I.C.)}
\end{center}

I Nibali sono una razza creata magicamente per essere schiava ai primi Patroni.

La leggenda dice che un antico Patrono, partendo da una coppia di umani (dopo che a migliaia erano morti atrocemente nei precedenti esperimenti) riuscì a creare manipolando con la magia, un razza più robusta, più forte, più intelligente ed allo stesso tempo più docile e disciplinata con pregio che ogni figlio generato sarebbe stato assolutamente identico fisicamente al padre o alla madre.

Quando i primi Patroni sono andati via i Nibali hanno continuato a prosperare ed usufruendo di quanto avevano già creato nella fredda tundra.

Per molti l'estrema efficienza e dedizione dei Nibali è odiosa, un giogo che non lascia spazio alle libertà personali, per i Nibali è solo un modo naturale di progredire.

Tutti i Nibali sono uguali tra loro a parità di sesso ma il fatto che non possano avere figli con altre razze non li rende un popolo chiuso o razzista, anzi l'assorbire il meglio di ogni cultura li rende migliori ed anche ottimi diplomatici. Ciò che veramente distingue un Nibali da un altro è l'acconciatura, i tatuaggi, il vestiario, l'essere se stessi. Il rispetto per gli altri e la Legge sono legati indissolubilmente alla loro natura eppure non c'è nulla di più libero di un Nibali.

Per un Nibali le regole e le leggi devono promuovere la pace e la libertà, devono essere giuste e coloro che le mantengono comprensivi e saggi. Per un Nibali la libertà non è fare ciò che si vuole ma il diritto di fare ciò che si deve.

Il Nibali maschio è calvo e dalla pelle di un acceso azzurro, gli occhi sono viola. Le donne hanno la pelle ambrata, capelli castano con riflessi biondi, occhi verdi.

\textbf{Modificatori razziali:} +1 Costituzione, +1 Intelligenza, - 1 Saggezza

\textbf{Caratteristiche fisiche}: altezza 183cm maschi, 170cm femmine, 50 - 120 kg, aspettativa di vita 130 anni

\textbf{Dimensioni}: Medie

\textbf{Velocità}: 9m

\textbf{Linguaggi}: Comune

\textbf{Vantaggio}: Quando passi di livello tiri due volte il dado per determinare i Punti Ferita e prendi il risultato migliore. +2 ai Tiri Salvezza che alterano la forma.
% Ogni round riduci di 1 il danno da Sanguinamento.

\subsection{Diversi}\index{Diversi}\label{diverso}\hypertarget{diverso}{}

\begin{center}
\includegraphics[width=0.7\linewidth]{immagini/diverso2-ai.png}

\emph{(B.I.C.)}
\end{center}

Benedetti o maledetti i Diversi non sono come noi. Un Diverso è frutto di una unione corrotta. Se i Patroni non dovrebbero agire direttamente sulla Terra, o almeno questo è quello che cerca di evitare Gradh, sovente invece usano i loro poteri per creare una stirpe a loro fedele.

Un Diverso è fedele al suo Patrono e non può fare diversamente. Per fortuna sono sterili con le altre razze, altrimenti avrebbero già dominato il mondo.

Un Diverso è più robusto e più intelligente. Purtroppo la loro vita frenetica è segnata da una breve durata. Solitamente un umano Diverso non supera i 50 anni di vita.

Un Diverso è marchiato, da qualche parte sul suo corpo c'è il simbolo, una voglia, del suo Patrono. Molti Diversi hanno 3 o più cerchi concentrici dorati sul polso sinistro che possono indicare il Patrono (o Patroni in rarissimi casi) di cui sono \emph{figli}.

Diverso è un attributo che può essere data a qualsiasi razza. Vengono sostituiti i modificatori razziali con quelli del Diverso ed l'aspettativa di vita viene dimezzata. Rimangono validi i vantaggi razziali originali e si aggiunge quello Speciale del Diverso.

\textbf{Modificatori razziali}: +1 a due Caratteristiche a scelta

\textbf{Caratteristiche fisiche}: altezza come razza originale, aspettativa di vita dimezzata rispetto a razza originale

\textbf{Dimensioni}: come razza originale

\textbf{Velocità}: come razza originale

\textbf{Linguaggi}: come razza originale

\textbf{Speciale}: Deve individuare un Patrono ed avere almeno 3 Tratti comuni. Accede al potere a somma Tratti 5 anche se ha meno punti. Una Abilità aggiuntiva a scelta.

\subsection{Sornelian}\index{Sornelian}\index{Furryman}\label{sornelian}\hypertarget{sornelian}{}

\begin{center}
\includegraphics[width=0.7\linewidth]{immagini/arpia-ai.png}

\emph{Arpia, donna uccello, solitamente molto arrabbiata (predatore/volante)...(B.I.C.)}

\end{center}

La genesi dei Sornelian è dovuta al Patrono \hyperlink{efrem}{Efrem} (pag. \pageref{efrem}). Efrem decise che la natura doveva avere maggiore voce nelle questioni terrestri e stabilì che esistessero animali antropomorfi affinché riequilibrassero la strapotenza delle altre creature umanoidi.

Un Sornelian ha una testa simile a quella di un animale antropomorfo ma il corpo è più simile ad un bipede umanoide. A seconda dell'animale il Sornelian potrebbe anche avere pelliccia, piume, squame ed artigli. Le dimensioni di un Sornelian dipendono molto dall'animale originario variando dalla taglia piccola a quella media. L'aspetto antropomorfo di un Sornelian è vario quanto sono vari gli animali a cui assomigliano.

Un Sornelian quasi mai nasce quale figlio di due Sornelian bensì è una spontanea \emph{mutazione} in grembo di una coppia di umanoidi, come nani, elfi o umani.

\textbf{Modificatori razziali}: +1 ad una Caratteristica a scelta

\textbf{Caratteristiche fisiche}: l'aspettativa di vita dipende dalla longevità della specie, solitamente attorno ai 60+6d10 anni.

\textbf{Dimensioni}: dipende dalla specie originale, dai 50cm ai 220cm, da taglia piccola a media.

\textbf{Velocità}: 6 metri

\textbf{Linguaggi}: Comune. Ottiene +1d6 alle prove per interagire con gli animali della sua linea di sangue.

\textbf{Vantaggi}: Alla creazione il giocatore sceglie 2 capacità tra quelle elencate che più caratterizzano il suo Sornelian. Sono indicati tra parentesi alcuni animali di esempio.

- \emph{Corazzato} (tartaruga, armadillo, granchio, pesce scatola cornuto, alligatore). Hai buona parte del corpo ricoperto da una robusta corazza. La tua Difesa naturale è 12. Se scegli due volte questa capacità gli avversari non hanno vantaggi al Tiro per Colpire alle spalle o quando ti fiancheggiano.

- \emph{Corridore} (cervo, levriero, saurovallo, velociraptor). Aumenti la tua velocità di Movimento di 3 metri. Se scegli due volte questa capacità il tuo Movimento diventa di 12 metri.

- \emph{Creatura notturna} (gatto, lucertola, pipistrello, gufo). Hai visione crepuscolare 9 metri. Se scegli due volte questa capacità la visione crepuscolare arriva fino a 18 metri.

- \emph{Nuotatore} (coccodrillo, delfino, rana, squalo). Puoi trattenere il respiro fino a 1 Turno per punto di Costituzione, minimo 1 Turno, hai una velocità di nuoto pari a metà del tuo Movimento. Hai Riduzione al danno da freddo pari a 4. Se scegli due volte questa capacità hai delle rudimentali branchie che ti permettono di respirare sott'acqua, la riduzione al danno da freddo diventa a 10.

- \emph{Predatore} (orso, felino). I tuoi attacchi naturali (artigli, fauci..) causano 1d6 di danno letale e non sono armi improvvisate. Questi attacchi ricadono nella Lista d'Armi Scuri ed Accette. Se scegli due volte questa capacità il tuo attacco naturale causa 1d8 di danno.

- \emph{Robusto} (rinoceronte, ippopotamo, elefante). Ad ogni passaggio di livello tiri il d8 invece che il d6 per determinare i Punti Ferita. Se scegli due volte questa capacità ogni punto di CA assegnato aumenta di 5 i Punti Ferita e non 3.

- \emph{Scalatore} (orso, gatto, lucertola, scoiattolo). Hai artigli uncinati, unghie affilate o una coda serpentina. Hai una velocità di arrampicata pari alla metà del tuo Movimento. Se scegli due volte questa capacità la velocità di scalata è pari al tuo Movimento.

- \emph{Sensi Eccellenti (udito, vista, olfatto...)} (cane, pipistrello, gufo). Hai un +2 di bonus alle prove di Consapevolezza basate sui sensi. Se scegli due volte questa capacità il bonus diventa +1d6.

- \emph{Volante} (pipistrello, aquila, gufo, corvo). Hai ali rudimentali. Quando cadi da almeno 3 metri puoi usare una Reazione per planare ed atterrare in sicurezza, come incantesimo \hyperlink{cadutapiuma}{Caduta Piuma} (pag. \pageref{cadutapiuma}), senza subire danni da caduta. Quando esegui una prova di Salto in Lungo o in Alto tiri 1d6 in più. Se scegli due volte questa capacità puoi volare per (CM+CA)/3 minuti, ad intervalli minimi di 1 minuto, al giorno.

\subsection{Golian}\index{Golian}\index{Uomini gigante}\label{golian}\hypertarget{golian}{}

I Golian, al pari dei Sornelian, discendono dalla volontà dei Patroni \hyperlink{erondil}{Erondil} (pag. \pageref{erondil}) e \hyperlink{gaya}{Gaya} (pag. \pageref{gaya}) ovvero dal desiderio di avere delle creature che potessero rappresentare i maestosi giganti, i loro piccoli figli.

I Golian hanno Caratteristiche fisiche che ricordano i giganti delle loro linee familiari. Alcuni Golian hanno la pelle grigia o quasi marmorizzata come i giganti delle pietre, altri sprizzano scintille schioccando le dita come i giganti del fuoco, altri ancora hanno le pelle azzurra come i giganti del cielo.

\textbf{Modificatori razziali:} +2 a Forza, -1 a una Caratteristica a scelta

\textbf{Caratteristiche fisiche}: alti circa 180/210cm. Aspettativa di vita circa 80 anni (60+2d10)

\textbf{Dimensioni}: taglia media

\textbf{Velocità}: 9 metri

\textbf{Linguaggi}: Comune, Gigante della loro stirpe.

\begin{center}
	\includegraphics[height=0.8\linewidth]{immagini/Herakles_Farnese_MAN_Napoli_Inv6001_n01.png}

	\medskip

	\emph{Ercole Farnese, Museo Archeologico Nazionale Napoli}

\end{center}

\textbf{Forma Grande}: a partire da CM+CA almeno 5 il Golian ha la capacità di ingrandirsi e diventare di taglia grande per un minuto al giorno ogni (CM+CA)/5, al costo di 2 Azioni, in intervalli minimi di 1 minuto, al termine dei quali il Golian aumenta il suo livello di affaticamento di 1. Mentre di taglia grande i Tiri per Colpire ed il danno basati su Forza aumentano di 1d6, la velocità di Movimento aumenta di 1 metro, la portata aumenta a 2 metri.

\textbf{Stabile}. Sei considerato di taglia Grande per resistere alle prove per essere \hyperlink{spingereavversario}{afferrato} o \hyperlink{spingereavversario}{spinto}.

\textbf{Vantaggi}: Ogni Golian discende da una linea di giganti e da questa eredita dei poteri peculiari. Il potere indicato è usabile (CM+CA)/3, arrotondato per eccesso, al giorno.

- \emph{Gigante delle nuvole}. Un passo nel cielo. Con il costo di due Azioni ti teletrasporti magicamente fino a 10 metri a un spazio non occupato che puoi vedere.

- \emph{Gigante di fuoco}. Braci ardenti. Quando colpisci un bersaglio in mischia puoi infliggere 1d10 danni da fuoco a quel bersaglio. Costo 1 Reazione.

- \emph{Gigante del gelo}. Gelo profondo. Quando colpisci un bersaglio in mischia puoi infliggere 1d6 di danni da freddo e la velocità di movimento della creatura diminuisce di 3 metri fino alla fine del tuo prossimo round. Costo 1 Reazione.

- \emph{Gigante della collina}. Colpo rabbioso. Tiri un 1d6 in più quando effettui il Tiro per Colpire. Costo 1 Azione Immediata da dichiararsi prima del Tiro per Colpire.

- \emph{Gigante di pietra}. Pelle di pietra . Quando subisci un danno puoi indurire la tua pelle fino a farla diventare di pietra. Riduci i danni subiti di (CA o CM + Costituzione)/2. Costo 1 Reazione.

- \emph{Gigante della Tempesta}. Risonanza del Tuono. Quando subisci danni in mischia puoi emettere un onda d'urto che causa 1d10 di danno da Suono o Elettricità a chi ti ha causato danno. Costo 1 Reazione.

\subsection{Sulian}\index{Sulian}\label{sulian}\hypertarget{sulian}{}

\begin{center}

\includegraphics[width=0.7\linewidth]{immagini/sulian4-ai.png}

\emph{(B.I.C.)}

\end{center}

%\begin{center}
%\includegraphics[width=0.7\linewidth]{immagini/Undine_Rising_from_the_Waters.png}
%
%\emph{Undine Rising from the Waters, ca. 1880-1892, by Chauncey Bradley Ives (1810-1894), in the Yale University Art Gallery}
%\end{center}

L'origine dei Sulian non è chiara, alcuni li fanno discendere dagli spiriti elementali altre voci, meno insistenti, dicono che sono figli del Patrono \hyperlink{ledyal}{Ledyal o Laydel} (pag. \pageref{ledyal}) a causa del loro mutevole aspetto e carattere.

Nei Sulian scorre la potenza, l'energia e la vitalità degli elementi, possa essere un unico tipo oppure più elementi.

I Sulian sono molto simili agli umani ma nei loro occhi e spesso sulla loro pelle si vede scorrere l'energia primordiale che li caratterizza.

\textbf{Modificatori razziali:} +1 ad una Caratteristica a propria scelta

\textbf{Caratteristiche fisiche}: alti circa 150-190cm. Aspettativa di vita circa 180 anni (160+2d10)

\textbf{Dimensioni}: taglia media

\textbf{Velocità}: 9 metri

\textbf{Linguaggi}: Comune. Possono comprendere il linguaggio elementale della loro linea di sangue ma non sanno parlarlo.

\textbf{Vantaggi}: Ogni Sulian discende da una linea o più linee elementali e da questa ereditano poteri e capacità uniche. Al primo punto di CA o CM assegnato e poi ogni ottavo punto assegnato totale (CM+CA=1,8,16...), il Sulian potenzia la sua linea di sangue elementale e selezione un potere oppure sblocca un altra linea elementale presente in lui per scegliere poteri diversi.

Il potere indicato è usabile (CM+CA)/3 al giorno.

- \emph{Scarica Primordiale}: il Sulian può al costo di 1 Reazione quando è colpito o colpisce in mischia scaricare parte della sua energia elementale. Il danno è pari a 2d6 per volte che questo potere è stato selezionato.

- \emph{Accesso alla Lista di Magia}: tramite questo potere il Sulian può accedere ad una Lista Elementale. Ogni volta che prende questo potere conosce spontaneamente fino a 3 incantesimi in quella lista con un livello massimo di incantesimo pari alle volte che si è preso questo potere nella medesima lista -1 (la prima volta lanci solo trucchetti).
Il Sulian non esegue Prove di Magia ne si può considerare Distratto quando lancia l'incantesimo. Per eventuale fattori si considera che la CM sia pari alla somma di CM+CA e Adepto della Magia sia stato preso un numero pari alle volte che si è stato preso questo potere.

- \emph{Resistenza Elementale}: tramite questo potere il Sulian acquisisce Resistenza all'elemento scelto.

\end{multicols}

\index{Razze}\index{Razza}


\begin{giocatore}[Nota sulle Razze]
Nessuna descrizione di una razza potrà mai imbrigliare e sottomettere un personaggio. Ogni giocatore è libero di creare il personaggio della razza preferita (concessa dal Narratore) e descriverlo, inquadrarlo, sentirlo, renderlo vivo come più gli piace.

Non limitatevi alle descrizioni qui proposte, vogliono essere solo spunti, non sentitevi limitati nelle scelte perché la descrizione della razza dice questo o quello.
Fate nascere i più belli e completi personaggi possibili. Ogni personaggio è vivo ed è una persona e come tale sarà sempre diverso l'uno dall'altro, ognuno fantastico in maniera diversa a discapito di qualsiasi razza e pregiudizio.
\end{giocatore}

\begin{giocatore}[Nota sul Sesso dei personaggi]\index{Sesso}
Casomai foste così ottusi ribadisco che non c'è differenza di capacità o caratteristiche in base al sesso. Ogni giocatrice e giocatore è invitato a fare il personaggio del genere che preferisce.
\end{giocatore}

\begin{enfasi}
Non dimenticare, nessun altro vede il mondo come lo vedi tu, quindi nessun altro può raccontare le storie che devi raccontare tu. (Ursula K. Le Guin)
\end{enfasi}

\vfill

\begin{center}
\includegraphics[keepaspectratio,width=0.9\textwidth]{immagini/Dragon_by_Henry_Justice_Ford_grey.png}

\emph{The End of the Dragon - Henry Justice Ford}
\end{center}

\pagebreak

\section{Caratteristiche Speciali}

\begin{enfasi}{
Non basta avere gli occhi per vedere (anonimo)
}\end{enfasi}

\begin{multicols}{2}

Ogni creatura è speciale ed unica eppure ci sono esseri ancora più unici e speciali per le loro caratteristiche. Queste sono le peculiarità di alcune di queste.

\subsection{Visione Crepuscolare}\index{Visione Crepuscolare}\label{visionecrepuscolare}

Quello che per molti é oscurità per chi ha \hypertarget{visionecrepuscolare}{visione crepuscolare} é vedere bene purché ci sia una fonte minima di luce.

La visione crepuscolare è una visione a colori.
Un incantatore dotato di visione crepuscolare può leggere una Pergamena fino a quando ha accanto come fonte di luce anche la più smorta delle candele.

I personaggi dotati di visione crepuscolare possono vedere all'esterno nelle notti illuminate dalla luna come se si trovassero alla luce del giorno.
Nella assoluta mancanza di luce la visione crepuscolare non aiuta, rimane buio pesto impenetrabile.

\subsection{Scurovisione}\index{Scurovisione}\label{scurovisione}

La Scurovisione è la capacità straordinaria di vedere senza fonti di luce, fino ad una distanza massima indicata per ogni creatura.

La Scurovisione è solo in bianco e nero (non consente alla creatura di distinguere i colori). Non permette ai personaggi di vedere nulla che non possano altrimenti vedere: gli oggetti Invisibili sono ancora Invisibili, e le Illusioni sono ancora visibili per quello che sembrano essere.

Alla stessa maniera, la Scurovisione rende una creatura soggetta agli attacchi con lo sguardo normalmente. La presenza di luce non altera la Scurovisione.
Effettuare una prova di Sopravvivenza per cercare trappole o di Consapevolezza solo visiva prende un -2 di penalità.

\subsection{Fiuto}\index{Fiuto}\label{fiuto}

Questa qualità speciale permette ad una creatura di sfruttare l'olfatto per individuare i nemici nascosti o in avvicinamento e di seguire le tracce. Le creature dotate di fiuto possono identificare con l'olfatto gli odori familiari come gli umani fanno con quello che vedono.

La creatura può individuare le creature entro 6 metri di distanza con l'olfatto. Se l'avversario è sottovento, il raggio aumenta a 18 metri; se è sopravento, il raggio diminuisce a distanza di 3 metri.
Gli odori più forti, come il fumo, spazzatura o corpi in decomposizione, possono essere individuati al doppio del raggio sopra indicato.

Quando una creatura individua un odore, non viene rivelata l'esatta posizione della sua fonte, ma solo la sua presenza entro il raggio d'azione. La creatura può utilizzare un'Azione per individuare la direzione da cui proviene l'odore. Quando si trova a distanza di mischia dalla fonte, ne individua la posizione.

Una creatura dotata di fiuto può seguire tracce utilizzando l'olfatto, effettuando una prova di Seguire Tracce per trovare e seguire una traccia. La tipica DC di una traccia fresca è 10 (a prescindere dalla superficie su cui si trova la traccia). La DC aumenta o diminuisce a seconda dell'intensità della traccia, del numero di creature che la lasciano e del tempo trascorso da quando è stata lasciata. Per ogni ora trascorsa la DC aumenta di 2.

\begin{center}
\includegraphics[width=0.7\linewidth]{immagini/mostro.png}

\emph{John D. Batten}
\end{center}

Per il resto, questa capacità segue le regole della competenza Sopravvivenza. Le creature che seguono tracce con il fiuto ignorano gli effetti delle superfici su cui si trova la traccia e della scarsa visibilità.

L'acqua, e in particolare l'acqua corrente, nega la capacità di seguire tracce delle creature.

Alcuni forti odori possono facilmente mascherarne altri. La presenza di un odore simile rende impossibile individuare o identificare esattamente una creatura mediante il Fiuto; la DC base della competenza Sopravvivenza per seguire tracce in presenza di odori coprenti passa da 10 a 20.

\subsection{Vista Cieca}\index{Vista Cieca}\label{vistacieca}

Utilizzando un insieme di sensi diversi dalla vista, come la percezione delle vibrazioni, un fiuto sensibile, un udito acuto o un sonar, una creatura dotata di vista cieca si muove e combatte bene quanto una creatura dotata della vista.

Invisibilità e buio sono indifferenti anche se la creatura dotata di vista cieca deve avere una linea di effetto per notare una determinata creatura o oggetto.

Una creatura con copertura continua comunque ad avere il suo vantaggio alla Difesa.

Il raggio della capacità è indicato nella descrizione della creatura. La creatura, in genere, non deve effettuare prove di Consapevolezza per notare creature entro il raggio della sua vista cieca.

A meno che non sia diversamente indicato, la vista cieca è sempre attiva e la creatura non deve compiere azioni per attivarla. Alcune forme di vista cieca devono essere attivate come Azione Immediata. In questo caso, viene indicato nella descrizione della creatura.

Una creatura eterea non è visibile alla vista cieca.

\subsection{Visione del Vero}\index{Visione del Vero}\label{cap Visione del Vero}\hypertarget{cap Visione del Vero}{}

Una creatura con Visione del Vero può, entro il raggio indicato, vedere nell'oscurità normale e magica, vedere creature e oggetti invisibili, rilevare automaticamente le illusioni visive e superare i Tiri Salvezza contro di esse. Percepisce la forma originale di un mutaforma o di una creatura trasformata dalla magia. La creatura con Visione del Vero può vedere nel Piano Eterico.

\medskip

\begin{center}
\includegraphics[height=0.65\linewidth]{immagini/grabroid.png}

\medskip

\emph{Grabroid. Conosciuti anche come Agguantatori. Tremors (Film)}
\end{center}

\subsection{Senso Tellurico}\index{Senso Tellurico}\label{sensotellurico}
Una creatura dotata di Senso Tellurico è sensibile alle vibrazioni del suolo e può automaticamente individuare qualsiasi cosa sia in contatto con il terreno entro il raggio specificato dal Senso Tellurico.

Le Creature Acquatiche dotate di Senso Tellurico (ecolocalizzazione) possono percepire la posizione di creature in contatto con l'acqua.

Il raggio della capacità è specificato nel testo descrittivo della creatura.

\end{multicols}

\vfill

\begin{center}
	\includegraphics[width=0.75\linewidth]{immagini/argus2.png}

	\emph{Argus Panoptes Guarding the Heifer (Io), Red Figure pitcher, c. 460 BC Museum of Fine Arts, Boston}
\end{center}

\pagebreak

\section{Le Caratteristiche}\index{Caratteristiche}

\begin{enfasi}{
Vivere non è respirare: è agire, è fare uso degli organi, dei sensi, delle facoltà, di tutte quelle parti di noi stessi per cui abbiamo il sentimento di esistere. (Jean-Jacques Rousseau)
}\end{enfasi}

\begin{multicols}{2}

Ogni personaggio ha 6 Caratteristiche (chiamate anche Statistiche) che rappresentano i suoi attributi base e costituiscono il suo potenziale talento e capacità innata.

Anche se non è comune che un personaggio effettui una prova usando soltanto una sua Caratteristica, i punteggi di Caratteristica influiscono praticamente su ogni aspetto delle capacità e competenze del personaggio.

\subsection{Descrizione delle Caratteristiche}\label{decrizionedellecaratteristiche}

Il punteggio delle Caratteristiche non è tutto in un personaggio ne tanto meno in un mostro.

I mostri più \emph{istintivi} ed aggressivi avranno probabilmente punteggi negativi di Intelligenza, ma non per questo sono \emph{stupidi}, semplicemente agiscono in base ai loro schemi naturali. Allo stesso tempo creature con basso valore di Costituzione non staranno per morire ma sono solo \emph{fragili}.

\subsubsection{Forza}\index{Forza}\label{forza}

\begin{enfasi}{
Ah, è cosa eccellente possedere la forza d'un gigante, ma usarla da gigante, è tirannia! (William Shakespeare, Isabella: da "Misura per misura", atto II, scena II)
}\end{enfasi}

La Forza misura la potenza fisica, l'atletismo e i limiti della forza bruta che puoi esprimere. La Forza si applica nel danno di mischia e per le armi tirate a mano.

Una prova di Forza può essere impiegata per qualsiasi tentativo di sollevare, spingere, tirare o spaccare qualcosa, per spingere il tuo corpo all'interno di uno spazio, o una qualsiasi altra applicazione di forza bruta.

Un mostro con Forza -4 non è prossimo a morire, semplicemente ha pochissima forza (immaginate di dare un valore di Forza ad un topo od uno scoiattolo se non ad un piccolo ragno...)

Un personaggio con un punteggio di Forza pari a -5 è morto.

\subsubsection{Destrezza}\index{Destrezza}\label{destrezza}

\begin{enfasi}{
Abbaiare stanca. La forza non conta niente nella vita. Saper schivare è quello che conta. (Daniel Pennac)
}\end{enfasi}

La Destrezza misura l'agilità, i riflessi, l'equilibrio ed il coordinamento; determina la Difesa ed i Tiri per Colpire con Armi da Lancio.

Una prova di Destrezza può essere impiegata per qualsiasi tentativo di muoversi agilmente, per evitare di perdere l'equilibrio o borseggiare.

Un personaggio con un punteggio di Destrezza pari a -5 è incapace di muoversi ed è completamente immobile (ma non privo di sensi).

\subsubsection{Costituzione}\index{Costituzione}\label{costituzione}

\begin{enfasi}{
Un pò di salute ogni tanto è il miglior rimedio per l'ammalato. (Friedrich Nietzsche)
}\end{enfasi}

La Costituzione misura la salute, il vigore e la forza vitale nonché la resistenza agli sforzi.

Una prova di Costituzione può essere impiegata per i tentativi di spingerti oltre i normali limiti del tuo corpo e per prove di resistenza e durata.

Un personaggio con Costituzione -5 non ha più il controllo del proprio corpo ed è morto.

\subsubsection{Intelligenza}\index{Intelligenza}\label{intelligenza}

\begin{enfasi}{
La forza senza intelligenza rovina sotto il suo stesso peso. (Orazio)
}\end{enfasi}

L'Intelligenza misura l'acume mentale, l'accuratezza dei ricordi e la capacità di ragionare.
Una prova di Intelligenza entra in gioco quando hai bisogno di affidarti alla logica, le conoscenze, la memoria o le capacità deduttive.

Le tue prove di Intelligenza misurano la tua capacità di ricordare informazioni su incantesimi, oggetti magici, simboli esoterici, tradizioni magiche, i piani dell'esistenza e gli abitanti di quei piani. Rovistare tra antiche pergamene alla ricerca di un frammento di conoscenza potrebbe richiedere una prova di Intelligenza.

Un personaggio con un punteggio di Intelligenza pari a -5 è in stato di coma.

\subsubsection{Saggezza}\index{Saggezza}\label{saggezza}

\begin{enfasi}{
La forza non deriva dalla capacità fisica. Deriva da una volontà indomita. (Mahatma Gandhi)}\end{enfasi}

La Saggezza riflette la tua sintonia con il mondo circostante e rappresenta la perspicacia, l'intuito, la forza di volontà ed il buon senso.

Una prova di Saggezza riflette uno sforzo per interpretare il linguaggio corporeo, comprendere i sentimenti di qualcuno, notare dettagli dell'ambiente o curare una persona ferita.

Un personaggio con un punteggio di Saggezza pari a -5 è incapace di pensiero razionale ed è privo di sensi.

\subsubsection{Carisma}\index{Carisma}\label{carisma}

\begin{enfasi}{
Kogami, tu sai che cos'è il carisma?

- Per come la vedo io, è un'attitudine innata, come quella di un eroe o di un leader.

- [...] Gli elementi che identificano il carisma sono tre: l'indole innata degli eroi e dei profeti, la capacità di infondere benessere agli altri con la sola presenza e una cultura che ti permetta una conversazione brillante su ogni argomento. (Psycho-Pass)
}\end{enfasi}

Il Carisma misura la tua capacità di interagire efficacemente con il prossimo. Comprende fattori come la sicurezza e l'eloquenza, può rappresentare una personalità affascinante o autoritaria.

Una prova di Carisma può essere richiesta quando cerchi di influenzare o intrattenere altre persone, quando cerchi di fare impressione o raccontare una menzogna, o quando devi barcamenarti in una complicata situazione sociale.

Il punteggio di Carisma influenza il numero di \emph{tizi} che conosci. Vedi \hyperlink{ioconoscountizio}{Io conosco un tizio....}.

\begin{center}
\includegraphics[width=0.7\linewidth]{immagini/dice4.png}
\end{center}

Tipiche situazioni di utilizzo del Carisma includono tentativi di raggirare una guardia, truffare un mercante, guadagnare soldi al gioco d'azzardo, farsi passare per qualcun altro grazie a un travestimento, fugare i sospetti di qualcuno con false rassicurazioni o mantenere un volto imperturbabile mentre si racconta una lampante menzogna.

Un personaggio con un punteggio di Carisma pari a -5 è privo di sensi.

\subsubsection{Leggere i punteggi delle Caratteristiche}\index{Leggere i punteggi delle Caratteristiche}\label{leggereipunteggidellecaratteristiche}

Ogni punteggio di Caratteristica in genere va da 0 a 3, un punteggio di Caratteristica buono è 1, 2 è ottima, 0 è "normale", 3 è giudicato \emph{eccezionale}.

Un punteggio di -1 e giudicato debole, un -2 subnormale, un -3 severamente problematico, un -4 porta quasi ad un non utilizzo della caratteristica, un -5 è opportuno che stia nel letto e basta (se non è già in una bara).

\subsubsection{Opzionale - Eta' del personaggio}\index{Opzionale - Età dei personaggio}\hypertarget{etadelpersonaggio}{} \label{etadelpersonaggio}

L'età del personaggio influisce sulle Caratteristiche fisiche e mentali.

\medskip

\noindent\begin{tabularx}{\linewidth}{llllll}
	\toprule
\rowcolor{gray!20}\textbf{Periodo} & \textbf{FOR} & \textbf{DES} & \textbf{COS} & \textbf{INT} & \textbf{SAG}\\
\toprule
Giovane& & & +1 & & -1 \\
\rowcolor{gray!20}Adulto& & &  & & \\
Maturo& & & -1 & & +1 \\
\rowcolor{gray!20}Anziano& -2 & -1 & -1 & +1 & +1 \\
Venerabile&-1 & -1 & -1 & -1 & +1 \\
\end{tabularx}

\medskip

I modificatori indicati si cumulano.

\subsection{I punteggi delle Caratteristiche} \hypertarget{assegnazione.punteggi.caratteristica}{}\label{assegnazionepunteggicaratteristica}

I punteggi delle Caratteristiche hanno un ruolo importante ma non fondamentale. Il giocatore deve capire che un punteggio basso non significa avere un pessimo personaggio, ma bensì si divertirà di più nel ruolarlo facendo leva sulle competenze, Abilità e capacità peculiari, usando ingegno e arguzia. Sono presentati più sistemi per tirare le Caratteristiche.

Personalmente suggerisco l'approccio della \textbf{Modalità Base}. In OBSS i personaggi non sono eroi, non sono i prescelti che si ergono a difensori del pianeta. I personaggi sono persone normali coinvolte spesso loro malgrado in situazioni al limite se non oltre la sopravvivenza.

L'indubbio vantaggio di tirare i valori in ordine delle Caratteristiche è che permette di scombinare gli schemi ed evitare \emph{build} fatte a tavolino.

E' probabile che non vengano i risultati che speravate od addirittura siano venuti in Caratteristiche che non vi interessavano. Va bene così. Cambiate idea, fatevi ispirare dai valori ottenuti! Divertitevi con il nuovo personaggio, costruite qualcosa di nuovo e diverso, lasciatevi stupire.

I \textbf{tiri per le Caratteristiche sono eseguiti in ordine}, quindi il primo tiro è per la Forza, poi per la Destrezza, Costituzione, Intelligenza, Saggezza ed infine Carisma.

In ultimo ricordate che OBSS è un gioco di ruolo dove la morte del personaggio capita, anche più spesso che in altri Giochi di Ruolo. Create dei validi e concreti personaggi e lasciate che sia l'avventura a forgiare i dettagli.

\textbf{\emph{I modificatori razziali o di background non possono alzare o abbassare i punteggio oltre +4/-4}}.

\subsubsection{Modalita' base}\index{Caratteristiche - Modalità base}\label{modalitabase}

Il giocatore tira 3d6 per ogni caratteristica ed in ordine, può ritirare una sola volta un 1 tirato per terzina (3d6). Tira poi una settima terzina che può sostituire ad un'altra terzina. Per ogni caratteristica tirata controlla la somma dei dadi tirati con la \textbf{Tabella: Tiro Caratteristiche}.

Il personaggio così generato acquisisce gratuitamente l'Abilita \hyperlink{Duro a morire}{Duro a morire} (pag. \pageref{Duro a morire}).

\subsubsection{Modalita' della Tradizione}\index{Caratteristiche - Modalita' della tradizione}\label{modalitadellatradizione}

Ogni giocatore tira 4d6 per 6 volte e somma i migliori 3 risultati ogni volta. Il risultato ottenuto viene controllato con la \textbf{Tabella: Tiro delle Caratteristiche} ed assegnato alle Caratteristiche a piacere.

\subsubsection{Modalita' opzionale (per i codardi!)}\index{Caratteristiche - Modalità opzionale per i codardi}\label{modalitapericodardi}

Ogni giocatore distribuisce 4 punti tra le 6 Caratteristiche, ogni Caratteristica deve avere come minimo un punteggio di -1 e come massimo 2 prima dei modificatori razziali.

\subsubsection{Tabella: Tiro delle Caratteristiche}\index[Tabelle]{Tabella Tiro delle Caratteristiche}

La somma dei dadi tirati per le Caratteristiche va confrontata con questa tabella per determinare i valori effettivi della Caratteristiche.

\medskip

\noindent\begin{tabularx}{\linewidth}{ll|ll}
\toprule
\rowcolor{gray!20}\textbf{Val. tirato} & \textbf{Caratt.} & \textbf{Val. tirato} & \textbf{Caratt.}\\
\toprule
3 (o meno)	&	-3	&	12-13-14	&	+1\\
\rowcolor{gray!20}4-5			&	-2	&	15-16-17	&	+2\\
6-7-8		&	-1	&	18 (o più)	&+3\\
\rowcolor{gray!20}9-10-11		&	+0	&	&\\
\end{tabularx}

\medskip

\textbf{Applicate i modificatori razziali!}

\medskip

\begin{giocatore}[Tiriamo le Caratteristiche di Tups]
		\textbf{Prima terzina}: \st{1},1,4,3 totale 8. Forza è -1

		\textbf{Seconda}: 5,6,6 totale 17. Destrezza è +2

		\textbf{Terza}: \st{1},2,1,4 totale 7. Costituzione è -1

		\textbf{Quarta}: 6,6,6 totale 18. Intelligenza è +3

		\textbf{Quinta}: 3,4,2 totale 9. Saggezza è +0

		\textbf{Sesta}: 3,4,4 totale 11. Carisma è +0

		\textbf{Settima}: 3,5,2 totale 10. Che sostituisce Forza (da -1 a +0)

		Come Umano \hyperlink{diverso}{Diverso}, Tups prende +1 in Costituzione e +1 in Intelligenza, portando il valore di Costituzione a 0 e quello di Intelligenza a +4.

\end{giocatore}

\subsection{Aumentare le Caratteristiche}\label{aumentarelecaratteristiche}\hypertarget{aumentarelecaratteristiche}{}\index{Caratteristica Potenziata}

Le Abilità hanno tutte segnate due \emph{\textbf{Caratteristiche Potenziata}}, quando segnate l'Abilità nella scheda sceglietene una sola. Ogni 4 Abilità che potenziano la medesima Caratteristica questa aumenta di 1 il suo valore.

Per aumentare la Caratteristica oltre a 4 sono necessari oggetti magici o incantesimi. L'aumento di Caratteristica ha effetto retroattivo solo per gli aumenti di Costituzione, andando a ripercuotersi sui Punti Ferita massimi.

L'aumento di Caratteristica applica immediatamente il modificatore alle Prove di Competenza, Tiri Salvezza, Tiri per Colpire, Iniziativa e Magia. L'aumento dell'intelligenza si ripercuote nel successivo livello per quanto riguarda l'aumento delle Competenze acquisite.

\medskip

\begin{center}
	\includegraphics[width=0.9\linewidth]{immagini/guerrieroispirato.png}

	\emph{Brian Boru, High King of Ireland}
\end{center}

\begin{narratore}[Non è tutto nelle Caratteristiche]
I giocatori comunque si lamenteranno delle Caratteristiche tirate, è normale, specialmente i giocatori più inesperti. Cercate di fargli capire che non deve limitarsi a guardare le Caratteristiche ma vedere l'insieme generale del personaggio. Suggeritegli Abilità che possano aiutarlo a sopperire ai valori delle Caratteristiche.
\end{narratore}

\medskip

\begin{giocatore}[Il Personaggio fa schifo!]\index{Caratteristiche basse}
Avere delle Caratteristiche basse non è la morte del personaggio! Cercate piuttosto di giocare affinché non sia necessario tirare dadi o fare prove! Sforzatevi di essere arguti, intuitivi, propositivi, furbi.. insomma tutto ciò che vi può fare risolvere la situazione senza dover per forza tirare dadi. In OBSS il Narratore premia i giocatori che descrivono e si esaltano in ciò che il personaggio fa!
\end{giocatore}

\end{multicols}

\vfill

\begin{enfasi}
Sono le nostre scelte che mostrano chi siamo veramente, molto più delle nostre capacità. (Albus Silente)
\end{enfasi}

\pagebreak

\begin{multicols}{2}

\section{Punti Ferita}\index{Punti Ferita}\index{Punti Ferita}\index{PF}\label{puntiferita}

\begin{enfasi}{Chi non stima la vita, non la merita. (Leonardo da Vinci)}\end{enfasi}

\medskip

I Punti Ferita rappresentano l'energia vitale del personaggio ma anche l'abilità, fortuna, la capacità del personaggio di resistere e combattere. Finché il personaggio/avversario ha almeno 1 Punto Ferita (PF) combatterà e lotterà al meglio delle sue capacità.

\begin{itemize}[leftmargin=*] \setlength{\itemsep}{0pt}
\item Ogni personaggio parte con 8 Punti Ferita al primo livello + il punteggio della Costituzione.

\item Ad ogni livello, oltre il primo, guadagna 1d6 Punti Ferita + il punteggio della Costituzione. Se il tiro di dado è inferiore a Costituzione, può prendere come risultato il valore di Costituzione.\index{Punti Ferita minimi}

\item Ogni punto preso in Competenza Armi aumenta i Punti Ferita presi di 3. Ulteriori Abilità possono aumentare i Punti Ferita.
\end{itemize}

Segna nella scheda i Punti Ferita massimi che hai ed indica il valore attuale di volta in volta che ne perdi o recuperi. Segna sulla scheda sempre qual é l'ammontare dei Punti Ferita attuale, dopo ogni danno subito. I Punti Ferita Massimi sono l'ammontare di Punti Ferita quando il personaggio è \emph{perfettamente sano}.

%\begin{tcolorbox}[title = Sto per morire!]\index{Sto per morire!}
%SCAPPA! Ritirati, nasconditi, esci dal combattimento. Non c'è gloria nell'essere morto. Meglio una ritirata che un TPK (Total Party Kill ovvero morte di tutto il gruppo).
%\end{tcolorbox}\end{changemargin}

\medskip

I \textbf{Punti Ferita si recuperano} in diversi modi:\index{Recupero Punti Ferita}\label{recuperarepf}

\begin{itemize}[leftmargin=*] \setlength{\itemsep}{0pt}

\item per ogni notte di riposo (almeno 8 ore) recuperi in Punti Ferita il valore di Costituzione*Livello, con un minimo di PF pari a Livello. \index{Recupero PF dormendo}

\item tramite magie curative (incantesimi, pozioni.. o altri effetti magici)

\item competenza \hyperlink{prontosoccorso}{Pronto Soccorso} (pag. \pageref{prontosoccorso}), tramite trattamenti più o meno lunghi

\end{itemize}

I \textbf{Punti Ferita} possono essere anche \textbf{temporanei}\index{Punti Ferita Temporanei} ovvero aggiunti o tolti temporaneamente ai tuoi attuali.

\noindent\begin{itemize}[leftmargin=*] \setlength{\itemsep}{0pt}

\item Una magia che conceda +10 Punti Ferita temporanei alzerà i Punti Ferita attuali di 10, se subisci 8 di danno ti rimarranno 2 Punti Ferita temporanei. Se invece subisci 13 di danno oltre a perdere tutti i Punti Ferita temporanei subirai anche 3 Punti Ferita \emph{normali}.

\item Quando ottieni Punti Ferita temporanei devi scegliere se l'effetto sostituisce il precedente. I Punti Ferita temporanei non si cumulano e non possono essere superiori alla metà dei Punti Ferita massimi. Una magia di cura fa recuperare i PF normali, non i PF temporanei persi.

\item Al termine dell'effetto che concede Punti Ferita temporanei questi scompaiono facendo rimanere la creatura ai suoi Punti Ferita.

\item Se non esplicitato diversamente i Punti Ferita Temporanei scompaiono dopo un ora da quando si sono aggiunti.

\item I Punti Ferita Temporanei vanno tolti per primi quando si viene feriti.

\end{itemize}

Un'arma o effetto che causa danni non letali vuole dire che causa \hyperlink{recuperopuntiferitanonletali}{ferite temporanee}\label{feritetemporanee}.

\section{Punti Fato}\index{Punti Fato}\index{Fortuna del Principiante}\label{puntifato}

\begin{enfasi}{Se il destino è contro di noi, peggio per lui. (motto del 1º Reggimento Carabinieri Paracadutisti "Tuscania")}\end{enfasi}
\medskip

%In un mondo non facile la Fortuna del Principiante aiuta chi non ha esperienza.
Ogni personaggio ha un numero di Punti Fato pari a (20 - Livello)/5, arrotondato all'intero più vicino, con un minimo di 1. I Punti Fato si azzerano e conteggiano per sessione di gioco.
Si recupera un Punto Fato ogni volta che si tirano almeno tre 1 in una prova.\index{Recuperare Punti Fato}

Non costa Azioni usare un Punto Fato e può essere usato per:

\begin{description}[labelwidth=1cm]
	\item[\FatePoint] o più, aggiungere 1d6 ad un Tiro Salvezza, Tiro per Colpire, Prova di Competenza. Da dichiarasi prima del tiro dei dadi. Il dado aggiunto può esplodere secondo le Golden Rules.
	\item[\FatePoint\FatePoint] ritirare 1d6 nelle prove
	\item[\FatePoint] negare un Tiro Critico d'arma subito
	\item[\FatePoint\FatePoint] ritirare completamente una prova
	\item[\FatePoint] trasformare prova fallita criticamente in fallita semplicemente
	\item[\FatePoint] far ritirare un Tiro Salvezza ad un obiettivo
	\item[\FatePoints{3}] tornare a 0 Punti Ferita (tutti i punti disponibili)
	\item[\FatePoint] o più, diminuire di 3 i danni subiti
\end{description}

\subsection*{Opzionale - Punti Caos}\index{Opzionale - Punti Caos}

Un sistema per aggiungere tensione è gestire un insieme di Punti Fato condivisi tra personaggi ed avversari al posto di quelli del singolo giocatore.

Si mette al centro del tavolo un contenitore, una piccola ciotola, con dentro un numero di d6 pari al numero dei personaggi. Ogni giocatore è libero di prendere un dado alla volta ed usarlo come se fossero Punti Fato.

I dadi usati dai giocatori vengono poi spostati in un altro contenitore che il Narratore, sempre massimo uno alla volta per avversario, userà a \emph{suo beneficio}. Una volta che il Narratore ha usato il dado lo rimette nel contenitore dei giocatori.

\end{multicols}

\pagebreak

\section{I Tratti}\index{Tratti}\hypertarget{tratti}{}\label{tratti}

\begin{enfasi}{Chi dunque sa fare il bene e non lo compie, commette peccato. (Giacomo il Giusto 4.17, Lettera di Giacomo)
\smallskip

E' un diritto naturale saziarsi l'anima con la vendetta. (Attila)
\smallskip

Est Sularus Oth Mithas. ("Il mio onore è la mia vita", Giuramento dei Cavalieri di Solamnia)
}\end{enfasi}

\begin{multicols}{2}

\index{Tratti}
In OBSS non c'è una netta distinzione tra bene e male, legge e caos, tra ciò che è giusto e ciò che è sbagliato.

In OBSS esistono i Tratti, aspetti e sfumature caratteriali che \textbf{contribuiscono} al background del personaggio, aiutano il giocatore a ruolare meglio e gli possono fornire quelle linee guida per interpretare in maniera più corretta il personaggio che si è voluto creare.

Un Tratto è un dettaglio che aiuta meglio a inquadrare il personaggio, ne delinea i caratteri principali concedendogli sfumature diverse.

\textbf{Ogni giocatore sceglie 5 Tratti per il proprio personaggio alla creazione del personaggio.} Questi che suggeriranno il personaggio nell'agire e nelle scelte.

\begin{giocatore}[Scegliere i Tratti]
I Tratti non sono il personaggio, non lo bloccano ne lo fissano eterno nel tempo. Un personaggio è sempre in costante evoluzione e così il suo carattere, morale, comportamento e desideri. Non essere rigido ma usa i Tratti per darti dei suggerimenti da cui farti ispirare.
\end{giocatore}

I Tratti non hanno accezione positiva o negativa, servono solo ad inquadrare il personaggio e capire quale Patrono è più interessato al personaggio. Non vogliono definire se sei buono o cattivo, ognuno ha la propria morale indipendentemente dai Tratti posseduti.

\textbf{Al primo livello scegli un Tratto maggiormente caratteristico per il personaggio, questo avrà valore 1, gli altri 4 Tratti avranno valore 0.}

Col passare del tempo e delle avventure i Tratti aumenteranno valore o potranno essere sostituiti, in concerto tra Narratore e giocatore in base a come giocato, da altri Tratti. \textbf{più è alto un valore di Tratto più questo è presente e permeante nelle scelte del personaggio}.

Durante le avventure il Narratore a seguito di particolari scene e recitazione potrà fare aumentare di un punto, o di una frazione di punto, un Tratto del personaggio.

Ad esempio a seguito di una particolare situazione e climax di avventura il Narratore potrebbe concedere a tutti o qualcuno il Tratto Coraggio o dare un +1 a Coraggio a chi ha già questo Tratto. Per i Tratti non presi si considera il valore base in punti di -1, ovvero il primo punto serve per prendere il Tratto ed i successivi per enfatizzarli.

Mentre è \emph{relativamente} facile acquisire nuovi Tratti è complicato cambiare quelli già presenti. Parlane con il Narratore, saprà preparare situazioni ed avventure che ti aiuteranno a comprendere come evolvere il personaggio ed eventualmente ad evolvere i Tratti scelti.

Nella scheda troverai dei \textbf{check} da mettere vicino ai Tratti, questi vengono segnati a seguito di azioni idonee ad accrescere il valore del Tratto; raggiunti i 10 check il Tratto aumenterà di 1 punto e si ricomincerà a segnare una nuova decina.

Sarà il Narratore durante l'avventura a dirti quando segnare, o cancellare, dei punti parziali. \textbf{In linea di massima si presume che un personaggio acquisisca almeno un punto Tratto a livello.}

Ogni azione particolarmente importante dove il personaggio abbia seguito un Tratto porta il personaggio ad avvicinarsi al \textbf{Patrono} competente per quel Tratto.

All'aumentare del valore della somma dei Tratti comuni con il Patrono il personaggio potrà acquisire dei poteri, indipendentemente sia un credente (Seguace o Devoto) o meno di quel Patrono.

\noindent\begin{itemize}[leftmargin=*] \setlength{\itemsep}{0pt}
\item A \textbf{'5'} punti si può incominciare a sentire la presenza di un Patrono

\item A \textbf{'10'} punti si sente la vicinanza di un Patrono

\item A \textbf{'15'} punti si è legati ad Patrono

\item A \textbf{'20'} punti si è un campione del Patrono
\end{itemize}

Non è necessario credere in un Patrono per sentirne la vicinanza, esserne legati o campione, semplicemente è la propria natura (i propri Tratti) che è affine al Patrono, che lo si voglia o meno. I poteri si prendono solo dal Patrono che ha somma tratti più alta rispetto agli altri.

Dato che lo scopo di un Patrono è fare che i propri Tratti siano dominanti sugli altri, avere persone di alto livello e potere che siano così affini a lui tornerà utile nel giudizio dei 100 anni. Usate a vostro vantaggio i Tratti ed il legame che il Patrono instaurerà con voi.

Per individuare il Patrono più affine, quello che vi darà i poteri, verificate il vostro Tratto a maggior valore sulla \hyperlink{tabellacollegamentopatronotratto}{Tabella Collegamento Patrono - Tratto} (pag. \pageref{tabellacollegamentopatronotratto}) ed individuate il Patrono che quel Tratto maggiormente caratterizzante, in caso il Tratto fosse condiviso tra più Patroni verificate gli altri Tratti ed in base alla somiglianza scegliete il Patrono.
Verificate poi in \hyperlink{cosmologia}{Cosmologia} (pag. \pageref{patroni}) i poteri concessi dal Patrono. Questo controllo è opportuno farlo ad ogni aumento di valore di Tratto.

Si è Devoti con almeno 2 Tratti e Seguaci con almeno 1 Tratto in comune con il Patrono. Non si può essere contemporaneamente Seguace o Devoti di più Patroni.

Il Narratore è libero di inserire nuovi Tratti a suo piacere o richiesti dai giocatori, si suggerisce di attribuire questi nuovi Tratti anche ai Patroni.

\medskip

\textbf{Lista dei Tratti}\index[Tabelle]{Tabella dei Tratti}

Ogni Tratto è brevemente descritto nel suo significato generico. Il personaggio è libero di interpretare il Tratto come più lo sente proprio.

\medskip

\noindent\begin{itemize}[leftmargin=*] \setlength{\itemsep}{0pt}
	\item \textbf{Altruista}: Persona che mette gli altri al primo posto, anche sacrificando i propri bisogni.
	\item \textbf{Ambizioso}: Pensa solo ai propri interessi e bisogni, senza considerare quelli degli altri
	\item \textbf{Arrogante}: Ha un'opinione esagerata di sé stesso e tende a sminuire gli altri.
	\item \textbf{Avaro}: Eccessivamente attaccato ai propri beni materiali e riluttante a condividere.
	\item \textbf{Cinico}: Tende a vedere il peggio nelle persone e nelle situazioni, spesso con un atteggiamento sprezzante.
	\item \textbf{Codardo}: Che manca di coraggio e tende a evitare situazioni di pericolo.
	\item \textbf{Compassionevole}: Mostra empatia e comprensione verso le sofferenze degli altri.
	\item \textbf{Coraggioso}: Affronta le paure e le sfide con determinazione.
	\item \textbf{Crudele}: Senza pietà e compassione, provoca sofferenza intenzionalmente.
	\item \textbf{Curioso}: Ha un forte desiderio di conoscere e imparare cose nuove.
	\item \textbf{Disonesto}: Non dice la verità e inganna gli altri per il proprio vantaggio.
	\item \textbf{Dissoluto}: Vive in modo sregolato e senza considerare le conseguenze morali delle proprie azioni.
	\item \textbf{Entusiasta}: Mostra grande energia e passione per ciò che fa.
	\item \textbf{Estroverso}: Socievole e a proprio agio in situazioni sociali.
	\item \textbf{Gentile}: Tratta gli altri con rispetto e considerazione.
	\item \textbf{Impulsivo}: Tende ad agire e reagire senza pensare troppo alle conseguenze.
	\item \textbf{Indeciso}: Non riesce rapidamente a prendere decisione soffermandosi troppo nel ponderare le scelte.
	\item \textbf{Intransigente}: Non è disposto a scendere a compromessi o a considerare punti di vista diversi.
	\item \textbf{Invidioso}: Prova risentimento verso chi possiede qualcosa che lui desidera.
	\item \textbf{Leale}: Fedele e affidabile nei confronti degli amici e delle persone care.
	\item \textbf{Paziente}: Capace di aspettare senza irritarsi o perdere la calma.
	\item \textbf{Prudente}: Pondera attentamente le situazioni difficili o pericolose.
	\item \textbf{Sospettoso}: Sei convito che tutti abbiano interesse a danneggiarti.
	\item \textbf{Testardo}: Determinato e persistente nel raggiungere i propri obiettivi, nonostante le difficoltà.
	\item \textbf{Vanitoso}: Sei certo delle tue eccezionali qualità, capacità ed aspetto.
	\item \textbf{Vendicativo}: Cerca di punire chi gli ha fatto un torto, spesso in modo sproporzionato.

\end{itemize}

\end{multicols}

%valutare le motivazioni, una tabella delle motivazioni

\smallskip

Se il personaggio è completamente difforme ai suoi Tratti non acquisirà punti esperienza.

\vfill

\begin{center}
\includegraphics[height=0.3\linewidth]{immagini/troll.png}
\end{center}

\medskip

\begin{enfasi}{Se un viaggiatore non riporta qualcosa da condividere, non è un \emph{Eroe} ma un impostore, un egoista privo di saggezza. (Il viaggio dell'Eroe, Christopher Vogler)}\end{enfasi}

\pagebreak

\section{Opzionale - Archetipi del Carattere}

\begin{multicols}{2}

Questa opzione presenta un sistema che integra gli archetipi junghiani \autocite{jung1971} con il framework esistente di Tratti del carattere e Patroni. Ispirandosi agli archetipi di Carl Jung e all'indicatore tipologico Myers-Briggs (MBTI) \autocite{myers1995}, vengono presentati 21 distinti pattern archetipici che possono essere utilizzati per la creazione del personaggio, lo sviluppo e la narrazione.

Ogni archetipo è presentato con un insieme di Tratti raccomandati che si allineano naturalmente con quell'energia archetipica. Insieme ai Tratti sono anche riportati quelli  generalmente incompatibili o contraddittori (e per questo stimolanti) alla natura fondamentale dell'archetipo. Inoltre vengono elencati quali Patroni condividono almeno due Tratti (e quindi permettono di essere Devoti) con ogni archetipo, suggerendo affinità spirituali naturali.

\begin{itemize}
\item Scegli un archetipo che ti attrae o si adatta alla tua idea del personaggio
\item Considera di adottare almeno 2-3 dei Tratti raccomandati per quell'archetipo
\item Evita i Tratti sconsigliati a meno che tu non stia specificamente mirando a creare conflitto interno
\item Guarda ai Patroni allineati per una guida su quali poteri spirituali potrebbero essere affini naturalmente con il tuo personaggio
\end{itemize}

Gli archetipi possono anche evolversi durante le avventure di un personaggio. Un personaggio potrebbe iniziare come un archetipo (L'Innocente) e trasformarsi in un altro (L'Eroe) attraverso le sue esperienze. Questa evoluzione può essere riflessa nel cambiamento graduale dei Tratti e delle affinità con i Patroni.

\subsection*{I 21 Archetipi}

\subsection*{L'Eroe}
Il personaggio coraggioso che supera gli ostacoli per raggiungere un obiettivo, spesso trasformando se stesso nel processo.

\noindent\begin{itemize}[leftmargin=*] \setlength{\itemsep}{0pt}
\item \textbf{Tratti Raccomandati:} Coraggioso, Testardo, Ambizioso, Leale
\item \textbf{Tratti Sconsigliati:} Codardo, Indeciso, Disonesto
\item \textbf{Tratti Contraddittori:} Cinico, Crudele, Dissoluto
\item \textbf{Patroni Allineati:} Gradh, Sumkjr, Nedraf, Ljust, Lynx, Orlaith
\end{itemize}

\subsection*{Il Mentore}
La guida saggia che fornisce conoscenza, intuizione e supporto agli altri, spesso all'Eroe.

\noindent\begin{itemize}[leftmargin=*] \setlength{\itemsep}{0pt}
\item \textbf{Tratti Raccomandati:} Paziente, Gentile, Prudente, Compassionevole
\item \textbf{Tratti Sconsigliati:} Impulsivo, Arrogante, Vendicativo
\item \textbf{Tratti Contraddittori:} Crudele, Disonesto, Avaro
\item \textbf{Patroni Allineati:} Ljust, Sumkjr, Thaft, Ledyal, Gaya
\end{itemize}

\subsection*{Il Guardiano della Soglia}
Il personaggio che mette alla prova i giocatori, presentando sfide che devono superare per procedere.

\noindent\begin{itemize}[leftmargin=*] \setlength{\itemsep}{0pt}
\item \textbf{Tratti Raccomandati:} Intransigente, Sospettoso, Prudente, Paziente
\item \textbf{Tratti Sconsigliati:} Impulsivo, Compassionevole, Altruista
\item \textbf{Tratti Contraddittori:} Codardo, Indeciso, Disonesto
\item \textbf{Patroni Allineati:} Atmos, Orlaith, Lynx, Krondal, Sixiser
\end{itemize}

\subsection*{L'Araldo}
Il personaggio che annuncia la chiamata all'avventura e segnala la necessità del cambiamento.

\noindent\begin{itemize}[leftmargin=*] \setlength{\itemsep}{0pt}
\item \textbf{Tratti Raccomandati:} Entusiasta, Estroverso, Curioso, Coraggioso
\item \textbf{Tratti Sconsigliati:} Prudente, Indeciso, Sospettoso
\item \textbf{Tratti Contraddittori:} Codardo, Cinico, Disonesto
\item \textbf{Patroni Allineati:} Nethergal, Sumkjr, Lynx, Nedraf
\end{itemize}

\subsection*{Il Mutaforma}
Il personaggio la cui lealtà e vera natura sono costantemente in discussione.

\noindent\begin{itemize}[leftmargin=*] \setlength{\itemsep}{0pt}
\item \textbf{Tratti Raccomandati:} Disonesto, Arrogante, Impulsivo, Ambizioso
\item \textbf{Tratti Sconsigliati:} Leale, Altruista, Gentile
\item \textbf{Tratti Contraddittori:} Intransigente, Prudente, Paziente
\item \textbf{Patroni Allineati:} Calicante, Orudjs, Tazher, Tàhil, Torbiorn
\end{itemize}

\subsection*{L'Ombra}
Il riflesso oscuro del personaggio, che rappresenta aspetti rifiutati o temuti del sé.

\noindent\begin{itemize}[leftmargin=*] \setlength{\itemsep}{0pt}
\item \textbf{Tratti Raccomandati:} Crudele, Vendicativo, Cinico, Arrogante
\item \textbf{Tratti Sconsigliati:} Compassionevole, Gentile, Altruista
\item \textbf{Tratti Contraddittori:} Leale, Paziente, Entusiasta
\item \textbf{Patroni Allineati:} Calicante, Cattalm, Tàhil, Shayalia, Torbiorn
\end{itemize}

\subsection*{L'Imbroglione}
Il personaggio malizioso che disturba lo status quo e porta trasformazione attraverso il caos.

\noindent\begin{itemize}[leftmargin=*] \setlength{\itemsep}{0pt}
\item \textbf{Tratti Raccomandati:} Impulsivo, Curioso, Disonesto, Entusiasta
\item \textbf{Tratti Sconsigliati:} Prudente, Paziente, Intransigente
\item \textbf{Tratti Contraddittori:} Leale, Altruista, Compassionevole
\item \textbf{Patroni Allineati:} Orudjs, Belevon, Nihar
\end{itemize}

\subsection*{L'Alleato}
Il personaggio fedele che supporta il gruppo nel suo viaggio.

\noindent\begin{itemize}[leftmargin=*] \setlength{\itemsep}{0pt}
\item \textbf{Tratti Raccomandati:} Leale, Coraggioso, Altruista, Gentile
\item \textbf{Tratti Sconsigliati:} Disonesto, Invidioso, Crudele
\item \textbf{Tratti Contraddittori:} Vendicativo, Arrogante, Avaro
\item \textbf{Patroni Allineati:} Ljust, Sumkjr, Efrem, Gradh, Thaft
\end{itemize}

\subsection*{L'Innocente}
Il personaggio puro e ingenuo che vede il mondo con meraviglia e ottimismo.

\noindent\begin{itemize}[leftmargin=*] \setlength{\itemsep}{0pt}
\item \textbf{Tratti Raccomandati:} Gentile, Entusiasta, Altruista, Curioso
\item \textbf{Tratti Sconsigliati:} Cinico, Sospettoso, Vendicativo
\item \textbf{Tratti Contraddittori:} Crudele, Disonesto, Dissoluto
\item \textbf{Patroni Allineati:} Ljust, Ledyal, Sumkjr, Gaya
\end{itemize}

\subsection*{Il Saggio}
Il personaggio custode della conoscenza che ha accumulato saggezza attraverso lo studio o l'esperienza.

\noindent\begin{itemize}[leftmargin=*] \setlength{\itemsep}{0pt}
\item \textbf{Tratti Raccomandati:} Prudente, Paziente, Curioso, Intransigente
\item \textbf{Tratti Sconsigliati:} Impulsivo, Disonesto, Dissoluto
\item \textbf{Tratti Contraddittori:} Indeciso, Codardo, Arrogante
\item \textbf{Patroni Allineati:} Atmos, Nethergal, Sixiser
\end{itemize}

\subsection*{Il Sovrano}
Il personaggio che cerca di stabilire ordine e controllo su suo dominio.

\noindent\begin{itemize}[leftmargin=*] \setlength{\itemsep}{0pt}
\item \textbf{Tratti Raccomandati:} Ambizioso, Intransigente, Arrogante, Testardo
\item \textbf{Tratti Sconsigliati:} Indeciso, Codardo, Impulsivo
\item \textbf{Tratti Contraddittori:} Altruista, Compassionevole, Gentile
\item \textbf{Patroni Allineati:} Calicante, Erondil, Cattalm, Krondal, Tàhil
\end{itemize}

\subsection*{Il Creatore}
Il personaggio costruttore, innovativo, che porta nuove cose all'esistenza.

\noindent\begin{itemize}[leftmargin=*] \setlength{\itemsep}{0pt}
\item \textbf{Tratti Raccomandati:} Curioso, Entusiasta, Ambizioso, Paziente
\item \textbf{Tratti Sconsigliati:} Indeciso, Codardo, Cinico
\item \textbf{Tratti Contraddittori:} Dissoluto, Intransigente, Vendicativo
\item \textbf{Patroni Allineati:} Erondil, Efrem, Gaya, Nethergal
\end{itemize}

\subsection*{Il Custode}
Il personaggio protettore, premuroso, che si prende cura e difende gli altri.

\noindent\begin{itemize}[leftmargin=*] \setlength{\itemsep}{0pt}
\item \textbf{Tratti Raccomandati:} Compassionevole, Altruista, Gentile, Paziente
\item \textbf{Tratti Sconsigliati:} Crudele, Avaro, Vendicativo
\item \textbf{Tratti Contraddittori:} Arrogante, Disonesto, Dissoluto
\item \textbf{Patroni Allineati:} Ljust, Ledyal, Atherim, Belevon, Sumkjr
\end{itemize}

\subsection*{Il Mago}
Il personaggio che sfrutta conoscenze uniche per alterare la realtà.

\noindent\begin{itemize}[leftmargin=*] \setlength{\itemsep}{0pt}
\item \textbf{Tratti Raccomandati:} Curioso, Ambizioso, Arrogante, Intransigente
\item \textbf{Tratti Sconsigliati:} Codardo, Indeciso, Impulsivo
\item \textbf{Tratti Contraddittori:} Altruista, Compassionevole, Leale
\item \textbf{Patroni Allineati:} Erondil, Orudjs, Nethergal, Nihar, Krondal
\end{itemize}

\subsection*{Il Fuorilegge}
Il personaggio ribelle che sfida le norme stabilite e combatte contro i vincoli.

\noindent\begin{itemize}[leftmargin=*] \setlength{\itemsep}{0pt}
\item \textbf{Tratti Raccomandati:} Coraggioso, Impulsivo, Arrogante, Vendicativo
\item \textbf{Tratti Sconsigliati:} Prudente, Paziente, Leale
\item \textbf{Tratti Contraddittori:} Altruista, Compassionevole, Gentile
\item \textbf{Patroni Allineati:} Lynx, Tàhil, Gradh, Tazher, Calicante
\end{itemize}

\subsection*{L'Amante}
Il personaggio cercatore appassionato di connessione, intimità e piacere sensuale.

\noindent\begin{itemize}[leftmargin=*] \setlength{\itemsep}{0pt}
\item \textbf{Tratti Raccomandati:} Compassionevole, Entusiasta, Impulsivo, Dissoluto
\item \textbf{Tratti Sconsigliati:} Sospettoso, Cinico, Arrogante
\item \textbf{Tratti Contraddittori:} Prudente, Intransigente, Avaro
\item \textbf{Patroni Allineati:} Shayalia, Ledyal, Orudjs
\end{itemize}

\subsection*{Il Giullare}
Il personaggio intrattenitore giocoso che porta gioia e leggerezza in situazioni difficili.

\noindent\begin{itemize}[leftmargin=*] \setlength{\itemsep}{0pt}
\item \textbf{Tratti Raccomandati:} Entusiasta, Estroverso, Impulsivo, Curioso
\item \textbf{Tratti Sconsigliati:} Prudente, Sospettoso, Intransigente
\item \textbf{Tratti Contraddittori:} Crudele, Vendicativo, Arrogante
\item \textbf{Patroni Allineati:} Nihar, Belevon, Nethergal, Orudjs
\end{itemize}

\subsection*{L'Uomo Comune}
Il personaggio ordinario che cerca appartenenza e connessione.

\noindent\begin{itemize}[leftmargin=*] \setlength{\itemsep}{0pt}
\item \textbf{Tratti Raccomandati:} Leale, Gentile, Prudente, Indeciso
\item \textbf{Tratti Sconsigliati:} Arrogante, Ambizioso, Dissoluto
\item \textbf{Tratti Contraddittori:} Crudele, Vendicativo, Disonesto
\item \textbf{Patroni Allineati:} Efrem, Thaft, Atherim
\end{itemize}

\subsection*{L'Esploratore}
Il personaggio avventuriero che cerca nuove esperienze e scoperte.

\noindent\begin{itemize}[leftmargin=*] \setlength{\itemsep}{0pt}
\item \textbf{Tratti Raccomandati:} Curioso, Coraggioso, Impulsivo, Entusiasta
\item \textbf{Tratti Sconsigliati:} Prudente, Indeciso, Sospettoso
\item \textbf{Tratti Contraddittori:} Codardo, Cinico, Avaro
\item \textbf{Patroni Allineati:} Lynx, Nihar, Nethergal, Sumkjr
\end{itemize}

\subsection*{Il Martire}
Il personaggio che si sacrifica e dà tutto per una causa o per gli altri.

\noindent\begin{itemize}[leftmargin=*] \setlength{\itemsep}{0pt}
\item \textbf{Tratti Raccomandati:} Altruista, Coraggioso, Intransigente, Testardo
\item \textbf{Tratti Sconsigliati:} Arrogante, Avaro, Ambizioso
\item \textbf{Tratti Contraddittori:} Cinico, Crudele, Disonesto
\item \textbf{Patroni Allineati:} Ljust, Sumkjr, Atherim, Lynx
\end{itemize}

\subsection*{Il Tiranno}
Il personaggio oppressore, controllore, che governa attraverso la paura e la dominazione.

\noindent\begin{itemize}[leftmargin=*] \setlength{\itemsep}{0pt}
\item \textbf{Tratti Raccomandati:} Crudele, Arrogante, Vendicativo, Avaro
\item \textbf{Tratti Sconsigliati:} Compassionevole, Altruista, Gentile
\item \textbf{Tratti Contraddittori:} Indeciso, Codardo, Leale
\item \textbf{Patroni Allineati:} Calicante, Tàhil, Cattalm, Torbiorn, Rezh
\end{itemize}

\subsection*{L'Eremita}
Il personaggio cercatore solitario che si ritira dalla società per trovare la verità interiore.

\noindent\begin{itemize}[leftmargin=*] \setlength{\itemsep}{0pt}
\item \textbf{Tratti Raccomandati:} Prudente, Paziente, Intransigente, Sospettoso
\item \textbf{Tratti Sconsigliati:} Estroverso, Impulsivo, Entusiasta
\item \textbf{Tratti Contraddittori:} Dissoluto, Arrogante, Avaro
\item \textbf{Patroni Allineati:} Atmos, Sixiser, Efrem
\end{itemize}

\subsection*{Cicli Archetipici e Sviluppo del Personaggio}

I personaggi raramente incarnano un singolo archetipo durante l'intero loro viaggio. Piuttosto, possono evolversi attraverso una serie di archetipi mentre crescono e si sviluppano. Di seguito sono riportate alcune progressioni archetipiche comuni che possono servire come modelli per lo sviluppo del personaggio:

\noindent\begin{itemize}[leftmargin=*] \setlength{\itemsep}{0pt}
\item \textbf{Il Viaggio dell'Eroe:} Innocente > Esploratore > Eroe > Saggio > Sovrano
\item \textbf{La Caduta e la Redenzione:} Innocente > Fuorilegge > Ombra > Martire > Eroe
\item \textbf{L'Arco della Corruzione:} Alleato > Mutaforma > Ombra > Tiranno
\item \textbf{Il Sentiero della Saggezza:} Esploratore > Guardiano della Soglia > Eremita > Saggio > Mentore
\item \textbf{L'Evoluzione della Leadership:} Alleato > Eroe > Mentore > Sovrano
\item \textbf{Il Risveglio Spirituale:} Uomo Comune > Guardiano della Soglia > Eremita > Mago > Creatore
\end{itemize}

Ogni transizione tra archetipi rappresenta un momento significativo di sviluppo del personaggio, spesso accompagnato dall'acquisizione o perdita di Tratti specifici e potenzialmente da un cambiamento nell'affinità con i Patroni. Narratori e giocatori possono utilizzare questi cicli archetipici per pianificare archi narrativi significativi del personaggio.

\subsection*{Affinità con i Patroni e Risonanze Archetipiche}

Alcuni Patroni incarnano naturalmente o risuonano con archetipi specifici più fortemente di altri. Queste connessioni possono informare sia lo sviluppo del personaggio che la narrazione:

\medskip

\noindent\begin{tabularx}{\columnwidth}{lX}
	\toprule
\rowcolor{gray!20}\textbf{Patrono} & \textbf{Risonanze Archetipiche} \\
\toprule
Ljust & L'Eroe, Il Custode, Il Martire, Il Mentore \\
\rowcolor{gray!20}Calicante & L'Ombra, Il Tiranno, Il Mutaforma, Il Sovrano \\
Atmos & Il Saggio, L'Eremita, Il Guardiano della Soglia \\
\rowcolor{gray!20}Lynx & L'Esploratore, Il Fuorilegge, L'Eroe \\
Gradh & L'Eroe, L'Alleato, Il Sovrano, Il Fuorilegge \\
\rowcolor{gray!20}Atherim & Il Custode, Il Guardiano, Il Martire \\
Belevon & L'Imbroglione, Il Giullare, Il Mutaforma \\
\rowcolor{gray!20}Cattalm & L'Ombra, Il Tiranno, Il Distruttore \\
Efrem & L'Uomo Comune, Il Creatore, L'Eremita \\
\rowcolor{gray!20}Erondil & Il Creatore, Il Mago, Il Sovrano \\
Gaya & Il Creatore, L'Innocente, Il Custode \\
\rowcolor{gray!20}Krondal & Il Sovrano, Il Guardiano della Soglia, Il Mago \\
Ledyal & Il Custode, L'Innocente, L'Amante \\
\rowcolor{gray!20}Laydel & L'Ombra, Il Mutaforma, Il Fuorilegge \\
Nethergal & L'Araldo, L'Esploratore, Il Mago \\
\rowcolor{gray!20}Nedraf & L'Eroe, L'Esploratore, L'Alleato \\
Nihar & Il Giullare, L'Esploratore, Il Giullare \\
\rowcolor{gray!20}Orudjs & Il Mutaforma, Il Giullare, L'Amante \\
Orlaith & Il Guardiano della Soglia, Il Sovrano, L'Eroe \\
\rowcolor{gray!20}Rezh & Il Tiranno, L'Ombra, Il Mutaforma \\
Shayalia & L'Ombra, L'Amante, Il Mutaforma \\
\rowcolor{gray!20}Sixiser & L'Eremita, Il Saggio, L'Ombra \\
Sumkjr & L'Eroe, L'Alleato, Il Mentore, Il Martire \\
\rowcolor{gray!20}Tàhil & L'Ombra, Il Tiranno, Il Fuorilegge \\
Tazher & L'Ombra, Il Fuorilegge, Il Mutaforma \\
\rowcolor{gray!20}Thaft & L'Uomo Comune, L'Alleato, Il Custode \\
Torbiorn & Il Tiranno, L'Ombra, Il Sovrano \\

\end{tabularx}

\subsection*{Utilizzare gli Archetipi nel Gioco}

\subsection*{Per i Giocatori}
\noindent\begin{itemize}[leftmargin=*] \setlength{\itemsep}{0pt}
\item Usa gli archetipi come punto di partenza per la creazione del personaggio, selezionando Tratti che si allineano con il tuo archetipo scelto
\item Considera il potenziale viaggio archetipico del tuo personaggio, come potrebbe evolversi nel tempo?
\item Guarda ai Patroni allineati del tuo archetipo quando consideri affiliazioni spirituali
\item Usa archetipi contrastanti all'interno del tuo gruppo per creare dinamiche inter-personaggio interessanti
\end{itemize}

\subsection*{Per il Narratore}
\noindent\begin{itemize}[leftmargin=*] \setlength{\itemsep}{0pt}
\item Popola il tuo mondo con PNG che incarnano pattern archetipici chiari
\item Progetta sfide che testano specificamente i personaggi contro le loro debolezze archetipiche
\item Crea antagonisti che servono come riflessi oscuri degli archetipi degli eroi
\item Usa interventi dei Patroni per evidenziare o sfidare la natura archetipica di un personaggio
\item Progetta archi di campagna che seguono cicli archetipici classici
\end{itemize}

\begin{narratore}[Gli Archetipi]\index{Archetipi Caratteriali}
Questo sistema archetipico fornisce un altro livello di profondità alla creazione e sviluppo del personaggio, collegando la teoria psicologica con i sistemi esistenti di Tratti e Patroni del gioco. Comprendendo questi pattern archetipici, giocatori e Narratore possono creare personaggi e storie più coerenti e psicologicamente consistenti.

\textbf{Piuttosto che limitare la creatività, questi archetipi servono come framework e linee guida utili che possono essere abbracciati, sovvertiti o trasformati durante il viaggio di un personaggio}. L'integrazione di Tratti e affinità con i Patroni con pattern archetipici crea un ricco arazzo di possibilità per lo sviluppo del personaggio e la narrazione.
\end{narratore}

\end{multicols}

\vfill

\begin{enfasi}
	La battaglia tra il bene e il male si svolge nel cuore umano di ciascuno e non necessariamente tra un esercito di persone vestite di bianco e un esercito di persone vestite di nero. (George R.R. Martin)
\end{enfasi}

\pagebreak

\section{Competenze}\index{Competenze}

\begin{enfasi}{
Chi dice che una cosa è impossibile, non dovrebbe disturbare chi la sta facendo. (Albert Einstein)

\medskip
Non hai veramente capito qualcosa fino a quando non sei in grado di spiegarlo a tua nonna. (Albert Einstein)}\end{enfasi}

\begin{multicols}{2}

Le Competenze rappresentano il cosa si conosce ed il cosa si sa fare. I punteggi delle stesse rappresentano quanto bene è conosciuta la competenza e quindi più è alto il valore più si è esperti.

\subsection{Competenze di Base}\index{Competenze di Base}\label{competenzebase}

\begin{enfasi}{
Lo studio è per i perdenti! (Lobo) }\end{enfasi}


Ogni personaggio ha una Professione iniziale, un percorso di vita e lavoro che l'ha portato ad apprendere determinate competenze.

Sono elencate alcune Professioni e le loro competenze relative, il personaggio acquisisce queste competenze con il punteggio indicato nella tabella.

Nella scheda va segnata la Professione iniziale e le competenze acquisite, in accordo con il Narratore è possibile selezionare delle competenze diverse e anche creare professioni diverse!

\end{multicols}

\textbf{Tabella: Elenco delle Professioni e relative Competenze}\index[Tabelle]{Tabella Elenco delle Professioni e relative Competenze}\index{Professioni}

\medskip

\noindent\begin{tabularx}{\textwidth}{>{\raggedright\arraybackslash}l|c|c|c|c}
	\toprule
\rowcolor{gray!20}\textbf{Professione}& \textbf{1 punto} & \textbf{2 punti} & \textbf{2 punti} & \textbf{3 punti}\\
\toprule
\textbf{Accolito}& Occulto& Storia o Geografia& Arcana& Religione\\
\rowcolor{gray!20}\textbf{Alchimista}& Valutare&Natura& Erboristeria& Arcana\\
\textbf{Allevatore}& Sopravvivenza&Seguire tracce& Gestire animali&Natura \\
\rowcolor{gray!20}\textbf{Allievo mago}& Storia e Geografia&Occulto&Miti e Leggende&Arcana\\
\textbf{Araldo} & Tradizioni Locali & Araldica & Lingue & Diplomazia \\
\rowcolor{gray!20}\textbf{Bibliotecario}& Natura e Geografia&Tradizioni locali&Religione e Arcana&Storia\\
\textbf{Boscaiolo}& Usare corda&Seguire tracce & Natura& Sopravvivenza\\
\rowcolor{gray!20}\textbf{Cacciatore}& Furtività&Seguire tracce&Sopravvivenza& Natura\\
\textbf{Carovaniere}&Tradizioni locali &Gestire animali&Sopravvivenza&Cavalcare\\
\rowcolor{gray!20}\textbf{Contadino} & Usare Corda & Erboristeria & Gestire Animali& Natura\\
\textbf{Delinquente}& Sopravvivenza&Cavalcare&Valutare&Furtività\\
\rowcolor{gray!20}\textbf{Erborista}& Miti e Leggende&Geografia&Natura&Erboristeria\\
\textbf{Fabbro} & Pronto Soccorso & Valutare&Atletica & Artigianato \\
\rowcolor{gray!20}\textbf{Forestale}& Miti e Leggende&Erboristeria&Cavalcare & Natura\\
\textbf{Guardia}& Percepire Emozioni&Conoscenza Legge&Cavalcare&Intimidire\\
\rowcolor{gray!20}\textbf{Guida}& Miti e Leggende&Dungeon&Natura&Geografia\\
\textbf{Leguleio}& Valutare&Ingannare&Percepire Emozioni&Legge\\
\rowcolor{gray!20}\textbf{Locandiere}& Pronto soccorso&Valutare&Diplomazia&Percepire Emozioni\\
\textbf{Mazziere}& Percepire Emozioni&Valutare&Intrattenere&Ingannare\\
\rowcolor{gray!20}\textbf{Medico}& Miti e Leggende&Natura&Erboristeria&Pronto soccorso\\
\textbf{Mercante}& Lingue&Tradizioni Locali&Ingannare&Valutare\\
\rowcolor{gray!20}\textbf{Mercenario} & Percepire Emozioni & Acrobatica & Intimidire & Atletica\\
\textbf{Minatore}& Usare corde&Pronto soccorso&Valutare&Dungeon\\
\rowcolor{gray!20}\textbf{Monaco} & Intrattenere &Pronto Soccorso & Natura & Religione\\
\textbf{Nomade}&  Erboristeria& Gestire Animali & Natura &  Sopravvivenza\\
\rowcolor{gray!20}\textbf{Orafo} & Tradizioni locali & Mani di fata & Valutare & Falsificare\\
\textbf{Pescatore}& Usare corde&Nuotare&Sopravvivenza&Natura\\
\rowcolor{gray!20}\textbf{Scriba} &Tradizioni Locali& Falsificare & Miti e Leggende & Lingue\\
\textbf{Sensale} & Intrattenere & Tradizioni locali & Percepire Emozioni & Diplomazia\\
\rowcolor{gray!20}\textbf{Soldato}& Nuotare&Gestire animali&Atletica&Cavalcare\\
\textbf{Tagliaborse} & Disattivare congegni&Artista della fuga&Furtività&Mani di fata\\
\rowcolor{gray!20}\textbf{Teatrante}& Percepire emozioni& Lingue&Acrobatica&Intrattenere\\
\end{tabularx}

\vfill

\begin{enfasi}{
Anche se indubbiamente il desiderio di conoscere è naturale per tutti gli uomini, la voglia di imparare non è cosa da tutti...(Richard de Bury)
}\end{enfasi}

\begin{multicols}{2}

Una professione non si esplicita in sole 4 competenze ma queste sono quelle che più entreranno in uso durante le avventure, il Narratore sarà aiutato dalla vostra professione a capire come il vostro personaggio potrà risolvere le situazioni e come interagirà con gli altri personaggi.

Qui sotto c'è la \textbf{Tabella elenco competenze} da cui scegliere per eventuali nuove professioni o personalizzazioni delle stesse.

\subsubsection{Personalizzare Competenza e Professione}\index{Personalizzare Competenza e Professione}

Ad ogni nuova professione che andrete a creare associate 4 competenze prese dalla \textbf{Tabella: Elenco Competenze e Relativa Caratteristica d'uso}. Una competenza partirà con un punteggio di 1, due competenze partiranno con il punteggio di 2 e quella più specifica e professionale partirà con il punteggio di 3.

In accordo con il Narratore è anche possibile cambiare l'ordine delle Competenze per le Professioni già elencate rendendo più capace il personaggio in alcune competenze piuttosto che altre.

\subsubsection{Competenze, Intelligenza e Background del personaggio}\label{quintacompetenza}\index{Il Competenze e Background del personaggio}

Il giocatore alla creazione del personaggio può decidere di prendere un +1 ad una Competenza già conosciuta oppure prendere una nuova Competenza, legata alla storia del personaggio, a punteggio 1.

Il personaggio acquisisce una Competenza a punteggio 1 per ogni punto di Intelligenza superiore a 2, e perde 1 punto in una Competenza per ogni punto di Intelligenza inferiore a 0.

\medskip

Il giocatore \textbf{aumenta di 1 il punteggio di una Caratteristica che si colleghi alla Professione od al background} fino al valore massimo di 4. Potrebbe essere Intelligenza per un Apprendista mago, ma se questo fa il culturista per hobby potrebbe essere anche Forza.

\begin{giocatore}[Professione ???]
Non sottovalutate la scelta della Professione! Non tutto può risolversi con asciate o magia. Sapere districare nodi, seguire tracce, riconoscere erbe o malattie fanno del personaggio un esperto, creano una professione. Non dovete definire il personaggio solo in base alle Abilità che ha ma in base a cosa e quanto bene sa farlo. Un personaggio di basso livello ma esperto di sopravvivenza sarà sempre più utile di un combattente esperto se si tratta di attraversare un deserto.\end{giocatore}

\end{multicols}

\medskip

\textbf{Tabella: Elenco Competenze e relativa Caratteristica d'uso}\index[Tabelle]{Tabella Elenco Competenze e relativa Caratteristica d'uso}

\medskip\label{competenzeelenco}\hypertarget{competenzeelenco}{}

\noindent\begin{tabularx}{\linewidth}{lllll}
\toprule
\rowcolor{gray!20}\textbf{Forza} & \textbf{Destrezza} & \textbf{Intelligenza} & \textbf{Saggezza} & \textbf{Carisma}\\
\toprule
Arrampicarsi	& Acrobatica		& Arcana		&Cavalcare				& Diplomazia \\
\rowcolor{gray!20}Atletica		& Artista della fuga& Artigianato 	&\emph{Consapevolezza} 	& Intrattenere \\
Intimidire		& Furtività 		& Conoscenza	&Gestire animali		& Ingannare \\
\rowcolor{gray!20}Nuotare 		& Mani di fata		& Disattivare congegni &Natura			& Tradizioni locali \\
- & Usare corda	& Erboristeria		& Percepire Emozioni & - \\
\rowcolor{gray!20}- & -				& Falsificare		& Pronto soccorso	 & - \\
- & -				& Valutare			& Seguire tracce	 & - \\
\rowcolor{gray!20}- & -		        & -        			& Sopravvivenza      & - \\

\end{tabularx}

\medskip

La \textbf{Conoscenza} va esplicitata su quale argomento verte: Architettura ed Ingegneria, Dungeon, Geografia, Legge, Lingue (terrestri o meno), Miti e Leggende, Nobiltà ed Araldica, Occulto, Piani, Religione, Storia, Tecnologia Antica ...

\begin{multicols}{2}

Ad ogni \textbf{livello successivo al primo} distribuisci un numero di punti pari la metà del punteggio di Intelligenza +1, $[(Int/2)+1]$, con un minimo di 1 punto, tra le competenze già conosciute o perfezionate nell'avventura od apprese ex novo.\index{Competenze per livello}

\textbf{Nessuna competenza di Base o Attiva può avere più del livello +3 punti assegnati.}\index{Massimo punteggio Competenza}

\subsubsection{Consapevolezza}\label{consapevolezza}\index{Consapevolezza}

Una competenza che hanno tutti i personaggi è \textbf{Consapevolezza}, ovvero la capacità di percepire l'ambiente intorno a loro. Questa competenza ha un punteggio fisso pari a 1/3 del livello del personaggio (arrotondato per eccesso) più Saggezza. Su questa competenza il personaggio non può assegnare punti, ma può scegliere certe Abilità per alzarne il punteggio.

I giocatori più che usare Consapevolezza per ricercare informazioni dovrebbero fare domande, indagare, curiosare, arguire ipotesi e confrontarsi e non limitarsi a chiedere un tiro di Consapevolezza per trovare qualcosa.

\subsubsection{Apprendere nuove competenze, professioni}\label{apprenderenuovecompetenze}

Un personaggio può apprendere una nuova competenza o migliorarla con un studio/pratica di almeno 4 ore al giorno per almeno 4 mesi con un insegnante che abbia un punteggio di competenza superiore a quello che mira il personaggio. Dopo questo lasso di tempo il giocatore può assegnare un punto alla competenza di base per cui si è applicato.

Per apprendere una nuova professione deve passare almeno 6 mesi per 6 ore al giorno con chi pratica quella professione. Passati i 6 mesi il personaggio acquisisce le 4 competenze della professione. Eventuali Competenze già conosciute aumenteranno di 1 punto.

\subsubsection{Competenze ed i loro ambiti di utilizzo}\label{competenzeambitidiutilizzo}

Sono descritte sommariamente le Competenze ed i loro utilizzi soliti. Viene anche indicato il numero di Azioni necessarie per svolgere la prova tipica, usi più complessi richiedono più tempo ed Azioni.

Le Azioni necessarie alla prova possono variare a seconda della capacità del personaggio e della complessità della prova.

In ogni caso ricordate sempre di valutare con attenzione come il giocatore dichiara di svolgere le azioni per capirne la durata ed effetti.

La Competenze con un \textbf{*} subiscono le penalità dovute all'\hyperlink{equipaggiamento.armature.scudi}{armatura} indossata (pag. \pageref{equipaggiamentoarmature}).

\medskip

\textbf{Acrobatica* (DES)}: Questa competenza serve per mantenere l'equilibrio su superfici strette o precarie, per tuffarsi, rotolare, fare capriole, salti mortali, superare degli ostacoli nonché cadere e non farsi male. 1 Azione.

\textbf{Arcana (INT)}: Con questa competenza si è esperti di magia e di incantesimi, di oggetti magici è si è grado di identificare gli incantesimi che vengono lanciati. 1 Azione.

\textbf{Arrampicarsi* (FOR)}: Con questa competenza si possono scalare superfici verticali, dalle mura cittadine alle pareti rocciose. E' legata all'Azione di Movimento. Con 8 punti il movimento di Scalare è solo dimezzato.

\textbf{Artigianato (INT)}: si esplicita su una capacità costruttiva, permette costruire l'oggetto dell'artigianato e di giudicare e valutare un lavoro nell'ambito della competenza.

\textbf{Artista della fuga (DES)}: Con questa competenza ci si può liberare da legacci (contrapposta all'Usare Corde) e manette. 1 Azione ogni 10 di DC. Con 6 punti il tempo è 1 Azione ogni 15 di DC, con 12 è 1 Azione ogni 20 DC.\label{artistadellafuga}

\textbf{Atletica* (FOR)}: Con questa competenza si è esperti atleti, capaci di prodigiosi salti ed eccezionali prove di Forza. 1 Azione.

\textbf{Cavalcare (SAG)}: Con questa competenza è possibile cavalcare in maniera professionale e dare comandi alla propria cavalcatura. Vedi Capitolo \hyperlink{cavalcare}{Cavalcare} (pag. \pageref{cavalcare}) 1 Azione.

\textbf{Consapevolezza (SAG)}: per cercare, accorgersi, notare. E' un qualcosa di attivo. 2 Azioni. \textbf{Usare 1 Azione impone una penalità -1d6 alla prova}.

\textbf{Conoscenza di Architettura ed Ingegneria (INT)}: Sei un esperto costruttore e sai valutare la struttura delle costruzioni. Sai anche riconoscere stili architettonici e creare progetti d'interno e d'arredo. 1 Azione.

\textbf{Conoscenza dei Dungeon (INT)}: Con questa competenza si hanno conoscenze di Aberrazioni, caverne, esplorazioni sotterranee, Melme. 1 Azione.

\textbf{Conoscenze di Geografia (INT)}: Con questa competenza si hanno conoscenze sul clima, popolazione, terreni, territori, nazioni e confini. 1 Azione.

\textbf{Conoscenza delle Legge (INT)}: Con questa competenza si conosce la Legge di una regione. Si è esperti nel conoscere le norme ed i cavilli. Si sanno citare casi e si conoscono altri azzeccagarbugli e giudici. 2 Azioni.

\textbf{Conoscenza Lingue (INT)}: Ogni punto in questa competenza permette di apprendere un nuovo linguaggio scritto e parlato. Un buon punteggio di Lingue aiuta a comprendere lingue non note ed a farsi comprendere. Viene usata anche per comprendere testi complessi. Costo variabile.

\textbf{Conoscenza dei Miti e Leggende (INT)}: Si ha una e vera propria passione per i miti e leggende tradizionali e più remoti. Conosci località, storia e creature leggendarie. 1 Azione.

\textbf{Conoscenza di Nobiltà e Araldica (INT)}: Conosci linee nobiliari, casate, dicerie, stemmi araldici, personalità ed i maggiori possedimenti e tesori. Si applica anche a personaggi famosi ed importanti. 1 Azione.

\textbf{Conoscenze dei Piani (INT)}: Con questa competenza si è esperti di Piani e relativi abitanti. 1 Azione.

\textbf{Conoscenze Occulte (INT)}: Con questa competenza si è esperti di occulto, creature immondi. 1 Azione.

\textbf{Conoscenze Religione (INT)}: Con questa competenza si hanno conoscenze su Patroni, mitologia, Celestiali, Non Morti, simboli sacri, tradizione ecclesiastica, feste e ricorrenze liturgiche. 1 Azione.

\textbf{Conoscenze di Storia (INT)}: Con questa competenza si hanno conoscenze di Storia quali guerre, migrazioni, colonie, fondazioni di città, accadimenti importanti.. 1 Azione.

\textbf{Diplomazia (CAR)}: Con questa competenza si possono risolvere diverbi e raccogliere preziose informazioni e dicerie dalle persone. La competenza è anche usata per negoziare in modo efficace con la giusta etichetta e condotta adatta alla situazione controversa. Costo variabile.

\textbf{Disattivare congegni (INT)}: Con questa competenza si possono disarmare Trappole e aprire serrature, sabotare congegni meccanici semplici, come le catapulte, le ruote di un carro o le porte. 1 Azione ogni 10 di DC. Con 6 punti il tempo è 1 Azione ogni 15 di DC, con 12 punti è 1 Azione ogni 20 DC.

\textbf{Erboristeria (INT)}: Con questa competenza si hanno conoscenze di come riconoscere e preparare pozioni e veleni naturali. Il punteggio si applica alle prove per distillare pozioni. Riconoscere Pozioni naturali 1 Azione ogni 10 di DC. Con 6 punti il tempo è 1 Azione ogni 15 di DC, con 12 punti è 1 Azione ogni 20 DC.

\textbf{Falsificare (INT)}: Con questa competenza si sa falsificare e riconoscere come falsi oggetti d'arte, mappe, firme... Costo variabile.

\textbf{Gestire animali (SAG)}: Con questa competenza è possibile addestrare e ammansire animali. 1 minuto ogni 5 di DC. Con 6 punti il tempo è 1 minuto ogni 10 di DC, con 12 è 1 minuto ogni 15 DC.

\textbf{Intimidire (FOR)}: Intimidire si basa sull'approccio fisico per convincere l'interessato. 1 Azione.

\textbf{Ingannare (CAR)}: La competenza Ingannare può essere usata per Raggirare (dicendo quindi fandonie) o Persuadere (adattando la verità) al fine di convincere delle proprie parole l'interessato. Costo variabile.

\textbf{Intrattenere (CAR)}: Con questa competenza si è esperti in una espressione artistica, dal canto alla recitazione, dal ballo a suonare strumenti musicali. E' necessario specificare la forma di intrattenimento. Costo variabile.

\textbf{Mani di fata* (DES)}: Con questa competenza si può borseggiare, estrarre un'arma nascosta, oppure compiere altre azioni senza essere notati(ad esempio barare a carte). 1 Azione.

\textbf{Furtività* (DES)}: Con questa competenza si è in grado di muoversi senza causare rumore o nascondersi nelle ombre. 1 Azione.

\textbf{Natura (SAG)}: Con questa competenza si hanno conoscenze di Animali, Fatati, stagioni e cicli, tempo atmosferico, vegetali. 1 Azione.

\textbf{Nuotare* (FOR)}: Con questa competenza si è in grado di nuotare, anche in acque tempestose. Senza competenza si sa stare a galla in acqua placide. Legata all'Azione di Movimento.

\textbf{Percepire Emozioni (SAG)}: Con questa competenza si può capire se qualcuno sta mentendo o si possono intuire le sue vere intenzioni. 1 Azione.

\textbf{Pronto soccorso (SAG)}: Con questa competenza si possono curare le ferite e le malattie. Costo variabile.

\textbf{Seguire tracce (SAG)}: Con questa competenza si sa seguire le tracce lasciate nell'ambiente. 1 Azione ogni 10 di DC. Con 6 punti il tempo è 1 Azione ogni 15 di DC, con 12 punti è 1 Azione ogni 20 DC.

\textbf{Sopravvivenza (SAG)}: Con questa competenza si può sopravvivere e orientarsi nelle terre selvagge. La competenza è usata anche per cercare attivamente trappole e fosse. 1 minuto per cercare trappole in 3x3 metri, con punteggio 6 costa 3 round, con punteggio di 12 costa 1 round, con punteggio 18 costa 1 Azione.

\textbf{Tradizioni locali (CAR)}: Con questa competenza si hanno conoscenze degli abitanti (più noti), costumi, leggende, leggi, personalità, tradizioni. E' necessario specificare una regione geografica dove è applicabile la conoscenza. 1 Azione.

\textbf{Usare corda (DES)}: Con questa competenza si è esperti in legacci e nodi per fissare e bloccare oggetti o persone. 2 Azioni.

\textbf{Valutare (INT)}: Con questa competenza si sa stimare il valore monetario di un oggetto. La difficoltà è in base alla rarità dell'oggetto, DC 12 + 2 comune, 4 non comune, 8 raro, 12 molto raro, 16 leggendario. 1 Azione ogni 5 di DC. Con 6 punti il tempo è 1 Azione ogni 10 di DC, con punteggio 12 è 1 Azione ogni 20 DC. \label{valutare}

\medskip

\subsection{Competenze Attive}\index{Competenze Attive}\label{competenzeattive}

\textbf{Il personaggio prende 1 punto, ad ogni livello, da distribuire tra le Competenze Attive od attribuirlo alle Competenze di Base}.

\medskip

Le \textbf{Competenze Attive} sono: Competenza Magica, Competenza Armi, Tiri Salvezza (Riflessi, Tempra, Volontà).

\noindent\begin{itemize}[leftmargin=*] \setlength{\itemsep}{0pt}

\item \textbf{Competenza Magica (CM)}: \index{CM}\index{Competenza Magica} indica la capacità e competenza nel lanciare un incantesimo.

\item \textbf{Competenza Armi (CA)}: \index{CA}\index{Competenza Armi} è la capacità e bravura di combattere con un'arma da mischia o da tiro/distanza.

\item I \textbf{Tiri Salvezza} rappresentano la resistenza, la capacità fisica e psichica del personaggio.
\end{itemize}

Attribuire il punto di Competenze Attive alle \textbf{Competenza di Base} significa distribuire 4 punti aggiuntivi su almeno 3 Competenze di Base a piacere. Attribuire il punto ai Tiri Salvezza significa aumentare di 1 punto un Tiro Salvezza.\index{Aumentare Competenze di Base}

\medskip

\begin{enfasi}{C'è solo un modo per allenarsi: quello giusto. (Carl Lewis)
\medskip

Wang Chi: Sei pronto?

Jack Burton: Io sono nato pronto! (Grosso guaio a Chinatown, Film 1986)
}\end{enfasi}

\subsubsection{Tiri Salvezza}\index{Tiri Salvezza}\label{tirisavellza}

I \textbf{Tiri Salvezza} (abbreviati in TS) sono usati quando il personaggio é sottoposto ad uno sforzo, vuoi di resistenza fisica, mentale o agilità eccezionale. Il punteggio dei Tiri Salvezza si base sulle Abilità scelte. Abilità più fisiche tenderanno a migliorare l'aspetto di resistenza del personaggio, Abilità più atletiche o di attenzione aumenteranno i riflessi, le Abilità prettamente mentali rafforzeranno la volontà del personaggio.

Il \textbf{Tiro Salvezza su Tempra} indica quanto si è in grado di sopportare le sofferenze fisiche o attacchi contro la propria vitalità e salute. Al valore dei Tiri Salvezza su Tempra si aggiunge il punteggio della \textbf{Costituzione}.

Il \textbf{Tiro Salvezza su Volontà} indica la resistenza contro l'influenza mentale ed altri effetti magici, ciò che vuole modificare il tuo libero arbitrio nelle scelte e nell'agire. Al valore dei Tiri Salvezza su Volontà si aggiunge il punteggio di \textbf{Saggezza}.

Il \textbf{Tiro Salvezza su Riflessi} indica quanto si è agili e pronti per evitare ostacoli o magie. Al valore dei Tiri Salvezza su Riflessi si aggiunge il punteggio di \textbf{Destrezza}.

Quando viene chiesto un Tiro Salvezza significa fare una prova sulla Competenza Attiva richiesta, possa essere Volontà, Tempra o Riflessi.
La prova si andrà ad eseguire tirando 3d6 + valore della Competenza Attiva richiesta ovvero il punteggio nel Tiro Salvezza su Volontà, Riflessi o Tempra + il valore della Caratteristica collegata alla tipologia Competenza Attiva (Saggezza, Destrezza o Costituzione) + Abilità + bonus magici (oggetti che influenzano il Tiro Salvezza) e modificatori vari presenti.

\medskip

\begin{giocatore}[Tiri Salvezza non standard]
E' possibile che vengano richiesti dei Tiri Salvezza con modificatori diversi, ovvero un Tiro Salvezza su Tempra con modificatore Forza oppure un Tiro Salvezza su Volontà con modificatore Carisma. Sarà il Narratore a dirvi quando si applica un modificatore diverso.
\end{giocatore}

\subsubsection{Competenza Armi}\label{competenzaarmi}

La \textbf{Competenza Armi} (abbreviata in \textbf{CA}) indica la capacità e bravura nell'usare un arma. La competenza si riflette direttamente nelle prove per colpire l'avversario con armi.

Il \textbf{Tiro per Colpire per le armi da mischia}\index{Armi da mischia} si risolve con una prova di Competenza Armi (\textbf{CA}) + \textbf{Forza} + eventuali Abilità + bonus da Lista d'Armi + bonus magici e modificatori contro la Difesa dell'avversario (Destrezza + armatura + scudo + modificatori).

Il \textbf{Tiro per Colpire con armi da distanza} \index{Armi da distanza}(archi, balestre, pugnali da lancio, giavellotti, sassi..) si risolve con una prova di Competenza Armi (\textbf{CA}) + \textbf{Destrezza} + bonus da Lista d'Armi + eventuali capacità, bonus magici e modificatori contro la Difesa dell'avversario (Destrezza + armatura + scudo + modificatori).

Quando si assegna un punto ad \textbf{CA} è necessario precisare su quale gruppo di arma si prende, se non si dichiara allora è come averlo preso nel gruppo Armi Semplici.
Controllare l'elenco \hyperlink{lista.armi}{Armi per Tipologia Omogenea} (pag. \pageref{lista.armi}).\index{Tipologia Omogenea}

Il personaggio può decidere di assegnare il suo punto ad una tipologia di armi che già conosce, migliorando così la sua capacità ed talento nell'uso od apprendere un altra tipologia di armi.

più è alto il punteggio in una tipologia d'armi più facilmente può usufruire di vantaggi con le armi della stessa, ma conoscerà meno armi.

Se il giocatore non ha assegnato alcun punto nella \textbf{CA} può utilizzare senza penalità al colpire solo le armi raggruppate come Armi Semplici.

Le \textbf{Armi Semplici} sono: Pugnale, Mazza Leggera, Mazza chiodata, Bastone, Balestra (Leggera), Giavellotto\index{Armi Semplici}

Usare un'\textbf{Arma senza conoscere la Lista d'Armi di appartenenza}, o che non sia un \textbf{Arma Semplice}, impone un -1d6 al Tiro per Colpire.\index{Arma senza competenza}

Per poter utilizzare \textbf{Armature Leggere} e \textbf{Scudi Leggeri} è necessario avere Forza almeno -1.\index{Armature Leggere}\index{Scudi Leggeri}\label{indossarearmature}\hypertarget{indossarearmature}{}

Per poter utilizzare \textbf{Armature Medie} e \textbf{Scudi Medi} è necessario avere almeno 2 punti in Competenza Armi.\index{Armature Medie}

Con almeno 3 punti in Competenza Armi ed 1 in Forza si possono usare senza penalità \textbf{Armature Pesanti} e \textbf{Scudi Pesanti}.\index{Armature Pesanti}\index{Scudi Pesanti}

Usare un'\textbf{Armatura senza l'adeguata competenza} impedisce di usare il valore di Destrezza in Difesa ed il bonus conferito dall'armatura alla Difesa si riduce di 1.\index{Armatura senza competenza}

Usare uno \textbf{Scudo senza l'adeguata competenza} peggiora il Tiro per Colpire di 1 e lo scudo conferisce un bonus massimo a Difesa di 1.\index{Scudo senza competenza}


\begin{giocatore}[Specializzarsi o meno in un arma]
I vantaggi di specializzarsi in una Lista d'Armi sono concreti e tangibili ma si portano dietro il limite di non sapere usare bene le altre armi.

Valutate il tipo di avventura, se avrete sempre a disposizione o potrete facilmente ottenere le vostre armi \emph{preferite}.

Valutate anche se la tipologia di nemici che affronterete possa avere delle resistenze o meno alla tipologia di danno che causate. Creare un ottimo arciere quando si affrontano quasi sempre scheletri non è una buona idea.
\end{giocatore}

\subsubsection{Competenza Magica}\label{competenzamagica}

La \textbf{Competenza Magica} (abbreviata in \textbf{CM}) permette al personaggio di poter conoscere più incantesimi, più potenti, più efficaci e più facilmente lanciabili.

Un personaggio con alta \textbf{Competenza Magica} sa manipolare più incantesimi e con risultati migliori.

Il valore di Competenza Magica stabilisce insieme all'Abilità Adepto della Magia ed al modificatore di caratteristica per incantesimi il livello massimo di incantesimi lanciabile.

Il punto di Competenza Magica, a differenza di quello di Competenza Armi, non è necessario dichiararlo su una Lista di Magia, Patrono o altro. Il punto si \emph{trasforma} in grezzo potere magico.

E' necessario avere almeno 1 punto in Competenza Magica se si vuole essere degli usufruitori di magia, se si vuole conoscere il come funziona la magia e la connessione di questa con i Patroni.

Non è strettamente necessario mettere sempre il punto di Competenza Attiva in Competenza Magia se si vuole fare un incantatore, anzi, qualche punto in Competenza Armi serve anche al mago più bravo se vuole sapere \emph{mirare e colpire} un avversario con qualcosa di diverso da un incantesimo.

Un punteggio di Competenza Magica 14 e aver preso 4 volte Adepto della Magia, avendo 4 nel modificatore di caratteristica per incantesimi, è sufficiente per poter lanciare incantesimi di massimo livello. CM 9 e Adepto della Magia preso tre volte vi possono garantire il sesto livello di incantesimi.

\subsubsection{Opzionale - Abilità come Competenze Attive}\index{Opzionale - Abilità come Competenze Attive}

Il Narratore può concedere su richiesta del giocatore di poter usare il punto di Competenza Attiva non per aumentare la Competenza Magica o delle Armi, bensì per selezionare una nuova Abilità, rispettando i requisiti.

\subsubsection{I punteggi delle Competenze Base e Attive}\label{punteggicompetenzebaseattive}\index{I punteggi delle Competenze Base e Attive}

Ogni punto attribuito nella Competenze di Base o Competenza Armi o Magica permette di usufruire di +1 nella prova relativa (Prove, Tiro per Colpire, Competenza Magica)

\end{multicols}

\vspace{2cm}

\begin{giocatore}[Tups arriva al 4' livello!]
Tups è arrivato al 4' livello! Ecco come ha distribuito i punti delle Competenze Attive.

\textbf{1 livello}: +1 Competenza Armi, Abilità: Armatura del Devoto (+2 Volontà, +1 Riflessi, \textbf{Costituzione}), Il Patrono è la mia Arma (+1 Riflessi, +1 Volontà, \textbf{Costituzione}), Conoscenza istintiva (+2 Volontà, +1 Tempra, \textbf{Saggezza})

\textbf{2 livello}: +1 Competenza Magica, Abilità: Colpi Poderosi (+2 Tempra, \textbf{Costituzione})

\textbf{3 livello}: +1 Competenza Magica, Abilità: Fedele (+2 Volontà, +1 Tempra, \textbf{Costituzione}). Il punteggio di Costituzione aumenta di 1.

\textbf{4 livello}: +1 Competenza Armi, Abilità: Incantatore Prudente (+2 Riflessi, +1 Tempra, \textbf{Intelligenza})

\textbf{\emph{Totale}}: +2 CA, +2 CM, +4 Tiro Salvezza Riflessi, +6 Tiro Salvezza Tempra, +6 Tiro Salvezza Volontà

\end{giocatore}

\vfill

\begin{center}
\includegraphics[width=0.45\linewidth]{immagini/attaccoallespalle.png}

\emph{Prova fallita di furtività...}
\end{center}


\bigskip

\begin{enfasi}
	Hai idea di quale grande potere hai nelle tue mani? (Morla, La Storia Infinita)
\end{enfasi}

\pagebreak

\section{Costruiamo il Personaggio}\index{Personaggio}

\begin{enfasi}{
Mai dimenticare chi sei, perché di certo il mondo non lo dimenticherà. Trasforma chi sei nella tua forza, così non potrà mai essere la tua debolezza. Fanne un'armatura, e non potrà mai essere usata contro di te. (Tyrion Lannister)
}\end{enfasi}

\begin{multicols}{2}

OBSS è un sistema duro, pericoloso, mortale ma anche ricco di soddisfazioni. I tuoi personaggi non sono eroi, non sono prescelti. Sono sfortunati che si trovano in imprese dove forse sopravviveranno e sarà a discapito di qualche compagno. Non sei tu a scegliere l'avventura ma è lei a trascinarti impetuosamente dentro. Sii forte, coraggioso, arguto ma non avventato.

Sopravvivi e reclama la Legge del Premio e vedrai che con il passare dei livelli acquisirai competenze ed abilità fuori dal comune!. \emph{Spes ultima dea}!

Come prima cosa prepara davanti a te la scheda ed un foglio dove prendere note ed appunti.

Per creare un personaggio prova rispondere a queste domande, potranno aiutarti ad immaginarlo e plasmarlo:

- Immagina che aspetto abbia

- Quale è il Tratto principale del carattere

- Quali sono i suoi tic, modi di fare, abitudini

- Quali sono i suoi obiettivi primari

- Una cosa curiosa, una buffa, una imbarazzante ed una espressione tipica del personaggio

- In cosa è bravo, in cosa si impegna, in cosa è negato

- I tre difetti ed i tre pregi principali del personaggio

\begin{center}

 \includegraphics[width=0.7\linewidth]{immagini/Leonidas_I_of_Sparta.png}

\emph{Leonida di Sparta}
\end{center}

E' cresciuto in famiglia, in un clan, vagabondo, per strada.. cosa l'ha portato e che scelte ha fatto per arrivare fino ad adesso ?

Quale è il suo stile di combattimento e strategia tipica ? Magia, Spada, dalle retrovie.. incitare i compagni.. scappare...

E non meno importante: quale è il suo scopo ? cosa lo ha fatto uscire di casa, dalle sue sicurezze.. da una vita normale e intrapreso quella di avventuriero ?

Ricorda sempre che questo è un mondo crudele, pieno di rischi, trappole e mostri, ma anche occasioni che possono renderti potente e ricchissimo.

Per incominciare leggi il capitolo sulle Razze ed individua quella del tuo personaggio.

Recupera un pò di d6 e tira!

Consulta il capitolo delle \hyperlink{assegnazione.punteggi.caratteristica}{Caratteristiche} per capire quanto sei stato fortunato (pag. \pageref{assegnazionepunteggicaratteristica}).

E se i valori delle Caratteristiche non sono venuti come che ti aspettavi allora lasciati guidare dal caos e crea qualcosa di diverso ma ugualmente divertente e magnifico.

Se hai Intelligenza pari o superiore a 2 scegli un altra \hyperlink{linguaggi}{lingua} (pag. \pageref{linguaggi}) parlata/scritta oltre al Comune, se hai 3 puoi sceglierne 2 di lingue in più.

Scegli al Professione del personaggio, le Competenze Base vengono assegnata in base a questa. Sceglila con attenzione e cura, oltre alle competenze previste dalla Professione scelta ne puoi prendere una quinta data dal tuo background oppure aumentare di uno il punteggio in una già presa.
In base al background e professione scelta aumenti una caratteristica di 1, fino ad un massimo di 4 + modificatore razziale.

Passa alle Competenze Attive: qui hai 1 punto  da distribuire tra Competenza Armi e Competenza Magica.

La Competenza Armi ti aiuta nel colpire meglio. La Competenza Magica è l'unica cosa che ti permette di usare la magia. Ricorda anche che i punti in Competenze Armi vanno dichiarati a quale \hyperlink{lista.armi}{Lista Armi} (pag. \pageref{lista.armi}) sono stati applicati.

Se non hai punti in Competenza Armi puoi usare solo le \hyperlink{armi.semplici}{armi semplici} (pag. \pageref{listaarmisemplice}) senza incorrere in penalità al Tiro per Colpire e non potrai usare armature medie o pesanti.

I Punti Ferita sono pari a 8 + Costituzione, aggiungi 3 se hai messo 1 punto in Competenza Armi (CA).

A questo punto scegli i \hyperlink{tratti}{Tratti} (pag. \pageref{tratti}). Fallo con attenzione, stai costruendo il tuo personaggio ed i Tratti delineano a forti pennellate il carattere. Ricordati che saranno fondamentali per la scelta del \hyperlink{patroni}{Patrono} (pag. \pageref{patroni}).

Nella scheda, nello specchietto dei Tratti, dove c'è la colonna Patrono scrivi il Patrono che ti collega a quel Tratto, indipendentemente che tu lo abbia scelto o meno.

Ricorda infine che un personaggio \emph{Dissoluto} e \emph{Leale} suona bene un in racconto dove è il solo protagonista ma qui si gioca in \textbf{gruppo}. Non prendere Tratti in ovvia opposizione agli altri o comunque non giocare da \emph{stronzo}, altrimenti il personaggio verrà naturalmente allontanato dagli altri personaggi e dal Narratore.

Se hai messo dei punti in Competenza Magica valuta anche di prendere l'Abilità Adepto della Magia per potenziare la tua magia (e \emph{apprendere} un incantesimo in più!).

\begin{center}
\includegraphics[width=0.85\linewidth]{immagini/Alexander_and_Bucephalus_-_Battle_of_Issus_mosaic.png}

\emph{Alessandro Magno}
\end{center}

Consulta il \hyperlink{tomocm1}{Tomo della Magia}, pag. \pageref{tomocm1}, per capire quanti incantesimi devi scrivere nel tuo Tomo.

Scelti gli incantesimi del Tomo devi decidere quali hai appreso e quindi puoi lanciare, vedi \hyperlink{incantesimicm1}{Regole della Magia} a pag. \pageref{incantesimicm1}.

Passa alle \hyperlink{abilita}{Abilità} (pag. \pageref{abilita}), al primo livello ne scegli due, stai attento ai prerequisiti ed anche ad eventuali Abilità che ti concede la tua razza.

Sono le Abilità che scegli ad aumentare il punteggio dei Tiri Salvezza. Ricorda che i Tiri Salvezza determinano la tua capacità di resistere a traumi e magie. Nella scheda indica la singola Caratteristica che vuoi che quella Abilità migliori (quando ne avrai quattro uguali).

Scegli l'\hyperlink{equipaggiamento}{equipaggiamento} (pag. \pageref{equipaggiamento}), \hyperlink{equipaggiamento.armature.scudi}{armatura} (pag. \pageref{equipaggiamentoarmature}), \hyperlink{equipaggiamento.armi}{armi} (pag. \pageref{equipaggiamentoarmi}), zaino, due torce, qualche razione di cibo.. un peluche.. quello che ti sembra indispensabile per l'avventura.
Aggiorna poi la parte di scheda relativa alla Difesa segnando che bonus ti da l'armatura e scudo indossata. Ricorda che parti con 100 mo, spendile in maniera accurata!

Entra nella parte, concediti di giocare questo straordinario personaggio. Se mai ti stufassi di giocarlo e volessi provare qualcosa di diverso parlane con il Narratore, saprà consigliarti e suggerirti la strada migliore.
Hai il vantaggio che in OBSS le classi non esistono, il personaggio cresce evolve ed impara in base a ciò che fai e sperimenti. Puoi prepararti la \emph{build} a tavolino ma non avrai mai la certezza che il tuo personaggio si evolva come hai pensato. Lascialo vivere e crescere!

In ultimo ricordati della Legge del Premio\index{Legge del Premio}. Questo mondo è feroce, spesso malvagio, ancor di più vorrà ucciderti, eppure per chi sopravvive c'è la Legge del Premio, una legge che neanche i Patroni possono violare. La Legge è piuttosto semplice nel suo concetto base \emph{A chi sopravvive andranno i tesori e la gloria}.

\begin{narratore}[Personaggi disfunzionali]\index{Personaggi disfunzionali}
Il Narratore può concedere in base alla tipologia di campagna che se il personaggio creato ha tutti i punteggi di Caratteristica a 0 o meno può essere creato nuovamente.
\end{narratore}

\subsection{Avanziamo di Livello}\index{Livello}\index{Avanzamento di Livello}\label{avanzamentodilivello}\index{Livellare}

\begin{enfasi}{
Ma ci sono cose che non si possono capire con la riflessione, bisogna viverle. (La storia infinita, Michael Ende)
}\end{enfasi}

Ogni qual volta il Narratore ti conferma il passaggio di livello sono da compiere diverse operazioni per aggiornare la scheda del personaggio.

\begin{itemize}[leftmargin=*] \setlength{\itemsep}{0pt}
\item Innanzitutto prendete la scheda, matita e gomma ed i dadi (almeno il d6)
\item Aggiornate i Punti Esperienza
\item Aggiornate il Livello aumentandolo di 1
\item Distribuite 1 punto tra Competenza Armi e Competenza Magica
\item Aumentate i Punti Ferita massimi ed attuali di 1d6+Costituzione ed aggiungetene altri 3 se avete dato 1 punto in Competenza Armi. Se il tiro di dado è inferiore a Costituzione, potete prendere come risultato il valore di Costituzione
\item Se avete assegnato un punto in Competenza Armi stabilite se prendete una nuova \hyperlink{lista.armi}{Lista d'Armi} (pag. \pageref{lista.armi}) o approfondite la conoscenza di una lista già appresa
\item Controllate se acquisite una nuova Abilità. Potete prenderne una nuova oppure migliorare una Abilità già appresa, state attenti ai prerequisiti. Vedi \hyperlink{abilita}{Abilità} (pag. \pageref{abilita}).
\item Aggiornate il punteggio dei Tiri Salvezza in base alle nuove Abilità prese.
\item Aggiornate il punteggio dei Tiri per Colpire in base al nuovo valore della Competenza Armi, Abilità, bonus dati dalla Lista d'Armi
\item Distribuite (Int/2)+1, con un minimo di 1 punto, tra le \hyperlink{competenzeelenco}{Competenze Base} (pag. \pageref{competenzeelenco}) conosciute o apprese durante le avventure. Verificate il punteggio di Consapevolezza.
\item Aggiornate il punteggio dei Punti Fato $(20-livello)/5$ , all'intero più vicino
\item Aumentate il punteggio dei Tratti come vi dirà il Narratore. Verificate se avete raggiunto un punteggio sufficiente per acquisire poteri legati ai Tratti
\item Verificate in base al nuovo punteggio di Competenza Magica ed all'Abilità Adepto della Magia il \hyperlink{scuoleelivelli}{livello massimo di incantesimo} (pag. \pageref{scuoleelivelli}) lanciabile ed i \hyperlink{magiepuntimagia}{Punti Magia disponibili} (pag. \pageref{magiepuntimagia})
\item Se avete aumentato la Competenza Magica apprendete 1 nuovo incantesimo dal Tomo di Magia oppure potete apprendere due Trucchetti (Incantesimi di livello 0)
\item Aggiornate la seconda parte della scheda in base al nuovo punteggio di Competenza Magica
\end{itemize}

Come avrete notato i punteggi delle Competenze sono ridotti, si prendono pochi punti da distribuire alla volta.
Come giocatori avete l'opportunità di prediligere un approccio specializzato ovvero \emph{puntare} su poche e specifiche Competenze oppure diluire i punti su più competenze per sapere un pò di tutto e non avere penalità nelle prove (la prova si fa solo con 2d6 + Caratteristica se non avete punti nella Competenza).

Un suggerimento è anche di usare le Abilità, ed in particolare Esperto, che vi concede un bonus di +2 alle prove di Competenze.

\begin{giocatore}[Potere, percepito]
Il livello di potere \textbf{percepito} dei personaggi in OBSS è inferiore a quello di altri GDR. La debolezza del personaggio è solo una percezione ed anzi vi accorgerete presto della vera potenza del personaggio. Giocate di gruppo e sopravviverete perché ricordate che questo è un mondo cattivo, dispettoso e mortale con gli \textbf{egoisti}.
\end{giocatore}

%\subsection{Suggerimenti per divertirsi e sopravvivere nelle avventure di OBSS}\index{Linee guida per i giocatori}\label{suggerimentigiocatori}

\subsection{Come Sopravvivere e Divertirsi}\index{Linee guida per i giocatori}\label{suggerimentigiocatori}

\begin{enfasi}{
-\noindent Ci vuole un piano.

-\noindent Da quando gli eroi hanno bisogno di piani? (Final Fantasy XIII)

\medskip

Vado matto per i piani ben riusciti! (Colonnello John \emph{Hannibal} Smith, A-Team)}
\end{enfasi}\medskip

\begin{itemize}[leftmargin=*] \setlength{\itemsep}{0pt}

\item
Ogni combattimento è potenzialmente letale. Decidi con raziocinio e approccialo con attenzione. Impara a scappare, non aver paura di sopravvivere.

\item
Non c'è tutto nella scheda. La scheda di un personaggio è il perimetro dello stesso ma non definisce ciò che può o non può fare. Spremiti le meningi e sii creativo, alternativo, curioso ma non suicida o avventato.

\item
Non si risolve tutto con un tiro di dado. Fai le domande giuste, parla con i compagni e descrivi con attenzione cosa intendi fare. Il Narratore premia le descrizioni accurate. Descrivere come e cosa si fa può evitare di fare la prova!.

\item
Delle basse Caratteristiche sono solo delle basse Caratteristiche e non il personaggio. Sfrutta le competenze, le Abilità, fai in modo di dover tirare meno dadi possibili per risolvere i problemi.

\item
Improvvisare, adattarsi e raggiungere lo scopo! (Tom Highway - Gunny, Film). Oppure come alcune mie giocatrici preferivano \emph{Improvvisare, \textbf{Ingannare} e raggiungere lo scopo}.

\item
Vivi appieno il tuo personaggio. Amplifica la sua storia porta nel presente il suo passato. Aiuta i compagni a conoscerti ed il Narratore a imbastire storie migliori intorno alle vostre storie.

\item
Una cosa che non potrà mai portarti via nessuno è l'essere eroico, intelligente, risoluto, caparbio, cocciuto ma non stupido.

\item
Descrivi in maniera realistica ciò che fai, aiuterai il Narratore e i compagni intorno a te. E' sicuramente meglio che dire \emph{faccio una prova di Consapevolezza}. Esaltati nel descrivere le azioni più importanti, il Narratore ne terrà conto.

%\item
%E finché non potrai dire "\emph{Io sono cattivo, incazzato e stanco. Sono uno che mangia filo spinato, piscia napalm e riesce a mettere una palla in culo ad una pulce a 200 metri}." (Tom Highway - Gunny, Film) allora stai al tuo posto e non fare lo sbruffone, c'è sempre qualcuno più grosso ed arrabbiato di te.

\item
Ricorda sempre che maggiore è il pericolo maggiore è l'esperienza maturata. più è profondo il dungeon maggiore saranno i tesori e l'esperienza acquisita!

\item
Lo scopo è divertirsi, fare divertire ed assaporare la sfida. Non creare un personaggio che sia contro gli altri personaggi o dia sempre fastidio e problemi. Media il tuo desiderio con le necessità del gruppo, perché sempre e \textbf{solo come gruppo} sopravviverete e mai solo come singolo.

\item
Pensa prima di agire, ma non farti aspettare dagli altri. Usa il tempo tra i tuoi round per pianificare come agire al meglio.

\item
Se hai difficoltà a capire o immaginare qualcosa, chiedi al Narratore maggiori informazioni e chiarimenti, gli farà solo piacere.

\item
Abbraccia il fallimento. Fallire con stile è molto meglio di una noiosa vittoria.

\item
Fai in modo che il tuo personaggio si preoccupi sempre di qualcosa di più che della sua vita.

\item
Non aver paura di discutere con gli altri personaggi, ma assicurati sempre di non andare sul personale con i giocatori.

\end{itemize}

\end{multicols}

\vfill

\begin{enfasi}{
La candela accesa da entrambe le parti dura la metà. (Anonimo)
}\end{enfasi}

%\vfill

%\begin{center}

%\includegraphics[width=0.9\linewidth]{immagini/Granblue.Fantasy.full.2108782.png}

%\filltopageendgraphics[width=0.7\linewidth]{immagini/Granblue.Fantasy.full.2108782.png}

%\emph{Attorno al fuoco, raccontando la giornata trascorsa.}
%\end{center}

%\begin{center}
%\includegraphics[width=0.45\linewidth]{immagini/threasure2.png}
%\end{center}

\pagebreak

\section{Regole per le Competenze}\index{Regole per le Competenze}\index{Competenze}

\begin{enfasi}{
Occorre che la legge sia breve, perché più facilmente i mal pratici la ricordino. (Lucio Anneo Seneca)}\end{enfasi}

\begin{multicols}{2}

Le prove (i check), per le Competenze o Caratteristiche, si eseguono tirando 3d6, al risultato dei dadi si somma il punteggio della Competenza (di base o attiva) e della Caratteristica collegata ed eventuali bonus magici e di circostanza o Abilità, il risultato ottenuto deve essere comunicato al Narratore, il quale lo confronterà con la difficoltà (DC) della prova.

Quando dovete stabilire una difficoltà partite pensando che la prova deve essere rapportata da una persona \emph{normale}. Non pensate \emph{se la dovessi fare io allora la prova sarebbe impossibile}, \emph{se la prova la fa Arsenio Lupin la prova è facilissima}. Partite dal presupposto che la difficoltà deve racchiudere in se tutti gli elementi circostanziali.

Pensate se piove, c'è poca luce, il personaggio sta correndo, è ferito, fa le cose di fretta ed anche alla complessità della cosa che deve fare, saltare un fosso di 3 metri non è come uno di 3 metri al buio, senza scarpe, sotto la pioggia ed inseguiti e con le tasche strapiene di monete...

Decifrare uno scritto antico potrà essere una passeggiata per un linguista esperto, ma per una \emph{persona normale} che non ha idea di cosa può avere davanti la prova è semplicemente impossibile. Questo \emph{impossibile} è la vostra DC, la difficoltà della prova.

E non spaventatevi se i personaggi falliscono le prove, renderà l'avventura più interessante e permetterà al Narratore di introdurre fatti, indizi e nuove avventure.

\begin{narratore}[Non serve sempre una Prova]
Evita di chiedere una prova qualora i giocatori dichiarino \textbf{come} effettuano la prova, come e dove cercano, che dialogo imbastiscono per intimidire l'obiettivo.. Valutate con attenzione come il giocatore descrive ciò che fa perché questa è già la prova. Non è solo per velocizzare il gioco, serve a stimolare i giocatori a pensare in maniera completa ed a calarsi nel personaggio e nell'ambiente.

Renderà il gioco più dinamico e tutti i giocatori parteciperanno alla situazione e collaboreranno dichiarando cosa e come agiscono. Usate sempre il buon senso e risparmiate tiri di dadi! Tirare un dado significa creare la possibilità di fallire!
\end{narratore}

%\medskip
%\begin{center}
%\includegraphics[width=0.8\linewidth]{immagini/master2.png}
%
%\emph{The Master of the Gamblers}
%\end{center}

\medskip

\textbf{Quando devi fare una prova per una Competenza di Base in cui non sei preparato, ovvero non hai punti, devi tirare solo 2d6 + punteggio della Caratteristica collegata}.\index{Prova competenza senza competenze}

Quando si scrive -1d6 significa che si tira un dado in meno (o due se è -2d6), viceversa se c'è scritto +1d6 si tira un dado a 6 in più e si somma.

La tabella qui sotto serve a rapportare la difficoltà alla capacità minima necessaria per riuscire la prova con un tiro medio (un punteggio di 10 lanciando 3d6). Usate queste indicazione per avere una idea delle scale di difficoltà.

Il Narratore non ti dirà fammi una prova a difficoltà 10, ma dirà che la prova non presenta elementi di particolare difficoltà.

%\begin{center}
%\includegraphics[width=0.9\linewidth]{immagini/difficulty.png}
%
%\emph{A City on a Rock, long attributed to Goya, is now thought to have been painted by 19th-century artist Eugenio Lucas Velázquez. Ottima prova di falsificazione}
%\end{center}
%\medskip

\medskip

\textbf{Tabella: Classe di difficoltà}\index[Tabelle]{Tabella Classe di difficoltà}\label{basedifficolta}

\medskip

\noindent\begin{tabularx}{\linewidth}{lll}
	\toprule
\rowcolor{gray!20}\textbf{Diff.} & \textbf{Descrizione} & \textbf{Livello}\\
\textbf{DC}&\textbf{difficoltà}& \textbf{Competenza}\\
\toprule
\rowcolor{gray!20}6 & Estremamente facile & Nulla\\
10 & Facile & Scarsa\\
\rowcolor{gray!20}15 & Normale & Normale\\
20 & Difficile & Buona\\
\rowcolor{gray!20}25 & Molto difficile & Ottima\\
30 & Eroica& Eccellente\\
\rowcolor{gray!20}35 & Quasi impossibile & Stupefacente\\
40 & Impossibile & Epica
\end{tabularx}

\medskip

Se devi fare una prova su una Caratteristica devi tirare 3d6 e sommare il punteggio della Caratteristica e altri modificatori. Comunica questo risultato al Narratore che la confronterà con la difficoltà (DC).

\subsection{Le Golden Rules}\index{Le Golden Rules}\label{goldenrules}

Se non specificato diversamente per tutte le prove di competenza (Base, Attive) valgono tre regole base \index{Regole Base} chiamate \textbf{Golden Rules}:\index{Golden Rules}

\begin{itemize}[leftmargin=*] \setlength{\itemsep}{0pt}
\item
I \textbf{6 esplodono}, ovvero se nella prova dei 3d6 un dado fa sei, somma il risultato e ritira, e se fa 6 nuovamente sommi il risultato e ritiri ancora e ancora..
\item
Gli \textbf{1 portano male}. Quando si tira 1 con un dado, quel dado non contribuisce al risultato. Il valore del dado che mostra 1 viene considerato zero.
\item
\textbf{Affidarsi alla sorte}. Ogni 4 punti tra Competenza (Base o Attiva) e Caratteristica che rinunci a sommare nella prova tiri un dado a 6 in più (Tiro per Colpire, Tiro Salvezza, prove Competenza). Questo valore non può essere tolto dal punteggio dato da Abilità o oggetti magici.



\begin{center}\textbf{\emph{Corollario}}\end{center}\index{Corollario Golden Rules}

\item \textbf{Tirare 3 volte 6 con i primi tre dadi è un successo}, sia nelle Prove di Competenza, Tiri Salvezza e Tiri per Colpire indipendentemente dal risultato finale.\index{Tirare tre volte 6}

\item \textbf{Tirare 3 volte 1 con i primi tre dadi è fallimento}, sia nelle Prove di Competenza, Tiri Salvezza e Tiri per Colpire indipendentemente dal risultato finale. \index{Tirare tre volte 1}

\item\textbf{Gli 1 tirati nelle Prove a seguito di un 6} valgono sempre zero

\end{itemize}

Sfruttate le \textbf{Golden Rules} a vostro vantaggio! Osate, tentate, rischiate quando la situazione non permette altre soluzioni!

\begin{giocatore}[Non c'è solo la scheda!]{
Non cercate per forza la soluzione nella scheda. Usate la vostra capacità di immaginare, di risolvere, di intuire per uscire e risolvere situazioni. La scheda rappresenta solo una piccola parte di ciò che il vostro personaggio può fare.
}\end{giocatore}

\subsection{Superare o Fallire la prova}\index{Superare o Fallire la prova di tanto}\label{superareofallirelaprova}\index{Successo critico nelle prove}

La prova è superata quando tirati i 3d6 e sommata la Competenza interessata e la Caratteristica nonché i vari modificatori il risultato è pari o superiore alla DC stabilita dal Narratore.

Se il risultato è inferiore alla difficoltà la prova è fallita.

Una prova può essere ripetuta\index{Ripetere una prova}\index{Rifare una prova} finché non mutano le condizioni che permettono alla prova di essere ripetuta.

\subsubsection{Successo Critico - Fallimento Critico}\index{Successo Critico - Fallimento Critico}

Se la prova viene \textbf{superata almeno di 8} rispetto alla difficoltà stabilita il Narratore allora si considerà come un Successo Critico.
Il Narratore può decidere di dare maggiori informazioni, concedere un bonus alle azioni successive (+1).. qualsiasi cosa possa valorizzare quanto agevolmente la prova è stata superata.\index{Superare la prova con un Critico}.

Viceversa se la prova fallisce \textbf{di almeno 8 punti} il Narratore potrebbe descrivere come miseramente la prova è fallita e come il risultato pessimo influenzi l'Azione e quelle successive.

Per ogni 8 punti superiori od inferiori alla difficoltà stabilita si conta un Successo Critico o un Fallimento Critico.  Quando nel manuale di parla di 2 Successi Critici significa superare la prova di almeno 16 punti.

A discrezione del Narratore una prova fallita criticamente non può essere ripetuta dallo stesso personaggio.

\begin{narratore}[Ripetere le Prove]
Se la prova può essere ripetuta fino all'eventuale successo senza problemi o interruzioni allora non fate fare la prova, descrivete i tentativi, le difficoltà incontrate e dichiarate il successo.
\end{narratore}

Ragionate su quanto è competente un personaggio al fine di evitare qualsiasi prova dal risultato scontato.

\subsection{Consapevolezza}\index{Consapevolezza}\label{consapevolezza2}

La Consapevolezza è una di quelle competenze che entra in gioco molto spesso.

Fate in modo che siano le domande ed i ragionamenti dei personaggi a rivelare gli indizi, una prova di Consapevolezza potrà essere fatta ogni qual volta ci sia da cercare qualcosa di non ovvio, qualcosa che deve essere cercato altrimenti non risulta immediatamente percettibile o intuibile, qualcosa che i giocatori desiderano trovare e che c'è ma non fanno la domanda giusta.

\begin{narratore}[Non sono le prove a comandare]
Non fate che siano le prove a governare il vostro gioco. \textbf{Fate giocare i giocatori}, fateli recitare, fateli partecipare ed in base a quanto dicono stabilite se la prova è passata o meno.

Se vi dicono \emph{convinco la guardia a farci passare} fate fare una prova di Intimidire (o Diplomazia), se invece intavolano un dialogo convincente potete considerare che la prova sia stata fatta con esito positivo (o negativo se non sono riusciti ad argomentare!) Premiate il COME più che il COSA.
\end{narratore}

\subsection{Le Prove}\index{Prove opposte}\label{proveopposte}\index{Prove Contrapposte}

\subsubsection{Prove di Competenza contrapposte ad un avversario}\index{Prove Contrapposte ad un avversario}

Ci sono situazioni in cui il personaggio deve effettuare una Prova Contrapposta ad un avversario ad esempio Furtività per muoversi silenziosamente alle spalle di una guardia, rubare dalle tasche del mercante, intimidire l'orchetto per farsi dare indicazioni, spingere un avversario..

In questo caso il personaggio effettua la prova indicata la cui \textbf{difficoltà (DC) è pari 10} + il punteggio della Caratteristica + Competenza oppure Tiro Salvezza (come indicato dalla prova) + modificatori (bonus/penalità) contingenti.

Chi ottiene il valore più alto vince, in caso di parità vince chi ha il valore più alto nella Competenza, poi nella Caratteristica ed infine l'eventuale \emph{avversario}. \index{DC Statica nelle prove contrapposte}\index{Prova Contrapposte}

\medskip

\textbf{Alcuni esempi di Prova Contrapposte}

\begin{description}[noitemsep, topsep=0pt, parsep=0pt, partopsep=0pt, leftmargin=0cm]
\item - \textbf{Ingannare qualcuno}: Ingannare Vs Percepire Emozioni
\item - \textbf{Travestirsi per sembrare qualcun altro}: Intrattenere Vs Consapevolezza
\item - \textbf{Creare una mappa falsa}: Falsificare Vs Valutare
\item - \textbf{Furtività} : Competenza Vs Consapevolezza, purché non visto
\item - \textbf{Intimidire}: Intimidire Vs Tiro Salvezza su Volontà (con modificatore Carisma)
\item - \textbf{Rubare}: Mani di Fata Vs Consapevolezza, o Mani di Fata se posseduta
\item - \textbf{Slegarsi da delle corde}: Usare Corde Vs Artista della fuga
\item - \textbf{Braccio di ferro}: Tiro Salvezza Tempra (con modificatore Forza)
\end{description}

\subsubsection{Prove di Caratteristica Contrapposte}\index{Prove di Caratteristica Contrapposte}

Ogni qual volta la \textbf{Prova} o \textbf{Prova Contrapposta} riguarda una \textbf{Caratteristica} e non anche una Competenza fate la prova (3d6) sommando alla Caratteristica più adeguata il Tiro Salvezza più adatto.

\smallskip

\textbf{Tabella: Prove Contrapposte}\index[Tabelle]{Tabella Prove Contrapposte e Caratteristiche}\label{Tabella Prove Contrapposte e Caratteristiche}

\smallskip

\noindent\begin{tabularx}{\linewidth}{Xl}
\toprule
\rowcolor{gray!20}\textbf{Prova Contrapposta}& \textbf{TS} \\
\toprule
Forza& Tempra \\
\rowcolor{gray!20}Destrezza&Riflessi\\
Costituzione& Tempra\\
\rowcolor{gray!20}Intelligenza, Saggezza, Carisma& Volontà
\end{tabularx}

\smallskip

E' possibile che siano chieste Prove Contrapposte con indicato modificatori diversi. Quelli riportati nella tabella sopra sono esempi di utilizzo tipici. E' possibile fare un prova contrapposta di Forza, facendo un Tiro Salvezza su Tempra e sommando il punteggio di Forza per capire chi vince in una gara di sollevamento pesi.

\subsubsection{Prove di Caratteristica non Contrapposte}\index{Prove di Caratteristica non Contrapposte}

Alcune prove possono essere indicate come \emph{Esegui prova di Destrezza a DC 20} senza indicare Tiro Salvezza o Competenze. In questo caso è necessario effettuare la prova sommando esclusivamente il punteggio della caratteristica indicata. Es. 3d6 + 1 (il valore della Destrezza).

\subsubsection{Prove contro una DC statica}\index{Prove non contrapposte, statiche}

Qualora la Prova sia contrapposta ad un \textit{avversario statico}, ovvero non una creatura dotata di Caratteristiche ed Competenze, ma ad una serratura, un salto da compiere.. allora si esegue la prova confrontando 3d6 + la Caratteristica interessata + la Competenza Attiva (TS/CM/CA) o Competenza Base (Disattivare Congegni, Atletica...) più idonea contro la difficoltà (\textbf{DC}) stabilita dal Narratore.

\begin{enfasi}{Audentes fortuna iuvat (\emph{La fortuna aiuta gli audaci}, Virgilio) }\end{enfasi}

\subsection{Bonus e Penalità} \index{Vantaggi}\index{Bonus}\index{Malus}\index{Penalità}\index{Svantaggi}\label{vantaggi}

A seconda delle circostanze potranno esserci dei bonus, vantaggio o penalità, svantaggi nella prove.

Il modificatore nelle \textbf{prove dinamiche}\index{Prove dinamiche} è da usarsi quando la prova viene fatta tirando i 3d6, in questo caso si potranno sommare bonus o penalità (-1, +2...) o addirittura tirare dadi in più od in meno (+1d6, -2d6), fino a non tirare dadi (con 3d6 di penalità)!.

Se le penalità accumulate portano i dadi della prova sotto zero si conta solo il valore della Competenza e Caratteristica.

Si intendono \textbf{prove a valore fisso} \index{Prove a valore fisso} quando il valore non dipende dal tiro di dadi (es. Difesa), in questo caso il punteggio si alza/abbassa del valore indicato.

Cercate di rimanere sempre tra questi valori di bonus e penalità, altrimenti potete direttamente dire che la prova è riuscita o fallita.

Il giocatore può richiedere di effettuare la prova anche se il risultato è certo.

\smallskip

\textbf{Tabella: Vantaggi e Svantaggi}:\index[Tabelle]{Tabella Vantaggi e Svantaggi}

\smallskip

%\noindent\begin{tabular}{lll}
%\multirow{2}*{\textbf{Vantaggio / Svantaggio}} & \multicolumn{2}{c}{\textbf{Prove}}\\
%\cmidrule(lr){2-3} & \textbf{Dinamiche} & \textbf{Fisse} \\
%\hline
%Bonus leggero & +1& +1\\
%Bonus normale & +2 & +2\\
%Bonus forte & +1d6 & +4\\
%Bonus molto forte & +2d6 & +8\\
%Svantaggio leggero & -1 & -1\\
%Svantaggio normale & -2 & -2\\
%Svantaggio forte & -1d6 & -4\\
%Svantaggio molto forte & -2d6 & -8
%\end{tabular}

\noindent\begin{tabular}{lcc}
	\toprule
\rowcolor{gray!20}\multirow{2}*{\textbf{Vantaggio / Svantaggio}} & \multicolumn{2}{c}{\textbf{Prove}}\\
\cmidrule(lr){2-3} & \textbf{Dinamiche} & \textbf{Fisse} \\
\toprule
Leggero & $\pm$1 & $\pm$1\\
\rowcolor{gray!20}Normale & $\pm$2 & $\pm$2\\
Forte & $\pm$1d6 & $\pm$4\\
\rowcolor{gray!20}Molto forte & $\pm$2d6 & $\pm$8
\end{tabular}

\begin{narratore}[Il valore dei dadi]
I bonus e penalità nel tiro di 3d6 hanno più \emph{effetto} che nella prova fatta con il d20. Cercate di rimanere entro i $\pm2$ e solo in situazioni particolari di effettivo e forte vantaggio o svantaggio applicate bonus o penalità maggiori.
\end{narratore}

\subsubsection{Fattore tempo}\index{Fattore tempo}\label{fattoretempo}

\textbf{Se un personaggio non è in difficoltà o pressione}\index{Senza problemi di tempo}\index{Prendere il 10} nell'effettuare la prova può prendere il 10 (+ Caratteristica + Competenze + Abilità..), ovvero considerare che abbia tirato 10 con i dadi. L'azione impiega 10 round. \label{prendere10}

\textbf{Se il personaggio non ha impellenti limiti di tempo}, ovvero può dedicare almeno 10 minuti per lavorarci (60 round) può considerare di prendere 14. Ovvero come se avesse fatto la prova e tirato 14 con i 3d6. \label{prendere14}

\textbf{Se il tempo diventa un fattore da non considerare}, ovvero il personaggio ha almeno 1 ora per pensare e lavorare e non ha alcuna penalità o rischio considerare di avere tirato 18 (ma non c'è nessuna esplosione di dadi o Successo Critico anche se il totale è 18).\label{prendere18}

\begin{narratore}[Consiglio di lettura]
Consiglio a tutti di leggere l'ottimo articolo di Lorenzo Bertini \href{https://dietroschermo.wordpress.com/2022/03/10/elogio-del-10-e-del-20}{Elogio del 10 e del 20} per una disamina critica ed intelligente sul successo e fallimento delle prove.
\end{narratore}

Se vuoi prendere questi valori chiedilo al Narratore, sarà lui che ti dirà se in base alla situazione, urgenza, pericolosità di ciò che ti circonda riesci a prendere il punteggio. Mettersi a scassinare una porta in un dungeon chiedendo il 10 richiede un estremo sangue freddo ed incoscienza. Prendere il 10/14/18 non deve essere concesso per le prove di conoscenza.

\subsubsection{Aiutare un Altro nelle Prove}\label{aiutarealtro}\index{Aiutare un altro nelle Prove}

Si può aiutare un amico in una prova dandogli supporto e suggerimenti. L'aiutante deve effettuare la \textbf{medesima prova} con un bonus di +1d6, se ci riesce non ottiene effetti ma concede un +1 alla prova del compagno. Se esegue un Successo Critico allora il bonus è di +2. L'aiutante esegue la prova come Reazione nel round in cui viene effettuata la prova effettiva.

più personaggi possono aiutare lo stesso personaggio; i bonus di questo tipo sono cumulabili fino ad un bonus pari ad un quarto della difficoltà da battere (es +6 nel caso di difficoltà 25).\index{Aiutare un altro}

\textbf{In caso di prove basate su Competenze chi aiuta deve aver assegnato almeno un punto nella Competenza coinvolta}.

Il Narratore valuterà la possibilità che più di un personaggio fornisca aiuto considerando spazi, modi e tempi (non è facile aiutare qualcuno ad infilare un filo nella cruna di un ago).

Se la prova per aiutare fallisce in modo critico chi doveva essere aiutato ha un -1 di penalità nella prova.

\subsection{Prove fatte dal Narratore}\label{provefattedalnarratore}

Evitate di fare voi le prove al posto dei Giocatori. Siate descrittivi ma non andate a dire al Giocatore che \emph{potrebbe} servire una prova di qualcosa. Qualora dovesse essere necessario eseguire delle prove di nascosto dal giocatore non tirate nessun dado ma aggiungete a 10 il valore della Caratteristica ed il punteggio Competenza o il valore del Tiro Salvezza in questione del personaggio e confrontate il risultato con la difficoltà della prova.

\subsection{Tirare o non Tirare dadi}\label{tirarenontiraredadi}

Non fate tirare dadi per prove che non hanno possibilità di fallire, per le prove che non hanno o generano \textbf{problemi} se sono fallite o possono essere ritentate senza problemi. Fate tirare i dadi ogni qual volta la prova può avere un risultato \textbf{spettacoloso}, \textbf{fallimentare} o innesca ulteriori scene. Fate godere il giocatore del successo o temere del fallimento critico.

\subsubsection{Opzionale - Successo Parziale}\index{Successo Parziale}\index{Prova con Rischio}\index{Opzionale - Successo Parziale}\hypertarget{successoparziale}{}\label{successoparziale}

Una \textbf{Prova con Rischio} si chiede in prove di particolare tensione ed urgenza in cui è più importante il risultato finale che il rischio che si corre. Questa richiesta va fatta prima di tirare i dadi.

Se la prova fallisce di 1 potrà considerarsi riuscita anche se con un problema leggero, se è fallita di 2 si porta dietro un problema serio se è fallita di 3 è riuscita con un problema critico, se è fallita di 4 o più la prova non è comunque riuscita. Applicata a competenze come Conoscenza si può decidere di fornire informazioni non complete oppure in parte vere e false, oppure ancora se si tratta di aprire una serratura si potrebbe rompere il grimaldello nella serratura!

\subsection{Prove di Gruppo}\label{provedigruppo}\hypertarget{provedigruppo}{}\index{Prove di Gruppo}

Ci sono situazioni in cui il gruppo deve fare una prova di competenza ma il risultato deve essere unico, in questo caso se almeno metà del gruppo riesce nella prova questa ha successo.

\subsection{Esempi Prove Competenza}\label{esempiprovecompetenza}\hypertarget{esempiprovecompetenze}{}\index{Esempi prove Competenza}

\textbf{Prove atipiche}\index{Prove atipiche}. Il giocatore è invitato a trovare usi, soluzioni, approcci che esulino dalle più ovvie prove. Siate creativi e descrivete al Narratore la meravigliosa azione che volete fare e come farla! Sarà lui a stabilire in base alla vostra descrizione dell'azione cosa provare e quanto potrà essere difficile.

Le Competenza che hanno un \textbf{*} a fianco al nome, come \textbf{Acrobatica*} hanno le penalità alla prova dovuta dall'armatura portata.

\titlespacing*{\subsubsection}{0pt}{0.5em}{0.5em}\subsubsection*{Acrobatica*}\index{Acrobatica} \label{acrobatica}
Una prova di Acrobatica riuscita con DC 15 permette al personaggio di ridurre di 3 il danno quando cade entro 6 metri (\textbf{Reazione}).

\textbf{Scendere o Salire} entro 50 cm è terreno difficile, tra i 50 e 150 cm è terreno doppiamente difficile, oltre è cadere o arrampicarsi. Il danno da caduta è 1d6 danni ogni 3 metri in caduta. \index{Scendere e Salire}

Vedi paragrafo \hyperlink{cadute}{Cadute} (pag. \pageref{cadute}) per i dettagli su come usare Acrobatica quando si cade.

\titlespacing*{\subsubsection}{0pt}{0.5em}{0.5em}\subsubsection*{Arrampicarsi/Scalare*} \index{Arrampicarsi}\index{Scalare}\label{arrampicarsi}

Arrampicarsi, scalari o scendere da una superficie impervia equivale a muoversi in un \textbf{terreno doppiamente difficile}.

\medskip

\noindent\begin{tabularx}{\linewidth}{Xl}
	\toprule
 \rowcolor{gray!20}\textbf{Esempio di Superficie} & \textbf{DC}\\
	\toprule
	Movimento solo dimezzato & -2d6\\
 \rowcolor{gray!20}Superficie scivolosa&+4\\
	{\small Parete grezza con appigli, mattoni sporgenti}&+12\\
 \rowcolor{gray!20}Un albero, una corda senza nodi&+15\\
	Un muro con pochi mattoni sporgenti &+20\\
 \rowcolor{gray!20}Un muro con pochissimi appigli&+25\\
	Una parete naturale liscia senza appigli&+30\\
 \rowcolor{gray!20}Ti puoi appoggiare a 2 pareti opposte&-8\\
	Ti puoi appoggiare a 2 pareti angolari&-4\\
	\midrule
 \rowcolor{gray!20}Esempi di prove con uso della corda&\\
	\midrule
	Usare una corda per calarsi&12\\
 \rowcolor{gray!20}Usare una corda per arrampicarsi&15\\
	La corda ha nodi & -3
\end{tabularx}

\medskip

In caso di fallimento della prova si consuma l'Azione senza spostarsi. Se si ottiene un Fallimento Critico perdi la presa e puoi fare un Tiro Salvezza su Riflessi alla stessa difficoltà per afferrarti a qualcosa, se fallisci anche il TS cadi fino in fondo.

In caso di \textbf{Successo Critico} nella prova scali, ti arrampichi o scendi come fosse terreno difficile. \textbf{Usare una corda} consente di trattare l'arrampicata come terreno difficile.

Vedi anche la \hyperlink{pareti}{Tabella: Pareti}, pag. \pageref{pareti}.

\titlespacing*{\subsubsection}{0pt}{0.5em}{0.5em}\subsubsection*{Arcana - Riconoscere un oggetto magico} \index{Riconoscere oggetto magico}\label{rinoscereoggettomagico}\hypertarget{rinoscereoggettomagico}{}

Per riconoscere un oggetto magico le sue capacità è necessaria una prova di \textbf{Arcana} a difficoltà 20 per avere indicazioni di massima sui poteri e ambiti di utilizzo, solo con un risultato di almeno 25 nella prova, puoi apprenderne i dettagli, bonus magici e cariche. \textbf{10 minuti}. Con punteggio Arcana 6 impiega 5 minuti, con 12 impiega 1 minuto, con Arcana 18 impiega 1 Round eseguire la prova.

\titlespacing*{\subsubsection}{0pt}{0.5em}{0.5em}\subsubsection*{Arcana - Riconoscere un incantesimo} \index{Riconoscere un incantesimo} \label{riconoscereincantesimo}\hypertarget{riconoscereincantesimo}{}
Mentre viene lanciato è una prova di \textbf{Arcana} a DC pari la 10 + livello dell'incantesimo. Costa una \textbf{Reazione}. Se fatto assieme al lancio di un \hyperlink{Controincantesimo}{Controincantesimo} non costa Reazione.

\titlespacing*{\subsubsection}{0pt}{0.5em}{0.5em}\subsubsection*{Atletica*}\index[Tabelle]{Tabella Saltare} \emph{Penalità dovuta all'Armatura.} \textbf{1 Azione}\index{Saltare}\label{atletica}\label{saltare}

La \textbf{distanza saltata in lungo} è pari a 30cm per risultato ottenuto nella prova, arrotondando all'intero più vicino. Es. se nella prova di saltare faccio 11, il salto sarà lungo 30cm*11=330cm=3 metri, con 16 nella prova è 30cm*16=480cm=5m.

La \textbf{distanza saltata in alto} è pari a 10cm per risultato ottenuto nella prova.

In un \textbf{salto in lungo} la punta più alta del salto è pari ad un 1/3 della lunghezza saltata. Se esegui un salto in lungo di 3 metri a metà salto sei in alto di 1 metro.

Se non si ha almeno 3 metri di rincorsa si salta la metà. In lungo si salta al massimo il proprio movimento ed in alto la metà.

Effettuare un Salto da fermo costa 1 Azione. Un Salto effettuato entro metà del proprio movimento (quindi si salta entro 4 metri percorsi per un umano) usa la stessa Azione del Movimento, altrimenti consumi una Azione per il Movimento ed una Azione per il Salto.

\titlespacing*{\subsubsection}{0pt}{0.5em}{0.5em}\subsubsection*{Conoscenza - Identificare una pozione o veleno naturale}\index{Identificare Veleno}\index{Erboristeria} \index{Identificare Pozione}\label{identificarepozioni}
è possibile con una prova di \textbf{Erboristeria} a DC uguale al fattore di rarità della pianta, oppure il TS che questa concede in caso di Veleni.

Impiega 1 Azione ogni 10 di DC. Con 6 in Erboristeria il tempo è 1 Azione ogni 15 di DC, con 12 punti è 1 Azione ogni 20 DC eseguire la prova. Se si fallisce con un Fallimento Critico si è venuti a contatto/ingerito parte della pozione e se ne subiscono gli effetti.

\titlespacing*{\subsubsection}{0pt}{0.5em}{0.5em}\subsubsection*{Conoscenza - Identificare una creatura} \index{Riconoscere una creatura}\label{riconosceremostro}\hypertarget{riconosceremostro}{}
si effettua una prova di Conoscenza. Controlla il capitolo \hyperlink{riconoscereimostri}{Riconoscere i Mostri} nel Mostruario (pag. \pageref{riconoscereimostri}). Costa 1 Azione.

\titlespacing*{\subsubsection}{0pt}{0.5em}{0.5em}\subsubsection*{Intimidire}\index{Intimidire}\label{intimidire}
Il personaggio usa \textbf{1 Azione} ed esegue una Prova Contrapposta al Tiro Salvezza su Volontà con bonus dato dal Carisma.
Se il Tiro Salvezza fallisce, l'avversario fino alla fine del suo round successivo ha -1 al Tiro per Colpire contro colui che l'ha intimidito. L'avversario deve avere Intelligenza pari o maggiore di -3. Il Tiro Salvezza prende un modificatore di $\pm2$ per taglia di differenza. In caso di Successo Critico il modificatore diventa -2.

Se chi tenta la prova di Intimidire esegue un Fallimento Critico subisce le medesime penalità come se fosse stato intimidito.

%\medskip

%\begin{center}
%	\includegraphics[width=0.7\linewidth]{immagini/Foster_Bible_Pictures.png}
%
%	\emph{Bible Pictures and What They Teach Us}
%\end{center}

\titlespacing*{\subsubsection}{0pt}{0.5em}{0.5em}\subsubsection*{Furtività*} \index{Nascondersi} \index{Muoversi silenziosamente}\index{Furtivita'}\label{furtivita}

Furtività raccoglie le capacità di muoversi silenziosamente, nascondersi nelle ombre, passare non visto e tutte quelle azioni che richiedono di non essere visti o sentiti.
Cercare di muoversi silenziosamente non costa Azioni, è \emph{compresa} nell'Azione di Movimento usata per spostarsi. Il terreno viene però trattato come difficile e se già lo fosse diviene doppiamente difficile. Muoversi a piena velocità cercando di non fare rumore impone alla prova di Furtività una penalità di 2d6.

Usando \textbf{1 Azione} puoi cercare di nasconderti dalla vista degli avversari. Non è possibile nascondersi se l'ambiente non lo permette, per quando la tua prova possa essere alta non puoi nasconderti se non c'è qualcosa che ti può nascondere od occultare. Per nascondersi dietro una creatura questa deve essere almeno di 3 taglie superiori alla tua (altrimenti la creatura fornisce solo copertura).

\titlespacing*{\subsubsection}{0pt}{0.5em}{0.5em}\subsubsection*{Gestire animali - Ammansire un animale}\index{Ammansire animale}\index{Gestire animali}\label{gestireanimali}
è una prova di \textbf{Gestire Animali} a DC 12+2*GS dell'animale. Impiega 1 minuto ogni 3 di DC, con 6 punti il tempo è 1 minuto ogni 6 di DC, con 12 è 1 minuto ogni 10 DC eseguire la prova. La creatura deve avere Intelligenza -3 o superiore.

\titlespacing*{\subsubsection}{0pt}{0.5em}{0.5em}\subsubsection*{Nuotare*}\index{Nuotare}\label{compnuotare}

In acqua calme DC 10, in acque mosse ha DC 15, in acque molto mosse DC 20, tempestose DC 25. La prova è necessaria per stare a galla o nuotare. Nuotare in acqua si considera \textbf{terreno difficile}.

Vedi Capitolo \hyperlink{combatteresottacqua}{Avventure in Acqua} (pag. \pageref{combatteresottacqua}).

\titlespacing*{\subsubsection}{0pt}{0.5em}{0.5em}\subsubsection*{Professione}
Un eventuale prova sulla \textbf{Professione} viene fatta con 3d6+Saggezza+metà del livello.

%\subsubsection*{Artista della Fuga}
%1 Azione ogni 10 di DC. 6p 1 Azione 15 di DC, 12p 1 Azione 20 DC.

\titlespacing*{\subsubsection}{0pt}{0.5em}{0.5em}\subsubsection*{Pronto Soccorso}\hypertarget{prontosoccorso}{}\label{prontosoccorso}\index{Pronto Soccorso}

Se il personaggio ha Punti Ferita negativi, è morente, la prova di Pronto Soccorso, 3 Azioni, a difficoltà 12 più il valore dei Punti Ferita negativi porterà il personaggio a 0 Punti Ferita, ovvero svenuto. Ogni volta successiva che il personaggio torna sotto 0 Punti Ferita la difficoltà della prova di Pronto Soccorso aumenta di 2.

Una prova riuscita (DC 15) fa recuperare 1d4 Punti Ferita \textbf{dopo uno scontro}, se il personaggio non è morente, o concede un +2 ad un Tiro Salvezza su Tempra per resistere ad un veleno. Da fare entro 1 Turno dal termine del combattimento. Costo \textbf{2 minuti}.

Con punteggio 6 costa 1 minuto e recuperi 1d4+4 PF. Con punteggio 12 costa 3 round e recuperi 2d4+8 PF, con punteggio 18 costa 1 round e recuperi 3d4+12 PF.

Una prova riuscita (base DC 12) riduce di 1 i danni da \hyperlink{sanguinamento}{\textbf{Sanguinamento}}. Per ogni valore di Sanguinamento sopra 1 la difficoltà aumenta di 2. Costo \textbf{2 Azioni}. Un trattamento di 1 minuto garantisce 1 successo, senza prova. Per ogni Successo Critico riduci il sanguinamento di un punto ulteriore.

Una prova riuscita (base DC 13) per \textbf{prendersi cura per 8 ore} di un paziente fa recuperare a questo il doppio dei Punti Ferita, con un minimo di 4, e concede un nuovo Tiro Salvezza su Tempra per debellare a Malattie naturali o Veleni già in corso.
Se effettuato durante le ore di riposo chi amministra la cura risulterà Affaticato.

Oggetti come \hyperlink{borsadaguaritore}{Borsa da Guaritore} (pag. \pageref{borsadaguaritore}) e \hyperlink{Fermasangue}{Fermasangue} (pag. \pageref{fermasangue}) possono essere utili nelle prove.

\titlespacing*{\subsubsection}{0pt}{0.5em}{0.5em}\subsubsection*{Seguire Tracce}\index{Seguire Tracce}\label{seguiretracce}

Alla \textbf{Difficoltà base di 15} si applicano i seguenti modificatori:

\medskip

\noindent\begin{tabularx}{\linewidth}{ll}
 \rowcolor{gray!20}Se il terreno è molto morbido& DC -4\\
	Se il terreno è stabile& DC +5\\
 \rowcolor{gray!20}Se il terreno è duro& DC +10\\
	A seconda della taglia& DC $\pm4$\\
 \rowcolor{gray!20}Ogni 3 creature inseguite& DC -2\\
	Ogni 24 ore passate& DC +4\\
 \rowcolor{gray!20}Ogni ora di pioggia& DC +4\\
	Visibilità scarsa& DC +2\\
 \rowcolor{gray!20}Cerca di occultare le tracce& DC +4
\end{tabularx}

\titlespacing*{\subsubsection}{0pt}{0.5em}{0.5em}\subsubsection*{Valutare}\label{compvalutare}\index{Valutare}
DC 12 + fattore rarità oggetto. Comune +0, Non Comune +2, Raro +6, Molto Raro +10, Leggendario +16. 3 Azioni

\titlespacing*{\subsubsection}{0pt}{0.5em}{0.5em}\subsubsection*{Sopravvivenza}\index{Sopravvivenza}\label{sopravvivenza}

Sopravvivenza può essere usata al posto di \textbf{Disattivare Congegni} con un -1d6 per disattivare trappole o serrature. 1 Azione per DC.

Ogni tre punti ottenuti nella prova di Sopravvivenza oltre la DC (solitamente 13) il personaggio è in grado di \textbf{procacciare cibo} per se stesso ed un altra persona purché si trovi in un ambiente capace di sostenere la vita.

Si può usare per cercare trappole: 1 minuto per cercare trappole in 3x3 metri, con punteggio 6 costa 3 round, con punteggio di 12 costa 1 round, con punteggio 18 costa 1 Azione.

\subsubsection*{Punteggio Competenze}\label{Punteggio Competenze}\index{Punteggio Competenze}

Quando nel manuale si parla di \emph{valore o punteggio Competenza} si intende sempre il valore della competenza compreso di tutti i punteggi e modificatori.

\begin{giocatore}[Prove Prove e Prove!]
	Ad essere cinici un gioco di ruolo è tutta una prova, vuoi per riuscire a fare un salto, per colpire qualcuno, per evitare una trappola od un incantesimo...!
	Devi essere più intelligente e furbo. Le prove possono essere spesso evitate o affrontate con vantaggio. Gioca con arguzia, usa la tua immaginazione, sii creativo!
\end{giocatore}


\begin{narratore}[Il ruolo delle Prove]
	L'esecuzione e la gestione delle prove determina il tipo di gioco. È fondamentale ascoltare i giocatori, percepire il loro entusiasmo e comprendere gli obiettivi delle loro azioni. Un giocatore coinvolto trasmette entusiasmo a tutto il gruppo.

	%Quando i giocatori chiedono semplicemente di fare una \emph{prova Consapevolezza} o di \emph{convincere la guardia}, assecondate le loro intenzioni ma cercate di coinvolgerli maggiormente.

	\textbf{Non c'è una regola per tutto ma divertimento e buon senso non devono mai mancare}!
\end{narratore}


\titlespacing*{\subsubsection}{0pt}{0.5em}{0.5em}\subsubsection*{Linguaggi}\index{Linguaggi}\hypertarget{linguaggi}{}\label{linguaggi}

Nel mondo ci sono le vecchie lingue umane, usate solo negli antichi tomi ed in comunità isolate e c'è la lingua Comune costruita dall'insieme dei vecchi idiomi terrestri e comprensibile più o meno a chiunque. Ogni personaggio che abbia almeno Intelligenza -2 parla il linguaggio della propria cultura, con 0 lo scrive. Per ogni punto pari o superiore a 2 parla e scrive un altra lingua che sarà scelta alla creazione del personaggio. Per ogni punto speso nella Competenza Conoscenza Lingua parla e scrive un altra lingua.

Le lingue segnate con un \textbf{*} possono essere parlate solo da creature appartenenti a quella specie o gruppo culturale.

Le creature extraplanari come Celestiali, Demoni, Diavoli, Draghi, Elfi, Nani, Gnomi ... parlano e scrivono le proprie lingue.

\smallskip

\textbf{Tabella delle Lingue}\index[Tabelle]{Tabella delle Lingue}

\smallskip

{\noindent\begin{tabularx}{\linewidth}{lXX}
		\toprule
\rowcolor{gray!20}\textbf{Ambito culturale}& \textbf{Parlato} & \textbf{Scritto}\\
\toprule
%Umano & Comune& Comune\\
%Nanico& Nanico& Nanico\\
%Elfico& Elfico & Elfico \\
%Gnomico& Gnomica & Gnomica \\
Vecchie lingue terrestri& varie & varie\\
%Gnoll & Gnoll & Goblinoide\\
%Giganti& Gigante & Gigante\\
%Orco& Orchesco & Orchesco \\
\rowcolor{gray!20}Creature marine* & Aquan& Elfico\\
%Creature marine senzienti* & Aquan& - \\
Uccelli senzienti& Ictun & -\\
\rowcolor{gray!20}Abitanti dei boschi& Silvano& - \\
%Celestiale& Celestiale & Celestiale\\
%Infernale & Infernale & Infernale\\
%Abissale & Abissale& Abissale\\
%Draghi& Draconico & Draconico\\
Elementali del Fuoco* & Ignan&-\\
\rowcolor{gray!20}Elementali della Terra*& Tremun &-\\
Elementali dell'Acqua* & Aquan & - \\
\rowcolor{gray!20}Elementali dell'Aria*& Ictun &-\\
Non-morti & Expiran & - \\
%Sottosuolo & Profondità& Profondità\\
%Lingua dei Segni*& dei Segni & - \\
\end{tabularx}}

\medskip

La \textbf{Telepatia}\index{Telepatia} è un mezzo per parlare con qualsiasi creatura che abbia Intelligenza almeno -3. Non c'è il vincolo del linguaggio, la telepatia funge da traduttore universale.

%\begin{center}
%\includegraphics[width=0.8\linewidth]{immagini/Pieter_Bruegel_the_Elder-The_Tower_of_Babel.png}
%
%\emph{La Torre di Babele, Pieter Bruegel il Vecchio.}
%\end{center}

%{\small

\end{multicols}

\pagebreak

\section{Combattimento Sociale}\index{Combattimento Sociale}

\begin{enfasi}{

Per formulare la dialettica in modo limpido bisogna considerarla, senza badare alla verità oggettiva (che è oggetto della logica), semplicemente come l'arte di ottenere ragione, la qual cosa sarà certo tanto più facile se si ha oggettivamente ragione. (Arthur Schopenhauer)

}\end{enfasi}

\begin{multicols}{2}

Per Combattimento Sociale si intende il tentativo da parte dei personaggi di convincere, forzare o raggirare i PNG o comunque creature tenute dal Narratore a fare o dire cose che non vorrebbero.

Può capitare che i giocatori tentino di corrompere una guardia, di ottenere informazioni in maniera diplomatica oppure intimidatoria, di ottenere una paga più alta, di raggirare un mercante o più semplicemente ogni qual volta lo \emph{scontro} o \emph{confronto} non è tramite armi ma a parole.

Per quanto il combattimento sociale possa riguardare una moltitudine di situazioni quello che accomuna tutte le prove è il metodo con cui si vuole ottenere il risultato finale

\begin{wrapfigure}[16]{r}[.5\width+.5\columnsep]{7.5cm}%\itshape

\centering
\includegraphics[width=6.5cm]{immagini/Greuter_Socrates.png}

\emph{Socrates and His Students. Johann Friedrich Greuter, 17th century.}
\end{wrapfigure}

, non tramite armi ma cercando di \emph{convincere} l'avversario.

In queste situazioni si possono seguire due approcci distinti, da una parte il Narratore valuta il risultato in base a quanto dicono i giocatori, dall'altra questo sistema imposta le regole come fosse un combattimento per stabilire chi vince nella prova finale.

Ogni Narratore sceglie l'approccio che preferisce, diciamo che in base all'esperienza con il sistema ed il gioco di ruolo in generale potrebbe preferire un sistema o l'altro. Per un approccio neutro usare delle regole può essere più indicato.

A seconda che il giocatore usi metodi più o meno coercitivi l'avversario resisterà di conseguenza.
Il giocatore eseguirà una Prova Contrapposta di Intimidire, Diplomazia od Ingannare e l'avversario cercherà di resistere con un Tiro Salvezza su Volontà con bonus Carisma.
Se si deve resistere ad una coercizione basata su minacce contrapponete un Tiro Salvezza su Volontà con bonus di Forza.

Il Narratore in base al livello del PNG stabilirà quanti successi consecutivi sono necessari per convincerlo. In linea di massima è necessario 1 + 1 successo ogni due livelli del PNG.\index{Successi necessari per convincere} Il numero di successi può essere modificato in base alle convinzioni, promesse, patti, rapporti interpersonali che l'avversario ha riguardo alla situazione.

Se si vincono tutte le prove si vincerà il \emph{combattimento} e si otterrà l'informazione o quanto richiesto. In caso di Successo Critico si conteranno due successi.

In caso di fallimento della prova questa può essere riprovata con un -1 di penalità se le conseguenze del fallimento non portano ad una scena successiva.

\begin{wrapfigure}[16]{l}[.5\width+.5\columnsep]{7.5cm}

\centering
\end{wrapfigure}

Se il fallimento è critico allora non solo la prova è fallita ma non sarà possibile effettuare ulteriori tentativi e l'avversario diverrà ancora meno amichevole. Molto probabilmente il Narratore deciderà l'evoluzione della situazione in base alla richiesta e scena originale.

In caso di Intimidazione molto probabilmente l'obiettivo del giocatore diventerà ostile, in caso di Inganno è possibile che sentendosi ingannata menta o che non dica nulla. In caso di Diplomazia è più probabile un silenzio o un cortese diniego.

Il Narratore deve usare queste prove, che siano risultate positive o negative, per fare evolvere la scena ed arricchire l'avventura.

Se c'è una informazione che non volete dare impostate una difficoltà più alta.
Ricordate che i personaggi si fidano del risultato ottenuto e se incominciate a dare informazioni false non potranno più fidarsi delle prove fatte.

Non dovete pensare che dare l'informazione sia un problema, alla fine i giocatori se la sono guadagnata e per voi è una nuova possibilità per arricchire l'avventura.

\end{multicols}

\vfill

\begin{enfasi}{
Colui che non vuole ragionare è un fanatico, colui che non sa ragionare è un pazzo e colui che non osa ragionare è uno schiavo. (William Drummond di Hawthornden)
}\end{enfasi}

\pagebreak

\section{Combattimento Armato}\label{combattimento-armato}\index{Combattimento Armato}

\begin{enfasi}{
Si vis pacem, para bellum (\emph{Se vuoi la pace, prepara la guerra}, Vegezio, libro III, Epitoma rei militaris)
\smallskip

Non conta come cadi, ma se e come ti rialzi (anonimo)

\smallskip

Non sono un eroe. No e non lo sarò mai. Sono solo un cattivo che viene pagato per pestare tipi peggiori di lui. (Deadpool)

\smallskip

Occhio per occhio... e il mondo diventa cieco (Mahatma Gandhi, NdA i suoi Tratti aborrivano la violenza!)}\end{enfasi}

\begin{multicols}{2}

Il combattimento è tra le fasi principali di un avventura ed è quando i personaggi cercano, con risultati alterni, di dare sfoggio della loro maestria con le armi o magie.

Il combattimento è diviso in 2 fasi:\index{Combattimento}
\begin{itemize}
\item verifica dell'iniziativa
\item risoluzione delle azioni (movimento, attacco, azioni varie..)
\end{itemize}

\begin{center}
\includegraphics[width=0.65\linewidth]{immagini/Achildbookofwarriors.png}

\emph{A child's book of warriors (1907), William Canton}
\end{center}

\subsection{L'Iniziativa}\index{Iniziativa}\label{iniziativa}

L'iniziativa è una prova (3d6) di Destrezza o Intelligenza ed Abilità inerenti che potete avere.

Il giocatore sceglie la Caratteristica che preferisce. Se viene scelta la Destrezza saranno i riflessi a determinare la reazione del personaggio, mentre l'Intelligenza guiderà la capacità di cogliere le tattiche dell'avversario ed anticiparle.

Chi ha l'iniziativa tra giocatori e nemici più alta incomincia per primo e successivamente agiscono gli altri in ordine decrescente, dichiarando le Azioni ed eseguendole. In caso di Iniziativa di pari punteggio agisce per primo chi ha il punteggio Caratteristica più alto, altrimenti lo scontro sarà in contemporanea. L'iniziativa vale per l'intero scontro e si ritira al cambio dell'avversario.\index{Iniziativa uguale}

\begin{narratore}[Gestire il combattimento] %box narratore
Cercate di fare fluire il combattimento in maniera naturale. Non interrompete il flusso delle azioni, bensì descrivendone gli effetti coinvolgete i giocatori (e nemici) nelle azioni seguenti. Vi consiglio la lettura dell'articolo \href{https://theangrygm.com/manage-combat-like-a-dolphin/}{How to Manage Combat Like a Dolphin} per capire nel dettaglio il metodo.
\end{narratore}

\textbf{Anche nella Prova di Iniziativa valgono le Golden Rules.}\index{Iniziativa e Golden Rules}

\subsubsection{Risoluzione delle Azioni}\index{Risoluzione delle Azioni}\label{risoluzionedelleazioni}

\begin{enfasi}{
Non è vero che abbiamo poco tempo: la verità è che ne perdiamo molto. (Lucio Anneo Seneca)
}
\end{enfasi}

Dal più veloce al più lento c'è la risoluzione delle Azioni.

Il Narratore chiederà al più veloce, quello con l'iniziativa più alta, di dichiarare le sue Azioni ed agire, proseguirà poi a chiedere e fare agire gli altri giocatori e nemici.

In questo modo la scelta dell'azione avviene quando è il round del giocatore che potrà agire anche in base alle Azioni e risoluzioni già avvenute.

%\begin{center}
%\includegraphics[width=0.9\linewidth]{immagini/Arthur-Pyle_Two_Knights.png}
%\emph{Howard Pyle, from the 1903 edition of The Story of King Arthur and His Knights}
%\end{center}

\subsubsection{Il Tempo (Round, Minuti e Turni)}\index{Round}\label{iltempo}

\begin{enfasi}{L'esitazione è la morte del vantaggio. Magic di V.E. Schwab} \end{enfasi}

Un \textbf{round} dura 10 secondi circa, è un lasso di tempo sufficiente per agire, correre, parlare.. combattere. Un Minuto sono quindi 6 round ed un Turno dura 10 Minuti (o 60 round).\index{Round e Turno, durata}

I round si usano nelle scene di combattimento o dove la tensione deve rimanere costantemente alta ed ad ogni Azione corrisponde un evolversi della situazione.

\subsubsection{Tempo di riattivazione Oggetti ed Abilita'}\index{Tempo di attivazione Oggetti ed Abilità}\label{temporiattivazioneoggetti}\index{Ricarica oggetti magici}

Se non specificato diversamente un oggetto o Abilità che prevede un certo numero di usi al giorno \emph{es. una volta al giorno} si ricarica all'alba successiva l'uso.

%\begin{center}
%\includegraphics[width=0.6\linewidth]{immagini/hjford-fight.png}

%\emph{\\Fairy book - Fairytale illustration, Henry Justice Ford}
%\end{center}

\subsection{Azioni nel Round}\index{Azioni nel Round}\index{Azione}\label{azioninelround}

%\begin{enfasi}{
%Un vero uomo d'azione vede subito dinanzi a sé tante cose da fare che il %lavoro non gli mancherà mai e riuscirà. (Fëdor Dostoevskij)
%} \end{enfasi}\end{changemargin}\medskip

Un personaggio può eseguire fino a 3 Azioni, 1 Azione Immediata, ed 1 Azione di Reazione per Round. Può anche usare 1 o più Azioni Gratuite se a disposizione.

Se l'iniziativa tirata è la più veloce ed è di +8 o più punti superiore alla seconda allora in quel round potrà usare una Azione di Reazione od Immediata in più, se il differenziale è di almeno +16 la sua grande reattività gli permette di eseguire una Azione in più.\index{Critico nell'Iniziativa}

Le Azioni possono essere eseguite nell'ordine preferito.

Nella tabella sottostante sono indicate le Azioni principali che un personaggio può fare, sono linee guida da seguire. Nel capitolo dedicato al combattimento e degli esempi d'uso delle competenze vengono elencate altre Azioni ed i loro costi relativi in Azioni.

\textbf{Una Azione non può essere interrotta da un altra Azione, ma può essere seguita da una Azione di Reazione o da una Azione Immediata}. \index{Interrompere Azioni}\index{Azioni, Interrompere}
Se un personaggio vuole fare più attacchi spostandosi nel campo di battaglia può usare una Azione per eseguire un attacco, usare una Azione di Movimento per spostarsi fino a tutto il suo movimento a disposizione, ed usare un ultima Azione di attacco per eseguire un ultimo singolo attacco, questo secondo attacco conta come attacco multiplo con le relative penalità.

E' possibile \textbf{ritardare} una o più Azioni\index{Ritardare Azioni} per aspettare lo svolgersi delle scene. Il personaggio che ritarda una sua Azione agisce per primo tra i soggetti che agiscono in quel valore di iniziativa, nei successivi round continuerà ad agire nel nuovo ordine di iniziativa. In questa maniera il giocatore ritarda volontariamente la sua iniziativa per inserirsi nell'ordine delle iniziative in un altro posto.

Un giocatore che dichiara di aspettare una certa situazione per poter agire equivale ad eseguire una o più \textbf{Azioni Preparate}\index{Azioni Preparate}. In questo caso il personaggio (o nemico) agisce \textbf{dopo} l'Azione scatenante con le sue Azioni ma rimane nel suo ordine di iniziativa al termine del round.

Se il personaggio ha già compiuto tutte le Azioni allora potrà agire nel round solo con un \textbf{Azione Immediata} e fuori dalla sua iniziativa solo tramite una Reazione, se a disposizione. L'\textbf{Azione di Reazione} si attiva sempre dopo l'Azione scatenante.
La \textbf{Azioni Gratuite} possono essere usate in qualsiasi momento.

\textbf{Azione di Attacco}: si intende sia l'uso di armi in mischia che l'uso di armi da lancio o tiro come archi, balestre o pugnali da lancio. Nel caso di armi da lancio ogni lancio/tiro conta come un attacco.

Il personaggio che esegue una Azione di Attacco ed il Lancio di un Incantesimo nel medesimo round si considera Distratto ovvero deve eseguire una Prova di Magia per lanciare l'incantesimo.\index{Attacco ed Incantesimo}

\textbf{Azione di Movimento*}: un Azione di Movimento è una Azione dedicata a spostarsi. Ci si può spostare fino a tutto il proprio movimento (9 metri per umani, 6 metri per nani..) per Azione usata. Ogni movimento consuma una Azione anche se non si sfrutta tutto il proprio movimento a disposizione.\index{Azione di Movimento}

Durante l'Azione di Movimento è possibile \textbf{Estrarre l'Arma} o Scudo o \textbf{Rinfoderare l'Arma} o lo Scudo.

\medskip

\textbf{Tabella: Azioni per Round}\index[Tabelle]{Tabella delle Azioni per Round}

\medskip

%\noindent\begin{tabularx}{1\linewidth}{lc}
\noindent\begin{tabular}{lc}
	\toprule
\rowcolor{gray!20}\textbf{Cosa si fa} & \textbf{Azioni}\\
\toprule
\hyperlink{tiropercolpireedifesa}{Eseguire un attacco}& 1\\
\rowcolor{gray!20}Eseguire due attacchi& 2\\
\hyperlink{attacchimultiplimischia}{Eseguire più di due attacchi}& 3\\
\rowcolor{gray!20}Estrarre o Rinfoderare l'arma o scudo& 1\\
\midrule
\hyperlink{tipodimovimento}{Eseguire una Azione di Movimento} &1*\\
\rowcolor{gray!20}\hyperlink{azionediscatto}{Scatto} & 1\\
\hyperlink{alzarsidaprono}{Alzarsi da prono}& 1\\
\midrule
\rowcolor{gray!20}\hyperlink{aiutare}{Aiutare qualcuno}& R\\
\hyperlink{esempiprovecompetenze}{Eseguire prova su una competenza}& 1*\\
\rowcolor{gray!20}\hyperlink{riconosceremostro}{Riconoscere una creatura}& 1\\
\hyperlink{copertura}{Nascondersi}& 1\\
\midrule
\rowcolor{gray!20}\hyperlink{cavalcare}{Salire o scendere dalla cavalcatura}& 2\\
\hyperlink{sfondare}{Sfondare una porta a spallate/calci}& 1\\
\rowcolor{gray!20}\hyperlink{piedediporco}{Forzare porta con piede di porco}& 2\\
\midrule
Cercare qualcosa nello zaino& 2\\
\rowcolor{gray!20}{\small Prendere qualcosa dalla cintura o di pronto} & 1\\
Usare un oggetto tenuto in mano& 1\\
\midrule
\rowcolor{gray!20}\hyperlink{insorgenzaveleno}{Bere una pozione tenuta in mano}& Imm.\\
\hyperlink{insorgenzaveleno}{Fare bere una pozione ad un altro} & 2\\
\midrule
\rowcolor{gray!20}Gettare un oggetto tenuto in mano& R\\
Gettarsi a terra prono& R\\
\midrule
\rowcolor{gray!20}\hyperlink{magietempodilancio}{Lanciare un Incantesimo}*& 2\\
\hyperlink{magieconcentrazione}{Concentrarsi su un Incantesimo}& 1\\
\rowcolor{gray!20}\hyperlink{magiedurata}{Interrompere un proprio incantesimo} & Imm.\\
\hyperlink{riconoscereincantesimo}{Riconoscere un Incantesimo}& R\\
\rowcolor{gray!20}\hyperlink{regoleoggettimagici}{Usare un oggetto magico}& 2\\
\midrule
Scambiare un dialogo con qualcuno& 3*\\
\rowcolor{gray!20}Scambiare poche battute con qualcuno& 0*\\
\midrule
\hyperlink{preparareladifesa}{Preparare la Difesa} & 1\\
\rowcolor{gray!20}\hyperlink{difesatotale}{Difesa Totale} & 2\\
\hyperlink{disingaggiare}{Disingaggiare} & 1\\
\rowcolor{gray!20}\hyperlink{colpopreciso}{Colpo preciso} & 2\\
\midrule
\hyperlink{disarmare}{Disarmare} & 2\\
\rowcolor{gray!20}\hyperlink{finta}{Finta} & 1\\
\hyperlink{spingereavversario}{Spingere un avversario} & 2\\
\rowcolor{gray!20}\hyperlink{afferrareunavversario}{Afferrare l'avversario} & 2\\
\hyperlink{farecadereavversario}{Fare cadere l'avversario} & 2
\end{tabular}

\medskip

\textbf{Lanciare un Incantesimo*}: solitamente sono necessarie 2 Azioni. Nella descrizione dell'incantesimo è indicato il numero di Azioni necessarie. Nel capitolo della Magia sono specificate le \hyperlink{piumagieround}{regole} (pag. \pageref{piumagieround}) per lanciare più incantesimi nel round.

\textbf{Scambiare un dialogo con qualcuno*}: Un dialogo può essere di pochi secondi se non di minuti. Il Narratore valuterà quanto questo dura.

\textbf{Scambiare poche battute con qualcuno*}: Finché sono veramente poche battute o uno sguardo non consuma Azioni, se questo diventa più articolato allora utilizza delle Azioni. L'obiettivo è non interrompere il flusso delle Azioni con un fitto dialogo ma comunque permettere l'interazione tra i personaggi.

\textbf{Eseguire prova su una competenza*}: se sfruttano una frazione del round costano 1 Azione, altrimenti 2 o più. Controllate negli \hyperlink{esempiprovecompetenze}{Esempi Prove Competenza} i costi riportati.

Una Azione di \textbf{Reazione (R)} \index{Azione di Reazione}può essere eseguita liberamente anche fuori dal proprio round. Questa Azione è solitamente dovuta ad Abilità o situazioni particolari. Se non indicato diversamente una Azione di Reazione accade immediatamente dopo la causa che la scatena.

Una Azione \textbf{Immediata (Imm.)} \index{Azione Immediata}può essere eseguita liberamente nel proprio round, primo o dopo la propria Azione. Una Azione Immediata è solitamente concessa da particolari Abilità.

E' possibile, se non descritto specificatamente nell'Abilità, eseguire solo una Azione Immediata ed una Azione di Reazione per round.

\smallskip

Questo \textbf{elenco non è completo}, prendetelo come linee guida per stabilire il peso delle decisioni ed azioni dei personaggi. Una Azione dura circa 3 secondi.

L'\textbf{ordine} con cui si eseguono le Azioni non è importante se non per correlazione logica e fisica. L'Azione di Movimento può essere tra altre Azioni (movimento, attacco/incantesimi/altra Azione, movimento).

Un personaggio potrebbe attaccare, muoversi ed ancora attaccare, questo secondo attacco avrebbe le penalità descritte negli attacchi multipli.

\end{multicols}

\vfill


\begin{center}

\includegraphics[width=0.8\linewidth]{immagini/Perseus_Fighting_Phineus_and_his_Companions.png}

	\emph{Luca Giordano: Perseus turning Phineas and his Followers to Stone}

\end{center}

\bigskip

\begin{enfasi}
I tesori non si vincono con cautela e previdenza, ma con uccisione rapida e attacco sconsiderato. (Michael Moorcock)
\end{enfasi}


\pagebreak

\subsection{Movimento}\index{Movimento}\label{movimento}

\begin{enfasi}{Un mobile più lento non può essere raggiunto da uno più rapido; giacché quello che segue deve arrivare al punto che occupava quello che è seguito e dove questo non è più (quando il secondo arriva); in tal modo il primo conserva sempre un vantaggio sul secondo. (Paradosso di Zenone)}
\end{enfasi}

\begin{multicols}{2}

Il movimento di un personaggio è dato dalla sua taglia e razza e da ciò che porta, dai pesi, ingombri ma anche magie ed oggetti magici.

Il Movimento scritto nella razza del personaggio è l'indicazione di quanti metri per Azione (di Movimento) il personaggio può fare.

Una creatura o personaggio potrebbe anche decidere di spostarsi più velocemente del solito ovvero correndo (Azione di Scatto).\label{azionediscatto}\hypertarget{azionediscatto}{}

L'Azione di Scatto è una Azione di Movimento particolare, consiste nel correre per quell'Azione.
Se si esegue un'Azione di \textbf{Scatto} \index{Scatto}si raddoppiano i metri percorsi (2x9 metri per un umano), per un nano (Movimento 6m) significa fare 12 metri, in una Azione.
E' anche possibile fare più Azioni di Scatto, fino a 3 in un round, ovvero correre per 6 volte il proprio movimento.

Il personaggio che fa una Azione di Scatto \index{Azione di Scatto}corre ed ha una penalità di 1d6 nel Tiro per Colpire, la Difesa diminuisce di 4 fino all'inizio del suo round successivo e si considera Distratto per il lancio di incantesimi.\index{Penalita' per correre}

Non è possibile spostarsi anche solo di 1 metro se non si spendono Azioni di Movimento.

Queste precisazioni hanno senso e vanno usate quando si tratta di combattere ed il dislocamento sul territorio, sulla mappa, è fondamentale. Durante gli spostamenti normali, mentre si cavalca o cammina liberi senza pericoli si usa la normale gestione del movimento orario.

Quando si parla di \textbf{\emph{quadretto}} \index{Quadretto}per indicare una distanza od una influenza si intende un quadretto di mappa di 1 metro x 1 metro.

%Nel caso di spostamento diagonale\index{Movimento diagonale}\index{Spostarsi di lato} si conta una distanza di 1,5 metri per quadretto, in caso di arrotondamenti sull'ultimo quadretto si fa per difetto, ovvero si torna indietro all'ultimo attraversato.

Nel caso di \textbf{spostamento diagonale}\index{Movimento diagonale}\index{Spostarsi di lato} per praticità contate un quadretto normalmente.

\textbf{Se ci si sposta su un terreno \emph{difficile}, si percorre la metà del movimento disponibile quindi un umano copre 4 metri per Azione di Movimento (ogni quadretto attraversato conta per due).}

Nel Mostruario sono indicate le dimensioni e relativi spazi occupati dalle creature di \hyperlink{tagliaedimensioni}{taglia} diversa (pag \pageref{tagliaedimensioni}).

\subsection{Distanza}\index{Distanza}\label{distanza}

Per \textbf{distanza di Tocco} \index{Distanza di Tocco} \index{Tocco}si intende una distanza che permette il toccare l'avversario, quindi non più di un metro per creature di taglia media senza armi lunghe o con portata. La distanza di tocco è distanza di mischia qualora non si usino armi lunghe.

Per \textbf{distanza di Mischia} \index{Distanza di Mischia} \index{Mischia}si intende una distanza che permette il combattimento corpo a corpo (entro 1 metro attorno al personaggio, oppure entro 2 metri in caso di arma lunga).

Nei mostri questa distanza è indicata dalla portata, per le armi da lancio è chiamata gittata.

Se non indicata nella scheda del avversario la \textbf{portata} è pari a metà dello spazio occupato arrotondato per eccesso. Un gigante delle colline, taglia enorme (3x3 quadretti sulla mappa), ha portata 2 quadretti, ovvero colpisce creature entro 2 quadretti/metri da lui.\index{Taglia e distanza di mischia}\index{Portata}

\begin{narratore}[Distanza in Combattimento]
Es. per una creatura armata di lancia la portata è 2 ovvero la distanza di mischia è 2 metri perché l'arma è lunga. Per uno gnomo armato di martello, o a mani nude, la distanza di mischia è 1 metro. Per creature particolarmente grandi (Enormi o più) con armi altrettanto grandi la portata viene indicata o si desume dalle dimensioni del mostro e dalla tipologia di arma.

\textbf{La portata indica fino a che distanza puoi colpire in mischia.}
\end{narratore}

\end{multicols}

\vfill

\begin{center}
	\includegraphics[width=0.5\textwidth]{immagini/camminata.png}
\end{center}

\pagebreak

\subsection{Vita e Morte}\index{Morire}\label{morire}

\begin{enfasi}{Chi non conosce la morte, non conosce la vita. (Grand Hotel, film 1932)

\medskip

Il meritevole Game Master non uccide mai volontariamente i personaggi dei giocatori. Lui presenta le opportunità per i giocatori frettolosi e sbadati di fare tutto da soli. (Gary Gygax)}\end{enfasi}

\begin{multicols}{2}

Il danno da arma si calcola come somma del dado dell'arma, Forza (o Destrezza se indicato da Abilità) che sia positiva o negativa, bonus dati da Lista d'Armi, bonus dati Abilità, bonus dati dall'arma e bonus circostanziali.\index{Come calcolare il danno dell'arma}\index{Danno dell'arma}

Quando una creatura raggiunge i 0 (zero) Punti Ferita si considera svenuto\index{Svenuto}, ovvero Indifeso ed Inabile a fare qualsiasi cosa. Una Cura magica (Incantesimo, Pozione..) lo porterà cosciente ed ai Punti Ferita curati. Una prova di \hyperlink{prontosoccorso}{Pronto Soccorso} (pag. \pageref{prontosoccorso}) (DC 12) potrà essere usata per riportarlo cosciente a 1 Punto Ferita.
Se lasciato svenuto dopo un ora se non è successo qualcosa a mutare la situazione il personaggio può fare un Tiro Salvezza su Tempra a DC 15, se riesce torna a 1 Punto Ferita, se fallisce va a -1 e diventa morente.\index{Zero PF, recupero da}\index{0 Punti Ferita}

Un personaggio morente ha Punti Ferita negativi (-1 o meno) ed è svenuto e \hyperlink{morente}{indifeso}. Continuerà a perdere 1 Punto Ferita a round fiche il valore non raggiungerà il doppio della Costituzione +10 ed il personaggio morirà, se non viene curato.

Una magia (incantesimo o pozione) di Cura, di qualsiasi ammontare di cura lo porterà a 1 Punto Ferita, successive cure funzioneranno normalmente.

Una prova di \hyperlink{prontosoccorso}{Pronto Soccorso}, 3 Azioni, a difficoltà 12 più il valore dei Punti Ferita negativi porterà il personaggio a 0 Punti Ferita, ovvero svenuto. Ogni volta successiva che il personaggio torna sotto 0 Punti Ferita la difficoltà della prova di Pronto Soccorso aumenta di 2.

\begin{giocatore}[Tups sta morendo]
Es. Tups é gravemente ferito ed ha attualmente -6 Punti Ferita, Jade decide di provare a curarlo (dopo averlo spostato in un posto più sicuro). Jade tenta una prova di Pronto Soccorso (3 Azioni) per stabilizzare il compagno, la sua difficoltà alla prova é 12+6 ovvero deve superare con Pronto Soccorso DC 18 per riportarlo a 0 Punti Ferita (svenuto)

Una successiva prova di Pronto Soccorso, svolta entro 10 minuti, potrà sanare ulteriori ferite.
\end{giocatore}

Un personaggio morente che subisce ulteriore danno, come nemici che infieriscono sul corpo od incantesimi diretti a lui od ad area, continua a sottrarre Punti Ferita con il rischio di morire.

\begin{center}
\includegraphics[width=0.8\linewidth]{immagini/Nuremberg_chronicles.png}

\emph{The Dance of Death (1493) by Michael Wolgemut, Nuremberg Chronicle of Hartmann Schedel}
\end{center}

\medskip

Le \textbf{Condizioni} \index{Condizioni mentali}\textbf{di tipo mentale} quali Affascinato, Confuso ma non Dominato, terminano quando il personaggio diventa morente.

Se un attacco o incantesimo porta il personaggio direttamente a -(10+COS*2), il personaggio muore\index{Morte immediata}\index{Danno Massivo} senza possibilità di essere curato.

Quando un personaggio torna a Punti Ferita positivi dopo che era andato negativo perde la metà dei Punti Magia rimanenti con una riduzione di almeno 10 Punti Magia e diviene ulteriormente \hyperlink{affaticato}{\textbf{Affaticato}} (pag. \pageref{affaticato}).

Quando un personaggio arriva a Punti Ferita negativi pari 10+doppio del suo punteggio di Costituzione é \hyperlink{morto}{\textbf{morto}} [-(10+(COS*2))].

Un personaggio con i Punti Ferita non Letali a 0 o meno sviene finché i Punti Ferita normali non sono tornati ad 1.

Es. Se ha Costituzione 2 morirà a -[10+4]=-14 Punti Ferita, se ha Costituzione 0 morirà a -10 Punti Ferita, se ha Costituzione -2 morirà a -[10-4]=-6 Punti Ferita. In caso di valori di Costituzione pari od inferiore a -3 il personaggio muore a -5 Punti Ferita.

Se il danno non letale di un personaggio arriva a Punti Ferita negativi pari 20+4*Costituzione il personaggio é morto.\hypertarget{puntiferitatemporaneimorte}{}

\begin{narratore}[Recitare]
Descrivete con pathos e trasporto la caduta del personaggio, fate capire la sofferenza provata. Enfatizzate la caduta a terra, il sangue che sgorga, i rantoli. Siate teatrali.
Se avete a che fare con giocatori facilmente impressionabili allora è meglio ridurre il \emph{gore}.
\end{narratore}

Un personaggio morto non può beneficiare delle cure normali o magiche, e non può essere riportato in vita da un incantesimo. Solo un Patrono ha sufficiente potere per riportare l'anima nel corpo e riportare in vita la creatura. L'incantesimo di \hyperlink{Animare Morti}{Animare Morti} può rianimare un corpo, ma come non morto.

\begin{giocatore}[La morte del personaggio]
Cerca di capire perché è morto, quali sono le cause, gli errori commessi. Quali sono le scelte che lo hanno portato fino a li. Ogni personaggio che muore è una ferita personale ma anche esperienza e consapevolezza. Fanne tesoro sia tu ma anche tutto il gruppo. Se qualcosa non ha funzionato cercate di capirlo insieme, senza accusarsi o darsi colpe ma con lo spirito consapevole che si può migliorare, tutti.
\end{giocatore}

\subsubsection{Opzionale - Recupero da 0 Punti Ferita} \index{Recupero} \index{Svenuto}\index{Opzionale - Recupero da 0 Punti Ferita}\label{recuperozeropf}

\begin{enfasi}{
Le notizie sulla mia morte sono molto esagerate. (Samuel Clemens)
}\end{enfasi}

\textbf{Nel caso vogliate un sistema meno letale potete applicare questa regola opzionale.}

Ogni round successivo ad essere andato a 0 Punti Ferita o meno, quindi svenuto o morente, il personaggio deve effettuare un Tiro Salvezza su Tempra a difficoltà 15, se riesce riprende coscienza e va ad 1 punto ferita.

Se fallisce il Tiro Salvezza può effettuarne un altro a DC +1 rispetto alla precedente il round successivo. Quanto la difficoltà raggiunge 18 (ovvero 3 prove fallite di seguito) il personaggio muore.

Appena la prova riesce (entro i 3 fallimenti) il personaggio torna ad 1 punto ferita ed è affaticato. Ogni volta che torna a meno di 0 Punti Ferita la difficoltà iniziale (15) aumenta di 1.

\subsubsection{Recupero punti Caratteristica}\index{Recupero punti caratteristica}\label{recuperopunticcaratteristica}

Eventuali punti Caratteristica persi si recuperano al ritmo di 1 punto al giorno, se non indicati come perdita permanente.

\subsubsection{Recupero Punti Ferita naturale}\index{Recupero Punti Ferita naturale}\label{recuperopuntiferitanaturale}

Per ogni notte di riposo (almeno 8 ore) recuperi in Punti Ferita il valore di Costituzione * CA o CM (a scelta del personaggio, con un minimo di PF pari a CA o CM).

\subsubsection{Recupero Punti Ferita non letali}\index{Recupero Punti Ferita non letali}\index{Punti Ferita non letali}\label{recuperopuntiferitanonletali}\hypertarget{recuperopuntiferitanonletali}{}

Ogni ora si recupera, con un minimo di 1 Punto Ferita, il proprio valore di Costituzione.

\subsubsection{Punti Ferita Massimi}\index{Recupero Punti Ferita Massimi}\index{Punti Ferita massimi}\label{puntiferitamassimi}

Se non indicato diversamente ogni qual volta il personaggio subisce un danno che abbassa i Punti Ferita Massimi oltre ad abbassare questi deve anche sottrarli dai Punti Ferita attuali. Un personaggio quando curato non può superare i Punti Ferita Massimi attuali.

Ogni 8 ore di riposo, nelle 24 ore, si recupera 1d4 + Costituzione in Punti Ferita Massimi, con un minimo di 1.

\end{multicols}

\vfill

\begin{center}

%\includegraphics[width=0.7\linewidth]{immagini/caravaggioSalomeLondon.png}

%\emph{Salomè con la testa del Battista è un dipinto di Caravaggio realizzato in olio su tela (91x106 cm) tra il 1607 e il 1610.\\ È conservato nella National Gallery di Londra.}
\includegraphics[width=0.45\linewidth]{immagini/giantdeath.png}

\emph{Henry Justice Ford}

\end{center}

\pagebreak

\subsection{Tiro per Colpire e Difesa}\index{Tiro per Colpire}\index{Difesa}\label{tiropercolpireedifesa}\hypertarget{tiropercolpireedifesa}{}

\begin{enfasi}{Applica sempre la giusta forza, mai troppa mai troppo poca. (Kano Jigoro)}\end{enfasi}

\begin{multicols}{2}

Il \textbf{Tiro per Colpire} è dato dall'insieme delle capacità combattive (Competenza Armi e bonus concessi da Lista d'Armi), Forza, armi magiche e tutto ciò che influisce nel combattimento. Se l'\textbf{attaccante} porta l'attacco con:

\begin{itemize}[leftmargin=*] \setlength{\itemsep}{0pt}
\item \textbf{Armi da Mischia o Contatto}: l'attaccante deve effettuare un \textbf{Tiro per Colpire (TC)}= 3d6 + Competenza Armi + Forza + eventuali bonus dati dalla Lista d'Armi + Abilità + bonus magici dell'arma e fattori circostanziali (ambiente, maledizioni..)

\item
\textbf{Armi da Distanza}: l'attaccante deve effettuare un Tiro per Colpire (TC) = 3d6 + Competenza Armi + Destrezza + eventuali bonus dati dalla Lista d'Armi + Abilità + bonus magici dell'arma e fattori circostanziali (ambiente, maledizioni..). Vale per archi, balestre, pugnali tirati, giavellotti...

\item
\textbf{Incantesimo}: vedi Capitolo sulla \hyperlink{magietiropercolpireconlemagie}{Magia} (pag. \pageref{magietiropercolpireconlemagie})
\end{itemize}

Il giocatore può decidere di rinunciare a parte del bonus dato dalla Competenza Armi per avere un migliore punteggio di Difesa. Questi punti non saranno a disposizione nell'attacco successivo (vedi Altre azioni e situazioni).

%\medskip

%\begin{center}
%\includegraphics[width=0.9\linewidth]{immagini/3fightmen.png}

%\emph{Henry Thomas Alken - Three Men, Yale Center for British Art}
%\end{center}

%\medskip

Chi si \textbf{difende} ha una \textbf{Difesa} pari a: 10 + Destrezza + Scudo + Armatura + bonus magici + Abilità e bonus circostanziali (copertura ad esempio), per i mostri il valore della Difesa è già calcolato dei valori normali, sarà eventualmente da modificare dai valori pertinenti alla situazione.

Si intende Difesa naturale il punteggio di 10 alla base del calcolo della Difesa, alcune eccezionali abilità possono aumentare questo valore.\index{Difesa naturale}

\subsection{La Difesa e l'Attacco}\index{Difesa}\index{Attacco}\label{difesaeattacco}

\begin{enfasi}{La difesa è sempre legittima (anonima vittima)}\end{enfasi}

Ogni Tiro per Colpire si raffronta la Difesa.

Se il \textbf{Tiro per Colpire} è pari o superiore al valore della Difesa l'avversario è stato colpito e si stabilirà il danno della ferita, dato dal dado dell'arma + punteggio Forza ed altri fattori quali bonus magici, Lista d'Armi ed Abilità.

Se il Tiro per Colpire (TC) è più basso della Difesa allora l'avversario avrà parato, schivato, evitato.. La scelta la si lascia al giocatore (o Narratore), evitato l'attacco non si subiscono ferite.

Ci sono situazioni che possono avvantaggiare la Difesa quali coperture, nascondigli, trincee, porte, compagni di taglia molto più grande della propria, invisibilità... Consultate i paragrafi relativi ai \hyperlink{copertura}{Nascondigli e Coperture} per capire il vantaggio che possono dare.

Ci sono occasioni in cui non è importante penetrare la difesa e ferire l'avversario ma semplicemente basta toccarlo.

Altre volte l'avversario è sorpreso e non può difendersi completamente.

Se è \textbf{sufficiente toccare l'avversario} il Tiro per Colpire ha +1d6 di bonus dato che non è necessario portare il colpo quanto solo sfiorarlo.\index{Toccare l'avversario}. Nel manuale è chiamato Attacco a Tocco.\index{Attacco a tocco}\label{attaccoatocco}\hypertarget{attaccoatocco}{}\label{difesaatocco}

Se \textbf{l'avversario è sorpreso} ovvero non si aspetta l'attacco la Difesa ed il Tiro Salvezza su Riflessi avranno una penalità di -2. Questo è il valore della \textbf{Difesa di sorpresa}.\label{difesasorpresi}

\textbf{Anche per il Tiro per Colpire valgono le Golden Rules}. I d6 esplodono in caso si tiri 6 con il dado, fare 1 porta male (vale zero) ed affidarsi alla sorte (ovvero togliere 4 punti tra Competenza Armi e Forza o Destrezza per aggiungere 1d6 al Tiro per Colpire, non dai bonus dati da Lista d'Armi od Abilità od oggetti magici).

Se i modificatori e circostanze portano il danno inflitto ad essere zero o negativo comunque farai 1 di danno.
Questa regola si applica ai modificatori del danno dell'arma che appunto non possono portare il danno totale ad essere inferiore a 1, se ci sono protezioni magiche o riduzioni del danno questo può diventare zero e quindi non ferirai l'avversario (ma se diventa negativo non lo curi!).

Quando effettui il Tiro per Colpire assicurati di aver conteggiato tutti i modificatori a te noti e ricorda, che per ogni 6 tirato (nei 3d6 del Tiro per Colpire) devi tirarne un altro e continuare a tirare finché continui a fare 6 con il dado.

\begin{center}
	\includegraphics[width=0.6\linewidth]{immagini/esplosionedanno.png}

	\emph{Henry Justice Ford}
\end{center}

Puoi \textbf{togliere 4} o multipli al tuo attacco per tirare un d6 in più. La scelta è da fare nelle situazioni più disperate dove solo la fortuna può risolvere il duello. Il valore lo togli dal punteggio di Competenza Armi o di Forza o Destrezza non da punteggi dati da Abilità, Liste d'Armi o bonus magici.

Se colpisci, per ogni margine di 8 superiore alla Difesa dell'avversario , l'arma fa del danno in più ovvero un Tiro Critico. Tira nuovamente solo il dado del danno dell'arma, senza altri modificatori.

Anche per il Tiro per Colpire valgono le regole base delle Competenze. La Difesa è un valore fisso e come tale usa i modificatori per le prove a valore fisso.

\begin{narratore}[Partecipazione]
OBSS vuole essere divertente da giocare, vuole che i giocatori si divertano e vedano i risultati ottenuti dai dadi (e ovviamente dalle loro scelte). Le Golden Rules e l'Esplosione del Danno vogliono proprio togliere la patina di polvere ai dadi e fare divertire. Un giocatore, apprezzerà, ancor più se di esperienza, come i tiri dei dadi non siano solo un numero ma bensì aprano la possibilità di fare la differenza. Chiedete al giocatore di descrivere il colpo critico e fatelo recitare nella sua gloria di potenza!
\end{narratore}

\subsection{Tirare 3 volte 1}\index{Tirare 2 volte 1 oppure 2 volte 2 ed una volta 1}\label{tiraretrevolteuno}\index{Tirare 3 volte 1}

Se hai tirato tre volte 1 hai mancato, indipendentemente dal risultato finale. Il Narratore potrebbe anche decidere che succedono brutte cose... (ad esempio vedi \hyperlink{tabellafallimentiarmi}{Tabella Fallimento Tiri per Colpire}, pag. \pageref{tabellafallimentiarmi})

\begin{center}
	\includegraphics[width=0.9\linewidth]{immagini/critico.png}

	\emph{Henry Justice Ford}
\end{center}


\subsection{Tirare 3 volte 6}\index{Tirare 3 volte 6}\label{tiraretrevoltesei}

Se nei primi 3 Tiri per Colpire fai tre volte 6 prenderai l'avversario indipendentemente dal risultato finale del Tiro per Colpire. Oltre ad avere la certezza di aver fatto un Tiro Critico, il Narratore potrebbe decidere di applicare qualche effetto descrittivo (od effettivo) ulteriore. E ricordati di continuare a tirare quei magnifici dadi nella speranza di fare ancora 6!

\subsection{Tiro Critico}\index{Tiro Critico}\index{Danno critico}\label{tirocritico}

Ogni qual volta hai colpito, tiri un \textbf{danno aggiuntivo della sola arma} per ogni margine di 8 che hai superato la Difesa con il tuto Tiro per Colpire, questo danno viene anche chiamato \textbf{danno critico}. Se hai fatto due Tiri Critici vuole dire che devi tirare 2 dadi di arma in più e che hai colpito con un margine tra +16 e +23!

\begin{giocatore}[Esempio Tiro Critico]
Es tiro 6 4 5, tiro in aggiunta 6, tiro in aggiunta un 6, tiro in aggiunta 4, totale 31. La Difesa dell'avversario è 15. Come danno tiri 3 volte il danno dell'arma, una volta perché ho colpito ed due perché l'hai colpito con un margine di 16!.
\end{giocatore}


\subsection{Esplosione del Danno}\index{Esplosione del Danno}\label{esplosionedeldanno}

Ogni qual volta nel tiro del dado dell'arma ottieni il valore massimo (nel classico d8 per la spada lunga ad esempio fai 8 ed è quindi il valore massimo del dado), ritiri il dado e sommi ancora il valore (del solo dado).

In caso di armi con più dadi (esempio 2d4, il valore massimo deve essere ottenuto come somma dei due dadi, ovvero 8). Non c'è esplosione del danno per le armi con danno massimo inferiore od uguale a 6.

Alcune armi hanno una esplosione del danno diversa. Nella tabella delle armi dove è segnato EDX (es ED9), il valore X sta per il valore minimo sufficiente per tirare un'altra volta il danno, quindi in caso di ED9 puoi fare l'esplosione del danno con 9 o più con il dado dell'arma.

Questa è una caratteristica di poche armi estremamente letali.

L'esplosione del danno non esplode a sua volta, anche se fai il massimo del dado con il dado aggiunto questo non esplode nuovamente.

I tiri di dado aggiunti grazie al Tiro Critico non hanno il vantaggio dell'esplosione del danno. Se il dado dell'arma tirato grazie al Tiro Critico fa il massimo non ritiri il dado. Eventualmente usa dadi diversi tra loro quando tiri i dadi dell'armi.


\subsection{Attacchi multipli}\index{Attacchi multipli}\label{attacchimultiplimischia}\hypertarget{attacchimultiplimischia}{}
Con \textbf{una Azione} il personaggio può eseguire un \textbf{singolo Tiro per Colpire}.
Con \textbf{due Azioni} il personaggio può effettuare fino a \textbf{due Tiri per Colpire}. \textbf{Se vuole fare 3 o più attacchi deve usare 3 Azioni}.

Ogni singola freccia, dardo, pugnale o arma con gittata scagliata conta come un attacco.\index{Attacchi multipli con armi da distanza}
La prima Azione di attacco non ha penalità mentre la seconda Azione di attacco ha -5 al Tiro per Colpire. Successivi Tiri per Colpire cumuleranno -5 al colpire, quindi un terzo attacco avrà -10 ed un quarto attacco -15...
Se la penalità al colpire cumulativa diventa maggiore del Tiro per Colpire non è più possibile fare ulteriori attacchi.

I personaggi con Tiro per Colpire meno di 6 possono scegliere di effettuare 2 attacchi spendendo 2 Azioni ma applicando una penalità di -4 ad entrambi gli attacchi invece della progressione standard. Questo permette anche ai personaggi di livello basso di sfruttare efficacemente le loro Azioni in combattimento anche se con significative penalità.

\begin{giocatore}[Esempio Attacco Multiplo]
Ad esempio se ho Competenza Armi 5, Forza 1, +2 al compire come bonus dalla Lista d'Armi e +1 al colpire dato da una Abilità, +2 perché fiancheggio e +1 per arma magica il primo Tiro per Colpire sarà 3d6+12, il secondo sarà 3d6+7, il terzo 3d6+2. Non è possibile effettuare un quarto attacco in quanto il bonus al colpire diventerebbe negativo.
\end{giocatore}

Eventuali bonus al colpire dinamici e non \emph{fissi}, es. +1d6, si applicano al solo primo Tiro per Colpire e non al computo del bonus per calcolare il numero di attacchi multipli. Nel caso di esempio il Tiro per Colpire diventa 4d6+12, 3d6+7 e poi 3d6+2.

Il giocatore può dichiarare di effettuare gli attacchi su bersagli differenti. Ogni attacco può essere inframmezzato da una Azione di Movimento, purché si abbiano abbastanza Azioni.

I personaggi che non possono effettuare attacchi multipli possono utilizzare le Azioni rimanenti per spostarsi, cercare di fiancheggiare con un compagno, mettersi sulla difensiva o recuperare un oggetto.


%\begin{center}
%\includegraphics[width=0.9\linewidth]{immagini/archer.png}
%
%\emph{Scythian archers in ancient attic vase painting}
%\end{center}

\subsection{Armi da Lancio}\index{Attacchi multipli armi da lancio}\label{armidatiro}\index{Armi da Tiro}\index{Armi da Lancio}

Le armi da lancio, o da tiro, sono tutte le armi con una gittata, ovvero che possono essere lanciate o lanciano dei proiettili. Le principali armi da lancio sono gli archi, balestre, fionde ma anche pugnali, giavellotti o lance qualora siano scagliate.

Il bonus al danno dato da Forza si applica in automatico per le fionde, pugnali, giavellotti..ovvero con tutte le armi che vengono scagliate di forza, gli archi applicano questo bonus solo se sono di tipo composito, le balestre non lo applicano mai.

La Destrezza modifica solo il Tiro per Colpire.

\textbf{I proiettili lanciati da Archi, Fionde, Balestre magiche non si considerano magici}.

\textbf{In caso di proiettili magici questi sommano il loro bonus magico al Tiro per Colpire ed al danno}.

In ogni arma da lancio è indicata la gittata ovvero entro che distanza è possibile tirare il proiettile senza penalità. Ogni arma da lancio può colpire entro tre volte la gittata indicata.

Se l'obiettivo è entro la gittata indicata non si hanno penalità al colpire, se il target è tra il primo e secondo incremento la penalità al colpire è -6. Se il target è tra il secondo è terzo incremento la penalità al colpire è di -12.\index{Penalita' distanza}

Un pugnale tirato entro 6 metri non ha penalità, tirato tra i 6 ed i 12 metri ha un -6 al colpire, a distanza tra 12 e 18 metri un -12 al colpire, oltre non può essere tirato.

%\begin{center}
%\includegraphics[width=0.75\linewidth]{immagini/fenice.png}
%
%\emph{Henry Justice Ford}
%\end{center}
\subsection{Arma Lunga} \index{Arma Lunga}\label{armalunga}

l'arma lunga permette di colpire un obiettivo a distanza di 2 metri.

Usare \textbf{Arma lunga a breve distanza} \index{Arma lunga a breve distanza}\label{armalungabrevedistanza}, inferiore ai 2 metri, comporta una penalità al Tiro per Colpire di -4, ad eccezione dell'utilizzo del Bastone.

\begin{giocatore}[Combattimento con Arma Lunga]
		Es. Tups armato di spada lunga affronta uno brigante armato di lancia lunga. Tups ha iniziativa 15, il brigante 12.

		Tups sfruttando la sua agilità arriva sotto il brigante colpendolo potentemente. Il brigante trovandosi in mischia con Tups non riesce a sfruttare la sua arma lunga che anzi lo penalizza.

		Usa una Azione per allontanarsi di due metri e poi attacca.

		Come terza azioni si allontana di altri 9 metri e urla imprechi verso Tups.

		Tups è a questo punto a 11 metri dall'avversario, decide di caricare aprendo così la propria difesa ma ottenendo un bonus al colpire.

		Carica il brigante colpendolo e arrivandogli addosso, con un ultima Azione decide di migliorare la sua Difesa (Preparare la Difesa).

		Il brigante molto ferito prova a colpirlo confidando che la sua difficoltà ad usare una arma lunga così da vicino sia bilanciata dalle penalità date dalla corsa di Tups. Tups viene colpito ed il brigante getta a terra la lancia ed estrae un corto pugnale e si mette sulla difensiva anche lui.

\end{giocatore}

\subsection{Arma Doppia} \index{Arma Doppia}\label{armadippia}

un'arma doppia è un'arma che è pericolosa da entrambe le estremità. Può essere usata come arma singola, oppure, incorrendo nelle penalità del combattimento con due armi, come appunto due armi.

Se non specificato un'arma doppia usata per combattimento con due armi equivale ad usare due armi medie.

\subsection{Armi Versatili} \index{Armi Versatili}\label{armiversatili}

le armi con il talento Versatile possono usare la Destrezza invece della Forza sui Tiri per Colpire. Al danno si usa sempre la Forza.

\subsection{Armi Leggere} \index{Armi Leggere}\label{armileggere}

queste armi sono leggere ed indicate per il \hyperlink{combattimentoaduemani}{combattimento a due armi}.

%\medskip

%\begin{center}
%\includegraphics[width=0.8\linewidth]{immagini/twoweapon.png}
%\end{center}

\subsection{Combattimento con due armi}\index{Combattimento con due armi}\hypertarget{combattimentoaduemani}{}\label{combattimentoduemani}

Gli attacchi fatti con l'arma secondaria si considerano attacchi multipli.
Se attacco una prima volta, indipendente che sia con l'arma primaria o secondaria, questo avrà il Tiro per Colpire a bonus pieno, gli altri attacchi cumuleranno il -5 al colpire.

Il bonus al danno dato dalla Forza sull'arma secondaria viene dimezzato. Se l'arma secondaria non è \textbf{Leggera} il Tiro per Colpire ha un ulteriore -3 al colpire (es. 0,-8,-10,-18..).

E' possibile usare l'arma secondaria per migliorare la Difesa di un punto ma non si possono fare attacchi con quell'arma.

\subsection{Carica} \index{Carica}\label{carica}\hypertarget{carica}{}

l'avversario deve essere entro 2 Azioni di movimento (18 o 12 metri solitamente) ed a non meno di 3 metri, il terreno non deve essere difficile (vedi anche Abilità \hyperlink{Rinoceronte}{Rinoceronte}, pag. \pageref{Rinoceronte}). Si deve correre fino ad essere a distanza di mischia.

Si ottiene un +1d6 a Tiro per Colpire, -4 alla Difesa fino all'inizio del proprio round successivo, l'attacco successivo al primo prende un -10 al colpire ed un eventuale successivo -15, 20...

Il movimento ed attacco costa 2 Azioni. Non si considerano altre penalità per avere corso oltre quelli indicati.

L'Azione di Carica ti porta addosso, in mischia, con l'avversario. L'attacco se fatto con arma lunga viene portato a distanza di 2 metri per poi finire a contatto con l'avversario.

%\begin{center}
%\includegraphics[width=0.9\linewidth]{immagini/carica.png}

%\emph{A Connecticut Yankee in King Arthur's Court / Samuel Clemens. New York : Charles L. Webster \& Co., 1889}
%\end{center}

\subsubsection{Arma da Controcarica}\index{Arma da Controcarica}\label{controcarica}\index{Controcarica}\label{caricaarmadacontrocarica}

se effettui una Carica ed il Tiro per Colpire ha successo, la tua arma con tratto Controcarica infligge un Tiro Critico aggiuntivo.


\subsubsection{Preparare una arma lunga/da controcarica contro una carica} \index{Preparare una arma lunga contro una carica}\label{prepararearmalungacontrocarica}

Solo un arma con il tratto controcarica può essere usata contro una carica. Preparare l'arma contro una carica costa una Reazione.

Se chi carica ha un portata minore dell'avversario allora chi prepara la controcarica può effettuare un attacco con l'arma, come Azione Gratuita (oltre alla Reazione per preparare l'arma) con un Tiro per Colpire con -1d6 di penalità, prima dell'avversario. Se colpisce infligge un Tiro Critico aggiunto.


%\begin{center}
%	\includegraphics[width=0.9\linewidth]{immagini/pilum.png}
%
%	\emph{Soldati romani armati di Pilum, pronti per una controcarica.}
%\end{center}

\subsection{Attacchi con armi a spargimento} \index{Armi a spargimento}\index{Acqua santa}\index{Olio Incediato}\label{attacchiarmidaspargimento}\hypertarget{spargimento}{}

sono armi a spargimenti quelle che \emph{spargono} il loro contenuto dove cadono, ad esempio olio incendiato/Acqua santa... Una arma a spargimento ha una gittata di 6 metri\index{Lanciare Armi a Spargimento}\index{Gittata armi a spargimento}.

In caso l'attacco manchi (di almeno 5) tirare un d8 e consultare questo schema per capire dove la fiala è caduta e tira un 2d6 per determinare lungo la direzione indicata dal d8 precedente a quanti metri è caduto distante dal bersaglio, ovvero contate i metri dal target.

\medskip
\begin{center}

\begin{tabular}{ccc}
 8& 1& 2\\
 \rowcolor{gray!20}	7 &\textbf{X}& 3\\
 6 &5 &4\\
	&\textbf{0}&
\end{tabular}
\end{center}

\smallskip

\textbf{X} si considera il bersaglio dell'oggetto tirato. \textbf{0} il punto di origine del lancio.

Ad esempio con il tiro del d8 faccio 5 e poi tirando 2d6 faccio 4, significa che la boccetta è caduta a destra del bersaglio a 4 metri.

E' anche possibile che ci si sia tirati sui piedi la boccetta (es faccio 7 e poi 6.. potrei averla tirata addosso ad un compagno o dietro di me!).

\subsection{Impreparato -- Colti di Sorpresa}\index{Impreparato}\index{Sorpresa}\label{coltidisorpresa}\hypertarget{sorpresa}{}

se una creatura viene colta di sorpresa, ovvero non si aspetta di essere attaccata, si deve considerare questo primo round come round di sorpresa. Chi è sorpreso ha un -2 alla Difesa ed ai Tiri Salvezza su Riflessi.

Non si possono usare Azioni o Reazioni se non esplicitamente permesse; dal round successivo si potrà dichiarare l'iniziativa ed agire normalmente. Le medesime considerazioni valgono per gli avversari se sorpresi.

Confrontate la prova di Furtività di chi si muove furtivo contro 10+Consapevolezza di chi potrebbe essere sorpreso. Se la prova è superiore allora la creatura è effettivamente sorpresa. Se chi dovrebbe essere sorpreso è sull'attenti e vigile concedete +2 alla prova di Consapevolezza.

Quando entrambe le creature sono colte di sorpresa per valutare chi effettivamente è sorpreso effettuate un Tiro Salvezza su Riflessi, chi ottiene più di 15 non è sorpreso.

\subsection{Magia in combattimento}\index{Magia in combattimento}\label{magiaincombattimento}

l'incantatore che lancia una magia mentre è in combattimento (ha un avversario in mischia o viene bersagliato da distanza) si considera Distratto.

%\begin{center}
%\includegraphics[width=0.45\linewidth]{immagini/shield-milan.png}
%
%\emph{Sconfitta di Johann Friedrich di Sassonia all'imperatore Charles V}
%\end{center}

\subsection{Modificatori in attacco o difesa} \index{Modificatori in Attacco e Difesa}\label{modificatoriattaccodifesaparticolari}

Il migliore suggerimento che si può dare nel gestire le situazioni di combattimento più caotiche è pensare a queste come ad un film, valutate la cinematicità della situazione.

Non è una questione di miniature, spazi, quadretti.. è una questione di divertimento e visualizzazione della scena. Soluzioni non ortodosse per situazioni non ortodosse.

Concedete un bonus o penalità ($\pm 1-2$) se non indicato diversamente) ogni qual volta il giocatore abbia un vantaggio o svantaggio ed allo stesso modo all'avversario.


\subsubsection{Lettura dei modificatori in attacco e difesa} \index{Lettura dei modificatori in attacco e difesa}\label{letturamodificatoriattaccodifesaparticolari}

Quando si scrive -1d6 significa che si tira un dado in meno (o due se è -2d6), parimente se c'è scritto +1d6 si tira un dado a 6 in più e si somma.

Quando la penalità è alla Difesa considerare ogni -1d6 come un -4 alla Difesa.

\medskip

In linea di principio in combattimento un bonus leggero è un +1, medio +2, alto +1d6 (o +4), un bonus molto alto è +2d6 (o +8), viceversa per le penalità.


\medskip

\end{multicols}

\noindent\begin{tabularx}{\linewidth}{l|X|X}
	\toprule
 \rowcolor{gray!20}\multicolumn{2}{c}{\textbf{Attaccante}}&\multicolumn{1}{c}{\textbf{Difensore}}\\
\textbf{Mod}.&\multicolumn{1}{c}{\emph{Situazione}}&\multicolumn{1}{c}{\emph{Situazione}}\\
\toprule
\rowcolor{gray!20}\textbf{-1}& Affaticato (1), Luce fioca&Affaticato (1)\\

\textbf{-2}& Affaticato (2), Intralciato & Affaticato (2), Afferrato, Intralciato, Sorpreso\\

\rowcolor{gray!20}\textbf{-4}& Affaticato (4), Prono, Arma Lunga a corta distanza, attacco non letale con arma letale& Affaticato (4), Prono, In ginocchio, Seduto, Ristretto, Stordito, Afferrato ad una parete, Bloccato\\

\textbf{-1d6}& Ristretto, Spaventato, Arma da Lancio contro avversario in mischia, Arma non conosciuta, Bersaglio invisibile ma Individuato, Afferrato ad una parete, Bloccato&\\

%\textbf{+1}& & \\
%
\rowcolor{gray!20}\textbf{+2}& Fiancheggia, Posizione Sopraelevata, Attacca al spalle& Copertura leggera\\

\textbf{+4}&& Copertura media\\

\rowcolor{gray!20}\textbf{+1d6}& Invisibile, Carica, avversario Indifeso& \\

\textbf{+8}&& Copertura completa
\end{tabularx}

\medskip

\begin{multicols}{2}

\begin{narratore}[Non è tutto Regole]
Cercate di non interrompere il gioco cercando la regola precisa, lasciatelo fluire, dite ai giocatori che per brevità gestite la situazione in una certa maniera; ci sarà poi tempo per ricordare la situazione e trovare la regola giusta. Interrompere il gioco di continuo spezza il \emph{pathos} della situazione.
\end{narratore}

\begin{narratore}[Motivi e Scopi]
	Ricordate sempre che lo scopo è divertirsi, a scapito (per il Narratore) di qualche mostro, non siate rigidi ma dinamici e adattatevi alle situazioni.
\end{narratore}

I \textbf{modificatori positivi indicati} nella \emph{Tabella: Modificatori in attacco o difesa} si sommano a partire da quello maggiore e si aggiunge un +1 per ogni ulteriore bonus presente. Se un avversario è sopra il personaggio, alle spalle, invisibile ed in carica avrà un bonus al colpire di +1d6 (carica o invisibilità) +1 perché sopra, +1 perché alle spalle, +1 perché in carica.\index{Somma bonus al colpire}

Le \textbf{penalità} si sommano per intero tra loro. Se il personaggio è sorpreso e prono ha un -6 alla Difesa.\index{Bonus comulativi}\index{Somma penalità}


\subsection{Altre azioni e situazioni} \label{AltreAzioni}\index{Altre azioni e situazioni}

\subsubsection{Attacco a mani nude} \index{Pugno}\index{Calci} \index{Fare a botte}\label{attaccomaninude}

due armi che non mancheranno mai a nessuno sono i propri pugni e calci, con queste armi si è sempre addestrato e non si considerano attacchi improvvisati.

Se non si ha preso la lista d'armi \emph{Pugno Vuoto} un pugno o calcio farà 1d3 + Forza di danno non letale. Solo con la Lista d'Armi Pugno Vuoto si diventa artisti marziali.

\subsubsection{Aiutare un altro in combattimento}\index{Aiutare in combattimento}\label{aiutare}\hypertarget{aiutare}{}

si può aiutare un compagno ad attaccare o a difendersi negli scontri in mischia, distraendo o interferendo con l'avversario. Si può portare un attacco in mischia (1 Azione) contro un avversario che ha già ingaggiato battaglia con un proprio alleato.

Si effettua un Tiro per Colpire contro la Difesa dell'avversario con 1d6 di bonus. Se l'attacco va a segno non si fa danno ma il compagno ottiene bonus di +1 al Tiro per Colpire verso quell'avversario od un bonus di +1 alla Difesa entro la fine del tuo round successivo contro quell'avversario sul primo attacco. Se l'aiutante ottiene un Tiro Critico allora chi è aiutato avrà un bonus di +2.

più personaggi possono aiutare lo stesso alleato; i bonus di questo tipo sono cumulabili (massimo 4 su taglia media), purché l'avversario sia circondato.

\subsubsection{Alzarsi da prono}\index{Alzarsi da prono}\label{alzarsidaprono}\hypertarget{alzarsidaprono}{}

costa 1 Azione. Con una Azione Immediata il personaggio può eseguire una prova di Acrobatica ed alzarsi se fa 13. Se fa un Fallimento Critico nella prova non può fare altre azioni quel round e rimane prono.

Quando sei prono puoi strisciare\index{Strisciare}\index{Carponi} o muoverti a carponi. Il terreno si considera difficile e sei comunque considerato ancora prono finché non ti alzi.

\subsubsection{Colpo di Grazia} \index{Colpo di Grazia}\label{colpodigrazia}

costa 3 Azioni, si può utilizzare un'arma da mischia per infliggere un colpo di grazia ad un bersaglio inabile o indifeso (svenuto o intrappolato). Si può anche usare un arco o una balestra, l'importante è che si sia adiacente al bersaglio.

L'attaccante colpisce automaticamente ed infligge tre colpi critici.

\subsubsection{Tiri Mirati}\index{Tiri Mirati}\label{tirimirati}\index{Mirare a parti specifiche}

OBSS non prevede la possibilità di effettuare tiri mirati con armi o incantesimi, tranne se questo lo specifica.

Quando si colpisce il bersaglio lo si colpisce genericamente, senza possibilità di specificare se alla testa, gamba o altro, medesimo concetto vale in caso di colpi ad oggetti, es. se miri ad un cardine di una porta colpisci tutta la porta. Questo non impedisce al Narratore di valutare conseguenze adeguate all'azione intrapresa.

\medskip

\begin{center}
	\includegraphics[width=0.6\linewidth]{immagini/Judith_Beheading_Holofernes_MET_MM27116.png}

	\emph{Judith Beheading Holofernes}
\end{center}

\subsubsection{Danno non letale}\index{Danno non letale}\label{dannononletale}

il danno non letale è una forma di danno causato da armi particolari o quando lo scopo è fare svenire l'avversario e non ucciderlo.

Il danno non letale si tratta come il danno normale se non che si recupera più velocemente e scendere sotto zero Punti Ferita causa svenimento e non l'avvicinarsi della morte.

\index{Danno non letale con arma non idonea} \label{dannononletalearmanonidonea}

Se vuoi fare danno non letale con un'arma non predisposta al danno non letale hai un -4 al Tiro per Colpire.

\subsubsection{Senza Competenza}\index{Senza Competenza}\label{senzacompetenza}

usare un'arma senza l'adeguata competenza, ovvero non avere la Lista d'Armi d'appartenenza dell'arma, impone un -1d6 al Tiro per Colpire.

Non puoi usare la capacità Versatile di un arma se non la sai usare. Calci e Pugni oppure un Arma Semplice sono usabili senza penalità anche senza competenze specifiche.

\subsubsection{Lanciare armi} \index{Lanciare armi}\label{lanciarearmi}

una spada o comunque un arma non fatta per essere lanciata, senza Gittata, può comunque essere scagliata contro l'avversario.

Il Tiro per Colpire prende un -1d6 e l'arma fa una categoria di danno inferiore (la spada lunga fa 1d6, una spada corta 1d4..). La gittata di lancio è 3 metri.

\subsubsection{Colpi Potenti}\index{Colpi Potenti}\label{colpipotenti}

Il personaggio al momento dell'attacco può dichiarare di aggiungere un +1 al danno togliendo 2 al Tiro per Colpire con l'arma da mischia (requisito Competenza Armi +1). Non si può togliere più di Competenza Armi/4 al Tiro per Colpire. Da dichiarare prima del Tiro per Colpire.

\subsubsection{Fiancheggiare, attaccare alle spalle}\hypertarget{fiancheggiare}{} \index{Fiancheggiare}\label{fiancheggiare}\index{Attaccare alle spalle}

se due personaggi sono attorno allo stesso bersaglio ma non sono a fianco tra loro prendono +2 al Tiro per Colpire o alla Difesa (a loro scelta quale bonus prendere).

Al massimo ci possono essere 4 personaggi attorno ad una creatura di taglia media che prendono il bonus di fiancheggiare. Il tipo di bonus si sceglie round per round, se non dichiarato vale come +2 al Tiro per Colpire.

Se tirando una ipotetica riga che collega i due personaggi questa attraversa in pieno il quadretto dell'avversario allora c'è la situazione di fiancheggiamento.

Un creatura può attaccare alle spalle se l'avversario non è in grado di fronteggiarlo. Attaccare alle spalle concede un +2 al Tiro per Colpire. Non si cumula con il Fiancheggiare.

\medskip

\textbf{Esempio di fiancheggiamento}\index{Esempi di Fiancheggiamento}

\medskip
\begin{center}

\begin{tabular}{lll}
\hline
\rowcolor{gray!20}A & G & D\\
B & \textbf{X} & E\\
\rowcolor{gray!20}C & H & F
\end{tabular}
\end{center}

\bigskip

In questo schema il fiancheggiamento è preso dalle coppie: A-F, B-E, C-D, G-H

\bigskip

Se la creatura può fronteggiare più creature contemporaneamente queste non godranno del bonus di fiancheggiamento.

\subsubsection{Maestria del combattimento} \index{Maestria del combattimento}\label{maestriacombattimento}

Il personaggio al momento dell'attacco può dichiarare di aggiungere un +1 alla Difesa togliendo 2 al Tiro per Colpire. Viceversa può prendere un -2 alla Difesa per alzare di +1 il Tiro per Colpire e quindi migliorare l'attacco.

Questi modificatori permangono fino all'inizio del proprio round successivo.

Non si può togliere/aggiungere più di Competenza Armi/4 al Tiro per Colpire/Difesa, Maestria del combattimento non consuma Azioni.

\subsubsection{Preparare la Difesa}\index{Preparare la Difesa}\label{preparareladifesa}\hypertarget{preparareladifesa}{}\index{Parata}

Il personaggio può usare una Azione per prepararsi meglio ai successivi attacchi degli avversari. Fino all'inizio del tuo prossimo round hai un +1 alla Difesa.

Se almeno un arma che usi ha il tratto \textbf{Parata} prendi un ulteriore +1 alla Difesa.

\subsubsection{Difesa totale} \index{Difesa totale}\label{difesatotale}\hypertarget{difesatotale}{}

Il personaggio usa 2 Azioni, prende un bonus di +4 alla Difesa e tratta come Difficile il terreno fino all'inizio del suo prossimo round.

\subsubsection{Disingaggiare} \index{Disingaggiare}\label{disingaggiare}\hypertarget{disingaggiare}{}

Il personaggio usando 1 Azione si sposta di 1 metro e non causa attacchi di opportunità.\index{Passo difensivo}

%\subsubsection{Capriola} \index{Capriola}\label{capriola}\hypertarget{capriola}{}

%Il personaggio può tentare di muoversi tra gli avversari non causando attacchi di opportunità se riesce in una prova di Acrobatica per avversario che vuole \emph{evitare}. La DC è pari a 10+GS avversario +2 per avversario già evitato. Il personaggio non consuma Azioni per la prova di Acrobatica ma per le Azioni di Movimento fatte. Se il personaggio fallisce la prova di Acrobatica interrompe la sua Azione di Movimento vicino all'avversario per il quale ha fallito la prova. Il terreno si considera difficile.

\subsubsection{Colpo Preciso} \index{Colpo Preciso}\label{colpopreciso}\hypertarget{colpopreciso}{}

Il personaggio usando 2 Azioni effettua un solo attacco in mischia. Su questo singolo attacco ottiene un bonus di +1d4 al Tiro per Colpire.

\subsubsection{Prendere la Mira (cecchino)} \index{Prendere la Mira (cecchino)}\label{cecchino}\hypertarget{cecchino}{}

Il personaggio che attacca con un arma da lancio può usare 2 Azioni a round, fino ad un massimo di 3 round, per prendere la mira contro un obiettivo. Ha un bonus al Tiro per Colpire pari a +1 il primo round, +2 il secondo round ed infine nel terzo round di Prendere la Mira il bonus arriva a +4.

Non può usare Azioni di Movimento mentre prende la mira.

\subsubsection{Arma da lancio contro obiettivi in combattimento} \index{Arma da lancio contro avversario impegnato in combattimento}\label{usarearmalancioinmischia}

In combattimento non è facile mirare un obiettivo che è in combattimento con un altra creatura.

Oltre le eventuali penalità date della \hyperlink{esempicopertura}{Copertura} (pag. \pageref{esempicopertura}) si ha una penalità aggiuntiva di -2 al Tiro per Colpire.

Es. Se voglio colpire un avversario coperto ed in combattimento con un mio compagno oltre alla penalità di -2 al Tiro per Colpire perché è in combattimento con qualcuno, questo avversario ha \emph{copertura leggera} (+2 Difesa), se ho due creature in fila e poi l'avversario da colpire questo avrà da copertura +4 ed un ulteriore +2 alla Difesa perché combatte con qualcuno.
Se sono 3 creature allora la \emph{copertura sarà completa} (+8 Difesa, +2 perché combatte con un altro).

Se si mira ad un obiettivo che è in combattimento ed entrambi sono frontali al personaggio si ha solo il -2 al Tiro per Colpire senza la penalità della copertura.

In caso di Fallimento Critico nel Tiro per Colpire si colpisce casualmente una creatura che dava copertura o chi era a fianco dell'obbiettivo.

Vedi anche Abilità \hyperlink{Precisino}{Precisino} (pag. \pageref{Precisino}).

\subsubsection{Usare un'arma da lancio sotto minaccia} \index{Usare un'arma da lancio sotto minaccia}\label{usarearmalanciosottominaccia}

usare un'arma da lancio come arco, balestra o pugnale (che si vuole lanciare) mentre si è minacciati in mischia impone una penalità di -1d6 al Tiro per Colpire.

Vedi anche Abilità \hyperlink{Uno con l'arco}{Uno con l'arco} (pag. \pageref{Uno con l'arco}.)

\medskip

\begin{center}
	\includegraphics[width=0.55\linewidth]{immagini/angelospadone.png}
\end{center}

\subsubsection{Arma troppo grande}\index{Arma troppo grande} \label{armatroppogrande}

La taglia indicata nelle tabella delle armi (vedi \hyperlink{dimensionediunarma}{Dimensioni delle Armi} ) è riferita ad una creature di taglia media. Per una creatura di taglia piccola la dimensione si deve intendere di una categoria superiore, es. una spada corta che è di dimensioni piccole per una creatura di taglia media, se usata da una creatura di taglia piccola si considera un'arma di dimensioni medie.

Allo stesso modo una arma grande, come uno spadone a due mani, nelle mani di un gigante diventa un'arma di dimensioni medie.

Questo non ne cambia il danno o il tipo di danno causato dall'arma.

Una creatura può usare un arma con dimensione della propria taglia o di un solo grado inferiore con una mano e deve usare due mani per brandire un arma di una taglia superiore alla propria.

Se l'arma è di taglia superiore a quella usabile con 2 mani, esempio un Alabarda (arma grande) per una creatura di taglia piccola la penalità al Tiro per Colpire è -1d6. Lo stesso principio è valido per una spadone a due mani di taglia grande (2d8 di danno) nelle mani di una creatura di taglia media.

Nella tabella delle armi la dimensione è segnata come P (piccola), M (media), G (grande), E (enorme) ed è riferita ad una creatura di taglia media. Una versione \emph{più grande} di un arma aumenta di una categoria il danno dell'arma (1d4->1d6, 1d6->1d8, 1d8->1d10, 1d10/1d12->2d6, 2d6->2d8, 2d8->2d10, 2d10->3d6...).

Es. uno spada lunga grande (+1 taglia) passa da 1d8 a 1d10 di danno.


\subsubsection{Usare un'arma con due mani} \index{Usare un'arma con due mani}\label{usarearmaconduemani}

un arma ad una mano che può (ma non deve) essere usata a due mani aumenta il dado di danno quando usata a due mani.

Es. Spada Lunga per una creatura media può causare 1d8 ad una mano o 1d10 a due mani. Una Spada Corta non può essere impugnata a due mani da una creatura media ma può essere tenuta a due mani da una creatura piccola.

Se l'arma deve essere tenuta a due mani perché troppo grande per la propria taglia questo modificatore non si considera (es. uno spadone a due mani per una creatura di taglia media, o una spada lunga per una creatura piccola).

Il valore di EDX se diverso dal massimo danno dell'arma aumenta di 2 (la Katana causerà 2d6 di danno ed avrà ED11) quando usata a due mani.

\subsubsection{Combattere al buio}\index{Combattere al Buio}

Combattere in condizioni di luminosità ridotta comporta delle difficoltà riassunte in questo schema.

\medskip

\noindent\begin{tabular}{lll}
	\toprule
\rowcolor{gray!20}\textbf{Visione} & \multicolumn{2}{c}{\textbf{Condizione}}\\
& Luce Fioca & Oscurità\\
\hline
Normale & -1 TC, -2 Cons. & Invis. (pag. \pageref{invisibilita})\\
\rowcolor{gray!20}Crepuscolare & Normale & Invis. (pag. \pageref{invisibilita})\\
Scurovisione & Normale & -2 Cons.
\end{tabular}

\medskip

Vedi anche Capitolo \hyperlink{visioneeluce}{Visione e Luce} (pag. \pageref{visioneeluce}).

%\subsection{Opzionale - L'Unica Regola}\index{Opzionale - L'Unica Regola}\hypertarget{lunicaregola}{}\label{lunicaregola}
%Questa opzione vuole semplificare la gestione di qualsiasi prova contrapposta, possa essere relativa alle Competenze di Base o Attiva.
%Quando una creatura o personaggio ha un vantaggio o svantaggio tira in aggiunta 1d6 alla prova. Se ha due vantaggi tira 2d6, se ne ha tre di vantaggi tira 3d6...
%Alla prova aggiunge o sottrae, in caso di bonus o penalità, il valore più alto tra i dadi tirati. Per questi dadi non vale l'esplosione del risultato.
%Uno Svantaggio annulla un Vantaggio se presente.
%Se il vantaggio/svantaggio è relativo ad un valore statico (come la Difesa) allora questo aumenta di 2 per ogni vantaggio/svantaggio accumulato.

\subsection{Manovre Opzionali in Combattimento}\label{azioniopzionaliincombattimento}

Queste Azioni di combattimento sono a discrezione del Narratore che può concederle o meno. \textbf{Ogni manovra conta come Azione di Attacco} per quanto riguardo le penalità del multiattacco.\index{Manovre ed Azioni di Attacco}

Quando queste manovre sono fatte dagli avversari e non sono segnati i valori di Tiro per Colpire, Atletica, Ingannare... contrapporre alla prova il Tiro Salvezza indicato dopo il costo in Azioni ed i modificatori suggeriti (Taglia...).

\medskip

\subsubsection{Disarmare*}\index{Disarmare}\label{disarmare}\hypertarget{disarmare}{}

fai una Prova Contrapposta di Competenza Armi + Destrezza o Forza (3d6+CA+For o Des).

Se chi tenta la manovra non riesce e ottiene un Fallimento Critico è lui che perde l'arma. Costa 2 Azioni (Riflessi).

\subsubsection{Finta*} \index{Finta}\label{finta}\hypertarget{finta}{}

fai una Prova Contrapposta di Competenza Armi + Ingannare (chi fa la finta) contro Competenza Armi + Percepire Emozioni (chi subisce la finta). Se la prova riesce l'avversario ha un -2 alla Difesa contro di te fino alla fine del tuo round.

Se chi tenta la manovra non riesce e ottiene un Fallimento Critico è lui che prende -2 alla Difesa fino alla fine del suo round successivo. Costa 1 Azione (Volontà).

%\begin{center}
%	\includegraphics[width=0.95\linewidth]{immagini/alfieri37.png}
%\end{center}

\subsubsection{Spingere/Tirare un avversario*} \index{Spingere un avversario}\label{spingereavversario}\hypertarget{spingereavversario}{}\index{Tirare un avversario}

è una prova di Atletica contrapposta ad un Tiro Salvezza su Tempra con Forza. Chi ha una taglia maggiore guadagna un bonus di +1d6 per taglia di differenza.

Chi vince può spingere o tirare l'avversario fino a 0.5 metri nella direzione che vuole per successo nella prova (fino al massimo del suo movimento). Es. se vinci la prova di 7 sposti l'avversario fino a 3 metri. Chi spinge si può muovere assieme a chi è spinto senza usare altre Azioni.

Se si vuole \emph{lanciare} l'avversario il movimento è di 0.3 metri per successo ottenuto.\index{Lanciare una creatura}

Se chi tenta la manovra non riesce e ottiene un fallimento critico, l'avversario usando una Reazione, può spostarlo secondo le regole di sopra. Costa 2 Azioni (Tempra).

\subsubsection{Afferrare un avversario*}\index{Afferrare un avversario}\label{afferrareunavversario}\hypertarget{afferrareunavversario}{}

è una prova di Atletica contrapposta ad un Tiro Salvezza su Tempra con Forza. Chi ha una taglia maggiore guadagna un bonus di +1d6 per taglia di differenza. Se chi riesce nella manovra ottiene un Successo Critico si considera che abbia \hyperlink{bloccato}{Bloccato} l'avversario.

Costa 2 Azioni (Tempra) fare e mantenere e liberarsi dalla presa. Si considera che chi afferra sia anche Afferrato ed abbia almeno una mano occupata nell'afferrare.
Muovere una creatura afferrata richiede \hyperlink{spingereavversario}{Spingere un avversario}.

Ogni contendente può attaccare l'altro afferrato con un arma piccola o armi naturali, la Difesa ha una penalità di -2 e si è considerati Distratti. Attaccare una creatura diversa da chi si afferra ha un -1d6 al Tiro per Colpire.

\subsubsection{Attraversare i nemici*}\index{Attraverso i nemici}\index{Attraversare quadretti occupati}\label{attraversonemici}\index{Destreggiarsi}\hypertarget{attraversarenemici}{}\hypertarget{destreggiarsi}{}

\index{Attraversare nemici}\index{Movimento attraverso}Un personaggio può \textbf{attraversare} ma non sostare in \textbf{una zona occupata} da una creatura senza essere \hyperlink{ristretto}{\textbf{ristretti}}\index{Ristretto}.

Per attraversare il terreno dove c'è una creatura ostile è necessario eseguire una Prova Contrapposta su Atletica o Acrobatica contro Tiro Salvezza su Riflessi della creatura al quale si vuole \textbf{attraversare} il terreno, per ogni creatura attraversata oltre la prima la difficoltà aumenta di +2.

Costa 1 Azione (Riflessi) la prova per attraversare oltre all'Azione di Movimento. Il terreno occupato dalla creatura ostile si considera difficile. Il terreno non si considera difficile solo nel caso in cui la creatura sia di due o più taglie inferiori. In caso di Successo Critico nella prova di Atletica o Acrobatica non si consuma l'Azione usata per attraversare.

Se si fallisce si rimane ne quadretto immediatamente precedente al nemico, con il rischio di essere \hyperlink{ristretti}{ristretti} (pag. \pageref{ristretti}). Si considera terminata sia l'Azione di Movimento che quella di oltrepassare.
Se il nemico ha l'Abilità \hyperlink{opportunista}{Opportunista} oltre ad ostacolare il passaggio può eseguire un attacco (usando una Reazione).

%\begin{center}
%\includegraphics[width=0.7\linewidth]{immagini/vantaggio.png}
%
%\emph{Henry Justice Ford}
%\end{center}

\subsubsection{Fare cadere un avversario*} \index{Fare cadere un avversari}\label{farecadereavversario}\hypertarget{farecadereavversario}{}

è una prova di Atletica contrapposta ad un Tiro Salvezza su Tempra con Forza per mettere Prono l'avversario.

Per ogni gamba/zampa in più il contendente ottiene un bonus di +1 alla prova e ottiene un +1d6 per differenza di Taglia.

Se chi tenta la manovra non riesce e ottiene un Fallimento Critico è lui che cade. Costa 2 Azioni (Tempra).

%\subsubsection{Modificare le proprie dimensioni*}\index{Modificare le proprie dimensioni}\label{modificatedimensioni}

%nel caso il personaggio cambi dimensione \index{Modificare le dimensioni} la sua Difesa cambia di conseguenza.

%\medskip
%\noindent\begin{tabular}{ll|ll}
%\textbf{Taglia} & \textbf{Difesa}&\textbf{Taglia} & \textbf{Difesa}\\
%\toprule
%Piccolissima& +8 & Grande & -1\\
%Minuta & +4 & Enorme & -2\\
%Minuscola & +2 & Mastodontica & -4\\
%Piccola & +1 & Colossale & -8\\
%Media & +0 &&
%\end{tabular}

\subsection{Opzionale - Azioni Tiro Critico}\index{Opzionale - Azioni Tiro Critico}\label{OpzionaleAzioniTiroCritico}\hypertarget{OpzionaleAzioniTiroCritico}{}

Questa Opzione permette un combattimento meno incentrato sul danno ma più sulle manovre e la tattica di utilizzo dei colpi.

Il giocatore tiene conto dei Tiri Critici che tira e che non applica al danno. Entro tre round dal loro tiro vanno usati.

Ogni round può \emph{usare} uno o più Tiri Critici accumulati per eseguire Azioni Critiche. L'uso delle Azioni Critiche deve essere contro l'avversario al quale si sono effettuati i Tiri Critici.

L'elenco propone l'elenco delle Azioni Critiche di Tiri Critici consumati. Non si possono avere più di 6 Tiri Critici accumulati per singolo round.
L'attivazione di queste Azioni Critiche costa una Reazione se non indicato diversamente.

Usate questi esempi come delle linee guida per stimolare il personaggio a creare un suo stile di combattimento. E' importante che il personaggio descriva come attiva l'Azione Critica.

\medskip

\noindent\textbf{Critici} \hskip 0.5cm \textbf{Effetto}

\medskip

\begin{itemize}[leftmargin=*]
	\setlength{\itemsep}{0pt}

	% ATTACCHI AGLI OCCHI
	\item \textbf{Attacchi agli occhi}
	\begin{itemize}[leftmargin=*]
		\setlength{\itemsep}{0pt}
		\item \textbf{1}: Sabbia negli occhi. Entro la fine del tuo prossimo round l'avversario ha -2 al primo Tiro per Colpire

		\item \textbf{2}: Graffio agli occhi. Entro la fine del tuo prossimo round l'avversario ha -4 al Tiro per Colpire

		\item \textbf{3}: Bersaglio abbagliato. Tira 1d6, con 1-2-3 l'avversario ha mancato il suo attacco. Dura fino alla fine del prossimo round.

		\item \textbf{4}: Bersaglio accecato. Per 1d6 round, l'avversario considera tutti come invisibili.

		\item \textbf{5}: Orbo. L'avversario esegue un Tiro Salvezza su Tempra con DC pari al tuo ultimo Tiro per Colpire, se fallisce è accecato permanentemente, altrimenti subisce gli effetti del punto 4.
	\end{itemize}

	% ATTACCHI ALL'ARMA
	\item \textbf{Attacchi all'arma}
	\begin{itemize}[leftmargin=*]
		\setlength{\itemsep}{0pt}
		\item \textbf{1}: Colpisci l'arma. L'avversario esegue un Tiro Salvezza su Tempra DC 15, $ \pm 2 $ per differenza di taglia dell'arma, oppure lascia cadere l'arma

		\item \textbf{2}: Arma danneggiata. L'arma dell'avversario infligge una categoria di danno in meno

		\item \textbf{3}: Colpo alla mano. A causa del dolore entro la fine del tuo prossimo round l'avversario perde i primi due attacchi

		\item \textbf{4}: Disarmi l'avversario. L'avversario lascia cadere l'arma

		\item \textbf{5}: Mano compromessa. L'avversario fino all'alba del giorno successivo ha -4 al Tiro per Colpire
	\end{itemize}

	% SPINTONI E SLANCI
	\item \textbf{Spintoni e Slanci}
	\begin{itemize}[leftmargin=*]
		\setlength{\itemsep}{0pt}
		\item \textbf{1}: Allontani l'avversario di 3 metri, $ \pm 1$ per taglia di differenza

		\item \textbf{1}: Ti puoi spostare di una Azione di Movimento come Reazione. Tratti il terreno come difficile

		\item \textbf{2}: Come punto 1 ma la distanza iniziale è di 6 metri

		\item \textbf{2}: Ti puoi spostare di una Azione di Movimento come Reazione

		\item \textbf{3}: Come punto 1 ma la distanza iniziale è di 9 metri

		\item \textbf{4}: Come punto 3 e puoi spostarti con l'avversario
	\end{itemize}

	% SGAMBETTI
	\item \textbf{Sgambetti}
	\begin{itemize}[leftmargin=*]
		\setlength{\itemsep}{0pt}
		\item \textbf{1}: Spallata. L'avversario esegue un Tiro Salvezza su Tempra DC 15, $ \pm 4 $ per taglia differenza o cadere prono

		\item \textbf{2}: Sgambetto. L'avversario esegue un Tiro Salvezza su Riflessi DC 19, $ \pm 2 $ per taglia differenza o cadere prono

		\item \textbf{3}: Spinta. L'avversario esegue un Tiro Salvezza su Tempra DC 23, $ \pm 2 $ per taglia differenza o cadere prono. Se il Tiro Salvezza riesce viene allontanato di 1d6 metri

		\item \textbf{4}: Urto. L'avversario esegue un Tiro Salvezza su Tempra con DC pari al tuo ultimo Tiro per Colpire, $ \pm 2 $ per taglia differenza o cadere prono. Se il Tiro Salvezza riesce viene allontanato di 1d10 metri
	\end{itemize}

	% PROIETTILI
	\item \textbf{Proiettili}
	\begin{itemize}[leftmargin=*]
		\setlength{\itemsep}{0pt}
		\item \textbf{Vari}: Per ogni Tiro Critico usato aggiungi una ulteriore gittata alla tua arma

		\item \textbf{Vari}: Per ogni due Tiro Critico usato aggiungi +4 al prossimo Tiro per Colpire entro la fine del round successivo

		\item \textbf{5}: \emph{Freccia Kennedy}. Entro la fine del tuo prossimo round il primo proiettile ignora qualsiasi copertura o ostacolo e se fisicamente possibile colpisce l'avversario
	\end{itemize}

	% FURIA
	\item \textbf{Furia}
	\begin{itemize}[leftmargin=*]
		\setlength{\itemsep}{0pt}
		\item \textbf{1}: Incitare i compagni. I tuoi compagni entro 6 metri hanno al loro primo attacco +2 al Tiro per Colpire

		\item \textbf{2}: Berserker. Un tuo compagno entro sei metri può eseguire una Azione di Attacco contro un avversario in mischia al costo di una Reazione

		\item \textbf{3}: Scossa. Un tuo compagno, entro 6 metri, può usare una Reazione per eseguire una Azione di Movimento trattando il terreno come difficile

		\item \textbf{4}: Gloria. I tuoi compagni entro 6 metri hanno +2 al Tiro per Colpire entro la fine del loro prossimo round

		\item \textbf{5}: Gloria!. I tuoi compagni entro 9 metri hanno +4 al Tiro per Colpire entro la fine del loro prossimo round
	\end{itemize}

	% DIFESA!
	\item \textbf{Difesa!}
	\begin{itemize}[leftmargin=*]
		\setlength{\itemsep}{0pt}
		\item \textbf{1}: Fino alla fine del tuo prossimo round hai +2 alla Difesa.

		\item \textbf{2}: Fino alla fine del tuo prossimo round tutti i compagni nel tuo raggio di mischia hanno +2 alla Difesa

		\item \textbf{3}: Fino alla fine del tuo prossimo round un compagno ha +8 alla Difesa.

		\item \textbf{4}: Fino alla fine del tuo prossimo round tutti i compagni nel raggio di 9 metri hanno +4 alla Difesa

		\item \textbf{5}: Per 1d6 round tutti i tuoi compagni hanno +4 alla Difesa
	\end{itemize}

\end{itemize}

\medskip

\begin{narratore}[Azioni Critiche]
Queste Azioni Critiche possono essere descritte come approfittare della distrazione dell'avversario, gettare terra negli occhi, costringere a colpi di arma a spostarsi...
\end{narratore}

\begin{enfasi}{
Onestà e Giustizia, Eroico Coraggio, Compassione, Gentile Cortesia, Completa Sincerità, Onore, Dovere e Lealtà (I sette princìpi del bushido)
}\end{enfasi}


\subsection{Opzionale - Elenco Manovre d'Arme}\hypertarget{elencotalentiarmi}{}\label{elencotalentiarmi}\index{Opzionale - Elenco Manovre d'Arme}

più il personaggio diventa competente con le armi maggiormente è in grado di sfruttare le occasioni d'attacco ed effettuare manovre d'armi. Ogni qual volta il personaggio esegua almeno due attacchi con armi nel round e \textbf{nessuno dei due vada a segno} è possibile consultare la lista Manovre d'Arme per comprendere quale manovra è possibile usare usando una Reazione.

Ogni Manovra ha indicata quale è la situazione che lo attiva (Attiv.) e quale è l'Effetto.

Può essere indicato anche un Effetto Critico, ovvero l'Effetto che si ha quando si ottiene un Fallimento Critico in almeno un Tiro per Colpire. Purché l'Attivatore sia sempre rispettato, il giocatore può scegliere tra l'Effetto e l'Effetto Critico.

L'Attivatore può specificare un valore pari o dispari che è da confrontare con il Tiro per Colpire.

Le Manovre d'Arme sono raggruppate per livello, ovvero il punteggio di Competenza Armi minima per poter usare quelle manovre, il giocatore può scegliere tra tutte le Manovre d'Arme a lui accessibili ed Attivabili.

\begin{giocatore}[Ho mancato!]
		Sfruttalo come e meglio che potete la vostra sfortuna! Fate che la sorte vi possa sorridere grazie alla vostra abilità nel scegliere la manovra da attivare. Valutate sempre l'ambiente, i nemici, i compagni e la situazione in genere prima di decidere cosa attivare.
\end{giocatore}

\begin{itemize}[leftmargin=*]
	\setlength{\itemsep}{0pt}

	% MANOVRE LIVELLO 4
	\item \textbf{Manovre livello 4}
	\begin{itemize}[leftmargin=*]
		\setlength{\itemsep}{0pt}
		\item \textbf{Schivata Acrobatica} - Attiv.: \textbf{\emph{Mancato}} con un \textbf{\emph{dispari}}. \emph{Effetto}: +1 ai Tiri Salvezza sui Riflessi fino al prossimo turno. \emph{Critico}: +2 ai Tiri Salvezza sui Riflessi.

		\item \textbf{Finta e Riposizionamento} - Attiv.: \textbf{\emph{Mancato}} con un \textbf{\emph{pari}}. \emph{Effetto}: Muovi 1 metro senza provocare attacchi di opportunità. \emph{Critico}: Come sopra ma di muovi 2 metri.

		\item \textbf{Colpo Distraente} - Attiv.: \textbf{\emph{Mancato}}. \emph{Effetto}: +1 alla Difesa contro l'avversario fino al fine del prossimo round. \emph{Critico}: L'avversario considera il personaggio come se avesse copertura leggera.

		\item \textbf{Intralcio Rapido} - Attiv.: \textbf{\emph{Mancato}} con un \textbf{\emph{dispari}}. \emph{Effetto}: L'avversario riduce di 1 metro il suo movimento nel prossimo round. \emph{Critico}: il prossimo movimento entro il prossimo round dell'avversario lo considera come terreno difficile.

		\item \textbf{Apertura Tattica} - Attiv.: \textbf{\emph{Mancato}} con un \textbf{\emph{pari}}. \emph{Effetto}: Un alleato ottiene +1 al prossimo Tiro per Colpire contro l'avversario. \emph{Critico}: L'alleato ottiene +2 al Tiro per Colpire.
	\end{itemize}

	% MANOVRE LIVELLO 6
	\item \textbf{Manovre livello 6}
	\begin{itemize}[leftmargin=*]
		\setlength{\itemsep}{0pt}
		\item \textbf{Reazione Rapida} - Attiv.: \textbf{\emph{Mancato}}. \emph{Effetto}: il personaggio può bere una pozione. \emph{Critico}: Il personaggio può somministrare una pozione ad un alleato adiacente.

		\item \textbf{Finta Avanzata} - Attiv.: \textbf{\emph{Pari}}. \emph{Effetto}: Aggiungi il modificatore di Intelligenza o Saggezza alla Difesa contro l'avversario fino all'inizio del prossimo round. \emph{Critico}: Aggiungi il modificatore di Intelligenza e Saggezza al prossimo Tiro per Colpire contro l'avversario fino all'inizio del prossimo round.

		\item \textbf{Supporto Coordinato} - Attiv.: \textbf{\emph{Pari}}. \emph{Effetto}: Un alleato adiacente ottiene un bonus al Tiro per Colpire pari al modificatore di Intelligenza del personaggio. \emph{Critico}: Il bonus si applica a due alleati a te adiacenti.

		\item \textbf{Contrattacco Immediato} - Attiv.: \textbf{\emph{Dispari}}. \emph{Effetto}: Se il Tiro per Colpire del fallimento avrebbe colpito un avversario adiacente, infliggi danni pari al modificatore di Forza. \emph{Critico}: Infliggi danni pari al doppio del modificatore di Forza.

		\item \textbf{Distrazione Sonica} - Attiv.: \textbf{\emph{Dispari}}. \emph{Effetto}: Un alleato afferrato o bloccato può effettuare una prova per liberarsi. \emph{Critico}: Come sopra e può muoversi di 1 metro.
	\end{itemize}

	% MANOVRE LIVELLO 8
	\item \textbf{Manovre livello 8}
	\begin{itemize}[leftmargin=*]
		\setlength{\itemsep}{0pt}
		\item \textbf{Concentrazione guerriera} - Attiv.: \textbf{\emph{Mancato}}. \emph{Effetto}: +2 ai Tiri per Colpire in mischia fino alla fine del prossimo round. \emph{Critico}: Fino alla fine del prossimo round se colpisci causi un Danno Critico in più.

		\item \textbf{Elusione Perfetta} - Attiv.: \textbf{\emph{Dispari}}. \emph{Effetto}: +2 alla Difesa fino al prossimo turno. \emph{Critico}: Il prossimo attacco in mischia da quell'avversario manca automaticamente.

		\item \textbf{Analisi dell'Avversario} - Attiv.: \textbf{\emph{Dispari}}. \emph{Effetto}: Puoi effettuare una Prova di Conoscenze per ottenere informazioni sull'avversario. \emph{Critico}: Un alleato può effettuare la stessa prova.

		\item \textbf{Terreno Instabile} - Attiv.: \textbf{\emph{Pari}}. \emph{Effetto}: Fino alla fine del prossimo round consideri il terreno come difficile ed hai +4 al Tiro per Colpire. \emph{Critico}: Non puoi muoverti entro la fine del prossimo round ma se colpisci causi due danni critici in più.

		\item \textbf{Attacco a Sorpresa} - Attiv.: \textbf{\emph{Pari}}. \emph{Effetto}: Usando una Reazione puoi effettuare un Attacco con il medesimo Tiro per Colpire dell'ultimo attacco. \emph{Critico}: Se sufficiente il tuo Tiro per Colpire colpisce un altro avversario in mischia.
	\end{itemize}

	% MANOVRE LIVELLO 10
	\item \textbf{Manovre livello 10}
	\begin{itemize}[leftmargin=*]
		\setlength{\itemsep}{0pt}
		\item \textbf{Manovra Laterale} - Attiv.: \textbf{\emph{Mancato}}. \emph{Effetto}: Ti muovi di 1 metro. \emph{Critico}: Ti muovi fino a 4 metri, ma il prossimo round esegui una Azione in meno.

		\item \textbf{Apertura Devastante} - Attiv.: \textbf{\emph{Dispari}}. \emph{Effetto}: Un alleato ottiene +4 al Tiro per Colpire contro l'avversario fino alla fine del tuo prossimo round. \emph{Critico}: Due alleati ottengono il bonus, ma il personaggio subisce -4 al Tiro per Colpire fino alla fine del prossimo round.

		\item \textbf{Intimidazione Superiore} - Attiv.: \textbf{\emph{Dispari}}. \emph{Effetto}: L'avversario subisce -4 al primo attacco contro di te entro la fine del prossimo round. \emph{Critico}: Entro la fine del prossimo round l'avversario in mischia non può fare danno critico contro di te.

		\item \textbf{Ferita Sanguinante} - Attiv.: \textbf{\emph{Pari}}. \emph{Effetto}: L'avversario subisce +1 al sanguinamento. \emph{Critico}: +2 al sanguinamento e il personaggio subisce danni pari al modificatore di Forza.

		\item \textbf{Valutazione Strategica} - Attiv.: \textbf{\emph{Pari}}. \emph{Effetto}: Il prossimo attacco a segno entro fine del round prossimo infligge un danno critico in più. \emph{Critico}: Come sopra e due danni critici ma il round successivo esegui una Azione in meno.
	\end{itemize}

	% MANOVRE LIVELLO 12
	\item \textbf{Manovre livello 12}
	\begin{itemize}[leftmargin=*]
		\setlength{\itemsep}{0pt}
		\item \textbf{Assalto Incessante} - Attiv.: \textbf{\emph{Mancato}}. \emph{Effetto}: Il prossimo round hai un +1 cumulativo al Tiro per Colpire per volta che attacchi. \emph{Critico}: Il prossimo round hai solo 1 Azione. Se la usi per attaccare e colpisci causi 2 danni critici in più.

		\item \textbf{Attacco Predittivo} - Attiv.: \textbf{\emph{Dispari}}. \emph{Effetto}: Il prossimo round il primo attacco portato manca, se colpisci con un attacco successivo infliggi 2 danni critici aggiuntivi. \emph{Critico}: Come sopra ma 3 danni critici ed esegui 1 Azione in meno nel round.

		\item \textbf{Grido di Guerra Potente} - Attiv.: \textbf{\emph{Dispari}}. \emph{Effetto}: Gli alleati entro 9 metri ottengono +1d6 al Tiro per Colpire entro la fine del tuo prossimo round. \emph{Critico}: Come sopra ma +2d6, e il personaggio esegue una sola Azione.

		\item \textbf{Intuizione Letale} - Attiv.: \textbf{\emph{Pari}}. \emph{Effetto}: Il prossimo attacco andato a segno entro la fine del round successivo infligge danni critici massimizzati. \emph{Critico}: Come sopra ma computi 2 danni critici massimizzati, ma esegui una Azione in meno.

		\item \textbf{Furia Incontenibile} - Attiv.: \textbf{\emph{Pari}}. \emph{Effetto}: Confronta il Tiro per Colpire con un avversario adiacente per capire se l'hai colpito, se si sommi anche un danno critico. \emph{Critico}: Fino alla fine del prossimo round hai -4 alla Difesa, +1d6 al Tiro per Colpire e ogni attacco andato a segno causa un danno critico aggiuntivo.
	\end{itemize}

\end{itemize}

%\begin{center}
%	\includegraphics[width=0.2\linewidth]{immagini/Bushido_Calligraphy.png}
%
%	\medskip
%
%	\emph{Trascrizione in kanji di} bushido

%\end{center}

\begin{narratore}[Partecipazione nel bene e nel male]
		Invitate il giocatore crei un suo stile di \emph{fallimento}, fatelo gioire di un \emph{fumble}!
\end{narratore}


%\begin{center}
%	\includegraphics[width=0.95\linewidth]{immagini/fauchard.png}
%
%	\emph{Evoluzione delle armi ad asta}
%\end{center}

\subsection{Cavalcature}\index{Combattimento a saurovallo}\index{Saurovallo}\label{cavalcature}\hypertarget{cavalcare}{}\label{cavalcare}

\begin{enfasi}{
- E ti puoi trovare un'altra moglie!

- Ah, questo sì. ma il guaio è che mi ha portato via il fucile e il cavallo! Peccato, era così bella, io mi ci ero affezionato. Le davo qualche frustata, ma lei non ci faceva caso.

- Chi, tua moglie?

- No, la mia cavalla. A trovare un'altra moglie si fa presto, ma una cavalla come quella non la ritrovo più. (Ombre rosse, film 1939)}\end{enfasi}

Per comandare una cavalcatura è necessario avere la competenza Cavalcare, altrimenti è solo possibile dare la direzione del movimento.

Una cavalcatura ha 2 Azioni e di norma sono usate per spostarsi o per reagire ed ubbidire ai tuoi comandi.

Una cavalcatura agisce nel tuo round e sei tu a decidere quando esegue le sue Azioni rispetto alle tue. Non tira l'iniziativa, usa la tua.

Gli attacchi verso un personaggio su un saurovallo (o cavalcatura in genere) se non dichiarati diversamente mirano al cavaliere e non al saurovallo.

\subsubsection{Situazioni e regole}\label{cavallosituazioniregole}

\begin{itemize}[leftmargin=*] \setlength{\itemsep}{0pt}
\item
Ogni qual volta la cavalcatura è colpita il cavaliere deve effettuare una prova di Cavalcare a DC 15 o essere disarcionato.

Se la cavalcatura è da guerra, addestrata al combattimento, la prova di Cavalcare ha difficoltà 12.

\item
Combattere da posizione sopraelevata concede un +2 al Tiro per Colpire se l'avversario non è alla tua altezza.

\item
Salire o Scendere dalla cavalcatura costa 1 Azione se si ha la competenza Cavalcare, altrimenti 2 Azioni.

\item
Se una magia o situazione sposta, bruscamente, la cavalcatura contro la tua volontà devi effettuare un Tiro Salvezza su Riflessi a DC 15 oppure una prova di Cavalcare (DC 15) o venire disarcionato.

\item
Se si è disarcionati si cade a terra proni e si subiscono 1d6 di danno.
\end{itemize}

\subsubsection{Controllare una Cavalcatura}\label{controllocavalcatura}

Mentre sei in sella, hai due scelte: dai ordini alla tua cavalcatura oppure gli permetti di agire da sola.

Cavalcature particolarmente intelligenti tendono a privilegiare l'autonomia di azione piuttosto che essere comandati.

Puoi controllare una cavalcatura solo se questa è stata addestrata ad accettare un cavaliere. Si presume che saurovalli addestrati da guerra creature abbiano ricevuto tale addestramento.

Spendendo 1 tua Azione puoi fare eseguire 2 di queste Azioni alla cavalcatura: Muoversi, Attaccare, Disingaggiare.

Se la cavalcatura è intelligente questa potrebbe agire e muoversi come preferisce a discapito delle indicazioni del cavaliere. Potrebbe fuggire dal combattimento, lanciarsi all'attacco e divorare un nemico ferito gravemente, o agire in qualche altro modo contro la volontà di chi la cavalca.

\end{multicols}

\vfill

\begin{figure}[h]
		\centering
	\begin{minipage}{0.45\textwidth}
		\centering
		\includegraphics[width=\textwidth]{immagini/napoleone.png}  % Sostituisci con il nome del file
\emph{Cavallo bianco, con Napoleone Bonaparte, mentre valica il Gran San Bernardo. (Jacques-Louis David, 1801, Castello di Malmaison)}
	\end{minipage}
	\hspace{0.5cm}
	\begin{minipage}{0.45\textwidth}
		\centering
		\includegraphics[width=\textwidth]{immagini/saurovallo1-ai.png} % Sostituisci con il nome del file
\emph{Saurovallo addestrato, senza Napoleone Bonaparte, con sella e finimenti, da qualche parte nella fu Italia (B.I.C.)}
	\end{minipage}
\end{figure}

\bigskip

\begin{enfasi}
Artax galoppava attraverso la Palude della Tristezza, e ad ogni passo i suoi zoccoli affondavano più profondamente.(La Storia Infinita, Michael Ende)

\medskip

Il cavallo conosce la strada verso casa anche quando il cavaliere ha smarrito la via. (Le Tombe di Atuan, Ursula K. Le Guin)

\end{enfasi}


\pagebreak

\section{Nascondigli e coperture} \index{Nascondigli}

\begin{enfasi}
Dove c'è molta luce, l'ombra è più nera. (Johann Wolfgang von Goethe)
\end{enfasi}

\begin{multicols}{2}

Non sempre l'avversario si palesa davanti a noi, spesso può essere nascosto se non addirittura invisibile.

Potrebbe essere nascosto dietro un muretto o dei barili, se non coperto da un muscoloso e gigantesco famiglio.
E se fosse alle nostre spalle e neanche l'abbiamo notato ?

\subsection{La Copertura}\index{Copertura} \label{copertura}\hypertarget{copertura}{}

Se l'obiettivo è noto che ci sia ma è occultato in qualche maniera allora si dice che ha \textbf{copertura}.

\begin{itemize}[leftmargin=*] \setlength{\itemsep}{0pt}
\item
Se l'obiettivo \textbf{ha più della metà} (ma non totale) della superficie \textbf{visibile} allora la copertura si definisce \textbf{leggera}, ovvero ha +2 alla Difesa. Può essere il caso di una creatura dietro un altra creatura della medesima taglia o di 1 taglia più grande.

Può essere il caso di un arciere in piedi dietro un muretto di 1 metro.

\begin{center}
\includegraphics[width=0.9\linewidth]{immagini/hide.png}

\emph{British Soldiers Hiding From Boer Fire At The Battle Of Majuba Hill.}
\end{center}

\item
Se l'obiettivo ha \textbf{meno della metà} (ma almeno un terzo) della superficie \textbf{visibile} allora la copertura si definisce \textbf{media}, ovvero ha +4 alla Difesa. Può essere il caso di una creatura dietro un altra creatura di 2 taglie più grande.

Può essere il caso di un nemico armato di balestra che si sporge quel tanto per tenere appoggiata la balestra al muretto e sparare (petto, spalle, braccia e testa visibili).

\item
Se l'obiettivo si sa dove è ma \textbf{si nasconde completamente} affacciandosi solo per controllare o tirare una freccia ogni tanto, dietro ad un muro, finestra, porta, tavolo, una creatura più grande di lui (almeno 3 taglie).. allora la copertura si definisce \textbf{completa}, ovvero ha +8 alla Difesa.

\end{itemize}

Metà del bonus di copertura si applica anche ai \textbf{Tiri Salvezza} contro Incantesimi che abbiano un \textbf{effetto ad area} (es. Palle di Fuoco che esplodano intorno..).\index{Copertura nei Tiri Salvezza}

\subsubsection{Combattimento con armi da tiro in caso di Copertura}\index{Esempi di Combattimento con armi da tiro con Copertura}\label{esempicopertura}\hypertarget{esempicopertura}{}

Quando si effettuano attacchi da lancio (arco, balestre, pugnali, giavellotti...) contro avversari con copertura è necessario verificare bene la linea di tiro e controllare quante creature ci sono all'interno.

Ogni creatura di taglia uguale all'avversario in linea, che \emph{copre} l'obiettivo aumenta di un grado la copertura fornita.

\textbf{\textit{Esempio}}. la quarta creatura sulla linea di tiro equivale, in caso di creature tutte delle stessa taglia, beneficia di una Copertura Completa. La prima creatura fornirà copertura leggera (+2 Difesa), la seconda copertura media (+4 Difesa) e la terza di copertura completa (+8 Difesa). Ogni ulteriore creatura della medesima taglia somma un ulteriore +6 alla Difesa di copertura e +6 ulteriori per taglia di differenza.

\textbf{\textit{Esempio}}. se la terza creatura sulla linea di tiro è coperta da una creatura media e poi da creatura piccola, godrà di una Copertura Leggera. La creatura piccola non fornisce bonus di copertura, se l'obiettivo è di taglia media.

Se chi si deve colpire è di taglia maggiore tra le creatura coinvolte nella copertura questa non beneficerà di alcuna copertura.

La copertura fornita da una creatura di taglia maggiore rispetto alla taglia dell'obiettivo si \emph{conta una creatura in più} per taglia di differenza per la protezione data da copertura.

\textbf{\textit{Esempio.}} se la la creatura da colpire è di taglia piccola ed è coperta da una creatura grande avrà bonus di copertura pari a Completa +8 (+2 per una copertura, +4 per prima taglia di differenza, +8 per seconda taglia di differenza).

\textbf{\textit{Esempio.}} se la la creatura da colpire è di taglia media ed è coperta da una creatura di taglia grande la copertura fornita sarà +4. Normalmente sarebbe copertura leggera perché è una sola creatura a coprire, ma essendo di una taglia maggiore conta come 2 creature a fornire copertura.

Vedi anche Abilità \hyperlink{Precisino}{Precisino} (pag. \pageref{Precisino}).

\subsection{Invisibilita'}\index{Invisibilità} \hypertarget{invisibilita}{}\label{invisibilita}

Se un avversario è invisibile o non si sa dove è si seguono le regole della Invisibilità.

Anche se si è invisibili non è detto che non si possa essere percepiti diversamente attraverso altri sensi, come l'olfatto, l'udito od il tatto. L'invisibilità rende una creatura non individuabile tramite la vista ma non rende di per sé una creatura non percepibile o immune ai Tiri Critici o Esplosioni del Danno.

Una creatura accecata o che combatta contro una creatura invisibile o che combatta nell'oscurità più completa senza scurovisione, può effettuare una prova di Consapevolezza, 1 Azione, a difficoltà 20, oppure 2 Azioni a Difficoltà 15, per \textbf{individuare} una creatura se questa è entro 6 metri da lei.\index{Individuare bersagli}

La prova di Consapevolezza può essere fatta contestualmente all'Azione di Movimento per avvicinarsi alla creatura a difficoltà base di 25.

A seconda della distanza della creatura invisibile o di ciò che questa ha fatto nel round precedente sono presenti diversi modificatori alla prova di Consapevolezza per individuarla.

\medskip

\begin{narratore}[Invisibilità]
La prova di Consapevolezza ha una difficoltà alta per un personaggio di basso livello. State ben attenti a considerare tutti i modificatori del caso altrimenti i personaggi difficilmente potranno individuarli ed attaccheranno quadretti a caso...
\end{narratore}

\bigskip

\textbf{Tabella: Modificatori alla DC di Consapevolezza per Rilevare Creature Invisibili}\index[Tabelle]{Tabella Modificatori Consapevolezza per Rilevare Creature Invisibili}

\medskip

\noindent\begin{tabularx}{\linewidth}{Xl}
	\toprule
\rowcolor{gray!20}\textbf{La Creatura Invisibile...} & \textbf{Mod.}\\
\toprule
Si è mossa& -4\\
\rowcolor{gray!20}Ha scagliato un proiettile & -4\\
Un compagno che lo vede ti indirizza & -4\\
\rowcolor{gray!20}Ha corso o caricato& -8\\
Usa Furtività & prova+10\\
\rowcolor{gray!20}E' ferma e non fa rumori & +4\\
Per ogni metro oltre i 6 metri & +2\\
\rowcolor{gray!20}Ha copertura Legg./Media/Compl. & +4/8/12
\end{tabularx}

\medskip

Questi modificatori sono cumulativi tra loro.

Se la creatura invisibile ha attaccato in mischia e non si è spostata si considera \textbf{automaticamente individuata}.

Se la prova per individuare riesce l'osservatore ha la sensazione che \emph{ci sia qualcosa} ma non può vederlo o prenderlo di mira in modo accurato con un attacco.

Chi attacca una creatura per lei \textbf{invisibile ma individuata} ha un -1d6 al Tiro per Colpire, la creatura che attacca colui che non la vede ha +1d6 al Tiro per Colpire.

Una creatura Accecata \index{Accecata}subisce penalità di -2 alle Prove di Competenza Base basate su Forza e Destrezza e fallisce automaticamente qualsiasi prova di Consapevolezza dipenda dalla vista.

Attaccare un bersaglio non individuato significa attaccare un \emph{quadretto} a caso della mappa. Permettete sempre il Tiro per Colpire, che ci sia un avversario o meno in quel quadretto. Se il bersaglio è in quel quadretto modificate la sua Difesa di +8, se il \emph{quadretto} è vuoto il Tiro per Colpire non colpirà nessuno ed informerete il personaggio che non si è colpito nulla.

\medskip
\begin{center}

	\includegraphics[width=0.8\linewidth]{immagini/DnD_Invisible_stalker.png}

	\emph{Persecutore Invisibile!, LadyofHats}

\end{center}

\medskip

%\begin{center}
%	\includegraphics[width=0.9\linewidth]{immagini/brickwall.png}
%
%	\emph{c'è qualcuno davanti a questo muro ?}
%\end{center}

\subsubsection{Note su invisibilità}

Se un personaggio invisibile raccoglie un oggetto visibile, l'oggetto resta visibile. Una creatura invisibile può raccogliere un piccolo oggetto visibile e nasconderselo addosso (mettendolo in una tasca o sotto il mantello, chiudendolo nel pugno) e renderlo effettivamente invisibile.

Qualcuno potrebbe spargere su un oggetto invisibile della farina per tenere traccia almeno della sua posizione (finché la farina non cade del tutto o viene soffiata via).

Le creature invisibili lasciano impronte. Le loro tracce possono essere seguite senza problemi. Impronte su sabbia, fango o altre superfici soffici possono dare ai nemici indicazioni sulla posizione della creatura invisibile rendendola individuata.

Una creatura invisibile nell'acqua muove il liquido, rivelando la propria posizione. La creatura invisibile rimane comunque difficile da colpire e gode dei benefici di una copertura media (+4 alla Difesa).

Una torcia accesa invisibile emana comunque luce (così come un oggetto invisibile soggetto ad una magia di luce).

Le creature invisibili non possono utilizzare gli attacchi con lo sguardo. L'invisibilità non influisce sull'essere obiettivo di un incantesimo di Divinazione.

\end{multicols}

%\vspace{4cm}

%\vfill
%
%\begin{center}
%\includegraphics[keepaspectratio,width=0.4\textwidth]{immagini/impronteneve.png}
%\emph{Può aiutare a trovare un lupo invisibile...}

%\end{center}

\pagebreak

\section{Lista Armi per Tipologia Omogenea}\index{Lista Armi}\index{Tipologia Omogenea}\hypertarget{lista.armi}{}\label{lista.armi}

\begin{enfasi}{La forza non risiede in una Spada, ma nelle braccia di un valoroso. (The Legend of Zelda: Twilight Princess)} \end{enfasi}

\begin{multicols}{2}

Ogni qual volta si assegna un punto a Competenza Armi si può decidere se continuare a perfezionarsi in una Lista di Armi già nota o apprendere una nuova, se non si dichiara l'uso questo è assegnato alla Lista delle Armi Semplici.

Nella scheda segnatevi a quale Lista d'Armi assegnate il punto di Competenza Armi.

Per riassegnare un punto di Competenza Armi in un altra lista sono necessari almeno 4 ore di allenamento per 4 mesi.

Usare un arma senza l'adeguata competenza impone un -1d6 al Tiro per Colpire.

\textbf{Tutte le Liste d'Armi} concedono, se non scritto diversamente, questi vantaggi cumulativi quando il punteggio nella Lista d'Armi raggiunge il valore indicato:

\begin{itemize}[leftmargin=*] \setlength{\itemsep}{0pt}\index{Bonus comuni Lista d'armi}

\item 6 punti: se affronti qualcuno che usa un arma in questa lista sei immediatamente in grado di capire la sua capacità di Competenza Armi (o bonus al Tiro per Colpire in caso di mostri).

\item 10 punti: se colpisci con almeno due attacchi il medesimo avversario nel round puoi spostarti da lui di un metro senza usare Azioni oppure il secondo attacco causa 1 danno critico se non ha generati

\item 14 punti: se colpisci con almeno un attacco l'avversario puoi spostarti da lui di un metro senza usare Azioni.

\item 18 punti: quando effettui un Tiro per Colpire consideri anche i 5 per il conteggio dei Critici (ma non ritiri il dado).

\item 20 punti: quando effettui un Tiro per Colpire consideri anche i 5 per il conteggio dei Critici e ritiri il dado.

\end{itemize}

I punti assegnati in una Lista d'Arma non si sommano al Tiro per Colpire! Bisogna verificare il punteggio nella Lista d'Arma con gli eventuali bonus che la stessa lista elenca.

I bonus indicati nelle Liste d'Armi si applicano solo quando combatti con le armi indicate dalla stessa lista.

I vantaggi indicati sono cumulativi se non indicato diversamente.

\subsection{Archi} \index{Archi} Arco Lungo, Arco Corto, Arco Lungo Composito, Arco Corto Composito\label{listaarmiarchi}

\begin{itemize}[leftmargin=*] \setlength{\itemsep}{0pt}

\item 4 punti: aggiungi il valore di Forza al danno, anche se l'arco non è composito. Su un arco corto puoi aggiungere fino a +1 di danno, su un arco lungo fino a +2 di danno.
\item 5 punti: riduci di 6 la penalità per tirare oltre la gittata standard.
\item 7 punti: la tua maestria nell'utilizzo dell'arco in combattimento è tale che non subisci nessuna penalità nel lanciare frecce a nemici con copertura leggera.
\item 9 punti: ti e' possibile con il primo Tiro per Colpire che esegui nel round scagliare due frecce. Il Tiro per Colpire parte da una penalità di -5.
\item 11 punti: riduci di 6 la penalità per tirare oltre la gittata standard.
\item 16 punti: la prima freccia che colpisce nel round aggiunge un danno critico.

\end{itemize}

\begin{center}
\includegraphics[width=0.7\linewidth]{immagini/arma-arco.png}
\end{center}

\subsection{Armature}\index{Lista Armature} \label{listaarmature}

Questa Lista conferisce solo i bonus cumulativi qui elencati quando si indossa una Armatura.

\begin{itemize}[leftmargin=*] \setlength{\itemsep}{0pt}

\item 1 punto: dimezzi il tempo necessario per indossare e togliere un'armatura
\item 2 punti: la Difesa concessa dall'armatura aumenta di 1 punto, dormire in armature medie non causa affaticamento. Diminuisci di 2 la penalità alla Prova di Magia.
\item 3 punti: la Penalità alla Competenza diminuisce di 1 punto, dormire in armature pesanti non causa affaticamento. Diminuisci di ulteriori 2 la penalità alla Prova di Magia.
\item 4 punti: la penalità al Movimento diminuisce di 1 metro, il bonus di la Difesa concessa dell'armatura aumenta di 1 punto. Diminuisci di ulteriori 2 la penalità alla Prova di Magia.
\item 5 punti: diminuisci di 1 i Tiri Critici subiti per attacco in mischia, la Penalità alla Competenza diminuiscono di 1 punto, la penalità al Movimento diminuisce di 1 metro. Diminuisci di ulteriori 2 la penalità alla Prova di Magia.
\item 6 punti: la diminuzione del Tiro Critico subito si applica anche agli attacchi a distanza. annulli la Penalità alla Competenza ed al Movimento.Diminuisci di ulteriori 2 la penalità alla Prova di Magia.
\item 7 punti: Diminuisci di ulteriori 2 la penalità alla Prova di Magia.
\end{itemize}

\subsection{Armi Leggere}\index{Armi Leggere} Spada Corta, Mazza leggera, Stocco, Scimitarra, Ascia ad una mano, Pugnale\label{listaarmileggere}

\begin{itemize}[leftmargin=*] \setlength{\itemsep}{0pt}

\item 4 punti: puoi usare la Destrezza al posto della Forza nel Tiro per Colpire.
\item 5 punti: puoi estrarre l'arma come parte dell'Azione di Movimento.
\item 7 punti: puoi estrarre l'arma come Azione Immediata.
\item 9 punti: aumenti di un grado il dado di danno dell'arma. Se il dado di danno diventa 8 o più l'arma acquisisce l'EDX sul massimo valore del dado.
\item 11 punti: aumenti di un grado il dado di danno dell'arma. L'EDX si riduce di 1.
\item 16 punti: usando una Reazione prendi +4 alla Difesa contro attacchi in mischia. Se eviti l'attacco puoi effettuare un attacco in risposta.

\end{itemize}

\subsection{Armi doppie} \index{Armi doppie} Grande Ascia Doppia, Flagello Doppio, Spada a due lame, Urgrosh\label{listaarmidoppie}

\begin{itemize}[leftmargin=*] \setlength{\itemsep}{0pt}
\item 4 punti: la tua competenza nell'uso di queste armi ti rende estremamente versatile dandoti la possibilità a inizio del tuo round di scegliere se essere difensivo o offensivo aumentando di 1 il Tiro per Colpire o la Difesa fino all'inizio del round successivo. Non costa Azioni.
\item 5 punti: prendendo -4 al Tiro per Colpire al primo attacco che esegui nel round prendi +4 alla Difesa fino all'inizio del tuo round successivo.
\item 7 punti: usare un arma doppia non leggera non cumula il -3 aggiuntivo al Tiro per Colpire.
\item 9 punti: la tua tecnica non lascia punti scoperti, per ogni Tiro per Colpire andato a segno nel round prendi +1 alla Difesa fino all'inizio del tuo round successivo.
\item 11 punti: colpisci vorticosamente con la tua arma. Il primo colpo andato a segno equivale a due colpi andati a segno.
\item 16 punti: ogni volta che colpisci con un tiro critico puoi portare, senza usare Azioni, un colpo con l'altra estremità dell'arma. Questo Tiro per Colpire non può a sua volta causare critici e ha -4 al Tiro per Colpire.

\end{itemize}

\subsection{Armi aggraziate}\index{Armi aggraziate} Stocco, Scimitarra, Falcione\label{listaarmiaggraziate}

\medskip

\begin{center}
\includegraphics[width=0.7\linewidth]{immagini/sciabole.png}
\end{center}

\begin{itemize}[leftmargin=*] \setlength{\itemsep}{0pt}
\item 4 punti: il tuo stile assomiglia molto ad una danza. Puoi usare il valore del Carisma o Destrezza al Tiro per Colpire.
\item 5 punti: puoi usare il punteggio di Intrattenere al posto di Competenza Armi nel Tiro per Colpire.
\item 7 punti: sai colpire dove fa veramente male. Il primo Colpo Critico somma un colpo critico aggiuntivo.
\item 9 punti: il dado dell'arma aumenta di una categoria.
\item 11 punti: usando una Azione di Reazione puoi cercare di intercettare gli attacchi dell'avversario, aggiungi +2 alla Difesa fino all'inizio del tuo round successivo.
\item 16 punti: la tua danza blocca l'avversario nell'affrontarti. Costringi l'avversario in mischia con te ad attaccare solo te fino alla fine del tuo prossimo round. 1 Azione.

\end{itemize}

\begin{center}
	\includegraphics[width=0.6\linewidth]{immagini/scythe-types.png}

	\emph{Eric Sloane. A Museum of Early American Tools.}

\end{center}

\subsection{Armi della morte}\index{Armi della morte} Picca Leggera, Picca Pesante, Falce, Falcetto\label{listaarmidelamorte}

\begin{itemize}[leftmargin=*] \setlength{\itemsep}{0pt}
\item 4 punti: puoi eseguire un Colpo di Grazia con il costo di 1 Azione.
\item 5 punti: il primo colpo critico che esegui sull'avversario somma un colpo critico aggiuntivo.
\item 7 punti: aumenti di un grado il dado di danno dell'arma.
\item 9 punti: il primo colpo critico che esegui sull'avversario somma 2 colpo critico aggiuntivi.
\item 11 punti: aumenti di un grado il dado di danno dell'arma.
\item 16 punti: aumenti di un grado il dado di danno dell'arma.
\end{itemize}

\subsection{Armi da stordimento}\index{Armi da stordimento} Manganello, Guanto chiodato\label{listaarmistordimento}

\begin{itemize}[leftmargin=*] \setlength{\itemsep}{0pt}
\item 4 punti: un avversario inconsapevole se colpito con queste armi (durante il round di sorpresa) deve effettuare un Tiro Salvezza Tempra DC 15 oppure essere Rallentato 1/1r.
\item 5 punti: per ogni Tiro Critico l'avversario deve fare un Tiro Salvezza su Tempra a DC 13 o essere allentato 1/1r.
\item 7 punti: raddoppi il tuo bonus di danno dato dalla Forza. Il Tiro Salvezza dell'abilità a 4 punti diventa 19.
\item 9 punti: la difficoltà dell'abilità al punto 4 diventa 19
\item 11 punti: la tua arma da stordimento fa 1d6 di danno non letale in più. Il Tiro Salvezza dell'abilità a 4 punti diventa 23
\item 16 punti: ogni volta che colpisci con un danno critico un avversario, un compagno in mischia con quell'avversario può usare una Reazione per effettuare un attacco contro di lui.

\end{itemize}

\begin{center}
	\includegraphics[width=0.7\linewidth]{immagini/david di Michelangelo.png}

	\emph{David di Michelangelo, Galleria dell'Accademia, Firenze}
\end{center}

\subsection{Armi da Lancio} Ascia ad una mano, Giavellotto, Tridente, Fionda, Pugnale\index{Armi da lancio}\label{listarmitiro}

\begin{itemize}[leftmargin=*] \setlength{\itemsep}{0pt}
\item 4 punti: sei diventato estremamente preciso nel lancio della tua arma hai un +1 al colpire e un +1 ai danni.
\item 5 punti: il primo Tiro Critico che esegui sull'avversario somma un colpo critico aggiuntivo.
\item 7 punti: la tua abilità ti permette di non avere tempi morti dopo il lancio di un arma puoi istantaneamente estrarne un altra senza consumare azioni.
\item 9 punti: il primo Tiro per Colpire scaglia 2 armi.
\item 11 punti: riduci di 6 la penalità alla gittata oltre lo standard.
\item 16 punti: sei diventato estremamente preciso nel lancio della tua arma hai un +4 al colpire e un +4 ai danni.
\end{itemize}

\subsection{Armi letali} Katana, Machete\index{Armi letali}\label{listarmiletali}

\begin{itemize}[leftmargin=*] \setlength{\itemsep}{0pt}

\item 4 punti: contro avversari sorpresi aggiungi al danno la tua Competenza Armi.
\item 5 punti: il primo Tiro Critico che esegui sull'avversario somma un Colpo Critico in aggiunta.
\item 7 punti: aumenti di un grado il dado di danno dell'arma. Se questo porta l'arma ad avere il d8 come dado di danno acquisisce anche EDX pari a 8.  Se l'arma ha già un EDX  questo diminuisce di 1.
\item 9 punti: il primo Colpo Critico che esegui sull'avversario aggiunge due Colpi Critici.
\item 11 punti: migliori EDX.  Se l'arma ha già un EDX  questo diminuisce di 1.
\item 16 punti: aumenti di un grado il dado di danno dell'arma.
\end{itemize}

%\begin{center}
%\includegraphics[width=0.7\linewidth]{immagini/katana3.png}

%\emph{Katana}
%\end{center}

\begin{center}
	\includegraphics[width=0.7\linewidth]{immagini/alabarda2.png}

	\emph{Fauchard, Partigiana, Spetum, Alabarda, Guisarma, Bardica}
\end{center}

\subsection{Aste} \index{Aste}Giavellotto, Tridente, Alabarda\label{listaarmiaste}

\begin{itemize}[leftmargin=*] \setlength{\itemsep}{0pt}

\item 4 punti: se fai almeno un tiro critico con il Tiro per Colpire puoi lasciare l'arma nel corpo dell'avversario, penalizzandolo con un -1 Destrezza. L'arma quando rimossa fa un danno critico.
\item 5 punti: puoi effettuare un attacco di opportunità contro gli avversari che attraversano la tua zona di mischia usando una Reazione.
\item 7 punti: puoi usare l'arma lunga in mischia entro un metro senza penalità. Il danno dell'abilità a 4 punti diventa pari a 2 danni critici.
\item 9 punti: il danno dell'abilità a 4 punti diventa pari a 3 danni critici.
\item 11 punti: la gittata se assente diventa 3 metri, se presente la raddoppi.
\item 16 punti: usando una Reazione puoi seguire l'avversario mantenendo la distanza attuale di mischia. Non puoi spostarti più del tuo Movimento.

\end{itemize}

\subsection{Balestre}\index{Balestre}Balestra leggera, Balestra pesante, Balestra ad una mano\label{listaarmibalestr}

\begin{itemize}[leftmargin=*] \setlength{\itemsep}{0pt}

\item 4 punti: guadagni l'Abilità \hyperlink{Tiro Rapido}{Tiro Rapido} (pag. \pageref{Tiro Rapido}).
\item 5 punti: il primo Tiro Critico che esegui sull'avversario somma un colpo critico aggiuntivo.
\item 7 punti: ogni Azione che dedichi a mirare, fino ad un massimo di 2, ti concede un +2 a colpire.
\item 9 punti: il primo Tiro Critico che esegui sull'avversario somma due colpi critici in aggiunta, non si cumula con il vantaggio al punto 5.
\item 11 punti: riduci di 6 la penalità per tirare oltre la gittata standard.
\item 16 punti: riduci di 6 la penalità per tirare oltre la gittata standard.

\end{itemize}

%\begin{center}
%\includegraphics[width=0.9\linewidth]{immagini/arma-balestra.png}
%\end{center}

\begin{center}
	\includegraphics[width=0.7\linewidth]{immagini/arma-asta.png}

	\emph{1 Spiedo dei lanzichenecchi; 2 Picca; 3 Lancia; 4 Spiedo da caccia; 5 Buttafuoco; 6 Falcione; 7 Partigiana ; 8 Alabarda; 9 Alabarda; 10 Roncone; 11 Mazzapicchio; 12 Berdica}
\end{center}

\subsection{Lance} \index{Lance}Alabarda, Urgrosh, Lancia da fante, Falcione in asta, Lancia

\begin{itemize}[leftmargin=*] \setlength{\itemsep}{0pt}
\item 4 punti: usata contro una carica od in carica, purché abbia l'abilità Controcarica, il danno critico aggiuntivo fa il massimo valore.
\item 5 punti: puoi usarla anche contro avversari a distanza di 1 metro senza penalità.
\item 7 punti: usata contro una carica od in carica, purché abbia l'abilità Controcarica fa un danno critico aggiuntivo.
\item 9 punti: rotei la tua arma. Usando 3 Azioni fai un unico Tiro per Colpire a -5. Confronta il tiro con la Difesa di tutte le creature in mischia con te per valutare se le hai colpite.
\item 11 punti: la portata della tua lancia diventa 3 metri.
\item 16 punti: usi 2 Azioni e fai un unico Tiro per Colpire. Se va a segno causi 3 colpi critici aggiuntivi.
\end{itemize}

\begin{center}
	\includegraphics[width=0.5\linewidth]{immagini/mazzafrusto.png}
\end{center}

\subsection{Palle rotanti} Flagello, Flagello Pesante, Flagello Doppio, Catena chiodata, Frusta\label{listaarmipallerotanti}

\begin{itemize}[leftmargin=*] \setlength{\itemsep}{0pt}
\item 4 punti: se il Tiro per Colpire va a segno puoi effettuare un ulteriore TC (senza consumare Azioni) a -5 contro un avversario in mischia con te che non sia l'avversario già colpito.
\item 5 punti: se colpisci due volte l'avversario nel round, il secondo Tiro per Colpire genera un danno critico aggiuntivo.
\item 7 punti: l'impatto dei tuo colpi è tale da stordire i nemici. Se colpisci l'avversario con un Tiro Critico questo subirà Rallentato 1/1r.
\item 9 punti: puoi usare una Azione Immediata ed usare la tua arma per cercare di deviare un Tiro per Colpire a te indirizzato su una creatura a distanza di mischia dall'avversario. Effettua un Tiro per Colpire, la manovra riesce solo se è superiore al Tiro per Colpire che vuoi deviare.
\item 11 punti: la precisione ed abilità nel roteare la tua arma è tale da confondere la difesa del nemico, ignori la protezione (Difesa) data dallo scudo.
\item 16 punti: puoi usare una Azione di Reazione ed usare la tua arma per cercare di proteggere una creatura in mischia con te fino all'inizio del tuo prossimo round. La creatura prende +4 alla Difesa.
\end{itemize}

\subsection{Pugno Vuoto} Pugni e Calci\index{Pugno Vuoto}\hypertarget{pugnovuoto}{}\label{listarmipugnonudo}

Hai addestrato il tuo corpo a diventare l'arma definitiva. Sei addestrato nell'usare calci e pugni in maniera efficace e letale.

La Lista Pugno Vuoto non beneficia del Colpo Critico, tranne per il vantaggio preso a 9 punti.

\textbf{Pugno Vuoto}: Ogni volta che prendi questa competenza il danno aumenta seguendo questa progressione: 1d6 (lista presa 2 volte), 1d8 (3), 2d6 (5), 2d8 (7), 2d10 (9), 3d6 (11), 3d8 (13), 3d10 (15), 4d6 (17).

Il giocatore può anche decidere di fare danno non letale non incorrendo in alcuna penalità, al danno può applicare a proprio piacere il valore di Forza o Destrezza.

\begin{itemize}[leftmargin=*] \setlength{\itemsep}{0pt}
\item 1 punto: tuoi pugni fanno danno letale (1d4). Puoi usare il valore di Forza o Destrezza al Tiro per Colpire ed al danno.
\item 4 punti: Saggezza della mano vuota. Puoi usare il valore della Saggezza al colpire ed al Danno al posto di Forza o Destrezza. Le penalità dell'attacco multiplo diventano -4 e non -5.
\item 5 punti: il punteggio di Difesa naturale aumenta di 1 punto.
\item 9 punti: colpo solitario. Usi tre Azioni per portare un solo colpo devastante, il colpo se va a segno somma 2 colpi critici aggiuntivi.
\item 11 punti: ottieni un bonus al colpire ed al danno pari al doppio della Caratteristica usata per determinare questo bonus.

\end{itemize}

Consultate \hyperlink{equivalenzaarmimagiche}{Vulnerabilità, Resistenza e Immunità} (pag. \pageref{equivalenzaarmimagiche}) per sapere quanto è magico il vostro colpo.

\subsection{Rompi Cranio} \index{Rompi Cranio}Flagello, Maglio da guerra, Martello da guerra, Mazza Leggera, Mazza flangiata, Mazza chiodata
\label{listaarmirompicranio}

\begin{center}
\includegraphics[width=0.7\linewidth]{immagini/arma-mazza3.png}
\end{center}

\begin{itemize}[leftmargin=*] \setlength{\itemsep}{0pt}
\item 4 punti: sei diventato cosi abile che puoi controllare la forza dei tuo colpi, puoi fare danno non letale senza penalità al colpire.

Puoi scegliere di ridurre di 4 il Tiro per Colpire per aumentare il danno di 8 (non cumulabile con Colpi Potenti).
\item 5 punti: il primo Tiro Critico che esegui sull'avversario somma un colpo critico aggiuntivo.
\item 7 punti: i tuoi colpi frastornano il nemico. Ogni Tiro Critico andato a segno abbassa la Difesa di 1 punto, fino ad un massimo di 3. L'avversario recupera all'inizio del suo round un punto di penalità.
\item 9 punti: aumenti di un grado il dado di danno dell'arma.
\item 11 punti: il vantaggio a 5 punti diventa di due colpi critici.
\item 16 punti: usando una Reazione, ogni volta che colpisci con un Tiro Critico, puoi effettuare un altro Tiro per Colpire con lo stesso punteggio contro diverso avversario purché in distanza di mischia.

\end{itemize}

\begin{center}
	\includegraphics[width=0.9\linewidth]{immagini/scudotorre.png}

	\emph{Henry Justice Ford. Scudo Pesante}
\end{center}

\subsection{Scudi}\index{Scudi} Scudi Leggeri, Medi, Pesanti\label{listaarmiscudi}

Sei un maestro nell'uso degli scudi, anche come arma.

Puoi usare lo scudo come arma, uno scudo piccolo fa 1d4 di danno (B/T), uno scudo medio fa 1d6 di danno (B/T), uno scudo pesante fa 1d8 di danno (B/T).
Non hai penalità al colpire con lo scudo, per te lo scudo non è un arma improvvisata. Questa Lista d'Armi non ha il bonus dei 6 punti e quello dei 18 comuni alle altre Liste d'Armi.

La tua tecnica mescola efficacemente difesa e attacco. Puoi lanciare il tuo scudo con una gittata di 6 metri.

\begin{itemize}[leftmargin=*] \setlength{\itemsep}{0pt}
\item 1 punto: sei competente in tutte le tipologie di scudo. Non hai il vincolo del limite di Forza 1 sugli Scudi Pesanti.
\item 2 punti: il bonus di Difesa quando usi lo scudo aumenta di 1 e ogni 4 volte che prendi questa Lista d'Armi (6,10,14,18..) Non usi Azioni per ripristinare lo scudo in Difesa dopo aver effettuato un attacco con lo stesso.
\item 3 punti: la penalità Competenza Magica data dallo scudo diminuisce di un 2
\item 4 punti: la penalità al Tiro per Colpire diminuisce di 1.
\item 5 punti: aumenta di 1 la categoria di danno dello scudo ed ogni 4 punti ulteriori in lista (9,13,17..).
\item 8 punti: ogni alleato adiacente (entro 1 metro) a te ha un +1 Difesa. Puoi lanciare lo scudo entro 6m per difendere un compagno garantendogli +2 alla Difesa, da usare come Reazione. Lo scudo cade a terra dove hai difeso il compagno. Puoi lanciare il tuo scudo con una gittata di 9 metri. La penalità Competenza Magica data dallo scudo diminuisce di ulteriori 2.
\item 12 punti: puoi lanciare il tuo scudo come fosse un arma con gittata 12 metri. Se colpisci ed ottieni un Tiro Critico nel lancio dello scudo questo torna nelle tue mani a fine round. Ogni alleato adiacente (entro 1 metro) a te ha un +2 Difesa.
\item 16 punti: se un avversario esegue almeno tue tiri per colpire mancandoti entrambi puoi effettuare come Reazione un attacco di scudo contro di lui.
\item 18 punti: lo scudo lanciato ha una gittata di 18 metri e torna nelle tue mani, se non impossibilitato. Questo ti permette di effettuare attacchi multipli anche da lancio con il medesimo scudo. Puoi lanciare lo scudo per difendere un compagno garantendogli +4 alla Difesa, da usare come Reazione. Lo scudo cade a terra dove hai difeso il compagno.

I bonus indicati si applicano una volta sola anche se si usano più scudi.

\end{itemize}

\subsection{Scuri e Accette}\index{Scuri e Accette} Ascia ad una mano, Ascia da battaglia, Ascia Martello, Grande Ascia Doppia, attacchi naturali del Sornelian\label{listaasce}

\begin{itemize}[leftmargin=*] \setlength{\itemsep}{0pt}

\item 4 punti: la furia dei tuoi attacchi è tale che guadagni un +2 al danno sul colpo.
\item 5 punti: se uccidi una creatura con un colpo critico il danno in eccesso, se il Tiro per Colpire è sufficiente, lo prende un'altra creatura in mischia con te.
\item 7 punti: le ferite che provochi sono cosi profonde che causi Sanguinamento. Ogni tuo attacco andato a segno aumenta di 1 il sanguinamento fino ad un massimo di Sanguinamento 5.
\item 9 punti: ogni colpo critico che provochi aumenta il Sanguinamento di 2, fino ad un massimo di 10.

\smallskip

\begin{center}
	\includegraphics[width=0.7\linewidth]{immagini/scurieaccette.png}
\end{center}

\item 11 punti: le ferite che provochi sono cosi profonde che causi molto Sanguinamento. Il valore di Sanguinamento massimo sale a 15.
\item 16 punti: consumi 3 Azioni, effettui un singolo Tiro per Colpire che confronti contro tutte le creature in un cono pari al tuo movimento per capire se le hai colpite. Al termine dell'attacco sei in fondo al cono.

\end{itemize}

\subsection{Spade}\index{Spade} Spada Corta, Spada Lunga, Spadone a due mani, Spada bastarda, Spada a due lame, Spada larga, Spada a due lame, Estoc

\begin{center}
	\includegraphics[width=0.8\linewidth]{immagini/arma-tipi-di-spade.png}

	\emph{A Sciabola, B Scimitarra, C Spada ad una mano, D Spada larga, E Stocco, F Spada lunga, G Spada a una mano e mezza o bastarda, H Spadone a due mani}
\end{center}

\medskip

\begin{itemize}[leftmargin=*] \setlength{\itemsep}{0pt}

\item 4 punti: la tua maestria nella tecnica della spada ti conferisce +1 a danno e Tiro per Colpire.
\item 5 punti: il primo Tiro Critico che esegui sull'avversario somma un colpo critico aggiuntivo.
\item 7 punti: la tua maestria nella tecnica della spada ti conferisce +2 a danno e Tiro per Colpire.
\item 9 punti: il primo colpo andato a segno nel round somma un colpo critico.
\item 12 punti: hai raggiunto l'apice della maestria con la spada i tuo colpi sono precisi e difficili da prevedere ottieni +1 a danno, Tiro per Colpire e Difesa. L'EDX della spada se presente si abbassa di 1.
\item 16 punti: il dado di danno della tua spada aumenta di una categoria.

La mano che non tiene la spada deve essere libera o usata sull'arma.

\end{itemize}

\subsection{Spade e Scudi}\index{Spade e Scudi} Spada Corta, Spada Lunga, Spada larga, Scudo Piccolo, Scudo Medio\label{listaarmispadescudi}

\begin{itemize}[leftmargin=*] \setlength{\itemsep}{0pt}

\item 4 punti: la tua maestria nella tecnica della spada e scudo ti conferisce +1 alla Difesa ed al Tiro per Colpire.
\item 5 punti: se vai a segno con due colpi consecutivi con la spada puoi effettuare un Tiro per Colpire, senza cumulo ulteriore di penalità da multiattacco o attacco improvvisato, con lo scudo consumando una Reazione.
\item 7 punti: la tua maestria nella tecnica della spada e scudo ti conferisce +2 alla Difesa ed al Tiro per Colpire.
\item 9 punti: usando una Reazione puoi usare lo scudo per proteggere una creatura a distanza di mischia con te. La sua Difesa aumenta di 2 punti fino all'inizio del round successivo.
\item 11 punti: il dado di danno della tua spada aumenta di una categoria, EDX si abbassa di 1.
\item 16 punti: sommi il valore di Difesa dello Scudo ai Tiri Salvezza su Riflessi.

\end{itemize}

Il personaggio deve tenere in una mano la spada e nell'altra lo scudo.

\subsection{Armi Semplici} Pugnale, Mazza Leggera, Mazza chiodata, Bastone, Balestra (Leggera), Giavellotto.\index{Armi Semplici}\hypertarget{armi.semplici}{}\label{listaarmisemplice}

\medskip

Questa suddivisione è sceglibile anche da chi non ha assegnato punti a Competenza Armi. Questa Lista d'Armi non concede bonus specifici.

\subsection{Armi in più Lista d'Armi}\index{Armi in più Liste d'Armi}\label{listaarmiinpiuliste}

Quando un personaggio usa un arma presente in più Liste d'Armi conosciute può applicare per avversario una sola tecnica (una Lista d'Armi) di combattimento, non cumula i vantaggi anche di eventuali altre liste.

Utilizzando 2 Azioni può concentrarsi e passare ad utilizzare i bonus derivanti dall'applicazione di una diversa Lista D'Armi.

\end{multicols}

\vfill

\begin{center}
\includegraphics[width=0.95\linewidth]{immagini/brancastle.png}

\emph{dettaglio dal Castello di Bran, Transilvania}
\end{center}

\pagebreak

\section{Abilità}\index{Abilità}\hypertarget{abilita}{}\label{abilita}

\begin{enfasi}{Il martirio è l'unica maniera per un uomo di diventare famoso se non ha abilità (George Bernard Shaw, The Devil's Disciple)} \end{enfasi}

\begin{multicols}{2}

Le Abilità sono capacità peculiari, frutto di allenamento o doti particolari. Le Abilità hanno sempre un effetto pratico.

Le Abilità costituiscono una buona parte di ciò che può fare il personaggio, vanno scelte con attenzione e cura. E' scegliendo le Abilità che si stabilisce lo stile e capacità del personaggio, se lo si vuole più guerriero o mago o curatore... o qualsivoglia combinazione e \emph{unicità}.

\textbf{Al primo livello si prendono due Abilità}. Successivamente si prende una Abilità ai livelli 2, 3, 4, 5, 6, 7, 9, 10, 12, 13, 15, 16, 18, 20. Questa può essere un'Abilità già conosciuta oppure una nuova Abilità appresa durante le avventure.\index{Abilita' al primo livello}\index{Abilità nei livelli}

E' possibile che siano indicati dei Requisiti sotto il nome dell'Abilità, in questo caso vanno rispettati per prendere l'Abilità in oggetto.
Eventuali requisiti successivi vengono indicati volta per volta.

Non prendete le Abilità in base al potere, forza, combinazione con altre ma perché in linea con la storia del personaggio.
Scegliere un'accozzaglia di Abilità solo perché forti non rende un personaggio potente ma sbilanciato, non fate il power-player ad ogni costo.

\medskip

\textbf{Le Abilità devono essere prese in base al percorso evolutivo del personaggio, in base a quanto vissuto ed appreso durante le avventure.}

\medskip

E' possibile cambiare una Abilità scelta, rispettando comunque i requisiti, riaddestrandosi per almeno una settimana per 4 ore al giorno con qualcuno che abbia la nuova Abilità.\index{Riaddestrarsi}\index{Cambiare le Abilita'}

Le capacità fornite dalle Abilità se non descritto diversamente sono cumulative o se si tratta dello stesso bonus si applica quello maggiore. Se non esplicitato una Abilità non può essere presa più volte.

\subsection{Tiri Salvezza ed Abilita'}\label{tirisalvezzaedabilita}

Ogni Abilità, tranne se diversamente indicato, concede dei bonus ai Tiri Salvezza che si cumulano tra loro, anche quando si prende più volte la stessa Abilità.

Quando scegliete un'Abilità fate anche caso a quali Tiri Salvezza aumenta!

\subsection{Caratteristiche ed Abilita'}\label{caratteristicheedabilita}\index{Caratteristiche ed Abilita'}

Ogni Abilità ha segnato una nota tipo \emph{\textbf{Caratteristica}} con a fianco l'indicazione di una o più Caratteristiche. Nella scheda segnate che Caratteristica, 1 sola, decidete di potenziare con quell'Abilità così che ogni 4 potenziamenti alla medesima Caratteristica potete aumentare di 1, entro il limite di 4 + modificatori razziali, il punteggio della Caratteristica in questione.

\subsection{Aggiungere nuove Abilita'}\label{aggiungereabilita}

Questo elenco non potrà mai essere esaustivo data la fantasia dei giocatori! Cercate però di capire se quello che il giocatore vuole è una Abilità o Competenza, l'avere una capacità o sapere fare qualcosa di particolare.
Valutate bene i prerequisiti ed i vantaggi che concede, cercate sempre di essere bilanciati, piuttosto concedete dei vantaggi a scalare, ovvero prendendo più volte l'Abilità.

Ricordatevi anche di segnare i bonus relativi ai Tiri Salvezza. Solitamente una Abilità concreta e pratica concede un bonus di +3 divisi tra 2 Tiri Salvezza, una Abilità più generica concede un 2 punti da dividere tra un solo Tiro Salvezza o due.

\subsection{Elenco Abilità}

\feat{Adepto della Magia}
\begin{description}[noitemsep, topsep=0pt, parsep=0pt, partopsep=0pt, leftmargin=0cm, labelwidth=2.5cm]
    \item[\textbf{Requisito}]: Competenza Magica 1
    \item[\textbf{Tiri Salvezza}]: +1 a due Tiri Salvezza a propria scelta.
    \item[\textbf{Caratteristica}]: Modificatore di caratteristica per incantesimi
\end{description}

Tramite questa Abilità si approfondisce la capacità di lanciare incantesimi.

L'Abilità Adepto della Magia permette di lanciare incantesimi di più alto livello, di rendere più difficile resistere ai propri incantesimi, di fallire meno nella Prova di Magia.

Ogni volta che prendi questa Abilità \emph{apprendi} un incantesimo in più presente nel Tomo di Magia.

L'Abilità è prendibile più volte purché sia inferiore a CM/2.

\feat{Ali della Fenice}
\begin{description}[noitemsep, topsep=0pt, parsep=0pt, partopsep=0pt, leftmargin=0cm, labelwidth=2.5cm]
    \item[\textbf{Requisito}]: Lista Pugno Vuoto 3, Gru d'Argento 1
    \item[\textbf{Tiri Salvezza}]: +2 Riflessi, +1 Tempra
    \item[\textbf{Caratteristica}]: Destrezza o Forza
\end{description}

Il tuo stile di combattimento enfatizza i colpi portati da lontano come pugni e calci volanti.

La \textbf{prima volta} che prendi questa Abilità la tua distanza di mischia con la Lista Pugno Vuoto diventa di 2 metri.

Le \textbf{seconda volta} che prendi questa Abilità, requisito Lista Pugno Vuoto 6, Gru d'Argento 3, Pugno di Ferro 1, la tua distanza di mischia diventa di 3 metri.

Le \textbf{terza volta} che prendi questa Abilità, requisito Lista Pugno Vuoto 12, Gru d'Argento 4, Pugno di Ferro 2, la tua distanza di mischia diventa di 4 metri.

\feat{Allungo}
\begin{description}[noitemsep, topsep=0pt, parsep=0pt, partopsep=0pt, leftmargin=0cm, labelwidth=2.5cm]
    \item[\textbf{Requisito}]: Competenza Armi 2
    \item[\textbf{Tiri Salvezza}]: +1 Volontà, +2 Tempra
    \item[\textbf{Caratteristica}]: Destrezza o Forza
\end{description}

Usi una Azione in concomitanza alla tua Azione di Attacco in mischia per aumentare la portata di 1 metro con il tuo colpo.

\feat{Animalia}
\begin{description}[noitemsep, topsep=0pt, parsep=0pt, partopsep=0pt, leftmargin=0cm, labelwidth=2.5cm]
    \item[\textbf{Requisito}]: Seguace o Devoto di Efrem oppure Shayalia, Competenza Magica 2.
    \item[\textbf{Tiri Salvezza}]: +2 Volontà, +1 Tempra
    \item[\textbf{Caratteristica}]: Modificatore di caratteristica per incantesimi
\end{description}

Si acquisisce la capacità di trasformarsi in una creatura conosciuta. Costo 2 Azioni.

I propri incantesimi di cura funzionano anche su Animali e Piante, normali e magiche.

\medskip

La \textbf{prima volta} che prendi questa Abilità puoi trasformarti in una creatura con queste caratteristiche:

\medskip

\textbf{Tipologia Creature}: Bestie

\textbf{Caratteristiche}: quelle fisiche, Difesa, Tiri Salvezza e forme di attacco sono dell'animale.

\textbf{Incantesimi}: non puoi lanciare incantesimi nella nuova forma.

\textbf{Equipaggiamento}: vedi Regole base per la trasformazione.

\medskip

La \textbf{seconda volta} che prendi questa Abilità puoi trasformarti anche in una creatura con queste caratteristiche:

\medskip

\textbf{Requisito}: Competenza Magica 4

\textbf{Tipologia Creature}: Piante e Melme

\textbf{Caratteristiche}: il Personaggio sceglie se le Caratteristiche fisiche, Difesa, Tiri Salvezza sono proprie o dell'animale.

\textbf{Incantesimi}: non puoi lanciare incantesimi nella nuova forma

\textbf{Equipaggiamento}: quello magico non ha effetto. Armature e Scudi applicano il bonus magico alla Difesa della creatura. Le capacità magiche degli oggetti non possono essere attivate.

\medskip

La \textbf{terza volta} che prendi questa Abilità puoi trasformarti anche in una creatura con queste caratteristiche:

\medskip

\textbf{Requisito}: Competenza Magica 10

\textbf{Tipologia Creature}: Elementali

\textbf{Caratteristiche}: il Personaggio sceglie se le Caratteristiche fisiche, Difesa, Tiri Salvezza sono proprie o dell'animale.

\textbf{Incantesimi}: puoi lanciare incantesimi nella nuova forma purché abbiano componenti solo Verbali

\textbf{Equipaggiamento}: quello magico continua ad avere effetto se possibile. Armature e Scudi applicano il bonus magico alla Difesa della creatura e le capacità magiche degli oggetti non possono essere attivate.

\medskip

La \textbf{quarta volta} che prendi questa Abilità puoi trasformarti anche in una creatura con queste caratteristiche:

\medskip

\textbf{Requisito}: Competenza Magica 16

\textbf{Tipologia Creature}: Mostruosità

\textbf{Caratteristiche}: il Personaggio sceglie se le Caratteristiche fisiche, Difesa, Tiri Salvezza sono proprie o dell'animale. I Punti Ferita rimangono quelli del personaggio.

\textbf{Incantesimi}: puoi lanciare incantesimi nella nuova forma purché abbiano componenti solo Verbali e Somatici

\textbf{Equipaggiamento}: quello magico continua ad avere effetto se possibile. Armature e Scudi applicano il bonus magico alla Difesa della creatura ed eventuali capacità magiche possono essere attivate.

\medskip

\textbf{Regole base per la trasformazione}

\medskip

L'\textbf{equipaggiamento} viene assorbito nella nuova forma e non può essere usato. Quello magico può funzionare ed essere usato come indicato.

La creatura in cui ti trasformi deve avere un \textbf{Grado di Sfida} entro un terzo del tuo punteggio di Competenza Magica + le volte che hai preso l'Abilità Animalia.

I \textbf{Punti Ferita} rimangono quelli attuali del personaggio. La \textbf{taglia} della creatura in cui ti trasformi deve essere entro la tua $\pm 1$ ogni 2 volte che hai preso questa Abilità.

Puoi \textbf{rimanere trasformato} per 1 minuto per somma Tratto in comune con il Patrono  o per punto in Competenza Magica, con utilizzo minimo di 1 minuto.

Costa 2 Azioni cambiare forma e prima di passare da una forma all'altra è necessario tornare in forma normale.

Il personaggio conserva i propri Tratti, personalità, Abilità (ma non è detto che la nuova forma gli permetta di usarle) e caratteristiche mentali.

Se la creatura possiede una competenza che anche il personaggio possiede ed il bonus della creatura è superiore a quello del personaggio allora usa il bonus della creatura anziché il proprio. Se la creatura possiede delle azioni aggiuntive o di tana, il personaggio non può usarle.

Qualsiasi azione che richieda le mani è limitata alle capacità della sua nuova forma. La trasformazione non interrompe la concentrazione del personaggio su un incantesimo che egli ha già lanciato e non gli impedisce di effettuare azioni che fanno parte di un incantesimo già lanciato, come per esempio Invocare il Fulmine.

Le forme di attacco sono sempre quelle delle creatura.

Della nuova forma acquisisce le caratteristiche e capacità, come sensi, movimento, lingue (ma non è detto che che possa parlare altre lingue oltre quella dell'animale).

Quando sei trasformato puoi canalizzare i tuoi Punti Magia per migliorare la trasformazione, per ogni Punto Magia usato prendi un +1 al Tiro per Colpire, al danno con gli attacchi, Difesa e Tiri Salvezza. La capacità va dichiarata all'inizio del round come Azione Immediata che dura fino all'inizio del tuo round successivo. Non puoi usare più Punti Magia alla volta di quante volte tu abbia preso l'Abilità Animalia.

\begin{center}
\includegraphics[width=0.9\linewidth]{immagini/animalia3.png}
\emph{Henry Justice Ford}
\end{center}

\feat{Animaletto / Famiglio}
\begin{description}[noitemsep, topsep=0pt, parsep=0pt, partopsep=0pt, leftmargin=0cm, labelwidth=2.5cm]
    \item[\textbf{Requisito}]: Competenza Magica 1
    \item[\textbf{Tiri Salvezza}]: +1 Volontà, +1 Tempra
    \item[\textbf{Caratteristica}]: Intelligenza o Modificatore di caratteristica per incantesimi
\end{description}

La \textbf{prima volta} che prendi questa Abilità guadagni un animale naturale. Questo animaletto ha un Grado di Sfida pari ad un quarto della tua Saggezza, con un minimo di 1/4. Puoi insegnare azioni di base al tuo animale e fargli fare dei compiti semplici.

La \textbf{seconda volta} che prendi questa Abilità guadagni un \hyperlink{famiglio}{Famiglio} (pag. \pageref{famiglio}).

\feat{Armato}
\begin{description}[noitemsep, topsep=0pt, parsep=0pt, partopsep=0pt, leftmargin=0cm, labelwidth=2.5cm]
    \item[\textbf{Requisito}]: Forza 3, Competenza Armi 1
    \item[\textbf{Tiri Salvezza}]: +2 Tempra
    \item[\textbf{Caratteristica}]: Forza o Costituzione
\end{description}

La \textbf{prima volta} che prendi questa Abilità quando usi un arma di una taglia troppo grande la penalità al colpire diventa -2.

La \textbf{seconda volta} che prendi questa Abilità, requisito Competenza Armi 6, non hai penalità nell'usare un arma di una taglia superiore.

\feat{Armatura del Devoto}
\begin{description}[noitemsep, topsep=0pt, parsep=0pt, partopsep=0pt, leftmargin=0cm, labelwidth=2.5cm]
    \item[\textbf{Requisito}]: Valore singolo Tratto in comune con il Patrono 4, essere Devoto o Seguace
    \item[\textbf{Tiri Salvezza}]: +2 Volontà, +1 Riflessi
    \item[\textbf{Caratteristica}]: Costituzione o Modificatore di caratteristica per incantesimi
\end{description}

La \textbf{prima volta} che prendi questa Abilità il costante allenamento con la tua armatura riduce di 2 la penalità alla Prova di Magia quando indossi armature leggere.

La \textbf{seconda volta} che si prende questa Abilità, requisito singolo Tratto 6, riduci di 4 la penalità alla Prova di Magia quando indossi armature medie.

La \textbf{terza volta} che si prende questa Abilità, requisito singolo Tratto 8, il portare armature leggere non ti obbliga a fare Prove di Magia, riduci la penalità per armature medie di 6 e riduci la penalità per le armature pesanti di 8.

La \textbf{quarta volta} che si prende l'Abilità, requisito singolo Tratto 12, la Prova di Magia per indossare armature è obbligatoria solo se indossi armature pesanti e riduci la penalità di 12.

\feat{Armatura della Montagna Incantata}
\begin{description}[noitemsep, topsep=0pt, parsep=0pt, partopsep=0pt, leftmargin=0cm, labelwidth=2.5cm]
    \item[\textbf{Requisito}]: Lista armi Pugno Vuoto 1, Competenza Armi 1, Costituzione 2, Saggezza 1
    \item[\textbf{Tiri Salvezza}]: +2 Tempra, +1 Volontà
    \item[\textbf{Caratteristica}]: Costituzione o Saggezza
\end{description}

Il constante allenamento nello spirito e corpo di permette di indurire la tua pelle e renderla più difficile da ferire. Per usufruire di questi bonus non devi portare armature o scudi od oggetti che migliorino la Difesa. Le capacità elencate non sono cumulabili con l'Abilità Gru d'Argento.

La \textbf{prima volta} che prendi questa Abilità la tua Difesa è 10 + Costituzione + 1/3 dei punti in Pugno Vuoto + varie ed eventuali.

La \textbf{seconda volta} che prendi questa Abilità, requisito Pugno Vuoto 5, acquisisci una riduzione al danno (DR) di 1/-

La \textbf{terza volta} che prendi questa Abilità, requisito Pugno Vuoto 7, riduci automaticamente il Sanguinamento di 1 a fine round.

La \textbf{quarta volta} che prendi questa Abilità, requisito Pugno Vuoto 8, acquisisci una riduzione al danno (DR) di 3/-

La \textbf{quinta volta} che prendi questa Abilità, requisito Pugno Vuoto 13, acquisisci una riduzione al danno (DR) di 5/-

\feat{Arciere su saurovallo}
\begin{description}[noitemsep, topsep=0pt, parsep=0pt, partopsep=0pt, leftmargin=0cm, labelwidth=2.5cm]
    \item[\textbf{Requisito}]: Competenza Armi 1
    \item[\textbf{Tiri Salvezza}]: +1 Riflessi, +1 Tempra
    \item[\textbf{Caratteristica}]: Destrezza o Saggezza
\end{description}

le penalità di tirare frecce da saurovallo diminuisce di 2 ogni volta che prendi questa Abilità.

Le penalità standard sono -4 e -6 a seconda che si trotti (movimento x2) o galoppi (movimento x3)

\feat{Arma Focalizzata}
\begin{description}[noitemsep, topsep=0pt, parsep=0pt, partopsep=0pt, leftmargin=0cm, labelwidth=2.5cm]
    \item[\textbf{Requisito}]: Competenza Armi 1
    \item[\textbf{Tiri Salvezza}]: +1 Riflessi, +1 Tempra
    \item[\textbf{Caratteristica}]: Forza o Destrezza
\end{description}

Scegli un arma in una Lista d'Armi che conosci. Ottieni un +1 a Iniziativa e Tiro per Colpire quando usi questa arma.


%\begin{center}
%\includegraphics[width=0.9\linewidth]{immagini/horsearcher.png}

%\emph{Arcere Assiro}
%\end{center}

\feat{Artista dell'Arma}
\begin{description}[noitemsep, topsep=0pt, parsep=0pt, partopsep=0pt, leftmargin=0cm, labelwidth=2.5cm]
    \item[\textbf{Requisito}]: Competenza Armi 2
    \item[\textbf{Tiri Salvezza}]: +1 Volontà, +1 Tempra
    \item[\textbf{Caratteristica}]: Destrezza o Forza
\end{description}

Scegli una Lista d'Armi, su queste armi ottieni un +1 al colpire.

L'Abilità può essere presa più volte, con almeno CA 5,9,13.

Se prendi \textbf{4 volte} questa Abilità sulla stessa Lista d'Armi i bonus al colpire si riducono a +2, invece che +4, ma effettui due Tiri per Colpire per il primo attacco del round e scegli il tiro da tenere.

\feat{Attacco Turbinante}
\begin{description}[noitemsep, topsep=0pt, parsep=0pt, partopsep=0pt, leftmargin=0cm, labelwidth=2.5cm]
    \item[\textbf{Requisito}]: Competenza Armi 12, Intrattenere 3
    \item[\textbf{Tiri Salvezza}]: +2 Riflessi, +1 Tempra
    \item[\textbf{Caratteristica}]: Destrezza o Carisma
\end{description}

La \textbf{prima volta} che prendi questa Abilità usando 3 Azioni puoi eseguire un singolo attacco (con penalità di 1d6 al Tiro per Colpire) contro tutti gli avversari in mischia attorno a te.

La \textbf{seconda volta} che prendi questa Abilità, Competenza Armi 15, Intrattenere 5, non hai la penalità al Tiro per Colpire.

\feat{Batteria Magica}
\begin{description}[noitemsep, topsep=0pt, parsep=0pt, partopsep=0pt, leftmargin=0cm, labelwidth=2.5cm]
    \item[\textbf{Requisito}]: Competenza Magica 3
    \item[\textbf{Tiri Salvezza}]: +2 Volontà, +1 Tempra
    \item[\textbf{Caratteristica}]: Modificatore di caratteristica per incantesimi
\end{description}

Hai una particolare connessione con la magia che permane la Terra.

La prima volta che prendi questa Abilità aumenti di 3 i punti Magia a disposizione.

L'Abilità può essere presa più volte ed il totale deve essere pari o inferiore a CM/3.

\feat{Batteria Estesa}
\begin{description}[noitemsep, topsep=0pt, parsep=0pt, partopsep=0pt, leftmargin=0cm, labelwidth=2.5cm]
    \item[\textbf{Requisito}]: Competenza Magica 1, Adepto della Magia
    \item[\textbf{Tiri Salvezza}]: + 1 Tempra, +1 Volontà
    \item[\textbf{Caratteristica}]: Modificatore di caratteristica per incantesimi
\end{description}

Riesci a sopportare meglio lo stress di lanciare incantesimi.

Quando effettui una Prova di Magia e riesci in almeno un Successo Critico Magico il costo dell'incantesimo diminuisce di un punto, con un costo minimo di 1.

\feat{Colosso}
\begin{description}[noitemsep, topsep=0pt, parsep=0pt, partopsep=0pt, leftmargin=0cm, labelwidth=2.5cm]
	\item[\textbf{Requisito}]: Costituzione 1
	\item[\textbf{Tiri Salvezza}]: +3 Tempra
	\item[\textbf{Caratteristica}]: Costituzione o Forza
\end{description}

Forse una volta eri gracile e debole, adesso sei una montagna di muscoli.

La \textbf{prima volta} che prendi questa Abilità quando prendi questa Abilità aumenti di 1d6 i Punti Ferita.

La \textbf{seconda volta} che prendi questa Abilità aumenti di 1 i Punti Ferita presi per livello.

La \textbf{terza volta} che prendi questa Abilità aumenti la taglia del dado per tirare i Punti Ferita.

I bonus sono cumulativi e retroattivi ai livelli precedenti, tranne che l'aumento del dado per determinare i PF.

La \textbf{quarta volta}, requisito Costituzione 3, che prendi questa Abilità aumenti di una taglia (P > M > G).


\begin{center}
	\includegraphics[width=0.65\linewidth]{immagini/elcolosso.png}

	\emph{The Colossus (also known as The Giant), is known in Spanish as El Coloso.}
\end{center}


\feat{Colpo Furtivo}
\begin{description}[noitemsep, topsep=0pt, parsep=0pt, partopsep=0pt, leftmargin=0cm, labelwidth=2.5cm]
    \item[\textbf{Requisito}]: Competenza Armi 3
    \item[\textbf{Tiri Salvezza}]: +2 Riflessi, +1 Volontà
    \item[\textbf{Caratteristica}]: Destrezza o Intelligenza
\end{description}

La \textbf{prima volta} che prendi questa Abilità quando l'avversario viene \hyperlink{sorpresa}{sorpreso} (vedi pag. \pageref{coltidisorpresa}) con un arma da mischia se il primo attacco del combattimento colpisce causa un danno critico aggiuntivo.

La \textbf{seconda volta} che si prende questa Abilità, requisito Competenza Armi 6, causi 2 danni critici aggiuntivi.

La \textbf{terza volta} che si prende questa Abilità, requisito Competenza Armi 10, causi 3 danni critici aggiuntivi.

La \textbf{quarta} che si prende questa Abilità, requisito Competenza Armi 12, causi 4 danni critici aggiuntivi.


\feat{Colpo Indebolente}
\begin{description}[noitemsep, topsep=0pt, parsep=0pt, partopsep=0pt, leftmargin=0cm, labelwidth=2.5cm]
    \item[\textbf{Requisito}]: Colpo furtivo 3, Competenza Armi 12
    \item[\textbf{Tiri Salvezza}]: +2 Riflessi, +1 Volontà
    \item[\textbf{Caratteristica}]: Intelligenza o Destrezza
\end{description}

Colpo Indebolente è una forma avanzata di colpo furtivo. Ogni Colpo Indebolente abbassa Forza o Destrezza (scelta giocatore) di quante volte si è preso Colpo Furtivo.

All'avversario è concesso un Tiro Salvezza Tempra con DC pari al Tiro per Colpire. Si causa il danno aggiuntivo del Colpo Furtivo o la perdita di punti caratteristica.

\feat{Colpo Mortale}
\begin{description}[noitemsep, topsep=0pt, parsep=0pt, partopsep=0pt, leftmargin=0cm, labelwidth=2.5cm]
    \item[\textbf{Requisito}]: Competenza Armi 5
    \item[\textbf{Tiri Salvezza}]: +2 Riflessi, +1 Volontà
    \item[\textbf{Caratteristica}]: Saggezza o Forza
\end{description}

Esegui il Tiro per Colpire con penalità di -1d6, se colpisci causi 2 danni critici. I Tiri per Colpire successivi partono da -10 al colpire.

\feat{Colpo Paralizzante}
\begin{description}[noitemsep, topsep=0pt, parsep=0pt, partopsep=0pt, leftmargin=0cm, labelwidth=2.5cm]
    \item[\textbf{Requisito}]: Colpo Indebolente, Colpo furtivo 4, Competenza Armi 18
    \item[\textbf{Tiri Salvezza}]: +2 Riflessi, +1 Tempra
    \item[\textbf{Caratteristica}]: Forza o Destrezza
\end{description}

Dedichi 2 Azioni a Round, per 5 round, a studiare un avversario che puoi minacciare. Nel sesto round utilizzando 2 Azioni porti un attacco in mischia o da distanza. L'avversario deve effettuare un Tiro Salvezza su Tempra con DC pari al Tiro per Compire o rimanere paralizzato per 3d6 round. La creatura non deve essere di 2 taglie più grande della tua.


\begin{center}
	\includegraphics[width=0.7\linewidth]{immagini/teseo.png}

	\emph{Henry Justice Ford - Colpo furtivo!}
\end{center}


\feat{Colpi Poderosi}
\begin{description}[noitemsep, topsep=0pt, parsep=0pt, partopsep=0pt, leftmargin=0cm, labelwidth=2.5cm]
    \item[\textbf{Requisito}]: Competenza Armi 1
    \item[\textbf{Tiri Salvezza}]: +2 Tempra
    \item[\textbf{Caratteristica}]: Forza o Costituzione
\end{description}

Il tuo stile enfatizza colpi poderosi.

Guadagni un +1 al danno con una Lista d'Arma.


\feat{Combattere alla Cieca}
\begin{description}[noitemsep, topsep=0pt, parsep=0pt, partopsep=0pt, leftmargin=0cm, labelwidth=2.5cm]
    \item[\textbf{Requisito}]: Consapevolezza 2
    \item[\textbf{Tiri Salvezza}]: +2 Riflessi, +1 Volontà
    \item[\textbf{Caratteristica}]: Destrezza o Saggezza
\end{description}

La \textbf{prima volta} che prendi questa Abilità un avversario con copertura leggera non ottiene bonus alla Difesa, con copertura media ha un +2 alla Difesa, con copertura completa ha un +6 alla Difesa.

Un attaccante invisibile in mischia non ottiene alcun vantaggio al colpire il personaggio in mischia.

La \textbf{seconda volta} che prendi l'Abilità, requisito Consapevolezza a 3, riduci di ulteriori due il bonus alla Difesa da creature con copertura completa.

Non c'è bisogno di effettuare prove di Acrobatica per muoversi a piena velocità mentre si è Accecati.

La penalità al Tiro per Colpire contro creature invisibili è -2.

\emph{Livello Zatoichi}, la \textbf{terza volta} che prendi l'Abilità, requisito Consapevolezza a 5, in mischia una creatura invisibile non ha alcun vantaggio contro di te ne tu hai penalità contro di lei.

\feat{Combattimento con due armi}
\begin{description}[noitemsep, topsep=0pt, parsep=0pt, partopsep=0pt, leftmargin=0cm, labelwidth=2.5cm]
    \item[\textbf{Requisito}]: Destrezza 2, Forza 1, Competenza Armi 2
    \item[\textbf{Tiri Salvezza}]: +2 Riflessi, +1 Tempra
    \item[\textbf{Caratteristica}]: Destrezza o Forza
\end{description}

La \textbf{prima volta} che prendi questa Abilità il constante è continuo allenamento ti permette di ridurre la penalità del multiattacco dato dall'attacco con l'arma secondaria. Quando attacchi con l'arma secondaria cumuli penalità al colpire di -4 al posto di -5 se l'arma è leggera.

\textbf{Requisito} Destrezza 3, Competenza Armi 12

La \textbf{seconda volta} se l'arma secondaria non è leggera non cumuli l'ulteriore -3 al colpire.

\textbf{Requisito} Competenza Armi 18

La \textbf{terza volta} il primo attacco effettuato con l'arma secondaria non cumula la penalità degli attacchi multipli.

\feat{Concentrato}
\begin{description}[noitemsep, topsep=0pt, parsep=0pt, partopsep=0pt, leftmargin=0cm, labelwidth=2.5cm]
    \item[\textbf{Requisito}]: Competenza Magica 2
    \item[\textbf{Tiri Salvezza}]: +1 Tempra, +1 Volontà
    \item[\textbf{Caratteristica}]: Modificatore di caratteristica per incantesimi
\end{description}

Scegli una Lista di Magia, la DC dei Tiri Salvezza dei tuoi incantesimi in quella lista aumenta di 1.

L'Abilità può essere presa più volte sulla stessa Lista di Magia o su altre liste ed il totale deve essere pari o inferiore a CM/4.

\feat{Conoscenza istintiva}
\begin{description}[noitemsep, topsep=0pt, parsep=0pt, partopsep=0pt, leftmargin=0cm, labelwidth=2.5cm]
    \item[\textbf{Requisito}]: Conoscenza 1
    \item[\textbf{Tiri Salvezza}]: +2 Volontà, +1 Tempra
    \item[\textbf{Caratteristica}]: Saggezza o Intelligenza
\end{description}

Non dimentichi mai un nemico.

Hai una istintiva capacità nel ricordare e valutare un nemico. Quando prendi questa Abilità puoi effettuare una prova di \hyperlink{riconoscereimostri}{Riconoscere un Mostro} (pag. \pageref{riconoscereimostri}) utilizzando una Reazione.


\feat{Creare Oggetti Magici}
\begin{description}[noitemsep, topsep=0pt, parsep=0pt, partopsep=0pt, leftmargin=0cm, labelwidth=2.5cm]
    \item[\textbf{Requisito}]: Competenza Magica 6
    \item[\textbf{Tiri Salvezza}]: +1 Tempra, +1 Volontà
    \item[\textbf{Caratteristica}]: Modificatore di caratteristica per incantesimi o a scelta
\end{description}

La \textbf{prima volta} che prendi questa Abilità tramite questa Abilità l'incantatore è in grado di infondere un incantesimo fino a livello 3 in un oggetto magico.

Le \textbf{seconda volta} che prendi questa Abilità, requisito Competenza Magica 12, l'incantatore è in grado di infondere un incantesimo fino a livello 5 in un oggetto magico.

Le \textbf{terza volta} che prendi questa Abilità, requisito Competenza Magica 16, l'incantatore è in grado di infondere un incantesimo fino a livello 8 in un oggetto magico.

Le \textbf{quarta volta} che prendi questa Abilità, requisito Competenza Magica 18, l'incantatore è in grado di infondere un incantesimo fino a livello 9 in un oggetto magico.


\begin{center}
	\includegraphics[width=0.8\linewidth]{immagini/oggettimagiciuomo.png}

	\emph{Henry Purcell - King Arthur}
\end{center}

\feat{Dadi Truccati}
\begin{description}[noitemsep, topsep=0pt, parsep=0pt, partopsep=0pt, leftmargin=0cm, labelwidth=2.5cm]
    \item[\textbf{Requisito}]: Competenza Magica 6
    \item[\textbf{Tiri Salvezza}]: +1 Tempra, +1 Riflessi
    \item[\textbf{Caratteristica}]: Saggezza o Carisma
\end{description}

La \textbf{prima volta} che prendi questa Abilità puoi aumentare di 1, entro il valore di 6, un dado nella Prova di Magia.

La \textbf{seconda volta} che prendi questa Abilità, requisito Competenza Magica 12, puoi aumentare di 1, entro il valore di 6, un ulteriore dado nella Prova di Magia.

\feat{Danno Coordinato}
\begin{description}[noitemsep, topsep=0pt, parsep=0pt, partopsep=0pt, leftmargin=0cm, labelwidth=2.5cm]
    \item[\textbf{Requisito}]: Competenza Armi 6, Saggezza 2
    \item[\textbf{Tiri Salvezza}]: +2 Volontà
    \item[\textbf{Caratteristica}]: Carisma o Intelligenza
\end{description}

La tua esperienza nel gestire gli alleati ti permette di massimizzare l'efficacia degli attacchi.

La \textbf{prima volta} che prendi questa Abilità puoi coordinare gli attacchi di due tuoi alleati, che siano a distanza di mischia tra loro, affinché il danno causato da uno colpisca il nemico dell'altro e vice versa. Costa 2 Azioni eseguire questo coordinamento.

La \textbf{seconda volta} che prendi questa Abilità, requisito Competenza Armi 8, Intelligenza 2, puoi coordinare e scambiare il danno di tre alleati purché in distanza di mischia tra loro. Costo 2 Azioni.

E' necessario che i Tiri per Colpire vadano a segno per poter applicare il danno all'altro avversario.

\feat{Danza della Lama}
\begin{description}[noitemsep, topsep=0pt, parsep=0pt, partopsep=0pt, leftmargin=0cm, labelwidth=2.5cm]
    \item[\textbf{Requisito}]: Lista d'Armi: Armi Aggraziate a 2, Destrezza o Carisma 1, Intrattenere 1
    \item[\textbf{Tiri Salvezza}]: +2 Riflessi, +1 Tempra
    \item[\textbf{Caratteristica}]: Carisma o Destrezza
\end{description}

La \textbf{prima volta} che prendi questa Abilità quando usi Armi Aggraziate puoi sostituire il solo danno dato dalla Forza negli attacchi di mischia con metà del valore del Carisma o Destrezza.

La \textbf{seconda volta}, requisito Armi Aggraziate 4, Intrattenere 3, che prendi l'Abilità puoi usare il Carisma come modificatore al danno dell'arma, ignorando il danno dato dalla Forza.

La \textbf{terza volta}, requisito Armi Aggraziate 7, Intrattenere 5, che prendi l'Abilità puoi usare la Destrezza od il Carisma come modificatore al danno dell'arma, ignorando il danno dato dalla Forza.

Il secondo e terzo vantaggio non sono cumulativi.

\feat{Daredevil}
\begin{description}[noitemsep, topsep=0pt, parsep=0pt, partopsep=0pt, leftmargin=0cm, labelwidth=2.5cm]
    \item[\textbf{Requisito}]: Competenza Armi +2, Destrezza 1
    \item[\textbf{Tiri Salvezza}]: +2 Riflessi, +1 Tempra
    \item[\textbf{Caratteristica}]: Destrezza o Costituzione
\end{description}

Ti piace buttarti nella mischia, specialmente se si corrono pericoli! I bonus sono cumulativi.

La \textbf{prima volta} che prendi questa Abilità hai un +1 al Tiro per Colpire in mischia ed alla Difesa sei in mischia con 3 o più avversari.

La \textbf{seconda volta} che prendi questa Abilità hai un +2 al Tiro per Colpire in mischia ed alla Difesa sei in mischia con 2 o più avversari.

\feat{Dattilografo}
\begin{description}[noitemsep, topsep=0pt, parsep=0pt, partopsep=0pt, leftmargin=0cm, labelwidth=2.5cm]
    \item[\textbf{Requisito}]: Competenza Magica 1
    \item[\textbf{Tiri Salvezza}]: +1 Tempra, +1 Volontà
    \item[\textbf{Caratteristica}]: Modificatore di caratteristica per incantesimi o Destrezza
\end{description}

Sei estremamente rapido nel copiare nuovi incantesimi sul tuo Tomo della Magia. Il tempo per copiare un incantesimo passa da 1 ora a 30 minuti a pagina (un incantesimo occupa un numero di pagine pari al proprio livello). Il costo in inchiostri passa da 10 mo a pagina a 5 mo a pagina.

\feat{Decifrare scritti magici}
\begin{description}[noitemsep, topsep=0pt, parsep=0pt, partopsep=0pt, leftmargin=0cm, labelwidth=2.5cm]
    \item[\textbf{Requisito}]: Competenza Magica 1
    \item[\textbf{Tiri Salvezza}]: +1 Tempra, +1 Volontà
    \item[\textbf{Caratteristica}]: Modificatore di caratteristica per incantesimi o Saggezza
\end{description}

Ha un bonus di +1d6 nel comprendere il contenuto di una pergamena e nel lanciare l'incantesimo contenuto. Il bonus si applica anche alla prova per copiare un incantesimo sul proprio Tomo della magia.

\feat{Difendere Cavalcatura}
\begin{description}[noitemsep, topsep=0pt, parsep=0pt, partopsep=0pt, leftmargin=0cm, labelwidth=2.5cm]
    \item[\textbf{Requisito}]: Cavalcare 1
    \item[\textbf{Tiri Salvezza}]: +1 Tempra, +1 Riflessi
    \item[\textbf{Caratteristica}]: Destrezza o Saggezza
\end{description}

Ogni qual volta la cavalcatura viene colpita, puoi effettuare una prova di Cavalcare per negare il colpo.

La tua prova di Cavalcare deve essere maggiore del Tiro per Colpire dell'avversario

L'Abilità è utilizzabile solo una volta per round, per un solo attacco, costa la Reazione.

\feat{Difesa pronta}
\begin{description}[noitemsep, topsep=0pt, parsep=0pt, partopsep=0pt, leftmargin=0cm, labelwidth=2.5cm]
    \item[\textbf{Requisito}]: Competenza Armi 2
    \item[\textbf{Tiri Salvezza}]: +2 Riflessi
    \item[\textbf{Caratteristica}]: Destrezza o Intelligenza
\end{description}

Sei sempre vigile ed attento quando rischi la vita.

Hai un +2 alla Difesa contro gli attacchi di opportunità, alle spalle, o da fiancheggiato.

\feat{Distillare pozioni}
\begin{description}[noitemsep, topsep=0pt, parsep=0pt, partopsep=0pt, leftmargin=0cm, labelwidth=2.5cm]
    \item[\textbf{Requisito}]: Competenza Magica 1
    \item[\textbf{Tiri Salvezza}]: +1 Tempra, +1 Volontà
    \item[\textbf{Caratteristica}]: Saggezza o Intelligenza
\end{description}

Sei più che competente nel distillare pozioni.

La \textbf{prima volta} che prendi questa Abilità acquisisci un bonus di +1d6 su Conoscenze Erboristeria, distillare creare pozioni e veleni naturali.

La \textbf{seconda volta} che prendi l'Abilità il tempo per preparare le pozioni/veleni viene dimezzato ed in caso di Fallimento Critico non ci si espone al prodotto. Dedicando un ora al giorno puoi creare una Pozione generica di Cura od una Indebolente con le erbe che trovi li intorno. Questa pozione \emph{scade} all'alba del giorno dopo la creazione.

\feat{Doppia porzione}
\begin{description}[noitemsep, topsep=0pt, parsep=0pt, partopsep=0pt, leftmargin=0cm, labelwidth=2.5cm]
    \item[\textbf{Requisito}]: Combattimento con due armi, Competenza Armi 4
    \item[\textbf{Tiri Salvezza}]: +2 Tempra, +1 Riflessi
    \item[\textbf{Caratteristica}]: Forza o Costituzione
\end{description}

Il costante allenamento con due armi ti permette di applicare il bonus al danno dovuto alla Forza in maniera piena anche all'arma secondaria.

\feat{Duro a morire}
\begin{description}[noitemsep, topsep=0pt, parsep=0pt, partopsep=0pt, leftmargin=0cm, labelwidth=2.5cm]
    \item[\textbf{Requisito}]: -
    \item[\textbf{Tiri Salvezza}]: +1 Tempra, +1 Volontà
    \item[\textbf{Caratteristica}]: Costituzione o Saggezza
\end{description}

Sei particolarmente ostinato nel non volere morire. Il personaggio aumenta di 3 Punti Ferita la tolleranza prima di morire, ovvero muore a 13+COS*2.

\feat{Energia Psichica}
\begin{description}[noitemsep, topsep=0pt, parsep=0pt, partopsep=0pt, leftmargin=0cm, labelwidth=2.5cm]
    \item[\textbf{Requisito}]: Forza 1, Saggezza 2, Competenza Armi 1, Competenza Magica 1
    \item[\textbf{Tiri Salvezza}]: +2 Volontà, +1 Tempra
    \item[\textbf{Caratteristica}]: Saggezza o Carisma
\end{description}

Dopo anni di allenamento, meditazione e stage a Panda Barbat sei in grado di raccogliere la tua Energia Chi.

La \textbf{prima volta} che prendi questa Abilità ogni giorno, dopo almeno 6 ore di riposo e 2 ore di meditazione/allenamento, riempi il tuo corpo di energia Chi pari a Competenza Armi+Competenza Magica+Saggezza/2

La \textbf{seconda volta} che prendi questa Abilità, requisito Forza 1, Saggezza 2, Competenza Armi 4, Competenza Magica 4

Recuperi 1 punto Chi ogni 10 minuti in cui il personaggio non effettua attività impegnative.

\feat{Colpo Psichico}
\begin{description}[noitemsep, topsep=0pt, parsep=0pt, partopsep=0pt, leftmargin=0cm, labelwidth=2.5cm]
    \item[\textbf{Requisito}]: Energia Psichica, Destrezza 1
    \item[\textbf{Tiri Salvezza}]: +2 Volontà, +1 Tempra
    \item[\textbf{Caratteristica}]: Saggezza o Forza
\end{description}

La \textbf{prima volta} che prendi questa Abilità concentri il tuo Chi nelle tue mani. Puoi concentrare un numero di punti Chi pari alla Saggezza.

Con un Attacco a Tocco andato a segno, nel round scarichi l'energia che causa 1d6 danni da forza per punto Chi usato, fino ad un massimo di punti Chi pari al punteggio di Saggezza.

Il colpo si considera come portato da un arma magica con un bonus pari ai punti Chi usati.

La \textbf{seconda volta} che prendi questa Abilità, requisito Colpo Psichico, Saggezza 3, Competenza Armi 2, se il Tiro per Colpire va a segno consumi un punto Chi in meno.

La \textbf{terza volta} che prendi questa Abilità, Competenza Armi 3, se il Tiro per Colpire va a segno consumi due punti Chi in meno. Puoi usare un numero di punti Chi contemporaneo pari ad una volta e mezza il valore della Saggezza.

La \textbf{quarta volta} che prendi questa Abilità, Competenza Armi 7, Saggezza 4, se il Tiro per Colpire va a segno consumi tre punti Chi in meno. Puoi usare un numero di punti Chi contemporaneo pari al doppio del valore della Saggezza.

\feat{Raggio Psichico}
\begin{description}[noitemsep, topsep=0pt, parsep=0pt, partopsep=0pt, leftmargin=0cm, labelwidth=2.5cm]
    \item[\textbf{Requisito}]: Colpo Psichico, Saggezza 3, Competenza Armi 5
    \item[\textbf{Tiri Salvezza}]: +2 Riflessi, +1 Volontà
    \item[\textbf{Caratteristica}]: Saggezza o Destrezza
\end{description}

La \textbf{prima volta} che prendi questa Abilità puoi effettuare un attacco a distanza entro 9 metri usando l'Energia Psichica. L'Attacco, a Tocco, causa 1d6 di danno da forza per punto Psichico speso focalizzato sul danno.

E' possibile focalizzare uno o più punti Psichici per aumentare la distanza ogni volta di 9 metri. Non puoi usare un numero di punti Chi totali (per distanza e e danno) superiore alla Saggezza.

La \textbf{seconda volta} che prendi questa Abilità requisito Saggezza 3, Competenza Armi 9, puoi utilizzare fino a doppio del tuo punteggio in Saggezza per potenziare il Raggio Psichico.

\feat{Elementalista}
\begin{description}[noitemsep, topsep=0pt, parsep=0pt, partopsep=0pt, leftmargin=0cm, labelwidth=2.5cm]
    \item[\textbf{Requisito}]: Competenza Magia 3, Almeno 4 incantesimi da Lista di Magia Elementale
    \item[\textbf{Tiri Salvezza}]: +1 Volontà, +1 Tempra
    \item[\textbf{Caratteristica}]: Modificatore di caratteristica per incantesimi o Costituzione
\end{description}

La \textbf{prima volta} che prendi questa Abilità scegli un tipo di Energia Elementale: Fuoco, Elettricità, Freddo, Suono.

Sei capace di scambiare gli elementi presenti nei tuoi incantesimi. Puoi sostituire un tipo di danno di energia elementale con il danno scelto quando ha preso questa Abilità.
Il tempo di lancio dell'incantesimo aumenta di 1 Azione, se il tempo totale di lancio supera le 3 Azioni non è possibile usare questa Abilità sull'incantesimo.

La \textbf{seconda volta} che prendi questa Abilità, requisito Competenza Magica 6, scegli un nuovo tipo di Energia. Quando effettui la sostituzione puoi scegliere tra i tipi di energia a disposizione. Non hai più la penalità al tempo di lancio dell'incantesimo.

\feat{Esperto}
\begin{description}[noitemsep, topsep=0pt, parsep=0pt, partopsep=0pt, leftmargin=0cm, labelwidth=2.5cm]
    \item[\textbf{Requisito}]: Caratteristica collegata almeno a -1
    \item[\textbf{Tiri Salvezza}]: +1 a due Tiri Salvezza a scelta.
    \item[\textbf{Caratteristica}]: a scelta
\end{description}

Sei un esperto in un argomento.

La \textbf{prima volta} che prendi questa Abilità guadagni un +1 alle prove su una Competenza base a tua scelta.

La \textbf{seconda volta} che prendi questa Abilità aggiungi +2 alla prova. Puoi prendere 10 alla prova impiegando 5 round anziché 10 (vedi pag. \pageref{prendere10}).

La \textbf{terza volta} che prendi questa Abilità aggiungi 1d6 alla prova. Puoi prendere 14 alla prova impiegando 5 minuti anziché 10.

La \textbf{quarta volta} che prendi questa Abilità consideri il totale dei dadi tirati come 10 se ha tirato da 4-9.

Il bonus sono cumulativi se riferiti sempre la stessa Competenza.

Non è usabile su Consapevolezza (vedi \hyperlink{Percettivo}{Percettivo}, pag. \pageref{Percettivo}).

\feat{Estrazione rapida}
\begin{description}[noitemsep, topsep=0pt, parsep=0pt, partopsep=0pt, leftmargin=0cm, labelwidth=2.5cm]
    \item[\textbf{Requisito}]: Competenza Armi 1
    \item[\textbf{Tiri Salvezza}]: +1 Riflessi, +1 Volontà
    \item[\textbf{Caratteristica}]: Destrezza o Intelligenza
\end{description}

La \textbf{prima volta} che prendi questa Abilità puoi estrarre un arma per te non grande con il costo di una Azione immediata.

La \textbf{seconda volta} che prendi questa Abilità puoi riporre l'arma attuale ed estrarne un'altra come Azione di Movimento.

La \textbf{terza volta} che prendi questa Abilità puoi riporre l'arma attuale ed estrarne un'altra come Azione Immediata.

\feat{Fare Infuriare}
\begin{description}[noitemsep, topsep=0pt, parsep=0pt, partopsep=0pt, leftmargin=0cm, labelwidth=2.5cm]
    \item[\textbf{Requisito}]: Competenza Armi 2 e Carisma o Forza 2
    \item[\textbf{Tiri Salvezza}]: +2 Volontà, +1 Tempra
    \item[\textbf{Caratteristica}]: Forza o Carisma
\end{description}

Impieghi 2 Azioni ad infamare ed inveire contro un avversario.

Il target deve fare una prova contrapposta di Tiro Salvezza Volontà contro la tua prova di competenza Intrattenere od Intimidire oppure perdere il bonus di Destrezza (Tiri Salvezza, Tiro Colpire e Difesa) fino alla fine del tuo round successivo.

L'avversario può non comprendere la tua lingua ma deve avere Intelligenza pari a -2 o più.

\feat{Fedele}
\begin{description}[noitemsep, topsep=0pt, parsep=0pt, partopsep=0pt, leftmargin=0cm, labelwidth=2.5cm]
    \item[\textbf{Requisito}]: Competenza Magica 1, Somma valore Tratti in comune 2, essere Devoto
    \item[\textbf{Tiri Salvezza}]: +2 Volontà, +1 Tempra
    \item[\textbf{Caratteristica}]: Modificatore di caratteristica per incantesimi o Saggezza
\end{description}

La tua connessione con il Patrono è forte ed energetica. Aumenti i tuoi Punti Magia di 3 punti.

L'Abilità può essere presa più volte ed il totale deve essere pari o inferiore alla somma dei Tratti comune con il Patrono/3.

Questa Abilità non si cumula con l'Abilità Batteria Magica.

\feat{Ferocia}
\begin{description}[noitemsep, topsep=0pt, parsep=0pt, partopsep=0pt, leftmargin=0cm, labelwidth=2.5cm]
    \item[\textbf{Requisito}]: Competenza Armi 1
    \item[\textbf{Tiri Salvezza}]: +2 Tempra, +1 Volontà
    \item[\textbf{Caratteristica}]: Costituzione o Forza
\end{description}

La tua rabbia è tale da sconfiggere, temporaneamente, la morte.

La \textbf{prima volta} che prendi questa Abilità quando scendi sotto lo 0 Punti Ferita non svieni ed incominci a perdere 1 punto ferita a round.

Una creatura dotata di Ferocia sviene quando ha un punteggio di Punti Ferita negativo pari al doppio dei punti di Costituzione e muore quando i suoi Punti Ferita scendono al punteggio negativo pari al suo triplo del punteggio di Costituzione+5 (COS*3+5)

La \textbf{seconda volta} che prendi questa Abilità, requisito Competenza Armi 4, puoi fare che la tua Forza aumenti di 2 e acquisisci 6 Punti Ferita temporanei per 1 minuto. A fine scontro il tuo livello di affaticamento aumenta di 1 per 10 minuti.

La \textbf{terza volta} che prendi questa Abilità, requisito Competenza Armi 7, puoi fare che la tua Forza aumenti di 3 e acquisisci 12 Punti Ferita temporanei per 3 minuti. A fine scontro il tuo livello di affaticamento aumenta di 2 per 20 minuti.

La \textbf{quarta volta} che prendi questa Abilità, requisito Competenza Armi 11, puoi fare che la tua Forza aumenti di 4 e acquisisci 24 Punti Ferita temporanei per 15 minuti. A fine scontro il tuo livello di affaticamento aumenta di 3 per 30 minuti.

Il giocatore può scegliere un solo grado di Ferocia da usare nello scontro (2, 3, 4).

\feat{Figlia di Shayalia}
\begin{description}[noitemsep, topsep=0pt, parsep=0pt, partopsep=0pt, leftmargin=0cm, labelwidth=2.5cm]
    \item[\textbf{Requisito}]: Devoto o Seguace di Shayalia
    \item[\textbf{Tiri Salvezza}]: +1 Tempra, +2 Volontà
    \item[\textbf{Caratteristica}]: Modificatore di caratteristica per incantesimi o Carisma
\end{description}

Hai una profonda ed istintiva connessione con il mondo naturale.

La \textbf{prima volta} che prendi questa Abilità ottieni un +2 alla prove di Natura ed un +2 ai Tiri Salvezza contro veleni naturali.

La \textbf{seconda volta} che prendi questa Abilità, requisito somma Tratti in comune 6, ottieni un +4 alla prove di Natura ed un +4 ai Tiri Salvezza contro effetti, anche magici, causati da Animali o Piante.

La \textbf{terza volta} che prendi questa Abilità, requisito somma Tratti in comune 12, sei sempre sotto l'effetto dell'incantesimo Santuario verso qualsiasi animale non magico.

La \textbf{quarta volta} che prendi questa Abilità, requisito Animalia preso 4 volte, puoi trasformati usando Animalia, in qualsiasi creatura purché non sia un immondo o drago.

\feat{Figlio di Tazher}
\begin{description}[noitemsep, topsep=0pt, parsep=0pt, partopsep=0pt, leftmargin=0cm, labelwidth=2.5cm]
    \item[\textbf{Requisito}]: Devoto o Seguace di Tazher, somma tratti comuni 10
    \item[\textbf{Tiri Salvezza}]: +2 Riflessi, +1 Volontà
    \item[\textbf{Caratteristica}]: Destrezza o Saggezza
\end{description}

Quando prendi questa Abilità la tua ombra diventa l'equivalente di un \hyperlink{Servitore Invisibile}{Servitore Invisibile}.
L'ombra ha Difesa pari a 10 + valore Tratto in comune con Tazher a più alto punteggio, i Tiri Salvezza sono pari ai tuoi, i Punti Ferita sono pari alla somma dei tratti in comune con Tazher.

Usando 2 Azioni puoi evocarla nuovamente se dovesse essere uccisa.

La tua ombra non può allontanarsi da te più della somma dei tratti in comune con Tazher in metri. Può essere usata per \hyperlink{Condividere Incantesimi}{Condividere Incantesimi}, \hyperlink{Trasmettere Incantesimi a Contatto}{Trasmettere Incantesimi a Contatto} (ma usi la tua capacità di attacco), vedi anche \hyperlink{famiglio}{Famiglio} (pag. \pageref{famiglio}).

Può essere usato come punto di lancio degli incantesimi.

Non puoi avere Famigli. Non puoi interagire con la tua Ombra se non ci sono le condizione per esserci un ombra.

\feat{Figlio Unico}
\begin{description}[noitemsep, topsep=0pt, parsep=0pt, partopsep=0pt, leftmargin=0cm, labelwidth=2.5cm]
    \item[\textbf{Requisito}]: Costituzione 0
    \item[\textbf{Tiri Salvezza}]: +1 Tempra, +1 Volontà
    \item[\textbf{Caratteristica}]: Modificatore di caratteristica per incantesimi o Saggezza
\end{description}

La \textbf{prima volta} che prendi questa Abilità scegli una Lista Magica e due Trucchetti da questa lista. Puoi lanciare questi due Trucchetti senza Prova di Magia anche se sei essere Distratto o con armatura.

Le volte successive che prendi questa Abilità puoi individuare un Patrono che abbia nelle Liste Privilegiate la Lista di Magia che hai individuato precedentemente.

La \textbf{seconda volta} che prendi questa Abilità individua altri due Trucchetti, o se sei un Seguace o Devoto scegli un incantesimo di primo livello dalla Lista Privilegiata del Patrono. Requisito Tratto in comune a punteggio 3.

La \textbf{terza volta} che prendi questa Abilità individua altri due Trucchetti, o se sei un Devoto scegli un incantesimo di secondo livello od inferiore dalla Lista Privilegiata del Patrono. Requisito Tratto in comune a punteggio 5.

La \textbf{quarta volta} che prendi questa Abilità individua altri due Trucchetti, o se sei un Devoto scegli un incantesimo di terzo livello od inferiore dalla Lista Privilegiata del Patrono. Requisito Tratto in comune a punteggio 9.

Le Abilità 2, 3, 4 possono essere prese più volte. Gli incantesimi di primo livello e successivo si lanciano pagando Punti Magia altrimenti si possono lanciare solo una volta al giorno per volta che si è preso l'Abilità 4.

\feat{Finta Morte}
\begin{description}[noitemsep, topsep=0pt, parsep=0pt, partopsep=0pt, leftmargin=0cm, labelwidth=2.5cm]
    \item[\textbf{Requisito}]: Costituzione 0
    \item[\textbf{Tiri Salvezza}]: +1 Volontà, +2 Tempra
    \item[\textbf{Caratteristica}]: Costituzione o Saggezza
\end{description}

Come Azione di Reazione sei in grado di cadere a terra (stramazzare!) morto. Solo una prova di Pronto Soccorso DC 20 può rivelare che sei vivo.

L'effetto dura al massimo 2 minuti. La finta morte non è ripetibile in intervalli inferiori ai 10 minuti l'una dall'altra.

\feat{Flagello Danzante}
\begin{description}[noitemsep, topsep=0pt, parsep=0pt, partopsep=0pt, leftmargin=0cm, labelwidth=2.5cm]
    \item[\textbf{Requisito}]: Competenza Armi 1, usare un arma della Lista Palle rotanti
    \item[\textbf{Tiri Salvezza}]: +1 Tempra, +1 Volontà
    \item[\textbf{Caratteristica}]: Forza o Carisma
\end{description}

Quando usi la tua arma della lista Palle Rotanti hai un bonus di +1 al Tiro per Colpire e +1 alla Difesa

\feat{Forgiato nella furia}
\begin{description}[noitemsep, topsep=0pt, parsep=0pt, partopsep=0pt, leftmargin=0cm, labelwidth=2.5cm]
    \item[\textbf{Requisito}]: Competenza Armi 5
    \item[\textbf{Tiri Salvezza}]: +1 Tempra, +1 Riflessi
    \item[\textbf{Caratteristica}]: Forza o Costituzione
\end{description}

Quando effettui un tiro critico con un attacco in mischia considera di aver colpito con un margine ulteriore di +2, per la verifica di ulteriori critici.

\feat{Fortunato}
\begin{description}[noitemsep, topsep=0pt, parsep=0pt, partopsep=0pt, leftmargin=0cm, labelwidth=2.5cm]
    \item[\textbf{Requisito}]: nessuno
    \item[\textbf{Tiri Salvezza}]: +1 Tempra, +1 Riflessi
    \item[\textbf{Caratteristica}]: Carisma o Destrezza
\end{description}

Una volta al giorno puoi fare ritirare 1d6 di una prova (Tiri per Colpire, Prove Competenze, Tiri Salvezza) al Narratore e prendere il valore più basso tra i due tiri.

L'Abilità può essere dichiarata anche dopo il tiro dei dadi.

\feat{Forma Elementale}
\begin{description}[noitemsep, topsep=0pt, parsep=0pt, partopsep=0pt, leftmargin=0cm, labelwidth=2.5cm]
    \item[\textbf{Requisito}]: Seguace o Devoto di Erondil, Gaya, Efrem oppure Shayalia. Almeno 3 incantesimi da 2 Liste di Magia Elementare diverse, Competenza Magica 6
    \item[\textbf{Tiri Salvezza}]: +1 Tempra, +1 Volontà
    \item[\textbf{Caratteristica}]: Modificatore di caratteristica per incantesimi o Costituzione
\end{description}

Quando ti trasformi con l'Abilità Animalia scegli un elemento tra gli incantesimi imparati presenti in una Lista di Magia Elementale.

La \textbf{prima volta} che prendi questa Abilità quando ti trasformi con Animalia i tuoi attacchi in mischia fanno il tipo di danno elementale scelto.

La \textbf{seconda volta} che prendi questa Abilità, Competenza Magica 11, quando ti trasformi con Animalia sei resistente al medesimo tipo di danno elementale che causi.

La \textbf{terza volta} che prendi questa Abilità, requisito Competenza Magica 14, i tuoi attacchi quando ti trasformi con Animalia causano 2d6 di danno in più del tipo elementale scelto.

Se sei un Devoto o Seguace di Gaya o Erondil non è necessario trasformarsi in animale, il tipo di danno si applica ai tuoi attacchi in mischia.

\feat{Freccia chiamata, freccia consegnata}
\begin{description}[noitemsep, topsep=0pt, parsep=0pt, partopsep=0pt, leftmargin=0cm, labelwidth=2.5cm]
    \item[\textbf{Requisito}]: Competenza Armi 2
    \item[\textbf{Tiri Salvezza}]: +2 Riflessi
    \item[\textbf{Caratteristica}]: Destrezza o Intelligenza
\end{description}

E' questione di un attimo perché ti accenda!

Puoi tirare 1 freccia/dardo, una volta al giorno, come Reazione, senza penalità al colpire date dal multiattacco.

L'arco/balestra deve già essere in mano.

\feat{Furia}
\begin{description}[noitemsep, topsep=0pt, parsep=0pt, partopsep=0pt, leftmargin=0cm, labelwidth=2.5cm]
    \item[\textbf{Requisito}]: Competenza Armi 1
    \item[\textbf{Tiri Salvezza}]: +2 Tempra, +1 Volontà
    \item[\textbf{Caratteristica}]: Forza o Costituzione
\end{description}

Il tuo stile di combattimento è rappresentato dalla cieca furia omicida.

Aggiungi +1d6 al danno ad ogni attacco andato a segno in mischia ed i tuoi avversari guadagnano +1d6 al colpire verso di te.

Puoi decidere di attivare questa Abilità round per round. Costa 1 Azione Immediata e dura fino all'inizio del tuo round successivo.

\begin{center}
	\includegraphics[width=0.9\linewidth]{immagini/Early_Egyptian_juggling_art.png}

	\emph{This ancient wall painting appears to depict jugglers.}
\end{center}

\feat{Giocoliere}
\begin{description}[noitemsep, topsep=0pt, parsep=0pt, partopsep=0pt, leftmargin=0cm, labelwidth=2.5cm]
    \item[\textbf{Requisito}]: Destrezza 2
    \item[\textbf{Tiri Salvezza}]: +2 Riflessi
    \item[\textbf{Caratteristica}]: Destrezza o Carisma
\end{description}

Hai un talento naturale per maneggiare gli oggetti.

Qualsiasi prova di Acrobatica che coinvolga il maneggiare oggetti o l'equilibrio ha un +2 di Bonus.

Puoi lanciare un secondo pugnale come Azione Immediata in seguito all'Azione di attacco di lancio di un pugnale, questo pugnale ha un -3 al Tiro per Colpire e non cumula penalità al multiattacco.

\feat{Guerriero della Magia}
\begin{description}[noitemsep, topsep=0pt, parsep=0pt, partopsep=0pt, leftmargin=0cm, labelwidth=2.5cm]
    \item[\textbf{Requisito}]: Competenza Armi 2, Competenza Magica 2
    \item[\textbf{Tiri Salvezza}]: +1 Volontà, +1 Riflessi
    \item[\textbf{Caratteristica}]: Modificatore di caratteristica per incantesimi o Forza
\end{description}

Non segui solo la via della magie e neanche quella della spada, il tuo stile abbraccia entrambi in un fendente di pura magia.

La \textbf{prima volta} che prendi questa Abilità sei in grado di scaricare un incantesimo con distanza di mischia con la tua arma. Effettui il Tiro per Colpire e se colpisci oltre al danno dell'attacco scarichi anche l'incantesimo. Devi riuscire in una Prova di Magia. In questa maniera puoi eseguire un solo attacco con l'arma. Costa 3 Azioni.

La \textbf{seconda volta} che prendi questa Abilità, requisito Competenza Armi 3, Competenza Magica 3, consumando 3 Azioni sei in grado di scaricare un incantesimo che non sia personale o a tocco con un arma a gittata. Devi riuscire in una Prova di Magia.

La \textbf{terza volta} che prendi questa Abilità, requisito Competenza Armi 6, Competenza Magica 9, puoi convogliare incantesimi fino al 3 livello tramite l'arma.

La \textbf{quarta volta} che prendi questa Abilità non è necessario più effettuare la Prova di Magia per scaricare l'incantesimo con l'arma.

Non puoi scaricare incantesimi di livello superiore a 3 con questa Abilità ed il tempo di lancio dell'incantesimo non può essere superiore alle 2 Azioni.

\feat{Gru d'Argento}
\begin{description}[noitemsep, topsep=0pt, parsep=0pt, partopsep=0pt, leftmargin=0cm, labelwidth=2.5cm]
    \item[\textbf{Requisito}]: Lista Pugno Vuoto 2, Destrezza 1
    \item[\textbf{Tiri Salvezza}]: +2 Riflessi, +1 Volontà
    \item[\textbf{Caratteristica}]: Destrezza o Intelligenza
\end{description}

Per usufruire di questi bonus non devi portare armature o scudi od oggetti anche magici che migliorino la Difesa. Le capacità elencate non sono cumulabili con l'Abilità Armatura della Montagna Incantata.

La \textbf{prima volta} che prendi questa Abilità la tua Difesa naturale aumenta di 1 + 1/3 dei punti in Pugno Vuoto + Destrezza + eventuali modificatori.

La \textbf{seconda volta} che prendi questa Abilità, requisito Lista Pugno Vuoto 4, la tua Iniziativa aumenta di 2 (solo con attacchi disarmati).

La \textbf{terza volta} che prendi questa Abilità, requisito Lista Pugno Vuoto 9 e Destrezza 2, hai un bonus nei Tiri salvezza su Volontà 2 (cumulativo).

La \textbf{quarta volta} che prendi questa Abilità, requisito Lista Pugno Vuoto 11, la tua Difesa naturale ed Iniziativa aumentano di 2 (cumulativo).

La \textbf{quinta volta} che prendi questa Abilità, requisito Lista Pugno Vuoto 13 e Destrezza 3, hai un bonus nei Tiri salvezza su Riflessi di 2 (cumulativo).

I bonus sono attivi anche se non stai combattendo.

\feat{Guarigione accelerata}
\begin{description}[noitemsep, topsep=0pt, parsep=0pt, partopsep=0pt, leftmargin=0cm, labelwidth=2.5cm]
    \item[\textbf{Requisito}]: Costituzione 0
    \item[\textbf{Tiri Salvezza}]: +2 Tempra, +1 Volontà
    \item[\textbf{Caratteristica}]: Costituzione
\end{description}

I tuoi naturali processi curativi sono estremamente efficienti. Le capacità si cumulano.

La \textbf{prima volta} che prendi questa Abilità dopo una notte di riposo recuperi 1d6 Punti Ferita in più.

La \textbf{seconda volta}, requisito Costituzione 1,  che prendi questa Abilità alla fine di ogni tuo round diminuisci di 1 la condizione di Sanguinamento.

La \textbf{terza volta}, requisito Costituzione 2, che prendi questa Abilità dopo una notte di riposo recuperi il doppio dei Punti Ferita.

\feat{Guaritore}
\begin{description}[noitemsep, topsep=0pt, parsep=0pt, partopsep=0pt, leftmargin=0cm, labelwidth=2.5cm]
    \item[\textbf{Requisito}]: Saggezza 1
    \item[\textbf{Tiri Salvezza}]: +2 Volontà, +1 Tempra
    \item[\textbf{Caratteristica}]: Saggezza
\end{description}

Hai un naturale talento nel curare le persone.

La \textbf{prima volta} che prendi questa Abilità hai +4 alle prove di Pronto soccorso.

La \textbf{seconda volta} che prendi questa Abilità, requisito somma tratti comuni con il Patrono Ledyal 8, ogni volta che usi un incantesimo di cura tu recuperi 1 Punte Ferita e la creatura curata +1d6 Punti Ferita in più.

\feat{Ho detto CADI!}
\begin{description}[noitemsep, topsep=0pt, parsep=0pt, partopsep=0pt, leftmargin=0cm, labelwidth=2.5cm]
    \item[\textbf{Requisito}]: Competenza Armi 4
    \item[\textbf{Tiri Salvezza}]: +2 Tempra, +1 Volontà
    \item[\textbf{Caratteristica}]: Forza o Costituzione
\end{description}

Se colpisci 3 volte entro 2 round un avversario questo deve fare una Tiro Salvezza su Tempra con DC pari al Tiro per Colpire dell'ultimo attacco o cadere prono. Il Tiro Salvezza ha 1d6 di modificatore per taglia di differenza.

\feat{Il Patrono e' con me}
\begin{description}[noitemsep, topsep=0pt, parsep=0pt, partopsep=0pt, leftmargin=0cm, labelwidth=2.5cm]
    \item[\textbf{Requisito}]: Devoto, Somma Tratti comuni con il Patrono 2
    \item[\textbf{Tiri Salvezza}]: +1 Volontà, +1 Riflessi
    \item[\textbf{Caratteristica}]: Modificatore di caratteristica per incantesimi o a scelta
\end{description}

La \textbf{prima volta} che prendi questa Abilità per 1 volta al giorno puoi ritirare un dado tirato nella Prova di Magia.

La \textbf{seconda volta} che prendi questa Abilità, requisito somma Tratti comuni con il Patrono 6, per 2 volte al giorno puoi ritirare fino a 2 dadi tirati nella Prova di Magia per il lancio di incantesimo.

La \textbf{terza volta} che prendi questa Abilità, requisito somma Tratti comuni con il Patrono 12, per 3 volte al giorno puoi ritirare fino a 3 dadi tirati nella Prova di Magia per il lancio di incantesimo.

L'Abilità può essere dichiarata anche dopo il lancio dei dadi. Qualsiasi nuovo valore ottenuto con il nuovo tiro va tenuto o si usa nuovamente questa Abilità.

\begin{center}
	\includegraphics[width=0.56\linewidth]{immagini/streghegoya.png}

	\emph{Sabba delle Streghe (Goya, 1798)}
\end{center}

\feat{Il Patrono è la mia Arma}
\begin{description}[noitemsep, topsep=0pt, parsep=0pt, partopsep=0pt, leftmargin=0cm, labelwidth=2.5cm]
    \item[\textbf{Requisito}]: Somma Tratti comuni con il Patrono 1, essere Seguaci
    \item[\textbf{Tiri Salvezza}]: +1 Volontà, +1 Riflessi
    \item[\textbf{Caratteristica}]: Modificatore di caratteristica per incantesimi o a scelta
\end{description}

Non hai penalità al colpire con l'arma del Patrono se non sei competente nella sua Lista d'Armi.

La \textbf{prima volta} che prendi questa Abilità hai un +1 di al Tiro per Colpire ed al Danno quando usi l'Arma preferita del tuo Patrono.

La \textbf{seconda volta} che prendi questa Abilità, requisito somma Tratti comune 5, Competenza Armi 1, la penalità per gli attacchi multipli con l'arma preferita del Patrono diviene -4.

La \textbf{terza volta} che prendi questa Abilità, requisito somma Tratti comuni con il Patrono 10, Competenza Armi 4, essere Devoti, puoi usare il tuo Modificatore di caratteristica per incantesimi per stabilire il Tiro per Colpire con l'arma del Patrono, al posto di Forza o Destrezza.

La \textbf{quarta volta} che prendi questa Abilità, requisito somma Tratti comuni con il Patrono 10, Competenza Armi 5, aggiungi +1d6 al Tiro per Colpire quando effettui il terzo attacco nel round con l'arma del Patrono.

La \textbf{quinta volta} che prendi questa Abilità, requisito somma Tratti comuni con il Patrono 13, Competenza Armi 6, aumenti di un grado il dado di danno della tua arma del Patrono.

La \textbf{sesta volta} che prendi questa Abilità, requisito somma Tratti comuni con il Patrono 16, ottieni un ulteriore +1 al Tiro per Colpire e +1 al Danno. Il primo attacco andato a segno nel round con l'arma del Patrono causa sempre danno critico.

\feat{Iaijutsu}
\begin{description}[noitemsep, topsep=0pt, parsep=0pt, partopsep=0pt, leftmargin=0cm, labelwidth=2.5cm]
    \item[\textbf{Requisito}]: Competenza Armi 2
    \item[\textbf{Tiri Salvezza}]: +2 Riflessi, +1 Volontà
    \item[\textbf{Caratteristica}]: Intelligenza o Destrezza
\end{description}

La \textbf{prima volta} che prendi questa Abilità fai un passo di un metro, attacchi una volta e torni dove eri prima, il tutto in meno di un battito di ciglia.

La \textbf{seconda volta}, requisito CA 6, che prendi questa Abilità puoi muoverti di 2 metri prima di attaccare e tornare dove eri.

La \textbf{terza volta}, requisito CA 12, che prendi questa Abilità puoi muoverti del tuo movimento prima di attaccare e tornare dove eri, tratti il terreno come difficile.

Consumi due Azioni.

\feat{Improvvisare}
\begin{description}[noitemsep, topsep=0pt, parsep=0pt, partopsep=0pt, leftmargin=0cm, labelwidth=2.5cm]
    \item[\textbf{Requisito}]: Competenza Armi 1
    \item[\textbf{Tiri Salvezza}]: +1 Tempra, +1 Riflessi
    \item[\textbf{Caratteristica}]: Destrezza o Saggezza
\end{description}

Qualsiasi oggetto che possa definirsi un arma improvvisata per te non è improvvisata.
Non soffri di penalità al colpire quando usi un \hyperlink{armaimprovvisata}{arma improvvisata}. Ottieni un +1 al danno quando usi un arma improvvisata.

\feat{Incantatore da Combattimento}
\begin{description}[noitemsep, topsep=0pt, parsep=0pt, partopsep=0pt, leftmargin=0cm, labelwidth=2.5cm]
    \item[\textbf{Requisito}]: Competenza Magica 1
    \item[\textbf{Tiri Salvezza}]: +1 Tempra, +1 Volontà
    \item[\textbf{Caratteristica}]: Modificatore di caratteristica per incantesimi o a scelta
\end{description}

La \textbf{prima volta} che prendi questa Abilità quando sei Distratto puoi ignorare un dado nella Prova di Magia.

La \textbf{seconda volta}, requisito Competenza Magica 6, che prendi questa Abilità, quando sei Distratto, puoi ignorare un ulteriore dado nella Prova di Magia.

La \textbf{terza volta}, requisito Competenza Magica 12, che prendi questa Abilità, quando sei Distratto, puoi ignorare un ulteriore dado nella Prova di Magia.

Questa Abilità si può usare nella Prova di Magia richiesta da \hyperlink{guerrierodellamagia}{Guerriero della Magia}. Le capacità indicate si cumulano.

\feat{Incantatore Prudente}
\begin{description}[noitemsep, topsep=0pt, parsep=0pt, partopsep=0pt, leftmargin=0cm, labelwidth=2.5cm]
    \item[\textbf{Requisito}]: Competenza Magica 8
    \item[\textbf{Tiri Salvezza}]: +2 Riflessi, +1 Tempra
    \item[\textbf{Caratteristica}]: Modificatore di caratteristica per incantesimi o a scelta
\end{description}

Quando una creatura ostile entra per la prima volta in uno spazio entro 1 metro da te puoi usare una Reazione per lanciare un trucchetto, entro 2 Azioni, senza potenziamenti o Prova di Magia.

Questa Abilità non influisce su fatto che si è comunque Distratti nel lancio di un successivo incantesimo.

\feat{Immunita' ai veleni}
\begin{description}[noitemsep, topsep=0pt, parsep=0pt, partopsep=0pt, leftmargin=0cm, labelwidth=2.5cm]
    \item[\textbf{Requisito}]: Costituzione 1
    \item[\textbf{Tiri Salvezza}]: +2 Tempra, +1 Volontà
    \item[\textbf{Caratteristica}]: Costituzione o Saggezza
\end{description}

\index{Mitridatismo}

La \textbf{prima volta} che prendi questa Abilità il corpo si abitua ai veleni, il personaggio guadagna un +2 Tiro Salvezza sui veleni.

La \textbf{seconda volta} che prendi l'Abilità divieni immune ai veleni naturali. Non riesci più ad ubriacarti normalmente.

La \textbf{terza volta} hai un +1d6 ai Tiro Salvezza ai veleni magici e subire gli effetti di fumi tossici (ma puoi sempre soffocare).

\medskip


\feat{Imposizione delle mani}
\begin{description}[noitemsep, topsep=0pt, parsep=0pt, partopsep=0pt, leftmargin=0cm, labelwidth=2.5cm]
    \item[\textbf{Requisito}]: Competenza Magica 1, Tratti comuni 3, essere Devoto o Seguace
    \item[\textbf{Tiri Salvezza}]: +2 Volontà, +1 Tempra
    \item[\textbf{Caratteristica}]: Carisma o Saggezza
\end{description}

Se i tuoi Tratti sono in comune con un Patrono positivo puoi convogliare energia curativa (effetto curativo/dannoso su non morti), se sono in comune con un Patrono neutrale o malvagio puoi convogliare energia negativa (effetto dannoso/curativo sui non morti).

Usabile un numero di volte al giorno pari alla (somma dei Tratti in comune con il Patrono)/2.

La \textbf{prima volta} che prendi questa Abilità attraverso l'imposizione delle mani puoi curare/ferire 5 Punti Ferita ad una creatura. Puoi applicare più usi con il singolo tocco.

L'Abilità presa più volte permette di togliere specifiche condizioni che affliggono la creatura facendo consumare più usi.

La \textbf{seconda volta} che prendi questa Abilità, somma Tratti in comune 4, puoi togliere la condizione Affaticato (3 usi).

La \textbf{terza volta} che prendi questa Abilità, somma Tratti in comune 6, puoi togliere le condizioni: Spaventato, Nauseato, Svenuto, Affaticato 2 (3 usi).

La \textbf{quarta volta} che prendi questa Abilità, somma Tratti in comune 8, puoi togliere le condizioni: Accecato, Assordato, Avvelenato, Malato* (3 usi).

La \textbf{quinta volta} che prendi questa Abilità, somma Tratti in comune 10, puoi togliere le condizioni: Confuso, Paralizzato, Affascinato (4 usi).

La \textbf{sesta volta} che prendi questa Abilità, somma Tratti in comune 12, puoi togliere le condizioni: Dominato, Pietrificato, Affaticato 3 (5 usi).

\textbf{Altri usi}:

\smallskip

Ogni uso dell'Abilità Imposizione delle mani per curare riduce il valore di Sanguinamento di 2 e permette di recuperare 3 Punti Ferita Massimi.

In caso di malattie magiche una eventuale prova di contrasto è fatta con 3d6 + somma punti Tratto in comune + Saggezza.

L'energia proviene dalle mani (non conta se ci sono guanti) e si applica con un Attacco a Tocco.

Tiro Salvezza su Tempra DC 10 + somma Tratti in comune con il Patrono + Saggezza per evitare l'effetto. 2 Azioni.


\begin{center}
	\includegraphics[width=0.75\linewidth]{immagini/Portrait_of_V_Greatrakesv2.png}

	\emph{Portrait of V. Greatrakes laying on his hands, window, in right-hand corner showing several successful cures, possibly. By W. Faithorne }
\end{center}

\feat{Incanalare Energia}
\begin{description}[noitemsep, topsep=0pt, parsep=0pt, partopsep=0pt, leftmargin=0cm, labelwidth=2.5cm]
    \item[\textbf{Requisito}]: Imposizione delle mani, Competenza Magica 1, Tratti comuni 4
    \item[\textbf{Tiri Salvezza}]: +2 Volontà, +1 Tempra
    \item[\textbf{Caratteristica}]: Saggezza o Costituzione
\end{description}

Sei in grado di usare l'energia di Imposizione delle Mani per creare un'aura energetica intorno a te.
Tramite l'Imposizione delle mani crei un'aura istantanea nel raggio di 3 metri attorno a te che cura o ferisce 5 Punti Ferita a tutte le creature presenti ogni 2 usi consumati.

Ogni volta che prendi questa Abilità, oltre la prima, aumenti il raggio di 1 metro e puoi escludere una creatura dall'effetto dell'aura.

L'energia proviene dal tuo corpo ed influenza te stesso e le creature intorno a te. Tiro Salvezza su Riflessi DC 10 + somma Tratti in comune con il Patrono + Saggezza per evitare l'effetto. \index{Incanalare energia su non morti} 2 Azioni.

%\begin{center}
%\includegraphics[width=0.8\linewidth]{immagini/kameame.png}
%
%\emph{Kamehameha!}
%\end{center}

\feat{Infondere Coraggio}
\begin{description}[noitemsep, topsep=0pt, parsep=0pt, partopsep=0pt, leftmargin=0cm, labelwidth=2.5cm]
    \item[\textbf{Requisito}]: Carisma 2, Intrattenere 1
    \item[\textbf{Tiri Salvezza}]: +2 Volontà, +1 Tempra
    \item[\textbf{Caratteristica}]: Carisma o Saggezza
\end{description}

Tramite la tua esibizione, canora, di balletto, oratoria od artistica in generale, sei in grado di infondere coraggio nei compagni in grado di sentirti o vederti, nel raggio di 6 metri.

La \textbf{prima volta} che prendi questa Abilità i tuoi compagni hanno un bonus di +1 al Tiro per Colpire ed al Danno in combattimento.

La \textbf{seconda volta} che prendi questa Abilità, requisito Intrattenere 4, puoi decidere di infondere fino a 2 di questi bonus. +2 Tiro per Colpire, +2 Difesa, +2 Danno, +2 Tiro Salvezza su Volontà. I tuoi compagni devono essere entro 12 metri di raggio.

La \textbf{terza volta} che prendi questa Abilità, requisito Intrattenere 12, puoi decidere di infondere fino a 2 di questi bonus. +1d6 Tiro per Colpire, +4 Difesa, +4 Danno, +1d6 TS. I tuoi compagni devono essere entro 24 metri di raggio.

Attivare, mantenere o cambiare effetto dell'Abilità richiede 2 Azioni.

Puoi mantenere l'Abilità un numero di round, anche non consecutivi, pari a punteggio Intrattenere x 3 al giorno.

Le creature per rimanere influenzate devono continuare a vedere/sentire la tua esibizione.

\feat{Infondere Energia Magica}
\begin{description}[noitemsep, topsep=0pt, parsep=0pt, partopsep=0pt, leftmargin=0cm, labelwidth=2.5cm]
    \item[\textbf{Requisito}]: Competenza Armi 1, Competenza Magica 2
    \item[\textbf{Tiri Salvezza}]: +1 Riflessi, +1 Tempra
    \item[\textbf{Caratteristica}]: Modificatore di caratteristica per incantesimi o a scelta
\end{description}

Sai manipolare le energie magiche in maniera istintiva ed infonderle nelle armi. Costa 1 Azione infondere la magia nell'arma.

La \textbf{prima volta} che prendi questa Abilità puoi usare due Punti Magia e canalizzarli nella tua arma.

Per la durata di 6 round la tua arma diviene un arma magica +1, se possiede già capacità magiche l'effetto non funziona.

La \textbf{seconda volta} che prendi questa Abilità, requisito Competenza Magica 4, puoi usare quattro Punti Magia ed un arma con cui vieni a contatto diventa un arma +2 per 6 round, se è già incantata acquisisce un bonus ulteriore di +1 fino ad un massimo di +3.

La \textbf{terza volta} che prendi questa Abilità, requisito Competenza Magica 8, puoi usare sei Punti Magia ed un arma con cui vieni a contatto diventa un arma +3 per 6 round, se è già incantata acquisisce un bonus ulteriore di +2 fino ad un massimo di +4.

\feat{Infondere Energia Magica Superiore}
\begin{description}[noitemsep, topsep=0pt, parsep=0pt, partopsep=0pt, leftmargin=0cm, labelwidth=2.5cm]
    \item[\textbf{Requisito}]: Competenza Armi 4, Competenza Magica 6
    \item[\textbf{Tiri Salvezza}]: +1 Riflessi, +1 Tempra
    \item[\textbf{Caratteristica}]: Modificatore di caratteristica per incantesimi o a scelta
\end{description}

Sai infondere l'arma di energia magica per fargli acquisire capacità fantastiche.

Costa 1 Azione attivare l'infusione della magia nell'arma. L'arma deve essere magica.

La \textbf{prima volta} che prendi questa Abilità usando un Punto Magia a round puoi rendere la tua arma fiammeggiante o elettrificata o cambiare la forma.

Ogni colpo portato a segna causa 1d6 di danni da fuoco o elettricità aggiuntivi. Se smetti di pagare il Punto Magia torna alla forma precedente e smette di causare danni aggiuntivi.

La \textbf{seconda volta} che prendi questa Abilità usando due Punti Magia a round puoi rendere un arma con cui vieni a contatto estremamente pericolosa. Ogni colpo portato a segno causa 1 danno critico aggiuntivo. Requisito Competenza Magica 7.

La \textbf{terza volta} che prendi questa Abilità usando tre Punti Magia a round puoi concedere ad un arma con cui vieni a contatto entrambi le Abilità precedenti.

Le Abilità non sono cumulative, devi scegliere quale applicare round per round.

\feat{Infondere Paura}
\begin{description}[noitemsep, topsep=0pt, parsep=0pt, partopsep=0pt, leftmargin=0cm, labelwidth=2.5cm]
    \item[\textbf{Requisito}]: Carisma 2
    \item[\textbf{Tiri Salvezza}]: +2 Volontà, +1 Tempra
    \item[\textbf{Caratteristica}]: Carisma o Costituzione
\end{description}

Tramite la tua esibizione, canora, di balletto, oratoria.. sei in grado di infondere paura negli avversari in grado di sentirti, nel raggio di 6 metri.

La \textbf{prima volta} che prendi questa Abilità i tuoi nemici hanno penalità di -1 al Tiro per Colpire ed al Danno in combattimento.

La \textbf{seconda volta} che prendi questa Abilità, requisito Intrattenere 4, la forza della tua arte aggredisce i nemici e puoi selezionare due effetti tra: -2 Tiro per Colpire, -2 Danno in combattimento, -2 Difesa, -2 al Tiri Salvezza su Volontà. I tuoi nemici devono essere entro 12 metri di raggio.

La \textbf{terza volta} che prendi questa Abilità, requisito Intrattenere 12, la forza della tua arte aggredisce i nemici e puoi selezionare due effetti tra: -1d6 Tiro per Colpire, -4 Difesa, -4 Danno, -1d6 TS. I tuoi nemici devono essere entro 24 metri di raggio.

All'avversario è concesso un Tiro Salvezza Volontà DC pari 10+CAR+punteggio Intrattenere. Una creatura che riesce nel Tiro Salvezza è immune per quel giorno a nuove manifestazioni di questo tuo potere.

Attivare, mantenere o cambiare effetto dell'Abilità richiede 2 Azioni e dura fino all'inizio del tuo round successivo. Puoi mantenere l'Abilità un numero di round, anche non consecutivi, pari a punteggio Intrattenere x 3 al giorno. Le creature per rimanere influenzate devono continuare a vedere/sentire la performance.

\feat{Iniziativa migliorata}
\begin{description}[noitemsep, topsep=0pt, parsep=0pt, partopsep=0pt, leftmargin=0cm, labelwidth=2.5cm]
    \item[\textbf{Requisito}]: Intelligenza o Destrezza 1
    \item[\textbf{Tiri Salvezza}]: +2 Riflessi
    \item[\textbf{Caratteristica}]: Destrezza o Intelligenza
\end{description}

Aumenti l'iniziativa di +1. L'Abilità può essere presa fino a 2 volte ed il bonus si cumula.

\feat{La mia pelle}
\begin{description}[noitemsep, topsep=0pt, parsep=0pt, partopsep=0pt, leftmargin=0cm, labelwidth=2.5cm]
    \item[\textbf{Requisito}]: Competenza Armi 1
    \item[\textbf{Tiri Salvezza}]: +3 Tempra
    \item[\textbf{Caratteristica}]: Costituzione o Forza
\end{description}

Hai un rapporto quasi simbiotico con la tua armatura.

La \textbf{prima volta} che prendi questa Abilità la Difesa che ti concede l'armatura che porti aumenta di 1.

La \textbf{seconda volta} che prendi questa Abilità, requisito Competenza Armi 6, la Difesa che ti concede l'armatura che porti aumenta di 2.

\feat{La mia morte la tua morte}
\begin{description}[noitemsep, topsep=0pt, parsep=0pt, partopsep=0pt, leftmargin=0cm, labelwidth=2.5cm]
    \item[\textbf{Requisito}]: Competenza Armi 1, Forza 1

    \item[\textbf{Tiri Salvezza}]: +2 Tempra, +1 Volontà
    \item[\textbf{Caratteristica}]: Carisma o Forza
\end{description}

Per ogni singolo avversario in combattimento puoi fare che il primo colpo a segno dello scontro causi un danno aggiuntivo pari al doppio di Competenza Armi. L'avversario guadagna un bonus al Tiro per Colpire ed al danno pari al valore della tua Competenza Armi al primo Tiro per Colpire effettuato entro la fine del round successivo.

L'Abilità va dichiarato prima del Tiro per Colpire e dura fino all'inizio del prossimo round.

\feat{La mia Testa e' piu' Dura}
\begin{description}[noitemsep, topsep=0pt, parsep=0pt, partopsep=0pt, leftmargin=0cm, labelwidth=2.5cm]
    \item[\textbf{Requisito}]: Competenza Armi 1
    \item[\textbf{Tiri Salvezza}]: +1 Tempra, +1 Volontà
    \item[\textbf{Caratteristica}]: Forza o Costituzione
\end{description}

La tua arma, presente nella Lista Armi Rompi Cranio, fa +2 danni

\feat{Lesto}
\begin{description}[noitemsep, topsep=0pt, parsep=0pt, partopsep=0pt, leftmargin=0cm, labelwidth=2.5cm]
    \item[\textbf{Requisito}]: Competenza Armi 3
    \item[\textbf{Tiri Salvezza}]: +1 Riflessi, +1 Volontà
    \item[\textbf{Caratteristica}]: Destrezza o Forza
\end{description}

La \textbf{prima volta} che prendi questa Abilità quando un alleato esegue un tiro critico puoi, usando una reazione, effettuare un Tiro per Colpire, con penalità di -1d6, contro il medesimo avversario purché in mischia anche con te.

La \textbf{seconda volta} che prendi questa Abilità, Competenza Armi 6, guadagni una reazione da poter usare solo per l'Abilità Lesto.

\feat{Litania versatile}
\begin{description}[noitemsep, topsep=0pt, parsep=0pt, partopsep=0pt, leftmargin=0cm, labelwidth=2.5cm]
    \item[\textbf{Requisito}]: Competenza Intrattenere 6
    \item[\textbf{Tiri Salvezza}]: +1 Volontà, +1 Riflessi
    \item[\textbf{Caratteristica}]: Carisma o Destrezza
\end{description}

Tramite la tua esibizione puoi decidere di infondere coraggio o paura alle creature entro 9 metri da te.

La \textbf{prima volta} che prendi questa Abilità ogni round, per tutte le creature influenzate, puoi decidere di applicare fino a 2 modificatori tra: bonus di +4 al Tiro per Colpire oppure +4 alla Difesa oppure -4 al Tiro per Colpire oppure -4 alla Difesa.

All'avversario è concesso un Tiro Salvezza Volontà DC pari 10+CAR+punteggio Intrattenere. Una creatura che riesce nel Tiro Salvezza è immune per quel giorno a nuove manifestazioni di questo tuo potere.

Attivare e mantenere l'Abilità richiede 2 Azioni. Puoi mantenere l'Abilità un numero di round, anche non consecutivi, pari a punteggio Intrattenere al giorno.

Le creature per rimanere influenzate devono continuare a vedere/sentire la performance.

La \textbf{seconda volta} che prendi questa Abilità puoi escludere dall'effetto un numero di creature fino al punteggio di Carisma

\feat{Lo scudo e' mio amico}
\begin{description}[noitemsep, topsep=0pt, parsep=0pt, partopsep=0pt, leftmargin=0cm, labelwidth=2.5cm]
    \item[\textbf{Requisito}]: Competenza Armi 1, Competenza Magica 1
    \item[\textbf{Tiri Salvezza}]: +1 Tempra, +1 Riflessi
    \item[\textbf{Caratteristica}]: Destrezza o Costituzione
\end{description}

Il tuo \emph{amico} è sempre al tuo fianco.

La \textbf{prima volta} che prendi questa Abilità il costante allenamento con lo scudo ti permette di usare scudi leggeri senza penalità alla Prova di Magia.

La \textbf{seconda volta} che prendi questa Abilità il costante allenamento con lo scudo ti permette di usare scudi medi riducendo la penalità di 2 alla Prova di Magia.

La \textbf{terza volta} che prendi questa Abilità il costante allenamento con lo scudo ti permette di usare scudi medi senza penalità alla Prova di Magia e riducendola di 2 quando usi gli scudi pesanti.

\feat{Magie Potenti}
\begin{description}[noitemsep, topsep=0pt, parsep=0pt, partopsep=0pt, leftmargin=0cm, labelwidth=2.5cm]
    \item[\textbf{Requisito}]: Competenza Magica 5
    \item[\textbf{Tiri Salvezza}]: +2 Volontà
    \item[\textbf{Caratteristica}]: Modificatore di caratteristica per incantesimi o a scelta
\end{description}

Le tue magie sono straordinariamente efficaci.

Scegli una Liste di Magia, ottieni un +1d6 alla Prova di Magia nel lancio di incantesimi da questa Lista. L'Abilità può essere presa più volte ma il totale deve essere inferiore od uguale a CM/4.

\feat{Molla}
\begin{description}[noitemsep, topsep=0pt, parsep=0pt, partopsep=0pt, leftmargin=0cm, labelwidth=2.5cm]
    \item[\textbf{Requisito}]: Forza 0
    \item[\textbf{Tiri Salvezza}]: +1 Riflessi, +1 Tempra
    \item[\textbf{Caratteristica}]: Destrezza o Forza
\end{description}

La \textbf{prima volta} che prendi questa Abilità puoi ignorare il requisito dei 3 metri di rincorsa prima di un salto.

La \textbf{seconda volta} che prendi questa Abilità quando effettui una prova per saltare in lungo od in alto tiri 1d6 in più.

\feat{Muro mentale}
\begin{description}[noitemsep, topsep=0pt, parsep=0pt, partopsep=0pt, leftmargin=0cm, labelwidth=2.5cm]
    \item[\textbf{Requisito}]: Saggezza +1
    \item[\textbf{Tiri Salvezza}]: +2 Volontà, +1 Tempra
    \item[\textbf{Caratteristica}]: Modificatore di caratteristica per incantesimi o Saggezza
\end{description}

La tua mente è addestrata contro chi vuole influenzarla. Ogni volta che prendi questa Abilità guadagni +1 ai Tiri Salvezza contro gli incantesimi della Lista di Magia di Ammaliamento.

\feat{Occhi della magia}
\begin{description}[noitemsep, topsep=0pt, parsep=0pt, partopsep=0pt, leftmargin=0cm, labelwidth=2.5cm]
    \item[\textbf{Requisito}]: Competenza Magica 1
    \item[\textbf{Tiri Salvezza}]: +1 Volontà, +1 Tempra
    \item[\textbf{Caratteristica}]: Modificatore di caratteristica per incantesimi o Carisma
\end{description}

La \textbf{prima volta} che prendi questa Abilità se lo puoi vedere sai anche se è magico. Costa una Azione attivare la vista magica e dura un round.

La \textbf{seconda volta}, requisito Competenza Magica 1, che prendi l'Abilità attivare la vista magica costa una Reazione.

\feat{Occhio Clinico}
\begin{description}[noitemsep, topsep=0pt, parsep=0pt, partopsep=0pt, leftmargin=0cm, labelwidth=2.5cm]
    \item[\textbf{Requisito}]: Competenza Armi 3
    \item[\textbf{Tiri Salvezza}]: +2 Riflessi
    \item[\textbf{Caratteristica}]: Destrezza o Forza
\end{description}

Sei in grado di fare danno critico a creature normalmente immuni ai critici.

\feat{Occhio di Falco}
\begin{description}[noitemsep, topsep=0pt, parsep=0pt, partopsep=0pt, leftmargin=0cm, labelwidth=2.5cm]
    \item[\textbf{Requisito}]: Competenza Armi 3
    \item[\textbf{Tiri Salvezza}]: +2 Riflessi, +1 Volontà
    \item[\textbf{Caratteristica}]: Destrezza o Intelligenza
\end{description}

La \textbf{prima volta} che prendi questa Abilità i proiettili, frecce o dardi, lanciati tra il primo ed il secondo incremento di gittata non hanno penalità al Tiro per Colpire.

La \textbf{seconda volta} che prendi questa Abilità, la penalità per i tiri entro il terzo incremento di portata è di 6.

La \textbf{terza volta} che prendi questa Abilità sei in grado di estendere ancora di più il tuo tiro e portarlo ad un quinto incremento con un -12 di penalità al colpire. Non hai penalità entro i primi 3 incrementi mentre hai -6 a colpire tra il terzo e quarto incremento.

\feat{Opportunista}
\begin{description}[noitemsep, topsep=0pt, parsep=0pt, partopsep=0pt, leftmargin=0cm, labelwidth=2.5cm]
    \item[\textbf{Requisito}]: Competenza Armi 2
    \item[\textbf{Tiri Salvezza}]: +2 Riflessi, +1 Volontà
    \item[\textbf{Caratteristica}]: Intelligenza o Destrezza
\end{description}

\label{attaccoopportunita}

Puoi tentare di colpire in mischia un avversario che \textbf{esca} o \textbf{attraversi} un area di mischia che tu minacci oppure che \textbf{usi un arma da lancio} nella tua area di mischia o \textbf{formuli un incantesimo}. L'Abilità è usabile come Reazione. Questo attacco viene anche chiamato attacco di opportunità.

\feat{Parata}
\begin{description}[noitemsep, topsep=0pt, parsep=0pt, partopsep=0pt, leftmargin=0cm, labelwidth=2.5cm]
    \item[\textbf{Requisito}]: Competenza Armi 3 oppure Pugno Vuoto 2
    \item[\textbf{Tiri Salvezza}]: +1 Riflessi, +1 Volontà
    \item[\textbf{Caratteristica}]: Forza o Destrezza
\end{description}

La \textbf{prima volta} che prendi questa Abilità quando usi l'Azione per \hyperlink{preparareladifesa}{Preparare la Difesa} (pag. \pageref{preparareladifesa}) questa aumenta di 1.

La \textbf{seconda volta} che prendi questa Abilità, requisito Competenza Armi 6 oppure Pugno Vuoto 4, usi una Azione Immediata per Preparare la Difesa.

La \textbf{terza volta} che prendi questa Abilità, requisito Competenza Armi 9 oppure Pugno Vuoto 6, puoi usare Preparare la Difesa su una creatura a distanza di mischia da te.

Usare l'Abilità Parata deve essere dichiarata nel proprio round e rimane attiva fino all'inizio del tuo round successivo.

%\medskip

%\begin{center}
%		\includegraphics[width=0.7\linewidth]{immagini/distillare.png}
%
%		\emph{The Alchemist Discovering Phosphorus. Joseph Wright of Derby (1771-1795)}
%\end{center}

\feat{Passo Felpato}
\begin{description}[noitemsep, topsep=0pt, parsep=0pt, partopsep=0pt, leftmargin=0cm, labelwidth=2.5cm]
    \item[\textbf{Requisito}]: Furtività 1
    \item[\textbf{Tiri Salvezza}]: +1 Riflessi, +1 Tempra
    \item[\textbf{Caratteristica}]: Destrezza o Saggezza
\end{description}

Il tuo passo è naturalmente silenzioso.

La \textbf{prima volta} che prendi questa Abilità la penalità per muoversi a piena velocità usando Furtività diviene di -1d6.

La \textbf{seconda volta} che prendi questa Abilità, requisito Destrezza 3, Furtività 8, non hai penalità a muoverti a piena velocità.

\feat{Passo rapido}
\begin{description}[noitemsep, topsep=0pt, parsep=0pt, partopsep=0pt, leftmargin=0cm, labelwidth=2.5cm]
    \item[\textbf{Requisito}]: Destrezza 1
    \item[\textbf{Tiri Salvezza}]: +2 Riflessi, +1 Tempra
    \item[\textbf{Caratteristica}]: Destrezza o Costituzione
\end{description}

Il tuo passo è naturalmente rapido.
Se hai movimento 6m passi a movimento 7m, se hai movimento 9m passi a movimento 10m.

Ogni ulteriori \textbf{due volte} che prendi l'Abilità il tuo movimento aumenta di 1 metro per Azione di Movimento, fino ad un massimo di +3 metri a round.

\feat{Passo Sicuro}
\begin{description}[noitemsep, topsep=0pt, parsep=0pt, partopsep=0pt, leftmargin=0cm, labelwidth=2.5cm]
    \item[\textbf{Requisito}]: Saggezza 1
    \item[\textbf{Tiri Salvezza}]: +2 Tempra, +1 Riflessi
    \item[\textbf{Caratteristica}]: Destrezza o Costituzione
\end{description}

E' la capacità di non essere rallentati in un ambiente ostile. E' necessario dichiarare su quale ambiente si prende l'Abilità. In questi ambienti il terreno naturale non è difficile. Finché ci si muove nell'ambiente scelto si ha un +1 alle prove di Sopravvivenza.

\medskip

\noindent\begin{tabular}{l|l}
	\toprule
\rowcolor{gray!20}\textbf{Ambiente} & \textbf{Ambiente}\\
\toprule
Giungla & Aquatico \& Costale\\
\rowcolor{gray!20}Palude & Collina \& Foresta \\
Pianura & Desertico \\
\rowcolor{gray!20}Montagna & Ghiacciai \& Tundra \\
Urbano& Sotterraneo
\end{tabular}

\medskip

Ogni qual volta si prende nuovamente questa Abilità si sceglie un ambiente diverso e si aggiunge al precedente o ci si specializza sullo stesso.

Le \textbf{seconda volta} che prendi questa Abilità sul medesimo terreno, specializzandoti, acquisisci una capacità a seconda del terreno.

\begin{itemize}[leftmargin=*] \setlength{\itemsep}{0pt}

\item \emph{Giungla / Foresta / Collina / Pianura}: il tuo movimento aumenta di 1 metro su questo terreno

\item \emph{Costale / Aquatico}: velocità di nuoto pari al tuo movimento

\item \emph{Palude}: +2 ai Tiri Salvezza contro Veleno

\item \emph{Desertico}: Riduzione danni da Fuoco pari al livello

\item \emph{Montagna / Ghiacciai / Tundra}: Riduzione danni da Freddo pari al livello

\item \emph{Sotterraneo}: Visione crepuscolare 9 metri

\item \emph{Urbano}: +1 Linguaggio, +1 a scelta in due Conoscenze

\end{itemize}

\feat{Pelle Coriacea}
\begin{description}[noitemsep, topsep=0pt, parsep=0pt, partopsep=0pt, leftmargin=0cm, labelwidth=2.5cm]
    \item[\textbf{Requisito}]: Costituzione 2
    \item[\textbf{Tiri Salvezza}]: +2 Tempra
    \item[\textbf{Caratteristica}]: Costituzione o Saggezza
\end{description}

La \textbf{prima volta} che prendi questa Abilità la tua pelle è estremamente resistente. Subisci 1 danno in meno quando colpito da armi taglienti.

La \textbf{seconda volta} che prendi questa Abilità, requisito Competenza Armi 6, subisci 1 danno in meno quando colpito da armi taglienti, perforanti, contundenti. Riduci di 1 la condizione di Sanguinamento quando acquisita.

La \textbf{terza volta} che prendi questa Abilità, requisito Competenza Armi 12 e Costituzione 3, subisci 1 danno in meno quando colpito da armi taglienti, perforanti o contundenti. Subisci 1 danno in meno quando colpito da magia. Riduci di 1 la condizione di Sanguinamento quando acquisita.

La \textbf{quarta volta} che prendi questa Abilità, requisito Competenza Armi 16, ignori 1 tiro critico quando colpito da armi taglienti, perforanti o contundenti e subisci 1 danno in meno quando colpito da magia. Riduci di 1 la condizione di Sanguinamento quando acquisita.

I bonus sono cumulativi.

\feat{Percettivo}
\begin{description}[noitemsep, topsep=0pt, parsep=0pt, partopsep=0pt, leftmargin=0cm, labelwidth=2.5cm]
    \item[\textbf{Requisito}]: Saggezza 0
    \item[\textbf{Tiri Salvezza}]: +1 Riflessi, +1 Volontà
    \item[\textbf{Caratteristica}]: Saggezza o Intelligenza
\end{description}

La tua Consapevolezza e attenzione ai particolari è sopra la media.

Prendi un bonus di +1 alla prove di Consapevolezza. L'Abilità può essere presa massimo 3 volte.

\feat{Persona veramente malvagia}
\begin{description}[noitemsep, topsep=0pt, parsep=0pt, partopsep=0pt, leftmargin=0cm, labelwidth=2.5cm]
    \item[\textbf{Requisito}]: Competenza Armi 1
    \item[\textbf{Tiri Salvezza}]: +1 Riflessi, +1 Volontà
    \item[\textbf{Caratteristica}]: Forza o Carisma
\end{description}

Quando vuoi sai essere cattivo.

CA/4 volte al giorno aggiungi il tuo valore di Competenza Armi al danno di un singolo attacco in mischia ad un singolo tuo avversario.

L'Abilità deve essere dichiarata prima di sapere l'esito del Tiro per Colpire. Costa una Azione.

\feat{Piu' sono grossi piu' fanno rumore quando cadono}
\begin{description}[noitemsep, topsep=0pt, parsep=0pt, partopsep=0pt, leftmargin=0cm, labelwidth=2.5cm]
    \item[\textbf{Requisito}]: Competenza Armi 1
    \item[\textbf{Tiri Salvezza}]: +2 Tempra, +1 Volontà
    \item[\textbf{Caratteristica}]: Forza o Carisma
\end{description}

Quando attacchi una creatura di almeno 2 taglie più grosse di te fai +1 danno aggiuntivo ogni 2 punti Competenza Armi. Se è solo una taglia superiore aggiungi 1 danno in più ogni 3 punti Competenza Armi.

\feat{Poliglotta}
\begin{description}[noitemsep, topsep=0pt, parsep=0pt, partopsep=0pt, leftmargin=0cm, labelwidth=2.5cm]
    \item[\textbf{Requisito}]: almeno Intelligenza -1, alla creazione del personaggio
    \item[\textbf{Tiri Salvezza}]: +2 Volontà
    \item[\textbf{Caratteristica}]: Intelligenza o Carisma
\end{description}

Hai una straordinaria capacità di imparare le lingue. Per ogni punto attribuito a Conoscenza Lingue conosci due lingue in più.

\feat{Potere del Patrono}
\begin{description}[noitemsep, topsep=0pt, parsep=0pt, partopsep=0pt, leftmargin=0cm, labelwidth=2.5cm]
    \item[\textbf{Requisito}]: Somma Tratti comuni con il Patrono 1, essere Devoto
    \item[\textbf{Tiri Salvezza}]: +1 Tempra, +2 Volontà
    \item[\textbf{Caratteristica}]: Modificatore di caratteristica per incantesimi o a scelta
\end{description}

La tua fede nel Patrono non conosce limiti ne crolli di fiducia.

1 volta al giorno per singola prova, come Reazione prima di effettuare la prova, usi come modificatore positivo unico la somma dei Tratti comune con il Patrono. Puoi usare questa Abilità su Tiri Salvezza, Tiri per Colpire e prove di competenze.

Se riescono tutte e tre le prove è probabile che si sia una Manifestazione del Patrono.

\feat{Primo Sangue}
\begin{description}[noitemsep, topsep=0pt, parsep=0pt, partopsep=0pt, leftmargin=0cm, labelwidth=2.5cm]
    \item[\textbf{Requisito}]: Competenza Armi 1
    \item[\textbf{Tiri Salvezza}]: +1 Tempra, +1 Volontà
    \item[\textbf{Caratteristica}]: Intelligenza o Carisma
\end{description}

Il primo Tiro per Colpire nel giorno ha un bonus di +1d6 e causa un tiro critico se colpisce.

\feat{Precisino}
\begin{description}[noitemsep, topsep=0pt, parsep=0pt, partopsep=0pt, leftmargin=0cm, labelwidth=2.5cm]
    \item[\textbf{Requisito}]: Competenza Armi 2
    \item[\textbf{Tiri Salvezza}]: +1 Riflessi, +1 Volontà
    \item[\textbf{Caratteristica}]: Destrezza o Charisma
\end{description}

La \textbf{prima volta} che prendi questa Abilità quando scagli un proiettile diminuisci di 2 la penalità data da Copertura.

La \textbf{seconda volta} che prendi questa Abilità, Competenza Armi 4, riduci la penalità data da Copertura di 4.

\feat{Prodigioso}
\begin{description}[noitemsep, topsep=0pt, parsep=0pt, partopsep=0pt, leftmargin=0cm, labelwidth=2.5cm]
    \item[\textbf{Requisito}]: Competenza Magica 3
    \item[\textbf{Tiri Salvezza}]: +2 Volontà
    \item[\textbf{Caratteristica}]: Modificatore di caratteristica per incantesimi
\end{description}

La tua mente non ha confini. Puoi apprendere due incantesimi presenti sul tuo Tomo di Magia, sempre rispettando i limiti del massimo livello di incantesimi lanciabile.

L'Abilità può essere presa più volte ed il totale deve essere pari o inferiore a CM/4.

\feat{Proseguire}
\begin{description}[noitemsep, topsep=0pt, parsep=0pt, partopsep=0pt, leftmargin=0cm, labelwidth=2.5cm]
    \item[\textbf{Requisito}]: Competenza Armi 1
    \item[\textbf{Tiri Salvezza}]: +1 Tempra, +1 Volontà
    \item[\textbf{Caratteristica}]: Forza o Destrezza
\end{description}

La \textbf{prima volta} che prendi questa Abilità se elimini un avversario con la tua ultima Azione di Attacco in mischia, puoi effettuare un'azione di attacco bonus.

L'attacco bonus utilizza gli stessi modificatori dell'ultima Azione di Attacco ed è diretto a un altro nemico entro la distanza di mischia.

Se elimini questa seconda creatura, non puoi effettuare ulteriori attacchi.

La \textbf{seconda volta}, requisiti Proseguire, Competenza Armi 6, se con l'attacco bonus di Proseguire elimini un avversario puoi effettuare una ulteriore azione di attacco bonus, utilizzando gli stessi modificatori dell'ultima Azione di Attacco. Se elimini questa creatura puoi continuare, spostandoti di massimo 1 metro, ad attaccare la creatura successiva.

Ogni attacco bonus oltre il primo subisce una penalità cumulativa: -2 al colpire e -1 al danno.

\feat{Pugno di Ferro}
\begin{description}[noitemsep, topsep=0pt, parsep=0pt, partopsep=0pt, leftmargin=0cm, labelwidth=2.5cm]
    \item[\textbf{Requisito}]: Lista Pugno Vuoto 3
    \item[\textbf{Tiri Salvezza}]: +2 Tempra, +1 Volontà
    \item[\textbf{Caratteristica}]: Forza o Destrezza
\end{description}

La tua tecnica di combattimento senza armi è estremamente precisa e potente.

La \textbf{prima volta} che prendi questa Abilità il danno causato dai tuoi colpi ed il Tiro per Colpire aumenta di 1.

La \textbf{seconda volta} che prendi questa Abilità, requisito Pugno Vuoto 6. Il danno aumenta di +2, il Tiro per Colpire +1.

La \textbf{terza volta} che prendi questa Abilità, requisito Pugno Vuoto 9. Danno +1, Tiro per Colpire +2.

La \textbf{quarta volta} che prendi questa Abilità, requisito Pugno Vuoto 12. Danno +2, Tiro per Colpire +1.

La \textbf{quinta volta} che prendi questa Abilità, requisito Pugno Vuoto 15. Danno +1, Tiro per Colpire +2.

La \textbf{sesta volta} che prendi questa Abilità, Requisito Pugno Vuoto 18. Danno +2, Tiro per Colpire +1.

I bonus indicati sono cumulativi.

\feat{Pugno Potente}
\begin{description}[noitemsep, topsep=0pt, parsep=0pt, partopsep=0pt, leftmargin=0cm, labelwidth=2.5cm]
    \item[\textbf{Requisito}]: Lista Pugno Vuoto 3
    \item[\textbf{Tiri Salvezza}]: +1 Tempra, +2 Volontà
    \item[\textbf{Caratteristica}]: Forza o Costituzione
\end{description}

Consumi 2 Azioni. Effettui un unico Tiro per Colpire con -5 di penalità.
Se colpisci, oltre al danno ed un danno critico, l'avversario che deve essere massimo di due taglie superiore alla tua deve effettuare un Tiro Salvezza su Tempra con DC pari al tuo Tiro per Colpire oppure essere spinto di 3 metri in una direzione a tua scelta.

Se fallisce il Tiro Salvezza in maniera critica subisce un ulteriore danno critico.

\feat{Questo e' il mio pugnale}
\begin{description}[noitemsep, topsep=0pt, parsep=0pt, partopsep=0pt, leftmargin=0cm, labelwidth=2.5cm]
    \item[\textbf{Requisito}]: Competenza Armi 1
    \item[\textbf{Tiri Salvezza}]: +2 Tempra, +1 Riflessi
    \item[\textbf{Caratteristica}]: Destrezza o Carisma
\end{description}

Quando fai un danno critico con il tuo pugnale sommi un ulteriore danno critico. L'Abilità è usabile 1 volta per avversario e si applica automaticamente al primo danno critico effettuato.

\feat{Questa e' la mia arma!}
\begin{description}[noitemsep, topsep=0pt, parsep=0pt, partopsep=0pt, leftmargin=0cm, labelwidth=2.5cm]
    \item[\textbf{Requisito}]: Competenza Armi 1
    \item[\textbf{Tiri Salvezza}]: +2 Tempra, +1 Volontà
    \item[\textbf{Caratteristica}]: Forza o Carisma
\end{description}

La \textbf{prima volta} che prendi questa Abilità ogni volta che colpisci il medesimo avversario, a partire dal secondo round, fai un danno aggiuntivo (Max +1 per round di combattimento, anche se lo colpisci più volte nel round) fino ad un massimo +5. La prima volta che non colpisci nel round l'avversario il bonus torna a +0. Il bonus si può mantenere su un solo avversario alla volta.

La \textbf{seconda volta} che prendi questa Abilità, Competenza Armi 5, puoi mancare l'avversario con un colpo e non perdere i benefici.

\feat{Radici magiche}
\begin{description}[noitemsep, topsep=0pt, parsep=0pt, partopsep=0pt, leftmargin=0cm, labelwidth=2.5cm]
    \item[\textbf{Requisito}]: Competenza Magica 1
    \item[\textbf{Tiri Salvezza}]: +2 Volontà, +1 Tempra
    \item[\textbf{Caratteristica}]: Modificatore di caratteristica per incantesimi o a scelta
\end{description}

Finché sei influenzato da un tuo incantesimo, utilizzando un'Azione la tua arma guadagna un +1 al colpire ed al danno e si considera un'arma magica fino alla fine del round.


\begin{center}
	%	\includegraphics[width=0.7\linewidth]{immagini/distillare.png}
	%
	%	\emph{The Alchemist Discovering Phosphorus. Joseph Wright of Derby (1771-1795)}

	\includegraphics[width=0.7\linewidth]{immagini/Theatrum_Chemicum_Britannicum.png}

	\emph{Theatrum Chemicum Britannicum, 1652}
\end{center}

\feat{Rappresaglia}
\begin{description}[noitemsep, topsep=0pt, parsep=0pt, partopsep=0pt, leftmargin=0cm, labelwidth=2.5cm]
    \item[\textbf{Requisito}]: Competenza Armi 1, almeno Seguace di Gradh o Nedraf o Orlaith o Sumkjr
    \item[\textbf{Tiri Salvezza}]: +2 Volontà, +1 Tempra
    \item[\textbf{Caratteristica}]: Carisma o Saggezza
\end{description}

Vedere i tuoi amici feriti ti riempie di rabbia.

Quanto un compagno (o te stesso) scende sotto metà dei Punti Ferita guadagni un +1 al Tiro per Colpire e Tiri Salvezza.

La durata massima dell'effetto é 1 minuto (6 round) al giorno e deve essere consecutiva. Il giocatore sceglie se attivare o meno l'Abilità ed il compagno ferito deve essere entro 9 metri.

Puoi prendere questa Abilità \textbf{fino a 3 volte}, ogni volta il bonus al Tiro per Colpire e Tiro Salvezza aumentano di 1.

\feat{Recupero}
\begin{description}[noitemsep, topsep=0pt, parsep=0pt, partopsep=0pt, leftmargin=0cm, labelwidth=2.5cm]
    \item[\textbf{Requisito}]: Costituzione 0
    \item[\textbf{Tiri Salvezza}]: +2 Tempra
    \item[\textbf{Caratteristica}]: Costituzione
\end{description}

Il tuo corpo produce spontaneamente caffeina!

Impieghi la metà del tempo per recuperare dalle condizioni Affaticato.

\feat{Resistenza della pietra}
\begin{description}[noitemsep, topsep=0pt, parsep=0pt, partopsep=0pt, leftmargin=0cm, labelwidth=2.5cm]
    \item[\textbf{Requisito}]: Costituzione 0
    \item[\textbf{Tiri Salvezza}]: nessuno
    \item[\textbf{Caratteristica}]: Costituzione o Saggezza
\end{description}

Nel tempo hai allenato la tua Costituzione a reggere gli urti, trasformazioni, veleni e quant'altro volesse modificare il tuo corpo.

La \textbf{prima volta} che prendi questa Abilità ottieni un bonus di +2 al Tiro Salvezza su Tempra. Il bonus è cumulativo, +2 la prima volta, +1 la \textbf{seconda}, +1 la \textbf{terza}.

La \textbf{quarta volta} che prendi questa Abilità puoi decidere di riuscire automaticamente in un Tiro Salvezza su Tempra una volta al giorno come Reazione. Deve essere dichiarata e non fa tirare il Tiro Salvezza.

\begin{center}
	\includegraphics[width=0.9\linewidth]{immagini/Historia_Mundi_Naturalis.png}

	\emph{Woodcut illustration from an edition of Pliny the Elder's Naturalis Historia (1582)}
\end{center}

\feat{Riflessi fulminei}
\begin{description}[noitemsep, topsep=0pt, parsep=0pt, partopsep=0pt, leftmargin=0cm, labelwidth=2.5cm]
    \item[\textbf{Requisito}]: Destrezza 1
    \item[\textbf{Tiri Salvezza}]: nessuno
    \item[\textbf{Caratteristica}]: Destrezza o Intelligenza
\end{description}

Nel tempo hai allenato i tuoi riflessi a schivare e prevedere qualsiasi ostacolo. Il bonus è cumulativo, +2 la \textbf{prima volta}, +1 la \textbf{seconda}, +1 la \textbf{terza} ai Tiri Salvezza su Riflessi.

La \textbf{quarta volta} che prendi questa Abilità puoi decidere di riuscire automaticamente in un Tiro Salvezza su Riflessi una volta al giorno come Reazione. Deve essere dichiarata e non fa tirare il Tiro Salvezza.

\feat{Rinoceronte}
\begin{description}[noitemsep, topsep=0pt, parsep=0pt, partopsep=0pt, leftmargin=0cm, labelwidth=2.5cm]
    \item[\textbf{Requisito}]: Costituzione 1
    \item[\textbf{Tiri Salvezza}]: +2 Tempra
    \item[\textbf{Caratteristica}]: Costituzione o Saggezza
\end{description}

La tua carica è distruttiva! Dietro di te lasci solo una scia di macerie e non ti lasci fermare da nulla.

La \textbf{prima volta} che prendi questa Abilità puoi effettuare una Carica anche se il terreno è difficile.

La \textbf{seconda volta} che prendi questa Abilità non consideri il terreno come difficile quando carichi.

La \textbf{terza volta} che prendi questa Abilità hai indurito la tua pelle talmente tanto che la tua Difesa naturale aumenta di +1.

\feat{Robusto}
\begin{description}[noitemsep, topsep=0pt, parsep=0pt, partopsep=0pt, leftmargin=0cm, labelwidth=2.5cm]
    \item[\textbf{Requisito}]: Costituzione -1
    \item[\textbf{Tiri Salvezza}]: +2 Tempra
    \item[\textbf{Caratteristica}]: Costituzione o Saggezza
\end{description}

La \textbf{prima volta} che prendi questa Abilità aumenti i Punti Ferita di 3.

La \textbf{seconda volta} che prendi questa Abilità aumenti di 1 i Punti Ferita presi per livello.

I bonus sono cumulativi e retroattivi ai livelli precedenti.

\feat{Sangue Puro}
\begin{description}[noitemsep, topsep=0pt, parsep=0pt, partopsep=0pt, leftmargin=0cm, labelwidth=2.5cm]
    \item[\textbf{Requisito}]: Animalia, Devoto di Efrem oppure Shayalia
    \item[\textbf{Tiri Salvezza}]: +1 Volontà , +2 Tempra
    \item[\textbf{Caratteristica}]: Modificatore di caratteristica per incantesimi o Costituzione
\end{description}

La \textbf{prima volta} che prendi questa Abilità con questa Abilità ogni tuo attacco quando ti trasformi con Animalia causa 1 danno aggiuntivo ed è considerato un attacco magico +1. Concentrandoti sul tuo passo puoi lasciare le impronte di un animale in cui ti puoi trasformare ed il terreno si considera doppiamente difficile.

Le \textbf{seconda volta} che prendi questa Abilità, Competenza Magica 8, quando usi l'Abilità di Animalia puoi eseguire una trasformazione parziale ovvero prendere il tipo di Movimento oppure Sensi della creatura in cui ti trasformi. Quando usi l'Abilità Animalia puoi selezionare una creatura con un Grado di Sfida aumentato di 1. Lasciare impronte diverse è considerato terreno difficile.

Le \textbf{terza volta} che prendi questa Abilità, Competenza Magica 12, quando ti trasformi in un animale, puoi usare la tua Competenza Magica al posto della Competenza Armi negli attacchi naturali. Quando usi l'Abilità Animalia puoi selezionare una creatura con un Grado di Sfida aumentato di 1. Lasciare impronte diverse non è considerato terreno difficile.

Le Abilità due e tre sono cumulative.

\feat{Sapiente}
\begin{description}[noitemsep, topsep=0pt, parsep=0pt, partopsep=0pt, leftmargin=0cm, labelwidth=2.5cm]
    \item[\textbf{Requisito}]: Competenza Magica 1, solo alla creazione del personaggio
    \item[\textbf{Tiri Salvezza}]: +2 Volontà
    \item[\textbf{Caratteristica}]: Modificatore di caratteristica per incantesimi o a scelta
\end{description}

Il tuo interesse e connessione con la magia non ha eguali. Puoi conoscere 4 incantesimi in più (pur rispettando i vincoli di massimo livello copiabile sul Tomo).

\feat{Scacciare i non morti}
\begin{description}[noitemsep, topsep=0pt, parsep=0pt, partopsep=0pt, leftmargin=0cm, labelwidth=2.5cm]
    \item[\textbf{Requisito}]: Somma Tratti comune 2, essere Devoto o Seguace
    \item[\textbf{Tiri Salvezza}]: +2 Volontà, +1 Tempra
    \item[\textbf{Caratteristica}]: Carisma o Saggezza
\end{description}

Concentrandoti sulla potenza del tuo Patrono convogli l'energia positiva ed allontani o distruggi i non-morti.

Tira 1d6 + somma dei Tratti in comune con il Patrono, questo totale è il tuo Potere Divino.

Partendo dai non morti più deboli intorno a te, nel raggio di 9 metri, controlla il punteggio del Potere Divino ed il Grado di Sfida del non morto.

Se il Potere Divino è almeno il doppio del Grado di Sfida, il non-morto viene distrutto e si sottrae il doppio del Grado di Sfida dal valore del Potere Divino.

Se il non morto non viene distrutto allora esegue un Tiro Salvezza su Volontà a DC pari a 10 + Potere Divino per resistere allo scaccio. Se il Tiro Salvezza fallisce il non morto è scacciato, se riesce non viene influenzato. Sia che riesca il Tiro Salvezza o meno sottrai il Grado di Sfida dal punteggio del Potere Divino prima di passare a verificare un nuovo non morto.

L'Abilità è usabile un numero di volte al giorno pari alla Saggezza ma un non morto può essere influenzato solo una volta al giorno dal tuo effetto.

Un non-morto che viene scacciato è sotto \hyperlink{condizionepaura}{Paura} per 1d4 round, un non-morto distrutto viene ridotto in polvere ed energia divina.

Un Devoto di Sixiser al posto di scacciare può dominare il non morto per 2d4 round, oppure per oppure per 1 ora reale al posto di distruggerlo.

Un Devoto di Thaft ottiene +1d6 al Potere Divino.

\begin{center}
	\includegraphics[width=0.6\linewidth]{immagini/turning-undead-six.png}
\end{center}

\feat{Schivare trappole}
\begin{description}[noitemsep, topsep=0pt, parsep=0pt, partopsep=0pt, leftmargin=0cm, labelwidth=2.5cm]
    \item[\textbf{Requisito}]: Destrezza 2
    \item[\textbf{Tiri Salvezza}]: +2 Riflessi, +1 Tempra
    \item[\textbf{Caratteristica}]: Destrezza o Intelligenza
\end{description}

La \textbf{prima volta} che prendi l'Abilità ottieni un bonus di +1d6 al Tiro Salvezza per evitare l'effetto delle trappole.

La \textbf{seconda volta} che prendi l'Abilità, requisito Competenza Armi 5, anche se la trappola non concede Tiro Salvezza la tua naturale propensione ad evitare i danni ti concede un Tiro Salvezza su Riflessi per dimezzare i danni.

E' anche possibile usare questa Abilità, usa una Reazione, per evitare Attacco furtivo (TS Riflessi superiore a Tiro Colpire avversario).

%La \textbf{terza volta} che prendi l'Abilità requisiti Competenza Armi 9, il Tiro Salvezza se riuscito ti permette di evitare qualsiasi effetto della trappola, se fisicamente possibile.

\feat{Schivata prodigiosa}
\begin{description}[noitemsep, topsep=0pt, parsep=0pt, partopsep=0pt, leftmargin=0cm, labelwidth=2.5cm]
    \item[\textbf{Requisito}]: Destrezza 3
    \item[\textbf{Tiri Salvezza}]: +2 Riflessi
    \item[\textbf{Caratteristica}]: Destrezza o Saggezza
\end{description}

La \textbf{prima volta} che prendi questa Abilità come Reazione ad una Azione di attacco avversaria puoi aggiungere +1 alla tua Difesa. Puoi usare l'Abilità fino a 3 volte al giorno.

La \textbf{seconda volta} che prendi l'Abilità, requisito Competenza Armi 4, un avversario non prende il bonus al colpire da fiancheggiamento contro di te.

La \textbf{terza volta} che prendi l'Abilità, requisito Competenza Armi 8, Destrezza 4, fino a due avversari non prendono il bonus al colpire dato fa fiancheggiamento.

Puoi usare l'Abilità anche dopo che si è saputo di quanto si è stati colpiti.

\feat{Seconda pelle}
\begin{description}[noitemsep, topsep=0pt, parsep=0pt, partopsep=0pt, leftmargin=0cm, labelwidth=2.5cm]
    \item[\textbf{Requisito}]: Competenza Armi 1
    \item[\textbf{Tiri Salvezza}]: +2 Tempra
    \item[\textbf{Caratteristica}]: Costituzione o Forza
\end{description}

Il costante utilizzo dell'armatura ti permette di indossarle senza particolari penalità.

La \textbf{prima volta} che prendi questa Abilità le penalità alle prove di competenza di Base dato dall'armatura diminuisce di 1.

La \textbf{seconda volta} che si prende questa Abilità, requisito Competenza Armi 6, la penalità alle prove di competenza diminuisce di ulteriori 1. La penalità al movimento diminuisce di 1 metro. Puoi dormire in armature medie senza essere affaticato.

La \textbf{terza volta} che si prende questa Abilità, requisito Competenza Armi 11, la penalità alle prove di competenza diminuisce di ulteriori 1. La penalità al movimento diminuisce di un ulteriore 1 metro. Puoi dormire in armature pesanti senza essere affaticato.

\feat{Segugio}
\begin{description}[noitemsep, topsep=0pt, parsep=0pt, partopsep=0pt, leftmargin=0cm, labelwidth=2.5cm]
    \item[\textbf{Requisito}]: Intelligenza 1, Saggezza 1, Competenza Armi 1
    \item[\textbf{Tiri Salvezza}]: +1 Riflessi, +1 Volontà
    \item[\textbf{Caratteristica}]: Saggezza o Intelligenza
\end{description}

Hai un talento naturale per seguire le persone

La \textbf{prima volta} che prendi questa Abilità con due Azioni ti focalizzi su un target che puoi vedere e finché lo vedi rimani focalizzato. Le tue Azioni che coinvolgono quel target hanno un +1 di bonus. Mantenere la focalizzazione costa 1 Azione per round.

La \textbf{seconda} volta che prendi questa Abilità, requisito Competenza Armi 10, Saggezza 2, il bonus sale a +2.

La \textbf{terza} volta che prendi questa Abilità, requisito Competenza Armi 16, Saggezza 3, il bonus sale a +3.

Il bonus può essere usato al Tiro per Colpire, Tiri Salvezza causati dall'avversario e prove di competenza di Base, non al danno.

\feat{Senza Traccia}
\begin{description}[noitemsep, topsep=0pt, parsep=0pt, partopsep=0pt, leftmargin=0cm, labelwidth=2.5cm]
    \item[\textbf{Requisito}]: Passo Sicuro
    \item[\textbf{Tiri Salvezza}]: +2 Volontà, +1 Riflessi
    \item[\textbf{Caratteristica}]: Saggezza o Destrezza
\end{description}

La capacità di non lasciare impronte nell'ambiente scelto. Ogni volta che prendi questa Abilità puoi scegliere un ambiente diverso (vedi Abilità \hyperlink{passosicuro}{Passo Sicuro} di cui hai preso l'Abilità. La difficoltà della prova di Seguire Tracce per inseguirti aumentata di 10.

\feat{Sifone Nero}
\begin{description}[noitemsep, topsep=0pt, parsep=0pt, partopsep=0pt, leftmargin=0cm, labelwidth=2.5cm]
    \item[\textbf{Requisito}]: Competenza magia 6, Seguace o Devoto di Tazher, punti Tratto in comune 6
    \item[\textbf{Tiri Salvezza}]: +1 Tempra, +2 Volontà
    \item[\textbf{Caratteristica}]: Modificatore di caratteristica per incantesimi o a scelta
\end{description}

Quando lanci un incantesimo che abbia durata istantanea e che causi danno ai Punti Ferita ad uno o più soggetti, aumentando di metà, arrotondato per eccesso, i Punti Magia usati nell'incantesimo, recuperi un ammontare di Punti Ferita pari a metà della creatura che ne ha persi di più.

Il tempo di lancio dell'incantesimo aumenta a 3 Azioni.

\feat{Sfortunato}
\begin{description}[noitemsep, topsep=0pt, parsep=0pt, partopsep=0pt, leftmargin=0cm, labelwidth=2.5cm]
    \item[\textbf{Requisito}]: Fortunato, almeno 6 punti nella somma dei Tratti
    \item[\textbf{Tiri Salvezza}]: +1 Tempra, +1 Volontà
    \item[\textbf{Caratteristica}]: una caratteristica a scelta
\end{description}

Una volta al giorno puoi trasformare un 6 tirato dal Narratore (Tiri per Colpire, Prove Competenze, Tiri Salvezza) in un 1. L'Abilità si può dichiarare una volta venuto a sapere del tiro effettuato.

\feat{Spara e Scappa}
\begin{description}[noitemsep, topsep=0pt, parsep=0pt, partopsep=0pt, leftmargin=0cm, labelwidth=2.5cm]
    \item[\textbf{Requisito}]: Lista Balestre 3
    \item[\textbf{Tiri Salvezza}]: +1 Tempra, +1 Riflessi
    \item[\textbf{Caratteristica}]: Destrezza o Carisma
\end{description}

Mentre esegui una Azione di Movimento puoi ridurre di 1 Azione il tempo di caricamento della tua balestra. In caso di Balestre Leggere od a una mano puoi puoi quindi ricaricarla mentre ti muovi, in caso di Balestre pesanti riduci di 1 Azione il tempo di caricamento.

\feat{Specialista}
\begin{description}[noitemsep, topsep=0pt, parsep=0pt, partopsep=0pt, leftmargin=0cm, labelwidth=2.5cm]
    \item[\textbf{Requisito}]: Competenza Magica 3
    \item[\textbf{Tiri Salvezza}]: +2 Tempra
    \item[\textbf{Caratteristica}]: Modificatore di caratteristica per incantesimi o a scelta
\end{description}

Scegli un incantesimo che conosci, i Punti Magia spesi per lanciare questo incantesimo diminuiscono di 1, con un minimo costo di 1.

L'Abilità può essere presa più volte su incantesimi ogni volta diversi.

\feat{Stai giu'!}
\begin{description}[noitemsep, topsep=0pt, parsep=0pt, partopsep=0pt, leftmargin=0cm, labelwidth=2.5cm]
    \item[\textbf{Requisito}]: Competenza Armi 3
    \item[\textbf{Tiri Salvezza}]: +2 Tempra, +1 Volontà
    \item[\textbf{Caratteristica}]: Forza o Carisma
\end{description}

La \textbf{prima volta} che prendi questa Abilità quando un tuo attacco causa due tiri critici su un avversario, la forza del colpo è tale da metterlo prono. L'avversario deve fare un Tiro Salvezza Tempra (DC pari all'ultimo Tiro per Colpire con l'arma che ha causato l'ultimo tiro critico) o cadere prono. L'Abilità funziona su creature di taglia pari o inferiore a quella del personaggio.

La \textbf{seconda volta} che prendi l'Abilità puoi influenzare anche creature di una taglia superiore.

La \textbf{terza volta} che prendi l'Abilità puoi influenzare anche creature di due taglie superiori. Il grado 3 non è cumulabile con il grado 2.

\feat{Tattico}
\begin{description}[noitemsep, topsep=0pt, parsep=0pt, partopsep=0pt, leftmargin=0cm, labelwidth=2.5cm]
    \item[\textbf{Requisito}]: Competenza Armi 1, Intelligenza 1
    \item[\textbf{Tiri Salvezza}]: +1 Tempra , +1 Volontà
    \item[\textbf{Caratteristica}]: Intelligenza o Carisma
\end{description}

Hai una capacità quasi istintiva di gestire e prevedere l'esito dei combattimenti. Usare l'Abilità costa 1 Azione.

Scambi l'esito della iniziativa tra due creature di cui almeno una sia una tua alleata.

Se nessuno dei due bersagli ha ancora agito allora l'effetto della tua Abilità si attiva subito. Se almeno una delle due ha già agito allora la tua Abilità prende effetto il round successivo.

L'effetto dura solo un round, poi i bersagli tornano alle loro iniziative precedenti.

La \textbf{prima volta} che prendi questa Abilità puoi influire su una coppia di bersagli che siano in mischia tra loro.

La \textbf{seconda volta} che prendi questa Abilità, requisito Intelligenza 2, Competenza Armi 6, puoi scambiare l'iniziativa su due coppie che siano reciprocamente in mischia tra loro.

\feat{Tempesta di Furia}
\begin{description}[noitemsep, topsep=0pt, parsep=0pt, partopsep=0pt, leftmargin=0cm, labelwidth=2.5cm]
    \item[\textbf{Requisito}]: Lista Pugno Vuoto 7, Destrezza 1, Forza 1
    \item[\textbf{Tiri Salvezza}]: +2 Riflessi, +1 Volontà
    \item[\textbf{Caratteristica}]: Destrezza o Forza
\end{description}

Quando usi questa Abilità puoi dichiarare di usare la Tempesta di Furia come tua unica Azione (3 Azioni).

Fai un unico Tiro per Colpire con -1d6 e se colpisci con il tuo attacco naturale causi un numero di danno critico pari a Lista Pugno Vuoto/4.

\feat{Testa cava}
\begin{description}[noitemsep, topsep=0pt, parsep=0pt, partopsep=0pt, leftmargin=0cm, labelwidth=2.5cm]
    \item[\textbf{Requisito}]: Lista Balestre 4
    \item[\textbf{Tiri Salvezza}]: +2 Tempra
    \item[\textbf{Caratteristica}]: Destrezza o Intelligenza
\end{description}

Riesci a imprimere un mortale effetto ai tuoi proiettili.

Il tuo dardo da balestra aumenta di una taglia di danno.

\feat{Tiro Preciso}
\begin{description}[noitemsep, topsep=0pt, parsep=0pt, partopsep=0pt, leftmargin=0cm, labelwidth=2.5cm]
    \item[\textbf{Requisito}]: Destrezza 3, Competenza Armi 1
    \item[\textbf{Tiri Salvezza}]: +2 Riflessi
    \item[\textbf{Caratteristica}]: Destrezza o Saggezza
\end{description}

Da vicino sai colpire dove fa male.

Guadagni +1 al Tiro per Colpire ed al danno, con armi da lancio quando il bersaglio è entro 9 metri.

%\begin{center}
%\includegraphics[width=0.7\linewidth]{immagini/kenilguerriero.png}
%\end{center}

\feat{Tiro Rapido}
\begin{description}[noitemsep, topsep=0pt, parsep=0pt, partopsep=0pt, leftmargin=0cm, labelwidth=2.5cm]
    \item[\textbf{Requisito}]: Destrezza 3, Tiro Preciso, Competenza Armi 2
    \item[\textbf{Tiri Salvezza}]: +2 Riflessi
    \item[\textbf{Caratteristica}]: Destrezza o Intelligenza
\end{description}

Non hai rivale nella precisione con cui scagli i tuoi proiettili.

Quando usi arco, balestre o lanci un arma le penalità per l'attacco multiplo sono inferiori.

Ogni proiettile lanciato oltre il primo prende un -4 al Tiro per Colpire cumulativo (e non il -5).

Il primo colpo ha un Tiro per Colpire normale, il secondo ha un -4, il terzo un -8 ...

\feat{Toccata e fuga}
\begin{description}[noitemsep, topsep=0pt, parsep=0pt, partopsep=0pt, leftmargin=0cm, labelwidth=2.5cm]
    \item[\textbf{Requisito}]: Destrezza 1, Competenza Armi 1
    \item[\textbf{Tiri Salvezza}]: +2 Riflessi
    \item[\textbf{Caratteristica}]: Destrezza o Intelligenza
\end{description}

Se nel round esegui almeno un attacco questi hanno una penalità base di -5 e puoi effettuare una Azione di movimento in più. Non è possibile eseguire in questa maniera più di una Azione di Movimento bonus. Costa una Azione Immediata.

\feat{Tutt'uno con la magia}
\begin{description}[noitemsep, topsep=0pt, parsep=0pt, partopsep=0pt, leftmargin=0cm, labelwidth=2.5cm]
    \item[\textbf{Requisito}]: Adepto della Magia
    \item[\textbf{Tiri Salvezza}]: +1 in due Tiri Salvezza a propria scelta
    \item[\textbf{Caratteristica}]: Modificatore di caratteristica per incantesimi o a scelta
\end{description}

Il tuo modificatore di caratteristica per incantesimi ha un +1 al valore per determinare gli effetti dell'incantesimo, Punti Magia, incantesimi conosciuti e livello massimo di incantesimo lanciabile.

\feat{Un braccio, un arma}
\begin{description}[noitemsep, topsep=0pt, parsep=0pt, partopsep=0pt, leftmargin=0cm, labelwidth=2.5cm]
    \item[\textbf{Requisito}]: Competenza Armi 2
    \item[\textbf{Tiri Salvezza}]: +1 Tempra, +1 Volontà
    \item[\textbf{Caratteristica}]: Forza o Costituzione
\end{description}

Scegli una Lista d'Armi. Il danno da Forza applicato dalle armi di quella lista aumenta di 1.

L’Abilità può essere presa più volte, con almeno CA 5,9,13.

Se prendi \textbf{4 volte} questa Abilità sulla stessa Lista d'Armi il bonus al danno si riduce a +2 ma tiri due volte il danno e scegli il risultato migliore. Non si applica sull'esplosione del danno o sul danno critico.

\feat{Un colpo un morto}
\begin{description}[noitemsep, topsep=0pt, parsep=0pt, partopsep=0pt, leftmargin=0cm, labelwidth=2.5cm]
    \item[\textbf{Requisito}]: Competenza Magica 1, Adepto della Magia 1
    \item[\textbf{Tiri Salvezza}]: +2 Riflessi
    \item[\textbf{Caratteristica}]: Modificatore di caratteristica per incantesimi o a scelta
\end{description}

La \textbf{prima volta} che prendi questa Abilità ottieni un +1 ai Tiri per Colpire agli incantesimi che richiedono un Tiro per Colpire.

La \textbf{seconda volta} il bonus al Tiro per Colpire per Incantesimi diventa +1 per ogni volta che hai preso l'Abilità Adepto della Magia. Non si cumula con il bonus preso la volta precedente.

\feat{Un solo corpo, una sola mente, un solo spirito}
\begin{description}[noitemsep, topsep=0pt, parsep=0pt, partopsep=0pt, leftmargin=0cm, labelwidth=2.5cm]
    \item[\textbf{Requisito}]: nessuno
    \item[\textbf{Tiri Salvezza}]:  +1 a scelta
    \item[\textbf{Caratteristica}]: nessuna
\end{description}

Assegnate un punto a Competenza Armi oppure Competenza Magica. Questa Abilità può essere presa al massimo 2 volte.

\feat{Uno con l'arco}
\begin{description}[noitemsep, topsep=0pt, parsep=0pt, partopsep=0pt, leftmargin=0cm, labelwidth=2.5cm]
    \item[\textbf{Requisito}]: Competenza Armi 4
    \item[\textbf{Tiri Salvezza}]: +1 Tempra, +1 Riflessi
    \item[\textbf{Caratteristica}]: Destrezza o Costituzione
\end{description}

A discapito del nome questa Abilità si applica a tutte le armi che lanciano proiettili.

La \textbf{prima volta} che prendi questa Abilità la penalità per scagliare proiettili mentre si usa un Arma da Lancio sotto minaccia diventa -2.

La \textbf{seconda volta} che prendi questa Abilità, Destrezza 3 e Competenza Armi 7, le penalità vengono annullate.

\feat{Un solo credo}
\begin{description}[noitemsep, topsep=0pt, parsep=0pt, partopsep=0pt, leftmargin=0cm, labelwidth=2.5cm]
    \item[\textbf{Requisito}]: Competenza Magica 2
    \item[\textbf{Tiri Salvezza}]: +1 Volontà, +1 Tempra
    \item[\textbf{Caratteristica}]: Modificatore di caratteristica per incantesimi
\end{description}

Il personaggio dedica la propria vita allo studio e perfezionamento di una sola Lista di Magia.

La \textbf{prima volta} che prendi questa Abilità devi scegliere una Lista di Magia \emph{preferita} e 3 \emph{opposte}.

Nella lista \emph{preferita} il costo di lancio degli incantesimi diminuisce di 1 rimanendo un costo minimo di 1. Per le 3 Liste di Magia \emph{opposte} il costo di lancio degli incantesimi aumenta di 1.

La \textbf{seconda volta} che prendi questa Abilità, requisito Competenza Magica 6, nella Lista di Magia \emph{preferita} le Prove di Magia vengono effettuate con 1d6 aggiuntivo e puoi scartare 1 dado dalla stessa.

La \textbf{terza volta} che prendi questa Abilità, requisito Competenza Magica 11, nella Lista di Magia \emph{preferita} quando fai una Prova di Magia conti un 6 in più rispetto a quanti tirati.

La \textbf{quarta volta} che prendi questa Abilità, requisito Competenza Magica 14, nella Lista di Magia \emph{preferita} puoi ritirare una volta la Prova di Magia in caso di fallimento critico.

La \textbf{quinta volta} che prendi questa Abilità, requisito Competenza Magica 17, nella Lista di Magia \emph{preferita} ogni qual volta devi tirare una Prova di Magia puoi non tirare e considerare di aver fatto un Successo Critico Magico.

La \textbf{sesta volta} che prendi questa Abilità, requisito Competenza Magica 20, nella Lista di Magia \emph{preferita} gli incantesimi inferiori al 4 livello non costano Punti Magia nella formulazione base.

\medskip

\textbf{Regole}:

\smallskip

\begin{itemize}[leftmargin=*] \setlength{\itemsep}{0pt}
\item Ogni volta che l'Abilità viene presa, oltre la prima, si devono selezionare due nuove Lista di Magia \emph{opposta} e il livello massimo lanciabile di incantesimi per tutte le Liste di magia \emph{opposte} diminuisce di 2.  La Lista di Magia Universale non è sceglibile tra le \emph{opposte}.

\item L'Abilità \emph{Un solo credo} non può essere presa assieme a: \hyperlink{figliounico}{Figlio Unico}, \hyperlink{magiepotenti}{Magie Potenti}, \hyperlink{specialista}{Specialista}.

\item Se usi l'Abilità \emph{Un solo credo} non puoi usare usare le \hyperlink{abilitadilista}{Abilità di Lista} (pag. \pageref{abilitadilista}).
\end{itemize}

\feat{Volonta' Ferrea}
\begin{description}[noitemsep, topsep=0pt, parsep=0pt, partopsep=0pt, leftmargin=0cm, labelwidth=2.5cm]
    \item[\textbf{Requisito}]: Saggezza 0
    \item[\textbf{Tiri Salvezza}]: nessuno
    \item[\textbf{Caratteristica}]: Saggezza o Carisma
\end{description}

Nel tempo hai allenato la tua volontà per resistere a qualsiasi debolezza e paura.

La \textbf{prima volta} prendi questa Abilità ottieni un bonus di +2 ai Tiri Salvezza su Volontà. Il bonus è cumulativo, +2 la prima volta, +1 la \textbf{seconda volta}, +1 la \textbf{terza} volta.

La \textbf{quarta volta} che prendi questa Abilità puoi decidere di riuscire automaticamente in un Tiro Salvezza su Volontà una volta al giorno prima di aver tirato i dadi.

%Superumano: una serie di abilita' a scelta, quando poi prendi due volte sbocchi altre, quando 3 volte sblocchi altre..
% 1 Duro da soggiogare : +2 Tiro Salvezza su Volontà su incantesimi della Lista Ammaliamento.
% 1 Arcobaleno : sei un artista. Le tue ditaspontaneamente producono colore
% 1 Consumi ridotti : bevi e mangi la metà di unuomo normale. Sei sotto peso
% 1 Super piastrine : Riduci il danno da Sanguinamento di 1 a fine di ogni round
% 1 Direzione Assoluta : sai sempre dove è il nord magnetico. Hai un +1d6 alle prove di orientamento.
% 1 Il fegato non si conta 10: puoi bere tanto e non ti ubriachi
% 1 Mano Piede palmata 5: +1d6 alle prove di nuotare
% 2 Lento e Fermo 5: Sei eccezionalmente stabile sui tuoi piedi. Non puoi essere mosso o sollevato se non da una creatura di 2 taglie superiori
% 2 Empatia Animale 10: +1d6 alle prove per gestire gli animali (anche selvaggi).
% 2 Polmoni di ferro 5: puoi trattenere il respiro 2*Costituzione minuti (minimo 2 minuti)
% 2 Magnetico 5-10: sprigioni luce quando vuoi. per fortuna non letteralmente. ±1/2 alle prove basate sul Carisma.
% 4 Vedere l’invisibile 15: Meglio la vista a raggi X
% 5 Rigenerazione 30: +1 Punti Ferita per Turno (non rigeneri arti)
% 5 Senso delle vibrazioni (Senso Tellurico) 30: tutto fa tremare un poco la terra, o quasi, raggio di 18 metri intorno a te.
% 5 Anfibio : puoi respirare sia sott’acqua che l’aria
% 6 Parlare con gli animali 20: scegli una famiglia (ovini, marsupiali, caviette..)
% 7 Parlare con le piante 25: ho sempre voluto parlare con le zucchine
% 8 Rigenerazione veloce 40: +1 Punti Ferita per round (non rigeneri arti). Muori se distruggono il tuo corpo (o non rimane che cenere
% 6 Tocco gelido 10: Toccando un morto (entro 1 giorno per livello) puoi vedere e sentire cosa è successo nel suo ultimo round di vita.

\end{multicols}




\vfill

\begin{center}

\includegraphics[width=0.9\linewidth]{immagini/Granblue.Fantasy.full.2108782.png}

%%\filltopageendgraphics[width=0.7\linewidth]{immagini/Granblue.Fantasy.full.2108782.png}

\emph{Attorno al fuoco, raccontando la giornata trascorsa.}
\end{center}


\bigskip

\begin{enfasi}

	L'eccellenza non è un atto, ma un'abitudine. (Etica Nicomachea, Aristotele)

	\medskip

	Avere grandi poteri non significa saperli usare saggiamente. (Earthsea, Ursula K. Le Guin)

	\medskip

	Il potere senza saggezza è la più pericolosa delle combinazioni. (Elric di Melnibonè Michael Moorcock)

\end{enfasi}


\pagebreak

\subsection{Raggruppamento Abilita' per Stile}

Per facilitare la transizione da chi viene da altri giochi di ruolo con classi sono qui suddivise le Abilità per le classi più canoniche. Sono chiaramente solo indicazioni, in OBSS il personaggio può essere costruito come meglio si preferisce e come la storia che vive lo sta istruendo.

\begin{multicols}{3}

{\small

\begin{flushleft}

\titlespacing*{\subsubsection}{0pt}{0.5em}{0.5em}\subsubsection*{Guerriero}

\hyperlink{Allungo}{Allungo}\\
\hyperlink{Arma Focalizzata}{Arma Focalizzata}\\
\hyperlink{Artista dell'Arma}{Artista dell'Arma}\\
\hyperlink{Attacco Turbinante}{Attacco Turbinante}\\
\hyperlink{Colpi Poderosi}{Colpi Poderosi}\\
\hyperlink{Combattere alla Cieca}{Combattere alla Cieca}\\
\hyperlink{Combattimento con due armi}{Combattimento con due armi}\\
\hyperlink{Difesa pronta}{Difesa pronta}\\
\hyperlink{Flagello Danzante}{Flagello Danzante}\\
\hyperlink{Iaijutsu}{Iaijutsu}\\
\hyperlink{Iniziativa migliorata}{Iniziativa migliorata}\\
\hyperlink{La mia pelle}{La mia pelle}\\
\hyperlink{Lesto}{Lesto}\\
\hyperlink{Parata}{Parata}\\
\hyperlink{Pelle Coriacea}{Pelle Coriacea}\\
\hyperlink{Primo Sangue}{Primo Sangue}\\
\hyperlink{Proseguire}{Proseguire}\\
\hyperlink{Questa è la mia arma!}{Questa è la mia arma!}\\
\hyperlink{Resistenza della pietra}{Resistenza della pietra}\\
\hyperlink{Riflessi fulminei}{Riflessi fulminei}\\
\hyperlink{Robusto}{Robusto}\\
\hyperlink{Seconda pelle}{Seconda pelle}\\
\hyperlink{Stai giù!}{Stai giù!}\\
\hyperlink{Un braccio, un arma}{Un braccio, un arma}

\hyperlink{Armato}{Armato}\\
\hyperlink{Colpo Mortale}{Colpo Mortale}\\
\hyperlink{Conoscenza istintiva}{Conoscenza istintiva}\\
\hyperlink{Daredevil}{Daredevil}\\
\hyperlink{Duro a morire}{Duro a morire}\\
\hyperlink{Ferocia}{Ferocia}\\
\hyperlink{Forgiato nella furia}{Forgiato nella furia}\\
\hyperlink{Furia}{Furia}\\
\hyperlink{Ho detto CADI!}{Ho detto CADI!}\\
\hyperlink{La mia morte la tua morte}{La mia morte la tua morte}\\
\hyperlink{La mia Testa è più Dura}{La mia Testa è più Dura}\\
\hyperlink{Colosso}{Colosso}\\
\hyperlink{Pelle Coriacea}{Pelle Coriacea}\\
\hyperlink{Persona veramente malvagia}{Persona veramente malvagia}\\
\hyperlink{più sono grossi più fanno rumore quando cadono}{più sono grossi più fanno rumore quando cadono}\\
\hyperlink{Primo Sangue}{Primo Sangue}\\
\hyperlink{Recupero}{Recupero}\\
\hyperlink{Rinoceronte}{Rinoceronte}\\
\hyperlink{Un braccio, un arma}{Un braccio, un arma}\\
\hyperlink{Volonta' Ferrea}{Volonta' Ferrea}

\titlespacing*{\subsubsection}{0pt}{0.5em}{0.5em}\subsubsection*{Ladro}

\hyperlink{Colpo Furtivo}{Colpo Furtivo}\\
\hyperlink{Colpo Indebolente}{Colpo Indebolente}\\
\hyperlink{Colpo Paralizzante}{Colpo Paralizzante}\\
\hyperlink{Estrazione rapida}{Estrazione rapida}\\
\hyperlink{Difesa pronta}{Difesa pronta}\\
\hyperlink{Fare Infuriare}{Fare Infuriare}\\
\hyperlink{Freccia chiamata, freccia consegnata}{Freccia chiamata, freccia consegnata}\\
\hyperlink{Giocoliere}{Giocoliere}\\
\hyperlink{Improvvisare}{Improvvisare}\\
\hyperlink{Lesto}{Lesto}\\
\hyperlink{Occhio Clinico}{Occhio Clinico}\\
\hyperlink{Opportunista}{Opportunista}\\
\hyperlink{Passo Felpato}{Passo Felpato}\\
\hyperlink{Percettivo}{Percettivo}\\
\hyperlink{Questo è il mio pugnale}{Questo è il mio pugnale}\\
\hyperlink{Schivare trappole}{Schivare trappole}\\
\hyperlink{Schivata prodigiosa}{Schivata prodigiosa}\\
\hyperlink{Toccata e fuga}{Toccata e fuga}

\titlespacing*{\subsubsection}{0pt}{0.5em}{0.5em}\subsubsection*{Paladino}

\hyperlink{Armatura del Devoto}{Armatura del Devoto}\\
\hyperlink{Il Patrono è la mia Arma}{Il Patrono è la mia Arma}\\
\hyperlink{Imposizione delle mani}{Imposizione delle mani}\\
\hyperlink{Incanalare Energia}{Incanalare Energia}\\
\hyperlink{Lo scudo è mio amico}{Lo scudo è mio amico}\\
\hyperlink{Muro mentale}{Muro mentale}\\
\hyperlink{Rappresaglia}{Rappresaglia}\\

\titlespacing*{\subsubsection}{0pt}{0.5em}{0.5em}\subsubsection*{Bardo}

\hyperlink{Dadi Truccati}{Dadi Truccati}\\
\hyperlink{Danno Coordinato}{Danno Coordinato}\\
\hyperlink{Danza della Lama}{Danza della Lama}\\
\hyperlink{Esperto}{Esperto}\\
\hyperlink{Fare Infuriare}{Fare Infuriare}\\
\hyperlink{Figlio Unico}{Figlio Unico}\\
\hyperlink{Fortunato}{Fortunato}\\
\hyperlink{Guerriero della Magia}{Guerriero della Magia}\\
\hyperlink{Incantatore da Combattimento}{Incantatore da Combattimento}\\
\hyperlink{Infondere Coraggio}{Infondere Coraggio}\\
\hyperlink{Infondere Energia Magica}{Infondere Energia Magica}\\
\hyperlink{Infondere Energia Magica Superiore}{Infondere Energia Magica Superiore}\\
\hyperlink{Infondere Paura}{Infondere Paura}\\
\hyperlink{Litania versatile}{Litania versatile}\\
\hyperlink{Poliglotta}{Poliglotta}\\
\hyperlink{Radici magiche}{Radici magiche}\\
\hyperlink{Tattico}{Tattico}\\
\hyperlink{Sfortunato}{Sfortunato}

\titlespacing*{\subsubsection}{0pt}{0.5em}{0.5em}\subsubsection*{Ranger}

\hyperlink{Combattimento con due armi}{Combattimento con due armi}\\
\hyperlink{Conoscenza istintiva}{Conoscenza istintiva}\\
\hyperlink{Difendere Cavalcatura}{Difendere Cavalcatura}\\
\hyperlink{Doppia porzione}{Doppia porzione}\\
\hyperlink{Freccia chiamata, freccia consegnata}{Freccia chiamata, freccia consegnata}\\
\hyperlink{Occhio di Falco}{Occhio di Falco}\\
\hyperlink{Passo Sicuro}{Passo Sicuro}\\
\hyperlink{Passo rapido}{Passo rapido}\\
\hyperlink{Precisino}{Precisino}\\
\hyperlink{più sono grossi più fanno rumore quando cadono}{più sono grossi più fanno rumore quando cadono}\\
\hyperlink{Segugio}{Segugio}\\
\hyperlink{Senza Traccia}{Senza Traccia}\\
\hyperlink{Spara e Scappa}{Spara e Scappa}\\
\hyperlink{Testa cava}{Testa cava}\\
\hyperlink{Tiro Preciso}{Tiro Preciso}\\
\hyperlink{Tiro Rapido}{Tiro Rapido}\\
\hyperlink{Uno con l'arco}{Uno con l'arco}

\titlespacing*{\subsubsection}{0pt}{0.5em}{0.5em}\subsubsection*{Druido}

\hyperlink{Adepto della Magia}{Adepto della Magia}\\
\hyperlink{Animalia}{Animalia}\\
\hyperlink{Distillare pozioni}{Distillare pozioni}\\
\hyperlink{Elementalista}{Elementalista}\\
\hyperlink{Figlia di Shayalia}{Figlia di Shayalia}\\
\hyperlink{Forma Elementale}{Forma Elementale}\\
\hyperlink{Il Patrono è con me}{Il Patrono è con me}\\
\hyperlink{Sangue Puro}{Sangue Puro}

\titlespacing*{\subsubsection}{0pt}{0.5em}{0.5em}\subsubsection*{Chierico}

\hyperlink{Adepto della Magia}{Adepto della Magia}\\
\hyperlink{Armatura del Devoto}{Armatura del Devoto}\\
\hyperlink{Dattilografo}{Dattilografo}\\
\hyperlink{Fedele}{Fedele}\\
\hyperlink{Guaritore}{Guaritore}\\
\hyperlink{Il Patrono è con me}{Il Patrono è con me}\\
\hyperlink{Il Patrono è la mia Arma}{Il Patrono è la mia Arma}\\
\hyperlink{Imposizione delle mani}{Imposizione delle mani}\\
\hyperlink{Incanalare Energia}{Incanalare Energia}\\
\hyperlink{Potere del Patrono}{Potere del Patrono}\\
\hyperlink{Scacciare i non morti}{Scacciare i non morti}\\
\hyperlink{Specialista}{Specialista}\\
\hyperlink{Tutt'uno con la magia}{Tutt'uno con la magia}

\titlespacing*{\subsubsection}{0pt}{0.5em}{0.5em}\subsubsection*{Mago/Stregone}

\hyperlink{Adepto della Magia}{Adepto della Magia}\\
\hyperlink{Animaletto / Famiglio}{Animaletto / Famiglio}\\
\hyperlink{Batteria Magica}{Batteria Magica}\\
\hyperlink{Batteria Estesa}{Batteria Estesa}\\
\hyperlink{Concentrato}{Concentrato}\\
\hyperlink{Creare Oggetti Magici}{Creare Oggetti Magici}\\
\hyperlink{Dadi Truccati}{Dadi Truccati}\\
\hyperlink{Dattilografo}{Dattilografo}\\
\hyperlink{Decifrare scritti magici}{Decifrare scritti magici}\\
\hyperlink{Elementalista}{Elementalista}\\
\hyperlink{Figlio di Tazher}{Figlio di Tazher}\\
\hyperlink{Incantatore Prudente}{Incantatore Prudente}\\
\hyperlink{Il Patrono è con me}{Il Patrono è con me}\\
\hyperlink{Magie Potenti}{Magie Potenti}\\
\hyperlink{Occhi della magia}{Occhi della magia}\\
\hyperlink{Prodigioso}{Prodigioso}\\
\hyperlink{Robusto}{Robusto}\\
\hyperlink{Sapiente}{Sapiente}\\
\hyperlink{Sifone Nero}{Sifone Nero}\\
\hyperlink{Specialista}{Specialista}\\
\hyperlink{Tutt'uno con la magia}{Tutt'uno con la magia}\\
\hyperlink{Un colpo un morto}{Un colpo un morto}\\
\hyperlink{Un solo credo}{Un solo credo}

\titlespacing*{\subsubsection}{0pt}{0.5em}{0.5em}\subsubsection*{Monaco}

\hyperlink{Ali della Fenice}{Ali della Fenice}\\
\hyperlink{Armatura della Montagna Incantata}{Armatura della Montagna Incantata}\\
\hyperlink{Colpo Psichico}{Colpo Psichico}\\
\hyperlink{Energia Psichica}{Energia Psichica}\\
\hyperlink{Finta Morte}{Finta Morte}\\
\hyperlink{Gru d'Argento}{Gru d'Argento}\\
\hyperlink{Immunita' ai veleni}{Immunita' ai veleni}\\
\hyperlink{Molla}{Molla}\\
\hyperlink{Colosso}{Colosso}\\
\hyperlink{Muro mentale}{Muro mentale}\\
\hyperlink{Passo rapido}{Passo rapido}\\
\hyperlink{Pugno di Ferro}{Pugno di Ferro}\\
\hyperlink{Pugno Potente}{Pugno Potente}\\
\hyperlink{Raggio Psichico}{Raggio Psichico}\\
\hyperlink{Tempesta di Furia}{Tempesta di Furia}\\
\hyperlink{Volonta' Ferrea}{Volonta' Ferrea}

\end{flushleft}
}

\end{multicols}

\subsection{Esempi di costruzione del personaggio}

Sono qui presentati alcuni esempi di personaggi secondo i canoni standard fantasy, prendeteli come uno spunto per costruire i vostri personaggi. Ricordate di aggiungere il valore della Costituzione per livello ai Punti Ferita.

{\small


\begin{multicols}{2}


\textbf{Guerriero Tank}

\noindent\begin{tabularx}{\linewidth}{c|>{\hsize=0.08\hsize}X>{\hsize=0.08\hsize}X>{\hsize=0.33\hsize}X|X|}
	\toprule
 \rowcolor{gray!20}	\textbf{Lv} & \multicolumn{3}{c|}{\textbf{Guerriero Tank}} & \textbf{Abilità} \\
& \centering\arraybackslash \textbf{CA} & \centering\arraybackslash \textbf{CM} & \centering\arraybackslash \textbf{PF} & \\
	\toprule
	1 &1	& 0	&	1d6+3	&\hyperlink{La mia pelle}{La mia pelle} - \hyperlink{Seconda pelle}{Seconda pelle}\\
 \rowcolor{gray!20}2	&	2	& 0	&	2d6+6	&\hyperlink{Difesa pronta}{Difesa pronta}\\
	3	&	3	& 0	&	3d6+9	&\hyperlink{Parata}{Parata}\\
 \rowcolor{gray!20}4	&	4	& 0	&	4d6+12	&\hyperlink{Pelle Coriacea}{Pelle Coriacea}\\
	5	&	5	& 0	&	5d6+15	&\hyperlink{Opportunista}{Opportunista}\\
 \rowcolor{gray!20}6	&	6	& 0	&	6d6+18	&\hyperlink{Iniziativa migliorata}{Iniziativa migliorata}\\
	7	&	7	& 0	&	7d6+21	&\hyperlink{Resistenza della pietra}{Resistenza della pietra}\\
 \rowcolor{gray!20}8	&	8	& 0	&	8d6+24	&\\
	9	&	9	& 0	&	9d6+27	&\hyperlink{Seconda pelle}{Seconda pelle} (2°)\\
 \rowcolor{gray!20}10	&	10	& 0	&	10d6+30	&\hyperlink{Robusto}{Robusto}\\
	11	&	11	& 0	&	11d6+33	&\\
 \rowcolor{gray!20}12	&	12	& 0	&	12d6+36	&\hyperlink{Pelle Coriacea}{Pelle Coriacea} (2°)\\
	13	&	13	& 0	&	13d6+39	&\hyperlink{Volontà Ferrea}{Volontà Ferrea}\\
 \rowcolor{gray!20}14	&	14	& 0	&	14d6+42	&\\
	15	&	15	& 0	&	15d6+45	&\hyperlink{Seconda pelle}{Seconda pelle} (3°)\\
 \rowcolor{gray!20}16	&	16	& 0	&	16d6+48	&\hyperlink{Duro a morire}{Duro a morire}\\
	17	&	17	& 0	&	17d6+51	&\\
 \rowcolor{gray!20}18	&	18	& 0	&	18d6+54	&\hyperlink{Riflessi fulminei}{Riflessi fulminei}\\
	19	&	19	& 0	&	19d6+57	&\\
 \rowcolor{gray!20}20	&	20	& 0	&	20d6+60	&\hyperlink{Colosso}{Colosso}\\
\end{tabularx}

\textbf{Barbaro}

\noindent\begin{tabularx}{\linewidth}{c|>{\hsize=0.08\hsize}X>{\hsize=0.08\hsize}X>{\hsize=0.33\hsize}X|X|}
	\toprule
 \rowcolor{gray!20}	\textbf{Lv} & \multicolumn{3}{c|}{\textbf{Barbaro}} & \textbf{Abilità} \\
& \centering\arraybackslash \textbf{CA} & \centering\arraybackslash \textbf{CM} & \centering\arraybackslash \textbf{PF} & \\
	\toprule
	1 &1	& 0	&	1d6+3	&\hyperlink{Ferocia}{Ferocia} - \hyperlink{Furia}{Furia}\\
 \rowcolor{gray!20}2	&	2	& 0	&	2d6+6	&\hyperlink{Colpi Poderosi}{Colpi Poderosi}\\
	3	&	3	& 0	&	3d6+9	&\hyperlink{Proseguire}{Proseguire}\\
 \rowcolor{gray!20}4	&	4	& 0	&	4d6+12	&\hyperlink{Ferocia}{Ferocia} (2°)\\
	5	&	5	& 0	&	5d6+15	&\hyperlink{Un braccio, un arma}{Un braccio, un arma}\\
 \rowcolor{gray!20}6	&	6	& 0	&	6d6+18	&\hyperlink{La mia morte la tua morte}{La mia morte la tua morte}\\
	7	&	7	& 0	&	7d6+21	&\hyperlink{Ferocia}{Ferocia} (3°)\\
 \rowcolor{gray!20}8	&	8	& 0	&	8d6+24	&\\
	9	&	9	& 0	&	9d6+27	&\hyperlink{Proseguire}{Proseguire} (2°)\\
 \rowcolor{gray!20}10	&	10	& 0	&	10d6+30	&\hyperlink{Primo Sangue}{Primo Sangue}\\
	11	&	11	& 0	&	11d6+33	&\\
 \rowcolor{gray!20}12	&	12	& 0	&	12d6+36	&\hyperlink{Ferocia}{Ferocia} (4°)\\
	13	&	13	& 0	&	13d6+39	&\hyperlink{Un braccio, un arma}{Un braccio, un arma} (2°)\\
 \rowcolor{gray!20}14	&	14	& 0	&	14d6+42	&\\
	15	&	15	& 0	&	15d6+45	&\hyperlink{Più sono grossi più fanno rumore quando cadono}{Più sono grossi più fanno rumore quando cadono}\\
 \rowcolor{gray!20}16	&	16	& 0	&	16d6+48	&\hyperlink{Forgiato nella furia}{Forgiato nella furia}\\
	17	&	17	& 0	&	17d6+51	&\\
 \rowcolor{gray!20}18	&	18	& 0	&	18d6+54	&\hyperlink{Persona veramente malvagia}{Persona veramente malvagia}\\
	19	&	19	& 0	&	19d6+57	&\\
 \rowcolor{gray!20}20	&	20	& 0	&	20d6+60	&\hyperlink{Colosso}{Colosso}\\
\end{tabularx}

\columnbreak

\textbf{Spadaccino}

\noindent\begin{tabularx}{\linewidth}{c|>{\hsize=0.08\hsize}X>{\hsize=0.08\hsize}X>{\hsize=0.33\hsize}X|X|}
	\toprule
 \rowcolor{gray!20}	\textbf{Lv} & \multicolumn{3}{c|}{\textbf{Spadaccino}} & \textbf{Abilità} \\
& \centering\arraybackslash \textbf{CA} & \centering\arraybackslash \textbf{CM} & \centering\arraybackslash \textbf{PF} & \\
	\toprule
	1 &1	& 0	&	1d6+3	&\hyperlink{Arma Focalizzata}{Arma Focalizzata} - \hyperlink{Danza della Lama}{Danza della Lama}\\
 \rowcolor{gray!20}2	&	2	& 0	&	2d6+6	&\hyperlink{Iniziativa migliorata}{Iniziativa migliorata}\\
	3	&	3	& 0	&	3d6+9	&\hyperlink{Parata}{Parata}\\
 \rowcolor{gray!20}4	&	4	& 0	&	4d6+12	&\hyperlink{Estrazione rapida}{Estrazione rapida}\\
	5	&	5	& 0	&	5d6+15	&\hyperlink{Iaijutsu}{Iaijutsu}\\
 \rowcolor{gray!20}6	&	6	& 0	&	6d6+18	&\hyperlink{Artista dell'Arma}{Artista dell'Arma}\\
	7	&	7	& 0	&	7d6+21	&\hyperlink{Danza della Lama}{Danza della Lama} (2°)\\
 \rowcolor{gray!20}8	&	8	& 0	&	8d6+24	&\\
	9	&	9	& 0	&	9d6+27	&\hyperlink{Parata}{Parata} (2°)\\
 \rowcolor{gray!20}10	&	10	& 0	&	10d6+30	&\hyperlink{Schivata prodigiosa}{Schivata prodigiosa}\\
	11	&	11	& 0	&	11d6+33	&\\
 \rowcolor{gray!20}12	&	12	& 0	&	12d6+36	&\hyperlink{Iaijutsu}{Iaijutsu} (2°)\\
	13	&	13	& 0	&	13d6+39	&\hyperlink{Artista dell'Arma}{Artista dell'Arma} (2°)\\
 \rowcolor{gray!20}14	&	14	& 0	&	14d6+42	&\\
	15	&	15	& 0	&	15d6+45	&\hyperlink{Danza della Lama}{Danza della Lama} (3°)\\
 \rowcolor{gray!20}16	&	16	& 0	&	16d6+48	&\hyperlink{Riflessi fulminei}{Riflessi fulminei}\\
	17	&	17	& 0	&	17d6+51	&\\
 \rowcolor{gray!20}18	&	18	& 0	&	18d6+54	&\hyperlink{Iaijutsu}{Iaijutsu} (3°)\\
	19	&	19	& 0	&	19d6+57	&\\
 \rowcolor{gray!20}20	&	20	& 0	&	20d6+60	&\hyperlink{Attacco Turbinante}{Attacco Turbinante}\\
\end{tabularx}

\textbf{Ladro Acrobata}

\noindent\begin{tabularx}{\linewidth}{c|>{\hsize=0.08\hsize}X>{\hsize=0.08\hsize}X>{\hsize=0.33\hsize}X|X|}
	\toprule
 \rowcolor{gray!20}	\textbf{Lv} & \multicolumn{3}{c|}{\textbf{Ladro Acrobata}} & \textbf{Abilità} \\
& \centering\arraybackslash \textbf{CA} & \centering\arraybackslash \textbf{CM} & \centering\arraybackslash \textbf{PF} & \\
	\toprule
	1 &1	& 0	&	1d6+3	&\hyperlink{Estrazione rapida}{Estrazione rapida} - \hyperlink{Passo Felpato}{Passo Felpato}\\
 \rowcolor{gray!20}2	&	2	& 0	&	2d6+6	&\hyperlink{Opportunista}{Opportunista}\\
	3	&	3	& 0	&	3d6+9	&\hyperlink{Colpo Furtivo}{Colpo Furtivo}\\
 \rowcolor{gray!20}4	&	4	& 0	&	4d6+12	&\hyperlink{Schivare trappole}{Schivare trappole}\\
	5	&	5	& 0	&	5d6+15	&\hyperlink{Occhio Clinico}{Occhio Clinico}\\
 \rowcolor{gray!20}6	&	6	& 0	&	6d6+18	&\hyperlink{Colpo Furtivo}{Colpo Furtivo} (2°)\\
	7	&	7	& 0	&	7d6+21	&\hyperlink{Percettivo}{Percettivo}\\
 \rowcolor{gray!20}8	&	8	& 0	&	8d6+24	&\\
	9	&	9	& 0	&	9d6+27	&\hyperlink{Passo Felpato}{Passo Felpato} (2°)\\
 \rowcolor{gray!20}10	&	10	& 0	&	10d6+30	&\hyperlink{Schivata prodigiosa}{Schivata prodigiosa}\\
	11	&	11	& 0	&	11d6+33	&\\
 \rowcolor{gray!20}12	&	12	& 0	&	12d6+36	&\hyperlink{Colpo Furtivo}{Colpo Furtivo} (3°)\\
	13	&	13	& 0	&	13d6+39	&\hyperlink{Schivare trappole}{Schivare trappole} (2°)\\
 \rowcolor{gray!20}14	&	14	& 0	&	14d6+42	&\\
	15	&	15	& 0	&	15d6+45	&\hyperlink{Colpo Furtivo}{Colpo Furtivo} (4°)\\
 \rowcolor{gray!20}16	&	16	& 0	&	16d6+48	&\hyperlink{Toccata e fuga}{Toccata e fuga}\\
	17	&	17	& 0	&	17d6+51	&\\
 \rowcolor{gray!20}18	&	18	& 0	&	18d6+54	&\hyperlink{Colpo Indebolente}{Colpo Indebolente}\\
	19	&	19	& 0	&	19d6+57	&\\
 \rowcolor{gray!20}20	&	20	& 0	&	20d6+60	&\hyperlink{Colpo Paralizzante}{Colpo Paralizzante}\\
\end{tabularx}

\columnbreak

\textbf{Assassino}

\noindent\begin{tabularx}{\linewidth}{c|>{\hsize=0.08\hsize}X>{\hsize=0.08\hsize}X>{\hsize=0.33\hsize}X|X|}
	\toprule
 \rowcolor{gray!20}	\textbf{Lv} & \multicolumn{3}{c|}{\textbf{Assassino}} & \textbf{Abilità} \\
& \centering\arraybackslash \textbf{CA} & \centering\arraybackslash \textbf{CM} & \centering\arraybackslash \textbf{PF} & \\
	\toprule
	1 &1	& 0	&	1d6+3	&\hyperlink{Colpo Furtivo}{Colpo Furtivo} - \hyperlink{Passo Felpato}{Passo Felpato}\\
 \rowcolor{gray!20}2	&	2	& 0	&	2d6+6	&\hyperlink{Estrazione rapida}{Estrazione rapida}\\
	3	&	3	& 0	&	3d6+9	&\hyperlink{Opportunista}{Opportunista}\\
 \rowcolor{gray!20}4	&	4	& 0	&	4d6+12	&\hyperlink{Questo è il mio pugnale}{Questo è il mio pugnale}\\
	5	&	5	& 0	&	5d6+15	&\hyperlink{Schivare trappole}{Schivare trappole}\\
 \rowcolor{gray!20}6	&	6	& 0	&	6d6+18	&\hyperlink{Colpo Furtivo}{Colpo Furtivo} (2°)\\
	7	&	7	& 0	&	7d6+21	&\hyperlink{Percettivo}{Percettivo}\\
 \rowcolor{gray!20}8	&	8	& 0	&	8d6+24	&\\
	9	&	9	& 0	&	9d6+27	&\hyperlink{Passo Felpato}{Passo Felpato} (2°)\\
 \rowcolor{gray!20}10	&	10	& 0	&	10d6+30	&\hyperlink{Occhio Clinico}{Occhio Clinico}\\
	11	&	11	& 0	&	11d6+33	&\\
 \rowcolor{gray!20}12	&	12	& 0	&	12d6+36	&\hyperlink{Colpo Furtivo}{Colpo Furtivo} (3°)\\
	13	&	13	& 1	&	13d6+39	&\hyperlink{Adepto della Magia}{Adepto della Magia}\\
 \rowcolor{gray!20}14	&	14	& 1	&	14d6+42	&\\
	15	&	15	& 1	&	15d6+45	&\hyperlink{Colpo Furtivo}{Colpo Furtivo} (4°)\\
 \rowcolor{gray!20}16	&	15	& 2	&	16d6+45	&\hyperlink{Incantatore da Combattimento}{Incantatore da Combattimento}\\
	17	&	15	& 3	&	17d6+45	&\\
 \rowcolor{gray!20}18	&	15	& 4	&	18d6+45	&\hyperlink{Colpo Indebolente}{Colpo Indebolente}\\
	19	&	15	& 5	&	19d6+45	&\\
 \rowcolor{gray!20}20	&	15	& 5	&	20d6+45	&\hyperlink{Colpo Paralizzante}{Colpo Paralizzante}\\
\end{tabularx}

\textbf{Ranger Esploratore}

\noindent\begin{tabularx}{\linewidth}{c|>{\hsize=0.08\hsize}X>{\hsize=0.08\hsize}X>{\hsize=0.33\hsize}X|X|}
	\toprule
 \rowcolor{gray!20}	\textbf{Lv} & \multicolumn{3}{c|}{\textbf{Ranger Esploratore}} & \textbf{Abilità} \\
& \centering\arraybackslash \textbf{CA} & \centering\arraybackslash \textbf{CM} & \centering\arraybackslash \textbf{PF} & \\
	\toprule
	1 &1	& 0	&	1d6+3	&\hyperlink{Passo Sicuro}{Passo Sicuro} - \hyperlink{Tiro Preciso}{Tiro Preciso}\\
 \rowcolor{gray!20}2	&	2	& 0	&	2d6+6	&\hyperlink{Segugio}{Segugio}\\
	3	&	3	& 0	&	3d6+9	&\hyperlink{Occhio di Falco}{Occhio di Falco}\\
 \rowcolor{gray!20}4	&	4	& 0	&	4d6+12	&\hyperlink{Passo rapido}{Passo rapido}\\
	5	&	5	& 0	&	5d6+15	&\hyperlink{Uno con l'arco}{Uno con l'arco}\\
 \rowcolor{gray!20}6	&	6	& 0	&	6d6+18	&\hyperlink{Precisino}{Precisino}\\
	7	&	7	& 0	&	7d6+21	&\hyperlink{Passo Sicuro}{Passo Sicuro} (specializzazione)\\
 \rowcolor{gray!20}8	&	8	& 0	&	8d6+24	&\\
	9	&	9	& 0	&	9d6+27	&\hyperlink{Senza Traccia}{Senza Traccia}\\
 \rowcolor{gray!20}10	&	10	& 0	&	10d6+30	&\hyperlink{Combattere alla Cieca}{Combattere alla Cieca}\\
	11	&	11	& 0	&	11d6+33	&\\
 \rowcolor{gray!20}12	&	12	& 0	&	12d6+36	&\hyperlink{Tiro Rapido}{Tiro Rapido}\\
	13	&	13	& 1	&	13d6+39	&\hyperlink{Adepto della Magia}{Adepto della Magia}\\
 \rowcolor{gray!20}14	&	14	& 1	&	14d6+42	&\\
	15	&	15	& 1	&	15d6+45	&\hyperlink{Segugio}{Segugio} (2°)\\
 \rowcolor{gray!20}16	&	15	& 2	&	16d6+45	&\hyperlink{Distillare pozioni}{Distillare pozioni}\\
	17	&	15	& 3	&	17d6+45	&\\
 \rowcolor{gray!20}18	&	15	& 4	&	18d6+45	&\hyperlink{Occhio di Falco}{Occhio di Falco} (2°)\\
	19	&	15	& 5	&	19d6+45	&\\
 \rowcolor{gray!20}20	&	15	& 5	&	20d6+45	&\hyperlink{Uno con l'arco}{Uno con l'arco} (2°)\\
\end{tabularx}

\columnbreak

\textbf{Ranger Bestiale}

\noindent\begin{tabularx}{\linewidth}{c|>{\hsize=0.08\hsize}X>{\hsize=0.08\hsize}X>{\hsize=0.33\hsize}X|X|}
	\toprule
 \rowcolor{gray!20}	\textbf{Lv} & \multicolumn{3}{c|}{\textbf{Ranger Bestiale}} & \textbf{Abilità} \\
& \centering\arraybackslash \textbf{CA} & \centering\arraybackslash \textbf{CM} & \centering\arraybackslash \textbf{PF} & \\
	\toprule
	1 &1	& 0	&	1d6+3	&\hyperlink{Animaletto / Famiglio}{Animaletto / Famiglio} - \hyperlink{Passo Sicuro}{Passo Sicuro}\\
 \rowcolor{gray!20}2	&	2	& 0	&	2d6+6	&\hyperlink{Segugio}{Segugio}\\
	3	&	2	& 1	&	3d6+6	&\hyperlink{Adepto della Magia}{Adepto della Magia}\\
 \rowcolor{gray!20}4	&	3	& 1	&	4d6+9	&\hyperlink{Animaletto / Famiglio}{Animaletto / Famiglio} (2°)\\
	5	&	3	& 2	&	5d6+9	&\hyperlink{Animalia}{Animalia}\\
 \rowcolor{gray!20}6	&	4	& 2	&	6d6+12	&\hyperlink{Tiro Preciso}{Tiro Preciso}\\
	7	&	4	& 3	&	7d6+12	&\hyperlink{Figlia di Shayalia}{Figlia di Shayalia}\\
 \rowcolor{gray!20}8	&	5	& 3	&	8d6+15	&\\
	9	&	5	& 4	&	9d6+15	&\hyperlink{Passo Sicuro}{Passo Sicuro} (specializzazione)\\
 \rowcolor{gray!20}10	&	6	& 4	&	10d6+18	&\hyperlink{Occhio di Falco}{Occhio di Falco}\\
	11	&	6	& 5	&	11d6+18	&\\
 \rowcolor{gray!20}12	&	7	& 5	&	12d6+21	&\hyperlink{Animalia}{Animalia} (2°)\\
	13	&	7	& 6	&	13d6+21	&\hyperlink{Segugio}{Segugio} (2°)\\
 \rowcolor{gray!20}14	&	8	& 6	&	14d6+24	&\\
	15	&	8	& 7	&	15d6+24	&\hyperlink{Figlia di Shayalia}{Figlia di Shayalia} (2°)\\
 \rowcolor{gray!20}16	&	9	& 7	&	16d6+27	&\hyperlink{Senza Traccia}{Senza Traccia}\\
	17	&	9	& 8	&	17d6+27	&\\
 \rowcolor{gray!20}18	&	10	& 8	&	18d6+30	&\hyperlink{Tiro Rapido}{Tiro Rapido}\\
	19	&	11	& 8	&	19d6+33	&\\
 \rowcolor{gray!20}20	&	12	& 8	&	20d6+36	&\hyperlink{Uno con l'arco}{Uno con l'arco}\\
\end{tabularx}

\textbf{Paladino Tradizionale}

\noindent\begin{tabularx}{\linewidth}{c|>{\hsize=0.08\hsize}X>{\hsize=0.08\hsize}X>{\hsize=0.33\hsize}X|X|}
	\toprule
 \rowcolor{gray!20}	\textbf{Lv} & \multicolumn{3}{c|}{\textbf{Paladino Tradizionale}} & \textbf{Abilità} \\
& \centering\arraybackslash \textbf{CA} & \centering\arraybackslash \textbf{CM} & \centering\arraybackslash \textbf{PF} & \\
	\toprule
	1 &1	& 0	&	1d6+3	&\hyperlink{Il Patrono è la mia Arma}{Il Patrono è la mia Arma} - \hyperlink{Rappresaglia}{Rappresaglia}\\
 \rowcolor{gray!20}2	&	1	& 1	&	2d6+3	&\hyperlink{Adepto della Magia}{Adepto della Magia}\\
	3	&	2	& 1	&	3d6+6	&\hyperlink{Imposizione delle mani}{Imposizione delle mani}\\
 \rowcolor{gray!20}4	&	2	& 2	&	4d6+6	&\hyperlink{Armatura del Devoto}{Armatura del Devoto}\\
	5	&	3	& 2	&	5d6+9	&\hyperlink{Il Patrono è la mia Arma}{Il Patrono è la mia Arma} (2°)\\
 \rowcolor{gray!20}6	&	3	& 3	&	6d6+9	&\hyperlink{Scacciare i non morti}{Scacciare i non morti}\\
	7	&	4	& 3	&	7d6+12	&\hyperlink{Imposizione delle mani}{Imposizione delle mani} (2°)\\
 \rowcolor{gray!20}8	&	4	& 4	&	8d6+12	&\\
	9	&	5	& 4	&	9d6+15	&\hyperlink{Lo scudo è mio amico}{Lo scudo è mio amico}\\
 \rowcolor{gray!20}10	&	5	& 5	&	10d6+15	&\hyperlink{Il Patrono è con me}{Il Patrono è con me}\\
	11	&	6	& 5	&	11d6+18	&\\
 \rowcolor{gray!20}12	&	6	& 6	&	12d6+18	&\hyperlink{Armatura del Devoto}{Armatura del Devoto} (2°)\\
	13	&	7	& 6	&	13d6+21	&\hyperlink{Potere del Patrono}{Potere del Patrono}\\
 \rowcolor{gray!20}14	&	7	& 7	&	14d6+21	&\\
	15	&	8	& 7	&	15d6+24	&\hyperlink{Il Patrono è la mia Arma}{Il Patrono è la mia Arma} (3°)\\
 \rowcolor{gray!20}16	&	8	& 8	&	16d6+24	&\hyperlink{Imposizione delle mani}{Imposizione delle mani} (3°)\\
	17	&	9	& 8	&	17d6+27	&\\
 \rowcolor{gray!20}18	&	9	& 9	&	18d6+27	&\hyperlink{Incanalare Energia}{Incanalare Energia}\\
	19	&	10	& 9	&	19d6+30	&\\
 \rowcolor{gray!20}20	&	10	& 10	&	20d6+30	&\hyperlink{Il Patrono è la mia Arma}{Il Patrono è la mia Arma} (4°)\\
\end{tabularx}

\columnbreak

\textbf{Paladino Templare}

\noindent\begin{tabularx}{\linewidth}{c|>{\hsize=0.08\hsize}X>{\hsize=0.08\hsize}X>{\hsize=0.33\hsize}X|X|}
	\toprule
 \rowcolor{gray!20}	\textbf{Lv} & \multicolumn{3}{c|}{\textbf{Paladino Templare}} & \textbf{Abilità} \\
& \centering\arraybackslash \textbf{CA} & \centering\arraybackslash \textbf{CM} & \centering\arraybackslash \textbf{PF} & \\
	\toprule
	1 &1	& 0	&	1d6+3	&\hyperlink{Il Patrono è la mia Arma}{Il Patrono è la mia Arma} - \hyperlink{Rappresaglia}{Rappresaglia}\\
 \rowcolor{gray!20}2	&	2	& 0	&	2d6+6	&\hyperlink{Questa è la mia arma!}{Questa è la mia arma!}\\
	3	&	3	& 0	&	3d6+9	&\hyperlink{Seconda pelle}{Seconda pelle}\\
 \rowcolor{gray!20}4	&	4	& 0	&	4d6+12	&\hyperlink{La mia pelle}{La mia pelle}\\
	5	&	5	& 0	&	5d6+15	&\hyperlink{Il Patrono è la mia Arma}{Il Patrono è la mia Arma} (2°)\\
 \rowcolor{gray!20}6	&	6	& 0	&	6d6+18	&\hyperlink{Opportunista}{Opportunista}\\
	7	&	7	& 0	&	7d6+21	&\hyperlink{Volontà Ferrea}{Volontà Ferrea}\\
 \rowcolor{gray!20}8	&	8	& 0	&	8d6+24	&\\
	9	&	9	& 0	&	9d6+27	&\hyperlink{Resistenza della pietra}{Resistenza della pietra}\\
 \rowcolor{gray!20}10	&	10	& 0	&	10d6+30	&\hyperlink{Il Patrono è la mia Arma}{Il Patrono è la mia Arma} (3°)\\
	11	&	11	& 0	&	11d6+33	&\\
 \rowcolor{gray!20}12	&	12	& 0	&	12d6+36	&\hyperlink{Seconda pelle}{Seconda pelle} (2°)\\
	13	&	13	& 1	&	13d6+39	&\hyperlink{Adepto della Magia}{Adepto della Magia}\\
 \rowcolor{gray!20}14	&	14	& 1	&	14d6+42	&\\
	15	&	15	& 1	&	15d6+45	&\hyperlink{Il Patrono è la mia Arma}{Il Patrono è la mia Arma} (4°)\\
 \rowcolor{gray!20}16	&	15	& 2	&	16d6+45	&\hyperlink{Imposizione delle mani}{Imposizione delle mani}\\
	17	&	15	& 3	&	17d6+45	&\\
 \rowcolor{gray!20}18	&	15	& 4	&	18d6+45	&\hyperlink{Il Patrono è la mia Arma}{Il Patrono è la mia Arma} (5°)\\
	19	&	15	& 5	&	19d6+45	&\\
 \rowcolor{gray!20}20	&	15	& 5	&	20d6+45	&\hyperlink{Il Patrono è la mia Arma}{Il Patrono è la mia Arma} (6°)\\
\end{tabularx}


\textbf{Mago Accademico}

\noindent\begin{tabularx}{\linewidth}{c|>{\hsize=0.08\hsize}X>{\hsize=0.08\hsize}X>{\hsize=0.33\hsize}X|X|}
	\toprule
 \rowcolor{gray!20}	\textbf{Lv} & \multicolumn{3}{c|}{\textbf{Mago Accademico}} & \textbf{Abilità} \\
& \centering\arraybackslash \textbf{CA} & \centering\arraybackslash \textbf{CM} & \centering\arraybackslash \textbf{PF} & \\
	\toprule
	1 &0	& 1	&	1d6	&\hyperlink{Adepto della Magia}{Adepto della Magia} - \hyperlink{Tutt'uno con la magia}{Tutt'uno con la magia}\\
 \rowcolor{gray!20}2	&	0	& 2	&	2d6	&\hyperlink{Sapiente}{Sapiente}\\
	3	&	0	& 3	&	3d6	&\hyperlink{Prodigioso}{Prodigioso}\\
 \rowcolor{gray!20}4	&	0	& 4	&	4d6	&\hyperlink{Adepto della Magia}{Adepto della Magia} (2°)\\
	5	&	0	& 5	&	5d6	&\hyperlink{Magie Potenti}{Magie Potenti}\\
 \rowcolor{gray!20}6	&	0	& 6	&	6d6	&\hyperlink{Concentrato}{Concentrato}\\
	7	&	0	& 7	&	7d6	&\hyperlink{Dattilografo}{Dattilografo}\\
 \rowcolor{gray!20}8	&	0	& 8	&	8d6	&\\
	9	&	0	& 9	&	9d6	&\hyperlink{Batteria Magica}{Batteria Magica}\\
 \rowcolor{gray!20}10	&	0	& 10	&	10d6	&\hyperlink{Specialista}{Specialista}\\
	11	&	0	& 11	&	11d6	&\\
 \rowcolor{gray!20}12	&	0	& 12	&	12d6	&\hyperlink{Creare Oggetti Magici}{Creare Oggetti Magici}\\
	13	&	0	& 13	&	13d6	&\hyperlink{Prodigioso}{Prodigioso} (2°)\\
 \rowcolor{gray!20}14	&	0	& 14	&	14d6	&\\
	15	&	0	& 15	&	15d6	&\hyperlink{Adepto della Magia}{Adepto della Magia} (3°)\\
 \rowcolor{gray!20}16	&	0	& 16	&	16d6	&\hyperlink{Concentrato}{Concentrato} (2°)\\
	17	&	0	& 17	&	17d6	&\\
 \rowcolor{gray!20}18	&	0	& 18	&	18d6	&\hyperlink{Magie Potenti}{Magie Potenti} (2°)\\
	19	&	0	& 19	&	19d6	&\\
 \rowcolor{gray!20}20	&	0	& 20	&	20d6	&\hyperlink{Un solo credo}{Un solo credo}\\
\end{tabularx}

\columnbreak

\textbf{Stregone}

\noindent\begin{tabularx}{\linewidth}{c|>{\hsize=0.08\hsize}X>{\hsize=0.08\hsize}X>{\hsize=0.33\hsize}X|X|}
	\toprule
 \rowcolor{gray!20}	\textbf{Lv} & \multicolumn{3}{c|}{\textbf{Stregone}} & \textbf{Abilità} \\
& \centering\arraybackslash \textbf{CA} & \centering\arraybackslash \textbf{CM} & \centering\arraybackslash \textbf{PF} & \\
	\toprule
	1 &0	& 1	&	1d6	&\hyperlink{Adepto della Magia}{Adepto della Magia} - \hyperlink{Batteria Magica}{Batteria Magica}\\
 \rowcolor{gray!20}2	&	0	& 2	&	2d6	&\hyperlink{Elementalista}{Elementalista}\\
	3	&	0	& 3	&	3d6	&\hyperlink{Batteria Estesa}{Batteria Estesa}\\
 \rowcolor{gray!20}4	&	0	& 4	&	4d6	&\hyperlink{Concentrato}{Concentrato}\\
	5	&	0	& 5	&	5d6	&\hyperlink{Magie Potenti}{Magie Potenti}\\
 \rowcolor{gray!20}6	&	0	& 6	&	6d6	&\hyperlink{Dadi Truccati}{Dadi Truccati}\\
	7	&	0	& 7	&	7d6	&\hyperlink{Elementalista}{Elementalista} (2°)\\
 \rowcolor{gray!20}8	&	0	& 8	&	8d6	&\\
	9	&	0	& 9	&	9d6	&\hyperlink{Un colpo un morto}{Un colpo un morto}\\
 \rowcolor{gray!20}10	&	0	& 10	&	10d6	&\hyperlink{Batteria Magica}{Batteria Magica} (2°)\\
	11	&	0	& 11	&	11d6	&\\
 \rowcolor{gray!20}12	&	0	& 12	&	12d6	&\hyperlink{Dadi Truccati}{Dadi Truccati} (2°)\\
	13	&	0	& 13	&	13d6	&\hyperlink{Concentrato}{Concentrato} (2°)\\
 \rowcolor{gray!20}14	&	0	& 14	&	14d6	&\\
	15	&	0	& 15	&	15d6	&\hyperlink{Adepto della Magia}{Adepto della Magia} (2°)\\
 \rowcolor{gray!20}16	&	0	& 16	&	16d6	&\hyperlink{Magie Potenti}{Magie Potenti} (2°)\\
	17	&	0	& 17	&	17d6	&\\
 \rowcolor{gray!20}18	&	0	& 18	&	18d6	&\hyperlink{Incantatore Prudente}{Incantatore Prudente}\\
	19	&	0	& 19	&	19d6	&\\
 \rowcolor{gray!20}20	&	0	& 20	&	20d6	&\hyperlink{Tutt'uno con la magia}{Tutt'uno con la magia}\\
\end{tabularx}

\textbf{Chierico Guaritore}

\noindent\begin{tabularx}{\linewidth}{c|>{\hsize=0.08\hsize}X>{\hsize=0.08\hsize}X>{\hsize=0.33\hsize}X|X|}
	\toprule
 \rowcolor{gray!20}	\textbf{Lv} & \multicolumn{3}{c|}{\textbf{Chierico Guaritore}} & \textbf{Abilità} \\
& \centering\arraybackslash \textbf{CA} & \centering\arraybackslash \textbf{CM} & \centering\arraybackslash \textbf{PF} & \\
	\toprule
	1 &1	& 0	&	1d6+3	&\hyperlink{Guaritore}{Guaritore} - \hyperlink{Adepto della Magia}{Adepto della Magia}\\
 \rowcolor{gray!20}2	&	1	& 1	&	2d6+3	&\hyperlink{Imposizione delle mani}{Imposizione delle mani}\\
	3	&	2	& 1	&	3d6+6	&\hyperlink{Fedele}{Fedele}\\
 \rowcolor{gray!20}4	&	2	& 2	&	4d6+6	&\hyperlink{Tutt'uno con la magia}{Tutt'uno con la magia}\\
	5	&	3	& 2	&	5d6+9	&\hyperlink{Imposizione delle mani}{Imposizione delle mani} (2°)\\
 \rowcolor{gray!20}6	&	3	& 3	&	6d6+9	&\hyperlink{Incanalare Energia}{Incanalare Energia}\\
	7	&	4	& 3	&	7d6+12	&\hyperlink{Scacciare i non morti}{Scacciare i non morti}\\
 \rowcolor{gray!20}8	&	4	& 4	&	8d6+12	&\\
	9	&	5	& 4	&	9d6+15	&\hyperlink{Guaritore}{Guaritore} (2°)\\
 \rowcolor{gray!20}10	&	5	& 5	&	10d6+15	&\hyperlink{Il Patrono è con me}{Il Patrono è con me}\\
	11	&	5	& 6	&	11d6+15	&\\
 \rowcolor{gray!20}12	&	5	& 7	&	12d6+15	&\hyperlink{Imposizione delle mani}{Imposizione delle mani} (3°)\\
	13	&	5	& 8	&	13d6+15	&\hyperlink{Potere del Patrono}{Potere del Patrono}\\
 \rowcolor{gray!20}14	&	5	& 9	&	14d6+15	&\\
	15	&	5	& 10	&	15d6+15	&\hyperlink{Imposizione delle mani}{Imposizione delle mani} (4°)\\
 \rowcolor{gray!20}16	&	5	& 11	&	16d6+15	&\hyperlink{Incanalare Energia}{Incanalare Energia} (2°)\\
	17	&	5	& 12	&	17d6+15	&\\
 \rowcolor{gray!20}18	&	5	& 13	&	18d6+15	&\hyperlink{Imposizione delle mani}{Imposizione delle mani} (5°)\\
	19	&	5	& 14	&	19d6+15	&\\
 \rowcolor{gray!20}20	&	5	& 15	&	20d6+15	&\hyperlink{Imposizione delle mani}{Imposizione delle mani} (6°)\\
\end{tabularx}

\columnbreak

\textbf{Chierico Guerriero}

\noindent\begin{tabularx}{\linewidth}{c|>{\hsize=0.08\hsize}X>{\hsize=0.08\hsize}X>{\hsize=0.33\hsize}X|X|}
	\toprule
\rowcolor{gray!20}	\textbf{Lv} & \multicolumn{3}{c|}{\textbf{Chierico Guerriero}} & \textbf{Abilità} \\
 & \centering\arraybackslash \textbf{CA} & \centering\arraybackslash \textbf{CM} & \centering\arraybackslash \textbf{PF} & \\
	\toprule
	1 &1	& 0	&	1d6+3	&\hyperlink{Il Patrono è la mia Arma}{Il Patrono è la mia Arma} - \hyperlink{Adepto della Magia}{Adepto della Magia}\\
 \rowcolor{gray!20}2	&	1	& 1	&	2d6+3	&\hyperlink{Imposizione delle mani}{Imposizione delle mani}\\
	3	&	2	& 1	&	3d6+6	&\hyperlink{Armatura del Devoto}{Armatura del Devoto}\\
 \rowcolor{gray!20}4	&	2	& 2	&	4d6+6	&\hyperlink{Scacciare i non morti}{Scacciare i non morti}\\
	5	&	3	& 2	&	5d6+9	&\hyperlink{Il Patrono è la mia Arma}{Il Patrono è la mia Arma} (2°)\\
 \rowcolor{gray!20}6	&	3	& 3	&	6d6+9	&\hyperlink{Lo scudo è mio amico}{Lo scudo è mio amico}\\
	7	&	4	& 3	&	7d6+12	&\hyperlink{Tutt'uno con la magia}{Tutt'uno con la magia}\\
 \rowcolor{gray!20}8	&	4	& 4	&	8d6+12	&\\
	9	&	5	& 4	&	9d6+15	&\hyperlink{Imposizione delle mani}{Imposizione delle mani} (2°)\\
 \rowcolor{gray!20}10	&	5	& 5	&	10d6+15	&\hyperlink{Il Patrono è con me}{Il Patrono è con me}\\
	11	&	6	& 5	&	11d6+18	&\\
 \rowcolor{gray!20}12	&	6	& 6	&	12d6+18	&\hyperlink{Armatura del Devoto}{Armatura del Devoto} (2°)\\
	13	&	7	& 6	&	13d6+21	&\hyperlink{Potere del Patrono}{Potere del Patrono}\\
 \rowcolor{gray!20}14	&	7	& 7	&	14d6+21	&\\
	15	&	8	& 7	&	15d6+24	&\hyperlink{Il Patrono è la mia Arma}{Il Patrono è la mia Arma} (3°)\\
 \rowcolor{gray!20}16	&	8	& 8	&	16d6+24	&\hyperlink{Incanalare Energia}{Incanalare Energia}\\
	17	&	9	& 8	&	17d6+27	&\\
 \rowcolor{gray!20}18	&	9	& 9	&	18d6+27	&\hyperlink{Armatura del Devoto}{Armatura del Devoto} (3°)\\
	19	&	10	& 9	&	19d6+30	&\\
 \rowcolor{gray!20}20	&	10	& 10	&	20d6+30	&\hyperlink{Il Patrono è la mia Arma}{Il Patrono è la mia Arma} (4°)\\
\end{tabularx}

\textbf{Druido Naturalista}

\noindent\begin{tabularx}{\linewidth}{c|>{\hsize=0.08\hsize}X>{\hsize=0.08\hsize}X>{\hsize=0.33\hsize}X|X|}
	\toprule
 \rowcolor{gray!20}	\textbf{Lv} & \multicolumn{3}{c|}{\textbf{Druido Naturalista}} & \textbf{Abilità} \\
& \centering\arraybackslash \textbf{CA} & \centering\arraybackslash \textbf{CM} & \centering\arraybackslash \textbf{PF} & \\
	\toprule
	1 &1	& 0	&	1d6+3	&\hyperlink{Distillare pozioni}{Distillare pozioni} - \hyperlink{Adepto della Magia}{Adepto della Magia}\\
 \rowcolor{gray!20}2	&	1	& 1	&	2d6+3	&\hyperlink{Figlia di Shayalia}{Figlia di Shayalia}\\
	3	&	2	& 1	&	3d6+6	&\hyperlink{Passo Sicuro}{Passo Sicuro}\\
 \rowcolor{gray!20}4	&	2	& 2	&	4d6+6	&\hyperlink{Elementalista}{Elementalista}\\
	5	&	3	& 2	&	5d6+9	&\hyperlink{Animaletto / Famiglio}{Animaletto / Famiglio}\\
 \rowcolor{gray!20}6	&	3	& 3	&	6d6+9	&\hyperlink{Tutt'uno con la magia}{Tutt'uno con la magia}\\
	7	&	4	& 3	&	7d6+12	&\hyperlink{Distillare pozioni}{Distillare pozioni} (2°)\\
 \rowcolor{gray!20}8	&	4	& 4	&	8d6+12	&\\
	9	&	5	& 4	&	9d6+15	&\hyperlink{Figlia di Shayalia}{Figlia di Shayalia} (2°)\\
 \rowcolor{gray!20}10	&	5	& 5	&	10d6+15	&\hyperlink{Concentrato}{Concentrato}\\
	11	&	5	& 6	&	11d6+15	&\\
 \rowcolor{gray!20}12	&	5	& 7	&	12d6+15	&\hyperlink{Elementalista}{Elementalista} (2°)\\
	13	&	5	& 8	&	13d6+15	&\hyperlink{Forma Elementale}{Forma Elementale}\\
 \rowcolor{gray!20}14	&	5	& 9	&	14d6+15	&\\
	15	&	5	& 10	&	15d6+15	&\hyperlink{Animaletto / Famiglio}{Animaletto / Famiglio} (2°)\\
 \rowcolor{gray!20}16	&	5	& 11	&	16d6+15	&\hyperlink{Figlia di Shayalia}{Figlia di Shayalia} (3°)\\
	17	&	5	& 12	&	17d6+15	&\\
 \rowcolor{gray!20}18	&	5	& 13	&	18d6+15	&\hyperlink{Magie Potenti}{Magie Potenti}\\
	19	&	5	& 14	&	19d6+15	&\\
 \rowcolor{gray!20}20	&	5	& 15	&	20d6+15	&\hyperlink{Prodigioso}{Prodigioso}\\
\end{tabularx}

\columnbreak

\textbf{Druido Mutaforma}

\noindent\begin{tabularx}{\linewidth}{c|>{\hsize=0.08\hsize}X>{\hsize=0.08\hsize}X>{\hsize=0.33\hsize}X|X|}
	\toprule
 \rowcolor{gray!20}	\textbf{Lv} & \multicolumn{3}{c|}{\textbf{Druido Mutaforma}} & \textbf{Abilità} \\
& \centering\arraybackslash \textbf{CA} & \centering\arraybackslash \textbf{CM} & \centering\arraybackslash \textbf{PF} & \\
	\toprule
	1 &1	& 0	&	1d6+3	&\hyperlink{Distillare pozioni}{Distillare pozioni} - \hyperlink{Adepto della Magia}{Adepto della Magia}\\
 \rowcolor{gray!20}2	&	1	& 1	&	2d6+3	&\hyperlink{Animalia}{Animalia}\\
	3	&	2	& 1	&	3d6+6	&\hyperlink{Figlia di Shayalia}{Figlia di Shayalia}\\
 \rowcolor{gray!20}4	&	2	& 2	&	4d6+6	&\hyperlink{Sangue Puro}{Sangue Puro}\\
	5	&	3	& 2	&	5d6+9	&\hyperlink{Animalia}{Animalia} (2°)\\
 \rowcolor{gray!20}6	&	3	& 3	&	6d6+9	&\hyperlink{Forma Elementale}{Forma Elementale}\\
	7	&	4	& 3	&	7d6+12	&\hyperlink{Passo Sicuro}{Passo Sicuro}\\
 \rowcolor{gray!20}8	&	4	& 4	&	8d6+12	&\\
	9	&	5	& 4	&	9d6+15	&\hyperlink{Sangue Puro}{Sangue Puro} (2°)\\
 \rowcolor{gray!20}10	&	5	& 5	&	10d6+15	&\hyperlink{Animalia}{Animalia} (3°)\\
	11	&	6	& 5	&	11d6+18	&\\
 \rowcolor{gray!20}12	&	6	& 6	&	12d6+18	&\hyperlink{Figlia di Shayalia}{Figlia di Shayalia} (2°)\\
	13	&	7	& 6	&	13d6+21	&\hyperlink{Forma Elementale}{Forma Elementale} (2°)\\
 \rowcolor{gray!20}14	&	7	& 7	&	14d6+21	&\\
	15	&	8	& 7	&	15d6+24	&\hyperlink{Sangue Puro}{Sangue Puro} (3°)\\
 \rowcolor{gray!20}16	&	8	& 8	&	16d6+24	&\hyperlink{Animalia}{Animalia} (4°)\\
	17	&	8	& 9	&	17d6+24	&\\
 \rowcolor{gray!20}18	&	8	& 10	&	18d6+24	&\hyperlink{Figlia di Shayalia}{Figlia di Shayalia} (3°)\\
	19	&	8	& 11	&	19d6+24	&\\
 \rowcolor{gray!20}20	&	8	& 12	&	20d6+24	&\hyperlink{Figlia di Shayalia}{Figlia di Shayalia} (4°)\\
\end{tabularx}


\textbf{Bardo Incantatore}

\noindent\begin{tabularx}{\linewidth}{c|>{\hsize=0.08\hsize}X>{\hsize=0.08\hsize}X>{\hsize=0.33\hsize}X|X|}
	\toprule
 \rowcolor{gray!20}	\textbf{Lv} & \multicolumn{3}{c|}{\textbf{Bardo Incantatore}} & \textbf{Abilità} \\
& \centering\arraybackslash \textbf{CA} & \centering\arraybackslash \textbf{CM} & \centering\arraybackslash \textbf{PF} & \\
	\toprule
	1 &1	& 0	&	1d6+3	&\hyperlink{Infondere Coraggio}{Infondere Coraggio} - \hyperlink{Adepto della Magia}{Adepto della Magia}\\
 \rowcolor{gray!20}2	&	1	& 1	&	2d6+3	&\hyperlink{Incantatore da Combattimento}{Incantatore da Combattimento}\\
	3	&	2	& 1	&	3d6+6	&\hyperlink{Infondere Paura}{Infondere Paura}\\
 \rowcolor{gray!20}4	&	2	& 2	&	4d6+6	&\hyperlink{Esperto}{Esperto}\\
	5	&	3	& 2	&	5d6+9	&\hyperlink{Infondere Coraggio}{Infondere Coraggio} (2°)\\
 \rowcolor{gray!20}6	&	3	& 3	&	6d6+9	&\hyperlink{Litania versatile}{Litania versatile}\\
	7	&	4	& 3	&	7d6+12	&\hyperlink{Concentrato}{Concentrato}\\
 \rowcolor{gray!20}8	&	4	& 4	&	8d6+12	&\\
	9	&	5	& 4	&	9d6+15	&\hyperlink{Infondere Paura}{Infondere Paura} (2°)\\
 \rowcolor{gray!20}10	&	5	& 5	&	10d6+15	&\hyperlink{Guerriero della Magia}{Guerriero della Magia}\\
	11	&	6	& 5	&	11d6+18	&\\
 \rowcolor{gray!20}12	&	6	& 6	&	12d6+18	&\hyperlink{Infondere Coraggio}{Infondere Coraggio} (3°)\\
	13	&	7	& 6	&	13d6+21	&\hyperlink{Tutt'uno con la magia}{Tutt'uno con la magia}\\
 \rowcolor{gray!20}14	&	7	& 7	&	14d6+21	&\\
	15	&	8	& 7	&	15d6+24	&\hyperlink{Litania versatile}{Litania versatile} (2°)\\
 \rowcolor{gray!20}16	&	8	& 8	&	16d6+24	&\hyperlink{Magie Potenti}{Magie Potenti}\\
	17	&	8	& 9	&	17d6+24	&\\
 \rowcolor{gray!20}18	&	8	& 10	&	18d6+24	&\hyperlink{Infondere Energia Magica}{Infondere Energia Magica}\\
	19	&	8	& 11	&	19d6+24	&\\
 \rowcolor{gray!20}20	&	8	& 12	&	20d6+24	&\hyperlink{Prodigioso}{Prodigioso}\\
\end{tabularx}

\columnbreak

\textbf{Bardo della Lama}

\noindent\begin{tabularx}{\linewidth}{c|>{\hsize=0.08\hsize}X>{\hsize=0.08\hsize}X>{\hsize=0.33\hsize}X|X|}
	\toprule
 \rowcolor{gray!20}	\textbf{Lv} & \multicolumn{3}{c|}{\textbf{Bardo della Lama}} & \textbf{Abilità} \\
& \centering\arraybackslash \textbf{CA} & \centering\arraybackslash \textbf{CM} & \centering\arraybackslash \textbf{PF} & \\
	\toprule
	1 &1	& 0	&	1d6+3	&\hyperlink{Danza della Lama}{Danza della Lama} - \hyperlink{Infondere Coraggio}{Infondere Coraggio}\\
 \rowcolor{gray!20}2	&	2	& 0	&	2d6+6	&\hyperlink{Arma Focalizzata}{Arma Focalizzata}\\
	3	&	3	& 0	&	3d6+9	&\hyperlink{Tattico}{Tattico}\\
 \rowcolor{gray!20}4	&	4	& 0	&	4d6+12	&\hyperlink{Danza della Lama}{Danza della Lama} (2°)\\
	5	&	5	& 0	&	5d6+15	&\hyperlink{Fare Infuriare}{Fare Infuriare}\\
 \rowcolor{gray!20}6	&	6	& 0	&	6d6+18	&\hyperlink{Danno Coordinato}{Danno Coordinato}\\
	7	&	7	& 0	&	7d6+21	&\hyperlink{Danza della Lama}{Danza della Lama} (3°)\\
 \rowcolor{gray!20}8	&	8	& 0	&	8d6+24	&\\
	9	&	9	& 0	&	9d6+27	&\hyperlink{Infondere Coraggio}{Infondere Coraggio} (2°)\\
 \rowcolor{gray!20}10	&	10	& 0	&	10d6+30	&\hyperlink{Attacco Turbinante}{Attacco Turbinante}\\
	11	&	11	& 0	&	11d6+33	&\\
 \rowcolor{gray!20}12	&	12	& 0	&	12d6+36	&\hyperlink{Iaijutsu}{Iaijutsu}\\
	13	&	12	& 1	&	13d6+36	&\hyperlink{Adepto della Magia}{Adepto della Magia}\\
 \rowcolor{gray!20}14	&	12	& 2	&	14d6+36	&\\
	15	&	12	& 3	&	15d6+36	&\hyperlink{Danno Coordinato}{Danno Coordinato} (2°)\\
 \rowcolor{gray!20}16	&	12	& 4	&	16d6+36	&\hyperlink{Guerriero della Magia}{Guerriero della Magia}\\
	17	&	12	& 5	&	17d6+36	&\\
 \rowcolor{gray!20}18	&	12	& 6	&	18d6+36	&\hyperlink{Infondere Energia Magica}{Infondere Energia Magica}\\
	19	&	12	& 7	&	19d6+36	&\\
 \rowcolor{gray!20}20	&	12	& 8	&	20d6+36	&\hyperlink{Litania versatile}{Litania versatile}\\
\end{tabularx}

\vspace{4cm}

\textbf{Monaco Mistico}

\noindent\begin{tabularx}{\linewidth}{c|>{\hsize=0.08\hsize}X>{\hsize=0.08\hsize}X>{\hsize=0.33\hsize}X|X|}
	\toprule
 \rowcolor{gray!20}	\textbf{Lv} & \multicolumn{3}{c|}{\textbf{Monaco Mistico}} & \textbf{Abilità} \\
& \centering\arraybackslash \textbf{CA} & \centering\arraybackslash \textbf{CM} & \centering\arraybackslash \textbf{PF} & \\
	\toprule
	1 &1	& 0	&	1d6+3	&\hyperlink{Gru d'Argento}{Gru d'Argento} - \hyperlink{Energia Psichica}{Energia Psichica}\\
 \rowcolor{gray!20}2	&	1	& 1	&	2d6+3	&\hyperlink{Pugno di Ferro}{Pugno di Ferro}\\
	3	&	2	& 1	&	3d6+6	&\hyperlink{Colpo Psichico}{Colpo Psichico}\\
 \rowcolor{gray!20}4	&	2	& 2	&	4d6+6	&\hyperlink{Armatura della Montagna Incantata}{Armatura della Montagna Incantata}\\
	5	&	3	& 2	&	5d6+9	&\hyperlink{Gru d'Argento}{Gru d'Argento} (2°)\\
 \rowcolor{gray!20}6	&	3	& 3	&	6d6+9	&\hyperlink{Pugno di Ferro}{Pugno di Ferro} (2°)\\
	7	&	4	& 3	&	7d6+12	&\hyperlink{Passo rapido}{Passo rapido}\\
 \rowcolor{gray!20}8	&	4	& 4	&	8d6+12	&\\
	9	&	5	& 4	&	9d6+15	&\hyperlink{Colpo Psichico}{Colpo Psichico} (2°)\\
 \rowcolor{gray!20}10	&	5	& 5	&	10d6+15	&\hyperlink{Raggio Psichico}{Raggio Psichico}\\
	11	&	6	& 5	&	11d6+18	&\\
 \rowcolor{gray!20}12	&	6	& 6	&	12d6+18	&\hyperlink{Volontà Ferrea}{Volontà Ferrea}\\
	13	&	7	& 6	&	13d6+21	&\hyperlink{Pugno di Ferro}{Pugno di Ferro} (3°)\\
 \rowcolor{gray!20}14	&	7	& 7	&	14d6+21	&\\
	15	&	8	& 7	&	15d6+24	&\hyperlink{Armatura della Montagna Incantata}{Armatura della Montagna Incantata} (2°)\\
 \rowcolor{gray!20}16	&	8	& 8	&	16d6+24	&\hyperlink{Tempesta di Furia}{Tempesta di Furia}\\
	17	&	9	& 8	&	17d6+27	&\\
 \rowcolor{gray!20}18	&	9	& 9	&	18d6+27	&\hyperlink{Gru d'Argento}{Gru d'Argento} (3°)\\
	19	&	10	& 9	&	19d6+30	&\\
 \rowcolor{gray!20}20	&	10	& 10	&	20d6+30	&\hyperlink{Ali della Fenice}{Ali della Fenice}\\
\end{tabularx}

\columnbreak

\textbf{Devoto dell'occulto}

\noindent\begin{tabularx}{\linewidth}{c|>{\hsize=0.08\hsize}X>{\hsize=0.08\hsize}X>{\hsize=0.33\hsize}X|X|}
	\toprule
 \rowcolor{gray!20}	\textbf{Lv} & \multicolumn{3}{c|}{\textbf{Devoto dell'occulto}} & \textbf{Abilità} \\
& \centering\arraybackslash \textbf{CA} & \centering\arraybackslash \textbf{CM} & \centering\arraybackslash \textbf{PF} & \\
	\toprule
	1 &1	& 0	&	1d6+3	&\hyperlink{Figlio di Tazher}{Figlio di Tazher} - \hyperlink{Fedele}{Fedele}\\
 \rowcolor{gray!20}2	&	1	& 1	&	2d6+3	&\hyperlink{Adepto della Magia}{Adepto della Magia}\\
	3	&	2	& 1	&	3d6+6	&\hyperlink{Il Patrono è con me}{Il Patrono è con me}\\
 \rowcolor{gray!20}4	&	2	& 2	&	4d6+6	&\hyperlink{Sifone Nero}{Sifone Nero}\\
	5	&	3	& 2	&	5d6+9	&\hyperlink{Potere del Patrono}{Potere del Patrono}\\
 \rowcolor{gray!20}6	&	3	& 3	&	6d6+9	&\hyperlink{Concentrato}{Concentrato}\\
	7	&	4	& 3	&	7d6+12	&\hyperlink{Batteria Magica}{Batteria Magica}\\
 \rowcolor{gray!20}8	&	4	& 4	&	8d6+12	&\\
	9	&	5	& 4	&	9d6+15	&\hyperlink{Il Patrono è con me}{Il Patrono è con me} (2°)\\
 \rowcolor{gray!20}10	&	5	& 5	&	10d6+15	&\hyperlink{Muro mentale}{Muro mentale}\\
	11	&	5	& 6	&	11d6+15	&\\
 \rowcolor{gray!20}12	&	5	& 7	&	12d6+15	&\hyperlink{Creare Oggetti Magici}{Creare Oggetti Magici}\\
	13	&	5	& 8	&	13d6+15	&\hyperlink{Fedele}{Fedele} (2°)\\
 \rowcolor{gray!20}14	&	5	& 9	&	14d6+15	&\\
	15	&	5	& 10	&	15d6+15	&\hyperlink{Adepto della Magia}{Adepto della Magia} (2°)\\
 \rowcolor{gray!20}16	&	5	& 11	&	16d6+15	&\hyperlink{Il Patrono è con me}{Il Patrono è con me} (3°)\\
	17	&	5	& 12	&	17d6+15	&\\
 \rowcolor{gray!20}18	&	5	& 13	&	18d6+15	&\hyperlink{Magie Potenti}{Magie Potenti}\\
	19	&	5	& 14	&	19d6+15	&\\
 \rowcolor{gray!20}20	&	5	& 15	&	20d6+15	&\hyperlink{Un solo credo}{Un solo credo}\\
\end{tabularx}


\vspace{4cm}


\textbf{Magus}

\noindent\begin{tabularx}{\linewidth}{c|>{\hsize=0.08\hsize}X>{\hsize=0.08\hsize}X>{\hsize=0.33\hsize}X|X|}
	\toprule
 \rowcolor{gray!20}	\textbf{Lv} & \multicolumn{3}{c|}{\textbf{Magus}} & \textbf{Abilità} \\
& \centering\arraybackslash \textbf{CA} & \centering\arraybackslash \textbf{CM} & \centering\arraybackslash \textbf{PF} & \\
	\toprule
	1 &1	& 0	&	1d6+3	&\hyperlink{Arma Focalizzata}{Arma Focalizzata} - \hyperlink{Incantatore da Combattimento}{Incantatore da Combattimento}\\
 \rowcolor{gray!20}2	&	1	& 1	&	2d6+3	&\hyperlink{Adepto della Magia}{Adepto della Magia}\\
	3	&	2	& 1	&	3d6+6	&\hyperlink{Guerriero della Magia}{Guerriero della Magia}\\
 \rowcolor{gray!20}4	&	2	& 2	&	4d6+6	&\hyperlink{Infondere Energia Magica}{Infondere Energia Magica}\\
	5	&	3	& 2	&	5d6+9	&\hyperlink{Radici magiche}{Radici magiche}\\
 \rowcolor{gray!20}6	&	3	& 3	&	6d6+9	&\hyperlink{Guerriero della Magia}{Guerriero della Magia} (2°)\\
	7	&	4	& 3	&	7d6+12	&\hyperlink{Incantatore da Combattimento}{Incantatore da Combattimento} (2°)\\
 \rowcolor{gray!20}8	&	4	& 4	&	8d6+12	&\\
	9	&	5	& 4	&	9d6+15	&\hyperlink{Infondere Energia Magica Superiore}{Infondere Energia Magica Superiore}\\
 \rowcolor{gray!20}10	&	5	& 5	&	10d6+15	&\hyperlink{Un colpo un morto}{Un colpo un morto}\\
	11	&	6	& 5	&	11d6+18	&\\
 \rowcolor{gray!20}12	&	6	& 6	&	12d6+18	&\hyperlink{Guerriero della Magia}{Guerriero della Magia} (3°)\\
	13	&	7	& 6	&	13d6+21	&\hyperlink{Concentrato}{Concentrato}\\
 \rowcolor{gray!20}14	&	7	& 7	&	14d6+21	&\\
	15	&	8	& 7	&	15d6+24	&\hyperlink{Guerriero della Magia}{Guerriero della Magia} (4°)\\
 \rowcolor{gray!20}16	&	8	& 8	&	16d6+24	&\hyperlink{Infondere Energia Magica Superiore}{Infondere Energia Magica Superiore} (2°)\\
	17	&	9	& 8	&	17d6+27	&\\
 \rowcolor{gray!20}18	&	9	& 9	&	18d6+27	&\hyperlink{Incantatore da Combattimento}{Incantatore da Combattimento} (3°)\\
	19	&	10	& 9	&	19d6+30	&\\
 \rowcolor{gray!20}20	&	10	& 10	&	20d6+30	&\hyperlink{Tutt'uno con la magia}{Tutt'uno con la magia}\\
\end{tabularx}


\end{multicols}

}

\pagebreak

\section{Famiglio}\index{Famiglio}\label{famiglio}\hypertarget{famiglio}{}

\medskip

\begin{enfasi}{
Nipote del signor Wing: Senta mister, ci sono tre regole da seguire, però.

Rand: Ah, sì? E quali sarebbero?

Nipote del signor Wing: Lo tenga lontano dalla luce, lui odia la luce forte, specialmente quella del sole. Morirebbe. E lo tenga lontano dall'acqua, non lo faccia bagnare. Ma la cosa più importante, la regola che non dovrà mai dimenticare è che anche se lui piange, anche se fa scena e la supplica lei non dovrà mai, mai dargli da mangiare dopo la mezzanotte. Ha capito? (Gremlins, Film, 1984)}
\end{enfasi}

\begin{multicols}{2}

I famigli sono animali scelti dal personaggio, tramite l'Abilità Famiglio, perché gli siano d'aiuto nelle avventure e per compagnia. Un famiglio ha un legame speciale con il suo padrone.

Un famiglio è un normale animale ma viene trattato come creatura magica al fine di determinare qualsiasi effetto che dipenda dal suo tipo.

Solo un normale animale, non modificato, può diventare un famiglio.

Un famiglio conferisce delle Capacità Speciali al suo padrone, queste Capacità Speciali si applicano solo quando il padrone e il famiglio sono entro 100 m l'uno dall'altro.

E' necessario un particolare rito di 4 ore nell'ambiente nativo dell'animale per farlo diventare un famiglio.

Se un famiglio viene congedato, perso oppure muore, può essere sostituito una settimana dopo con uno speciale rituale che costa 2 punti di Costituzione temporanea del personaggio. Per completare il rituale occorrono 8 ore.

\medskip

\textbf{Tabella: Tipi di Famiglio}\index[Tabelle]{Tabella Tipi di Famiglio}

\medskip

\noindent\begin{tabularx}{\linewidth}{lX}
	\toprule
\rowcolor{gray!20}\textbf{Famiglio} & \textbf{Capacità acquisita dal padrone}\\
\toprule
Civetta & +2 alle prove di Arcana\\
\rowcolor{gray!20}Corvo & +2 alle prove di Intimidire\\
\emph{Dobi}& +2 al Tiro Salvezza vs Ammaliamento\\
%Donnola & +1 alle prove su Conoscenza\\
\rowcolor{gray!20}Falco & +2 su Consapevolezza sulla vista\\
Gatto & +2 alle prove di Furtività\\
\rowcolor{gray!20}Gufo & +2 su Consapevolezza sul udito\\
Lontra & +2 alle prove di Nuotare\\
\rowcolor{gray!20}Lucertola & +2 alle prove di Sopravvivenza\\
Pipistrello & +1 Tiro Salvezza su Tempra\\
\rowcolor{gray!20}Ratto & +2 al Tiro Salvezza contro Malattie\\
Riccio & +1 al Tiro Salvezza su Volontà\\
\rowcolor{gray!20}Rospo & +2 al Tiro Salvezza su Veleno\\
Scimmia & +2 alle prove di Mani di Fata\\
\rowcolor{gray!20}\emph{Topi} & diventi il famiglio della Topi!!!\\
Volpe & +1 al Tiro Salvezza su Riflessi
\end{tabularx}

\bigskip

Utilizzare le statistiche base di una creatura della specie del famiglio, apportando i seguenti cambiamenti.

\medskip

\textbf{Attacchi}: Utilizzare la Competenza Armi del padrone se più alta. Utilizzare il modificatore di Destrezza o Forza del famiglio, quale dei due sia più alto per calcolare il bonus di attacco del famiglio con gli Attacchi Naturali. Il danno è uguale a quello di una normale creatura della specie del famiglio. Il famiglio agisce nel round del padrone.

\medskip

\begin{center}
\includegraphics[width=0.63\linewidth]{immagini/donnadrago2.png}

\emph{Henry Justice Ford}
\end{center}

\medskip

\textbf{Difesa}: il famiglio ha una Difesa pari a quello dell'animale standard più un bonus dovuto alla Competenza Magica del padrone. Vedi tabella Abilità del Famiglio.

\medskip

\textbf{Tiro Salvezza}: Per ogni Tiro Salvezza, utilizzare i Tiri Salvezza del famiglio o quelli del padrone quali siano i migliori. Il famiglio applica i suoi valori di Caratteristica come bonus ai Tiri Salvezza e non prende nessuno dei bonus che il suo padrone può avere.

\textbf{Iniziativa del Famiglio}\index{Iniziativa del Famiglio}: il famiglio agisce nel tuo round, non tira l'iniziativa, usa la tua.

\textbf{Azioni del Famiglio}\index{Azioni del Famiglio}: comandare un famiglio impiega 1 Azione. Il famiglio esegue 2 Azioni a round. Senza comandi il Famiglio non fa nulla se non difendersi ed attaccare chi lo attacca.

\textbf{Descrizione delle Capacità del Famiglio}

Tutti i famigli possiedono Capacità Speciali e le attribuiscono ai loro padroni a seconda del punteggio di Competenza Magica del padrone. Le capacità Speciali elencate nella tabella sono cumulative.

\end{multicols}

\textbf{Tabella: Abilità e Bonus del Famiglio}\index[Tabelle]{Tabella Abilità del Famiglio}

\medskip
{
%\setlength{\tabcolsep}{4pt}
%\setlength{\extrarowheight}{-2pt}
%\renewcommand{\arraystretch}{0.9}

\noindent\begin{tabularx}{\linewidth}{cccX}
	\toprule
\rowcolor{gray!20}\textbf{CM del Padrone} & \textbf{Difesa} & \textbf{Intelligenza} & \textbf{Speciale}\\
\toprule
1-2 & +1 & 0 & Allerta, Condividere Incantesimi\\
\rowcolor{gray!20}& && Legame Empatico\\
3-4 & +1 & +1 & Trasmettere Incantesimi a contatto\\
\rowcolor{gray!20}5-6 & +2 & +1 & Parlare con gli Animali della Sua Specie\\
7-8 & +3 & +1 & Parlare con il Padrone\\
\rowcolor{gray!20}9-10 & +3 & +2 & -\\
11-12 & +4 & +2 & Vedere attraverso Famiglio\\
\rowcolor{gray!20}13-14 & +5 & +2 & Trasmettere Incantesimi a contatto migliorato\\
15-16 & +5 & +3 & -\\
\rowcolor{gray!20}17-18 & +6 & +3 & -\\
19-20 & +7 & +3 & -
\end{tabularx}}

\begin{multicols}{2}

\medskip

\textbf{Competenza Magica del Padrone}: il numero indicato qui è il valore di Competenza Magica del padrone del famiglio.

\textbf{Difesa}: il bonus indicato è da sommare alla Difesa del famiglio.

\textbf{Intelligenza}: il bonus indicato è da sommare al punteggio di Intelligenza del famiglio.

\textbf{Speciale}: le capacità speciali acquisite dal famiglio (e/o dal padrone).

\emph{\textbf{Allerta}}: quando il famiglio è a portata di braccio dal padrone, questi guadagna +1 alle prove di Consapevolezza

\emph{\textbf{Condividere Incantesimi}}\label{Condividere Incantesimi}\hypertarget{Condividere Incantesimi}{}: a propria discrezione il padrone può lanciare qualsiasi Incantesimo che abbia effetto su se stesso sul suo famiglio, anche se il tipo di creatura non è previsto.

\emph{\textbf{Legame Empatico}}: il padrone ha un legame empatico con il suo famiglio fino a una distanza di 1 km. Il padrone non può vedere attraverso gli occhi del famiglio, ma può comunicare empaticamente con esso. A causa della natura limitata del legame, si possono comunicare solo emozioni generiche (paura, nervoso, tranquillità, gioia...).

\emph{\textbf{Trasmettere Incantesimi a Contatto}}\label{Trasmettere Incantesimi a Contatto}\hypertarget{Trasmettere Incantesimi a Contatto}{}: il famiglio può trasmettere Incantesimi a contatto per il padrone. Se il padrone e il famiglio sono entro 9 metri quando il padrone lancia un Incantesimo con Gittata a contatto, egli può designare il suo famiglio come \emph{colui che consegna l'Incantesimo}.

Il famiglio può trasmettere l'Incantesimo proprio come il padrone. Il famiglio usa una sua Azione per effettuare un attacco.

\emph{\textbf{Parlare col Padrone}}: il famiglio e il padrone possono comunicare verbalmente, come se utilizzassero un linguaggio comune. Le altre creature o animali non sono in grado di comprendere la loro conversazione, se non utilizzando ausili magici. La capacità funziona entro i 50m e devono sentirsi.

\emph{\textbf{Parlare con Animali}}: il famiglio è in grado di comunicare con animali della sua specie specifica: pipistrelli con pipistrelli, ratti con ratti... La comunicazione è limitata dall'Intelligenza delle creature con cui il famiglio comunica.

\emph{\textbf{Vedere attraverso Famiglio}}: il padrone può vedere attraverso il famiglio. Attivare questa Abilità costa 1 Azione e dura fino all'inizio del round successivo. Il famiglio deve essere entro 50 metri.

\emph{\textbf{Trasmettere Incantesimi a Contatto Migliorato}}: come \emph{Trasmettere Incantesimi a Contatto} ma il famiglio può essere entro 18 metri dal padrone.

\textbf{NOTE}: intelligente ed unico un famiglio rimane un animale e come tale non può usare oggetti magici o pergamene, può arrivare ad usare una pozione se ne ha le capacità per berla. Un famiglio particolarmente intelligente potrebbe eseguire semplici ed immediati compiti.

\end{multicols}

\vfill

\begin{center}
\includegraphics[width=0.5\linewidth]{immagini/familiar.png}

\emph{Henry Justice Ford, un ottimo famiglio...}
\end{center}

\pagebreak

\section{Altre Abilità Speciali}

\begin{multicols}{2}

Queste Abilità non sono selezionabili da parte del giocatore, bensì possono essere innate nelle creature.

\subsection{Etereo}\index{Etereo}\label{etereo}

Una creatura diventata Eterea è situata nel Piano Etereo che è sovrapposto a quello Materiale.

Una creatura eterea è Invisibile, senza sostanza e capace di muoversi in qualsiasi direzione, persino su e giù. Una creatura eterea può muoversi attraverso oggetti solidi, incluse altre creature viventi. Una creatura eterea può vedere e udire ciò che accade sul Piano Materiale, ma ogni cosa appare grigia ed inconsistente. La vista e l'udito di una creatura eterea che si trova sul Piano Materiale sono limitati a una distanza di 9 metri.

Gli Incantesimi se non opportunamente formulati e modificati non agiscono su creature eteree. Una creatura eterea ha Resistenza al Danno verso Luce o Vuoto, ed ignora tutte le altre forme di Energia.

Una creatura eterea non può attaccare una creatura materiale ed Incantesimi lanciati mentre ci si trova in condizione di etereo possono influenzare solo elementi eterei. Alcune creature o oggetti materiali hanno attacchi o effetti speciali che funzionano anche sul Piano Etereo. Una creatura eterea considera tutte le altre creature eteree come se tutti fossero materiali.

\subsection{Resistenza al Danno}\index{Resistenza al Danno}\label{resistenzaaldanno}

Determinate creature o protezioni conferiscono la capacità di Resistere ad una tipologia di Danno.

Essere Resistenti al Danno significa automaticamente dimezzare il danno ricevuto prima di applicare qualsiasi altra protezione o Tiro Salvezza.

La Resistenza al Danno può assumere anche dei valori. Quando viene scritto Resistenza al Danno: Elettricità, il soggetto dimezza automaticamente i danni da elettricità, se scritto Resistenza al Danno: Elettricità 10, significa che riduce il danno da elettricità di 10 punti prima di applicare il Tiro Salvezza o altri bonus.

Un creatura con una Resistenza al Fuoco dimezza (riduce) tutto il danno che riceve dalla fiamme, magiche o meno se non specificato diversamente.

Possono esistere Abilità o incantesimi che ignorano questa Resistenza. più resistenze uguali non si sommano, per il fatto che due oggetti mi danno resistenza al fuoco non riduco ad un quarto il danno, se ne applica solo una.
Se una capacità ignora la resistenza al danno passerà la resistenza anche se ho due o più fonti di resistenza.

\subsection{Riduzione del Danno - DR}\index{Riduzione del Danno}\label{resistenzaaldannodr}\hypertarget{riduzionedeldanno}{}

Determinate creature o Abilità conferiscono la capacità soprannaturale di resistere al danno di certe tipologie di armi o fino ad un certo ammontare (per attacco).

Solitamente assume il valore di XX/ZZ ovvero quanto danno (XX) è ignorato se non si è attaccati con (ZZ). Ignorare il danno significa anche che effetti connessi all'attacco non funzionano, come veleni sull'arma. La Riduzione si applica dopo Resistenze e Tiri Salvezza.

\begin{center}
%\includegraphics[width=0.8\linewidth]{immagini/morteachille.png}
%\emph{Paris shot Achilles with an arrow - Pieter Paul Rubens - Data 1630-1632}

\includegraphics[width=0.8\linewidth]{immagini/Archilles_Wilhelm_Wandschneider_grayscale.png}

\emph{Achilles, Wilhelm Wandschneider 1909}

\end{center}

Determinate armi, particolarmente magiche possono ignorare la DR \index{Ignorare la DR}

\medskip

\textbf{Proiettili (frecce, dardi, sassi) tirati da \emph{propulsori} magici NON sono considerati magici.}\index{Frecce magiche}

\subsection{Resistenza alla Magia}\index{Resistenza alla Magia}\label{resistenzaallamagia}

La Resistenza alla magia può essere indicata in due modi diversi.

Può venire indicata con un dado, es. \emph{Resistenza alla Magia. Il deva ha +1d6 ai Tiri Salvezza contro incantesimi e altri effetti magici}. In questo caso si applica il bonus come indicato.

Oppure seguita da un numero e livello, es. \emph{Resistenza alla Magia: 3lv}. In questo caso la creatura non è influenzata da incantesimi di quel livello o meno. Un incantesimo viene considerato di un livello superiore per ogni Critico magico ottenuto nella Prova di Magia.

L'obiettivo della magia pur se non influenzato dagli effetti diretti viene comunque interessato dagli effetti indiretti, ad esempio può cadere nella fossa creata da un incantesimo di Disintegrazione.

La Resistenza alla Magia non può essere abbassata neanche dalla creatura che la possiede.

\subsection{Immunita' al danno}\index{Immunità al danno}\label{immunitaaldanno}

E' estremamente raro ma ci sono creature o effetti magici che rendono immune ad una forma di danno, possa essere fisica (danno da arma..) o magica (le varie forme di energia).

Una creatura immune ad una forma di danno non subisce danno da quell'attacco. Una creatura che ha invece la capacità di avere i propri danni irresistibili, ovvero che non possono essere ridotti da resistenza, penetrerà solo in parte l'immunità della creatura rendendola soltanto resistente a quel danno.

Una creatura che riporta \emph{Immunità al Danno Vuoto, Veleno; armi +2} significa che non subisce danno da Vuoto, da Veleno e che per ferirlo serve un arma con un bonus magico +3 o superiore, oppure un personaggio che attacchi con armi naturali e sia di livello 12 o superiore oppure che abbia preso la Lista d'Armi Pugno Vuoto almeno 6 volte.

Vedi lo schema delle \hyperlink{equivalenzaarmimagiche}{Equivalenze Armi magiche} (pag. \pageref{equivalenzaarmimagiche})

\subsection{Vulnerabilita' al Danno}\index{Vulnerabilità al Danno}\label{vulnerabilitadanno}

Determinate creature o magie rendono più efficaci alcuni effetti causando maggiore danno al soggetto vulnerabile.

Essere Vulnerabili ad un tipo specifico di Danno significa automaticamente raddoppiare il danno ricevuto prima di applicare qualsiasi altra protezione o Tiro Salvezza.

Un creatura con una Vulnerabilità al Fuoco raddoppia tutto il danno subito poi se possibile effettua il Tiro Salvezza indicato dall'incantesimo o effetto.

\subsection{Paura}\index{Paura}\label{paura}

Incantesimi, Oggetti Magici e certe creature possono influenzare i personaggi con l'effetto di Paura. Una creatura dotata di Paura non può sopprimerne l'aura se questa è innata tranne che sia descritto diversamente. E' sempre segnata la difficoltà con cui fare il Tiro Salvezza su Volontà. Una creatura immune alla Paura non può essere spaventata sia che la sorgente sia naturale che magica.

\textbf{Spaventato}\label{spaventato}\index{Spaventato}

Una creatura spaventata ha -1d6 ai Tiri per Colpire, Tiri Salvezza e Prove Competenza finché la sorgente della sua paura è visibile. Una creatura spaventata non può avvicinarsi volontariamente alla sorgente della sua paura.

\subsection{Paralizzato}\index{Paralizzato}\label{paralizzato}

Ci sono diversi metodi per Paralizzare una creatura, sia magici che naturali. Mentre quelli naturali spesso hanno sistemi per liberarsi successivamente, sistemi magici possono prevedere di liberarsi dalla paralisi o meno, magari solo dopo un certo lasso di tempo.

Un personaggio paralizzato non può compiere Azioni o Reazioni ne parlare, gli attacchi in mischia contro di lei hanno +1d6 di bonus e perde il bonus alla Difesa dato dalla Destrezza. La creatura è consapevole di ciò che ha intorno, non lascia cadere gli oggetti. La creatura fallisce automaticamente i Tiri Salvezza su Riflessi.

\end{multicols}

\vfill

\begin{center}
\includegraphics[width=0.45\linewidth]{immagini/the-scream.png}

\emph{L'urlo (titolo originale: Skrik)\\ Edvard Munch - Data 1893-1910}
\end{center}

\pagebreak

\section{La Magia}\index{Magia}\label{lamagia}

\begin{enfasi}{
La magia non è nel pendolino, ma in chi lo usa. (NCIS - Unità anticrimine)

\medskip

Non lascerai vivere colei che pratica la magia. (Libro dell'Esodo)(Sempre a seconda dei propri Tratti...)

\medskip

Uno stregone non è mai in ritardo, Frodo Baggins. Né in anticipo. Arriva precisamente quando intende farlo. (Gandalf, Il Signore degli Anelli - La Compagnia dell'Anello. J.R.R. Tolkien)} \end{enfasi}

\begin{multicols}{2}

La magia permea i mondi di gioco e la sua forma più comune è quella di un incantesimo. Questo capitolo fornisce le regole per lanciare incantesimi.

\medskip

\subsection{Cos'è un Incantesimo?}\index{Cos'è incantesimo}\index{Incantesimo definizione}

Un incantesimo è una manifestazione del potere di un Patrono. Ogni incantesimo è frutto di potere e conoscenza, l'incantatore è un tramite superiore che canalizza la potenza dei Patroni. Nel lanciare un incantesimo, un personaggio compone gesti, parole ed usa oggetti che altro non fanno che collegarlo alla fonte, il Patrono che sovraintende quella lista di magia.

\subsection{Come fare magie! (In sintesi)}\index{Come fare magie! (In sintesi)}

Il tuo personaggio deve avere investito un punto in Competenza Magica.

La Competenza Magica ti permette di avere più Punti Magia, più incantesimi conosciuti e grazie all'Abilità Adepto della Magia anche di rendere i tuoi incantesimi più difficili da resistere. Un punteggio alto di Competenza Magica insieme all'Abilità Adepto della Magia ti permette di accedere a incantesimi di più alto livello.

Non scordarti di cercare antichi tomi e pergamene! Gli incantesimi sono un bene raro e prezioso non perdere l'occasione per trovarne di nuovi e copiarli sul tuo Tomo della Magia.

\subsection{Le caratteristiche degli incantesimi}\index{Le caratteristiche degli incantesimi}\label{caratteristicheincantesimi}

La descrizione di ciascun incantesimo inizia con un blocco di informazioni che comprende il livello, le Liste di Magia a cui appartiene, tempo di lancio, gittata e durata dell'incantesimo. Il resto della descrizione ci informa dell'effetto dell'incantesimo.

Quando un personaggio lancia qualsiasi incantesimo, si usano le seguenti regole base indipendentemente dall'effetto dell'incantesimo.

\subsubsection{Tempo di Lancio}\index{Tempo di Lancio Incantesimi}\label{magietempodilancio}\index{Incantesimi, Azioni per lanciare}\index{Casting Time}\hypertarget{magietempodilancio}{}

La maggior parte degli incantesimi possono essere lanciati con due Azioni. Alcuni incantesimi richiedono un'Azione Immediata, una Azione di Reazione o molto più tempo per essere lanciati. Nella descrizione degli incantesimi è indicata come \emph{T. di Lancio}.

Durante lo stesso round non puoi lanciare un altro incantesimo, a meno che non si tratti di un incantesimo di livello 0 (chiamati Trucchetti).

\smallskip \textbf{Azione Immediata} \smallskip

Un incantesimo lanciato con un'Azione Immediata è particolarmente rapido. Puoi lanciare un solo incantesimo a round come Azione Immediata e non devi averla già usata.

\smallskip \textbf{Reazioni} \smallskip

Alcuni incantesimi possono essere lanciati come Reazioni. Questi incantesimi richiedono una frazione di secondo per essere formulati e possono essere lanciati in risposta a un evento. Se un incantesimo può essere lanciato come Reazione, la descrizione dell'incantesimo ti dice esattamente quando puoi farlo. Devi avere a disposizione una Azione di Reazione e non averla già usata.

\smallskip \textbf{Tempo di Lancio più Lungo} \smallskip

Certi incantesimi richiedono più tempo per essere lanciati: round, minuti o addirittura ore.
Quando lanci un incantesimo con un tempo di lancio di 1 round, significa che passi un intero round a formulare l'incantesimo e il round successivo usando 1 Azione tiri l'incantesimo con l'iniziativa che hai tirato all'inizio della formulazione.

Quando la formulazione dura più di 1 round devi spendere 1 Azione a round per continuare la formulazione. Per quei round è come se dovessi mantenere la Concentrazione per determinare eventuali effetti.

\subsubsection{Le Liste di Magia}\hypertarget{lescuoledimagia}{} \index{Le Liste di Magia}\label{magielistadimagia}\index{Incantesimi, Liste di Magia}

Le Liste qui presentate sono quelle codificate ed insegnate nelle poche scuole di magia.

%Si racconta di ulteriori liste create, curate e diffuse in circoli ristretti o sette. Una di queste liste segrete è quella degli gnomi Devoti di Shayalia, una lista prettamente naturale che mescola la Lista tradizionale degli Animali e Piante con alcuni incantesimi dalle Liste degli elementi.
%Altre liste più oscure sono quelle demoniache o degli Aboleth, alcune altre sono legate all'appartenenza a gruppi di Devoti. Altre liste più nefaste arrivano a corrompere l'anima dei personaggi imponendo anche i Tratti. Al personaggio queste liste saranno normalmente chiuse ma non è detto che con l'aumentare della Competenza Magica non sia lui stesso a creare nuove liste di incantesimi.

Le Liste di Magia aiutano a descrivere gli incantesimi; non hanno delle proprie regole, sebbene alcune regole possano fare riferimento a queste liste.

\begin{itemize}[leftmargin=*] \setlength{\itemsep}{0pt}
\item
\emph{Abiurazione} riguarda incantesimi di natura protettiva, sebbene ne contenga anche alcuni dall'uso aggressivo. Questi incantesimi creano barriere magiche, negano effetti dannosi o bandiscono le creature in altri piani di esistenza.

\item
\emph{Acqua} sono gli incantesimi che agiscono sull'elemento acqua e freddo ed in minima parte anche sulle cure

\item
\emph{Aria} riguarda gli incantesimi che manipolano ed usano l'aria ed anche l'elettricità.

\item
\emph{Ammaliamento} riguarda incantesimi che agiscono sulla mente altrui, influenzandone o controllandone il comportamento. Questi incantesimi possono far sì che i nemici considerino l'incantatore un amico od addirittura controllare un'altra creatura come fosse una marionetta.

\item
\emph{Animali e Piante} questi sono gli incantesimi che agiscono su animali e piante, naturali o magiche.

\item
\emph{Cura} riguarda gli incantesimi che permettono di recuperare le energie fisiche, mentali ed annullare debolezze e veleni.

\item
\emph{Divinazione} riguarda incantesimi che rivelano informazioni perdute nel tempo, dimenticate, visioni del futuro, la posizione di oggetti nascosti, la verità dietro le illusioni od immagini di persone e luoghi lontani.

\item
\emph{Evocazione} riguarda incantesimi che trasportano oggetti e creature da un luogo all'altro. Alcuni incantesimi richiamano creature o oggetti al fianco dell'incantatore, mentre altri permettono all'incantatore di teletrasportarsi da un luogo a un altro. Alcune evocazioni creano oggetti o effetti dal nulla.

\emph{Fuoco} Gli incantesimi più pericolosi sono qua dentro con tutto ciò che serve a bruciare ed incenerire.

\begin{center}
	\includegraphics[width=0.6\linewidth]{immagini/Leonids-1833.png}

	\emph{The most famous depiction of the famous 1833 Leonids \hyperlink{sciamedimeteore}{Meteor Storm} (Pioggia di Meteore!)}
\end{center}

\emph{Illusione} riguarda incantesimi che ingannano i sensi e la mente altrui. Fanno vedere alle persone cose che non esistono, non gli fanno notare le cose che esistono, fanno udire rumori fasulli o ricordare cose che non sono mai accadute. Alcune illusioni creano immagini spettrali che chiunque può vedere.

\item
\emph{Invocazione} riguarda incantesimi che manipolano l'energia magica per produrre un effetto desiderato.

\item
\emph{Necromanzia} riguarda incantesimi che manipolano le energie della vita e della morte. Questi incantesimi possono conferire una riserva aggiuntiva di forza vitale, risucchiare l'energia vitale da un'altra creatura, creare non morti o addirittura riportare in vita i morti (se concesso).

\emph{In OBSS solo un Patrono ha sufficiente potere per poter riportare in vita un morto}.

\item
\emph{Terra} Gli incantesimi che agiscono e muovono la terra

\item
\emph{Trasmutazione} riguarda incantesimi che cambiano le proprietà di una creatura, oggetto o ambiente.

\item
\emph{Universale} alcuni incantesimi sono capisaldi della magia in se e come tali accessibili a tutti gli incantatori. Per accedere agli incantesimi contenuti in questa Lista di Magia è necessario avere almeno un punto in Competenza Magica. Il massimo livello di incantesimi lanciabile in questa lista è pari numero di volte che si è presa l'Abilità Adepto della Magia, con un minimo di 1.

\end{itemize}

\subsubsection{Gittata}\index{Gittata}\label{magiegittata}\index{Incantesimi, Gittata}

Il bersaglio di un incantesimo deve essere nella gittata dell'incantesimo. Per un incantesimo come Dardo arcano, il bersaglio è una creatura. Per un incantesimo come palla di fuoco, il bersaglio è il punto nello spazio da cui la sfera di fuoco esplode. La maggior parte degli incantesimi hanno una gittata espressa in metri. Alcuni incantesimi possono prendere a bersaglio solo una creatura (te compreso) con cui sei in contatto fisico. Altri incantesimi, come l'incantesimo scudo, agiscono solo su di te: questi incantesimi hanno come gittata \emph{personale}. Un incantesimo che ha come area di effetto \emph{un alleato} può essere lanciato anche su se stesso.

Gli incantesimi che creano coni o linee di effetto che originano da te, hanno anch'essi gittata personale\index{Gittata Personale}, a indicare che sei tu il punto di origine dell'effetto dell'incantesimo (vedi \emph{Aree di Effetto} più avanti in questo capitolo).

\subsubsection{Durata}\index{Durata Incantesimi}\label{magiedurata}\index{Incantesimi, Durata}\hypertarget{magiedurata}{}

La durata di un incantesimo è la lunghezza di tempo per cui esso persiste. La durata può essere espressa in round, minuti, ore o addirittura anni. Alcuni incantesimi specificano che i loro effetti durano finché l'incantesimo non viene dissolto o distrutto. Un \textbf{incantesimo può essere interrotto dal proprio incantatore come Azione Immediata}.\index{Interrompere un proprio incantesimo}

Qualora un critico magico raddoppi la durata si intente sempre riferita alla durata iniziale. Es. se la durata è 2 ore dopo il primo raddoppio diventa 4 ore, con il secondo diventa di 6 ore e poi 8 ore..\index{Successo critico magico sulla durata}

\begin{itemize}[leftmargin=*] \setlength{\itemsep}{0pt}

\item
\emph{Istantanea}

Molti incantesimi sono istantanei. L'incantesimo ferisce, cura, crea o altera una creatura o un oggetto in modo che non possa essere dissolto, dato che la sua magia esiste solo per un istante.

\item

\emph{Concentrazione}\index{Concentrazione}\index{Incantesimi, Durata Concentrazione}

Alcuni incantesimi richiedono che tu mantenga la concentrazione per tenerne la magia attiva. Se non puoi mantenere la concentrazione, l'incantesimo avrà fine. Se un incantesimo deve essere mantenuto tramite concentrazione, la cosa è indicata alla voce Durata, l'incantesimo specifica quanto a lungo vi potrai mantenere la concentrazione. Puoi terminare la concentrazione in qualsiasi momento usando una Reazione.

Normali attività, come muoversi e attaccare, non interferiscono con la concentrazione. Mantenere la concentrazione costa 1 Azione a round.
\end{itemize}

\subsubsection{Formulare gli incantesimi}\index{Formulare gli incantesimi}\label{magiecomponenti}\index{Incantesimi, Formulare gli incantesimi}

Ogni incantesimo prevede che l'incantatore abbia le mani libere e possa parlare.

La maggior parte degli incantesimi richiede di intonare parole mistiche e gesticolare in maniera particolare. Le parole ed i gesti, il ritmo, la cadenza e risonanza permettono la sintonia con il Patrono che fornisce la magia.

E' possibile consumare oggetti al momento di lancio dell'incantesimo come offerta al proprio Patrono, o quello che sovraintende la Lista di Magia dell'incantesimo, per ottenere vantaggi. A seconda della \emph{preziosità} e \emph{storia} dell'oggetto offerto, a discrezione del Narratore, la Prova di Magia può prendere $\pm2d6$\ di modificatore.

\subsubsection{Recuperare da morente}\index{Recuperare da morente}\label{magieessereucciso}\index{Incantesimi, Inabile}

Se scendi a zero o sotto gli zero Punti Ferita perdi la metà dei Punti Magia rimanenti, con un minimo di 10 Punti Magia persi. Tutti gli incantesimi su cui stai tenendo la concentrazione vengono interrotti.

\subsubsection{Lanciare Incantesimi in Armatura}\index{Lanciare Incantesimi in Armatura}\label{magielanciareincantesimiinarmatura}\index{Incantesimi, in Armatura}

Data la concentrazione mentale e i gesti precisi richiesti l'armatura distrae e sbilancia i flussi. La Prova di Magia nel lancio dell'incantesimo è obbligatoria e viene modificata come indicato nella sezione delle \hyperlink{armatureemagie}{armature} (pag. \pageref{armatureemagie}).

\subsubsection{Opzionale - Incantesimi in Armatura}\index{Opzionale - Incantesimi in armatura}

E' proposta questa opzione per gestire gli incantatori in armatura:

- Questa opzione prevede tutti gli incantesimi lanciati dall'incantatore diventino con Gittata a Contatto, ovvero scaricabili solo tramite la mano dell'incantatore. Non sono richieste Prove di Magia per il fatto di portare l'armatura.


\subsubsection{Bersagli}\index{Bersagli}\label{magiebersagli}\index{Incantesimi, Bersagli}

Un normale incantesimo richiede che tu scelga uno o più bersagli che siano affetti dalla sua magia. La descrizione dell'incantesimo ti dice se l'incantesimo prende a bersaglio creature, oggetti o un punto di origine per generare un'area di effetto. A meno che l'incantesimo non abbia un effetto percepibile, una creatura potrebbe non capire mai di essere stata bersaglio di un incantesimo. Un effetto come un fulmine crepitante è palese, ma un effetto più subdolo, come il tentativo di leggere i pensieri di una creatura, di solito non viene notato, a meno che l'incantesimo non dica altrimenti.

Lanciare un incantesimo è una azione che non passa inosservata. Una prova di Furtività a difficoltà 15 oppure lanciare l'incantesimo come se si fosse Distratto permettono di celare il lancio, se non avviene proprio davanti all'osservatore.

\subsubsection*{Traiettoria Sgombra Verso il Bersaglio}\index{Incantesimi, vedere bersaglio}

\textbf{Per prendere come bersaglio una creatura od oggetto}, devi vederlo ed avere la traiettoria sgombera verso di essa, e quindi questa \textbf{non può trovarsi dietro una copertura completa}. Se piazzi un'area di effetto in un punto che non puoi vedere e un'ostruzione, come un muro, si trova tra di te e quel punto, il punto di origine si crea dal tuo lato più vicino dell'ostruzione (una Palla di Fuoco dietro una porta chiusa esplode al contatto con la porta dalla tua parte e non si manifesta oltre la porta).\index{Magia vedere il bersaglio}

\subsubsection*{Prendere Te stesso come Bersaglio}\index{Se stesso come bersaglio}\index{Incantesimi, se stesso come bersaglio}

Se un incantesimo prende come bersaglio una creatura a tua scelta od un alleato, puoi scegliere anche te stesso, a meno che la creatura non debba essere ostile o sia specificato che non possa essere tu. Se ti trovi nell'area di effetto di un incantesimo lanciato da te, anche tu ne sarai influenzato.

\subsubsection{Aree di Effetto}\index{Area di Effetto incantesimi}\label{magieareedieffetto}\index{Incantesimi, Area di effetto}

Incantesimi come Onda rovente e cono di freddo coprono un'area, permettendogli di colpire più creature alla volta.

La descrizione di un incantesimo specifica la sua area di effetto, che di solito rientra in una di queste cinque forme: cilindro, cono, cubo, linea o sfera. Ogni area di effetto ha un punto di origine, un luogo da cui si manifesta l'energia dell'incantesimo. Le regole per ciascuna forma specificano come posizionare il suo punto di origine. Di solito il punto di origine è un punto nello spazio, ma alcuni incantesimi hanno un'area la cui origine è una creatura o un oggetto. Il punto di origine deve essere sempre valido.

\begin{itemize}[leftmargin=*] \setlength{\itemsep}{0pt}
\item \emph{\textbf{Cilindro}}: il punto di origine di un cilindro è il centro di un cerchio di specifico raggio come indicato nella descrizione dell'incantesimo. L'energia in un cilindro si espande in linea retta dal punto di origine al perimetro del cerchio, formando la base del cilindro. L'effetto dell'incantesimo parte poi dal basso verso l'alto o dall'alto verso il basso fino a una distanza uguale all'altezza del cilindro. Il punto di origine del cilindro è incluso nella sua area di effetto.

\item \emph{\textbf{Cono}}: un cono si estende in una direzione a tua scelta dal suo punto di origine. La larghezza del un cono in un dato punto della sua lunghezza è uguale alla distanza di quel punto dal punto di origine. L'area di effetto di un cono specifica la sua lunghezza massima. Il punto di origine del cono non è incluso nella sua area di effetto a meno che tu non decida altrimenti.

Es. Un Cono di Freddo di 9 metri è largo al termine 9 metri e si allunga dal punto di origine di 9 metri, a 3 metri di distanza dal punto di origine è largo 3 metri.

\item \emph{\textbf{Cubo}}: selezioni il punto di origine di un angolo del cubo. Le dimensioni del cubo vengono espresse come lunghezza di ciascun suo spigolo. Il punto di origine del cubo non è incluso nella sua area di effetto, a meno che tu non decida altrimenti.

\item \emph{\textbf{Linea}}: una linea si estende dal suo punto di origine in un percorso dritto per tutta la sua lunghezza e copre un'area definita dalla sua larghezza. Il punto di origine della linea non è incluso nella sua area di effetto, a meno che tu non decida altrimenti. Una linea se non specificato diversamente è larga un quadretto.


\begin{center}
	\includegraphics[width=0.6\linewidth]{immagini/3dformev2.png}

	\emph{Cono, Sfera, Cilindro, Cubo. Il punto nero indica l'origine dell'incantesimo. Nella sfera è al centro della stessa.}
\end{center}

\item \emph{\textbf{Sfera}}: selezioni il punto di origine di una sfera, che deve essere valido (vedi Gittata e Bersagli) e la sfera si estenderà da quel punto fino ad incontrare un ostacolo insormontabile o la sua dimensione espressa nel raggio. La misura della sfera è indicata come raggio in metri che si estende da quel punto. Il punto di origine della sfera è incluso nella sua area di effetto.

Una palla di fuoco che viene generata in una stanza di 9x9 m ne prenderà una buona parte e in una stanza di 6x6 m la riempirà tutta. In una stanza di 3x3 m se ha modo di uscire da una porta od una finestra continuerà la sua esplosione fino ad arrivare ai 6 metri di raggio. Una palla di fuoco in un corridoio di 3x3 m lo saturerà per 6 metri avanti e indietro dal punto di origine.

\end{itemize}

\subsubsection{Rarità degli Incantesimi}\index{Rarità degli Incantesimi}\label{magieraritaincantesimi}\index{Incantesimi, Rarita' incantesimi}

Negli incantesimi è indicata la Rarità ovvero quanto è probabile trovare questo incantesimo o quanto può essere conosciuto. La rarità dipende non solo dal livello stesso dell'incantesimo, ovviamente gli incantesimi più potenti sono anche i più rari, ma anche da quanto normalmente sono diffusi e conosciuti. Il Narratore userà questa scala su 3d6 per valutare cosa può essere trovato più facilmente: % Comune (1-75\%) - Non Comune (76-93\%) - Raro (94-97\%) - Molto Raro (98-99\%) - Leggendario (100\%).
%, (1-70,71-93,94-97,98-99,100)
Comune (1-14) - Non Comune (15) - Raro (16) - Molto Raro (17) - Leggendario (18).

\subsubsection{Combinare Effetti Magici}\index{Combinare Effetti Magici}\label{magiecombinareeffettimagici}\index{Incantesimi, Combinare effetti}

Gli effetti di incantesimi diversi si sommano fino a che la loro durata si sovrappone. Gli effetti dello stesso incantesimo o che danno lo stesso bonus lanciato più volte sullo stesso bersaglio non si combinano. Sarà invece l'incantesimo più potente fra quelli lanciati, quello di livello più alto ed a parità quello che ha ottenuto più Critici Magici ad applicarsi finché le durate si sovrappongono.

In caso di incantesimi istantanei gli effetti agiscono singolarmente se agiscono nel medesimo segmento di iniziativa. Es. Se vengo colpito da un fulmine a segmento di iniziativa 4 e poi da un altro fulmine a segmento di iniziativa 8 farò due distinti Tiri Salvezza con relativa gestione del danno, se fossero nel medesimo segmento di iniziativa subirei solo quello più potente (vedi sopra).

%\subsection{Regole di base}\index{Regole Base per la Magia}\label{magieregoledibase}\index{Incantesimi, Regole base}

\subsection{Regole di base}\label{magieregoledibase}

\begin{itemize}[leftmargin=*] \setlength{\itemsep}{0pt}

\item
L'incantatore al lancio del suo primo incantesimo sceglie se utilizzare come modificatore alla Prova di Magia l'Intelligenza oppure se è un Devoto può scegliere la Caratteristica indicata dal Patrono. Una volta fatta la scelta non è più possibile cambiarla.

Questo modificatore viene chiamato \textbf{modificatore di caratteristica per incantesimi}.\index{Modificatore di caratteristica per incantesimi}
\item
Il personaggio quando assegna il primo punto di Competenza Magica \textbf{conosce} (sono presenti) nel suo Tomo della Magia un numero di Trucchetti pari al modificatore di caratteristica per incantesimi +2 (con un minimo di 4 Trucchetti) ed un numero di incantesimi di primo livello pari allo stesso modificatore, con un minimo di 4.
\item
Ogni giorno, dopo il riposo, il personaggio \textbf{apprende} dal sul suo Tomo di Magia un numero di incantesimi pari a Competenza Magica/2 (minimo 1) + modificatore di caratteristica per incantesimi + Adepto della Magia.\label{incantesimicm1}\hypertarget{incantesimicm1}{}
\item
Il numero di incantesimi formulabile al giorno dipende dalla capacità dell'incantatore. Vedi \textbf{Tabella Punti magia e Competenza Magica}. Un incantesimo ha un costo in Punti Magia pari al suo livello.
\item
Un Seguace aggiunge +1d6 alle Prove di Magia negli incantesimi delle liste privilegiate dal Patrono. I tuoi incantesimi possono usare una delle forme energetiche preferite dal Patrono.\index{Liste Privilegiate}\label{listeprivilegiate}\hypertarget{listeprivilegiate}{}
\item
Un Devoto aggiunge +1d6 alle Prova di Magia negli incantesimi delle liste privilegiate dal Patrono e può ignorare un dado tirato nella Prova di Magia. I tuoi incantesimi usano una delle forme energetiche preferite dal Patrono.
\item
Con il termine \textbf{appreso}\index{Incatesimi, Appreso} si intende un incantesimo presente sul Tomo della Magia che si è memorizzato e si può lanciare quando voluto.
\item
Con il termine \textbf{conosciuto}\index{Incantesimi, Conosciuto} si intente un incantesimo presente sul Tomo della Magia che però non si è appreso, ovvero non si è memorizzato e non si può lanciare quando voluto.
\end{itemize}

\subsection{Massimo livello di incantesimo lanciabile}\hypertarget{scuoleelivelli}{}\index{Livello Incantesimi per Abilità}\label{magieaccessoallelistedimagia}\index{Incantesimi, Massimo livello di incantesimo lanciabile}\index{Massimo livello di incantesimi lanciabili}\label{scuoleelivelli}

Mentre la Competenza Magica indica lo studio e dedizione alla Magia nella forma più astratta è l'Abilità Adepto della Magia che permette di capire quanto si è \emph{votati} al formulare gli incantesimi.

Per stabilire il livello massimo lanciabile di incantesimi sommate il punteggio di Competenza Magica ed Adepto della Magia, dividendo per due ed arrotondando per eccesso. Confrontate il risultato con il (doppio del punteggio del modificatore di caratteristica per incantesimi)+1, prendendo il valore minore.

Es. CM=8, Adepto della Magia preso 4 volte, (8+4)/2=6lv.

Es. CM=16 e Adepto della Magia 1 volta, (16+1)/2=9 livello di incantesimi.

Se l'incantatore ha come modificatore di caratteristica per incantesimo 0 non potrà lanciare incantesimi superiori al primo livello (vedi anche Abilità \hyperlink{Tutt'uno con la magia}{Tutt'uno con la magia}, pag. \pageref{Tutt'uno con la magia}).

Negli esempi sopra se il modificatore di caratteristica per incantesimi è 3 il massimo livello lanciabile sarà il 6lv e 7lv rispettivamente.

\subsection{Distratto - Problemi nel lancio dell'incantesimo}\index{Distratto - Problemi nel lancio dell'incantesimo}\index{Distratto}\label{magiedistratto}

Se l'incantatore è \textbf{Distratto}, cerca di nascondere il lancio della magia, è impedito, severamente disturbato, è sanguinante, afferrato, è sotto attacco/minacciato mentre cerca di lanciare un incantesimo, \textbf{che non sia un Trucchetto}, deve effettuare una \textbf{Prova di Magia}.

Per ogni tiro critico o critico magico che si è subito nel round la Prova di Magia viene fatta con +4 di difficoltà aggiuntiva.\index{Danno critico se si lancia incantesimo}. Eventuali Fallimenti Critici o Successi Critici vengono presi in considerazione.

\subsection{Prova di Magia}\index{Prova di Magia}\index{Successo critico magico}\index{Fallimento critico magico}\label{magieprovadimagia}\index{Incantesimi, Prova di Magia}

Non sempre lanciare un incantesimo è sufficiente, molte volte è necessario che questo funzioni bene ed anzi agisca oltre normali aspettative. L'incantatore può decidere di richiamare più energia nel lancio dell'incantesimo, ovvero effettuare un \emph{\textbf{Prova di Magia}} e confidare nelle sue capacità.

L'incantatore tira \textbf{3d6 + 1d6 ogni due punti di Competenza Magica} (arrotondato per eccesso) più eventuali bonus, Abilità o penalità (armatura, scudi, critici subiti).

L'incantatore può \textbf{ignorare un dado tirato} nella Prova di Magia per \textbf{ogni due volte} che ha preso \textbf{Adepto della Magia}.\index{Scartare dadi nella Prova di Magia} Questo per evitare di tirare tre volte 1.

La Prova di Magia si considera superata se il tiro è superiore a 10 + Livello dell'Incantesimo*2 + eventuali penalità. I Successi Critici o Fallimenti Magici si raffrontano a questo valore.\index{Fallimento Critico Magico}. In caso di Successo Critico Magico il costo dell'incantesimo diminuisce di 1 con un minimo di costo di 1.

Quando viene richiesto di superare o fare una Prova di Magia è sufficiente superare la difficoltà data dall'incantesimo e non fare un Fallimento Critico Magico. Se viene richiesto di ottenere un Successo Critico e la Prova di Magia non lo ottiene allora qualsiasi risultato ottenuto sarà considerato un Fallimento Critico.

La Prova di Magia come tutte le Prove segue le \hyperlink{goldenrules}{Golden Rules}, pag.\pageref{goldenrules}.

\begin{narratore}[Declamare la Magia]
Concedete un +1d6 nella Prova di Magia quando il personaggio declama con perizia e trasporto il lancio dell'incantesimo. Se dice \emph{Lancio una palla di fuoco} non otterrà vantaggi ma se con trasporto declama \emph{Per la Fiamma della Genesi possa Nedraf distruggervi con le sue sacre fiamme. Bruciate indegni. Palla di Fuoco!} allora si!.

\medskip

Fate che la partecipazione e recitazione guidi sempre il personaggio, coinvolgendo anche gli altri giocatori.
\end{narratore}

\medskip

%\begin{center}
%	\includegraphics[width=1\linewidth]{immagini/spellbook.png}
%\end{center}



\begin{giocatore}[Osare la Prova di Magia]
La Prova di Magia è una parte importante e integrante del sistema magico, usatela a vostro vantaggio. Non è solo questione di fortuna! Con le giuste Abilità potete evolvere un personaggio capace di dominare la sorte!\\

Non rinunciate al divertimento per paura di sbagliare. Meglio fallire clamorosamente in una esplosione di colori che rinunciare e basta!
\end{giocatore}


\subsection{Fallimento Critico nella Prova di Magia}\index{Fallimento Critico nella Prova di Magia}\label{magiefallimentocriticonellaprovadimagia}\index{Incantesimi, Fallimento Prova di Magia}\hypertarget{magiefallimentocriticonellaprovadimagia}{}

Se la Prova di Magia ha avuto almeno un Fallimento Critico Magico, tirato tre 1 oppure tirato veramente basso, tira 3d6 e consulta la seguente tabella. Per ogni Fallimento Critico Magico oltre il primo che si è manifestato nel lancio dell'incantesimo e per ogni Tiro Critico subito, tira un 1d6 in meno, fino a tirare un solo 1d6.

\medskip

\textbf{Tabella: Effetti Fallimento Critico magico}\index[Tabelle]{Tabella Effetti Fallimento Critico Prova di Magia}

\medskip

\noindent\begin{tabularx}{\linewidth}{l|X}
	\toprule
\rowcolor{gray!20}\textbf{Dadi} & \textbf{Effetti}\\
\toprule
1 & Per 1 giorno non sei più in grado di canalizzare energie magiche. Non puoi lanciare incantesimi se non facendo un successo magico critico nella Prova di Magia\\
\rowcolor{gray!20}2 & Aumenti la condizione di Affaticato di 2 gradi, fino ad un massimo di Affaticato 5\\
3 & Manifesti una modifica corporea minore\\
\rowcolor{gray!20}4 & Vieni investito da una roboante colonna di Luce e Vuoto. In un raggio di 6 metri centrato su di te, chiunque deve fare un Tiro Salvezza su Riflessi DC 15 per dimezzare o subire 3d10 di danni da forza non resistibili\\
5 & Per 3 round sei sotto l'influenza dell'incantesimo Confusione\\
\rowcolor{gray!20}6 & Per 1 minuto non sei più in grado di concentrarti e parli in rima\\
7 & Vieni teletrasportato di 3d10 metri in una direzione casuale\\
\rowcolor{gray!20}8 & Diventi Invisibile e paralizzato per 6 round\\
9 & Solo tu vieni avvolto da una cortina di oscurità magica impenetrabile per 6 round\\
\rowcolor{gray!20}10 & Non riesci a parlare bene, sei balbuziente. Ogni lancio di incantesimi ti costringe a superare una Prova di Magia. Durata 3 round\\
11 & Manifesti l'incantesimo Unto sotto i tuoi piedi\\
\rowcolor{gray!20}12 & Il prossimo incantesimo che lanci ha effetti se possibile minimizzati\\
13 & Il battito del tuo cuore è come il battito di un tamburo, si può sentire entro 36 metri\\
\rowcolor{gray!20}14 & Tutte le creature nel raggio di 36 metri sanno esattamente dove sei e cosa tentavi di fare\\
15 & Tutte le creature in una sfera di 9 metri di raggio centrata su di te subiscono 1d10 danni da Vuoto\\
\rowcolor{gray!20}16 & Guadagni 2d6 Punti Magia\\
17 & Una incudine cade, 3d6 di danno Tiro Salvezza su Riflessi DC 15 per dimezzare, su una creatura a caso, escluso te, entro sei metri\\
\rowcolor{gray!20}18 & Le creature, te escluso, nel raggio di 6 metri da te subiscono 3d10 danni da forza non resistibili
\end{tabularx}

\subsection{Modificare la Prova di Magia}

\textbf{Prima di effettuare} la Prova di Magia l'incantatore può decidere investire ulteriori Punti Magia per migliorare la sua Prova di Magia.

Per ogni volta, fino ad un massimo di tre volte, che paga il costo dell'incantesimo, può \textbf{aggiungere} 1d6 in più nella Prova di magia. \index{Prova di Magia, più dadi}

\textbf{Dopo aver effettuato} la Prova di Magia, usando una Reazione, per ogni due volte che paga il costo dell'incantesimo (fino ad un massimo di sei volte), può \textbf{ignorare} un dado tirato nella Prova di magia. \index{Prova di Magia, ignorare i dadi}

Un incantatore può anche \textbf{volontariamente fallire la Prova di Magia}.


\subsection{I Punti Magia}\index{I Punti Magia}\label{magiepuntimagia}\index{Incantesimi, Punti Magia}\hypertarget{magiepuntimagia}{}

A seconda del punteggio in Competenza Magica l'incantatore ha a disposizione un certo ammontare di Punti Magia.

\textbf{Gli incantesimi hanno un costo in Punti Magia pari al loro livello}\index{Punti Magia e costo incantesimi}

Ogni qual volta si lanci un incantesimo si sottrae il costo ai Punti Magia a disposizione per il giorno.
In caso di Trucchetti questi non consumano Punti Magia ma è necessario avere almeno 1 Punto Magia residuo.

L'incantatore ha un \textbf{bonus} al punteggio di Punti Magia pari al suo modificatore di caratteristica per incantesimi.

I Punti Magia si recuperano tutti con 8 ore di riposo. \index{Incantesimi, Recupero Ponti Magia}


\medskip

\textbf{Tabella: Competenza Magica (CM) e Punti Magia (PM)}\index[Tabelle]{Tabella Competenza Magica (CM) e Punti Magia (PM)}

\medskip

\noindent\begin{tabularx}{\linewidth}{XX|XX|XX}
	\toprule
 \rowcolor{gray!20}\textbf{CM} & \textbf{PM}&\textbf{CM} & \textbf{PM}&\textbf{CM} & \textbf{PM}\\
	\toprule
	1&	4  &	8&	28&	15&	53\\
 \rowcolor{gray!20}2&	7  &	9&	32&	16&	56\\
	3&	11 &	10&	35&	17&	60\\
 \rowcolor{gray!20}4&	14 &	11&	39&	18&	63\\
	5&	18 &	12&	42&	19&	67\\
 \rowcolor{gray!20}6&	21 &	13&	46&	20&	70\\
	7&	25 & 	14&	49&	20+&	prec.+ 3
\end{tabularx}

PM = (CM × 3) + (CM ÷ 2 arrotondato per eccesso) + Modificatore Caratteristica


\begin{giocatore}[Scegliere gli Incantesimi]
	Ogni incantesimo è un tesoro prezioso che si deve trovare ed imparare.

	Ogni incantesimo è alla stregua di un oggetto magico, un vero tesoro da cercare e ottenere!

	Dovrai intraprendere perigliose avventure, pagare mercenari, cercare i tomi antichi e svelare i segreti più oscuri e dimenticati per poter imparare nuovi incantesimi.
\end{giocatore}


\subsection*{Incantesimi come Rituali}\index{Incantesimi come Rituali}\index{Rituali, Incantesimi}

Specialmente ai primi livelli può essere molto fastidioso non aver appreso un incantesimo pur avendolo a disposizione nel Tomo della Magia.

L'incantatore può lanciare un incantesimo che sia presente sul suo Tomo di Magia e che sia entro il 3 livello ed entro il massimo livello di incantesimo lanciabile, allungandone il tempo di lancio ad 1 ora per costo in Punto Magia. In caso di incantesimo così lanciato non si usano Punti Magia, ma è necessario superare una Prova di Magia al termine della formulazione.


\subsection{Successo Critico Auto Magico}\index{Successo Critico Auto Magico}\index{Nova}\label{magienova}

L'incantatore può decidere di spendere, in aggiunta ai \textbf{dei Punti Magia} dell'incantesimo, un uguale ammontare per avere in automatico un \textbf{Successo Critico Magico}.
Ogni volta che voglio applicare un Successo Critico Magico aggiuntivo oltre il primo il costo in Punti Magia aumenta di 1. La dichiarazione di volere usare il Successo Critico Auto Magico è da dichiarare prima di effettuare, e superare, la Prova di Magia.


%\medskip

%\begin{center}
%	\includegraphics[width=0.7\linewidth]{immagini/Arthur-Pyle_The_Enchanter_Merlin.png}
%
%	\emph{Merlin. Howard Pyle, The Story of King Arthur and His Knights (1903)}
%\end{center}

Il tempo di lancio di un incantesimo potenziato in questa maniera aumenta di 1 Azione.

Es. \hyperlink{Velocità}{Velocità}, voglio che faccia 2 critici magici, pago 3 Punti Magia per lanciarla, più 3 per il primo Successo Critico Magico più 4 per il secondo Successo Critico Magico, ed eventualmente 5 per un terzo Successo Critico Magico. Si pagano sempre tutti i Punti Magia usati indipendentemente dal risultato della Prova di Magia.

Non si possono spendere più di metà dei Punti Magia attuali per potenziare un incantesimo, non si possono fare più Autocritici del modificatore di caratteristica per incantesimo.

%\subsection{L'Essenza della Magia}
%Diversi incantesimi hanno la possibilità di essere potenziati direttamente sfruttando il grezzo potere magico.
%Quando dopo il livello e la Lista di Magia è presente un \textbf{*} in fondo allora valgono queste regole:
%\begin{itemize}[leftmargin=*] \setlength{\itemsep}{0pt}
%\item ogni Punto Magia speso in più il danno aumenta di 1d8
%\item ogni 3 Punti Magia spesi in più la DC del Tiro Salvezza aumenta di 1
%\item il tempo di lancio aumenta di 1 Azione
%\item non è possibile usare più Punti Magia della metà del punteggio di Competenza Magica
%\item non è possibile lanciare nuovamente quell'incantesimo per 10 round
%\item non è possibile combinare l'\emph{Essenza della magia} con il \emph{Successo Critico %Automagico}
%\end{itemize}


\subsection{Opzionale - Il vero costo della Magia}\index{Opzionale - Il vero costo della Magia}

Il sistema magico può diventare sbilanciato abusando sempre degli stessi incantesimi. Per limitare questo sono proposti due approcci, da stabilire nella Sessione Zero:

\begin{itemize}[leftmargin=*] \setlength{\itemsep}{0pt}
\item Il costo in Punti Magia dell'incantesimo aumenta del costo stesso ogni volta che viene rilanciato (\emph{metodo suggerito})
\item Un incantatore può lanciare lo stesso incantesimo al massimo 1 volta al giorno
\end{itemize}

\subsection{Il Tomo della Magia}\index{Tomo della Magia}\index{Il Tomo della Magia}\label{magietomodellamagia}\index{Incantesimi, Tomo della Magia}

Se i Patroni sono la sorgente della magia è solo l'applicazione di antichi riti e formule che permette di manifestare questa energia grezza in una forma ed espressione che chiamiamo incantesimo.

Ogni usufruitore di magia ha uno o più \textbf{Tomo} degli incantesimi, non pensate solo a un grosso Tomo antico rilegato in pelle, le diverse culture hanno sviluppato nel tempo la capacità di iscrivere le rune degli incantesimi in carte, bastoni, lastre di pietra, tatuaggi... fate la vostra scelta quando create il personaggio.
Questa scelta non vi impedirà di copiare incantesimi da \textbf{Tomi} fatti diversamente, per voi sarà sempre facile (prova di Arcana DC 12) capire se si è di fronte ad un Tomo di qualche tipo.

Un nuovo personaggio con Competenza Magica 1, avrà un Tomo di Magia con un certo elenco di incantesimi. In questo Tomo sono presenti un numero di Trucchetti pari al modificatore di caratteristica per incantesimi +2 (con un minimo di 4 Trucchetti) ed un numero di incantesimi di primo livello sempre pari allo stesso modificatore, con un minimo di 4.\index{Incantesimi al primo livello}\label{tomocm1}\hypertarget{tomocm1}{}

Ogni incantesimo occupa un numero di pagine nel Tomo pari al proprio livello, con un minimo di una, \textbf{copiare una pagina di incantesimo} porta via 1 ora di lavoro e 10 mo di preziosi inchiostri.\index{Copiare Incantesimi sul Tomo}

Un Tomo (libro) di incantesimi costa 5 mo per pagina.

Un incantatore può copiare sul suo Tomo incantesimi il cui livello è di uno in più rispetto al suo massimo lanciabile (vedi \hyperlink{scuoleelivelli}{Massimo livello di incantesimi lanciabile}).

Se l'incantesimo è di più di due livelli più alto l'incantatore deve fare una Prova di Magia ed ottenere un Successo Critico Magico. Se il personaggio è un Devoto e l'incantesimo appartiene ad una Lista di Magia preferita del Patrono allora la Prova di Magia si esegue solo se l'incantesimo è di tre o più livelli superiori al massimo lanciabile.\index{Incantesimo non conosciuto}

Se non ottiene almeno un Successo Critico Magico non potrà tentare di copiare quell'incantesimo fino al prossimo punto di Competenza Magica acquisito. Se ottiene un Fallimento Critico Magico accadranno brutte cose al Tomo e 1d4 incantesimi casuali verranno cancellati dal Tomo stesso.

La sorgente di nuovi incantesimi può essere un altro Tomo o pergamena.. insomma qualsiasi cosa che il precedente incantatore usasse per custodire gli incantesimi. Un oggetto magico (bastone magico, anello, verga..bacchetta..) non è idoneo quale fonte da cui copiare l'incantesimo che contiene, si deve copiare dall'equivalente Tomo o pergamena di un altro incantatore. Un incantesimo quando copiato sul nuovo Tomo svanisce dalla sorgente originale.

\begin{narratore}[Magie vero tesoro]
Gli incantesimi diventano oggetti e premi magici a tutti gli effetti. Sfruttate la sete di conoscenza e potere dei personaggi per costruire avventure interessanti che possano ruotare attorno tomi antichi e leggendari incantesimi perduti.
\end{narratore}

\subsection{Studiare gli incantesimi}\index{Studiare gli incantesimi}\label{magiestudiareincantesimi}\index{Incantesimi, Studiare}

Il personaggio che vuole lanciare magie deve ogni giorno ripassare le antiche formule sul suo Tomo. Questa operazione è piuttosto rapida, impiegando solo 3 minuti per Competenza Magica.

Se l'incantatore non ha ripassato gli incantesimi dopo aver riposato almeno 6 ore deve superare una Prova di Magia per ogni incantesimo formulato finché non avrà ripassato.

L'incantatore può studiare gli incantesimi anche da più Tomi..

\subsection{Tiro per Colpire con le Magie}\index{Tiro per Colpire con Incantesimi}\label{magietiropercolpireconlemagie}\index{Incantesimi, Tiro per Colpire}\hypertarget{magietiropercolpireconlemagie}{}

Diversi incantesimi devono essere scagliati e colpire un avversario per funzionare.

Quando l'incantesimo ti dice di fare un \emph{Tiro per colpire con incantesimo} oppure \emph{attacco a distanza con incantesimo} devi effettuare un Tiro per Colpire contro la Difesa dell'avversario.

Questo Tiro per Colpire, che sia in mischia od a distanza, è effettuato con 3d6 + \textbf{Competenza Magica} + \textbf{Modificatore di caratteristica per incantesimi} + \textbf{Abilità} e \textbf{modificatori vari}.

E' anche possibile che sia richiesto un \textbf{Tiro per Colpire con incantesimo a tocco} ovvero l'attacco viene effettuato con un bonus di +1d6, come per Attacco a Tocco.\index{Attacco a Tocco con incantesimo}

Tiro per Colpire con incantesimo o con arma cumulano le penalità dell'attacco multiplo.\index{Penalita' attacco multiplo con incantesimo}

\medskip

Quando la magia è ad area non è necessario effettuare un Tiro per Colpire se non per mirare a difficili e minute specificate aree, ovvero si mira in una area ben circoscritta e si vuole evitare di colpire qualcuno con un incantesimo ad area.

\subsection{Tiro Salvezza - Resistere all'incantesimo}\index{Tiro Salvezza - Resistere all'Incantesimo}\index{Tiro Salvezza Incantesimi}\label{magietirosalvezza}\hypertarget{magietirosalvezza}{}

Il \textbf{Tiro Salvezza} imposto dal personaggio ha difficoltà (DC) pari a \textbf{10} + \textbf{Competenza Magica} + \textbf{modificatore caratteristica per incantesimo} + \textbf{numero di volte che si è presa l'Abilità Adepto della Magia} +\textbf{1 per ogni Successo Critico Magico} nella Prova di Magia.

Questa DC è usata per misurare la \emph{forza ed efficacia} dell'incantesimo quando confrontato con altri effetti.

Nella descrizione dell'incantesimo è scritto se è necessario un Tiro Salvezza e quale eseguire.\index{DC di una magia}

Se è il personaggio a dover resistere ad una magia il Narratore non ti dirà di fare un Tiro Salvezza a difficoltà 18, è lui che confronta il tuo tiro con la difficoltà, potrà dirti che la prova è complessa, difficile o facile...

\begin{itemize}[leftmargin=*] \setlength{\itemsep}{0pt}

\item
Se nel Tiro Salvezza tiri 3 volte 6 sei riuscito a passarlo, indipendentemente dal totale, ed ottieni un \textbf{Successo Critico Salvezza}.

\item
Se il Tiro Salvezza riesce per ogni margine di riuscita di 8 ottieni un \textbf{Successo Critico Salvezza}\index{Successo Critico Incantesimi}.

\item
Se nel Tiro Salvezza tiri 3 volte 1 hai fallito il tiro, indipendentemente dal totale, ed ottieni un \textbf{Fallimento Critico Salvezza}.\index{Tre 1 nei Tiri Salvezza magici}

\item
Se il Tiro Salvezza fallisce ed il margine di fallimento è almeno 8, per ogni margine di fallimento di 8 ottieni un \textbf{Fallimento Critico Salvezza}\index{Fallimento Critico Incantesimi}.

\end{itemize}

\begin{giocatore}[Tups lancia Dardo Tracciante!]
Tups che ha Intelligenza 4, Competenza Magica 6 e ha preso 2 volta Adepto della Magia, lancia l'incantesimo \hyperlink{Dardo Tracciante}{Dardo Tracciante}. La difficoltà (DC) del Tiro Salvezza su Riflessi sarà pari a 10 + 6 (CM) + 4 (modificatore caratteristica per incantesimo, Intelligenza) + 2 (ha preso 2 volte Adepto della Magia) ovvero 10+6+4+2 = 22 per dimezzare i danni. Se avesse fatto una Prova di Magia e questa avesse avuto un Successo Critico magico la DC sarebbe diventata 23.
\end{giocatore}

E' anche possibile che nella descrizione dell'incantesimo sia riportato cosa succede in caso di Successo o Fallimento Critico del Tiro Salvezza.

Per i \textbf{mostri} o comunque per un lancio di incantesimi dato da abilità magiche innate, se non specificato la \textbf{DC del Tiro Salvezza è pari alla 12 + 2 x livello dell'incantesimo + Intelligenza o modificatore di incantesimi indicato}.\index{DC Tiro Salvezza Incantesimo mostri}\index{Difficoltà incantesimi dei mostri}\label{tirosalvezzainccmostro}

\subsection{Contrastare gli Incantesimi}\index{Contrastare gli incantesimi}\label{contrastareincantesimi}\hypertarget{contrastareincantesimi}{}

Diversi incantesimi interagiscono con altri effetti annullandoli o modificandoli. Quando è scritto che un incantesimo \textbf{contrasta} o è \textbf{contrastato} un altro è necessario verificare la DC degli incantesimi o effetti per accertarsi quale effetto domini sull'altro.

Ad esempio l'incantesimo \hyperlink{Lentezza}{Lentezza} contrasta \hyperlink{Velocità}{Velocità}, \hyperlink{Rimuovi Maledizione}{Rimuovi Maledizione} sulle maledizioni, \hyperlink{Rimuovi Veleno}{Rimuovi Veleno} sui veleni...

Il \textbf{proprio valore di contrasto} si computa con una prova di 3d6 + CM + modificatore di caratteristica per incantesimi + volte che si è preso Adepto della Magia.\index{Valore di Contrasto} + 1 per Successo Critico Magico ottenuto nella Prova di Magia.\index{Annullare gli incantesimi}

Il valore di contrasto si \textbf{confronta} con la DC dell'effetto magico da contrastare.

Quando per un incantesimo non è fornita la DC da contrastare allora considerate la stessa pari a 12+Livello incantesimo*2.

\subsection{Concentrazione}\index{Colpito mentre concentrato}\index{Concentrazione}\label{magieconcentrazione}\hypertarget{magieconcentrazione}{}

Perdi la concentrazione su di un incantesimo se lanci un altro incantesimo che richieda concentrazione. Non ti puoi concentrare su due incantesimi alla volta. Interrompere la concentrazione costa una Azione Immediata.

Se vieni colpito\index{Colpito mentre tieni la concentrazione} mentre sei concentrato su un incantesimo devi effettuare una Prova di Magia con difficoltà pari all'incantesimo su cui ti stai concentrando ed ottenere almeno 1 Successo Critico Magico, +1 per ogni critico subito oppure perdere la concentrazione.

%Mentre sei concentrato puoi lanciare solo Trucchetti senza difficoltà altrimenti se riesci in una Prova di Magia con un Successo Critico Magico puoi anche lanciare incantesimi non Trucchetti. Mantenere la Concentrazione costa 1 Azione a round.

Mentre sei concentrato puoi lanciare solo Trucchetti senza difficoltà altrimenti se riesci in una Prova di Magia puoi anche lanciare incantesimi non Trucchetti.

Ogni 6 punti di Competenza Magica si può mantenere la concentrazione su un incantesimo aggiuntivo. Se vieni colpito devi fare una prova per ogni incantesimo su cui stai mantenendo la concentrazione. In caso di fallimento anche su un solo incantesimo tutte le concentrazioni mantenute vengono interrotte.

Per ogni incantesimo su cui si mantiene la concentrazione si usa 1 Azione a round.

\subsection{Conservare la magia}\index{Conservare la magia}\label{magieconservare}

L'incantatore può lanciare l'incantesimo e trattenerlo nel suo pugno, senza manifestarlo. Per poter trattenere un incantesimo l'incantatore formula la magia e spende subito una Azione per rimanere Concentrato e paga 1 Punto Magia addizionale.
L'incantesimo può essere trattenuto fino ad 1 round per punteggio di modificatore di caratteristica da incantesimi +1 round per volte che ha preso Adepto della Magia.

Per trattenere l'incantesimo l'incantatore deve rimanere Concentrato e pagare 1 Punto Magia per round.
Per lanciare l'incantesimo conservato è sufficiente tirare l'iniziativa ed usare 1 Azione. Non è possibile lanciare ulteriori incantesimi che non siano Trucchetti finché si conserva un incantesimo.

\subsection{Regole per le creature evocate}\index{Regole per le creature evocate}\index{Evocazione creature}

Queste regole valgono per tutte le creature evocate magicamente.

La creatura evocata agisce nel tuo round, non deve tirare l'iniziativa, bensì usa la tua ed è amichevole verso di te e i tuoi compagni. Una creatura evocata può già agire nel round in cui è stata evocata.

Una creature evocata ha 2 azioni a round, se non vengono specificati ordini al momento dell'evocazione la creatura si difende e contrattacca chi l'ha attaccata.

La creatura evocata comprende i comandi che gli vengono dati al meglio delle sue capacità mentali. Per cambiare l'ordine si deve usare una Azione.

\subsection{Tentare più Magie nello stesso round}\index{Tentare più Magie nello stesso round}\index{Magie multiple nello stesso round}\hypertarget{piumagieround}{}\label{piumagieround}

E' possibile lanciare più magie nel round purché il tempo totale di lancio non superi le Azioni a disposizione e si superi una Prova di Magia al lancio del secondo incantesimo. più incantesimi che richiedano il Tiro per Colpire o attaccare con un arma e lanciare un incantesimo cumulano le penalità del multiattacco.\index{Incantesimi e multiattaco}

\subsection{Alterare le Magie}\index{Alterare le Magie}\label{magiealteraremagie}

L'incantatore può modificare gli incantesimi in diversi modi. Queste possibilità aggiungono versatilità all'incantatore ed è opportuno che il personaggio le abbia sempre presenti nelle situazioni più critiche.

%- \textbf{Magie Punitive}\index{Magie Punitive}\label{magiepunitive}\hypertarget{magiepunitive}{}: pagando due volte il costo dell'incantesimo in Punti Magia puoi tirare un dado in più nella Prova di Magia ed ignorare un dado tirato. La capacità è usabile fino a 3 dadi in più per incantesimo. Da parte del compagno è una Azione di Reazione da dichiararsi prima della Prova di Magia.

- \textbf{Magia eterea}\index{Magia eterea}: aumentando di 3 i Punti Magia spesi nell'incantesimo le proprie magie hanno pieno effetto su creature eteree o incorporee. Azione Immediata da dichiararsi prima del lancio dell'incantesimo.

- \textbf{Sacrificio Magico}\index{Sacrificio magico}: l'incantatore riducendo i suoi Punti Ferita attuali e Massimi di 4 acquisisce 1 Punto Magia da usare contestualmente al lancio di una magia. Non puoi sacrificare più di metà dei Punti Ferita attuali alla volta. Azione Immediata.

- \textbf{Magia pietosa}\index{Magia pietosa}: aumentando di 3 i Punti Magia spesi le magie infliggono danni temporanei. Le magie che infliggono danni di un tipo particolare (come da fuoco) infliggono danni temporanei dello stesso tipo. 1 Azione.

- \textbf{Magia mirata}\index{Magia mirata}: ogni 2 Punti Magia che paghi in aggiunta al costo dell'incantesimo puoi escludere una persona dall'area di effetto dell'incantesimo. Non puoi escludere più persone di quante volte non abbia preso l'Abilità Adepto della Magia. 1 Azione. %1 punto magia per livello incantesimo per creatura esclusa

- \textbf{Magia lontana}\index{Magia lontana}: aumentando di 1 i Punti Magia usati aumenti fino a 9 metri la distanza di lancio dell'incantesimo. 1 Azione.

- \textbf{Aumentare il tempo}\index{Aumentare il tempo} di lancio da 2 Azioni a 3 Azioni diminuisce di 1 i Punti Magia spesi per il lancio di incantesimo, con un minimo di costo di 1 Punto Magia.

%- \textbf{Circolo del Potere}\index{Circolo del Potere}: più incantatori che siano tutti Devoti o Seguaci dello stesso Patrono possono collaborare affinché uno di loro riesca meglio nel lancio di un incantesimo.
%Ogni incantatore sacrifica metà dei Punti Magia dell'incantesimo lanciato dal compagno e supera una Prova di Magia. Ogni due compagni che superano la Prova di Magia si genera un Successo critico magico, fino ad un massimo di 7 successi critici magici. Il tempo di lancio di un incantesimo tramite Circolo di Potere diviene almeno 1 Turno. Requisito Competenza Magica 5.

\medskip

Le possibilità concesse da Alterare le Magie sono cumulabili tra loro.

\medskip

\textbf{Modifiche lievi} \index{Modifiche lievi agli incantesimi} alla manifestazione dell'incantesimo possono essere concordati con il Narratore per un costo di Punti Magia aggiuntivi o con una Prova di Magia riuscita.

\subsection{Tentare Incantesimi con impedimenti}\index{Tentare Incantesimi con impedimenti} \index{Impedimenti}\label{magieconimpedimenti}\hypertarget{magieconimpedimenti}{}

Il lancio di un incantesimo è vincolato a gesti e parole particolari e unici. Quando il personaggio si trova in una situazione in cui non può gesticolare o parlare allora può tentare di lanciare l'incantesimo comunque anche se diventa molto più difficile.

I Punti Magia richiesti per il lancio di incantesimi se non può gesticolare vengono triplicati e se non può parlare vengono ulteriormente triplicati, è necessario in ogni caso superare una Prova di Magia.

\subsection{Definizioni obiettivi degli incantesimi}\index{Obiettivi degli incantesimi}\label{magiedefinizioniobiettivi}

Negli incantesimi sotto elencati troverete spesso i riferimenti alle tipologie di soggetti ed obiettivi influenzabili nonché a diverse tipologie di energia ed elementi.

- Le \textbf{Creature Naturali} sono Insetti, Rettili, Bestie, Umanoidi, Piante, Creature acquatiche, Mostruosità, Melme.

- Le \textbf{Creature Magiche} sono: Immondi (Diavoli e Demoni), Fatati, Spiriti, Non morti, Giganti, Celestiali, Elementali, Costrutti, Aberrazioni (tutto ciò che è alieno o innaturale) ed i Draghi.

Se una Creatura Naturale ha poteri magici allora si considera anche come Creatura Magica. Una descrizione più completa di queste categorie la trovate nel Capitolo del Mostruario.

- \textbf{Energia} comprende: Forza, Fuoco, Luce, Suono, Elettricità, Energia Positiva, Energia Negativa, Freddo, Vuoto.\label{elencoenergia}\hypertarget{elencoenergia}{}

\subsection{Danni da Energia, Luce e Vuoto}

Il danno causato da \textbf{Luce}\index{Luce} è per metà da fuoco e per metà da energia positiva, ovvero una resistenza al fuoco od all'energia positiva si applica solo su metà del danno causato dall'attacco.

Il danno causato da \textbf{Vuoto}\index{Vuoto} è per metà da freddo e per metà da energia negativa, eventuali protezioni si applicano alle rispettive metà del danno.

Essere Immuni o avere una Resistenza alla Luce o Vuoto non rende immune o resistenti a sua volta ai danni da Fuoco/Energia Positiva o Freddo/Energia Negativa.

La sola \textbf{energia negativa} danneggia\index{Energia Negativa} i viventi e cura i non morti, la sola \textbf{energia positiva}\index{Energia Positiva} danneggia i non morti ma non cura i viventi (a discrezione del Narratore l'esposizione per un round potrebbe equivalere ad un incantesimo di Ristorare Inferiore), vedi anche descrizioni dei Piani. Un obiettivo prende danno pieno da Luce o da Vuoto se non ha resistenze inerenti.

Un caso particolare è l'\textbf{energia positiva Curativa}\index{Energia positiva curativa} che cura i viventi e danneggia i non morti. Questa energia è quella dell'Imposizione delle mani, Incanalare energia e degli incantesimi di Cura.\index{Energia positiva su non morti}

\subsection{Abilità di Lista}\label{abilitadilista}\index{Abilità di Lista}\hypertarget{abilitadilista}{}

%\begin{enfasi}{
%Volli, e volli sempre, e fortissimamente volli (Vittorio Alfieri, 06/09/1783, Lettera a Ranieri dè Calzabigi)
%}\end{enfasi}\end{changemargin}

Lo studio della magia e l'approfondita conoscenza delle Liste di Magia porta l'incantatore ad imparare aspetti non sempre conosciuti della stessa.

Le capacità indicate sono attivabili se con le Azioni precedenti l'incantatore ha lanciato un incantesimo, non Trucchetto, dalla Lista di Magia indicata. L'incantatore può scegliere il potere che vuole purché abbia preso Adepto della Magia almeno il numero indicato di volte.

\emph{\textbf{Lista Abiurazione}}

\textbf{3: Scudo Minore.} Usando una Azione Immediata sei in grado di canalizzare le energie magiche che ti pervadono manifestando una protezione. Fino all'inizio del tuo prossimo round hai un +1 alla Difesa.

\textbf{4: Protezione Maggiore.} Usando una Azione Immediata sei in grado di canalizzare le energie magiche che ti pervadono manifestando una protezione. Scegli fino a 2 creature nel raggio di 6 metri, fino all'inizio del tuo prossimo round prendono +1 Difesa oppure +1 ai Tiri Salvezza, da scegliere al momento dell'utilizzo del potere.

\emph{\textbf{Lista dell'Acqua}}

\textbf{3: Acque profonde.} Usando una Azione Immediata acquisisci resistenza 5 al freddo ed al fuoco fino all'inizio del tuo prossimo round.

\textbf{5: Acque limpide.} Usando una Azione Immediata puoi toccare una creatura e lo aiuti a liberarsi da veleni e tossine. Viene concesso un nuovo Tiro Salvezza (se possibile) per perdere la condizione di avvelenato.

\emph{\textbf{Lista dell'Aria}}

\textbf{3: Fra le nuvole.} Usando una Reazione sei in grado di lanciare su te stesso l'incantesimo Caduta Piuma senza usare Punti Magia.

\textbf{4: Scossa.} La tua mano manifesta un crepitio di elettricità, il prossimo incantesimo che lanci entro la fine del prossimo round che abbia un Tiro per Colpire causa 2d8 danni in più da elettricità. Costa una Azione Immediata.

\textbf{Lista Ammaliamento}

\textbf{3: Distrazione.} Quando una creatura che puoi osservare entro 9 metri da te effettua un attacco con armi o magie, puoi usare una Reazione per distrarlo fino a fine round. La creatura ha -2 al Tiro per Colpire.

\textbf{4: Distrazione maggiore.} Quando una creatura che puoi osservare entro 9 metri da te effettua un attacco con armi o magie, puoi usare una Reazione per distrarlo fino a fine round. Tira 1d6, se il risultato è 3-4-5 la creatura ha -2 al Tiro per Colpire, se il risultato è 6 il bersaglio dell'attacco della creatura diventa casuale tra i bersagli colpibili.

\emph{\textbf{Lista Animali e Piante}}

\textbf{3: Corteccia.} Usando una Azione Immediata rendi la tua pelle più dura e resistente. Hai una riduzione del danno pari a 2 fino all'inizio del tuo prossimo round.

\textbf{4: Artigli.} Usando una Azione Immediata rendi ancora più affilati i tuoi attacchi naturali fino alla fine del prossimo round. Ogni attacco naturale andato a segno causa Sanguinamento 1, cumulabile fino a Sanguinamento 5.

\emph{\textbf{Lista Cura}}

\textbf{3: Mano calda.} Usando una Azione Immediata fai che l'incantesimo di cura che hai lanciato ti curi un numero di Punti Ferita aggiuntivi pari al livello dell'incantesimo stesso.

\textbf{4: Spirito benevolo.} Usando una Azione canalizzi l'energia residua di un tuo incantesimo. Usando una Azione Immediata puoi toccare una creatura e curarla di 1d4 Punti Ferita.

\emph{\textbf{Lista Divinazione}}

\textbf{3: Premonizione.} Usando una Azione Immediata ha una fugace previsione degli accadimenti futuri. Fino all'inizio del tuo prossimo round hai un +1 ai Tiri Salvezza su Riflessi.

\textbf{4: Punto Cieco.} Usando una Azione Immediata puoi toccare una creatura, fino all'inizio del tuo prossimo ha un +2 al Tiro per Colpire.

\emph{\textbf{Lista Evocazione}}

\textbf{3: Mano cava.} Con una Azione Immediata, puoi fare scomparire e riapparire quando vuoi un oggetto di volume L. Non puoi trattenere più di tre oggetti in questa maniera.

\textbf{4: Passo cauto.} Con una Azione Immediata fai che la successiva Azione di Movimento non causi attacchi d'opportunità.

\emph{\textbf{Lista del Fuoco}}

\textbf{3: Gola arrossata.} Con una Azione Immediata sputi un getto di fuoco in un quadretto entro 1 metro di distanza. Il terreno si considera difficile ed attraversarlo o sostare causa 1d6 Punti Ferita da Fuoco. Dura fino all'inizio del tuo round successivo.

\textbf{4: Napalm.} Con una Azione Immediata tocchi un arma. L'arma viene avvolta da fiamme, fino alla fine del tuo round successivo causa 1d8 di danno da Fuoco in più.

\emph{\textbf{Lista Illusione}}

\textbf{2: Prestidigitatore.} Puoi usare l'incantesimo Prestidigitazione con una Reazione.

\textbf{5: Abbondanza} Con una Reazione puoi creare un oggetto inorganico di volume 1 o inferiore del valore di 1 mo o meno. L'oggetto permane per 1 Turno o finché non viene usata nuovamente questa capacità.

\emph{\textbf{Lista Invocazione}}

\textbf{3: Speranza.} Con una Azione Immediata puoi illuminare la tua mano fino all'inizio del tuo round successivo. La mano illumina il solo tuo quadretto ed è luce fioca nel quadretto dopo.

\textbf{4: Augurio.} Con una Azione Immediata tocchi una creatura cedendogli un tuo Punto Fato.

\emph{\textbf{Lista Necromanzia}}

\textbf{3: Sangue Nero.} Usando una Azione Immediata fino all'inizio del tuo round successivo ignori la condizione di affaticato.

\textbf{4: Sangue Morto.} Usando una Azione Immediata puoi toccare una creatura. Questa prende +2 ai Tiri Salvezza su Tempra e -1 ai Tiri Salvezza su Riflessi fino all'inizio del tuo round successivo.

\emph{\textbf{Lista della Terra}}

\textbf{3: Colla.} Sei in grado di lanciare l'incantesimo Riparare come Azione Immediata senza spendere Punti Magia.

\textbf{4: Titano.} Usando una Azione, ogni volta che lanci un incantesimo della Lista della Terra, purché tu sia in contatto con la terra solida recuperi un numero di Punti Ferita pari al livello dell'incantesimo lanciato.

\emph{\textbf{Lista Trasmutazione}}

\textbf{3: Condivisione.} Usando una Azione Immediata tocchi una creatura, la creatura guadagna una Reazione in più da usarsi entro la fine del tuo round prossimo.

\textbf{4: Salto.} Con una Azione Immediata ti sposti nello spazio. Tira un 1d6, appari in un quadretto non occupato a quella distanza.

\textbf{5: Apparenze.} Con una Azione Immediata alteri la tua presenza nello spazio. Tira un 1d6, se fai 6 fino all'inizio del tuo prossimo round diventi invisibile.

\emph{\textbf{Lista Universale}}

\textbf{3: Vista.} Acquisisci l'Abilità Occhi della Magia, come se l'avessi scelta 2 volte.

\textbf{4: Precisione.} Esegui la Prova di Magia con un d6 in più e puoi ignorare un dado nella Prova di Magia.

\textbf{5: Conoscere.} Puoi lanciare l'incantesimo Identificare con una Azione Immediata, senza usare Punti Magia.

\textbf{Queste capacità non sono concesse a chi prende l'Abilità \hyperlink{Un solo credo}{Un solo credo}} (pag. \pageref{Un solo credo}).

%\begin{center}

%\includegraphics[width=0.6\linewidth]{immagini/Hex32.png}

%\emph{The Witchcraft Art of Jacques de Gheyn II}

%\end{center}

\subsection*{Circa OBSS e la dimensione degli incantesimi}

Partiamo da un assunto: in OBSS una creatura media occupa uno spazio di $(1*1)\si{m^{2}}$ mentre nella 5e la stessa creatura occupa $(1.5*1.5)\si{m{^2}}$ (2.25\si{m{^2}}).

Una \hyperlink{Palla di Fuoco}{Palla di Fuoco} agisce in un raggio di 6 metri, ovvero \emph{occupa} ($6*6*\pi$)\si{m^{2}}, in questo spazio ci stanno \emph{circa 110} creature in OBSS e 50 creature, sempre medie, della 5e!

A fare due conti la Palla di Fuoco dovrebbe avere un raggio di 4 metri per influenzare lo stesso numero di creature!

Se questo fattore influisce troppo sul vostro gioco riducete della metà ($r-r/2$) i raggi delle esplosioni e le lunghezze dei coni.

\end{multicols}

\vfill

%\begin{center}
%	\includegraphics[width=0.40\linewidth]{immagini/The_Chariot_of_Hermes.png}
%
%	\emph{Tarocchi - il Carro}
%\end{center}



%\vfill

%\begin{center}
%\includegraphics[width=0.8\linewidth]{immagini/Binsfeld_witches_grayscale.png}
%\medskip
%\emph{Artwork depicting various witch practices, 1592, Peter Binsfeld}
%\end{center}

\begin{center}

\includegraphics[width=0.55\linewidth]{immagini/David_Ryckaert_III_La_Ronde_des_farfadets_grayscale.png}

\medskip

\emph{Peinture à l'huile sur toile de 1651}

\end{center}

%\vfill
%
%\begin{center}
%\includegraphics[width=0.47\linewidth]{immagini/Voynich_Manuscript.png}
%
%\smallskip
%
%\emph{Una pagina dal manoscritto di Voynich, tuttora indecifrato.}
%\end{center}

\pagebreak

\section{Gli Incantesimi}

\begin{multicols}{2}

%\begin{tcolorbox}[title = più effetti speciali!]
%Gli incantesimi elencati sono quelli della 5ed più alcune mie proposte ed altre rivisitazioni. Se avete suggerimenti per il Narratore per gestire critici non previsti parlatene con lui! Lo spirito di collaborazione deve essere sempre costruttivo.
%\end{tcolorbox}\end{changemargin}

%\begin{narratore}\index{Opzionale - Alternativa al danno da Successo Critico Magico}
%Una alternativa agli effetti del Successo Critico Magico può essere che con incantesimi che causino danno diretto o curino, al posto del dado aggiuntivo si sommi direttamente metà del valore del dado arrotondato per eccesso. Quindi 1d10 diventa 6, 1d8 diventa 5, 1d6 diventa 4, 1d4 diventa 3. Es. Un danno da 1d10 + 1d8 per effetto critico diventa 1d10+5.
%\end{narratore}\end{changemargin}

\subsection{Elenco incantesimi}

\end{multicols}

\begin{enfasi}
La magia è credere in te stesso. Se riesci a farlo, puoi fare accadere qualsiasi cosa. (Johann Wolfgang von Goethe)
\end{enfasi}

\begin{multicols}{2}

%\smallskip\noindent\rule{\linewidth}{2pt} \index[Incantesimi]{Aiuto} \hypertarget{Aiuto}{}\medskip\noindent{\textbf{Aiuto}}\pdfbookmark[3]{Aiuto}{Aiuto}\noindent
%\noindent\begin{tabular}{p{2.8cm}|p{\dimexpr 0.5\textwidth - 3cm - 2\tabcolsep - 1pt\relax}}
%	\textbf{Lista di Magia} & Cura, Necromanzia \\
%	\textbf{Livello} & 2, Non Comune \\
%	\textbf{Tempo di Lancio} & 2 Azioni \\
%	\textbf{Gittata} & 9 metri \\
%	\textbf{Componenti} & V, S, M (una sottile striscia di tessuto bianco) \\
%	\textbf{Durata} & 1 ora per Competenza Magica \\
%\end{tabular}

\incantesimo{Aiuto}
\noindent\colorbox{OBSSgold!10}{
\begin{minipage}{0.95\linewidth}
\begin{description}[noitemsep, topsep=0pt, parsep=0pt, partopsep=0pt, leftmargin=0cm, labelwidth=1.3cm]
\item[\textbf{Lista}]: Cura, Necromanzia
\item[\textbf{Livello}]: 2, Non Comune
\item[\textbf{Lancio}]: 2 Azioni
\item[\textbf{Gittata}]: 9 metri
\item[\textbf{Durata}]: 1 ora per CM
\end{description}
\end{minipage}}\smallskip


Il tuo incantesimo aumenta la robustezza e risolutezza dei tuoi alleati. Scegli fino a tre creature a gittata, per la durata prendono 5 Punti Ferita Temporanei. Non è possibile ricevere più di un incantesimo Aiuto al giorno.

\textbf{Per ogni Successo Critico Magico} ottenuto nella Prova di Magia puoi influenzare una persona in più o aumentare i PF temporanei di 5.

\incantesimo{Allarme}
\noindent\colorbox{OBSSgold!10}{
\begin{minipage}{0.95\linewidth}
\begin{description}[noitemsep, topsep=0pt, parsep=0pt, partopsep=0pt, leftmargin=0cm, labelwidth=1.3cm]
\item[\textbf{Lista}]: Abiurazione
\item[\textbf{Livello}]: 1, Comune
\item[\textbf{Lancio}]: 1 minuto
\item[\textbf{Gittata}]: 9 metri
\item[\textbf{Durata}]: 2 ore per CM (massimo 24 ore)
\end{description}
\end{minipage}}\smallskip

Predisponi un allarme contro intrusioni indesiderate. Scegli una porta, una finestra o un'area a gittata che non sia più grande di un sfera di 3 metri di raggio. Fino al termine dell'incantesimo, sarai avvertito da un allarme ogni volta che una creatura di taglia Minuscola o superiore entri in contatto o acceda all'area protetta. Quando lanci l'incantesimo, puoi indicare delle creature che non faranno scattare l'allarme Scegli anche se l'allarme è udibile o solo mentale. Un allarme mentale, qualora ti trovi entro 1,5 chilometri dall'area protetta, ti avverte con un rumore nella tua mente. Il rumore è in grado di svegliarti se stai dormendo. Un allarme udibile produce il suono di una campanella per 10 secondi, udibile entro 18 metri.

\textbf{Per ogni Successo Critico Magico} ottenuto nella Prova di Magia la durata aumenta di 2 ore.

\incantesimo{Allucinazione Mortale}
\noindent\colorbox{OBSSgold!10}{
\begin{minipage}{0.95\linewidth}
\begin{description}[noitemsep, topsep=0pt, parsep=0pt, partopsep=0pt, leftmargin=0cm, labelwidth=1.3cm]
\item[\textbf{Lista}]: Illusione
\item[\textbf{Livello}]: 4, Non Comune
\item[\textbf{Lancio}]: 2 Azioni
\item[\textbf{Gittata}]: 36 metri
\item[\textbf{Durata}]: Istantanea
\end{description}
\end{minipage}}\smallskip

Attingi agli incubi di una creatura a gittata e che puoi vedere e crei una manifestazione illusoria delle sue più terrificanti paure, visibile solo per quella creatura. Il bersaglio deve effettuare un Tiro Salvezza su Volontà.

Se fallisce il Tiro Salvezza, il bersaglio è spaventato per 1 minuto e subisce 4d10 di danno.

\textbf{Tiro Salvezza Successo/Fallimento Critico}: In caso di Fallimento Critico il danno raddoppia, in caso di Successo Critico il danno viene ulteriormente dimezzato

\textbf{Per ogni Successo Critico Magico} ottenuto nella Prova di Magia il danno aumenta di 2d10

\incantesimo{Alterare Sé Stesso}\label{Alter Self}
\noindent\colorbox{OBSSgold!10}{
\begin{minipage}{0.95\linewidth}
\begin{description}[noitemsep, topsep=0pt, parsep=0pt, partopsep=0pt, leftmargin=0cm, labelwidth=2.8cm, labelsep=0.2cm]
\item[\textbf{Lista}]: Trasmutazione
\item[\textbf{Livello}]: 2, Non Comune
\item[\textbf{Lancio}]: 2 Azioni
\item[\textbf{Gittata}]: Personale
\item[\textbf{Durata}]: 1 minuto per CM
\end{description}
\end{minipage}}\smallskip

Assumi una forma diversa. Quando lanci questo incantesimo, scegli una della seguenti opzioni, l'effetto della quale permane per la durata dell'incantesimo. Per la durata dell'incantesimo puoi terminare un'opzione per ottenere i benefici di un'altra.

\emph{Adattamento Acquatico}. Adatti il tuo corpo a un ambiente acquatico, sviluppando branchie e dita palmate. Puoi respirare sott'acqua e ottieni velocità di nuoto pari alla metà della tua velocità di movimento.

\emph{Armi Naturali}. Sviluppi degli artigli, zanne, spuntoni, corna o una diversa arma naturale a tua scelta. I tuoi colpi senz'armi infliggono 1d6 danni contundenti, perforanti o taglienti, come appropriato all'arma naturale scelta con la quale sei competente. Infine, l'arma naturale è considerata come un arma +1. In questa forma usa la CA per calcolare i Tiri per Colpire.

\emph{Cambio di Aspetto}. Trasformi il tuo aspetto. Decidi il tuo aspetto esteriore, compresa l'altezza, il peso, i lineamenti facciali, il suono della tua voce, la lunghezza dei capelli, il colorito e qualsiasi peculiarità tu desideri. Puoi apparire come membro di un'altra razza, sebbene nessuna delle tue statistiche cambi. Inoltre non puoi apparire come una creatura di taglia diversa dalla tua, e la tua forma base resta la medesima; se sei bipede, non puoi usare questo incantesimo per diventare quadrupede, per esempio.

In qualsiasi momento della durata dell'incantesimo, puoi usare due Azioni per cambiare nuovamente di aspetto in questo modo.

\textbf{Per ogni Successo Critico Magico} ottenuto nella Prova di Magia puoi alterare un altro soggetto o raddoppiare la durata.

\incantesimo{Amicizia con gli Animali}
\noindent\colorbox{OBSSgold!10}{
\begin{minipage}{0.95\linewidth}
\begin{description}[noitemsep, topsep=0pt, parsep=0pt, partopsep=0pt, leftmargin=0cm, labelwidth=1.3cm]
\item[\textbf{Lista}]: Animali e Piante
\item[\textbf{Livello}]: 1, Non Comune
\item[\textbf{Lancio}]: 2 Azioni
\item[\textbf{Gittata}]: 9 metri
\item[\textbf{Durata}]: 24 ore
\end{description}
\end{minipage}}\smallskip

Questo incantesimo ti permette di convincere una bestia naturale che non vuoi arrecargli danno. Scegli una bestia a gittata che puoi vedere. Questa deve vederti e udirti. Se l'Intelligenza della bestia è 2 o più l'incantesimo fallisce. Altrimenti la bestia deve superare un Tiro Salvezza su Volontà o restare affascinata da te per la durata dell'incantesimo. Se tu o uno dei tuoi compagni danneggiate il bersaglio l'incantesimo ha termine.

\textbf{Per ogni Successo Critico Magico} ottenuto nella Prova di Magia puoi agire su di una bestia aggiuntiva.

\incantesimo{Anatema}
\noindent\colorbox{OBSSgold!10}{
\begin{minipage}{0.95\linewidth}
\begin{description}[noitemsep, topsep=0pt, parsep=0pt, partopsep=0pt, leftmargin=0cm, labelwidth=1.3cm]
\item[\textbf{Lista}]: Ammaliamento
\item[\textbf{Livello}]: 1, Comune
\item[\textbf{Lancio}]: 2 Azioni
\item[\textbf{Gittata}]: 9 metri
\item[\textbf{Durata}]: 1 minuto
\end{description}
\end{minipage}}\smallskip

Fino a tre creature di tua scelta che puoi vedere, e che sono a gittata, devono effettuare un Tiro Salvezza su Volontà. Ogni bersaglio che fallisca questo Tiro Salvezza ed effettui un Tiro per Colpire o un Tiro Salvezza prima del termine dell'incantesimo ha un -1 di penalità.

\textbf{Per ogni Successo Critico Magico} ottenuto nella Prova di Magia puoi prendere come bersaglio una creatura aggiuntiva.

\incantesimo{Animale Messaggero}\label{Animal Messenger}
\noindent\colorbox{OBSSgold!10}{
\begin{minipage}{0.95\linewidth}
\begin{description}[noitemsep, topsep=0pt, parsep=0pt, partopsep=0pt, leftmargin=0cm, labelwidth=1.3cm]
\item[\textbf{Lista}]: Animali e Piante
\item[\textbf{Livello}]: 2, Comune
\item[\textbf{Lancio}]: 2 Azioni
\item[\textbf{Gittata}]: 9 metri
\item[\textbf{Durata}]: 24 ore
\end{description}
\end{minipage}}\smallskip

Tramite questo incantesimo, usi un animale per consegnare un messaggio. Scegli una bestia Minuscola a gittata e che puoi vedere, come uno scoiattolo, una ghiandaia o un pipistrello, che abbia un GS inferiore a 0. Specifichi un luogo, che devi aver visitato in passato, e un destinatario che corrisponda a una descrizione generica, come \emph{un uomo o una donna che vesta l'uniforme della guardia cittadina} o \emph{un nano dai capelli rossi che indossa una coppola}. Pronuncia anche un messaggio di massimo venticinque parole. La bestia bersaglio viaggia per la durata dell'incantesimo verso il luogo specificato, coprendo circa 75 chilometri in 24 ore per un messaggero volante, o 40 chilometri per gli altri animali. Quando il messaggero arriva a destinazione, consegna il messaggio alla creatura da te descritta, replicando il suono della tua voce. Il messaggero parla solo a una creatura corrispondente alla descrizione da te fornita. Se il messaggero non riesce a raggiungere la destinazione prima del termine dell'incantesimo, il messaggio è perduto, e la bestia ritorna verso il punto in cui hai lanciato l'incantesimo.

\textbf{Per ogni Successo Critico Magico} ottenuto nella Prova di Magia la durata dell'incantesimo aumenta di 8 ore

\incantesimo{Animare Morti}
\noindent\colorbox{OBSSgold!10}{
\begin{minipage}{0.95\linewidth}
\begin{description}[noitemsep, topsep=0pt, parsep=0pt, partopsep=0pt, leftmargin=0cm, labelwidth=1.3cm]
\item[\textbf{Lista}]: Necromanzia
\item[\textbf{Livello}]: 3, Comune
\item[\textbf{Lancio}]: 1 minuto
\item[\textbf{Gittata}]: 3 metri
\item[\textbf{Durata}]: Istantanea
\end{description}
\end{minipage}}\smallskip

Questo incantesimo crea un servitore non morto. Scegli una pila di ossa o un cadavere di un umanoide Medio o Piccolo a gittata. Il tuo incantesimo imbeve il bersaglio di una nefanda parvenza di vita rianimandolo come creatura non morta. Il bersaglio diventa uno scheletro se scegli le ossa o uno zombi se scegli un cadavere. Durante ciascun tuo round, puoi usare un'Azione per comandare mentalmente qualsiasi creatura da te creata con questo incantesimo che si trovi entro 18 metri da te (se controlli più creature, puoi comandarle tutte o solo alcune di loro allo stesso momento, inviando lo stesso comando a tutte). Decidi quale azione la creatura svolgerà e dove si muoverà durante il suo prossimo round oppure inviale un comando generale come quello di stare di guardia a una particolare stanza o corridoio. Se non invii alcun comando, la creatura si limita a difendersi dalle creature ostili. Una volta ricevuto un ordine, la creatura continuerà a svolgerlo fino al suo compimento. La creatura è sotto il tuo controllo per 24 ore, dopodiché smetterà di eseguire i comandi che le impartirai. Per mantenere il controllo sulla creatura per altre 24 ore, devi lanciare di nuovo questo incantesimo su di essa prima del termine dell'attuale periodo di 24 ore. Questo impiego dell'incantesimo riafferma il tuo controllo su di un massimo di quattro creature che hai animato con questo incantesimo, piuttosto che animarne una nuova.

\textbf{Per ogni Successo Critico Magico} ottenuto nella Prova di Magia animi o riaffermi il controllo su due creature non morte. Ciascuna di queste creature deve provenire da un cadavere o pila di ossa differenti.

\incantesimo{Animare Oggetti}\label{Animate Object}
\noindent\colorbox{OBSSgold!10}{
\begin{minipage}{0.95\linewidth}
\begin{description}[noitemsep, topsep=0pt, parsep=0pt, partopsep=0pt, leftmargin=0cm, labelwidth=1.3cm]
\item[\textbf{Lista}]: Trasmutazione
\item[\textbf{Livello}]: 5, Comune
\item[\textbf{Lancio}]: 1 minuto
\item[\textbf{Gittata}]: 36 metri
\item[\textbf{Durata}]: Concentrazione, massimo 1 minuto
\end{description}
\end{minipage}}\smallskip

Gli oggetti prendono vita al tuo comando. Scegli fino a CM/2 oggetti non magici a gittata e che non siano indossati o trasportati. I bersagli Medi contano come due oggetti, i bersagli Grandi contano come quattro oggetti, i bersagli Enormi contano come otto oggetti. Non puoi animare oggetti di taglia più grossa di Enorme. Ogni bersaglio si anima e diventa una creatura sotto il tuo controllo fino al termine dell'incantesimo o finché non viene ridotto a 0 Punti Ferita.

Ogni bersaglio che si anima fa spuntare le gambe e diventa un Costrutto che utilizza il blocco statistiche di Oggetto Animato, questa creatura è sotto il tuo controllo finché l'incantesimo non termina o finché non viene ridotta a 0 Punti Ferita. Ogni creatura che crei con questo incantesimo è un alleato tuo e dei tuoi alleati. In combattimento, condivide il tuo conteggio di Iniziativa e agisce al tuo comando.

Con un'Azione puoi comandare mentalmente qualsiasi creatura che hai generato con questo incantesimo e che si trovi entro 150 metri da te (se controlli più creature, puoi comandarne solo alcune o tutte allo stesso tempo, impartendo lo stesso comando a ciascuna).
Decidi tu quale azione intraprenderà la creatura e dove si muoverà durante il suo round successivo, o puoi emettere un comando generico, come quello di fare la guardia a una particolare stanza o corridoio.
Se non impartisci comandi, la creatura si limiterà a difendersi dalle creature ostili. Una volta dato un ordine, la creatura continuerà a seguirlo finché non avrà completato il suo compito.

\textbf{Per ogni Successo Critico Magico} ottenuto nella Prova di Magia la durata massima raddoppia.

\medskip

\noindent\rule{\linewidth}{2pt} \index[Mostruario]{Oggetto Animato} \hypertarget{Oggetto Animato}{}\medskip \noindent{\large\textbf{Oggetto Animato}}
\noindent
\begin{description}[noitemsep, topsep=0pt, parsep=0pt, partopsep=0pt, leftmargin=0cm, labelwidth=2.2cm]
	\item[\textbf{Tipo:}] varie, costrutto, disallineato
	\item[\textbf{Caratt.:}] For 3 Des 0 Cos 0 Int -4 Sag -4 Car -5
	\item[\textbf{Punti Ferita:}] 10/20/40,  \textbf{Difesa:} 15,  \textbf{Iniziativa:} +0
	\item[\textbf{Movimento:}] 6 m
	\item[\textbf{Tiri Salvezza:}] Tempra +3, Riflessi +1, Volontà 0
	\item[\textbf{Imm. Danni:}] Veleno, Energia Positiva, Energia Negativa
	\item[\textbf{Immunità:}] accecato, affascinato, assordato, paralizzato, pietrificato, spaventato
	\item[\textbf{Sensi:}] Vista Cieca 18 m (cieco oltre questo raggio)
	\item[\textbf{Sfida:}] nessuno\smallskip
\end{description}

\textbf{Azioni}

\textit{Schianto.} Tiro per Colpire in Mischia Bonus pari al tuo modificatore di attacco con incantesimo, portata 1 m.

\emph{Colpito}: Danni da botta pari a 1d4 + 3 (taglia Media o più piccola), 2d6 + 3 + il tuo modificatore di caratteristica da incantatore (Grande),  2d12 + 3 + il tuo modificatore di caratteristica da incantatore (Enorme).

Quando l'oggetto animato scende a 0 Punti Ferita, ritorna alla sua normale forma di oggetto, e tutti i danni in eccesso vengono inflitti alla sua forma originale.

Se comandi a un oggetto di attaccare, questo può effettuare un singolo attacco da mischia contro una creatura entro 1 metro da esso. Un oggetto potrebbe invece infliggere danni taglienti o perforanti a seconda della sua forma.

\textbf{Per ogni Successo Critico Magico} ottenuto nella Prova di Magia puoi animare due oggetti aggiuntivi.

\incantesimo{Anti-Individuazione}
\noindent\colorbox{OBSSgold!10}{
\begin{minipage}{0.95\linewidth}
\begin{description}[noitemsep, topsep=0pt, parsep=0pt, partopsep=0pt, leftmargin=0cm, labelwidth=1.3cm]
	\item[\textbf{Lista}]: Abiurazione
	\item[\textbf{Livello}]: 3, Non Comune
	\item[\textbf{Lancio}]: 2 Azioni
	\item[\textbf{Gittata}]: Contatto
	\item[\textbf{Durata}]: 8 ore
\end{description}
\end{minipage}}\smallskip

Per la durata nascondi il bersaglio con cui sei stato in contatto dalla magia di divinazione. Il bersaglio può essere una creatura consenziente o un luogo o un oggetto che occupi uno spazio equivalente ad una sfera di 2 metri di raggio. Il bersaglio non può divenire bersaglio di alcuna magia di divinazione o essere percepito tramite sensi di scrutamento magici.

\incantesimo{Antipatia/Simpatia}
\noindent\colorbox{OBSSgold!10}{
\begin{minipage}{0.95\linewidth}
\begin{description}[noitemsep, topsep=0pt, parsep=0pt, partopsep=0pt, leftmargin=0cm, labelwidth=1.3cm]
	\item[\textbf{Lista}]: Ammaliamento
	\item[\textbf{Livello}]: 8, Raro
	\item[\textbf{Lancio}]: 1 ora
	\item[\textbf{Gittata}]: 18 metri
	\item[\textbf{Durata}]: 10 giorni
\end{description}
\end{minipage}}\smallskip

Questo incantesimo attrae o repelle delle creature di tua scelta. Prendi un bersaglio a gittata, che sia un oggetto Enorme o più piccolo o una creatura o un'area non più grande di una sfera di 30 metri di raggio. Poi specifica una specie di creature intelligenti, come i draghi rossi, i goblin o i vampiri. Investi il bersaglio di un'aura che attrae o respinge le creature specificate per la durata. Scegli antipatia o simpatia come effetto dell'aura.

\emph{Antipatia}. L'ammaliamento fa sì che le creature del tipo da te indicato provino un forte impulso a lasciare l'area ed evitare il bersaglio. Quando una creatura del genere può vedere il bersaglio o si avvicina entro 18 metri da esso, la creatura deve superare un Tiro Salvezza su Volontà o diventare spaventata. La creatura rimane spaventata finché può vedere il bersaglio o resta entro 18 metri da esso. Mentre è spaventata dal bersaglio, la creatura deve impiegare il suo movimento per muoversi verso il posto sicuro più vicino dal quale non possa più vedere il bersaglio. Se la creatura si muove più di 18 metri lontano dal bersaglio e non può vederlo, la creatura non è più spaventata, ma torna a essere spaventata se torna a vedere il bersaglio o si muove entro 18 metri da esso.

\emph{Simpatia}. L'ammaliamento fa sì che le creature specificate provino un forte impulso ad avvicinarsi al bersaglio se si trovano entro 18 metri da esso o possono vederlo. Quando una simile creatura può vedere il bersaglio o si avvicina entro 18 metri da esso, la creatura deve superare un Tiro Salvezza su Volontà o usare il suo movimento durante ciascun round per entrare nell'area, o muoversi a portata del bersaglio. Quando la creatura l'avrà fatto, non potrà più volontariamente muoversi lontano dal bersaglio. Se il bersaglio danneggia o altrimenti nuoce alla creatura soggetta, questa può effettuare un Tiri Salvezza su Volontà per terminare l'effetto, come descritto di seguito.

\emph{Terminare l'Effetto}. Se una creatura soggetta termina il suo round mentre si trova più lontana di 18 metri dal bersaglio o non può vederlo, la creatura effettua un Tiro Salvezza su Volontà. Se supera il Tiro Salvezza, la creatura non è più soggetta al bersaglio e riconosce la sensazione di ripugnanza o attrazione come magica. Inoltre, una creatura soggetta all'incantesimo, ha diritto a un altro Tiro Salvezza su Volontà ogni 24 ore di durata dell'incantesimo. Una creatura che supera il Tiro Salvezza contro questo effetto è immune a esso per 1 minuto, dopodiché può subirlo nuovamente.

\incantesimo{Arma Energetica}
\noindent\colorbox{OBSSgold!10}{
\begin{minipage}{0.95\linewidth}
\begin{description}[noitemsep, topsep=0pt, parsep=0pt, partopsep=0pt, leftmargin=0cm, labelwidth=1.3cm]
	\item[\textbf{Lista}]: Aria, Acqua, Terra, Fuoco
	\item[\textbf{Livello}]: 1, Molto Raro
	\item[\textbf{Lancio}]: 1 Azione
	\item[\textbf{Gittata}]: Contatto
	\item[\textbf{Durata}]: 6 round, Concentrazione
\end{description}
\end{minipage}}\smallskip

Lanci l'incantesimo a contatto di un'arma e questa acquisisce dei poteri a seconda della Lista di Magia dal quale hai lanciato l'incantesimo. L'arma si considera magica, come avesse un bonus di +1.
Se Arma Energetica viene lanciato usando la Lista dell'\emph{Aria} l'arma diviene percorsa da elettricità, in caso di \emph{Acqua} l'arma diventa estremamente fredda, in caso di \emph{Terra} dall'arma sgorga acido, in caso di \emph{Fuoco} questa diventa fiammeggiante. Qualsiasi sia la Lista usata l'effetto è tale che l'arma causa 1d6 di danno aggiuntivo del tipo indicato per colpo andato a segno.
Un arma può avere solo un effetto di Arma Energetica attivo contemporaneamente.

\textbf{Per ogni due Successo Critico Magico ottenuto} nella Prova di Magia il danno aumenta di +1d6.

\incantesimo{Arma Magica}
\noindent\colorbox{OBSSgold!10}{
\begin{minipage}{0.95\linewidth}
\begin{description}[noitemsep, topsep=0pt, parsep=0pt, partopsep=0pt, leftmargin=0cm, labelwidth=1.3cm]
	\item[\textbf{Lista}]: Trasmutazione
	\item[\textbf{Livello}]: 2, Comune
	\item[\textbf{Lancio}]: 1 Azione
	\item[\textbf{Gittata}]: Contatto
	\item[\textbf{Durata}]: 10 minuti
\end{description}
\end{minipage}}\smallskip

Lanci l'incantesimo a contatto di un'arma non magica. Fino al termine dell'incantesimo, l'arma diventa un'arma magica con un bonus di +1 ai Tiri per Colpire e di danno.

\textbf{Per ogni Successo Critico Magico} ottenuto nella Prova di Magia il bonus aumenta a +1.

\incantesimo{Arma Spirituale}
\noindent\colorbox{OBSSgold!10}{
\begin{minipage}{0.95\linewidth}
\begin{description}[noitemsep, topsep=0pt, parsep=0pt, partopsep=0pt, leftmargin=0cm, labelwidth=1.3cm]
	\item[\textbf{Lista}]: Invocazione
	\item[\textbf{Livello}]: 2, Comune
	\item[\textbf{Lancio}]: 2 Azioni
	\item[\textbf{Gittata}]: 18 metri
	\item[\textbf{Durata}]: 3 minuti, Concentrazione
\end{description}
\end{minipage}}\smallskip

In un punto nella gittata, crei un'arma spettrale fluttuante, che resta per la durata o finché non lanci di nuovo questo incantesimo. Quando lanci l'incantesimo, puoi effettuare un attacco da incantesimo in mischia contro una creatura entro 1 metro dall'arma con un bonus al colpire pari a Competenza Magica/4. Se colpisci, il bersaglio subisce danni da forza pari a 1d4 + il tuo modificatore di caratteristica per incantesimi da incantatore + Competenza Magica/4. Durante il tuo round, con una Azione, puoi spostare l'arma di 6 metri ed effettuare l'attacco contro una creatura entro 1 metro dall'arma. L'arma può assumere qualsiasi forma tu voglia, magari affine al Patrono. L'arma ha un equivalente bonus magico pari a Competenza Magica/4.

I bonus concessi da Competenza Magica/4 possono essere sostituiti dalla somma dei Tratti in comune con il Patrono/4 se si è un Seguace.

\textbf{Per ogni Successo Critico Magico} ottenuto nella Prova di Magia il danno aumenta di 2.

\incantesimo{Armatura Magica}
\noindent\colorbox{OBSSgold!10}{
\begin{minipage}{0.95\linewidth}
\begin{description}[noitemsep, topsep=0pt, parsep=0pt, partopsep=0pt, leftmargin=0cm, labelwidth=1.3cm]
	\item[\textbf{Lista}]: Abiurazione
	\item[\textbf{Livello}]: 1, Non Comune
	\item[\textbf{Lancio}]: 2 Azioni
	\item[\textbf{Gittata}]: Contatto
	\item[\textbf{Durata}]: 8 ore
\end{description}
\end{minipage}}\smallskip

Lanci l'incantesimo a contatto di una creatura consenziente che non indossa un'armatura. Una forza magica protettiva circonda il bersaglio fino al termine dell'incantesimo. La Difesa naturale del bersaglio aumenta di 3 + Destrezza + 1/6 Competenza Magica. L'incantesimo termina se il bersaglio indossa un'armatura o interrompe l'incantesimo con un'Azione.

\textbf{Per ogni Successo Critico Magico} ottenuto nella Prova di Magia la Difesa aumenta di 1.

\incantesimo{Artificio Druidico}
\noindent\colorbox{OBSSgold!10}{
\begin{minipage}{0.95\linewidth}
\begin{description}[noitemsep, topsep=0pt, parsep=0pt, partopsep=0pt, leftmargin=0cm, labelwidth=1.3cm]
	\item[\textbf{Lista}]: Universale
	\item[\textbf{Livello}]: 0, Non Comune
	\item[\textbf{Lancio}]: 2 Azioni
	\item[\textbf{Gittata}]: 9 metri
	\item[\textbf{Durata}]: Istantanea
\end{description}
\end{minipage}}\smallskip

Sussurrando agli spiriti della natura, crei, a gittata, uno dei seguenti effetti:

\begin{itemize}[leftmargin=*] \setlength{\itemsep}{0pt}
	\item Crei un minuscolo e innocuo effetto sensoriale che predice quale clima ci sarà nel luogo in cui ti trovi per le prossime 24 ore. L'effetto potrebbe manifestarsi come una sfera dorata per i cieli limpidi, una nube per la pioggia, fiocchi di neve per la neve, e così via. L'effetto persiste per 1 round.
	\item Fai immediatamente sbocciare un fiore, un seme o simile pianta.
	\item Crei un istantaneo e innocuo effetto sensoriale, come foglie che cadono, uno sbuffo di vento, il suono di un piccolo animale, o il lieve tanfo di una puzzola. L'effetto deve entrare in una sfera di 1 metro di raggio.
	\item Accendi o spegni istantaneamente una candela, una torcia o un piccolo falò.
\end{itemize}

Questo incantesimo può essere lanciato solo da Seguaci o Devoti di Efrem, Erondil, Gaya, Shayalia.

\incantesimo{Aura Magica dell'Arcanista}
\noindent\colorbox{OBSSgold!10}{
\begin{minipage}{0.95\linewidth}
\begin{description}[noitemsep, topsep=0pt, parsep=0pt, partopsep=0pt, leftmargin=0cm, labelwidth=1.3cm]
	\item[\textbf{Lista}]: Illusione
	\item[\textbf{Livello}]: 2, Non Comune
	\item[\textbf{Lancio}]: 2 Azioni
	\item[\textbf{Gittata}]: Contatto
	\item[\textbf{Durata}]: 24 ore
\end{description}
\end{minipage}}\smallskip

Poni un'illusione su di una creatura od oggetto con cui sei in contatto, così che gli incantesimi di divinazione rivelino false informazioni su di esso. Il bersaglio può essere una creatura consenziente o un oggetto che non sia trasportato o indossato da un'altra creatura. Quando lanci questo incantesimo, scegli uno o entrambi i seguenti effetti. L'effetto permane per la durata. Se esegui questo incantesimo sulla stessa creatura od oggetto ogni giorno per 30 giorni, piazzando ogni volta lo stesso effetto, l'illusione permarrà finché non viene dissolta.

\emph{Aura Falsa}. Cambi il modo in cui il bersaglio risulta a incantesimi ed effetti magici, come individuazione del magico, che individuano le aure magiche. Puoi far apparire magico un oggetto normale, non magico un oggetto magico, o cambiare l'aura magica dell'oggetto così che sembri appartenere a una Liste di Magia di tua scelta. Quando impieghi questo effetto su di un oggetto, puoi far sì che la falsa magia sia apparente a qualsiasi creatura che lo manipoli.

\emph{Mascherare}. Cambi il modo in cui il bersaglio risulta a incantesimi ed effetti magici che individuano il tipo di creatura o Tratti, come l'attivazione dell'incantesimo simbolo. Scegli un tipo di creatura o Tratto, e gli altri incantesimi ed effetti magici considereranno il bersaglio come fosse una creatura di quel tipo o di quel Tratto, e non più di quello originale.

\incantesimo{Aura Sacra}
\noindent\colorbox{OBSSgold!10}{
\begin{minipage}{0.95\linewidth}
\begin{description}[noitemsep, topsep=0pt, parsep=0pt, partopsep=0pt, leftmargin=0cm, labelwidth=1.3cm]
	\item[\textbf{Lista}]: Abiurazione
	\item[\textbf{Livello}]: 8, Comune
	\item[\textbf{Lancio}]: 2 Azioni
	\item[\textbf{Gittata}]: Personale
	\item[\textbf{Durata}]: Concentrazione, 1 minuto
\end{description}
\end{minipage}}\smallskip

Irradi da te luce divina che si raccoglie in una debole luminosità con raggio di 9 metri intorno a te. Quando lanci l'incantesimo, le creature da te scelte in questo raggio emanano luce fioca con un raggio di 1 metro e hanno +8 a tutti i Tiri Salvezza, mentre le altre creature hanno -8 sui Tiri per Colpire contro di loro fino al termine dell'incantesimo. Inoltre, quando un demone o non morto colpisce una creatura bersaglio con un attacco in mischia, l'aura risplende di una luce intensa e deve superare un Tiro Salvezza su Tempra o restare accecato fino al termine dell'incantesimo.

\incantesimo{Bacche Benefiche}
\noindent\colorbox{OBSSgold!10}{
\begin{minipage}{0.95\linewidth}
\begin{description}[noitemsep, topsep=0pt, parsep=0pt, partopsep=0pt, leftmargin=0cm, labelwidth=1.3cm]
	\item[\textbf{Lista}]: Animali e Piante
	\item[\textbf{Livello}]: 2, Comune
	\item[\textbf{Lancio}]: 2 Azioni
	\item[\textbf{Gittata}]: Contatto
	\item[\textbf{Durata}]: Istantanea
\end{description}
\end{minipage}}\smallskip

Incanti fino a 2d4 bacche nella tua mano che vengono infuse di magia per la durata. Una creatura può usare 1 Azione Immediata per mangiare una bacca. Mangiare una bacca ripristina 1 punto ferita e provvede nutrimento, ma non acqua, sufficiente per alimentare una creatura per un giorno. Solo la prima bacca è efficace nel giorno.

Le bacche perdono la loro efficacia se non vengono consumate entro 8 ore dal lancio dell'incantesimo.

\textbf{Per ogni Successo Critico Magico} ottenuto nella Prova di Magia le bacche durano un giorno in più oppure incanti una bacca in più (fino ad un massimo totale di 8).

\incantesimo{Bagliore Solare}\index{Cannone a onde moventi Yamato}
\noindent
\begin{description}[noitemsep, topsep=0pt, parsep=0pt, partopsep=0pt, leftmargin=0cm, labelwidth=1.3cm]
	\item[\textbf{Lista}]: Invocazione
	\item[\textbf{Livello}]: 6, Non Comune
	\item[\textbf{Lancio}]: 2 Azioni
	\item[\textbf{Gittata}]: Personale (linea di 18 metri)
	\item[\textbf{Durata}]: Concentrazione, massimo 1 minuto
\end{description}

Una fascio di luce brillante esplode dalla tua mano in una linea larga 1 metro e lunga 18 metri. Ogni creatura sulla linea deve effettuare un Tiro Salvezza su Tempra. Se fallisce il Tiro Salvezza, la creatura subisce 6d8 danni da Luce e rimane accecata fino al tuo prossimo round. Se supera il Tiro Salvezza, subisce la metà dei danni e non è accecata. I non morti e le melme hanno -1d6 su questo Tiro Salvezza. Puoi creare una nuova linea di luminosità spendendo 3 Azioni durante qualsiasi tuo round fino al termine dell'incantesimo.

Per la durata, una particella di luce brillante risplende nella tua mano. Produce luce in un raggio di 9 metri e penombra per ulteriori 9 metri. Questa luce è considerata luce solare.

\textbf{In caso di due Successo Critico Magico ottenuto} l'incantesimo termina dopo il primo raggio ma la linea è larga 6 metri, lunga 108 metri, il danno da Luce diviene 12d8.

\incantesimo{Banchetto degli Eroi}
\noindent\colorbox{OBSSgold!10}{
\begin{minipage}{0.95\linewidth}
\begin{description}[noitemsep, topsep=0pt, parsep=0pt, partopsep=0pt, leftmargin=0cm, labelwidth=1.3cm]
	\item[\textbf{Lista}]: Evocazione
	\item[\textbf{Livello}]: 6, Non Comune
	\item[\textbf{Lancio}]: 10 minuti
	\item[\textbf{Gittata}]: 9 metri
	\item[\textbf{Durata}]: Istantanea
\end{description}
\end{minipage}}\smallskip

Crei un magnifico banchetto, comprensivo di cibi e bevande prelibate. Il banchetto viene consumato in 1 ora e scompare al termine di questo periodo, ma gli effetti benefici non si faranno sentire fino al termine dell'ora. Fino ad altre dodici creature possono
partecipare al banchetto. Una creatura che partecipi al banchetto ottiene diversi benefici. La creatura viene guarita da tutte le malattie e i veleni non magici. Diventa immune al veleno e all'essere spaventata, ha +1d6 su tutti i Tiri Salvezza su Volontà e Tempra ed acquisisce 2010 Punti Ferita temporanei, questi benefici durano 24 ore.

\textbf{In caso di due Successo Critico Magico ottenuto} nella Prova di Magia la ciotola non è consumata.

\incantesimo{Barriera Antianimali}
\noindent\colorbox{OBSSgold!10}{
\begin{minipage}{0.95\linewidth}
\begin{description}[noitemsep, topsep=0pt, parsep=0pt, partopsep=0pt, leftmargin=0cm, labelwidth=1.3cm]
	\item[\textbf{Lista}]: Animali e Piante
	\item[\textbf{Livello}]: 5, Raro
	\item[\textbf{Lancio}]: 2 Azioni
	\item[\textbf{Gittata}]: Personale
	\item[\textbf{Durata}]: 1 turno per CM
\end{description}
\end{minipage}}\smallskip

Barriera antianimali crea una barriera invisibile in grado di proteggere tutte le creature al suo interno, come se si trovassero dietro un muro, dagli attacchi degli animali, normali e giganti. La barriera, centrata sull'incantatore, ha un diametro di 6 metri.

\textbf{Per ogni Successo Critico Magico} ottenuto nella Prova di Magia raddoppi la durata o allarghi il raggio di 2 metri.

\incantesimo{Barriera di Lame}
\noindent\colorbox{OBSSgold!10}{
\begin{minipage}{0.95\linewidth}
\begin{description}[noitemsep, topsep=0pt, parsep=0pt, partopsep=0pt, leftmargin=0cm, labelwidth=1.3cm]
	\item[\textbf{Lista}]: Invocazione
	\item[\textbf{Livello}]: 6, Comune
	\item[\textbf{Lancio}]: 2 Azioni
	\item[\textbf{Gittata}]: 18 metri
	\item[\textbf{Durata}]: 10 minuti
\end{description}
\end{minipage}}\smallskip

Crei un muro verticale di lame rotanti fatte di energia magica, affilate come rasoi. Il muro compare a gittata e resta per la durata. Puoi creare un muro diritto lungo fino a 30 metri, alto 6 metri e spesso 1 metro, o un muro circolare di 18 metri massimo di diametro e spesso 1 metro. Il muro fornisce copertura completa alle creature dietro di esso e il suo spazio è terreno difficile.

Quando una creatura entra per la prima volta in un round nell'area del muro o comincia il suo round lì deve effettuare un Tiro Salvezza su Riflessi. Se la creatura fallisce il Tiro Salvezza subisce 6d10 danni taglienti, o la metà se lo supera.

Un incantatore che è ad una distanza di un metro dalla Barriera di Lame si considera Distratto.

\incantesimo{Bastoni in Serpenti}
\noindent\colorbox{OBSSgold!10}{
\begin{minipage}{0.95\linewidth}
\begin{description}[noitemsep, topsep=0pt, parsep=0pt, partopsep=0pt, leftmargin=0cm, labelwidth=1.3cm]
	\item[\textbf{Lista}]: Animali e Piante
	\item[\textbf{Livello}]: 3, Non Comune
	\item[\textbf{Lancio}]: 2 Azioni
	\item[\textbf{Gittata}]: 18 metri
		\item[\textbf{Durata}]: Concentrazione fino ad 1 minuto
\end{description}
\end{minipage}}\smallskip

Trasformi 1d4 bastoncini, +1 per ogni volta che hai preso Adepto della Magia, in serpenti velenosi. I serpenti agiscono, nel tuo round, sempre all'unisono e compiono la medesima Azione contro lo stesso avversario.

Questi serpenti, considerati oggetti minuscoli, hanno Difesa 13, 10 Punti Ferita, tutti i Tiri Salvezza a 5. Se scendono sotto 0 Punti Ferita tornano dei bastoncini ma rotti.

Con una Azione puoi comandare i serpenti di attaccare. Esegui un Tiro per Colpire come da attacco con incantesimo in mischia per ogni Serpente contro una creatura entro 1 metro da loro. Ogni serpente che colpisce causa 1 danno da perforazione ed obbliga un Tiro Salvezza su Tempra a DC 14, se il Tiro Salvezza fallisce la creatura subisce 2d4 di danno da veleno o la metà se riesce.

Con una Azione puoi comandare i serpenti di spostarsi fino a 6 metri.

\textbf{Ogni Successo Critico Magico ottenuto} nella Prova di Magia crei un nuovo serpente.

\incantesimo{Beffa Crudele}
\noindent\colorbox{OBSSgold!10}{
\begin{minipage}{0.95\linewidth}
\begin{description}[noitemsep, topsep=0pt, parsep=0pt, partopsep=0pt, leftmargin=0cm, labelwidth=1.3cm]
	\item[\textbf{Lista}]: Ammaliamento
	\item[\textbf{Livello}]: 0, Comune
	\item[\textbf{Lancio}]: 1 Azione
	\item[\textbf{Gittata}]: 18 metri
	\item[\textbf{Durata}]: Istantanea
\end{description}
\end{minipage}}\smallskip

Scateni una serie di insulti avvolti da una subdola malia contro una creatura a gittata e che puoi vedere. Se il bersaglio ti può udire (sebbene non sia necessario che ti comprenda), deve superare un Tiro Salvezza su Volontà o subire 1d4 danni e avere -2 al prossimo Tiro per Colpire che effettuerà prima del termine del suo prossimo round.

Il danno dell'incantesimo aumenta di 1d4 quando raggiungi CM 5, CM 11 e CM 17, ma costa 2 Azioni lanciarlo potenziato e 1 Punti Magia, è altresì necessario avere preso Adepto della Magia un numero di volte pari ai potenziamenti che si vogliono applicare.

\textbf{Ogni 2 Successo Critico Magico ottenuto} nella Prova di Magia influenzi un altra creatura.

\incantesimo{Benedici Acqua}
\noindent\colorbox{OBSSgold!10}{
\begin{minipage}{0.95\linewidth}
\begin{description}[noitemsep, topsep=0pt, parsep=0pt, partopsep=0pt, leftmargin=0cm, labelwidth=1.3cm]
	\item[\textbf{Lista}]: Universale
	\item[\textbf{Livello}]: 2, Comune
	\item[\textbf{Lancio}]: 10 Minuti
	\item[\textbf{Gittata}]: Contatto
	\item[\textbf{Durata}]: Istantanea
\end{description}
\end{minipage}}\smallskip

Benedici fino ad un litro di liquido, sufficiente a creare 5 boccette di Acqua santa.

Devi essere un Seguace o Devoto per poter lanciare questo incantesimo.

\textbf{Per ogni Successo Critico Magico} ottenuto nella Prova di Magia benedici un litro di liquido in più.\index{Acqua Benedetta, creare}

\incantesimo{Benedizione}
\noindent\colorbox{OBSSgold!10}{
\begin{minipage}{0.95\linewidth}
\begin{description}[noitemsep, topsep=0pt, parsep=0pt, partopsep=0pt, leftmargin=0cm, labelwidth=1.3cm]
	\item[\textbf{Lista}]: Universale
	\item[\textbf{Livello}]: 1, Comune
	\item[\textbf{Lancio}]: 2 Azioni
	\item[\textbf{Gittata}]: 9 metri
	\item[\textbf{Durata}]: 1 minuto
\end{description}
\end{minipage}}\smallskip

Benedici fino a tre creature a gittata, scelte da te. I bersagli guadagnano +1 ai Tiri Salvezza e Tiro per Colpire.

più benedizioni, anche da Patroni diversi non si sommano. Devi essere un Seguace o Devoto per poter lanciare questo incantesimo.

\textbf{Per ogni Successo Critico Magico} ottenuto nella Prova di Magia puoi aggiungere una creatura come bersaglio.

\incantesimo{Benedizione della Vita}
\noindent\colorbox{OBSSgold!10}{
\begin{minipage}{0.95\linewidth}
\begin{description}[noitemsep, topsep=0pt, parsep=0pt, partopsep=0pt, leftmargin=0cm, labelwidth=1.3cm]
	\item[\textbf{Lista}]: Cura
	\item[\textbf{Livello}]: 3, Raro
	\item[\textbf{Lancio}]: 2 Azioni
	\item[\textbf{Gittata}]: 9 metri
	\item[\textbf{Durata}]: 1 minuto, Concentrazione
\end{description}
\end{minipage}}\smallskip

Questo incantesimo conferisce speranza e vitalità. Scegli fino a 6 creature a gittata. Per la durata, ciascun bersaglio ha +2 ai Tiri Salvezza su Volontà e recupera 1 Punto Ferita a round.

\textbf{Se ottieni 2 Successo Critico Magico e sei un Devoto o Seguace di un Patrono buono} ogni round le creature scelte recuperano 2 Punti Ferita in più.

\incantesimo{Benedizione di Cattalm}
\noindent\colorbox{OBSSgold!10}{
\begin{minipage}{0.95\linewidth}
\begin{description}[noitemsep, topsep=0pt, parsep=0pt, partopsep=0pt, leftmargin=0cm, labelwidth=1.3cm]
	\item[\textbf{Lista}]: Ammaliamento, Fuoco
	\item[\textbf{Livello}]: 3, Molto Raro
	\item[\textbf{Lancio}]: 2 Azioni
	\item[\textbf{Gittata}]: 18 metri
	\item[\textbf{Durata}]: Istantanea
\end{description}
\end{minipage}}\smallskip

Invochi l'ira di Cattalm sul tuo avversario. La creatura bersaglio subisce 4d6 di danno da fuoco, deve effettuare un Tiro Salvezza su Volontà o subire alla prima successiva prova di competenza oppure Tiro per Colpire o Tiro Salvezza una penalità di -1d6 e l'incantatore aumenta di uno la sua riserva di Punti Fato.

\textbf{Per ogni due Successo Critico Magico ottenuto} nella Prova di Magia puoi influenzare un'altra creatura.

\incantesimo{Benedizioni di Efrem}\label{Animal Shapes}
\noindent\colorbox{OBSSgold!10}{
\begin{minipage}{0.95\linewidth}
\begin{description}[noitemsep, topsep=0pt, parsep=0pt, partopsep=0pt, leftmargin=0cm, labelwidth=1.3cm]
	\item[\textbf{Lista magia}] : Animali e Piante
	\item[\textbf{Livello}] : 8, Raro
	\item[\textbf{T. di Lancio}] : 2 Azioni
	\item[\textbf{Gittata}] : 9 metri
	\item[\textbf{Durata}] : 1 ora per CM
\end{description}
\end{minipage}}\smallskip

Scegli fino a CM creature consenzienti che puoi vedere entro gittata. Ogni bersaglio si trasforma in una Bestia, di taglia piccola, media o grande, di tua scelta che abbia un Grado di Sfida pari o inferiore a 4. Puoi scegliere una forma diversa per ogni bersaglio. Nei round successivi puoi usare 2 Azioni per trasformare di nuovo i bersagli.

La trasformazione segue le regole standard della trasformazione animale.

Le statistiche di gioco di un bersaglio vengono sostituite dalle statistiche della Bestia scelta, ma il bersaglio conserva il suo tipo di creatura; Punti Ferita; Tratti, capacità di comunicare e punteggi di Intelligenza, Saggezza e Carisma. Le azioni del bersaglio sono limitate dall'anatomia della forma Bestia e non può lanciare incantesimi. L'equipaggiamento del bersaglio si fonde nella nuova forma e non può essere usato mentre è in quella forma.

L'incantesimo termina sulla creature se questa perde conoscenza.

\textbf{NOTA}: è necessario essere un Devoto o Seguace di Efrem o Shayalia per poter lanciare questo incantesimo.

\incantesimo{Benedizione di Ledyal}\label{Aura of Vitality}
\noindent\colorbox{OBSSgold!10}{
\begin{minipage}{0.95\linewidth}
\begin{description}[noitemsep, topsep=0pt, parsep=0pt, partopsep=0pt, leftmargin=0cm, labelwidth=1.3cm]
	\item[\textbf{Lista di Magia}] : Necromanzia
	\item[\textbf{Livello}] : 3, Molto Raro
	\item[\textbf{T. di Lancio}] : 2 Azioni
	\item[\textbf{Gittata}] : Personale
	\item[\textbf{Durata}] : Concentrazione, fino a 6 round
\end{description}
\end{minipage}}\smallskip

Un aura sacra si irradia da te. Qualsiasi creatura che incominci il round entro 9 metri da te viene curato di 1d6 Punti Ferita. Una creatura non viene curata per più di 3 round per incantesimo.

\textbf{NOTA}: è necessario essere un Devoto o Seguace di Ledyal o Sumkjr per poter lanciare questo incantesimo.

\incantesimo{Benedizione Superiore}
\noindent\colorbox{OBSSgold!10}{
\begin{minipage}{0.95\linewidth}
\begin{description}[noitemsep, topsep=0pt, parsep=0pt, partopsep=0pt, leftmargin=0cm, labelwidth=1.3cm]
	\item[\textbf{Lista}]: Invocazione
	\item[\textbf{Livello}]: 2, Non Comune
	\item[\textbf{Lancio}]: 1 Minuto
	\item[\textbf{Gittata}]: 18 metri
	\item[\textbf{Durata}]: 1 ora
\end{description}
\end{minipage}}\smallskip

Benedici una creatura a tua scelta. La creatura entro la durata può aggiungere 1d6 ad un tiro prima di sapere se la prova (TC/TS/Prova) ha avuto successo o meno. Questo bonus può essere usato 2 volte nell'ora. Devi essere un Seguace o Devoto per poter lanciare questo incantesimo.

\textbf{Per ogni Successo Critico Magico} ottenuto nella Prova di Magia puoi aggiungere una creatura come bersaglio o aggiungere un ora alla durata.

\incantesimo{Benedizione Suprema}
\noindent\colorbox{OBSSgold!10}{
\begin{minipage}{0.95\linewidth}
\begin{description}[noitemsep, topsep=0pt, parsep=0pt, partopsep=0pt, leftmargin=0cm, labelwidth=1.3cm]
	\item[\textbf{Lista}]: Invocazione
	\item[\textbf{Livello}]: 3, Raro
	\item[\textbf{Lancio}]: 1 Reazione
	\item[\textbf{Gittata}]: 27 metri
	\item[\textbf{Durata}]: Istantanea
\end{description}
\end{minipage}}\smallskip

Benedici una creatura a tua scelta. La creatura può ritirare due dadi di una singola prova prima di sapere se la prova ha avuto successo o meno. La creatura sceglie se prendere i nuovi tiri ottenuti o tenere i vecchi. Devi essere un Seguace o Devoto per poter lanciare questo incantesimo.

\textbf{Per ogni Successo Critico Magico} ottenuto nella Prova di Magia la creatura prende anche un +1 di bonus alla prova.

\incantesimo{Blocca Mostri}
\noindent\colorbox{OBSSgold!10}{
\begin{minipage}{0.95\linewidth}
\begin{description}[noitemsep, topsep=0pt, parsep=0pt, partopsep=0pt, leftmargin=0cm, labelwidth=1.3cm]
	\item[\textbf{Lista}]: Ammaliamento
	\item[\textbf{Livello}]: 5, Comune
	\item[\textbf{Lancio}]: 2 Azioni
	\item[\textbf{Gittata}]: 27 metri
	\item[\textbf{Durata}]: 1 minuto
\end{description}
\end{minipage}}\smallskip

Scegli una creatura a gittata e che puoi vedere. Il bersaglio deve superare un Tiro Salvezza su Volontà, o restare paralizzato per la durata. Questo incantesimo non ha effetto su non morti o costrutti. Al termine di ciascun suo round, il bersaglio può effettuare un altro Tiro Salvezza su Volontà. Se lo supera, per quel bersaglio l'incantesimo ha termine.

\textbf{Per ogni Successo Critico Magico} ottenuto nella Prova di Magia puoi aggiungere una creatura come bersaglio purché siano entro 9 metri l'una dall'altra.

\incantesimo{Blocca Persona}
\noindent\colorbox{OBSSgold!10}{
\begin{minipage}{0.95\linewidth}
\begin{description}[noitemsep, topsep=0pt, parsep=0pt, partopsep=0pt, leftmargin=0cm, labelwidth=1.3cm]
	\item[\textbf{Lista}]: Ammaliamento
	\item[\textbf{Livello}]: 2, Comune
	\item[\textbf{Lancio}]: 2 Azioni
	\item[\textbf{Gittata}]: 18 metri
	\item[\textbf{Durata}]: 1 minuto
\end{description}
\end{minipage}}\smallskip

Scegli un umanoide a gittata e che puoi vedere. L'incantesimo non ha effetto su creature con GS 4 o più. Il bersaglio deve superare un Tiro Salvezza su Volontà o restare paralizzato per la durata.

\textbf{Per ogni Successo Critico Magico} ottenuto nella Prova di Magia puoi aggiungere una creatura come bersaglio purché siano entro 9 metri l'una dall'altra.

\incantesimo{Blocca Persona Avanzato}
\noindent\colorbox{OBSSgold!10}{
\begin{minipage}{0.95\linewidth}
\begin{description}[noitemsep, topsep=0pt, parsep=0pt, partopsep=0pt, leftmargin=0cm, labelwidth=1.3cm]
	\item[\textbf{Lista}]: Ammaliamento
	\item[\textbf{Livello}]: 4, Non Comune
	\item[\textbf{Lancio}]: 2 Azioni
	\item[\textbf{Gittata}]: 18 metri, raggio 6 metri
	\item[\textbf{Durata}]: 1 minuto
\end{description}
\end{minipage}}\smallskip

Blocchi fino a 2d4 creature entro 18 metri da te in un raggio di 6 metri. L'incantesimo non ha effetto su creature con GS 6 o più. I bersagli devono superare un Tiro Salvezza su Volontà o restare paralizzati per la durata, il Tiro Salvezza può essere ripetuto quando subiscono degli attacchi.

\textbf{Per ogni Successo Critico Magico} ottenuto nella Prova di Magia puoi aggiungere 2 punti ai 2d4 tirati.

\incantesimo{Bocca Magica}
\noindent\colorbox{OBSSgold!10}{
\begin{minipage}{0.95\linewidth}
\begin{description}[noitemsep, topsep=0pt, parsep=0pt, partopsep=0pt, leftmargin=0cm, labelwidth=1.3cm]
	\item[\textbf{Lista}]: Illusione
	\item[\textbf{Livello}]: 2, Comune
	\item[\textbf{Lancio}]: 1 minuto
	\item[\textbf{Gittata}]: 9 metri
	\item[\textbf{Durata}]: Fino a che dissolto
\end{description}
\end{minipage}}\smallskip

Impianti un messaggio in un oggetto a gittata, messaggio che viene pronunciato quando si soddisfa la condizione di attivazione. Scegli un oggetto che puoi vedere e che non sia indossato o trasportato da un'altra creatura. Poi pronuncia il messaggio che deve essere di 25 parole o meno ma può essere distribuito in un periodo di massimo 10 minuti. Infine determina la circostanza che attiverà l'incantesimo affinché questo trasmetta il tuo messaggio.

Quando la circostanza si manifesta, una bocca magica appare sull'oggetto e recita il messaggio con la tua voce e allo stesso volume con cui l'hai pronunciato. Se l'oggetto da te scelto ha una bocca o qualcosa che assomiglia a una bocca (per esempio, la bocca di una statua), la bocca magica appare così che le parole sembrino provenire dalla bocca dell'oggetto. Quando lanci questo incantesimo, puoi far sì che l'incantesimo termini dopo aver trasmesso il suo messaggio o che perduri e ripeta il messaggio ogni volta che la condizione si attiva.

La circostanza di attivazione può essere generica o dettagliata quanto desideri, ma deve essere basata su condizioni visibili o udibili che avvengono entro 9 metri dall'oggetto. Per esempio potresti istruire la bocca di parlare quando una qualsiasi creatura si avvicina entro 9 metri dall'oggetto o quando una campanella d'argento suona entro 9 metri da esso.

\incantesimo{Bolla vitale}
\noindent\colorbox{OBSSgold!10}{
\begin{minipage}{0.95\linewidth}
\begin{description}[noitemsep, topsep=0pt, parsep=0pt, partopsep=0pt, leftmargin=0cm, labelwidth=1.3cm]
	\item[\textbf{Lista}]: Aria, Abiurazione
	\item[\textbf{Livello}]: 4, Non Comune
	\item[\textbf{Lancio}]: 1 minuto
	\item[\textbf{Gittata}]: 9 metri
	\item[\textbf{Durata}]: 1 ora per CM
\end{description}
\end{minipage}}\smallskip

Puoi creare fino a 6 bolle che circondano le creature da te designate.
La durata totale è di 1 ora per punto in Competenza Magica suddivisa a piacimento tra le creature nelle bolle.
Questa bolla permette ai soggetti di respirare liberamente, anche sott'acqua o nel vuoto, e li rende immuni ai gas e ai vapori nocivi, incluse le malattie e i veleni da inalazione e gli incantesimi come Nebbia Nauseante e Nebbia mortale. La bolla protegge i soggetti dalle temperature estreme (ma non che causino danno ogni round) e dalle pressioni estreme.

Bolla vitale non fornisce protezione dall'energia negativa o positiva (ad esempio sui piani dell'Energia Negativa e Positiva), la capacità di vedere in condizioni di scarsa visibilità (come nel fumo o nella nebbia), né la capacità di muoversi o agire normalmente in condizioni che impediscono il movimento (come sott'acqua).


\incantesimo{Caduta Piuma}\label{cadutapiuma}\hypertarget{cadutapiuma}
\noindent\colorbox{OBSSgold!10}{
\begin{minipage}{0.95\linewidth}
\begin{description}[noitemsep, topsep=0pt, parsep=0pt, partopsep=0pt, leftmargin=0cm, labelwidth=1.3cm]
	\item[\textbf{Lista}]: Aria
	\item[\textbf{Livello}]: 1, Comune
	\item[\textbf{Lancio}]: 1 Reazione, che effettui quando tu o una creatura entro 18 metri da te cade
	\item[\textbf{Gittata}]: 18 metri
	\item[\textbf{Durata}]: 1 minuto
\end{description}
\end{minipage}}\smallskip


Scegli fino a 2 creature a gittata. La velocità di discesa di una creatura che cade diminuisce a 18 metri per round fino al termine dell'incantesimo. Se la creatura atterra prima del termine dell'incantesimo, non subisce danni da caduta e può atterrare sui suoi piedi; per quella creatura l'incantesimo ha termine.

\textbf{Per ogni Successo Critico Magico} ottenuto nella Prova di Magia puoi spostarti lateralmente di 1 metro od influenzare un altra creatura.

\incantesimo{Calmare Emozioni}
\noindent\colorbox{OBSSgold!10}{
\begin{minipage}{0.95\linewidth}
\begin{description}[noitemsep, topsep=0pt, parsep=0pt, partopsep=0pt, leftmargin=0cm, labelwidth=1.3cm]
	\item[\textbf{Lista}]: Ammaliamento
	\item[\textbf{Livello}]: 2, Comune
	\item[\textbf{Lancio}]: 2 Azioni
	\item[\textbf{Gittata}]: 18 metri
	\item[\textbf{Durata}]: Concentrazione, massimo 1 minuto
\end{description}
\end{minipage}}\smallskip

Tenti di sopprimere le forti emozioni in un gruppo di persone. Ogni umanoide in una sfera di 6 metri di raggio centrata su di un punto a gittata da te scelto, deve effettuare un Tiro Salvezza su Volontà; se lo desidera, una creatura può scegliere di fallire questo Tiro Salvezza. Se una creatura fallisce il Tiro Salvezza, scegli uno di questi due effetti.

\emph{Placare}. Puoi sopprimere qualsiasi effetto che renda il bersaglio Affascinato o spaventato. Quando questo incantesimo termina gli effetti soppressi riprendono purché la loro durata non sia nel frattempo esaurita.

\emph{Indifferenza}. Puoi rendere un bersaglio indifferente nei confronti di una creatura di tua scelta, verso la quale è ostile. Questa indifferenza termina se il bersaglio viene attaccato o danneggiato da un incantesimo o se vede uno dei suoi amici venir danneggiato. Quando l'incantesimo termina la creatura diventa di nuovo ostile a meno che il Narratore non determini diversamente.

\incantesimo{Camminare nell'aria}
\noindent\colorbox{OBSSgold!10}{
\begin{minipage}{0.95\linewidth}
\begin{description}[noitemsep, topsep=0pt, parsep=0pt, partopsep=0pt, leftmargin=0cm, labelwidth=1.3cm]
	\item[\textbf{Lista}]: Aria
	\item[\textbf{Livello}]: 4, Non Comune
	\item[\textbf{Lancio}]: 2 Azioni
	\item[\textbf{Gittata}]: 18 metri
	\item[\textbf{Durata}]: 1 Turno
\end{description}
\end{minipage}}\smallskip

Per la durata una creatura da te scelta a gittata, che puoi vedere, e consenziente può camminare nell'aria come se comminasse su solido terreno. Se una creatura è per aria quando l'effetto ha termine, la creatura scende 18 metri per round per un minuto dopo di che cade per la distanza rimanente.

\textbf{Per ogni Successo Critico Magico} ottenuto nella Prova di Magia puoi puoi aggiungere una creatura come bersaglio. Quando lanci l'incantesimo, le creature bersaglio devono trovarsi entro 9 metri l'una dall'altra.

\incantesimo{Camminare nel Vento}
\noindent\colorbox{OBSSgold!10}{
\begin{minipage}{0.95\linewidth}
\begin{description}[noitemsep, topsep=0pt, parsep=0pt, partopsep=0pt, leftmargin=0cm, labelwidth=1.3cm]
	\item[\textbf{Lista}]: Aria
	\item[\textbf{Livello}]: 6, Non Comune
	\item[\textbf{Lancio}]: 1 minuto
	\item[\textbf{Gittata}]: 9 metri
	\item[\textbf{Durata}]: 8 ore
\end{description}
\end{minipage}}\smallskip

Per la durata, tu e fino ad altre dieci creature consenzienti a gittata, che puoi vedere, assumete forma gassosa, diventando nubi. Mentre è in forma di nube una creatura ha velocità di volo 90 metri e ha resistenza ai danni dalle armi non magiche. Ritornare alla forma normale o ritornare a nube richiede 1 minuto, durante il quale la creatura è inabile e non può muoversi. Se una creatura è in forma di nube e sta volando quando l'effetto ha termine, la creatura scende 18 metri per round al minuto finché non atterra, al sicuro. Se non riesce ad atterrare dopo 1 minuto, la creatura cadrà per la distanza rimanente.

\incantesimo{Camminare sull'Acqua}
\noindent\colorbox{OBSSgold!10}{
\begin{minipage}{0.95\linewidth}
\begin{description}[noitemsep, topsep=0pt, parsep=0pt, partopsep=0pt, leftmargin=0cm, labelwidth=1.3cm]
	\item[\textbf{Lista}]: Acqua
	\item[\textbf{Livello}]: 3, Comune
	\item[\textbf{Lancio}]: 2 Azioni
	\item[\textbf{Gittata}]: 9 metri
	\item[\textbf{Durata}]: 1 ora
\end{description}
\end{minipage}}\smallskip

Questo incantesimo conferisce la capacità di muoversi attraverso superfici liquide (come acqua, acido, fango, neve, sabbie mobili o lava) come se fossero innocuo terreno solido (le creature che attraversano la lava fusa possono comunque subire danni dal calore o sciogliersi nell'acido). Fino a dieci creature consenzienti a gittata, e che puoi vedere, ricevono questa capacità per tutta la durata. Se il tuo bersaglio è immerso in un liquido, l'incantesimo riporta il bersaglio in superficie del liquido a una velocità di 9 metri per round.

\incantesimo{Charme su Persone}
\noindent\colorbox{OBSSgold!10}{
\begin{minipage}{0.95\linewidth}
\begin{description}[noitemsep, topsep=0pt, parsep=0pt, partopsep=0pt, leftmargin=0cm, labelwidth=1.3cm]
	\item[\textbf{Lista}]: Ammaliamento
	\item[\textbf{Livello}]: 1, Comune
	\item[\textbf{Lancio}]: 2 Azioni
	\item[\textbf{Gittata}]: 9 metri
	\item[\textbf{Durata}]: 1 ora
\end{description}
\end{minipage}}\smallskip

Cerchi di affascinare un umanoide a gittata e che puoi vedere. Egli deve effettuare un Tiro Salvezza su Volontà e avrà +1d6 se sta combattendo contro di te o i tuoi alleati. Se fallisce il Tiro Salvezza è Affascinato da te fino al termine dell'incantesimo o finché tu o i tuoi alleati non gli facciate qualcosa di nocivo. La creatura affascinata ti considera un amichevole conoscente. Quando l'incantesimo termina la creatura è consapevole di essere stata affascinata da te. Ogni qual volta la creatura è minacciata da te o da un tuo amico può rifare il Tiro Salvezza con un bonus di +2.

\textbf{Per ogni Successo Critico Magico} ottenuto nella Prova di Magia puoi puoi aggiungere una creatura come bersaglio. Quando lanci l'incantesimo, le creature bersaglio devono trovarsi entro 9 metri l'una dall'altra.

\incantesimo{Campo Anti-Magia}
\noindent\colorbox{OBSSgold!10}{
\begin{minipage}{0.95\linewidth}
\begin{description}[noitemsep, topsep=0pt, parsep=0pt, partopsep=0pt, leftmargin=0cm, labelwidth=1.3cm]
	\item[\textbf{Lista}]: Abiurazione
	\item[\textbf{Livello}]: 8, Raro
	\item[\textbf{Lancio}]: 2 Azioni
	\item[\textbf{Gittata}]: Personale (sfera di 3 metri di raggio)
	\item[\textbf{Durata}]: Concentrazione, massimo 1 ora
\end{description}
\end{minipage}}\smallskip

Vieni circondato da una sfera invisibile di anti-magia di 3 metri di raggio. Quest'area è separata dall'energia magica che permea la Terra. All'interno della sfera non si possono lanciare incantesimi, le creature richiamate scompaiono e anche gli oggetti magici diventano normali. Fino al termine dell'incantesimo la sfera si muove con te centrata su di te. Gli incantesimi e altri effetti magici, eccetto quelli creati da un artefatto o Patrono, sono soppressi all'interno della sfera né vi possono penetrare. Uno slot speso per lanciare un incantesimo soppresso è consumato. Mentre un effetto è soppresso non funziona ma il tempo che trascorre soppresso è conteggiato per la sua durata.

\medskip

\noindent\emph{Effetti con Bersaglio}. Incantesimi e altri effetti magici, come Dardo arcano e \hyperlink{Charme su Persone}{Charme su Persone}, che prendono come bersaglio una creatura o un oggetto all'interno della sfera non hanno effetto su quel bersaglio.

\emph{Aree di Magia}. L'area di un altro incantesimo o effetto magico, come palla di fuoco, non può estendersi all'interno della sfera. Se la sfera si sovrappone a un'area di magia, la parte di quell'area coperta dalla sfera viene soppressa. Per esempio, le fiamme generate da un muro di fuoco vengono soppresse all'interno della sfera, creando un buco nel muro se la sovrapposizione è sufficientemente grande. Incantesimi. Qualsiasi incantesimo o altro effetto magico attivo su di una creatura od oggetto all'interno della sfera viene soppresso finché la creatura o l'oggetto si trovano all'interno della sfera.

\emph{Oggetti Magici}. Le proprietà e poteri degli oggetti magici vengono soppressi dalla sfera. Per esempio, una spada lunga +1 all'interno della sfera funziona come una spada lunga non magica. Le proprietà e i poteri delle armi magiche vengono soppressi se sono usati contro un bersaglio all'interno della sfera o impugnate da un attaccante dentro la sfera. Se un'arma magica o munizione magica lascia interamente la sfera (per esempio, se tiri una freccia magica o scagli una lancia magica a un bersaglio al di fuori della sfera), la magia dell'oggetto non è più soppressa non appena esce dalla sfera.

\emph{Magia di Viaggio}. Il teletrasporto e il viaggio planare non funzionano all'interno della sfera, che la sfera sia il punto di destinazione o di partenza di questo viaggio magico. All'interno della sfera, un portale verso un altro luogo, mondo, o piano di esistenza, così come uno spazio extradimensionale come quello creato dall'incantesimo trucco della corda, resta chiuso.

\emph{Creature e Oggetti}. All'interno della sfera, una creatura o oggetto evocati o creati dalla magia svaniscono temporaneamente dall'esistenza. La creatura od oggetto riappare istantaneamente una volta che lo spazio occupato da essa non si trova più all'interno della sfera.

\emph{Dissolvi magie}. Gli incantesimi e gli effetti magici come dissolvi magie non hanno effetto sulla sfera. Allo stesso modo le sfere create da altri incantesimi campo antimagia non si annullano vicendevolmente.

\incantesimo{Camuffare Sé Stesso}
\noindent\colorbox{OBSSgold!10}{
\begin{minipage}{0.95\linewidth}
\begin{description}[noitemsep, topsep=0pt, parsep=0pt, partopsep=0pt, leftmargin=0cm, labelwidth=1.3cm]
	\item[\textbf{Lista}]: Illusione
	\item[\textbf{Livello}]: 1, Comune
	\item[\textbf{Lancio}]: 2 Azioni
	\item[\textbf{Gittata}]: Personale
	\item[\textbf{Durata}]: 1 ora
\end{description}
\end{minipage}}\smallskip

Cambi il tuo aspetto, assieme a quello dei tuoi abiti, armatura, armi e altri oggetti che indossi, fino al termine dell'incantesimo o finché non impieghi un'Azione per interrompere l'incantesimo. Puoi apparire 30 centimetri più basso o più alto, magro, grasso o una via di mezzo. Non puoi modificare la tua conformazione fisica quindi devi adottare una forma che abbia la medesima distribuzione di arti. Per tutto il resto, l'illusione è limitata solo dalla tua fantasia.

I cambi apportati da questo incantesimo non sono in grado di sostenere un'ispezione fisica. Per esempio, se usi questo incantesimo per aggiungere un cappello al tuo abbigliamento, gli oggetti attraversano il cappello, e chiunque lo tocchi non avvertirebbe nulla e finirebbe per toccarti la testa e i capelli. Se usi questo incantesimo per apparire più magro di quello che sei, la mano di una persona che provasse a toccarti rimbalzerebbe su di te, mentre alla vista sembrerebbe fermarsi a mezz'aria. Per distinguere il tuo camuffamento, una creatura può usare 2 Azioni per ispezionare il tuo aspetto e deve superare una prova di Consapevolezza+4 contro la DC del Tiro Salvezza dell'incantesimo.

\incantesimo{Capanna}
\noindent\colorbox{OBSSgold!10}{
\begin{minipage}{0.95\linewidth}
\begin{description}[noitemsep, topsep=0pt, parsep=0pt, partopsep=0pt, leftmargin=0cm, labelwidth=1.3cm]
	\item[\textbf{Lista}]: Invocazione
	\item[\textbf{Livello}]: 3, Non Comune
	\item[\textbf{Lancio}]: 1 minuto
	\item[\textbf{Gittata}]: Personale (semisfera di 3 metri di raggio)
	\item[\textbf{Durata}]: 8 ore
\end{description}
\end{minipage}}\smallskip

Una mezza sfera di forza immobile del raggio di 3 metri si forma intorno e sopra di te, restando stazionaria per la durata. L'incantesimo termina se lasci l'area. Otto creature di taglia Media o inferiore possono entrare nella cupola insieme a te. L'incantesimo fallisce se l'area include una creatura più grande o più di nove creature. Le creature e gli oggetti all'interno della cupola, quando lanci questo incantesimo, la possono attraversare liberamente. Tutte le altre creature e oggetti devono effettuare un Tiro Salvezza su Tempra o sono impossibilitati dall'attraversarla per quel round. Incantesimi e altri effetti magici possono estendersi oltre la cupola o attraversarla se non sono Trucchetti. L'atmosfera all'interno dello spazio è confortevole e asciutta quale che sia il clima all'esterno.

Fino al termine dell'incantesimo puoi comandare che l'illuminazione interna sia piena, fioca o buio. La cupola è opaca dall'esterno, fornisce copertura media, di qualsiasi colore tu scelga, ma è trasparente dall'interno.

\textbf{Per ogni Successo Critico Magico} ottenuto nella Prova di Magia l'incantesimo dura 2 ore in più.

\incantesimo{Caratteristica Potenziata}
\noindent\colorbox{OBSSgold!10}{
\begin{minipage}{0.95\linewidth}
\begin{description}[noitemsep, topsep=0pt, parsep=0pt, partopsep=0pt, leftmargin=0cm, labelwidth=1.3cm]
	\item[\textbf{Lista}]: Trasmutazione
	\item[\textbf{Livello}]: 2, Comune
	\item[\textbf{Lancio}]: 2 Azioni
	\item[\textbf{Gittata}]: Contatto
	\item[\textbf{Durata}]: massimo 10 minuti
\end{description}
\end{minipage}}\smallskip

Conferisci un potenziamento magico a una creatura con cui sei in contatto. Scegli uno degli effetti seguenti; il bersaglio ottiene quell'effetto fino al termine dell'incantesimo.

\begin{itemize}[leftmargin=*] \setlength{\itemsep}{0pt}
	\item \emph{Astuzia della Volpe}. Il bersaglio ha +1d6 alle Competenze base basate su di Intelligenza e Forza
	\item \emph{Forza del Toro}. Il bersaglio ha +1d6 alle Competenze base basate su Forza e la sua capacità di Ingombro raddoppia.
	\item \emph{Grazia del Energia Luminosa}. Il bersaglio ha +1d6 alle Competenze base basate su Destrezza. Inoltre, qualora non sia inabile, non subisce danni dalle cadute di 6 metri o meno.
	\item \emph{Resistenza della Nutria}. Il bersaglio ha +1d6 alle Competenze base basate su  Costituzione. Ottiene anche 2d6 Punti Ferita temporanei, che vengono persi alla fine dell'incantesimo.
	\item \emph{Saggezza del Dobi}. Il bersaglio ha +1d6 alle Competenze base basate su  Saggezza e +2 alla Consapevolezza.
	\item \emph{Splendore della Topi}. Il bersaglio ha +1d6 alle Competenze base basate su  Carisma.
\end{itemize}

\textbf{Per ogni Successo Critico Magico} ottenuto nella Prova di Magia puoi prendere come bersaglio un'ulteriore creatura

\incantesimo{Carne in Pietra}
\incantesimo{Pietra in Carne}
\noindent\colorbox{OBSSgold!10}{
	\begin{minipage}{0.95\linewidth}
\begin{description}[noitemsep, topsep=0pt, parsep=0pt, partopsep=0pt, leftmargin=0cm, labelwidth=1.3cm]
	\item[\textbf{Lista}]: Terra
	\item[\textbf{Livello}]: 6, Non Comune - Raro
	\item[\textbf{Lancio}]: 2 Azioni
	\item[\textbf{Gittata}]: 18 metri
	\item[\textbf{Durata}]: Permanente
\end{description}
\end{minipage}}\smallskip

Cerchi di trasformare in pietra una creatura a gittata che puoi vedere. Se il corpo del bersaglio è fatto di carne la creatura diventa Rallentata 1/6r e deve effettuare un Tiro Salvezza su Tempra. Se fallisce il Tiro Salvezza diventa invece Rallentata 1/10 minuti e la sua carne comincia a indurirsi.
La creatura che fallisce il Tiro Salvezza iniziale il round dopo deve effettuare un nuovo Tiro Salvezza su Tempra. Se supera il Tiro Salvezza con successo non ci sono ulteriori effetti. Se fallisce questo nuovo Tiro Salvezza viene trasformata in pietra e resta vittima della condizione pietrificato per la durata.

Se supera il primo Tiro Salvezza la creatura non subisce ulteriori effetti.

Se la creatura viene danneggiata fisicamente mentre è pietrificata, soffre di deformità simili ai danni arrecati alla pietra, se ritorna al suo stato originale.

L'incantesimo \emph{Pietra in Carne} fa tornare una creatura di carne purché non sia stata trasformata da più di un anno. L'incantesimo Dissolvi Magia non è in grado di annullarne gli effetti.

\incantesimo{Cecità/Sordità}
\noindent\colorbox{OBSSgold!10}{
\begin{minipage}{0.95\linewidth}
\begin{description}[noitemsep, topsep=0pt, parsep=0pt, partopsep=0pt, leftmargin=0cm, labelwidth=1.3cm]
	\item[\textbf{Lista}]: Necromanzia
	\item[\textbf{Livello}]: 2, Comune
	\item[\textbf{Lancio}]: 2 Azioni
	\item[\textbf{Gittata}]: 9 metri
	\item[\textbf{Durata}]: 1 minuto
\end{description}
\end{minipage}}\smallskip

Puoi accecare o assordare un nemico. Scegli una creatura a gittata e che puoi vedere. Il bersaglio deve effettuare un Tiro Salvezza su Tempra. Se lo fallisce il bersaglio è accecato o assordato (a tua scelta) per la durata.

\textbf{Per ogni due Successo Critico Magico ottenuto} nella Prova di Magia puoi aggiungere un altro bersaglio in gittata. Se fai 3 Successo Critico Magico il bersaglio è influenzato dall'incantesimo per tutto il giorno.

\incantesimo{Cecità/Sordità Avanzata}
\noindent\colorbox{OBSSgold!10}{
\begin{minipage}{0.95\linewidth}
\begin{description}[noitemsep, topsep=0pt, parsep=0pt, partopsep=0pt, leftmargin=0cm, labelwidth=1.3cm]
	\item[\textbf{Lista}]: Necromanzia
	\item[\textbf{Livello}]: 3, Non Comune
	\item[\textbf{Lancio}]: 2 Azioni
	\item[\textbf{Gittata}]: 36 metri
	\item[\textbf{Durata}]: 10 minuti
\end{description}
\end{minipage}}\smallskip

Puoi accecare o assordare un nemico. Scegli una creatura a gittata e che puoi vedere. Il bersaglio deve effettuare un Tiro Salvezza su Tempra. Se lo fallisce, il bersaglio è accecato o assordato (a tua scelta) per la durata.

\textbf{Per ogni Successo Critico Magico} ottenuto nella Prova di Magia puoi prendere come bersaglio una creatura aggiuntiva.

\textbf{Tiro Salvezza Fallimento Critico}: in caso di Fallimento Critico l'effetto è permanente.

\incantesimo{Celare}
\noindent\colorbox{OBSSgold!10}{
\begin{minipage}{0.95\linewidth}
\begin{description}[noitemsep, topsep=0pt, parsep=0pt, partopsep=0pt, leftmargin=0cm, labelwidth=1.3cm]
	\item[\textbf{Lista}]: Trasmutazione
	\item[\textbf{Livello}]: 7, Raro
	\item[\textbf{Lancio}]: 2 Azioni
	\item[\textbf{Gittata}]: Contatto
	\item[\textbf{Durata}]: Fino a che dissolto
\end{description}
\end{minipage}}\smallskip

Tramite questo incantesimo, una creatura consenziente o un oggetto può essere nascosto, impossibile da individuare per la durata. Eseguendo questo incantesimo ed entrando in contatto con un bersaglio questo diventa invisibile e non può essere preso come bersaglio dagli incantesimi di divinazione, né percepito da sensori di scrutamento creati da incantesimi di divinazione.

Se il bersaglio è una creatura, cade in uno stato di animazione sospesa. Per lui il tempo cessa di scorrere e non invecchia.

Puoi predisporre una condizione per cui l'incantesimo termini anticipatamente. La condizione può essere qualsiasi cosa tu voglia, ma deve avvenire o essere visibile entro 1,5 chilometri dal bersaglio. Esempi includono \emph{al prossimo giudizio dei Patroni} o \emph{quando il tarrasque si risveglia}. Questo incantesimo termina anche qualora il bersaglio subisca danni.

\incantesimo{Cerchio d'Invisibilità}
\noindent\colorbox{OBSSgold!10}{
\begin{minipage}{0.95\linewidth}
\begin{description}[noitemsep, topsep=0pt, parsep=0pt, partopsep=0pt, leftmargin=0cm, labelwidth=1.3cm]
	\item[\textbf{Lista}]: Illusione
	\item[\textbf{Livello}]: 3, Non Comune
	\item[\textbf{Lancio}]: 2 Azioni
	\item[\textbf{Gittata}]: Tocco
	\item[\textbf{Durata}]: 1 minuto per CM
\end{description}
\end{minipage}}\smallskip

Questo incantesimo funziona come invisibilità, ma ha effetto su tutte le creature toccate dall'incantatore che in seguito restano entro 3 metri da lui. I soggetti che si allontanano oltre il cerchio d'invisibilità tornano immediatamente visibili o se rompono le condizioni per mantenere l'invisibilità.


\incantesimo{Cerchio Magico}
\noindent\colorbox{OBSSgold!10}{
\begin{minipage}{0.95\linewidth}
\begin{description}[noitemsep, topsep=0pt, parsep=0pt, partopsep=0pt, leftmargin=0cm, labelwidth=1.3cm]
	\item[\textbf{Lista}]: Abiurazione
	\item[\textbf{Livello}]: 3, Comune
	\item[\textbf{Lancio}]: 1 minuto
	\item[\textbf{Gittata}]: 3 metri
	\item[\textbf{Durata}]: 1 ora
\end{description}
\end{minipage}}\smallskip

Crei un cilindro di energia magica di 3 metri di raggio e alto 6 metri, centrato su di un punto del terreno a gittata e che puoi vedere. Rune luminose compaiono dovunque il cilindro si intersechi con il pavimento o altra superficie.

Scegli uno o più dei seguenti tipi di creature: celestiali, elementali, fatati, demoni o non morti. Il circolo influisce su di una creatura del tipo scelto nei modi seguenti:

\medskip

- La creatura non può entrare consapevolmente nel cilindro tramite alcun mezzo non magico. Se la creatura prova a usare il teletrasporto o il viaggio tra i piani per farlo deve prima superare un Tiro Salvezza su Volontà.

- La creatura ha -1d6 ai Tiri per Colpire contro i bersagli all'interno del cilindro.

- I bersagli all'interno del cilindro non possono essere affascinati, spaventati o posseduti dalla creatura. Quando lanci questo incantesimo, puoi decidere che la magia operi in direzione opposta, impedendo a una creatura del tipo specificato di lasciare il cilindro e proteggendo i bersagli all'esterno.

\textbf{Per ogni Successo Critico Magico} ottenuto nella Prova di Magia puoi aumentare la durata di 1 ora.

\incantesimo{Cerchio di Morte}
\noindent\colorbox{OBSSgold!10}{
\begin{minipage}{0.95\linewidth}
\begin{description}[noitemsep, topsep=0pt, parsep=0pt, partopsep=0pt, leftmargin=0cm, labelwidth=1.3cm]
	\item[\textbf{Lista}]: Invocazione
	\item[\textbf{Livello}]: 6, Molto Raro
	\item[\textbf{Lancio}]: 2 Azioni
	\item[\textbf{Gittata}]: 45 metri
	\item[\textbf{Durata}]: Istantanea
\end{description}
\end{minipage}}\smallskip

Una sfera di energia negativa del raggio di 18 metri, erutta in un punto a gittata. Ogni creatura in quell'area deve effettuare un Tiro Salvezza su Tempra. La creatura subisce 8d6 danni da Vuoto se fallisce il Tiro Salvezza, o la metà di questi danni se lo supera.

\textbf{Per ogni Successo Critico Magico} ottenuto nella Prova di Magia il danno aumenta di 4d6.

\textbf{Tiro Salvezza Successo/Fallimento Critico}: In caso di Fallimento Critico il danno raddoppia, in caso di Successo Critico il danno viene ulteriormente dimezzato

\incantesimo{Cerchio di Teletrasporto}
\noindent\colorbox{OBSSgold!10}{
\begin{minipage}{0.95\linewidth}
\begin{description}[noitemsep, topsep=0pt, parsep=0pt, partopsep=0pt, leftmargin=0cm, labelwidth=1.3cm]
	\item[\textbf{Lista}]: Evocazione
	\item[\textbf{Livello}]: 5, Non Comune
	\item[\textbf{Lancio}]: 1 minuto
	\item[\textbf{Gittata}]: 3 metri
	\item[\textbf{Durata}]: 1 round
\end{description}
\end{minipage}}\smallskip

Mentre lanci l'incantesimo, tracci un cerchio di 3 metri di diametro sul pavimento, inscritto con sigilli che collegano il posto in cui ti trovi a un cerchio di teletrasporto permanente di tua scelta, di cui conosci la sequenza dei sigilli e che si trovi sullo stesso piano di esistenza in cui ti trovi tu. Un portale luminoso si apre all'interno del cerchio tracciato da te e resta aperto fino al termine del tuo prossimo round. Qualsiasi creatura che entri nel portale, riappare istantaneamente entro 1 metro dal cerchio di destinazione o nello spazio non occupato più vicino, se non può comparire entro 1 metro da esso.

Molti grandi templi, gilde, e altri luoghi importanti possiedono dei cerchi di teletrasporto permanenti, incisi da qualche parte nelle loro prossimità. Ciascuno di questi cerchi possiede una sequenza di sigilli unica: una serie di rune magiche disposte seguendo una trama precisa.

Quando ottieni la capacità di lanciare questo incantesimo, apprendi le sequenze di sigilli di due destinazioni sul Piano Materiale, determinate dal Narratore. Nel corso delle tue avventure puoi imparare nuove sequenze di sigilli. Puoi mandare a memoria una sequenza di sigilli dopo averla studiata per almeno 1 minuto.

Puoi creare un cerchio di teletrasporto permanente eseguendo questo incantesimo nello stesso luogo ogni giorno per un anno. Non devi usare il cerchio di teletrasporto quando lanci l'incantesimo in questo modo.

\textbf{NOTA}: Teletrasportarsi per più di 500 km ha solo il 5\% di riuscita.

\incantesimo{Chiaroveggenza}
\noindent\colorbox{OBSSgold!10}{
\begin{minipage}{0.95\linewidth}
\begin{description}[noitemsep, topsep=0pt, parsep=0pt, partopsep=0pt, leftmargin=0cm, labelwidth=1.3cm]
	\item[\textbf{Lista}]: Divinazione
	\item[\textbf{Livello}]: 3, Comune
	\item[\textbf{Lancio}]: 10 minuti
	\item[\textbf{Gittata}]: 1,5 chilometri
	\item[\textbf{Durata}]: Concentrazione, massimo 10 minuti
\end{description}
\end{minipage}}\smallskip

Crei un sensore invisibile in un luogo a te familiare e che sia a gittata (un luogo che hai già visitato o visto precedentemente) o in un luogo ovvio ma che non ti è familiare (come dietro una porta o un angolo, o in mezzo un boschetto di alberi). Il sensore rimane sul posto per la durata, e non può essere attaccato né altrimenti vi si può interagire. Quando lanci questo incantesimo, scegli se vedere o udire. Puoi usare il senso scelto tramite il sensore, come ti trovassi nel suo spazio. Con due azioni, puoi passare da udire a sentire e viceversa. Una creatura che può vedere il sensore (una creatura munita di \hyperlink{Vedere l'invisibile}{Vedere l'invisibile} o di visione del vero) lo percepisce come un orbe intangibile e luminoso delle dimensioni del tuo pugno.

\textbf{Per ogni Successo Critico Magico} ottenuto nella Prova di Magia la durata aumenta di 10 minuti o la gittata aumenta di 500m.

\incantesimo{Chiudi Portale}
\noindent\colorbox{OBSSgold!10}{
\begin{minipage}{0.95\linewidth}
\begin{description}[noitemsep, topsep=0pt, parsep=0pt, partopsep=0pt, leftmargin=0cm, labelwidth=1.3cm]
	\item[\textbf{Lista}]: Abiurazione
	\item[\textbf{Livello}]: 2, Raro
	\item[\textbf{Lancio}]:  1 Turno
	\item[\textbf{Gittata}]: 18 metri
	\item[\textbf{Durata}]: Istantanea
\end{description}
\end{minipage}}\smallskip

L'incantatore si pone entro distanza da un Portale. Formulato l'incantesimo il Portale se ha una DC inferiore a quella dell'incantatore viene chiuso e scompare.

\textbf{Per ogni Successo Critico Magico} ottenuto nella Prova di Magia la tua DC aumenta di 4.

\textbf{NOTA}: l'incantesimo è Comune per i Devoti di Lynx

\incantesimo{Colpo Accurato}
\noindent\colorbox{OBSSgold!10}{
\begin{minipage}{0.95\linewidth}
\begin{description}[noitemsep, topsep=0pt, parsep=0pt, partopsep=0pt, leftmargin=0cm, labelwidth=1.3cm]
	\item[\textbf{Lista}]: Divinazione
	\item[\textbf{Livello}]: 0, Comune
	\item[\textbf{Lancio}]: 2 Azioni
	\item[\textbf{Gittata}]: 9 metri
	\item[\textbf{Durata}]: 1 round
\end{description}
\end{minipage}}\smallskip

Allunghi la mano e punti il dito verso un bersaglio a gittata. La tua magia ti conferisce una breve comprensione delle difese del bersaglio. Entro la fine del prossimo round ottieni +1d6 al primo Tiro per Colpire contro quel bersaglio.

\textbf{Per ogni Successo Critico Magico} ottenuto il bonus perdura per un round in più.

\incantesimo{Colpo Accecante}
\noindent\colorbox{OBSSgold!10}{
\begin{minipage}{0.95\linewidth}
\begin{description}[noitemsep, topsep=0pt, parsep=0pt, partopsep=0pt, leftmargin=0cm, labelwidth=1.3cm]
	\item[\textbf{Lista}]: Invocazione
	\item[\textbf{Livello}]: 3, Raro
	\item[\textbf{Lancio}]: 2 Azioni

	\item[\textbf{Durata}]: 1 minuto
\end{description}
\end{minipage}}\smallskip

Il bersaglio colpito dal colpo subisce 3d8 danni da Luce e deve superare il Tiro Salvezza su Tempra o diventare Accecato fino al termine dell'incantesimo. Alla fine di ciascuno dei suoi round, il bersaglio accecato ripete il Tiro Salvezza terminando l'incantesimo su se stesso in caso di successo.

\textbf{Per ogni Successo Critico Magico} ottenuto nella Prova di Magia infliggi +2d6 danni da Luce in più.

\incantesimo{Colpo Infuocato}
\noindent\colorbox{OBSSgold!10}{
\begin{minipage}{0.95\linewidth}
\begin{description}[noitemsep, topsep=0pt, parsep=0pt, partopsep=0pt, leftmargin=0cm, labelwidth=1.3cm]
	\item[\textbf{Lista}]: Fuoco
	\item[\textbf{Livello}]: 5, Comune
	\item[\textbf{Lancio}]: 2 Azioni
	\item[\textbf{Gittata}]: 18 metri
	\item[\textbf{Durata}]: Istantanea
\end{description}
\end{minipage}}\smallskip

Una colonna verticale di fuoco divino scende dal cielo e si abbatte sul luogo da te specificato. Ogni creatura in un cilindro di 3 metri di raggio e alto 12 metri centrato su di un punto a gittata deve effettuare un Tiro Salvezza su Riflessi. Una creatura subisce 8d6 danni da Luce se fallisce il Tiro Salvezza, o la metà di questi danni se lo supera.

\textbf{Per ogni Successo Critico Magico} ottenuto nella Prova di Magia il danno Luce aumenta di 4d6.

\textbf{Tiro Salvezza Successo/Fallimento Critico}: In caso di Fallimento Critico il danno raddoppia, in caso di Successo Critico il danno viene ulteriormente dimezzato

\incantesimo{Colpo Fiammeggiante}
\noindent\colorbox{OBSSgold!10}{
\begin{minipage}{0.95\linewidth}
\begin{description}[noitemsep, topsep=0pt, parsep=0pt, partopsep=0pt, leftmargin=0cm, labelwidth=1.3cm]
	\item[\textbf{Lista}]: Invocazione
	\item[\textbf{Livello}]: 1, Raro
	\item[\textbf{Lancio}]: 1 Azione
	\item[\textbf{Gittata}]: personale
	\item[\textbf{Durata}]: 1 minuto
\end{description}
\end{minipage}}\smallskip

Tocchi un bersaglio, questo subisce 1d6 danni da Fuoco. Ogni round deve effettuare un Tiro Salvezza su Tempra o subire 1d6 di danno da fuoco, questo effetto termina dopo un minuto oppure quando il Tiro Salvezza riesce.

\textbf{Per ogni Successo Critico Magico} ottenuto nella Prova di Magia infliggi +1d6 danni da Fuoco iniziali.

\incantesimo{Colpo Luccicante}
\noindent\colorbox{OBSSgold!10}{
\begin{minipage}{0.95\linewidth}
\begin{description}[noitemsep, topsep=0pt, parsep=0pt, partopsep=0pt, leftmargin=0cm, labelwidth=1.3cm]
	\item[\textbf{Lista}]: Invocazione
	\item[\textbf{Livello}]: 2, Non Comune
	\item[\textbf{Lancio}]: 1 Azione
	\item[\textbf{Gittata}]: personale
	\item[\textbf{Durata}]: 1 minuto
\end{description}
\end{minipage}}\smallskip

Tocchi un bersaglio, questo subisce 2d6 danni da Luce e diventa visibile per tutta la durata dell'incantesimo. In aggiunta la creatura emana luce in 1 metro di raggio.

\textbf{Per ogni Successo Critico Magico} ottenuto nella Prova di Magia infliggi 1d6 danni da Luce aggiuntivi.

\incantesimo{Comando}
\noindent\colorbox{OBSSgold!10}{
\begin{minipage}{0.95\linewidth}
\begin{description}[noitemsep, topsep=0pt, parsep=0pt, partopsep=0pt, leftmargin=0cm, labelwidth=1.3cm]
	\item[\textbf{Lista}]: Ammaliamento
	\item[\textbf{Livello}]: 1, Comune
	\item[\textbf{Lancio}]: 2 Azioni
	\item[\textbf{Gittata}]: 18 metri
	\item[\textbf{Durata}]: 1 round
\end{description}
\end{minipage}}\smallskip

Pronunci un comando di una parola verso una creatura a gittata e che puoi vedere ed un gesto. Il bersaglio deve superare un Tiro Salvezza su Volontà o eseguire il comando entro il suo prossimo round. L'incantesimo non ha effetto se il bersaglio è non morto, se non capisce la tua lingua o se il tuo comando gli recherebbe danni.

Sono elencati alcuni tipici comandi e i loro effetti. Puoi dare comandi diversi da quelli descritti qui, in quel caso il Narratore determinerà il comportamento del bersaglio. Se il bersaglio non può eseguire il tuo comando, l'incantesimo ha fine.

\begin{itemize}[leftmargin=*] \setlength{\itemsep}{0pt}
	\item \emph{Avvicinati}. Il bersaglio si muove verso di te per il tragitto più breve e diretto, terminando il suo round se si avvicina a 1 metro da te.
	\item \emph{Fermo}. Il bersaglio non si muove e poi termina il suo round. Una creatura volante resta sul posto, purché le sia possibile. Se deve muoversi per restare in aria, vola la distanza minima necessaria per farlo.
	\item \emph{Getta}. Il bersaglio getta qualsiasi cosa stia tenendo in mano e poi termina il suo round.
	\item \emph{Scappa}. Il bersaglio spende il suo round a muoversi lontano da te con il mezzo più veloce a sua disposizione.
	\item \emph{Striscia}. Il bersaglio si getta prono e poi termina il suo round.
\end{itemize}

\textbf{Per ogni Successo Critico Magico} ottenuto nella Prova di Magia puoi agire su di un'ulteriore creatura. Nel momento in cui lanci l'incantesimo le creature bersaglio devono trovarsi entro 9 metri l'una da l'altra ed eseguono il medesimo comando.

\incantesimo{Conoscere i Tratti}
\noindent\colorbox{OBSSgold!10}{
\begin{minipage}{0.95\linewidth}
\begin{description}[noitemsep, topsep=0pt, parsep=0pt, partopsep=0pt, leftmargin=0cm, labelwidth=1.3cm]
	\item[\textbf{Lista}]: Divinazione
	\item[\textbf{Livello}]: 1, Leggendario
	\item[\textbf{Lancio}]: 2 Azioni
	\item[\textbf{Gittata}]: Personale
	\item[\textbf{Durata}]: Istantanea
\end{description}
\end{minipage}}\smallskip

Questo incantesimo permette di conoscere i Tratti di una creatura. Al soggetto è concesso un Tiro Salvezza su Volontà per resistere. Indipendentemente dal risultato del Tiro Salvezza la creatura sa di per certo chi ha formulato l'incantesimo.

\textbf{Per due Successo Critico Magico} ottenuto nella Prova di Magia puoi analizzare un altra creatura.

\incantesimo{Comprensione dei Linguaggi}
\noindent\colorbox{OBSSgold!10}{
\begin{minipage}{0.95\linewidth}
\begin{description}[noitemsep, topsep=0pt, parsep=0pt, partopsep=0pt, leftmargin=0cm, labelwidth=1.3cm]
	\item[\textbf{Lista}]: Divinazione
	\item[\textbf{Livello}]: 1, Comune
	\item[\textbf{Lancio}]: 2 Azioni
	\item[\textbf{Gittata}]: Personale
	\item[\textbf{Durata}]: 1 ora
\end{description}
\end{minipage}}\smallskip

Per la durata, capisci il significato letterale di qualsiasi linguaggio parlato che ascolti.

\textbf{Per ogni Successo Critico Magico} ottenuto nella Prova di Magia la durata raddoppia. Con tre successi critici sei anche in grado di leggere.

\textbf{NOTA}: se sei un Devoto di Nethergal l'incantesimo dura 2 ore.

\incantesimo{Comprensione degli Scritti}
\noindent\colorbox{OBSSgold!10}{
\begin{minipage}{0.95\linewidth}
\begin{description}[noitemsep, topsep=0pt, parsep=0pt, partopsep=0pt, leftmargin=0cm, labelwidth=1.3cm]
	\item[\textbf{Lista}]: Divinazione
	\item[\textbf{Livello}]: 2, Non Comune
	\item[\textbf{Lancio}]: 2 Azioni
	\item[\textbf{Gittata}]: Personale
	\item[\textbf{Durata}]: 1 ora
\end{description}
\end{minipage}}\smallskip

Per la durata comprendi il significato letterale di qualsiasi linguaggio scritto non magico che vedi. Devi essere a contatto con la superficie su cui le parole sono scritte. Per leggere una pagina di testo impieghi 1 minuto. Questo incantesimo non decodifica i messaggi segreti in un testo o i glifi, come un sigillo arcano, che non faccia parte di un linguaggio scritto.

\textbf{Per ogni Successo Critico Magico} ottenuto nella Prova di Magia la durata raddoppia.

\textbf{NOTA}: se sei un Devoto di Nethergal l'incantesimo è comune e dura 2 ore.

\incantesimo{Compulsione}
\noindent\colorbox{OBSSgold!10}{
\begin{minipage}{0.95\linewidth}
\begin{description}[noitemsep, topsep=0pt, parsep=0pt, partopsep=0pt, leftmargin=0cm, labelwidth=1.3cm]
	\item[\textbf{Lista}]: Ammaliamento
	\item[\textbf{Livello}]: 4, Non Comune
	\item[\textbf{Lancio}]: 2 Azioni
	\item[\textbf{Gittata}]: 9 metri
	\item[\textbf{Durata}]: Concentrazione, massimo 1 minuto
\end{description}
\end{minipage}}\smallskip

Le creature di tua scelta entro la gittata che puoi vedere e che ti possono sentire devono effettuare un Tiro Salvezza su Volontà. Un bersaglio supera automaticamente il Tiro Salvezza se non può essere Affascinato. Fino al termine dell'incantesimo puoi usare un'Azione durante ciascun tuo round per indicare una direzione orizzontale rispetto a te. Ogni bersaglio soggetto all'incantesimo deve usare quanto più possibile del suo movimento, durante il suo prossimo round, per muoversi in quella direzione. Il bersaglio non può effettuare nessun'altra Azione prima di muoversi. Dopo essersi mosso in questo modo il bersaglio può effettuare un altro Tiro Salvezza su Volontà per tentare di terminare l'effetto.

Un bersaglio non può essere obbligato a muoversi dentro un pericolo palesemente letale, come fiamme o crepacci.

\incantesimo{Comunione}
\noindent\colorbox{OBSSgold!10}{
\begin{minipage}{0.95\linewidth}
\begin{description}[noitemsep, topsep=0pt, parsep=0pt, partopsep=0pt, leftmargin=0cm, labelwidth=1.3cm]
	\item[\textbf{Lista}]: Divinazione
	\item[\textbf{Livello}]: 5, Raro
	\item[\textbf{Lancio}]: 1 minuto
	\item[\textbf{Gittata}]: Personale
	\item[\textbf{Durata}]: 1 minuto
\end{description}
\end{minipage}}\smallskip

Comunichi con il tuo Patrono e gli poni fino a tre domande a cui si può dare risposta con un sì o un no. Devi porre le domande prima della fine dell'incantesimo. Riceverai la risposta corretta a ciascuna domanda. Le creature divine non sono necessariamente onniscienti, quindi potresti ricevere un \emph{non è chiaro} come risposta a una domanda che riguarda informazioni non pertinenti alle conoscenze del Patrono. Nel caso in cui una risposta di una parola potrebbe essere fuorviante o contraria agli interessi del Patrono, il Narratore potrebbe invece dare una breve frase come risposta.

Se lanci l'incantesimo due o più volte prima che sia sorta la nuova alba c'è una probabilità cumulativa del 25\% che per ogni lancio dopo il primo tu non ottenga alcuna risposta. Il Narratore effettua questo tiro in segreto.

\textbf{NOTA:} è necessario essere almeno Seguace per poter formulare questo incantesimo.

\incantesimo{Comunione con la Natura}
\noindent\colorbox{OBSSgold!10}{
\begin{minipage}{0.95\linewidth}
\begin{description}[noitemsep, topsep=0pt, parsep=0pt, partopsep=0pt, leftmargin=0cm, labelwidth=1.3cm]
	\item[\textbf{Lista}]: Divinazione
	\item[\textbf{Livello}]: 5, Molto Raro
	\item[\textbf{Lancio}]: 1 minuto
	\item[\textbf{Gittata}]: Personale
	\item[\textbf{Durata}]: Istantanea
\end{description}
\end{minipage}}\smallskip

Per un istante diventi tutt'uno con la natura e ottieni informazioni sul territorio circostante. In ambienti esterni, l'incantesimo ti fornisce informazioni sul territorio entro 5 chilometri da te. In grotte e altri ambienti naturali sotterranei, il raggio è limitato a 100 metri. L'incantesimo non funziona nei luoghi in cui la natura è stata soppiantata da costruzioni, come in sotterranei e paesi.

Apprendi immediatamente informazioni su un massimo di tre argomenti a tua scelta su uno dei seguenti soggetti, in relazione all'area:

\begin{itemize}\setlength{\itemsep}{-1pt}
	\item terreno e corpi d'acqua
	\item piante, minerali, animali e popolazioni prevalenti
	\item potenti celestiali, elementali, fatati, demoni o non morti
	\item influenze da altri piani di esistenza
	\item edifici
\end{itemize}

\textbf{Per ogni Successo Critico Magico} ottenuto nella Prova di Magia apprendi un argomento aggiuntivo.

\textbf{NOTA}: Se sei un Devoto di Efrem ottiene sempre almeno un Successo Critico Magico.

\incantesimo{Confusione}\hypertarget{incconfusione}{}\label{incconfusione}
\noindent
\begin{description}[noitemsep, topsep=0pt, parsep=0pt, partopsep=0pt, leftmargin=0cm, labelwidth=1.3cm]
	\item[\textbf{Lista}]: Ammaliamento
	\item[\textbf{Livello}]: 4, Comune
	\item[\textbf{Lancio}]: 2 Azioni
	\item[\textbf{Gittata}]: 27 metri
	\item[\textbf{Durata}]: 1 minuto
\end{description}

Questo incantesimo assale e piega la mente delle creature, generando illusioni e provocando azioni incontrollate. Quando lanci questo incantesimo ogni creatura, in una sfera di 3 metri di raggio centrata su di un punto da te scelto entro la gittata, deve superare un Tiro Salvezza su Volontà o subirne gli effetti. Un bersaglio soggetto all'incantesimo non può effettuare reazioni e deve tirare un d10 all'inizio di ciascun suo round per determinare il proprio comportamento per quel round.

Al termine di ciascun suo round, un bersaglio soggetto all'incantesimo può effettuare un Tiro Salvezza su Volontà. Se lo supera per lui l'effetto ha termine.

\textbf{Per ogni Successo Critico Magico} ottenuto nella Prova di Magia il raggio della sfera aumenta di 1 metro.

\medskip

\noindent\begin{tabularx}{\linewidth}{lX}
	\toprule
 \rowcolor{gray!20}d10 & Comportamento\\
	\toprule
	1 & La creatura usa tutte le sue Azioni per per muoversi in una direzione casuale. Per determinare la direzione tira un d8\\
 \rowcolor{gray!20}2-5 & La creatura non fa nulla per tutto il round\\
	6 & La creatura effettua un attacco contro se stessa e finisce il round\\
 \rowcolor{gray!20}7-8 & La creatura effettua un attacco contro una creatura determinata a caso entro 1 Azione di Movimento. Se è stata colpita il round precedente attaccherà la creatura che l'ha colpito. Fatto l'attacco il round termina.\\
	9-10 & La creatura può agire e muoversi normalmente.
\end{tabularx}

\incantesimo{Confusione Contagiosa}
\noindent\colorbox{OBSSgold!10}{
\begin{minipage}{0.95\linewidth}
\begin{description}[noitemsep, topsep=0pt, parsep=0pt, partopsep=0pt, leftmargin=0cm, labelwidth=1.3cm]
	\item[\textbf{Lista}]: Ammaliamento
	\item[\textbf{Livello}]: 8, Molto raro
	\item[\textbf{Lancio}]: 10 minuti
	\item[\textbf{Gittata}]: Contatto
	\item[\textbf{Durata}]: 1 minuto
\end{description}
\end{minipage}}\smallskip

Questo incantesimo assale e piega la mente delle creature, generando illusioni e provocando azioni incontrollate. Una volta formulato questo incantesimo hai poi un minuto per toccare la prima creatura. Questa creatura può fare un Tiro Salvezza su Volontà per annullare gli effetti.

Qualsiasi creatura toccata dalla prima creatura trasmette l'effetto Confusione, con Tiro Salvezza come la prima creatura, la durata dell'effetto su questa creatura sarà di un minuto.

Se l'incantatore non tocca entro un minuto una creatura allora sarà lui stesso vittima dell'incantesimo confusione, senza possibilità di Tiro Salvezza.

\incantesimo{Cono di Freddo}
\noindent\colorbox{OBSSgold!10}{
\begin{minipage}{0.95\linewidth}
\begin{description}[noitemsep, topsep=0pt, parsep=0pt, partopsep=0pt, leftmargin=0cm, labelwidth=1.3cm]
	\item[\textbf{Lista}]: Acqua
	\item[\textbf{Livello}]: 5, Comune
	\item[\textbf{Lancio}]: 2 Azioni
	\item[\textbf{Gittata}]: Personale (cono di 18 metri)
	\item[\textbf{Durata}]: Istantanea
\end{description}
\end{minipage}}\smallskip

Un'esplosione di aria fredda erutta dalle tue mani. Ogni creatura in un cono di 18 metri deve effettuare un Tiro Salvezza su Tempra. Una creatura subisce 8d8 danni da freddo se fallisce il Tiro Salvezza o la metà di questi danni se lo supera. Una creatura uccisa da questo incantesimo diventa una statua di ghiaccio fino a quando scongela.

\textbf{Per ogni Successo Critico Magico} ottenuto nella Prova di Magia il danno aumenta di 4d8

\textbf{Tiro Salvezza Successo/Fallimento Critico}: In caso di Fallimento Critico il danno raddoppia, in caso di Successo Critico il danno viene ulteriormente dimezzato

\incantesimo{Conoscenza delle Leggende}
\noindent\colorbox{OBSSgold!10}{
\begin{minipage}{0.95\linewidth}
\begin{description}[noitemsep, topsep=0pt, parsep=0pt, partopsep=0pt, leftmargin=0cm, labelwidth=1.3cm]
	\item[\textbf{Lista}]: Divinazione
	\item[\textbf{Livello}]: 5, Comune
	\item[\textbf{Lancio}]: 10 minuti
	\item[\textbf{Gittata}]: Personale
	\item[\textbf{Durata}]: Istantanea
\end{description}
\end{minipage}}\smallskip

Nomina o descrivi una persona, luogo od oggetto. L'incantesimo ti porta alla mente un breve riassunto delle conoscenze più importanti sull'argomento da te nominato. Se la cosa da te nominata non ha alcuna rilevanza leggendaria, non ottieni alcuna informazione. Maggiori informazioni hai sull'argomento, più precise e dettagliate saranno le informazioni che riceverai. L'informazione che riceverai sarà accurata, ma celata magari in linguaggio metaforico.

\incantesimo{Contagio}
\noindent\colorbox{OBSSgold!10}{
\begin{minipage}{0.95\linewidth}
\begin{description}[noitemsep, topsep=0pt, parsep=0pt, partopsep=0pt, leftmargin=0cm, labelwidth=1.3cm]
	\item[\textbf{Lista}]: Necromanzia
	\item[\textbf{Livello}]: 5, Non Comune
	\item[\textbf{Lancio}]: 2 Azioni
	\item[\textbf{Gittata}]: Contatto
	\item[\textbf{Durata}]: 7 giorni
\end{description}
\end{minipage}}\smallskip

Tramite il contatto puoi infliggere malattie. Effettua un attacco da mischia contro una creatura a portata. Se colpisci, infetti la creatura con una malattia a tua scelta tra quelle descritte di seguito. Al termine di ciascun round del bersaglio, esso deve effettuare un Tiro Salvezza su Tempra. Dopo aver fallito tre di questi Tiri Salvezza, gli effetti della malattia permangono per la durata e la creatura non effettua più Tiri Salvezza. Dopo aver superato tre di questi Tiri Salvezza la creatura recupera dalla malattia e l'incantesimo ha termine. Nel mentre che esegue i Tiri Salvezza la creatura subisce gli effetti della malattia.

Dato che questo incantesimo induce nel suo bersaglio una malattia naturale, qualsiasi effetto che rimuova le malattie o migliori gli effetti delle malattie si applica a essa.

\begin{itemize}[leftmargin=*] \setlength{\itemsep}{0pt}
	\item \emph{Carne Putrida}. La pelle della creatura marcisce. La creatura ha -1d6 alle prove di Carisma e ogni danno è raddoppiato.
	\item \emph{Debolezza Accecante}. Il dolore attanaglia la mente della creatura mentre i suoi occhi diventano bianco latte. La creatura ha -1d6 alle prove di Saggezza e ai Tiri Salvezza su Volontà ed è accecata.
	\item \emph{Febbre Lurida}. Una febbre devastante sconvolge il corpo della creatura. La creatura ha -1d6 alle prove di Forza e ai Tiri Salvezza su Tempra e ai Tiri per Colpire che usano la Forza.
	\item \emph{Fitte}. La creatura è sopraffatta dai tremiti. La creatura ha -1d6 alle prove di Destrezza, ai Tiri Salvezza su Riflessi ed ai Tiri per Colpire che usano la Destrezza.
	\item \emph{Fuoco Mentale}. La mente della creatura è preda della febbre. La creatura ha -1d6 alle prove di Intelligenza e ai Tiri Salvezza su Volontà e si comporta come se in combattimento fosse sotto l'effetto dell'incantesimo confusione.
	\item \emph{Morte Melmosa}. La creatura inizia a sanguinare incessantemente. La creatura ha -1d6 alle prove di Costituzione ed ai Tiri Salvezza su Tempra. Inoltre ogni qualvolta la creatura subisce danni è Rallentata 1/1r fino alla fine del suo prossimo round.
\end{itemize}

\incantesimo{Contingenza}
\noindent\colorbox{OBSSgold!10}{
\begin{minipage}{0.95\linewidth}
\begin{description}[noitemsep, topsep=0pt, parsep=0pt, partopsep=0pt, leftmargin=0cm, labelwidth=1.3cm]
	\item[\textbf{Lista}]: Invocazione
	\item[\textbf{Livello}]: 6, Non Comune
	\item[\textbf{Lancio}]: 10 minuti
	\item[\textbf{Gittata}]: Personale
	\item[\textbf{Durata}]: 10 giorni
\end{description}
\end{minipage}}\smallskip

Scegli un incantesimo di Livello 4 o più basso che puoi lanciare, che abbia il tempo di lancio di 2 Azioni e che può avere te come bersaglio. Lanci quell'incantesimo (detto incantesimo contingente) come parte del lancio di contingenza, spendendo gli slot incantesimo di entrambi, ma senza che l'incantesimo contingente abbia effetto. Avrà invece effetto quando si avvererà una determinata circostanza. Descrivi questa circostanza mentre lanci i due incantesimi. Per esempio, contingenza lanciato assieme a respirare sott'acqua potrebbe stipulare che respirare sott'acqua entra in azione quando sei immerso nell'acqua o simile liquido.

L'incantesimo contingente ha effetto immediatamente dopo che la circostanza si verifica per la prima volta, che tu lo voglia o no, e poi contingenza termina. L'incantesimo contingente agisce solo su di te, anche se normalmente può prendere come bersaglio anche altri. Puoi usare un solo incantesimo contingenza alla volta. Se lanci di nuovo questo incantesimo, l'effetto di un altro incantesimo contingenza su di te avrà termine.

\textbf{Per ogni Successo Critico Magico} ottenuto nella Prova di Magia la contingenza dura 10 giorni in più.

\incantesimo{Controincantesimo}
\noindent\colorbox{OBSSgold!10}{
\begin{minipage}{0.95\linewidth}
\begin{description}[noitemsep, topsep=0pt, parsep=0pt, partopsep=0pt, leftmargin=0cm, labelwidth=1.3cm]
	\item[\textbf{Lista}]: Abiurazione
	\item[\textbf{Livello}]: 3, Comune
	\item[\textbf{Lancio}]: 1 Reazione, che effettui quando vedi una creatura/oggetto entro 18 metri manifestare un incantesimo
	\item[\textbf{Gittata}]: 18 metri
	\item[\textbf{Durata}]: Istantanea
\end{description}
\end{minipage}}\smallskip

Usi una Azione di Reazione per fare una prova di Arcana a DC 13. Se la prova riesce comprendi se puoi annullare l'effetto dell'incantesimo tramite Controincantesimo. L'incantesimo annullato deve essere di Livello 2 o più basso, indipendentemente che sia manifestato da un incantatore od oggetto. Ogni successo Critico magico o potenziamento ottenuto dall'incantesimo originale alza il livello dell'incantesimo di 1.

\textbf{Per ogni due Successo Critico Magico ottenuto} nella Prova di Magia puoi annullare un incantesimo di un livello superiore.

\incantesimo{Controllare Acqua}
\noindent\colorbox{OBSSgold!10}{
\begin{minipage}{0.95\linewidth}
\begin{description}[noitemsep, topsep=0pt, parsep=0pt, partopsep=0pt, leftmargin=0cm, labelwidth=1.3cm]
	\item[\textbf{Lista}]: Acqua
	\item[\textbf{Livello}]: 4, Comune
	\item[\textbf{Lancio}]: 2 Azioni
	\item[\textbf{Gittata}]: 90 metri
\end{description}
\end{minipage}}\smallskip

Fino al termine dell'incantesimo, controlli qualsiasi acqua libera all'interno dell'area che hai scelto fino a un cubo di 30 metri di spigolo. Quando lanci questo incantesimo puoi scegliere qualsiasi tra i seguenti effetti. Come Azione, durante il tuo round, puoi ripetere lo stesso effetto o sceglierne uno diverso.

\medskip

- \emph{Allagamento}. Fai sì che il livello di tutta l'acqua nell'area aumenti fino a 6 metri. Se l'area include una costa, l'acqua inonda la terraferma. Se scegli un'area all'interno di un grosso corpo d'acqua, crei invece un'onda alta 6 metri che viaggia da un lato all'altro dell'area prima di infrangersi. Qualsiasi veicolo di taglia Enorme o inferiore sul percorso dell'onda viene trasportato dall'altro lato. Qualsiasi veicolo di taglia Enorme o inferiore colpito dall'acqua ha una percentuale del 25\% di cappottarsi.

Il livello dell'acqua resta elevato fino al termine dell'incantesimo o finché non scegli un effetto diverso. Se questo effetto ha prodotto un'onda, l'onda si ripete all'inizio del tuo round successivo, finché perdura l'effetto di allagamento.

- \emph{Dividere le Acque}. Fai sì che l'acqua nell'area si sposti a lato per creare un varco. Il varco si estende per l'area dell'incantesimo, e l'acqua divisa forma un muro su entrambi i lati del varco. Il varco resta fino al termine dell'incantesimo o finché non scegli un effetto diverso. L'acqua tornerà poi lentamente a riempire il varco nel corso del round successivo, fino a che non sarà risalita al suo normale livello.

- \emph{Ridirigere il Flusso}. Fai sì che l'acqua corrente nell'area si muova in una direzione a tua scelta, anche se l'acqua deve superare degli ostacoli, risalire muri o dirigersi verso altre direzioni improbabili. L'acqua nell'area si muove secondo le tue indicazioni, ma una volta giunta oltre l'area dell'incantesimo, riprende il suo flusso in base alle condizioni del terreno. L'acqua continua a muoversi nella direzione da te scelta fino al termine dell'incantesimo o finché non scegli un effetto diverso.

- \emph{Turbine}. Questo effetto richiede un corpo d'acqua che copra un quadrato di 15 metri di lato e abbia una profondità di 7 metri. Fai sì che si formi un turbine al centro dell'area. Il turbine produce un vortice largo 1 metro alla base, largo fino a 15 metri in cima e alto 7 metri. Qualsiasi creatura od oggetto nell'acqua e che si trovi entro 7 metri dal vortice viene trascinato 3 metri verso di esso. Una creatura può nuotare per allontanarsi dal vortice effettuando una prova di Nuotare contro la DC del Tiro Salvezza dell'incantesimo.

Quando una creatura entra nel vortice per la prima volta durante un round o inizia lì il suo round, deve effettuare un Tiro Salvezza su Tempra. Se lo fallisce, la creatura subisce 2d8 danni contundenti e viene catturata dal vortice fino al termine dell'incantesimo. Se supera il Tiro Salvezza, la creatura subisce la metà di questi danni, e non è catturata dal vortice. Una creatura catturata dal vortice può usare 3 Azioni per cercare di nuotare via dal vortice come descritto sopra, ma ha -4 alle prove di Nuotare per farlo. La prima volta durante ciascun round in un cui un oggetto entra nel vortice, l'oggetto subisce 2d8 danni contundenti; questo danno viene subito ogni round in cui l'oggetto rimane nel vortice.

\incantesimo{Controllare Tempo Atmosferico}
\noindent\colorbox{OBSSgold!10}{
\begin{minipage}{0.95\linewidth}
\begin{description}[noitemsep, topsep=0pt, parsep=0pt, partopsep=0pt, leftmargin=0cm, labelwidth=1.3cm]
	\item[\textbf{Lista}]: Acqua, Aria
	\item[\textbf{Livello}]: 8, Molto Raro
	\item[\textbf{Lancio}]: 10 minuti
	\item[\textbf{Gittata}]: Personale (raggio di 1,5 chilometri)
	\item[\textbf{Durata}]: Concentrazione, massimo 8 ore
\end{description}
\end{minipage}}\smallskip

Per la durata, assumi il controllo del clima entro 7,5 chilometri da te. Per lanciare questo incantesimo devi essere all'esterno. Muoversi in un posto dove non hai la visuale aperta verso il cielo termina l'incantesimo anticipatamente. Quando lanci questo incantesimo, cambia le attuali condizioni climatiche determinate dal Narratore in base alla stagione e la latitudine. Puoi modificare le precipitazioni, la temperatura e il vento. Ci vogliono 1d4 x 10 minuti perché la nuova condizione prenda effetto. Una volta che la condizione avrà preso effetto, potrai cambiarla di nuovo. Quando l'incantesimo termina il clima tornerà gradualmente alla norma.

\medskip

Quando cambi le condizioni climatiche, trova l'attuale condizione sulla seguente tabella e cambiala di uno stadio, verso l'alto o il basso. Quando cambi il vento, puoi cambiarne anche la direzione.

\medskip

\emph{Precipitazione}

- 1 Limpido

- 2 Qualche nuvola

- 3 Coperto o foschia a terra

- 4 Pioggia, grandine o neve

- 5 Pioggia torrenziale, grandinata pesante, tormenta

\medskip

\emph{Temperatura}

- 1 Caldo insopportabile

- 2 Caldo

- 3 Tiepido

- 4 Fresco

- 5 Freddo

- 6 Freddo polare

\medskip

\emph{Vento}

- 1 Calmo

- 2 Vento moderato

- 3 Vento moderato

- 4 Fortunale

- 5 Tempesta

\medskip

\textbf{Per ogni Successo Critico Magico} ottenuto nella Prova di Magia la durata aumenta di 8 ore.

\incantesimo{Costrizione}
\noindent\colorbox{OBSSgold!10}{
\begin{minipage}{0.95\linewidth}
\begin{description}[noitemsep, topsep=0pt, parsep=0pt, partopsep=0pt, leftmargin=0cm, labelwidth=1.3cm]
	\item[\textbf{Lista}]: Ammaliamento
	\item[\textbf{Livello}]: 5, Raro
	\item[\textbf{Lancio}]: 1 minuto
	\item[\textbf{Gittata}]: 18 metri
	\item[\textbf{Durata}]: 30 giorni
\end{description}
\end{minipage}}\smallskip

Imponi un comando magico a una creatura a gittata che puoi vedere, obbligandolo ad adempiere un determinato compito o vietandole di svolgere un'azione o corso d'attività deciso da te. Se la creatura ti può capire, deve superare un Tiro Salvezza su Volontà o restare affascinata da te per la durata. Mentre la creatura è affascinata da te, subisce 3d10 danni ogni volta che agisce in maniera direttamente contraria alle tue istruzioni, ma non più di una volta al giorno. Una creatura che non ti può capire ignora gli effetti di questo incantesimo. Puoi dare qualsiasi comando di tua scelta, tranne un'attività che provocherebbe morte certa. Dovessi tu pronunciare un comando suicida, l'incantesimo avrebbe termine.

Puoi terminare l'incantesimo usando un'Azione. Anche Rimuovi Maledizione, Ristorare superiore o Desiderio vi pongono termine.

\textbf{Se ottieni almeno due Critici Magici} nella Prova di Magia la durata è 1 anno. Se ottieni 3 Critici l'incantesimo dura finché non viene terminato da uno degli incantesimi sopra menzionati.

\incantesimo{Creare Cibo e Acqua}
\noindent\colorbox{OBSSgold!10}{
\begin{minipage}{0.95\linewidth}
\begin{description}[noitemsep, topsep=0pt, parsep=0pt, partopsep=0pt, leftmargin=0cm, labelwidth=1.3cm]
	\item[\textbf{Lista}]: Evocazione
	\item[\textbf{Livello}]: 3, Comune
	\item[\textbf{Lancio}]: 2 Azioni
	\item[\textbf{Gittata}]: 9 metri
	\item[\textbf{Durata}]: Istantanea
\end{description}
\end{minipage}}\smallskip

Crei cibo e acqua in contenitori a gittata, sufficienti a sostenere fino a cinque umanoidi o 2 cavalcature per 24 ore. Il cibo è blando ma nutriente e marcisce dopo 24 ore se non viene consumato, come anche l'acqua.

\textbf{Per ogni Successo Critico Magico} ottenuto nella Prova di Magia crei cibo per altre 3 persone oppure 1 cavalcatura.

\textbf{Nota}: se sei un Seguace di Nihar il cibo è succulento e saporito.

\incantesimo{Creare Birra}
\noindent\colorbox{OBSSgold!10}{
\begin{minipage}{0.95\linewidth}
\begin{description}[noitemsep, topsep=0pt, parsep=0pt, partopsep=0pt, leftmargin=0cm, labelwidth=1.3cm]
	\item[\textbf{Lista}]: Evocazione
	\item[\textbf{Livello}]: 0, Raro
	\item[\textbf{Lancio}]: variabile
	\item[\textbf{Gittata}]: 9 metri
	\item[\textbf{Durata}]: 1 ora
\end{description}
\end{minipage}}\smallskip

Crei un boccale di birra, 0.5 litri. La qualità e tipologia di birra dipende dal lievito, malto e acqua usata.
Maggiore è il tempo di lancio dell'incantesimo più viene alta la gradazione alcolica, con un tempo di lancio di due azioni la gradazione è di 4.3, se viene impiegata 1 Azione la birra generata è analcolica, ogni Azione spesa dopo le 2 aumenta la gradazione di 0.3 vol fino ad un massimo di 12.5 vol.
Dopo un ora la birra svanisce, quando consumata dopo un ora terminano anche eventuali effetti alcolici della stessa sulle persone che l'hanno bevuta.

\textbf{Per ogni Successo Critico Magico} ottenuto nella Prova di Magia aumenti di un litro o di un ora la durata.

\incantesimo{Creare o Distruggere Acqua}
\noindent\colorbox{OBSSgold!10}{
\begin{minipage}{0.95\linewidth}
\begin{description}[noitemsep, topsep=0pt, parsep=0pt, partopsep=0pt, leftmargin=0cm, labelwidth=1.3cm]
	\item[\textbf{Lista}]: Acqua
	\item[\textbf{Livello}]: 1, Comune
	\item[\textbf{Lancio}]: 2 Azioni
	\item[\textbf{Gittata}]: 9 metri
	\item[\textbf{Durata}]: Istantanea
\end{description}
\end{minipage}}\smallskip

Crei o distruggi l'acqua.

\emph{Creare Acqua}. Crei fino a 40 litri di acqua limpida dalle tue mani che spruzzano fino a 9 metri. In alternativa l'acqua cade come pioggia in una sfera di 3 metri di raggio che si trovi entro la gittata, estinguendo le fiamme esposte nell'area.

L'incantesimo non può essere usato per spegnere fiamme magiche.

\emph{Distruggere Acqua}. Distruggi fino a 40 litri di acqua in un contenitore aperto a gittata. In alternativa, puoi distruggere la nebbia in una sfera di 4 metri di raggio entro la gittata. Usato su una elementale dell'acqua l'incantesimo causa 4d6 di danno con un Tiro Salvezza su Tempra per dimezzare.

\textbf{Per ogni Successo Critico Magico} ottenuto nella Prova di Magia crei o distruggi ulteriori 40 litri d'acqua, o le dimensioni della sfera aumentano di 1 metro di raggio in caso di nebbia.

L'acqua è potabile e disseta se bevuta entro un round dalla creazione.

\incantesimo{Creare Fossa}
\noindent\colorbox{OBSSgold!10}{
\begin{minipage}{0.95\linewidth}
\begin{description}[noitemsep, topsep=0pt, parsep=0pt, partopsep=0pt, leftmargin=0cm, labelwidth=1.3cm]
	\item[\textbf{Lista}]: Invocazione
	\item[\textbf{Livello}]: 2, Non Comune
	\item[\textbf{Lancio}]: 1 Azione
	\item[\textbf{Gittata}]: 30 metri più 3 metri per livello
	\item[\textbf{Durata}]: 1 round per CM
\end{description}
\end{minipage}}\smallskip

Crei una buca extradimensionale di 3 metri per 3 metri con una profondità di 3 metri per ogni due punti di CM fino ad un massimo di 9 metri. Devi creare la fossa su una superficie orizzontale di dimensioni sufficienti. Poiché si estende in un'altra dimensione, la fossa non ha peso e non sposta il materiale sottostante originale.

Qualsiasi creatura che si trova nell'area dove hai evocato la fossa deve effettuare un Tiro Salvezza su Riflessi per saltare in sicurezza nello spazio aperto più vicino. Inoltre, i bordi della fossa sono inclinati, e qualsiasi creatura che termina il suo turno in una casella adiacente alla fossa deve effettuare un Tiro Salvezza su Riflessi con bonus +2 per evitare di caderci dentro.

Le creature che cadono nella fossa subiscono danni da caduta normali. Le pareti rocciose lisce della fossa hanno DC 25 per Scalare. Quando la durata dell'incantesimo termina, le creature dentro la buca si alzano con il fondo della fossa fino a trovarsi sulla superficie nel corso di un singolo round.

\textbf{Per ogni Successo Critico Magico} ottenuto nella Prova di Magia raddoppi la profondità della fossa o la allarghi di 1 metro.

\incantesimo{Creare Non Morti}
\noindent\colorbox{OBSSgold!10}{
\begin{minipage}{0.95\linewidth}
\begin{description}[noitemsep, topsep=0pt, parsep=0pt, partopsep=0pt, leftmargin=0cm, labelwidth=1.3cm]
	\item[\textbf{Lista}]: Necromanzia
	\item[\textbf{Livello}]: 6, Non Comune
	\item[\textbf{Lancio}]: 2 Azioni
	\item[\textbf{Gittata}]: 3 metri
	\item[\textbf{Durata}]: Istantanea
\end{description}
\end{minipage}}\smallskip

Puoi lanciare questo incantesimo solo di notte. Scegli fino a tre cadaveri di umanoidi Medi o Piccoli a gittata. Ogni cadavere diventa un ghoul sotto il tuo controllo (il Narratore possiede le statistiche di gioco di queste creature). Durante il tuo round, con due Azioni, puoi comandare mentalmente una qualsiasi creatura da te animata con questo incantesimo, se la creatura si trova entro 36 metri da te (se controlli più creature, puoi comandarle tutte o solo una nello stesso momento impartendo lo stesso comando). Decidi tu quale azione effettuerà la creatura e dove si muoverà durante il suo prossimo round, oppure puoi impartire un comando generico, come quello di fare la guardia a una specifica stanza o corridoio. Se non impartisci comandi, le creature si limiteranno a difendersi dalle creature ostili. Una volta ricevuto un comando, la creatura continuerà a eseguirlo finché il compito sarà completo. La creatura è sotto il tuo controllo per 24 ore, dopodiché smetterà di rispondere ai comandi che gli impartisci. Per mantenere il controllo della creatura per altre 24 ore, devi lanciare questo incantesimo sulla creatura prima che l'attuale periodo di 24 ore abbia termine. Questo impiego dell'incantesimo riasserisce il tuo controllo su di un massimo di tre creature che hai animato con questo incantesimo, anziché animarne di nuove.

\textbf{Se ottieni un Critico Magico} nella Prova di Magia puoi rianimare o riasserire il controllo su quattro ghoul. Con due Critici puoi animare o riasserire il controllo su cinque ghoul o due ghast o wight. Con tre Critici puoi animare o riasserire il controllo su sei ghoul, tre ghast o wight, o due mummie.

\incantesimo{Creazione}
\noindent\colorbox{OBSSgold!10}{
\begin{minipage}{0.95\linewidth}
\begin{description}[noitemsep, topsep=0pt, parsep=0pt, partopsep=0pt, leftmargin=0cm, labelwidth=1.3cm]
	\item[\textbf{Lista}]: Illusione
	\item[\textbf{Livello}]: 5, Raro
	\item[\textbf{Lancio}]: 1 minuto
	\item[\textbf{Gittata}]: 9 metri
	\item[\textbf{Durata}]: Speciale
\end{description}
\end{minipage}}\smallskip

Afferri pezzi di materia d'ombra dal piano delle Ombre per creare, a gittata, oggetti non viventi di materia vegetale: beni morbidi, corda, legno o qualcosa di simile. Puoi usare questo incantesimo anche per creare oggetti minerali come pietra, cristallo o metallo. L'oggetto creato non può essere più grosso di un cubo di 1 metro di spigolo e l'oggetto deve essere di una forma e materiale che hai già visto in passato.

La durata dipende dal materiale dell'oggetto. Se l'oggetto è composto da più materiali, usa la durata più breve.
\medskip
Tabella Materiale - Durata
\medskip

\noindent\begin{tabularx}{\linewidth}{lX}
	\toprule
 \rowcolor{gray!20}Materia vegetale &1 giorno\\
	Pietra o cristallo &12 ore\\
 \rowcolor{gray!20}Metalli preziosi &1 ora\\
	Gemme &10 minuti\\
 \rowcolor{gray!20}Adamantio o mithral &1 minuto
\end{tabularx}
\medskip

Usare qualsiasi materiale creato da questo incantesimo come componente materiale di un altro incantesimo farà fallire il nuovo incantesimo.

\textbf{Per ogni Successo Critico Magico} ottenuto nella Prova di Magia il cubo aumenta di 1 metro di spigolo.

\incantesimo{Crescita di Spuntoni}
\noindent\colorbox{OBSSgold!10}{
\begin{minipage}{0.95\linewidth}
\begin{description}[noitemsep, topsep=0pt, parsep=0pt, partopsep=0pt, leftmargin=0cm, labelwidth=1.3cm]
	\item[\textbf{Lista}]: Animali e Piante
	\item[\textbf{Livello}]: 2, Comune
	\item[\textbf{Lancio}]: 2 Azioni
	\item[\textbf{Gittata}]: 45 metri
	\item[\textbf{Durata}]: 10 minuti
\end{description}
\end{minipage}}\smallskip

Il terreno in un raggio di 6 metri centrato su di un punto a gittata si contorce e genera spuntoni e spine molto acuminate. Per la durata, l'area diventa terreno difficile. Quando una creatura entra o si muove all'interno dell'area, subisce 2d4 danni per ogni 1 metro percorsi.
La trasformazione del terreno è talmente ben camuffata da sembrare naturale. Qualsiasi creatura che non abbia visto l'area al momento del lancio dell'incantesimo deve effettuare una prova di Consapevolezza contro la DC del Tiro Salvezza dell'incantesimo, per riconoscere il pericolo rappresentato dal terreno prima di entrarvi.

\incantesimo{Crescita Vegetale}
\noindent\colorbox{OBSSgold!10}{
\begin{minipage}{0.95\linewidth}
\begin{description}[noitemsep, topsep=0pt, parsep=0pt, partopsep=0pt, leftmargin=0cm, labelwidth=1.3cm]
	\item[\textbf{Lista}]: Animali e Piante
	\item[\textbf{Livello}]: 3, Non Comune
	\item[\textbf{Lancio}]: 2 Azioni o 8 ore
	\item[\textbf{Gittata}]: 45 metri
	\item[\textbf{Durata}]: Istantanea
\end{description}
\end{minipage}}\smallskip

Questo incantesimo incanala vitalità nei vegetali entro una specifica area. Esistono due usi possibili per questo incantesimo, che conferiscono benefici immediati o a lungo termine. Se lanci questo incantesimo impiegando 1 Azione, scegli un punto a gittata. Tutte i vegetali normali in un raggio di 30 metri centrato su quel punto diventano densi e folti. Una creatura che attraversa l'area quadruplica il costo del suo movimento.

Puoi escludere dai suoi effetti una o più aree di qualsiasi dimensione all'interno dell'area dell'incantesimo.

Se lanci questo incantesimo nel corso di 8 ore, nutri la terra. Tutti i vegetali in un raggio di 750 metri centrato su di un punto a gittata diventano super produttivi per 1 anno. I vegetali producono il doppio del normale ammontare di cibo al momento del raccolto.

\textbf{Se ottiene due Successo Critico Magico} sortisci gli effetti delle 8 ore di lancio anche se l'incantesimo è stato lanciato con 2 Azioni.

\incantesimo{CTRLC+CTRLV}
\noindent\colorbox{OBSSgold!10}{
\begin{minipage}{0.95\linewidth}
\begin{description}[noitemsep, topsep=0pt, parsep=0pt, partopsep=0pt, leftmargin=0cm, labelwidth=1.3cm]
	\item[\textbf{Lista}]: Universale
	\item[\textbf{Livello}]: 1, Molto Raro
	\item[\textbf{Lancio}]: 2 Azioni
	\item[\textbf{Gittata}]: Personale
	\item[\textbf{Durata}]: 1 minuto per CM
\end{description}
\end{minipage}}\smallskip

Questo incantesimo permette di copiare un testo da una sorgente ad ud altra. In caso di sorgente non magica questa può essere un libro, una pergamena, delle rune su una lastra od un bastone. La destinazione che va appoggiata sulla sorgente andrà a copiare i simboli nella forma e dimensione fino alla sua capienza, per un massimo di 1 pagina (di destinazione) al minuto.

Se lo scritto è un incantesimo, quindi su un Tomo o Pergamena, devono essere comunque rispettate le regole e limitazioni previste per la copia di Incantesimi sul Tomo. Questo incantesimo permette di evitare la Prova di Magia in caso di Incantesimo entro un livello superiore al massimo consentito. Copiato un incantesimo questo incantesimo termina.

\incantesimo{Cuoco Invisibile}
\noindent\colorbox{OBSSgold!10}{
\begin{minipage}{0.95\linewidth}
\begin{description}[noitemsep, topsep=0pt, parsep=0pt, partopsep=0pt, leftmargin=0cm, labelwidth=1.3cm]
	\item[\textbf{Lista}]: Evocazione
	\item[\textbf{Livello}]: 1, Comune
	\item[\textbf{Lancio}]: 2 Azioni
	\item[\textbf{Gittata}]: 18 metri
	\item[\textbf{Durata}]: 2 ore
\end{description}
\end{minipage}}\smallskip

Questo incantesimo crea una forza quasi invisibile solo delimitata da una leggera aura (di colore a tua scelta) capace e competente nel cucinare. Assieme al cuoco si manifesta anche un set di pentole e padelle nonché stoviglie ed un piccolo fornello da campo.

In base agli ingredienti a disposizione o vegetali commestibili nel raggio di 100 metri (il cuoco non va a caccia) il cuoco cucinerà al meglio degli ingredienti preparando delle ottime vivande fino a 4 persone. L'incantesimo non crea cibo o acqua, questo deve essere a disposizione al momento del lancio dell'incantesimo.

Una volta a disposizione gli ingredienti entro le due ore il cuoco invisibile preparerà da mangiare. E' possibile anche affrettare l'esecuzione ma a discapito della qualità.

Nessuna delle pentole, stoviglie o fuoco può essere usato fuorché dal cuoco invisibile.

\textbf{Se ottiene due Successo Critico Magico} il Cuoco viene evocato con cibo necessario a sfamare 2 persone

\incantesimo{Cura Ferite}
\noindent\colorbox{OBSSgold!10}{
\begin{minipage}{0.95\linewidth}
\begin{description}[noitemsep, topsep=0pt, parsep=0pt, partopsep=0pt, leftmargin=0cm, labelwidth=1.3cm]
	\item[\textbf{Lista}]: Acqua, Cura
	\item[\textbf{Livello}]: 1, Comune
	\item[\textbf{Lancio}]: 2 Azioni
	\item[\textbf{Gittata}]: Contatto
	\item[\textbf{Durata}]: Istantanea
\end{description}
\end{minipage}}\smallskip

La tua mano si riempie di energia positiva curativa, una creatura che tocchi recupera un numero di Punti Ferita uguale a 1d8 + modificatore di caratteristica per incantesimi. Questo incantesimo se usato su un non morto, Tiro per Colpire con incantesimo a tocco, lo danneggia dello stesso ammontare.

Questo incantesimo se non esplicitato diversamente non può essere usato su animali o piante.

Usando 3 Punti Magia alla formulazione dell'incantesimo curi un ammontare di Punti Ferita pari a 3d8 + 2*modificatore di caratteristica per incantesimi.

Usando 5 Punti Magia alla formulazione dell'incantesimo curi un ammontare di Punti Ferita pari a 5d8 + 3*modificatore di caratteristica per incantesimi.

Spendendo il triplo dei Punti Magia puoi curare fino a 4 creature che si trovino entro 6 metri da te.

\textbf{Per ogni Successo Critico Magico} ottenuto nella Prova di Magia curi 1d8 Punti Ferita in più.

\textbf{NOTA}: Se incantatore e creatura curata sono entrambi Seguaci dello stesso Patrono l'incantesimo cura 1d8 in più.

\textbf{NOTA}: Se incantatore e creatura curata sono entrambi Devoti dello stesso Patrono ogni valore sul dado pari a 1,2,3 sarà considerato 4.

\textbf{NOTA}: l'incantesimo quando lanciato dalla Lista Elementale dell'Acqua non può essere usato con più di 1 Punto Magia.

\incantesimo{Dardo arcano}
\noindent\colorbox{OBSSgold!10}{
\begin{minipage}{0.95\linewidth}
\begin{description}[noitemsep, topsep=0pt, parsep=0pt, partopsep=0pt, leftmargin=0cm, labelwidth=1.3cm]
	\item[\textbf{Lista}]: Universale
	\item[\textbf{Livello}]: 1, Comune
	\item[\textbf{Lancio}]: 1 Azioni
	\item[\textbf{Gittata}]: 36 metri
	\item[\textbf{Durata}]: 1 Turno, Concentrazione
\end{description}
\end{minipage}}\smallskip

Crei un dardo luminoso di forza magica. Il dardo colpisce una creatura a gittata che puoi vedere, scelta da te. Un dardo infligge 1d4 + 1 danni da forza al suo bersaglio e li puoi dirigere perché colpiscano una o più creature.

Il danno aumenta di 1 ogni due volte che hai preso Adepto della Magia fino ad un massimo di 4 aumenti.

Lanciare uno o più dardi già evocati costa 1 Azione.

Puoi creare un dardo aggiuntivo quando raggiungi CM 3, CM 5, CM 7 e CM 9, ma l'incantesimo costa un Punto Magia addizionale.

Per ogni Azione nel round dedicata al lancio dell'incantesimo oltre la prima manifesti 1 dardo in più.

\textbf{Per ogni Successo Critico Magico} ottenuto nella Prova di Magia l'incantesimo crea un dardo aggiuntivo.

\incantesimo{Dardo di Fuoco}
\noindent\colorbox{OBSSgold!10}{
\begin{minipage}{0.95\linewidth}
\begin{description}[noitemsep, topsep=0pt, parsep=0pt, partopsep=0pt, leftmargin=0cm, labelwidth=1.3cm]
	\item[\textbf{Lista}]: Fuoco
	\item[\textbf{Livello}]: 1, Comune
	\item[\textbf{Lancio}]: 1 Azione
	\item[\textbf{Gittata}]: 36 metri
	\item[\textbf{Durata}]: Istantanea
\end{description}
\end{minipage}}\smallskip

Scagli una scintilla infuocata a una creatura od oggetto a gittata. Effettua un attacco a distanza con incantesimo contro il bersaglio. Se colpisci il bersaglio subisce 1d10 danni da fuoco. Un oggetto infiammabile colpito da questo incantesimo prende fuoco, se non è indossato o trasportato.

Puoi aumentare il danno dell'incantesimo di 1d8 quando raggiungi CM 5, CM 11 e CM 17 ma costa 2 Azioni lanciarlo potenziato e 2 Punti Magia, è altresì necessario avere preso Adepto della Magia un numero di volte pari ai potenziamenti che si vogliono applicare.

\textbf{Per ogni Successo Critico Magico ottenuto} nella Prova di Magia scagli una scintilla ulteriore.

\incantesimo{Dardo occulto}
\noindent\colorbox{OBSSgold!10}{
\begin{minipage}{0.95\linewidth}
\begin{description}[noitemsep, topsep=0pt, parsep=0pt, partopsep=0pt, leftmargin=0cm, labelwidth=1.3cm]
	\item[\textbf{Lista}]: Invocazione
	\item[\textbf{Livello}]: 1, Comune
	\item[\textbf{Lancio}]: 1 Azione
	\item[\textbf{Gittata}]: 36 metri
	\item[\textbf{Durata}]: Istantanea
\end{description}
\end{minipage}}\smallskip

Un fascio di energia crepitante si dirige verso una creatura a gittata. Effettua un attacco a distanza con incantesimo contro il bersaglio. Se colpisci, il bersaglio subisce 1d8 danni da forza.

Puoi aumentare il danno dell'incantesimo di 1d8 quando raggiungi CM 5, CM 11 e CM 17 ma costa 2 Azioni lanciarlo potenziato e 2 Punti Magia, è altresì necessario avere preso Adepto della Magia un numero di volte pari ai potenziamenti che si vogliono applicare.

\textbf{Ogni Successo Critico Magico ottenuto} nella Prova di Magia crei un altro fascio di energia.

\incantesimo{Dardo Tracciante}
\noindent\colorbox{OBSSgold!10}{
\begin{minipage}{0.95\linewidth}
\begin{description}[noitemsep, topsep=0pt, parsep=0pt, partopsep=0pt, leftmargin=0cm, labelwidth=1.3cm]
	\item[\textbf{Lista}]: Invocazione
	\item[\textbf{Livello}]: 1, Non Comune
	\item[\textbf{Lancio}]: 2 Azioni
	\item[\textbf{Gittata}]: 36 metri
	\item[\textbf{Durata}]: 1 round
\end{description}
\end{minipage}}\smallskip

Un lampo di luce viaggia verso una creatura a gittata, scelta da te. Effettua un attacco a distanza con incantesimo contro il bersaglio. Se colpisci, il bersaglio subisce 2d6 danni da Luce ed il prossimo Tiro per Colpire effettuato contro di lui prima del termine del tuo prossimo round ha +1d6 al TC, grazie alla mistica luce fioca che continuerà a brillare intorno al bersaglio fino ad allora.

\textbf{Per ogni Successo Critico Magico} ottenuto nella Prova di Magia il danno aumenta di 1d6.

\incantesimo{Danza Irresistibile}
\noindent\colorbox{OBSSgold!10}{
\begin{minipage}{0.95\linewidth}
\begin{description}[noitemsep, topsep=0pt, parsep=0pt, partopsep=0pt, leftmargin=0cm, labelwidth=1.3cm]
	\item[\textbf{Lista}]: Ammaliamento
	\item[\textbf{Livello}]: 8, Leggendario
	\item[\textbf{Lancio}]: 2 Azioni
	\item[\textbf{Gittata}]: 9 metri
	\item[\textbf{Durata}]: 1 minuto
\end{description}
\end{minipage}}\smallskip

Scegli una creatura a gittata e che puoi vedere. Il bersaglio comincia un comico balletto sul posto: agitando le gambe, battendo i piedi e saltellando per la durata. Le creature che non possono essere affascinate sono immuni a questo incantesimo.

Una creatura che balla deve usare 2 Azioni di Movimento per ballare senza lasciare il suo spazio e ha -1d6 ai Tiri Salvezza su Riflessi e i Tiri per Colpire. Mentre il bersaglio è soggetto a questo incantesimo le altre creature hanno +1d6 ai Tiri per Colpire contro di lui. Spendendo 1 Azione la creatura che balla può effettuare un nuovo Tiro Salvezza su Volontà per recuperare il controllo di se stessa. Se lo supera, l'incantesimo ha fine. Mentre balla si considera Distratto.

\textbf{Se ottieni 2 Successo Critico Magico} la durata aumenta di 1 ora

\incantesimo{Desiderio}
\noindent\colorbox{OBSSgold!10}{
\begin{minipage}{0.95\linewidth}
\begin{description}[noitemsep, topsep=0pt, parsep=0pt, partopsep=0pt, leftmargin=0cm, labelwidth=1.3cm]
	\item[\textbf{Lista}]: Evocazione
	\item[\textbf{Livello}]: 9, Leggendario
	\item[\textbf{Lancio}]: 2 Azioni
	\item[\textbf{Gittata}]: Personale
	\item[\textbf{Durata}]: Istantanea
\end{description}
\end{minipage}}\smallskip

Desiderio è il più potente incantesimo che una creatura mortale possa lanciare. Semplicemente parlando ad alta voce e consumando le gemme tenute in mano, puoi modificare le stesse fondamenta della realtà a seconda dei tuoi bisogni.

L'uso basilare di questo incantesimo è quello di riprodurre l'effetto di qualsiasi altro incantesimo con livello 8 o meno. Non devi soddisfare nessuno dei requisiti dell'incantesimo, comprese le componenti materiali costose. L'incantesimo ha semplicemente effetto.

In alternativa, puoi creare uno dei seguenti effetti a tua scelta:

\begin{itemize}[leftmargin=*] \setlength{\itemsep}{0pt}
	\item Crei un oggetto del valore massimo di 25000 mo, che non sia un oggetto magico. L'oggetto non può avere dimensioni superiori ai 90 metri in qualsiasi dimensione, e compare in uno spazio non occupato sul terreno.
	\item Permetti fino a venti creature che puoi vedere di recuperare tutti i Punti Ferita, e termini tutti gli effetti su di loro descritti dall'incantesimo ristorare superiore.
	\item Conferisci a un massimo di dieci creature che puoi vedere la resistenza a un tipo di danno a tua scelta per 8 ore.
	\item Conferisci a un massimo di dieci creature che puoi vedere l'immunità a un singolo incantesimo o altro effetto magico per 8 ore. Per esempio, potresti rendere te e tutti tuoi compagni immuni all'attacco risucchia vita del lich.
	\item Annulli un qualsiasi evento recente obbligando a ritirare qualsiasi tiro effettuato nell'ultimo round (compreso il tuo ultimo round). La realtà si rimodella per assecondare il nuovo risultato. Puoi far sì che il nuovo tiro abbia +2d6 o -2d6, puoi scegliere se usare il tiro originale o il nuovo tiro. Potresti anche riuscire a ottenere altro, oltre gli obiettivi negli esempi di cui sopra.
\end{itemize}

\medskip
Definisci i tuoi desideri quanto più possibile al Narratore. Il Narratore ha grande spazio di manovra nel decidere cosa accada in questi casi; maggiore il desiderio, più grosse le probabilità che qualcosa vada storto. L'incantesimo potrebbe semplicemente fallire, l'effetto desiderato manifestarsi solo in parte, oppure potresti subire delle conseguenze impreviste, in base a come hai proferito il desiderio. Lo stress del lanciare questo incantesimo per creare qualsiasi effetto che non sia riprodurre un altro incantesimo ti indebolisce.

Dopo averne retto lo stress, ogni volta che lancerai un incantesimo, fino a che non avrai terminato una notte di riposo, subirai 1d10 danni da Vuoto per livello/2 dell'incantesimo lanciato. Questo danno non può essere ridotto o diminuito in alcun modo. Inoltre, la tua Costituzione scende a -3, se non è già a -3 o meno, per 2d4 giorni.

Per ciascun giorno che trascorri a riposare e non svolgere altro che un'attività leggera, il tuo tempo di recupero rimanente diminuisce di 2 giorni.

Tira 1d100, se fai da 1 a 33\% tu non sarai mai più in grado di lanciare desiderio a causa dello stress sofferto, 34\%-66\% invecchi di 5 anni, 67\%-99\% non succede alcun altro effetto particolare, 100\% recuperi immediatamente lo stress del lancio.

\textbf{In caso di 2 Successi Critici magico ottenuti} non subisci effetti collaterali dal lancio di Desiderio.


\incantesimo{Desiderio limitato}
\noindent\colorbox{OBSSgold!10}{
\begin{minipage}{0.95\linewidth}
\begin{description}[noitemsep, topsep=0pt, parsep=0pt, partopsep=0pt, leftmargin=0cm, labelwidth=1.3cm]
	\item[\textbf{Lista}]: Evocazione
	\item[\textbf{Livello}]: 7, Molto Raro
	\item[\textbf{Lancio}]: 2 Azioni
	\item[\textbf{Gittata}]: Personale
	\item[\textbf{Durata}]: Istantanea
\end{description}
\end{minipage}}\smallskip

\emph{Desiderio limitato} è un incantesimo estremamente potente e versatile, che permette all'incantatore di realizzare "quasi qualsiasi cosa".

L'uso basilare di questo incantesimo è quello di riprodurre l'effetto di qualsiasi altro incantesimo con livello 7 o meno. Non devi soddisfare nessuno dei requisiti dell'incantesimo, comprese le componenti materiali costose. L'incantesimo ha semplicemente effetto.

In alternativa, puoi creare uno dei seguenti effetti a tua scelta:

\begin{itemize}[leftmargin=*] \setlength{\itemsep}{0pt}
	\item Crei un oggetto del valore massimo di 3000 mo, che non sia un oggetto magico. L'oggetto non può avere dimensioni superiori ai 90 metri in qualsiasi dimensione, e compare in uno spazio non occupato sul terreno.
	\item Permetti fino a 5 creature che puoi vedere di recuperare tutti i Punti Ferita, e termini tutti gli effetti su di loro descritti dall'incantesimo ristorare superiore.
	\item Conferisci a un massimo di 5 creature che puoi vedere la resistenza a un tipo di danno a tua scelta per 8 ore.
	\item Conferisci a un massimo di 4 creature che puoi vedere l'immunità a un singolo incantesimo o altro effetto magico per 8 ore. Per esempio, potresti rendere te e tutti tuoi compagni immuni all'attacco risucchia vita del lich.
\end{itemize}

\medskip
Definisci i tuoi desideri quanto più possibile al Narratore. Il Narratore ha grande spazio di manovra nel decidere cosa accada in questi casi; maggiore il desiderio, più grosse le probabilità che qualcosa vada storto. L'incantesimo potrebbe semplicemente fallire, l'effetto desiderato manifestarsi solo in parte, oppure potresti subire delle conseguenze impreviste, in base a come hai proferito il desiderio.


\incantesimo{Destriero Fantasma}
\noindent\colorbox{OBSSgold!10}{
\begin{minipage}{0.95\linewidth}
\begin{description}[noitemsep, topsep=0pt, parsep=0pt, partopsep=0pt, leftmargin=0cm, labelwidth=1.3cm]
	\item[\textbf{Lista}]: Illusione
	\item[\textbf{Livello}]: 3, Comune
	\item[\textbf{Lancio}]: 1 minuto
	\item[\textbf{Gittata}]: 9 metri
	\item[\textbf{Durata}]: 1 ora
\end{description}
\end{minipage}}\smallskip

Una creatura quasi reale simile a un saurovallo di taglia Grande, appare sul terreno in uno spazio non occupato di tua scelta e a gittata. Decidi tu l'aspetto della creatura, e questa compare equipaggiata di sella, morso e briglia. Qualsiasi equipaggiamento creato dall'incantesimo svanisce in una nuvola di fumo se viene portato a più di 3 metri di distanza dal destriero. Per la durata, tu o una creatura di tua scelta potete cavalcare il destriero. La creatura usa le statistiche del Saurovallo da Galoppo, eccetto che ha velocità 30 metri e può percorrere 15 chilometri in un'ora, o 20 chilometri ad andatura veloce. Quando l'incantesimo termina, il destriero inizia gradualmente a svanire, dando al fantino 1 minuto per smontare di sella. L'incantesimo termina se usi un'Azione per interromperlo o se il destriero subisce danni.

\textbf{Per ogni Successo Critico Magico} ottenuto nella Prova di Magia la durata aumenta di un ora oppure crei una cavalcatura in più.

\incantesimo{Disco Fluttuante}
\noindent\colorbox{OBSSgold!10}{
\begin{minipage}{0.95\linewidth}
\begin{description}[noitemsep, topsep=0pt, parsep=0pt, partopsep=0pt, leftmargin=0cm, labelwidth=1.3cm]
	\item[\textbf{Lista}]: Evocazione
	\item[\textbf{Livello}]: 1, Comune
	\item[\textbf{Lancio}]: 2 Azioni
	\item[\textbf{Gittata}]: 9 metri
	\item[\textbf{Durata}]: 2 ora
\end{description}
\end{minipage}}\smallskip

Questo incantesimo crea un piano di forza orizzontale leggermente concavo, perfettamente circolare, di 1 metro di diametro e 2,5 centimetri di spessore che fluttua a 1 metro da terra, in uno spazio non occupato di tua scelta a gittata e che puoi vedere. Il disco rimane attivo per la durata, e può sostenere 250 chili o 50 di Ingombro. Se gli viene poggiato sopra un peso superiore, l'incantesimo termina e tutto quello che vi si trova sopra cade a terra. Finché ti trovi entro 6 metri da esso, il disco è immobile. Se ti muovi più di 6 metri lontano da esso, il disco ti segue in modo da rimanere sempre a 6 metri da te. Può muoversi attraverso terreno disomogeneo, su e giù per le scale, pendenze e simili, ma non può superare cambi di altitudine di 3 o più metri. Per esempio, il disco non può attraversare un fossato profondo 3 metri, né potrebbe lasciare il fossato se fosse creato in fondo a esso. Il disco può essere afferrato dall'incantatore e spostato manualmente. Se ti allontani più di 30 metri dal disco (di solito perché non riesce ad aggirare un ostacolo nel seguirti) l'incantesimo termina.

\textbf{Per ogni Successo Critico Magico} ottenuto nella Prova di Magia la durata aumenta di 2 ore.

\incantesimo{Disintegrazione}
\noindent\colorbox{OBSSgold!10}{
\begin{minipage}{0.95\linewidth}
\begin{description}[noitemsep, topsep=0pt, parsep=0pt, partopsep=0pt, leftmargin=0cm, labelwidth=1.3cm]
	\item[\textbf{Lista}]: Trasmutazione
	\item[\textbf{Livello}]: 6, Non Comune
	\item[\textbf{Lancio}]: 2 Azioni
	\item[\textbf{Gittata}]: 18 metri
	\item[\textbf{Durata}]: Istantanea
\end{description}
\end{minipage}}\smallskip

Un sottile raggio verde parte dal tuo dito puntato verso un bersaglio a gittata e che puoi vedere. Il bersaglio può essere una creatura, un oggetto o una creazione di forza magica, come un muro creato da muro di forza. Una creatura bersaglio di questo incantesimo deve effettuare un Tiro Salvezza su Tempra. Il bersaglio subisce 10d6 + 40 danni da forza se fallisce il Tiro Salvezza, la metà del danno se riesce. Se questo danno riduce il bersaglio a 0 Punti Ferita, questi è disintegrato. Una creatura disintegrata e tutto quello che indossa e trasporta, eccetto gli oggetti magici, viene ridotta a un mucchietto di sottile polvere grigia. La creatura può essere riportata in vita solo tramite l'intervento di un Patrono.

Questo incantesimo disintegra automaticamente gli oggetti non magici o una creazione di forza magica di taglia Grande o più piccola. Se il bersaglio è un oggetto non magico o una creazione di forza di taglia Enorme o più grossa, questo incantesimo disintegra una porzione di essa pari a una sfera di 1 metro di raggio.

\textbf{Per ogni Successo Critico Magico} ottenuto nella Prova di Magia danno aumenta di 5d6.

\textbf{Tiro Salvezza Successo/Fallimento Critico}: In caso di Fallimento Critico il danno raddoppia, in caso di Successo Critico il danno viene ulteriormente dimezzato


\incantesimo{Dissolvi Magie}\hypertarget{dissolvimagie}{}
\noindent
\begin{description}[noitemsep, topsep=0pt, parsep=0pt, partopsep=0pt, leftmargin=0cm, labelwidth=1.3cm]
	\item[\textbf{Lista}]: Abiurazione
	\item[\textbf{Livello}]: 3, Comune
	\item[\textbf{Lancio}]: 2 Azioni
	\item[\textbf{Gittata}]: 36 metri
	\item[\textbf{Durata}]: Istantanea
\end{description}

Scegli una creatura, oggetto o effetto magico a gittata. Qualsiasi incantesimo di livello 2 o più basso sul bersaglio ha fine.

Se l'incantesimo è tra il 3 ed il 5 livello è necessaria una prova di \hyperlink{contrastareincantesimi}{contrastare incantesimi} (pag. \pageref{contrastareincantesimi}).

Un effetto magico permanente viene soppresso temporaneamente per 10 minuti.


\incantesimo{Dissolvi Magie Avanzato}\hypertarget{dissolvimagieavanzato}{}
\noindent
\begin{description}[noitemsep, topsep=0pt, parsep=0pt, partopsep=0pt, leftmargin=0cm, labelwidth=1.3cm]
	\item[\textbf{Lista}]: Abiurazione
	\item[\textbf{Livello}]: 5, Raro
	\item[\textbf{Lancio}]: 3 Azioni
	\item[\textbf{Gittata}]: 36 metri
	\item[\textbf{Durata}]: Istantanea
\end{description}

Scegli una creatura, oggetto o effetto magico a gittata. Qualsiasi incantesimo di livello 4 o più basso sul bersaglio ha fine.

Se l'incantesimo è di livello maggiore al quarto è necessaria una prova di \hyperlink{contrastareincantesimi}{contrastare incantesimi} (pag. \pageref{contrastareincantesimi}).

Un effetto magico permanente viene soppresso temporaneamente per 10 minuti.

\textbf{Nota}: se si ottengono 3 Successi Critici Magici disperde permanentemente un effetto su un oggetto non artefatto.

\incantesimo{Distruggere nonmorto}
\noindent\colorbox{OBSSgold!10}{
\begin{minipage}{0.95\linewidth}
\begin{description}[noitemsep, topsep=0pt, parsep=0pt, partopsep=0pt, leftmargin=0cm, labelwidth=1.3cm]
	\item[\textbf{Lista}]: Cura
	\item[\textbf{Livello}]: 3, Non Comune
	\item[\textbf{Lancio}]: 2 Azioni
	\item[\textbf{Gittata}]: 36 metri
	\item[\textbf{Durata}]: Istantanea
\end{description}
\end{minipage}}\smallskip

Scegli un nonmorto entro 36 metri. Un raggio luminoso si propaga dalla tua mano avvolgendo la creatura. Il nonmorto effettua un Tiro Salvezza su Tempra per dimezzare 4d12 danni da energia positiva.

\textbf{Per ogni Successo Critico Magico} ottenuto nella Prova di Magia il danno aumenta di 2d12.

\textbf{Tiro Salvezza Successo/Fallimento Critico}: In caso di Fallimento Critico il danno raddoppia, in caso di Successo Critico il danno viene ulteriormente dimezzato

\incantesimo{Dito}
\noindent\colorbox{OBSSgold!10}{
\begin{minipage}{0.95\linewidth}
\begin{description}[noitemsep, topsep=0pt, parsep=0pt, partopsep=0pt, leftmargin=0cm, labelwidth=1.3cm]
	\item[\textbf{Lista}]: Ammaliamento
	\item[\textbf{Livello}]: 0, Raro
	\item[\textbf{Lancio}]: 1 Azione
	\item[\textbf{Gittata}]: 18 metri
	\item[\textbf{Durata}]: 3 round
\end{description}
\end{minipage}}\smallskip

Fai il dito (o pernacchia o gesto dell'ombrello) all'avversario che deve poterlo vedere.

Questo deve fare un Tiro Salvezza su Volontà, se riesce non succede nulla.

Se fallisce il Tiro Salvezza in maniera critica, per i prossimi 2 round ha una penalità di 2 ai Tiri per Colpire, TS ed alle prove di Competenza di Base.

Se fallisce il TS di 3 o 4, viene mortificato, fino alla fine del prossimo round ha una penalità di 1 ai Tiri per Colpire e Competenza.

Se fallisce il TS di 2 o 1, è punito, fino alla fine del prossimo round ha una penalità di 1 ai Tiri per Colpire o Difesa (scelta dell'obiettivo).

\textbf{Per ogni Successo Critico Magico} ottenuto nella Prova di Magia puoi influenzare una altra creatura che possa vedere il gesto.

\incantesimo{Dito della Morte}
\noindent\colorbox{OBSSgold!10}{
\begin{minipage}{0.95\linewidth}
\begin{description}[noitemsep, topsep=0pt, parsep=0pt, partopsep=0pt, leftmargin=0cm, labelwidth=1.3cm]
	\item[\textbf{Lista}]: Necromanzia
	\item[\textbf{Livello}]: 6, Raro
	\item[\textbf{Lancio}]: 2 Azioni
	\item[\textbf{Gittata}]: 18 metri
	\item[\textbf{Durata}]: Istantanea
\end{description}
\end{minipage}}\smallskip

Invii una scarica di energia negativa a una creatura a gittata e che puoi vedere, provocandole profondo dolore. Il bersaglio deve effettuare un Tiro Salvezza su Tempra. Il bersaglio subisce 7d8 + 30 danni da Vuoto se fallisce il Tiro Salvezza, o la metà di questi danni se lo supera.

Un umanoide ucciso da questo incantesimo si rianima come zombi sotto il tuo comando permanente all'inizio del tuo prossimo round, e seguirà i tuoi ordini verbali al meglio delle sue capacità.

\textbf{Per ogni Successo Critico Magico} ottenuto nella Prova di Magia il danno aumenta di 4d8.

\textbf{Tiro Salvezza Successo/Fallimento Critico}: In caso di Fallimento Critico il danno raddoppia, in caso di Successo Critico il danno viene ulteriormente dimezzato

\incantesimo{Divinazione}
\noindent\colorbox{OBSSgold!10}{
\begin{minipage}{0.95\linewidth}
\begin{description}[noitemsep, topsep=0pt, parsep=0pt, partopsep=0pt, leftmargin=0cm, labelwidth=1.3cm]
	\item[\textbf{Lista}]: Divinazione
	\item[\textbf{Livello}]: 6, Raro
	\item[\textbf{Lancio}]: 2 Azioni
	\item[\textbf{Gittata}]: Personale
	\item[\textbf{Durata}]: Istantanea
\end{description}
\end{minipage}}\smallskip

La tua magia e un'offerta votiva ti mettono in comunicazione con un Patrono o il servitore di un Patrono. Gli puoi porre una singola domanda in merito a uno specifico obiettivo, evento o attività che debba verificarsi entro 7 giorni. Il Narratore dà una risposta veritiera. La replica potrebbe essere una breve frase, una rima criptica o un presagio.

L'incantesimo non tiene conto di ogni possibile circostanza che possa modificare il risultato, come il lancio di ulteriori incantesimi o la perdita o l'arrivo di un alleato.

Se lanci l'incantesimo due o più volte prima di aver riposato almeno 8 ore, c'è una probabilità cumulativa del 25\% che per ogni lancio dopo il primo tu ottenga una lettura erronea. Il Narratore effettua questo tiro in segreto.

\textbf{NOTA}: l'incantesimo deve essere formulato da almeno un Seguace

\incantesimo{Dominare Bestie}
\noindent\colorbox{OBSSgold!10}{
\begin{minipage}{0.95\linewidth}
\begin{description}[noitemsep, topsep=0pt, parsep=0pt, partopsep=0pt, leftmargin=0cm, labelwidth=1.3cm]
	\item[\textbf{Lista}]: Ammaliamento, Animali e Piante
	\item[\textbf{Livello}]: 4, Molto Raro - Comune
	\item[\textbf{Lancio}]: 2 Azioni
	\item[\textbf{Gittata}]: 18 metri
	\item[\textbf{Durata}]: Concentrazione, massimo 1 minuto
\end{description}
\end{minipage}}\smallskip

Cerchi di affascinare una bestia a gittata che puoi vedere. Essa deve superare un Tiro Salvezza su Volontà o restare affascinata per la durata, ricevendo +1d6 al tiro se tu o i tuoi alleati la state combattendo.

Mentre la bestia è affascinata, finché voi due vi trovate sullo stesso piano di esistenza mantieni un collegamento telepatico con essa. Puoi usare questo collegamento telepatico per inviare comandi alla creatura mentre sei cosciente (richiede 1 Azione), a cui essa obbedirà al suo meglio. Puoi specificare un corso d'azione semplice e generico, come \emph{Attacca quella creatura}, \emph{Corri laggiù}, o \emph{Prendi quell'oggetto}. Se la creatura completa l'ordine e non riceve ulteriori indicazioni da te, si difenderà e preserverà al meglio delle sue capacità.

Puoi impiegare 2 tue azioni per assumere il totale e preciso controllo del bersaglio. Fino al termine del tuo prossimo round, il bersaglio effettuerà solo le azioni decise da te, e non farà nulla che tu non gli permetta di fare. Durante questo periodo, puoi anche far usare una Azione di Reazione al bersaglio, ma ciò richiede l'uso della tua Reazione.

Ogni volta che il bersaglio subisce danni, effettua un nuovo Tiro Salvezza su Volontà contro l'incantesimo. Se supera il Tiro Salvezza, l'incantesimo termina. La bestia non può avere GS superiore a 4.

\textbf{Per ogni Successo Critico Magico} ottenuto nella Prova di Magia la durata raddoppia fino ad un massimo di 8 ore. Ogni 2 Successi Magici Critici puoi comandare una bestia in più oppure aumenti il GS comandabile di 1.

\incantesimo{Dominare Mostri}
\noindent\colorbox{OBSSgold!10}{
\begin{minipage}{0.95\linewidth}
\begin{description}[noitemsep, topsep=0pt, parsep=0pt, partopsep=0pt, leftmargin=0cm, labelwidth=1.3cm]
	\item[\textbf{Lista}]: Ammaliamento
	\item[\textbf{Livello}]: 8, Non Comune
	\item[\textbf{Lancio}]: 2 Azioni
	\item[\textbf{Gittata}]: 18 metri
	\item[\textbf{Durata}]: Concentrazione, massimo 1 ora
\end{description}
\end{minipage}}\smallskip

Cerchi di affascinare una creatura a gittata che puoi vedere. Essa deve superare un Tiro Salvezza su Volontà o restare affascinata per la durata, ricevendo +1d6 al tiro se tu o i tuoi alleati la state combattendo.

Mentre la creatura è affascinata, finché voi due vi trovate sullo stesso piano di esistenza mantieni un collegamento telepatico con essa. Puoi usare questo collegamento telepatico per inviare comandi alla creatura mentre sei cosciente (richiede 1 Azione), a cui essa obbedirà al suo meglio. Puoi specificare un corso d'azione semplice e generico, come \emph{Attacca quella creatura}, \emph{Corri laggiù}, o \emph{Prendi quell'oggetto}. Se la creatura completa l'ordine e non riceve ulteriori indicazioni da te, si difenderà e preserverà al meglio delle sue capacità.

Puoi impiegare due tua Azioni per assumere il totale e preciso controllo del bersaglio. Fino al termine del tuo prossimo round la creatura effettuerà solo le azioni decise da te, e non farà nulla che tu non le permetta di fare. Durante questo periodo, puoi anche far usare una Azione di Reazione alla creatura, ma ciò richiede l'uso della tua Reazione. Ogni volta che il bersaglio subisce danni, effettua un nuovo Tiro Salvezza su Volontà contro l'incantesimo. Se supera il Tiro Salvezza, l'incantesimo termina.

\textbf{Per ogni Successo Critico Magico} ottenuto nella Prova di Magia la durata raddoppia fino ad un massimo di 8 ore.

\incantesimo{Dominare Persone}
\noindent\colorbox{OBSSgold!10}{
\begin{minipage}{0.95\linewidth}
\begin{description}[noitemsep, topsep=0pt, parsep=0pt, partopsep=0pt, leftmargin=0cm, labelwidth=1.3cm]
	\item[\textbf{Lista}]: Ammaliamento
	\item[\textbf{Livello}]: 5, Non Comune
	\item[\textbf{Lancio}]: 2 Azioni
	\item[\textbf{Gittata}]: 18 metri
	\item[\textbf{Durata}]: Concentrazione, massimo 1 minuto
\end{description}
\end{minipage}}\smallskip

Cerchi di affascinare un umanoide a gittata che puoi vedere. Esso deve superare un Tiro Salvezza su Volontà o restare Affascinato per la durata, ricevendo +1d6 al tiro se tu o i tuoi alleati lo state combattendo.

Mentre il bersaglio è affascinato, finché voi due vi trovate sullo stesso piano di esistenza mantieni un collegamento telepatico con esso. Puoi usare questo collegamento telepatico per inviare comandi al bersaglio mentre sei cosciente (richiede 1 Azione), a cui esso obbedirà al suo meglio. Puoi specificare un corso d'azione semplice e generico, come \emph{Attacca quella creatura}, \emph{Corri laggiù}, o \emph{Prendi quell'oggetto}. Se il bersaglio completa l'ordine e non riceve ulteriori indicazioni da te, si difenderà al meglio delle sue capacità.

Puoi impiegare 2 Azioni per assumere il totale e preciso controllo del bersaglio. Fino al termine del tuo prossimo round, il bersaglio effettuerà solo le azioni decise da te, e non farà nulla che tu non gli permetta di fare. Durante questo periodo, puoi anche far usare una Azione di Reazione al bersaglio, ma ciò richiede l'uso della tua Reazione. Ogni volta che il bersaglio subisce danni effettua un nuovo Tiro Salvezza su Volontà contro l'incantesimo. Se supera il Tiro Salvezza l'incantesimo termina.

\textbf{Per ogni Successo Critico Magico} ottenuto nella Prova di Magia la durata raddoppia fino ad un massimo di 8 ore.

\incantesimo{Eroismo}
\noindent\colorbox{OBSSgold!10}{
\begin{minipage}{0.95\linewidth}
\begin{description}[noitemsep, topsep=0pt, parsep=0pt, partopsep=0pt, leftmargin=0cm, labelwidth=1.3cm]
	\item[\textbf{Lista}]: Ammaliamento
	\item[\textbf{Livello}]: 1, Non Comune
	\item[\textbf{Lancio}]: 2 Azioni
	\item[\textbf{Gittata}]: Contatto
	\item[\textbf{Durata}]: 1 minuto
\end{description}
\end{minipage}}\smallskip

Una creatura consenziente con cui sei in contatto vene infusa di coraggio. Fino al termine dell'incantesimo, la creatura è immune all'essere spaventata e all'inizio di ciascun suo round ottiene Punti Ferita temporanei pari al tuo valore di modificatore da incantesimo. Quando l'incantesimo ha termine il bersaglio perde tutti i Punti Ferita temporanei rimasti.

\textbf{Per ogni Successo Critico Magico} ottenuto nella Prova di Magia puoi influenzare un altra creatura.

\incantesimo{Esilio}
\noindent\colorbox{OBSSgold!10}{
\begin{minipage}{0.95\linewidth}
\begin{description}[noitemsep, topsep=0pt, parsep=0pt, partopsep=0pt, leftmargin=0cm, labelwidth=1.3cm]
	\item[\textbf{Lista}]: Abiurazione
	\item[\textbf{Livello}]: 4, Comune
	\item[\textbf{Lancio}]: 2 Azioni
	\item[\textbf{Gittata}]: 18 metri
	\item[\textbf{Durata}]: 1 minuto
\end{description}
\end{minipage}}\smallskip

Cerchi di spedire una creatura a gittata e che puoi vedere in un altro piano di esistenza. Il bersaglio deve superare un Tiro Salvezza su Volontà o venire esiliato. Se il bersaglio è natio del piano di esistenza in cui ti trovi, esili il bersaglio in un semipiano innocuo. Mentre è lì, il bersaglio è inabile. Il bersaglio rimane lì fino al termine dell'incantesimo, quando riapparirà nello spazio che aveva lasciato o nello spazio non occupato più vicino, se il suo spazio originale adesso è occupato. Se il bersaglio è natio di un diverso piano di esistenza da quello in cui ti trovi, il bersaglio svanisce emettendo un lieve scoppio, tornando al suo piano natio. Se l'incantesimo termina prima che sia trascorso 1 minuto, il bersaglio riappare nello spazio che aveva lasciato o nello spazio non occupato più vicino, se il suo spazio originale è occupato.

\textbf{Per ogni Successo Critico Magico} ottenuto nella Prova di Magia puoi influenzare un altra creatura, oppure la creatura è bandita per una settimana.

\incantesimo{Esplosione Solare}
\noindent\colorbox{OBSSgold!10}{
\begin{minipage}{0.95\linewidth}
\begin{description}[noitemsep, topsep=0pt, parsep=0pt, partopsep=0pt, leftmargin=0cm, labelwidth=1.3cm]
	\item[\textbf{Lista}]: Invocazione
	\item[\textbf{Livello}]: 8, Raro
	\item[\textbf{Lancio}]: 2 Azioni
	\item[\textbf{Gittata}]: 45 metri
	\item[\textbf{Durata}]: Istantanea
\end{description}
\end{minipage}}\smallskip

Un'intensa luce solare illumina in un raggio di 18 metri centrato su di un punto a gittata, scelto da te. Tutte le creature all'interno della luce devono effettuare un Tiro Salvezza su Tempra. Se fallisce il Tiro Salvezza, una creatura subisce 12d6 danni da Luce e resta accecata per 1 minuto. Se lo supera, subisce la metà dei danni e non resta accecata dall'incantesimo. Non morti e melme hanno -8 a questo Tiro Salvezza. Una creatura accecata da questo incantesimo effettua un altro Tiro Salvezza su Tempra alla fine di ciascun suo round. Se supera il Tiro Salvezza, non è più accecata.

Nella sua area, questo incantesimo dissolve qualsiasi oscurità generata da un incantesimo.

\textbf{Per ogni Successo Critico Magico} ottenuto nella Prova di Magia il danno aumenta di 6d6.

\textbf{NOTA}: un Devoto di Ljust o Sumkjr ottiene un Successo Critico Magico automatico

\incantesimo{Estasiare}
\noindent\colorbox{OBSSgold!10}{
\begin{minipage}{0.95\linewidth}
\begin{description}[noitemsep, topsep=0pt, parsep=0pt, partopsep=0pt, leftmargin=0cm, labelwidth=1.3cm]
	\item[\textbf{Lista}]: Ammaliamento
	\item[\textbf{Livello}]: 2, Comune
	\item[\textbf{Lancio}]: 2 Azioni
	\item[\textbf{Gittata}]: Personale
	\item[\textbf{Durata}]: 1 minuto
\end{description}
\end{minipage}}\smallskip

Intessi una serie di parole svianti, facendo sì che delle creature di tua scelta entro la gittata, che puoi vedere e possano sentirti, effettuino un Tiro Salvezza su Volontà. Qualsiasi creatura che non può restare affascinata supera il Tiro Salvezza automaticamente, e se tu o i tuoi compagni state combattendo una creatura, questa ha +1d6 al Tiro Salvezza. Se fallisce il Tiro Salvezza, il bersaglio ha -1d6 sulle prove di Consapevolezza effettuate per percepire una qualsiasi creatura diversa da te fino al termine dell'incantesimo o finché il bersaglio non può più sentirti.

L'incantesimo termina se sei reso inabile o non puoi più parlare.

\incantesimo{Evoca Animali}
\noindent\colorbox{OBSSgold!10}{
\begin{minipage}{0.95\linewidth}
\begin{description}[noitemsep, topsep=0pt, parsep=0pt, partopsep=0pt, leftmargin=0cm, labelwidth=1.3cm]
	\item[\textbf{Lista}]: Animali e Piante
	\item[\textbf{Livello}]: 3, Non Comune
	\item[\textbf{Lancio}]: 3 Azioni
	\item[\textbf{Gittata}]: 18 metri
	\item[\textbf{Durata}]: 1 Turno
\end{description}
\end{minipage}}\smallskip

Evochi spiriti magici che assumono l'aspetto di bestie e compaiono in spazi non occupati a gittata e che puoi vedere. Scegli una delle seguenti opzioni per determinare ciò che appare:

\begin{itemize}[leftmargin=*] \setlength{\itemsep}{-1pt}
	\item Una bestia di grado di sfida 2 o inferiore
	\item Due bestie di grado di sfida 1 o inferiore
	\item Quattro bestie di grado di sfida 1/2 o inferiore
	\item Otto bestie di grado di sfida 1/4 o inferiore
\end{itemize}

Ogni bestia è considerata anche magica e sparisce quando scende a 0 Punti Ferita o quando l'incantesimo termina.

Le creature evocate sono amichevoli verso di te e i tuoi compagni e ubbidiscono al meglio delle loro capacità.

\textbf{Per ogni Successo Critico Magico} ottenuto nella Prova di Magia appariranno due bestie in più di grado inferiore o 1 bestia in più di grado superiore a quello inizialmente scelto.

\incantesimo{Evoca Cavalcatura}
\noindent\colorbox{OBSSgold!10}{
\begin{minipage}{0.95\linewidth}
\begin{description}[noitemsep, topsep=0pt, parsep=0pt, partopsep=0pt, leftmargin=0cm, labelwidth=1.3cm]
	\item[\textbf{Lista}]: Animali e Piante
	\item[\textbf{Livello}]: 2, Comune
	\item[\textbf{Lancio}]: 10 minuti
	\item[\textbf{Gittata}]: 9 metri
	\item[\textbf{Durata}]: 1 ora
\end{description}
\end{minipage}}\smallskip

Evochi uno spirito che assume la forma di una cavalcatura insolitamente intelligente, forte e leale, stabilendo un legame duraturo con esso. Apparendo in uno spazio a gittata, non occupato, il destriero assume la forma di tua scelta, come quella di un saurovallo da guerra, un saurovallo nano, un cammello, un alce o un mastino (il Narratore potrebbe darti la possibilità di evocare come destrieri anche altri tipi di animali). Il destriero ha le statistiche della forma scelta, sebbene sia di tipo celestiale, fatato o demone (a tua scelta) invece del suo normale tipo. Inoltre, se il tuo destriero ha Intelligenza -3 o meno, la sua Intelligenza diventa -2, e ottiene la capacità di comprendere un linguaggio a tua scelta tra quelli che sei in grado di parlare. Il tuo destriero serve da cavalcatura, sia in combattimento che fuori da esso, e possiedi un legame istintivo con esso, che vi permette di combattere come foste un unico insieme.

Quando il destriero scende a 0 Punti Ferita, scompare, non lasciandosi dietro alcuna forma fisica. puoi congedare il destriero in qualsiasi momento con un'Azione, facendolo sparire. In entrambi i casi, lanciare di nuovo questo incantesimo evoca lo stesso destriero, ripristinato al massimo dei suoi Punti Ferita.

Non puoi avere più di un destriero legato da questo incantesimo alla volta. Con un'Azione, puoi liberare il destriero da questo legame in qualsiasi momento, facendolo sparire.

\textbf{Per ogni Successo Critico Magico} ottenuto nella Prova di Magia l'incantesimo dura 2 ore in più.

\incantesimo{Evoca Elementale}
\noindent\colorbox{OBSSgold!10}{
\begin{minipage}{0.95\linewidth}
\begin{description}[noitemsep, topsep=0pt, parsep=0pt, partopsep=0pt, leftmargin=0cm, labelwidth=1.3cm]
	\item[\textbf{Lista}]: Aria, Acqua, Terra, Fuoco
	\item[\textbf{Livello}]: 5, Raro
	\item[\textbf{Lancio}]: 1 minuto
	\item[\textbf{Gittata}]: 27 metri
	\item[\textbf{Durata}]: Concentrazione, 1 Turno
\end{description}
\end{minipage}}\smallskip

Evochi un servitore elementale. Scegli un'area a gittata composta di acqua, aria, fuoco o terra e che riempia una sfera di 1 metro di raggio. Un elementale di grado di sfida 3 o minore appropriato all'area da te scelta compare in uno spazio non occupato entro 3 metri da essa. L'elementale sparisce quando scende a 0 Punti Ferita o l'incantesimo termina.

Ogni Lista di Magia può evocare solo il proprio Elementale specifico

\textbf{Ogni due Successo Critico Magico ottenuto} nella Prova di Magia il grado di sfida dell'elementale evocato aumenta di 1

\incantesimo{Evoca Elementali Minori}
\noindent\colorbox{OBSSgold!10}{
\begin{minipage}{0.95\linewidth}
\begin{description}[noitemsep, topsep=0pt, parsep=0pt, partopsep=0pt, leftmargin=0cm, labelwidth=1.3cm]
	\item[\textbf{Lista}]: Aria, Acqua, Terra, Fuoco
	\item[\textbf{Livello}]: 4, Non Comune
	\item[\textbf{Lancio}]: 1 minuto
	\item[\textbf{Gittata}]: 27 metri
	\item[\textbf{Durata}]: 1 Turno
\end{description}
\end{minipage}}\smallskip

Evochi degli elementali che compariranno in spazi non occupati a gittata e che puoi vedere. Scegli una della seguenti opzioni per decidere cosa appare:

\begin{itemize}[leftmargin=*] \setlength{\itemsep}{-1pt}
	\item Un elementale di grado di sfida 2 o meno
	\item Due elementali di grado di sfida 1 o meno
	\item Quattro elementali di grado di sfida 1/2 o meno
	\item Otto elementali di grado di sfida 1/4 o meno
\end{itemize}

Un elementale evocato sparisce quando scende a 0 Punti Ferita o l'incantesimo termina.

Ogni Lista di Magia può evocare solo il proprio Elementale specifico. L'elementale è amichevole verso di te ed i tuoi compagni e ubbidisce al meglio delle sue capacità.

\textbf{Per ogni Successo Critico Magico} ottenuto nella Prova di Magia appariranno due elementale in più di grado inferiore o 1 elementale in più di grado superiore a quello inizialmente scelto.

\incantesimo{Evocazioni Istantanee}
\noindent\colorbox{OBSSgold!10}{
\begin{minipage}{0.95\linewidth}
\begin{description}[noitemsep, topsep=0pt, parsep=0pt, partopsep=0pt, leftmargin=0cm, labelwidth=1.3cm]
	\item[\textbf{Lista}]: Evocazione
	\item[\textbf{Livello}]: 6, Raro
	\item[\textbf{Lancio}]: 1 minuto
	\item[\textbf{Gittata}]: Contatto
	\item[\textbf{Durata}]: Fino a che dissolto
\end{description}
\end{minipage}}\smallskip

Entri a contatto con un oggetto del peso di 5 chili o meno e la cui dimensione più grossa non superi i 180 centimetri. L'incantesimo lascia un marchio sulla superficie dell'oggetto e ne incide invisibilmente il nome sullo zaffiro usato come componente materiale. Ogni volta che lanci questo incantesimo, devi usare uno zaffiro diverso.

In qualsiasi momento successivo, puoi usare 2 Azioni per pronunciare il nome dell'oggetto e frantumare lo zaffiro. L'oggetto appare istantaneamente nella tua mano quale che sia la distanza fisica o planare che vi separa, e l'incantesimo ha termine.

Se un'altra creatura sta impugnando o trasportando l'oggetto, frantumare lo zaffiro non trasporterà l'oggetto da te, ma invece apprenderai chi sia la creatura che ne è in possesso e indicativamente dove si trovi in questo momento.

Dissolvi magie, o un effetto simile applicato con successo allo zaffiro, termina l'effetto dell'incantesimo.

\incantesimo{Fabbricare}
\noindent\colorbox{OBSSgold!10}{
\begin{minipage}{0.95\linewidth}
\begin{description}[noitemsep, topsep=0pt, parsep=0pt, partopsep=0pt, leftmargin=0cm, labelwidth=1.3cm]
	\item[\textbf{Lista}]: Trasmutazione
	\item[\textbf{Livello}]: 4, Comune
	\item[\textbf{Lancio}]: 10 minuti
	\item[\textbf{Gittata}]: 36 metri
	\item[\textbf{Durata}]: Istantanea
\end{description}
\end{minipage}}\smallskip

Converti le materie prime in prodotti finiti dello stesso materiale. Per esempio, puoi fabbricare un piccolo ponte di legno da un cumulo di alberi, una corda da un mucchio di canapa, e abiti dal lino o la lana. Scegli le materie prima che puoi vedere a gittata. Puoi fabbricare un oggetto di taglia Grande o inferiore (contenuto in un cubo di 3 metri di spigolo, o otto cubi connessi di 1 metro di spigolo) data una sufficiente quantità di materie prime. Se stai lavorando con il metallo, la pietra o altre sostanze minerali, l'oggetto fabbricato non può essere più grande di taglia Media (contenuto in un singolo cubo di 1 metro di spigolo). La qualità degli oggetti creati da questo incantesimo è commisurata alla qualità delle materie prime.

Tramite questo incantesimo non si possono creare o trasmutare creature od oggetti magici. Inoltre non puoi usarlo per creare oggetti che normalmente richiedono un alto livello di lavorazione, come i gioielli, le armi, il vetro o le armature, a meno che tu non abbia la competenza con il tipo di strumenti da artigiano utilizzati per costruire questi oggetti. In caso di critico nella Prova di Magia si possono processare più volumi o produrre con maggiore qualità.

\incantesimo{Fatale}
\noindent\colorbox{OBSSgold!10}{
\begin{minipage}{0.95\linewidth}
\begin{description}[noitemsep, topsep=0pt, parsep=0pt, partopsep=0pt, leftmargin=0cm, labelwidth=1.3cm]
	\item[\textbf{Lista}]: Illusione
	\item[\textbf{Livello}]: 9, Raro
	\item[\textbf{Lancio}]: 2 Azioni
	\item[\textbf{Gittata}]: 36 metri
	\item[\textbf{Durata}]: 1 minuto
\end{description}
\end{minipage}}\smallskip

Attingendo alle paure più intime di un gruppo di creature, crei delle creature illusorie nella loro mente, visibili solo a loro. Ogni creatura in una sfera di 9 metri di raggio centrata su di un punto a tua scelta nella gittata, deve effettuare un Tiro Salvezza su Volontà. Se fallisce il Tiro Salvezza, la creatura diventa spaventata per la durata L'illusione affonda nelle paure più intime della creatura, manifestando i suoi incubi peggiori come una implacabile minaccia. Alla fine di ciascun round della creatura spaventata, questa deve superare un Tiro Salvezza su Volontà o subire 4d10 danni da forza. Se supera il Tiro Salvezza, per quella creatura l'incantesimo ha termine.

\textbf{Per due Successo Critico Magico} ottenuto nella Prova di Magia il danno aumenta di 4d10.

\incantesimo{Favore Divino}
\noindent\colorbox{OBSSgold!10}{
\begin{minipage}{0.95\linewidth}
\begin{description}[noitemsep, topsep=0pt, parsep=0pt, partopsep=0pt, leftmargin=0cm, labelwidth=1.3cm]
	\item[\textbf{Lista}]: Invocazione
	\item[\textbf{Livello}]: 1, Non Comune
	\item[\textbf{Lancio}]: 1 Azione
	\item[\textbf{Gittata}]: Personale
	\item[\textbf{Durata}]: 1 minuto
\end{description}
\end{minipage}}\smallskip

Le tue preghiere potenziano te e la tua arma. Fino al termine dell'incantesimo, quando colpisce, la tua arma infligge 1d4 danni da Luce aggiuntivi.

\textbf{Per ogni Successo Critico Magico} ottenuto nella Prova di Magia la tua arma causa +2 danno aggiuntivo da Luce.

\incantesimo{Ferire}
\noindent\colorbox{OBSSgold!10}{
\begin{minipage}{0.95\linewidth}
\begin{description}[noitemsep, topsep=0pt, parsep=0pt, partopsep=0pt, leftmargin=0cm, labelwidth=1.3cm]
	\item[\textbf{Lista}]: Necromanzia
	\item[\textbf{Livello}]: 6, Non Comune
	\item[\textbf{Lancio}]: 2 Azioni
	\item[\textbf{Gittata}]: 18 metri
	\item[\textbf{Durata}]: Istantanea
\end{description}
\end{minipage}}\smallskip

Scateni una malattia virulenta su di una creatura a gittata che puoi vedere. Il bersaglio deve effettuare un Tiro Salvezza su Tempra. Il bersaglio subisce 12d6 danni da Vuoto se fallisce il Tiro Salvezza, o la metà di questi danni se lo supera. Se il bersaglio fallisce il Tiro Salvezza, i suoi Punti Ferita massimi sono ridotti per 1 ora di un ammontare uguale al danno da Vuoto subito. Qualsiasi effetto che rimuova una malattia permette ai Punti Ferita massimi del personaggio di poter tornare al valore normale prima che trascorra quel tempo.

\textbf{Per ogni Successo Critico Magico} ottenuto nella Prova di Magia causi 6d6 di danno da vuoto aggiuntivo.

\incantesimo{Fermare il Tempo}
\noindent\colorbox{OBSSgold!10}{
\begin{minipage}{0.95\linewidth}
\begin{description}[noitemsep, topsep=0pt, parsep=0pt, partopsep=0pt, leftmargin=0cm, labelwidth=1.3cm]
	\item[\textbf{Lista}]: Trasmutazione
	\item[\textbf{Livello}]: 9, Molto Raro
	\item[\textbf{Lancio}]: 2 Azioni
	\item[\textbf{Gittata}]: Personale
	\item[\textbf{Durata}]: Istantanea
\end{description}
\end{minipage}}\smallskip

Fermi brevemente il flusso del tempo per tutti, tranne che per te. Il tempo non scorre per le altre creature, mentre tu effettui 1d4 + 1 round di fila, durante i quali puoi effettuare azioni e muoverti come sempre. Questo incantesimo termina se una delle azioni che usi durante questo periodo, o qualsiasi effetto che crei durante questo periodo, ha effetto su di una creatura diversa da te o su di un oggetto indossato o trasportato da qualcuno che non sia tu. Inoltre, l'incantesimo termina se ti muovi in un posto lontano più di 300 metri da quello in cui lo hai lanciato.

\textbf{Per ogni Successo Critico Magico} ottenuto nella Prova di Magia la durata aumenta di 1 round. In caso di due Successo Magico Critico puoi escludere un altra creatura dal fermarsi del tempo.

\incantesimo{Fiamma Perenne}
\noindent\colorbox{OBSSgold!10}{
\begin{minipage}{0.95\linewidth}
\begin{description}[noitemsep, topsep=0pt, parsep=0pt, partopsep=0pt, leftmargin=0cm, labelwidth=1.3cm]
	\item[\textbf{Lista}]: Universale
	\item[\textbf{Livello}]: 2, Leggendario
	\item[\textbf{Lancio}]: 2 Azioni
	\item[\textbf{Gittata}]: Contatto
	\item[\textbf{Durata}]: 1 giorno
\end{description}
\end{minipage}}\smallskip

Una luminosità simile a quella prodotta da una torcia si sprigiona da un oggetto con cui sei in contatto. L'effetto sembra quello di una normale fiamma, ma non produce calore né necessita ossigeno. Una fiamma perenne può essere celata o nascosta ma non può essere smorzata né spenta.

\incantesimo{Fiamma Sacra}
\noindent\colorbox{OBSSgold!10}{
\begin{minipage}{0.95\linewidth}
\begin{description}[noitemsep, topsep=0pt, parsep=0pt, partopsep=0pt, leftmargin=0cm, labelwidth=1.3cm]
	\item[\textbf{Lista}]: Universale
	\item[\textbf{Livello}]: 0, Comune
	\item[\textbf{Lancio}]: 1 Azione
	\item[\textbf{Gittata}]: 18 metri
	\item[\textbf{Durata}]: Istantanea
\end{description}
\end{minipage}}\smallskip

Una luminosità simile a quella prodotta da una fiaccola discende su di una creatura a gittata che puoi vedere. Il bersaglio deve superare un Tiro Salvezza su Riflessi o subire 1d8 danni da Luce. Il bersaglio non riceve il beneficio della copertura per questo Tiro Salvezza.

Puoi aumentare il danno dell'incantesimo di 1d8 quando la somma dei Tratti in comune con il Patrono raggiunge 5, 11 e 17, ma costa 2 Azioni lanciarlo potenziato e 1 Punti Magia.

\textbf{Per ogni Successo Critico Magico ottenuto} nella Prova di Magia discende una fiamma in più che deve colpire un bersaglio diverso entro gittata.

\incantesimo{Fiotto Acido}\label{Acid Splash}
\noindent\colorbox{OBSSgold!10}{
\begin{minipage}{0.95\linewidth}
\begin{description}[noitemsep, topsep=0pt, parsep=0pt, partopsep=0pt, leftmargin=0cm, labelwidth=1.3cm]
	\item[\textbf{Lista}]: Evocazione
	\item[\textbf{Livello}]: 0, Comune
	\item[\textbf{Lancio}]: 1 Azione
	\item[\textbf{Gittata}]: 18 metri
	\item[\textbf{Durata}]: Istantanea
\end{description}
\end{minipage}}\smallskip

Scagli una bolla di acido. Scegli una creatura a gittata o due creature a gittata che siano entro 1 metro l'una dall'altra. Il bersaglio deve superare un Tiro Salvezza su Riflessi o subire 1d6 danni da acido.

Puoi aumentare il danno dell'incantesimo di 1d8 quando raggiungi CM 5, CM 11 e CM 17, ma costa 2 Azioni lanciarlo potenziato e 1 Punti Magia, è altresì necessario avere preso Adepto della Magia un numero di volte pari ai potenziamenti che si vogliono applicare.

\textbf{Per ogni Successo Critico Magico ottenuto} nella Prova di Magia scagli una bolla di acido in più entro gittata.

\incantesimo{Folata di Vento}
\noindent\colorbox{OBSSgold!10}{
\begin{minipage}{0.95\linewidth}
\begin{description}[noitemsep, topsep=0pt, parsep=0pt, partopsep=0pt, leftmargin=0cm, labelwidth=1.3cm]
	\item[\textbf{Lista}]: Aria
	\item[\textbf{Livello}]: 2, Comune
	\item[\textbf{Lancio}]: 2 Azioni
	\item[\textbf{Gittata}]: Personale (linea di 18 metri)
	\item[\textbf{Durata}]: Concentrazione, massimo 1 minuto
\end{description}
\end{minipage}}\smallskip

Una linea di forte vento lunga 18 metri e larga 3 metri esplode partendo da te in una direzione a tua scelta, per la durata dell'incantesimo. Ogni creatura che inizia il suo round dentro la linea deve superare un Tiro Salvezza su Tempra o venire spinta lontano da te di 3 metri, seguendo la direzione della linea.

Qualsiasi creatura sulla linea deve spendere il doppio del movimento per avvicinarsi a te.

La folata disperde gas o vapori, estingue candele, torce e simili fiamme non protette nell'area. Le fiamme protette, come quelle della lanterne, si agitano, e hanno una probabilità del 50\% di estinguersi. Come 1 Azione durante ciascun tuo round, prima del termine dell'incantesimo, puoi cambiare la direzione in cui la linea si proietta da te.

Un arma da lancio che attraversa una folata di vento ha il 50\% di mancare il bersaglio.

\incantesimo{Forma Eterea}
\noindent\colorbox{OBSSgold!10}{
\begin{minipage}{0.95\linewidth}
\begin{description}[noitemsep, topsep=0pt, parsep=0pt, partopsep=0pt, leftmargin=0cm, labelwidth=1.3cm]
	\item[\textbf{Lista}]: Trasmutazione
	\item[\textbf{Livello}]: 7, Raro
	\item[\textbf{Lancio}]: 2 Azioni
	\item[\textbf{Gittata}]: Personale
	\item[\textbf{Durata}]: Massimo 8 ore
\end{description}
\end{minipage}}\smallskip

Entri nelle regioni di confine del Piano Etereo, nell'area che si sovrappone al tuo piano attuale. Resti sul Confine Etereo per la durata o finché non usi un'Azione per interrompere l'incantesimo. Se ti muovi verso l'alto o il basso, il costo del movimento è raddoppiato, se ti muovi invece orizzontalmente il movimento è raddoppiato per Azione di Movimento. Puoi vedere e udire il piano da cui provieni, ma tutto quello che si trova lì ti appare grigio, e non puoi vedere a più di 18 metri di distanza.

Mentre sei sul Piano Etereo, può interagire solo con altre creature su quel piano. Le creature che non sono sul Piano Etereo non ti possono percepire né interagire con te, a meno che una capacità speciale o la magia gli fornisca la possibilità di farlo.

Ignori tutti gli oggetti e gli effetti che non sono sul Piano etereo, potendo così attraversare gli oggetti che percepisci sul piano da cui provieni. Quando l'incantesimo termina, ritorni immediatamente al piano da cui provieni nel punto che occupi attualmente. Se quando accade occupi lo stesso spazio di un oggetto solido o di una creatura, vieni immediatamente spostato nel più vicino spazio non occupato che puoi occupare e subisci 6 danni da forza per ogni metro di cui vieni spostato (o sua frazione). Questo incantesimo non ha effetto se lo esegui mentre sei già nel Piano Etereo o su di un piano che non vi confina, come uno dei Piani Esterni.

\textbf{Per ogni Successo Critico Magico} ottenuto nella Prova di Magia puoi portare con te un altra creatura.

\incantesimo{Forma Gassosa}
\noindent\colorbox{OBSSgold!10}{
\begin{minipage}{0.95\linewidth}
\begin{description}[noitemsep, topsep=0pt, parsep=0pt, partopsep=0pt, leftmargin=0cm, labelwidth=1.3cm]
	\item[\textbf{Lista}]: Trasmutazione
	\item[\textbf{Livello}]: 3, Non Comune
	\item[\textbf{Lancio}]: 2 Azioni
	\item[\textbf{Gittata}]: Contatto
	\item[\textbf{Durata}]: Concentrazione, massimo 1 ora
\end{description}
\end{minipage}}\smallskip

Trasformi una creatura consenziente insieme a tutto ciò che sta indossando e trasportando, in una nube vaporosa per la durata. L'incantesimo termina se la creatura scende a 0 Punti Ferita. Le creature incorporee ignorano questo effetto. Mentre è in questa forma, l'unico metodo di movimento del bersaglio è una velocità di volo 3 metri. Il bersaglio può entrare e occupare lo spazio di un'altra creatura. Il bersaglio ha resistenza ai danni non magici, e ha +1d6 ai Tiri Salvezza su Tempra e Riflessi. Il bersaglio può attraversare piccoli buchi, strettoie, e anche semplici fori, sebbene consideri i liquidi come superfici solide. Il bersaglio non può cadere e resta fluttuante nell'aria anche se stordito o altrimenti reso inabile.

Mentre è nella forma di una nube vaporosa, il bersaglio non può parlare né manipolare oggetti, e qualsiasi oggetto stesse indossando o trasportando non può essere gettato, usato o altrimenti impiegato. Il bersaglio non può attaccare né lanciare incantesimi.

\textbf{Per ogni due Successo Critico Magico ottenuto} nella Prova di Magia puoi influenzare una altra creatura.

\incantesimo{Frantumare}
\noindent\colorbox{OBSSgold!10}{
\begin{minipage}{0.95\linewidth}
\begin{description}[noitemsep, topsep=0pt, parsep=0pt, partopsep=0pt, leftmargin=0cm, labelwidth=1.3cm]
	\item[\textbf{Lista}]: Invocazione
	\item[\textbf{Livello}]: 2, Comune
	\item[\textbf{Lancio}]: 2 Azioni
	\item[\textbf{Gittata}]: 18 metri
	\item[\textbf{Durata}]: Istantanea
\end{description}
\end{minipage}}\smallskip

Un forte rombo, molto intenso, erutta da un punto a gittata di tua scelta. Ogni creatura in una sfera di 3 metri di raggio centrata su quel punto deve effettuare un Tiro Salvezza su Tempra. Una creatura subisce 3d8 danni da suono se fallisce il Tiro Salvezza, o la metà di questi danni se lo supera. Una creatura composta di materiale inorganico, come pietra, cristallo o metallo, ha -1d6 sul Tiro Salvezza. Un oggetto non magico che non è indossato né trasportato subisce anch'esso danni se si trova nell'area dell'incantesimo.

\textbf{Per ogni Successo Critico Magico} ottenuto nella Prova di Magia il danno aumenta di 2d6.

\textbf{Tiro Salvezza Successo/Fallimento Critico}: In caso di Fallimento Critico il danno raddoppia, in caso di Successo Critico il danno viene ulteriormente dimezzato

\incantesimo{Freccia Acida di Restser}
\noindent\colorbox{OBSSgold!10}{
\begin{minipage}{0.95\linewidth}
\begin{description}[noitemsep, topsep=0pt, parsep=0pt, partopsep=0pt, leftmargin=0cm, labelwidth=1.3cm]
	\item[\textbf{Lista}]: Acqua, Terra
	\item[\textbf{Livello}]: 2, Comune
	\item[\textbf{Lancio}]: 2 Azioni
	\item[\textbf{Gittata}]: 27 metri
	\item[\textbf{Durata}]: Istantanea
\end{description}
\end{minipage}}\smallskip

Una freccia verde luminosa saetta verso un bersaglio a gittata ed esplode con uno spruzzo d'acido. Effettua un attacco a distanza con incantesimo contro il bersaglio. Se colpisci, il bersaglio subisce immediatamente 4d4 danni da acido e 2d4 danni da acido al termine del suo prossimo round. Se manchi, la freccia spruzza il bersaglio di acido infliggendo la metà dei danni iniziali e non arrecando danni al termine del prossimo round del bersaglio.

\textbf{Per ogni Successo Critico Magico} ottenuto nella Prova di Magia il danno aumenta di 2d4.

\incantesimo{Fulmine}
\noindent\colorbox{OBSSgold!10}{
\begin{minipage}{0.95\linewidth}
\begin{description}[noitemsep, topsep=0pt, parsep=0pt, partopsep=0pt, leftmargin=0cm, labelwidth=1.3cm]
	\item[\textbf{Lista}]: Aria
	\item[\textbf{Livello}]: 3, Comune
	\item[\textbf{Lancio}]: 2 Azioni
	\item[\textbf{Gittata}]: Personale (linea di 30 metri)
	\item[\textbf{Durata}]: Istantanea
\end{description}
\end{minipage}}\smallskip

Esplodi un fulmine che forma una linea lunga 30 metri e larga 1 metro che parte da dove ti trovi in una direzione scelta da te. Ogni creatura sulla linea deve superare un Tiro Salvezza su Riflessi. La creatura subisce 8d6 danni da elettricità se fallisce il Tiro Salvezza, o la metà di questi danni se lo supera.
Il fulmine incendia gli oggetti infiammabili nell'area che non sono indossati o trasportati.

Il fulmine se lanciato contro della pietra dura lavorata rimbalza con un angolo di uscita uguale a quello di entrata (\textbackslash|/) (180-angolo di entrata). Un fulmine lanciato in acqua crea una sfera di 3 metri di raggio di elettricità nel punto in cui entra.

\textbf{Per ogni Successo Critico Magico} ottenuto nella Prova di Magia il danno aumenta di 4d6.

\textbf{Tiro Salvezza Successo/Fallimento Critico}: In caso di Fallimento Critico il danno raddoppia, in caso di Successo Critico il danno viene ulteriormente dimezzato

\incantesimo{Fulmine a catena}
\noindent\colorbox{OBSSgold!10}{
\begin{minipage}{0.95\linewidth}
\begin{description}[noitemsep, topsep=0pt, parsep=0pt, partopsep=0pt, leftmargin=0cm, labelwidth=1.3cm]
	\item[\textbf{Lista}]: Aria
	\item[\textbf{Livello}]: 6, Raro
	\item[\textbf{Lancio}]: 2 Azioni
	\item[\textbf{Gittata}]: 45 metri
	\item[\textbf{Durata}]: Istantanea
\end{description}
\end{minipage}}\smallskip

Crei una saetta di elettricità che colpisce un bersaglio a gittata che puoi vedere scelto da te. Da questo si genera una ulteriore saetta che colpisce il più vicino bersaglio entro 6 metri. Il processo continua finché non sono state colpite 7 bersagli o non c'è più nessun nuovo avversario a distanza. Un bersaglio può essere una creatura o oggetto almeno di taglia piccola e può essere bersaglio di una sola saetta. Un bersaglio deve effettuare un Tiro Salvezza su Riflessi oppure subisce 8d6 danni da elettricità o la metà di questi danni se lo supera.

\textbf{Per ogni Successo Critico Magico} ottenuto nella Prova di Magia la saetta si protende su tre ulteriori bersagli.

\textbf{Tiro Salvezza Successo/Fallimento Critico}: In caso di Fallimento Critico il danno raddoppia, in caso di Successo Critico il danno viene ulteriormente dimezzato

\incantesimo{Fuorviare}
\noindent\colorbox{OBSSgold!10}{
\begin{minipage}{0.95\linewidth}
\begin{description}[noitemsep, topsep=0pt, parsep=0pt, partopsep=0pt, leftmargin=0cm, labelwidth=1.3cm]
	\item[\textbf{Lista}]: Illusione
	\item[\textbf{Livello}]: 5, Non Comune
	\item[\textbf{Lancio}]: 2 Azioni
	\item[\textbf{Gittata}]: Personale
	\item[\textbf{Durata}]: 1 ora
\end{description}
\end{minipage}}\smallskip

Diventi invisibile nello stesso momento in cui un tuo doppione illusorio compare nel posto in cui ti trovi. Il doppione resta per la durata dell'incantesimo, ma l'invisibilità termina se attacchi o lanci un incantesimo. Puoi usare 2 Azioni per far muovere il doppione illusorio fino al doppio della tua velocità e fargli compiere un gesto, parlare e comportarsi in qualsiasi maniera tu voglia.

Puoi vedere attraverso i suoi occhi e udire tramite le sue orecchie come se fossi nello spazio in cui si trova lui. Durante ciascun tuo round, con un'Azione, puoi passare dall'usare i suoi sensi all'usare i tuoi, o viceversa. Mentre stai usando i suoi sensi, sei accecato e assordato riguardo i tuoi dintorni.

\incantesimo{Gabbia di Forza}
\noindent\colorbox{OBSSgold!10}{
\begin{minipage}{0.95\linewidth}
\begin{description}[noitemsep, topsep=0pt, parsep=0pt, partopsep=0pt, leftmargin=0cm, labelwidth=1.3cm]
	\item[\textbf{Lista}]: Invocazione
	\item[\textbf{Livello}]: 6, Raro
	\item[\textbf{Lancio}]: 2 Azioni
	\item[\textbf{Gittata}]: 30 metri
	\item[\textbf{Durata}]: 1 ora
\end{description}
\end{minipage}}\smallskip

Una prigione cubica, immobile e invisibile, composta di forza magica compare intorno a un'area a gittata da te scelta. La prigione può essere una gabbia o una scatola solida, a tua scelta. Una prigione nella forma di una gabbia può avere 6 metri di lato ed essere composta da sbarre di 1,5 centimetri separate di 1,5 centimetri tra di loro e fornisce una copertura completa alle creature all'interno. Una prigione a forma di scatola può avere 3 metri di lato, creando una barriera solida che impedisce a qualsiasi materia di attraversarla e bloccando qualsiasi incantesimo lanciato dall'interno o l'esterno dell'area. Quando lanci questo incantesimo, qualsiasi creatura che è completamente all'interno della gabbia deve effettuare un Tiro Salvezza su Riflessi o rimanere intrappolata. Le creature solo parzialmente nell'area della gabbia, o quelle troppo grosse per entrarvi, vengono spinte via dal centro dell'area finché non ne sono completamente fuori.

Una creatura all'interno della gabbia non può lasciarla tramite mezzi non magici. Se la creatura prova a usare il teletrasporto o il viaggio interplanare per lasciare la gabbia, deve prima effettuare un Tiro Salvezza su Volontà. Se lo supera, la creatura può usare quella magia per sfuggire alla gabbia. Se lo fallisce, la creatura non può uscire dalla gabbia e spreca l'uso dell'incantesimo o dell'effetto. La gabbia si estende anche sul Piano Etereo, bloccando così il viaggio etereo.

Questo incantesimo non può essere dissolto da \hyperlink{dissolvimagie}{Dissolvi Magie} ma solo con \hyperlink{dissolvimagieavanzato}{Dissolvi Magie Avanzato}.

\incantesimo{Giara Magica}
\noindent\colorbox{OBSSgold!10}{
\begin{minipage}{0.95\linewidth}
\begin{description}[noitemsep, topsep=0pt, parsep=0pt, partopsep=0pt, leftmargin=0cm, labelwidth=1.3cm]
	\item[\textbf{Lista}]: Necromanzia
	\item[\textbf{Livello}]: 6, Molto Raro
	\item[\textbf{Lancio}]: 1 minuto
	\item[\textbf{Gittata}]: Personale
	\item[\textbf{Durata}]: Finché a che dissolto
\end{description}
\end{minipage}}\smallskip

Il tuo corpo entra in uno stato catatonico mentre la tua anima lo abbandona ed entra nel contenitore da te usato come componente materiale. Mentre la tua anima occupa il contenitore, sei consapevole dei tuoi dintorni come se fossi nello spazio del contenitore. Non puoi muoverti né usare reazioni. L'unica Azione che puoi effettuare è quella di proiettare la tua anima fino a 30 metri di distanza, fuori dal contenitore, ritornando al tuo corpo vivente (e terminando l'incantesimo) o cercando di possedere un corpo umanoide.

Puoi tentare di possedere qualsiasi umanoide entro 30 metri da te e che tu possa vedere (le creature protette da cerchio magico non possono essere possedute). Il bersaglio deve effettuare un Tiro Salvezza su Volontà e, se lo fallisce, la tua anima entra nel corpo del bersaglio, mentre l'anima del bersaglio resta intrappolata nel contenitore. Se lo supera, il bersaglio resiste ai tuoi tentativi di possederlo, e non puoi tentare di possederlo nuovamente prima che siano trascorse 24 ore.

Una volta che possiedi il corpo di una creatura, lo puoi controllare. Le tue statistiche di gioco sono rimpiazzate dalle statistiche della creatura, a eccezione dei tuoi Tratti e dei tuoi punteggi di Intelligenza, Saggezza e Carisma. Mantieni i benefici forniti dalle Abilità. Se il bersaglio possiede delle Abilità non puoi usarne nessuna.

Nel frattempo, l'anima della creatura posseduta può percepire i dintorni del contenitore usando i propri sensi, ma non può muoversi né effettuare alcuna azione.

Mentre possiedi un corpo, puoi usare 2 Azioni per ritornare dal corpo ospite al contenitore, se ti trovi entro 30 metri da esso, riportando l'anima della creatura ospite nel suo corpo. Se il corpo ospite muore mentre sei al suo interno, la creatura muore, e tu devi effettuare un Tiro Salvezza su Volontà contro la tua DC dei Tiri Salvezza degli incantesimi. Se lo superi, ritorni al contenitore, se si trova entro 30 metri da te. Altrimenti, morirai.

Se il contenitore viene distrutto o l'incantesimo termina, la tua anima ritorna immediatamente al tuo corpo. Se il tuo corpo è più di 30 metri lontano o se è morto mentre cerchi di farvi ritorno, morirà anche la tua anima. Se l'anima di un'altra creatura è nel contenitore quando viene distrutto, l'anima della creatura ritorna al suo corpo, se il corpo è vivo e si trova entro 30 metri, altrimenti, la creatura muore. Quando l'incantesimo termina, il contenitore viene distrutto.

\incantesimo{Glifo di Interdizione}
\noindent\colorbox{OBSSgold!10}{
\begin{minipage}{0.95\linewidth}
\begin{description}[noitemsep, topsep=0pt, parsep=0pt, partopsep=0pt, leftmargin=0cm, labelwidth=1.3cm]
	\item[\textbf{Lista}]: Abiurazione
	\item[\textbf{Livello}]: 3, Comune
	\item[\textbf{Lancio}]: 2 Azioni
	\item[\textbf{Gittata}]: Contatto
	\item[\textbf{Durata}]: Fino a che dissolto o attivato
\end{description}
\end{minipage}}\smallskip

Quando lanci questo incantesimo, inscrivi un glifo che danneggia altre creature su di una superficie (come un tavolo o una sezione di pavimento o muro) o all'interno di un oggetto che può essere chiuso (come un libro, una pergamena o un forziere) per celare il glifo. Se scegli una superficie, il glifo può coprire un'area di superficie non maggiore di 3 metri di diametro. Se scegli un oggetto, quell'oggetto deve restare al suo posto; se l'oggetto viene spostato più di 3 metri dal punto in cui è stato lanciato l'incantesimo, il glifo è spezzato, e l'incantesimo termina senza essere stato attivato.

Il glifo è quasi invisibile e può essere trovato con una prova Consapevolezza contro la DC del Tiro Salvezza dei tuoi incantesimi. Decidi tu cosa attivi il glifo al momento del lancio dell'incantesimo.

Per i glifi inscritti su di una superficie, l'attivazione tipica comprende entrare in contatto o stare sopra il glifo, rimuovere un altro oggetto che copra il glifo, avvicinarsi a una certa distanza dal glifo, o manipolare l'oggetto su cui è inscritto il glifo. Per i glifi inscritti su di un oggetto, l'attivazione tipica comprende aprire l'oggetto, avvicinarsi a una certa distanza dall'oggetto, o vedere o leggere il glifo. Una volta che il glifo è stato attivato, l'incantesimo ha termine.

Puoi definire meglio l'attivazione così che l'incantesimo si attivi solo in determinate circostanze o secondo certe peculiarità fisiche (come l'altezza o il peso), specie di creatura (per esempio, l'interdizione potrebbe agire contro le aberrazioni o gli elfi oscuri), o specifici Tratti. Puoi anche predisporre condizioni per evitare che il glifo venga attivato, come la pronuncia di una parola d'ordine.

Quando inscrivi il glifo scegli rune esplosive o glifo incantesimo.

\medskip

- \emph{Glifo Incantesimo}. Puoi inserire un incantesimo preparato di livello 2 o inferiore nel glifo lanciandolo come parte della creazione del glifo. L'incantesimo deve prendere come bersaglio una singola creatura o un'area. L'incantesimo che viene inserito non ha effetto immediato se lanciato in questo modo. Quando il glifo è attivato, l'incantesimo inserito viene lanciato. Se l'incantesimo ha un bersaglio, prende come bersaglio la creatura che ha attivato il glifo. Se l'incantesimo agisce su di un'area, l'area è incentrata su quella creatura. Se l'incantesimo evoca creature ostili o crea oggetti o trappole nocive, questi appaiono quanto più vicino possibile all'intruso e lo attaccano. Se l'incantesimo richiede concentrazione, questa è mantenuta fino al termine della sua normale durata.

- \emph{Rune Esplosive}. Quando attivato, il glifo erutta energia magica in una sfera di raggio 6 metri centrata sul glifo. La sfera si propaga intorno agli angoli. Ogni creatura nell'area deve effettuare un Tiro Salvezza su Riflessi. Una creatura subisce 5d8 danni da acido, fulmine, fuoco, freddo o suono se fallisce il Tiro Salvezza (a tua scelta quando crei il glifo), o la metà di questi danni se supera il Tiro Salvezza.

Non è possibile avere contemporaneamente più di CM/4 Glifi attivi contemporaneamente.

\textbf{Per ogni Successo Critico Magico} ottenuto nella Prova di Magia il danno del glifo rune esplosive aumenta di 1d8.

\textbf{Per ogni due Successo Critico Magico} ottenuto nella Prova di Magia è possibile inserire un incantesimo di livello superiore nel Glifo Incantesimo.

\incantesimo{Globo di Invulnerabilità}
\noindent\colorbox{OBSSgold!10}{
\begin{minipage}{0.95\linewidth}
\begin{description}[noitemsep, topsep=0pt, parsep=0pt, partopsep=0pt, leftmargin=0cm, labelwidth=1.3cm]
	\item[\textbf{Lista}]: Abiurazione
	\item[\textbf{Livello}]: 6, Comune
	\item[\textbf{Lancio}]: 2 Azioni
	\item[\textbf{Gittata}]: Personale (raggio di 3 metri)
	\item[\textbf{Durata}]: Concentrazione, massimo 1 minuto
\end{description}
\end{minipage}}\smallskip

Una barriera immobile e lievemente scintillante si erge in un raggio di 3 metri intorno a te e vi rimane per la durata.

Qualsiasi incantesimo di Livello 4 (ad esclusione di risultati superiori grazie a critici magici) o più basso lanciato dall'esterno della barriera non può agire sulle creature o gli oggetti al suo interno. Questi incantesimi vengono soppressi se prendono come bersaglio creature e oggetti all'interno della barriera o coinvolgono l'area su cui è la barriera.

\textbf{Per ogni due Successo Critico Magico ottenuto} nella Prova di Magia puoi bloccare un livello superiore di incantesimo.

\incantesimo{Goffaggine}
\noindent\colorbox{OBSSgold!10}{
\begin{minipage}{0.95\linewidth}
\begin{description}[noitemsep, topsep=0pt, parsep=0pt, partopsep=0pt, leftmargin=0cm, labelwidth=1.3cm]
	\item[\textbf{Lista}]: Trasmutazione
	\item[\textbf{Livello}]: 2, Raro
	\item[\textbf{Lancio}]: 1 Azione
	\item[\textbf{Gittata}]: 9 metri
	\item[\textbf{Durata}]: 1 round per CM, Concentrazione
\end{description}
\end{minipage}}\smallskip

Il bersaglio effettua un Tiro Salvezza su Volontà con bonus Carisma, se fallisce ogni qual volta che effettua una Prova di Competenza, Tiro Salvezza o Tiro per Colpire conta sempre un 1 tirato in più per verificare i fallimenti critici.

\incantesimo{Gragnola di Ghiande di Kyrin}
\noindent\colorbox{OBSSgold!10}{
\begin{minipage}{0.95\linewidth}
\begin{description}[noitemsep, topsep=0pt, parsep=0pt, partopsep=0pt, leftmargin=0cm, labelwidth=1.3cm]
	\item[\textbf{Lista}]: Animali e Piante
	\item[\textbf{Livello}]: 2, Non Comune
	\item[\textbf{Lancio}]: 1 Azione
	\item[\textbf{Gittata}]: 50 metri
	\item[\textbf{Durata}]: 1 minuto per CM, Concentrazione
\end{description}
\end{minipage}}\smallskip

Incanti 9 ghiande di energia magica e queste incominciano a vorticare 30 centimetri sopra la tua spalla.
Ogni round, spendendo 1 Azione, puoi lanciare fino a 5 ghiande contro uno o più bersagli.
Esegui un solo Tiro per Colpire con incantesimi da distanza e confronta il risultato con la Difesa di ogni bersaglio indipendentemente dal numero di ghiande che gli tiri. Ogni ghianda se colpisce fa 1d4 di danni contundenti.

\textbf{Per ogni Successo Critico Magico} ottenuto nella Prova di Magia puoi incantare due ghiande in più.

\incantesimo{Gragnola di Ghiande Infuocate di Kyrin}
\noindent\colorbox{OBSSgold!10}{
\begin{minipage}{0.95\linewidth}
\begin{description}[noitemsep, topsep=0pt, parsep=0pt, partopsep=0pt, leftmargin=0cm, labelwidth=1.3cm]
	\item[\textbf{Lista}]: Animali e Piante, Fuoco
	\item[\textbf{Livello}]: 3, Raro
	\item[\textbf{Lancio}]: 2 Azione
	\item[\textbf{Gittata}]: 50 metri
	\item[\textbf{Durata}]: 1 minuto per CM, Concentrazione
\end{description}
\end{minipage}}\smallskip

Incanti 9 ghiande di energia magica e queste incominciano a vorticare 30 centimetri sopra la tua spalla.
Ogni round, spendendo 1 Azione, puoi lanciare fino a 5 ghiande contro uno o più bersagli.
Esegui un solo Tiro per Colpire con incantesimi da distanza e confronta il risultato con la Difesa di ogni bersaglio indipendentemente dal numero di ghiande che gli tiri. Ogni ghianda se colpisce fa 1d4 di danni contundenti + 1d4 di danno da fuoco.

\textbf{Per ogni Successo Critico Magico} ottenuto nella Prova di Magia puoi incantare due ghiande in più.

\incantesimo{Gragnola di Limoni di Kyrin}
\noindent\colorbox{OBSSgold!10}{
\begin{minipage}{0.95\linewidth}
\begin{description}[noitemsep, topsep=0pt, parsep=0pt, partopsep=0pt, leftmargin=0cm, labelwidth=1.3cm]
	\item[\textbf{Lista}]: Animali e Piante, Terra
	\item[\textbf{Livello}]: 3, Raro
	\item[\textbf{Lancio}]: 2 Azioni
	\item[\textbf{Gittata}]: 30 metri
	\item[\textbf{Durata}]: 1 round per CM, Concentrazione
\end{description}
\end{minipage}}\smallskip

Incanti una boccetta con dentro almeno 9 gocce di limone.
Ogni round, spendendo 1 Azione, puoi spruzzare fino a 2 gocce di limone, delle 9 totali, contro uno o più bersagli entro 30 metri.
Esegui un solo Tiro per Colpire con incantesimi da distanza e confronta il risultato con la Difesa di ogni bersaglio indipendentemente dal numero di ghiande che gli tiri. Ogni goccia se colpisce fa 1d6+1 danni da acido.

\textbf{Per ogni Successo Critico Magico} ottenuto nella Prova di Magia puoi creare due gocce di limone in più.

\incantesimo{Gragnola di Marroni di Kyrin}
\noindent\colorbox{OBSSgold!10}{
\begin{minipage}{0.95\linewidth}
\begin{description}[noitemsep, topsep=0pt, parsep=0pt, partopsep=0pt, leftmargin=0cm, labelwidth=1.3cm]
	\item[\textbf{Lista}]: Animali e Piante
	\item[\textbf{Livello}]: 5, Molto Raro
	\item[\textbf{Lancio}]: 1 Azione
	\item[\textbf{Gittata}]: 60 metri
	\item[\textbf{Durata}]: 1 minuto per CM, Concentrazione
\end{description}
\end{minipage}}\smallskip

Incanti 9 marroni di energia magica e queste incominciano a vorticare 60 centimetri sopra la tua spalla.
Ogni round, spendendo 1 Azioni, puoi lanciare fino a 5 marroni contro uno o più bersagli.
Esegui un solo Tiro per Colpire con incantesimi da distanza e confronta il risultato con la Difesa di ogni bersaglio indipendentemente dal numero di ghiande che gli tiri. Ogni ghianda se colpisce fa 2d8+4 di danni contundenti

\textbf{Per ogni Successo Critico Magico} ottenuto nella Prova di Magia puoi incantare due marroni in più.

\incantesimo{Grido di dolore}
\noindent\colorbox{OBSSgold!10}{
\begin{minipage}{0.95\linewidth}
\begin{description}[noitemsep, topsep=0pt, parsep=0pt, partopsep=0pt, leftmargin=0cm, labelwidth=1.3cm]
	\item[\textbf{Lista}]: Necromanzia
	\item[\textbf{Livello}]: 1, Raro
	\item[\textbf{Lancio}]: 1 Reazione
	\item[\textbf{Gittata}]: personale
	\item[\textbf{Durata}]: Istantanea
\end{description}
\end{minipage}}\smallskip

Come Azione di Reazione emetti un grido di dolore quando colpito in mischia. La creatura che ti ha colpito deve effettuare un Tiro Salvezza su Tempra o subito 2d4 di danno da Vuoto.

\textbf{Per ogni Successo Critico Magico} ottenuto nella Prova di Magia causi 1d6 di danno in più.

\incantesimo{Guarigione}
\noindent\colorbox{OBSSgold!10}{
\begin{minipage}{0.95\linewidth}
\begin{description}[noitemsep, topsep=0pt, parsep=0pt, partopsep=0pt, leftmargin=0cm, labelwidth=1.3cm]
	\item[\textbf{Lista}]: Cura
	\item[\textbf{Livello}]: 6, Raro
	\item[\textbf{Lancio}]: 2 Azioni
	\item[\textbf{Gittata}]: 18 metri
	\item[\textbf{Durata}]: Istantanea
\end{description}
\end{minipage}}\smallskip

Scegli una creatura a gittata e che puoi vedere. un'ondata di energia positiva curativa travolge la creatura, facendole recuperare 70 Punti Ferita. L'incantesimo prova anche a \hyperlink{contrastareincantesimi}{contrastare} a qualsiasi cecità, sordità e malattia (anche magica) che affligga il bersaglio. Questo incantesimo causa 50 Punti Ferita di danno ad un non morto con un Tiro per Colpire con incantesimo a tocco.

\textbf{Per ogni Successo Critico Magico} ottenuto nella Prova di Magia l'ammontare guarito aumenta di 20.

\textbf{NOTA}: Se incantatore e creatura curata sono entrambi \textbf{Seguaci} dello stesso Patrono l'incantesimo cura 90 Punti Ferita.

\textbf{NOTA}: Se incantatore e creatura curata sono entrambi \textbf{Devoti} dello stesso Patrono l'incantesimo riporta a pieno di Punti Ferita.

\incantesimo{Guarigione di Massa}
\noindent\colorbox{OBSSgold!10}{
\begin{minipage}{0.95\linewidth}
\begin{description}[noitemsep, topsep=0pt, parsep=0pt, partopsep=0pt, leftmargin=0cm, labelwidth=1.3cm]
	\item[\textbf{Lista}]: Cura
	\item[\textbf{Livello}]: 9, Leggendario
	\item[\textbf{Lancio}]: 2 Azioni
	\item[\textbf{Gittata}]: 18 metri
	\item[\textbf{Durata}]: Istantanea
\end{description}
\end{minipage}}\smallskip

Un effluvio di energia guaritrice scorre da te verso le creature ferite che ti circondano. Ripristini fino a 700 Punti Ferita, divisi come preferisci tra qualsiasi creatura a gittata e che puoi vedere (con un massimo di 70 Punti Ferita a creatura). Le creature guarite da questo incantesimo sono curate anche di tutte le malattie e da qualsiasi effetto che le renda accecate o assordate. Questo incantesimo può infliggere fino a 120 Punti Ferita di danno ad un non morto. TS su Tempra per annullare l'effetto.

Se l'incantatore e creatura curata sono entrambi \textbf{Seguaci} dello stesso Patrono la cura assegnata aumenta di 20\%

Se l'incantatore e creatura curata sono entrambi \textbf{Devoti} dello stesso Patrono la cura assegnata aumenta di 50\%

\incantesimo{Guida}
\noindent\colorbox{OBSSgold!10}{
\begin{minipage}{0.95\linewidth}
\begin{description}[noitemsep, topsep=0pt, parsep=0pt, partopsep=0pt, leftmargin=0cm, labelwidth=1.3cm]
	\item[\textbf{Lista}]: Divinazione
	\item[\textbf{Livello}]: 0, Comune
	\item[\textbf{Lancio}]: 1 Reazione
	\item[\textbf{Gittata}]: 3 metri
		\item[\textbf{Durata}]: 1 Round
\end{description}
\end{minipage}}\smallskip

Lanci l'incantesimo a contatto di una creatura consenziente. Una volta, prima che l'incantesimo termini, il bersaglio può tirare un d4 e sommare il risultato tirato a una prova di competenza a sua scelta. Può tirare il dado prima o dopo aver effettuato la prova di Competenza. L'incantesimo ha poi termine. Non è possibile lanciare Guida sulla stessa creatura ad intervalli inferiori ad 1 ora.

\incantesimo{Guscio Anti-Vita}
\noindent\colorbox{OBSSgold!10}{
\begin{minipage}{0.95\linewidth}
\begin{description}[noitemsep, topsep=0pt, parsep=0pt, partopsep=0pt, leftmargin=0cm, labelwidth=1.3cm]
	\item[\textbf{Lista}]: Animali e Piante
	\item[\textbf{Livello}]: 5, Non Comune
	\item[\textbf{Lancio}]: 2 Azioni
	\item[\textbf{Gittata}]: Personale (raggio di 3 metri)
	\item[\textbf{Durata}]: massimo 1 ora
\end{description}
\end{minipage}}\smallskip

Una barriera luminosa si estende fino a un raggio di 3 metri intorno a te, muovendosi con te e rimanendo centrata su di te, tenendo distanti le creature che non siano non morti o costrutti. La barriera permane per la durata.

La barriera impedisce a una creatura soggetta di attraversarla in alcun modo. Una creatura soggetta può lanciare incantesimi o effettuare attacchi con armi a distanza o con portata attraverso la barriera. Se ti muovi in modo che una creatura soggetta venga forzata ad attraversare la barriera, l'incantesimo termina.

\textbf{Per ogni Successo Critico Magico} ottenuto nella Prova di Magia la durata raddoppia.

\incantesimo{Identificare}
\noindent\colorbox{OBSSgold!10}{
\begin{minipage}{0.95\linewidth}
\begin{description}[noitemsep, topsep=0pt, parsep=0pt, partopsep=0pt, leftmargin=0cm, labelwidth=1.3cm]
	\item[\textbf{Lista}]: Universale
	\item[\textbf{Livello}]: 1, Comune
	\item[\textbf{Lancio}]: variabile
	\item[\textbf{Gittata}]: Contatto
	\item[\textbf{Durata}]: Variabile
\end{description}
\end{minipage}}\smallskip

Scegli un oggetto con cui devi restare a contatto per tutto il lancio dell'incantesimo. Se si tratta di un oggetto magico o altro oggetto imbevuto di magia effettua una prova di Arcana, con un bonus di +2d6, come parte di lancio dell'incantesimo.

Se la DC che fatti è superiore a 20 ne apprendi le caratteristiche principali, con DC 25 ne apprendi le proprietà e come usarle e quante cariche abbia, se ne ha.

Apprendi se degli incantesimi stiano agendo sull'oggetto e cosa siano. Se l'oggetto è stato creato da un incantesimo, apprendi quale incantesimo lo abbia creato. Se invece durante l'esecuzione resti a contatto con una creatura, apprendi se degli incantesimi stiano agendo su di essa e quali siano.

La prova di Arcana dura 10 minuti. Con punteggio Arcana 6 dura 5 minuti, con 12 dura 1 minuto, con Arcana 18 è sufficiente 1 Round.

\textbf{Solo se ottieni un Successo Critico Magico} apprendi se l'oggetto è \hyperlink{oggettimaledettiid}{maledetto}.

\incantesimo{Illusione Minore}
\noindent\colorbox{OBSSgold!10}{
\begin{minipage}{0.95\linewidth}
\begin{description}[noitemsep, topsep=0pt, parsep=0pt, partopsep=0pt, leftmargin=0cm, labelwidth=1.3cm]
	\item[\textbf{Lista}]: Universale
	\item[\textbf{Livello}]: 0, Comune
	\item[\textbf{Lancio}]: 2 Azioni
	\item[\textbf{Gittata}]: 9 metri
	\item[\textbf{Durata}]: 1 minuto
\end{description}
\end{minipage}}\smallskip

Crei l'immagine di un oggetto o un suono a gittata per la durata dell'incantesimo. L'illusione ha termine se la interrompi con un'Azione o lanci di nuovo questo incantesimo.

Se crei un suono, il suo volume può variare da quello di un bisbiglio a un urlo. Può essere la tua voce, la voce di qualcun altro, il ruggito di un leone, un battito di tamburi, o qualsiasi altro suono tu scelga. Il suono continua incessante per tutta la durata, oppure puoi produrre suoni diversi in momenti diversi prima del termine dell'incantesimo.

Se crei l'immagine di un oggetto (come una sedia, un'impronta fangosa o un piccolo forziere) non può essere più grande di un cubo di 1 metro di spigolo. L'immagine non può produrre suoni, luci, odori o qualsiasi altro effetto sensoriale. L'interazione fisica con l'oggetto lo rivela come illusione, perché le cose lo possono attraversare.

Una creatura che usa 3 Azioni per esaminare il suono o l'immagine può determinare che si tratta di un'illusione con una prova riuscita di Intelligenza (Indagare) contro la DC del Tiro Salvezza del tuo incantesimo. Se una creatura riconosce l'illusione per quello che è, per lei l'illusione sbiadisce.

\incantesimo{Illusione Programmata}
\noindent\colorbox{OBSSgold!10}{
\begin{minipage}{0.95\linewidth}
\begin{description}[noitemsep, topsep=0pt, parsep=0pt, partopsep=0pt, leftmargin=0cm, labelwidth=1.3cm]
	\item[\textbf{Lista}]: Illusione
	\item[\textbf{Livello}]: 6, Non Comune
	\item[\textbf{Lancio}]: 2 Azioni
	\item[\textbf{Gittata}]: 36 metri
	\item[\textbf{Durata}]: Fino a che dissolto
\end{description}
\end{minipage}}\smallskip

Crei, a gittata, l'illusione di un oggetto, creatura o qualche altro fenomeno visibile che si attiva quando viene soddisfatta una specifica condizione. Fino ad allora l'illusione è impercettibile. Non può essere più grande di un cubo di 9 metri di spigolo, e decidi tu quando lanci l'incantesimo, come si comporti l'illusione e che suoni produca. L'esibizione programmata può durare fino a 5 minuti. Quando occorrono le condizioni da te specificate, l'illusione si manifesta e si comporta nel modo da te descritto. Una volta che l'illusione ha terminato la sua esibizione, scompare e rimane dormiente per 10 minuti. Dopo questo periodo, l'illusione può essere attivata di nuovo.

La condizione di attivazione può essere generica o dettagliata quanto vuoi, sebbene debba essere basata su condizioni visibili o udibili che avvengano entro 9 metri dall'area. Per esempio, potresti creare un'illusione di te stesso che appare e avverta chi tenti di aprire una porta munita di trappola, oppure potresti predisporre l'illusione perché si attivi solo quando una creatura pronunci la parola o la frase giusta.

L'interazione fisica con l'immagine la rivela come illusione, dato che le cose le passano attraverso. Una creatura che usi 3 Azioni per esaminare l'immagine può determinare che è un'illusione con una prova riuscita di Intelligenza (Indagare) contro la DC del Tiro Salvezza dell'incantesimo. Se una creatura riconosce l'illusione per quello che è, essa può vedere attraverso l'immagine, e qualsiasi suono prodotto dall'immagine le suona artefatto.

\incantesimo{Immagine Maggiore}
\noindent\colorbox{OBSSgold!10}{
\begin{minipage}{0.95\linewidth}
\begin{description}[noitemsep, topsep=0pt, parsep=0pt, partopsep=0pt, leftmargin=0cm, labelwidth=1.3cm]
	\item[\textbf{Lista}]: Illusione
	\item[\textbf{Livello}]: 3, Comune
	\item[\textbf{Lancio}]: 2 Azioni
	\item[\textbf{Gittata}]: 36 metri
	\item[\textbf{Durata}]: Concentrazione, massimo 1 minuto per CM
\end{description}
\end{minipage}}\smallskip

Crei l'immagine di un oggetto, una creatura o qualche altro fenomeno visibile non più grande di un cubo di 6 metri di spigolo. L'immagine appare in un punto a gittata che puoi vedere e vi rimane per la durata dell'incantesimo. L'immagine sembra completamente reale, e comprende suoni, odori e la temperatura appropriata alla cosa raffigurata. Non puoi generare calore o freddo sufficiente a provocare danni, né un suono abbastanza forte da infliggere danno da suono o assordare una creatura, o un odore che possa far star male una creatura (come il fetore di un troglodita). Finché resti a gittata dell'illusione, puoi usare un'Azione per far muovere l'immagine in qualsiasi altro punto a gittata.

Quando l'immagine cambia posizione, puoi alterarne l'aspetto così che i suoi movimenti appaiano naturali. Per esempio, se crei l'immagine di una creatura e la muovi, puoi alterare l'immagine in modo che sembri camminare. Allo stesso modo, puoi impiegare l'illusione per produrre suoni diversi in momenti diversi, fino a farle portare avanti una conversazione.

L'interazione fisica con l'immagine la rivela come illusione, dato che le cose vi passano attraverso. Una creatura che usa 3 Azioni per esaminare l'immagine può determinare che si tratta di un'illusione con una prova riuscita di Intelligenza (Indagare) contro la DC del Tiro Salvezza del tuo incantesimo. Se una creatura riconosce l'illusione per quello che è, la creatura può vedervi attraverso, e per quella creatura tutte le altre qualità sensoriali svaniscono.

\textbf{Se ottieni un Successo Critico Magico} nella Prova di Magia l'incantesimo dura finché non viene dissolto, senza richiedere la tua concentrazione.

\incantesimo{Immagine Proiettata}
\noindent\colorbox{OBSSgold!10}{
\begin{minipage}{0.95\linewidth}
\begin{description}[noitemsep, topsep=0pt, parsep=0pt, partopsep=0pt, leftmargin=0cm, labelwidth=1.3cm]
	\item[\textbf{Lista}]: Illusione
	\item[\textbf{Livello}]: 7, Non Comune
	\item[\textbf{Lancio}]: 2 Azioni
	\item[\textbf{Gittata}]: 750 chilometri
	\item[\textbf{Durata}]: 1 giorno
\end{description}
\end{minipage}}\smallskip

Crei una copia illusoria di te stesso che permane per la durata. La copia può apparire in qualsiasi luogo entro la gittata che tu abbia già visto, ignorando qualsiasi ostacolo frapposto. L'illusione riproduce il tuo aspetto e i tuoi rumori ma è intangibile. Se l'illusione subisce danni, scompare, e l'incantesimo ha termine.

Puoi usare 2 Azioni per far muovere questa illusione fino al doppio della tua velocità e farle compiere un gesto, parlare e comportarsi in qualsiasi maniera tu voglia. Imita alla perfezione i tuoi comportamenti.

Puoi vedere attraverso i suoi occhi e udire tramite le sue orecchie come se fossi nello spazio in cui essa si trova. Durante ciascun tuo round, con un'Azione, puoi passare dall'usare i suoi sensi all'usare i tuoi, o viceversa. Mentre stai usando i suoi sensi, sei accecato e assordato riguardo i tuoi dintorni.

L'interazione fisica con l'immagine la rivela come illusione, dato che le cose le passano attraverso. Una creatura che usi 3 Azioni per esaminare l'immagine può determinare che è un'illusione con una prova riuscita di Consapevolezza contro la DC del Tiro Salvezza dell'incantesimo. Se una creatura riconosce l'illusione per quello che è, essa può vedere attraverso l'immagine, e qualsiasi suono prodotto dall'immagine le suona artefatto.

\incantesimo{Immagine Silenziosa}
\noindent\colorbox{OBSSgold!10}{
\begin{minipage}{0.95\linewidth}
\begin{description}[noitemsep, topsep=0pt, parsep=0pt, partopsep=0pt, leftmargin=0cm, labelwidth=1.3cm]
	\item[\textbf{Lista}]: Illusione
	\item[\textbf{Livello}]: 1, Comune
	\item[\textbf{Lancio}]: 2 Azioni
	\item[\textbf{Gittata}]: 36 metri
	\item[\textbf{Durata}]: Concentrazione, massimo 3 minuti per CM
\end{description}
\end{minipage}}\smallskip

Crei l'immagine di un oggetto, una creatura o qualche altro fenomeno visibile non più grande di un cubo di 3 metri di spigolo. L'immagine appare in un punto a gittata che puoi vedere e resta per la durata dell'incantesimo. L'immagine è puramente visiva; non è accompagnata da suoni, odori o altri effetti sensoriali. Puoi usare un'Azione per far muovere l'immagine in qualsiasi altro punto a gittata. Quando l'immagine cambia posizione, puoi alterarne l'aspetto così che i suoi movimenti appaiano naturali. Per esempio, se crei l'immagine di una creatura e la muovi, puoi alterare l'immagine in modo che sembri camminare.

L'interazione fisica con l'immagine la rivela come illusione, dato che le cose vi passano attraverso. Una creatura che usa 3 Azioni per esaminare l'immagine può determinare che si tratta di un'illusione con una prova di Consapevolezza contro la DC del Tiro Salvezza del tuo incantesimo. Se una creatura riconosce l'illusione per quello che è, la creatura può vedervi attraverso.

\incantesimo{Immagine Speculare}
\noindent\colorbox{OBSSgold!10}{
\begin{minipage}{0.95\linewidth}
\begin{description}[noitemsep, topsep=0pt, parsep=0pt, partopsep=0pt, leftmargin=0cm, labelwidth=1.3cm]
	\item[\textbf{Lista}]: Illusione
	\item[\textbf{Livello}]: 2, Comune
	\item[\textbf{Lancio}]: 2 Azioni
	\item[\textbf{Gittata}]: Personale
	\item[\textbf{Durata}]: 1 minuto
\end{description}
\end{minipage}}\smallskip

Nel tuo spazio compaiono 2d4 duplicati illusori di te stesso. Fino al termine dell'incantesimo, i duplicati si muovono con te e imitano le tue azioni, scambiandosi di posto in modo da rendere impossibile determinare quale sia l'immagine reale. Puoi usare 1 Azione per congedare i duplicati illusori.

Ogni volta che una creatura ti prende in realtà ha colpito una immagine illusoria.
Se una creatura fa più attacchi a round può disperdere una immagine per ogni attacco andato a buon fine. Se vieni colpito da un incantesimo ad area tutte le immagini svaniscono.

Una creatura che non può vedere, o si affida a sensi diversi dalla vista (come la vista cieca), o che può distinguere le illusioni come false (come la visione del vero), ignora gli effetti di questo incantesimo.

\textbf{Per ogni Successo Critico Magico} ottenuto nella Prova di Magia crei una immagine duplicata in più fino ad un massimo totale di 8 immagini.

\incantesimo{Imprigionare}
\noindent\colorbox{OBSSgold!10}{
\begin{minipage}{0.95\linewidth}
\begin{description}[noitemsep, topsep=0pt, parsep=0pt, partopsep=0pt, leftmargin=0cm, labelwidth=1.3cm]
	\item[\textbf{Lista}]: Abiurazione
	\item[\textbf{Livello}]: 9, Raro
	\item[\textbf{Lancio}]: 2 Azioni
	\item[\textbf{Gittata}]: 9 metri
	\item[\textbf{Durata}]: Fino a dissolvimento
\end{description}
\end{minipage}}\smallskip

Crei dei vincoli magici per bloccare una creatura a gittata e che puoi vedere. Il bersaglio deve superare un Tiro Salvezza su Volontà o essere avvinto dall'incantesimo; se lo supera, è immune all'incantesimo qualora lo lanci di nuovo. Mentre è soggetta a questo incantesimo, la creatura non ha bisogno di respirare, mangiare o bere e non invecchia. Gli incantesimi di divinazione non possono localizzare né percepire il bersaglio.

Quando lanci questo incantesimo, scegli una delle seguenti forme di prigionia.

\begin{itemize}[leftmargin=*] \setlength{\itemsep}{0pt}
	\item \emph{Incatenamento}. Catene pesanti, ben saldate al terreno, tengono il bersaglio ancorato. Il bersaglio è intralciato fino al termine dell'incantesimo, e non può muoversi né essere mosso in alcun modo fino ad allora. La componente speciale per questa versione dell'incantesimo è una catenella di metallo prezioso.
	\item \emph{Isolamento Minimo}. Il bersaglio rimpicciolisce fino a 2,5 centimetri di altezza ed è imprigionato in una gemma o simile oggetto. La luce può attraversare normalmente la gemma (permettendo al bersaglio di vedere all'esterno e ad altre creature di vedere dentro), ma null'altro può attraversarla, neppure tramite teletrasporto o viaggio planare. La gemma non può essere tagliata né infranta finché l'incantesimo rimane in atto. La componente speciale per questa versione dell'incantesimo è una grande gemma trasparente, come il corindone, il diamante o il rubino.
	\item \emph{Prigione Confinata}. L'incantesimo trasporta il bersaglio in un minuscolo semipiano interdetto al teletrasporto e al viaggio planare. Il semipiano può essere un labirinto, una gabbia, una torre, o qualsiasi altra struttura chiusa scelta da te. La componente speciale per questa versione dell'incantesimo è una rappresentazione in miniatura della prigione fatta di giada.
	\item \emph{Sepoltura}. Il bersaglio viene sepolto nelle profondità della terra in una sfera di forza magica grande a sufficienza da contenere il bersaglio. Nulla può attraversare la sfera, né alcuna creatura può teletrasportarsi o usare il viaggio planare per entrarvi o uscire. La componente speciale per questa versione dell'incantesimo è una piccola sfera di mithral.
	\item \emph{Sonno}. Il bersaglio cade addormentato e non può essere risvegliato. La componente speciale per questa versione dell'incantesimo consiste di rare erbe soporifere.
\end{itemize}

\emph{\textbf{Terminare l'incantesimo}}. Durante il lancio dell'incantesimo, in qualsiasi delle sue versioni, puoi specificare una condizione che possa porre fine all'incantesimo e liberare il bersaglio. La condizione può essere tanto specifica o elaborata quanto desideri, ma il Narratore deve concordare che la condizione sia ragionevole e possa avverarsi. Le condizioni possono essere basate sul nome, l'identità o il Patrono di una creatura, ma comunque basate su azioni o qualità percepibili e non su cose intangibili come il livello, le Abilità o i Punti Ferita.

Un incantesimo dissolvi magie può porre fine all'incantesimo solo se lanciato da un personaggio con Competenza Magica almeno 18, che prenda come bersaglio la prigione o la componente materiale usata per crearla.

Puoi usare una particolare componente speciale per creare solo una prigione alla volta. Se lanci l'incantesimo di nuovo usando la stessa componente, il bersaglio del primo lancio dell'incantesimo viene immediatamente liberato dal suo vincolo.

\incantesimo{Inaridire}
\noindent\colorbox{OBSSgold!10}{
\begin{minipage}{0.95\linewidth}
\begin{description}[noitemsep, topsep=0pt, parsep=0pt, partopsep=0pt, leftmargin=0cm, labelwidth=1.3cm]
	\item[\textbf{Lista}]: Necromanzia
	\item[\textbf{Livello}]: 4, Non Comune
	\item[\textbf{Lancio}]: 2 Azioni
	\item[\textbf{Gittata}]: 9 metri
	\item[\textbf{Durata}]: Istantanea
\end{description}
\end{minipage}}\smallskip
Energia necromantica avvolge una creatura di tua scelta a gittata e che puoi vedere, deprivandola di linfa e vitalità. Il bersaglio deve effettuare un Tiro Salvezza su Tempra. Se fallisce il Tiro Salvezza il bersaglio subisce 8d8 danni da Vuoto, o la metà di questi danni se supera il Tiro Salvezza. L'incantesimo non ha effetto su non morti o costrutti.

Se il bersaglio è un vegetale non magico che non sia anche una creatura, come un albero o un cespuglio, non effettua alcun Tiro Salvezza, avvizzisce e muore all'istante.

\textbf{Per ogni Successo Critico Magico} ottenuto nella Prova di Magia il danno aumenta di 4d8.

\textbf{Tiro Salvezza Successo/Fallimento Critico}: In caso di Fallimento Critico il danno raddoppia, in caso di Successo Critico il danno viene ulteriormente dimezzato

\incantesimo{Individuazione del Magico}\index{Occhi della Magia}
\begin{description}[noitemsep, topsep=0pt, parsep=0pt, partopsep=0pt, leftmargin=0cm, labelwidth=1.3cm]
	\item[\textbf{Lista}]: Universale
	\item[\textbf{Livello}]: 1, Comune
	\item[\textbf{Lancio}]: 2 Azioni
	\item[\textbf{Gittata}]: Personale
	\item[\textbf{Durata}]: 1d4 +1 round per CM
\end{description}

Per la durata, percepisci la presenza della magia entro 9 metri da te. Puoi usare 1 Azione per vedere una flebile aura che si estende intorno a qualsiasi creatura o oggetto visibile nell'area che rechi magia. Con due Azioni ne apprendi anche la Liste di Magia, se ce l'ha.

L'incantesimo può penetrare la maggior parte delle barriere, ma è bloccato da 30 centimetri di pietra, 2,5 centimetri di metallo comune, un sottile foglio di piombo o 1 metro di legno o terra. Questo incantesimo non permette di vedere cose influenzate dall'incantesimo \hyperlink{Invisibilità}{Invisibilità}.

\textbf{Per ogni Successo Critico Magico} ottenuto nella Prova di Magia la durata aumenta di 3 round.

\incantesimo{Individuazione delle Malattie e dei Veleni}
\noindent\colorbox{OBSSgold!10}{
\begin{minipage}{0.95\linewidth}
\begin{description}[noitemsep, topsep=0pt, parsep=0pt, partopsep=0pt, leftmargin=0cm, labelwidth=1.3cm]
	\item[\textbf{Lista}]: Divinazione
	\item[\textbf{Livello}]: 1, Non Comune
	\item[\textbf{Lancio}]: 2 Azioni
	\item[\textbf{Gittata}]: Personale
	\item[\textbf{Durata}]: 1 round per CM
\end{description}
\end{minipage}}\smallskip

Per la durata, percepisci la presenza e posizione di veleni, creature velenose e malattie entro 9 metri da te. Inoltre riesci a identificare il tipo di veleno, creatura velenosa o malattia. L'incantesimo può penetrare la maggior parte delle barriere, ma è bloccato da 30 centimetri di pietra, 2,5 centimetri di metallo comune, un sottile foglio di piombo o 1 metro di legno o terra.

\textbf{Per ogni Successo Critico Magico} ottenuto nella Prova di Magia durata raddoppia.

\incantesimo{Individuazione dei Pensieri}
\noindent\colorbox{OBSSgold!10}{
\begin{minipage}{0.95\linewidth}
\begin{description}[noitemsep, topsep=0pt, parsep=0pt, partopsep=0pt, leftmargin=0cm, labelwidth=1.3cm]
	\item[\textbf{Lista}]: Divinazione
	\item[\textbf{Livello}]: 2, Raro
	\item[\textbf{Lancio}]: 2 Azioni
	\item[\textbf{Gittata}]: Personale
	\item[\textbf{Durata}]: 1 minuto
\end{description}
\end{minipage}}\smallskip

Per la durata, puoi leggere i pensieri di certe creature. Quando lanci questo incantesimo e con altre due Azioni in ciascun round successivo sino al termine dell'incantesimo, puoi concentrare la tua mente su qualsiasi creatura che tu possa vedere e si trovi entro 9 metri da te. Se la creatura che hai scelto ha un punteggio di Intelligenza -3 o meno o non parla nessun linguaggio, la creatura ignora l'effetto.

Inizialmente, apprendi solo i pensieri di superficie della creatura: quelli più ricorrenti. Con un'Azione, puoi o spostare la tua attenzione sui pensieri di un'altra creatura o tentare di sondare più a fondo la mente della stessa creatura. Se sondi più a fondo, il bersaglio deve effettuare un Tiro Salvezza su Volontà. Se lo fallisce, ottieni una percezione dei suoi ragionamenti (se ve ne sono), del suo stato emotivo, e di ogni cosa abbia prevalenza nei suoi pensieri (come una preoccupazione, l'amore, o l'odio). Se supera il Tiro Salvezza, l'incantesimo termina. A ogni modo, il bersaglio sa che stai sondando la sua mente e, a meno che non sposti la tua attenzione verso la mente di un'altra creatura, nel suo round la creatura può usare la 2 Azioni per effettuare un Tiro Salvezza su Volontà contrapposto; se la vince, l'incantesimo termina.

Le domande poste verbalmente alla creatura bersaglio, ovviamente, modellano il corso dei suoi pensieri, cosicché questo incantesimo risulta particolarmente efficace negli interrogatori.

Puoi anche usare questo incantesimo per individuare la presenza di creature pensanti che non puoi vedere. Quando lanci questo incantesimo o con 2 Azioni nella sua durata, puoi cercare pensieri entro 9 metri da te. L'incantesimo può penetrare le barriere, ma è bloccato da 60 centimetri di pietra, 5 centimetri di metallo che non sia il piombo, o un sottile foglio di piombo. Non puoi individuare una creatura con Intelligenza -3 o meno, o una creatura che non parla alcun linguaggio. Una volta individuata in questo modo la presenza di una creatura, puoi leggerne i pensieri per la durata dell'incantesimo finché resta nella gittata, come descritto sopra, anche se non puoi vederla.
Mentre hai attivo questo incantesimo per il lancio di altri incantesimo risulterai Distratto.

\incantesimo{Infliggi Ferite}
\noindent\colorbox{OBSSgold!10}{
\begin{minipage}{0.95\linewidth}
\begin{description}[noitemsep, topsep=0pt, parsep=0pt, partopsep=0pt, leftmargin=0cm, labelwidth=1.3cm]
	\item[\textbf{Lista}]: Necromanzia
	\item[\textbf{Livello}]: 2, Comune
	\item[\textbf{Lancio}]: 2 Azioni
	\item[\textbf{Gittata}]: Contatto
	\item[\textbf{Durata}]: Istantanea
\end{description}
\end{minipage}}\smallskip

Effettua un attacco in mischia con incantesimo contro una creatura a portata. Se colpisci, il bersaglio subisce 3d10 danni da Vuoto, Tiro Salvezza su Tempra per dimezzare.

\textbf{Per ogni Successo Critico Magico} ottenuto nella Prova di Magia il danno aumenta di 2d8.

\incantesimo{Ingrandire/Ridurre}
\noindent\colorbox{OBSSgold!10}{
\begin{minipage}{0.95\linewidth}
\begin{description}[noitemsep, topsep=0pt, parsep=0pt, partopsep=0pt, leftmargin=0cm, labelwidth=1.3cm]
	\item[\textbf{Lista}]: Trasmutazione
	\item[\textbf{Livello}]: 2, Comune
	\item[\textbf{Lancio}]: 2 Azioni
	\item[\textbf{Gittata}]: 9 metri
	\item[\textbf{Durata}]: 1 minuto
\end{description}
\end{minipage}}\smallskip

Fai sì che una creatura od oggetto a gittata e che puoi vedere ingrandisca o rimpicciolisca per la durata dell'incantesimo. Scegli una creatura o un oggetto che non sia né indossato né trasportato. Se il bersaglio non è consenziente, può effettuare un Tiro Salvezza su Tempra, se lo supera, l'incantesimo non ha effetto. Se il bersaglio è una creatura, tutto ciò che sta indossando e trasportando cambia taglia assieme a essa. Qualsiasi oggetto lasciato cadere da una creatura soggetta a questo incantesimo ritorna subito alla sua taglia normale.

\begin{itemize}[leftmargin=*] \setlength{\itemsep}{0pt}
	\item \emph{Ingrandire}. La dimensione del bersaglio raddoppia in tutte le dimensioni, e il suo peso è moltiplicato per otto. Questa crescita aumenta la sua taglia di una categoria: da Media a Grande, per esempio. Se non c'è spazio sufficiente perché il bersaglio raddoppi la sua taglia, la creatura od oggetto assume la taglia più grossa possibile permessagli dallo spazio disponibile. Fino al termine dell'incantesimo, il bersaglio ha +1d6 alle Azioni basate su Forza e ai Tiri Salvezza su Tempra. Le armi del bersaglio crescono per raggiungere la nuova taglia. Mentre queste armi sono ingrandite, gli attacchi del bersaglio con esse faranno una categoria di danno ulteriore.
	\item \emph{Ridurre}. La dimensione del bersaglio si dimezza in tutte le dimensioni, e il suo peso è ridotto a un ottavo. Questa crescita diminuisce la sua taglia di una categoria: da Media a Piccola, per esempio. Fino al termine dell'incantesimo, il bersaglio ha -1d6 alle Azioni basate su Forza e ai Tiri Salvezza su Tempra. Le armi del bersaglio rimpiccioliscono per raggiungere la nuova taglia. Mentre queste armi sono rimpicciolite, gli attacchi del bersaglio con esse faranno una categoria di danno inferiore (ma senza ridurre il danno dell'arma a meno di 1).
\end{itemize}

\textbf{Per ogni due Critici ottenuti} nella Prova di Magia la creatura aumenta di un altra taglia, oppure influenzi un altra creature entro 6 metri dalla prima.

\incantesimo{Insetto Gigante}
\noindent\colorbox{OBSSgold!10}{
\begin{minipage}{0.95\linewidth}
\begin{description}[noitemsep, topsep=0pt, parsep=0pt, partopsep=0pt, leftmargin=0cm, labelwidth=1.3cm]
	\item[\textbf{Lista}]: Animali e Piante
	\item[\textbf{Livello}]: 4, Non Comune
	\item[\textbf{Lancio}]: 2 Azioni
	\item[\textbf{Gittata}]: 9 metri
	\item[\textbf{Durata}]: 10 minuti
\end{description}
\end{minipage}}\smallskip

Per la durata dell'incantesimo, trasformi fino a dieci centopiedi, tre ragni, cinque vespe o uno scorpione a gittata, in versioni giganti della loro forma naturale. Un centopiedi diventa un centopiedi gigante, un ragno diventa un ragno gigante, una vespa diventa una vespa gigante e uno scorpione diventa uno scorpione gigante. Ogni creatura obbedisce ai tuoi comandi vocali e, in combattimento, agisce in ciascun round durante il tuo round. Il Narratore possiede le statistiche di queste creature, e sarà sempre Il Narratore a risolvere le loro azioni e i loro movimenti. Una creatura resta nella sua forma gigante per la durata, finché non scende a 0 Punti Ferita, o finché non usi un'Azione per interrompere l'effetto su di essa.

Il Narratore può permetterti di scegliere bersagli differenti. Per esempio, se trasformi un'ape, la sua versione gigante potrebbe avere le stesse statistiche della vespa gigante.

\incantesimo{Interdizione alla Morte}
\noindent\colorbox{OBSSgold!10}{
\begin{minipage}{0.95\linewidth}
\begin{description}[noitemsep, topsep=0pt, parsep=0pt, partopsep=0pt, leftmargin=0cm, labelwidth=1.3cm]
	\item[\textbf{Lista}]: Necromanzia
	\item[\textbf{Livello}]: 4, Non Comune
	\item[\textbf{Lancio}]: 2 Azioni
	\item[\textbf{Gittata}]: Contatto
	\item[\textbf{Durata}]: 1 ora
\end{description}
\end{minipage}}\smallskip

Lanci l'incantesimo a contatto con una creatura. Conferisci al bersaglio protezione dalla morte. La prima volta che il bersaglio dovesse scendere a 0 Punti Ferita in seguito al danno subito, il bersaglio scende invece a 1 punto ferita e l'incantesimo ha fine. Se l'incantesimo è ancora attivo quando il bersaglio è vittima di un effetto che lo ucciderebbe all'istante senza infliggere danni,quell'effetto viene invece negato sul bersaglio e l'incantesimo ha fine.

\textbf{Per ogni due Successo Critico Magico ottenuto} nella Prova di Magia l'incantesimo protegge una volta in più o protegge un altra creatura.

\incantesimo{Intermittenza}
\noindent\colorbox{OBSSgold!10}{
\begin{minipage}{0.95\linewidth}
\begin{description}[noitemsep, topsep=0pt, parsep=0pt, partopsep=0pt, leftmargin=0cm, labelwidth=1.3cm]
	\item[\textbf{Lista}]: Trasmutazione
	\item[\textbf{Livello}]: 3, Non Comune
	\item[\textbf{Lancio}]: 2 Azioni
	\item[\textbf{Gittata}]: Personale
	\item[\textbf{Durata}]: 1 round per CM
\end{description}
\end{minipage}}\smallskip

Tira un 1d6 alla fine di ciascun tuo round per la durata di questo incantesimo. Se ottieni un numero dispari svanisci dal tuo attuale piano di esistenza e riappari sul Piano Etereo (l'incantesimo fallisce e il lancio è sprecato qualora tu fossi già su quel piano). All'inizio del tuo prossimo round, e quando l'incantesimo termina, qualora tu fossi sul Piano Etereo, ritorni in uno spazio non occupato di tua scelta e che puoi vedere, entro 3 metri dallo spazio da cui sei svanito. Se nessuno spazio non occupato è disponibile entro questa gittata, compari nello spazio non occupato più vicino (determinato casualmente se è disponibile più di uno spazio). Puoi interrompere l'incantesimo con un'Azione.

Mentre sei sul Piano Etereo, puoi vedere e udire il piano da cui provieni, che percepisci in sfumature di grigio, ma non puoi comunque percepire nulla che si trovi a più di 18 metri di distanza. Puoi interagire solo con creature che si trovano sul Piano Etereo. Le creature che non si trovano lì non possono né percepirti né interagire con te, a meno che non siano provviste della capacità di farlo.

\incantesimo{Intralciare}
\noindent\colorbox{OBSSgold!10}{
\begin{minipage}{0.95\linewidth}
\begin{description}[noitemsep, topsep=0pt, parsep=0pt, partopsep=0pt, leftmargin=0cm, labelwidth=1.3cm]
	\item[\textbf{Lista}]: Animali e Piante
	\item[\textbf{Livello}]: 1, Comune
	\item[\textbf{Lancio}]: 2 Azioni
	\item[\textbf{Gittata}]: 27 metri
	\item[\textbf{Durata}]: 1 minuto
\end{description}
\end{minipage}}\smallskip

Rampicanti e rami stritolanti spuntano dal terreno in un quadrato di 6 metri di lato a partire da un punto a gittata. Per la durata, questi vegetali trasformano il terreno nell'area in terreno difficile.

Una creatura nell'area nel momento in cui lanci questo incantesimo deve superare un Tiro Salvezza su Tempra o restare intralciata da questi vegetali fino al termine dell'incantesimo. Una creatura \hyperlink{intralciato}{intralciata} (vedi pag. \pageref{intralciato}) dai vegetali può usare due Azioni per effettuare un nuovo Tiro Salvezza. Se la supera, si libera. Quando l'incantesimo ha termine, i vegetali evocati svaniscono.

\incantesimo{Inversione della Gravità}
\noindent\colorbox{OBSSgold!10}{
\begin{minipage}{0.95\linewidth}
\begin{description}[noitemsep, topsep=0pt, parsep=0pt, partopsep=0pt, leftmargin=0cm, labelwidth=1.3cm]
	\item[\textbf{Lista}]: Trasmutazione
	\item[\textbf{Livello}]: 7, Raro
	\item[\textbf{Lancio}]: 2 Azioni
	\item[\textbf{Gittata}]: 30 metri
	\item[\textbf{Durata}]: 1 minuto
\end{description}
\end{minipage}}\smallskip

Questo incantesimo inverte la gravità in un cilindro di raggio 15 metri, alto 30 metri, centrato in un punto a gittata. Quando lanci questo incantesimo, tutte le creature e gli oggetti che non sono in qualche modo ancorati al terreno cadono verso l'alto e raggiungono la cima dell'area. Una creatura può tentare un Tiro Salvezza su Riflessi per afferrare un oggetto fisso a portata, per evitare di cadere in questo modo, in caso lo superi.

Se lungo questa caduta si incontra un oggetto solido (il soffitto), gli oggetti e le creature che cadono vi impattano come accadrebbe durante una normale caduta. Se un oggetto o creatura raggiunge la cima dell'area senza colpire nulla, rimane lì, oscillando lievemente, per la durata.

Al termine della durata, gli oggetti e le creature colpite ricadono verso il basso.

\incantesimo{Inviare}
\noindent\colorbox{OBSSgold!10}{
\begin{minipage}{0.95\linewidth}
\begin{description}[noitemsep, topsep=0pt, parsep=0pt, partopsep=0pt, leftmargin=0cm, labelwidth=1.3cm]
	\item[\textbf{Lista}]: Invocazione
	\item[\textbf{Livello}]: 3, Comune
	\item[\textbf{Lancio}]: 2 Azioni
	\item[\textbf{Gittata}]: Illimitata
	\item[\textbf{Durata}]: 1 round
\end{description}
\end{minipage}}\smallskip

Invii un breve messaggio di 25 parole o meno a una creatura con cui sei familiare. La creatura sente il messaggio nella sua mente, ti riconosce come mittente e può risponderti in modo simile. L'incantesimo permette a creature con un punteggio di Intelligenza almeno di -2 di comprendere il significato del tuo messaggio anche se non comprende la tua lingua.

Puoi inviare il messaggio attraverso qualsiasi distanza e anche su altri piani di esistenza, ma se il bersaglio è su di un piano diverso dal tuo, c'è una probabilità del 5\% che il messaggio non arrivi.

\textbf{Per ogni Successo Critico Magico} ottenuto nella Prova di Magia aumenti di 25 parole il messaggio o di un round la durata.

\textbf{NOTA}: i Devoti di Nethergal ottengono un Successo Critico Magico in automatico al lancio dell'incantesimo

\incantesimo{Invisibilità}
\noindent\colorbox{OBSSgold!10}{
\begin{minipage}{0.95\linewidth}
\begin{description}[noitemsep, topsep=0pt, parsep=0pt, partopsep=0pt, leftmargin=0cm, labelwidth=1.3cm]
	\item[\textbf{Lista}]: Illusione
	\item[\textbf{Livello}]: 2, Comune
	\item[\textbf{Lancio}]: 2 Azioni
	\item[\textbf{Gittata}]: Contatto
	\item[\textbf{Durata}]: 1 minuto per CM
\end{description}
\end{minipage}}\smallskip

Lanci l'incantesimo a contatto di una creatura. Il bersaglio diventa invisibile fino alla fine dell'incantesimo. Qualsiasi cosa il bersaglio stia indossando o trasportando diventa invisibile finché resta sul bersaglio. L'incantesimo ha fine per il bersaglio che attacca od esegue un incantesimo.

\textbf{Per ogni Successo Critico Magico} ottenuto nella Prova di Magia puoi scegliere un'ulteriore creatura bersaglio o aumenti 1 minuto la durata.

\incantesimo{Invisibilità Superiore}
\noindent\colorbox{OBSSgold!10}{
\begin{minipage}{0.95\linewidth}
\begin{description}[noitemsep, topsep=0pt, parsep=0pt, partopsep=0pt, leftmargin=0cm, labelwidth=1.3cm]
	\item[\textbf{Lista}]: Illusione
	\item[\textbf{Livello}]: 4, Non Comune
	\item[\textbf{Lancio}]: 2 Azioni
	\item[\textbf{Gittata}]: Contatto
	\item[\textbf{Durata}]: 1 minuto
\end{description}
\end{minipage}}\smallskip

Lanci l'incantesimo a contatto di una creatura. Il bersaglio diventa invisibile fino alla fine dell'incantesimo. Qualsiasi cosa indossata o trasportata dal bersaglio diventa invisibile finché resta addosso al bersaglio.

Eseguire incantesimi o azioni di attacco non fa diventare visibile.

\incantesimo{Invocare il Fulmine}
\noindent\colorbox{OBSSgold!10}{
\begin{minipage}{0.95\linewidth}
\begin{description}[noitemsep, topsep=0pt, parsep=0pt, partopsep=0pt, leftmargin=0cm, labelwidth=1.3cm]
	\item[\textbf{Lista}]: Aria
	\item[\textbf{Livello}]: 3, Comune
	\item[\textbf{Lancio}]: 1 round
	\item[\textbf{Gittata}]: 36 metri
	\item[\textbf{Durata}]: Concentrazione, massimo 10 minuti
\end{description}
\end{minipage}}\smallskip

Una nube di tempesta compare nella forma di un cilindro alto 3 metri con un raggio di 18 metri, centrato su di un punto che puoi vedere, 30 metri sopra di te. L'incantesimo fallisce automaticamente se non puoi vedere il punto nell'aria dove apparirà la nube di tempesta (per esempio, se sei in una stanza che non può accogliere la nube). Quando lanci l'incantesimo, scegli un punto che puoi vedere entro la gittata. Un fulmine si abbatterà dalla nuvola su quel punto. Ogni creatura entro 1 metro da quel punto deve effettuare un Tiro Salvezza su Riflessi. Una creatura subisce 3d10 danni da elettricità se fallisce il Tiro Salvezza, o la metà di questi danni se lo supera. Durante ciascun tuo round fino al termine dell'incantesimo, puoi usare due Azioni per richiamare un altro fulmine in questo modo, prendendo come bersaglio lo stesso punto o uno diverso.

Se quando lanci questo incantesimo ti trovi all'esterno in condizioni di tempesta, l'incantesimo ti fornisce il controllo della tempesta esistente invece di crearne una nuova. Sotto queste condizioni, il danno dell'incantesimo aumenta di 1d10.

\textbf{Per ogni Successo Critico Magico} ottenuto nella Prova di Magia il danno aumenta di 2d6.

\incantesimo{Labirinto}
\noindent\colorbox{OBSSgold!10}{
\begin{minipage}{0.95\linewidth}
\begin{description}[noitemsep, topsep=0pt, parsep=0pt, partopsep=0pt, leftmargin=0cm, labelwidth=1.3cm]
	\item[\textbf{Lista}]: Evocazione
	\item[\textbf{Livello}]: 8, Raro
	\item[\textbf{Lancio}]: 2 Azioni
	\item[\textbf{Gittata}]: 18 metri
	\item[\textbf{Durata}]: massimo 10 minuti
\end{description}
\end{minipage}}\smallskip

Bandisci una creatura a gittata e che puoi vedere in un semipiano labirintico. Il bersaglio rimane lì per la durata dell'incantesimo o finché non fugge dal labirinto. Il bersaglio può impiegare 3 Azioni per tentare di fuggire a partire dal secondo round. Quando lo fa, effettua un Tiro Salvezza su Volontà. Se la supera, fugge, e l'incantesimo termina (un minotauro o un demone goristro riescono automaticamente).

Quando l'incantesimo termina, il bersaglio riappare nello spazio che aveva lasciato o, se quello spazio è occupato nel più vicino spazio non occupato.

\textbf{Per ogni Successo Critico Magico} ottenuto nella Prova di Magia la durata aumenta di 10 minuti. Con due Successo Critico Magico puoi influenzare un altra creatura.

\incantesimo{Lacrima di Laydel}
\noindent\colorbox{OBSSgold!10}{
\begin{minipage}{0.95\linewidth}
\begin{description}[noitemsep, topsep=0pt, parsep=0pt, partopsep=0pt, leftmargin=0cm, labelwidth=1.3cm]
	\item[\textbf{Lista}]: Invocazione
	\item[\textbf{Livello}]: 2, Molto Raro/Comune
	\item[\textbf{Lancio}]: 2 Azione/1 Azione
	\item[\textbf{Gittata}]: 36 metri
	\item[\textbf{Durata}]: Istantaneo
\end{description}
\end{minipage}}\smallskip

L'incantatore permea di magia una lacrima che getta contro l'avversario, è necessario un Tiro per Colpire con Incantesimi a distanza.
La creatura subisce 1d6+2d6 di danno, per stabilire il tipo di danno consultare la tabella con i valori del primo d6 tirato.

\medskip

\noindent\begin{tabular}{l|l}
	\toprule
 \rowcolor{gray!20}\textbf{1d6}&\textbf{Energia}\\
	\toprule
	1 &Fuoco\\
 \rowcolor{gray!20}2 &Elettricità\\
	3 &Freddo\\
 \rowcolor{gray!20}4 &Suono\\
	5 &Vuoto\\
 \rowcolor{gray!20}6 &Forza
\end{tabular}

\medskip

Il danno che l'obiettivo subisce è del tipo di Energia che risulta dal primo d6. Se il primo dado è un 6 ed è 6 anche uno degli altri dadi allora tira nuovamente 1d6 e somma al danno.

Per un Devoto di Laydel questo incantesimo è Comune e ha un tempo di lancio di 1 Azione. Inoltre può continuare a tirare ulteriori d6 di danno purché continui a tirare 6 con quel dado.

\incantesimo{Lacrima di Ljust}
\noindent\colorbox{OBSSgold!10}{
\begin{minipage}{0.95\linewidth}
\begin{description}[noitemsep, topsep=0pt, parsep=0pt, partopsep=0pt, leftmargin=0cm, labelwidth=1.3cm]
	\item[\textbf{Lista}]: Universale
	\item[\textbf{Livello}]: 0, Non Comune
	\item[\textbf{Lancio}]: 1 Azione
	\item[\textbf{Gittata}]: Personale
	\item[\textbf{Durata}]: 10 round
\end{description}
\end{minipage}}\smallskip

L'incantatore permea di magia un piccolo oggetto che incomincia a brillare di luce. La luce si illumina il suo quadretto ed un ulteriore metro attorno, oltre non genera luce fioca. La durata dell'incantesimo è 10 round. L'incantatore può lanciare l'oggetto entro 18 metri e deve rimanere entro questa distanza. Non è possibile lanciare l'incantesimo più volte al giorno di quanti Punti Fato si possiedono.

\incantesimo{Lama Infuocata}
\noindent\colorbox{OBSSgold!10}{
\begin{minipage}{0.95\linewidth}
\begin{description}[noitemsep, topsep=0pt, parsep=0pt, partopsep=0pt, leftmargin=0cm, labelwidth=1.3cm]
	\item[\textbf{Lista}]: Fuoco
	\item[\textbf{Livello}]: 2, Comune
	\item[\textbf{Lancio}]: 1 Azione
	\item[\textbf{Gittata}]: Personale
	\item[\textbf{Durata}]: Concentrazione, massimo 10 minuti
\end{description}
\end{minipage}}\smallskip

Crei nella tua mano una lama infuocata. La lama è simile in dimensioni e forma a una scimitarra e rimane per la durata. Se lasci andare la lama, questa sparisce, ma ne puoi creare un'altra con un'Azione. Puoi usare 2 Azioni per effettuare un attacco in mischia con la lama infuocata. Se colpisci, il bersaglio subisce 3d6 danni da fuoco. La lama infuocata emana luce intensa in un raggio di 3 metri e luce fioca per 6 metri.

\textbf{Per ogni due Critici ottenuti} nella Prova di Magia il danno aumenta di 1d6.

\incantesimo{Lanciafiamme}
\noindent\colorbox{OBSSgold!10}{
\begin{minipage}{0.95\linewidth}
\begin{description}[noitemsep, topsep=0pt, parsep=0pt, partopsep=0pt, leftmargin=0cm, labelwidth=1.3cm]
	\item[\textbf{Lista}]: Fuoco
	\item[\textbf{Livello}]: 2, Raro
	\item[\textbf{Lancio}]: 2 Azioni
	\item[\textbf{Gittata}]: Personale
	\item[\textbf{Durata}]: 1 minuto
\end{description}
\end{minipage}}\smallskip

Una fiammella compare al termine del tubo di metallo che tieni nella mano. La fiamma resta lì per la durata dell'incantesimo e non danneggia né te né il tuo equipaggiamento. La fiamma produce luce intensa nel raggio di 1 metro e luce fioca nel raggio di 1 metro. L'incantesimo termina se lo interrompi con un'Azione o se lo lanci di nuovo.

Con un Tiro per Colpire con incantesimo a distanza e spendendo 1 Azione puoi allungare la fiamma fino a 9 metri per colpire un bersaglio. Se colpisci, il bersaglio subisce 2d6 danni da fuoco, finché mantieni lo stesso bersaglio hai un +2 al colpire con l'attacco successivo.

\textbf{Per ogni Critico ottenuto} nella Prova di Magia il danno aumenta di 1d6.

\incantesimo{Legame Telepatico}
\noindent\colorbox{OBSSgold!10}{
\begin{minipage}{0.95\linewidth}
\begin{description}[noitemsep, topsep=0pt, parsep=0pt, partopsep=0pt, leftmargin=0cm, labelwidth=1.3cm]
	\item[\textbf{Lista}]: Divinazione
	\item[\textbf{Livello}]: 5, Raro
	\item[\textbf{Lancio}]: 2 Azioni
	\item[\textbf{Gittata}]: 9 metri
	\item[\textbf{Durata}]: 1 ora
\end{description}
\end{minipage}}\smallskip

Stabilisci un collegamento telepatico tra un massimo di otto creature consenzienti a gittata di tua scelta, collegando psichicamente ciascuna creatura alle altre per la durata dell'incantesimo. Le creature con punteggio di Intelligenza -3 o meno ignorano questo incantesimo. Fino al termine dell'incantesimo, i bersagli possono comunicare telepaticamente tramite questo legame, che condividano o meno un linguaggio comune. La comunicazione è possibile a qualsiasi distanza, ma non può estendersi su differenti piani di esistenza.

\textbf{Per ogni Successo Critico Magico} ottenuto nella Prova di Magia la durata aumenta di 1 ora.

\incantesimo{Lentezza}\hypertarget{lentezza}{}
\noindent
\begin{description}[noitemsep, topsep=0pt, parsep=0pt, partopsep=0pt, leftmargin=0cm, labelwidth=1.3cm]
	\item[\textbf{Lista}]: Trasmutazione
	\item[\textbf{Livello}]: 3, Non Comune
	\item[\textbf{Lancio}]: 2 Azioni
	\item[\textbf{Gittata}]: 36 metri
	\item[\textbf{Durata}]: 1 minuto, Concentrazione
\end{description}

Rallenti il metabolismo di massimo 2 più le volte che hai preso Adepto della Magia creature a tua scelta in un raggio di 3 metri a gittata. Al lancio dell'incantesimo ciascun bersaglio deve superare un Tiro Salvezza su Volontà od eseguire una Azione in meno a round per la durata dell'incantesimo.

Questo incantesimo \hyperlink{contrastareincantesimi}{contrasta ed è contrastato} da \hyperlink{Velocità}{Velocità}.

\textbf{Per ogni Successo Critico Magico} ottenuto nella Prova di Magia puoi influenzare una creatura in più.

\textbf{Tiro Salvezza Fallimento Critico}: In caso di Fallimento Critico si viene rallentati di una ulteriore Azione.

\incantesimo{Lettura della terra di Kyrin}
\noindent\colorbox{OBSSgold!10}{
\begin{minipage}{0.95\linewidth}
\begin{description}[noitemsep, topsep=0pt, parsep=0pt, partopsep=0pt, leftmargin=0cm, labelwidth=1.3cm]
	\item[\textbf{Lista}]: Terra
	\item[\textbf{Livello}]: 2, Non Comune
	\item[\textbf{Lancio}]: 1 Round
	\item[\textbf{Gittata}]: Personale (raggio di 30 metri)
	\item[\textbf{Durata}]: Istantanea
\end{description}
\end{minipage}}\smallskip

Appoggi le mani sulla terra e formulato l'incantesimo hai una fugace visione dell'ambiente intorno a te nel raggio sferico di 30 metri.
Riesci a percepire la posizione e relativa forma delle creature e delle strutture che poggiano a terra.

\textbf{Per ogni Successo Critico Magico} ottenuto nella Prova di Magia il raggio aumenta di 10 metri.

\incantesimo{Levitazione}
\noindent\colorbox{OBSSgold!10}{
\begin{minipage}{0.95\linewidth}
\begin{description}[noitemsep, topsep=0pt, parsep=0pt, partopsep=0pt, leftmargin=0cm, labelwidth=1.3cm]
	\item[\textbf{Lista}]: Aria
	\item[\textbf{Livello}]: 2, Comune
	\item[\textbf{Lancio}]: 2 Azioni
	\item[\textbf{Gittata}]: 18 metri
	\item[\textbf{Durata}]: 10 minuti
\end{description}
\end{minipage}}\smallskip

Una creatura o oggetto a gittata che puoi vedere, scelto da te, si alza verticalmente fino a 6 metri e rimane sospeso per la durata dell'incantesimo. L'incantesimo può levitare un bersaglio pesante fino a 250 chili. Una creatura non consenziente che superi un Tiro Salvezza su Tempra ignora l'effetto.

Il bersaglio può muoversi solo spingendo o tirando verso un oggetto fisso o superficie a portata (per esempio un muro o un soffitto). Durante il tuo round puoi cambiare l'altitudine del bersaglio fino a 6 metri in entrambe le direzioni. Se sei tu il bersaglio, ti puoi muovere verso l'alto o il basso come parte del tuo movimento. Altrimenti puoi usare 1 Azione per muovere il bersaglio, che deve rimanere nella gittata dell'incantesimo. Quando l'incantesimo termina, qualora sia ancora in aria, il bersaglio fluttua dolcemente a terra.

Mentre sei sotto l'influenza di questo incantesimo sei considerato Distratto nel lancio di incantesimi.

\textbf{Per ogni Successo Critico Magico} ottenuto nella Prova di Magia puoi spostarti di 1 metro lateralmente o influenzi un altra creatura.

\incantesimo{Lettura del Magico}
\noindent\colorbox{OBSSgold!10}{
\begin{minipage}{0.95\linewidth}
\begin{description}[noitemsep, topsep=0pt, parsep=0pt, partopsep=0pt, leftmargin=0cm, labelwidth=1.3cm]
	\item[\textbf{Lista}]: Universale
	\item[\textbf{Livello}]: 1, Comune
	\item[\textbf{Lancio}]: 1 Azione
	\item[\textbf{Gittata}]: Contatto
	\item[\textbf{Durata}]: 1 minuto, finché usato
\end{description}
\end{minipage}}\smallskip

L'incantatore conferisce la capacità di leggere una pergamena o una scritta magica ad un bersaglio. Per la durata di 1 minuto o finché usato una volta, quale venga prima, la creatura riesce automaticamente a comprendere una pergamena magica od a lanciare il contenuto della pergamena rispettando i criteri e regole di lancio incantesimi da pergamena.

\textbf{Per ogni Successo Critico Magico} ottenuto nella Prova di Magia puoi leggere o comprendere una pergamena in più.

\incantesimo{Libertà di Movimento}
\noindent\colorbox{OBSSgold!10}{
\begin{minipage}{0.95\linewidth}
\begin{description}[noitemsep, topsep=0pt, parsep=0pt, partopsep=0pt, leftmargin=0cm, labelwidth=1.3cm]
	\item[\textbf{Lista}]: Abiurazione
	\item[\textbf{Livello}]: 4, Comune
	\item[\textbf{Lancio}]: 2 Azioni
	\item[\textbf{Gittata}]: Contatto
	\item[\textbf{Durata}]: 1 ora
\end{description}
\end{minipage}}\smallskip

Lanci l'incantesimo a contatto di una creatura consenziente. Per la sua durata, il movimento del bersaglio ignora il terreno difficile naturale, mentre gli incantesimi o altri effetti magici non possono ridurre la sua velocità né far sì che il bersaglio sia paralizzato o intralciato se hanno un DC inferiore a quella dell'incantesimo stesso.

Il bersaglio può usare due Azioni per liberarsi automaticamente da qualsiasi restrizione non magica, come manette o una creatura da cui è afferrato. Infine, trovarsi sott'acqua non comporta penalità al movimento o gli attacchi del bersaglio.

\textbf{Per due Successo Critico Magico ottenuto} nella Prova di Magia puoi influenzare un altra creatura.

\incantesimo{Lingue}
\noindent\colorbox{OBSSgold!10}{
\begin{minipage}{0.95\linewidth}
\begin{description}[noitemsep, topsep=0pt, parsep=0pt, partopsep=0pt, leftmargin=0cm, labelwidth=1.3cm]
	\item[\textbf{Lista}]: Divinazione
	\item[\textbf{Livello}]: 3, Comune
	\item[\textbf{Lancio}]: 2 Azioni
	\item[\textbf{Gittata}]: Contatto
	\item[\textbf{Durata}]: 1 ora
\end{description}
\end{minipage}}\smallskip

Questo incantesimo conferisce alla creatura con cui sei stato in contatto al momento del lancio dell'incantesimo la capacità di comprendere qualsiasi linguaggio parlata che ode. Inoltre, quando il bersaglio parla, qualsiasi creatura che conosca almeno un linguaggio e può udire il bersaglio, comprende ciò che dice.

\textbf{Per ogni Successo Critico Magico} ottenuto nella Prova di Magia la durata raddoppia o influenzi un'altra creatura.

\incantesimo{Localizza Animali e Piante}
\noindent\colorbox{OBSSgold!10}{
\begin{minipage}{0.95\linewidth}
\begin{description}[noitemsep, topsep=0pt, parsep=0pt, partopsep=0pt, leftmargin=0cm, labelwidth=1.3cm]
	\item[\textbf{Lista}]: Animali e Piante
	\item[\textbf{Livello}]: 2, Non Comune
	\item[\textbf{Lancio}]: 2 Azioni
	\item[\textbf{Gittata}]: Personale
	\item[\textbf{Durata}]: Istantanea
\end{description}
\end{minipage}}\smallskip

Descrivi o nomina uno specifico tipo di bestia o vegetale. Concentrandoti sulla voce della natura nei tuoi dintorni, apprendi la direzione e la distanza dalla più vicina creatura o vegetale di quella specie, se ce ne sono entro 7,5 chilometri.

\textbf{Per ogni Successo Critico Magico} ottenuto nella Prova di Magia aumenti di 1 km l'area controllata

\incantesimo{Localizza Creatura}
\noindent\colorbox{OBSSgold!10}{
\begin{minipage}{0.95\linewidth}
\begin{description}[noitemsep, topsep=0pt, parsep=0pt, partopsep=0pt, leftmargin=0cm, labelwidth=1.3cm]
	\item[\textbf{Lista}]: Divinazione
	\item[\textbf{Livello}]: 4, Comune
	\item[\textbf{Lancio}]: 2 Azioni
	\item[\textbf{Gittata}]: Personale
	\item[\textbf{Durata}]: Concentrazione, massimo 1 ora
\end{description}
\end{minipage}}\smallskip

Descrivi o nomina una creatura che ti è familiare. Percepisci la direzione della posizione della creatura, purché quella creatura si trovi entro 300 metri da te. Se la creatura si muove, conosci anche la direzione del suo movimento.

L'incantesimo può localizzare una specifica creatura a te nota, o la più vicina creatura di una specie (come umano o unicorno), purché tu abbia visto una simile creatura da vicino (entro 9 metri) almeno una volta. Se la creatura che descrivi o nomini ha una forma diversa, per esempio è sotto gli effetti dell'incantesimo metamorfosi, questo incantesimo non sarà in grado di localizzare la creatura.

Questo incantesimo non può localizzare una creatura se un flusso di acqua corrente largo almeno 3 metri blocca un percorso diretto tra te e la creatura.

\textbf{Per ogni Successo Critico Magico} ottenuto nella Prova di Magia aumenta la distanza di altri 300m.

\incantesimo{Localizza Oggetto}
\noindent\colorbox{OBSSgold!10}{
\begin{minipage}{0.95\linewidth}
\begin{description}[noitemsep, topsep=0pt, parsep=0pt, partopsep=0pt, leftmargin=0cm, labelwidth=1.3cm]
	\item[\textbf{Lista}]: Divinazione
	\item[\textbf{Livello}]: 2, Comune
	\item[\textbf{Lancio}]: 2 Azioni
	\item[\textbf{Gittata}]: Personale
	\item[\textbf{Durata}]: Concentrazione, massimo 10 minuti
\end{description}
\end{minipage}}\smallskip

Descrivi o nomina un oggetto che ti è familiare. Percepisci la direzione della posizione dell'oggetto, purché quell'oggetto si trovi entro 300 metri da te. Se l'oggetto si muove, conosci anche la direzione del suo movimento.

L'incantesimo può localizzare uno specifico oggetto a te noto, purché tu lo abbia visto da vicino (entro 9 metri) almeno una volta. In alternativa, l'incantesimo può localizzare l'oggetto più vicino di un particolare tipo, come certi tipi di abbigliamento, gioielleria, mobili, attrezzi o armi.

Questo incantesimo non può localizzare un oggetto se qualsiasi spessore di piombo, anche un foglio sottile, blocca un percorso diretto tra di te e l'oggetto.

\textbf{Per ogni Successo Critico Magico} ottenuto nella Prova di Magia la durata aumenta di 1 ora.

\incantesimo{Loquacità}
\noindent\colorbox{OBSSgold!10}{
\begin{minipage}{0.95\linewidth}
\begin{description}[noitemsep, topsep=0pt, parsep=0pt, partopsep=0pt, leftmargin=0cm, labelwidth=1.3cm]
	\item[\textbf{Lista}]: Trasmutazione
	\item[\textbf{Livello}]: 8, Raro
	\item[\textbf{Lancio}]: 2 Azioni
	\item[\textbf{Gittata}]: Personale
	\item[\textbf{Durata}]: 1 ora
\end{description}
\end{minipage}}\smallskip

Fino al termine dell'incantesimo, quando effettui una prova basata sul Carisma puoi rimpiazzare il numero tirato con 15. Inoltre, non importa quello che dici, la magia o l'analisi che determina se stai dicendo la verità indicherà sempre che sei onesto.

\textbf{Per ogni Successo Critico Magico} ottenuto nella Prova di Magia aumenti di 1 ora la durata.

\incantesimo{Luce}
\noindent\colorbox{OBSSgold!10}{
\begin{minipage}{0.95\linewidth}
\begin{description}[noitemsep, topsep=0pt, parsep=0pt, partopsep=0pt, leftmargin=0cm, labelwidth=1.3cm]
	\item[\textbf{Lista}]: Universale
	\item[\textbf{Livello}]: 1, Comune
	\item[\textbf{Lancio}]: 2 Azioni
	\item[\textbf{Gittata}]: Contatto
	\item[\textbf{Durata}]: 30 minuti di tempo reale di gioco
\end{description}
\end{minipage}}\smallskip

Lanci l'incantesimo a contatto di un oggetto che non sia più grosso di 3 metri in qualsiasi direzione. Fino al termine dell'incantesimo, l'oggetto irradia una luce intensa in un raggio di 3 metri e penombra per ulteriori 6 metri. La luce può essere di qualsiasi colore tu voglia. Coprire completamente l'oggetto con qualcosa di opaco blocca la luce. Se un oggetto bersaglio è tenuto o indossato da una creatura ostile, quella creatura deve superare un Tiro Salvezza su Riflessi per evitare l'incantesimo. Una creatura colpita dall'incantesimo deve effettuare un Tiro Salvezza su Tempra o rimanere accecato fino alla fine del round successivo. Non puoi avere attivo più di un incantesimo Luce alla volta, un lancio successivo spegne la precedente Luce.

\textbf{Per ogni Critico ottenuto} nella Prova di Magia la durata aumenta di 10 minuti reali.

\incantesimo{Luce Diurna}
\noindent\colorbox{OBSSgold!10}{
\begin{minipage}{0.95\linewidth}
\begin{description}[noitemsep, topsep=0pt, parsep=0pt, partopsep=0pt, leftmargin=0cm, labelwidth=1.3cm]
	\item[\textbf{Lista}]: Invocazione
	\item[\textbf{Livello}]: 3, Comune
	\item[\textbf{Lancio}]: 2 Azioni
	\item[\textbf{Gittata}]: 18 metri
	\item[\textbf{Durata}]: 1 ora di tempo reale di gioco
\end{description}
\end{minipage}}\smallskip

Una sfera di luce con raggio 6 metri si espande da un punto a tua scelta entro la gittata. La sfera irradia luce intensa e luce fioca per ulteriori 12 metri. Se scegli un punto su di un oggetto che stai reggendo o che non è indossato o trasportato, la luce si irradia dall'oggetto e si muove con esso. Coprire completamente un oggetto con qualcosa di opaco, come un vaso o un elmo, blocca la luce. Se qualsiasi parte dell'area di questo incantesimo si sovrappone con l'area di oscurità creata da un incantesimo di livello 3 o più basso, l'incantesimo che ha creato l'oscurità viene dissolto. La luce creata si considera luce solare.

\textbf{NOTA}: i Devoti di Ljust o Sumkjr prendono +1 ai Tiri Salvezza finché illuminati da questo incantesimo

\incantesimo{Luci Danzanti}
\noindent\colorbox{OBSSgold!10}{
\begin{minipage}{0.95\linewidth}
\begin{description}[noitemsep, topsep=0pt, parsep=0pt, partopsep=0pt, leftmargin=0cm, labelwidth=1.3cm]
	\item[\textbf{Lista}]: Invocazione
	\item[\textbf{Livello}]: 1, Non Comune
	\item[\textbf{Lancio}]: 2 Azioni
	\item[\textbf{Gittata}]: 36 metri
	\item[\textbf{Durata}]: 10 minuti di tempo reale di gioco
\end{description}
\end{minipage}}\smallskip

Crei, a gittata, fino a quattro luci delle dimensioni di una torcia, facendole apparire come torce, lanterne o sfere luminose che fluttuano nell'aria per la durata dell'incantesimo. Puoi anche combinare le quattro luci in un'unica forma luminosa vagamente umanoide di taglia Media. Qualsiasi forma scegli, ciascuna luce emette una luce fioca in un raggio di 3 metri. Come 1 Azione di Movimento durante il tuo round, puoi spostare le luci fino a 18 metri in un nuovo punto a gittata.

Una luce deve trovarsi entro 6 metri da un'altra luce creata con questo incantesimo, e le luci svaniscono se eccedono la gittata dell'incantesimo.

\textbf{Per ogni Successo Critico Magico} nella Prova di Magia la durata aumenta di 10 minuti o crei una nuova luce.

\incantesimo{Luminescenza}
\noindent\colorbox{OBSSgold!10}{
\begin{minipage}{0.95\linewidth}
\begin{description}[noitemsep, topsep=0pt, parsep=0pt, partopsep=0pt, leftmargin=0cm, labelwidth=1.3cm]
	\item[\textbf{Lista}]: Invocazione
	\item[\textbf{Livello}]: 1, Non Comune
	\item[\textbf{Lancio}]: 2 Azioni
	\item[\textbf{Gittata}]: 18 metri
	\item[\textbf{Durata}]: 1 minuto di tempo reale di gioco
\end{description}
\end{minipage}}\smallskip

Tutti gli oggetti in una sfera di 3 metri di raggio a gittata vengono circondati da una luce blu, verde o viola (a tua scelta). Qualsiasi creatura nell'area quando l'incantesimo viene lanciato, viene anch'essa circondata dalla luce se fallisce un Tiro Salvezza su Riflessi. Per la durata dell'incantesimo, gli oggetti e le creature soggette emettono una luce fioca con raggio di 3 metri. Qualsiasi Tiro per Colpire contro una creatura od oggetto soggetto ha +2 se l'attaccante può vederlo e la creatura od oggetto non può beneficiare dell'invisibilità.

\incantesimo{Onda rovente}
\noindent\colorbox{OBSSgold!10}{
\begin{minipage}{0.95\linewidth}
\begin{description}[noitemsep, topsep=0pt, parsep=0pt, partopsep=0pt, leftmargin=0cm, labelwidth=1.3cm]
	\item[\textbf{Lista}]: Fuoco
	\item[\textbf{Livello}]: 1, Comune
	\item[\textbf{Lancio}]: 2 Azioni
	\item[\textbf{Gittata}]: Personale (cono di 3 metri)
	\item[\textbf{Durata}]: Istantanea
\end{description}
\end{minipage}}\smallskip

Tieni le mani chiuse davanti a te, una potente onda rovente si genera da ogni tuo pugno. Ogni creatura in un cono di 3 metri deve effettuare un Tiro Salvezza su Riflessi. Una creatura subisce 1d4 di danno per Competenza Magica, fino ad un massimo di 5d4, danni da fuoco se fallisce il Tiro Salvezza, o la metà se lo supera. Il calore incendia gli oggetti infiammabili nell'area che non siano indossati o trasportati.

\textbf{Per ogni Successo Critico Magico} ottenuto nella Prova di Magia il danno aumenta di 2d4.

\textbf{Tiro Salvezza Successo/Fallimento Critico}: In caso di Fallimento Critico il danno raddoppia, in caso di Successo Critico il danno viene ulteriormente dimezzato

\incantesimo{Mano Arcana}
\noindent\colorbox{OBSSgold!10}{
\begin{minipage}{0.95\linewidth}
\begin{description}[noitemsep, topsep=0pt, parsep=0pt, partopsep=0pt, leftmargin=0cm, labelwidth=1.3cm]
	\item[\textbf{Lista}]: Invocazione
	\item[\textbf{Livello}]: 5, Non Comune
	\item[\textbf{Lancio}]: 2 Azioni
	\item[\textbf{Gittata}]: 36 metri
	\item[\textbf{Durata}]: Concentrazione, 1 minuto
\end{description}
\end{minipage}}\smallskip

Crei una mano Grande, composta di energia trasparente e luminosa, in uno spazio non occupato a gittata e che puoi vedere. La mano permane per la durata dell'incantesimo, e si muove al tuo comando, imitando i movimenti della tua mano.

La mano è un oggetto che ha Difesa 25 e Punti Ferita uguali ai tuoi Punti Ferita massimi. Ha Forza 4 e Destrezza 0. La mano non riempie il suo spazio.
Quando lanci l'incantesimo e come 2 Azioni durante i tuoi round successivi, puoi muovere la mano fino a 18 metri e poi generare uno dei seguenti effetti.

\medskip

- \emph{Mano Afferrante}. La mano cerca di afferrare una creatura di taglia Enorme o più piccola che si trovi entro 1 metro da essa. Per risolvere l'azione di afferrare usi la DC dell'incantesimo contro un Tiro Salvezza su Tempra con bonus Forza dell'avversario.
Chi ha una taglia maggiore guadagna un bonus di +1d6 per taglia differenza.
Mentre la mano tiene afferrato il bersaglio, puoi usare un'Azione per fare stritolare il bersaglio dalla mano. Quando lo fai, il bersaglio subisce danni contundenti pari a 2d6 + 1d6 per taglia di differenza + il tuo modificatore di caratteristica per incantesimo.

- \emph{Mano di Forza}. La mano cerca di spingere una creatura di 1 metro in una direzione a tua scelta. Per risolvere l'azione di afferrare usi la DC dell'incantesimo contro un Tiro Salvezza su Tempra con bonus Forza dell'avversario. Chi ha una taglia maggiore guadagna un bonus di +1d6 per taglia differenza.
Se vinci la contesa, la mano spinge il bersaglio di 1 metro più 1 metro per modificatore di caratteristica per incantesimi. La mano si muove assieme al bersaglio per restare entro 1 metro da lui.

- \emph{Mano Frapposta}. La mano si frappone tra di te e una creatura di tua scelta finché non le dai un comando diverso. La mano si muove di modo da restare tra di te e il bersaglio, fornendoti copertura media contro il bersaglio. Il bersaglio non può muoversi attraverso lo spazio della mano se il suo punteggio di Forza è uguale o inferiore al punteggio di Forza della mano. Se il suo punteggio di Forza è superiore al punteggio di Forza della mano, il bersaglio può muoversi attraverso lo spazio della mano, ma considera quello spazio come fosse terreno difficile.

- \emph{Pugno Serrato}. La mano colpisce una creatura o un oggetto entro 1 metro da essa. Effettua un attacco in mischia con incantesimo usando la mano. Se colpisci, il bersaglio subisce 4d8 danni da forza.

\textbf{Per ogni Successo Critico Magico} ottenuto nella Prova di Magia il danno dell'opzione pugno serrato aumenta di 2d8 e il danno dell'opzione mano afferrante aumenta di 2d6.

\incantesimo{Mano Magica}
\noindent\colorbox{OBSSgold!10}{
\begin{minipage}{0.95\linewidth}
\begin{description}[noitemsep, topsep=0pt, parsep=0pt, partopsep=0pt, leftmargin=0cm, labelwidth=1.3cm]
	\item[\textbf{Lista}]: Evocazione
	\item[\textbf{Livello}]: 0, Comune
	\item[\textbf{Lancio}]: 2 Azioni
	\item[\textbf{Gittata}]: 9 metri
	\item[\textbf{Durata}]: 1d4 round +1 per punto di Competenza Magica
\end{description}
\end{minipage}}\smallskip

Una mano spettrale fluttuante compare in un punto a gittata, scelto da te. La mano resta per la durata dell'incantesimo o finché non viene interrotta con un'Azione. La mano svanisce se si dovesse trovare a più di 9 metri da te o se lanci nuovamente l'incantesimo.

Le Azioni necessarie a muovere ed usare la mano magica sono le stesse che useresti per usare la tua mano. Puoi usare la mano per manipolare un oggetto, aprire una porta o un contenitore non chiusi a chiave, inserire o recuperare un oggetto da un contenitore aperto, o versare fuori i contenuti di una fiala. Puoi muovere la mano di 9 metri ogni volta che la usi. La mano non può attaccare, attivare oggetti magici o trasportare oggetti con Ingombro maggiore di 1.

\textbf{Per ogni Successo Critico Magico} ottenuto nella Prova di Magia l'Ingombro sollevato aumenta di 1 o raddoppi la durata.


\incantesimo{Marchio Magico}
\noindent\colorbox{OBSSgold!10}{
\begin{minipage}{0.95\linewidth}
\begin{description}[noitemsep, topsep=0pt, parsep=0pt, partopsep=0pt, leftmargin=0cm, labelwidth=1.3cm]
	\item[\textbf{Lista}]: Universale
	\item[\textbf{Livello}]: 0, Comune
	\item[\textbf{Lancio}]: 2 Azioni
	\item[\textbf{Gittata}]: Contatto
	\item[\textbf{Durata}]: Permanente
\end{description}
\end{minipage}}\smallskip

Questo incantesimo permette di iscrivere una personale runa o marchio su un oggetto. La scritta non può essere lunga più di 15 cm. La scritta può essere visibile od invisibile a seconda di come si decide al momento del lancio della magia.
Un incantesimo di Individuazione del Magico o Lettura del magico mostra la scritta se invisibile.
Se la scritta è posta su una creatura questa scompare nel giro di un mese.

\textbf{Per ogni Successo Critico Magico} ottenuto nella Prova di Magia scrivi un logo in più.

\incantesimo{Messaggio}
\noindent\colorbox{OBSSgold!10}{
\begin{minipage}{0.95\linewidth}
\begin{description}[noitemsep, topsep=0pt, parsep=0pt, partopsep=0pt, leftmargin=0cm, labelwidth=1.3cm]
	\item[\textbf{Lista}]: Trasmutazione
	\item[\textbf{Livello}]: 0, Comune
	\item[\textbf{Lancio}]: 2 Azioni
	\item[\textbf{Gittata}]: 36 metri
	\item[\textbf{Durata}]: 1 round
\end{description}
\end{minipage}}\smallskip

Punti il dito verso una creatura a gittata e sussurri un messaggio breve. Il bersaglio (e solo il bersaglio) ode il messaggio e può replicare con un sussurro che solo tu puoi udire.

Puoi lanciare questo incantesimo anche attraverso oggetti solidi, se sei familiare col bersaglio e sai che questi si trova dietro la barriera. Il silenzio magico, 30 centimetri di pietra, 2,5 centimetri di metallo normale, un sottile foglio di piombo o 1 metro di legno bloccano l'incantesimo. L'incantesimo non deve seguire una linea retta, e può liberamente aggirare gli angoli o attraversare gli spiragli.

\textbf{Per ogni Successo Critico Magico} ottenuto nella Prova di Magia l'incantesimo dura 1 round in più.

\incantesimo{Metamorfosi}
\noindent\colorbox{OBSSgold!10}{
\begin{minipage}{0.95\linewidth}
\begin{description}[noitemsep, topsep=0pt, parsep=0pt, partopsep=0pt, leftmargin=0cm, labelwidth=1.3cm]
	\item[\textbf{Lista}]: Animali e Piante
	\item[\textbf{Livello}]: 4, Comune
	\item[\textbf{Lancio}]: 2 Azioni
	\item[\textbf{Gittata}]: 18 metri
	\item[\textbf{Durata}]: 1 ora
\end{description}
\end{minipage}}\smallskip

Questo incantesimo trasforma una creatura a gittata, che puoi vedere, in una nuova forma. Una creatura non consenziente deve superare un Tiro Salvezza su Volontà per evitare l'effetto. I mutaforma superano automaticamente il Tiro Salvezza. L'incantesimo non ha effetto su di un bersaglio con 0 Punti Ferita.

La trasformazione permane per la durata dell'incantesimo o finché il bersaglio non scende a 0 Punti Ferita o muore. La nuova forma può essere quella di qualsiasi bestia il cui grado di sfida sia la metà del punteggio di Competenza Magica (o somma dei Tratti se Devoto di Shayalia) di chi lancia l'incantesimo. Le statistiche di gioco del bersaglio, compresi i punteggi delle caratteristiche mentali, vengono rimpiazzate dalle statistiche della bestia scelta. Egli mantiene però i suoi Tratti e personalità.

La creatura è limitata nelle azioni che può svolgere dalla natura della sua nuova forma, e non può dialogare, lanciare incantesimi, o effettuare qualsiasi altra azione che richieda mani o di parlare. L'equipaggiamento del bersaglio si fonde nella nuova forma. La creatura non può attivare, usare, impugnare o beneficiare in alcun modo del suo equipaggiamento.

\incantesimo{Metamorfosi Pura}
\noindent\colorbox{OBSSgold!10}{
\begin{minipage}{0.95\linewidth}
\begin{description}[noitemsep, topsep=0pt, parsep=0pt, partopsep=0pt, leftmargin=0cm, labelwidth=1.3cm]
	\item[\textbf{Lista}]: Animali e Piante
	\item[\textbf{Livello}]: 9, Raro
	\item[\textbf{Lancio}]: 2 Azioni
	\item[\textbf{Gittata}]: 9 metri
	\item[\textbf{Durata}]: 1 ora
\end{description}
\end{minipage}}\smallskip

Scegli una creatura od oggetto non magico a gittata e che puoi vedere. L'incantesimo non ha effetto su di un bersaglio con 0 Punti Ferita. Trasformi la creatura in una creatura diversa, la creatura in un oggetto, o l'oggetto in una creatura (l'oggetto non deve essere indossato né trasportato da un'altra creatura). La trasformazione permane per la durata dell'incantesimo o finché il bersaglio non scende a 0 Punti Ferita o muore. Se ti concentri su questo incantesimo per l'intera durata, la trasformazione diventa permanente.

I mutaforma ignorano questo incantesimo. Una creatura non consenziente può effettuare un Tiro Salvezza su Volontà e, se lo supera, ignora l'effetto di questo incantesimo.

\begin{itemize}[leftmargin=*] \setlength{\itemsep}{0pt}
	\item \emph{Creatura in Creatura}. Se trasformi una creatura in un'altra specie di creatura, la nuova forma può essere quella di qualsiasi specie tu voglia, il cui grado di sfida sia pari o inferiore al tuo punteggio di Competenza Magica (o somma Tratti in comune se Devoto di Shayalia). Le statistiche di gioco del bersaglio, compresi i punteggi delle caratteristiche mentali, vengono rimpiazzate dalle statistiche della nuova forma. Egli mantiene però il suoi Tratti e personalità.

	Il bersaglio mantiene i medesimi Punti Ferita e ne recupera 1d12 Punti Ferita nella sua nuova forma. Quando ritorna alla sua forma normale, la creatura mantiene i Punti Ferita che ha attualmente. Se arriva a 0, o meno, Punti Ferita nella nuova forma allora torna normale e qualsiasi effetto si ripercuote anche nella forma corrente. La creatura è limitata nelle azioni che può svolgere dalla natura della sua nuova forma, e non può dialogare, lanciare incantesimi, o effettuare qualsiasi altra azione che richieda mani o di parlare, a meno che la nuova forma non sia capace di svolgere queste azioni. L'equipaggiamento del bersaglio si fonde nella nuova forma. La creatura non può attivare, usare, impugnare o beneficiare in alcun modo del suo equipaggiamento.

	\item \emph{Oggetto in Creatura.} Puoi trasformare un oggetto in un qualsiasi tipo di creatura, purché la taglia della creatura non sia maggiore della taglia dell'oggetto e il grado di sfida della creatura sia 9 o meno. La creatura è amichevole verso di te e i tuoi compagni. Essa agisce durante i tuoi round. Decidi tu quali azioni essa compirà e come si muove. Il Narratore possiede le statistiche della creatura e risolverà tutte le sue azioni e i suoi movimenti.
	Se l'incantesimo diventa permanente, perdi il controllo della creatura. A seconda di come l'hai trattata, potrebbe restare amichevole nei tuoi confronti.

	\item \emph{Creatura in Oggetto}. Se trasformi una creatura in un oggetto, essa si trasforma assieme a qualsiasi cosa stia indossando o trasportando. Le statistiche della creatura diventano quelle dell'oggetto, e, dopo che l'incantesimo termina e la creatura ritorna alla sua forma normale, questa non ha più ricordi del tempo trascorso in forma di oggetto.

\end{itemize}

\incantesimo{Miraggio Arcano}
\noindent\colorbox{OBSSgold!10}{
\begin{minipage}{0.95\linewidth}
\begin{description}[noitemsep, topsep=0pt, parsep=0pt, partopsep=0pt, leftmargin=0cm, labelwidth=1.3cm]
	\item[\textbf{Lista}]: Illusione
	\item[\textbf{Livello}]: 7, Raro
	\item[\textbf{Lancio}]: 10 minuti
	\item[\textbf{Gittata}]: Vista
	\item[\textbf{Durata}]: 10 giorni
\end{description}
\end{minipage}}\smallskip

Fai sì che un pezzo di terreno a gittata, in un'area quadrata fino a 1,5 chilometri, appaia, risuoni e odori come qualche altro tipo di terreno. La conformazione generale del terreno rimane tuttavia la stessa. Campi aperti o una strada possono essere trasformati in un acquitrino, colline, un crepaccio o qualche altro tipo di terreno difficile o invalicabile. Un laghetto può essere trasformato in una radura erbosa, un precipizio in una lieve pendenza, un burrone cosparso di rocce in una strada ampia e liscia.

Allo stesso modo, puoi modificare l'aspetto delle strutture, o aggiungerne dove non ve ne sono. L'incantesimo non camuffa, occulta né aggiunge creature.

L'illusione comprende elementi uditivi, visivi, tattili e olfattivi, così da poter trasformare un terreno sgombro in terreno difficile (o viceversa) o impedire altrimenti il movimento nell'area. Qualsiasi pezzo di terreno illusorio (come una pietra o un bastone), che venga rimosso dall'area dell'incantesimo, svanisce immediatamente. Le creature con visione del vero possono vedere oltre l'illusione e distinguere la vera forma del terreno; tuttavia, gli altri elementi dell'illusione rimangono, così, sebbene la creatura sia consapevole della presenza dell'illusione, vi può comunque interagire fisicamente.

\textbf{Con tre Successo Critico Magico ottenuto} nella Prova di Magia la durata è permanente.

\incantesimo{Modificare Memoria}
\noindent\colorbox{OBSSgold!10}{
\begin{minipage}{0.95\linewidth}
\begin{description}[noitemsep, topsep=0pt, parsep=0pt, partopsep=0pt, leftmargin=0cm, labelwidth=1.3cm]
	\item[\textbf{Lista}]: Ammaliamento
	\item[\textbf{Livello}]: 5, Molto Raro
	\item[\textbf{Lancio}]: 3 Azioni
	\item[\textbf{Gittata}]: 9 metri
	\item[\textbf{Durata}]: Istantanea
\end{description}
\end{minipage}}\smallskip

Tenti di rimodellare i ricordi di un'altra creatura. Una creatura che puoi vedere deve effettuare un Tiro Salvezza su Volontà. Se la stai combattendo, la creatura ha +1d6 sul Tiro Salvezza. Se fallisce il Tiro Salvezza puoi agire sui ricordi del bersaglio in merito a un evento che abbia vissuto nelle ultime 24 ore e che non sia durato più di 10 minuti.

Puoi eliminare permanentemente tutti i ricordi dell'evento, permettere al bersaglio di ricordare l'evento con perfetta chiarezza e dettagli particolareggiati, modificare il ricordo dei dettagli dell'evento, o creare il ricordo di un altro evento. Devi poter parlare al bersaglio per descrivere il modo in cui i suoi ricordi saranno colpiti, e questi deve essere in grado di comprendere il tuo linguaggio, affinché i ricordi modificati si instaurino nella sua memoria.
I ricordi modificati si instaurano al termine dell'incantesimo.

Una memoria modificata non influisce necessariamente sul comportamento della creatura, in particolare se i suoi ricordi contraddicono le inclinazioni naturali, i Tratti o la fede della creatura. Una memoria modificata in modo illogico, come impiantare il ricordo di quanto la creatura ami immergersi nell'acido, viene rimossa, come fosse un brutto sogno.

Il Narratore può giudicare un ricordo modificato troppo insensato perché abbia alcun effetto su di una creatura. Un incantesimo rimuovi maledizione o ristorare superiore lanciato sul bersaglio ne ripristina i veri ricordi.

\textbf{Con un Successo Critico Magico} ottenuto nella Prova di Magia puoi alterare i ricordi di un bersaglio riguardo un evento svoltosi fino a 7 giorni prima. Con due fino a 30 giorni prima, con tre fino ad 1 anno prima. Con 4 successi Critici Magici in qualsiasi punto nel passato della creatura.

\incantesimo{Morte Apparente}
\noindent\colorbox{OBSSgold!10}{
\begin{minipage}{0.95\linewidth}
\begin{description}[noitemsep, topsep=0pt, parsep=0pt, partopsep=0pt, leftmargin=0cm, labelwidth=1.3cm]
	\item[\textbf{Lista}]: Necromanzia
	\item[\textbf{Livello}]: 3, Non Comune
	\item[\textbf{Lancio}]: 1 Reazione
	\item[\textbf{Gittata}]: 18 metri
	\item[\textbf{Durata}]: 6 round più 1 round per CM
\end{description}
\end{minipage}}\smallskip

L'incantatore induce in sé stesso, o in una creatura consenziente, uno stato di paralisi totale che sembra identico alla morte, anche in caso di esame approfondito. Una creatura influenzata sente suoni e odori, ma non si può muovere ed è completamente priva di percezioni tattili; se il corpo viene danneggiato, non percepirà alcun fastidio, né avrà alcuna reazione fisica. Tutto il danno inflitto a una creatura in questo stato viene ridotto della metà; i veleni e gli attacchi che paralizzano o prosciugano la vita non fanno effetto prima della scadenza di questo incantesimo. È necessario un round, terminato l'incantesimo, prima che il corpo possa riprendere le sue normali funzioni vitali.

\textbf{Per ogni Successo Critico Magico} ottenuto nella Prova di Magia raddoppi la durata o influenzi un altra creatura.

\incantesimo{Movimenti del Ragno}
\noindent\colorbox{OBSSgold!10}{
\begin{minipage}{0.95\linewidth}
\begin{description}[noitemsep, topsep=0pt, parsep=0pt, partopsep=0pt, leftmargin=0cm, labelwidth=1.3cm]
	\item[\textbf{Lista}]: Trasmutazione
	\item[\textbf{Livello}]: 2, Non Comune
	\item[\textbf{Lancio}]: 2 Azioni
	\item[\textbf{Gittata}]: Contatto
	\item[\textbf{Durata}]: 10 minuti
\end{description}
\end{minipage}}\smallskip

Lanci l'incantesimo a contatto di una creatura consenziente. Fino al termine dell'incantesimo, la creatura ottiene la capacità di spostarsi verso l'alto, il basso e lungo superfici verticali o stando a testa in giù sul soffitto, tenendo le mani libere. Il bersaglio ottiene anche velocità di scalata pari alla sua velocità di movimento. La creatura soggetta all'incantesimo si considera Distratta nel lancio di altri incantesimi.

\incantesimo{Muovere il Terreno}
\noindent\colorbox{OBSSgold!10}{
\begin{minipage}{0.95\linewidth}
\begin{description}[noitemsep, topsep=0pt, parsep=0pt, partopsep=0pt, leftmargin=0cm, labelwidth=1.3cm]
	\item[\textbf{Lista}]: Terra
	\item[\textbf{Livello}]: 6, Non Comune
	\item[\textbf{Lancio}]: 2 Azioni
	\item[\textbf{Gittata}]: 36 metri
	\item[\textbf{Durata}]: Concentrazione, massimo 2 ore
\end{description}
\end{minipage}}\smallskip

Scegli un'area sul terreno a gittata, non più grande di 12 metri di lato. Per la durata, puoi rimodellare terriccio, sabbia o argilla nell'area in qualsiasi modo tu voglia. Puoi innalzare o abbassare l'altitudine dell'area, creare o riempire un fossato, erigere o abbassare un muro, o formare un pilastro. La portata di questi cambiamenti non può eccedere metà della dimensione più grossa dell'area. Così, se operi su di un quadrato di 12 metri di lato, puoi creare un pilastro alto 6 metri, innalzare o abbassare l'altitudine del terreno di 6 metri, scavare un fossato profondo 6 metri, e così via. Ci vogliono 10 minuti per completare questi mutamenti. Al termine di ogni 10minuti trascorsi a concentrarsi sull'incantesimo, puoi scegliere una nuova area di terreno su cui operare.

Dato che la trasformazione del terreno avviene lentamente, le creature nell'area di solito non possono restare intrappolate o ferite dal movimento del terreno. L'incantesimo non può manipolare la pietra naturale o le costruzioni in pietra. Le rocce e le strutture si muovono per adattarsi al nuovo terreno. Se il modo in cui modelli il terreno renderebbe una struttura instabile, questa potrebbe crollare. Allo stesso modo, questo incantesimo non influenza direttamente la crescita dei vegetali. La terra smossa trasporta con sé qualsiasi vegetale presente.

\incantesimo{Muro di Forza}
\noindent\colorbox{OBSSgold!10}{
\begin{minipage}{0.95\linewidth}
\begin{description}[noitemsep, topsep=0pt, parsep=0pt, partopsep=0pt, leftmargin=0cm, labelwidth=1.3cm]
	\item[\textbf{Lista}]: Invocazione
	\item[\textbf{Livello}]: 5, Comune
	\item[\textbf{Lancio}]: 2 Azioni
	\item[\textbf{Gittata}]: 36 metri
	\item[\textbf{Durata}]: 10 minuti
\end{description}
\end{minipage}}\smallskip

Un invisibile muro di forza si forma in un punto a gittata scelto da te. Il muro appare in qualsiasi orientamento da te desiderato, come una barriera orizzontale o verticale oppure angolata. Può fluttuare nell'aria o appoggiarsi su di una superficie solida. Puoi darle la forma di una cupola semisferica o di una sfera con un raggio massimo di 3 metri, oppure darle l'aspetto di una superficie piana composta da un massimo di dieci pannelli di 3 metri per 3 metri. Ogni pannello deve essere contiguo a un altro pannello. In qualsiasi forma, il muro ha uno spessore di 75 centimetri e resta per tutta la durata dell'incantesimo. Se il muro taglia uno spazio di una creatura, quando compare, la creatura viene spinta da un lato del muro (a tua discrezione). Nulla può attraversare fisicamente il muro, chiunque al di là del muro ha copertura completa. È immune a tutti i danni e non può essere dissolto da dissolvi magie. Tuttavia, il muro è distrutto all'istante dall'incantesimo disintegrazione. Il muro si estende anche sul Piano Etereo, impedendo ai viaggiatori eterei di attraversarlo.

\incantesimo{Muro di Fuoco}
\noindent\colorbox{OBSSgold!10}{
\begin{minipage}{0.95\linewidth}
\begin{description}[noitemsep, topsep=0pt, parsep=0pt, partopsep=0pt, leftmargin=0cm, labelwidth=1.3cm]
	\item[\textbf{Lista}]: Fuoco
	\item[\textbf{Livello}]: 4, Comune
	\item[\textbf{Lancio}]: 2 Azioni
	\item[\textbf{Gittata}]: 36 metri
	\item[\textbf{Durata}]: 1 minuto
\end{description}
\end{minipage}}\smallskip

Crei un muro di fuoco su di una superficie solida a gittata. Puoi creare un muro lungo fino a 18 metri, alto fino a 6 metri e spesso 30 centimetri, o un muro circolare di 6 metri di diametro, 6 metri di altezza e 30 centimetri di spessore. Il muro è opaco e rimane per la durata dell'incantesimo.

Quando il muro appare, ogni creatura nella sua area deve effettuare un Tiro Salvezza su Riflessi. Una creatura subisce 5d8 danni da fuoco se fallisce il Tiro Salvezza, o la metà se lo supera. Un lato del muro, selezionato da te quando lanci questo incantesimo, infligge 5d8 danni da fuoco a ciascuna creatura che termini il suo round entro 3 metri da quel lato o all'interno del muro. Una creatura subisce lo stesso danno quando entra nel muro per la prima volta durante un round. L'altro lato del muro non infligge danni.

\textbf{Per ogni Successo Critico Magico} ottenuto nella Prova di Magia il danno aumenta di 3d6.

\incantesimo{Muro di Ghiaccio}
\noindent\colorbox{OBSSgold!10}{
\begin{minipage}{0.95\linewidth}
\begin{description}[noitemsep, topsep=0pt, parsep=0pt, partopsep=0pt, leftmargin=0cm, labelwidth=1.3cm]
	\item[\textbf{Lista}]: Acqua
	\item[\textbf{Livello}]: 6, Comune
	\item[\textbf{Lancio}]: 2 Azioni
	\item[\textbf{Gittata}]: 36 metri
	\item[\textbf{Durata}]: 10 minuti
\end{description}
\end{minipage}}\smallskip

Crei un muro di ghiaccio su di una superficie solida a gittata. Puoi creare una cupola semisferica o una sfera con un raggio massimo di 3 metri, o puoi creare una superficie piana composta di un massimo di dieci panelli quadrati di 3 metri di lato. Ogni pannello deve essere contiguo ad almeno un altro pannello. In ogni forma, il muro è spesso 30 centimetri e rimane per la durata dell'incantesimo.

Se, quando compare, il muro attraversa lo spazio di una creatura, la creatura viene spinta da una parte del muro (a tua scelta) e deve effettuare un Tiro Salvezza su Riflessi. Se fallisce il Tiro Salvezza, la creatura subisce 10d6 danni da freddo, o la metà di questi danni se lo supera.

Il muro è un oggetto che può essere danneggiato e sfondato. Ogni sezione di 3 metri ha Difesa 12 e 30 Punti Ferita, ed è vulnerabile al danno da fuoco. Ridurre una sezione di 3 metri a 0 Punti Ferita la distrugge e lascia nello spazio che era occupato dal muro una brezza di vento gelido. Una creatura che si muova attraverso questa brezza di vento gelido per la prima volta in un round, deve effettuare un Tiro Salvezza su Tempra. Se lo fallisce, la creatura subisce 5d6 danni da freddo, o la metà di questi danni se lo supera.

\textbf{Per ogni Successo Critico Magico} ottenuto nella Prova di Magia il danno aumentano di 5d6.

\incantesimo{Muro di Pietra}
\noindent\colorbox{OBSSgold!10}{
\begin{minipage}{0.95\linewidth}
\begin{description}[noitemsep, topsep=0pt, parsep=0pt, partopsep=0pt, leftmargin=0cm, labelwidth=1.3cm]
	\item[\textbf{Lista}]: Invocazione
	\item[\textbf{Livello}]: 5, Comune
	\item[\textbf{Lancio}]: 2 Azioni
	\item[\textbf{Gittata}]: 36 metri
	\item[\textbf{Durata}]: 10 minuti
\end{description}
\end{minipage}}\smallskip

Un muro di pietra solida non magico si forma in un punto a gittata, scelto da te. Il muro è spesso 15 centimetri ed è composto da 10 pannelli di 3 per 3 metri. Ogni pannello deve essere contiguo ad almeno un altro pannello. In alternativa, puoi creare pannelli 3 x 6 metri di soli 7,5 centimetri di spessore.

Se, quando compare, il muro attraversa lo spazio di una creatura, la creatura viene spinta da una parte del muro (a tua scelta). Se la creatura fosse circondata da tutte le parti dal muro (o dal muro e un'altra superficie solida), la creatura può effettuare un Tiro Salvezza su Riflessi. Se lo supera, può usare una Azione di Reazione per muoversi della sua velocità in modo da non essere più intrappolata nel muro.

Il muro può aver qualsiasi forma tu desideri, sebbene non possa occupare lo stesso spazio di una creatura od oggetto. Il muro può anche non essere verticale o poggiare su di un piano. Deve, tuttavia, fondersi con ed essere sostenuto da pietra già esistente. Quindi, puoi usare questo incantesimo per creare un ponte su di un baratro o creare un rampa.

Se crei un muro non verticale del genere, più lungo di 6 metri, devi dimezzare le dimensioni di ciascun pannello per creare dei supporti. Puoi modellare rozzamente la pietra per creare merlature, spalti e così via. Il muro è un oggetto fatto di pietra che può essere danneggiato e sfondato. Ogni pannello ha Difesa 15, Durezza 15 e 15 Punti Ferita ogni 2,5 centimetri di spessore. Ridurre un pannello a 0 Punti Ferita lo distrugge e potrebbe far crollare i pannelli connessi, a discrezione del Narratore. Se mantieni la concentrazione su questo incantesimo per la sua intera durata, il muro diventa permanente e non può essere dissolto. Altrimenti, il muro sparisce al termine dell'incantesimo.

\incantesimo{Muro Prismatico}
\noindent\colorbox{OBSSgold!10}{
\begin{minipage}{0.95\linewidth}
\begin{description}[noitemsep, topsep=0pt, parsep=0pt, partopsep=0pt, leftmargin=0cm, labelwidth=1.3cm]
	\item[\textbf{Lista}]: Abiurazione
	\item[\textbf{Livello}]: 9, Raro
	\item[\textbf{Lancio}]: 2 Azioni
	\item[\textbf{Gittata}]: 18 metri
	\item[\textbf{Durata}]: 10 minuti
\end{description}
\end{minipage}}\smallskip

Un piano di luci brillanti e multicolore forma un muro verticale opaco, largo fino a 27 metri, alto 9 metri e spesso 2,5 centimetri, centrato su di un punto a gittata e che puoi vedere. In alternativa, puoi modellare il muro in una sfera, fino a 9 metri di diametro, centrata su di un punto a gittata di tua scelta. Il muro resta fisso sul posto per la durata dell'incantesimo. Se posizioni il muro in modo che attraversi lo spazio occupato da una creatura, l'incantesimo fallisce e lo slot incantesimo sono sprecati. Il muro irradia luce intensa fino a una gittata di 18 metri e luce fioca per 36 metri. Tu e le creature indicate da te al momento del lancio dell'incantesimo potete attraversare e restare vicini al muro senza pericolo. Se un'altra creatura che può vedere il muro si muove entro 6 metri da esso o inizia lì il suo round, deve superare un Tiro Salvezza su Tempra o restare accecata per 1 minuto. Il muro consiste di sette strati, ognuno di un diverso colore. Quando una creatura cerca di immergersi o attraversare il muro, lo fa uno strato alla volta, attraverso tutti gli strati del muro. Mentre si immerge o attraversa ciascuno strato, la creatura deve superare un Tiro Salvezza su Riflessi o subire le proprietà di ciascuno strato, uno alla volta, come descritto di seguito.

Il muro può essere distrutto, uno strato alla volta, in ordine dal rosso al violetto, in un modo specifico per ogni strato. Una volta che uno strato è distrutto, lo sarà per la durata dell'incantesimo. Una verga di cancellazione distrugge un Muro Prismatico, ma un campo anti-magia non ha effetto su di esso.

\begin{itemize}[leftmargin=*] \setlength{\itemsep}{0pt}
	\item \emph{1. Rosso}. Il bersaglio subisce 10d6 danni da fuoco se fallisce il Tiro Salvezza, o la metà di questi danni se lo supera. Finché questo strato esiste, gli attacchi a distanza non magici non possono attraversare il muro. Lo strato può essere distrutto infliggendogli 25 danni da freddo.
	\item \emph{2. Arancio}. Il bersaglio subisce 10d6 danni da acido se fallisce il Tiro Salvezza, o la metà di questi danni se lo supera. Finché questo strato esiste, gli attacchi a distanza magici non possono attraversare il muro. Lo strato può essere distrutto da un forte vento. 3. Giallo. Il bersaglio subisce 10d6 danni da elettricità se fallisce il Tiro Salvezza, o la metà di questi danni se lo supera. Questo strato può essere distrutto infliggendogli 60 danni di forza.
	\item \emph{4. Verde}. Il bersaglio subisce 10d6 danni da veleno se fallisce il Tiro Salvezza, o la metà di questi danni se lo supera. Un incantesimo Passa Porta, o un altro incantesimo di pari livello o più alto che può aprire un portale su di una superficie solida, distrugge questo strato.
	\item \emph{5. Blu}. Il bersaglio subisce 10d6 danni da freddo se fallisce il Tiro Salvezza, o la metà di questi danni se lo supera. Lo strato può essere distrutto infliggendogli almeno 25 danni da fuoco.
	\item \emph{6. Indaco}. Se fallisce il Tiro Salvezza, il bersaglio è intralciato. Deve poi effettuare un Tiro Salvezza su Tempra all'inizio di ciascun suo round. Se supera il Tiro Salvezza tre volte,l'incantesimo termina. Se fallisce il Tiro Salvezza tre volte, viene permanentemente trasformato in pietra e diventa vittima della condizione pietrificato. I successi e i fallimenti non devono essere consecutivi; tieni traccia di entrambi finché il bersaglio non ne ha ottenuti tre dello stesso tipo. Finché questo strato esiste, non si possono lanciare incantesimi attraverso il muro. Lo strato viene distrutto dalla luce intensa emanata dall'incantesimo luce diurna o da un simile incantesimo di livello più alto.
	\item \emph{7. Violetto}. Se fallisce il Tiro Salvezza, il bersaglio è accecato. Deve poi effettuare un Tiro Salvezza su Volontà all'inizio del tuo prossimo round. Se supera il Tiro Salvezza, la cecità termina. Se fallisce il Tiro Salvezza, la creatura viene trasportata su di un altro piano di esistenza a scelta del Narratore e non è più accecata (di solito, una creatura che non è sul suo piano natio, viene esiliata su di esso, mentre le altre creature sono di solito gettate nei piani Astrale o Etereo). Questo strato è distrutto dall'incantesimo dissolvi magie o da un incantesimo simile di pari livello o più alto che possa porre fine a incantesimi ed effetti magici.

\end{itemize}

\incantesimo{Muro di Spine}
\noindent\colorbox{OBSSgold!10}{
\begin{minipage}{0.95\linewidth}
\begin{description}[noitemsep, topsep=0pt, parsep=0pt, partopsep=0pt, leftmargin=0cm, labelwidth=1.3cm]
	\item[\textbf{Lista}]: Animali e Piante
	\item[\textbf{Livello}]: 6, Non Comune
	\item[\textbf{Lancio}]: 2 Azioni
	\item[\textbf{Gittata}]: 36 metri
	\item[\textbf{Durata}]: massimo 10 minuti
\end{description}
\end{minipage}}\smallskip

Crei un muro di cespugli robusti, malleabili e impigliati, ricolmi di spine appuntite. Il muro compare a gittata su di una superficie solida e rimane per la durata dell'incantesimo. Il muro che puoi creare può essere lungo fino a 18 metri, alto fino a 3 metri, e spesso fino a 1 metro o un circolo che abbia un diametro di 6 metri e sia alto fino a 6 metri e spesso 1 metro. Il muro blocca la linea di visuale.

Quando il muro compare, ogni creatura nella sua area deve effettuare un Tiro Salvezza su Riflessi. Se fallisce il Tiro Salvezza, una creatura subisce 7d8 danni perforanti, o la metà di questi danni se lo supera. Una creatura può muoversi attraverso il muro, seppure in maniera lenta e dolorosa. Il terreno coperto dal muro si considera doppiamente difficile. Inoltre, la prima volta che una creatura entra nel muro durante un round o vi termina il suo round dentro, la creatura deve effettuare un Tiro Salvezza su Riflessi. Subisce 7d8 danni taglienti se fallisce il Tiro Salvezza, o la metà di questi danni se lo supera.

\textbf{Per ogni Successo Critico Magico} ottenuto nella Prova di Magia il danno aumenta di 3d8.

\incantesimo{Muro di Vento}
\noindent\colorbox{OBSSgold!10}{
\begin{minipage}{0.95\linewidth}
\begin{description}[noitemsep, topsep=0pt, parsep=0pt, partopsep=0pt, leftmargin=0cm, labelwidth=1.3cm]
	\item[\textbf{Lista}]: Aria
	\item[\textbf{Livello}]: 3, Non Comune
	\item[\textbf{Lancio}]: 2 Azioni
	\item[\textbf{Gittata}]: 36 metri
	\item[\textbf{Durata}]: 1 minuto
\end{description}
\end{minipage}}\smallskip

Un muro di forte vento si leva dal terreno in un punto a gittata di tua scelta. Puoi creare un muro lungo fino a 15 metri, alto 3 metri e spesso 30 centimetri. Puoi modellare il muro in qualsiasi maniera desideri purché componga un percorso continuo sul terreno. Il muro rimane per la durata dell'incantesimo. Quando il muro appare, ogni creatura all'interno della sua area deve effettuare un Tiro Salvezza su Tempra. Una creatura subisce 3d8 danni contundenti se fallisce il Tiro Salvezza, o la metà di questi danni se lo supera. Il forte vento tiene lontana foschia, fumo e altri gas. Le creature volanti di taglia Piccola o minore non possono attraversare il muro. I materiali leggeri trascinati nel muro volano verso l'alto. Frecce, quadrelli e altre munizioni normali vengono deviati e mancano automaticamente il bersaglio (i macigni scagliati dai giganti e dalle macchine d'assedio, e munizioni simili, ne ignorano invece gli effetti). Le creature in forma gassosa non possono attraversarlo.

\textbf{Per ogni Successo Critico Magico} ottenuto nella Prova di Magia la durata aumenta di 1 minuto.

\incantesimo{Nube Incendiaria}
\noindent\colorbox{OBSSgold!10}{
\begin{minipage}{0.95\linewidth}
\begin{description}[noitemsep, topsep=0pt, parsep=0pt, partopsep=0pt, leftmargin=0cm, labelwidth=1.3cm]
	\item[\textbf{Lista}]: Fuoco
	\item[\textbf{Livello}]: 8, Raro
	\item[\textbf{Lancio}]: 2 Azioni
	\item[\textbf{Gittata}]: 45 metri
	\item[\textbf{Durata}]: 1 minuto
\end{description}
\end{minipage}}\smallskip

Una nube di fumo turbinante attraversata da lapilli incandescenti si forma in una sfera di 6 metri di raggio centrata su di un punto a gittata. La nube si propaga intorno agli angoli ed è in penombra. Rimane per la durata dell'incantesimo o finché un vento di velocità moderata o superiore (almeno 15 chilometri all'ora) non la disperde.

Quando la nube appare, ogni creatura al suo interno deve effettuare un Tiro Salvezza su Riflessi. Una creatura subisce 10d8 danni da fuoco se fallisce il Tiro Salvezza, e la metà di questi danni se lo supera. Una creatura deve effettuare il Tiro Salvezza anche quando entra per la prima volta nell'area o termina lì il suo round.

All'inizio di ciascun tuo round, la nube si muove di 3 metri lontano da te in una direzione a tua scelta.

\incantesimo{Nebbia Nauseante}
\noindent\colorbox{OBSSgold!10}{
\begin{minipage}{0.95\linewidth}
\begin{description}[noitemsep, topsep=0pt, parsep=0pt, partopsep=0pt, leftmargin=0cm, labelwidth=1.3cm]
	\item[\textbf{Lista}]: Acqua, Aria
	\item[\textbf{Livello}]: 3, Non Comune
	\item[\textbf{Lancio}]: 2 Azioni
	\item[\textbf{Gittata}]: 27 metri
	\item[\textbf{Durata}]: 10 minuti
\end{description}
\end{minipage}}\smallskip

Crei, in un punto a gittata, una sfera di 6 metri di raggio composta di un gas giallo e nauseabondo. La nebbia si propaga dietro gli angoli e la sua area è in penombra. La nebbia permane nell'aria per la durata. Ogni creatura che si trovi completamente all'interno della nebbia all'inizio del proprio round, deve effettuare un Tiro Salvezza su Tempra contro il veleno. Se il Tiro Salvezza fallisce, la creatura spende 2 Azioni di quel round a vomitare e barcollare. Le creature che non hanno bisogno di respirare o che sono immuni al veleno superano automaticamente il Tiro Salvezza. Un vento moderato (almeno 15 chilometri all'ora) disperde la nebbia dopo 4 round. Un vento forte (almeno 30 chilometri all'ora) lo disperde dopo 1 round.

\incantesimo{Nebbia mortale}
\noindent\colorbox{OBSSgold!10}{
\begin{minipage}{0.95\linewidth}
\begin{description}[noitemsep, topsep=0pt, parsep=0pt, partopsep=0pt, leftmargin=0cm, labelwidth=1.3cm]
	\item[\textbf{Lista}]: Acqua
	\item[\textbf{Livello}]: 5, Raro
	\item[\textbf{Lancio}]: 2 Azioni
	\item[\textbf{Gittata}]: 36 metri
	\item[\textbf{Durata}]: 10 minuti
\end{description}
\end{minipage}}\smallskip

Crei una sfera di 6 metri di raggio formata da una nebbia velenosa giallo-verde centrata in un punto a gittata di tua scelta. La nebbia si propaga dietro gli angoli. Rimane per la durata dell'incantesimo o finché un forte vento non disperde la nebbia, terminando l'incantesimo. La sua area è in penombra. Quando una creatura entra nell'area dell'incantesimo per la prima volta in un round o inizia lì il suo round, quella creatura deve effettuare un Tiro Salvezza su Tempra. La creatura subisce 5d8 danni da veleno se fallisce il Tiro Salvezza, o la metà di questi danni se lo supera. Le creature ne sono soggette anche se trattengono il respiro o non hanno bisogno di respirare. La nebbia si allontana di 3 metri da te all'inizio di ogni tuo round, spostandosi lungo la superficie del terreno. I vapori, essendo più pesanti dell'aria, tendono a scendere verso il basso, arrivando addirittura a insinuarsi nelle aperture.

\textbf{Per ogni Successo Critico Magico} ottenuto nella Prova di Magia il danno aumenta di 3d8.

\incantesimo{Nube di Nebbia}
\noindent\colorbox{OBSSgold!10}{
\begin{minipage}{0.95\linewidth}
\begin{description}[noitemsep, topsep=0pt, parsep=0pt, partopsep=0pt, leftmargin=0cm, labelwidth=1.3cm]
	\item[\textbf{Lista}]: Acqua, Aria
	\item[\textbf{Livello}]: 1, Comune
	\item[\textbf{Lancio}]: 2 Azioni
	\item[\textbf{Gittata}]: 36 metri
	\item[\textbf{Durata}]: 1 ora
\end{description}
\end{minipage}}\smallskip

Crei una sfera di foschia del raggio di 6 metri centrata su di un punto a gittata. La sfera si propaga intorno agli angoli, e la sua area è in penombra. Rimane per la durata dell'incantesimo o finché un vento di velocità moderata o superiore (almeno 15 chilometri all'ora) non la disperde.

\textbf{Per ogni Successo Critico Magico} ottenuto nella Prova di Magia il raggio della foschia aumenta di 6 metri.

\incantesimo{Occhio Arcano}
\noindent\colorbox{OBSSgold!10}{
\begin{minipage}{0.95\linewidth}
\begin{description}[noitemsep, topsep=0pt, parsep=0pt, partopsep=0pt, leftmargin=0cm, labelwidth=1.3cm]
	\item[\textbf{Lista}]: Divinazione
	\item[\textbf{Livello}]: 4, Comune
	\item[\textbf{Lancio}]: 2 Azioni
	\item[\textbf{Gittata}]: 9 metri
	\item[\textbf{Durata}]: Concentrazione, massimo 1 ora
\end{description}
\end{minipage}}\smallskip

Crei a gittata un occhio magico e invisibile, che fluttua nell'aria per la durata dell'incantesimo.

Ricevi mentalmente le informazioni visive dall'occhio, che ha vista normale e scurovisione fino a 9 metri. L'occhio può guardare in tutte le direzioni. Con un'Azione di Movimento, puoi spostare l'occhio di 9 metri in qualsiasi direzione. Non c'è limite a quanto lontano possa spostarsi l'occhio, ma non può entrare in un altro piano di esistenza. Una barriera solida blocca il movimento dell'occhio, ma questo può attraversare un'apertura di una grandezza minima di 2,5 centimetri di diametro.

\incantesimo{Onda Tonante}
\noindent\colorbox{OBSSgold!10}{
\begin{minipage}{0.95\linewidth}
\begin{description}[noitemsep, topsep=0pt, parsep=0pt, partopsep=0pt, leftmargin=0cm, labelwidth=1.3cm]
	\item[\textbf{Lista}]: Aria
	\item[\textbf{Livello}]: 1, Comune
	\item[\textbf{Lancio}]: 2 Azioni
	\item[\textbf{Gittata}]: Personale
	\item[\textbf{Durata}]: Istantanea
\end{description}
\end{minipage}}\smallskip

Un'onda di forza tonante si proietta da te. Ogni creatura in una sfera di 2 metri di raggio che origina da te deve effettuare un Tiro Salvezza su Tempra. Se fallisce il Tiro Salvezza una creatura subisce 2d8 danni da suono e viene allontana 3 metri da te. Se supera il Tiro Salvezza, la creatura subisce la metà dei danni e non viene allontanata. Gli oggetti non ancorati che sono totalmente all'interno dell'area vengono spinti 3 metri lontano da te dall'effetto dell'incantesimo. L'incantesimo produce un rimbombo tonante udibile fino a 90 metri.

\textbf{Per ogni Successo Critico Magico} ottenuto nella Prova di Magia il danno aumenta di 1d8.

\incantesimo{Oscurità}
\noindent\colorbox{OBSSgold!10}{
\begin{minipage}{0.95\linewidth}
\begin{description}[noitemsep, topsep=0pt, parsep=0pt, partopsep=0pt, leftmargin=0cm, labelwidth=1.3cm]
	\item[\textbf{Lista}]: Invocazione
	\item[\textbf{Livello}]: 1, Comune
	\item[\textbf{Lancio}]: 2 Azioni
	\item[\textbf{Gittata}]: 18 metri
	\item[\textbf{Durata}]: 10 minuti
\end{description}
\end{minipage}}\smallskip

L'oscurità magica si propaga da un punto a gittata, scelto da te, per riempire una sfera di 3 metri di raggio per la durata dell'incantesimo. L'oscurità si propaga intorno agli angoli. Una creatura con scurovisione non può vedere in questa oscurità, e la luce non magica non può illuminarla.

Se il punto che hai scelto è su di un oggetto che stai trasportando o uno che non è indossato o trasportato, l'oscurità emana dall'oggetto e si muove con esso. Coprire completamente la fonte dell'oscurità con un oggetto opaco, come un vaso o un elmo, blocca l'oscurità.

Se qualsiasi parte dell'area di questo incantesimo si sovrappone con l'area di luce creata da un incantesimo con livello 2 o più basso, l'incantesimo che ha creato la luce viene dissolto.

\incantesimo{Palla di fango di Eithne}
\noindent\colorbox{OBSSgold!10}{
\begin{minipage}{0.95\linewidth}
\begin{description}[noitemsep, topsep=0pt, parsep=0pt, partopsep=0pt, leftmargin=0cm, labelwidth=1.3cm]
	\item[\textbf{Lista}]: Terra
	\item[\textbf{Livello}]: 1, Non Comune
	\item[\textbf{Lancio}]: 2 Azioni
	\item[\textbf{Gittata}]: 24 metri
	\item[\textbf{Durata}]: Istantanea
\end{description}
\end{minipage}}\smallskip

L'incantatore mima il gesto di tirare un sasso con una fionda in direzione del bersaglio ed esegue un Tiro per Colpire con incantesimi a distanza.
Se il Tiro per Colpire va a segno il bersaglio subisce 2d6 di danni contundenti e deve effettuare un Tiro Salvezza su Riflessi. Se il tiro salvezza fallisce il movimento del bersaglio diminuisce di 2 metri per 1 minuto.

\textbf{Per ogni Successo Critico Magico} ottenuto nella Prova di Magia scagli un sasso in più.

\begin{enfasi}{
			Mi sparpaglio in giro per evitare incantesimi ad area (detta da un giocatore per evitare una Palla di Fuoco)
}\end{enfasi}

\incantesimo{Palla di Fuoco}
\noindent\colorbox{OBSSgold!10}{
\begin{minipage}{0.95\linewidth}
\begin{description}[noitemsep, topsep=0pt, parsep=0pt, partopsep=0pt, leftmargin=0cm, labelwidth=1.3cm]
	\item[\textbf{Lista}]: Fuoco
	\item[\textbf{Livello}]: 3, Comune
	\item[\textbf{Lancio}]: 2 Azioni
	\item[\textbf{Gittata}]: 45 metri
	\item[\textbf{Durata}]: Istantanea
\end{description}
\end{minipage}}\smallskip

Un fascio di luce gialla parte dal tuo dito puntato verso un punto a gittata scelto da te e poi esplode con un boato roboante e si trasforma in lingua di fiamme.

Ogni creatura in una sfera di 6 metri di raggio centrata in quel punto deve effettuare un Tiro Salvezza su Riflessi. Una creatura subisce 8d6 danni da fuoco se fallisce il Tiro Salvezza, o la metà di questi danni se lo supera.

Il fuoco si propaga ed occupa tutto il volume disponibile entro i 6 metri dal punto di esplosione. Il fuoco incendia gli oggetti infiammabili nell'area che non sono indossati o trasportati.

\textbf{Per ogni Successo Critico Magico} ottenuto nella Prova di Magia il danno base aumenta di 4d6.

\textbf{Tiro Salvezza Successo/Fallimento Critico}: In caso di Fallimento Critico il danno raddoppia, in caso di Successo Critico il danno viene ulteriormente dimezzato

\incantesimo{Palla di Fuoco Ritardata}
\noindent\colorbox{OBSSgold!10}{
\begin{minipage}{0.95\linewidth}
\begin{description}[noitemsep, topsep=0pt, parsep=0pt, partopsep=0pt, leftmargin=0cm, labelwidth=1.3cm]
	\item[\textbf{Lista}]: Fuoco
	\item[\textbf{Livello}]: 7, Raro
	\item[\textbf{Lancio}]: 2 Azioni
	\item[\textbf{Gittata}]: 45 metri
	\item[\textbf{Durata}]: Concentrazione, 1 minuto
\end{description}
\end{minipage}}\smallskip

Un fascio di luce gialla parte dal tuo dito puntato, per condensarsi per la durata dell'incantesimo nella forma di una pallina luminosa in un punto a gittata, scelto da te. Quando l'incantesimo termina, o perché la tua concentrazione è spezzata o perché decidi tu di porgli fine, la pallina esplode con un boato sommesso e si trasforma in un getto di fiamme che si propaga dietro gli angoli. Ogni creatura in una sfera di 6 metri di raggio centrata in quel punto deve effettuare un Tiro Salvezza su Riflessi. Una creatura subisce danni da fuoco pari al danno totale accumulato se fallisce il Tiro Salvezza, o la metà di questi danni se lo supera. Il danno base dell'incantesimo è 12d6. Se al termine del tuo round la pallina non è ancora detonata, il danno aumenta di 1d6.

Se la pallina luminosa viene toccata prima che l'incantesimo abbia avuto fine la pallina esplode.

Il fuoco danneggia gli oggetti nell'area e incendia gli oggetti infiammabili che non sono indossati o trasportati.

\textbf{Per ogni Successo Critico Magico} ottenuto nella Prova di Magia il danno aumento di 6d6.

\textbf{Tiro Salvezza Successo/Fallimento Critico}: In caso di Fallimento Critico il danno raddoppia, in caso di Successo Critico il danno viene ulteriormente dimezzato.

\incantesimo{Parlare con gli Animali}
\noindent\colorbox{OBSSgold!10}{
\begin{minipage}{0.95\linewidth}
\begin{description}[noitemsep, topsep=0pt, parsep=0pt, partopsep=0pt, leftmargin=0cm, labelwidth=1.3cm]
	\item[\textbf{Lista}]: Animali e Piante
	\item[\textbf{Livello}]: 1, Comune
	\item[\textbf{Lancio}]: 2 Azioni
	\item[\textbf{Gittata}]: Personale
	\item[\textbf{Durata}]: 10 minuti
\end{description}
\end{minipage}}\smallskip

Per la durata dell'incantesimo, ottieni la capacità di comprendere e comunicare verbalmente con le bestie. Il sapere e la consapevolezza di molte bestie sono limitati dal loro intelletto ma, come minimo, le bestie possono fornirti informazioni riguardo luoghi e mostri nelle vicinanze, compresi quelli che possono percepire o hanno percepito nei giorni passati. A discrezione del Narratore potresti riuscire a convincere una bestia a farti un piccolo favore.

\textbf{Per ogni Successo Critico Magico} ottenuto nella Prova di Magia la durata raddoppia.

\incantesimo{Parlare con i Morti}
\noindent\colorbox{OBSSgold!10}{
\begin{minipage}{0.95\linewidth}
\begin{description}[noitemsep, topsep=0pt, parsep=0pt, partopsep=0pt, leftmargin=0cm, labelwidth=1.3cm]
	\item[\textbf{Lista}]: Necromanzia
	\item[\textbf{Livello}]: 3, Raro
	\item[\textbf{Lancio}]: 2 Azioni
	\item[\textbf{Gittata}]: 3 metri
	\item[\textbf{Durata}]: 10 minuti
\end{description}
\end{minipage}}\smallskip

Conferisci un'apparenza di vita e Intelligenza a un cadavere a gittata, scelto da te, permettendogli di rispondere alle domande che gli poni. Il cadavere deve avere ancora una bocca e non può essere non morto. L'incantesimo fallisce se il cadavere è già stato bersaglio di questo incantesimo negli ultimi 10 giorni. Fino al termine dell'incantesimo, puoi porre al cadavere fino a cinque domande. Il cadavere conosce solo quello che già sapeva in vita, compresi i linguaggi parlati. Le risposte sono di solito brevi, criptiche o ripetitive, e il cadavere non è sotto nessun obbligo a darti risposte veritiere se gli sei ostile o ti riconosce come suo nemico. Questo incantesimo non riporta l'anima della creatura nel corpo, ma solo lo spirito che lo muove. Di conseguenza, il cadavere non può apprendere nuove informazioni, non capisce nulla di quello che è successo da quando è morto, e non può fare valutazioni su eventi futuri.

\incantesimo{Parlare con le Creature}
\noindent\colorbox{OBSSgold!10}{
\begin{minipage}{0.95\linewidth}
\begin{description}[noitemsep, topsep=0pt, parsep=0pt, partopsep=0pt, leftmargin=0cm, labelwidth=1.3cm]
	\item[\textbf{Lista}]: Animali e Piante, Divinazione
	\item[\textbf{Livello}]: 6, Raro
	\item[\textbf{Lancio}]: 2 Azioni
	\item[\textbf{Gittata}]: 9 metri
	\item[\textbf{Durata}]: 1 ora
\end{description}
\end{minipage}}\smallskip

Questo incantesimo è una versione più potente di parlare con gli animali, che consente di parlare con qualsiasi creatura entro la gittata, indipendentemente dalla sua natura o intelligenza (che deve essere maggiore di -5).

\incantesimo{Parlare con le Piante}
\noindent\colorbox{OBSSgold!10}{
\begin{minipage}{0.95\linewidth}
\begin{description}[noitemsep, topsep=0pt, parsep=0pt, partopsep=0pt, leftmargin=0cm, labelwidth=1.3cm]
	\item[\textbf{Lista}]: Animali e piante
	\item[\textbf{Livello}]: 3, Raro
	\item[\textbf{Lancio}]: 2 Azioni
	\item[\textbf{Gittata}]: Personale (raggio di 9 metri)
	\item[\textbf{Durata}]: 10 minuti
\end{description}
\end{minipage}}\smallskip

Infondi i vegetali entro 9 metri da te di capacità senziente e di limitata mobilità, dandole la capacità di comunicare con te ed eseguire dei semplici comandi. Puoi interrogare i vegetali in merito a eventi avvenuti nell'ultimo giorno nell'area dell'incantesimo, ottenendo informazioni sulle creature di passaggio, il clima e altro. Puoi anche trasformare il terreno difficile prodotto dalla crescita dei vegetali (come cespugli e fitto sottobosco) in terreno ordinario per la durata dell'incantesimo.

Oppure puoi trasformare del terreno normale in cui siano presenti dei vegetali in terreno difficile che rimane per la durata dell'incantesimo facendo sì, per esempio, che rampicanti e rami rallentino gli inseguitori.

A discrezione del Narratore i vegetali potrebbero svolgere anche altri compiti per tuo conto. L'incantesimo non permette ai vegetali di sradicarsi e muoversi, ma possono muovere liberamente rami, steli e gambi. Se una creatura vegetale si trova nell'area, puoi comunicare con essa come se parlaste lo stesso linguaggio, ma non ottieni alcuna capacità magica per influenzarla. Questo incantesimo può far sì che i vegetali creati dall'incantesimo intralciare rilascino una creatura intralciata.

\incantesimo{Parola Divina}
\noindent\colorbox{OBSSgold!10}{
\begin{minipage}{0.95\linewidth}
\begin{description}[noitemsep, topsep=0pt, parsep=0pt, partopsep=0pt, leftmargin=0cm, labelwidth=1.3cm]
	\item[\textbf{Lista}]: Invocazione
	\item[\textbf{Livello}]: 7, Molto Raro
	\item[\textbf{Lancio}]: 2 Azioni
	\item[\textbf{Gittata}]: 9 metri
	\item[\textbf{Durata}]: Istantanea
\end{description}
\end{minipage}}\smallskip

Pronunci una parola divina, infusa del potere del tuo Patrono. Scegli un qualsiasi numero di creature a gittata e che puoi vedere. Ogni creatura che può udirti deve effettuare un Tiro Salvezza su Volontà. Se fallisce il Tiro Salvezza, la creatura subisce un effetto in base ai suoi attuali Punti Ferita:

\begin{itemize}[leftmargin=*] \setlength{\itemsep}{0pt}
	\item 100 Punti Ferita o meno: assordata per 1 minuto
	\item 40 Punti Ferita o meno: assordata e accecata per 10 minuti
	\item 30 Punti Ferita o meno: accecata, assordata e stordita per 1 ora
	\item 20 Punti Ferita o meno: uccisa all'istante
\end{itemize}

Quali che siano i suoi attuali Punti Ferita, un celestiale, elementale, fatato o demone che fallisca il Tiro Salvezza è obbligato a tornare al suo piano di origine (se non vi si trova già) e non può tornare sul tuo attuale piano prima che siano passate 24 ore, a meno dell'uso dell'incantesimo desiderio.

\incantesimo{Parola del Potere Stordire}
\noindent\colorbox{OBSSgold!10}{
\begin{minipage}{0.95\linewidth}
\begin{description}[noitemsep, topsep=0pt, parsep=0pt, partopsep=0pt, leftmargin=0cm, labelwidth=1.3cm]
	\item[\textbf{Lista}]: Ammaliamento
	\item[\textbf{Livello}]: 8, Non Comune
	\item[\textbf{Lancio}]: 2 Azioni
	\item[\textbf{Gittata}]: 18 metri
	\item[\textbf{Durata}]: 1 minuti
\end{description}
\end{minipage}}\smallskip

Pronunci una parola di potere che può travolgere la mente di una creatura a gittata e che puoi vedere. Se il bersaglio ha 150 Punti Ferita o meno, è stordito per 2d4 round, altrimenti l'incantesimo non ha effetto.

\incantesimo{Parola del Potere Uccidere}
\noindent\colorbox{OBSSgold!10}{
\begin{minipage}{0.95\linewidth}
\begin{description}[noitemsep, topsep=0pt, parsep=0pt, partopsep=0pt, leftmargin=0cm, labelwidth=1.3cm]
	\item[\textbf{Lista}]: Ammaliamento
	\item[\textbf{Livello}]: 9, Raro
	\item[\textbf{Lancio}]: 2 Azioni
	\item[\textbf{Gittata}]: 18 metri
	\item[\textbf{Durata}]: Istantanea
\end{description}
\end{minipage}}\smallskip

Pronunci una parola di potere che costringe a morire all'istante una creatura a gittata che puoi vedere. Se la creatura che scegli ha 100 Punti Ferita o meno, muore. Altrimenti l'incantesimo non ha effetto.

\incantesimo{Parola del Ritiro}
\noindent\colorbox{OBSSgold!10}{
\begin{minipage}{0.95\linewidth}
\begin{description}[noitemsep, topsep=0pt, parsep=0pt, partopsep=0pt, leftmargin=0cm, labelwidth=1.3cm]
	\item[\textbf{Lista}]: Evocazione
	\item[\textbf{Livello}]: 6, Raro
	\item[\textbf{Lancio}]: 2 Azioni
	\item[\textbf{Gittata}]: 1 metro
	\item[\textbf{Durata}]: Istantanea
\end{description}
\end{minipage}}\smallskip

Te e fino a cinque creature consenzienti entro 1 metro da te vi teletrasportate istantaneamente in un luogo sicuro indicato precedentemente, detto santuario. Tu e tutte le creature teletrasportate con te, riapparite nello spazio non occupato più vicino al punto indicato quando hai preparato questo santuario (vedi sotto). Se lanci questo incantesimo senza aver prima preparato un santuario, l'incantesimo non ha effetto.

Devi indicare un santuario, che sia dedicato o fortemente collegato al tuo Patrono. Se tenti di lanciare l'incantesimo perché ti porti in un'area che non sia dedicata dal tuo Patrono, l'incantesimo non ha effetto.

\textbf{NOTA}: devi essere un Devoto o Seguace per poter lanciare questo incantesimo.

\incantesimo{Passa Porta}
\noindent\colorbox{OBSSgold!10}{
\begin{minipage}{0.95\linewidth}
\begin{description}[noitemsep, topsep=0pt, parsep=0pt, partopsep=0pt, leftmargin=0cm, labelwidth=1.3cm]
	\item[\textbf{Lista}]: Terra
	\item[\textbf{Livello}]: 5, Non Comune
	\item[\textbf{Lancio}]: 2 Azioni
	\item[\textbf{Gittata}]: 9 metri
	\item[\textbf{Durata}]: 1 ora
\end{description}
\end{minipage}}\smallskip

Per la durata dell'incantesimo, compare un passaggio in un punto a gittata che puoi vedere, su di una superficie di legno, muro o pietra (come una parete, un soffitto o un pavimento) scelta da te. Scegli le dimensioni dell'apertura: al massimo larga 1 metro, alta 2 metri e profonda 6 metri. Il passaggio non crea instabilità nella struttura che lo circonda.

Quando l'apertura sparisce, qualsiasi creatura od oggetto ancora nel passaggio creato dall'incantesimo viene espulso al sicuro nello spazio non occupato più vicino alla superficie su cui hai lanciato l'incantesimo.

\textbf{Per ogni Successo Critico Magico} puoi creare un altra porta pur nella durata dell'incantesimo.

\incantesimo{Passare Senza Tracce}
\noindent\colorbox{OBSSgold!10}{
\begin{minipage}{0.95\linewidth}
\begin{description}[noitemsep, topsep=0pt, parsep=0pt, partopsep=0pt, leftmargin=0cm, labelwidth=1.3cm]
	\item[\textbf{Lista}]: Terra, Animali e Piante
	\item[\textbf{Livello}]: 2, Comune
	\item[\textbf{Lancio}]: 2 Azioni
	\item[\textbf{Gittata}]: Creatura toccata
	\item[\textbf{Durata}]: 1 ora
\end{description}
\end{minipage}}\smallskip

Per la durata dell'incantesimo la creatura toccata non lascia tracce sul terreno.

\textbf{Per ogni Successo Critico Magico} ottenuto nella Prova di Magia puoi influenzare un altra creatura.

\incantesimo{Passo Velato}
\noindent\colorbox{OBSSgold!10}{
\begin{minipage}{0.95\linewidth}
\begin{description}[noitemsep, topsep=0pt, parsep=0pt, partopsep=0pt, leftmargin=0cm, labelwidth=1.3cm]
	\item[\textbf{Lista}]: Evocazione
	\item[\textbf{Livello}]: 2, Raro
	\item[\textbf{Lancio}]: 1 Azione
	\item[\textbf{Gittata}]: Personale
	\item[\textbf{Durata}]: Istantanea
\end{description}
\end{minipage}}\smallskip

Avvolto rapidamente da una foschia argentata, ti teletrasporti di massimo 9 metri in uno spazio non occupato che puoi vedere.

\textbf{Se ottieni due Successo Critico Magico ottenuto} nella Prova di Magia puoi scambiarti di posto con una creatura consenziente.

\textbf{NOTA}: se sei un devoto di Lynx l'incantesimo ha tempo di lancio di 1 Azione Immediata e la rarità è Non Comune.

\incantesimo{Passo Veloce}
\noindent\colorbox{OBSSgold!10}{
\begin{minipage}{0.95\linewidth}
\begin{description}[noitemsep, topsep=0pt, parsep=0pt, partopsep=0pt, leftmargin=0cm, labelwidth=1.3cm]
	\item[\textbf{Lista}]: Trasmutazione
	\item[\textbf{Livello}]: 1, Molto Raro
	\item[\textbf{Lancio}]: 2 Azioni
	\item[\textbf{Gittata}]: Contatto
	\item[\textbf{Durata}]: 1 ora
\end{description}
\end{minipage}}\smallskip

Il movimento di una creatura aumenta di 1 metro fino al termine dell'incantesimo.

\textbf{Per ogni Successo Critico Magico} ottenuto nella Prova di Magia puoi prendere come bersaglio un'ulteriore creatura.

\incantesimo{Paura}
\noindent\colorbox{OBSSgold!10}{
\begin{minipage}{0.95\linewidth}
\begin{description}[noitemsep, topsep=0pt, parsep=0pt, partopsep=0pt, leftmargin=0cm, labelwidth=1.3cm]
	\item[\textbf{Lista}]: Illusione
	\item[\textbf{Livello}]: 3, Non Comune
	\item[\textbf{Lancio}]: 2 Azioni
	\item[\textbf{Gittata}]: Personale (cono di 9 metri)
	\item[\textbf{Durata}]: 1 minuto
\end{description}
\end{minipage}}\smallskip

Proietti un'immagine illusoria delle peggiori paure di una creatura. Ogni creatura in un cono di 9 metri deve superare un Tiro Salvezza su Volontà o far cadere qualsiasi cosa stia impugnando e restare Spaventata per la durata dell'incantesimo.

Mentre è \hyperlink{condizionepaura}{spaventata} da questo incantesimo, una creatura deve, durante ciascun suo round, muoversi lontano da te tramite il tragitto più sicuro, a meno che non abbia spazio per muoversi. Se la creatura termina il suo round in un posto dove non può vederti, può effettuare un Tiro Salvezza su Volontà, se lo supera, l'incantesimo, per quella creatura, ha termine.

\incantesimo{Pelle di Corteccia}
\noindent\colorbox{OBSSgold!10}{
\begin{minipage}{0.95\linewidth}
\begin{description}[noitemsep, topsep=0pt, parsep=0pt, partopsep=0pt, leftmargin=0cm, labelwidth=1.3cm]
	\item[\textbf{Lista}]: Animali e Piante
	\item[\textbf{Livello}]: 2, Comune
	\item[\textbf{Lancio}]: 2 Azioni
	\item[\textbf{Gittata}]: Contatto
	\item[\textbf{Durata}]: 1 ora
\end{description}
\end{minipage}}\smallskip

La pelle del bersaglio con cui sei in contatto al momento del lancio dell'incantesimo diventa ruvida e dall'aspetto simile alla corteccia fino al termine dell'incantesimo e la Difesa naturale del bersaglio aumenta di 1 + 1/6 Competenza Magica indipendentemente dall'armatura che stia indossando.

\incantesimo{Pelle di Pietra}
\noindent\colorbox{OBSSgold!10}{
\begin{minipage}{0.95\linewidth}
\begin{description}[noitemsep, topsep=0pt, parsep=0pt, partopsep=0pt, leftmargin=0cm, labelwidth=1.3cm]
	\item[\textbf{Lista}]: Terra
	\item[\textbf{Livello}]: 4, Non Comune
	\item[\textbf{Lancio}]: 2 Azioni
	\item[\textbf{Gittata}]: Contatto
	\item[\textbf{Durata}]: 1 ora
\end{description}
\end{minipage}}\smallskip

Lanci l'incantesimo a contatto di una creatura consenziente, la cui pelle si tramuta in una sostanza dura come la pietra. Tira 1d4+metà del valore di CM, la somma risultante è le volte che un attacco con arma di mischia o distanza viene annullato (indipendentemente che di colpisca o meno).

\textbf{Per ogni Successo Critico Magico} ottenuto nella Prova di Magia aumenti di 2 gli attacchi annullati.

\incantesimo{Piaga degli Insetti}
\noindent\colorbox{OBSSgold!10}{
\begin{minipage}{0.95\linewidth}
\begin{description}[noitemsep, topsep=0pt, parsep=0pt, partopsep=0pt, leftmargin=0cm, labelwidth=1.3cm]
	\item[\textbf{Lista}]: Animali e Piante
	\item[\textbf{Livello}]: 5, Raro
	\item[\textbf{Lancio}]: 2 Azioni
	\item[\textbf{Gittata}]: 90 metri
	\item[\textbf{Durata}]: 10 minuti
\end{description}
\end{minipage}}\smallskip

Uno sciame di locuste affamate riempie una sfera di 6 metri di raggio centrata in un punto a gittata scelto da te. La sfera si propaga intorno agli angoli. La sfera rimane per la durata dell'incantesimo, e la sua area è in penombra. L'area della sfera è terreno difficile.

Quando l'area appare, ogni creatura al suo interno deve effettuare un Tiro Salvezza su Tempra. Una creatura subisce 4d10 danni se fallisce il Tiro Salvezza, o la metà di questi danni se lo supera. Una creatura deve effettuare questo Tiro Salvezza anche quando entra per la prima volta nell'area dell'incantesimo durante un round o se termina il proprio round al suo interno.

\textbf{Per ogni Successo Critico Magico} ottenuto nella Prova di Magia il danno aumenta di 2d10.

\incantesimo{Pietra in Fango}

\incantesimo{Fango in Pietra}

\noindent
\begin{description}[noitemsep, topsep=0pt, parsep=0pt, partopsep=0pt, leftmargin=0cm, labelwidth=1.3cm]
	\item[\textbf{Lista}]: Terra
	\item[\textbf{Livello}]: 5, Non Comune - Molto Raro
	\item[\textbf{Lancio}]: 2 Azioni
	\item[\textbf{Gittata}]: 45 metri
	\item[\textbf{Durata}]: Istantanea
\end{description}

Questo incantesimo trasforma qualsiasi tipo di roccia naturale in un eguale volume di fango. La pietra magica non viene influenzata dall'incantesimo. L'incantesimo ha effetto fino a 2 cubi di 3x3x3 metri. La profondità del fango creato non può superare i 3 metri. Le creature incapaci di Volare, levitare o allontanarsi in qualche modo dal fango affondano fino alla vita o fino al petto; le creatura sono intralciate ed il terreno diviene doppiamente difficile. Le creature abbastanza grandi da camminare sul fondo della pozza di fango possono guadare l'area come terreno difficile.

Se Pietra in Fango viene lanciato sul soffitto di una caverna o di un tunnel, il fango si riversa sul pavimento e si espande fino a formare una pozza della profondità di 1 metro. Il fango in caduta e la frana che ne segue infliggono 8d6 danni contundenti a chiunque si trovi direttamente sotto l'area se non dimezza i danni con un Tiro Salvezza su Riflessi.

I castelli e i grandi edifici in pietra sono generalmente immuni agli effetti dell'incantesimo, in quanto Trasformare Pietra in Fango non arriva abbastanza in profondità da minare le fondamenta. Tuttavia altri edifici più piccoli spesso poggiano su fondamenta abbastanza superficiali da poter essere danneggiate o perfino distrutte dagli effetti dell'incantesimo.

Il fango rimane finché non viene usato con successo un incantesimo Dissolvi Magie o Fango in Pietra, che ripristina la sua sostanza, ma non necessariamente la sua forma. L'evaporazione naturale trasforma il fango in terreno normale nel giro di diversi giorni a seconda dell'esposizione al sole, al vento ed all'essiccazione naturale.
Se una creatura è nel fango al momento dell'incantesimo Fango in Pietra può effettuare un Tiro Salvezza su Riflessi per liberarsi altrimenti è necessario un Tiro Salvezza Tempra con Forza a DC 26 oppure 30 di danno, per rompere la pietra.

\textbf{Per ogni Successo Critico Magico} ottenuto nella Prova di Magia influenzi un cubo di 3x3x3 metri in più.

\incantesimo{Pietre Parlanti}
\noindent\colorbox{OBSSgold!10}{
\begin{minipage}{0.95\linewidth}
\begin{description}[noitemsep, topsep=0pt, parsep=0pt, partopsep=0pt, leftmargin=0cm, labelwidth=1.3cm]
	\item[\textbf{Lista}]: Terra, Divinazione
	\item[\textbf{Livello}]: 6, Raro
	\item[\textbf{Lancio}]: 2 Azioni
	\item[\textbf{Gittata}]: Tocco
	\item[\textbf{Durata}]: 1 turno
\end{description}
\end{minipage}}\smallskip

L'incantatore acquisisce la capacità di parlare con le pietre, le quali possono dire chi o cosa le abbia toccate e rivelare ciò che nascondono dietro o sotto di loro. Le pietre forniscono descrizioni accurate a richiesta, anche se potrebbero non fornire i dettagli desiderati in certe situazioni. L'incantatore può parlare con pietre sia naturali che lavorate.

\textbf{Per ogni Successo Critico Magico} ottenuto nella Prova di Magia la durata raddoppia oppure può parlare con un altra pietra entro 18 metri dalla prima.

\incantesimo{Pioggia di Meteore}\hypertarget{sciamedimeteore}{}
\begin{description}[noitemsep, topsep=0pt, parsep=0pt, partopsep=0pt, leftmargin=0cm, labelwidth=1.3cm]
	\item[\textbf{Lista}]: Fuoco, Terra
	\item[\textbf{Livello}]: 9, Leggendario
	\item[\textbf{Lancio}]: 3 Azioni
	\item[\textbf{Gittata}]: 1,5 chilometri
	\item[\textbf{Durata}]: Istantanea
\end{description}

4 meteoriti di fuoco si schiantano a terra in quattro punti differenti a gittata e che puoi vedere. Ogni meteorite colpisce in un raggio di 3 metri. Ogni creatura interessata deve effettuare un Tiro Salvezza su Riflessi. Una creatura subisce 20d6 danni da fuoco e 20d6 danni contundenti se fallisce il Tiro Salvezza, o la metà di
questi danni se lo supera. Una creatura se nell'area di più di un meteorite ne subisce gli effetti solo do uno.

\textbf{Tiro Salvezza Successo/Fallimento Critico}: In caso di Fallimento Critico il danno raddoppia, in caso di Successo Critico il danno viene ulteriormente dimezzato

\textbf{Ogni 3 critici ottenuti} nella Prova di Magia scagli un altro meteorite.

\incantesimo{Piroesperto}
\noindent\colorbox{OBSSgold!10}{
\begin{minipage}{0.95\linewidth}
\begin{description}[noitemsep, topsep=0pt, parsep=0pt, partopsep=0pt, leftmargin=0cm, labelwidth=1.3cm]
	\item[\textbf{Lista}]: Fuoco
	\item[\textbf{Livello}]: 2, Non Comune
	\item[\textbf{Lancio}]: 2 Azioni
	\item[\textbf{Gittata}]: 18 metri
	\item[\textbf{Durata}]: Istantanea
\end{description}
\end{minipage}}\smallskip

L'incantatore sceglie un'area con un fuoco, di almeno 1 metro di spigolo, entro gittata che sia a lui direttamente visibile. Estinguendo le fiamme può creare fuochi d'artificio o fumo.

\begin{itemize}[leftmargin=*] \setlength{\itemsep}{0pt}
	\item \emph{Fuochi d'Artificio}. Il fuoco bersaglio esplode in uno spettacolo luminoso di fiamme e colori. Ogni creatura entro 3 metri dal bersaglio deve superare un Tiro Salvezza su Tempra oppure diventa accecata fino alla fine round successivo.
	\item \emph{Fumo}. Un fumo nero e denso scaturisce dal fuoco bersaglio e si diffonde in un raggio di 6 metri, muovendosi oltre gli angoli. L'area del fumo è pesantemente oscurata e fornisce copertura media. Il fumo persiste per 1 minuto o finché un vento forte non lo disperde.
\end{itemize}

\incantesimo{Polvere luccicante}
\noindent\colorbox{OBSSgold!10}{
\begin{minipage}{0.95\linewidth}
\begin{description}[noitemsep, topsep=0pt, parsep=0pt, partopsep=0pt, leftmargin=0cm, labelwidth=1.3cm]
	\item[\textbf{Lista}]: Fuoco, Aria
	\item[\textbf{Livello}]: 2, Non Comune
	\item[\textbf{Lancio}]: 2 Azioni
	\item[\textbf{Gittata}]: 36 metri
	\item[\textbf{Durata}]: 1 round per CM
\end{description}
\end{minipage}}\smallskip

In una sfera di 3 metri di diametro chiunque si trovi viene ricoperto da polvere luccicante e luminosa. La nuvola delinea le creature presenti, anche quelle invisibili e chiunque permanga nell'area deve fare ad inizio round un Tiro Salvezza su Riflessi od essere accecata per il round. La polvere scompare naturalmente dopo la durata o se portata via da un vento anche leggero.

\incantesimo{Porta Dimensionale}
\noindent\colorbox{OBSSgold!10}{
\begin{minipage}{0.95\linewidth}
\begin{description}[noitemsep, topsep=0pt, parsep=0pt, partopsep=0pt, leftmargin=0cm, labelwidth=1.3cm]
	\item[\textbf{Lista}]: Evocazione
	\item[\textbf{Livello}]: 4, Comune
	\item[\textbf{Lancio}]: 2 Azioni
	\item[\textbf{Gittata}]: 150 metri
	\item[\textbf{Durata}]: Istantanea
\end{description}
\end{minipage}}\smallskip

Ti teletrasporti dalla tua attuale posizione in qualsiasi altro posto a gittata. Arrivi esattamente nel posto desiderato. Può essere un luogo che puoi vedere, uno che puoi visualizzare, o uno che puoi descrivere indicando distanza e direzione, come \emph{30 metri verso il basso} o \emph{90 metri in alto a nordovest con un angolo di 45 gradi}.

Puoi portare con te oggetti il cui peso non ecceda la tua capacità di Ingombro. Puoi portare con te anche una creatura consenziente della tua taglia o più piccola con equipaggiamento fino al limite della sua capacità di carico. La creatura deve essere entro 1 metro da te quando lanci questo incantesimo.

Se dovessi arrivare in un posto già occupato da un oggetto o creatura, tu e la creatura che viaggia con te subite ciascuno 4d6 danni da forza, e l'incantesimo non riesce a teletrasportarvi.

\textbf{Per ogni due Successo Critico Magico ottenuto} nella Prova di Magia puoi portare una ulteriore creatura.

\incantesimo{Preghiera}
\noindent\colorbox{OBSSgold!10}{
\begin{minipage}{0.95\linewidth}
\begin{description}[noitemsep, topsep=0pt, parsep=0pt, partopsep=0pt, leftmargin=0cm, labelwidth=1.3cm]
	\item[\textbf{Lista}]: Invocazione
	\item[\textbf{Livello}]: 3, Non Comune
	\item[\textbf{Lancio}]: 2 Azioni
	\item[\textbf{Gittata}]: Personale
	\item[\textbf{Durata}]: 1 round per livello, Concentrazione
\end{description}
\end{minipage}}\smallskip

Intoni un canto al tuo Patrono e ne invochi la benedizione. Le creature entro 9 metri da te prendono un +1 al Tiro per Colpire e Tiro Salvezza per tratto in comune con il Patrono.

Le creature con Tratti diversi dal Patrono prendono un -1 hai Tiri per Colpire e Tiri Salvezza.

\textbf{Nota}: devi essere un Devoto per poter lanciare questo incantesimo, e puoi influenzare un numero di creature pari al tuo punteggio Tratti più alto condiviso con il Patrono.

\incantesimo{Preghiera di Guarigione}
\noindent\colorbox{OBSSgold!10}{
\begin{minipage}{0.95\linewidth}
\begin{description}[noitemsep, topsep=0pt, parsep=0pt, partopsep=0pt, leftmargin=0cm, labelwidth=1.3cm]
	\item[\textbf{Lista}]: Cura
	\item[\textbf{Livello}]: 2, Comune
	\item[\textbf{Lancio}]: 10 minuti
	\item[\textbf{Gittata}]: 9 metri
	\item[\textbf{Durata}]: Istantanea
\end{description}
\end{minipage}}\smallskip

Fino a sei creature a gittata che puoi vedere, scelte da te, recuperano ciascuna Punti Ferita pari a 2d6 + il tuo modificatore di caratteristica per incantesimi. Questo incantesimo causa lo stesso ammontare di danno sui non morti.

\textbf{Per ogni Successo Critico Magico} ottenuto nella Prova di Magia la cura aumenta di 1d8.

\incantesimo{Presagio}
\noindent\colorbox{OBSSgold!10}{
\begin{minipage}{0.95\linewidth}
\begin{description}[noitemsep, topsep=0pt, parsep=0pt, partopsep=0pt, leftmargin=0cm, labelwidth=1.3cm]
	\item[\textbf{Lista}]: Divinazione
	\item[\textbf{Livello}]: 2, Comune
	\item[\textbf{Lancio}]: 1 minuto
	\item[\textbf{Gittata}]: Personale
	\item[\textbf{Durata}]: Istantanea
\end{description}
\end{minipage}}\smallskip

Gettando bastoncini intarsiati con gemme, facendo rotolare ossa di drago, impilando carte elaborate o impiegando qualche altro strumento di divinazione, ricevi un presagio da un'entità ultraterrena riguardo il risultato di uno specifico corso di azione che intendi intraprendere nei prossimi 30 minuti. Il Narratore sceglie tra i seguenti presagi:

\medskip

- Prosperità, per i risultati positivi

- Calamità, per i risultati negativi

- Prosperità e calamità, per i risultati sia positivi che negativi

- Nulla, per i risultati che non sono né particolarmente positivi né negativi

L'incantesimo non tiene conto di ogni possibile circostanza che possa modificare il risultato, come il lancio di ulteriori incantesimi o la perdita o l'arrivo di un alleato. Se lanci l'incantesimo due o più volte prima che sia sorto il nuovo sole, c'è una probabilità cumulativa del 25\% che per ogni lancio dopo il primo tu ottenga una lettura erronea. Il Narratore effettua questo tiro in segreto.

\incantesimo{Prestidigitazione}
\noindent\colorbox{OBSSgold!10}{
\begin{minipage}{0.95\linewidth}
\begin{description}[noitemsep, topsep=0pt, parsep=0pt, partopsep=0pt, leftmargin=0cm, labelwidth=1.3cm]
	\item[\textbf{Lista}]: Universale
	\item[\textbf{Livello}]: 0, Comune
	\item[\textbf{Lancio}]: 2 Azioni
	\item[\textbf{Gittata}]: 3 metri
	\item[\textbf{Durata}]: Massimo 1 ora
\end{description}
\end{minipage}}\smallskip

Questo incantesimo è un trucco magico minore che gli incantatori novizi impiegano per fare pratica. Crei a gittata uno dei seguenti effetti magici:

\begin{itemize}[leftmargin=*] \setlength{\itemsep}{0pt}
	\item Crei un effetto sensoriale innocuo e istantaneo come una pioggia di scintille, un soffio di vento, una debole nota musicale o uno strano odore.
	\item Illumini o spegni istantaneamente una candela, una torcia o piccolo fuoco da campo.
	\item Ripulisci o insozzi istantaneamente un oggetto non più grosso di 30 cm di lato.
	\item Raffreddi, riscaldi o insapori per 1 ora un cubo di 30 cm di lato di materiale non vivente.
	\item Fai comparire per 1 ora un colore, un piccolo segno o un simbolo su di un oggetto o una superficie.
	\item Crei un ninnolo non magico o un'immagine illusoria che entri nella tua mano e che resta fino al termine del tuo prossimo round.
\end{itemize}

Se lanci questo incantesimo più volte, puoi tenere attivi fino a tre effetti non istantanei alla volta, e puoi interrompere uno di questi effetti con un'Azione.

\textbf{Per ogni Successo Critico Magico} ottenuto nella Prova di Magia puoi attivare un effetto magico in più.

\incantesimo{Previsione}
\noindent\colorbox{OBSSgold!10}{
\begin{minipage}{0.95\linewidth}
\begin{description}[noitemsep, topsep=0pt, parsep=0pt, partopsep=0pt, leftmargin=0cm, labelwidth=1.3cm]
	\item[\textbf{Lista}]: Divinazione
	\item[\textbf{Livello}]: 9, Non Comune
	\item[\textbf{Lancio}]: 1 minuto
	\item[\textbf{Gittata}]: Contatto
	\item[\textbf{Durata}]: 8 ore
\end{description}
\end{minipage}}\smallskip

Lanci l'incantesimo a contatto di una creatura consenziente per conferirle una limitata capacità di vedere nell'immediato futuro. Per la durata, il bersaglio non può essere sorpreso e ha +1d6 sui Tiri per Colpire, prove su competenze di base e Tiri Salvezza. Inoltre, sempre per la durata, le altre creature hanno -1d6 sui Tiri per Colpire contro il bersaglio. L'incantesimo ha immediatamente termine se lo lanci di nuovo prima che la sua durata abbia fine.

\incantesimo{Produrre Fiamma}
\noindent\colorbox{OBSSgold!10}{
\begin{minipage}{0.95\linewidth}
\begin{description}[noitemsep, topsep=0pt, parsep=0pt, partopsep=0pt, leftmargin=0cm, labelwidth=1.3cm]
	\item[\textbf{Lista}]: Fuoco
	\item[\textbf{Livello}]: 0, Comune
	\item[\textbf{Lancio}]: 1 Azione
	\item[\textbf{Gittata}]: Personale
	\item[\textbf{Durata}]: 10 minuti
\end{description}
\end{minipage}}\smallskip

Una fiammella compare nella tua mano. La fiammella resta lì per la durata dell'incantesimo e non danneggia né te né il tuo equipaggiamento. La fiamma produce luce fioca nel raggio di 1 metro. L'incantesimo termina se lo interrompi con un'Azione o se lo lanci di nuovo.

Puoi usare la fiamma anche per attaccare, sebbene farlo ponga termine all'incantesimo. Quando lanci questo incantesimo, o con un'Azione in un round successivo, puoi scagliare la fiamma a una creatura entro 9 metri da te. Effettua un attacco a distanza con incantesimo. Se colpisci, il bersaglio subisce 1d8 danni da fuoco.

Il danno dell'incantesimo aumenta di 1d8 quando raggiungi CM 5, CM 11 e CM 17, ma costa 2 Azioni lanciarlo potenziato e 1 Punti Magia, è altresì necessario avere preso Adepto della Magia un numero di volte pari ai potenziamenti che si vogliono applicare.

\textbf{Per ogni Successo Critico Magico} ottenuto nella Prova di Magia puoi attaccare una creatura in più senza terminare l'incantesimo.

\incantesimo{Profumo di Atherim}\label{Aura of Purity}
\noindent\colorbox{OBSSgold!10}{
\begin{minipage}{0.95\linewidth}
\begin{description}[noitemsep, topsep=0pt, parsep=0pt, partopsep=0pt, leftmargin=0cm, labelwidth=1.3cm]
	\item[\textbf{Lista di Magia}] : Cura
	\item[\textbf{Livello}] : 4, Raro
	\item[\textbf{T. di Lancio}] : 2 Azioni
	\item[\textbf{Gittata}] : Personale
	\item[\textbf{Durata}] : Concentrazione, fino a 10 minuti
\end{description}
\end{minipage}}\smallskip

Un profumo si irradia da te in un raggio di 4 metri per tutta la durata. Mentre si trovano in questo profumo tu e i tuoi alleati avete Resistenza ai danni da veleno e +4 ai Tiri Salvezza per evitare o terminare effetti che includono le condizioni di Accecato, Affascinato, Assordato, Spaventato, Paralizzato, Avvelenato o Stordito.

\textbf{Per ogni Successo Critico Magico} ottenuto nella Prova di Magia allarghi il raggio del profumo di 1 metro.

\incantesimo{Proibizione}
\noindent\colorbox{OBSSgold!10}{
\begin{minipage}{0.95\linewidth}
\begin{description}[noitemsep, topsep=0pt, parsep=0pt, partopsep=0pt, leftmargin=0cm, labelwidth=1.3cm]
	\item[\textbf{Lista}]: Abiurazione
	\item[\textbf{Livello}]: 6, Non Comune
	\item[\textbf{Lancio}]: 10 minuti
	\item[\textbf{Gittata}]: Contatto
	\item[\textbf{Durata}]: 1 giorno
\end{description}
\end{minipage}}\smallskip

Crei una interdizione al viaggio magico che protegge fino a 4000 metri quadri di pavimento, fino a un'altezza di 9 metri dal suolo. Per la durata dell'incantesimo, le creature non possono teletrasportarsi nell'area o usare passaggi, come quello creato dall'incantesimo portale, per entrare nell'area. L'incantesimo protegge l'area dal viaggio planare, e quindi impedisce alle creature di accedere all'area tramite il Piano Astrale, il Piano Etereo od il Piano delle Ombre.

Inoltre, l'incantesimo danneggia i tipi di creatura scelti da te durante il lancio. Scegli uno o più dei seguenti: celestiali, elementali, fatati, demoni e non morti. Quando una creatura selezionata entra nell'area dell'incantesimo per la prima volta in un round o inizia qui il suo round, la creatura subisce 5d10 danni da Luce o da Vuoto (a tua scelta, quando lanci l'incantesimo).

Quando lanci questo incantesimo, puoi stabilire una parola d'ordine. Una creatura che pronuncia la parola d'ordine mentre entra nell'area dell'incantesimo, non subisce danni da esso.

L'area dell'incantesimo non può sovrapporsi all'area di un altro incantesimo proibizione. Se esegui proibizione ogni giorno per 30 giorni nello stesso posto, l'incantesimo durerà finché non viene dissolto, e le componenti materiali saranno consumate durante l'ultimo lancio.

\incantesimo{Protezione dall'Energia}
\noindent\colorbox{OBSSgold!10}{
\begin{minipage}{0.95\linewidth}
\begin{description}[noitemsep, topsep=0pt, parsep=0pt, partopsep=0pt, leftmargin=0cm, labelwidth=1.3cm]
	\item[\textbf{Lista}]: Abiurazione
	\item[\textbf{Livello}]: 3, Comune
	\item[\textbf{Lancio}]: 2 Azioni
	\item[\textbf{Gittata}]: Contatto
	\item[\textbf{Durata}]: 10 minuti
\end{description}
\end{minipage}}\smallskip

Lanci l'incantesimo a contatto di una creatura consenziente. Per la durata dell'incantesimo, il bersaglio ha resistenza a un tipo di danno scelto da te: acido, freddo, fuoco, fulmine o suono. Puoi sacrificare tutta la durata dell'incantesimo, terminandolo, per annullare completamente il danno subito da una fonte di energia.

\textbf{Per ogni Successo Critico Magico} ottenuto nella Prova di Magia puoi influenzare un altra persona o aumentare la durata di 10 minuti.

\incantesimo{Protezione dall'Energia minore}
\noindent\colorbox{OBSSgold!10}{
\begin{minipage}{0.95\linewidth}
\begin{description}[noitemsep, topsep=0pt, parsep=0pt, partopsep=0pt, leftmargin=0cm, labelwidth=1.3cm]
	\item[\textbf{Lista}]: Abiurazione
	\item[\textbf{Livello}]: 1, Raro
	\item[\textbf{Lancio}]: 1 Reazione
	\item[\textbf{Gittata}]: Contatto
	\item[\textbf{Durata}]: 1 minuto
\end{description}
\end{minipage}}\smallskip

Lanci l'incantesimo a contatto di una creatura consenziente. Per la durata dell'incantesimo, il bersaglio ha Riduzione al Danno dall'energia scelta pari a 5. Puoi sacrificare tutta la durata dell'incantesimo, terminandolo, per ridurre di 20 il danno subito da una fonte di energia (come se avessi Resistenza al Danno 20 da quella fonte di energia).

\textbf{Per ogni Successo Critico Magico} ottenuto nella Prova di Magia puoi influenzare un altra persona o aumentare la durata di 1 minuto.

\incantesimo{Protezione dai Veleni}
\noindent\colorbox{OBSSgold!10}{
\begin{minipage}{0.95\linewidth}
\begin{description}[noitemsep, topsep=0pt, parsep=0pt, partopsep=0pt, leftmargin=0cm, labelwidth=1.3cm]
	\item[\textbf{Lista}]: Abiurazione
	\item[\textbf{Livello}]: 2, Non Comune
	\item[\textbf{Lancio}]: 2 Azioni
	\item[\textbf{Gittata}]: Contatto
	\item[\textbf{Durata}]: 1 ora
\end{description}
\end{minipage}}\smallskip

Per la durata dell'incantesimo, il bersaglio ha +1d6 ai Tiri Salvezza contro l'essere avvelenato, e ha resistenza al danno da veleno.

\textbf{In caso di due Successo Critico Magico ottenuto} nella Prova di Magia puoi annullare un veleno in circolo sul bersaglio.

\incantesimo{Punizione Marchiante}
\noindent\colorbox{OBSSgold!10}{
\begin{minipage}{0.95\linewidth}
\begin{description}[noitemsep, topsep=0pt, parsep=0pt, partopsep=0pt, leftmargin=0cm, labelwidth=1.3cm]
	\item[\textbf{Lista}]: Invocazione
	\item[\textbf{Livello}]: 2, Comune
	\item[\textbf{Lancio}]: 1 Azione
	\item[\textbf{Gittata}]: Personale
	\item[\textbf{Durata}]: 1 minuto
\end{description}
\end{minipage}}\smallskip

La prossima volta che colpisci una creatura con un attacco in mischia con arma nella durata dell'incantesimo, l'arma riluce di un bagliore magico mentre colpisci. L'attacco infligge 1d6 danni da Luce aggiuntivi al bersaglio, che diventa visibile qualora sia invisibile ed emette luce fioca in un raggio di 1 metro. Inoltre il bersaglio non può diventare invisibile fino al termine dell'incantesimo.

\textbf{Per ogni Successo Critico Magico} ottenuto nella Prova di Magia il danno aggiuntivo aumenta di 1d6.

\incantesimo{Purificare Cibo e Bevande}
\noindent\colorbox{OBSSgold!10}{
\begin{minipage}{0.95\linewidth}
\begin{description}[noitemsep, topsep=0pt, parsep=0pt, partopsep=0pt, leftmargin=0cm, labelwidth=1.3cm]
	\item[\textbf{Lista}]: Animali e Piante
	\item[\textbf{Livello}]: 1, Comune
	\item[\textbf{Lancio}]: 2 Azioni
	\item[\textbf{Gittata}]: 3 metri
	\item[\textbf{Durata}]: Istantanea
\end{description}
\end{minipage}}\smallskip

Tutti i cibi e le bevande non magiche in una sfera di 1 metro di raggio, centrata in un punto a gittata di tua scelta, vengono purificati e liberati da veleni e malattie. Un cibo in decomposizione viene ripulito e reso commestibile.

\incantesimo{Raggio di Gelo}
\noindent\colorbox{OBSSgold!10}{
\begin{minipage}{0.95\linewidth}
\begin{description}[noitemsep, topsep=0pt, parsep=0pt, partopsep=0pt, leftmargin=0cm, labelwidth=1.3cm]
	\item[\textbf{Lista}]: Acqua
	\item[\textbf{Livello}]: 0, Comune
	\item[\textbf{Lancio}]: 1 Azione
	\item[\textbf{Gittata}]: 18 metri
	\item[\textbf{Durata}]: Istantanea
\end{description}
\end{minipage}}\smallskip

Un fascio gelato di luce azzurra colpisce una creatura a gittata. Effettua un attacco a distanza con incantesimo contro il bersaglio. Se colpisci, egli subisce 1d8 danni da freddo, e la sua velocità è ridotta di 3 metri fino all'inizio del tuo prossimo round.

Puoi aumentare il danno dell'incantesimo di 1d8 quando raggiungi CM 5, CM 11 e CM 17, ma costa 2 Azioni lanciarlo potenziato e 1 Punti Magia, è altresì necessario avere preso Adepto della Magia un numero di volte pari ai potenziamenti che si vogliono applicare.

\textbf{Per ogni due Successo Critico Magico ottenuto} nella Prova di Magia crei un fascio gelato aggiuntivo.

\incantesimo{Raggio di Indebolimento}
\noindent\colorbox{OBSSgold!10}{
\begin{minipage}{0.95\linewidth}
\begin{description}[noitemsep, topsep=0pt, parsep=0pt, partopsep=0pt, leftmargin=0cm, labelwidth=1.3cm]
	\item[\textbf{Lista}]: Necromanzia
	\item[\textbf{Livello}]: 1, Comune
	\item[\textbf{Lancio}]: 2 Azioni
	\item[\textbf{Gittata}]: 18 metri
	\item[\textbf{Durata}]: 1 minuto
\end{description}
\end{minipage}}\smallskip

Un fascio nero di energia debilitante parte dal tuo dito diretto contro una creatura a gittata. Effettua un attacco a distanza con incantesimo contro il bersaglio. Se colpisci, il bersaglio, quando attacca con un arma che usa la Forza come modificatore, tirerà due volte per il danno prendendo il risultato inferiore fino al termine dell'incantesimo. Non si può essere influenzati da più di un Raggio di Indebolimento al giorno.

\textbf{Per ogni due Successo Critico Magico ottenuto} nella Prova di Magia aumenti di 1 il livello di Affaticamento del bersaglio.

\incantesimo{Raggio mortale}
\noindent\colorbox{OBSSgold!10}{
\begin{minipage}{0.95\linewidth}
\begin{description}[noitemsep, topsep=0pt, parsep=0pt, partopsep=0pt, leftmargin=0cm, labelwidth=1.3cm]
	\item[\textbf{Lista}]: Necromanzia
	\item[\textbf{Livello}]: 2, Raro
	\item[\textbf{Lancio}]: 2 Azioni
	\item[\textbf{Gittata}]: 6 metri
	\item[\textbf{Durata}]: Istantaneo
\end{description}
\end{minipage}}\smallskip

Un fascio nero di energia crepitante parte dalle tue mani verso una creatura a gittata. Effettua un attacco a distanza con incantesimo contro il bersaglio. Indipendentemente dal fatto se colpisci o meno, il bersaglio deve effettuare un Tiro Salvezza su Tempra.

Se il Tiro per Colpire va a segno la creatura effettua il Tiro Salvezza con -2 di penalità, se il Tiro per Colpire genera un Tiro Critico il Tiro Salvezza viene fatto con -4 di penalità.

Se manchi il Tiro Salvezza viene fatto senza modificatori aggiuntivi.

La condizione Affaticato della creatura aumenta di 1

\textbf{Per ogni tre Successo Critico Magico ottenuto} nella Prova di Magia aumenti di 1 il livello di Affaticamento del bersaglio.

\incantesimo{Raggio Rovente}
\noindent\colorbox{OBSSgold!10}{
\begin{minipage}{0.95\linewidth}
\begin{description}[noitemsep, topsep=0pt, parsep=0pt, partopsep=0pt, leftmargin=0cm, labelwidth=1.3cm]
	\item[\textbf{Lista}]: Fuoco
	\item[\textbf{Livello}]: 2, Comune
	\item[\textbf{Lancio}]: 2 Azioni
	\item[\textbf{Gittata}]: 36 metri
	\item[\textbf{Durata}]: Istantanea
\end{description}
\end{minipage}}\smallskip

Crei tre raggi di fuoco e li proietti verso tre bersagli a gittata. Puoi proiettarli contro lo stesso bersaglio o bersagli diversi. Effettua un attacco a distanza con incantesimo per ciascun raggio. Se colpisci, il bersaglio subisce 2d6 danni da fuoco.

\textbf{Per ogni Successo Critico Magico} ottenuto nella Prova di Magia crei un raggio aggiuntivo.

\incantesimo{Ragnatela}
\noindent\colorbox{OBSSgold!10}{
\begin{minipage}{0.95\linewidth}
\begin{description}[noitemsep, topsep=0pt, parsep=0pt, partopsep=0pt, leftmargin=0cm, labelwidth=1.3cm]
	\item[\textbf{Lista}]: Animali e Piante
	\item[\textbf{Livello}]: 2, Comune
	\item[\textbf{Lancio}]: 2 Azioni
	\item[\textbf{Gittata}]: 18 metri
	\item[\textbf{Durata}]: 1 ora
\end{description}
\end{minipage}}\smallskip

Evochi una spessa massa di tela densa e appiccicosa in un punto a gittata, scelto da te. Per la durata, la ragnatela riempie una sfera di 3 metri di raggio dal punto designato. La ragnatela è terreno difficile e rende quell'area oscurata leggermente.

Se la tela non è ancorate tra due masse solide (come pareti o alberi) o stesa lungo un pavimento, parete o soffitto, la ragnatela evocata crolla su se stessa, e l'incantesimo termina all'inizio del tuo prossimo round. Le tele distese su di una superficie piatta hanno una profondità di 1 metro.

Ogni creatura che inizia il suo round nella ragnatela o che vi entra durante il proprio round deve effettuare un Tiro Salvezza su Riflessi. Se lo fallisce, la creatura è intralciata finché rimane nella ragnatela o finché non si libera.

Una creatura intralciata dalle ragnatele può usare 2 Azioni per effettuare un nuovo Tiro Salvezza. Se lo supera, non è più intralciata.

La ragnatela è infiammabile e se esposta alle fiamme prende fuoco immediatamente e brucia per 2 round causando a ogni creatura dentro la sua area 2d4 di danno da fuoco.

\incantesimo{Randello Incantato}
\noindent\colorbox{OBSSgold!10}{
\begin{minipage}{0.95\linewidth}
\begin{description}[noitemsep, topsep=0pt, parsep=0pt, partopsep=0pt, leftmargin=0cm, labelwidth=1.3cm]
	\item[\textbf{Lista}]: Animali e Piante
	\item[\textbf{Livello}]: 0, Comune
	\item[\textbf{Lancio}]: 1 Azione
	\item[\textbf{Gittata}]: Contatto
	\item[\textbf{Durata}]: 1 minuto
\end{description}
\end{minipage}}\smallskip

Il legno di un randello o bastone da combattimento che stai impugnando viene infuso del potere della natura. Per la durata dell'incantesimo, usando quell'arma puoi usare la tua caratteristica da incantatore al posto della Forza per il danno e Tiri per Colpire, il dado di danno dell'arma diventa un d8. L'arma diventa anche magica, se già non lo è. L'incantesimo ha termine se lo lanci di nuovo o se lasci l'arma.

\textbf{Per ogni Successo Critico Magico} ottenuto nella Prova di Magia la durata raddoppia oppure aggiungi un +1 al danno.

\incantesimo{Reggia Meravigliosa}
\noindent\colorbox{OBSSgold!10}{
\begin{minipage}{0.95\linewidth}
\begin{description}[noitemsep, topsep=0pt, parsep=0pt, partopsep=0pt, leftmargin=0cm, labelwidth=1.3cm]
	\item[\textbf{Lista}]: Evocazione
	\item[\textbf{Livello}]: 7, Leggendario
	\item[\textbf{Lancio}]: 1 minuto
	\item[\textbf{Gittata}]: 90 metri
	\item[\textbf{Durata}]: 24 ore
\end{description}
\end{minipage}}\smallskip

Entro la gittata, evochi un'abitazione extradimensionale che rimane per la durata dell'incantesimo. Scegli dove è posizionato il suo portone d'ingresso. Il portone d'ingresso emette una lieve luminosità ed è largo 1 metro per 3 metri di altezza. Tu e tutte le creature da te indicate quando hai lanciato l'incantesimo potete entrare nell'abitazione extradimensionale, fino a quando il portone resta aperto. Puoi aprire o chiudere il portone se ti trovi entro 9 metri da esso. Mentre è chiuso, il portone è invisibile.

Oltre il portone si trova un magnifico ingresso, oltre il quale si dipanano numerose stanze. L'atmosfera è pulita, fresca e accogliente. Puoi creare quanti piani desideri, ma lo spazio non può eccedere 50 cubi ognuno di 3 metri di spigolo. Il luogo è ammobiliato e decorato come preferisci. Contiene cibo sufficiente a soddisfare un banchetto di 9 portate per 100 persone. Uno staff di 100 servitori quasi trasparenti è al servizio di chiunque vi faccia ingresso. Sta a te decidere l'aspetto visivo di questi servitori e il loro abbigliamento. Essi obbediscono assolutamente ai tuoi ordini. Ogni servitore può svolgere qualsiasi compito un normale servitore umano possa svolgere, ma non possono attaccare o effettuare alcuna azione che potrebbe arrecare direttamente danno a un'altra creatura. I servitori possono quindi raccogliere oggetti, pulire, riparare, ripiegare vestiti, accendere fuochi, servire cibi, versare vini e così via. I servitori possono recarsi in qualsiasi punto della dimora, ma non possono uscirne. I mobili e gli altri oggetti creati da questo incantesimo diventano fumo quando vengono portati fuori dalla dimora. Quando l'incantesimo termina, qualsiasi creatura all'interno dello spazio extradimensionale viene espulsa nello spazio aperto più vicino all'uscita.

\textbf{Per ogni Successo Critico Magico} ottenuto nella Prova di Magia la durata raddoppia o togli un mese dal conteggio per renderlo permanente.

\textbf{NOTA}: l'incantesimo lanciato per un anno tutti i giorni sempre nello stesso luogo diventa permanente.

\textbf{Per ogni Successo Critico Magico} ottenuto nella Prova di Magia la durata raddoppia o togli un mese dal conteggio per renderlo permanente.

\incantesimo{Regressione Mentale}
\noindent\colorbox{OBSSgold!10}{
\begin{minipage}{0.95\linewidth}
\begin{description}[noitemsep, topsep=0pt, parsep=0pt, partopsep=0pt, leftmargin=0cm, labelwidth=1.3cm]
	\item[\textbf{Lista}]: Ammaliamento
	\item[\textbf{Livello}]: 8, Raro
	\item[\textbf{Lancio}]: 2 Azioni
	\item[\textbf{Gittata}]: 45 metri
	\item[\textbf{Durata}]: Istantanea
\end{description}
\end{minipage}}\smallskip

Assalti la mente di una creatura a gittata e che puoi vedere, cercando di frammentarne l'intelletto e la personalità. Il bersaglio subisce 4d6 danni e deve effettuare un Tiro Salvezza su Volontà. Se fallisce il Tiro Salvezza, i punteggi di Intelligenza e Carisma della creatura scendono a -4. La creatura non può lanciare incantesimi, attivare oggetti magici, comprendere linguaggi, o comunicare in alcun modo comprensibile. La creatura può, tuttavia, identificare i suoi amici, seguirli e anche proteggerli. Dopo 30 giorni, la creatura può ripetere il Tiro Salvezza contro l'incantesimo. Se lo supera, l'incantesimo ha termine se fallisce l'effetto è permanete.

L'incantesimo può essere terminato entro i 30 giorni da ristorare superiore, guarigione o desiderio.

\incantesimo{Reincarnazione}
\noindent\colorbox{OBSSgold!10}{
\begin{minipage}{0.95\linewidth}
\begin{description}[noitemsep, topsep=0pt, parsep=0pt, partopsep=0pt, leftmargin=0cm, labelwidth=1.3cm]
	\item[\textbf{Lista}]: Animali e Piante
	\item[\textbf{Livello}]: 5, Raro
	\item[\textbf{Lancio}]: 1 ora
	\item[\textbf{Gittata}]: Contatto
	\item[\textbf{Durata}]: Istantanea
\end{description}
\end{minipage}}\smallskip

Entri a contatto con un umanoide morto o un frammento di umanoide morto. Purché la creatura non sia morta da più di 10 giorni, l'incantesimo gli forma un nuovo corpo adulto e poi ne richiama l'anima affinché entri nel corpo. Se l'anima del bersaglio non è libera o consenziente a farlo, l'incantesimo fallisce.

La magia modella un nuovo corpo, che probabilmente provocherà un cambio di razza alla creatura. Il Narratore tira un d10 e consulta la seguente tabella per determinare quale forma assuma la creatura una volta riportata in vita, oppure sarà Il Narratore a scegliere la forma.

\medskip

\noindent\begin{tabular}{ll}
	\toprule
 \rowcolor{gray!20}\textbf{d10} &\textbf{Razza}\\
	\toprule
	0 & Lupo/Aquila/Volpe/Lince (tirate 1d4)\\
 \rowcolor{gray!20}1&Nano\\
	2&Elfo\\
 \rowcolor{gray!20}3&Mezzelfo\\
	4&Mezzorco\\
 \rowcolor{gray!20}5&Cinghiale/Tasso/Cane/Ratto (tirate 1d4)\\
	6&Nibali\\
 \rowcolor{gray!20}7&Orso/Gufo/Procione/Gatto (tirate 1d4)\\
	8&Umano\\
 \rowcolor{gray!20}9&Stessa razza precedente
\end{tabular}

\medskip

La creatura reincarnata ricorda la sua vita e le sue esperienze passate (stesso punteggio di CA e CM, Abilità e Competenze). Mantiene le capacità che aveva nella sua forma originale se è in grado di applicarle.

\textbf{Questo incantesimo non non è disponibile se non ai Devoti e Seguaci di Shayalia o Efrem}

\emph{NOTA}: un Devoto o Seguace di Shayalia od Efrem reincarnerà la creatura sempre in un animale, però potendo scegliere il tipo.

Non è possibile reincarnarsi in uno gnomo se non si era prima uno gnomo.

\incantesimo{Resistenza}
\noindent\colorbox{OBSSgold!10}{
\begin{minipage}{0.95\linewidth}
\begin{description}[noitemsep, topsep=0pt, parsep=0pt, partopsep=0pt, leftmargin=0cm, labelwidth=1.3cm]
	\item[\textbf{Lista}]: Abiurazione
	\item[\textbf{Livello}]: 0, Comune
	\item[\textbf{Lancio}]: 1 Reazione
	\item[\textbf{Gittata}]: Contatto
	\item[\textbf{Durata}]: Istantanea
\end{description}
\end{minipage}}\smallskip

Lanci l'incantesimo a contatto con una creatura consenziente. Una volta prima del termine dell'incantesimo, il bersaglio può tirare un 1d4 e sommare il risultato ottenuto a un Tiro Salvezza a sua scelta. Può tirare il dado prima o dopo aver effettuato il Tiro Salvezza. Poi l'incantesimo termina. Non si può ricevere l'incantesimo Resistenza ad intervalli inferiori ad 1 ora.

\textbf{Per ogni Successo Critico Magico} ottenuto nella Prova di Magia puoi fare usufruire del bonus un altra creatura.

\incantesimo{Respirare Sott'Acqua}
\noindent\colorbox{OBSSgold!10}{
\begin{minipage}{0.95\linewidth}
\begin{description}[noitemsep, topsep=0pt, parsep=0pt, partopsep=0pt, leftmargin=0cm, labelwidth=1.3cm]
	\item[\textbf{Lista}]: Acqua, Aria
	\item[\textbf{Livello}]: 3, Comune
	\item[\textbf{Lancio}]: 2 Azioni
	\item[\textbf{Gittata}]: 9 metri
	\item[\textbf{Durata}]: 1 Ora per CM
\end{description}
\end{minipage}}\smallskip

Puoi dividere equamente il totale della durata dell'incantesimo tra CM/2 creature consenzienti.

Questo incantesimo consente  alle creature selezionate entro gittata di respirare sott'acqua per la durata concessa. Le creature soggette mantengono anche il loro normale metodo di respirazione.

\textbf{Per ogni Successo Critico Magico} ottenuto nella Prova di Magia hai un bonus di +2 per la verifica della tua CM.

\incantesimo{Rigenerazione}
\noindent\colorbox{OBSSgold!10}{
\begin{minipage}{0.95\linewidth}
\begin{description}[noitemsep, topsep=0pt, parsep=0pt, partopsep=0pt, leftmargin=0cm, labelwidth=1.3cm]
	\item[\textbf{Lista}]: Trasmutazione
	\item[\textbf{Livello}]: 7, Leggendario
	\item[\textbf{Lancio}]: 1 minuto
	\item[\textbf{Gittata}]: Contatto
	\item[\textbf{Durata}]: 1 ora
\end{description}
\end{minipage}}\smallskip

Lanci l'incantesimo a contatto di una creatura per stimolare la sua capacità di guarigione naturale. Il bersaglio recupera 4d8 + 15 Punti Ferita. Per la durata dell'incantesimo, il bersaglio recupera 10 Punti Ferita all'inizio di ciascun suo round. Le membra recise del corpo del bersaglio (dita, gambe, code e così via), se ne ha, vengono ripristinate in 2 minuti. Se hai la parte recisa e la tieni appoggiata al moncherino, l'incantesimo fa sì che l'arto si ricucia in 3 round col moncherino.

\textbf{Per ogni Successo Critico Magico} ottenuto nella Prova di Magia raddoppi i Punti Ferita recuperati per round.

\incantesimo{Rimuovi Malattia}\label{rimuovimalattie}\hypertarget{rimuovimalattie}{}
\noindent
\begin{description}[noitemsep, topsep=0pt, parsep=0pt, partopsep=0pt, leftmargin=0cm, labelwidth=1.3cm]
	\item[\textbf{Lista}]: Cura
	\item[\textbf{Livello}]: 2, Comune
	\item[\textbf{Lancio}]: 1 turno
	\item[\textbf{Gittata}]: Contatto
	\item[\textbf{Durata}]: Istantanea
\end{description}

Puoi porre fine a una malattia naturale. In caso di malattie magiche la tua DC di incantesimo deve essere superiore alla DC della malattia.

\textbf{Per ogni Successo Critico Magico} ottenuto nella Prova di Magia puoi guarire una persona in più oppure considerare un +4 per superare la DC della malattia.

\incantesimo{Rimuovi Maledizione}
\noindent\colorbox{OBSSgold!10}{
\begin{minipage}{0.95\linewidth}
\begin{description}[noitemsep, topsep=0pt, parsep=0pt, partopsep=0pt, leftmargin=0cm, labelwidth=1.3cm]
	\item[\textbf{Lista}]: Abiurazione
	\item[\textbf{Livello}]: 3, Comune
	\item[\textbf{Lancio}]: 2 Azioni
	\item[\textbf{Gittata}]: Contatto
	\item[\textbf{Durata}]: Istantanea
\end{description}
\end{minipage}}\smallskip

Se l'oggetto o persona è stato maledetto tramite l'incantesimo \hyperlink{Scagliare Maledizione}{Scagliare Maledizione}, o comunque il Narratore decide che l'oggetto ha una maledizione particolare allora la DC di chi lancia Rimuovi Maledizione deve essere superiore a quella della Maledizione.

\textbf{Per ogni Successo Critico Magico} ottenuto nella Prova di Magia puoi guarire una persona in più oppure considerare un +4 per superare la DC della maledizione.

Sia che sia stato sufficiente lanciare l'incantesimo oppure sia stato lanciato con una Prova di Magia, la maledizione resta, ma l'incantesimo permette di rimuovere l'oggetto e gettarlo.


\incantesimo{Rimuovi Paura}
\noindent\colorbox{OBSSgold!10}{
\begin{minipage}{0.95\linewidth}
\begin{description}[noitemsep, topsep=0pt, parsep=0pt, partopsep=0pt, leftmargin=0cm, labelwidth=1.3cm]
	\item[\textbf{Lista}]: Abiurazione
	\item[\textbf{Livello}]: 1, Comune
	\item[\textbf{Lancio}]: 1 Azione
	\item[\textbf{Gittata}]: Tocco
	\item[\textbf{Durata}]: 2 turni
\end{description}
\end{minipage}}\smallskip

Questo incantesimo infonde coraggio nel soggetto e può rimuovere gli effetti della paura indotta magicamente permettendogli di fare un nuovo Tiro Salvezza. Il soggetto toccato riceve un bonus al Tiro Salvezza di +1 per volte che l'incantatore ha preso Adepto della Magia.

\textbf{Per ogni Successo Critico Magico} ottenuto nella Prova di Magia il soggetto prende un +2 al Tiro Salvezza.


\incantesimo{Rimuovi Veleno}\label{incrimuoviveleno}\hypertarget{incrimuoviveleno}{}
\noindent
\begin{description}[noitemsep, topsep=0pt, parsep=0pt, partopsep=0pt, leftmargin=0cm, labelwidth=1.3cm]
	\item[\textbf{Lista}]: Acqua, Cura
	\item[\textbf{Livello}]: 3, Comune
	\item[\textbf{Lancio}]: 1 Round
	\item[\textbf{Gittata}]: Contatto
	\item[\textbf{Durata}]: Istantanea
\end{description}

Puoi porre fine ad un veleno naturale. In caso di veleni magici la tua DC di incantesimo deve essere superiore alla DC (o Tiro Salvezza) del veleno.

\textbf{Per ogni Successo Critico Magico} ottenuto nella Prova di Magia aggiungi +4 alla propria DC per capire se ha superato quella dello del veleno.

\incantesimo{Rinascita}
\noindent\colorbox{OBSSgold!10}{
\begin{minipage}{0.95\linewidth}
\begin{description}[noitemsep, topsep=0pt, parsep=0pt, partopsep=0pt, leftmargin=0cm, labelwidth=1.3cm]
	\item[\textbf{Lista}]: Cura, Necromanzia
	\item[\textbf{Livello}]: 3, Molto Raro
	\item[\textbf{Lancio}]: 10 Minuti
	\item[\textbf{Gittata}]: Contatto
	\item[\textbf{Durata}]: Istantanea
\end{description}
\end{minipage}}\smallskip

Una creatura morta nell'ultimo minuto e con cui sei in contatto, ritorna in vita con 1 punto ferita e nessun Punto Magia. Questo incantesimo non può riportare in vita le persone morte di vecchiaia, né può ripristinare le parti del corpo mancanti.

La creatura riportata in vita deve effettuare un Tiro Salvezza su Tempra a DC 15 oppure per il trauma subito non torna in vita, se torna in vita è Affaticato 3.

\textbf{NOTA}: a discrezione del Narratore questo potrebbe essere l'unico incantesimo concesso per riportare in vita una creatura, altrimenti vale la regola che solo un Patrono può riportare in vita.

\incantesimo{Riparare}
\noindent\colorbox{OBSSgold!10}{
\begin{minipage}{0.95\linewidth}
\begin{description}[noitemsep, topsep=0pt, parsep=0pt, partopsep=0pt, leftmargin=0cm, labelwidth=1.3cm]
	\item[\textbf{Lista}]: Terra
	\item[\textbf{Livello}]: 0, Comune
	\item[\textbf{Lancio}]: 1 minuto
	\item[\textbf{Gittata}]: Contatto
	\item[\textbf{Durata}]: Istantanea
\end{description}
\end{minipage}}\smallskip

Questo incantesimo ripara una singola rottura o spaccatura in un oggetto con cui sei a contatto, come una catenella spezzata, due metà di una chiave rotta, un mantello lacerato, o un otre che perde. Purché la rottura o la spaccatura non sia più grande di 30 centimetri in qualsiasi dimensione, sei in grado di ripararle, senza lasciare traccia dei danni subiti. Questo incantesimo può riparare fisicamente un oggetto magico o un costrutto, ma non è in grado di ripristinare le funzioni magiche di questi oggetti.

\incantesimo{Riposo Inviolato}
\noindent\colorbox{OBSSgold!10}{
\begin{minipage}{0.95\linewidth}
\begin{description}[noitemsep, topsep=0pt, parsep=0pt, partopsep=0pt, leftmargin=0cm, labelwidth=1.3cm]
	\item[\textbf{Lista}]: Necromanzia
	\item[\textbf{Livello}]: 2, Non Comune
	\item[\textbf{Lancio}]: 2 Azioni
	\item[\textbf{Gittata}]: Contatto
	\item[\textbf{Durata}]: 10 giorni
\end{description}
\end{minipage}}\smallskip

Entri a contatto con un cadavere o altri resti. Per la durata, il bersaglio è protetto dalla putrefazione e non può diventare non morto.

\textbf{Per ogni Successo Critico Magico} ottenuto nella Prova di Magia raddoppi la durata fino ad un massimo di un anno.

\incantesimo{Risata Incontenibile}
\noindent\colorbox{OBSSgold!10}{
\begin{minipage}{0.95\linewidth}
\begin{description}[noitemsep, topsep=0pt, parsep=0pt, partopsep=0pt, leftmargin=0cm, labelwidth=1.3cm]
	\item[\textbf{Lista}]: Ammaliamento
	\item[\textbf{Livello}]: 1, Non Comune
	\item[\textbf{Lancio}]: 2 Azioni
	\item[\textbf{Gittata}]: 9 metri
	\textbf{Durata}: 1 minuto
\end{description}
\end{minipage}}\smallskip

Una creatura a gittata di tua scelta e che puoi vedere percepisce tutto come tremendamente ilare e divertente, scoppiando in fragorose risate finché è soggetta a questo incantesimo. Il bersaglio deve superare un Tiro Salvezza su Volontà o cadere prono ed i round successivi perdere 1 Azione a round per ridere. Le creature con un punteggio di Intelligenza -2 o meno, ignorano l'effetto.

Al termine di ciascun suo round e ogni volta che subisce danni, il bersaglio può effettuare un altro Tiro Salvezza su Volontà. Il bersaglio ha +1d6 al Tiro Salvezza se ha subito danni nel round. Se lo supera, l'incantesimo termina.

\incantesimo{Riscaldare il Metallo}
\noindent\colorbox{OBSSgold!10}{
\begin{minipage}{0.95\linewidth}
\begin{description}[noitemsep, topsep=0pt, parsep=0pt, partopsep=0pt, leftmargin=0cm, labelwidth=1.3cm]
	\item[\textbf{Lista}]: Fuoco
	\item[\textbf{Livello}]: 2, Non Comune
	\item[\textbf{Lancio}]: 2 Azioni
	\item[\textbf{Gittata}]: 18 metri
	\item[\textbf{Durata}]: 1 minuto, Concentrazione
\end{description}
\end{minipage}}\smallskip

Scegli un manufatto di metallo, come un'arma di metallo o un'armatura di metallo media o pesante, a gittata e che puoi vedere. Fai sì che l'oggetto risplenda di rosso per il calore. Qualsiasi creatura in contatto fisico con l'oggetto subisce 1d8 danni da fuoco quando lanci questo incantesimo. Finché mantieni la Concentrazione infliggi nuovamente questo danno nel round.

Se una creatura sta impugnando o indossando l'oggetto e subisce danno da esso la creatura deve superare un Tiro Salvezza su Tempra o gettare l'oggetto se ne è in grado. Se non getta l'oggetto, ha -2 ai Tiri per Colpire e le prove su competenze di base fino all'inizio del suo prossimo round. Se l'oggetto è oltre i 18 metri dall'incantatore l'incantesimo non termina ma smette di essere rovente.

\textbf{Per ogni Successo Critico Magico} ottenuto nella Prova di Magia il danno aumenta di 1d6.

\incantesimo{Ristorare Inferiore}
\noindent\colorbox{OBSSgold!10}{
\begin{minipage}{0.95\linewidth}
\begin{description}[noitemsep, topsep=0pt, parsep=0pt, partopsep=0pt, leftmargin=0cm, labelwidth=1.3cm]
	\item[\textbf{Lista}]: Cura
	\item[\textbf{Livello}]: 2, Comune
	\item[\textbf{Lancio}]: 2 Azioni
	\item[\textbf{Gittata}]: Contatto
	\item[\textbf{Durata}]: Istantanea
\end{description}
\end{minipage}}\smallskip

Puoi porre fine a una condizione che affligge una creatura con cui sei a contatto. La condizione può essere \textbf{accecato}, \textbf{assordato} o \textbf{paralizzato}. Può ridurre di un grado il livello di \textbf{Affaticamento}. Si recupera 2d6 Punti Ferita Massimi persi, ma non aumentano i Punti Ferita attuali. Puoi recuperare 1 punto di Caratteristica perso non permanentemente.

Non è possibile usufruire di più di un Ristorare Inferiore al giorno.

In caso di condizioni magiche esegui una prova di \hyperlink{contrastareincantesimi}{contrastare} (pag. \pageref{contrastareincantesimi}) con la DC della condizione.

\incantesimo{Ristorare Superiore}
\noindent\colorbox{OBSSgold!10}{
\begin{minipage}{0.95\linewidth}
\begin{description}[noitemsep, topsep=0pt, parsep=0pt, partopsep=0pt, leftmargin=0cm, labelwidth=1.3cm]
	\item[\textbf{Lista}]: Cura
	\item[\textbf{Livello}]: 5, Non Comune
	\item[\textbf{Lancio}]: 2 Azioni
	\item[\textbf{Gittata}]: Contatto
	\item[\textbf{Durata}]: Istantanea
\end{description}
\end{minipage}}\smallskip

Imbevi una creatura a contatto di energia positiva curativa per annullare un effetto debilitante, non è possibile usufruire di più di un Ristorare Superiore al giorno:

\begin{itemize}[leftmargin=*] \setlength{\itemsep}{0pt}

	\item Un effetto che ha Affascinato o Dominato il bersaglio.
	\item Fai recuperare 2 punti ad una statistica al bersaglio. Recuperi 1 punto se la perdita era permanente.
	\item I Punti Ferita massimi tornano al valore normale, ma non aumentano i Punti Ferita attuali.
	\item Sei in grado di alleviare di due gradi le condizioni di Affaticamento.
\end{itemize}

Non è possibile usufruire di più di un Ristorare Superiore al giorno.

In caso di condizioni magiche esegui una prova di \hyperlink{contrastareincantesimi}{contrastare} (pag. \pageref{contrastareincantesimi}) con la DC della condizione.

\incantesimo{Risveglio}
\noindent\colorbox{OBSSgold!10}{
\begin{minipage}{0.95\linewidth}
\begin{description}[noitemsep, topsep=0pt, parsep=0pt, partopsep=0pt, leftmargin=0cm, labelwidth=1.3cm]
	\item[\textbf{Lista}]: Animali e Piante
	\item[\textbf{Livello}]: 5, Raro
	\item[\textbf{Lancio}]: 8 ore
	\item[\textbf{Gittata}]: Contatto
	\item[\textbf{Durata}]: Istantanea
\end{description}
\end{minipage}}\smallskip

Dopo aver trascorso il tempo di lancio a disegnare tracciati magici con una gemma preziosa, entri a contatto con una bestia o vegetale Enorme o di taglia inferiore. Il bersaglio deve essere privo di punteggio di Intelligenza o avere Intelligenza -3 o meno. Il bersaglio ottiene Intelligenza 0. Il bersaglio ottiene anche la capacità di parlare un linguaggio che conosci. Se il bersaglio è un vegetale, ottiene la capacità di muovere i suoi arti, radici, liane, rampicanti e così via, e ottiene sensi simili a quelli di un umano. Il Narratore sceglierà le statistiche appropriate al tipo di vegetale risvegliato, come le statistiche per il cespuglio risvegliato o l'albero risvegliato.

La bestia o vegetale risvegliato è Affascinato da te per 30 giorni o finché tu o i tuoi compagni non gli arrecherete danno. Quando la condizione Affascinato termina, la creatura risvegliata sceglie se rimanerti amichevole, in base a come l'hai trattata mentre era affascinata.

\textbf{Per ogni Successo Critico Magico} ottenuto nella Prova di Magia aumenti la durata della fascinazione di 30 giorni, fino ad un massimo di 1 anno.

\incantesimo{Ritirata Rapida}
\noindent\colorbox{OBSSgold!10}{
\begin{minipage}{0.95\linewidth}
\begin{description}[noitemsep, topsep=0pt, parsep=0pt, partopsep=0pt, leftmargin=0cm, labelwidth=1.3cm]
	\item[\textbf{Lista}]: Trasmutazione
	\item[\textbf{Livello}]: 1, Non Comune
	\item[\textbf{Lancio}]: 1 Azione
	\item[\textbf{Gittata}]: Personale
	\item[\textbf{Durata}]: Concentrazione, 1 minuto
\end{description}
\end{minipage}}\smallskip

Questo incantesimo ti permette di muoverti a un'andatura incredibile. Quando lanci questo incantesimo il tuo movimento aumenta di 2 metri per Azione di Movimento.

\textbf{Per ogni Successo Critico Magico} ottenuto nella Prova di Magia la durata aumenta di 1 round.

\incantesimo{Saltare}
\noindent\colorbox{OBSSgold!10}{
\begin{minipage}{0.95\linewidth}
\begin{description}[noitemsep, topsep=0pt, parsep=0pt, partopsep=0pt, leftmargin=0cm, labelwidth=1.3cm]
	\item[\textbf{Lista}]: Aria
	\item[\textbf{Livello}]: 1, Comune
	\item[\textbf{Lancio}]: 2 Azioni
	\item[\textbf{Gittata}]: Contatto
	\item[\textbf{Durata}]: 1 minuto
\end{description}
\end{minipage}}\smallskip

La distanza di salto della creatura con cui sei in contatto al momento del lancio è triplicata, rispetto al risultato ottenuto e senza limite di lunghezza/altezza, fino al termine dell'incantesimo.

\incantesimo{Santificare}
\noindent\colorbox{OBSSgold!10}{
\begin{minipage}{0.95\linewidth}
\begin{description}[noitemsep, topsep=0pt, parsep=0pt, partopsep=0pt, leftmargin=0cm, labelwidth=1.3cm]
	\item[\textbf{Lista}]: Invocazione
	\item[\textbf{Livello}]: 5, Raro
	\item[\textbf{Lancio}]: 24 ore
	\item[\textbf{Gittata}]: Contatto
	\item[\textbf{Durata}]: Fino a che dissolto
\end{description}
\end{minipage}}\smallskip

Infondi l'area circostante a un punto con cui sei in contatto del potere del tuo Patrono. L'area può avere un raggio massimo di 18 metri, e l'incantesimo fallisce se include un'area già sotto l'effetto di un incantesimo santificare. L'area soggetta all'incantesimo genera i seguenti effetti.

\emph{Per prima cosa}, celestiali, elementali, fatati, demoni e non morti non possono entrare nell'area, né una simile creatura può affascinare, spaventare o possederne altre al suo interno. Qualsiasi creatura affascinata, spaventata o posseduta da una creatura simile non è più affascinata,spaventata o posseduta dal momento in cui entra in quest'area. Puoi escludere uno o più tipi di queste creature da questo effetto.

\emph{Seconda cosa}, puoi vincolare un effetto ulteriore all'area. Scegli l'effetto dalla lista seguente, o scegline uno presentatoti dal Narratore. Alcuni di questi effetti si applicano alle creature nell'area; puoi decidere se gli effetti si applichino a tutte le creature, le creature Devote o Seguaci di specifica Patrono, o le creature di un tipo specifico, come orchi o troll. Quando una creatura soggetta all'incantesimo entra in quest'area per la prima volta durante un round o inizia il suo round qui, deve effettuare un Tiro Salvezza su Volontà. Se lo supera, la creatura ignora l'effetto aggiuntivo finché non lascia l'area.

\begin{itemize}[leftmargin=*] \setlength{\itemsep}{0pt}
	\item \emph{Coraggio}. Le creature soggette non possono essere spaventate mentre restano in quest'area. Interferenza Extradimensionale. Le creature soggette non possono muoversi o viaggiare usando il teletrasporto o altri mezzi extradimensionali o interplanari.
	\item \emph{Lingue}. Le creature soggette possono comunicare con qualsiasi altra creatura nell'area, anche se non condividono un linguaggio comune.
	\item \emph{Luce Diurna}. Luce intensa riempie l'area. L'oscurità magica creata da incantesimi di più basso livello di quella usata per lanciare questo incantesimo non possono estinguere la luce. La durata in questo caso è di una settimana.
	\item \emph{Oscurità}. L'oscurità riempie l'area. La luce normale, e anche la luce magica creata da incantesimi di più basso livello di quello usato per lanciare questo incantesimo, non possono illuminare l'area.
	\item \emph{Paura}. Le creature soggette sono spaventate mentre restano in quest'area.
	\item \emph{Protezione dall'Energia}. Le creature soggette ricevono resistenza a un tipo di danno a tua scelta (a eccezione dei danni contundenti, perforanti o taglienti), finché restano nell'area.
	\item \emph{Riposo Inviolato}. I corpi morti seppelliti nell'area non possono essere trasformati in non morti.
	\item \emph{Silenzio}. Nessun suono può emanare dall'interno dell'area, e nessun suono può entrarvi.
	\item \emph{Vulnerabilità all'Energia}. Le creature soggette ricevono vulnerabilità a un tipo di danno a tua scelta (a eccezione dei danni contundenti, perforanti o taglienti), finché restano nell'area.
\end{itemize}

\incantesimo{Santuario}
\noindent\colorbox{OBSSgold!10}{
\begin{minipage}{0.95\linewidth}
\begin{description}[noitemsep, topsep=0pt, parsep=0pt, partopsep=0pt, leftmargin=0cm, labelwidth=1.3cm]
	\item[\textbf{Lista}]: Abiurazione
	\item[\textbf{Livello}]: 1, Comune
	\item[\textbf{Lancio}]: 2 Azioni
	\item[\textbf{Gittata}]: 9 metri
	\item[\textbf{Durata}]: 1 minuto
\end{description}
\end{minipage}}\smallskip

Proteggi una creatura a gittata dagli attacchi. Fino al termine dell'incantesimo, qualsiasi creatura che prenda come bersaglio la creatura protetta con un attacco o incantesimo dannoso deve prima effettuare un Tiro Salvezza su Volontà. Se fallisce il Tiro Salvezza, l'attaccante deve scegliere un nuovo bersaglio o perdere l'attacco o l'incantesimo. Questo incantesimo non protegge la creatura protetta dagli effetti ad area, come lo scoppio di una palla di fuoco. Se la creatura protetta effettua un attacco o lancia un incantesimo che agisce su creature nemiche, l'incantesimo termina.

\incantesimo{Santuario Privato}
\noindent\colorbox{OBSSgold!10}{
\begin{minipage}{0.95\linewidth}
\begin{description}[noitemsep, topsep=0pt, parsep=0pt, partopsep=0pt, leftmargin=0cm, labelwidth=1.3cm]
	\item[\textbf{Lista}]: Abiurazione
	\item[\textbf{Livello}]: 4, Molto Raro
	\item[\textbf{Lancio}]: 10 minuti
	\item[\textbf{Gittata}]: 36 metri
	\item[\textbf{Durata}]: 24 ore
\end{description}
\end{minipage}}\smallskip

Proteggi con la magia un'area. L'area è una sfera che può essere piccola fino a 1 metro di raggio o grande fino a 15 metri di raggio. L'incantesimo agisce fino al termine della durata o finché non usi un'Azione per interromperlo.

Quando lanci l'incantesimo, decidi che tipo di protezione questo fornisce, scegliendo una o più delle seguenti proprietà:

\noindent- Il suono non può attraversare il perimetro dell'area protetta.

\begin{itemize}[leftmargin=*] \setlength{\itemsep}{0pt}
	\item Il perimetro dell'area protetta appare buio e nebbioso, impedendo di vedervi attraverso (anche alla scurovisione)
	\item Sensori creati da incantesimi di divinazione non possono apparire all'interno dell'area protetta o attraversare la sua barriera perimetrale
	\item Le creature nell'area non possono essere bersaglio di incantesimi di divinazione
	\item Nulla può teletrasportarsi dentro o fuori dell'area protetta.
	\item All'interno dell'area protetta, il viaggio planare è interdetto.
\end{itemize}

Lanciare questo incantesimo sullo stesso punto ogni giorno per un anno, rende l'effetto permanente.

\textbf{Per ogni Successo Critico Magico} ottenuto nella Prova di Magia puoi aumentare le dimensioni della sfera di 3 metri di raggio oppure aumentare la durata di 12 ore.

\incantesimo{Scagliare Maledizione}
\noindent\colorbox{OBSSgold!10}{
\begin{minipage}{0.95\linewidth}
\begin{description}[noitemsep, topsep=0pt, parsep=0pt, partopsep=0pt, leftmargin=0cm, labelwidth=1.3cm]
	\item[\textbf{Lista}]: Necromanzia
	\item[\textbf{Livello}]: 3, Non Comune
	\item[\textbf{Lancio}]: 2 Azioni
	\item[\textbf{Gittata}]: Contatto
	\item[\textbf{Durata}]: 1 minuto
\end{description}
\end{minipage}}\smallskip

Una creatura con cui sei a contatto deve superare un Tiro Salvezza su Volontà o restare maledetta per la durata dell'incantesimo. Quando lanci questo incantesimo, scegli la natura della maledizione tra le seguenti opzioni:

\begin{itemize}[leftmargin=*] \setlength{\itemsep}{0pt}
	\item Scegli un punteggio di caratteristica. Mentre è maledetto, il bersaglio ha -1d6 alle prova di competenza base basate su quella caratteristica ed i Tiri Salvezza basati su quella caratteristica.
	\item Mentre è maledetto, il bersaglio ha -1d6 ai Tiri per Colpire e -3 al danno in mischia, contro di te.
	\item Mentre è maledetto, il bersaglio deve effettuare un Tiro Salvezza su Volontà all'inizio di ciascun suo round. Se lo fallisce, spreca 1 Azione di quel suo round senza fare nulla.
	\item Mentre il bersaglio è maledetto, ogni tuo attacco ed incantesimo infliggono 1d8 danni da Vuoto aggiuntivi contro di lui.
\end{itemize}

L'incantesimo rimuovi maledizione (vedi descrizione) termina questo effetto. A discrezione del Narratore, puoi scegliere una maledizione dall'effetto diverso, ma non dovrebbe essere comunque più potente di quelle descritte qui sopra. Il Narratore detiene il giudizio finale sull'effetto di una maledizione.

\textbf{Per ogni Successo Critico Magico} ottenuto nella Prova di Magia raddoppi la durata. Se ottieni 3 successi critici la durata è permanente.

\incantesimo{Scagliare Maledizione Minore}
\noindent\colorbox{OBSSgold!10}{
\begin{minipage}{0.95\linewidth}
\begin{description}[noitemsep, topsep=0pt, parsep=0pt, partopsep=0pt, leftmargin=0cm, labelwidth=1.3cm]
	\item[\textbf{Lista}]: Universale
	\item[\textbf{Livello}]: 1, Comune
	\item[\textbf{Lancio}]: 2 Azioni
	\item[\textbf{Gittata}]: Contatto
	\item[\textbf{Durata}]: 1 minuto
\end{description}
\end{minipage}}\smallskip

Una creatura con cui sei a contatto deve superare un Tiro Salvezza su Volontà o restare maledetta per la durata dell'incantesimo. Quando lanci questo incantesimo, scegli la natura della maledizione tra le seguenti opzioni:

\begin{itemize}[leftmargin=*] \setlength{\itemsep}{0pt}
	\item Scegli un punteggio di caratteristica. Mentre è maledetto, il bersaglio ha -1 alle prove di competenza di Base e i Tiri Salvezza basati su quel punteggio di caratteristica.
	\item Mentre è maledetto, il bersaglio ha -2 ai Tiri per Colpire contro di te.
	\item Mentre è maledetto, il bersaglio deve effettuare un Tiro Salvezza su Volontà all'inizio di ciascun suo round. Se lo fallisce, spreca 1 Azione di quel suo round senza fare nulla.
\end{itemize}

L'incantesimo \hyperlink{Rimuovi Maledizione}{Rimuovi Maledizione} termina questo effetto. A discrezione del Narratore, puoi scegliere una maledizione dall'effetto diverso, ma non dovrebbe essere comunque più potente di quelle descritte qui sopra. Il Narratore detiene il giudizio finale sull'effetto di una maledizione.

\textbf{Per ogni Successo Critico Magico} ottenuto nella Prova di Magia scegli un altra creatura entro 6 metri dalla prima.

\incantesimo{Scassinare}\label{knock}
\noindent\colorbox{OBSSgold!10}{
\begin{minipage}{0.95\linewidth}
\begin{description}[noitemsep, topsep=0pt, parsep=0pt, partopsep=0pt, leftmargin=0cm, labelwidth=1.3cm]
	\item[\textbf{Lista}]: Trasmutazione
	\item[\textbf{Livello}]: 2, Comune
	\item[\textbf{Lancio}]: 2 Azioni
	\item[\textbf{Gittata}]: 18 metri
	\item[\textbf{Durata}]: Istantanea
\end{description}
\end{minipage}}\smallskip

Scegli un oggetto a gittata e che puoi vedere. L'oggetto può essere una porta, scatola, delle manette, una serratura o un altro oggetto che possieda un metodo comune o magico per prevenirne l'accesso.

Un bersaglio che è chiuso da una serratura comune o che è bloccato o sbarrato viene aperto, sbloccato o liberato. Se l'oggetto ha più serrature, solo una di queste viene aperta.

Se scegli un bersaglio che è tenuto chiuso con Serratura Magica (o simile) esegui una prova di \hyperlink{contrastareincantesimi}{contrastare incantesimi} tra le due DC. Se Scassinare è più efficace di Serratura Magia allora l'incantesimo di chiusura viene annullato, altrimenti Scassinare non ha avuto effetto.

Quando lanci questo incantesimo, un sonoro bussare, udibile fino a 90 metri di distanza, emana dall'oggetto bersaglio.

\textbf{Per ogni Successo Critico Magico} ottenuto nella Prova di Magia puoi aprire un altro lucchetto/serratura entro gittata o aumentare di 4 la DC dell'incantesimo.

\incantesimo{Schiaffo di Cattalm}
\noindent\colorbox{OBSSgold!10}{
\begin{minipage}{0.95\linewidth}
\begin{description}[noitemsep, topsep=0pt, parsep=0pt, partopsep=0pt, leftmargin=0cm, labelwidth=1.3cm]
	\item[\textbf{Lista}]: Evocazione
	\item[\textbf{Livello}]: 1, Non Comune
	\item[\textbf{Lancio}]: 1 Reazione, che puoi effettuare in risposta al danno arrecatoti da una creatura entro 18 metri da te che puoi vedere
	\item[\textbf{Gittata}]: 18 metri
	\item[\textbf{Durata}]: Istantanea
\end{description}
\end{minipage}}\smallskip

Punti il dito, e la creatura che ti ha danneggiato viene schiaffeggiata da fuoco divina. La creatura deve effettuare un Tiro Salvezza su Riflessi e se fallisce subisce 2d10 danni da fuoco o la metà di questi danni se lo supera.

\textbf{Per ogni Successo Critico Magico} ottenuto nella Prova di Magia i danno aumenta di 1d10.



\incantesimo{Scolpire Pietra}
\noindent\colorbox{OBSSgold!10}{
\begin{minipage}{0.95\linewidth}
\begin{description}[noitemsep, topsep=0pt, parsep=0pt, partopsep=0pt, leftmargin=0cm, labelwidth=1.3cm]
	\item[\textbf{Lista}]: Terra
	\item[\textbf{Livello}]: 4, Comune
	\item[\textbf{Lancio}]: 2 Azioni
	\item[\textbf{Gittata}]: Contatto
	\item[\textbf{Durata}]: Istantanea
\end{description}
\end{minipage}}\smallskip

Scolpisci in qualsiasi forma che si presti ai tuoi scopi un cubo di 3 metri di lato di pietra con cui sei a contatto.

Così, per esempio, potresti scolpire una grossa pietra in un'arma, idolo o feretro, o creare un passaggio attraverso il muro. Potresti anche modellare una porta di pietra o la sua cornice per sigillare la porta. L'oggetto che crei può avere fino a due cardini e un chiavistello, ma è impossibile creare meccanismi più complessi.

\incantesimo{Scopri il Percorso}
\noindent\colorbox{OBSSgold!10}{
\begin{minipage}{0.95\linewidth}
\begin{description}[noitemsep, topsep=0pt, parsep=0pt, partopsep=0pt, leftmargin=0cm, labelwidth=1.3cm]
	\item[\textbf{Lista}]: Divinazione
	\item[\textbf{Livello}]: 6, Non Comune
	\item[\textbf{Lancio}]: 1 minuto
	\item[\textbf{Gittata}]: Personale
	\item[\textbf{Durata}]: 1 giorno
\end{description}
\end{minipage}}\smallskip

Questo incantesimo ti permette di trovare la rotta fisica più breve e diretta verso uno specifico luogo fisso con cui hai familiarità ed è sullo stesso piano di esistenza. Se indichi una destinazione su di un altro piano di esistenza, una destinazione che si muove (come una fortezza mobile) o una destinazione non specifica (come \emph{la tana di un drago verde}), l'incantesimo fallisce.

Per la durata dell'incantesimo, finché sei nello stesso piano di esistenza della destinazione, saprai quanto è distante e in che direzione si trovi. Mentre sei in viaggio verso di essa, ogni volta che ti si presenterà la possibilità di scegliere tra percorsi diversi, determinerai automaticamente qual è la via più breve e la rotta più diretta (ma non necessariamente la più sicura) per raggiungere la destinazione.

\textbf{Per ogni critico} ottenuto nella Prova di Magia l'incantesimo dura 8 ore in più.

\incantesimo{Scopri Piante}
\noindent\colorbox{OBSSgold!10}{
\begin{minipage}{0.95\linewidth}
\begin{description}[noitemsep, topsep=0pt, parsep=0pt, partopsep=0pt, leftmargin=0cm, labelwidth=1.3cm]
	\item[\textbf{Lista}]: Divinazione
	\item[\textbf{Livello}]: 2, Non Comune
	\item[\textbf{Lancio}]: 2 Azioni
	\item[\textbf{Gittata}]: Personale
	\item[\textbf{Durata}]: 1 turno per CM
\end{description}
\end{minipage}}\smallskip

L'incantatore che lancia questo incantesimo è in grado di trovare una specifica pianta entro un cerchio del diametro di 3 metri per CM centrato sull'incantatore. L'incantatore può concentrarsi su un diverso tipo di pianta ogni round e può muoversi, dal momento che l'area di effetto si sposta con lui.

\textbf{Nota}: per i Devoti di Shayalia l'incantesimo è Comune ed il cerchio ha un diametro di 10 metri per somma Tratti in comune con il Patrono.

\incantesimo{Scopri Trappole}
\noindent\colorbox{OBSSgold!10}{
\begin{minipage}{0.95\linewidth}
\begin{description}[noitemsep, topsep=0pt, parsep=0pt, partopsep=0pt, leftmargin=0cm, labelwidth=1.3cm]
	\item[\textbf{Lista}]: Divinazione
	\item[\textbf{Livello}]: 2, Comune
	\item[\textbf{Lancio}]: 2 Azioni
	\item[\textbf{Gittata}]: 36 metri
	\item[\textbf{Durata}]: 10 minuti di tempo reale
\end{description}
\end{minipage}}\smallskip

Per la durata dell'incantesimo avverti la presenza di qualsiasi trappola a gittata che sia nella tua linea di visuale. Una trappola, ai fini di questo incantesimo, comprende qualsiasi cosa che sia in grado di infliggere un effetto improvviso o inaspettato che tu possa considerare dannoso o indesiderabile e che è stato espressamente inteso come tale dal suo creatore. Di conseguenza, l'incantesimo percepirebbe un'area sotto l'incantesimo allarme, un glifo di interdizione, o una botola meccanica, ma non rivelerebbe una debolezza naturale del pavimento, un soffitto instabile o una buca nascosta.

La trappola viene evidenziata alla tua vista con un segno viola.

\incantesimo{Scrigno Segreto}
\noindent\colorbox{OBSSgold!10}{
\begin{minipage}{0.95\linewidth}
\begin{description}[noitemsep, topsep=0pt, parsep=0pt, partopsep=0pt, leftmargin=0cm, labelwidth=1.3cm]
	\item[\textbf{Lista}]: Evocazione
	\item[\textbf{Livello}]: 4, Raro
	\item[\textbf{Lancio}]: 2 Azioni
	\item[\textbf{Gittata}]: Contatto
	\item[\textbf{Durata}]: Istantanea
\end{description}
\end{minipage}}\smallskip

Nascondi un forziere e tutti i suoi contenuti sul Piano Etereo. Quando lanci questo incantesimo devi essere in contatto con il forziere e la replica in miniatura che serve da componente materiale. Il forziere può contenere fino a 0,25 metri cubi di materiale non vivente (1 x metro x 50 centimetri x 50 centimetri). Mentre il forziere rimane sul Piano Etereo, puoi usare un'Azione per entrare in contatto con la replica e richiamare il forziere. Esso riapparirà in uno spazio non occupato sul terreno entro 1 metro da te. Puoi rispedire il forziere nel Piano Etereo, usando un'Azione ed entrando in contatto sia col forziere che con la replica.

Dopo 60 giorni, c'è una percentuale cumulativa del 5\% al giorno che l'effetto dell'incantesimo abbia termine.

L'effetto termina se l'incantesimo viene lanciato nuovamente, se la replica del forziere viene distrutta, o se decidi di terminare l'incantesimo con un'Azione. Se l'incantesimo termina e il forziere si trova sul Piano Etereo, viene irrimediabilmente perduto.

\incantesimo{Scritto Illusorio}
\noindent\colorbox{OBSSgold!10}{
\begin{minipage}{0.95\linewidth}
\begin{description}[noitemsep, topsep=0pt, parsep=0pt, partopsep=0pt, leftmargin=0cm, labelwidth=1.3cm]
	\item[\textbf{Lista}]: Illusione
	\item[\textbf{Livello}]: 1, Comune
	\item[\textbf{Lancio}]: 1 minuto
	\item[\textbf{Gittata}]: Contatto
	\item[\textbf{Durata}]: 10 giorni
\end{description}
\end{minipage}}\smallskip

Scrivi su di una pergamena, un pezzo di carta o qualche altro materiale adatto a scrivere e lo infondi di una potente illusione che permane per la durata dell'incantesimo.

Per te e qualsiasi creatura da te indicata al lancio dell'incantesimo, la scritta appare normale, con la tua grafia, e trasmette qualsiasi significato volevi comunicare quando hai vergato il testo. Per tutti gli altri, la scritta appare come se fosse redatta in una scrittura ignota o magica, che risulta incomprensibile. In alternativa, puoi far sì che la scritta sembri un messaggio totalmente diverso, in una grafia e linguaggio differente, sebbene debba essere un linguaggio a te conosciuto.

In caso l'incantesimo venisse dissolto, sia la scritta originale che l'illusione svaniscono. Una creatura con visione del vero può leggere il messaggio nascosto.

\incantesimo{Scrutare}
\noindent\colorbox{OBSSgold!10}{
\begin{minipage}{0.95\linewidth}
\begin{description}[noitemsep, topsep=0pt, parsep=0pt, partopsep=0pt, leftmargin=0cm, labelwidth=1.3cm]
	\item[\textbf{Lista}]: Divinazione
	\item[\textbf{Livello}]: 5, Raro
	\item[\textbf{Lancio}]: 10 minuti
	\item[\textbf{Gittata}]: Personale
	\item[\textbf{Durata}]: Concentrazione, massimo 10 minuti
\end{description}
\end{minipage}}\smallskip

Puoi vedere e udire una particolare creatura a tua scelta che si trovi sul tuo stesso piano di esistenza. Il bersaglio deve effettuare un Tiro Salvezza su Volontà, modificato da quanto bene conosci il bersaglio e la tua connessione fisica a esso. Se il bersaglio sa che stai lanciando l'incantesimo, può fallire volontariamente il Tiro Salvezza, in caso desiderasse essere osservato da
te.

\medskip

\noindent\begin{tabularx}{\linewidth}{Xl}
	\toprule
 \rowcolor{gray!20}\textbf{Conoscenza} & \textbf{Mod. al TS}\\
	\toprule
	Ne hai sentito parlare &+4\\
 \rowcolor{gray!20}Hai incontrato il bersaglio &+0\\
	Conosci bene il bersaglio &-4
\end{tabularx}

\noindent\begin{tabularx}{\linewidth}{Xl}
	\toprule
 \rowcolor{gray!20}\textbf{Connessione} & \textbf{Mod. TS}\\
	\toprule
	Descrizione o immagine &-2\\
 \rowcolor{gray!20}Proprietà o indumento & -4\\
	Parte del corpo (capelli...)&-10
\end{tabularx}

\medskip

Se supera il Tiro Salvezza, il bersaglio ignora gli effetti dell'incantesimo, e non potrai usare di nuovo questo incantesimo contro di lui prima che siano passate 24 ore.

Se il Tiro Salvezza fallisce, l'incantesimo crea un sensore invisibile entro 3 metri dal bersaglio. Tramite il sensore puoi udire e vedere come se fossi sul posto. Il sensore si muove assieme al bersaglio, rimanendo entro 3 metri da lui per la durata dell'incantesimo. Una creatura che può vedere oggetti invisibili vede il sensore come una sfera luminosa delle dimensioni all'incirca di un pugno.

Invece di prendere come bersaglio una creatura, puoi scegliere come bersaglio dell'incantesimo un luogo che hai già visto in passato. Quando scegli questa opzione, il sensore compare in quel luogo ma non si muove.

\incantesimo{Scudo}
\noindent\colorbox{OBSSgold!10}{
\begin{minipage}{0.95\linewidth}
\begin{description}[noitemsep, topsep=0pt, parsep=0pt, partopsep=0pt, leftmargin=0cm, labelwidth=1.3cm]
	\item[\textbf{Lista}]: Universale
	\item[\textbf{Livello}]: 0, Comune
	\item[\textbf{Lancio}]: 1 Azione
	\item[\textbf{Gittata}]: Personale
	\item[\textbf{Durata}]: 1 round
\end{description}
\end{minipage}}\smallskip

Compare una barriera di forza magica invisibile a proteggerti. Fino all'inizio del tuo prossimo round hai un bonus di +1 alla Difesa e non subisci danno dal primo Dardo arcano e Dardo occulto che ti colpisce nel round.

Se lanciato spendendo 1 Punto Magia il bonus alla Difesa aumenta a +2, il tempo di lancio diventa una Reazione. Spendendo 2 Punti Magia oltre alle modifiche precedenti l'obiettivo diventa una qualsiasi creatura entro 6 metri.

\textbf{Per ogni Successo Critico Magico} ottenuto nella Prova di Magia aumenti la durata di 1 round.

\incantesimo{Scudo di Fuoco}
\noindent\colorbox{OBSSgold!10}{
\begin{minipage}{0.95\linewidth}
\begin{description}[noitemsep, topsep=0pt, parsep=0pt, partopsep=0pt, leftmargin=0cm, labelwidth=1.3cm]
	\item[\textbf{Lista}]: Fuoco, Acqua
	\item[\textbf{Livello}]: 4, Non Comune
	\item[\textbf{Lancio}]: 2 Azioni
	\item[\textbf{Gittata}]: Personale
	\item[\textbf{Durata}]: 10 minuti
\end{description}
\end{minipage}}\smallskip

Fiamme sottili e vaporose avvolgono il tuo corpo per la durata dell'incantesimo, emettendo luce intensa in un raggio di 3 metri e luce fioca per 6 metri. Puoi terminare l'incantesimo in anticipo, usando un'Azione per interromperlo.

Le fiamme ti forniscono uno scudo caldo o uno scudo freddo, a tua scelta. Lo scudo caldo ti conferisce resistenza al danno da freddo, mentre lo scudo freddo ti fornisce resistenza al danno da caldo.

Inoltre, ogni qualvolta una creatura entro 1 metro da te ti colpisce con un attacco in mischia, lo scudo erutta l'elemento. L'attaccante subisce 2d8 danni da fuoco da uno scudo caldo, o 2d8 danni da freddo da uno scudo freddo.

\incantesimo{Scurovisione}
\noindent\colorbox{OBSSgold!10}{
\begin{minipage}{0.95\linewidth}
\begin{description}[noitemsep, topsep=0pt, parsep=0pt, partopsep=0pt, leftmargin=0cm, labelwidth=1.3cm]
	\item[\textbf{Lista}]: Trasmutazione
	\item[\textbf{Livello}]: 2, Comune
	\item[\textbf{Lancio}]: 2 Azioni
	\item[\textbf{Gittata}]: Contatto
	\item[\textbf{Durata}]: 1 ora di tempo reale di gioco
\end{description}
\end{minipage}}\smallskip

Una creatura consenziente con cui sei in contatto ottiene la capacità di vedere al buio. Per la durata dell'incantesimo, quella creatura ha scurovisione fino a una gittata di 9 metri.

\textbf{Per ogni Successo Critico Magico} ottenuto nella Prova di Magia raddoppi la durata.

\incantesimo{Segugio Fedele}
\noindent\colorbox{OBSSgold!10}{
\begin{minipage}{0.95\linewidth}
\begin{description}[noitemsep, topsep=0pt, parsep=0pt, partopsep=0pt, leftmargin=0cm, labelwidth=1.3cm]
	\item[\textbf{Lista}]: Evocazione
	\item[\textbf{Livello}]: 4, Raro
	\item[\textbf{Lancio}]: 2 Azioni
	\item[\textbf{Gittata}]: 9 metri
	\item[\textbf{Durata}]: 8 ore
\end{description}
\end{minipage}}\smallskip

Puoi evocare un cane da guardia fantasma in uno spazio non occupato a gittata che puoi vedere dove rimarrà per la durata dell'incantesimo, finché non viene congedato con un'Azione, o finché non si allontanerà più di 30 metri da te.

Il segugio è invisibile a tutte le creature eccetto che a te e non può essere danneggiato. Quando una creatura di taglia Piccola o superiore si avvicina entro 9 metri da esso senza aver prima pronunciato la parola d'ordine da te specificata quando hai lanciato l'incantesimo, il segugio inizia ad abbaiare a grande volume, ma puoi sentirlo solo tu.

Il segugio vede le creature invisibili e può vedere nel Piano Etereo. Esso ignora le illusioni. All'inizio di ciascun tuo round, il segugio tenta di mordere una creatura entro 1 metro da esso e che ti sia ostile.

Il bonus di attacco del segugio è uguale al tuo modificatore di caratteristica per incantesimi + CM. Se colpisce, infligge 2d8 danni perforanti, ha CM*2 Punti Ferita, Difesa 10+modificatore da incantesimi, Tiri Salvezza pari al tuo modificatore da incantesimi.

\incantesimo{Sembrare}
\noindent\colorbox{OBSSgold!10}{
\begin{minipage}{0.95\linewidth}
\begin{description}[noitemsep, topsep=0pt, parsep=0pt, partopsep=0pt, leftmargin=0cm, labelwidth=1.3cm]
	\item[\textbf{Lista}]: Illusione
	\item[\textbf{Livello}]: 5, Non Comune
	\item[\textbf{Lancio}]: 2 Azioni
	\item[\textbf{Gittata}]: 9 metri
	\item[\textbf{Durata}]: 8 ore
\end{description}
\end{minipage}}\smallskip

Questo incantesimo ti permette di cambiare l'aspetto di un qualsiasi numero di creature a gittata e che puoi vedere. Fornisci a ciascun bersaglio un nuovo aspetto illusorio. Una creatura non consenziente può effettuare un Tiro Salvezza su Volontà e, se lo supera, ignora l'incantesimo.

L'incantesimo camuffa l'aspetto fisico oltre che gli abiti, le armature, le armi e l'equipaggiamento. Puoi far sembrare ciascuna creatura 30 centimetri più bassa o più alta, sembrare magra, grassa o una via di mezzo. Non puoi cambiare la conformazione del corpo del bersaglio, e quindi devi scegliere una forma che abbia la stessa distribuzione basilare di arti.

Per tutto il resto, l'illusione è limitata solo dalla tua fantasia. L'incantesimo permane per la sua durata, a meno che tu non usi una Azione per interromperlo prima. I cambi apportati da questo incantesimo non sono in grado di sostenere un'ispezione fisica. Per esempio, se usi questo incantesimo per aggiungere un cappello all'abbigliamento di una creatura, gli oggetti attraversano il cappello, e chiunque lo tocchi non avvertirebbe nulla e finirebbe per toccare la testa e i capelli della creatura.
Se usi questo incantesimo per apparire più magro di quello che sei, la mano di una persona che provasse a toccarti rimbalzerebbe su di te, mentre alla vista sembrerebbe fermarsi a mezz'aria. Una creatura può usare 2 Azioni per ispezionare un bersaglio ed effettuare una prova di Consapevolezza contro la DC del Tiro Salvezza dell'incantesimo, se impiega 3 Azione ha +1d6 di bonus. Se la riesce, capisce che il bersaglio è camuffato.

\incantesimo{Serratura Magica}\label{Arcane Lock}
\noindent\colorbox{OBSSgold!10}{
\begin{minipage}{0.95\linewidth}
\begin{description}[noitemsep, topsep=0pt, parsep=0pt, partopsep=0pt, leftmargin=0cm, labelwidth=1.3cm]
	\item[\textbf{Lista}]: Abiurazione
	\item[\textbf{Livello}]: 2, Comune
	\item[\textbf{Lancio}]: 2 Azioni
	\item[\textbf{Gittata}]: Contatto
	\item[\textbf{Durata}]: Fino a che dissolto
\end{description}
\end{minipage}}\smallskip

Lanci l'incantesimo a contatto di una porta, finestra, portale, forziere o altro ingresso chiuso, e questo diventa chiuso a chiave per la durata. Tu e le creature che hai indicato, quando hai lanciato questo incantesimo, potete aprire l'oggetto normalmente. Puoi anche predisporre una parola d'ordine che, quando pronunciata entro 1 metro dall'oggetto, sopprime l'incantesimo per 1 minuto. Altrimenti l'apertura è invalicabile fino a che non viene distrutta o l''incantesimo è dissolto o soppresso. Lanciare scassinare sull'oggetto sopprime Serratura Magica per 10 minuti.

Mentre è soggetto a questo incantesimo, l'oggetto è più difficile da distruggere o aprire a forza; la DC per romperlo o scassinare una serratura su di esso aumenta di 10.

\textbf{Per ogni Successo Critico Magico} ottenuto nella Prova di Magia puoi influenzare un altra chiusura o aumentare la difficoltà di apertura di 4.

\incantesimo{Servitore Invisibile}
\noindent\colorbox{OBSSgold!10}{
\begin{minipage}{0.95\linewidth}
\begin{description}[noitemsep, topsep=0pt, parsep=0pt, partopsep=0pt, leftmargin=0cm, labelwidth=1.3cm]
	\item[\textbf{Lista}]: Evocazione
	\item[\textbf{Livello}]: 1, Comune
	\item[\textbf{Lancio}]: 2 Azioni
	\item[\textbf{Gittata}]: 18 metri
	\item[\textbf{Durata}]: 1 ora
\end{description}
\end{minipage}}\smallskip

Questo incantesimo crea una forza quasi invisibile solo delimitata da una leggera aura (di colore a tua scelta) che svolge dei semplici compiti al tuo comando, fino al termine dell'incantesimo. Il servitore si forma in uno spazio sul terreno non occupato, entro la gittata. Ha Difesa 10, 1 punto ferita, Forza 0 e non può attaccare. Se scende a 0 Punti Ferita, l'incantesimo ha termine.

Come Azione Immediata, durante ciascun tuo round, puoi comandare mentalmente il servitore di muoversi fino a 3 metri e interagire con un oggetto. Il servitore può svolgere dei semplici compiti alla stregua di un servitore umano, come raccogliere cose, pulire, riparare, piegare abiti, accendere fuochi, servire il cibo e versare il vino. Una volta impartito il comando, il servitore svolgerà il compito al meglio delle sue capacità finché non l'avrà completato, e poi aspetterà il tuo prossimo comando.

Se comandi al servitore di svolgere un compito che lo farà muovere a più di 18 metri da te, l'incantesimo termina.

\incantesimo{Sfera Congelante}
\noindent\colorbox{OBSSgold!10}{
\begin{minipage}{0.95\linewidth}
\begin{description}[noitemsep, topsep=0pt, parsep=0pt, partopsep=0pt, leftmargin=0cm, labelwidth=1.3cm]
	\item[\textbf{Lista}]: Acqua
	\item[\textbf{Livello}]: 6, Raro
	\item[\textbf{Lancio}]: 2 Azioni
	\item[\textbf{Gittata}]: 90 metri
	\item[\textbf{Durata}]: Istantanea
\end{description}
\end{minipage}}\smallskip

Un globo gelido di energia fredda parte dalla punta delle tue dita verso un punto di tua scelta a gittata, dove esplode in una sfera di 18 metri di raggio. Ogni creatura nell'area deve effettuare un Tiro Salvezza su Tempra. Se fallisce il Tiro Salvezza, una creatura subisce 10d6 danni da freddo. Se lo supera, subisce la metà di questi danni.

Se il globo colpisce un corpo d'acqua o un liquido composto principalmente d'acqua (escluse però le creature a base d'acqua), congela il liquido fino a una profondità di 15 centimetri in un'area quadrata di 9 metri di lato. Il ghiaccio dura 1 minuto. Le creature che stavano nuotando sulla superficie dell'acqua congelata restano intrappolate nel ghiaccio. Una creatura intrappolata può usare due azioni per effettuare un nuovo Tiro Salvezza al fine di liberarsi.

Se lo desideri, dopo aver completato l'incantesimo, puoi trattenerti dallo sparare il globo. Un piccolo globo, circa delle dimensioni di una pietra da fionda, freddo al contatto, appare nella tua mano. In qualsiasi momento, tu, o una creatura a cui hai dato il globo, potete lanciare il globo (fino a una gittata di 12 metri). Questo si frantumerà all'impatto, con lo stesso effetto del normale lancio dell'incantesimo. Puoi anche poggiare il globo a terra senza che si frantumi. Dopo 1 minuto, se il globo non è già stato frantumato, esploderà.

\textbf{Per ogni Successo Critico Magico} ottenuto nella Prova di Magia il danno aumenta di 5d6.

\textbf{Tiro Salvezza Successo/Fallimento Critico}: In caso di Fallimento Critico il danno raddoppia, in caso di Successo Critico il danno viene ulteriormente dimezzato

\incantesimo{Sfera Infuocata}
\noindent\colorbox{OBSSgold!10}{
\begin{minipage}{0.95\linewidth}
\begin{description}[noitemsep, topsep=0pt, parsep=0pt, partopsep=0pt, leftmargin=0cm, labelwidth=1.3cm]
	\item[\textbf{Lista}]: Fuoco
	\item[\textbf{Livello}]: 2, Comune
	\item[\textbf{Lancio}]: 2 Azioni
	\item[\textbf{Gittata}]: 18 metri
	\item[\textbf{Durata}]: 1 minuto
\end{description}
\end{minipage}}\smallskip

Per la durata dell'incantesimo compare una sfera di 1 metro di diametro in uno spazio a gittata, scelto da te. Qualsiasi creatura che termini il suo round entro 1 metro dalla sfera deve effettuare un Tiro Salvezza su Riflessi. La creatura subisce 2d6 danni da fuoco se fallisce il Tiro Salvezza, o la metà di questi danni se lo supera.

Con un'Azione puoi spostare la sfera di 9 metri. Se fai schiantare la sfera contro una creatura, la creatura deve effettuare un Tiro Salvezza contro il danno della sfera, e la sfera smetterà di muoversi per quel round.
Quando muovi la sfera, la puoi spostare oltre barriere alte fino a 1 metro, e farle saltare spazi larghi fino a 3 metri. La sfera incendia gli oggetti infiammabili non indossati o trasportati, e irradia una luce intensa in un raggio di 3 metri e una luce fioca per ulteriori 3 metri.

Mentre hai questo incantesimo attivo sei Distratto nel lancio di altri incantesimi.

\textbf{Per ogni Successo Critico Magico} ottenuto nella Prova di Magia il danno aumenta di 1d8.

\incantesimo{Sfocatura}
\noindent\colorbox{OBSSgold!10}{
\begin{minipage}{0.95\linewidth}
\begin{description}[noitemsep, topsep=0pt, parsep=0pt, partopsep=0pt, leftmargin=0cm, labelwidth=1.3cm]
	\item[\textbf{Lista}]: Illusione
	\item[\textbf{Livello}]: 2, Comune
	\item[\textbf{Lancio}]: 2 Azioni
	\item[\textbf{Gittata}]: Personale
	\item[\textbf{Durata}]: 1 minuto
\end{description}
\end{minipage}}\smallskip

Il tuo corpo diventa sfocato, indistinto e tremolante a chiunque ti veda. Per la durata dell'incantesimo, tutte le creature hanno ha -1d6 ai Tiri per Colpire contro di te. Gli attaccanti che non si affidano alla vista sono immuni a questo effetto, per esempio se hanno vista cieca o sono in grado di distinguere le illusioni, come per visione del vero.

\incantesimo{Sguardo Penetrante}
\noindent\colorbox{OBSSgold!10}{
\begin{minipage}{0.95\linewidth}
\begin{description}[noitemsep, topsep=0pt, parsep=0pt, partopsep=0pt, leftmargin=0cm, labelwidth=1.3cm]
	\item[\textbf{Lista}]: Necromanzia
	\item[\textbf{Livello}]: 6, Molto Raro
	\item[\textbf{Lancio}]: 2 Azioni
	\item[\textbf{Gittata}]: Personale
	\item[\textbf{Durata}]: Concentrazione, massimo 1 minuto
\end{description}
\end{minipage}}\smallskip

Per la durata dell'incantesimo i tuoi occhi si tramutano in un vuoto nero infuso di terribile potere. Una creatura a tua scelta entro 18 metri da te e che puoi vedere deve superare un Tiro Salvezza su Volontà o, per la durata, subire uno dei seguenti effetti di tua scelta. Durante ciascun tuo round, fino al termine dell'incantesimo, puoi usare due Azioni per prendere come bersaglio un'altra creatura, ma non puoi prendere di nuovo come bersaglio una creatura che abbia superato un Tiro Salvezza contro questo lancio di sguardo penetrante.

\begin{itemize}[leftmargin=*] \setlength{\itemsep}{0pt}
	\item \emph{Addormentato}. Il bersaglio cade privo di sensi. Si risveglia qualora subisca qualsiasi ammontare di danno o se un'altra creatura usa 2 Azioni per scuoterlo dal sonno.
	\item \emph{Ammalato}. Il bersaglio ha -1d6 ai Tiri per Colpire e le prove su competenze di base. Al termine di ciascun suo round, può effettuare un altro Tiro Salvezza su Volontà. Se lo supera, l'effetto ha termine.
	\item \emph{Impanicato}. Il bersaglio è spaventato da te. Durante ciascun suo round, la creatura spaventata deve effettuare usare due Azioni di Movimento e muoversi lontano da te tramite il tragitto più breve e sicuro possibile, a meno che non abbia spazio per muoversi. Se il bersaglio si muove in un luogo lontano almeno 18 metri da te, dove non ti possa vedere, questo effetto ha termine.
\end{itemize}

\incantesimo{Silenzio}
\noindent\colorbox{OBSSgold!10}{
\begin{minipage}{0.95\linewidth}
\begin{description}[noitemsep, topsep=0pt, parsep=0pt, partopsep=0pt, leftmargin=0cm, labelwidth=1.3cm]
	\item[\textbf{Lista}]: Illusione
	\item[\textbf{Livello}]: 2, Comune
	\item[\textbf{Lancio}]: 2 Azioni
	\item[\textbf{Gittata}]: 36 metri
	\item[\textbf{Durata}]: 10 minuti
\end{description}
\end{minipage}}\smallskip

Per la durata dell'incantesimo, nessun suono può essere creato all'interno o attraversare una sfera di 6 metri di raggio centrata su di un punto a gittata, scelto da te. Qualsiasi creatura o oggetto che si trovi completamente all'interno della sfera è immune al danno da suono e le creature che sono completamente al suo interno sono assordate. È estremamente difficile (vedi \hyperlink{magieconimpedimenti}{Tentare magie con impedimenti}, pag. \pageref{magieconimpedimenti} ) lanciare un incantesimo che comprende una componente verbale mentre si è al suo interno. Lanciato su una creatura è concesso un Tiro Salvezza su Volontà per annullarne gli effetti.

\textbf{Per ogni Successo Critico Magico} ottenuto nella Prova di Magia la durata aumenta di 10 minuti.

\textbf{Con due Successo Critico Magico} puoi lanciare l'incantesimo su un oggetto o creatura che diventa il centro della sfera. In caso di creature è concesso un Tiro Salvezza su Volontà.

\incantesimo{Simbolo}
\noindent\colorbox{OBSSgold!10}{
\begin{minipage}{0.95\linewidth}
\begin{description}[noitemsep, topsep=0pt, parsep=0pt, partopsep=0pt, leftmargin=0cm, labelwidth=1.3cm]
	\item[\textbf{Lista}]: Abiurazione
	\item[\textbf{Livello}]: 7, Non Comune
	\item[\textbf{Lancio}]: 2 Azioni
	\item[\textbf{Gittata}]: Contatto
	\item[\textbf{Durata}]: Fino a che dissolto o attivato
\end{description}
\end{minipage}}\smallskip

Quando lanci questo incantesimo, inscrivi un glifo dannoso su di una superficie (come una sezione di pavimento, muro o un tavolo) o all'interno di un oggetto che può essere chiuso per nascondere il glifo (come un libro, una pergamena o un forziere). Se scegli una superficie, il glifo può coprire un'area di superficie non maggiore di 3 metri di diametro. Se scegli un oggetto,quell'oggetto deve restare al suo posto; se l'oggetto viene spostato più di 3 metri dal punto in cui è stato lanciato l'incantesimo, il glifo è spezzato, e l'incantesimo termina senza essere stato attivato.

Il glifo è quasi invisibile e può essere trovato con una prova di Sopravvivenza contro la DC del Tiro Salvezza dei tuoi incantesimi.

Decidi tu cosa attivi il glifo al momento del lancio dell'incantesimo.

Per i glifi inscritti su di una superficie, l'attivazione tipica comprende entrare in contatto o stare sopra il glifo, rimuovere un altro oggetto che copra il glifo, avvicinarsi a una certa distanza dal glifo, o manipolare l'oggetto su cui è inscritto il glifo.

Per i glifi inscritti su di un oggetto, l'attivazione tipica comprende aprire l'oggetto, avvicinarsi a una certa distanza dall'oggetto, o vedere o leggere il glifo.

Puoi definire meglio l'attivazione così che l'incantesimo si attivi solo in determinate circostanze o secondo certe peculiarità fisiche (come l'altezza o il peso) o specie di creatura (per esempio, la protezione potrebbe agire contro le megere o i mutaforma). Puoi anche predisporre condizioni per evitare che il glifo venga attivato, come la pronuncia di una parola d'ordine.

Quando inscrivi il glifo scegli una delle opzioni seguenti come suo effetto. Una volta attivato, il glifo riluce, riempiendo una sfera di 18 metri di raggio di luce fioca per 10 minuti, dopo i quali l'incantesimo termina. Ogni creatura nella sfera quando il glifo si attiva diventa bersaglio del suo effetto, così come una creatura che entri per la prima volta nella sfera durante un round o termine lì il suo round.

\begin{itemize}[leftmargin=*] \setlength{\itemsep}{0pt}
	\item \emph{Demenza}. Ogni bersaglio deve effettuare un Tiro Salvezza su Volontà. Se fallisce il Tiro Salvezza, il bersaglio diventa demente per 1 minuto. Una creatura demente non può effettuare azioni, non comprende quello che gli altri le dicono, non può leggere, e parla solo farfugliando. Il Narratore ne controlla i movimenti, che risultano erratici.
	\item \emph{Discordia}. Ogni bersaglio deve effettuare un Tiro Salvezza su Tempra. Se lo fallisce, il bersaglio inizia a bisticciare e discutere con un'altra creatura per 1 minuto. In questo periodo, è incapace di effettuare qualsiasi comunicazione significativa e ha -1d6 ai Tiri per Colpire e le prove su competenze di base.
	\item \emph{Dolore}. Ogni bersaglio deve effettuare un Tiro Salvezza su Tempra. Se lo fallisce, il bersaglio diventa inabile a causa del dolore lacerante.
	\item \emph{Morte}. Ogni bersaglio deve effettuare un Tiro Salvezza su Tempra, subendo 10d10 danni da Vuoto se lo fallisce, o la metà di questi danni se lo supera.
	\item \emph{Paura}. Ogni bersaglio deve effettuare un Tiro Salvezza su Volontà e, se lo fallisce, restare spaventato per 1 minuto. Mentre è spaventato, il bersaglio getta qualsiasi cosa stesse tenendo e deve muoversi almeno 9 metri lontano dal glifo durante ciascuno suo round, se in grado.
	\item \emph{Sfiducia}. Ogni bersaglio deve effettuare un Tiro Salvezza su Volontà. Se fallisce il Tiro Salvezza, il bersaglio è sopraffatto dalla disperazione per 1 minuto. Durante questo periodo, non può attaccare o prendere come bersaglio nessuna creatura con capacità, incantesimi o altri effetti magici nocivi.
	\item \emph{Sonno}. Ogni bersaglio deve effettuare un Tiro Salvezza su Volontà, e cadere privo di sensi per 10 minuti se lo fallisce. Una creatura si risveglia se subisce danni o se qualcuno usa un'Azione per risvegliarla.
	\item \emph{Stordimento}. Ogni bersaglio deve effettuare un Tiro Salvezza su Volontà, e restare stordito per 1 minuto se lo fallisce.
\end{itemize}

\incantesimo{Sogno}
\noindent\colorbox{OBSSgold!10}{
\begin{minipage}{0.95\linewidth}
\begin{description}[noitemsep, topsep=0pt, parsep=0pt, partopsep=0pt, leftmargin=0cm, labelwidth=1.3cm]
	\item[\textbf{Lista}]: Illusione
	\item[\textbf{Livello}]: 5, Non Comune
	\item[\textbf{Lancio}]: 2 Azioni
	\item[\textbf{Gittata}]: Speciale
	\item[\textbf{Durata}]: 8 ore
\end{description}
\end{minipage}}\smallskip

Questo incantesimo modella i sogni di una creatura. Scegli una creatura a te nota come bersaglio dell'incantesimo. Il bersaglio deve trovarsi sul tuo stesso piano di esistenza. Le creature che non dormono non possono essere soggette a questo incantesimo. Tu o una creatura consenziente con cui sei a contatto entrate in uno stato di trance, agendo da messaggero. Mentre è in trance, il messaggero è consapevole di ciò che lo circonda, ma non può effettuare azioni o muoversi.

Per la durata dell'incantesimo, se il bersaglio è addormentato, il messaggero appare nei sogni del bersaglio e può conversare con lui finché questi rimane addormentato. Il messaggero può anche modellare l'ambiente del sogno, creando terreni, oggetti e altre immagini. Il messaggero può emergere dalla trance in qualsiasi momento, terminando anticipatamente l'effetto dell'incantesimo. Al risveglio, il bersaglio ricorda perfettamente il suo sogno. Se il bersaglio è sveglio quando lanci l'incantesimo, il messaggero ne viene a conoscenza e può porre fine alla trance (e all'incantesimo) o aspettare che il bersaglio si addormenti. A quel punto il messaggero potrà comparire nei sogni del bersaglio.

Puoi fare apparire il messaggero al bersaglio con un aspetto mostruoso e terrificante. Se lo fai, il messaggero può consegnare un messaggio di al massimo dieci parole e poi il bersaglio deve effettuare un Tiro Salvezza su Volontà. Se fallisce il Tiro Salvezza, gli echi della spaventosa mostruosità generano un incubo per la durata del sonno del bersaglio che gli impedisce di ottenere qualsiasi beneficio da quel riposo. Inoltre, quando il bersaglio si sveglia, subisce 3d6 danni.

Se possiedi una ciocca di capelli, delle unghie tagliate, o simile porzione del corpo del bersaglio, egli effettuerà il suo Tiro Salvezza con -1d6.

\incantesimo{Sonnellino}
\noindent\colorbox{OBSSgold!10}{
\begin{minipage}{0.95\linewidth}
\begin{description}[noitemsep, topsep=0pt, parsep=0pt, partopsep=0pt, leftmargin=0cm, labelwidth=1.3cm]
	\item[\textbf{Lista}]: Alterazione
	\item[\textbf{Livello}]: 2, Leggendario
	\item[\textbf{Lancio}]: 1 round
	\item[\textbf{Gittata}]: 6 metri
	\item[\textbf{Durata}]: 1 minuto
\end{description}
\end{minipage}}\smallskip

Questo incantesimo permette all'incantatore di mettere a riposo per 1 ora fino ad 1 creatura per (Competenza Magica/6 + Adepto della Magia). La creatura deve essere consenziente.

Quest'ora di riposo è equivalente a 8 ore di riposo per quanto riguarda il recupero dei Punti Magia e Punti Ferita. L'incantesimo non è usufruibile ad intervalli inferiori alle 35 ore.

\textbf{Per ogni Successo Critico Magico} ottenuto nella Prova di Magia influenzi 1 creatura in più.

\incantesimo{Sonno}
\noindent\colorbox{OBSSgold!10}{
\begin{minipage}{0.95\linewidth}
\begin{description}[noitemsep, topsep=0pt, parsep=0pt, partopsep=0pt, leftmargin=0cm, labelwidth=1.3cm]
	\item[\textbf{Lista}]: Ammaliamento
	\item[\textbf{Livello}]: 1, Comune
	\item[\textbf{Lancio}]: 2 Azioni
	\item[\textbf{Gittata}]: 27 metri
	\item[\textbf{Durata}]: 1 minuto
\end{description}
\end{minipage}}\smallskip

In un raggio di 2 metri le creature devono fare un Tiro Salvezza su Volontà o cadere addormentate. Qualsiasi cosa danneggi le creature o le influenzi causa l'immediato scioglimento del sonno magico. Le creature con Grado di Sfida superiore a 3 non sono influenzate.

\textbf{Per ogni due Successo Critico Magico} ottenuto nella Prova di Magia puoi influenzare creature 1 Grado Sfida più alto.

\incantesimo{Spada Arcana}
\noindent\colorbox{OBSSgold!10}{
\begin{minipage}{0.95\linewidth}
\begin{description}[noitemsep, topsep=0pt, parsep=0pt, partopsep=0pt, leftmargin=0cm, labelwidth=1.3cm]
	\item[\textbf{Lista}]: Invocazione
	\item[\textbf{Livello}]: 7, Raro
	\item[\textbf{Lancio}]: 2 Azioni
	\item[\textbf{Gittata}]: 18 metri
	\item[\textbf{Durata}]: Concentrazione, massimo 1 minuto
\end{description}
\end{minipage}}\smallskip

Per la durata dell'incantesimo, crei a gittata una fluttuante spada di forza. Quando la spada appare, effettui un attacco in mischia con modificatore CM + modificatore da incantesimo contro un bersaglio scelto da te entro 1 metro dalla spada. Se colpisci, il bersaglio subisce 4d10 danni da forza. Fino al termine dell'incantesimo,puoi usare un'Azione ogni tuo round per muovere la spada di 6 metri in un punto che puoi vedere e ripetere questo attacco contro lo stesso bersaglio o uno differente.

\incantesimo{Spruzzo Colorato}
\noindent\colorbox{OBSSgold!10}{
\begin{minipage}{0.95\linewidth}
\begin{description}[noitemsep, topsep=0pt, parsep=0pt, partopsep=0pt, leftmargin=0cm, labelwidth=1.3cm]
	\item[\textbf{Lista}]: Illusione
	\item[\textbf{Livello}]: 1, Comune
	\item[\textbf{Lancio}]: 2 Azioni
	\item[\textbf{Gittata}]: Personale (cono di 3 metri)
	\item[\textbf{Durata}]: 1 round
\end{description}
\end{minipage}}\smallskip

Uno spruzzo di luci e colori erutta dalla tua mano. Le creature in un cono di 3 metri devono fare un Tiro Salvezza su Volontà.

Se il Tiro Salvezza riesce non si subisce alcun effetto, se fallisce la creatura \hyperlink{confusionecondizione}{confusa} per 1 round.

\textbf{Per ogni Successo Critico Magico} ottenuto nella Prova di Magia la durata aumenta di 1 round.

\incantesimo{Spruzzo Prismatico}
\noindent\colorbox{OBSSgold!10}{
\begin{minipage}{0.95\linewidth}
\begin{description}[noitemsep, topsep=0pt, parsep=0pt, partopsep=0pt, leftmargin=0cm, labelwidth=1.3cm]
	\item[\textbf{Lista}]: Invocazione
	\item[\textbf{Livello}]: 7, Raro
	\item[\textbf{Lancio}]: 2 Azioni
	\item[\textbf{Gittata}]: Personale (cono di 18 metri)
	\item[\textbf{Durata}]: Istantanea
\end{description}
\end{minipage}}\smallskip

Otto raggi di luce multicolore partono dalla tua mano. Ogni raggio è di un diverso colore e ha un potere e uno scopo diverso. Ogni creatura in un cono di 18 metri deve effettuare un Tiro Salvezza su Riflessi. Per ogni bersaglio, tirare un d8 per determinare quale sia il colore del raggio che lo ha colpito.

\begin{itemize}[leftmargin=*] \setlength{\itemsep}{0pt}
	\item \emph{1. Rosso}. Il bersaglio subisce 10d6 danni da fuoco se fallisce il Tiro Salvezza, o la metà di questi danni se lo supera.
	\item \emph{2. Arancio}. Il bersaglio subisce 10d6 danni da acido se fallisce il Tiro Salvezza, o la metà di questi danni se lo supera.
	\item \emph{3. Giallo}. Il bersaglio subisce 10d6 danni da elettricità se fallisce il Tiro Salvezza, o la metà di questi danni se lo supera.
	\item \emph{4. Verde}. Il bersaglio subisce 10d6 danni da veleno se fallisce il Tiro Salvezza, o la metà di questi danni se lo supera.
	\item \emph{5. Blu}. Il bersaglio subisce 10d6 danni da freddo se fallisce il Tiro Salvezza, o la metà di questi danni se lo supera.
	\item \emph{6. Indaco}. Se fallisce il Tiro Salvezza, il bersaglio è intralciato. Deve poi effettuare un Tiro Salvezza su Tempra all'inizio di ciascun suo round. Se supera il Tiro Salvezza tre volte, l'incantesimo termina. Se fallisce il Tiro Salvezza tre volte, viene permanentemente trasformato in pietra e diventa vittima della condizione pietrificato. I successi e i fallimenti non devono essere consecutivi; tieni traccia di entrambi finché il bersaglio non ne ha ottenuti tre dello stesso tipo.
	\item \emph{7. Violetto}. Se fallisce il Tiro Salvezza, il bersaglio è accecato. Deve poi effettuare un Tiro Salvezza su Volontà all'inizio del tuo prossimo round. Se supera il Tiro Salvezza, la cecità termina. Se fallisce il Tiro Salvezza, la creatura viene trasportata su di un altro piano di esistenza a scelta del Narratore e non è più accecata (di solito, una creatura che non è sul suo piano natio, viene esiliata su di esso, mentre le altre creature sono di solito portate nei piani Astrale o Etereo).
	\item \emph{8. Speciale}. Il bersaglio è colpito da due raggi. Tira altre due volte, ritirando gli 8.

\end{itemize}

\incantesimo{Spruzzo Velenoso}
\noindent\colorbox{OBSSgold!10}{
\begin{minipage}{0.95\linewidth}
\begin{description}[noitemsep, topsep=0pt, parsep=0pt, partopsep=0pt, leftmargin=0cm, labelwidth=1.3cm]
	\item[\textbf{Lista}]: Animali e Piante
	\item[\textbf{Livello}]: 0, Non Comune
	\item[\textbf{Lancio}]: 1 Azione
	\item[\textbf{Gittata}]: 3 metri
	\item[\textbf{Durata}]: Istantanea
\end{description}
\end{minipage}}\smallskip

Stendi la mano verso una creatura a gittata e che puoi vedere, e proietti una nube di gas velenoso dal tuo palmo. La creatura deve superare un Tiro Salvezza su Tempra o subire 1d12 danni da veleno.

Puoi aumentare il danno dell'incantesimo di 1d8 quando raggiungi CM 5, CM 11 e CM 17, ma costa 2 Azioni lanciarlo potenziato e 1 Punti Magia, è altresì necessario avere preso Adepto della Magia un numero di volte pari ai potenziamenti che si vogliono applicare.

\textbf{Per ogni due Successo Critico Magico ottenuto} nella Prova di Magia influenzi un altra creatura entro gittata.


\incantesimo{Statua}
\noindent\colorbox{OBSSgold!10}{
\begin{minipage}{0.95\linewidth}
\begin{description}[noitemsep, topsep=0pt, parsep=0pt, partopsep=0pt, leftmargin=0cm, labelwidth=1.3cm]
	\item[\textbf{Lista}]: Terra, Trasmutazione
	\item[\textbf{Livello}]: 7, Raro
	\item[\textbf{Lancio}]: 2 Azioni
	\item[\textbf{Gittata}]: Tocco
	\item[\textbf{Durata}]: 1 ora per livello
\end{description}
\end{minipage}}\smallskip

Questo incantesimo trasforma l'incantatore o il soggetto consenziente in pietra, insieme a qualsiasi abito o oggetto trasportato. Il soggetto può vedere e percepire suoni e odori, ma non ha bisogno di mangiare o respirare. Il senso del tatto è limitato alle sensazioni percepibili dalla sostanza granitica di cui è composto il corpo del soggetto. Una scheggiatura è paragonabile a un semplice graffio, ma spezzare un braccio della statua equivale a una mutilazione. Il soggetto di statua può tornare allo stato normale e ridiventare di pietra tutte le volte che vuole durante la durata dell'incantesimo. La statua ha durezza 15 ed il doppio dei Punti Ferita della creatura originaria.

\textbf{Per ogni Successo Critico Magico} ottenuto nella Prova di Magia raddoppi la durata o influenzi un altra creatura.

\incantesimo{Stretta Folgorante}
\noindent\colorbox{OBSSgold!10}{
\begin{minipage}{0.95\linewidth}
\begin{description}[noitemsep, topsep=0pt, parsep=0pt, partopsep=0pt, leftmargin=0cm, labelwidth=1.3cm]
	\item[\textbf{Lista}]: Aria
	\item[\textbf{Livello}]: 0, Comune
	\item[\textbf{Lancio}]: 1 Azione
	\item[\textbf{Gittata}]: Contatto
	\item[\textbf{Durata}]: Istantanea
\end{description}
\end{minipage}}\smallskip

Dalle tue mani saettano lampi che infliggono una scossa a una creatura con cui provi a entrare in contatto. Effettua un attacco in mischia con incantesimo contro il bersaglio. Hai +1d6 sul Tiro per Colpire se il bersaglio sta indossando un'armatura fatta di metallo. Se colpisci, il bersaglio subisce 1d8 danni da elettricità, e non può effettuare reazioni fino all'inizio del suo prossimo round.

Puoi aumentare il danno dell'incantesimo di 1d8 quando raggiungi CM 5, CM 11 e CM 17, ma costa 2 Azioni lanciarlo potenziato e 1 Punti Magia, è altresì necessario avere preso Adepto della Magia un numero di volte pari ai potenziamenti che si vogliono applicare.

\textbf{Per ogni Successo Critico Magico ottenuto} l'incantesimo \emph{può saltare} su un altra creatura nemica entro 1 metro da quella iniziale o somma 1d6 di danno aggiuntivo.

\incantesimo{Succo concentrato di Ribes di Kyrin}
\noindent\colorbox{OBSSgold!10}{
\begin{minipage}{0.95\linewidth}
\begin{description}[noitemsep, topsep=0pt, parsep=0pt, partopsep=0pt, leftmargin=0cm, labelwidth=1.3cm]
	\item[\textbf{Lista}]: Animali e Piante, Terra
	\item[\textbf{Livello}]: 2, Non Comune
	\item[\textbf{Lancio}]: 2 Azioni
	\item[\textbf{Gittata}]: 9 metri
	\item[\textbf{Durata}]: 1 minuto
\end{description}
\end{minipage}}\smallskip

Estrai la linfa acida dai ribes e proietti una linea di spruzzo d'acido lunga 9 metri e larga 1 metro in una direzione a tua scelta. Ogni creatura nella linea deve superare un Tiro Salvezza su Riflessi o essere ricoperta di acido per la durata dell'incantesimo o finché una creatura non usa due Azioni per togliere via l'acido da sé o da un'altra creatura. Una creatura coperta dall'acido subisce 2d4 danni da acido all'inizio di ciascuno dei suoi round.

\textbf{Per ogni Successo Critico Magico} ottenuto nella Prova di Magia il danno aumenta di 2d4

\incantesimo{Suggestione}
\noindent\colorbox{OBSSgold!10}{
\begin{minipage}{0.95\linewidth}
\begin{description}[noitemsep, topsep=0pt, parsep=0pt, partopsep=0pt, leftmargin=0cm, labelwidth=1.3cm]
	\item[\textbf{Lista}]: Ammaliamento
	\item[\textbf{Livello}]: 2, Raro
	\item[\textbf{Lancio}]: 2 Azioni
	\item[\textbf{Gittata}]: 9 metri
	\item[\textbf{Durata}]: 8 ore
\end{description}
\end{minipage}}\smallskip

Suggerisci un corso di attività (limitato a una o due frasi) e influenzi magicamente una creatura a gittata e che puoi vedere e udire e ti possa capire, scelta da te. Le creature che non possono essere affascinate sono immuni a questo effetto. La suggestione deve essere pronunciata in modo che il corso d'azione suoni ragionevole. Chiedere a una creatura di pugnalarsi,gettarsi su una lancia, darsi fuoco, o fare qualche altro atto palesemente dannoso nega automaticamente gli effetti dell'incantesimo.

Il bersaglio deve effettuare un Tiro Salvezza su Volontà. Se fallisce il Tiro Salvezza, esso segue il corso d'azione da te descritto al meglio delle sue capacità. Il corso d'azione suggerito può proseguire per l'intera durata dell'incantesimo. Se l'attività suggerita può essere completata in un tempo più breve,l'incantesimo ha termine quando il soggetto termina di fare ciò che gli è stato chiesto.

Puoi anche specificare condizioni che attiveranno un'attività speciale per la durata dell'incantesimo. Per esempio, potresti suggerire a un cavaliere di cedere il suo saurovallo da guerra al primo mendicante che incontri. Se la condizione non viene soddisfatta prima del termine dell'incantesimo, l'attività non verrà svolta. Se tu o uno qualsiasi dei tuoi compagni danneggia il bersaglio, l'incantesimo ha termine.

\incantesimo{Suggestione di Massa}
\noindent\colorbox{OBSSgold!10}{
\begin{minipage}{0.95\linewidth}
\begin{description}[noitemsep, topsep=0pt, parsep=0pt, partopsep=0pt, leftmargin=0cm, labelwidth=1.3cm]
	\item[\textbf{Lista}]: Ammaliamento
	\item[\textbf{Livello}]: 6, Molto Raro
	\item[\textbf{Lancio}]: 2 Azioni
	\item[\textbf{Gittata}]: 18 metri
	\item[\textbf{Durata}]: 24 ore
\end{description}
\end{minipage}}\smallskip

Suggerisci un corso di attività (limitato a una o due frasi) e influenzi magicamente fino a dodici creature a gittata che puoi vedere e udire e ti possano capire, scelte da te. Le creature che non possono essere affascinate sono immuni a questo effetto. La suggestione deve essere pronunciata in modo che il corso d'azione suoni ragionevole. Chiedere a una creatura di pugnalarsi, gettarsi su di una lancia, darsi fuoco, o fare qualche altro atto palesemente dannoso nega automaticamente gli effetti dell'incantesimo.

Ogni bersaglio deve effettuare un Tiro Salvezza su Volontà. Se fallisce il Tiro Salvezza, esso segue il corso d'azione da te descritto al meglio delle sue capacità. Il corso d'azione suggerito può proseguire per l'intera durata dell'incantesimo. Se l'attività suggerita può essere completata in un tempo più breve, l'incantesimo ha termine quando il soggetto termina di fare ciò che gli è stato chiesto.

Puoi anche specificare condizioni che attiveranno un'attività speciale per la durata dell'incantesimo. Per esempio, potresti suggerire a un gruppo di soldati di cedere tutti i loro soldi al primo mendicante che incontrino. Se la condizione non viene soddisfatta prima del termine dell'incantesimo, l'attività non verrà svolta. Se tu o uno qualsiasi dei tuoi compagni danneggia una creatura soggetta a questo incantesimo, per quella creatura l'incantesimo ha termine.

\textbf{Per ogni Successo Critico Magico} ottenuto nella Prova di Magia aggiungi un giorno alla durata.

\incantesimo{Taumaturgia}
\noindent\colorbox{OBSSgold!10}{
\begin{minipage}{0.95\linewidth}
\begin{description}[noitemsep, topsep=0pt, parsep=0pt, partopsep=0pt, leftmargin=0cm, labelwidth=1.3cm]
	\item[\textbf{Lista}]: Universale
	\item[\textbf{Livello}]: 0, Non Comune
	\item[\textbf{Lancio}]: 2 Azioni
	\item[\textbf{Gittata}]: 9 metri
	\item[\textbf{Durata}]: Massimo 1 minuto
\end{description}
\end{minipage}}\smallskip

Manifesti a gittata una trucco minore, un segno di potere soprannaturale. Crei a gittata uno dei seguenti effetti magici:

\begin{itemize}[leftmargin=*] \setlength{\itemsep}{0pt}
	\item La tua voce risuona tre volte più forte del normale per 1 minuto.
	\item Fai sì che le fiamme tremolino, si intensifichino, affievoliscano o cambino colore per 1 minuto.
	\item Provochi innocui tremori sul terreno per 1 minuto.
	\item Crei un rumore istantaneo, come un rombo di tuono, il verso di un corvo, o un sussurro inquietante, che origina da un punto a gittata scelto da te.
	\item Fai sì che una porta o una finestra non chiusa a chiave si spalanchi o si chiuda di colpo.
	\item Modifichi l'aspetto dei tuoi occhi per 1 minuto.
\end{itemize}

Se lanci questo incantesimo più volte, puoi tenere attivi fino a tre effetti da un minuto alla volta e puoi interrompere questi effetti con un'Azione.

\textbf{Per ogni Successo Critico Magico} ottenuto nella Prova di Magia puoi manifestare un effetto magico aggiuntivo.

\incantesimo{Telecinesi}
\noindent\colorbox{OBSSgold!10}{
\begin{minipage}{0.95\linewidth}
\begin{description}[noitemsep, topsep=0pt, parsep=0pt, partopsep=0pt, leftmargin=0cm, labelwidth=1.3cm]
	\item[\textbf{Lista}]: Trasmutazione
	\item[\textbf{Livello}]: 5, Non Comune
	\item[\textbf{Lancio}]: 2 Azioni
	\item[\textbf{Gittata}]: 18 metri
	\item[\textbf{Durata}]: Concentrazione, massimo 10 minuti
\end{description}
\end{minipage}}\smallskip

Ottieni la capacità di muovere o manipolare creature o oggetti tramite il pensiero. Quando lanci questo incantesimo e con 2 Azioni durante ciascun round, puoi esercitare la tua volontà su di una creatura od oggetto a gittata e che puoi vedere, provocando l'effetto appropriato tra quelli seguenti. Puoi agire round dopo round sempre sullo stesso bersaglio, o sceglierne uno nuovo ogni volta. Se cambi bersaglio, il bersaglio precedente non è più soggetto all'incantesimo.

\emph{Creatura}. Puoi tentare di muovere una creatura di taglia Enorme o più piccola. Effettua un Tiro Salvezza contrapposto tra la tua Volontà con modificatore la tua caratteristica da incantatore contro un Tiro Salvezza su Tempra. Se vinci la contesa, muovi la creatura di 9 metri in qualsiasi direzione, compreso verso l'alto, ma senza eccedere la gittata dell'incantesimo. Fino al termine del tuo prossimo round, la creatura è intralciata dalla tua presa telecinetica. Una creatura sollevata in alta, resta sospesa a mezz'aria.

Nei round successivi, puoi usare 2 Azioni per tentare di mantenere la tua presa telecinetica sulla creatura ripetendo la contesa.

\emph{Oggetto}. Puoi tentare di muovere un oggetto che pesa fino a 500 chili. Se l'oggetto non è indossato o trasportato, lo sposti automaticamente di 9 metri in qualsiasi direzione, ma senza superare la gittata dell'incantesimo.

Se l'oggetto è indossato o trasportato da una creatura, devi effettuare un Tiro Salvezza contrapposto tra la tua Volontà con modificatore la tua caratteristica da incantatore contro un Tiro Salvezza su Tempra modificato da Forza della creatura che lo trattiene. Se vinci la contesa, trascini via l'oggetto da quella creatura e lo muovi di 9 metri in una qualsiasi direzione, senza però superare la gittata dell'incantesimo.

Puoi esercitare un controllo preciso sugli oggetti tramite la tua presa telecinetica, riuscendo così a manipolare un attrezzo semplice, aprire una porta o un contenitore,inserire o recuperare un oggetto da un contenitore aperto, o versare del materiale in una fiala.

\incantesimo{Teletrasporto}
\noindent\colorbox{OBSSgold!10}{
\begin{minipage}{0.95\linewidth}
\begin{description}[noitemsep, topsep=0pt, parsep=0pt, partopsep=0pt, leftmargin=0cm, labelwidth=1.3cm]
	\item[\textbf{Lista}]: Evocazione
	\item[\textbf{Livello}]: 7, Comune
	\item[\textbf{Lancio}]: 2 Azioni
	\item[\textbf{Gittata}]: 3 metri
	\item[\textbf{Durata}]: Istantanea
\end{description}
\end{minipage}}\smallskip

Questo incantesimo teletrasporta istantaneamente te e altre otto creature consenzienti (oppure un singolo oggetto) a gittata e che puoi vedere, scelte da te, in una destinazione di tua scelta. Se il bersaglio è un oggetto, deve poter entrare in una sfera di 2 metri di raggio e non può essere tenuto o trasportato da una creatura non consenziente.

La destinazione che scegli ti deve essere nota, e deve essere sullo stesso piano di esistenza in cui ti trovi. La tua familiarità con la destinazione determina se vi riesce ad arrivare.

Il Narratore tira un d100 e consulta la tabella.

\begin{itemize}[leftmargin=*] \setlength{\itemsep}{0pt}
	\item \emph{Cerchio permanente} indica un cerchio di teletrasporto permanente di cui conosci la sequenza dei sigilli.
	\item \emph{Oggetto associato} indica che possiedi uno oggetto preso negli ultimi sei mesi dalla destinazione desiderata, come il libro della biblioteca di un mago, biancheria della suite reale, o un pezzo di marmo della tomba segreta di un lich.
	\item \emph{Molto familiare} è un luogo in cui sei stato molto spesso, un posto che hai studiato attentamente, o un posto che puoi vedere quando lanci l'incantesimo.
	\emph{Visto casualmente} è un posto che hai visto più di una volta ma con cui non sei molto familiare.
	\item \emph{Visto una volta} è un posto che hai visto una volta sola, magari tramite la magia.\\ \emph{Descrizione} è un luogo la cui posizione e aspetto conosci solo tramite la descrizione di qualcun altro, magari una mappa.
	\item \emph{Falsa destinazione} è un posto che non esiste. Magari hai cercato di scrutare il nascondiglio di un nemico ma hai invece visto un'illusione, oppure stai cercando di teletrasportarti in un posto familiare che non esiste più.
	\item \emph{Sul Bersaglio}. Tu e il tuo gruppo (o l'oggetto bersaglio) apparite dove desideri.
	\item \emph{Fuori Bersaglio}. Tu e il tuo gruppo (o l'oggetto bersaglio) apparite a una distanza casuale dalla destinazione in una direzione casuale. La distanza fuori bersaglio è 1d10 x 1d10 percento della distanza viaggiata. Per esempio, se hai provato a viaggiare per 180 chilometri, atterri fuori bersaglio e tiri 5 e 3 su due d10, allora saresti fuori bersaglio del 15\%, ovvero 27 chilometri. Il Narratore determina la direzione fuori bersaglio casualmente, tirando un d8 e indicando l'1 come nord, il 2 come nordest, il 3 come est e così via seguendo le direzioni della bussola. Se ti stai teletrasportando in una città costiera e finisci 27 chilometri al largo in mare, potresti essere nei guai!
	\item \emph{Area Simile}. Tu e il tuo gruppo (o l'oggetto bersaglio) finite in un'area diversa che è visualmente o tematicamente simile all'area bersaglio. Per esempio, se sei diretto al tuo laboratorio personale, potresti finire nel laboratorio di un altro incantatore o in un negozio di oggetti alchemici che possiede molti degli attrezzi e strumenti del tuo laboratorio. In genere, compari nel luogo simile più vicino, ma dato che l'incantesimo non ha limiti di gittata, potresti finire praticamente dovunque sullo stesso piano.
	\item \emph{Errore}. L'imprevedibile magia dell'incantesimo provoca un viaggio difficile. Ogni creatura teletrasportata (o l'oggetto bersaglio) subisce 3d10 danni da forza, e il Narratore ritira sulla tabella per vedere dove finiscano (possono capitare più errori, che infliggono danni ogni volta).
\end{itemize}

\end{multicols}

\medskip

\noindent\begin{tabularx}{\linewidth}{lllll}
	\toprule
\rowcolor{gray!20}d100 &Errore&Area Simile&Fuori Bersaglio&Sul Bersaglio\\
\toprule
Cerchio permanente&-&-&-&01-100\\
\rowcolor{gray!20}Oggetto Associato&-&-&-&01-100\\
Molto Familiare&01-05&06-13&14-24&25-100\\
\rowcolor{gray!20}Visto per caso&01-33&34-43&44-53&54-100\\
Visto una volta&01-43&44-53&54-73&74-100\\
\rowcolor{gray!20}Descrizione&01-43&44-53&54-73&74-100\\
Falsa Destinazione&01-50&51-100&-&-
\end{tabularx}

\medskip

\begin{multicols}{2}

\incantesimo{Tempesta di Fuoco}
\noindent\colorbox{OBSSgold!10}{
\begin{minipage}{0.95\linewidth}
\begin{description}[noitemsep, topsep=0pt, parsep=0pt, partopsep=0pt, leftmargin=0cm, labelwidth=1.3cm]
	\item[\textbf{Lista}]: Fuoco
	\item[\textbf{Livello}]: 7, Raro
	\item[\textbf{Lancio}]: 2 Azioni
	\item[\textbf{Gittata}]: 45 metri
	\item[\textbf{Durata}]: Istantanea
\end{description}
\end{minipage}}\smallskip

Una tempesta composta di fiamme roboanti compare in un punto a gittata, scelto da te. L'area della tempesta consiste di un massimo di dieci cubi di 3 metri di spigolo adiacenti, che puoi disporre come preferisci. Ogni cubo deve avere almeno una faccia adiacente a quella di un altro cubo. Ogni creatura nell'area deve effettuare un Tiro Salvezza su Riflessi. Se lo fallisce subisce 7d10 danni da fuoco, o la metà di questi danni se lo supera. Il fuoco danneggia gli oggetti nell'area e incendia gli oggetti infiammabili che non sono indossati o trasportati. Se lo desideri, la vita vegetale nell'area resta illesa dagli effetti di questo incantesimo.

\textbf{Per ogni Successo Critico Magico} ottenuto nella Prova di Magia aumenti l'area di un cubo di 3 metri di spigolo.

\textbf{Tiro Salvezza Successo/Fallimento Critico}: In caso di Fallimento Critico il danno raddoppia, in caso di Successo Critico il danno viene ulteriormente dimezzato

\incantesimo{Tempesta di Ghiaccio}
\noindent\colorbox{OBSSgold!10}{
\begin{minipage}{0.95\linewidth}
\begin{description}[noitemsep, topsep=0pt, parsep=0pt, partopsep=0pt, leftmargin=0cm, labelwidth=1.3cm]
	\item[\textbf{Lista}]: Acqua, Aria
	\item[\textbf{Livello}]: 4, Non Comune
	\item[\textbf{Lancio}]: 2 Azioni
	\item[\textbf{Gittata}]: 90 metri
	\item[\textbf{Durata}]: Istantanea
\end{description}
\end{minipage}}\smallskip

Una grandinata di ghiaccio si abbatte a terra in un cilindro di 6 metri di raggio e 12 metri di altezza centrato su di un punto a gittata. Ogni creatura nel cilindro deve effettuare un Tiro Salvezza su Riflessi. La creatura subisce 2d8 danni contundenti e 4d6 danni da freddo se fallisce il Tiro Salvezza, o la metà se lo supera. La grandine trasforma l'area di effetto della tempesta in terreno difficile fino al termine del tuo prossimo round.

\textbf{Per ogni Successo Critico Magico} ottenuto nella Prova di Magia il danno contundente aumenta di 2d8 e quello da freddo di 2d6.

\textbf{Tiro Salvezza Successo/Fallimento Critico}: In caso di Fallimento Critico il danno raddoppia, in caso di Successo Critico il danno viene ulteriormente dimezzato

\incantesimo{Tempesta di Nevischio}
\noindent\colorbox{OBSSgold!10}{
\begin{minipage}{0.95\linewidth}
\begin{description}[noitemsep, topsep=0pt, parsep=0pt, partopsep=0pt, leftmargin=0cm, labelwidth=1.3cm]
	\item[\textbf{Lista}]: Acqua
	\item[\textbf{Livello}]: 3, Molto Raro
	\item[\textbf{Lancio}]: 2 Azioni
	\item[\textbf{Gittata}]: 45 metri
	\item[\textbf{Durata}]: 1 minuto
\end{description}
\end{minipage}}\smallskip

Fino al termine dell'incantesimo, pioggia gelida e nevischio si abbattono in un cilindro alto 6 metri e del raggio di 12 metri centrato in un punto da te scelto a gittata. L'area è in penombra, mentre le fiamme esposte vengono spente. Il terreno nell'area è coperto di ghiaccio scivoloso, rendendolo terreno difficile. Quando una creatura entra nell'area dell'incantesimo per la prima volta durante un round o inizia il suo round lì, deve effettuare un Tiro Salvezza su Riflessi. Se lo fallisce, cade prona. Se una creatura nell'area dell'incantesimo si sta concentrando, deve superare un Tiro Salvezza su Tempra contro la DC del Tiro Salvezza dell'incantesimo o perdere la concentrazione.

\incantesimo{Tentacoli Neri}
\noindent\colorbox{OBSSgold!10}{
\begin{minipage}{0.95\linewidth}
\begin{description}[noitemsep, topsep=0pt, parsep=0pt, partopsep=0pt, leftmargin=0cm, labelwidth=1.3cm]
	\item[\textbf{Lista}]: Evocazione
	\item[\textbf{Livello}]: 4, Non Comune
	\item[\textbf{Lancio}]: 2 Azioni
	\item[\textbf{Gittata}]: 27 metri
	\item[\textbf{Durata}]: 1 minuto
\end{description}
\end{minipage}}\smallskip

Viscidi tentacoli d'ebano riempiono un quadrato di 6 metri di lato sul terreno, a gittata e che puoi vedere. Per la durata dell'incantesimo, questi tentacoli trasformano l'area in terreno difficile.

Quando una creatura entra nell'area soggetta per la prima volta in un round o comincia qui il suo round, deve superare un Tiro Salvezza su Riflessi o subire 3d6 danni contundenti e rimanere \hyperlink{intralciato}{intralciata} dai tentacoli fino al termine dell'incantesimo. Una creatura intralciata dai tentacoli può usare 2 Azioni per effettuare un nuovo Tiro Salvezza per essere libera in quel round.

\incantesimo{Terremoto}
\noindent\colorbox{OBSSgold!10}{
\begin{minipage}{0.95\linewidth}
\begin{description}[noitemsep, topsep=0pt, parsep=0pt, partopsep=0pt, leftmargin=0cm, labelwidth=1.3cm]
	\item[\textbf{Lista}]: Terra
	\item[\textbf{Livello}]: 8, Molto Raro
	\item[\textbf{Lancio}]: 2 Azioni
	\item[\textbf{Gittata}]: 150 metri
	\item[\textbf{Durata}]: Concentrazione, massimo 1 minuto
\end{description}
\end{minipage}}\smallskip

Provochi un disturbo sismico in un punto sul terreno a gittata e che puoi vedere. Per la durata, un intenso tremore scuote il terreno in un cerchio di 30 metri di raggio centrato su quel punto e scuote le creature e le strutture in quell'area che sono a contatto del terreno.Il terreno nell'area diventa terreno difficile. Ogni creatura a terra che si sta concentrando deve effettuare un Tiro Salvezza su Tempra. Se lo fallisce, la sua concentrazione è infranta.

Quando lanci questo incantesimo e alla fine di ogni round che hai speso a concentrarti su di esso, ogni creatura nell'area che si trovi a terra deve effettuare un Tiro Salvezza su Riflessi. Se lo fallisce, la creatura cade prona.

Questo incantesimo ha effetti aggiuntivi a seconda del tipo di terreno nell'area, a discrezione del Narratore. Fenditure. All'inizio del round successivo a quello in cui hai lanciato l'incantesimo si aprono delle fenditure per tutta l'area dell'incantesimo. Un totale di 1d6 fenditure si aprono in punti scelti dal Narratore. Ognuna di esse è profonda 1d10 x 3 metri, larga 3 metri e si estende da un lato dell'area dell'incantesimo all'altro. Una creatura che si trova sul punto in cui si apre una fenditura deve superare un Tiro Salvezza su Riflessi o cadervi dentro. Una creatura che riesca il Tiro Salvezza si sposta sul bordo della fenditura, nel momento in cui questa si apre.

Una fenditura che si apre sotto una struttura la fa crollare immediatamente (vedi sotto). Strutture. Il tremore infligge 50 danni contundenti a qualsiasi struttura in contatto col terreno nell'area quando lanci l'incantesimo e alla fine di ciascuno dei tuoi round fino al termine dell'incantesimo. Se una struttura scende a 0 Punti Ferita, crolla e potrebbe danneggia le creature vicine. Una creatura distante dalla struttura metà della altezza o meno della struttura, deve effettuare un Tiro Salvezza su Riflessi. Se lo fallisce, la creatura subisce 5d6 danni contundenti, cade prona ed è sommersa dalle macerie. Dovrà poi impiegare 2 azioni riuscendo una prova di Atletica DC 20 per liberarsi. Il Narratore può modificare verso l'alto o il basso la DC, a seconda della natura delle macerie. Se supera il Tiro Salvezza, la creatura subisce solo la metà dei danni e non cade né resta sepolta.

\incantesimo{Terreno Illusorio}
\noindent\colorbox{OBSSgold!10}{
\begin{minipage}{0.95\linewidth}
\begin{description}[noitemsep, topsep=0pt, parsep=0pt, partopsep=0pt, leftmargin=0cm, labelwidth=1.3cm]
	\item[\textbf{Lista}]: Illusione
	\item[\textbf{Livello}]: 4, Non Comune
	\item[\textbf{Lancio}]: 10 minuti
	\item[\textbf{Gittata}]: 90 metri
	\item[\textbf{Durata}]: 24 ore
\end{description}
\end{minipage}}\smallskip

Fai sì che un pezzo di terreno naturale a gittata, in un cubo di 150 metri di spigolo, appaia, risuoni e odori come qualche altro tipo di terreno naturale. Di conseguenza, campi aperti o una strada possono essere trasformati in un acquitrino, colline, un crepaccio o qualche altro tipo di terreno difficile o invalicabile. Un laghetto può essere trasformato in una radura erbosa, un precipizio in una lieve pendenza, un burrone cosparso di rocce in una strada ampia e liscia. Le strutture edificate, l'equipaggiamento e le creature all'interno dell'area non mutano d'aspetto.

Le peculiarità tattili del terreno sono immutate, così che le creature che entrano nell'area è probabile che svelino l'illusione. Se al contatto la differenza non è ovvia, una creatura che esamina con cautela l'illusione può tentare una prova di Consapevolezza contro la DC del Tiro Salvezza dei tuoi incantesimi per dubitare di essa. Una creatura che riconosca l'illusione per quello che è, la percepisce come una vaga immagine sovrapposta al terreno.

\incantesimo{Tocco Gelido}
\noindent\colorbox{OBSSgold!10}{
\begin{minipage}{0.95\linewidth}
\begin{description}[noitemsep, topsep=0pt, parsep=0pt, partopsep=0pt, leftmargin=0cm, labelwidth=1.3cm]
	\item[\textbf{Lista}]: Necromanzia
	\item[\textbf{Livello}]: 0, Comune
	\item[\textbf{Lancio}]: 1 Azione
	\item[\textbf{Gittata}]: 36 metri
	\item[\textbf{Durata}]: 1 round
\end{description}
\end{minipage}}\smallskip

Crei una scheletrica mano spettrale nello spazio di una creatura a gittata. Effettua un attacco a distanza con incantesimo contro la creatura, per aggredirla con il gelo della morte. Se colpisci, il bersaglio subisce 1d8 danni da Vuoto, e non può recuperare Punti Ferita fino all'inizio del tuo prossimo round. Fino ad allora, la mano resterà serrata sul bersaglio. Se colpisci un bersaglio non morto, esso avrà anche -1d6 ai Tiri per Colpire contro di te fino alla fine del suo prossimo round.

Puoi aumentare il danno dell'incantesimo di 1d8 quando raggiungi CM 5, CM 11 e CM 17, ma costa 2 Azioni lanciarlo potenziato e 1 Punti Magia, è altresì necessario avere preso Adepto della Magia un numero di volte pari ai potenziamenti che si vogliono applicare.

\textbf{Per ogni due Successo Critico Magico ottenuto} nella Prova di Magia crei una mano scheletrica aggiuntiva che deve attaccare una creatura diversa entro gittata.

\incantesimo{Tocco Vampirico}
\noindent\colorbox{OBSSgold!10}{
\begin{minipage}{0.95\linewidth}
\begin{description}[noitemsep, topsep=0pt, parsep=0pt, partopsep=0pt, leftmargin=0cm, labelwidth=1.3cm]
	\item[\textbf{Lista}]: Necromanzia
	\item[\textbf{Livello}]: 3, Raro
	\item[\textbf{Lancio}]: 2 Azioni
	\item[\textbf{Gittata}]: Personale
	\item[\textbf{Durata}]: 1 minuto
\end{description}
\end{minipage}}\smallskip

Il contatto con la tua mano avvolta dall'ombra può risucchiare la forza vitale altrui per curare le tue ferite. Ogni round puoi effettuare un attacco in mischia, 2 Azioni, con incantesimo contro una creatura a portata. Se colpisci, il bersaglio subisce 3d6 danni da Vuoto e tu recuperi un numero di Punti Ferita pari alla metà del danno da Vuoto che hai inflitto.

Mentre hai questo incantesimo attivo sei considerato Distratto per il lancio di altri incantesimi.

\textbf{Per ogni Successo Critico Magico} ottenuto nella Prova di Magia il danno aumento di 1d8.

\textbf{NOTA}: l'incantesimo è Comune tra i Devoti di Tazher. Un seguace di Tazher sostituisce il d6 di danno con il d8.

\incantesimo{Trama Ipnotica}
\noindent\colorbox{OBSSgold!10}{
\begin{minipage}{0.95\linewidth}
\begin{description}[noitemsep, topsep=0pt, parsep=0pt, partopsep=0pt, leftmargin=0cm, labelwidth=1.3cm]
	\item[\textbf{Lista}]: Illusione
	\item[\textbf{Livello}]: 3, Comune
	\item[\textbf{Lancio}]: 2 Azioni
	\item[\textbf{Gittata}]: 36 metri
	\item[\textbf{Durata}]: 1 minuto
\end{description}
\end{minipage}}\smallskip

Crei a gittata una trama contorta di colori che si muove nell'aria all'interno di un cubo di 9 metri di spigolo. La trama appare per un momento e poi svanisce. Ogni creatura nell'area che veda la trama deve effettuare un Tiro Salvezza su Volontà. Se fallisce il Tiro Salvezza, una creatura rimane affascinata per la durata. Mentre è affascinata da questo incantesimo, la creatura è inabile e ha velocità 0. L'incantesimo termina per la creatura soggetta, qualora questa subisca danni o se qualcuno usa un'Azione per scuoterla dal suo stato confusionale.

\incantesimo{Trasformazione}
\noindent\colorbox{OBSSgold!10}{
\begin{minipage}{0.95\linewidth}
\begin{description}[noitemsep, topsep=0pt, parsep=0pt, partopsep=0pt, leftmargin=0cm, labelwidth=1.3cm]
	\item[\textbf{Lista}]: Trasmutazione
	\item[\textbf{Livello}]: 9, Raro
	\item[\textbf{Lancio}]: 2 Azioni
	\item[\textbf{Gittata}]: Personale
	\item[\textbf{Durata}]: 1 ora
\end{description}
\end{minipage}}\smallskip

Per la durata assumi la forma di una creatura differente. La nuova forma può essere quella di qualsiasi creatura il cui grado di sfida sia pari o inferiore alla tua CM/2+Adepto della magia. La creatura non può essere un costrutto o un non morto e devi averla vista almeno una volta. Ti trasformi in un esemplare medio di quella creatura, uno senza Abilità uniche. Puoi restare nella forma assunta fino al termine dell'incantesimo. Ti ritrasformi automaticamente se cadi privo di sensi, scendi a 0 Punti Ferita o muori. Le tue statistiche di gioco sono rimpiazzate dalle statistiche della creatura scelta, fatta accezione per i tuoi Tratti, e dei tuoi punteggi di Intelligenza, Saggezza e Carisma. Mantieni tutte le tue competenze nelle Abilità e i Tiri Salvezza, oltre a ottenere quelle della creatura. Se la creatura possiede le tue stesse competenze e il bonus indicato nelle sue statistiche è più alto del tuo, usa il bonus della creatura al posto del tuo. Non puoi usare nessuna Azione aggiuntiva o Azione da tana della nuova forma.

Quando ti trasformi mantieni i Punti Ferita. Quando ritorni alla tua forma normale, ritorni al numero di Punti Ferita che avevi prima di trasformarti. Tuttavia, se ti ritrasformi perché sei stato ridotto a 0 Punti Ferita, tutto il danno in eccesso viene riportato alla tua forma originale. A meno che il danno in eccesso non riduca la tua forma normale a 0 Punti Ferita, non cadrai privo di sensi.

Mantieni tutti i benefici di qualsiasi Abilità possedessi, razza, o altra fonte e puoi usarli se la nuova forma è fisicamente capace di farne uso. Tuttavia, non puoi usare nessuno dei tuoi sensi speciali, come la scurovisione, a meno che la nuova forma non possieda anch'essa lo stesso senso. Puoi parlare solo se la creatura è normalmente in grado di parlare.

Quando ti trasformi scegli se il tuo equipaggiamento cade a terra nel tuo spazio, si fonde con la nuova forma o sia indossato da essa. L'equipaggiamento indossato funziona come di norma, ma sta al Narratore decidere se sia comodo per la nuova forma indossare un simile pezzo di equipaggiamento, in base alla taglia e le dimensioni della creatura. Il tuo equipaggiamento non cambia dimensioni né si adatta alla nuova forma, e qualsiasi equipaggiamento che la nuova forma non può indossare deve essere fatto cadere a terra o fondersi con la nuova forma. L'equipaggiamento che si fonde è inefficace.

Nella durata dell'incantesimo, puoi usare due azioni per assumere una forma diversa seguendo le stesse restrizioni e regole della forma originale.

\textbf{NOTA}: devi essere un Devoto di Efrem o Shayalia per imparare questo incantesimo

\incantesimo{Trasformazione Furiosa di Restser}
\noindent\colorbox{OBSSgold!10}{
\begin{minipage}{0.95\linewidth}
\begin{description}[noitemsep, topsep=0pt, parsep=0pt, partopsep=0pt, leftmargin=0cm, labelwidth=1.3cm]
	\item[\textbf{Lista}]: Trasmutazione
	\item[\textbf{Livello}]: 6, Molto Raro
	\item[\textbf{Lancio}]: 2 Azioni
	\item[\textbf{Gittata}]: Personale
	\item[\textbf{Durata}]: 1 round per CM
\end{description}
\end{minipage}}\smallskip

Questo incantesimo permette ad un incantatore di convogliare le sue energie magiche per trasformarsi in un potente combattente.

Fino alla fine della durata dell'incantesimo la Competenza Armi dell'incantatore diviene pari alla sua Competenza Magica.

In base all'arma magica che si tiene in mano al momento dell'incantesimo si diviene competente nella Lista d'Armi in cui appartiene quell'arma, se l'arma è presente in più liste sarà l'incantatore a scegliere la lista. L'incantatore acquisisce le capacità di quella Lista d'Armi come se l'avesse scelta un numero di volte pari al doppio delle volte che ha preso Adepto della Magia.

L'incantatore acquisisce 2 Punti Ferita Temporanei per punto di Competenza Magica posseduto.
Il punteggio non modificati delle caratteristiche fisiche (Forza, Destrezza e Costituzione) se inferiori a 2 diventano 2.

Per tutta la durata dell'incantesimo l'incantatore non è più in grado di lanciare incantesimi.

\incantesimo{Traslazione Arborea}
\noindent\colorbox{OBSSgold!10}{
\begin{minipage}{0.95\linewidth}
\begin{description}[noitemsep, topsep=0pt, parsep=0pt, partopsep=0pt, leftmargin=0cm, labelwidth=1.3cm]
	\item[\textbf{Lista}]: Animali e Piante
	\item[\textbf{Livello}]: 5, Raro
	\item[\textbf{Lancio}]: 2 Azioni
	\item[\textbf{Gittata}]: Personale
	\item[\textbf{Durata}]: massimo 1 minuto
\end{description}
\end{minipage}}\smallskip

Ottieni la capacità di entrare in un albero e muoverti dal suo interno all'interno di un altro albero della stessa specie entro 150 metri. Entrambi gli alberi devono essere vivi e almeno della tua stessa taglia. Ogni Azione di Movimento ti permetti di entrare ed uscire da un nuovo albero.

Apprendi istantaneamente la posizione di tutti gli altri alberi della stessa specie entro 150 metri. Quando esci riappari in un punto a tua scelta entro 1 metro dall'albero di destinazione.

Per la durata dell'incantesimo puoi usare questa capacità di trasporto una volta per round. Devi terminare ogni round al di fuori di un albero.

\incantesimo{Trasporto Vegetale}
\noindent\colorbox{OBSSgold!10}{
\begin{minipage}{0.95\linewidth}
\begin{description}[noitemsep, topsep=0pt, parsep=0pt, partopsep=0pt, leftmargin=0cm, labelwidth=1.3cm]
	\item[\textbf{Lista}]: Animali e Piante
	\item[\textbf{Livello}]: 6, Molto Raro
	\item[\textbf{Lancio}]: 2 Azioni
	\item[\textbf{Gittata}]: 3 metri
	\item[\textbf{Durata}]: 1 round
\end{description}
\end{minipage}}\smallskip

Questo incantesimo crea un legame magico tra un vegetale inanimato di taglia Grande o maggiore a gittata e un altro vegetale, a qualsiasi distanza, sullo stesso piano di esistenza. Devi aver visto o essere entrato in contatto almeno una volta con il vegetale di destinazione. Per la durata dell'incantesimo, qualsiasi creatura può entrare nel vegetale bersaglio e uscire dal vegetale di destinazione usando 1 Azione di Movimento.

\incantesimo{Trucco della Corda}
\noindent\colorbox{OBSSgold!10}{
\begin{minipage}{0.95\linewidth}
\begin{description}[noitemsep, topsep=0pt, parsep=0pt, partopsep=0pt, leftmargin=0cm, labelwidth=1.3cm]
	\item[\textbf{Lista}]: Trasmutazione
	\item[\textbf{Livello}]: 2, Comune
	\item[\textbf{Lancio}]: 1 minuto
	\item[\textbf{Gittata}]: Contatto
	\item[\textbf{Durata}]: 1 ora +1 Turno per CM
\end{description}
\end{minipage}}\smallskip

Entri a contatto con un pezzo di corda lungo fino a 18 metri. Un'estremità della corda si leva nell'aria finché la corda non pende perpendicolare al terreno. All'estremità opposta della corda, un'entrata invisibile si apre su di uno spazio extradimensionale che resta fino al termine dell'incantesimo

Lo spazio extradimensionale può essere raggiunto arrampicandosi fino alla cima della corda (prova di Arrampicarsi DC 15). Lo spazio può contenere fino a 2 creature di taglia Media o inferiore +1 per ogni volta che si è preso Adepto della Magia. La corda può essere trascinata nello spazio, facendola sparire dalla vista di chi è fuori di esso.

Chi si trova al suo interno o sotto l'ingresso può vedere fuori come se vedesse attraverso una finestra di 1 x 1 metro centrata sulla corda. L'incantesimo di Individuazione del Magico permette di vedere l'apertura. Qualsiasi cosa si trovi nello spazio extradimensionale ne cade fuori al termine dell'incantesimo.

Al termine dell'incantesimo la corda usata scompare.

\textbf{Per ogni Successo Critico Magico} ottenuto nella Prova di Magia la durata aumenta di due ore o può contenere un'altra creatura media o più piccola.

\incantesimo{Uno con la pietra}
\noindent\colorbox{OBSSgold!10}{
\begin{minipage}{0.95\linewidth}
\begin{description}[noitemsep, topsep=0pt, parsep=0pt, partopsep=0pt, leftmargin=0cm, labelwidth=1.3cm]
	\item[\textbf{Lista}]: Terra
	\item[\textbf{Livello}]: 3, Comune
	\item[\textbf{Lancio}]: 2 Azioni
	\item[\textbf{Gittata}]: Contatto
	\item[\textbf{Durata}]: 8 ore
\end{description}
\end{minipage}}\smallskip

Entri in un oggetto o superficie di pietra grossa abbastanza da contenere tutto il tuo corpo fondendoti con la pietra assieme a tutto l'equipaggiamento che trasporti per la durata.

Usando il tuo movimento, entri nella pietra in un punto con cui sei in contatto. Non resta nulla della tua presenza che rimanga visibile o altrimenti possa essere individuato da sensi non magici. Mentre sei fuso con la pietra, non puoi vedere ciò che avviene all'esterno, e qualsiasi prova di Consapevolezza che effettui per ascoltare i suoni prodotti fuori da essa è fatta con -1d6. Resti consapevole del passare del tempo e puoi lanciare incantesimi su di te mentre sei fuso con la pietra. Puoi usare il tuo movimento per lasciare la pietra e ricomparire nel punto in cui vi sei entrato, terminando così l'incantesimo. Altrimenti non puoi muoverti.

I danni minori alla pietra non ti danneggiano, ma la sua parziale distruzione o cambio di forma (di modo che tu non vi entri più) ti espellono da essa e ti infliggono 6d6 danni contundenti. La completa distruzione della pietra (o la sua trasmutazione in un'altra sostanza) ti fa espellere e ti infligge 50 danni contundenti. Se vieni espulso, cadi prono in uno spazio non occupato, nel punto più vicino a quello in cui sei entrato nella pietra.

\textbf{Per ogni Successo Critico Magico} ottenuto nella Prova di Magia la durata massima aumenta di 1 ora.

\incantesimo{Unto}
\noindent\colorbox{OBSSgold!10}{
\begin{minipage}{0.95\linewidth}
\begin{description}[noitemsep, topsep=0pt, parsep=0pt, partopsep=0pt, leftmargin=0cm, labelwidth=1.3cm]
	\item[\textbf{Lista}]: Animali e Piante
	\item[\textbf{Livello}]: 1, Comune
	\item[\textbf{Lancio}]: 2 Azioni
	\item[\textbf{Gittata}]: 18 metri
	\item[\textbf{Durata}]: 1 minuto
\end{description}
\end{minipage}}\smallskip

Del grasso scivoloso ricopre il terreno in un quadrato di 3 metri di lato, centrato su di un punto a gittata, e lo trasforma in terreno difficile per la durata dell'incantesimo.

Quando compare il grasso, ciascun bersaglio che si trova in piedi nell'area deve superare un Tiro Salvezza su Riflessi o cadere prono. Una creatura che entra nell'area o termina il suo round lì, deve superare un Tiro Salvezza su Riflessi o cadere prona.

\incantesimo{Vedere l'invisibile}
\noindent\colorbox{OBSSgold!10}{
\begin{minipage}{0.95\linewidth}
\begin{description}[noitemsep, topsep=0pt, parsep=0pt, partopsep=0pt, leftmargin=0cm, labelwidth=1.3cm]
	\item[\textbf{Lista}]: Divinazione
	\item[\textbf{Livello}]: 2, Comune
	\item[\textbf{Lancio}]: 2 Azioni
	\item[\textbf{Gittata}]: Personale
	\item[\textbf{Durata}]: 1 ora
\end{description}
\end{minipage}}\smallskip

Per la durata dell'incantesimo, vedi le creature e gli oggetti invisibili come se fossero visibili, e inoltre puoi vedere nel Piano Etereo. Le creature e gli oggetti eterei ti appaiono spettrali e trasparenti.

\incantesimo{Velocità}
\noindent\colorbox{OBSSgold!10}{
\begin{minipage}{0.95\linewidth}
\begin{description}[noitemsep, topsep=0pt, parsep=0pt, partopsep=0pt, leftmargin=0cm, labelwidth=1.3cm]
	\item[\textbf{Lista}]: Trasmutazione
	\item[\textbf{Livello}]: 3, Non Comune
	\item[\textbf{Lancio}]: 2 Azioni
	\item[\textbf{Gittata}]: 9 metri
	\item[\textbf{Durata}]: 1 minuto
\end{description}
\end{minipage}}\smallskip

Acceleri il metabolismo di massimo 2 più le volte che hai preso Adepto della Magia creature a tua scelta in un raggio di 3 metri a gittata. Fino al termine dell'incantesimo i bersagli possono eseguire una Azione aggiuntiva di Attacco, senza penalità di multiattacco, o di Movimento. L'Azione aggiuntiva può essere parte di un altra Azione.

Questo incantesimo \hyperlink{contrastareincantesimi}{contrasta ed è contrastato} da \hyperlink{lentezza}{Lentezza}.

Quando l'incantesimo termina, i bersagli sono Rallentati 2/1r mentre sono preda di un'improvvisa sonnolenza.

\textbf{Per ogni Successo Critico Magico} ottenuto nella Prova di Magia puoi influenzare una creatura in più.

\textbf{Per ogni tre Successo Critico Magico ottenuto} nella Prova di Magia puoi aumentare di un ulteriore 1 le Azioni a round ad una creatura.

\incantesimo{Ventriloquio}
\noindent\colorbox{OBSSgold!10}{
\begin{minipage}{0.95\linewidth}
\begin{description}[noitemsep, topsep=0pt, parsep=0pt, partopsep=0pt, leftmargin=0cm, labelwidth=1.3cm]
	\item[\textbf{Lista}]: Illusione
	\item[\textbf{Livello}]: 1, Comune
	\item[\textbf{Lancio}]: 1 Azione
	\item[\textbf{Gittata}]: 9 metri
	\item[\textbf{Durata}]: 1 minuto
\end{description}
\end{minipage}}\smallskip

Puoi far sembrare che la tua voce (o qualsiasi suono che puoi normalmente produrre vocalmente) provenga da un altro luogo. Puoi parlare in qualsiasi lingua tu conosca. Chiunque senta il suono può tentare una prova di Consapevolezza contro la tua DC dell'incantesimo. In caso di successo, riconoscono che è illusorio, ma lo sentono comunque. Puoi interromperlo a volontà per la durata, senza azioni richieste.

\textbf{Per ogni Successo Critico Magico} ottenuto nella Prova di Magia puoi allontanare l'origine della voce di altri 9 metri.

\incantesimo{Vigilanza e Interdizione}
\noindent\colorbox{OBSSgold!10}{
\begin{minipage}{0.95\linewidth}
\begin{description}[noitemsep, topsep=0pt, parsep=0pt, partopsep=0pt, leftmargin=0cm, labelwidth=1.3cm]
	\item[\textbf{Lista}]: Abiurazione
	\item[\textbf{Livello}]: 6, Non Comune
	\item[\textbf{Lancio}]: 10 minuti
	\item[\textbf{Gittata}]: Contatto
	\item[\textbf{Durata}]: 24 ore
\end{description}
\end{minipage}}\smallskip

Crei una interdizione che protegge fino a 225 metri quadri di pavimento (un'area quadrata di 15 metri di lato, o cento quadrati di 1 metro di lato o venticinque quadrati di 3 metri di lato). L'area interdetta può essere alta fino a 6 metri, e modellata come preferisci. Puoi interdire diversi piani di una roccaforte dividendo l'area tra di essi, purché tu possa camminare ininterrottamente in ogni area adiacente, mentre lanci l'incantesimo.

Quando lanci questo incantesimo, puoi specificare gli individui che ignorano qualcuno o tutti gli effetti di questo incantesimo. Puoi anche specificare una parola d'ordine che, pronunciata ad alta voce, rende chi la proferisce immune a questi effetti.

Vigilanza e interdizione crea i seguenti effetti all'interno dell'area interdetta.

\begin{itemize}[leftmargin=*] \setlength{\itemsep}{0pt}

	\item \emph{Corridoi}. La nebbia riempie tutti i corridoi interdetti, rendendoli oscurati pesantemente. Inoltre, a ogni intersezione o biforcazione del passaggio che offre una scelta di direzione, c'è una probabilità del 50\% che una creatura, escluso te, creda di stare andando nella direzione opposta a quella che ha scelto.
	\item \emph{Porte}. Tutte le porte nell'area interdetta sono chiuse magicamente, come se fossero sigillate dall'incantesimo Serratura Magica. Inoltre, puoi coprire fino a dieci porte con un'illusione (equivalente della funzione oggetto illusorio dell'incantesimo illusione minore) per farle sembrare delle semplici sezioni di muro.
	\item \emph{Scale}. Ragnatele ricoprono da cima a fondo tutte le scale nell'area interdetta, come per l'incantesimo ragnatela. Questi fili ricrescono in 10 minuti se vengono bruciati o strappati mentre vigilanza e interdizione resta attivo.
\end{itemize}

Altri Incantesimi in Effetto. Puoi piazzare uno dei seguenti effetti magici di tua scelta all'interno dell'area interdetta dell'edificio

\begin{itemize}[leftmargin=*] \setlength{\itemsep}{0pt}
	\item Piazza luci danzanti in quattro corridoi. Puoi indicare un semplice programma che le luci ripeteranno per la durata di vigilanza e interdizione.
	\item Piazza bocca magica in due posti.
	\item Piazza Nebbia Nauseante in due posti. I vapori appaiono nel posto da te indicato; ritornano entro 10 minuti se dispersi dal vento mentre vigilanza e interdizione è ancora attivo.
	\item Piazza una folata di vento costante in un corridoio o stanza.
	\item Piazza una suggestione in un luogo. Seleziona un'area quadrata di 1 metro di lato, e qualsiasi creatura che entra o passa attraverso quell'area riceve mentalmente la suggestione.
\end{itemize}

L'intera area interdetta irradia magia. Un incantesimo dissolvi magie lanciato contro uno specifico effetto, se riesce, rimuove solo quell'effetto Puoi creare una struttura perennemente vigilata e interdetta lanciandovi questo incantesimo ogni giorno per un anno.

\textbf{Se effettui tre critici} la durata è permanente.

\incantesimo{Vigore}
\noindent\colorbox{OBSSgold!10}{
\begin{minipage}{0.95\linewidth}
\begin{description}[noitemsep, topsep=0pt, parsep=0pt, partopsep=0pt, leftmargin=0cm, labelwidth=1.3cm]
	\item[\textbf{Lista}]: Cura
	\item[\textbf{Livello}]: 4, Raro
	\item[\textbf{Lancio}]: 2 Azioni
	\item[\textbf{Gittata}]: Contatto
	\item[\textbf{Durata}]: 1 round per CM
\end{description}
\end{minipage}}\smallskip

La creatura influenzata da questo incantesimo recupera un livello di Affaticamento, acquisisce 3d6 Punti Ferita Temporanei. Può concentrare le energie per effettuare una Azione di Attacco senza penalità di multiattacco o eseguire una Azione di Movimento in più.

\incantesimo{Vincolo di Interdizione}
\noindent\colorbox{OBSSgold!10}{
\begin{minipage}{0.95\linewidth}
\begin{description}[noitemsep, topsep=0pt, parsep=0pt, partopsep=0pt, leftmargin=0cm, labelwidth=1.3cm]
	\item[\textbf{Lista}]: Abiurazione
	\item[\textbf{Livello}]: 2, Comune
	\item[\textbf{Lancio}]: 2 Azioni
	\item[\textbf{Gittata}]: Contatto
	\item[\textbf{Durata}]: 1 ora
\end{description}
\end{minipage}}\smallskip

Lanci l'incantesimo a contatto di una creatura che vuoi proteggere. Crei una connessione mistica tra di te e il bersaglio fino al termine dell'incantesimo. Finché il bersaglio è entro 18 metri da te, ottiene un bonus di +1 alla Difesa e ai Tiri Salvezza e ha resistenza a tutti i danni. Inoltre, ogni volta che il bersaglio subisce danni, tu ne subisci la stessa quantità. L'incantesimo ha fine se scendi a 0 Punti Ferita o tu e il bersaglio vi allontanate più di 18 metri. Ha fine anche se lo lanci di nuovo sulla stessa creatura su cui è già in atto. Puoi interrompere l'incantesimo con un'Azione.

\incantesimo{Visione del Vero}
\noindent\colorbox{OBSSgold!10}{
\begin{minipage}{0.95\linewidth}
\begin{description}[noitemsep, topsep=0pt, parsep=0pt, partopsep=0pt, leftmargin=0cm, labelwidth=1.3cm]
	\item[\textbf{Lista}]: Divinazione
	\item[\textbf{Livello}]: 6, Raro
	\item[\textbf{Lancio}]: 2 Azioni
	\item[\textbf{Gittata}]: Contatto
	\item[\textbf{Durata}]: 1 ora
\end{description}
\end{minipage}}\smallskip

Lanci l'incantesimo a contatto di una creatura consenziente. Il bersaglio riceve la capacità di vedere le cose come sono realmente. Per la durata dell'incantesimo, la creatura ha visione del vero, nota porte segrete nascoste dalla magia e può vedere nel Piano Etereo, fino a una gittata di 36 metri. Vedi anche \hyperlink{cap Visione del Vero}{Visione del Vero} pag. \pageref{cap Visione del Vero}.

\incantesimo{Vita Falsata}
\noindent\colorbox{OBSSgold!10}{
\begin{minipage}{0.95\linewidth}
\begin{description}[noitemsep, topsep=0pt, parsep=0pt, partopsep=0pt, leftmargin=0cm, labelwidth=1.3cm]
	\item[\textbf{Lista}]: Necromanzia
	\item[\textbf{Livello}]: 1, Comune
	\item[\textbf{Lancio}]: 2 Azioni
	\item[\textbf{Gittata}]: Personale
	\item[\textbf{Durata}]: 1 ora
\end{description}
\end{minipage}}\smallskip

Potenziandoti con una parvenza necromantica di vitalità, ottieni 1d4 + 4 Punti Ferita temporanei per la durata.

\textbf{Per ogni Successo Critico Magico} ottenuto nella Prova di Magia ottieni 5 Punti Ferita temporanei.

\incantesimo{Viticci Perforanti}
\noindent\colorbox{OBSSgold!10}{
\begin{minipage}{0.95\linewidth}
\begin{description}[noitemsep, topsep=0pt, parsep=0pt, partopsep=0pt, leftmargin=0cm, labelwidth=1.3cm]
	\item[\textbf{Lista}]: Ammali e Piante
	\item[\textbf{Livello}]: 0, Non Comune
	\item[\textbf{Lancio}]: 1 Azione
	\item[\textbf{Gittata}]: 9 metri
	\item[\textbf{Durata}]: Istantanea
\end{description}
\end{minipage}}\smallskip

Scateni dal palmo della tua mano 1 viticcio puntuto e spinato. Esegui un Tiro per Colpire a distanza con incantesimi sul bersaglio designato.
Se il Tiro per Colpire ha successo il bersaglio subisce 1d4 Punti Ferita da danno da penetrazione.

Ogni Azione che dedichi in più al lancio dell'incantesimo puoi decidere di manifestare un viticcio aggiuntivo, con un proprio Tiro per Colpire, oppure aumentare la portata di 9 metri di un viticcio creato.

Se spendi 1 Punto Magia nel lancio dell'incantesimo il viticcio diventa velenoso e se colpisce causa 2 danni da Veleno aggiuntivi.

\textbf{Per ogni Successo Critico Magico} ottenuto nella Prova di Magia l'incantesimo crea un viticcio aggiuntivo.

\incantesimo{Volare}
\noindent\colorbox{OBSSgold!10}{
\begin{minipage}{0.95\linewidth}
\begin{description}[noitemsep, topsep=0pt, parsep=0pt, partopsep=0pt, leftmargin=0cm, labelwidth=1.3cm]
	\item[\textbf{Lista}]: Aria
	\item[\textbf{Livello}]: 3, Comune
	\item[\textbf{Lancio}]: 2 Azioni
	\item[\textbf{Gittata}]: Contatto
	\item[\textbf{Durata}]: 10 minuti
\end{description}
\end{minipage}}\smallskip

Lanci l'incantesimo a contatto di una creatura consenziente. Per la durata dell'incantesimo, il bersaglio ottiene velocità di volo 18 metri. Quando l'incantesimo ha fine, qualora sia ancora in aria, il bersaglio cade, a meno che non riesca a frenare la discesa.

Lanciare un incantesimo mentre si vola è più complesso, si è Distratti se non si riesce in una prova di Volare a DC 11.

\textbf{Per ogni Successo Critico Magico} ottenuto nella Prova di Magia puoi prendere come bersaglio un'ulteriore creatura oppure aumentare la durata di 10 minuti.

\incantesimo{Scudo Mentale}
\noindent\colorbox{OBSSgold!10}{
\begin{minipage}{0.95\linewidth}
\begin{description}[noitemsep, topsep=0pt, parsep=0pt, partopsep=0pt, leftmargin=0cm, labelwidth=1.3cm]
	\item[\textbf{Lista}]: Abiurazione
	\item[\textbf{Livello}]: 8, Non Comune
	\item[\textbf{Lancio}]: 2 Azioni
	\item[\textbf{Gittata}]: Contatto
	\item[\textbf{Durata}]: 24 ore
\end{description}
\end{minipage}}\smallskip

Fino al termine dell'incantesimo, una creatura consenziente con cui sei in contatto durante il lancio è immune a qualsiasi effetto che ne percepirebbe le emozioni o leggerebbe i pensieri, incantesimi di divinazione e la condizione Affascinato. l'incantesimo nega anche gli incantesimi desiderio e altri incantesimi o effetti di simili potenza impiegati per
influenzare la mente del bersaglio o per ottenere informazioni su di esso.

\textbf{Per ogni Successo Critico Magico} ottenuto nella Prova di Magia la durata raddoppia. Se ottieni tre critici la durata è permanente.

\incantesimo{Zona di Verità}
\noindent\colorbox{OBSSgold!10}{
\begin{minipage}{0.95\linewidth}
\begin{description}[noitemsep, topsep=0pt, parsep=0pt, partopsep=0pt, leftmargin=0cm, labelwidth=1.3cm]
	\item[\textbf{Lista}]: Ammaliamento
	\item[\textbf{Livello}]: 2, Non Comune
	\item[\textbf{Lancio}]: 2 Azioni
	\item[\textbf{Gittata}]: 18 metri
	\item[\textbf{Durata}]: 10 minuti
\end{description}
\end{minipage}}\smallskip

Crei una zona magica che protegge contro i raggiri in una sfera di 3 metri di raggio centrata su di un punto a gittata di tua scelta. Fino al termine dell'incantesimo, una creatura che entra nell'area dell'incantesimo per la prima volta durante un round, o inizia il suo round al suo interno, deve effettuare un Tiro Salvezza su Volontà. Se fallisce il Tiro Salvezza, la creatura non può pronunciare bugie deliberatamente mentre è nel raggio dell'incantesimo. Sei a conoscenza se una creatura ha superato o fallito il Tiro Salvezza. Una creatura soggetta all'incantesimo ne è consapevole e può quindi evitare di rispondere a domande a cui risponderebbe normalmente con una bugia. Questa creatura può dare risposte elusive purché rimanga entro i confini della verità.

%\vspace{2cm}
%\begin{center}
%\includegraphics[width=0.4\linewidth]{immagini/Bocca_della_Verita.png}
%\medskip
%\emph{La Bocca della Verita', Chiesa di Santa Maria in Cosmedin, Roma}
%\end{center}

\bigskip

%\begin{narratore}
%Gli incantesimi elencati sono in parte della 5ed più altre mie proposte e rivisitazioni. Se avete suggerimenti per il Narratore per gestire critici non previsti parlatene con lui! Lo spirito di collaborazione deve essere sempre costruttivo.
%\end{narratore}\end{changemargin}

\end{multicols}

\vfill

\begin{center}
\includegraphics[keepaspectratio,width=0.5\linewidth]{immagini/Goetic_circle_from_The_Lesser_Key_of_Solomon.png}

\medskip

\emph{The Circle of Solomon and Triangle of Solomon from The Goetia. L. W. De Laurence}
\end{center}

\pagebreak

\subsection{Incantesimi per Lista con Livello e Rarita'}\hypertarget{elencoscuole}{}\index{Incantesimi elenco}

A fianco di ogni incantesimo è indicata la Rarità ed il livello dell'incantesimo.

\begin{multicols}{3}

\small{

\setlength{\parindent}{0cm}{

\textbf{Lista dell'Acqua}

\hyperlink{Raggio di Gelo}{Raggio di Gelo}, Comune, 0\\
\hyperlink{Arma Energetica}{Arma Energetica}, Molto Raro, 1\\
\hyperlink{Creare o Distruggere Acqua}{Creare o Distruggere Acqua}, Comune, 1\\
\hyperlink{Cura Ferite}{Cura Ferite}, Comune, 1\\
\hyperlink{Nube di Nebbia}{Nube di Nebbia}, Comune, 1\\
\hyperlink{Freccia Acida di Restser}{Freccia Acida di Restser}, Comune, 2\\
\hyperlink{Camminare sull'Acqua}{Camminare sull'Acqua}, Comune, 3\\
\hyperlink{Nebbia Nauseante}{Nebbia Nauseante}, Non Comune, 3\\
\hyperlink{Respirare Sott'Acqua}{Respirare Sott'Acqua}, Comune, 3\\
\hyperlink{Rimuovi Veleno}{Rimuovi Veleno}, Comune, 3\\
\hyperlink{Tempesta di Nevischio}{Tempesta di Nevischio}, Molto Raro, 3\\
\hyperlink{Controllare Acqua}{Controllare Acqua}, Comune, (4)\\
\hyperlink{Evoca Elementali Minori}{Evoca Elementali Minori}, Non Comune, 4\\
\hyperlink{Scudo di Fuoco}{Scudo di Fuoco}, Non Comune, 4\\
\hyperlink{Tempesta di Ghiaccio}{Tempesta di Ghiaccio}, Non Comune, 4\\
\hyperlink{Cono di Freddo}{Cono di Freddo}, Comune, 5\\
\hyperlink{Evoca Elementale}{Evoca Elementale}, Raro, 5\\
\hyperlink{Nebbia mortale}{Nebbia mortale}, Raro, 5\\
\hyperlink{Muro di Ghiaccio}{Muro di Ghiaccio}, Comune, (6)\\
\hyperlink{Sfera Congelante}{Sfera Congelante}, Raro, 6\\
\hyperlink{Controllare Tempo Atmosferico}{Controllare Tempo Atmosferico}, Raro, 8

\medskip\textbf{Lista dell'Aria}

\hyperlink{Stretta Folgorante}{Stretta Folgorante}, Comune, 0\\
\hyperlink{Arma Energetica}{Arma Energetica}, Molto Raro, 1\\
\hyperlink{Caduta Piuma}{Caduta Piuma}, Comune, 1\\
\hyperlink{Nube di Nebbia}{Nube di Nebbia}, Comune, 1\\
\hyperlink{Onda Tonante}{Onda Tonante}, Comune, 1\\
\hyperlink{Saltare}{Saltare}, Comune, 1\\
\hyperlink{Folata di Vento}{Folata di Vento}, Comune, 2\\
\hyperlink{Levitazione}{Levitazione}, Comune, 2\\
\hyperlink{Polvere luccicante}{Polvere luccicante}, Non Comune, 2\\
\hyperlink{Fulmine}{Fulmine}, Comune, 3\\
\hyperlink{Invocare il Fulmine}{Invocare il Fulmine}, Comune, 3\\
\hyperlink{Muro di Vento}{Muro di Vento}, Non Comune, 3\\
\hyperlink{Nebbia Nauseante}{Nebbia Nauseante}, Non Comune, 3\\
\hyperlink{Respirare Sott'Acqua}{Respirare Sott'Acqua}, Comune, 3\\
\hyperlink{Volare}{Volare}, Comune, 3\\
\hyperlink{Bolla vitale}{Bolla vitale}, Non Comune, 4\\
\hyperlink{Camminare nell'aria}{Camminare nell'aria}, Non Comune, 4\\
\hyperlink{Evoca Elementali Minori}{Evoca Elementali Minori}, Non Comune, 4\\
\hyperlink{Tempesta di Ghiaccio}{Tempesta di Ghiaccio}, Non Comune, 4\\
\hyperlink{Evoca Elementale}{Evoca Elementale}, Raro, 5\\
\hyperlink{Camminare nel Vento}{Camminare nel Vento}, Non Comune, 6\\
\hyperlink{Fulmine a catena}{Fulmine a catena}, Raro, 6\\
\hyperlink{Controllare Tempo Atmosferico}{Controllare Tempo Atmosferico}, Molto Raro, 8\\
\hyperlink{Muro Prismatico}{Muro Prismatico}, Raro, (9)

\medskip\textbf{Lista del Fuoco}

\hyperlink{Produrre Fiamma}{Produrre Fiamma}, Comune, 0\\
\hyperlink{Arma Energetica}{Arma Energetica}, Molto Raro, 1\\
\hyperlink{Dardo di Fuoco}{Dardo di Fuoco}, Comune, 1\\
\hyperlink{Onda rovente}{Onda rovente}, Comune, 1\\
\hyperlink{Lama Infuocata}{Lama Infuocata}, Comune, 2\\
\hyperlink{Lanciafiamme}{Lanciafiamme}, Raro, 2\\
\hyperlink{Piroesperto}{Piroesperto}, Non Comune, 2\\
\hyperlink{Polvere luccicante}{Polvere luccicante}, Non Comune, 2\\
\hyperlink{Raggio Rovente}{Raggio Rovente}, Comune, 2\\
\hyperlink{Riscaldare il Metallo}{Riscaldare il Metallo}, Non Comune, 2\\
\hyperlink{Sfera Infuocata}{Sfera Infuocata}, Comune, 2\\
\hyperlink{Gragnola di Ghiande Infuocate di Kyrin}{Gragnola di Ghiande Infuocate di Kyrin}, Raro, 3\\
\hyperlink{Benedizione di Cattalm}{Benedizione di Cattalm}, Molto Raro, 3\\
\hyperlink{Palla di Fuoco}{Palla di Fuoco}, Comune, 3\\
\hyperlink{Evoca Elementali Minori}{Evoca Elementali Minori}, Non Comune, 4\\
\hyperlink{Muro di Fuoco}{Muro di Fuoco}, Non Comune, 4\\
\hyperlink{Scudo di Fuoco}{Scudo di Fuoco}, Non Comune, 4\\
\hyperlink{Colpo Infuocato}{Colpo Infuocato}, Comune, 5\\
\hyperlink{Evoca Elementale}{Evoca Elementale}, Raro, 5\\
\hyperlink{Palla di Fuoco Ritardata}{Palla di Fuoco Ritardata}, Raro, 7\\
\hyperlink{Tempesta di Fuoco}{Tempesta di Fuoco}, Raro, 7\\
\hyperlink{Nube Incendiaria}{Nube Incendiaria}, Raro, 8\\
\hyperlink{Pioggia di Meteore}{Pioggia di Meteore}, Leggendario, 9

\medskip\textbf{Lista della Terra}

\hyperlink{Riparare}{Riparare}, Comune, 0\\
\hyperlink{Arma Energetica}{Arma Energetica}, Molto Raro, 1\\
\hyperlink{Palla di fango di Eithne}{Palla di fango di Eithne}, Non Comune, 1\\
\hyperlink{Lettura della terra di Kyrin}{Lettura della terra di Kyrin}, Non Comune, 2\\
\hyperlink{Passare Senza Tracce}{Passare Senza Tracce}, Comune, 2\\
\hyperlink{Uno con la pietra}{Uno con la pietra}, Comune, 3\\
\hyperlink{Freccia Acida di Restser}{Freccia Acida di Restser}, Comune, 2\\
\hyperlink{Gragnola di Limoni di Kyrin}{Gragnola di Limoni di Kyrin}, Molto Raro, 2\\
\hyperlink{Succo concentrato di Ribes di Kyrin}{Succo concentrato di Ribes di Kyrin}, Non Comune, 2\\
\hyperlink{Evoca Elementali Minori}{Evoca Elementali Minori}, Non Comune, 4\\
\hyperlink{Pelle di Pietra}{Pelle di Pietra}, Non Comune, 4\\
\hyperlink{Scolpire Pietra}{Scolpire Pietra}, Comune, 4\\
\hyperlink{Evoca Elementale}{Evoca Elementale}, Raro, 5\\
\hyperlink{Muro di Pietra}{Muro di Pietra}, Comune, 5\\
\hyperlink{Passa Porta}{Passa Porta}, Non Comune, 5\\
\hyperlink{Pietra in Fango - Fango in Pietra}{Pietra in Fango - Fango in Pietra}, Non Comune - Molto Raro, 5\\
\hyperlink{Carne in Pietra - Pietra in Carne}{Carne in Pietra - Pietra in Carne}, Non Comune - Raro, 6\\
\hyperlink{Muovere il Terreno}{Muovere il Terreno}, Non Comune, 6\\
\hyperlink{Pietre Parlanti}{Pietre Parlanti}, Raro, 6\\
\hyperlink{Statua}{Statua}, Raro, 7\\
\hyperlink{Terremoto}{Terremoto}, Molto Raro, 8\\
\hyperlink{Pioggia di Meteore}{Pioggia di Meteore}, Leggendario, 9

\medskip\textbf{Abiurazione}

\hyperlink{Resistenza}{Resistenza}, Comune, 0\\
\hyperlink{Allarme}{Allarme}, Comune, 1\\
\hyperlink{Armatura Magica}{Armatura Magica}, Non Comune, 1\\
\hyperlink{Protezione dall'Energia minore}{Protezione dall'Energia minore}, Raro, 1\\
\hyperlink{Santuario}{Santuario}, Comune, 1\\
\hyperlink{Rimuovi Paura}{Rimuovi Paura}, Comune, 1\\
\hyperlink{Chiudi Portale}{Chiudi Portale}, Raro, 2\\
\hyperlink{Protezione dai Veleni}{Protezione dai Veleni}, Non Comune, 2\\
\hyperlink{Serratura Magica}{Serratura Magica}, Comune, 2\\
\hyperlink{Vincolo di Interdizione}{Vincolo di Interdizione}, Comune, 2\\
\hyperlink{Anti-Individuazione}{Anti-Individuazione}, Non Comune, 3\\
\hyperlink{Cerchio Magico}{Cerchio Magico}, Comune, 3\\
\hyperlink{Controincantesimo}{Controincantesimo}, Comune, 3\\
\hyperlink{Dissolvi Magie}{Dissolvi Magie}, Comune, 3\\
\hyperlink{Glifo di Interdizione}{Glifo di Interdizione}, Comune, 3\\
\hyperlink{Protezione dall'Energia}{Protezione dall'Energia}, Comune, 3\\
\hyperlink{Rimuovi Maledizione}{Rimuovi Maledizione}, Comune, 3\\
\hyperlink{Bolla vitale}{Bolla vitale}, Non Comune, 4\\
\hyperlink{Esilio}{Esilio}, Comune, 4\\
\hyperlink{Libertà di Movimento}{Libertà di Movimento}, Comune, 4\\
\hyperlink{Santuario Privato}{Santuario Privato}, Molto Raro, 4\\
\hyperlink{Dissolvi Magie Avanzato}{Dissolvi Magie Avanzato}, Raro, 5\\
\hyperlink{Globo di Invulnerabilità}{Globo di Invulnerabilità}, Comune, 6\\
\hyperlink{Proibizione}{Proibizione}, Non Comune, 6\\
\hyperlink{Vigilanza e Interdizione}{Vigilanza e Interdizione}, Non Comune, 6\\
\hyperlink{Simbolo}{Simbolo}, Non Comune, 7\\
\hyperlink{Aura Sacra}{Aura Sacra}, Comune, 8\\
\hyperlink{Campo Anti-Magia}{Campo Anti-Magia}, Raro, 8\\
\hyperlink{Scudo Mentale}{Scudo Mentale}, Non Comune, 8\\
\hyperlink{Imprigionare}{Imprigionare}, Raro, 9

\medskip\textbf{Animali e Piante}

\hyperlink{Randello Incantato}{Randello Incantato}, Comune, 0\\
\hyperlink{Spruzzo Velenoso}{Spruzzo Velenoso}, Non Comune, 0\\
\hyperlink{Viticci Perforanti}{Viticci Perforanti}, Non Comune, 0\\
\hyperlink{Intralciare}{Intralciare}, Comune, 1\\
\hyperlink{Parlare con gli Animali}{Parlare con gli Animali}, Comune, 1\\
\hyperlink{Purificare Cibo e Bevande}{Purificare Cibo e Bevande}, Comune, 1\\
\hyperlink{Unto}{Unto}, Comune, 1\\
\hyperlink{Amicizia con gli Animali}{Amicizia con gli Animali}, Non Comune, 1\\
\hyperlink{Animale Messaggero}{Animale Messaggero}, Comune, 2\\
\hyperlink{Bacche Benefiche}{Bacche Benefiche}, Comune, 2\\
\hyperlink{Crescita di Spuntoni}{Crescita di Spuntoni}, Comune, 2\\
\hyperlink{Evoca Cavalcatura}{Evoca Cavalcatura}, Comune, 2\\
\hyperlink{Gragnola di Ghiande di Kyrin}{Gragnola di Ghiande di Kyrin}, Non Comune, 2\\
\hyperlink{Gragnola di Limoni di Kyrin}{Gragnola di Limoni di Kyrin}, Molto Raro, 2\\
\hyperlink{Localizza Animali e Piante}{Localizza Animali e Piante}, Non Comune, 2\\
\hyperlink{Movimenti del Ragno}{Movimenti del Ragno}, Non Comune, 2\\
\hyperlink{Passare Senza Tracce}{Passare Senza Tracce}, Comune, 2\\
\hyperlink{Pelle di Corteccia}{Pelle di Corteccia}, Comune, 2\\
\hyperlink{Ragnatela}{Ragnatela}, Comune, 2\\
\hyperlink{Succo concentrato di Ribes di Kyrin}{Succo concentrato di Ribes di Kyrin}, Non Comune, 2\\
\hyperlink{Bastoni in Serpenti}{Bastoni in Serpenti}, Non Comune, 3\\
\hyperlink{Crescita Vegetale}{Crescita Vegetale}, Non Comune, 3\\
\hyperlink{Evoca Animali}{Evoca Animali}, Non Comune, 3\\
\hyperlink{Gragnola di Ghiande Infuocate di Kyrin}{Gragnola di Ghiande Infuocate di Kyrin}, Raro, 3\\
\hyperlink{Parlare con le Piante}{Parlare con le Piante}, Raro, 3\\
\hyperlink{Dominare Bestie}{Dominare Bestie}, Comune, 4\\
\hyperlink{Insetto Gigante}{Insetto Gigante}, Non Comune, 4\\
\hyperlink{Localizza Creatura}{Localizza Creatura}, Comune, 4\\
\hyperlink{Metamorfosi}{Metamorfosi}, Comune, 4\\
\hyperlink{Gragnola di Marroni di Kyrin}{Gragnola di Marroni di Kyrin}, Molto Raro, 5\\
\hyperlink{Guscio Anti-Vita}{Guscio Anti-Vita}, Non Comune, 5\\
\hyperlink{Piaga degli Insetti}{Piaga degli Insetti}, Raro, 5\\
\hyperlink{Reincarnazione}{Reincarnazione}, Raro, 5\\
\hyperlink{Risveglio}{Risveglio}, Raro, 5\\
\hyperlink{Traslazione Arborea}{Traslazione Arborea}, Raro, 5\\
\hyperlink{Barriera Antianimali}{Barriera Antianimali}, Raro, 5\\
\hyperlink{Parlare con le Creature}{Parlare con le Creature}, Raro, 6\\
\hyperlink{Muro di Spine}{Muro di Spine}, Non Comune, 6\\
\hyperlink{Trasporto Vegetale}{Trasporto Vegetale}, Molto Raro, 6\\
\hyperlink{Benedizioni di Efrem}{Benedizioni di Efrem}, Raro, 8\\
\hyperlink{Metamorfosi Pura}{Metamorfosi Pura}, Raro, 9

\medskip\textbf{Ammaliamento}

\hyperlink{Beffa Crudele}{Beffa Crudele}, Comune, 0\\
\hyperlink{Dito}{Dito}, Raro, 0\\
\hyperlink{Charme su Persone}{Charme su Persone}, Comune, 1\\
\hyperlink{Comando}{Comando}, Comune, 1\\
\hyperlink{Eroismo}{Eroismo}, Non Comune, 1\\
\hyperlink{Risata Incontenibile}{Risata Incontenibile}, Non Comune, 1\\
\hyperlink{Sonno}{Sonno}, Comune, 1\\
\hyperlink{Anatema}{Anatema}, Comune, 1\\
\hyperlink{Blocca Persona}{Blocca Persona}, Comune, 2\\
\hyperlink{Calmare Emozioni}{Calmare Emozioni}, Comune, 2\\
\hyperlink{Estasiare}{Estasiare}, Comune, 2\\
\hyperlink{Sonnellino}{Sonnellino}, Leggendario, 2\\
\hyperlink{Suggestione}{Suggestione}, Raro, 2\\
\hyperlink{Zona di Verità}{Zona di Verità}, Non Comune, 2\\
\hyperlink{Benedizione di Cattalm}{Benedizione di Cattalm}, Molto Raro, 3\\
\hyperlink{Blocca Persona Avanzato}{Blocca Persona Avanzato}, Non Comune, 4\\
\hyperlink{Compulsione}{Compulsione}, Non Comune, 4\\
\hyperlink{Confusione}{Confusione}, Comune, 4\\
\hyperlink{Dominare Bestie}{Dominare Bestie}, Molto Raro, 4\\
\hyperlink{Costrizione}{Costrizione}, Raro, 5\\
\hyperlink{Dominare Persone}{Dominare Persone}, Non Comune, 5\\
\hyperlink{Modificare Memoria}{Modificare Memoria}, Molto Raro, 5\\
\hyperlink{Suggestione di Massa}{Suggestione di Massa}, Molto Raro, 6\\
\hyperlink{Antipatia/Simpatia}{Antipatia/Simpatia}, Raro, 8\\
\hyperlink{Confusione Contagiosa}{Confusione Contagiosa}, Molto Raro, 8\\
\hyperlink{Danza Irresistibile}{Danza Irresistibile}, Leggendario, 8\\
\hyperlink{Dominare Mostri}{Dominare Mostri}, Non Comune, 8\\
\hyperlink{Parola del Potere Stordire}{Parola del Potere Stordire}, Non Comune, 8\\
\hyperlink{Regressione Mentale}{Regressione Mentale}, Raro, 8\\
\hyperlink{Parola del Potere Uccidere}{Parola del Potere Uccidere}, Raro, 9

\medskip\textbf{Cura}

\hyperlink{Cura Ferite}{Cura Ferite}, Comune, 1\\
\hyperlink{Preghiera di Guarigione}{Preghiera di Guarigione}, Comune, 2\\
\hyperlink{Ristorare Inferiore}{Ristorare Inferiore}, Comune, 2\\
\hyperlink{Aiuto}{Aiuto}, Non Comune, 2\\
\hyperlink{Rimuovi Malattia}{Rimuovi Malattia}, Comune, 2\\
\hyperlink{Benedizione della Vita}{Benedizione della Vita}, Raro, 3\\
\hyperlink{Distruggere nonmorto}{Distruggere Nonmorto}, Non Comune, 3\\
\hyperlink{Rimuovi Veleno}{Rimuovi Veleno}, Comune, 3\\
\hyperlink{Rinascita}{Rinascita}, Molto Raro, 3\\
\hyperlink{Profumo di Atherim}{Profumo di Atherim}, Molto Raro, 3\\
\hyperlink{Vigore}{Vigore}, Raro, 4\\
\hyperlink{Ristorare Superiore}{Ristorare Superiore}, Non Comune, 5\\
\hyperlink{Guarigione}{Guarigione}, Raro, 6\\
\hyperlink{Rigenerazione}{Rigenerazione}, Leggendario, 7\\
\hyperlink{Guarigione di Massa}{Guarigione di Massa}, Leggendario, 9

\medskip\textbf{Divinazione}

\hyperlink{Colpo Accurato}{Colpo Accurato}, Comune, 0\\
\hyperlink{Comprensione dei Linguaggi}{Comprensione dei Linguaggi}, Comune, 1\\
\hyperlink{Conoscere i Tratti}{Conoscere i Tratti}, Leggendario, 1\\
\hyperlink{Guida}{Guida}, Comune, 1\\
\hyperlink{Comprensione degli Scritti}{Comprensione degli Scritti}, Non Comune, 2\\
\hyperlink{Individuazione dei Pensieri}{Individuazione dei Pensieri}, Raro, 2\\
\hyperlink{Individuazione delle Malattie e dei Veleni}{Individuazione delle Malattie e dei Veleni}, Non Comune, 2\\
\hyperlink{Localizza Oggetto}{Localizza Oggetto}, Comune, 2\\
\hyperlink{Scopri Piante}{Scopri Piante}, Non Comune, 2\\
\hyperlink{Presagio}{Presagio}, Comune, 2\\
\hyperlink{Scopri Trappole}{Scopri Trappole}, Comune, 2\\
\hyperlink{Vedere l'invisibile}{Vedere l'invisibile}, Comune, 2\\
\hyperlink{Lingue}{Lingue}, Comune, 3\\
\hyperlink{Chiaroveggenza}{Chiaroveggenza}, Comune, 3\\
\hyperlink{Occhio Arcano}{Occhio Arcano}, Comune, 4\\
\hyperlink{Comunione}{Comunione}, Raro, 5\\
\hyperlink{Comunione con la Natura}{Comunione con la Natura}, Molto Raro, 5\\
\hyperlink{Conoscenza delle Leggende}{Conoscenza delle Leggende}, Comune, 5\\
\hyperlink{Legame Telepatico}{Legame Telepatico}, Raro, 5\\
\hyperlink{Scrutare}{Scrutare}, Raro, 5\\
\hyperlink{Divinazione}{Divinazione}, Comune, 6\\
\hyperlink{Parlare con le Creature}{Parlare con le Creature}, Raro, 6\\
\hyperlink{Pietre Parlanti}{Pietre Parlanti}, Raro, 6\\
\hyperlink{Scopri il Percorso}{Scopri il Percorso}, Non Comune, 6\\
\hyperlink{Visione del Vero}{Visione del Vero}, Raro, 6\\
\hyperlink{Previsione}{Previsione}, Non Comune, 9

\medskip\textbf{Evocazione}

\hyperlink{Creare Birra}{Creare Birra}, Raro, 0\\
\hyperlink{Fiotto Acido}{Fiotto Acido}, Comune, 0\\
\hyperlink{Mano Magica}{Mano Magica}, Comune, 0\\
\hyperlink{Cuoco Invisibile}{Cuoco Invisibile}, Comune, 1\\
\hyperlink{Disco Fluttuante}{Disco Fluttuante}, Comune, 1\\
\hyperlink{Schiaffo di Cattalm}{Schiaffo di Cattalm}, Non Comune, 1\\
\hyperlink{Servitore Invisibile}{Servitore Invisibile}, Comune, 1\\
\hyperlink{Passo Velato}{Passo Velato}, Non Comune, 2\\
\hyperlink{Creare Cibo e Acqua}{Creare Cibo e Acqua}, Comune, 3\\
\hyperlink{Porta Dimensionale}{Porta Dimensionale}, Comune, 4\\
\hyperlink{Scrigno Segreto}{Scrigno Segreto}, Raro, 4\\
\hyperlink{Segugio Fedele}{Segugio Fedele}, Raro, 4\\
\hyperlink{Tentacoli Neri}{Tentacoli Neri}, Non Comune, 4\\
\hyperlink{Cerchio di Teletrasporto}{Cerchio di Teletrasporto}, Non Comune, 5\\
\hyperlink{Evocazioni Istantanee}{Evocazioni Istantanee}, Raro, 6\\
\hyperlink{Parola del Ritiro}{Parola del Ritiro}, Raro, 6\\
\hyperlink{Desiderio limitato}{Desiderio limitato}, Molto Raro, 7\\
\hyperlink{Reggia Meravigliosa}{Reggia Meravigliosa}, Leggendario, 7\\
\hyperlink{Teletrasporto}{Teletrasporto}, Comune, 7\\
\hyperlink{Labirinto}{Labirinto}, Raro, 8\\
\hyperlink{Desiderio}{Desiderio}, Non Comune, 9

\medskip\textbf{Illusione}

\hyperlink{Camuffare Sé Stesso}{Camuffare Sé Stesso}, Comune, 1\\
\hyperlink{Immagine Silenziosa}{Immagine Silenziosa}, Comune, 1\\
\hyperlink{Scritto Illusorio}{Scritto Illusorio}, Comune, 1\\
\hyperlink{Spruzzo Colorato}{Spruzzo Colorato}, Comune, 1\\
\hyperlink{Ventriloquio}{Ventriloquio}, Comune, 1\\
\hyperlink{Aura Magica dell'Arcanista}{Aura Magica dell'Arcanista}, Non Comune, 2\\
\hyperlink{Bocca Magica}{Bocca Magica}, Comune, 2\\
\hyperlink{Immagine Speculare}{Immagine Speculare}, Comune, 2\\
\hyperlink{Invisibilità}{Invisibilità}, Comune, 2\\
\hyperlink{Lacrima di Laydel}{Lacrima di Laydel}, Molto Raro, 2\\
\hyperlink{Sfocatura}{Sfocatura}, Comune, 2\\
\hyperlink{Silenzio}{Silenzio}, Comune, 2\\
\hyperlink{Cerchio d'Invisibilità}{Cerchio d'Invisibilità}, Non Comune, 3\\
\hyperlink{Destriero Fantasma}{Destriero Fantasma}, Comune, 3\\
\hyperlink{Immagine Maggiore}{Immagine Maggiore}, Comune, 3\\
\hyperlink{Paura}{Paura}, Non Comune, 3\\
\hyperlink{Trama Ipnotica}{Trama Ipnotica}, Comune, 3\\
\hyperlink{Invisibilità Superiore}{Invisibilità Superiore}, Non Comune, 4\\
\hyperlink{Terreno Illusorio}{Terreno Illusorio}, Non Comune, 4\\
\hyperlink{Allucinazione Mortale}{Allucinazione Mortale}, Non Comune, 4\\
\hyperlink{Creazione}{Creazione}, Raro, 5\\
\hyperlink{Fuorviare}{Fuorviare}, Non Comune, 5\\
\hyperlink{Sembrare}{Sembrare}, Non Comune, 5\\
\hyperlink{Sogno}{Sogno}, Non Comune, 5\\
\hyperlink{Illusione Programmata}{Illusione Programmata}, Non Comune, 6\\
\hyperlink{Immagine Proiettata}{Immagine Proiettata}, Non Comune, 7\\
\hyperlink{Miraggio Arcano}{Miraggio Arcano}, Raro, 7\\
\hyperlink{Fatale}{Fatale}, Raro, 9

\medskip\textbf{Invocazione}

\hyperlink{Colpo Fiammeggiante}{Colpo Fiammeggiante}, Raro, 1\\
\hyperlink{Dardo Tracciante}{Dardo Tracciante}, Non Comune, 1\\
\hyperlink{Dardo occulto}{Dardo occulto}, Comune, 1\\
\hyperlink{Favore Divino}{Favore Divino}, Non Comune, 1\\
\hyperlink{Luci Danzanti}{Luci Danzanti}, Non Comune, 1\\
\hyperlink{Luminescenza}{Luminescenza}, Non Comune, 1\\
\hyperlink{Oscurità}{Oscurità}, Comune, 1\\
\hyperlink{Benedizione Superiore}{Benedizione Superiore}, Non Comune, 2\\
\hyperlink{Frantumare}{Frantumare}, Comune, 2\\
\hyperlink{Punizione Marchiante}{Punizione Marchiante}, Comune, 2\\
\hyperlink{Arma Spirituale}{Arma Spirituale}, Comune, 2\\
\hyperlink{Colpo Luccicante}{Colpo Luccicante}, Non Comune, 2\\
\hyperlink{Creare Fossa}{Creare Fossa}, Non Comune, 2\\
\hyperlink{Benedizione Suprema}{Benedizione Suprema}, Raro, 3\\
\hyperlink{Capanna}{Capanna}, Non Comune, 3\\
\hyperlink{Colpo Accecante}{Colpo Accecante}, 3\\
\hyperlink{Inviare}{Inviare}, Comune, 3\\
\hyperlink{Luce Diurna}{Luce Diurna}, Comune, 3\\
\hyperlink{Preghiera}{Preghiera}, Non Comune, 3\\
\hyperlink{Mano Arcana}{Mano Arcana}, Non Comune, 5\\
\hyperlink{Muro di Forza}{Muro di Forza}, Comune, 5\\
\hyperlink{Santificare}{Santificare}, Raro, 5\\
\hyperlink{Bagliore Solare}{Bagliore Solare}, Non Comune, 6\\
\hyperlink{Banchetto degli Eroi}{Banchetto degli Eroi}, Non Comune, 6\\
\hyperlink{Barriera di Lame}{Barriera di Lame}, Comune, 6\\
\hyperlink{Cerchio di Morte}{Cerchio di Morte}, Molto Raro, 6\\
\hyperlink{Contingenza}{Contingenza}, Comune, 6\\
\hyperlink{Parola Divina}{Parola Divina}, Molto Raro, 7\\
\hyperlink{Spada Arcana}{Spada Arcana}, Raro, 7\\
\hyperlink{Spruzzo Prismatico}{Spruzzo Prismatico}, Raro, 7\\
\hyperlink{Esplosione Solare}{Esplosione Solare}, Raro, 8\\
\hyperlink{Gabbia di Forza}{Gabbia di Forza}, Raro, 8\\

\medskip\textbf{Necromanzia}

\hyperlink{Tocco Gelido}{Tocco Gelido}, Comune, 0\\
\hyperlink{Vita Falsata}{Vita Falsata}, Comune, 1\\
\hyperlink{Grido di dolore}{Grido di dolore}, Raro, 1\\
\hyperlink{Raggio di Indebolimento}{Raggio di Indebolimento}, Comune, 1\\
\hyperlink{Cecità/Sordità}{Cecità/Sordità}, Comune, 2\\
\hyperlink{Infliggi Ferite}{Infliggi Ferite}, Comune, 2\\
\hyperlink{Raggio mortale}{Raggio mortale}, Raro, 2\\
\hyperlink{Riposo Inviolato}{Riposo Inviolato}, Non Comune, 2\\
\hyperlink{Aiuto}{Aiuto}, Non Comune, 2\\
\hyperlink{Animare Morti}{Animare Morti}, Comune, 3\\
\hyperlink{Benedizione di Ledyal}{Benedizione di Ledyal}, Molto Raro, 3\\
\hyperlink{Cecità/Sordità Avanzata}{Cecità/Sordità Avanzata}, Non Comune, 3\\
\hyperlink{Parlare con i Morti}{Parlare con i Morti}, Raro, 3\\
\hyperlink{Morte Apparente}{Morte Apparente}, Non Comune, 3\\
\hyperlink{Rinascita}{Rinascita}, Molto Raro, 3\\
\hyperlink{Scagliare Maledizione}{Scagliare Maledizione}, Non Comune, 3\\
\hyperlink{Tocco Vampirico}{Tocco Vampirico}, Raro, 3\\
\hyperlink{Inaridire}{Inaridire}, Non Comune, 4\\
\hyperlink{Interdizione alla Morte}{Interdizione alla Morte}, Non Comune, 4\\
\hyperlink{Contagio}{Contagio}, Non Comune, 5\\
\hyperlink{Creare Non Morti}{Creare Non Morti}, Non Comune, 6\\
\hyperlink{Dito della Morte}{Dito della Morte}, Raro, 6\\
\hyperlink{Ferire}{Ferire}, Non Comune, 6\\
\hyperlink{Giara Magica}{Giara Magica}, Molto Raro, 6\\
\hyperlink{Sguardo Penetrante}{Sguardo Penetrante}, Molto Raro, 6

\medskip\textbf{Trasmutazione}

\hyperlink{Messaggio}{Messaggio}, Comune, 0\\
\hyperlink{Passo Veloce}{Passo Veloce}, Molto Raro, 1\\
\hyperlink{Ritirata Rapida}{Ritirata Rapida}, Non Comune, 1\\
\hyperlink{Alterare Sé Stesso}{Alterare Sé Stesso}, Non Comune, 1\\
\hyperlink{Arma Magica}{Arma Magica}, Comune, 2\\
\hyperlink{Caratteristica Potenziata}{Caratteristica Potenziata}, Comune, 2\\
\hyperlink{Goffaggine}{Goffaggine}, Raro, 2\\
\hyperlink{Ingrandire/Ridurre}{Ingrandire/Ridurre}, Comune, 2\\
\hyperlink{Scassinare}{Scassinare}, Comune, 2\\
\hyperlink{Scurovisione}{Scurovisione}, Comune, 2\\
\hyperlink{Trucco della Corda}{Trucco della Corda}, Comune, 2\\
\hyperlink{Forma Gassosa}{Forma Gassosa}, Non Comune, 3\\
\hyperlink{Intermittenza}{Intermittenza}, Non Comune, 3\\
\hyperlink{lentezza}{Lentezza}, Non Comune, 3\\
\hyperlink{Velocità}{Velocità}, Non Comune, 3\\
\hyperlink{Fabbricare}{Fabbricare}, Comune, 4\\
\hyperlink{Animare Oggetti}{Animare Oggetti}, Comune, 5\\
\hyperlink{Telecinesi}{Telecinesi}, Non Comune, 5\\
\hyperlink{Disintegrazione}{Disintegrazione}, Non Comune, 6\\
\hyperlink{Trasformazione Furiosa di Restser}{Trasformazione Furiosa di Restser}, Molto Raro, 6\\
\hyperlink{Celare}{Celare}, Raro, 7\\
\hyperlink{Forma Eterea}{Forma Eterea}, Raro, 7\\
\hyperlink{Inversione della Gravità}{Inversione della Gravità}, Raro, 7\\
\hyperlink{Statua}{Statua}, Raro, 7\\
\hyperlink{Loquacità}{Loquacità}, Raro, 8\\
\hyperlink{Fermare il Tempo}{Fermare il Tempo}, Molto Raro, 9\\
\hyperlink{Trasformazione}{Trasformazione}, Raro, 9

\medskip\textbf{Universale}

\hyperlink{Fiamma Sacra}{Fiamma Sacra}, Comune, 0\\
\hyperlink{Lacrima di Ljust}{Lacrima di Ljust}, Non Comune, 0\\
\hyperlink{Marchio Magico}{Marchio Magico}, Comune, 0\\
\hyperlink{Prestidigitazione}{Prestidigitazione}, Comune, 0\\
\hyperlink{Scudo}{Scudo}, Comune, 0\\
\hyperlink{Taumaturgia}{Taumaturgia}, Non Comune, 0\\
\hyperlink{Artificio Druidico}{Artificio Druidico}, Non Comune, 0\\
\hyperlink{Benedizione}{Benedizione}, Comune, 1\\
\hyperlink{Dardo arcano}{Dardo arcano}, Comune, 1\\
\hyperlink{Identificare}{Identificare}, Comune, 1\\
\hyperlink{Illusione Minore}{Illusione Minore}, Comune, 1\\
\hyperlink{Individuazione del Magico}{Individuazione del Magico}, Comune, 1\\
\hyperlink{Lettura del Magico}{Lettura del Magico}, Comune, 1\\
\hyperlink{Luce}{Luce}, Comune, 1\\
\hyperlink{Scagliare Maledizione Minore}{Scagliare Maledizione Minore}, Comune, 1\\
\hyperlink{Benedici Acqua}{Benedici Acqua}, Comune, 2\\
\hyperlink{Fiamma Perenne}{Fiamma Perenne}, Leggendario, 2
}}

\end{multicols}

\subsection{Incantesimi per Livello}\hypertarget{elencoinc}{}\index{Incantesimi per livello}

Sono elencati gli incantesimi in ordine per livello e alfabetico. Vedi Capitolo \hyperlink{CPergamene}{Generazione Oggetti Magici} (pag. \pageref{CPergamene}) per generare Tomi casuali.

\begin{multicols}{3}

\small{

\setlength{\parindent}{0cm}{

\textbf{Livello 0 - Trucchetti} \newcounter{inclvzero}

\stepcounter{inclvzero}\hyperlink{Artificio Druidico}{Artificio Druidico}, Non Comune, 0\\
\stepcounter{inclvzero}\hyperlink{Beffa Crudele}{Beffa Crudele}, Comune, 0\\
\stepcounter{inclvzero}\hyperlink{Colpo Accecante}{Colpo Accecante}, 3\\
\stepcounter{inclvzero}\hyperlink{Colpo Accurato}{Colpo Accurato}, Comune, 0\\
\stepcounter{inclvzero}\hyperlink{Creare Birra}{Creare Birra}, Raro, 0\\
\stepcounter{inclvzero}\hyperlink{Dito}{Dito}, Raro, 0\\
\stepcounter{inclvzero}\hyperlink{Fiamma Sacra}{Fiamma Sacra}, Comune, 0\\
\stepcounter{inclvzero}\hyperlink{Fiotto Acido}{Fiotto Acido}, Comune, 0\\
\stepcounter{inclvzero}\hyperlink{Lacrima di Ljust}{Lacrima di Ljust}, Non Comune, 0\\
\stepcounter{inclvzero}\hyperlink{Mano Magica}{Mano Magica}, Comune, 0\\
\stepcounter{inclvzero}\hyperlink{Marchio Magico}{Marchio Magico}, Comune, 0\\
\stepcounter{inclvzero}\hyperlink{Messaggio}{Messaggio}, Comune, 0\\
\stepcounter{inclvzero}\hyperlink{Prestidigitazione}{Prestidigitazione}, Comune, 0\\
\stepcounter{inclvzero}\hyperlink{Produrre Fiamma}{Produrre Fiamma}, Comune, 0\\
\stepcounter{inclvzero}\hyperlink{Raggio di Gelo}{Raggio di Gelo}, Comune, 0\\
\stepcounter{inclvzero}\hyperlink{Randello Incantato}{Randello Incantato}, Comune, 0\\
\stepcounter{inclvzero}\hyperlink{Resistenza}{Resistenza}, Comune, 0\\
\stepcounter{inclvzero}\hyperlink{Riparare}{Riparare}, Comune, 0\\
\stepcounter{inclvzero}\hyperlink{Scudo}{Scudo}, Comune, 0\\
\stepcounter{inclvzero}\hyperlink{Spruzzo Velenoso}{Spruzzo Velenoso}, Non Comune, 0\\
\stepcounter{inclvzero}\hyperlink{Stretta Folgorante}{Stretta Folgorante}, Comune, 0\\
\stepcounter{inclvzero}\hyperlink{Taumaturgia}{Taumaturgia}, Non Comune, 0\\
\stepcounter{inclvzero}\hyperlink{Tocco Gelido}{Tocco Gelido}, Comune, 0\\
\stepcounter{inclvzero}\hyperlink{Viticci Perforanti}{Viticci Perforanti}, Non Comune, 0\\

Totale incantesimi: \theinclvzero\\

\textbf{Livello 1} \newcounter{inclvuno}

\stepcounter{inclvuno}\hyperlink{Allarme}{Allarme}, Comune, 1\\
\stepcounter{inclvuno}\hyperlink{Alterare Sé Stesso}{Alterare Sé Stesso}, Non Comune, 1\\
\stepcounter{inclvuno}\hyperlink{Amicizia con gli Animali}{Amicizia con gli Animali}, Non Comune, 1\\
\stepcounter{inclvuno}\hyperlink{Anatema}{Anatema}, Comune, 1\\
\stepcounter{inclvuno}\hyperlink{Armatura Magica}{Armatura Magica}, Non Comune, 1\\
\stepcounter{inclvuno}\hyperlink{Benedizione}{Benedizione}, Comune, 1\\
\stepcounter{inclvuno}\hyperlink{Caduta Piuma}{Caduta Piuma}, Comune, 1\\
\stepcounter{inclvuno}\hyperlink{Camuffare Sé Stesso}{Camuffare Sé Stesso}, Comune, 1\\
\stepcounter{inclvuno}\hyperlink{Charme su Persone}{Charme su Persone}, Comune, 1\\
\stepcounter{inclvuno}\hyperlink{Colpo Fiammeggiante}{Colpo Fiammeggiante}, Raro, 1\\
\stepcounter{inclvuno}\hyperlink{Comando}{Comando}, Comune, 1\\
\stepcounter{inclvuno}\hyperlink{Comprensione dei Linguaggi}{Comprensione dei Linguaggi}, Comune, 1\\
\stepcounter{inclvuno}\hyperlink{Conoscere i Tratti}{Conoscere i Tratti}, Leggendario, 1\\
\stepcounter{inclvuno}\hyperlink{Creare o Distruggere Acqua}{Creare o Distruggere Acqua}, Comune, 1\\
\stepcounter{inclvuno}\hyperlink{Cuoco Invisibile}{Cuoco Invisibile}, Comune, 1\\
\stepcounter{inclvuno}\hyperlink{Cura Ferite}{Cura Ferite}, Comune, 1\\
\stepcounter{inclvuno}\hyperlink{Dardo arcano}{Dardo arcano}, Comune, 1\\
\stepcounter{inclvuno}\hyperlink{Dardo di Fuoco}{Dardo di Fuoco}, Comune, 1\\
\stepcounter{inclvuno}\hyperlink{Dardo occulto}{Dardo occulto}, Comune, 1\\
\stepcounter{inclvuno}\hyperlink{Dardo Tracciante}{Dardo Tracciante}, Non Comune, 1\\
\stepcounter{inclvuno}\hyperlink{Disco Fluttuante}{Disco Fluttuante}, Comune, 1\\
\stepcounter{inclvuno}\hyperlink{Eroismo}{Eroismo}, Non Comune, 1\\
\stepcounter{inclvuno}\hyperlink{Favore Divino}{Favore Divino}, Non Comune, 1\\
\stepcounter{inclvuno}\hyperlink{Grido di dolore}{Grido di dolore}, Raro, 1\\
\stepcounter{inclvuno}\hyperlink{Guida}{Guida}, Comune, 1\\
\stepcounter{inclvuno}\hyperlink{Identificare}{Identificare}, Comune, 1\\
\stepcounter{inclvuno}\hyperlink{Illusione Minore}{Illusione Minore}, Comune, 1\\
\stepcounter{inclvuno}\hyperlink{Immagine Silenziosa}{Immagine Silenziosa}, Comune, 1\\
\stepcounter{inclvuno}\hyperlink{Individuazione del Magico}{Individuazione del Magico}, Comune, 1\\
\stepcounter{inclvuno}\hyperlink{Intralciare}{Intralciare}, Comune, 1\\
\stepcounter{inclvuno}\hyperlink{Lettura del Magico}{Lettura del Magico}, Comune, 1\\
\stepcounter{inclvuno}\hyperlink{Luce}{Luce}, Comune, 1\\
\stepcounter{inclvuno}\hyperlink{Luci Danzanti}{Luci Danzanti}, Non Comune, 1\\
\stepcounter{inclvuno}\hyperlink{Luminescenza}{Luminescenza}, Non Comune, 1\\
\stepcounter{inclvuno}\hyperlink{Onda rovente}{Onda rovente}, Comune, 1\\
\stepcounter{inclvuno}\hyperlink{Onda Tonante}{Onda Tonante}, Comune, 1\\
\stepcounter{inclvuno}\hyperlink{Oscurità}{Oscurità}, Comune, 1\\
\stepcounter{inclvuno}\hyperlink{Palla di fango di Eithne}{Palla di fango di Eithne}, Non Comune, 1\\
\stepcounter{inclvuno}\hyperlink{Parlare con gli Animali}{Parlare con gli Animali}, Comune, 1\\
\stepcounter{inclvuno}\hyperlink{Passo Veloce}{Passo Veloce}, Molto Raro, 1\\
\stepcounter{inclvuno}\hyperlink{Protezione dall'Energia minore}{Protezione dall'Energia minore}, Raro, 1\\
\stepcounter{inclvuno}\hyperlink{Purificare Cibo e Bevande}{Purificare Cibo e Bevande}, Comune, 1\\
\stepcounter{inclvuno}\hyperlink{Raggio di Indebolimento}{Raggio di Indebolimento}, Comune, 1\\
\stepcounter{inclvuno}\hyperlink{Risata Incontenibile}{Risata Incontenibile}, Non Comune, 1\\
\stepcounter{inclvuno}\hyperlink{Ritirata Rapida}{Ritirata Rapida}, Non Comune, 1\\
\stepcounter{inclvuno}\hyperlink{Saltare}{Saltare}, Comune, 1\\
\stepcounter{inclvuno}\hyperlink{Santuario}{Santuario}, Comune, 1\\
\stepcounter{inclvuno}\hyperlink{Rimuovi Paura}{Rimuovi Paura}, Comune, 1 \\
\stepcounter{inclvuno}\hyperlink{Scagliare Maledizione Minore}{Scagliare Maledizione Minore}, Comune, 1\\
\stepcounter{inclvuno}\hyperlink{Schiaffo di Cattalm}{Schiaffo di Cattalm}, Non Comune, 1\\
\stepcounter{inclvuno}\hyperlink{Scritto Illusorio}{Scritto Illusorio}, Comune, 1\\
\stepcounter{inclvuno}\hyperlink{Servitore Invisibile}{Servitore Invisibile}, Comune, 1\\
\stepcounter{inclvuno}\hyperlink{Sonno}{Sonno}, Comune, 1\\
\stepcounter{inclvuno}\hyperlink{Spruzzo Colorato}{Spruzzo Colorato}, Comune, 1\\
\stepcounter{inclvuno}\hyperlink{Unto}{Unto}, Comune, 1\\
\stepcounter{inclvuno}\hyperlink{Ventriloquio}{Ventriloquio}, Comune, 1\\
\stepcounter{inclvuno}\hyperlink{Vita Falsata}{Vita Falsata}, Comune, 1\\


\medskip Totale incantesimi: \theinclvuno\\

\textbf{Livello 2} \newcounter{inclvdue}

\stepcounter{inclvdue}\hyperlink{Animale Messaggero}{Animale Messaggero}, Comune, 2\\
\stepcounter{inclvdue}\hyperlink{Arma Magica}{Arma Magica}, Comune, 2\\
\stepcounter{inclvdue}\hyperlink{Arma Spirituale}{Arma Spirituale}, Comune, 2\\
\stepcounter{inclvdue}\hyperlink{Aura Magica dell'Arcanista}{Aura Magica dell'Arcanista}, Non Comune, 2\\
\stepcounter{inclvdue}\hyperlink{Bacche Benefiche}{Bacche Benefiche}, Comune, 2\\
\stepcounter{inclvdue}\hyperlink{Benedici Acqua}{Benedici Acqua}, Comune, 2\\
\stepcounter{inclvdue}\hyperlink{Benedizione Superiore}{Benedizione Superiore}, Non Comune, 2\\
\stepcounter{inclvdue}\hyperlink{Blocca Persona}{Blocca Persona}, Comune, 2\\
\stepcounter{inclvdue}\hyperlink{Bocca Magica}{Bocca Magica}, Comune, 2\\
\stepcounter{inclvdue}\hyperlink{Calmare Emozioni}{Calmare Emozioni}, Comune, 2\\
\stepcounter{inclvdue}\hyperlink{Caratteristica Potenziata}{Caratteristica Potenziata}, Comune, 2\\
\stepcounter{inclvdue}\hyperlink{Cecità/Sordità}{Cecità/Sordità}, Comune, 2\\
\stepcounter{inclvdue}\hyperlink{Chiudi Portale}{Chiudi Portale}, Raro, 2\\
\stepcounter{inclvdue}\hyperlink{Colpo Luccicante}{Colpo Luccicante}, Non Comune, 2\\
\stepcounter{inclvdue}\hyperlink{Comprensione degli Scritti}{Comprensione degli Scritti}, Non Comune, 2\\
\stepcounter{inclvdue}\hyperlink{Creare Fossa}{Creare Fossa}, Non Comune, 2\\
\stepcounter{inclvdue}\hyperlink{Crescita di Spuntoni}{Crescita di Spuntoni}, Comune, 2\\
\stepcounter{inclvdue}\hyperlink{Estasiare}{Estasiare}, Comune, 2\\
\stepcounter{inclvdue}\hyperlink{Evoca Cavalcatura}{Evoca Cavalcatura}, Comune, 2\\
\stepcounter{inclvdue}\hyperlink{Fiamma Perenne}{Fiamma Perenne}, Leggendario, 2\\
\stepcounter{inclvdue}\hyperlink{Folata di Vento}{Folata di Vento}, Comune, 2\\
\stepcounter{inclvdue}\hyperlink{Frantumare}{Frantumare}, Comune, 2\\
\stepcounter{inclvdue}\hyperlink{Goffaggine}{Goffaggine}, Raro, 2\\
\stepcounter{inclvdue}\hyperlink{Gragnola di Ghiande di Kyrin}{Gragnola di Ghiande di Kyrin}, Non Comune, 2\\
\stepcounter{inclvdue}\hyperlink{Immagine Speculare}{Immagine Speculare}, Comune, 2\\
\stepcounter{inclvdue}\hyperlink{Individuazione dei Pensieri}{Individuazione dei Pensieri}, Raro, 2\\
\stepcounter{inclvdue}\hyperlink{Individuazione delle Malattie e dei Veleni}{Individuazione delle Malattie e dei Veleni}, Non Comune, 2\\
\stepcounter{inclvdue}\hyperlink{Infliggi Ferite}{Infliggi Ferite}, Comune, 2\\
\stepcounter{inclvdue}\hyperlink{Ingrandire/Ridurre}{Ingrandire/Ridurre}, Comune, 2\\
\stepcounter{inclvdue}\hyperlink{Invisibilità}{Invisibilità}, Comune, 2\\
\stepcounter{inclvdue}\hyperlink{Lacrima di Laydel}{Lacrima di Laydel}, Molto Raro, 2\\
\stepcounter{inclvdue}\hyperlink{Lama Infuocata}{Lama Infuocata}, Comune, 2\\
\stepcounter{inclvdue}\hyperlink{Lanciafiamme}{Lanciafiamme}, Raro, 2\\
\stepcounter{inclvdue}\hyperlink{Lettura della terra di Kyrin}{Lettura della terra di Kyrin}, Non Comune, 2\\
\stepcounter{inclvdue}\hyperlink{Levitazione}{Levitazione}, Comune, 2\\
\stepcounter{inclvdue}\hyperlink{Localizza Animali e Piante}{Localizza Animali e Piante}, Non Comune, 2\\
\stepcounter{inclvdue}\hyperlink{Localizza Oggetto}{Localizza Oggetto}, Comune, 2\\
\stepcounter{inclvdue}\hyperlink{Movimenti del Ragno}{Movimenti del Ragno}, Non Comune, 2\\
\stepcounter{inclvdue}\hyperlink{Passo Velato}{Passo Velato}, Non Comune, 2\\
\stepcounter{inclvdue}\hyperlink{Pelle di Corteccia}{Pelle di Corteccia}, Comune, 2\\
\stepcounter{inclvdue}\hyperlink{Piroesperto}{Piroesperto}, Non Comune, 2\\
\stepcounter{inclvdue}\hyperlink{Preghiera di Guarigione}{Preghiera di Guarigione}, Comune, 2\\
\stepcounter{inclvdue}\hyperlink{Presagio}{Presagio}, Comune, 2\\
\stepcounter{inclvdue}\hyperlink{Protezione dai Veleni}{Protezione dai Veleni}, Non Comune, 2\\
\stepcounter{inclvdue}\hyperlink{Punizione Marchiante}{Punizione Marchiante}, Comune, 2\\
\stepcounter{inclvdue}\hyperlink{Raggio mortale}{Raggio mortale}, Raro, 2\\
\stepcounter{inclvdue}\hyperlink{Raggio Rovente}{Raggio Rovente}, Comune, 2\\
\stepcounter{inclvdue}\hyperlink{Ragnatela}{Ragnatela}, Comune, 2\\
\stepcounter{inclvdue}\hyperlink{Rimuovi Malattia}{Rimuovi Malattia}, Comune, 2\\
\stepcounter{inclvdue}\hyperlink{Riposo Inviolato}{Riposo Inviolato}, Non Comune, 2\\
\stepcounter{inclvdue}\hyperlink{Riscaldare il Metallo}{Riscaldare il Metallo}, Non Comune, 2\\
\stepcounter{inclvdue}\hyperlink{Ristorare Inferiore}{Ristorare Inferiore}, Comune, 2\\
\stepcounter{inclvdue}\hyperlink{Scassinare}{Scassinare}, Comune, 2\\
\stepcounter{inclvdue}\hyperlink{Scopri Piante}{Scopri Piante}, Non Comune, 2\\
\stepcounter{inclvdue}\hyperlink{Scopri Trappole}{Scopri Trappole}, Comune, 2\\
\stepcounter{inclvdue}\hyperlink{Scurovisione}{Scurovisione}, Comune, 2\\
\stepcounter{inclvdue}\hyperlink{Serratura Magica}{Serratura Magica}, Comune, 2\\
\stepcounter{inclvdue}\hyperlink{Sfera Infuocata}{Sfera Infuocata}, Comune, 2\\
\stepcounter{inclvdue}\hyperlink{Sfocatura}{Sfocatura}, Comune, 2\\
\stepcounter{inclvdue}\hyperlink{Silenzio}{Silenzio}, Comune, 2\\
\stepcounter{inclvdue}\hyperlink{Sonnellino}{Sonnellino}, Leggendario, 2\\
\stepcounter{inclvdue}\hyperlink{Suggestione}{Suggestione}, Raro, 2\\
\stepcounter{inclvdue}\hyperlink{Trucco della Corda}{Trucco della Corda}, Comune, 2\\
\stepcounter{inclvdue}\hyperlink{Vedere l'invisibile}{Vedere l'invisibile}, Comune, 2\\
\stepcounter{inclvdue}\hyperlink{Vincolo di Interdizione}{Vincolo di Interdizione}, Comune, 2\\
\stepcounter{inclvdue}\hyperlink{Zona di Verità}{Zona di Verità}, Non Comune, 2\\

\medskip Totale incantesimi: \theinclvdue\\

\textbf{Livello 3}  \newcounter{inclvtre}

\stepcounter{inclvtre}\hyperlink{Animare Morti}{Animare Morti}, Comune, 3\\
\stepcounter{inclvtre}\hyperlink{Anti-Individuazione}{Anti-Individuazione}, Non Comune, 3\\
\stepcounter{inclvtre}\hyperlink{Bastoni in Serpenti}{Bastoni in Serpenti}, Non Comune, 3\\
\stepcounter{inclvtre}\hyperlink{Benedizione di Ledyal}{Benedizione di Ledyal}, Molto Raro, 3\\
\stepcounter{inclvtre}\hyperlink{Benedizione della Vita}{Benedizione della Vita}, Raro, 3\\
\stepcounter{inclvtre}\hyperlink{Benedizione Suprema}{Benedizione Suprema}, Raro, 3\\
\stepcounter{inclvtre}\hyperlink{Camminare sull'Acqua}{Camminare sull'Acqua}, Comune, 3\\
\stepcounter{inclvtre}\hyperlink{Capanna}{Capanna}, Non Comune, 3\\
\stepcounter{inclvtre}\hyperlink{Cecità/Sordità Avanzata}{Cecità/Sordità Avanzata}, Non Comune, 3\\
\stepcounter{inclvtre}\hyperlink{Cerchio Magico}{Cerchio Magico}, Comune, 3\\
\stepcounter{inclvtre}\hyperlink{Cerchio d'Invisibilità}{Cerchio d'Invisibilità}, Non Comune, 3\\
\stepcounter{inclvtre}\hyperlink{Chiaroveggenza}{Chiaroveggenza}, Comune, 3\\
\stepcounter{inclvtre}\hyperlink{Controincantesimo}{Controincantesimo}, Comune, 3\\
\stepcounter{inclvtre}\hyperlink{Creare Cibo e Acqua}{Creare Cibo e Acqua}, Comune, 3\\
\stepcounter{inclvtre}\hyperlink{Crescita Vegetale}{Crescita Vegetale}, Non Comune, 3\\
\stepcounter{inclvtre}\hyperlink{Destriero Fantasma}{Destriero Fantasma}, Comune, 3\\
\stepcounter{inclvtre}\hyperlink{Dissolvi Magie}{Dissolvi Magie}, Comune, 3\\
\stepcounter{inclvtre}\hyperlink{Distruggere nonmorto}{Distruggere Nonmorto}, Non Comune, 3\\
\stepcounter{inclvtre}\hyperlink{Evoca Animali}{Evoca Animali}, Non Comune, 3\\
\stepcounter{inclvtre}\hyperlink{Forma Gassosa}{Forma Gassosa}, Non Comune, 3\\
\stepcounter{inclvtre}\hyperlink{Fulmine}{Fulmine}, Comune, 3\\
\stepcounter{inclvtre}\hyperlink{Glifo di Interdizione}{Glifo di Interdizione}, Comune, 3\\
\stepcounter{inclvtre}\hyperlink{Immagine Maggiore}{Immagine Maggiore}, Comune, 3\\
\stepcounter{inclvtre}\hyperlink{Intermittenza}{Intermittenza}, Non Comune, 3\\
\stepcounter{inclvtre}\hyperlink{Inviare}{Inviare}, Comune, 3\\
\stepcounter{inclvtre}\hyperlink{Invocare il Fulmine}{Invocare il Fulmine}, Comune, 3\\
\stepcounter{inclvtre}\hyperlink{lentezza}{Lentezza}, Non Comune, 3\\
\stepcounter{inclvtre}\hyperlink{Lingue}{Lingue}, Comune, 3\\
\stepcounter{inclvtre}\hyperlink{Luce Diurna}{Luce Diurna}, Comune, 3\\
\stepcounter{inclvtre}\hyperlink{Morte Apparente}{Morte Apparente}, Non Comune, 3\\
\stepcounter{inclvtre}\hyperlink{Muro di Vento}{Muro di Vento}, Non Comune, 3\\
\stepcounter{inclvtre}\hyperlink{Palla di Fuoco}{Palla di Fuoco}, Comune, 3\\
\stepcounter{inclvtre}\hyperlink{Parlare con i Morti}{Parlare con i Morti}, Raro, 3\\
\stepcounter{inclvtre}\hyperlink{Parlare con le Piante}{Parlare con le Piante}, Raro, 3\\
\stepcounter{inclvtre}\hyperlink{Paura}{Paura}, Non Comune, 3\\
\stepcounter{inclvtre}\hyperlink{Preghiera}{Preghiera}, Non Comune, 3\\
\stepcounter{inclvtre}\hyperlink{Protezione dall'Energia}{Protezione dall'Energia}, Comune, 3\\
\stepcounter{inclvtre}\hyperlink{Rimuovi Maledizione}{Rimuovi Maledizione}, Comune, 3\\
\stepcounter{inclvtre}\hyperlink{Scagliare Maledizione}{Scagliare Maledizione}, Non Comune, 3\\
\stepcounter{inclvtre}\hyperlink{Tempesta di Nevischio}{Tempesta di Nevischio}, Molto Raro, 3\\
\stepcounter{inclvtre}\hyperlink{Tocco Vampirico}{Tocco Vampirico}, Raro, 3\\
\stepcounter{inclvtre}\hyperlink{Trama Ipnotica}{Trama Ipnotica}, Comune, 3\\
\stepcounter{inclvtre}\hyperlink{Uno con la pietra}{Uno con la pietra}, Comune, 3\\
\stepcounter{inclvtre}\hyperlink{Velocità}{Velocità}, Non Comune, 3\\
\stepcounter{inclvtre}\hyperlink{Volare}{Volare}, Comune, 3\\

\medskip Totale incantesimi: \theinclvtre\\

\textbf{Livello 4}   \newcounter{inclvquattro}

\stepcounter{inclvquattro}\hyperlink{Allucinazione Mortale}{Allucinazione Mortale}, Non Comune, 4\\
\stepcounter{inclvquattro}\hyperlink{Blocca Persona Avanzato}{Blocca Persona Avanzato}, Non Comune, 4\\
\stepcounter{inclvquattro}\hyperlink{Camminare nell'aria}{Camminare nell'aria}, Non Comune, 4\\
\stepcounter{inclvquattro}\hyperlink{Compulsione}{Compulsione}, Non Comune, 4\\
\stepcounter{inclvquattro}\hyperlink{Confusione}{Confusione}, Comune, 4\\
\stepcounter{inclvquattro}\hyperlink{Controllare Acqua}{Controllare Acqua}, Comune, (4)\\
\stepcounter{inclvquattro}\hyperlink{Dominare Bestie}{Dominare Bestie}, Vario, 4\\
\stepcounter{inclvquattro}\hyperlink{Esilio}{Esilio}, Comune, 4\\
\stepcounter{inclvquattro}\hyperlink{Fabbricare}{Fabbricare}, Comune, 4\\
\stepcounter{inclvquattro}\hyperlink{Inaridire}{Inaridire}, Non Comune, 4\\
\stepcounter{inclvquattro}\hyperlink{Insetto Gigante}{Insetto Gigante}, Non Comune, 4\\
\stepcounter{inclvquattro}\hyperlink{Interdizione alla Morte}{Interdizione alla Morte}, Non Comune, 4\\
\stepcounter{inclvquattro}\hyperlink{Invisibilità Superiore}{Invisibilità Superiore}, Non Comune, 4\\
\stepcounter{inclvquattro}\hyperlink{Libertà di Movimento}{Libertà di Movimento}, Comune, 4\\
\stepcounter{inclvquattro}\hyperlink{Localizza Creatura}{Localizza Creatura}, Comune, 4\\
\stepcounter{inclvquattro}\hyperlink{Metamorfosi}{Metamorfosi}, Comune, 4\\
\stepcounter{inclvquattro}\hyperlink{Muro di Fuoco}{Muro di Fuoco}, Non Comune, 4\\
\stepcounter{inclvquattro}\hyperlink{Occhio Arcano}{Occhio Arcano}, Comune, 4\\
\stepcounter{inclvquattro}\hyperlink{Pelle di Pietra}{Pelle di Pietra}, Non Comune, 4\\
\stepcounter{inclvquattro}\hyperlink{Porta Dimensionale}{Porta Dimensionale}, Comune, 4\\
\stepcounter{inclvquattro}\hyperlink{Profumo di Atherim}{Profumo di Atherim}, Raro, 4\\
\stepcounter{inclvquattro}\hyperlink{Santuario Privato}{Santuario Privato}, Molto Raro, 4\\
\stepcounter{inclvquattro}\hyperlink{Scolpire Pietra}{Scolpire Pietra}, Comune, 4\\
\stepcounter{inclvquattro}\hyperlink{Scrigno Segreto}{Scrigno Segreto}, Raro, 4\\
\stepcounter{inclvquattro}\hyperlink{Segugio Fedele}{Segugio Fedele}, Raro, 4\\
\stepcounter{inclvquattro}\hyperlink{Tentacoli Neri}{Tentacoli Neri}, Non Comune, 4\\
\stepcounter{inclvquattro}\hyperlink{Terreno Illusorio}{Terreno Illusorio}, Non Comune, 4\\
\stepcounter{inclvquattro}\hyperlink{Vigore}{Vigore}, Raro, 4\\

\medskip Totale incantesimi: \theinclvquattro\\

\textbf{Livello 5} \newcounter{inclvcinque}

\stepcounter{inclvcinque}\hyperlink{Animare Oggetti}{Animare Oggetti}, Comune, 5\\
\stepcounter{inclvcinque}\hyperlink{Barriera Antianimali}{Barriera Antianimali}, Raro, 5\\
\stepcounter{inclvcinque}\hyperlink{Cerchio di Teletrasporto}{Cerchio di Teletrasporto}, Non Comune, 5\\
\stepcounter{inclvcinque}\hyperlink{Colpo Infuocato}{Colpo Infuocato}, Comune, 5\\
\stepcounter{inclvcinque}\hyperlink{Comunione}{Comunione}, Raro, 5\\
\stepcounter{inclvcinque}\hyperlink{Comunione con la Natura}{Comunione con la Natura}, Molto Raro, 5\\
\stepcounter{inclvcinque}\hyperlink{Cono di Freddo}{Cono di Freddo}, Comune, 5\\
\stepcounter{inclvcinque}\hyperlink{Conoscenza delle Leggende}{Conoscenza delle Leggende}, Comune, 5\\
\stepcounter{inclvcinque}\hyperlink{Contagio}{Contagio}, Non Comune, 5\\
\stepcounter{inclvcinque}\hyperlink{Costrizione}{Costrizione}, Raro, 5\\
\stepcounter{inclvcinque}\hyperlink{Creazione}{Creazione}, Raro, 5\\
\stepcounter{inclvcinque}\hyperlink{Dissolvi Magie Avanzato}{Dissolvi Magie Avanzato}, Raro, 5\\
\stepcounter{inclvcinque}\hyperlink{Dominare Persone}{Dominare Persone}, Non Comune, 5\\
\stepcounter{inclvcinque}\hyperlink{Fuorviare}{Fuorviare}, Non Comune, 5\\
\stepcounter{inclvcinque}\hyperlink{Gragnola di Marroni di Kyrin}{Gragnola di Marroni di Kyrin}, Molto Raro, 5\\
\stepcounter{inclvcinque}\hyperlink{Guscio Anti-Vita}{Guscio Anti-Vita}, Non Comune, 5\\
\stepcounter{inclvcinque}\hyperlink{Legame Telepatico}{Legame Telepatico}, Raro, 5\\
\stepcounter{inclvcinque}\hyperlink{Mano Arcana}{Mano Arcana}, Non Comune, 5\\
\stepcounter{inclvcinque}\hyperlink{Modificare Memoria}{Modificare Memoria}, Molto Raro, 5\\
\stepcounter{inclvcinque}\hyperlink{Muro di Forza}{Muro di Forza}, Comune, 5\\
\stepcounter{inclvcinque}\hyperlink{Muro di Pietra}{Muro di Pietra}, Comune, 5\\
\stepcounter{inclvcinque}\hyperlink{Passa Porta}{Passa Porta}, Non Comune, 5\\
\stepcounter{inclvcinque}\hyperlink{Piaga degli Insetti}{Piaga degli Insetti}, Raro, 5\\
\stepcounter{inclvcinque}\hyperlink{Pietra in Fango - Fango in Pietra}{Pietra in Fango - Fango in Pietra}, Non Comune - Molto Raro, 5\\
\stepcounter{inclvcinque}\hyperlink{Reincarnazione}{Reincarnazione}, Raro, 5\\
\stepcounter{inclvcinque}\hyperlink{Ristorare Superiore}{Ristorare Superiore}, Non Comune, 5\\
\stepcounter{inclvcinque}\hyperlink{Risveglio}{Risveglio}, Raro, 5\\
\stepcounter{inclvcinque}\hyperlink{Santificare}{Santificare}, Raro, 5\\
\stepcounter{inclvcinque}\hyperlink{Scrutare}{Scrutare}, Raro, 5\\
\stepcounter{inclvcinque}\hyperlink{Sembrare}{Sembrare}, Non Comune, 5\\
\stepcounter{inclvcinque}\hyperlink{Sogno}{Sogno}, Non Comune, 5\\
\stepcounter{inclvcinque}\hyperlink{Telecinesi}{Telecinesi}, Non Comune, 5\\
\stepcounter{inclvcinque}\hyperlink{Traslazione Arborea}{Traslazione Arborea}, Raro, 5\\

\medskip Totale incantesimi: \theinclvcinque\\

\textbf{Livello 6} \newcounter{inclvsei}

\stepcounter{inclvsei}\hyperlink{Bagliore Solare}{Bagliore Solare}, Non Comune, 6\\
\stepcounter{inclvsei}\hyperlink{Banchetto degli Eroi}{Banchetto degli Eroi}, Non Comune, 6\\
\stepcounter{inclvsei}\hyperlink{Barriera di Lame}{Barriera di Lame}, Comune, 6\\
\stepcounter{inclvsei}\hyperlink{Camminare nel Vento}{Camminare nel Vento}, Non Comune, 6\\
\stepcounter{inclvsei}\hyperlink{Carne in Pietra - Pietra in Carne}{Carne in Pietra - Pietra in Carne}, Non Comune - Raro, 6\\
\stepcounter{inclvsei}\hyperlink{Cerchio di Morte}{Cerchio di Morte}, Molto Raro, 6\\
\stepcounter{inclvsei}\hyperlink{Contingenza}{Contingenza}, Comune, 6\\
\stepcounter{inclvsei}\hyperlink{Creare Non Morti}{Creare Non Morti}, Non Comune, 6\\
\stepcounter{inclvsei}\hyperlink{Disintegrazione}{Disintegrazione}, Non Comune, 6\\
\stepcounter{inclvsei}\hyperlink{Dito della Morte}{Dito della Morte}, Raro, 6\\
\stepcounter{inclvsei}\hyperlink{Divinazione}{Divinazione}, Comune, 6\\
\stepcounter{inclvsei}\hyperlink{Evocazioni Istantanee}{Evocazioni Istantanee}, Raro, 6\\
\stepcounter{inclvsei}\hyperlink{Ferire}{Ferire}, Non Comune, 6\\
\stepcounter{inclvsei}\hyperlink{Fulmine a catena}{Fulmine a catena}, Raro, 6\\
\stepcounter{inclvsei}\hyperlink{Giara Magica}{Giara Magica}, Molto Raro, 6\\
\stepcounter{inclvsei}\hyperlink{Globo di Invulnerabilità}{Globo di Invulnerabilità}, Comune, 6\\
\stepcounter{inclvsei}\hyperlink{Guarigione}{Guarigione}, Raro, 6\\
\stepcounter{inclvsei}\hyperlink{Illusione Programmata}{Illusione Programmata}, Non Comune, 6\\
\stepcounter{inclvsei}\hyperlink{Muovere il Terreno}{Muovere il Terreno}, Non Comune, 6\\
\stepcounter{inclvsei}\hyperlink{Muro di Ghiaccio}{Muro di Ghiaccio}, Comune, (6)\\
\stepcounter{inclvsei}\hyperlink{Muro di Spine}{Muro di Spine}, Non Comune, 6\\
\stepcounter{inclvsei}\hyperlink{Parlare con le Creature}{Parlare con le Creature}, Raro, 6\\
\stepcounter{inclvsei}\hyperlink{Parola del Ritiro}{Parola del Ritiro}, Raro, 6\\
\stepcounter{inclvsei}\hyperlink{Pietre Parlanti}{Pietre Parlanti}, Raro, 6\\
\stepcounter{inclvsei}\hyperlink{Proibizione}{Proibizione}, Non Comune, 6\\
\stepcounter{inclvsei}\hyperlink{Scopri il Percorso}{Scopri il Percorso}, Non Comune, 6\\
\stepcounter{inclvsei}\hyperlink{Sfera Congelante}{Sfera Congelante}, Raro, 6\\
\stepcounter{inclvsei}\hyperlink{Sguardo Penetrante}{Sguardo Penetrante}, Molto Raro, 6\\
\stepcounter{inclvsei}\hyperlink{Suggestione di Massa}{Suggestione di Massa}, Molto Raro, 6\\
\stepcounter{inclvsei}\hyperlink{Trasformazione Furiosa di Restser}{Trasformazione Furiosa di Restser}, Molto Raro, 6\\
\stepcounter{inclvsei}\hyperlink{Trasporto Vegetale}{Trasporto Vegetale}, Molto Raro, 6\\
\stepcounter{inclvsei}\hyperlink{Vigilanza e Interdizione}{Vigilanza e Interdizione}, Non Comune, 6\\
\stepcounter{inclvsei}\hyperlink{Visione del Vero}{Visione del Vero}, Raro, 6\\

\medskip Totale incantesimi: \theinclvsei\\

\textbf{Livello 7}\newcounter{inclvsette}

\stepcounter{inclvsette}\hyperlink{Celare}{Celare}, Raro, 7\\
\stepcounter{inclvsette}\hyperlink{Desiderio limitato}{Desiderio limitato}, Raro, 7\\
\stepcounter{inclvsette}\hyperlink{Forma Eterea}{Forma Eterea}, Raro, 7\\
\stepcounter{inclvsette}\hyperlink{Immagine Proiettata}{Immagine Proiettata}, Non Comune, 7\\
\stepcounter{inclvsette}\hyperlink{Inversione della Gravità}{Inversione della Gravità}, Raro, 7\\
\stepcounter{inclvsette}\hyperlink{Miraggio Arcano}{Miraggio Arcano}, Raro, 7\\
\stepcounter{inclvsette}\hyperlink{Palla di Fuoco Ritardata}{Palla di Fuoco Ritardata}, Raro, 7\\
\stepcounter{inclvsette}\hyperlink{Parola Divina}{Parola Divina}, Molto Raro, 7\\
\stepcounter{inclvsette}\hyperlink{Reggia Meravigliosa}{Reggia Meravigliosa}, Leggendario, 7\\
\stepcounter{inclvsette}\hyperlink{Rigenerazione}{Rigenerazione}, Leggendario, 7\\
\stepcounter{inclvsette}\hyperlink{Simbolo}{Simbolo}, Non Comune, 7\\
\stepcounter{inclvsette}\hyperlink{Spada Arcana}{Spada Arcana}, Raro, 7\\
\stepcounter{inclvsette}\hyperlink{Spruzzo Prismatico}{Spruzzo Prismatico}, Raro, 7\\
\stepcounter{inclvsette}\hyperlink{Statua}{Statua}, Raro, 7\\
\stepcounter{inclvsette}\hyperlink{Teletrasporto}{Teletrasporto}, Comune, 7\\
\stepcounter{inclvsette}\hyperlink{Tempesta di Fuoco}{Tempesta di Fuoco}, Raro, 7\\

\medskip Totale incantesimi: \theinclvsette\\

\textbf{Livello 8}\newcounter{inclvotto}

\stepcounter{inclvotto}\hyperlink{Antipatia/Simpatia}{Antipatia/Simpatia}, Raro, 8\\
\stepcounter{inclvotto}\hyperlink{Aura Sacra}{Aura Sacra}, Comune, 8\\
\stepcounter{inclvotto}\hyperlink{Benedizioni di Efrem}{Benedizioni di Efrem}, Raro, 8\\
\stepcounter{inclvotto}\hyperlink{Campo Anti-Magia}{Campo Anti-Magia}, Raro, 8\\
\stepcounter{inclvotto}\hyperlink{Confusione Contagiosa}{Confusione Contagiosa}, Molto Raro, 8\\
\stepcounter{inclvotto}\hyperlink{Controllare Tempo Atmosferico}{Controllare Tempo Atmosferico}, Raro, 8\\
\stepcounter{inclvotto}\hyperlink{Danza Irresistibile}{Danza Irresistibile}, Leggendario, 8\\
\stepcounter{inclvotto}\hyperlink{Dominare Mostri}{Dominare Mostri}, Non Comune, 8\\
\stepcounter{inclvotto}\hyperlink{Esplosione Solare}{Esplosione Solare}, Raro, 8\\
\stepcounter{inclvotto}\hyperlink{Gabbia di Forza}{Gabbia di Forza}, Raro, 8\\
\stepcounter{inclvotto}\hyperlink{Labirinto}{Labirinto}, Raro, 8\\
\stepcounter{inclvotto}\hyperlink{Loquacità}{Loquacità}, Raro, 8\\
\stepcounter{inclvotto}\hyperlink{Nube Incendiaria}{Nube Incendiaria}, Raro, 8\\
\stepcounter{inclvotto}\hyperlink{Parola del Potere Stordire}{Parola del Potere Stordire}, Non Comune, 8\\
\stepcounter{inclvotto}\hyperlink{Regressione Mentale}{Regressione Mentale}, Raro, 8\\
\stepcounter{inclvotto}\hyperlink{Scudo Mentale}{Scudo Mentale}, Non Comune, 8\\
\stepcounter{inclvotto}\hyperlink{Terremoto}{Terremoto}, Molto Raro, 8\\

\medskip Totale incantesimi: \theinclvotto\\

\textbf{Livello 9} \newcounter{inclvnove}

\stepcounter{inclvnove}\hyperlink{Desiderio}{Desiderio}, Non Comune, 9\\
\stepcounter{inclvnove}\hyperlink{Fatale}{Fatale}, Raro, 9\\
\stepcounter{inclvnove}\hyperlink{Fermare il Tempo}{Fermare il Tempo}, Molto Raro, 9\\
\stepcounter{inclvnove}\hyperlink{Guarigione di Massa}{Guarigione di Massa}, Leggendario, 9\\
\stepcounter{inclvnove}\hyperlink{Imprigionare}{Imprigionare}, Raro, 9\\
\stepcounter{inclvnove}\hyperlink{Metamorfosi Pura}{Metamorfosi Pura}, Raro, 9\\
\stepcounter{inclvnove}\hyperlink{Muro Prismatico}{Muro Prismatico}, Raro, (9)\\
\stepcounter{inclvnove}\hyperlink{Parola del Potere Uccidere}{Parola del Potere Uccidere}, Raro, 9\\
\stepcounter{inclvnove}\hyperlink{Previsione}{Previsione}, Non Comune, 9\\
\stepcounter{inclvnove}\hyperlink{Trasformazione}{Trasformazione}, Raro, 9\\

\medskip Totale incantesimi: \theinclvnove

}}

\end{multicols}

\vfill

\begin{center}
	\includegraphics[width=0.5\linewidth]{immagini/the-discovery-of-witchcraft.png}

	\emph{"The Discoverie of Witchcraft' by Reginald Scot, 16 secolo }
\end{center}

%\titlespacing*{\subsubsection}{0pt}{*1}{*1}

\pagebreak

\section{Opzionale - Capacità Iconiche}\index{Opzionale - Abilità Iconiche}\hypertarget{abilitaiconiche}{}\label{abilitaiconiche}

\begin{enfasi}{
La vita è diventata incommensurabilmente migliore da quanto sono stato forzato a smettere di prenderla seriamente. (Daniel Day Lewis)
}\end{enfasi}

\begin{multicols}{2}

%{\small

Queste capacità rappresentano l'apice di un personaggio, non inteso come le abilità finali del 20esimo livello, ma come capacità legate al modo di ruolare, al tipo di personaggio che si è andato a creare e crescere. Queste capacità andrebbero date solo a personaggi che sono stati portati dal primo ad almeno il 15esimo livello, è un riconoscimento al giocatore.

Ogni personaggio può avere una sola capacità iconica, una capacità che contraddistingue gli eroi, capaci di azioni al limite e oltre l'umano. I giocatori sono invitati a creare nuove Abilità Iconiche in base allo sviluppo del personaggio.

\medskip

\subsubsection{Una Luce contro le tenebre}\index{Una Luce contro le tenebre}

\textbf{Requisiti suggeriti}: Patrono Ljust, Sumkjr

Una volta al giorno emetti per 60 minuti luce sacra intorno a te che ti conferisce +1d6 ai Tiri Salvezza e Tiri per Colpire contro i Devoti e Seguaci non del tuo Patrono. Puoi convogliare 1 volta al giorno la luce e tutte le creature Seguaci o Devoti di altri Patroni in raggio di 10 metri da te devono effettuare un TS Tempra a DC 10 + somma Tratti in comune con il Patrono + Saggezza o essere storditi per 2d6 round.

\subsubsection{Il Fabbro}\index{Il Fabbro}

\textbf{Requisiti suggeriti}: abilità nel lavorare il metallo

Le tue capacità di lavorare con le armi ed armature sono leggendarie.
Ogni armatura da te fatta ingombra e pesa come una categoria inferiore, le armi fanno un danno di una categoria superiore di dado.

\subsubsection{L'Oracolo della Guerra}\index{L'Oracolo della Guerra}

\textbf{Requisiti suggeriti}: maestro combattente di mischia

Ogni arma nelle tue mani è letale. Il dado dell'arma raddoppia come raddoppia il danno causato dalla Forza. Es. una spada lunga fa 2d8 di danno e se hai Forza +3 il danno totale diventa 2d8+6

\subsubsection{L'Eroe senza paura}\index{L'Eroe senza paura}

\textbf{Requisiti suggeriti}: coraggioso e risoluto

Una volta per avversario puoi ignorare (per 1d4 round) le condizioni che ti affliggono come Reazione.

\subsubsection{Mindmaster}\index{Mindmaster}

\textbf{Requisiti suggeriti}: una vita a avventurosa gestita con intelligenza e sangue freddo

Puoi usare il punteggio in una Caratteristica mentale (Intelligenza, Saggezza o Carisma) al posto di una fisica (Forza, Destrezza, Costituzione) per quanto riguarda tutte le prove.

\subsubsection*{Su un saurovallo pallido}\index{Su un saurovallo pallido}

\textbf{Requisiti suggeriti}: non temere al morte, aver ucciso tantissimi avversari

Sei la cosa più simile alla morte che i tuoi nemici vedranno mai.
Quando uccidi un nemico tutti gli avversari (che possono aver visto la scena) in 10m di raggio devono fare un TS Volontà con DC pari al Tiro per Colpire, costo una Reazione, od essere influenzati come dall'incantesimo Paura. La capacità è usabile 3 volte al giorno.

\subsubsection{La Furia Magica}\index{La Furia Magica}

\textbf{Requisiti suggeriti}: una vita dedicata alla magia esplosiva

Sei capace di scatenare l'inferno con la magia. La difficoltà (DC) di ogni tuo incantesimo aumenta di 2, quando fai una Prova di Magia tiri 3d6 in più ed ignori 2 dadi tirati.

\subsubsection{L'Ombra}\index{L'Ombra}

\textbf{Requisiti suggeriti}: una vita dedicata a nascondersi e sorprendere i nemici

Tre volte al giorno puoi teletrasportati sull'ombra di un altra creatura che sia entro 30 metri. Costo 1 Reazione.

\subsubsection{La Madre}\index{La Madre}

\textbf{Requisiti suggeriti}: passato più tempo in forma animale che propria

Ha la capacità innata di lasciare le orme di qualsiasi animale, compatibile con la tua taglia, anche se non sei trasformato. Puoi parlare con qualsiasi animale come se fossi sempre sotto effetto dell'incantesimo Parlare con gli Animali.

\subsubsection{Il Morto}\index{Il Morto}

\textbf{Requisiti suggeriti}: una vita di sempre sul baratro della morte

Tre volte al giorno quando i tuoi Punti Ferita scendono sotto 1, con una Azione di Reazione recuperi 3d12 Punti Ferita. Questa capacità può essere usata anche quando i Punti Ferita sono negativi o si dovrebbe essere direttamente morti.

\subsubsection{Il Cacciatore}\index{Il Cacciatore}

\textbf{Requisiti suggeriti}: una vita dedicata a cacciare ed inseguire

Le tue prove di Sopravvivenza hanno un +2d6 di bonus. Il primo colpo che va a segno contro un avversario ottiene automaticamente 2 critici.

\end{multicols}

\pagebreak

\section{Cosmologia}\index{Cosmologia}\hypertarget{cosmologia}{}\label{cosmologia}

\begin{enfasi}{
E' più facile dominare su chi non crede in niente (La Storia Infinita, Kmorf)

\medskip

Tu credi che c'è un Dio solo? Fai bene; anche i demoni lo credono e tremano! (Giacomo Il Giusto 2, 19)}\end{enfasi}

\begin{narratore}[I Patroni]

In OBSS le divinità sono diverse dai tradizionali dei nei giochi di ruolo.

Le divinità, i Patroni, amano sporcarsi le mani, partecipare nelle faccende delle creature che le adorano, per loro è una sfida continua ad avere più credenti, adepti e persone più simili, per Tratti, a loro.

I Patroni sono stati creati come \href{https://www.treccani.it/vocabolario/parossismo/}{parossismo} dell'animo umano, dove tutto è un eccesso. Come spiriti liberati dal vaso di Pandora hanno il solo scopo di portare i loro Tratti al dominio rendendoli i più comuni e presenti tra le creature, specialmente tra le più potenti.
\end{narratore}

\medskip

\begin{multicols}{2}

In principio era il nulla che in sé conteneva il tutto.

L'Energia dei Patroni della Genesi, derivante dalle più primordiali pulsioni esplodeva in tutta la sua potenza senza alcun controllo

Manifestate come due lingue di una fiamma unica Ljust e Calicante sono la fonte infinita di infinita energia.

Ljust l'energia positiva, il calore, la luce, la vita e sintropia; Calicante, l'energia negativa, gelido odio, distruzione, morte ed entropia.

\textbf{Ljust} \index{Ljust}è la rappresentazione di ciò che luce ed vita portano sempre con se. Rappresenta la purezza del sentimento d'amore, la protezione della vita, il rispetto per l'altro, la curiosità per il nuovo, la voglia di migliorarsi sempre, la forza di combattere con coraggio e valore per il bene comune. E' la spinta vitale al cambiamento, il caos che evolve ma non distrugge.

\textbf{Calicante}\index{Calicante} è la rappresentazione del buio, dell'odio, della rabbia e violenza. Calicante è vendetta e fredda distruzione, non c'è interesse in alcuna forma di vita piuttosto le usa, le sfrutta e solo in certi casi ne subisce la presenza. Ama sadicamente la sofferenza. E' l'entropia che annienta e annichilisce e trova piacere nel farlo.

\textbf{Atmos} \index{Atmos}è il testimone, colui che segna il passaggio del tempo e trascrive ogni accadimento sulla Terra e trai Patroni della Genesi. Una entità nata dalla creazione perché prevenisse la distruzione assoluta. Vigila e trascrive quanto fanno i Patroni della Genesi, le divinità che hanno generato il creato.

Un quarto essere fu presente, il cui nome non ci è mai giunto, ma che conosciamo come \emph{Ominiessenza}, colui che era il primo nato dai Patroni della Genesi.

Il Primogenito dei Patroni della Genesi curiosando nell'universo tutto giunse sulla Terra e qui, non si sa come, venne catturato dalla Freten.
Catturato, fatto a pezzi, collegato a macchine per poter attingere alla sua eterna energia. Non si sa come è stato possibile e tuttora è uno dei più grandi misteri irrisolti.

Quando il primo reattore della Frenten venne collegato si aprì una \emph{frattura} e fu così che i Patroni della Genesi scoprirono cosa era accaduto e l'ira di Calicante fu immensa e totale.

La sua volontà e furia fu talmente soverchiante che Ljust non poté intervenire immediatamente, non poté impedire la morte indiscriminata di innocenti e la quasi distruzione di un interno mondo.

Cosa successe in quell'anno lo possiamo ancora vedere nelle macerie e distruzione rimaste sulla Terra.

Ljust riuscì infine a placare, con sommo sacrificio, Calicante e questo richiamò le sue manifestazioni.

La morte del primo Patrono aveva creato numerosi squarci, fratture, nella realtà, Portali che si aprivano in altri mondi, altri universi. Portali che continuavano ad aprirsi e chiudersi cambiando posizione di continuo. In ognuno di questi Portali c'era un pò del loro figlio.

Ljust e Calicante decisero di fare di questo nostro mondo, la Terra, il loro spazio di gioco e qui decisero di fare divertire la loro futura progenie, i Patroni.

Decisero di comune accordo di generare un Patrono che sovrintendesse a questi Portali. Crearono un Patrono che fosse capace di percepire, aprire e bloccarli Così venne creato \textbf{Lynx}, il Guardiano dei Portali.

Lynx \index{Lynx}sovrintende al vuoto cosmico, all'accesso ai Piani, ai portali che con l'avvicendarsi di caos e ordine, di bene e male, di luce e tenebra stanno sempre più creando fratture al confine esistente la la Terra e l'Oltre.

Lynx li percepisce, li sente, sa dove si stanno generando o spegnendo, con il passare del tempo infatti alcuni di questi Portali sono divenuti stabili e definitivi, altri invece continuano a generarsi casualmente e sempre in modo totalmente ignoto rimangono attivi o si esauriscono. Viaggiando di continuo nel non luogo Lynx chiude i portali più grandi ma per uno che ne chiude un altro si apre. Lynx ha privato le Liste di Magia da molti degli incantesimi che agiscono sui piani, per proteggere la Terra ed i futuri Patroni dalle minacce esterne.

Lynx è il custode e carceriere di tutta la Terra.

Atmos, preoccupato per l'equilibrio sulla Terra, incanalò le energie primordiali e divine dei Patroni della Genesi andando a creare un Patrono che potesse rivaleggiare e tenere a bada gli altri futuri Patroni.

Il primo creato da Atmos, con l'aiuto di Ljust, e l'intervento di Calicante, fu \textbf{Gradh}\index{Gradh}, Patrono dell'Umanità (e di tutte le razze senzienti), colui che avrebbe difeso la Terra dalle creature esterne e dagli altri Patroni. Gradh racchiude in sé il dualismo dei due Patroni della Genesi, l'istinto innato alla protezione, alla difesa ed alla cura propri di Ljust e l'istinto di vendetta, violenza e furia di Calicante.

Si getta con coraggio nelle battaglie, attacca il nemico senza paura, protegge il più debole, difende la vita ma non teme di percorrere la strada della vendetta più distruttiva verso chi sfrutta e distrugge vite senza motivo. Gradh ama calarsi fra la gente e vivere con loro, come loro. Non si sente totalmente a suo agio nel pantheon con gli altri Patroni ne fra la gente comune, lui è Umano tra i Patroni e Patrono tra gli Umani. Passionale e gentile è il Patrono che maggiormente ha a cuore le sorti del nostro pianeta e delle sue creature. Gradh è il Patrono che ci protegge dai Patroni, forse l'unico vero figlio della Terra e dei Patroni della Genesi.

Le lingue di energie divine erano troppo intense, caotiche e pure perché Atmos potesse governarle per plasmare da solo i Patroni. Usando il grezzo potere dei Patroni della Genesi creò altri Patroni, la seconda venuta, ognuno influenzato in maniera diversa da Calicante o Ljust. Questi Patroni risultarono meno perfetti e divini rispetto alle sue intenzioni, più imperfetti ed \emph{umani} in quanto originati dalle emozioni, incontrollabili e pure dei Patroni della Genesi. Questi nuovi Patroni plasmano le volontà, fondano regni, comandano nell'ombra le pedine che osano chiedere i loro favori.

In questa apparente calma i Patroni perpetuano i loro interessi, diventare i più forti, i più importanti, coloro che hanno maggiori seguaci. Lo scopo è uno solo: avere più persone possibili che seguono i loro Tratti.

Se un Patrono agisce in prima persona o in modo indiscriminato sa che scatenerà la reazione di Gradh o l'intervento di Atmos che gli impediranno un uso incontrollato e massivo dei suoi poteri direttamente sulla Terra, come fecero i primissimi Patroni. Questo raramente li ferma e la stessa natura e creature tutte vengono spesso influenzate dal volere dei Patroni.

Ed è così che nascono sempre più spesso aberrazioni, malattie sempre nuove, terre maledette dove non può crescere nulla, per non parlare di pazzie che spesso coinvolgono chi invece dovrebbe proteggere i cittadini.
E' una dura vita quella dell'uomo comune che continuamente deve affrontare siccità o alluvioni, morie di animali ed un meteo irregolare se non assurdo, orde di creature venute dal nulla che vogliono solo sterminare tutti,

Ad ogni passo deve guardarsi intorno perché non può mai sapere chi ha venduto l'anima ad un Patrono per vivere un giorno in più.

Ovunque i nemici più forti sono i Draghi, generati e convocati nella prima venuta da Tàhil, il primo servitore di Calicante, unico, forse, dei primi Patroni ad essere rimasto.
Fanno incursioni al solo scopo di portare distruzione e morte, seminare paura ed orrore.

Molte regioni stanno diventando incubatrici di razze oscure e malvagie, legioni di non morti guidate da potenti negromanti si ammassano sui confini, i Draghi addestrano i loro corrotti adepti e scure spire nel cielo promettono tempesta.

Come la Freten riuscì a catturare il Primo figlio ed a soggiogarlo è tuttora un mistero.

\subsection{Patroni}\index{Patroni}\hypertarget{patroni}{}\label{patroni}

\begin{enfasi}{
Conan: A quali dei preghi?

Subotai: Io prego ai quattro venti e tu?

Conan: Io prego Crom, ma solo raramente... lui non ascolta. (Conan il Barbaro, film 1982)

\medskip

Infatti, come il corpo senza lo spirito è morto, così anche la fede senza le opere è morta. (Giacomo Il Giusto 2, 26. NdA Riferendosi ai punteggi dei Tratti collegati al Patrono...)
}
\end{enfasi}

Le creature tutte, anche chi non usa la magia possono sentire l'influenza di questi Poteri, di questi Patroni.

Ogni personaggio per il suo modo di essere (giocare) e comportarsi ha almeno un Tratto in comune con un Patrono e nel corso delle avventure e sua evoluzione matura e potenzia queste convinzioni potrà sentire maggiormente l'influsso ed effetti di un Patrono.

Non è necessario che abbia giurato fedeltà ad un Patrono o che ne sia Seguace o Devoto, sentirà comunque l'influenza del Patrono e riceverà dei doni da esso.

Un Patrono è ben contento se le creature seguono i suoi dettami, Tratti, e dona a coloro che lo fanno dei piccoli poteri come riconoscimento per la fedeltà a lui riservata, volutamente o meno. I poteri indicati sotto \emph{Tratti in Comune} sono cumulativi. Se non indicato diversamente i poteri sono usabili 1 volta al giorno e costano 2 Azioni.
Quando è indicato un incantesimo questo viene manifestato senza Prove di Magia o penalità dovute ad armatura.

Ogni \textbf{Patrono predilige uno o più forme energetiche}, se sei Seguace puoi usare quell'energia nelle tue magie, se sei un Devoto invece i tuoi incantesimi useranno una della forme energetiche indicate. Sono indicate le \hyperlink{listeprivilegiate}{\emph{Liste Privilegiate}} (pag. \pageref{listeprivilegiate})\index{Liste Magia privilegiate}, ovvero liste nelle quale il Devoto ha dei vantaggi d'uso.

Le forme di Energia vengono distinte tra fonti positive, neutrali e negative, servono per inquadrare meglio il vostro padrone, pardon, il Patrono che servite.

Fate la somma delle energie, se positiva il Patrono si può considerare buono, se a valore zero il Patrono è neutrale, se a valore negativo il Patrono è malvagio, per comodità nella lista delle forme di energia è indicato se il Patrono è \textbf{B}uono, \textbf{N}eutrale o \textbf{M}alvagio.

Nella descrizione del Patrono troverete anche la sua \textbf{manifestazione}, ovvero cosa accade quando un personaggio agisce in maniera particolarmente e significativamente consona ai Tratti seguiti dal Patrono. L'effetto è puramente scenico e di circostanza ma lascia sempre colpito chiunque lo possa osservare, solitamente garantisce un punto avanzamento in un Tratto collegato al Patrono. Non è necessario essere Devoto o Seguace, basta aver seguito in maniera particolarmente \emph{epica} quel Tratto.

E' presente anche l'indicazione dell'\textbf{arma preferita} dal Patrono. Non ci sono vantaggi nell'usarla, a meno di avere la specifica \hyperlink{Il Patrono è la mia Arma}{Abilità} (pag. \pageref{Il Patrono è la mia Arma}), la scelta è puramente personale e lasciata alla devozione del personaggio.

Sotto l'indicazione dell'arma preferita c'è l'indicazione della Regola \index{Regola} ovvero il comportamento che il Devoto deve cercare di rispettare.

Un incantatore che si affida ad un Patrono, almeno \textbf{2 Tratti} in comune, diventa un \textbf{Devoto}. Se ha almeno \textbf{1 Tratto} in comune e si affida ad un Patrono allora si dice che è un \textbf{Seguace}. Il \textbf{Vantaggio} indicato è solo per il Devoto.

\begin{giocatore}[Devoti e Seguaci]
Essere Devoti o Seguaci è una scelta vostra, nessuno ve la impone. Dovete sentirla come una occasione di gioco di ruolo, come un arricchimento del personaggio e non una costrizione. Essere Devoti o Seguaci non significa essere proni al volere del Patrono, anzi, significa essere ancora più convinto di propri Tratti, della propria personalità. \textbf{Un Patrono non chiede preghiere, ma chiede di essere se stessi}.
\end{giocatore}

Il personaggio potrebbe anche non seguire alcun Patrono pur avendo più Tratti in comune oppure potrebbe essere un Devoto o Seguace non del Patrono con cui hai più Tratti in comune od i Tratti a punteggio più alto. La scelta è sempre del personaggio e della sua sensibilità.

Le capacità acquisite legate ai Tratti in comune sono indipendenti dall'essere un Devoto, Seguace o semplicemente ateo, rappresentano i doni del Patrono a chi segue i suoi Tratti.

Nulla vieta che un personaggio riceva più poteri da Patroni diversi! Ad alti livelli quando il personaggio ha un alto punteggio nei vari Tratti posseduti questo capiterà frequentemente. \index{Vantaggi}

\begin{narratore}[Adattarsi]
Il Narratore può comunque concedere l'essere Seguace o Devoto pur se i Tratti non collimano perfettamente. Su richiesta del giocatore ed a sua discrezione può valutare la somiglianza di alcuni Tratti del personaggio a quelli del Patrono e valutarli idonei per esserne un Seguace o Devoto. In queste situazioni è necessario comprendere come il giocatore inquadra il personaggio e capire non solo se i Tratti ma anche il sentimento del personaggio è affine al Patrono scelto.
\end{narratore}

%\begin{narratore}
%In termini di Gioco di Ruolo canonico i Patroni sono divinità minori, creature di altri piani estremamente potenti e come tali \emph{affrontabili}, vedi ad esempio scheda di Tàhil. I Patroni della Genesi invece potendo creare i Patroni sono Dei Maggiori.
%\end{narratore}\end{changemargin}

\medskip

\textbf{Tabella Energia - Elementi}\index[Tabelle]{Tabella degli Elementi}

\medskip

\noindent\begin{tabularx}{\linewidth}{lXX}
	\toprule
\rowcolor{gray!20}\textbf{Positivi} (+1) & \textbf{Neutrali} (0) & \textbf{Negativi} (-1)\\
\toprule
En. Positiva& Fuoco& En. Negativa\\
\rowcolor{gray!20}Luce & Freddo& Vuoto\\
& Suono & \\
\rowcolor{gray!20}& Elettricità &
\end{tabularx}

\medskip

Nessun Patrono è completamente buono o completamente malvagio. Come la natura di un uomo è sfaccettata anche a seconda delle occasioni, così un Patrono potrà avere anomali slanci positivi o negativi. Ricordati che i Patroni sono Dei ma generati dalle emozioni e desideri.

\subsubsection{Miracoli, Interventi e Prodigi}\index{Miracoli, Interventi e Prodigi}

In un mondo dove le divinità sono così capricciose, volubili e assetati di devoti fa loro gioco dimostrarsi generosi con coloro che possono poi diffondere i loro Tratti.

Un favore chiesto ad un Patrono ha sempre un prezzo non ovvio ne scontato. Il Narratore deve valutare attentamente la supplica del personaggio e giudicare se la richiesta è pertinente con i Tratti del Patrono, in caso positivo tirare 1d100 e fare meno della metà del punteggio più alto di Tratto in comune con il Patrono. Oppure decidere autonomamente secondo il corso dell'avventura.

\addcontentsline{toc}{subsection}{Elenco Patroni}

%\subsubsection{Elenco Patroni}\label{elencopatroni}\index{Elenco Patroni}

\subsubsection{Ljust}\label{ljust}\index{Empatia}\hypertarget{ljust}{}

\index{Ljust}

\begin{enfasi}{
Solo la luce che uno accende a se stesso, risplende in seguito anche per gli altri. (Arthur Schopenhauer)
}\end{enfasi}


La Dama della Luce, colei che irradia calore e amore. Generatrice delle pulsioni d'amore, protezione, gentilezza, gioia e perdono. Racchiude in sé l'aspetto protettivo di una madre, la forza e l'audacia di una combattente, la passionalità di una giovane amante, l'allegria e la ricerca del nuovo, la fantasia di una bambina. Ljust incarna la bellezza della vita ed ogni creatura che la contempla vede quella che per lei è la massima armonia e cade prona al suo fascino.

Ljust può essere scelta solo da un personaggio con 4 Tratti in comune con lei, fondamentalmente si nasce per essere Devoti di Ljust. Nel corso delle ere Ljust decise di selezionare, scegliere e premiare le creature che più mostravano in modo innato e profondo amore per la vita, curiosità per il nuovo, forza incrollabile, dedizione, fiducia, rispetto e cura degli altri donando loro i poteri e la possibilità di studiare e crescere come Allieve della Luce. Queste Allieve devono seguire la regola degli 8 Passi.

\begin{tikzpicture}[remember picture, overlay]
	\node[anchor=center, opacity=0.3] at (0.45\columnwidth,0)
	{\includegraphics[width=1\columnwidth]{immagini/patroni-stella_luminosa.pdf}};
\end{tikzpicture}

\begin{itemize}[leftmargin=*] \setlength{\itemsep}{0pt}
\item \textbf{Simbolo}: Una stella a 8 punte con 8 raggi luminosi
\item \textbf{Caratteristica}(Devoto): Saggezza o Carisma
\item \textbf{Tratti}: Compassionevole, Testardo, Coraggioso, Estroverso, Altruista, Leale, Paziente. Il Devoto di Ljust ha 4 Tratti in comune con il Patrono.
\item \textbf{Manifestazione}: luce dorata inonda l'incantatore.
\item \textbf{Somma dei Tratti in comune a 5 punti}: puoi lanciare l'incantesimo Luce come Reazione, 3 volte al giorno
\item \textbf{Somma dei Tratti in comune a 10 punti}: guadagni un +2 ai Tiri Salvezza su Tempra

\item \textbf{Somma dei Tratti in comune a 15 punti}: una armatura di luce ti protegge, guadagni un +2 a tutti i Tiri Salvezza e Difesa, l'effetto è permanente.
\item \textbf{Somma dei Tratti in comune a 20 punti}: puoi lanciare l'incantesimo Bagliore Solare. 1 volta al giorno.
\item \textbf{Energia/B} (Seguace/Devoto): Energia Positiva, Luce
\item \textbf{Vantaggio} (Devoto): Ogni volta che fai una Cura magica curi un Punto Ferita in più.
\item \textbf{Liste Magia Privilegiate}(Seguace/Devoto): Cura, Abiurazione
\item \textbf{Arma Preferita}: Spada Bastarda
\item \textbf{Regola}: Accettare l'invito ad un ballo
\end{itemize}

\medskip

\textbf{Gli 8 Passi delle Allieve}\index{8 Passi delle Allieve}\index{Allieve}

\medskip

Le Allieve della Luce sono una gruppo segreto di Devote che per totale affinità con Ljust hanno intrapreso il difficile percorso del bene e dell'amore. E' tra i gruppi più antichi fondati sulla Terra. Le Allieve, 99 come numero massimo, ma purtroppo spesso meno numerose, sono Devote di Ljust e devono seguire gli 8 Passi della Luce

\begin{enumerate}[leftmargin=*] \setlength{\itemsep}{0pt}
\item Ama e proteggi con tutta te stessa, con totale e sincera dedizione chi hai attorno a te.
\item Non lasciare che la tua inazione generi sofferenza.
\item Si un punto di paragone. Fai che la tua Luce elevi le persone che hai intorno e possano vedere in te speranza, serenità, calma, protezione e sicurezza.
\item Usa l'intelligenza, la furbizia e l'arguzia. Si lungimirante e risoluta nell'azione.
\item La tua opera è per il bene comune. Fa che la tua Luce sia sempre alta ed intensa.
\item Non cercare altra Luce se non la tua e quella delle tue sorelle.
\item Sii luminosa ma non accecare chi è intorno a te.
\item Sii la differenza tra la disperazione e la speranza.
\end{enumerate}

Le Allieve hanno costruito un ballo armonioso trasformando in danza i passi della loro Regola.

Esistono anche Allieve di altro genere, rari ma storicamente accertati.

\subsubsection{Calicante}\index{Calicante}\label{calicante}\index{Egoismo}\hypertarget{calicante}{}

\begin{enfasi}{
La superstizione è la religione degli spiriti deboli. (Edmund Burke)
}\end{enfasi}

E' oscuro, gelido e arrabbiato. Racchiude in sé odio, violenza, distruzione, vendetta e perenne insoddisfazione. Raccoglie la personalità capricciosa e scontenta di un bambino, la noia violenta e sadica di un giovane uomo, la forza distruttiva di un uragano e la rabbia di un combattente che non ha più nulla da perdere. Calicante solo con la presenza mette a disagio, ti fa sentire in pericolo, affascina ma con le armi della paura e della incostanza.

Calicante può essere scelto solo dai personaggi che hanno 4 Tratti in comune con lui. I suoi Devoti sono i migliori assassini, sua professione più affine. Coloro che mostrano il maggiore sprezzo del pericolo e della vita altrui. I suoi prediletti sono coloro che sono temuti, odiati, coloro che sono violenti e crudeli ma mortalmente efficienti e decisivi in ogni situazione di combattimento.

\begin{itemize}[leftmargin=*] \setlength{\itemsep}{0pt}
\item \textbf{Simbolo}: Un turbine nero
\item \textbf{Caratteristica}: Forza o Destrezza
\item \textbf{Tratti}: Ambizioso, Disonesto, Vendicativo, Cinico, Dissoluto, Arrogante, Avaro. Il Devoto di Calicante ha 4 Tratti in comune con il Patrono
\item \textbf{Manifestazione}: spada grondante di sangue nero
\item \textbf{Somma dei Tratti in comune a 5 punti}: Puoi lanciare l'incantesimo Oscurità. Una volta al giorno
\item \textbf{Somma dei Tratti in comune a 10 punti}: La tua arma si ammanta di ombra. Guadagni un +2 al Tiro per Colpire e +1d4 di danno da Vuoto per 2d6 round, Una volta al giorno.
\item \textbf{Somma dei Tratti in comune a 15 punti}: Crei 4 dardi di Vuoto, ogni dardo fa 2d6 di danno, colpisce automaticamente entro 18 metri. Una volta al giorno.
\item \textbf{Somma dei Tratti in comune a 20 punti}: Crei una zona di energia protettiva attorno a te nel raggio di 3 metri, dimezzi tutto il danno che ricevi, non è possibile recuperare Punti Ferita nell'area. Durata 10 minuti consecutivi, 1 volta al giorno.
\item \textbf{Energia/M}: Energia Negativa, Vuoto
\item \textbf{Vantaggio}: +2 TS contro Lista Incantamento
\item \textbf{Liste Magia Privilegiate}: Fuoco, Necromanzia
\item \textbf{Arma Preferita}: Machete
\item \textbf{Regola}: Mai lasciare impunita una offesa diretta
\end{itemize}

\subsubsection{Atmos}\index{Atmos}\label{atmos}\index{Indifferenza}\hypertarget{atmos}{}

\begin{enfasi}{
Che cos'è dunque il tempo? Se nessuno me lo chiede, lo so; se voglio spiegarlo a chi me lo chiede, non lo so più. (Agostino da Ippona)
}\end{enfasi}

Il custode del Tempo e della Torre dell'Orologio, come ha avviato il tempo e la creazione dei nuovi Patroni così fermerà la sfida fra loro ed i Patroni sopravvissuti saranno giudicati, le loro opere valutate e Ljust o Calicante ne trarranno giovamento. Come una sfida da una singola moneta di rame nuovi Patroni, nuovi ideali saranno creati e noi, piccole creature vedremo nascere nuove civiltà e regni fiorenti. La storia è poco nota, solo i pochi Devoti di Atmos, scribi e studiosi della biblioteca del Tempo, conoscono il segreto e lo scorrere del tempo e della gara, gli altri, ignoranti, vivranno il loro tempo con un padrone sicuramente guidato da un Patrono.

Atmos, il Patrono del Tempo è il custode della storia, è colui che tiene traccia degli infiniti e più mondi che sono stati creati.

Atmos ha il potere unico e riservato solo a lui di poter bandire dal creato un Patrono qualora questo diventasse troppo forte e minacciasse Calicante e Ljust. Atmos ha già usato questo potere in passato. Atmos sia per la sua natura totalmente neutrale sia per il suo ruolo non si è mai schierato.

Tutti i Patroni temono Atmos per il suo potere, il più terribile per loro, ovvero il loro alienamento, l'oblio, la dimenticanza, l'essere distolti dal tempo e dalla sfida.

Per essere un Devoto di Atmos al momento del rito è necessario che il futuro Devoto possieda almeno quattro Tratti in comune con lui, amare la storia e la conoscenza.

Vestito di un morbido saio marrone e calzari di cuoio si muove tra gli infiniti scaffali della Biblioteca del Sapere con sempre uno strano misuratore del tempo appeso alla vita.


\begin{itemize}[leftmargin=*] \setlength{\itemsep}{0pt}
\item \textbf{Simbolo}: Un libro bianco con un orologio da taschino appoggiato sopra
\item \textbf{Caratteristica}: Intelligenza o Saggezza
\item \textbf{Tratti}: Indeciso, Prudente, Intransigente, Paziente, Vendicativo, Curioso, Avaro. Il Devoto di Atmos ha 4 Tratti in comune con il Patrono.
\item \textbf{Manifestazione}: l'incantesimo si sviluppa come a rallentatore, è solo un effetto illusorio
\item \textbf{Somma dei Tratti in comune a 5 punti}: Conosci sempre la data esatta e l'ora.
\item \textbf{Somma dei Tratti in comune a 10 punti}: Hai una intuizione innata per la conoscenza. Hai +1d6 alle prove di Conoscenza
\item \textbf{Somma dei Tratti in comune a 15 punti}: Puoi lanciare l'incantesimo Globo di Invulnerabilità, 1 volta al giorno.
\item \textbf{Somma dei Tratti in comune a 20 punti}: Ogni qual volta che devi fare una prova di Arcana puoi prendere il 18 come prendessi 10
\item \textbf{Energia/N}: Suono, Freddo
\item \textbf{Vantaggio}: Sai sempre che ora è
\item \textbf{Liste Magia Privilegiate}: Divinazione, Abiurazione
\item \textbf{Arma Preferita}: Mazza leggera
\item \textbf{Regola}: Non accettare o dare un compenso se non meritato
\end{itemize}

\subsubsection{Lynx}\index{Lynx}\label{lynx}\index{Responsabilità}\hypertarget{lynx}{}

\begin{enfasi}{
Le persone non fanno i viaggi, sono i viaggi che fanno le persone. (John Steinbeck)}
\end{enfasi}

Patrono dei Portali, è sceglibile solo da personaggi che abbiano almeno 3 Tratti in comune. E' il primo Patrono generato da Ljust e Calicante, creato per proteggere la Terra dagli attacchi esterni.

Serio, occhi gelidi di un azzurro chiarissimo è il Custode dei Portali e di ciò che è Oltre. Letale guardiano per chi cerca di passarli senza permesso, guida attenta per chi chiede il suo aiuto ed il suo permesso. Si fa scudo delle sue cicatrici per allontanare tutti. E' il solitario controllore del mondo.

I suoi Devoti sono i viaggiatori per eccellenza, coloro che presidiano e proteggono la Terra da ciò che è alieno, da ciò che potrebbe disturbare la creazione.

\begin{itemize}[leftmargin=*] \setlength{\itemsep}{0pt}
\item \textbf{Simbolo}: Un portale sull'oscurità
\item \textbf{Caratteristica}: Destrezza o Intelligenza
\item \textbf{Tratti}: Testardo, Coraggioso, Cinico, Intransigente, Vendicativo, Estroverso, Vanitoso
\item \textbf{Manifestazione}: come se il panorama non avesse più orizzonte
\item \textbf{Somma dei Tratti in comune a 5 punti} punti: Una volta al giorno puoi eseguire una Azione di Movimento in più
\item \textbf{Somma dei Tratti in comune a 10 punti}: Puoi lanciare Porta Dimensionale una volta al giorno
\item \textbf{Somma dei Tratti in comune a 15 punti}: Puoi lanciare l'incantesimo Esilio, 1 volta al giorno, DC 30.
\item \textbf{Somma dei Tratti in comune a 20 punti}: Puoi teletrasportarti per 500km al giorno (anche più teletrasporti o teletrasportati purché la somma totale non superi 500km)
\item \textbf{Energia/N}: Fuoco, Elettricità
\item \textbf{Vantaggio}: Sei considerato di una taglia maggiore quando devono spostarti o spingerti
\item \textbf{Liste Magia Privilegiate}: Evocazione, Acqua
\item \textbf{Arma Preferita}: Spada Corta
\item \textbf{Regola}: Non lasciare un ambiente non esplorato
\end{itemize}

\subsubsection{Gradh}\index{Gradh}\label{gradh}\index{Coraggio}\hypertarget{gradh}{}

\begin{enfasi}
L'uomo che ha cessato di avere paura ha cessato di preoccuparsi. (Francis Herbert Bradley)
\end{enfasi}

Il primo Patrono creato da Atmos sotto la guida di Ljust e l'influenza di Calicante.

Gradh racchiude in sé l'istinto innato alla protezione, alla difesa ed alla cura propri di Ljust. Gradh è quanto di più simile e profondamente legato a Ljust sia stato generato. Lui è equilibrio, razionalità ed empatia.
Dove vi è difesa, cura e protezione vi è Gradh.

Ma Calicante non poteva permettere la creazione di un Patrono totalmente votato ad Ljust e così infuse in Gradh la freddezza della vendetta e la furia della rabbia. Ecco che allora Gradh nell'atto di difendere l'umanità, spesso la deve in primis proteggere da sé stesso.

Gradh non ama sfidare apertamente Cattalm perché sa che farebbe esattamente il suo gioco, ecco che con astuzia cerca di attirarlo nel suo terreno di gioco, dove nessuna vita sarà in pericolo e lì da sfoggio a della sua superiorità strategica e di combattimento.

Passionale e freddo è forse il Patrono più umano del pantheon attuale. Il suo sguardo carismatico e protettivo può divenire freddo e tagliente quando è preda della furia della battaglia o della vendetta. Gradh ama studiare il mondo attorno a sé e passare inosservato. Spesso si nasconde fra la gente e \emph{vive} la sua vita umana, ma non si lascia avvicinare veramente da nessuno. Gradh attira a sé con la stessa facilità con cui lui allontana da se.

Il Devoto di Gradh è fiero ed orgoglioso, indomito e protettivo, ed addolorato, perché per quanto si sforzi di portare equilibrio e pace il male continua sempre a prosperare.

\begin{itemize}[leftmargin=*] \setlength{\itemsep}{0pt}
\item \textbf{Simbolo}: Uno scudo con incise sopra due spirali intrecciate.
\item \textbf{Caratteristica}: Forza
\item \textbf{Tratti}: Coraggioso, Vanitoso, Arrogante, Gentile, Invidioso, Leale, Sospettoso
\item \textbf{Manifestazione}: due spire una nera come ombra ed una lucente come scintilla circondano la tua arma intrecciandosi
\item \textbf{Somma dei Tratti in comune a 5 punti} punti: Puoi lanciare l'incantesimo Cura Ferite da 3 Punti Magia, ma ti causa 1d6 di danno. 1 volta al giorno
\item \textbf{Somma dei Tratti in comune a 10 punti}: Per 10 minuti consecutivi hai un bonus di +1d6 Tiro Salvezza su Riflessi e Tempra. Una volta al giorno.
\item \textbf{Somma dei Tratti in comune a 15 punti}: Emani un aura che concede a tutti i tuoi compagni entro raggio 3 metri un +2 Tiri Salvezza. Una volta al giorno, per 30 minuti consecutivi
\item \textbf{Somma dei Tratti in comune a 20 punti}: Lanci l'incantesimo Palla di Fuoco. L'incantesimo causa 60 di danno da energia negativa. DC 25 Riflessi per dimezzare. 2 volte al giorno
\item \textbf{Energia/N}: Energia Positiva - Energia Negativa
\item \textbf{Vantaggio}: +2 Consapevolezza
\item \textbf{Liste Magia Privilegiate}: Abiurazione, Invocazione
\item \textbf{Arma Preferita}: Mazza flangiata
\item \textbf{Regola}: Non permettere ad un immondo di camminare sulla Terra
\end{itemize}

\subsubsection{Atherim}\index{Atherim}\label{atherim}\index{Onestà}\hypertarget{atherim}{}

\begin{enfasi}{
Colui al quale confidate il vostro segreto, diventa il padrone della vostra libertà. (François de La Rochefoucauld)

\medskip

Non c'è nulla di nascosto che non sarà svelato, né di segreto che non sarà conosciuto. (Luca, 12, 1-7)
}\end{enfasi}

Il Patrono custode. Molti vedono nel seno generoso di Atherim un segno di voluttà e passione. Si lasciano incantare dalla sua procace bellezza e non vedono gli occhi di cristallo che incutono timore a chi osa anche solo pensare di avvicinarla.

Atherim è la custode dei sogni e delle speranze, colei alla quale affidare, come ad una madre, i desideri. E' il Patrono dei Bambini, dei Segreti e delle Levatrici.

Dal sorriso allegro e dall'animo buono sarà sempre pronta ad aiutarti a realizzare i tuoi sogni. E come una madre Atherim protegge e custodisce i segreti e le passioni. Atherim è muta. E' colei che custodisce per sempre, dentro il suo animo i segreti.

Il Devoto di Atherim si prende a cuore coloro che hanno fatto una promessa, punisce chi le infrange e chi svela i segreti. Molti Devoti di Atherim sono diplomatici, notai e levatrici.

\begin{itemize}[leftmargin=*] \setlength{\itemsep}{0pt}
\item \textbf{Simbolo}: Una mano di donna guantata che tiene un'ampolla ricca di flussi
\item \textbf{Caratteristica}: Saggezza
\item \textbf{Tratti}: Sospettoso, Compassionevole, Altruista, Intransigente, Coraggioso, Entusiasta, Vanitoso
\item \textbf{Manifestazione}: un silenzio sereno e tranquillizzante cala attorno all'incantatore
\item \textbf{Somma dei Tratti in comune a 5 punti}: Puoi aggiungere 1d6 ad un Tiro salvezza dopo averlo tirato ma prima di sapere se ha avuto successo o meno. Una volta al giorno, come Reazione.
\item \textbf{Somma dei Tratti in comune a 10 punti}: Guadagni 30 Punti Ferita temporanei. Durata 1 ora, una volta al giorno, come Azione Immediata.
\item \textbf{Somma dei Tratti in comune a 15 punti}: Puoi lanciare l'incantesimo \hyperlink{Zona di Verità}{Zona di Verità} 3 volte al giorno, senza Tiro Salvezza.
\item \textbf{Somma dei Tratti in comune a 20 punti}: Ogni pozione che bevi ha il doppio come durata o effetto se immediata.
\item \textbf{Energia/B}: Energia Positiva, Elettricità
\item \textbf{Vantaggio}: Riduci il Sanguinamento di 1 a fine round
\item \textbf{Liste Magia Privilegiate}: Ammaliamento
\item \textbf{Arma Preferita}: Pugnale
\item \textbf{Regola}: Non rivelare un segreto confidato
\end{itemize}

\subsubsection{Belevon}\index{Belevon}\label{belevon}\index{Invidia}\hypertarget{belevon}{}

\begin{enfasi}{
Dove c'è un uomo, c'è anche la bugia. (Robert Louis Stevenson)

\medskip

Nessuno ha una memoria tanto buona da poter essere un perfetto bugiardo. (Abraham Lincoln)
}\end{enfasi}

E' il Patrono che meglio incarna la bugia e la finzione al fine di un proprio tornaconto. Lui ama solo se stesso. E' un narcisista che si circonda solo di persone che lo assecondano e lo adulano. Aborrisce la solitudine ma allo stesso tempo odia essere toccato da qualcuno.

E' sempre alla ricerca di nuove cose, di oggetti meravigliosi che scambia e ricambia con altri oggetti. Gli piace discutere e mercanteggiare fino ad ottenere sempre quello che vuole a costo della vita altrui.

Belevon è un orrendo e deforme Patrono adorato da abiette creature delle caverne più profonde, invidiose di quanto posseduto dagli altri.

Il Devoto di Belevon è ben descritto da un uomo lucertola circondato da chincaglieria e resti umani.

\begin{itemize}[leftmargin=*] \setlength{\itemsep}{0pt}
\item \textbf{Simbolo}: Una gabbia dorata
\item \textbf{Caratteristica}: Intelligenza
\item \textbf{Tratti}: Invidioso, Ambizioso, Dissoluto, Disonesto, Compassionevole, Paziente, Altruista
\item \textbf{Manifestazione}: come se le sbarre dorate di una gabbia si intrecciassero attorno all'incantatore
\item \textbf{Somma dei Tratti in comune a 5 punti} punti: Puoi lanciare l'incantesimo Prestidigitazione, 3 volte al giorno.
\item \textbf{Somma dei Tratti in comune a 10 punti}: Acquisisci la capacità lanciare l'incantesimo \hyperlink{Immagine Maggiore}{Immagine Maggiore} una volta al giorno.
\item \textbf{Somma dei Tratti in comune a 15 punti}: Puoi lanciare l'incantesimo Allucinazione Mortale. 1 volta al giorno
\item \textbf{Somma dei Tratti in comune a 20 punti}: Toccando un oggetto vieni a conoscenza per sommi capi della storia di chi l'ha creato. Una volta al giorno. Costa 3 Azioni.
\item \textbf{Energia/N}: Fuoco, Suono
\item \textbf{Vantaggio}: Vedi Abilità Fortunato
\item \textbf{Liste Magia Privilegiate}: Illusione
\item \textbf{Arma Preferita}: Picca leggera
\item \textbf{Regola}: Contratta sul prezzo che sia in acquisto o vendita
\end{itemize}

\subsubsection{Cattalm}\index{Cattalm}\label{cattalm}\index{Cinismo}\hypertarget{cattalm}{}

\begin{enfasi}{
Non è l'essere arrabbiati che conta, è l'essere arrabbiati per le cose giuste. Le dissi: guardalo dalla prospettiva darwiniana. La rabbia serve a renderti efficiente. Questa è la sua funzione per la sopravvivenza. Ecco perché ti è stata data. Se ti rende inefficiente, mollala come una patata bollente. (Philip Roth)
}\end{enfasi}

Generato direttamente da Calicante, come risposta alla creazione di Gradh da parte di Ljust, è pura distruzione, caos ed entropia. Cattalm si prefigura il solo scopo di distruggere, portare caos e malattie, terremoti ed alluvioni.

Cattalm è tra i pochi Patroni che osa sfidare apertamente Gradh e lo fa con gioia perché sa che la loro battaglia altro non farà che portare ulteriore distruzione. Cattalm accetta ed invita ad essere suo Devoto ogni creatura capace di odio, capace di distruggere e ferire. Molti suoi Devoti sono creature mostruose o aberrazioni.

Cattalm invece è tra i Patroni più meravigliosi, con una candida pelle lucente, ali di piuma soffice ed una leggera armatura argentata. Per quanto i lineamenti delicati ne facciano un essere bellissimo per quanto ambisca alla distruzione.

Cattalm adora il caos che manifesta nei modi più violenti con terremoti, alluvioni, maremoti, malattie se non direttamente piogge infuocate. Non agisce quasi mai direttamente ma lascia che caos e distruzione lavorino per lui.

Ljust non poteva non intervenire nella creazione di un Patrono così esplicitamente malvagio e, di nascosto da Calicante, instillò in Cattalm l'amore e protezione per i bambini. Cattalm distrugge, avvelena, indebolisce ma non i bambini, neanche indirettamente, piuttosto si attiva lui stesso per annullare i malefici causati dalla sua natura.

Ogni qual volta succede una calamità si suole dire che \emph{Cattalm ha battuto il piede}.

\begin{itemize}[leftmargin=*] \setlength{\itemsep}{0pt}
\item \textbf{Simbolo}: Un'onda gigante che sovrasta la costa
\item \textbf{Caratteristica}: Forza
\item \textbf{Tratti}: Cinico, Arrogante, Ambizioso, Intransigente, Dissoluto, Sospettoso, Paziente
\item \textbf{Manifestazione}: il rombo del tuono
\item \textbf{Somma dei Tratti in comune a 5 punti} punti: Attraverso le tue armi indebolisci l'avversario designato. A seguito di un colpo critico puoi aumentare di un livello l'affaticamento. Una volta al giorno come Reazione.
\item \textbf{Somma dei Tratti in comune a 10 punti}: Il tuo tocco imputridisce cibo (fino a 50kg/Ingombro 10) e acqua (un cubo con uno spigolo di 10 m). Una volta al giorno
\item \textbf{Somma dei Tratti in comune a 15 punti}: Il tuo sguardo accieca di collera. Lanci l'incantesimo Confusione, ma l'unico risultato possibile è che i target attaccano dei soggetti a caso. DC 25. 1 volta al giorno
\item \textbf{Somma dei Tratti in comune a 20 punti}: Lanci l'incantesimo Cono di Freddo 60 di danno, ma il danno è da Vuoto. DC 25. Una volta al giorno
\item \textbf{Energia/M}: Energia Negativa, Vuoto
\item \textbf{Vantaggio}: aumenta di 10 i PF necessari ad ucciderti
\item \textbf{Liste Magia Privilegiate}: Fuoco
\item \textbf{Arma Preferita}: Ascia da battaglia
\item \textbf{Regola}: Non fare buone azioni senza un tornaconto
\end{itemize}

\subsubsection{Efrem}\index{Efrem}\label{efrem}\index{Rispetto}\hypertarget{efrem}{}

\begin{enfasi}{
Non deviare dalla natura ed il formarci sulle sue leggi e sui suoi esempi, è sapienza. (Lucio Anneo Seneca)
}\end{enfasi}

E' il Patrono di chi fa della natura la propria casa. Incarna in sé gli aspetti più puri della natura stessa, aggressivo come solo i felini più letali sanno essere; ma anche selvaggio come le radure più nascoste e rigorosa come solo la natura può essere.

Efrem si prefigge di difendere la Natura dalla contaminazione dell'uomo, da questa specie infestante che distrugge tutto ciò che incontra.

I Devoti di Efrem. chiamati anche druidi\index{Druido}, sono legati maggiormente all'elemento naturale. Manipolano la magia principalmente elementale e si difendono o attaccano usando anche animali e creature naturali. Si dice che i più potenti costringano anche i Draghi alla ubbidienza.

I Devoti di Efrem hanno l'obiettivo supremo di proteggere gli animali e le piante, i luoghi e tutto ciò che è naturale e non artificiale. Solitamente solitario e scontroso non riesce a capire il perché dell'odio che, dal suo punto di vista, l'uomo scarica sulla Terra.

Un Devoto di Efrem rispetta la vita come la morte, nel processo naturale che è l'evoluzione ed il ciclo vitale. A volte decide di stabilirsi in un certo ambiente e lo elegge come suo territorio e come fosse la sua casa lo protegge. Altre volte decide di essere ramingo ed intervenire in tutto il mondo per proteggere le sue amate piante ed animali.

Nelle terre più desolate, nelle regioni più naturali i Devoti di Efrem costruiscono utopie tra umanoidi e animali, dove l'equilibrio viene mantenuto con il sangue di qualsiasi che si ribelli al loro volere.

\begin{itemize}[leftmargin=*] \setlength{\itemsep}{0pt}
\item \textbf{Simbolo}: Una staffa con un rampicante attorcigliato attorno
\item \textbf{Caratteristica}: Costituzione
\item \textbf{Tratti}: Leale, Indeciso, Prudente, Impulsivo, Testardo, Paziente, Ambizioso
\item \textbf{Manifestazione}: spire di foglie avvolgono l'arma
\item \textbf{Somma dei Tratti in comune a 5 punti} punti: Il tuo tocco rende docili gli animali non magici. Tiro Salvezza su Volontà 20 per resistere. 3 volte al giorno. Costo 2 Azioni.
\item \textbf{Somma dei Tratti in comune a 10 punti}: Guadagni un +1d6 a tutte le prove di Sopravvivenza che si effettuano in un ambiente naturale.
\item \textbf{Somma dei Tratti in comune a 15 punti}: Puoi lanciare l'incantesimo Bacche Benefiche 1 volta al giorno. Ogni bacca cura 1d6 Punti Ferita e rimuove le malattie o veleni non magici.
\item \textbf{Somma dei Tratti in comune a 20 punti}: Il tuo tocco è quello del padrone. Puoi ammansire creature anche magiche come Aberrazioni o Draghi che tocchi. Tiro Salvezza su Volontà DC 30. Una volta al giorno. Costo 2 Azioni
\item \textbf{Energia/N}: Elettricità, Suono
\item \textbf{Vantaggio}: +1d6 a Gestire Animali
\item \textbf{Liste Magia Privilegiate}: Animali e Piante ed una Liste Magia Elementale.
\item \textbf{Arma Preferita}: Bastone
\item \textbf{Regola}: La Natura è sempre la tua prima scelta
\end{itemize}

\subsubsection{Erondil}\index{Erondil}\label{erondil}\hypertarget{erondil}{}\index{Superbia}

\begin{enfasi}{
I buoni ragionamenti sono più forti di due mani robuste. (Sofocle)
}\end{enfasi}

Patrono di Terra e Aria, Erondil è il Signore degli elementi più concreti e razionali. Colui che dotato di infinito potere e razionalità concede ai suoi Devoti il potere della manipolazione della terra, il dono di creare dal semplice fango costruzioni gigantesche e di millenaria forza. Conclude le sue opere con attenzione e precisione.
Pur con fatica perché se il risultato finale non lo soddisfa scatena i suoi fulmini per distruggerlo all'istante. Perfezionista ed incontentabile, difficilmente qualcosa è esattamente come lui se la immaginava.

Ordinato ed esuberante è il signore delle tempeste, dei tuoni e dei fulmini, dei terremoti e distruzioni. Ama circondarsi del fragore del tuono, del rombo della terra che si sgretola. Sa essere distruttivo verso coloro che non rispettano la Terra.
Ha braccia e petto ricoperti da tatuaggi quasi argentei che narrano le leggende della Terra e dell'Aria.

I Devoti di Erondil sono gli ingegneri dell'impossibile.

\begin{itemize}[leftmargin=*] \setlength{\itemsep}{0pt}
\item \textbf{Simbolo}: un castello di sabbia con un fulmine sopra
\item \textbf{Caratteristica}: Saggezza
\item \textbf{Tratti}: Arrogante, Vendicativo, Ambizioso, Compassionevole, Entusiasta, Leale, Avaro
\item \textbf{Manifestazione}: suono di tempesta e rombo di frana
\item \textbf{Somma dei Tratti in comune 5 punti}: Non temi più le cadute. Puoi lanciare l'incantesimo Caduta Piuma 3 volte al giorno, solo su di te.
\item \textbf{Somma dei Tratti in comune a 10 punti}: Il tuo tocco plasma la pietra. Puoi lanciare l'incantesimo Passa Porta 1 volta al giorno.
\item \textbf{Somma dei Tratti in comune a 15 punti}: Puoi scagliare l'incantesimo Fulmine dalle tue mani. Tiro Salvezza Riflessi DC 30 per dimezzare. Costo 2 Azioni.
\item \textbf{Somma dei Tratti in comune a 20 punti}: Sei in grado di creare una fossa profondissima (1km) sotto il tuo avversario (taglia fino a grande). Tiro Salvezza Riflessi 30 o cadere ogni round di 100 metri. Una volta al giorno. Dopo 1 minuto la fossa si chiude con chi c'è dentro. Costo 2 Azioni.
\item \textbf{Energia/N}: Suono, Elettricità
\item \textbf{Vantaggio}: Riduzione 10 ai danni da Elettricità
\item \textbf{Liste Magia Privilegiate}: Aria, Terra
\item \textbf{Arma Preferita}: Martello da guerra
\item \textbf{Regola}: Non devi consentire la distruzione di monumenti architettonici
\end{itemize}

\subsubsection{Gaya}\index{Gaya}\label{gaya}\hypertarget{gaya}{}\index{Generosità}

\begin{enfasi}{
{Splendore del giorno concluso, che mi sollevi e mi colmi,

ora profetica, ora che il passato riadduci!

E mi gonfi la gola, te, divino egualitarissimo,

voi, terra e vita finché brilli l'ultimo raggio, io canto. (Song at Sunset, Walt Whitman)}
}\end{enfasi}

Patrono di Acqua e Fuoco, nelle profondità della terra, dove acqua e lava si incontrano, Gaya si diverte a dipingere. Adora circondarsi dei flussi di fuoco e acqua quasi a creare una danza in mezzo a loro. Adora i suoni della natura, l'infrangersi delle onde sugli scogli, il cadere delle gocce di pioggia sull'acciottolato, il borbottare di un fuoco scoppiettante.

Dipinge mescolando il caldo ed il freddo. L'acqua cristallina ed impetuosa al fuoco intrigante ed ardente. Gelosa del bello e delle arti tiene tutte le sue opere al sicuro in un ordine quasi maniacale e protette. Da vera artista utilizza gli elementi per far risplendere le meraviglie della natura. Gaya è la pittrice di tramonti e delle tempeste.

I Devoti di Gaya sono artisti volubili e sopra le righe. Sono coloro che ricreano la magia dell'alba o del tramonto o del mare in tempesta nelle loro opere, sono coloro che mettono poesia e follia nella normalità.

Ma Gaya ha anche un lato molto più subdolo e violento, una vena di follia malvagia che adora portare distruzione con fiamme e acqua. Nelle profondità delle caverne creature affini all'acqua od al fuoco adorano Gaya e uccidono chiunque non sia d'accordo con loro.

\begin{narratore}[Gaia ed Erondil]
\textbf{Gaia} ed \textbf{Erondil} sono come le due facce della stessa medaglia e sovraintendono agli elementi, Gaia acqua e fuoco ed Erondil Aria e Terra; agiscono come espressione diretta dei Patroni maggiori, sono piccole manifestazione del loro immenso potere.
\end{narratore}

\begin{itemize}[leftmargin=*] \setlength{\itemsep}{0pt}
\item \textbf{Simbolo}: un pennello sul cielo
\item \textbf{Caratteristica}: Intelligenza
\item \textbf{Tratti}: Altruista, Gentile, Sospettoso, Cinico, Invidioso, Disonesto, Arrogante
\item \textbf{Manifestazione}: spire di fuoco e acqua avvolgono all'incantatore
\item \textbf{Somma dei Tratti in comune a 5 punti} punti: Puoi creare fino a 5 litri di acqua o 1 litro di liquore di buona qualità. Una volta al giorno. Costo 2 Azioni.
\item \textbf{Somma dei Tratti in comune a 10 punti}: Il tuo metabolismo non teme il freddo. Resisti al Danno magico da freddo e sei immune a quello naturale.
\item \textbf{Somma dei Tratti in comune a 15 punti}: Puoi respirare sott'acqua come respiri l'aria. Resisti al danno da fuoco non magico
\item \textbf{Somma dei Tratti in comune a 20 punti}: Generi una pioggia di fuoco. Lanci l'incantesimo Colpo Infuocato, DC 25 una volta al giorno. Resisti il danno da fuoco magico.
\item \textbf{Energia/N}: Freddo, Fuoco
\item \textbf{Vantaggio}: Riduzione 10 ai danni da Fuoco
\item \textbf{Liste Magia Privilegiate}: Acqua, Fuoco
\item \textbf{Arma Preferita}: Tridente
\item \textbf{Regola}: Non impedirti di ascoltare della buona musica
\end{itemize}

\subsubsection{Krondal}\index{Krondal}\label{krondal}\index{Intransigenza}\hypertarget{krondal}{}

\begin{enfasi}{
La libertà, Sancio, è uno dei più preziosi doni che i cieli abbiano mai dato agli uomini; né i tesori che racchiude la terra né che copre il mare sono da paragonare ad essa; per la libertà, come per l'onore, si può e si deve mettere a repentaglio la vita. (Miguel de Cervantes)
}\end{enfasi}

Krondal il folle, Krondal l'assassino, Krondal il salvatore.

Questi e molti altri sono gli appellativi di Krondal, il Patrono che non potrai mai veramente capire completamente.

Krondal abbraccia lo spirito anarchico e libero nella maniera più assoluta. Tutti secondo Krondal devono fare solo ciò che vogliono.

Il suo motto è \emph{Nessuno conosce Nessuno} perché non puoi sapere il futuro e quello che ti aspetta.

Krondal nutre un profondo rispetto per la libertà e non può criticare le scelte, estreme o meno che vengano fatte, eppure per suo dettato divino perseguita intransigente per portare giustizia.

Un Devoto di Krondal è tipicamente una guardia del corpo, un protettore, lo sceriffo al quale non interessano i motivi della scelta ma che sa giudicare le azioni compiute.

\begin{itemize}[leftmargin=*] \setlength{\itemsep}{0pt}
\item \textbf{Simbolo}: Una spada tenuta verticalmente davanti a se
\item \textbf{Caratteristica}: Carisma
\item \textbf{Tratti}: Intransigente, Vanitoso, Arrogante, Sospettoso, Paziente, Ambizioso, Testardo
\item \textbf{Manifestazione}: il mantello o veste del Devoto diventa pulito e lucente
\item \textbf{Somma dei Tratti in comune a 5 punti} punti: Maledici il tuo avversario. Lanci una volta al giorno l'incantesimo \hyperlink{Scagliare Maledizione}{Scagliare Maledizione}. DC 20 per resistere.
\item \textbf{Somma dei Tratti in comune a 10 punti}: Non vuoi essere legato o ammanettato. Due volte al giorno puoi lanciare solo su te stesso Libertà di Movimento.
\item \textbf{Somma dei Tratti in comune a 15 punti}: La tua presenza toglie la vista agli avversari. Designa fino a 6 creature entro 9 metri, queste devono fare un Tiro Salvezza Tempra a DC 30 o essere ciechi solo nei tuoi confronti per 1d4 round.
\item \textbf{Somma dei Tratti in comune a 20 punti}: La tua arma è più efficace contro i nemici. Ogni creatura colpita deve fare un Tiro Salvezza Volontà DC 20 o rimanere paralizzata per 3 round. Una volta che la creatura riesce nel Tiro Salvezza non può essere più influenzata per le successive 24 ore. Una volta la giorno, attivare l'abilità costa 1 Azione e dura 1 minuto.
\item \textbf{Energia/B}: Energia Positiva, Fuoco
\item \textbf{Vantaggio}: +2 TS su Tempra
\item \textbf{Liste Magia Privilegiate}: Abiurazione
\item \textbf{Arma Preferita}: Spada lunga
\item \textbf{Regola}: Non permettere soprusi
\end{itemize}

\subsubsection{Ledyal}\index{Ledyal}\index{Laydel}\label{ledyal}\label{laydel}\hypertarget{ledyal}{} \hypertarget{laydel}{}\index{Gratitudine}\index{Rancore}

\begin{enfasi}{
L'anima di un uomo è immortale e incorruttibile. (Platone)

\medskip

Non è il corpo che mi definisce (anonima creatura)
}\end{enfasi}

E' il Patrono senza un volto preciso, senza una voce se non un canto. Mutevole nel corpo e senza una definizione chiara del suo essere. Si manifesta con un lungo mantello color rosso fuoco dal tessuto fatto da mille farfalle. Il suo tocco è vita e pace, protegge chi necessita dei suoi favori indipendentemente dal fatto che li chieda o meno. Desidera un mondo senza sofferenza, con solo felicità ed armonia. Sospettoso/a e profondamente introverso non crede a coloro che gli danno ragione. Ha il cuore pieno di vita e di bontà ma non ha un corpo con cui amare.

Ledyal ha anche una sorella gemella, o forse un'altra personalità, o forse sono lo stesso Patrono, nessuno le ha mai viste insieme. La \emph{gemella} \textbf{Laydel}\index{Laydel} non tollera la sofferenza, disprezza chi causa dolore, uccide senza timore qualunque creatura abbia peccato contro un innocente, chiunque abbia causato sofferenza.

\begin{itemize}[leftmargin=*] \setlength{\itemsep}{0pt}
\item \textbf{Simbolo}: Una farfalla che gronda sangue mentre vola
\item \textbf{Caratteristica}: Saggezza (Ledyal) - Forza (Laydel)
\item \textbf{Tratti Ledyal}: Entusiasta, Compassionevole, Prudente, Gentile, Curioso, Codardo, Testardo
\item \textbf{Tratti Laydel}: Vendicativo, Paziente, Ambizioso, Intransigente, Invidioso, Cinico, Arrogante
\item \textbf{Manifestazione}: come se un mantello di farfalle avvolgesse il Devoto
\item \textbf{Somma dei Tratti in comune a 5 punti} punti: Il tuo tocco è vita/attacco. 3 volte al giorno puoi toccare una creatura vivente e curarla/causare 1d6 Punti Ferita. Costo 2 Azioni (comprende anche l'Azione di tocco)
\item \textbf{Somma dei Tratti in comune a 10 punti}: Il tuo tocco è pace. Puoi lanciare 2 volta al giorno l'incantesimo Santuario.
\item \textbf{Somma dei Tratti in comune a 15 punti}: La tua aura protegge i tuoi compagni. Entro raggio 6 metri i tuoi compagni hanno un +4 alla Difesa ed un +2 ai Tiri Salvezza. Durata 10 minuti consecutivi, una volta al giorno. Costo 2 Azioni.
\item \textbf{Somma dei Tratti in comune a 20 punti}: Irradi una sfera curativa intorno a te. Ogni creatura nel raggio di 6 metri viene curata di 60 Punti Ferita. Una volta al giorno. In caso di Laydel l'effetto è opposto. Costo 2 Azioni
\item \textbf{Energia/B}: Energia Positiva, Elettricità
\item \textbf{Vantaggio}: +1d6 alle prove di Pronto Soccorso (Ledyal) oppure sei +4 TS contro Paura (Laydel)
\item \textbf{Liste Magia Privilegiate}: Cura o Invocazione
\item \textbf{Arma Preferita}: Manganello/Catena chiodata
\item \textbf{Regola}: Non permettere violenze contro il genere di una creatura
\end{itemize}

\subsubsection{Nethergal}\index{Nethergal}\label{nethergal}\index{Sincerità}\hypertarget{nethergal}{}

\begin{enfasi}{
I sogni sono risposte a domande che non siamo ancora in grado di formulare. (X-Files)

\medskip

Un sogno non interpretato è come una lettera non letta. (Talmud)
}\end{enfasi}

\medskip

Il Patrono Messaggero. Sulla piuma di un'oca vola la lettera di Nethergal. Rapida, impetuosa, diretta, Nethergal è la messaggera, colei alla quale affidare pensieri e scritti. Sarcastica e logorroica curioserà sui tuoi scopi, ti chiederà informazioni sugli scritti affidatole con esplicita franchezza ed avrà sempre qualcosa da ridire sul messaggio da portare ma sarà anche altrettanto diretta e precisa nel consegnarlo.

Nethergal non è solo chiacchiere e pettegolezzi, qualsiasi testo venga scritto lei lo conosce, non esiste codice o segreto scritto che lei non conosca o non sappia decifrare.

Il Devoto di Nethergal è un fine linguista, un esperto di indovinelli e rebus, un Devoto che a differenza di Atmos non si limita a custodire gli scritti ma ne diffonde la conoscenza.

Un Devoto di Nethergal è un maestro, una professoressa di lingue di un Collegio, un dotto esperto di mille argomenti, probabilmente vanitoso e qualche volta arrogante se sfidato sulle sue materie.

Nethergal conosce la locazione di ogni documento ed è probabilmente la chiave per comprendere cosa successe alla Freten. Pochissimi sanno che Nethergal era uno dei Patroni della prima venuta, quella incaricata di distruggere le infrastrutture di comunicazione, colei che cifrò i contenuti degli archivi elettronici nel più grande ransoware della storia.

Nethergal ha anche un altro ruolo è il Patrono dei sogni e delle visioni, divide questo compito con Sixiser che invece domina gli incubi.

\begin{itemize}[leftmargin=*] \setlength{\itemsep}{0pt}
\item \textbf{Simbolo}: una piuma bianca cangiante
\item \textbf{Caratteristica}: Destrezza
\item \textbf{Tratti}: Estroverso, Curioseffettuareo, Testardo, Vanitoso, Vendicativo, Arrogante, Paziente
\item \textbf{Manifestazione}: cascata di piume, un oca in volo
\item \textbf{Somma dei Tratti in comune a 5 punti} punti: Puoi inviare un messaggio di massimo 144 caratteri ad un soggetto che puoi vedere entro 50 metri senza essere udito/visto. Una volta all'ora. Costo 1 Azione. Il soggetto deve comprendere la lingua usata.
\item \textbf{Somma dei Tratti in comune a 10 punti}: Mettendo la mano su un libro ne apprendi il contenuto come se lo avessi letto. Un libro a settimana. Perdi le conoscenze così acquisite dopo una settimana. Tempo 1 round. La lingua scritta del tomo deve essere nota.
\item \textbf{Somma dei Tratti in comune a 15 punti}: Puoi volare, come omonimo incantesimo, 1 ora al giorno. Costo 2 Azioni.
\item \textbf{Somma dei Tratti in comune a 20 punti}: Comprendi ogni scritto che non sia magico o codificato.
\item \textbf{Energia/N}: Elettricità, Suono
\item \textbf{Vantaggio}: sai sempre dove è il nord magnetico
\item \textbf{Liste Magia Privilegiate}: Trasmutazione, Aria
\item \textbf{Arma Preferita}: Balestra leggera
\item \textbf{Regola}: Non distruggere un libro o una lettera
\end{itemize}

\subsubsection{Nedraf}\index{Nedraf}\label{nedraf}\index{Perseveranza}\hypertarget{nedraf}{}

\begin{enfasi}{
Pensa veramente di lottare per qualcosa a parte la sua sopravvivenza? (Matrix Revolutions, Film)

\medskip

Non importa quanto stretto sia il passaggio,

Quanto piena di castighi la vita.

Io sono il padrone del mio destino:

Io sono il capitano della mia anima (Invictus, William Ernest Henley)
}\end{enfasi}

Il Patrono Sopravvissuto, il vecchio lupo mai stanco che ha attraversato e combattuto innumerevoli battaglie. La sua carne è ferita, il suo corpo ricoperto di cicatrici di guerra e lividi ma nulla lo farà crollare. Tenacia, passione, esperienza e tanta rabbia rendono Nedraf non solo un combattente eccellente in qualsiasi occasione ma un conoscitore dell'ambiente attorno a sé. Grazie al suo impeccabile allenamento sa sfruttare al meglio le risorse a disposizione. Sa spronare con passione gli uomini a suoi ordini.
Nedraf rappresenta colui che vorresti sempre accanto in ogni battaglia.

Molti capitani di ventura e ufficiali al comando sono Devoti di Nedraf. Il Devoto di Nedraf non si arrende, non rinuncia, non abbandona i compagni ma non per questo è avventato o irrazionale nelle scelte.

\begin{itemize}[leftmargin=*] \setlength{\itemsep}{0pt}
\item \textbf{Simbolo}: una mano forte, avvolta in una benda sporca di sangue che brandisce una spada
\item \textbf{Caratteristica}: Costituzione
\item \textbf{Tratti}: Paziente, Vanitoso, Coraggioso, Intransigente, Entusiasta, Arrogante, Cinico
\item \textbf{Manifestazione}: si spande nell'aria odore di sangue e metallo
\item \textbf{Somma dei Tratti in comune a 5 punti} punti: Puoi portare armature leggere senza penalità alla Prova di Magia
\item \textbf{Somma dei Tratti in comune a 10 punti}: Acquisisci un punto bonus su una Lista armi. Può essere nota o meno
\item \textbf{Somma dei Tratti in comune a 15 punti}: Puoi portare armature medie senza penalità alla Prova di Magia ed Destrezza
\item \textbf{Somma dei Tratti in comune a 20 punti}: Acquisisci un punto bonus su una Lista armi. Può essere nota o meno
\item \textbf{Energia/B}: Energia positiva, Suono
\item \textbf{Vantaggio}: recuperi il doppio di PF quando riposi
\item \textbf{Liste Magia Privilegiate}: Ammaliamento, Terra
\item \textbf{Arma Preferita}: Spadone a due mani
\item \textbf{Regola}: Non abbandonare i compagni
\end{itemize}

\subsubsection{Nihar}\index{Nihar}\label{nihar}\index{Umiltà}\hypertarget{nihar}{}

\begin{enfasi}{
Che cos'è un eroe? È un individuo dotato di un grande talento e straordinario coraggio, che sa scegliere il bene al posto del male, che sacrifica se stesso per salvare gli altri, ma soprattutto... che agisce quando ha tutto da perdere e nulla da guadagnare. (Lo chiamavano Jeeg Robot, film)
}\end{enfasi}

E' il Patrono degli eroi per caso. Ponderato e tranquillo è amante del buon vino e del gozzovigliare. E' colui che non sceglieresti mai come compagno d'armi a causa del suo aspetto \emph{comune} e del suo atteggiamento goliardico. Ma poi al momento di esserci, di combattere, di far la differenza ecco che con un colpo fortunato risolve la sfida.

Ha le sembianze di un piccolo uomo, dai vestiti sfarzosi e ricercati e dall'espressione guardinga ed allegra. Si protegge sempre e a qualunque costo, mostrando al mondo esattamente ciò che il mondo vuole vedere. Controlla attentamente la realtà attorno a sé e anche se è sempre più facile vederlo con un calice in mano, se non ci si lascia ingannare dalle apparenze si noterà come i suoi occhi non perdano mai di vista il pericolo, il problema. Sta attento, non si fida di nulla e di nessuno. Ha fatto dei suoi difetti i suoi punti di forza.

\begin{itemize}[leftmargin=*] \setlength{\itemsep}{0pt}
\item \textbf{Simbolo}: Una daga appoggiata vicino ad un calice di vino
\item \textbf{Caratteristica}: Intelligenza
\item \textbf{Tratti}: Curioso, Coraggioso, Compassionevole, Vanitoso, Invidioso, Avaro, Crudele
\item \textbf{Manifestazione}: il suono di un brindisi o lo stappare di una bottiglia
\item \textbf{Somma dei Tratti in comune a 5 punti} punti: Puoi trasformare l'acqua in vino. Un litro al giorno. Costo 2 Azioni. 2 volte al giorno.
\item \textbf{Somma dei Tratti in comune a 10 punti}: Una Azione Immediata, ottieni un bonus di +2d6 ad una prova di Competenza in quel round. 3 volta al giorno.
\item \textbf{Somma dei Tratti in comune a 15 punti}: La tua arma leggera causa sempre un danno critico quando colpisci. Il bonus è sempre attivo.
\item \textbf{Somma dei Tratti in comune a 20 punti}: I manicaretti che prepari sono buonissimi. Chiunque si sazi con una pietanza da te preparata recupera 2d6 Punti Ferita e viene curato dai veleni anche magici. Max 6 persone al giorno. 0.5 ore di preparazione per persona.
\item \textbf{Energia/B}: Energia Positiva, Fuoco
\item \textbf{Vantaggio}: Riduzione 1 al danno non magico (RD 1-magico)
\item \textbf{Liste Magia Privilegiate}: Ammaliamento, Divinazione
\item \textbf{Arma Preferita}: Spada corta
\item \textbf{Regola}: Non rifiutare un buon bicchiere di vino
\end{itemize}

\subsubsection{Orudjs}\index{Orudjs}\label{orudjs}\index{Impulsivo}\hypertarget{orudjs}{}

\begin{enfasi}{
Nulla è più facile che illudersi. Perché l'uomo crede vero ciò che desidera. (Demostene)
}\end{enfasi}

Orudjs striscia nelle profondità delle caverne, circondato da gemme, tesori, servitori zelanti.

Descritto come una melma informe da chi ha percepito una parvenza della forma, Orudjs è Patrono della illusione e della finzione.

Con il solo pensiero convince chiunque di qualsiasi cosa voglia. Adora il teatro per ciò che per lui è, la rappresentazione della falsità, l'essere tante persone ed in realtà nessuna.

Dove domina regna il caos dove ognuno è convinto di essere nel giusto e guerre tra clan nutrono la sua fame senza fine.

Finge di ascoltare chi gli sta vicino ma in realtà non è interessato alle storie altrui perché le sue sono sempre le migliori. E' un codardo senza limiti ed un bugiardo con sempre un tornaconto.

I suoi Devoti sono creature deboli, che hanno bisogno di un padrone, di una voce che gli dica costantemente di cosa hanno bisogno e cosa vogliono.

Ma anche abili attori ed intrattenitori, spie sotto copertura, diplomatici o politicanti.

\begin{itemize}[leftmargin=*] \setlength{\itemsep}{0pt}
\item \textbf{Simbolo}: Una maschera teatrale bianca con solo la bocca aperta e gli occhi chiusi
\item \textbf{Caratteristica}: Carisma
\item \textbf{Tratti}: Impulsivo, Dissoluto, Ambizioso, Indeciso, Crudele, Compassionevole, Disonesto
\item \textbf{Manifestazione}: il suono di una risata profonda e contagiosa
\item \textbf{Somma dei Tratti in comune a 5 punti} punti: Il tuo eloquio è già leggendario. +2 alle prove di Intrattenere.
\item \textbf{Somma dei Tratti in comune a 10 punti}: Puoi lanciare Immagine Silenziosa 3 volte al giorno.
\item \textbf{Somma dei Tratti in comune a 15 punti}: Il tuo eloquio è già leggendario. +1d6 aggiuntivo alle prove di Intrattenere. Puoi lanciare \hyperlink{Immagine Maggiore}{Immagine Maggiore} 1 volta al giorno.
\item \textbf{Somma dei Tratti in comune a 20 punti}: La tua voce è suadente. Una creatura, da te individuata, che ti ascolti per almeno un minuto deve fare un Tiro Salvezza Volontà DC 30 oppure essere sotto l'influenza di Dominare Persone/Mostri. Una volta al giorno
\item \textbf{Energia/N}: Elettricità, Fuoco
\item \textbf{Vantaggio}: +4 alle prove di Ingannare
\item \textbf{Liste Magia Privilegiate}: Ammaliamento, Illusione
\item \textbf{Arma Preferita}: Stocco
\item \textbf{Regola}: Devi avere sempre l'ultima parola
\end{itemize}

\subsubsection{Orlaith}\index{Orlaith}\label{orlaith}\index{Giustizia}\hypertarget{orlaith}{}

\begin{enfasi}{
Io sono fatto per combattere il crimine, non per governarlo. Non è ancora giunto il tempo in cui gli uomini onesti possono servire impunemente la patria. I difensori della libertà saranno sempre dei proscritti finché la masnada dei furfanti dominerà. (Maximilien de Robespierre)
}\end{enfasi}

Ovvero il Patrono della Giustizia e della Vendetta. Lui segue pedissequamente le leggi e pretende che i suoi sottoposti eseguano senza alcuna discussione gli ordini impartiti. E' mosso da uno spirito gentile e buono che però tiene ben nascosto dietro le sue azioni dirette ed incisive, spudorate e mortali. Orlaith è vendetta che si fa legge. Agisce per senso di giustizia con i suoi metodi. Di lui attirano il portamento e lo sguardo fiero.

I Devoti di Orlaith spesso sono giudici e giustizieri, persone che hanno deciso di portare la giustizia ovunque, perché Orlaith non può stare fermo, c'è sempre qualcuno da giudicare e punire.

Attenti ai Seguaci di Orlaith, vanità, vendetta ed intransigenza li rendono odiosi e mal disposti verso tutti.

\begin{itemize}[leftmargin=*] \setlength{\itemsep}{0pt}
\item \textbf{Simbolo}: Una mano stesa su un libro chiuso
\item \textbf{Caratteristica}: Forza
\item \textbf{Tratti}: Vanitoso, Intransigente, Coraggioso, Testardo, Dissoluto, Vendicativo, Curioso
\item \textbf{Manifestazione}: l'immagine di una stadera, sbilanciata.
\item \textbf{Somma dei Tratti in comune a 5 punti} punti: Richiami a te 1 \hyperlink{Mastino}{mastino} che obbedisce ai tuoi comandi. Durata 1 minuto. Una volta al giorno. Costo 2 Azioni.
\item \textbf{Somma dei Tratti in comune a 10 punti}: Un paio di manette si manifesta attorno ai polsi della creatura (massimo taglia grande) entro 27 metri. Tiro Salvezza Riflessi DC 25 per annullare. Costo 2 Azioni. Una volta al giorno. Forza/Artista della Fuga DC 20 per liberarsi.
\item \textbf{Somma dei Tratti in comune a 15 punti}: Crei un raggio di Luce lungo 27 metri e largo pochi centimetri. Ogni creatura attraversata subisce 8d6 di danno, DC 25 Riflessi per dimezzare. Una volta al giorno. Costo 2 Azioni.
\item \textbf{Somma dei Tratti in comune a 20 punti}: Il tuo udito è solo per la verità. Intorno a te per 3 metri, te compreso, è sempre attiva \hyperlink{Zona di Verità}{Zona di Verità}.
\item \textbf{Energia/B}: Luce, Suono
\item \textbf{Vantaggio}: Qualsiasi arma non improvvisata nelle tue mani fa almeno 1d6 di danno
\item \textbf{Liste Magia Privilegiate}: Illusione, Fuoco
\item \textbf{Arma Preferita}: Lancia da fante
\item \textbf{Regola}: Non rifiutare un ordine da una legittima autorità
\end{itemize}

\subsubsection{Rezh}\index{Rezh}\label{rezh}\index{Avarizia}\hypertarget{rezh}{}

\begin{enfasi}{
L'avidità, non trovo una parola migliore, è valida, l'avidità è giusta, l'avidità funziona, l'avidità chiarifica, penetra e cattura l'essenza dello spirito evolutivo. L'avidità in tutte le sue forme: l'avidità di vita, di amore, di sapere, di denaro, ha improntato lo slancio in avanti di tutta l'umanità. E l'avidità, ascoltatemi bene, non salverà solamente la Teldar Carta, ma anche l'altra disfunzionante società che ha nome America. (Gordon Gekko dal film Wall Street, 1987)
}\end{enfasi}

Il Patrono che disprezza tutto. Rezh ama, vuole, tocca, rimira solo le sue monete lucide e brillanti. Non sono mai abbastanza, nessuna ricchezza è mai sufficiente per lei. Rezh, l'avara tiene tutto per se, non conosce compassione, non conosce carità, non conosce condivisione. La sua fame di denaro, di ricchezze la rende prona a qualsiasi bassezza. Disprezza tutto e tutti e giudica tutto e tutti seguendo solo il suo personale metro di giudizio. In ogni moneta c'è un pò di Rezh. Nella ossidatura di ogni moneta si può vedere l'impronta di Rezh.

Nelle profondità delle caverne i devoti di Rezh scavano cercando tesori, profanano catacombe e cacciano, cacciano ed uccidono chiunque abbia con se qualcosa di prezioso, anche il migliore amico.

Rezh si è preoccupata di distruggere in verdi fiamme qualsiasi documento finanziario potesse esistere prima della venuta dei Patroni. In un gesto che può sembrare di estrema generosità ha cancellato tutti i debite delle persone.

Tra gli umani i Devoti di Rezh diventano esploratori, tombaroli, persone sempre alla ricerca di un tesoro e di una moneta in più.

\begin{itemize}[leftmargin=*] \setlength{\itemsep}{0pt}
\item \textbf{Simbolo}: una pila di monete con un ratto vicino
\item \textbf{Caratteristica}: Intelligenza
\item \textbf{Tratti}: Avaro, Indeciso, Ambizioso, Invidioso, Crudele, Cinico, Paziente
\item \textbf{Manifestazione}: un rumore di monete che cadono avvolge l'incantatore
\item \textbf{Somma dei Tratti in comune a 5 punti} punti: Sei un esperto di monete e gemme, nessun falsario può ingannarti. +1d6 alle prove di Consapevolezza e Conoscenza relative.
\item \textbf{Somma dei Tratti in comune a 10 punti}: Usi le gemme come ricettacoli. Puoi scaricare un incantesimo di 3 livello o inferiore in una gemma, che deve avere valore minimo di 10mo x livello dell'incantesimo. La gemma conserva l'incantesimo per 6 ore. Per attivare la gemma usi 2 azioni e viene eseguito l'incantesimo che contiene consumando la gemma.
\item \textbf{Somma dei Tratti in comune a 15 punti}: Puoi tirare fuori dalle tasche 1 moneta d'oro ogni volta che vuoi. Max 10 mo al giorno. Costo 1 Azione.
\item \textbf{Somma dei Tratti in comune a 20 punti}: La tua armatura viene coperta da scintillio dorato e di gemme. Guadagni +4 alla Difesa e +2d6 Tiro Salvezza Tempra per 1 ora. Costo 2 Azioni, una volta al giorno.
\item \textbf{Energia/M}: Vuoto, Elettricità
\item \textbf{Vantaggio}: quando trovi un tesoro aumentane il valore del 1\%
\item \textbf{Liste Magia Privilegiate}: Abiurazione
\item \textbf{Arma Preferita}: Falcetto
\item \textbf{Regola}: Non lasciare un tesoro incustodito
\end{itemize}

\subsubsection{Shayalia}\index{Shayalia}\label{shayalia}\index{Lussuria}\hypertarget{shayalia}{}

\begin{enfasi}{
Chi pianta un giardino semina felicità (proverbio cinese)

\medskip

Nulla di grande al mondo è stato compiuto senza passione. (Georg Wilhelm Friedrich Hegel)

}\end{enfasi}

Patrono dell'Arcano di Tenebra. Shayalia è l'anima oscura della perdizione, del tradimento, della lussuria più sordida e peccaminosa. Adora i bordelli. Le piace l'odore acre del sudore, la pelle lucida di oli e profumi. Le passioni, le vendette che li si consumano, la distruzione fisica e morale che in quei luoghi viene perpetrata è la sua vita.

Shayalia è una donna, spesso sola, che gode e vive i piaceri più fisici. Vive di vendette lungamente e ben dettagliatamente progettate. Vendicativa ed amorale, non giudica con metro di giudizio umano, il suo godere non è neppure lontanamente comprensibile. Shayalia è quanto di più vicino a Calicante sia stato creato. Sono le passioni, le pulsioni, i liquidi umorali che la fanno inebriare.

Shayalia è la concubina che ti ammalia e ti distrugge, goccia dopo goccia. I veleni sono le sue armi, le debolezze umane il suo campo.

I Devoti di Shayalia sono spie, figli bastardi, amanti di potenti signori che agiscono all'ombra.

Ljust disgustata dalla visione di un Patrono del genere instillò in Shayalia l'amore e passione per le piante ed animali. E così molti dei più famosi botanici, erboristi e zoologi sono Devoti di Shayalia, forse le uniche cose che Shayalia veramente può amare.

\medskip

\begin{narratore}[Efrem e Shayalia]
Mentre \textbf{Efrem} è il patrono della natura selvaggia, Shayalia incarna la devozione e l'amore umano per la natura. Il primo sovraintende dall'alto la natura incontaminata, la seconda si cala e diventa tutt'uno con essa costruendo magnifici giardini.

\medskip

\textbf{Sumkjr} e \textbf{Shayalia} sono complementari nel tenere in mano gli umori sfuggenti delle creature. Agiscono come espressione diretta dei Patroni della Genesi.

\end{narratore}

\medskip

\begin{itemize}[leftmargin=*] \setlength{\itemsep}{0pt}
\item \textbf{Simbolo}: un cuscino stropicciato sporco di sangue
\item \textbf{Caratteristica}: Carisma
\item \textbf{Tratti}: Dissoluto, Cinico, Crudele, Vendicativo, Paziente, Compassionevole, Vanitoso
\item \textbf{Manifestazione}: il Devoto è avvolto da un mantello di velluto nero
\item \textbf{Somma dei Tratti in comune a 5 punti} punti: I tempi per preparare una pozione sono dimezzati. Gli incantesimi di Cura hanno effetto anche su animali e piante.
\item \textbf{Somma dei Tratti in comune a 10 punti}: Il tuo tocco è vita per la natura. I tuoi incantesimi di cura agiscono su animali e piante naturali in maniera massimizzata.
\item \textbf{Somma dei Tratti in comune a 15 punti}: Dal tuo palmo secerni veleno. Il tuo tocco, o tramite arma in mischia veicola il veleno. Tiro Salvezza Tempra DC 25 o -2 a Saggezza e Destrezza per 10 minuti, un soggetto avvelenato non può esserlo nuovamente per 24 ore. Costo 1 Azione.
\item \textbf{Somma dei Tratti in comune a 20 punti}: Il tuo tocco è vita per la natura. Puoi curare animali e piante magiche. Sei immune ai veleni naturali. +1d6 Conoscenza Natura.
\item \textbf{Energia/M}: Vuoto, Elettricità
\item \textbf{Vantaggio}: +4 TS contro Veleni
\item \textbf{Liste Magia Privilegiate}: Illusione oppure Animali e Piante ed una Liste Magia Elementale
\item \textbf{Arma Preferita}: Frusta
\item \textbf{Regola}: Non rinunciare ad umiliare
\end{itemize}

\subsubsection{Sixiser}\index{Sixiser}\label{sixiser}\index{Ingordigia}\hypertarget{sixiser}{}

\begin{enfasi}{
La forza che si oppone al destino è in realtà una debolezza. (Franz Kafka)

\medskip

Ora stai zitto, bimbo, non piangere,

mamma realizzerà tutti i tuoi incubi. (Mother, 1979 The Wall, Pink Floyd)

}\end{enfasi}

Il Patrono che è indifferente al presente in quanto totalmente, compulsivamente ossessionato dal futuro e dal suo destino. Negli angoli più remoti dei mondi conosciuti si narra che Sixiser accumuli di tutto, indifferente a tutto e tutti.

Terrorizzato dal futuro che vede, da una ipotetica fine di sé e del tutto vive una vita di ritiro, spirituale e fisico. Si priva volontariamente di tutto il necessario. Ma allo stesso accumula qualunque oggetto incroci la sua strada nella speranza di un ritorno.

E' paranoico e non si fida di nessuno. Usa i suoi poteri di divinazione per conoscere e scrutare tutti.

Sixiser è padrone degli incubi, dei sogni più spaventosi delle visioni di morte. Spesso usa gli incubi quale mezzo di comunicazione con i suoi seguaci.

I Devoti di Sixiser sono spesso negromanti circondati da non morti ed altre creature silenziose ed ubbidienti. Chi si rifugia alla ricerca della solitudine e dello studio, chi mira ad espandere e governare intere città e nazioni al fine di sentirsi più sicuro, è Devoto a Sixiser.

\begin{itemize}[leftmargin=*] \setlength{\itemsep}{0pt}
\item \textbf{Simbolo}: Un forziere straripante di ogni cosa che non si può chiudere
\item \textbf{Caratteristica}: Saggezza
\item \textbf{Tratti}: Prudente, Indeciso, Intransigente, Impulsivo, Disonesto, Cinico, Sospettoso
\item \textbf{Manifestazione}: due mani che circondano, come a nascondere, la testa dell'incantatore
\item \textbf{Somma dei Tratti in comune a 5 punti} punti: acquisisci la visione crepuscolare fino 9 metri, o 18 metri se già presente.
\item \textbf{Somma dei Tratti in comune a 10 punti}: vedi nell'oscurità anche magica entro 9 metri. Individui automaticamente le trappole non magiche entro 3 metri da te.
\item \textbf{Somma dei Tratti in comune a 15 punti}: Toccando un oggetto sei in grado di capirne tutte le proprietà magiche e non, anche se è maledetto. 3 volte al giorno.
\item \textbf{Somma dei Tratti in comune a 20 punti}: Sei in grado di animare una creatura morta da non più di un giorno come non morto da 1 grado di Sfida (tipo zombi/scheletro a secondo dello stato). Una volta al giorno. Costo 2 Azioni.
\item \textbf{Energia/M}: Elettricità, Energia Negativa
\item \textbf{Vantaggio}: immune alle malattie naturali
\item \textbf{Liste Magia Privilegiate}: Necromanzia
\item \textbf{Arma Preferita}: Falcione
\item \textbf{Regola}: Non fidarti
\end{itemize}

\subsubsection{Sumkjr}\index{Sumkjr}\label{sumkjr}\index{Gentilezza}\hypertarget{sumkjr}{}

\begin{enfasi}{
Ogni uomo è colpevole di tutto il bene che non ha fatto. (Voltaire)

\medskip

Tutto ciò che non viene donato va perduto. (Dominique Lapierre)

}\end{enfasi}

Patrono dell'Arcano di Luce. Sumkjr è bontà, correttezza, lealtà, giustizia, protezione.

Sumkjr è il cavaliere che protegge gli innocenti, è la spada di Ljust nella battaglia finale. Difende i deboli e lenisce le ferite.

Sumkjr porta la Luce di Ljust ovunque, nessun pericolo potrà mai fermare Sumkjr dalla sua continua, infinita, cerca del bene.

Un Devoto di Sumkjr agisce lealmente e con onore, sempre perseguendo il bene ultimo, il suo essere non può essere piegato al male, all'ingiustizia, al disonore.

Con coraggio e determinazione il Devoto affronta ogni sfida ma non solo per senso del dovere, ma perché profondamente votato al suo destino. Sumkjr sa che poche persone reggono tale standard perché a differenza dei Devoti della Patrona delle Genesi i suoi Devoti non nascono per essere tali, ma lo diventano grazie alla loro profonda e determinata forza di volontà.

Per questo motivo Ljust interviene in loro favore con l'elaborato Rito del Rinnovo, grazie al quale ogni anno al Devoto meritevole viene fatto aggiungere un punto Tratto.

Sumkjr è un soldato valoroso, il migliore amico del giusto.

Calicante, preso dall'orrore alla vista di un Patrono così fatto, lo privò della capacità di amare e provare veri sentimenti d'affetto. Portare il bene per un Devoto di Sumkjr è un qualcosa di normale come è normale non riuscire ad essere empatico con chi soffre. Il Devoto sa cosa deve fare e perché, ma non riesce a commuoversi od amare di fronte alle sofferenze od alle carezze di una donna/uomo.

\begin{itemize}[leftmargin=*] \setlength{\itemsep}{0pt}
\item \textbf{Simbolo}: tre gocce di sangue che cadono una dietro l'altra
\item \textbf{Caratteristica}: Carisma
\item \textbf{Tratti}: Gentile, Coraggioso, Testardo, Sospettoso, Altruista, Curioso, Estroverso
\item \textbf{Manifestazione}: il Devoto è avvolto da un mantello di broccato dorato
\item \textbf{Somma dei Tratti in comune a 2 punti} punti: Il tocco della tua spada é vita. Una creatura toccata con la tua arma recupera 3d6 Punti Ferita. Una volta al giorno. Costo 2 Azioni.
\item \textbf{Somma dei Tratti in comune a 7 punti}: La tua Volontà é più forte del metallo. Guadagni un +2 ai Tiri Salvezza su Volontà
\item \textbf{Somma dei Tratti in comune a 11 punti}: Puoi lanciare l'incantesimo Cono di Freddo, 40 danni da Elettricità. DC 25 per dimezzare. Una volta al giorno. Costo 2 Azioni.
\item \textbf{Somma dei Tratti in comune a 15 punti}: Sacrifichi la tua vita per portare in vita una creatura morta da non più di 1 settimana. Una volta. Costo 3 Azioni.\index{Resuscitare}
\textbf{Nota}: Sumkjr è l'unico Patrono a concedere benefici per una somma Tratti minore del normale
\item \textbf{Energia/B}: Energia Positiva, Elettricità
\item \textbf{Vantaggio}: +1 a tutti i Tiri Salvezza
\item \textbf{Liste Magia Privilegiate}: Cura
\item \textbf{Arma Preferita}: Spada Bastarda
\item \textbf{Regola}: Non compiere atti sessuali
\end{itemize}

\textbf{Le 7 Regole Luminose}\index{7 Regole Luminose}

Le Sette regole Luminose sono un insieme di norme e comportamenti tenuti, a vario titolo, dai Devoti che vogliono seguire la strada della Luce di Ljust.

I Devoti di Sumkjr devono seguirle tutte e 7, altri Devoti di altri Patroni, sempre positivi od almeno neutrali, seguono solo alcune di questi dettami, come regola per non cadere nelle braccia di Calicante.

\begin{enumerate}[leftmargin=*] \setlength{\itemsep}{0pt}
\item Proteggi i deboli e chi non sa difendersi dai soprusi
\item Ama la vita e proteggila
\item Combatti contro le ingiustizie e chi porta sofferenze e dolore
\item Lenisci le ferite ed i dolori. Placa gli animi e favorisci la pace ed armonia
\item Onestà e Lealtà sono le tua fondamenta
\item Sei un maestro di virtù. Fa che gli altri possano prendere ispirazione dalle tue gesta
\item Non lasciare che la tua inazione generi sofferenza
\end{enumerate}

\subsubsection{Tàhil}\index{Tàhil}\label{tahil}\index{Vendicativo}\hypertarget{tahil}{}

\begin{enfasi}{
Le persone malvagie fanno cose malvagie perché possono. (Sherlock Holmes - Gioco di ombre)
}\end{enfasi}

Il Patrono che brama solo infliggere dolore. Tàhil è l'incarnazione della furia di vendetta, della volontà di fare del male per il piacere della sofferenza.

Tàhil è stato creato quando Calicante ha scoperto la morte del suo primo figlio, è stata una creazione di puro istinto e furia, di rabbia e desiderio di distruzione totale, di sofferenza ed umiliazione; nulla di meglio che creare il Drago.

Tàhil ha ben pochi Devoti o Seguaci che si dichiarino tali, ma a differenza di tutti gli altri Patroni a Tàhil la cosa non interessa, lui è la manifestazione dell'odio puro e questo gli basta.

\begin{itemize}[leftmargin=*] \setlength{\itemsep}{0pt}
\item \textbf{Simbolo}: Una svastica
\item \textbf{Caratteristica}: Forza
\item \textbf{Tratti}: Vendicativo, Disonesto, Arrogante, Cinico, Ambizioso, Testardo, Impulsivo
\item \textbf{Manifestazione}: un rumore di tuono
\item \textbf{Somma dei Tratti in comune a 5 punti}: Puoi aggiungere 1d6 ad un Tiro per Colpire. Una volta al giorno, prima di effettuare il Tiro per Colpire, come Azione Immediata.
\item \textbf{Somma dei Tratti in comune a 10 punti}: Tre volte al giorno, prima di effettuare il Tiro per Colpire, puoi dichiarare di colpire. Azione Immediata
\item \textbf{Somma dei Tratti in comune a 15 punti}: Un tuo colpo andato a segno causa almeno un Tiro Critico.
\item \textbf{Somma dei Tratti in comune a 20 punti}: Il dado del Critico dell'arma aumenta di una taglia.
\item \textbf{Energia/M}: Fuoco, Vuoto
\item \textbf{Vantaggio}: Riduzione danno da Luce 10
\item \textbf{Liste Magia Privilegiate}: Ammaliamento
\item \textbf{Arma Preferita}: Spadone
\item \textbf{Regola}: Mai essere clemente
\end{itemize}

\subsubsection{Tazher}\index{Tazher}\label{tazher}\index{Malizia}\hypertarget{tazher}{}

\begin{enfasi}{
Una persona spesso finisce con l'assomigliare alla sua ombra. (Rudyard Kipling)
}\end{enfasi}

Il Patrono delle Ombre; colui che silenzioso, ti uccide. Non saprai mai il perché. Non conoscerai mai il suo aspetto ma, se improvvisamente hai una sensazione di gelo, Tazher è dietro di te pronto a prendere la tua vita.

Doppiogiochista dall'animo cattivo, chiedi il suo aiuto solo se sei disposto a pagarne il prezzo che lui e lui solo deciderà.

Vive di oscurità e sangue. Le ombre sono le sue amiche e la tenebra il suo mantello.

Si circonda di assassini, mercenari, chiunque uccida senza provare sentimenti. Nelle profondità del sottosuolo nutre i suoi adepti di dolore, sangue e morte.

Ljust inorridita da tanto odio e nichilismo instillò nel Patrono il rispetto per i morti. Un Devono si Tazher non si accanirà contro un defunto ne violerà il suo cadavere. Molti cacciatori di non morti sono devoti di Tazher.

L'umano Devoto di Tazher è il ladro, l'assassino, il bandito, chiunque viva per l'oscurità ed il proprio tornaconto. Un Devoto di Tazher è estremamente pericoloso in combattimento.

\begin{itemize}[leftmargin=*] \setlength{\itemsep}{0pt}
\item \textbf{Simbolo}: Lo scintillio della lama nel buio
\item \textbf{Caratteristica}: Destrezza
\item \textbf{Tratti}: Disonesto, Ambizioso, Paziente, Cinico, Indeciso, Arrogante, Crudele
\item \textbf{Manifestazione}: l'ombra del Devoto prende vita muovendo l'arma
\item \textbf{Somma dei Tratti in comune a 5 punti} punti: Guadagni +2 alle prove di Furtività.
\item \textbf{Somma dei Tratti in comune a 10 punti}: La tua Scurovisione diventa di 6 metri.
\item \textbf{Somma dei Tratti in comune a 15 punti}: finché cammini sopra delle ombre o al buio (oscurità) sei invisibile. Puoi essere comunque rilevato con la luce o incantesimi di divinazione.
\item \textbf{Somma dei Tratti in comune a 20 punti}: L'oscurità non è più un problema. Vedi nell'oscurità anche magica come se fosse giorno. Quando sei in un ambiente in piena luce sei abbagliato con un -2 al Tiro per Colpire.
\item \textbf{Energia/M}: Vuoto, Ghiaccio
\item \textbf{Vantaggio}: Scurovisione 3 metri
\item \textbf{Liste Magia Privilegiate}: Trasmutazione
\item \textbf{Arma Preferita}: Falcione in asta
\item \textbf{Regola}: 5 Secondi. Il tempo per rubare ad un morto, non di più.
\end{itemize}

\subsubsection{Thaft}\index{Thaft}\label{thaft}\index{Pazienza}\hypertarget{thaft}{}

\begin{enfasi}{
La morte, il più atroce dunque di tutti i mali, non esiste per noi. Quando noi viviamo la morte non c'è, quando c'è lei non ci siamo noi. Non è nulla né per i vivi né per i morti. Per i vivi non c'è, i morti non sono più. (Epicuro)
}\end{enfasi}

Il Patrono che accompagna nella nascita e nella morte. Silenzioso, resta in disparte e osserva lo scorrere della vita degli uomini. Quasi umile nella sua semplicità, Thaft è ovunque. Testimone silenzioso della vita umana; nel momento in cui una vita scivola via, Thaft assiste, nell'attimo in cui una vita nasce, Thaft è presente.

Thaft sa anche che non si può essere sempre e solo osservatori. Attraverso il suo taccuino sacro e magico può decidere e giudicare della vita degli uomini, perché se una spada ferisce, è solo Thaft che ne decide la morte.

I Devoti di Thaft sono i sacerdoti dell'ultimo viaggio, coloro che proteggono e vegliano sulle anime e corpi dei morti. Profondamente contrari all'utilizzo dei non-morti ne perseguono la distruzione.

Un Devoto di Thaft rispetta la vita come la morte e non teme di arrecare distruzione per un equilibrio maggiore.

Thaft è stato plasmato da Atmos.

\begin{itemize}[leftmargin=*] \setlength{\itemsep}{0pt}
\item \textbf{Simbolo}: Un libro aperto con un teschio sopra
\item \textbf{Caratteristica}: Saggezza
\item \textbf{Tratti}: Codardo, Paziente, Estroverso, Leale, Gentile, Vanitoso, Vendicativo
\item \textbf{Manifestazione}: si sente il pianto di un bambino appena nato o il sospiro della morte
\item \textbf{Somma dei Tratti in comune a 5 punti} punti: Il tuo tocco è letale per i non morti. Un tuo tocco infligge 2d6 di danno ad un non morto. Costo 2 Azioni compreso il tocco. Fino a 3 volte al giorno.
\item \textbf{Somma dei Tratti in comune a 10 punti}: Il tuo tocco lenisce. Una volta al giorno puoi rimuovere Cecità o Sordità. Costo 2 Azioni.
\item \textbf{Somma dei Tratti in comune a 15 punti}: Un non morto, con GS inferiore alla somma dei tuoi Tratti in comune, deve effettuare un Tiro Salvezza Tempra DC 30 o essere distrutto se toccato dalla tua mano. Costo 2 Azioni.
\item \textbf{Somma dei Tratti in comune a 20 punti}: Uccidi la creatura toccata. Tiro Salvezza su Volontà DC 30 o morte. Una volta alla settimana. Costo 2 Azioni.
\item \textbf{Energia/N}: Suono, Elettricità
\item \textbf{Vantaggio}: Riduzione danno 5 a Vuoto e Luce
\item \textbf{Liste Magia Privilegiate}: Necromanzia, Animali e piante
\item \textbf{Arma Preferita}: Arco
\item \textbf{Regola}: Non creare un non morto
\end{itemize}

\subsubsection{Torbiorn}\index{Torbiorn}\label{torbion}\index{Crudele}\hypertarget{torbiorn}{}

\begin{enfasi}{
Riguardo all'arroganza, i violenti la soffrono, ma i saggi la deridono. (Tito Livio, attribuita ad Astimede)
}\end{enfasi}

Il Patrono che meglio incarna il concetto \emph{non è mai abbastanza}. Alto, bello come una statua classica ma, proprio come quest'ultima, senza calore e vita, Torbiorn rasenta la perfezione maniacale nel vestirsi, nell'atteggiarsi.

Nulla è mai abbastanza per lui. Nessuno è mai alla sua altezza. Ed eccolo che con arroganza e ironia va a modificare tutto il modificabile per poter placare questa profonda insoddisfazione. Qualora il risultato finale raggiunto non lo soddisfi, e accade molto spesso, ecco che prende il sopravvento il suo cinismo e distrugge tutto senza curarsi della sofferenza che sta arrecando a chi gli sta attorno.

Nelle lande abbandonate il Devoto di Torbiorn è il Tiranno dal pugno di ferro che agisce solo per proprio desiderio e piacere senza curarsi di nessun'altro.

Il Devoto di Torbiorn è il tipico aristocratico ricco e svogliato,colui che cerca sempre la strada più facile e meno rischiosa.

Incurante degli altri si diverte nello sfruttare i lavori altrui e trarne giovamento.

\begin{itemize}[leftmargin=*] \setlength{\itemsep}{0pt}
\item \textbf{Simbolo}: Uno specchio opaco
\item \textbf{Caratteristica}: Carisma
\item \textbf{Tratti}: Crudele, Impulsivo, Arrogante, Disonesto, Cinico, Indeciso, Compassionevole
\item \textbf{Manifestazione}: schegge di specchio rotto tutto intorno al Devoto come un turbine
\item \textbf{Somma dei Tratti in comune a 5 punti} punti: Con un gesto puoi rinfrescare i tuoi vestiti e te stesso rendendoli puliti e profumati. Costo 1 Azione. 3 volte al giorno.
\item \textbf{Somma dei Tratti in comune a 10 punti}: Il tuo sputo è velenoso. Se il Tiro per Colpire a tocco va a segno -2 Forza, non cumulabile. Durata 1 minuto. Tre volte al giorno. Costo 1 Azione.
\item \textbf{Somma dei Tratti in comune a 15 punti}: Fissando l'obiettivo negli occhi lo costringi a fermarsi. Il soggetto non può eseguire Azioni di Movimento. Tiro Salvezza su Volontà DC 30. Una volta al giorno. Costo 2 Azioni.
\item \textbf{Somma dei Tratti in comune a 20 punti}: Dalle tue dita partono dei viticci che pungono fino a 10 avversari. Ogni viticcio, lungo fino a 18 metri causa 2d6 di danno, Tiro Salvezza Riflessi DC 25 per dimezzare. Costo 2 Azioni.
\item \textbf{Energia/N}: Fuoco, Suono
\item \textbf{Vantaggio}: +3 contro gli incantesimi della Lista Divinazione.
\item \textbf{Liste Magia Privilegiate}: Trasmutazione
\item \textbf{Arma Preferita}: Ascia ad una mano
\item \textbf{Regola}: Non essere sciatto, mal vestito o disordinato.
\end{itemize}

\medskip

\begin{narratore}[Adeguare ed adeguarsi]

In accordo con il Narratore, ed adeguatamente motivato, è possibile cambiare Vantaggio e Liste di Magia Privilegiate.

\end{narratore}

\subsubsection{Elenco Patrono - Tratto}\index{Elenco Patrono - Tratto}\hypertarget{tabellacollegamentopatronotratto}{}\label{tabellacollegamentopatronotratto}

\smallskip

I Patroni sono indicati in ordine alfabetico per Tratto maggiormente caratterizzante.

\medskip

\setlength{\parindent}{0cm}{

\textbf{\hyperlink{gaya}{Gaya}}: Altruista, Gentile, Sospettoso, Cinico, Invidioso, Disonesto, Arrogante

\smallskip

\textbf{\hyperlink{calicante}{Calicante}}: Ambizioso, Disonesto, Vendicativo, Cinico, Dissoluto, Arrogante, Avaro

\smallskip

\textbf{\hyperlink{erondil}{Erondil}}: Arrogante, Vendicativo, Ambizioso, Compassionevole, Entusiasta, Leale, Avaro

\smallskip

\textbf{\hyperlink{rezh}{Rezh}}: Avaro, Indeciso, Ambizioso, Invidioso, Crudele, Cinico, Paziente

\smallskip

\textbf{\hyperlink{sixiser}{Sixiser}}: Prudente, Indeciso, Intransigente, Impulsivo, Disonesto, Cinico, Sospettoso

\smallskip

\textbf{\hyperlink{cattalm}{Cattalm}}: Cinico, Arrogante, Ambizioso, Intransigente, Dissoluto, Sospettoso, Paziente

\smallskip

\textbf{\hyperlink{ljust}{Ljust}}: Compassionevole, Testardo, Coraggioso, Estroverso, Altruista, Leale, Paziente

\smallskip

\textbf{\hyperlink{gradh}{Gradh}}: Coraggioso, Vanitoso, Arrogante, Gentile, Invidioso, Leale, Sospettoso

\smallskip

\textbf{\hyperlink{nihar}{Nihar}}: Curioso, Coraggioso, Compassionevole, Vanitoso, Invidioso, Avaro, Crudele

\smallskip

\textbf{\hyperlink{tazher}{Tazher}}: Disonesto, Ambizioso, Paziente, Cinico, Indeciso, Arrogante, Crudele

\smallskip

\textbf{\hyperlink{shayalia}{Shayalia}}: Dissoluto, Cinico, Crudele, Vendicativo, Paziente, Compassionevole, Vanitoso

\smallskip

\textbf{\hyperlink{ledyal}{Ledyal}}: Entusiasta, Compassionevole, Prudente, Gentile, Curioso, Codardo, Testardo

\smallskip

\textbf{\hyperlink{nethergal}{Nethergal}}: Estroverso, Curioso, Testardo, Vanitoso, Vendicativo, Arrogante, Paziente

\smallskip

\textbf{\hyperlink{sumkjr}{Sumkjr}}: Gentile, Coraggioso, Testardo, Sospettoso, Altruista, Curioso, Estroverso

\smallskip

\textbf{\hyperlink{atmos}{Atmos}}: Indeciso, Prudente, Intransigente, Paziente, Vendicativo, Curioso, Avaro

\smallskip

\textbf{\hyperlink{krondal}{Krondal}}: Intransigente, Vanitoso, Arrogante, Sospettoso, Paziente, Ambizioso, Testardo

\smallskip

\textbf{\hyperlink{belevon}{Belevon}}: Invidioso, Ambizioso, Dissoluto, Disonesto, Compassionevole, Paziente, Altruista

\smallskip

\textbf{\hyperlink{orudjs}{Orudjs}}: Impulsivo, Dissoluto, Ambizioso, Indeciso, Crudele, Compassionevole, Disonesto

\smallskip

\textbf{\hyperlink{efrem}{Efrem}}: Leale, Indeciso, Prudente, Impulsivo, Testardo, Paziente, Ambizioso

\smallskip

\textbf{\hyperlink{torbiorn}{Torbiorn}}: Crudele, Impulsivo, Arrogante, Disonesto, Cinico, Indeciso, Compassionevole

\smallskip

\textbf{\hyperlink{nedraf}{Nedraf}}: Paziente, Vanitoso, Coraggioso, Intransigente, Entusiasta, Arrogante, Cinico

\smallskip

\textbf{\hyperlink{atherim}{Atherim}}: Sospettoso, Compassionevole, Altruista, Intransigente, Coraggioso, Entusiasta, Vanitoso\

\smallskip

\textbf{\hyperlink{thaft}{Thaft}}: Codardo, Paziente, Estroverso, Leale, Gentile, Vanitoso, Vendicativo

\smallskip

\textbf{\hyperlink{lynx}{Lynx}}: Testardo, Coraggioso, Cinico, Intransigente, Vendicativo, Estroverso, Vanitoso

\smallskip

\textbf{\hyperlink{orlaith}{Orlaith}}: Vanitoso, Intransigente, Coraggioso, Testardo, Dissoluto, Vendicativo, Curioso

\smallskip

\textbf{\hyperlink{laydel}{Laydel}}: Vendicativo, Paziente, Ambizioso, Intransigente, Invidioso, Cinico, Arrogante

\smallskip

\textbf{\hyperlink{tahil}{Tàhil}}: Vendicativo, Disonesto, Arrogante, Cinico, Ambizioso, Testardo, Impulsivo

}

\end{multicols}

\vfill

\begin{enfasi}
	Gli dei tessono sventure per gli uomini, perché le generazioni future abbiano qualcosa da cantare. Iliade, Omero.
\end{enfasi}

%\vfill

%\begin{center}
%\includegraphics[keepaspectratio,width=0.6\textwidth]{immagini/archetipijung.png}

%\emph{I 12 Archetipi di Jung}
%\end{center}

\pagebreak

\section{Equipaggiamento}\hypertarget{equipaggiamento}{}\label{equipaggiamento}

\subsection{Ricchezza e Denaro}\index{Ricchezza e Denaro}
\begin{enfasi}{\begin{center}Sono pronto ad andare. Ho uno zaino! (Morgan Grimes, Chuck, Serie TV)\end{center}
}\end{enfasi}

\begin{multicols}{2}

\label{ricchezza-e-denaro}

\textit{Euro ? Dollaro ? Yen ? No.. non sono accettati.}

In questo nuova \emph{economia} il valore delle vecchie monete è nullo se non per uno scarso valore storico numismatico.

Potremmo dire che dopo la prima venuta dei Patroni, lo smantellamento e distruzione dell'infrastruttura statale ed economica, nei due secoli a venire non sono stati riproposti gli \href{https://it.wikipedia.org/wiki/Accordi_di_Bretton_Woods}{accordi di Bretton Woods} ed anzi a fatica arriviamo nelle città più importanti ad avere un \href{https://it.wikipedia.org/wiki/Sistema_aureo}{sistema aureo}.

L'estrema difficoltà dei trasporti internazionali, limitati tramite pericolosi Portali oppure tornati ad usare vecchie nave a vela e rarissime navi e treni a vapore, residuati storici di pezzi di alta ingegneria \emph{moderna}, hanno portato il commercio a lavorare in maniera più semplice, più basilare.

La finanza come la conoscevamo non esiste più se non nelle piccole comunità dove sono gli stessi cittadini ad investire nelle attività locali. Scordatevi la Borsa, non c'è internet, telefono, le poche radio sono di appassionati marconisti che come antichi alchimisti riescono a produrre l'elettricità ed utilizzare attrezzature dalla minima portata tenute insieme con del prezioso fili di rame.

Si è quindi tornati alla moneta come espressione del valore intrinseco della stessa: una moneta vale non perché c'è una Società, Stato, Nazione che garantisce, bensì vale per il valore intrinseco che possiede.

Bentornate quindi monete d'oro, d'argento, di electrum (lega argento/oro), o di semplice rame.

Quando i mercanti discutono accordi che riguardano merci o servizi del valore di centinaia o migliaia di monete d'oro lo scambio viene effettuato in lingotti d'oro, lettere di credito o merci di valore.

Una moneta d'oro vale dieci monete d'argento, il tipo di moneta più usato tra la popolazione. Una moneta d'argento può coprire il salario giornaliero di un manovale e comperare un'ampolla d'olio per lanterna o un frugale pasto e giaciglio per una notte in una locanda scadente.

Una moneta d'argento vale dieci monete di rame, che vengono normalmente usate dagli operai e i mendicanti. Una singola moneta di rame può comperare una candela oppure un pezzo di gesso.

A volte, però, in mezzo ai tesori compaiono monete insolite appartenute agli antichi stati. Queste monete spesso hanno un valore solo storico se non sono di materiali preziosi, il valore quindi dipende totalmente da chi la deve incassare.

Una moneta comune pesa circa dieci grammi cosicché cinquanta monete pesano mezzo chilo.

Un personaggio che inizia a giocare generalmente ha monete d'oro sufficienti per acquistare gli elementi di base: qualche arma, un'armatura di seconda mano (quella meno costosa) ed un pò di attrezzatura varia. \textbf{Al primo livello} i personaggi hanno monete ed equipaggiamento per un totale di \textbf{circa 100 mo}.\index{Ricchezza al primo livello}

\subsubsection{Monete}\index{Monete}

\textbf{Tabella: Equivalenza delle Monete}\index[Tabelle]{Tabella Equivalenza delle Monete}

\medskip

\noindent\begin{tabularx}{\linewidth}{lXXXXX}
\toprule
\rowcolor{gray!20}\textbf{Moneta} & \textbf{MR}&\textbf{MA}&\textbf{ME}&\textbf{MO}&\textbf{MP}\\
\toprule
Rame& 1& 1/10& 1/50& 1/100& 1/1000\\
\rowcolor{gray!20}Argento & 10 & 1 & 1/5& 1/10 & 1/100\\
Electrum & 50 & 5 & 1 & 1/2& 1/10\\
\rowcolor{gray!20}Oro & 100 & 10 & 2& 1 & 1/10\\
Platino & 1000 & 100 & 20& 10 & 1
\end{tabularx}

\medskip

Di solito pagamenti oltre le 100 monete d'oro avvengono in lingotti da 1, 2, 5 kilogrammi d'oro, equivalenti a 100, 200 e 500 monete d'oro o meglio ancora in gemme. In caso di somme ancora più cospicue è possibile che sia rilasciata una lettera di credito di qualche istituto bancario (ma valido in pochissime ed importanti città).

\subsubsection{Altre Ricchezze - Merci di scambio}\index{Altre Ricchezze}

I mercanti di solito scambiano merci anche senza l'uso di monete.
Per farsi un'idea delle transazioni commerciali, alcune merci di scambio sono descritte nella tabella.

\medskip

\textbf{Tabella: Esempi altre ricchezze}\index[Tabelle]{Tabella Esempi altre ricchezze}

\medskip

\noindent\begin{tabular}{ll}
	\toprule
\rowcolor{gray!20}\textbf{Costo} & \textbf{Oggetto}\\
\toprule
1 mr & Frumento (0.5 kg)\\
\rowcolor{gray!20}2 mr & Farina (0.5 kg) o pollo (1)\\
1 ma & Ferro (0.5 kg)\\
\rowcolor{gray!20}1 mo & Cannella (0.5 kg) o capra \\
2 mo & Zenzero o pepe (0.5 kg) o pecora (1)\\
\rowcolor{gray!20}3 mo & Maiale (1) \\
4 mo & Lino (1 m\textsuperscript{2}\\
\rowcolor{gray!20}5 mo & Sale o argento (0.5 kg) \\
10 mo& Seta (1 m) o mucca (1)\\
\rowcolor{gray!20}15 mo& Zafferano(0.5 kg)/bue (1)
\end{tabular}

\medskip

Consultate anche il capitolo sull'\hyperlink{ingombro}{Ingombro} (pag. \pageref{ingombro} in Movimento e Trasporto.

\end{multicols}

\pagebreak

\section{Equipaggiamento - Armi}\index{Equipaggiamento}\index{Armi}\label{equipaggiamentoarmi}
\hypertarget{equipaggiamento.armi}{}

\label{equipaggiamento---armi}
\begin{enfasi}{
Questo è il mio fucile. Ce ne sono tanti come lui, ma questo è il mio. Il mio fucile è il mio migliore amico, è la mia vita. Io debbo dominarlo come domino la mia vita. Senza di me il mio fucile non è niente; senza il mio fucile io sono niente. Debbo saper colpire il bersaglio, debbo sparare meglio del mio nemico che cerca di ammazzare me, debbo sparare io prima che lui spari a me e lo farò. Al cospetto di Dio giuro su questo credo: il mio fucile e me stesso siamo i difensori della patria, siamo i dominatori dei nostri nemici, siamo i salvatori della nostra vita e così sia, finché non ci sarà più nemico ma solo pace, amen. (Full Metal Jacket, Film, 1987)

\medskip

La spada davvero buona è quella che rimane nel suo fodero. (Sanjuro)}\end{enfasi}

\medskip

Usare un'Arma senza l'adeguata competenza impone un -1d6 al colpire

La tabella presenta il nome dell'arma, il suo costo in monete d'oro, il danno ed il tipo di danno (se da Taglio, Contundente o Perforante), la gittata, la Lista d'Arma appartenente e le caratteristiche speciali che può avere. Vedi anche \hyperref[sec:capacita-di-carico-e-trasporto-ingombro]{Capacità di Carico e Trasporto.}

\medskip

\textbf{Tabella: Lista della Armi}\index[Tabelle]{Tabella Lista della Armi}

\noindent\begin{xltabular}{\linewidth}{lllX}
\rowcolor{gray!20}\textbf{Arma}&\textbf{Costo}&\textbf{Dim./Danno} & \textbf{Gittata, Lista, Speciale}\\
\hline
Alabarda& 10 & G/1d10 P/T& \textbf{Lance}, \textbf{Aste}, Controcarica, Arma lunga, ED9 \\
\rowcolor{gray!20}Arco corto composito& note*& M/Frecce& 20 metri, \textbf{Archi}\\
Arco corto& 30 & M/1d6 P& 15 metri, \textbf{Archi}\\
\rowcolor{gray!20}Arco lungo composito& note*& G/Frecce& 36 metri, \textbf{Archi}\\
Arco lungo& 75 & G/Frecce& 20 metri, \textbf{Archi}\\
\rowcolor{gray!20}Ascia martello& 16 & M/1d6 T/C& \textbf{Scuri e Accette}\\
Ascia ad una mano& 6 & M/1d6 T& 6 metri, \textbf{Scuri e Accette}, \textbf{Armi da Lancio}, Versatile\\
\rowcolor{gray!20}Ascia da battaglia& 10 & G/1d10 T&\textbf{Scuri e Accette}\\
Balestra ad una mano& 100& M/Dardi& 6 metri, \textbf{Balestre}\\
\rowcolor{gray!20}Balestra leggera& 35 & P/Dardi& 15 metri, \textbf{Armi Semplici}, \textbf{Balestre}\\
Balestra pesante& 50 & G/Dardi& 30 metri, \textbf{Balestre}\\
\rowcolor{gray!20}Bastone& 3& M/1d6 C& \textbf{Armi Semplici}, Arma lunga, Versatile, Parata\\
%Brandistocco& 10 & M/2d4 P/T& \textbf{Lance}, Controcarica, Arma lunga\\
Catena chiodata& 25 & G/2d4 P& 3 metri, \textbf{Palle rotanti}, Arma lunga\\
\rowcolor{gray!20}Estoc& 25& G/1d8 P& \textbf{Spade}, Arma lunga, Parata\\
Falce& 18 & G/2d4 P/T& \textbf{Armi della Morte}, Arma lunga\\
\rowcolor{gray!20}Falcetto& 6& P/1d6 T& \textbf{Armi della Morte}\\
Falcione in asta& 12 & G/1d10 P/T& \textbf{Lance}, Controcarica, Arma lunga, ED9\\
\rowcolor{gray!20}Falcione& 75 & M/2d4 T& \textbf{Armi Aggraziate}, ED7\\
Fionda& -& P/1d4 B& 10 metri, \textbf{Armi da lancio}\\
\rowcolor{gray!20}Flagello doppio& 90 & G/1d10 C& \textbf{Palle Rotanti}, \textbf{Armi doppie}\\
Flagello pesante& 15 & G/1d10 C& \textbf{Palle Rotanti}\\
\rowcolor{gray!20}Flagello& 8& M/1d8 C& \textbf{Palle Rotanti}, \textbf{Rompi Cranio}\\
Frusta& 1& M/1d3 T& \textbf{Palle Rotanti}, Arma lunga\\
\rowcolor{gray!20}Giavellotto& 1& P/1d6 P& 12 metri, \textbf{Armi Semplici}, \textbf{Aste}, \textbf{Armi da Lancio}\\
Grande ascia doppia& 25 & G/1d10 T& \textbf{Scuri e Accette}, \textbf{Armi doppie}\\
%Grosso randello& 2& M/1d8 C&\textbf{Rompi Cranio}\\
\rowcolor{gray!20}Guanto chiodato& 5& P/1d4 P&\textbf{Armi da Stordimento}, non letale\\
Katana& 300& M/1d10 T& \textbf{Armi letali}, ED9\\
\rowcolor{gray!20}Lancia da fante& 2& M/1d8 P&3 metri, \textbf{Lance}, Arma lunga, Controcarica\\
Lancia& 10 & G/1d10 P&\textbf{Lance}, Arma lunga, Controcarica\\
\rowcolor{gray!20}Machete& 10 & M/1d6 T&\textbf{Armi letali}\\
Maglio da guerra& 7& G/1d10 C& \textbf{Rompi Cranio}\\
\rowcolor{gray!20}Manganello& 1& P/1d6 C& \textbf{Armi da stordimento}, non letale\\
Martello da guerra& 5& M/1d8 C/P& 6 metri, \textbf{Rompi Cranio}\\
\rowcolor{gray!20}Mazza leggera& 3& P/1d6 C/T& \textbf{Armi Semplici}, \textbf{Armi Leggere}, \textbf{Rompi Cranio} \\
Mazza flangiata& 5& M/1d8 C/T& \textbf{Rompi Cranio}\\
\rowcolor{gray!20}Mazza chiodata& 6& M 1d8 C/P& \textbf{Armi Semplici}, \textbf{Rompi Cranio}\\
%Naginata& 8& G/1d10 T&\textbf{Lance}, Arma lunga, ED9\\
Picca leggera& 4& M/1d4 P&\textbf{Armi della morte}\\
\rowcolor{gray!20}Picca pesante& 8& G/1d6 P&\textbf{Armi della morte}, Arma lunga\\
Pugnale& 2& P/1d4 P& 6 metri, \textbf{Armi Semplici}, \textbf{Armi leggere}, \textbf{Armi da Lancio}\\
\rowcolor{gray!20}Pugno/Calcio & note*& P/1d4 C&Versatile\\
%Randello& 1& P/1d6 C& \textbf{Armi Semplici}, \textbf{Rompi Cranio}\\
Scimitarra& 15 & M/1d6 T&\textbf{Armi Leggere}, \textbf{Armi Aggraziate}, Versatile\\
\rowcolor{gray!20}Spada corta& 10 & P/1d6 P&\textbf{Armi Leggere}, \textbf{Spade}, Versatile, Parata\\
Spada lunga& 15 & M/1d8 T&\textbf{Spade}, Parata\\
\rowcolor{gray!20}Spada a due lame& 100& G/1d8 T& \textbf{Armi doppie}, \textbf{Spade}, Parata\\
Spada bastarda& 35 & M/1d8 T&\textbf{Spade}, Parata, 1d8 ad una mano, 2d6 a 2 mani\\
\rowcolor{gray!20}Spada larga& 12 & M/2d4 T&\textbf{Spade}, Parata, 2d4 ad una mano, 1d10 a 2 mani\\
Spadone a due mani& 50 & G/2d6 T&\textbf{Spade}, Parata\\
\rowcolor{gray!20}Stocco& 20 & P/1d6 P& \textbf{Armi Leggere}, \textbf{Armi Aggraziate}, Versatile\\
Tridente& 15 & M/1d6 P/T& 3 metri, \textbf{Aste}, \textbf{Armi da Lancio}, Arma Lunga, Controcarica\\
\rowcolor{gray!20}Urgrosh& 18 & M/1d6 T/P& \textbf{Lance}, \textbf{Armi doppie}\\
\end{xltabular}

\medskip

Un \textbf{Arma} Piccola ha \textbf{Ingombro} 1, una Arma Media ha Ingombro 2, un Arma Grande ha Ingombro 4, un Arma Enorme ha Ingombro 8.\index{Ingombro Armi}\index{Ingombro Armi}

\medskip

\textbf{Tabella: Lista dei proiettili - Archi - Balestre - Fionde}\index[Tabelle]{Tabella Lista dei proiettili - Archi - Balestre - Fionde}\label{proiettili}

\noindent\begin{tabular}{lcc}
	\toprule
\rowcolor{gray!20}\textbf{Nome Proiettile}& \textbf{Numero di colpi/Costo (mo)} & \textbf{Danno / Tipo}\\
\toprule
Dardi da balestra (una mano, leggera) & 10/1 mo & 1d6 P\\
\rowcolor{gray!20}Dardi per balestra (pesante) & 3/1 mo & 1d10 P\\
Frecce da caccia (Arco Corto, Arco Lungo)& 20/1 mo & 1d6 P\\
\rowcolor{gray!20}Frecce da guerra (Arco Lungo)& 10/1 mo & 1d8 P\\
Biglie di Marmo (fionde)& 15/1 mo & 1d4 B\\
\rowcolor{gray!20}Sasso (fionde)& -& 1d2 B
\end{tabular}

\medskip

Una \textbf{Faretra} (piena o vuota) di Proiettili (Frecce o Dardi) ha \textbf{Ingombro} 2.\index{Ingombro Proiettili}

Un \textbf{dardo pesante} per Balestra penetra più facilmente le armature di metallo causando +2 danni aggiuntivi.\index{Quadrello da Balestra pesante}\index{Balestra pesante}

\begin{multicols}{2}

Un Arma +1 costa 1500 mo, +2 5000 mo. Non è possibile acquistare armi con incantamenti superiore a +2, devono essere trovate.

Una Freccia/Dardo/Sasso magico con un bonus +1 costa 25 mo, se +2 costa 100 mo. Proiettili con bonus magico superiori a +2 sono quasi impossibili da trovare.

\textbf{Un proiettile non acquisisce proprietà magiche perché il suo lanciatore è magico.}

\medskip

\textbf{Pugno Vuoto}: \hyperlink{pugnovuoto}{vedi Lista d'Armi}

\medskip

\textbf{Arco Composito}\index{Arco Composito}
Un arco composito è un arco particolarmente robusto e rigido che richiede un certo minimo di Forza per essere usato efficacemente.
Un \textbf{arco composito} lungo ha un modificatore fisso, da +1 a +5, il bonus si applica solo al danno e non al Tiro per Colpire. Un arco composito applica al danno un bonus pari al \textbf{minimo valore tra Forza ed il suo bonus}.

Un arco composito +3 usato da un personaggio con Forza 2 non può essere tirato completamente e quindi la freccia che parte avrà un modificatore al danno di +2.
Un arco composito +1 usato da un personaggio con Forza 4 può essere tirato completamente e quindi la freccia che parte avrà un modificatore al danno di +1.

Il costo di un arco composito dipende dal sul modificatore.
Un arco composito con modificatore di +1 costa 75 mo, +2 150 mo, +3 300 mo, +4 600 mo, +5 1500 mo. Non è possibile acquistare archi compositi con bonus superiori a +3, devono essere \emph{trovati}.

Un arco corto composito ha come massimo modificatore di Forza +3.

\textbf{Balestra}\index{Balestre}\index{Ricarica Balestra}
Una balestra pesante richiede due Azioni per essere ricaricata. Una balestra leggera od a una mano richiede 1 Azione per essere ricaricata.

\textbf{Gittata}\index{Gittata}\index{Tirare lontano}
La distanza indicata è quello a pieno Tiro per Colpire. Ogni arma a distanza può colpire entro tre volte la distanza indicata.

Se il target è entro la distanza indicata non si hanno penalità al colpire, se il target è tra il primo e secondo incremento la penalità al colpire è -1d6. Se il target è tra il secondo è terzo incremento la penalità al colpire è di -2d6.

Un giavellotto tirato entro 12 metri non ha penalità, ma tirato entro 24 metri ha un -6 al colpire, a distanza tra 24 e 36 metri un -12 al colpire, oltre non può essere tirato.

Un \textbf{Proiettile che colpisce si considera distrutto}, se manca ha un 50\% (4-5-6 su un d6) di probabilità che sia ancora integra.

Un Proiettile magico somma i suoi bonus a quelli del lanciatore per determinare il Tiro per Colpire ed il Danno.

La \textbf{Dimensione dell'Arma}\label{dimensionediunarma}\hypertarget{dimensionediunarma}{} è indicata come P (piccola), M (media), G (grande) ed è riferito ad una creatura media. \hyperref[armatroppogrande]{Vedi sezione Arma troppo grande}

Una \textbf{arma di dimensione superiore} \index{Arma di dimensione superiore} come ad esempio una Spada Lunga forgiata per un Ogre aumenta di una categoria il suo dado di danno.

Le Armi hanno indicato una \textbf{Tipologia di danno}\index{Tipologia di danno}, ovvero T/C/P.

Queste lettere stanno ad indicare se il danno è di tipo Taglio, Contundente o da Perforazione. Questa caratteristica può essere importante perché determinate creature possono essere immuni o subire meno danno da un particolare tipo di ferita (es uno scheletro contro un'arma da perforazione o un cubo gelatinoso contro un arma da taglio..).

Un arma può essere usata per causare un tipo di danno diverso (da taglio a perforazione o contundente) riducendo di una categoria il dado di danno (es. Spada Lunga per fare danno contundente causa 1d6).

\medskip

\begin{center}
	\includegraphics[width=0.7\linewidth]{immagini/bow2.png}
\end{center}

\medskip

\textbf{Armi Perfette}\index{Armi Perfette}

Un arma perfetta è un arma creata da un abilissimo armaiolo che pur non essendo magica, grazie al suo perfetto bilanciamento ed affilatura, ha un +1 al Tiro per Colpire.

Un armaiolo per creare un arma perfetta deve superare con un Successo Critico la DC impostata per la creazione dell'arma (DC = 12 + Ingombro dell'arma).\index{Armi Perfette}

Un arma perfetta costa il doppio di un arma normale.

\textbf{Armi Improvvisate}\index{Armi Improvvisate}\label{armaimprovvisata}\hypertarget{armaimprovvisata}{}

Talvolta oggetti che non sono stati creati per essere armi possono avere una certa efficacia in combattimento. Dal momento che non si tratta di oggetti pensati per questo utilizzo, la creatura che attacca con uno di essi subisce una penalità -1d6 al Tiro per Colpire. Un'arma improvvisata di piccole dimensioni (bottiglia) fa 1d3 di danno, di medie dimensioni (la gamba di una sedia) da 1d6, di grandi dimensioni (la gamba di un tavolo) fa 1d8 di danno.

Un'arma da lancio improvvisata ha una gittata 3 metri.

\medskip

\textbf{Lanciare armi}\index{Lanciare armi}

Una spada o comunque un arma non fatta per essere lanciata può comunque essere scagliata contro l'avversario. Il Tiro per Colpire prende un -1d6 e l'arma fa una categoria di danno inferiore (la spada lunga fa 1d6, una spada corta 1d4..). La gittata è 3 metri.

\medskip

\textbf{Usare un'Arma senza l'adeguata competenza se non è un Arma Semplice} Impone un -1d6 al Tiro per Colpire.

\textbf{Esempio}: Una creatura piccola che usa un alabarda in combattimento ravvicinato ha -1d6 perché l'arma è grande, -1d6 perché non è competente, -1d6 perché usa l'arma in mischia.

In questo caso essendo le penalità superiori ai 3d6 il personaggio non tira dadi ma usa solo la sua Competenza Armi e Forza come valore per colpire.

\subsubsection{Le armi antiche}\index{Armi Antiche}\index{Revolver}\index{Shotgun}\index{Fucile}\index{Armi da fuoco}

E' possibile trovare ancora delle armi antiche funzionanti, armi che dopo 100 anni ancora possono essere usate.

La maggior parte delle armi da fuoco dopo un lasso di tempo così lungo richiedono pezzi di ricambio ed una continua manutenzione. Questi pezzi di ricambio sono molto rari da trovare integri ed ancora più difficile è trovare un artigiano che sappia farli.

Le armi che potrete trovare funzionanti sono i revolver, gli shotgun, i fucili semi automatici ed i fucili automatici.

\medskip

\begin{description}[noitemsep, topsep=0pt, parsep=0pt, partopsep=0pt, leftmargin=0cm, labelwidth=2cm]
\item[\textbf{Revolver}]
\item[\textbf{Azioni:}] 1 Azione per un singolo colpo sparato
\item[\textbf{Caricatore:}] 6 proiettili
\item[\textbf{Gittata:}] 12 metri
\item[\textbf{Danno:}] 1d10 (P) danni a proiettile
\item[\textbf{Regole:}] è necessario un Tiro per Colpire con armi a distanza.
\end{description}

\medskip

\begin{description}[noitemsep, topsep=0pt, parsep=0pt, partopsep=0pt, leftmargin=0cm, labelwidth=2cm]
	\item[\textbf{Shotgun}]
	\item[\textbf{Azioni:}] 1 Azione per un singolo colpo sparato
	\item[\textbf{Caricatore:}] 4 proiettili
	\item[\textbf{Gittata:}] cono 6 metri
	\item[\textbf{Danno:}] 2d8 (P) danni
	\item[\textbf{Regole:}] le creature nel cono possono effettuare un Tiro Salvezza Riflessi contro il tuo Tiro per Colpire. In caso di Successo o Successo Critico il danno è dimezzato, in caso di Fallimento Critico il danno è raddoppiato. In caso di fallimento il danno è normale.
\end{description}

\medskip

\begin{description}[noitemsep, topsep=0pt, parsep=0pt, partopsep=0pt, leftmargin=0cm, labelwidth=2cm]
	\item[\textbf{Fucile Semi-automatico}]
	\item[\textbf{Azioni:}] 1 Azione per 3 colpi sparati
	\item[\textbf{Caricatore:}] 21 proiettili
	\item[\textbf{Gittata:}] 18 metri
	\item[\textbf{Danno:}] 1d8 (P) danni a proiettile
	\item[\textbf{Regole:}] è necessario un Tiro per Colpire con armi a distanza ogni 3 colpi sparati. +1 al Tiro per Colpire per Azione in cui si spara sempre al medesimo obiettivo.
\end{description}

\medskip

\begin{description}[noitemsep, topsep=0pt, parsep=0pt, partopsep=0pt, leftmargin=0cm, labelwidth=2cm]
	\item[\textbf{Fucile Automatico}]
	\item[\textbf{Azioni:}] 1 Azione per 6 colpi sparati
	\item[\textbf{Caricatore:}] 30 proiettili
	\item[\textbf{Gittata:}] 12 metri
	\item[\textbf{Danno:}] 1d6 (P) danni a proiettile
	\item[\textbf{Regole:}] è necessario un Tiro per Colpire con armi a distanza ogni 6 colpi sparati. -1 al Tiro per Colpire per Azione in cui si spara sempre al medesimo obiettivo.
\end{description}

\medskip

\textbf{Regola generale per i fucili} semi-automatici ed automatici: va a segno 1 proiettile per per differenza tra Tiro per Colpire e Difesa avversario. Es. Tiro per Colpire 16 e Difesa 14, la differenza è 2; in quella scarica di proiettili sono andati a segno 2 colpi. Non possono andare a segno più proiettili dei colpi sparati per Azione.

\textbf{Regola sui proiettili}: ogni arma usa dei proiettili diversi. Non puoi utilizzare i proiettili del revolver su un fucile semi automatico o quelli del fucile automatico su uno shotgun o fucile semi automatico.

\index{Proiettili} Inserire un nuovo caricatore o caricare un arma usa 2 Azioni.\index{Ricaricare armi da fuoco}

\textbf{Tiro Critico}: per ogni Tiro Critico ottenuto si considera un danno da singolo proiettile in più.

\textbf{Lista d'Armi}: le armi antiche sono armi improvvisate a meno di creare una Lista d'Armi da Fuoco ed assegnargli almeno 1 punto.

\subsubsection*{Proiettili}\index{Proiettili}

I proiettili sono la cosa in assoluto più difficile da trovarsi. Nessun proiettile veniva costruito con l'idea di essere sparato 100 anni dopo la sua creazione.
La polvere da sparo si è inumidita, ha perso la carica, la camicia di metallo si è corrosa con il tempo.. ci sono tantissimi fattori che rendono i proiettili estremamente rari, quasi e più delle armi magiche.

Un eventuale costo non sarebbe inferiore alle 30 mo a proiettile.

\subsubsection*{Problemi di fuoco}\index{Problemi di fuoco}\index{Inceppamento armi da fuoco}

Ogni qual volta il Tiro per Colpire sia un Fallimento Critico c'è stato un problema con l'arma e non ha sparato con successo.

\medskip

\textbf{Tira e somma 2d10, consulta la tabella}

\medskip

\noindent\begin{tabularx}{\linewidth}{lX}
	\toprule
\rowcolor{gray!20}\textbf{\#}& \textbf{Effetto}\\
\toprule
2 & Il proiettile è difettoso, per fortuna non ci sono altri problemi. Costa una Azione togliere il proiettile inceppato.\\
\rowcolor{gray!20}3 & Il proiettile si è incastrato. Costa due Azioni togliere il proiettile inceppato.\\
4 & L'aver mancato ti lascia esitante, perdi 1 Azione.\\
\end{tabularx}
\noindent\begin{tabularx}{\linewidth}{lX}
\rowcolor{gray!20}\textbf{\#}& \textbf{Effetto}\\
\toprule
5 & Il mirino è impreciso. Il prossimo colpo ha -2 al Tiro per Colpire.\\
\rowcolor{gray!20}6 & Spari inavvertitamente due colpi/raffiche. Il secondo non colpisce nessuno e fa solo consumare proiettili.\\
7 & L'arma non è adeguatamente lubrificata. Inserire il prossimo caricatore costa 1 round intero.\\
\rowcolor{gray!20}8 & Il caricatore si sgancia/cade. Devi caricare un nuovo caricatore (2 Azioni) o recuperare e rimettere il precedente caricatore.\\
9 & Il rinculo è così forte che il personaggio cade a terra prono.\\
\rowcolor{gray!20}10 & Proiettile molto incastrato. Per liberare il colpo devi eseguire una prova di Artigianato DC 15, 1 round.\\
11 & L'arma si surriscalda e non può essere usata fino alla fine del prossimo round.\\
\rowcolor{gray!20}12 & L'arma emette una fitta coltre di fumo attorno a te che fornisce copertura leggera verso te e da te verso gli altri.\\
13 & Parziale ostruzione della canna. Il prossimo colpo/raffica sparato fa metà danno.\\
\rowcolor{gray!20}14 & Il grilletto si incastra. Costa 3 Azioni sparare il prossimo colpo/raffica.\\
15 & Il rinculo è tale che ti cade l'arma per terra entro 1d4 metri di distanza.\\
\rowcolor{gray!20}16 & Il proiettile è esploso nella canna. Per liberare il colpo devi eseguire una prova di Artigianato DC 17, 1d4 round.\\
17 & Colpisci un altro. Il colpo prende una traiettoria non voluta e colpisci una creatura a caso nella direzione del colpo.\\
\rowcolor{gray!20}18 & Il rumore è talmente forte che il personaggio è assordato per 1 minuto.\\
19 & Il proiettile è esploso nella canna danneggiandola. È necessaria una prova di Artigianato DC 21 per ripristinare l'arma, 1 turno.\\
\rowcolor{gray!20}20 & L'intero caricatore è esploso. Subisci danno come se metà dei proiettili rimasti nell'arma ti colpissero. L'arma e proiettili sono distrutti.
\end{tabularx}

%Agile: usare l'arma senza scudo conferisce bonus +1
%Devastante: se crit fai danno da sanguinamento
%Fendente: +1 critico a soggetti proni
%Potente: x1.5 danno da forza
%Impatto: +2 tc contro armature pesanti
%Rompicostole: +2 tc contro armature leggere
%Precisa: solo 1 arma in mano, +1 tc
%Intralcio: -3 m movimento per colpo a segno

\end{multicols}

\vfill

\begin{center}
%\includegraphics[width=0.4\linewidth]{immagini/armiriempitivo3.png}
\includegraphics[width=0.8\linewidth]{immagini/Double-barreled_Shotgun.png}

\emph{Shotgun}
\end{center}

\pagebreak

\section{Equipaggiamento - Armature e Scudi} \index{Armature}\index{Scudi}\hypertarget{equipaggiamento.armature.scudi}{}\label{equipaggiamentoarmature}

\label{equipaggiamento---armature-e-scudi}

\begin{enfasi}{
Armatura (s.f.). Abito che si indossa se il proprio sarto è un fabbro. (Ambrose Bierce)

\medskip

Armatura Fantozzi: banderuola 4 venti in funzione di pennacchio, pauroso elmo vichingo con visibilità azzerata, sospensorio in bronzo sottratto alla statua di Pipino il Breve e, ai piedi, ferroni da stiro a carbonella di piombo fuso. Peso complessivo armatura Fantozzi: 4 quintali, 32 chili e 7 etti e mezzo. (Superfantozzi, Film)} \end{enfasi}

Le armature aiutano ad essere non colpiti (alzano la Difesa) e penalizzano la Prova di Magia e le prove di competenza di Base.

La Penalità Competenze è la penalità che si applica alle prove di competenza di Base influenzate dal peso ed Ingombro dell'armatura. Armature diverse, specifiche o magiche hanno punteggio diversi, questa tabella serve come linea guida per il Narratore.\index{Penalità armatura}

\subsubsection{Tabella Armature}\index[Tabelle]{Tabella Armature}

\label{tabella-armature}
\noindent\begin{tabularx}{\linewidth}{llcccccc}
	\toprule
\rowcolor{gray!20}\textbf{Armatura} & \textbf{Costo} & \textbf{Difesa} & \textbf{Pen. Comp.} & \textbf{Tipo} & \textbf{Mov.} & \textbf{Prova Magia}&\textbf{Ingombro}\\
\toprule
Imbottita & 5 mo & 1 & 0 & L 	& 		0 		& -		&2\\
\rowcolor{gray!20}Cuoio & 10 mo & 2 & 0 & L 		& 		0 		& +4		&2\\
Cuoio rinforzato& 25 mo& 3 & 0& L 	& 0 		& +4		&2\\
\rowcolor{gray!20}Giaco di Maglia & 15 mo & 4 & -1 & M & 0 		&+8			&4\\
Scaglie& 50 mo & 5& -1& M & 0 					&+8			&4\\
\rowcolor{gray!20}Anelli & 150 mo & 6& -1& M & 0 					&+8			&4\\
Pettorale& 200 mo & 6& -2& M & 0 				&+8			&4\\
\rowcolor{gray!20}Bande & 250 mo & 7& -2& P & 0 					&+16		&8\\
Mezza armatura& 1200 mo& 8& -2& P & 1 			&+16		&8\\
\rowcolor{gray!20}da Campo& 1350 mo& 9& -3& P & 2 				&+16		&8\\
Completa& 1500 mo& 10 & -4& P & 3 				&+16		&8
\end{tabularx}

\begin{multicols}{2}

\medskip

\textbf{Costo}: è per un armatura di taglia media.

\textbf{Difesa}: è il bonus data alla Difesa

\textbf{Penalità Comp.}: è la penalità dato alle prove di Competenza di Base dato dal peso ed Ingombro dell'armatura.

\textbf{Tipo}: indica se l'armatura è \textbf{L}eggera, \textbf{M}edia oppure \textbf{P}esante.

\textbf{Ingombro}: indica l'ingombro dell'armatura.

\textbf{Mov. (movimento)}: è la riduzione in metri di movimento da applicare per Azione di Movimento.

\textbf{Prova Magia}: Il - indica che non si è obbligati a farla. I numeri indicati è l'aumento di difficoltà della Prova di Magia.

\textbf{Costo}: il costo di un'armatura o scudo +1 è di 2250mo, +2 10000mo. Non è praticamente possibile acquistare armature o scudi od armi con incantamenti superiore a +2, devono essere \emph{trovate}.

\textbf{Controllate i requisiti} per indossare un \hyperlink{indossarearmature}{Armatura o Scudo} a pag. \pageref{indossarearmature}.


\subsubsection{Armature, Scudi e Magia}\index{Armature e Magia}\index{Scudi e Magia}\hypertarget{armatureemagie}{}\label{armatureemagie}

Tutte le Armature, ad esclusione dell'armatura Imbottita forzano chi lancia incantesimi ad superare una Prova di Magia con un aumento di difficoltà indicato nella tabella.

Es. Tups indossa una armatura Pettorale (armatura media) e lancia un incantesimo. E' obbligato dal portare l'armatura ad effettuare la Prova di Magia. Tira 3d6 +4 dadi (perché ha 9 punti in CM), ignora 2 dadi (perché ha preso 4 volte Adepto della Magia).

La difficoltà dell'incantesimo che lancia, Fulmine, e' 10+ 2*3 (livello Incantesimo) +8 (difficoltà aggiuntiva data dall'armatura)

Nella prova esce 4,5,5 / 3,4,\st{1},\st{1}. Toglie i due 1 per Adepto della Magia, valgono zero e se ne avesse fatti 3 sarebbe stato un Fallimento Critico!. Totale 4+5+5+3+4=21 , contro una difficoltà di 10+6+8=24. Tups non riesce a lanciare l'incantesimo!!!

Se Tups avesse indossato una Cuoio Rinforzato la difficoltà sarebbe aumentata \emph{solo} di 4 e quindi sarebbe riuscito a lanciare l'incantesimo.

\subsubsection{Descrizione delle Armature}

\textbf{Armature Leggere}

Fatte di materiali leggeri e flessibili, le armature leggere favoriscono gli avventurieri agili dato che offrono protezione senza sacrificare la mobilità.

\emph{Imbottita}. Le armature imbottite consistono di strati di tessuto e imbottitura cuciti insieme.

\emph{Cuoio}. Il corpetto e le protezioni delle spalle di questa armatura sono fatte di cuoio indurito dopo essere stato bollito nell'olio. Il resto dell'armatura è composto di
materiali più morbidi e flessibili.

\emph{Cuoio Rinforzato}. Fatta di cuoio duro ma flessibile, l'armatura di cuoio rinforzato è arricchita da rivetti o spuntoni.

\textbf{Armature Medie}

Le armature medie offrono più protezione di quelle leggere, ma limitano i movimenti.

\emph{Giaco di Maglia}. Composto di anelli metallici intrecciati tra di loro, un giaco di maglia viene indossato sopra strati di abiti o cuoio. Questo tipo di armatura offre una protezione modesta alla parte superiore del corpo, mentre il rumore degli anelli che strusciano fra di loro viene attutito dagli altri strati.

\emph{Scaglie}. Quest'armatura consiste in una cotta e gambali (a volte anche di una gonna separata) di cuoio coperti da pezzi di metallo sovrapposti, in maniera simile alle scaglie di un pesce. L'armatura è completa di guanti.

\emph{Anelli}. Quest'armatura è un'armatura di cuoio con dei pesanti anelli cuciti sopra. Gli anelli servono a rinforzare l'armatura contro i colpi di spada e d'ascia. L'armatura è completa di guanti.

\emph{Pettorale}. Questa armatura consiste di un corpetto di metallo indossato su uno strato di cuoio. Sebbene lasci braccia e gambe relativamente scoperte, l'armatura fornisce una buona protezione agli organi vitali del personaggio, senza procurargli grande ingombro.

\textbf{Armature Pesanti}

\emph{Bande}. Questa armatura e fatta di strisce di metallo cucite ad un robusto schienale di cuoio e maglia di ferro. Le dimensioni delle piastre metalliche, interconnesse alle bande di metallo e gli strati di armatura sottostanti la rendono una delle più protettive tra le armature.

\emph{Mezza Armatura}. La mezza armatura consiste in piastre di metallo sagomate che coprono gran parte del corpo del personaggio. Non comprende protezioni per le gambe oltre a dei semplici schinieri legati con lacci di cuoio.

\emph{da Campo}. Molto simile all'armatura completa ma più leggera in costruzione sacrificando un poco di protezione a favore di una maggiore flessibilità e mobilità.

\emph{Completa}. Quest'armatura consiste di piastre di metallo sagomate a incastro che coprono l'intero corpo. Un'armatura completa comprende guanti, stivali di cuoio pesanti, un elmo con visiera, e uno spesso strato di imbottitura sotto l'armatura. Fibbie e lacci distribuiscono il peso dell'armatura su tutto il corpo.


\subsubsection{Regole base per l'utilizzo dell'armatura}

\textbf{Usare un'Armatura senza l'adeguata competenza} impedisce di usare il bonus di Destrezza e diminuisce il bonus alla Difesa fornito di 1.

\textbf{Usare uno Scudo senza l'adeguata competenza} peggiora il Tiro per Colpire di 1 e diminuisce di 1 il Bonus Difesa concesso.

\textbf{Dormire in Armatura}: se si dorme in un'armatura media o pesante, il giorno seguente si è automaticamente \hyperlink{affaticato}{Affaticati}.

Dormire in un'armatura leggera non provoca Affaticamento.

La \textbf{capacità di movimento} del personaggio rimarrà la medesima fino all'armatura a bande poi calerà progressivamente. Il valore indicato nella colonna Mov. sono i metri in meno che il personaggio fa per Azione di Movimento.

Ad esempio un umano in armatura completa ha movimento 6 metri, un nano 3 metri.

\textbf{Peso}: il peso indicato si riferisce alla versione per personaggi di taglia Media. Le armature adattate per personaggi di taglia Piccola pesano la metà, mentre per quelli di taglia Grande pesano il doppio.

\textbf{Armature Perfette}\index{Armature Perfette}

Un'armatura perfetta è un armatura creata da un abilissimo fabbro che pur non essendo magica, grazie al suo perfetto bilanciamento, ha un +1 alla Difesa. Un fabbro per creare un armatura perfetta deve superare con un Successo Critico la DC impostata per la creazione dell'armatura. Un Armatura perfetta costa il doppio di un armatura normale.

\medskip

\begin{center}
	\includegraphics[width=0.75\linewidth]{immagini/donnacavalierecavallo.png}

	\emph{Rara immagine di un cavallo, ora estinti}
\end{center}

\subsubsection{Armature e Scudi magici}\index{Armature magiche}\index{Scudi magici}\label{armaturaescudimagici}\hypertarget{armaturaescudimagici}{}

Un armatura magica o scudo magico non solo protegge meglio ma è anche più leggera e affine alla magia.

Una armatura +1 abbassa di 1 la penalità di Competenza e di 1 metro la quella al movimento.
Una armatura o scudo +2 inoltre diminuisce la penalità alla Prova di Magia di 2. Una armatura +3 ulteriormente toglie 1 alla penalità di Competenza, riduce di 1m la penalità Movimento e riduce di ulteriori 2 la penalità dalla Prova di Magia.

\subsubsection{Gli Scudi}

Gli \textbf{Scudi} \index{Scudi}permettono di aumentare la propria Difesa, più lo scudo è imponente e pesante più protegge, più aumentano le penalità alle prove di competenza magica e meno rende facile combattere (penalità Tiro per Colpire).

Gli Scudi possono essere di tipo Leggero, Medio, Pesante.

\textbf{Bonus Difesa}: è il bonus che si applica alla Difesa quando lo scudo è indossato.

\textbf{Penalità TC}: è la penalità al Tiro per Colpire che si ha quando lo scudo è indossato e non si ha Forza almeno 3.

\textbf{Tipo}: indica la taglia dello scudo. \textbf{L}eggero, \textbf{M}edio, \textbf{P}esante.

Uno \textbf{Scudo} Leggero ha \textbf{Ingombro} 1, uno Scudo Medio ha Ingombro 2, uno Scudo Pesante ha Ingombro 4.\index{Imgombro per Scudi}

La penalità alla \textbf{Prova di Magia} si somma con quella eventualmente dovuta dall'armatura e si applica quando lo scudo è indossato.\index{Penalità Magia Scudo ed Armatura}

Uno scudo può essere usato come \hyperlink{armaimprovvisata}{arma improvvisata}. Uno scudo piccolo fa 1d4 di danno (B/T), uno scudo medio fa 1d6 di danno (B/T), uno scudo pesante fa 1d8 di danno (B/T).

Usare lo scudo come arma improvvisata non fa applicare il suo bonus alla Difesa se non si usa una Reazione per reimpostarlo alla Difesa dopo aver attaccato.

Imbracciare uno scudo occupa la mano ed il braccio.

\end{multicols}

\subsubsection{Tabella Scudi}\index[Tabelle]{Tabella Scudi}

\label{tabella-scudi}

\noindent\begin{tabular}{lcccccc}
	\toprule
\rowcolor{gray!20}\textbf{Scudi} & \textbf{Costo} & \textbf{Difesa} & \textbf{penalità TC} & \textbf{Prova magia} & \textbf{Tipo} & \textbf{Ingombro}\\
\toprule
Scudo leggero di legno& 3 mo&+1& 0& +2& L & 1\\
\rowcolor{gray!20}Scudo leggero di metallo & 9 mo&+1& 0& +2& L& 1\\
Scudo medio legno &5 mo &+2& 0& +4& M& 2\\
\rowcolor{gray!20}Scudo medio metallo&12 mo&+2& 0& +4& M& 2\\
Scudo pesante di legno & 9mo&+3 & 1& +6& P& 4\\
\rowcolor{gray!20}Scudo pesante di metallo & 20 mo&+3& 1& +6& P& 4\\
\end{tabular}

\begin{multicols}{2}

\subsubsection{Indossare e Togliere Armature}\index{Indossare e Togliere Armature}

Indossare e togliere armature è una operazione che richiede tempo ed attenzione, farlo in fretta non aiuta ed anzi tende a peggiorare la protezione fornita.

\end{multicols}

\textbf{Tabella: Tempi per indossare e togliere l'armatura}\index[Tabelle]{Tabella Tempi per indossare e togliere l'armatura}

\medskip

\noindent\begin{tabularx}{\linewidth}{Xlll}
	\toprule
\rowcolor{gray!20}\textbf{Tipo di Armatura}& \textbf{Indossare} & \textbf{Indossare in fretta} & \textbf{Togliere}\\
\toprule
Scudo& 1 Azione & - & 1 Azione\\
\rowcolor{gray!20}Imbottita, Cuoio, Cuoio rinforzata& 1 minuto& 3 round& - \\
Giaco di Maglia& 1 minuto& 5 round& 5 round\\
\rowcolor{gray!20}Scaglie, Anelli, Pettorale, Bande & 4 minuti & 1 minuto{*}& 1 minuto\\
Mezza armatura, da Campo, Completa& 4 minuti{*}{*}& 4 minuti{*}& 1d4+1 minuti
\end{tabularx}

\bigskip

\begin{multicols}{2}

{*} Se qualcuno aiuta, il tempo si dimezza. Un singolo personaggio che non sta facendo altro può aiutare uno o due personaggi adiacenti a lui. Due personaggi non possono aiutarsi l'un l'altro a indossare un'armatura contemporaneamente.

{*}{*} Bisogna essere aiutati per indossare questa armatura. Senza aiuto è possibile indossarla solo in fretta.

\textbf{Indossare un'armatura in fretta} implica una penalità di -1 alla Difesa fornita dall'Armatura ed una penalità aggiuntiva di +1 alle prove di Competenza di Base.

\end{multicols}




\vfill
%%
\begin{center}
	\includegraphics[width=0.65\linewidth]{immagini/buckler.png}

	\emph{Buckler, fronte e retro}
\end{center}


\pagebreak

\section{Merci e Servizi}\index{Merci}\index{Servizi}

\subsection{Ricchezza, Denaro ed Equipaggiamento}\index{Ricchezza, Denaro ed Equipaggiamento}

\begin{enfasi}{
- Doc... c'è soltanto bisogno di un pochino di plutonio.

\medskip

- Ah, sono certo che nell'85 il plutonio si compra nella drogheria sotto casa, ma nel '55 la faccenda è molto più complicata! (Ritorno al futuro, Film 1985)}
\end{enfasi}

\begin{multicols}{2}

\subsubsection{Vendere Tesori}

Nei sotterranei che esplorerai avrai ampie opportunità di trovare tesori, equipaggiamento, armi, armature e altro ancora. Di solito, potrai vendere tesori e ninnoli quando raggiungerai un paese o altro insediamento, purché tu riesca a trovare acquirenti e mercanti interessati al tuo bottino.

\medskip

\textbf{Armi, Armature e Altro Equipaggiamento }\index{Vendere Tesori}

Come regola generale, le armi, le armature ed il resto dell'equipaggiamento non danneggiato quando viene venduto viene pagato la metà del valore originale. Difficilmente le armi e le armature utilizzate dai mostri sono in condizioni ottimali per la vendita.

\medskip

\textbf{Oggetti Magici}

La vendita di oggetti magici è un problema. Trovare qualcuno che voglia comprare una pozione o pergamena non comporta grandi difficoltà ma la maggior parte degli oggetti sono fuori della portata delle tasche di chiunque salvo dei nobili più ricchi. Inoltre, a parte alcuni oggetti magici comuni, è difficile riuscire a trovare oggetti magici o incantesimi in vendita. Il valore della magia trasale la semplice moneta e dovrebbe essere sempre trattata con riguardo.

\medskip

\textbf{Gemme, Gioielli e Oggetti d'Arte}

Questi oggetti sono di più facile scambio e puoi decidere di scambiarli per soldi o usarli come moneta corrente nelle transazioni. Nel caso di tesori di eccezionale valore, il Narratore potrebbe richiedere che tu riesca prima a trovare un acquirente in un grosso paese o addirittura una comunità più grande. Una gemma viene sempre scambiata al suo valore pieno e non a metà come l'equipaggiamento.

\medskip

\textbf{Merci}

Sulle terre di confine, la maggior parte delle transazioni avvengono tramite baratto. Molte merci di pratico consumo come lingotti di ferro, sacchi di sale, bestiame e così via possono essere scambiate come moneta corrente al loro pieno valore.

\medskip

\textbf{Attrezzature Perfette}

Un attrezzo perfetto oltre a costare 10 volte tanto la versione normale, concede un +1 alla prova in cui si utilizza.

\medskip

%\end{multicols}

%\begin{center}
%	\includegraphics[width=0.6\linewidth]{immagini/armaturacorpetto.png}
%\end{center}

%\vfill
%\begin{center}
%\includegraphics[width=0.7\linewidth]{immagini/jewelry-box-2931784_1280.png}
%\end{center}
%\pagebreak

%\begin{multicols}{2}

\subsubsection{Equipaggiamento da Avventura}\index{Equipaggiamento avventura}\index{Cose da comprare}

Questo è un breve e non esaustivo elenco di equipaggiamento che i vostri personaggi potrebbero essere interessati a comprare. L'elenco non è certo esaustivo o completo ma potrà fornirvi linee guida sui prezzi.

Come Narratore usate sempre il buon senso nelle richieste valutate bene la tipologia di richiesta, la necessità dell'oggetto, il luogo dove si compra e come lo si compra.

In base alla tipologia di compagna potrebbero essere disponibili ulteriori oggetti quali armi da fuoco o alchemici.

\medskip\label{equipaggiamentolista}

\noindent\begin{tabularx}{\linewidth}{Xll}
	\toprule
\rowcolor{gray!20}\textbf{Oggetto}& \textbf{Costo} & \textbf{Ing.}\\
\toprule
Abaco&2 mo&L\\
\rowcolor{gray!20}Abito da Monaco & 5 mo& 1\\
Abito da artigiano& 1 mo& 1\\
\rowcolor{gray!20}Abito da contadino& 1 ma& 1\\
Abito da cortigiano& 30 mo& 1\\
\rowcolor{gray!20}Abito da esploratore& 10 mo& 1\\
Abito da intrattenitore & 3 mo& 1\\
\rowcolor{gray!20}Abito da nobile & 75 mo& 2\\
Abito da studioso & 5 mo& 1\\
\rowcolor{gray!20}Abito da viaggiatore& 1 mo& 2\\
Abito invernale & 8 mo& 2\\
\rowcolor{gray!20}Abito reale & 200 mo & 3\\
\hyperlink{Acido Intenso}{Acido Intenso} (ampolla) & 30 mo& L \\
\rowcolor{gray!20}\hyperlink{Acqua santa}{Acqua santa} (ampolla) & 25 mo& L\\
Ago da cucito & 5 ma &- \\
\rowcolor{gray!20}Agrifoglio e vischio & & -\\
Amo da pesca & 1 ma & - \\
\rowcolor{gray!20}Ampolla (vuota) & 3 mr & L \\
Anello con sigillo& 5 mo& - \\
\rowcolor{gray!20}\hyperlink{Anello per veleno}{Anello per veleno} & +20 mo&-\\
\hyperlink{Antitossina}{Antitossina} (boccetta)& 50 mo& L\\
\rowcolor{gray!20}\hyperlink{Ariete portatile}{Ariete portatile}& 10 mo& 3 \\
Arnesi da artigiano& 5 mo& 2\\
\rowcolor{gray!20}\hyperlink{Attrezzi da scasso}{Arnesi da scasso}& 30 mo& 1\\
Asta (3 m)& 5 mr& 2\\
\rowcolor{gray!20}Attrezzi da scalatore & 80 mo& 1\\
Banchetto (a persona) & 10 mo& -\\
\rowcolor{gray!20}\hyperlink{Bandoliera}{Bandoliera} & 3 mo & L\\
\hyperlink{Barca a remi}{Barca a remi}& 50 mo& 12\\
\rowcolor{gray!20}Barcone & 3000 mo& -\\
Barile (vuoto)& 2 mo& 4\\
\rowcolor{gray!20}Bastone & 2 mo& 1\\
Bilancia da mercante& 2 mo& 1\\
\rowcolor{gray!20}Birra Boccale& 5 mr& L\\
Birra Caraffa & 2 ma & L\\
\rowcolor{gray!20}Boccale di ceramica & 2 mr & L \\
\end{tabularx}
\noindent\begin{tabular}{p{5cm}p{1.5cm}p{0.7cm}}
\toprule
\rowcolor{gray!20}\textbf{Oggetto} & \textbf{Costo} & \textbf{Ing.} \\
\toprule
Boccetta di inchiostro o pozione & 1 mo & L\\
\rowcolor{gray!20}Borsa & 5 ma & L \\
Alchimista (laboratorio) & 200 mo & 5 \\
\rowcolor{gray!20}Borsa da cintura & 1 mo & L \\
\hyperlink{borsadaguaritore}{Borsa da Guaritore} & 20 mo & 1 \\
\rowcolor{gray!20}\hyperlink{Borsa dei Componenti}{Borsa dei Componenti} & 25 mo & L \\
Bottiglia di vetro & 2 mo & L \\
\rowcolor{gray!20}Brocca di ceramica (5lt) & 2 mr & L \\
Campanella & 1 mo & - \\
\rowcolor{gray!20}Candela & 1 mr & - \\
Canna da pesca & 1 mo & 1 \\
\rowcolor{gray!20}Cannocchiale & 900 mo & 1 \\
Caraffa di ceramica & 2 mr & L \\
\rowcolor{gray!20}Carne (1 pezzo) & 3 ma & L \\
Carretto & 15 mo & 10 \\
\rowcolor{gray!20}Carro & 35 mo & - \\
Carrozza & 300 mo & - \\
\rowcolor{gray!20}\hyperlink{Carrucola e paranco}{Carrucola e paranco} & 20 mo & 2 \\
Carta (foglio) & 4 ma & - \\
\rowcolor{gray!20}Cassa (vuota) & 2 mo & 3 \\
Catena (3 m) & 30 mo & 1 \\
\rowcolor{gray!20}Ceralacca & 1 mo & - \\
Cerata & 5 ma & 1 \\
\rowcolor{gray!20}Cesto (vuoto) & 4 ma & 1 \\
Chiodo da rocciatore & 1 ma & L \\
\rowcolor{gray!20}\hyperlink{Cintura porta oggetti}{Cintura porta oggetti} & 2 mo & L \\
Clessidra & 25 mo & - \\
\rowcolor{gray!20}Coperta invernale & 5 ma & 1 \\
\hyperlink{Corda}{Corda} di canapa (15 m) & 1 mo & 1 \\
\rowcolor{gray!20}Corda di canapa grossa (15 m) & 2 mo & 2 \\
Corda di seta di ragno (15 m) & 10 mo & L \\
\rowcolor{gray!20}Cote per affilare & 2 mr & L \\
Custodia per pergamene & 1 mo & 1 \\
\rowcolor{gray!20}\hyperlink{Esca ed Acciarino}{Esca ed Acciarino} & 5 ma & L \\
\hyperlink{Faretra}{Faretra} & 3 mo & 1 \\
\rowcolor{gray!20}Fischietto & 8 ma & - \\
Formaggio (1 pezzo) & 1 ma & - \\
\rowcolor{gray!20}Forziere & 5 mo & 4 \\
Fuoco dell'Alchimista (ampolla) & 20 mo & L \\
\rowcolor{gray!20}Galea & 30k mo & - \\
Gancio di metallo & 1 mo & L \\
\rowcolor{gray!20}Gessetto (1 pezzo) & 1 mr & - \\
Giaciglio & 1 ma & 1 \\
\rowcolor{gray!20}Inchiostro (boccetta da 30 g) & 8 mo & - \\
\hyperlink{Lanterna}{Lanterna} & 1 mo & 2 \\
\rowcolor{gray!20}Lanterna a lente sporgente & 12 mo & 1 \\
Lanterna schermabile & 7 mo & 1 \\
\rowcolor{gray!20}Legna da ardere (per giorno) & 1 mr & 4 \\
Lente del cacciatore & 100 mo & - \\
\rowcolor{gray!20}Locanda Buona (dormire) & 2 mo & - \\
Locanda Modesta (dormire) & 5 ma & - \\
\rowcolor{gray!20}Locanda Scadente (dormire) & 1 ma & - \\
Maglio & 1 mo & 2 \\
\rowcolor{gray!20}Martello & 5 ma & 1 \\
Morso e briglie & 2 mo & 1 \\
\rowcolor{gray!20}Nave a vela & 10k mo & - \\
Nave da guerra & 25k mo & - \\
\rowcolor{gray!20}Nave lunga & 10k mo & - \\
\hyperlink{Olio da lanterna}{Olio da lanterna} & 1 ma & 1 \\
\rowcolor{gray!20}Orologio ad acqua & 900 mo & - \\
Otre & 1 mo & 2 \\
\end{tabular}
\noindent\begin{tabular}{p{5.1cm}p{1.5cm}p{0.7cm}}
\toprule
\rowcolor{gray!20}\textbf{Oggetto} & \textbf{Costo} & \textbf{Ing.} \\
\toprule
\hyperlink{Manette}{Manette} & 15 mo & L \\
\rowcolor{gray!20}Pala o badile & 2 mo & 1 \\
Pane (a pagnotta) & 2 mr & - \\
\rowcolor{gray!20}Pasti (al giorno) Buono & 5 ma & - \\
Pasti (al giorno) Modesto & 3 ma & - \\
\rowcolor{gray!20}Pasti (al giorno) Povero & 6 mr & - \\
Peluche & 2 ma & - \\
\rowcolor{gray!20}Pennino & 1 ma & - \\
Pentola di ferro & 8 ma & 1 \\
\rowcolor{gray!20}Pergamena (Foglio) & 2 ma & - \\
Piccone da minatore & 3 mo & 2 \\
\rowcolor{gray!20}\hyperlink{piedediporco}{Piede di porco} & 2 mo & 1 \\
\hyperlink{Equip Pozione di Cura}{Pozione di Cura} & 50 mo & L \\
\rowcolor{gray!20}\hyperlink{Equip Pozione di Cura potenziata}{Pozione di Cura potenziata} & 125 mo & L \\
Profumo & 5 mo & L \\
\rowcolor{gray!20}Rampino & 1 mo & 1 \\
Razioni da viaggio (al giorno) & 3 ma & L \\
\rowcolor{gray!20}Remo & 2 mo & 2 \\
Rete da pesca (2,25 m) & 4 mo & 1 \\
\rowcolor{gray!20}Sacche da sella & 4 mo & 2 \\
Sacco (vuoto) & 1 ma & L \\
\rowcolor{gray!20}Sacco a pelo & 3 mo & 2 \\
Sapone (per 0,5 kg) & 5 ma & - \\
\rowcolor{gray!20}Scala a pioli (3 m) & 2 ma & 3 \\
Secchio (vuoto) & 5 ma & L \\
\rowcolor{gray!20}Sella Da galoppo & 30 mo & 2 \\
Sella da carico & 15 mo & 2 \\
\rowcolor{gray!20}Sella esotica & 60 mo & 3 \\
Sella Militare & 50 mo & 3 \\
\rowcolor{gray!20}Serratura/lucchetto Media & 40 mo & - \\
Serratura/lucchetto Semplice & 20 mo & - \\
\rowcolor{gray!20}Serratura/lucchetto Superiore & 150 mo & - \\
Sfere Metalliche (100) & 3 mo & 1 \\
\rowcolor{gray!20}Simbolo sacro d'argento & 25 mo & L \\
Simbolo sacro di legno & 1 mo & L \\
\rowcolor{gray!20}Slitta & 20 mo & 3 \\
Specchio piccolo di metallo & 10 mo & L \\
\rowcolor{gray!20}Stallaggio (al giorno) & 1 ma & - \\
Strumento musicale comune & 5 mo & 2 \\
\rowcolor{gray!20}Tagliola & 5 mo & 3 \\
Tela (per mq) & 1 ma & L \\
\rowcolor{gray!20}Tenda & 10 mo & 3 \\
\hyperlink{Torcia}{Torcia} & 1 ma & 1 \\
\rowcolor{gray!20}\hyperlink{Trappola da Caccia}{Trappola da Caccia} & 12 mo & 2\\
\hyperlink{Tribolo}{Tribolo} (20) & 1 ma & L \\
\rowcolor{gray!20}Trucchi per il camuffamento & 50 mo & L \\
Vanga o Badile & 1 mo & 1 \\
\rowcolor{gray!20}Veste da Devoto & 5 mo & 1 \\
Vino Buono (bottiglia) & 10 mo & 1 \\
\rowcolor{gray!20}Vino della casa (caraffa) & 2 ma & 1 \\
Zaino & 2 mo & 1 \\
\end{tabular}

\medskip

\begin{enfasi}{
Ogni tecnologia sufficientemente avanzata è indistinguibile dalla magia. (Arthur C. Clarke, da Profiles of the Future)
}\end{enfasi}

\medskip

\textbf{Acido Intenso}\label{Acido Intenso}\index{Acido intenso}\hypertarget{Acido Intenso}{}. Con un'Azione, puoi spargere il contenuto di questa fiala su di una creatura entro 1 metro da te o lanciare la fiala fino a 6 metri, fracassandola all'impatto. In entrambi i casi, effettua un Tiro per Colpire a distanza contro la creatura o l'oggetto, trattando l'acido come un'\hyperlink{armaimprovvisata}{arma improvvisata}. Se colpisci, il bersaglio subisce 2d6 danni da acido il primo round ed 1d6 il secondo round.

\textbf{Acqua santa}\label{Acqua santa}\index{Acqua santa}\hypertarget{Acqua santa}{}. Con un'Azione, puoi spargere il contenuto di questa ampolla su di una creatura entro 1 metro da te o lanciare l'ampolla fino a 6 metri, fracassandola all'impatto. In entrambi i casi, effettua un Tiro per Colpire a distanza contro la creatura o l'oggetto, trattando l'Acqua santa come un'arma improvvisata. Se colpisci, e il bersaglio è un immondo o un non morto, subisce 2d4 danni da energia positiva.

\textbf{Ampolla (vuota)}: piccola anfora in vetro o ceramica con collo sottile.

\textbf{Anello con Sigillo}: cerchietto di metallo, generalmente pregiato, con un'incisione atta ad imprimere sigilli su ceralacca.

\textbf{Anello per Veleno}\label{Anello per veleno}\hypertarget{Anello per veleno}{}: +20 mo, rispetto a costo anello, questo anello ha un piccolo scompartimento sotto la gemma, di solito utilizzato per contenere del veleno. Aprirlo e chiuderlo richiede un'Azione; farlo senza essere notati richiede una prova di Mani di Fata con DC 15.

\textbf{Antitossina}\index{Antitossina}\label{Antitossina}\hypertarget{Antitossina}{}. Una creatura che beve da questa fiala di liquido ottiene +1d6 sui Tiri Salvezza contro il veleno per 1 ora. Non conferisce alcun bonus ai non morti e ai costrutti.

\textbf{Ariete portatile}\index{Ariete portatile}\label{Ariete portatile}\hypertarget{Ariete portatile}{}. Puoi usare un ariete portatile per abbattere le porte con un bonus di +1d6. Un altro personaggio può aiutarti con l'uso dell'ariete, dandoti +2 sulla prova.

\textbf{Attrezzatura da Pesca}. Questo kit comprende un'asta di legno, filo di seta, taglierino di legno, ami d'acciaio, peso di piombo, esche di velluto e un retino.

\textbf{Bandoliera}\label{Bandoliera}\hypertarget{Bandoliera}{}. Questa cintura specializzata per contenere piccoli oggetti quali pozioni o pergamene si porta al collo e porta fino a 6 oggetti. Bere una pozione contenuta nella Bandoliera è una Azione Immediata se viene usata un altra Azione per estrarla.\index{Bandoliera}

\textbf{Biglie di Metallo}. Con un'Azione, puoi spargere una singola borsa di queste minuscole biglie di metallo per coprire un quadrato di 3 metri di lato. Una creatura che attraversa l'area coperta deve superare un Tiro Salvezza su Riflessi con DC 13 o cadere prona. Una creatura che attraversa l'area a metà velocità non deve effettuare il Tiro Salvezza.

\textbf{Bilancia da Mercante}. Una bilancia da mercante include un piccolo bilanciere, un piatto, e un assortimento di pesi fino a 1 chilo. Con essa, puoi misurare il peso esatto di piccoli oggetti, come metalli preziosi o merci, per aiutarti a determinarne il valore.

\textbf{Cintura porta oggetti}. Questa cintura specializzata per contenere fino a 4 piccoli oggetti quali pozioni o pergamene. Bere una pozione contenuta nella Cintura porta oggetti è una Azione Immediata se viene usata un altra Azione per estrarla.\index{Cintura porta oggetti}\label{Cintura porta oggetti}\hypertarget{Cintura porta oggetti}{}\index{Cintura porta oggetti}

\textbf{Borsa dei Componenti}\label{Borsa dei Componenti}\hypertarget{Borsa dei Componenti}{}. Una borsa dei componenti è un piccolo borsello da cinta di cuoio impermeabile munito di compartimenti contenenti tutte le componenti materiali e altri oggetti speciali di cui hai bisogno per lanciare i tuoi incantesimi, eccetto per quelle componenti che hanno un costo specifico o sono materiali particolari.

\begin{center}
	\includegraphics[height=0.8\linewidth]{immagini/stadera.png}
\end{center}

\textbf{Borsa da cintura}. Un borsello di tessuto o cuoio può contenere, tra le altre cose, fino a 20 proiettili da fionda o 500 monete. Un borsello diviso in compartimenti per contenere componenti per incantesimi viene detto borsa dei componenti.

\textbf{Candela}\hypertarget{Candela}{}\label{Candela}. Per 1 ora di tempo reale di gioco, una candela proietta luce fioca in un raggio di 1 metro.

\textbf{Cerata}. E' un mantello trattato per essere idrorepellente, ti permette di rimanere asciutto anche sotto la pioggia.

\textbf{Cannocchiale}. Gli oggetti osservati tramite un cannocchiale sono ingranditi al doppio delle loro dimensioni.

\textbf{Carrucola e Paranco}. Una serie di leve collegate da un cavo e un gancio per attaccarsi ad oggetti, carrucola e paranco ti permettono di tirare su fino a quattro volte il peso che puoi normalmente sollevare.\label{Carrucola e Paranco}\index{Carrucola e Paranco}\hypertarget{Carrucola e paranco}{}

\textbf{Catena}\index{Catena}\hypertarget{Catena}{}. Una catena ha 15 Punti Ferita e durezza 6. Può essere spezzata superando un Tiro Salvezza Tempra con Forza con DC 24. Costa 7 ma per metro.

\textbf{Chiodi da rocciatore}. Se ne deve usare 1 almeno ogni 6 metri per fissare la corda alla parete.

\textbf{Colonia di Scarafaggi Necrofagi}: 3 mo, questa giara di vetro contiene scarafaggi necrofagi carnivori. Gli scarafaggi devono essere nutriti con almeno 125 grammi di carne al giorno oppure muoiono. Quando rilasciati su un organismo morto, ne divorano le carni in 1d4 giorni, lasciando solo le ossa. Gli scarafaggi necrofagi mangiano soltanto la carne morta e non possono danneggiare le creature viventi. Una volta rilasciati, gli scarafaggi non possono essere rimessi nella giara.

\textbf{Corda}. Una corda, lunga solitamente 18 metri, è fatta di canapa, ha 2 Punti Ferita, e può essere spezzata superando un Tiro Salvezza Tempra con Forza con DC 19. La versione grossa ha 6 Punti Ferita, DC 22.\index{Corda}\label{Corda}\hypertarget{Corda}{}

\textbf{Corda di Seta} (15 m): 10 mo, questa corda di seta di ragno ha 8 Punti Ferita e può essere spezzata con un Tiro Salvezza Tempra con Forza con DC 23

\textbf{Faretra}. 3 mo, una faretra può contenere fino a 12 frecce\index{Faretra} o dardi.\label{Faretra}\hypertarget{Faretra}{}

\textbf{Frecce}. Con la Professione Guardiaboschi o Falegname è possibile fare 1d6 frecce da caccia in un ora di lavoro, un fabbro con l'attrezzatura può preparare 1d4 frecce da guerra in 1 ora di lavoro.

\textbf{Borsa da Guaritore}. 20 mo. \label{borsadaguaritore}\hypertarget{borsadaguaritore}{}\index{Borsa da Guaritore} Questo kit è una borsa di cuoio contenente bende, unguenti e stecche. Il kit può essere usato dieci volte. Concede un +2 alle prove di Pronto Soccorso.

\textbf{Kit da Pranzo}. 4 mo. Questa piccola scatola di latta contiene una ciotola e delle semplici posate. Le due parti della scatola possono essere staccate, e un lato impiegato come pentola per cucinare e l'altro come piatto o contenitore

\textbf{Kit da Scalatore}. 8 mo. Un kit da scalatore comprende chiodi speciali, punte per stivali, guanti e un'imbracatura. Puoi ancorarti usando il kit da scalatore con un'Azione; quando lo fai, non puoi cadere per più di 7 metri dal punto in cui ti sei ancorato, e non puoi arrampicarti a più di 7 metri di distanza dal punto a cui ti sei ancorato senza prima disfare l'ancora.

\textbf{Lanterna}\label{Lanterna}\index{Lanterna}\hypertarget{Lanterna}{}. Una Lanterna proietta luce intensa in un raggio di 3 metri e luce fioca per 6 metri. Una volta accesa, brucia per 3 ore di tempo reale di gioco con un'ampolla (0,5 litri) d'olio da lanterna. Una lanterna si accende in un round solo con Esca e Acciarino oppure con una altra fonte di fiamma.

\textbf{Lanterna a Lente Sporgente}. Una lanterna a lente sporgente proietta luce in un cono di 3 metri e luce fioca per 9 metri. Una volta accesa brucia per 3 di tempo reale di gioco ore con un'ampolla (0,5 litri) d'olio.Una lanterna si accende in un round solo con Esca e Acciarino oppure con una altra fonte di fiamma.

\textbf{Lanterna Schermabile}. Una lanterna schermabile proietta luce in un raggio di 6 metri e luce fioca per 6 metri. Una volta accesa brucia per 1 ora di tempo reale di gioco con un'ampolla (0,5 litri) d'olio. Con un'Azione, puoi abbassare la schermatura, riducendo la luce a fioca con un raggio di 1 metro. Una lanterna si accende in un round solo con Esca e Acciarino oppure con una altra fonte di fiamma.

\textbf{Lente del Cacciatore}: 100 mo, questa complessa lente viene posta su un occhio e occupa lo slot occhi quando è in uso. Quando la si utilizza con un attacco a distanza, si riduce di 1d6 le penalità da attacchi a distanza. Gli oggetti entro 9 metri diventano difficili da vedere, e si subisce penalità -1d6 alle prove di Consapevolezza basate sulla vista e Tiri per Colpire.

\textbf{Manette}\label{Manette}\index{Manette}\hypertarget{Manette}{}. Questi strumenti di metallo possono imprigionare una creatura Piccola o Media. Per romperle bisogna superare un Tiro Salvezza Tempra con Forza con DC 24. Ogni set di manette è fornito di una chiave. Senza la chiave, una creatura può usare Artista della Fuga o Disattivare Congegni per aprire la serratura superando una prova DC 18. Le manette hanno 15 Punti Ferita e Durezza 3

\textbf{Olio da lanterna}\hypertarget{Olio da lanterna}{}. Di solito si compra in un'ampolla d'argilla che contiene 0,5 litri. Si usa per ricaricare le lanterne, Usata come arma improvvisata vedere il \hyperlink{Fuoco dell'Alchimista}{Fuoco dell'Alchimista}, considerando che il danno è 1d4.

\begin{center}
	\includegraphics[width=0.7\linewidth]{immagini/forziere.png}
\end{center}

\textbf{Piede di porco}. Utilizzare un piede di porco dà +1d6 alle prove di Forza ogni volta si possa applicare la leva del piede di porco.\hypertarget{piedediporco}{}\label{piedediporco}\index{Piede di porco}

\textbf{Pozione di Cura}\index{Pozione di Cura}\hypertarget{Equip Pozione di Cura}{}. Questa pozione generica di cura consente di recuperare 1d8+1 Punti Ferita. Vedi anche \hyperlink{pozionigeneriche}{Pozioni generiche} pag. \pageref{pozionigeneriche}.

\textbf{Pozione di Cura potenziata}\label{Pozione di Cura potenziata}\hypertarget{Equip Pozione di Cura potenziata}{}. Questa pozione generica di cura consente di recuperare 3d8+3 Punti Ferita.

\begin{narratore}[Cure disponibili]
Per quanto sia personalmente contrario all'acquisto di oggetti magici da parte dei personaggi le Pozioni di Cura devono essere disponibili.
\end{narratore}

\textbf{Razioni}\index{Razioni}\hypertarget{Razioni}{}\label{Razioni}. Le razioni consistono di cibo secco adatto a lunghi viaggi, e includono carne secca, frutta secca, gallette e noci.

\textbf{Esca ed Acciarino}\index{Esca ed Acciarino}\label{Esca ed Acciarino}\hypertarget{Esca ed Acciarino}{}. Questo piccolo contenitore contiene pietra, acciarino ed esca (di solito uno straccio secco imbevuto d'olio) impiegati per appiccare un fuoco. Utilizzarlo per accendere una torcia (o qualsiasi altro oggetto facilmente incendiabile) richiede due Azioni.\index{Accendere una torcia}

\textbf{Scatola per Mappe o Pergamene}. Questa scatola cilindrica di cuoio può contenere, arrotolati, fino a dieci pezzi di carta o cinque fogli di pergamena.

\textbf{Serratura}\hypertarget{Serratura}{}. Insieme alla serratura viene fornita una chiave. Senza la chiave, una creatura può scassinare questa serratura superando una prova di Disattivare Congegni con DC variabile a seconda della bontà della stessa.

%\begin{center}
%\includegraphics[width=0.6\linewidth]{immagini/serratura.png}
%\end{center}

\textbf{Simbolo Sacro}\index{Simbolo Sacro}\label{Simbolo Sacro}. Un simbolo sacro è la raffigurazione di un Patrono. Potrebbe essere un amuleto che raffigura il simbolo di un Patrono, lo stesso simbolo accuratamente inciso o intrecciato su di un emblema o scudo, o una minuscola scatola contenente una reliquia sacra.

\textbf{Tappi per Orecchie} 3 mr, fatti di cotone o sughero cerato, i tappi per orecchie concedono Bonus +2 al Tiro Salvezza contro gli effetti che richiedono l'udito ma infliggono penalità -4 alle prove di Consapevolezza basate sull'udito.

\textbf{Tenda}\label{Tenda}\index{Tenda}\hypertarget{Tenda}{}. Un semplice riparo portabile di tela, una tenda può contenere due persone. Ci vogliono circa 20 minuti per montare una tenda.

\textbf{Tomo di Magia}. Un Tomo di Magia di 50 pagine, ovvero che può contenere 50 livelli di incantesimo costa 250 mo. Solitamente se l'incantatore proviene da una Accademia di Magia o da una congrega di Devoti può acquistarlo a metà del prezzo.\index{Tomo di Magia, acquistare}

\textbf{Torcia}\label{Torcia}\hypertarget{Torcia}{}. Una torcia brucia per \textbf{1 ora di tempo di gioco reale}, fornendo luce in un raggio di 3 metri e luce fioca per 6 metri. Se effettui un Tiro per Colpire con una torcia accesa, arma improvvisata, e colpisci, infliggi 1d6 di danno più 1 danno da fuoco aggiuntivo ma consumi 10 minuti della sua durata.

Sono necessarie 2 Azioni usando Esca ed Acciarino per accendere una Torcia, 1 Azione se la accendi tramite un altro fuoco già acceso, altrimenti è necessaria una prova di Sopravvivenza a DC 17 e ci si impiega 1 minuto.\index{Torcia}\index{Accendere una torcia}

\textbf{Trappola da Caccia}\hypertarget{Trappola da Caccia}{}\index{Trappola da Caccia}\label{Trappola da Caccia}. 12 mo, 2. Usi due azioni per disporre questa trappola, formata da un anello d'acciaio seghettato, che scatta quando una creatura calpesta la piastra metallica al centro di essa. La trappola è fissata tramite una catena pesante a un oggetto immobile, come un albero o uno spuntone conficcato nel terreno. Una creatura che calpesti la piastra deve superare un Tiro Salvezza su Riflessi con DC 15 o subire 1d4 danni perforanti e interrompere il movimento. Una creatura può usare 2 azioni per superare un Tiro Salvezza Tempra con Forza con DC 15, e se la riesce si libera o libera un'altra creatura a portata. Ogni tentativo fallito infligge 1 danno perforante alla creatura intrappolata.

%\begin{wrapfigure}{r}{0.3\textwidth}
\begin{center}
\includegraphics[width=0.3\textwidth]{immagini/tribolo.png}
\end{center}
%\end{wrapfigure}

%\begin{center}
%\includegraphics[width=0.2\linewidth]{immagini/tribolo.png}
%\end{center}

\textbf{Tribolo}\label{Tribolo}\index{Tribolo}\hypertarget{Tribolo}{}. Con un'Azione, puoi spargere una singola borsa di questi minuscoli triboli per coprire un'area quadrata di 1 metro di lato. Una creatura che attraversa l'area coperta deve superare un Tiro Salvezza su Riflessi con DC 15 o subire 1 danno perforante. Finché la creatura non recupera almeno 1 punto ferita, la sua velocità a piedi è diminuita di 3 metri. Una creatura che attraversa l'area a metà velocità non deve effettuare il Tiro Salvezza.

\textbf{Veleno Base}\index{Veleno Base}\label{Veleno Base}. Puoi usare il veleno in questa fiala per coprire un'arma tagliente o perforante o fino a tre pezzi di munizioni. Applicare il veleno necessita un'azione. Una creatura colpita da un'arma o munizione avvelenata deve superare un Tiro Salvezza su tempra con DC 12 o subire 1d4 danni da veleno.
Una volta applicato, il veleno mantiene la sua efficacia per 1 minuto prima di seccarsi.

\subsubsection{Dotazioni di base}\index{Dotazioni di base}\hypertarget{Dotazioni di base}{}
Se il personaggio sceglie di acquistare il suo equipaggiamento di partenza, può acquistare una dotazione al prezzo indicato, che generalmente è più conveniente rispetto all'acquisto dei singoli oggetti separati.

\textbf{Dotazione da Avventuriero (18 mo)}\index{Dotazione da Avventuriero}. Include uno zaino, un piede di porco, un martello, 10 chiodi da rocciatore, 10 torce, Esca ed Acciarino, un giaciglio, 10 razioni giornaliere e un otre. La dotazione include anche 15 metri di corda di canapa legata allo zaino.

\textbf{Dotazione da Cacciatore (24 mo)}\index{Dotazione da Cacciatore}: contiene Esca ed Acciarino, una borsa da cintura, una corda 18m, un giaciglio, una cerata, un otre, una pentola di ferro, razioni da viaggio (5 giorni), torce (10) e uno zaino.

\textbf{Dotazione da Diplomatico (57 mo)}\index{Dotazione da Diplomatico}. Include un forziere, 2 custodie per mappe e pergamene, un abito pregiato, una boccetta di inchiostro, un pennino, una lanterna, 2 ampolle di olio, 5 fogli di carta, una fiala di profumo, cera per sigillo e sapone.

\textbf{Dotazione da Devoto (30 mo)}\index{Dotazione da Devoto}: contiene Esca ed Acciarino, una borsa da cintura, una Borsa per Componenti di Incantesimi, candele (10), corda 18m, un giaciglio, una pentola di ferro, un otre, razioni da viaggio (per 5 giorni), sapone, un simbolo sacro di legno, un testo sacro economico, torce (10) e uno zaino.

%\textbf{Dotazione da Esploratore (16 mo)}\index{Dotazione da Esploratore}. Include uno zaino, un giaciglio, una gavetta, Esca ed Acciarino, 10 torce, 10 razioni giornaliere, un piede di porco e un otre. La dotazione include anche 15 metri di corda di canapa legata allo zaino.

\textbf{Dotazione da Esploratore di caverne (24 mo)}\index{Dotazione da Esploratore di caverne}: contiene un insieme di attrezzi di base per esplorare rovine e città abbandonate include 2 candele, un piede di porco, un gessetto, un martello e 4 Chiodi da Rocciatore, 18 metri di corda, una lanterna schermabile con 5 ampolle d'olio, 2 sacchi, 2 torce, razioni da viaggio (per 3 giorni)

\textbf{Dotazione da Intrattenitore (60 mo})\index{Dotazione da Intrattenitore}. Include uno zaino, un giaciglio, 2 costumi, 5 candele, 5 razioni giornaliere, un otre e trucchi per il camuffamento.

\textbf{Dotazione da Scassinatore (24 mo)}\index{Dotazione da Scassinatore}. Include uno zaino, un sacchetto con 1000 sfere metalliche, 3 metri di spago, una campanella, 5 candele, un piede di porco, un martello, 10 chiodi da rocciatore, una lanterna schermabile, 2 ampolle di olio, 5 razioni giornaliere, Esca ed Acciarino e un otre. La dotazione include anche 15 metri di corda di canapa legata allo zaino.

\textbf{Dotazione da Studioso (60 mo)}\index{Dotazione da Studioso}. Include uno zaino, un libro di studio, una boccetta d'inchiostro, un pennino, 10 fogli di pergamena, un sacchetto di sabbia e un coltellino.

\end{multicols}

\subsubsection{Capienza dei Contenitori}

\noindent\begin{tabularx}{\linewidth}{lXl|lXl}
	\toprule
\rowcolor{gray!20}\textbf{Oggetto}&\textbf{Capienza}&\textbf{CdC}&\textbf{Oggetto}&\textbf{Capienza}&\textbf{CdC}\\
\toprule
Borsa&1 cubo con spigolo di 30 cm/3 kg di equipaggiamento&1&Barile&160 litri liquidi, 4 cubi con spigolo di 30 cm&35\\
\rowcolor{gray!20}Boccale&0,5 litri&L&Bottiglia&1 litro di liquido&L\\
Secchio&12 litri liquidi, 1 cubo con spigolo di 25 cm&3&Canestro&2 cubi con spigolo di 30 cm/20 kg di equipaggiamento&5\\
\rowcolor{gray!20}Sacco&1 cubo con spigolo di 30 cm/15 kg di equipaggiamento&3&Forziere&12 cubi con spigolo di 30 cm/150 kg di equipaggiamento&35\\
Fiala&120 ml di liquidi&L&Otre&2 litri liquidi&1\\
\rowcolor{gray!20}Caraffa&4 litri liquidi&2&Zaino&2 cubi con spigolo di 30 cm/30 kg di equipaggiamento&6
\end{tabularx}

\medskip

Esiste anche la versione dello Zaino Perfetto (+100 mo) che concede un +2 al valore di Ingombro trasportabile.

\begin{multicols}{2}

\subsubsection{Strumenti}\index{Strumenti}\index{Attrezzi}

L'elenco degli strumenti o attrezzi presentati aiuta i personaggi ad eseguire le prove legate alle loro professioni o situazioni.

Le prove legate alle professioni sono solitamente calcolate come 3d6+LV/2 + Saggezza.

Ad esempio una prova di \emph{Calligrafia} si risolve con una prova di Saggezza, se il personaggio ha a disposizione gli strumenti idonei (\emph{Scorte da Calligrafo}) ottiene un bonus di +2 alla prova.

Se il personaggio effettua una prova utilizzando strumenti o attrezzi specifici ha un bonus di +2.

\end{multicols}

\medskip

\noindent\begin{tabularx}{\linewidth}{llX|llX}
	\toprule
\rowcolor{gray!20}\textbf{Oggetto}&\textbf{Costo}&\textbf{Ing.}&\textbf{Oggetto}&\textbf{Costo}&\textbf{Ing.}\\
\toprule
Arnesi da Scasso/da Falsario&25 mo&1&Borsa da Erborista&5 mo&1\\
\rowcolor{gray!20}Dadi&1 ma&-&Mazzo di Carte&5 ma&-\\
Sostanze da Avvelenatore&50 mo&1&Scorte da Alchimista&50 mo&2\\
\rowcolor{gray!20}Scorte da Calligrafo&10 mo&1&Scorte da Mescitore&20 mo&2\\
Strumenti da Calzolaio&5 mo&2&Strumenti da Cartografo&15 mo&2\\
\rowcolor{gray!20}Strumenti da Conciatore&5 mo&2&Strumenti da Costruttore&10 mo&2\\
Strumenti da Fabbro&20 mo&3&Strumenti da Falegname&8 mo&2\\
\rowcolor{gray!20}Strumenti da Gioielliere&25 mo&1&Strumenti da Intagliatore&1 mo&2\\
Strumenti da Inventore&50 mo&2&Strumenti da Pittore&10 mo&1\\
\rowcolor{gray!20}Strumenti da Soffiatore&30 mo&2&Strumenti da Tessitore&1 mo&2\\
Strumenti da Vasaio&10 mo&2&Utensili da Cuoco&1 mo&2\\
\rowcolor{gray!20}Strumenti da Navigatore&25 mo&2&Lira&30 mo&1\\
Cornamusa&30 mo&1&Corno&3 mo&L\\
\rowcolor{gray!20}Dulcimer&25 mo&2&Flauto&2 mo&L\\
Flauto di Pan&12 mo&L&Ciaramella &1 mo &2\\
\rowcolor{gray!20}Liuto&35 mo&1&Tamburo&6 mo&1\\
Viola&30 mo&1&Trucchi per il Camuffamento&25 mo&1\\
\end{tabularx}

\begin{multicols}{2}

\subsubsection{Cavalcature e Veicoli}

Una buona cavalcatura può consentire a un personaggio di attraversare rapidamente un territorio selvaggio, ma il suo scopo primario è trasportare l'equipaggiamento che altrimenti rallenterebbe il suo padrone.

La tabella \emph{Cavalcature e Altri Animali} indica i costi degli animali da trasporto, per le indicazioni sul movimento e la capacità di trasporto di ogni animale vedi indicazioni in capitolo \hyperlink{tabella-cavalcature-e-veicoli}{Tabella Cavalcature e Veicoli} (pag. \pageref{tabella-cavalcature-e-veicoli}).

Nei mondi di fantasy esistono altre cavalcature oltre a quelle elencate in questa sezione, ma si tratta di cavalcature rare che normalmente non sono disponibili per l'acquisto, come certe cavalcature volanti (pegasi, grifoni, ippogrifi e altri animali simili) o perfino alcune cavalcature acquatiche (come per esempio i cavallucci marini giganti).

Per entrare in possesso di una cavalcatura del genere spesso è necessario rubare un uovo e crescere la creatura di persona, stipulare un patto con una potente entità o negoziare con la cavalcatura stessa.

\textbf{Bardatura}. Una bardatura è un'armatura concepita per proteggere la testa, il collo, il petto e il corpo di un animale. Ogni tipo di armatura elencato nella tabella Armature di questo capitolo può essere acquistata come bardatura. Il costo è pari al quadruplo dell'armatura equivalente fabbricata per gli umanoidi, mentre il peso è pari al doppio, in caso di animali di taglia grande. Vedi anche \hyperlink{ArmaturedaCavallo}{Bardature da Cavalcatura}, pag. \pageref{ArmaturedaCavallo}.

\textbf{Sella}\index{Sella}\hypertarget{Sella}{}. Un cavalcatore può agganciarsi a una sella militare per rimanere al suo posto su una cavalcatura attiva, nel corso di una battaglia. Una sella militare conferisce +1d6 alle prove che il personaggio effettua per rimanere in sella. Una sella da carico è una sella solo adatta a permettere il trasporto di bisacce alla cavalcatura. E' necessaria una sella esotica per cavalcare una creatura acquatica o volante.

\begin{center}
	\includegraphics[width=0.8\linewidth]{immagini/bardatura.png}

	\emph{Bardatura completa}
\end{center}

\textbf{Imbarcazioni a Remi}\hypertarget{Barca a remi}{}\label{Barca a remi}\index{Barca a remi}. I barconi e le barche a remi sono solitamente usati sui laghi e sui fiumi. Se un'imbarcazione segue la corrente, si aggiunge la velocità della corrente (solitamente 4,5 km all'ora) alla sua velocità. In genere non è possibile remare controcorrente se la corrente ha un'intensità rilevante, ma è possibile far risalire un corso d'acqua a queste imbarcazioni portandole a riva e facendole trainare da una o più bestie da soma. Una barca a remi pesa 50 kg (Ingombro 15) qualora gli avventurieri debbano trasportarla via terra.

\medskip

\textbf{Tabella: Cavalcature e Altri Animali}\index[Tabelle]{Tabella Cavalcature e Altri Animali}\label{costicavalcature}\index{Cavalcature}

\noindent\begin{tabularx}{\linewidth}{Xl}
\toprule
\rowcolor{gray!20}\textbf{Cavalcatura}&\textbf{Costo}\\
\toprule
Cammello&50 mo\\
\rowcolor{gray!20}Elefante&500 mo\\
Mastino&25 mo\\
\rowcolor{gray!20}Saurovallo da Galoppo&75 mo\\
Saurovallo da Guerra&400 mo\\
\rowcolor{gray!20}Saurovallo da Tiro&50 mo\\
Saurovallo nano&30 mo
\end{tabularx}

\medskip

\textbf{Finimenti e Veicoli da Tiro}\index{Veicoli}\label{Veicoli}\hypertarget{Veicoli}{}

\noindent\begin{tabularx}{\linewidth}{Xll}
\toprule
\rowcolor{gray!20}\textbf{Oggetto}&\textbf{Costo}&\textbf{Peso}\\
\toprule
Bardatura&x4&x2\\
\rowcolor{gray!20}Biga&250 mo&50 kg\\
Bisacce&4 mo&4 kg\\
\rowcolor{gray!20}Carretto&15 mo&100 kg\\
Carro&35 mo&200 kg\\
\rowcolor{gray!20}Carrozza&300 mo&300 kg\\
Morso e Briglie&2 mo&0,5 kg\\
\rowcolor{gray!20}Nutrimento (al giorno)&5 mr&2 kg
\end{tabularx}

%\medskip

%\begin{center}
%	\includegraphics[height=0.8\linewidth]{immagini/sella2.png}
%\end{center}

\subsubsection{Servizi}\index{Servizi}

\textbf{Sella}\index{Sella}\index{Sella lista}

\noindent\begin{tabularx}{\linewidth}{Xll}
\toprule
\rowcolor{gray!20}\textbf{Oggetto}&\textbf{Costo}&\textbf{Peso}\\
\toprule
Da Carico&15 mo&7,5 kg\\
\rowcolor{gray!20}Da Galoppo&30 mo&12,5 kg\\
Esotica&60 mo&20 kg\\
\rowcolor{gray!20}Militare&50 mo&15 kg\\
Slitta&20 mo&150 kg\\
\rowcolor{gray!20}Stallaggio (al giorno)&1 ma&
\end{tabularx}

\medskip

\textbf{Imbarcazioni}\index{Imbarcazioni}\hypertarget{Imbarcazioni}{}

\noindent\begin{tabularx}{\linewidth}{Xll}
\toprule
\rowcolor{gray!20}\textbf{Oggetto}&\textbf{Costo}&\textbf{Velocità}\\
\toprule
Barca a Remi&50 mo&2,25 km orari\\
\rowcolor{gray!20}Barcone&3000 mo&1,5 km orari\\
Galea&30000 mo&6 km orari\\
\rowcolor{gray!20}Nave a Vela&10000 mo&3 km orari\\
Nave da Guerra&25000 mo&3,75 km orari\\
\rowcolor{gray!20}Nave Lunga&10000 mo&4,5 km orari
\end{tabularx}

\medskip

Gli avventurieri possono pagare i personaggi non giocanti affinché li aiutino o agiscano in loro vece nelle circostanze più disparate. La maggior parte di questi gregari è dotata di abilità pressoché ordinarie, mentre altri hanno padroneggiato un'arte o un mestiere e alcuni si sono specializzati in qualche abilità da avventuriero.

Altri gregari comuni includono i numerosi abitanti di un tipico paese o città che gli avventurieri possono ingaggiare per svolgere un compito specifico. Per esempio, un incantatore potrebbe pagare un falegname per farsi costruire un pregiato scrigno (e la sua replica in miniatura) da usare per un incantesimo.
Un guerriero potrebbe commissionare a un fabbro la forgiatura di una spada speciale.

\medskip

\textbf{Servizi}

\bigskip

\noindent\begin{tabularx}{\linewidth}{Xl}
	\toprule
\rowcolor{gray!20}\textbf{Servizio}&\textbf{Costo}\\
\toprule
Carrozza all'interno di una città&5 mr/1 km\\
\rowcolor{gray!20}Carrozza tra due paesi&1 ma/1 km\\
Gregario Abile&2 mo al giorno\\
\rowcolor{gray!20}Gregario Inesperto&5 ma al giorno\\
Messaggero&5 mr/1,5 km\\
\rowcolor{gray!20}Passaggio in nave&1 ma/1,5 km\\
Pedaggio di ingresso&5 mr/5 ma
\end{tabularx}

\subsubsection{Servizi Magici}\index{Servizi Magici}

\textbf{Livello Incantesimo x Livello Incantesimo ×100 mo}

Questo è il costo per avere un incantatore che manipola la magia, più eventuali componenti dell'incantesimo. Questo costo presuppone che si possa andare dall'incantatore e chiedergli di eseguire una certa magia a proprio piacimento (solitamente gli servono almeno 8 ore per prepararsi). Se si vuole portare da qualche parte l'incantatore per fargli usare la magia è necessario negoziare con lui, e la risposta di base è \emph{no}.

Se l'incantesimo ha conseguenze pericolose, l'incantatore deve ricevere delle prove certe che il personaggio ha la possibilità di pagare e che non mancherà di farlo nel caso queste conseguenze si verifichino (sempre che accetti di lanciare l'incantesimo richiesto, cosa nient'affatto sicura). Quando si tratta di incantesimi che trasportano il personaggio e l'incantatore lungo una distanza, è necessario pagare l'incantesimo due volte anche se il personaggio non desidera tornare indietro con l'incantatore.

Non tutti i villaggi e i paesi hanno un incantatore abbastanza capace a manipolare la magia. Come regola generale, è necessario spostarsi almeno in un piccolo paese per essere abbastanza sicuri di trovare un incantatore. In un piccolo paese si potrebbe trovare un incantatore in grado di lanciare incantesimi a livello 2, in un grande paese quelli a livello 3, una piccola città per quelli a livello 5, in una grande città per quelli di livello 6, in una metropoli per quelli di livello 8. Nemmeno in una metropoli si è certi di trovare un incantatore capaci di lanciare magie con livello 9 o più.

%\begin{center}
%\includegraphics[width=0.8\linewidth]{immagini/riempitivocavalieriapranzo.png}
%\end{center}

\subsubsection{Oggetti e Sostanze Speciali}\index{Sostanze Speciali}

\textbf{Antiemetico} 25 mo, questo liquido verde dolce e saporito crea un senso di calore e conforto. Lo sciroppo protegge lo stomaco e lo rende più resistente. Per 1 ora dopo averlo bevuto si ottiene Bonus +4 ai Tiri Salvezza per resistere agli effetti che rendono Nauseati o contro i veleni da Ingestione.

\textbf{Antibiotico} (fiala) 50 mo, bevendo una fiala di questo liquido bianco latte dal pessimo sapore si ottiene Bonus +4 ai Tiri Salvezza contro le Malattie, effettuati nell'ora successiva. Se già infetti, si possono effettuare due Tiro Salvezza per resistere alla Malattia in quella determinata giornata (senza il bonus +4) e tenere il risultato migliore. Monodose.

\textbf{Antitossina} (boccetta) 50 mo, se si beve l'antitossina, si ottiene Bonus +4 a tutti i Tiri Salvezza su Tempra contro Veleni per 1 ora. Monodose.

\textbf{Bastone del Fumo} 20 mo, questo bastone di legno trattato con procedimento alchemico crea istantaneamente un denso fumo opaco quando viene infiammato. Il fumo riempie un cubo con spigolo di 3 metri, tranne che il fumo viene dissipato in 1 round da un vento moderato o più intenso. Il bastone si consuma in 1 round e il fumo si dissolve poi naturalmente. Tutte le creature nell'area influenzata hanno copertura totale.

\textbf{Caffettone dell'Alchimista} 1 mo, molto amata dai giovani si tratta di una polvere cristallina bruna. Mischiata con l'acqua crea una bevanda amara che cura gli effetti della sbornia. Monodose. Lavoro DC 15

\textbf{Borsa dell'Impedimento} 50 mo, questa borsa di cuoio rotonda è piena di melassa, resina o altra sostanza appiccicosa. Quando si scaglia la borsa contro una creatura (come attacco di contatto a distanza con gittata 3 metri), la borsa si apre e la sostanza contenuta invischia ed \hyperlink{intralciato}{intralcia} la vittima per 1 minuto, diventando resistente ed elastica con l'esposizione all'aria. Sono necessarie 3 Azioni, anche in round diversi, per liberarsi.

La sostanza non agisce su creature di taglia Enorme o superiore. Una creatura volante non viene appiccicata al suolo, ma deve effettuare un Tiro Salvezza su Riflessi con DC 15 o perde la capacità di Volare (sempre che usi le ali per farlo), cadendo a terra. La borsa dell'impedimento non funziona sott'acqua.

\textbf{Fermasangue}\hypertarget{Fermasangue}{}\label{fermasangue} 25 mo, questa sostanza rosa e appiccicosa aiuta a curare le ferite. Utilizzarne una dose concede Bonus +4 alle prove di Pronto Soccorso. 4 Usi.

\textbf{Fiasco Alcalino} 15 mo, questo fiasco di liquidi caustici reagisce con gli acidi naturali delle melme. E' possibile lanciare un fiasco alcalino come arma a spargimento con gittata 3 metri. Contro le creature non melme un fiasco alcalino funziona come un'Ampolla d'acido. Contro le melme e altre creature acide il fiasco alcalino infligge i danni raddoppiati indicati da Ampolla d'Acido.

\textbf{Fumogeno} 25 mo, questa piccola sfera di argilla contiene due sostanze alchemiche separate da una sottile barriera. Quando si rompe la sfera, le sostanze si uniscono e riempiono un area di mischia con una nuvola di fumo nerastro e innocuo. Il fumogeno funziona come un bastone del fumo, ma il fumo rimane per 1 round prima di disperdersi. E' possibile lanciare un fumogeno come attacco di contatto con gittata 3 metri.

\textbf{Fuoco dell'Alchimista}\index{Fuoco dell'Alchimista}\hypertarget{Fuoco dell'Alchimista}{} 20 mo, si può lanciare un'ampolla di fuoco dell'alchimista come arma a spargimento. Si consideri l'attacco come un attacco di contatto a distanza con arma improvvisata, con gittata 3 metri.

Il colpo diretto provoca 1d6 danni da fuoco. Tutte le creature entro raggio di mischia dal punto in cui è caduta l'ampolla subiscono 1 danno da fuoco come effetto dello spargimento. Nel round successivo al colpo diretto la vittima subisce 1d6 danni da fuoco aggiuntivi. La vittima può sfruttare 1 Azione per tentare di spegnere le fiamme prima di subire questi danni aggiuntivi. Occorre superare un Tiro Salvezza su Riflessi con DC 15 per spegnere le fiamme. Usare 2 Azioni dà al personaggio bonus +2 al Tiro Salvezza. Tuffarsi in acqua o smorzare le fiamme con mezzi magici spegne automaticamente le fiamme.

\textbf{Gesso per Calchi}: 5 ma, questa polvere bianca e secca, mischiata con l'acqua, si addensa nel giro di un'ora per creare un materiale solido. Può essere utilizzato per creare un calco di un'orma o di un bassorilievo, riempire buchi o crepe nei muri o (se applicato ad una copertura di stoffa) per fermare un osso rotto. Il gesso indurito ha Durezza 1 e 5 Punti Ferita ogni 2.5 centimetri di spessore. Un vaso di 2 kg di gesso può coprire un raggio di mischia per la profondità di 2.5 centimetri, creare cinque ingessature per l'avambraccio o il polpaccio di una creatura di taglia Media o due ingessature complete per braccio o gamba. Monodose.

\textbf{Ghiaccio Liquido} (fiala) 40 mo, detto anche \emph{ghiaccio dell'alchimista}, questo fluido blu cristallino inizia ad evaporare appena tolto dal contenitore. Nei successivi 1d6 round è possibile utilizzarlo per congelare un liquido o coprire un oggetto con un sottile strato di ghiaccio. E' possibile anche lanciare il ghiaccio liquido come arma a spargimento. Un colpo diretto infligge 1d6 danni da freddo, mentre le creature entro raggio di mischia subiscono 1 danno da freddo per lo spargimento. La confezione contiene 3 dosi.

\textbf{Grasso Alchemico} 5 mo, ogni vaso di questa sostanza nerastra può coprire una creatura Media o due Piccole. Coprendosi di grasso alchemico si ottiene Bonus +4 alle prove di sfuggire alle prese. L'effetto dura 4 ore o finché si lava via il grasso.

\textbf{Individua Luce} 1 mo, questa piastra di metallo grande quanto una mano è coperta da una crema trasparente sensibile alla luce. Se esposta alla luce, la crema si scurisce e diviene opaca a seconda di quanta luce sia presente. La luce intensa la fa scurire in 1 round, quella normale in 3 round, quella fioca in 10 round.
La piastra viene venduta avvolta in un panno pesante per evitare esposizioni accidentali.

\textbf{Individua Luce avanzata} 50 mo, questa piastra di metallo simile alla piastra Individua luce è grande circa 50cm*50 cm. Se esposta alla luce imprime su di essa l'immagine dell'ambiente circostante entro 3 metri.

\textbf{Pietra del Tuono} 30 mo, si può scagliare questa pietra con un attacco a distanza con gittata 6 metri. Quando colpisce una superficie dura (o è colpita con forza), crea un rumore assordante che equivale a un attacco sonoro. Le creature presenti entro una distanza di 3 metri devono effettuare un Tiro Salvezza su Tempra con DC 15 o restano Assordate per 1 ora. Monouso.

\textbf{Polvere Lampo} 50 mo, questa polvere argentea brucia ed esplode quasi istantaneamente se esposta al fuoco, frizionandola o lanciandola con forza contro una superficie (1 Azione). Le creature entro raggio 3 metri sono Accecate per 1 round (Tempra DC 13 nega). La confezione contiene 3 dosi.

\textbf{Proteggilama} 40 mo, questa resina trasparente protegge un'arma dagli attacchi di Melme, Rugginofagi ed effetti che corrodono o sciolgono le armi, rendendola immune a tali attacchi per 24 ore. Un vasetto può coprire un'arma grande, 2 medie, 4 leggere o 50 munizioni. Applicarla richiede 2 Azioni. La confezione contiene 3 dosi.

\textbf{Solvente Universale} (fiala) 20 mo, questa gelatina viola ribollente divora gli adesivi. Ogni fiala può coprire un quadretto di 1 metro. Distrugge i normali adesivi (come la pece, la resina o la colla) in 1 round, ma richiede 1d4+1 round per dissolvere adesivi più potenti (borse dell'impedimento, ragnatele, ecc.). Non ha effetti sugli adesivi magici.

\textbf{Tizzone Ardente} 1 mo, la sostanza alchemica sulla punta di questo piccolo bastone di legno si infiamma quando viene sfregata contro una superficie ruvida. Creare una fiamma con un tizzone ardente è molto più rapido che crearla con Esca ed Acciarino (o lente d'ingrandimento) e esca. Accendere una torcia con un tizzone ardente costa 1 Azione (invece che 2 Azioni) e per accendere qualsiasi altro fuoco occorre almeno 3 Azioni. Monouso.

\subsubsection{Attrezzature Alchemiche}

\textbf{Cartina tornasole} 1 mo, questo pezzo di carta può aiutare a identificare i liquidi. Il suo colore cambia a seconda di tratti come acidità, salinità e magia. Consumare un foglio conferisce Bonus +2 alle prove di Lavoro (alchimia) o Arcano per identificare Pozioni o altri liquidi.

\textbf{Inchiostro Esplosivo} (fiala) 40 mo, questo inchiostro infuso alchemicamente aiuta ad assicurarsi che un messaggio segreto venga distrutto dopo essere stato letto. Se la luce colpisce l'inchiostro dopo che quest'ultimo si è asciugato, le sostanze chimiche lo fanno bruciare spontaneamente nel giro di 1 minuto
Questa combustione è di piccole dimensioni: non è abbastanza significativa da dar fuoco ad altro che alla carta. L'inchiostro usato su altri materiali come pietra o legno semplicemente svanisce, non lasciando alcuna traccia della scrittura.
Una fiala di questo inchiostro ne contiene abbastanza da scrivere 10 brevi messaggi di non più di 50 parole ciascuno.

%\textbf{Olio dei Liutai} 50 mo, quest'olio dorato profuma di legno antico. Quando lo si applica sulla cassa di uno strumento musicale di legno ne migliora la qualità del suono. Per 1 ora, chiunque suoni lo strumento ottiene Bonus +2 alla prova di Intrattenere appropriata. 3 usi.

\textbf{Pastiglia ai propoli} 50 mo, questa caramella ricoperta di miele è fatta di reagenti calmanti. Se mangiata, ha bisogno di 1 round per iniziare ad avere effetto, dopodiché conferisce Bonus +2 alle prove di Intrattenere per 1 ora. 3 caramelle.

\textbf{Sassolini luminosi} 50 mo, questi piccoli sassolini bianchi sono trattati alchemicamente in modo che emanino una luce tenue quando sfreganti gli uni contro gli altri. La luminescenza è fioca, appena sufficiente a illuminare la pietra. La durata è di 8 ore. 10 sassolini.

\textbf{Rivela Traccie} 30 mo, quando sparsa per terra, questa sottilissima polvere blu chiaro rivela le tracce di qualsiasi creatura o individuo che sia passato nell'area nelle ultime 48 ore.
La polvere fornisce anche Bonus +8 alle prove di Sopravvivenza per individuare le tracce. Una singola applicazione può coprire un'area di 3 metri. La polvere tracciante viene venduta in piccole borse di cuoio che contengono 10 applicazioni ciascuna.

\subsubsection{Rimedi Alchemici}\index{Rimedi Alchemici}

\label{rimedi-alchemici}

\textbf{Aiuto amaro} 25 mo, questo pacchetto è pieno di foglie secche amaricanti, dall'odore acre di alcool. Mentre si masticano le foglie si ignorano gli effetti dell'essere affaticati. Le foglie durano per 10 round, dopodiché ne rimane solo un mucchietto di poltiglia.
Quando l'effetto dell'aiuto amaro si esaurisce si aumenta di 1 grado il livello di affaticamento. Un pacchetto basta per 1 sola volta.

%\textbf{Crema Anti-veleno} 15 mo, questa crema alle erbe può essere applicato direttamente sulla pelle per prevenire gli effetti dei Veleni a contatto. Se una creatura tocca un veleno a contatto, ma applica su di sé il balsamo entro 1 round dal contatto, effettua il Tiro Salvezza due volte e tiene il risultato migliore. Monouso.

\textbf{Balsamo Coagulante} 5 ma, applicando questo balsamo alle erbe su una ferita diminuisce di 3 il danno da sanguinamento. Non è possibile usare più di due dosi al giorno sullo stesso paziente. La confezione è per 3 usi.

\textbf{Bombardino} 20 mo, questo liquido fortemente alcolico genera una piacevole sensazione di calore quando ingerito. Per l'ora successiva si ottiene Bonus +2 ai Tiri Salvezza contro Paura. Usare più dosi nell'arco delle stesse 24 ore rende Nauseati per 1 ora. La confezione è per 3 usi.

\begin{center}
\includegraphics[width=0.7\linewidth]{immagini/zaino.png}
\end{center}

\subsubsection{Lo Zaino Standard}\index{Zaino Standard}

Lo Zaino Standard\textregistered \space è una lista di oggetti che ho segnato nel tempo andando ad aggiungere ogni cosa che nel corso delle avventure mi era servito.
Prendetela come spunto per capire che oggetti avere dietro, non segnateveli tutti altrimenti il Narratore incomincerà seriamente a guardare le regole dell'Ingombro!

Questo il contenuto dello zaino dell'avventuriero: cintura, 3 candele, 6 torce, Esca ed Acciarino, 7 razioni secche, tenda da 2 persone, otre per l'acqua, materasso arrotolato, sacco a pelo, cerata, tenda, 18 metri corda, rete, specchio di metallo, piede di porco, bussola, 3 olio da lanterna, inchiostro, gesso, carboncino, uncino, vanga, amo da pesca, stracci, cavo di metallo 2m, fischietto, 6 fiale da pozione vuote, biglie di marmo, campanella in ottone, 1kg di farina in sacchetto, 3 zeppe, catena di metallo 12 metri, 2 manette, 8 chiodi da rocciatore, martello, carrucola, rampino, bandoliera.

%\begin{center}
%\includegraphics[width=0.8\linewidth]{immagini/mercante.png}
%\end{center}

\subsection{Spese e Stile di Vita}\index{Spese e Stile di Vita}

Quando non si calano nelle viscere della terra, non esplorano rovine in cerca di tesori perduti o non muovono guerra alle forze dell'oscurità incombente, anche gli avventurieri devono pensare ai bisogni più comuni. Anche in un modo fantastico, la gente deve soddisfare bisogni basilari come un vitto, alloggio e vestiario. Tutto questo ha un costo e certi stili di vita costano più di altri.

Per semplificare le spese giornaliere il giocatore può dichiarare di tenere un certo tenore di vita e sottrarre giornalmente le spese in un unicum, senza sottrarre le singole spese effettuate.

I prezzi elencati sono giornalieri, quindi chi desidera calcolare il suo costo di sostentamento per un periodo di trenta giorni dovrà moltiplicare il prezzo indicato per 30.

La scelta dello stile di vita può avere delle conseguenze. Un personaggio che mantiene uno stile di vita ricco può stringere più facilmente contatti con i ricchi e i potenti, ma corre il rischio di attirare qualche ladro. Analogamente, uno stile di vita povero può aiutarlo a evitare i criminali, ma difficilmente gli consentirà di stringere contatti importanti.

\medskip

\textbf{Spese dello Stile di Vita, al giorno}

\medskip

\noindent\begin{tabular}{ll|ll}
	\toprule
\rowcolor{gray!20}\textbf{Stile}&\textbf{Prezzo}&\textbf{Stile}&\textbf{Prezzo}\\
\toprule
Miserabile&-&Squallido&1 ma\\
\rowcolor{gray!20}Povero&2 ma&Modesto&1 mo\\
Agiato&2 mo&Ricco&4 mo\\
\rowcolor{gray!20}Aristocratico&10+ mo&Regale &50 mo+\\
\end{tabular}

%\begin{center}
%\includegraphics[width=0.7\linewidth]{immagini/mendicante.png}
%
%\emph{Mendicante - Francesco Londonio}
%\end{center}

%\end{multicols}

%\vfill

\begin{center}
\includegraphics[width=0.8\linewidth]{immagini/carrozza.png}
\end{center}

%\pagebreak

\subsubsection{Lavorare in città}\index{Lavorare in città}\index{Downtime}

Durante le pause tra un avventura ed un altra o perché deve passare un certo lasso di tempo perché una certa cosa accada, i personaggi possono cercare di mettere a frutto le loro Competenze per guadagnare qualche moneta.

I personaggi effettuano una prova al giorno della loro competenza professionale (es Artigianato oppure Erboristeria od Intrattenere...) in base al successo guadagneranno o meno.

La prova di professione eseguitela con 3d6+Saggezza+1/2 livello, se questa ottiene un valore superiore a 15 allora il personaggio ha ottenuto un compenso. Sottraete alla prova effettuata 15 ed elevate al quadrato questa differenza, saranno le monete d'argento guadagnate nel giorno ( ($(15-Prova)^2$) ).

\end{multicols}

\pagebreak

\subsection{Materiali Speciali}\index{Materiali Speciali}

\begin{enfasi}{
All'uopo il capitano De Medici ha fatto brunire tutte le armature, per sorprendere il nemico anche col buio. (Il mestiere delle armi, Ermanno Olmi, film 2001)}\end{enfasi}

\begin{multicols}{2}

Le armature e le armi si possono costruire con materiali che possiedono delle innate qualità speciali. Se si costruisce un'armatura o arma con più di un materiale speciale, si ricevono i benefici solo del materiale prevalente. Si può però costruire un'arma doppia con ogni testa fatta di un materiale speciale diverso.

\subsubsection{Acciaio Vivente}\index{Acciaio Vivente}\index[Tabelle]{Tabella Costo Acciaio Vivente}

\label{acciaio-vivente}

\noindent\begin{tabularx}{\linewidth}{Xl}
	\toprule
\rowcolor{gray!20}\textbf{Tipo di oggetto} & \textbf{Mod. al costo}\\
\toprule
per Munizione & +40 mo \\
\rowcolor{gray!20}Arma & +1000 mo\\
Armatura leggera & +3000 mo\\
\rowcolor{gray!20}Armatura media & +8000 mo\\
Armatura pesante & +12000 mo\\
\rowcolor{gray!20}Scudo & +600 mo\\
Altri oggetti & 3000 mo/kg
\end{tabularx}

\medskip
Un albero di acciaio vivente si caratterizza dal un legno particolarmente duro alla stregua dell'acciaio. L'origine di questi alberi rimane un mistero per quasi tutti. Un albero di acciaio vivente è un normale albero piantato da un Devoto di Efrem o Shayalia che è stato dal Devoto benedetto.

Le armature e gli scudi di acciaio vivente sono formalmente di legno ma hanno le medesime caratteristiche dell'adamantio. Questo particolare legno è il preferito da chi combatte e vive per la natura. Non è facile individuare un albero di acciaio vivente per un non esperto ed anche per questo è estremamente raro trovarlo grezzo, al più è possibile trovare armi o armature già fatte.

L'acciaio vivente ha 35 Punti Ferita per 2,5 cm di spessore e Durezza 15.

\subsubsection{Adamantio}\index{Adamantio}\index[Tabelle]{Tabella Costo Adamantio}

\label{adamantio}

\noindent\begin{tabularx}{\linewidth}{Xl}
	\toprule
\rowcolor{gray!20}\textbf{Tipo di oggetto} & \textbf{Mod. al costo}\\
\toprule
per Munizione & +60 mo\\
\rowcolor{gray!20}Arma & +1500 mo\\
Armatura leggera & +5000 mo\\
\rowcolor{gray!20}Armatura media & +10000 mo\\
Armatura pesante & +15000 mo\\
\rowcolor{gray!20}Scudo & +1000 mo\\
Altri oggetti & 5000 mo/kg
\end{tabularx}

\medskip
Questo metallo durissimo si trova solo nei meteoriti e contribuisce alla qualità di un'arma o di un'armatura.

Le armi e le munizioni in adamantio hanno Bonus di +1 ai Tiri per Colpire e la penalità date dall'armatura (Penalità Competenze e Prove di Magia) viene diminuita di 1 rispetto ad una normale armatura del suo stesso tipo. Gli oggetti senza parti metalliche non possono essere costruiti con l'adamantio. Una freccia può essere in adamantio, ma un bastone ferrato no.

\begin{center}
%	\includegraphics[width=0.8\linewidth]{immagini/mithral.png}
	\includegraphics[width=0.8\linewidth]{immagini/armature-praga_grayscale.png}

	\emph{Praga, Vicolo d'oro, esposizione armature e armi}
\end{center}

Armi e armature fatte normalmente d'acciaio e costruite con l'adamantio hanno un terzo dei Punti Ferita in più del normale. L'adamantio ha 40 Punti Ferita per 2,5 cm di spessore e Durezza 20.

\subsubsection{Argento Alchemico}\index{Argento Alchemico}\index[Tabelle]{Tabella Costo Armi Argentate}

\label{argento-alchemico}

\noindent\begin{tabularx}{\linewidth}{Xl}
	\toprule
\rowcolor{gray!20}\textbf{Tipo di Oggetto} & \textbf{Mod. al costo}\\
\toprule
per Munizione & +2 mo\\
\rowcolor{gray!20}Arma leggera & +20 mo\\
Arma media & +90 mo\\
\rowcolor{gray!20}Arma pesante & +180 mo\\
Scudo & +100 mo
\end{tabularx}

\medskip

Il processo di argentatura alchemica può essere applicato solo alle armi metalliche e non funziona sui metalli speciali come ad esempio l'adamantio, il ferro freddo ed il mithral.

Un complesso processo che coinvolge la metallurgia e l'alchimia può legare l'argento a un'arma fatta d'acciaio in modo che oltrepassi la Riduzione del Danno di creature come i Licantropi.

Un arma in argento alchemico mantiene la Durezza e Punti Ferita dell'arma originale.

\subsubsection{Ferro Freddo}\index{Ferro Freddo}

\label{ferro-freddo}

Questo ferro viene estratto nelle profondità del sottosuolo ed è noto per la sua efficacia contro demoni e folletti. Viene forgiato ad una temperatura inferiore per conservare le sue delicate proprietà. Costruire armi fatte di ferro freddo costa il doppio rispetto alle loro normali controparti. Inoltre qualsiasi potenziamento magico costa 2000 mo addizionali. Questo aumento viene applicato la prima volta che l'oggetto viene potenziato, non una volta per qualità aggiunta.

Gli oggetti senza parti di metallo non possono essere costruiti in ferro freddo. Una freccia potrebbe essere fatta di ferro freddo ma un randello no (tranne se tutto di metallo). Un'arma doppia che è fatta solo per metà di ferro freddo aumenta il suo costo del 50\%.

Il ferro freddo ha 30 Punti Ferita per 2,5 cm di spessore e Durezza 10.

\subsubsection{Mithral}\index{Mithral}\index[Tabelle]{Tabella Costo Armi Mithral}

\label{mithral}

\noindent\begin{tabularx}{\linewidth}{Xl}
	\toprule
\rowcolor{gray!20}\textbf{Tipo di Oggetto} & \textbf{Mod. al costo}\\
\toprule
Armatura leggera & +1000 mo\\
\rowcolor{gray!20}Armatura media & +4000 mo\\
Armatura pesante & +9000 mo\\
\rowcolor{gray!20}Scudo & +1000 mo\\
Altri oggetti & +1000 mo/kg
\end{tabularx}

\medskip

Il mithral è un metallo molto raro, luccicante, simile all'argento, più leggero del ferro ma altrettanto duro. Quando viene lavorato come l'acciaio, diventa un meraviglioso materiale con cui creare armature e occasionalmente anche per altri oggetti. La maggior parte delle armature in mithral è più leggera di una categoria ed è più agevole per il movimento e le altre limitazioni. Le armature pesanti sono trattate come armature medie, e le armature medie sono trattate come leggere, ma le armature leggere restano leggere.

Questa diminuzione non si applica alla competenza necessaria per indossare l'armatura in questione, un armatura pesante in mithril richiede comunque Competenza Armi 3. Occorre essere competenti nel tipo di armatura appropriato, altrimenti si incorre nelle relative penalità come di norma.

Le Prova di Magia per armature e scudi in mithral diminuiscono di 2 (rimanendo comunque necessaria la prova) e la penalità alla prove competenza diminuiscono di 2 (fino a un minimo di 0), le penalità al movimento diminuiscono di 1 metro.

Il mithral ha 30 Punti Ferita per ogni 2,5 cm di spessore e Durezza 15.

\subsubsection{Pelle di Drago}\index{Pelle di Drago}

\label{pelle-di-drago}

I fabbricanti di armature possono lavorare le pelli dei draghi per produrre armature o scudi.
Un drago fornisce scaglie sufficiente per una singola armatura completa, equivalente ad una armatura pesante, per una creatura di una taglia più piccola del drago, oppure due armature medie per una creatura di due taglie più piccole o 4 armature leggere per creature di 3 taglie più piccole.

Un armatura o scudo in pelle di Drago non si compra, è sempre necessario portare la materia prima, possibilmente non viva, all'artigiano che si preoccuperà di costruire l'armatura.

In ogni caso, c'è sempre pelle sufficiente per produrre uno scudo leggero o pesante in aggiunta all'armatura, purché il drago sia almeno Grande.
Se la pelle di drago proviene da un Drago che ha immunità ad un tipo di energia, anche l'armatura è immune a quel tipo di energia, sebbene non conferisca alcuna protezione a chi la indossa. Se allo scudo o all'armatura viene conferita in seguito la capacità di proteggere chi la indossa da un tipo di energia specifico, il costo di questo potenziamento viene ridotto del 25\%.

Un Armatura in Pelle di Drago riduce la penalità alla Prova di Magia di 4 quando lanci un  incantesimo, la penalità alla Competenze diminuiscono di 1 (fino a un minimo di 0), le penalità al movimento diminuiscono di 1 metro.

Le armature di pelle di drago costano 10 volte un'armatura di quel tipo, ma non richiedono più tempo per essere costruite. Un armatura di drago non è mai in vendita.

La pelle di drago ha 10 Punti Ferita per 2,5 cm di spessore e Durezza 10. Solitamente la pelle di drago è spessa da 1,25 a 2,5 cm.

\end{multicols}

\vfill

\begin{center}
\includegraphics[width=0.6\linewidth]{immagini/dragonhide.png}

\emph{Pelle di drago, dettaglio.}
\end{center}

\pagebreak

\section{Sfondare ed Entrare}\index{Sfondare}\index{Entrare}\hypertarget{sfondare}{}\label{sfondarecap}

\begin{enfasi}{
Nella vita di un uomo prima o poi arriva un giorno in cui, per andare dove deve andare, se non ci sono porte né finestre, gli tocca sfondare la parete. (Bernard Malamud)

\medskip

Il reato di furto sarà punito con il marchio a fuoco dei ladri, in pieno petto. In caso di reiterazione del reato saranno mozzate prima le orecchie e poi due dita delle mani. (Twoslad, Diritti e Doveri Cittadini)

}\end{enfasi}

\begin{multicols}{2}

\label{sfondare-ed-entrare}

Quando si tenta di spaccare un oggetto le scelte sono due: colpirlo con un'oggetto (arma?) o romperlo con la forza bruta.

\smallskip

\subsection{Le dimensioni contano...}

A seconda delle dimensioni dell'oggetto questo può essere più o meno facile da colpire.

\medskip

\textbf{Tabella: Taglia e Difesa degli Oggetti - Colpire un Oggetto}\index[Tabelle]{Tabella Taglia e Difesa degli Oggetti - Colpire un Oggetto}

\medskip

\noindent\begin{tabularx}{\linewidth}{Xll}
	\toprule
\rowcolor{gray!20}\textbf{Taglia} & \textbf{Mod. Difesa} & \textbf{Dimensioni}\\
\toprule
Colossale & -8 &18m+\\
\rowcolor{gray!20}Mastodontica & -6 &9-18m\\
Enorme & -4 &4-9m\\
\rowcolor{gray!20}Grande & -2 &2.4-4m\\
Media & +0 &1.2-2.4m\\
\rowcolor{gray!20}Piccola & +2 &60-120cm\\
Minuscola & +4 &30-60cm\\
\rowcolor{gray!20}Minuta & +6 &15-30cm\\
Piccolissima & +8 &5-20cm
\end{tabularx}

\medskip

\textbf{Modificatore Difesa}

Gli oggetti sono più facili da colpire delle creature poiché di solito non si muovono ma molti sono abbastanza resistenti da ignorare del danno ad ogni colpo. La Difesa di un oggetto è pari a 10 + il suo modificatore di Taglia (vedi Tabella: Colpire un Oggetto) + il suo modificatore di Destrezza (caso mai ne avesse uno).

Se si usano 3 Azioni per prendere la mira si colpisce automaticamente con un'arma da mischia.\index{Colpire un oggetto}

\subsection{Spaccare}

Nella Tabella seguente sono indicati materiali ed oggetti con relativa Durezza, Punti Ferita e DC per rompere o sfondare\index{Sfondare}\index{Rompere}

Quando si tenta di rompere o sfondare qualcosa con forza bruta piuttosto che infliggendo danni bisogna effettuare un Tiro Salvezza Tempra con Forza per capire se ci si riesce. Poiché la Durezza non influisce sulla DC per rompere l'oggetto, questo valore dipende più dal modo in cui è costruito l'oggetto che non dal materiale. La DC indicata è per oggetti comuni, un vetro spesso 20 cm non avrà DC 6 per rompersi.

Vedi anche \hyperlink{tabellaporte}{Tabella: Porte}, pag. \pageref{tabellaporte}

\end{multicols}

\textbf{Tabella: Durezza e Punti Ferita oggetti}\index[Tabelle]{Tabella Durezza e Punti Ferita oggetti}\label{durezzaoggetti}\index{Rombere oggetti}\index{Spaccare cose}\index{Rompere bauli}

\noindent\begin{tabularx}{\linewidth}{Xllll}
\toprule
\rowcolor{gray!20}\textbf{Materiale} &\textbf{Dur.}&\textbf{PF} &\textbf{DC} & \textbf{Oggetti di Esempio}\\
\toprule
Carta, Vetro, Stoffa	&0	& 1	&	3&	Fogli di carta, vetro di finestra, stoffa leggera\\
\rowcolor{gray!20}Vetro					&1	& 4 &	6	&Blocco di vetro, tavolo di vetro, vaso pesante\\
Stoffa pesante 			&1	& 4 & 	12	&Armatura di stoffa, giacca pesante, sacco, tenda\\
\rowcolor{gray!20}Legno sottile			&3	& 12 &	14	&Sedia\\
Legno					&5	& 20 &	18	&Baule, tavolo\\
\rowcolor{gray!20}Corda, Cuoio			&2 	& 4	&	19	&Corda di canapa\\
Pietra sottile			&4	& 16 &	20	&Ardesia, mattonelle, baule leggero\\
\rowcolor{gray!20}Struttura in legno		&10	& 40 &	20	&Muro di legno, forziere\\
Cuoio rinforzato		&4	& 16 &	22	&Armatura di cuoio, sella, corda di canapa grossa\\
\rowcolor{gray!20}Acciaio o ferro sottile &5	& 20 &	23	&Corda di seta, scudo d'acciaio, spada corta\\
Acciaio o ferro			&9	& 36 &	26	&Catena, Armatura di metallo, forziere rinforzato\\
\rowcolor{gray!20}Struttura in pietra		&14	& 56 &	35	&Muro di pietra, spada lunga\\
Pietra					&7	& 28 &	35	&Pietra per lastricato, statua\\
\rowcolor{gray!20}Struttura in acciaio o ferro	&18	&90	&45	&Muro di piastre di ferro, spadone a due mani

\end{tabularx}

\begin{multicols}{2}

\subsection{Danneggiare gli oggetti}\index{Danneggiare oggetti}\index{Durezza}

\textbf{Durezza}: rappresenta la resistenza dell'oggetto a essere scalfitto o danneggiato. Quando si calcola il danno ad un oggetto va \textbf{sottratta la Durezza} del materiale prima di applicare il danno.

\textbf{Attacchi di Energia}: quasi tutti gli oggetti hanno Resistenza al danno verso gli attacchi di energia (fuoco, elettricità..), dividete per 2 i danni prima di applicare la Durezza mentre altri oggetti potrebbero essere particolarmente vulnerabili.

Per esempio, il fuoco potrebbe infliggere il doppio del danno a pergamene, stoffa e altri oggetti che bruciano facilmente. Oggetti e creature in cristallo o ceramica potrebbero subire danno doppio (vulnerabilità) contro un attacco sonoro.

Energia Negativa o Positiva non danneggiano gli oggetti, solo le creature viventi o meno.

\textbf{Armi Inefficaci}: Certe armi semplicemente non possono infliggere danni a certi oggetti. Per esempio, un'arma contundente non è in grado di tagliare una corda.
Allo stesso modo è decisamente difficile abbattere una porta o un muro di pietra con la maggior parte delle armi da mischia a meno che non siano specificamente ideate per farlo come picconi e martelli.

\textbf{Immunità}\index{Oggetti e Danno Critici}: Gli oggetti inanimati sono immuni ai Danni Non Letali e ai Danno Critici (ma non all'esplosione del danno). Anche gli oggetti animati, se non considerati come delle creature, hanno queste immunità.\index{Immunità ai critici degli oggetti}

\textbf{Oggetti Danneggiati}: Un oggetto danneggiato rimane pienamente funzionale fino a quando i Punti Ferita non arrivano a 0, e a quel punto è considerato distrutto. Gli oggetti danneggiati (ma non quelli distrutti) possono essere riparati da una Professione Artigiano e alcuni Incantesimi.

\textbf{Tiro Salvezza}: Gli oggetti non magici incustoditi non effettuano mai Tiro Salvezza. Si considera che abbiano fallito i loro Tiro Salvezza se disponibile.

Un oggetto custodito da un personaggio (che lo tenga in mano, lo tocchi o lo indossi) riesce nel Tiro Salvezza se il personaggio riesce nello stesso.\index{Tiro Salvezzo degli oggetti}

\textbf{Gli Oggetti Magici hanno sempre Tiro Salvezza}. Il bonus ai Tiri Salvezza su Tempra, Riflessi o Volontà di un Oggetti Magico è pari a 2 + livello x2 dell'incantesimo più potente che ospitano. Se l'oggetto non ha un incantesimo si considera un bonus di +4 per ogni +1 di bonus posseduto. Gli Oggetti Magici custoditi (indossati) effettuano il Tiro Salvezza solo se il loro possessore fallisce il proprio. Se un effetto influenza specificatamente l'oggetto magico e non chi lo indossa allora è solo l'oggetto magico ad effettuare il Tiro Salvezza.

Un \textbf{oggetto incantato}\index{Danneggiare oggetti incantati} come un arma o armatura ha Durezza, Punti Ferita e DC per romperlo aumentati della metà rispetto all'equivalente non magico.

\textbf{Oggetti animati}: Gli oggetti animati contano come creature per determinarne la Difesa e Punti Ferita (non sono considerati oggetti inanimati).

\medskip

\begin{center}
\includegraphics[width=0.7\linewidth]{immagini/portarinforzata2.png}

\emph{Porta rinforzata}
\end{center}

\subsection{Le Dimensioni contano per Sfondare...}\index{Sfondare}\index{Le Dimensioni contano per Sfondare...}

\label{sfondare}

Creature di Taglia superiore o inferiore a quella Media hanno bonus o penalità dati dalla taglia sulla prova di Forza (TS Tempra con Forza) per sfondare una porta:

\medskip

\textbf{Tabella: Modificatori prova di Forza in base alla propria taglia}\index[Tabelle]{Tabella Modificatori prova di Forza per Sfondare porta}

\medskip

\noindent\begin{tabular}{ll|ll}
	\toprule
\rowcolor{gray!20}\textbf{Taglia} & \textbf{Mod.}&\textbf{Taglia} & \textbf{Mod.}\\
\toprule
Piccolissima & -16& Grande & +4\\
\rowcolor{gray!20}Minuta & -12 &Enorme & +8\\
Minuscola & -8& Mastodontica & +12\\
\rowcolor{gray!20}Piccola & -4 & Colossale & +16\\
Media & +0&&
\end{tabular}

\medskip

Un \textbf{piede di porco}\index{Piede di porco} o un \textbf{ariete portatile}\index{Ariete} aumentano la probabilità del personaggio di sfondare una porta di +1d6.

\end{multicols}

\vfill

\begin{center}
\includegraphics[width=0.60\linewidth]{immagini/kitladro.png}

\emph{Kit da furfante}
\end{center}

\pagebreak

\section{Ambiente}\index{Ambiente}

\label{ambiente}
\begin{enfasi}{
La natura non è crudele, è solo spietatamente indifferente. Questa è una delle più dure lezioni che un essere umano debba imparare. (Richard Dawkins)

\medskip

L'antidoto principale contro un cattivo ambiente consiste, naturalmente, nel sostituirlo con uno buono. (Robert Baden-Powell)}\end{enfasi}

\begin{multicols}{2}

Dai deserti senza vita ai dungeon zeppi di trappole, l'ambiente aiuta a definire il mondo, renderlo vivo, dinamico e ricco. Consente di creare un'esperienza di gioco emozionante e coinvolgente.

\subsection{Regole Ambientali}

\label{regole-ambientali}

\subsubsection{Visione e Luce}\index{Visione}\index{Luce}\hypertarget{visioneeluce}{}\label{visioneeluce}

\begin{narratore}[Luce]
Il diverso funzionamento delle fonti di luce vuole rendere più cupo, oscuro e difficile l'esplorazione, specialmente di caverne e zone prive di fonti luminose. Basta gruppi che castano Luce ogni minuto o gridano \emph{Scurovisione!}. L'oscurità aiuta l'immaginazione ed alza il livello di tensione. Enfatizzate il crepitare della fiamma della torcia, l'ondeggiare e a volte quasi spegnersi per le correnti improvvise. Rendete misterioso ciò che c'è intorno ai personaggi!.
\end{narratore}

\label{sec:visione-e-luce}

In un ambiente naturale l'illuminazione può assumere diverse gradazioni e queste gradazioni aiutano a capire fino a che distanza una creatura può vedere.

Le gradazione di luce possono essere:
\begin{itemize}[leftmargin=*] \setlength{\itemsep}{0pt}
\item
\textbf{Oscurità}': buio pesto, può essere naturale o magico
\item
\textbf{Luce Fioca / Scarsa / Penombra / Oscurata leggermente}: poca illuminazione, permette di riconoscere le sagome\index{Oscurata leggermente}\index{Luce Fioca}
\item
\textbf{Luce}: luce intensa, luce brillante, coprente, assolata
\end{itemize}

Saranno le fonti di luce, o la loro assenza, a stabilire quanta illuminazione c'è e fino a che distanza. La Tabella Fonti di Luce indica per le più comuni fonti di luce il raggio illuminato a pieno, quello illuminato in maniera inferiore (Luce Fioca) e la durata.

Molti incantesimi ed oggetti usano come durata il \emph{tempo di gioco reale}, ovvero non si contano i round o turni per stabilire la durata bensì si segna sulla scheda l'orario di accensione della torcia, lanterna, incantesimo. Un altro metodo può essere quello di impostare un timer nello smartphone. In questa maniera risulterà più facile la gestione e maggiore l'attenzione alle risorse consumabili.

\medskip

\textbf{Tabella: Fonti di luce}\index[Tabelle]{Tabella delle fonti di luce}\label{fontidiluce}

\medskip

\index{Luce Fioca}

\noindent\begin{tabularx}{\linewidth}{l|cc|c}
	\toprule
\rowcolor{gray!20}\textbf{Fonte di Luce} &\multicolumn{2}{c}{\textbf{Raggio (metri)}}& \textbf{Durata} \\
& \textbf{Luce} & \textbf{Fioca} &\\
\toprule
\hyperlink{Candela}{Candela} & - & 1 & 1 ora\\
\rowcolor{gray!20}\hyperlink{Torcia}{Torcia} & 3 & 6 & 1 ora\\
\hyperlink{Lanterna}{Lanterna} & 6 & 12 & 3 ore \\
\multicolumn{4}{c}{\textbf{Incantesimi}}\\
\rowcolor{gray!20}\hyperlink{Lacrima di Ljust}{Lacrima di Ljust} & 1 & - & 10 round\\
\hyperlink{Luce}{Luce}& 3 & 6 &30 min. \\
\rowcolor{gray!20}\hyperlink{Luce Diurna}{Luce Diurna} & 6 & 12 & 1 ora
\end{tabularx}

\smallskip

La durata indicata è espressa, quando in minuti o ore, come durata di tempo reale di gioco.\index{Durata Luce}

\medskip

\begin{center}
\includegraphics[width=0.8\linewidth]{immagini/oscurita.png}

\emph{Henry Justice Ford}
\end{center}

\begin{giocatore}[Visione Crepuscolare e Scurovisione ?]{
L'oscurità che permea la Terra non è solo mancanza di luce ma è viva e pulsante. Anche se molte delle nuove razze dovrebbero vedere al buio così non è. Tazher e Calicante hanno reso impenetrabile l'oscurità per chiunque.
}\end{giocatore}

\medskip

La \textbf{Luce fioca}\index{Luce Fioca} è la luce oltre una fonte di luce. E' il passare in un corridoio di 3 metri se è illuminato solo da leggere candele, è una notte di luna piena, è una zona oscurata leggermente.
In linea di massima una fonte di luce crea luce fioca in un raggio doppio rispetto al raggio di luce normale. \textbf{Una creatura in Luce Fioca ha un -2 alle prove di Consapevolezza ed un -1 ai Tiri per Colpire}.\index{Luce Fioca penalita' TC}

\medskip

\textbf{Oscurità}\index{Oscurità}: è il buio più completo senza alcuna fonte di luce. Per creature con visione normale l'oscurità è ciò che c'è oltre la Luce fioca.
Il \textbf{personaggio cieco}\index{Cieco} o che combatte nell'oscurità (e non può vedere nell'oscurità) ha -1d6 alla Consapevolezza e tutti gli avversari sono \hyperlink{invisibilita}{invisibili} (vedi pag. \pageref{invisibilita}).

\medskip

La \textbf{Luce}\index{Luce} è la luce all'aperto sotto il sole, ma anche se si tiene una torcia in mano o in un corridoio illuminato da lanterne. Se le fonti di luci non si susseguono si creano zone di luce fioca se non oscurità.

\subsubsection{Tipi di Visione e Illuminazione}

\begin{itemize}[leftmargin=*] \setlength{\itemsep}{0pt}
\item
Una creatura con \textbf{Visione Normale} \index{Visione Normale}vede fino alla distanza, come raggio circolare intorno alla fonte di luce, indicato in Luce. Oltre è Luce Fioca e oltre ancora Oscurità.

\item
Una creatura con \textbf{Visione Crepuscolare} \index{Visione Crepuscolare}vede senza problemi fino alla distanza, come raggio circolare intorno alla fonte di luce, indicato in Luce fioca, o indicato dalla razza se minore, oltre è oscurità.

\item
Una creatura con \textbf{Scurovisione} \index{Scurovisione} vede, entro il raggio indicato dalla sua Scurovisione, in condizioni di luce normale e luce fioca senza problemi, mentre nell'oscurità ha -2 Consapevolezza ed a Sopravvivenza per cercare trappole. La Scurovisione è una visione in bianco e nero.
\end{itemize}

\begin{giocatore}[Nota sulle fonti di luce]
Vi sarete accorti o lo farete presto, che le fonti di luce magiche funzionano in maniera diversa, molto spesso durano molto di meno o generano poca luce. Questo è dovuto al volere di un Patrono e come tale solo un Patrono può annullarne gli effetti (o il Narratore!).
\end{giocatore}

\subsubsection{Buio}\index{Buio}

\label{buio}

Le torce e le lanterne possono essere spente all'improvviso da una folata di vento, le fonti di luce magiche possono essere dissolte o contrastate ed alcune trappole magiche possono creare aree di buio impenetrabile.

In certi casi alcuni personaggi o mostri potrebbero essere in grado di vedere mentre gli altri sono Accecati. Ai fini delle regole che seguono, una creatura Accecata è semplicemente una creatura che non è in grado di vedere ciò che la circonda.

\subsubsection{Accecato}\index{Accecato}\index{Invisibile}

\label{accecato}

Le creature Accecate perdono la loro capacità di infliggere danni extra causati ad esempio dall'Abilità di pugnalare alle spalle (ma non da esplosione del danno o critico al colpire).

Le creature accecate considerano il terreno come difficile \index{Muoversi al buio}. Devono effettuare una prova di Acrobatica con DC 12 per Azione di Movimento per muoversi a velocità normale. Se la prova fallisce cadono a terra prone. Le creature accecate non possono caricare.

Una creatura accecata, o che combatte contro una creatura invisibile,\index{Invisibile} può effettuare una prova di Consapevolezza a difficoltà 20 (oppure 10+ prova di Furtività dell'avversario se questo non vuole farsi trovare) per individuare la creatura purché questa sia entro 6 metri dal personaggio.

Una creatura Accecata \index{Accecata}subisce penalità di -2 alle Prove di Competenza basate su Forza e Destrezza e fallisce automaticamente qualsiasi prova di Consapevolezza dipenda dalla vista.

Inoltre, una creatura accecata non può usare incantesimi che prevedano l'uso dello sguardo ed è immune agli incantesimi che prevedono lo sguardo.

Per i modificatori in dettaglio vedi in \hyperlink{invisibilita}{Invisibilità} (pag \pageref{invisibilita}).

\begin{center}
\includegraphics[width=0.75\linewidth]{immagini/oggetticadenti.png}

\emph{Henry Justice Ford}
\end{center}

\subsubsection{Cadute}\index{Cadute}\index{Cadere}\hypertarget{cadute}{}\label{cadute}

Le creature che cadono si fanno male. Dividi l'altezza di caduta (in metri) per 3, arrotonda per difetto, il numero che risulta sono i d6 di danno subiti. Es 16 metri di caduta sono 16/3=5d6 di danno. Per praticità si suggerisce di applicare 1 danno ogni metro di caduta.\index{Danno da caduta}

Le creature che subiscono danni da una caduta atterrano in posizione \textbf{prona}.

Una prova, usando una Reazione, di Acrobatica riuscita con DC 15 permette al personaggio di ridurre il danno da caduta di 3. Per ogni punto oltre il 15 nella prova riduce di un ulteriore 1 il danno. La caduta deve essete entro 6 metri.

Quando Acrobatica raggiunge punteggio 6 la riduzione si applica a cadute entro 9 metri, a punteggio 9 a cadute entro 12 metri.

Cadute su superfici morbide (terreno morbido, fango ecc.) riducono di 3 i danni.

Un personaggio termina una Azione di Movimento con una caduta, ma solo se non si è fatto danni può proseguire con la stessa Azione, altrimenti prima deve alzarsi da prono.

In un round di caduta libera si precipita di 150 metri (50d6 oppure 150 di danno), al termine del primo segmento cade a 20 metri, poi a 80m poi a 150m. Un personaggio non può lanciare incantesimi mentre cade, a meno che la caduta non sia superiore o pari a 100 metri. Si è Distratti mentre si prova a lanciare un incantesimo mentre si cade.\index{Lanciare incantesimi mentre si cade}

\medskip

\noindent \textbf{Cadere in Acqua}\index{Cadere in Acqua}

Le cadute in acqua sono gestite in modo leggermente diverso. Fino a quando l'acqua ha una profondità di almeno di 3 metri ed il tuffo è da una altezza entro 12 metri non si subiscono danni.

Per determinare il danno da caduta in acqua sottraete all'altezza di caduta 12 metri, aggiungete 1d6 di danno per ogni 3 metri rimanenti ($((H-12)/3)*1d6)$).

I personaggi che si tuffano volontariamente in acqua non subiscono danni se superano una prova di Nuotare con DC 15 e se l'acqua è profonda almeno 6 metri. La DC della prova aumenta di 1 ogni metro oltre i 12 metri di altezza.

\subsubsection{Effetti dell'Acido}\index{Acido}

\label{effetti-dellacido}

Gli acidi corrosivi infliggono 1d6 danni per round di esposizione, tranne nel caso di totale immersione (come in una vasca d'acido) che infligge 10d6 danni per round. Un attacco con l'acido, come quello di una boccetta lanciata o la saliva/soffio di un mostro, deve essere considerato come un round di esposizione.

I vapori prodotti dalla maggior parte degli acidi sono equivalenti a veleni inalati. Coloro che si avvicinano ad un grosso ammasso di acido devono effettuare un Tiro Salvezza su Tempra con DC 13 o subiranno 1 danno temporaneo alla Costituzione per round di esposizione. Questo veleno non ha frequenza, pertanto una creatura è salva se si allontana dall'acido.

Le creature immuni alle proprietà caustiche dell'acido potrebbero comunque annegare se vi vengono totalmente immerse (vedi Annegamento).

\subsubsection{Effetti del Fumo}\index{Fumo}

\label{effetti-del-fumo}

Un personaggio costretto a respirare del fumo denso deve superare un Tiro Salvezza su Tempra ogni round (DC 15, +1 per ogni prova precedente) oppure passa il round a tossire e soffocare. Un personaggio che continua a soffocare per 2 round consecutivi subisce 1d6 Danni Non Letali per ulteriore round di esposizione. Il fumo oscura la vista, fornendo Copertura leggera (+2 Difesa) ai personaggi che si trovano al suo interno.

\subsubsection{Fame e Sete}\index{Fame}\index{Sete}

\label{fame-e-sete}

I personaggi potrebbero trovarsi senz'acqua o cibo e privo dei mezzi per procurarsene. Nei climi normali, i personaggi di taglia Media hanno bisogno di almeno 2 litri di liquidi e 0.5 kg di cibo decente al giorno per evitare la fame, i personaggi di taglia Piccola necessitano della metà. Nei climi molto caldi i personaggi possono aver bisogno di due o tre volte quella quantità d'acqua per evitare la disidratazione.

Ogni giorno senza cibo è necessario effettuare un Tiro Salvezza su Tempra a difficoltà 11 +1 per giorno senza cibo, se non si ha da bere la difficoltà aumenta di +3.

Se si fallisce il Tiro Salvezza si subiscono 1d4 di danno e si diventa sempre più Affaticati. Le penalità date dall'affaticamento rimangono finché non si mangia e beve abbastanza.

\subsubsection{Oggetti cadenti}\index{Oggetti Cadenti}\index{Caduta oggetti}

\label{oggetti-cadenti}

Proprio come i personaggi subiscono danni dalle cadute superiori ai 3 metri, allo stesso modo subiscono danni se vengono colpiti da oggetti cadenti.

Gli oggetti che cadono addosso ai personaggi infliggono danni a seconda del loro peso e della distanza da cui sono caduti.

La \textbf{Tabella: Danno da Oggetti Cadenti} determina la quantità di danni inflitti da un oggetto in base alla sua taglia. Si presume che l'oggetto sia fatto di un materiale denso e pesante, come la pietra.
Gli oggetti fatti di materiali più leggeri potrebbero infliggere la metà o meno del danno indicato, a discrezione del Narratore. Per esempio un masso Enorme che colpisce un personaggio infligge 6d6 danni, mentre un carro di legno potrebbe infliggerne solo 3d6.

Inoltre, se l'oggetto cade da una distanza inferiore ai 3 metri, infligge la metà dei danni indicati. Se un oggetto cade da una distanza superiore ai 20 metri, infligge danni raddoppiati. L'oggetto che cade subisce la stessa quantità di danni che infligge.

\bigskip

\textbf{Tabella: Danno da Oggetti Cadenti}\index[Tabelle]{Tabella Danno da Oggetti Cadenti}

\medskip

\noindent\begin{tabularx}{\linewidth}{Xl}
	\toprule
\rowcolor{gray!20}\textbf{Taglia dell'Oggetto} & \textbf{Danno}\\
\toprule
Minuscola o più Piccola & 1d6\\
\rowcolor{gray!20}Piccola & 2d6\\
Media & 3d6\\
\rowcolor{gray!20}Grande & 4d6\\
Enorme & 6d6\\
\rowcolor{gray!20}Mastodontica & 8d6\\
Colossale & 10d6
\end{tabularx}

\bigskip

Lasciar cadere addosso ad una creatura un oggetto richiede un attacco a tocco a distanza (vedi \hyperlink{attaccoatocco}{Attacco a Tocco}, pag. \pageref{attaccoatocco}). Questi attacchi hanno di solito una gittata di 3 metri. Se un oggetto cade su una creatura la creatura deve effettuare, se colpita, un Tiro Salvezza su Riflessi con DC 15 per dimezzare il danno se è consapevole dell'oggetto che sta cadendo. Gli oggetti cadenti che sono parte di una trappola usano le regole relative alle trappole invece che quelle qui descritte.

\subsubsection{Pericoli dell'Acqua}\index{Pericoli dell'Acqua}\index{Acqua}\hypertarget{pericoli-dellacqua}{}\label{pericoli-dellacqua} \index{Nuotare}

Qualsiasi personaggio può attraversare acque relativamente calme che non abbiano una profondità maggiore alla sua altezza, senza bisogno di prove.

Una creatura con una velocità di Nuotare può muoversi attraverso l'acqua alla sua velocità indicata senza fare prove di Nuotare. Ha un bonus di +2d6 su qualsiasi prova di Nuotare per eseguire un'azione particolare o evitare un pericolo.
La creatura può sempre scegliere di prendere 10 su una prova di Nuotare, anche se distratti od in pericolo quando si nuota. Non può prendere il 10 solo in caso di acque tempestose. Una tale creatura può utilizzare l'azione di correre mentre nuota, a condizione che nuoti in linea retta.

Un incantatore si considera Distratto se lancia un incantesimo mentre è in acqua.

Se non si ha il tipo di movimento Nuotare \textbf{muoversi nell'acqua} si considera come \textbf{\emph{terreno} difficile}, e quindi ci si muovo a metà della velocità indicata da movimento.

Se la creatura sa nuotare non sono necessarie prove per muoversi normalmente in acque calme, tranne in cui voglia \emph{correre} (DC 13) o le acque siano mosse (DC 15) o siano tempestose (DC 20).

\medskip

\begin{center}
	\includegraphics[width=0.7\linewidth]{immagini/affogare.png}

	\emph{Henry Justice Ford}\end{center}

\medskip

Se la creatura non sa nuotare allora deve fare una Tiro Salvezza su Tempra a DC 13 ogni round che vuole muoversi, se le acque sono mosse la DC è 19 e se sono tempestose la DC è 24, se si vuole \emph{correre} la DC aumenta di 4.
In caso di fallimento non ci si sposta e si ha un -1 alla prova successiva, in caso di Fallimento Critico la successiva prova prende -4, le penalità si cumulano finché non si riesce nella prova.
Se la prova di nuotare fallisce la creature subisce 1d6 danni letali se le acque scorrono sopra rocce e avvallamenti.

Quando le penalità cumulate sono 9 o più si incomincia ad affondare ed annegare (vedi sotto).

L'acqua molto profonda non è solo nera come la pece ma infligge danni ancora peggiori a causa della pressione nell'ordine di 1d6 danni al minuto ogni 30 metri che separano il personaggio dalla superficie. Un Tiro Salvezza su Tempra superato con successo (DC 15, +1 per ogni prova precedente) indica che il personaggio immerso non subisce danni in quel minuto. L'acqua, oltre i 150 metri di profondità, è molto fredda ed infligge 1d6 Danni Non Letali per minuto di esposizione a causa dell'ipotermia.

\medskip

\textbf{Annegamento / Trattenere il fiato}\index{Annegamento}\index{Affogare}\index{Soffocare}\hypertarget{trattenereilfiato}{}\label{trattenereilfiato}\index{Trattenere il fiato}

\medskip

Qualsiasi personaggio può trattenere il fiato per un numero di round pari 10 + 10 round per il suo punteggio di Costituzione, con un minimo di 10 round. Per ogni Azione compiuta la durata restante diminuisce di 1 round, lanciare un incantesimo con componenti Verbali consuma 3 round di aria in più. Trascorso questo periodo di tempo, il personaggio deve effettuare un Tiro Salvezza su Tempra con DC 12 ogni round per continuare a trattenere il fiato. Ogni round, la DC aumenta di 2.

L'incantatore che lanci incantesimi sott'acqua si considera Distratto.

Se il Tiro Salvezza fallisce il personaggio va immediatamente a 0 Punti Ferita e sviene. Dal round successivo incomincia a perdere 1 Punto Ferita a round fino alla morte (o alla rianimazione!)

Si può annegare in sostanze diverse dall'acqua, come la sabbia, le sabbie mobili, la polvere molto fine o un silos pieno di farro o semplicemente trattenendo il respiro.

\subsubsection{Pericoli del Caldo}\index{Caldo}

\begin{center}
	\includegraphics[height=0.65\linewidth]{immagini/desert.png}
\end{center}

\label{pericoli-del-caldo}

Una creatura sottoposta a temperature molto elevate (sopra i 40° C) deve superare un Tiro Salvezza su Tempra ogni ora (DC 15, +1 per ogni prova precedente) oppure subisce 1d4 danni Non Letali. Se indossa abiti pesanti o qualsiasi tipo di armatura, subisce penalità -1d6 a questi Tiri Salvezza. Un personaggio somma i suoi punti assegnati in Sopravvivenza e può dare un bonus ai compagni pari alla metà del valore per lo stesso Tiro Salvezza. I personaggi Privi di Sensi iniziano a subire danni letali (1d4 danni all'ora).

Un personaggio che subisce Danni Non Letali a causa dell'esposizione al caldo, è soggetto ad un colpo di calore ed è Affaticato. Queste penalità terminano quando il personaggio recupera i Danni Non Letali subiti a causa del caldo.

Il caldo infernale (temperatura dell'aria sopra i 60° C, fuoco, acqua che bolle, lava) infligge danni letali. Respirare l'aria con queste temperature infligge 1d6 danni da fuoco al minuto (senza Tiro Salvezza).

L'acqua bollente infligge 1d6 danni da scottatura, a meno che il personaggio non vi venga completamente immerso, nel qual caso subirebbe 10d6 danni per round di esposizione.

\subsubsection{Prendere Fuoco}\index{Prendere Fuoco}\index{Fuoco}

\label{prendere-fuoco}

I personaggi esposti ad olio bollente, fuochi da campo, fuochi magici non istantanei possono vedere i loro abiti, capelli o equipaggiamento prendere fuoco. Gli incantesimi specificano se sono in grado di appiccare il fuoco.

I personaggi che rischiano di prendere fuoco possono effettuare un Tiro Salvezza su Riflessi con DC 15 per evitare questo fato. Se i vestiti o i capelli di un personaggio prendono fuoco, egli subisce immediatamente 1d6 danni. Per ogni round successivo il personaggio in fiamme deve effettuare un altro Tiro Salvezza su Riflessi. Il fallimento indica che subisce altri 1d6 danni in quel round. Il successo indica che il fuoco si è estinto (una volta che supera il Tiro Salvezza non sta più andando a fuoco).

\begin{center}
	\includegraphics[width=0.7\linewidth]{immagini/fuocopericolo.png}
\end{center}

Un personaggio che va a fuoco può estinguere automaticamente le fiamme saltando dentro a dell'acqua sufficiente a spegnerle. Se non ci sono grosse quantità d'acqua a disposizione, rotolarsi sul terreno o smorzare la fiamma con mantelli o simili può concedere al personaggio +1d6 al Tiro Salvezza.

Il personaggio in fiamme deve fare un Tiro Salvezza su Riflessi a DC 15 per ogni oggetto portato, se fallisce gli oggetti subiscono la stessa quantità di danni del personaggio.\\

\textbf{Effetti della Lava}\index{Lava}\\

La lava o il magma infliggono 2d6 danni per round di esposizione, tranne in caso di totale immersione (come quando un personaggio cade nel cratere di un vulcano attivo), che infligge 20d6 danni per round (più eventuali danni da caduta e magari trova un anello..).

I danni provocati dal magma continuano per 1d3 round dopo il termine dell'esposizione, ma questi danni addizionali sono solo la metà di quelli inflitti durante l'ultimo round di effettivo contatto (20/10/5). Un'Immunità o una Resistenza al fuoco serve anche come resistenza alla lava od al magma. Tuttavia, le creature Immuni o Resistenti al Fuoco potrebbero annegare se immerse nella lava (vedi Annegamento).

\subsubsection{Pericoli del Freddo}\index{Freddo}

\label{pericoli-del-freddo}

I personaggi non ben vestiti in climi freddi (sotto i 5° C) devono superare un Tiro Salvezza su Tempra ogni ora (DC 15, +1 per ogni prova precedente) oppure subiscono 1d6 Danni Non Letali.
In condizioni di freddo estremo o di esposizione sotto i -17° C, un personaggio non adeguatamente vestito deve effettuare un Tiro Salvezza su Tempra ogni 10 minuti (DC 15, +1 per ogni prova precedente), subendo 1d6 Danni Letali per ogni Tiro Salvezza fallito. I personaggi che indossano abiti invernali hanno bisogno di effettuare la prova per il freddo e l'esposizione solo una volta all'ora.

\begin{center}
	%\includegraphics[height=0.7\linewidth]{immagini/snowfall.png}
	\includegraphics[height=0.6\linewidth]{immagini/Forest_road_Slavne_2017_G4_grayscale.png}
\end{center}

Un personaggio somma i punti assegnati in Sopravvivenza ai Tiri Salvezza e può dare un bonus ai compagni pari alla metà del valore per lo stesso Tiro Salvezza.

Un personaggio che subisce Danni Non Letali a causa del freddo o dell'esposizione, è soggetto ai geloni o all'ipotermia (considerarlo come Affaticato). Queste penalità terminano quando il personaggio recupera dai Danni Non Letali subiti a causa del freddo e dell'esposizione.

Le condizioni di freddo intollerabile o di esposizione (sotto i -28° C) infliggono ai personaggi 1d6 danni letali per minuto (senza alcun Tiro Salvezza) se non specificatamente protetti.

\subsubsection{Effetti del Ghiaccio}\index{Ghiaccio}

I personaggi che camminano sul ghiaccio è come se camminassero su terreno difficile. Il movimento è dimezzato, eventuali prove di Acrobatica hanno un aumento di difficoltà +4. I personaggi che sono per lungo tempo a contatto con il ghiaccio potrebbero subire dei danni da freddo estremo.

\subsubsection{Soffocamento Lento}\index{Soffocamento}

Un personaggio di taglia Media può respirare tranquillamente per circa 6 ore in una camera sigillata che misura 3 metri di lato. Dopo questo tempo, subisce 1d6 Danni Non Letali ogni 15 minuti. Ogni ulteriore personaggio di taglia Media oppure ogni fuoco significativo (una torcia, per esempio) riducono proporzionalmente la durata dell'aria respirabile. Una volta privi di sensi per l'accumulo di Danni Non Letali, i personaggi iniziano a subire Danni Letali allo stesso ritmo. I personaggi di taglia Piccola consumano metà dell'aria dei personaggi di taglia Media.

\subsection{Tempo Atmosferico - Meteo}\index{Meteo}

\label{tempo-atmosferico---meteo}

A volte il tempo atmosferico può giocare un ruolo importante nel corso di un'avventura. La Tabella: Tempo Atmosferico Casuale è una tabella generica che può essere utilizzata per stabilire le condizioni atmosferiche locali. I termini della tabella sono definiti qui di seguito:

\end{multicols}

\medskip

\textbf{Tabella: Tempo Atmosferico Casuale}\index[Tabelle]{Tabella Tempo Atmosferico Casuale}

\medskip

\noindent\begin{tabularx}{\linewidth}{llXXl}
	\toprule
\rowcolor{gray!20}\textbf{d\%} & \textbf{Meteo} & \textbf{Clima Freddo}& \textbf{Clima Temperato {*}} & \textbf{Deserto}\\
\toprule
01-70 & Normale& Freddo, calmo & Normale per la stagione {*}{*} & Torrido, calmo\\
\rowcolor{gray!20}71-80 & Anormale & Ondata di Caldo (01-30) / Ondata di Freddo (31-100)&Ondata di Caldo (01-50) - Ondata di Freddo (51-100)& Torrido, ventilato \\
81-90 & Inclemente & Precipitazioni (neve)& Precipitazioni normali per la stagione& Torrido, ventilato \\
\rowcolor{gray!20}91-99 & Tempesta & Tempesta di neve& Tempesta di fulmini / Tempesta di neve& Tempesta di polvere \\
100& Tempesta violenta& Tormenta & Bufera, tormenta, uragano, tornado & Acquazzone
\end{tabularx}

\medskip

* Temperato comprende foreste, colline, paludi, montagne, pianure e zone marine calde.

** L'inverno è freddo, l'estate è calda, l'autunno e la primavera sono moderati. Le paludi sono sempre leggermente più calde d'inverno.

\begin{multicols}{2}


\textbf{Acquazzone}: Considerarlo come pioggia (vedi Precipitazioni sotto), ma offre copertura come la nebbia. Può provocare inondazioni e dura di solito 2d4 ore.

\textbf{Caldo}: La temperatura è tra 15° e 30° C di giorno, e tra 6 e 11 gradi in meno di notte.

\textbf{Calmo}: Vento leggero (tra 0 e 15 km/h).

\textbf{Freddo}: Temperatura tra -17° e 5° C durante il giorno, e tra 6 a 11 gradi in meno di notte.

\textbf{Moderato}: Temperatura tra i 5° e i 15° C durante il giorno, e tra 6 e 11 gradi in meno di notte.

\textbf{Ondata Caldo}: Fa aumentare la temperatura di 6° C.

\textbf{Ondata Freddo}: Abbassa la temperatura di 6° C.

\textbf{Precipitazioni}: Tirare un d100 per determinare se la precipitazione è nebbia (01-30), pioggia/neve (31-90), o nevischio/ grandine (91-00). La neve e il nevischio si verificano solo quando la temperatura è di 0° C o inferiore. La maggior parte delle precipitazioni dura 2d4 ore. La grandine, invece, dura solo 3d6 minuti ma di solito è accompagnata da 1d4 ore di pioggia.

\textbf{Tempesta} (di Fulmini/di Neve/di Polvere): Il vento è molto forte (da 45 a 75 km/h) e la visibilità è ridotta di tre quarti. Le tempeste durano 2d4-1 ore. Vedi Tempeste, sotto, per ulteriori dettagli.

\textbf{Tempesta} (Bufera/Tormenta/Uragano/Tornado): La velocità del vento è superiore ai 75 km/h (vedi Tabella: Effetti del Vento). Inoltre, le tormente sono accompagnate da pesanti nevicate (1d3 \texttimes{} 30 cm), e gli uragani sono accompagnati da acquazzoni. Le bufere durano 1d6 ore, le tormente 1d3 giorni. Gli uragani possono durare fino a una settimana, ma l'impatto maggiore per i personaggi avverrà in un periodo di tempo tra le 24 e le 48 ore, mentre il centro della tempesta si sposta nella loro zona. I tornado durano molto poco (1d6 \texttimes{} 10 minuti), e di solito si formano come parte di una tempesta di fulmini.

\textbf{Torrido}: Temperatura tra i 30° e i 43° C durante il giorno e tra 6 e 11 gradi in meno di notte.

\textbf{Ventilato}: La velocità del vento va da moderata a forte (da 15 a 45 km/h); vedi Tabella: Effetti del Vento.

\textbf{Pioggia, Neve, Nevischio e Grandine}: La brutta stagione frequentemente rallenta o blocca i trasporti via terra e rende praticamente impossibile la navigazione. acquazzoni torrenziali e bufere oscurano la visuale tanto quanto lo farebbe una nebbia densa.

La maggior parte delle precipitazioni si manifesta come pioggia, ma nei climi freddi possono manifestarsi anche come neve, nevischio o grandine. Le precipitazioni di qualsiasi tipo, seguite da un calo della temperatura da sopra a sotto gli 0° C possono produrre ghiaccio.

\begin{center}

	\includegraphics[width=0.95\linewidth]{immagini/Paesaggio-pioggia-Auvers.png}

	\emph{Vincent van Gogh, Paesaggio sotto la pioggia ad Auvers, 1890, olio su tela, cm 50 x 100}
\end{center}


\textbf{Pioggia intensa}\index{Pioggia intensa}: La pioggia dimezza la visibilità, e impone penalità -1d6 alle prove di Consapevolezza. Ha lo stesso effetto di un vento molto forte sulle fiamme, sugli attacchi con armi a distanza e sulle prove di Consapevolezza come vento molto forte.

\textbf{Neve}\index{Neve}: Mentre cade, la neve ha gli stessi effetti della pioggia su visibilità, attacchi con armi a distanza e prove di Consapevolezza ed il terreno è considerato difficile. Una nevicata della durata di un giorno lascia al suolo 3d6*2.5 centimetri di neve.

\textbf{Neve Fitta}: Una fitta nevicata ha gli stessi effetti di una nevicata normale, ma oscura la visibilità come la nebbia (vedi Nebbia). Un giorno di neve fitta lascia sul terreno 2d4 x 30 centimetri di neve ed il terreno viene considerato doppiamente difficile (movimento/4). Una fitta nevicata accompagnata da venti forti o molto forti può dare origine a cumuli di neve profondi 1d4 x 1 metro, specialmente sopra e intorno ad oggetti abbastanza grandi da deflettere il vento (una capanna o una grande tenda, per esempio).
C'è una probabilità del 10\% che una nevicata fitta sia accompagnata da fulmini (vedi Tempesta di Fulmini). La neve ha gli stessi effetti del vento moderato sulle fiamme.

\textbf{Nevischio}: Si tratta fondamentalmente di pioggia congelata, che ha gli stessi effetti della pioggia quando cade (eccetto che la probabilità di estinguere fiamme protette è del 75\%) e quelli della neve una volta depositatasi.

\textbf{Grandine}: La grandine non riduce la visibilità, ma il suono della grandine che cade rende più difficili le prove di Consapevolezza basate sull'udito (penalità -1d6). A volte (probabilità del 5\%) la grandine può essere talmente grossa da infliggere 1 danno letale (per tempesta) a qualsiasi cosa si trovi all'aperto. Una volta depositata, la grandine ha lo stesso effetto della neve sul movimento.

\subsubsection{Tempeste}\index{Tempeste}

\label{tempeste}

Gli effetti combinati delle precipitazioni (o della polvere) e del vento, che accompagnano tutte le tempeste, riducono la visibilità di tre quarti, imponendo penalità -8 a tutte le prove di Consapevolezza. Le tempeste rendono impossibili gli attacchi con le armi a distanza, tranne che con le armi da assedio, che subiscono penalità -1d6 i Tiri per Colpire.
Estinguono automaticamente le candele, le torce o simili fiamme non protette. Le fiamme protette, come quelle delle lanterne, vengono agitate violentemente e hanno una probabilità del 50\% di estinguersi. Vedi Tabella: Effetti del Vento per le possibili conseguenze sulle creature sorprese all'esterno senza ripari.

Le tempeste sono di tre tipi.

\textbf{Tempesta di Polvere (grado di Sfida 3)}

queste tempeste desertiche si differenziano dalle altre tempeste in quanto non hanno precipitazioni. Al contrario, le tempeste di polvere trasportano granelli di sabbia che oscurano la vista, soffocano le fiamme non protette e possono addirittura spegnere quelle protette (probabilità del 50\%). Molte tempeste di polvere sono accompagnate da venti molto forti e si lasciano alle spalle un deposito di 1d6 \texttimes{} 2.5 centimetri di sabbia.
Esiste anche una probabilità del 10\% di incontrare grandi tempeste di polvere con bufere di vento (vedi Tabella: Effetti del Vento). Queste violente tempeste di polvere infliggono 1d3 danni non letali per round a chiunque venga sorpreso all'aperto senza riparo e pongono anche il rischio del soffocamento (vedi Annegamento, eccetto che un personaggio con una sciarpa o simile protezione sulla bocca e il naso, non inizia a soffocare se non dopo un numero di round pari 10 \texttimes{} il suo punteggio di Costituzione). Le grandi tempeste di polvere si depositano alle spalle (2d3-1) x 30 centimetri di sabbia.

\textbf{Tempesta di Neve}

oltre ai venti e alle precipitazioni comuni alle altre tempeste, le tempeste di neve depositano 1d6 \texttimes{} 2.5 centimetri di neve sul terreno.

\textbf{Tempesta di Fulmini}

oltre ai venti e alle precipitazioni (di solito pioggia, ma a volte anche grandine), le tempeste di fulmini sono accompagnate da scariche elettriche che rappresentano un pericolo per i personaggi che si trovano all'aperto senza riparo (specialmente se indossano armature metalliche). Come regola generale, si può considerare un fulmine al minuto per un periodo di un'ora nel cuore della tempesta. Ogni fulmine infligge danni da elettricità tra 4d8 e 10d8. Una tempesta di fulmini su dieci viene accompagnata da un tornado.

\textbf{Tempeste Violente}

Venti molto forti e precipitazioni torrenziali riducono la visibilità a zero, e rendono impossibile effettuare prove di Consapevolezza e compiere attacchi con armi a distanza. Le fiamme non protette vengono automaticamente spente, e c'è una probabilità del 75\% che ciò si verifichi anche per quelle protette. Le creature sorprese in queste zone devono effettuare un Tiro Salvezza su Tempra o devono affrontare effetti a seconda della propria taglia (vedi Tabella: Effetti del Vento). Le tempeste violente sono suddivise nei seguenti quattro tipi.

%\begin{center}
%\includegraphics[width=0.95\linewidth]{immagini/Vincent_van_Gogh_tempesta.png}

%\emph{Vincent van Gogh, Campo di grano sotto un cielo tempestoso (Auvers-sur-Oise, luglio 1890)}

%\end{center}

\textbf{Bufera}

Sebbene abbiano poche o nessuna precipitazione, le bufere possono provocare danni ingenti a causa della forza del vento.

\textbf{Tormenta}

La combinazione di forti venti, neve fitta (di solito 1d3 \texttimes{} 30 cm) e freddo intenso rende le tormente letali per chiunque non vi sia preparato.

\textbf{Uragano}

Oltre ai venti molto forti e alla pioggia intensa, gli uragani sono seguiti da inondazioni. Molte attività in un'avventura sono impossibili in queste condizioni.

\textbf{Tornado}

Oltre ai venti molto forti, i tornado possono ferire gravemente ed uccidere quelli che vengono catturati al suo interno.

\subsubsection{Nebbia}\index{Nebbia}

\label{nebbia}

Sia nella forma di una nube a bassa altitudine che di una foschia che sale dal terreno, la nebbia ostacola la visuale oltre la distanza di 3 metri. Le creature più lontane di 3 metri godono di Copertura leggera (+2 Difesa).

La nebbia rende il terreno difficile.

La nebbia potrebbe essere anche molto fitta in quel caso le creature più lontane di 6 metri godono di Copertura completa, entro 4 metri hanno copertura media (+4 alla Difesa) e quelle entro 1 metro hanno comunque copertura leggera.

\subsubsection{Venti}\index{Venti}

\label{venti}

I venti possono creare turbini di sabbia o polvere, alimentare grossi incendi, rovesciare piccole imbarcazioni e disperdere gas o vapori. Se sono forti a sufficienza possono addirittura buttare a terra i personaggi (vedi Tabella: Effetti del Vento), interferire con gli attacchi a distanza, o imporre penalità ad alcune Prove di Competenze.

\medskip

\textbf{Tabella: Effetti del Vento Forza del Vento}\index[Tabelle]{Tabella Effetti del Vento Forza del Vento}

\medskip

\noindent\begin{tabularx}{\linewidth}{Xll}
	\toprule
\rowcolor{gray!20}\textbf{Intensità} & \textbf{Velocità} & \textbf{Attacchi a}\\
& & \textbf{Distanza}\\
\toprule
\rowcolor{gray!20}Leggero & 0-15km &\\
Moderato & 16,5-30 km/h& \\
\rowcolor{gray!20}Forte & 31.5-45 & -2 \\
Molto forte & 45.5-75km/h & -4 \\
\rowcolor{gray!20}Bufera & 76.5-111km/h & impossibile \\
Uragano & 12-261km/h & impossibile \\
\rowcolor{gray!20}Tornado & 262-450km/h & impossibile
\end{tabularx}

\bigskip

\textbf{Vento Leggero}

Una brezza gentile, che non ha effetti pratici sul gioco.

\textbf{Vento Moderato}

Un vento sostenuto, che ha una probabilità del 50\% di estinguere qualsiasi piccola fiamma non protetta, come quella di una candela.

\textbf{Vento forte}: Folate che spengono automaticamente le fiamme non protette (candele, torce e simili). Queste folate impongono penalità -2 ai Tiri per Colpire a distanza ed alle prove di Consapevolezza.

\textbf{Vento Molto Forte}

Oltre a spegnere automaticamente le fiamme non protette, i venti di questa intensità agitano violentemente le fiamme protette (come quelle di una lanterna) e hanno una probabilità del 50\% di estinguerle. Gli attacchi con le armi a distanza e le prove di Consapevolezza subiscono penalità -1d6.

\textbf{Bufera}\index{Bufera}

Abbastanza forti da abbattere i rami o addirittura interi alberi, le bufere estinguono automaticamente le fiamme non protette e hanno una probabilità del 75\% di estinguere quelle protette, come quelle delle lanterne. Gli attacchi con le armi a distanza sono impossibili, e anche le armi da assedio subiscono penalità -1d6 ai Tiri per Colpire. Le prove di Consapevolezza basate sull'udito subiscono penalità -2d6 per l'ululare del vento.

\textbf{Uragano}\index{Uragano}

Estingue tutte le fiamme. Gli attacchi a distanza sono impossibili (eccetto con le armi da assedio che subiscono penalità -2d6 ai Tiri per Colpire). Anche le prove di Consapevolezza basate sull'udito sono impossibili e tutto ciò che i personaggi possono udire è l'ululare del vento. Gli uragani spesso sono in grado di abbattere gli alberi.

\textbf{Tornado (grado di Sfida 10)}\index{Tornado}

Estingue tutte le fiamme. Tutti gli attacchi a distanza sono impossibili (compresi quelli con le armi da assedio), così come le prove di Consapevolezza basate sull'udito. Invece di essere portati via (vedi Tabella: Effetti del Vento), i personaggi che si trovano nelle immediate vicinanze di un tornado e che falliscono un Tiro Salvezza su Tempra (DC 20+) vengono risucchiati dentro il tornado.

Coloro che entrano in contatto con il tornado vengono sollevati da terra e sbatacchiati per 1d10 round, subendo 6d6 danni per round, prima di venirne espulsi violentemente. La creatura viene espulsa da un altezza di 1d6 metri per round di permanenza nel tornado.

Sebbene la velocità rotatoria di un tornado possa raggiungere i 450 km/h, il cono stesso si muove in avanti ad una media di 45 km/h (circa 75 metri per ogni round). Un tornado è in grado di sradicare alberi, distruggere edifici e provocare altre forme di simile devastazione.

\end{multicols}

\vfill

\OBSSseparator

%\vfill

%\begin{center}
%\includegraphics[keepaspectratio,width=0.6\textwidth]{immagini/blizzard.png}

%\end{center}

\pagebreak

\section{Avventure in Acqua}\index{Avventure in Acqua}

\label{avventure-in-acqua}
\begin{enfasi}{
Guardò il mare e capì fino a che punto era solo, adesso. (Il vecchio e il mare, Ernest Hemingway)}\end{enfasi}

\begin{multicols}{2}

L'acqua permette alle società di esistere, ma può anche distruggerle. La vita non potrebbe esistere senza di essa ma l'acqua può anche uccidere, sia annegando le persone, sia generando alluvioni e tsunami su larga scala.

\textbf{Avventure Acquatiche}

Un'avventura acquatica può aver luogo ovunque l'acqua rappresenti l'elemento principale del territorio: come paludi, fiumi, laghi, stagni, oceani, il Piano dell'Acqua e simili. Le avventure Acquatiche non richiedono che i personaggi abbiano la capacità di respirare sott'acqua, le sfide Acquatiche per avventurieri di basso livello portano ad un'avventura tensione e sensazione di pericolo.

\textbf{Adattarsi agli Ambienti Acquatici}

Le regole per il combattimento sott'acqua si applicano alle creature che non sono native di questo pericoloso ambiente come la maggior parte dei PG. Per avventure Acquatiche prolungate ed esplorazioni particolarmente in profondità, i personaggi necessiteranno dell'uso della magia per proseguire le proprie avventure. Il danno da pressione può essere totalmente evitato tramite incantesimi che offrano una resistenza o dal poter respirare sott'acqua.

\medskip

\begin{center}

%\includegraphics[width=0.4\linewidth]{immagini/avventure_acqua_grey.png}

\includegraphics[width=0.75\linewidth]{immagini/Poe's_Tales_of_Mystery-Rackham-047_grayscale.png}

\emph{Poe's Tales of Mystery-Rackham}
\end{center}

\subsection{Combattimento sott'acqua}\label{combatteresottacqua}\index{Combattimento sott'acqua}\hypertarget{combatteresottacqua}{}
Le creature che vivono sulla terra hanno considerevoli difficoltà a combattere sott'acqua. L'acqua influenza la Difesa di una creatura, i suoi Tiri per Colpire, i danni e il movimento.

\begin{itemize}[leftmargin=*] \setlength{\itemsep}{0pt}
\item
Una creatura sott'acqua perde il bonus di Destrezza alla Difesa.
\item
Una creatura sott'acqua che non sia sotto l'incantesimo \emph{Libertà di movimento} effettua i Tiri per Colpire con un -1d6 e l'avversario si considera che abbia Resistenza al danno contro armi da Taglio e Contundente.

Armi come Tridente, Lancia corta, Spada Corta, Giavellotto non hanno penalità al colpire sott'acqua in mischia.
\item
Muoversi o nuotare in acqua si considera \textbf{\emph{terreno} difficile}.
\end{itemize}

Queste penalità sono valide solo se non si ha un movimento di tipo Nuotare.

\subsubsection{Attacchi a distanza sott'acqua}\index{Attacchi sott'acqua}
Le armi da lancio sono inefficaci sott'acqua, anche quando vengono lanciate da terra. Gli attacchi con le armi a distanza subiscono -1 al danno per ogni 1 metro d'acqua che attraversano.

\subsubsection{Attacchi dalla terraferma}
Quei personaggi che nuotano, galleggiano o attraversano l'acqua in superficie, o guadano un tratto in cui l'acqua è alta almeno fino al petto, godono di copertura media.

Una creatura completamente sommersa dispone di copertura completa contro gli avversari sulla terraferma.

\subsubsection{Effetti magici in acqua}
Gli effetti magici non sono influenzati, tranne quelli che richiedono un Tiro per Colpire (vedi sopra) e gli effetti di fuoco.

Il \textbf{fuoco} non magico (incluso il fuoco dell'alchimista) non brucia sott'acqua. Gli incantesimi o gli effetti magici di fuoco sono inefficaci sott'acqua. Una creatura parzialmente sommersa ha resistenza al fuoco.

\textbf{Lanciare incantesimi sott'acqua}\index{Lanciare incantesimi sott'acqua}

Lanciare incantesimi sott'acqua può essere difficile per chi non ha la capacità di respirare sott'acqua.

Una creatura incapace di respirare sott'acqua spende tre round di trattenere il respiro per lanciare un incantesimo con componente Verbali.

Alcuni incantesimi potrebbero funzionare diversamente sott'acqua, a discrezione del Narratore.

Vedi Capitolo Ambiente per le regole sul \hyperlink{trattenereilfiato}{trattenere il respiro} (pag. \pageref{trattenereilfiato}).

\end{multicols}

\pagebreak

\section{Avventure nei Dungeon}\index{Dungeon}

\begin{enfasi}{

\st{Linux} Dungeon is user friendly. It's just very picky about who its friends are. (anonimo)

\medskip

Il dungeon è inclinato. Le creature sono infuriate perché non riescono a giocare a biglie (Dungeon Keeper 2, Videogioco, 1999)

}\end{enfasi}

\label{avventure-nei-dungeon}
\begin{multicols}{2}

Di tutti i luoghi strani che un avventuriero può esplorare, nessuno è più letale di un dungeon. Questi labirinti, pieni di trappole mortali, mostri affamati e tesori meravigliosi, provano ogni abilità e capacità dei personaggi. Queste regole si possono applicare a qualsiasi tipo di dungeon, dal relitto di una nave ad un vasto complesso di grotte sotterranee.

\begin{narratore}[Dungeon!!!]
Il dungeon, caverna, catacomba, spelonca, ecosistema sotterraneo... chiamatelo come preferite è un cardine dell'avventura!

Un dungeon è una ricetta fatta di umidità, fetore, aria stantia, sporcizia, fango, resti di creature, trappole, melme, trappole (abbondate...), mostri, nemici, mostri (abbondare!), oscurità, rumori sinistri, funghi, scricchiolii, guaiti, urla, gemiti.. ma anche di paura, tensione, brivido terrore \& raccapriccio, enfasi, rabbia, dolore, delusione e tesori!!!

Il vostro dungeon non è mai solo una caverna. MAI!
\end{narratore}


Che siano caverne, antri, cave, grotte, tane, spelonche i \emph{Dungeon} rappresentano spesso il centro focale dell'avventura, dell'esplorazione e sopravvivenza.

I personaggi passeranno molto tempo in questi ambienti ed il Narratore deve essere preparato e pronto sull'ambiente che incontreranno.

Quando si prepara una caverna è necessario ragionare in maniera intelligente sul tipo di caverna e sulle creature che si andranno ad incontrare, ogni caverna è un complesso ecosistema.
Mettere un gruppo di lucertoloidi senza pensare a cosa mangiano, dove dormono, che tipo di organizzazione hanno è pericoloso, per non parlare di inserire una chimera.
Avrà le ali atrofizzate perché la caverna è alta 3 metri e larga 3 e fa fatica a muoversi ? Di cosa si è nutrita in questo periodo ? Piuttosto meglio usare una gorgone che si nutre di minerali...

Se progettato con attenzione e cura una caverna può diventare un'ottima esperienza di incontri, situazioni ed avventura.

\subsection{Il sottosuolo}\index{Il sottosuolo}

Le condizione naturali del sottosuolo dipendono da vari fattori ma ci sono sicuramente dei punti comuni a tutti.

\begin{itemize}[leftmargin=*] \setlength{\itemsep}{0pt}
\item Nessuna luce per illuminare gli spazi. Possono esserci sporadici funghi fluorescenti, che irradiano luce fioca nelle vicinanze, ma nulla che possa illuminare tutto l'ambiente
\item Ambiente umido
\item Temperatura ambiente solitamente fresca, raramente ci sono caverne con temperature estreme sia nel caldo che nel freddo.
\end{itemize}

\subsubsection{Illuminazione}\index{Illuminazione}

In una caverna non ci sono fonti di luci artificiali o naturali se non quelle introdotte da creature senzienti. Possono esserci gruppi di funghi, licheni, che illuminano debolmente il terreno dove crescono, entro 1 metro, ma null'altro intorno.
Oltretutto se strappati dal terreno perdono la bioluminescenza dopo 2d4 Turni.

Le creature che vivono nelle caverne si abituano all'oscurità sviluppando qualche forma di visione alternativa, quale scurovisione, senso tellurico o vista cieca.

Anche la stessa torcia può dare un sollievo limitato dato che il suo raggio di luce è di 3 metri più 3 metri di luce fioca e dura un ora prima di spegnersi.

Controlla le sezioni su \hyperlink{copertura}{Coperture} ed \hyperlink{invisibilita}{Invisibilità} (pag. \pageref{invisibilita}) per maggiori informazioni.

\subsubsection{Movimento}\index{Movimento}

Se non si hanno mezzi per vedere il terreno si considera difficile e buche, precipizi ed ostacoli vari possono essere molto pericolosi.\index{Terreno difficile in oscurità}

In caso di totale buio ed in un ambiente naturale va fatta una prova di Acrobatica a DC 12 ogni Azione di Movimento o inciampare e subire 1 di danno temporaneo.

\subsection{Tipologie di caverne}

Si possono individuare diverse tipologie di caverne:

\begin{itemize}[leftmargin=*] \setlength{\itemsep}{0pt}

\item \textbf{creati dallo scorrere dell'acqua}. In questo caso il tunnel può essere parecchio caotico nel suo dipanarsi a causa del tipo di rocce che l'acqua ha incontrato. Ci possono essere ancora fiumi e laghi sotterranei.

\item \textbf{creati dalla erosione}. In questo caso l'acqua probabilmente non c'è più se non in minima parte, le caverne risultanti possono essere anche molto grandi con sale di decine se non centinaia di metri di ampiezza.

\item possono essere stati \textbf{creati da un vulcano} con lo scorrere della lava. In questo caso il tunnel scavato dalla roccia è spesso lineare e in qualche maniera levigato, la lava una volta rappresa si è poi sbriciolata nei millenni.

\item possono essere \textbf{caverne artiche}, scavate nel ghiaccio dall'acqua. In questo caso valutate bene l'ambiente circostante e la temperatura gelida.

\item possono essere \textbf{caverne artificiali}, costruite da creature di diverso tipo.

\end{itemize}

\subsubsection{Le quattro tipologie di Dungeon}\index{Le quattro tipologie di Dungeon}

I quattro tipi base di dungeon sono definiti dal loro stato attuale. Molti dungeon sono varianti di questi tipi base o combinazioni di più tipi. Occasionalmente, antichi dungeon vengono usati da nuovi abitanti per scopi diversi.

\textbf{Struttura in Rovina}: Un tempo abitato, questo luogo è ora abbandonato (completamente o in parte) dai suoi creatori originari ed è occupato da altre creature. Molte creature sotterranee vanno alla ricerca di costruzioni sotterrane ed abbandonate in cui stabilire le loro tane. Qualsiasi trappola che possa essere esistita è stata probabilmente già rimossa o attivata, è possibile trovare bestie erranti.

\medskip
\begin{center}
\includegraphics[width=0.9\linewidth]{immagini/avventure_dungeon.png}

\emph{The Red Romance Book, Henry Justice Ford}
\end{center}
\medskip

\textbf{Struttura Occupata}: Questo dungeon viene ancora utilizzato. Delle creature (di solito intelligenti) ancora lo abitano, anche se potrebbero non essere i creatori del dungeon. Una struttura occupata potrebbe essere una casa, una fortezza, un tempio, una miniera attiva, una prigione, un quartier generale...

Questo tipo di dungeon è meno probabile che abbia trappole o bestie erranti, e più probabilmente dispone di guardie organizzate, sia di guardia che di pattuglia. Le trappole e le bestie erranti che si possono incontrare sono spesso sotto il controllo degli occupanti. Le strutture occupate dispongono di arredo adatto agli abitanti, così come decorazioni, riserve di cibo, e la possibilità per gli abitanti di muoversi.

Gli abitanti possono disporre anche di un sistema di comunicazione, e quasi sempre controllano almeno un accesso verso l'esterno.

Alcuni dungeon sono parzialmente occupati e parzialmente vuoti o in rovina. In questi casi, gli occupanti di solito non sono gli originari costruttori del luogo, ma bensì un gruppo di creature intelligenti che hanno stabilito la loro base, tana o fortificazione all'interno del dungeon abbandonato.

\textbf{Riparo Sicuro}: Quando qualcuno vuole proteggere una cosa, spesso la seppellisce sottoterra. Che l'oggetto che vuole proteggere sia un favoloso tesoro, un artefatto proibito o il cadavere di un uomo importante, questi oggetti di valore vengono posti all'interno di un dungeon e circondati da barriere, trappole e guardiani.

Il dungeon del tipo riparo sicuro è quello che avrà più trappole e meno bestie erranti. E' normalmente costruito in base alla funzionalità piuttosto che all'aspetto, anche se a volte viene decorato con statue e pareti dipinte, specie per le tombe di personaggi importanti.

%\begin{center}
%\includegraphics[width=0.7\linewidth]{immagini/dungeon.png}
%\end{center}

A volte, però, una sala del tesoro o una cripta vengono costruite in modo da ospitare guardiani viventi. Il problema con questa strategia è che occorre tenere in vita le creature tra un tentativo di intrusione e un altro. La magia è di solito la soluzione migliore per rifornire di cibo e acqua queste creature. I costruttori di tombe e sepolcri, di solito, pongono non morti e costrutti, che non hanno bisogno di sostentamento o di riposo, a protezione dei loro dungeon. Le trappole magiche possono attaccare gli intrusi convocando mostri nel dungeon che scompaiono quando terminano il loro compito.

\textbf{Complesso di Caverne Naturali}: Le caverne sotterranee offrono riparo a qualsiasi tipo di creatura delle profondità. Create naturalmente e collegate da un sistema di passaggi labirintici, queste caverne mancano di qualsiasi parvenza di ordine, logica o decorazioni. Senza alcuna potenza intelligente che lo abbia costruito, questo tipo di dungeon è quello che ha minori probabilità di presentare trappole o porte.

Molteplici varietà di funghi vivono nelle caverne, a volte crescendo fino a formare enormi foreste di funghi e vesce, dove si aggirano predatori sotterranei a caccia di chi si nutre di questi vegetali. Alcune varietà di funghi producono un bagliore fosforescente in grado di fornire al complesso di caverne naturali una propria limitata fonte di illuminazione. In altre zone, l'uso di incantesimi di Luce Diurna può garantire luce sufficiente per la crescita di piante verdi.

Spesso, un complesso di caverne naturali è collegato ad altri tipi di dungeon, essendo stato scoperto quando è stato costruito il dungeon artificiale. Un complesso di caverne può collegare due dungeon indipendenti, producendo a volte uno strano ambiente misto. Un complesso di caverne naturali unito a un altro dungeon spesso offre un percorso che le creature sotterranee possono usare per raggiungere un dungeon artificiale e popolarlo.

\subsection{Esplorazione}\index{Esplorazione}\index{Muoversi con attenzione}

Muoversi all'interno di un dungeon richiede attenzione e sangue freddo. Pavimenti accidentati, rumori sinistri, botole e trappole, luci che appaiono e scompaiono rendono non facile avventurarsi in sicurezza in questi ambienti pericolosi.

I personaggi dovranno stare attenti, cercare attivamente trappole, osservare in lontananza e tenere un atteggiamento prudente. Tutto questo significa che il movimento è dimezzato se i personaggi \emph{mettono in essere precauzioni} per evitare problemi, ovvero avere un minimo bonus alle prove di Consapevolezza.

Descrivere ciò che il personaggio fa per cercare trappole, passaggi.. \emph{problemi} o richiedere una prova (Sopravvivenza oppure Consapevolezza) a DC 13 può dare indicazioni generiche sulla \emph{sensazione} che ci sia qualcosa che non va.

\subsection{Terreno del Dungeon}\index{Terreno del Dungeon}

Le regole seguenti riguardano i terreni di base che si possono trovare in un dungeon.

\subsubsection{Pareti}\index{Pareti}\label{pareti}\hypertarget{pareti}{}

A volte pareti in mattoni (pietre accatastate una sopra tenute insieme con la calce) dividono i dungeon in corridoi e stanze. Le pareti dei dungeon possono anche essere scolpite nella nuda roccia, ottenendo così un aspetto scalpellato, oppure possono essere composte di pietra liscia e semplice come si trova nelle caverne naturali. Le pareti dei dungeon sono difficili da danneggiare o da sfondare, ma di solito sono facilmente scalabili.

\end{multicols}
\textbf{Tabella: Pareti}\index[Tabelle]{Tabella Pareti}
\medskip

\noindent\begin{tabularx}{\linewidth}{Xccccc}
	\toprule
\rowcolor{gray!20}\textbf{Tipo di Parete} & \textbf{Spessore} & \textbf{Sfondare} & \textbf{Durezza} & \textbf{Punti Ferita} & \textbf{DC Scalare}\\
\toprule
Mattoni di pietra & 30 cm& 35 & 8 & 90& 20\\
\rowcolor{gray!20}Mattoni di pietra superiori & 30 cm& 35 & 8 & 120 & 25\\
Mattoni di pietra rinforzati & 30 & 45 & 8 & 180 & 20\\
\rowcolor{gray!20}Pietra Scolpita & 90 & 50 & 8 & 540 & 25\\
Pietra grezza & 150 cm & 65 & 8 & 900 & 25\\
\rowcolor{gray!20}Ferro & 7.5 cm & 30 & 10& 90& 25\\
Carta & variabile & 1 & --& 1 & 30\\
\rowcolor{gray!20}Legno & 15 cm& 20 & 5 & 60& 21
\end{tabularx}

\medskip

\textbf{Tabella: Scavare un tunnel}\index[Tabelle]{Tabella Scavare un tunnel}

\medskip

\noindent\begin{tabularx}{\linewidth}{Xccc}
	\toprule
\rowcolor{gray!20}\textbf{Minatore}&\multicolumn{3}{c}{\textbf{Materiale da Scavare (1 minuto)}}\\
&\textbf{Terreno}&\textbf{Pietra} \textbf{morbida}&\textbf{Pietra dura}\\
\toprule
Umano&50 cm&15 cm&7 cm\\
\rowcolor{gray!20}Gnomo &45 cm&30 cm&15 cm\\
Nano/Orco & 55 cm&45 cm&20 cm\\
\rowcolor{gray!20}Gigante della Pietra& 3 m& 1.5 m& 75 cm\\
Xorn &6 m&6 m& 6m\\
\rowcolor{gray!20}Elementale della Terra & 9 m&9 m&9 m
\end{tabularx}

\medskip

Le distanze scavate indicate si presume che siano ottenute con strumenti idonei come vanghe o picconi, altrimenti ridurre ad un terzo.

\begin{multicols}{2}

\textbf{Pareti in Mattoni di pietra}: Il tipo più comune di parete per un dungeon, le pareti in pietre di solito sono spesse almeno 30 centimetri. Spesso queste antiche pareti presentano fori e fessure, all'interno dei quali possono annidarsi fanghiglie e piccole creature, che aspettano lì le loro prede. Le pareti di mattone di pietra sono in grado di bloccare tutti i rumori, tranne quelli più forti. E' necessaria una prova di Arrampicarsi con DC 20 per muoversi lungo una parete in mattoni.

\textbf{Pareti in Mattoni in Pietra di Qualità Superiore}: Queste pareti sono spesso abbellite da dipinti, bassorilievi o altre decorazioni. Le pareti in mattoni di qualità superiore non sono più difficili da danneggiare delle normali pareti in mattoni, ma sono più difficili da Arrampicarsi (DC 25).

\textbf{Pareti rinforzate} Queste sono pareti in mattoni con sbarre di ferro su uno o entrambi i lati, o inserite all'interno della parete stessa per rinforzarla. La Durezza della parete rinforzata resta la stessa, ma i Punti Ferita vengono raddoppiati e la DC per sfondarla viene incrementata di 10.

\textbf{Pareti di Pietra Scolpita}: Queste pareti generalmente si trovano in stanze o passaggi scavati nella nuda roccia. La ruvida superficie di una parete scolpita presenta minuscole sporgenze su cui possono crescere funghi e crepe all'interno delle quali possono vivere parassiti, pipistrelli o serpenti sotterranei.

%Quando una parete di questo tipo ha un altro lato (la parete separa due stanze in un dungeon), la parete è spessa almeno 90 centimetri; se fosse più sottile rischierebbe di far crollare tutto perché non sarebbe in grado di sostenere il peso della volta di pietra. E' necessaria una prova di Arrampicarsi con DC 25 per scalare una parete di pietra scolpita.

\textbf{Pareti di Pietra Grezza}: Queste superfici sono irregolari e raramente piatte. Di solito sono bagnate o perlomeno umide, in quanto le caverne naturali sono in genere il prodotto di infiltrazioni d'acqua. Quando una parete di questo tipo da un altro lato, la parete è di solito spessa almeno 150 centimetri.

E' necessaria una prova di Arrampicarsi con DC 15 per muoversi lungo una parete di pietra grezza.

\textbf{Pareti di Ferro}: Queste pareti sono poste all'interno dei dungeon intorno a luoghi importanti come le sale del tesoro.

%\textbf{Pareti di Stoffa}: Le pareti di stoffa sono l'opposto di quelle di ferro, utilizzate come schermi per impedire la vista ma nulla più.

\textbf{Pareti di Legno}: Le pareti di legno si trovano spesso come recenti aggiunte a dungeon più antichi, utilizzate per creare recinti per animali, depositi, o anche solo per dividere in una serie di stanze più piccole una più grande.

\textbf{Pareti Trattate Magicamente}: Queste pareti sono più forti della media, con una Durezza maggiore, con più Punti Ferita e per sfondarle bisogna superare una DC maggiore. La magia può di solito raddoppiare la Durezza e i Punti Ferita della parete e aggiungere fino a +20 alla sua DC per sfondarla. Una parete trattata magicamente ottiene anche un Tiro Salvezza contro Incantesimi che potrebbero avere effetto su di essa, con il bonus al Tiro Salvezza pari a 2 + metà del livello dell'incantatore della magia che rinforza la parete. Creare una parete magica richiede il talento Creare Oggetti Meravigliosi e la spesa di 1.500 mo per ogni sezione di 3 per 3 metri.

\textbf{Pareti con Feritoie}: Le pareti con feritoie possono essere costruite con qualsiasi materiale resistente, ma sono di solito fatte in mattoni, pietra scolpita o legno. Permettono ai difensori di scagliare frecce o quadrelli da balestra contro gli intrusi restando dietro la relativa protezione di un muro. Gli arcieri dietro alle feritoie godono di una Copertura superiore che fornisce loro bonus +8 alla Difesa, bonus +1d6 ai Tiri Salvezza su Riflessi.\index{Feritoie e frecce}

\subsubsection{Pavimenti}\index{Pavimenti}

Così come per le pareti, esistono molti tipi di pavimenti per dungeon.

\textbf{Lastricato}: Come le pareti in mattoni, i pavimenti possono essere composti da pietre incastrate tra loro. Sono di solito piene di fessure e solitamente appena livellate. Fanghiglie e muffe crescono all'interno di queste fessure. In certi casi l'acqua scorre in piccoli scoli attraverso le pietre o forma pozze stagnanti. Il lastricato è il tipo di pavimento più comune nei dungeon.

\textbf{Lastricato Irregolare}: Col passare del tempo, alcuni pavimenti possono diventare talmente irregolari da richiedere una prova di Acrobatica con DC 10 per correre o Caricare sulla loro superficie. Coloro che falliscono la prova non possono muoversi durante quel round. Pavimenti così pericolosi dovrebbero essere in realtà l'eccezione e non la regola.

\textbf{Pavimento di Pietra Scolpita}: Ruvidi e irregolari, i pavimenti scolpiti nella pietra sono di solito coperti da pietre smosse, ghiaia, polvere e altri detriti. Una prova di Acrobatica con DC 10 è necessaria per correre o Caricare su un simile pavimento. Un fallimento significa che il personaggio può ancora agire, ma non può correre o Caricare in quel round.

\textbf{Pietrisco Scarso}: Piccoli e sparuti detriti sono presenti a terra. Un pavimento su cui sia presente del pietrisco scarso aggiunge 2 alla DC delle prove di Acrobatica.

\textbf{Pietrisco fitto}: Il terreno è ricoperto di detriti di tutte le dimensioni. Il pietrisco si considera terreno difficile. Un pavimento cosparso di pietrisco fitto aggiunge 5 alla DC delle prove di Acrobatica, e aggiunge 2 alla DC delle prove di Consapevolezza contro Furtività.

\textbf{Pavimento di Pietra Liscia}: Pavimenti lisci, perfetti e a volte anche levigati si trovano solo nei dungeon creati da costruttori capaci e attenti.

\medskip

\begin{center}
	\includegraphics[width=0.9\linewidth]{immagini/pavimento_grey.png}
\end{center}

\medskip

\textbf{Pavimento di Pietra Naturale}: Il pavimento di una caverna naturale è irregolare quanto le pareti. E' difficile che queste caverne presentino ampie superfici piane; è più probabile che i loro pavimenti siano disposti su più livelli.

Alcune superfici potrebbero variare in elevazione di appena 30 centimetri, cosicché lo spostamento da un punto all'altro non sia più difficile del salire un gradino di una scala, ma in certi punti il pavimento potrebbe scendere o salire di oltre 1.5 metri, obbligando il personaggio a una prova di Arrampicarsi (pag \pageref{arrampicarsi}) per spostarsi da una superficie a un'altra.

A meno che non ci sia un percorso scavato dal tempo o ben battuto il terreno è considerato difficile e quindi il movimento è dimezzato, per praticità gradoni sotto i 50cm considerateli terreno difficile e quelli entro 1.5m terreno doppiamente difficile.\index{Gradini} La Carica e la corsa in questi ambienti sono impossibili, tranne che sui percorsi in questione.

\textbf{Scivoloso}: Acqua, ghiaccio, melma o sangue possono rendere qualunque pavimento descritto in questa sezione più insidioso. I pavimenti scivolosi aumentano la DC delle prove di Acrobatica di 5.

\textbf{Grata}: Una grata spesso copre una fossa o una zona al di sotto del pavimento principale. Le grate sono di solito costruite in ferro, ma quelle più grosse potrebbero essere anche fatte di tronchi d'albero rinforzati. Molte grate hanno cardini che permettono l'accesso alla zona sottostante (queste grate possono essere chiuse a chiave come una porta), mentre altre sono fisse e create per non poter essere spostate. Una tipica grata di ferro spessa 3 centimetri ha 25 Punti Ferita, Durezza 10, e DC 27 per sfondarla o smuoverla.

\textbf{Sporgenze}: Le sporgenze permettono alle creature di camminare al di sopra di un'area sottostante. Spesso sono disposte intorno a fosse, lungo il corso di fiumi sotterranei, come balconate che circondano un'ampia stanza oppure forniscono una posizione dalla quale gli arcieri possono appostarsi per attaccare i nemici dall'alto.

Le sporgenze strette (di ampiezza inferiore a 30 centimetri) richiedono a coloro che vi si muovono sopra, 3 Azioni di Movimento, delle prove di Acrobatica (DC 15). Un fallimento implica che il personaggio che si stava muovendo cade dalla sporgenza.

A volte le sporgenze hanno una ringhiera. In questi casi i personaggi ottengono bonus +1d6 alle prove di Acrobatica per muoversi lungo la sporgenza. Un personaggio vicino alla ringhiera ha Bonus +2 alla propria Prova Contrapposta di Forza per evitare di essere spinto giù dalla sporgenza.

\textbf{Pavimenti Trasparenti}: I pavimenti trasparenti, fatti di vetro rinforzato o di materiali magici permettono di osservare un ambiente pericoloso dall'alto. I pavimenti trasparenti sono di solito posti al di sopra di pozze di lava, arene, tane di mostri e stanze di tortura. Possono essere usati dai difensori per sorvegliare un'area.

\textbf{Pavimenti Scorrevoli}: Un pavimento scorrevole è un tipo di botola, creato per essere spostato e rivelare qualcosa che si trova al di sotto. In genere un pavimento scorrevole si muove tanto lentamente che chiunque vi si trovi sopra può evitare di cadere nell'apertura, purché abbia spazio per spostarsi. Se un pavimento di questo tipo scorre velocemente che c'è la possibilità che un personaggio cada in quello che si trova sotto di esso (lance acuminate, una vasca con olio bollente, o una pozza infestata da squali, acido...) allora si tratta come una trappola.

\textbf{Pavimenti Trappola}: Questi pavimenti sono stati progettati per diventare di colpo pericolosi. Con l'applicazione della giusta quantità di peso o l'azionamento di una leva nelle vicinanze, spuntoni sbucano dal pavimento, fiammate o sbuffi di vapore partono da fori nascosti, o l'intero pavimento si muove. Questi strani pavimenti si trovano di solito dentro alle arene, progettati per rendere i combattimenti più appassionanti e letali. Questo tipo di pavimento si gestisce come una trappola.

\subsection{Le porte}\index{Porte}

\noindent\begin{itemize}[leftmargin=*] \setlength{\itemsep}{0pt}
\item \textbf{Bloccata / Incastrata}: DC per Sfondare (TS Tempra con Forza, +1d6 se viene usato un piede di porco). Sfondare una porta a spallate/calci costa 1 Azione, 2 Azioni se si usa un piede di porco.\index{Azione Sfondare porte}\index{Azione Forzare porte}
\item \textbf{Chiusa a Chiave}: DC per Scassinare (prova di Disattivare Congegni).
\item \textbf{Non bloccata}: una porta non chiusa a chiave o bloccata richiede 1 Azione per aprirla oppure la si può aprire con l'Azione di Movimento usata per attraversarla.\index{Aprire una porta}
\end{itemize}

\medskip

Il \textbf{fallimento critico} in una prova di Forza (TS Tempra con Forza) significa essersi fatti male nella manovra di sfondamento. Finché non passano almeno 10 minuti non è più possibile sfondare una porta.\index{Fallire prova di forza sfondare porte}\index{Sfondare porte}

\index{Porte}Le porte all'interno dei dungeon sono ben più che semplici entrate o uscite. Spesso possono essere dei veri e propri incontri. Le porte dei dungeon si presentano in tre tipi basilari: di legno, di pietra e di ferro.

\end{multicols}

%\textbf{Tabella: Porte}
\index[Tabelle]{Tabella Porte}\index{Scassinare una porta}\label{tabellaporte}\hypertarget{tabellaporte}{}

\medskip

\noindent\begin{tabular}{llllll}
	\toprule
\rowcolor{gray!20}\textbf{Tipo di porta} & \textbf{Spessore tipico} & \textbf{Durezza} & \textbf{Punti Ferita} & \multicolumn{2}{c}{\textbf{DC per sfondare}} \\
&\textbf{(cm)}&&& \textbf{Bloccata} & \textbf{Chiusa a chiave}\\
\toprule
Legno semplice & 2.5& 5 & 10& 15 & 18\\
\rowcolor{gray!20}Legno buono& 3.75 & 5 & 15& 18 & 21\\
Legno robusto& 5& 5 & 20& 25 & 28\\
\rowcolor{gray!20}Pietra& 10 & 8 & 60& 31 & 34\\
Ferro & 5& 10& 60& 30 & 33\\
\rowcolor{gray!20}Saracinesca di legno & 7.5& 5 & 30& 27& 30\\
Saracinesca di ferro & 5& 10& 60& 28& 31\\
\rowcolor{gray!20}Serratura& -& 15& 30& -& -\\
Cardini & -& 10& 30& -& -
\end{tabular}

\medskip

\begin{multicols}{2}

\medskip

\textbf{Porte di Legno}\index{Porte di Legno}: Costruite con spesse assi inchiodate, a volte rinforzate con sbarre di ferro (poste anche per impedire le deformazioni prodotte dall'umidità dei dungeon), quelle di legno sono il tipo più comune di porta. Le porte di legno variano per durezza: possono essere semplici, buone o robuste. Le porte semplici (DC 15 per sfondarle) non sono progettate per tenere alla larga assalitori motivati.

Le porte di buona fattura (DC 18 per sfondarle), sebbene forti e resistenti, non sono comunque progettate per subire una grande quantità di danni. Le porte robuste (DC 25 per sfondarle) sono rivestite in ferro e sono delle barriere discretamente resistenti contro coloro che cerchino di oltrepassarle. Cardini di ferro sorreggono la porta e di solito un anello circolare posto al centro serve ad aprirla. A volte, al posto di un anello, una porta dispone di una sbarra di ferro su uno o entrambi i lati che funziona come maniglia.

Nei dungeon abitati queste porte sono di solito ben tenute (non bloccate) e non chiuse a chiave, anche se le zone importanti probabilmente saranno chiuse a chiave.

\textbf{Porte di Pietra}\index{Porte di Pietra}: Costruite da blocchi di pietra solida, queste porte pesanti e poco maneggevoli sono spesso pensate in modo da ruotare su se stesse quando vengono aperte, anche se i nani e altri abili artigiani sono in grado di costruire cardini forti abbastanza da sostenere il peso di una porta di pietra.

\begin{center}
	\includegraphics[width=0.9\linewidth]{immagini/porta_grey.png}
\end{center}

Le porte segrete nascoste lungo una parete di pietra sono solitamente di pietra. Altrimenti, le porte di questo tipo sono studiate per diventare resistenti barriere che proteggono qualsiasi cosa si trovi al di là di esse. Di conseguenza si trovano spesso chiuse a chiave o sbarrate.

\textbf{Porte di Ferro}\index{Porte di Ferro}: Arrugginite ma resistenti, le porte di ferro in un dungeon sono dotate di cardini come quelle di legno. Queste porte sono le porte più resistenti del tipo non magico. Sono di solito chiuse a chiave o sbarrate.

\textbf{Sfondare}\index{Sfondare porte}: Le porte dei dungeon possono essere chiuse a chiave, munite di trappole, rinforzate, sbarrate, sigillate magicamente o, a volte, semplicemente bloccate.

Tutti, ad eccezione dei personaggi più deboli, riusciranno a buttar giù una porta con un pesante attrezzo come un maglio, numerosi incantesimi ed oggetti magici possono offrire ai personaggi un modo facile per superare una porta chiusa.

\textbf{DC 13 o inferiore}: Una porta che chiunque può sfondare.

\textbf{DC 13--15}: Una porta che una persona forte dovrebbe sfondare con un solo tentativo, e che una persona di forza media potrebbe avere qualche speranza di abbattere in un solo colpo.

\textbf{DC 16--20}: Una porta che praticamente chiunque potrebbe sfondare, avendo a disposizione il tempo necessario.

\textbf{DC 21--25}: Una porta che solo una persona forte o molto forte ha una speranza di sfondare, e probabilmente non al primo tentativo.

\textbf{DC 26 o superiore}: Una porta che solo una persona dotata di una forza eccezionale può avere una qualche speranza di sfondare.

\textbf{Serrature}\index{Serratura}: Le porte dei dungeon sono spesso chiuse a chiave e così torna utile la competenza Disattivare Congegni. Le serrature sono incassate sul bordo opposto ai cardini o dritte nel centro della porta. Le serrature di solito controllano una sbarra di ferro o legno che si estende dalla porta dentro il muro che la sostiene.

I lucchetti fissano tra due anelli, uno sulla porta e uno sul muro. Serrature più complesse, come quelle a combinazione o quelle ad enigma, sono di solito costruite dentro la porta stessa.

La DC per scassinare una serratura con una prova di Disattivare Congegni spesso ricade tra 15 e 30, anche se esistono serrature con DC maggiori o inferiori. Una porta può disporre di più di una serratura, ognuna delle quali da aprire separatamente.\index{Scassinare una porta}. Scassinare serratura senza attrezzi da scasso comporta una penalità di -1d6 alla prova.\index{Scassinare senza attrezzi}\hypertarget{Attrezzi da scasso}{}

Un Fallimento Critico nell'apertura di un porta o lucchetto causa la rottura degli attrezzi da scasso.\index{Rompere gli attrezzi da scasso}\index{Fallire l'apertura di un lucchetto}

Le serrature sono spesso dotate di trappole, di solito aghi avvelenati che scattano all'infuori per pungere le dita del ladro.

\subsubsection{Porte, passaggi ed aperture}\index{Porte, passaggi ed aperture}

Una porta speciale potrebbe avere una serratura senza chiave, ma che richiede che venga indovinata la giusta combinazione delle leve vicine o vengano premuti nell'ordine corretto i simboli su un pannello per riuscire ad aprirla.

\textbf{Porte Bloccate}: I dungeon sono spesso luoghi umidi, e in alcuni casi le porte rimangono bloccate, in modo particolare se sono fatte di legno. Di solito si suppone che all'incirca 1 su 6 delle porte di legno e il 1 su 10 delle altre porte siano bloccate. Questi valori possono essere raddoppiati (al 2 su 6 e 2 su 10 rispettivamente) nel caso di dungeon da tempo abbandonati o trascurati.

\textbf{Porte Sbarrate}: Quando un personaggio cerca di sfondare una porta sbarrata, è la qualità della sbarra che fa la differenza, non il materiale della porta in sé. Sfondare una porta chiusa da una sbarra di legno richiede un Tiro Salvezza Tempra con Forza con DC 25, e la DC sale a 30 nel caso di una sbarra metallica.

I personaggi possono attaccare la porta e distruggerla, lasciando la sbarra appesa nel passaggio sgombro. Usare un piede di porco per forzare una porta incastrata/bloccata concede un +1d6 alla prova.\index{Piede di Porco su porta}

\textbf{Sigilli Magici}: Incantesimi messi su una porta possono rendere ostico l'attraversamento di una porta.

Una porta su cui è stato lanciato un blocco magico si considera chiusa anche se non ha fisicamente una serratura. E' necessario un incantesimo che scassina o dissolve le magie oppure una prova riuscita di sfondare per oltrepassare una porta chiusa in questo modo.

\textbf{Cardini}: La maggior parte delle porte è dotata di cardini. Ovviamente le porte scorrevoli non lo sono (queste sono piuttosto dotate di solchi sul pavimento, che permettono loro di scorrere a lato con facilità).

Gli avventurieri possono rimuovere i cardini uno alla volta superando varie prove di Disattivare Congegni (solo se, naturalmente, sono davanti al lato della porta su cui si trovano i cardini). Una simile azione ha una DC di 20, in quanto molti dei cardini sono arrugginiti o bloccati.

Spaccare un cardine è difficile. La maggior parte ha Durezza 10 e 30 Punti Ferita. La DC per spaccare un cardine è la stessa che serve per abbattere la porta

%\begin{center}
%\includegraphics[width=0.8\linewidth]{immagini/cardini.png}
%\end{center}

\textbf{Cardini a Inserimento}: Questi cardini sono molto più complessi e si trovano solo in zone di eccellente costruzione. Questi cardini sono costruiti dentro la parete e permettono alla porta di aprirsi in entrambe le direzioni. I personaggi non possono raggiungere i cardini per rimuoverli a meno che non sfondino il sostegno della porta o la parete. I cardini a inserimento si trovano di solito sulle porte di pietra, ma a volte si vedono anche su porte di legno o di ferro.

\textbf{Perni}: I perni non sono veri cardini, ma semplici pioli che si protendono dal lato superiore e inferiore della porta e si infilano dentro i buchi nel suo sostegno, permettendole di girare. I vantaggi dei perni è che non possono essere rimossi come i cardini e che sono facili da realizzare. Lo svantaggio è che siccome la porta gira sul suo centro di gravità (di solito nel mezzo), nulla più grosso di metà dell'ampiezza della porta vi può passare attraverso.

Le porte dotate di perni sono di solito di pietra e spesso anche abbastanza larghe per ovviare allo svantaggio. Un'altra soluzione è quella di piazzare il perno verso un'estremità e fare la porta più spessa da quella parte e più sottile dall'altra, in modo che si apra più o meno come una porta normale.

Le porte segrete all'interno di muri spesso ruotano, in quanto la mancanza di cardini rende più facile occultare la presenza della porta. I perni permettono anche a oggetti come una libreria di essere usati come porte segrete.

\textbf{Porte Segrete}: Camuffata da comune porzione di muro (o di pavimento o di soffitto), da libreria, da focolare, da fontana, una porta segreta porta ad un passaggio segreto oppure ad una stanza.

Qualcuno che stia esaminando la zona può trovare una porta segreta (se ne esiste una) con una prova riuscita di Consapevolezza (con DC 20 per una porta segreta comune e DC 30 per una porta molto ben nascosta).

Molte porte segrete richiedono un metodo speciale per essere aperte, come un bottone nascosto o una piastra a pressione. Le porte segrete possono aprirsi come porte comuni, girare su un perno, scorrere, sprofondare, sollevarsi o anche calare come un ponte levatoio.

Un costruttore potrebbe piazzare una porta segreta molto bassa vicino al pavimento oppure molto in alto su un muro, in modo da rendere più difficile sia il rinvenimento che l'utilizzo della porta.

\textbf{Porte Magiche} Incantata dal costruttore originario, una porta può apostrofare gli esploratori invitandoli a non proseguire. Potrebbe essere protetta dai danni, con una Durezza maggiore o un numero maggiore di Punti Ferita, oltre che un bonus al Tiro Salvezza migliorato. Una porta magica potrebbe non condurre allo spazio che si trova dietro di essa, ma essere in realtà un portale verso un luogo molto distante o addirittura verso un altro piano di esistenza. Altre porte magiche potrebbero aver bisogno di una parola d'ordine o di chiavi speciali per aprirsi.
Le porte magiche sono apribili solo tramite comando specifico o annullando la magia che le pervade, pochissime hanno una serratura.
In tal caso il Narratore potrebbe decidere di aumentare la prova di Disattivare Congegni di 10, portandola a 30 o più e potrebbe essere necessario avere qualche punto in Arcana.

\begin{center}
	\includegraphics[width=0.8\linewidth]{immagini/arcoserpenti.png}

	\emph{Henry Justice Ford}
\end{center}

\textbf{Saracinesche}: Queste porte speciali sono fatte con aste di ferro o di spesso legno rinforzato che calano da un recesso nella parte superiore di un arco. A volte una saracinesca dispone di barre orizzontali a formare una griglia, altre volte no. Sollevate di solito con un argano o simile macchinario, le saracinesche possono esser fatte scendere in fretta, e le sbarre terminano in punte per scoraggiare chiunque dal passarci sotto (o dal tentare di attraversarle in corsa mentre calano). Una volta scesa, una saracinesca si chiude, a meno che non sia così grande che nessuna persona normale sarebbe in grado di sollevarla. In ogni caso, sollevare una tipica saracinesca richiede un Tiro Salvezza Tempra con Forza con DC 25.

\textbf{Pareti, Porte ed azioni di Individuazione}

Le pareti di pietra, di ferro e le porte di ferro sono generalmente sufficientemente spessi da bloccare la maggior parte delle Divinazioni. Le pareti di legno, le porte di legno e di pietra in genere non sono sufficientemente spesse da fare altrettanto. Tuttavia, una porta segreta di pietra costruita in un muro e spessa come il muro stesso (almeno 30 centimetri) bloccherà la maggior parte di queste Azioni.

\subsection{Pericoli nei Dungeon}\index{Pericoli nei Dungeon}

Nei dungeon e nelle caverne oltre ai mostri ci sono anche altri pericoli tra crolli, muffe, funghi e altro.

\subsubsection{Crolli e Cedimenti (grado di Sfida 8)}\index{Crolli e Cedimenti}

I crolli e i cedimenti nei tunnel sono estremamente pericolosi. Non solo gli esploratori di dungeon corrono il rischio di essere schiacciati da tonnellate di pietra ma anche qualora dovessero sopravvivere, di rimanere bloccati sotto un mucchio di detriti o di essere impossibilitati a raggiungere un'uscita.

Un crollo seppellisce chiunque si trovi nel mezzo della zona sepolta e i detriti che rotolano via infliggeranno danni a tutti coloro che si trovano nelle zone attorno alla zona sepolta. Un tipico corridoio soggetto a un crollo potrebbe avere una zona sepolta con raggio 3 metri e una zona di scorrimento detriti con raggio di 1 metro all'estremità di quella sepolta.

Un soffitto pericolante può essere identificato con una prova di Conoscenza Ingegneria con DC 20 o Professione Muratore con DC 20.

Un soffitto pericolante può crollare sotto l'impatto di una grossa forza. Un personaggio può provocare un crollo distruggendo la metà dei pilastri che reggono il soffitto.

I personaggi che si trovano nella zona sepolta subiscono 8d6 danni o danni dimezzati se superano un Tiro Salvezza su Riflessi con DC 15 e sono sepolti. I personaggi nella zona ai bordi subiscono 3d6 danni o nessun danno se superano un Tiro Salvezza su Riflessi con DC 15. I personaggi che si trovano nelle zone ai bordi sono anch'essi sepolti per 1 quadretto se falliscono il Tiro Salvezza.

I personaggi sepolti subiscono 1d6 danni non letali per ogni minuto che rimangono sotto le macerie. Se un personaggio in queste condizioni cade privo di sensi, deve effettuare un Tiro Salvezza su Tempra con DC 15, se fallisce la prova, inizia a subire 1d6 danni letali al minuto fino a quando non viene liberato o muore.\index{Sepolti vivi}

I personaggi che non sono stati sepolti possono estrarre i loro compagni da sotto le macerie. In 1 minuto una creatura, usando solo le mani, libera un quarto di un quadretto di macerie, se usa degli strumenti adatti, come un piccone, vanga o una pala può scavare mezzo quadretto al minuto. Un personaggio sepolto può tentare di liberarsi da solo superando un Tiro Salvezza Tempra con Forza con DC 30, una volta al minuto.

\subsubsection{Fanghiglie, Muffe e Funghi}\index{Fanghiglie, Muffe e Funghi}

Negli umidi e oscuri recessi dei dungeon, le muffe e i funghi prosperano, temete le colonne di muffa! Per quanto riguarda gli incantesimi ed altri effetti speciali, tutte le fanghiglie, le muffe e i funghi sono considerati vegetali. Come le trappole, le fanghiglie e le muffe pericolose sono dotate di un grado di Sfida e i personaggi guadagnano Punti Esperienza per averle incontrate.

Una lucida melma organica ricopre qualsiasi cosa che rimanga per troppo tempo immersa nell'oscurità e nell'umidità dei dungeon. Questo tipo di fanghiglia, benché possa essere repellente, non è pericolosa. Le muffe e i funghi abbondano nei luoghi bui, freddi e umidi. Sebbene alcuni siano innocui quanto le normali fanghiglie dei dungeon, altri sono alquanto pericolosi. Funghi commestibili, vesce, lieviti, muffe e altri tipi di funghi fibrosi, bulbosi o intere distese di spore fungine possono essere rinvenuti nella maggior parte dei dungeon. Di solito sono innocui e spesso sono anche commestibili (anche se la maggior parte è poco invitante o ha uno strano sapore).

\begin{center}
	\includegraphics[width=0.95\linewidth]{immagini/funghi.png}

	\emph{Sono luminosi al buio, credetemi! e fritti sono ancora meglio!}
\end{center}

\textbf{Boleto Stridente}\index{Boleto Stridente}: Questi funghi viola di grandezza umana emettono un suono penetrante che dura 1d3 round ogni volta che c'è un movimento o una sorgente di luce entro raggio 3 metri. Questo grido rende impossibile sentire altri suoni o rumori entro raggio di mischia. Il suono attira le creature nelle vicinanze che sono disposte ad investigare. Alcune creature che vivono vicino ai boleti stridenti hanno imparato che il rumore significa molto spesso cibo.

\textbf{Fanghiglia Verde}\index{Fanghiglia Verde} (grado di Sfida 4): Questo pericolo dei dungeon è una varietà insidiosa della normale fanghiglia.
La fanghiglia verde divora la carne e i materiali organici che vi entrano in contatto ed è addirittura capace di dissolvere i metalli. Di un verde splendente, bagnata ed appiccicosa, si distribuisce a chiazze su pareti, pavimenti e soffitti e si riproduce consumando materiale organico. Si lascia cadere dalle pareti e dai soffitti quando individua del movimento (e possibile nutrimento) sotto di sé.

La fanghiglia verde infligge 1 danno alla Costituzione per ogni round in cui divora la carne. Al primo round di contatto, la fanghiglia può essere asportata da una creatura (con la probabile distruzione dell'oggetto utilizzato per asportarla), ma dopo il primo round deve essere congelata, bruciata o tagliata (infliggendo danni anche alla sua vittima) per essere rimossa. Tutto ciò che infligge danni da fuoco o da freddo, la luce solare o un incantesimo di rimuovi malattia distruggono una chiazza di fanghiglia verde. Nel caso di legno o metallo, la fanghiglia verde infligge 2d6 danni per round, ignorando la Durezza del metallo ma non quella del legno. Non danneggia la pietra. Difesa 10, Punti Ferita 30, Tiri Salvezza T 3, R 0, V 1.

\textbf{Fungo Fosforescente}\index{Fungo Fosforescente}: Questo strano fungo sotterraneo emana una debole luminescenza violacea che illumina le caverne e i passaggi sotterranei

come una candela. Rare macchie di questo fungo illuminano come una torcia. Strappato dal suo ambiente si spegne in un 1d4 turni.

\textbf{Muffa Gialla} \index{Muffa Gialla}(grado di Sfida 6): Se disturbata nel raggio di 3 metri rilascia una nube di spore velenose. Tutti coloro entro raggio di 3 metri dalla muffa devono superare un Tiro Salvezza su Tempra con DC 15 o subiscono 1d3 danni a Costituzione. Un altro Tiro Salvezza su Tempra con DC 15 è necessario una volta per round per i successivi 5 round o per evitare di subire altri 1d3 danni a Costituzione. Un Tiro Salvezza riuscito blocca questo effetto. Il fuoco distrugge la muffa gialla, mentre la luce solare la rende inerte. Difesa 10, Punti Ferita 25, Tiri Salvezza T 3, R 0, V 1, Vulnerabilità al Fuoco.

\textbf{Muffa Marrone} \index{Muffa Marrone}(grado di Sfida 2): La muffa marrone si nutre di calore, estraendolo da tutto ciò che la circonda. Di solito si presenta in chiazze con diametro di dimensione di 1 metro e la temperatura attorno alla muffa risulta sempre fredda in un raggio di 3 metri. Le creature viventi entro 1 metro da essa subiscono 3d6 danni non letali da freddo. Se viene portata una fonte di fuoco entro 1 metro dalla muffa questa raddoppia immediatamente le proprie dimensioni. I danni da freddo, come quelli inflitti da un cono di freddo, la distruggono all'istante. Difesa 10, Punti Ferita 12, Tiri Salvezza T 3, R 0, V 1, Vulnerabilità Freddo, converte i danni da fuoco subiti in Punti Ferita.

\subsubsection{Esempio di Trappole da dungeon}\index{Esempio di Trappole da dungeon}

Viene indicato il nome della trappola, la DC per la prova di Sopravvivenza per trovare la trappola e le indicazioni d'uso della stessa. Vedi anche \hyperlink{trappoleesempio}{Trappole} (pag. \pageref{trappoleesempio}).

\medskip

\textbf{Stanza allagata, DC 17}: se i personaggi non notano la piastra a pressione sul pavimento questa farà sigillare la porta di ingresso e la stanza incomincerà a riempirsi d'acqua.
La stanza si riempie d'acqua in 10 round. Una prova di Sopravvivenza a DC 15, combinata con una prova di Nuotare DC 13, fa rilevare la piastra che attiva la fuoriuscita d'acqua.

\textbf{Stanza stritolante, DC 15}: se i personaggi non notano la piastra a pressione sul pavimento questa farà sigillare la porta di ingresso e fortissimi rumori di stridii ed ingranaggi riempiranno la stanza. Le pareti incominceranno ad avvicinarsi tra loro come il soffitto al pavimento. Se i personaggi non trovano la mattonella nascosta (DC 17) subiranno 10d6 di danno da stritolamento. La trappola è più facile da rilevare di altre perché le pareti sono più spesse rendendo la stanza più piccola.

\textbf{Soffitto schiacciante, DC 18}: se i personaggi non notano il sistema di attivazione ( piastra a pressione, cavo, raggio di luce interrotto..) una sezione di soffitto di 3m x 3m cadrà sui i personaggi con un danno di 3d6.

\textbf{Tunnel di ragnatele, DC 12}: questo tunnel è evidentemente pieno di ragnatele fitte, dense, robuste. Se i personaggi entrano si considerano Intralciati. Dopo 1d4 round di permanenza un attivatore genererà una scintilla dando fuoco alle ragnatele per 1d4 round. Ogni round all'interno del tunnel si subiscono 2d4 di danno da fuoco.

\textbf{Fossa, DC 15}: il personaggio disattento farà crollare una sezione di 3m x 3m di pavimento su una fossa. Questa può essere una semplice fossa (1d6 di danno da caduta), con spuntoni (1d6+2d4), con acido (1d6 per round), con non morti...

\textbf{Garrotte, DC 14}: questa trappola può essere molto insidiosa. Un filo affilato magicamente è a 1 metro da terra, tra una parete e quella opposta e scorre verso i giocatori.
E' necessario un Tiro Salvezza su Riflessi DC 14 oppure subire 2d6 di danno da taglio.

\textbf{Porta schiacciante, DC 16}: questa porta appena toccata rotea su dei cardini centrali e roteando picchia il personaggio (o personaggi se un grande portone). Causa 1d6 di danni contundenti e continua a roteare per 1d6 round.

\textbf{Trincia Dito, DC 14}: questa trappola è molto subdola. Si presenta con un foro di circa 1 cm di diametro e profondo 7 cm. Qualsiasi cosa che ne tocchi il fondo farà scattare la trappola, causando 2d4 di danno al dito/oggetto inserito. La lama potrebbe anche essere avvelenata.

\end{multicols}

\vfill

\begin{center}
\includegraphics[width=0.35\linewidth]{immagini/GBP14.png}

\emph{Carceri d'invenzione, XIV / The Gothic Arch, Giovanni Battista Piranesi}
\end{center}

\pagebreak

\section{Pericoli in Avventura}\index{Pericoli in Avventura}

\begin{enfasi}{
Un'avventura è un risultato ragionevole. Due sono meglio, tre meritano di essere tramandate, e quattro... nessuno potrà mai contestare quattro avventure. (John Steinbeck)

\medskip

Corre meno pericoli colui che, anche se è al sicuro, sta in guardia. (Publilio Siro)} \end{enfasi}

\label{pericoli-in-avventura}

\begin{multicols}{2}

Il mondo è pieno dì pericoli oltre che di draghi ed immondi famelici. I pericoli sono minacce presenti nell'ambiente e che hanno molto in comune con le trappole, ma che di solito fanno parte del posto anziché venir costruite. I pericoli si dividono in tre categorie principali: ambientali, viventi e magici.

I pericoli ambientali includono frane, incendi e simili. I pericoli viventi includono creature che pur non essendo considerate mostri, rappresentano una minaccia per gli avventurieri \emph{incauti}, come fanghiglie, funghi e muschi. I pericoli magici sono i più imprevedibili e possono essere residui di esperimenti arcani, strane radiazioni sotterraneo o antichi incantesimi falliti.

\medskip

\begin{center}
\includegraphics[width=0.75\linewidth]{immagini/boscopericoli.png}
\end{center}

\textbf{Zona di Antimagia (grado di Sfida 6)}\index{Zone di Antimagia}\index{Antimagia}

Zona di entropia magica che distruggono le magie, le zone di Antimagia si formano sui siti di grandi duelli magici, attraverso la distruzione di potenti artefatti o da vortici di energia mistica ai margini delle zone di antimagia. Le dimensioni variano da piccole bolle di appena pochi metri fino a grandi aree delle dimensioni di una città.

Una prova riuscita di Arcana con DC 20 rivela la vicinanza di una Zona di Antimagia con un formicolio nell'aria. Una magia attiva portata in una zona antimagica potrebbe venir dissolta, qualsiasi incantesimo lanciato al suo interno è soggetta ad un contro incantesimo immediato. Se si ottiene un critico nella Prova di Magia questo riesce a passare il contro incantesimo ma non genera ulteriori effetti.

Se l'incantesimo fallisce il rilascio di energia magica infligge 2d6 danni da forza in un'esplosione in un raggio di 3 metri centrata su chi ha tentato l'incantesimo; un Tiro Salvezza su Riflessi a DC 15 permetti di dimezzare questo danno.

Una magia manifestata da un oggetto, che non sia un Artefatto, fallisce sempre.

Se più scoppi sovrapposti colpiscono lo stesso bersaglio, si applica solo quello più dannoso. Una magia che ha resistito ad un tentativo di dissoluzione, non viene influenzato nuovamente a meno che non esca e rientri dalla zona.

Le zone antimagiche più potenti sono ancora più distruttive. Ogni +1 di incremento del grado di Sfida aumenta di 1d6 il danno e la DC del Tiro Salvezza di 1.

\medskip
\textbf{Aria Viziata (grado di Sfida 1 o 4)}\index{Aria Viziata}

Le sacche di gas sono un rischio per minatori, speleologi e avventurieri che investigano nelle caverne. I gas ininfiammabili hanno grado di Sfida 1 e richiedono una prova di Sopravvivenza con DC 25 per essere notati. Le creature che respirano quell'aria devono superare un Tiro Salvezza su Tempra (DC 15 +1 per ogni tiro precedente) ogni ora o diventano Affaticati. Le creature che trattengono il fiato possono evitare questi effetti.

I vapori infiammabili sono molto più pericolosi (grado di Sfida 4). Questo gas sostituisce l'aria respirabile nei polmoni, provocando affaticamento: inoltre, qualsiasi fiamma aperta o scintilla causa un'esplosione che infligge 6d6 danni (TS su Riflessi con DC 15 dimezza) a chi è nella caverna o entro 3 metri da un ingresso. Il fuoco brucia l'ossigeno nell'aria, rendendola irrespirabile per 2d4 minuti. Dopo un'esplosione, il gas infiammabile generalmente impiega molti giorni per ritornare a livelli pericolosi.

\medskip
\textbf{Parassiti}\index{Parassiti}

Parassiti come \emph{cercaorecchie} o \emph{larve necrofaghe} provocano parassitosi, un tipo particolare di Malattia. Le parassitosi possono essere guarite solo attraverso trattamenti specifici; indipendentemente da quanti Tiri Salvezza si effettuano, la parassitosi continua ad affliggere il bersaglio. Anche se un Rimuovi Malattia (o un effetto simile) uccide immediatamente una parassitosi, l'immunità alle Malattie non offre protezione, dato che è causata da parassiti.

\medskip
\noindent\emph{Cercaorecchie (grado di Sfida 5)}\index{Cercaorecchie}

I cercaorecchie sono minuscoli vermi bianchi che vivono nel legno marcio o altri detriti organici. Si possono notare con una prova di Consapevolezza (DC 15). Altrimenti, una creatura vivente che frughi nella loro tana, si trasferisce inavvertitamente addosso uno o più cercaorecchie, i quali poi cercano una zona calda sul corpo della creatura, prediligendo il condotto uditivo, e li depongono 2d8 uova prima di morire.

Le uova si schiudono 4d6 ore dopo e le larve divorano la carne intorno. Alla morte del loro ospite, i vermetti strisciano fuori e ne cercano uno nuovo.

Rimuovi Malattia uccide tutti i cercaorecchie o le uova non ancora schiuse su un ospite. Alcuni cercaorecchie preferiscono vivere nel legno corrotto, spesso nascondendosi nelle porte dei sotterranei. I piccoli fori lasciati da questa variante sono molto difficili da notare (Consapevolezza DC 20).

\medskip
\textbf{Cercaorecchie}

Tipo: Parassitosi

TS: Tempra DC 15

Insorgenza: 4d6 ore

Frequenza Tiro Salvezza: 1 ogni ora

Effetti: 1d3 a Costituzione se fallisce il Tiro Salvezza

\medskip
\noindent\emph{Larve Necrofaghe (grado di Sfida 4)}\index{Larve Necrofaghe}

Una volta occupato un corpo vivente, le larve scavano verso il cuore, il cervello e altri organi interni chiave dell'ospite, provocandone infine la morte.

Nel primo round di parassitosi, applicando del fuoco nel foro di ingresso si possono uccidere le larve e salvare l'ospite, ma questo subisce 1d6 danni da fuoco.

Anche estrarle funziona, ma più a lungo le larve restano nell'ospite, più danni provoca questo metodo. Per estrarre le larve occorre un'arma tagliente ed una prova di Pronto Soccorso con DC 20, infliggendo 1d6 danni per ogni round che l'ospite è stato afflitto da parassitosi. Se la prova di Pronto Soccorso riesce una larva viene rimossa. Rimuovi Malattia uccide tutte le larve necrofaghe presenti in un ospite.

\medskip
\textbf{Larve Necrofaghe}

Tipo: Parassitosi

TS: Tempra DC 17

Insorgenza: immediata

Frequenza: 1/round

Effetti: 1 danno a Costituzione per larva

\medskip
\textbf{Cristalli magici (grado di Sfida 3)}\index{Cristalli magici}

I cristalli magici sono grandi (3-12 metri d'altezza) grappoli di cristalli di quarzo viola che irradiano un'aura di alterazione forte. Per identificarli occorre una prova di Arcana con DC 25.

I cristalli magici accumulano energia magica per crescere e difendersi. Un cristallo magici assorbe gli incantesimi lanciati nei 3 metri intorno a lui. L'incantatore deve effettuare una Prova di Magia con un Successo Critico per evitare l'effetto.

Danneggiando o rompendo i cristalli le magie assorbite vengono espulsi con un'esplosione di energia magica che infligge 1d4 danni per Punti Magia assorbiti (solitamente 10d4) a tutti coloro che si trovano entro 6 metri di raggio.

I cristalli magici sono molto fragili (Durezza 0, 4 Punti Ferita).
In aree ricche di cristalli, le creature che vi passano attraverso devono superare una prova di Acrobatica con DC 10 per evitare di camminarci sopra o sfiorarli rompendoli.

\medskip
\textbf{Magnete (grado di Sfida 2)}\index{Magnete}

Le strane energie del mondo sotterraneo possono caricare pietre e vene di minerali con potenti campi magnetici, creando un pericolo per chi porta o indossa metalli ferrosi. Tutti gli oggetti di ferro o acciaio portate entro raggio di 3 metri dal minerale sono trascinate verso di esso.

%\medskip

%\begin{center}
%\includegraphics[width=0.65\linewidth]{immagini/neodimio.png}

%\emph{Neodimio}
%\end{center}

Ogni creatura che abbia più di 4 di ingombro in metallo viene inesorabilmente attirato verso il minerale magnetico. E' concesso un Tiro Salvezza su Tempra con modificatore Forza a DC 25 per non avvicinarsi o riuscire a staccarsi dalla grossa calamita.

\medskip
\textbf{Pozzo Maledetto (grado di Sfida 3)}\index{Pozzo Maledetto}

Un pozzo maledetto attira gli avventurieri nelle sue profondità attraverso un illusione (TS su Volontà con DC 16 per non crederci) di uno meraviglioso tesoro sul fondo profondo solo 3 metri. Qualsiasi creatura che giunga al \emph{tesoro} attiva la maledizione.

Una creatura all'interno del pozzo deve superare un Tiro Salvezza su Volontà con DC 18 o è colpita dalla maledizione, che distorce la sua percezione del pozzo. L'acqua sembra addensarsi in un viscosa melma che spinge la creatura verso il fondo a 12 metri.

E richiesta una prova di Nuotare a DC 16 ogni round, il fallimento indica che si incomincia ad affogare.

Un pozzo maledetto irradia una forte magia, e può essere distrutta da Dissolvi Magie o da Rimuovi Maledizione.

\medskip
\textbf{Quercia Velenosa (grado di Sfida 1 o 3)}\index{Quercia Velenosa}

Il contatto con una quercia velenosa (grado di Sfida 1) causa una dolorosa, 1d4 Punti Ferita di danno, eruzione cutanea che rende la vittima Affaticata finché i danni non guariscono. Un pieno contatto col corpo o l'inalazione del fumo di una quercia velenosa che brucia potrebbero essere fatali (grado di Sfida 3) causando 2 gradi di Affaticato ed 1d8 di danno.
Una prova di Natura (o Erboristeria) con DC 15 rivela i pericoli insiti nella pianta. Questo pericolo può essere usato anche per piante nocive simili (edera velenosa, sommaco velenoso od ortiche pungenti..)

\textbf{Quercia Velenosa}

Tipo: Veleno, contatto

TS: Tempra DC 13

Insorgenza: 1 ora

Effetti: 1d4 danni, la creatura è affaticata finché i danni non guariscono

Cura: 1 TS

\subsection{Prepararsi per il riposo}\index{Prepararsi per il riposo}\index{Dormire}\index{Turni di guardia}

Ogni avventuriero deve riposarsi ogni tanto, lo deve fare con attenzione e stando attento a non incorrere in brutte e pericolose sorprese.

Ogni volta che un personaggio termina un periodo di 24 ore senza dormire almeno 8 ore, deve superare un Tiro Salvezza su Tempra con DC 17, altrimenti diventa Affaticato.

Ogni riposo mancato ulteriore lo renderà ancora più Affaticato cumulando le penalità relative. Se il personaggio resta sveglio per più giorni, lottare contro il sonno diventa più difficile. Dopo le prime 24 ore, la DC aumenta di 4 per ogni periodo consecutivo di 24 ore trascorso senza aver dormito 8 ore. La DC torna a 17 quando il personaggio completa un riposo di almeno 8 ore.

Dormire in armatura media o pesanti rende Affaticati, tranne se hai l'Abilità \hyperlink{secondapelle}{Seconda pelle}.

Non si riesce a dormire le 8 ore ad intervalli minori di 16 ore.\index{Dormire più volte al giorno}

Se il personaggio viene svegliato e coinvolto in una attività impegnativa come combattere, lanciare incantesimi, cavalcare... se questa si protrae per più di 10 minuti obbliga il personaggio a riprendere completamente il riposo.

\subsubsection{Organizzare i Turni di Guardia}

Se il gruppo è numeroso i turni di guardia per vegliare e controllare l'ambiente diventano più corti.

\medskip{}

\textbf{Tabella: Durata turni di guardia}\index[Tabelle]{Tabella Durata turni di guardia}

In questa tabella vengono indicate la durata dei turni di guardia ed il tempo totale di riposo del gruppo, nell'ipotesi di riposare almeno 8 ore.

\medskip{}

\noindent\begin{tabularx}{\linewidth}{XXX}
	\toprule
\rowcolor{gray!20}\textbf{Membri} &\textbf{Durata}&\textbf{Durata}\\
\textbf{del gruppo}&\textbf{del Turno}&\textbf{Totale}\\
\toprule
\rowcolor{gray!20}2& 8 h& 16 h\\
3& 4 h & 12 h\\
\rowcolor{gray!20}4& 2 h e 30 min. & 10 h e 30 min.\\
5& 2 h& 10 h\\
\rowcolor{gray!20}6& 1 h e 30 min. & 9 h e 30 min.
\end{tabularx}

\medskip{}

Un \textbf{rumore brusco} concede una prova di Consapevolezza a DC 15, oppure pari alla prova di Furtività +8 dell'avversario, per svegliarsi.\index{Svegliarsi per rumore}\index{Rumore nella notte}

\end{multicols}

\vfill

\begin{center}
\includegraphics[width=0.9\linewidth]{immagini/mappaparigi.png}

\emph{Antica mappa di Parigi}
\end{center}

\pagebreak

\subsection{Avventure e Trappole}\index{Trappole}\label{trappole}

\begin{enfasi}{
Chi pone la trappola sempre allo stesso posto non prenderà alcun'iguana. (Proverbio Africano)}\end{enfasi}

\begin{multicols}{2}

Quasi ovunque si può incontrare una trappola. Le trappole possono essere di natura magica o meccanica. Le trappole meccaniche comprendono fosse, frecce, massi che cadono, stanze piene d'acqua, lame rotanti e qualsiasi altra cosa che dipenda da un meccanismo per operare. Le trappole magiche sono congegni magici trappola o incantesimi trappola. I congegni magici trappola quando attivati generano gli effetti di un incantesimo, mentre gli incantesimi trappola sono incantesimi come glifo di interdizione e simbolo che funzionano come trappole.

Quando gli avventurieri si imbattono in una trappola, dovreste sapere come la trappola si attiva e cosa faccia, oltre ad avere un'idea di come i personaggi possano individuare la trappola e riuscire a disarmarla o evitarla.

\subsubsection{Attivare una Trappola}
La maggior parte delle trappole si attivano quando una creatura giunge in un punto o tocca qualcosa che il creatore della trappola voleva proteggere. Normali sistemi di attivazione sono pedane a pressione o false sezioni di pavimento, tirare un cavo, girare una maniglia e usare la chiave sbagliata nella serratura. Le trappole magiche spesso si attivano quando una creatura entra in un'area o tocca un oggetto.

Alcune trappole magiche (come l'incantesimo glifo di interdizione) possiedono delle condizioni di attivazione più complesse, tra cui l'impiego di parole d'ordine per impedire l'attivazione della trappola.

Una volta attivata la trappola esegue l'effetto indicato: il personaggio cade nella fossa, parte il dardo avvelenato, viene attivato l'incantesimo collegato. Al personaggio che subisce l'attivazione della trappola è concesso un Tiro Salvezza o ulteriore prova solo se specificato nella descrizione della trappola stessa.

\subsubsection{Individuare e Disabilitare una Trappola}
Di solito, alcuni elementi di una trappola sono ben visibili a un'attenta ispezione.

La descrizione della trappola specifica le prove e le DC necessarie per individuarla, disabilitarla o entrambe. Un personaggio che cerchi attivamente una trappola può tentare una prova di \textbf{Sopravvivenza} contro la DC della trappola.

Il Narratore può anche comparare la DC per individuare la trappola contro il punteggio di Sopravvivenza (a tiro dadi 8) dei personaggi al fine di determinare se un membro del gruppo noti la trappola. Se gli avventurieri notano la trappola prima di attivarla, potrebbero tentare di disarmarla, in maniera permanente o abbastanza a lungo da permettergli il passaggio.

Il Narratore potrebbe richiedere una prova di Disattivare Congegni. Se non si hanno \textbf{attrezzi da scasso}\index{Attrezzi da scasso} o adeguati la prova la fai con un -1d6 di penalità. \index{Disattivare congegni senza attrezzi}Può essere usata anche la competenza Sopravvivenza seppure con un -1d6 per disattivare una trappola, lucchetto..., in questo caso la durata dell'operazione è pari ad 1 Azione per DC della trappola.

Se si vuole disattivare temporaneamente\index{Trappola disattivare temporaneamente} una trappola aggiungete 6 alla difficoltà. Questo disattiverà la trappola per 2d4 minuti.

Una \textbf{trappola magica può essere disattivata} con una prova di Disattivare Congegni purché il valore di Arcana sia almeno 1/5 della DC della trappola in aggiunta a qualsiasi altra prova indicata nella descrizione della trappola. La prova di Arcana può essere fatta anche da chi non ha Disattivare Congegni ma con la difficoltà segnata nella trappola ed insieme a chi fa la prova di Disattivare Congegni. L'incantesimo Dissolvi Magie ha una probabilità di annullare la maggior parte delle trappole magiche.\index{Disattivare trapple magiche}. Un Dissolvi Magie può annullare la parte magica di trappola la cui DC di Arcana richiesta sia 3 o o inferiore, se richiede 4 allora la componente magica viene disattivata per 10 minuti. Un Dissolvi Magia Superiore può annullare la parte magica di trappola la cui DC di Arcana richiesta sia 4 o o inferiore, se richiede 5 allora la componente magica viene disattivata per 10 minuti. Per ogni Critico Magico nella Prova di Magia si alza di uno la DC di Arcana richiesta.

Se la prova per disattivare o disabilitare la trappola fallisce\index{Fallimento disattivare trappola} e si ottiene un Fallimento Critico la trappola scatta.

Di solito, la descrizione della trappola è abbastanza chiara da permettere al Narratore di giudicare se le azioni di un personaggio riescano a individuare o disattivare la trappola.

Usate il buon senso e basatevi sulla descrizione della trappola per determinare cosa accade. Nessun progetto di trappola può prevedere ogni possibile azione che i personaggi potrebbero tentare.

Il Narratore dovrebbe consentire a un personaggio di scoprire una trappola senza dover effettuare prove di competenza, se le sue azioni o la descrizione di ciò che fa rivelano chiaramente la presenza della trappola.

Disattivare le trappole può essere un pò più complicato. Prendiamo, ad esempio, un forziere protetto da una trappola. Se il forziere viene aperto senza tirare le due maniglie laterali, un meccanismo interno spara una raffica di aghi avvelenati verso chiunque si trovi di fronte.

Dopo aver ispezionato il forziere e fatto qualche prova, i personaggi non sono ancora sicuri che sia trappolato. Invece di aprirlo direttamente, puntano uno scudo davanti al forziere e lo aprono a distanza con un'asta di ferro. In questo caso, la trappola si attiva, ma la raffica di aghi colpisce lo scudo senza ferire nessuno.

Le trappole sono spesso progettate con meccanismi che permettono di disattivarle o aggirarle.

\medskip

\begin{center}
\includegraphics[width=0.7\linewidth]{immagini/medusa.png}
\end{center}

\subsubsection{Effetti delle Trappole}
Gli effetti delle trappole possono essere da semplici inconvenienti a letali. La descrizione di una trappola specifica cosa accade quando viene attivata.
Il bonus di attacco di una trappola, la DC del Tiro Salvezza per resistere ai suoi effetti, e il danno che infligge possono variare in base alla pericolosità della trappola.

Usare la tabella DC dei Tiri Salvezza e Bonus di Attacco delle Trappole e la tabella Gravità del Danno per Livello come suggerimenti sui tre livelli di gravità delle trappole.

\medskip

\textbf{Tabella: DC dei Tiri Salvezza e Bonus di Attacco delle Trappole}\index[Tabelle]{Tabella DC dei Tiri Salvezza e Bonus di Attacco delle Trappole}

\medskip

\noindent\begin{tabularx}{\linewidth}{X|X|X}
	\toprule
\rowcolor{gray!20}Pericolosità della Trappola&DC Tiro Salvezza& Bonus di Attacco\\
\toprule
Minima&13-14&+4 a +6\\
\rowcolor{gray!20}Pericolosa&16-20&+8 a +10\\
Mortale&21-26&+12 a +15
\end{tabularx}

\medskip

\textbf{Tabella: Gravità del Danno per Livello}\index[Tabelle]{Tabella Gravità del Danno per Livello}

\medskip

\noindent\begin{tabularx}{\linewidth}{X|X|X|X}
	\toprule
\rowcolor{gray!20}Livello PG&Minima&Pericolosa&Mortale\\
\toprule
1°-4°&1d10&2d10&4d10\\
\rowcolor{gray!20}5°-10°&2d10&4d10&10d10\\
11°-16°&4d10&10d10&18d10\\
\rowcolor{gray!20}17°-20°&10d10&18d10&24d10
\end{tabularx}

\medskip

\subsubsection{Trappole di Esempio}\hypertarget{trappoleesempio}{}\label{trappoleesempio}
\textbf{Ago Avvelenato}

Trappola meccanica

Un ago avvelenato è nascosto all'interno della serratura di un forziere, o altro oggetto che si possa aprire. Aprire il forziere senza la chiave adeguata farebbe scattare l'ago, che dispensa una dose di veleno.

Quando la trappola viene attivata, l'ago si estende per 7 centimetri dalla serratura. Una creatura a gittata subisce 1 danno perforante e 11 (2d10) danni da veleno, e deve superare un Tiro Salvezza su Tempra con DC 20 o prendere -1d6 al Tiro per Colpire e -1d6 alle prove di Competenza di Base per 1 ora.

Il personaggio che superi una prova di Sopravvivenza con DC 22, può dedurre la presenza della trappola dalle modifiche apportate alla serratura per ospitare l'ago. Una prova superata di Disattivare Congegni disarma la trappola rimuovendo l'ago dalla serratura. \textbf{Una prova fallita per scassinare la serratura fa scattare la trappola}\index{Trappola, fallire prova}. Dichiarare di incastrare un bastone nella serratura è altrettanto efficace nel disattivare la trappola.

\medskip

\textbf{Dardi Avvelenati}

Trappola meccanica

Quando una creatura calpesta una pedana a pressione nascosta, dei dardi avvelenati vengono sparati da un meccanismo a molla o da tubi pressurizzati astutamente nascosti all'interno delle pareti circostanti. Un'area potrebbe presentare più pedane a pressione, ciascuna collegata alla propria serie di dardi.

I minuscoli fori nelle pareti sono celati da polvere e ragnatele, oppure astutamente celati tra i bassorilievi, murali o affreschi che adornano la stanza. La DC della prova per notarli (Sopravvivenza) è 18.

Il personaggio che superi una prova di Sopravvivenza con DC 18, può dedurre la presenza della pedana a pressione nascosta dalle differenze nella pavimentazione di cui è composta rispetto al resto del pavimento.

Incuneare una punta di ferro o altro oggetto sotto la pedana a pressione previene l'attivazione della trappola. Riempire i fori di tessuto o cera impedisce la fuoriuscita dei dardi contenuti all'interno.

La trappola si attiva quando più di 10 chili di peso vengono posti sulla pedana a pressione, facendo così sparare quattro dardi. Ogni dardo effettua un attacco a distanza con un bonus di attacco +10 contro un bersaglio casuale entro 3 metri dalla pedana a pressione (la visuale non ha alcun impatto su questo tiro di attacco).

Se non ci sono bersagli nell'area, il dardo non colpisce nulla. Un bersaglio colpito subisce 2 (1d4) danni perforanti e deve effettuare un Tiro Salvezza su Tempra con DC 18 e subire 11 (2d10) danni da veleno se lo fallisce, o la metà di questi danni se lo supera.

\medskip

\textbf{Fosse}

Trappola meccanica

Presentiamo di seguito quattro tipi base di fosse.

\medskip

\emph{Fossa Semplice}

La fossa semplice è un buco scavato nel terreno. Il buco è coperto da un grosso tessuto ancorato ai margini della fossa e mimetizzato con terra e detriti.
La DC per notare la fossa è 14. Chiunque metta piede sul tessuto cade all'interno della buca e si tira dietro il tessuto, subendo danni in base alla profondità della fossa (di solito 3 metri, ma alcune fosse sono più profonde).

\medskip

\emph{Fossa Nascosta}

Questa fossa possiede una copertura fatta di materiale identico a quello del pavimento circostante.
Superando una prova di Consapevolezza con DC 18 si nota l'assenza di tracce nella sezione di pavimento che forma la copertura della fossa.

È necessario superare una prova di Sopravvivenza con DC 18 per confermare che quella sezione di pavimento copra in realtà una fossa.

Quando una creatura mette piede sulla copertura, questa si spalanca come una botola, facendo precipitare l'intruso nella fossa sottostante. La fossa è profonda solitamente tra i 3 e i 6 metri, ma può esserlo anche di più.

Una volta che la fossa è stata individuata, uno spuntone di ferro o simile oggetto può essere conficcato tra la copertura della fossa e il terreno circostante per impedire che la copertura si apra, rendendo sicuro il passaggio. La copertura può anche essere tenuta chiusa magicamente tramite l'incantesimo Serratura Magica o magie simili.

\medskip
\emph{Fossa a Scatto}

Questa fossa è identica alla trappola fossa nascosta, con un'eccezione fondamentale: la botola che copre la fossa nasconde un meccanismo a molla. Dopo che una creatura è caduta nella fossa, la copertura si richiude di scatto per intrappolare la vittima al suo interno.

È necessario superare un Tiro Salvezza Tempra con Forza DC 20 per aprire a forza la copertura. La copertura può essere anche distrutta. Un personaggio all'interno della fossa può anche tentare di disabilitare il meccanismo a molla dall'interno superando una prova di Disattivare Congegni con DC 18 purché possa raggiungere e vedere il meccanismo in questione. In alcuni casi, un altro meccanismo fa riaprire la fossa.

\medskip

\emph{Fossa con Spuntoni}

La fossa è una fossa semplice, nascosta o a scatto, sul cui fondo si trovano delle punte di legno o degli spuntoni di ferro. Una creatura che caschi nella fossa subisce 11 (2d10) danni perforanti dagli spuntoni, oltre al danno da caduta.

Versioni più crudeli di questa trappola sono munite di veleno cosparso sulle punte collocate in fondo alla fossa. In quel caso, chiunque subisca danni perforanti dagli spuntoni deve anche effettuare un Tiro Salvezza su Tempra con DC 16 e subire 22 (4d10) danni da veleno se lo fallisce, o la metà di questi danni se lo supera.

\medskip

\textbf{Rete che Casca}

Trappola meccanica

Questa trappola usa un cavetto per liberare una rete appesa al soffitto.

Il cavetto è collocato a 7 centimetri dal terreno e si estende tra due colonne o alberi. La rete è nascosta da ragnatele o fogliame. La DC (Sopravvivenza) per notare il cavetto e la rete è 15. Una prova superata di Disattivare Congegni con DC 20 disabilita il cavetto.

Un personaggio privo degli attrezzi da scasso può tentare comunque la prova con -1d6 usando un'arma o un attrezzo affilati. Se la prova fallisce, la trappola si attiva.

Quando la trappola viene attivata, la rete viene rilasciata coprendo un'area quadrata di 3 metri di lato. Tutte le creature nell'area vengono intrappolate dalla rete e sono intralciate, mentre quelle che falliscono un Tiro Salvezza su Tempra, con modificatore Forza, con DC 13 cadono anche prone.

Una creatura può usare 2 Azioni per effettuare un Tiro Salvezza Tempra con Forza DC 13, liberando se stessa o un'altra creatura a portata se la supera.

La rete ha Difesa 10 e 20 Punti Ferita. infliggere 5 danni taglienti alla rete ne distrugge una sezione quadrata di 1 metro di lato, liberando qualsiasi creatura intrappolata in quella sezione.

\medskip

\textbf{Sfera Rotolante}

Trappola meccanica

Quando 10 o più chili vengono posti sulla pedana a pressione della trappola, una botola nascosta nel soffitto si apre, rilasciando una sfera di 3 metri di diametro interamente fatta di pietra.

Superando una prova di Sopravvivenza con DC 20 un personaggio può notare la botola e la pedana a pressione. Se un esame del pavimento è accompagnato da una prova superata di Sopravvivenza con DC 20, rivelerà la presenza della pedana a pressione tramite la differenza di struttura della pavimentazione che la accomoda. La stessa prova effettuata mentre si controlla il soffitto, rivelerà la presenza di una botola. Incuneare uno spuntone di ferro o un altro oggetto sotto la pedana a pressione impedirà l'attivazione della trappola.

L'attivazione della sfera fa sì che tutte le creature presenti tirino per l'iniziativa. La sfera tira l'iniziativa con un bonus di +8.

Durante il suo round, la sfera si muove di 18 metri in linea retta. La sfera può muoversi attraverso lo spazio di una creatura, e le creature possono muoversi attraverso lo spazio che occupa, considerandolo terreno difficile.

Ogni qualvolta la sfera entri nello spazio di una creatura o una creatura entri nel suo spazio mentre la sfera sta rotolando, la creatura deve superare un Tiro Salvezza su Riflessi con DC 15 o subire 55 (10d10) danni contundenti e cadere prona.

La sfera si ferma quando colpisce un muro o una barriera simile. Non può girare gli angoli, ma gli abili costruttori di sotterranei incorporano lievi curve e svolte curvilinee nei passaggi limitrofi che permettono alla sfera di continuare a muoversi.

Con 2 Azioni, una creatura entro 1 metro dalla sfera può tentare di rallentarla superando un Tiro Salvezza Tempra con Forza DC 20. Se la prova viene superata, la velocità della sfera viene ridotta di 3 metri. Se la velocità della sfera scende a 0, arresta il movimento e non è più una minaccia.

\medskip

\textbf{Soffitto che Crolla}

Trappola meccanica

Questa trappola usa un cavetto per fare crollare i sostegni che sorreggono una sezione instabile di soffitto.

Il cavetto è collocato a 7 centimetri dal terreno e si estende tra i due sostegni. La DC (Sopravvivenza) per notare il cavetto è 13. Una prova superata di Disattivare Congegni con DC 20 disabilita il cavetto.

Un personaggio privo degli attrezzi da scasso può tentare comunque la prova con -1d6 usando un'arma o un attrezzo affilati. Se la prova fallisce, la trappola si attiva.

Chiunque ispezioni i sostegni può facilmente dedurre che sono solo appoggiati. Con un'Azione, il personaggio può far cadere un sostegno e attivare la trappola.

Il soffitto sopra il cavetto è in cattivo stato, e chiunque possa vederlo può capire che rischia di crollare. Quando la trappola viene attivata, il soffitto instabile crolla. Tutte le creature nell'area sotto la sezione instabile devono effettuare un Tiro Salvezza su Riflessi con DC 20, subendo 22 (4d10) danni contundenti se lo falliscono o la metà di questi danni se lo superano. Una volta attivata la trappola, il pavimento dell'area è pieno di macerie e diventa terreno difficile.

\medskip

\textbf{Statua Soffia Fuoco}

Trappola magica

Questa trappola si attiva quando un intruso calpesta una pedana a pressione nascosta, liberando una vampata di fiamme magiche da una statua vicina.

La DC (Sopravvivenza) per notare la pedana a pressione o segni di bruciature sul pavimento e le pareti è 20. Un incantesimo o altro effetto che può percepire la presenza di magia, come individuazione del magico, rivela un'aura magica di invocazione intorno alla statua.

La trappola si attiva quando più di 10 chili di peso vengono posti sulla pedana a pressione, facendo sì che dalla statua scaturisca un cono di fuoco di 9 metri. Tutte le creature nel cono devono effettuare un Tiro Salvezza su Riflessi con DC 17, subendo 22 (4d10) danni da fuoco se lo falliscono o la metà di questi danni se lo superano.

Infilare uno spuntone di ferro o altro oggetto sotto la pedana a pressione impedisce alla trappola di attivarsi. Una prova di Disattivare Congegni a DC 20 (ed è necessario avere 3 in Arcana) disattiva la trappola. Un dissolvi magie (DC 17) lanciato sulla statua distrugge la trappola.

\medskip

\textbf{Trappole ad incantesimo e Dissolvi Magia}\index{Trappole ad incantesimo e Dissolvi Magia}

Le trappole di cui sopra possono essere dotate di un incantesimo che si attiva con la trappola.
I Tiri Salvezza per resistere all'incantesimo sono i medesimi dell'incantesimo lanciato da oggetto o come indicato nella descrizione della trappola.

Un Dissolvi Magia cancella l'incantesimo sulla trappola se questa ha la competenza Arcana richiesta a 3 o meno e ne disabilita l'effetto magico per 10 minuti se Arcana è 4.
Un Dissolvi Magie Avanzato cancella l'incantesimo sulla trappola se questa richiede Arcana 5 o meno e ne disabilita l'effetto magico per 10 minuti se richiede Arcana 6. In caso di Critico Magico nel lancio dell'incantesimo si agisce su un grado maggiore di trappola.

\subsubsection{Altri esempi di trappole}

Sono qui presentate ulteriori trappole per la vostra gioia.

\medskip

\textbf{Piccola legenda}:

\textbf{Grado di Sfida (GS)}: indica quale è il grado di sfida della trappola

\textbf{Tipo}: se la trappola è di tipo Meccanico (Mec.) o Magico (Mag.)

\textbf{DC Sopravvivenza (SOP)}: quale è la prova e difficoltà per rivelare la trappola

\textbf{DC Disattivare Congegni (DIS)}: quale è la prova e difficoltà per disattivare la trappola. Il punteggio dopo la sbarra (es DC 26/6) indica il requisito minimo di conoscenza Arcana per disattivarla.

\textbf{Attivatore}: se si attiva a contatto o distanza o tramite un incantesimo come Allarme (per GS < 3), Occhio arcano (per GS tra 3 e 10) o Visione del Vero (per GS oltre 10).

\textbf{Ripristino (Ripr.)}: se è possibile ripristinare la trappola una volta scattata

\textbf{Effetto}: quale è l'effetto della trappola

\bigskip



\noindent\begin{tabularx}{\linewidth}{p{2.3cm}X}
\rowcolor{gray!20}\textbf{GS:} 1&\textbf{Dardo Avvelenato} \\
\textbf{Tipo:} & Meccanico \\
\rowcolor{gray!20}\textbf{Sopravviv.:} & DC 13 \\
\textbf{Dis. Cong.:} & DC 15 \\
\rowcolor{gray!20}\textbf{Attivatore:} & Contatto \\
\textbf{Ripristino:} & Nessuno \\
\rowcolor{gray!20}\textbf{Effetto:} & Attacco a distanza 12 metri +10 (1d3 di danno più \hyperlink{bavadilucos}{Bava fermentata di Lucos}, pag \pageref{bavadilucos})
\end{tabularx}\\

\medskip

\noindent\begin{tabularx}{\linewidth}{p{2.3cm}X}
 \rowcolor{gray!20}\textbf{GS:} 1&\textbf{Freccia} \\
	\textbf{Tipo:} & Meccanico \\
 \rowcolor{gray!20}\textbf{Sopravviv.:} & DC 13 \\
	\textbf{Dis. Cong.:} & DC 15 \\
 \rowcolor{gray!20}\textbf{Attivatore:} & Contatto \\
	\textbf{Ripristino:} & Nessuno \\
 \rowcolor{gray!20}\textbf{Effetto:} & Attacco a distanza 12 metri +15 (1d8+3)
\end{tabularx}\\

\medskip

\noindent\begin{tabularx}{\linewidth}{p{2.3cm}X}
 \rowcolor{gray!20}\textbf{GS:} 1&\textbf{Fossa} \\
	\textbf{Tipo:} & meccanico \\
 \rowcolor{gray!20}\textbf{Sopravviv.:} & DC 14 \\
	\textbf{Dis. Cong.:} & DC 16 \\
 \rowcolor{gray!20}\textbf{Attivatore:} & posizione \\
	\textbf{Ripristino:} & manuale \\
 \rowcolor{gray!20}\textbf{Effetto:} & fossa profonda 3 metri (2d6 danni da caduta)
\end{tabularx}\\

\medskip

\noindent\begin{tabularx}{\linewidth}{p{2.3cm}X}
 \rowcolor{gray!20}\textbf{GS:} 1 & \textbf{Trancia Dita}\\
	\textbf{Tipo:} & meccanico \\
 \rowcolor{gray!20}	\textbf{Sopravviv.:} & DC 17 \\
	\textbf{Dis. Cong.:} & DC 14 \\
 \rowcolor{gray!20}	\textbf{Attivatore:} & posizione \\
	\textbf{Ripristino:} & manuale \\
 \rowcolor{gray!20}	\textbf{Effetto:} & trancia la prima falange (1d8+1)
\end{tabularx}\\

\medskip

\noindent\begin{tabularx}{\linewidth}{p{2.3cm}X}
\rowcolor{gray!20}\textbf{GS:}  1  & \textbf{Lama Falciante}\\
	\textbf{Tipo:} & meccanico \\
 \rowcolor{gray!20}\textbf{Sopravviv.:} & DC 13 \\
	\textbf{Dis. Cong.:} & DC 15 \\
 \rowcolor{gray!20}\textbf{Attivatore:} & posizione \\
	\textbf{Ripristino:} & manuale \\
 \rowcolor{gray!20}\textbf{Effetto:} & 2 attacchi in mischia +10 (1d8+1)
\end{tabularx}\\

\noindent\begin{tabularx}{\linewidth}{p{2.3cm}X}
\rowcolor{gray!20} \textbf{GS:} 2 & \textbf{Fossa con Spuntoni}\\
	\textbf{Tipo:} & meccanico \\
 \rowcolor{gray!20}\textbf{Sopravviv.:} & DC 16 \\
	\textbf{Dis. Cong.:} & DC 18 \\
 \rowcolor{gray!20}\textbf{Attivatore:} & posizione \\
	\textbf{Ripristino:} & manuale \\
 \rowcolor{gray!20}\textbf{Effetto:} & fossa profonda 3 metri + spuntoni (Attacco in mischia +10, 1d4 spuntoni per bersaglio per 1d4+2 danni ciascuno)
\end{tabularx}\\

\medskip

\noindent\begin{tabularx}{\linewidth}{p{2.3cm}X}
\rowcolor{gray!20}  \textbf{GS:} 2& \textbf{Onda rovente} \\
	\textbf{Tipo:} & magico \\
 \rowcolor{gray!20}\textbf{Sopravviv.:} & DC 16 \\
	\textbf{Dis. Cong.:} & DC 18/4 \\
 \rowcolor{gray!20}\textbf{Attivatore:} & prossimità (Allarme) \\
	\textbf{Ripristino:} & nessuno \\
 \rowcolor{gray!20}\textbf{Effetto:} & 5d4 danni da fuoco
\end{tabularx}\\

\medskip

\noindent\begin{tabularx}{\linewidth}{p{2.3cm}X}
 \rowcolor{gray!20}\textbf{GS:} &2\textbf{Giavellotto} \\
	\textbf{Tipo:} & meccanico \\
 \rowcolor{gray!20}\textbf{Sopravviv.:} & DC 15 \\
	\textbf{Dis. Cong.:} & DC 17 \\
 \rowcolor{gray!20}\textbf{Attivatore:} & posizione \\
	\textbf{Ripristino:} & nessuno \\
 \rowcolor{gray!20}\textbf{Effetto:} & Attacco a distanza 12 metri +15 (2d6+6), entro raggio 6 metri
\end{tabularx}\\

\medskip

\noindent\begin{tabularx}{\linewidth}{p{2.3cm}X}
 \rowcolor{gray!20}\textbf{GS:} 2&\textbf{Fossa con non morti} \\
	\textbf{Tipo:} & meccanico \\
 \rowcolor{gray!20}\textbf{Sopravviv.:} & DC 14 \\
	\textbf{Dis. Cong.:} & DC 16 \\
 \rowcolor{gray!20}\textbf{Attivatore:} & posizione \\
	\textbf{Ripristino:} & nessuno \\
 \rowcolor{gray!20}\textbf{Effetto:} & Sul fondo della fossa (2d6 di danno da caduta) sono presenti due zombi
\end{tabularx}\\

\medskip

\noindent\begin{tabularx}{\linewidth}{p{2.3cm}X}
 \rowcolor{gray!20}\textbf{GS:}  3&\textbf{Freccia Acida} \\
	\textbf{Tipo:} & magico \\
 \rowcolor{gray!20}\textbf{Sopravviv.:} & DC 18 \\
	\textbf{Dis. Cong.:} & DC 20/4 \\
 \rowcolor{gray!20}\textbf{Attivatore:} & prossimità (Allarme) \\
	\textbf{Ripristino:} & nessuno \\
 \rowcolor{gray!20}\textbf{Effetto:} & Attacco distanza di 16 metri (4d4 danni da acido per 4 round)
\end{tabularx}\\

\medskip

\noindent\begin{tabularx}{\linewidth}{p{2.3cm}X}
 \rowcolor{gray!20}\textbf{GS:} 3&\textbf{Fossa Celata} \\
	\textbf{Tipo:} & meccanico \\
 \rowcolor{gray!20}\textbf{Sopravviv.:} & DC 19 \\
	\textbf{Dis. Cong.:} & DC 20 \\
 \rowcolor{gray!20}\textbf{Attivatore:} & posizione \\
	\textbf{Ripristino:} & manuale \\
 \rowcolor{gray!20}\textbf{Effetto:} & 3d6 danni da caduta
\end{tabularx}\\

\medskip

\noindent\begin{tabularx}{\linewidth}{p{2.3cm}X}
 \rowcolor{gray!20}\textbf{GS:} 4&\textbf{Arco Elettrico} \\
	\textbf{Tipo:} & magico \\
 \rowcolor{gray!20}\textbf{Sopravviv.:} & DC 21 \\
	\textbf{Dis. Cong.:} & DC 20/4 \\
 \rowcolor{gray!20}\textbf{Attivatore:} & contatto \\
	\textbf{Ripristino:} & nessuno \\
 \rowcolor{gray!20}\textbf{Effetto:} & Arco elettrico. 2 bersagli entro 3 metri tra loro, 5d6 danni da elettricità
\end{tabularx}\\

\medskip

\noindent\begin{tabularx}{\linewidth}{p{2.3cm}X}
 \rowcolor{gray!20}\textbf{GS:} 4 &\textbf{Falce a Parete}\\
	\textbf{Tipo:} & meccanico \\
 \rowcolor{gray!20}\textbf{Sopravviv.:} & DC 20 \\
	\textbf{Dis. Cong.:} & DC 18 \\
 \rowcolor{gray!20}\textbf{Attivatore:} & posizione \\
	\textbf{Ripristino:} & automatico \\
 \rowcolor{gray!20}\textbf{Effetto:} & tre attacchi in mischia +10 (2d8+3)
\end{tabularx}\\

\medskip

\noindent\begin{tabularx}{\linewidth}{p{2.3cm}X}
 \rowcolor{gray!20}\textbf{GS:} 5&\textbf{Blocco in Caduta} \\
	\textbf{Tipo:} & meccanico \\
 \rowcolor{gray!20}\textbf{Sopravviv.:} & DC 23 \\
	\textbf{Dis. Cong.:} & DC 22 \\
 \rowcolor{gray!20}\textbf{Attivatore:} & posizione \\
	\textbf{Ripristino:} & manuale \\
 \rowcolor{gray!20}\textbf{Effetto:} & Attacco in mischia +15 (6d6)
\end{tabularx}\\

\medskip

\noindent\begin{tabularx}{\linewidth}{p{2.3cm}X}
 \rowcolor{gray!20}\textbf{GS:} 6 &\textbf{Colpo Infuocato}\\
	\textbf{Tipo:} & magico \\
 \rowcolor{gray!20}\textbf{Sopravviv.:} & DC 25 \\
	\textbf{Dis. Cong.:} & DC 24/4 \\
 \rowcolor{gray!20}\textbf{Attivatore:} & prossimità (Allarme) \\
	\textbf{Ripristino:} & nessuno \\
 \rowcolor{gray!20}\textbf{Effetto:} & cono 3 metri, 8d6 danni da fuoco
\end{tabularx}\\

\medskip

\noindent\begin{tabularx}{\linewidth}{p{2.3cm}X}
 \rowcolor{gray!20}	\textbf{GS:} 6&\textbf{Freccia Avvelenata}  \\
	\textbf{Tipo:} & meccanico \\
 \rowcolor{gray!20}	\textbf{Sopravviv.:} & DC 25 \\
	\textbf{Dis. Cong.:} & DC 20 \\
 \rowcolor{gray!20}	\textbf{Attivatore:} & posizione \\
	\textbf{Ripristino:} & nessuno \\
 \rowcolor{gray!20}	\textbf{Effetto:} & 6 frecce, attacco a distanza 18 metri +10 (1d6 più 1d6 Veleno)
\end{tabularx}\\

\medskip

\noindent\begin{tabularx}{\linewidth}{p{2.3cm}X}
 \rowcolor{gray!20}\textbf{GS:} 7&\textbf{Acque bollenti} \\
	\textbf{Tipo:} & meccanico \\
 \rowcolor{gray!20}\textbf{Sopravviv.:} & DC 25 \\
	\textbf{Dis. Cong.:} & DC 20/4 \\
 \rowcolor{gray!20}\textbf{Attivatore:} & posizione \\
	\textbf{Ripristino:} & nessuno \\
 \rowcolor{gray!20}\textbf{Effetto:} & cono di 6 metri (spruzzo di acqua bollente, 8d6 danni da fuoco).
\end{tabularx}\\

\medskip

\noindent\begin{tabularx}{\linewidth}{p{2.3cm}X}
 \rowcolor{gray!20}\textbf{GS:} 8&\textbf{Trappola a Gas} \\
	\textbf{Tipo:} & meccanico \\
 \rowcolor{gray!20}\textbf{Sopravviv.:} & DC 28 \\
	\textbf{Dis. Cong.:} & DC 26 \\
 \rowcolor{gray!20}\textbf{Attivatore:} & posizione \\
	\textbf{Ripristino:} & riparabile \\
 \rowcolor{gray!20}\textbf{Effetto:} & Gas velenoso, cubo 6 metri di spigolo. TS Tempra DC 17 oppure rallentati 1/1 minuto.
\end{tabularx}\\

\medskip

\noindent\begin{tabularx}{\linewidth}{p{2.3cm}X}
 \rowcolor{gray!20}\textbf{GS:} 9&\textbf{Raffica di Frecce} \\
	\textbf{Tipo:} & meccanico \\
 \rowcolor{gray!20}\textbf{Sopravviv.:} & DC 30 \\
	\textbf{Dis. Cong.:} & DC 27 \\
 \rowcolor{gray!20}\textbf{Attivatore:} & visivo ( Occhio Arcano) \\
	\textbf{Ripristino:} & riparabile \\
 \rowcolor{gray!20}\textbf{Effetto:} & 6 frecce. Attacco a distanza +14 (1d8+1)
\end{tabularx}\\

\medskip

\noindent\begin{tabularx}{\linewidth}{p{2.3cm}X}
 \rowcolor{gray!20}\textbf{GS:} 8&\textbf{Fossa Celata con spuntoni} \\
	\textbf{Tipo:} & meccanico \\
 \rowcolor{gray!20}\textbf{Sopravviv.:} & DC 29 \\
	\textbf{Dis. Cong.:} & DC 20 \\
 \rowcolor{gray!20}\textbf{Attivatore:} & posizione \\
	\textbf{Ripristino:} & manuale \\
 \rowcolor{gray!20}\textbf{Effetto:} & Fossa profonda 15 m + spuntoni (Attacco in mischia +9, 3 spuntoni per bersaglio, 1d6+5 danni ciascuno)
\end{tabularx}\\

\medskip

\noindent\begin{tabularx}{\linewidth}{p{2.3cm}X}
 \rowcolor{gray!20}\textbf{GS:} 9&\textbf{Pavimento Elettrico} \\
	\textbf{Tipo:} & magico \\
 \rowcolor{gray!20}\textbf{Sopravviv.:} & DC 30 \\
	\textbf{Dis. Cong.:} & DC 26/5 \\
 \rowcolor{gray!20}\textbf{Attivatore:} & prossimità (Allarme) \\
	\textbf{Ripristino:} & nessuno \\
 \rowcolor{gray!20}\textbf{Effetto:} & 6mx9m. 8d6 danni da Elettricità.
\end{tabularx}\\

\medskip

\noindent\begin{tabularx}{\linewidth}{p{2.3cm}X}
 \rowcolor{gray!20}\textbf{GS:} 10&\textbf{Risucchio di Energia} \\
	\textbf{Tipo:} & magico \\
 \rowcolor{gray!20}\textbf{Sopravviv.:} & DC 34 \\
	\textbf{Dis. Cong.:} & DC 34/6 \\
 \rowcolor{gray!20}\textbf{Attivatore:} & visivo (Visione del Vero) \\
	\textbf{Ripristino:} & nessuno \\
 \rowcolor{gray!20}\textbf{Effetto:} & Attacco di contatto a distanza 18 metri +14 da Vuoto, Punti Ferita max calano di 10d4 + Affaticato.
\end{tabularx}\\

\medskip

\noindent\begin{tabularx}{\linewidth}{p{2.3cm}X}
 \rowcolor{gray!20}\textbf{GS:} 10&\textbf{Stanza delle Lame} \\
	\textbf{Tipo:} & meccanico \\
 \rowcolor{gray!20}\textbf{Sopravviv.:} & DC 25 \\
	\textbf{Dis. Cong.:} & DC 20 \\
 \rowcolor{gray!20}\textbf{Attivatore:} & posizione \\
	\textbf{Ripristino:} & riparabile \\
 \rowcolor{gray!20}\textbf{Effetto:} & Attacco in mischia +15 (a tutti tre attacchi 3d8+3)
\end{tabularx}\\

\medskip

\noindent\begin{tabularx}{\linewidth}{p{2.3cm}X}
 \rowcolor{gray!20}\textbf{GS:} 11&\textbf{Cono di Ghiaccio} \\
	\textbf{Tipo:} & magico \\
 \rowcolor{gray!20}\textbf{Sopravviv.:} & DC 30 \\
	\textbf{Dis. Cong.:} & DC 30/6 \\
 \rowcolor{gray!20}\textbf{Attivatore:} & prossimità (Allarme) \\
	\textbf{Ripristino:} & nessuno \\
 \rowcolor{gray!20}\textbf{Effetto:} & come \hyperlink{Cono di Freddo}{Cono di Freddo} da 9d6 danni. TS Riflessi DC 22 per dimezzare
\end{tabularx}\\

\medskip

\noindent\begin{tabularx}{\linewidth}{p{2.3cm}X}
 \rowcolor{gray!20}\textbf{GS:} 11&\textbf{Fossa Avvelenata} \\
	\textbf{Tipo:} & meccanico \\
 \rowcolor{gray!20}\textbf{Sopravviv.:} & DC 25 \\
	\textbf{Dis. Cong.:} & DC 20 \\
 \rowcolor{gray!20}\textbf{Attivatore:} & posizione \\
	\textbf{Ripristino:} & manuale \\
 \rowcolor{gray!20}\textbf{Effetto:} & Fossa 6mx3m, 15 m profonda + spuntoni (3 attacchi in mischia +15 per bersaglio. 1d6+5 danni + veleno 2d6 danni)
\end{tabularx}\\

\medskip

\noindent\begin{tabularx}{\linewidth}{p{2.3cm}X}
 \rowcolor{gray!20}\textbf{GS:} 13&\textbf{Galleria dei Fulmini} \\
	\textbf{Tipo:} & magico \\
 \rowcolor{gray!20}\textbf{Sopravviv.:} & DC 29 \\
	\textbf{Dis. Cong.:} & DC 29/5 \\
 \rowcolor{gray!20}\textbf{Attivatore:} & prossimità (Allarme) \\
	\textbf{Ripristino:} & nessuno \\
 \rowcolor{gray!20}\textbf{Effetto:} & come incantesimo \hyperlink{Fulmine a catena}{Fulmine a catena}. DC 22.
\end{tabularx}\\

\medskip

\noindent\begin{tabularx}{\linewidth}{p{2.3cm}X}
 \rowcolor{gray!20}\textbf{GS:} 15&\textbf{Inferno di fuoco} \\
	\textbf{Tipo:} & magico \\
 \rowcolor{gray!20}\textbf{Sopravviv.:} & DC 31 \\
	\textbf{Dis. Cong.:} & DC 31/6 \\
 \rowcolor{gray!20}\textbf{Attivatore:} & prossimità (Allarme) \\
	\textbf{Ripristino:} & nessuno \\
 \rowcolor{gray!20}\textbf{Effetto:} & 60 danni da fuoco. TS Riflessi DC 23 per dimezzare.
\end{tabularx}\\

\medskip

\noindent\begin{tabularx}{\linewidth}{p{2.3cm}X}
 \rowcolor{gray!20}\textbf{GS:} 16&\textbf{Distruzione} \\
	\textbf{Tipo:} & N/A \\
 \rowcolor{gray!20}\textbf{Sopravviv.:} & DC 34 \\
	\textbf{Dis. Cong.:} & DC 34/6 \\
 \rowcolor{gray!20}\textbf{Attivatore:} & prossimità (Allarme) \\
	\textbf{Ripristino:} & nessuno \\
 \rowcolor{gray!20}\textbf{Effetto:} & come incantesimo \hyperlink{Disintegrazione}{Disintegrazione} con 1 Critico Magico. DC 24
\end{tabularx}

\medskip

\noindent\begin{tabularx}{\linewidth}{lX}
\rowcolor{gray!20}	\textbf{GS:} 19 &\textbf{Pioggia di Meteore}\\
\textbf{Tipo:} & magico \\
\rowcolor{gray!20}	\textbf{Sopravviv.:} & DC 34 \\
\textbf{Dis. Cong.:} & DC 34/6 \\
\rowcolor{gray!20}	\textbf{Attivatore:} & visivo \\
\textbf{Ripristino:} & nessuno \\
\textbf{Effetto:} & come incantesimo \hyperlink{Pioggia di Meteore}{Pioggia di Meteore}. DC 28
\end{tabularx}\\

\end{multicols}

\begin{narratore}[Tups e la trappola]
In questo esempio vi porto l'approccio vecchia scuola quando si presumeva che ci fossero delle trappole. Nulla vieta al Narratore di permettere prove di Sopravvivenza o Disattivare Congegni. Posso solo dire che questo approccio è però più coinvolgente.

\bigskip

\emph{Narratore}: un corridoio largo 3 metri porta a nord, nell'oscurità.

\emph{Tups}: Avanziamo tastando il pavimento con la nostra pertica di 3 metri.

\emph{Narratore}: la pertica è stata lasciata incastrata nello scontro con l'idolo di pietra.
[\emph{Se avesse usato la pertica la trappola sarebbe stata scoperta facilmente}.]
Prosegui nel corridoio?

\emph{Tups}: No, sono sospettoso. Posso vedere qualche crepa nel pavimento, magari di forma quadrata ?

\emph{Narratore}: No, ci sono milioni di crepe, non riesci a vedere una fossa così chiaramente [\emph{il Narratore valuta che la fossa è ben mimetizzata e Tups ha scarsa illuminazione per vedere bene}]

\emph{Tups}: Ok, prendo la mia fiasca d'acqua dallo zaino. Vado a versare un pò d'acqua sul pavimento. Sembra infilarsi nel pavimento in qualche punto o rivelare qualche forma di trama ?

\emph{Narratore}: Si, l'acqua sembra convogliare intorno ad una forma quadrata, leggermente rialzata sul pavimento.

\emph{Tups}: Sembra una fossa coperta ?

\emph{Narratore}: potrebbe essere

\emph{Tups}: posso disattivarla ?

\emph{Narratore}: come ? [\emph{Il Narratore volutamente non fa fare una prova, ma coinvolge il giocatore}]

\emph{Tups}: ci incastro il piede di porco così ché il meccanismo non faccia aprire la botola [\emph{Tups non chiede di tirare un dado per capire come disarmarla o disarmarla direttamente, spiega al Narratore come lo fa e basta}]

\emph{Narratore}: attraversi la zona adesso in sicurezza e vedi che si apre su una piccola stanza con due porte di legno rinforzato...

\medskip

Liberamente ispirato da \href{https://friendorfoe.com/d/Old%20School%20Primer.pdf}{ \textbf{Quick Primer for Old School Gaming}}

\end{narratore}

\bigskip

\begin{narratore}[Trappole ovvie]
Una trappola visibile/ovvia obbliga i giocatori ad interagire con essa, a sforzarsi per capirne il funzionamento ed ingegnarsi per evitarla o disattivarla. Evitate quando potete risoluzioni solo basate sul tiro di dado (Cerco trappole/Disattivo trappole), piuttosto premiate l'ingegnosità anche semplice ma creativa del giocatore per evitare il pericolo... e magari prima o poi si ricorderanno di recuperare il piede di porco...!
\end{narratore}

\vfill

\begin{center}
	\includegraphics[width=0.65\linewidth]{immagini/Bear_trap.png}

	\emph{Non tutte le trappole sono così ben segnalate...}
\end{center}

\pagebreak

\section{Veleni, Pozioni e Malattie}\index{Veleni}\index{Pozioni}\index{Malattie}

\label{veleni-e-pozioni}

\begin{enfasi}{
Un giorno, un uomo fu colpito da una freccia avvelenata. Gli amici e i parenti, in ansia, chiamarono un medico. Quando gli si avvicinarono per prendere la freccia, l'uomo disse loro: "Prima di farlo, vorrei sapere chi mi ha trafitto con questa freccia... Era uno schiavo, un re, o un bramino? Era grande? Piccolo? Di che colore era la sua pelle? Dove viveva? E la freccia com'è stata costruita? Quale veleno è stato impiegato? ..."

Mentre si stava ponendo tutte queste domande... il veleno fece il suo effetto e l'uomo ferito finì per morire. (Budda)}\end{enfasi}

\begin{multicols}{2}

\subsection{Tipo di Veleno e Pozione}\label{tipidiveleno}

I veleni e pozioni possono distinguersi in base all'effetto scatenato.
Non tutti i veleni sono tossici se ingeriti o inalati.

Per identificare una pozione naturale è necessario una prova di Erboristeria a DC uguale alla rarità della pianta oppure in caso di Veleni la difficoltà è pari al Tiro Salvezza dello stesso. Costa 1 Azione ogni 10 di DC, con Erboristeria a 6 o più costa 1 Azione ogni 15 DC, con 12 punti costa 1 Azione ogni 20 DC. Le pozioni se non descritto diversamente devono essere bevute (ingestione).

\textbf{Contatto}: sono contratti nel momento in cui qualcuno tocca il veleno con la pelle nuda. I veleni a contatto hanno solitamente un tempo di insorgenza di 1 round. Un veleno a contatto può essere un unguento, balsamo, liquido di qualsiasi densità o anche polvere se specifica per contatto e non inalazione.

\textbf{Ingestione}: si attivano quando una creatura le mangia o le beve. I veleni ad ingestione hanno solitamente un tempo di insorgenza di 10 minuti.

\textbf{Ferimento}: vengono trasferiti soprattutto con gli attacchi di alcune creature e tramite armi cosparse di veleno. I veleni a ferimento hanno solitamente un tempo di insorgenza istantaneo.

\textbf{Inalazione (R)}: si attivano nel momento in cui una creatura entra in un'area che contiene tali veleni. Molti veleni ad inalazione riempiono un volume pari ad un cubo con lato di 3x3x3 metri per dose. Le creature possono tentare di trattenere il fiato mentre si trovano all'interno dell'area per evitare di inalare la tossina.
Vedi regole per trattenere il fiato e soffocare in \hyperlink{trattenereilfiato}{Ambiente} (pag. \pageref{trattenereilfiato}).

\subsection{Insorgenza ed Effetto}\index{Insorgenza veleno}\index{Tempo di attivazione veleno}\label{insorgenzaveleno}\hypertarget{insorgenzaveleno}{}

Per insorgenza si intende quanto tempo ci mette il veleno o pozione a fare effetto. Se il tempo di insorgenza è 1 Turno significa che per gli effetti del veleno/pozione ed il Tiro Salvezza si aspetta 10 minuti. Se nella tabella del veleno/pozione insorgenza non è specificata significa che l'effetto è immediato dopo l'entrata in contatto con il veleno.

L'effetto di un veleno/pozione è immediato dopo l'insorgenza. Verificare la descrizione del veleno per capirne l'effetto. Se il Tiro Salvezza su Tempra riesce il veleno non ha fatto effetto e si può ritenere neutralizzato.

Ci sono alcuni casi in cui è presente la voce Frequenza, in queste occasioni il Tiro Salvezza va ripetuto ogni volta che passa la Frequenza indicata, in caso di fallimento del Tiro Salvezza gli effetti indicati vengono nuovamente applicati.

Bere una pozione tenuta in mano costa 1 Azione Immediata, farla bere ad un compagno privo di sensi costa 2 Azioni.

Se il personaggio \textbf{dedica 1 minuto} a bere una Pozione di Cura o Naturale questa avrà effetto massimizzato.\index{Pozioni effetto massimizzato}

%\begin{center}
%\includegraphics[height=0.6\linewidth]{immagini/potion.png}
%\end{center}

\begin{narratore}[Veleni o meno]
I veleni qui proposti sono alcuni dei tanti presenti e possibili. Usali come linee guida. Se per tua etica e stile non ti piacciono i veleni, specialmente quelli più cattivi, suggerisco di usare le Pozioni Generiche che trovi in fondo al capitolo. Sono veleni più blandi e meno personali, probabilmente più facilmente usabili anche dai giocatori.
\end{narratore}

\subsubsection{Avvelenati}\index{Avvelenati}\label{avvelenato}

\textbf{Prima dose}: Quando si viene esposti a un veleno per la prima volta (durante la propria azione o quella di qualcun altro), è necessario effettuare un Tiro Salvezza all'insorgenza per evitare di venire avvelenati.

\textbf{Successo}: Si resiste al veleno. Non si subiscono effetti negativi e non sono necessari ulteriori Tiri Salvezza.

\textbf{Fallimento}: Siete stati avvelenati e si subisce subito l'effetto indicato.

\textbf{più dosi}: Se si vieni esposti a più dosi dello stesso veleno prima dell'insorgenza la difficoltà del Tiro Salvezza aumenta di 1 per dose aggiuntiva.\index{Veleno più dosi}

\textbf{In tempi diversi}: se si viene esposti al veleno in tempi diversi dopo la prima insorgenza, ogni volta ci sarà un nuovo Tiro Salvezza e si subiranno gli eventuali effetti previsti, se invece l'esposizione avviene prima dell'insorgenza allora il Tiro Salvezza è unico e si conta come \emph{più Dosi}.

Se si viene esposti a veleni diversi è necessario effettuare un Tiro Salvezza per ogni tipo di veleno assunto.

\begin{giocatore}[Veleno ?]
{Il veleno è un arma a doppio taglio. Finché la usi tu va bene ma se te la usano contro, magari lo stesso, diventa un problema. Ci sono anche degli aspetti etici nell'usare i veleni, valuta se i tuoi Tratti ti permettono di usare dei veleni e di che tipi.}\end{giocatore}

\subsection{Applicare il Veleno}\index{Applicare il Veleno}\label{applicareveleno}

Applicare il veleno ad un'arma o ad una munizione richiede 1 Azione.

Ogni volta che un personaggio applica o prepara un veleno per l'uso deve effettuare una Prova di Erboristeria (DC 11) e se ottiene un fallimento e' entrato in contatto con il veleno e ne subisce gli effetti. Se la prova fallisce criticamente ha anche consumato una intera dose del veleno.

Ogni volta che un personaggio attacca con un'arma avvelenata, se esegue un Fallimento Critico con il Tiro per Colpire esegue un nuovo Tiro per Colpire e se si \emph{colpisce} allora si espone agli effetti del veleno. Ciò consuma il veleno sull'arma.\index{Fallimento critico con arma avvelenata}

Un pozione di veleno è sufficiente per coprire di veleno un arma media oppure 3 frecce. Il veleno viene così consumato e rimane attivo sull'arma finché questa non colpisce.

Una creatura sotto gli effetti di un veleno, anche se non manifestato, ha la condizione di Avvelenato.

\subsection{Rimuovere il veleno}

L'incantesimo \hyperlink{incrimuoviveleno}{Rimuovi Veleno} (pag. \pageref{incrimuoviveleno}) esegue una prova di \hyperlink{contrastareincantesimi}{contrasto} con il veleno, e quindi la condizione avvelenato.

Se la DC del veleno non è espressa allora si considera che sia sufficiente il semplice lancio dell'incantesimo per annullarne gli effetti.

Una prova di Pronto Soccorso\index{Pronto Soccorso e Veleni}\index{Veleni e Pronto Soccorso}, 3 Azioni, che sia almeno la metà della DC del veleno entro il tempo dell'insorgenza, permette di effettuare subito un Tiro Salvezza con un bonus di +2.

Questo Tiro Salvezza se riesce annulla gli effetti del veleno, se fallisce la creatura rimane avvelenata, se fallisce criticamente il TS avrà una penalità di -2.
Una volta fatta la prova di Pronto Soccorso non è più possibile rifarla se non dopo l'insorgenza.

Un trattamento continuativo di Pronto Soccorso per 8 ore permette di effettuare un nuovo Tiro Salvezza con +1d6 bonus dopo l'attivazione del veleno se questo è ancora attivo.

\medskip

\begin{enfasi}{
Io credo che una foglia d'erba non sia meno di una giornata di lavoro compiuto dagli astri. (Walt Whitman)}
\end{enfasi}

\subsection{Creare Veleni e Pozioni Naturali}\index{Creare Veleni Naturali}\label{crearevelenonaturale}\index{Creare Pozioni Naturali}

Pozioni e veleni naturali possono essere realizzati usando Erboristeria. La DC per preparare le pozioni è uguale alla rarità per e per i velini è pari al Tiro Salvezza -2.

Un Erborista può preparare contemporaneamente fino al suo valore in (Erboristeria/2)+1 di dosi di pozioni naturali o veleni in 8 ore di lavoro.

Un Fallimento Critico nella prova di Erboristeria renderà inutili i materiali usati.

Se gli ingredienti si comprano il costo per preparare il veleno è metà del costo di vendita indicato, se si cercano in natura il costo di produzione scende ad un quarto. Il tempo per preparare queste pozioni/veleni è pari alla DC/3 in ore.

Una Pozione che \emph{\textbf{Rimuove}} una condizione è efficace se la sua DC è superiore a quella della Condizione stessa. \index{Pozioni e rimozioni condizioni}

Gli esempi seguenti rappresentano solo alcuni dei possibili veleni. Tutti i costi sono espressi in Monete d'Oro. I Veleni sono presentati, specialmente nel Mostruario con questa dizione: Nome Veleno, Uso (I/R/F/C), tempo Insorgenza, DC del Tiro Salvezza, Effetto.

\begin{narratore}[Anche Veleni]
I veleni fanno parte della lunga tradizione dei problemi ed avversità nei giochi di ruolo. Quando volete usare un veleno pensate innanzitutto perché si trova li, per chi doveva essere usato, per quale scopo. Non è detto che tutti i veleni debbano uccidere, un abile ladro potrebbe anche usare veleni stordenti o che indeboliscono la volontà del suo obiettivo giusto quel tanto che basta a farsi aprire la cassaforte.
\end{narratore}

\subsection{Come trovare le pianticine...}

Per \emph{trovare} gli ingredienti per preparare i \textbf{veleni} è necessario superare con la prova di Erboristeria la DC indicata dal TS.

Per \emph{trovare} gli ingredienti per preparare le \textbf{pozioni naturali} è necessario superare con la prova di Erboristeria la Rarità indicata nella colonna \emph{Rar.}

\end{multicols}

\begin{center}

	\begin{tikzpicture}[scale=1.2]
	% Stem
	\draw[thick, black] (0,0) -- (0,2);

	% Main leaves
	\draw[fill=gray!30, black]
	(-0.3,1.5) .. controls (-0.8,1.7) and (-0.8,2.1) .. (-0.3,2.0)
	.. controls (0.1,1.9) .. (0,1.5);

	\draw[fill=gray!30, black]
	(0.3,1.3) .. controls (0.8,1.5) and (0.8,1.9) .. (0.3,1.8)
	.. controls (-0.1,1.7) .. (0,1.3);

	% Small leaves
	\draw[fill=gray!20, black]
	(-0.2,1.0) .. controls (-0.5,1.1) and (-0.5,1.3) .. (-0.2,1.2)
	.. controls (0.0,1.15) .. (0,1.0);

	\draw[fill=gray!20, black]
	(0.2,0.8) .. controls (0.5,0.9) and (0.5,1.1) .. (0.2,1.0)
	.. controls (0.0,0.95) .. (0,0.8);

	% Flower
	\draw[fill=white, black] (0,2.2) circle (0.15);
	\foreach \angle in {0,45,90,135,180,225,270,315} {
		\draw[fill=gray!40, black]
		([shift={(\angle:0.15)}]0,2.2)
		.. controls ([shift={(\angle:0.25)}]0,2.2)
		and ([shift={(\angle+20:0.25)}]0,2.2)
		.. ([shift={(\angle+40:0.15)}]0,2.2);
	}

	% Roots
	\draw[thick, black] (0,0) -- (-0.2,-0.3);
	\draw[thick, black] (0,0) -- (0.1,-0.25);
	\draw[thick, black] (0,0) -- (0.3,-0.2);

	% Ground line
	\draw[thick, black] (-1,0) -- (1,0);
\end{tikzpicture}

\end{center}

\vfill

\begin{center}
	\includegraphics[height=0.25\linewidth]{immagini/poison.png}
\end{center}

\textbf{Tabella: Veleni. (costo in Monete d'Oro)}\index[Tabelle]{Tabella Veleni}\label{tabellaveleni}

\medskip

	\noindent\begin{tabularx}{\linewidth}{lcccXc}
	\toprule
 \rowcolor{gray!20}\textbf{Nome Veleno} & \textbf{Uso} & \textbf{Ins.} & \textbf{TS} & \textbf{Effetto (danno)} & \textbf{Costo}\\
\toprule
	Mistura Rossa \index{Mistura Rossa} & F& -& 13& -1d6 TC/TS per 10 minuti & 10\\

 \rowcolor{gray!20}Concentrato Viola \index{Concentrato Viola} & F& -& 15& 2d6 Punti Ferita & 15\\

	Grasso di Toporagno Viola \index{Grasso di Toporagno Viola} & C& 1 r& 13& 2d12 Punti Ferita & 15\\

 \rowcolor{gray!20}Nocciolo di Dennar \index{Nocciolo di Dennar}& I& 1 r& 13& -1d2 Forza, per 3gg& 15\\

	Succo di Ythis\index{Succo di Ythis} & I& 1 T& 14& -1d2 Intelligenza, per 1g& 20\\

	Bava fermentata di Lucos \index{Bava fermentata di Lucos}\label{bavadilucos}\hypertarget{bavadilucos}{}& F& - & 15& 1d8 Punti Ferita& 25\\

 \rowcolor{gray!20}Cenere di Corteccia Gialla \index{Cenere di Corteccia Gialla} & F& 6 r& 15& Privo di sensi per 1d3 ore& 25\\

	Polline di Rosa di Omro\index{Polline di Rosa di Omro} & I& - & 15& -1d3 Costituzione e Destrezza, per 1 ora & 25\\

 \rowcolor{gray!20}Profumo di Ragmor \index{Profumo di Ragmor}& R& - & 16& -1d3 Carisma, per 1 giorno & 30\\

	Dita di Daraka\index{Dita di Daraka} & F& - & 17& -1d6 Forza, per 1 ora & 35\\

 \rowcolor{gray!20}Bacca Viola di Barsar\index{Bacca Viola di Barsar}& I& 1 r& 18& Incapace di violenze per 3d8 ore& 40 \\

	Gelo blu \index{Gelo blu} & F& -& 18& 3d6 Punti Ferita da freddo& 25\\

 \rowcolor{gray!20}Fumi di Curna\index{Fumi di Curna} & R& - & 18& -1d3 Saggezza & 40\\

	Fiocco bianco di Mucot \index{Fiocco bianco di Mucot}& C& - & 20& Dorme per 2d12 ore& 20\\

 \rowcolor{gray!20}Lingua di Kreex \index{Lingua di Kreex} & F& - & 20& La ferita sanguina. +1 danno da sanguinamento. 1 uso nelle 24 ore. & 50 \\

	Muschio Giallo \index{Muschio Giallo}& I& 1 r& 20& la creatura guadagna una taglia. -2 Int e Sag. Durata 10 minuti& 50\\

 \rowcolor{gray!20}Olio di Nabar \index{Olio di Nabar}& R-F& - & 20& Confuso per 2d6 round& 50\\

	Veleno di Ottalm\index{Veleno di Ottalm}& F& - & 20& Morte o -1d2 Costituzione permanente& 50\\

 \rowcolor{gray!20}Bacche Azzurre di fosso \index{Bacche Azzurre di fosso}& I& 1 T& 21& -1d3 Intelligenza e Saggezza per 6 ore& 55\\

	Erba puntuta rosa \index{Erba puntuta rosa}& I& 1 r& 22& -1d6 Destrezza, per 1 ora& 60\\

 \rowcolor{gray!20}Pelle di Rospo Azzurro \index{Pelle di Rospo Azzurro}& C& 10 r & 22& Paralizzato per 1d6 turni& 60\\

	Fegato di Toporagno Viola \index{Fegato di Toporagno Viola} & I& 1 ora& 25& 2d6 di danno a Saggezza e Intelligenza. Permanente & 75 \\

 \rowcolor{gray!20}Veleno di Serpe del Sangue \index{Veleno di Serpe del Sangue} & F& - & 25& Paralisi per 1d6 ore -1d4 punti Forza per 7 giorni & 75\\

	Sangue di Thrun \index{Sangue di Thrun} & C& - & 26& -1d3 Costituzione & 80\\

\end{tabularx}

\medskip

\textbf{Applicazione}: \textbf{I}(ngestione), \textbf{F}(erimento), \textbf{C}(ontatto), \textbf{R}(espirazione).

Il Tiro Salvezza è sempre su Tempra se non specificato diversamente

I punti caratteristica persi si recuperano al ritmo di 1 al giorno se non permanenti o indicato diversamente.

\medskip

\textbf{Tabella: Pozioni Naturali (costo in monete d'oro)}\index[Tabelle]{Pozioni Naturali}\label{tabellapozioni}

\medskip

\noindent\begin{xltabular}{\linewidth}{llllXll}
	\toprule
 \rowcolor{gray!20}\textbf{Nome}& \textbf{Uso} & \textbf{Ins.} & \textbf{DC} & \textbf{Effetto}& \textbf{Rar.} & \textbf{Costo} \\
 	\toprule
	Wickalim & I& 1 ora& 15& Cura 2 Punti Ferita& 9 & 5 \\
 \rowcolor{gray!20}Darsirion & C& 1 r & 25& Cura 1d4 Punti Ferita& 9 & 5 \\
	Harfy & C& I & 12& -1 al sanguinamento& 10 & 3 \\
 \rowcolor{gray!20}Barannie & I& 10 r& 15& Rimuove nausea & 11 & 3 \\
	Estratto radice Gisenosa & I& 3 T& 15& Cura tosse e raffreddore& 11 & 3\\
 \rowcolor{gray!20}Klagul & C& 1 T & 20& Pulisce i denti & 11 & 2 \\
	Nelthalion & I& I& 15& Fa vomitare& 11 & 1\\
 \rowcolor{gray!20}Delrean & C& 1 r & 15& Allontana insetti per 1 giorno & 12 & 2\\
	Uovo di Urk & I& 1 T & 12& 1 giorno di cibo& 12 & 1\\
 \rowcolor{gray!20}Delrean Plus & I& 1 r & 18& Allontana insetti per 3 giorni & 13 & 5\\
	Corteccia di Aklent & I& 1 T & 10& La corteccia masticata per almeno 10 round concede per le 24 ore successive un +1 Tiro Salvezza vs Veleno& 13 & 1\\
 \rowcolor{gray!20}Lievito Muschio Bianco & I& 10 r& 12& I prodotti da forno che usano questo lievito causano meteorismo incontrollabile ed incredibilmente puzzolente per 12 ore & 14 & 1\\
	Harfindar & I& 1 T & 15& Fa abortire& 14 & 3 \\
 \rowcolor{gray!20}Gusperboon & C& 1 r & 20& La pelle diventa camaleontica concedendo un +1d6 alla prove di Furtività & 15 & 8\\
	Bacche di Ljust & I& 1 r & 16& Preso la sera recuperi il doppio dei PF, minimo 4 & 15 & 10 \\
 \rowcolor{gray!20}Mirenna & I& 1 r & 20& Cura 5 Punti Ferita& 16 & 30 \\
	Klynkyx & C& 1 ora & 15& Fa cadere tutti i capelli per 1d6+4 gg & 16 & 4\\
 \rowcolor{gray!20}Miscela 31 & I& 1 T & 20&La cavalcatura è estremamente resistente. +4 ore di galoppo al giorno & 17 & 15\\
	Culcoa & C& 1 r & 16& Recuperi 2d6 da danno da fuoco & 18 & 8 \\
 \rowcolor{gray!20}Jojopo & C& 1 r & 15& Resistenza a Freddo per 1 ora & 18 & 45 \\
	Silea & C& 5 r& 15& Cura 1d6+3 Punti Ferita & 18 & 50 \\
 \rowcolor{gray!20}Draaf & C& 1 r & 20& Cura 1d8 Punti Ferita& 19 & 50 \\
	Estratto di Bacca Illa & I& 1 r & 15& +2 Iniziativa, +2 Destrezza, -1d6 Tiro Salvezza su Volontà, per 1 minuto& 19 & 15\\
 \rowcolor{gray!20}Petali di Lisbeth & I& 1 T & 15&+2 Intelligenza, -2 Destrezza per 10 minuti & 19 & 20 \\
	Kelventare & I& 2 r & 28& Recuperi 2d6 Punti Ferita & 20 & 100 \\
 \rowcolor{gray!20}Garioe & I& 1 r & 25& Cura 2d6 Punti Ferita& 20 & 95 \\
	Yajeth & I& 1 T& 20& Cura 2d8 Punti Ferita& 20 & 100 \\
 \rowcolor{gray!20}Corteccia Dagmather & R& 1 r & 25& Rimuove un livello di Affaticamento& 21 & 15 \\
	Radice di Kathaus & R& 1 r& 20& +2 Forza e Destrezza per 1 ora & 21 & 100 \\
 \rowcolor{gray!20}Klandor & I& I& 15& Rimuove paralisi. Aumenta di 1 il livello di affaticamento& 22 & 18 \\
	Eldrin'tail & I& 1 r& 15& Concede un nuovo Tiro Salvezza su Veleni& 22 & 18 \\
 \rowcolor{gray!20}Attarna & I& 1 T & 20& Concede un nuovo Tiro Salvezza per Malattie con un +1d6& 22 & 50 \\
	Lingua Rossa di Xabax & C& 1 T & 20& Cura 2d6 Punti Ferita ma se c'è malattia o veleno tenta la rimozione causando 2d6 PF di danno & 23 & 13 \\
 \rowcolor{gray!20}Arduur & I& 1 r & 25& Rimuove Veleni& 24 & 75 \\
	Arksun & C& 1 T& 25& Cura 1d6 PF a Turno per 3 turni& 24 & 75 \\
 \rowcolor{gray!20}Uscaboo & R& 1 T & 25& Rimuove cecità& 25 & 125 \\
	Melandrir & I& 1 r & 15& Concede un nuovo Tiro Salvezza per Malattie con +4& 26 & 100 \\
 \rowcolor{gray!20}Gylvert & I& 10 r& 25& Concede respirare sott'acqua per 4 ore & 27 & 3 \\
	Callynthine & C& 1 ora& 24& Rinsalda le fratture, cura 2d8+8 PF & 28 & 200\\
 \rowcolor{gray!20}Gelfnus & I& 5 r & 28& Cura 3d8+3 Punti Ferita, rende Affaticato & 28 & 150 \\
	Polline di Rosa Verde & R& 3 T & 25& Recuperi 2d4 danni Intelligenza e Saggezza& 29 & 350 \\
 \rowcolor{gray!20}Muschio argentato & I& I& 25& Rimuove Malattie magiche & 30 & 250\\
	Nazamuse & I& I& 30& Rimuove Veleni e Malattie naturali & 32 & 175\\
 \rowcolor{gray!20}Bacio di Ljust & C& 1 r & 35& Cura 100 Punti Ferita& 34 & 500\\
	Estratto 100 erbe & I& I& 24& Rimuove Cecità, Sordità, Paralisi, Veleno &35&150 \\
\end{xltabular}

\begin{multicols}{2}

\subsubsection{Note sui Veleni e Pozioni}

\textbf{Arduur}: Un'erba sacra, raccolta principalmente nei boschi di Sangzhar. Viene macerata e ridotta in un decotto, bevuto per rimuovere i veleni più persistenti.

\textbf{Arksun}: Pozione dai riflessi dorati, distillata dalle lacrime dei fiori di sole. Ogni druido custodisce gelosamente la ricetta.

\textbf{Attarna}: Estratto dalla corteccia degli alberi di Tarna. La sua assunzione garantisce una possibilità contro malattie debilitanti.

\textbf{Bacca Viola di Barsar}: curiosità, il Toporagno viola è schifato da queste bacche.

\textbf{Bacio di Ljust}: Leggendario rimedio. Si dice che ogni pianta, da cui si ricava questo elisir, cresca al centro di un cerchio di funghetti magici.

\textbf{Barannie}: Un'erba semplice ma efficace, spesso usata dai pastori nomadi per alleviare la nausea.

\textbf{Bava fermentata di Lucos}: Lucos è una lucertola erbivora e pacifica. La bava raccolta va fatta fermentare al buio per 1 settimana prima di essere utilizzabile.

\textbf{Callynthine}: Creato dalle radici della vite di Callyntha, il suo colore azzurro scuro riflette la nobiltà del suo effetto curativo.

\textbf{Cenere di Corteccia Gialla}: la corteccia va prima macerata e battuta in acqua e sale. La poltiglia risultante va seccata e poi fatta scaldare senza bruciarla direttamente.

\textbf{Corteccia Dagmather}: Piccolo arbusto sempre verde.

\textbf{Corteccia di Aklent}: chiamato anche \emph{Cespuglio Puzzola} per il suo pungente e caratteristico odore.

\textbf{Culcoa}: Un unguento rosso lucente usato principalmente per curare le ustioni da fuoco.

\textbf{Darsirion}: Una pozione leggera, fatta con i fiori argentei risplendenti alla luna. Utilizzata dai guaritori per curare piccole ferite.

\textbf{Delrean Plus}: Una formula avanzata che mescola l'estratto di foglie di Delrean con cera d'api. È un repellente infallibile.

\textbf{Delrean}: Estratto dalle foglie di Delrean viene utilizzato come repellente per insetti fastidiosi durante il raccolto.

\textbf{Dita di Daraka}: le Dita di Daraka sono il frutto dell'albero di Daraka. Il baccello di forma allungata e nera ricorda le dita dell'antica dea dell'oscurità.

\textbf{Draaf}: Un preparato basato su antiche ricette alchemiche. La pozione è conosciuta per la sua capacità di rigenerazione rapida.

\textbf{Eldrin'tail}: Il tè preparato con questa erba argentata ha un sapore amaro, ma efficace per contrastare i veleni più forti.

\textbf{Estratto 100 erbe}: Una miscela complessa e poco conosciuta

\textbf{Estratto di Bacca Illa}: Le bacche usate in questa preparazione devono essere bruciate delicatamente, per preservare i nutrienti.

\textbf{Estratto di radice Gisenosa}: pianta tipo carciofo, estremamente spinosa. Tende a crescere circondata dal \emph{Tribulus terrestris} o \emph{baciapiedi}.

\textbf{Fegato di Toporagno Viola}: avvelenamento riconoscibile dai tipici occhi iniettati di sangue

\textbf{Fumi di Curna}: la Curna è l'inflorescenza del cardo rosso.

\textbf{Garioe}: Un frutto raro delle colline sud-orientali. Viene macerato in acqua e alcool per creare questa pozione curativa potente.

\textbf{Gelfnus}: creato dai fiori che sbocciano solo nella notte della luna piena.

\textbf{Gusperboon}: una piccola margherita, all'apparenza.

\textbf{Gylvert}: Un composto derivato dalle alghe di mare profondo

\textbf{Harfy}: Una crema composta da muschi, radici e lumache.

\textbf{Jojopo}: Un elisir caldo, zuccherino leggermente alcolico.

\textbf{Kelventare}: Uno sciroppo puro estratto dalla linfa dell’Albero della Vita.

\textbf{Klagul}: Una pasta abrasiva di antica ricette goblin. Usata come dentifricio in molte tribù selvagge.

\textbf{Klandor}: Polvere piccante e salata.

\textbf{Klynkyx}: un intruglio di miele e sterco di toporagno viola.

\textbf{Lingua Rossa di Xabax}: è il petalo lungo della Xabax. Dei 7 petali solo quello lungo ha i sostanze necessarie a preparare l'unguento.

\textbf{Melandrir}: Prodotto con una rara orchidea

\textbf{Mirenna}: Estratto da piante officinali.

\textbf{Miscela 31}: un insieme studiato di droghe per i saurovalli. Terminato l'effetto la creatura deve fare un Tiro Salvezza su Tempra DC 23 o cadere svenuto per 12 ore.

\textbf{Muschio Argentato}: molto simile, per un non esperto, al Muschio Bianco. Si raccolgono le bacche.

\textbf{Nazamuse}: Creato con tecniche avanzate dai Devoti di Atherim.

\textbf{Nelthalion}: L'estratto rosso-brunastro induce vomito.

\textbf{Olio di Nabar}: le piccole bacche di Nabar sono esclusivamente mangiate dai Toporagni, immuni ai loro malefici effetti. Bollito a lungo diviene un ottimo unguento per la pelle.

\textbf{Petali di Lisbeth}: Bellissimi e nerissimi.

\textbf{Polline di Rosa Verde}: Solo dalle Rose verdi più rare.

\textbf{Radice secca di Kathaus}: piccolo tubero nero estremamente duro e legnoso. Solitamente si lascia seccare al sole prima di macinarla.

%\textbf{Rewky}: Estratto alcolico blu traslucido.

\textbf{Silea}: Un balsamo verde pastoso.

\textbf{Toporagno Viola}: secondo molti il Toporagno è l'animaletto preferito di Cattalm. Aggressivo, violento, pericoloso in ogni sua fibra.

\textbf{Uovo di Urk}: Urk è un grosso coleottero, l'uovo è poco più grande di una nocciola. Solitamente viene prima affumicato con legno di faggio, mangiato crudo il sapore è di muffa e terra.

\textbf{Uscaboo}: Distillato dalle foglie di un tubero rosa

\textbf{Veleno di Ottalm}: l'Ottalm è una variante di Toporagno viola dotato di un pungiglione velenoso.

\textbf{Wickalim}: Usato spesso per piccole ferite, lasciano un sapore salato in bocca.

\textbf{Yajeth}: Estratto trito del fiore.

\subsection{Pozioni generiche}\index{Pozioni generiche}\index{Pozioni}

Il Narratore è libero di usare tutte le pozioni e veleni già elencate oppure usare delle pozioni generiche pronte all'uso. Nella tabella sono indicati i costi ed effetti di queste pozioni generiche.

L'insorgenza è sempre immediata, la durata per le cure è immediata, per le altre è 10 minuti (quindi la pozione Rimuovi Veleno ti protegge per 1 Turno contro un veleno).

\end{multicols}

%\vfill

%\begin{center}
%\includegraphics[width=0.37\linewidth]{immagini/mandragola2.png}
%
%\emph{Pianta di Mandragola}
%\end{center}

\textbf{Tabella: delle pozioni generiche. Costo in Monete d'Oro.}\index[Tabelle]{Tabella delle pozioni generiche}\label{pozionigeneriche}\hypertarget{pozionigeneriche}{}

\medskip

\noindent\begin{tabularx}{\linewidth}{lXcc}
	\toprule
\rowcolor{gray!20}\textbf{Nome Pozione}& \textbf{Effetto}&\textbf{Costo}& \textbf{Appl.}\\
\toprule
Cura& cura 1d8+1 Punti Ferita & 50 & I\\
\rowcolor{gray!20}Cura potenziata& cura 3d8+3 Punti Ferita & 125& I\\
Cura maggiore& cura 3d10+15 Punti Ferita & 300& I\\
\rowcolor{gray!20}Indebolente& -2 TC. TS DC 15 Tempra per annullare gli effetti& 35 & F\\
Indebolente potenziata& -1d6 TC. TS DC 18 Tempra per annullare gli effetti& 50 & F/I \\
\rowcolor{gray!20}Veleno&2d6+2 di danno. TS DC 15 Tempra per dimezzare& 30 & I/F \\
Veleno potenziata& 2d8+4 di danno. TS DC 18 Tempra per dimezzare & 50 & F \\
\rowcolor{gray!20}Veleno maggiore& 4d10+8 di danno. TS DC 24 Tempra per dimezzare& 125 & F \\
Rimuovi Veleno& concede un nuovo TS con +1d6 & 75 & I\\
\rowcolor{gray!20}Pozione Generica* & vedi \hyperlink{crearepozioni}{Creare Pozioni} (pag. \pageref{crearepozioni}) &Lv*Lv*50&I

\end{tabularx}

\subsection{Opzionale - Droghe}\index{Droga}\index{Opzionale - Droghe}\hypertarget{droghe}{}\label{droghe} \index[Tabelle]{Tabella Elenco Droghe}

\medskip

\noindent\begin{tabularx}{\linewidth}{lcccXcc}
	\toprule
\rowcolor{gray!20}\textbf{Nome}& \textbf{Uso} & \textbf{Ins.} & \textbf{DC} & \textbf{Effetto}& \textbf{Rar.} & \textbf{Costo} \\
\toprule
Foglie fermentate di Luside\index{Foglie fermentate di Luside}& I& 1 T& 17&+4 Carisma ed Intelligenza per 1d4 ore& 17 & 5 mo\\
\rowcolor{gray!20}Ferpillon \index{Ferpillon}& I& 1 r & 20& Fa dormire per 24 ore& 15 & 50 mo \\
Unto Grigio \index{Unto Grigio} & I& 1 r & 24& Contrasta DC 25 effetti Lista Ammaliamento& 19 & 80 mo\\
\rowcolor{gray!20}Cenere di Arpasur \index{Cenere di Arpasur}& R& 1 r & 20& Rimuove 2 livelli di affaticato& 16 & 10 mo\\
%Carne secca di Toporagno Viola \index{Carne secca di Ragno Viola} & I& 1 round & 24& +4 Forza -4 Intelligenza (minimo -3) per 1 turno& SH7 & 30 \\
%
Estratto alcolico di Melzaa\index{Estratto alcolico di Melzaa}& I& 1 r & 20& +2 For, +2 Des, -2 Sag. Per 1d4 ore & 16 & 25 mo \\
\rowcolor{gray!20}Essenza profumata di Inut\index{Essenza profumata do Inut} & R& I& 15& +4 Destrezza per 1d8 ore& 16 & 15 mo \\
Polline di Julnnaus\index{Polline di Julnnaus} & R& I& 20& +3 Costituzione per 2 ore & 16 & 25 mo\\
\rowcolor{gray!20}Polline del fiore di Erain \index{Polline del fiore di Erain} & R& 1 r& 20& +2 For, Int, Des, +3d6 PF temporanei per 1 ora& 17 & 75 mo\\
\end{tabularx}

\begin{multicols}{2}

\medskip

\textbf{L'utilizzo delle droghe è completamente opzionale. E' il Narratore a decidere la loro presenza e disponibilità in base alla sensibilità dei giocatori}.

Le droghe danno dipendenza. Terminato l'effetto è necessario effettuare un Tiro Salvezza su Volontà a difficoltà 15 o prenderne un altra dose entro 24 ore, il successivo Tiro Salvezza avrà difficoltà +1 e così via.

Se il Tiro Salvezza riesce è comunque necessario effettuarne uno nuovo il giorno dopo con le stesse conseguenze.

Non prendere una dose aumenta il livello di Affaticamento di uno. Sono necessari 7 Tiri Salvezza riusciti di seguito per terminare l'effetto di dipendenza.

La DC indicata è per resistere agli effetti.

\subsection{Opzionale - Bere troppo}\index{Bere troppo}\index{Opzionale - Bere troppo}\hypertarget{alcolismo}{}\label{beretroppo}\index{Ubriachi}

Una creatura può bere un numero di boccali di birra o bicchierini di liquore pari al suo punteggio di Costituzione +1, con un minimo di 1. Ulteriori boccali costringono la creatura a fare un Tiro Salvezza su Tempra a DC 11, ogni successivo boccale aumenta il Tiro Salvezza di +2. Quando il Tiro Salvezza fallisce la creatura è ubriaca e si considera sotto l'effetto dell'incantesimo Confusione per un numero di Turni pari al modificatore aggiuntivo alla difficoltà.

Birre o liquori a maggiore gradazione impongono un modificatore più alto al Tiro Salvezza.

Il Narratore può decidere di gestire una creatura \emph{alticcia} solo attraverso il \emph{roleplaying}.

\subsection{Malattie}\index{Malattie}\hypertarget{malattie}{}\label{malattie}

Le malattie in linea di principio si gestiscono eseguendo un Tiro Salvezza per verificare se ci si è contagiati ed altri Tiri Salvezza per guarire.
Solitamente il tempo di scatenamento di una malattia non è così immediato come un veleno eppure quelle magiche possono essere dirompenti ed agire in pochi minuti.

Ogni malattia ha indicato il tempo di insorgenza, il Tiro Salvezza iniziale, ogni quanto va rifatto il Tiro Salvezza, quanti successi ai Tiri Salvezza sono necessari per guarire, gli effetti che si subiscono. Se non specificato diversamente le malattie si trasferiscono tramite il ferimento di un infetto.

Es. \emph{Febbre Demoniaca minore}: 1 minuto, TS Tempra DC 18, 6 ore, 3 successi, -1 Costituzione e Saggezza

La Febbre Demoniaca minore costringe al Tiro Salvezza su Tempra a DC 18 dopo un solo minuto che la si è presa. Successivamente ogni 6 ore va rifatto il Tiro Salvezza e la malattia permane finché non si sono fatti almeno 3 successi consecutivi al TS. Ogni 6 ore il malato perde 1 punto di Costituzione e Saggezza.

Per guarire da una malattia, non naturale, come quelle afflitte dai mostri è necessario superare i Tiri Salvezza richiesti oppure usare l'incantesimo di \hyperlink{rimuovimalattie}{Rimuovi Malattie} (pag. \pageref{rimuovimalattie}).

Si effettua una prova di \hyperlink{contrastareincantesimi}{contrasto} (pag. \pageref{contrastareincantesimi}) tra l'incantesimo Rimuovi Malattie e la DC della malattia.

Se la DC della malattia non è espressa allora si considera che sia sufficiente il semplice lancio dell'incantesimo per annullarne gli effetti.

Una prova di \textbf{Pronto Soccorso}\index{Pronto Soccorso e Malattie}\index{Malattie e pronto soccorso}, 10 minuti, con DC pari almeno alla metà della DC della malattia (o 15 se non indicata), effettuata tra un Tiro Salvezza e successivo, permette di avere un +2 al Tiro Salvezza per resistere agli effetti della malattia. Un trattamento per una intera notte concede +1d6 al Tiro Salvezza successivo.

Essere colpiti più volte dalla stessa malattia non ne aumenta la difficoltà di guarigione ne cambia i tempi ed effetti della stessa.

Esempi di Malattie:

\textbf{Influenza Demoniaca}: 1 minuto, TS Tempra DC 16, 12 ore, 2 successi, -1 Costituzione\index{Influenza Demoniaca}

\textbf{Corruzione di Rezh}: 1 giorno, TS Volontà DC 18, 1 ora, 2 successi, -1d6 Punti Ferita Massimi\index{Corruzione di Rezh}

\textbf{Maledizione di Efrem}\index{Maledizione di Efrem}: 8 ore, TS Tempra DC 24, 12 ore, 2 successi, -1 punto a Destrezza e +1 Difesa

\textbf{Torpore Violento}\index{Torpore Violento}: 24 ore, TS Volontà DC 12, 12 ore, 1 successo, +1 al Danno con Armi da Mischia e -1 Saggezza

\textbf{Febbre Demoniaca minore}\index{Febbre Demoniaca minore}: 1 minuto, TS Tempra DC 18, 6 ore, 3 successi, -1 Costituzione e Saggezza

\textbf{Sangue Nero}\index{Sangue Nero}: 10 minuti, TS Tempra 28, 12 ore, 1 successo, perdita metà Punti Ferita rimasti

\textbf{Peste T}\index{Peste T}: 1 minuto, TS Tempra 30, 2 ore, 3 successi, esegui 3 successi consecutivi altrimenti vieni trasformato in uno zombi di pari GS. Si trasmette attraverso il contatto.

\end{multicols}

\vfill

\begin{center}
\includegraphics[width=0.8\linewidth]{immagini/funeralebarca.png}
\end{center}

%\begin{center}
%\includegraphics[width=0.6\linewidth]{immagini/plaguedoctor.png}

%\emph{Engraving of the Plague Doctor, Paul Furst, 1656}
%\end{center}

\pagebreak

\section{Movimento e Trasporto}\index{Trasporto}\index{Movimento}\label{movimentocap}

\label{movimento-e-trasporto}

\begin{enfasi}{
Il mio piede sinistro funziona benissimo, ma non riuscirei comunque a camminare se non ci fosse il destro! (Madagascar 3 - Ricercati in Europa, Film )

\medskip

Quando non puoi più correre, cammina veloce; quando non puoi più camminare veloce, cammina; quando non puoi più camminare, usa il bastone; però non trattenerti mai. (Madre Teresa di Calcutta)}
\end{enfasi}

\begin{multicols}{2}

Il movimento si può distinguere in base a quale situazione si applica.

\medskip

\begin{itemize}[leftmargin=*] \setlength{\itemsep}{0pt}
\item Tattico, quando si combatte, si usano le distanze precise, mappa ed i quadretti di 1 metro di lato
\item Locale, per esplorare una zona, misurato in metri al minuto.
\item Via Terra, per muoversi da un posto all'altro, misurato in km all'ora o al giorno.
\end{itemize}

\subsection{Tipi di Movimento}\label{tipodimovimento}\hypertarget{tipodimovimento}{}

Quando si muovono nelle differenti situazioni di movimento (Tattico, Locale, Via Terra), le creature generalmente camminano o corrono.

\textbf{Camminare}:\index{Camminare} Camminare rappresenta un movimento non affrettato ma deciso di circa 4 km all'ora per un umano non ingombrato. Per Azione di Movimento la creatura percorre la distanza indicata in Movimento.

\textbf{Correre}\index{Correre}: Significa muoversi di circa 12 km all'ora per un umano.

Correre come Azione di Movimento raddoppia la velocità di movimento.
Il personaggio che corre ha penalità di 1d6 nel Tiro per Colpire e di 4 nella Difesa fino all'inizio del suo round successivo.
Solo in situazioni di non combattimento la corsa triplica il movimento, ovvero quando si usa il Movimento locale o Via Terra.

\textbf{Tabella: Movimento e Distanza e Velocità: a Piedi} \index{Movimento a Piedi}\index[Tabelle]{Tabella Movimento e Distanza e Velocità : a Piedi}

Questa tabella mostra i valori base di movimento a terra in situazioni di non combattimento.

\medskip

\noindent\begin{tabularx}{\linewidth}{lccc}
	\toprule
\rowcolor{gray!20}\multirow{2}*{\textbf{Tipo di movimento}} &
\multicolumn{3}{c}{\textbf{Movimento}}\\
\cmidrule(lr){2-4} & \textbf{6m}& \textbf{9m} & \textbf{12m}\\
\midrule
\rowcolor{gray!20}\textbf{Movimento (Tattico)}&&&\\
Camminare& 6m & 9m & 12m\\
\rowcolor{gray!20}Correre (x2) & 12m& 18m& 24m\\
\textbf{Un minuto (Locale)}&&& \\
\rowcolor{gray!20}Camminare & 36m& 54m& 72m \\
Correre (x3) & 108m & 162m & 216m \\
\rowcolor{gray!20}\textbf{Un'ora (Via Terra)}&&& \\
Camminare& 3km& 4km& 6km\\
\rowcolor{gray!20}Correre (x3) & 9km& 12km & 18km \\
\textbf{Un giorno (Via Terra)}&&&\\
\rowcolor{gray!20}Camminare& 24km & 32km & 54km
\end{tabularx}

\subsection{Movimento Tattico}\index{Movimento Tattico}\label{movimentotattico}

Durante un combattimento si utilizza il Movimento tattico.
Le distanze vengono misurate in quadretti da un metro, il movimento è gestito tramite le Azioni di Movimento.

Un personaggio può usare 1 Azione (di Movimento) per muoversi fino a tutto il proprio movimento. Può effettuare più volte nel round, fino a 3 volte, l'Azione Movimento, spostandosi quindi del triplo del suo movimento.

Può anche effettuare una Azione di Scatto\index{Scatto} ovvero \textbf{Correre} e quindi muoversi del doppio del suo Movimento in una sola Azione. Incappa così però nelle penalità per chi corre (-1d6 al Tiro per Colpire, -4 Difesa).

Un personaggio può effettuare fino a 3 Azioni di Scatto, ovvero corre per tutto il round percorrendo quindi il suo movimento * 6.

\subsubsection{Movimento Ostacolato - Terreno Difficile}\index{Terreno difficile}\label{terrenodifficile}

Terreno difficile, innevato, ghiacciato, con rapide salite e discese, pieno di macerie o con ostacoli o scarsa visibilità possono impedire i movimenti. Quando il movimento è ostacolato ci si muove a metà della velocità, sono necessarie 2 Azioni per coprire la propria distanza di 9 metri (se si è umano senza ingombro..), oppure con una Azione di Movimento si copre solo 4 metri.

Se esiste più di una condizione particolare, aggiungere tra loro tutti i costi aggiuntivi applicabili, ovvero se un terreno è difficile e ci si muove a carponi significa muoversi di un quarto del proprio movimento. \index{Terreno doppiamente difficile}

In alcune situazioni il movimento è talmente ostacolato\index{Movimento quasi impossibile} che la distanza percorribile per Azione è minima, in tal caso si possono utilizzare tutte e 3 le Azioni per muoversi di solo 1 metro in qualsiasi direzione.

Non applicate questa regola per attraversare terreni impraticabili o per muoversi quando non è possibile farlo in alcun modo.

Non si può \textbf{Scattare} (Correre) su un terreno difficile, il giocatore può tentare una prova di Acrobatica a DC 20 per riuscire a correre ma non a caricare, se fallisce tratta il terreno come Difficile, se Fallisce Criticamente cade prono sul posto. \index{Scatto su terreno difficile}

Non si può \hyperlink{carica}{\textbf{Caricare}}\index{Carica su Terreno difficile} attraverso un terreno difficile a meno di avere l'Abilità \hyperlink{Rinoceronte}{Rinoceronte} (pag. \pageref{Rinoceronte}).

Muoversi da prono\index{Muoversi da prono}\index{Muoversi a carponi}, Nuotare o Strisciare\index{Strisciare} è considerato terreno difficile, Arrampicarsi è doppiamente difficile.

Il terreno dove sono presenti dei corpi di creature si considera difficile. \index{Muoversi su corpi}

\begin{giocatore}[Tups nel cunicolo] %box giocatore
Tups è con i suoi compagni in uno stretto cunicolo in fila indiana. E' alla quarta posizione.

Improvvisamente un nemico si para davanti e Tups è il più veloce a reagire, usando una Azione di Movimento \emph{\textbf{attraversa}} i 3 compagni che ha davanti rimanendo \textbf{ristretto} con il primo della fila.

Potrebbe decidere di (tra le varie possibilità):

\begin{itemize}[leftmargin=*] \setlength{\itemsep}{-1pt}
	\item stare fermo senza scattare in avanti, lasciare agire prima i compagni davanti a lui.
	\item rimanere ristretto ed attaccare. Rimangono 2 Azioni.
	\item spingere il compagno (1 Azione) nel quadretto precedente, facendolo stringere con un altro compagno. Rimane 1 Azione.
	\item spingere il compagno (1 Azione) nel quadretto successivo! facendo attraversare a lui il quadretto del nemico. Rimane 1 Azione.
	\item tornare indietro (1 Azione) al suo quadretto iniziale. Rimane 1 Azione.
	\item provare ad attraversare l'avversario (1 Azione), ma se fallisce sarebbe ristretto con il compagno, danneggiando entrambi e gli rimarrebbe soltanto 1 altra Azione
\end{itemize}

\end{giocatore}

\subsubsection{Condividere gli Spazi}\index{Condividere il quadretto}\index{più creature nello stesso spazio}\label{condividereglispazi}\hypertarget{condividereglispazi}{}

Una creatura di taglia media o più piccola può condividere lo stesso quadretto con una creatura di taglia piccola.

Una creatura di taglia superiore a media può condividere i propri quadretti solo se l'altra creatura è di almeno 2 taglie inferiore.

Es. un mostro di taglia Grande può condividere il suo spazio solo con una creatura di taglia Piccola o inferiore, se fosse Enorme potrebbe condividerlo con una creatura di taglia Media o inferiore.

\subsubsection{Scambiarsi di posto}\index{Scambiarsi di posto}
Un personaggio a contatto con un altra creatura può usare \textbf{una Azione} per \textbf{scambiarsi di posto} con questa. Se la creatura è ostile è necessaria una Prova Atletica contrapposta ad un Tiro Salvezza su Tempra per riuscire a scambiarsi. Per ogni taglia di differenza chi ha quella maggiore prende +1d6 di bonus alla prova. Costa una Reazione alla creatura amichevole.

\begin{narratore}[Amici fastidiosi]
Se volete un crudo realismo allora è terreno difficile attraversare anche zone dove ci sono creature amichevoli. \end{narratore}

\subsubsection{Essere ristretti con qualcuno}\index{Essere ristretti con qualcuno}
Due creature ristrette, ovvero che condividono lo stesso quadretto e non rispettano le regole di \hyperlink{condividereglispazi}{Condividere gli Spazi} subiscono un -1d6 al Tiro per Colpire ed un -4 alla Difesa finché ristretti.\index{Ristretti}\hypertarget{ristretti}{}\label{ristretti}

\subsubsection{Passare per strettoie o restringimenti}\index{Passare per strettoie o restringimenti}\index{Strettoie}\index{Restringimenti}

Passare attraverso uno spazio di una taglia più piccola equivale a muoversi in terreno difficile. Es. una creatura media, che occupa 1 quadretto, che deve passare per una strettoia da mezzo quadretto (mezzo metro) tratta quel percorso come terreno difficile.

Non è possibile attraversare restringimenti più stretti di una taglia.

\subsection{Movimento Locale}\index{Movimento Locale}\label{movimentolocale}

I personaggi che esplorano una zona usano il movimento locale, misurato in metri al minuto.

In queste situazione non è fondamentale misurare la distanza in maniera precisa ma appena la situazione diventa \emph{problematica} o richiede attenzione la mappa si converte in movimento tattico, quadrettata e misurata.

\begin{itemize}[leftmargin=*] \setlength{\itemsep}{0pt}
\item
Camminare: Un personaggio può camminare senza problemi in Movimento locale per 8 ore al giorno.
\item
Correre: Un personaggio può Correre per un numero di minuti pari suo valore di Tiro Salvezza Tempra senza bisogno di riposarsi (minimo 1 round).
\end{itemize}

\subsection{Movimento Via Terra}\index{Movimento Via Terra}\label{movimentoviaterra}

I personaggi che percorrono lunghe distanze usano il movimento Via terra. Il movimento Via terra è misurato in ore o giorni. Un giorno rappresenta 8 ore di tempo di viaggio reale. Per imbarcazioni a remi, un giorno significa remare per 10 ore. Per navi a vela rappresenta 24 ore di movimento.

Camminare più a lungo può sfinire (vedi Marcia forzata, sotto).

\textbf{Andare Veloci}\index{Andare Veloci}\label{andareveloci}

Si può andare veloci (movimento*2) per 1 ora senza problemi. Andare veloci per una seconda ora compresa tra due cicli di sonno provoca 1 Danno Non Letale e ogni ora aggiuntiva provoca il doppio dei danni subiti nell'ora precedente. Un personaggio che subisce Danni Non Letali da andatura veloce è considerato Affaticato per quel giorno.

Un personaggio Affaticato non può Correre o Caricare.

\textbf{Correre}\index{Correre}\label{correre}

Non è possibile Correre per un lungo periodo di tempo. Tentativi di Correre e riposarsi a cicli funzionano come Andare Veloci.

\textbf{Marcia Forzata}\index{Marcia Forzata}\label{marciaforzata}

In un giorno di cammino normale, si può camminare per 8 ore. Il resto del giorno viene sfruttato per fare e disfare il campo, riposarsi e mangiare.

Se si cammina di più di 8 ore è necessario effettuare un Tiro Salvezza su Tempra a difficoltà 11 +1 per ogni giorno consecutivo di marcia forzata o si diventa Affaticati. Il Tiro Salvezza viene effettuato ogni 2 ore oltre le 8 di cammino altrimenti aumenta il livello di Affaticato.

La marcia forzata può essere tenuta per un numero di giorni pari al valore di Costituzione+1 prima di incorrere nell'Affaticamento indipendentemente dall'esito del Tiro Salvezza.

\textbf{Terreno}\index{Terreno}\label{terreno}

Il terreno su cui si viaggia influenza quanta distanza viene percorsa in un'ora o in un giorno. A seconda dell'ambiente, clima, qualità della strada il Narratore può valutare che il movimento può essere normale, ridotto di un terzo, ridotto di metà oppure talmente impervio e difficile da ridurlo ad un quarto del movimento totale possibile.

\textbf{Movimento in sella}\index{Movimento in sella}\label{movimentoacavallo}

Una cavalcatura che porta un cavaliere può muoversi con andatura veloce. Tuttavia, i danni che subisce sono danni normali invece che non letali. Può anche essere costretta a una marcia forzata, ma le sue prove di Costituzione falliscono automaticamente ed i danni che subisce sono danni normali. Anche le cavalcature sono considerate Affaticate quando subiscono danni da andatura veloce o marcia forzata.

\textbf{Bardature da Cavalcatura}\index{Bardature da Cavalcatura}\index{Armature da Saurovallo}\label{ArmaturedaCavallo}\hypertarget{ArmaturedaCavallo}{}

Una cavalcatura può essere bardata con un armatura. Un armatura leggera conferirà un bonus alla Difesa di +2, una armatura Media concederà un bonus di +4 alla Difesa riducendo il movimento del 25\%, una armatura Pesante darà un +6 alla Difesa abbassando il movimento del 33\%.

\end{multicols}

%\medskip
%\begin{center}
%\includegraphics[height=0.3\linewidth]{immagini/carretto.png}
%\end{center}

\subsection{Tabella: Cavalcature e Veicoli}\index{Cavalcature}\index{Veicoli}\index[Tabelle]{Tabella Cavalcature e Veicoli}\index{Saurovallo movimento}\index{Movimento al giorno su saurovallo}\label{TabellaCavalcatureeVeicoli}\hypertarget{TabellaCavalcatureeVeicoli}{}

\medskip

\label{tabella-cavalcature-e-veicoli}\index{Cane}\index{Saurovallonano}\index{Carretto}\index{Zattera}\index{Barca}\index{Nave}\hypertarget{tabella-cavalcature-e-veicoli}{}

\noindent\begin{tabularx}{\linewidth}{llXX}
	\toprule
\rowcolor{gray!20}\textbf{Cavalcatura o} & \textbf{Ingombro trasportato} & \textbf{Movimento} & \textbf{Movimento}\\
\textbf{Veicolo}&\textbf{(CdC)}&\textbf{All'ora} & \textbf{Al giorno}\\
\toprule
\rowcolor{gray!20}Cane da Galoppo & 30 & 6km & 36km \\
Saurovallo da Galoppo& 60& 8km & 48km \\
\rowcolor{gray!20}Saurovallo da Guerra & 80& 7km & 42km \\
Saurovallo Nano& 30& 5km & 30km \\
\rowcolor{gray!20}Saurovallo da Tiro& 70& 5km & 30km \\
Cammello& 50& 8km & 48km \\
\rowcolor{gray!20}Elefante& 160& 6km & 36km \\
\hline
\textbf{Imbarcazione} &&& \\
\hline
\rowcolor{gray!20}Zattera o Chiatta (pertica o rimorchio)& 225 & 0.75km & 7.5km \\
Barcone a Remi**& 425 & 1.5km& 15km\\
\rowcolor{gray!20}Barca a Remi**& 200 & 2.25km & 22.5km\\
Nave a Vela& 800 & 3km& 72km\\
\rowcolor{gray!20}Nave da Guerra (vele e remi) & 2200 & 3.5km & 90km\\
Nave Lunga (vele e remi)& 600& 5km& 108km \\
\rowcolor{gray!20}Galea (remi e vele) & 3300 & 6km& 144km
\end{tabularx}

\begin{multicols}{2}

\bigskip

Una cavalcatura può portare in groppa una creatura solo se di taglia inferiore alla sua. Il movimento al giorno si intende per 6 ore di cavalcata, oltre queste ore la cavalcatura si sfianca richiedendo un giorno intero di riposo.\index{Ore di cavalcata al giorno}

Zattere, chiatte e barconi sono usati su laghi e fiumi. Se seguono la corrente, sommare la velocità della corrente (di solito 4,5 km/h) alla velocità dell'imbarcazione. Oltre a essere spinta con pertiche o remi per 10 ore, l'imbarcazione può anche essere trasportata dalla corrente per altre 14 ore, se qualcuno è in grado di guidarla, e quindi si aggiungono altri 100 km nelle 24 ore. Queste imbarcazioni non possono essere spinte a remi contro una corrente molto forte, ma possono essere tirate controcorrente da animali da soma sulla riva.

Le Zattere e Chiatte attrezzate per il trasporto sono delle piccole locande che permettono un pasto frugale del pescato giornaliero ed un pò di frutta e verdura portata da riva. Non ci sono stanze per dormire. A chi ne fa richiesta, dietro un piccolo compenso, vengono stese delle stuoie e srotolati vissuti materassi e se il clima lo rende necessario vengono fornite coperte.

La guida della Zattera o Chiatta avviene su turni di 8 ore giornalieri, per permettere anche la continua navigazione. Quando è notte la navigazione si ferma o prosegue con la sola forza della corrente se non impetuosa e non ci sono pericoli noti. Pagando un sovrapprezzo è possibile navigare anche sulle 24 ore.

Se il viaggio dura più giorni diventa una occasione di conoscenza tra i personaggi quando nelle lunghe sere si raccolgono assieme agli altri ospiti e marinai per consumare il pasto e raccontarsi storie.

\subsection{Fuga ed Inseguimento}\index{Fuga}\index{Inseguimento}\label{fugainseguimento}

Nel movimento round per round è impossibile per un personaggio lento sfuggire ad un personaggio veloce senza qualche tipo di aiuto. Allo stesso modo non è un problema per un personaggio veloce sfuggire ad uno più lento.

Quando l'inseguimento avviene in città o comunque in un ambiente che permette di nascondersi o fare perdere le proprie tracce, se la velocità dei due personaggi coinvolti è uguale occorre che inseguitore ed inseguito effettuino 3 Tiri Salvezza consecutivi su Riflessi con modificatore Furtività contrapposti. Chi vince la sfida riesce a fare perdere le proprie tracce od agguantare il fuggitivo.

Se l'inseguimento avviene all'aperto dove non c'è modo di nascondersi o fare perdere le proprie tracce eseguite 3 Tiri Salvezza su Tempra contrapposti per determinare quale delle due parti può mantenere più a lungo il ritmo. Chi vince la sfida riesce a seminare l'inseguitore od agguantare il fuggitivo.

\subsection{Capacita' di Carico e Trasporto: Ingombro}\index{Capacità di Carico}\index{Ingombro}\hypertarget{ingombro}{}\label{ingombro}

\label{sec:capacita-di-carico-e-trasporto-ingombro}

\subsubsection{Peso e Ingombro}\index{Ingombro}\index{Peso}

Portare tesori, pezzi di drago, armature complete per non parlare di armi sproporzionate o arieti da sfondamento, carrucole e paranco, rendono difficile il movimento.

Quando valutate il peso trasportato ragionate anche sull'ingombro!
Portare un rotolo di 12 metri x 6 metri di seta non è una attività fisica impegnativa, saranno pochi chili, ma l'ingombro è tale da non poter permettere ulteriore carico.

Ci possono essere oggetti leggeri ma estremamente ingombranti (tronchi cavi, tappeti di seta appunto..) oppure piccoli ma pesantissimi (sfere di mercurio, vestiti intessuti d'oro), per tutti questi oggetti il valore del peso deve essere ragionato anche in funzione dell'ingombro.

Ogni oggetto ha un proprio valore di Ingombro, in linea di massima \textbf{ogni 3 kg si ha 1 come fattore di Ingombro}. Questo valore può diventare anche 5Kg se l'oggetto è facilmente trasportabile. I valori di Ingombro degli oggetti si sommano tra di loro per dare il carico totale portato che si confronta con la Capacità di Carico della creatura.\index{Kili ed Ingombro}

Gli oggetti con scarso peso e volume hanno ingombro \textbf{Leggero} (L). Questi oggetti contano come 1 di Ingombro ogni 10 oggetti. Ogni 500 monete si ha 1 di Ingombro.\index{Ingombro delle monete}

\subsubsection{Capacita' di Carico}\label{capacitadicarico}\index{Capacita' di Carico}

La Capacità di Carico di una creatura è data dalla \textbf{somma di Taglia, Forza e Costituzione}.

La Taglia di una creatura concede un bonus alla \textbf{CdC} (Capacità di Carico) pari a 6 se Piccola, 12 se Media, 24 se Grande. L'Ingombro di una creatura se trascinata\index{Trascinare un corpo} di peso è pari alla metà della sua Capacità di Carico, data dalla taglia, più il suo ingombro.\index{Ingombro Creature trasportate} Se trasportata di peso la CdC sarà pari all'Ingombro che ha la creatura.

Quando la \textbf{CdC totale viene} superata allora muoversi ed effettuare prove di competenza basate sulla Destrezza diventa problematico. Si diviene appesantiti e tutto il terreno viene trattato come difficile e le prove di Competenza basate su Destrezza hanno un -3 di penalità.

Se la CdC \textbf{viene doppiata} allora non ci si può più muovere per l'ingombro dei pesi portati.

Es. Tups ha indosso una Armatura ad Anelli (ingombro 4), una spada lunga (arma media, ingombro 2), una mazza chiodata (ing. 2), 18 oggetti leggeri (ing. 1), uno zaino (ing. 1), una tenda (ing. 2), una lanterna (ing 1). Totale Ingombro = 13.

Tups è una creatura Media con Forza -1 e Costituzione +0 (è un pò gracile e debole..) questo gli concede una Capacità di Carico di 12-1=11.

Il peso trasportato da Tups è superiore alla sua CdC ! E' meglio se lascia la tenda sul suo saurovallo...

Se il carico viene appoggiato su un carro puoi spingerlo a movimento pieno se entro la tua CdC, a metà movimento se entro il doppio della CdC ed ad un quarto del movimento se entro il quadruplo della CdC.

In caso più creature spingano o trainino un carro considerate come CdC quella più alta ed aggiungete metà delle altre creature. Un carro può essere spinto da 2 creature +1 per taglia del carro superiore a media.

\subsubsection{Creature più Grandi e più Piccole}\label{tagliaeportata}

Nella \textbf{Tabella: CdC trasportato in base alla taglia}\index{Ingombro trasportato in base alla taglia} viene riportata la Capacità di Carico in base alla taglia. Al valore dato dalla taglia vanno sommati i valori di Forza e Costituzione.

\medskip

\noindent\begin{tabularx}{\linewidth}{Xl|ll}
	\toprule
\rowcolor{gray!20}\textbf{Taglia}& \textbf{Ing.}&\textbf{Taglia} & \textbf{Ing.}\\
\toprule
Piccolissima &1/4& Grande & 24\\
\rowcolor{gray!20}Minuta & 1 & Enorme& 36\\
Minuscola & 3& Mastodontica&49\\
\rowcolor{gray!20}Piccola & 6 & Colossale&77\\
Media & 12&&
\end{tabularx}

\medskip

Creature con 4 zampe o più possono trasportare carichi maggiori.

%\begin{center}
%\includegraphics[height=0.5\linewidth]{immagini/cavallo.png}
%\end{center}

\textbf{Tabella: modificatori trasporto per creature con più zampe}\index[Tabelle]{Tabella modificatori trasporto per creature con più zampe}

\medskip

\noindent\begin{tabularx}{\linewidth}{Xl}
	\toprule
\rowcolor{gray!20}\textbf{Zampe Creatura}&\textbf{CdC}\\
\toprule
4 zampe & x2\\
\rowcolor{gray!20}6 zampe & x2.5\\
8 zampe & x3\\
\rowcolor{gray!20}12 zampe & x4\\
ogni altre 2 zampe & +0.5
\end{tabularx}

\medskip

Queste Tabelle sono da usare per gli animali insoliti non indicati o assimilabili a quelli nella \hyperlink{TabellaCavalcatureeVeicoli}{Tabella: Cavalcature e Veicoli}, pag. \pageref{TabellaCavalcatureeVeicoli}.

\subsection{Altri Tipi di Movimento}

\begin{enfasi}{
Uno dei problemi riguarda la velocità della luce e le difficoltà che comporta il tentare di superarla. Non la si può superare. Niente viaggia più in fretta della velocità della luce, con la possibile eccezione delle cattive notizie, che seguono proprie leggi specifiche. (Douglas Adams)
}\end{enfasi}

\label{altri-tipi-di-movimento}

\subsubsection{Nuotare}\index{Nuotare}\label{nuotare}

Vedi Capito Ambiente per \hyperlink{pericoli-dellacqua}{prove nuotare} (pag. \pageref{pericoli-dellacqua}) e \hyperlink{combatteresottacqua}{combattimento sott'acqua} (pag. \pageref{combatteresottacqua}).

\subsubsection{Scalare}\index{Scalare}\label{scalare}

Una creatura con una velocità di Scalare ha un bonus di +2d6 su tutte le prove di Arrampicarsi, quando necessario. La creatura se deve fare una prova di Arrampicarsi per arrampicarsi su qualsiasi parete o pendenza può sempre scegliere di prendere 10, anche se di fretta o minacciata durante la salita.

Se una creatura con una velocità di Scalare tenta una scalata rapida (vedi sopra), è come se eseguisse una Azione di Scatto e fa una singola prova di Arrampicarsi a DC 13. Se la creatura non ha un punteggio di Scalare indicato si considera il valore pari al suo GS + Movimento in metri per Scalare.

Una creatura con Velocità di Scalare non ha penalità alla Difesa durante la salita e non ha penalità ai Tiri per Colpire mentre attacca.

Se non si ha il tipo di \textbf{movimento Scalare} si considera come \textbf{terreno doppiamente difficile}, e quindi ci si muove ad un quarto del del Movimento.

\subsubsection{Scavare}\index{Scavare}\label{scavare}

Una creatura con una velocità di Scavare può scavare tunnel attraverso la terra, ma non attraverso la roccia a meno che il testo descrittivo non dica il contrario. Le creature non possono caricare o correre mentre scavano.

La maggior parte delle creature scavatrici non lascia tunnel che altre creature possono utilizzare (sia perché il materiale attraverso cui scavano riempie il tunnel dietro di loro o perché in realtà non spostano materiale quando scavano), vedere la descrizione della singola creatura per i dettagli.

\subsubsection{Camminare - Velocità Su Terreno}

La Velocità sul terreno é la normale velocità per personaggi che non scalano, nuotano o volano.

\subsubsection{Volare}\index{Volare}\label{volare}

Volare per una creatura dotata di questa abilità è come camminare per una creatura terrestre. Una creatura dotata di volo usa le sua azioni per muoversi ma difficilmente sarà influenzata dal terreno difficile.

Una creatura volante che viene danneggiata in un singolo colpo della metà dei suoi Punti Ferita massimi deve fare un Tiro Salvezza su Tempra a DC 17 o cadere a terra.

\subsubsection{Fluttuare}\index{Fluttuare}\label{Fluttuare}\hypertarget{Fluttuare}{}

Fluttuare è la capacità che consente di rimanere fluttuante nell'aria, all'altezza voluta, anche se non ci si muove o si è privi di sensi.

\end{multicols}

\vfill

\begin{center}
\includegraphics[width=0.45\linewidth]{immagini/grifonicastello.png}
\end{center}

\pagebreak

\section{Masterizzare}\index{Masterizzare}\index{Narratore}

\label{masterizzare}

\begin{enfasi}{
Chi comanda al racconto non è la voce: è l'orecchio. (Italo Calvino)

\medskip

Per fare ciò che si vuole bisogna nascere re o stupidi. (\emph{o Narratore}, NdA) (Lucio Anneo Seneca)

\medskip

%"Il gioco di ruolo di Dungeons\& Dragons (\emph{ed anche OBSS}) è sulla narrazione in mondi di spade e stregoneria. Condivide elementi con i giochi di finzione dell'infanzia. Come quei giochi, D\&D è guidato dall'immaginazione. Si tratta di immaginare l'imponente castello sotto il tempestoso cielo notturno e immaginare come un avventuriero fantasy potrebbe reagire alle sfide che la scena presenta." (DnD 5e Basic Rules)

%\medskip

"Non è compito del solo DM (\emph{Narratore}) intrattenere i giocatori e assicurarsi che si divertano. Ogni persona che gioca è responsabile del divertimento del gioco di tutti. Tutti accelerano il gioco, aumentano il dramma, aiutano a stabilire quanto il gruppo si sente a suo agio nel gioco di ruolo e danno vita al mondo di gioco con la loro immaginazione. Tutti dovrebbero anche trattarsi reciprocamente con rispetto e considerazione: i litigi personali e le liti tra i personaggi ostacolano il divertimento.

Persone diverse hanno idee diverse su ciò che è divertente in D\&D. Ricorda che il \emph{modo giusto} di giocare a D\&D è il modo in cui tu e i tuoi giocatori siete d'accordo e vi divertite. Se tutti si mettono al tavolo pronti a contribuire al gioco, tutti si divertono". (Dungeon Master Guide, 4ed)

}\end{enfasi}

\begin{multicols}{2}

\subsection{Il Narratore}

\label{il-narratore}

Mentre il giocatore interpreta un personaggio in un'avventura, il Narratore è colui che la gestisce. Ha certamente molto più lavoro, ma creare un mondo intero affinché i propri amici lo esplorino, può dare molte soddisfazioni.

Il ruolo del Narratore non è facile ma concede enormi privilegi. Vedere i propri amici giocare, divertirsi, ammattirsi dietro dubbi, indovinelli e situazioni da te create dà tantissimo divertimento e momenti di vera convivialità.

Il tuo ruolo è quello del grande orchestratore, pianificatore o anche paesaggista se preferisci, con poche semplici pennellate delinei la struttura e saranno poi i giocatori ad aggiungere dettagli e situazioni.

\begin{narratore}[Divertirsi sempre]
OBSS vuole aiutare te e gli altri giocatori a divertirsi. Usa sempre il buon senso quando devi applicare una regola. Il tuo scopo non è ammazzare i personaggi ma creare mondi e campagne che si evolvono attorno ai personaggi ed al mondo che crei, alle loro azioni e decisioni. Incorpora le cose che interessano i giocatori, tienili partecipi, fagli comprendere che il mondo è vivo e ne fanno parte. Se sei bravo le tue avventure, le situazioni riecheggeranno in altre sessioni e fuori dal tavolo.
\end{narratore}

Il tuo \emph{lavoro e divertimento} è fondamentale ed importantissimo, la bontà della sessione di gioco dipende anche da te. Il tuo scopo è innanzitutto divertirti, essere creativo, improvvisare, recitare, creare ingegnose situazioni. Finché tu ti diverti è estremamente probabile che anche i giocatori si stiano divertendo!

\textbf{Ricorda che non sei tu il protagonista ne l'avventura, ma i personaggi}, non rubare la scena ma come un gran ballo sii il direttore d'orchestra dove gli strumenti sono le possibilità offerte dal OBSS, la musica è l'avventura ed i ballerini i personaggi.

\subsection{Punti Esperienza}\index{Punti Esperienza}\index{PX}

\label{punti-esperienza}

In OBSS i Punti Esperienza che prendono i personaggi servono a determinare il livello e quindi le capacità ed abilità a loro disposizione.

I personaggi prenderanno Punti Esperienza in base ai mostri sconfitti ma anche ad altri fattori quali obiettivi, idee, azioni particolari, difficoltà superate.. ma anche tesori recuperati!

Il suggerimento principale è premiare i personaggi che più si sono impegnati per il gruppo, quelli che maggiormente hanno contribuito al buon esito dell'avventura e della sessione. I Punti Esperienza non misurano solo il successo ma anche la partecipazione al gioco.
E' quindi possibile avere personaggi con Punti Esperienza diversi e potenzialmente anche livelli diversi.

I Punti Esperienza che assegna la sconfitta di un mostro sono indicati nel Mostruario es. Sfida 13 (10000 PX). Questi Punti Esperienza vanno divisi tra tutti i personaggi che hanno partecipato allo scontro in qualsiasi maniera.

La Tabella Punti Esperienza per Livello indica i Punti Esperienza necessari per passare da un livello a successivo.

Non esagerate mai nell'assegnazione dei Punti Esperienza altrimenti rischierete di sbilanciare il gioco e dover modificare anche in maniera significativa l'avventura. Siate chiari anche con i giocatori all'inizio della campagna, nella Sessione Zero, come i Punti Esperienza saranno calcolati, distribuiti e cosa è possibile fare per poterne avere di più.

%\begin{tabularx}{\linewidth}{lX|lX}
%\textbf{Livello} & \textbf{Punti Esperienza}&\textbf{Livello} & \textbf{Punti Esperienza}\\
%\hline
%1&0&11&122970\\
%2&1610&12&198960\\
%3&2620&13&321920\\
%4&4240&14&600000\\
%5&6850&15&520860\\
%6&11090&16&842750\\
%7&17940&17&1363570\\
%8&29030&18&2206260\\
%9&46970&19&3569730\\
%10&76000&20&5775820\\
%&+prec*1.680&&\\
%\end{tabularx}

\medskip

Non dovete però tenere conto solo dei Punti Esperienza concessi dalle sfide ma dovete valutare i personaggi e gruppo durante la sessione.

Ogni qual volta il personaggio od il gruppo:\index{Bonus Punti Esperienza}\label{puntiesperienzabonus}\index{Punti Esperienza Bonus}
\begin{itemize}[leftmargin=*] \setlength{\itemsep}{0pt}
\item \textbf{Raggiunga gli obiettivi prefissati} (premio al gruppo od al personaggio);
\item \textbf{Sfrutti a pieno ed anzi sia alternativo nell'uso delle proprie Abilità e capacità} (premio al personaggio);
\item \textbf{Risolva i problemi in maniera creativa, fantasiosa e funzionale} (premio al personaggio);
\item \textbf{Proponga piani e azioni e funzionanti ed alternative a quanto previsto} (premio al personaggio);
\item \textbf{Scopra o avvii indizi di avventura e creazione di nuovi plot} (premio al personaggio);
\item \textbf{Usi in maniera intelligente e furba una competenza od oggetto} (premio al personaggio);
\item \textbf{Usi in maniera geniale (ed alternativo) un incantesimo} (premio al personaggio);
\item \textbf{Compia una azione che mette a repentaglio la propria vita per il gruppo} (premio al personaggio);
\item \textbf{Compia azioni seguendo il credo del proprio Patrono (per Devoti). Queste dovrebbero dare punti Tratto} (premio al personaggio);
\item \textbf{Converta un PNG, di livello equivalente, al suo Patrono (solo per Devoti)} (premio al personaggio);
\item \textbf{Raccolga almeno 500*Livello in monete d'oro (o tesoro equivalente)} (premio al gruppo, una volta per sessione massimo);
\end{itemize}

\medskip

\textbf{Tabella: Punti Esperienza per Livello}\index[Tabelle]{Tabella Punti Esperienza per Livello}\label{tabellapuntiesperienza}

\noindent\begin{tabularx}{\linewidth}{lX|lX}
	\toprule
 \rowcolor{gray!20}\textbf{Livello} & \textbf{PX}&\textbf{Livello} & \textbf{PX}\\
	\toprule
	1&0&11&300000\\
 \rowcolor{gray!20}2&2000&12&390000\\
	3&8000&13&490000\\
 \rowcolor{gray!20}4&15000&14&600000\\
	5&35000&15&740000\\
 \rowcolor{gray!20}6&60000&16&890000\\
	7&90000&17&1050000\\
 \rowcolor{gray!20}8&120000&18&1250000\\
	9&170000&19&1470000\\
 \rowcolor{gray!20}10&220000&20&1730000\\
	+&prec*0.2&-&-\\
\end{tabularx}

\medskip

Vi suggerisco anche di valutare queste azioni per premiare l'impegno del giocatore
\begin{itemize}[leftmargin=*] \setlength{\itemsep}{0pt}
\item \textbf{Sia collaborativo con gli altri giocatori} (premio al gruppo od al personaggio);
\item \textbf{Aiuti un giocatore in difficoltà} (premio al personaggio od al gruppo);
\end{itemize}

\medskip

\textbf{Concedete 200 Punti Esperienza * Livello del personaggio o Livello medio del gruppo.}

\medskip

%Concedete in Punti Esperienza 1\% dei Punti Esperienza del livello successivo (segnatevi su un foglio le volte che il premio viene guadagnato per poi sommare i Punti Esperienza una sola volta).\index{Premio Punti Esperienza}

Un ulteriore approccio, ma da considerare solo nei gruppi più affiatati e maturi, è a fine sessione chiedere ai giocatori di scegliere chi tra loro ha giocato meglio, in un insieme di ruolo, ispirazione, incisività e collaborazione. Premiate il suo personaggio con 200 PX per Livello del Personaggio.

Questi Punti Esperienza potranno essere assegnati al gruppo e quindi a tutti i personaggi od al singolo personaggio.
Non c'è bisogno di dare i Punti Esperienza a fine sessione di gioco, tenetene traccia e informate i giocatori quando c'è un momento di pausa, di riflessione su quanto accaduto e fatto.
In questo sistema sono necessarie circa 8/12 sessioni per passare di livello, potenzialmente anche molte meno se i giocatori si dimostrano bravi ed affrontano le situazioni in maniera efficace.

Costruite la sessione perché tutti i personaggi possano essere partecipi e nessuno si senta escluso.

%\begin{center}
%\includegraphics[width=0.7\linewidth]{immagini/deathbeowulf.png}

%\emph{Henry Justice Ford}
%\end{center}

Quando dico \emph{incontro} non pensate al semplice scontro con i mostri, per incontro si intende qualsiasi evento di ruolo che sfidi e metta alla prova i personaggi. Questa sfida può essere una arguta discussione con il nobile che non li vuole pagare al termine di una missione, alla sfida di un indovinello, rebus, delle trappole ben piazzate. In base alla difficoltà della sfide ricavate i Punti Esperienza.

Un mostro non deve essere per forza ucciso per averne i Punti Esperienza, è sufficiente sconfiggerlo, catturarlo, vincere in maniera diversa. In caso di ritirata da parte dei personaggi o nemico accordate la metà dei Punti Esperienza previsti per lo scontro se c'è almeno stato il tentativo di sfida.

Nel limite del possibile ogni sessione dovrebbe includere una parte di ruolo, una parte di esplorazione, tre parti di combattimento (anche molte più di tre), una parte di riposo.

\medskip

\begin{narratore}[Esperienza e OSR]
Sia chiaro che nulla vi vieta di approntare un passaggio di livello basato su punti fissi (milestone) durante l'avventura. Vostro il tavolo, vostre le regole!

\medskip

Può sembrare anacronistico quando c'è già in sviluppo la sesta edizione del più famoso gioco di ruolo tornare a premiare i personaggi in base all'oro preso ai mostri.

Posso però garantirvi che qualora il vostro gruppo sia particolarmente \emph{povero} di giocate di ruolo o semplicemente preferisca uno stile più combattivo, sapere che l'oro raccolto equivale ad esperienza può rendere molto più dinamico ed avvincente l'andare in avventura.

OBSS si rifà ai principi dell'OSR e come tale la fase di esplorazione e combattimento ha un proprio peso importante e vitale.
\end{narratore}

\subsection{Incontri}\index{Incontri}\label{incontri}

%\begin{enfasi}Che è la vita senza speranza? Una gittata di dadi fra le tenebre, fra i deliri. (Ambrogio Bazzero)\end{enfasi}


Un incontro è un momento di tensione e speranza, paura e sfida. E' l'occasione di mostrare e manifestare le proprie capacità e di lavorare come gruppo.

Un incontro non è l'occasione per fare sfoggio del proprio potere assoluto, sia come Narratore, che come Giocatore. Il Narratore saprà \st{punire} educare il giocatore che vuole essere oltre il gruppo e non parte di esso.

Troverete nelle pagine seguenti le istruzioni per creare delle sfide facili, medie, alte, straordinarie, mortali ed epiche.

Attraverso gli strumenti forniti dal manuale e dalla vostra esperienza con il gruppo saprete quale livello la sfida propone e ne valuterete sia l'impatto come punti esperienza che come ricompense.

Un incontro è un evento che mette i personaggi di fronte ad un problema specifico che devono risolvere. Molti sono combattimenti con i mostri o i PNG ostili, ma ce ne sono altri tipi: un corridoio irto di trappole, un'interazione politica con un re sospettoso, un passaggio pericoloso sopra un ponticello di corda traballante, un argomento scomodo con un PNG amichevole che ritiene che un personaggio lo abbia tradito, o qualsiasi cosa che aggiunga un pò di drammaticità al gioco.

Rompicapi, sfide interpretative e prove di competenza sono i metodi classici per la risoluzione degli incontri. Gli incontri più complessi da costruire e bilanciare saranno gli incontri di combattimento. Fidatevi del vostro istinto e dei suggerimenti forniti in OBSS.

Uno scontro può anche nascere palesemente sbilanciato, sarà l'accortezza dei giocatori a capire quando scappare!

Nel progettare un incontro di combattimento in primo luogo decidete che livello di sfida volete far fronteggiare ai PG, quindi seguite i punti descritti qui di seguito.

\textbf{Determinare APL}: \index{APL}Determinate il livello medio dei personaggi: questo è il Livello Medio del Gruppo (APL in breve, Average Party Level). Dovreste arrotondate questo valore al numero intero più vicino (questa è una delle poche eccezioni alla regola dell'arrotondamento per difetto).


\medskip

\textbf{Tabella: Determinare gli Incontri}\index[Tabelle]{Tabella Determinare gli Incontri}

\medskip

\noindent\begin{tabularx}{\linewidth}{Xl} % @{} removes extra padding
	\toprule
\rowcolor{gray!20}\textbf{difficoltà} & \textbf{Grado di Sfida}\\
\toprule
Facile& APL\\
\rowcolor{gray!20}Media& APL +1\\
Alta& APL +2\\
\rowcolor{gray!20}Straordinaria& APL +3\\
Mortale& APL +4\\
\rowcolor{gray!20}Epica& APL +6
\end{tabularx}

\medskip

Si noti che questa guida di riferimento alla creazione di un incontro presuppone un gruppo di quattro o cinque personaggi. Se il vostro gruppo ha sei o più giocatori, aggiungete uno al loro livello medio. Se il vostro gruppo contiene tre o meno giocatori, sottraete uno dal loro livello medio. Per esempio, se il vostro gruppo consiste di sei giocatori, due di 5° livello e quattro di 7° livello, il APL è il 7° (38 livelli totali, diviso per sei giocatori, arrotondando all'intero più vicino, ed aggiungendo uno al risultato finale).


%\begin{center}
%\includegraphics[width=0.7\linewidth]{immagini/impegnativa.png}
%
%\emph{Henry Justice Ford}
%\end{center}

\textbf{Determinare il grado di Sfida}: Il Grado di Sfida (GS) è un numero di convenienza usato per indicare i rischi relativi presentati da un mostro, una trappola, un pericolo o un altro incontro: più il grado di Sfida è alto, più pericoloso è l'incontro. Riferitevi alla Tabella: Determinare gli Incontri per determinare il Grado di Sfida che il vostro gruppo dovrebbe affrontare, in base alla difficoltà della sfida che volete e al APL.

\subsubsection{Quanti scontri affrontare}\index{Quanti scontri affrontare}\label{quantiincontri}

Non c'è una risposta unica. E' a vostra scelta, il sistema trova un suo equilibrio tra i 3 ed i 5 scontri al giorno. Ovvio che non devono essere tutti a difficoltà Alta!.

Gli scontri sono alla fine una gestione di risorse da usare contro un nemico. Queste risorse sono i Punti Ferita, gli incantesimi, le pozioni, pergamene ed oggetti consumabili posseduti.

Se piazzate una sfida Straordinaria come primo incontro è probabile che i giocatori poi decidano di riposarsi per recuperare le energie, diversamente potreste optare per stancarli pian piano con incontri medi e poi provarli con una difficoltà maggiore. Ricorda infine che uno \emph{scontro} non deve essere per forza fisico, ma anche trappole, puzzle/indovinelli, sfide alternative.. qualsiasi cosa che faccia consumare risorse e ragionare.

Valutate sempre dove si muovono e cosa c'è intorno, verrà naturale trovare il giusto numero e tipi di scontri e nemici.

\subsubsection{Costruire l'Incontro}\index{Costruire l'Incontro}\label{costruireincontro}

Per costruire un incontro come prima cosa calcolate il valore dell' APL (il livello medio del vostro gruppo).

Per sviluppare il vostro incontro, aggiungete le creature, le trappole ed i pericoli finché non arrivate a vostro APL programmato.

Parti calcolando le sfide con grado di Sfida più alto dell'incontro, completando il resto con sfide minori.

Per esempio, volete che il vostro gruppo di sei personaggi di 7° livello abbia una sfida Media ed affronti alcuni Gargoyle (grado di Sfida 2 ciascuno), degli Xorn (grado di Sfida 5) e il loro capo, un Gigante delle Pietre (grado di Sfida 7). I personaggi hanno APL 8 e la Tabella: Determinare gli Incontri stabilisce che una sfida Media per un APL 8 è un incontro di grado di Sfida 9 (Difficoltà Media = APL+1).

Partendo da un grado di Sfida stabilito (9) seguite questa tabella per stabilire quanti mostri inserire nello scontro.

\medskip

\textbf{Tabella: Peso grado di Sfida per calcolo incontro}\index[Tabelle]{Tabella Peso grado di Sfida per calcolo incontro}

\medskip

\noindent\begin{tabularx}{\linewidth}{X|l|X|l|}
	\toprule
\rowcolor{gray!20}\textbf{Fattore} & \textbf{\% Peso} &\textbf{Fattore} & \textbf{\% Peso}\\
\toprule
-6/-7& 5  & -2 & 65 \\
\rowcolor{gray!20}-5& 10    & -1 & 80 \\
-4& 20    & +0 & 100\\
\rowcolor{gray!20}-3& 45    &    & \\
\end{tabularx}

\medskip

Per \textbf{Fattore} si intende la differenza tra il GS del mostro rispetto al Grado di Sfida scelto. Il Peso è la \% relativa che il mostro apporta per raggiungere l'obiettivo del 100\%.

\textbf{Per raggiungere l'obiettivo dobbiamo sommare \emph{le percentuali} di ogni singolo avversario fino a raggiungere 100, ovvero il 100\% della sfida.}

Nel nostro esempio un Gigante delle Pietre ha grado di Sfida 7, ovvero un grado di Sfida -2 rispetto al nostro obiettivo di difficoltà grado di Sfida 9, lo Xorn ha grado di Sfida 5 ovvero -4 rispetto al grado di Sfida 9, i Gargoyle hanno grado di Sfida 2 ovvero -7 rispetto al grado di Sfida 9.

Un nemico con grado di Sfida -2 ha peso 65, un grado di Sfida -4 ha peso 20, un grado di Sfida -7 ha un peso di 5.

Per raggiungere l'obiettivo di un grado di Sfida 9 metterò 1 grado di Sfida -2 ( ovvero un gigante delle pietre), 1 grado Sfida -4 (ovvero uno Xorn) e 3 grado di Sfida -7 (ovvero tre gargoyle). Il Totale sarà 1*65 (un Gigante di Pietra) + 1*20 (uno Xorn) + 3*5 (tre gargoyle) = 65+20+15 = 100. Obiettivo raggiunto!

Il totale dei Punti Esperienza sarà : 2900+1800+450*3 = 6050 Punti Esperienza / 6 Personaggi = 2015 Punti Esperienza a personaggio!

Avversari con grado di Sfida inferiore a 8 rispetto al APL si contano, pesano, solo se sono superiori a 20 come unità.

Avversari con GS pari a 1/2 considerateli come di GS 1, con GS inferiore ad 1/2 considerateli di GS 0.

\begin{narratore}[Ricordatevi i Tratti]
Ricordatevi al termine di ogni \emph{incontro} o \emph{sfida} di segnarvi i Tratti che hanno caratterizzato le azioni dei personaggi. Questi punti parziali li potete concedere alla prima occasione di riposo dei personaggi.
\end{narratore}

\subsubsection{Scontri troppo veloci}

Un problema a cui potreste andare incontro è che lo scontro si risolve troppo velocemente. Possono esserci diversi motivi ed altrettante soluzioni.

Se i giocatori si aspettano pochi incontri è probabile che useranno le loro migliori risorse ed opzioni subito ad inizio combattimento andando così a sconfiggere velocemente i nemici, in questo caso prendeteli di sorpresa con ondate successivi di nemici.

E' possibile che ci siano troppi pochi nemici e quindi anche se questi sono \emph{forti} canalizzando su di loro tutti gli attacchi risultano facile preda dei personaggi, in questo caso dei gregari o l'impedire di riposare e quindi recuperare Punti Magie ed Punti Ferita sarà di aiuto.

E' ovviamente possibile che lo scontro non sia tarato bene ed effettivamente abbiate bilanciato l'incontro perché sia troppo facile, questo è il caso più facile da risolvere, l'esperienza vi insegnerà a meglio costruire gli incontri vuoi aggiungendo o sostituendo gli avversari.

Ricordate che i \emph{mostri} possono anche loro eseguire Azioni come \hyperlink{spingereavversario}{Spingere}, \hyperlink{afferrareunavversario}{Afferrare}, \hyperlink{farecadereavversario}{Buttare a terra}, \hyperlink{fiancheggiare}{Fiancheggiare}, non limitatevi nelle scelte.

\begin{enfasi}{
L'essenza del mondo è il gioco ... noi giochiamo il serio, giochiamo l'autentico, giochiamo la realtà, il lavoro e la lotta, giochiamo l'amore e la morte e giochiamo perfino il gioco. (Eugen Fink)
}\end{enfasi}

\subsubsection{Lo scontro con il Boss}\index{Lo scontro con il Boss}\index{BBEG}

Quando preparate uno scontro il boss, ovvero con quello che potete definire un nemico significativo che ha un certo peso nello svolgimento della campagna dovete preoccuparvi di rendere interessante la sfida!

Se lo scontro deve essere memorabile non basta piazzare il cattivo, organizzate il tutto perché tutti gli avvenimenti risultino coinvolgenti ed emozionanti.

Organizzate i nemici affinché:

\begin{itemize}[leftmargin=*] \setlength{\itemsep}{-1pt}

\item arrivino in più ondate così che ci sia un senso di falso successo
\item che i nemici arrivino da più parti per non fare concentrare le forze solo da un lato
\item che siano intervallati nemici più o meno ostici così che ci sia un senso di falsa sicurezza
\item che l'ambiente sia significativo e giochi un ruolo importante nel combattimento
\item dividi i personaggi su più fronti
\item fai che l'attacco non sembri un attacco
\item gioca con arguzia e non farti demoralizzare.
\end{itemize}

Ed in ogni caso ricorda sempre: non è uno scontro tra Narratore e Giocatori! L'obiettivo è creare momenti memorabili!!!

\subsubsection{Aggiungere i PNG}\index{Aggiungere i PNG}

Una creatura che possiede livelli, Abilità, competenze, che potrebbe essere un personaggio ma viene gestito dal Narratore si considera un PNG. Queste creature possono svolgere un ruolo molto importante e non vanno usate come semplici mostri. Dategli uno spessore e creerete delle figure indimenticabili.

\subsubsection{Modifiche ad Hoc del grado di Sfida}\index{Modifiche ad Hoc del grado di Sfida}

Mentre potete modificare il grado di Sfida specifico del mostro avanzandolo, applicando modifiche o livelli, potete anche aggiustare la difficoltà dell'incontro applicando modifiche ad hoc all'incontro o alla creatura in sé.

Qui descritti ci sono tre modi aggiuntivi con cui potete alterare la difficoltà dell'incontro.

\medskip

\textbf{Terreno Favorevole ai PG}\index{Terreno Favorevole ai PG}

Un incontro contro un mostro che non è nel suo elemento preferito (come uno Yeti incontrato in una caverna piena di lava, o un Drago enorme incontrato in una stanza molto piccola) da ai personaggi un vantaggio. Considerate l'incontro come se avesse un grado di Sfida più basso del suo grado di Sfida reale.

\medskip

\textbf{Terreno Sfavorevole ai PG}\index{Terreno Sfavorevole ai PG}

I mostri sono progettati con il presupposto che siano incontrati nel loro terreno preferito: incontrare un Aboleth sott'acqua non aumenta il grado di Sfida dell'incontro, anche se nessun personaggio è in grado di respirare sott'acqua.

Se, d'altra parte, il terreno ha un impatto più significativo sull'incontro (come un incontro contro una creatura con Vista Cieca in una zona che sopprime ogni fonte di luce), si possono, aumentare il grado di Sfida dell'incontro fosse di un grado più alto.

\medskip

\textbf{Modifiche all'Equipaggiamento dei PNG}\index{Modifiche all'Equipaggiamento dei PNG}

Potete aumentare o diminuire la difficoltà data dai PNG modificandone l'Equipaggiamento. Un PNG incontrato senza equipaggiamento dovrebbe avere un grado di Sfida ridotto di 1 (a condizione che la perdita di equipaggiamento sia realmente controproducente per il PNG), mentre un PNG che ha un equipaggiamento equivalente a quello di un personaggio (come indicato sulla Tabella: Ricchezza dei Personaggi per Livello) ha un grado di Sfida superiore di 1 al suo grado di Sfida reale.

Occorre prestare attenzione ad assegnare ai PNG questo equipaggiamento supplementare, specie ai livelli più alti, in cui potete consumare l'intero tesoro della vostra avventura in un colpo solo!

\subsubsection{Assegnare i PX}\index{Assegnare i PX}\label{assegnarepuntiesperienza}

I personaggi avanzano di livello sconfiggendo mostri, superando sfide, divertendosi,completando l'avventura ed arraffando tesori: nel farlo guadagnano i Punti Esperienza (PX in breve). Potete assegnare Punti Esperienza non appena una sfida viene superata, ma ciò potrebbero interrompere il flusso del gioco. E' più facile assegnare i punti esperienza alla fine di una sessione di gioco (o più sessioni) che permetta ai personaggi di riflettere su quanto accaduto. Il giocatore può usare il tempo a disposizione fra le sessioni di gioco per aggiornare la scheda.

%\begin{center}
%\includegraphics[width=0.7\linewidth]{immagini/tesoro2.png}
%\end{center}

\subsection{Ricchezza dei Personaggi per Livello}\index{Richezza per Livello}

\textbf{Tabella: Ricchezza dei Personaggi per Livello}\index[Tabelle]{Tabella Ricchezza dei Personaggi per Livello}

\medskip

\noindent\begin{tabularx}{\linewidth}{lX|lX}
	\toprule
\rowcolor{gray!20}\textbf{Livello} & \textbf{Ricchezza (mo)} & \textbf{Livello} & \textbf{Ricchezza (mo)}\\
\toprule
1 & 100 & 11 & 13900\\
\rowcolor{gray!20}2 & 160 & 12 & 19900\\
3 & 220 & 13 & 25900\\
\rowcolor{gray!20}4 & 340 & 14 & 37900\\
5 & 530 & 15 & 49800\\
\rowcolor{gray!20}6 & 2030 & 16 & 67700\\
7 & 3660 & 17 & 85700\\
\rowcolor{gray!20}8 & 5780 & 18 & 142000\\
9 & 8100 & 19 & 253000\\
\rowcolor{gray!20}10 & 11000 & 20 & 365000
\end{tabularx}

\medskip

La \textbf{Tabella: Ricchezza dei Personaggi per Livello} per Livello indica la quantità di monete d'oro equivalenti in tesori ed oggetti che ogni personaggio dovrebbe avere ad un livello specifico. Si noti che questa tabella si basa su un modello standard di gioco.

Le avventure con magia rara potrebbero assegnare soltanto la metà di questo valore, mentre avventure più epiche potrebbero raddoppiarlo. Si presume che parte del tesoro sia consumato nel corso di un'avventura (come pozioni e pergamene) e che alcuni degli oggetti meno utilizzati siano venduti per metà del loro valore per acquistare un equipaggiamento più utile.

La Tabella: Ricchezza dei Personaggi per Livello può anche essere usata per stabilire l'equipaggiamento per i personaggi che cominciano dopo il 1° livello, come un nuovo personaggio creato per sostituirne uno morto. I personaggi non dovrebbero spendere più di un terzo della loro ricchezza totale su un singolo oggetto.

Per un metodo equilibrato, i personaggi che vengono creati dopo il 1° livello dovrebbero spendere il 25\% della loro ricchezza per le armi, il 25\% per armatura e oggetti di protezione, il 25\% per altri oggetti magici, il 15\% per oggetti che si consumano come bacchette, pergamene e pozioni e il 10\% per un equipaggiamento normale e monete. Tipi di personaggio differenti potrebbero spendere diversamente la loro ricchezza rispetto a come suggerito; ad esempio, gli incantatori arcani potrebbero spendere di più per oggetti magici e a consumo che per le armi.

\subsection{Io conosco un tizio...}\index{Io conosco un tizio}\label{ioconoscountizio}\hypertarget{ioconoscountizio}{}

Per agevolare lo spirito di avventura e non lasciare i personaggi incapaci o indecisi nell'agire, permettetegli di conoscere un certo numero di PNG pari al loro punteggio di Carisma +1. Il giocatore in qualsiasi momento potrà dichiarare di conoscere questo PNG e dovrà tenerne traccia. Questi PNG potranno essere \emph{sfruttati} quando i personaggi si trovano in situazioni difficili, di pericolo o semplicemente bisognosi si supporto. Il personaggio che si appella al \emph{io conosco un tizio...} deve descrivere adeguatamente il soggetto ed il rapporto che c'è tra loro. Il Narratore adatterà la situazione per includere questo personaggio al meglio delle possibilità.

Il tizio potrebbe essere un commerciante che gli deve un favore, se non un ladro od un burocrate. I personaggi sono invitati a non inventarsi amicizie o favori da personaggi troppo importanti.

\subsection{Recitare}\index{Recitare}\label{ruolare}

\label{recitare}

Un gioco di ruolo non è un semplice tirare dadi, è un incontro di pensieri, opinioni, sfide, lotte. E' un gioco catartico, liberatorio, evolutivo ed istruttivo.

E' giusto che ci sia combattimento, lotta, sangue paura ed azione, allo stesso modo deve esserci la possibilità di giocare i propri personaggi con i loro svantaggi, vantaggi, poteri e storie e anche drammi personali.

Il giocatore deve sempre impersonare il personaggio, immedesimarsi e partecipare attivamente.

Ci possono essere situazioni di contorno, gestite velocemente, che vengono fatte in terza persona, eppure ogni volta che si renda necessario giocare allora deve essere vero, fatto dal giocatore calandosi appieno nel personaggio.

\medskip

\textbf{Quando un giocatore interpreta bene e descrive l'azione che va a svolgere in maniera \textbf{partecipativa}, \textbf{coinvolgente}, \textbf{ispirata}, dategli un premio, concedete un bonus di +1 all'azione che sta svolgendo}

\medskip

Fatelo presente al giocatore che grazie alla sua interpretazione ha quel bonus.

Allo stesso tempo potrebbero esserci situazioni che si rivelano sgradevoli da gestire e giocare per qualche giocatore. Fate molta attenzione in questo caso, andare contro la sensibilità di un giocatore, di un amico, non è come andare contro l'etica o morale di un personaggio. Se percepite un senso di disagio ed imbarazzo fermate subito il gioco e chiarite la situazione con i giocatori e riprendete solo quando vi sarete accordati su come modificare la situazione per evitare che accada di nuovo.\index{Veli e Divieti}\index{Giocare e non spaventare}

\begin{enfasi}
{Concentratevi sulle persone, non sulle regole. Spingete per uno stile di gioco di gruppo; l'interpretazione è divertente ma non deve ostacolare il piano; sostenete i vostri compagni. (Frank Mentzer)}\end{enfasi}

\subsection{Cambiare Personaggio}\index{Cambiare personaggio}

Per quanto il sistema favorisca la libertà di costruzione e sviluppo del personaggio se un giocatore è in difficoltà con il personaggio creato permettetegli, entro il 4 livello di cambiare personaggio e crearne uno nuovo. Ricordate che l'obiettivo è divertirsi tutti.

\subsection{Circa OBSS ed i tiro di dadi}\index{Circa OBSS ed i tiro di dadi}\label{obssedadi}

OBSS usa un sistema di tiro di dadi peculiare andando a mescolare una distribuzione 3d6 ad il potenziale dei 6 che esplodono. Questo sistema riesce a garantire una buona varianza e pur se concentrando i risultati intorno ai valori centrali della distribuzione lascia aperto il limite superiore a tiri particolarmente fortunati.

Se volete divertirvi a studiare la curva corrispondente vi consiglio il sito\href{https://anydice.com/}{Anydice}. Questo lo pseudo codice da inserire (o cliccate \href{https://anydice.com/program/2610e}{qui} per il codice già inserito):

\medskip

\noindent{

function: explode ROLLED:n \{

if ROLLED = 6 \{ result: 6 + [explode d6] \}

if ROLLED = 1 \{ result: 0 \}

result: ROLLED\}

output 3d[explode d6]}

\medskip

oppure cliccate \href{https://anydice.com/program/2610e}{qui} per il codice già inserito.

\subsubsection{Opzionale - Variante Consumo Risorse}\index{Opzionale - Variante Consumo Risorse}\label{varianteconsumorisorse}\hypertarget{varianteconsumorisorse}{}

Ogni qual volta il personaggio usi delle risorse \emph{contate}, quali Frecce, Razioni di cibo, Torce, se non si ha pressione di fare consumare gli oggetti si può optare per questa regola opzionale.

Al termine di un combattimento, dopo una giornata di avventura, il giocatore tira 1d12 per ogni tipo di risorsa che ha consumato. Se fa 1 o 2 con il dado ha diminuito la sua scorta.
La volta dopo tirerà invece che 1d12 un 1d10 e poi 1d8 e poi 1d6 e poi 1d4. Quando arriva a tirare il d4 e fa 1 o 2 ha finito completamente la risorsa e deve ricomprare 20 frecce, 7 giorni di cibo, 6 torce...

Nella scheda a fianco a quelle risorse segna il dado da usare per il successivo tiro.

Il Narratore potrebbe decidere di tracciare in questa maniera i soli consumabili comuni e non di quelli di pregio.

\subsection{Le avventure in OBSS} \hypertarget{OSR}{} \index{OSR}\index{Avventura in OBSS}\label{avventureinobss}

Suggerisco la lettura integrale dell'articolo: \href{https://lithyscaphe.blogspot.com/p/principia-apocrypha.html} {Principia Apocrypha}

https://lithyscaphe.blogspot.com/p/principia-apocrypha.html quello che segue è un sunto da me adattato e modificato delle linee guida che seguo quando masterizzo OBSS.

OBSS segue i principi dell' \href{https://it.wikipedia.org/wiki/Old_School_Renaissance}{OSR} (wikipedia). Le avventure in OBSS mirano ad essere letali, avere un mondo liberamente esplorabile, una trama abbozzata, spingere sul problem-solving ed avere un sistema di ricompense incentrato sull'esplorazione, sui tesori e sulla partecipazione al gruppo. OBSS non si cura troppo del bilanciamento degli incontri e apprezza l'intraprendenza dei giocatori e cattura le loro idee mettendole nell'avventura.

Per me l'OSR non sono tabelle di incontri casuali e randomizzazione caotica ne un regolamento specifico, è piuttosto lo spirito di avventura, meraviglia, paura, gloria, stupore e sfida che si sviluppa nelle avventure. Non siate troppo lineari, troppo prevedibili, aggiungete nelle avventure quel giusto mix che le rendono sempre uniche.

Se il metodo può non piacere usate quello che più vi aggrada, personalmente nel corso dei decenni ho imparato ad apprezzare e vedere apprezzato la spontaneità e naturalezza che i cardini dell'OSR portano nel gioco.

\bigskip

\textbf{Queste sono regole di base per il Narratore che suggerisco per condurre le avventure.}\index{Linee guida per i Narratori}\index{Principi OSR}

\medskip

\begin{itemize}[leftmargin=*] \setlength{\itemsep}{0pt}

\item
Tu sei il Narratore, tue le Regole, tuo il Mondo.

Non farti limitare dall'avventura, dal sistema, dall'elenco dei mostri, sentiti sempre libero di modificare e adattare in base alle necessità dell'avventura e del gruppo

\item
Ricordati di esser giusto e corretto. Improvvisa, adatta quanto vuoi ma sii coerente. Se stabilisci una regola (od una modifica ad una regola) seguila fino in fondo.

Allo stesso tempo se ti serve una regola e non la trovi usa il buon senso, è sicuramente la scelta giusta in quel momento.

Rispetta i dadi ed i risultati ottenuti, come capiteranno ai giocatori capiteranno risultati particolari anche a te. E' giusto così.

\item
Non devi salvare il \emph{culo} ai personaggi. Non sei il loro amico ne il loro nemico. Il tuo ruolo è di raccontare storie che nascono dalle storie dei personaggi, dalle loro azioni ed inazioni.

\item
Abbozza la storia, scrivi le parti centrali o da leggere ai giocatori ma non farti dominare o vincolare da quello che ti aspetti. Spesso e volentieri i giocatori ti stupiranno, meglio sapere dove si muovono e cosa hanno intorno per poter reagire sempre puntualmente.

Sono i giocatori a dare la direzione all'avventura e tu a dipanarla.

\item
Apprezza il caso e crea situazioni diverse dove i giocatori possono scegliere strade diverse o intrecciarne di nuove. E' la tua fortuna avere dei giocatori creativi che sanno sorprenderti.

\item
Non costringere nessuno a fare qualcosa, lascia sbagliare i giocatori, lascia che paghino le loro scelte. Non devi ostacolarli ne devi imbeccarli per una direzioni. Richiede da parte tua una immaginazione e capacità di adattamento non indifferente, ma sicuramente l'avventura ed il divertimento ne gioverà.

\item
I personaggi sono esploratori, per definizione. Focalizza sull'esplorazione, più si esplora più si creano situazioni, più si creano agganci nell'avventura, più si conoscono altri png più ci sono zone da esplorare.

Fa capire che i tesori sono esperienza, in senso letterale e pratico. Non dovrai mai spingerli tu in un dungeon ma la loro brama di esperienza e tesoro.

\item
Fai risolvere i problemi ai giocatori e non ai personaggi. Lascia ruolare le scene, sono sempre meglio di un tiro di dado. Incoraggia il giocatori ad interagire e chiedere una prova solo come ultima chance. Proponi problemi che non debbano essere risolti per forza con un tiro di dado bensì piuttosto tramite più azioni, anche complesse.

Premia le azioni creative e le scelte coraggiose prima più di tutto l'arguzia e il volere trovare situazioni alternative e creative.

\item
Fa che i giocatori ti chiedano informazioni, si confrontino con l'ambiente e tra di loro. Incoraggia l'interazione con il mondo esterno e solo come ultima possibilità concedi un tiro di dado.

\item
Grandi sfide e rischi danno sempre grandi ricompense. Non deludere i giocatori (se non per scopo di avventura) negandogli il giusto tesoro o esperienza, più si addentreranno in profondità più i pericoli saranno letali maggiore sarà la ricompensa (Legge del Premio).

\item
Non devono esistere abitudini o consuetudini. Non creare uno standard.
Cerca sempre di sorprendere i giocatori con mostri fuori luogo (ma che abbiano un senso), trappole anomale, ambienti alternativi. Situazioni diverse stimoleranno i giocatori a risolvere in maniera diversa ogni problema.

Prepara soluzioni diverse e accetta soluzioni diverse. Metti nell'avventura problemi e situazioni che nell'insieme permettano la soluzione, ogni stanza non dovrà essere un asettico ambiente ma contenere indizi e soluzioni per altri problemi anche senza una diretta soluzione.

\item
Accetta la morte. Un combattimento se tale è sempre letale, non avere paura di ferire o uccidere i personaggi. Falli ragionare, studiare il nemico, capire quale è il migliore approccio; ed infine stupiscili. I personaggi devono prima battere i nemici in astuzia e pianificazione, se vogliono sopravvivere.
Se proteggi i personaggi la partita mancherà di tensione e i giocatori risolveranno tutti i problemi con la forza bruta.
%I dungeon non devono essere ambienti per forza da svuotare dai mostri. Lo scopo dei mostri e limitare e orientare le azioni, consumare le opzioni.

Se i giocatori cercano sempre e comunque lo scontro frontale allora daglielo, come richiedono.

\item
Mantieni l'attenzione alta. Fa in modo che il passare del tempo abbia conseguenze, se i giocatori temono lo scorrere del tempo faranno scelte più ardite o forse sbagliate. Mantieni la tensione fra il desiderio di esplorare e fare bottino e il terrore di restare fermi troppo a lungo.

\item
Tu sei la sorgente delle informazioni, i giocatori le elaborano, i personaggi le usano.

Non nascondere informazioni che i personaggi devono sapere o sanno già, non dovrai fare il professore ma allo stesso modo fa in modo che siano consapevoli di ciò che hanno intorno.
Allo stesso tempo non devi rivelare tutto subito, falli indagare, curiosare. Come una cipolla le informazioni che otterranno saranno nascoste sotto strati di altre informazioni magari di minore importanza.

\item
Gli indizi creano situazioni. Lascia che i tuoi indizi, specifici e curiosi, attirino l'attenzione dei giocatori.% Come un esca su un amo attira i giocatori in situazioni di dubbio, dove indagare e capire cosa succede.

Non infarcire l'avventura di dettagli inutili, lascia spazio alla creatività e immaginazione dei giocatori. %, i dettagli che però fornirai dovranno non solo avere un senso ma essere necessari all'avventura.

\item
Se i giocatori tendono a dimenticare le informazioni utili date cerca di sfruttare un PNG che abbia memoria o invitali a prendere appunti, non c'è nulla di male nell'essere preparati.

\item
L'avventura non è mai statica ne tanto meno il mondo dove si muovono i personaggi.
Il mondo ha la stessa importanza se non di più dell'avventura stessa. Azioni dei giocatori possono scatenare accadimenti a livello globale. Pensate sempre alle conseguenze dei gesti.

\item
Se usi i PNG non farli essere delle semplici macchiette, fa in modo che i personaggi si possano affezionare e considerare il PNG uno del gruppo alla pari di tutti gli altri.

\item
I mostri non devono essere stupidi per forza. Falli parlare, ragionare, scappare.. anche loro vogliono vivere!

\item
Ricordati la Legge del Premio. Premia gli audaci, premia che si spinge più in profondità nelle caverne. Premia chi sopravvive.

\end{itemize}

\subsection{Sessione Zero}\index{Sessione Zero}\label{sessionezero}\hypertarget{sessionezero}{}

La Sessione Zero, la prima sessione di gioco, ha una valenza ed importanza particolare. Può essere la sessione in cui ci si conosce per la prima volta, spesso è la sessione in cui si creano i personaggi che si andranno a giocare, sempre è la sessione in cui si vanno a stabilire le regole ed aspettative comuni.

La Sessione Zero serve a stabilire cosa e come si andrà a giocare, quali saranno le principali caratteristiche della campagna e del gruppo che si va a creare.

Per partire bene come gruppo di giocatori è importante conoscersi personalmente e avere fiducia e rispetto negli altri. Non devi dire tutto di te ma almeno le passioni, interessi, curiosità, ciò che almeno all'inizio serve a creare fiducia.

Suggerisco ai Narratori di stabilire delle regole chiare per il buon gioco. Purtroppo l'esperienza insegna che siamo tutte persone diverse con stili, prospettive ed aspettative diverse. Conoscersi serve anche a questo, a capire se il proprio personaggio può stare bene insieme agli altri e capire se la propria persona e personalità è in qualche maniera affine o meno alle altre persone.

\textbf{Il Narratore prima di incominciare è opportuno che chiarisca quali sono le regole essenziali al suo tavolo}. Un esempio di regole possono essere:

\begin{itemize}[leftmargin=*] \setlength{\itemsep}{0pt}

\item Che ogni giocatore \textbf{conosca} la parte del regolamento del manuale che maggiormente andrà ad usare (combattimento, magia, patroni...).
\item Rispetti i \textbf{limiti} degli altri. Ogni persona ha una diversa sensibilità a certi argomenti (stupri, schiavitù, razzismo, violenza...) è fondamentale che si chiarisca insieme quali sono i limiti da non superare mai.
\item \textbf{Rispetta} ogni persona con cui giochi. Ciò include essere puntuali e non annullare senza un motivo importante.
\item I giocatori devono creare un gruppo \textbf{coeso} fatto da individualità che collaborano.
\end{itemize}

\textbf{Vanno condivise e stabilite le informazioni base dell'avventura.}

\medskip

\begin{itemize}[leftmargin=*] \setlength{\itemsep}{0pt}
\item Introduci in linea di massima la campagna o avventure che si andranno a svolgere. Indica la tipologia (eroica, dark, gothic, horror, politica, caverne infinite, esplorazione, sopravvivenza..) e grado di difficoltà.
\item Introduci le informazioni necessarie relative all'ambientazione o fornisci dispense e manuali sull'argomento. Indica se ci sono delle Abilità suggerite.
\item Stabilite le regole opzionali e che siano chiare a tutti.
\item Indica la lista o la tipologia di Tratti accettati e se ci sono dei limiti nella scelta dei Patroni.
\item Incentiva la creazione di personaggi con background condivisi così che il gruppo sia già coeso alla formazione.
\item Come Narratore devi comprendere il grado di conoscenza del sistema da parte dei giocatori e se necessario quando serve soffermarti a spiegare le regole.
\end{itemize}

\textbf{Altre indicazioni utili riguardano}:

\medskip

\begin{itemize}[leftmargin=*] \setlength{\itemsep}{0pt}
\item Cosa è permesso portare ed usare al tavolo e cosa no (bibite, mangiare, cellulari, alcolici, fumare..). Sapere se ci sono animali in casa.
\item Stabilite il numero minimo di giocatori per fare la sessione, giorno di gioco ed orari.
\end{itemize}

In definitiva, la Sessione Zero è fondamentale per stabilire una solida base per il buon gioco di ruolo. Aiuta a creare un ambiente collaborativo in cui tutti si sentono partecipi e contribuisce a evitare problemi e disaccordi durante il corso della campagna.

Anche nel miglior gruppo già affiatato è sempre bene ricordare e condividere questi suggerimenti ad ogni inizio di campagna.

\end{multicols}

%\vfill

%\begin{center}
%\includegraphics[width=0.95\linewidth]{immagini/fognelondra.png}

%\emph{Mappa delle fogne di Londra, 1880}

%\emph{fondamentale per tutti i cacciatori di ratti...}
%\end{center}

%\bigskip

\vfill

\begin{enfasi}
{
I problemi più complessi hanno soluzioni semplici e facili da comprendere ma sbagliate (Arthur Bloch).... ma se sono divertenti e piacciono a tutti allora usatele! (NdA)
}
\end{enfasi}

%\vfill

%\begin{center}
%\includegraphics[keepaspectratio,width=0.55\textwidth]{immagini/dungeonsample.png}
%
%\emph{Dettaglio di un dungeon}
%\end{center}

\pagebreak

\section{I Tesori}\index{Disporre Tesori}\label{disporretesori}

\begin{multicols}{2}

Mentre i personaggi avanzano di livello anche la quantità di tesori che trasportano ed usano aumenta. In OBSS si suppone che tutti i personaggi di pari livello abbiano più o meno la stessa quantità di tesoro e oggetti magici. Poiché il reddito primario per un personaggio deriva dai tesori e dai bottini ricavati dalle avventure è importante stare attenti alla ricchezza e i tesori delle avventure.

\subsection{Dove sono i Tesori}

Alcuni tesori li avranno i mostri (vedi sotto), altri saranno dispersi e nascosti nel dungeon ed altri ancora saranno sul fondo di trappole e tunnel nascosti.

Come distribuire un tesoro è una faccenda importante. I tesori non devono essere sbattuti in faccia ai personaggi, ne tanto meno nascosti che non è possibile trovarli.

Un consiglio è fare in modo che i tesori (oggetti e monete) trovati nei dungeon siano distribuiti secondo questo criterio:

\begin{itemize}[leftmargin=*] \setlength{\itemsep}{0pt}
\item \textbf{un terzo li avranno i mostri addosso}
\item \textbf{un terzo saranno nascosti dietro passaggi segreti o trappole}
\item \textbf{un terzo sarà sparso in giro}
\end{itemize}

Questo stimolerà i giocatori a continuare l'esplorazione, affrontare i mostri e cercare attivamente nel dungeon.

Animali, Vegetali, Costrutti, Non Morti non intelligenti, Melme e trappole sono ottimi \emph{incontri con poco tesoro}.
In alternativa, se i personaggi affrontano un certo numero di creature con poco o nessun tesoro, dovrebbero avere l'occasione di ottenere un certo numero di oggetti di valore più significativo nell'immediato futuro per compensare lo squilibrio.

Come regola generale, i personaggi non dovrebbero possedere alcun oggetto magico di valore superiore alla metà della ricchezza totale del personaggio, pertanto controllate bene prima di ricompensare i personaggi con oggetti molti costosi.

\subsection{Costruire un Bottino}\index{Costruire un Bottino}\label{costruireunbottino}

Spesso è sufficiente dire ai vostri giocatori che hanno trovato 5000 mo in gemme e 10000 mo in gioielli, ma è più interessante fornire dei particolari. Dare a un tesoro una personalità può non solo aiutare la verosimiglianza del gioco, ma può a volte innescare nuove avventure.

Nelle pagine seguenti troverete le regole e tabelle per attribuire i tesori ai nemici così da poter generare casualmente cosa trovano i personaggi.

\subsubsection{Monete e Gemme}\index{Elenco Gemme}\index{Gemme}\label{gemme}

\textbf{Monete}: Le monete in un tesoro possono essere di rame, argento, oro e platino: quelle d'argento e d'oro sono le più comuni, ma potete decidere diversamente. Per le monete ed il loro valore di cambio andate all'Equipaggiamento.

Le monete in possesso di mostri e creature selvagge non saranno certo fior di conio e saranno probabilmente segnate da morsi o bave appiccicose. Monete invece trovate nei tesori o in fondo a qualche tana potrebbero essere di altri regni, se non mondi.. ed in quel caso quello che li fa valere è lo stretto valore metallurgico. 10 grammi di oro sono sempre 10 grammi di oro anche se su una faccia della moneta c'è un fiore ed in un altra un castello.

Usate la Tabella \hyperlink{valoredellegemme}{Valore delle gemme} (pag. \pageref{valoredellegemme}) per determinare il valore delle gemme trovate. Qui sono elencate le gemme per valore.

\textbf{Gemme}: Anche se potete assegnare qualsiasi valore ad una gemma, alcune possono valere di più delle altre. Utilizzate le categorie di valore qui sotto (e le pietre preziose associate) come guida di riferimento quando assegnate i valori alle pietre preziose. Solitamente le gemme vengono vendute ed acquistate a valore pieno.

\textbf{Gemme di Bassa Qualità} (10 mo): agata; azzurrite; quarzo blu; ematite; lapislazzuli; malachite; ossidiana; rodocrosite; occhio di tigre; turchese; perla di fiume (irregolare).

\textbf{Gemme Semi Preziose} (50 mo): eliotropio, corniola; calcedonio; crisoprasio; citrino; diaspro; lunaria; onice; crisolito; cristallo di roccia (quarzo chiaro); sardonica; sardonice; quarzo rosato, affumicato o rosa di stella; zircone.

\textbf{Pietre Preziose di Media Qualità} (100 mo): ambra; ametista; crisoberillo; corallo; granato rosso o verde-marrone; giada; giaietto; perla bianca, dorata, rosa o argentata; spinello rosso, marrone-rosso o verde scuro; tormalina.

\textbf{Pietre Preziose di Alta Qualità} (500 mo): alessandrite; acquamarina; granato viola; perla nera; spinello blu scuro; topazio giallo oro.

\textbf{Gioielli} (1000 mo): smeraldo; opale bianco, nero, o di fuoco; zaffiro blu; corindone giallo fuoco o vermiglio; zaffiro a stella blu o nero; tanzanite.

\textbf{Gioielli Eccezionali} (5000 mo o più): smeraldo verde brillante, diamante, giacinto, rubino, miele topetto cristallino.

\textbf{Tesori non Magici} Questa categoria include monili, abiti raffinati, merci, oggetti alchemici, oggetti perfetti e altri.

Diversamente delle gemme, molti di questi oggetti hanno valori stabiliti, ma potete sempre aumentare il valore dell'oggetto decorandolo con pietre preziose o con fatture particolarmente artistiche.

\textbf{Oggetti d'Arte Raffinati} (100 mo o più): Anche se alcuni oggetti d'arte sono composti di materiali preziosi, il valore della maggior parte di pitture, sculture, opere letterarie, abiti raffinati, e simili consiste nella fattura con cui sono realizzati e nella bravura di chi li ha realizzati. Gli oggetti d'arte sono spesso ingombranti o difficili da spostare, e fragili, rendendone il recupero ed il trasporto un'avventura a sé.

\textbf{Monili Minori} (50 mo): Questa categoria comprende monili realizzati con materiali come ottone, bronzo, rame, avorio, o legni esotici, a volte impreziositi con gemme di bassa qualità molto piccole o difettate. I monili minori includono anelli, braccialetti e orecchini.

\textbf{Monili Normali} (100-500 mo): La maggior parte dei monili è realizzata con argento, oro, giada, o corallo, e decorata spesso con gemme semi preziose o pietre preziose di qualità media. I monili normali comprendono tutti i tipi di monili minori più bracciali, collane e spille.

\begin{narratore}[Attenzione ai tesori]
Non esagerate mai con i tesori, specialmente quelli magici. Un tesoro non deve diventare un abitudine. Un conto possono essere le monete, gemme e consumabili un conto sono i veri tesori, quelli magici, particolari, unici.

Rispettare la Legge del Premio non significa riempire le tasche ai personaggi, altrimenti gli verrà a noia il rischiare la vita per nuovi tesori ed oggetti. Quando fate trovare un oggetto magico ragionate sempre in prospettiva. E' vero che può essere bello vedere i giocatori felici per quello che hanno trovato ma poi sarete costretti l'avventura successiva a dare qualcosa di ancora più potente.
\end{narratore}

\textbf{Monili Preziosi} (500 mo o più): I monili preziosi sono realizzati in oro, mithral, platino, o simili metalli rari. Tali oggetti comprendono i tipi di monili normali più scettri, pendenti ed altri grandi oggetti.

\textbf{Attrezzi fatti ottimamente} (100-300 mo): Questa categoria include attrezzi per le Professioni o Competenze: vedi Equipaggiamento per i dettagli e i costi di questi oggetti.

\textbf{Oggetti Comuni} (fino a 1000 mo): Ci sono molti oggetti di valore di natura alchemica o comune che possono essere utilizzati come tesoro. La maggior parte degli oggetti alchemici sono oggetti portabili e stimabili, ma anche altri come serrature, simboli sacri, cannocchiali, vini prelibati o abiti raffinati possono costituire parti interessanti di un tesoro. Anche le merci commerciali possono servire da tesoro: 5 kg di zafferano, per esempio, valgono 150 mo.

\textbf{Mappe del Tesoro e Oggetti d'Informazione} (variabili): Gli oggetti come mappe del tesoro, documenti legali di navi e case, liste di informatori o dei turni di guardia, parole d'accesso, e simili possono essere divertenti oggetti da trovare in un tesoro: potete stabilire il valore di questi oggetti come volete e possono essere di doppia utilità in quanto possono generare idee per nuove avventure.

\textbf{Tesori Accidentali}: sono i tesori che la creatura ha con se o nella tana per puro caso o perché non voluti. Possono essere i resti \emph{non digeriti} di un lauto pasto o qualcosa che ha attirato l'attenzione della creatura. Un tesoro accidentale \index{Tesoro Accidentale} va valutato caso per caso a seconda dell'ambiente e creatura che lo possiede.

\subsection{Oggetti Magici}

Naturalmente, la scoperta di un Oggetto Magico è il vero premio per qualsiasi avventuriero. Fate attenzione a collocare gli Oggetti Magici in un tesoro: è molto più soddisfacente per molti giocatori trovare un oggetto magico piuttosto che comprarlo.

Anche se in genere dovreste collocare gli oggetti con attenta riflessione sui loro probabili effetti sulla vostra campagna, può essere divertente generare gli oggetti magici in un tesoro a caso. Fate attenzione, comunque! è facile, con un pò di fortuna (o sfortuna) dei dadi gonfiare il vostro gioco con troppo tesoro o privarlo dello stesso. Il collocamento di oggetti magici casuali dovrebbe essere temperato sempre dal buon senso del Narratore.

Anche gli incantesimi sono veri e propri tesori e premi al pari di oggetti magici. Valutate con attenzione quali possono essere trovati. Ricordate che una abilità magica non è un incantesimo copiabile, solo quelli presenti nei tomi, pergamene e quant'altro appositamente creato per essere un ricettacolo di incantesimi è idoneo alla copia.

\subsubsection{Tesori Magici}\index{Tesori Magici}\index{Trovare tesori magici}\label{tesorimagici}

I Tesori magici possono essere trovati dai personaggi in tre modi:

\medskip

- \textbf{addosso ai nemici o nelle loro tane}

- \textbf{dispersi o nascosti nel dungeon}

- \textbf{comprati} (!!!)

\medskip

Quale sia la situazione il Narratore deve sempre prestare attenzione agli oggetti magici che i personaggi \emph{troveranno}.

I Tesori magici vanno inseriti, se addosso ai nemici o dungeon, con parsimonia e ragionando, cercate di resistere alla tentazione di essere generosi con i personaggi perché facilmente si abitueranno e difficilmente potrete recuperare la situazione.

Ancora di più è necessario che gli oggetti magici, specialmente quelli più potenti, non possano essere comprati come \emph{vili} oggetti comuni. Non lesinate su Pozioni di Cura o piccoli ninnoli magici che hanno una loro utilità, eppure gli oggetti più meravigliosi (dalla spada +2 in su..) devono essere trovati, deve essere affrontato colui che attualmente possiede quell'oggetto, altrimenti lo scopo dell'avventura e del pericolo va a scemare.

Nel caso preferite una distribuzione stabilita e bilanciata seguite le indicazioni sottostanti.\index{Oggetti magici per livello}

Per quanto riguarda gli oggetti magici \textbf{permanenti} come armi, armature, altri oggetti senza cariche o con cariche giornaliere, potete distribuire gli oggetti secondo questo schema, cumulativo per ogni livello:

\medskip

- livelli 1-4: un oggetto non comune

- livelli 5-7: un secondo oggetto non comune

- livelli 8-10: un oggetto raro

- livelli 11-13: un secondo oggetto raro

- livelli 14-16: un oggetto molto raro

- oltre il 17: oggetto molto raro o leggendario.

\medskip

Per quanto riguarda i \textbf{consumabili} come pozioni, pergamene od oggetti con un uso a scalare di cariche, potete distribuire gli oggetti secondo questo schema, cumulativo per ogni livello:

\medskip

- livelli 1-5: un consumabile comune

- livelli 6-10: un consumabile non comune

- livelli 11-15: un consumabile raro

- livelli 16-19: un consumabile molto raro

- oltre il 20: un consumabile leggendario

\medskip

Tutto questo chiaramente dipende dal livello di magia che si vuole dare all'avventura.

In questa maniera piloterete gli oggetti trovati in base alle necessità del gruppo ed al bilanciamento dell'avventura.

\subsection*{Ma perché dungeon senza fine ?}

I primi Patroni detestavano questo nuovo pianeta ed ancora di più il Sole come manifestazione di Ljust.
Distruggevano la Terra per ordine divino e per poter depredare e costruire un \emph{nuovo mondo nel sottosuolo}, il più lontano possibile dalla superficie. Se un anno può sembrare poco ricordatevi che la loro volontà era assoluta e la realtà si piegava a ciò che volevano.

Furono costruite innumerevoli sistemi, ambienti, strutture, città sotterranee la cui dimensione rivaleggiava con interi stati e qui cumularono i \emph{tesori} e qualunque oggetto attirasse la loro attenzione. Riempirono questi luoghi di creature \emph{aliene} e infinite mostruosità e oggetti moderni e creazioni prese dall'infinita fantasia della cultura di innumerevoli pianeti.

Per renderle infine più sicure, se mai fosse stato necessario, nella zona più profonda ed inaccessibile crearono quello che viene chiamato il \emph{cuore del dungeon} una frattura, un portale costruito specificatamente per rimpinguare le orde di mostri che ci abitano. Questi portali sono da estirpare e chiudere se si vuole che i mostri terminino e si possa veramente ricominciare a costruire un nuovo mondo.

Se i dungeon più piccoli (quelli entro 5, 6 piani) hanno un solo \emph{cuore} quelli più ampi e profondi ne hanno molti di più, ed ogni volta protetti da creature più forti e difficili da sconfiggere.

\medskip

\begin{enfasi}
%Come sarebbe bello dire \emph{per caso}? .. "Tu credi davvero che ci sia qualcosa che succede \emph{per caso}?" (Alessandro Baricco)
%\medskip

Tesoro è qualunque cosa mobile di pregio, nascosta o sotterrata, di cui nessuno può provare d'essere proprietario. (Codice civile italiano)

\end{enfasi}

\subsection*{Ma come fanno ad esistere oggetti magici sulla Terra ?}

I primi Patroni non hanno solo distrutto, stravolto, modificata tutta la geografia della Terra ma hanno anche portato nuove razze, creature, demoni e mostri!

Queste creature sono state portate con il loro bagaglio di conoscenze ed oggetti. Non solo, i Patroni hanno creato e fornito a questi le armi ed \emph{attrezzature} necessarie al loro scopo.

Hanno poi attinto alla tradizione, folklore, letteratura ed incubi, non solo nostri, per creare gli oggetti più straordinari e unici e seminarli nelle profondità del pianeta in quelle che dovevano diventare le loro dimore.

Questi oggetti sono nelle profondità e custoditi dalle aberrazioni evocati dai Patroni, come sempre più si scende in profondità più c'è possibilità di trovare qualcosa, perché quella che è per noi la Legge del Premio, per i primi Patroni è il modo di nascondere i preziosi, nelle profondità delle loro \emph{case}.

Le nuove razze, a differenza degli umani, possedevano già le conoscenze per creare questi oggetti unici e hanno continuato a produrli.

Nel secolo che è passato parte della conoscenza per creare gli oggetti magici è stata appresa da Elfi, Nani e Gnomi ed in parte dai Patroni stessi. Gli umani hanno incominciato a costruire loro stessi oggetti fantastici anche se alcuni dei tesori più preziosi rimangono nelle terrorizzanti gotiche cattedrali e città elfiche o nelle miniere infinite che sono le case dei nani.

Come Narratore approfittate della conoscenza che può avere un anziano mago elfo, per coinvolgere i personaggi in avventure per recuperare rari ingredienti, leggendari incantesimi, oggetti mitici.. ed imparare una coltura così antica e diversa.

Usate anche gli oggetti mitici della cultura terrestre, sicuramente un Patrono si è divertito a crearli per poi annoiarsene qualche attimo dopo ed averlo gettato nelle viscere di qualche caverna.

\subsection{Tesori e Mostri}\index{Tesori e Mostri}

Ogni mostro ha assegnato una \textbf{Categoria Tesoro} (CT), questa categoria indica in linea di massima che tipo di tesoro, monete ed eventuali oggetti magici, la creatura porta con se o sono disponibili nella sua tana. Potete sempre decidere autonomamente o modificare le percentuali indicate per favorire un certo tipo di tesoro.

Quando è indicata una percentuale significa che è necessario fare uguale o meno con il d100 per trovare il tesoro indicato. Quando è indicato +1 \emph{oggetto} allora significa che indipendentemente dal tiro percentuale fatto ci sarà almeno 1 oggetto, pozione o pergamena, del tipo indicato.

Es. una creatura è segnata come \emph{Tesoro CT} \textbf{F} significa che nella sua tana, nascondiglio, ci sarà il 10\% di trovare 3d6 monete l'argento, il 40\% di trovare 1d6 monete d'oro ed 5 volte il 30\% di trovare un oggetto magici che non siano armi.

%Non tutti i tesori pecuniari devono necessariamente essere costituiti da monete, gemme, gioielli, potete distribuire i tesori sotto forma di opere d'arte, arazzi, sculture e altri oggetti. Gli stessi oggetti magici possono avere funzioni aggiunte dal Narratore per arricchire l'avventura ed introdurre nuove situazioni.

Consultate poi la \hyperlink{tipologiaoggettomagico}{Tabella Tipologia Oggetto magico} (pag. \pageref{tipologiaoggettomagico}) per tirare e scoprire quale oggetti magici aveva la creatura.

\end{multicols}


\index[Tabelle]{Tabella Composizione dei tesori}\label{valoretesoroincontro}\hypertarget{valoretesoroincontro}{}\index{Lettere dei Tesori}

\medskip

\noindent\begin{tabularx}{\textwidth}{>{\bfseries}l|>{\small}c|>{\small}c|>{\small}c|>{\small}c|>{\small}c|>{\small}c|>{\small}X}
	\toprule
\rowcolor{gray!20}	\multicolumn{8}{c}{\textbf{Tesori da tana o nascondigli di creature}} \\
	\midrule
 \rowcolor{gray!20}CT & \textbf{mr} & \textbf{ma} & \textbf{mo} & \textbf{mp} & \textbf{Gemme} & \textbf{Gioielli} & \textbf{Ogg. magici} \\
	\hline
& x1000 & x1000 & x1000 & x100 & & & \\
\rowcolor{gray!20}A & 1d3, 25\% & 2d10, 30\% & 1d6, 40\% & 3d6, 35\% & 1d4, 60\% & 2d6, 50\% & 3 qualsiasi, 30\% \\
B & 1d6, 50\% & 1d3, 25\% & 2d10, 25\% & 1d10, 25\% & 1d8, 30\% & 1d4, 20\% & Armature o armi, 10\% \\
\rowcolor{gray!20}C & 1d10, 20\% & 1d6, 30\% & - & 1d6, 10\% & 1d6, 25\% & 1d3, 20\% & 2 qualsiasi, 10\% \\
D & 1d6, 10\% & 1d10, 15\% & 1d3, 50\% & 1d6, 15\% & 1d10, 30\% & 1d6, 25\% & 2 qualsiasi, 15\%, + 1 pozione\\
\rowcolor{gray!20}E & 1d6, 5\% & 1d10, 25\% & 1d4, 25\% & 3d6, 25\% & 1d12, 15\% & 1d6, 10\% & 3 qualsiasi, 25\%, +1 pergamena \\
F & - & 3d6, 10\% & 1d6, 40\% & 1d4, 15\% & 2d10, 20\% & 1d8, 10\% & 5 qualsiasi, non armi 30\% \\
\rowcolor{gray!20}G & - & - & 2d10, 50\% & 1d10, 50\% & 3d6, 30\% & 1d6, 25\% & 5 qualsiasi 35\% \\
H & 3d6, 25\% & 2d10, 40\% & 2d10, 55\% & 1d8, 40\% & 3d10, 50\% & 2d10, 50\% & 6 qualsiasi 15\% \\
\rowcolor{gray!20}I & - & - & - & 1d6, 30\% & 2d6, 55\% & 2d4, 50\% & 1 qualsiasi 15\% \\
\end{tabularx}



\medskip

\begin{center}
	\includegraphics[width=0.5\linewidth]{immagini/Hoxne_Hoard_1.png}

	\emph{Riproduzione del tesoro di Hoxne}
\end{center}

\bigskip

\noindent\begin{tabularx}{\textwidth}{>{\bfseries}l|>{\small}c|>{\small}c|>{\small}c|>{\small}c|>{\small}c|>{\small}c|>{\small}X}
	\toprule
\rowcolor{gray!20}\multicolumn{8}{c}{\textbf{Tesori Individuali, piccole tane, zaini e borse}} \\
	\midrule
  \rowcolor{gray!20}CT & \textbf{mr} & \textbf{ma} & \textbf{mo} & \textbf{mp} & \textbf{Gemme} & \textbf{Gioielli} & \textbf{Ogg. magici} \\
	 \hline
J & 3d8 & - & - & - & - & - & - \\
\rowcolor{gray!20}K & - & 3d6 & - & - & - & - & - \\
L & - & - & - & 2d6 & - & - & - \\
\rowcolor{gray!20}M & - & - & 2d4 & - & - & - & - \\
N & - & - & - & 1d6 & - & - & - \\
\rowcolor{gray!20}O & 1d4*10 & 1d3*10 & - & - & - & - & - \\
P & - & 1d6*10 & - & 3d6 & - & - & - \\
\rowcolor{gray!20}Q & - & - & - & - & 1d4 & - & - \\
R & - & - & 2d10 & 1d6*10 & 2d4 & 1d3 & - \\
\rowcolor{gray!20}S & - & - & - & - & - & - & 1d8 pozioni \\
T & - & - & - & - & - & - & 1d4 pergamene \\
\rowcolor{gray!20}U & - & - & - & - & 2d8, 90\% & 1d6, 80\% & 1 qualsiasi, 70\% \\
V & - & - & - & - & - & - & 2 qualsiasi \\
\rowcolor{gray!20}W & - & - & 5d6 & 1d8 & 2d8, 60\% & 1d8, 50\% & 2 qualsiasi, 60\% \\
X & - & - & - & - & - & - & 2 pozioni \\
\rowcolor{gray!20}Y & - & - & 1d6*100 & - & - & - & - \\
Z & 1d3*100 & 1d4*100 & 1d6*100 & 1d4*100 & 1d6, 50\% & 2d6, 50\% & 3 qualsiasi, 50\% \\
\end{tabularx}

\bigskip

Quando il tesoro è indicato da più lettere la creatura possiede entrambi i tesori indicati.

Alcune creature particolarmente \emph{ricche} potrebbero avere più volte lo stesso tesoro (2 H ovvero 2 volte il tesoro H).

\pagebreak

\section{Generazione casuale dei Tesori}\index{Generazione casuale dei Tesori}\label{generazionetesorimagici}\hypertarget{generazionetesorimagici}{}

\begin{enfasi}
{Come ogni amore non corrisposto, anche quello per le cose alla lunga si paga. (Adolfo Bioy Casares)}
\end{enfasi}

\begin{multicols}{2}

Il Narratore nella preparazione dell'avventura può posizionare gli oggetti magici che preferisce, che ce ne sia bisogno, ed in puro stile OSR affidarsi ad una generazione casuale.

L'approccio esclusivamente casuale non è sempre suggerito, i risultati potrebbero stravolgere l'avventura se non tutta la campagna!
Eppure Trovare una spada ammazzadraghi al primo livello siate certi che genererà avventure a non finire per i personaggi!

\begin{giocatore}[Tessssori!]
La Terra è un mondo a \emph{raro} profilo magico, gli oggetti magici esistono ma sono rari ed ancor di più quelli più potenti. Mentre pozioni naturali e piccoli ninnoli possono essere trovati ovunque è solo cercando attivamente, andando nelle profondità che si possono trovare i tesori migliori.
\end{giocatore}

\subsubsection{Valore delle Gemme e Gioielli}\index{Valore delle gemme}\hypertarget{valoredellegemme}{}\label{valoredellegemme}\index{Valore dei gioielli}\hypertarget{valoredeigioielli}{}\label{valoredeigioielli}

Quando vengono trovate delle Gemme o Gioielli il Narratore deve tirare per determinarne il valore in monete d'oro. Si può assegnare a tutti gli oggetti lo stesso valore, assegnare un valore individuale a ciascuna gemma/gioiello, oppure e tirare casualmente per ciascuno.

\medskip

\setlength{\parindent}{0cm}{

\noindent\begin{tabularx}{\linewidth}{Xl|l}
	\toprule
 \rowcolor{gray!20}\textbf{3d6}&\textbf{Gemma (mo)}&\textbf{Gioielli (mo)}\\
\toprule
	3&10& 1d4*10\\
 \rowcolor{gray!20}4-5&25& 2d4*10\\
	6-8&50& 1d4*100\\
 \rowcolor{gray!20}9-12&75& 2d4*100\\
	13-15&100& 3d4*100\\
 \rowcolor{gray!20}16&200& 3d6*100\\
	17&400& 4d6*100\\
 \rowcolor{gray!20}18&800& 5d6*100\\
\end{tabularx}

\medskip

\textbf{Tabella: tipologie di Gemme, ordinate per valore}\index[Tabelle]{Tabella tipologie di Gioielli}

\noindent\begin{tabularx}{\linewidth}{l|l|l|l}
	\toprule
  \rowcolor{gray!20}\textbf{4d6} & \textbf{Gemme} & \textbf{4d6} & \textbf{Gemme} \\
\toprule
		4 & Quarzo & 16 &  Topazio\\
  \rowcolor{gray!20}5-6 & Ambra & 17 & Turchese \\
		7-8 & Zircone & 18 & Zaffiro \\
  \rowcolor{gray!20}9-10 & Ametista & 19 &  Acquamarina\\
		11 & Corallo & 20 & Smeraldo \\
  \rowcolor{gray!20}12 & Lapislazzuli & 21 & Perla \\
		13 & Giada & 22 & Rubino\\
  \rowcolor{gray!20}14 & Peridoto & 23 &  Diamante\\
		15 & Granato & 24 & Pietra Leggendaria\\
\end{tabularx}

\textbf{Tabella: tipologie di Gioielli}\index[Tabelle]{Tabella tipologie di Gioielli}

\medskip

{\small \begin{tabularx}{\linewidth}{X|l|X|l}
		\toprule
\rowcolor{gray!20}\textbf{d100} & \textbf{Gioiello} & \textbf{d100} & \textbf{Gioiello}\\
\toprule
1-5 & Cavigliera & 6-10 & Orecchino \\
\rowcolor{gray!20}11-15 & Cintura& 16-20 & Coppa\\
21-25 & Ciondolo& 26-30 & Calice\\
\rowcolor{gray!20}31-35 & Braccialetto& 36-40 & Coltello\\
41-45 & Spilla& 46-50 & Tagliacarte\\
\rowcolor{gray!20}51-55 & Fibbia& 56-60 & Medaglione\\
61-64 & Catena& 65-68 & Medaglia\\
\rowcolor{gray!20}69-72 & Girocollo& 73-76 & Collana\\
77-80 & Tiara& 90& Piatto\\
\rowcolor{gray!20}81-84 & Diadema& 85-88 & Uovo\\
89-92 & Fermaglio& 96 & Scettro\\
\rowcolor{gray!20}52    & Corona& 93-96 & Statuetta\\
97-100 & Scrigno& 100 & Globo
\end{tabularx}}

\subsubsection{Capacità Speciali ed Oggetti Maledetti}

Quando nel \emph{Bonus Magico} c'è scritto \textbf{ritira + Capacità Speciale Armi/Armature Tipo...} significa che devi ritirare il 1d100, ignorando altri risultato sopra 80 e tenere il bonus magico ottenuto, poi potrai tirare sulla \emph{Tabella Capacità Speciale Armi Tipo...} risultante.

Quando un Arma, Armatura o Scudo è indicata come \textbf{Maledetta} può essere indicato la penalità al colpire ed al danno. L'oggetto può essere abbandonato senza grossi problemi.

Quando invece è segnato \textbf{Arma}, \textbf{Armatura}, \textbf{Verga} , \textbf{Bastone}, \textbf{Anello}... \textbf{Maledetta} (es. \emph{Arma Maledetta}) è necessario ritirare sulla tabella e verificare se il nuovo oggetto indicato a una versione maledetta. In caso fosse disponibile selezionarlo altrimenti l'oggetto si comporta come un oggetto maledetto -2 (attacco/ danno o Difesa) o non funzionante.\hypertarget{Arma Maledetta}{}\hypertarget{Maledetta}{}\hypertarget{Armatura Maledetta}{}

\medskip

\textbf{Tabella: Tipologia di Oggetto magico}\index[Tabelle]{Tabella Tipologia di Oggetto magico}\label{tipologiaoggettomagico}\hypertarget{tipologiaoggettomagico}{}

{\small \begin{tabularx}{\linewidth}{ll}
		\toprule
\rowcolor{gray!20}\textbf{3d6}& \textbf{Tipologia di oggetto magico}\\
\toprule
1-20 &\hyperlink{amuleticollanegioielli}{Amuleti, Collane, Gioielli}\\
\rowcolor{gray!20}21-40 &\hyperlink{cintureelmi}{Cinture, Elmi, Stivali e Guanti}\\
41-60 &\hyperlink{armatureescudi}{Armature e Scudi}\\
\rowcolor{gray!20}61-80 &\hyperlink{armimagiche}{Armi Magiche}\\
81-100 &\hyperlink{pozionifiltri}{Pozioni, Filtri e Olii}\\
\rowcolor{gray!20}14&\hyperlink{bastonibacchette}{Bacchette, Bastoni e Verghe}\\
15&\hyperlink{anellimagici}{Anelli}\\
\rowcolor{gray!20}16&\hyperlink{Cappelli}{Cappelli}, \hyperlink{Mantelli}{Mantelli}, \hyperlink{OcchialidaNotte}{Occhiali}, \hyperlink{Tuniche}{Tuniche}\\
17&\hyperlink{manualitomi}{Manuali, Tomi e Pergamene}\\
\rowcolor{gray!20}18&\hyperlink{oggettimagicivari}{Oggetti Magici vari}
\end{tabularx}}

\subsubsection{Armi}

\textbf{Tabella: Generazione Armi}\index[Tabelle]{Tabella Generazione Armi}\hypertarget{armimagiche}{}\label{armimagiche}

\medskip

{\small\begin{tabularx}{\linewidth}{ll}
		\toprule
\rowcolor{gray!20}\textbf{4d6} & \textbf{Bonus magico}\\
\toprule
4-6 & -2 \hyperlink{Arma maledetta}{Maledetta}\\
\rowcolor{gray!20}7-11 &+1\\
12-16 & -1 \hyperlink{Arma maledetta}{Maledetta}\\
\rowcolor{gray!20}17-21 & +2\\
22 & +3\\
\rowcolor{gray!20}23 & +4\\
24 &+5\\
\end{tabularx}}

\textbf{Tabella: Capacità Speciale Armi Tipo 1}\index[Tabelle]{Tabella Capacità Speciale Armi Tipo 1}\hypertarget{Capacità Speciale Armi Tipo 1}{}

\medskip

{\small \begin{tabularx}{\linewidth}{ll}
		\toprule
\rowcolor{gray!20}\textbf{1d100} & \textbf{Capacità Speciale Armi Tipo 1}\\
\toprule
1-15 &\hyperlink{Accumula Incantesimi}{Accumula Incantesimi}\\
\rowcolor{gray!20}16-30 &\hyperlink{Arma Anatema}{Arma Anatema}\\
31-44 & \hyperlink{Distruttrice dei Giganti}{Distruttrice dei Giganti}\\
\rowcolor{gray!20}45-58 & \hyperlink{Energia Luminosa}{Energia Luminosa}\\
59-72 & \hyperlink{Ladra delle NoveVite}{Ladra delle Nove Vite}\\
\rowcolor{gray!20}73-86 & \hyperlink{Tocco Fantasma}{Tocco Fantasma}\\
87-100 & \hyperlink{Arma della Velocita'}{Arma della Velocita'}\\
\end{tabularx}
}
\medskip

%\begin{center}
%\includegraphics[width=0.6\linewidth]{immagini/shield1.png}
%\end{center}

\textbf{Tabella: Capacità Speciale Armi Tipo 2}\index[Tabelle]{Tabella Capacità Speciale Armi Tipo 2}\hypertarget{Capacità Speciale Armi Tipo 2}{}

\medskip
{\small \begin{tabularx}{\linewidth}{ll}
		\toprule
\rowcolor{gray!20}\textbf{1d100} & \textbf{Capacità Speciale Armi Tipo 2}\\
\toprule
1-25 & \hyperlink{deldolore}{del dolore}\\
\rowcolor{gray!20}26-50 & \hyperlink{Mazza della Punizione}{Mazza della Punizione}\\
51-75 & \hyperlink{Anello d'Arma maggiore}{Anello d'Arma maggiore}\\
\rowcolor{gray!20}76-100 & \hyperlink{Anello d'Arma}{Anello d'Arma}\\
\end{tabularx}}

\medskip

\textbf{Tabella: Capacità Speciale Armi Tipo 3}\index[Tabelle]{Tabella Capacità Speciale Armi Tipo 3}\hypertarget{Capacità Speciale Armi Tipo 3}{}

\medskip

{\small\begin{tabularx}{\linewidth}{ll}
		\toprule
\rowcolor{gray!20}\textbf{1d100} & \textbf{Capacità Speciale Armi Tipo 3}\\
\toprule
1-17 & \hyperlink{Ammazza Draghi}{Ammazza Draghi}\\
\rowcolor{gray!20}18-34 & \hyperlink{Ammazza Giganti}{Ammazza Giganti}\\
35-51 & \hyperlink{Estingui Fuoco}{Estingui Fuoco}\\
\rowcolor{gray!20}52-68 & \hyperlink{Mutaforma}{Mutaforma}\\
69-84 & \hyperlink{Munizione Fantasma}{Munizione Fantasma}\\
\rowcolor{gray!20}85-100 & \hyperlink{Munizioni Infinite}{Munizioni Infinite}\\
\end{tabularx}}

%\begin{center}
%\includegraphics[width=0.8\linewidth]{immagini/romanring.png}
%\end{center}

\subsubsection{Armature e Scudi}

\textbf{Tabella: Generazione Armature/Scudi}\index[Tabelle]{Tabella Generazione Armature/Scudi}\hypertarget{armatureescudi}{}\label{armatureescudi}

\medskip

{\small\begin{tabularx}{\linewidth}{ll}
		\toprule
  \rowcolor{gray!20}\textbf{4d6} & \textbf{Bonus magico}\\
\toprule
		4-6 & -2 \hyperlink{Armatura maledetta}{Maledetta}\\
  \rowcolor{gray!20}7-11 &+1\\
		12-16 & -1 \hyperlink{Armatura maledetta}{Maledetta}\\
  \rowcolor{gray!20}17-21 & +2\\
		22 & +3\\
  \rowcolor{gray!20}23 & +4\\
		24 &+5\\
\end{tabularx}}

\medskip

\textbf{Tabella: Capacità Speciale Armature/Scudi Tipo 1}\index[Tabelle]{Tabella Capacità Speciale Armature/Scudi Tipo 1}\hypertarget{Capacità Speciale Armature / Scudi Tipo 1}{}

\medskip

{\small\begin{tabularx}{\linewidth}{ll}
		\toprule
\rowcolor{gray!20}\textbf{1d100} & \textbf{Capacità Speciale Armature/Scudi Tipo 1}\\
\toprule
11-15& \hyperlink{Bracciali dell'Arciere}{Bracciali dell'Arciere}\\
\rowcolor{gray!20}16-20& \hyperlink{Bracciali della Difesa}{Bracciali della Difesa}\\
21-25& \hyperlink{Braccialidella Difesa Maggiore}{Bracciali della Difesa Maggiore}\\
\rowcolor{gray!20}36-40& \hyperlink{Difesa dagli Incantesimi}{Difesa dagli Incantesimi}\\
51-55& \hyperlink{Resistenza al Veleno}{Resistenza al Veleno}\\
\rowcolor{gray!20}56-60& \hyperlink{Resistenza all'Energia}{Resistenza all'Energia}\\
61-65& \hyperlink{Resistenza all'Energia Superiore}{Resistenza all'Energia Superiore}\\
\rowcolor{gray!20}71-75& \hyperlink{Scaglie di Drago}{Scaglie di Drago}\\
76-80& \hyperlink{Scudo Animato}{Scudo Animato}\\
\rowcolor{gray!20}81-85& \hyperlink{Scudo dell'Attrazione dei Proiettili}{Scudo dell'Attrazione dei Proiettili}\\
86-90& \hyperlink{Soffio del Dragone}{Soffio del Dragone}\\
\rowcolor{gray!20}91-95& \hyperlink{Tocco Fantasma}{Tocco Fantasma}\\
96-100 & Maledetta\\
\end{tabularx}}

%\begin{center}
%\includegraphics[width=0.8\linewidth]{immagini/gauntlet.png}\\
%\end{center}

\textbf{Tabella: Capacità Speciale Armature/Scudi Tipo 2}\index[Tabelle]{Tabella Capacità Speciale Armature/Scudi Tipo 2}\hypertarget{Capacità Speciale Armature / Scudi Tipo 2}{}

\medskip

{\small\begin{tabularx}{\linewidth}{ll}
		\toprule
\rowcolor{gray!20}\textbf{1d100} & \textbf{Capacità Speciale Armature/Scudi Tipo 2}\\
\toprule
1-5 & Maledetta\\
\rowcolor{gray!20}6-10 &\hyperlink{Adamantio}{Adamantio}\\
31-35& \hyperlink{Armatura Demoniaca}{Armatura Demoniaca}\\
\rowcolor{gray!20}46-50& \hyperlink{Armatura della Forma Eterea}{Armatura della Forma Eterea}\\
61-65 &\hyperlink{Mithral}{Mithral}\\
\rowcolor{gray!20}66-70 &\hyperlink{Armatura d'Ombra}{Armatura d'Ombra}\\
71-80 &\hyperlink{Armatura Titanica}{Armatura Titanica}\\
\rowcolor{gray!20}81-100 & Maledetta\\
\end{tabularx}}

\subsubsection{Amuleti, Collane e Gioielli}\index[Tabelle]{Tabella Generazione Amuleti, Collane e Gioielli}\hypertarget{amuleticollanegioielli}{}\label{amuleticollanegioielli}

{\small\begin{tabular}{ll}
		\toprule
\rowcolor{gray!20}\textbf{Tipo Oggetto}&\textbf{1d8}\\
\toprule
1-6&Amuleti, Collane e Gioielli Tipo 1\\
7-8&Amuleti, Collane e Gioielli Tipo 2\\
\end{tabular}}

\medskip\hypertarget{amuleticollanegioielli1}{}

{\small\begin{tabularx}{\linewidth}{lX}
		\toprule
\rowcolor{gray!20}\textbf{1d100} & \textbf{Amuleti, Collane e Gioielli Tipo 1}\\
\toprule
1-8 & \hyperlink{Amuleto Antiveleno}{Amuleto Antiveleno}\\
\rowcolor{gray!20}8-12& \hyperlink{Amuleto della Cancrena}{Amuleto della Cancrena}\\
12-18 & \hyperlink{Amuleto Cicatrizzante}{Amuleto Cicatrizzante}\\
\rowcolor{gray!20}19-26 & \hyperlink{Amuleto Controla Possessione}{Amuleto Contro la Possessione}\\
27-34 & \hyperlink{Amuleto della Localizzazione inevitabile}{Amuleto della Localizzazione inevitabile}\\
\rowcolor{gray!20}35& \hyperlink{Amuleto dei Piani}{Amuleto dei Piani}\\
36-42 & \hyperlink{Amuleto di Protezione dalla Individuazione e Localizzazione}{Amuleto di Protezione dalla Individuazione e Localizzazione}\\
\rowcolor{gray!20}42-46 & \hyperlink{Amuleto della Resistenza Fisica}{Amuleto della Resistenza Fisica}\\
47-53 & \hyperlink{Cerchietto dell'Esplosione}{Cerchietto dell'Esplosione}\\
\rowcolor{gray!20}53-60 & \hyperlink{Collana dell'Adattamento}{Collana dell'Adattamento}\\
61-70 & \hyperlink{Collana dello Strangolamento}{Collana dello Strangolamento}\\
\rowcolor{gray!20}71-77 & \hyperlink{Collana delle Palle di Fuoco}{Collana delle Palle di Fuoco}\\
78-83 & \hyperlink{Collana del Rosario}{Collana del Rosario}\\
\rowcolor{gray!20}84-90 & \hyperlink{Scarabeo della Morte}{Scarabeo della Morte}\\
91-100& \hyperlink{Scarabeo di Protezione}{Scarabeo di Protezione}
\end{tabularx}}

\medskip\hypertarget{amuleticollanegioielli2}{}

{\small\begin{tabularx}{\linewidth}{lX}
		\toprule
\rowcolor{gray!20}\textbf{1d100} & \textbf{Amuleti, Collane e Gioielli Tipo 2}\\
\toprule
1-7 & \hyperlink{Gemma Elementale}{Gemma Elementale}\\
\rowcolor{gray!20}8-13& \hyperlink{Gemma della Luminosita'}{Gemma della Luminosita'}\\
9-16& \hyperlink{Gemma della Vista}{Gemma della Vista}\\
\rowcolor{gray!20}17-26& \hyperlink{Gioiello Attira mostri}{Gioiello Attiramostri}\\
27-33& \hyperlink{Medaglione dei Pensieri}{Medaglione dei Pensieri}\\
\rowcolor{gray!20}34-41& \hyperlink{Medaglione della Caduta piuma}{Medaglione della Caduta piuma}\\
42-49& \hyperlink{Perla della Saggezza}{Perla della Saggezza}\\
\rowcolor{gray!20}50-57& \hyperlink{Spilla della Difesa}{Spilla della Difesa}\\
58-60& \hyperlink{Talismano del Bene puro}{Talismano del Bene puro}\\
\rowcolor{gray!20}61-62& \hyperlink{Talismano del Male estremo}{Talismano del Male estremo}\\
63-70& \hyperlink{Talismano di Protezione dal Veleno}{Talismano di Protezione dal Veleno}\\
\rowcolor{gray!20}71-80 & Talismano rotto\\
81-85& \hyperlink{Talismano della Sfera}{Talismano della Sfera}\\
\rowcolor{gray!20}86-100& Gioiello senza valore
\end{tabularx}}

\subsubsection{Cinture, Elmi, Stivali e Guanti}\index[Tabelle]{Tabella Generazione Cinture, Elmi, Stivali e Guanti}\hypertarget{cintureelmi}{}\label{cintureelmi}

{\small\begin{tabularx}{\linewidth}{ll}
	\toprule
\rowcolor{gray!20}\textbf{1d100} & \textbf{Cinture, Elmi, Stivali e Guanti}\\
\toprule
1-3 & \hyperlink{Cintura dei Giganti}{Cintura dei Giganti}\\
\rowcolor{gray!20}3-6 & \hyperlink{Cintura dei Nani}{Cintura dei Nani}\\
6-11 & \hyperlink{Elmo della Comprensione dei Linguaggi}{Elmo della Comprensione dei Linguaggi}\\
\rowcolor{gray!20}12 & \hyperlink{Elmo della Lucentezza}{Elmo della Lucentezza}\\
13-17 & \hyperlink{Elmo del Movimento subacqueo}{Elmo del Movimento subacqueo}\\
\rowcolor{gray!20}18-22 & \hyperlink{Elmo della Telepatia}{Elmo della Telepatia}\\
23-26 & \hyperlink{Elmo del Teletrasporto}{Elmo del Teletrasporto}\\
\rowcolor{gray!20}27-31 & \hyperlink{Guanti Afferra Proiettili}{Guanti Afferra Proiettili}\\
31-35 & \hyperlink{Guanti del Potere orchesco}{Guanti del Potere orchesco}\\
\rowcolor{gray!20}36-41 & \hyperlink{Guanti del Nuoto edella Scalata}{Guanti del Nuoto e della Scalata}\\
41-46 & \hyperlink{Guanti della Destrezza}{Guanti della Destrezza}\\
\rowcolor{gray!20}47-52 & \hyperlink{Guanti Maldestri}{Guanti Maldestri}\\
53-58 & \hyperlink{Pantofole del Ragno}{Pantofole del Ragno}\\
\rowcolor{gray!20}59-63 & \hyperlink{Stivali Alati}{Stivali Alati}\\
64-66 & \hyperlink{Stivali della Corsa e del Salto}{Stivali della Corsa e del Salto}\\
\rowcolor{gray!20}67-77 & \hyperlink{Stivali degli Elfi}{Stivali degli Elfi}\\
78-83 & \hyperlink{Stivali dell'Inverno}{Stivali dell'Inverno}\\
\rowcolor{gray!20}84-90 & \hyperlink{Stivali della Levitazione}{Stivali della Levitazione}\\
91-95 & \hyperlink{Stivalid ella Velocita'}{Stivali della Velocita'}\\
\rowcolor{gray!20}96-100 & \hyperlink{StivaliDanzanti}{Stivali Danzanti}\\
\end{tabularx}}

\subsubsection{Bacchette, Bastoni e Verghe}\index[Tabelle]{Tabella Generazione Bacchette, Bastoni e Verghe}\hypertarget{bastonibacchette}{}\label{bastonibacchette}

Tirare 1d8 per determinare se si trova una Bacchetta o Bastone o Verga.

\medskip

{\small\begin{tabularx}{\linewidth}{ll}
		\toprule
\rowcolor{gray!20}\textbf{Tipo Oggetto}&\textbf{1d8}\\
\toprule
1-4&\hyperlink{Bacchette}{Bacchette}\\
\rowcolor{gray!20}5-7&\hyperlink{Bastoni}{Bastoni}\\
8  &\hyperlink{Verghe}{Verghe}\\
\end{tabularx}}

\medskip

\textbf{Tabella: Generazione Bacchette}\index[Tabelle]{Tabella Generazione Bacchette}\hypertarget{Bacchette}{}

\medskip

{\small\begin{tabularx}{\linewidth}{lX}
		\toprule
\rowcolor{gray!20}\textbf{1d100} & \textbf{Bacchetta}\\
\toprule
1-5 & \hyperlink{Bacchetta Cerca metalli}{Bacchetta Cerca metalli}\\
\rowcolor{gray!20}6-10 & \hyperlink{Bacchetta dei DardiArcani}{Bacchetta dei Dardi Arcani}\\
11-15 & \hyperlink{Bacchetta delle Comodita'}{Bacchetta delle Comodita'}\\
\rowcolor{gray!20}16-20 & \hyperlink{Bacchetta dei Fulmini}{Bacchetta dei Fulmini}\\
21-25 & \hyperlink{Bacchetta del Fuoco}{Bacchetta del Fuoco}\\
\rowcolor{gray!20}26-30 & \hyperlink{Bacchetta del Ghiaccio}{Bacchetta del Ghiaccio}\\
36-38 & \hyperlink{Bacchetta di Individ. dei Nemici}{Bacchetta di Individ. dei Nemici}\\
\rowcolor{gray!20}45-48 & \hyperlink{Bacchetta dell'Individuazione delle portese grete}{Bacchetta dell'Individuazione delle porte segrete}\\
46-50 & \hyperlink{Bacchetta della Luce}{Bacchetta della Luce}\\
\rowcolor{gray!20}51 & \hyperlink{Bacchetta del Mago da Guerra}{Bacchetta del Mago da Guerra}\\
52 & \hyperlink{Bacchetta della Metamorfosi}{Bacchetta della Metamorfosi}\\
\rowcolor{gray!20}53 & \hyperlink{Bacchetta delle Meraviglie}{Bacchetta delle Meraviglie}\\
54 & \hyperlink{Bacchetta della Negazione}{Bacchetta della Negazione}\\
\rowcolor{gray!20}55-60 & \hyperlink{Bacchetta delle Palle di Fuoco}{Bacchetta delle Palle di Fuoco}\\
61-65 & \hyperlink{Bacchetta della Paralisi}{Bacchetta della Paralisi}\\
\rowcolor{gray!20}66-70 & \hyperlink{Bacchetta della Paura}{Bacchetta della Paura}\\
71-75 & \hyperlink{Bacchetta Scopri trappole}{Bacchetta Scopri trappole}\\
\rowcolor{gray!20}76-80 & \hyperlink{Bacchetta dei Segreti}{Bacchetta dei Segreti}\\
81-85 & \hyperlink{Bacchetta della Ragnatela}{Bacchetta della Ragnatela}\\
\rowcolor{gray!20}86-90 & \hyperlink{Bacchetta del Vincolo}{Bacchetta del Vincolo}\\
91-95 & \hyperlink{Bacchetta della Fuga Assistita}{Bacchetta della Fuga Assistita}\\
\rowcolor{gray!20}96-100 & Bacchetta Maledetta
\end{tabularx}}

\medskip

\textbf{Tabella: Generazione Bastoni}\index[Tabelle]{Tabella Generazione Bastoni}\hypertarget{Bastoni}{}

\medskip

{\small\begin{tabularx}{\linewidth}{ll}
		\toprule
\rowcolor{gray!20}\textbf{3d6} & \textbf{Bastone}\\
\toprule
3 & \hyperlink{Bastone dell'Arcimago}{Bastone dell'Arcimago}\\
\rowcolor{gray!20}4 & \hyperlink{Bastone dei Boschi}{Bastone dei Boschi}\\
5-6 & \hyperlink{Bastone dello Charme}{Bastone dello Charme}\\
\rowcolor{gray!20}7-8 & \hyperlink{Bastone del Colpire}{Bastone del Colpire}\\
9-10 & \hyperlink{Bastone del Fuoco}{Bastone del Fuoco}\\
\rowcolor{gray!20}11 & \hyperlink{Bastone del Gelo}{Bastone del Gelo}\\
12 & \hyperlink{Bastoned egli Insetti Sciamanti}{Bastone degli Insetti Sciamanti}\\
\rowcolor{gray!20}13 & \hyperlink{Bastone del Pitone}{Bastone del Pitone}\\
14 & \hyperlink{Bastone del Potere}{Bastone del Potere}\\
\rowcolor{gray!20}15 & \hyperlink{Bastone dei Tuoni e Fulmini}{Bastone dei Tuoni e Fulmini}\\
16 & \hyperlink{Bastone della Stregoneria}{Bastone della Stregoneria}\\
\rowcolor{gray!20}17-18 & Bastone rotto\\
\end{tabularx}}

\medskip

\textbf{Tabella: Generazione Verghe}\index[Tabelle]{Tabella Generazione Verghe}\hypertarget{Verghe}{}

\medskip

{\small\begin{tabularx}{\linewidth}{ll}
		\toprule
\rowcolor{gray!20}\textbf{1d100} & \textbf{Verga}\\
\toprule
1-10 & \hyperlink{Verga dell'Ammaliamento}{Verga dell'Ammaliamento}\\
\rowcolor{gray!20}11-20 & \hyperlink{Verga dell'Assorbimento}{Verga dell'Assorbimento}\\
21-30 & \hyperlink{Verga Inamovibile}{Verga Inamovibile}\\
\rowcolor{gray!20}31-41 & \hyperlink{Verga del Colpo possente}{Verga del Colpo possente}\\
42-50 & \hyperlink{Verga della Forza Sovrana}{Verga della Forza Sovrana}\\
\rowcolor{gray!20}51-60 & \hyperlink{Verga della Prontezza}{Verga della Prontezza}\\
61-70 & \hyperlink{Verga della Sicurezza}{Verga della Sicurezza}\\
\rowcolor{gray!20}71-80 & \hyperlink{Verga della Sovranita'}{Verga della Sovranita'}\\
81-90 & \hyperlink{Verga Tentacolare}{Verga Tentacolare}\\
\rowcolor{gray!20}91-100 & Verga Maledetta\\
\end{tabularx}}

\subsubsection{Pozioni, Filtri e Olii}\index[Tabelle]{Tabella Generazione Pozioni, Filtri e Olii}\hypertarget{pozionifiltri}{}\label{pozionifiltri}

{\small\begin{tabularx}{\linewidth}{ll}
		\toprule
\rowcolor{gray!20}\textbf{Pozione}&\textbf{1d8}\\
\toprule
 1-4&\hyperlink{pozionifiltri}{Pozione Tipo 1}\\
 \rowcolor{gray!20}5-7&\hyperlink{pozionifiltri}{Pozione Tipo 2}\\
 8  &\hyperlink{pozionifiltri}{Pozione Tipo 3}
\end{tabularx}}

\medskip\hypertarget{Pozione Tipo 1}{}

{\small\begin{tabularx}{\linewidth}{ll}
		\toprule
\rowcolor{gray!20}\textbf{1d100} & \textbf{Pozione Tipo 1}\\
\toprule
1-8 & \hyperlink{Pozione di Arrampicata}{Pozione di Arrampicata}\\
\rowcolor{gray!20}9-15 & \hyperlink{Pozione di Crescita}{Pozione di Crescita}\\
16-23 & \hyperlink{Pozione di Eroismo}{Pozione di Eroismo}\\
\rowcolor{gray!20}24-29 & \hyperlink{Pozione di FormaGassosa}{Pozione di Forma Gassosa}\\
30-35 & \hyperlink{Pozione di ForzadeiGiganti}{Pozione di Forza dei Giganti}\\
\rowcolor{gray!20}36-46 & \hyperlink{Pozione di Guarigione}{Pozione di Guarigione}\\
47-53 & \hyperlink{Pozione dell'Inganno}{Pozione dell'Inganno}\\
\rowcolor{gray!20}54-64 & \hyperlink{Pozione di Invisibilita'}{Pozione di Invisibilita'}\\
65-74 & \hyperlink{Pozione della Levitazione}{Pozione della Levitazione}\\
\rowcolor{gray!20}77-78 & \hyperlink{Pozione di Resistenza}{Pozione di Resistenza}\\
79-84 & \hyperlink{Pozione di RespirareSott'Acqua}{Pozione di Respirare Sott'Acqua}\\
\rowcolor{gray!20}84-90 & \hyperlink{Pozione di Rimpicciolimento}{Pozione di Rimpicciolimento}\\
91-95 & \hyperlink{Pozione di Velocita'}{Pozione di Velocita'}\\
\rowcolor{gray!20}96-100 & \hyperlink{Pozione di Volo}{Pozione di Volo}
\end{tabularx}}

\begin{center}
\includegraphics[width=0.8\linewidth]{immagini/cupdrinking.png}

\emph{Drinking cup depicting scenes from the Odyssey, Athens 550-525 B.C.}
\end{center}
\hypertarget{Pozione Tipo 2}{}
{\small\begin{tabularx}{\linewidth}{ll}
		\toprule
\rowcolor{gray!20}\textbf{1d100} & \textbf{Pozione Tipo 2}\\
\toprule
1-10 & \hyperlink{Pozione della Chiaraudienzaa nimale}{Pozione della Chiaraudienza animale}\\
\rowcolor{gray!20}11-20 & \hyperlink{Pozione della Chiaroveggenza animale}{Pozione della Chiaroveggenza animale}\\
21-28 & \hyperlink{Pozione di Controllo degli animali}{Pozione di Controllo degli animali}\\
\rowcolor{gray!20}29-33 & \hyperlink{Pozione di Controllo dei draghi}{Pozione di Controllo dei draghi}\\
34-38 & \hyperlink{Pozione di Controllo dei nonmorti}{Pozione di Controllo dei non morti}\\
\rowcolor{gray!20}39-49 & \hyperlink{Pozione di Controllo delle persone}{Pozione di Controllo delle persone}\\
50-55 & \hyperlink{Pozione di Controllo delle piante}{Pozione di Controllo delle piante}\\
\rowcolor{gray!20}56-66 & \hyperlink{Pozione dell'invulnerabilita'}{Pozione dell'invulnerabilita'}\\
67-77 & \hyperlink{Pozione di Letturadel Pensiero}{Pozione di Lettura del Pensiero}\\
\rowcolor{gray!20}78-85 & \hyperlink{Pozione di Veleno}{Pozione di Veleno}\\
86-95 & \hyperlink{pozionifiltri}{Pozione di Cura Maggiore}\\
\rowcolor{gray!20}96-100 & \hyperlink{pozionifiltri}{Pozione di Veleno Potenziata}
\end{tabularx}}

\medskip\hypertarget{Pozione Tipo 3}{}

{\small\begin{tabularx}{\linewidth}{ll}
		\toprule
\rowcolor{gray!20}\textbf{1d100} & \textbf{Pozione Tipo 3}\\
\toprule
1-13& \hyperlink{Filtro d'Amore}{Filtro d'Amore}\\
\rowcolor{gray!20}14-27 & \hyperlink{Filtro Scopritesori}{Filtro Scopritesori}\\
28-40 & \hyperlink{Olio di Affilatezza}{Olio di Affilatezza}\\
\rowcolor{gray!20}41-53 & \hyperlink{Olio di Forma Eterea}{Olio di Forma Eterea}\\
54-66 & \hyperlink{Olio di Scivolosita'}{Olio di Scivolosita'}\\
\rowcolor{gray!20}67-79 & \hyperlink{Pozione di Amicizia congli Animali}{Pozione di Amicizia con gli Animali}\\
80-85 & \hyperlink{Pozione della Longevita'}{Pozione della Longevita'}\\
\rowcolor{gray!20}86-95 & \hyperlink{Pozione della Metamorfosi}{Pozione della Metamorfosi}\\
96-100&\hyperlink{pozionifiltri}{Pozione di Veleno Maggiore}
\end{tabularx}}

\subsubsection{Anelli}\index[Tabelle]{Tabella Generazione Anelli}\hypertarget{anellimagici}{}\label{anellimagici}

{\small\begin{tabularx}{\linewidth}{ll}
		\toprule
\rowcolor{gray!20}\textbf{Anello}&\textbf{3d6}\\
\toprule
3-16&\hyperlink{Pozione Tipo 3}{Anello Tipo 1} \\
\rowcolor{gray!20}17-18&\hyperlink{Pozione Tipo 3}{Anello Tipo 2}
\end{tabularx}}

\medskip\hypertarget{Anello Tipo 1}{}

{\small\begin{tabularx}{\linewidth}{ll}
		\toprule
\rowcolor{gray!20}\textbf{1d100} & \textbf{Anelli Tipo 1}\\
\toprule
1-5 & \hyperlink{Anello Accumula Incantesimi}{Anello Accumula Incantesimi}\\
\rowcolor{gray!20}6-13& \hyperlink{Anello dell'Ariete}{Anello dell'Ariete}\\
14-21 & \hyperlink{Anello di CadutaPiuma}{Anello di Caduta Piuma}\\
\rowcolor{gray!20}22-28 & \hyperlink{Anello di Camminaresull'Acqua}{Anello di Camminare sull'Acqua}\\
29-35 & \hyperlink{Scarfatotto del Calore}{Anello del Calore}\\
\rowcolor{gray!20}36-41 & \hyperlink{Anello della Debolezza}{Anello della Debolezza}\\
42-47 & \hyperlink{Anello di Elusione}{Anello di Elusione}\\
\rowcolor{gray!20}48-50 & \hyperlink{Anello di Influenza sugli Animali}{Anello di Influenza sugli Animali}\\
51-55 & \hyperlink{Anello dell'Inganno}{Anello dell'Inganno}\\
\rowcolor{gray!20}56-61 & \hyperlink{Anello di Liberta' di Azione}{Anello di Liberta' di Azione}\\
61-67 & \hyperlink{Anello delNuoto}{Anello del Nuoto}\\
\rowcolor{gray!20}68-77 & \hyperlink{Anello di Protezione}{Anello di Protezione}\\
76-84 & \hyperlink{Anello di Resistenza}{Anello di Resistenza}\\
\rowcolor{gray!20}85-93 & \hyperlink{Anello del Salto}{Anello del Salto}\\
93-100 & \hyperlink{Anello di Telecinesi}{Anello di Telecinesi}
\end{tabularx}}

\medskip\hypertarget{Anello Tipo 2}{}

{\small\begin{tabularx}{\linewidth}{ll}
		\toprule
\rowcolor{gray!20}\textbf{1d100} & \textbf{Anelli Tipo 2}\\
\toprule
1-8 & \hyperlink{Anello del Controllo delle persone}{Anello del Controllo delle persone}\\
\rowcolor{gray!20}9-17 & \hyperlink{Anello del Controllo delle piante}{Anello del Controllo delle piante}\\
18-23 & \hyperlink{Anello degli Elementali dell'Acqua}{Anello degli Elementali dell'Acqua}\\
\rowcolor{gray!20}24-29 & \hyperlink{Anello degli Elementali dell'Aria}{Anello degli Elementali dell'Aria}\\
31-36 & \hyperlink{Anello degli Elementali del Fuoco}{Anello degli Elementali del Fuoco}\\
\rowcolor{gray!20}37-42 & \hyperlink{Anello degli Elementali della Terra}{Anello degli Elementali della Terra}\\
43-48 & \hyperlink{Anello di Elusione}{Anello di Elusione}\\
\rowcolor{gray!20}49-56 & \hyperlink{Anello Respingi Incantesimi}{Anello Respingi Incantesimi}\\
57-65 & \hyperlink{Anello di Invisibilita'}{Anello di Invisibilita'}\\
\rowcolor{gray!20}66-75 & \hyperlink{Anello di Rigenerazione}{Anello di Rigenerazione}\\
76-83 & \hyperlink{Anello dello Scudo Mentale}{Anello dello Scudo Mentale}\\
\rowcolor{gray!20}84-90 & \hyperlink{Anello delle Stelle Cadenti}{Anello delle Stelle Cadenti}\\
91-92 & \hyperlink{Anello dei Tre Desideri}{Anello dei Tre Desideri}\\
\rowcolor{gray!20}92-96 & Anello dei Tre Desideri esaurito\\
97-100 & \hyperlink{AnellodellaVistaaiRaggiX}{Anello della Vista ai Raggi X}

\end{tabularx}}

\subsubsection{Cappelli, Mantelli, Occhiali, Tuniche}\index[Tabelle]{Tabella Generazione Cappelli, Mantelli, Occhiali, Tuniche}\label{cappellimantelli}\hypertarget{Mantelli}{}\hypertarget{Tuniche}{}\hypertarget{Occhiali}{Occhiali}\hypertarget{Cappelli}{Cappelli}

{\small\begin{tabularx}{\linewidth}{ll}
		\toprule
\rowcolor{gray!20}\textbf{1d100} & \textbf{Cappelli, Mantelli, Occhiali, Tuniche}\\
\toprule
1-3 & \hyperlink{Bandana dell'Intelligenza}{Bandana dell'Intelligenza}\\
\rowcolor{gray!20}4-10 & \hyperlink{Cappello delCamuffamento}{Cappello del Camuffamento}\\
11-17 & \hyperlink{Mantello dell'Aracnide}{Mantello dell'Aracnide}\\
\rowcolor{gray!20}18-23 & \hyperlink{Mantella del Ciarlatano}{Mantella del Ciarlatano}\\
24-29 & \hyperlink{Mantello di Distorsione}{Mantello di Distorsione}\\
\rowcolor{gray!20}30-40 & \hyperlink{Mantello degli Elfi}{Mantello degli Elfi}\\
41-45 & \hyperlink{Mantello della Manta}{Mantello della Manta}\\
\rowcolor{gray!20}46-50 & \hyperlink{Mantello del Pipistrello}{Mantello del Pipistrello}\\
51-57 & \hyperlink{Mantello di Protezione}{Mantello di Protezione}\\
\rowcolor{gray!20}58-62 & \hyperlink{Mantello della Resistenza agli Incantesimi}{Mantello della Resistenza agli Incantesimi}\\
63-68 & \hyperlink{Mantello della velenosita'}{Mantello della velenosita'}\\
\rowcolor{gray!20}69-72 & \hyperlink{Occhi della pietrificazione}{Occhi della pietrificazione}\\
73-75 & \hyperlink{Occhi Affascinanti}{Occhi Affascinanti}\\
\rowcolor{gray!20}76-77 & \hyperlink{Occhi dell'Aquila}{Occhi dell'Aquila}\\
78-80 & \hyperlink{Occhi della Vista Dettagliata}{Occhi della Vista Dettagliata}\\
\rowcolor{gray!20}80-82 & \hyperlink{Occhiali da Notte}{Occhiali da Notte}\\
83-86 & \hyperlink{Tunicadel Mimetismo}{Tunica del Mimetismo}\\
\rowcolor{gray!20}87 & \hyperlink{Tunica dell'Arcimago}{Tunica dell'Arcimago}\\
88 & \hyperlink{Tunica deiColori Scintillanti}{Tunica dei Colori Scintillanti}\\
\rowcolor{gray!20}89-91 & \hyperlink{Tunica dell'Indebolimento}{Tunica dell'Indebolimento}\\
92-94 & \hyperlink{Tunica degli Occhi}{Tunica degli Occhi}\\
\rowcolor{gray!20}95-99 & \hyperlink{Tunica degli OggettiUtili}{Tunica degli Oggetti Utili}\\
100 & \hyperlink{Tunica delle Stelle}{Tunica delle Stelle}
\end{tabularx}}

\subsubsection{Manuali, Tomi e Pergamene}\index[Tabelle]{Tabella Generazione Manuali e Tomi}\hypertarget{manualitomi}{}\label{manualitomi}\label{CPergamene}\hypertarget{CPergamene}{}

{\small \begin{tabularx}{\linewidth}{lll}
		\toprule
\rowcolor{gray!20}\textbf{3d6} & \textbf{Rarita' Pergamena} & \textbf{Pagine della Pergamena}\\
\toprule
3-11 &Comune & 1d4 \\
\rowcolor{gray!20}12-13 &Non Comune & 1d6 \\
14-16 &Rara & 1d8 \\
\rowcolor{gray!20}17 & Molto Rara & 1d10\\
18 & Leggendaria & 2d6\\
\end{tabularx}}

\medskip

{\small \begin{tabularx}{\linewidth}{ll}
		\toprule
 \rowcolor{gray!20}\textbf{3d6} & \textbf{Livello Incantesimo}\\
	\toprule
	3-9 & 1 \\
 \rowcolor{gray!20}10-12 & 2 \\
	13-16 & 3  \\
 \rowcolor{gray!20}17 & 4  \\
	18 & 5 \\
 \rowcolor{gray!20}18 & 6 \\
\end{tabularx}}

\medskip

{\small\begin{tabularx}{\linewidth}{ll}
		\toprule
  \rowcolor{gray!20}\textbf{Manuali, Tomi e Pergamene}&\textbf{3d6}\\
		\toprule
3-14 &	\hyperlink{manualitomi}{Pergamene}\\
\rowcolor{gray!20}15-16&	\hyperlink{manualitomi}{Manuali} \\
17-18&	\hyperlink{manualitomi}{Tomi}
\end{tabularx}}

\medskip\hypertarget{Manuali}{}

\noindent\begin{tabularx}{\linewidth}{ll}
	\toprule
\rowcolor{gray!20}\textbf{3d6} & \textbf{Manuali}\\
\toprule
3-12 & Manuale vuoto\\
\rowcolor{gray!20}13 & \hyperlink{Manuale dei Golem}{Manuale dei Golem}\\
14 & \hyperlink{Manuale della Buonasalute}{Manuale della Buona salute}\\
\rowcolor{gray!20}15 & \hyperlink{Manuale della Velocita' di azione}{Manuale della Velocita' di azione}\\
16 & \hyperlink{Manuale dell'Esercizio fisico}{Manuale dell'Esercizio fisico}\\
\rowcolor{gray!20}17 & Manuale della Buona Salute maledetto\\
18 & Manuale dell'Esercizio fisico maledetto
\end{tabularx}

\medskip\hypertarget{Tomi}{}

{\small\begin{tabularx}{\linewidth}{ll}
		\toprule
\rowcolor{gray!20}\textbf{3d6} & \textbf{Tomi}\\
\toprule
3-12 & Tomo vuoto\\
\rowcolor{gray!20}13 & \hyperlink{Tomo dell'Autorita' e dell'Influenza}{Tomo dell'Autorita' e dell'Influenza}\\
14 & \hyperlink{Tomo della Comprensione}{Tomo della Comprensione}\\
\rowcolor{gray!20}15 & \hyperlink{Tomo del Pensiero Limpido}{Tomo del Pensiero Limpido}\\
16 & Tomo dell'Autorita' e dell'Influenza maledetto\\
\rowcolor{gray!20}17 & Tomo della Comprensione maledetto\\
18 & Tomo del Pensiero Limpido maledetto
\end{tabularx}}

\subsubsection{Oggetti Magici vari}\index[Tabelle]{Tabella Generazione Oggetti Magici vari}\hypertarget{oggettimagicivari}{}\label{oggettimagicivari}

{\small\begin{tabularx}{\linewidth}{ll}
		\toprule
\rowcolor{gray!20}\textbf{Tipo Oggetto} & \textbf{1d12}\\
\toprule
1-4 & \hyperlink{Oggetti Magici Vari 1}{Oggetti Magici Vari 1}\\
\rowcolor{gray!20}5-7 & \hyperlink{Oggetti Magici Vari 2}{Oggetti Magici Vari 2}\\
8-10&\hyperlink{Oggetti Magici Vari 3}{Oggetti Magici Vari 3}\\
\rowcolor{gray!20}11  &\hyperlink{Oggetti Magici Vari 4}{Oggetti Magici Vari 4}\\
12  &\hyperlink{Rari e Leggendari}{Rari e Leggendari}
\end{tabularx}}

\subsubsection{Oggetti magici vari 1}\index[Tabelle]{Tabella Generazione Oggetti magici vari 1}\hypertarget{Oggetti Magici Vari 1}{}

{\small\begin{tabularx}{\linewidth}{ll}
		\toprule
\rowcolor{gray!20}\textbf{1d100} & \textbf{Oggetti magici vari 1}\\
\toprule
1-8 & \hyperlink{Acqua purificatrice}{Acqua purificatrice}\\
\rowcolor{gray!20}9-17 & \hyperlink{Battaglio dell'Apertura}{Battaglio dell'Apertura}\\
18-27 & \hyperlink{Borsa Conservante TipoI}{Borsa Conservante Tipo I}\\
\rowcolor{gray!20}28-34 & \hyperlink{Corda da Arrampicata}{Corda da Arrampicata}\\
35-43 & \hyperlink{Faretra Efficiente}{Faretra Efficiente}\\
\rowcolor{gray!20}44-48 & \hyperlink{Freccia localizzante}{Freccia localizzante}\\
49-52 & \hyperlink{Balestra dei Dardi Arcani}{Balestra dei Dardi Arcani}\\
\rowcolor{gray!20}53-57 & \hyperlink{Lanterna della Rivelazione}{Lanterna della Rivelazione}\\
58-60 & \hyperlink{Pergamena contro gli elementali}{Pergamena contro gli elementali}\\
\rowcolor{gray!20}61-63 & \hyperlink{Pergamena contro i nonmorti}{Pergamena contro i non morti}\\
63-65 & \hyperlink{Collana dell'Aria Salubre}{Collana dell'Aria Salubre}\\
\rowcolor{gray!20}66-69 & \hyperlink{Perla del Potere}{Perla del Potere}\\
70-73 & \hyperlink{Pietra della Buona Sorte}{Pietra della Buona Sorte}\\
\rowcolor{gray!20}74-80 & \hyperlink{Filatterio contro i nonmorti}{Filatterio contro i non morti}\\
81-83 & \hyperlink{Solvente Universale}{Solvente Universale}\\
\rowcolor{gray!20}84-89 & \hyperlink{Polvere dell'Aridita'}{Polvere dell'Aridita'}\\
90-94 & \hyperlink{Unguento di Ljust}{Unguento di Ljust}\\
\rowcolor{gray!20}95-100 & \hyperlink{Zainetto Pratico}{Zainetto Pratico}
\end{tabularx}}

\subsubsection{Oggetti magici vari 2}\index[Tabelle]{Tabella Generazione Oggetti magici vari 2}\hypertarget{Oggetti Magici Vari 2}{}

{\small\begin{tabularx}{\linewidth}{ll}
		\toprule
\rowcolor{gray!20}\textbf{1d100} & \textbf{Oggetti magici vari 2}\\
\toprule
1-8 &\hyperlink{Braciere degli Elementali del Fuoco}{Braciere degli Elementali del Fuoco}\\
\rowcolor{gray!20}9-17 &\hyperlink{Braciere del Sonno maledetto}{Braciere del Sonno maledetto}\\
18-23 & \hyperlink{Cubo di protezione dal freddo}{Cubo di protezione dal freddo}\\
\rowcolor{gray!20}24-29 & \hyperlink{Incensiere degli Elementali dell'Aria}{Incensiere degli Elementali dell'Aria}\\
30-34 & \hyperlink{Incenso della meditazione}{Incenso della meditazione}\\
\rowcolor{gray!20}35-43 & \hyperlink{Rete Intralciante}{Rete Intralciante}\\
44-52 & \hyperlink{Rete Intrappolante}{Rete Intrappolante}\\
\rowcolor{gray!20}53-58 & \hyperlink{Scopa dell'Attaccoanimato}{Scopa dell'Attacco animato}\\
59-63 & \hyperlink{Scopa Volante}{Scopa Volante}\\
\rowcolor{gray!20}64-67 & \hyperlink{Borsa Divorante}{Borsa Divorante}\\
68-70 & \hyperlink{Pergamena protettiva contro la magia}{Pergamena protettiva contro la magia}\\
\rowcolor{gray!20}71-73 & \hyperlink{Pergamena contro i licantropi}{Pergamena contro i licantropi}\\
74-76 & \hyperlink{Scopa del Volo maledetto}{Scopa del Volo maledetto}\\
\rowcolor{gray!20}77-84 & \hyperlink{Pietre parlanti}{Pietre parlanti}\\
85-88 & \hyperlink{Specchio della Duplicazione}{Specchio della Duplicazione}\\
\rowcolor{gray!20}89-90 & \hyperlink{Specchio IntrappolaVita}{Specchio Intrappola Vita}\\
91-94 & \hyperlink{Corno del Valhalla}{Corno del Valhalla}\\
\rowcolor{gray!20}95-98 & \hyperlink{Tappe to Volante}{Tappeto Volante}\\
99-100 & \hyperlink{Zappa dei Titani}{Zappa dei Titani}
\end{tabularx}}

\subsubsection{Oggetti magici vari 3}\index[Tabelle]{Tabella Generazione Oggetti magici vari 3}\hypertarget{Oggetti Magici Vari 3}{}

{\small\begin{tabularx}{\linewidth}{ll}
		\toprule
\rowcolor{gray!20}\textbf{1d100} & \textbf{Oggetti magici vari 3}\\
\toprule
1-15 & \hyperlink{Borsa dell'Annullamento}{Borsa dell'Annullamento}\\
\rowcolor{gray!20}16-25 & \hyperlink{Brocca dell'AcquaInfinita}{Brocca dell'Acqua Infinita}\\
26-35 & \hyperlink{Ceppi Dimensionali}{Ceppi Dimensionali}\\
\rowcolor{gray!20}36-46 & \hyperlink{Colla Suprema}{Colla Suprema}\\
47-56 & \hyperlink{Polvere Rivelatrice}{Polvere Rivelatrice}\\
\rowcolor{gray!20}57-68 & \hyperlink{Polvere della Sparizione}{Polvere della Sparizione}\\
69-70 & \hyperlink{Palla di Cristallo ipnotica}{Palla di Cristallo ipnotica}\\
\rowcolor{gray!20}71-81 & \hyperlink{Polvere delloS tarnuto e del Soffocamento}{Polvere dello Starnuto e del Soffocamento}\\
82-90 & \hyperlink{Pietra Arcana}{Pietra Arcana}\\
\rowcolor{gray!20}91-96 & \hyperlink{Pietra del Peso}{Pietra del Peso}\\
97-100 & \hyperlink{Ventaglio Arcano}{Ventaglio Arcano}
\end{tabularx}}

%\begin{center}
%\includegraphics[width=0.8\linewidth]{immagini/ancientdrum.png}
%\end{center}

\subsubsection{Oggetti magici vari 4}\index[Tabelle]{Tabella Generazione Oggetti magici vari 4}\hypertarget{Oggetti Magici Vari 4}{}

{\small\begin{tabularx}{\linewidth}{ll}
		\toprule
\rowcolor{gray!20}\textbf{1d100} & \textbf{Oggetti magici vari 4}\\
\toprule
1-10 & \hyperlink{Ampolla delle maledizioni}{Ampolla delle maledizioni}\\
\rowcolor{gray!20}11-18 & \hyperlink{Battaglio del Cannibalismo}{Battaglio del Cannibalismo}\\
19-28 & \hyperlink{Borsa Conservante TipoII}{Borsa Conservante Tipo II}\\
\rowcolor{gray!20}29-37 & \hyperlink{Buco Portatile}{Buco Portatile}\\
38-43 & \hyperlink{Corda dell'Intralciamento}{Corda dell'Intralciamento}\\
\rowcolor{gray!20}44-50 & \hyperlink{Corda Strozzatrice}{Corda Strozzatrice}\\
51-55 & \hyperlink{Corno di Distruzione}{Corno di Distruzione}\\
\rowcolor{gray!20}56-60 & \hyperlink{Cubo di Forza}{Cubo di Forza}\\
61-64 & \hyperlink{Fasce di Ferro del Vincolo}{Fasce di Ferro del Vincolo}\\
\rowcolor{gray!20}65-73 & \hyperlink{Incenso dell'Ossessione}{Incenso dell'Ossessione}\\
74-82 & \hyperlink{Mazzo delle Illusioni}{Mazzo delle Illusioni}\\
\rowcolor{gray!20}83-84 & \hyperlink{Pietra degli Elementali della Terra}{Pietra degli Elementali della Terra}\\
85-91 & \hyperlink{Piffero dello Spavento}{Piffero dello Spavento}\\
\rowcolor{gray!20}92-94 & \hyperlink{Piuma Arcana}{Piuma Arcana}\\
95-96 & \hyperlink{Tamburi del Panico}{Tamburi del Panico}\\
\rowcolor{gray!20}97-98 & \hyperlink{Tamburi dello Stordimento}{Tamburi dello Stordimento}\\
99-100 & \hyperlink{Turibolo dell'Evocazione maledetta}{Turibolo dell'Evocazione maledetta}
\end{tabularx}}

%\begin{center}
%\includegraphics[width=0.9\linewidth]{immagini/armatura-med.png}
%\end{center}

\subsubsection{Rari e Leggendari}\index[Tabelle]{Tabella Generazione Rari e Leggendari}\hypertarget{Rari e Leggendari}{}

{\small\begin{tabularx}{\linewidth}{ll}
		\toprule
\rowcolor{gray!20}\textbf{1d100} & \textbf{Oggetto Magico}\\
\toprule
1-3 & \hyperlink{Ali del Volo}{Ali del Volo}\\
\rowcolor{gray!20}4-6 & \hyperlink{Ampolla di Ferro}{Ampolla di Ferro}\\
7-10 & \hyperlink{Anfora elementale dell'acqua}{Anfora elementale dell'acqua}\\
\rowcolor{gray!20}11-12 & \hyperlink{Apparato del Granchio}{Apparato del Granchio}\\
13-15 & \hyperlink{Barca Pieghevole}{Barca Pieghevole}\\
\rowcolor{gray!20}17-20 & \hyperlink{Borsa Conservante Tipo III}{Borsa Conservante Tipo III}\\
21-24 & \hyperlink{Borsa Conservante Tipo IV}{Borsa Conservante Tipo IV}\\
\rowcolor{gray!20}25-28 & \hyperlink{Borsa dei Fagioli}{Borsa dei Fagioli}\\
29-30 & \hyperlink{Bottiglia dell'Efreeti}{Bottiglia dell'Efreeti}\\
\rowcolor{gray!20}31 & \hyperlink{Brocca delle Pozioni}{Brocca delle Pozioni}\\
32-35 & \hyperlink{Candela di Invocazione}{Candela di Invocazione}\\
\rowcolor{gray!20}36-39 & \hyperlink{Filatterio della giovinezza}{Filatterio della giovinezza}\\
40-42 & \hyperlink{Fortezza Istantanea}{Fortezza Istantanea}\\
\rowcolor{gray!20}43-45 & \hyperlink{Mazzo delle Meraviglie}{Mazzo delle Meraviglie}\\
46-50 & \hyperlink{Miniatura dal Potere Meraviglioso}{Miniatura dal Potere Meraviglioso}\\
\rowcolor{gray!20}51-55 & \hyperlink{Munizione dell'Uccisione}{Munizione dell'Uccisione}\\
56-60 & \hyperlink{Palla di Cristallo}{Palla di Cristallo}\\
\rowcolor{gray!20}61-67 & \hyperlink{Piffero delle Fogne}{Piffero delle Fogne}\\
68-75 & \hyperlink{Pigmenti delle Meraviglie}{Pigmenti delle Meraviglie}\\
\rowcolor{gray!20}76-83 & \hyperlink{Portale Cubico}{Portale Cubico}\\
84-86 & \hyperlink{Pozzo dei Molti Mondi}{Pozzo dei Molti Mondi}\\
\rowcolor{gray!20}87-89 & \hyperlink{Specchio dell'Abilita' mentale}{Specchio dell'Abilita' mentale}\\
90-91 & \hyperlink{Sfera dell'Annientamento}{Sfera dell'Annientamento}\\
\rowcolor{gray!20}92-94 & \hyperlink{Turibolo Elementale dell'aria}{Turibolo Elementale dell'aria}\\
95-96 & \hyperlink{Vano Portatile}{Vano Portatile}\\
\rowcolor{gray!20}97-98 & \hyperlink{Zoccoli della Velocita'}{Zoccoli della Velocita'}\\
99-100 & \hyperlink{Zoccoli dello Zefiro}{Zoccoli dello Zefiro}
\end{tabularx}}

}

%\begin{center}
%\includegraphics[width=0.6\linewidth]{immagini/ancientbraziers2.png}
%
%\emph{Teotihuacano Old God vessels: Top - stone brazier in Natural History Museum of Los Angeles County}
%\end{center}

\end{multicols}

\pagebreak

\section{Descrizione degli Oggetti Magici}\index{Descrizione degli Oggetti Magici}

\begin{multicols}{2}

Gli oggetti magici sono presentati in ordine alfabetico. La descrizione di un oggetto magico fornisce il nome dell'oggetto, la sua categoria, rarità e le sue proprietà magiche.

Benché i costi siano riportati è sempre bene concedere gli oggetti magici come premi, tesoro, a seguito di missione.

In linea di massima un oggetto Comune, l'unico che potrebbe trovarsi facilmente in una grande città, puoi costare dai 50 ai 100 mo, uno Non Comune tra i 150 ed i 500 mo, uno Raro tra i 500 e i 5000 mo, uno Molto Raro fino a 30000 mo e oltre c'è solo la leggenda...

Oggetti con un bonus oltre il +2, o Leggendari, non si comprano mai, deve essere un epica avventura a farli trovare.

\medskip

Anche gli incantesimi sono oggetti magici e come tali, se il Narratore permette, possono essere acquistati (orrore! non c'è nulla di più bello trovare un nuovo incantesimo tra i tesori di un'avventura).

Un incantesimo costa in monete d'oro livello*livello*livello*80\index{Comprare un incantesimo}

\medskip

\oggettomagico{Accecante}

\textbf{Aura:} Invocazione moderata; \textbf{Costo:} 3000 mo

\textbf{Requisiti:} Creare Oggetti Magici 2, Luce diurna

Uno \textbf{scudo} dotato di questo incantamento emana una luce accecante per un massimo di due volte al giorno su comando di chi lo impugna. Tutti coloro che si trovano entro 6 metri dallo scudo, eccetto chi lo impugna, devono superare un Tiro Salvezza su Riflessi con DC 14 o restano Accecati per 1d4 round.

\oggettomagico{Accumula Incantesimi}

\textbf{Aura:} Invocazione forte e variabile; \textbf{Costo:} 3000 mo

\textbf{Requisiti:} Creare Oggetti Magici 2

Un'\textbf{arma} Accumula Incantesimi permette a un incantatore di immagazzinare un incantesimo con bersaglio fino livello 3 nell'arma. L'incantesimo deve avere un tempo di lancio standard di 2 Azioni. Ogni volta che l'arma colpisce una creatura e quest'ultima subisce dei danni, chi impugna l'arma con una Azione Immediata può liberare l'incantesimo.

Una volta che l'incantesimo viene lanciato, un incantatore può immagazzinarvi all'interno qualsiasi altro incantesimo con bersaglio, sempre fino a livello 3.

L'arma rivela magicamente a chi la impugna il nome dell'incantesimo attualmente contenuto. Un'arma Accumula Incantesimi creata casualmente ha una probabilità del 50\% di avere già un incantesimo contenuto al suo interno. Questa capacità speciale può essere aggiunta solo ad armi da mischia.

Un'arma Accumula Incantesimi emette una forte aura della scuola Invocazione, più l'aura dell'incantesimo contenuto.

\oggettomagico{Acqua purificatrice}

\textbf{Rarità:} Raro; \textbf{Costo:} 500 mo

questo \textbf{liquido} dolce può essere usato per purificare l'acqua (anche per desalinizzare quella di mare) e per trasformare veleni, acidi e altri liquidi nocivi in una bevanda potabile. Inoltre, l'acqua purificatrice neutralizza l'efficacia di ogni altra pozione. Questa pozione può trasformare fino a 1000 metri cubi di quasi tutti i liquidi a base d'acqua, ma solo 10 metri cubi di acido. Gli effetti sono permanenti e un liquido purificato non può essere deteriorato o contaminato di nuovo per un periodo di 5d4 round.

\oggettomagico{Adamantio}

\textbf{Rarità:} Non Comune; \textbf{Costo:} +700 mo

Chi indossa questa \textbf{armatura} per il primo colpo critico del round diventa un colpo normale (ma non protegge dall'esplosione del danno).

\oggettomagico{Adattiva}

\textbf{Aura:} Trasmutazione debole; \textbf{Costo:} 1500 mo

\textbf{Requisiti:} Creare Oggetti Magici, Lista Animali e Piante

Questa capacità può essere aggiunta solo agli \textbf{archi} compositi. Un arco Adattivo reagisce alla forza di chi lo impugna, agendo come un arco con un bonus di Forza pari a quello di chi lo sta impugnando. Chi lo impugna può scoccare con un bonus di Forza inferiore (e causare meno danni) se lo desidera.

\oggettomagico{Affilata}

\textbf{Aura:} Trasmutazione moderata; \textbf{Costo:} 5000 mo

\textbf{Requisiti:} Creare Oggetti Magici 2, Lista della Terra

Questa capacità in caso di critico consente di contare il numero di 6 tirati aumentandolo di 1. Solo le \textbf{armi} da mischia taglienti o perforanti possono essere Affilate.

\oggettomagico{Ali del Volo}

\textbf{Rarità:} Leggendario; \textbf{Costo:} 54000 mo

mentre indossi questo \textbf{Mantello}, puoi usare due azioni per pronunciare la sua parola di comando, trasformandola in un paio di ali da pipistrello o da uccello che spuntano dalla tua schiena per 1 ora o finché non ripeti la parola di comando con un'Azione. Le ali ti forniscono velocità di volo 18 metri. Quando scompaiono, non potrai più usarle fino all'alba del giorno dopo.

\oggettomagico{Ammazza Draghi}

\textbf{Aura:} Invocazione moderata; \textbf{Costo:} 8000 mo

\textbf{Requisiti:} Creare Oggetti Magici 2

Quando colpisci un drago con quest'\textbf{arma}, il drago subisce 3d6 danni aggiuntivi del tipo dell'arma. Ai fini di quest'arma, drago è qualsiasi creatura del tipo drago.

\oggettomagico{Ammazza Giganti}

\textbf{Aura:} Invocazione moderata; \textbf{Costo:} 8000 mo

\textbf{Requisiti:} Creare Oggetti Magici 2

Quando colpisci un gigante con quest'\textbf{arma}, il gigante subisce 2d6 danni aggiuntivi del tipo dell'arma e deve superare un Tiro Salvezza su Tempra con DC 18 o cadere prono. Ai fini di quest'arma, gigante è qualsiasi creatura del tipo gigante.

\oggettomagico{Amorfa}

\textbf{Aura:} Trasmutazione moderata; \textbf{Costo:} 2250 mo

\textbf{Requisiti:} Creare Oggetti Magici 2, Metamorfosi,

Una volta al giorno a comando, chi indossa l'\textbf{armatura} (insieme a qualsiasi equipaggiamento indossi) può assumere la forma di un liquido viscoso che è in grado di passare attraverso qualsiasi spazio nel quale potrebbe ragionevolmente scorrere del fango denso. Mentre si usa questa capacità, la propria velocità viene ridotta a 3 metri e si possono effettuare solo azioni di movimento. Si può assumere questa forma per 1 minuto o finché non si spende un'Azione di Movimento per tornare alla propria forma naturale. Un'armatura Amorfa deve essere fatta principalmente di cuoio, stoffa o altro materiale organico e flessibile.

\oggettomagico{Ampolla delle maledizioni}

\textbf{Rarità:} Raro; \textbf{Costo:} 800 mo

questo oggetto ha l'aspetto di una ampolla, bottiglia, caraffa, contenitore, fiasco, o brocca. Può contenere un liquido o emanare fumo. Quando l'\textbf{ampolla} viene stappata per la prima volta, tutte le creature entro 9 m vengono maledette.

\oggettomagico{Ampolla di Ferro}

\textbf{Rarità:} Leggendario; \textbf{Costo:} 35000 mo

questa \textbf{bottiglia} di ferro ha un tappo di ottone. Puoi usare due azioni per pronunciare la parola di comando dell'ampolla, prendendo come bersaglio una creatura visibile entro 18 metri da te. Se il bersaglio è nativo di un piano di esistenza diverso da quello in cui ti trovi, deve superare un Tiro Salvezza su Volontà con DC 21 o venir intrappolato nell'ampolla. Se il bersaglio è già stato intrappolato nell'ampolla, riceve +1d6 al Tiro Salvezza. Una volta intrappolata, la creatura rimarrà nell'ampolla finché non sarà liberata. L'ampolla può contenere solo una creatura alla volta. Una creatura intrappolata nell'ampolla non ha bisogno di respirare, mangiare o dormire e non invecchia. Puoi usare due azioni per rimuovere il tappo dell'ampolla e liberare la creatura che contiene. La creatura sarà amichevole verso di te e i tuoi compagni per 1 ora e obbedirà ai vostri comandi per quella durata. Se non le impartisci comandi o gliene dai uno che provocherebbe la sua morte, si difenderà ma non compirà altre azioni. Al termine della durata, la creatura agirà in base al suo normale comportamento

L'incantesimo \emph{identificare} rivela che una creatura si trova all'interno dell'ampolla, ma l'unico modo per determinare che sorta di creatura sia è di aprire l'ampolla. Un'ampolla di ferro appena scoperta potrebbe già contenere una creatura scelta dal Narratore o determinata casualmente.

\medskip

\noindent\begin{tabularx}{\linewidth}{ll}
	\toprule
\rowcolor{gray!20}\textbf{d100} &\textbf{Contiene}\\
\toprule
1-50 &Vuota\\
\rowcolor{gray!20}51-66 &Demone \\
67 &Angelo Deva\\
\rowcolor{gray!20}68-69 &Diavolo (superiore)\\
70-73 &Diavolo (inferiore)\\
\rowcolor{gray!20}74-75 &Genio Djinni\\
76-77 &Genio Efreeti\\
\rowcolor{gray!20}78-83 &Elementale (qualsiasi)\\
84-86 &Persecutore invisibile\\
\rowcolor{gray!20}87-90 &Megera notturna\\
91 &Angelo Planetar\\
\rowcolor{gray!20}92-95 &Salamandra\\
96 &Angelo Solar\\
\rowcolor{gray!20}97-99 &Succube/Incubo\\
100 &La Topi
\end{tabularx}
\medskip

\oggettomagico{Amuleto Antiveleno}

\textbf{Aura:} Necromantica moderata; \textbf{Costo:} 3000 mo

\textbf{Requisiti:} Creare Oggetti Magici, \hyperlink{Rimuovi Veleno}{Rimuovi Veleno}; \textbf{Rarità:} Rara

questa \textbf{gemma} appesa a una catenella d'argento è nera e lucida. Chi la indossa ha una +2 al Tiro Salvezza contro veleno.

\oggettomagico{Amuleto Cicatrizzante}

\textbf{Aura:} Necromantica forte; \textbf{Costo:} 25000 mo

\textbf{Requisiti:} Creare Oggetti Magici, \hyperlink{Cura Ferite}{Cura Ferite}; \textbf{Rarità:} Molto Raro

questa \textbf{gemma} appesa a una catenella d'oro è rossa e brillante. Chi la indossa recupera i Punti Ferita due volte più rapidamente del normale (anche Punti Ferita Massimi). L'amuleto impedisce di subire danni da Sanguinamento.

\oggettomagico{Amuleto Contro la Possessione}

\textbf{Aura:} Abiurativa forte; \textbf{Costo:} 32000 mo

\textbf{Requisiti:} Creare Oggetti Magici 3, \hyperlink{AnellodelloScudoMentale}{Scudo Mentale}; \textbf{Rarità:} Molto Raro

il possessore di questo \textbf{amuleto} di rame diviene immune agli incantesimi di possessione e dominazione.

\oggettomagico{Amuleto della Cancrena}

questa \textbf{gemma} incisa appesa a una catenella sembra essere di scarso valore. Se un personaggio la tiene con se per più di 1 giorno, viene colpito da una terribile cancrena che gli fa perdere permanentemente 1 punto di Destrezza, Costituzione e Carisma alla settimana. La gemma (e la cancrena) possono essere neutralizzate solo da \hyperlink{Rimuovi Maledizione}{Rimuovi Maledizione} e \hyperlink{Rimuovi Malattia}{Rimuovi Malattia}, seguiti da \hyperlink{BastonedellaGuarigione}{Guarigione} o \hyperlink{Desiderio}{Desiderio}. La cancrena può anche essere sconfitta macinando un amuleto della salute e spargendone la polvere sul personaggio afflitto

\oggettomagico{Amuleto della Localizzazione inevitabile}

\textbf{Aura:} Divinazione forte; \textbf{Costo:} 8000 mo

\textbf{Requisiti:} Creare Oggetti Magici 2, \hyperlink{PozionedellaChiaroveggenzaanimale}{Chiaroveggenza}; \textbf{Rarità:} Molto Raro

questo \textbf{amuleto} maledetto ha l'apparenza di un amuleto dell'introvabilità. Al contrario, rende il possessore vulnerabile a questo tipo di magia. La probabilità di osservare il possessore e la durata di incantesimi usati per tale scopo raddoppiano.

\oggettomagico{Amuleto della Resistenza Fisica}

\textbf{Aura:} Abiurazione forte; \textbf{Costo:} 8000 mo

\textbf{Requisiti:} Creare Oggetti Magici 1, \hyperlink{ResistenzaalVeleno}{Resistenza}; \textbf{Rarità:} Raro

non mentre indossi questo \textbf{amuleto} hai un +2 ai Tiri Salvezza su Tempra.

\oggettomagico{Amuleto di Protezione dalla Individuazione e Localizzazione}

\textbf{Aura:} Abiurazione forte; \textbf{Costo:} 20000 mo

\textbf{Requisiti:} Creare Oggetti Magici 3, \hyperlink{AnellodelloScudoMentale}{Scudo Mentale}; \textbf{Rarità:} Raro

mentre indossi questo \textbf{amuleto} sei celato alla magia di divinazione. Non puoi essere preso come bersaglio da queste magie o percepito tramite sensori magici di scrutamento.

\oggettomagico{Arma Anatema}

\textbf{Aura:} Evocazione moderata; \textbf{Costo:} 3000 mo

\textbf{Requisiti:} Creare Oggetti Magici 2, Lista Abiurazione

Un'\textbf{arma} Anatema eccelle nell'attaccare certe creature. Contro il nemico prescelto, il suo bonus effettivo diventa di +2. L'arma, inoltre, infligge un danno critico automaticamente contro tale nemico. Per determinare casualmente il nemico prescelto dell'arma si usa la tabella seguente:

\medskip

\noindent\begin{tabularx}{\linewidth}{ll}
	\toprule
\rowcolor{gray!20}d\% &Nemico prescelto\\
\toprule
01-05 &Aberrazioni\\
\rowcolor{gray!20}06-09 &Bestie\\
10-16 &Costrutti\\
\rowcolor{gray!20}17-22 &Draghi\\
23-27 &Fatati\\
\rowcolor{gray!20}28-60 &Umanoidi (scegliere sottotipo)\\
61-70 &Creature Magiche\\
\rowcolor{gray!20}71-72 &Melme\\
73-88 &Immondi\\
\rowcolor{gray!20}89-90 &Piante\\
91-98 &Non Morti\\
\rowcolor{gray!20}99-100 &Insetti
\end{tabularx}

\medskip

\oggettomagico{Anello Accumula Incantesimi}

\textbf{Rarità:} Raro; \textbf{Costo:} 24000 mo

questo \textbf{anello} immagazzina gli incantesimi lanciati su di esso, conservandoli fino a che chi lo indossa non ne faccia uso. L'anello può accumulare fino a 3 Incantesimi fino ad un massimo di 15 Punti Magia, con un massimo di 6 Punti Magia per singolo incantesimo.

Qualsiasi creatura può lanciare un incantesimo accumulato di livello da 1 a 5 sull'anello toccandolo. L'incantesimo ha una DC pari a 10 + 2 x Livello incantesimo, l'eventuale Tiro per Colpire viene effettuato da chi lancia l'incantesimo.

Per caricare l'incantesimo nell'anello l'incantatore deve mirare all'anello per farlo assorbire. Se l'anello non può contenere l'incantesimo, l'incantesimo si manifesta normalmente. Un incantesimo lanciato tramite questo anello non è più contenuto al suo interno e libera spazio per altri incantesimi.

\oggettomagico{Anello d'Arma}

\textbf{Aura:} Ammaliamento moderato; \textbf{Costo:} 4000 mo

\textbf{Requisiti:} Creare Oggetti Magici 2, Aura Magica

Un Anello d'\textbf{arma} può essere applicato solo ad armi da mischia. Quando inserito nell'elsa dell'arma il personaggio può due volte al giorno usare 1 Azione e fare diventare l'arma magica +1 per 10 round. Tale abilità non può essere attivata su oggetti che abbiano già un bonus magico.
La capacità dell'Anello d'Arma si ricarica all'alba. L'anello d'arma non si conta nei limiti per indossare anelli.

\oggettomagico{Anello d'Arma maggiore}

\textbf{Aura:} Abiurazione forte; \textbf{Costo:} 8000 mo

\textbf{Requisiti:} Creare Oggetti Magici 2, Aura Magica

Un Anello d'\textbf{arma} maggiore può essere applicato solo ad armi da mischia ed ha un solo potere. Quando viene trovato è necessario tirare sulla tabella sottostante per determinare il tipo di talento che conferisce l'anello. L'anello d'arma non si conta nei limiti per indossare anelli. Un arma può avere un numero di Anelli d'Arma massimo pari al suo bonus magico.

\medskip

\noindent\begin{tabularx}{\linewidth}{lX|lX}
	\toprule
\rowcolor{gray!20}\textbf{1d100} & \textbf{Talento} & \textbf{1d100} & \textbf{Talento} \\
\toprule
1-6 & Maledetta -2 & 55-60 & Folgorante \\
\rowcolor{gray!20}7-12 & Coraggiosa & 61-66 & Gelida \\
13-18 & Del Dolore & 67-72 & Lingua di fuoco \\
\rowcolor{gray!20}19-24 & Perfida & 73-78 & Marina \\
25-30 & +3 & 79-84 & Mutaforma \\
\rowcolor{gray!20}31-36 & Maledetta -1 & 85-90 & +2 \\
37-42 & Arma Titanica & 91-96 & Tonante \\
\rowcolor{gray!20}43-48 & Caos & 97-100 & Trasformante \\
49-54 & Corrosiva & & \\

\end{tabularx}

\medskip

Quando inserito nell'elsa dell'\textbf{arma} il personaggio può due volte al giorno usare 1 Azione e aggiungere all'arma il talento tirato, per 10 round. Se l'arma non è almeno magica +1 diventa magica +1. La capacità dell'Anello d'Arma maggiore si ricarica all'alba.

\oggettomagico{Anello degli Elementali del Fuoco}

\textbf{Rarità:} Leggendario; \textbf{Costo:} 250000 mo

questo \textbf{anello} è collegato al Piano Elementale del Fuoco. Mentre lo indossi, hai +1d6 ai Tiri per Colpire contro gli elementali del Piano Elementale del Fuoco, ed essi hanno -1d6 ai Tiri per Colpire effettuati contro di te.

Puoi spendere 2 cariche dell'anello per lanciare dominare mostri su di un elementale del fuoco. Inoltre, hai resistenza ai danni da fuoco. Puoi parlare e comprendere l'Ignan.

Se aiuti a uccidere un elementale del fuoco mentre indossi l'anello, ottieni accesso alle seguenti proprietà aggiuntive:

\smallskip- Hai immunità ai danni da fuoco.

\smallskip- Puoi lanciare tramite l'anello i seguenti incantesimi, spendendo il numero di cariche richieste: Onda rovente (1 carica), muro di fuoco (3 cariche) o palla di fuoco (2 cariche).

\medskip

L'anello ha 5 cariche. Recupera 1d4 + 1 cariche ogni giorno all'alba.

Gli incantesimi lanciati tramite l'anello hanno DC del Tiro Salvezza 21.

\oggettomagico{Anello degli Elementali dell'Acqua}

\textbf{Rarità:} Leggendario; \textbf{Costo:} 250000 mo

questo \textbf{anello} è collegato al Piano Elementale dell'Acqua. Mentre lo indossi, hai +1d6 ai Tiri per Colpire contro gli elementali del Piano Elementale dell'Acqua ed essi hanno -1d6 ai Tiri per Colpire effettuati contro di te.

Puoi spendere 2 cariche dell'anello per lanciare dominare mostri su di un elementale dell'acqua. Inoltre, puoi stare in piedi e camminare sulle superfici liquide come se fossero terreno solido. Puoi parlare e comprendere l'Aquan.

Se aiuti a uccidere un elementale dell'acqua mentre indossi l'anello, ottieni accesso alle seguenti proprietà aggiuntive:

\smallskip- Puoi respirare sott'acqua e hai velocità di nuovo pari alla tua velocità di movimento.

\smallskip- Puoi lanciare tramite l'anello i seguenti incantesimi, spendendo il numero di cariche richieste: creare o distruggere acqua (1 carica), controllare tempo atmosferico (3 cariche), muro di ghiaccio (3 cariche) o tempesta di ghiaccio (2 cariche).

\medskip
L'anello ha 5 cariche. Recupera 1d4 + 1 cariche ogni giorno all'alba. Gli incantesimi lanciati tramite l'anello hanno DC del Tiro Salvezza 21.

\oggettomagico{Anello degli Elementali dell'Aria}

\textbf{Rarità:} Leggendario; \textbf{Costo:} 250000 mo

questo \textbf{anello} è collegato al Piano Elementale dell'Aria. Mentre lo indossi, hai +1d6 ai Tiri per Colpire contro gli elementali del Piano Elementale dell'Aria, ed essi hanno -1d6 ai Tiri per Colpire effettuati contro di te.

Puoi spendere 2 cariche dell'anello per lanciare dominare mostri su di un elementale dell'aria. Inoltre, quando cadi, scendi di 18 metri per round e non subisci danni dalla caduta. Puoi parlare e comprendere l'Ictun.

Se aiuti a uccidere un elementale dell'aria mentre indossi l'anello, ottieni accesso alle seguenti proprietà aggiuntive:

\smallskip- Hai resistenza ai danni da elettricità.

\smallskip- Hai velocità di volo pari alla tua velocità di movimento e puoi fluttuare.

\smallskip- Puoi lanciare tramite l'anello i seguenti incantesimi, spendendo il numero di cariche richieste: Fulmine a catena (3 cariche), folata di vento (2 cariche) o muro di vento (1 carica).

\medskip

L'anello ha 5 cariche. Recupera 1d4 + 1 cariche ogni giorno all'alba.

Gli incantesimi lanciati tramite l'anello hanno DC del Tiro Salvezza 21.

\oggettomagico{Anello degli Elementali della Terra}

\textbf{Rarità:} Leggendario; \textbf{Costo:} 250000 mo

questo \textbf{anello} è collegato al Piano Elementale della Terra. Mentre lo indossi, hai +1d6 ai Tiri per Colpire contro gli elementali del Piano Elementale della Terra, ed essi hanno -1d6 ai Tiri per Colpire effettuati contro di te.

Puoi spendere 2 cariche dell'anello per lanciare dominare mostri su di un elementale della terra. Inoltre, puoi muoverti su terreno difficile composto da macerie, pietre o terra come se fosse terreno normale. Puoi parlare e comprendere il Tremun.

Se aiuti a uccidere un elementale della terra mentre indossi l'anello, ottieni accesso alle seguenti proprietà aggiuntive:

\smallskip- Hai resistenza ai danni da acido.

\smallskip- Puoi muoverti attraverso la terra o la roccia solida come se fossero terreno difficile. Se vi termini il tuo round, vieni proiettato fuori nello spazio non occupato più vicino che hai occupato per ultimo.

\smallskip- Puoi lanciare tramite l'anello i seguenti incantesimi, spendendo il numero di cariche richieste: \hyperlink{Scolpire Pietra}{Scolpire Pietra} (2 cariche), muro di pietra (3 cariche) o pelle di pietra (1 carica).

\medskip

L'anello ha 5 cariche. Recupera 1d4 + 1 cariche ogni giorno all'alba.

Gli incantesimi lanciati tramite l'anello hanno DC del Tiro Salvezza 21.

\oggettomagico{Anello dei Tre Desideri}

\textbf{Rarità:} Leggendario; \textbf{Costo:} 75000 mo

mentre indossi quest'\textbf{anello}, puoi usare due azioni per spendere 1 delle sue 1d3 cariche per lanciare tramite esso l'incantesimo desiderio. L'anello perde la sua magia quando usi l'ultima carica.

\oggettomagico{Scarfarotto del Calore}

\textbf{Rarità:} Non Comune; \textbf{Costo:} 5000 mo

mentre indossi questo \textbf{calzino}, hai resistenza ai danni da freddo. Inoltre, tu e tutto quello che indossi e trasporti siete immuni agli effetti delle temperature basse fino a -45° C.

\oggettomagico{Anello del Controllo delle persone}

\textbf{Rarità:} Raro; \textbf{Costo:} 2500 mo

questo \textbf{anello} conferisce a chi lo indossa la capacità di usare l'incantesimo \emph{charme} una volta la giorno. L'effetto dura finché chi esercita il controllo non vi pone termine, passa 1 ora o l'incantesimo viene disperso.

\oggettomagico{Anello del Controllo delle piante}

\textbf{Rarità:} Molto Raro; \textbf{Costo:} 5000 mo

chi indossa questo \textbf{anello} può controllare le piante e le creature vegetali in un'area quadrata di 3x3 m entro una distanza di 18 metri. Anche se una pianta è immobile, essa si può spostare mentre è sotto l'effetto di questo anello. Il controllo dura fintanto che chi lo esercita mantiene una concentrazione totale, che impedisce ogni altra Azione.

\oggettomagico{Anello del Nuoto}

\textbf{Rarità:} Non Comune; \textbf{Costo:} 3000 mo

mentre indossi questo \textbf{anello}, hai velocità di nuoto pari al tuo Movimento.

\oggettomagico{Anello del Salto}

\textbf{Rarità:} Non Comune; \textbf{Costo:} 2500 mo

mentre indossi questo \textbf{anello}, con due azioni puoi lanciare tramite esso l'incantesimo salto\emph{saltare} a volontà, ma il bersaglio puoi essere solo tu.

\oggettomagico{Anello dell'Ariete}

\textbf{Rarità:} Raro; \textbf{Costo:} 5000 mo

mentre indossi questo \textbf{anello}, puoi usare due azioni per spendere da 1 a 3 cariche per attaccare una creatura visibile entro 18 metri da te.

L'anello produce una testa di ariete spettrale ed effettua il suo Tiro per Colpire a distanza con un bonus di +7. Se colpisce, per ogni carica spesa, il bersaglio subisce 2d10 danni da forza e viene spinto di 1 metro lontano da te.

In alternativa, puoi spendere da 1 a 3 cariche dell'anello, con due azioni per carica, per tentare di sfondare una porta entro 18 metri da te. L'ariete ha DC 18 +6 per ogni carica spesa per sfondare.

Questo anello ha 3 cariche, e recupera 1 carica spesa ogni mattina all'alba.

\oggettomagico{Anello dell'Evocazione dello Djinni}

\textbf{Rarità:} Leggendario; \textbf{Costo:} 35000 mo

mentre indossi quest'\textbf{anello}, puoi pronunciarne la parola di comando, con due azioni, per evocare uno specifico djinni del Piano Elementale dell'Aria. Lo djinni compare in uno spazio non occupato a tua scelta, entro 36 metri da te. Resta finché rimani concentrato (come se ti concentrassi su di un incantesimo), per un massimo di 1 ora, o finché non scende a 0 Punti Ferita. Poi ritorna al suo piano natio.

Finché resta evocato, lo djinni è amichevole verso di te e i tuoi compagni. Obbedisce a qualsiasi comando gli dai, non importa la lingua usata. Se non gli impartisci ordini, lo djinni si difenderà dagli attacchi ma non effettuerà nessun'altra Azione.

Dopo la partenza dello djinni, esso non potrà più essere evocato prima che siano passate 24 ore e se lo djinni muore l'anello perde la sua magia.

\oggettomagico{Anello dell'Inganno}

\textbf{Rarità:} Raro

chi indossa questo \textbf{anello} maledetto è convinto che abbia un potere scelto dal Narratore o determinato casualmente.

\oggettomagico{Anello della Debolezza}

\textbf{Rarità:} Raro

una volta indossato, questo \textbf{anello} può essere rimosso solo da \emph{\hyperlink{Rimuovi Maledizione}{Rimuovi Maledizione}}. Il portatore perde un punto di Forza a round finché non si riduce a -3.

\oggettomagico{Anello della Vista ai Raggi X}

\textbf{Rarità:} Comune; \textbf{Costo:} 6000 mo

mentre indossi questo \textbf{anello}, puoi usare due azioni per pronunciarne la parola di comando. Quando lo fai, puoi vedere attraverso la materia solida per 1 minuto. Questa vista ha un raggio di 9 metri. Per te, gli oggetti solidi all'interno del raggio appaiono trasparenti e non impediscono alla luce di attraversarli.

Questa vista può penetrare 30 centimetri di pietra, 2,5 centimetri di metallo comune o fino a 90 centimetri di legno o terra. Le sostanze più dense bloccano la vista, così come un sottile foglio di piombo. Ogni qualvolta usi di nuovo l'anello prima di aver terminato una notte di riposo devi superare un Tiro Salvezza su Tempra con DC 18 o diventare affaticato.

\oggettomagico{Anello delle Stelle Cadenti}

\textbf{Rarità:} Molto Raro; \textbf{Costo:} 14000 mo

mentre indossi questo \textbf{anello} a luce fioca o all'oscurità, puoi lanciare tramite esso \emph{luci danzanti} e \emph{luce} a volontà. Lanciare uno dei due incantesimi tramite l'anello richiede due azioni. L'anello ha 6 cariche per le seguenti altre proprietà.

L'anello recupera 1 carica spesa ogni giorno all'alba.

\emph{Luminescenza}. Spendi 1 carica con due azioni per lanciare tramite l'anello l'incantesimo \emph{luminescenza}.

\emph{Palla di fulmini}. Puoi spendere 2 cariche con due azioni per creare da una a quattro sfere di fulmini di 1 metro di diametro. più sfere crei, meno potente sarà ciascuna sfera individualmente.
Ogni sfera compare in uno spazio non occupato visibile entro 36 metri da te. La sfera dura finché ti concentri su di essa (come se ti concentrassi su di un incantesimo), fino a un massimo di 1 minuto. Ogni sfera irradia luce fioca in un raggio di 9 metri. Con due azioni puoi muovere ciascuna sfera di massimo 9 metri, ma senza superare i 36 metri di distanza da te. Quando una creatura, a parte te, si trova entro 1 metro da una sfera, la sfera scarica i fulmini contro quella creatura e poi scompare. Quella creatura deve effettuare un Tiro Salvezza su Riflessi con DC 18. Se fallisce il Tiro Salvezza, la creatura subisce danni da elettricità in base al numero di sfere da te creato (4 sfere, 2d4 danni; 3 sfere, 2d6 danni; 2 sfere, 5d4 danni; 1 sfera, 4d12 danni).

\emph{Stelle Cadenti}. Puoi spendere da 1 a 3 cariche con due azioni. Per ogni carica spesa, scagli una scintilla di luce dall'anello in un punto visibile entro 18 metri da te. Ogni creatura, in cubo di 3 metri di lato originante da quel punto, viene ricoperta di scintille e deve effettuare un Tiro Salvezza di Destrezza DC 15, subendo 5d4 danni da fuoco se lo fallisce, o la metà di questi danni se lo supera.

\oggettomagico{Anello dello Scudo Mentale}

\textbf{Rarità:} Non Comune; \textbf{Costo:} 16000 mo

mentre indossi questo \textbf{anello}, sei immune alla magia che permette alle altre creature di leggere i tuoi pensieri, determinare se stai mentendo, conoscere i tuoi Tratti o apprendere che tipo di creatura sei. Le creature possono comunicare telepaticamente con te solo se glielo concedi.

\oggettomagico{Anello di Caduta Piuma}

\textbf{Rarità:} Raro; \textbf{Costo:} 2000 mo

se cadi da più di 1 metro e indossi questo \textbf{anello} si attiva in automatico l'incantesimo Caduta Piuma

\oggettomagico{Anello di Camminare sull'Acqua}

\textbf{Rarità:} Non Comune; \textbf{Costo:} 1500 mo

mentre indossi questo \textbf{anello}, puoi stare in piedi o muoverti su qualsiasi superficie liquida come se fosse terreno solido.

\oggettomagico{Anello di Elusione}

\textbf{Rarità:} Raro; \textbf{Costo:} 5000 mo

mentre indossi questo \textbf{anello} e fallisci un Tiro Salvezza su Riflessi, puoi usare la tua Azione di Reazione per spendere 1 carica per ripetere il Tiro Salvezza che hai appena fallito. Questo anello ha 3 cariche, e recupera 1 carica spesa ogni mattina all'alba.

\oggettomagico{Anello di Influenza sugli Animali}

\textbf{Rarità:} Raro; \textbf{Costo:} 4000 mo

\emph{Anello} 4000 mo

Mentre indossi questo \textbf{anello}, puoi usare due azioni per spendere 1 delle sue cariche per lanciare tramite esso uno dei seguenti incantesimi: \emph{amicizia con gli animali} (DC del Tiro Salvezza 15), \emph{parlare con gli animali}, \emph{paura} (DC del Tiro Salvezza 15, prende come bersaglio solo bestie che hanno Intelligenza -2 o meno).

Questo anello ha 3 cariche, e recupera 1 carica spesa ogni giorno all'alba.

\oggettomagico{Anello di Invisibilita'}

\textbf{Rarità:} Molto Raro; \textbf{Costo:} 10000 mo

mentre indossi quest'\textbf{anello}, puoi renderti invisibile con due azioni. Tutto ciò che indossi o trasporti diventa invisibile assieme a te. Resti invisibile finché l'anello non viene rimosso, attacchi o lanci un incantesimo, o finché non usi due azioni per tornare visibile. L'anello è usabile 3 volte al giorno.

\oggettomagico{Anello di Liberta' di Azione}

\textbf{Rarità:} Raro; \textbf{Costo:} 20000 mo

mentre indossi questo \textbf{anello}, il terreno difficile non ti costa movimento aggiuntivo. Inoltre, la magia non può né ridurre la tua velocità né renderti paralizzato o intralciato.

\oggettomagico{Anello di Protezione}

\textbf{Rarità:} Molto Raro; \textbf{Costo:} 5000 mo

costo vario, rarità varia, mentre indossi questo \textbf{anello}, hai un bonus da +1 (5000 mo, raro), +2 (7500 mo, raro), +3 (12000mo, molto raro) alla Difesa e ai Tiri Salvezza.

\oggettomagico{Anello di Resistenza}

\textbf{Rarità:} Raro; \textbf{Costo:} 12000 mo

mentre indossi questo \textbf{anello}, hai resistenza a un tipo di danno. La gemma incastonata nell'anello indica il tipo di danno, che viene scelto o determinato casualmente dal Narratore.

\medskip

\noindent\begin{tabularx}{\linewidth}{lll}
	\toprule
\rowcolor{gray!20}\textbf{d10} & \textbf{Tipo di Danno} & \textbf{Gemma}\\
\toprule
1 &Acido &Perla\\
\rowcolor{gray!20}2& Forza &Zaffiro\\
3& Freddo &Tormalina\\
\rowcolor{gray!20}4& Fulmine &Citrino\\
5& Fuoco &Granato\\
\rowcolor{gray!20}6& Vuoto& Giaietto\\
7& Energia Positiva &Giada\\
\rowcolor{gray!20}8& Luce &Topazio\\
9& Suono &Spinello\\
\rowcolor{gray!20}10& Energia Negativa &Ardesia
\end{tabularx}

\medskip

\oggettomagico{Anello di Rigenerazione}

\textbf{Rarità:} Molto Raro; \textbf{Costo:} 12000 mo

mentre indossi questo \textbf{anello}, recuperi 1d6 Punti Ferita ogni 10 minuti, purché ti rimanga almeno 1 Punto Ferita. Se perdi una parte del corpo, l'anello fa sì che la parte mancante ricresca e ritorni alla sua completa funzionalità in 1d6 + 1 giorni, purché per tutto il periodo ti rimanga sempre almeno 1 Punto Ferita.

\oggettomagico{Anello di Telecinesi}

\textbf{Rarità:} Molto Raro; \textbf{Costo:} 80000 mo

mentre indossi questo \textbf{anello}, puoi lanciare a volontà l'incantesimo \emph{telecinesi}, ma puoi prendere come bersaglio solo oggetti che non siano indossati o trasportati.

\oggettomagico{Anello Respingi Incantesimi}

\textbf{Rarità:} Leggendario; \textbf{Costo:} 35000 mo

mentre indossi quest'\textbf{anello}, hai +1d6 ai Tiri Salvezza contro qualsiasi incantesimo che prende come bersaglio solo te e non un'area di effetto. Inoltre, se fai un Successo Critico Salvezza e l'incantesimo ha livello 5 o più basso, l'incantesimo non ha effetto su di te e invece prende come bersaglio l'incantatore che ha lanciato l'incantesimo.

\oggettomagico{Anfora elementale dell'acqua}

\textbf{Rarità:} Raro; \textbf{Costo:} 2500 mo

quest'\textbf{anfora} può essere usata per evocare e controllare un elementale dell'acqua in modo analogo all'incantesimo evoca elementale. È necessario preparare l'oggetto magico e condurre un rituale per un round prima dell'evocazione vera e propria, che richiede un round. Dopo che l'elementale è stato evocato, occorre mantenere la concentrazione per potergli impartire gli ordini. L'anfora è usabile una volta al giorno.

\oggettomagico{Antiemorragica}

\textbf{Aura:} di Cura moderata; \textbf{Costo:} 3000 mo

\textbf{Requisiti:} Creare Oggetti Magici 2, Cura Ferite 5, Cura Ferite

Un'\textbf{armatura} Antiemorragica aiuta a fermare la perdita di sangue dalle ferite di chi la indossa. Un'armatura Antiemorragica riduce i danni da Sanguinamento di 1 ogni fine round.

\oggettomagico{Apparato del Granchio}

\textbf{Rarità:} Leggendario; \textbf{Costo:} 15000 mo

quest'oggetto appare come un \textbf{barile} di ferro sigillato di taglia Grande e del peso di 250 chili. Il barile nasconde un fermo, che può essere trovato superando una prova di Consapevolezza con DC 25. Rimuovere il fermo apre uno scomparto a una delle estremità dell'apparato, che permette a due creature di taglia Media o inferiore di entrarvi dentro. All'estremità opposta sono disposte dieci leve, ciascuna in posizione neutrale, in grado di muoversi verso l'alto o il basso. Quando vengono impiegate determinate leve, l'apparato si trasforma e assomiglia a un'aragosta gigante.

L'apparato è un oggetto Grande con le seguenti statistiche.

Difesa: 20, Punti Ferita: 200, Velocità: 9 m, nuoto 9 m (o 0 m entrambi se le gambe e la coda non vengono estese)

Immunità ai Danni: veleno

Per essere usato come veicolo, l'apparato necessita un pilota. Quando lo sportello dell'apparato viene chiuso, il compartimento è a tenuta stagna, e non fa filtrare aria o acqua. I compartimenti conservano aria sufficiente per 10 ore, divise per il numero di creature all'interno. L'apparato galleggia in acqua e può anche spingersi sott'acqua fino a una profondità di 270 metri. Al di sotto di questa soglia, l'apparato subisce 2d6 danni contundenti al minuto a causa della pressione. Una creatura all'interno del compartimento può usare due azioni per muovere verso l'alto o il basso fino a due leve. Dopo ciascun uso, la leva torna alla sua posizione neutrale. Ogni leva, da sinistra a destra, funziona come mostrato sulla tabella seguente.

1: Estende gambe e coda, permettendo all'apparato di camminare e nuotare. Ritrae gambe e coda, riducendo la velocità dell'apparato a 0 e rendendolo incapace di beneficiare di bonus alla velocità.

2: Apre l'oblò frontale. Chiude l'oblò frontale.

3: Apre gli oblò laterali (due per lato). Chiude gli oblò laterali (due per lato).

4: Estende due chele dal lato frontale dell'apparato. Ritrae le chele.

5: Effettua un attacco con arma da mischia con ciascuna chela estesa: +8 al Tiro per Colpire, portata 1 m, un bersaglio. Colpisce: 7 (2d6) danni contundenti. Effettua un attacco con arma da mischia con ciascuna chela estesa: +8 al Tiro per Colpire, portata 1 m, un bersaglio. Colpisce: Il bersaglio è afferrato (DC 18 per fuggire).

6: L'apparato cammina o nuota in avanti. L'apparato cammina o nuota indietro.

7: L'apparato svolta di 90 gradi a sinistra. L'apparato svolta di 90 gradi a destra.

8: Delle fessure frontali emettono luce intensa in un raggio di 9 metri e luce fioca per 18 metri. Spegne le luci.

9: L'apparato affonda di 6 metri nei liquidi. L'apparato risale di 6 metri dai liquidi.

10: Sblocca e apre il portellone posteriore. Chiude e sigilla il portellone posteriore.

\oggettomagico{Ariete}

\textbf{Aura:} Invocazione debole; \textbf{Costo:} 3000 mo

\textbf{Requisiti:} Creare Oggetti Magici 2,

Questi \textbf{scudi} sono molto solidi e spesso recano l'emblema di un ariete o un toro. Quando chi indossa uno scudo ariete effettua un attacco con lo scudo come parte di una Carica, il bonus alla Difesa dello scudo si applica ai Tiri per Colpire e per i danni. Questo non si cumula con nessun altro potenziamento che possegga lo scudo. Questa capacità non è applicabile agli scudi di tipo leggero.

\oggettomagico{Arma magica}

\textbf{Costo:} 1800 mo

\emph{Arma (qualsiasi)} +1 1800 mo, +2 6000 mo, +3 17000 mo, +4 45000 mo, +5 80000 mo

Hai un bonus ai Tiri per Colpire e ai tiri di danno effettuati con quest'\textbf{arma}. Il bonus è determinato dalla rarità dell'arma. Alcune armi magiche possiedono delle ulteriori proprietà, come l'emettere luce.

\oggettomagico{Armatura / Scudo Magico}

\emph{Armatura (qualsiasi)} +1 2500 mo, +2 10000 mo, +3 18000 mo, +4 35000 mo, +5 80000 mo

\emph{Scudi (piccoli, medi, pesanti)}: +1 1500 mo, +2 4000 mo, +3 9000 mo, +4 20000 mo, +5 35000 mo

Mentre impugni questo \textbf{scudo}/\textbf{armatura}, hai un bonus alla Difesa determinato dal bonus magico dello scudo/armatura. Questo bonus è in aggiunta al normale bonus alla Difesa fornito dallo scudo/armatura.

\oggettomagico{Armatura Demoniaca}

\textbf{Aura:} Evocazione forte; \textbf{Costo:} 5000 mo

\textbf{Requisiti:} Creare Oggetti Magici 2 ; \textbf{Rarità:} Rara

Mentre indossi l'\textbf{armatura} puoi comprendere e parlare l'Abissale. Inoltre, le manopole artigliate dell'armatura trasformano i colpi disarmati effettuati con le tue mani in armi magiche che infliggono danni taglienti, con un bonus di +1 ai Tiri per Colpire e ai tiri di danno e il d6 come dado di danno.

\oggettomagico{Bacchetta Cerca metalli}

\textbf{Rarità:} Non Comune; \textbf{Costo:} 500 mo

quando viene spesa una carica, la \textbf{bacchetta} punta nella direzione di qualunque massa metallica di almeno 100 kg che si trovi entro 6 metri. Chi impugna la bacchetta ha una percezione intuitiva del tipo di metallo individuato.

\oggettomagico{Bacchetta dei Dardi Arcani}

\textbf{Rarità:} Raro; \textbf{Costo:} 8000 mo

mentre impugni questa \textbf{bacchetta}, puoi usare due azioni per spendere 1 o più delle sue cariche per lanciare tramite essa un Dardo arcano, come l'omonimo incantesimo. Ogni carica genera 1 dardo. La bacchetta ha 7 cariche. La bacchetta recupera 1d3+1 cariche spese all'alba.

\oggettomagico{Bacchetta dei Fulmini}

\textbf{Rarità:} Raro; \textbf{Costo:} 32000 mo

mentre impugni questa \textbf{bacchetta}, puoi usare due azioni per spendere 1 carica lanciare tramite essa l'incantesimo fulmine (DC del Tiro Salvezza 18).
Questa bacchetta ha 7 cariche. La bacchetta recupera 1d3 + 1 cariche spese all'alba.

\oggettomagico{Bacchetta dei Segreti}

\textbf{Rarità:} Non Comune; \textbf{Costo:} 500 mo

mentre impugni questa \textbf{bacchetta}, puoi usare due azioni per spendere 1 carica e rilevare se porta segreta o trappola si trova entro 9 metri da te, la bacchetta pulsa e punta a quella più vicina a te. La bacchetta ha 3 cariche. La bacchetta recupera tutte le cariche spese all'alba.

\oggettomagico{Bacchetta del Fuoco}

\textbf{Rarità:} Molto Raro; \textbf{Costo:} 18000 mo

una \textbf{bacchetta} del fuoco produce diversi incantesimi e consuma 1 carica + livello dell'incantesimo manifestato. Gli incantesimi manifestabili sono: Onda rovente, piroesperto, palla di fuoco, muro di fuoco. Finché la bacchetta è tenuta in mano ogni 1 sui dadi per danno da fuoco che infligge viene considerato come 2. La bacchetta ha 7 cariche e ne recupera 1 all'alba.

\oggettomagico{Bacchetta del Ghiaccio}

\textbf{Rarità:} Molto Raro; \textbf{Costo:} 15000 mo

una \textbf{bacchetta} del fuoco produce diversi incantesimi e consuma 1 carica + livello dell'incantesimo manifestato. Gli incantesimi manifestabili sono: \hyperlink{Raggio di Gelo}{Raggio di Gelo}, \hyperlink{Tempesta di Nevischio}{Tempesta di Nevischio}, tempesta di ghiaccio, cono di freddo. Finché la bacchetta è tenuta in mano ogni 1 sui dadi per danno da freddo che infligge viene considerato come 2. La bacchetta ha 7 cariche e ne recupera 1 all'alba.

\oggettomagico{Bacchetta del Mago da Guerra}

\textbf{Rarità:} Molto Raro; \textbf{Costo:} 25000 mo

1500 / 5500 / 25000 mo, non comune (+1), raro (+2), o molto raro (+3), mentre impugni questa \textbf{bacchetta}, ottieni un bonus ai Tiri per Colpire con gli incantesimi determinato dalla rarità della bacchetta. Inoltre, ignori la copertura leggera quando effettui un attacco con incantesimo.

\oggettomagico{Bacchetta del Vincolo}

\textbf{Rarità:} Raro; \textbf{Costo:} 10000 mo

questa \textbf{bacchetta} ha 7 cariche per le seguenti proprietà. La bacchetta recupera 1 carica spese all'alba. Mentre impugni questa bacchetta, puoi usare due azioni e spendere alcune delle sue cariche per lanciare uno dei seguenti incantesimi (DC del Tiro Salvezza 21):

\emph{blocca mostri} (5 cariche) o \emph{\hyperlink{Blocca Persona}{Blocca Persona}} (2 cariche).

\oggettomagico{Bacchetta dell'Individuazione del Magico}

\textbf{Rarità:} Non Comune; \textbf{Costo:} 1500 mo

mentre impugni questa \textbf{bacchetta}, con due azioni puoi spendere 1 carica per lanciare tramite essa l'incantesimo individuazione del magico. Questa bacchetta ha 7 cariche, e recupera 1d3 cariche spese ogni mattina all'alba.

\oggettomagico{Bacchetta dell'Individuazione delle porte segrete}

\textbf{Rarità:} Non Comune; \textbf{Costo:} 300 mo

questa \textbf{bacchetta} punta al passaggio segreto più vicino entro 6 metri. L'effetto consuma una carica delle 7 disponibili, ogni giorno all'alba si recuperano tutte le cariche.

\oggettomagico{Bacchetta della Fuga Assistita}

\textbf{Rarità:} Raro; \textbf{Costo:} 2000 mo

mentre impugni questa \textbf{bacchetta}, puoi usare la Azione di Reazione e spendere 1 carica per ottenere +1d6 ai Tiri Salvezza che effettui per evitare di restare paralizzato o intralciato, o puoi spendere 1 carica per ottenere +1d6 su qualsiasi prova effettuata per sfuggire un tentativo di afferrare.

\oggettomagico{Bacchetta della Luce}

\textbf{Rarità:} Raro; \textbf{Costo:} 3500 mo

una \textbf{bacchetta} della luce manifesta diversi incantesimi e consuma 1 carica + livello dell'incantesimo manifestato. Gli incantesimi manifestabili sono: luci danzanti, luce, fiamma perenne, luce diurna. Infine, spendendo 5 cariche, il possessore può creare un raggio di intensa luce solare. La luce di colore giallo-dorato intenso, ha una gittata di 36 m, e forma una sfera di luce del diametro di 12 m. Chiunque si trovi nell'area di effetto viene accecato e stordito per 1 round se fallisce un Tiro Salvezza su Tempra DC 17. La sfera dorata ha un effetto devastante sui non morti, infliggendo 6d6 ferite da Luce senza possibilità di Tiro Salvezza. Questa bacchetta ha 7 cariche. La bacchetta recupera 2 cariche spese all'alba.

\oggettomagico{Bacchetta della Metamorfosi}

\textbf{Rarità:} Molto Raro; \textbf{Costo:} 32000 mo

mentre impugni questa \textbf{bacchetta}, puoi usare due azioni per spendere 1 carica per lanciare tramite essa l'incantesimo metamorfosi (DC 18, Tiro Salvezza su Volontà). Questa bacchetta ha 3 cariche. La bacchetta recupera 1 carica spesa all'alba.

\oggettomagico{Bacchetta della Negazione}

\textbf{Rarità:} Molto Raro; \textbf{Costo:} 35000 mo

questa \textbf{bacchetta} nega gli incantesimi o effetti simili prodotti da oggetti magici. Il possessore punta la bacchetta verso l'oggetto entro 36 metri, ed essa emana un raggio di colore grigio chiaro che colpisce il bersaglio. Il raggio annulla automaticamente le manifestazione di incantesimi o effetti simili di livello 3 o meno. Ciascun uso della bacchetta costa 1 carica ed essa può essere usata solo una volta per round. Questa bacchetta ha 3 cariche. La bacchetta recupera 1 carica ogni giorno all'alba.

\oggettomagico{Bacchetta della Paralisi}

\textbf{Rarità:} Raro; \textbf{Costo:} 16000 mo

mentre impugni questa \textbf{bacchetta}, puoi usare due azioni per spendere 1 carica per far sì che un sottile raggio parta dalla sua punta verso una creatura visibile entro 18 metri da te. Il bersaglio deve superare un tiro Salvezza su Tempra con DC 17 o restare paralizzato per 1 minuto. Al termine di ciascun round del bersaglio, questi può effettuare un tiro Salvezza su Tempra DC 15, terminando l'effetto su di sé in caso lo superi. Questa bacchetta ha 7 cariche. La bacchetta recupera 1 carica spese all'alba.

\oggettomagico{Bacchetta della Paura}

\textbf{Rarità:} Raro; \textbf{Costo:} 13000 mo

questa \textbf{bacchetta} ha 7 cariche per le seguenti proprietà. La bacchetta recupera 1 carica spese all'alba.

\textbf{Comando} mentre impugni questa bacchetta, puoi usare due azioni per spendere 1 carica e comandare a un'altra creatura di scappare o strisciare, come per l'incantesimo comando (DC del Tiro Salvezza 18)

\textbf{Cono di Paura} mentre impugni questa bacchetta, puoi usare due azioni per spendere 2 cariche, facendo sì che la punta della bacchetta emetta luce in un cono di 18 metri. Ogni creatura nel cono deve superare un Tiro Salvezza su Volontà con DC 18 o restare spaventata da te per 1 minuto. Mentre è spaventata in questo modo, una creatura deve spendere i suoi round cercando di muoversi più lontano possibile da te, e non può muoversi volontariamente entro 9 metri da te.

Inoltre non può effettuare reazioni. Come sua azione, la creatura può usare solo l'Azione Scattare o cercare di liberarsi da un effetto che le impedisca di muoversi. Se non può muoversi da nessuna parte, la creatura può usare l'Azione Difesa Totale. Al termine di ciascun suo round, la creatura può ripetere il Tiro Salvezza, terminando l'effetto su di sé in caso lo superi. Questa bacchetta ha 7 cariche. La bacchetta recupera 1 carica spese all'alba.
cast
\oggettomagico{Bacchetta della Ragnatela}

\textbf{Rarità:} Non Comune; \textbf{Costo:} 8000 mo

mentre la impugni la \textbf{bacchetta}, puoi usare due azioni per spendere 1 carica per lanciare tramite essa l'incantesimo ragnatela (DC del Tiro Salvezza 18). Questa bacchetta ha 7 cariche. La bacchetta recupera 1 carica spesa all'alba.

\oggettomagico{Bacchetta delle Comodita'}

\textbf{Rarità:} Comune; \textbf{Costo:} 300 mo

Il possessore della \textbf{bacchetta} può spendere 1 carica per lanciare gli incantesimi servitore invisibile o cuoco invisibile o disco fluttuante. La bacchetta ha 7 cariche che sono recuperate all'alba.

\oggettomagico{Bacchetta delle Illusioni}

\textbf{Rarità:} Raro; \textbf{Costo:} 3000 mo

chi impugna questa \textbf{bacchetta} può lanciare \hyperlink{Immagine Maggiore}{Immagine Maggiore} (3), Immagine Silenziosa (1), Immagine Speculare (2). Ogni incantesimo costo un numero di cariche pari al livello +1. Mentre si concentra sull'effetto, il personaggio può solo muoversi a velocità dimezzata. Se viene colpito deve riuscire con successo in una Prova di Magia o l'illusione svanisce immediatamente.

\oggettomagico{Bacchetta delle Meraviglie}

\textbf{Rarità:} Molto Raro; \textbf{Costo:} 25000 mo

mentre impugni questa \textbf{bacchetta}, puoi spendere 1 carica con due azioni e scegliere un bersaglio entro 36 metri da te. Il bersaglio può essere una creatura, un oggetto o un punto nello spazio. Il Narratore decide o determina casualmente cosa accadrà quando fai uso della bacchetta. Gli incantesimi lanciati tramite la bacchetta hanno DC del Tiro Salvezza 18. Se l'incantesimo normalmente ha una gittata espressa in metri, la gittata diventa 36 metri qualora non lo sia già. Se un effetto copre un'area, devi centrare l'incantesimo sul bersaglio e includervelo. Se un effetto più agire su più soggetti possibili, il Narratore determina casualmente chi sia affetto.

Questa bacchetta ha 7 cariche. La bacchetta recupera 1 carica ogni giorno all'alba.

Ogni volta che fai uso della bacchetta delle meraviglie tira un d100 e consulta questa tabella.

%\end{multicols}

%\vfill
%
%\begin{center}
%\includegraphics[width=0.6\linewidth]{immagini/bacchette.png}
%\end{center}

\medskip

\noindent\begin{tabularx}{\linewidth}{lX}
	\toprule
\rowcolor{gray!20}\textbf{d100}& \textbf{Contenuti}\\
\toprule
01-05 &Lanci l'incantesimo Lentezza.\\
\rowcolor{gray!20}06-10 &Lanci l'incantesimo Luminescenza.\\
11-15 &Sei stordito fino all'inizio del tuo prossimo round, e ritieni che sia accaduto qualcosa di stupefacente.\\
\rowcolor{gray!20}16-20 &Lanci l'incantesimo Folata di vento.\\
21-25 &Lanci l'incantesimo Individuazione dei pensieri sul bersaglio da te scelto. Se il tuo a bersaglio non è una creatura, subisci invece 1d6 danni.\\
\rowcolor{gray!20}26-30 &Lanci l'incantesimo Nebbia Nauseante.\\
31-33 &Pioggia abbondante precipita in un raggio di 18 metri centrato sul bersaglio. L'area diventa oscurata leggermente. La pioggia continua a cadere fino all'inizio del tuo prossimo round.\\
\rowcolor{gray!20}34-36 &Compare un animale nello spazio non occupato più vicino al bersaglio. L'animale non è sotto il tuo controllo e agisce come di norma. \\
&Tira un d100 per determinare che specie di animale compaia 01-25, un rinoceronte; 26-50, un elefante; 51-100, un ratto.\\
\rowcolor{gray!20}37-46 &Lanci Fulmine.\\
47-49 &Una nube di 600 enormi farfalle riempie un raggio di 9 metri intorno al bersaglio. L'area diventa oscurata pesantemente e fornisce copertura totale. Le farfalle restano per 10 minuti.\\
\rowcolor{gray!20}50-53 &Ingrandisci il bersaglio come se avessi lanciato l'incantesimo ingrandire/ridurre. Se il bersaglio non può essere soggetto all'incantesimo, o se non è una creatura, tu diventi il bersaglio.\\
54-58 &Lanci l'incantesimo oscurità.\\
\rowcolor{gray!20}59-62 &Erba folta spunta in un raggio di 18 metri intorno al bersaglio.Se vi è già dell'erba, questa cresce di dieci volte e resta così per 1 minuto.\\
63-65 &Un oggetto a scelta del Narratore scompare sul Piano Etereo. L'oggetto non deve essere né indossato né trasportato, deve essere entro 36 metri dal bersaglio, e non più grande di 3 metri in ciascuna dimensione.\\
\rowcolor{gray!20}66-69 &Ti rimpicciolisci come se avessi lanciato su di te l'incantesimo ingrandire/ridurre.\\
70-79 &Lanci l'incantesimo Palla di fuoco.\\
\rowcolor{gray!20}80-84 &Lanci l'incantesimo Invisibilità su di te.\\
85-87 &Sul bersaglio crescono delle foglie. Se hai scelto un punto nello spazio come bersaglio, le foglie spunteranno sulla creatura più vicina a quel punto. A meno che non vengano strappate, le foglie diventeranno marroni e cadranno dopo 24 ore.\\
\rowcolor{gray!20}88-90& Un flusso di 1d4 x 10 gemme del valore di 1 mo ciascuna scaturisce dalla punta della bacchetta in una linea lunga 9 metri e larga 1 metro.\\
\end{tabularx}
\noindent\begin{tabularx}{\linewidth}{lX}
\rowcolor{gray!20}& Ogni gemma infligge 1 danno contundente, e il loro danno totale è diviso equamente tra tutte le creature sulla linea.\\
91-95 &Una raffica di luci scintillanti e colorate si estende da te in un raggio di 9 metri. Tu e tutte le creature nell'area dovete superare un Tiro Salvezza su Tempra con DC 15 o restare accecati per 1 minuto. Una creatura può ripetere il Tiro Salvezza al termine di ciascun suo round, terminando l'effetto su di sé in caso lo superi.\\
\rowcolor{gray!20}96-97 &La pelle del bersaglio assume un colorito blu intenso per 1d10 giorni. Se hai scelto un punto nello spazio, il soggetto sarà la creatura più vicina a quel punto.\\
\end{tabularx}
\noindent\begin{tabularx}{\linewidth}{lX}
\rowcolor{gray!20}98-00 &Se il bersaglio è una creatura, deve effettuare un Tiro Salvezza di Tempra con DC 18. Se il bersaglio non è una creatura, il bersaglio diventi tu e sarai tu a effettuare il Tiro Salvezza. Se il Tiro Salvezza fallisce di 5 o più, il bersaglio è pietrificato. Se il Tiro Salvezza fallisce di meno, il bersaglio è intralciato e inizia a trasformarsi in pietra. Mentre è intralciato a questo modo, il bersaglio deve ripetere il Tiro Salvezza al termine di ciascun suo round, diventando pietrificato in caso di fallimento o terminando l'effetto in caso di successo. Il bersaglio resta pietrificato finché non sarà liberato dall'incantesimo pietra in carne o simili magie.\\
\end{tabularx}

%\medskip

%\begin{center}
%\includegraphics[width=0.7\linewidth]{immagini/bacchette.png}
%\end{center}

%\medskip

%\begin{multicols}{2}

\oggettomagico{Bacchetta delle Palle di Fuoco}

\textbf{Rarità:} Raro; \textbf{Costo:} 32000 mo

mentre impugni questa \textbf{bacchetta}, puoi usare due azioni per spendere 1 carica per lanciare tramite essa l'incantesimo palla di fuoco (DC del Tiro Salvezza 18). Questa bacchetta ha 7 cariche. La bacchetta recupera 1 carica spesa all'alba.

\oggettomagico{Bacchetta di Individ. dei Nemici}

\textbf{Rarità:} Raro; \textbf{Costo:} 4000 mo

mentre impugni questa \textbf{bacchetta}, puoi usare due azioni e spendere 1 carica per pronunciarne la parola di comando. Per il minuto successivo, conosci in che direzione si trovi la creatura ostile più vicina entro 18 metri da te, ma non la distanza che vi separa. La bacchetta può percepire la presenza di creature ostili che siano eteree, invisibili, camuffate, o nascoste, oltre che di quelle in piena vista. L'effetto termina se smetti di impugnare la bacchetta. Questa bacchetta ha 7 cariche. La bacchetta recupera 2 cariche spese all'alba.

\oggettomagico{Bacchetta Scopri trappole}

\textbf{Rarità:} Non Comune; \textbf{Costo:} 400 mo

questa \textbf{bacchetta} punta alla trappola più vicina entro 6 m. L'effetto consuma una carica. Questa bacchetta ha 7 cariche. La bacchetta recupera tutte le cariche spese all'alba.

\oggettomagico{Bacinella dell'annegamento}

questa \textbf{bacinella} maledetta ha l'apparenza di un'anfora elementale dell'acqua. Tuttavia, invece di evocare un elementale, sprigiona un globo d'acqua che avvolge la testa del personaggio. Questi affoga in 2d4 round a meno che non riesca in un Tiro Salvezza Riflessi DC 19. L'acqua è "appiccicosa" e può essere rimossa solo con la magia (\emph{Dissolvi magia} o \emph{distruggere acqua}).

\oggettomagico{Balestra dei Dardi Arcani}

questa \textbf{balestra} piccola ad una mano ha la capacità di manifestare un dardo magico.
Spendendo 1 Azione è possibile sparare un dardo magico come se fosse un singolo Dardo arcano.

\oggettomagico{Bandana dell'Intelligenza}

\textbf{Rarità:} Raro; \textbf{Costo:} 8000 mo

mentre indossi questa \textbf{bandana} la tua Intelligenza è +4. La bandana non ha effetto se hai già Intelligenza è già +4 o più alta.

\oggettomagico{Barca Pieghevole}

\textbf{Rarità:} Raro; \textbf{Costo:} 12000 mo

questo \textbf{oggetto} sembra una scatola di legno che misura 30 centimetri di lunghezza, 15 centimetri di larghezza e 15 centimetri di profondità. Pesa 2 chili, ingombro 2, e galleggia. Può essere aperta per porvi oggetti all'interno. Quest'oggetto possiede tre parole di comando, ciascuna delle quali richiede due azioni per essere pronunciata. Una parola di comando fa sì che la scatola si dispieghi in una barca lunga 3 metri, larga 1,5 metri e profonda 50 centimetri. La barca ha un paio di remi, un'ancora, un albero e una vela. La barca può contenere fino a quattro creature di taglia Media.

La seconda parola di comando fa sì che la scatola si dispieghi in una nave lunga 7,2 metri, larga 2,5 metri e profonda 2 metri. La nave ha un ponte, file di voga, cinque serie di remi, un timone direzionale, un'ancora, una cabina e un albero con la vela quadrata. La nave può contenere quindici creature di taglia Media.

La terza parola di comando fa sì che la barca pieghevole ritorni a piegarsi nella scatola, purché nessuna creatura sia a bordo. Qualsiasi oggetto a bordo che non possa entrare nella scatola resta fuori della scatola mentre questa si piega. Qualsiasi oggetto a bordo che possa entrare nella scatola, vi entra.

\oggettomagico{Bastone degli Insetti Sciamanti}

\textbf{Rarità:} Raro; \textbf{Costo:} 160000 mo

questo \textbf{bastone} ha 10 cariche che puoi impiegare per usare le proprietà sotto descritte e recupera 1 carica ogni giorno all'alba.

- \emph{Incantesimi}. Mentre impugni questo bastone, puoi usare due azioni per spendere le sue cariche ed lanciare uno dei seguenti incantesimi: insetto gigante (4 cariche, DC 19) o piaga degli insetti (5 cariche, DC 21).

- \emph{Nube di Insetti}. Mentre impugni questo bastone, puoi usare due azioni e spendere 1 carica per fa sì che uno sciame di insetti innocui si diffonda in un raggio di 9 metri intorno a te. Gli insetti rimangono per 10 minuti, rendendo l'area oscurata pesantemente per tutti tranne te. Lo sciame si muove assieme a te, rimanendo centrato su di te. Un vento di almeno 15 chilometri all'ora disperde lo sciame e termina l'effetto.

\oggettomagico{Bastone dei Boschi}

\textbf{Rarità:} Raro; \textbf{Costo:} 44000 mo

il \textbf{bastone} può essere impugnato come un bastone da combattimento magico che conferisce un bonus di +2 ai Tiri per Colpire e danno effettuati con esso. Quando lo impugni hai anche un bonus di +2 ai Tiri per Colpire con incantesimi.
Questo bastone ha 10 cariche per le seguenti proprietà. Recupera 1 cariche spese ogni giorno all'alba.

- \emph{Incantesimi}. Puoi usare due azioni per spendere 1 o più cariche del bastone per lanciare tramite esso uno dei seguenti incantesimi, utilizzando la tua DC del Tiro Salvezza degli incantesimi: amicizia con gli animali (1 carica), localizza animali e piante (1 carica), muro di spine (6 cariche), parlare con gli animali (3 cariche), Pelle di Corteccia (2 cariche) o risveglio (5 cariche). Puoi inoltre usare due azioni per lanciare tramite il bastone l'incantesimo \hyperlink{Passare Senza Tracce}{Passare Senza Tracce} senza spendere cariche.

- \emph{Forma d'Albero}. Puoi usare due azioni per piantare un'estremità del bastone su terreno fertile e spendere 1 carica per trasformare il bastone in un albero da frutto vigoroso. L'albero è alto 18 metri, con un tronco di 1 metro di diametro; in cima i suoi rami si estendono per 6 metri. L'albero sembra un albero normale ma irradia una debole aura di magia di trasmutazione, qualora sia bersaglio dell'incantesimo individuazione del magico. Mentre sei in contatto con l'albero e usi un'altra Azione per pronunciarne la parola di comando, riporti il bastone alla sua forma normale. Qualsiasi creatura sull'albero, cade quando questo si ritrasforma in bastone.

\oggettomagico{Bastone dei Tuoni e Fulmini}

\textbf{Rarità:} Molto Raro; \textbf{Costo:} 10000 mo

il \textbf{bastone} può essere impugnato come un bastone da combattimento magico che conferisce un bonus di +2 ai Tiri per Colpire e danno effettuati con esso. Inoltre ha le seguenti proprietà. Quando viene usata una di queste proprietà, non se ne potrà più far uso fino all'alba successiva.

- \emph{Fulmine}. Quando colpisci con un attacco in mischia usando il bastone, puoi far sì che il bersaglio subisca 2d6 danni da elettricità aggiuntivi.

- \emph{Tuono}. Quando colpisci con un attacco in mischia usando il bastone, puoi far sì che il bastone emetta il suono di un tuono, udibile fino a 90 metri di distanza. Il bersaglio colpito deve superare un Tiro Salvezza su Tempra con DC 21 o restare stordito fino al termine del tuo prossimo round.

- \emph{Colpo Fulminante}. Puoi usare due azioni per far sì che una fulmine balzi dalla punta del bastone in una linea larga 1 metro e lunga 36 metri. Ogni creatura sulla linea deve effettuare un Tiro Salvezza su Riflessi con DC 21, subendo 9d6 danni da elettricità se lo fallisce, o la metà di questi danni se lo supera.

- \emph{Rombo di Tuono}. Puoi usare due azioni per far sì che il bastone produca un rombo di tuono assordante, udibile fino a 180 metri di distanza. Ogni creatura entro 18 metri da te (te escluso) deve effettuare un Tiro Salvezza su Tempra con DC 21. Se fallisce il Tiro Salvezza, la creatura subisce 2d6 danni da suono e resta assordata per 1 minuto. Se supera il Tiro Salvezza, subisce la metà dei danni e non è assordata.

- \emph{Tuoni e Fulmini}. Puoi usare due Azioni per usare le proprietà Colpo Fulminante e Rombo di Tuono assieme. Farlo non consuma l'uso giornaliero di quelle proprietà, ma solo l'uso di questa.

\oggettomagico{Bastone del Colpire}

\textbf{Rarità:} Molto Raro; \textbf{Costo:} 25000 mo

questo \textbf{bastone} può essere impugnato come un bastone da combattimento magico che conferisce un bonus di +3 ai Tiri per Colpire e di danno effettuati con esso. Quando colpisci con un attacco da mischia facendo uso del bastone, puoi spendere fino a 3 delle sue cariche. Per ogni carica spesa, il bersaglio subisce 1d6 danni da forza aggiuntivi. Il bastone ha 10 cariche, e recupera 1 carica spese ogni giorno all'alba.

\oggettomagico{Bastone del Fuoco}

\textbf{Rarità:} Molto Raro; \textbf{Costo:} 16000 mo

mentre impugni questo \textbf{bastone}, hai resistenza al danno da fuoco.
Inoltre, puoi usare due azioni per spendere 1 o più delle sue cariche per lanciare tramite esso uno dei seguenti incantesimi: Onda rovente (1 carica, DC 13), muro di fuoco (4 cariche, DC 19) o palla di fuoco (3 cariche, DC 17).

Il bastone ha 10 cariche, e recupera 1 carica spesa ogni giorno all'alba.

\oggettomagico{Bastone del Gelo}

\textbf{Rarità:} Molto Raro; \textbf{Costo:} 26000 mo

mentre impugni questo \textbf{bastone}, hai resistenza ai danni da freddo.
Inoltre, puoi usare due azioni per spendere 1 o più delle sue cariche per lanciare tramite esso uno dei seguenti incantesimi.

- \emph{Incantesimi}: cono di freddo (5 cariche, DC 21), muro di ghiaccio (4 cariche, DC 19), nube di nebbia (1 carica, DC 13) o tempesta di ghiaccio (4 cariche, DC 19).

Il bastone ha 10 cariche, e recupera 1 carica spesa ogni giorno all'alba.

\oggettomagico{Bastone del Pitone}

\textbf{Rarità:} Non Comune; \textbf{Costo:} 2000 mo

puoi usare due azioni per pronunciare la parola di comando del \textbf{bastone} e scagliarlo sul terreno fino a 3 metri di distanza. Il bastone diventa un serpente costrittore gigante sotto il tuo controllo e agisce al proprio conteggio di iniziativa. Utilizzando due azioni per pronunciare di nuovo la parola di comando, riporti il bastone alla sua forma normale nello spazio precedentemente occupato dal serpente.

Durante il tuo round puoi impartire ordini mentali al serpente finché si trova entro 18 metri da te e non sei inabile. Decidi tu quali azioni effettuerà il serpente e dove si muoverà durante il suo prossimo round, oppure puoi impartirgli un comando generico, come quello di attaccare i tuoi nemici o difendere un luogo. Se il serpente viene ridotto a 0 Punti Ferita, muore e ritorna alla sua forma di bastone. Poi, il bastone si frantuma ed è distrutto. Se il serpente si ritrasforma in forma di bastone prima di perdere tutti i suoi Punti Ferita, recupera tutti quelli persi.

\oggettomagico{Bastone del Potere}

\textbf{Rarità:} Leggendario; \textbf{Costo:} 150000 mo

questo \textbf{bastone} può essere impugnato come un bastone da combattimento magico che conferisce un bonus di +2 ai Tiri per Colpire e danno effettuati con esso. Mentre lo impugni, ricevi un bonus di +2 alla Difesa, ai Tiri Salvezza, e ai Tiri per Colpire con incantesimi. Questo bastone ha 20 cariche per le seguenti proprietà. Recupera 1d8 + 1 cariche spese ogni giorno all'alba. Se spendi l'ultima carica del bastone, tira 1d6 se ottieni 1 o meno il bastone mantiene il suo bonus di +2 ai Tiri per Colpire e danno ma perde tutte le altre proprietà.

- \emph{Colpo di Potere}. Quando colpisci con un attacco in mischia usando questo bastone, puoi spendere 1 carica per infliggere 1d6 danni da forza aggiuntivi al bersaglio.

- \emph{Incantesimi}. Mentre impugni questo bastone, puoi usare due azioni per spendere 1 o più delle sue cariche per lanciare tramite esso uno dei seguenti incantesimi: blocca mostri (5 cariche DC 21), cono di freddo (5 cariche, DC 21), globo di invulnerabilità (6 cariche, DC 22), levitazione (2 cariche DC 15), muro di forza (5 cariche, DC 21), palla di fuoco (3 cariche DC 17), Dardo arcano (1 carica), raggio di indebolimento (1 carica DC 11) o fulmine (3 cariche DC 17).

- \emph{Colpo di Vendetta}. Puoi usare due azioni per spezzare il bastone sul tuo ginocchio o contro una superficie solida, eseguendo un colpo di vendetta. Il bastone viene distrutto e libera la sua magia rimanente in un'esplosione che si espande fino a riempire una sfera di 9 metri di raggio centrata su di esso.

Hai il 50\% di probabilità di viaggiare istantaneamente in un piano di esistenza a caso, evitando così l'esplosione. Se non riesci a evitare l'effetto, subisci danni da forza pari a 16 x il numero di cariche nel bastone. Ogni altra creatura nell'area deve effettuare un Tiro Salvezza su Riflessi con DC 27. Se il Tiro Salvezza fallisce, la creatura subisce un ammontare di danno basato sulla distanza dal punto di origine dell'esplosione, come mostrato sulla tabella seguente.

Se il Tiro Salvezza riesce, la creatura subisce la metà di questi danni.

\medskip

\noindent\begin{tabularx}{0.49\textwidth}{ll}
	\toprule
\rowcolor{gray!20}\textbf{Distanza dall'origine} &\textbf{Danno}\\
\toprule
3 metri o meno &8 x cariche nel bastone\\
\rowcolor{gray!20}Fino a 6 metri& 6 x cariche nel bastone\\
Fino a 9 metri& 4 x cariche nel bastone
\end{tabularx}

\medskip

Nota: il Bastone dell'Archimago e del Potere sono simili, questo perché preparati da due acerrimi nemici che volevano creare il Bastone più potente.

\oggettomagico{Bastone dell'Arcimago}

\textbf{Rarità:} Leggendario; \textbf{Costo:} 125000 mo

il \textbf{bastone} dell'arcimago è una versione molto potente del bastone della stregoneria.

Esso mette a disposizione del possessore diversi incantesimi. Il bastone può essere usato per manifestare incantesimi: Serratura Magica, individuazione del magico, ingrandire/ridurre e luce. Queste capacità non richiedono il consumo di cariche.

In aggiunta, il bastone possiede le seguenti capacità che costano 1 carica per uso: dissolvi magie, fulmine, invisibilità, muro di fuoco, palla di fuoco, passa porta, piroesperto, ragnatela, scassinare e tempesta di ghiaccio.

Le seguenti, potenti capacità costano 2 cariche per uso: evoca elementale, telecinesi. Il possessore del bastone riceve un bonus +2 ai tiri salvezza contro incantesimi.

Il bastone può essere ricaricato, ma soltanto assorbendo le energie magiche lanciate contro il possessore, il quale può assorbirle in quantità pari ad 1 carica per livello dell'incantesimo. Questa operazione è la sola Azione possibile in un round, ed il bastone non può essere usato per altri effetti nello stesso round in cui esso assorbe energia.

Ciascun bastone ha un numero massimo di cariche possibili, ed esso assorbirà cariche solo fino al suo limite senza incorrere in effetti deleteri. Il possessore non ha modo di conoscere tale limite, o quante cariche sono state usate, a meno di non usare qualche metodo magico.

Se il bastone assorbe energia in eccesso, esso esplode come nel caso di un colpo definitivo, descritto di seguito.

Un bastone dell'arcimago può essere usato per un colpo definitivo, il che richiede che esso venga spezzato dal suo possessore. La rottura non deve essere accidentale e deve essere dichiarata. Tutte le cariche immagazzinate nel bastone vengono rilasciate istantaneamente nel raggio di 9 m. Tutte le creature entro 3 m subiscono ferite pari a 10 volte il numero di cariche nel bastone; tra i 3 m ed i 6 m le ferite sono 6 volte il numero di cariche; e tra i 6 m ed i 9 m le ferite sono 4 volte il numero di cariche. Un Tiro Salvezza su Tempra a DC 25 riduce il danno a metà.

Il personaggio che spezza il bastone ha il 50\% di probabilità di andare su un altro piano di esistenza, altrimenti il rilascio esplosivo di energia magica lo distrugge. Quando tutte le cariche sono state consumate, il bastone diviene un bastone +2. Se le cariche sono esaurite non può essere usato per un colpo definitivo.

\oggettomagico{Bastone dell'Avvizzimento}

\textbf{Rarità:} Raro; \textbf{Costo:} 3000 mo

il \textbf{bastone} può essere impugnato come un bastone da combattimento magico. Se colpisci, infligge danni come un normale bastone da combattimento, e puoi spendere 1 carica per infliggere 2d10 danni da Vuoto aggiuntivi al bersaglio. Inoltre, il bersaglio deve superare un Tiro Salvezza su Tempra con DC 18 o avere -1d6 per 1 ora a qualsiasi prova di Competenza o Tiro Salvezza che richieda Costituzione Questo bastone ha 3 cariche e recupera 1d3 cariche spese a mezzanotte.

\oggettomagico{Bastone della Guarigione}

\textbf{Rarità:} Raro; \textbf{Costo:} 13000 mo

mentre impugni il \textbf{bastone}, puoi usare due azioni per spendere 1 o più delle sue cariche per lanciare tramite esso uno dei seguenti incantesimi: Cura Ferite (1 carica), ristorare inferiore (2 cariche), rimuovi malattie (3 cariche). Questo bastone ha 10 cariche, e recupera 1 carica spesa ogni giorno all'alba.

\oggettomagico{Bastone della Stregoneria}

\textbf{Rarità:} Molto Raro; \textbf{Costo:} 85000 mo

in combattimento, questo \textbf{bastone} funziona come un bastone +1. Può essere usato per lanciare evoca elementale, invisibilità, Passa Porta e ragnatela. Il bastone può essere usato come una bacchetta della paralisi. Ciascuno di questi poteri richiede una carica. E' possibile spezzare il bastone per produrre un "colpo finale", il cui effetto dipende dal numero di cariche residue. Il bastone esplode in una grande sfera di fiamme, colpendo tutte le creature entro 9 m (compreso il proprietario del bastone) e infliggendo 8 ferite per carica rimasta, Tiro Salvezza su Tempra DC 27 per dimezzare.

\oggettomagico{Bastone dello Charme}

\textbf{Rarità:} Raro; \textbf{Costo:} 12000 mo

mentre impugni questo \textbf{bastone}, puoi usare due azioni per spendere 1 carica per lanciare tramite esso \hyperlink{Charme su Persone}{Charme su Persone}, comando o comprendere linguaggi, utilizzando la tua DC dei Tiri Salvezza degli incantesimi. Il bastone può essere usato come bastone da combattimento magico.

Se stai impugnando il bastone e fallisci un Tiro Salvezza contro un incantesimo di ammaliamento che prende come bersaglio solo te e non un'area, puoi trasformare il Tiro Salvezza fallito in un successo. Non potrai più usare questa proprietà del bastone fino all'alba del giorno successivo.

Se riesci in un Tiro Salvezza contro un incantesimo di ammaliamento che prende come bersaglio solo te, con o senza l'intervento del bastone, puoi usare una Azione di Reazione per spendere 3 carica dal bastone e rivolgere l'incantesimo contro chi lo ha lanciato, come se l'incantesimo fosse stato lanciato da te.

Il bastone ha 7 cariche, e recupera 1 carica spesa ogni giorno all'alba.

\oggettomagico{Battaglio del Cannibalismo}

questo \textbf{oggetto} sembra una Battaglio dell'apertura. Funziona come tale per il primo round di uso (ed ha 1d4x10 cariche per questo scopo). Tuttavia, al secondo tintinnio tutte le creature entro 18 m devono riuscire in un Tiro Salvezza su Volontà DC 21 o cadere preda di una fame vorace, attaccando il più vicino umanoide per ucciderlo e divorarlo. A round alterni è permesso un nuovo Tiro Salvezza. Se non sono presenti umanoidi, le creature affette attaccheranno le altre creature presenti.

\oggettomagico{Battaglio dell'Apertura}

\textbf{Rarità:} Raro; \textbf{Costo:} 1500 mo

questo \textbf{tubo} metallico cavo misura circa 30 centimetri di lunghezza e pesa 0,5 chili, ingombro 1. Puoi batterlo con due azioni, puntandolo verso un oggetto entro 36 metri che può essere aperto, come una porta o una serratura non magica. Il battaglio emette un suono limpido e una serratura o chiusura dell'oggetto si apre a meno che il suono sia impedito dal raggiungere l'oggetto. Se non rimangono serrature o lacci da aprire, l'oggetto si apre da sé.

Il battaglio può essere usato dieci volte. Dopo la decima, si spacca e diventa inutilizzabile.

\oggettomagico{Borsa Conservante}

\textbf{Rarità:} Comune

Sussistono diverse tipologie di Borse Conservanti e tutte hanno in comune la capacità di poter contenere molto di più di quello che dovrebbero date le loro dimensioni.

Le Borse Conservanti si suddividono in 4 tipi (Tipo I, II, III; IV) a seconda dalla capacità di conservazione che hanno.

Se la borsa è sovraccarica, perforata o strappata, la borsa si rompe ed è distrutta e il suo contenuto sparpagliato per il Piano Astrale. Se la borsa viene rivoltata, i suoi contenuti vengono espulsi, illesi, ma la borsa dev'essere rimessa nel verso giusto prima che possa essere riutilizzata. Le creature che respirano, piazzate nella borsa, possono sopravvivervi per un numero di minuti pari a 10 diviso il numero di creature (minimo 1 minuto), dopodiché inizieranno a soffocare.

Piazzare una borsa conservante all'interno dello spazio extradimensionale generato da uno zainetto pratico, un buco portatile o simile oggetto, distrugge entrambi gli oggetti e apre un portale verso il Piano Astrale. Il portale origina nel punto in cui un oggetto è stato posto all'interno dell'altro. Qualsiasi creatura entro 3 metri dal portale viene risucchiata al suo interno e ricompare in un posto a caso sul Piano Astrale, poi il portale si richiude. Il portale è a senso unico e non può essere riaperto.

Alcuni incantatori preferiscono creare Bauli Conservanti, che funzionano nell'identico modo delle borse conservanti.

\oggettomagico{Borsa Conservante Tipo I}

\textbf{Rarità:} Non Comune; \textbf{Costo:} 500 mo

questo è il modello più piccolo delle \textbf{borse} conservanti. All'apparenza è un piccolo sacchetto di 20 cm di diametro con una bocca larga circa altrettanto.
Non è possibile fare entrare oggetti che abbiano una larghezza superiore ai 20 cm ed una lunghezza superiore ai 50cm.
La capacità massima è di 20 kg/Ingombro 7.

\oggettomagico{Borsa Conservante Tipo II}

\textbf{Rarità:} Non Comune; \textbf{Costo:} 1000 mo

questo è il modello medio delle \textbf{borse} conservanti. All'apparenza è un sacchetto di 40 cm di diametro con una bocca larga circa altrettanto.
Non è possibile fare entrare oggetti che abbiano una larghezza superiore ai 40 cm ed una lunghezza superiore ai 100cm.
La capacità massima è di 100 kg/Ingombro 25.

\oggettomagico{Borsa Conservante Tipo III}

\textbf{Rarità:} Raro; \textbf{Costo:} 1500 mo

all'apparenza è un \textbf{sacco} di 80 cm di diametro con una bocca larga circa altrettanto.
Non è possibile fare entrare oggetti che abbiano una larghezza superiore ai 80 cm ed una lunghezza superiore ai 150cm. La capacità massima è di 200 kg/Ingombro 50.

\oggettomagico{Borsa Conservante Tipo IV}

\textbf{Rarità:} Molto Raro; \textbf{Costo:} 5000 mo

all'apparenza è un \textbf{saccone} di 120 cm di diametro con una bocca larga circa altrettanto.
Non è possibile fare entrare oggetti che abbiano una larghezza superiore ai 120 cm ed una lunghezza superiore ai 200cm. La capacità massima è di 300 kg/Ingombro 75.

\oggettomagico{Borsa dei Fagioli}

\textbf{Rarità:} Raro; \textbf{Costo:} 5000 mo

all'interno di questa \textbf{borsa} si trovano 3d4 fagioli secchi. La borsa pesa 250 grammi più 125 grammi per ogni fagiolo che contiene.

Se riversi il contenuto della borsa sul terreno, i fagioli esplodono in un raggio di 3 metri. Ogni creatura nell'area, te compreso, deve effettuare un Tiro Salvezza di Riflessi con DC 18, subendo 5d4 danni da fuoco se lo fallisce, o la metà di questi danni se lo supera.

Il fuoco incendia gli oggetti infiammabili nell'area che non siano indossati o trasportati. Se rimuovi il fagiolo dalla borsa, lo pianti nel terreno o la sabbia, e lo innaffi, il fagiolo produrrà un effetto 1 minuto dopo, a partire dal punto del terreno in cui è stato piantato. Il Narratore sceglie l'effetto o lo determina casualmente.

\medskip

\noindent\begin{tabularx}{\linewidth}{lX}
	\toprule
\rowcolor{gray!20}\textbf{d100} & \textbf{Effetto}\\
\toprule
01 &Spuntano 5d4 funghi. Se una creatura mangia un fungo, tira un dado. Se il risultato è dispari esegui un tiro Salvezza su Tempra con DC 18 o subire 5d6 danni da veleno e restare avvelenato per 1 ora. Se il risultato è pari ottiene 5d6 Punti Ferita temporanei per 1 ora.\\
\rowcolor{gray!20}02-10 &Erutta un geyser che sputa acqua, birra, succo di frutta, tè, aceto, vino od olio (a discrezione del Narratore) a 9 metri in aria per 1d12 round.\\
11-20 &Spunta un \hyperlink{Uomo Albero (Arborom)}{uomo albero}. C'è una probabilità del 50\% che l'uomo albero sia malvagio e ti attacchi.\\
\rowcolor{gray!20}21-30 &Una statua di pietra animata con le tue fattezze si leva dal terreno. Essa comincerà a minacciarti verbalmente. Se dovessi andartene e altre persone giungessero sul posto, la statua ti descriverebbe come il più pericoloso dei criminali e li esorterebbe ad cercarti e attaccarti. Se ti trovi sullo stesso piano di esistenza della statua, essa saprà sempre dove sei. Dopo 24 ore la statua diventerà inanimata.\\
31-40 &Un fuoco da campo che produce fiamme blu spunta dal terreno e brucia per 24 ore (o finché non viene spento).\\
\rowcolor{gray!20}41-50 &Sputano 1d6 + 6 \hyperlink{Fungo Stridente}{Fungo Stridente}.\\
51-60 &Compaiono 1d4 + 8 rospi fucsia. Ogniqualvolta un rospo viene toccato, si trasforma in un mostro di taglia Grande o inferiore a scelta del Narratore. Il mostro resta per 1 minuto e poi scompare in un sbuffo di fumo fucsia. \\
\rowcolor{gray!20}61-70 & Una bulette esce dal terreno e attacca.\\
\end{tabularx}
\noindent\begin{tabularx}{\linewidth}{lX}
\toprule
\rowcolor{gray!20}\textbf{d100} & \textbf{Effetto}\\
\toprule
71-80 &Cresce un albero da frutta. Possiede 1d10+20 frutti, ogni frutto ha la possibilità (50/50) di funzionare come un veleno potenziato o come una pozione naturale a caso. L'albero svanisce dopo 1 ora. I frutti raccolti invece rimangono e mantengono la propria magia per 30 giorni. \\
\rowcolor{gray!20}81-90 &Compare un nido con 1d4+3 uova. Qualsiasi creatura che mangi un uovo deve effettuare un Tiro Salvezza su Tempra con DC 28. Se riesce aumenta permanentemente il suo punteggio di caratteristica più basso di 1, scegliendo casualmente in caso di parità, se fallisce subisce 10d6 danni da forza.\\
91-99 &Spunta dal terreno una piramide dalla base quadrata di 18 metri. All'interno c'è un sarcofago che contiene una mummia sovrana. Il suo sarcofago contiene un tesoro a scelta del Narratore.\\
\rowcolor{gray!20}100 &Un enorme pianta di fagioli cresce sul posto, fino a un'altezza a scelta del Narratore. La cima conduce dovunque voglia il Narratore, che sia il castello di un gigante delle nuvole o un altro piano di esistenza.
\end{tabularx}
%\noindent\begin{tabularx}{\linewidth}{lX}
%\textbf{d100} & \textbf{Effetto}\\
%
%\end{tabularx}

%\begin{multicols}{2}

\medskip

\oggettomagico{Borsa dell'Annullamento}

\textbf{Rarità:} Raro; \textbf{Costo:} 9000 mo

questa \textbf{borsa} magica funziona come una borsa conservante per 1d6 giorni. Trascorso questo periodo, tutto il materiale al suo interno o nuovo materiale aggiunto è soggetto ad una trasformazione dipendente dalla sua natura. Pietre preziose diventano inutili sassi, e metalli preziosi si trasformano in metalli di minor valore tipo piombo. Gli oggetti magici perdono il loro potere senza alcun Tiro Salvezza e si trasformano in oggetti mondani del loro tipo. Solo oggetti magici estremamente potenti sono possibilmente immuni a questo effetto.

\oggettomagico{Borsa Divorante}

\textbf{Rarità:} Raro; \textbf{Costo:} 2000 mo

la \textbf{borsa} appare come una borsa conservante. Quando una parte di una creatura vivente viene posta nella borsa, c'è una probabilità del 50\% che la creatura venga trascinata dentro la borsa. Una creatura all'interno della borsa può usare due azioni per cercare di fuggirne superando un Tiro Salvezza Tempra con Forza DC 25.

Un'altra creatura può usare due azioni per afferrare la creatura all'interno della borsa e tirarla fuori, superando un Tiro Salvezza Tempra con Forza DC 25 (e sempre che non venga a sua volta trascinata dentro la borsa). La materia animale o vegetale posta dentro la borsa viene divorata ed è persa per sempre se inizia il proprio round all'interno della borsa.

\oggettomagico{Bottiglia dell'Efreeti}

\textbf{Rarità:} Molto Raro; \textbf{Costo:} 15000 mo

questa \textbf{bottiglia} di ottone dipinta pesa 500 grammi. Quando usi due azioni per rimuoverne il tappo, una nube di denso fumo fuoriesce dalla bottiglia. Al termine del tuo round, il fumo si dissipa in un lampo di fuoco innocuo ed un efreeti compare in uno spazio non occupato entro 9 metri da te. La prima volta che la bottiglia viene aperta, il Narratore determina casualmente cosa accade.

\medskip

\noindent\begin{tabularx}{\linewidth}{lX}
	\toprule
\rowcolor{gray!20}\textbf{3d6} &\textbf{Effetto}\\
\toprule
3-5 & L'efreeti ti attacca. Dopo aver combattuto per 5 round, l'efreeti scompare e la bottiglia perde la sua magia.\\
\rowcolor{gray!20}6-16 &L'efreeti ti obbedisce per 1 ora, agendo ai tuoi comandi. Poi torna nella bottiglia ed un nuovo tappo appare e la chiude. Il tappo non potrà essere rimosso prima che siano passate 24 ore. Le prossime due volte che la bottiglia viene aperta, si ripresenta lo stesso effetto. Se la bottiglia viene aperta una quarta volta, l'efreeti scappa e scompare e la bottiglia perde la sua magia.\\
\end{tabularx}
\noindent\begin{tabularx}{\linewidth}{lX}
\rowcolor{gray!20}\textbf{3d6} &\textbf{Effetto}\\

17-18 & L'efreeti può lanciare l'incantesimo desiderio a tuo favore per tre volte. Scompare quando conferisce il desiderio finale o dopo 1 ora, allorché la bottiglia perde la sua magia.
\end{tabularx}

\begin{center}
	\includegraphics[width=0.7\linewidth]{immagini/genielamp.png}
\end{center}

\oggettomagico{Bottiglia Fumante}

\textbf{Rarità:} Non Comune; \textbf{Costo:} 1200 mo

dalla bocca di questa \textbf{bottiglia} di ottone fuoriesce continuamente del fumo, trattenuto dal suo tappo di piombo. La bottiglia pesa 500 grammi. Quando usi due azioni per rimuovere il tappo, una nube di denso fumo si sparge in un raggio di 18 metri intorno alla bottiglia. L'area della nube è oscurata pesantemente. Per ciascun minuto in cui la bottiglia resta aperta e all'interno della nube, il raggio aumenta di 3 metri finché non raggiunge il raggio massimo di 36 metri.

La nube persiste fino a quando la bottiglia resta aperta. Chiudere la bottiglia richiede che tu pronunci la sua parola di comando con due azioni. Una volta chiusa la bottiglia, la nube si disperde dopo 10 minuti. Un vento moderato (dai 15 ai 30 km/h) può disperdere il fumo in1 minuto, e un vento forte (più di 30 km/h) può disperderlo in 1 round.

\oggettomagico{Bracciali dell'Arciere}

\textbf{Aura:} Trasmutazione debole; \textbf{Costo:} 3000 mo

\textbf{Requisiti:} Creare Oggetti Magici,

Mentre indossi questi \textbf{bracciali}, hai competenza con la Lista d'armi Archi, ottieni un bonus di +2 ai tiri di danno degli attacchi a distanza effettuati con queste armi.

\oggettomagico{Bracciali della Difesa}

\textbf{Aura:} Abiurazione; \textbf{Costo:}  Costo +6000 mo, 15000 mo, 30000 mo, 45000 mo, 60000 mo.

\textbf{Requisiti:} Creare Oggetti Magici 2; \textbf{Rarità:} Raro

Mentre indossi questi \textbf{bracciali}, hai un bonus di +1, +2, +3, +4+, +5 alla tua Difesa se non indossi nessuna armatura e non usi nessuno scudo.

\oggettomagico{Bracciali della Difesa Maggiore}

\textbf{Aura:} Abiurazione;  \textbf{Costo:}  +12000 mo, 24000 mo, 36000 mo, 50000 mo, 75000 mo

\textbf{Requisiti:} Creare Oggetti Magici 2; \textbf{Rarità:} Leggendario

\emph{Oggetto meraviglioso}

Questi \textbf{bracciali} funzionano come un armatura pur non essendo tali. Vieni avvolto in uno scudo magico invisibile che ti concede Difesa 15, 17, 19, 21, 23. La Difesa può essere aumentata con oggetti magici che migliorano la Difesa, tranne armature e scudi.

\oggettomagico{Braciere del Comando degli Elementali del Fuoco}

\textbf{Rarità:} Raro; \textbf{Costo:} 8000 mo

mentre il fuoco arde all'interno di questo \textbf{braciere} di ottone, puoi usare due azioni per pronunciare la parola di comando del braciere ed evocare un elementale del fuoco, come se avessi lanciato l'incantesimo \hyperlink{Evoca Elementale}{Evoca Elementale}. Il braciere non può di nuovo essere usato a questo modo, fino alla prossima alba.

Il braciere pesa 2,5 chili, ingombro 3.

\oggettomagico{Braciere del Sonno maledetto}

questo \textbf{braciere} ha l'apparenza di, e funziona come, un braciere del comando degli elementi del fuoco. Tuttavia, quando viene attivato, il fumo si addensa per 3 m di raggio intorno al braciere, inducendo a un sonno maledetto chiunque si trovi nell'area, a meno che non riesca in un Tiro Salvezza su Volontà DC 21. Un elementale del fuoco compare normalmente, ma è ostile ed attacca tutte le creature presenti. Creature soggette al sonno maledetto dormono indefinitamente fino a che non vengono uccise, a meno che non venga usato Rimuovi Maledizione.

\oggettomagico{Brillante}

\textbf{Aura:} Invocazione moderata; \textbf{Costo:} 3750 mo

\textbf{Requisiti:} Creare Oggetti Magici, Luce Diurna; \textbf{Rarità:} Leggendario

\textbf{Armature} e \textbf{scudi} con la capacità speciale Brillante irradiano luce come una torcia quando indossati, che può essere soppressa o riattivata a comando. L'aspetto dell'oggetto di solito è caratterizzato da colori vivaci e una brillante lucentezza anche quando non illuminato. Solitamente riporta i simboli di Ljust e/o Sunkjr. Una volta al giorno, il portatore può comandare all'armatura o allo scudo di brillare con l'intensità di un incantesimo Luce Diurna per 10 minuti o finché non gli viene comandato di affievolirla.

Questa armatura va pulita almeno 1 volta al giorno o perde i poteri per una settimana.

\oggettomagico{Brocca dell'Acqua Infinita}

\textbf{Rarità:} Non Comune; \textbf{Costo:} 12000 mo

12000 mo, quest'\textbf{ampolla} tappata emette un suono di liquido quando viene smossa, come se contenesse acqua. La brocca pesa 1 chilo, ingombro 2. Puoi usare due azioni per rimuovere il tappo e pronunciare una delle tre parole di comando, e a quel punto un ammontare di acqua fresca o acqua salata (a tua scelta) si riverserà fuori dell'ampolla, fino all'inizio del tuo prossimo round. Scegli una delle opzioni seguenti:

\begin{itemize}[leftmargin=*] \setlength{\itemsep}{0pt}
\item
\emph{Ruscello} produce 4 litri d'acqua.
\item
\emph{Fontana} produce 20 litri d'acqua.
\item
\emph{Geyser} produce 150 litri d'acqua che vengono proiettati da un geyser lungo 9 metri e largo 30 centimetri. Con due azioni, mentre impugni la brocca, puoi prendere come bersaglio del geyser una creatura visibile entro 9 metri da te.

Il bersaglio deve superare un Tiro Salvezza su Tempra con DC 15 o subire 1d4 danni contundenti e cadere prono. Invece di una creatura, puoi prendere come bersaglio un oggetto che non sia indossato o trasportato e che non pesi più di 100 chili. L'oggetto viene ribaltato o spinto 3 metri lontano da te.
\end{itemize}

\oggettomagico{Brocca delle Pozioni}

\textbf{Rarità:} Leggendaria; \textbf{Costo:} 18000 mo

questa \textbf{brocca} di ceramica azzurra ha un tappo d'oro massiccio. La brocca contiene 1d4+1 pozioni magiche, ognuna delle quali può essere versata ogni 2 giorni. Le specifiche pozioni sono determinate a caso, rimangono le stesse nel tempo e devono essere versate sempre nello stesso ordine. Non tutte sono necessariamente benefiche.

\oggettomagico{Buco Portatile}

\textbf{Rarità:} Raro; \textbf{Costo:} 10000 mo

questo elegante \textbf{tessuto} nero, soffice come la seta, si piega fino alle dimensioni di un fazzoletto. Si dispiega in uno strato circolare di 1 metro di diametro. Puoi usare 1 round per dispiegare un buco portatile e piazzarlo sopra o contro una superficie solida, sulla quale il Buco portatile crea un foro profondo 3 metri. Qualsiasi creatura abbastanza piccola può usare il Buco Portatile per attraversare la parete o superficie su cui è appoggiato purché sia profonda meno di 3 metri.

Puoi usare 1 round per chiudere un Buco portatile prendendo i margini del tessuto e ripiegandolo. Piegare il tessuto chiude il buco, e qualsiasi creatura od oggetto al suo interno viene espulso con una probabilità del 50\% di uscire da una parte o dall'altra.

Piazzare un buco portatile all'interno dello spazio extradimensionale creato da una borsa conservante, Vano portatile, uno zainetto pratico o simile oggetto distrugge istantaneamente entrambi gli oggetti e apre un portale verso il Piano Astrale. Il portale origina dal punto in cui un oggetto è stato piazzato all'interno dell'altro. Qualsiasi creatura entro 3 metri dal portale viene risucchiata al suo interno e depositata in un luogo casuale del Piano Astrale. Poi il portale si chiude. Il portale è a senso unico e non può essere riaperto.

\oggettomagico{Candela di Invocazione}

\textbf{Rarità:} Molto Raro; \textbf{Costo:} 8000 mo

questa lunga e sottile \textbf{candela} è dedicata a un Patrono e ne condivide i Tratti. I Tratti della candela possono essere individuati tramite un rituale di 1 ora di affiancamento alla candela.

Il Narratore sceglie il Patrono e i Tratti associato a esso o lo determina casualmente.

La magia della candela si attiva quando la candela viene accesa con due azioni. Dopo aver bruciato per 4 ore, la candela è distrutta. Puoi decidere di spegnerla anticipatamente per riutilizzarla più tardi. Dedurre il tempo che rimane alla candela prima di estinguersi a incrementi di 1 minuto, per determinare per quanto abbia bruciato la candela.

Quando è accesa, la candela irradia luce fioca in un raggio di 9 metri. Qualsiasi creatura all'interno della luce Devota o Seguace a quello della candela effettua Tiri per Colpire, Tiri Salvezza e prove di competenza con +1d6.

In alternativa, quando accendi la candela per la prima volta, puoi lanciare l'incantesimo portale. Farlo distrugge la candela.

\oggettomagico{Caos}

\textbf{Arma} Maledetta. Questa capacità conferisce un bonus +2 agli attacchi, tuttavia, all'inizio della battaglia, fa sì che il portatore venga preso da una rabbia incontenibile. Il personaggio attaccherà la creatura più vicina, nemica o amica che sia, finché non ne resterà nessuna in vita entro 18 m. E' necessario l'incantesimo Rimuovi Maledizione per liberarsi dall'arma.

\oggettomagico{Cappello del Camuffamento}

\textbf{Rarità:} Non Comune; \textbf{Costo:} 5000 mo

mentre indossi questo \textbf{cappello}, puoi usare due azioni per lanciare a volontà l'incantesimo \emph{\hyperlink{Camuffare Sé Stesso}{Camuffare Sé Stesso}}. L'incantesimo termina quando il cappello viene rimosso.

\oggettomagico{Ceppi Dimensionali}

\textbf{Rarità:} Raro; \textbf{Costo:} 4000 mo

puoi usare 2 Azioni per piazzare queste \textbf{manette} su di una creatura inabile. Le manette si adattano a qualsiasi creatura da taglia Piccola a Grande. Oltre a servire da comuni manette, i ceppi impediscono a una creatura legata con essi dall'usare qualsiasi metodo di movimento extradimensionale, compreso il teletrasporto o il viaggio verso piani diversi dell'esistenza. Tuttavia non impediscono a una creatura di attraversare un portale interdimensionale.

Tu e qualsiasi creatura da te indicata quando fai uso dei ceppi potete usare due azioni per rimuoverli. Una volta ogni 30 giorni, la creatura legata può effettuare un Tiro Salvezza Tempra con Forza DC 40. Se la supera, la creatura si libera e distrugge i ceppi.

\oggettomagico{Cerchietto dell'Esplosione}

\textbf{Aura:} Lista Fuoco; \textbf{Costo:} 1500 mo

\textbf{Requisiti:} Creare Oggetti Magici 1, \hyperlink{Raggio Roventi}{Raggio Rovente}; \textbf{Rarità:} Raro

mentre indossi questo \textbf{cerchietto}, puoi usare due azioni, portando due dita sulla fronte, per lanciare tramite esso l'incantesimo \hyperlink{Raggio Rovente}{Raggio Rovente}. Il cerchietto non potrà essere usato di nuovo a questo modo fino alla prossima alba.

\oggettomagico{Cintura dei Giganti}

\textbf{Rarità:} Molto Raro; \textbf{Costo:} 45000 mo

10000 / 15000 / 20000 / 30000/ mo, mentre indossi questa \textbf{cintura}, il tuo punteggio raggiunge il punteggio conferito dalla cinta. Se il tuo punteggio di Forza è già pari o superiore al punteggio della cinta, l'oggetto non ha effetto su di te.

Esistono quattro varianti di questa cinta, corrispondenti ciascuna a una specie di veri giganti. La cinta del gigante di pietra e la cinta del gigante del gelo appaiono diverse, ma hanno lo stesso effetto.

\medskip

\noindent\begin{tabularx}{\linewidth}{lll}
	\toprule
\rowcolor{gray!20}\textbf{Tipo di Gigante}& \textbf{Forza} &\textbf{Rarità}\\
\toprule
\textbf{Collina} &5& Raro\\
\rowcolor{gray!20}\textbf{Pietra/del Gelo}& 6 &Molto raro\\
\textbf{Fuoco} &7& Molto raro\\
\rowcolor{gray!20}\textbf{Nuvole} &8& Leggendario\\
\textbf{Tempeste}& 9& Leggendario
\end{tabularx}

\oggettomagico{Cintura dei Nani}

\textbf{Rarità:} Raro; \textbf{Costo:} 86000 mo

mentre indossi questa \textbf{cintura}, ottieni i seguenti benefici:

- il tuo punteggio di Costituzione aumenta di 1, fino a un massimo di 5.

- hai +2 alle prove di Carisma effettuate per interagire con i nani.

Mentre indossi la cintura hai il 50\% di probabilità ogni giorno all'alba di vederti spuntare una folta barba, se può crescerti, oppure di vedere la tua ancora più folta, se già la hai.

Se non sei un nano, ottieni i seguenti benefici aggiuntivi quando indossi questa cintura:

- hai +2 ai Tiri Salvezza contro veleno e hai resistenza ai danni da veleno. Hai scurovisione con una gittata di 18 metri. Puoi parlare, leggere e scrivere in Nanico.

\oggettomagico{Colla Suprema}

\textbf{Rarità:} Non Comune; \textbf{Costo:} 400 mo

questa \textbf{sostanza} bianco lattea e viscosa può formare un legame adesivo permanente tra qualsiasi due oggetti. Deve essere contenuto in una giara o ampolla che è stata ricoperta all'interno di olio di scivolosità. Quando viene trovata, il suo contenitore ne tiene 1d6 + 1 per 30 grammi. 30 grammi di colla possono coprire una superficie quadrata di 30 centimetri di lato. La colla ci mette 1 minuto per fissarsi. Una volta fissata la colla, il legame creato può essere spezzato solo dal solvente universale o l'olio della forma eterea, o tramite l'incantesimo desiderio.

\oggettomagico{Collana del Rosario}

\textbf{Aura:} Lista Animali e Piante; \textbf{Costo:} 3000 mo

\textbf{Requisiti:} Creare Oggetti Magici 3, \hyperlink{Palla di Fuoco}{Palla di Fuoco}; \textbf{Rarità:} Raro

questa \textbf{collana} possiede 1d4 + 2 sfere magiche fatte di acquamarina, perla nera o topazio. Possiede anche diverse sfere non magiche. Se una sfera magica venisse rimossa dalla collana, quella sfera perderebbe la sua magia.

Esistono sei tipi di sfere magiche. Il Narratore decide il tipo di ciascuna sfera facente parte della collana. Una collana può avere più di una sfera dello stesso tipo. Per usarla, devi indossare la collana. Ogni sfera contiene un incantesimo che puoi lanciare con due azioni, con DC dell'Incantesimo pari a 12 + 2x Livello in caso di Tiro Salvezza. Una volta che l'incantesimo di una sfera magica è stato lanciato, non potrai usare di nuovo quella sfera fino all'alba successiva.

\medskip

\noindent\begin{tabularx}{\linewidth}{llX}
	\toprule
\rowcolor{gray!20}\textbf{3d6} &\textbf{Sfera di...} &\textbf{Incantesimo}\\
\toprule
3-5 &Benedizione &Benedizione\\
\rowcolor{gray!20}6-11& Cura &Cura ferite 5 o Ristorare inferiore\\
12-14 &Favore Divino& Ristorare superiore\\
\rowcolor{gray!20}15-16& Punire &Punizione marchiante\\
17 &Vento& Camminare nel vento\\
\rowcolor{gray!20}18 &Convocare &Barriera di Lame
\end{tabularx}

\oggettomagico{Collana dell'Adattamento}

\textbf{Aura:} Lista Aria; \textbf{Costo:} 1500 mo

\textbf{Requisiti:} Creare Oggetti Magici 1, \hyperlink{ResistenzaalVeleno}{Resistenza}; \textbf{Rarità:} Non Comune

mentre indossi questa \textbf{collana}, puoi respirare normalmente in qualsiasi ambiente che abbia aria e hai +1d6 ai Tiri Salvezza effettuati contro gas e vapori nocivi.

\oggettomagico{Collana dell'Aria Salubre}

\textbf{Rarità:} Non Comune; \textbf{Costo:} 2500 mo

questa \textbf{collana} è una catena con un medaglione di platino. La magia della collana circonda chi la indossa con una bolla di aria pura, rendendolo immune agli effetti dei vapori e dei gas. La bolla consente di sopravvivere in un ambiente senz'aria per una settimana.

\oggettomagico{Collana delle Palle di Fuoco}

\textbf{Aura:} Lista Animali e Piante; \textbf{Costo:} 500 mo * Palla di Fuoco

\textbf{Requisiti:} Creare Oggetti Magici 2, \hyperlink{Palla di Fuoco}{Palla di Fuoco}; \textbf{Rarità:} Molto Raro

da questa \textbf{collana} pendono 1d6 + 3 sfere. Puoi usare due azioni per staccare una sfera e lanciarla fino a 18 metri di distanza. Quando essa raggiunge il termine della sua traiettoria, la sfera detona come un incantesimo palla di fuoco (DC 18).

\oggettomagico{Collana dello Strangolamento}

\textbf{Aura:} Lista Animali e Piante; \textbf{Costo:} 3000 mo

\textbf{Requisiti:} Creare Oggetti Magici 2, \hyperlink{Crescita Spuntoni}{Crescita Spuntoni}; \textbf{Rarità:} Raro

questa \textbf{collana} sembra un gioiello di grande valore. Appena indossata, si stringe fulmineamente intorno al collo, infliggendo 6 danni a round. Non può essere rimossa in alcun modo se non con un Desiderio o Rimuovi Maledizione, rimanendo stretta al collo della sua vittima anche dopo la morte. La collana si allenterà solo quando la vittima sarà diventata uno scheletro, pronta per essere raccolta da un ignaro cacciatore di tesori.

\oggettomagico{Coraggiosa}

\textbf{Aura:} Ammaliamento debole; \textbf{Costo:} 3000 mo

\textbf{Requisiti:} Creare Oggetti Magici, Eroismo, Paura

Questa capacità speciale può essere aggiunta solo a un'\textbf{arma} da mischia. Un'arma Coraggiosa fortifica il coraggio e il morale in battaglia di chi la indossa. Chi la impugna ottiene un Bonus ai Tiri Salvezza contro Paura pari al bonus dell'arma.

\oggettomagico{Corda da Arrampicata}

\textbf{Rarità:} Non Comune; \textbf{Costo:} 2000 mo

questa \textbf{corda} di seta lunga 18 metri, pesa 1,5 chili, ingombro 1, e può sostenere fino a 1.500 chili. Se impugni un'estremità della corda e usi due azioni per pronunciare la parola di comando, la corda si anima. Con due azioni puoi comandare all'altra estremità di muoversi verso una destinazione a tua scelta. Quell'estremità si muove di 3 metri durante il tuo round quando riceve il tuo primo comando, e di 3 metri durante ciascun round successivo finché non raggiunge la sua destinazione, fino alla sua lunghezza massima, o finché non le dici di fermarsi. Puoi anche dire alla corda di stringersi o sganciarsi da un oggetto, annodarsi o snodarsi o riavvolgersi per essere trasportata. Se dici alla corda di compiere un nodo, grossi nodi compariranno a intervalli di 30 centimetri lungo la corda. Mentre è annodata, la corda diminuisce fino ad un lunghezza di 15 metri e conferisce +1d6 alle prove effettuate per arrampicarvisi.

La corda ha Difesa 20, Durezza 3 e 20 Punti Ferita. Recupera 1 punto ferita ogni 5 minuti finché ha almeno 1 punto ferita. Se la corda scende a 0 Punti Ferita, è distrutta.

\oggettomagico{Corda dell'Intralciamento}

\textbf{Rarità:} Raro; \textbf{Costo:} 4000 mo

questa \textbf{corda} è lunga 9 metri e pesa 1,5 chili, ingombro 1. Se tieni un'estremità della corda e usi due azioni per pronunciare la sua parola di comando, l'altra estremità scatterà in avanti per impigliare una creatura visibile entro 6 metri da te. Il bersaglio deve superare un Tiro Salvezza su Riflessi con DC 18 o restare intralciato. Puoi rilasciare la creatura usando due azioni per pronunciare una seconda parola di comando. Un bersaglio intralciato dalla corda può usare due azioni per effettuare una prova contrapposta di Forza DC 25 o Artista della Fuga DC 18 (a scelta del bersaglio). Se la supera, la creatura non è più intralciata dalla corda.

La corda ha Difesa 20 e 20 Punti Ferita. Recupera 1 punto ferita ogni 5 minuti finché ha almeno 1 punto ferita. se la corda scende a 0 Punti Ferita, è distrutta.

\oggettomagico{Corda Strozzatrice}

\textbf{Rarità:} Raro

questa \textbf{corda} magica, benché normale all'apparenza, può animarsi e aggredire chi cerca di usarla, stringendosi intorno al collo e cercando di strangolare la sua vittima. La corda strozzatrice è lunga abbastanza da poter strangolare fimo a 1d4 creature in un raggio di 3 m, infliggendo 2d6 ferite a round a ognuna di loro. E' necessario superare un Tiro Salvezza su Riflessi DC 19 per evitare di essere presi. La corda ha Difesa 22 e 25 Punti Ferita, ma solo chi non viene strangolato può attaccarla. Le vittime non possono liberarsi da sole in alcun modo, né lanciare incantesimi.

\oggettomagico{Corno del Valhalla}

\textbf{Rarità:} Raro; \textbf{Costo:} 6000 mo

puoi usare due Azioni per suonare questo \textbf{corno}. Come risposta, entro 18 metri da te appaiono gli spiriti guerrieri di Asgard. Questi spiriti usano le statistiche dei berserker. Essi ritornano ad Asgard dopo 1 ora o quando scendono a 0 Punti Ferita. Una volta usato, il corno non potrà essere usato di nuovo prima che siano passati 7 giorni.

%\begin{center}
%\includegraphics[width=0.65\linewidth]{immagini/cube_grayscale.png}

%\emph{Cubo di Forza}
%\end{center}

\oggettomagico{Corno di Distruzione}

\textbf{Rarità:} Raro; \textbf{Costo:} 750 mo

puoi usare due azioni per pronunciare la parola di comando del \textbf{corno} e poi suonarlo, emettendo uno scoppio tonante in un cono di 9 metri e udibile fino a 180 metri di distanza. Ogni creatura all'interno del cono deve effettuare un tiro Salvezza su Tempra con DC 18. Se il Tiro Salvezza fallisce, la creatura subisce 5d6 danni da suono e resta assordata per 1 minuto. Se il Tiro Salvezza riesce, la creatura subisce la metà dei danni e non è assordata. Le creature e gli oggetti fatti di vetro o cristallo hanno -1d6 al Tiro Salvezza e subiscono 10d6 danni da suono anziché 5d6.

Ogni uso della magia del corno ha il 20\% di probabilità di farlo esplodere. L'esplosione infligge 10d6 danni da fuoco a chi lo suona e distrugge il corno.

\oggettomagico{Corrosiva}

\textbf{Aura:} Lista di Terra moderata; \textbf{Costo:} 3000 mo

\textbf{Requisiti:} Creare Oggetti Magici, \hyperlink{Freccia Acida di Restser}{Freccia Acida di Restser}

Un'\textbf{arma} Corrosiva è coperta da uno strato di acido che infligge 1d6 danni aggiuntivi da acido quando colpisce il bersaglio. L'acido non danneggia chi la impugna.

\oggettomagico{Cubo di Forza}

\textbf{Rarità:} Raro; \textbf{Costo:} 16000 mo

questo \textbf{cubo} ha 2,5 centimetri di spigolo. Ogni faccia ha un marchio unico che può essere premuto. Il cubo inizia con 36 cariche, e recupera 3d6 cariche spese ogni giorno all'alba. Puoi usare due Azioni per premere una delle facce del cubo, spendendo un numero di cariche basate sulla faccia del cubo.

Ogni faccia ha un effetto diverso. Se al cubo non rimangono più cariche, non succede nulla. Altrimenti, si erge una barriera di forza invisibile, che forma un cubo di 3 metri di spigolo. La barriera è centrata su di te, si muove con te, e dura per 1 minuto, fino a che non usi due azioni per premere la sesta faccia del cubo, o il cubo esaurisce le cariche. Puoi cambiare l'effetto della barriera premendo una faccia diversa del cubo e spendendo il numero di cariche richiesto, resettandone la durata.

Se il tuo movimento fa sì che la barriera entri a contatto con un oggetto solido che non può attraversare il cubo, finché rimane la barriera non potrai avvicinarti all'oggetto.

Il cubo perde cariche quando la barriera viene presa come bersaglio da certi incantesimi o entra a contatto con certi incantesimi o effetti di oggetti magici, come indicato nella tabella seguente.

\medskip

\noindent\begin{tabularx}{\linewidth}{ll}
	\toprule
\rowcolor{gray!20}\textbf{Incantesimo o Oggetto} &\textbf{Cariche Perse}\\
\toprule
Dardo arcano (3 colpi) &1\\
\rowcolor{gray!20}Disintegrazione &1d12\\
Muro di fuoco &1d4\\
\rowcolor{gray!20}Passa Porta& 1d6\\
Spruzzo Prismatico &3d6
\end{tabularx}

\medskip

Qui sotto le proprietà di ogni faccia attivabile ed il relativo costo in cariche.

\medskip

\noindent\begin{tabularx}{\linewidth}{ccX}
	\toprule
\rowcolor{gray!20}\textbf{Faccia} & \textbf{Costo}& \textbf{Effetto}\\
\toprule
1&1 & Gas, vento e nebbia non possono penetrare la barriera\\
\rowcolor{gray!20}2&2 & La materia non vivente non può attraversare la barriera. Muri, pavimenti e soffitti possono attraversarla a tua discrezione.\\
3&3 & La materia vivente non può attraversare la barriera.\\
\rowcolor{gray!20}4&4 & Gli effetti dell'incantesimo non possono attraversare la barriera.\\
5&5 & Nulla può attraversare la barriera. Muri, pavimenti e soffitti possono attraversarla a tua discrezione.\\
\rowcolor{gray!20}6&0 & La barriera si disattiva.
\end{tabularx}

\medskip

Ogni attivazione dura un minuto.

\medskip

\oggettomagico{Cubo di protezione dal freddo}

\textbf{Rarità:} Raro; \textbf{Costo:} 2500 mo

questo \textbf{ciondolo} cubico si attiva e si disattiva premendone una faccia (Azione Immediata). Quando è attivato, emana un campo protettivo cubico con lo spigolo di 3 m (simile a quello di un cubo di forza ma dall'effetto diverso). La temperatura all'interno del campo protettivo si mantiene a 21 °C. Il campo assorbe tutti gli attacchi di freddo, negandoli completamente. Se nega più di 50 danni da freddo in un round (sia da un singolo attacco che da attacchi multipli), però, il campo magico collassa e non può essere riattivato per un'ora. Se il campo nega più di 100 ferite da freddo in un round, il cubo viene distrutto.

\oggettomagico{Danzante}

\textbf{Aura:} Trasmutazione forte; \textbf{Costo:} 25000 mo

\textbf{Requisiti:} Creare Oggetti Magici 2, Animare Oggetti

Con 2 Azioni un'\textbf{arma} Danzante può essere lasciata libera in modo che combatta da sola. L'arma combatte per 4 round usando il bonus al Tiro per Colpire di colui che l'ha lasciata libera e poi cade a terra.

Rimane sempre accanto alla persona che l'ha liberata, anche se si sposta con mezzi fisici o magici. Se colui che l'ha lasciata libera ha una mano libera può riprendere l'arma che sta attaccando da sola, come Azione Immediata, ma una volta ripresa, la spada non potrà più danzare (attaccare da sola) prima di 4 round.

Questa capacità può essere aggiunta solo ad armi da mischia.

\oggettomagico{Del Dolore}

\textbf{Aura:} Invocazione moderata; \textbf{Costo:} 3000 mo

\textbf{Requisiti:} Creare Oggetti Magici 2, Ferire

Questa capacità può essere aggiunta solo ad \textbf{armi} da mischia. Quando un'arma del dolore colpisce un avversario, produce un lampo di Vuoto che riecheggia tra chi la impugna e il suo bersaglio. L'energia infligge 2d6 danni addizionali all'avversario e 1d6 danni a chi la impugna.

\oggettomagico{Denegante}

\textbf{Aura:} Abiurazione forte; \textbf{Costo:} 25000 mo

\textbf{Requisiti:} Creare Oggetti Magici 2; \textbf{Rarità:} Rara

Quando colui che indossa l'\textbf{armatura} è bersaglio di un Colpo Critico o Esplosione del danno effettuato con un'arma da mischia, può automaticamente negare questo Critico e renderlo un attacco normale. Questa capacità può essere applicata solo alle armature pesanti. L'abilità è usabile un numero di volte al giorno pari al bonus magico dell'armatura/scudo.

\oggettomagico{Determinazione}

\textbf{Aura:} Lista di Cura moderata; \textbf{Costo:} 15000 mo

\textbf{Requisiti:} Creare Oggetti Magici 2, Cura ferite; \textbf{Rarità:} Non Comune

Uno \textbf{scudo} o un'\textbf{armatura} concede la capacità di combattere in circostanze apparentemente impossibili. Una volta al giorno, quando il possessore raggiungere 0 o meno Punti Ferita, l'oggetto attiva automaticamente l'incantesimo Cure ferite 2.

\oggettomagico{Difensiva}

\textbf{Aura:} Abiurazione moderata; \textbf{Costo:} 3000 mo

\textbf{Requisiti:} Creare Oggetti Magici 2, Scudo

Un'\textbf{arma} Difensiva permette a chi la impugna di trasferire una parte o tutto il bonus dell'arma alla propria Difesa come un bonus cumulabile con eventuali altri bonus. Come Azione Immediata ad inizio del proprio round chi la impugna può scegliere di trasferire parte o tutto del bonus al colpire dell'arma come bonus alla Difesa e Tiri Salvezza. I punteggio dei bonus rimangono validi fino all'inizio del round successivo.

\oggettomagico{Difesa dagli Incantesimi}

\textbf{Aura:} Abiurazione forte; \textbf{Costo:} 5000 mo

\textbf{Requisiti:} Creare Oggetti Magici 2; \textbf{Rarità:} Rara

Hai +1d6 ai Tiri Salvezza contro incantesimi e altri effetti magici.

\oggettomagico{Distanza}

\textbf{Aura:} Divinazione moderata; \textbf{Costo:} 3000 mo

\textbf{Requisiti:} Creare Oggetti Magici, Chiaroveggenza

Questa capacità speciale può essere aggiunta solo a \textbf{proiettili}. Un proiettile della Distanza ignora la penalità al colpire fino al triplo della gittata. Il costo si intendo per 20 proiettili.

\oggettomagico{Distruttrice dei Giganti}

\textbf{Aura:} Invocazione forte; \textbf{Costo:} 16000 mo

\textbf{Requisiti:} Creare Oggetti Magici 3

Devi indossare una \emph{\textbf{cintura} dei giganti} (qualsiasi varietà) e i \emph{guanti del potere orchesco} per poter usare quest'arma.

Mentre usi il martello il tuo punteggio di Forza aumenta di 2 (fino ad un massimo di 7).

Quando ottieni un critico sul Tiro per Colpire effettuato con quest'arma contro un gigante, il gigante deve superare un tiro Salvezza su Tempra con DC 21 o morire.

Puoi spendere 1 carica ed effettuare un attacco con arma a distanza scagliandolo come se avesse gittata di 6 metri. Se l'attacco colpisce, il martello produce un tuono udibile fino a 90 metri di distanza. Il bersaglio e tutte le creature entro 9 metri da esso devono superare un tiro Salvezza su Tempra con DC 21 o restare stordite fino al termine del tuo prossimo round.

Il martello ha 5 cariche, e recupera 1 carica spesa ogni giorno all'alba.

\oggettomagico{Distruzione}

\textbf{Aura:} Evocazione forte; \textbf{Costo:} 6000 mo

\textbf{Requisiti:} Creare Oggetti Magici 2, Guarigione

Un'\textbf{arma} della Distruzione è la rovina di tutti i Non Morti. Ogni creatura Non Morta colpita in combattimento deve superare un Tiro Salvezza su Volontà con DC 14 o viene distrutta o subire 2d8 di danni aggiuntivi da Luce. Un'arma della Distruzione deve essere un'arma da mischia contundente.

\oggettomagico{Elegante}

\textbf{Aura:} Illusione moderata; \textbf{Rarità:} Comune; \textbf{Costo:} 3000 mo

\textbf{Requisiti:} Creare Oggetti Magici 2; \textbf{Rarità:} Comune

Puoi usare due azioni per pronunciare la parola di comando per ottenere che l'\textbf{armatura} assuma l'aspetto di un comune abito o qualche altro tipo di armatura. Decidi tu l'aspetto, compreso il colore, lo stile e gli accessori, ma l'armatura / scudo mantiene il suo normale Ingombro e peso. L'aspetto illusorio dura finché non usi di nuovo questa proprietà o ti togli l'armatura.

\oggettomagico{Elmo del Movimento subacqueo}

\textbf{Rarità:} Raro; \textbf{Costo:} 4000 mo

questo \textbf{elmo}, solitamente in pelle di pesce, conferisce la capacità di respirare sott'acqua, movimento Nuotare 18 metri, ecolocalizzazione 18 metri. Il potere è usabile per 6 ore al giorno e si ricarica all'alba.

\oggettomagico{Elmo del Teletrasporto}

\textbf{Rarità:} Raro; \textbf{Costo:} 64000 mo

mentre indossi questo \textbf{elmo}, puoi usare due azioni e spendere 1 carica per lanciare l'incantesimo teletrasporto tramite esso. L'elmo ha 3 cariche, e ne recupera 1 ogni mattina all'alba.

\oggettomagico{Elmo della Comprensione dei Linguaggi}

\textbf{Rarità:} Comune; \textbf{Costo:} 600 mo

mentre indossi questo \textbf{elmo}, puoi usare due azioni per lanciare a volontà tramite esso l'incantesimo comprendere linguaggi.

\oggettomagico{Elmo della Lucentezza}

\textbf{Rarità:} Leggendario; \textbf{Costo:} 75000 mo

questo \textbf{elmo} luminoso è incastonato con 1d10 diamanti, 2d10 rubini, 3d10 opali di fuoco e 4d10 opali. Qualsiasi gemma estratta dall'elmo si riduce in polvere. Quando tutte le gemme sono rimosse o distrutte, l'elmo perde la sua magia. Mentre lo indossi ottieni i seguenti benefici:

\smallskip

\begin{itemize}[leftmargin=*] \setlength{\itemsep}{0pt}
\item
Puoi usare due azioni per lanciare uno dei seguenti incantesimi, usando una delle gemme dell'elmo del tipo specificato come componente: \hyperlink{Luce Diurna}{Luce Diurna} (opale), \hyperlink{Muro di Fuoco}{Muro di Fuoco} (rubino), \hyperlink{Palla di Fuoco}{Palla di Fuoco} (opale di fuoco) o \hyperlink{Spruzzo Prismatico}{Spruzzo Prismatico} (diamante). Quando l'incantesimo viene lanciato la gemma è distrutta e scompare dall'elmo.

\item
Finché possiede almeno un diamante, l'elmo emette luce in un raggio di 9 metri quando almeno un non morto si trova entro quest'area. Qualsiasi non morto che inizi il suo round all'interno dell'area subisce 1d6 danni da Luce.

\item
Finché l'elmo possiede almeno un rubino, hai resistenza ai danni da fuoco.
\end{itemize}

\smallskip

Finché l'elmo possiede almeno un opale di fuoco, puoi usare due azioni e pronunciare una parola di comando per far sì che un'arma che stai impugnando venga avvolta dalle fiamme. Le fiamme emettono luce in un raggio di 3 metri e luce fioca per 6 metri. Le fiamme sono innocue per te e per l'arma. Quando colpisci con un attacco sferrato con l'arma infiammata, il bersaglio subisce 1d6 danni da fuoco aggiuntivi. Le fiamme perdurano fino a quando non userai due azioni per pronunciare la parola di comando di nuovo o fino a quando non lascerai cadere o rinfodererai l'arma.

Se stai indossando l'elmo e subisci danni da fuoco in seguito al Fallimento Critico di un Tiro Salvezza contro un incantesimo, l'elmo emette un fascio di luce tramite le gemme rimanenti. Ogni creatura entro 18 metri dall'elmo, a parte te, deve superare un Tiro Salvezza su Riflessi con DC 21 o venire colpita dal fascio, subendo danni di Luce uguali al numero di gemme nell'elmo x 5. Poi, le gemme e l'elmo vengono distrutti.

\oggettomagico{Elmo della Telepatia}

\textbf{Rarità:} Raro; \textbf{Costo:} 12000 mo

mentre indossi questo \textbf{elmo}, puoi usare due azioni per lanciare tramite esso l'incantesimo individuazione dei pensieri (DC del Tiro Salvezza 13). Finché mantieni la concentrazione sull'incantesimo, puoi usare due azioni per inviare un messaggio telepatico alla creatura su cui sei concentrato. Essa può replicare (usando due azioni per farlo) fino a quando continui a concentrarti su di lei.

Mentre ti concentri su di una creatura con individuazione dei pensieri, puoi usare due azioni per lanciare tramite l'elmo l'incantesimo suggestione (DC del Tiro Salvezza 13) su quella creatura. Una volta usata, la proprietà suggestione non potrà essere usata di nuovo fino alla prossima alba.

\oggettomagico{Energia Luminosa}

\textbf{Aura:} Trasmutazione forte; \textbf{Costo:} 45000 mo

\textbf{Requisiti:} Creare Oggetti Magici 3, Fiamma Perenne, Esplosione Solare

Quest'oggetto sembra l'impugnatura di una \textbf{spada} lunga, ma senza lama. Quando ne afferri l'impugnatura, puoi usare due azioni per far sì che una lama di pura luminescenza si formi, o faccia sparire la lama inserita nell'impugnatura.

Finché la spada esiste, questa spada lunga magica ha la proprietà Versatile. Se sei competente con le spade corte o le spade lunghe, sei competente anche con questa arma.

L'arma ha un bonus di +2 ai Tiri per Colpire e danno, infligge danni da Luce anziché danni taglienti. Quando colpisci con essa una creatura non morta, il bersaglio subisce 1d8 danni da Luce aggiuntivi.

La lama luminosa della spada emette luce intensa in un raggio di 3 metri e luce fioca per 6 metri. La luce è luce solare. Finché la lama è attiva, puoi usare 1 Azione per espandere o ridurre il raggio della luce intensa e fioca fino ad 1 metro ciascuno, per 3 round.

\oggettomagico{Faretra Efficiente}

\textbf{Rarità:} Raro; \textbf{Costo:} 2500 mo

ciascuno dei tre compartimenti della \textbf{faretra} è collegato a uno spazio extradimensionale che le permetta di trasportare numerosi oggetti non pesando mai più di 1 chilo.

Il compartimento più piccolo può contenere fino a 60 frecce, dardi od oggetti simili. Il compartimento mediano può contenere fino a 18 giavellotti od oggetti simili. Il compartimento più lungo può contenere fino a 6 oggetti lunghi, come archi, bastoni da combattimento o lance. Puoi estrarre qualsiasi oggetto contenuto nella faretra come se lo stessi prendendo da una normale faretra o fodero.

\oggettomagico{Fasce di Ferro del Vincolo}

\textbf{Rarità:} Raro; \textbf{Costo:} 5000 mo

questa \textbf{sfera} di ferro arrugginita misura 7,5 centimetri di diametro e pesa 500 grammi. Puoi usare due azioni per pronunciare una parola di comando e scagliare la sfera contro una creatura visibile di taglia Enorme o inferiore entro 18 metri da te. La sfera si muove nell'aria, aprendosi in un reticolato di fasce metalliche. Effettua un Tiro per Colpire a distanza, se colpisci, il bersaglio è intralciato fino a quando non effettuerai due azioni per pronunciare una parola di comando e liberarlo. Farlo, o mancare l'attacco, fa sì che le fasce si contraggano e ritornino a essere una sfera.

Una creatura, compresa quella intralciata, può usare due azioni per effettuare un Tiro Salvezza Tempra con Forza DC 25 per spezzare le fasce di ferro. Se la riesce, l'oggetto viene distrutto, e la creatura intralciata è libera. Se la prova fallisce, qualsiasi ulteriore tentativo effettuato dalla creatura fallisce automaticamente fino a quando non saranno trascorse 24 ore. Una volta che le fasce sono state usate non potranno più esserlo fino alla prossima alba.

\oggettomagico{Felpa}

\textbf{Aura:} Trasmutazione forte; \textbf{Costo:} 6000 mo

\textbf{Requisiti:} Creare Oggetti Magici 2; \textbf{Rarità:} Non Comune

Un \textbf{armatura} con la capacità Felpa conta per quanto concerne le penalità di indossare ed ingombro come un armatura leggera. Il personaggio riesce a muoversi quasi senza difficoltà con questa armatura.

\oggettomagico{Ferimento}

\textbf{Aura:} Necromantica moderata; \textbf{Costo:} 6000 mo

\textbf{Requisiti:} Creare Oggetti Magici 2, Contagio

Questa capacità può essere aggiunta solo ad \textbf{armi} da mischia da taglio o perforante. Un'arma da Ferimento infligge 1 danno da Sanguinamento quando colpisce una creatura. Danni multipli di quest'arma sommano il danno da Sanguinamento fino ad un massimo di 10.
Le creature sanguinanti subiscono il danno da Sanguinamento all'inizio del loro round.

Le creature immuni ai Colpi Critici sono immuni ai danni da Sanguinamento inflitti da quest'arma.

\oggettomagico{Filatterio contro i non morti}

\textbf{Rarità:} Comune; \textbf{Costo:} 1000 mo

questo \textbf{oggetto} sacro permette di usare l'Abilità Scacciare non morti con un bonus di +2 alla somma dei Tratti in comune con il Patrono.

\oggettomagico{Filatterio della giovinezza}

\textbf{Rarità:} Leggendario; \textbf{Costo:} 10000 mo

la striscia di \textbf{pergamena} di questo filatterio è di solito rinchiusa in un tubetto metallico da portare appeso al collo. Quando un personaggio lo indossa, la sua naturale velocità di invecchiamento diminuisce ed invecchia di 1 mese in un anno, mentre un eventuale invecchiamento magico è ridotto della metà.

\oggettomagico{Filtro d'Amore}

\textbf{Rarità:} Non Comune; \textbf{Costo:} 120 mo

resterai Affascinato per 1 ora dalla prima creatura che vedrai entro 10 minuti da quando avrai bevuto questo \textbf{filtro}. Se la creatura è di una specie o genere da cui sei normalmente attratto, finché sei Affascinato la considererai il tuo unico e grande amore.

\oggettomagico{Filtro Scopritesori}

\textbf{Rarità:} Raro; \textbf{Costo:} 500 mo

chi beve questa \textbf{pozione} può percepire entro 72 metri i tesori che contengono metalli preziosi o gemme, purché abbiano un valore di almeno 50 monete d'oro. Si può percepire la direzione del tesoro, ma non la sua esatta distanza. Nessuna barriera non magica può impedire di percepire i tesori, tranne una lastra di piombo.

\oggettomagico{Folgorante}

\textbf{Aura:} Lista Aria; \textbf{Costo:} 3000 mo

\textbf{Requisiti:} Creare Oggetti Magici 2, Fulmine

A comando, un'\textbf{arma} Folgorante viene avvolta da elettricità crepitante che infligge 1d6 danni addizionali da elettricità per ogni colpo andato a segno. Quest'elettricità non danneggia chi impugna l'arma. L'effetto rimane sempre attivo finché l'arma è sguainata, l'arma è immune ai danni da elettricità.

\oggettomagico{Armatura della Forma Eterea}

\textbf{Aura:} Trasmutazione forte; \textbf{Costo:} 24500 mo

\textbf{Requisiti:} Creare Oggetti Magici 3, Forma Eterea; \textbf{Rarità:} Rara

A comando, questa proprietà permette a chi indossa l'\textbf{armatura} di diventare Etereo (come per l'incantesimo Forma Eterea) una volta al giorno. Il personaggio può rimanere Etereo per quanto tempo desidera ma, una volta tornato alla normalità, per quel giorno non può più diventare Etereo.

\oggettomagico{Fortezza Istantanea}

\textbf{Rarità:} Molto Raro; \textbf{Costo:} 75000 mo

puoi usare due azioni per porre questo \textbf{cubo} di metallo di 2,5 centimetri di spigolo sul terreno e pronunciarne la parola di comando. Il cubo cresce rapidamente fino a diventare una fortezza che resterà fino a quando userai due azioni per pronunciare la parola di comando che la congeda, la quale funziona solo quando la fortezza è vuota.

La fortezza è una torre quadrata, 6 metri per lato e alta 9 metri, con feritoie su tutti i lati e spalti in cima. Il suo interno è diviso in due piani, con una scala che corre lungo una parete a congiungerli. La scala termina con una botola che si apre sul tetto. Quando viene attivata, la torre presenta una piccola porta sul lato rivolto verso di te. La porta si apre solo al tuo comando, che puoi pronunciare con due azioni. È immune all'incantesimo scassinare e magie simili, come quella del battaglio dell'apertura.

Ogni creatura nell'area in cui la fortezza compare deve effettuare un Tiro Salvezza su Riflessi con DC 17, subendo 10d10 danni contundenti se lo fallisce, o la metà di questi danni se lo riesce. In entrambi i casi, la creatura viene spinta in uno spazio fuori della fortezza ma in sua prossimità. Gli oggetti nell'area che non sono indossati o trasportati subiscono gli stessi danni e vengono spinti automaticamente.

La torre è fatta di adamantio, e la sua magia le impedisce di venir ribaltata. Il tetto, la porta e le mura hanno 100 Punti Ferita ognuno, immunità ai danni dalle armi non magiche a eccezione delle armi da assedio, e resistenza a tutti gli altri danni.

Solo l'incantesimo desiderio può riparare la fortezza. Ciascun lancio di desiderio fa sì che il tetto, la porta o una delle pareti recuperi 50 Punti Ferita.

\oggettomagico{Fortunata}

\textbf{Aura:} Invocazione molto forte; \textbf{Costo:} 30000 mo

\textbf{Requisiti:} Creare Oggetti Magici 4, Desiderio

Finché hai addosso la \textbf{arma} ricevi anche un bonus di +1 ai Tiri Salvezza.

- \emph{Fortuna}. Se hai addosso l'arma, puoi affidarti alla sua fortuna (non richiede azioni) per ripetere un Tiro per Colpire, prova di Competenza o Tiro Salvezza il cui risultato non ti soddisfa. Sei obbligato a usare il secondo risultato del dado. Questa proprietà non può essere usata di nuovo fino alla prossima alba.

- \emph{Desiderio}. Mentre la impugni, puoi usare due azioni per spendere 1 carica e lanciare tramite essa l'incantesimo desiderio. Questa proprietà non può essere usata di nuovo fino alla prossima alba. La spada ha 1 carica e perde questa proprietà se finisce le cariche.

\oggettomagico{Freccia localizzante}

\textbf{Rarità:} Non Comune; \textbf{Costo:} 400 mo

questa \textbf{freccia} può essere usata fino ad 8 volte al giorno. Essa viene lanciata per aria, e quando atterra indica una direzione o un luogo desiderati. Possibili indicazioni includono l'uscita o l'ingresso più vicini, scale, passaggi, caverne e simili aree.

\oggettomagico{Gelida}

\textbf{Aura:} Lista Acqua moderata; \textbf{Costo:} 3000 mo

\textbf{Requisiti:} Creare Oggetti Magici 2, Lista dell'Acqua

A comando, un'\textbf{arma} Gelida viene avvolta da un gelo terribile che infligge 1d6 danni da freddo per ogni colpo andato a segno. Questo freddo non danneggia chi impugna l'arma. L'effetto rimane sempre attivo finché l'arma è sguainata. L'arma è immune ai danni da freddo.

\oggettomagico{Gemma della Luminosita'}

\textbf{Aura:} Lista Fuoco; \textbf{Costo:} 5000 mo

\textbf{Requisiti:} Creare Oggetti Magici 1; \textbf{Rarità:} Comune

questo \textbf{prisma} ha 50 cariche. Mentre lo impugni, puoi usare due azioni per pronunciare una delle tre parole di comando per provocare uno dei seguenti effetti:

\begin{itemize}[leftmargin=*] \setlength{\itemsep}{0pt}
\item
La prima parola di comando fa sì che la gemma produca una luce intensa nel raggio di 9 metri e luce fioca per 18 metri. L'effetto consuma 1 carica. Dura finché non userai due azioni per ripetere la parola di comando o finché non impiegherai un'altra funzione della gemma, oppure sono passate 6 ore.

\item
La seconda parola di comando spende 1 carica e fa sì che la gemma proietti una fascio di luce luminoso contro una creatura visibile entro 18 metri da te. La creatura deve superare un Tiro Salvezza su Tempra con DC 17 o restare accecata per 1 minuto.

\item
La terza parola di comando spende 5 cariche e fa sì che la gemma irradi una luce accecante in un cono di 9 metri originante da te. Ogni creatura all'interno del cono deve effettuare un Tiro Salvezza come se fosse stata colpita dal fascio creato dalla seconda parola di comando.

\end{itemize}

\medskip

Quando tutte le cariche della gemma sono state spese, la gemma diventa un comune gioiello del valore di 50 mo.

\oggettomagico{Gemma della Vista}

\textbf{Rarità:} Molto Raro; \textbf{Costo:} 32000 mo

con due azioni, puoi pronunciare la parola di comando della \textbf{gemma} e spendere 1 carica. Per i successivi 10 minuti, quando guardi attraverso la gemma possiedi la visione del vero fino a 36 metri di distanza. La gemma ha 3 cariche, e recupera 1 carica spese ogni giorno all'alba.

\oggettomagico{Gemma Elementale}

\textbf{Aura:} Lista Elementi; \textbf{Costo:} 12000 mo

\textbf{Requisiti:} Creare Oggetti Magici 2, \hyperlink{Evoca Elementale}{Evoca Elementale}; \textbf{Rarità:} Non Comune

questa \textbf{gemma} contiene una scintilla di energia elementale. Quando usi due azioni per infrangere la gemma, questa evoca un elementale come se tu avessi lanciato l'incantesimo \hyperlink{Evoca Elementale}{Evoca Elementale}, e la magia della gemma svanisce. Il tipo di gemma determina l'elementale evocato dall'incantesimo.

\medskip

\noindent\begin{tabularx}{\linewidth}{ll}
	\toprule
\rowcolor{gray!20}\textbf{Gemma} &\textbf{Elementale evocato}\\
\toprule
Corindone rosso& Elementale del fuoco\\
\rowcolor{gray!20}Diamante giallo& Elementale della terra\\
Smeraldo &Elementale dell'acqua\\
\rowcolor{gray!20}Zaffiro blu&Elementale dell'aria
\end{tabularx}

\medskip

\oggettomagico{Gioiello Attiramostri}

questo \textbf{gioiello} magico è maledetto, il possessore attrae i mostri vaganti con il doppio della probabilità. I mostri, inoltre, lo inseguiranno al doppio della probabilità qualora egli fugga. Il gioiello non può essere abbandonato e riapparirà immediatamente sulla persona del possessore ogni volta che questi proverà a liberarsene. Solo Rimuovi Maledizione permetterà al possessore di lasciarsi indietro il gioiello.

\oggettomagico{Gloriosa}

\textbf{Aura:} Invocazione moderata; \textbf{Costo:} 6000 mo

\textbf{Requisiti:} Creare Oggetti Magici, Cecità/Sordità, Luce Diurna

Un'\textbf{arma} Gloriosa illumina con una luce abbagliante pari a quella un incantesimo Luce Diurna quando viene estratta. Chi la impugna non può sopprimere questa luce, anche se può essere soppressa temporaneamente da qualsiasi effetto che può sopprimere Luce Diurna.

Quando un'arma Gloriosa effettua un Colpo Critico, il bersaglio è Accecato fino all'inizio del round successivo del possessore (Tempra DC 14 nega). Solo un'arma da mischia può avere la capacità Gloriosa.

\oggettomagico{Guanti Afferra Proiettili}

\textbf{Rarità:} Non Comune; \textbf{Costo:} 3000 mo

questi \textbf{quanti} sembrano quasi fondersi con la tua pelle quando li indossi. Quando un attacco con arma a distanza ti colpisce mentre li indossi, puoi usare una Azione di Reazione per ridurre il danno di 1d10 + Destrezza, purché tu abbia una mano libera. Se riduci il danno a 0 ed il proiettile è piccolo a sufficienza da essere tenuto in mano, puoi afferrarlo.

\oggettomagico{Guanti del Nuoto e della Scalata}

\textbf{Rarità:} Non Comune; \textbf{Costo:} 2000 mo

mentre indossi entrambi questi \textbf{guanti}, la scalata e il nuoto non ti costano movimento aggiuntivo. Inoltre, hai un bonus di +1d6 alle prove di Costituzione e Saggezza effettuate mentre scali o nuoti.

\oggettomagico{Guanti del Potere orchesco}

\textbf{Rarità:} Raro; \textbf{Costo:} 9000 mo

mentre indossi queste \textbf{manopole} la tua Forza è 4. I guanti non hanno effetto se la tua Forza è già 4 o più.

\oggettomagico{Guanti della Destrezza}

\textbf{Costo:} 12000 mo

rari, questi \textbf{guanti} impartiscono al possessore una Destrezza minima di +2 e nel caso abbia già un punteggio di +2 questa aumenta di 1 (fino ad un massimo +4). Inoltre il possessore acquisisce +1d6 nella Competenza Mani di Fata

\oggettomagico{Guanti Maldestri}

questi \textbf{guanti} possono essere di morbido cuoio o pesante materiale protettivo adatto per l'uso con armature. Nel primo caso sembrano essere guanti della destrezza. Nel secondo caso essi sembrano essere guanti del potere orchesco. Ad ogni prova i guanti sembrano avere le funzioni di cui sopra fino a quando chi li indossa non è sotto attacco o in una situazione di vita o di morte. In quel momento la maledizione si attiva. Il personaggio diviene maldestro, con una probabilità del 50\% ad ogni round di lasciar cadere un oggetto che tiene nelle mani. I guanti riducono la Destrezza di 2 punti. Una volta che la maledizione è attiva, i guanti possono essere rimossi soltanto con un incantesimo Rimuovi Maledizione od un desiderio.

\oggettomagico{Guardiana}

\textbf{Aura:} Abiurazione moderata; \textbf{Costo:} 3000 mo

\textbf{Requisiti:} Creare Oggetti Magici 2, Resistenza

Questa capacità può essere aggiunta solo ad \textbf{armi} da mischia. Un'arma Guardiana permette a chi la impugna di trasferire, come Azione Immediata, una parte o tutto il bonus dell'arma alla Difesa di una creatura a distanza di mischia. Il bonus ceduto perdura fino all'inizio del round successivo.

\oggettomagico{Incensiere degli Elementali dell'Aria}

\textbf{Rarità:} Raro; \textbf{Costo:} 8000 mo

mentre l'\textbf{incenso} brucia all'interno di questo incensiere, puoi usare due azioni per pronunciare la parola di comando del braciere ed evocare un elementale dell'aria, come se avessi lanciato l'incantesimo \hyperlink{Evoca Elementale}{Evoca Elementale}. L'incensiere non può di nuovo essere usato a questo modo fino alla prossima alba. Questo incensiere largo 15 centimetri e alto 30 centimetri assomiglia a un calice dalla copertura decorata. Pesa 0,5 chili, ingombro L.

\oggettomagico{Incenso dell'Ossessione}

\textbf{Rarità:} Raro

raro, del tutto simile all'\textbf{incenso} della meditazione, questo incenso dà a chi lo usa anche l'impressione del suo effetto, ma sarà sotto Confusione per 24 ore se fallisce un Tiro Salvezza su Volontà DC 23.

\oggettomagico{Incenso della meditazione}

\textbf{Rarità:} Raro; \textbf{Costo:} 5000 mo

questo blocchetto d'\textbf{incenso} dal profumo dolce è indistinguibile dal normale incenso finché non viene acceso. Quando brucia, la sua particolare fragranza e il suo fumo chiaro sono riconoscibili con una prova di Arcana a DC 15. Dopo che un incantatore avrà trascorso 8 ore ripassando sul Tomo e meditando nei pressi di un blocchetto acceso il costo in Punti Magia degli incantesimi diminuirà di 1 punto per le successive 8 ore. Ogni blocchetto d'incenso brucia per 8 ore e l'effetto persiste per altre 8 ore. Di solito si trovano 2d4 blocchetti di incenso nella stessa custodia.

\oggettomagico{Invulnerabilita'}

\textbf{Aura:} Abiurazione forte; \textbf{Costo:} 15000 mo

\textbf{Requisiti:} Creare Oggetti Magici 4, Desiderio; \textbf{Rarità:} Leggendario

Questa \textbf{armatura} garantisce a chi la indossa una Riduzione del Danno di 5/magia.

\oggettomagico{Irrintracciabile}

\textbf{Aura:} Trasmutazione debole; \textbf{Costo:} 3750 mo

\textbf{Requisiti:} Creare Oggetti Magici, \hyperlink{Passare Senza Tracce}{Passare Senza Tracce}; \textbf{Rarità:} Rara

Un'\textbf{armatura} Irrintracciabile alleggerisce i passi di chi la indossa e ne camuffa l'aspetto. Le prove di Sopravvivenza per seguire le tracce del portatore subiscono penalità -4, e chi indossa l'armatura ottiene Bonus di +4 alle prove di Furtività. Soltanto le armature di cuoio o di pelle possono essere Irrintracciabili.

\oggettomagico{Ladra delle Nove Vite}

\textbf{Aura:} Necromantica forte; \textbf{Costo:} 25000 mo

\textbf{Requisiti:} Creare Oggetti Magici 2, Tocco Vampirico

Ottieni un bonus di +2 ai Tiri per Colpire e danno effettuati con quest'\textbf{arma} magica. Se ottieni un colpo critico contro una creatura che ha meno di 100 Punti Ferita, questa deve superare un tiro Salvezza su Tempra con DC 17 o venire immediatamente uccisa, mentre la spada ne risucchia la forza vitale dal corpo (i costrutti e i non morti sono immuni a questa proprietà).

La spada ha 1d8 + 1 cariche e perde 1 carica quando una creatura viene uccisa. Quando la spada non ha più cariche, perde questa proprietà.

\oggettomagico{Lanterna della Rivelazione}

\textbf{Rarità:} Non Comune; \textbf{Costo:} 5000 mo

mentre è accesa, questa \textbf{lanterna} brucia per 6 ore con 1 fiasca d'olio, irradiando luce intensa in un raggio di 9 metri e luce fioca per 18 metri. Le creature e gli oggetti invisibili sono resi visibili mentre si trovano sotto la luce intensa della lanterna.

Puoi usare due azioni per abbassare la copertura, riducendo la luce a fioca con un raggio di 1 metro.

\oggettomagico{Lingua di fuoco}

\textbf{Aura:} Lista di Fuoco; \textbf{Costo:} 3000 mo

\textbf{Requisiti:} Creare Oggetti Magici 2, Palla di Fuoco

A comando, un'\textbf{arma} fiammeggiante viene avvolta da fiamme che infliggono 1d6 danni da fuoco per ogni colpo andato a segno. Questo fuoco non danneggia chi impugna l'arma. L'effetto rimane attivo finché non viene disattivato con un altro comando. L'arma è immune ai danni da fuoco.

\oggettomagico{Mantella del Ciarlatano}

\textbf{Rarità:} Raro; \textbf{Costo:} 8000 mo

mentre indossi questa \textbf{mantella} che odora lievemente di zolfo, puoi usarla per lanciare l'incantesimo \emph{porta dimensionale} con due azioni. La proprietà di questa mantella non può essere usata di nuovo fino all'alba. Quando scompari, ti lasci alle spalle una nube di fumo e riappari alla tua destinazione all'interno di una simile nube di fumo. Questo fumo oscura leggermente lo spazio che hai lasciato e quello dove riappari e si dissipa alla fine del tuo prossimo round. Un vento leggero o più forte disperde il fumo.

\oggettomagico{Mantello degli Elfi}

\textbf{Rarità:} Non Comune; \textbf{Costo:} 5000 mo

mentre indossi questo \textbf{Mantello} tirando su il cappuccio, le prove di Consapevolezza effettuate per notarti hanno -1d6, e hai +1d6 alle prove di Furtività effettuate per nasconderti. Tirare su o giù il cappuccio richiede due Azioni.

\oggettomagico{Mantello del Pipistrello}

\textbf{Rarità:} Raro; \textbf{Costo:} 6000 mo

in aree di luce fioca o oscurità, puoi afferrare i bordi della \textbf{Mantello} con entrambe le mani e usarla per muoverti a velocità di volo 12 metri. Se dovessi smettere di tenere i bordi del Mantello mentre voli a questo modo, perdi la tua velocità di volo. Mentre indossi la Mantello in un'area di luce fioca o oscurità, puoi usare due Azioni per lanciare \emph{metamorfosi} su di te, trasformandoti in un pipistrello. Quando sei in forma di pipistrello, mantieni i tuoi punteggi di Intelligenza, Saggezza e Carisma. La Mantello non può essere impiegata di nuovo in questo modo fino alla prossima alba.

\oggettomagico{Mantello dell'Aracnide}

\textbf{Rarità:} Molto Raro; \textbf{Costo:} 8000 mo

mentre indossi questo elegante \textbf{abito} di seta nera intessuto con fili d'argento, ottieni i seguenti benefici:

\medskip

\begin{itemize}[leftmargin=*] \setlength{\itemsep}{0pt}
\item
Hai resistenza ai danni da veleno.
\item
Hai velocità di scalata pari alla tua velocità di movimento.
\item
Puoi muoverti verso l'alto, il basso e lungo superfici verticali e a testa in giù sui soffitti, tenendo le mani libere.
\item
Non puoi essere catturato da alcuna sorta di ragnatela e ti muovi attraverso le ragnatele come fossero terreno difficile.
\item
Puoi usare due azioni per lanciare l'incantesimo \emph{ragnatela} (DC del Tiro Salvezza 15). La ragnatela creata dall'incantesimo riempie il doppio della sua normale area. Una volta usata, questa proprietà della Mantello non può essere usata di nuovo fino alla prossima alba.
\end{itemize}

\oggettomagico{Mantello della Manta}

\textbf{Rarità:} Non Comune; \textbf{Costo:} 6000 mo

mentre indossi questa \textbf{Mantello} con il cappuccio tirato su, puoi respirare sott'acqua e hai velocità di nuoto 18 metri. Tirare su o giù il cappuccio richiede 2 Azioni.

\oggettomagico{Mantello della Resistenza agli Incantesimi}

\textbf{Rarità:} Non Comune; \textbf{Costo:} 3000 mo

non comune, mentre indossi questo \textbf{mantello}, hai +1 ai Tiri Salvezza contro incantesimi.

\oggettomagico{Mantello della velenosita'}

\textbf{Rarità:} Raro; \textbf{Costo:} 4000 mo

questo \textbf{mantello} è di solito fatto di lana, sebbene possa essere anche di pelle. L'indumento può essere manipolato senza pericolo, ma appena viene indossato causa 5d6 di danno da veleno. Ogni round successivo può essere fatto un Tiro Salvezza su Tempra DC 21 per ridurre ad 1d6 il danno. Il mantello può essere rimosso soltanto con un incantesimo \emph{Rimuovi Maledizione} o \emph{Desiderio}.

\oggettomagico{Mantello di Distorsione}

\textbf{Rarità:} Raro; \textbf{Costo:} 60000 mo

mentre indossi questo \textbf{mantello}, esso proietta un'illusione che ti fa apparire come se stessi in un punto vicino alla tua reale posizione, facendo sì che tutte le creature abbiano -1d6 ai Tiri per Colpire contro di te. Se subisci danni, la proprietà cessa di funzionare fino all'inizio del tuo prossimo round. Questa proprietà è soppressa mentre sei inabile, intralciato o altrimenti impossibilitato a muoverti.

\oggettomagico{Mantello di Protezione}

\textbf{Rarità:} Molto Raro; \textbf{Costo:} 3500 mo

mentre indossi questa \textbf{mantello}, ottieni un bonus di +1 (non comune, 3500 mo),+2 (raro, 6000 mo),+3 (molto raro, 15000 mo) alla Difesa e ai Tiri Salvezza.

\oggettomagico{Manuale dei Golem}

\textbf{Rarità:} Molto Raro; \textbf{Costo:} 10000 mo

questo \textbf{tomo} contiene le informazioni e incantamenti necessari a costruire un tipo particolare di golem. Il Narratore sceglie il tipo di golem che è possibile costruire o lo determina casualmente. Per decifrare e usare il manuale devi avere almeno Competenza Magica 10. Una creatura che non possa usare il manuale dei golem e provi a leggerlo, subisce 6d6 danni da forza.

Per creare un golem, devi trascorrere il tempo sopra indicato, lavorando senza interruzione con il manuale a disposizione e riposando per non più di 8 ore al giorno. Devi anche pagare il costo specificato per acquistare i materiali necessari.

Una volta finito di creare il golem, il libro viene consumato da fiamme arcane. Il golem si anima quando le ceneri del manuale saranno sparse su di esso. Sarà sotto il tuo controllo e comprende e obbedisce gli ordini pronunciati da te.

\medskip

\noindent\begin{tabularx}{\linewidth}{llll}
	\toprule
\rowcolor{gray!20}3d6 &Golem &Tempo &Costo\\
\toprule
3-4 &Argilla &30 giorni &65000 mo\\
\rowcolor{gray!20}5-16 &Carne &60 giorni& 50000 mo\\
17 &Ferro &120 giorni &100000 mo\\
\rowcolor{gray!20}18 &Pietra& 90 giorni &80000 mo
\end{tabularx}

\medskip

\oggettomagico{Manuale dell'Esercizio fisico}

\textbf{Rarità:} Molto Raro; \textbf{Costo:} 15000 mo

questo \textbf{tomo} funziona esattamente come il manuale della salute, ma conferisce al lettore un punto di Forza. La \emph{versione maledetta} di questo manuale pur sembrando assolutamente identico a quello originale al termine delle 4 settimana fa perdere un punto di Forza.

\oggettomagico{Manuale della Buona salute}

\textbf{Rarità:} Molto Raro; \textbf{Costo:} 15000 mo

questo \textbf{tomo} contiene istruzioni per rafforzare il corpo e la salute. Per leggere il libro occorrono 24 ore in un minimo di 3 giorni. Le sue istruzioni andranno seguite per 4 settimane, al termine delle quali il lettore guadagnerà permanentemente un punto di Costituzione. Una volta letto il manuale perde la sua magia, per recuperarla dopo un secolo. La \emph{versione maledetta} di questo manuale pur sembrando assolutamente identico a quello originale al termine delle 4 settimana fa perdere un punto di Costituzione.

\oggettomagico{Manuale della Velocita' di azione}

\textbf{Rarità:} Molto Raro; \textbf{Costo:} 15000 mo

questo \textbf{tomo} contiene esercizi per l'equilibrio e la coordinazione. Funziona come un manuale della buona salute, ma fa ottenere un punto di Destrezza. La \emph{versione maledetta} di questo manuale pur sembrando assolutamente identico a quello originale al termina delle 4 settimana fa perdere un punto di Destrezza.

\oggettomagico{Marina}

\textbf{Aura:} Lista Acqua moderata; \textbf{Costo:} 3000 mo

\textbf{Requisiti:} Creare Oggetti Magici 2, Libertà di Movimento,

Questa capacità speciale può essere aggiunta solo ad \textbf{armi} da mischia. Un'arma Marina funziona tranquillamente negli ambienti acquatici. Con l'arma in mano, chi la impugna ottiene un bonus alle prove di Nuotare pari al doppio del bonus dell'arma.

\oggettomagico{Mazza della Punizione}

\textbf{Aura:} Invocazione forte; \textbf{Costo:} 7000 mo

\textbf{Requisiti:} Creare Oggetti Magici 2

Ottieni un ulteriore +3 al colpire e danno quando usi quest'\textbf{arma} per attaccare un costrutto.

Quando ottieni un critico al Tiro per Colpire con quest'arma, il bersaglio subisce un critico aggiuntivo, o 2 Tiri Critici aggiuntivi se è un costrutto. Se, dopo aver subito questi danni, a un costrutto restano 25 Punti Ferita o meno, viene distrutto.

\oggettomagico{Mazzo delle Illusioni}

\textbf{Rarità:} Non Comune; \textbf{Costo:} 6500 mo

questa scatola contiene un \textbf{set di carte}. Un mazzo completo contiene 34 carte, ognuna raffigurante una creatura diversa. Le creature rappresentate vengono lasciate alla discrezionalità del Narratore. Di solito i mazzi trovati in giro sono privi di 3d6-3 carte.

La magia del mazzo funziona solo se le carte vengono pescate a caso (potete usare un mazzo di normali carte da gioco modificato per simulare il mazzo delle illusioni). Puoi usare due azioni per pescare una carta dal mazzo e scagliarla in un punto sul terreno a 9 metri da te.

L'illusione di una o più creature si forma sopra la carta lanciata e rimane finché non viene dissolta. La creatura illusoria sembra reale, della taglia appropriata, e si comporta come fosse una vera creatura, eccetto che non può recare danni. Finché resti entro 36 metri dalla creatura illusoria e puoi vederla, puoi usare due azioni per muoverla magicamente in qualsiasi punto entro 9 metri dalla carta. Qualsiasi interazione fisica con la creatura illusoria la rivela come illusione, dato che gli oggetti le passano attraverso. Qualcuno che usi due azioni per ispezionare visivamente la creatura, la identifica come illusoria superando un Tiro Salvezza su Volontà con Intelligenza con DC 17. La creatura le apparirà quindi trasparente.
L'illusione permane finché la carta non viene mossa o l'illusione dissolta. Quando l'illusione termina, l'immagine sulla carta scompare e quella carta non potrà più essere usata.

\medskip

\noindent\begin{tabularx}{\linewidth}{ll}
	\toprule
\rowcolor{gray!20}\textbf{Carta da Gioco} & \textbf{Illusione} \\
\toprule
Asso di cuori & Drago rosso \\
\rowcolor{gray!20}Re di cuori & Cavaliere e quattro guardie \\
Regina di cuori & Succube o incubo \\
\rowcolor{gray!20}Fante di cuori & Druido \\
Dieci di cuori & Gigante delle nuvole \\
\rowcolor{gray!20}Nove di cuori & Ettin \\
Otto di cuori & Bugbear \\
\rowcolor{gray!20}Due di cuori & Goblin \\
Asso di picche & Lich \\
\rowcolor{gray!20}Re di picche & Sacerdote e due accoliti \\
Regina di picche & Medusa \\
\rowcolor{gray!20}Fante di picche & Veterano \\
Dieci di picche & Gigante del gelo \\
\rowcolor{gray!20}Nove di picche & Troll \\
Otto di picche & Hobgoblin \\
\rowcolor{gray!20}Due di picche & Goblin \\
Asso di quadri & Beholder \\
\rowcolor{gray!20}Re di quadri & Arcimago e mago apprendista \\
Regina di quadri & Megera notturna \\
\rowcolor{gray!20}Fante di quadri & Assassino \\
Dieci di quadri & Gigante del fuoco \\
\rowcolor{gray!20}Nove di quadri & Oni \\
Otto di quadri & Gnoll \\
\rowcolor{gray!20}Due di quadri & Coboldo \\
Asso di fiori & Golem di ferro \\
\rowcolor{gray!20}Re di fiori & Capitano bandito e tre banditi \\
Regina di fiori & Erinni \\
\rowcolor{gray!20}Fante di fiori & Berserker \\
Dieci di fiori & Gigante delle Colline \\
\rowcolor{gray!20}Nove di fiori & Ogre \\
Otto di fiori & Orco \\
\rowcolor{gray!20}Due di fiori & Coboldo \\
Jolly (2) & Tu (il proprietario del mazzo) \\
\end{tabularx}

\oggettomagico{Mazzo delle Meraviglie}

\textbf{Rarità:} Leggendario; \textbf{Costo:} 100000 mo

di solito lo si trova in un \textbf{borsello} o una \textbf{scatola}, che contiene delle carte fatte d'avorio o vello. La maggior parte di questi mazzi (il 75\%) ha solo tredici carte, mentre i restanti mazzi ne hanno ventidue.

Prima di pescare una carta, devi dichiarare quante carte intendi pescare e poi pescarle casualmente (puoi usare un mazzo di carte da gioco modificato per simulare il mazzo). Qualsiasi carta pescata in eccesso di questo numero non ha effetto. Altrimenti, appena peschi una carta dal mazzo, la sua magia ha effetto.

Devi pescare ciascuna carta entro 1 ora dalla pescata precedente. Se non peschi il numero scelto di carte, il numero di carte rimanenti uscirà fuori dal mazzo spontaneamente e avrà effetto in contemporanea. Una volta estratta una carta, questa svanirà dall'esistenza. A meno che la carta non sia il Matto o il Buffone, la carta ricompare nel mazzo, rendendo possibile pescare due volte la stessa carta.

\medskip

\noindent\begin{tabularx}{\linewidth}{ll}
	\toprule
\rowcolor{gray!20}\textbf{Carta da Gioco} & \textbf{Carta}\\
\toprule
	Asso di quadri & Visir* \\
 \rowcolor{gray!20}Re di quadri & Sole \\
	Regina di quadri & Luna \\
	\end{tabularx}
\noindent\begin{tabularx}{\linewidth}{ll}
	\toprule
\rowcolor{gray!20}\textbf{Carta da Gioco} & \textbf{Carta}\\
	\toprule
	Fante di quadri & Stella \\
 \rowcolor{gray!20}Due di quadri & Cometa* \\
	Asso di fiori & Artigli* \\
 \rowcolor{gray!20}Re di fiori & Vuoto \\
	Regina di fiori & Fiamme \\
 \rowcolor{gray!20}Fante di fiori & Teschio \\
	Due di fiori & Idiota \\
 \rowcolor{gray!20}Jolly & Giullare \\
	Asso di cuori & Destino* \\
 \rowcolor{gray!20}Re di cuori & Trono \\
	Regina di cuori & Chiave \\
 \rowcolor{gray!20}Fante di cuori & Cavaliere \\
	Due di cuori & Gemma* \\
 \rowcolor{gray!20}Asso di picche & Prigione* \\
	Re di picche & Rovina \\
 \rowcolor{gray!20}Regina di picche & Eurialo \\
	Due di picche & Bilancia* \\
 \rowcolor{gray!20}Jolly & Matto* \\
	\end{tabularx}

\medskip

* Solo in mazzo da 22 \textbf{carte}

\noindent\begin{itemize}[leftmargin=*] \setlength{\itemsep}{0pt}
\item \textbf{Artigli}: Tutti gli oggetti magici indossati o trasportati dal personaggio si disintegrano. Gli artefatti posseduti dal personaggio non si disintegrano ma svaniscono.
\item \textbf{Bilancia}: La mente del personaggio è sottoposta a una sofferta alterazione che modifica il suoi tratti.
\item \textbf{Cavaliere}: Il personaggio ottiene i servigi di un combattente con CA 4 che appare in un spazio a sua scelta entro 9 metri da lui. Il guerriero è della stessa razza del personaggio e lo serve fedelmente fino alla morte, convinto che sia stato il destino a condurlo da lui. Il personaggio ha il controllo del guerriero.
\item \textbf{Chiave}: Nella mano del personaggio appare un’arma magica rara, molto rara o leggendaria in cui è competente. L’arma viene scelta dal DM.
\item \textbf{Cometa}: Se il personaggio sconfigge da solo il successivo mostro o gruppo di mostri ostili che incontra, ottiene abbastanza punti esperienza da aumentare di livello. Altrimenti, la carta non ha alcun effetto.
\item \textbf{Destino}: La trama della realtà si disfa e si ricompone, permettendo al personaggio di evitare o annullare un evento come se non fosse mai avvenuto. Il personaggio può usare la magia della carta non appena la pesca o in qualsiasi altro momento prima della sua morte.
\item \textbf{Eurialo}: Il volto di medusa raffigurato sulla carta maledice il personaggio. Finché è maledetto in questo modo, il personaggio subisce una penalità di 2 ai tiri salvezza. Solo un Patrono o la magia della carta Destino può porre fine a questa maledizione.
\item \textbf{Fiamme}: Il personaggio si inimica un potente diavolo, che desidera condurlo alla rovina e corromperà la sua vita in ogni modo. Questa ostilità dura finché uno dei due tra il personaggio e il diavolo muore.
\item \textbf{Gemma}: Ai piedi del personaggio appaiono venticinque gemme del valore di 2000 mo ognuna o cinquanta gemme del valore di 1000 mo ognuna.
\item \textbf{Giullare}: Il personaggio ottiene abbastanza Punti Esperienza per passare di livello oppure può pescare un altra carta oltre al numero dichiarato di carte da pescare.
\item \textbf{Idiota}: Il punteggio di Intelligenza del personaggio si riduce permanentemente di 1d2 + 1 (fino a un punteggio minimo di -4). Il personaggio può pescare un’altra carta oltre al numero dichiarato di carte da pescare.
\item \textbf{Ladro}: Un PNG a scelta del Narratore diventa ostile nei confronti del personaggio. L'identità del nuovo nemico è ignota al personaggio finché lo stesso PNG o qualcun altro non la rivela. Solo un incantesimo desiderio o un intervento divino possono annullare l’ostilità del PNG.
\item \textbf{Luna}: Il personaggio ottiene la capacità di lanciare l'incantesimo desiderio per 1d3 volte.
\item \textbf{Matta}: Il personaggio perde un livello, scarta questa carta e ne pesca un’altra dal mazzo; la pesca di entrambe queste carte conta solo come una pesca ai fini del numero dichiarato di carte da pescare.
\item \textbf{Prigione}: Il personaggio scompare e resta intrappolato in stato di animazione sospesa all’interno di una sfera extradimensionale. Tutto ciò che indossava e trasportava resta nello spazio che occupava al momento della sua scomparsa. Il personaggio resta imprigionato finché non viene trovato e rimosso dalla sfera. Non è possibile localizzare il personaggio attraverso alcuna magia di divinazione, ma un incantesimo desiderio può rivelare l’ubicazione della prigione. Il personaggio non può pescare altre carte.
\item \textbf{Rovina}: Ogni forma di ricchezza trasportata o posseduta dal personaggio, ad eccezione degli oggetti magici, va perduta. Le proprietà trasportabili svaniscono. Le attività commerciali, gli edifici e le terre possedute vanno perdute nel modo che richiede l'alterazione della realtà più limitata. Tutti i documenti che potrebbero dimostrare il possesso delle proprietà perdute a causa della carta svaniscono a loro volta.
\item \textbf{Sole}: Ottieni abbastanza Punti Esperienza per passare di livello e un oggetto meraviglioso, determinato dal Narratore, compare tra le tue mani.
\item \textbf{Stella}: Un punteggio di caratteristica del personaggio aumenta di 2. Il nuovo punteggio può essere superiore a 5, ma non a 7.
\item \textbf{Teschio}: Il personaggio evoca una manifestazione della morte: uno spettrale scheletro umanoide avvolto in una veste nera stracciata e armato di una falce spettrale. Lo scheletro appare in uno spazio a scelta del Narratore entro 3 metri dal personaggio e lo attacca, intimando agli altri presenti di non immischiarsi nel combattimento. La manifestazione combatte finché non uccide il personaggio o non scende a 0 punti ferita nel qual caso scompare. Se qualcuno tenta di aiutare il personaggio, evoca a sua volta un altra manifestazione della morte. Una creatura uccisa da una manifestazione della morte non può essere riportata in vita.
\item \textbf{Trono}: Il personaggio ottiene competenza nella competenza Ingannare e il suo bonus alle prove di Ingannare raddoppia. Inoltre, il personaggio diventa il legittimo proprietario di un piccolo castello situato da qualche parte nel mondo. Tuttavia, il castello è infestato da mostri e il personaggio dovrà scacciarli prima di poterlo rivendicare come suo.
\item \textbf{Visir}: In qualunque momento desideri, entro un anno da quando pesca questa carta, il personaggio può porre una domanda mentre medita e ottenere mentalmente una risposta veritiera a quella domanda. Oltre alle informazioni la risposta lo aiuta a comprendere come risolvere.
\item \textbf{Vuoto}: Questa carta nera è foriera di sventura. L'anima del personaggio viene strappata dal corpo e rinchiusa in un oggetto situato in un luogo scelto dal Narratore. Tale luogo è protetto da uno o più potenti guardiani. Finché l’anima del personaggio è intrappolata in questo modo, il suo corpo è incapacitato. Un incantesimo desiderio non può riportare l’anima del personaggio nel corpo, ma può rivelare il luogo in cui si trova l’oggetto che la custodisce. Il personaggio non pesca altre carte.
\end{itemize}

\emph{Manifestazione della Morte}

Non morto media, neutrale malvagio

\textbf{Forza} +3

\textbf{Destrezza} +3

\textbf{Intelligenza} +3

\textbf{Saggezza} +3

\textbf{Carisma} +3

\textbf{Difesa} 23

\textbf{Punti Ferita} metà dei Punti Ferita del suo evocatore

\textbf{Movimento}: Velocità 18 m, volo 18 m, Fluttuare

\textbf{Immunità ai Danni}: Vuoto, veleno

\textbf{Immunità alle Condizioni}: Affascinato, avvelenato, paralizzato, pietrificato, spaventato, svenuto

\textbf{Sensi}: scurovisione 18 m, visione del vero 18 m

\textbf{Linguaggi}: tutti i linguaggi conosciuti dal suo evocatore

\textbf{Sfida} (0 PX)

\textbf{Movimento Incorporeo}. La Manifestazione può attraversare creature e oggetti come fossero terreno difficile. Subisce 5 (1d10) danni da forza se termina il proprio round all'interno di un oggetto.

\textbf{Immunità allo Scacciare}. La Manifestazione è immune agli effetti che scacciano i non morti.

\textbf{Azioni}

\textbf{Falce Mietitrice}. La Manifestazione affonda la sua falce spettrale in una creatura entro 1 metro da esso, infliggendo 7 (1d8 + 3) danni perforanti più 4 (1d8) danni da Vuoto. Non può mancare.

\smallskip*\hypertarget{Miniatura dal Potere Meraviglioso}{} \textbf{Miniatura dal Potere Meraviglioso} rarità variabile, costo variabile, una \textbf{miniatura} dal potere meraviglioso è una statuetta di una bestia, piccola a sufficienza da entrare in tasca. Se usi due azioni per pronunciare una parola di comando e lanciare la miniatura in un punto del terreno entro 18 metri da te, la miniatura diventa una creatura vivente. Se lo spazio in cui la creatura dovesse apparire è occupato da un'altra creatura od oggetto, o se non c'è spazio sufficiente per la creatura, la miniatura non si trasforma.

La creatura è amichevole nei confronti tuoi e dei tuoi compagni. Comprende i tuoi linguaggi e obbedisce agli ordini impartitele. Se non le impartisci ordini, la creatura si difende ma non effettua altre azioni. Vedi il Bestiario per le altre statistiche della creatura.

La creatura resta per la durata specificata per ciascuna miniatura. Al termine della durata, la creatura ritorna alla sua forma di miniatura. Si trasforma anticipatamente se scende a 0 Punti Ferita o se usi due azioni per pronunciare la parola di comando di nuovo mentre la tocchi. Dopo che la creatura è tornata a essere una miniatura, le sue proprietà non possono più essere usate fino a quando non sarà trascorso un certo ammontare di tempo, come specificato nella descrizione della miniatura.

\emph{Cane di Onice} (Raro, 500 mo). Questa statuetta di onice raffigura un cane. Può diventare un mastino per un massimo di 6 ore. Il mastino ha Intelligenza -2 e può parlare Comune. Inoltre ha scurovisione 18 metri e può vedere le creature e gli oggetti invisibili entro quella distanza. Una volta usata, non può essere usata di nuovo prima che siano passati 7 giorni.

\emph{Caprone d'Avorio (Raro. 1000 mo)}. Queste statuette d'avorio di caproni sono sempre create in set da tre. Ogni caprone ha un aspetto unico e funziona in modo diverso dagli altri. Le loro proprietà sono le seguenti:

Il caprone del terrore può diventare un caprone gigante per un massimo di 3 ore. Il caprone non può attaccare, ma puoi rimuoverne i corni e usarli come armi. Un corno diventa una lancia da cavaliere +1 mentre l'altro diventa una spada lunga +2.

Rimuovere un corno richiede due azioni, e le armi scompaiono e i corni ricompaiono quando il caprone torna alla sua forma di miniatura. Inoltre, il caprone irradia un'aura di terrore con raggio 9 metri finché lo cavalchi. Qualsiasi creatura a te ostile che inizi il proprio round all'interno dell'aura deve superare un Tiro Salvezza su Volontà con DC 17 o restare
spaventata dal caprone per 1 minuto, o finché il caprone non torna alla forma di miniatura. La creatura spaventata può ripetere il Tiro Salvezza al termine di ciascun suo round, terminando l'effetto se lo supera. Una volta che ha riuscito il Tiro Salvezza contro questo effetto, una creatura è immune all'aura del caprone per le successive 24 ore. Una volta usata, la miniatura non può essere usato di nuovo prima che siano passati 15 giorni.

Il caprone del travaglio può diventare un caprone gigante per un massimo di 3 ore. Una volta usato, non può essere usato di nuovo prima che siano passati 30 giorni.
Il caprone del viaggio può diventare un caprone Grande con le stesse statistiche di un Saurovallo da Galoppo. Ha 24 cariche, e ciascuna ora o porzione di essa che trascorre in forma di bestia costa 1 carica. Finché ha cariche, lo puoi usare quanto ti pare. Una volta terminate le cariche, ritorna a essere una miniatura e non può essere usato di nuovo prima che siano passati 7 giorni, allorché avrà recuperato tutte le sue cariche.

\emph{Corvo d'Argento} (Non Comune, 300 mo). Questa statuetta d'argento raffigura un corvo. Può diventare un corvo per un massimo di 6 ore. Una volta usata, non può essere usata di nuovo prima che siano passati 2 giorni. Mentre è in forma di corvo, la miniatura ti permette di lanciare a volontà l'incantesimo messaggero animale su di essa.

\emph{Destriero di Ossidiana} (Molto Raro, 1000 mo). Questa statuetta di ossidiana liscia diventa un incubo per un massimo di 24 ore. L'incubo combatte solo per difendersi. Una volta usata, non può essere usata di nuovo prima che siano passati 5 giorni.

\emph{Elefante di Marmo} (Raro, 1500 mo). Questa statuetta di marmo è larga e alta circa 10 centimetri. Può diventare un elefante per un massimo di 24 ore. Una volta usata, non può essere usata di nuovo prima che siano passati 7 giorni.

\emph{Grifone di Bronzo} (Raro, 1250 mo). Questa statuetta di bronzo raffigura un grifone rampante. Può diventare un grifone per un massimo di 6 ore. Una volta usata, non può essere usata di nuovo prima che siano passati 5 giorni.

\emph{Gufo Serpentino} (Raro, 400 mo). Questa statuetta serpentina di un gufo può diventare un gufo gigante per un massimo di 8 ore. Una volta usata, non può essere usata di nuovo prima che siano passati 2 giorni. Se vi trovate sullo stesso piano di esistenza, il gufo può comunicare telepaticamente con te a qualsiasi distanza.

\emph{Leoni d'Oro} (Raro, 800 mo). Queste statuette d'oro di leoni sono sempre create a coppie. Puoi usare una o entrambe le miniature contemporaneamente. Ciascuna può diventare un leone per un massimo di 1 ora. Una volta usato uno dei leoni, questi non può essere usato di nuovo prima che siano passati 7 giorni.

\oggettomagico{Medaglione dei Pensieri}

\textbf{Rarità:} Non Comune; \textbf{Costo:} 3000 mo

mentre indossi questo \textbf{medaglione}, puoi usare due azioni e spendere 1 carica per lanciare tramite esso l'incantesimo individuazione dei pensieri (DC del Tiro Salvezza 15). Il medaglione ha 3 cariche, e recupera 1 carica spese ogni giorno all'alba.

\oggettomagico{Medaglione della Caduta piuma}

\textbf{Rarità:} Non Comune; \textbf{Costo:} 400 mo

questo \textbf{medaglione} attiva in automatico l'incantesimo Caduta Piuma quando il possessore cade da una altezza di 2 metri o più.

\oggettomagico{Mithral}

\textbf{Rarità:} Non Comune; \textbf{Costo:} 800 mo; \textbf{Rarità:} Rara

\textbf{Armatura} media o pesante, ma non di pelle +800 mo oltre il prezzo base dell'armatura. Il mithral è un metallo leggero e flessibile. Un giaco di maglia o un pettorale di mithral possono essere indossati sotto abiti normali. Riduce di 1 la categoria di peso per determinare penalità alle prove di competenza di Base e Magia.

\oggettomagico{Munizione dell'Uccisione}

\textbf{Rarità:} Molto Raro; \textbf{Costo:} 700 mo

se una creatura appartenente al tipo, razza o gruppo a cui la \textbf{freccia} dell'uccisione è associata subisce danni dalla freccia, la creatura deve effettuare un tiro Salvezza su Tempra con DC 21, subendo 6d10 danni perforanti aggiuntivi se lo fallisce, o la metà di questi danni se lo riesce.

Una volta che la freccia dell'uccisione è stata scoccata diventa una freccia non magica.

\oggettomagico{Munizione Fantasma}

\textbf{Aura:} Trasmutazione moderata

\textbf{Requisiti:} Creare Oggetti Magici 2, Disintegrazione, Riparare

Questa capacità può essere conferita solo alle \textbf{munizioni}. Una munizione con questa capacità speciale si dissolve 1 round dopo essere stata scagliata. In aggiunta, se il Proiettile colpisce un bersaglio, la ferita causata si richiude non appena la munizione si disintegra. Il Proiettile infligge danni normalmente, ma non lascia alcuna traccia visibile.

Il prezzo si riferisce a 50 Munizioni Fantasma.

\oggettomagico{Munizioni Infinite}

\textbf{Aura:} Evocazione moderata; \textbf{Costo:} 6000 mo

\textbf{Requisiti:} Creare Oggetti Magici 2, Creazione

Solo \textbf{archi} e \textbf{balestre} possono essere rese armi dalle Munizioni Infinite. Ogni volta che un'arma dalle Munizioni Infinite viene incoccata, una singola freccia o quadrello non magico viene creato spontaneamente dalla sua magia, quindi chi lo impugna non ha mai bisogno di caricare l'arma con delle munizioni.

A differenza di una normale munizione per arco o balestra, queste frecce e quadrelli vengono sempre distrutti quando scagliati.

\oggettomagico{Mutaforma}

\textbf{Aura:} Illusione moderata; \textbf{Costo:} 2000 mo

\textbf{Requisiti:} Creare Oggetti Magici 2, Arma Magica, Camuffare Se Stesso

Ad un'\textbf{arma} Mutaforma può essere comandato di mutare la sua forma e apparire come un altro oggetto di taglia simile. L'arma conserva tutte le sue proprietà (compreso il peso) anche quando è mascherata, ma non irradia magia. Solo Visione del Vero o altre magie simili rivelano la reale natura dell'arma trasformata. Dopo che un'arma Mutaforma è stata usata per attaccare, questa capacità speciale viene soppressa per 1 minuto.

\oggettomagico{Occhi Affascinanti}

\textbf{Rarità:} Non Comune; \textbf{Costo:} 3000 mo

mentre indossi queste \textbf{lenti} di cristallo davanti agli occhi, puoi spendere 1 carica con due azioni per lanciare l'incantesimo \emph{\hyperlink{Charme su Persone}{Charme su Persone}} (DC del Tiro Salvezza 15) su di un umanoide entro 9 metri da te, purché tu e il bersaglio vi possiate vedere. Le lenti hanno 3 cariche e recuperano 1 carica di quelle spese ogni giorno all'alba.

\oggettomagico{Occhi dell'Aquila}

\textbf{Rarità:} Non Comune; \textbf{Costo:} 4500 mo

mentre indossi queste \textbf{lenti} di cristallo davanti agli occhi, hai +1d6 alle prove di Consapevolezza basate sulla vista. In condizioni di visibilità limpida, puoi distinguere i dettagli anche di creature e oggetti molto distanti delle dimensioni di 50 centimetri.

\oggettomagico{Occhi della pietrificazione}

queste due \textbf{lenti} di cristallo magico si sovrappongono alle pupille degli occhi. Quando una creatura mette queste lenti, viene immediatamente pietrificata senza Tiro Salvezza. Circa un quarto di questi oggetti (probabilità del 25\%) consentono invece a chi le mette di pietrificare con lo sguardo, ma in questo caso le vittime hanno diritto a un Tiro Salvezza DC 17. Non è possibile combinare due tipi di lenti magiche.

\oggettomagico{Occhi della Vista Dettagliata}

\textbf{Rarità:} Non Comune; \textbf{Costo:} 2500 mo

mentre indossi queste \textbf{lenti} di cristallo davanti agli occhi, puoi vedere molto meglio del normale fino a una distanza di 30 centimetri. Hai +1d6 alle prove di Consapevolezza basata su vista mentre perlustri un'area o studi un oggetto a distanza ravvicinata.

\oggettomagico{Occhiali da Notte}

\textbf{Rarità:} Non Comune; \textbf{Costo:} 3500 mo

mentre indossi queste \textbf{lenti} scure, possiedi la scurovisione, con una gittata di 9 metri. Se già possiedi la scurovisione, indossare questi occhiali ne aumenta la gittata di 18 metri.

\oggettomagico{Olio di Affilatezza}

\textbf{Rarità:} Molto Raro; \textbf{Costo:} 3200 mo

quest'\textbf{olio} può ricoprire un'arma tagliente o perforante o fino a 5 munizioni taglienti o perforanti. Applicare l'olio richiede 1 minuto. Per 1 ora, l'arma ricoperta dall'olio è magica e ha un bonus di +3 ai Tiri per Colpire e danno.

\oggettomagico{Olio di Forma Eterea}

\textbf{Rarità:} Raro; \textbf{Costo:} 2000 mo

una dose di \textbf{olio} è sufficiente a ricoprire una creatura di taglia Media o inferiore, e l'equipaggiamento che indossa e trasporta (è necessaria un'ulteriore fiala per ogni categoria di taglia sopra la Media). Applicare l'olio richiede 10 minuti. Dopodiché la creatura ottiene l'effetto dell'incantesimo forma eterea per 1 ora.

\oggettomagico{Olio di Scivolosita'}

\textbf{Rarità:} Non Comune; \textbf{Costo:} 500 mo

l'\textbf{olio} può coprire una creatura di taglia Media o inferiore, insieme a tutto l'equipaggiamento che indossa o trasporta (è necessaria un'ulteriore fiala per ogni categoria di taglia sopra la Media). Applicare l'olio richiede 10 minuti. La creatura ottiene poi il beneficio dell'incantesimo libertà di movimento per 8 ore. In alternativa, con due azioni si può versare l'olio sul terreno, duplicando per 8 ore l'effetto dell'incantesimo unto su quell'area.

\oggettomagico{Armatura d'Ombra}

\textbf{Aura:} Illusione debole; \textbf{Costo:} 1875 mo

\textbf{Requisiti:} Creare Oggetti Magici, Invisibilità, Silenzio; \textbf{Rarità:} Non Comune

Quest'\textbf{armatura} rende chi la indossa sfocato ogni volta che tenta di nascondersi, fornendo bonus di +4 alle sue prove di Furtività per nascondersi. La penalità di armatura alla prova si applica normalmente.

\oggettomagico{Ospitale}

\textbf{Aura:} Lista Animali e Piante moderata; \textbf{Costo:} 3750 mo

\textbf{Requisiti:} Creare Oggetti Magici 2, Scrigno Segreto; \textbf{Rarità:} Leggendario

Un'\textbf{armatura} o uno scudo con questa capacità speciale nasconde animali vivi all'interno della sua iconografia per tenerli al sicuro. Il portatore con una parola di comando immagazzina magicamente un animale a cui è legato, come un Famiglio o una Cavalcatura. L'animale immagazzinato appare come simbolo sull'armatura o sullo scudo, che si tratti di un'imitazione dell'aspetto dell'animale o di una rappresentazione più simbolica e astratta.

Mentre è immagazzinato, l'animale dorme e non dà alcun beneficio (come il bonus alle competenze di un Famiglio) a chi la indossa. La taglia degli animali immagazzinabili dipende dal tipo di armatura o scudo. Le armature leggere o medie e gli scudi leggeri o pesanti possono immagazzinare un animale di taglia massima pari a quella di chi li indossa. Un'armatura pesante o uno scudo torre possono immagazzinare un animale fino a una categoria taglia superiore rispetto a chi li indossa. Una seconda parola di comando rilascia l'animale immagazzinato nell'armatura o nello scudo ospitale. Un animale liberato si risveglia immediatamente, appare in uno spazio adiacente al portatore e può intraprendere azioni nel round in cui appare.

Dato che l'animale immagazzinato dorme anziché essere in animazione sospesa (o persino in letargo), invecchia e ha fame al ritmo normale mentre è immagazzinato. Un'armatura o uno scudo Ospitale rilascia automaticamente un animale immagazzinato 24 ore dopo che vi è stato immagazzinato all'interno.

\oggettomagico{Palla di Cristallo}

\textbf{Rarità:} Molto Raro; \textbf{Costo:} 50000 mo

una tipica \textbf{palla} di cristallo ha il diametro di circa 15 centimetri. Mentre la tocchi, puoi lanciare tramite essa l'incantesimo scrutare (DC del Tiro Salvezza 21). Le seguenti palle di cristallo varianti sono oggetti leggendari e hanno proprietà aggiuntive.

\emph{Palla di Cristallo di Lettura del Pensiero}. Questa palla di cristallo è di circa 12 centimetri di diametro. Mentre la tocchi, puoi lanciare tramite di essa l'incantesimo scrutare (DC del Tiro Salvezza 21). Puoi usare due azioni per lanciare l'incantesimo individuazione dei pensieri (DC del Tiro Salvezza 21) mentre stai scrutando tramite questa palla di cristallo, prendendo come bersaglio le creature che puoi vedere e si trovano entro 9 metri dal sensore dell'incantesimo. Non devi concentrarti su questo individuazione dei pensieri per mantenerlo per la sua durata, che termina quando termina scrutare.

\emph{Palla di Cristallo di Telepatia}. Questa palla di cristallo è di circa 12 centimetri di diametro. Mentre la tocchi, puoi lanciare tramite di essa l'incantesimo scrutare (DC del Tiro Salvezza 21). Mentre scruti tramite questa palla di cristallo, puoi comunicare telepaticamente con le creature che puoi vedere e si trovano entro 9 metri dal sensore dell'incantesimo. Puoi anche usare due azioni per lanciare l'incantesimo suggestione (DC del Tiro Salvezza 21) su una di queste creature tramite il sensore. Non devi concentrarti su questa suggestione per mantenerla per la sua durata, che termina se termina scrutare. Una volta usato, il potere suggestione della palla di cristallo non può essere usato di nuovo fino alla prossima alba.

\emph{Palla di Cristallo di Visione del Vero}. Questa palla di cristallo è di circa 12 centimetri di diametro. Mentre la tocchi, puoi lanciare tramite di essa l'incantesimo scrutare (DC del Tiro Salvezza 21). Mentre scruti con questa palla di cristallo, hai visione del vero con un raggio di 36 metri centrato sul sensore dell'incantesimo.

\oggettomagico{Palla di Cristallo ipnotica}

\textbf{Rarità:} Raro

questo oggetto \textbf{maledetto} è indistinguibile da una normale Palla di cristallo. Tuttavia chiunque tenti di usare il dispositivo rimane affascinato per 1d6 turni, ed una suggestione telepatica viene impiantata nella sua mente se fallisce un Tiro Salvezza su Volontà DC 27. L'utilizzatore del dispositivo crede di aver visto la creatura o scena desiderata, ma in realtà è sotto l'influenza di un potente incantatore, o addirittura una potenza o essere da un altro piano di esistenza. Ad ogni uso ulteriore l'utilizzatore cade sempre più sotto l'influenza del controllore, come servo o come strumento. L'utilizzatore è sempre ignaro di essere soggiogato.

\oggettomagico{Pantofole del Ragno}

\textbf{Rarità:} Non Comune; \textbf{Costo:} 5000 mo

mentre indossi queste \textbf{scarpe} leggere, puoi muoverti verso l'alto, il basso, e lungo superfici verticali e a testa in giù sul soffitto, lasciando libere le mani. Hai una velocità di scalata pari alla velocità di movimento. Tuttavia, le pantofole non ti permettono di muoverti a questo modo su terreno difficile, come pareti coperte da ghiaccio, da olio, macerie...

\oggettomagico{Perfida}

\textbf{Aura:} Evocazione debole; \textbf{Costo:} 3000 mo

\textbf{Requisiti:} Creare Oggetti Magici, Causa Ferite Leggere

Quando ottieni un Tiro Critico con quest'\textbf{arma} magica infliggi un Tiro Critico aggiuntivo.

\oggettomagico{Pergamena contro gli elementali}

\textbf{Rarità:} Raro; \textbf{Costo:} 800 mo

questa \textbf{pergamena} protegge da tutti gli elementali per 20 round, concedendo +4 alla Difesa e Tiri Salvezza contro attacchi o effetti prodotti dagli elementali.

\oggettomagico{Pergamena contro i licantropi}

\textbf{Rarità:} Non Comune; \textbf{Costo:} 700 mo

questa \textbf{pergamena} protegge da tutti i licantropi per 20 round, concedendo +4 alla Difesa e Tiri Salvezza contro attacchi o effetti prodotti dai licantropi.

\oggettomagico{Pergamena contro i non morti}

\textbf{Rarità:} Non Comune; \textbf{Costo:} 900 mo

questa \textbf{pergamena} protegge da tutti i non morti per 20 round, concedendo +4 alla Difesa e Tiri Salvezza contro attacchi o effetti prodotti dai non morti.

\oggettomagico{Pergamena degli Incantesimi}

questa \textbf{pergamena} degli incantesimi riporta le parole di un singolo incantesimo, scritte in un codice mistico. Vedi \hyperlink{crearepergamene}{Creare Pergamene}, pag. \pageref{crearepergamene}.

Lanciare l'incantesimo leggendolo da una pergamena richiede il normale tempo di lancio dell'incantesimo. Una volta che l'incantesimo è stato lanciato, le parole sulla pergamena svaniscono, e la pergamena viene ridotta in polvere. Se il lancio viene interrotto, la pergamena non si dissolve.

\oggettomagico{Pergamena protettiva contro la magia}

\textbf{Rarità:} Raro; \textbf{Costo:} 1500 mo

la \textbf{pergamena} lancia un incantesimo di Campo Anti-Magia.

\oggettomagico{Perla del Potere}

\textbf{Rarità:} Non Comune; \textbf{Costo:} 6000 mo

mentre hai la \textbf{perla} con te, puoi usare due azioni per recuperare 1d4 Punti Magia. Una volta usata, la perla non potrà essere usata di nuovo fino alla prossima alba. Esistono varianti più potenti che fanno recuperare più punti.

\oggettomagico{Perla della Saggezza}

\textbf{Rarità:} Raro; \textbf{Costo:} 20000 mo

questa \textbf{perla} magica dona un punto di Saggezza extra che la tiene con sé per 4 settimane. Trascorso questo tempo la perla dovrà essere sempre portata per non perderne i benefici. C'è un 5\% di probabilità che una perla sia maledetta e sortisca l'effetto opposto. In questo caso, dopo 4 settimane, l'effetto negativo è permanente e cancellabile solo da desiderio.

\oggettomagico{Pietosa}

\textbf{Aura:} Evocazione debole; \textbf{Costo:} 3000 mo

\textbf{Requisiti:} Creare Oggetti Magici, Cura Ferite

Tutto il danno inflitto dall'\textbf{arma} è temporaneo.

A comando, come Azione Immediata, l'arma sopprime questa capacità fino a quando non le viene ordinato di riattivarla.

\oggettomagico{Pietra Arcana}

\textbf{Rarità:} Molto Raro; \textbf{Costo:} 3000 mo

\index{Ioun Stone}
esistono diversi tipi di \textbf{pietra} arcana, ogni tipo una specifica combinazione di forme e colori.

Quando usi due azioni per lanciare una di queste pietre in aria, la pietra inizia a orbitare intorno alla tua testa alla distanza di 1d3 x 30 centimetri e ti conferisce un beneficio.
Un'altra creatura può usare due azioni per afferrare la pietra e separarla da te riuscendo in un Tiro per Colpire disarmato contro Difesa 24. Puoi usare due azioni per afferrare e mettere da parte la pietra terminandone l'effetto.

Una pietra ha Difesa 24, 10 Punti Ferita e resistenza a tutti i danni. Mentre orbita intorno alla tua testa è considerata un oggetto indossato.

\emph{Destrezza} (molto raro, 3000 mo). Mentre orbita intorno alla tua testa il tuo punteggio di Destrezza aumenta di 1, fino a un massimo di 5.

\emph{Assorbimento} (molto raro, 6000 mo). Mentre orbita intorno alla tua testa, puoi usare una tua Azione per cancellare un incantesimo di livello 4 o inferiore lanciato da una creatura visibile e che prende a bersaglio solo te. Una volta che la pietra ha cancellato 5 Incantesimi, si esaurisce e diventa grigia opaca, perdendo la sua magia.

\emph{Autorità} (molto raro, 3000 mo). Mentre orbita intorno alla tua testa il tuo punteggio di Carisma aumenta di 1, fino a un massimo di 5.

\emph{Consapevolezza} (raro, 12000 mo). Mentre orbita intorno alla tua testa non puoi essere sorpreso.

\emph{Forza} (molto raro, 3000 mo). Mentre orbita intorno alla tua testa il tuo punteggio di Forza aumenta di 1, fino a un massimo di 5.

\emph{Intelligenza} (molto raro, 3000 mo). Mentre orbita intorno alla tua testa il tuo punteggio di Intelligenza aumenta di 1, fino a un massimo di 5.

\emph{Intuizione} (molto raro, 3000 mo). Mentre orbita intorno alla tua testa il tuo punteggio di Saggezza aumenta di 1, fino a un massimo di 5.

\emph{Protezione} (raro, 10000 mo). Mentre orbita intorno alla tua testa ottieni un bonus di +1 alla Difesa.

\emph{Sostentamento} (raro, 3500 mo). Mentre orbita intorno alla tua testa non hai bisogno di mangiare né di bere.

\oggettomagico{Pietra degli Elementali della Terra}

\textbf{Rarità:} Raro; \textbf{Costo:} 8000 mo

se la \textbf{pietra} tocca terra, puoi usare due azioni per pronunciare la parola di comando ed evocare un elementale della terra, come se avessi lanciato l'incantesimo evocare elementali. La pietra non può di nuovo essere usata a questo modo, fino alla prossima alba. La pietra pesa 2,5 chili, ingombro 3.

\oggettomagico{Pietra del Peso}

questo oggetto sembra un \textbf{sasso} nero liscio e lucido. Quando chi la porta è coinvolto in un combattimento o in una fuga, subisce improvvisamente gli effetti dell'incantesimo \emph{lentezza}. Una volta presa, la pietra non può essere buttata via normalmente, poiché dopo poco tempo riappare magicamente sulla persona del possessore. Per liberarsi definitivamente della pietra occorre l'incantesimo Rimuovi Maledizione.

\oggettomagico{Pietra della Buona Sorte}

\textbf{Rarità:} Non Comune; \textbf{Costo:} 4500 mo

finché la \textbf{pietra} è con te, ottieni un bonus di +1 alle prove su competenze di base e ai Tiri Salvezza.

\oggettomagico{Pietre parlanti}

\textbf{Rarità:} Raro; \textbf{Costo:} 4000 mo

queste due \textbf{pietre} vendute sempre a coppia permettono la comunicazione a distanza tra i loro utilizzatori. Non c'è limite di distanza o Piano. Le pietre possono essere usate per 10 minuti al giorno, poi vanno esposte alla luce solare per almeno 3 ore. Ingombro L.

\oggettomagico{Piffero delle Fogne}

\textbf{Rarità:} Non Comune; \textbf{Costo:} 2000 mo

devi essere competente con gli strumenti a fiato per usare questo \textbf{piffero}. Mentre usi questo piffero, i ratti normali e i ratti giganti sono indifferenti nei tuoi confronti e non ti attaccheranno a meno che non li minacci o li danneggi. Mentre suoni il piffero, puoi usare due azioni per spendere da 1 a 3 cariche richiamando uno sciame di ratti per ogni carica spesa, purché ci siano abbastanza ratti entro 750 metri da te da richiamare in questa maniera (a discrezione del Narratore). Se non ci sono abbastanza ratti da formare uno sciame, la carica è sprecata. Gli sciami richiamati si muovono verso la musica tramite la rotta più breve possibile, ma non sono in alcun altro modo sotto il tuo controllo. Il piffero ha 3 cariche e recupera 1 carica ogni giorno all'alba.

Ogni qualvolta uno sciame di ratti che non sia sotto il controllo di un'altra creatura si avvicina entro 9 metri da te mentre stai suonando il piffero, puoi effettuare una prova di Intrattenere contrapposta alla Volontà dello sciame. Se perdi la contesa, lo sciame si comporta come di norma e non può essere di nuovo distratto dalla musica del piffero per le successive 24 ore. Se vinci la contesa, lo sciame è attratto dalla musica del piffero e diventa amichevole nei confronti tuoi e dei tuoi compagni fino a che continui a suonare il piffero con due azioni ogni round. Uno sciame amichevole obbedisce ai tuoi comandi. Se non impartisci ordini a uno sciame amichevole, questo si difenderà ma non compirà altre azioni.

Se uno sciame amichevole all'inizio del round non può udire la musica del piffero, il tuo controllo su quello sciame termina, e lo sciame si comporta come farebbe normalmente e non può essere attirato nuovamente dalla musica del piffero per le successive 24 ore.

\oggettomagico{Piffero dello Spavento}

\textbf{Rarità:} Non Comune; \textbf{Costo:} 6000 mo

devi essere competente con gli strumenti a fiato per usare questo \textbf{piffero}. Puoi usare due azioni per suonarlo e spendere 1 carica per creare un suono incantevole e spettrale. Ogni creatura entro 9 metri da te e che ti oda suonare deve superare un Tiro Salvezza su Volontà con DC 21 o restare spaventata da te per 1 minuto. Se lo desideri, tutte le creature nell'area che non ti siano ostili possono superare automaticamente il loro Tiro Salvezza. Una creatura che fallisca il Tiro Salvezza può ripeterlo alla fine del suo round, terminando l'effetto su di sé in caso lo superi. Una creatura che superi il Tiro Salvezza è immune all'effetto di questo piffero per 24 ore. Il piffero ha 3 cariche e recupera 1 carica ogni giorno all'alba.

\oggettomagico{Pigmenti delle Meraviglie}

\textbf{Rarità:} Molto Raro; \textbf{Costo:} 400 mo

trovati solitamente in 1d4 vasetti all'interno di eleganti scatole di legno assieme a un pennello (1 ingombro, peso totale di 500 grammi), questi \textbf{pigmenti} ti permettono di creare oggetti tridimensionali, dipingendoli a due dimensioni. La pittura fluisce dal pennello per formare l'oggetto desiderato mentre ti concentri sull'immagine

Ogni vasetto di pittura è sufficiente a coprire 90 m quadri di una superficie, permettendoti di creare oggetti inanimati e caratteristiche del terreno (porte, fosse, fiori, alberi, celle, stanze o armi) che occupino un totale di 270 metri cubi. Ci vogliono 10 minuti per coprire 90 quadri.

Quando completi il dipinto, l'oggetto o la caratteristica del terreno dipinta diventa un oggetto reale, non magico. Quindi, dipingere una porta su di una parete crea una vera porta che può essere aperta per accedere a ciò che si trova oltre di essa. Dipingere una fossa sul pavimento crea una vera fossa, la cui profondità è conteggiata nell'area totale degli oggetti che puoi creare.

Nulla di ciò che viene creato dai pigmenti può avere un valore superiore ai 25 mo. Se dipingi un oggetto di valore superiore (un diamante o una pila d'oro), l'oggetto sembrerà autentico, ma un attento esame rivelerà che è fatto di gesso, ossa o qualche altro materiale privo di valore.

Se dipingi una forma di energia, come fuoco o fulmine, l'energia compare ma si dissipa non appena completi il dipinto, senza recare danni a niente.

\oggettomagico{Piuma Arcana}

\textbf{Rarità:} variabile; \textbf{Costo:} 50 mo

questo minuscolo oggetto assomiglia a una \textbf{piuma}. Esistono diversi tipi di piume arcane, ciascuno dotato di un singolo effetto monouso. Il Narratore sceglie il tipo di piuma arcana.

\emph{Albero}. Devi trovarti all'aperto per poter usare questa piuma arcana. Puoi usare due azioni per appoggiarla a uno spazio non occupato sul terreno. La piuma svanisce e al suo posto spunta un albero di quercia non magico. L'albero è alto 18 metri e ha un tronco di 1 metro di diametro. In cima, i suoi rami si estendono per un massimo di 6 metri. 50 mo

\emph{Ancora}. Puoi usare due azioni per appoggiare la piuma arcana a una barca o nave. Per le successive 24 ore, il vascello non potrà essere mosso in alcun modo. Toccare di nuovo il vascello con la piuma arcana termina questo effetto. Quando l'effetto termina, la piuma svanisce. 50 mo

\emph{Frusta}. Puoi usare due azioni per lanciare la piuma arcana verso un punto entro 3 metri da te. La piuma svanisce e al suo posto compare una frusta fluttuante. Puoi poi usare due azioni per effettuare un attacco con incantesimo in mischia contro una creatura entro 3 metri dalla frusta, con un bonus di attacco +9. Se colpisci, il bersaglio subisce 1d6 + 5 danni da forza. Durante il tuo round, con due azioni puoi dirigere la frusta affinché voli per un massimo di 6 metri e ripeta l'attacco contro una creatura entro 3 metri da essa. La frusta svanisce dopo 1 ora, quando usi due azioni per congedarla, o quando sei inabile o muori. 250 mo

\emph{Nave Cigno}. Puoi usare due azioni per appoggiare la piuma arcana su di una massa d'acqua di almeno 18 metri di diametro. La piuma svanisce e al suo posto compare una barca lunga 15 metri e larga 6 metri dalla forma di cigno. La barca si sposta da sola e si muove in acqua alla velocità di 9 chilometri all'ora. Puoi usare due azioni, mentre sei a bordo per comandarle di muoversi o voltare di 90 gradi. La barca può trasportare fino a trentadue creature di taglia Media o inferiore. Una creatura Grande conta come quattro creature Medie, mentre una creatura Enorme conta come nove creature Medie. La barca svanisce dopo 24 ore. Puoi congedare la barca con due azioni. 3000 mo

\emph{Uccello}. Puoi usare due azioni per lanciare la piuma arcana 1 metro nell'aria. La piuma svanisce e un enorme uccello multicolore ne prende il posto. L'uccello ha le statistiche di un Roc, ma obbedisce a comandi semplici e non può attaccare. Può trasportare fino a 250 chili mentre vola alla sua velocità massima (24 chilometri all'ora per un massimo di 216 chilometri al giorno, con un'ora di riposo ogni 3 ore di volo), o 500 chili di peso a metà velocità. L'uccello svanisce dopo aver volato per la distanza massima possibile in un giorno o se scende a 0 Punti Ferita. Puoi congedare l'uccello con due azioni. 3000 mo

\emph{Ventaglio}. Se ti trovi su di una barca o una nave, puoi usare due azioni per lanciare la piuma arcana fino a 3 metri in aria. La piuma svanisce e un gigantesco ventaglio compare al suo posto. Il ventaglio fluttua e crea un vento forte abbastanza da gonfiare le vele della nave, aumentandone la velocità di 7,5 chilometri all'ora per 8 ore. Puoi congedare il ventaglio con due azioni. 250 mo

\oggettomagico{Polvere dell'Aridita'}

\textbf{Rarità:} Raro; \textbf{Costo:} 120 mo

questa piccola confezione contiene 1d6 + 4 pizzichi di \textbf{polvere}. Puoi usare due azioni per spargere un pizzico di polvere sull'acqua La polvere trasforma un cubo d'acqua di 3 metri di spigolo in una pallina di polvere delle dimensioni di una biglia, che fluttua o si deposita nel punto in cui è stata gettata la polvere. Il peso della pallina è trascurabile.

Chiunque può usare due azioni per spaccare la pallina contro una superficie dura, facendo sì che la pallina si rompa e liberi l'acqua assorbita dalla polvere. Farlo esaurisce la magia della pallina.

Un elementale composto principalmente d'acqua e che venga esposto a un pizzico di questa polvere, deve effettuare un tiro Salvezza su Tempra con DC 15, subendo 10d6 danni da Vuoto se lo fallisce, o la metà di questi danni se lo riesce.

\oggettomagico{Polvere della Sparizione}

\textbf{Rarità:} Raro; \textbf{Costo:} 700 mo

rinvenuta in piccoli sacchetti, questa \textbf{polverina} sembra sabbia molto sottile. In un sacchetto ce n'è a sufficienza per un uso. Quando usi due azioni per lanciare la polvere in aria, tu e ciascuna creatura e oggetto entro 3 metri da te diventate invisibili per 2d4 minuti. La durata è la stessa per tutti i soggetti e quando la magia prende effetto la polvere si consuma. Se una creatura sotto l'effetto della polvere attacca o lancia un incantesimo, l'invisibilità ha fine solo per quella creatura.

\oggettomagico{Polvere dello Starnuto e del Soffocamento}

\textbf{Rarità:} Non Comune; \textbf{Costo:} 480 mo

trovata in piccoli contenitori, questa \textbf{polverina} sembra sabbia sottile. Appare simile alla polvere della sparizione e l'incantesimo identificare la rivela come tale. Ce n'è a sufficienza per un uso. Quando usi due azioni per lanciare una manciata di polvere in aria tu e tutte le creature che necessitano di respirare e si trovino entro 9 metri da te dovete superare un tiro Salvezza su Tempra con DC 17 o smettere di respirare, e iniziare a starnutire in maniera incontrollabile. Una creatura afflitta a questo modo è inabile e soffoca. Finché è cosciente, la creatura può ripetere il Tiro Salvezza alla fine di ciascun suo round, terminando l'effetto in caso lo superi. Anche l'incantesimo ristorare inferiore può terminare l'effetto che affligge la creatura.

\oggettomagico{Polvere Rivelatrice}

\textbf{Rarità:} Non Comune; \textbf{Costo:} 500 mo

questa polverina sottile sembra un \textbf{pulviscolo} metallico molto leggero. Una manciata di questa sostanza spruzzata in aria ricopre tutti gli oggetti in un raggio di 3 metri, rendendo ogni cosa visibile. Se viene spruzzata attraverso una cerbottana, la polvere riempie un cono lungo 6 metri e largo 1 metro all'estremità. La polvere annulla gli effetti di potere illusorio, del mantello distorcente, del mantello elfico e le capacità speciali di creature come i molossi instabili e le pantere distorcenti; l'effetto dura 2d10 turni. La polvere rivelatrice di solito viene conservata in piccoli sacchetti di seta o in tubetti cavi fatti d'osso; normalmente si trovano 5d10 dosi di polvere.

\oggettomagico{Portale Cubico}

\textbf{Rarità:} Leggendario; \textbf{Costo:} 40000 mo

questo \textbf{cubo} di 7,5 centimetri di spigolo irradia una palpabile energia magica. Le sei facce del cubo sono ciascuna collegata a un diverso piano di esistenza, uno dei quali è il Piano Materiale. Le altre facce sono collegate a piani determinati dal Narratore.

Puoi usare due azioni per premere una faccia del cubo per lanciare tramite esso l'incantesimo portale, aprendo un passaggio verso il piano collegato a quella faccia. In alternativa, se usi due azioni per premere una faccia due volte, puoi spostarti di un piano (DC del Tiro Salvezza 17) tramite il cubo e trasportarne i bersagli al piano collegato a quella faccia. Il cubo ha 3 cariche. Ogni uso del cubo spende 1 carica. Il cubo recupera 1 carica spesa ogni giorno all'alba.

\oggettomagico{Pozione dell'Inganno}

\textbf{Rarità:} Rara; \textbf{Costo:} 500 mo

questa \textbf{pozione} ha un nome quanto mai appropriato, poiché convince chi la beve di aver ingerito una pozione di un altro tipo. Per esempio, una finta "pozione di chiaraudienza" potrebbe far sentire a chi l'ha bevuta suoni che in realtà non esistono. Se più persone assaggiano questa pozione, c'è una probabilità del 90\% che concordino nel ritenerla dello stesso tipo.

\oggettomagico{Pozione dell'invulnerabilita'}

\textbf{Rarità:} Rara; \textbf{Costo:} 800 mo

una \textbf{pozione} di invulnerabilità conferisce a chi la beve un bonus +2 ai Tiri Salvezza e un miglioramento di 2 punti la Difesa per 10 minuti.

\oggettomagico{Pozione della Chiaraudienza animale}

\textbf{Rarità:} Non Comune; \textbf{Costo:} 500 mo

non comune, questa \textbf{pozione} conferisce a chi la beve la facoltà di percepire i suoni attraverso le orecchie di un animale che si trovi in un raggio di 18 metri. Una barriera di piombo blocca questo effetto.

\oggettomagico{Pozione della Chiaroveggenza animale}

\textbf{Rarità:} Non Comune; \textbf{Costo:} 500 mo

non comune,questa \textbf{pozione} conferisce a chi la beve la facoltà di vedere attraverso gli occhi di un animale che si trovi in un raggio di 18 metri. Una barriera di piombo blocca questo effetto.

\oggettomagico{Pozione della Levitazione}

\textbf{Rarità:} Non Comune; \textbf{Costo:} 200 mo

questa \textbf{pozione} ha lo stesso effetto dell'incantesimo levitazione.

\oggettomagico{Pozione della Longevita'}

\textbf{Rarità:} Leggendaria; \textbf{Costo:} 15000 mo

questa \textbf{pozione} fa ringiovanire di 1d12 anni. La giovinezza riguadagnata non annulla soltanto l'invecchiamento naturale, ma anche l'invecchiamento causato da effetti magici o creature. Esiste un pericolo nell'usare questa pozione, poiché ogni volta che si beve una pozione di longevità, c'è una probabilità cumulativa dell'1\% che tutti i benefici precedentemente guadagnati con pozioni di questo tipo siano annullati. Non è possibile consumare una dose parziale di questa pozione.

\oggettomagico{Pozione della Metamorfosi}

\textbf{Rarità:} Rara; \textbf{Costo:} 2500 mo

questa \textbf{pozione} conferisce un potere analogo all'incantesimo metamorfosi.

\oggettomagico{Pozione di Amicizia con gli Animali}

\textbf{Rarità:} Non Comune; \textbf{Costo:} 200 mo

quando bevi questa \textbf{pozione}, per 1 ora puoi lanciare a volontà l'incantesimo Amicizia con gli Animali (DC del Tiro Salvezza 15).

\oggettomagico{Pozione di Arrampicata}

\textbf{Rarità:} Comune; \textbf{Costo:} 250 mo

quando bevi questa \textbf{pozione}, per 1 ora ottieni velocità di scalata pari alla tua velocità di movimento. Durante questo periodo hai +1d6 alle prove di Resistenza che compi per effettuare una scalata.

\oggettomagico{Pozione di Controllo degli animali}

\textbf{Rarità:} Rara; \textbf{Costo:} 1500 mo

chiunque beva questa \textbf{posizione} è come avesse lanciato Dominare Bestie

\oggettomagico{Pozione di Controllo dei draghi}

\textbf{Rarità:} Leggendaria; \textbf{Costo:} 5000 mo

questa \textbf{pozione} conferisce un potere equivalente all'incantesimo dominare mostri su un singolo tipo di drago. E' possibile controllare un drago entro 18 metri per 5d4 round.

\oggettomagico{Pozione di Controllo dei non morti}

\textbf{Rarità:} Rara; \textbf{Costo:} 2500 mo

anche se normalmente i non morti sono immuni a questo tipo di effetti, questa \textbf{pozione} permette a chi la beve di influenzare fino a 4 non morti con GS 3 o meno (intelligenti o no) come se usasse l'incantesimo charme. La durata dell'effetto è di 5d4 round.

\oggettomagico{Pozione di Controllo delle persone}

\textbf{Rarità:} Non Comune; \textbf{Costo:} 500 mo

una volta ingerita, questa \textbf{pozione} conferisce a chi la beve un potere analogo all'incantesimo charme.

\oggettomagico{Pozione di Controllo delle piante}

\textbf{Rarità:} Rara; \textbf{Costo:} 1500 mo

chi beve questa \textbf{pozione} è in grado di controllare tutte le piante e le creature vegetali (compresi i funghi) in un'area quadrata di 6x6 m ed entro una distanza di 27 metri. L'effetto dura 5d4 round. Le piante obbediscono secondo le loro possibilità (ad esempio, le liane possono attorcigliarsi e infittirsi, causando lentezza o impedimento alla vista). E' possibile dare ordini a creature vegetali senzienti, ma queste hanno diritto a un Tiro Salvezza su Volontà DC 19. Come per altri tipi di ammaliamento, non si può ordinare a una creatura controllata di farsi male da sola.

\oggettomagico{Pozione di Crescita}

\textbf{Rarità:} Raro; \textbf{Costo:} 300 mo

quando bevi questa \textbf{pozione} per 1d4 ore ottieni l'effetto ingrandire o ridurre dell'incantesimo ingrandire (non richiede concentrazione).

\oggettomagico{Pozione di Eroismo}

\textbf{Rarità:} Raro; \textbf{Costo:} 200 mo

quando bevi questa \textbf{pozione}, ottieni 10 Punti Ferita temporanei che durano 1 ora. Per la stessa durata sei sotto l'effetto dell'incantesimo benedizione (non richiede concentrazione).

\oggettomagico{Pozione di Forma Gassosa}

\textbf{Rarità:} Raro; \textbf{Costo:} 1500 mo

quando bevi questa \textbf{pozione}, per 1 ora o finché non terminerai l'effetto con due azioni ottieni l'effetto dell'incantesimo \hyperlink{PozionediFormaGassosa}{forma gassosa} (non richiede concentrazione).

\oggettomagico{Pozione di Forza dei Giganti}

\textbf{Rarità:} Molto Raro; \textbf{Costo:} 500 mo

quando bevi questa \textbf{pozione}, per 1 ora il tuo punteggio di Forza cambia. Il tipo di gigante determina il punteggio (vedi la tabella seguente). La pozione non ha effetto se il tuo punteggio di Forza è pari o superiore al nuovo punteggio. La pozione della forza del gigante del gelo e la pozione della forza del gigante di pietra hanno lo stesso effetto.

\begin{itemize} \setlength\itemsep{0em}
\item delle colline, Forza 5 500 mo
\item di pietra o del gelo, Forza 6 1000 mo
\item del fuoco, Forza 7 2000 mo
\item delle nuvole, Forza 8 5000 mo
\item delle tempeste, Forza 9 10000 mo
\end{itemize}

\oggettomagico{Pozione di Guarigione}

\textbf{Rarità:} Non Comune; \textbf{Costo:} 50 mo

quando bevi da \textbf{questa} pozione, recuperi un numero di Punti Ferita che varia a seconda della rarità della pozione di guarigione.

\begin{itemize} \setlength\itemsep{0em}
\item \textbf{Comune}, Punti Ferita 1d8 + 1. Comune, 50 mo
\item \textbf{Potenziata}, Punti Ferita 3d8 +3. Non Comune, 125 mo
\item \textbf{Maggiore}, Punti Ferita 3d10 +30. Rara, 300 mo
\item \textbf{Suprema}, Punti Ferita 5d10 + 50. Molto Rara, 1500 mo
\end{itemize}

\oggettomagico{Pozione di Invisibilita'}

\textbf{Rarità:} Molto Raro; \textbf{Costo:} 200 mo

quando bevi questa \textbf{pozione}, per 1 ora diventi invisibile. Mentre sei invisibile, tutto ciò che trasporti o indossi resta anch'esso invisibile assieme a te. L'effetto ha termine qualora tu attacchi o lanci un incantesimo.

\oggettomagico{Pozione di Lettura del Pensiero}

\textbf{Rarità:} Raro; \textbf{Costo:} 200 mo

quando bevi questa \textbf{pozione}, ottieni l'effetto dell'incantesimo individuazione dei pensieri (DC del Tiro Salvezza 15).

\oggettomagico{Pozione di Resistenza}

\textbf{Rarità:} Non Comune; \textbf{Costo:} 300 mo

quando bevi questa \textbf{pozione}, per 1 ora ottieni resistenza a un tipo di danno. Il Narratore sceglie il tipo di danno o lo determina casualmente (Acido, Freddo, Fuoco, Fulmine, Veleno, Suono, Luce, Vuoto)

\oggettomagico{Pozione di Respirare Sott'Acqua}

\textbf{Rarità:} Non Comune; \textbf{Costo:} 200 mo

dopo aver bevuto questa \textbf{pozione}, puoi respirare sott'acqua per 1 ora.

\oggettomagico{Pozione di Rimpicciolimento}

\textbf{Rarità:} Raro; \textbf{Costo:} 300 mo

quando bevi questa \textbf{pozione} per 1d4 ore ottieni l'effetto ridurre dell'incantesimo  ridurre (non richiede concentrazione).

\oggettomagico{Pozione di Veleno}

\textbf{Rarità:} Non Comune; \textbf{Costo:} 35mo

Sono presenti diversi tipi di \textbf{veleni}:

\begin{itemize} \setlength\itemsep{0em}
\item \textbf{Indebolente}, DC 15 Tempra oppure -2 Tiri per Colpire e Tiri Salvezza per 1 minuto. Non comune, 35mo
\item \textbf{Indebolente potenziata}, DC 18 Tempra, -1d6 Tiri per Colpire e Tiri Salvezza per 1 Turno, 50 mo
\item \textbf{Veleno}, subisci 2d6+2 di danno. TS DC 15 Tempra. Non Comune, 30 mo
\item \textbf{Veleno potenziata}, subisci 2d8+4 di danno. TS DC 18 Tempra. Rara, 50 mo
\item \textbf{Veleno maggiore}, subisci 4d10+8 di danno. TS DC 24 Tempra. Molto Rara, 125 mo
\end{itemize}

\oggettomagico{Pozione di Velocita'}

\textbf{Rarità:} Molto Raro; \textbf{Costo:} 400 mo

quando bevi questa \textbf{pozione}, ottieni l'effetto dell'incantesimo \emph{Velocità} per 1 minuto.

\oggettomagico{Pozione di Volo}

\textbf{Rarità:} Molto Raro; \textbf{Costo:} 500 mo

quando bevi questa \textbf{pozione}, per 10 minuti ottieni velocità di volo pari alla tua normale velocità di movimento e puoi fluttuare. Se la pozione ha termine mentre stai volando, cadi a meno che non possiedi qualche altro metodo per restare in aria.

\oggettomagico{Pozzo dei Molti Mondi}

\textbf{Rarità:} Leggendario; \textbf{Costo:} 75000 mo

questo elegante \textbf{tessuto} nero, soffice come la seta, è avvolto fino alle dimensioni di un fazzoletto. Si dispiega in un foglio circolare di 1,8 metri di diametro. Puoi usare due azioni per dispiegare e piazzare il pozzo dei molti mondi su di una superficie solida, su cui crea un portale bidirezionale verso un altro mondo o piano di esistenza. Ogni volta che l'oggetto apre un portale, il Narratore decide il posto a cui conduce. Puoi usare due azioni per chiudere un portale aperto afferrando i margini del tessuto e ripiegandoli. Una volta che un pozzo dei molti mondi ha aperto un portale, non potrà farlo di nuovo prima che siano passate 1d8 ore.

\oggettomagico{Resistenza al Veleno}

\textbf{Aura:} Trasmutazione debole; \textbf{Costo:} 1125 mo

\textbf{Requisiti:} Creare Oggetti Magici 2, Rimuovi Veleno; \textbf{Rarità:} Rara

Un'\textbf{armatura} o uno \textbf{scudo} con questa capacità speciale conferisce a chi lo indossa Bonus di +2 ai Tiri Salvezza contro il veleno.

\oggettomagico{Resistenza all'Energia}

\textbf{Aura:} Abiurazione debole; \textbf{Costo:} 9000 mo

\textbf{Requisiti:} Creare Oggetti Magici, Protezione dall'Energia; \textbf{Rarità:} Non Comune

Questo tipo di \textbf{armatura} o \textbf{scudo} protegge contro un tipo di energia (Fuoco, Luce, Suono, Elettricità, Energia Positiva, Energia Negativa, Freddo, Vuoto) ed è decorata da disegni che raffigurano l'elemento dal quale protegge. L'armatura o lo scudo concedono Riduzione 6 ai danni di energia per attacco che verrebbero subiti normalmente da chi li indossa.

\oggettomagico{Resistenza all'Energia Superiore}

\textbf{Aura:} Abiurazione moderata; \textbf{Costo:} 21000 mo

\textbf{Requisiti:} Creare Oggetti Magici 2, Protezione all'Energia; \textbf{Rarità:} Rara

Questo tipo di \textbf{armatura} o \textbf{scudo} protegge contro un tipo di energia (Fuoco, Luce, Suono, Elettricità, Energia Positiva, Energia Negativa, Freddo, Vuoto) ed è decorata da disegni che raffigurano l'elemento dal quale protegge. L'armatura o lo scudo concedono Resistenza all'energia indicata.

\oggettomagico{Rete Intralciante}

\textbf{Rarità:} Raro; \textbf{Costo:} 800 mo

questa \textbf{rete} quadrata di 3 metri di lato può essere lanciata a un avversario per intralciarlo. La rete è molto resistente e occorre la forza di un gigante (DC 20) per strapparla a mani nude. La rete resiste anche ai tagli, e deve essere colpita con estrema precisione (Difesa 25, Punti Ferita 30) affinché ceda. La rete può anche essere appesa o messa sul terreno come una trappola, che si attiverà magicamente al comando del possessore.

\oggettomagico{Rete Intrappolante}

\textbf{Rarità:} Raro; \textbf{Costo:} 900 mo

questa \textbf{rete} può essere usata solo sott'acqua, ma funziona esattamente come una rete intralciante in superficie, fluttuando se occorre fino a 9 m di distanza per avvinghiare un avversario.

\oggettomagico{Ricercante}

\textbf{Aura:} Divinazione forte; \textbf{Costo:} 3000 mo

\textbf{Requisiti:} Creare Oggetti Magici 2, Visione del Vero

Questa capacità può essere aggiunta solo ad \textbf{armi a distanza}. Un arma Ricercante vira verso il suo bersaglio individuato, negando qualsiasi bonus dovuto alla copertura.

\oggettomagico{Ritornante}

\textbf{Aura:} Evocazione moderata; \textbf{Costo:} 3000 mo

\textbf{Requisiti:} Creare Oggetti Magici 2, Teletrasporto

Un'\textbf{arma} Ritornante può teletrasportarsi nelle mani del suo possessore come Azione Immediata, anche se si trova in possesso di un'altra creatura. Questa capacità ha un raggio massimo di 30 metri e gli effetti che bloccano il teletrasporto impediscono il ritorno di un'arma Ritornante. Un'arma Ritornante deve essere in possesso di una creatura per almeno 24 ore perché questa capacità funzioni.

\oggettomagico{Sacra}

\textbf{Aura:} Invocazione forte; \textbf{Rarità:} Rara; \textbf{Costo:} 10000 mo

\textbf{Requisiti:} Tratti  12; Creare Oggetti Magici 2

Quando con essa colpisci un immondo o un non morto, quella creatura subisce 2d10 danni da Luce aggiuntivi.

Mentre impugni l'\textbf{arma} sguainata essa crea un'aura di 3 metri di raggio attorno a te. Tu e tutte le creature a te amichevoli all'interno dell'aura ottenete +1d6 ai Tiri Salvezza contro incantesimi e altri effetti magici generati da Seguaci o Devoti di altri Patroni. Se hai Tratti in comune con il Patrono 13 o più, il raggio dell'aura aumenta a 9 metri.

\oggettomagico{Scaglie di Drago}

\textbf{Aura:} Abiurazione moderata; \textbf{Costo:} 8000 mo; \textbf{Rarità:} Leggendaria

\textbf{Requisiti:} Creare Oggetti Magici 3

Questa \textbf{armatura} o \textbf{scudo} è fatta con le scaglie di una specie di drago.

Mentre la indossi hai +1d6 ai Tiri Salvezza contro la Presenza Spaventosa e le armi a soffio dei draghi ed hai resistenza a un tipo di danno determinato dalla specie di drago che ha fornito le scaglie.

Inoltre, con due azioni puoi focalizzare i tuoi sensi per determinare magicamente la distanza e la direzione in cui si trovi il drago più vicino entro 45 chilometri che sia della stessa specie dell'armatura. Quest'azione speciale non può essere più usata fino alla prossima alba.

\oggettomagico{Scarabeo della Morte}

questa \textbf{spilla} a forma di scarabeo sembra un semplice portafortuna. Tuttavia, se tenuto in mano per 1 round o portato per 1 Turno, si trasforma in un orrendo insetto carnivoro. Dotata di potenti mandibole, la famelica creatura penetra attraverso il cuoio e il tessuto, affondando nella carme e raggiungendo il cuore in 1 round. Dopo aver ucciso la sua vittima, la creatura riassume la forma di spilla. Solo il calore che viene dal contatto con un essere vivente può animare l'insetto mostruoso, quindi mettere la spilla in una scatola o in una teca è una precauzione sufficiente per evitare ogni pericolo.

\oggettomagico{Scarabeo di Protezione}

\textbf{Rarità:} Leggendario; \textbf{Costo:} 36000 mo

se tieni questo \textbf{medaglione} a forma di scarabeo tra le tue mani per 1 round, su di esso compare un'iscrizione che ne rivela la natura magica. Mentre è addosso a te, fornisce due benefici

- Hai +2 ai Tiri Salvezza contro incantesimi.

- Lo scarabeo ha 12 cariche. Se fallisci un Tiro Salvezza contro un incantesimo di necromanzia o un effetto nocivo originante da una creatura non morta, puoi usare una Azione di Reazione per spendere 1 carica e trasformare il Tiro Salvezza fallito in un successo. Lo scarabeo si riduce in polvere ed è distrutto quando viene spesa la sua ultima carica.

\oggettomagico{Scopa del Volo maledetto}

questa \textbf{scopa} magica sembra una scopa volante. Tuttavia, quando viene attivata, vola fino a 15 m di altezza o fino al soffitto (se più basso) e poi smette di funzionare, facendo precipitare chi la cavalca.

\oggettomagico{Scopa dell'Attacco animato}

questo oggetto è indistinguibile in apparenza da una \textbf{scopa} normale. A tutti i test risulta identica ad una scopa volante, fino vola a 6 metri d'altezza. Quando ciò avviene la scopa esegue una piroetta e fa cadere il suo pilota sulla testa da un'altezza di 1d4+5 x 30 cm (non viene inflitto danno da caduta poiché la distanza è inferiore a 3 m). La scopa quindi attacca la vittima, colpendola in viso con la spazzola e battendola con il manico. La scopa effettua due attacchi per round con ciascuna estremità (due attacchi con la spazzola e due col manico per un totale di quattro attacchi). La spazzola acceca la vittima per 1 round quando colpisce. Il manico infligge 1d3 ferite. La scopa ha Difesa 13, 18 Punti Ferita, e ha +4 al Tiro per Colpire.

\oggettomagico{Scopa Volante}

\textbf{Rarità:} Non Comune; \textbf{Costo:} 8000 mo

questa \textbf{scopa} di legno, del peso di circa 1,5 chili (ingombro 2), funziona come una normale scopa fino a quando non vi siedi sopra e ne pronunci la parola di comando. Essa inizia così a fluttuare sotto di te e può essere cavalcata in aria. Ha velocità di volo 15 metri. Può trasportare fino a 200 chili, ma la sua velocità di volo diventa 9 metri se dovesse trasportare più di 100 chili. Quando atterri, la scopa smette di fluttuare.

Pronunciando la parola di comando, nominando il posto e se vi sei familiare, puoi inviare la scopa da sola in un posto fino a 1,5 chilometri da te. La scopa tornerà da te quando pronuncerai un'altra parola di comando, purché si trovi ancora entro 1,5 chilometri da te.

\oggettomagico{Scudo Animato}

\textbf{Aura:} Trasmutazione forte; \textbf{Costo:} 6000 mo

\textbf{Requisiti:} Creare Oggetti Magici 2, Animare Oggetti; \textbf{Rarità:} Rara

Mentre impugni questo \textbf{scudo}, con due azioni puoi pronunciare una parola di comando e farlo animare. Lo scudo fluttuerà nell'aria all'interno del tuo spazio per proteggerti come se lo stessi impugnando, lasciandoti libera la mano.

Lo scudo resta animato per 1 minuto, finché non usi due azioni per terminarne l'effetto, sei inabile o muori, a quel punto lo scudo cadrà a terra o tornerà nella tua mano se ne hai una libera.

\oggettomagico{Scudo dell'Attrazione dei Proiettili}

\textbf{Aura:} Trasmutazione forte; \textbf{Costo:} 2000 mo

\textbf{Requisiti:} Creare Oggetti Magici 2, Animare Oggetti; \textbf{Rarità:} Non Comune

Mentre impugni questo \textbf{scudo} apparentemente hai resistenza ai danni da parte degli attacchi con arma a distanza.

\emph{Versione maledetta}.

Togliersi lo scudo non pone fine alla maledizione (DC 28). Ogni qualvolta un attacco con arma a distanza viene effettuato contro un bersaglio entro 3 metri da te, la maledizione fa sì che diventi tu il bersaglio dell'attacco.

\oggettomagico{Sfera dell'Annientamento}

\textbf{Rarità:} Leggendario; \textbf{Costo:} 250000 mo

artefatto, questa \textbf{sfera} nera di 50 centimetri di diametro è in realtà un foro nella struttura del multiverso, che fluttua nello spazio ed è stabilizzata dal campo magico che la circonda.

La sfera annienta tutta la materia che attraversa e tutta la materia che l'attraversa. L'unica eccezione sono gli artefatti. A meno che l'artefatto non sia suscettibile ai danni della sfera dell'annientamento esso può attraversare la sfera senza problemi. Qualsiasi altra cosa tocchi la sfera e non ne sia completamente avvolta e annientata da essa, subisce 4d10 danni da forza a round.

La sfera resta immobile fino a quando qualcuno non la controlla. Se ti trovi entro 18 metri da una sfera incontrollata, puoi impiegare due azioni per effettuare una prova di Arcana con DC 30. Se la superi, la sfera levita in una direzione a tua scelta, per un numero di metri pari a 1 x Intelligenza (minimo 1 metro). Se fallisci, la sfera si muove di 3 metri verso di te. Una creatura nel cui spazio entri la sfera deve superare un Tiro Salvezza di Riflessi con DC 15 o venire toccata da essa subendo 4d10 danni da forza.

Se tenti di controllare una sfera che si trova sotto il controllo di un'altra creatura, effettui una prova contrastata di arcana contro arcana dell'altra creatura. Il vincitore della contesa ottiene il controllo della sfera e può farla levitare come di norma.

%Se la sfera entra in contatto con un portale planare, come quello creato dall'incantesimo portale, o uno spazio extradimensionale, come quello all'interno di un buco portatile, il Narratore determina casualmente ciò che accade, utilizzando la tabella seguente.

%\medskip

%\noindent\begin{tabularx}{\linewidth}{lX}
%\textbf{3d6}& \textbf{Risultato}\\
%
%3-10 &La sfera è distrutta\\
%12-16& La sfera si muove attraverso il portale o all'interno dello spazio extradimensionale.\\
%17-18 &Un squarcio spaziale spedisce ogni creatura e oggetto entro 54 metri dalla sfera, sfera inclusa, in un piano dell'esistenza casuale.\\
%\end{tabularx}

%\medskip

\oggettomagico{Soffio del Dragone}

\textbf{Aura:} Invocazione debole; \textbf{Costo:} 2500 mo

\textbf{Requisiti:} Creare Oggetti Magici, Onda rovente; \textbf{Rarità:} Rara

Uno \textbf{scudo} con questa capacità speciale di solito è realizzato con le fauci di un drago spalancate sulla parte anteriore. Uno scudo con la capacità speciale Soffio del Dragone è legato a un tipo di energia (veleno, elettricità, freddo o fuoco). Lo scudo recupera 1 carica ad ogni alba e ne può tenere fino a 10.

A comando, 2 Azioni, chi lo indossa può consumare da 1 a 5 cariche dello scudo per fargli emettere un Soffio in un cono di 3 metri che infligge 1d4 danni da energia per carica consumata (Riflessi DC 13 dimezza). Questo danno è dello stesso tipo di energia legato allo scudo. Uno scudo non può avere più di una capacità Soffio del Dragone.

\oggettomagico{Solvente Universale}

\textbf{Rarità:} Leggendario; \textbf{Costo:} 300 mo

questo \textbf{tubetto} contiene un liquido bianco con un forte odore di alcool. Puoi usare due azioni per versarne i contenuti su di una superficie a portata. Il liquido dissolve istantaneamente 1000 cm x cm di adesivo con cui entra in contatto, compresa la colla suprema.

\oggettomagico{Specchio dell'Abilita' mentale}

\textbf{Rarità:} Molto Raro; \textbf{Costo:} 15000 mo

questo oggetto sembra un normale \textbf{specchio} alto un metro e mezzo e largo 60 cm. A comando, il possessore può usarlo nei seguenti modi:

- Leggerei pensieri di una persona riflessa sulla sua superficie con la telepatia (senza bisogno di capire una lingua sconosciuta).

- Vedere altri luoghi come con una palla di cristallo, con la possibilità di vedere in altri piani, purché siano sufficientemente familiari all'osservatore.

- Creare un portale per visitare altri luoghi. Il possessore dovrà prima visualizzare il luogo, poi entrare fisicamente nello specchio, da solo o con gli accompagnatori che desidera. Lo specchio creerà un portale invisibile dall'altra parte, attraverso cui il possessore, o chiunque riesca a individuarlo, potrà attraversarlo.

- Una volta alla settimana, lo specchio può rispondere con precisione a una domanda riguardante una persona riflessa sulla sua superficie (un effetto simile all'incantesimo conoscenza delle leggende).

\oggettomagico{Specchio della Duplicazione}

\textbf{Rarità:} Leggendario

leggendario, questo \textbf{specchio} è alto un pò più di un metro e largo un pò di meno. Quando una creatura si riflette sulla superficie dello specchio, la sua immagine riflessa (un duplicato identico in tutto e per tutto) esce per attaccare l'originale. Il duplicato ha tutto l'equipaggiamento e i poteri dell'originale, compresa la magia. Il duplicato sparisce immediatamente, assieme a tutti i suoi oggetti, alla morte sua o dell'originale.

\oggettomagico{Specchio Intrappola Vita}

\textbf{Rarità:} Raro; \textbf{Costo:} 18000 mo

quando questo \textbf{specchio} alto 120 centimetri viene guardato in maniera indiretta, la sua superficie mostra una vaga immagine della creatura. Lo specchio pesa 25 chili, ingombro 7, ha Difesa 11, 10 Punti Ferita e vulnerabilità ai danni contundenti. Si frantuma ed è distrutto quando viene ridotto a 0 Punti Ferita.

Se lo specchio è appeso a una superficie verticale e ti trovi entro 1 metro da esso, puoi usare due azioni per pronunciare la sua parola di comando e attivarlo. Rimarrà attivo fino a quando non pronuncerai di nuovo la parola di comando.

Qualsiasi creatura, a parte te, che veda il suo riflesso nello specchio attivato mentre si trova entro 9 metri da esso deve superare un Tiri Salvezza su Volontà con DC 17 o finire intrappolata, insieme a tutto ciò che indossa o trasporta, in una delle dodici celle extradimensionali dello specchio. Questo Tiro Salvezza riceve +1d6 se la creatura conosce la natura dello specchio ed i costrutti riescono automaticamente il Tiro Salvezza.

Una cella extradimensionale è uno spazio infinito colmo di una densa foschia che riduce la visibilità a 3 metri. Le creature intrappolate nelle celle dello specchio non invecchiano, e non hanno bisogno di mangiare, bere o dormire. Una creatura intrappolata all'interno di una cella può fuggirne usando la magia che permette di viaggiare tra i piani. Altrimenti, la creatura è confinata nella cella fino a quando non sarà liberata.

Se lo specchio intrappola una creatura ma le sue dodici celle extradimensionali sono già occupate, lo specchio libera una delle creature intrappolate a caso per alloggiare il nuovo prigioniero. La creatura liberata compare in uno spazio non occupato in vista dello specchio ma rivolta dalla parte opposta. Se lo specchio viene infranto, tutte le creature che contiene sono liberate e ricompaiono in uno spazio non occupato in sua prossimità.

Mentre ti trovi entro 1 metro dallo specchio, puoi usare due azioni per pronunciare il nome di una delle creature intrappolate al suo interno o richiamare un particolare numero di cella. La creatura nominata o contenuta nella cella nominata appare come immagine sulla superficie dello specchio. Dopodiché tu e la creatura nominata potete comunicare normalmente.

In un modo simile, puoi usare due azioni per pronunciare una seconda parola di comando e liberare una delle creature intrappolate nello specchio. La creatura liberata compare, insieme a tutte le sue proprietà, nello spazio non occupato più vicino allo specchio e rivolta nella direzione opposta a esso.

\oggettomagico{Spilla della Difesa}

\textbf{Rarità:} Non Comune; \textbf{Costo:} 7500 mo

la \textbf{spilla} può assorbire 101 danni da incantesimi di Forza, poi perde le sue proprietà magiche.

\oggettomagico{Stivali Alati}

\textbf{Rarità:} Raro; \textbf{Costo:} 15000 mo

mentre indossi questi \textbf{stivali}, hai una velocità di volo pari alla tua velocità di movimento. Puoi usare questi stivali per volare per un massimo di 4 ore, tutte insieme o divise in brevi voli, ciascuno dei quali impiega un minimo di 1 minuto di durata. Se la durata termina mentre stai volando, scendi alla velocità di 9 metri per round finché non atterri. Gli stivali recuperano 2 ore di capacità di volo ogni alba.

\oggettomagico{Stivali Danzanti}

questi \textbf{stivali} maledetti funzionano come altri stivali magici. Tuttavia, quando il personaggio entra in combattimento o tenta di fuggire da un potenziale combattimento, egli viene affetto da un incantesimo danza irresistibile, senza possibilità di Tiro Salvezza. E' possibile rimuovere gli Stivali Danzanti con l'incantesimo Rimuovi Maledizione o Desiderio.

\oggettomagico{Stivali degli Elfi}

\textbf{Rarità:} Non Comune; \textbf{Costo:} 3000 mo

mentre indossi questi \textbf{stivali}, i tuoi passi non emettono suoni, quale che sia la superficie che stai attraversando. Il muoverti silenziosamente non ti obbliga a trattare il terreno come difficile (ma potrebbe comunque esserlo).

\oggettomagico{Stivali dell'Inverno}

\textbf{Costo:} 10000 mo

mentre indossi questi \textbf{stivali} hai resistenza ai danni da freddo, ignori il terreno difficile prodotto da neve o ghiaccio. Puoi tollerare le temperature fino ai -45° C senza bisogno di ulteriori protezioni. Se indossi abiti pesanti, puoi tollerare temperature fino a -75° C.

\oggettomagico{Stivali della Corsa e del Salto}

\textbf{Rarità:} Non Comune; \textbf{Costo:} 5000 mo

mentre indossi questi \textbf{stivali}, la tua velocità di movimento diventa 9 metri, a meno che non sia superiore, e la tua velocità non viene ridotta qualora tu sia ingombrato o stia indossando un'armatura pesante. Inoltre, salti tre volte la normale distanza, fino ad un massimo di 9 metri.

\oggettomagico{Stivali della Levitazione}

\textbf{Rarità:} Raro; \textbf{Costo:} 5000 mo

mentre indossi questi \textbf{stivali}, puoi usare a volontà due azioni per lanciare l'incantesimo levitazione su di te.

\oggettomagico{Stivali della Velocita'}

\textbf{Rarità:} Raro; \textbf{Costo:} 5000 mo

mentre indossi questi \textbf{stivali}, puoi usare una Reazione per ottenere un Azione di Movimento in più. La capacità è usabile per un massimo di 10 minuti al giorno. La capacità si ricarica all'alba.

\oggettomagico{Talismano del Bene puro}

\textbf{Rarità:} Leggendario; \textbf{Costo:} 50000 mo

un Devoto di Gradh o Sumkjr in possesso di questo oggetto può far sì che una voragine di fiamme appaia ai piedi di un Devoto di Calicante o Shayalia entro 30 m. La vittima viene inghiottita dal fuoco e precipita urlando verso il centro della Terra. Un talismano del bene puro ha 6 cariche e non può essere ricaricato. Se un Devoto di Calicante o Shayalia lo tocca subisce 6d6 ferite. Qualsiasi altro Devoto o Seguace non subisce alcun effetto. Il \textbf{Talismano} pulsa di luce entro 36 metri da un Devoto o Seguace di Calicante o Shayalia.

\oggettomagico{Talismano del Male estremo}

\textbf{Rarità:} Leggendario; \textbf{Costo:} 50000 mo

questo \textbf{talismano} funziona esattamente come il talismano del bene puro ma con i Patroni invertiti.

\oggettomagico{Talismano della Protezione dal Veleno}

\textbf{Rarità:} Raro; \textbf{Costo:} 5000 mo

mentre indossi questo \textbf{pendente} i veleni non hanno alcun effetto su di te. Sei immune alla condizione avvelenato e hai immunità ai danni da veleno.

\oggettomagico{Talismano della Salute}

\textbf{Rarità:} Raro; \textbf{Costo:} 5000 mo

mentre indossi questo \textbf{pendente} sei immune alla possibilità di contrarre qualsiasi malattia. Se sei già infetto da una malattia, i suoi effetti vengono sospesi finché indossi questo pendente.

\oggettomagico{Talismano della Sfera}

\textbf{Rarità:} Leggendario; \textbf{Costo:} 75000 mo

quando effettui una prova di Arcana per controllare una sfera dell'annientamento mentre stai impugnando questo \textbf{talismano} hai un bonus di 5. Inoltre, quando inizi il round con il controllo di una sfera dell'annientamento, puoi usare due azioni per farla levitare di 3 metri più un numero di metri aggiuntivi pari a 3 x il tuo valore di Intelligenza.

\oggettomagico{Tamburi del Panico}

\textbf{Rarità:} Non Comune; \textbf{Costo:} 1500 mo

questi \textbf{tamburi} sono simili a timpani (piccoli strumenti a percussione facilmente trasportabili). Si trovano a coppie e hanno un aspetto poco appariscente. Se vengono suonati entrambi, tutte le creature entro 72 m (tranne quelle all'interno di un cerchio di 3 m centrato sui tamburi) vengono assalite da Paura e fuggono per 30 round alla massima velocità. È consentito un Tiro Salvezza su Volontà a DC 21 per salvarsi dagli effetti.

\oggettomagico{Tamburi dello Stordimento}

\textbf{Rarità:} Raro

questi due \textbf{tamburi} accoppiati somigliano ai tamburi del panico; quando vengono suonati entrambi, tutte le creature entro 3 m devono riuscire in un Tiro Salvezza su Tempra DC 21 essere stordite per 2d4 round. Tutte le creature entro 21 m sono immediatamente assordate. Gli incantesimi ristorare superiore, guarigione, rigenerazione o effetti simili possono curare la sordità.

\oggettomagico{Tappeto Volante}

\textbf{Rarità:} Molto Raro; \textbf{Costo:} 15000 mo

puoi pronunciare la parola di comando del \textbf{tappeto} con due azioni per far fluttuare e volare il tappeto. Esso si muove in base alle direzioni indicategli a voce, purché ti trovi entro 9 metri da esso.

Esistono quattro taglie di tappeto volante. Il Narratore sceglie la taglia del tappeto o la determina casualmente.

\medskip

\noindent\begin{tabular}{cccc}
	\toprule
\textbf{d100} & \multicolumn{1}{c}{\textbf{Taglia}} & \textbf{Capacità} & \multicolumn{1}{c}{\textbf{Velocità}} \\
& \multicolumn{1}{c}{\textbf{(cm)}} &\textbf{(Ing.)} & \multicolumn{1}{c}{\textbf{di Volo}} \\
\toprule
\rowcolor{gray!20}01--20 & 90 × 150 & 100 kg / 25 & 23 metri \\
21--55 & 120 × 180 & 200 kg / 50 & 18 metri \\
\rowcolor{gray!20}56--80 & 150 × 210 & 300 kg / 75 & 12 metri \\
81--100 & 180 × 270 & 400 kg / 100 & 9 metri \\
\end{tabular}

\medskip
Il valore di Capacità indica sia il peso trasportato che l'Ingombro. Il tappeto può trasportare fino al doppio del carico indicato sulla tabella, ma vola a velocità dimezzata se trasporta di più.

\oggettomagico{Arma Titanica}

\textbf{Aura:} Trasmutazione moderata; \textbf{Costo:} 3000 mo

\textbf{Requisiti:} Creare Oggetti Magici 2, Ingrandire/Ridurre

Quest'\textbf{arma} gigante è lunga 3 m e pesa quasi 40 kg (8 Ingombro). Se usato come un'arma ha un bonus +2 al colpire e infligge 2 categorie di dado superiore di danno. Può essere usato anche per piantare rapidamente pali grossi come tronchi d'albero e per divellere con pochi colpi porte e cancelli. A causa della fatica nell'usarla non può essere brandita per più di 10 round al giorno.

\oggettomagico{Armatura Titanica}

\textbf{Aura:} Trasmutazione moderata; \textbf{Costo:} 15000 mo

\textbf{Requisiti:} Creare Oggetti Magici 2, Ingrandire; \textbf{Rarità:} Rara

Un'\textbf{armatura} con la proprietà Titanica è fuori misura, anche se l'effetto è solo esteriore e l'interno accoglie una creatura come di norma, senza necessitare modifiche. Una creatura che indossa un'armatura Titanica è considerata di una categoria di taglia superiore, questo anche al fine dell'uso di oggetti ed armi o dell'essere influenzati da attacchi speciali che dipendono dalla taglia, come Inghiottire e Travolgere.

\oggettomagico{Tocco Fantasma}

\textbf{Aura:} Evocazione moderata; \textbf{Costo:} 3000 mo

\textbf{Requisiti:} Creare Oggetti Magici 2, Desiderio Limitato

Un'\textbf{arma} del Tocco Fantasma infligge un danno critico aggiuntivo quando colpisce le creature con Movimento Incorporeo. Finché imbracci una arma Tocco Fantasma puoi vedere le creature invisibili.

\oggettomagico{Tomo del Pensiero Limpido}

\textbf{Rarità:} Molto Raro; \textbf{Costo:} 15000 mo

questo \textbf{libro} contiene esercizi di memoria e logica, e le sue parole sono soffuse di magia. Se trascorri 48 ore in un periodo di 6 giorni o meno a studiare i contenuti del libro e praticarne le indicazioni, il tuo punteggio di Intelligenza aumenta di 1. Una volta letto il manuale perde la sua magia, per recuperarla dopo un secolo.La \emph{versione maledetta} di questo manuale pur sembrando assolutamente identico a quello originale fa perdere un punto di Intelligenza.

\oggettomagico{Tomo dell'Autorita' e dell'Influenza}

\textbf{Rarità:} Molto Raro; \textbf{Costo:} 15000 mo

questo \textbf{libro} contiene indicazioni su come influenzare e affascinare il prossimo, e le sue parole sono soffuse di magia. Se trascorri 48 ore in un periodo di 6 giorni o meno a studiare i contenuti del libro e praticarne le indicazioni, il tuo punteggio di Carisma aumenta di 1. Una volta letto il manuale perde la sua magia, per recuperarla dopo un secolo. La \emph{versione maledetta} di questo manuale pur sembrando assolutamente identico a quello originale fa perdere un punto di Carisma.

\oggettomagico{Tomo della Comprensione}

\textbf{Rarità:} Molto Raro; \textbf{Costo:} 15000 mo

questo \textbf{libro} contiene esercizi di intuizione e discernimento, e le sue parole sono soffuse di magia. Se trascorri 48 ore in un periodo di 6 giorni o meno a studiare i contenuti del libro e praticarne le indicazioni, il tuo punteggio di Saggezza aumenta di 1, e così fa il tuo punteggio massimo per quella caratteristica. Una volta letto il manuale perde la sua magia, per recuperarla dopo un secolo. La \emph{versione maledetta} di questo manuale pur sembrando assolutamente identico a quello originale fa perdere un punto di Saggezza.

\oggettomagico{Tonante}

\textbf{Aura:} Necromantica debole; \textbf{Costo:} 3000 mo

\textbf{Requisiti:} Creare Oggetti Magici, Cecità/Sordità

Un'\textbf{arma} Tonante crea un tremendo frastuono simile a quello di un tuono, quando mette a segno un Colpo Critico. L'energia sonora non danneggia chi tiene in mano l'arma e infligge 1d8 danni sonori addizionali per ogni Colpo Critico andato a segno. Chi è soggetto a un Colpo Critico da un'arma Tonante deve effettuare un Tiro Salvezza su Tempra con DC 14 o resta Sordo in modo permanente.

\oggettomagico{Trasformante}

\textbf{Aura:} Trasmutazione moderata; \textbf{Costo:} 5000 mo

\textbf{Requisiti:} Creare Oggetti Magici 2, Creazione Maggiore

Questa capacità si può aggiungere solo ad \textbf{armi} da mischia. Un'arma Trasformante altera la sua forma a comando di chi la impugna, diventando una qualsiasi altra arma da mischia con la stessa dimensione. Ad esempio, una Spada Lunga trasformante può assumere la forma di una qualsiasi altra arma da mischia a una mano media, come una Scimitarra, un Flagello od un Tridente, ma non un'arma da mischia piccola o grande (come una Spada Corta o uno Spadone a due mani).

L'arma conserva tutte le sue capacità, compresi bonus e capacità speciali dell'arma, ad eccezione di quelle proibite dalla sua nuova forma attuale. Se lasciata incustodita, l'arma ritorna alla sua forma originaria.

\oggettomagico{Trovacose}

\textbf{Aura:} Divinazione leggera; \textbf{Costo:} 1000 mo

\textbf{Requisiti:} Creare Oggetti Magici, Localizza oggetto

Questa capacità concede a chi impugna quest'\textbf{arma} di lanciare l'incantesimo \hyperlink{Localizza Oggetto}{Localizza Oggetto} una volta al giorno

\oggettomagico{Tunica degli Occhi}

\textbf{Rarità:} Leggendario; \textbf{Costo:} 30000 mo

questa \textbf{vestaglia} è adornata da un disegno di occhi. Mentre la indossi, ottieni i seguenti benefici:

- La vestaglia ti permette di vedere in tutte le direzioni e hai +1d6 alle prove di Consapevolezza basate sulla vista.

- Hai scurovisione con una portata di 18 metri.

- Puoi vedere creature e oggetti invisibili, oltre che nel Piano Etereo, fino a una gittata di 36 metri.

Gli occhi della vestaglia non possono essere chiusi o distolti, e mentre indossi questa vestaglia non viene mai considerato a occhi chiusi o distolti.

L'incantesimo \emph{luce} lanciato sulla vestaglia o l'incantesimo luce diurna lanciato entro 1 metro dalla vestaglia ti rendono accecato per 1 minuto. Al termine di ciascun tuo round, puoi effettuare un tiro Salvezza su Tempra (DC 15 per luce o DC 19 per luce diurna), ponendo fine alla condizione accecato in caso lo superi.

\oggettomagico{Tunica degli Oggetti Utili}

\textbf{Rarità:} Non Comune; \textbf{Costo:} 300 mo

mentre indossi questa \textbf{vestaglia} ricoperta da toppe di varie forme e colori, puoi usare due azioni per staccare una delle toppe, facendola diventare l'oggetto o la creatura che rappresenta. Quando l'ultima toppa viene rimossa, la vestaglia diventa un indumento normale. La vestaglia possiede due di ciascuna delle seguenti toppe:

Asta di 3 metri, Corda di canapa (15 metri, arrotolata), Lanterna a lente sporgente (piena e accesa), Pugnale, Sacco, Specchio d'acciaio.

Inoltre, la vestaglia ha 4d4 altre toppe. Il Narratore sceglie le toppe o le determina a caso, scegliendo tra proprietà totalmente diverse da quelle già presenti.

Tira un d100 sulla tabella seguente per scoprire le proprietà delle altre 4d4 toppe della vestaglia degli oggetti utili.

%\end{multicols}

\medskip

\noindent\begin{tabularx}{\linewidth}{lX}
	\toprule
\rowcolor{gray!20}\textbf{d100} & \textbf{Effetto}\\
\toprule
01-08 &Borsello con 100 mo.\\
\rowcolor{gray!20}09-15& Forziere d'argento (lungo 30 cm, largo e profondo 15 cm) del valore di 500 mo.\\
16-22& Porta di ferro (larga e alta massimo 3 metri, sbarrata dal lato di tua scelta), che puoi piazzare su qualsiasi apertura a portata; si adatta per entrare nell'apertura, fissandosi e creando dei cardini.\\
\rowcolor{gray!20}23-30 &10 gemme del valore di 100 mo ciascuna.\\
31-44 &Una scala di legno (8 metri).\\
\rowcolor{gray!20}45-51 &Un Saurovallo da Galoppo con sacche da sella \\
\end{tabularx}
\noindent\begin{tabularx}{\linewidth}{lX}
\toprule
\rowcolor{gray!20}\textbf{d100} & \textbf{Effetto}\\
\toprule
52-59 & Fossa (un cubo di 3 metri di spigolo), che puoi piazzare sul terreno entro 3 metri da te.\\
\rowcolor{gray!20}60-68 &4 pozioni di guarigione. \\
69-75 &Barca a remi (lunga 4 metri).\\
\rowcolor{gray!20}76-83& Pergamena degli incantesimi contenente un incantesimo di livello dal 1° al 3°.\\
84-90& Due mastini.\\
\rowcolor{gray!20}91-96 &Finestra (60 x 120 cm, profonda massimo 60 cm), che puoi piazzare su qualsiasi superficie verticale a portata.\\
97-100 &Ariete portatile.
\end{tabularx}

%\begin{multicols}{2}

\medskip

\oggettomagico{Tunica dei Colori Scintillanti}

\textbf{Rarità:} Molto Raro; \textbf{Costo:} 6000 mo

questa \textbf{vestaglia} ha 3 cariche, e recupera 1 carica spesa ogni giorno all'alba. Quando la indossi, puoi usare due azioni e spendere 1 carica per far sì che l'indumento produca una trama mutevole di colori abbaglianti fino al termine del tuo prossimo round. Durante questo periodo, la vestaglia emana luce intensa in un raggio di 9 metri e luce fioca per 18 metri. Le creature che ti vedono hanno -1d6 ai Tiri per Colpire contro di te. Inoltre, qualsiasi creatura sotto la luce intensa e che ti veda quando il potere della vestaglia viene attivato, deve superare un Tiri Salvezza su Volontà con DC 17 o restare stordita fino al termine dell'effetto.

\oggettomagico{Tunica del Mimetismo}

\textbf{Rarità:} Raro; \textbf{Costo:} 2500 mo

quando indossa questa \textbf{tunica}, un personaggio capisce immediatamente il suo potere. Una tunica del mimetismo permette al personaggio di confondersi con l'ambiente circostante, qualunque esso sia, e di nascondersi. Ha +1d6 nelle prove di Furtività per nascondersi nelle ombre. Il possessore può assumere a volontà l'aspetto di un altro umanoide, come per l'incantesimo \emph{Alterare se stesso} (Cambio Aspetto). In questo caso, gli amici del possessore e chi lo conosce molto bene sono istintivamente consci della sua vera identità.

\oggettomagico{Tunica dell'Arcimago}

\textbf{Rarità:} Leggendario; \textbf{Costo:} 8000 mo

questo \textbf{abito} apparentemente normale può essere giallo (01-45 su 1d100), grigio (46-75) o nero (76-00). Può essere indossato solo da un incantatore con Competenza Magica 2 o superiore. Conferisce le seguenti bonus:

- Difesa naturale +4

- +2 ai Tiri Salvezza contro incantesimi e oggetti magici

\oggettomagico{Tunica dell'Indebolimento}

\textbf{Rarità:} Raro; \textbf{Costo:} 5000 mo

una \textbf{tunica} dell'indebolimento sembra un abito magico di un altro tipo. Appena un personaggio la indossa, la sua Forza e la sua Intelligenza scendono a -3 ed egli perde la capacità di lanciare incantesimi. La tunica può essere rimossa facilmente, ma per ripristinare gli attributi occorre \emph{Rimuovi Maledizione} seguito da \emph{guarigione}.

\oggettomagico{Tunica delle Stelle}

\textbf{Rarità:} Raro; \textbf{Costo:} 60000 mo

mentre indossi questa \textbf{vestaglia}, ottieni un bonus di +1 ai Tiri Salvezza. Sei stelle, posizionate sulla parte superiore frontale della vestaglia, sono più grosse delle altre. Mentre indossi questa vestaglia, puoi usare 1 Azione per estrarre una delle stelle e usarla per lanciare Dardo arcano (1d4+1). Ogni giorno al tramonto, la stella rimossa ricompare sulla vestaglia. Mentre indossi la vestaglia, puoi usare due azioni per entrare nel Piano Astrale assieme a tutto ciò che indossi o trasporti. Resterai lì fino a quando userai due azioni per ritornare al piano in cui ti trovavi prima. Ricompari nell'ultimo spazio da te occupato, o se quello spazio è occupato, nello spazio non occupato più vicino.

\oggettomagico{Turibolo dell'Evocazione maledetta}

\textbf{Rarità:} Raro

questo \textbf{turibolo} ha l'aspetto di, e sembra funzionare come, un turibolo elementale dell'aria. Tuttavia, una volta acceso è impossibile spegnerlo per 1d4 round. In ciascun round un elementale dell'aria emerge ed attacca tutte le creature vicine.

\oggettomagico{Turibolo Elementale dell'aria}

\textbf{Rarità:} Raro; \textbf{Costo:} 1500 mo

questo \textbf{turibolo} può essere usato per evocare e controllare un elementale dell'aria in modo analogo all'incantesimo evoca elementale. È necessario preparare l'oggetto magico e condurre un rituale per un round prima dell'evocazione vera e propria, che richiede un round. Dopo che l'elementale è stato evocato, occorre mantenere la concentrazione per potergli impartire gli ordini.

\oggettomagico{Unguento di Ljust}

\textbf{Rarità:} Non Comune; \textbf{Costo:} 5000 mo

questa \textbf{giara} di vetro, 7,5 centimetri di diametro, contiene 1d4 + 1 dosi di una densa mistura. La giara e i suoi contenuti pesano 250 grammi, ingombro 1. Con due azioni, si può inghiottire o applicare sulla pelle una dose di unguento. La creatura che lo riceve recupera 2d8 + 2 Punti Ferita, smette di essere avvelenata e viene curata da qualsiasi malattia.

\oggettomagico{Vampira}

\textbf{Aura:} Necromantica moderata; \textbf{Costo:} 8000 mo

\textbf{Requisiti:} Creare Oggetti Magici 2, Tocco Vampiro

Quando attacchi una creatura con quest'\textbf{arma} magica e ottieni un critico al Tiro per Colpire, il bersaglio, a parte i costrutti e i non morti, subisce un danno critico aggiuntivo e chi imbraccia l'arma recupera 1d6 Punti Ferita.

\oggettomagico{Vano Portatile}

\textbf{Rarità:} Raro; \textbf{Costo:} 10000 mo

questo elegante \textbf{tessuto} nero, soffice come la seta, si piega fino alle dimensioni di un fazzoletto e si dispiega fino ad cerchio di 2 metri di diametro. Puoi usare due azioni per dispiegare un Vano portatile e piazzarlo sopra o contro una superficie solida, sulla quale il Vano portatile crea un foro extradimensionale profondo 3 metri. Lo spazio cilindrico all'interno del foro si trova su di un piano diverso, e quindi non può essere usato per aprire dei passaggi. Qualsiasi creatura all'interno di un Vano portatile aperto può uscirne fuori arrampicandosi fuori di esso.

Puoi usare due azioni per chiudere un Vano portatile prendendo i margini del tessuto e ripiegandolo. Piegare il tessuto chiude il Vano, e qualsiasi creatura od oggetto al suo interno rimane nello spazio extradimensionale. Non importa quello che contiene, il Vano non pesa nulla.

Se il Vano viene ripiegato, una creatura all'interno dello spazio dimensionale del Vano può usare due azioni per effettuare un Tiro Salvezza Tempra con Forza DC 10. Se la prova riesce, la creatura riesce a liberarsi e ricompare entro 1 metro dal Vano portatile o della creatura che lo trasporta. Una creatura che respira può sopravvivere all'interno di un buco portatile chiuso per un massimo di 10 minuti, dopodiché iniziare a soffocare.

Piazzare un Vano portatile all'interno dello spazio extradimensionale creato da una borsa conservante, uno zainetto pratico o simile oggetto distrugge istantaneamente entrambi gli oggetti e apre un portale verso il Piano Astrale. Qualsiasi creatura entro 3 metri dal portale viene risucchiata al suo interno e depositata in un luogo casuale del Piano Astrale. Poi il portale scompare.

\oggettomagico{Arma della Velocita'}

\textbf{Aura:} Trasmutazione moderata; \textbf{Costo:} 15000 mo

\textbf{Requisiti:} Creare Oggetti Magici 2, Velocità

Quando compie più attacchi (2 Azioni), chi impugna un'\textbf{arma} di Velocità può compiere un attacco addizionale con l'arma senza usare ulteriori Azioni. L'attacco aggiuntivo non ha le penalità degli attacchi multipli. Questa capacità non è cumulabile con incantesimi o effetti simili.

\oggettomagico{Ventaglio Arcano}

\textbf{Rarità:} Non Comune; \textbf{Costo:} 1500 mo

mentre impugni questo \textbf{ventaglio}, puoi usare due azioni per lanciare tramite esso l'incantesimo folata di vento (DC del Tiro Salvezza 15). Una volta usato, il ventaglio
non dovrebbe essere usato di nuovo fino alla prossima alba. Ogni volta che venga usato prima di allora, c'è una probabilità cumulativa del 20\% che non funzioni e si rompa in inutili brandelli privi di magia.

\oggettomagico{Verga del Colpo possente}

\textbf{Rarità:} Molto Rara; \textbf{Costo:} 30000 mo

una \textbf{verga} del colpo possente infligge 1d8+3 ferite, e funziona come una mazza leggera magica +3. Quando la verga è usata contro i golem, consuma 1 carica per colpo inflitto, ed infligge 2d8+6 ferite. Si noti che quando la verga è usata come arma contro un golem un Tiro per Colpire Critico lo annienta istantaneamente. In aggiunta, questa verga infligge ferite addizionali a immondi e non morti. Quando si attaccano questi mostri, un Tiro per Colpire Critico causa il consumo di 1 carica, e la verga infligge il triplo delle ferite.

\oggettomagico{Verga dell'Ammaliamento}

\textbf{Rarità:} Raro; \textbf{Costo:} 28000 mo

spendendo 1 carica, il possessore della \textbf{verga} può lanciare dominare bestie, con 2 cariche \hyperlink{Dominare Persone}{Dominare Persone} e con 3 cariche dominare mostri.

\oggettomagico{Verga dell'Assorbimento}

\textbf{Rarità:} Molto Raro; \textbf{Costo:} 50000 mo

mentre impugni questa \textbf{verga}, puoi usare una Azione per assorbire un incantesimo che prenda come bersaglio solo te e privo di un'area di effetto. L'effetto dell'incantesimo assorbito è cancellato, e l'energia dell'incantesimo (non l'incantesimo stesso) viene assorbita dalla verga. Nel corso della sua esistenza la verga può assorbire e contenere fino ad una somma di 31 Livelli di incantesimi. Una volta che la verga ha assorbito 8 incantesimi (max livello 4), non ne potrà più assorbire. Se sei il bersaglio di un incantesimo che la verga non può contenere, la verga non ha alcun effetto sull'incantesimo. Quando prendi in mano la verga, sai quanti incantesimi la verga ha assorbito finora. Se sei un incantatore e impugni la verga, puoi convertire tutta l'energia contenuta per avere 10 Punti Magia in più.

\oggettomagico{Verga della Forza Sovrana}

\textbf{Costo:} 50000 mo

leggendaria, questa \textbf{verga} ha una testa flangiata, e funziona come una mazza magica che conferisce un bonus di +3 ai Tiri per Colpire e danno effettuati con essa. La verga ha delle proprietà associate ai sei diversi pulsanti che sono disposti lungo il manico. Possiede anche altre tre proprietà descritte di seguito.

\textbf{Sei Pulsanti}. Puoi premere uno dei sei pulsanti della verga con due azioni. L'effetto del pulsante dura finché non premi un pulsante differente o premi di nuovo lo stesso pulsante, facendo tornare la verga alla sua forma normale.

- Se premi il \emph{pulsante 1}, la verga diventa un'arma lingua di fuoco, e una lama infuocata fuoriesce dall'estremità opposta alla testa flangiata.

- Se premi il \emph{pulsante 2}, la testa flangiata della verga si ripiega e fuoriescono due lame a mezzaluna, che trasformano la verga in un'ascia da battaglia magica che conferisce un bonus di +3 ai Tiri per Colpire e danno effettuati con essa.

- Se premi il \emph{pulsante 3}, la testa flangiata della verga si ripiega, e una punta di lancia esce fuori dall'estremità della verga, mentre il manico si allunga fino a 1,8 metri, trasformando la verga in una lancia magica che conferisce un bonus di +3 ai Tiri per Colpire e danno effettuati con essa.

- Se premi il \emph{pulsante 4}, la verga si trasforma in un'asta per scalare lunga fino a 15 metri, come richiesto da te. Sulle superfici dure come il granito, uno spuntone sul fondo e tre in cima tengono l'asta fissa sul posto. Sbarre orizzontali lunghe 7,5 centimetri si dipanano lungo i lati della verga, a 30 centimetri di distanza l'uno dall'altro, per formare una scala. L'asta può sostenere 2000 chili. Un peso superiore o la mancanza di un ancoraggio solido fa sì che la verga torni alla sua forma normale.

- Se premi il \emph{pulsante 5}, la verga si trasforma in un ariete da sfondamento e conferisce a chi lo usa un bonus di +10 alle prove di Forza effettuate per sfondare porte, barricate o altre barriere.

- Se premi il \emph{pulsante 6}, la verga assume o rimane nella sua forma normale e indica il nord magnetico (non accade nulla se questa funzione della verga viene impiegata in zone prive di un nord magnetico). La verga ti fornisce anche un'approssimativa conoscenza della profondità sottoterra e della tua altezza sul livello del mare.

\emph{Risucchiare Vita}. Quando colpisci una creatura con un attacco in mischia utilizzando la verga, puoi obbligare il bersaglio a effettuare un tiro Salvezza su Tempra con DC 21. Se lo fallisce, il bersaglio subisce 4d6 danni da Vuoto aggiuntivi e vengono tolti dal massimo dei Punti Ferita, tu recuperi un numero di Punti Ferita pari alla metà del danno da Vuoto inflitto. Una volta usata, questa proprietà non più essere usata fino all'alba del giorno successivo.

\textbf{Paralizzare}. Quando colpisci una creatura con un attacco da mischia utilizzando la verga, puoi obbligare il bersaglio a effettuare un Tiro Salvezza su Tempra con DC 21. Se lo fallisce, il bersaglio è paralizzato per 1 minuto. Il bersaglio può ripetere il Tiro Salvezza al termine di ciascun suo round, terminando l'effetto su di sé in caso lo superi. Una volta usata, questa proprietà non può più essere usata fino all'alba del giorno successivo.

\emph{Terrorizzare}. Mentre impugni questa verga, puoi obbligare ogni creatura che vedi entro 9 metri da te a effettuare un Tiri Salvezza su Volontà con DC 21. Se lo fallisce, il bersaglio è spaventato da te per 1 minuto. Il bersaglio spaventato può ripetere il Tiro Salvezza al termine di ciascun suo round, terminando l'effetto su di sé in caso lo superi. Una volta usata, questa proprietà non può più essere usata fino all'alba del giorno successivo.

Questa verga non può essere ricaricata. Quando le cariche finiscono rimane una

\oggettomagico{Verga della Prontezza}

\textbf{Rarità:} Molto Raro; \textbf{Costo:} 25000 mo

questa \textbf{verga} dalla testa flangiata ha le seguenti proprietà.

\emph{Prontezza}. Mentre impugni questa verga, hai +2 alle prove di Saggezza e ai tiri di iniziativa.

\emph{Incantesimi}. Mentre impugni questa verga, puoi usare due azioni per lanciare tramite essa uno dei seguenti incantesimi: \hyperlink{Bacchettadell'IndividuazionedelMagico}{Individuazione del Magico}, \hyperlink{Individuazione delle Malattie e dei Veleni}{Individuazione delle Malattie e dei Veleni} o \hyperlink{Vedere l'invisibile}{Vedere l'invisibile}.

\emph{Aura Protettiva}. Con due azioni, puoi piantare l'estremità appuntita della verga nel terreno. A quel punto la testa della verga irradierà luce intensa in un raggio di 18 metri e luce fioca per 36 metri. All'interno di questa luce intensa, tu e qualsiasi creatura a te amichevole otterrete un bonus di +1 alla Difesa e ai Tiri Salvezza e potrete percepire la posizione di qualsiasi creatura invisibile ostile che si trovi anch'essa all'interno della luce intensa. La testa della verga smette di emettere luce e termina l'effetto dopo 10 minuti, o quando una creatura usa due azioni per estrarre la verga dal terreno. Questa proprietà non può essere usata di nuovo fino all'alba del giorno successivo.

\oggettomagico{Verga della Sicurezza}

\textbf{Rarità:} Molto Raro; \textbf{Costo:} 90000 mo

mentre impugni questa \textbf{verga}, puoi usare due azioni per attivarla. Di conseguenza la verga trasporta te e fino ad altre 199 altre creature consenzienti visibili in un paradiso collocato in uno spazio extraplanare. Sarai tu a scegliere la forma di questo paradiso. Potrebbe essere un placido giardino, una gradevole radura, un'allegra taverna, un immenso palazzo, un'isola tropicale, o una fantastica fiera o qualsiasi altra cosa tu riesca a immaginare. Quale che sia la sua natura, il paradiso contiene cibo e bevande sufficienti ad alimentare i suoi visitatori. Tutto ciò con cui si può interagire nello spazio extraplanare può esistere solo al suo interno.

Per ogni ora trascorsa in questo paradiso, un visitatore recupera Punti Ferita come se avesse avesse riposato una notte. Inoltre, finché le creature restano nel paradiso non invecchiano, sebbene il tempo trascorra normalmente. I visitatori possono restare nel paradiso per un massimo di 200 giorni diviso il numero di creature presenti (arrotondare per difetto).

Quando il tempo termina o usi due azioni per farlo terminare, tutti i visitatori ricompaiono nel luogo da loro occupato quando hai attivato la verga, o nello spazio non occupato più vicino a quello. La verga non potrà essere usata di nuovo prima che siano passati dieci giorni.

\oggettomagico{Verga della Sovranita'}

\textbf{Rarità:} Raro; \textbf{Costo:} 16000 mo

puoi usare due azioni e presentare la \textbf{verga} e richiedere obbedienza a ciascuna creatura visibile entro 36 metri da te di tua scelta. Ogni bersaglio deve superare un Tiri Salvezza su Volontà con DC 17 o restare Affascinato da te per 8 ore. Mentre è affascinata in questa maniera, la creatura ti considera un capo fidato. Se le viene recato danno da te o dai tuoi compagni, o le viene ordinato di fare qualcosa contrario alla sua natura, il bersaglio smetterà di essere affascinato in questa maniera. La verga non può essere usata di nuovo prima della prossima alba.

\oggettomagico{Verga Inamovibile}

\textbf{Rarità:} Non Comune; \textbf{Costo:} 5000 mo

questa \textbf{verga} di ferro piatta ha un pulsante a un'estremità. Puoi usare due azioni per premere il pulsante, che fa sì che la verga resti magicamente fissata sul posto. Fino a quando tu o un'altra creatura userete due azioni per premere di nuovo il pulsante, la verga non si muoverà, anche se dovesse sfidare la gravità. La verga può sostenere fino a 4000 chili di peso. Un peso maggiore fa sì che la verga si disattivi e cada. Una creatura può usare due azioni per effettuare un Tiro Salvezza Tempra con Forza DC 30, spostando la verga di 3 metri in caso di successo.

\oggettomagico{Verga Tentacolare}

\textbf{Rarità:} Raro; \textbf{Costo:} 5000 mo

questa \textbf{verga} è un'arma magica che termina in tre tentacoli di cuoio. Mentre impugni la verga, puoi usare due azioni per dirigere ciascun tentacolo per attaccare una creatura visibile entro 3 metri da te. Ogni tentacolo effettua un Tiro per Colpire da mischia con un bonus di +9. Se colpisci, il tentacolo infligge 1d6 danni contundenti. Se colpisci un bersaglio con tutti e tre i tentacoli, esso deve effettuare un Tiro Salvezza su Tempra con DC 15. Se lo fallisce, la velocità della creatura è dimezzata, ha -1d6 ai Tiri Salvezza di Riflessi, e per 1 minuto non può usare le sue reazioni. Inoltre, durante ciascun suo round, egli può effettuare due azioni o due azioni ma non entrambe. Il bersaglio può ripetere il Tiro Salvezza al termine di ciascun suo round, terminando l'effetto su di sé in caso lo superi.

\oggettomagico{Vorpal}

\textbf{Aura:} Invocazione molto forte; \textbf{Costo:} 150000 mo

\textbf{Requisiti:} Creare Oggetti Magici 4

Pur essendo un \textbf{arma} magica +1 viene considerata un arma magica +5 per valutare immunità e bonus al Tiro per Colpire e danno. Inoltre, l'arma ignora la resistenza ai danni taglienti. Quando attacchi una creatura che abbia almeno una testa con quest'arma e ottieni almeno 2 Tiri Critici, tagli una delle teste della creatura. La creatura muore se non può sopravvivere senza la perdita della testa.

Una creatura è immune a questo effetto se è immune ai danni taglienti, non possiede o non ha bisogno di una testa o il Narratore decide che la creatura è troppo grossa perché la sua testa sia recisa da quest'arma.

Una creatura del genere subisce invece 6d8 danni taglienti aggiuntivi dal colpo subito.

\oggettomagico{Vulnerabilita'}

\textbf{Aura:} Necromantica moderata; \textbf{Costo:} 3000 mo

\textbf{Requisiti:} Creare Oggetti Magici, \hyperlink{Scagliare Maledizione}{Scagliare Maledizione}; \textbf{Rarità:} Rara

Mentre indossi questa \textbf{armatura}, hai resistenza a uno dei seguenti tipi di danno: contundente, perforante o tagliente. Il Narratore sceglie il tipo. L'armatura è maledetta, mentre sei maledetto, hai vulnerabilità a due dei tre tipi di danno associati con l'armatura (che non sia quello a cui hai resistenza).

\oggettomagico{Zainetto Pratico}

\textbf{Rarità:} Raro; \textbf{Costo:} 7000 mo

questo \textbf{zaino} ha una sacca centrale e due laterali, ciascuna delle quali è in realtà uno spazio extradimensionale.
Le due sacche laterali funzionano come due Borse Conservanti di Tipo I, la grande sacca centrale funziona come una Borsa Conservante di Tipo III.

Piazzare un oggetto all'interno dello zainetto segue le normali regole di interazione con gli oggetti. Recuperare un oggetto dallo zainetto richiede l'uso di due azioni. Quando cerchi un oggetto nello zainetto, questo magicamente si troverà sempre in cima alla pila degli oggetti che questo contiene.

Lo zainetto ha alcune limitazioni. Se sovraccarico, o un oggetto affilato lo taglia o si strappa, lo zainetto si spacca e viene distrutto. Se lo zainetto è distrutto, ciò che conteneva è perso per sempre, sebbene un artefatto ricomparirà sempre da qualche parte nel multiverso. Se lo zainetto viene rivoltato, ciò che contiene viene espulso, senza recargli danno e lo zainetto deve essere rimesso al verso giusto prima che possa essere usato di nuovo. Se una creatura che respira viene posta all'interno dello zainetto, vi può sopravvivere per al massimo 10 minuti prima di cominciare a soffocare.

Piazzare lo zainetto all'interno dello spazio extradimensionale creato da una borsa conservante, un buco portatile o un oggetto simile distrugge immediatamente entrambi gli oggetti e apre un portale verso il Piano Astrale. Il portale origina dal punto in cui gli oggetti sono stati posti l'uno dentro l'altro. Qualsiasi creatura entro 3 metri dal portale viene risucchiata attraverso di esso e trascinata in un luogo casuale del Piano Astrale. Poi il portale si chiude. Il portale è a senso unico e non può essere riaperto.

\oggettomagico{Zappa dei Titani}

\textbf{Rarità:} Non Comune; \textbf{Costo:} 2000 mo

questo strumento sovradimensionato è lungo 3 m e pesante 120 kg (30 Ingombro), e può essere usato solo da un gigante (o da un personaggio ingrandito) per spostare grandi quantità di terriccio e costruire terrapieni (un cubo di 3 m per Turno). La \textbf{zappa} può anche essere usata per spaccare la pietra con grande rapidità. Se usata come un'arma ha un bonus +3 al colpire e infligge 5d6 ferite.

\oggettomagico{Zoccoli della Velocita'}

\textbf{Rarità:} Raro; \textbf{Costo:} 5000 mo

questi \textbf{zoccoli} di ferro si trovano in set da quattro. Quando tutti e quattro gli zoccoli sono fissati a un saurovallo o creatura simile, aumentano la velocità di movimento di quella creatura di 9 metri.

\oggettomagico{Zoccoli dello Zefiro}

\textbf{Rarità:} Molto Raro; \textbf{Costo:} 1500 mo

questi \textbf{zoccoli} di ferro si trovano in set da quattro. Quando tutti e quattro gli zoccoli sono fissati a un saurovallo o creatura simile, permettono a quella creatura di muoversi normalmente, mentre fluttua a circa 10 centimetri dal terreno. Questo effetto vuol dire che la creatura può attraversare o passare sopra superfici non solide o instabili, come l'acqua o la lava. La creatura non lascia tracce e ignora il terreno difficile. Inoltre, la creatura può muoversi alla sua normale velocità per un massimo di 12 ore al giorno senza subire l'affaticamento a causa della marcia forzata.

\end{multicols}

%\vfill
%\begin{center}
%\includegraphics[width=0.6\linewidth]{immagini/borsetta.png}

%\emph{Borsetta conservante, modello classico, Tipo II}

%\includegraphics[width=0.8\linewidth]{immagini/The_erection_of_the_Tabernacle_and_the_Sacred.png}

%\emph{Ark of the Covenantat the erection of the Tabernacle and the sacred vessels, as in Exodus 40:17–19. Figures de la Bible, Gerard Hoet (1648–1733) and others, published by P. de Hondt in The Hague in 1728}

%\end{center}

\pagebreak

\section{Regole su Oggetti Magici}\index{Regole su Oggetti Magici}\hypertarget{identificareom}{}\label{regoleoggettimagici}\hypertarget{regoleoggettimagici}{}

\begin{multicols}{2}

Queste sono le indicazioni su l'utilizzo degli oggetti magici.

\label{oggetti-magici}
\begin{itemize}[leftmargin=*] \setlength{\itemsep}{0pt}
\item
Un personaggio può \textbf{tenere attivi numerosi (fino a 10) oggetti magici} su di sé.

\item Per determinare il bonus alla \textbf{Difesa} non si possono sommare più di 2 oggetti (es. 1 anello magico ed un braccialetto). Armatura e Scudo non si considerano in questo conteggio.
\item
Se hai più oggetti magici che concedono bonus allo stesso \textbf{Tiro Salvezza} si applicano solo i due con il bonus maggiore.
\item
Se hai più oggetti magici che concedono bonus alla stessa \textbf{Caratteristica} allora si applica solo il bonus maggiore.
\item
Un personaggio \textbf{non può indossare più di due anelli magici} altrimenti entrano in risonanza abbassando i Punti Ferita massimi di 1d6 (non riducibile o curabile) a round per ogni anello oltre il secondo.
\item
Per \textbf{riconoscere un oggetto magico} vedi \hyperlink{rinoscereoggettomagico}{Riconoscere un oggetto magico} (pag. \pageref{rinoscereoggettomagico}) e \hyperlink{oggettimaledettiid}{Oggetti Maledetti} (pag. \pageref{oggettimaledettiid}).
\item
Un \textbf{oggetto magico che manifesta incantesimi} non esegue alcuna Prova di Magia. Il \textbf{Tiro Salvezza} che impone, se non specificato, è pari a 12 + livello*2 dell'incantesimo che manifesta.\index{Tiro Salvezza per incantesimi da oggetto}\label{tirosalvezzaincoggetto}
\item
Per \textbf{Attivare delle capacità magiche} di un oggetto se non indicato diversamente  costa 2 Azioni.
\item
Un oggetto magico che fornisce un \textbf{bonus (o penalità) statico} applica il suo valore anche se l'oggetto non è stato identificato, sarà il Narratore ad applicare silenziosamente questo bonus alla Difesa, Tiro per Colpire, Tiri Salvezza... informando il giocatore che percepisce come l'oggetto interagisca con la situazione.
\item
Un oggetto magico che ha degli usi giornalieri si ricarica all'alba del giorno successivo all'uso.
\item
Bacchette, Bastoni, Pergamene (non Isy), Verghe sono usabili solo da personaggi che abbiano il punteggio di CM pari al livello di incantesimo più alto formulabile dall'oggetto.
\item
Se spendi l'ultima carica della bacchetta, tira 1d6 se ottieni 1 la bacchetta si riduce in polvere ed è distrutta
\item
Un oggetto magico impiega 10 minuti di tempo una volta indossato prima di permettere di usare i suoi modificatori, bonus o talenti.

\end{itemize}

\subsubsection{Armi}

\textbf{Abilità Speciali}: un'arma con una capacità speciale deve avere almeno bonus di +1. Le armi non possono avere la stessa capacità speciale più di una volta.

Il bonus magico di un \textbf{arma può essere compreso} tirando di due critici in un Tiro per Colpire oppure dedicando 1 ora di allenamento, eventuali talenti o abilità magiche rimangono celate.

\subsubsection{Armature e Scudi}

\textbf{Abilità Speciali}: un'armatura o scudo con una capacità speciale deve avere almeno bonus di +1. Armature e Scudi non possono avere la stessa capacità speciale più di una volta. Vedi anche sezione \hyperlink{armaturaescudimagici}{Armatura, Scudi e Magia} (pag. \pageref{armaturaescudimagici}).

Un'\emph{armatura} +1 abbassa di 1 la penalità di Competenza e di 1 metro la quella al movimento

Una armatura o \emph{scudo} +2 riduce di 2 la penalità data dall'armatura alla Prova di Magia.

Una armatura +3 ulteriormente toglie 1 alla penalità di Competenza, riduce di 1m la penalità al Movimento e riduce di ulteriori due la penalità data dall'armatura alla Prova di Magia.

\textbf{Il costo di Armi e Armature}: di dimensioni superiori alle Medie è almeno il doppio (o quadruplo in base alla taglia). Armature piccole o Armi piccole pur richiedendo meno materiale costano la medesima cifra delle armi e armature medie.

\subsection{Taglia e Oggetti Magici}\label{tagliaoggettimagici}

\label{taglia-e-oggetti-magici}

Quando un capo di vestiario o un gioiello magici vengono scoperti, il più delle volte la taglia non è un problema: molti vestiti magici sono di facile utilizzo per tutti oppure si adattano magicamente a chi li indossa. Di regola, la taglia non dovrebbe impedire ai personaggi di varia tipologia fisica l'utilizzo di un oggetto magico.

Ci possono essere delle rare eccezioni, specie con gli oggetti realizzati per una razza specifica.

Le armi e le armature rinvenute casualmente hanno una probabilità del 30\% di essere Piccole (01--30), del 60\% di essere Medie (31--90), e del 10\% di essere di un'altra taglia. Le armature non si adattano alla taglia del possessore se non indicato diversamente.

\subsection{Oggetti Magici sul Corpo}\index{Oggetti Magici sul Corpo}

\label{oggetti-magici-sul-corpo}

Molti oggetti magici devono essere indossati da un personaggio che voglia usarli o beneficiare delle loro capacità. Per una creatura di forma umanoide è possibile indossare fino a 10 oggetti magici alla volta. Ognuno di questi oggetti deve essere indossato sopra una parte specifica del corpo denominata \textbf{slot}.

Un corpo di forma umanoide può indossare l'equipaggiamento magico in queste parti del corpo:

\textbf{Dita}: anelli (due al massimo).

\textbf{Vesti}: corazze, armature, tuniche e vesti

\textbf{Cintura}: cinture.

\textbf{Collo}: amuleti, collane, medaglioni, scarabei, spille, talismani e sciarpe

\textbf{Mani}: armi, guanti e guanti d'arme.

\textbf{Occhi}: occhi, occhiali e lenti.

\textbf{Piedi}: scarpe, stivali e pantofole.

\textbf{Polso}: braccialetti e bracciali.

\textbf{Braccia}: scudi.

\textbf{Spalle}: cappe e mantelli.

\textbf{Testa}: cappelli, diademi, elmi, maschere, corone, fasce e filatteri.

\textbf{Torace}: camicie, giubbe, maglie e manti.

\subsection{Tiri Salvezza Contro i Poteri degli Oggetti Magici}\index{Tiri Salvezza}

\label{tiri-salvezza-contro-i-poteri-degli-oggetti-magici}

Gli oggetti magici normalmente riproducono incantesimi o altri effetti magici. Per un Tiro Salvezza contro la magia o un effetto magico generato da un oggetto magico, la DC è 12 + livello dell'incantesimo manifestato x2 se non specificato diversamente.

\subsection{Danneggiare gli Oggetti Magici}\index{Danneggiare gli Oggetti Magici}

\label{danneggiare-gli-oggetti-magici}

Un oggetto magico non deve compiere un Tiro Salvezza a meno che non sia incustodito, sia il bersaglio specifico dell'effetto, o il suo possessore ottenga un 0 (tre volte 1) naturale al suo Tiro Salvezza.

Gli oggetti magici hanno sempre diritto a un Tiro Salvezza contro qualcosa che potrebbe danneggiarli, anche quando un oggetto normale dello stesso tipo non avrebbe alcuna possibilità di effettuare un Tiro Salvezza. Gli oggetti magici usano sempre lo stesso bonus ai Tiri Salvezza, indipendentemente dal tipo (Tempra, Riflessi o Volontà). Il bonus ai Tiri Salvezza di un oggetto magico è pari a 2 + 2x livello dell'incantesimo più potente che ospitano (oppure un +6 per ogni +1 che hanno). Le sole eccezioni a questa regola sono gli oggetti magici intelligenti, che effettuano i Tiri Salvezza su Volontà basandosi sul loro punteggio di Saggezza.

\subsection{Riparare gli Oggetti Magici}\index{Riparare gli Oggetti Magici}
\label{riparare-gli-oggetti-magici}

Per riparare un oggetto magico occorrono materiali e tempo, pari alla metà del tempo e del costo per crearlo.

\subsection{Cariche, Dosi e Usi Multipli}\index{Cariche}\index{Dosi}\index{Usi Multipli}

\label{cariche-dosi-e-usi-multipli}

Molti oggetti, ed in modo particolare le bacchette e i bastoni, hanno un potere limitato al numero di cariche che contengono. Normalmente gli oggetti dotati di cariche non superano mai il massimo di 20 cariche (10 per i bastoni). Se oggetti simili vengono trovati come parte casuale di un tesoro, si tira un 1d10+10 per determinare il numero delle cariche rimaste. Se un oggetto ha un numero massimo di cariche diverso da 20, si tira casualmente per stabilire quante cariche sono rimaste.

I prezzi indicati si riferiscono agli oggetti al massimo delle loro cariche (quando un oggetto viene creato, ha sempre il massimo delle cariche). Il valore di un oggetto dipende dal numero di cariche residue, in caso di oggetti che possono avere un uso anche con poche o senza cariche, il valore rimane più alto.

\subsection{Ricaricare gli oggetti magici}\index{Ricaricare gli oggetti magici}\label{Ricaricare gli oggetti magici}

Oggetti magici \emph{a carica} come i Bacchette e Bastoni hanno un numero di usi, cariche, ovvero ogni volta che si attinge al suo potere si usa una carica.

Per ricaricare una bacchetta od un bastone un incantatore deve infondere lo stesso incantesimo che vuole ricaricare spendendo il doppio dei Punti Magia del costo dell'incantesimo e superare una Prova di Magia.

\end{multicols}

\subsection{Acquisire Oggetti Magici}\index{Acquisire Oggetti Magici}\index[Tabelle]{Tabella Acquisire Oggetti Magici}

\label{acquisire-oggetti-magici}

\bigskip

\noindent\begin{tabularx}{\linewidth}{lXXXX}
	\toprule
\rowcolor{gray!20}\textbf{Dimensioni Comunità} & \textbf{Valore Base} & \textbf{Comune} & \textbf{Non Comune} & \textbf{Raro}\\
\toprule
Insediamento& 20 mo & 1d2 oggetti && \\
\rowcolor{gray!20}Borgo & 150 mo& 1d4 oggetti && \\
Villaggio & 300 mo& 1d6 oggetti & 1d2 oggetti& \\
\rowcolor{gray!20}Piccolo paese & 700 mo & 1d4 oggetti & 1d2 oggetti& \\
Grande paese& 1500 mo & 1d6 oggetti & 1d4 oggetti& 1d2 oggetti\\
\rowcolor{gray!20}Piccola città & 2500 mo & 2d4 oggetti & 1d6 oggetti& 1d4 oggetti\\
Grande città& 6000 mo & 3d4 oggetti& 2d4 oggetti& 1d6 oggetti\\
\rowcolor{gray!20}Metropoli & 12000 mo& {*} & 3d4 oggetti& 2d4 oggetti
\end{tabularx}

{*} In una metropoli si trovano quasi tutti gli oggetti magici minori.

\begin{multicols}{2}

\bigskip

Gli oggetti magici sono preziosi e la maggior parte delle grandi città ha almeno uno o due fornitori di oggetti magici, dal semplice venditore di pozioni ad un fabbro specializzato nel forgiare spade magiche. Ovviamente non ogni oggetto in questo manuale è disponibile in ogni città.

Le linee guida seguenti aiutano i Narratori a determinare quali oggetti sono disponibili in una specifica comunità. Esse presuppongono una campagna con un livello medio di magia. Alcune città potrebbero deviare di molto da questa linea di base a discrezione del Narratore. Il Narratore dovrebbe tenere una lista degli oggetti disponibili da ogni mercante e dovrebbe rimpinguare occasionalmente le scorte con nuove acquisizioni.

Il numero ed i tipi di oggetti magici disponibili in una comunità dipendono dalla sua dimensione. Ogni comunità ha un valore base legato ad essa (vedi Tabella: Oggetti Magici Disponibili).

c'è una probabilità del 75\% che qualsiasi oggetto di quel valore o inferiore si possa trovare in vendita facilmente in quella comunità. Inoltre, la comunità ha un certo numero di altri oggetti in vendita. Questi oggetti sono determinati a caso e sono ripartiti in categorie (minore, medio o maggiore).

Dopo aver determinato il numero di oggetti disponibili in ogni categoria, consultate il capitolo Generazione casuale degli Oggetti Magici per determinare il tipo di ogni oggetto (pozione, pergamena, anello, arma,ecc.) prima di passare alle tabelle specifiche per stabilire l'oggetto esatto. Ritirate ogni volta che gli oggetti non si adeguano al valore base della comunità.

Se l'uso della magia nella campagna in cui si gioca è raro, occorre dimezzare il valore base e il numero di oggetti in ogni comunità. Nelle campagne con magia estremamente rara o senza magia potrebbero non esserci affatto oggetti magici in vendita I Narratori che conducono questo tipo di campagne dovrebbe prevedere delle modifiche alle sfide affrontate dai personaggi data la mancanza di oggetti magici.

Le campagne con abbondanti oggetti magici potrebbero avere comunità con il doppio del valore base stabilito e degli oggetti magici casuali disponibili. In alternativa, si potrebbe stabilire che tutte le comunità siano di una categoria di dimensione maggiore allo scopo di stabilire gli oggetti magici disponibili. In una campagna con magia molto comune, tutti gli oggetti magici si possono acquistare in una metropoli.

Oggetti e attrezzi non magici sono in genere disponibili in una comunità di qualsiasi dimensione a meno che l'oggetto non sia molto costoso, come un'armatura completa, o fatto di un materiale insolito, come una spada lunga in adamantio. Questi oggetti dovrebbero seguire la linea guida del valore base per determinare la loro disponibilità, a discrezione del Narratore.

\subsection{Gli artefatti del vecchio mondo}

Nel corso delle avventure i giocatori troveranno degli oggetti del passato dimenticato. Potranno essere chincaglieria senza utilità se non come reperto storico di un era che non tornerà più. Potranno spesso essere apparati elettronici che senza una fonte di energia non funzioneranno mai..

Potranno essere altrimenti strumenti creati negli ultimi giorni della prima era quando appresi i rudimenti della magia qualche genio e Devoto riuscì a sfruttare la tecnologia con la magia, riuscì ad attivare \emph{magicamente} un apparato tecnologico.

Non stupiamoci allora di trovare oggetti che possano essere ricaricati, anche se per poco tempo, con \emph{Stretta Folgorante}, o veicoli funzionanti se colpiti da un \emph{Fulmine}.

Questi oggetti saranno rari e preziosi, quasi come una dose di antibiotico ancora attiva. Sbizzarritevi nel creare oggetti \emph{tecnomagici}, prendete ispirazione dal mondo moderno e dalla fantascienza per creare strabilianti oggetti utili alla vostra avventura.

\end{multicols}

\vfill

\begin{center}
\includegraphics[keepaspectratio,width=0.8\textwidth]{immagini/Alchemical_laboratory_Wellcome_M0005193.png}

\emph{Alchemical laboratory}
\end{center}

\pagebreak

\section{Creare Oggetti Magici}\index{Creare Oggetti Magici}

\begin{enfasi}{
Creare è vivere due volte. (Albert Camus)
}\end{enfasi}

\begin{multicols}{2}

\label{creare-oggetti-magici}

Per Creare Oggetti Magici è necessario avere le Abilità Creazione oggetti magici ed essere capace in una competenza specifica (oreficeria, erboristeria, calligrafia...).

I costi qui elencati sono quelli di produzione, il ricavo si può attestare attorno al 20\%-50\% del prezzo di produzione.

La \textbf{DC}, basata sulla competenza indicata, per creare un oggetto è 15 +2*Livelli dell'incantesimo contenuto, oppure +6 per ogni +1 dell'oggetto.\index{DC per creare oggetto magico}

Conoscere l'incantesimo (o averlo a disposizione tramite Pergamena) che si applica all'oggetto è un requisito di ogni oggetto magico che si crea. I giorni di lavoro indicati non possono essere frazionati in meno di 6 ore al giorno dedicate alla creazione.

\begin{narratore}[Creare oggetti magici]
La creazione di oggetti magici può rompere gli equilibri del gioco. Un personaggio con risorse abbondanti e tempo può arrivare a creare oggetti che spezzano gli equilibri dell'avventura. Suggerisco siano gli NPC, i personaggi non giocanti in gestione al Narratore, a creare gli oggetti più meravigliosi. Allo stesso tempo la vendita di oggetti di valore sopra le 2000mo dovrebbe essere il più limitata possibile.
\end{narratore}

\subsubsection{Modificatori al costo degli oggetti magici}\index{Modificatori al costo degli oggetti magici}\label{modificatoricostooggettimagici}

Gli oggetti magici hanno come componente base l'applicazione di un incantesimo nell'oggetto stesso.

E' importante valutare la rarità dell'incantesimo per che viene usato per determinare il costo dell'oggetto.

I \textbf{costi} riportati per la creazione dei vari tipi di oggetti sono riferiti all'utilizzo di un incantesimo con \textbf{rarità} comune. Se la rarità è Non comune moltiplicare il prezzo x1.5, se è Raro x2, Molto Raro x5, Leggendario x10.

\begin{center}
	\includegraphics[width=0.3\linewidth]{immagini/onering2.png}

	\emph{Non c'è bisogno di dire che Unico Anello sia...}
\end{center}

\subsection{Creare Anelli Magici}\index{Anelli Magici}\label{creareanellimagici}

Per creare un anello magico, un personaggio ha bisogno di una fonte di calore. Ha anche bisogno di una provvista di materiali, di cui il più ovvio è un anello o pezzi di anello da assemblare.

Il costo di produzione dell'anello è pari a livello*livello*2000, un Anello con Invisibilità costa 2*2*2000=8000 mo

Un anello permette di fissare un incantesimo per rendere l'effetto sempre attivo.
L'anello deve avere un valore intrinseco pari almeno a 100mo*somma dei livelli dell'incantesimo che deve ospitare.

Un anello può ospitare un incantesimo di livello 9 o se più incantesimi il massimo livello è 7.

E' anche possibile inserire un incantesimo ad attivazione, in questo caso consultare i costi delle Verghe.

Forgiare un anello richiede 1 giorno per ogni 500 mo del prezzo base. In caso di più incantesimi i costi e tempi si sommano.

\medskip

\textbf{Talento di creazione oggetto richiesto}: Creare Oggetti Magici 2, Competenza Oreficeria.

\begin{center}
	\includegraphics[width=0.55\linewidth]{immagini/Rustning_Gustav_Vasa.png}

	\emph{Armour for Gustav I of Sweden by Kunz Lochner, c. 1540 (Livrustkammaren)}
\end{center}

\subsection{Creare Armature e Scudi Magici}\index{Creare Armature Magiche}\label{crearearmaturemagiche}

Per creare un'armatura o scudo magico, un personaggio ha bisogno di una fonte di calore e di alcuni attrezzi per lavorare il ferro, il legno o il cuoio. Ha anche bisogno di una provvista di materiali di cui il più ovvio è l'armatura/scudo stessa o i pezzi di armatura da assemblare. Un'armatura/scudo che va incantata deve essere di qualità.

Se i prerequisiti per la creazione dell'armatura comprendono degli incantesimi, l'incantatore deve conoscere detti incantesimi.

Il costo di produzione di un'armatura magica +1 costa 2050 mo, +2 7500 mo, +3 12000 mo, +4 25000 mo, +5 45000 mo più il prezzo dell'armatura stessa.

Infondere un incantesimo in una armatura ha un costo come se si andasse a creare un anello con quell'incantesimo.

Creare armature/scudi magiche richiede un giorno per ogni 1000 mo del valore del prezzo base.

\textbf{Talento di creazione oggetto richiesto}: Creare Oggetti Magici, Competenza Fabbro.

\subsection{Creare Armi Magiche}\index{Creare Armi Magiche}\label{crearearmimagiche}

Per creare un'arma magica, un personaggio ha bisogno di una fonte di calore e alcuni attrezzi per lavorare il ferro od il materiale con cui è fatta l'arma. Ha anche bisogno di una provvista di materiali, di cui il più ovvio è l'arma stessa o i pezzi di arma da assemblare. Solo un'arma di qualità può essere incantata per diventare un'arma magica, e il suo costo va aggiunto al costo totale di incantamento per determinare il valore finale di mercato.

Un'arma magica deve avere almeno bonus di +1 per avere una qualsiasi capacità speciale o incantesimo.

\medskip

\begin{center}
\includegraphics[width=0.6\linewidth]{immagini/exacaliburfuori.png}

\emph{The drawing of the sword from the stone, Henrietta Elizabeth Marshall's Our Island Story (1906)}
\end{center}

\medskip

Se i prerequisiti per la creazione dell'arma comprendono degli incantesimi, l'incantatore deve conoscere detti incantesimi.

Nel momento della creazione, l'incantatore deve decidere se l'arma emana luce o meno, come effetto secondario della magia infusa nell'arma. Questa decisione non influenza il prezzo o il tempo di creazione, ma una volta che l'oggetto è completato, la decisione è definitiva.

Creare armi doppie viene considerato analogo a creare due armi per quanto riguarda il costo, il tempo e le Capacità Speciali.

Il costo di produzione di un Arma +1 è 1200 mo, +2 4000 mo, +3 11000 mo, +4 25000 mo, +5 45000 mo più il prezzo dell'arma (influente solo se è di un qualche materiale raro o prezioso).

Il costo di produzione di una Freccia +1 è 20 mo, +2 75 mo, +3 325 mo. Incantamenti più potenti sono estremamente rari.

Infondere un incantesimo in un arma ha un costo come se si andasse a creare un anello con quell'incantesimo, se continuativo, altrimenti se mono uso come una pozione.

Creare un'arma magica richiede una giornata per ogni 1000 mo del valore del prezzo base.

\medskip

\textbf{Talento di creazione oggetto richiesto}: Creare Oggetti Magici, Competenza Fabbro

\begin{center}
	\includegraphics[width=0.5\linewidth]{immagini/wand.png}
\end{center}

\subsection{Creare Bacchette}\index{Creare Bacchette}\label{crearebacchette}

Il costo di produzione della Bacchetta è pari a livello*livello*400, una Bacchetta con Invisibilità costa 2*2*400=1600 mo

Una bacchetta è un oggetto magico che conserva in se un incantesimo caricato in precedenza.

Una Bacchetta può contenere come massimo livello di incantesimo 5.

Per creare una bacchetta, un personaggio ha bisogno di una provvista di materiali di cui il più ovvio è una bacchetta o i pezzi di una bacchetta da assemblare. Le bacchette sono sempre pienamente cariche (20 cariche) all'atto della creazione.

L'incantatore deve conoscere l'incantesimo che inserisce nella Bacchetta.

Creare una bacchetta richiede 1 giorno per ogni 500 mo del valore del prezzo base.

\medskip

\textbf{Talento di creazione oggetto richiesto}: Creare Oggetti Magici, Competenza Arcana.

\begin{center}
	\includegraphics[width=0.35\linewidth]{immagini/staff2.png}
\end{center}

\subsection{Creare Bastoni}\index{Creare Bastoni}\label{crearebastoni}

\textbf{Costi Base dei Bastoni}

\bigskip

Il costo di produzione del Bastone è pari a livello*livello*600, un Bastone con Invisibilità costa 2*2*600=2400 mo

\bigskip

Un Bastone è un oggetto magico dove si caricano una o più incantesimi.

Quando un bastone viene attivato è possibile usare un incantesimo alla volta.

Per creare un bastone un personaggio ha bisogno di una provvista di materiali di cui il più ovvio è un bastone o i pezzi di un bastone da assemblare.

I bastoni sono sempre pienamente carichi, 10 cariche, all'atto della creazione.

Una Bastone può contenere come massimo livello di incantesimo 8, o in caso di diversi incantesimi il massimo livello è il 6.

Creare un bastone richiede 1 giorno per ogni 500 mo del prezzo base.

\medskip

\textbf{Talento di creazione oggetto richiesto}: Creare Oggetti Magici 2, Competenza Falegnameria.

\subsection{Creare Pergamene}\index{Creare Pergamene}\index{Pergamene}\index{Isy Scroll}
\index{Pergamene facili}\index{Comprare incantesimi}\label{crearepergamene}\hypertarget{crearepergamene}{}

Le pergamene sono oggetti magici che custodiscono all'interno la formula e componenti di uno o più incantesimi. Il tempo di lancio di un incantesimo da una pergamena è pari al tempo di lancio del medesimo incantesimo.

Esistono due tipologie di Pergamene magiche, quelle eseguibili da tutti (dette ISY SCROLL, o Facili) e quelle invece che richiedono la capacità magica di lanciare incantesimi, ovvero Competenza Magica maggiore o uguale a 1.

Le pergamene facili hanno un costo di produzione pari a livello*livello*rarità dell'incantesimo*80 mo.

Le pergamene normali, non facili, hanno un costo di produzione pari a livello*livello*rarità dell'incantesimo*40 mo

Se una pergamena include più incantesimi il costo è pari alla somma dei vari incantesimi. Su una pergamena ISY SCROLL non possono esserci incantesimi da pergamena normali e vice versa.

L'incantatore deve conoscere gli incantesimi che inserisce nella pergamena. Per preparare una pergamena è necessario 30 minuti di lavoro per livello di incantesimo presente.

Una pergamena ISY può contenere incantesimi di livello 3 come massimo, mentre una pergamena normale può contenere come massimo livello di incantesimo 9, in caso di più incantesimi il massimo livello è 8.

\medskip

Per leggere una pergamena è necessario:\label{leggerepergamena}

\medskip

\textbf{in caso di pergamene ISY SCROLL}:

\begin{itemize}[leftmargin=*] \setlength{\itemsep}{0pt}
\item per comprendere il contenuto è sufficiente una prova di Intelligenza (o Arcana se conosciuta) a difficoltà DC 10
\item per poter leggere e lanciare l'incantesimo della pergamena è necessaria una prova di Intelligenza (o Arcana se conosciuta) a difficoltà 12.
\end{itemize}

\textbf{in caso di pergamene normali}:

\begin{itemize}[leftmargin=*] \setlength{\itemsep}{0pt}
\item per comprenderne il contenuto è necessaria una prova di Arcana a difficoltà 15
\item per poter leggere e lanciare l'incantesimo della pergamena è necessaria una prova di Arcana a difficoltà 11+Livello dell'incantesimo e l'incantesimo deve avere un livello pari al massimo lanciabile +2
\end{itemize}

\begin{center}
\includegraphics[width=0.4\linewidth]{immagini/scroll3.png}
\end{center}

Il \textbf{tempo di lancio} di un incantesimo da una pergamena è pari al tempo di lancio dell'incantesimo presente.

\textbf{Talento di creazione dell'oggetto richiesto}: Creare Oggetti Magici, Competenza Calligrafia.

Una pergamena quando viene usata o copiata si distrugge.

\textbf{Nota}: un Tomo della Magia è equivalente ad un insieme di pergamene normali. Un personaggio in situazione disperate può leggere la pagina dell'incantesimo dal Tomo della Magia e manifestare la magia come se fosse da una pergamena. Le pagine contenenti l'incantesimo si polverizzeranno e l'incantatore dovrà trovare una sorgente da dove copiare nuovamente l'incantesimo su Tomo. Non può ricopiare sul Tomo lo stesso l'incantesimo perché l'ha appreso. \index{Tomo di Magia come pergamena}

\subsection{Creare Pozioni}\index{Creare Pozioni}\index{Pozioni}\label{crearepozioni}\hypertarget{crearepozioni}{}

Una pozione contiene l'infuso di un solo incantesimo, ogni pozione è quindi monouso.

\medskip

Il costo di produzione della Pozione è pari a livello*livello*40, una Pozione con Invisibilità costa 2*2*40=160 mo

\medskip

Per creare una pozione un personaggio ha bisogno di un piano di lavoro orizzontale e alcuni contenitori per mescere i liquidi oltre a una fonte di calore per bollire l'infuso.

Una Pozione può contenere di norma come massimo livello di incantesimo 3. A discrezione del Narratore pozioni di livello superiore potrebbero essere possibili al prezzo di livello*livello*livello*20 mo.

Tutti gli ingredienti e i materiali per preparare una pozione devono essere freschi e mai usati.

L'incantatore deve conoscere l'incantesimo che inserisce nella pozione. Il tempo di preparazione di una pozione è pari al doppio del livello dell'incantesimo contenuto in ore.

\textbf{Talento di creazione dell'oggetto richiesto}: Distillare Pozioni, Competenza Erboristeria.

\subsection{Creare Verghe}\index{Creare Verghe}\index{Verghe}\label{creareverghe}

Una verga è una bacchetta speciale che è capace di rigenerare le proprie cariche. Sono oggetti preziosi e molto costosi.

Per creare una verga, un personaggio ha bisogno di una provvista di materiali, di cui il più ovvio è una verga o i pezzi di una verga da assemblare.

\medskip

Il costo di produzione della Verga è pari a livello*livello*1600, una Verga con Invisibilità costa 2*2*1600=6400 mo

\bigskip

Una verga è in grado di lanciare 1 volta al giorno il proprio incantesimo.

Moltiplicare il costo per 4 se è in grado di lanciarla 2 volte, moltiplicare per 8 se è in grado di lanciarla 3 volte al giorno.

Si può anche lanciare una volta in più nel giorno l'incantesimo contenuta nella verga, dopo di che la verga si distrugge.

Una Verga può contenere come massimo livello di incantesimo 3.

L'incantatore deve conoscere l'incantesimo che inserisce nella Verga.

Creare una verga richiede 1 giorno per ogni 500 mo del prezzo base.

\textbf{Talento di creazione oggetto richiesto}: Creare Oggetti Magici 2, Competenza Arcana.

\subsection{Aggiungere Nuove Capacità}\index{Aggiungere Nuove Capacita}\label{aggiungerecapacitamagiche}

A volte la mancanza di fondi o tempo rende impossibile realizzare l'oggetto magico voluto ma fortunatamente è possibile potenziare o modificare un oggetto magico creato. Solo il tempo, l'oro ed i vari prerequisiti richiesti dalla nuova capacità che si vuole aggiungere all'oggetto magico pongono delle restrizione sul tipo di poteri addizionali che uno può infondere.

Il costo per aggiungere capacità addizionali ad un oggetto è lo stesso che se l'oggetto non fosse magico, meno il valore dell'oggetto originale. Quindi una spada lunga +1 può diventare una spada lunga vorpal +2 e il costo della creazione è uguale a quello di una spada lunga vorpal +2 meno il costo di una spada lunga +1.

Quando si determina il prezzo di un oggetto magico inventato bisogna considerare molti fattori. Il modo più semplice per decidere il prezzo è confrontare il nuovo oggetto a un oggetto che ha già un prezzo, e usare tale prezzo come guida.

\end{multicols}

\vfill

\medskip

\begin{center}
\includegraphics[width=0.5\linewidth]{immagini/potion2.png}
\smallskip

\emph{A witch, raising her arm above a flaming cauldron, recites a spell; a young woman kneels in front of the cauldron. Mezzotint by J. Dixon after J.H. Mortimer, 1773}
\end{center}

%\vfill

%\begin{center}
%\includegraphics[width=0.13\linewidth]{immagini/Rod_of_asclepius.png}
%
%\emph{Il bastone di Asclepio è un antico simbolo greco associato alla medicina. Consiste in un serpente attorcigliato intorno ad una verga.}
%\end{center}

\pagebreak

\pagebreak

\section{Oggetti Maledetti}\index{Oggetti Maledetti}

\begin{enfasi}{Quando un empio maledice l'avversario, maledice se stesso. (Siracide)

\medskip

Se maledici una persona ci saranno due fosse. (Proverbio giapponese)}
\end{enfasi}

\begin{multicols}{2}

\label{oggetti-maledetti}

Gli oggetti maledetti sono oggetti magici dotati di un'influenza potenzialmente negativa sul personaggio.

Gli oggetti maledetti non sono quasi mai realizzati intenzionalmente, piuttosto sono il risultato di un lavoro mal riuscito, di artigiani con poca esperienza o della mancanza di componenti adeguati o patti non rispettati con qualche Patrono.

Il Narratore può chiedere una prova di Arcana con una DC pari a 10+giorni impiegati per costruire l'oggetto magico in caso di oggetti particolarmente complessi o se ci siano state situazioni problematiche nella creazione e se la prova fallisce tirate sulla tabella seguente per determinare il tipo di maledizione che l'oggetto possiede.

Una maledizione può manifestarsi anche a seguito dalle influenze negative od emozionali estreme che coinvolgono un oggetto.

\medskip

\textbf{Maledizioni Comuni degli Oggetti}

\medskip

\noindent\begin{tabularx}{\linewidth}{ll}
	\toprule
\rowcolor{gray!20}\textbf{\%} & \textbf{Maledizione}\\
\toprule
01-15 & Inganno\\
\rowcolor{gray!20}16-40 & Effetto o Bersaglio Opposto\\
41-50 & Funzionamento Discontinuo\\
\rowcolor{gray!20}51-65 & Requisito\\
66-90 & Inconveniente\\
\rowcolor{gray!20}91-100& Effetto completamente diverso
\end{tabularx}

\medskip

Gli oggetti maledetti sono \hypertarget{oggettimaledettiid}{\textbf{identificati}}\label{oggettimaledettiid} come qualsiasi altro oggetto magico con una sola eccezione: a meno che non la prova di Arcana per identificare l'oggetto non superi 35 o l'incantesimo \hyperlink{Identificare}{Identificare} sia lanciato con una Prova di Magia ed ottenga un critico magico la maledizione non viene individuata. Se la prova è sotto 35 o senza critico magico tutto quello che viene rivelato è l'originale scopo dell'oggetto magico.

Se si sa che l'oggetto è maledetto, la natura della maledizione può essere determinata usando la DC \hyperlink{identificareom}{standard} per identificare l'oggetto.

\begin{center}
\includegraphics[width=0.75\linewidth]{immagini/vasobasano.png}

\emph{Vaso di Basano. Questo vaso è stato realizzato nella seconda metà del XV secolo ed è realizzato in argento. DC 35}
\end{center}

\begin{narratore}[Maledizioni e perché]
Una maledizione è sempre un \emph{inconveniente} particolare, che non si usa a caso. Ragionate attentamente sugli oggetti maledetti che farete trovare ai personaggi perché vi chiederanno molte informazioni e dovrete essere pronti.

Non c'è bisogno che la maledizione sia eccessiva e limitante può essere benissimo ridicola o particolare, fate in modo che sia caratterizzante. Il personaggio non deve sentirsi (tranne se lo volete) condannato in eterno, sfruttate l'occasione per costruire nuove avventure e spirito di gruppo.
\end{narratore}

\subsection{Rimuovere Oggetti Maledetti}\index{Rimuovere Oggetti Maledetti}

Mentre alcuni oggetti maledetti possono essere semplicemente posati, altri esercitano una forte compulsione sul possessore a tenerli con sé, a qualsiasi costo. Altri riappaiono anche se abbandonati o è impossibile gettarli via.

Questi oggetti possono essere rimossi solo dopo che sul personaggio o l'oggetto viene lanciato l'incantesimo Rimuovi Maledizione.

L'incantesimo \hyperlink{Dissolvi Magie}{Dissolvi Magia} è inutile per rimuovere una maledizione, solo un \hyperlink{Dissolvi Magie Avanzato}{Dissolvi Magie Avanzato} con 3 Successi Magici Critici può essere sufficiente.

Se l'oggetto è stato maledetto tramite l'incantesimo \hyperlink{Scagliare Maledizione}{Scagliare Maledizione}, o comunque il Narratore decide che l'oggetto ha una maledizione particolare allora si deve effettuare una prova di \hyperlink{contrastareincantesimi}{contrasto} (pag. \pageref{contrastareincantesimi})tra chi lancia Rimuovi Maledizione e la DC della maledizione dell'oggetto.

Se la prova di contrasto ha successo allora l'oggetto può essere rimosso nel round successivo e la maledizione rimane e colpisce nuovamente se l'oggetto viene usato/indossato un'altra volta.

Ogni oggetto maledetto ha un proprio metodo per essere distrutto, dall'essere gettato in un vulcano attivo, ad essere colpito dal martello del dio del Tuono (o Patrono...) oppure divorato da un Verme colossale delle sabbie se non colpito dal soffio di un drago rosso e un drago d'oro contemporaneamente...

Se la DC della maledizione non è indicata è sufficiente il lancio dell'incantesimo Rimuovi Maledizioni.

\subsection{Effetti Comuni degli Oggetti Maledetti}

Gli effetti più comuni degli oggetti maledetti sono i seguenti, il Narratore può inventare nuovi effetti particolari per specifici oggetti maledetti.

\subsubsection{Inganno}

Chi utilizza l'oggetto continua a credere che sia ciò che sembra a prima vista, ma in realtà non ha alcun potere, a parte quello di ingannare. Chi lo usa è mentalmente spinto a credere che funzioni e non può essere convinto del contrario se non con l'uso di Rimuovi maledizione

\medskip

\begin{center}
\includegraphics[width=0.70\linewidth]{immagini/mirror.png}

\emph{The mirror in The Myrtles Plantation. DC 28}
\end{center}

\subsubsection{Effetto o Bersaglio Opposto}

Questi oggetti maledetti tendono ad avere dei difetti di funzionamento che in alcuni casi generano effetti diametralmente opposti a quelli desiderati dal loro creatore, mentre in altri casi tendono a colpire chi li utilizza invece di qualcun altro.

La categoria degli oggetti magici dagli effetti opposti include anche le armi che infliggono penalità ai Tiri per Colpire ed ai danni, invece che bonus.

La cosa più interessante è che questi oggetti potrebbero anche non essere uno svantaggio per chi li possiede.

Visto che un personaggio non dovrebbe sapere immediatamente quale sia il bonus od effetto di un oggetto magico, non dovrebbe venire a conoscenza nemmeno della natura della sua maledizione. Una volta che lo verrà a sapere per liberarsi dall'oggetto sarà necessario l'Incantesimo Rimuovi maledizione.

\subsection{Funzionamento Discontinuo}

Gli oggetti discontinui funzionano esattamente come dovrebbero, quando funzionano. Stabilite se l'oggetto è Inaffidabile, Condizionato oppure Incontrollabile.

\medskip
\subsubsection{Inaffidabile}

Ogni volta che l'oggetto viene attivato, c'è una probabilità del 5\% che non funzioni.

\subsubsection{Condizionato}

Questo oggetto funziona solo in determinate situazioni. Per determinare quali siano, scegliete una condizione di attivazione o consultate la tabella poco sotto.

\subsubsection{Incontrollabile}

Un oggetto incontrollabile tende ad attivarsi casualmente. Tirare un d\% ogni giorno. Con un risultato di 01--05 l'oggetto si attiva spontaneamente in un certo momento del giorno.

\medskip

\noindent\begin{tabularx}{\linewidth}{lX}
	\toprule
\rowcolor{gray!20}\textbf{\%} & \textbf{Situazione}\\
\toprule
01-03 & Temperatura sotto lo zero\\
\rowcolor{gray!20}04-05 & Temperatura sopra lo zero\\
06-10 & Durante il giorno\\
\rowcolor{gray!20}11-15 & Durante la notte\\
16-20 & Esposto alla luce solare\\
\rowcolor{gray!20}21-25 & In assenza di luce solare\\
26-34 & Sott'acqua\\
\rowcolor{gray!20}35-37 & Fuori dall'acqua\\
38-45 & Sottoterra\\
\rowcolor{gray!20}46-55 & In superficie\\
56-60 & Entro 3 metri da un tipo di creatura\\
\rowcolor{gray!20}61-64 & Entro 3 metri da una razza o tipo di creatura\\
65-72 & Entro 3 metri da un incantatore\\
\rowcolor{gray!20}73-80 & Entro 3 metri da un Seguace o Devoto di un Patrono specifico\\
81-85 & Nelle mani di un personaggio non incantatore\\
\rowcolor{gray!20}86-90 & Nelle mani di un personaggio incantatore\\
91-95 & Nelle mani di una creatura con particolare Tratto\\
\rowcolor{gray!20}96& Nelle mani di una creatura di un particolare genere\\
97-99 & Nei giorni non sacri o durante particolari ricorrenze astronomiche\\
\rowcolor{gray!20}100 & A più di 150 km da un determinato luogo
\end{tabularx}

\subsection{Requisito}

Alcuni oggetti hanno requisiti molto più difficili da soddisfare perché funzionino. Per far funzionare l'oggetto in questione, potrebbe essere necessario soddisfare una delle seguenti condizioni:

\begin{itemize}[leftmargin=*] \setlength{\itemsep}{0pt}
\item Il personaggio deve mangiare il doppio del normale.
\item Il personaggio deve dormire il doppio del normale.
\item Il personaggio deve compiere almeno una missione specifica.
\item Il personaggio deve sacrificare (distruggere) un valore pari a 100 mo di oggetti o materiali preziosi al giorno.
\item Il personaggio deve giurare lealtà ad un nobile in particolare o alla sua famiglia.
\item Il personaggio deve abbandonare tutti gli altri oggetti magici.
\item Il personaggio deve essere un Seguace o Devoto di uno specifico Patrono
\item Il personaggio deve avere un numero minimo di gradi in una particolare competenza.
\item Il personaggio deve sacrificare parte della propria energia vitale (1 punto di Costituzione permanente) la prima volta che usa l'oggetto.
\item L'oggetto deve essere purificato con l'Acqua santa di uno specifico Patrono ogni giorno.
\item L'oggetto deve essere bagnato in almeno mezzo litro di sangue (animale o umanoide) al giorno.
\item L'oggetto deve essere usato per uccidere una creatura vivente al giorno.
\item L'oggetto deve essere usato almeno una volta al giorno, o smette di funzionare per il suo attuale possessore.
\item Quando viene brandito, l'oggetto deve spillare sangue (solo armi). Non può essere messo da parte o cambiato con un altro oggetto finché non ha messo a segno un colpo.
\end{itemize}

I requisiti dipendono dalla convenienza dell'oggetto che non dovrebbero mai essere determinati a caso. Un oggetto intelligente con un requisito spesso impone il proprio requisito grazie alla sua personalità.

Se il requisito non viene soddisfatto, l'oggetto smette di funzionare. Se invece viene soddisfatto, di solito l'oggetto funziona per un giorno intero prima di dover di nuovo soddisfare il requisito (anche se alcuni requisiti vanno soddisfatti una volta sola, altri una volta al mese e altri ancora in continuazione).

\subsection{Inconveniente}

Gli oggetti che hanno degli inconvenienti hanno solitamente degli effetti positivi su chi li usa, ma hanno anche degli aspetti negativi. Anche se a volte gli inconvenienti vengono alla luce solo quando gli oggetti sono utilizzati (o tenuti in mano, nel caso di oggetti come le armi), di solito rimangono presenti fino a quando il personaggio non si libera dell'oggetto in questione.

A meno che non sia indicato diversamente, gli inconvenienti rimangono attivi per tutto il tempo in cui l'oggetto rimane in possesso del personaggio. La DC dei Tiro Salvezza per evitare questi effetti è pari a 10 + DC della maledizione (se non avete stabilito la difficoltà impostate il Tiro Salvezza, solitamente su Volontà, a DC 25)

\medskip

\begin{narratore}[Creativi ma non punire]
L'elenco è di esempio per poter generare casualmente degli effetti sul possessore dell'oggetto. Prendete spunto e siate creativi!Non fate però che una maledizione renda impossibile giocare il personaggio piuttosto deve essere vissuta come l'occasione per provare, fare, qualcosa di diverso. Non gettate mai un oggetto maledetto a caso nel mucchio dei tesori, pensate sempre cosa potrà accadere e quali conseguenze si genereranno. Un oggetto maledetto richiede sempre un alto livello di attenzione e pianificazione da parte del Narratore\end{narratore}

\medskip

\end{multicols}

\vfill

\begin{center}
\includegraphics[width=0.35\linewidth]{immagini/donnalemb.png}

\emph{Donna di Lemb o Statua della Dea della Morte, 3500 AC. DC 40}
\end{center}

\pagebreak

\textbf{Tabella: Effetti degli oggetti magici maledetti}\index[Tabelle]{Tabella Effetti degli oggetti magici maledetti}

\medskip

%{\small
\noindent\begin{tabularx}{\linewidth}{lX}
	\toprule
\rowcolor{gray!20}\textbf{\%} & \textbf{Inconveniente}\\
\toprule
01-02& I capelli del personaggio crescono di 2,5 cm all'ora.\\
\rowcolor{gray!20}02-04& Le unghie del personaggio crescono di 1 cm ogni 8 ore\\
05-06 & L'altezza del personaggio diminuisce di 5d10 cm \\
\rowcolor{gray!20}07-09 & L'altezza del personaggio aumenta di 5d10 cm \\
10-11 & La temperatura intorno all'oggetto è di 5° C più fredda del normale.\\
\rowcolor{gray!20}12-13 & La temperatura intorno all'oggetto è di 20° C più fredda del normale.\\
14-15 & La temperatura intorno all'oggetto è di 5° C più calda del normale.\\
\rowcolor{gray!20}16-17 & La temperatura intorno all'oggetto è di 20° C più calda del normale.\\
18-20 & Il colore dei capelli del personaggio cambia.\\
\rowcolor{gray!20}21-23 & II colore della pelle del personaggio cambia.\\
24& Il colore dei capelli del personaggio cambia ogni ora\\
\rowcolor{gray!20}25& Il colore della pelle del personaggio cambia ogni ora\\
26& Delle corna come un montone crescono sulla testa del personaggio\\
\rowcolor{gray!20}27& Un palco di corna come un alce crescono sulla testa del personaggio\\
28-29 & II personaggio ora porta un segno distintivo (un tatuaggio, una strana luminescenza ecc.).\\
\rowcolor{gray!20}30-32 & II sesso del Personaggio cambia ogni giorno all'alba.\\
33-34 & La razza o la specie del Personaggio cambiano.\\
\rowcolor{gray!20}35& II PG viene colpito da una Malattia determinata casualmente, che non può essere curata.\\
36-39 & L'oggetto emette costantemente suoni sgradevoli (lamenti, maledizioni, insulti...).\\
\rowcolor{gray!20}40& L'oggetto ha un aspetto ridicolo (colori sgargianti, forma, brilla di un alone rosa ecc.).\\
41& Un unicorno blu, visibile solo con la magia, di dimensioni piccole vola sempre attorno al Personaggio dando consigli inutili e facendo battute stupide.\\
\rowcolor{gray!20}42& Ogni giorno ti prende una improvvisa voglia e capacità di fare l'uncinetto per almeno 1 ora.\\
43-45 & II personaggio diventa estremamente possessivo nei confronti dell'oggetto.\\
\rowcolor{gray!20}46-49 & II personaggio ha una paura incontrollabile di perdere l'oggetto o che venga danneggiato.\\
50& Un Tratto viene sostituito\\
\rowcolor{gray!20}51& Il metabolismo del personaggio cambia e diventa esclusivamente carnivoro\\
52& Il metabolismo del personaggio cambia e diventa esclusivamente vegetariano\\
\rowcolor{gray!20}53-54 & II personaggio deve attaccare la creatura a lui più vicina (probabilità del 5\% ogni giorno).\\
55-57 & II personaggio rimane Stordito per 1d4 round ogni volta che l'oggetto è servito al suo scopo\\
\rowcolor{gray!20}58-60 & Il personaggio diventa sordo\\
61-64 & I Punti Ferita massimi calano di 10 permanentemente (rimanendo con un minimo di 1).\\
\rowcolor{gray!20}65& I Punti Ferita massimi calano di 20 permanentemente (rimanendo con un minimo di 1).\\
66-68 & Il PG acquisisce una Fobia a caso.\\
\rowcolor{gray!20}69-71 & TS su Volontà ogni giorno all'alba con mod. Intelligenza o subisce 1 danno a Intelligenza permanente.\\
72-74 & TS su Volontà ogni giorno all'alba o subisce 1 danno a Saggezza permanente.\\
\rowcolor{gray!20}75-77 & TS su Volontà ogni giorno all'alba con mod. Carisma o subisce 1 danno a Carisma permanente.\\
78-80 & TS su Tempra ogni giorno all'alba con mod. Forza o subisce 1 danno a Forza permanente.\\
\rowcolor{gray!20}81-83 & TS su Tempra ogni giorno all'alba con mod. Destrezza o subisce 1 danno a Destrezza permanente.\\
84-85 & TS su Tempra ogni giorno all'alba o subisce 1 danno a Costituzione permanente.\\
\rowcolor{gray!20}86-89& Il PG incomincia a parlare di se in terza persona.\\
90-92& Saurovalli, cani e gatti domestici diventano ostili.\\
\rowcolor{gray!20}93& Un Patrono farà di tutto per ucciderti.\\
94& Il PG viene teletrasportato a 10d100 Km di distanza ogni giorno all'alba.\\
\rowcolor{gray!20}95& II personaggio viene trasformato in una creatura a caso di una specie specifica (probabilità del 5\% ogni giorno).\\
96& II personaggio viene trasformato in una creatura specifica (probabilità del 5\% ogni giorno).\\
\rowcolor{gray!20}97& II personaggio non può più usare oggetti magici o Incantesimi con livello oltre 5\\
98& II personaggio non può più usare oggetti magici o Incantesimi con livello oltre 3\\
\rowcolor{gray!20}99& II personaggio non può più usare Incantesimi\\
100 & Tira due volte
\end{tabularx}
%}

\pagebreak

\section*{Licantropia}\index{Licantropia}\label{Licantropia}\hypertarget{Licantropia}{}

\begin{multicols}{2}

Le creature mannare sono umanoidi condannati a trasformarsi in animali o in ibridi animali-umanoidi al sorgere della luna piena. Questa maledizione è trasmissibile tramite morso/ferita e procreazione. Una creatura mannara è potenzialmente inconsapevole della sua maledizione fino all'arrivo della prima luna piena.

\subsection{La forma ibrida}\index{La forma ibrida}

Dopo la prima mutazione completa la creatura acquisisce la capacità di trasformarsi a volere nella creatura mannara di forma ibrida.

La forma \textbf{ibrida}, a parte le ovvie trasformazioni fisiche, concede:

\begin{itemize}[leftmargin=*] \setlength{\itemsep}{0pt}
	\item la creatura acquisisce anche il tipo Bestia e Mutaforma
	\item Aumenta di una taglia se il tipo di mannaro è più grande della taglia della creatura originaria
	\item Le Caratteristiche fisiche e Difesa aumentano di 2.
	\item Il Tiro Salvezza su Volontà aumenta di 1.
	\item Può attaccare con artiglio o morso causando 1d6 + Forza. Usa la Lista Armi Accette e Scuri ed è competente nell'arma.
	\item Acquisisce il doppio del livello in Punti Ferita temporanei indipendentemente dall'aumento di Costituzione.
	\item Acquisisce i sensi riportati dalla versione mannara della creatura.
	\item Ottiene +2d6 ad interagire con gli animali della sua forma mannara.
	\item La creatura diventa Vulnerabile all'argento.
\end{itemize}

\subsection{La forma mannara completa}\index{La forma mannara completa}

La forma \textbf{mannara completa} a parte le ovvie ed importanti trasformazioni fisiche, concede e modifica quelle della forma ibrida in questa maniera:

\begin{itemize}[leftmargin=*] \setlength{\itemsep}{0pt}
	\item Le Caratteristiche fisiche e Difesa aumentano di 3.
	\item Il Tiro Salvezza su Volontà aumenta di 2.
	\item Il danno con artiglio o morso causa 1d8 + Forza.
	\item I Punti Ferita temporanei aumentano del quadruplo del livello della creatura.
\end{itemize}

\subsection{Trasformarsi in mannaro}

Al sorgere della luna piena, dalle 22.00 alle 06.00, la creatura si trasforma nella sua versione mannara completa e la mattina non ha ricordi di ciò che ha fatto.

La creatura può resistere con un Tiro Salvezza su Tempra con DC pari a 15 + livello della creatura stessa. La trasformazione impiega 1 minuto e tornare in forma originale lascia Affaticato 2.

Solo dopo essersi trasformato in un mannaro completo sarà possibile attivare la trasformazione in forma ibrida a volere, usando 2 Azioni.

La creatura maledetta dalla licantropia da un altro mannaro tramite ferita acquisisce la capacità di trasformarsi spontaneamente in mannaro completo solo dopo 1 anno dalla prima trasformazione, solo di notte e con luna presente.

Figli di creature mannare hanno un 33\% di possibilità, per genitore mannaro, di essere licantrope naturali e quindi sin dalla nascita poter comandare i loro poteri.

\subsection*{Curare la licantropia}

La licantropia è una maledizione antica e potente e non è facile da rimuovere.
Se ferito da un mannaro deve effettuare un Tiro Salvezza come descritto dalla descrizione del mostro. Se il Tiro Salvezza iniziale fallisce allora è necessario un Rimuovi Maledizione contrastato a DC 21 + livello della creatura stessa.

\end{multicols}

\vfill

\begin{center}
	\includegraphics[width=0.45\linewidth]{immagini/William_Blake_-_Nebuchadnezzar.png}

	\medskip

	\emph{William Blake - Nebuchadnezzar, Tate Museum}
\end{center}

\pagebreak

\section{La Terra}\index{Terra}\index{Atilantis}

\begin{enfasi}{
Così la Terra è davvero tonda. Però non immaginavo che fosse azzurra. Perché gli uomini che vivono su un pianeta tanto bello non fanno altro che combattere tra loro? (Nadia - Il mistero della pietra azzurra)

\medskip

Il pianeta non ci appartiene, siamo noi ad appartenergli. Noi siamo di passaggio, lui rimane. (Pierre Rabhi)}\end{enfasi}

\begin{multicols}{2}

\label{terra}

La Terra, inteso proprio come pianeta, per come lo abbiamo conosciuto è un ricordo oramai sbiadito.

Tàhil fu strumento diretto di Calicante per sfogare la rabbia dell'uccisione di un suo figlio.

Nessuna città fu risparmiata, ogni insediamento fu vittima di terremoti, inondazioni, epidemie. Da un giorno all'altro le città vennero letteralmente ribaltate sottosopra.

Passati 30 giorni Calicante decise che non bastava distruggere tutto e tutti ma era necessario cambiare una volta e per sempre la Terra, era necessario che nuovi Patroni intervenissero, era necessario una Rifondazione.

Questi attinsero alla cultura e tradizione, alla paure più recondite agli incubi più terrorizzanti e da migliaia di portali arrivarono orde demoniache, orchi, draghi e abomini come generati dalla mente fervida di qualche pazzo. I morti si sollevarono dalle tombe e presero a cacciare i vivi.

Anche se durarono solo 1 anno le manifestazioni dirette di potere dei Patroni l'umanità perse oltre il 90\% della sua popolazione nel tentativo di difendersi, di sopravvivere.

L'industria, la conoscenze vennero distrutte e nel secolo successivo la barbarie ed ignoranza non ci ha certo aiutato a riprenderci.
Fu l'Editto della Dimenticanza che distrusse tutto. L'Editto emise una potente onda magica che fuse i componenti di ogni apparato elettrico e allo stesso tempo cancellò qualsiasi dato potesse essere li custodito, ma non fu questo il peggio: l'Editto confuse le parole di ogni libro scritto.

Molti degli apparati sono ancora li, dove erano in origine, la maggior parte vandalizzata per recuperare materiali, altri invece chiusi chissà in quale segreto posto. Qualche speranza c'è ancora di trovare un apparato funzionante, pur se remota o magari un libro ancora leggibile.

Le vecchie rovine delle città sono spesso ricettacolo di orde di nefaste creature che aspettano solo di potersi nutrire. Molte zone non sono state rimappate e pur sapendo cosa c'era non si sa più cosa c'è adesso.

Poche sono le città che superano i 50000 abitanti. Ogni stato ha una capitale che per il beffardo destino della Terra molto spesso viene distrutta o scompare. La legge è spesso assente e solo quella del più forte vige.

Ampie lande inesplorate si dipanano dove antichi resti di civiltà scomparse sono rifugio di nuovi abitanti. Ci sono strati su strati di civiltà sepolte sotto i propri piedi con tesori, segreti, caverne e protettori.

\subsection{Società}\index{Società sulla Terra}\label{societasullaterra}

Le dinastie raramente sono destinati a regnare per più di qualche generazione, guerre fratricide, attacchi dall'esterno, voleri di Patroni, fanno che le forme societarie più rigide facciano fatica a prosperare e lo spirito \emph{democratico} non è sempre così sviluppato da permettere di creare società evolute che possano adattarsi alla situazione.

Le nazioni hanno così confini molto labili, spesso definiti dalla geografia più che dalle conquiste. Non sempre gli eserciti possono difenderli da attacchi esterni e ancora più spesso le milizie devono concentrarsi a difendere la città principale da attacchi interni, ribellioni o improvvise orde di mostri usciti da chissà quale impronta di Cattalm.

In tutto il mondo la forma di governo e società più diffusa sono le Città Stato, roccaforti e terreni raccolti attorno ad una città in grado di difenderli e proteggerli dagli assalti esterni, governate da un leader forte con l'appoggio di un Patrono.

Piccoli e grandi villaggi sorgono ovunque nel territorio, attorno a fonti d'acqua e risorse naturali, spesso sono in balia di bande di disperati se non di gablin.
Spesso è qui che i nostri eroi hanno la prima formazione nel tentativo di difendere prima la loro casa, poi il villaggio, dall'assalto di qualche astuto e sanguinario nemico.

Ancora si ergono i resti di magnifiche città del passato e spesso queste tornano ad essere popolate anche se sugli abitanti spesso aleggia la maledizione che ha condannato la Prima Era.

Il sottosuolo possa essere caverne, catacombe o infiniti cunicoli se non vere e proprie città sotterranee sono per tutta la Terra, memoria imperitura, stratificata e ristratificata, della sua storia. Non c'è mai una fine a quanto si può andare in profondità, c'è sempre qualcos'altro sotto di ancora più magnifico e pericoloso.

Le leggende parlano di intere regioni inghiottite sottoterra, città che dal giorno alla notte sono scomparse in una nube di polvere. Ovunque sono presenti accessi alle profondità dove si favoleggiano tesori e ricchezze, dove la Legge del Premio aspetta chi osa raccogliere la sfida.

\subsection{Le Religioni}\index{Le altre Religioni}

Le \emph{vecchie} religioni esistono ancora. I Patroni non ti fanno guerra perché adori un altro Patrono o Dio, non è il loro scopo o interesse. La religione ha acquisito un ruolo molto più concreto e reale, con numerosi estremisti che vanno profetizzando la venuta di altri patroni, dei veri Dei. E' sempre molto pericoloso manifestare il proprio credo, non sai mai come potrebbero reagire gli altri.

Le stessi Devoti o Seguaci se pubblicamente declamano i loro Patroni rischiano a volte il linciaggio, quasi tutti gli usufruitori di magia vengono visti come persone che hanno venduto l'anima al \emph{diavolo}. La fiducia è un bene prezioso che va guadagnato.

\subsection{Avventure}\index{Avventure}\label{avventure}

In un'ambientazione così complessa, le possibilità per le avventure sono praticamente infinite. Ecco alcune idee che potrebbero ispirarti:

\begin{itemize}[leftmargin=*] \setlength{\itemsep}{0pt}
\item I personaggi sono inviati a esplorare le zone devastate e mutate, alla ricerca di risorse, sopravvissuti o antiche tecnologie
\item Le diverse fazioni umane e non umane sono in in guerra per il controllo delle risorse o dei territori. I personaggi potrebbero scegliere una fazione o cercare di mediare la pace
\item Proteggere un villaggio dagli attacchi di entità ostili o mostri evocati dai Portali.
\item I Patroni vogliono manipolare i leader dei paesi per obbligare il loro culto. I personaggi devono sventare complotti e tradimenti.
\item Formare alleanze con i Patroni per ottenere poteri magici o protezione, navigando tra le diverse richieste e aspettative di questi \emph{esseri}.
\item Indagare sull'origine e la natura dell'Omniessenza, cercando di capire come sfruttarne i residui senza causare ulteriori catastrofi.
\item I personaggi sono apprendisti maghi che devono recuperare un qualche incantesimo fondamentale per salvare il loro villaggio/città.
\item Aiutare a ricostruire le comunità devastate, trovando risorse, costruendo infrastrutture e difendendole dalle minacce.
\item Sopravvivere in ambienti mutati e pericolosi, come deserti nucleari o foreste infestate da creature mostruose.
\item Esplorare miti e leggende per scoprire verità nascoste e artefatti potenti, scoprire cosa è rimasto mito e leggenda e cosa è stato creato
\item Utilizzare i Portali per esplorare altri mondi o dimensioni, affrontando le sfide e scoprendo i segreti di questi luoghi.
\item Missioni per salvare gruppi di sopravvissuti intrappolati in zone pericolose o sotto l'influenza di creature maligne.
\item Esplorare eventi inspiegabili e fenomeni paranormali causati dalle \emph{entità} (uscite da portali ?), cercando di scoprire la verità dietro di essi.
\item Rappresentare una comunità o una fazione in missioni diplomatiche presso altri popoli, cercando di ottenere supporto o risorse.
\item Attraverso i Portali è possibile viaggiare nel tempo. Cercare di prevenire catastrofi o correggere errori del passato, affrontando i rischi dei paradossi temporali.
\item Organizzare e guidare una ribellione contro leader/nemici oppressivi, cercando di liberare territori occupati.
\item Cercare di comprendere i Patroni, cercando di ottenere il loro favore.
\item Supportare l'espansione della propria comunità nel territorio, conquistando nuove terre e difendendole dagli attacchi nemici.
\item Proteggere avamposti strategici da assalti nemici, utilizzando tattiche e risorse limitate.
\item Alcune regioni e città hanno subito uno sfasamento temporale, tornando ad epoche antiche.
\item Strani accadimenti stanno succedendo nel territorio, tutte situazioni allineate lungo le linee di Ley.
\end{itemize}

Il \emph{problema} per gli avventurieri ed esploratori è l'estrema diversificazione e mutevolezza non solo dell'ambiente ma anche delle culture e civiltà che si possono incontrare.

La Terra non si potrà più dire esplorata, la stessa zona può cambiare da un giorno all'altro perché un Patrono ha deciso così. Curiosi, ineffabili, volubili sono capaci di costruire in un battito di mani l'avventura della vita solo per godersi lo spettacolo.

Saranno orde di gablin affamati nel dedalo delle profondità della città, saranno orde barbariche devote a Cattalm ad uccidere e rapire la prole, saranno regni di ghoul che spuntati dal nulla vorranno mangiare tutto e tutti, saranno carestie e pestilenze risolvibili sono ritrovando antichi artefatti, potranno essere antiche città spuntate da una impronta di Cattalm, putride paludi in piena espansione cariche di mostri...

I Patroni faranno di tutto per sconfiggerti ed umiliarti ma ricorda bene la Legge del Premio è superiore anche a loro!

\subsection{Nuovi luoghi notevoli sulla Terra}\label{luoghinotevoli}

\subsubsection{Deserto di Kranguran}

In questo immenso deserto si celano giganteschi mostri. Alcuni nascosti sotto la sabbia usano il senso tellurico per cacciare le loro prede, altri giganteschi dinosauri cacciano qualsiasi preda possa capitare.

Ogni creatura in questo deserto è gigantesca, mostruosa e sproporzionata nell'aspetto, quasi fosse nato dall'incubo di qualcuno.

La stessa vegetazione nelle poche oasi presenti è enorme ed ipertrofica. Nessuna creatura che qui nasce può uscire, nessuna creature che qui muore lascia veramente il posto. La leggenda narra che ogni \emph{mostro} in questo deserto sia in realtà una creatura qui morta. A seconda del potere della creatura questa è rinata come più gigantesco mostro affamato.

Secondo i più il Deserto di Kranguran è un luogo di gioco di Cattalm.

\subsubsection{Città di Knandir}

Questa ricca, prosperosa e popolosa città antica fu distrutta nell'arco di una notte da un gigantesco cataclisma.

Si dice che il volere del Patrono fu talmente pervasivo che tutti gli edifici vennero distrutti o severamente danneggiati. Non soddisfatto dell'opera condanno la città a essere sfasata rispetto alla realtà, facendola scomparire agli occhi di tutti gli altri.

I pochi abitanti sopravvissuti perirono tra atroci sofferenze, condannati a non poter uscire, a non aver da magiare o bere.

La città venne maledetta e nei pochi giorni dell'anno in cui è possibile raggiungerla ogni persona che ci mette piede per saccheggiare gli immensi tesori contenuti sembra condannato a non uscire più, vittima della maledizione o dei numerosi fantasmi, spiriti e non morti dei precedenti abitanti.

La città oltretutto non appare mai nello stesso posto ma si sposta seguendo uno schema non ben compreso. Si dice che la Pergamena perduta di Knandir ne spieghi gli spostamenti, un peculiare oggetto vetroso con una mappa che ne indica la posizione con una luce che appare e scompare.

%\subsubsection{Il Mare silente}

%C'è una particolare zona di mare, compresa tra tre isole maggiori e contenente diverse isole minori, dove qualsiasi suono viene silenziato. Un suono che venga generato in quelle acque, e non sulla terra ferma, viene zittito.

\subsubsection{La torre dei gorilla blu}

L'origine di questo antico e magico edificio è ormai dimenticata, si dice che fosse stata creata per sfidare un Patrono, probabilmente Gradh. La torre, a base quadrata di 20 metri di lato è apparentemente alta 7 piani. In ogni piano, la cui mappa sembra essere costantemente mutevole, appaiono dei gorilla blu, assolutamente brutali e con l'intenzione di uccidere chiunque sia nella torre. Una volta sconfitto l'ultimo gorilla del piano la porta che conduce alle scale per il piano successivo si apre e i personaggi possono salire. Ad ogni piano i gorilla diventano più forti, resistenti e più intelligenti.

E' noto che già al 4 piano acquisiscano anche poteri magici. I personaggi entrati possono uscire quando vogliono, se dovessero morire all'interno della torre verranno automaticamente teletrasporati fuori, ma vivi ad 1 Punto Ferita ed estremamente affaticati, senza l'oggetto più prezioso che avevano addosso al momento della morte. Non ci sono oggetti magici all'interno della torre, almeno nei piani noti, l'unica cosa che i personaggi guadagnano è esperienza per i combattimenti fatti. L'attuale record è stato raggiungere il 7 piano. Riusciranno dei nuovi eroi ad arrivare alla fine (???) della torre, e che premi ci saranno per chi sopravvive?

\subsubsection{e la Terra ?}\index{La Terra}

Il nostro mondo esiste e persiste, a discapito di ogni distruzione e tragedia. Per quanto la descrizione fatta fin'ora possa sembrare apocalittica, alcuni luoghi sono ancora quasi integri, piccoli angoli di un paradiso dimenticato.

Non troverete una industria o fabbrica funzionante ma piuttosto una piccola impresa, un valente artigiano, qualcuno che porta avanti le tradizioni familiari arrabattandosi come può.

Molti luoghi d'altronde sono stati svuotati, depredati, parzialmente distrutti, città abitate non più da umani ma da \emph{altro}, metropolitane che sono diventate i nuovi dungeon, catacombe dimenticate che sono tornate ad essere luogo di rifugio, città seppellite da terremoti eppure ancora \emph{abitate}. I massicci rifugi antiatomici, i poderosi condomini sottoterra, hanno resistito eppure hanno dovuto dovuto dividere gli spazi vitali con chi è stato mutato, cambiato.

Si dice che da qualche parte, dove la Freten è esplosa ci sia il portale che conduce agli stessi Patroni della Genesi, quel che è certo è che ovunque ci sono Portali che portano in mondi fantastici, stupendi o mortalmente terrorizzanti.

Molti cercano il nuovo eden saltando da un portale all'altro sperando di trovare il mondo giusto sempre alla ricerca di una avventura o della tranquillità.

Ricordate che la natura è stata mutata, nuove specie ispirate ai manuali dei giochi sono state create per diletto, mostri di ogni genere forma e morale sono presenti solo per farli combattere con i sopravvissuti. Foreste gigantesche, lussureggianti, piene di vita raramente amichevole, sono spuntate in mezzo ai deserti. La grande jungla del Sahara è tra le più temute ed evitate anche se all'interno ci sono immensi giacimenti di minerali preziosi.

E ovunque, Draghi! Innumerevoli, affamati, cattivi.

\subsubsection{I vecchi Stati}

E' impossibile in queste poche righe descrivervi come tutto il pianeta sia stato \emph{riscritto}. La magia dei Patroni è assoluta ed il loro volere Legge, non stupiamoci se quello che era il Deserto del Sahara adesso è la più fitta e lussureggiante jungla del pianeta, conosciuta come Giardino di Shayalia.
Buona parte della zona est della Russia, quella ai confini con gli ex stati dell'est Europa è diventa l'Impero dei Ghoul, uno dei luoghi più terribili dove vivere, se non si è devoti di Sixiser.

Molto del Nord America è un deserto nucleare con le poche popolazioni che si sono rifugiate nelle coste est ed ovest, cacciati da bande di predoni cannibali e mutati sputa acido.

Quello che era il Brasile è diventato una zone totalmente selvaggia in mano a dinosauri e popolazioni folli che adorano Orudjs.

La parte dell'Italia centrale è sotto la teocrazia di Rezh mentre numerosissime contee, baroniee e semplici città si sono auto proclamate sotto la protezione di un qualche signorotto locale.

La Francia è comandata direttamente dal nuovo Re Sole, pardon, Re Torbion XXIII che invaghitosi della storia e cultura ha voluto riproporre, con volere questa volta veramente divino, gli sfarzi ed atteggiamenti di quella corte e periodo, rendendo il tutto tremendamente più pericoloso ed infido.

La Germania, quella che era il motore della vecchia Europa, ha subito tra i danni maggiori, ritornando ad uno stato barbarico, con una involuzione culturale e naturale forzata da Efrem.

Buona parte delle terre tra Francia e Germania sono tornate ad uno spirito più primitivo ed ancestrale, qui Gaya ed Erondil hanno creato i loro culti maggiori ispirati a quella che era la tradizione celtica.

Le fredde terre del nord Europa si sono isolate dopo che i loro morti sono risorti. Questa volta per volontà delle persone è stato chiesto aiuto a Krondal e Nedraf perché li potessero salvare. Nedraf gli diede le armi e l'esperienza per usarle, Krondal, da vero folle fece tornare gli ancestrali ricordi di un passato guerriero fatto di miti e Dei dimenticati, o meglio ignorati, dai più.
Così Krondal ha ricreato come suoi servitori Aegir, Alfadur, Hel, Idhunn, Norne per non citare i più noti Thor, Loki, Valchirie...

\begin{narratore}[Mappe alternative]
Usate le mappe geografiche fisiche reali terrestri per aiutarvi con l'ambiente. Cercate online le mappe delle antiche città. Avete a disposizione il più grande setting mai creato, si tratta solo di popolarlo con i miti, leggende, storie, fantasia che già sono intorno a voi.

Ogni città ha le sue leggende storiche, scopritele e giocatele insieme!
\end{narratore}

\end{multicols}

\subsection{I Portali}\index{Portali}

\begin{enfasi}{
Non aprire mai le porte a coloro che le aprono anche senza il tuo permesso. (Stanislaw Jerzy Lec)}\end{enfasi}

\begin{multicols}{2}

Questo proliferare di piccoli, grandi, duraturi o istantanei Portali ha causato uno squarcio nel tessuto dimensionale della Terra generando a sua volta un proliferare di tunnel spontanei più o meno grandi e duraturi.

Questi occasionali Portali saranno spesso la causa di situazioni da affrontare e sconfiggere. Ogni Portale ha una propria DC da superare per poter essere chiuso con l'incantesimo \hyperlink{Chiudi Portale}{Chiudi Portale} (pag. \pageref{Chiudi Portale}), solitamente questa DC va dai 20 per i più facili ai 40 per quelli permanenti.

Ci sono portali conosciuti e stabili, fino ad ora, che collegano continenti. I Portali più interessanti per il commercio sono quasi tutti sotto il controllo, per non dire dentro la roccaforte, di reali e potenti.

Non tutti i Portali sono pericolosi, molti fungono semplicemente da varchi di trasporto da un paese all'altro, possa essere anche a centinaia di kilometri. Altri Portali porteranno in altri mondi, da esplorare e forse vivere, altri invece porteranno in luoghi che è bene evitare e che dovranno essere sigillati e chiusi per evitare ulteriori invasioni di \emph{xenomorfi alieni}.

Potete inventare mille ed una avventura dietro ai Portali, ognuno è una possibilità di un mondo diverso e di una moltitudine di avventure.

\begin{narratore}[I Portali]
Dovete intendere i portali come chiave per mille ed una avventura. Ogni portale vi condurrà in un posto diverso, fantastico come voi lo intendete. Volete un avventura in un mondo primitivo, ambientata nella società moderna, in un pianeta chissà dove? Usate i portale per spalancare le porte della vostra immaginazione.

Gli stessi personaggi potrebbero essere non \emph{terrestri} e cercare un modo per tornare a casa...
\end{narratore}

\begin{narratore}[Ambientazione]
Usate l'ambientazione che più preferite! Questo mondo è un esempio di un mondo caotico e leggermente anarchico dominato dai continui cambiamenti di umori di divinità capricciose.
Scegliete voi l'ambientazione, usate Greyhawk, Dark Sun, Mystara quello che preferite. Siete voi il Narratore, siete voi il mondo, siete voi a proiettare luce ed oscurità, OBSS vi fornirà gli strumenti per condurre le vostre campagne!

%Il primo suggerimento che vi do è di conoscere bene l'ambientazione, maggiore sarà la vostra conoscenza, più facilmente saprete adattarvi alle situazione che vi andranno a capitare.
\end{narratore}

%\begin{center}
%	\includegraphics[width=0.7\linewidth]{immagini/ancientwell.png}
%
%	\emph{Antico pozzo e portale}
%\end{center}

\end{multicols}

\vfill

\begin{enfasi}
In principio fu il Caos, poi nacque Gaia dall'ampio petto, sede sicura per sempre di tutti gli immortali. (Teogonia, Esiodo)
\end{enfasi}

%\vfill

%\begin{enfasi}{
%Una volta eliminato l'impossibile ciò che rimane, per quanto improbabile, dev'essere la verità. (Sir Arthur Conan Doyle)
%} \end{enfasi}\end{changemargin}

%Tàhil rosso
%Elysan argento
%Curyan vita
%Tiya scuro

\pagebreak

\subsection{Il Calendario}\index{Calendario}

\begin{enfasi}{
Mi è capitato spesso di finire su un calendario. Ma mai per una data precisa. (Marilyn Monroe)

\medskip

Tutto ebbe inizio la tredicesima ora del tredicesimo giorno del tredicesimo mese... Eravamo lì per discutere degli errori di stampa dei calendari acquistati dalla scuola. (I Simpson)} \end{enfasi}

\medskip

\begin{multicols}{2}

\label{il-calendario}

Basato sul ciclo lunare presenta 12 mesi da 28 giorni.

Questi i nomi dei mesi a partire da quello che si definisce inizio anno:

1°) Ianas (stagione: primavera)

2°) Prineva (stagione: primavera)

3°) Marc (stagione: estate)

4°) Epral (stagione: estate)

5°) Meea (stagione: estate)

6°) Vernam (stagione: autunno)

7°) Ilai (stagione: autunno)

8°) Arkast (stagione: autunno)

9°) Cester (stagione: inverno)

10°) Koper (stagione: inverno)

11°) Narava (stagione: inverno)

0°) Raanant* (speciale)

12°) Kartan (stagione: primavera)

\medskip

\emph{Raanant} è il mese che si festeggia alla fine del Ciclo secolare, ogni cento anni. E' un mese di libertà dai Patroni, dalle Leggi, è il mese della catarsi e della violenza, della libertà e della rinascita.

\medskip

Un ciclo di sette giorni, settimana, è composta da giorni di nome:

\medskip

1°) Kalint (solitamente festivo)

2°) Iratam

3°) Munrat

4°) Arai

5°) Venran

6°) Kittam

7°) Viltar

\medskip

Il giorno è diviso in 24 ore. L'anno corrente è il 125 del nuovo calendario.

\subsection{Oltre la morte}

Gli abitanti hanno una visione abbastanza pessimistica di ciò che succede dopo la morte. Per i più dopo la morte non c'è nulla se non la dissoluzione del corpo.

I Devoti e Seguaci credono che il loro spirito si riunirà con il Patrono, rendendolo più forte.

Altri ancora credono ancora che ogni spirito si incarni per 4 volte per essere poi giudicato dai Patroni della Genesi e mandato nel piano a lui assegnato.

Quale sia la verità non è dato saperlo.

\end{multicols}

\vfill

\begin{center}
\includegraphics[width=0.5\linewidth]{immagini/Laminas_8_y_9_del_Codice_de_Dresden2.png}

\medskip

\emph{Codice di Dresda, pagine 10 e 11}
\end{center}

\pagebreak

\subsection{I Cicli Secolari}\index{I cicli secolari}

\begin{enfasi}{
Vidi poi un angelo che scendeva dal cielo con la chiave dell'Abisso e una gran catena in mano.

Afferrò il dragone, il serpente antico - cioè il diavolo, satana - e lo incatenò per mille anni; lo gettò nell'Abisso, ve lo rinchiuse e ne sigillò la porta sopra di lui, perché non seducesse più le nazioni, fino al compimento dei mille anni. Dopo questi dovrà essere sciolto per un pò di tempo. (Apocalisse 20,1-3, apostolo Giovanni)
}\end{enfasi}

\begin{multicols}{2}

Dice il mito che ogni cento anni la Terra muoia per rinascere nuovamente, più bella di prima. Non è proprio così ma ci si avvicina molto.

E' noto a pochi eruditi di Atmos che ogni secolo i Patroni riconosciuti, e da cui molti traggono i poteri, scompaiano e lascino il posto, dopo esattamente 1 anno a nuovi Patroni.

Improvvisamente gli incantesimi cessano di funzionare, solo gli oggetti magici che possono assorbire e conservare la magia funzionano (come ad esempio una Pozione, una Armatura o Arma se non un Anello od un Bastone che abbia delle cariche, ma non oggetti che si ricaricano automaticamente come le Verghe), neanche i Devoti o Seguaci hanno più accesso a nessun incantesimo.

\begin{wrapfigure}[20]{r}[.5\width+.5\columnsep]{7cm}

\centering
\includegraphics[width=6cm]{immagini/Aztec_calendar.png}

\medskip

\emph{Antico calendario Azteco}
\end{wrapfigure}

Con qualche eccezione. I Patroni della Genesi, Atmos e Lynx ed il Patrono Vincitore sono gli unici a rimanere costanti e non cambiare. Solo i loro Devoti e Seguaci possono continuare ad usare gli incantesimi a disposizione nell'anno di intermezzo.

A partire dal sesto mese i Seguaci e Devoti dei precedenti patroni incominciano a sentire delle voci, a sognare nuovi volti e nomi di nuovi Patroni.

Questo è quello che successe alla fine della prima venuta, con la sola differenza che Ljust concesse la vittoria a Calicante per poter interrompere immediatamente il ciclo e salvare dalla distruzione il nostro mondo dalla vendetta per aver ucciso il Primo Patrono.\index{Il primo ciclo}\index{I primi Patroni}

Ogni nuovo Patrono, in base ai Tratti che comanda, avvicina un Seguace o Devoto e cerca di convincerlo ad accettarlo come nuovo Patrono. Gli incantatori solo al termine dell'anno potranno usare gli incantesimi, indipendentemente che seguano un Patrono o meno.

E' un periodo estremamente turbolento ed agitato dove scoppiano guerre e vendette approfittando dell'assenza della magia. Per molti è un periodi di puro odio e violenza dove vengono sfogati gli istinti più bassi sapendo poi che non si sarà giudicati da nessun Patrono.

La verità è che ogni cento anni i Patroni delle Genesi giudicano i loro figli, i Patroni, valutando chi ha fatto meglio e chi peggio. E' una sfida tra Calicante ed Ljust a chi ha, tramite i Patroni, ottenuto più Seguaci e Devoti.

Il Patrono che più di tutti si è dimostrato capace di conquistare più persone rimarrà anche nel secolo successivo, questo sarà il Vincitore ed i suoi credenti ne canteranno per altri cento anni la gloria e la potenza.

Inebriato dalla vittoria il Patrono della Genesi esprimerà un desiderio che l'altro dovrà cercare di rispettare il più possibile.

Ovvio che il Patrono stesso potrebbe soddisfarlo ma la gioia di obbligare l'altro

\begin{wrapfigure}[20]{l}[.5\width+.5\columnsep]{7cm}

	\centering
\end{wrapfigure}

a fare qualcosa che detesta è superiore a ogni cosa. Ed è per questo che ogni cento anni succede l'impossibile, oltre alla nascita di nuovi Patroni.

Può essere un nuovo continente, un mare che si apre tra le terre, nuove razze, animali... qualcosa di imponente cambia per tutti i terrestri. E' un periodo di sconvolgimenti globali.

Solo i sommi Devoti di Atmos conoscono questa verità come conoscono che i Patroni della Genesi dopo la vittoria giacciono insieme per sei mesi generando i nuovi Patroni.

Un altra verità sconosciuta purtroppo è che in realtà il nostro pianeta è sotto il gioco dei Patroni da molto più di un secolo e che è solo per desiderio di Ljust che non si ha memoria di tutti i cicli precedenti. La Patrona della Luce per non fare perdere la speranza all'umanità ha ottenuto che ci dimenticassimo dei secoli di soprusi causati dalle vittorie continue dei Patroni di Calicante, dalle distruzioni perpetrate dai draghi e mantenessimo una flebile e vitale speranza in un mondo che possa essere più gentile e amorevole verso tutte le sue creature.

Traccia di questi cicli passati si possono trovare negli innumerevoli ed altrimenti ingiustificati pericoli, dungeon, mostri, draghi, città sotterrane che riempiono ogni angolo della Terra.

\end{multicols}

\pagebreak

\section{Mostruario di OBSS}\index{Mostruario}

\begin{enfasi}{Chi lotta con i mostri deve guardarsi di non diventare, così facendo, un mostro. E se tu scruterai a lungo in un abisso, anche l'abisso scruterà dentro di te. (Friedrich Nietzsche)

\medskip

I mostri possono essere sconfitti soltanto dai loro simili. (Claymore)

\medskip

La tragedia dei mostri è di essere troppo grandi e potenti per essere accettati dal genere umano. (Ishiro Honda)

\medskip

Per aspera ad astra! ("attraverso le asperità sino alle stelle")

}\end{enfasi}

\begin{multicols}{2}

Benvenuti in un universo ricco di avversari, spesso cattivi, altre volte violenti, pure subdoli, anche intelligenti, forse meschini e quasi sempre giganteschi.. e quant'altro tu vorrai. I mostri sono il caposaldo di qualsiasi gioco di ruolo fantasy.

Vengono qui spiegati e presentati dei mostri, non certo tutti ne tanto meno esaustivi, usateli per popolare di incubi le avventure dei vostri compagni.

\medskip

\begin{center}

\includegraphics[width=0.9\linewidth]{immagini/sangiorgioedrago.png}

\emph{San Giorgio e il drago (1460 circa) di Paolo Uccello. National Gallery di Londra}
\end{center}

\subsection{Introduzione}

Un avventura non è solo un insieme di avversari ma di situazioni, di luoghi, di sorprese, insomma di tutto ciò che può affascinare, coinvolgere stupire, impegnare i personaggi. Ma anche i mostri servono. Picchiare ha un aspetto catartico, liberatorio.

Inserite nell'avventura mostri difficili e letali dove serve ma ogni tanto, raramente, fate sentire i personaggi potenti, fategli affrontare mostri che in pochissimi round possono risolvere. Descrivete il combattimento enfatizzando i colpi, i critici, il dolore ed il sangue dei mostri. Fate capire quanto possano essere potenti i personaggi.

Altre volte fate che i mostri incutano timore perché' sono grossi, affamati, magici e cattivi, è necessario che i giocatori abbiano paura per i loro personaggi, che non diano mai per scontato la vittoria.

La forza dell'avversario è nella sicurezza nel descrivere la situazione, in poche battute, il fissare negli occhi i giocatori. Coinvolgete i giocatori ed una volta che avrete la loro attenzione anche i personaggi saranno più attenti. Cercate di mettere mostri coerenti all'ambiente, all'avventura, alla situazione. Non tirate a caso su tabelle, uno scontro ben organizzato da molta più soddisfazione che mostri a caso che \emph{spawnano}.

Non riducete tutto a un MMORG dove l'obiettivo è solo uccidere tutto e tutti, ci possono essere sempre tante scelte se ti impegni un pò.

\begin{giocatore}[Affrontare i mostri]
{
Lascia che questo vecchio ti dia un paio di consigli giovane avventuriero!

- Non tutti i nemici si sconfiggono con la spada, molte volte serve anche una mazza!

- A volte le armi e la forza bruta non bastano. Se non hai compagni che possono lanciare incantesimi assicurati di avere sempre la possibilità di appiccare un fuoco.

- Scappa. E' sempre una opzione valida se hai modo e vedi che la situazione non promette niente di buono.

- Organizzati! non entrare nel dungeon a testa bassa senza mai fermarti tranne quando sei morto! Riposati, esplora, controlla l'ambiente e quando sei sicuro e stai meglio prosegui! anche i tuoi nemici si organizzano e si riposano intanto, stai attento!

- A volte si può anche parlare con i nemici, anche loro non vogliono morire sempre.

- Se devi uccidere fallo con cattiveria e velocità. Non perdere tempo e ottimizza i colpi, risparmia le energie e preparati immediatamente ad un altro scontro.

}\end{giocatore}

\subsection{Modificare le Creature}

Nonostante la variopinta collezione di incontri presente in questo manuale, potresti comunque trovarti in imbarazzo quando si tratta di trovare la creatura perfetta per una tua avventura. Sentiti libro di modificare le creature esistenti e trasformarle in qualcosa che ti sia più utile, magari prendendo in prestito uno o due caratteristiche da un mostro diverso.

Tieni a mente che modificare un avversario potrebbe cambiarne il grado di sfida.

\subsection{Taglia e Dimensioni}

Un mostro può essere di taglia Minuscola, Piccola, Media, Grande, Enorme o Mastodontica e Colossale. La tabella Categorie di Taglia mostra la grandezza media di una creatura e quanto spazio occupi sulla griglia.

Se non indicata la portata di una creatura dipende dalla taglia e dall'arma usata (pensate ad un gigantesco spadone brandito da un titano..)

\end{multicols}

\textbf{Tabella: Categorie di Taglia, Quadretti occupati e Portata}\index[Tabelle]{Tabella Categorie di Taglia, Quadretti occupati e Portata}\index{Portata per creature}\index{Quadretti per creature}\index{Taglia e quadretti}\index{Creature per quadretto}\label{tagliaedimensioni}\hypertarget{tagliaedimensioni}{}

\medskip

\noindent\begin{tabularx}{\linewidth}{Xllll}
	\toprule
\rowcolor{gray!20}\textbf{Taglia}& \textbf{Dimensione} & \textbf{Esempio}&\textbf{Quadretti}&\textbf{Portata}\\
\toprule
Minuscola & 25 x 25 cm&Gatto, spiritello& 1/4&0m\\
\rowcolor{gray!20}Piccola & 0,5 x 0,5 m & Goblin, cane, Gnomo&1/2&1m\\
Media & 1 x 1 m & Orco, Umano, Elfo, Nano, Nibali &1&1m\\
\rowcolor{gray!20}Grande & 2 x 2 m& Ogre&2x2&1m\\
Enorme & 3 x 3 m & Gigante, Ent&3x3&2m\\
\rowcolor{gray!20}Mastodontico & 4 x 4 m&Kraken, Drago&4x4&2m\\
Colossale & 12 x 12 m&Drago anziano, Tarrasque&6x6&6m
\end{tabularx}

\medskip

I più avvezzi avranno notato che le dimensioni delle creature sono inferiori alle solite, questo perché le miniature in commercio sono fatte per scala 1 quadretto=1.5 metri, mentre in OBSS 1 quadretto=1 metro.

\begin{multicols}{2}

\subsection{Tipo}

Il tipo di un mostro si riferisce alla sua natura basilare. Certi incantesimi, oggetti magici, Abilità e altri effetti del gioco interagiscono in modi speciali con le creature di un tipo specifico. Ad esempio, una \emph{freccia ammazza draghi} infligge danni extra non solo ai draghi ma anche a tutte le altre creature del tipo drago, come i draghi tartaruga e le viverne.

Il gioco comprende i seguenti tipi di mostri:

\smallskip\textbf{Aberrazioni}, creature totalmente aliene. Molte di esse possiedono innate abilità magiche che attingono alla mente aliena della creatura anziché dalle forze mistiche del mondo. Esempi classici di aberrazioni sono aboleti, divora cervelli ed i fustigatori.

\smallskip\textbf{Bestie}, creature non umanoidi che sono una componente naturale di un mondo fantasy. Alcune possiedono poteri magici, ma la maggior parte è priva di Intelligenza e non ha alcuna forma di società o linguaggio. Esempi classici di bestie sono tutte le specie di animali comuni, i dinosauri e le versioni giganti degli animali.

\smallskip\textbf{Celestiali}, creature native dei Piani Superiori. Molti di loro sono servitori delle divinità, impiegati come messaggeri o agenti nel mondo dei mortali e per i piani.

\medskip

I celestiali sono di natura buona, esempi classici di celestiali sono angeli, couatl e pegasi.

\smallskip\textbf{Costrutti}, sono creati e non partoriti. Alcuni sono programmati dai loro creatori per seguire una semplice serie di istruzioni, mentre altri sono senzienti e capaci di pensare per proprio conto. I golem sono i costrutti più rappresentativi.

\smallskip\textbf{Draghi}, sono grandi creature rettili di antica origine ed enorme potere. I veri draghi, compresi i buoni draghi di Ljust e i malvagi draghi di Tàhil, sono molto intelligenti e possiedono doti magiche innate. In questa categoria si collocano anche creature lontanamente imparentate con i veri draghi, ma meno potenti, meno intelligenti e meno magiche, come le viverne e gli pseudodraghi.

\smallskip\textbf{Elementali}, sono creature native dei piani elementali. Alcune creature di questo tipo sono poco più che masse animate del rispettivo elemento, e includono le creature chiamate semplicemente elementali. Altre creature possiedono forme biologiche infuse di energia elementale. Le razze dei geni, compresi djinn ed efreet, formano le civiltà più importanti dei piani elementali. Altre creature elementali sono gli azer, i persecutori invisibili e le bizzarrie d'acqua.

\smallskip\textbf{Fatati}, sono creature magiche strettamente legate alle forze della natura. Vivono in radure nascoste e foreste nebbiose. Esempi di fatati sono driadi, pixie, fate e satiri e La Topi.

\smallskip\textbf{Giganti}, troneggiano sugli umani e i loro simili. Sono di forma umana, sebbene alcuni abbiano più teste (ettin) o deformità. Le sei varianti dei veri giganti sono Gigante delle Colline, gigante di pietra, gigante del gelo, gigante del fuoco, gigante delle nuvole, gigante delle tempeste. Oltre questi, anche ogri e troll sono giganti.

\smallskip\textbf{Immondi}, vengono genericamente chiamati immondi le creature malvagie provenienti da altri piani. A volte sacerdoti e incantatori malvagi evocano gli immondi nel mondo materiale perché eseguano le loro volontà. Se un celestiale malvagio è una rarità, un immondo buono è praticamente inconcepibile. Gli immondi includono demoni, diavoli, segugi infernali, rakshasa, gablin...

\smallskip\textbf{Melme}, sono creature gelatinose che difficilmente hanno una forma fissa. Vivono principalmente sottoterra, stabilendosi in grotte e sotterranei, nutrendosi di rifiuti, carcasse o creature tanto sfortunate da incapparvi. I protoplasmi neri e i cubi gelatinosi sono tra le melme più riconoscibili.

\smallskip\textbf{Mostruosità}, sono mostri nel senso più stretto del termine creature spaventose che non sono comuni, né davvero naturali, e quasi mai benigne. Alcune sono il risultato di esperimenti magici andati male, mentre altri sono il prodotto di terribili maledizioni (tra cui ricordiamo il minotauro). Sfuggono a qualsiasi categorizzazione, e in qualche modo servono da categoria onnicomprensiva per quelle creature che non corrispondono a nessun altro tipo di mostro.

\smallskip\textbf{Non Morti}, sono creature un tempo vive condotte ad un orribile stato di non morte tramite la pratica della magia negromantica o qualche blasfema maledizione. Tra i non morti si annoverano cadaveri ambulanti, come vampiri e zombi, oppure spiriti incorporei, come fantasmi e spettri. Alcuni non morti più intelligenti parlano Expiran, una lingua fatti di oscuri sussurri.

\smallskip\textbf{Piante}, in questo contesto si tratta di creature vegetali, non della normale flora. La maggior parte di esse sono mobili e alcune sono carnivore. L'esempio più classico di piante sono i Cumulo Strisciante e gli Uomini Albero. Anche le creature fungoidi e i miconidi rientrano in questa categoria.

\begin{center}
\includegraphics[width=0.7\linewidth]{immagini/sanmichelesatana.png}\\
\emph{San Michele sconfigge Satana. Raffaello ed aiuti (1518). Museo del Louvre}
\end{center}

\smallskip\textbf{Umanoidi}, sono la popolazione principale dei mondi di gioco, civilizzati e selvaggi, comprendono gli umani e un'ampia gamma di altre specie. Possiedono una lingua e una cultura, poche o nessuna abilità magica innata (sebbene molti umanoidi possano apprendere gli incantesimi), ed una forma bipede. Le razze più comuni di umanoide sono quelle più adatte come personaggi del giocatore: umani, nani, elfi e nibali, diversi. Quasi altrettanto numerose, ma più brutali e selvagge, e quasi tutte malvagie, sono le razze goblinoidi (goblin, hobgoblin e bugbear), orchi, gnoll, lucertoloidi e coboldi.

\medskip

Queste categorie possono essere a loro volta raggruppate in tipologie di Creature:
\smallskip
\begin{itemize}[leftmargin=*] \setlength{\itemsep}{0pt}
\item
Le \textbf{Creature Naturali}: sono Insetti, Rettili, Bestie, Umanoidi, Piante, Creature acquatiche, Mostrusità, Melme
\item
Le \textbf{Creature Magiche} sono: Immondi, Demoni, Diavoli, Fatati, Spiriti, Non morti, Giganti, Celestiali, Costrutti, Aberrazioni (tutto ciò che è alieno o innaturale) e Draghi.

Se una Creatura Naturale ha poteri magici allora si considera anche come Creatura Magica.
\end{itemize}

\medskip\textbf{Etichette}

Un mostro può presentare una o più etichette indicate tra parentesi, a seguire il suo tipo. Ad esempio un orco ha il tipo \emph{umanoide (orco)}. Le etichette tra parentesi forniscono ulteriori categorizzazioni per determinate creature. Le etichette non hanno delle proprie regole specifiche, ma alcuni elementi del gioco, come gli oggetti magici, vi possono fare riferimento. Ad esempio, una lancia particolarmente efficace contro i demoni, funzionerebbe contro qualsiasi mostro che abbia l'etichetta demone.

\subsection{Tratti}

I mostri non presentano l'elenco dettagliato dei Tratti, troverete solo l'indicazione sugli assi del Caos, Legge, Bene e Male. Ricordatevi che sono indicazioni, le eccezioni possono capitare specialmente nelle specie più intelligenti.
Determinate creature sono \textbf{disallineate}, ovvero non hanno una condotta morale o etica.

\subsection{Difesa}

Un mostro che indossa un'armatura o trasporta uno scudo ha una Difesa che tiene conto dell'armatura, dello scudo e della Destrezza. Altrimenti, la Difesa di un mostro è basata sul suo valore di Destrezza e l'armatura naturale se la possiede (la "\emph{pellaccia}"). Se un mostro possiede un'armatura naturale, indossa armature o trasporta uno scudo, viene indicato tra parentesi dopo il valore della sua Difesa.

Qualora il mostro fosse \textbf{colto di sorpresa} sottraete alla Difesa -4.

\subsection{Punti Ferita}

Di solito quando scende a 0 Punti Ferita, un mostro muore o viene considerato morto.

I Punti Ferita di un mostro sono presentati con il suo valore.

Anche il valore di Costituzione di un mostro influenza il numero di Punti Ferita che possiede. Il suo valore di Costituzione viene moltiplicato per il Grado di Sfida che possiede e il risultato viene sommato ai suoi Punti Ferita. Ad esempio, un mostro che ha Costituzione 1 e Grado di Sfida 2 avrà, \emph{mediamente} (GS+1)*15+GS*Cos = 47 Punti Ferita.

Capiterà che i giocatori vi chiedano \textbf{\emph{come sta il mostro}}, vi suggerisco di non scendere mai nei dettagli dicendo quanti Punti Ferita ha in tutto o ne ha persi, bensì rimanere in questi gradi: Non ferito (Punti Ferita pieni), Ferito (30\% Punti Ferita subiti), Gravemente ferito (almeno 50\% Punti Ferita subiti), ovvero dare una descrizione generica dello stato. \index{Come sta il mostro}\index{Chiedere Punti Ferita del Mostro}

\subsubsection{Arrabbiato}\index{Arrabbiato}\index{Bloodied}\label{mostroarrabbiato}

A discrezione del Narratore una creatura che abbia perso almeno i 50\% dei Punti Ferita totali innesca una furia che gli permette azioni particolari.
I mostri con Grado di Sfida 5 o più possono avere una scheda \textbf{Arrabbiato}. L'abilità Arrabbiato si può usare una volta per scontro al costo, se non segnato diversamente, di 1 Azione.

Creature particolarmente feroci e potenti potrebbero avere più note di Arrabbiato ed entrambe, rispettando le eventuali condizioni segnate, sono attivabili.

%\subsubsection{Opzionale - Tutti Arrabbiati}\index{Opzionale - Tutti Arrabbiati}
%Per una maggiore aggressività potete fare che il mostro quando scende sotto la metà dei Punti Ferita prende +1d6 al Tiro per Colpire oppure al Danno oppure ai Tiri Salvezza a seconda del tipo di creatura.\index{Sanguinante}\index{Bloodied} \index{Arrabbiato}

\medskip

Potete anche decidere che la creatura annulla una condizione che ha su di se.

\subsection{Movimento}

Il Movimento di un mostro ti dice di quanto si possa muovere durante il suo round per Azione di Movimento

Tutte le creature possiedono un movimento di passeggio, detto semplicemente movimento del mostro. Le creature che non possiedono alcuna forma di spostamento terreno hanno velocità di movimento 0 metri.

Alcune creature possiedono uno o più dei seguenti modi di movimento aggiuntivi.

%\begin{center}
%\includegraphics[width=0.65\linewidth]{immagini/roc.png}\\
%\emph{Henry Justice Ford}
%\end{center}

\smallskip\textbf{Nuoto}

Un mostro che possieda una velocità di nuoto non deve spendere movimento extra per nuotare (non è terreno difficile)

\smallskip\textbf{Scalata}

Un mostro che possieda una velocità di scalata può usare tutto o solo parte del suo movimento per muoversi su superfici verticali. Il mostro non deve spendere movimento extra (x4) per scalare.

\smallskip\textbf{Scavo}

Un mostro che possieda una velocità di scavo può usare la sua velocità per attraversare sabbia, terra, fango, ecc. Un mostro non può scavare attraverso la roccia solida a meno che non possieda un tratto speciale che glielo permetta.

\smallskip\textbf{Volo}

Un mostro che possieda una velocità di volo può usare tutto o solo parte del suo movimento per volare. Alcuni mostri hanno l'abilità di \hyperlink{Fluttuare}{\textbf{Fluttuare}} (pag. \pageref{Fluttuare}), che li rende difficili da abbattere. Il mostro smette di fluttuare quando muore.

\subsection{Punteggi di Caratteristica}

Ogni mostro possiede sei punteggi di caratteristica (Forza, Destrezza, Costituzione, Intelligenza, Saggezza, Carisma)

\subsection{Competenze}\index{Competenza Armi dei mostri}

La voce Competenze è riservata a quei mostri che sono capaci in una o più competenze peculiari diverse da quelle che normalmente userebbe per vivere. Ad esempio, un mostro che è molto attento e furtivo potrebbe avere bonus alle prove di Consapevolezza e Destrezza.

Si possono applicare anche altri modificatori, ad esempio, un mostro potrebbe avere un bonus maggiore del previsto per tenere conto della sua grande perizia.

Se non indicata, ma necessaria per le prove (non al Tiro per Colpire, dove si usa il valore già segnato) la Competenza Armi di un Mostro è pari al suo Grado di Sfida.

\medskip

\textbf{Tabella: Equivalenza Armi Magiche}\index[Tabelle]{Tabella Equivalenza Armi Magiche}\label{equivalenzaarmimagiche}\hypertarget{equivalenzaarmimagiche}{}

\medskip

\noindent\begin{tabularx}{\linewidth}{lXXX}
		\toprule
 \rowcolor{gray!20}\textbf{Immunità} & \textbf{Magia Arma} & \textbf{Attacco Naturale}& \textbf{Pugno Vuoto}\\
	\toprule
	+1 & +1 & 3& 2\\
 \rowcolor{gray!20}+2 & +2 & 6& 4\\
	Ferro Freddo & +1 & 4& 2\\
 \rowcolor{gray!20}Argento & +1 & 4& 2\\
	Adamantio & +2 & 6& 4\\
 \rowcolor{gray!20}+3 & +3 & 12& 8\\
	+4 & +4 & 16& 12\\
 \rowcolor{gray!20}+5 & +5 & - & 16
\end{tabularx}

\subsection{Vulnerabilità, Resistenze e Immunità}\index{Equivalenze armi}\index{Pugni magici}\label{vulnerabilitaresistenze}
Alcune creature possiedono vulnerabilità, resistenze o immunità ad un certo tipo di danno. Creature particolari sono addirittura resistenti o immuni agli attacchi non magici (un attacco magico è un attacco sferrato tramite un incantesimo, un oggetto magico o arma, o un'altra fonte di magia).

Quando è indicata una immunità alle armi magiche (es. +1 oppure +2) significa che bisogna usare un arma con un incantamento maggiore per poter danneggiare la creatura. In caso di creature immune ai critici questo vale sia per incantesimi che per armi, rimane efficace l'esplosione del danno. \index{Critico sui mostri}.

Una creatura immune alle armi non magiche o +1 ma vulnerabile al ferro freddo o all'argento applica prima le sue immunità poi se passate applica le vulnerabilità all'attacco subito, e quindi un arma d'argento non farà danno, ma se d'argento +1 farà il doppio del danno.

Inoltre, certe creature sono immuni a determinate condizioni. Se un mostro è immune ad un effetto di gioco che non viene considerato danno o condizione, possiede invece un tratto speciale.

Nella tabella sottostante viene indicato quale incantamento magico dell'arma è necessario per superare l'immunità indicata. E' anche indicato il punteggio minimo di Competenza Armi nel caso si colpisca con calci e pugni.

In caso di personaggio con Lista d'Armi \textbf{Pugno Vuoto} si controlla quanto volte si è presa la lista.

\subsection{Consapevolezza}

Tutti i mostri, quando non segnato, hanno un valore di Consapevolezza pari a \textbf{Grado di Sfida/2 + Saggezza}.

\subsection{Sensi}

La voce Sensi elenca qualsiasi senso speciale di cui il mostro sia in possesso. I sensi speciali sono descritti di seguito. Se non è presente la voce Sensi, la creatura ha dei sensi standard (visione, olfatto, gusto, tatto...) non particolarmente evoluti.

\begin{center}
	\includegraphics[width=0.7\linewidth]{immagini/ciclope.png}

	\emph{Henry Justice Ford}
\end{center}

\subsubsection{Percezione Tellurica}

Un mostro con percezione tellurica può individuare e trovare le origini delle vibrazioni entro uno specifico raggio, purché il mostro e la fonte della vibrazione siano in contatto con lo stesso terreno o sostanza. La percezione tellurica non può essere impiegata per individuare creature volanti o incorporee. Molte creature scavatrici, come gli ankheg e i colossi di terra, possiedono questo senso speciale.

\subsubsection{Visione Crepuscolare o Scurovisione}

Una creatura con Visione Crepuscolare può vedere nella più tenue delle luci, ma non nell'oscurità completa a differenza di quelle che possiedono scurovisione. Molte creature che vivono sottoterra possiedono questo senso speciale. Vedi capitolo \hyperlink{visioneeluce}{Caratteristiche Speciali}.

\subsubsection{Visione del Vero}

Un mostro con la visione del vero può, fino ad una specifica gittata, vedere attraverso l'oscurità normale e magica, vedere creature e oggetti invisibili, automaticamente individuare le illusioni e riuscire i Tiri Salvezza contro di loro, percepire la forma originale di un mutaforma o di una creatura trasformata dalla magia. Inoltre, la creatura può vedere nel Piano Etereo fino alla stessa gittata.

\subsubsection{Vista Cieca}

Una creatura con vista cieca può percepire l'ambiente circostante, senza fare affidamento alla vista, fino ad una specifica gittata.

Le creature senza occhi come i grimlock e le melme e le creature con ecolocazione o sensi potenziati, come i pipistrelli ed i draghi, possiedono questo senso.

Se un mostro è cieco di natura, la cosa viene annotata tra parentesi, in questo caso la portata della sua vista cieca definisce anche la portata massima della sua percezione.

\subsection{Linguaggi}

Le lingue che un mostro può parlare sono riportate in ordine alfabetico. Se un mostro può capire una lingua ma non parlarla la cosa viene indicata in questa voce. Se un mostro non ha la nota \emph{Linguaggi} significa che non conosce linguaggi diversi dalla propria lingua (se applicabile).

\subsection{Telepatia}

La telepatia è un'abilità che permette ad un mostro di comunicare mentalmente con un'altra creatura nel raggio di azione specificato. La creatura contattata non è necessario che parli la stessa lingua del mostro per comunicare in questo modo. Una creatura senza telepatia può ricevere e rispondere a messaggi telepatici ma non può iniziare o terminare una conversazione telepatica.

Un mostro telepatico non ha bisogno di vedere la creatura contattata e può terminare il contatto telepatico in qualsiasi momento. Il contatto è infranto non appena le due creature non si trovano più entro il raggio di azione o se il mostro telepatico contatta un'altra creatura a gittata. Un mostro telepatico può iniziare o terminare una conversazione telepatica senza dover usare un'azione, ma mentre il mostro è inabile non può dare inizio ad un contatto telepatico, e qualsiasi contatto in corso viene terminato. Per avviare una comunicazione telepatica l'obiettivo deve essere stato almeno individuato.

Una creatura nell'area di un \emph{campo anti-magia} o in qualsiasi altro posto in cui la magia non funziona può inviare o ricevere messaggi telepatici.

\subsection{Sfida}

Il \textbf{grado di sfida} (GS) di un mostro vi dice quanto sia grande la minaccia che pone. Una compagnia di quattro avventurieri equipaggiata in maniera appropriata e riposata deve essere in grado di sconfiggere un mostro dal grado di sfida pari al proprio livello medio senza subire perdite. Ad esempio, una compagnia di quattro personaggi di 3° livello dovrebbe ritenere un mostro di grado di sfida 3 una sfida normale e non pericolosa.

I mostri che sono significativamente più deboli dei personaggi di 1° livello hanno un grado di sfida inferiore ad 1. I mostri con un grado di sfida 0 non presentano problemi eccetto in grandi numeri, quelli privi di reali attacchi non valgono punti esperienza.

\subsection{Riconoscere i Mostri}\label{riconoscereimostri}\hypertarget{riconoscereimostri}{} \index{Riconoscere i Mostri}

Sapere riconoscere un mostro può essere estremamente utile ed è qualcosa che non andrebbe mai sottovalutato.

Per \textbf{riconoscere un mostro} si effettua una prova di Conoscenza. (\textbf{1 Azione}) su:

\medskip

\noindent\begin{itemize}[leftmargin=*] \setlength{\itemsep}{0pt}
\item \textbf{Arcana}: Giganti, Costrutti, Spiriti, Mostruosità, Aberrazioni, Draghi
\item \textbf{Piani}: Elementali, Giganti
\item \textbf{Occulto}: Immondi (Diavoli e Demoni), Spiriti, Non Morti
\item \textbf{Religione}: Spiriti, Non Morti, Celestiali
\item \textbf{Dungeon}: Aberrazioni, Mostruosità, Melme e creature sotterranee
\item \textbf{Natura}: Bestie, Piante, Fatati
\end{itemize}

\medskip

La DC delle prova è pari \textbf{10 + grado di Sfida} della creatura + \textbf{fattore di rarità}/notorietà (comune (0), non comune (+1), raro (+2), molto raro (+4), leggendario +(10)).

Le informazioni ottenibili dipendono dal margine di successo ottenuto.

\noindent\begin{itemize} \setlength{\itemsep}{0pt}
\item \textbf{entro 2}: nome, tipo, la Caratteristica principale
\item \textbf{fino 7}: quale è il migliore Tiro Salvezza, una immunità a Condizioni, una vulnerabilità a Danni, attacco tipico
\item \textbf{fino 12}: quale è il peggiore Tiro Salvezza, una immunità a Condizioni, una immunità a Danni, una vulnerabilità a Condizioni, una vulnerabilità a tipo di Danno
\item \textbf{fino 15}: due immunità a Condizioni, una immunità a Danni, una vulnerabilità a Condizioni, una vulnerabilità a tipo di Danno
\item \textbf{fino 17}: grado di sfida relativo ovvero se è una scontro facile, medio, alto, straordinario, mortale o epico
\item \textbf{oltre 17}: attacco e difese speciali
\end{itemize}

\medskip

Le informazioni ottenute sono cumulative, ovvero se la prova riesce di 15 ottieni le informazioni entro 2, 7 e 12.

\subsection{Tratti Speciali}

I tratti speciali (che compaiono dopo il grado di sfida di un mostro ma prima di qualsiasi Azione o reazione) sono peculiarità che avranno probabilmente un ruolo in un incontro di combattimento e che richiedono delle spiegazioni.

\subsection{Incantesimi}

Un mostro con il privilegio Incantesimi o Incantesimi Innati è in grado di lanciare Incantesimi.

La \textbf{DC è 12 + livello incantesimo x2 + Intelligenza o Saggezza a seconda della caratteristica migliore oppure indicata}. Un mostro non necessita di eseguire Prove di Magia ma può farle se ha un valore di Competenza Magica (es. Lich, Mummia, Naga...).

Il Tiro per Colpire con Incantesimi è pari al valore di Competenza Magica se segnata, se non è segnata è  pari a metà del GS + Intelligenza o modificatore di caratteristica indicato. Se è necessario calcolare la Competenza Armi per l'uso di incantesimi e questa non è specificata, allora è pari alla metà del punteggio di Competenza Magica.

Un mostro con incantesimi può lanciare un numero di incantesimi segnati di quel livello pari a \emph{slot}, scegliendoli tra quelli indicati.

%\begin{center}
%	\includegraphics[width=0.65\linewidth]{immagini/lich2.png}

%	\emph{Lich - Battle of Wesnoth}

%\end{center}

\subsection{Azioni}

Anche i mostri agiscono secondo lo schema delle 3 Azioni disponibili per round. Possono essere segnate abilità e capacità che gli permettono di eseguire un numero più elevato di Azioni.

Quando un mostro svolge le sue azioni, può scegliere tra le opzioni della sezione Azioni del suo blocco statistiche o impiegare una delle Azioni disponibili a tutte le creature, come Scattare, Nascondersi, \hyperlink{preparareladifesa}{Preparare la difesa}, se non indicato diversamente usare una Azione (non facente parte del Multiattacco o segnata come \emph{Attacco}) costa 2 Azioni.

\subsubsection{Attacchi da Mischia e a Distanza}\index{Mostri e Attacchi}

L'azione più comune che un mostro effettuerà in combattimento sarà un attacco in mischia o a distanza. Possono essere attacchi con incantesimi o attacchi con armi, dove l'arma può essere un manufatto o un'arma naturale, come gli artigli o la coda chiodata.

\emph{\textbf{Creatura contro Bersaglio}.} Il bersaglio di un attacco da mischia o a distanza è di solito una creatura o un bersaglio.

\textbf{Portata}: la portata indicata è la distanza \textbf{entro} quanti metri la creatura può colpire l'avversario. Una creatura con portata 0 deve esserti addosso per colpirti, solitamente hanno portata 0 le creature estremamente piccole.

\emph{\textbf{Colpisce.}} Qualsiasi danno inflitto o altro effetto che avviene come risultato di un attacco che colpisce il bersaglio viene descritto nell'annotazione \emph{Colpisce}. Puoi scegliere se prendere il danno medio o tirare i dadi; per questo motivo vengono presentati sia il danno medio che una formula di dadi.

Anche nel Tiro per Colpire per i Mostri valgono le Golden Rules.

Il Tiro per Colpire del mostro \textbf{non ha applica danno critico ne esplosione del danno}, ma \textbf{non subisce penalità per il multiattacco}. Ogni attacco del mostro, quindi anche 3 attacchi a round, viene effettuato con il Tiro per Colpire senza penalità del Multiattacco.\index{Attacco dei mostri}

\textbf{\emph{Manca}.} Se un attacco ha un effetto prodotto da un colpo a vuoto, quell'informazione viene fornita dall'annotazione \emph{Manca}.

\emph{\textbf{Danni.}} Se un mostro impugna armi manufatte, infligge danni appropriati all'arma. I mostri più grossi di solito impugnano armi di dimensioni superiori che infliggono danni extra quando colpiscono. Se usano questo tipo di armi il danno è già segnato, altrimenti se raccolgono o usano un arma non prevista raddoppiare i dadi dell'arma se la creatura è Grande, triplicarli se Enorme e quadruplicarli se Mastodontica qualora usino armi della loro taglia.

%\begin{narratore}
%Una creatura che abbia almeno Grado di Sfida 6 a discrezione del Narratore può sferrare un attacco di Opportunità al costo di una Reazione (vedi \hyperlink{opportunista}{Opportunista} pag. \pageref{opportunista}).

%\medskip
%Per valorizzare i mostri e renderli più incisivi potete decidere che ogni mostro abbia una \hyperlink{riduzionedeldanno}{Riduzione del Danno} pari a metà del suo Grado di Sfida ( $\frac{GS}{2}$/- )

%\end{narratore}\end{changemargin}

\subsubsection{Multiattacco e Attacco}\index{Multiattacco dei mostri}

L'\textbf{Azione Multiattacco consuma 2 Azioni} anche se porta più di 2 attacchi.

Come per l'Attacco normale i mostri non cumulano le penalità del Multiattacco e non hanno il Tiro Critico o l'Esplosione del Danno. Ogni attacco è portato con il bonus al Tiro per Colpire segnato.

Altre \emph{Azioni di Attacco} elencate sotto Multiattacco ma non facente parte di quelle elencate nella descrizione del Multiattacco costano 1 Azione se non descritto diversamente e non cumulano le penalità del Multiattacco.\index{Multiattacco dei Mostri}

Ad esempio il lancio del Sasso per il \hyperlink{Gigante delle Colline}{Gigante delle Colline} costa 1 Azione perché e' un \emph{Attacco} e non fa parte delle Azioni elencate nel Multiattacco.

Il Soffio infuocato di una \hyperlink{Chimera}{Chimera} non ha l'attributo \emph{Attacco} e quindi costa 2 Azioni.

Usare \textbf{1 solo Attacco} della catena del Multiattacco \textbf{costa 1 Azione}.

\subsubsection{Regole dell'Afferrare per i Mostri}

Molti mostri possiedono un attacco speciale che gli permette di afferrare rapidamente la preda. Quando un mostro colpisce con un simile attacco, non deve effettuare un'ulteriore prova per determinare se l'afferrare riesce a meno che l'attacco non dica altrimenti.

Una creatura afferrata dal mostro segue le indicazioni di \hyperlink{afferrareunavversario}{Afferrare un avversario} (pag. \pageref{afferrareunavversario}).

Se non viene fornita una \textbf{DC di fuga} assumere che sia ia uguale a 10 + (Tiro Salvezza su Tempra  + Forza) del mostro +1d6 per Taglia di differenza.

\subsubsection{Munizioni}

Un mostro porta con sé munizioni sufficienti per effettuare i suoi attacchi a distanza. Puoi presumere che un mostro abbia 2d4 proiettili per un attacco con armi da lancio (giavellotti, macigni..), e 2d10 proiettili per un'arma a proiettili come un arco o una balestra.

\begin{center}
	\includegraphics[width=0.65\linewidth]{immagini/polpo.png}

	\emph{Alphonse de Neuville - Hetzel edition of 20000 Lieues Sous les Mers}
\end{center}

\subsubsection{Reazioni}

Se un mostro può compiere qualcosa di speciale con le sue reazioni, è riportato qui. Se una creatura non ha reazioni speciali, questa sezione è assente.

\subsubsection{Uso Limitato}

Alcune abilità speciali hanno restrizioni sul numero di volte che possono essere usate.

\textbf{\emph{X/Giorno}.} L'annotazione "X/Giorno" indica un'abilità speciale che può essere usata X volte prima che il sorga l'alba per recuperare gli usi consumati. Ad esempio, \emph{1/Giorno} indica un'abilità speciale che può essere usata una volta prima che il mostro debba aspettare la nuova alba.

\emph{\textbf{Ricarica X-Y.}} L'annotazione "Ricarica X-Y" indica che il mostro può usare un'abilità speciale una volta e che l'abilità ha una probabilità casuale di ricaricarsi ogni round seguente di combattimento. All'inizio di ciascun round del mostro, tira un d6. Se il risultato è uno dei numeri dell'annotazione di ricarica, il mostro recupera l'uso dell'abilità speciale. L'abilità si ricarica anche all'alba di un nuovo giorno.

Ad esempio, \emph{Ricarica 5-6} indica che un mostro può usare la sua abilità speciale una volta. Poi, all'inizio del round del mostro, recupera l'uso dell'abilità se tira 5 o 6 su di un d6.

\subsection{Azioni Aggiuntive}

Certe creature possono possono eseguire azioni speciali al di fuori del proprio round, ed alcune possono estendere il proprio potere all'ambiente, provocando la manifestazione di effetti magici straordinari nelle loro vicinanze.

Una creatura con azioni aggiuntive può effettuare un certo numero di azioni speciali, dette \emph{azioni aggiuntive}, al di fuori del suo round. Solo un'azione aggiuntiva può essere usata alla volta e solo al termine del round di un'altra creatura. Non costa Azioni o Reazioni usare una Azione Aggiuntiva. Una creatura con azioni aggiuntive recupera all'inizio del suo round le azioni aggiuntive che ha usato. Non è obbligata ad usare le sue azioni aggiuntive e non può usare le azioni aggiuntive mentre è inabile o altrimenti incapace di effettuare Reazioni. Se sorpresa, non può usarle fin dopo il suo primo round di combattimento.

Se una creatura assume la forma di una creatura con azioni aggiuntive, magari tramite un incantesimo, non ne ottiene però le azioni aggiuntive o le azioni da tana.

\subsubsection{La Tana di una Creatura}\index{Tana di una Creatura}\index{Azioni da tana}

Una creatura con Azioni aggiuntive può presentare una sezione che ne descrive la tana e gli effetti speciali che vi può creare mentre si trova lì, o per propria volontà o semplicemente grazie alla sua presenza. Questa sezione si applica solo alle creature leggendarie che trascorrono molto tempo nelle loro tane dove è altamente probabile che li vengano incontrate.

Se una creatura con azioni aggiuntive ha un' \textbf{Azione da tana}, può usarla per imbrigliare la magia ambientale della sua tana. Al conteggio di iniziativa 10, perdendo i pareggi, la creatura può usare una delle sue opzioni di azioni da tana. Non può farlo mentre è inabile o altrimenti incapace di effettuare azioni. Se sorpresa, non può farne uso fino a dopo il suo primo round di combattimento.

%\medskip

%\begin{center}
%\includegraphics[width=0.7\linewidth]{immagini/cupido.png}

%\emph{Eros con il suo arco. Musei Capitolini}
%\end{center}

\subsection{Equipaggiamento}

Il blocco statistiche si riferisce all'equipaggiamento, oltre le armi o le armature utilizzate dal mostro. Una creatura che normalmente indossa abiti, come un umanoide, si assume sia vestito in maniera appropriata.

Puoi equipaggiare i mostri con ulteriore equipaggiamento come preferisci, utilizzando il capitolo \hyperlink{equipaggiamento}{Equipaggiamento} come fonte di ispirazione, sei tu a decidere quanto dell'equipaggiamento del mostro è recuperabile dopo che la creatura è stata uccisa o se qualsiasi parte del suo equipaggiamento sia ancora utilizzabile. Ad esempio, un'armatura ammaccata fatta per un mostro difficilmente sarà utilizzabile da qualcun altro. Se un mostro incantatore necessita di componenti materiali per lanciare i suoi incantesimi, dai per scontato che abbia le componenti materiali per lanciare gli incantesimi indicati nella sua scheda.

\subsection{Punti Esperienza per GS}

Ogni mostro se \emph{sconfitto} concede un certo ammontare di Punti Esperienza da suddividere tra tutti i partecipanti allo scontro.

Questa tabella indica per GS i Punti Esperienza relativi.

\medskip

\textbf{Tabella: Grado di Sfida e Punti Esperienza}\index[Tabelle]{Tabella Punti Esperienza per Grado di Sfida}

\medskip

\noindent\begin{tabularx}{\linewidth}{Xl|Xl|Xl}
	\toprule
\rowcolor{gray!20}\textbf{GS} & \textbf{PX} &\textbf{GS} & \textbf{PX} &\textbf{GS} & \textbf{PX}\\
\toprule
0& 10 &9& 5000& 21&33000\\
\rowcolor{gray!20}1/8& 25 &10& 5900&22&41000\\
1/4& 50 &11& 7200&23&50000\\
\rowcolor{gray!20}1/2& 100 &12& 8400&24&62000\\
1& 200 &13& 10000&25&75000\\
\rowcolor{gray!20}2& 450&14& 11500&26&90000\\
3& 700&15& 13000&27&105000\\
\rowcolor{gray!20}4& 1100&16& 15000&28&120000\\
5& 1800&17& 18000&29&135000\\
\rowcolor{gray!20}6& 2300&18& 20000&30&155000\\
7& 2900&19& 22000&&\\
\rowcolor{gray!20}8& 3900&20& 25000&&
\end{tabularx}

\subsection{Tipologie di Tesoro}

Ogni tipologia di creatura può preferire un tipo di tesoro (inteso come oggetti, monete, gemme...) diverso. Questi sono solo suggerimenti su come costruire il tesoro del mostro.

Vedi anche \hyperlink{valoretesoroincontro}{Tabella: Valori del Tesoro per Incontro} (pag. \pageref{valoretesoroincontro}).

\medskip

\begin{itemize}[leftmargin=*] \setlength{\itemsep}{0pt}

	\item \textbf{Aberrazione}
	Molte aberrazioni hanno scarsa considerazione per i tesori, possedendo solo quel che prendono dai resti delle loro vittime. Alcune sono nemici intelligenti che usano vari oggetti magici e tesori per potenziare le loro capacità.

	\item \textbf{Animale - Bestia Magica} Gli animali non prestano alcuna attenzione ai tesori, lasciando monete e oggetti tra i resti dei loro pasti. Per quelli che possiedono un tesoro, questo si trova solitamente nelle loro tane, sparso tra ossa e altri rifiuti.

	\item \textbf{Costrutto}
	Spesso è il costrutto stesso il tesoro più di valore. I costrutti sono solitamente usati per sorvegliare tesori o oggetti magici di grande valore.

	\item \textbf{Drago}
	Noti per i loro preziosi tesori, i draghi spesso riposano su pile di monete, gemme, oggetti magici e costosi.

	\item \textbf{Esterno}
	Gli esterni sono tra le creature più diverse e, di conseguenza, possono possedere qualsiasi tipo di tesoro, sia su di loro che nascosto nei loro rifugi. Il Narratore dovrebbe valutare ogni singola creatura per determinare il tipo di tesoro più adatto a ciascun esterno.

	\item \textbf{Folletto}
	I folletti danno valore agli oggetti belli e magici. Hanno scarsa considerazione per monete e merci.

	\item \textbf{Melma - Parassita - Vegetale}
	Le melme non sanno cosa è un tesoro e lasciano dove trovano tutto ciò che non è digeribile. Qualsiasi tesoro possano trasportare è completamente accidentale.

	\item \textbf{Non Morto}
	I tesori posseduti dai non morti dipendono dalla loro intelligenza. I non morti senza intelletto di solito hanno solo i pochi beni che avevano in vita, raramente utili come tesori. Al contrario, i non morti intelligenti utilizzano una varietà di oggetti magici per annientare i viventi.

	\item \textbf{Umanoide}
	Queste creature sono molto diverse tra loro, ma anche gli umanoidi più primitivi utilizzano equipaggiamenti e oggetti magici in qualche misura. In gruppi più grandi, come le comunità, gli umanoidi spesso possiedono una notevole quantità di tesori che custodiscono collettivamente.

\end{itemize}

\subsubsection{Opzionale - Esperienza per Sfida}\index{Opzionale - Esperienza per Sfida}

Con questo sistema i Punti Esperienza sono dati in base alla difficoltà relativa della Sfida dato il livello dei personaggi. Uno scontro con 5 Troll non darà (1800 x 5) Punti Esperienza, ma a seconda della sfida relativa concederà un ammontare diverso.

Il gruppo di Troll (Sfida 5, 1800 PX) non da sempre 1800 PX a troll sconfitto; se viene affrontato da un gruppo di basso livello, ovvero per una sfida di difficoltà Straordinaria, ne darà di più mentre affrontato da un gruppo di alto livello, dove 5 troll sono una sfida Alta, ne darà di meno.

Con questo sistema ogni 1000 Punti Esperienza si passa di livello. Valgono tutte le considerazione del capitolo Masterizzare per preparare gli scontri.

\medskip

\textbf{Tabella: Punti Esperienza per Grado di Sfida}\index[Tabelle]{Tabella Punti Esperienza per Grado di Sfida}

\medskip

\noindent\begin{tabularx}{\linewidth}{ll|ll}
\rowcolor{gray!20}\textbf{Grado di Sfida} & \textbf{PX}&\textbf{Grado di Sfida} & \textbf{PX}\\
\hline
Facile& 20& Media& 30\\
\rowcolor{gray!20}Alta& 50& Straordinaria& 80\\
Mortale& 120& Epica& 160
\end{tabularx}

\medskip

Anche per trappole o sfide superate si usa questo sistema per calcolare i PX guadagnati. I Punti Esperienza premio per ogni obiettivo personale o di gruppo raggiunto sono 10.

\subsubsection{Opzionale - Metodo alternativo per costruire gli Incontri}\index{Opzionale - Sistema alternativo per costruire gli Incontri}

\begin{enumerate}[leftmargin=*] \setlength{\itemsep}{0pt}

\item \textbf{Definire l'APL (Average Party Level):} Calcolare il livello medio del gruppo. Sommare i livelli di tutti i personaggi e dividere per il numero di personaggi, come già spiegato.

%Questa guida presuppone un gruppo di quattro o cinque personaggi. Se il vostro gruppo ha sei o più giocatori, aggiungete uno al loro livello medio. Se il vostro gruppo contiene tre o meno giocatori, sottraete uno dal loro livello medio. Per esempio, se il vostro gruppo consiste di sei giocatori, due di 5° livello e quattro di 7° livello, il APL è il 7° (38 livelli totali, diviso per sei giocatori, arrotondando all'intero più vicino, ed aggiungendo uno al risultato finale).

\item \textbf{Stabilire la Difficoltà Desiderata:} Decidere quale livello di sfida si vuole presentare al gruppo.

%\begin{itemize}
%\setlength{\itemsep}{-1pt} % Set item separation to zero
%\setlength{\itemindent}{-1pt}
%\item \makebox[2.5cm][l]{Facile:} 75\% - 125\% \\
%\item \makebox[2.5cm][l]{Media:} 126\% - 175\% \\
%\item \makebox[2.5cm][l]{Impegnativa:} 176\% - 225\% \\
%\end{itemize}

\medskip

\noindent\begin{tabularx}{\linewidth}{lX}
\rowcolor{gray!20}\textbf{Difficoltà} & \textbf{Peso \%}\\
	Facile 		& 75\% - 105\%\\
 \rowcolor{gray!20}Media		& 106\% - 145\%\\
	Impegnativa	& 146\% - 195\%\\
 \rowcolor{gray!20}Alta		& 196\% - 255\%\\
Straordinaria	& 256\% - 325\%\\
 \rowcolor{gray!20}Mortale		& 326\% - 405\%\\
	Epica		& 406\% e oltre\\
\end{tabularx}

%\begin{itemize}[leftmargin=*]
%\setlength{\itemsep}{0pt}    % Riduce distanza tra gli item
%\setlength{\parsep}{0pt}     % Riduce spazio extra nei paragrafi interni
%\setlength{\topsep}{0pt}     % Riduce spazio prima della lista
%\setlength{\partopsep}{0pt}  % Elimina spazio extra sopra la lista
%\item \makebox[2.5cm][l]{Facile:} 75\% - 105\%
%\item \makebox[2.5cm][l]{Media:} 106\% - 145\%
%\item \makebox[2.5cm][l]{Impegnativa:} 146\% - 195\%
%\item \makebox[2.5cm][l]{Alta:} 196\% - 255\%
%\item \makebox[2.5cm][l]{Straordinaria:} 256\% - 325\%
%\item \makebox[2.5cm][l]{Mortale:} 326\% - 405\%
%\item \makebox[2.5cm][l]{Epica:} 406\% e oltre
%\end{itemize}

%\begin{tabular}{@{}ll@{}} % @{} removes extra padding
%Facile: & 75\% - 105\% \\
%Media: & 106\% - 145\% \\
%Impegnativa: & 146\% - 195\% \\
%Alta: & 196\% - 255\% \\
%Straordinaria: & 256\% - 325\% \\
%Mortale: & 326\% - 405\% \\
%Epica: & 406\%+
%\end{tabular}

\item \textbf{Assegnare un valore Percentuale ai Mostri:} Utilizzare la tabella seguente per determinare il \emph{peso} (percentuale) di ciascun Avversario (\emph{Avv.}) in base alla differenza tra il suo Grado di Sfida (GS) e all'APL del gruppo (\emph{Rapp.})

\medskip

\noindent\begin{tabularx}{\linewidth}{c|c|c|c}
	\toprule
\rowcolor{gray!20}\textbf{Rapp.} & \textbf{\% per Avv.} &\textbf{Rapp.} & \textbf{\% per Avv.}\\
\toprule
-6 & 3\% &  0 & 70\% \\
\rowcolor{gray!20}-5 & 5\% & +1 & 105\% \\
-4 & 10\% & +2 & 160\% \\
\rowcolor{gray!20}-3 & 15\% & +3 & 240\% \\
-2 & 25\% & +4 & 360\% \\
\rowcolor{gray!20}-1 & 45\% & +5 & 480\% \\
\end{tabularx}

\medskip

Partite dai mostri con Grado di Sfida più alto e poi aggiungere mostri con GS più basso per raggiungere la percentuale desiderata.

\end{enumerate}

%\begin{enfasi}{
%Io sono il mostro che gli uomini che respirano bramerebbero uccidere. Io sono Dracula. (Dracula, Bram Stoker)}\end{enfasi}\end{changemargin}

\subsection{Opzionale - Cosa farci con i mostri}\index{In cucina con i mostri}

Questa sezione vuole essere un divertito omaggio a certi tipi di avventure e anche una occasione di ingegno e divertimento. Il suggerimento principale e' di evitare descrizione sanguinolente, macabre o \emph{schifose}, mantenete sempre un velo e cercate di rimanere sullo scherzoso. I tempi di cottura possono essere nell'ordine di 1d4 ore per 4/5 porzioni.

\smallskip

\textbf{Tabella: 1d4-12 Parti Anatomiche dei Mostri}\index[Tabelle]{Tabella 1d4-12 Parti Anatomiche dei Mostri}

\smallskip

\noindent\begin{tabularx}{\linewidth}{lX}
	\toprule
 \rowcolor{gray!20}\textbf{d4-12}&\textbf{Parte}\\
	\toprule
	11 & Sacco digestivo \\
 \rowcolor{gray!20}12 & Gelatina fotoreattiva \\
	13 & Bulbo traslucido \\
 \rowcolor{gray!20}14 & Enzimi coagulanti superficiali \\
	15 & Cellule mimetiche iridescenti \\
 \rowcolor{gray!20}16 & Tessuto elastico \\
	17 & Organo ottico polifasico \\
 \rowcolor{gray!20}18 & Midollo cerebrale ipertrofico \\
	19 & Tendini a conduzione potenziata \\
 \rowcolor{gray!20}110 & Sangue lattiginoso reattivo \\
	111 & Pelle dura infusa di minerali \\
 \rowcolor{gray!20}112 & Tendine robusto con fibre concentriche \\
	21 & Osso cavo \\
 \rowcolor{gray!20}22 & Ghiandola sudoripara tossica \\
	23 & Uncino osseo bifocale \\
 \rowcolor{gray!20}24 & Derma fotosensibile \\
	25 & Emolinfa profonda mutagena \\
 \rowcolor{gray!20}26 & Vertebra cava ripiena di magia \\
	27 & Squama cristallina lucida \\
 \rowcolor{gray!20}28 & Ghiandola del soffio elementale \\
	29 & Cuore di riserva \\
 \rowcolor{gray!20}210 & Muscolo a potenziamento magico \\
	211 & Nervi ottici ipersensibili \\
 \rowcolor{gray!20}212 & Polpastrelli succosi\\
	31 & Nasonte con setole ipersensoriali \\
 \rowcolor{gray!20}32 & Membrana retrattile di volo \\
	33 & Ghiandola di adrenalina magica \\
 \rowcolor{gray!20}34 & Pelle cicatrizzata a strati sovrapposti \\
	35 & Nucleo dentale rigenerativo \\
 \rowcolor{gray!20}36 & Appendice prensile multifocale \\
	37 & Unghia multi articolata \\
 \rowcolor{gray!20}38 & Pinna morbida fredda \\
	39 & Retina con riflesso interno rigido \\
 \rowcolor{gray!20}310 & Ciste magica piena \\
	311 & Sensi alternativi \\
 \rowcolor{gray!20}312 & Tessuto adiposo con riserva di magia \\
	41 & Ghiandola linfatica ipersviluppata \\
		\end{tabularx}
\noindent\begin{tabularx}{\linewidth}{lX}
 \rowcolor{gray!20}42 & Tessuto visivo focalizzato \\
	43 & Sacca di gas multipolare \\
 \rowcolor{gray!20}44 & Nervatura senso-tattile \\
	45 & Pelle ad assorbimento energetico \\
 \rowcolor{gray!20}46 & Tentacolo con ventose attive \\
	47 & Ghiandola di secrezione corrosiva \\
 \rowcolor{gray!20}48 & Midollo secondario autonomo \\
	49 & Tessuto cardiaco a ritmo alternato \\
 \rowcolor{gray!20}410 & Placche auricolari addestrate \\
	411 & Articolazione in osso flessibile \\
 \rowcolor{gray!20}412 & Polmone secondario con vesciche d’aria \\
\end{tabularx}

\medskip

\textbf{Tabella: d12-4 Effetti Magici da Ingestione o Preparazione}\index[Tabelle]{d12-4 Effetti Magici da Ingestione o Preparazione}

\medskip

\noindent\begin{tabularx}{\linewidth}{lX}
	\toprule
 \rowcolor{gray!20}\textbf{d12-4} & \textbf{Effetti}\\
	\toprule
11 & Visione notturna per 1d6 ore \\
\rowcolor{gray!20}12 & Gli animali evitano il personaggio \\
13 & +1 Difesa naturale per 1 ora \\
\rowcolor{gray!20}14 & +2 a Ingannare per 1 ora \\
21 & +1 a Forza per 1d6 Turni \\
\rowcolor{gray!20}22 & Rigenerazione (1 PF ogni 1T per 1 ora) \\
23 & +1d6 nelle prove di Salto per 2d4 ore\\
\rowcolor{gray!20}24 & +2 a Consapevolezza per 30 minuti \\
31 & Resistenza all'acido per 1d6 ore \\
\rowcolor{gray!20}32 & +1d6 a Intimidire per 1 ora \\
33 & Levitazione per 10 minuti \\
\rowcolor{gray!20}34 & +2 a Acrobazia per 1 ora \\
41 & Movimento Scalare per 10 minuti \\
\rowcolor{gray!20}42 & Mimesi: +1 a Furtività per 1d6 Turni\\
43 & L'olfatto rivela veleni/malattie per 1d6 ore \\
\rowcolor{gray!20}44 & Respira sott’acqua per 1 ora \\
51 & +1 metro al Movimento per 1d4 ore\\
\rowcolor{gray!20}52 & Linguaggio Elementale per 1 ora\\
52 & Polpastrelli: +2 a Cercare o Trappole per 1d6 ore\\
\rowcolor{gray!20}53 & Artigli: 1d6 danno per 1d6 ore\\
54 & Senza ossa: puoi restringerti di 1 taglia per 1d6 ore\\
61 & Immunità al freddo per 1d4 ore \\
\rowcolor{gray!20}62 & Forma gassosa per 1 ora\\
\end{tabularx}
\noindent\begin{tabularx}{\linewidth}{lX}
\toprule
\rowcolor{gray!20}\textbf{d12-4} & \textbf{Effetti}\\
\toprule
63 & Curato di 3d6 Punti Ferita\\
\rowcolor{gray!20}64 & Visione dell’invisibile per 1d4 Turni \\
71 & Parassita benigno assorbe veleno e muore \\
\rowcolor{gray!20}72 & Soffio acido (2d6 danni in cono, 4 metri, DC 10+LV) 1xT, 1d4 Ore\\
73 & Soffio rovente (vedi soffio acido)\\
\rowcolor{gray!20}74 & Recupera 1d6 Punti magia \\
81 & Immunità ad illusioni ottiche e abbagliamento per 1d6 ore\\
\rowcolor{gray!20}82 & Soffio acido (vedi soffio acido)\\
83 & Sacche d'aria. Non respiri per 2d4 Turni\\
\rowcolor{gray!20}84 & Il sudore respinge gli insetti per 2d8 ore \\
91 & +2 a Intrattenere (bella voce) per 1 ora \\
\rowcolor{gray!20}92 & Cieco per 1d4 ore\\
93 & Visione crepuscolare 9 metri per 1 ora \\
\rowcolor{gray!20}94 & Liquido: come un gelatina per 1d10 Turni\\
101 & Minerale: DR 4/magico a danni fisici, per 2d4 T\\
\rowcolor{gray!20}102 & Sonno: ti addormenti per 1d6 ore\\
103 & Velocizzato: +1 Reazione a round per 1d6 minuti\\
\rowcolor{gray!20}104 & Scurovisione 9 metri per 1d4 ore\\
111 & Rallentato: -1 Azione per 1d8 minuti\\
\rowcolor{gray!20}112 & Iperveloce: +1 Azione a round per 1d6 minuti \\
113 & Vista ceca 6 metri per 1d4 ore \\
\rowcolor{gray!20}114 & Affaticato: +1 livello di affaticamento\\
121 & Imbarazzo di stomaco: -4 a tutte le Azioni per 1 ore\\
\rowcolor{gray!20}122 & Dermatite violente: -2 alle Azioni che usano le mani per 1d4 ore\\
123 & Vesciche ai piedi: tutto il terreno è difficile per 1d10 minuti\\
\rowcolor{gray!20}124 & Terzo occhio: +2 alle Prove di Consapevolezza per 1d4 ore\\
\end{tabularx}


\end{multicols}

\vfill

\begin{center}
\includegraphics[width=0.5\linewidth]{immagini/trex.png}

\emph{Tyrannosaurus Rex skeleton (the specimen RTMP 81.6.1), California Academy of Sciences, San Francisco}

\emph{(Tirannosauro, scheletro, GS 12!)}
\end{center}

%\vfill


\pagebreak

\section{I Mostri}

\begin{narratore}[Un pò di mostri...]
Le creature qui presentate vogliono essere un esempio, corposo, degli avversari che i tuoi personaggi potrebbero incontrare. Attenzione, non è detto che siano tutti nemici o per forza che abbiano intenzione negative.

Creature più civilizzate avranno una loro condotta etica e morale individuale, anche all'interno di uno stesso gruppo di avversari c'è chi potrebbe essere più nemico o semplicemente indifferente.

Sfrutta le peculiarità e unicità delle creature per creare incontri non scontati e sfidanti dal punto di vista tattico. Non essere scontato ma neanche assurdo nelle scelte, deve sempre esserci della coerenza nel scegliere le creature.
\end{narratore}

\bigskip

\begin{enfasi}{

Amon Goth: Il controllo è potere. Questo è il potere.

Oskar Schindler: è per questo che ci temono?

Amon Goth: Abbiamo il potere di uccidere. Per questo ci temono.

Oskar Schindler: Ci temono perché abbiamo il potere di uccidere arbitrariamente. Un uomo commette un reato, doveva pensarci, lo facciamo uccidere e ci sentiamo in pace. O lo uccidiamo noi stessi e ci sentiamo ancora meglio. Questo non è il potere però! Questa è giustizia, è una cosa diversa dal potere. Il potere è quando abbiamo ogni giustificazione per uccidere e non lo facciamo.

Amon Goth: è questo il potere?

Oskar Schindler: L'avevano gli imperatori questo. Un uomo ruba qualcosa, viene portato davanti all'imperatore e si lascia cadere per terra tremante, implora per avere pietà. E' conscio che sta per andarsene. E l'imperatore lo perdona, invece. Quell'uomo, immeritevole, lo lascia libero.

(Schindler's list - La lista di Schindler, Film, 1993)

\medskip

I mostri sono la poesia della paura. (Stephen King)
}\end{enfasi}

\begin{multicols}{2}

\setlength{\parindent}{0cm}{

%\small{

\pdfbookmark[3]{A}{A}
%\addcontentsline{toc}{subsubsection}{A}

\mostro{Aboleth}
\begin{description}[noitemsep, topsep=0pt, parsep=0pt, partopsep=0pt, leftmargin=0cm, labelwidth=2.2cm]
	\item[\textbf{Taglia/Tipo:}] Grande aberrazione, malvagio
	\item[\textbf{Caratt.:}] \resizebox{0.5\linewidth+1.8cm}{!}{For 5 Des -1 Cos 2 Int 4 Sag 2 Car 4}
	\item[\textbf{Punti Ferita:}] 199,  \textbf{Difesa:} 24,  \textbf{Iniziativa:} +4
	\item[\textbf{Movimento:}] 3 m, nuoto 12 m
	\item[\textbf{Tiri Salvez.:}] \resizebox{0.5\linewidth+1.8cm}{!}{\resizebox{0.5\linewidth+1.8cm}{!}{Tempra +12, Riflessi +9, Volontà +12}}
	\item[\textbf{Incant.:}] immune incantesimi di Illusione inferiori al 2 livello
	\item[\textbf{Comp.:}] Consapevolezza +10, Storia +12
	\item[\textbf{Sensi:}] Scurovisione 36 m
	\item[\textbf{Linguaggi:}] Aquan, telepatia 36 m
	\item[\textbf{Sfida:}] 10 (5900 PX)\smallskip
\end{description}

\emph{\textbf{Anfibio.}} L'aboleth può respirare aria e acqua.

\emph{\textbf{Nube di Muco.}} Mentre è sott'acqua, l'aboleth è avvolto da muco mutante. Una creatura che entri a contatto con l'aboleth, o che lo colpisca con un attacco da mischia mentre si trova entro 1 metro da esso, deve effettuare un Tiro Salvezza di Tempra DC 24. Se lo fallisce, la creatura resta ammalata per 1d4 ore. La creatura ammalata può respirare solo sott'acqua.

\emph{\textbf{Sonda Telepatica.}} Se una creatura comunica telepaticamente con l'aboleth, e l'aboleth può vederla, l'aboleth ne apprende i più grandi desideri.

\textbf{Azioni}

\emph{\textbf{Multiattacco.}} L'aboleth effettua tre attacchi con i tentacoli

\emph{\textbf{Tentacolo.} Attacco con arma da mischia}: +10 a colpire, portata 3 m, un bersaglio.

\emph{Colpisce}: 12 (2d6 + 5) danni contundenti. Se il bersaglio è una creatura, deve riuscire un Tiro Salvezza di Tempra DC 24 o divenire ammalato. La malattia non produce alcun effetto per 1 minuto e può essere rimossa da qualsiasi magia che curi le malattie. Dopo 1 minuto, la pelle della creatura ammalata diventa trasparente e viscida, la creatura non può recuperare Punti Ferita a meno che non sia sott'acqua, e la malattia può essere rimossa solo da \emph{guarire} o un altro incantesimo cura malattie di livello 3 o più. Quando la creatura si trova al di fuori di un corpo d'acqua, subisce 6 (1d12) danni da acido ogni 10 minuti a meno che la sua pelle non venga bagnata prima che siano passati questi 10 minuti.

\emph{\textbf{Coda.} Attacco con arma da mischia}: +9 a colpire, portata 3 m, un bersaglio.

\emph{Colpisce:} 15 (3d6 + 5) danni contundenti.

\emph{\textbf{Schiavizzare (3/Giorno).}} L'aboleth prende a bersaglio una creatura che può vedere entro 9 metri da esso. Il bersaglio deve riuscire un Tiro Salvezza di Volontà DC 24 o restare affascinato magicamente dall'aboleth finché l'aboleth muore o i due si trovano su piani di esistenza differenti. Il bersaglio affascinato è sotto il controllo dell'aboleth e non può effettuare reazioni. L'aboleth e il bersaglio possono comunicare telepaticamente tra di loro a qualsiasi distanza.

Ogniqualvolta il bersaglio affascinato subisce danni, può ripetere il Tiro Salvezza. Se lo riesce, l'effetto termina. Non più di una volta ogni 24 ore, può ripetere il Tiro Salvezza quando si trova almeno a 1,5 chilometri di distanza dall'aboleth.

\textbf{Azioni Aggiuntive}

L'aboleth può effettuare 3 Azioni aggiuntive, scelte tra le opzioni seguenti. Può usare solo un'opzione Aggiuntive alla volta e solo al termine del round di un'altra creatura. L'aboleth recupera le Azioni aggiuntive spese all'inizio del proprio round.

\textbf{Individuare.} L'aboleth effettua una prova Consapevolezza.

\textbf{Risucchio Psichico (Costa 2 Azioni).} Una creatura affascinata dall'aboleth subisce 10 (3d6) danni e l'aboleth recupera un numero di Punti Ferita pari al danno subito dalla creatura.

\textbf{Spazzata di Coda.} L'aboleth effettua un attacco di coda.

\textbf{Ecologia}\\
Ambiente: Qualsiasi Acquatico\\
Organizzazione: Solitario, coppia, nidiata (3-6) o branco (7-19)\\
\textbf{Categoria Tesoro}: F\\
\textbf{Descrizione}\\
Come suggerisce il loro aspetto primitivo gli aboleth sono fra le più antiche forme di vita al mondo. Un aboleth è lungo 7 metri e pesa circa 3,2 tonnellate. Gli aboleth abitano in fondo ai mari nelle loro enormi città, serviti da innumerevoli schiavi.

\mostro{Angelo Deva}
\begin{description}[noitemsep, topsep=0pt, parsep=0pt, partopsep=0pt, leftmargin=0cm, labelwidth=2.2cm]
	\item[\textbf{Taglia/Tipo:}] Media celestiale, buono
	\item[\textbf{Caratt.:}] \resizebox{0.5\linewidth+1.8cm}{!}{For 4 Des 4 Cos 4 Int 3 Sag 5 Car 5}
	\item[\textbf{Punti Ferita:}] 203,  \textbf{Difesa:} 29,  \textbf{Iniziativa:} +4
	\item[\textbf{Movimento:}] 9 m, volo 27 m
	\item[\textbf{Tiri Salvez.:}] \resizebox{0.5\linewidth+1.8cm}{!}{\resizebox{0.5\linewidth+1.8cm}{!}{Tempra +14, Riflessi +14, Volontà +15}}
	\item[\textbf{Comp.:}] Percepire Emozioni +9
	\item[\textbf{Res. Danni:}] da Luce; da arma non magica
	\item[\textbf{Immunità:}] affascinato, affaticato, spaventato
	\item[\textbf{Sensi:}] Scurovisione 36 m
	\item[\textbf{Linguaggi:}] tutte, telepatia 36 m
	\item[\textbf{Sfida:}] 10 (5900 PX)\smallskip
\end{description}

\emph{\textbf{Armi Angeliche.}} Gli attacchi con arma del deva sono magici. Quando il deva colpisce con qualsiasi arma, l'arma infligge 4d8 danni da Luce aggiuntivi (già compresi nell'attacco).

\emph{\textbf{Incantesimi Innati.}} La caratteristica da incantatore innato del deva è il Carisma (DC 17 per i Tiri Salvezza degli incantesimi). Il deva può lanciare in maniera innata i seguenti incantesimi, con l'uso delle sole componenti verbali:

A volontà: \emph{\hyperlink{Conoscere i Tratti}{Conoscere i Tratti}}

1/giorno: \emph{\hyperlink{Comunione}{Comunione}}

\emph{\textbf{Resistenza alla Magia.}} Il deva ha +1d6 ai Tiri Salvezza contro incantesimi e altri effetti magici.

\textbf{Azioni}

\emph{\textbf{Multiattacco.}} Il deva effettua due attacchi da mischia.

\emph{\textbf{Mazza.} Attacco con arma da mischia}: +10 a colpire, portata 1 m, un bersaglio.

\emph{Colpisce:} 7 (1d6 + 4) danni contundenti più 18 (4d8) danni da Luce.

\emph{\textbf{Tocco Guaritore (3/Giorno).}} Il deva entra a contatto con un'altra creatura. Il bersaglio recupera magicamente 20 (4d8 + 2) Punti Ferita ed è libero da qualsiasi cecità, malattia, maledizione, sordità o veleno.

\emph{\textbf{Mutare Forma.}} Il deva può trasformarsi magicamente in un umanoide o bestia il cui grado di sfida sia pari o inferiore al proprio, o tornare alla sua vera forma. Alla morte ritorna alla sua vera forma. Qualsiasi equipaggiamento stia indossando o trasportando viene assorbito o trasportato nella nuova forma (a scelta del deva).

Nella nuova forma, il deva mantiene le sue statistiche di gioco e la facoltà di parlare, ma la sua Difesa, metodi di movimento, Forza, Destrezza e sensi speciali vengono rimpiazzati da quelli della nuova forma, e ottiene qualsiasi statistica o capacità (Azioni aggiuntive e azioni da tana) possedute dalla sua nuova forma e non dalla sua originale.

\textbf{Ecologia}\\
Ambiente: Qualsiasi piano di con Tratti buono\\
Organizzazione: Solitario, coppia, o squadriglia (3-6)\\
\textbf{Categoria Tesoro}: (Spadone Infuocato +1, altro tesoro)\\
\textbf{Descrizione}\\
I deva movanici compongono i ranghi della fanteria delle armate celesti, sebbene trascorrano la maggior parte del loro tempo pattugliando il Piano Positivo, quello Negativo e quello Materiale. Sul Piano Positivo sorvegliano le anime buone erranti. Sul Piano Negativo combattono i non morti e altri strani esseri che cacciano nel famelico vuoto. Le loro rare volte sul Piano Materiale hanno solitamente lo scopo di portare aiuto a potenti mortali, quando un grande pericolo minaccia di far cadere nelle mani del male un intero regno.

\mostro{Angelo Planetar}
\begin{description}[noitemsep, topsep=0pt, parsep=0pt, partopsep=0pt, leftmargin=0cm, labelwidth=2.2cm]
	\item[\textbf{Taglia/Tipo:}] Grande celestiale, buono
	\item[\textbf{Caratt.:}] \resizebox{0.5\linewidth+1.8cm}{!}{For 7 Des 5 Cos 7 Int 4 Sag 6 Car 7}
	\item[\textbf{Punti Ferita:}] 325,  \textbf{Difesa:} 38,  \textbf{Iniziativa:} +5
	\item[\textbf{Movimento:}] 12 m, volo 36 m
	\item[\textbf{Tiri Salvez.:}] \resizebox{0.5\linewidth+1.8cm}{!}{\resizebox{0.5\linewidth+1.8cm}{!}{Tempra +23, Riflessi +21, Volontà +22}}
	\item[\textbf{Comp.:}] Consapevolezza +13
	\item[\textbf{Res. Danni:}] da Luce;
	\item[\textbf{Immunità:}] affascinato, affaticato, spaventato, armi +1
	\item[\textbf{Sensi:}] visione del vero 36 m
	\item[\textbf{Linguaggi:}] tutte, telepatia 36 m
	\item[\textbf{Sfida:}] 16 (15000 PX)\smallskip
\end{description}

\emph{\textbf{Armi Angeliche.}} Gli attacchi con arma del planetar sono magici. Quando colpisce con qualsiasi arma, l'arma infligge 5d8 danni da Luce aggiuntivi (già indicati nell'attacco).

\emph{\textbf{Consapevolezza Divina.}} Il planetar riconosce immediatamente le bugie.

\emph{\textbf{Incantesimi Innati.}} La caratteristica da incantatore innato del planetario è il Carisma (DC 20 per i Tiri Salvezza degli incantesimi). Il planetario può lanciare in maniera innata i seguenti incantesimi, senza bisogno di componenti materiali:

A volontà: \emph{\hyperlink{Conoscere i Tratti}{Conoscere i Tratti}}, \emph{\hyperlink{Invisibilità}{Invisibilità}} (solo personale)

3/giorno: \emph{\hyperlink{Barriera di Lame}{Barriera di Lame}, \hyperlink{Colpo Infuocato}{Colpo Infuocato}}

1/giorno: \emph{\hyperlink{Comunione}{Comunione}, \hyperlink{Controllare Tempo Atmosferico}{Controllare Tempo Atmosferico}, \hyperlink{Piaga degli Insetti}{Piaga degli Insetti}}

\emph{\textbf{Resistenza alla Magia.}} Il planetar ha +1d6 ai Tiri Salvezza contro incantesimi e altri effetti magici.

\textbf{Azioni}

\emph{\textbf{Multiattacco.}} Il planetar effettua due attacchi da mischia.

\emph{\textbf{Spadone.} Attacco con arma da mischia}: +13 a colpire, portata 2 m, un bersaglio.

\emph{Colpisce:} 21 (4d6 + 7) danni taglienti più 22 (5d8) danni da Luce.

\emph{\textbf{Tocco Guaritore (4/Giorno).}} Il planetar entra a contatto con un'altra creatura. Il bersaglio recupera magicamente 30 (6d8 + 3) Punti Ferita ed è libero da qualsiasi cecità, malattia, maledizione, sordità o veleno.

\emph{\textbf{Arrabbiato:}}

- il Planetar evoca le potenze angeliche in suo aiuto. Usando 3 Azioni evoca 1d4 Angeli Deva.

- Il Planetar causa un danno critico (2d6) ogni volta che colpisce fino a fine combattimento. Costa 1 Azione.

\textbf{Ecologia}\\
Ambiente: Qualsiasi piano con Tratti buono\\
Organizzazione: Solitario o coppia\\
\textbf{Categoria Tesoro}: Spadone Sacro +3\\
\textbf{Descrizione}\\
I planetar sono i generali delle armate celestiali volti alla distruzione del male. Un planetar è di norma alto 2,7 metri e pesa circa 250 kg. Sono ottimi diplomatici, ma contro gli immondi preferiscono la guerra piuttosto che negoziare una pace.

\mostro{Angelo Solar}
\begin{description}[noitemsep, topsep=0pt, parsep=0pt, partopsep=0pt, leftmargin=0cm, labelwidth=2.2cm]
	\item[\textbf{Taglia/Tipo:}] Grande celestiale, buono
	\item[\textbf{Caratt.:}] \resizebox{0.5\linewidth+1.8cm}{!}{For 8 Des 6 Cos 8 Int 7 Sag 7 Car 10}
	\item[\textbf{Punti Ferita:}] 426,  \textbf{Difesa:} 46,  \textbf{Iniziativa:} +7
	\item[\textbf{Movimento:}] 15 m, volo 45 m
	\item[\textbf{Tiri Salvez.:}] \resizebox{0.5\linewidth+1.8cm}{!}{\resizebox{0.5\linewidth+1.8cm}{!}{Tempra +29, Riflessi +27, Volontà +28}}
	\item[\textbf{Comp.:}] Consapevolezza +16
	\item[\textbf{Res. Danni:}] da Luce; Fuoco, Freddo, Elettricità
	\item[\textbf{Immunità:}] affascinato, avvelenato, affaticato, spaventato, arma +2
	\item[\textbf{Sensi:}] visione del vero 36 m
	\item[\textbf{Linguaggi:}] tutte, telepatia 36 m
	\item[\textbf{Sfida:}] 21 (33000 PX)\smallskip
\end{description}

\emph{\textbf{Armi Angeliche.}} Gli attacchi con arma del solar sono magici. Quando colpisce con qualsiasi arma, l'arma infligge 6d8 danni da Luce aggiuntivi (già compresi nell'attacco).

\emph{\textbf{Consapevolezza Divina.}} Il solar riconosce immediatamente le bugie.

\emph{\textbf{Incantesimi Innati.}} La caratteristica da incantatore innato del solar è il Carisma (DC 25 per i Tiri Salvezza degli incantesimi). Il solar può lanciare in maniera innata i seguenti incantesimi, senza bisogno di componenti materiali:

A volontà: \emph{\hyperlink{Conoscere i Tratti}{Conoscere i Tratti}}, \emph{\hyperlink{Invisibilità}{Invisibilità}} (solo personale)

3/giorno: \emph{\hyperlink{Barriera di Lame}{Barriera di Lame}, \hyperlink{Colpo Infuocato}{Colpo Infuocato}}

1/giorno: \emph{\hyperlink{Comunione}{Comunione}, \hyperlink{Controllare Tempo Atmosferico}{Controllare Tempo Atmosferico}}

\emph{\textbf{Resistenza alla Magia.}} Il solar ha +1d6 ai Tiri Salvezza contro incantesimi e altri effetti magici.

\textbf{Azioni}

\emph{\textbf{Multiattacco.}} Il solar effettua due attacchi con lo spadone.

\emph{\textbf{Spadone.} Attacco con arma da mischia}: +16 a colpire, portata 1 m, un bersaglio.

\emph{Colpisce:} 22 (4d6 + 8) danni taglienti più 27 (6d8) danni da Luce.

\emph{\textbf{Arco Lungo dell'Uccisione.} Attacco con arma a distanza}: +17 a colpire, gittata 45m, un bersaglio.

\emph{Colpisce:} 15 (2d8 + 6) danni perforanti più 27 (6d8) danni da Luce. Se il bersaglio è una creatura con 100 Punti Ferita o meno, deve riuscire un Tiro Salvezza di Tempra DC 35 o morire.

\emph{\textbf{Spada Volante.}} Il solar libera il suo spadone perché fluttui magicamente in uno spazio non occupato entro 1 metro da lui. Se il solar può vedere la spada, con un'azione gratuita le può ordinare mentalmente di volare per un massimo di 15 metri ed effettuare un attacco contro un bersaglio o ritornare nella mano del solar. Se la spada fluttuante è bersaglio di un effetto, si considera come se fosse impugnata dal solar. Se il solar muore, la spada fluttuante cade a terra.

\emph{\textbf{Tocco Guaritore (4/Giorno).}} Il solar entra a contatto con un'altra creatura. Il bersaglio recupera magicamente 40 (8d8 + 4) Punti Ferita ed è libero da qualsiasi cecità, malattia, maledizione, sordità o veleno.

Il solar può effettuare 3 azioni aggiuntive, scelte tra le opzioni seguenti. Può usare solo un'Azione Aggiuntiva alla volta e solo al termine del round di un'altra creatura. Il solar recupera le azioni aggiuntive spese all'inizio del proprio round.

\textbf{Esplosione Incandescente (Costa 2 Azioni).} Il solar emette energia magica divina. Ogni creatura di sua scelta, in un raggio di 3 metri, deve effettuare un Tiro Salvezza su Riflessi DC 35, subendo 14 (4d6) danni da fuoco più 14 (4d6) danni da Luce se fallisce il Tiro Salvezza, o la metà se lo riesce.

\textbf{Sguardo Accecante (Costa 3 Azioni).} Il solar prende a bersaglio una creatura entro 9 metri e che possa vedere. Se il bersaglio può vedere il solar, il bersaglio deve riuscire un Tiro Salvezza su Tempra DC 35 o restare accecato finché un incantesimo come \emph{\hyperlink{Ristorare Inferiore}{Ristorare Inferiore}} non rimuoverà la cecità.

\textbf{Teletrasporto.} Il solar si teletrasporta magicamente fino a 36 metri di distanza, insieme a tutto l'equipaggiamento che sta indossando o trasportando, in uno spazio non occupato e che può vedere.

\textbf{Ecologia}\\
Ambiente: Qualsiasi piano con Tratti buono\\
Organizzazione: Solitario o coppia\\
\textbf{Categoria Tesoro}: Armatura Completa +5, Spadone Danzante +5, Arco Lungo Composito +5\\
\textbf{Descrizione}\\
I solar sono i più potenti tra gli angeli, solitamente braccio destro delle divinità o campioni di cause importanti. Hanno un aspetto quasi umano e sono alti circa 2,7 metri, con pelle argentata o dorata. Sono benedetti con potenti capacità magiche e sono in grado di uccidere le creature malvagie più forti.

I solar possono seguire tracce lasciate da potenti nemici e sono conosciuti come uccisori di mostri. Alcuni proteggono concetti soprannaturali, oggetti o creature di grande importanza, come i condotti di energia solare o le catene che imprigionano divinità malvagie. Possono diventare profeti e guru, gettando le basi per grandi religioni.

Pur non essendo divinità, il loro potere si avvicina a quello dei semidei e spesso fanno da consiglieri per le giovani divinità. Sono creati come servitori degli dei, amalgamando anime buone ed energia divina pura. Alcuni solar provengono dalla promozione di angeli minori come deva e planetar.

I solar che passano tempo sul Piano Materiale possono influenzare linee di sangue umano, creando discendenti mezzo-celestiali. Tendono a evitare il contatto diretto con la loro progenie per evitare vergogna, ma controllano e aiutano da lontano.

Rispettati da tutti gli angeli, i solar a volte comandano armate contro le legioni dell'Inferno e le orde dell'Abisso.

\mostro{Ankheg}
\begin{description}[noitemsep, topsep=0pt, parsep=0pt, partopsep=0pt, leftmargin=0cm, labelwidth=2.2cm]
	\item[\textbf{Taglia/Tipo:}] Grande mostruosità, disallineato
	\item[\textbf{Caratt.:}] \resizebox{0.5\linewidth+1.8cm}{!}{For 3 Des 0 Cos 1 Int -5 Sag 1 Car -2}
	\item[\textbf{Punti Ferita:}] 51,  \textbf{Difesa:} 14,  \textbf{Iniziativa:} +0
	\item[\textbf{Movimento:}] 9 m, scavo 3 m
	\item[\textbf{Tiri Salvez.:}] \resizebox{0.5\linewidth+1.8cm}{!}{Tempra +3, Riflessi +3, Volontà +3}
	\item[\textbf{Sensi:}] Scurovisione 18 m, percezione tellurica 18 m
	\item[\textbf{Sfida:}] 2 (450 PX)\smallskip
\end{description}

\textbf{Azioni}

\emph{\textbf{Morso.} Attacco con arma da mischia}: +5 a colpire, portata 1 m, un bersaglio.

\emph{Colpisce:} 10 (2d6 + 3) danni taglienti più 3 (1d6) danni da acido. Se il bersaglio è una creatura di taglia Grande o inferiore, è afferrata (DC 13 per fuggire). Fino al termine dell'afferrare, l'ankheg può mordere solo la creatura afferrata e ha +1d6 ai tiri di attacco contro di essa.

\emph{\textbf{Spruzzo Acido}} L'ankheg sputa acido in una linea lunga 9 metri e larga 1 metro, purché non stia afferrando nessuna creatura. Ogni creatura su quella linea deve effettuare un Tiro Salvezza di Riflessi DC 14, e subire 10 (3d6) danni da acido se fallisce il Tiro Salvezza, o la metà di questi danni se lo riesce.

\textbf{Ecologia}\\
Ambiente: Pianure temperate o calde\\
Organizzazione: Solitario, coppia o nido (3-6)\\
\textbf{Categoria Tesoro}: C\\
\textbf{Descrizione}\\
Gli ankheg sono mostri scavatori che prediligono le campagne per via del terreno morbido che facilita i loro spostamenti. Solitamente evitano le aree densamente popolate ma la loro predilezione per la carne del bestiame e degli umani li tiene lontani dalle zone disabitate. Nei deserti remoti, si trovano ankheg più grandi che si nutrono di scorpioni e cammelli.

Alcuni ankheg sono addestrabili e possono diventare animali da carico, sebbene il loro comportamento imprevedibile li renda pericolosi nelle regioni civilizzate.

\mostro{Arpia}
\begin{description}[noitemsep, topsep=0pt, parsep=0pt, partopsep=0pt, leftmargin=0cm, labelwidth=2.2cm]
	\item[\textbf{Taglia/Tipo:}] Media mostruosità, Arrogante, Impulsivo
	\item[\textbf{Caratt.:}] \resizebox{0.5\linewidth+1.8cm}{!}{For 1 Des 1 Cos 1 Int -2 Sag 0 Car 1}
	\item[\textbf{Punti Ferita:}] 33,  \textbf{Difesa:} 14,  \textbf{Iniziativa:} +1
	\item[\textbf{Movimento:}] 6 m, volo 12 m
	\item[\textbf{Tiri Salvez.:}] \resizebox{0.5\linewidth+1.8cm}{!}{Tempra +3, Riflessi +3, Volontà +3}
	\item[\textbf{Linguaggi:}] Comune
	\item[\textbf{Sfida:}] 1 (200 PX)\smallskip
\end{description}

\textbf{Azioni}

\emph{\textbf{Multiattacco.}} L'arpia effettua due attacchi: uno con gli artigli e uno con il randello.

\emph{\textbf{Artigli.} Attacco con arma da mischia}: +4 a colpire, portata 1 m, un bersaglio.

\emph{Colpisce:} 5 (2d4 + 1) danni taglienti, 1 danno da Sanguinamento.

\emph{\textbf{Randello.} Attacco con arma da mischia}: +4 a colpire, portata 1 m, un bersaglio.

\emph{Colpisce:} 3 (1d4 + 1) danni contundenti.

\emph{\textbf{Venti affamati} (2 Azioni)}: L'arpia usa il vento per avvicinare le sue prede. Un bersaglio entro 6 metri deve fare un Tiro Salvezza su Tempra DC 12 o essere tirato a fianco all'arpia. Se il bersaglio è stato alzato da terra e non può volare, poi cade normalmente.

\emph{\textbf{Canto Ammaliatore.}} L'arpia canta una melodia magica. Ogni umanoide e gigante entro 90 metri dall'arpia e che possa udire la canzone deve riuscire un Tiro Salvezza di Volontà DC 13 o restare affascinato fino al termine della canzone. L'arpia deve effettuare un'Azione Immediata durante il suo prossimo round per continuare a cantare. Può smettere di cantare in qualsiasi momento. Il canto ha termine se l'arpia è inabile.

Mentre è affascinato dall'arpia, un bersaglio è inabile e ignora le canzoni di altre arpie. Se il bersaglio affascinato si trova a più di 1 metro dall'arpia, il bersaglio deve muoversi durante il proprio round per dirigersi verso l'arpia usando la via più diretta. Prima di muoversi in un terreno pericoloso, come lava o un pozzo, e prima di subire danno da qualsiasi fonte che non sia l'arpia, il bersaglio potrà ripetere il Tiro Salvezza. Una creatura può ripetere il Tiro Salvezza al termine di ciascun proprio round. Se il Tiro Salvezza ha successo, l'effetto ha termine per quel bersaglio.

Un bersaglio che riesce il Tiro Salvezza è immune al canto di quell'arpia per le successive 24 ore.

\textbf{Ecologia}\\
Ambiente: Paludi Temperate\\
Organizzazione: Solitario, coppia o stormo (3-12)\\
\textbf{Categoria Tesoro}: R (C)\\
\textbf{Descrizione}\\
Spesso viste come creature malvagie e corrotte, le arpie sanno come gli altri pensano e agiscono. Questa capacità percettiva offre loro un vantaggio nel trovare i loro pasti preferiti. Sebbene le creature selvatiche cadano facilmente vittime del canto ammaliatore, queste malvagie donne-uccello preferiscono pasti conditi con complessi pensieri senzienti. Le facili prede rendono il pasto noioso.

Anche se in definitiva selvagge e senza alcun rimorso per le loro azioni, diverse arpie vivono presso le società umanoidi e si divertono a sfruttare le creature che reputano potenziali pasti.

Le arpie tendono ad indossare ninnoli e ciondoli rubati alle loro vittime, perché amano compiacersi dei brillanti ornamenti degli uomini. Da vicino queste creature trasudano del puzzo delle loro vittime divorate e raramente lasciano che le creature non ancora ammaliate si avvicinino troppo, cosicché non sentano l'odore del sangue e della putrefazione sulle loro penne. Per questo motivo, molte arpie si cospargono di profumi e oli aromatici.

Le arpie sono marcatamente differenti a seconda della regione in cui vivono. Alcune assomigliano ad una mescolanza di avvoltoi e donne, mentre altri portano sulle penne i tratti regali di falchi e falconi. Rare nidiate di arpie, in luoghi isolati e tropicali del mondo, hanno anche piume colorate come i pappagalli.

\mostro{Azer}
\begin{description}[noitemsep, topsep=0pt, parsep=0pt, partopsep=0pt, leftmargin=0cm, labelwidth=2.2cm]
	\item[\textbf{Taglia/Tipo:}] Media elementale, legale
	\item[\textbf{Caratt.:}] \resizebox{0.5\linewidth+1.8cm}{!}{For 3 Des 1 Cos 2 Int 1 Sag 1 Car 0}
	\item[\textbf{Punti Ferita:}] 51,  \textbf{Difesa:} 15,  \textbf{Iniziativa:} +1
	\item[\textbf{Movimento:}] 9 m
	\item[\textbf{Tiri Salvez.:}] \resizebox{0.5\linewidth+1.8cm}{!}{Tempra +4, Riflessi +3, Volontà +3}
	\item[\textbf{Imm. Danni:}] Fuoco, Veleno
	\item[\textbf{Linguaggi:}] Ignan
	\item[\textbf{Sfida:}] 2 (450 PX)\smallskip
\end{description}

\emph{\textbf{Armi Riscaldate.}} Quando l'azer colpisce con un'arma da mischia in metallo, infligge 3 (1d6) danni da fuoco aggiuntivi (già inclusi nell'attacco).

\emph{\textbf{Corpo Riscaldato.}} Una creatura che entri a contatto con l'azer o lo colpisca con un attacco da mischia mentre si trova entro 1 metro da lui subisce 5 (1d10) danni da fuoco.

\emph{\textbf{Fuoco Vivente.}} Un azer non ha bisogno di cibo, bevande o di dormire.

\emph{\textbf{Illuminazione.}} L'azer irradia luce intensa in un raggio di 3 metri e luce fioca per 6 metri.

\textbf{Azioni}

\emph{\textbf{Martello da Guerra.} Attacco con arma da mischia}: +6 a colpire, portata 1 m, un bersaglio.

\emph{Colpisce:} 7 (1d8 + 3) danni contundenti, o 8 (1d10 + 3) danni contundenti se usato a due mani per effettuare un attacco da mischia, più 3 (1d6) danni da fuoco.

\textbf{Ecologia}\\
Ambiente: Qualsiasi terreno (Piano del Fuoco)\\
Organizzazione: Solitario, coppia, gruppo (3-6), squadra (11-20 più 2 sergenti di 3° livello e 1 capo di 3°-6° livello) o clan (30-100 più 50\% di non combattenti più 1 sergente di 3° livello ogni 20 adulti, 5 tenenti di 5° livello e 3 capitani di 7° livello)\\
\textbf{Categoria Tesoro}: Armatura a Scaglie Perfetta, Martello da Guerra Perfetto, Martello Leggero, N\\
\textbf{Descrizione}\\
Una Razza orgogliosa e industriosa proveniente dal Piano del Fuoco, gli Azer lavorano nelle loro fortezze di bronzo e d'ottone, sempre pronti a combattere la loro lunga e ribollente guerra contro gli Efreet. Gli Azer vivono in una società in cui ogni membro sa qual è il suo posto. Nati con specifici doveri, solitamente legati alle attività del padre o della madre, gli Azer si dedicano a queste occupazioni per tutta la vita. Un sistema di caste provvede a tenere ulteriormente in riga la società Azer. I nobili, che regnano senza dover rendere conto a nessuno, indossano kilt di ottone decorato come simbolo della loro casta, mentre quelli dei mercanti e dei proprietari di negozi sono in resistente bronzo. I kilt di rame sono indossati dalla casta lavoratrice, composta da servitori, artigiani e braccianti.

Capaci di incanalare calore tramite le Armi e gli attrezzi in metallo, gli Azer non utilizzano quasi mai Armi non metalliche, e prediligono il corpo a corpo agli attacchi a distanza. Sono soliti fare prigionieri, riportandoli alle loro fortezze e obbligandoli a lavorare per loro per un anno e un giorno.

Nella leggendaria Città d'Ottone abitano più di mezzo milione di Azer. La maggior parte di questi sfortunati Azer vive una vita di Schiavitù sotto gli Efreet. Gli Azer soggiogati a questa Schiavitù continuano a eseguire i loro doveri senza porre domande, preferendo aspettare la conclusione dei loro contratti o sperando che i loro padroni muoiano o vengano sconfitti. La dedizione all'ordine arde intensa in questa Razza, al punto che alcuni degli Schiavi Azer fungono da supervisori sulla loro stessa gente. Al di fuori della Città d'Ottone, gli Azer sono liberi di vivere le loro vite, spesso in altre metropoli Planari, creando oggetti, vendendo merci e gestendo taverne.

A un occhio non allenato gli Azer si somigliano tra loro in modo impressionante. Sono alti 1,2 metri ma pesano 100 kg.

%\addcontentsline{toc}{subsubsection}{B}
\pdfbookmark[3]{B}{B}

\mostro{Banshee}
\begin{description}[noitemsep, topsep=0pt, parsep=0pt, partopsep=0pt, leftmargin=0cm, labelwidth=2.2cm]
	\item[\textbf{Taglia/Tipo:}] Media non morto, Arrogante, Vanitoso
	\item[\textbf{Caratt.:}] \resizebox{0.5\linewidth+1.8cm}{!}{For -5 Des 5 Cos 0 Int 1 Sag 1 Car 4}
	\item[\textbf{Punti Ferita:}] 87,  \textbf{Difesa:} 22,  \textbf{Iniziativa:} +5
	\item[\textbf{Movimento:}] 0 m, volo 18 m, Fluttuare
	\item[\textbf{Tiri Salvez.:}] \resizebox{0.5\linewidth+1.8cm}{!}{Tempra +4, Riflessi +9, Volontà +5}
	\item[\textbf{Res. Danni:}] Acido, Elettricità, Fuoco, Suono; arma magica +1
	\item[\textbf{Imm. Danni:}] da Vuoto, Veleno, Freddo, da arma non magica
	\item[\textbf{Immunità:}] affascinato, afferrato, intralciato, paralizzato, pietrificato, prono, affaticato, sanguinamento
	\item[\textbf{Sensi:}] Scurovisione 18 m
	\item[\textbf{Linguaggi:}] Elfico, Comune, Expiran
	\item[\textbf{Sfida:}] 4 (1100 PX)\smallskip
\end{description}

\emph{\textbf{Individuazione della Vita}}. La Banshee percepisce la presenza di creature che non siano non morti e costrutti entro un raggio di 5 chilometri. Conosce la direzione generale in cui si trovano, ma non la loro precisa ubicazione.

\emph{\textbf{Movimento Incorporeo}}. La Banshee può muoversi attraverso altre creature e oggetti come se fossero terreno difficile. Subisce 5 (1d10) danni da forza se termina il suo round all'interno di un oggetto.

\emph{\textbf{Natura Non Morta.}} La Banshee non ha bisogno di aria, cibo, bevande o sonno.

\emph{\textbf{Sensibilità alla Luce}}. Mentre è alla luce del sole, la Banshee è Rallentata 1.

\textbf{Azioni}

\emph{\textbf{Tocco Corruttore}}. Attacco a Tocco: +6 al Tiro per Colpire, portata 1 m, un bersaglio.

\emph{Colpito}: 12 (3d6 +2) danni da Vuoto.

\textbf{Reazione: \emph{Urlo d'odio}}: usando una reazione la Banshee urla contro un avversario che ha fatto un Tiro Critico contro di lei. L'avversario deve fare un Tiro Salvezza su Tempra con bonus Carisma DC 16 per dimezzare 2d8 di danno da forza.

\emph{\textbf{Volto Terrificante}}. Ogni creatura che non sia un non morto situata entro 18 metri dalla Banshee e che sia in grado di vederla deve superare un Tiro Salvezza su Volontà con modificatore Carisma con DC 18, altrimenti è spaventata per 1 minuto. Un bersaglio spaventato può ripetere il Tiro Salvezza alla fine di ogni suo round, subendo -1d6 se la Banshee si trova entro linea di vista; se supera il tiro, l'effetto per lui termina. Se un bersaglio supera il Tiro Salvezza o l'effetto per lui termina, quel bersaglio è immune al Volto Terrificante della Banshee per le 24 ore successive.

\emph{\textbf{Lamento (1/Giorno)}}. La Banshee emette un lamento funesto, purché non sia esposta alla luce del sole. Questo lamento non ha alcun effetto sui costrutti e sui non morti. Ogni altra creatura situata entro 9 metri da lei e in grado di udirla deve effettuare un Tiro Salvezza su Tempra con bonus Carisma DC 17; se lo fallisce, scende a O Punti Ferita, mentre se lo supera, subisce 35 (10d6) danni da forza.

\textbf{Ecologia}\\
Ambiente: Qualsiasi\\
Organizzazione: Solitario\\
\textbf{Categoria Tesoro}: D\\
\textbf{Descrizione}\\
La Banshee è lo spirito infuriato di una donna che ha tradito i propri cari o è stata a sua volta tradita. Impazzita per il dolore, la Banshee riversa la propria vendetta su ogni creatura vivente (innocente o colpevole) con il suo temibile tocco e le sue grida mortali.

\mostro{Basilisco}
\noindent
\begin{description}[noitemsep, topsep=0pt, parsep=0pt, partopsep=0pt, leftmargin=0cm, labelwidth=2.2cm]
	\item[\textbf{Taglia/Tipo:}] Media mostruosità, disallineato
	\item[\textbf{Caratt.:}] \resizebox{0.5\linewidth+1.8cm}{!}{For 3 Des -1 Cos 2 Int -4 Sag -1 Car -2}
	\item[\textbf{Punti Ferita:}] 70,  \textbf{Difesa:} 15,  \textbf{Iniziativa:} -1
	\item[\textbf{Movimento:}] 6 m
	\item[\textbf{Tiri Salvez.:}] \resizebox{0.5\linewidth+1.8cm}{!}{Tempra +5, Riflessi +3, Volontà +3}
	\item[\textbf{Sensi:}] Scurovisione 18 m
	\item[\textbf{Sfida:}] 3 (700 PX)\smallskip
\end{description}

\emph{\textbf{Sguardo Pietrificante.}} Se una creatura comincia il suo round entro 9 metri dal basilisco e i due si possono vedere vicendevolmente, se non inabile il basilisco può obbligare la creatura ad effettuare un Tiro Salvezza di Tempra DC 14. Se la creatura fallisce il Tiro Salvezza diventa Rallentato 1. La creatura deve ripetere il Tiro Salvezza al termine del suo prossimo round. Se lo riesce, l'effetto termina. Se lo fallisce, la creatura è pietrificata finché non viene liberata dall'incantesimo \emph{\hyperlink{Ristorare Superiore}{Ristorare Superiore}} o altra magia.

Una creatura che non sia sorpresa e che voglia attaccare il basilisco senza guardarla direttamente ha -1d6 al Tiro per Colpire.

Se il basilisco si trova entro 9 metri dal suo riflesso a luce intensa e lo vede, lo scambia per un rivale e diventa il bersaglio del proprio sguardo.

\textbf{Azioni}

\emph{\textbf{Morso.} Attacco con arma da mischia}: +6 a colpire, portata 1 m, un bersaglio.

\emph{Colpisce:} 10 (2d6 + 3) danni perforanti più Veleno del Basilisco.

\emph{Veleno:} Veleno del Basilisco, F, istantaneo, TS Tempra 14 oppure Rallentato 1/3r.

\textbf{Ecologia}\\
Ambiente: Qualsiasi\\
Organizzazione: Solitario, coppia o colonia (3-6)\\
\textbf{Categoria Tesoro}: F\\
\textbf{Descrizione}\\
Il basilisco, spesso chiamato \emph{Re dei Serpenti} è un rettile a otto zampe di indole aggressiva che ha la capacità di trasformare le creature in pietra con il suo sguardo. La leggenda narra che, come la Cockatrice, i primi basilischi nacquero da uova deposte da serpenti e covate da galli, ma ben poco nella fisiologia del basilisco lascia spazio a questa teoria.

I basilischi vivono in quasi tutti gli ambienti asciutti, dalla foresta al deserto, e la loro pelle tende a rispecchiare l'ambiente che li circonda

Tendono a usare come rifugio le grotte, le tane o altre zone riparate. Questi rifugi sono spesso segnalati da statue raffiguranti persone e animali in pose naturali, che non sono altro che i resti pietrificati degli sventurati imbattutisi in un basilisco.

I basilischi hanno la capacità di consumare le creature pietrificate. Quando non sono in attesa dei piccoli mammiferi, uccelli o rettili che fanno parte della loro dieta, i basilischi passano il tempo a dormire nelle tane. Coloro che sono abbastanza coraggiosi da catturare i basilischi o da nascondere un tesoro vicino a loro, scoprono che questi esseri possono fare da custodi o da cani da guardia.

Un basilisco adulto è lungo quasi 3 metri, di cui la metà occupata dalla lunga coda, e pesa 135 chili. Alcune razze presentano delle piccole corna ricurve sul naso o piccole creste di pungiglioni ossuti sopra la testa simili a una corona. Sebbene siano creature in genere solitarie che si riuniscono solo per accoppiarsi e deporre le uova, in zone particolarmente pericolose possono riunirsi in piccoli gruppi per proteggersi e attaccare gli intrusi in massa.

Per motivi ignoti, le donnole, furetti e topine sono immuni allo sguardo del basilisco, e a volte si intrufolano nelle tane mentre l'adulto è a caccia per cibarsi dei suoi piccoli.

%\begin{center}
%\includegraphics[width=0.45\textwidth]{immagini/head-of-a-war-hammer.png}
%\end{center}

\mostro{Behir}
\noindent
\begin{description}[noitemsep, topsep=0pt, parsep=0pt, partopsep=0pt, leftmargin=0cm, labelwidth=2.2cm]
	\item[\textbf{Taglia/Tipo:}] Enorme mostruosità, malvagio
	\item[\textbf{Caratt.:}] \resizebox{0.5\linewidth+1.8cm}{!}{For 6 Des 3 Cos 4 Int -2 Sag 2 Car 1}
	\item[\textbf{Punti Ferita:}] 221,  \textbf{Difesa:} 29,  \textbf{Iniziativa:} +3
	\item[\textbf{Movimento:}] 15 m, scalata 12 m
	\item[\textbf{Tiri Salvez.:}] \resizebox{0.5\linewidth+1.8cm}{!}{\resizebox{0.5\linewidth+1.8cm}{!}{Tempra +15, Riflessi +14, Volontà +13}}
	\item[\textbf{Comp.:}] Furtività +7
	\item[\textbf{Imm. Danni:}] Elettricità
	\item[\textbf{Sensi:}] Scurovisione 27 m
	\item[\textbf{Linguaggi:}] Draconico
	\item[\textbf{Sfida:}] 11 (7200 PX)\smallskip
\end{description}

\textbf{Azioni}

\emph{\textbf{Multiattacco.}} Il behir effettua due attacchi: uno con il morso e uno per stritolare.

\emph{\textbf{Morso.} Attacco con arma da mischia}: +13 a colpire, portata 3 m, un bersaglio.

\emph{Colpisce:} 22 (3d10 + 6) danni perforanti.

\emph{\textbf{Stritolare.} Attacco con arma da mischia}: +13 a colpire, portata 1 m, una creatura di taglia Grande o inferiore.

\emph{Colpisce:} 17 (2d10 + 6) danni contundenti più 17 (2d10 + 6) danni taglienti. Il bersaglio è afferrato (DC 16 per fuggire).

\emph{\textbf{Inghiottire.}} Il behir effettua una attacco di morso contro un bersaglio di taglia Media o inferiore che sta afferrando. Se l'attacco colpisce, il bersaglio è inghiottito, e l'afferrare ha termine. Il bersaglio inghiottito è accecato e intralciato, ha copertura completa contro gli attacchi e altri effetti all'esterno del behir, e subisce 21 (6d6) danni da acido all'inizio di ciascun round del behir. Il behir può inghiottire solo una creatura alla volta.

Se il behir subisce 30 o più danni in un singolo round da una creatura che ha inghiottito, deve riuscire un Tiro Salvezza di Tempra DC 19 al termine di quel round o vomitare la creatura, che ricade prona in uno spazio entro 3 metri dal behir. Se il behir muore, una creatura inghiottita non è più intralciata da esso e può uscire dal cadavere utilizzando 2 Azioni e uscendo prona.

\emph{\textbf{Soffio di Fulmine (Ricarica 5-6).}} Il behir esala fulmini in una linea lunga 6 metri e larga 1 metro. Ogni creatura su quella linea deve effettuare un Tiro Salvezza di Riflessi DC 24 e subire 66 (12d10) danni da elettricità se fallisce il Tiro Salvezza, o la metà di questi danni se lo riesce.

\emph{\textbf{Arrabbiato:}} il Behir ricarica il soffio di fulmine. Costa 2 Azioni.

\textbf{Ecologia}\\
Ambiente: Colline e Deserti Caldi\\
Organizzazione: Solitario o coppia\\
\textbf{Categoria Tesoro}: U\\
\textbf{Descrizione}\\
Il behir è una creatura territoriale che striscia per le colline sabbiose e le rocce del deserto, cacciando tutte le creature che osano entrare nel suo territorio. Con una lunghezza media di 12 metri e un peso di circa 1800 kg, il behir è dotato di sei paia di zampe con artigli che utilizza in combattimento per afferrare i nemici, correre o scalare.

Nonostante la sua furia bestiale, il behir non è necessariamente malvagio e può essere convinto da negoziatori intrepidi. I behir sono spesso legati ai draghi blu, ma la natura di questo legame è sconosciuta. Questo legame suscita rancore nei behir, rendendoli pronti ad attaccare qualunque drago entri nel loro territorio.

\mostro{Blatta Esplosiva}
\noindent
\begin{description}[noitemsep, topsep=0pt, parsep=0pt, partopsep=0pt, leftmargin=0cm, labelwidth=2.2cm]
	\item[\textbf{Taglia/Tipo:}] Piccolo Elementale, neutrale
	\item[\textbf{Caratt.:}] \resizebox{0.5\linewidth+1.8cm}{!}{For 1 Des 2 Cos 1 Int -5 Sag -1 Car -2}
	\item[\textbf{Punti Ferita:}] 51,  \textbf{Difesa:} 16,  \textbf{Iniziativa:} +2
	\item[\textbf{Movimento:}] 4 m, salto 9 m, scavare 2 m
	\item[\textbf{Tiri Salvez.:}] \resizebox{0.5\linewidth+1.8cm}{!}{Tempra +3, Riflessi +4, Volontà +3}
	\item[\textbf{Res. Danni:}] contundenti
	\item[\textbf{Imm. Danni:}] Fuoco
	\item[\textbf{Immunità:}] affaticato, spaventato
	\item[\textbf{Sensi:}] Vista Cieca 5 m
	\item[\textbf{Sfida:}] 2 (450 PX)\smallskip
\end{description}

\emph{Individuazione del fuoco}: la Blatta Esplosiva può percepire fuochi entro 100 metri di distanza, purché pari o superiori ad una torcia

\emph{Scavare}: la blatta esplosiva può scavare nel terreno solido a metà del proprio movimento.

\textbf{Azioni}

\emph{\textbf{Multiattacco.}} la Blatta Esplosiva può effettuare 1 attacco di carica oppure emettere una poltiglia di fuoco.

\emph{\textbf{Carica.}} Attacco da mischia: +6 a a colpire, portata 1 metro, un bersaglio.

\emph{Colpisce:} 12 (3d6 + 3) danni contundenti. La creature deve effettuare un Tiro Salvezza su Tempra a DC 12 o cadere prona.

\emph{\textbf{Poltiglia di Fuoco}} Attacco da distanza: +5 al colpire, gittata 3 metri. La Blatta Esplosiva rigurgita un liquido appiccicoso e infiammabile all'aria. Ricarica 3-6.

\emph{Colpisce:} 18 (4d6 + 6) danni da fuoco. Tiro Salvezza su Riflessi DC 14 per dimezzare.

\emph{\textbf{Morte:}} Quando la Blatta Esplosiva muore la gelatina all'interno a contatto con l'aria esplode tutto intorno, nel raggio di 1 metro attorno alla blatta le fiamme causano 12 (4d6) di danno, Tiro Salvezza su Riflessi DC 15 per dimezzare.

\textbf{Ecologia}\\
Ambiente: caverne calde\\
Organizzazione: Solitario, nido (8-64)\\
\textbf{Categoria Tesoro}: Diamante 1d4x1d50mo\\
\textbf{Descrizione}\\
Le Blatta Esplosive sono creature native tra il piano elementale del fuoco e della terra. Solitamente attirati da ambienti ricchi di fiamme, pietra o almeno caldo e terra.
Dalla forma proporzionata a quelli di una comune blatta se non lunga circa 40 cm e pensante circa 4 kg, è una creatura completamente priva di intelletto agendo solo per puro istinto.
Sono ormai comuni nella caverne prossime a vulcani o tane di drago rosso essendosi abituate a vivere sulla Terra.

Nel nido dove dimorano c'è almeno una regina che comanda le blatte, estremamente più grossa e forte. Le Blatte Esplosive si nutrono di carbone, legni bruciati, carcasse bruciate. Sono estremamente golosi di diamanti che una volta bruciati sono delle vere e proprie leccornie.

\mostro{B.O.C.}
\noindent
\begin{description}[noitemsep, topsep=0pt, parsep=0pt, partopsep=0pt, leftmargin=0cm, labelwidth=2.2cm]
	\item[\textbf{Taglia/Tipo:}] grande mostruosità, malvagio
	\item[\textbf{Caratt.:}] \resizebox{0.5\linewidth+1.8cm}{!}{For 4 Des 3 Cos 2 Int -2 Sag 1 Car -1}
	\item[\textbf{Punti Ferita:}] 88,  \textbf{Difesa:} 20,  \textbf{Iniziativa:} +3
	\item[\textbf{Movimento:}] 13 m
	\item[\textbf{Tiri Salvez.:}] \resizebox{0.5\linewidth+1.8cm}{!}{Tempra +6, Riflessi +7, Volontà +5}
	\item[\textbf{Comp.:}] Furtività +8
	\item[\textbf{Sensi:}] Scurovisione 20 m, Visione Crepuscolare 18 m
	\item[\textbf{Linguaggi:}] Comune, può solo comprenderlo
	\item[\textbf{Sfida:}] 4 (1100 PX)\smallskip
\end{description}

\textbf{Azioni}

\emph{\textbf{Multiattacco.}} Il B.O.C effettua due attacchi con artigli ed uno con il morso, oppure effettua due attacchi con i tentacoli

\emph{\textbf{Artigli.} Attacco con arma da mischia}: +7 a colpire, portata 3 m, un bersaglio, 1 danno da Sanguinamento.

\emph{Colpisce:} 7 (1d6 + 4) danni da taglio.

\emph{\textbf{Morso.} Attacco con arma da mischia}: per ogni artiglio che ha colpito il B.O.C ottiene +2 al colpire con il morso. +6 a colpire, portata 3 m, un bersaglio.

\emph{Colpisce:} 10 (1d8 + 6) danni da taglio.

\emph{\textbf{Tentacoli.} Attacco con arma da mischia}: Ogni tentacolo può colpire fino a 6 metri di distanza ed ognuno può colpire un bersaglio diverso, +6 al colpire.

\emph{Colpisce:} 6 (1d4 + 4) danni contundenti

\textbf{Reazione: \emph{Frustata}}: quando colpito da un avversario in mischia il B.O.C. usa una Reazione per effettuare un colpo con i Tentacoli.

\emph{\textbf{Deflettere la luce.}} Il B.O.C. è costantemente influenzato da un effetto che ne altera la posizione, ogni Tiro per Colpire ha -1d6. Questa penalità si elimina se si può attaccare il B.O.C. senza usare la vista per individuarlo.

Il B.O.C. piega costantemente la luce intorno a se apparendo quasi un metro spostato rispetto alla sua reale posizione. Questa abilità non è influenzata da visioni di tipo normali, solo \hyperlink{Visione del Vero}{Visione del Vero}, vista cieca o senso tellurico possono percepire correttamente il B.O.C.

\textbf{Ecologia}\\
Ambiente: Colline e foreste\\
Organizzazione: Solitario, coppia oppure branco (2d4)\\
\textbf{Categoria Tesoro}: I\\
\textbf{Descrizione}\\
Il Black Ops Cat meglio conosciuto con B.O.C. è un grande felino predatore, ovviamente di colore nero. Feroce, insaziabile, uccide per il gusto di cacciare. Agisce solitamente in branco ed è estremamente fedele al gruppo.

\mostro{Bugbear}
\noindent
\begin{description}[noitemsep, topsep=0pt, parsep=0pt, partopsep=0pt, leftmargin=0cm, labelwidth=2.2cm]
	\item[\textbf{Taglia/Tipo:}] Media umanoide (goblinoide), Arrogante, Impulsivo
	\item[\textbf{Caratt.:}] \resizebox{0.5\linewidth+1.8cm}{!}{For 2 Des 2 Cos 1 Int -1 Sag 0 Car -1}
	\item[\textbf{Punti Ferita:}] 33,  \textbf{Difesa:} 15,  \textbf{Iniziativa:} +2
	\item[\textbf{Movimento:}] 9 m
	\item[\textbf{Tiri Salvez.:}] \resizebox{0.5\linewidth+1.8cm}{!}{Tempra +3, Riflessi +3, Volontà +3}
	\item[\textbf{Comp.:}] Furtività +6, Sopravvivenza +2
	\item[\textbf{Sensi:}] Scurovisione 18 m
	\item[\textbf{Linguaggi:}] Comune, Goblin
	\item[\textbf{Sfida:}] 1 (200 PX)\smallskip
\end{description}

\emph{\textbf{Attacco di Sorpresa.}} Se il bugbear sorprende una creatura e la colpisce con un attacco durante il primo round di combattimento, il bersaglio subisce 7 (2d6) danni aggiuntivi dall'attacco.

\emph{\textbf{Bruto.}} Un'arma da mischia infligge un dado aggiuntivo di danno quando il bugbear colpisce con essa (già incluso nell'attacco).

\textbf{Azioni}

\emph{\textbf{Mazza Chiodata.} Attacco con arma da mischia}: +4 a colpire, portata 1 m, un bersaglio.

\emph{Colpisce:} 11 (2d8 + 2) danni perforanti.

\emph{\textbf{Giavellotto.} Attacco con arma da mischia o a Distanza}: +4 a colpire, portata 1 m o gittata 12m, un bersaglio.

\emph{Colpisce:} 9 (2d6 + 2) danni perforanti in mischia o 5 (1d6 + 2) danni perforanti a gittata.

\textbf{Ecologia}\\
Ambiente: Montagne temperate\\
Organizzazione: Solitario, coppia, gruppo (3-6) o banda da guerra (7-12 più 2 Guerrieri di 1° livello e 1 capitano di 3°-5° livello)\\
\textbf{Categoria Tesoro}: Equipaggiamento da PNG (Armatura di Cuoio, Scudo Leggero di Legno, Mazza chiodata, 3 Giavellotti, O)\\
\textbf{Descrizione}\\
Il bugbear è il più grande degli esponenti della razza Goblinoide, un bruto dai movimenti pesanti che supera di almeno una testa la maggior parte degli Umani. Sono solitari che preferiscono vivere ed uccidere da soli piuttosto che in tribù, sebbene non sia insolito trovare una piccola banda di Bugbear che collabora o vive con una tribù di Goblin od Hobgoblin fungendo da guardia d'élite o carnefici.

I bugbear non formano grandi insediamenti come i goblin o nazioni come gli hobgoblin; preferiscono qualcosa di più piccolo e caotico che li lasci liberi di fare quello che preferiscono (uccidere e torturare) a un livello più personale. Gli umani sono le prede preferite dei bugbear, e la maggior parte di essi annovera la carne umana come uno degli alimenti principali della propria dieta. Macabri trofei quali orecchie e dita sono decorazioni comuni tra i bugbear.

I bugbear, quando si rivolgono alla religione, prediligono le divinità dell'omicidio e della violenza, con i vari signori dei demoni tra i preferiti. Un tipico bugbear è alto 2,1 metro e pesa 200 kg.

\mostro{Bulette}
\noindent
\begin{description}[noitemsep, topsep=0pt, parsep=0pt, partopsep=0pt, leftmargin=0cm, labelwidth=2.2cm]
	\item[\textbf{Taglia/Tipo:}] Grande bestia, disallineato
	\item[\textbf{Caratt.:}] \resizebox{0.5\linewidth+1.8cm}{!}{For 4 Des 0 Cos 5 Int -4 Sag 0 Car -3}
	\item[\textbf{Punti Ferita:}] 110,  \textbf{Difesa:} 18,  \textbf{Iniziativa:} +0
	\item[\textbf{Movimento:}] 12 m, scavo 12 m
	\item[\textbf{Tiri Salvez.:}] \resizebox{0.5\linewidth+1.8cm}{!}{Tempra +10, Riflessi +5, Volontà +5}
	\item[\textbf{Comp.:}] Consapevolezza +6
	\item[\textbf{Sensi:}] Scurovisione 18 m, percezione tellurica 18 m
	\item[\textbf{Sfida:}] 5 (1800 PX)\smallskip
\end{description}

\emph{\textbf{Salto da Fermo.}} Un bulette può saltare in lungo fino a 9 metri e in alto fino a 5 m con o senza la rincorsa.

\textbf{Azioni}

\emph{\textbf{Morso.} Attacco con arma da mischia}: +7 a colpire, portata 1 m, un bersaglio.

\emph{Colpisce:} 30 (4d12 + 4) danni perforanti.

\emph{\textbf{Salto Letale.}} Se il bulette può saltare di almeno 3 metri come parte del suo movimento, può usare poi questa azione per atterrare in piedi in uno spazio che contiene una o più creature. Ciascuna di queste creature deve riuscire un Tiro Salvezza di Tempra o Riflessi DC 18 (a scelta del bersaglio) o venire gettata prona e subire 14 (3d6 + 4) danni contundenti più 14 (3d6 + 4) danni taglienti. Se il Tiro Salvezza riesce, la creatura subisce solo la metà dei danni, non è gettata prona, e viene spinta di 1 metro fuori dello spazio del bulette in uno spazio non occupato a scelta della creatura. Se non ci sono spazi non occupati a gittata, la creatura cade prona nello spazio del bulette.

\emph{\textbf{Fiuto del sangue.}} la bulette concentra la sua attenzione su una creatura che ha ferito, 1 Azione, fino alla fine del combattimento o finché la creatura non è totalmente guarita ha +2 al Tiro per Colpire.

\emph{\textbf{Arrabbiato:}} La bulette si ricarica delle ultime energie, recupera tre volte il suo GS in punti ferita. Costa 1 Azione.

\textbf{Ecologia}\\
Ambiente: Colline Temperate\\
Organizzazione: Solitario o coppia\\
\textbf{Categoria Tesoro}: Nessuno\\
\textbf{Descrizione}\\
Creazione di uno sconosciuto mago del passato, il bulette ora è diventato un feroce predatore di collina. Scavando rapidamente sotto il terreno, fende la superficie con la sua pinna dorsale lasciandosi dietro una scia caratteristica. Il bulette balza fuori, liberandosi da pietre e terriccio, per fare a pezzi la sua preda senza rimorsi, dando così origine al suo soprannome di \emph{squalo terrestre}.

I bulette sono noti per il pessimo carattere, e attaccano creature molto più grandi di loro senza alcuna paura. Bestie solitarie tranne per le occasionali coppie in fase riproduttiva, passano la maggior parte del tempo pattugliando i loro territori, che possono superare i 4 km2, cacciando e punendo gli intrusi con una furia in grado di scuotere i pendii delle colline.

I bulette sono macchine perfette per divorare e distruggere ossa, armature e anche oggetti magici con le loro possenti mascelle e l'acido ribollente del loro stomaco. In mancanza d'altro, un bulette potrebbe sgranocchiare oggetti comuni, ma per qualche ragione non mangia volontariamente carne di elfo, segno forse di un coinvolgimento della magia elfica nella loro creazione, o di nani, anche se può far strage dei membri di entrambe le razze. Gli Gnomi, invece, sono tra i cibi preferiti di queste bestie, e non ci sono Gnomi assennati che si avventurino nel territorio di un bulette a cuor leggero.

Il bulette è un combattente astuto, e sorprende i nemici con agilità impressionante. Una delle sue tattiche preferite è lanciarsi alla carica e balzare sulla preda attaccando con i suoi artigli affilati come rasoi. Si dice che la carne dietro la cresta dorsale della bestia sia particolarmente tenera, e che quanti vogliano o riescano ad attendere che la pinna venga sollevata nella concitazione del combattimento o dell'accoppiamento possano tentare di sferrare un colpo mortale in quel punto, anche se la maggior parte di quelli che hanno affrontato uno squalo terrestre concordano sul fatto che il miglior modo per vincere un combattimento con un bulette sia evitarlo del tutto.

%\addcontentsline{toc}{subsubsection}{C}
\pdfbookmark[3]{C}{C}

\mostro{Cavaliere Nero}
\noindent
\begin{description}[noitemsep, topsep=0pt, parsep=0pt, partopsep=0pt, leftmargin=0cm, labelwidth=2.2cm]
	\item[\textbf{Taglia/Tipo:}] Media non morto, Arrogante, Paziente
	\item[\textbf{Caratt.:}] \resizebox{0.5\linewidth+1.8cm}{!}{For 5 Des 1 Cos 5 Int 1 Sag 2 Car 3}
	\item[\textbf{Punti Ferita:}] 357,  \textbf{Difesa:} 37,  \textbf{Iniziativa:} +1
	\item[\textbf{Movimento:}] 9 metri
	\item[\textbf{Tiri Salvez.:}] \resizebox{0.5\linewidth+1.8cm}{!}{\resizebox{0.5\linewidth+1.8cm}{!}{Tempra +23, Riflessi +19, Volontà +20}}
	\item[\textbf{Comp.:}] Intimidire +12, Religione +8, Conoscenza Piani +8, Conoscenza Arcana +5
	\item[\textbf{Res. Danni:}] Freddo, Elettricità
	\item[\textbf{Imm. Danni:}] da Vuoto, Veleno; armi +1
	\item[\textbf{Immunità:}] affascinato, paralizzato, affaticato, spaventato, sanguinamento
	\item[\textbf{Sensi:}] Scurovisione 36 m
	\item[\textbf{Linguaggi:}] Comune, Abissale, Expiran
	\item[\textbf{Sfida:}] 18 (20000 PX)\smallskip
\end{description}

\emph{\textbf{Incantesimi.}} Il Cavaliere Nero ha CM 7. La sua caratteristica da incantatore è il Carisma. Il Cavaliere Nero conosce i seguenti incantesimi:

livello 1 (4 slot): \emph{\hyperlink{Comando}{Comando},  \hyperlink{Dardo arcano}{Dardo arcano}, Onda rovente, \hyperlink{Scudo}{Scudo}}

livello 2 (3 slot): \emph{\hyperlink{Blocca Persona}{Blocca Persona}, arma magica}

livello 3 (3 slot): \emph{\hyperlink{Controincantesimo}{Controincantesimo}, \hyperlink{Dissolvi Magie}{Dissolvi Magie}, \hyperlink{Palla di Fuoco}{Palla di Fuoco}}

livello 4 (3 slot): \emph{\hyperlink{Esilio}{Esilio}, Punizione marchiante (con 1 critico magico automatico, danno da Vuoto)}

\emph{\textbf{Natura Non Morta.}} Il Cavaliere Nero non ha bisogno di aria, cibo, bevande o sonno.

\emph{\textbf{Resistenza Leggendaria (1/Giorno).}} Se il Cavaliere Nero fallisce un Tiro Salvezza, può scegliere invece di riuscirvi.

\emph{\textbf{Resistenza allo Scacciare.}} Il Cavaliere Nero ha +1d6 ai Tiri Salvezza contro gli effetti che scacciano i non morti.

\textbf{Azioni}

\emph{\textbf{Multiattacco.} 3 attacchi con spada lunga +3}: +27 al colpire, portata 1 m, fino a tre creature differenti, oppure 1 colpo di spada con Corruzione

\emph{Colpisce:} 13 (1d10+5+3) danni da taglio + Colpo Fiammeggiante (danno da Vuoto)

\emph{Corruzione:} 15 (1d10+10) danni da taglio. L'obiettivo deve fare un Tiro Salvezza su Volontà DC 30 oppure perdere un 1/10 di un punto Tratto legato ad un Patrono buono se presente.

\textbf{Reazione: \emph{Attacco d'opportunità}}: il Cavaliere nero effettua un attacco ad una creatura che attraversi o esca dalla sua portata di 1 metro.

\textbf{Ecologia}\\
Ambiente: Qualsiasi\\
Organizzazione: Solitario\\
\textbf{Categoria Tesoro}: spada lunga +3 od armatura completa +3, il resto dell'equipaggiamento scompare con la morte del Cavaliere Nero.\\
\textbf{Descrizione}\\
Dannato fin nel profondo della sua anima il Cavaliere Nero è l'antitesi del cavaliere di Sumkjr, anzi spesso nasce dalla corruzione di un cavaliere di Sumkjr. Avversario temibile, furbo, tattico, adora comportarsi e ragionare malignamente, come una persona ancora viva. La sua tattica è di lanciare la \hyperlink{Palla di Fuoco}{Palla di Fuoco} il prima possibile per poi consumare la vittima con Punizione marchiante.

\mostro{Centauro}
\noindent
\begin{description}[noitemsep, topsep=0pt, parsep=0pt, partopsep=0pt, leftmargin=0cm, labelwidth=2.2cm]
	\item[\textbf{Taglia/Tipo:}] Grande mostruosità, buono
	\item[\textbf{Caratt.:}] \resizebox{0.5\linewidth+1.8cm}{!}{For 4 Des 2 Cos 2 Int -1 Sag 1 Car 0}
	\item[\textbf{Punti Ferita:}] 51,  \textbf{Difesa:} 16,  \textbf{Iniziativa:} +2
	\item[\textbf{Movimento:}] 15 m
	\item[\textbf{Tiri Salvez.:}] \resizebox{0.5\linewidth+1.8cm}{!}{Tempra +4, Riflessi +4, Volontà +3}
	\item[\textbf{Comp.:}] Atletica +6, Consapevolezza +3, Sopravvivenza +3
	\item[\textbf{Linguaggi:}] Elfico, Silvano
	\item[\textbf{Sfida:}] 2 (450 PX)\smallskip
\end{description}

\emph{\textbf{Carica.}} Se il centauro si muove di almeno 9 metri diretto verso il bersaglio e colpisce con un attacco di picca durante lo stesso round, il bersaglio subisce 10 (3d6) danni perforanti aggiuntivi.

\textbf{Azioni}

\emph{\textbf{Multiattacco.}} Il centauro effettua due attacchi: uno con la picca e uno con gli zoccoli o due con l'arco lungo.

\emph{\textbf{Picca.} Attacco con arma da mischia}: +5 a colpire, portata 3 m, un bersaglio.

\emph{Colpisce:} 9 (1d10 + 4) danni perforanti.

\emph{\textbf{Zoccoli.} Attacco con arma da mischia}: +6 a colpire, portata 1 m, un bersaglio.

\emph{Colpisce:} 11 (2d6 + 4) danni contundenti.

\emph{\textbf{Arco Lungo.} Attacco con arma a Distanza}: +4 a colpire, gittata 45m, un bersaglio.

\emph{Colpisce:} 6 (1d8 + 2) danni perforanti.

\textbf{Ecologia}\\
Ambiente: Pianure e foreste temperate\\
Organizzazione: Solitario, coppia, banda (3-10), tribù (11-30 più 3 cacciatori di 3° livello e 1 capo di 6° livello)\\
\textbf{Categoria Tesoro}: B\\
\textbf{Descrizione}\\
Leggendari cacciatori e abili guerrieri, i centauri sono in parte uomini e in parte cavalli. Generalmente collocata ai margini della civilizzazione, questa stoica popolazione varia enormemente come aspetto: di solito il colore della pelle è molto abbronzato ma simile a quello degli umani delle regioni limitrofe, mentre la parte inferiore del corpo ha le tonalità degli equini locali. Hanno capelli e occhi di colore scuro e i tratti del volto piuttosto marcati, mentre la loro stazza totale dipende dalla taglia del cavallo di cui hanno la parte inferiore del corpo. Quindi, anche se un centauro medio è alto in piedi 2,1 metro e pesa più di 1000 kg, esistono molteplici varianti regionali, dagli esili corridori delle pianure ai massicci cacciatori di montagna.

I centauri vivono in media circa 60 anni. Distanti dalle altre razze e in conflitto con gli altri della loro specie, i centauri sono una razza antica che lentamente comincia ad accettare il mondo moderno. Anche se la maggioranza dei centauri vive ancora in tribù vagando per vaste pianure o ai margini di mistiche foreste, alcuni hanno abbandonato i modi isolazionisti dei loro antenati per stabilirsi in città cosmopolite. Spesso questi spiriti liberi sono considerati dei reietti e vengono disprezzati dalle loro tribù, e pertanto la decisione di abbandonarle è una scelta pesante. In alcuni casi, comunque, intere tribù guidate da capi progressisti hanno cominciato a commerciare o stringere alleanze con altre comunità di umanoidi, specie Elfi, a volte Gnomi, e più raramente Umani o Nani. Molte razze rimangono caute nei confronti dei centauri, però, per lo più a causa di leggende che li ritraggono come creature territoriali e feroci e dei periodici scontri violenti che essi hanno con i coloni testardi e i paesi in via di espansione.

La leggenda vuole che i Centauri dovessero esplodere come tutti gli equini, per volere di Calicante. Ljust inorridita da tanta morte intercesse su Calicante perché lasciasse stare queste creature. Questo salvataggio ha portato molti tribù di Centauri ad essere devoti della Signora della Luce, anche se altri hanno preferito invece dedicare il loro culto a Calicante nella speranza che non li uccida tutti in una notte.

\mostro{Chimera}
\noindent
\begin{description}[noitemsep, topsep=0pt, parsep=0pt, partopsep=0pt, leftmargin=0cm, labelwidth=2.2cm]
	\item[\textbf{Taglia/Tipo:}] Grande mostruosità, Arrogante, Vanitoso
	\item[\textbf{Caratt.:}] \resizebox{0.5\linewidth+1.8cm}{!}{For 4 Des 0 Cos 4 Int -4 Sag 2 Car 0}
	\item[\textbf{Punti Ferita:}] 127,  \textbf{Difesa:} 20,  \textbf{Iniziativa:} +0
	\item[\textbf{Movimento:}] 9 m, volo 18 m
	\item[\textbf{Tiri Salvez.:}] \resizebox{0.5\linewidth+1.8cm}{!}{Tempra +10, Riflessi +6, Volontà +8}
	\item[\textbf{Comp.:}] Consapevolezza +8
	\item[\textbf{Sensi:}] Scurovisione 18 m
	\item[\textbf{Linguaggi:}] comprende il Draconico ma non può parlare
	\item[\textbf{Sfida:}] 6 (2300 PX)\smallskip
\end{description}

\textbf{Azioni}

\emph{\textbf{Multiattacco.}} La chimera effettua tre attacchi: uno con il morso, uno con le corna e uno con gli artigli. Quando il soffio infuocato è disponibile, può usare il soffio al posto del morso o delle corna.

\emph{\textbf{Artigli.} Attacco con arma da mischia}: +7 a colpire, portata 1 m, un bersaglio.

\emph{Colpisce:} 11 (2d6 + 4) danni taglienti, 1 danno da Sanguinamento.

\emph{\textbf{Corna.} Attacco con arma da mischia}: +7 a colpire, portata 1 m, un bersaglio.

\emph{Colpisce:} 10 (1d12 + 4) danni contundenti.

\emph{\textbf{Morso.} Attacco con arma da mischia}: +6 a colpire, portata 1 m, un bersaglio.

\emph{Colpisce:} 11 (2d6 + 4) danni perforanti.

\emph{\textbf{Soffio Infuocato (Ricarica 5-6).}} La testa di drago esala fuoco in un cono di 5 metri. Ogni creatura in quell'area deve effettuare un Tiro Salvezza di Riflessi DC 21 e subire 31 (7d8) danni da fuoco se fallisce il Tiro Salvezza, o la metà di questi danni se lo riesce.

\textbf{Reazione: \emph{Attacco d'opportunità}}: la Chimera effettua un attacco ad una creatura che attraversi o esca dalla sua portata di 1 metro.

\emph{\textbf{Arrabbiato:}} la Chimera brilla di energia. Ricarica il Soffio Infuocato. Costa 1 Azione.

\textbf{Ecologia}\\
Ambiente: Colline Temperate\\
Organizzazione: Solitario, coppia, branco (3-6) o stormo (7-12)\\
\textbf{Categoria Tesoro}: F\\
\textbf{Descrizione}\\
Le chimere sono mostruose creature nate dal male primordiale. Odiose e fameliche, cacciano sia a terra che in aria. La testa di drago di una chimera può essere di qualunque tipo di drago malvagio, con il soffio corrispondente e le ali generalmente dotate delle stesse scaglie della testa. Le chimere parlano con tre voci che si sovrappongono, ma lo fanno raramente, tipicamente solo per adulare una creatura più potente. Una chimera è alta al garrese 1 metro, raggiungendo la lunghezza di 4 metri e il peso di 350 kg.\\
Le chimere preferiscono la carne, ma possono sopravvivere di vegetali se necessario (anche se quando sono costrette a farlo il loro umore peggiora ulteriormente). Il fatto che volino significa che possono scegliere con attenzione le loro prede, e generalmente cacciano in vaste aree cercando quelle facili. Sono troppo stupide e belligeranti per acquisire seguaci, anche se a volte una tribù di coboldi può far loro delle offerte. Al contrario, sono abbastanza intelligenti e caparbie da essere mediocri animali domestici, e solo una creatura molto più potente di loro può riuscire a sottometterle. Possono formare collaborazioni paritarie con umanoidi rispettosi o creature simili, e acconsentono anche ad essere usate come cavalcature. Un branco di chimere ha una gerarchia simile a quella dei leoni, con un maschio dominante che comanda il gruppo e la maggior parte delle cacce svolte dalle femmine. Una chimera solitaria può essere un giovane maschio solitario o una femmina con i cuccioli nelle vicinanze.

\mostro{Chuul}
\noindent
\begin{description}[noitemsep, topsep=0pt, parsep=0pt, partopsep=0pt, leftmargin=0cm, labelwidth=2.2cm]
	\item[\textbf{Taglia/Tipo:}] Grande aberrazione, malvagio
	\item[\textbf{Caratt.:}] \resizebox{0.5\linewidth+1.8cm}{!}{For 4 Des 0 Cos 3 Int -3 Sag 0 Car -3}
	\item[\textbf{Punti Ferita:}] 89,  \textbf{Difesa:} 17,  \textbf{Iniziativa:} +0
	\item[\textbf{Movimento:}] 9 m, nuoto 9 m
	\item[\textbf{Tiri Salvez.:}] \resizebox{0.5\linewidth+1.8cm}{!}{Tempra +7, Riflessi +4, Volontà +4}
	\item[\textbf{Comp.:}] Consapevolezza +4
	\item[\textbf{Imm. Danni:}] Veleno
	\item[\textbf{Sensi:}] Scurovisione 18 m
	\item[\textbf{Linguaggi:}] comprende la Linguaggio delle Profondità ma non può parlare
	\item[\textbf{Sfida:}] 4 (1100 PX)\smallskip
\end{description}

\emph{\textbf{Anfibio.}} Il chuul può respirare aria e acqua.

\emph{\textbf{Senso della Magia.}} Il chuul percepisce la magia entro 36 metri da sé. Questo tratto funziona come l'incantesimo \emph{individuazione} \emph{del magico} ma di per sé non è magico.

\textbf{Azioni}

\emph{\textbf{Multiattacco.}} Il chuul effettua due attacchi con le chele. Se il chuul sta afferrando una creatura, può anche usare i suoi tentacoli una volta.

\emph{\textbf{Chele.} Attacco con arma da mischia}: +7 a colpire, portata 3 m, un bersaglio.

\emph{Colpisce:} 11 (2d6 + 4) danni contundenti. Un bersaglio è afferrato (DC 14 per fuggire) se è di taglia Grande o inferiore e il chuul non sta già afferrando altre due creature.

\emph{\textbf{Tentacoli.}} Una creatura afferrata dal chuul deve riuscire un Tiro Salvezza di Tempra DC 16 o restare avvelenata per 1 minuto. Fino al termine dell'avvelenamento, il bersaglio è paralizzato. Il bersaglio può ripetere il Tiro Salvezza al termine di ciascun suo round, terminando l'effetto per sé in caso di successo.

\textbf{Reazione: \emph{Attacco d'opportunità}}: il chuul effettua un attacco ad una creatura che attraversi o esca dalla sua portata di 3 metri.

\textbf{Ecologia}\\
Ambiente: Paludi Temperate\\
Organizzazione: Solitario, coppia o branco (3-6)\\
\textbf{Categoria Tesoro}: U (B)\\
\textbf{Descrizione}\\
I chuul sono predatori corazzati simili ai crostacei, sempre in agguato sotto la superficie degli stagni e dei pantani poco profondi, che escono dal loro nascondiglio per afferrare le loro prede con le loro chele e poi paralizzarle con i tentacoli della bocca prima di mangiarle vive.

I chuul sono eccellenti nuotatori, ma preferiscono attaccare le creature terrestri o abituate ad acque poco profonde. Una volta afferrate le loro vittime, i chuul spesso le trascinano nell'acqua profonda. I lucertoloidi sono le prede preferite dei chuul, anche se le pallide specie di chuul che vivono nei sotterranei preferiscono morlock, nani oscuri, incauti elfi e altri sfortunati che si avvicinano troppo ai loro corsi d'acqua sotterranei, ad eccezione dei trogloditi il cui sapore i chuul trovano particolarmente disgustoso.

I chuul sono sorprendentemente intelligenti e molti si impegnano in inutili speculazioni sulle loro origini e motivazioni. Parlano un cinguettante e gorgogliante dialetto del Comune, ma pochi di essi sono inclini a chiacchierare con quanti non siano della loro razza, e se esiste una società chuul al di fuori del frenetico periodo degli amori, nessuno lo ha ancora scoperto. Al contrario, le menti dei chuul sembrano dedite solo alla ricerca del luogo perfetto in cui tendere un'imboscata per attaccare altre creature intelligenti e a come decorare le loro elaborate tane con trofei delle loro vittime. Anche se i chuul sembrano non interessati all'utilizzo di utensili, hanno un bisogno compulsivo di collezionare quelli delle loro vittime. Un tipico chuul è alto 2,3 metri e pesa 325 kg.

\mostro{Coboldo}
\noindent
\begin{description}[noitemsep, topsep=0pt, parsep=0pt, partopsep=0pt, leftmargin=0cm, labelwidth=2.2cm]
	\item[\textbf{Taglia/Tipo:}] Piccola umanoide (coboldo), malvagio
	\item[\textbf{Caratt.:}] \resizebox{0.5\linewidth+1.8cm}{!}{For -2 Des 2 Cos -1 Int -1 Sag -2 Car -1}
	\item[\textbf{Punti Ferita:}] 17,  \textbf{Difesa:} 14,  \textbf{Iniziativa:} +2
	\item[\textbf{Movimento:}] 9 m
	\item[\textbf{Tiri Salvez.:}] \resizebox{0.5\linewidth+1.8cm}{!}{Tempra +3, Riflessi +3, Volontà +3}
	\item[\textbf{Sensi:}] Scurovisione 18 m
	\item[\textbf{Linguaggi:}] Comune, Draconico
	\item[\textbf{Sfida:}] 1/8 (25 PX)\smallskip
\end{description}

\emph{\textbf{Sensibilità alla Luce}}. Mentre è alla luce del sole, il coboldo ha -1d6 ai tiri per colpire, oltre che alle prove di Consapevolezza basate sulla vista.

\emph{\textbf{Tattiche di Branco.}} Il coboldo ha +1d6 ai tiri per colpire contro una creatura se almeno uno degli alleati del coboldo si trova entro 1 metro dalla creatura e quell'alleato non è inabile.

\textbf{Azioni}

\emph{\textbf{Pugnale.} Attacco con arma da mischia}: +3 a colpire, portata 1 m, un bersaglio.

\emph{Colpisce:} 4 (1d4 + 2) danni perforanti.

\emph{\textbf{Fionda.} Attacco con arma a distanza}: +3 a colpire, gittata 9m, un bersaglio.

\emph{Colpisce:} 4 (1d4 + 2) danni contundenti.

\textbf{Ecologia}\\
Ambiente: Foreste temperate o sotterranei\\
Organizzazione: solitario, gruppo (2-4), nido (5-30 più un ugual numero di non combattenti, 1 sergente di 3° livello ogni 20 adulti e 1 capo di 4°-6° livello) o tribù (31-300 più di 35\% di non combattenti, 1 sergente di 3° livello ogni 20 adulti, 2 tenenti di 4° livello, 1 capo di 6°-8° livello e 5-16 Ratti Crudeli)\\
\textbf{Categoria Tesoro}: Equipaggiamento da PNG (Armatura di Cuoio, Lancia, Fionda, altro tesoro), 2d6 monete d'argento\\
\textbf{Descrizione}\\
I coboldi sono creature dell'oscurità, che si incontrano più facilmente in enormi dedali sotterranei o negli angoli bui delle foreste dove il sole non batte mai. A causa della somiglianza fisica i coboldi si proclamano a gran voce discendenti della stirpe draconica e destinati a governare la terra sotto l'ala dei loro grandi cugini divini, ma la maggior parte dei draghi li considera poco più che insetti fastidiosi. Codardi ed intriganti, non lottano mai apertamente se possono evitarlo tendendo invece imboscate e trappole, rintanandosi nelle loro labirintiche costruzioni dietro una coltre di rozzi ma ingegnosi trabocchetti, o rovesciandosi sul nemico in vaste orde ululanti.

La tonalità dei coboldi varia anche tra i fratelli della stessa covata, spaziando tra i colori dei draghi di Tàhil, con una predominanza del rosso e porpora, e più di rado bianco, verde, blu e nero.

I coboldi hanno un debole per l'argento ma essendo pessimi minatori preferiscono predare dalle monete d'argento gli avventurieri e ne mangiano come fossero biscotti al burro. I coboldi possono digerire l'argento piuttosto velocemente e più mangiano più le loro squame sono luminose ed i coboldi sembrano sani.

\mostro{Cockatrice}
\begin{description}[noitemsep, topsep=0pt, parsep=0pt, partopsep=0pt, leftmargin=0cm, labelwidth=2.2cm]
	\item[\textbf{Taglia/Tipo:}] Piccola mostruosità, disallineato
	\item[\textbf{Caratt.:}] \resizebox{0.5\linewidth+1.8cm}{!}{For -2 Des 1 Cos 1 Int -4 Sag 1 Car -3}
	\item[\textbf{Punti Ferita:}] 24,  \textbf{Difesa:} 13,  \textbf{Iniziativa:} +1
	\item[\textbf{Movimento:}] 6 m, volo 12 m
	\item[\textbf{Tiri Salvez.:}] \resizebox{0.5\linewidth+1.8cm}{!}{Tempra +3, Riflessi +3, Volontà +3}
	\item[\textbf{Sensi:}] Scurovisione 18 m
	\item[\textbf{Sfida:}] 1/2 (100 PX)\smallskip
\end{description}

\textbf{Azioni}

\emph{\textbf{Morso.} Attacco con arma da mischia}: +4 a colpire, portata 1 m, una creatura.

\emph{Colpisce:} 3 (1d4 + 1) danni perforanti, e il bersaglio deve riuscire un Tiro Salvezza di Tempra DC 11 o essere Rallentato 1/1r per via della progressiva pietrificazione. Se successivi morsi portano la creatura a non avere più Azioni la creatura è pietrificata per 24 ore.

\textbf{Ecologia}\\
Ambiente: Pianure temperate\\
Organizzazione: Solitario, coppia, squadriglia (3-5) o stormo (6-12)\\
\textbf{Categoria Tesoro}: D\\
\textbf{Descrizione}\\
Stupide, malevole e repellenti, le cockatrici sono evitate dalle altre creature per la loro capacità di trasformare la carne in pietra. I maschi si distinguono solo per barbigli e creste. Una tipica cockatrice è alta poco più di 60 centimetri e pesa 2,5 kg.

Anche se la loro dieta consiste principalmente di semi e insetti pietrificati (che nello stomaco della creatura fungono sia da gastroliti che da nutrimento), le cockatrici difendono ferocemente il loro territorio da tutto ciò che ritengono una minaccia, e i vagabondaggi dei maschi raminghi in cerca di nuovi luoghi dove costruire tane a volte li portano ad involontari contatti con gli umani, con risultati devastanti.

La strana capacità della cockatrice di trasformare le altre creature in pietra è la sua miglior difesa e la sua tana è invariabilmente piena di resti dei nemici pietrificati. Per ironia della sorte, tuttavia, donnole e furetti, le creature che più probabilmente finiscono nei nidi delle cockatrici per mangiarne le uova, sembrano completamente immuni a questo effetto. Per ragioni sconosciute, le cockatrici sono sia terrorizzate che furiose con i galli comuni e c'è la stessa probabilità che fuggano o attacchino quando avviene un confronto.

\mostro{Couatl}
\noindent
\begin{description}[noitemsep, topsep=0pt, parsep=0pt, partopsep=0pt, leftmargin=0cm, labelwidth=2.2cm]
	\item[\textbf{Taglia/Tipo:}] Media celestiale, buono
	\item[\textbf{Caratt.:}] \resizebox{0.5\linewidth+1.8cm}{!}{For 3 Des 5 Cos 3 Int 4 Sag 5 Car 4}
	\item[\textbf{Punti Ferita:}] 89,  \textbf{Difesa:} 22,  \textbf{Iniziativa:} +5
	\item[\textbf{Movimento:}] 9 m, volo 9 m
	\item[\textbf{Tiri Salvez.:}] \resizebox{0.5\linewidth+1.8cm}{!}{Tempra +7, Riflessi +9, Volontà +9}
	\item[\textbf{Res. Danni:}] da Luce
	\item[\textbf{Imm. Danni:}] da arma non magica
	\item[\textbf{Sensi:}] visione del vero 36 m
	\item[\textbf{Linguaggi:}] tutte, telepatia 36 m
	\item[\textbf{Sfida:}] 4 (1100 PX)\smallskip
\end{description}

\emph{\textbf{Armi Magiche.}} Gli attacchi con armi del couatl sono magici.

\emph{\textbf{Incantesimi Innati.}} La caratteristica da incantatore innato del couatl è il Carisma. Il couatl può lanciare questi incantesimi in maniera innata, usando solo componenti verbali:

A volontà: \emph{\hyperlink{Conoscere i Tratti}{Conoscere i Tratti}, \hyperlink{Individuazione del Magico}{Individuazione del Magico}, \hyperlink{Individuazione dei Pensieri}{Individuazione dei Pensieri}}

3/giorno ciascuno: \emph{\hyperlink{Benedizione}{Benedizione}, \hyperlink{Creare cibo e acqua}{Creare cibo e acqua}, \hyperlink{Cura Ferite}{Cura Ferite}, protezione dai veleni, \hyperlink{Ristorare Inferiore}{Ristorare Inferiore}, \hyperlink{Santuario}{Santuario}, \hyperlink{Scudo}{Scudo}} 1/giorno ciascuno: \emph{\hyperlink{Ristorare Superiore}{Ristorare Superiore}, \hyperlink{Scrutare}{Scrutare}, \hyperlink{Sogno}{Sogno}}

\emph{\textbf{Mente Protetta.}} Il couatl è immune allo scrutare e qualsiasi effetto che percepisca le sue emozioni, legga i suoi pensieri o individui la sua posizione.

\textbf{Azioni}

\emph{\textbf{Morso.} Attacco con arma da mischia}: +8 a colpire, portata 1 m, una creatura.

\emph{Colpisce:} 8 (1d6 + 5) danni perforanti, e il bersaglio deve riuscire un Tiro Salvezza di Tempra DC 16 o restare avvelenato per 24 ore. Fino al termine dell'avvelenamento, il bersaglio è privo di sensi. Un'altra creatura può effettuare un'Azione per risvegliare il bersaglio.

\emph{\textbf{Stritolare.} Attacco con arma da mischia}: +7 a colpire, portata 3 m, una creatura di taglia Media o inferiore.

\emph{Colpisce:} 10 (2d6 + 3) danni contundenti, e il bersaglio è afferrato (DC 15 per fuggire).

\emph{\textbf{Mutare Forma.}} Il couatl può trasformarsi magicamente in un umanoide o bestia il cui grado di sfida sia pari o inferiore al proprio, o tornare alla sua vera forma. Alla morte ritorna alla sua vera forma. Qualsiasi equipaggiamento stia indossando o trasportando viene assorbito o trasportato nella nuova forma (a scelta del couatl).

Nella nuova forma, il couatl mantiene le sue statistiche di gioco e la facoltà di parlare, ma la sua Difesa, metodi di movimento, Forza, Destrezza e altre azioni vengono rimpiazzati da quelli della nuova forma, e ottiene qualsiasi statistica o capacità (Azioni aggiuntive e azioni da tana) possedute dalla sua nuova forma e non dalla sua originale. Se la nuova forma ha un attacco di morso, il couatl può usare il proprio morso nella nuova forma.

\textbf{Ecologia}\\
Ambiente: foreste calde\\
Organizzazione: Solitario, coppia o stormo (3-6)\\
\textbf{Categoria Tesoro}: I\\
\textbf{Descrizione}\\
Rispettati ed ammirati per la loro saggezza e bellezza, cercano di portare i mortali sulla retta via e usano i loro poteri per combattere il male, specie quelli noti per viaggiare tra i piani. Un couatl è lungo circa 3,6 metri, con un'apertura alare di circa 5 metri e pesa 900 kg.

Preferiscono gli stessi alimenti dei veri serpenti, come mammiferi e uccelli, anche se è noto che divorano gli umanoidi malvagi. Poiché preferiscono passare il tempo a perseguire i loro intenti anziché cacciare, apprezzano le offerte di cibo, in particolare piccoli cinghiali e volatili. Un couatl talvolta mostra il suo apprezzamento a un avventuriero o a un gruppo che gli ha reso un servizio donandogli 1d4 delle sue brillanti piume colorate.

\mostro{Cumulo Strisciante}
\noindent
\begin{description}[noitemsep, topsep=0pt, parsep=0pt, partopsep=0pt, leftmargin=0cm, labelwidth=2.2cm]
	\item[\textbf{Taglia/Tipo:}] Grande pianta, disallineato
	\item[\textbf{Caratt.:}] \resizebox{0.5\linewidth+1.8cm}{!}{For 4 Des -1 Cos 3 Int -3 Sag 0 Car -3}
	\item[\textbf{Punti Ferita:}] 108,  \textbf{Difesa:} 17,  \textbf{Iniziativa:} -1
	\item[\textbf{Movimento:}] 6 m, nuoto 6 m
	\item[\textbf{Tiri Salvez.:}] \resizebox{0.5\linewidth+1.8cm}{!}{Tempra +8, Riflessi +4, Volontà +5}
	\item[\textbf{Comp.:}] Furtività +2
	\item[\textbf{Res. Danni:}] Freddo, Fuoco
	\item[\textbf{Imm. Danni:}] Elettricità
	\item[\textbf{Immunità:}] accecato, assordato, affaticato
	\item[\textbf{Sensi:}] Vista Cieca 18 m (cieco oltre questo raggio)
	\item[\textbf{Sfida:}] 5 (1800 PX)\smallskip
\end{description}

\emph{\textbf{Assorbimento dei Fulmini.}} Ogni qual volta il cumulo strisciante subisce danni da elettricità, non subisce danni e recupera un numero di Punti Ferita pari al danno da elettricità inferto.

\textbf{Azioni}

\emph{\textbf{Multiattacco.}} Il cumulo strisciante effettua due attacchi di schianto. Se entrambi gli attacchi colpiscono una creatura di taglia Media o inferiore, il bersaglio è afferrato (DC 14 per fuggire) e il cumulo strisciante usa Avvolgere su di esso.

\emph{\textbf{Schianto.} Attacco con arma da mischia}: +7 a colpire, portata 1 m, un bersaglio.

\emph{Colpisce:} 13 (2d8 + 4) danni contundenti.

\emph{\textbf{Avvolgere.}} Il cumulo strisciante avvolge una creatura di taglia Media o inferiore che ha afferrato. Il bersaglio avvolto è accecato e impossibilitato a respirare, e deve riuscire un Tiro Salvezza di Tempra DC 17 all'inizio di ciascun round del tumulo o subire 13 (2d8 + 4) danni contundenti. Se il cumulo si muove, il bersaglio avvolto si muove con esso. Il cumulo può avvolgere solo una creatura alla volta.

\emph{\textbf{Arrabbiato:}} Il Cumulo strisciante rilascia un'onda di elettricità. Tutte le creature entro 3 metri subiscono 3d6 di danno da elettricità. Costa 2 Azioni.

\textbf{Ecologia}\\
Ambiente: Foreste o Paludi Temperate\\
Organizzazione: Solitario\\
\textbf{Categoria Tesoro}: A\\
\textbf{Descrizione}\\
I cumuli striscianti, chiamati anche soltanto striscianti, sembrano masse vegetali in decomposizione. Sono piante carnivore intelligenti, con un debole per la carne elfica. Il cervello e gli organi sensoriali si trovano nella parte superiore del corpo. Di solito i cumuli striscianti hanno una circonferenza di 2,3 metri e sono alti da 1,8 a 2,7 metri. Pesano circa 1.900 kg.

I cumuli striscianti sono strane creature, più simili a un groviglio di rampicanti parassiti che ad una singola pianta dotata di radici. Sono onnivori, capaci di trarre sostentamento da qualsiasi cosa, avvinghiandosi agli alberi per succhiarne la linfa, inserendo le radici nel terreno per assorbire nutrienti semplici o consumando la carne e le ossa dalle prede.

I cumuli striscianti sono incredibilmente furtivi nel loro ambiente naturale. Si confondono con il terreno circostante e possono attendere immobili per giorni l'arrivo di una potenziale preda. Possono essere praticamente ovunque ed attaccare in qualsiasi momento senza alcun preavviso e senza curarsi che ci siano o meno sopravvissuti, fintanto che hanno da mangiare.

Di solito i cumuli striscianti conducono un'esistenza nomade e solitaria in profonde foreste e fetide paludi ma possono essere trovati anche sottoterra, in mezzo a boschetti di funghi. Voci preoccupanti parlano di gruppi di cumuli striscianti che si radunano intorno a grandi tumuli nelle profondità di giungle e paludi, spesso durante violente tempeste di fulmini. Il motivo di questo comportamento è sconosciuto, e molti saggi si chiedono se dietro non ci sia uno scopo oscuro ed alieno.

%\addcontentsline{toc}{subsubsection}{D}
\pdfbookmark[3]{D}{D}

\mostro{Balor}
\noindent
\begin{description}[noitemsep, topsep=0pt, parsep=0pt, partopsep=0pt, leftmargin=0cm, labelwidth=2.2cm]
	\item[\textbf{Taglia/Tipo:}] Enorme demone, malvagio
	\item[\textbf{Caratt.:}] \resizebox{0.5\linewidth+1.8cm}{!}{For 8 Des 2 Cos 6 Int 5 Sag 3 Car 6}
	\item[\textbf{Punti Ferita:}] 379,  \textbf{Difesa:} 39,  \textbf{Iniziativa:} +5
	\item[\textbf{Movimento:}] 12 m, volo 24 m
	\item[\textbf{Tiri Salvez.:}] \resizebox{0.5\linewidth+1.8cm}{!}{\resizebox{0.5\linewidth+1.8cm}{!}{Tempra +25, Riflessi +21, Volontà +22}}
	\item[\textbf{Res. Danni:}] Freddo, Elettricità;
	\item[\textbf{Imm. Danni:}] Fuoco, Veleno, armi +1
	\item[\textbf{Vulnerabilità:}] ferro freddo, Luce
	\item[\textbf{Sensi:}] visione del vero 36 m
	\item[\textbf{Linguaggi:}] Abissale, telepatia 36 m
	\item[\textbf{Sfida:}] 19 (22000 PX)\smallskip
\end{description}

\emph{\textbf{Armi Magiche.}} Gli attacchi con arma del demone sono magici.

\emph{\textbf{Aura di Fuoco.}} All'inizio di ciascun round del demone, ciascuna creatura entro 1 metro da lui subisce 10 (3d6) danni da fuoco, e gli oggetti infiammabili che si trovano nell'aura e che non sono indossati o trasportati prendono fuoco. Una creatura che entri a contatto con il demone o lo colpisca con un attacco da mischia mentre si trova entro 1 metro da esso subisce 10 (3d6) danni da fuoco.

\emph{\textbf{Resistenza alla Magia.}} Il demone ha +1d6 ai Tiri Salvezza contro incantesimi e altri effetti magici.

\emph{\textbf{Spasmo Mortale.}} Quando il demone muore, esplode; ciascuna creatura entro 9 metri da esso deve effettuare un Tiro Salvezza di Riflessi DC 31, subendo 70 (20d6) danni da fuoco se fallisce il Tiro Salvezza, o la metà di questi danni se lo riesce. L'esplosione appicca il fuoco agli oggetti infiammabili che non sono indossati o trasportati, e distrugge le armi del demone.

\textbf{Azioni}

\emph{\textbf{Multiattacco.}} Il demone effettua due attacchi: uno con la spada lunga e uno con la frusta.

\emph{\textbf{Frusta.} Attacco con arma da mischia}: +14 a colpire, portata 9 m, un bersaglio.

\emph{Colpisce:} 15 (2d6 + 8) danni taglienti più 10 (3d6) danni da fuoco, e il bersaglio deve riuscire un Tiro Salvezza di Tempra DC 32 o venire trascinato 7 metri verso il demone.

\emph{\textbf{Spada Lunga.} Attacco con arma da mischia}: +14 a colpire, portata 3 m, un bersaglio.

\emph{Colpisce:} 21 (3d8 + 8) danni taglienti più 13 (3d8) danni da elettricità.

\textbf{Reazione: \emph{Attacco d'opportunità}}: il demone effettua un attacco ad una creatura che attraversi o esca dalla sua portata di 6 metri.

\emph{\textbf{Teletrasporto.}} Il demone si teletrasporta magicamente, insieme a tutto l'equipaggiamento che indossa o trasporta, in uno spazio non occupato e che può vedere entro 36 metri.

\textbf{Ecologia}\\
Ambiente: Qualsiasi (Abisso)\\
Organizzazione: Solitario o banda di guerra (1 Balor e 2-5 Glabrezu)\\
\textbf{Categoria Tesoro}: Standard (Spada Lunga Sacrilega+1, Frusta Infuocata+1, R)\\
\textbf{Descrizione}\\
Quando la gente sussurra terrificanti racconti di creature demoniache, immagina per lo più un'imponente figura di fuoco e carne, un incubo cornuto armato di frusta e spada fiammeggianti, che vola nella notte in cerca delle sue prede. Il demone che queste persone temono è il Balor, e questa paura è pienamente giustificata, dal momento che pochi demoni possono eguagliare il possente Balor in forza o in brutalità.

Nell'Abisso, i Balor sono per lo più al servizio dei signori dei demoni, in qualità di generali o capitani (quando non si tratti di balor estremamente potenti, noti come signori dei balor). Un balor solitamente comanda vaste legioni di demoni e, sebbene spesso consenta a questi servi bramosi e sbavanti di combattere le sue battaglie, è tutt'altro che un codardo. Se si presenta l'opportunità di unirsi ad uno scontro, sono pochi i balor che scelgono di trattenersi.

Un Balor è alto 4,2 metri e pesa 2.250 kg. Solo le anime mortali più crudeli possono alimentare la creazione di un balor: a differenza degli altri demoni, spesso occorrono numerose anime di potenti malvagi per far nascere un nuovo balor.

\mostro{Demogorgone}
\noindent
\begin{description}[noitemsep, topsep=0pt, parsep=0pt, partopsep=0pt, leftmargin=0cm, labelwidth=2.2cm]
	\item[\textbf{Taglia/Tipo:}] Enorme principe demone, malvagio
	\item[\textbf{Caratt.:}] \resizebox{0.5\linewidth+1.8cm}{!}{For 9 Des 2 Cos 8 Int 5 Sag 3 Car 7}
	\item[\textbf{Punti Ferita:}] 524,  \textbf{Difesa:} 48,  \textbf{Iniziativa:} +5
	\item[\textbf{Movimento:}] 15 metri, nuotare 9m
	\item[\textbf{Tiri Salvez.:}] \resizebox{0.5\linewidth+1.8cm}{!}{\resizebox{0.5\linewidth+1.8cm}{!}{Tempra +34, Riflessi +28, Volontà +29}}
	\item[\textbf{Comp.:}] tutte +15
	\item[\textbf{Res. Danni:}] Freddo, Elettricità, Fuoco
	\item[\textbf{Imm. Danni:}] da Vuoto, Veleno; armi +2
	\item[\textbf{Immunità:}] affascinato, paralizzato, affaticato, spaventato
	\item[\textbf{Vulnerabilità:}] ferro freddo, Luce
	\item[\textbf{Sensi:}] Visione del vero 40 m
	\item[\textbf{Linguaggi:}] tutti, telepatia 45 m
	\item[\textbf{Sfida:}] 26 (90000 PX)\smallskip
\end{description}

\emph{\textbf{Incantesimi.}} Il Demogorgone ha CM 20. La sua caratteristica da incantatore è la Forza. Il Demogorgon conosce i seguenti incantesimi:

A volontà: \hyperlink{Individuazione del Magico}{Individuazione del Magico}, \hyperlink{Immagine Maggiore}{Immagine Maggiore}

livello 3 (4 slot): \emph{\hyperlink{Dissolvi Magie}{Dissolvi Magie}, \hyperlink{Paura}{Paura}, \hyperlink{Telecinesi}{Telecinesi}}

livello 4 (1 slot): \emph{\hyperlink{Immagine Proiettata}{Immagine Proiettata}, \hyperlink{Regressione Mentale}{Regressione Mentale}}

\emph{\textbf{Natura Demoniaca.}} Il Demogorgone non ha bisogno di aria, cibo, bevande o sonno.

\emph{\textbf{Resistenza Leggendaria (3/Giorno).}} Se il Demogorgone fallisce un Tiro Salvezza, può scegliere invece di riuscirvi.

\emph{\textbf{Resistenza allo Scacciare.}} Il Demogorgone ha +1d6 ai Tiri Salvezza contro gli effetti che scacciano i non morti.

\emph{\textbf{Due teste.}} Demogorgone ha +1d6 ai Tiri Salvezza contro essere cieco, sordo, svenuto

\textbf{Azioni}

\emph{\textbf{Multiattacco.} 2 attacchi con tentacolo}: +19, portata 3 metri, una creatura. Tutti gli attacchi di Demogorgone sono considerati magici +2.

\emph{Colpisce:} 35 (4d12 +9) danni contundenti. La creatura colpita deve fare un Tiro Salvezza su Tempra a DC 33 od i suoi Punti Ferita massimi scendono dello stesso ammontare.

\textbf{Reazione: \emph{Attacco d'opportunità}}: il Demogorgone effettua un attacco ad una creatura che attraversi o esca dalla sua portata di 3 metri.

\emph{\textbf{Sguardo}} Demogorgone fissa una creatura che può vedere entro 40 metri. Il bersaglio deve fare un Tiro Salvezza su Volontà a DC 33.

\emph{Effetto Sguardo:} Demogorgone sceglie uno di questi effetti oppure è a caso:

1. Sguardo Potente. Il bersaglio è svenuto fino al prossimo round o finché il Demogorgon è fuori dalla linea di vista

2. Sguardo Ipnotico. Il bersaglio è in dominato dal Demogorgone che ne stabilisce ogni azione. Questo sguardo necessità dell'utilizzo di entrambe le teste del Demogorgon.

3. Sguardo della Follia. Il bersaglio è sotto l'influenza dell'incantesimo Confusione che permane, senza Tiro Salvezza ulteriore, finché Demogorgone è in area di vista. Il Demogorgone non deve rimanere concentrato per il perdurare dell'effetto.


\textbf{Azioni Aggiuntive}

Il Demogorgone può effettuare 3 azioni aggiuntive, scelte da quelle sottostanti ed una per round solo al termine del round di un altra creatura.

\textbf{Coda.} Il Demogorgone attacca con la coda. +19 to al colpire, portata 5 metri, un obiettivo. Se colpisce 31 Punti Ferita di danni contundenti più 4d6 danni da Vuoto

\textbf{Sguardo di Follia.} Demogorgone usa o lo sguardo Potente o lo Sguardo della Follia

\textbf{Ecologia}\\
Ambiente: Abisso\\
Organizzazione: Unico\\
\textbf{Tesoro}: R, S, T, V\\
\textbf{Descrizione}\\
Demogorgone è un enorme demone, principe dell'abisso e della follia alto circa 5 metri. Appare come un rettiloide bipede con due teste da babbuino, i colli sono lunghi e serpentini come le braccia tentacolari. Le due teste di Demogorgone sono hanno personalità distinte che si detestano. Spesso tentano di dominarsi a vicenda e molte delle storie che riguardano il Demogorgone trattano proprio su come una o l'altra testa cechi di dominare il tutto. Tra il Demogorgone ed Orcus c'è una forte rivalità.


\begin{center}
	%\includegraphics[width=1\linewidth]{immagini/banner.png}
	\includegraphics[width=0.9\linewidth]{immagini/ercole-cerbero_grayscale.png}

	\emph{Hercules and Cerberus: Hercules grasps the collar of Cerberus. Antonio Tempesta}
\end{center}

\mostro{Dretch}
\noindent
\begin{description}[noitemsep, topsep=0pt, parsep=0pt, partopsep=0pt, leftmargin=0cm, labelwidth=2.2cm]
	\item[\textbf{Taglia/Tipo:}] Piccolo demone, malvagio
	\item[\textbf{Caratt.:}] \resizebox{0.5\linewidth+1.8cm}{!}{For 0 Des 0 Cos 1 Int -3 Sag -1 Car -4}
	\item[\textbf{Punti Ferita:}] 19,  \textbf{Difesa:} 12,  \textbf{Iniziativa:} +0
	\item[\textbf{Movimento:}] 6 m
	\item[\textbf{Tiri Salvez.:}] \resizebox{0.5\linewidth+1.8cm}{!}{Tempra +3, Riflessi +3, Volontà +3}
	\item[\textbf{Res. Danni:}] Freddo, Elettricità, Fuoco
	\item[\textbf{Imm. Danni:}] Veleno
	\item[\textbf{Vulnerabilità:}] ferro freddo, Luce
	\item[\textbf{Sensi:}] Scurovisione 18 m
	\item[\textbf{Linguaggi:}] Abissale, telepatia 18 m (funziona solo con le creature che comprendono l'Abissale)
	\item[\textbf{Sfida:}] 1/4 (50 PX)\smallskip
\end{description}

\textbf{Azioni}

\emph{\textbf{Multiattacco.}} Il demone effettua due attacchi: uno con il morso e uno con gli artigli.

\emph{\textbf{Artigli.} Attacco con arma da mischia}: +3 a colpire, portata 1 m, un bersaglio.

\emph{Colpisce:} 5 (2d4) danni taglienti.

\emph{\textbf{Morso.} Attacco con arma da mischia}: +3 a colpire, portata 1 m, un bersaglio.

\emph{Colpisce:} 3 (1d6) danni perforanti.

\textbf{Reazione: \emph{Anatomia opportunistica}} il dretch riorganizza la propria anatomia demoniaca dimezzando fine alla fine del round ogni danni critico subito.

\emph{\textbf{Nube Fetida (1/Giorno).}} Un disgustoso gas verde si estende in un raggio di 3 metri dal demone. Il gas si propaga intorno agli angoli e la sua area è oscurata leggermente. Rimane per 1 minuto o finché non viene disperso da un forte vento. Qualsiasi creatura che inizi il proprio round in quell'area deve riuscire un Tiro Salvezza di Tempra DC 11 o restare avvelenata fino all'inizio del suo prossimo round. Mentre è avvelenato in questo modo, il bersaglio, durante il suo round, è Rallentato 1.

\textbf{Ecologia}\\
Ambiente: Qualsiasi (Abisso)\\
Organizzazione: Solitario, coppia, banda (3-5), gruppo (6-12) o folla (13+)\\
\textbf{Categoria Tesoro}: Nessuno\\
\textbf{Descrizione}\\
Anche il più infimo demone dell'Abisso è pericoloso e possiede la necessità impellente di spargere rovina e sgomento. Il miserabile dretch è tanto orripilante e fetido quanto crudele, anche se non possiede la forza ed il potere per riuscire a soddisfare la sua voglia di brutalizzare gli altri nel suo reame nativo. Lo scopo dell'esistenza dei dretch è quello di servire demoni più potenti come vittime sacrificabili, e solo pochi fortunati riescono a sopravvivere abbastanza a lungo da evolversi.

I dretch sono i bersagli preferiti dai dilettanti in evocazioni abissali. Relativamente deboli e facili da intimorire, i dretch spesso possono essere obbligati a lunghi periodi di servitù utilizzando vaghe promesse di opportunità di sfogare le loro frustrazioni e la loro rabbia contro avversari più deboli. Eppure il potenziale evocatore di dretch farebbe meglio a ricordarsi che questi demoni sono codardi ed infidi quanto gli altri demoni. Un dretch che si trova di fronte un nemico più potente sarà assai lieto di scambiare qualsiasi informazione di cui disponga in cambio della sua miserevole vita.

A differenza della maggior parte dei demoni, la sciatta personalità del dretch ed il suo disprezzo per il lavoro fisico prolungato raramente danno dei risultati. I dretch avanzati sono rari, ma quelli che riescono a trovare la forza in se stessi per diventare più di quello che erano al momento della loro creazione divengono i sovrani poveri dell'Abisso, crudeli ed amareggiati, che regnano su parassiti, anime spezzate, non morti privi di intelletto e altri dretch. I loro imperi sono limitati a tratti abbandonati di fogne sotto città dimenticate, instabili distese paludose evitate dalle menti più sensate ed altri sgraditi angoli dell'Abisso che persino i demoni considerano scomodi o ripugnanti. Eppure per i signori dei dretch questi regni sono i loro imperi, e li difendono con pietosa tenacia.

Un dretch è alto 1,2 metri e pesa 90 kg. I dretch solitamente si formano dalle anime di mortali malvagi ed indolenti: è sufficiente solo un piccolo frammento di anima per dare origine ad una nascita così orripilante. Una sola anima spesso può causare l'apparizione di una piccola armata di dretch e la vista di un'orda di dretch appena nati che si liberano dalla protomateria pulsante dell'Abisso è al contempo nauseante e terrificante.


\begin{center}
	\includegraphics[width=0.9\linewidth]{immagini/Demone_Alato.png}

	\emph{Tomba dei demoni alati}
\end{center}

\mostro{Glabrezu}
\noindent
\begin{description}[noitemsep, topsep=0pt, parsep=0pt, partopsep=0pt, leftmargin=0cm, labelwidth=2.2cm]
	\item[\textbf{Taglia/Tipo:}] Grande demone, malvagio
	\item[\textbf{Caratt.:}] \resizebox{0.5\linewidth+1.8cm}{!}{For 5 Des 2 Cos 5 Int 4 Sag 3 Car 3}
	\item[\textbf{Punti Ferita:}] 186,  \textbf{Difesa:} 26,  \textbf{Iniziativa:} +4
	\item[\textbf{Movimento:}] 12 m
	\item[\textbf{Tiri Salvez.:}] \resizebox{0.5\linewidth+1.8cm}{!}{\resizebox{0.5\linewidth+1.8cm}{!}{Tempra +14, Riflessi +11, Volontà +12}}
	\item[\textbf{Res. Danni:}] Freddo, Elettricità, Fuoco; da arma non magica
	\item[\textbf{Imm. Danni:}] Veleno
	\item[\textbf{Vulnerabilità:}] ferro freddo, Luce
	\item[\textbf{Sensi:}] visione del vero 36 m
	\item[\textbf{Linguaggi:}] Abissale, telepatia 36 m
	\item[\textbf{Sfida:}] 9 (5000 PX)\smallskip
\end{description}

\emph{\textbf{Incantesimi Innati.}} La caratteristica da incantatore del demone è l'Intelligenza. Il demone può lanciare questi incantesimi in maniera innata, senza bisogno di componenti materiali:

A volontà: \emph{\hyperlink{Dissolvi Magie}{Dissolvi Magie}, \hyperlink{Individuazione del Magico}{Individuazione del Magico}, \hyperlink{Oscurità}{Oscurità}}

1/giorno ciascuno: \emph{\hyperlink{Confusione}{Confusione}, \hyperlink{Parola del Potere Stordire}{Parola del Potere Stordire}, \hyperlink{Volare}{Volare}}

\emph{\textbf{Resistenza alla Magia.}} Il demone ha +1d6 ai Tiri Salvezza contro incantesimi e altri effetti magici.

\textbf{Azioni}

\emph{\textbf{Multiattacco.}} Il demone effettua quattro attacchi: due con le chele e due con i pugni. In alternativa, può effettuare due attacchi con le chele e lanciare un incantesimo.

\emph{\textbf{Chela.} Attacco con arma da mischia}: +9 a colpire, portata 3 m, un bersaglio.

\emph{Colpisce:} 16 (2d10 + 5) danni contundenti. Se il bersaglio è una creatura di taglia Media o inferiore, è afferrato (DC 15 per fuggire). Il glabrezu possiede due chele, ciascuna delle quali può afferrare un bersaglio.

\emph{\textbf{Pugno.} Attacco in mischia con arma}: +9 a colpire, portata 1 m, un bersaglio.

\emph{Colpisce:} 7 (2d4 + 2) danni contundenti.

\textbf{Reazione: \emph{Attacco d'opportunità}}: il demone effettua un attacco ad una creatura che attraversi o esca dalla sua portata di 3 metri.

\emph{\textbf{Arrabbiato:}} il glabrezu crea un duplicato di se stesso dal piano delle ombre. Questo duplicato ha le stesse caratteristiche del glabrezu ma non attacca. Quando si attacca il glabrezu si ha un 50\% di attaccare il duplicato d'ombra.

\textbf{Ecologia}\\
Ambiente: Qualsiasi (Abisso)\\
Organizzazione: Solitario o drappello (1 glabrezu, 1 Succube e 2-5 Vrock)
\textbf{Categoria Tesoro}: U\\
\textbf{Descrizione}\\
Mentre la Succube è un demone che adesca la sua preda sfruttandone i desideri e le necessità carnali, il glabrezu è un tentatore di altro genere. Feroce e dalla forma bestiale, il glabrezu è in realtà un maestro di inganni e bugie. Con la sua abilità di nascondere la sua vera forma dietro piacenti illusioni, usa la sua magia per esaudire i desideri degli umanoidi mortali, come forma di ricompensa per coloro che soccombono ai suoi inganni e raggiri. Un desiderio esaudito da un glabrezu appaga la necessità di chi lo esprime nel modo più rovinoso possibile, sebbene queste conseguenze possano non rivelarsi immediatamente tali. Un fabbro che fatica ad affermarsi potrebbe desiderare fama ed abilità nella professione scelta, solo per scoprire che il suo miglior patrono è un crudele e sadico omicida che usa le armi per promuovere i propri distruttivi desideri. Un uomo solo che esprime il desiderio di avere una compagna, potrebbe vedere il suo desiderio avverarsi con una sua vecchia fiamma ritornata alla vita in forma di vampiro, ed altri esempi di questo tipo. Il glabrezu è assai creativo nel soddisfare i desideri di un mortale.

Un glabrezu è alto 5,3 metri e pesa poco più di 3000 kg. Questi perfidi demoni si originano dalle anime dei traditori, dei falsi e dei sovversivi: anime di mortali che, in vita, giurarono il falso o utilizzarono il tradimento e l'inganno per rovinare le vite altrui.

\mostro{Hezrou}
\noindent
\begin{description}[noitemsep, topsep=0pt, parsep=0pt, partopsep=0pt, leftmargin=0cm, labelwidth=2.2cm]
	\item[\textbf{Taglia/Tipo:}] Grande demone, malvagio
	\item[\textbf{Caratt.:}] \resizebox{0.5\linewidth+1.8cm}{!}{For 4 Des 3 Cos 5 Int 5 Sag 1 Car 1}
	\item[\textbf{Punti Ferita:}] 167,  \textbf{Difesa:} 25,  \textbf{Iniziativa:} +5
	\item[\textbf{Movimento:}] 9 m
	\item[\textbf{Tiri Salvez.:}] \resizebox{0.5\linewidth+1.8cm}{!}{Tempra +13, Riflessi +11, Volontà +9}
	\item[\textbf{Res. Danni:}] Freddo, Elettricità, Fuoco; da arma non magica
	\item[\textbf{Imm. Danni:}] Veleno
	\item[\textbf{Vulnerabilità:}] ferro freddo, Luce
	\item[\textbf{Sensi:}] Scurovisione 36 m
	\item[\textbf{Linguaggi:}] Abissale, telepatia 36 m
	\item[\textbf{Sfida:}] 8 (3900 PX)\smallskip
\end{description}

\emph{\textbf{Fetore.}} Qualsiasi creatura che inizi il suo round entro 3 metri dal demone, deve riuscire un Tiro Salvezza di Tempra DC 21 o restare avvelenata, -1 Forza e Destrezza, fino all'inizio del proprio round. Se riesce il Tiro Salvezza, la creatura è immune al fetore del demone per 24 ore.

\emph{\textbf{Resistenza alla Magia.}} Il demone ha +1d6 ai Tiri Salvezza contro incantesimi e altri effetti magici.

\textbf{Azioni}

\emph{\textbf{Multiattacco.}} Il demone effettua tre attacchi: uno con il morso e due con gli artigli.

\emph{\textbf{Artiglio.} Attacco con arma da mischia}: +8 a colpire, portata 1 m, un bersaglio.

\emph{Colpisce:} 11 (2d6 + 4) danni taglienti, 2 danni da Sanguinamento.

\emph{\textbf{Morso.} Attacco con arma da mischia}: +8 a colpire, portata 1 m, un bersaglio.

\emph{Colpisce:} 15 (2d10 + 4) danni perforanti e malattia Febbre Demoniaca minore.

\emph{Febbre Demoniaca minore}: 1 minuto, TS Tempra DC 18, 6 ore, 3 successi, -1 Costituzione e Saggezza.

\textbf{Reazione: \emph{Pustola esplosiva}} il demone quando colpito da un critico fa esplodere una pustola fetida che infligge alla creatura che ha portato il critico, entro 2 metri, 2d8 danni da acido.

\emph{\textbf{Arrabbiato:}} Hezrou rilascia una nube di fetore incendiaria. Tutte le creature attorno a lui entro 3 metri devono fare un Tiro Salvezza su Riflessi DC 21 per dimezzare i 4d10 di danno da fuoco. Costa 2 Azioni.

\textbf{Ecologia}\\
Ambiente: Qualsiasi (Abisso)\\
Organizzazione: Solitario o banda (2-4)\\
\textbf{Categoria Tesoro}: W\\
\textbf{Descrizione}\\
L'hezrou vive nelle vaste paludi, acquitrini e corsi d'acqua dell'Abisso. E' a suo agio sia nell'acqua che sulla terraferma. La presenza di un hezrou ha un effetto dannoso su flora, causando mutazioni e rendendo maleodoranti e dal sapore salmastro le acque. Spesso intere comunità isolate di mutanti deformi devono il loro aspetto contorto non tanto ai loro depravati costumi quanto alla vicinanza di un hezrou.

Queste mostruose e bestiali creature nascono dalle anime di mortali malvagi che hanno avvelenato se stessi, i loro parenti o il loro ambiente, ad esempio, drogati, assassini ed alchimisti che non si sono preoccupati di come i loro esperimenti avvelenassero il mondo naturale.

\mostro{Marilith}
\noindent
\begin{description}[noitemsep, topsep=0pt, parsep=0pt, partopsep=0pt, leftmargin=0cm, labelwidth=2.2cm]
	\item[\textbf{Taglia/Tipo:}] Grande demone, malvagio
	\item[\textbf{Caratt.:}] \resizebox{0.5\linewidth+1.8cm}{!}{For 4 Des 5 Cos 5 Int 4 Sag 3 Car 5}
	\item[\textbf{Punti Ferita:}] 319,  \textbf{Difesa:} 38,  \textbf{Iniziativa:} +5
	\item[\textbf{Movimento:}] 12 m
	\item[\textbf{Tiri Salvez.:}] \resizebox{0.5\linewidth+1.8cm}{!}{\resizebox{0.5\linewidth+1.8cm}{!}{Tempra +21, Riflessi +21, Volontà +19}}
	\item[\textbf{Res. Danni:}] Freddo, Elettricità, Fuoco
	\item[\textbf{Imm. Danni:}] Veleno, armi +1
	\item[\textbf{Vulnerabilità:}] ferro freddo, Luce
	\item[\textbf{Sensi:}] visione del vero 36 m
	\item[\textbf{Linguaggi:}] Abissale, telepatia 36 m
	\item[\textbf{Sfida:}] 16 (15000 PX)\smallskip
\end{description}

\emph{\textbf{Armi Magiche.}} Gli attacchi con armi del demone sono magici.

\emph{\textbf{Reattivo.}} il marlith può effettuare una Reazione di Parata durante ciscuno round.

\emph{\textbf{Resistenza alla Magia.}} Il demone ha +1d6 ai Tiri Salvezza contro incantesimi e altri effetti magici.

\textbf{Azioni}

\emph{\textbf{Multiattacco.}} Il demone effettua sette attacchi: sei con le spade lunghe e uno con la coda.

\emph{\textbf{Coda.} Attacco con arma da mischia}: +13 a colpire, portata 3 m, una creatura.

\emph{Colpisce:} 15 (2d10 + 4) danni contundenti. Se il bersaglio è di taglia Media o inferiore, è afferrato (DC 19 per fuggire). Fino al termine dell'afferrare il demone può colpire automaticamente il bersaglio con la coda, ma non può effettuare attacchi di coda contro altri bersagli.

\emph{\textbf{Spada Lunga.} Attacco con arma da mischia}: +13 a colpire, portata 1 m, un bersaglio.

\emph{Colpisce:} 13 (2d8 + 4) danni taglienti.

\textbf{Reazione: \emph{Parata.}} Il demone somma 5 alla sua Difesa contro un attacco da mischia che lo colpirebbe. Per farlo il demone deve poter vedere il suo attaccante e impugnare un'arma da mischia.

\textbf{Reazione: \emph{Attacco d'opportunità}}: il demone effettua un attacco ad una creatura che attraversi o esca dalla sua portata di 3 metri.

\emph{\textbf{Arrabbiato:}}

- la marilith affila le sue spade tra loro, ogni attacco con la spada lunga guadagna Sanguinamento 1/20. Costa 2 Azioni, dura fino alla fine del combattimento.

- la marilith condanna l'avversario all'abisso. Costo 2 Azioni. L'avversario deve effettuare un Tiro Salvezza su Volontà DC 28 o essere trasportato nell'abisso.

\textbf{Ecologia}\\
Ambiente: Qualsiasi (Abisso)\\
Organizzazione: Solitario, coppia o plotone (1 marilith, 1-3 Glabrezu e 3-14 Babau)\\
\textbf{Categoria Tesoro}: C\\
\textbf{Descrizione}\\
Sovrane di orde demoniache e regine di nazioni abissali, le temibili marilith servono i signori dei demoni come governanti, consigliere e persino amanti, eppure la loro supremazia come strateghe le rende particolarmente richieste come generali e comandanti d'armate. Le marilith più potenti non sono al servizio di nessuno e comandano invece fameliche legioni demoniache.

Una marilith è alta da 1,8 a 2,7 metri, lunga 6 metri dalla testa alla punta della coda, e pesa 2000 kg. Solo le anime malvagie più arroganti ed orgogliose, solitamente quelle di crudeli sovrani, sadici generali e signori della guerra particolarmente violenti, possono causare la nascita di una marilith.

\mostro{Nalfeshnee}
\noindent
\begin{description}[noitemsep, topsep=0pt, parsep=0pt, partopsep=0pt, leftmargin=0cm, labelwidth=2.2cm]
	\item[\textbf{Taglia/Tipo:}] Grande demone, malvagio
	\item[\textbf{Caratt.:}] \resizebox{0.5\linewidth+1.8cm}{!}{For 5 Des 0 Cos 6 Int 4 Sag 1 Car 2}
	\item[\textbf{Punti Ferita:}] 264,  \textbf{Difesa:} 29,  \textbf{Iniziativa:} +4
	\item[\textbf{Movimento:}] 6 m, volo 9 m
	\item[\textbf{Tiri Salvez.:}] \resizebox{0.5\linewidth+1.8cm}{!}{\resizebox{0.5\linewidth+1.8cm}{!}{Tempra +19, Riflessi +13, Volontà +14}}
	\item[\textbf{Res. Danni:}] Freddo, Elettricità, Fuoco; da arma non magica
	\item[\textbf{Imm. Danni:}] Veleno
	\item[\textbf{Vulnerabilità:}] ferro freddo, Luce
	\item[\textbf{Sensi:}] Scurovisione 36 m
	\item[\textbf{Linguaggi:}] Abissale, telepatia 36 m
	\item[\textbf{Sfida:}] 13 (10000 PX)\smallskip
\end{description}

\emph{\textbf{Resistenza alla Magia.}} Il demone ha +1d6 ai Tiri Salvezza contro incantesimi e altri effetti magici.

\textbf{Azioni}

\emph{\textbf{Multiattacco.}} Il demone usa, se possibile, Aureola di Orrore. Poi effettua tre attacchi: uno con il morso e due con gli artigli.

\emph{\textbf{Artiglio.} Attacco con arma da mischia}: +12 a colpire, portata 3 m, un bersaglio.

\emph{Colpisce:} 15 (3d6 + 5) danni taglienti, 2 danni da Sanguinamento.

\emph{\textbf{Morso.} Attacco con arma da mischia}: +12 a colpire, portata 1 m, un bersaglio.

\emph{Colpisce:} 32 (5d10 + 5) danni perforanti e Febbre Demoniaca.

\emph{Febbre Demoniaca}: 1 minuto, TS Tempra DC 23, 4 ore, 3 successi, -1 Costituzione e Saggezza/4 ore.

\emph{\textbf{Aureola di Orrore (Ricarica 5-6).}} Il demone emette una luce magica multicolore e scintillante. Ogni creatura entro 5 metri dal demone e che possa vedere la luce deve riuscire un Tiro Salvezza su Volontà DC 25 o restare spaventata per 1 minuto.

Una creatura può ripetere il Tiro Salvezza al termine di ciascun suo round, terminando l'effetto per sé se lo riesce.

Se il Tiro Salvezza della creatura riesce o l'effetto ha termine per essa la creatura è immune all'Aureola di Orrore del demone per le successive 24 ore.

\emph{\textbf{Teletrasporto.}} Il demone si teletrasporta, insieme a tutto l'equipaggiamento che sta indossando o trasportando, in uno spazio non occupato che possa vedere fino a 36 metri di distanza. E' una Azione di Movimento.

\textbf{Reazione: \emph{Attacco d'opportunità}}: il demone effettua un attacco ad una creatura che attraversi o esca dalla sua portata di 1 metro.

\emph{\textbf{Arrabbiato:}} il nalfeshnee mima le parole arcane ed in gesti visti entro 3 round precedenti e lancia un incantesimo di cui è stato testimone. Costa 3 Azioni.

\textbf{Ecologia}
Ambiente: Qualsiasi (Abisso)\\
Organizzazione: Solitario o banda di guerra (1 nalfeshnee, 1 Hezrou e 2-5 Vrock)\\
\textbf{Categoria Tesoro}: G\\
\textbf{Descrizione}\\
Sono pochi i demoni che comprendono le meccaniche interne che regolano l'Abisso come i nalfeshnee, e non è raro che questi demoni servano l'Abisso stesso invece che un signore dei demoni. Alcuni sovrintendono i reami organici che generano i nuovi demoni, mentre altri custodiscono luoghi di particolare importanza nei recessi nascosti del piano. Spesso il regno di un nalfeshnee nell'Abisso è superiore per forze e dimensioni al più grande dei regni mortali, in quanto questi demoni hanno una predisposizione naturale a governare ed imporre una sorta di ordine al caos dell'Abisso. Gli evocatori mortali spesso li richiamano per il loro folle ma impareggiabile intelletto, esaminando accuratamente gli accordi presi con questi demoni onde evitare eventuali conseguenze nascoste e risvolti non voluti, in quanto un nalfeshnee raramente accetta qualcosa che, in qualche modo contorto, non gli consenta di soddisfare le necessità ed i desideri dell'Abisso.

I nalfeshnee sono alti 6 metri e pesano 4000 kg. Sono creati dalle anime di malvagi mortali avari o bramosi, in particolare di coloro che hanno regnato su imperi di schiavitù, furto, brigantaggio e altri vizi ancora più violenti.

\mostro{Orcus}
\noindent
\begin{description}[noitemsep, topsep=0pt, parsep=0pt, partopsep=0pt, leftmargin=0cm, labelwidth=2.2cm]
	\item[\textbf{Taglia/Tipo:}] Enorme principe demone, malvagio
	\item[\textbf{Caratt.:}] \resizebox{0.5\linewidth+1.8cm}{!}{For 8 Des 2 Cos 7 Int 5 Sag 5 Car 7}
	\item[\textbf{Punti Ferita:}] 519,  \textbf{Difesa:} 48,  \textbf{Iniziativa:} +5
	\item[\textbf{Movimento:}] 15 metri, volare 15 metri
	\item[\textbf{Tiri Salvez.:}] \resizebox{0.5\linewidth+1.8cm}{!}{\resizebox{0.5\linewidth+1.8cm}{!}{Tempra +33, Riflessi +28, Volontà +31}}
	\item[\textbf{Comp.:}] tutte +13
	\item[\textbf{Res. Danni:}] Freddo, Elettricità, Fuoco
	\item[\textbf{Imm. Danni:}] da Vuoto, Veleno; armi +2
	\item[\textbf{Immunità:}] affascinato, paralizzato, affaticato, spaventato
	\item[\textbf{Vulnerabilità:}] Luce
	\item[\textbf{Sensi:}] Visione del vero 40 m
	\item[\textbf{Linguaggi:}] tutti, telepatia 45 m
	\item[\textbf{Sfida:}] 26 (90000 PX)\smallskip
\end{description}

\emph{\textbf{Incantesimi.}} Orcus ha CM 17, DC 30. La sua caratteristica da incantatore è il Carisma. Orcus conosce i seguenti incantesimi:

A volontà: \hyperlink{Individuazione del Magico}{Individuazione del Magico}, \hyperlink{Tocco Gelido}{Tocco Gelido}

livello 3 (3 slot): \emph{\hyperlink{Dissolvi Magie}{Dissolvi Magie}}

livello 6 (3 slot): \emph{\hyperlink{Creare Non Morti}{Creare Non Morti}}

livello 9 (1 slot): \emph{\hyperlink{Fermare il Tempo}{Fermare il Tempo}}

\emph{\textbf{Natura Demoniaca.}} Orcus non ha bisogno di aria, cibo, bevande o sonno.

\emph{\textbf{Resistenza Leggendaria (3/Giorno).}} Se il Orcus fallisce un Tiro Salvezza, può scegliere invece di riuscirvi.

\emph{\textbf{Signore dei non morti.}} Orcus può sempre decidere il tipo di non morto che crea e questo rimane sotto il suo controllo per tempo indefinito, oltretutto può lanciare l'incantesimo in qualsiasi condizione si trovi.

\textbf{Azioni}

\emph{\textbf{Multiattacco.} 2 attacchi con bacchetta}: +19, portata 3 metri, una creatura. Tutti gli attacchi di Orcus sono considerati magici +3.

\emph{Colpisce:} 21 (3d8 + 8) danni contundenti + 13 (2d12) da Vuoto

\emph{\textbf{Coda}} Orcus colpisce con la sua coda. +19, portata 3 metri, una creatura

\emph{Colpisce:} 21 (3d8 + 8) danni contundenti + 18 (4d8) da Veleno

\textbf{Reazione: \emph{Attacco d'opportunità}}: il demone effettua un attacco ad una creatura che attraversi o esca dalla sua portata di 3 metri.

\textbf{Azioni Aggiuntive}

Il Orcus può effettuare 3 azioni aggiuntive, scelte da quelle sottostanti ed una per round solo al termine del round di un altra creatura.

\textbf{Coda.} Il Orcus attacca con la coda. +19 to al colpire, portata 5 metri, un obiettivo. Se colpisce 21 (3d8 + 8) danni contundenti + 18 (4d8) da Veleno

\textbf{Assaggio di Morte.} Orcus lancia l'incantesimo \hyperlink{Colpo Infuocato}{Colpo Infuocato}, in maniera blasfema, con danni da Vuoto

\textbf{Ecologia}\\
Ambiente: Abisso\\
Organizzazione: Unico\\
\textbf{Categoria Tesoro}: Z\\
\textbf{Descrizione}\\
Orcus è il Principe Demone dei non morti. Predilige la compagnia e servizio dei non morti. Desidera vedere scomparire tutta la vita e questa trasformarsi tutta in una gigantesca necropoli di non morti sotto il suo comando. Orcus ha la testa e le gambe da capra, corna simili a montoni, un corpo gonfio, ali da pipistrello e una lunga coda.

\mostro{Silku}
\noindent
\begin{description}[noitemsep, topsep=0pt, parsep=0pt, partopsep=0pt, leftmargin=0cm, labelwidth=2.2cm]
	\item[\textbf{Taglia/Tipo:}] Media demone, malvagio
	\item[\textbf{Caratt.:}] \resizebox{0.5\linewidth+1.8cm}{!}{For 2 Des 2 Cos 3 Int 1 Sag 0 Car 2}
	\item[\textbf{Punti Ferita:}] 52,  \textbf{Difesa:} 16,  \textbf{Iniziativa:} +2
	\item[\textbf{Movimento:}] 9 m
	\item[\textbf{Tiri Salvez.:}] \resizebox{0.5\linewidth+1.8cm}{!}{Tempra +5, Riflessi +4, Volontà +3}
	\item[\textbf{Comp.:}] Furtività +2, Ingannare +5
	\item[\textbf{Res. Danni:}] Freddo, Elettricità; da arma non magica o non argentata
	\item[\textbf{Imm. Danni:}] Veleno
	\item[\textbf{Vulnerabilità:}] ferro freddo, Luce
	\item[\textbf{Sensi:}] Scurovisione 36 m
	\item[\textbf{Linguaggi:}] Abissale, Comune
	\item[\textbf{Sfida:}] 2 (200 PX)\smallskip
\end{description}

\emph{\textbf{Resistenza alla Magia.}} Il demone ha +1d6 ai Tiri Salvezza contro incantesimi e altri effetti magici.

\textbf{Azioni}

\emph{\textbf{Artigli.} Attacco con arma da mischia}: +4 a colpire, portata 1 m, un bersaglio.

\emph{Colpisce:} 6 (1d6 + 3) danni taglienti. Se il bersaglio è una creatura, deve riuscire un Tiro Salvezza di Tempra DC 14 o subire 5 (2d4) danni da veleno

\emph{\textbf{Cambiare aspetto (a volontà).}} Il Silku può apparire come un umanoide di taglia media a suo piacimento. E' necessario una prova di Consapevolezza DC 16 per percepire il vero aspetto. 2 Azioni

\emph{\textbf{Rigenerazione.}} Il Silku rigenera 2 PF alla fine del suo round a meno che non sia stato ferito con acido o fuoco, l'effetto è attivo anche se il Silku ha Punti Ferita negativi.

\textbf{Ecologia}\\
Ambiente: Qualsiasi (Abisso)\\
Organizzazione: Piccoli gruppi (3-6)\\
\textbf{Categoria Tesoro}: P\\
\textbf{Descrizione}\\
"...I loro volti avevano qualcosa di strano, erano come ... sfocati, era l'unica parte del loro corpo che non riuscivo a mettere a fuoco. Stupita dalla stranezza, sbattei le palpebre più volte e concentrai lo sguardo sui volti di entrambi. Sentii uno strano pizzicore sul volto e poi la vista si schiarì.
Sgranai gli occhi e indietreggiai di un passo, mentre il terrore si impadroniva di me. I loro visi non erano umani. Avevano entrambi la pelle grigiastra e grinzosa, il naso schiacciato e lunghi canini che uscivano dalla bocca, grandi orecchie e occhi piccoli e neri. Sembrava il muso di un pipistrello."

Da \emph{Il Guardiano di Falkonia}, romanzo di Federica Angeli

\mostro{Quasit}
\noindent
\begin{description}[noitemsep, topsep=0pt, parsep=0pt, partopsep=0pt, leftmargin=0cm, labelwidth=2.2cm]
	\item[\textbf{Taglia/Tipo:}] Minuscola demone, mutaforma, malvagio
	\item[\textbf{Caratt.:}] \resizebox{0.5\linewidth+1.8cm}{!}{For -3 Des 3 Cos 0 Int -2 Sag 0 Car 0}
	\item[\textbf{Punti Ferita:}] 33,  \textbf{Difesa:} 16,  \textbf{Iniziativa:} +3
	\item[\textbf{Movimento:}] 12 m (3 m, volo 12 m in forma di pipistrello; 12 m, scalata 12 m in forma di centopiedi; 12 m, nuoto 12 m in forma di rospo)
	\item[\textbf{Tiri Salvez.:}] \resizebox{0.5\linewidth+1.8cm}{!}{Tempra +3, Riflessi +4, Volontà +3}
	\item[\textbf{Comp.:}] Furtività +5
	\item[\textbf{Res. Danni:}] Freddo, Elettricità, Fuoco; da arma non magica
	\item[\textbf{Imm. Danni:}] Veleno
	\item[\textbf{Vulnerabilità:}] ferro freddo, Luce
	\item[\textbf{Sensi:}] Scurovisione 36 m
	\item[\textbf{Linguaggi:}] Abissale, Comune
	\item[\textbf{Sfida:}] 1 (200 PX)\smallskip
\end{description}

\emph{\textbf{Mutaforma.}} Il demone può usare una Azione per trasformarsi in una forma bestiale da pipistrello, centopiedi o rospo, o per tornare alla sua vera forma. Le sue statistiche sono le stesse in tutte le forme, sebbene gli attacchi possano variare per alcune di esse. Qualsiasi equipaggiamento stia indossando o trasportando non viene trasformato. Alla morte ritorna alla sua vera forma.

\emph{\textbf{Resistenza alla Magia.}} Il demone ha +1d6 ai Tiri Salvezza contro incantesimi e altri effetti magici.

\textbf{Azioni}

\emph{\textbf{Artigli (Morso in Forma di Bestia).} Attacco con arma da mischia}: +4 a colpire, portata 1 m, un bersaglio.

\emph{Colpisce:} 5 (1d4 + 3) danni perforanti. Se il bersaglio è una creatura, deve riuscire un Tiro Salvezza di Tempra DC 12 o subire 5 (2d4) danni da veleno e restare avvelenato, -1 Forza e Destrezza, per 1 minuto. La creatura può ripetere il Tiro Salvezza al termine di ciascun suo round, ponendo termine all'effetto se lo riesce.

\emph{\textbf{Invisibilità.}} Il demone resta invisibile finché non attacca o termina la sua concentrazione. Qualsiasi cosa che il demone stia trasportando o indossando resta invisibile finché rimane in contatto con il demone.

\emph{\textbf{Spavento (1/Giorno).}} Una creatura scelta dal demone che si trovi entro 6 metri da lui, deve riuscire un Tiro Salvezza su Volontà DC 12 o restare spaventata per 1 minuto. Il bersaglio può ripetere il Tiro Salvezza al termine di ciascun suo round, con -1d6 se il demone è in linea di visuale, ponendo termine all'effetto prematuramente se riesce il Tiro Salvezza.

\textbf{Ecologia}\\
Ambiente: Qualsiasi (Abisso)\\
Organizzazione: Solitario o stormo (2-12)\\
\textbf{Categoria Tesoro}: Nessuno\\
\textbf{Descrizione}\\
Il quasit è forse il demone meno potente, ma non è tra i meno rispettati: persino i quasit si ritengono superiori alle orde di Dretch e, fedeli alla propria natura, i Dretch mancano del coraggio o degli stimoli necessari a dimostrare loro che si sbagliano. Il ruolo primario in vita di un quasit è quello di famiglio al servizio di un incantatore, ma quei quasit che sfuggono a questa umiliante servitù acquisiscono una volontà propria e sono molto più pericolosi. Un quasit tipico è alto 45 centimetri e pesa solo 4 kg.

Unici tra le orde demoniache, i quasit non nascono dalle anime di malvagi mortali deceduti, ma da anime viventi: quando un incantatore cerca di richiamare a sé un quasit come famiglio, la sua anima sfiora l'Abisso ed esso reagisce, creando dalla sua materia un quasit collegato all'anima dell'incantatore e generando un potente legame tra i due.

I quasit appena creati vengono alla luce direttamente nel Piano Materiale, dove diventano famigli e, finché sono soggetti alla volontà del loro padrone, lo odiano e disprezzano, dal momento che possono percepire il pulsare delle sua anima e sanno che potrebbero aspirare a qualcosa di più. Un quasit serve, eppure osserva e vigila nell'attesa di errori che possano costare la vita al suo signore, o meglio, che gli consentano di rivoltarsi contro il proprio padrone. Alla morte del proprio padrone il quasit spesso decide di rimanere nel Piano Materiale in cerca di altri modi per divertirsi, solitamente insediandosi in un'area urbana dove ci sono molti individui da tormentare.

\mostro{Succube}
\noindent
\begin{description}[noitemsep, topsep=0pt, parsep=0pt, partopsep=0pt, leftmargin=0cm, labelwidth=2.2cm]
	\item[\textbf{Taglia/Tipo:}] Media demone, malvagio
	\item[\textbf{Caratt.:}] \resizebox{0.5\linewidth+1.8cm}{!}{For -1 Des 3 Cos 1 Int 2 Sag 1 Car 5}
	\item[\textbf{Punti Ferita:}] 87,  \textbf{Difesa:} 20,  \textbf{Iniziativa:} +3
	\item[\textbf{Movimento:}] 9 m, volo 18 m
	\item[\textbf{Tiri Salvez.:}] \resizebox{0.5\linewidth+1.8cm}{!}{Tempra +5, Riflessi +7, Volontà +5}
	\item[\textbf{Comp.:}] Furtività +5, Percepire Emozioni +5, Consapevolezza +5, Ingannare +9
	\item[\textbf{Res. Danni:}] Freddo, Elettricità, Fuoco, Veleno; da arma non magica
	\item[\textbf{Sensi:}] Scurovisione 18 m
	\item[\textbf{Vulnerabilità:}] ferro freddo, Luce
	\item[\textbf{Linguaggi:}] Abissale, Comune, Infernale, telepatia 18 m
	\item[\textbf{Sfida:}] 4 (1100 PX)
\end{description}

\noindent\rule{\linewidth}{2pt}

\medskip

\emph{\textbf{Legame Telepatico.}} L'immondo ignora le restrizioni di raggio di azione della sua telepatia quando comunica con una creatura che ha affascinato. I due non sono neppure costretti a trovarsi sullo stesso piano di esistenza.

\emph{\textbf{Mutaforma.}} L'immondo può usare una Azione per trasformarsi in un umanoide di taglia Piccola o Media, o per tornare alla sua vera forma. Senza le ali, l'immondo perde la velocità di volo. A parte la taglia e la velocità, le sue statistiche sono le stesse in tutte le forme. Qualsiasi equipaggiamento stia indossando o trasportando non viene trasformato. Alla morte ritorna alla sua vera forma.

\textbf{Azioni}

\emph{\textbf{Artiglio (Solo Forma Immonda).} Attacco con arma da mischia}: +6 a colpire, portata 1 m, un bersaglio.

\emph{Colpisce:} 6 (1d6 + 3) danni taglienti.

\emph{\textbf{Affascinare.}} Un umanoide visibile all'immondo entro 9 metri da esso deve riuscire un Tiro Salvezza di Volontà DC 16 o restare magicamente affascinato per 1 giorno. Il bersaglio affascinato obbedisce ai comandi verbali o telepatici dell'immondo. Se il bersaglio subisce danni o riceve un comando suicida, può ripetere il Tiro Salvezza, terminando l'effetto se lo riesce. Se il bersaglio riesce il Tiro Salvezza contro l'effetto, o se l'effetto termina, il bersaglio è immune all'Affascinare dell'immondo per le successive 24 ore.

L'immondo può tenere affascinato solo un bersaglio alla volta. Se ne affascina un altro, l'effetto sul bersaglio precedente termina.

\emph{\textbf{Bacio Risucchiante.}} L'immondo bacia una creatura affascinata o una creatura consenziente. Il bersaglio deve effettuare un Tiro Salvezza di Tempra DC 16 contro questa magia, subendo 32 (5d10 + 5) danni se lo fallisce, o la metà di questi danni se lo riesce. L'immondo recupera metà dei Punti Ferita persi dalla creatura. I Punti Ferita massimi del bersaglio vengono ridotti di un ammontare pari ai danni subiti. Questa riduzione perdura finché non sorge l'alba. Il bersaglio muore se questo effetto riduce i suoi Punti Ferita massimi a 0.

\emph{\textbf{Forma Eterea.}} L'immondo entra magicamente nel Piano Etereo dal Piano Materiale, e viceversa.

\textbf{Ecologia}\\
Ambiente: Qualsiasi (Abisso)\\
Organizzazione: Solitario, coppia o harem (3-12)\\
\textbf{Categoria Tesoro}: I\\
\textbf{Descrizione}\\
Tra le orde demoniache una succube spesso può raggiungere altissimi livelli di potere, utilizzando le sue manipolazioni ed il suo fascino sensuale, e molte guerre demoniache imperversano a causa delle subdole macchinazioni di queste creature. Una succube si origina dalle anime di malvagi mortali particolarmente libidinosi ed avidi.

\mostro{Vrock}
\noindent
\begin{description}[noitemsep, topsep=0pt, parsep=0pt, partopsep=0pt, leftmargin=0cm, labelwidth=2.2cm]
	\item[\textbf{Taglia/Tipo:}] Grande demone, malvagio
	\item[\textbf{Caratt.:}] \resizebox{0.5\linewidth+1.8cm}{!}{For 3 Des 2 Cos 4 Int -1 Sag 1 Car -1}
	\item[\textbf{Punti Ferita:}] 127,  \textbf{Difesa:} 22,  \textbf{Iniziativa:} +2
	\item[\textbf{Movimento:}] 12 m, volo 18 m
	\item[\textbf{Tiri Salvez.:}] \resizebox{0.5\linewidth+1.8cm}{!}{Tempra +10, Riflessi +8, Volontà +7}
	\item[\textbf{Res. Danni:}] Freddo, Elettricità, Fuoco; da arma non magica
	\item[\textbf{Imm. Danni:}] Veleno
	\item[\textbf{Vulnerabilità:}] ferro freddo, Luce
	\item[\textbf{Sensi:}] Scurovisione 36 m
	\item[\textbf{Linguaggi:}] Abissale, telepatia 36 m
	\item[\textbf{Sfida:}] 6 (2300 PX)\smallskip
\end{description}

\emph{\textbf{Resistenza alla Magia.}} Il demone ha +1d6 ai Tiri Salvezza contro incantesimi e altri effetti magici.

\textbf{Azioni}

\emph{\textbf{Multiattacco.}} Il demone effettua due attacchi: uno con il becco e uno con gli speroni.

\emph{\textbf{Becco.} Attacco con arma da mischia}: +8 a colpire, portata 1 m, un bersaglio.

\emph{Colpisce:} 10 (2d6 + 3) danni perforanti.

\emph{\textbf{Speroni.} Attacco con arma da mischia}: +8 a colpire, portata 1 m, un bersaglio.

\emph{Colpisce:} 14 (2d10 + 3) danni taglienti.

\emph{\textbf{Spore (Ricarica 6).}} Una nube di spore tossiche si diffonde in un raggio di 5 metri intorno al demone. Ogni creatura in quell'area deve riuscire un Tiro Salvezza su Tempra DC 18 o restare avvelenata. Mentre avvelenato in questo modo, un bersaglio subisce 5 (1d10) danni da veleno all'inizio di ciascun suo round. Il bersaglio può ripetere il Tiro Salvezza al termine di ciascun suo round, ponendo termine all'effetto se lo riesce. Svuotare una fiala di Acqua santa sul bersaglio pone termine all'effetto.

\emph{\textbf{Strillo Stordente (1/Giorno).}} Il demone emette uno strillo orripilante. Ogni creatura entro 6 metri da esso e che lo possa udire, e non sia un demone, deve riuscire un Tiro Salvezza su Tempra DC 18 o restare stordita fino al termine del prossimo round del demone.

\textbf{Reazione: \emph{Attacco d'opportunità}}: il Vrock effettua un attacco ad una creatura che attraversi o esca dalla sua portata di 1 metro.

\emph{\textbf{Arrabbiato:}} Il Vrock striscia il becco con gli speroni rendendoli ancora più affilati. Fino alla fine del combattimento il danno causato da Becco e Speroni causa 1 danno da Sanguinamento fino ad un massimo di 10 danni. 1 Azione.

\textbf{Ecologia}\\
Ambiente: Qualsiasi (Abisso)\\
Organizzazione: Solitario, coppia o banda (3-10)\\
\textbf{Categoria Tesoro}: B\\
\textbf{Descrizione}\\
Profani campioni dell'Abisso, i vrock incarnano tutta la rabbia, l'odio e la violenza di questo reame. Tanto voraci e grottescamente opportunisti quanto il saprofago a cui assomigliano, i vrock si deliziano nello spargimento di sangue, godendo del suono e delle sensazioni derivanti dallo strappare gli intestini ancora pulsanti da una creatura vivente.\\
Un vrock tipico è alto 2,3 metri e pesa 200 kg. Queste creature solitamente si originano dalle anime di malvagi mortali colmi di odio e di collera, in particolare coloro che erano criminali professionisti, mercenari o assassini.

\mostro{Destriero dell'Incubo}
\begin{description}[noitemsep, topsep=0pt, parsep=0pt, partopsep=0pt, leftmargin=0cm, labelwidth=2.2cm]
	\item[\textbf{Taglia/Tipo:}] Grande immondo, malvagio
	\item[\textbf{Caratt.:}] \resizebox{0.5\linewidth+1.8cm}{!}{For 4 Des 2 Cos 3 Int 0 Sag 1 Car 2}
	\item[\textbf{Punti Ferita:}] 70,  \textbf{Difesa:} 18,  \textbf{Iniziativa:} +2
	\item[\textbf{Movimento:}] 18 m, volo 24 m
	\item[\textbf{Tiri Salvez.:}] \resizebox{0.5\linewidth+1.8cm}{!}{Tempra +6, Riflessi +5, Volontà +4}
	\item[\textbf{Imm. Danni:}] Fuoco
	\item[\textbf{Comp.:}] Consapevolezza +6
	\item[\textbf{Vulnerabilità:}] Luce
	\item[\textbf{Sensi:}] Scurovisione 36 m
	\item[\textbf{Linguaggi:}] comprende Abissale, Comune e Infernale ma non può parlare
	\item[\textbf{Sfida:}] 3 (700 PX)\smallskip
\end{description}

\emph{\textbf{Conferire Resistenza al Fuoco.}} Il destriero dell'incubo può conferire resistenza al danno da fuoco a chiunque lo cavalchi.

\emph{\textbf{Illuminazione.}} Il destriero da incubo irradia luce intensa in un raggio di 3 metri e luce fioca per 6 metri.

\textbf{Azioni}

\emph{\textbf{Zoccoli.} Attacco con arma da mischia}: +6 a colpire, portata 3 m, un bersaglio.

\emph{Colpisce:} 13 (2d8 + 4) danni contundenti più 7 (2d6) danni da fuoco.

\emph{\textbf{Passo Etereo.}} Il destriero da incubo e fino a tre creature consenzienti entro 3 metro da esso possono entrare magicamente nel Piano Etereo dal Piano Materiale e viceversa.

\textbf{Ecologia}\\
Ambiente: Qualsiasi\\
Organizzazione: Solitario\\
\textbf{Categoria Tesoro}: Nessuno\\
\textbf{Descrizione}\\
Gli incubi sono fiammeggianti messaggeri di morte. Permettono solo alle creature più malvagie di cavalcarli, e non sono mai soltanto cavalcature, ma collaborano nella distruzione provocata dai loro cavalieri.

\begin{enfasi}{L'inferno è vuoto, tutti i diavoli sono qui. (William Shakespeare, La Tempesta)}\end{enfasi}

\mostro{Diavolo Barbuto}
\noindent
\begin{description}[noitemsep, topsep=0pt, parsep=0pt, partopsep=0pt, leftmargin=0cm, labelwidth=2.2cm]
	\item[\textbf{Taglia/Tipo:}] Media diavolo, malvagio
	\item[\textbf{Caratt.:}] \resizebox{0.5\linewidth+1.8cm}{!}{For 3 Des 2 Cos 2 Int -1 Sag 0 Car 0}
	\item[\textbf{Punti Ferita:}] 70,  \textbf{Difesa:} 18,  \textbf{Iniziativa:} +2
	\item[\textbf{Movimento:}] 9 m
	\item[\textbf{Tiri Salvez.:}] \resizebox{0.5\linewidth+1.8cm}{!}{Tempra +5, Riflessi +5, Volontà +3}
	\item[\textbf{Res. Danni:}] Freddo; da arma non magica o non argentata
	\item[\textbf{Imm. Danni:}] Fuoco, Veleno
	\item[\textbf{Vulnerabilità:}] argento, Luce
	\item[\textbf{Sensi:}] Scurovisione 36 m
	\item[\textbf{Linguaggi:}] Infernale, telepatia 36 m
	\item[\textbf{Sfida:}] 3 (700 PX)\smallskip
\end{description}

\emph{\textbf{Resistenza alla Magia.}} Il diavolo ha +1d6 ai Tiri Salvezza contro incantesimi e altri effetti magici.

\emph{\textbf{Risoluto.}} Il diavolo non può essere spaventato finché riesce a vedere una creatura alleata entro 9 metri da lui.

\emph{\textbf{Vista del Diavolo.}} La Scurovisione del diavolo non è limitata dall'oscurità magica.

\textbf{Azioni}

\emph{\textbf{Multiattacco.}} Il diavolo effettua due attacchi: uno con la barba e uno con il falcione.

\emph{\textbf{Barba.} Attacco con arma da mischia}: +5 a colpire, portata 1 m, una creatura.

\emph{Colpisce:} 6 (1d8 + 2) danni perforanti e il bersaglio deve riuscire un Tiro Salvezza di Tempra DC 14 o restare avvelenato per 1 minuto. Mentre è avvelenato in questo modo, il bersaglio non può recuperare Punti Ferita. Il bersaglio può ripetere il Tiro Salvezza al termine di ciascun suo round, terminando l'effetto se riesce il Tiro Salvezza.

\emph{\textbf{Falcione.} Attacco con arma da mischia}: +6 a colpire, portata 3, un bersaglio.

\emph{Colpisce:} 8 (1d10 + 3) danni taglienti. Se il bersaglio è una creatura, ad esclusione di costrutti e non morti, deve riuscire un Tiro Salvezza su Tempra 15 o perdere 5 (1d10) Punti Ferita all'inizio di ciascun suo round a causa della ferita infernale. Ogni volta che il diavolo colpisce il bersaglio ferito con questo attacco, il danno inflitto dalla ferita aumenta di 5 (1d10). Qualsiasi creatura può effettuare due Azioni per bloccare la ferita con una prova riuscita di Saggezza (Pronto Soccorso) DC 12. La ferita si richiude anche nel caso in cui il bersaglio riceva della magia guaritrice.

\textbf{Reazione: \emph{Attacco d'opportunità}}: il diavolo effettua un attacco ad una creatura che attraversi o esca dalla sua portata di 1 metro.

\textbf{Ecologia}\\
Ambiente: Qualsiasi (Inferno)\\
Organizzazione: Solitario, coppia, squadra (3-10) o truppa (10-40)\\
\textbf{Categoria Tesoro}: Falcione, L\\
\textbf{Descrizione}\\
Guerrieri scelti delle legioni infernali, i diavoli barbuti, o barbazu, combattono selvaggiamente in nome dei loro signori infernali e in battaglia comandano orde brutali di dannati. Si radunano e si addestrano con i loro falcioni forgiati negli inferi, tra le volte del terzo girone dell'Inferno, Erebo, ma ritornano inevitabilmente nel primo girone, Averno, per servire al fianco del temibile signore Barbatos.

I barbazu amano effettuare attacchi di carica con i loro falcioni e cercano di mantenere una distanza di 3 metri tra loro ed i loro avversari, così che possono utilizzare le loro caratteristiche armi ad asta con la massima efficacia. In posizione eretta i diavoli barbuti sono alti più di 1,8 metri (sebbene la posizione accovacciata che tengono in battaglia li faccia spesso sembrare più bassi) e pesano più di 100 kg.


\begin{center}
	\includegraphics[width=0.9\linewidth]{immagini/Diavoli_giotto_2.png}

	\emph{Diavoli di Giotto}
\end{center}


\mostro{Diavolo delle Catene}
\noindent
\begin{description}[noitemsep, topsep=0pt, parsep=0pt, partopsep=0pt, leftmargin=0cm, labelwidth=2.2cm]
	\item[\textbf{Taglia/Tipo:}] Media diavolo, malvagio
	\item[\textbf{Caratt.:}] \resizebox{0.5\linewidth+1.8cm}{!}{For 4 Des 2 Cos 4 Int 0 Sag 1 Car 2}
	\item[\textbf{Punti Ferita:}] 165,  \textbf{Difesa:} 24,  \textbf{Iniziativa:} +2
	\item[\textbf{Movimento:}] 9 m
	\item[\textbf{Tiri Salvez.:}] \resizebox{0.5\linewidth+1.8cm}{!}{\resizebox{0.5\linewidth+1.8cm}{!}{Tempra +12, Riflessi +10, Volontà +9}}
	\item[\textbf{Res. Danni:}] Freddo; da arma non magica o non argentata
	\item[\textbf{Imm. Danni:}] Fuoco, Veleno
	\item[\textbf{Vulnerabilità:}] argento, Luce
	\item[\textbf{Sensi:}] Scurovisione 36 m
	\item[\textbf{Linguaggi:}] Infernale, telepatia 36 m
	\item[\textbf{Sfida:}] 8 (3900 PX)\smallskip
\end{description}

\emph{\textbf{Resistenza alla Magia.}} Il diavolo ha +1d6 ai Tiri Salvezza contro incantesimi e altri effetti magici.

\emph{\textbf{Vista del Diavolo.}} La Scurovisione del diavolo non è limitata dall'oscurità magica.

\textbf{Azioni}

\emph{\textbf{Multiattacco.}} Il diavolo effettua due attacchi con la catena.

\emph{\textbf{Catena.} Attacco con arma da mischia}: +9 a colpire, portata 3 m, un bersaglio.

\emph{Colpisce:} 11 (2d6 + 4) danni taglienti. Il bersaglio è afferrato (DC 14 per fuggire) se il diavolo non sta già afferrando un'altra creatura. Fino al termine dell'afferrare, il bersaglio subisce 7 (2d6) danni perforanti all'inizio di ciascun suo round.

\emph{\textbf{Animare Catene (Ricarica dopo un 1 ora).}} Fino a quattro catene che il diavolo possa vedere e si trovano entro 18 metri da lui producono dei bordi affilati e si animano sotto il controllo del diavolo, purché quelle catene non siano né indossate né trasportate da qualcun altro.

Ogni catena animata è un oggetto con Difesa 20, 20 Punti Ferita, resistenza ai danni perforanti, e immunità ai danni da suono. Quando il diavolo usa Multiattacco durante il suo round, può usare ciascuna catena animata per effettuare un ulteriore attacco di catena. Una catena animata può afferrare una creatura per conto proprio ma non può effettuare attacchi mentre afferra. Una catena animata ritorna al suo stato inanimato se viene ridotta a 0 Punti Ferita o se il diavolo è reso inabile o muore.

\textbf{Reazione: \emph{Maschera Snervante.}} Quando una creatura che il diavolo può vedere inizia il proprio round entro 9 metri dal diavolo, il diavolo può creare un'illusione per assomigliare all'amore perduto o un acerrimo rivale di quella creatura. Se la creatura può vedere il diavolo, deve riuscire un Tiro Salvezza di Volontà DC 21 o rimanere spaventata fino al termine del suo round.

\textbf{Reazione: \emph{Attacco d'opportunità}}: il diavolo effettua un attacco ad una creatura che attraversi o esca dalla sua portata di 3 metri.

\emph{\textbf{Arrabbiato:}} il Diavolo delle Catene agita le catene davanti a se. Fino alla fine del combattimento la Difesa è 27. Costa 1 Azione a round mantenere l'effetto.

\textbf{Ecologia}\\
Ambiente: Qualsiasi\\
Organizzazione: Solitario, coppia, anello (3-6) o catena (7-20)\\
\textbf{Categoria Tesoro}: R\\
\textbf{Descrizione}\\
Spesso classificati dai profani tra le fila dei diavoli infernali, i Diavolo delle Catene non sono veri diavoli. Anche se alcuni sono noti per vivere all'Inferno, essi esistono al di fuori delle gerarchie stabilite dagli dei degli inferi e dai suoi arcidiavoli e a volte si possono trovare su altri piani, in particolare sul Piano delle Ombre. Molti suggeriscono che siano nativi dell'Inferno che esisteva prima dell'avvento della stirpe diabolica, anche se altri ipotizzano che siano stati portati sul piano da qualche sadica potenza. Indipendentemente dalle loro origini vagano per i piani assecondano il loro desiderio di causare e ricevere sofferenza, ricercando il dolore attraverso violenti rapimenti e sadiche depravazioni.

\mostro{Diavolo Cornuto}
\noindent
\begin{description}[noitemsep, topsep=0pt, parsep=0pt, partopsep=0pt, leftmargin=0cm, labelwidth=2.2cm]
	\item[\textbf{Taglia/Tipo:}] Grande diavolo, malvagio
	\item[\textbf{Caratt.:}] \resizebox{0.5\linewidth+1.8cm}{!}{For 6 Des 3 Cos 5 Int 1 Sag 3 Car 3}
	\item[\textbf{Punti Ferita:}] 224,  \textbf{Difesa:} 29,  \textbf{Iniziativa:} +3
	\item[\textbf{Movimento:}] 6 m, volo 18 m
	\item[\textbf{Tiri Salvez.:}] \resizebox{0.5\linewidth+1.8cm}{!}{\resizebox{0.5\linewidth+1.8cm}{!}{Tempra +16, Riflessi +14, Volontà +14}}
	\item[\textbf{Res. Danni:}] Freddo;
	\item[\textbf{Imm. Danni:}] Fuoco, Veleno, armi +1
	\item[\textbf{Vulnerabilità:}] argento, Luce
	\item[\textbf{Sensi:}] Scurovisione 36 m
	\item[\textbf{Linguaggi:}] Infernale, telepatia 36 m
	\item[\textbf{Sfida:}] 11 (7200 PX)\smallskip
\end{description}

\emph{\textbf{Resistenza alla Magia.}} Il diavolo ha +1d6 ai Tiri Salvezza contro incantesimi e altri effetti magici.

\emph{\textbf{Vista del Diavolo.}} La Scurovisione del diavolo non è limitata dall'oscurità magica.

\emph{\textbf{Sguardo del comandante.}} i diavoli a più bassa Sfida entro 9 metri prendono +1 al Tiro per Colpire, Difesa e Tiri Salvezza. Non è cumulabile.

\textbf{Azioni}

\emph{\textbf{Multiattacco.}} Il diavolo effettua tre attacchi da mischia: due con il forcone e uno con il pungiglione. Può usare Scagliare Fiamma al posto di qualsiasi attacco da mischia.

\emph{\textbf{Coda.} Attacco con arma da mischia}: +10 a colpire, portata 3 m, un bersaglio.

\emph{Colpisce:} 10 (1d8 + 6) danni perforanti. Se il bersaglio è una creatura, ad esclusione di costrutti e non morti, deve riuscire un Tiro Salvezza su Tempra 25 o Sanguinare 10 (3d6). Ogni volta che il diavolo ferisce il bersaglio con questo attacco, il danno inflitto dal Sanguinamento aumenta di 10 (3d6).

\emph{\textbf{Forcone.} Attacco con arma da mischia}: +11 colpire, portata 3 m, un bersaglio.

\emph{Colpisce:} 15 (2d8 + 6) danni perforanti.

\emph{\textbf{Pungiglione.} Attacco con arma da mischia}: +9 a colpire, portata 3 m, un bersaglio.

\emph{Colpisce:} 13 (2d8 + 4) danni perforanti più 17 (5d6) danni da veleno, e il bersaglio deve riuscire un Tiro Salvezza di Tempra DC 24, o restare avvelenato, -1 Forza e Destrezza, per 1 minuto. Il bersaglio può ripetere il Tiro Salvezza al termine di ciascun suo round, terminando l'effetto se lo riesce.

\emph{\textbf{Scagliare Fiamma.} Attacco con incantesimo a Distanza}: +10 a colpire, gittata 45 m, un bersaglio.

\emph{Colpisce:} 14 (4d6) danni da fuoco. Se il bersaglio è un oggetto infiammabile che non sia indossato o trasportato, prende fuoco.

\textbf{Reazione: \emph{Attacco d'opportunità}}: il diavolo effettua un attacco ad una creatura che attraversi o esca dalla sua portata di 3 metri.

\emph{\textbf{Arrabbiato:}} il Diavolo Cornuto risucchia la vita che i nemici stanno perdendo. Fino alla fine del round successivo recupera tutti i Punti Ferita persi da Sanguinamento da ferite da lui causate.

\textbf{Ecologia}\\
Ambiente: Qualsiasi (Inferno)\\
Organizzazione: Solitario, coppia o stormo (3-10)\\
\textbf{Categoria Tesoro}: Catena Chiodata Sacrilega +1, P\\
\textbf{Descrizione}\\
Tra i più letali guerrieri degli arcidiavoli ed abili comandanti dei diavoli minori, i diavoli cornuti divulgano le regole dell'Inferno dovunque passano. Questi diavoli maggiori sono addestrati, forgiati e riforgiati per essere tra i più implacabili ed obbedienti guerrieri del multiverso. I diavoli cornuti delle truppe degli eserciti infernali sono noti come cornugon, mentre i più grandi tra loro sono chiamati malebranche.

Un diavolo cornuto tipico raggiunge la ragguardevole altezza di 2,7 metri, è dotato di ali con un'apertura di 4,2 metri, e pesa 350 kg.

\medskip

\begin{enfasi}{Il SIGNORE disse a Satana: "Da dove vieni?" Satana rispose al SIGNORE: "Dal percorrere la terra e dal passeggiare per essa". Giobbe 1,6-12}\end{enfasi}

\mostro{Diavolo della Fossa}
\noindent
\begin{description}[noitemsep, topsep=0pt, parsep=0pt, partopsep=0pt, leftmargin=0cm, labelwidth=2.2cm]
	\item[\textbf{Taglia/Tipo:}] Grande diavolo, malvagio
	\item[\textbf{Caratt.:}] \resizebox{0.5\linewidth+1.8cm}{!}{For 8 Des 2 Cos 7 Int 6 Sag 4 Car 7}
	\item[\textbf{Punti Ferita:}] 403,  \textbf{Difesa:} 40,  \textbf{Iniziativa:} +6
	\item[\textbf{Movimento:}] 9 m, volo 18 m
	\item[\textbf{Tiri Salvez.:}] \resizebox{0.5\linewidth+1.8cm}{!}{\resizebox{0.5\linewidth+1.8cm}{!}{Tempra +27, Riflessi +22, Volontà +24}}
	\item[\textbf{Res. Danni:}] Freddo;
	\item[\textbf{Imm. Danni:}] Fuoco, Veleno, armi +2
	\item[\textbf{Vulnerabilità:}] Luce
	\item[\textbf{Sensi:}] visione del vero 36 m
	\item[\textbf{Linguaggi:}] Infernale, telepatia 36 m
	\item[\textbf{Sfida:}] 20 (25000 PX)\smallskip
\end{description}

\emph{\textbf{Arma Magica.}} Gli attacchi con arma del diavolo della fossa sono magici.

\emph{\textbf{Aura di Paura.}} Qualsiasi creatura ostile al diavolo che inizi il suo round entro 6 metri da esso, deve effettuare un Tiro Salvezza su Volontà DC 35, a meno che il diavolo non sia inabile. Se fallisce il Tiro Salvezza, la creatura è spaventata fino all'inizio del suo prossimo round. Se il Tiro Salvezza della creatura riesce, la creatura è immune all'Aura di Paura del diavolo per le successive 24 ore.

\emph{\textbf{Incantesimi Innati.}} La caratteristica da incantatore diavolo della fossa è il Carisma. Il diavolo della fossa può lanciare questi incantesimi in maniera innata, senza bisogno di componenti materiali:

A volontà: \emph{\hyperlink{Individuazione del Magico}{Individuazione del Magico}, \hyperlink{Palla di Fuoco}{Palla di Fuoco}}

3/giorno ciascuno: \emph{blocca mostri, \hyperlink{Muro di Fuoco}{Muro di Fuoco}}

\emph{\textbf{Resistenza alla Magia.}} Il diavolo ha +1d6 ai Tiri Salvezza contro incantesimi e altri effetti magici.

\textbf{Azioni}

\emph{\textbf{Multiattacco.}} Il diavolo effettua quattro attacchi: uno con il morso, uno con l'artiglio, uno con la mazza e uno con la coda.

\emph{\textbf{Artiglio.} Attacco con arma da mischia}: +16 a colpire, portata 3 m, un bersaglio.

\emph{Colpisce:} 17 (2d8 + 8) danni taglienti, 3/20 danni da Sanguinamento.

\emph{\textbf{Coda.} Attacco con arma da mischia}: +16 a colpire, portata 3 m, un bersaglio.

\emph{Colpisce:} 24 (3d10 + 8) danni contundenti.

\emph{\textbf{Mazza.} Attacco con arma da mischia}: +17 a colpire, portata 3 m, un bersaglio.

\emph{Colpisce:} 15 (2d6 + 8) danni contundenti più 21 (6d6) danni da fuoco.

\emph{\textbf{Morso.} Attacco con arma da mischia}: +15 a colpire, portata 1 m, un bersaglio.

\emph{Colpisce:} 22 (4d6 + 8) danni perforanti. Il bersaglio deve riuscire un Tiro Salvezza di Tempra DC 33 o restare avvelenato. Mentre è avvelenato in questo modo, il bersaglio non può recuperare Punti Ferita, e subisce 21 (6d6) danni da veleno all'inizio di ciascun suo round. Il bersaglio avvelenato può ripetere il Tiro Salvezza al termine di ciascun suo round, terminando l'effetto su di sé.

\textbf{Reazione: \emph{Attacco d'opportunità}}: il diavolo effettua un attacco ad una creatura che attraversi o esca dalla sua portata di 3 metri.

\textbf{Ecologia}\\
Ambiente: Qualsiasi (Inferno)\\
Organizzazione: Solitario, coppia o concilio (3-9)\\
\textbf{Categoria Tesoro}: G\\
\textbf{Descrizione}\\
I diavoli della fossa sono potenti sovrani infernali, generali delle armate dell'Inferno e consiglieri degli arcidiavoli. Con corpi massicci e intelletti malvagi, dominano distese infernali e sottomettono mondi mortali. Alti oltre 4 metri e pesanti più di 500 kg, sono corazzati e dotati di ali imponenti.

I diavoli della fossa radunano eserciti, trasformando i lemure in veri diavoli e puntano a conquistare semipiani e mondi mortali vulnerabili. Obbediscono alla gerarchia infernale ma possono deporre padroni indegni. Solo i più potenti incantatori osano evocarli rischiando la dannazione eterna.

\mostro{Diavolo del Ghiaccio}
\noindent
\begin{description}[noitemsep, topsep=0pt, parsep=0pt, partopsep=0pt, leftmargin=0cm, labelwidth=2.2cm]
	\item[\textbf{Taglia/Tipo:}] Grande diavolo, malvagio
	\item[\textbf{Caratt.:}] \resizebox{0.5\linewidth+1.8cm}{!}{For 5 Des 2 Cos 4 Int 4 Sag 2 Car 4}
	\item[\textbf{Punti Ferita:}] 278,  \textbf{Difesa:} 32,  \textbf{Iniziativa:} +4
	\item[\textbf{Movimento:}] 12 m
	\item[\textbf{Tiri Salvez.:}] \resizebox{0.5\linewidth+1.8cm}{!}{\resizebox{0.5\linewidth+1.8cm}{!}{Tempra +18, Riflessi +16, Volontà +16}}
	\item[\textbf{Imm. Danni:}] Freddo, Fuoco, Veleno, armi +1
	\item[\textbf{Vulnerabilità:}] argento, Luce
	\item[\textbf{Sensi:}] Vista Cieca 18 m, Scurovisione 36 m
	\item[\textbf{Linguaggi:}] Infernale, telepatia 36 m
	\item[\textbf{Sfida:}] 14 (11500 PX)\smallskip
\end{description}

\emph{\textbf{Resistenza alla Magia.}} Il diavolo ha +1d6 ai Tiri Salvezza contro incantesimi e altri effetti magici.

\emph{\textbf{Vista del Diavolo.}} La Scurovisione del diavolo non è limitata dall'oscurità magica.

\textbf{Azioni}

\emph{\textbf{Multiattacco.}} Il diavolo effettua tre attacchi: uno con il morso, uno con gli artigli e uno con la coda. In alternativa effettua due attacchi: uno con la coda e uno con lancia.

\emph{\textbf{Artigli.} Attacco con arma da mischia}: +12 a colpire, portata 1 m, un bersaglio.

\emph{Colpisce:} 10 (2d4 + 5) danni taglienti più 10 (3d6) danni da freddo, 1 danno da Sanguinamento.

\emph{\textbf{Coda.} Attacco con arma da mischia}: +12 a colpire, portata 3 m, un bersaglio.

\emph{Colpisce:} 12 (2d6 + 5) danni contundenti più 10 (3d6) danni da freddo.

\emph{\textbf{Lancia di Ghiaccio.} Attacco con arma da mischia}: +12 a colpire, portata 3 m, un bersaglio.

\emph{Colpisce:} 14 (2d8 + 5) danni perforanti più 10 (3d6) danni da freddo. Se il bersaglio è una creatura, deve riuscire un Tiro Salvezza su Tempra DC 26, o avere per 1 minuto la velocità ridotta di 3 metri; durante ciascun suo round può effettuare solo un'Azione o un'Azione Immediata, ma non entrambe; non può effettuare reazioni. Il bersaglio può ripetere il Tiro Salvezza al termine di ciascun suo round, terminando l'effetto su di sé in caso di successo.

\emph{\textbf{Morso.} Attacco con arma da mischia}: +12 a colpire, portata 1 m, un bersaglio.

\emph{Colpisce:} 12 (2d6 + 5) danni perforanti più 10 (3d6) danni da freddo. TS su Tempra DC 18 o Rallentato 1/1r.

\emph{\textbf{Muro di Ghiaccio (Ricarica 6).}} Il diavolo forma magicamente un muro di ghiaccio opaco su di una superficie solida che possa vedere entro 18 metri da lui. Il muro è spesso 30 centimetri e largo fino a 9 metri per un massimo di 3 metri di altezza, oppure una cupola semisferica di massimo 6 metri di diametro. Quando la parete appare, ogni creatura nel suo spazio viene spinta fuori da esso tramite la via più breve. La creatura sceglie su quale lato del muro finire, a meno che la creatura non sia inabile. La creatura poi effettua un Tiro Salvezza di Riflessi DC 25, subendo 35 (10d6) danni da freddo se lo fallisce, o la metà di questi danni se lo riesce.

Il muro rimane per 1 minuto o finché il diavolo non è reso inabile o muore. Il muro può essere danneggiato e bucato; ogni sezione di 3 metri ha Difesa 5, 30 Punti Ferita, vulnerabilità al danno da fuoco, e Immune al Danno da acido, freddo, da Vuoto e da veleno. Se una sezione viene distrutta, lascia una patina di aria gelida nello spazio che occupava prima il muro. Ogni volta che una creatura finisce per muoversi attraverso quest'aria gelida durante un round, consenziente o meno, deve effettuare un Tiro Salvezza di Tempra DC 25, subendo 17 (5d6)danni da freddo se lo fallisce, o la metà di questi danni se lo riesce. L'aria gelida si dissipa quando il resto del muro svanisce.

\textbf{Reazione: \emph{Attacco d'opportunità}}: il diavolo effettua un attacco ad una creatura che attraversi o esca dalla sua portata di 1 metro.

\emph{\textbf{Arrabbiato:}} il Diavolo di Ghiaccio punta al cuore del nemico e cerca di strapparlo. La creatura, entro 1 metro, deve fare un Tiro Salvezza su Tempra DC 26 od avere il cuore strappato.

\textbf{Ecologia}\\
Ambiente: Qualsiasi (Inferno)\\
Organizzazione: Solitario, squadra (2-3), concilio (4-10) o contingente (1-3 diavoli del ghiaccio, 2-6 diavoli cornuti e 1-4 diavoli d'ossa\\
\textbf{Categoria Tesoro}: Lancia Gelida +1, R\\
\textbf{Descrizione}\\
Strateghi illuminati delle armate dell'Inferno, gli insettoidi diavoli del ghiaccio sono tra le menti più ingegnose e crudeli dell'infermo. Un diavolo del ghiaccio nasconde nel suo petto un cuore ghiacciato trafugato ad un mortale, che gli permette di prendere decisioni libero da emozioni. Nati nel girone ghiacciato di Cocito, il settimo girone infernale, la maggior parte dei diavoli del ghiaccio migra a Caina, l'ottavo girone, dove complotta per dannare il mondo. Sebbene abbiano le sembianze più aliene e mostruose tra tutti i diavoli, a pochi altri viene accordato un maggiore rispetto.

In combattimento manda avanti i suoi sottoposti, così da poter valutare le tattiche, i punti di forza e le debolezze dell'avversario nelle retrovie, e fornire loro supporto con le sue capacità magiche, evitando di coglierli nell'area di effetto dei suoi incantesimi: atteggiamento non dovuto ad un senso di cameratismo, bensì alla fredda e logica verità che i suoi alleati possono sopravvivere più a lungo in uno scontro se non sono esposti a fuoco amico.

I Diavoli del Ghiaccio sono alti 3,6 metri e pesano approssimativamente 350 kg.

\mostro{Diavolo d'Ossa}
\noindent
\begin{description}[noitemsep, topsep=0pt, parsep=0pt, partopsep=0pt, leftmargin=0cm, labelwidth=2.2cm]
	\item[\textbf{Taglia/Tipo:}] Grande diavolo, malvagio
	\item[\textbf{Caratt.:}] \resizebox{0.5\linewidth+1.8cm}{!}{For 4 Des 3 Cos 4 Int 1 Sag 2 Car 3}
	\item[\textbf{Punti Ferita:}] 184,  \textbf{Difesa:} 27,  \textbf{Iniziativa:} +3
	\item[\textbf{Movimento:}] 12 m, volo 12 m
	\item[\textbf{Tiri Salvez.:}] \resizebox{0.5\linewidth+1.8cm}{!}{\resizebox{0.5\linewidth+1.8cm}{!}{Tempra +13, Riflessi +12, Volontà +11}}
	\item[\textbf{Comp.:}] Ingannare +7, Percepire Emozioni +6
	\item[\textbf{Res. Danni:}] Freddo; da arma non magica o non argentata
	\item[\textbf{Imm. Danni:}] Fuoco, Veleno
	\item[\textbf{Vulnerabilità:}] argento, Luce
	\item[\textbf{Sensi:}] Scurovisione 36 m
	\item[\textbf{Linguaggi:}] Infernale, telepatia 36 m
	\item[\textbf{Sfida:}] 9 (5000 PX)\smallskip
\end{description}

\emph{\textbf{Resistenza alla Magia.}} Il diavolo ha +1d6 ai Tiri Salvezza contro incantesimi e altri effetti magici.

\emph{\textbf{Vista del Diavolo.}} La Scurovisione del diavolo non è limitata dall'oscurità magica.

\textbf{Azioni}

\emph{\textbf{Multiattacco.}} Il diavolo effettua tre attacchi: due con gli artigli e uno con il pungiglione oppure uno con la sua arma inastata uncinata e uno con il pungiglione.

\emph{\textbf{Arma Inastata Uncinata.} Attacco con arma da mischia}: +11 a colpire, portata 3 m, un bersaglio.

\emph{Colpisce:} 17 (2d12 + 4) danni perforanti. Se il bersaglio è una creatura di taglia Enorme o inferiore, può essere afferrata (DC 14 per fuggire). Fino al termine dell'afferrare, il diavolo non può usare la sua arma inastata su di un altro bersaglio.

\emph{\textbf{Artiglio.} Attacco con arma da mischia}: +9 a colpire, portata 3 m, un bersaglio.

\emph{Colpisce:} 8 (1d8 + 4) danni taglienti, 1 danno da Sanguinamento.

\emph{\textbf{Pungiglione.} Attacco con arma da mischia}: +9 a colpire, portata 3 m, un bersaglio.

\emph{Colpisce:} 13 (2d8 + 4) danni perforanti più 17 (5d6) danni da veleno, e il bersaglio deve riuscire un Tiro Salvezza di Tempra DC 21, o restare avvelenato, -1 Forza e Destrezza, per 1 minuto. Il bersaglio può ripetere il Tiro Salvezza al termine di ciascun suo round, terminando l'effetto se lo riesce.

\emph{\textbf{Arrabbiato:}} il Diavolo d'Ossa attacca tutte le creature intorno a lui con l'arma inastata. Tutte le creature ne raggio di 3 metri subiscono un attacco di Arma Inastata Uncinata, senza essere afferrati. Costo 2 Azioni. Il Diavolo d'ossa può decidere di diventare invisibile come sotto l'incantesimo di Invisibilità superiore. 2 Azioni.

\textbf{Ecologia}\\
Ambiente: Qualsiasi (Inferno)\\
Organizzazione: Solitario, squadra (2-3), concilio (4-10) o contingente (1-3 diavoli del ghiaccio, 2-6 diavoli cornuti e 1-4 diavoli d'ossa\\
\textbf{Categoria Tesoro}: I\\
\textbf{Descrizione}\\
I diavoli d'ossa sono inquisitori delle razze diaboliche, noti per la loro passione per la tortura di mortali, anime e altri diavoli. Nati nelle paludi di Stigia, nel quinto girone dell'Inferno, fanno rispettare gli ordini degli arcidiavoli con devozione assoluta.

I diavoli d'ossa viaggiano spesso fino al piano mortale per servire malvagi incantatori, recuperando informazioni preziose. Alti 2,7 metri e pesanti oltre 200 kg, con coda e ali terrificanti, sono imponenti e temuti.

\mostro{Diavolo Spinoso}
\noindent
\begin{description}[noitemsep, topsep=0pt, parsep=0pt, partopsep=0pt, leftmargin=0cm, labelwidth=2.2cm]
	\item[\textbf{Taglia/Tipo:}] Piccola diavolo, malvagio
	\item[\textbf{Caratt.:}] \resizebox{0.5\linewidth+1.8cm}{!}{For 0 Des 2 Cos 1 Int 0 Sag 2 Car -1}
	\item[\textbf{Punti Ferita:}] 51,  \textbf{Difesa:} 16,  \textbf{Iniziativa:} +2
	\item[\textbf{Movimento:}] 6 m, volo 12 m
	\item[\textbf{Tiri Salvez.:}] \resizebox{0.5\linewidth+1.8cm}{!}{Tempra +3, Riflessi +4, Volontà +4}
	\item[\textbf{Res. Danni:}] Freddo; da arma non magica o non argentata
	\item[\textbf{Imm. Danni:}] Fuoco, Veleno
	\item[\textbf{Vulnerabilità:}] argento, Luce
	\item[\textbf{Sensi:}] Scurovisione 36 m
	\item[\textbf{Linguaggi:}] Infernale, telepatia 36 m
	\item[\textbf{Sfida:}] 2 (450 PX)\smallskip
\end{description}

\emph{\textbf{Resistenza alla Magia.}} Il diavolo ha +1d6 ai Tiri Salvezza contro incantesimi e altri effetti magici.

\emph{\textbf{Sorvolare.}} Il diavolo non provoca attacchi di opportunità quando vola via dalla portata di un nemico.

\emph{\textbf{Spine Limitate.}} Il diavolo possiede dodici spine caudali. Le spine usate ricrescono a mezzanotte.

\emph{\textbf{Vista del Diavolo.}} La Scurovisione del diavolo non è limitata dall'oscurità magica.

\textbf{Azioni}

\emph{\textbf{Multiattacco.}} Il diavolo effettua due attacchi: uno con il morso e uno con il suo forcone o due con le sue spine caudali.

\emph{\textbf{Forcone.} Attacco con arma da mischia}: +4 a colpire, portata 1 m, un bersaglio.

\emph{Colpisce:} 3 (1d6) danni perforanti.

\emph{\textbf{Morso.} Attacco con arma da mischia}: +4 a colpire, portata 1 m, un bersaglio.

\emph{Colpisce:} 5 (2d4) danni taglienti.

\emph{\textbf{Spina Caudale.} Attacco con arma a Distanza}: +4 a colpire, gittata 6m, un bersaglio.

\emph{Colpisce:} 4 (1d4 + 2) danni perforanti più 3 (1d6) danni da fuoco.

\textbf{Ecologia}\\
Ambiente: Qualsiasi (Inferno)\\
Organizzazione: Solitario, coppia, gruppo (3-5) o plotone (6-11)\\
\textbf{Categoria Tesoro}: J\\
\textbf{Descrizione}\\
Sentinelle delle volte dell'Inferno, carcerieri delle anime più nere e armi viventi delle forge infernali. Un Diavolo Spinoso ama sentire il sangue caldo sulle proprie spine e preferisce gettarsi nella mischia quando gli viene offerta l'opportunità di combattere.
I Diavoli Spinoso sono collezionisti ed organizzatori. Se lasciati agire liberamente, nei nascondigli di questi diavoli spesso fanno mostra i trofei trafitti di vecchie vittime.
La maggior parte dei diavoli spinosi è alta dai 2,1 metri in su e pesa 150 kg, sebbene i loro corpi asciutti e muscolosi sembrino più grossi per via degli spuntoni in continua crescita che fuoriescono dai loro corpi, taglienti come lame.

\mostro{Erinni}
\noindent
\begin{description}[noitemsep, topsep=0pt, parsep=0pt, partopsep=0pt, leftmargin=0cm, labelwidth=2.2cm]
	\item[\textbf{Taglia/Tipo:}] Media diavolo, malvagio
	\item[\textbf{Caratt.:}] \resizebox{0.5\linewidth+1.8cm}{!}{For 4 Des 3 Cos 4 Int 2 Sag 2 Car 4}
	\item[\textbf{Punti Ferita:}] 240,  \textbf{Difesa:} 31,  \textbf{Iniziativa:} +3
	\item[\textbf{Movimento:}] 9 m, volo 18 m
	\item[\textbf{Tiri Salvez.:}] \resizebox{0.5\linewidth+1.8cm}{!}{\resizebox{0.5\linewidth+1.8cm}{!}{Tempra +16, Riflessi +15, Volontà +14}}
	\item[\textbf{Res. Danni:}] Freddo; da arma non magica o non argentata
	\item[\textbf{Imm. Danni:}] Fuoco, Veleno
	\item[\textbf{Vulnerabilità:}] argento, Luce
	\item[\textbf{Sensi:}] visione del vero 36 m
	\item[\textbf{Linguaggi:}] Infernale, telepatia 36 m
	\item[\textbf{Sfida:}] 12 (8400 PX)\smallskip
\end{description}

\emph{\textbf{Armi Diaboliche.}} Gli attacchi con arma dell'erinni sono magici e infliggono 13 (3d8) danni da veleno aggiuntivi quando colpiscono (già incluso negli attacchi).

\emph{\textbf{Resistenza alla Magia.}} L'erinni ha +1d6 ai Tiri Salvezza contro incantesimi e altri effetti magici.

\textbf{Azioni}

\emph{\textbf{Multiattacco.}} L'erinni effettua tre attacchi.

\emph{\textbf{Spada Lunga.} Attacco con arma da mischia}: +11 a colpire, portata 1 m, un bersaglio.

\emph{Colpisce:} 8 (1d8 + 4) danni taglienti, o 9 (1d10 + 4) danni taglienti se usata con due mani, più 13 (3d8) danni da veleno.

\emph{\textbf{Arco Lungo.} Attacco con arma a Distanza}: +11 a colpire, gittata 45m, un bersaglio.

\emph{Colpisce:} 7 (1d8 + 4) danni perforanti più 13 (3d8) danni da veleno, e il bersaglio deve riuscire un Tiro Salvezza di Tempra DC 25 o restare avvelenato, -1 Forza e Destrezza. Il veleno rimane finché non viene rimosso da un incantesimo \emph{ristorazione inferiore} o simile.

\textbf{Reazione: \emph{Attacco d'opportunità}}: il diavolo effettua un attacco ad una creatura che attraversi o esca dalla sua portata di 1 metro.

\textbf{Reazione: \emph{Parata.}} L'erinni somma 4 alla sua Difesa contro un attacco da mischia che lo colpirebbe. Per farlo, l'erinni deve poter vedere il suo attaccante e impugnare un'arma da mischia.

\emph{\textbf{Arrabbiato:}} L'Erinni incanala la sua energia magica in un attacco. Il bersaglio dell'attacco viene colpito da una infernale fiamma che causa 12d6 di danno da Vuoto. Tiro Salvezza DC 25 Riflessi per dimezzare. Costa 2 Azioni.

\textbf{Ecologia}\\
Ambiente: Qualsiasi (Inferno)\\
Organizzazione: Solitario o trio\\
\textbf{Categoria Tesoro}: Arco Lungo Composito Infuocato +1 [Forza +5], corda, Spada Lunga+1\\
\textbf{Descrizione}\\
Le erinni, note anche come Caduti, Ali Cineree e Furie, sono diavoli che insultano la loro forma angelica con la loro sete di vendetta e giustizia sanguinosa. Volteggiano sopra i cornicioni di Dite, il secondo girone dell'Inferno, sempre pronte alla battaglia per difendere l'inferno o per i capricci dei loro signori diabolici.

Questi angeli bellissimi e oscuri accrescono la loro sensualità con cicatrici e lividi, ma preferiscono risolvere i problemi con violenza rapida e atroce. Utilizzano corde viventi fatte dei loro capelli per intralciare e sollevare i nemici, prolungando le loro sofferenze.

Le erinni sono alte circa 1,8 metri, pesano 70 kg e hanno ali nere con un'apertura di oltre 3 metri. Sono abili nel mantenere i nemici in vita per prolungare il tormento, e le più potenti possono far perdurare le sofferenze anche dopo la morte del soggetto.

\mostro{Imp}
\noindent
\begin{description}[noitemsep, topsep=0pt, parsep=0pt, partopsep=0pt, leftmargin=0cm, labelwidth=2.2cm]
	\item[\textbf{Taglia/Tipo:}] Minuscola diavolo, mutaforma, malvagio
	\item[\textbf{Caratt.:}] \resizebox{0.5\linewidth+1.8cm}{!}{For -2 Des 3 Cos 1 Int 0 Sag 1 Car 2}
	\item[\textbf{Punti Ferita:}] 33,  \textbf{Difesa:} 16,  \textbf{Iniziativa:} +3
	\item[\textbf{Movimento:}] 6 m, volo 12 m (6 m in forma di ratto; 6 m, volo 18 m in forma di corvo; 6 m, scalata 6 m in forma di ragno)
	\item[\textbf{Tiri Salvez.:}] \resizebox{0.5\linewidth+1.8cm}{!}{Tempra +3, Riflessi +4, Volontà +3}
	\item[\textbf{Comp.:}] Furtività +5, Ingannare +4, Percepire Emozioni +3
	\item[\textbf{Res. Danni:}] Freddo; da arma non magica o non argentata
	\item[\textbf{Imm. Danni:}] Fuoco, Veleno
	\item[\textbf{Vulnerabilità:}] argento, Luce
	\item[\textbf{Sensi:}] Scurovisione 36 m
	\item[\textbf{Linguaggi:}] Infernale, Comune
	\item[\textbf{Sfida:}] 1 (200 PX)\smallskip
\end{description}

\emph{\textbf{Mutaforma.}} Il diavolo può usare una Azione per trasformarsi in una forma bestiale da ratto, corvo o ragno, o per tornare alla sua vera forma. Le sue statistiche sono le stesse in tutte le forme, sebbene gli attacchi possano variare per alcune di esse. Qualsiasi equipaggiamento stia indossando o trasportando non viene trasformato. Alla morte ritorna alla sua vera forma.

\emph{\textbf{Resistenza alla Magia.}} Il diavolo ha +1d6 ai Tiri Salvezza contro incantesimi e altri effetti magici.

\emph{\textbf{Vista del Diavolo.}} La Scurovisione del diavolo non è limitata dall'oscurità magica.

\textbf{Azioni}

\emph{\textbf{Pungiglione (Morso in Forma di Bestia).} Attacco con arma da mischia}: +4 a colpire, portata 1 m, una creatura.

\emph{Colpisce:} 5 (1d4 + 3) danni perforanti e il bersaglio deve effettuare un Tiro Salvezza di Tempra DC 12, subendo 10 (3d6) danni da veleno se lo fallisce, o la metà di questi danni se lo riesce.

\emph{\textbf{Invisibilità.}} Il diavolo resta invisibile finché non attacca o termina la sua concentrazione. Qualsiasi cosa che il diavolo stia trasportando o indossando, resta invisibile finché rimane in contatto con il diavolo.

\textbf{Ecologia}\\
Ambiente: Qualsiasi (Inferno)\\
Organizzazione: Solitario, coppia o stormo (3-10)\\
\textbf{Categoria Tesoro}: K\\
\textbf{Descrizione}\\
Nati direttamente dalle fosse dell'Inferno, gli imp sono i diavoli meno potenti, anche se queste crudeli ed invadenti creature svolgono un ruolo importante nella corruzione delle anime mortali. Libere dalle gerarchie e dai doveri delle armate infernali, gli imp si dilettano ad ogni opportunità di viaggiare fino al Piano Materiale e di tentare astutamente i mortali, spingendoli a compiere atti sempre più depravati.

Volontariamente al servizio di incantatori nel ruolo di famigli, gli imp recitano la parte dei fedeli servitori, offrendo spesso ai loro padroni astuti consigli ed infernali intuizioni. In realtà, gli imp operano per inviare anime all'Inferno, accertandosi che l'anima del loro padrone, insieme a molte altre, sia destinata alla dannazione dopo la morte.

Gli imp variano molto in aspetto, in un ampio spettro di tratti bestiali e grotteschi, sebbene molti di essi abbiano la forma di un umanoide alato dalla pelle rossiccia, con lineamenti bulbosi. Il tipico imp è alto solamente 60 centimetri, ha un'apertura alare di 90 centimetri e pesa 5 kg.

Diversamente dagli altri diavoli, gli imp si ritrovano spesso liberi e soli nel Piano Materiale, in particolare dopo che sono stati evocati per servire come famigli ed i loro padroni sono morti (spesso, indirettamente, a causa delle macchinazioni dell'imp stesso). Senza alcun mezzo per poter fare ritorno a casa questi imp, liberi da ogni legame con padroni arcani, possono diventare pericolosi seccatori o persino porsi a capo di piccole tribù di sanguinosi umanoidi, quali Gablin o Coboldi.

\mostro{Lemure}
\noindent
\begin{description}[noitemsep, topsep=0pt, parsep=0pt, partopsep=0pt, leftmargin=0cm, labelwidth=2.2cm]
	\item[\textbf{Taglia/Tipo:}] Media diavolo, malvagio
	\item[\textbf{Caratt.:}] \resizebox{0.5\linewidth+1.8cm}{!}{For 0 Des -3 Cos 0 Int -5 Sag 0 Car -4}
	\item[\textbf{Punti Ferita:}] 15,  \textbf{Difesa:} 9,  \textbf{Iniziativa:} -3
	\item[\textbf{Movimento:}] 5 metri
	\item[\textbf{Tiri Salvez.:}] \resizebox{0.5\linewidth+1.8cm}{!}{Tempra +3, Riflessi +3, Volontà +3}
	\item[\textbf{Res. Danni:}] Freddo
	\item[\textbf{Imm. Danni:}] Fuoco, Veleno
	\item[\textbf{Immunità:}] affascinato, spaventato
	\item[\textbf{Vulnerabilità:}] argento, Luce
	\item[\textbf{Sensi:}] Scurovisione 36 m
	\item[\textbf{Linguaggi:}] comprende l'Infernale ma non può parlare
	\item[\textbf{Sfida:}] 0 (10 PX)\smallskip
\end{description}

\emph{\textbf{Rinvigorimento Diabolico.}} Un lemure che muore nei Nove Inferi ritorna in vita con tutti i suoi Punti Ferita in 1d10 giorni a meno che non venga ucciso da una creatura con tratti buoni su cui sia stato eseguito l'incantesimo \emph{benedire} o i suoi resti vengano cosparsi di Acqua santa.

\emph{\textbf{Vista del Diavolo.}} La Scurovisione del diavolo non è limitata dall'oscurità magica.

\textbf{Azioni}

\emph{\textbf{Pugno.} Attacco con arma da mischia}: +3 a colpire, portata 1 m, un bersaglio.

\emph{Colpisce:} 2 (1d4) danni contundenti.

\textbf{Ecologia}\\
Ambiente: Qualsiasi (Inferno)\\
Organizzazione: Solitario, coppia, gruppo (3-5), sciame (6-17) o schiera (10-40 o più)\\
\textbf{Categoria Tesoro}: Nessuno\\
\textbf{Descrizione}\\
I lemure sono i diavoli più infimi, nati dalle anime dannate all'inferno. Sono masse informi di carne tremolante, con tratti grotteschi che imitano i loro torturatori. Alti più di 1,2 metri e pesanti oltre 100 kg, sono creature rivoltanti che distruggono qualsiasi forma di vita non infernale.

Essi rivestono un ruolo vitale nell'ecologia dell'Inferno. Le anime dannate vengono tormentate per secoli, dimenticando le loro vite e diventando automi guidati dall'odio e dalla paura. Alla fine, queste anime vengono trasformate in lemure, la forma di vita più elementare dei diavoli.

I diavoli maggiori possono riconoscere i lemure più corrotti e trasformarli in veri diavoli, pronti a servire nelle legioni dei dannati.

\mostro{Plesiosauro}
\noindent
\begin{description}[noitemsep, topsep=0pt, parsep=0pt, partopsep=0pt, leftmargin=0cm, labelwidth=2.2cm]
	\item[\textbf{Taglia/Tipo:}] Grande bestia, disallineato
	\item[\textbf{Caratt.:}] \resizebox{0.5\linewidth+1.8cm}{!}{For 4 Des 2 Cos 3 Int -4 Sag 1 Car -3}
	\item[\textbf{Punti Ferita:}] 52,  \textbf{Difesa:} 16,  \textbf{Iniziativa:} +2
	\item[\textbf{Movimento:}] 6 m, nuoto 12 m
	\item[\textbf{Tiri Salvez.:}] \resizebox{0.5\linewidth+1.8cm}{!}{Tempra +5, Riflessi +4, Volontà +3}
	\item[\textbf{Comp.:}] Furtività +4, Consapevolezza +3
	\item[\textbf{Sfida:}] 2 (450 PX)\smallskip
\end{description}

\emph{\textbf{Trattenere il Fiato.}} Il plesiosauro può trattenere il fiato per 1 ora.

\textbf{Azioni}

\emph{\textbf{Morso.} Attacco con arma da mischia}: +5 a colpire, portata 3 m, un bersaglio.

\emph{Colpisce:} 14 (3d6 + 4) danni perforanti.

\textbf{Ecologia}\\
Ambiente: Acquatico Caldo\\
Organizzazione: Solitario, coppia o branco (3-6)\\
\textbf{Categoria Tesoro}: Nessuno\\
\textbf{Descrizione}\\
Il plesiosauro è un rettile acquatico dal lungo collo. Sebbene tecnicamente non sia un dinosauro, questa creatura ed i suoi simili si trovano spesso a cacciare in laghi ed oceani nei quali è facile trovare dei dinosauri.

\mostro{Tirannosauro}
\noindent
\begin{description}[noitemsep, topsep=0pt, parsep=0pt, partopsep=0pt, leftmargin=0cm, labelwidth=2.2cm]
	\item[\textbf{Taglia/Tipo:}] Enorme bestia, disallineato
	\item[\textbf{Caratt.:}] \resizebox{0.5\linewidth+1.8cm}{!}{For 7 Des 0 Cos 4 Int -4 Sag 1 Car -1}
	\item[\textbf{Punti Ferita:}] 165,  \textbf{Difesa:} 22,  \textbf{Iniziativa:} +0
	\item[\textbf{Movimento:}] 15 m
	\item[\textbf{Tiri Salvez.:}] \resizebox{0.5\linewidth+1.8cm}{!}{\resizebox{0.5\linewidth+1.8cm}{!}{Tempra +12, Riflessi +8, Volontà +9}}
	\item[\textbf{Sfida:}] 8 (3900 PX)\smallskip
\end{description}

\textbf{Azioni}

\emph{\textbf{Multiattacco.}} Il tirannosauro effettua due attacchi: uno con il morso e uno con la coda. Non può effettuare entrambi gli attacchi contro lo stesso bersaglio.

\emph{\textbf{Coda.} Attacco con arma da mischia}: +10 a colpire, portata 3 m, un bersaglio.

\emph{Colpisce:} 20 (3d8 + 7) danni contundenti.

\emph{\textbf{Morso.} Attacco con arma da mischia}: +10 a colpire, portata 3 m, un bersaglio.

\emph{Colpisce:} 33 (4d12 + 7) danni perforanti. Se il bersaglio è una creatura di taglia Media o inferiore, è afferrato (DC 17 per fuggire). Fino al termine dell'afferrare il tirannosauro non può usare il morso contro un altro bersaglio.

\emph{\textbf{Zampata.} Attacco con arma da mischia}: +10 a colpire, portata 6 m, fino a due bersagli. Il Tirannosauro concentra le azioni e salta su un avversario. 2 Azioni

\emph{Colpisce:} 30 (4d10 + 8) danni contundenti.

\textbf{Reazione: \emph{Attacco d'opportunità}}: il Tirannosauro effettua un attacco ad una creatura che attraversi o esca dalla sua portata di 3 metri.

\emph{\textbf{Arrabbiato:}} il Tirannosauro è pervaso da furia assassina. Attacca qualsiasi creatura amica o nemica. Il Tiro per Colpire guadagna +1d6 ed il morso causa Sanguinamento 2/15.

\textbf{Ecologia}\\
Ambiente: Foreste e Pianure Calde\\
Organizzazione: Solitario, coppia o branco (3-6)\\
\textbf{Categoria Tesoro}: Nessuno\\
\textbf{Descrizione}\\
Il tirannosauro è un predatore primario che misura 12 metri di lunghezza e pesa 7000 kg.

\mostro{Triceratopo}
\noindent
\begin{description}[noitemsep, topsep=0pt, parsep=0pt, partopsep=0pt, leftmargin=0cm, labelwidth=2.2cm]
	\item[\textbf{Taglia/Tipo:}] Enorme bestia, disallineato
	\item[\textbf{Caratt.:}] \resizebox{0.5\linewidth+1.8cm}{!}{For 6 Des -1 Cos 3 Int -4 Sag 0 Car -3}
	\item[\textbf{Punti Ferita:}] 108,  \textbf{Difesa:} 17,  \textbf{Iniziativa:} -1
	\item[\textbf{Movimento:}] 15 m
	\item[\textbf{Tiri Salvez.:}] \resizebox{0.5\linewidth+1.8cm}{!}{Tempra +8, Riflessi +4, Volontà +5}
	\item[\textbf{Sfida:}] 5 (1800 PX)\smallskip
\end{description}

\emph{\textbf{Carica Travolgente.}} Se il triceratopo si muove di almeno 6 metri diretto verso una creatura e la colpisce con un attacco di incornata durante lo stesso round, il bersaglio deve riuscire un Tiro Salvezza su Tempra DC 19 o cadere prono. Se il bersaglio è prono, il triceratopo può effettuare un attacco di pestone contro di lui come Azione Immediata.

\textbf{Azioni}

\emph{\textbf{Incornata.} Attacco con arma da mischia}: +7 a colpire, portata 1 m, un bersaglio.

\emph{Colpisce:} 24 (3d10 + 6) danni perforanti.

\emph{\textbf{Pestone.} Attacco con arma da mischia}: +7 a colpire, portata 1 m, una creatura prona.

\emph{Colpisce:} 22 (3d10 + 6) danni contundenti.

\textbf{Reazione: \emph{Collare protettivo}} il Triceratopo piega la testa sotto il collare e guadagna +4 alla Difesa.

\textbf{Ecologia}\\
Ambiente: Pianure Calde\\
Organizzazione: Solitario, coppia o branco (5-8)\\
\textbf{Categoria Tesoro}: Nessuno\\
\textbf{Descrizione}\\
Il triceratopo è un erbivoro irascibile e caparbio. Un tipico triceratopo è lungo 9 metri e pesa 10000 kg.

\mostro{Divora Cervelli}
\noindent
\begin{description}[noitemsep, topsep=0pt, parsep=0pt, partopsep=0pt, leftmargin=0cm, labelwidth=2.2cm]
	\item[\textbf{Taglia/Tipo:}] Piccola aberrazione, malvagio
	\item[\textbf{Caratt.:}] \resizebox{0.5\linewidth+1.8cm}{!}{For 1 Des 6 Cos 5 Int 3 Sag 0 Car 3}
	\item[\textbf{Punti Ferita:}] 186,  \textbf{Difesa:} 30,  \textbf{Iniziativa:} +6
	\item[\textbf{Movimento:}] 12 m
	\item[\textbf{Tiri Salvez.:}] \resizebox{0.5\linewidth+1.8cm}{!}{\resizebox{0.5\linewidth+1.8cm}{!}{Tempra +14, Riflessi +15, Volontà +9}}
	\item[\textbf{Imm. Danni:}] Fuoco
	\item[\textbf{Immunità:}] incantesimi dalle liste di magia Illusione e Charme
	\item[\textbf{Sensi:}] Vista Cieca 18 m
	\item[\textbf{Linguaggi:}] telepatia 50 m
	\item[\textbf{Sfida:}] 9 (3900 PX)\smallskip
\end{description}

\emph{\textbf{Occhi della Magia.}} Il Divora Cervelli ha \hyperlink{Individuazione del Magico}{Individuazione del Magico} sempre attivo.

\emph{\textbf{Incantesimi Innati.}} La caratteristica da incantatore del Divora Cervelli è il Carisma. Il Divora Cervelli può lanciare in maniera innata i seguenti incantesimi, senza bisogno di componenti materiali:

A volontà: \emph{\hyperlink{Confusione}{Confusione} (un unico bersaglio), \hyperlink{Infliggi Ferite}{Infliggi Ferite} con un critico, \hyperlink{Invisibilità}{Invisibilità}}

3/giorno: \emph{\hyperlink{Cura Ferite}{Cura Ferite} 3, \hyperlink{Globo di Invulnerabilità}{Globo di Invulnerabilità}}

\textbf{Azioni}

\emph{\textbf{Multiattacco.}} Il Divora Cervelli può effettuare 4 attacchi, uno per artiglio

\emph{\textbf{Artiglio.} Attacco con arma da mischia}: +9 al a colpire, portata 1 m, una creatura.

\emph{Colpisce:} 3 danni da taglio (1d4+1), 1 danno da Sanguinamento.

\textbf{Abilità speciali}

\emph{\textbf{Furto del corpo}}

Spendendo 3 Azioni un Divora Cervelli può diventare minuscolo e strisciare nella bocca/naso/orecchie di una creatura indifesa o morta ed arrivare al cervello per nutrirsene. Si tratta di una azione che uccide la creatura.

Il Divora Cervelli assume il controllo del corpo e lo può usare a suo piacimento, come se controllasse la vittima con un incantesimo Dominare Mostri. Il Divora Cervelli ha pieno accesso a tutte le capacità difensive e offensive dell'ospite tranne che per le capacità magiche e gli incantesimi (anche se il Divora Cervelli può comunque usare le proprie capacità magiche).

Un corpo ospite non deve essere morto da più di 1 giorno perché questa capacità funzioni, e anche dopo essere stati occupati con successo i corpi si decompongono diventando inutilizzabili in 7 giorni (a meno che questo periodo venga prolungato con l'incantesimo Riposo Inviolato).

Finché il Divora Cervelli occupa il corpo, conosce (e può parlare) i linguaggi conosciuti dalla vittima e le informazioni sulla sua identità e personalità, ma non può possederne gli specifici ricordi e conoscenze.

Il danno inflitto al corpo, che ha il doppio dei Punti Ferita originali, ospite non danneggia il Divora Cervelli e se il corpo ospite viene distrutto il Divora Cervelli esce ed è Stordito per 1 round.

\textbf{Ecologia}\\
Ambiente: Qualsiasi sotterraneo\\
Organizzazione: Solitario, covata (2-6) o tribù (7-16)\\
\textbf{Categoria Tesoro}: G\\
\textbf{Descrizione}\\
Un Divora Cervelli altro non è che un cervello di circa 50 cm dotato di 4 potenti zampe artigliate.

I Divora Cervelli sono certamente una tra le razze più crudeli del mondo. Incapaci di provare emozioni o di sguazzare nei peccati del proprio piacere fisico, i divora cervelli sono costretti a rubare corpi per soddisfare la loro golosità, lussuria e crudeltà. Esistono storie che narrano di intere città sotterranee di queste creature che indossano corpi come se fossero vestiti per consumare spaventose orge e macabri festini.

I Divora Cervelli solitari spesso vivono in rovine o caverne ai margini delle regioni civilizzate per poter fare periodiche scorrerie in città per "acquistare" un nuovo allettante corpo.

Si dice che il giardino di Shayalia sia pieno di Divora Cervelli. Un Divora Cervelli è lungo 90 cm e pesa circa 30 kg.

\mostro{Dobi}
\noindent
\begin{description}[noitemsep, topsep=0pt, parsep=0pt, partopsep=0pt, leftmargin=0cm, labelwidth=2.2cm]
	\item[\textbf{Taglia/Tipo:}] Minuscola fatata, neutrale
	\item[\textbf{Caratt.:}] \resizebox{0.5\linewidth+1.8cm}{!}{For -3 Des -1 Cos 2 Int -2 Sag 1 Car 3}
	\item[\textbf{Punti Ferita:}] 15,  \textbf{Difesa:} 11,  \textbf{Iniziativa:} -1
	\item[\textbf{Movimento:}] 3 m, Nuotare 9 m
	\item[\textbf{Tiri Salvez.:}] \resizebox{0.5\linewidth+1.8cm}{!}{Tempra +3, Riflessi +3, Volontà +3}
	\item[\textbf{Sensi:}] Visione Crepuscolare 18 m
	\item[\textbf{Linguaggi:}] comprende il Comune, ma non lo parla
	\item[\textbf{Imm. Danni:}] al danno delle armi non magiche
	\item[\textbf{Sfida:}] 0 (10 PX)\smallskip
\end{description}

\emph{\textbf{Dobi}} Il Dobi si appiccica, per spostarlo è necessario essere gentili e chiederglielo.\\
\emph{\textbf{Dobi Dobi Dobi}} Quando il Dobi subisce più di 3 punti ferita di danno con un arma non contundente si divide in due Dobi più piccoli ognuno con lo stesso ammontare di Punti Ferita rimasti al Dobi precedente.\\
\smallskip\textbf{Azioni}\\
\emph{\textbf{Dobi Dobi}} il Dobi proietta un aura di \hyperlink{Calmare Emozioni}{Calmare Emozioni} come l'omonimo incantesimo ma non è concesso il Tiro Salvezza. Il Dobi può influenzare una sola creatura alla volta con il suo potere.\\
\textbf{Ecologia}\\
Ambiente: Paludi\\
Organizzazione: gruppo\\
\textbf{Categoria Tesoro}: Accidentale\\
\textbf{Descrizione}\\
{\small "...Smossi le foglie dell'acquitrino e vidi a terra una strana palla di pelo, di circa dieci centimetri di diametro, di colore chiaro. Incuriosito lo raccolsi, accarezzando il suo pelo soffice e lo scrutai con attenzione. Sembrava non avere arti o segni di possedere un muso con occhi, orecchie, bocca, ma non appena lo accarezzai la palla vibrò, emettendo uno squittio.

	Finalmente scorsi due occhietti neri e vispi aprirsi in tutto quel pelo e poi due orecchiette tonde spuntare, quindi due zampette corte ma robuste, adatte al salto, appoggiate a terra e altre due, sempre corte ma dotate di ben cinque dita ognuna, a mezza altezza.

	- Dobi! - rispose l'animaletto, esprimendo una sorta di gioia ed entusiasmo. - Dobi dobi! -.

	- Che carino! - esclamai, accarezzandolo. Era l'animaletto più tenero che avessi mai visto. - Ora però ti rimetto giù -.

	- Dobi - rispose la palla di pelo.
	Portai a terra la mano, ma l'animale non si mosse. Provai a staccarmelo dalla mano, ma rimase appiccicato all'altra. Lo presi con due dita, tirando forte e lo appoggiai veloce a terra, ma subito mi saltò sul piede e vi rimase attaccato. Dovetti attraversare l'acquitrino con il dobi attaccato al piede, senza contare gli altri quattro che trovai avvinghiati all'armatura."}

Da \emph{Viaggio nel primo mondo.} Romanzo di Federica Angeli

\mostro{Doppelganger}
\noindent
\begin{description}[noitemsep, topsep=0pt, parsep=0pt, partopsep=0pt, leftmargin=0cm, labelwidth=2.2cm]
	\item[\textbf{Taglia/Tipo:}] Media mostruosità (mutaforma), neutrale
	\item[\textbf{Caratt.:}] \resizebox{0.5\linewidth+1.8cm}{!}{For 0 Des 4 Cos 2 Int 0 Sag 1 Car 2}
	\item[\textbf{Punti Ferita:}] 70,  \textbf{Difesa:} 20,  \textbf{Iniziativa:} +4
	\item[\textbf{Movimento:}] 9 m
	\item[\textbf{Tiri Salvez.:}] \resizebox{0.5\linewidth+1.8cm}{!}{Tempra +5, Riflessi +7, Volontà +4}
	\item[\textbf{Comp.:}] Ingannare +6, Percepire Emozioni +3
	\item[\textbf{Immunità:}] affascinato
	\item[\textbf{Sensi:}] Scurovisione 18 m
	\item[\textbf{Linguaggi:}] Comune
	\item[\textbf{Sfida:}] 3 (700 PX)\smallskip
\end{description}

\emph{\textbf{Mutaforma.}} Il doppelganger può usare una Azione per cambiare la propria forma in quella di un umanoide Piccolo o Medio che abbia visto, o per tornare alla sua vera forma. Le sue statistiche, a parte la taglia, sono le stesse in tutte le forme. Qualsiasi equipaggiamento stia indossando o trasportando non viene trasformato. Alla morte ritorna alla sua vera forma.

\emph{\textbf{Appostato.}} Nel primo round di combattimento, il doppelganger ha +1d6 ai tiri di attacco contro qualsiasi creatura abbia preso di sorpresa.

\emph{\textbf{Attacco di Sorpresa.}} Se il doppelganger sorprende una creatura e la colpisce con un attacco durante il primo round di combattimento, il bersaglio subisce 10 (3d6) danni aggiuntivi dall'attacco.

\textbf{Azioni}

\emph{\textbf{Multiattacco.}} Il doppelganger effettua due attacchi da mischia.

\emph{\textbf{Schianto.} Attacco con arma da mischia}: +5 a colpire, portata 1 m, un bersaglio.

\emph{Colpisce:} 7 (1d6 + 4) danni contundenti.

\emph{\textbf{Leggere Pensieri.}} Il doppelganger legge magicamente i pensieri di superficie di una creatura entro 18 metri da lui. L'effetto può penetrare le barriere, ma 1 metro di legno o terra, 50 centimetri di pietra, 5 centimetri di metallo, o un sottile foglio di piombo lo blocca. Mentre il bersaglio è a gittata, il doppelganger può continuare a leggerne i pensieri, purché la concentrazione del doppelganger non venga infranta (come la concentrazione di un incantesimo). Mentre legge la mente di un bersaglio, il doppelganger ha +1d6 alle prove di Saggezza e Carisma contro il bersaglio.

\textbf{Ecologia}\\
Ambiente: Qualsiasi\\
Organizzazione: Solitario, coppia o banda (3-6)\\
\textbf{Categoria Tesoro}: Equipaggiamento da PNG\\
\textbf{Descrizione}\\
I doppelganger sono esseri che possono assumere la forma di chiunque incontrino. Nella loro forma naturale, sembrano umanoidi snelli e fragili, con tratti facciali non del tutto formati e carnagione pallida.

Preferiscono infiltrarsi in società complesse per accumulare ricchezza e potere, usando la loro abilità mimetica per tendere imboscate e trappole. Sebbene non siano necessariamente malvagi, sono egoisti e vedono gli altri come giocattoli da manipolare.

Alcuni doppelganger amano i giochi politici, mentre altri cambiano continuamente razza, sesso e partner amorosi. Sono noti per le loro capacità di cambiare forma e per evitare la cattura. I più potenti possono assumere anche abilità e ricordi delle creature che impersonano.

\medskip


\begin{enfasi}
	Conosci il tuo nemico e conosci te stesso; in cento battaglie non correrai mai pericolo. L'Arte della Guerra, Sun Tzu
\end{enfasi}

\rule{\linewidth}{2pt}

\medskip

\pdfbookmark[3]{Draghi}{Draghi}



I Draghi sono creature temibili, pericolose, antiche; rappresentano il potere stesso.

Ogni Drago ha pieno accesso a tutti gli incantesimi di una specifica lista di magia a seconda del proprio colore.

Questo accesso è garantito da Tàhil o Ljust a seconda che siano draghi fedeli ad uno o all'altro.

Ed è da questa distinzione che i draghi vengono suddivisi tra Draghi di Tàhil e di Ljust. I primi rappresentano a vario titolo e grado Caos, distruzione, violenza e morte, mentre i Draghi di Ljust sono l'emblema del buono, giusto, corretto, protettivo. Mentre i draghi di Tàhil sono solitamente definiti anche cromatici quelli di Ljust sono definiti metallici.

I Draghi di Ljust sono errori di trasporto, magari perché il portale di Tàhil si è aperto mentre un drago malvagio combatteva con un drago buono.

\textbf{Fallire il Tiro Salvezza} contro il soffio di un drago in maniera critica ne raddoppia i danni subiti mentre riuscire in maniera critica non ne dimezza ulteriormente il danno ricevuto.\index{Soffio del drago}

\medskip

\textbf{Draghi e Magia}

\begin{itemize}[leftmargin=*] \setlength{\itemsep}{0pt}
	\item Ogni Drago può lanciare incantesimi sino ad un livello massimo pari ad un quarto del suo Grado si Sfida, con un minimo accesso al primo livello.
	\item Ogni Drago ha un numero di Punti Magia pari 5 volte il suo Grado di Sfida
	\item Ogni Drago ha un punteggio di Competenza Magica pari alla metà del suo Grado di Sfida
\end{itemize}

\medskip

\textbf{Tabella: accesso Lista di Magia per Draghi}\index[Tabelle]{Tabella accesso Lista di Magia per Draghi}

\medskip

\noindent\begin{tabularx}{\linewidth}{ll}
\toprule
 \rowcolor{gray!20}\textbf{Colore} & \textbf{Lista} \\
\toprule
	Bianco & Acqua \\
 \rowcolor{gray!20}Blu & Aria \\
	Giallo & Fuoco, Evocazione \\
 \rowcolor{gray!20}Porpora & Terra \\
	Nero & Acqua, Necromanzia \\
 \rowcolor{gray!20}Rosso & Fuoco \\
	Verde & Animali e Piante \\
 \rowcolor{gray!20}Bronzo & Abiurazione \\
	Ottone & Illusione, Divinazione \\
 \rowcolor{gray!20}Argento & Trasmutazione \\
	Oro & Cura, Evocazione \\
 \rowcolor{gray!20}Rame & Invocazione \\

\end{tabularx}

\medskip

Tutti i Draghi hanno accesso alla lista di magia Universale e prediligono certi incantesimi che sono segnati nella loro descrizione.

\medskip

Nella \textbf{Descrizione di ogni Drago Antico} troverete una breve descrizione del tipo di drago.

\medskip

\textbf{Poteri speciali dei Draghi}\medskip

Ogni Drago a seconda dell'età ha uno o più poteri speciale casuali.
Se è un Drago Cucciolo ha 1 potere casuale, 2 se è Giovane o Adulto e 3 se è Antico. In caso di poteri ripetuti non ripetere il tiro.

\medskip

\noindent\begin{tabularx}{\linewidth}{lX}
	\toprule
 \rowcolor{gray!20}\textbf{3d6}& \textbf{Potere del Drago}\\
\toprule
	3	& Ricarica veloce. Il Drago ricarica il soffio con 4-6. \\
	\end{tabularx}
\noindent\begin{tabularx}{\linewidth}{lX}
 \rowcolor{gray!20}4& Agilità sorprendente. La Difesa del Drago aumenta di un ulteriore +4. \\
	5-7 	& Signore dei Serpenti. La coda ha un pungiglione velenoso che infligge 2xGS PF di danno da veleno. TS Tempra DC 10+GS per dimezzare.\\
 \rowcolor{gray!20}8-10 	& Benedetto di Tàhil. Il Drago ha migliori Tiri Salvezza. +1d6 ad ogni Tiro Salvezza.\\
	11-13 	& Regina Lucertola. Lo sguardo del Drago ha lo stesso effetto di quello del \hyperlink{Basilisco}{Basilisco}.\\
 \rowcolor{gray!20}14-15 	& Potere del Ferro. Il Drago ha \emph{Arrugginire Metallo} come il \hyperlink{Rugginofago}{Rugginofago}.\\
	16 	& Resistenza alla magia. Il Drago è immune agli incantesimi sotto GS/5 livello.\\
 \rowcolor{gray!20}17 	& Immunità aggiuntiva. Il Drago è immune ai danni da una forma di \hyperlink{elencoenergia}{Energia} (pag. \pageref{elencoenergia}) in più a \hyperlink{fontienergia}{caso} (pag. \pageref{fontienergia}).\\
	18 	& Pelle corazzata. Il Drago ha riduzione del danno pari a GS/3 ai danni T/P/C.
\end{tabularx}

\medskip

\begin{enfasi}{
Oh maledetti possa Lynx chiudervi tutti i portali\\
Oh assassini possa Sumkjir sterminarvi\\
Oh devastatori possa Nedraf rompervi le ossa!\\
(imprecazioni popolari contro i Draghi)}\end{enfasi}

\medskip

\rule{\linewidth}{2pt}

\medskip

\textbf{Draghi di Tàhil}

\pdfbookmark[3]{Draghi di Tahil}{Draghi di Tahil}

\mostro{Drago Bianco Antico}
\noindent
\begin{description}[noitemsep, topsep=0pt, parsep=0pt, partopsep=0pt, leftmargin=0cm, labelwidth=2.2cm]
	\item[\textbf{Taglia/Tipo:}] Mastodontica drago, malvagio
	\item[\textbf{Caratt.:}] \resizebox{0.5\linewidth+1.8cm}{!}{For 8 Des 0 Cos 8 Int 3 Sag 1 Car 2}
	\item[\textbf{Punti Ferita:}] 407,  \textbf{Difesa:} 38,  \textbf{Iniziativa:} +3
	\item[\textbf{Movimento:}] 12 m, nuoto 12 m, volo 24 m
	\item[\textbf{Tiri Salvez.:}] \resizebox{0.5\linewidth+1.8cm}{!}{\resizebox{0.5\linewidth+1.8cm}{!}{Tempra +28, Riflessi +20, Volontà +21}}
	\item[\textbf{Comp.:}] Furtività +6, Consapevolezza +13
	\item[\textbf{Imm. Danni:}] Freddo, armi +1
	\item[\textbf{Sensi:}] Scurovisione 36 m, Vista Cieca 18 m
	\item[\textbf{Linguaggi:}] Comune, Draconico
	\item[\textbf{Sfida:}] 20 (25000 PX)\smallskip
\end{description}

\emph{\textbf{Aura di gelo.}} il drago emette nel raggio di 3 metri un gelo magico che causa 2d6 danni da freddo a round.

\emph{\textbf{Camminare sul Ghiaccio.}} Il drago può muoversi e arrampicarsi su superfici ghiacciate senza bisogno di effettuare prove su competenze di base. Inoltre, il terreno difficile composto di ghiaccio o neve non gli costa movimento aggiuntivo.

\emph{\textbf{Resistenza Leggendaria (3/Giorno).}} Se il drago fallisce un Tiro Salvezza, può scegliere invece di riuscire.

\emph{\textbf{Resistenza alla Magia:}} 3lv

\textbf{Azioni}

\emph{\textbf{Multiattacco.}} Il drago può usare la sua Presenza Spaventosa e poi effettuare tre attacchi: uno con il morso e due con gli artigli.

\emph{\textbf{Artiglio.} Attacco con arma da mischia}: +16 a colpire, portata 3 m, un bersaglio.

\emph{Colpisce:} 15 (2d6 + 8) danni taglienti, 3/20 danni da Sanguinamento.

\emph{\textbf{Coda.} Attacco con arma da mischia}: +16 a colpire, portata 6 m, un bersaglio.

\emph{Colpisce:} 17 (2d8 + 8) danni contundenti.

\emph{\textbf{Morso.} Attacco con arma da mischia}: +16 a colpire, portata 5 metri, un bersaglio.

\emph{Colpisce:} 19 (2d10 + 8) danni perforanti più 9 (2d8) danni da freddo.

\emph{\textbf{Presenza Spaventosa.}} Ogni creatura scelta dal drago, che si trovi entro 36 metri da esso e consapevole della sua presenza, deve riuscire un Tiro Salvezza di Volontà DC 35 o restare spaventata per 1 minuto. Una creatura può ripetere il Tiro Salvezza al termine di ciascun suo round, terminando l'effetto se lo riesce. Se il Tiro Salvezza della creatura ha successo o l'effetto ha termine per essa, la creatura è immune alla Presenza Spaventosa del drago per le successive 24 ore.

\emph{\textbf{Soffio Gelido (Ricarica 5-6).}} Il drago esala un'esplosione di ghiaccio in un cono di 27 metri. Ogni creatura in quell'area deve effettuare un Tiro Salvezza di Tempra DC 35 e subire 72 (16d8) danni da freddo se fallisce il Tiro Salvezza, o la metà di questi danni se lo riesce.

\textbf{Azioni Aggiuntive}

Il drago può effettuare 3 Azioni aggiuntive, scelte tra le opzioni seguenti. Può usare solo un'opzione Aggiuntive alla volta e solo al termine del round di un'altra creatura. Il drago recupera le Azioni aggiuntive spese all'inizio del proprio round.

\textbf{Attacco di Ala (Costa 2 Azioni).} Il drago batte le ali. Ogni creatura entro 5 metri dal drago deve riuscire un Tiro Salvezza su Riflessi DC 33 o subire 15 (2d6 + 8) danni contundenti e venir gettato prono. Il drago può poi volare fino a metà del suo movimento di volo.

\textbf{Attacco di Coda.} Il drago effettua un attacco di coda.

\textbf{Individuare.} Il drago effettua una prova Consapevolezza.

\textbf{Ecologia}\\
Ambiente: Montagne Fredde\\
Organizzazione: Solitario\\
\textbf{Categoria Tesoro}: H\\
\textbf{Descrizione}\\
I Draghi Bianchi sono tra i più selvaggi e animali di tutti i draghi.
Amano i posti freddi e ghiacciati, trovando rifugio nelle valli più fredde come i picchi ghiacciati delle montagne e le steppe gelide.

I Draghi Bianchi hanno un aspetto selvaggio quasi sempre mostrano i denti e gli artigli sono estratti per muoversi agilmente sul terreno ghiacciato.
Non hanno penalità di movimento su questi terreni.

Sfruttano il loro naturale camuffamento per aggredire e catturare le prede, sono ottimi cacciatori, molto intelligenti nello sfruttare l'ambiente.

Poco inclini alla magia sanno però soffiare schegge di ghiaccio molto più frequentemente di altri draghi. E' immune gli attacchi basati sul freddo e ghiaccio.

Le loro tane sono caverne ghiacciate nelle montagne o scavate nei ghiacciai più massicci.

I Draghi Bianchi hanno +1d6 nelle prove di magia e possono ignorare un dado tirato nella prova con la Lista dell'Acqua ed è immune al freddo.\\

\textbf{Incantesimi}\index{Incantesimi da Drago Bianco}\\
Gli incantesimi preferiti di questo Drago sono:\\
- \hyperlink{Scudo di Fuoco}{Scudo di Fuoco}\\
- \hyperlink{Tempesta di Ghiaccio}{Tempesta di Ghiaccio}\\
- \hyperlink{Tempesta di Nevischio}{Tempesta di Nevischio}

\mostro{Drago Bianco Adulto}
\noindent
\begin{description}[noitemsep, topsep=0pt, parsep=0pt, partopsep=0pt, leftmargin=0cm, labelwidth=2.2cm]
	\item[\textbf{Taglia/Tipo:}] Enorme drago, malvagio
	\item[\textbf{Caratt.:}] \resizebox{0.5\linewidth+1.8cm}{!}{For 6 Des 0 Cos 6 Int 2 Sag 1 Car 1}
	\item[\textbf{Punti Ferita:}] 264,  \textbf{Difesa:} 29,  \textbf{Iniziativa:} +2
	\item[\textbf{Movimento:}] 12 m, nuoto 12 m, volo 24 m
	\item[\textbf{Tiri Salvez.:}] \resizebox{0.5\linewidth+1.8cm}{!}{\resizebox{0.5\linewidth+1.8cm}{!}{Tempra +19, Riflessi +13, Volontà +14}}
	\item[\textbf{Comp.:}] Furtività +5, Consapevolezza +8
	\item[\textbf{Imm. Danni:}] Freddo
	\item[\textbf{Sensi:}] Scurovisione 36 m, Vista Cieca 18 m
	\item[\textbf{Linguaggi:}] Comune, Draconico
	\item[\textbf{Sfida:}] 13 (10000 PX)\smallskip
\end{description}

\emph{\textbf{Aura di gelo.}} il drago emette nel raggio di 3 metri un gelo magico che causa 1d6 danni da freddo a round.

\emph{\textbf{Camminare sul Ghiaccio.}} Il drago può muoversi e arrampicarsi su superfici ghiacciate senza bisogno di effettuare prove su competenze di base. Inoltre, il terreno difficile composto di ghiaccio o neve non gli costa movimento aggiuntivo.

\emph{\textbf{Resistenza Leggendaria (3/Giorno).}} Se il drago fallisce un Tiro Salvezza, può scegliere invece di riuscire.

\textbf{Azioni}

\emph{\textbf{Multiattacco.}} Il drago può usare la sua Presenza Spaventosa e poi effettuare tre attacchi: uno con il morso e due con gli artigli.

\emph{\textbf{Artiglio.} Attacco con arma da mischia}: +12 a colpire, portata 1 m, un bersaglio, 1 danno da Sanguinamento.

\emph{Colpisce:} 13 (2d6 + 6) danni taglienti.

\emph{\textbf{Coda.} Attacco con arma da mischia}: +12 a colpire, portata 5 metri, un bersaglio.

\emph{Colpisce:} 15 (2d8 + 6) danni contundenti.

\emph{\textbf{Morso.} Attacco con arma da mischia}: +12 a colpire, portata 3 m, un bersaglio.

\emph{Colpisce:} 17 (2d10 + 6) danni perforanti più 4 (1d8) danni da freddo.

\emph{\textbf{Presenza Spaventosa.}} Ogni creatura scelta dal drago, che si trovi entro 36 metri da esso e consapevole della sua presenza, deve riuscire un Tiro Salvezza di Volontà DC 27 o restare spaventata per 1 minuto. Una creatura può ripetere il Tiro Salvezza al termine di ciascun suo round, terminando l'effetto se lo riesce. Se il Tiro Salvezza della creatura ha successo o l'effetto ha termine per essa, la creatura è immune alla Presenza Spaventosa del drago per le successive 24 ore.

\emph{\textbf{Soffio Gelido (Ricarica 5-6).}} Il drago esala un'esplosione di ghiaccio in un cono di 18 metri. Ogni creatura in quell'area deve effettuare un Tiro Salvezza di Tempra DC 27 e subire 54 (12d8) danni da freddo se fallisce il Tiro Salvezza, o la metà di questi danni se lo riesce.

\textbf{Azioni Aggiuntive}

Il drago può effettuare 3 Azioni aggiuntive, scelte tra le opzioni seguenti. Può usare solo un'opzione Aggiuntiva alla volta e solo al termine del round di un'altra creatura. Il drago recupera le Azioni aggiuntive spese all'inizio del proprio round.

\textbf{Attacco di Ala (Costa 2 Azioni).} Il drago batte le ali. Ogni creatura entro 3 metri dal drago deve riuscire un Tiro Salvezza su Riflessi DC 27 o subire 13 (2d6 + 6) danni contundenti e venir gettato prono. Il drago può poi volare fino a metà del suo movimento di volo. \textbf{Attacco di Coda.} Il drago effettua un attacco di coda
.
\textbf{Individuare.} Il drago effettua una prova di Consapevolezza.

\emph{\textbf{Arrabbiato:}} Il Drago Bianco Adulto può eseguire queste azioni a costo 2 Azioni.

\emph{Focalizzare}: la creatura interrompe un effetto mentale su di se in corso

\emph{Brutalità}: la creatura attacca con ferocia inaudita. +1d6 al Tiro per Colpire, 1 danno critico automatico quando colpisce.

\textbf{Ecologia}\\
Ambiente: Montagne Fredde\\
Organizzazione: Solitario\\
\textbf{Categoria Tesoro}: E\\
\textbf{Descrizione}\\
Vedi descrizione Drago Bianco Antico.\\
\textbf{Incantesimi}\index{Incantesimi da Drago Bianco}\\
Gli incantesimi preferiti di questo Drago sono:\\
- \hyperlink{Scudo di Fuoco}{Scudo di Fuoco}\\
- \hyperlink{Tempesta di Ghiaccio}{Tempesta di Ghiaccio}\\
- \hyperlink{Tempesta di Nevischio}{Tempesta di Nevischio}

\mostro{Drago Bianco Giovane}
\noindent
\begin{description}[noitemsep, topsep=0pt, parsep=0pt, partopsep=0pt, leftmargin=0cm, labelwidth=2.2cm]
	\item[\textbf{Taglia/Tipo:}] Grande drago, malvagio
	\item[\textbf{Caratt.:}] \resizebox{0.5\linewidth+1.8cm}{!}{For 4 Des 0 Cos 4 Int -2 Sag 0 Car 1}
	\item[\textbf{Punti Ferita:}] 127,  \textbf{Difesa:} 20,  \textbf{Iniziativa:} +0
	\item[\textbf{Movimento:}] 12 m, nuoto 12 m, volo 24 m
	\item[\textbf{Tiri Salvez.:}] \resizebox{0.5\linewidth+1.8cm}{!}{Tempra +10, Riflessi +6, Volontà +6}
	\item[\textbf{Comp.:}] Furtività +3, Consapevolezza +6
	\item[\textbf{Imm. Danni:}] Freddo
	\item[\textbf{Sensi:}] Scurovisione 36 m, Vista Cieca 9 m
	\item[\textbf{Linguaggi:}] Comune, Draconico
	\item[\textbf{Sfida:}] 6 (2300 PX)\smallskip
\end{description}

\emph{\textbf{Camminare sul Ghiaccio.}} Il drago può muoversi e arrampicarsi su superfici ghiacciate senza bisogno di effettuare prove su competenze di base. Inoltre, il terreno difficile composto di ghiaccio o neve non gli costa movimento aggiuntivo.

\textbf{Azioni}

\emph{\textbf{Multiattacco.}} Il drago può usare la sua Presenza Spaventosa e poi effettuare tre attacchi: uno con il morso e due con gli artigli.

\emph{\textbf{Artiglio.} Attacco con arma da mischia}: +9 a colpire, portata 1 m, un bersaglio.

\emph{Colpisce:} 11 (2d6 + 4) danni taglienti, 1 danno da Sanguinamento.

\emph{\textbf{Morso.} Attacco con arma da mischia}: +9 a colpire, portata 3 m, un bersaglio.

\emph{Colpisce:} 15 (2d10 + 4) danni perforanti più 4 (1d8) danni da freddo.

\emph{\textbf{Soffio Gelido (Ricarica 5-6).}} Il drago esala un'esplosione di ghiaccio in un cono di 9 metri. Ogni creatura in quell'area deve effettuare un Tiro Salvezza di Tempra DC 18 e subire 45 (10d8) danni da freddo se fallisce il Tiro Salvezza, o la metà di questi danni se lo riesce.

\textbf{Ecologia}\\
Ambiente: Montagne Fredde\\
Organizzazione: Solitario\\
\textbf{Categoria Tesoro}: D\\
\textbf{Descrizione}\\
Vedi descrizione Drago Bianco Antico.\\
\textbf{Incantesimi}\index{Incantesimi da Drago Bianco}\\
Gli incantesimi preferiti di questo Drago sono:\\
- \hyperlink{Scudo di Fuoco}{Scudo di Fuoco}\\
- \hyperlink{Tempesta di Ghiaccio}{Tempesta di Ghiaccio}\\
- \hyperlink{Tempesta di Nevischio}{Tempesta di Nevischio}

\mostro{Drago Bianco Cucciolo}
\noindent
\begin{description}[noitemsep, topsep=0pt, parsep=0pt, partopsep=0pt, leftmargin=0cm, labelwidth=2.2cm]
	\item[\textbf{Taglia/Tipo:}] Media drago, malvagio
	\item[\textbf{Caratt.:}] \resizebox{0.5\linewidth+1.8cm}{!}{For 2 Des 0 Cos 2 Int -3 Sag 0 Car 0}
	\item[\textbf{Punti Ferita:}] 51,  \textbf{Difesa:} 14,  \textbf{Iniziativa:} +0
	\item[\textbf{Movimento:}] 9 m, nuoto 9 m, volo 18 m
	\item[\textbf{Tiri Salvez.:}] \resizebox{0.5\linewidth+1.8cm}{!}{Tempra +4, Riflessi +3, Volontà +3}
	\item[\textbf{Comp.:}] Furtività +2, Consapevolezza +4
	\item[\textbf{Imm. Danni:}] Freddo
	\item[\textbf{Sensi:}] Scurovisione 18 m, Vista Cieca 3 m
	\item[\textbf{Linguaggi:}] Draconico
	\item[\textbf{Sfida:}] 2 (450 PX)\smallskip
\end{description}

\textbf{Azioni}

\emph{\textbf{Morso.} Attacco con arma da mischia}: +5 a colpire, portata 3 m, un bersaglio.

\emph{Colpisce:} 15 (2d10 + 4) danni perforanti più 4 (1d8) danni da freddo.

\emph{\textbf{Soffio Gelido (Ricarica 5-6).}} Il drago esala un'esplosione di ghiaccio in un cono di 5 metri. Ogni creatura in quell'area deve effettuare un Tiro Salvezza di Tempra DC 15 e subire 22 (5d8) danni da freddo se fallisce il Tiro Salvezza, o la metà di questi danni se lo riesce.

\textbf{Ecologia}\\
Ambiente: Montagne Fredde\\
Organizzazione: Solitario\\
\textbf{Categoria Tesoro}: C\\
\textbf{Descrizione}\\
Vedi descrizione Drago Bianco Antico.

\mostro{Drago Blu Antico}
\noindent
\begin{description}[noitemsep, topsep=0pt, parsep=0pt, partopsep=0pt, leftmargin=0cm, labelwidth=2.2cm]
	\item[\textbf{Taglia/Tipo:}] Mastodontica drago, malvagio
	\item[\textbf{Caratt.:}] \resizebox{0.5\linewidth+1.8cm}{!}{For 9 Des 0 Cos 8 Int 4 Sag 3 Car 5}
	\item[\textbf{Punti Ferita:}] 465,  \textbf{Difesa:} 42,  \textbf{Iniziativa:} +4
	\item[\textbf{Movimento:}] 12 m, scavo 12 m, volo 24 m
	\item[\textbf{Tiri Salvez.:}] \resizebox{0.5\linewidth+1.8cm}{!}{\resizebox{0.5\linewidth+1.8cm}{!}{Tempra +31, Riflessi +23, Volontà +26}}
	\item[\textbf{Comp.:}] Furtività +7, Consapevolezza +17
	\item[\textbf{Imm. Danni:}] Elettricità, armi +1
	\item[\textbf{Sensi:}] Scurovisione 36 m, Vista Cieca 18 m
	\item[\textbf{Linguaggi:}] Comune, Draconico
	\item[\textbf{Sfida:}] 23 (50000 PX)\smallskip
\end{description}

\emph{\textbf{Scarica elettrica.}} il drago emette nel raggio di 3 metri scariche elettriche magiche che causano 2d6 danni da elettricità a round.

\emph{\textbf{Resistenza Leggendaria (3/Giorno).}} Se il drago fallisce un Tiro Salvezza, può scegliere invece di riuscire.

\emph{\textbf{Resistenza alla Magia:}} 3lv

\textbf{Azioni}

\emph{\textbf{Multiattacco.}} Il drago può usare la sua Presenza Spaventosa e poi effettuare tre attacchi: uno con il morso e due con gli artigli.

\emph{\textbf{Artiglio.} Attacco con arma da mischia}: +16 a colpire,
portata 3 m, un bersaglio.

\emph{Colpisce:} 16 (2d6 + 9) danni taglienti, 3/20 danno da Sanguinamento.

\emph{\textbf{Coda.} Attacco con arma da mischia}: +16 a colpire, portata 6 m, un bersaglio.

\emph{Colpisce:} 18 (2d8 + 9) danni contundenti.

\emph{\textbf{Morso.} Attacco con arma da mischia}: +16 a colpire, portata 5 metri, un bersaglio.

\emph{Colpisce:} 20 (2d10 + 9) danni perforanti più 11 (2d10) danni da elettricità.

\emph{\textbf{Presenza Spaventosa.}} Ogni creatura scelta dal drago, che si trovi entro 36 metri da esso e consapevole della sua presenza, deve riuscire un Tiro Salvezza di Volontà DC 35 o restare spaventata per 1 minuto. Una creatura può ripetere il Tiro Salvezza al termine di ciascun suo round, terminando l'effetto se lo riesce. Se il Tiro Salvezza della creatura ha successo o l'effetto ha termine per essa, la creatura è immune alla Presenza Spaventosa del drago per le successive 24 ore.

\emph{\textbf{Soffio Fulminante (Ricarica 5-6).}} Il drago esala fulmini in una linea lunga 36 metri e larga 3 metri. Ogni creatura su quella linea deve effettuare un Tiro Salvezza di Riflessi DC 35 e subire 88 (16d10) danni da elettricità se fallisce il Tiro Salvezza, o la metà di questi danni se lo riesce.

\textbf{Azioni Aggiuntive}

Il drago può effettuare 3 Azioni aggiuntive, scelte tra le opzioni seguenti. Può usare solo un'opzione Aggiuntiva alla volta e solo al termine del round di un'altra creatura. Il drago recupera le Azioni aggiuntive spese all'inizio del proprio round.

\textbf{Attacco di Ala (Costa 2 Azioni).} Il drago batte le ali. Ogni creatura entro 5 metri dal drago deve riuscire un Tiro Salvezza su Riflessi DC 35 o subire 16 (2d6 + 9) danni contundenti e venir gettato prono. Il drago può poi volare fino a metà del suo movimento di volo.

\textbf{Attacco di Coda.} Il drago effettua un attacco di coda.

\textbf{Individuare.} Il drago effettua una prova di Consapevolezza.\\
\textbf{Ecologia}\\
Ambiente: Picchi montuosi\\
Organizzazione: Solitario\\
\textbf{Categoria Tesoro}: H\\
\textbf{Descrizione}\\
I Draghi Blu abitano tra le nuvole, volando (e levitando) tra le tempeste.

I Draghi Blu hanno un aspetto serpentiforme, allungato ed legante, con corna lunghe all'indietro.

La faccia di un Drago Blu è meno segnata da increspature e rimane liscia.
Sono gli unici draghi a non avere ali pur volando meglio di ogni altro drago.

La loro magica ma naturale capacità di volo unita al fatto di nutrirsi di elettricità ne fa creature prettamente volanti che quasi mai scendono a terra (e mai toccano terra considerandola impura e sporca!), preferiscono rimanere tra le nubi, specialmente tra quelle più scure e cariche di energia per nutrirsi

La tana del Drago Blu solitamente è tra i picchi più alti delle montagne possibilmente tanto alte da arrivare alle nubi. Questa non è mai coperta e spesso assomiglia a giganteschi nidi.

I Draghi Blu possono assimilare carne ma non vegetali, non traggono nutrimenti da ciò che mangiano avendo un metabolismo puramente elettrico.

Sono draghi sociali, che amano stare con i loro simili e sono molto protettivi con la loro prole.
Solitamente non si trova mai un nido da solo, ma interi altopiani dominati da decine di draghi.

Non vanno d'accordo con i draghi viola che disprezzano per la scelta di aver rinunciato al volo per vivere sottoterra.

I Draghi Blu hanno +1d6 nelle prove di magia e possono ignorare un dado tirato nella prova con la Lista dell'Aria ed è immune all'elettricità.\\
\textbf{Incantesimi}\index{Incantesimi da Drago Blu}\\
Gli incantesimi preferiti di questo Drago sono:\\
- \hyperlink{Nebbia mortale}{Nebbia mortale}\\
- \hyperlink{Invocare il Fulmine}{Invocare il Fulmine}\\
- \hyperlink{Tempesta di Ghiaccio}{Tempesta di Ghiaccio}

\mostro{Drago Blu Adulto}
\noindent
\begin{description}[noitemsep, topsep=0pt, parsep=0pt, partopsep=0pt, leftmargin=0cm, labelwidth=2.2cm]
	\item[\textbf{Taglia/Tipo:}] Enorme drago, malvagio
	\item[\textbf{Caratt.:}] \resizebox{0.5\linewidth+1.8cm}{!}{For 7 Des 0 Cos 6 Int 3 Sag 2 Car 4}
	\item[\textbf{Punti Ferita:}] 322,  \textbf{Difesa:} 33,  \textbf{Iniziativa:} +3
	\item[\textbf{Movimento:}] 12 m, scavo 12 m, volo 24 m
	\item[\textbf{Tiri Salvez.:}] \resizebox{0.5\linewidth+1.8cm}{!}{\resizebox{0.5\linewidth+1.8cm}{!}{Tempra +22, Riflessi +16, Volontà +18}}
	\item[\textbf{Comp.:}] Furtività +5, Consapevolezza +13
	\item[\textbf{Imm. Danni:}] Elettricità
	\item[\textbf{Sensi:}] Scurovisione 36 m, Vista Cieca 18 m
	\item[\textbf{Linguaggi:}] Comune, Draconico
	\item[\textbf{Sfida:}] 16 (15000 PX)\smallskip
\end{description}

\emph{\textbf{Scarica elettrica.}} il drago emette nel raggio di 3 metri scariche elettriche magiche che causano 1d6 danni da elettricità a round.

\emph{\textbf{Resistenza Leggendaria (3/Giorno).}} Se il drago fallisce un Tiro Salvezza, può scegliere invece di riuscire.

\textbf{Azioni}

\emph{\textbf{Multiattacco.}} Il drago può usare la sua Presenza Spaventosa e poi effettuare tre attacchi: uno con il morso e due con gli artigli.

\emph{\textbf{Artiglio.} Attacco con arma da mischia}: +14 a colpire, portata 1 m, un bersaglio.

\emph{Colpisce:} 14 (2d6 + 7) danni taglienti, 1 danno da Sanguinamento.

\emph{\textbf{Coda.} Attacco con arma da mischia}: +14 a colpire, portata 5 metri, un bersaglio.

\emph{Colpisce:} 16 (2d8 + 7) danni contundenti.

\emph{\textbf{Morso.} Attacco con arma da mischia}: +14 a colpire, portata 3 m, un bersaglio.

\emph{Colpisce:} 18 (2d10 + 7) danni perforanti più 5 (1d10) danni da elettricità.

\emph{\textbf{Presenza Spaventosa.}} Ogni creatura scelta dal drago, che si trovi entro 36 metri da esso e consapevole della sua presenza, deve riuscire un Tiro Salvezza di Volontà DC 30 o restare spaventata per 1 minuto. Una creatura può ripetere il Tiro Salvezza al termine di ciascun suo round, terminando l'effetto se lo riesce. Se il Tiro Salvezza della creatura ha successo o l'effetto ha termine per essa, la creatura è immune alla Presenza Spaventosa del drago per le successive 24 ore.

\emph{\textbf{Soffio Fulminante (Ricarica 5-6).}} Il drago esala fulmini in una linea lunga 27 metri e larga 1 metro. Ogni creatura su quella linea deve effettuare un Tiro Salvezza di Riflessi DC 30 e subire 66 (12d10) danni da elettricità se fallisce il Tiro Salvezza, o la metà di questi danni se lo riesce.

\textbf{Azioni Aggiuntive}

Il drago può effettuare 3 Azioni aggiuntive, scelte tra le opzioni seguenti. Può usare solo un'opzione Aggiuntiva alla volta e solo al termine del round di un'altra creatura. Il drago recupera le Azioni aggiuntive spese all'inizio del proprio round.

\textbf{Attacco di Ala (Costa 2 Azioni).} Il drago batte le ali. Ogni creatura entro 3 metri dal drago deve riuscire un Tiro Salvezza su Riflessi DC 30 o subire 14 (2d6 + 7) danni contundenti e venir gettato prono. Il drago può poi volare fino a metà della del suo movimento di volo.

\textbf{Attacco di Coda.} Il drago effettua un attacco di coda.

\textbf{Individuare.} Il drago effettua una prova di Consapevolezza.

\emph{\textbf{Arrabbiato:}} Il Drago Blu Adulto può eseguire queste azioni a costo 2 Azioni.

\emph{Focalizzare}: la creatura interrompe un effetto mentale su di se in corso

\emph{Brutalità}: la creatura attacca con ferocia inaudita. +1d6 al Tiro per Colpire, 1 danno critico automatico quando colpisce.

\textbf{Ecologia}\\
Ambiente: Picchi montuosi\\
Organizzazione: Solitario\\
\textbf{Categoria Tesoro}: E\\
\textbf{Descrizione}\\
Vedi Descrizione Drago Blu Antico.\\
\textbf{Incantesimi}\index{Incantesimi da Drago Blu}\\
Gli incantesimi preferiti di questo Drago sono:\\
- \hyperlink{Nebbia mortale}{Nebbia mortale}\\
- \hyperlink{Invocare il Fulmine}{Invocare il Fulmine}\\
- \hyperlink{Tempesta di Ghiaccio}{Tempesta di Ghiaccio}

\mostro{Drago Blu Giovane}
\noindent
\begin{description}[noitemsep, topsep=0pt, parsep=0pt, partopsep=0pt, leftmargin=0cm, labelwidth=2.2cm]
	\item[\textbf{Taglia/Tipo:}] Enorme drago, malvagio
	\item[\textbf{Caratt.:}] \resizebox{0.5\linewidth+1.8cm}{!}{For 5 Des 0 Cos 4 Int 2 Sag 1 Car 3}
	\item[\textbf{Punti Ferita:}] 184,  \textbf{Difesa:} 24,  \textbf{Iniziativa:} +2
	\item[\textbf{Movimento:}] 12 m, scavo 12 m, volo 24 m
	\item[\textbf{Tiri Salvez.:}] \resizebox{0.5\linewidth+1.8cm}{!}{\resizebox{0.5\linewidth+1.8cm}{!}{Tempra +13, Riflessi +9, Volontà +10}}
	\item[\textbf{Comp.:}] Furtività +4, Consapevolezza +9
	\item[\textbf{Imm. Danni:}] Elettricità
	\item[\textbf{Sensi:}] Scurovisione 36 m, Vista Cieca 9 m
	\item[\textbf{Linguaggi:}] Comune, Draconico
	\item[\textbf{Sfida:}] 9 (5000 PX)\smallskip
\end{description}

\textbf{Azioni}

\emph{\textbf{Multiattacco.}} Il drago può effettuare tre attacchi: uno con il morso e due con gli artigli.

\emph{\textbf{Artiglio.} Attacco con arma da mischia}: +12 a colpire, portata 1 m, un bersaglio.

\emph{Colpisce:} 12 (2d6 + 5) danni taglienti, 1 danno da Sanguinamento.

\emph{\textbf{Morso.} Attacco con arma da mischia}: +12 a colpire, portata 3 m, un bersaglio.

\emph{Colpisce:} 16 (2d10 + 5) danni perforanti più 5 (1d10) danni da elettricità.

\emph{\textbf{Soffio Fulminante (Ricarica 5-6).}} Il drago esala fulmini in una linea lunga 18 metri e larga 1 metro. Ogni creatura su quella linea deve effettuare un Tiro Salvezza di Riflessi DC 21 e subire 55 (10d10) danni da elettricità se fallisce il Tiro Salvezza, o la metà di questi danni se lo riesce.

\emph{\textbf{Arrabbiato:}} il Drago Blu Giovane ricarica il suo soffio fulminante.

\textbf{Ecologia}\\
Ambiente: Picchi montuosi\\
Organizzazione: Solitario\\
\textbf{Categoria Tesoro}: D\\
\textbf{Descrizione}\\
Vedi Descrizione Drago Blu Antico.\\
\textbf{Incantesimi}\index{Incantesimi da Drago Blu}\\
Gli incantesimi preferiti di questo Drago sono:\\
- \hyperlink{Nebbia mortale}{Nebbia mortale}\\
- \hyperlink{Invocare il Fulmine}{Invocare il Fulmine}\\
- \hyperlink{Tempesta di Ghiaccio}{Tempesta di Ghiaccio}

\mostro{Drago Blu Cucciolo}
\noindent
\begin{description}[noitemsep, topsep=0pt, parsep=0pt, partopsep=0pt, leftmargin=0cm, labelwidth=2.2cm]
	\item[\textbf{Taglia/Tipo:}] Enorme drago, malvagio
	\item[\textbf{Caratt.:}] \resizebox{0.5\linewidth+1.8cm}{!}{For 3 Des 0 Cos 2 Int 1 Sag 0 Car 2}
	\item[\textbf{Punti Ferita:}] 70,  \textbf{Difesa:} 16,  \textbf{Iniziativa:} +1
	\item[\textbf{Movimento:}] 9 m, scavo 5 metri, volo 18 m
	\item[\textbf{Tiri Salvez.:}] \resizebox{0.5\linewidth+1.8cm}{!}{Tempra +5, Riflessi +3, Volontà +3}
	\item[\textbf{Comp.:}] Furtività +2, Consapevolezza +4
	\item[\textbf{Imm. Danni:}] Elettricità
	\item[\textbf{Sensi:}] Scurovisione 18 m, Vista Cieca 3 m
	\item[\textbf{Linguaggi:}] Draconico
	\item[\textbf{Sfida:}] 3 (700 PX)\smallskip
\end{description}

\textbf{Azioni}

\emph{\textbf{Morso.} Attacco con arma da mischia}: +5 a colpire, portata 1 m, un bersaglio.

\emph{Colpisce:} 8 (1d10 + 3) danni perforanti più 3 (1d6) danni da elettricità.

\emph{\textbf{Soffio Fulminante (Ricarica 5-6).}} Il drago esala fulmini in una linea lunga 9 metri e larga 1 metro. Ogni creatura su quella linea deve effettuare un Tiro Salvezza di Riflessi DC 14 e subire 22 (4d10) danni da elettricità se fallisce il Tiro Salvezza, o la metà di questi danni se lo riesce.

\textbf{Ecologia}\\
Ambiente: Picchi montuosi\\
Organizzazione: Solitario\\
\textbf{Categoria Tesoro}: C\\
\textbf{Descrizione}\\
Vedi Descrizione Drago Blu Antico.

\mostro{Drago Giallo Antico}
\noindent
\begin{description}[noitemsep, topsep=0pt, parsep=0pt, partopsep=0pt, leftmargin=0cm, labelwidth=2.2cm]
	\item[\textbf{Taglia/Tipo:}] Mastodontica drago, malvagio
	\item[\textbf{Caratt.:}] \resizebox{0.5\linewidth+1.8cm}{!}{For 10 Des 1 Cos 8 Int 3 Sag 2 Car 4}
	\item[\textbf{Punti Ferita:}] 465,  \textbf{Difesa:} 43,  \textbf{Iniziativa:} +3
	\item[\textbf{Movimento:}] 12 m, scavo 24 m, scalata 24, volo 12 m
	\item[\textbf{Tiri Salvez.:}] \resizebox{0.5\linewidth+1.8cm}{!}{\resizebox{0.5\linewidth+1.8cm}{!}{Tempra +31, Riflessi +24, Volontà +25}}
	\item[\textbf{Comp.:}] Furtività +7, Consapevolezza +17
	\item[\textbf{Imm. Danni:}] Fuoco, armi +1
	\item[\textbf{Sensi:}] Scurovisione 36 m, Vista Cieca 18 m
	\item[\textbf{Linguaggi:}] Comune, Draconico
	\item[\textbf{Sfida:}] 23 (50000 PX)\smallskip
\end{description}

\emph{\textbf{Calore rovente.}} il drago emette nel raggio di 3 metri calore magico che causa 2d6 danni da fuoco a round.

\emph{\textbf{Resistenza Leggendaria (3/Giorno).}} Se il drago fallisce un Tiro Salvezza, può scegliere invece di riuscire. \\
\emph{\textbf{Resistenza alla Magia:}} 3lv\\
\smallskip\textbf{Azioni}\\
\emph{\textbf{Multiattacco.}} Il drago può usare la sua Presenza Spaventosa e poi effettuare tre attacchi: uno con il morso e due con gli artigli.\\
\emph{\textbf{Artiglio.} Attacco con arma da mischia}: +16 al colpire, portata 3 m, un bersaglio.\\
\emph{Colpisce:} 16 (2d6 + 9) danni taglienti, 3/20 danno da Sanguinamento.\\
\emph{\textbf{Coda.} Attacco con arma da mischia}: +16 al colpire, portata 6 m, un bersaglio.\\
\emph{Colpisce:} 18 (2d8 + 9) danni contundenti.\\
\emph{\textbf{Morso.} Attacco con arma da mischia}: +16 al colpire, portata 5 metri, un bersaglio.\\
\emph{Colpisce:} 20 (2d10 + 9) danni perforanti più 11 (2d10) danni da elettricità.\\
\emph{\textbf{Presenza Spaventosa.}} Ogni creatura scelta dal drago, che si trovi entro 36 metri da esso e consapevole della sua presenza, deve riuscire un Tiro Salvezza su Volontà DC 35 o restare spaventata per 1 minuto. Una creatura può ripetere il Tiro Salvezza al termine di ciascun suo round, terminando l'effetto se lo riesce. Se il Tiro Salvezza della creatura ha successo o l'effetto ha termine per essa, la creatura è immune alla Presenza Spaventosa del drago per le successive 24 ore.\\
\emph{\textbf{Soffio Incendiario (Ricarica 5-6).}} Il drago esala aria rovente in una linea lunga 36 metri e larga 3 metri. Ogni creatura su quella linea deve effettuare un Tiro Salvezza su Riflessi DC 35 e subire 88 (16d10) danni da fuoco se fallisce il Tiro Salvezza, o la metà di questi danni se lo riesce.\\
\textbf{Azioni Aggiuntive}\\
Il drago può effettuare 3 azioni aggiuntive, scelte tra le opzioni seguenti. Può usare solo un'Azione Aggiuntiva alla volta e solo al termine del round di un'altra creatura. Il drago recupera le Azioni Aggiuntive spese all'inizio del proprio round.\\
\textbf{Attacco di Ala (Costa 2 Azioni).} Il drago batte le ali. Ogni creatura entro 5 metri dal drago deve riuscire un Tiro Salvezza su Riflessi DC 35 o subire 16 (2d6 + 9) danni contundenti e venir gettato prono. Il drago può poi volare fino a metà della sua velocità di volo.\\
\textbf{Attacco di Coda.} Il drago effettua un attacco di coda.\\
\textbf{Individuare.} Il drago effettua una prova di Consapevolezza.\\
\textbf{Ecologia}\\
Ambiente: Deserti Caldi\\
Organizzazione: Solitario\\
\textbf{Categoria Tesoro}: H\\
\textbf{Descrizione}\\
I Draghi Gialli hanno scaglie di vari toni di giallo che con la crescita prendono ad assomigliare sempre di più al colore delle sabbie dove dimorano, dal giallo chiaro all'ocra mattone.

Sono molto intelligenti ma essendo per natura solitari non hanno interesse a comunicare con le altre razze.

Vivono nei deserti dove spesso tendono agguati alle loro prede nascondendosi sul fondo di ampie buche scavate nella sabbia.
Appena percepiscono un movimento sopra di loro escono e divorano qualunque creatura.
Hanno una passione per la carne dei nani che trovano saporita anche se asciutta.

Il Drago Giallo pur se intelligente è una macchina di morte e difficilmente scende a patti, solo se si trova in serio pericolo.

Un Drago Giallo hanno +1d6 nelle prove di magia e possono ignorare un dado tirato nella prova con la Lista del Fuoco ed è immune al fuoco.
\\
\textbf{Incantesimi}\index{Incantesimi da Drago Giallo}\\
Gli incantesimi preferiti di questo Drago sono:\\
- \hyperlink{Creare Cibo e Acqua}{Creare Cibo e Acqua}\\
- \hyperlink{Muro di Fuoco}{Muro di Fuoco}\\
- \hyperlink{Scudo di Fuoco}{Scudo di Fuoco}

\mostro{Drago Nero Antico}
\noindent
\begin{description}[noitemsep, topsep=0pt, parsep=0pt, partopsep=0pt, leftmargin=0cm, labelwidth=2.2cm]
	\item[\textbf{Taglia/Tipo:}] Mastodontica drago, malvagio
	\item[\textbf{Caratt.:}] \resizebox{0.5\linewidth+1.8cm}{!}{For 8 Des 2 Cos 7 Int 3 Sag 2 Car 4}
	\item[\textbf{Punti Ferita:}] 422,  \textbf{Difesa:} 42,  \textbf{Iniziativa:} +3
	\item[\textbf{Movimento:}] 12 m, scalata 12 m, volo 24 m
	\item[\textbf{Tiri Salvez.:}] \resizebox{0.5\linewidth+1.8cm}{!}{\resizebox{0.5\linewidth+1.8cm}{!}{Tempra +28, Riflessi +23, Volontà +23}}
	\item[\textbf{Comp.:}] Furtività +9, Consapevolezza +16
	\item[\textbf{Imm. Danni:}] Acido, armi +1
	\item[\textbf{Sensi:}] Scurovisione 36 m, Vista Cieca 18 m
	\item[\textbf{Linguaggi:}] Comune, Draconico
	\item[\textbf{Sfida:}] 21 (33000 PX)\smallskip
\end{description}

\emph{\textbf{Gas corrosivi.}} il drago emette nel raggio di 3 metri gas corrosivi che causano 2d6 danni da acido a round.

\emph{\textbf{Anfibio.}} Il drago può respirare aria e acqua.

\emph{\textbf{Resistenza Leggendaria (3/Giorno).}} Se il drago fallisce un Tiro Salvezza, può scegliere invece di riuscire.

\emph{\textbf{Resistenza alla Magia:}} 3lv

\textbf{Azioni}

\emph{\textbf{Multiattacco.}} Il drago può usare la sua Presenza Spaventosa e poi effettuare tre attacchi: uno con il morso e due con gli artigli.

\emph{\textbf{Artiglio.} Attacco con arma da mischia}: +16 a colpire, portata 3 m, un bersaglio.

\emph{Colpisce:} 15 (2d6 + 8) danni taglienti, 3/20 danno da Sanguinamento.

\emph{\textbf{Coda.} Attacco con arma da mischia}: +16 a colpire, portata 6 m, un bersaglio.

\emph{Colpisce:} 17 (2d8 + 8) danni contundenti.

\emph{\textbf{Morso.} Attacco con arma da mischia} : +16 a colpire, portata 5 metri, un bersaglio.

\emph{Colpisce:} 19 (2d10 + 8) danni perforanti più 9 (4d6) danni da acido.

\emph{\textbf{Presenza Spaventosa.}} Ogni creatura scelta dal drago, che si trovi entro 36 metri da esso e consapevole della sua presenza, deve riuscire un Tiro Salvezza di Volontà DC 33 o restare spaventata per 1 minuto. Una creatura può ripetere il Tiro Salvezza al termine di ciascun suo round, terminando l'effetto se lo riesce. Se il Tiro Salvezza della creatura ha successo o l'effetto ha termine per essa, la creatura è immune alla Presenza Spaventosa del drago per le successive 24 ore.

\emph{\textbf{Soffio Acido (Ricarica 5-6).}} Il drago esala acido in una linea di 27 metri larga 3 metri. Ogni creatura in quell'area deve effettuare un Tiro Salvezza di Riflessi DC 33 e subire 67 (15d8) danni da acido se fallisce il Tiro Salvezza, o la metà di questi danni se lo riesce.

\textbf{Azioni Aggiuntive}

Il drago può effettuare 3 Azioni aggiuntive, scelte tra le opzioni seguenti. Può usare solo un'opzione Aggiuntiva alla volta e solo al termine del round di un'altra creatura. Il drago recupera le Azioni aggiuntive spese all'inizio del proprio round.

\textbf{Attacco di Ala (Costa 2 Azioni).} Il drago batte le ali. Ogni creatura entro 5 metri dal drago deve riuscire un Tiro Salvezza su Riflessi DC 33 o subire 15 (2d6 + 8) danni contundenti e venir gettato prono. Il drago può poi volare fino a metà del suo movimento di volo.

\textbf{Attacco di Coda.} Il drago effettua un attacco di coda.

\textbf{Individuare.} Il drago effettua una prova di Consapevolezza.

\textbf{Ecologia}\\
Ambiente: Paludi Calde\\
Organizzazione: Solitario\\
\textbf{Categoria Tesoro}: D\\
\textbf{Descrizione}\\
I Draghi Neri sono violenti ed aggressivi, vivono in paludi e acquitrini e che generalmente governano come padroni indiscussi.

I Draghi Neri sono creature minacciose che hanno grandi corna curve in avanti.
La testa si collega ad un collo relativamente corto e ad un corpo da lucertola grossa e muscoloso.

Hanno ali piccolissime che si trovano sui lati, ma riescono comunque a volare grazie alla magia.
Hanno le zampe palmate per permettere loro di nuotare con maggiore facilità nelle zone paludose dove vivono.

I Draghi Neri tendono a fare le loro tane al centro della palude o acquitrino.
Considerano quel territorio il loro e nessuno può bagnarsi senza subire la loro ira.

Una tana di drago nero può essere un ammasso gigantesco di tronchi ma anche una caverna sotterranea sommersa d'acqua, se non il fondo di un lago.
Potendo respirare sott'acqua non si fanno preoccupazione su dove costruire la loro dimora.

La loro casa è sempre protetta da trappole e dai loro seguaci malvagi che gli portano cibo, possibilmente vivo.

L'ambiente dove vive un drago nero ne subisce i suoi effetti, vapori acidi, distruzione, corruzione sono immediatamente percepibili.

Il Drago Nero rappresentano i Tratti dell'egoismo e violenza odiando ogni forma di vita, compreso gli stessi draghi neri.

I Draghi neri hanno +1d6 nelle prove di magia e possono ignorare un dado tirato nella prova con la Lista di Necromanzia ed è immune all'acido.\\
\textbf{Incantesimi}\index{Incantesimi da Drago Nero}\\
Gli incantesimi preferiti di questo Drago sono:\\
- \hyperlink{Animare Morti}{Animare Morti}\\
- \hyperlink{Creare Non Morti}{Creare Non Morti}\\
- \hyperlink{Scagliare Maledizione}{Scagliare Maledizione}

Ebbene si, il Drago Nero è l'unica creatura sulla Terra che può portare in vita un morto a discapito di tutti i vincoli imposti dai Patroni.

\mostro{Drago Nero Adulto}
\noindent
\begin{description}[noitemsep, topsep=0pt, parsep=0pt, partopsep=0pt, leftmargin=0cm, labelwidth=2.2cm]
	\item[\textbf{Taglia/Tipo:}] Enorme drago, malvagio
	\item[\textbf{Caratt.:}] \resizebox{0.5\linewidth+1.8cm}{!}{For 6 Des 2 Cos 5 Int 2 Sag 1 Car 3}
	\item[\textbf{Punti Ferita:}] 338,  \textbf{Difesa:} 36,  \textbf{Iniziativa:} +2
	\item[\textbf{Movimento:}] 12 m, scalata 12 m, volo 24 m
	\item[\textbf{Tiri Salvez.:}] \resizebox{0.5\linewidth+1.8cm}{!}{\resizebox{0.5\linewidth+1.8cm}{!}{Tempra +22, Riflessi +19, Volontà +18}}
	\item[\textbf{Comp.:}] Furtività +7, Consapevolezza +11
	\item[\textbf{Imm. Danni:}] Acido
	\item[\textbf{Sensi:}] Scurovisione 36 m, Vista Cieca 18 m
	\item[\textbf{Linguaggi:}] Comune, Draconico
	\item[\textbf{Sfida:}] 17 (18000 PX)\smallskip
\end{description}

\emph{\textbf{Gas corrosivi.}} il drago emette nel raggio di 3 metri gas corrosivi che causano 1d6 danni da acido a round.

\emph{\textbf{Anfibio.}} Il drago può respirare aria e acqua.

\emph{\textbf{Resistenza Leggendaria (3/Giorno).}} Se il drago fallisce un Tiro Salvezza, può scegliere invece di riuscire.

\textbf{Azioni}

\emph{\textbf{Multiattacco.}} Il drago può usare la sua Presenza Spaventosa e poi effettuare tre attacchi: uno con il morso e due con gli artigli.

\emph{\textbf{Artiglio.} Attacco con arma da mischia}: +14 a colpire, portata 1 m, un bersaglio.

\emph{Colpisce:} 13 (2d6 + 6) danni taglienti, 1 danno da Sanguinamento.

\emph{\textbf{Coda.} Attacco con arma da mischia}: +14 a colpire, portata 5 metri, un bersaglio.

\emph{Colpisce:} 15 (2d8 + 6) danni contundenti.

\emph{\textbf{Morso.} Attacco con arma da mischia}: +14 a colpire, portata 3 m, un bersaglio.

\emph{Colpisce:} 17 (2d10 + 6) danni perforanti più 4 (1d8) danni da acido.

\emph{\textbf{Presenza Spaventosa.}} Ogni creatura scelta dal drago, che si trovi entro 36 metri da esso e consapevole della sua presenza, deve riuscire un Tiro Salvezza di Volontà DC 30 o restare spaventata per 1 minuto. Una creatura può ripetere il Tiro Salvezza al termine di ciascun suo round, terminando l'effetto se lo riesce. Se il Tiro Salvezza della creatura ha successo o l'effetto ha termine per essa, la creatura è immune alla Presenza Spaventosa del drago per le successive 24 ore.

\emph{\textbf{Soffio Acido (Ricarica 5-6).}} Il drago esala acido in una linea di 18 metri larga 1 metro. Ogni creatura in quell'area deve effettuare un Tiro Salvezza di Riflessi DC 30 e subire 54 (12d8) danni da acido se fallisce il Tiro Salvezza, o la metà di questi danni se lo riesce.

\textbf{Azioni Aggiuntive}

Il drago può effettuare 3 Azioni aggiuntive, scelte tra le opzioni seguenti. Può usare solo un'opzione Aggiuntiva alla volta e solo al termine del round di un'altra creatura. Il drago recupera le Azioni aggiuntive spese all'inizio del proprio round.

\textbf{Attacco di Ala (Costa 2 Azioni).} Il drago batte le ali. Ogni creatura entro 3 metri dal drago deve riuscire un Tiro Salvezza su Riflessi DC 30 o subire 13 (2d6 + 6) danni contundenti e venir gettato prono. Il drago può poi volare fino a metà della del suo movimento di volo.

\textbf{Attacco di Coda.} Il drago effettua un attacco di coda.

\textbf{Individuare.} Il drago effettua una prova di Consapevolezza.

\emph{\textbf{Arrabbiato:}} Il Drago Nero Adulto può eseguire queste azioni a costo 2 Azioni.

\emph{Focalizzare}: la creatura interrompe un effetto mentale su di se in corso

\emph{Brutalità}: la creatura attacca con ferocia inaudita. +1d6 al Tiro per Colpire, 1 danno critico automatico quando colpisce.

\textbf{Ecologia}\\
Ambiente: Paludi Calde\\
Organizzazione: Solitario\\
\textbf{Categoria Tesoro}: D\\
\textbf{Descrizione}\\
Vedi Descrizione Drago Nero Antico.\\
\textbf{Incantesimi}\index{Incantesimi da Drago Nero}\\
Gli incantesimi preferiti di questo Drago sono:\\
- \hyperlink{Animare Morti}{Animare Morti}\\
- \hyperlink{Creare Non Morti}{Creare Non Morti}\\
- \hyperlink{Scagliare Maledizione}{Scagliare Maledizione}

\mostro{Drago Nero Giovane}
\noindent
\begin{description}[noitemsep, topsep=0pt, parsep=0pt, partopsep=0pt, leftmargin=0cm, labelwidth=2.2cm]
	\item[\textbf{Taglia/Tipo:}] Grande drago, malvagio
	\item[\textbf{Caratt.:}] \resizebox{0.5\linewidth+1.8cm}{!}{For 4 Des 2 Cos 3 Int 1 Sag 0 Car 2}
	\item[\textbf{Punti Ferita:}] 145,  \textbf{Difesa:} 23,  \textbf{Iniziativa:} +2
	\item[\textbf{Movimento:}] 12 m, scalata 12 m, volo 24 m
	\item[\textbf{Tiri Salvez.:}] \resizebox{0.5\linewidth+1.8cm}{!}{Tempra +10, Riflessi +9, Volontà +7}
	\item[\textbf{Comp.:}] Furtività +5, Consapevolezza +6
	\item[\textbf{Imm. Danni:}] Acido
	\item[\textbf{Sensi:}] Scurovisione 36 m, Vista Cieca 9 m
	\item[\textbf{Linguaggi:}] Comune, Draconico
	\item[\textbf{Sfida:}] 7 (2900 PX)\smallskip
\end{description}

\emph{\textbf{Anfibio.}} Il drago può respirare aria e acqua.

\textbf{Azioni}

\emph{\textbf{Multiattacco.}} Il drago può effettuare tre attacchi: uno con il morso e due con gli artigli.

\emph{\textbf{Artiglio.} Attacco con arma da mischia}: +8 a colpire, portata 1 m, un bersaglio.

\emph{Colpisce:} 11 (2d6 + 4) danni taglienti, 1 danno da Sanguinamento.

\emph{\textbf{Morso.} Attacco con arma da mischia}: +8 a colpire, portata 3 m, un bersaglio.

\emph{Colpisce:} 11 (2d10 + 4) danni perforanti più 4 (1d8) danni da acido.

\emph{\textbf{Soffio Acido (Ricarica 5-6).}} Il drago esala acido in una linea di 9 metri larga 1 metro. Ogni creatura in quell'area deve effettuare un Tiro Salvezza di Riflessi DC 19 e subire 49 (11d8) danni da acido se fallisce il Tiro Salvezza, o la metà di questi danni se lo riesce.

\emph{\textbf{Arrabbiato:}} Il Drago Nero Giovane ricarica il soffio acido. Costa 1 Azione.

\textbf{Ecologia}\\
Ambiente: Paludi Calde\\
Organizzazione: Solitario\\
\textbf{Categoria Tesoro}: C\\
\textbf{Descrizione}\\
Vedi Descrizione Drago Nero Antico.\\
\textbf{Incantesimi}\index{Incantesimi da Drago Nero}\\
Gli incantesimi preferiti di questo Drago sono:\\
- \hyperlink{Animare Morti}{Animare Morti}\\
- \hyperlink{Creare Non Morti}{Creare Non Morti}\\
- \hyperlink{Scagliare Maledizione}{Scagliare Maledizione}

\medskip

\begin{center}
	\includegraphics[width=0.9\linewidth]{immagini/Friedrich-Johann-Justin-Bertuch_Mythical-Creature-Dragon_1806.png}

	\emph{Drago, Friedrich Johann Justin Bertuch}
\end{center}

\mostro{Drago Nero Cucciolo}
\noindent
\begin{description}[noitemsep, topsep=0pt, parsep=0pt, partopsep=0pt, leftmargin=0cm, labelwidth=2.2cm]
	\item[\textbf{Taglia/Tipo:}] Media drago, malvagio
	\item[\textbf{Caratt.:}] \resizebox{0.5\linewidth+1.8cm}{!}{For 2 Des 2 Cos 1 Int 0 Sag 0 Car 1}
	\item[\textbf{Punti Ferita:}] 51,  \textbf{Difesa:} 16,  \textbf{Iniziativa:} +2
	\item[\textbf{Movimento:}] 9 m, scalata 9 m, volo 18 m
	\item[\textbf{Tiri Salvez.:}] \resizebox{0.5\linewidth+1.8cm}{!}{Tempra +3, Riflessi +4, Volontà +3}
	\item[\textbf{Comp.:}] Furtività +4, Consapevolezza +4
	\item[\textbf{Imm. Danni:}] Acido
	\item[\textbf{Sensi:}] Scurovisione 18 m, Vista Cieca 3 m
	\item[\textbf{Linguaggi:}] Draconico
	\item[\textbf{Sfida:}] 2 (450 PX)\smallskip
\end{description}

\emph{\textbf{Anfibio.}} Il drago può respirare aria e acqua.

\textbf{Azioni}

\emph{\textbf{Morso.} Attacco con arma da mischia}: +5 a colpire, portata 1 m, un bersaglio.

\emph{Colpisce:} 7 (1d10 + 2) danni perforanti più 2 (1d4) danni da acido.

\emph{\textbf{Soffio Acido (Ricarica 5-6).}} Il drago esala acido in una linea di 5 metri larga 1 metro. Ogni creatura in quell'area deve effettuare un Tiro Salvezza di Riflessi DC 14 e subire 22 (5d8) danni da acido se fallisce il Tiro Salvezza, o la metà di questi danni se lo riesce.

\textbf{Ecologia}\\
Ambiente: Paludi Calde\\
Organizzazione: Solitario\\
\textbf{Categoria Tesoro}: H\\
\textbf{Descrizione}\\
Vedi Descrizione Drago Nero Antico.

\mostro{Drago Porpora Antico}
\noindent
\begin{description}[noitemsep, topsep=0pt, parsep=0pt, partopsep=0pt, leftmargin=0cm, labelwidth=2.2cm]
	\item[\textbf{Taglia/Tipo:}] Mastodontica drago, malvagio
	\item[\textbf{Caratt.:}] \resizebox{0.5\linewidth+1.8cm}{!}{For 8 Des 3 Cos 4 Int 4 Sag 4 Car 4}
	\item[\textbf{Punti Ferita:}] 428,  \textbf{Difesa:} 44,  \textbf{Iniziativa:} +4
	\item[\textbf{Movimento:}] 12 m, scavare 24 m
	\item[\textbf{Tiri Salvez.:}] \resizebox{0.5\linewidth+1.8cm}{!}{\resizebox{0.5\linewidth+1.8cm}{!}{Tempra +26, Riflessi +25, Volontà +26}}
	\item[\textbf{Comp.:}] Conoscenza Dungeon +8, Intimidazione +11, Percepire Emozioni +10, Consapevolezza + 15
	\item[\textbf{Imm. Danni:}] Suono, armi +1
	\item[\textbf{Sensi:}] Scurovisione 36 m, Senso Tellurico 72 m
	\item[\textbf{Linguaggi:}] Comune, Draconico
	\item[\textbf{Sfida:}] 22 (41000 PX)\smallskip
\end{description}

\emph{\textbf{Onde esplosive.}} il drago emette nel raggio di 3 metri vibrazioni sonore che causano 2d6 danni da suono a round.

\emph{\textbf{Terrestre.}} Il drago finché è sotto terra può non respirare ne mangiare.

\emph{\textbf{Resistenza Leggendaria (3/Giorno).}} Se il drago fallisce un Tiro Salvezza, può scegliere invece di riuscire.

\emph{\textbf{Resistenza alla Magia:}} 3lv

\textbf{Azioni}

\emph{\textbf{Multiattacco.}} Il drago può usare la sua Presenza Spaventosa e poi effettuare tre attacchi: uno con il morso e due con gli artigli.

\emph{\textbf{Artiglio.} Attacco con arma da mischia}: +17 a colpire, portata 3 m, un bersaglio.

\emph{Colpisce:} 15 (2d6 + 8) danni taglienti, 3/20 danno da Sanguinamento.

\emph{\textbf{Coda.} Attacco con arma da mischia}: +17 a colpire, portata 6 m, un bersaglio.

\emph{Colpisce:} 17 (2d8 + 8) danni contundenti.

\emph{\textbf{Morso.} Attacco con arma da mischia}: +17 a colpire, portata 5 metri, un bersaglio.

\emph{Colpisce:} 19 (2d10 + 8) danni perforanti più 10 (3d6) danni da veleno.

\emph{\textbf{Presenza Spaventosa.}} Ogni creatura scelta dal drago, che si trovi entro 36 metri da esso e consapevole della sua presenza, deve riuscire un Tiro Salvezza di Volontà DC 35 o restare spaventata per 1 minuto. Una creatura può ripetere il Tiro Salvezza al termine di ciascun suo round, terminando l'effetto se lo riesce. Se il Tiro Salvezza della creatura ha successo o l'effetto ha termine per essa, la creatura è immune alla Presenza Spaventosa del drago per le successive 24 ore.

\emph{\textbf{Soffio Sonico (Ricarica 5-6).}} Il drago emette un cono di 27 metri. Ogni creatura in quell'area deve effettuare un Tiro Salvezza di Tempra DC 35 e subire 77 (22d6) danni da suono se fallisce il Tiro Salvezza, o la metà di questi danni se lo riesce.

\textbf{Azioni Aggiuntive}

Il drago può effettuare 3 Azioni aggiuntive, scelte tra le opzioni seguenti. Può usare solo un'opzione Aggiuntiva alla volta e solo al termine del round di un'altra creatura. Il drago recupera le Azioni aggiuntive spese all'inizio del proprio round.

\textbf{Schianto (Costa 2 Azioni).} Il drago salta sul posto. Ogni creatura entro 5 metri dal drago deve riuscire un Tiro Salvezza su Riflessi DC 35 o subire 15 (2d6 + 8) danni contundenti e venire gettato prono. Il drago può poi spostarsi fino a metà del suo movimento.

\textbf{Attacco di Coda.} Il drago effettua un attacco di coda.

\textbf{Individuare.} Il drago effettua una prova di Consapevolezza.

\textbf{Ecologia}\\
Ambiente: Caverne\\
Organizzazione: Solitario, creature sotterranee\\
\textbf{Categoria Tesoro}: E\\
\textbf{Descrizione}\\
I Draghi Porpora vivono sotto terra e si sono perfettamente adattati alla vita sotterranea.
Capaci di vedere al buio come se fosse pieno giorno, dotati di Senso Tellurico, hanno perso la capacità di volare ma acquisito quella di scavare con la stessa velocità come se corressero.

Un Drago Porpora è molto territoriale e stabilirà un perimetro (di circa 5 km di raggio) dove crea, se non già presente un intricata serie di cunicoli e caverne per i suoi servi.

Un Drago Porpora è molto protettivo con le sue creature, con chi gli porta da mangiare e gli offre tesori.

Dall'aspetto tozzo hanno denti fini e lunghi ed artigli enormi che continuamente crescono. Hanno un potentissimo attacco sonico che spesso crea crolli nelle caverne, crolli che sono completamente indifferente a lui.

E' forte e coraggioso, arrogante ma non sfrontato. Non ha paura di combattere se pensa di vincere. Porta sempre la battaglia sottoterra dove può creare fosse per fare precipitare i nemici o scappare se necessario.

Un Drago Porpora ha +1d6 nelle prove di magia e può ignorare un dado tirato nella prova con la Lista della Terra, è immune al danno ed effetti sonori.\\
\textbf{Incantesimi}\index{Incantesimi da Drago Porpora}\\
Gli incantesimi preferiti di questo Drago sono:\\
- \hyperlink{Freccia Acida di Restser}{Freccia Acida di Restser}\\
- \hyperlink{Passare Senza Tracce}{Passare Senza Tracce}\\
- \hyperlink{Scolpire Pietra}{Scolpire Pietra}

\mostro{Drago Rosso Antico}
\noindent
\begin{description}[noitemsep, topsep=0pt, parsep=0pt, partopsep=0pt, leftmargin=0cm, labelwidth=2.2cm]
	\item[\textbf{Taglia/Tipo:}] Mastodontica drago, malvagio
	\item[\textbf{Caratt.:}] \resizebox{0.5\linewidth+1.8cm}{!}{For 10 Des 0 Cos 9 Int 4 Sag 2 Car 6}
	\item[\textbf{Punti Ferita:}] 490,  \textbf{Difesa:} 44,  \textbf{Iniziativa:} +4
	\item[\textbf{Movimento:}] 12 m, scalata 12 m, volo 24 m
	\item[\textbf{Tiri Salvez.:}] \resizebox{0.5\linewidth+1.8cm}{!}{\resizebox{0.5\linewidth+1.8cm}{!}{Tempra +33, Riflessi +24, Volontà +26}}
	\item[\textbf{Comp.:}] Furtività +7, Consapevolezza +16
	\item[\textbf{Imm. Danni:}] Fuoco, armi +1
	\item[\textbf{Sensi:}] Scurovisione 36 m, Vista Cieca 18 m
	\item[\textbf{Linguaggi:}] Comune, Draconico
	\item[\textbf{Sfida:}] 24 (62000 PX)\smallskip
\end{description}

\emph{\textbf{Aura di fiamma.}} il drago emette nel raggio di 3 metri calore magico che causa 2d6 danni da fuoco a round.

\emph{\textbf{Resistenza Leggendaria (3/Giorno).}} Se il drago fallisce un Tiro Salvezza, può scegliere invece di riuscire.

\emph{\textbf{Resistenza alla Magia:}} 3lv

\textbf{Azioni}

\emph{\textbf{Multiattacco.}} Il drago può usare la sua Presenza Spaventosa e poi effettuare tre attacchi: uno con il morso e due con gli artigli.

\emph{\textbf{Artiglio.} Attacco con arma da mischia}: +18 a colpire, portata 3 m, un bersaglio.

\emph{Colpisce:} 17 (2d6 + 10) danni taglienti, 3/20 danno da Sanguinamento.

\emph{\textbf{Coda.} Attacco con arma da mischia}: +18 a colpire, portata 6 m, un bersaglio.

\emph{Colpisce:} 19 (2d8 + 10) danni contundenti.

\emph{\textbf{Morso.} Attacco con arma da mischia}: +18 a colpire, portata 5 metri, un bersaglio.

\emph{Colpisce:} 21 (2d10 + 10) danni perforanti più 14 (4d6) danni da fuoco.

\emph{\textbf{Presenza Spaventosa.}} Ogni creatura scelta dal drago, che si trovi entro 36 metri da esso e consapevole della sua presenza, deve riuscire un Tiro Salvezza di Volontà DC 38 o restare spaventata per 1 minuto. Una creatura può ripetere il Tiro Salvezza al termine di ciascun suo round, terminando l'effetto se lo riesce. Se il Tiro Salvezza della creatura ha successo o l'effetto ha termine per essa, la creatura è immune alla Presenza Spaventosa del drago per le successive 24 ore.

\emph{\textbf{Soffio Infuocato (Ricarica 5-6).}} Il drago esala fuoco in un cono di 27 metri. Ogni creatura in quell'area deve effettuare un Tiro Salvezza su Riflessi DC 38 e subire 91 (26d6) danni da fuoco se fallisce il Tiro Salvezza, o la metà di questi danni se lo riesce.

\textbf{Azioni Aggiuntive}

Il drago può effettuare 3 Azioni aggiuntive, scelte tra le opzioni seguenti. Può usare solo un'opzione Aggiuntiva alla volta e solo al termine del round di un'altra creatura. Il drago recupera le Azioni aggiuntive spese all'inizio del proprio round.

\textbf{Attacco di Ala (Costa 2 Azioni).} Il drago batte le ali. Ogni creatura entro 5 metri dal drago deve riuscire un Tiro Salvezza su Riflessi DC 38 o subire 17 (2d6 + 10) danni contundenti e venir gettato prono. Il drago può poi volare fino a metà del suo movimento di volo.

\textbf{Attacco di Coda.} Il drago effettua un attacco di coda.

\textbf{Individuare.} Il drago effettua una prova di Consapevolezza.

\emph{\textbf{Arrabbiato}}: Il drago rosso si scuote e ruggisce. 1 volta al giorno, la prima volta che è Arrabbiato, termina tutte le condizioni negative su di se e tutte le abilità si ricaricano. Il soffio si ricarica con 3-6.

\textbf{Ecologia}\\
Ambiente: Montagne calde\\
Organizzazione: Solitario\\
\textbf{Categoria Tesoro}: H\\
\textbf{Descrizione}\\
Il Drago Rosso si crede il Re dei Draghi per via della sua potenza fisica e del soffio capace di sciogliere la pietra.

I Draghi Rossi sono i draghi più grandi sia per corporatura che per apertura alare.
Spesso le scaglie, di un rosso scuro quasi di sangue, hanno bordi affilati ed allungati.

I Draghi Rossi prediligono le montagne calde e se possibile direttamente direttamente dentro un vulcano.

Combattono sfruttando la loro mole, le ali, il morso artigli.. insomma tutto ciò che sono ed hanno a disposizione. Un Drago Rosso combatte sempre fino alla morte non si ritira ne scappa ne rinuncia ad una sfida, l'orgoglio di cui sono tronfi non gli permette di mostrarsi deboli.

Un Drago Rosso hanno +1d6 nelle prove di magia e possono ignorare un dado tirato nella prova con la Lista del Fuoco ed è immune al fuoco.\\
\textbf{Incantesimi}\index{Incantesimi da Drago Rosso}\\
Gli incantesimi preferiti di questo Drago sono:\\
- \hyperlink{Palla di Fuoco}{Palla di Fuoco}\\
- \hyperlink{Nube Incendiaria}{Nube Incendiaria}\\
- \hyperlink{Muro di Fuoco}{Muro di Fuoco}


\begin{center}
	\includegraphics[width=0.9\linewidth]{immagini/Pair_of_winged_dragons.png}

	\emph{China, Pair of winged dragons, 4th-5th century}
\end{center}

\mostro{Drago Rosso Adulto}
\noindent
\begin{description}[noitemsep, topsep=0pt, parsep=0pt, partopsep=0pt, leftmargin=0cm, labelwidth=2.2cm]
	\item[\textbf{Taglia/Tipo:}] Enorme drago, malvagio
	\item[\textbf{Caratt.:}] \resizebox{0.5\linewidth+1.8cm}{!}{For 8 Des 0 Cos 7 Int 3 Sag 1 Car 5}
	\item[\textbf{Punti Ferita:}] 344,  \textbf{Difesa:} 34,  \textbf{Iniziativa:} +3
	\item[\textbf{Movimento:}] 12 m, scalata 12 m, volo 24 m
	\item[\textbf{Tiri Salvez.:}] \resizebox{0.5\linewidth+1.8cm}{!}{\resizebox{0.5\linewidth+1.8cm}{!}{Tempra +24, Riflessi +17, Volontà +18}}
	\item[\textbf{Comp.:}] Furtività +6, Consapevolezza +13
	\item[\textbf{Imm. Danni:}] Fuoco
	\item[\textbf{Sensi:}] Scurovisione 36 m, Vista Cieca 18 m
	\item[\textbf{Linguaggi:}] Comune, Draconico
	\item[\textbf{Sfida:}] 17 (18000 PX)\smallskip
\end{description}

\emph{\textbf{Aura di fiamma.}} il drago emette nel raggio di 3 metri calore magico che causa 1d6 danni da fuoco a round.

\emph{\textbf{Resistenza Leggendaria (3/Giorno).}} Se il drago fallisce un Tiro Salvezza, può scegliere invece di riuscire.

\textbf{Azioni}

\emph{\textbf{Multiattacco.}} Il drago può usare la sua Presenza Spaventosa e poi effettuare tre attacchi: uno con il morso e due con gli artigli.

\emph{\textbf{Artiglio.} Attacco con arma da mischia}: +14 a colpire, portata 1 m, un bersaglio.

\emph{Colpisce:} 15 (2d6 + 8) danni taglienti, 1 danno da Sanguinamento.

\emph{\textbf{Coda.} Attacco con arma da mischia}: +14 a colpire, portata 5 metri, un bersaglio.

\emph{Colpisce:} 17 (2d8 + 8) danni contundenti.

\emph{\textbf{Morso.} Attacco con arma da mischia}: +14 a colpire, portata 3 m, un bersaglio.

\emph{Colpisce:} 19 (2d10 + 8) danni perforanti più 7 (2d6) danni da fuoco.

\emph{\textbf{Presenza Spaventosa.}} Ogni creatura scelta dal drago, che si trovi entro 36 metri da esso e consapevole della sua presenza, deve riuscire un Tiro Salvezza di Volontà DC 30 o restare spaventata per 1 minuto. Una creatura può ripetere il Tiro Salvezza al termine di ciascun suo round, terminando l'effetto se lo riesce. Se il Tiro Salvezza della creatura ha successo o l'effetto ha termine per essa, la creatura è immune alla Presenza Spaventosa del drago per le successive 24 ore.

\emph{\textbf{Soffio Infuocato (Ricarica 5-6).}} Il drago esala fuoco in un cono di 18 metri. Ogni creatura in quell'area deve effettuare un Tiro Salvezza su Riflessi DC 30 e subire 63 (18d6) danni da fuoco se fallisce il Tiro Salvezza, o la metà di questi danni se lo riesce.

\textbf{Azioni Aggiuntive}

Il drago può effettuare 3 Azioni aggiuntive, scelte tra le opzioni seguenti. Può usare solo un'opzione Aggiuntiva alla volta e solo al termine del round di un'altra creatura. Il drago recupera le Azioni aggiuntive spese all'inizio del proprio round.

\textbf{Attacco di Ala (Costa 2 Azioni).} Il drago batte le ali. Ogni creatura entro 3 metri dal drago deve riuscire un Tiro Salvezza su Riflessi DC 30 o subire 15 (2d6 + 8) danni contundenti e venir gettato prono.

Il drago può poi volare fino a metà del suo movimento di volo.

\textbf{Attacco di Coda.} Il drago effettua un attacco di coda.

\textbf{Individuare.} Il drago effettua una prova di Consapevolezza.

\emph{\textbf{Arrabbiato:}} Il Drago Rosso Adulto può eseguire queste azioni a costo 2 Azioni.

\emph{Focalizzare}: la creatura interrompe un effetto mentale su di se in corso

\emph{Brutalità}: la creatura attacca con ferocia inaudita. +1d6 al Tiro per Colpire, 1 danno critico automatico quando colpisce.

\textbf{Ecologia}\\
Ambiente: Montagne calde\\
Organizzazione: Solitario\\
\textbf{Categoria Tesoro}: C\\
\textbf{Descrizione}\\
Vedi Descrizione Drago Rosso Antico.\\
\textbf{Incantesimi}\index{Incantesimi da Drago Rosso}\\
Gli incantesimi preferiti di questo Drago sono:\\
- \hyperlink{Palla di Fuoco}{Palla di Fuoco}\\
- \hyperlink{Nube Incendiaria}{Nube Incendiaria}\\
- \hyperlink{Muro di Fuoco}{Muro di Fuoco}


\mostro{Drago Rosso Giovane}
\noindent
\begin{description}[noitemsep, topsep=0pt, parsep=0pt, partopsep=0pt, leftmargin=0cm, labelwidth=2.2cm]
	\item[\textbf{Taglia/Tipo:}] Grande drago, malvagio
	\item[\textbf{Caratt.:}] \resizebox{0.5\linewidth+1.8cm}{!}{For 6 Des 0 Cos 5 Int 2 Sag 0 Car 4}
	\item[\textbf{Punti Ferita:}] 205,  \textbf{Difesa:} 25,  \textbf{Iniziativa:} +2
	\item[\textbf{Movimento:}] 12 m, scalata 12 m, volo 24 m
	\item[\textbf{Tiri Salvez.:}] \resizebox{0.5\linewidth+1.8cm}{!}{\resizebox{0.5\linewidth+1.8cm}{!}{Tempra +15, Riflessi +10, Volontà +10}}
	\item[\textbf{Comp.:}] Furtività +4, Consapevolezza +8
	\item[\textbf{Imm. Danni:}] Fuoco
	\item[\textbf{Sensi:}] Scurovisione 36 m, Vista Cieca 9 m
	\item[\textbf{Linguaggi:}] Comune, Draconico
	\item[\textbf{Sfida:}] 10 (5900 PX)\smallskip
\end{description}

\textbf{Azioni}

\emph{\textbf{Multiattacco.}} Il drago può effettuare tre attacchi: uno con il morso e due con gli artigli.

\emph{\textbf{Artiglio.} Attacco con arma da mischia}: +11 a colpire, portata 1 m, un bersaglio.

\emph{Colpisce:} 13 (2d6 + 6) danni taglienti, 1 danno da Sanguinamento.

\emph{\textbf{Morso.} Attacco con arma da mischia}: +11 a colpire, portata 3 m, un bersaglio.

\emph{Colpisce:} 17 (2d10 + 6) danni perforanti più 3 (1d6) danni da fuoco.

\emph{\textbf{Soffio Infuocato (Ricarica 5-6).}} Il drago esala fuoco in un cono di 9 metri. Ogni creatura in quell'area deve effettuare un Tiro Salvezza su Riflessi DC 22 e subire 56 (16d6) danni da fuoco se fallisce il Tiro Salvezza, o la metà di questi danni se lo riesce.

\emph{\textbf{Arrabbiato:}} il giovane drago rosso ricarica il soffio infuocato.

Costa 1 Azione.

\textbf{Ecologia}\\
Ambiente: Montagne calde\\
Organizzazione: Solitario\\
\textbf{Categoria Tesoro}: D\\
Vedi Descrizione Drago Rosso Antico.\\
\textbf{Incantesimi}\index{Incantesimi da Drago Rosso}\\
Gli incantesimi preferiti di questo Drago sono:\\
- \hyperlink{Palla di Fuoco}{Palla di Fuoco}\\
- \hyperlink{Nube Incendiaria}{Nube Incendiaria}\\
- \hyperlink{Muro di Fuoco}{Muro di Fuoco}

\mostro{Drago Rosso Cucciolo}
\noindent
\begin{description}[noitemsep, topsep=0pt, parsep=0pt, partopsep=0pt, leftmargin=0cm, labelwidth=2.2cm]
	\item[\textbf{Taglia/Tipo:}] Media drago, malvagio
	\item[\textbf{Caratt.:}] \resizebox{0.5\linewidth+1.8cm}{!}{For 4 Des 0 Cos 3 Int 1 Sag 0 Car 2}
	\item[\textbf{Punti Ferita:}] 89,  \textbf{Difesa:} 17,  \textbf{Iniziativa:} +1
	\item[\textbf{Movimento:}] 9 m, scalata 9 m, volo 18 m
	\item[\textbf{Tiri Salvez.:}] \resizebox{0.5\linewidth+1.8cm}{!}{Tempra +7, Riflessi +4, Volontà +4}
	\item[\textbf{Comp.:}] Furtività +2, Consapevolezza +4
	\item[\textbf{Imm. Danni:}] Fuoco
	\item[\textbf{Sensi:}] Scurovisione 18 m, Vista Cieca 3 m
	\item[\textbf{Linguaggi:}] Draconico
	\item[\textbf{Sfida:}] 4 (1100 PX)\smallskip
\end{description}

\textbf{Azioni}

\emph{\textbf{Morso.} Attacco con arma da mischia}: +6 a colpire, portata 1 m, un bersaglio.

\emph{Colpisce:} 9 (1d10 + 4) danni perforanti più 3 (1d6) danni da fuoco.

\emph{\textbf{Soffio Infuocato (Ricarica 5-6).}} Il drago esala fuoco in un cono di 5 metri. Ogni creatura in quell'area deve effettuare un Tiro Salvezza di Riflessi DC 16 e subire 24 (7d6) danni da fuoco se fallisce il Tiro Salvezza, o la metà di questi danni se lo riesce.

\textbf{Ecologia}\\
Ambiente: Montagne calde\\
Organizzazione: Solitario\\
\textbf{Categoria Tesoro}: C\\
Vedi Descrizione Drago Rosso Antico.

\mostro{Drago Verde Antico}
\noindent
\begin{description}[noitemsep, topsep=0pt, parsep=0pt, partopsep=0pt, leftmargin=0cm, labelwidth=2.2cm]
	\item[\textbf{Taglia/Tipo:}] Mastodontica drago, malvagio
	\item[\textbf{Caratt.:}] \resizebox{0.5\linewidth+1.8cm}{!}{For 8 Des 1 Cos 7 Int 5 Sag 3 Car 4}
	\item[\textbf{Punti Ferita:}] 441,  \textbf{Difesa:} 42,  \textbf{Iniziativa:} +5
	\item[\textbf{Movimento:}] 12 m, nuoto 12 m, volo 24 m
	\item[\textbf{Tiri Salvez.:}] \resizebox{0.5\linewidth+1.8cm}{!}{\resizebox{0.5\linewidth+1.8cm}{!}{Tempra +29, Riflessi +23, Volontà +25}}
	\item[\textbf{Comp.:}] Furtività +8, Ingannare +11, Percepire Emozioni +10, Consapevolezza + 15
	\item[\textbf{Imm. Danni:}] Veleno, armi +1
	\item[\textbf{Sensi:}] Scurovisione 36 m, Vista Cieca 18 m
	\item[\textbf{Linguaggi:}] Comune, Draconico
	\item[\textbf{Sfida:}] 22 (41000 PX)\smallskip
\end{description}

\emph{\textbf{Aria mefitica.}} il drago emette nel raggio di 3 metri gas magici che causano 2d6 danni da veleno a round.

\emph{\textbf{Anfibio.}} Il drago può respirare aria e acqua.

\emph{\textbf{Resistenza Leggendaria (3/Giorno).}} Se il drago fallisce un Tiro Salvezza, può scegliere invece di riuscire.

\emph{\textbf{Resistenza alla Magia:}} 3lv

\textbf{Azioni}

\emph{\textbf{Multiattacco.}} Il drago può usare la sua Presenza Spaventosa e poi effettuare tre attacchi: uno con il morso e due con gli artigli.

\emph{\textbf{Artiglio.} Attacco con arma da mischia}: +17 a colpire, portata 3 m, un bersaglio.

\emph{Colpisce:} 15 (2d6 + 8) danni taglienti, 3/20 danno da Sanguinamento.

\emph{\textbf{Coda.} Attacco con arma da mischia}: +17 a colpire, portata 6 m, un bersaglio.

\emph{Colpisce:} 17 (2d8 + 8) danni contundenti.

\emph{\textbf{Morso.} Attacco con arma da mischia}: +17 a colpire, portata 5 metri, un bersaglio.

\emph{Colpisce:} 19 (2d10 + 8) danni perforanti più 10 (3d6) danni da veleno.

\emph{\textbf{Presenza Spaventosa.}} Ogni creatura scelta dal drago, che si trovi entro 36 metri da esso e consapevole della sua presenza, deve riuscire un Tiro Salvezza di Volontà DC 25 o restare spaventata per 1 minuto. Una creatura può ripetere il Tiro Salvezza al termine di ciascun suo round, terminando l'effetto se lo riesce. Se il Tiro Salvezza della creatura ha successo o l'effetto ha termine per essa, la creatura è immune alla Presenza Spaventosa del drago per le successive 24 ore.

\emph{\textbf{Soffio Velenoso (Ricarica 5-6).}} Il drago esala gas velenosi in un cono di 27 metri. Ogni creatura in quell'area deve effettuare un Tiro Salvezza di Tempra DC 35 e subire 77 (22d6) danni da veleno se fallisce il Tiro Salvezza, o la metà di questi danni se lo riesce.

\textbf{Azioni Aggiuntive}

Il drago può effettuare 3 Azioni aggiuntive, scelte tra le opzioni seguenti. Può usare solo un'opzione Aggiuntiva alla volta e solo al termine del round di un'altra creatura. Il drago recupera le Azioni aggiuntive spese all'inizio del proprio round.

\textbf{Attacco di Ala (Costa 2 Azioni).} Il drago batte le ali. Ogni creatura entro 5 metri dal drago deve riuscire un Tiro Salvezza su Riflessi DC 35 o subire 15 (2d6 + 8) danni contundenti e venire gettato prono. Il drago può poi volare fino a metà del suo movimento di volo.

\textbf{Attacco di Coda.} Il drago effettua un attacco di coda.

\textbf{Individuare.} Il drago effettua una prova di Consapevolezza.

\textbf{Ecologia}\\
Ambiente: Foreste Temperate\\
Organizzazione: Solitario\\
\textbf{Categoria Tesoro}: H\\
\textbf{Descrizione}\\
I Draghi verdi amano le foreste e la natura incontaminata dove si reputano i padroni e re indiscussi.

I potenti draghi verdi hanno la testa tondeggiante e pronunciate orecchie all'indietro, le corna sono corte e non appuntite.
Gli artigli e le fauci sono devastanti, potenti e capace di tranciare qualsiasi cosa.
Il naso è largo e le narici aperte come se dovesse soffiare in qualsiasi momento.

Il soffio dei draghi verde è veleno, così che possa uccidere le creature viventi ma non le piante.

La tana di un drago verde è sempre vicino ad una sorgente d'acqua, possibilmente nella parte più lussureggiante ed incontaminata della foresta.

Un Drago verde non ama volare e preferisce saltare schiacciando con il suo peso e dilaniare con i suoi artigli.

Tra i tanti draghi quello verde è forse quello che farà parlare gli avventurieri se si dimostrano rispettosi ed impauriti dalla sua regalità.

I Draghi Verdi hanno +1d6 nelle prove di magia e possono ignorare un dado tirato nella prova con la Lista di Animali e Piante ed è immune i Veleni sia a quelli magici che naturali.\\
\textbf{Incantesimi}\index{Incantesimi da Drago Verde}\\
Gli incantesimi preferiti di questo Drago sono:\\
- \hyperlink{Guscio Anti-Vita}{Guscio Anti-Vita}\\
- \hyperlink{Localizza Animali e Piante}{Localizza Animali e Piante}\\
- \hyperlink{Rimuovi Veleno}{Rimuovi Veleno}

\mostro{Drago Verde Adulto}
\noindent
\begin{description}[noitemsep, topsep=0pt, parsep=0pt, partopsep=0pt, leftmargin=0cm, labelwidth=2.2cm]
	\item[\textbf{Taglia/Tipo:}] Enorme drago, malvagio
	\item[\textbf{Caratt.:}] \resizebox{0.5\linewidth+1.8cm}{!}{For 6 Des 1 Cos 5 Int 4 Sag 2 Car 3}
	\item[\textbf{Punti Ferita:}] 300,  \textbf{Difesa:} 33,  \textbf{Iniziativa:} +4
	\item[\textbf{Movimento:}] 12 m, nuoto 12 m, volo 24 m
	\item[\textbf{Tiri Salvez.:}] \resizebox{0.5\linewidth+1.8cm}{!}{\resizebox{0.5\linewidth+1.8cm}{!}{Tempra +20, Riflessi +16, Volontà +17}}
	\item[\textbf{Comp.:}] Furtività +6, Ingannare +8, Percepire Emozioni +7, Consapevolezza +12
	\item[\textbf{Imm. Danni:}] Veleno
	\item[\textbf{Sensi:}] Scurovisione 36 m, Vista Cieca 18 m
	\item[\textbf{Linguaggi:}] Comune, Draconico
	\item[\textbf{Sfida:}] 15 (13000 PX)\smallskip
\end{description}

\emph{\textbf{Aria mefitica.}} il drago emette nel raggio di 3 metri gas magici che causano 1d6 danni da veleno a round.

\emph{\textbf{Anfibio.}} Il drago può respirare aria e acqua.

\emph{\textbf{Resistenza Leggendaria (3/Giorno).}} Se il drago fallisce un Tiro Salvezza, può scegliere invece di riuscire.

\textbf{Azioni}

\emph{\textbf{Multiattacco.}} Il drago può usare la sua Presenza Spaventosa e poi effettuare tre attacchi: uno con il morso e due con gli artigli.

\emph{\textbf{Artiglio.} Attacco con arma da mischia}: +13 a colpire, portata 1 m, un bersaglio.

\emph{Colpisce:} 13 (2d6 + 6) danni taglienti, 1 danno da Sanguinamento.

\emph{\textbf{Coda.} Attacco con arma da mischia}: +13 a colpire, portata 5 metri, un bersaglio.

\emph{Colpisce:} 15 (2d8 + 6) danni contundenti.

\emph{\textbf{Morso.} Attacco con arma da mischia}: +13 a colpire, portata 3 m, un bersaglio.

\emph{Colpisce:} 17 (2d10 + 6) danni perforanti più 7 (2d6) danni da veleno.

\emph{\textbf{Presenza Spaventosa.}} Ogni creatura scelta dal drago, che si trovi entro 36 metri da esso e consapevole della sua presenza, deve riuscire un Tiro Salvezza di Volontà DC 28 o restare spaventata per 1 minuto. Una creatura può ripetere il Tiro Salvezza al termine di ciascun suo round, terminando l'effetto se lo riesce. Se il Tiro Salvezza della creatura ha successo o l'effetto ha termine per essa, la creatura è immune alla Presenza Spaventosa del drago per le successive 24 ore.

\emph{\textbf{Soffio Velenoso (Ricarica 5-6).}} Il drago esala gas velenosi in un cono di 18 metri. Ogni creatura in quell'area deve effettuare un Tiro Salvezza di Tempra DC 28 e subire 56 (16d6) danni da veleno se fallisce il Tiro Salvezza, o la metà di questi danni se lo riesce.

\textbf{Azioni Aggiuntive}

Il drago può effettuare 3 Azioni aggiuntive, scelte tra le opzioni seguenti. Può usare solo un'opzione Aggiuntiva alla volta e solo al termine del round di un'altra creatura. Il drago recupera le Azioni aggiuntive spese all'inizio del proprio round.

\textbf{Attacco di Ala (Costa 2 Azioni).} Il drago batte le ali. Ogni creatura entro 3 metri dal drago deve riuscire un Tiro Salvezza su Riflessi DC 28 o subire 13 (2d6 + 6) danni contundenti e venir gettato prono. Il drago può poi volare fino a metà del suo movimento di volo.

\textbf{Attacco di Coda.} Il drago effettua un attacco di coda.

\textbf{Individuare.} Il drago effettua una prova di Consapevolezza.

\emph{\textbf{Arrabbiato:}} Il Drago Verde Adulto può eseguire queste azioni a costo 2 Azioni.

\emph{Focalizzare}: la creatura interrompe un effetto mentale su di se in corso

\emph{Brutalità}: la creatura attacca con ferocia inaudita. +1d6 al Tiro per Colpire, 1 danno critico automatico quando colpisce.

\textbf{Ecologia}\\
Ambiente: Foreste Temperate\\
Organizzazione: Solitario\\
\textbf{Categoria Tesoro}: E\\
\textbf{Descrizione}\\
Vedi Descrizione Drago Verde Antico.\\
\textbf{Incantesimi}\index{Incantesimi da Drago Verde}\\
Gli incantesimi preferiti di questo Drago sono:\\
- \hyperlink{Guscio Anti-Vita}{Guscio Anti-Vita}\\
- \hyperlink{Localizza Animali e Piante}{Localizza Animali e Piante}\\
- \hyperlink{Rimuovi Veleno}{Rimuovi Veleno}

\mostro{Drago Verde Giovane}
\noindent
\begin{description}[noitemsep, topsep=0pt, parsep=0pt, partopsep=0pt, leftmargin=0cm, labelwidth=2.2cm]
	\item[\textbf{Taglia/Tipo:}] Grande drago, malvagio
	\item[\textbf{Caratt.:}] \resizebox{0.5\linewidth+1.8cm}{!}{For 4 Des 1 Cos 3 Int 3 Sag 1 Car 2}
	\item[\textbf{Punti Ferita:}] 163,  \textbf{Difesa:} 23,  \textbf{Iniziativa:} +3
	\item[\textbf{Movimento:}] 12 m, nuoto 12 m, volo 24 m
	\item[\textbf{Tiri Salvez.:}] \resizebox{0.5\linewidth+1.8cm}{!}{\resizebox{0.5\linewidth+1.8cm}{!}{Tempra +11, Riflessi +9, Volontà +9}}
	\item[\textbf{Comp.:}] Furtività +4, Ingannare +5, Consapevolezza +7
	\item[\textbf{Imm. Danni:}] Veleno
	\item[\textbf{Sensi:}] Scurovisione 36 m, Vista Cieca 9 m
	\item[\textbf{Linguaggi:}] Comune, Draconico
	\item[\textbf{Sfida:}] 8 (3900 PX)\smallskip
\end{description}

\emph{\textbf{Anfibio.}} Il drago può respirare aria e acqua.

\textbf{Azioni}

\emph{\textbf{Multiattacco.}} Il drago può effettuare tre attacchi: uno con il morso e due con gli artigli.

\emph{\textbf{Artiglio.} Attacco con arma da mischia}: +9 a colpire, portata 1 m, un bersaglio.

\emph{Colpisce:} 11 (2d6 + 4) danni taglienti, 1 danno da Sanguinamento.

\emph{\textbf{Morso.} Attacco con arma da mischia}: +9 a colpire, portata 3 m, un bersaglio.

\emph{Colpisce:} 15 (2d10 + 4) danni perforanti più 7 (2d6) danni da veleno.

\emph{\textbf{Soffio Velenoso (Ricarica 5-6).}} Il drago esala gas velenosi in un cono di 9 metri. Ogni creatura in quell'area deve effettuare un Tiro Salvezza di Tempra DC 20 e subire 42 (12d6) danni da veleno se fallisce il Tiro Salvezza, o la metà di questi danni se lo riesce.

\emph{\textbf{Arrabbiato:}} il Drago Verde Giovane ricarica il suo soffio Velenoso.

\textbf{Ecologia}\\
Ambiente: Foreste Temperate\\
Organizzazione: Solitario\\
\textbf{Categoria Tesoro}: D\\
\textbf{Descrizione}\\
Vedi Descrizione Drago Verde Antico.\\
\textbf{Incantesimi}\index{Incantesimi da Drago Verde}\\
Gli incantesimi preferiti di questo Drago sono:\\
- \hyperlink{Guscio Anti-Vita}{Guscio Anti-Vita}\\
- \hyperlink{Localizza Animali e Piante}{Localizza Animali e Piante}\\
- \hyperlink{Rimuovi Veleno}{Rimuovi Veleno}

\mostro{Drago Verde Cucciolo}
\noindent
\begin{description}[noitemsep, topsep=0pt, parsep=0pt, partopsep=0pt, leftmargin=0cm, labelwidth=2.2cm]
	\item[\textbf{Taglia/Tipo:}] Media drago, malvagio
	\item[\textbf{Caratt.:}] \resizebox{0.5\linewidth+1.8cm}{!}{For 2 Des 1 Cos 1 Int 2 Sag 0 Car 1}
	\item[\textbf{Punti Ferita:}] 51,  \textbf{Difesa:} 15,  \textbf{Iniziativa:} +2
	\item[\textbf{Movimento:}] 9 m, nuoto 9 m, volo 18 m
	\item[\textbf{Tiri Salvez.:}] \resizebox{0.5\linewidth+1.8cm}{!}{Tempra +3, Riflessi +3, Volontà +3}
	\item[\textbf{Comp.:}] Furtività +3, Consapevolezza +4
	\item[\textbf{Imm. Danni:}] Veleno
	\item[\textbf{Sensi:}] Scurovisione 18 m, Vista Cieca 3 m
	\item[\textbf{Linguaggi:}] Draconico
	\item[\textbf{Sfida:}] 2 (450 PX)\smallskip
\end{description}

\emph{\textbf{Anfibio.}} Il drago può respirare aria e acqua.

\textbf{Azioni}

\emph{\textbf{Morso.} Attacco con arma da mischia}: +5 a colpire, portata 1 m, un bersaglio.

\emph{Colpisce:} 7 (1d10 + 2) danni perforanti più 3 (1d6) danni da veleno.

\emph{\textbf{Soffio Velenoso (Ricarica 5-6).}} Il drago esala gas velenosi in un cono di 5 metri. Ogni creatura in quell'area deve effettuare un Tiro Salvezza di Tempra DC 13 e subire 21 (6d6) danni da veleno se fallisce il Tiro Salvezza, o la metà di questi danni se lo riesce.

\textbf{Ecologia}\\
Ambiente: Foreste Temperate\\
Organizzazione: Solitario\\
\textbf{Categoria Tesoro}: C\\
\textbf{Descrizione}\\
Vedi Descrizione Drago Verde Antico.

\medskip

\rule{\linewidth}{2pt}

\medskip

\textbf{Draghi di Ljust}

\pdfbookmark[3]{Draghi di Ljust}{Draghi di Ljust}

Pochissimi draghi buoni o di Ljust come vengono chiamati, sono presenti nella Terra.
Elysan è probabilmente il più noto e potente, un antico drago d'argento.

\mostro{Drago di Argento Antico}
\noindent
\begin{description}[noitemsep, topsep=0pt, parsep=0pt, partopsep=0pt, leftmargin=0cm, labelwidth=2.2cm]
	\item[\textbf{Taglia/Tipo:}] Mastodontica drago, buono
	\item[\textbf{Caratt.:}] \resizebox{0.5\linewidth+1.8cm}{!}{For 10 Des 0 Cos 9 Int 4 Sag 2 Car 6}
	\item[\textbf{Punti Ferita:}] 470,  \textbf{Difesa:} 42,  \textbf{Iniziativa:} +4
	\item[\textbf{Movimento:}] 12 m, volo 24 m
	\item[\textbf{Tiri Salvez.:}] \resizebox{0.5\linewidth+1.8cm}{!}{\resizebox{0.5\linewidth+1.8cm}{!}{Tempra +32, Riflessi +23, Volontà +25}}
	\item[\textbf{Comp.:}] Arcana +11, Furtività +7, Consapevolezza +16, Storia +11
	\item[\textbf{Imm. Danni:}] Freddo, armi +1
	\item[\textbf{Sensi:}] Scurovisione 36 m, Vista Cieca 18 m
	\item[\textbf{Linguaggi:}] Comune, Draconico
	\item[\textbf{Sfida:}] 23 (50000 PX)\smallskip
\end{description}

\emph{\textbf{Aura rallentante.}} il drago emette nel raggio di 3 metri una aura magica che causa Rallentato 1.

\emph{\textbf{Resistenza Leggendaria (3/Giorno).}} Se il drago fallisce un Tiro Salvezza, può scegliere invece di riuscire.

\emph{\textbf{Resistenza alla Magia:}} 3lv

\textbf{Azioni}

\emph{\textbf{Multiattacco.}} Il drago può usare la sua Presenza Spaventosa e poi effettuare tre attacchi: uno con il morso e due con gli artigli.

\emph{\textbf{Artiglio.} Attacco con arma da mischia}: +17 a colpire, portata 3 m, un bersaglio.

\emph{Colpisce:} 17 (2d6 + 10) danni taglienti, 3/20 danno da Sanguinamento.

\emph{\textbf{Coda.} Attacco con arma da mischia}: +17 a colpire, portata 6 m, un bersaglio.

\emph{Colpisce:} 19 (2d8 + 10) danni contundenti.

\emph{\textbf{Morso.} Attacco con arma da mischia}: +17 a colpire, portata 5 metri, un bersaglio.

\emph{Colpisce:} 21 (2d10 + 10) danni perforanti.

\emph{\textbf{Presenza Spaventosa.}} Ogni creatura scelta dal drago, che si trovi entro 36 metri da esso e consapevole della sua presenza, deve riuscire un Tiro Salvezza di Volontà DC 36 o restare spaventata per 1 minuto. Una creatura può ripetere il Tiro Salvezza al termine di ciascun suo round, terminando l'effetto se lo riesce. Se il Tiro Salvezza della creatura ha successo o l'effetto ha termine per essa, la creatura è immune alla Presenza Spaventosa del drago per le successive 24 ore.

\emph{\textbf{Arma a Soffio (Ricarica 5-6).}} Il drago usa una delle seguenti armi a soffio:

\emph{Soffio Gelido.} Il drago esala un'esplosione ghiacciata in un cono di 27 metri. Ogni creatura nell'area deve effettuare un Tiro Salvezza su Tempra DC 36, subendo 67 (15d8) danni da freddo se fallisce il Tiro Salvezza, o la metà di questi danni se lo riesce.

\emph{Soffio Paralizzante.} Il drago esala un gas paralizzante in un cono di 23 metri. Ogni creatura nell'area deve riuscire un Tiro Salvezza su Tempra 36 o restare paralizzata per 1 minuto. Una creatura può ripetere il Tiro Salvezza al termine di ciascun suo round, terminando l'effetto per sé in caso di successo.

\emph{\textbf{Mutare Forma.}} Il drago può trasformarsi magicamente in un umanoide o bestia il cui grado di sfida sia pari o inferiore al proprio, o tornare alla sua vera forma. Alla morte ritorna alla sua vera forma.

Qualsiasi equipaggiamento stia indossando o trasportando viene assorbito o trasportato nella nuova forma (a scelta del drago).

Nella nuova forma, il drago mantiene i suoi Tratti, Punti Ferita, la facoltà di parlare, le competenze, la Resistenza Leggendaria, le azioni da tana, e i punteggi di Intelligenza, Saggezza e Carisma, oltre a questa azione. Le sue statistiche e capacità vengono altrimenti rimpiazzate da quelle della nuova forma, eccetto Azioni aggiuntive della nuova forma.

\textbf{Azioni Aggiuntive}

Il drago può effettuare 3 Azioni aggiuntive, scelte tra le opzioni seguenti. Può usare solo un'opzione Aggiuntiva alla volta e solo al termine del round di un'altra creatura. Il drago recupera le Azioni aggiuntive spese all'inizio del proprio round.

\textbf{Attacco di Ala (Costa 2 Azioni).} Il drago batte le ali. Ogni creatura entro 5 metri dal drago deve riuscire un Tiro Salvezza su Riflessi DC 36 o subire 17 (2d6 + 10) danni contundenti e venir gettato prono. Il drago può poi volare fino a metà della sua velocità di volo.

\textbf{Attacco di Coda.} Il drago effettua un attacco di coda.

\textbf{Individuare.} Il drago effettua una prova di Consapevolezza.

\textbf{Ecologia}\\
Ambiente: Montagne Temperate\\
Organizzazione: Solitario\\
\textbf{Categoria Tesoro}: H\\
\textbf{Descrizione}\\
Tra tutti i draghi, quelli d'argento sono i più coraggiosi, e si attengono ad un codice cavalleresco che impone loro di aiutare i deboli, sconfiggere il male e comportarsi in modo onorevole.\\
\textbf{Incantesimi}\index{Incantesimi da Drago Argento}\\
Gli incantesimi preferiti di questo Drago sono:\\
- \hyperlink{lentezza}{Lentezza}\\
- \hyperlink{Fabbricare}{Fabbricare}\\
- \hyperlink{Sogno}{Sogno}

%\begin{center}
%\includegraphics[width=0.9\linewidth]{immagini/silver.png}
%
%\textit{Argento, grezzo}
%\end{center}

\mostro{Drago di Argento Adulto}
\noindent
\begin{description}[noitemsep, topsep=0pt, parsep=0pt, partopsep=0pt, leftmargin=0cm, labelwidth=2.2cm]
	\item[\textbf{Taglia/Tipo:}] Enorme drago, buono
	\item[\textbf{Caratt.:}] \resizebox{0.5\linewidth+1.8cm}{!}{For 8 Des 0 Cos 7 Int 3 Sag 1 Car 5}
	\item[\textbf{Punti Ferita:}] 325,  \textbf{Difesa:} 33,  \textbf{Iniziativa:} +3
	\item[\textbf{Movimento:}] 12 m, volo 24 m
	\item[\textbf{Tiri Salvez.:}] \resizebox{0.5\linewidth+1.8cm}{!}{\resizebox{0.5\linewidth+1.8cm}{!}{Tempra +23, Riflessi +16, Volontà +17}}
	\item[\textbf{Comp.:}] Arcana +8, Furtività +5, Consapevolezza +11, Storia +8
	\item[\textbf{Imm. Danni:}] Freddo
	\item[\textbf{Sensi:}] Scurovisione 36 m, Vista Cieca 18 m
	\item[\textbf{Linguaggi:}] Comune, Draconico
	\item[\textbf{Sfida:}] 16 (15000 PX)\smallskip
\end{description}

\emph{\textbf{Resistenza Leggendaria (3/Giorno).}} Se il drago fallisce un Tiro Salvezza, può scegliere invece di riuscire.

\textbf{Azioni}

\emph{\textbf{Multiattacco.}} Il drago può usare la sua Presenza Spaventosa e poi effettuare tre attacchi: uno con il morso e due con gli artigli.

\emph{\textbf{Artiglio.} Attacco con arma da mischia}: +14 a colpire, portata 1 m, un bersaglio.

\emph{Colpisce:} 15 (2d6 + 8) danni taglienti, 1 danno da Sanguinamento.

\emph{\textbf{Coda.} Attacco con arma da mischia}: +14 a colpire, portata 5 metri, un bersaglio.

\emph{Colpisce:} 17 (2d8 + 8) danni contundenti.

\emph{\textbf{Morso.} Attacco con arma da mischia}: +14 a colpire, portata 3 m, un bersaglio.

\emph{Colpisce:} 19 (2d10 + 8) danni perforanti.

\emph{\textbf{Presenza Spaventosa.}} Ogni creatura scelta dal drago, che si trovi entro 36 metri da esso e consapevole della sua presenza, deve riuscire un Tiro Salvezza di Volontà DC 28 o restare spaventata per 1 minuto. Una creatura può ripetere il Tiro Salvezza al termine di ciascun suo round, terminando l'effetto se lo riesce. Se il Tiro Salvezza della creatura ha successo o l'effetto ha termine per essa, la creatura è immune alla Presenza Spaventosa del drago per le successive 24 ore.

\emph{\textbf{Arma a Soffio (Ricarica 5-6).}} Il drago usa una delle seguenti armi a soffio:

\emph{Soffio Gelido.} Il drago esala un'esplosione ghiacciata in un cono di 18 metri. Ogni creatura nell'area deve effettuare un Tiro Salvezza su Tempra DC 28, subendo 58 (13d8) danni da freddo se fallisce il Tiro Salvezza, o la metà di questi danni se lo riesce.

\emph{Soffio Paralizzante.} Il drago esala un gas paralizzante in un cono di 18 metri. Ogni creatura nell'area deve riuscire un Tiro Salvezza su Tempra 28 o restare paralizzata per 1 minuto. Una creatura può ripetere il Tiro Salvezza al termine di ciascun suo round, terminando l'effetto per sé in caso di successo.

\emph{\textbf{Mutare Forma.}} Il drago può trasformarsi magicamente in un umanoide o bestia il cui grado di sfida sia pari o inferiore al proprio, o tornare alla sua vera forma. Alla morte ritorna alla sua vera forma. Qualsiasi equipaggiamento stia indossando o trasportando viene assorbito o trasportato nella nuova forma (a scelta del drago).

Nella nuova forma, il drago mantiene i suoi Tratti, Punti Ferita la facoltà di parlare, le competenze, la Resistenza Leggendaria, le azioni da tana, e i punteggi di Intelligenza, Saggezza e Carisma, oltre a questa azione. Le sue statistiche e capacità vengono altrimenti rimpiazzate da quelle della nuova forma, eccetto Azioni aggiuntive della nuova forma.

\textbf{Azioni Aggiuntive}

Il drago può effettuare 3 Azioni aggiuntive, scelte tra le opzioni seguenti. Può usare solo un'opzione Aggiuntiva alla volta e solo al termine del round di un'altra creatura. Il drago recupera le Azioni aggiuntive spese all'inizio del proprio round.

\textbf{Attacco di Ala (Costa 2 Azioni).} Il drago batte le ali. Ogni creatura entro 3 metri dal drago deve riuscire un Tiro Salvezza su Riflessi DC 28 o subire 15 (2d6 + 8) danni contundenti e venir gettato prono. Il drago può poi volare fino a metà del suo movimento di volo.

\textbf{Attacco di Coda.} Il drago effettua un attacco di coda.

\textbf{Individuare.} Il drago effettua una prova di Consapevolezza.

\emph{\textbf{Arrabbiato:}} Il Drago d'Argento Adulto può eseguire queste azioni a costo 2 Azioni.

\emph{Focalizzare}: la creatura interrompe un effetto mentale su di se in corso

\emph{Brutalità}: la creatura attacca con ferocia inaudita. +1d6 al Tiro per Colpire, 1 danno critico automatico quando colpisce.

\textbf{Ecologia}\\
Ambiente: Montagne Temperate\\
Organizzazione: Solitario\\
\textbf{Categoria Tesoro}: E\\
\textbf{Descrizione}\\
Tra tutti i draghi, quelli d'argento sono i più coraggiosi, e si attengono ad un codice cavalleresco che impone loro di aiutare i deboli, sconfiggere il male e comportarsi in modo onorevole.\\
\textbf{Incantesimi}\index{Incantesimi da Drago Argento}\\
Gli incantesimi preferiti di questo Drago sono:\\
- \hyperlink{lentezza}{Lentezza}\\
- \hyperlink{Fabbricare}{Fabbricare}\\
- \hyperlink{Sogno}{Sogno}

\medskip

\begin{center}
	\includegraphics[width=0.9\linewidth]{immagini/Dragon_Ljubljana.png}

	\emph{Dragon Bridge, Ljubljana}
\end{center}

\mostro{Drago di Argento Giovane}
\noindent
\begin{description}[noitemsep, topsep=0pt, parsep=0pt, partopsep=0pt, leftmargin=0cm, labelwidth=2.2cm]
	\item[\textbf{Taglia/Tipo:}] Grande drago, buono
	\item[\textbf{Caratt.:}] \resizebox{0.5\linewidth+1.8cm}{!}{For 6 Des 0 Cos 5 Int 2 Sag 0 Car 4}
	\item[\textbf{Punti Ferita:}] 186,  \textbf{Difesa:} 24,  \textbf{Iniziativa:} +2
	\item[\textbf{Movimento:}] 12 m, volo 24 m
	\item[\textbf{Tiri Salvez.:}] \resizebox{0.5\linewidth+1.8cm}{!}{\resizebox{0.5\linewidth+1.8cm}{!}{Tempra +14, Riflessi +9, Volontà +9}}
	\item[\textbf{Comp.:}] Arcana +6, Furtività +4, Consapevolezza +8, Storia +6
	\item[\textbf{Imm. Danni:}] Freddo
	\item[\textbf{Sensi:}] Scurovisione 36 m, Vista Cieca 9 m
	\item[\textbf{Linguaggi:}] Comune, Draconico
	\item[\textbf{Sfida:}] 9 (5000 PX)\smallskip
\end{description}

\textbf{Azioni}

\emph{\textbf{Multiattacco.}} Il drago può effettuare tre attacchi: uno con il morso e due con gli artigli.

\emph{\textbf{Artiglio.} Attacco con arma da mischia}: +10 a colpire, portata 1 m, un bersaglio.

\emph{Colpisce:} 13 (2d6 + 6) danni taglienti, 1 danno da Sanguinamento.

\emph{\textbf{Morso.} Attacco con arma da mischia}: +10 a colpire, portata 3 m, un bersaglio.

\emph{Colpisce:} 17 (2d10 + 6) danni perforanti.

\emph{\textbf{Arma a Soffio (Ricarica 5-6).}} Il drago usa una delle seguenti armi a soffio:

\emph{Soffio Gelido.} Il drago esala un'esplosione ghiacciata in un cono di 9 metri. Ogni creatura nell'area deve effettuare un Tiro Salvezza su Tempra DC 21, subendo 54 (12d8) danni da freddo se fallisce il Tiro Salvezza, o la metà di questi danni se lo riesce.

\emph{Soffio Paralizzante.} Il drago esala un gas paralizzante in un cono di 9 metri. Ogni creatura nell'area deve riuscire un Tiro Salvezza su Tempra 21 o restare paralizzata per 1 minuto. Una creatura può ripetere il Tiro Salvezza al termine di ciascun suo round, terminando l'effetto per sé in caso di successo.

\emph{\textbf{Arrabbiato:}} il giovane drago d'argento ricarica uno dei suoi soffi.

\textbf{Ecologia}\\
Ambiente: Montagne Temperate\\
Organizzazione: Solitario\\
\textbf{Categoria Tesoro}: D\\
\textbf{Descrizione}\\
Tra tutti i draghi, quelli d'argento sono i più coraggiosi, e si attengono ad un codice cavalleresco che impone loro di aiutare i deboli, sconfiggere il male e comportarsi in modo onorevole.\\
\textbf{Incantesimi}\index{Incantesimi da Drago Argento}\\
Gli incantesimi preferiti di questo Drago sono:\\
- \hyperlink{lentezza}{Lentezza}\\
- \hyperlink{Fabbricare}{Fabbricare}\\
- \hyperlink{Sogno}{Sogno}

\mostro{Drago di Argento Cucciolo}
\noindent
\begin{description}[noitemsep, topsep=0pt, parsep=0pt, partopsep=0pt, leftmargin=0cm, labelwidth=2.2cm]
	\item[\textbf{Taglia/Tipo:}] Media drago, buono
	\item[\textbf{Caratt.:}] \resizebox{0.5\linewidth+1.8cm}{!}{For 4 Des 0 Cos 3 Int 1 Sag 0 Car 2}
	\item[\textbf{Punti Ferita:}] 52,  \textbf{Difesa:} 14,  \textbf{Iniziativa:} +1
	\item[\textbf{Movimento:}] 9 m, volo 18 m
	\item[\textbf{Tiri Salvez.:}] \resizebox{0.5\linewidth+1.8cm}{!}{Tempra +5, Riflessi +3, Volontà +3}
	\item[\textbf{Comp.:}] Furtività +2, Consapevolezza +4
	\item[\textbf{Imm. Danni:}] Freddo
	\item[\textbf{Sensi:}] Scurovisione 18 m, Vista Cieca 3 m
	\item[\textbf{Linguaggi:}] Draconico
	\item[\textbf{Sfida:}] 2 (450 PX)\smallskip
\end{description}

\textbf{Azioni}

\emph{\textbf{Morso.} Attacco con arma da mischia}: +5 a colpire, portata 1 m, un bersaglio.

\emph{Colpisce:} 9 (1d10 + 4) danni perforanti.

\emph{\textbf{Arma a Soffio (Ricarica 5-6).}} Il drago usa una delle seguenti armi a soffio:

\emph{Soffio Gelido.} Il drago esala un'esplosione ghiacciata in un cono di 5 metri. Ogni creatura nell'area deve effettuare un Tiro Salvezza su Tempra 14, subendo 18 (4d8) danni da freddo se fallisce il Tiro Salvezza, o la metà di questi danni se lo riesce.

\emph{Soffio Paralizzante.} Il drago esala un gas paralizzante in un cono di 5 metri. Ogni creatura nell'area deve riuscire un Tiro Salvezza su Tempra 14 o restare paralizzata per 1 minuto. Una creatura può ripetere il Tiro Salvezza al termine di ciascun suo round, terminando l'effetto per sé in caso di successo.

\textbf{Ecologia}\\
Ambiente: Montagne Temperate\\
Organizzazione: Solitario\\
\textbf{Categoria Tesoro}: C\\
\textbf{Descrizione}\\
Tra tutti i draghi, quelli d'argento sono i più coraggiosi, e si attengono ad un codice cavalleresco che impone loro di aiutare i deboli, sconfiggere il male e comportarsi in modo onorevole.

\mostro{Drago di Bronzo Antico}
\noindent
\begin{description}[noitemsep, topsep=0pt, parsep=0pt, partopsep=0pt, leftmargin=0cm, labelwidth=2.2cm]
	\item[\textbf{Taglia/Tipo:}] Mastodontica drago, buono
	\item[\textbf{Caratt.:}] \resizebox{0.5\linewidth+1.8cm}{!}{For 9 Des 0 Cos 8 Int 4 Sag 3 Car 5}
	\item[\textbf{Punti Ferita:}] 446,  \textbf{Difesa:} 41,  \textbf{Iniziativa:} +4
	\item[\textbf{Movimento:}] 12 m, nuoto 12 m, volo 24 m
	\item[\textbf{Tiri Salvez.:}] \resizebox{0.5\linewidth+1.8cm}{!}{\resizebox{0.5\linewidth+1.8cm}{!}{Tempra +30, Riflessi +22, Volontà +25}}
	\item[\textbf{Comp.:}] Furtività +7, Percepire Emozioni +10, Consapevolezza +17
	\item[\textbf{Imm. Danni:}] Elettricità, armi +1
	\item[\textbf{Sensi:}] Scurovisione 36 m, Vista Cieca 18 m
	\item[\textbf{Linguaggi:}] Comune, Draconico
	\item[\textbf{Sfida:}] 22 (41000 PX)\smallskip
\end{description}

\emph{\textbf{Aura repulsiva.}} il drago emette nel raggio di 3 metri un aura che disturba le creature. Ogni attacco portato subisce una penalità di 3 - distanza attacco.

\emph{\textbf{Anfibio.}} Il drago può respirare aria e acqua.

\emph{\textbf{Resistenza Leggendaria (3/Giorno).}} Se il drago fallisce un Tiro Salvezza, può scegliere invece di riuscire.

\emph{\textbf{Resistenza alla Magia:}} 3lv

\textbf{Azioni}

\emph{\textbf{Multiattacco.}} Il drago può usare la sua Presenza Spaventosa e poi effettuare tre attacchi: uno con il morso e due con gli artigli.

\emph{\textbf{Artiglio.} Attacco con arma da mischia}: +17 a colpire, portata 3 m, un bersaglio.

\emph{Colpisce:} 16 (2d6 + 9) danni taglienti, 3/20 danno da Sanguinamento.

\emph{\textbf{Coda.} Attacco con arma da mischia}: +17 a colpire, portata 6 m, un bersaglio.

\emph{Colpisce:} 18 (2d8 + 9) danni contundenti.

\emph{\textbf{Morso.} Attacco con arma da mischia}: +17 a colpire, portata 5 metri, un bersaglio.

\emph{Colpisce:} 20 (2d10 + 9) danni perforanti.

\emph{\textbf{Presenza Spaventosa.}} Ogni creatura scelta dal drago, che si trovi entro 36 metri da esso e consapevole della sua presenza, deve riuscire un Tiro Salvezza di Volontà DC 35 o restare spaventata per 1 minuto. Una creatura può ripetere il Tiro Salvezza al termine di ciascun suo round, terminando l'effetto se lo riesce. Se il Tiro Salvezza della creatura ha successo o l'effetto ha termine per essa, la creatura è immune alla Presenza Spaventosa del drago per le successive 24 ore.

\emph{\textbf{Arma a Soffio (Ricarica 5-6).}} Il drago usa una delle seguenti armi a soffio:

\emph{Soffio Fulminante.} Il drago esala fulmini in una linea lunga 36 metri e larga 3 metri. Ogni creatura sulla linea deve effettuare un Tiro Salvezza su Riflessi DC 35, subendo 88 (16d10) danni da elettricità se fallisce il Tiro Salvezza, o la metà di questi danni se lo riesce.

\emph{Soffio Repulsivo.} Il drago esala dell'energia repulsiva in un cono di 9 metri. Ogni creatura in quell'area deve riuscire un Tiro Salvezza su Tempra DC 35, altrimenti viene allontana di 18 metri dal drago.

\emph{\textbf{Mutare Forma.}} Il drago può trasformarsi magicamente in un umanoide o bestia il cui grado di sfida sia pari o inferiore al proprio, o tornare alla sua vera forma. Alla morte ritorna alla sua vera forma. Qualsiasi equipaggiamento stia indossando o trasportando viene assorbito o trasportato nella nuova forma (a scelta del drago).

Nella nuova forma, il drago mantiene i suoi Tratti, Punti Ferita la facoltà di parlare, le competenze, la Resistenza Leggendaria, le azioni da tana, e i punteggi di Intelligenza, Saggezza e Carisma, oltre a questa azione. Le sue statistiche e capacità vengono altrimenti rimpiazzate da quelle della nuova forma, eccetto Azioni aggiuntive della nuova forma.

\textbf{Azioni Aggiuntive}

Il drago può effettuare 3 Azioni aggiuntive, scelte tra le opzioni seguenti. Può usare solo un'opzione Aggiuntiva alla volta e solo al termine del round di un'altra creatura. Il drago recupera le Azioni aggiuntive spese all'inizio del proprio round.

\textbf{Attacco di Ala (Costa 2 Azioni).} Il drago batte le ali. Ogni creatura entro 5 metri dal drago deve riuscire un Tiro Salvezza su Riflessi DC 35 o subire 16 (2d6 + 9) danni contundenti e venir gettato prono. Il drago può poi volare fino a metà del suo movimento di volo.

\textbf{Attacco di Coda.} Il drago effettua un attacco di coda.

\textbf{Individuare.} Il drago effettua una prova di Consapevolezza.

\textbf{Ecologia}\\
Ambiente: Zone Costiere Temperate\\
Organizzazione: Solitario\\
\textbf{Categoria Tesoro}: H\\
\textbf{Descrizione}\\
I draghi di bronzo sono noti per allearsi con viaggiatori ed avventurieri se causa e ricompensa sono giuste e adeguate\\
\textbf{Incantesimi}\index{Incantesimi da Drago Bronzo}\\
Gli incantesimi preferiti di questo Drago sono:\\
- \hyperlink{Globo di Invulnerabilità}{Globo di Invulnerabilità}\\
- \hyperlink{Libertà di Movimento}{Libertà di Movimento}

\mostro{Drago di Bronzo Adulto}
\noindent
\begin{description}[noitemsep, topsep=0pt, parsep=0pt, partopsep=0pt, leftmargin=0cm, labelwidth=2.2cm]
	\item[\textbf{Taglia/Tipo:}] Enorme drago, buono
	\item[\textbf{Caratt.:}] \resizebox{0.5\linewidth+1.8cm}{!}{For 7 Des 0 Cos 6 Int 3 Sag 2 Car 4}
	\item[\textbf{Punti Ferita:}] 303,  \textbf{Difesa:} 32,  \textbf{Iniziativa:} +3
	\item[\textbf{Movimento:}] 12 m, nuoto 12 m, volo 24 m
	\item[\textbf{Tiri Salvez.:}] \resizebox{0.5\linewidth+1.8cm}{!}{\resizebox{0.5\linewidth+1.8cm}{!}{Tempra +21, Riflessi +15, Volontà +17}}
	\item[\textbf{Comp.:}] Furtività +5, Percepire Emozioni +7, Consapevolezza +12
	\item[\textbf{Imm. Danni:}] Elettricità
	\item[\textbf{Sensi:}] Scurovisione 36 m, Vista Cieca 18 m
	\item[\textbf{Linguaggi:}] Comune, Draconico
	\item[\textbf{Sfida:}] 15 (13000 PX)\smallskip
\end{description}

\emph{\textbf{Anfibio.}} Il drago può respirare aria e acqua.

\emph{\textbf{Resistenza Leggendaria (3/Giorno).}} Se il drago fallisce un Tiro Salvezza, può scegliere invece di riuscire.

\textbf{Azioni}

\emph{\textbf{Multiattacco.}} Il drago può usare la sua Presenza Spaventosa e poi effettuare tre attacchi: uno con il morso e due con gli artigli.

\emph{\textbf{Artiglio.} Attacco con arma da mischia}: +13 a colpire, portata 1 m, un bersaglio.

\emph{Colpisce:} 14 (2d6 + 7) danni taglienti, 1 danno da Sanguinamento.

\emph{\textbf{Coda.} Attacco con arma da mischia}: +13 a colpire, portata 5 metri, un bersaglio.

\emph{Colpisce:} 16 (2d8 + 7) danni contundenti.

\emph{\textbf{Morso.} Attacco con arma da mischia}: +13 a colpire, portata 3 m, un bersaglio.

\emph{Colpisce:} 18 (2d10 + 7) danni perforanti.

\emph{\textbf{Presenza Spaventosa.}} Ogni creatura scelta dal drago, che si trovi entro 36 metri da esso e consapevole della sua presenza, deve riuscire un Tiro Salvezza di Volontà DC 29 o restare spaventata per 1 minuto. Una creatura può ripetere il Tiro Salvezza al termine di ciascun suo round, terminando l'effetto se lo riesce. Se il Tiro Salvezza della creatura ha successo o l'effetto ha termine per essa, la creatura è immune alla Presenza Spaventosa del drago per le successive 24 ore.

\emph{\textbf{Arma a Soffio (Ricarica 5-6).}} Il drago usa una delle seguenti armi a soffio:

\emph{Soffio Fulminante.} Il drago esala fulmini in una linea lunga 27 metri e larga 1 metro. Ogni creatura sulla linea deve effettuare un Tiro Salvezza di Riflessi DC 29, subendo 66 (12d10) danni da elettricità se fallisce il Tiro Salvezza, o la metà di questi danni se lo riesce.

\emph{Soffio Repulsivo.} Il drago esala dell'energia repulsiva in un cono di 9 metri. Ogni creatura in quell'area deve riuscire un Tiro Salvezza su Tempra DC 29, altrimenti viene allontana di 18 metri dal drago.

\emph{\textbf{Mutare Forma.}} Il drago può trasformarsi magicamente in un umanoide o bestia il cui grado di sfida sia pari o inferiore al proprio, o tornare alla sua vera forma. Alla morte ritorna alla sua vera forma. Qualsiasi equipaggiamento stia indossando o trasportando viene assorbito o trasportato nella nuova forma (a scelta del drago).

Nella nuova forma, il drago mantiene i suoi Tratti, Punti Ferita la facoltà di parlare, le competenze, la Resistenza Leggendaria, le azioni da tana, e i punteggi di Intelligenza, Saggezza e Carisma, oltre a questa azione. Le sue statistiche e capacità vengono altrimenti rimpiazzate da quelle della nuova forma, eccetto Azioni aggiuntive della nuova forma.

\textbf{Azioni Aggiuntive}

Il drago può effettuare 3 Azioni aggiuntive, scelte tra le opzioni seguenti. Può usare solo un'opzione Aggiuntiva alla volta e solo al termine del round di un'altra creatura. Il drago recupera le Azioni aggiuntive spese all'inizio del proprio round.

\textbf{Attacco di Ala (Costa 2 Azioni).} Il drago batte le ali. Ogni creatura entro 3 metri dal drago deve riuscire un Tiro Salvezza su Riflessi DC 29 o subire 14 (2d6 + 7) danni contundenti e venir gettato prono. Il drago può poi volare fino a metà del suo movimento di volo.

\textbf{Attacco di Coda.} Il drago effettua un attacco di coda.

\textbf{Individuare.} Il drago effettua una prova di Consapevolezza.

\emph{\textbf{Arrabbiato:}} Il Drago di Bronzo Adulto può eseguire queste azioni a costo 2 Azioni.

\emph{Focalizzare}: la creatura interrompe un effetto mentale su di se in corso

\emph{Brutalità}: la creatura attacca con ferocia inaudita. +1d6 al Tiro per Colpire, 1 danno critico automatico quando colpisce.

\textbf{Ecologia}\\
Ambiente: Zone Costiere Temperate\\
Organizzazione: Solitario\\
\textbf{Categoria Tesoro}: E\\
\textbf{Descrizione}\\
I draghi di bronzo sono noti per allearsi con viaggiatori ed avventurieri se causa e ricompensa sono giuste e adeguate\\
\textbf{Incantesimi}\index{Incantesimi da Drago Bronzo}\\
Gli incantesimi preferiti di questo Drago sono:\\
- \hyperlink{Globo di Invulnerabilità}{Globo di Invulnerabilità}\\
- \hyperlink{Libertà di Movimento}{Libertà di Movimento}

%\begin{center}
%	\includegraphics[width=0.7\linewidth]{immagini/Ancient_bronze_greek_helmet_-South_Italy.png}
%\end{center}

\mostro{Drago di Bronzo Giovane}
\noindent
\begin{description}[noitemsep, topsep=0pt, parsep=0pt, partopsep=0pt, leftmargin=0cm, labelwidth=2.2cm]
	\item[\textbf{Taglia/Tipo:}] Grande drago, buono
	\item[\textbf{Caratt.:}] \resizebox{0.5\linewidth+1.8cm}{!}{For 5 Des 0 Cos 4 Int 2 Sag 1 Car 3}
	\item[\textbf{Punti Ferita:}] 165,  \textbf{Difesa:} 22,  \textbf{Iniziativa:} +2
	\item[\textbf{Movimento:}] 12 m, nuoto 12 m, volo 24 m
	\item[\textbf{Tiri Salvez.:}] \resizebox{0.5\linewidth+1.8cm}{!}{\resizebox{0.5\linewidth+1.8cm}{!}{Tempra +12, Riflessi +8, Volontà +9}}
	\item[\textbf{Comp.:}] Furtività +3, Percepire Emozioni +4, Consapevolezza +7
	\item[\textbf{Imm. Danni:}] Elettricità
	\item[\textbf{Sensi:}] Scurovisione 36 m, Vista Cieca 9 m
	\item[\textbf{Linguaggi:}] Comune, Draconico
	\item[\textbf{Sfida:}] 8 (3900 PX)\smallskip
\end{description}

\emph{\textbf{Anfibio.}} Il drago può respirare aria e acqua.

\textbf{Azioni}

\emph{\textbf{Multiattacco.}} Il drago può usare effettuare tre attacchi: uno con il morso e due con gli artigli.

\emph{\textbf{Artiglio.} Attacco con arma da mischia}: +10 a colpire, portata 1 m, un bersaglio.

\emph{Colpisce:} 12 (2d6 + 5) danni taglienti, 1 danno da Sanguinamento.

\emph{\textbf{Morso.} Attacco con arma da mischia}: +10 a colpire, portata 3 m, un bersaglio.

\emph{Colpisce:} 16 (2d10 + 5) danni perforanti.

\emph{\textbf{Arma a Soffio (Ricarica 5-6).}} Il drago usa una delle seguenti armi a soffio:

\emph{Soffio Fulminante.} Il drago esala fulmini in una linea lunga 18 metri e larga 1 metro. Ogni creatura sulla linea deve effettuare un Tiro Salvezza di Riflessi DC 20, subendo 55 (10d10) danni da elettricità se fallisce il Tiro Salvezza, o la metà di questi danni se lo riesce.

\emph{Soffio Repulsivo.} Il drago esala dell'energia repulsiva in un cono di 9 metri. Ogni creatura in quell'area deve riuscire un Tiro Salvezza su Tempra DC 20, altrimenti viene allontana di 12 metri dal drago.

\emph{\textbf{Arrabbiato:}} il Drago di Bronzo Giovane ricarica uno dei suoi soffi.

\textbf{Ecologia}\\
Ambiente: Zone Costiere Temperate\\
Organizzazione: Solitario\\
\textbf{Categoria Tesoro}: D\\
\textbf{Descrizione}\\
I draghi di bronzo sono noti per allearsi con viaggiatori ed avventurieri se causa e ricompensa sono giuste e adeguate\\
\textbf{Incantesimi}\index{Incantesimi da Drago Bronzo}\\
Gli incantesimi preferiti di questo Drago sono:\\
- \hyperlink{Globo di Invulnerabilità}{Globo di Invulnerabilità}\\
- \hyperlink{Libertà di Movimento}{Libertà di Movimento}

\mostro{Drago di Bronzo Cucciolo}
\noindent
\begin{description}[noitemsep, topsep=0pt, parsep=0pt, partopsep=0pt, leftmargin=0cm, labelwidth=2.2cm]
	\item[\textbf{Taglia/Tipo:}] Media drago, buono
	\item[\textbf{Caratt.:}] \resizebox{0.5\linewidth+1.8cm}{!}{For 3 Des 0 Cos 2 Int 1 Sag 0 Car 2}
	\item[\textbf{Punti Ferita:}] 51,  \textbf{Difesa:} 14,  \textbf{Iniziativa:} +1
	\item[\textbf{Movimento:}] 9 m, nuoto 9 m, volo 18 m
	\item[\textbf{Tiri Salvez.:}] \resizebox{0.5\linewidth+1.8cm}{!}{Tempra +4, Riflessi +3, Volontà +3}
	\item[\textbf{Comp.:}] Furtività +2, Consapevolezza +4
	\item[\textbf{Imm. Danni:}] Elettricità
	\item[\textbf{Sensi:}] Scurovisione 18 m, Vista Cieca 3 m
	\item[\textbf{Linguaggi:}] Draconico
	\item[\textbf{Sfida:}] 2 (450 PX)\smallskip
\end{description}

\emph{\textbf{Anfibio.}} Il drago può respirare aria e acqua.

\textbf{Azioni}

\emph{\textbf{Morso.} Attacco con arma da mischia}: +5 a colpire,
portata 1 m, un bersaglio.

\emph{Colpisce:} 8 (1d10 + 3) danni perforanti.

\emph{\textbf{Arma a Soffio (Ricarica 5-6).}} Il drago usa una delle seguenti armi a soffio:

\emph{Soffio Fulminante.} Il drago esala fulmini in una linea lunga 12 metri e larga 1 metro. Ogni creatura sulla linea deve effettuare un Tiro Salvezza di Riflessi DC 16, subendo 16 (3d10) danni da elettricità se fallisce il Tiro Salvezza, o la metà di questi danni se lo riesce.

\emph{Soffio Repulsivo.} Il drago esala dell'energia repulsiva in un cono di 9 metri. Ogni creatura in quell'area deve riuscire un Tiro Salvezza su Tempra DC 16, altrimenti viene allontana di 9 metri dal drago.

\textbf{Ecologia}\\
Ambiente: Zone Costiere Temperate\\
Organizzazione: Solitario\\
\textbf{Categoria Tesoro}: C\\
\textbf{Descrizione}\\
I draghi di bronzo sono noti per allearsi con viaggiatori ed avventurieri se causa e ricompensa sono giuste e adeguate.

\mostro{Drago d'Oro Antico}
\noindent
\begin{description}[noitemsep, topsep=0pt, parsep=0pt, partopsep=0pt, leftmargin=0cm, labelwidth=2.2cm]
	\item[\textbf{Taglia/Tipo:}] Mastodontica drago, buono
	\item[\textbf{Caratt.:}] \resizebox{0.5\linewidth+1.8cm}{!}{For 10 Des 2 Cos 9 Int 4 Sag 3 Car 9}
	\item[\textbf{Punti Ferita:}] 490,  \textbf{Difesa:} 46,  \textbf{Iniziativa:} +4
	\item[\textbf{Movimento:}] 12 m, nuoto 12 m, volo 24 m
	\item[\textbf{Tiri Salvez.:}] \resizebox{0.5\linewidth+1.8cm}{!}{\resizebox{0.5\linewidth+1.8cm}{!}{Tempra +33, Riflessi +26, Volontà +27}}
	\item[\textbf{Comp.:}] Furtività +9, Percepire Emozioni +10, Consapevolezza +17, Ingannare +16
	\item[\textbf{Imm. Danni:}] Fuoco, armi +1
	\item[\textbf{Sensi:}] Scurovisione 36 m, Vista Cieca 18 m
	\item[\textbf{Linguaggi:}] Comune, Draconico
	\item[\textbf{Sfida:}] 24 (62000 PX)\smallskip
\end{description}

\emph{\textbf{Aura indebolente.}} il drago emette nel raggio di 3 metri un aura che causa Affaticato 2. Rimanere nell'aura non aumenta il livello di affaticato.

\emph{\textbf{Anfibio.}} Il drago può respirare aria e acqua.

\emph{\textbf{Resistenza Leggendaria (3/Giorno).}} Se il drago fallisce un Tiro Salvezza, può scegliere invece di riuscire.

\emph{\textbf{Resistenza alla Magia:}} 3lv

\textbf{Azioni}

\emph{\textbf{Multiattacco.}} Il drago può usare la sua Presenza Spaventosa e poi effettuare tre attacchi: uno con il morso e due con gli artigli.

\emph{\textbf{Artiglio.} Attacco con arma da mischia}: +18 a colpire, portata 3 m, un bersaglio.

\emph{Colpisce:} 17 (2d6 + 10) danni taglienti, 3/20 danno da Sanguinamento.

\emph{\textbf{Coda.} Attacco con arma da mischia}: +18 a colpire, portata 6 m, un bersaglio.

\emph{Colpisce:} 19 (2d8 + 10) danni contundenti.

\emph{\textbf{Morso.} Attacco con arma da mischia}: +18 a colpire, portata 5 metri, un bersaglio.

\emph{Colpisce:} 21 (2d10 + 10) danni perforanti.

\emph{\textbf{Presenza Spaventosa.}} Ogni creatura scelta dal drago, che si trovi entro 36 metri da esso e consapevole della sua presenza, deve riuscire un Tiro Salvezza di Volontà DC 37 o restare spaventata per 1 minuto. Una creatura può ripetere il Tiro Salvezza al termine di ciascun suo round, terminando l'effetto se lo riesce. Se il Tiro Salvezza della creatura ha successo o l'effetto ha termine per essa, la creatura è immune alla Presenza Spaventosa del drago per le successive 24 ore.

\emph{\textbf{Arma a Soffio (Ricarica 5-6).}} Il drago usa una delle seguenti armi a soffio:

\emph{Soffio Infuocato.} Il drago esala fuoco in un cono di 27 metri. Ogni creatura nell'area deve effettuare un Tiro Salvezza di Riflessi DC 37, subendo 71 (13d10) danni da fuoco se fallisce il Tiro Salvezza, o la metà di questi danni se lo riesce.

\emph{Soffio Indebolente.} Il drago esala del gas in un cono di 27 metri. Ogni creatura in quell'area deve riuscire un Tiro Salvezza su Tempra DC 37 o avere -1d6 ai tiri di attacco basati sulla Forza, prove di Forza, e Tiri Salvezza su Tempra per 1 minuto. Una creatura può ripetere il Tiro Salvezza al termine di ciascun suo round, terminando l'effetto su di sé in caso di successo.

\emph{\textbf{Mutare Forma.}} Il drago può trasformarsi magicamente in un umanoide o bestia il cui grado di sfida sia pari o inferiore al proprio, o tornare alla sua vera forma. Alla morte ritorna alla sua vera forma. Qualsiasi equipaggiamento stia indossando o trasportando viene assorbito o trasportato nella nuova forma (a scelta del drago).

Nella nuova forma, il drago mantiene i suoi Tratti, Punti Ferita la facoltà di parlare, le competenze, la Resistenza Leggendaria, le azioni da tana, e i punteggi di Intelligenza, Saggezza e Carisma, oltre a questa azione. Le sue statistiche e capacità vengono altrimenti rimpiazzate da quelle della nuova forma, eccetto Azioni aggiuntive della nuova forma.

\textbf{Azioni Aggiuntive}

Il drago può effettuare 3 Azioni aggiuntive, scelte tra le opzioni seguenti. Può usare solo un'opzione Aggiuntiva alla volta e solo al termine del round di un'altra creatura. Il drago recupera le Azioni aggiuntive spese all'inizio del proprio round.

\textbf{Attacco di Ala (Costa 2 Azioni).} Il drago batte le ali. Ogni creatura entro 5 metri dal drago deve riuscire un Tiro Salvezza su Riflessi DC 37 o subire 17 (2d6 + 10) danni contundenti e venir gettato prono. Il drago può poi volare fino a metà del suo movimento di volo.

\textbf{Attacco di Coda.} Il drago effettua un attacco di coda.

\textbf{Individuare.} Il drago effettua una prova di Consapevolezza.

\textbf{Ecologia}\\
Ambiente: Pianure calde\\
Organizzazione: Solitario\\
\textbf{Categoria Tesoro}: H\\
\textbf{Descrizione}\\
I draghi d'oro sono l'emblema della virtù. Gli altri draghi di Ljust li riveriscono come agenti delle potenze divine e membri esemplari della razza draconica, e spesso li cercano per consigli o aiuto.\\
\textbf{Incantesimi}\index{Incantesimi da Drago d'Oro}\\
Gli incantesimi preferiti di questo Drago sono:\\
- \hyperlink{Guarigione}{Guarigione}\\
- \hyperlink{Ristorare Superiore}{Ristorare Superiore}\\
- \hyperlink{Tentacoli Neri}{Tentacoli Neri}

\mostro{Drago d'Oro Adulto}
\noindent
\begin{description}[noitemsep, topsep=0pt, parsep=0pt, partopsep=0pt, leftmargin=0cm, labelwidth=2.2cm]
	\item[\textbf{Taglia/Tipo:}] Enorme drago, buono
	\item[\textbf{Caratt.:}] \resizebox{0.5\linewidth+1.8cm}{!}{For 8 Des 2 Cos 7 Int 3 Sag 2 Car 7}
	\item[\textbf{Punti Ferita:}] 344,  \textbf{Difesa:} 36,  \textbf{Iniziativa:} +3
	\item[\textbf{Movimento:}] 12 m, nuoto 12 m, volo 24 m
	\item[\textbf{Tiri Salvez.:}] \resizebox{0.5\linewidth+1.8cm}{!}{\resizebox{0.5\linewidth+1.8cm}{!}{Tempra +24, Riflessi +19, Volontà +19}}
	\item[\textbf{Comp.:}] Furtività +8, Percepire Emozioni +8, Consapevolezza +14, Ingannare +13
	\item[\textbf{Imm. Danni:}] Fuoco
	\item[\textbf{Sensi:}] Scurovisione 36 m, Vista Cieca 18 m
	\item[\textbf{Linguaggi:}] Comune, Draconico
	\item[\textbf{Sfida:}] 17 (18000 PX)\smallskip
\end{description}

\emph{\textbf{Aura indebolente.}} il drago emette nel raggio di 3 metri un aura che causa Affaticato 1. Rimanere nell'aura non aumenta il livello di affaticato.

\emph{\textbf{Anfibio.}} Il drago può respirare aria e acqua.

\emph{\textbf{Resistenza Leggendaria (3/Giorno).}} Se il drago fallisce un Tiro Salvezza, può scegliere invece di riuscire.

\textbf{Azioni}

\emph{\textbf{Multiattacco.}} Il drago può usare la sua Presenza Spaventosa e poi effettuare tre attacchi: uno con il morso e due con gli artigli.

\emph{\textbf{Artiglio.} Attacco con arma da mischia}: +14 a colpire, portata 1 m, un bersaglio.

\emph{Colpisce:} 15 (2d6 + 8) danni taglienti, 1 danno da Sanguinamento.

\emph{\textbf{Coda.} Attacco con arma da mischia}: +14 a colpire, portata 5 metri, un bersaglio.

\emph{Colpisce:} 17 (2d8 + 8) danni contundenti.

\emph{\textbf{Morso.} Attacco con arma da mischia}: +14 a colpire, portata 3 m, un bersaglio.

\emph{Colpisce:} 19 (2d10 + 8) danni perforanti.

\emph{\textbf{Presenza Spaventosa.}} Ogni creatura scelta dal drago, che si trovi entro 36 metri da esso e consapevole della sua presenza, deve riuscire un Tiro Salvezza di Volontà DC 30 o restare spaventata per 1 minuto. Una creatura può ripetere il Tiro Salvezza al termine di ciascun suo round, terminando l'effetto se lo riesce. Se il Tiro Salvezza della creatura ha successo o l'effetto ha termine per essa, la creatura è immune alla Presenza Spaventosa del drago per le successive 24 ore.

\emph{\textbf{Arma a Soffio (Ricarica 5-6).}} Il drago usa una delle seguenti armi a soffio:

\emph{Soffio Infuocato.} Il drago esala fuoco in un cono di 18 metri. Ogni creatura nell'area deve effettuare un Tiro Salvezza di Riflessi DC 30, subendo 66 (12d10) danni da fuoco se fallisce il Tiro Salvezza, o la metà di questi danni se lo riesce.

\emph{Soffio Indebolente.} Il drago esala del gas in un cono di 18 metri. Ogni creatura in quell'area deve riuscire un Tiro Salvezza su Tempra DC 30 o avere -1d6 ai tiri di attacco basati sulla Forza, prove di Forza, e Tiri Salvezza su Tempra per 1 minuto. Una creatura può ripetere il Tiro Salvezza al termine di ciascun suo round, terminando l'effetto su di sé in caso di successo.

\emph{\textbf{Mutare Forma.}} Il drago può trasformarsi magicamente in un umanoide o bestia il cui grado di sfida sia pari o inferiore al proprio, o tornare alla sua vera forma. Alla morte ritorna alla sua vera forma. Qualsiasi equipaggiamento stia indossando o trasportando viene assorbito o trasportato nella nuova forma (a scelta del drago).

Nella nuova forma, il drago mantiene i suoi Tratti, Punti Ferita la facoltà di parlare, le competenze, la Resistenza Leggendaria, le azioni da tana, e i punteggi di Intelligenza, Saggezza e Carisma, oltre a questa azione. Le sue statistiche e capacità vengono altrimenti rimpiazzate da quelle della nuova forma, eccetto Azioni aggiuntive della nuova forma.

\textbf{Azioni Aggiuntive}

Il drago può effettuare 3 Azioni aggiuntive, scelte tra le opzioni seguenti. Può usare solo un'opzione Aggiuntiva alla volta e solo al termine del round di un'altra creatura. Il drago recupera le Azioni aggiuntive spese all'inizio del proprio round.

\textbf{Attacco di Ala (Costa 2 Azioni).} Il drago batte le ali. Ogni creatura entro 3 metri dal drago deve riuscire un Tiro Salvezza su Riflessi DC 30 o subire 15 (2d6 + 8) danni contundenti e venir gettato prono. Il drago può poi volare fino a metà del suo movimento di volo.

\textbf{Attacco di Coda.} Il drago effettua un attacco di coda.

\textbf{Individuare.} Il drago effettua una prova di Consapevolezza.

\emph{\textbf{Arrabbiato:}} Il Drago d'Oro Adulto può eseguire queste azioni a costo 2 Azioni.

\emph{Focalizzare}: la creatura interrompe un effetto mentale su di se in corso

\emph{Brutalità}: la creatura attacca con ferocia inaudita. +1d6 al Tiro per Colpire, 1 danno critico automatico quando colpisce.

\textbf{Ecologia}\\
Ambiente: Pianure calde\\
Organizzazione: Solitario\\
\textbf{Categoria Tesoro}: E\\
\textbf{Descrizione}\\
I draghi d'oro sono l'emblema della virtù. Gli altri draghi di Ljust li riveriscono come agenti delle potenze divine e membri esemplari della razza draconica, e spesso li cercano per consigli o aiuto.\\
\textbf{Incantesimi}\index{Incantesimi da Drago d'Oro}\\
Gli incantesimi preferiti di questo Drago sono:\\
- \hyperlink{Guarigione}{Guarigione}\\
- \hyperlink{Ristorare Superiore}{Ristorare Superiore}\\
- \hyperlink{Tentacoli Neri}{Tentacoli Neri}

\mostro{Drago d'Oro Giovane}
\noindent
\begin{description}[noitemsep, topsep=0pt, parsep=0pt, partopsep=0pt, leftmargin=0cm, labelwidth=2.2cm]
	\item[\textbf{Taglia/Tipo:}] Grande drago, buono
	\item[\textbf{Caratt.:}] \resizebox{0.5\linewidth+1.8cm}{!}{For 6 Des 2 Cos 5 Int 3 Sag 1 Car 5}
	\item[\textbf{Punti Ferita:}] 205,  \textbf{Difesa:} 27,  \textbf{Iniziativa:} +3
	\item[\textbf{Movimento:}] 12 m, nuoto 12 m, volo 24 m
	\item[\textbf{Tiri Salvez.:}] \resizebox{0.5\linewidth+1.8cm}{!}{\resizebox{0.5\linewidth+1.8cm}{!}{Tempra +15, Riflessi +12, Volontà +11}}
	\item[\textbf{Comp.:}] Furtività +6, Percepire Emozioni +5, Consapevolezza +9, Ingannare +9
	\item[\textbf{Imm. Danni:}] Fuoco
	\item[\textbf{Sensi:}] Scurovisione 36 m, Vista Cieca 9 m
	\item[\textbf{Linguaggi:}] Comune, Draconico
	\item[\textbf{Sfida:}] 10 (5900 PX)\smallskip
\end{description}

\emph{\textbf{Anfibio.}} Il drago può respirare aria e acqua.

\textbf{Azioni}

\emph{\textbf{Multiattacco.}} Il drago può effettuare tre attacchi: uno con il morso e due con gli artigli.

\emph{\textbf{Artiglio.} Attacco con arma da mischia}: +12 a colpire, portata 1 m, un bersaglio.

\emph{Colpisce:} 13 (2d6 + 6) danni taglienti, 1 danno da Sanguinamento.

\emph{\textbf{Morso.} Attacco con arma da mischia}: +12 a colpire, portata 3 m, un bersaglio.

\emph{Colpisce:} 17 (2d10 + 6) danni perforanti.

\emph{\textbf{Arma a Soffio (Ricarica 5-6).}} Il drago usa una delle seguenti armi a soffio:

\emph{Soffio Infuocato.} Il drago esala fuoco in un cono di 9 metri. Ogni creatura nell'area deve effettuare un Tiro Salvezza di Riflessi DC 23, subendo 55 (10d10) danni da fuoco se fallisce il Tiro Salvezza, o la metà di questi danni se lo riesce.

\emph{Soffio Indebolente.} Il drago esala del gas in un cono di 9 metri. Ogni creatura in quell'area deve riuscire un Tiro Salvezza di Tempra DC 23 o avere -1d6 ai tiri di attacco basati sulla Forza, prove di Forza, e Tiri Salvezza su Tempra per 1 minuto. Una creatura può ripetere il Tiro Salvezza al termine di ciascun suo round, terminando l'effetto su di sé in caso di successo.

\emph{\textbf{Arrabbiato:}} il giovane drago d'oro ricarica uno dei suoi soffi. Costa 1 Azione.

\textbf{Ecologia}\\
Ambiente: Pianure calde\\
Organizzazione: Solitario\\
\textbf{Categoria Tesoro}: D\\
\textbf{Descrizione}\\
I draghi d'oro sono l'emblema della virtù. Gli altri draghi di Ljust li riveriscono come agenti delle potenze divine e membri esemplari della razza draconica, e spesso li cercano per consigli o aiuto.\\
\textbf{Incantesimi}\index{Incantesimi da Drago d'Oro}\\
Gli incantesimi preferiti di questo Drago sono:\\
- \hyperlink{Guarigione}{Guarigione}\\
- \hyperlink{Ristorare Superiore}{Ristorare Superiore}\\
- \hyperlink{Tentacoli Neri}{Tentacoli Neri}

\mostro{Drago d'Oro Cucciolo}
\noindent
\begin{description}[noitemsep, topsep=0pt, parsep=0pt, partopsep=0pt, leftmargin=0cm, labelwidth=2.2cm]
	\item[\textbf{Taglia/Tipo:}] Media drago, buono
	\item[\textbf{Caratt.:}] \resizebox{0.5\linewidth+1.8cm}{!}{For 4 Des 2 Cos 3 Int 2 Sag 0 Car 3}
	\item[\textbf{Punti Ferita:}] 70,  \textbf{Difesa:} 18,  \textbf{Iniziativa:} +2
	\item[\textbf{Movimento:}] 9 m, nuoto 9 m, volo 18 m
	\item[\textbf{Tiri Salvez.:}] \resizebox{0.5\linewidth+1.8cm}{!}{Tempra +6, Riflessi +5, Volontà +3}
	\item[\textbf{Comp.:}] Furtività +4, Consapevolezza +4
	\item[\textbf{Imm. Danni:}] Fuoco
	\item[\textbf{Sensi:}] Scurovisione 18 m, Vista Cieca 3 m
	\item[\textbf{Linguaggi:}] Draconico
	\item[\textbf{Sfida:}] 3 (700 PX)\smallskip
\end{description}

\emph{\textbf{Anfibio.}} Il drago può respirare aria e acqua.

\textbf{Azioni}

\emph{\textbf{Morso.} Attacco con arma da mischia}: +6 a colpire, portata 1 m, un bersaglio.

\emph{Colpisce:} 9 (1d10 + 4) danni perforanti.

\emph{\textbf{Arma a Soffio (Ricarica 5-6).}} Il drago usa una delle seguenti armi a soffio:

\emph{Soffio Infuocato.} Il drago esala fuoco in un cono di 5 metri. Ogni creatura nell'area deve effettuare un Tiro Salvezza di Riflessi DC 15, subendo 22 (4d10) danni da fuoco se fallisce il Tiro Salvezza, o la metà di questi danni se lo riesce.

\emph{Soffio Indebolente.} Il drago esala del gas in un cono di 5 metri. Ogni creatura in quell'area deve riuscire un Tiro Salvezza su Tempra DC 15 o avere -1d6 ai tiri di attacco basati sulla Forza, prove di Forza, e Tiri Salvezza su Tempra per 1 minuto. Una creatura può ripetere il Tiro Salvezza al termine di ciascun suo round, terminando l'effetto su di sé in caso di successo.

\textbf{Ecologia}\\
Ambiente: Pianure calde\\
Organizzazione: Solitario\\
\textbf{Categoria Tesoro}: C\\
\textbf{Descrizione}\\
I draghi d'oro sono l'emblema della virtù. Gli altri draghi di Ljust li riveriscono come agenti delle potenze divine e membri esemplari della razza draconica, e spesso li cercano per consigli o aiuto.

\mostro{Drago di Ottone Antico}
\begin{description}[noitemsep, topsep=0pt, parsep=0pt, partopsep=0pt, leftmargin=0cm, labelwidth=2.2cm]
	\item[\textbf{Taglia/Tipo:}] Mastodontica drago, buono
	\item[\textbf{Caratt.:}] \resizebox{0.5\linewidth+1.8cm}{!}{For 8 Des 0 Cos 7 Int 3 Sag 2 Car 4}
	\item[\textbf{Punti Ferita:}] 403,  \textbf{Difesa:} 38,  \textbf{Iniziativa:} +3
	\item[\textbf{Movimento:}] 12 m, scavo 12 m, volo 24 m
	\item[\textbf{Tiri Salvez.:}] \resizebox{0.5\linewidth+1.8cm}{!}{\resizebox{0.5\linewidth+1.8cm}{!}{Tempra +27, Riflessi +20, Volontà +22}}
	\item[\textbf{Imm. Danni:}] Fuoco, armi +1
	\item[\textbf{Comp.:}] Consapevolezza +14
	\item[\textbf{Sensi:}] Scurovisione 36 m, Vista Cieca 18 m
	\item[\textbf{Linguaggi:}] Comune, Draconico
	\item[\textbf{Sfida:}] 20 (25000 PX)\smallskip
\end{description}

\emph{\textbf{Aura soporifera.}} il drago emette nel raggio di 3 metri una aura magica che causa Rallentato 1 o Affaticato 1, casualmente per creatura.

\emph{\textbf{Resistenza Leggendaria (3/Giorno).}} Se il drago fallisce un Tiro Salvezza, può scegliere invece di riuscire.

\emph{\textbf{Resistenza alla Magia:}} 3lv

\textbf{Azioni}

\emph{\textbf{Multiattacco.}} Il drago può usare la sua Presenza Spaventosa e poi effettuare tre attacchi: uno con il morso e due con gli artigli.

\emph{\textbf{Artiglio.} Attacco con arma da mischia}: +16 a colpire, portata 3 m, un bersaglio.

\emph{Colpisce:} 15 (2d6 + 8) danni taglienti, 3/20 danno da Sanguinamento.

\emph{\textbf{Coda.} Attacco con arma da mischia}: +16 a colpire, portata 6 m, un bersaglio.

\emph{Colpisce:} 17 (2d8 + 8) danni contundenti.

\emph{\textbf{Morso.} Attacco con arma da mischia}: +16 a colpire, portata 5 metri, un bersaglio.

\emph{Colpisce:} 19 (2d10 + 8) danni perforanti.

\emph{\textbf{Presenza Spaventosa.}} Ogni creatura scelta dal drago, che si trovi entro 36 metri da esso e consapevole della sua presenza, deve riuscire un Tiro Salvezza di Volontà DC 34 o restare spaventata per 1 minuto. Una creatura può ripetere il Tiro Salvezza al termine di ciascun suo round, terminando l'effetto se lo riesce. Se il Tiro Salvezza della creatura ha successo o l'effetto ha termine per essa, la creatura è immune alla Presenza Spaventosa del drago per le successive 24 ore.

\emph{\textbf{Arma a Soffio (Ricarica 5-6).}} Il drago usa una delle seguenti armi a soffio:

\emph{Soffio Infuocato.} Il drago esala fuoco in una linea lunga 27 metri e larga 3 metri. Ogni creatura sulla linea deve effettuare un Tiro Salvezza su Riflessi DC 34, subendo 56 (16d6) danni da fuoco se fallisce il Tiro Salvezza, o la metà di questi danni se lo riesce.

\emph{Soffio Soporifero.} Il drago esala del gas soporifero in un cono di 27 metri. Ogni creatura in quell'area deve riuscire un Tiro Salvezza su Tempra 34 o cadere svenuta per 10 minuti. Questo effetto termina se la creatura svenuta subisce danni o qualcuno impiega un'azione per risvegliarla.

\emph{\textbf{Mutare Forma.}} Il drago può trasformarsi magicamente in un umanoide o bestia il cui grado di sfida sia pari o inferiore al proprio, o tornare alla sua vera forma. Alla morte ritorna alla sua vera forma. Qualsiasi equipaggiamento stia indossando o trasportando viene assorbito o trasportato nella nuova forma (a scelta del drago).

Nella nuova forma, il drago mantiene i suoi Tratti, Punti Ferita la facoltà di parlare, le competenze, la Resistenza Leggendaria, le azioni da tana, e i punteggi di Intelligenza, Saggezza e Carisma, oltre a questa azione. Le sue statistiche e capacità vengono altrimenti rimpiazzate da quelle della nuova forma, eccetto Azioni aggiuntive della nuova forma.

\textbf{Azioni Aggiuntive}

Il drago può effettuare 3 Azioni aggiuntive, scelte tra le opzioni seguenti. Può usare solo un'opzione Aggiuntiva alla volta e solo al termine del round di un'altra creatura. Il drago recupera le Azioni aggiuntive spese all'inizio del proprio round.

\textbf{Attacco di Ala (Costa 2 Azioni).} Il drago batte le ali. Ogni creatura entro 5 metri dal drago deve riuscire un Tiro Salvezza su Riflessi DC 34 o subire 15 (2d6 + 8) danni contundenti e venir gettato prono. Il drago può poi volare fino a metà del suo movimento di volo.

\textbf{Attacco di Coda.} Il drago effettua un attacco di coda.

\textbf{Individuare.} Il drago effettua una prova di Consapevolezza.

\textbf{Ecologia}\\
Ambiente: Deserti Caldi\\
Organizzazione: Solitario\\
\textbf{Categoria Tesoro}: H\\
\textbf{Descrizione}\\
Ottimi conversatori, i draghi d'ottone preferiscono parlare invece che combattere. I draghi d'ottone fanno la tana vicino agli insediamenti umanoidi, dove possono udire le notizie e i pettegolezzi più recenti.\\
\textbf{Incantesimi}\index{Incantesimi da Drago d'Ottone}\\
Gli incantesimi preferiti di questo Drago sono:\\
- \hyperlink{Visione del Vero}{Visione del Vero}\\
- \hyperlink{Conoscenza delle Leggende}{Conoscenza delle Leggende}\\
- \hyperlink{Scrutare}{Scrutare}

\mostro{Drago d'Ottone Adulto}
\begin{description}[noitemsep, topsep=0pt, parsep=0pt, partopsep=0pt, leftmargin=0cm, labelwidth=2.2cm]
	\item[\textbf{Taglia/Tipo:}] Enorme drago, buono
	\item[\textbf{Caratt.:}] \resizebox{0.5\linewidth+1.8cm}{!}{For 6 Des 0 Cos 5 Int 2 Sag 1 Car 3}
	\item[\textbf{Punti Ferita:}] 262,  \textbf{Difesa:} 29,  \textbf{Iniziativa:} +2
	\item[\textbf{Movimento:}] 12 m, scavo 9 m, volo 24 m
	\item[\textbf{Tiri Salvez.:}] \resizebox{0.5\linewidth+1.8cm}{!}{\resizebox{0.5\linewidth+1.8cm}{!}{Tempra +18, Riflessi +13, Volontà +14}}
	\item[\textbf{Imm. Danni:}] Fuoco
	\item[\textbf{Comp.:}] Furtività +5, Consapevolezza +11, Ingannare +8, Storia +7
	\item[\textbf{Sensi:}] Scurovisione 36 m, Vista Cieca 18 m
	\item[\textbf{Linguaggi:}] Comune, Draconico
	\item[\textbf{Sfida:}] 13 (10000 PX)\smallskip
\end{description}

\emph{\textbf{Resistenza Leggendaria (3/Giorno).}} Se il drago fallisce un Tiro Salvezza, può scegliere invece di riuscire.

\textbf{Azioni}

\emph{\textbf{Multiattacco.}} Il drago può usare la sua Presenza Spaventosa e poi effettuare tre attacchi: uno con il morso e due con gli artigli.

\emph{\textbf{Artiglio.} Attacco con arma da mischia}: +12 a colpire, portata 1 m, un bersaglio.

\emph{Colpisce:} 13 (2d6 + 6) danni taglienti, 1 danno da Sanguinamento.

\emph{\textbf{Coda.} Attacco con arma da mischia}: +12 a colpire, portata 5 metri, un bersaglio.

\emph{Colpisce:} 15 (2d8 + 6) danni contundenti.

\emph{\textbf{Morso.} Attacco con arma da mischia}: +12 a colpire, portata 3 m, un bersaglio.

\emph{Colpisce:} 17 (2d10 + 6) danni perforanti.

\emph{\textbf{Presenza Spaventosa.}} Ogni creatura scelta dal drago, che si trovi entro 36 metri da esso e consapevole della sua presenza, deve riuscire un Tiro Salvezza di Volontà DC 26 o restare spaventata per 1 minuto. Una creatura può ripetere il Tiro Salvezza al termine di ciascun suo round, terminando l'effetto se lo riesce. Se il Tiro Salvezza della creatura ha successo o l'effetto ha termine per essa, la creatura è immune alla Presenza Spaventosa del drago per le successive 24 ore.

\emph{\textbf{Arma a Soffio (Ricarica 5-6).}} Il drago usa una delle seguenti armi a soffio:

\emph{Soffio Infuocato.} Il drago esala fuoco in una linea lunga 18 metri e larga 1 metro. Ogni creatura sulla linea deve effettuare un Tiro Salvezza di Riflessi DC 26, subendo 45 (13d6) danni da fuoco se fallisce il Tiro Salvezza, o la metà di questi danni se lo riesce.

\emph{Soffio Soporifero.} Il drago esala del gas soporifero in un cono di 18 metri. Ogni creatura in quell'area deve riuscire un Tiro Salvezza su Tempra 26 o cadere svenuta per 10 minuti. Questo effetto termina se la creatura svenuta subisce danni o qualcuno impiega un'Azione per risvegliarla.

\textbf{Azioni Aggiuntive}

Il drago può effettuare 3 Azioni aggiuntive, scelte tra le opzioni seguenti. Può usare solo un'opzione Aggiuntiva alla volta e solo al termine del round di un'altra creatura. Il drago recupera le Azioni aggiuntive spese all'inizio del proprio round.

\textbf{Attacco di Ala (Costa 2 Azioni).} Il drago batte le ali. Ogni creatura entro 3 metri dal drago deve riuscire un Tiro Salvezza su Riflessi DC 26 o subire 13 (2d6 + 6) danni contundenti e venir gettato prono. Il drago può poi volare fino a metà del suo movimento di volo.

\textbf{Attacco di Coda.} Il drago effettua un attacco di coda.

\textbf{Individuare.} Il drago effettua una prova di Consapevolezza.

\emph{\textbf{Arrabbiato:}} Il Drago d'ottone Adulto può eseguire queste azioni a costo 2 Azioni.

\emph{Focalizzare}: la creatura interrompe un effetto mentale su di se in corso

\emph{Brutalità}: la creatura attacca con ferocia inaudita. +1d6 al Tiro per Colpire, 1 danno critico automatico quando colpisce.

\textbf{Ecologia}\\
Ambiente: Deserti Caldi\\
Organizzazione: Solitario\\
\textbf{Categoria Tesoro}: E\\
\textbf{Descrizione}\\
Ottimi conversatori, i draghi d'ottone preferiscono parlare invece che combattere. I draghi d'ottone fanno la tana vicino agli insediamenti umanoidi, dove possono udire le notizie e i pettegolezzi più recenti.\\
\textbf{Incantesimi}\index{Incantesimi da Drago d'Ottone}\\
Gli incantesimi preferiti di questo Drago sono:\\
- \hyperlink{Visione del Vero}{Visione del Vero}\\
- \hyperlink{Conoscenza delle Leggende}{Conoscenza delle Leggende}\\
- \hyperlink{Scrutare}{Scrutare}

\mostro{Drago d'Ottone Giovane}
\begin{description}[noitemsep, topsep=0pt, parsep=0pt, partopsep=0pt, leftmargin=0cm, labelwidth=2.2cm]
	\item[\textbf{Taglia/Tipo:}] Grande drago, buono
	\item[\textbf{Caratt.:}] \resizebox{0.5\linewidth+1.8cm}{!}{For 4 Des 0 Cos 3 Int 1 Sag 0 Car 2}
	\item[\textbf{Punti Ferita:}] 126,  \textbf{Difesa:} 20,  \textbf{Iniziativa:} +1
	\item[\textbf{Movimento:}] 12 m, scavo 6 m, volo 24 m
	\item[\textbf{Tiri Salvez.:}] \resizebox{0.5\linewidth+1.8cm}{!}{Tempra +9, Riflessi +6, Volontà +6}
	\item[\textbf{Imm. Danni:}] Fuoco
	\item[\textbf{Comp.:}] Furtività +3, Consapevolezza +6, Ingannare +5
	\item[\textbf{Sensi:}] Scurovisione 18 m, Vista Cieca 3 m
	\item[\textbf{Linguaggi:}] Comune, Draconico
	\item[\textbf{Sfida:}] 6 (2300 PX)\smallskip
\end{description}

\textbf{Azioni}

\emph{\textbf{Multiattacco.}} Il drago può effettuare tre attacchi: uno con il morso e due con gli artigli.

\emph{\textbf{Artiglio.} Attacco con arma da mischia}: +7 a colpire, portata 1 m, un bersaglio.

\emph{Colpisce:} 11 (2d6 + 4) danni taglienti, 1 danno da Sanguinamento.

\emph{\textbf{Morso.} Attacco con arma da mischia}: +7 a colpire, portata 3 m, un bersaglio.

\emph{Colpisce:} 15 (2d10 + 4) danni perforanti.

\emph{\textbf{Arma a Soffio (Ricarica 5-6).}} Il drago usa una delle seguenti armi a soffio:

\emph{Soffio Infuocato.} Il drago esala fuoco in una linea lunga 12 metri e larga 1 metro. Ogni creatura sulla linea deve effettuare un Tiro Salvezza di Riflessi DC 18, subendo 42 (12d6) danni da fuoco se fallisce il Tiro Salvezza, o la metà di questi danni se lo riesce.

\emph{Soffio Soporifero.} Il drago esala del gas soporifero in un cono di 9 metri. Ogni creatura in quell'area deve riuscire un Tiro Salvezza su Tempra 18 o cadere svenuta per 5 minuti. Questo effetto termina se la creatura svenuta subisce danni o qualcuno impiega un'Azione per risvegliarla.

\textbf{Ecologia}\\
Ambiente: Deserti Caldi\\
Organizzazione: Solitario\\
\textbf{Categoria Tesoro}: D\\
\textbf{Descrizione}\\
Ottimi conversatori, i draghi d'ottone preferiscono parlare invece che combattere. I draghi d'ottone fanno la tana vicino agli insediamenti umanoidi, dove possono udire le notizie e i pettegolezzi più recenti.\\
\textbf{Incantesimi}\index{Incantesimi da Drago d'Ottone}\\
Gli incantesimi preferiti di questo Drago sono:\\
- \hyperlink{Visione del Vero}{Visione del Vero}\\
- \hyperlink{Conoscenza delle Leggende}{Conoscenza delle Leggende}\\
- \hyperlink{Scrutare}{Scrutare}

\mostro{Drago di Ottone Cucciolo}
\begin{description}[noitemsep, topsep=0pt, parsep=0pt, partopsep=0pt, leftmargin=0cm, labelwidth=2.2cm]
	\item[\textbf{Taglia/Tipo:}] Media drago, buono
	\item[\textbf{Caratt.:}] \resizebox{0.5\linewidth+1.8cm}{!}{For 2 Des 0 Cos 1 Int 0 Sag 0 Car 1}
	\item[\textbf{Punti Ferita:}] 33,  \textbf{Difesa:} 13,  \textbf{Iniziativa:} +0
	\item[\textbf{Movimento:}] 9 m, scavo 5 m, volo 18 m
	\item[\textbf{Tiri Salvez.:}] \resizebox{0.5\linewidth+1.8cm}{!}{Tempra +3, Riflessi +3, Volontà +3}
	\item[\textbf{Imm. Danni:}] Fuoco
	\item[\textbf{Comp.:}] Furtività +2, Consapevolezza +4
	\item[\textbf{Sensi:}] Scurovisione 18 m, Vista Cieca 3 m
	\item[\textbf{Linguaggi:}] Draconico
	\item[\textbf{Sfida:}] 1 (200 PX)\smallskip
\end{description}

\textbf{Azioni}

\emph{\textbf{Morso.} Attacco con arma da mischia}: +4 a colpire, portata 1 m, un bersaglio.

\emph{Colpisce:} 7 (1d10 + 2) danni perforanti.

\emph{\textbf{Arma a Soffio (Ricarica 5-6).}} Il drago usa una delle seguenti armi a soffio:

\emph{Soffio Infuocato.} Il drago esala fuoco in una linea lunga 6 metri e larga 1 metro. Ogni creatura sulla linea deve effettuare un Tiro Salvezza su Riflessi DC 12, subendo 14 (4d6) danni da fuoco se fallisce il Tiro Salvezza, o la metà di questi danni se lo riesce.

\emph{Soffio Soporifero.} Il drago esala del gas soporifero in un cono di 5 metri. Ogni creatura in quell'area deve riuscire un Tiro Salvezza su Tempra 12 o cadere svenuta per 1 minuto. Questo effetto termina se la creatura svenuta subisce danni o qualcuno impiega un'Azione per risvegliarla.

\textbf{Ecologia}\\
Ambiente: Deserti Caldi\\
Organizzazione: Solitario\\
\textbf{Categoria Tesoro}: C\\
\textbf{Descrizione}\\
Ottimi conversatori, i draghi d'ottone preferiscono parlare invece che combattere. I draghi d'ottone fanno la tana vicino agli insediamenti umanoidi, dove possono udire le notizie e i pettegolezzi più recenti.

\mostro{Drago di Rame Antico}
\noindent
\begin{description}[noitemsep, topsep=0pt, parsep=0pt, partopsep=0pt, leftmargin=0cm, labelwidth=2.2cm]
	\item[\textbf{Taglia/Tipo:}] Mastodontica drago, buono
	\item[\textbf{Caratt.:}] \resizebox{0.5\linewidth+1.8cm}{!}{For 8 Des 1 Cos 7 Int 5 Sag 3 Car 4}
	\item[\textbf{Punti Ferita:}] 422,  \textbf{Difesa:} 41,  \textbf{Iniziativa:} +5
	\item[\textbf{Movimento:}] 12 m, scalata 12 m, volo 24 m
	\item[\textbf{Tiri Salvez.:}] \resizebox{0.5\linewidth+1.8cm}{!}{\resizebox{0.5\linewidth+1.8cm}{!}{Tempra +28, Riflessi +22, Volontà +24}}
	\item[\textbf{Comp.:}] Furtività +8, Ingannare +11, Consapevolezza +17
	\item[\textbf{Imm. Danni:}] Acido, armi +1
	\item[\textbf{Sensi:}] Scurovisione 36 m, Vista Cieca 18 m
	\item[\textbf{Linguaggi:}] Comune, Draconico
	\item[\textbf{Sfida:}] 21 (33000 PX)\smallskip
\end{description}

\emph{\textbf{Gas corrosivi.}} il drago emette nel raggio di 3 metri gas corrosivi che causano 2d6 danni da acido a round.

\emph{\textbf{Resistenza Leggendaria (3/Giorno).}} Se il drago fallisce un Tiro Salvezza, può scegliere invece di riuscire.

\emph{\textbf{Resistenza alla Magia:}} 3lv

\textbf{Azioni}

\emph{\textbf{Multiattacco.}} Il drago può usare la sua Presenza Spaventosa e poi effettuare tre attacchi: uno con il morso e due con gli artigli.

\emph{\textbf{Artiglio.} Attacco con arma da mischia}: +16 a colpire, portata 3 m, un bersaglio.

\emph{Colpisce:} 15 (2d6 + 8) danni taglienti, 3/20 danno da Sanguinamento.

\emph{\textbf{Coda.} Attacco con arma da mischia}: +16 a colpire, portata 6 m, un bersaglio.

\emph{Colpisce:} 17 (2d8 + 8) danni contundenti.

\emph{\textbf{Morso.} Attacco con arma da mischia}: +16 a colpire, portata 5 metri, un bersaglio.

\emph{Colpisce:} 19 (2d10 + 8) danni perforanti.

\emph{\textbf{Presenza Spaventosa.}} Ogni creatura scelta dal drago, che si trovi entro 36 metri da esso e consapevole della sua presenza, deve riuscire un Tiro Salvezza di Volontà DC 34 o restare spaventata per 1 minuto. Una creatura può ripetere il Tiro Salvezza al termine di ciascun suo round, terminando l'effetto se lo riesce. Se il Tiro Salvezza della creatura ha successo o l'effetto ha termine per essa, la creatura è immune alla Presenza Spaventosa del drago per le successive 24 ore.

\emph{\textbf{Arma a Soffio (Ricarica 5-6).}} Il drago usa una delle seguenti armi a soffio:

\emph{Soffio Acido.} Il drago esala acido in una linea lunga 27 metri e larga 3 metri. Ogni creatura sulla linea deve effettuare un Tiro Salvezza su Riflessi DC 34, subendo 63 (14d8) danni da acido se fallisce il Tiro Salvezza, o la metà di questi danni se lo riesce.

\emph{Soffio Rallentante.} Il drago esala del gas in un cono di 27 metri. Ogni creatura in quell'area deve riuscire un Tiro Salvezza su Tempra DC 34. Se fallisce il Tiro Salvezza, la creatura ha una Azione in meno a round ed ha la velocità dimezzata. Questi effetti permangono 1 minuto. La creatura può ripetere il Tiro Salvezza al termine di ciascun suo round, terminando l'effetto su di sé in caso di successo.

\emph{\textbf{Mutare Forma.}} Il drago può trasformarsi magicamente in un umanoide o bestia il cui grado di sfida sia pari o inferiore al proprio, o tornare alla sua vera forma. Alla morte ritorna alla sua vera forma. Qualsiasi equipaggiamento stia indossando o trasportando viene assorbito o trasportato nella nuova forma (a scelta del drago).

Nella nuova forma, il drago mantiene i suoi Tratti, Punti Ferita la facoltà di parlare, le competenze, la Resistenza Leggendaria, le azioni da tana, e i punteggi di Intelligenza, Saggezza e Carisma, oltre a questa Azione. Le sue statistiche e capacità

vengono altrimenti rimpiazzate da quelle della nuova forma, eccetto Azioni aggiuntive della nuova forma.

\textbf{Azioni Aggiuntive}

Il drago può effettuare 3 Azioni aggiuntive, scelte tra le opzioni seguenti. Può usare solo un'opzione Aggiuntiva alla volta e solo al termine del round di un'altra creatura. Il drago recupera le Azioni aggiuntive spese all'inizio del proprio round.

\textbf{Attacco di Ala (Costa 2 Azioni).} Il drago batte le ali. Ogni creatura entro 5 metri dal drago deve riuscire un Tiro Salvezza su Riflessi DC 34 o subire 15 (2d6 + 8) danni contundenti e venir gettato prono. Il drago può poi volare fino a metà del suo movimento di volo.

\textbf{Attacco di Coda.} Il drago effettua un attacco di coda.

\textbf{Individuare.} Il drago effettua una prova di Consapevolezza.

\textbf{Ecologia}\\
Ambiente: Colline Calde\\
Organizzazione: Solitario\\
\textbf{Categoria Tesoro}: H\\
\textbf{Descrizione}\\
Questo drago capriccioso durante il combattimento cerca di ostacolare e frustrare i suoi nemici.\\
\textbf{Incantesimi}\index{Incantesimi da Drago di Rame}\\
Gli incantesimi preferiti di questo Drago sono:\\
- \hyperlink{Barriera di Lame}{Barriera di Lame}\\
- \hyperlink{Muro di Forza}{Muro di Forza}\\
- \hyperlink{Scudo di Fuoco}{Scudo di Fuoco}

\mostro{Drago di Rame Adulto}
\noindent
\begin{description}[noitemsep, topsep=0pt, parsep=0pt, partopsep=0pt, leftmargin=0cm, labelwidth=2.2cm]
	\item[\textbf{Taglia/Tipo:}] Enorme drago, buono
	\item[\textbf{Caratt.:}] \resizebox{0.5\linewidth+1.8cm}{!}{For 6 Des 1 Cos 5 Int 4 Sag 2 Car 3}
	\item[\textbf{Punti Ferita:}] 281,  \textbf{Difesa:} 31,  \textbf{Iniziativa:} +4
	\item[\textbf{Movimento:}] 12 m, scalata 12 m, volo 24 m
	\item[\textbf{Tiri Salvez.:}] \resizebox{0.5\linewidth+1.8cm}{!}{\resizebox{0.5\linewidth+1.8cm}{!}{Tempra +19, Riflessi +15, Volontà +16}}
	\item[\textbf{Comp.:}] Furtività +6, Ingannare +8, Consapevolezza +12
	\item[\textbf{Imm. Danni:}] Acido
	\item[\textbf{Sensi:}] Scurovisione 36 m, Vista Cieca 18 m
	\item[\textbf{Linguaggi:}] Comune, Draconico
	\item[\textbf{Sfida:}] 14 (11.500 PX)\smallskip
\end{description}

\emph{\textbf{Resistenza Leggendaria (3/Giorno).}} Se il drago fallisce un Tiro Salvezza, può scegliere invece di riuscire.

\textbf{Azioni}

\emph{\textbf{Multiattacco.}} Il drago può usare la sua Presenza Spaventosa e poi effettuare tre attacchi: uno con il morso e due con gli artigli.

\emph{\textbf{Artiglio.} Attacco con arma da mischia}: +13 a colpire, portata 1 m, un bersaglio.

\emph{Colpisce:} 13 (2d6 + 6) danni taglienti, 1 danno da Sanguinamento.

\emph{\textbf{Coda.} Attacco con arma da mischia}: +13 a colpire, portata 5 metri, un bersaglio.

\emph{Colpisce:} 15 (2d8 + 6) danni contundenti.

\emph{\textbf{Morso.} Attacco con arma da mischia}: +13 a colpire, portata 3 m, un bersaglio.

\emph{Colpisce:} 17 (2d10 + 6) danni perforanti.

\emph{\textbf{Presenza Spaventosa.}} Ogni creatura scelta dal drago, che si trovi entro 36 metri da esso e consapevole della sua presenza, deve riuscire un Tiro Salvezza di Volontà DC 27 o restare spaventata per 1 minuto. Una creatura può ripetere il Tiro Salvezza al termine di ciascun suo round, terminando l'effetto se lo riesce. Se il Tiro Salvezza della creatura ha successo o l'effetto ha termine per essa, la creatura è immune alla Presenza Spaventosa del drago per le successive 24 ore.

\emph{\textbf{Arma a Soffio (Ricarica 5-6).}} Il drago usa una delle seguenti armi a soffio:

\emph{Soffio Acido.} Il drago esala acido in una linea lunga 18 metri e larga 1 metro. Ogni creatura sulla linea deve effettuare un Tiro Salvezza su Riflessi DC 27, subendo 54 (12d8) danni da acido se fallisce il Tiro Salvezza, o la metà di questi danni se lo riesce.

\emph{Soffio Rallentante.} Il drago esala del gas in un cono di 27 metri. Ogni creatura in quell'area deve riuscire un Tiro Salvezza su Tempra DC 27. Se fallisce il Tiro Salvezza, la creatura ha una Azione in meno a round ed ha la velocità dimezzata. Questi effetti permangono 1 minuto. La creatura può ripetere il Tiro Salvezza al termine di ciascun suo round, terminando l'effetto su di sé in caso di successo.

\textbf{Azioni Aggiuntive}

Il drago può effettuare 3 Azioni aggiuntive, scelte tra le opzioni seguenti. Può usare solo un'opzione Aggiuntiva alla volta e solo al termine del round di un'altra creatura. Il drago recupera le Azioni aggiuntive spese all'inizio del proprio round.

\textbf{Attacco di Ala (Costa 2 Azioni).} Il drago batte le ali. Ogni creatura entro 3 metri dal drago deve riuscire un Tiro Salvezza su Riflessi DC 27 o subire 13 (2d6 + 6) danni contundenti e venir gettato prono. Il drago può poi volare fino a metà del suo movimento di volo.

\textbf{Attacco di Coda.} Il drago effettua un attacco di coda.

\textbf{Individuare.} Il drago effettua una prova di Consapevolezza.

\emph{\textbf{Arrabbiato:}} Il Drago di Rame Adulto può eseguire queste azioni a costo 2 Azioni.

\emph{Focalizzare}: la creatura interrompe un effetto mentale su di se in corso

\emph{Brutalità}: la creatura attacca con ferocia inaudita. +1d6 al Tiro per Colpire, 1 danno critico automatico quando colpisce.

\textbf{Ecologia}\\
Ambiente: Colline Calde\\
Organizzazione: Solitario\\
\textbf{Categoria Tesoro}: E\\
\textbf{Descrizione}\\
Questo drago capriccioso durante il combattimento cerca di ostacolare e frustrare i suoi nemici.\\
\textbf{Incantesimi}\index{Incantesimi da Drago di Rame}\\
Gli incantesimi preferiti di questo Drago sono:\\
- \hyperlink{Barriera di Lame}{Barriera di Lame}\\
- \hyperlink{Muro di Forza}{Muro di Forza}\\
- \hyperlink{Scudo di Fuoco}{Scudo di Fuoco}

%\begin{center}
%\includegraphics[width=0.45\textwidth]{immagini/Media_età_del_bronzo,_pugnale_in_bronzo,_02.png}
%\end{center}

\mostro{Drago di Rame Giovane}
\noindent
\begin{description}[noitemsep, topsep=0pt, parsep=0pt, partopsep=0pt, leftmargin=0cm, labelwidth=2.2cm]
	\item[\textbf{Taglia/Tipo:}] Grande drago, buono
	\item[\textbf{Caratt.:}] \resizebox{0.5\linewidth+1.8cm}{!}{For 4 Des 1 Cos 3 Int 3 Sag 1 Car 2}
	\item[\textbf{Punti Ferita:}] 145,  \textbf{Difesa:} 22,  \textbf{Iniziativa:} +3
	\item[\textbf{Movimento:}] 12 m, scalata 12 m, volo 24 m
	\item[\textbf{Tiri Salvez.:}] \resizebox{0.5\linewidth+1.8cm}{!}{Tempra +10, Riflessi +8, Volontà +8}
	\item[\textbf{Comp.:}] Furtività +4, Ingannare +5, Consapevolezza +7
	\item[\textbf{Imm. Danni:}] Acido
	\item[\textbf{Sensi:}] Scurovisione 36 m, Vista Cieca 18 m
	\item[\textbf{Linguaggi:}] Comune, Draconico
	\item[\textbf{Sfida:}] 7 (2900 PX)\smallskip
\end{description}

\textbf{Azioni}

\emph{\textbf{Multiattacco.}} Il drago può effettuare tre attacchi: uno con il morso e due con gli artigli.

\emph{\textbf{Artiglio.} Attacco con arma da mischia}: +8 a colpire, portata 1 m, un bersaglio.

\emph{Colpisce:} 11 (2d6 + 4) danni taglienti, 1 danno da Sanguinamento.

\emph{\textbf{Morso.} Attacco con arma da mischia}: +8 a colpire, portata 3 m, un bersaglio.

\emph{Colpisce:} 15 (2d10 + 4) danni perforanti.

\emph{\textbf{Arma a Soffio (Ricarica 5-6).}} Il drago usa una delle seguenti armi a soffio:

\emph{Soffio Acido.} Il drago esala acido in una linea lunga 12 metri e larga 1 metro. Ogni creatura sulla linea deve effettuare un Tiro Salvezza su Riflessi DC 19, subendo 40 (9d8) danni da acido se fallisce il Tiro Salvezza, o la metà di questi danni se lo riesce.

\emph{Soffio Rallentante.} Il drago esala del gas in un cono di 27 metri. Ogni creatura in quell'area deve riuscire un Tiro Salvezza su Tempra DC 19. Se fallisce il Tiro Salvezza, la creatura ha una Azione in meno a round ed ha la velocità dimezzata. Questi effetti permangono 1 minuto. La creatura può ripetere il Tiro Salvezza al termine di ciascun suo round, terminando l'effetto su di sé in caso di successo.

\emph{\textbf{Arrabbiato:}} il Drago di Rame Giovane ricarica uno dei due soffi. Costo 1 Azione.

\textbf{Ecologia}\\
Ambiente: Colline Calde\\
Organizzazione: Solitario\\
\textbf{Categoria Tesoro}: D\\
\textbf{Descrizione}\\
Questo drago capriccioso durante il combattimento cerca di ostacolare e frustrare i suoi nemici.\\
\textbf{Incantesimi}\index{Incantesimi da Drago di Rame}\\
Gli incantesimi preferiti di questo Drago sono:\\
- \hyperlink{Barriera di Lame}{Barriera di Lame}\\
- \hyperlink{Muro di Forza}{Muro di Forza}\\
- \hyperlink{Scudo di Fuoco}{Scudo di Fuoco}

\mostro{Drago di Rame Cucciolo}
\noindent
\begin{description}[noitemsep, topsep=0pt, parsep=0pt, partopsep=0pt, leftmargin=0cm, labelwidth=2.2cm]
	\item[\textbf{Taglia/Tipo:}] Media drago, buono
	\item[\textbf{Caratt.:}] \resizebox{0.5\linewidth+1.8cm}{!}{For 2 Des 1 Cos 1 Int 2 Sag 0 Car 1}
	\item[\textbf{Punti Ferita:}] 33,  \textbf{Difesa:} 14,  \textbf{Iniziativa:} +2
	\item[\textbf{Movimento:}] 9 m, scalata 9 m, volo 18 m
	\item[\textbf{Tiri Salvez.:}] \resizebox{0.5\linewidth+1.8cm}{!}{Tempra +3, Riflessi +3, Volontà +3}
	\item[\textbf{Comp.:}] Furtività +3, Ingannare +3, Consapevolezza +4
	\item[\textbf{Imm. Danni:}] Acido
	\item[\textbf{Sensi:}] Scurovisione 36 m, Vista Cieca 18 m
	\item[\textbf{Linguaggi:}] Comune, Draconico
	\item[\textbf{Sfida:}] 1 (200 PX)\smallskip
\end{description}

\textbf{Azioni}

\emph{\textbf{Morso.} Attacco con arma da mischia}: +5 a colpire, portata 1 m, un bersaglio.

\emph{Colpisce:} 7 (1d10 + 2) danni perforanti.

\emph{\textbf{Arma a Soffio (Ricarica 5-6).}} Il drago usa una delle seguenti armi a soffio:

\emph{Soffio Acido.} Il drago esala acido in una linea lunga 6 metri e larga 1 metro. Ogni creatura sulla linea deve effettuare un Tiro Salvezza su Riflessi DC 12, subendo 18 (4d8) danni da acido se fallisce il Tiro Salvezza, o la metà di questi danni se lo riesce.

\emph{Soffio Rallentante.} Il drago esala del gas in un cono di 27 metri. Ogni creatura in quell'area deve riuscire un Tiro Salvezza su Tempra DC 12. Se fallisce il Tiro Salvezza, la creatura ha una Azione in meno a round ed ha la velocità dimezzata. Questi effetti permangono 1 minuto. La creatura può ripetere il Tiro Salvezza al termine di ciascun suo round, terminando l'effetto su di sé in caso di successo.

\textbf{Ecologia}\\
Ambiente: Colline Calde\\
Organizzazione: Solitario\\
\textbf{Categoria Tesoro}: C\\
\textbf{Descrizione}\\
Questo drago capriccioso durante il combattimento cerca di ostacolare e frustrare i suoi nemici.

\begin{enfasi}{Il gran dragone, il serpente antico, che è chiamato diavolo e Satana, il seduttore di tutto il mondo, fu gettato giù; fu gettato sulla terra, e con lui furono gettati anche i suoi angeli." Giovanni, Apocalisse 12:9}\end{enfasi}

\mostro{Tàhil}
\noindent
\begin{description}[noitemsep, topsep=0pt, parsep=0pt, partopsep=0pt, leftmargin=0cm, labelwidth=2.2cm]
	\item[\textbf{Taglia/Tipo:}] Colossale drago, Patrono
	\item[\textbf{Caratt.:}] \resizebox{0.5\linewidth+1.8cm}{!}{For 10 Des 0 Cos 10 Int 8 Sag 8 Car 9}
	\item[\textbf{Punti Ferita:}] 615,  \textbf{Difesa:} 52,  \textbf{Iniziativa:} +8
	\item[\textbf{Movimento:}] 20 metri, volare 20 metri
	\item[\textbf{Tiri Salvez.:}] \resizebox{0.5\linewidth+1.8cm}{!}{\resizebox{0.5\linewidth+1.8cm}{!}{Tempra +40, Riflessi +30, Volontà +38}}
	\item[\textbf{Comp.:}] tutte +18
	\item[\textbf{Imm. Danni:}] Freddo, Elettricità, Fuoco, Acido, Veleno, Suono, armi +3
	\item[\textbf{Immunità:}] affascinato, paralizzato, affaticato, spaventato
	\item[\textbf{Sensi:}] Scurovisione 60 m, Visione del vero 40 m
	\item[\textbf{Linguaggi:}] tutti
	\item[\textbf{Sfida:}] 30 (155000 PX)\smallskip
\end{description}

\emph{\textbf{Aura distruttiva.}} il drago emette nel raggio di 6 metri un aura che causa 1 danno da forza cumulativo per round di permanenza. Il danno si azzera dopo 1 ora di allontanamento.

\textbf{Immortale sulla Terra.} Quando il corpo di Tàhil viene ucciso sulla Terra questo si riforma in 3d6 giorni nella tana fatta da Calicante.

\emph{\textbf{Incantesimi.}} Tàhil ha CM 20. La sua caratteristica da incantatore è il Carisma. Tàhil conosce i seguenti incantesimi:

A volontà: Parola Divina

\emph{\textbf{Natura Divina.}} Tàhil non ha bisogno di aria, cibo, bevande o sonno. Gli incantesimi di 5 livello o inferiore non hanno effetto su Tàhil tranne se lo vuole.

\emph{\textbf{Padrone dei Draghi.}} Ogni Drago non di Ljust sulla Terra è fedele ed ubbidiente al volere di Tàhil.

\emph{\textbf{Voce del Padrone.}} Tàhil può dialogare con ogni drago di Tàhil presente in sulla Terra, indipendentemente dalla distanza.

\emph{\textbf{Richiamo del Padrone.}} Tàhil apre un portale ed escono 1d2+1 draghi di Tàhil di età e colore casuale. Il potere è usabile 1 volta al giorno.

\emph{\textbf{Resistenza Leggendaria (5/Giorno).}} Se il Tàhil fallisce un Tiro Salvezza, può scegliere invece di riuscirvi.

\emph{\textbf{più teste.}} Tàhil ha +1d6 ai Tiri Salvezza contro essere cieco, sordo, svenuto. Tàhil può eseguire fino a 6 Reazioni per round.

\emph{\textbf{Rigenerazione.}} Tàhil rigenera 30 Punti Ferita all'inizio del suo round

\textbf{Azioni}

\emph{\textbf{Multiattacco.}} Tàhil può usare la sua Presenza Spaventosa oppure effettuare 3 attacchi (2 con artigli ed uno con la coda) oppure uno solo con il morso. Artiglio +30, portata 5 metri. Coda +19 portata 8 metri. Morso +19, portata 6 metri. Tutti gli attacchi di Tàhil sono considerati magici +5.

\emph{Colpisce:} Artiglio, 24 (4d6 +10, 5/40 danni da sanguinamento) da taglio. Coda, 28 (4d8 +10) contundenti. Morso 48 (8d6 +10) tagliente. Se colpisce con un margine di 10 con il morso mozza il corpo a metà della creatura se non si riesce un TS su Tempra a DC 30.

\emph{\textbf{Presenza Spaventosa}} Ogni creatura che possa vedere Tàhil e sia entro 80 metri deve fare un Tiro Salvezza su Volontà a DC 45 o essere Spaventato per 1 minuto. Ogni round la creatura può effettuare il Tiro Salvezza, se questo riesce è immune alla Presenza Spaventosa di Tàhil per le successive 24 ore.

\textbf{Azioni Aggiuntive}

Il Tàhil può effettuare 3 azioni aggiuntive, scelte da quelle sottostanti ed una per round solo al termine del round di un altra creatura. Tàhil può cambiare il colore della sua testa per accedere ai poteri degli altri tipi di drago. Le azioni dipendono dalla testa scelta.

\textbf{Attacco con Artiglio.}: +19, portata 6 metri, un obiettivo. Se colpisce 32 (4d10 + 10, 3 da Sanguinamento) danno da taglio più 14 (4d6) danni da acido (testa Nera) oppure Elettricità (testa Blu) oppure da Veleno (testa Verde) oppure da Fuoco (testa Rossa) oppure da Freddo (testa Bianca) oppure da Fuoco (testa Gialla) oppure da Suono (testa Viola)

\textbf{Testa Nera.}: Costa 2 azioni Aggiuntive, Tàhil soffia Acido in un cono di 40 metri. Tiro Salvezza su Riflessi DC 40 o prendere 68 (15d8) di danno da acido oppure dimezzare.

\textbf{Testa Blu.}: Costa 2 azioni Aggiuntive, Tàhil soffia Elettricità in un cono di 40 metri. Tiro Salvezza su Riflessi DC 40 o prendere 88 (16d10) di danno da Elettricità oppure dimezzare.

\textbf{Testa Verde.}: Costa 2 azioni Aggiuntive, Tàhil soffia Veleno in un cono di 30 metri. Tiro Salvezza su Riflessi DC 40 o prendere 77 (22d6) di danno da Veleno oppure dimezzare.

\textbf{Testa Rossa.}: Costa 2 azioni Aggiuntive, Tàhil soffia Fuoco in un cono di 30 metri. Tiro Salvezza su Riflessi DC 40 o prendere 91 (26d6) di danno da Fuoco oppure dimezzare.

\textbf{Testa Bianca.}: Costa 2 azioni Aggiuntive, Tàhil soffia Ghiaccio in un cono di 30 metri. Tiro Salvezza su Riflessi DC 40 o prendere 72 (16d8) di danno da Ghiaccio oppure dimezzare.

\textbf{Testa Viola.}: Costa 2 azioni Aggiuntive, Tàhil soffia Suono in un cono di 30 metri. Tiro Salvezza su Riflessi DC 40 o prendere 90 (18d8) di danno da Suono oppure dimezzare.

\textbf{Testa Gialla.}: Costa 2 azioni Aggiuntive, Tàhil soffia sabbia rovente in un cono di 60 metri. Tiro Salvezza su Riflessi DC 40 o prendere 72 (16d8) di danno da Fuoco oppure dimezzare.

\textbf{Ecologia}\\
Ambiente: Sconosciuto\\
Organizzazione: Unico\\
\textbf{Categoria Tesoro}: 6 H\\
\textbf{Descrizione}\\
Tàhil è il Patrono dei Draghi incarnato. Nulla resiste alla sua furia, follia, rabbia e distruzione. Tàhil è una mastodontica creatura con 7 teste di drago, ognuna colorata in modo diverso, ognuna a rappresentare un colore di un Drago. Vedi capitolo sulla Cosmologia per i dettagli della sua storia.

\mostro{Drider}
\noindent
\begin{description}[noitemsep, topsep=0pt, parsep=0pt, partopsep=0pt, leftmargin=0cm, labelwidth=2.2cm]
	\item[\textbf{Taglia/Tipo:}] Grande mostruosità, malvagio
	\item[\textbf{Caratt.:}] \resizebox{0.5\linewidth+1.8cm}{!}{For 3 Des 3 Cos 4 Int 1 Sag 2 Car 1}
	\item[\textbf{Punti Ferita:}] 127,  \textbf{Difesa:} 23,  \textbf{Iniziativa:} +3
	\item[\textbf{Movimento:}] 9 m, scalata 9 m
	\item[\textbf{Tiri Salvez.:}] \resizebox{0.5\linewidth+1.8cm}{!}{\resizebox{0.5\linewidth+1.8cm}{!}{Tempra +10, Riflessi +9, Volontà +8}}
	\item[\textbf{Comp.:}] Furtività +9, Consapevolezza +5
	\item[\textbf{Sensi:}] Scurovisione 36 m
	\item[\textbf{Linguaggi:}] Elfico, Linguaggio delle Profondità
	\item[\textbf{Sfida:}] 6 (2300 PX)\smallskip
\end{description}

\emph{\textbf{Camminare sulla Tela.}} Il drider ignora le restrizioni al movimento provocate dalle ragnatele.

\emph{\textbf{Discendenza Fatata.}} Il drider ha +1d6 ai Tiri Salvezza per non restare affascinato e la magia non può far addormentare un drider.

\emph{\textbf{Incantesimi Innati.}} La caratteristica da incantatore innato del drider è la Saggezza. Il drider può lanciare in maniera innata i seguenti incantesimi, senza bisogno di componenti materiali:

A volontà: \emph{luci danzanti}

1/Giorno: \emph{luce diurna, \hyperlink{Oscurità}{Oscurità}}

\emph{\textbf{Scalare come Ragno.}} Il drider può scalare superfici difficili, compreso lo stare a testa in giù sul soffitto, senza bisogno di effettuare una prova di competenza.

\textbf{Azioni}

\emph{\textbf{Multiattacco.}} Il drider effettua tre attacchi con la spada lunga o con l'arco lungo. Può rimpiazzare uno di questi attacchi con un attacco di morso.

\emph{\textbf{Morso.} Attacco con arma da mischia}: +8 a colpire, portata 1 m, una creatura.

\emph{Colpisce:} 2 (1d4) danni perforanti più 9 (2d8) danni da veleno.

\emph{\textbf{Spada Lunga.} Attacco con arma da mischia}: +7 a colpire, portata 1 m, un bersaglio.

\emph{Colpisce:} 7 (1d8 + 3) danni taglienti, o 8 (1d8 + 3) danni taglienti se usata con due mani.

\emph{\textbf{Arco Lungo.} Attacco con arma a Distanza}: +9 a colpire, gittata 45m, un bersaglio.

\emph{Colpisce:} 7 (1d8 + 3) danni perforanti più 4 (1d8) danni da veleno.

\emph{\textbf{Ragnatela.}} il Drider usando 1 sola Azione lancia l'incantesimo \hyperlink{Ragnatela}{Ragnatela}.

\emph{\textbf{Arrabbiato:}} il Drider raccoglie la saliva velenosa e la sputa sulle sue armi. Fino alla fine del combattimento l'attacco di Spada Lunga causa anche 1d8 di danni da veleno. Costa 1 Azione.

\textbf{Ecologia}\\
Ambiente: Qualsiasi sotterraneo\\
Organizzazione: Solitario, coppia o gruppo (3-8)\\
\textbf{Categoria Tesoro}: Mazza flangiata Perfetta, Arco Lungo Composito Perfetto [Forza +2] con 20 Frecce, Y\\
\textbf{Descrizione}\\
Creato dal corpo di un elfo, alterato e mutato attraverso speciali veleni ed elisir per assumere le caratteristiche di un ragno gigante, il drider è una creatura pericolosa.\\
I drider sono sessualmente dimorfici. La parte inferiore da ragno del corpo di un drider femmina è lucente ed aggraziata, spesso simile al corpo di una vedova nera, mentre il busto superiore di elfo mantiene le sue curve allettanti e il bel viso (con l'eccezione delle venefiche zanne acuminate). La parte inferiore del corpo di un drider maschio è tozza come una tarantola, mentre quella superiore ha un fisico asciutto e supporta un'orrenda faccia più da ragno che da elfo, completa di mandibole zannute.


\begin{center}
	\includegraphics[width=0.9\linewidth]{immagini/James_Paterson_-_Dryade.png}

	\emph{Dryade, James Paterson}
\end{center}

\mostro{Driade}
\noindent
\begin{description}[noitemsep, topsep=0pt, parsep=0pt, partopsep=0pt, leftmargin=0cm, labelwidth=2.2cm]
	\item[\textbf{Taglia/Tipo:}] Media fatato, neutrale
	\item[\textbf{Caratt.:}] \resizebox{0.5\linewidth+1.8cm}{!}{For 0 Des 1 Cos 0 Int 2 Sag 2 Car 4}
	\item[\textbf{Punti Ferita:}] 33,  \textbf{Difesa:} 14,  \textbf{Iniziativa:} +2
	\item[\textbf{Movimento:}] 9 m
	\item[\textbf{Tiri Salvez.:}] \resizebox{0.5\linewidth+1.8cm}{!}{Tempra +3, Riflessi +3, Volontà +3}
	\item[\textbf{Comp.:}] Furtività +5, Consapevolezza +4
	\item[\textbf{Sensi:}] Scurovisione 18
	\item[\textbf{Linguaggi:}] Elfico, Silvano
	\item[\textbf{Sfida:}] 1 (200 PX)\smallskip
\end{description}

\emph{\textbf{Camminata Arborea.}} Uno volta durante il suo round, la driade può usare 1 Azione per entrare magicamente in un albero vivo a sua portata

ed emergere da un altro albero vivo entro 18 metri dal primo albero, ricomparendo in uno spazio non occupato entro 1 metro dal secondo albero. Entrambi gli alberi devono essere di taglia Grande o superiore.

\emph{\textbf{Incantesimi Innati.}} La caratteristica da incantatore innato della driade è il Carisma (DC 14 per i Tiri Salvezza degli incantesimi). La driade può lanciare in maniera innata i seguenti incantesimi, senza aver bisogno di componenti materiali. A volontà:

\emph{\hyperlink{Artificio Druidico}{Artificio Druidico}}

3/giorno ciascuno: \emph{\hyperlink{Bacche Benefiche}{Bacche Benefiche}, \hyperlink{Intralciare}{Intralciare}}

1/giorno: \emph{\hyperlink{Passare Senza Tracce}{Passare Senza Tracce}, \hyperlink{Pelle di Corteccia}{Pelle di Corteccia}, \hyperlink{Randello incantato}{Randello Incantato}}

\emph{\textbf{Parlare con Animali e Piante.}} La driade può comunicare con bestie e piante come se parlassero la stessa lingua.

\emph{\textbf{Resistenza alla Magia.}} La driade ha +1d6 ai Tiri Salvezza contro incantesimi e altri effetti magici.


\textbf{Azioni}

\emph{\textbf{Randello.} Attacco con arma da mischia}: +4 a colpire (+6 a colpire con il randello incantato), portata 1 m, un bersaglio.

\emph{Colpisce:} 2 (1d4) danni contundenti, o 8 (1d8 + 4) danni contundenti con randello incantato

\emph{\textbf{Fascino Fatato.}} La driade può prendere a bersaglio un umanoide o bestia entro 9 metri da lei e che possa vedere. Se il bersaglio può vedere la driade, deve riuscire un Tiro Salvezza su Volontà DC 14 o restare affascinato dalla magia. Le creature affascinate considerano la driade un'amica fidata da ascoltare e proteggere. Sebbene il bersaglio non sia sotto il controllo della driade, interpreterà le richieste o le azioni della driade nel modo più favorevole possibile.

Ogni volta che la driade o i suoi alleati arrecano danno al bersaglio, esso può ripetere il Tiro Salvezza, terminando l'effetto in caso di successo. Altrimenti, l'effetto permane 24 ore o finché la driade muore, si trova su di un piano di esistenza diverso rispetto al bersaglio, o termina l'effetto con un'Azione Immediata.

Se il Tiro Salvezza del bersaglio riesce, il bersaglio sarà immune al Fascino Fatato della driade per le successive 24 ore.

La driade non può tenere affascinati più di un umanoide o tre bestie alla volta.

\textbf{Ecologia}\\
Ambiente: Foreste Temperate\\
Organizzazione: Solitario, coppia o boschetto (3-8)\\
\textbf{Categoria Tesoro}: Arco Lungo Perfetto con 20 Frecce, Pugnale,D\\
\textbf{Descrizione}\\
Le driadi sono spiriti della natura che amano i boschi appartati lontani dagli umanoidi bisognosi di legname. L'interesse principale delle driadi è la propria sopravvivenza e quella delle adorate foreste e sono note per costringere magicamente i viaggiatori ad aiutarle in quei compiti che non possono espletare.
Sono amichevoli con druidi e guardiaboschi non malvagi, dato che riconoscono la loro empatia o il loro rispetto per la natura.
Le driadi sono benevole guardiane degli alberi, e sebbene non siano violente di natura, possono bloccare e sventare le minacce alle loro dimore o trasformare i nemici in alleati.

%\addcontentsline{toc}{subsubsection}{E}
\pdfbookmark[3]{E}{E}

\mostro{Elementale dell'Acqua Generico}
\noindent


\begin{description}[noitemsep, topsep=0pt, parsep=0pt, partopsep=0pt, leftmargin=0cm, labelwidth=2.2cm]
	\item[\textbf{Taglia/Tipo:}] Elementale
	\item[\textbf{Caratt.:}] For 2+GS/3 Des 0+GS/6 Cos 2+GS/3 Int -2+GS/6 Sag 0+GS/6 Car 0+GS/6
	\item[\textbf{Punti Ferita:}] (GS+2)*15, \textbf{Difesa:} GS+Des, \textbf{Iniziativa:} +Des
	\item[\textbf{Movimento:}] 9 m, nuoto GS*4 m
	\item[\textbf{Tiri Salvez.:}] Tempra GS+GS/5+COS, Riflessi GS+DES, Volontà GS+SAG
	\item[\textbf{Res. Danni:}] Acido; da arma non magica
	\item[\textbf{Imm. Danni:}] Veleno
	\item[\textbf{Immunità:}] afferrato, intralciato, paralizzato, pietrificato, privo di sensi, prono, affaticato
	\item[\textbf{Sensi:}] Scurovisione 18 m
	\item[\textbf{Linguaggi:}] Aquan
	\item[\textbf{Sfida:}] GS \\
\end{description}

\emph{\textbf{Congelamento.}} Se l'elementale subisce danno da freddo, gela parzialmente; il suo movimento è ridotto di 6 metri fino al termine del suo prossimo round.\\
\emph{\textbf{Forma d'Acqua.}} L'elementale può entrare nello spazio di una creatura ostile e fermarsi lì. Può muoversi attraverso uno spazio stretto fino a 3 centimetri senza doversi stringere.\\
\emph{\textbf{Natura Elementale.}} Un elementale non ha bisogno di aria,cibo, bevande o sonno.\\
\textbf{Azioni}\\
\emph{\textbf{Multiattacco.}} L'elementale effettua due attacchi di schianto.\\
\emph{\textbf{Schianto.} Attacco con arma da mischia}: +GS/2+FOR a colpire, portata GS/3 metri, un bersaglio.\\
\emph{Colpisce:} GS*1d8 danni contundenti.\\


\begin{center}
	\includegraphics[width=0.9\linewidth]{immagini/geyser.png}
\end{center}

\textbf{Reazione: \emph{Attacco d'opportunità}}: l'elementale effettua un attacco ad una creatura che attraversi o esca dalla sua portata di GS/3 metri.

\emph{\textbf{Sommergere (Ricarica 4-6).}} Ogni creatura nello spazio dell'elementale deve effettuare un Tiro Salvezza di Tempra DC 10+GS+GS/5. Se lo fallisce, il bersaglio subisce (1d8+1)*GS/2 danni contundenti. Se è di taglia GS/3 >=4, il bersaglio è anche afferrato (DC CR*2 per fuggire). Fino al termine dell'afferrare, il bersaglio non può respirare a meno che non sia in grado di respirare acqua. Se il Tiro Salvezza riesce, il bersaglio viene spinto fuori dallo spazio dell'elementale.\\
L'elementale può afferrare una creatura di taglia GS/3 oppure 2 di GS/2 oppure. All'inizio di ciascun round dell'elementale, ogni bersaglio afferrato subisce (1d6)*GS/2 danni contundenti. Una creatura entro 3 metri dall'elementale può trascinare fuori da esso una creatura o oggetto, impiegando un'Azione per tentare di riuscire una prova di Tiro Salvezza Tempra con Forza DC 2+GS*2.\\

\begin{center}
	\includegraphics[width=0.9\linewidth]{immagini/tornado_Elie_Manitoba_2007.png}
\end{center}

\mostro{Elementale dell'Aria Generico}
\noindent
\begin{description}[noitemsep, topsep=0pt, parsep=0pt, partopsep=0pt, leftmargin=0cm, labelwidth=2.2cm]
	\item[\textbf{Taglia/Tipo:}] GS/3 (Piccola, Media, Grande, Enorme, Mastodontico, Colossale)
	\item[\textbf{Caratt.:}] For 0+GS/6 Des 3+GS/3 Cos 0+GS/6 Int -2+GS/6 Sag -1+GS/6 Car 0+GS/6
	\item[\textbf{Punti Ferita:}] (GS+1)*15, \textbf{Difesa:} GS+Des+2, \textbf{Iniziativa:} +Des
	\item[\textbf{Movimento:}] 0 m, volare GS*4 m
	\item[\textbf{Tiri Salvez.:}] Tempra GS+COS, Riflessi GS+GS/5 + DES, Volontà GS+SAG
	\item[\textbf{Res. Danni:}] Elettricità, Suono; da arma non magica
	\item[\textbf{Imm. Danni:}] Veleno
	\item[\textbf{Immunità:}] afferrato, intralciato, paralizzato, pietrificato, privo di sensi, prono, affaticato
	\item[\textbf{Sensi:}] Scurovisione 18 m
	\item[\textbf{Linguaggi:}] Ictun
	\item[\textbf{Sfida:}] GS \\
\end{description}

\emph{\textbf{Forma d'Aria.}} L'elementale può entrare nello spazio di una creatura ostile e fermarsi lì. Può muoversi attraverso uno spazio stretto fino a 3 centimetri senza doversi stringere.\\
\emph{\textbf{Natura Elementale.}} Un elementale non ha bisogno di aria, cibo, bevande o sonno.\\
\textbf{Azioni}\\
\emph{\textbf{Multiattacco.}} L'elementale effettua due attacchi di schianto.\\
\emph{\textbf{Schianto.} Attacco con arma da mischia}: +GS/2+FOR a colpire, portata GS/3 metri, un bersaglio.\\
\emph{Colpisce:} 1d6*GS/3 danni contundenti.\\

\textbf{Reazione: \emph{Attacco d'opportunità}}: l'elementale effettua un attacco ad una creatura che attraversi o esca dalla sua portata di GS/3 metri.

\emph{\textbf{Turbine (Ricarica 4-6).}} Ogni creatura nello spazio dell'elementale deve effettuare un Tiro Salvezza di Tempra DC 10+GS*1.5. Se lo fallisce, il bersaglio subisce 1d8*GS/3 danni contundenti e viene scagliato a GS metri di distanza dall'elementale in una direzione casuale e cadere prono. Se un bersaglio lanciato colpisce un oggetto, come un muro o il pavimento, subisce 3 (1d6) danni contundenti per ogni 3 metri per cui è stato lanciato. Se il bersaglio viene lanciato contro un'altra creatura, quella creatura deve riuscire un Tiro Salvezza di Riflessi DC 13 o subire lo stesso danno e cadere prona.
Se il Tiro Salvezza riesce, il bersaglio subisce la metà del danno contundente e non viene scagliato via né cade prono.

\begin{center}
	%\includegraphics[width=0.9\linewidth]{immagini/elefuoco.png}
	\includegraphics[width=0.7\linewidth]{immagini/wildfire_grayscale.png}
\end{center}

\mostro{Elementale del Fuoco Generico}
\noindent
\begin{description}[noitemsep, topsep=0pt, parsep=0pt, partopsep=0pt, leftmargin=0cm, labelwidth=2.2cm]
	\item[\textbf{Taglia/Tipo:}] GS/3 (Piccola, Media, Grande, Enorme, Mastodontico, Colossale)
	\item[\textbf{Caratt.:}] For 0+GS/3 Des 2+GS/3 Cos 1+GS/6 Int -2+GS/6 Sag -1+GS/6 Car -2+GS/6
	\item[\textbf{Punti Ferita:}] (GS+2)*15, \textbf{Difesa:} GS+1+Des, \textbf{Iniziativa:} +Des
	\item[\textbf{Movimento:}] 15 m
	\item[\textbf{Tiri Salvez.:}] Tempra GS+COS, Riflessi GS+DES, Volontà GS+SAG
	\item[\textbf{Res. Danni:}] da arma non magica
	\item[\textbf{Imm. Danni:}] Fuoco, Veleno
	\item[\textbf{Immunità:}] afferrato, intralciato, paralizzato, pietrificato, privo di sensi, prono, affaticato
	\item[\textbf{Sensi:}] Scurovisione 18 m
	\item[\textbf{Linguaggi:}] Ignan
	\item[\textbf{Sfida:}] GS \\
\end{description}

\emph{\textbf{Forma di Fuoco.}} L'elementale può spostarsi attraverso uno spazio fino a 3 centimetri di larghezza senza stringersi. Una creatura che entri a contatto o colpisca l'elementale con un attacco da mischia mentre si trova entro 1 metro da esso subisce 5 (1d10) danni da fuoco. Inoltre, l'elementale può entrare nello spazio di una creatura ostile e fermarsi lì. La prima volta che entra nello spazio di una creatura in un round, la creatura subisce GS danni da fuoco e prende fuoco; finché qualcuno non impiega un'Azione per spegnere le fiamme, la creatura subirà GS danni da fuoco all'inizio di ciascun proprio round.\\

\emph{\textbf{Illuminazione.}} L'elementale emette luce intensa in un raggio di GS*2 metri e luce fioca per GS*4 metri.\\
\emph{\textbf{Natura Elementale.}} Un elementale non ha bisogno di aria, cibo, bevande o sonno.\\
\emph{\textbf{Suscettibilità all'Acqua.}} L'elementale subisce 1 danno da freddo per ogni 1 metro che si muove in acqua o per ogni 4 litri d'acqua che gli vengono spruzzati addosso.
\textbf{Azioni}\\
\emph{\textbf{Multiattacco.}} L'elementale effettua due attacchi di contatto.\\
\emph{\textbf{Schianto.} Attacco con arma da mischia}: +GS/2+FOR a colpire, portata GS/3 metri, un bersaglio.\\
\emph{Colpisce:} GS*2 danni da fuoco. Se il bersaglio è una creatura o un oggetto infiammabile, prende fuoco. Finché una creatura non impiega un'Azione per spegnere le fiamme, la creatura subirà CR danni da fuoco all'inizio di ciascun proprio round.

\textbf{Reazione: \emph{Attacco d'opportunità}}: l'elementale effettua un attacco ad una creatura che attraversi o esca dalla sua portata di GS/3 metri.


\begin{center}
	\includegraphics[width=0.6\linewidth]{immagini/eleterra.png}
\end{center}

\mostro{Elementale della Terra Generico}
\noindent
\begin{description}[noitemsep, topsep=0pt, parsep=0pt, partopsep=0pt, leftmargin=0cm, labelwidth=2.2cm]
	\item[\textbf{Taglia/Tipo:}] GS/3 (Piccola, Media, Grande, Enorme, Mastodontico, Colossale)
	\item[\textbf{Caratt.:}] For GS Des -2+GS/6 Cos 1+GS/3 Int -3+GS/6 Sag -1+GS/6 Car -3+GS/6
	\item[\textbf{Punti Ferita:}] (GS+3)*15, \textbf{Difesa:} GS+Des, \textbf{Iniziativa:} +Des
	\item[\textbf{Movimento:}] 9 m, arrampicarsi 9 m, scavo 9 m
	\item[\textbf{Tiri Salvez.:}] Tempra GS+COS+GS/5, Riflessi GS+DES, Volontà GS+SAG
	\item[\textbf{Res. Danni:}] da arma non magica
	\item[\textbf{Imm. Danni:}] Veleno, Suono
	\item[\textbf{Immunità:}] afferrato, intralciato, paralizzato, pietrificato, privo di sensi, prono, affaticato
	\item[\textbf{Sensi:}] percezione tellurica 18 m, Scurovisione 18 m
	\item[\textbf{Linguaggi:}] Tremun
	\item[\textbf{Sfida:}] GS \\
\end{description}

\emph{\textbf{Mostro d'Assedio.}} L'elementale infligge danni doppi agli oggetti e le strutture.\\
\emph{\textbf{Natura Elementale.}} Un elementale non ha bisogno di aria, cibo, bevande o sonno.\\
\emph{\textbf{Planata Terrestre.}} L'elementale può scavare attraversa la terra e la pietra non magiche e non lavorate. Quando lo fa, l'elementale non disturba il materiale che sposta.\\
\textbf{Azioni}\\
\emph{\textbf{Multiattacco.}} L'elementale effettua due attacchi di schianto.\\
\emph{\textbf{Schianto.} Attacco con arma da mischia}: +GS a colpire, portata GS/6 metri, un bersaglio.\\
\emph{Colpisce:} GS*3 danni contundenti.

\textbf{Reazione: \emph{Attacco d'opportunità}}: l'elementale effettua un attacco ad una creatura che attraversi o esca dalla sua portata di GS/3 metri.


\mostro{Ettercap}
\noindent
\begin{description}[noitemsep, topsep=0pt, parsep=0pt, partopsep=0pt, leftmargin=0cm, labelwidth=2.2cm]
	\item[\textbf{Taglia/Tipo:}] Media mostruosità, malvagio
	\item[\textbf{Caratt.:}] \resizebox{0.5\linewidth+1.8cm}{!}{For 2 Des 2 Cos 1 Int -2 Sag 1 Car -2}
	\item[\textbf{Punti Ferita:}] 51,  \textbf{Difesa:} 16,  \textbf{Iniziativa:} +2
	\item[\textbf{Movimento:}] 9 m, scalata 9 m
	\item[\textbf{Tiri Salvez.:}] \resizebox{0.5\linewidth+1.8cm}{!}{Tempra +3, Riflessi +4, Volontà +3}
	\item[\textbf{Comp.:}] Furtività +4, Consapevolezza +3, Sopravvivenza +3
	\item[\textbf{Sensi:}] Scurovisione 18 m
	\item[\textbf{Sfida:}] 2 (450 PX)\smallskip
\end{description}

\emph{\textbf{Camminare sulla Tela.}} L'ettercap ignora le restrizioni al movimento provocate dalle ragnatele.

\emph{\textbf{Scalare come Ragno.}} L'ettercap può scalare superfici difficili, compreso lo stare a testa in giù sul soffitto, senza bisogno di effettuare una prova.

\emph{\textbf{Senso della Tela.}} Mentre è in contatto con una ragnatela, l'ettercap sa l'esatta posizione di qualsiasi altra creatura in contatto con la stessa ragnatela.

\textbf{Azioni}

\emph{\textbf{Multiattacco.}} L'ettercap effettua due attacchi: uno con il morso e uno con gli artigli

\emph{\textbf{Artigli.} Attacco con arma da mischia}: +6 a colpire, portata 1 m, un bersaglio.

\emph{Colpisce:} 7 (2d4 + 2) danni taglienti, 1 danno da Sanguinamento.

\emph{\textbf{Morso.} Attacco con arma da mischia}: +6 a colpire, portata 1 m, un bersaglio.

\emph{Colpisce:} 6 (1d8 + 2) danni perforanti più 4 (1d8) danni da veleno. Il bersaglio deve riuscire un Tiro Salvezza di Tempra DC 11 o restare avvelenato, -1 Forza e Destrezza, per 1 minuto. La creatura può ripetere il Tiro Salvezza al termine di ciascun suo round, terminando l'effetto se riesce il Tiro Salvezza.

\emph{\textbf{Ragnatela (Ricarica 5-6).} Attacco con arma a Distanza}: +5 a colpire, gittata 9m, una creatura di taglia Grande o minore.

\emph{Colpisce:} La creatura è intralciata dalla ragnatela. Con un'Azione, la creatura intralciata può effettuare un Tiro Salvezza Tempra con Forza DC 11, liberandosi dalla tela se la riesce. L'effetto termina se la tela è distrutta. La tela ha Difesa 10, 5 Punti Ferita, vulnerabilità ai danni da fuoco, e immunità ai danni contundenti e da veleno.

\textbf{Ecologia}\\
Ambiente: Foreste Temperate\\
Organizzazione: solitario, coppia o nido (3-6 più 2-8 ragni giganti)\\
\textbf{Categoria Tesoro}: C\\
\textbf{Descrizione}\\
Gli ettercap sono umanoidi alti di solito 1,8 metri e pesano circa 100 kg, con braccia allungate fino a terra ed un orrendo volto con elementi ragneschi. Sono solitari e raramente si uniscono ad altri della loro razza, tranne per l'accoppiamento. Quando fanno gruppo, tendono ad attrarre varie specie di ragni, formando uno strano connubio di ettercap e aracnidi.\\
Gli ettercap sono noti per la costruzione di astute trappole fatte di ragnatele e altri materiali naturali, che usano per catturare prede. Costruiscono rifugi di ragnatela, tra i rami più alti gli alberi lontano dagli altri predatori terrestri, e usano ragni mostruosi come vedette e guardiani.\\
Gli ettercap non sono coraggiosi, ma le loro trappole spesso impediscono al nemico di estrarre le armi. Un ettercap attacca con artigli e morsi velenosi. In genere evita la mischia con gli avversari che possono ancora muoversi e fugge se si liberano.

\mostro{Ettin}
\noindent
\begin{description}[noitemsep, topsep=0pt, parsep=0pt, partopsep=0pt, leftmargin=0cm, labelwidth=2.2cm]
	\item[\textbf{Taglia/Tipo:}] Grande gigante, malvagio
	\item[\textbf{Caratt.:}] \resizebox{0.5\linewidth+1.8cm}{!}{For 5 Des -1 Cos 3 Int -2 Sag 0 Car -1}
	\item[\textbf{Punti Ferita:}] 89,  \textbf{Difesa:} 16,  \textbf{Iniziativa:} -1
	\item[\textbf{Movimento:}] 12 m
	\item[\textbf{Tiri Salvez.:}] \resizebox{0.5\linewidth+1.8cm}{!}{Tempra +7, Riflessi +3, Volontà +4}
	\item[\textbf{Comp.:}] Consapevolezza +4
	\item[\textbf{Sensi:}] percezione tellurica 18 m
	\item[\textbf{Linguaggi:}] Gigante, Goblinoide
	\item[\textbf{Sfida:}] 4 (1100 PX)\smallskip
\end{description}

\emph{\textbf{Due Teste.}} L'ettin ha +1d6 alle prove di Consapevolezza e sui Tiri Salvezza contro le condizioni accecato, affascinato, assordato, privo di sensi, spaventato e stordito.

\emph{\textbf{Veglia.}} Quando una delle due teste dell'ettin è addormentata, l'altra è sveglia.

\textbf{Azioni}

\emph{\textbf{Multiattacco.}} L'ettin effettua due attacchi: uno con l'ascia da battaglia e uno con la mazza chiodata.

\emph{\textbf{Ascia da Battaglia.} Attacco con arma da mischia}: +7 a colpire, portata 1 m, un bersaglio.

\emph{Colpisce:} 14 (2d8 + 5) danni taglienti.

\emph{\textbf{Mazza Chiodata.} Attacco con arma da mischia}: +7 a colpire, portata 1 m, un bersaglio.

\emph{Colpisce:} 14 (2d8 + 5) danni perforanti.

\textbf{Ecologia}\\
Ambiente: Colline fredde\\
Organizzazione: Solitario, coppia, gruppo (3-6), truppa (1-2 più 1-2 Orsi Bruni, banda (3-6 più 1-2 Orsi Bruni) o colonia (3-6 più 1-2 Orsi Bruni e 7-12 Orchi, o 9-16 Goblin)\\
\textbf{Categoria Tesoro}: Armatura di Cuoio, 2 Mazzafrusti Leggeri, 4 Giavellotti, P\\
\textbf{Descrizione}\\
Gli ettin, o giganti a due teste, sono cacciatori notturni malevoli e imprevedibili. Le due teste gli concedono impareggiabili poteri di percezione, facendone dei guardiani eccellenti.\\
Gli ettin sembrano Giganti di Collina o Giganti di Pietra, ma il volto zannuto tradisce una discendenza orchesca. Hanno pelle marrone rosata e non si lavano mai se non vi sono costretti, cosa che li rende così sporchi e sudici che la loro pelle sembra spessa e grigia.\\
Gli adulti sono alti 3,9 metri e pesano 2.600 kg. Gli ettin vivono circa 75 anni.\\
Gli ettin non hanno un loro linguaggio ma parlano un gergo misto di Gigante, Goblin e Orchesco. Le creature che parlano uno qualsiasi di questi linguaggi possono comunicare con un ettin effettuando una prova di Intelligenza con DC 15. La prova si effettua una volta per ogni frammento di informazione; se l'altra creatura parla due di questi linguaggi la DC è 10, mentre per qualcuno che li parla tutti e tre è 5.\\
Sebbene gli ettin non siano molto intelligenti, sono guerrieri astuti. Preferiscono tendere imboscate alle loro vittime anziché ingaggiarle in combattimento, ma una volta che la battaglia è cominciata, un ettin combatte furiosamente fino alla morte del nemico.\\
Gli ettin sono creature solitarie, si stabiliscono nella sicurezza di cave rocciose e cavità, spesso circondate da buche e fossi, e tengono a volte degli orsi delle caverne come animali da compagnia o guardiani.\\
Un ettin particolarmente potente può attrarre un gruppo di seguaci, specie Goblin o Orchi. Comunque, questi assembramenti sono più che altro delle eccezioni, e raramente durano a lungo, con gli individualisti ettin che vanno per la loro strada appena le opportunità di saccheggio e rapina diminuiscono o se il capo viene ucciso.\\
In genere formano delle coppie riproduttive per allevare la prole solo per brevi periodi prima di riprendere ognuno la propria strada. I giovani ettin maturano rapidamente, raggiungendo la taglia adulta in un anno, potendo così provvedere a se stessi.

%\addcontentsline{toc}{subsubsection}{F}
\pdfbookmark[3]{F}{F}

\mostro{Fantasma}
\noindent
\begin{description}[noitemsep, topsep=0pt, parsep=0pt, partopsep=0pt, leftmargin=0cm, labelwidth=2.2cm]
	\item[\textbf{Taglia/Tipo:}] Media non morto, qualsiasi tratto
	\item[\textbf{Caratt.:}] \resizebox{0.5\linewidth+1.8cm}{!}{For -2 Des 1 Cos 0 Int 0 Sag 1 Car 3}
	\item[\textbf{Punti Ferita:}] 87,  \textbf{Difesa:} 18,  \textbf{Iniziativa:} +1
	\item[\textbf{Movimento:}] 0 m, volo 12 m, Fluttuare
	\item[\textbf{Tiri Salvez.:}] \resizebox{0.5\linewidth+1.8cm}{!}{Tempra +4, Riflessi +5, Volontà +5}
	\item[\textbf{Res. Danni:}] Acido, Elettricità, Fuoco, Suono, Veleno; da armi non magiche
	\item[\textbf{Immunità:}] affascinato, afferrato, intralciato, paralizzato, pietrificato, prono, affaticato, spaventato, sanguinamento
	\item[\textbf{Sensi:}] Scurovisione 18 m
	\item[\textbf{Linguaggi:}] qualsiasi lingua conosciuta in vita, Expiran
	\item[\textbf{Sfida:}] 4 (1100 PX)\smallskip
\end{description}

\emph{\textbf{Movimento Incorporeo.}} Il fantasma può attraversare altre creature e oggetti come se fossero terreno difficile. Subisce 5 (1d10) danni da forza se termina il suo round all'interno di un oggetto.

\emph{\textbf{Natura Non Morta.}} Il fantasma non ha bisogno di aria, cibo, bevande o di dormire.

\emph{\textbf{Vista Eterea.}} Il fantasma può vedere 18 metri nel Piano Etereo quando si trova sul Piano Materiale, e vice versa.

\textbf{Azioni}

\emph{\textbf{Tocco Avvizzente.} Attacco con arma da mischia}: +5 a colpire, portata 1 m, un bersaglio.

\emph{Colpisce:} 17 (4d6 + 3) danni da Vuoto. Il bersaglio deve fare un Tiro Salvezza su Tempra a DC 15 o divenire Affaticato.

\emph{\textbf{Eterealità.}} Il fantasma entra nel Piano Etereo dal Piano Materiale, o vice versa. È visibile sul Piano Materiale mentre è nel Piano Etereo, e vice versa, ma non può interagire con nulla che si trovi sull'altro piano.

\emph{\textbf{Possessione (Ricarica 6).}} Un umanoide, entro 1 metro e visibile al fantasma, deve riuscire un Tiro Salvezza di Volontà DC 15 o venire posseduto dal fantasma; il fantasma poi scompare, e il bersaglio è inabile e perde il controllo del suo corpo. Il fantasma ora controlla il corpo ma non priva il bersaglio della sua consapevolezza. Il fantasma non può essere bersaglio di attacchi, incantesimi, o altri effetti, eccetto quelli che scacciano i non morti, e mantiene i suoi Tratti, Intelligenza, Saggezza, Carisma e immunità all'essere affascinato e spaventato. Per il resto usa altrimenti le statistiche del bersaglio posseduto, ma non accede al sapere e competenze del bersaglio.

La possessione dura finché il corpo scende a 0 Punti Ferita, il fantasma la termina con un'Azione Immediata, o il fantasma viene scacciato o espulso. Quando la possessione termina, il fantasma riappare in uno spazio non occupato entro 1 metro dal corpo. Il bersaglio è immune alla Possessione di questo fantasma per 24 ore dopo aver riuscito il Tiro Salvezza o al termine della possessione.

\emph{\textbf{Viso Orripilante.}} Ogni creatura che non sia non morta, entro 18 metri dal fantasma e che lo possa vedere, deve riuscire un Tiro Salvezza di Volontà DC 15 o essere spaventata per 1 minuto. Se il Tiro Salvezza fallisce di 5 o più, il bersaglio invecchia anche di 1d4 x 10 anni. Un bersaglio spaventato può ripetere il Tiro Salvezza al termine di ciascun proprio round, terminando l'effetto per sé, qualora riuscisse il Tiro Salvezza. Se il Tiro Salvezza del bersaglio riesce e per lui l'effetto ha fine, il bersaglio è immune al Viso Orripilante del fantasma per le successive 24 ore. Tramite l'incantesimo \hyperlink{Ristorare Superiore}{Ristorare Superiore} si può recuperare 1 anno di invecchiamento, ma solo se eseguito entro 24 dall'effetto di invecchiamento.

\textbf{Ecologia}
Ambiente: qualsiasi\\
Organizzazione: solitario\\
\textbf{Categoria Tesoro}: Nessuno\\
\textbf{Descrizione}\\
Quando ad un'anima non è concesso il riposo a causa di qualche grave ingiustizia, vera o presunta, a volte essa torna come fantasma. Questi esseri sono eternamente angosciati, privi di sostanza e incapaci di rimettere le cose a posto. Sebbene i fantasmi possano avere qualsiasi Tratto, molti si aggrappano al mondo dei viventi con un forte senso di odio e rabbia, e come risultato diventano malvagi; anche una creatura buona dopo morta può diventare un fantasma odioso e crudele.

più di altri mostri, il fantasma deve avere un background ben delineato. Perché questo personaggio è diventato un fantasma? Quali leggende lo circondano? Un incontro con un fantasma non dovrebbe mai avvenire in modo accidentale: ci sono molti altri non morti incorporei, come Wraith e Spettri, per questo. Un incontro adeguato con un fantasma dovrebbe avvenire in una scena al culmine di un lungo periodo di tensione costruito con servitori minori o manifestazioni di spiriti non morti. L'esempio di fantasma sopra rappresenta una principessa umana assassinata da un amante infedele; dopo un confronto, lui la legò con delle catene e la gettò nel pozzo del castello, dove morì annegata. Le capacità del fantasma sono state selezionate in base al background, mostrando come si possa creare un potente antagonista. Applicando l'archetipo a creature con livelli e quindi Abilità proprie o con capacità razziali significative si possono creare fantasmi molto più potenti.

Quando viene creato un fantasma, questi ottiene le copie degli oggetti a cui in vita dava particolare valore (a condizione che gli originali non siano in possesso di altre creature). L'equipaggiamento funziona normalmente per il fantasma ma passa attraverso gli oggetti o le creature materiali. Un'arma +1 o con un potenziamento superiore, tuttavia, può danneggiare le creature materiali. Un fantasma può usare scudi e armature solo se hanno la capacità Tocco Fantasma.

Gli oggetti originali vengono lasciati indietro, proprio come le spoglie fisiche del fantasma. Se un'altra creatura impugna l'originale, la copia incorporea svanisce. Questa perdita fa inevitabilmente infuriare il fantasma, che non si ferma davanti a nulla per riportare l'oggetto nel posto in cui giaceva originariamente (e riguadagnarne l'utilizzo).

\mostro{Fauci Gorgoglianti}
\noindent
\begin{description}[noitemsep, topsep=0pt, parsep=0pt, partopsep=0pt, leftmargin=0cm, labelwidth=2.2cm]
	\item[\textbf{Taglia/Tipo:}] Media aberrazione, neutrale
	\item[\textbf{Caratt.:}] \resizebox{0.5\linewidth+1.8cm}{!}{For 0 Des -1 Cos 3 Int -4 Sag 0 Car -2}
	\item[\textbf{Punti Ferita:}] 52,  \textbf{Difesa:} 13,  \textbf{Iniziativa:} -1
	\item[\textbf{Movimento:}] 3 m, nuoto 3 m
	\item[\textbf{Tiri Salvez.:}] \resizebox{0.5\linewidth+1.8cm}{!}{Tempra +5, Riflessi +3, Volontà +3}
	\item[\textbf{Immunità:}] prono
	\item[\textbf{Sensi:}] Scurovisione 18 m
	\item[\textbf{Sfida:}] 2 (450 PX)\smallskip
\end{description}

\emph{\textbf{Gorgoglio.}} Finché la fauce è in grado di vedere una creatura e non è inabile, pronuncia frasi incoerenti. Ogni creatura che inizi il suo round entro 6 metri dalla fauce e può udire il suo gorgoglio deve effettuare un Tiro Salvezza di Volontà DC 12. Se lo fallisce, la creatura non può effettuare reazioni fino all'inizio del suo prossimo round e tira un d8 per determinare cosa farà durante il proprio round. Da 1 a 4, la creatura non fa nulla. Con 5 o 6, la creatura non svolge nessun'Azione o Reazione e usa tutto il suo movimento per muoversi in una direzione determinata casualmente. Con 7 o 8, la creatura effettua un attacco da mischia contro una creatura determinata a caso entro la sua portata o non fa nulla se non è in grado di effettuare un simile attacco.

\emph{\textbf{Terreno Aberrante.}} Il terreno in un raggio di 3 metri intorno alla fauce è considerato terreno difficile. Ogni creatura che inizi il suo round in quell'area deve riuscire un Tiro Salvezza di Tempra DC 11 o vedere il suo movimento ridotto a 0 fino all'inizio del suo round successivo.

\textbf{Azioni}

\emph{\textbf{Multiattacco.}} La fauce gorgogliante effettua un attacco di morso e, se può, uno Sputo Accecante.

\emph{\textbf{Morso.} Attacco con arma da mischia}: +4 a colpire, portata 1 m, una creatura.

\emph{Colpisce:} 17 (5d6) danni perforanti. Se il bersaglio è di taglia Media o inferiore, deve riuscire un Tiro Salvezza di Tempra DC 11 o venir gettato prono. Se il bersaglio viene ucciso da questo danno, viene assorbito dalla fauce.

\emph{\textbf{Sputo Accecante (Ricarica 5-6).}} La fauce sputa un globo chimico ad un punto visibile entro 5 metri da essa. Il globo esplode all'impatto in un lampo accecante di luce. Ogni creatura entro 1 metro dal lampo deve riuscire un Tiro Salvezza di Riflessi DC 13 o restare accecata fino al termine del prossimo round della fauce.

\textbf{Reazione: \emph{Sputo opportunistico}} la fauce, quando colpita con un danno critico sputa un globo acido alla creatura che l'ha ferita causando 2d6 di danno da acido.

\textbf{Ecologia}\\
Ambiente: Qualsiasi Sotterraneo\\
Organizzazione: Solitario\\
\textbf{Categoria Tesoro}: accidentale (O)\\
\textbf{Descrizione}\\
Disgustosa, nauseante e affamata: queste sono le uniche parole che descrivono in modo appropriato la fauce gorgogliante. Bestie ripugnanti che si nascondono nelle grotte, nelle fogne e negli incubi, le fauci non hanno altro senso sociale, ecologico o religioso diverso dalla loro capacità di far impazzire coloro che le ascoltano. Alcuni studiosi credono che le fauci gorgoglianti siano una variante più piccola del molto più pericoloso shoggoth, mentre altri teorizzano che sia una punizione di Orudjs inflitta a coloro che l'hanno offesa.

\mostro{Fenice}
\noindent
\begin{description}[noitemsep, topsep=0pt, parsep=0pt, partopsep=0pt, leftmargin=0cm, labelwidth=2.2cm]
	\item[\textbf{Taglia/Tipo:}] Mastodontica celestiale, Coraggioso, Protettivo, Buono
	\item[\textbf{Caratt.:}] \resizebox{0.5\linewidth+1.8cm}{!}{For 8 Des 6 Cos 5 Int 5 Sag 6 Car 6}
	\item[\textbf{Punti Ferita:}] 300,  \textbf{Difesa:} 38,  \textbf{Iniziativa:} +6
	\item[\textbf{Movimento:}] 9 m, volare 27 m (buono)
	\item[\textbf{Tiri Salvez.:}] \resizebox{0.5\linewidth+1.8cm}{!}{\resizebox{0.5\linewidth+1.8cm}{!}{Tempra +20, Riflessi +21, Volontà +21}}
	\item[\textbf{Imm. Danni:}] Fuoco, Luce, Veleno, armi +1
	\item[\textbf{Immunità:}] afferrato, intralciato, paralizzato, pietrificato, prono, privo di sensi, affaticato, sanguinamento
	\item[\textbf{Sensi:}] Scurovisione 18 m, Visione Crepuscolare 18 m
	\item[\textbf{Linguaggi:}] Ictun, Celestiale, Comune, Ignan
	\item[\textbf{Sfida:}] 15 (13000 PX)\smallskip
\end{description}

\emph{\textbf{Consapevolezza della Luce.}} La Fenice ha sempre attivi i seguenti incantesimi \emph{\hyperlink{Individuazione del Magico}{Individuazione del Magico}, \hyperlink{Individuazione delle Malattie e dei Veleni}{Individuazione delle Malattie e dei Veleni}, \hyperlink{Vedere l'invisibile}{Vedere l'invisibile}}

\emph{\textbf{Incantesimi Innati.}} La caratteristica da incantatore della Fenice è il Carisma. La Fenice può lanciare in maniera innata i seguenti incantesimi, senza bisogno di componenti materiali:

A volontà: \emph{\hyperlink{Cura Ferite}{Cura Ferite} 1, \hyperlink{Dissolvi Magie}{Dissolvi Magie}, \hyperlink{Fiamma Perenne}{Fiamma Perenne}, \hyperlink{Rimuovi Maledizione}{Rimuovi Maledizione}, \hyperlink{Metamorfosi}{Metamorfosi} (solo in umanoidi)}

3/giorno: \emph{\hyperlink{Cura Ferite}{Cura Ferite} 5 di Massa, \hyperlink{Guarigione}{Guarigione}, \hyperlink{Muro di Fuoco}{Muro di Fuoco}, \hyperlink{Ristorare Superiore}{Ristorare Superiore}, \hyperlink{Tempesta di Fuoco}{Tempesta di Fuoco}}

1 volta: \emph{\hyperlink{Resurrezione}{Resurrezione}} la Fenice sacrificando la sua vita in maniera definitiva può riportare in vita una creatura.

\textbf{Azioni}

\emph{\textbf{Multiattacco.}} La Fenice può attaccare con due artigli ed il morso

\emph{\textbf{Morso.} Attacco con arma da mischia}: +12 al a colpire, portata 6 m, una creatura.

\emph{Colpisce:} 19 danni perforanti (2d8+8 + 1d6 da Luce)

\emph{\textbf{Artiglio.} Attacco con arma da mischia}: +12 al a colpire, portata 6 m, una creatura.

\emph{Colpisce:} 17 danni da taglio (2d6+8 + 1d6 da Luce)

\textbf{Reazione: \emph{Attacco d'opportunità}}: la Fenice effettua un attacco ad una creatura che attraversi o esca dalla sua portata di 6 metri.

\textbf{Abilità speciali}

\emph{\textbf{Rinascita}}

Una Fenice uccisa si riduce ad un falò di 3 metri cubi dove giace al centro un uovo di fenice. Dopo 1d4+4 round questo uovo si schiude e diventa una Fenice perfettamente sana. L'unico modo per evitare la rinascita è togliere l'uovo dal falò (20d6 di danno da Luce) od usare un incantesimo di Disintegrazione sull'uovo.
Una Fenice può resuscitare in questo modo una volta all'anno, se muore prima che sia trascorso questo tempo, la morte è definitiva. Uccidere una Fenice scatena l'ira delle Allieve della Luce e dei cavalieri di Sumkjr.

\emph{\textbf{Ali di fiamma}}

La Fenice può trasformare le sue piume in fiamma senza usare Azioni. Queste piume infliggono 1d6 danni da fuoco + 1d6 danni da Luce a tutte le creature entro 6 metri all'inizio del suo round.

\emph{\textbf{Arrabbiato:}} solo le leggende narrano di una Fenice arrabbiata e si dice che sia intervenuto direttamente un Patrono.

\textbf{Ecologia}\\
Ambiente: Deserti e colline calde\\
Organizzazione: Solitario\\
\textbf{Categoria Tesoro}: Nessuno\\
\textbf{Descrizione}\\
La leggenda narra che le Fenici siano gli uccelli da compagnia di Ljust, sicuramente sono creature maestose e bellissime ed emanano una Luce simile a quella della Patrona della Genesi. Il movimento delle loro ali non produce rumore mentre la loro voce è canto. La fenice è un leggendario uccello di fuoco e luce che vive solitamente nei deserti. Sono creature molto intelligenti e sagge ed a volte usando la loro capacità di metamorfosi si recano nelle città dove aiutano chi combatte contro l'oscurità.

La leggenda racconta che le fenici si generino quando un Cavaliere di Sumkjir o un Allieva della Luce compia l'estremo sacrificio.

\mostro{Fioritura Ossea}
\noindent
\begin{description}[noitemsep, topsep=0pt, parsep=0pt, partopsep=0pt, leftmargin=0cm, labelwidth=2.2cm]
	\item[\textbf{Taglia/Tipo:}] Grande non morto, non allineato
	\item[\textbf{Caratt.:}] \resizebox{0.5\linewidth+1.8cm}{!}{For 3 Des 2 Cos 4 Int -2 Sag -2 Car -3}
	\item[\textbf{Punti Ferita:}] 127,  \textbf{Difesa:} 22,  \textbf{Iniziativa:} +2
	\item[\textbf{Movimento:}] 12 m
	\item[\textbf{Tiri Salvez.:}] \resizebox{0.5\linewidth+1.8cm}{!}{Tempra +10, Riflessi +8, Volontà +4}
	\item[\textbf{Imm. Danni:}] Veleno
	\item[\textbf{Res. Danni:}] perforante, tagliente, Veleno, da Luce
	\item[\textbf{Immunità:}] affaticato, sanguinamento, rallentato, lentezza
	\item[\textbf{Sensi:}] Vista cieca 18 m
	\item[\textbf{Linguaggi:}] comprende il Comune, druidico, silvano ma non può parlare
	\item[\textbf{Sfida:}] 6 (2300 PX)\smallskip
\end{description}

\emph{\textbf{Un piede nella Natura.}} finché Fioritura Ossea è a contatto con la terra rigenera all'inizio del suo round 6 Punti Ferita.

\emph{\textbf{Uno nella Natura.}} finché Fioritura Ossea è in un ambiente naturale e non si muove attacca di sorpresa se non notato. E' richiesta una prova di Consapevolezza 21 per notarlo.

\emph{\textbf{Natura Non Morta.}} Fioritura Ossea non necessita aria, cibo, bevande o sonno.

\textbf{Azioni}

\emph{\textbf{Multiattacco}} Fioritura Ossea può attaccare con il Grande Randello 3 volte oppure usa Soffiare Spore ed esegue un attacco con il Grande Randello

\emph{\textbf{Grande Randello.} Attacco con arma da mischia}: +8 a colpire, portata 2 m, un bersaglio.

\emph{Colpisce:} 17 (2d10 + 6) danni contundenti

\textbf{Reazione: \emph{Attacco d'opportunità}}: la Fioritura ossea effettua un attacco ad una creatura che attraversi o esca dalla sua portata di 1 metro.

\emph{\textbf{Soffio di Spore}}: raggio di 6 metri. Fioritura Ossea emana spore e pollini tutto intorno a se. Qualsiasi creatura che respiri nel raggio di 6 metri dalla Fioritura Ossea deve effettuare un Tiro Salvezza su Tempra a DC 18. Se il Tiro Salvezza fallisce la creatura subisce 3d8 danni da veleno ed è sotto l'influenza dell'incantesimo \hyperlink{lentezza}{Lentezza} per 1 minuto. Se il Tiro Salvezza riesce subisce metà del danno ed è rallentato fino alla fine del round successivo.

\emph{\textbf{Arrabbiato:}} la Fioritura Ossea raccoglie le energie della natura intorno a se avvizzendola. Recupera 50 Punti Ferita. Costa 2 Azioni.

\textbf{Ecologia}\\
Ambiente: Qualsiasi foresta\\
Organizzazione: Solitario, gruppi (2d12)\\
\textbf{Categoria Tesoro}: Accidentale\\
\textbf{Descrizione}\\
Le Fioriture Ossee sono creature morte nel fitto della foresta per i più disparati motivi. La Natura non volendo sprecare nulla anima la creatura per farne suo difensore. A prima vista una Fioritura Ossea non è diverso da un tronco coperto di licheni colorati, piccoli funghi e manto erboso tanto è uno con la natura.

\mostro{Fungo Stridente}
\noindent
\begin{description}[noitemsep, topsep=0pt, parsep=0pt, partopsep=0pt, leftmargin=0cm, labelwidth=2.2cm]
	\item[\textbf{Taglia/Tipo:}] Media pianta, disallineato
	\item[\textbf{Caratt.:}] \resizebox{0.5\linewidth+1.8cm}{!}{For -5 Des -5 Cos 0 Int -5 Sag -4 Car -5}
	\item[\textbf{Punti Ferita:}] 15,  \textbf{Difesa:} 7,  \textbf{Iniziativa:} -5
	\item[\textbf{Movimento:}] 0 m
	\item[\textbf{Tiri Salvez.:}] \resizebox{0.5\linewidth+1.8cm}{!}{Tempra +3, Riflessi +3, Volontà +3}
	\item[\textbf{Immunità:}] accecato, assordato, spaventato
	\item[\textbf{Sensi:}] Vista Cieca 9 m (cieco oltre questo raggio)
	\item[\textbf{Sfida:}] 0 (10 PX)\smallskip
\end{description}

\emph{\textbf{Falso Aspetto.}} Mentre il fungo stridente rimane immobile, è indistinguibile da un normale fungo.

\textbf{Azioni}

\emph{\textbf{Strillo.}} Quando una luce intensa o una creatura si trova entro 9 metri dal fungo stridente, esso emette un strillo udibile fino a 90 metri di distanza. Il fungo stridente continua a strillare finché la fonte del disturbo non si è portata fuori gittata e per altri 1d4 round successivi, ovvero finché non si è sgonfiato il cappello.

\textbf{Ecologia}\\
Ambiente: Qualsiasi sotterraneo\\
Organizzazione: Solitario, coppia o macchia (3-12)\\
\textbf{Categoria Tesoro}: Accidentale\\
\textbf{Descrizione}\\
Un fungo stridente è alto circa 50 cm, dall'ampio cappello marrone. Una volta emesso l'urlo il cappello si rigonfia in 1d3 minuti.

Si racconta di cuochi delle profondità specializzati nel cuocere questi funghi in pietanze sopraffine. I più bravi riescono anche a non fare sgonfiare il cappello.

\mostro{Fungo Violetto}
\noindent
\begin{description}[noitemsep, topsep=0pt, parsep=0pt, partopsep=0pt, leftmargin=0cm, labelwidth=2.2cm]
	\item[\textbf{Taglia/Tipo:}] Media pianta, disallineato
	\item[\textbf{Caratt.:}] \resizebox{0.5\linewidth+1.8cm}{!}{For -4 Des -5 Cos 0 Int -5 Sag -4 Car -5}
	\item[\textbf{Punti Ferita:}] 19,  \textbf{Difesa:} 7,  \textbf{Iniziativa:} -5
	\item[\textbf{Movimento:}] 2 m
	\item[\textbf{Tiri Salvez.:}] \resizebox{0.5\linewidth+1.8cm}{!}{Tempra +3, Riflessi +3, Volontà +3}
	\item[\textbf{Immunità:}] accecato, assordato, spaventato
	\item[\textbf{Sensi:}] Vista Cieca 9 m (cieco oltre questo raggio)
	\item[\textbf{Sfida:}] 1/4 (50 PX)\smallskip
\end{description}

\emph{\textbf{Falso Aspetto.}} Mentre il fungo violetto rimane immobile, è indistinguibile da un normale fungo.

\textbf{Azioni}

\emph{\textbf{Multiattacco.}} Il fungo effettua 1d4 attacchi con Contatto Putrido.

\emph{\textbf{Contatto Putrido.} Attacco con arma da mischia}: +3 a colpire, portata 3 m, un bersaglio.

\emph{Colpisce:} 4 (1d8) danni da Vuoto.

\textbf{Ecologia}\\
Ambiente: Qualsiasi sotterraneo\\
Organizzazione: Solitario, coppia o macchia (3-12)\\
\textbf{Categoria Tesoro}: Accidentale\\
\textbf{Descrizione}\\
I funghi viola sono uno dei più noti e temuti pericoli delle caverne. Un viaggiatore può spesso notare i segni lasciati dal fungo viola su coloro che vivono o cacciano nei luoghi in cui questi funghi carnivori si appostano. Queste profonde e orribili cicatrici sembrano solchi scavati nella carne: i segni di un incontro ravvicinato con un fungo viola.

Un fungo viola si nutre della materia organica putrefatta, ma a differenza della maggioranza dei funghi non è un consumatore passivo. I viticci di un fungo viola possono colpire con inaspettata rapidità e sono ricoperti di un veleno virulento che causa la putrefazione delle carni con nauseante velocità. Questo potente veleno, se trascurato, può far marcire rapidamente un intero braccio o una gamba, lasciandosi dietro solo ossa che presto si corroderanno anch'esse.

Sebbene i funghi viola possano muoversi, lo fanno solo per attaccare o cacciare la preda. Un fungo viola con un flusso regolare di putredine di cui nutrirsi si accontenta di restare in un posto. Molti abitanti del sottosuolo, in particolare Trogloditi e Vegepigmei, sfruttano questo comportamento a loro vantaggio e posizionano molteplici funghi viola in giunzioni ed entrate chiave delle loro caverne come guardiani, assicurandosi di fornire loro cadaveri a sufficienza per evitare che si addentrino nel rifugio in cerca di cibo.

Alcune specie di Boleto Stridente hanno un aspetto piuttosto simile a quello dei funghi viola, sebbene manchino di ramificazioni tentacolari. Non è strano trovare boleti stridenti e funghi viola nello stesso groviglio, specialmente nelle aree dove altre creature coltivano questi funghi come guardiani.

Un fungo viola è alto 1,2 metri e pesa 25 kg.

\mostro{Fuoco Fatuo}
\noindent
\begin{description}[noitemsep, topsep=0pt, parsep=0pt, partopsep=0pt, leftmargin=0cm, labelwidth=2.2cm]
	\item[\textbf{Taglia/Tipo:}] Minuscola non morto, malvagio
	\item[\textbf{Caratt.:}] \resizebox{0.5\linewidth+1.8cm}{!}{For -5 Des 9 Cos 0 Int 1 Sag 2 Car 0}
	\item[\textbf{Punti Ferita:}] 51,  \textbf{Difesa:} 23,  \textbf{Iniziativa:} +9
	\item[\textbf{Movimento:}] 0 m, volo 15 m, Fluttuare
	\item[\textbf{Tiri Salvez.:}] \resizebox{0.5\linewidth+1.8cm}{!}{Tempra +3, Riflessi +11, Volontà +4}
	\item[\textbf{Res. Danni:}] Acido, Veleno, Freddo, Fuoco, da Vuoto, Suono; armi che non siano magiche
	\item[\textbf{Immunità:}] afferrato, intralciato, paralizzato, privo di sensi, prono, affaticato, sanguinamento
	\item[\textbf{Sensi:}] Scurovisione 36 m
	\item[\textbf{Linguaggi:}] le lingue che conosceva in vita
	\item[\textbf{Sfida:}] 2 (450 PX)\smallskip
\end{description}

\emph{\textbf{Consumare Vita.}} Con un'Azione Immediata, il fuoco fatuo può prendere a bersaglio una creatura che può vedere entro 1 metro da esso e che abbia 0 Punti Ferita o meno e sia ancora in vita. Il bersaglio deve riuscire un Tiro Salvezza di Tempra DC 12 contro questa magia o morire. Se il bersaglio muore, il fuoco fatuo recupera 10 (3d6) Punti Ferita.

\emph{\textbf{Effimero.}} Il fuoco fatuo non può indossare né trasportare nulla.

\emph{\textbf{Illuminazione Variabile.}} Il fuoco fatuo promana luce intensa in un raggio da 1 a 6 metri e luce fioca per un numero di metri pari al doppio del raggio scelto. Il fuoco fatuo può modificare questo raggio con una Reazione.

\emph{\textbf{Movimento Incorporeo.}} Il fuoco fatuo può muoversi attraverso altre creature e oggetti come se fossero terreno difficile. Subisce 5 (1d10) danni da forza se termina il suo round all'interno di un oggetto.

\emph{\textbf{Natura Non Morta.}} Il fuoco fatuo non ha bisogno di aria, cibo o bevande.

\textbf{Azioni}

\emph{\textbf{Scossa.} Attacco con incantesimo in mischia}: +6 a colpire, portata 1 m, una creatura.

\emph{Colpisce:} 9 (2d8) danni da elettricità.

\emph{\textbf{Invisibilità.}} Il fuoco fatuo e la sua luce diventano magicamente invisibili finché non attacca o usa Consumare Vita, o finché la sua concentrazione non termina (come se si stesse concentrando su di un incantesimo).

\textbf{Ecologia}
Ambiente: Qualsiasi Palude\\
Organizzazione: Solitario, coppia o sequenza (3-4)\\
\textbf{Categoria Tesoro}: Accidentale\\
\textbf{Descrizione}\\
Malvagie creature che si nutrono delle forti emanazioni psichiche delle creature terrorizzate, i fuochi fatui traggono piacere nel mettere i viaggiatori creduloni in situazioni pericolose. Nelle terre selvagge, dove sono molto comuni, i fuochi fatui preferiscono tattiche semplici come posizionarsi su scogli o sabbie mobili dove possono essere scambiati facilmente per lanterne (specialmente se possono predisporre la trappola nei pressi di vere lanterne di segnalazione), così da attirare i viaggiatori verso il pericolo.

I fuochi fatui possono contare solo sulla loro scossa elettrica in situazioni pericolose, quindi preferiscono lasciare che altre creature o pericoli si occupino delle loro vittime mentre loro fluttuano nelle vicinanze e banchettano.

I fuochi fatui possono brillare di qualunque colore desiderino, ma sono più spesso gialli, bianchi, verdi o blu. Possono anche variare la loro luminosità per creare un disegno: molti fuochi fatui amano creare forme che somigliano vagamente a teschi nella loro luminescenza per aumentare il terrore nelle loro vittime. I loro veri corpi sono globi di materiale spugnoso traslucido appena visibili di circa 30 centimetri che pesano 1,5 kg e possono emettere luce su tutta la loro superficie. La luce dei fuochi fatui brilla approssimativamente come una torcia, e sebbene non sembrino utilizzare suoni per comunicare, sentono perfettamente e possono far vibrare i loro corpi così rapidamente da imitare il linguaggio.

Nonostante siano denigrati dalla maggioranza delle creature senzienti, i fuochi fatui sono in realtà alquanto intelligenti, sebbene ragionino in modo completamente alieno.

I fuochi fatui non hanno età e sono di fatto immortali, a meno che non muoiano di morte violenta; i fuochi fatui più antichi possono essere ottimi depositari di conoscenze del passato, sebbene convincere una di queste crudeli creature a cooperare possa essere piuttosto complicato.

\mostro{Fustigatore}
\noindent
\begin{description}[noitemsep, topsep=0pt, parsep=0pt, partopsep=0pt, leftmargin=0cm, labelwidth=2.2cm]
	\item[\textbf{Taglia/Tipo:}] Grande mostruosità, malvagio
	\item[\textbf{Caratt.:}] \resizebox{0.5\linewidth+1.8cm}{!}{For 4 Des -1 Cos 3 Int 3 Sag 3 Car -2}
	\item[\textbf{Punti Ferita:}] 108,  \textbf{Difesa:} 17,  \textbf{Iniziativa:} +3
	\item[\textbf{Movimento:}] 3 m, scalata 3 m
	\item[\textbf{Tiri Salvez.:}] \resizebox{0.5\linewidth+1.8cm}{!}{Tempra +8, Riflessi +4, Volontà +8}
	\item[\textbf{Comp.:}] Furtività +5, Consapevolezza +6
	\item[\textbf{Sensi:}] Scurovisione 18 m
	\item[\textbf{Linguaggi:}] comune, lingue antiche (latino, greco, celtico..)
	\item[\textbf{Sfida:}] 5 (1800 PX)\smallskip
\end{description}

\emph{\textbf{Falso Aspetto.}} Quando il fustigatore rimane immobile, è indistinguibile da una normale formazione rocciosa, come una stalagmite.

\emph{\textbf{Scalare come Ragno.}} Il fustigatore può scalare superfici difficili, compreso lo stare a testa in giù sul soffitto, senza bisogno di effettuare una prova di competenza.

\emph{\textbf{Viticci Afferranti.}} Il fustigatore può avere fino a sei viticci alla volta. Ogni viticcio può essere attaccato (Difesa 16; 10 Punti Ferita; immunità ai danni da veleno). Distruggere un viticcio non infligge danni al fustigatore, che può produrre un viticcio di rimpiazzo nel suo prossimo round. Un viticcio può essere anche rotto se una creatura effettua un'Azione e riesce un Tiro Salvezza Tempra con Forza DC 17 contro di esso.

\textbf{Azioni}

\emph{\textbf{Multiattacco.}} Il fustigatore può effettuare quattro attacchi con i suoi viticci, usare avvolgere ed effettuare un attacco con il morso.

\emph{\textbf{Morso.} Attacco con arma da mischia}: +7 a colpire, portata 2 m, un bersaglio.

\emph{Colpisce:} 22 (4d8 + 4) danni perforanti e Malattia Necrosi Purulenta

\emph{Necrosi Purulenta:} 1 giorno, TS Tempra DC 15, 12 ore, 1 successo, -1 Costituzione.

\emph{\textbf{Viticcio.} Attacco con arma da mischia}: +7 a colpire, portata 15 m, una creatura.

\emph{Colpisce:} Il bersaglio è afferrato (DC 15 per fuggire). Fino al termine dell'afferrare il fustigatore non può usare lo stesso viticcio contro un altro bersaglio.

\emph{\textbf{Avvolgere.}} Il fustigatore trascina le creature afferrate da lui di 7 metri verso di lui. TS Tempra DC 17 per non farsi spostare.

\textbf{Reazione: \emph{Attacco d'opportunità}}: il fustigatore effettua un attacco con Viticcio ad una creatura che attraversi o esca dalla sua portata di 6 metri.

\emph{\textbf{Arrabbiato:}} il fustigatore emette un onda cacofonica nauseabonda. Tutte le creature nel raggio di 6 metri devono eseguire un Tiro Salvezza su Tempra DC 18 o essere Nauseato fino alla fine del round successivo. Costa 2 Azioni.

\textbf{Ecologia}
Ambiente: Qualsiasi Sotterraneo\\
Organizzazione: Solitario, coppia o gruppo (3-6)\\
\textbf{Categoria Tesoro}: D\\
\textbf{Descrizione}\\
Il fustigatore è un cacciatore da agguato. Capace di modificare la colorazione e la forma del suo corpo, un fustigatore nascosto sembra una stalagmite di pietra o ghiaccio (o in luoghi dal soffitto basso, una colonna di pietra o ghiaccio). Nelle aree prive di questi tratti per nascondersi un fustigatore può comprimere il suo corpo fino a sembrare un masso. Le sferze che può estroflettere non sono di carne ma di uno spesso materiale semiliquido simile a cera parzialmente fusa ma con la resistenza di una catena di ferro e la capacità di intirizzire la carne e indebolire le forze. Il fustigatore può usare queste sferze con grande maestria e farle volare fino a 15 metri per rubare gli oggetti che attraggono la sua attenzione.

Nonostante la sua forma aliena e mostruosa, il fustigatore è uno degli abitanti più intelligenti del sottosuolo. Non formano vaste società (anche se spesso si trovano a vivere insieme ad altre creature del sottosuolo come i Divora Cervelli, con cui a volte si alleano), ma spesso si aggregano in piccoli gruppi. Particolarmente interessato alla filosofia della vita e della morte, e agli aspetti più sottili delle religioni più sinistre e crudeli del mondo, un fustigatore può parlare o discutere per ore con quelli che inizialmente aveva semplicemente cercato di mangiare. Alcune storie parlano di oratori e filosofi particolarmente dotati che sono stati tenuti per giorni o anche anni come animali domestici o compagni di conversazione da gruppi di fustigatori; alla fine, però, se non riescono a scappare, l'appetito dei fustigatori finisce per avere la meglio sulla loro curiosa intelligenza, specialmente nei casi in cui questi animali da compagnia superano costantemente l'arguzia e la pazienza dei loro guardiani.
Un fustigatore è alto 2,7 metri e pesa 1.100 kg.

%\addcontentsline{toc}{subsubsection}{G}
\pdfbookmark[3]{G}{G}

\mostro{Gablin}
\noindent
\begin{description}[noitemsep, topsep=0pt, parsep=0pt, partopsep=0pt, leftmargin=0cm, labelwidth=2.2cm]
	\item[\textbf{Taglia/Tipo:}] Piccolo immondo, malvagio
	\item[\textbf{Caratt.:}] \resizebox{0.5\linewidth+1.8cm}{!}{For 2 Des 1 Cos 1 Int -2 Sag -1 Car -2}
	\item[\textbf{Punti Ferita:}] 19,  \textbf{Difesa:} 13,  \textbf{Iniziativa:} +1
	\item[\textbf{Movimento:}] 9 m
	\item[\textbf{Tiri Salvez.:}] \resizebox{0.5\linewidth+1.8cm}{!}{Tempra +3, Riflessi +3, Volontà +3}
	\item[\textbf{Sensi:}] Scurovisione 18 m
	\item[\textbf{Linguaggi:}] comprendono il Comune ma non lo parlano, Abissale
	\item[\textbf{Sfida:}] 1/4 (50 PX)\smallskip
\end{description}

\emph{\textbf{Sensibilità alla Luce}}. Mentre è alla luce del sole, il gablin ha -1d6 ai tiri per colpire oltre che alle prove di Consapevolezza basate sulla vista.

\textbf{Azioni}

\emph{\textbf{Spada Corta.} Attacco con arma da mischia}: +4 a colpire, portata 1 m, un bersaglio.

\emph{Colpisce:} 5 (1d6 + 2) danni taglienti.

\emph{\textbf{Morso.} Attacco con arma da mischia}: +5 a colpire, contatto, un bersaglio.

\emph{Colpisce:} 3 (1d1 + 2) danni perforanti.

\textbf{Ecologia}\\
Ambiente: Ovunque\\
Organizzazione: Gruppo (8-12), banda da guerra (10-24) o tribù (50+, 1 sergente di 3° livello per 20 adulti, 1 o 2 luogotenenti di 4° o 5° livello, 1 capo di 6°-8° livello, 6-12 lupi selvatici e 1-4 Ogre o 1-2 Campione Gablin)\\
\textbf{Categoria Tesoro}: Accidentale\\
\textbf{Descrizione}\\
I Gablin sono la feccia della feccia, si dice che un Gablin nasce ad ogni pensiero cattivo e sicuramente sono veramente tanti.
I Gablin sono piccoli umanoidi dalla pelle scura, con striature verdi generati inizialmente per volontà di Cattalm con l'unico scopo di portare distruzione, morte e sofferenza.
I Gablin si possono nascondere ovunque purché in prossimità di una fonte di cibo, solitamente prediligono le fogne oppure strutture abbandonate vicino ai villaggi.
Lo scopo unico di un Gablin è uccidere e perpetuare la specie. I Gablin sono tutti maschi e la loro natura immonda li rende capaci di impregnare qualsiasi femmina umanoide.
Solitamente la gestazione dura solo 3 settimane durante le quali le donne vengono torturate per rafforzare gli 1d6+2 piccoli che porta in grembo. Il parto solitamente si conclude con i piccoli di Gablin che sventrano la madre e ne fanno il primo loro pasto.
Questo metodo di procreazione unita alla loro voracia famelica di sangue e carne ne fanno tra le creature più odiate e temute.
Anche se singolarmente non sono particolarmente temibili i Gablin si muovono sempre in gruppo e se questo supera le due dozzine allora c'è quasi sempre un Gablin Incantatore o addirittura un Campione Gablin a guidarli.

\mostro{Campione Gablin}
\noindent
\begin{description}[noitemsep, topsep=0pt, parsep=0pt, partopsep=0pt, leftmargin=0cm, labelwidth=2.2cm]
	\item[\textbf{Taglia/Tipo:}] Media immondo, malvagio
	\item[\textbf{Caratt.:}] \resizebox{0.5\linewidth+1.8cm}{!}{For 4 Des 2 Cos 3 Int 1 Sag 0 Car -1}
	\item[\textbf{Punti Ferita:}] 70,  \textbf{Difesa:} 18,  \textbf{Iniziativa:} +2
	\item[\textbf{Movimento:}] 12 m
	\item[\textbf{Tiri Salvez.:}] \resizebox{0.5\linewidth+1.8cm}{!}{Tempra +6, Riflessi +5, Volontà +3}
	\item[\textbf{Sensi:}] Scurovisione 18 m
	\item[\textbf{Linguaggi:}] Comune, Abissale
	\item[\textbf{Sfida:}] 3 (700 PX)\smallskip
\end{description}

\textbf{Azioni}

\emph{\textbf{Randello Pesante.} Attacco con arma da mischia}: +6 a colpire, portata 2 m, un bersaglio.

\emph{Colpisce:} 11 (2d6 + 4) danni contundenti.

\emph{\textbf{Evocare Gablin}}: 3 Azioni. Il Gablin spilla il suo sangue a terra e a questo sorgono 2d4 Gablin, perde 1 Punto Ferita

\textbf{Ecologia}\\
Ambiente: Qualsiasi\\
Organizzazione: a capo di un gruppo di Gablin\\
\textbf{Categoria Tesoro}: Armatura di Pelle, Randello pesante, B\\
\textbf{Descrizione}\\
I Campioni Gablin vengono generati spontaneamente quando il numero di Gablin presente raggiunge le 20 unità. Enormemente più grossi, più forti ed intelligenti di un Gablin i Campioni sono i leader del gruppo, coloro che pianificano le battaglie e gli scontri.
Non hanno remore a mandare al massacro i Gablin o ad uccidere qualsiasi cosa che respiri. Pervasi dello spirito di Cattalm il loro scopo è sempre e solo distruggere ed uccidere.

\mostro{Paladino Gablin}
\noindent
\begin{description}[noitemsep, topsep=0pt, parsep=0pt, partopsep=0pt, leftmargin=0cm, labelwidth=2.2cm]
	\item[\textbf{Taglia/Tipo:}] Grande immondo, malvagio
	\item[\textbf{Caratt.:}] \resizebox{0.5\linewidth+1.8cm}{!}{For 5 Des 2 Cos 3 Int 2 Sag 3 Car 3}
	\item[\textbf{Punti Ferita:}] 126,  \textbf{Difesa:} 22,  \textbf{Iniziativa:} +2
	\item[\textbf{Movimento:}] 12 m
	\item[\textbf{Tiri Salvez.:}] \resizebox{0.5\linewidth+1.8cm}{!}{\resizebox{0.5\linewidth+1.8cm}{!}{Tempra +9, Riflessi +8, Volontà +9}}
	\item[\textbf{Sensi:}] Scurovisione 18 m
	\item[\textbf{Linguaggi:}] Comune, Abissale
	\item[\textbf{Sfida:}] 6 (2300 PX)\smallskip
\end{description}

\textbf{Azioni}

\emph{\textbf{Multiattacco.}} Il Paladino Gablin attacca con 2 colpi di spada bastarda.

\emph{\textbf{Spada Bastarda.} Attacco con arma da mischia}: +8 a colpire, portata 2 m, un bersaglio.

\emph{Colpisce:} 10 (1d10 + 5) danni contundenti, più 1d6 danno da Vuoto. Se la creatura colpita è un Seguace o Devoto di Gradh il danno aumenta di un ulteriore 1d6.

\emph{\textbf{Evocare Gablin}}: 3 Azioni. Il Gablin spilla il suo sangue a terra e a questo sorgono 3d4 Gablin.

\textbf{Reazione: \emph{Attacco d'opportunità}}: il Paladino Gablin effettua un attacco ad una creatura che attraversi o esca dalla sua portata di 1 metro.

\textbf{Aura immonda}: il Paladino Gablin emana un aura di 6 metri di raggio intorno a lui che conferisce +2 al Tiro per Colpire ed al Danno a tutti gli altri Gablin ed impone -2 al Tiro per Colpire e TS alle altre creature non Devoti o Seguaci di Cattalm.

\textbf{Ecologia}\\
Ambiente: Qualsiasi\\
Organizzazione: a capo di un armata di Gablin\\
\textbf{Categoria Tesoro}: Armatura da campo, Spada Bastarda +1, S\\
\textbf{Descrizione}\\
I Paladini Gablin sono tra i più potenti gablin che si conoscano, i veri eletti di Cattalm. Evocati da più potenti seguaci di Cattalm possono da soli guidare centinaia di Gablin e grazia al loro acume preparare accurati piani e portare scompiglio e distruzione in intere regioni.

\mostro{Gargoyle}
\noindent
\begin{description}[noitemsep, topsep=0pt, parsep=0pt, partopsep=0pt, leftmargin=0cm, labelwidth=2.2cm]
	\item[\textbf{Taglia/Tipo:}] Media elementale, malvagio
	\item[\textbf{Caratt.:}] \resizebox{0.5\linewidth+1.8cm}{!}{For 2 Des 0 Cos 3 Int -2 Sag 0 Car -2}
	\item[\textbf{Punti Ferita:}] 52,  \textbf{Difesa:} 14,  \textbf{Iniziativa:} +0
	\item[\textbf{Movimento:}] 9 m, volo 18 m
	\item[\textbf{Tiri Salvez.:}] \resizebox{0.5\linewidth+1.8cm}{!}{Tempra +5, Riflessi +3, Volontà +3}
	\item[\textbf{Res. Danni:}] Veleno, da arma non magica o che non siano di adamantio
	\item[\textbf{Immunità:}] pietrificato, affaticato
	\item[\textbf{Sensi:}] Scurovisione 18 m
	\item[\textbf{Linguaggi:}] Tremun
	\item[\textbf{Sfida:}] 2 (450 PX)\smallskip
\end{description}

\emph{\textbf{Falso Aspetto.}} Mentre il gargoyle rimane immobile, è indistinguibile da una statua inanimata.

\emph{\textbf{Natura Elementale.}} Una gargoyle non ha bisogno di aria, cibo, bevande o sonno.

\textbf{Azioni}

\emph{\textbf{Multiattacco.}} Il gargoyle effettua due attacchi: uno con il morso e uno con gli artigli.

\emph{\textbf{Artigli.} Attacco con arma da mischia}: +5 a colpire, portata 1 m, un bersaglio.

\emph{Colpisce:} 5 (1d6 + 2) danni taglienti, 1 danno da Sanguinamento.

\emph{\textbf{Morso.} Attacco con arma da mischia}: +5 a colpire, portata 1 m, un bersaglio.

\emph{Colpisce:} 5 (1d6 + 2) danni perforanti.

\textbf{Reazione: \emph{Attacco d'opportunità}}: il gargoyle attacca se sta volando ed una creatura esce o attraversa la sua portata di 1 m.

\textbf{Ecologia}
Ambiente: Qualsiasi\\
Organizzazione: Solitario, coppia o stormo (3-12)\\
\textbf{Categoria Tesoro}: Q\\
\textbf{Descrizione}\\
I gargoyle spesso sembrano essere statue alate di pietra, poiché possono rimanere immobili indefinitamente per poi sorprendere i nemici. I gargoyle tendono a comportamenti ossessivo-compulsivi, tanto diversi quanto abbondante è la loro specie. Libri, ninnoli rubati, armi e trofei raccolti dai nemici caduti sono solo alcuni esempi dei tipi di oggetti che un gargoyle può collezionare per decorare la sua tana e il suo territorio.

I gargoyle tendono ad avere uno stile di vita solitario, anche se a volte formano temibili stormi detti ali per protezione e divertimento. In certe condizioni, una tribù di gargoyle può persino allearsi con altre creature, ma anche la più stabile di queste alleanze può crollare per ragioni infime; i gargoyle sono solo traditori, meschini e vendicativi.

I gargoyle sono noti per abitare nel cuore delle città più grandi, accovacciati tra le decorazioni di pietra delle cattedrali e degli edifici dove si nascondono in bella vista di giorno piombando giù per nutrirsi di vagabondi, mendicanti e altri sfortunati la notte.

più a lungo una tribù di gargoyle dimora in un'area di edifici o rovine, più i suoi membri cominciano ad assomigliare allo stile architettonico della zona. I cambiamenti subiti dall'aspetto di un gargoyle sono lenti e sottili, ma nel corso degli anni possono diventare radicali.

Un'insolita variante del gargoyle non abita tra edifici e rovine ma sotto le onde del mare. Queste creature sono note come kapoacinth; hanno le stesse statistiche base dei gargoyle normali, eccetto che hanno il sottotipo acquatico e le loro ali gli garantiscono una velocità di nuotare di 12 metri (ma sono inutili per volare). I kapoacinth abitano nelle regioni costiere poco profonde dove possono strisciare fuori dalla spuma per dare la caccia ai residenti della zona. È più probabile che formino stormi, poiché i kapoacinth preferiscono la vita di gruppo a quella solitaria.

\mostro{G.E.C.}
\noindent
\begin{description}[noitemsep, topsep=0pt, parsep=0pt, partopsep=0pt, leftmargin=0cm, labelwidth=2.2cm]
	\item[\textbf{Taglia/Tipo:}] Grande aberrazione, malvagio
	\item[\textbf{Caratt.:}] \resizebox{0.5\linewidth+1.8cm}{!}{For 6 Des 1 Cos 5 Int 3 Sag 1 Car -1}
	\item[\textbf{Punti Ferita:}] 205,  \textbf{Difesa:} 26,  \textbf{Iniziativa:} +3
	\item[\textbf{Movimento:}] 9 m, scavare 9 m
	\item[\textbf{Tiri Salvez.:}] \resizebox{0.5\linewidth+1.8cm}{!}{\resizebox{0.5\linewidth+1.8cm}{!}{Tempra +15, Riflessi +11, Volontà +11}}
	\item[\textbf{Comp.:}] Consapevolezza +10
	\item[\textbf{Sensi:}] Scurovisione 18 m, senso tellurico 18 m
	\item[\textbf{Sfida:}] 10 (5900 PX)\smallskip
\end{description}

\textbf{Azioni}

\emph{\textbf{Multiattacco.}} Il G.E.C. può attaccare con due artigli oppure con il morso

\textbf{Artigli}: Attacco con arma naturale da mischia: +11 a colpire, portata 3 m, un bersaglio.

\emph{Colpisce:} 20 (6d6 + 5) danni taglienti, 1 danno da Sanguinamento.

\textbf{Morso}: Attacco con arma naturale da mischia: +11 al colpire, portata 3 m, un bersaglio

\emph{Colpisce:} 22 (6d6 + 8) danni taglienti, 1 danno da Sanguinamento, Visione Offuscata.

\textbf{Visione Offuscata:} è un effetto da Veleno, TS Volontà DC 18 oppure fino alla fine del round successivo il bersaglio ha -1d6 al Tiro per Colpire.

\emph{\textbf{Sguardo.}} E' sufficiente guardare il G.E.C. per essere affetti da \hyperlink{Confusione}{Confusione}, come omonimo incantesimo. Per resistere è necessario effettuare un Tiro Salvezza su Volontà a DC 22. Ogni round è possibile ripetere il Tiro Salvezza per resistere all'effetto.

Combattere senza guardare il G.E.C. impone -1d6 al Tiro per Colpire.

\emph{\textbf{Arrabbiato:}} il G.E.C. emette un ruggito cacofonico. Le creature entro 6 metri da lui devono fare un Tiro Salvezza su Volontà a DC 22 o essere affetti da Confusione per 2 round. Costa 2 Azioni.

\textbf{Ecologia}\\
Ambiente: Sotterraneo\\
Organizzazione: solitario, gruppo (2-4) \\
\textbf{Categoria Tesoro}: Accidentale\\
\textbf{Descrizione}\\
Il Grande Essere Chitinoso, o G.E.C, è un insetto dal vago aspetto umanoide di quasi 3 metri di altezza, possente e dotato di due chele fortissime e resistenti capaci di scavare e tranciare qualsiasi materiale. 4 occhi piccoli, centrali e multi faccettati emanano un fioca luminescenza cangiante che confondono le creature che incrociano il loro sguardo.

Probabilmente frutto di una qualche incantesimo di trasformazione andato a male i G.E.C. sono padroni del sottosuolo. Creature dotate di una reale intelligenza amano la carne di elfo e combattono in maniera tattica ed accorta.

\mostro{Djinni}
\noindent
\begin{description}[noitemsep, topsep=0pt, parsep=0pt, partopsep=0pt, leftmargin=0cm, labelwidth=2.2cm]
	\item[\textbf{Taglia/Tipo:}] Grande elementale, buono
	\item[\textbf{Caratt.:}] \resizebox{0.5\linewidth+1.8cm}{!}{For 5 Des 2 Cos 6 Int 2 Sag 3 Car 5}
	\item[\textbf{Punti Ferita:}] 226,  \textbf{Difesa:} 28,  \textbf{Iniziativa:} +2
	\item[\textbf{Movimento:}] 9 m, volo 27 m
	\item[\textbf{Tiri Salvez.:}] \resizebox{0.5\linewidth+1.8cm}{!}{\resizebox{0.5\linewidth+1.8cm}{!}{Tempra +17, Riflessi +13, Volontà +14}}
	\item[\textbf{Imm. Danni:}] Elettricità, Suono
	\item[\textbf{Sensi:}] Scurovisione 36 m
	\item[\textbf{Linguaggi:}] Ictun
	\item[\textbf{Sfida:}] 11 (7200 PX)\smallskip
\end{description}

\emph{\textbf{Decesso Elementale.}} Se il djinni muore, il suo corpo si disintegra in una brezza calda, lasciando dietro di sé solo l'equipaggiamento che il djinni stava indossando o trasportando.

\emph{\textbf{Incantesimi Innati.}} La caratteristica da incantatore innato del djinni è il Carisma 17. Può lanciare in maniera innata i seguenti incantesimi, senza bisogno di componenti materiali:

A volontà: \emph{\hyperlink{Conoscere i Tratti}{Conoscere i Tratti}, \hyperlink{Individuazione del Magico}{Individuazione del Magico}, \hyperlink{Onda Tonante}{Onda Tonante}}

3/giorno ciascuno: \emph{\hyperlink{Camminare nel Vento}{Camminare nel Vento}, \hyperlink{Creare Cibo e Acqua}{Creare Cibo e Acqua}} (può creare vino al posto dell'acqua), \emph{\hyperlink{Lingue}{Lingue}}

1/giorno ciascuno: \emph{\hyperlink{Creazione}{Creazione}}, \emph{\hyperlink{Evoca Elementale}{Evoca Elementale}} (solo elementale dell'aria), \emph{\hyperlink{Forma Gassosa}{Forma Gassosa}, \hyperlink{Immagine Maggiore}{Immagine Maggiore}, \hyperlink{Invisibilità}{Invisibilità}}

\textbf{Azioni}

\emph{\textbf{Multiattacco.}} Il djinni effettua tre attacchi di scimitarra.

\emph{\textbf{Scimitarra.} Attacco con arma da mischia}: +11 a colpire, portata 1 m, un bersaglio.

\emph{Colpisce:} 12 (2d6 + 5) danni taglienti più 3 (1d6) danni da elettricità o suono (a scelta del gin).

\textbf{Reazione: \emph{Nube improvvisa}} il djinni subisce un colpo critico, diventa immediatamente di vapore fino alla fine del round. Costa 1 Azione tornare in forma solida.

\emph{\textbf{Creare Turbine.}} Un cilindro d'aria turbinante di 1 metro di raggio e alto 9 metri si forma magicamente in un punto visibile al djinni entro 36 metri da esso. Il turbine resta finché il djinni mantiene la concentrazione (come se si stesse concentrando su di un incantesimo). Qualsiasi creatura salvo il djinni che entri nel turbine deve riuscire un Tiro Salvezza di Tempra DC 23 o restare intralciata da esso. Il djinni può muovere il turbine di massimo 18 metri con un'Azione, e le creature intralciate dal turbine si muovono con esso. Il turbine termina se il djinni lo perde di vista.

Una creatura può usare una Azione per liberare una creatura intralciata dal turbine, compresa se stessa, riuscendo un Tiro Salvezza Tempra con Forza DC 22. Se la prova riesce, la creatura non è più intralciata e si sposta nello spazio più vicino all'esterno del turbine.

\textbf{Ecologia}
Ambiente: Qualsiasi (Piano dell'Aria)\\
Organizzazione: Solitario, coppia, compagnia (3-6) o banda (7-10)\\
\textbf{Categoria Tesoro}: Scimitarra Perfetta, U\\
\textbf{Descrizione}\\
I Djinn (singolare djinni) sono Geni provenienti dal Piano Elementale dell'Aria. Si dice che siano fatti di nuvole e abbiano la forza delle tempeste più potenti. Un Djinni è alto circa 3 metri e pesa circa 500 kg.

I Djinn disdegnano il Combattimento fisico, preferendo usare i loro poteri Magici e capacità aeree contro i nemici. Un Djinni sconfitto in Combattimento generalmente prende il volo e diventa un turbine per molestare chi lo insegue. Quando non ha altra scelta che combattere in mischia, la maggioranza dei Djinn preferisce impugnare Scimitarre a Due Mani Perfette.

Verso gli altri Geni, i Djinn vanno d'accordo con gli Janni e i Marid. Sono frequentemente in contrasto con gli Shaitan, e sono nemici giurati degli Efreeti, disprezzando questi Geni feroci più di qualsiasi altra delle Razze di Geni. Il conflitto tra gli Efreeti e i Djinn è così leggendario che molti incantatori tentano (con vari gradi di successo) di assicurarsi il servizio di un Djinni promettendogli aiuto nella causa contro gli odiati nemici.

\mostro{Efreeti}
\noindent
\begin{description}[noitemsep, topsep=0pt, parsep=0pt, partopsep=0pt, leftmargin=0cm, labelwidth=2.2cm]
	\item[\textbf{Taglia/Tipo:}] Grande elementale, malvagio
	\item[\textbf{Caratt.:}] \resizebox{0.5\linewidth+1.8cm}{!}{For 6 Des 1 Cos 7 Int 3 Sag 2 Car 3}
	\item[\textbf{Punti Ferita:}] 228,  \textbf{Difesa:} 27,  \textbf{Iniziativa:} +3
	\item[\textbf{Movimento:}] 12 m, volo 18 m
	\item[\textbf{Tiri Salvez.:}] \resizebox{0.5\linewidth+1.8cm}{!}{\resizebox{0.5\linewidth+1.8cm}{!}{Tempra +18, Riflessi +12, Volontà +13}}
	\item[\textbf{Imm. Danni:}] Fuoco
	\item[\textbf{Sensi:}] Scurovisione 36 m
	\item[\textbf{Linguaggi:}] Ignan
	\item[\textbf{Sfida:}] 11 (7200 PX)\smallskip
\end{description}

\emph{\textbf{Decesso Elementale.}} Se l'efreeti muore, il suo corpo si disintegra in un lampo di fuoco e uno sbuffo di fumo, lasciando dietro di sé solo l'equipaggiamento che l'efreeti stava indossando o trasportando.

\emph{\textbf{Incantesimi Innati.}} La caratteristica da incantatore innato dell'efreeti è il Carisma. Può lanciare in maniera innata i seguenti incantesimi, senza bisogno di componenti materiali:

A volontà: \emph{\hyperlink{Individuazione del Magico}{Individuazione del Magico}}

3/giorno ciascuno: \emph{\hyperlink{Ingrandire/Ridurre}{Ingrandire/Ridurre}, \hyperlink{Lingue}{Lingue}}

1/giorno ciascuno: \emph{\hyperlink{Evoca Elementale}{Evoca Elementale}} (solo elementale del fuoco), \emph{\hyperlink{Forma Gassosa}{Forma Gassosa}, \hyperlink{Immagine Maggiore}{Immagine Maggiore}}, \emph{\hyperlink{Invisibilità}{Invisibilità}, \hyperlink{Muro di Fuoco}{Muro di Fuoco}}

\textbf{Azioni}

\emph{\textbf{Multiattacco.}} L'efreeti effettua due attacchi di scimitarra o usa due volte Scagliare Fiamma.

\emph{\textbf{Scimitarra.} Attacco con arma da mischia}: +11 a colpire, portata 1 m, un bersaglio.

\emph{Colpisce:} 13 (2d6 + 6) danni taglienti più 7 (2d6) danni da fuoco.

\emph{\textbf{Scagliare Fiamma.} Attacco con arma a Distanza}: +12 a colpire, gittata 36 m, un bersaglio.

\emph{Colpisce:} 17 (5d6) danni da fuoco.

\textbf{Reazione: \emph{Attacco d'opportunità}}: efreeti effettua un attacco ad una creatura che attraversi o esca dalla sua portata di 1 metro.

\textbf{Ecologia}
Ambiente: Qualsiasi (Piano del Fuoco)\\
Organizzazione: Solitario, coppia, compagnia (3-6) o banda (7-12)\\
\textbf{Categoria Tesoro}: Falcione Perfetto, U\\
\textbf{Descrizione}\\
Gli Efreet (singolare Efreeti) sono Geni provenienti dal Piano del Fuoco. Sono alti 3,6 metri e pesano circa 1000 kg.

Gli Efreet hanno pochi alleati tra gli altri Geni: odiano i Djinni e li attaccano a vista, non sopportano i Marid e vedono i Janni come deboli e fragili. Gli Efreet spesso cooperano bene con gli Shaitan, eppure anche queste alleanze sono temporanee.

\mostro{Ghast}
\noindent
\begin{description}[noitemsep, topsep=0pt, parsep=0pt, partopsep=0pt, leftmargin=0cm, labelwidth=2.2cm]
	\item[\textbf{Taglia/Tipo:}] Media non morto, malvagio
	\item[\textbf{Caratt.:}] \resizebox{0.5\linewidth+1.8cm}{!}{For 3 Des 3 Cos 0 Int 0 Sag 0 Car -1}
	\item[\textbf{Punti Ferita:}] 51,  \textbf{Difesa:} 17,  \textbf{Iniziativa:} +3
	\item[\textbf{Movimento:}] 9 m
	\item[\textbf{Tiri Salvez.:}] \resizebox{0.5\linewidth+1.8cm}{!}{Tempra +3, Riflessi +5, Volontà +3}
	\item[\textbf{Res. Danni:}] da Vuoto
	\item[\textbf{Imm. Danni:}] Veleno
	\item[\textbf{Immunità:}] affascinato, affaticato
	\item[\textbf{Sensi:}] Scurovisione 18 m
	\item[\textbf{Linguaggi:}] Comune, Expiran
	\item[\textbf{Sfida:}] 2 (450 PX)\smallskip
\end{description}

\emph{\textbf{Fetore.}} Qualsiasi creatura che inizi il suo round entro 1 metro dal ghast deve riuscire un Tiro Salvezza di Tempra DC 14 o restare Nauseata (-1d6 a TC, TS e Prove) fino all'inizio del suo prossimo round. Se riesce il Tiro Salvezza, la creatura è immune al Fetore del ghast per le successive 24 ore.

\emph{\textbf{Ribellione allo Scacciare.}} Il ghast e tutti i ghoul entro 9 metri da esso hanno +1d6 ai Tiri Salvezza contro gli effetti che scacciano i non morti.

\textbf{Azioni}

\emph{\textbf{Artigli.} Attacco con arma da mischia}: +5 a colpire, portata 1 m, un bersaglio.

\emph{Colpisce:} 10 (2d6 + 3) danni taglienti. Se il bersaglio è una creatura, diversa da un non morto, deve riuscire un Tiro Salvezza su Tempra DC 14 o restare paralizzata per 1 minuto. Il bersaglio può ripetere il Tiro Salvezza al termine di ciascun suo round, terminando l'effetto se riesce il Tiro Salvezza.

\emph{\textbf{Morso.} Attacco con arma da mischia}: +5 a colpire, portata 1 m, una creatura.

\emph{Colpisce:} 12 (2d8 + 3) danni perforanti.

\emph{\textbf{Morso affamato.} Attacco con arma da mischia}: +5 a colpire, portata 1 m, una creatura. 2 Azioni

\emph{Colpisce:} 14 (3d6 + 3) danni perforanti, il ghast recupera la metà del danno in Punti Ferita.

\textbf{Ecologia}\\
Ambiente: Qualsiasi terreno\\
Organizzazione: Solitario, gruppo (2-4) o branco (7-12)\\
\textbf{Categoria Tesoro}: B\\
\textbf{Descrizione}\\
I ghast sono Ghoul con un legame più profondo con il Vuoto. La paralisi di un ghast ha effetto anche sugli Elfi. I ghast si aggirano in branchi o comandano gruppi di Ghoul comuni. Il fetore di morte e putrefazione che circonda queste creature è travolgente.

\mostro{Ghoul}
\noindent
\begin{description}[noitemsep, topsep=0pt, parsep=0pt, partopsep=0pt, leftmargin=0cm, labelwidth=2.2cm]
	\item[\textbf{Taglia/Tipo:}] Media non morto, malvagio
	\item[\textbf{Caratt.:}] \resizebox{0.5\linewidth+1.8cm}{!}{For 1 Des 2 Cos 0 Int -2 Sag 0 Car -2}
	\item[\textbf{Punti Ferita:}] 33,  \textbf{Difesa:} 15,  \textbf{Iniziativa:} +2
	\item[\textbf{Movimento:}] 9 m
	\item[\textbf{Tiri Salvez.:}] \resizebox{0.5\linewidth+1.8cm}{!}{\resizebox{0.5\linewidth+1.8cm}{!}{Tempra +3, Riflessi +3, Volontà +3}}
	\item[\textbf{Imm. Danni:}] Veleno
	\item[\textbf{Immunità:}] affascinato, affaticato
	\item[\textbf{Sensi:}] Scurovisione 18 m
	\item[\textbf{Linguaggi:}] Comune
	\item[\textbf{Sfida:}] 1 (200 PX)\smallskip
\end{description}

\textbf{Azioni}

\emph{\textbf{Artigli.} Attacco con arma da mischia}: +4 a colpire, portata 1 m, un bersaglio.

\emph{Colpisce:} 7 (2d4 + 2) danni taglienti, 1 danno da Sanguinamento. Se il bersaglio è una creatura, diversa da un elfo o un non morto, deve riuscire un Tiro Salvezza su Tempra DC 13 o restare paralizzata per 1 minuto. Il bersaglio può ripetere il Tiro Salvezza al termine di ciascun suo round, terminando l'effetto se riesce il Tiro Salvezza.

\emph{\textbf{Morso.} Attacco con arma da mischia}: +4 a colpire, portata 1 m, una creatura.

\emph{Colpisce:} 9 (2d6 + 2) danni perforanti.

\textbf{Ecologia}
Ambiente: Qualsiasi terreno\\
Organizzazione: Solitario, gruppo (2-4) o branco (7-12)\\
\textbf{Categoria Tesoro}: K\\
\textbf{Descrizione}\\
I ghoul sono non morti che frequentano i cimiteri e mangiano i cadaveri. Le leggende sostengono che i primi ghoul fossero umani cannibali che una fame innaturale ha riportato indietro dalla morte, oppure umani che in vita si nutrivano dei resti in decomposizione dei loro simili e che morirono (e poi rinacquero) a causa di un'orrenda malattia; la vera origine di questi non morti necrofagi è incerta.

I ghoul si appostano ai margini della civilizzazione (dentro o nei pressi dei cimiteri o nelle fogne cittadine) dove possono reperire ampie scorte del loro cibo preferito. Sebbene preferiscano i corpi in putrefazione e spesso seppelliscano le loro vittime per migliorarne il sapore, mangiano i morti freschi se hanno abbastanza fame.

Anche se molti ghoul di superficie vivono in modo primitivo, delle voci parlano di città di ghoul nelle profondità del sottosuolo comandate da sacerdoti che adorano antiche divinità crudeli o strani signori dei demoni della fame. Questi ghoul \emph{civilizzati} non sono meno orribili nelle loro abitudini alimentari, e in effetti il loro concetto di tavola ben imbandita per banchetti è forse anche più orrendo dell'idea di un pasto fresco prelevato da una bara.

\mostro{Ghoul, Nero}
\noindent
\begin{description}[noitemsep, topsep=0pt, parsep=0pt, partopsep=0pt, leftmargin=0cm, labelwidth=2.2cm]
	\item[\textbf{Taglia/Tipo:}] Media non morto, malvagio
	\item[\textbf{Caratt.:}] \resizebox{0.5\linewidth+1.8cm}{!}{For 4 Des 2 Cos 2 Int 0 Sag 1 Car -2}
	\item[\textbf{Punti Ferita:}] 125,  \textbf{Difesa:} 22,  \textbf{Iniziativa:} +2
	\item[\textbf{Movimento:}] 12 m
	\item[\textbf{Tiri Salvez.:}] \resizebox{0.5\linewidth+1.8cm}{!}{Tempra +8, Riflessi +8, Volontà +7}
	\item[\textbf{Imm. Danni:}] Veleno, da Vuoto, da critico, sanguinamento,
	\item[\textbf{Immunità:}] affascinato, affaticato,
	\item[\textbf{Res. Danni:}] armi non magiche o d'argento
	\item[\textbf{Sensi:}] Scurovisione 18 m
	\item[\textbf{Linguaggi:}] Comune, Expiran
	\item[\textbf{Sfida:}] 6 (2300 PX)\smallskip
\end{description}

\textbf{\emph{Aura nefasta}}: il Ghoul Nero emana costantemente un aura attorno a se che indebolisce le difese di chiunque tranne che di altri ghoul. Ogni due round di permanenza nell'aura di 12 metri di raggio attorno al Ghoul Nero si cumula un -1 a tutti i TS, quando ci si allontana dal Ghoul Nero si recupera 1 punto a round.

\textbf{Azioni}

\emph{\textbf{Artigli.} Attacco con arma da mischia}: +8 a colpire, portata 1 m, un bersaglio.

\emph{Colpisce:} 15 (2d10 + 4) danni taglienti, 2 danno da Sanguinamento. Se il bersaglio è una creatura, diversa da un elfo o un non morto, deve riuscire un Tiro Salvezza su Tempra DC 17 o restare paralizzata per 1 minuto. Il bersaglio può ripetere il Tiro Salvezza al termine di ciascun suo round, terminando l'effetto se riesce il Tiro Salvezza.

\emph{\textbf{Morso.} Attacco con arma da mischia}: +9 a colpire, portata 1 m, una creatura.

\emph{Colpisce:} 18 (3d8 + 6) danni perforanti, 1 da Sanguinamento, Malattia del Ghoul

\textbf{Reazione: \emph{Attacco d'opportunità}}: il Ghoul Nero effettua un attacco ad una creatura che attraversi o esca dalla sua portata di 1 metro.

\emph{Malattia del Ghoul:} 3 giorni, TS Tempra DC 18, 6 ore, 3 successi, -1 Costituzione, ti trasformi in un Ghoul

\textbf{Ecologia}
Ambiente: Qualsiasi terreno\\
Organizzazione: Gruppo (4-8) o branco (14-24)\\
\textbf{Categoria Tesoro}: B\\
\textbf{Descrizione}\\
Il Ghoul Nero rappresenta una delle elite evolutive dei Ghoul. Solitamente a capo di un gruppo almeno un ghoul putrescente di circa 18 ghoul.

%\begin{center}
%\includegraphics[width=0.45\textwidth]{immagini/Pazin_Burgmuseum_-_Waffen_1.png}
%\end{center}

\mostro{Ghoul, Madre}
\noindent
\begin{description}[noitemsep, topsep=0pt, parsep=0pt, partopsep=0pt, leftmargin=0cm, labelwidth=2.2cm]
	\item[\textbf{Taglia/Tipo:}] Media non morto, malvagio
	\item[\textbf{Caratt.:}] \resizebox{0.5\linewidth+1.8cm}{!}{For 0 Des 3 Cos 2 Int 2 Sag 1 Car 2}
	\item[\textbf{Punti Ferita:}] 107,  \textbf{Difesa:} 21,  \textbf{Iniziativa:} +3
	\item[\textbf{Movimento:}] 9 m
	\item[\textbf{Tiri Salvez.:}] \resizebox{0.5\linewidth+1.8cm}{!}{Tempra +7, Riflessi +8, Volontà +6}
	\item[\textbf{Imm. Danni:}] Veleno, Vuoto, da critico, sanguinamento
	\item[\textbf{Immunità:}] affascinato, affaticato
	\item[\textbf{Res. Danni:}] armi non magiche
	\item[\textbf{Sensi:}] Scurovisione 18 m
	\item[\textbf{Linguaggi:}] Comune, Expiran
	\item[\textbf{Sfida:}] 5 (1800 PX)\smallskip
\end{description}

\textbf{Azioni}

\emph{\textbf{Artigli.} Attacco con arma da mischia}: +6 a colpire, portata 1 m, un bersaglio.

\emph{Colpisce:} 12 (2d6 + 6) danni taglienti, 2 danno da Sanguinamento. Se il bersaglio è una creatura diverso da un non morto, deve riuscire un Tiro Salvezza su Tempra DC 17 o restare paralizzata per 1 minuto. Il bersaglio può ripetere il Tiro Salvezza al termine di ciascun suo round, terminando l'effetto se riesce il Tiro Salvezza. Se la creatura fallisce il TS allora è vittima della maledizione del Ghoul. Entro 1d3+1 giorni si trasformerà in un Ghoul. E' necessario un Scacciare Maledizioni DC 19 entro la trasformazione per evitare la trasformazione.

\emph{\textbf{Morso.} Attacco con arma da mischia}: +6 a colpire, portata 1 m, una creatura.

\emph{Colpisce:} 8 (2d6 + 2) danni perforanti.

\textbf{Ecologia}
Ambiente: Qualsiasi terreno\\
Organizzazione: Clan (7-12+)\\
\textbf{Categoria Tesoro}: I\\
\textbf{Descrizione}\\
La Madre Ghoul è solitamente a capo di un clan di ghoul che può raggiungere anche diverse decine di membri. Rispettata e temuta è solitamente tra i ghoul evoluti più intelligenti e molto apprezzata per la sua capacità di poter trasformare in ghoul i viventi. La loro tattica prevede di ferire e non uccidere diverse persone così che tornate a casa e poi trasformati possano attaccare ed uccidere tutto il villaggio.

%\begin{center}
%\includegraphics[width=0.45\textwidth]{immagini/Mexican_machete.png}
%\end{center}

\mostro{Ghoul, putrescente}
\noindent
\begin{description}[noitemsep, topsep=0pt, parsep=0pt, partopsep=0pt, leftmargin=0cm, labelwidth=2.2cm]
	\item[\textbf{Taglia/Tipo:}] Grande non morto, malvagio
	\item[\textbf{Caratt.:}] \resizebox{0.5\linewidth+1.8cm}{!}{For 1 Des 2 Cos 3 Int -1 Sag 0 Car -2}
	\item[\textbf{Punti Ferita:}] 89,  \textbf{Difesa:} 19,  \textbf{Iniziativa:} +2
	\item[\textbf{Movimento:}] 6 m
	\item[\textbf{Tiri Salvez.:}] \resizebox{0.5\linewidth+1.8cm}{!}{Tempra +7, Riflessi +6, Volontà +4}
	\item[\textbf{Imm. Danni:}] Veleno, sanguinamento, da critico, da Vuoto
	\item[\textbf{Immunità:}] affascinato, affaticato
	\item[\textbf{Res. Danni:}] armi non magiche o d'argento
	\item[\textbf{Sensi:}] Scurovisione 36 m
	\item[\textbf{Linguaggi:}] Comune, Expiran
	\item[\textbf{Sfida:}] 4 (1100 PX)\smallskip
\end{description}

\textbf{\emph{Rigenerazione}}. Il Ghoul Putrescente rigenera 5 Punti Ferita a round tranne se è in piena luce solare o ha subito danni da Luce nel round precedente. Se il Ghoul Putrescente è in un cimitero recupera 10 Punti Ferita a round.

\textbf{Azioni}

\emph{\textbf{Artigli.} Attacco con arma da mischia}: +6 a colpire, portata 2 m, un bersaglio.

\emph{Colpisce:} 12 (2d10 + 2) danni taglienti, 1 danno da Sanguinamento. Se il bersaglio è una creatura, diversa da un non morto, deve riuscire un Tiro Salvezza su Tempra DC 15 o restare paralizzata per 1 minuto.

\emph{\textbf{Morso.} Attacco con arma da mischia}: +6 a colpire, portata 1 m, una creatura.

\emph{Colpisce:} 10 (2d8 + 2) danni perforanti.

\textbf{Reazione: \emph{Attacco d'opportunità}}: il Ghoul Putrescente effettua un attacco ad una creatura che attraversi o esca dalla sua portata di 1 metro.

\emph{\textbf{Aura di Sofferenza.}}: il Ghoul Putrescente emana un aura di 6 metri intorno a lui, ogni attacco di ghoul andato a segno causa automaticamente un danno critico. Attivare questa aura costa 2 Azioni e dura fino all'inizio del round successivo.

\textbf{Ecologia}
Ambiente: Qualsiasi terreno\\
Organizzazione: Gruppo (4-8) o branco (10-18)\\
\textbf{Categoria Tesoro}: Nessuno\\
\textbf{Descrizione}\\
I Ghoul Putrescenti sono una delle tante l'evoluzione dei Ghoul. Il contatto continuo con l'energia negativa ed il nutrirsi per secoli di cadaveri di ogni genere lo hanno reso più grande, forte e capace di infliggere e fare infliggere le ferite più pericolose.

\mostro{Gigante delle Colline}
\noindent
\begin{description}[noitemsep, topsep=0pt, parsep=0pt, partopsep=0pt, leftmargin=0cm, labelwidth=2.2cm]
	\item[\textbf{Taglia/Tipo:}] Enorme gigante, malvagio
	\item[\textbf{Caratt.:}] \resizebox{0.5\linewidth+1.8cm}{!}{For 5 Des -1 Cos 4 Int -3 Sag -1 Car -2}
	\item[\textbf{Punti Ferita:}] 109,  \textbf{Difesa:} 17,  \textbf{Iniziativa:} -1
	\item[\textbf{Movimento:}] 12 m
	\item[\textbf{Tiri Salvez.:}] \resizebox{0.5\linewidth+1.8cm}{!}{Tempra +9, Riflessi +4, Volontà +4}
	\item[\textbf{Linguaggi:}] Gigante
	\item[\textbf{Sfida:}] 5 (1800 PX)\smallskip
\end{description}

\textbf{Azioni}

\emph{\textbf{Multiattacco.}} Il gigante effettua due attacchi con il randello pesante.

\emph{\textbf{Randello Pesante.} Attacco con arma da mischia}: +7 a colpire, portata 3 m, un bersaglio.

\emph{Colpisce:} 18 (3d8 + 5) danni contundenti.

\emph{\textbf{Ampio Fendente.} Attacco con arma da mischia}: +7 a colpire, portata 3 metri, con un singolo attacco può colpire due creature in mischia vicine tra loro.

\emph{\textbf{Sasso.} Attacco con arma a Distanza}: +5 a colpire, gittata 18m, un bersaglio.

\emph{Colpisce:} 21 (3d10 + 5) danni contundenti.

\textbf{Ecologia}\\
Ambiente: Colline Temperate\\
Organizzazione: Solitario, gruppo (2-5), banda (6-8), gruppo di razzia (9-12 più 1d4 Lupi Crudeli) o tribù (13-30 più 35\% non combattente più 1 capo combattente di 4°-6° livello, 11-16 Lupi Crudeli, 1-4 Ogre e 13-20 schiavi orchi)\\
\textbf{Categoria Tesoro}: Armatura di Pelle, Randello Pesante, B\\
\textbf{Descrizione}\\
I giganti di Collina hanno pelle che varia dal marrone chiaro al rossastro, capelli castani o neri, ed occhi dello stesso colore. Indossano strati di pelli rozzamente conciate con ancora il pelo. Raramente lavano o riparano i propri indumenti, e preferiscono semplicemente aggiungere nuovi strati man mano che i vecchi si logorano. Gli adulti sono alti circa 3 metri e pesano più o meno 550 kg. I giganti di Collina possono vivere fino a 200 anni, anche se raramente raggiungono quest'età.

I giganti di Collina preferiscono combattere dall'alto di sporgenze e rupi, da dove possono colpire gli avversari con rocce e massi, limitando così il rischio personale. Amano effettuare attacchi di oltrepassare contro creature più piccole all'inizio del combattimento, e solo dopo prendono posizione e iniziano a roteare i loro massicci randelli.

I giganti di Collina sono per natura nomadi e preferiscono viaggiare da un luogo all'altro per razziare e saccheggiare. Sebbene gradiscano di più i climi temperati, non disdegnano di viaggiare lontano dal loro ambiente favorito, se la razzia è abbondante e prospera. Si tratta, nel complesso, di creature molto egoiste, che raramente affrontano battaglie che non siano sicuri di vincere. I giganti delle colline sono noti per l'abitudine di spingersi l'un l'altro se devono confrontarsi con avversari temibili e non esitano a sacrificare un compagno per salvarsi la pelle. Bande erranti di giganti di Collina sono diffuse sulle colline temperate, e la loro costante aggressività li rende uno dei pericoli più temuti in questo ambiente.

I giganti di Collina solitari e non malvagi sono molto rari, ma li si può trovare qualche volta in altre società umanoidi, anche se non sono quasi mai accettati nelle città principali o nei centri popolati. Si trovano a proprio agio come lavoratori e soldati nelle remote città di frontiera, e spesso fungono da rudimentali diplomatici per negoziare con le bande di giganti di Collina razziatori. Sfortunatamente, i giganti di Collina che abbandonano il proprio stile di vita razziale per la civiltà vengono derisi e spesso uccisi a vista dai loro fratelli nomadi. Tuttavia, questi giganti di Collina \emph{civilizzati} possono trovare il proprio posto nella società e molti sono riusciti a vivere un'esistenza pacifica e tranquilla.

\mostro{Gigante del Fuoco}
\noindent
\begin{description}[noitemsep, topsep=0pt, parsep=0pt, partopsep=0pt, leftmargin=0cm, labelwidth=2.2cm]
	\item[\textbf{Taglia/Tipo:}] Enorme gigante, malvagio
	\item[\textbf{Caratt.:}] \resizebox{0.5\linewidth+1.8cm}{!}{For 7 Des -1 Cos 6 Int 0 Sag 2 Car 1}
	\item[\textbf{Punti Ferita:}] 187,  \textbf{Difesa:} 23,  \textbf{Iniziativa:} +0
	\item[\textbf{Movimento:}] 9 m
	\item[\textbf{Tiri Salvez.:}] \resizebox{0.5\linewidth+1.8cm}{!}{\resizebox{0.5\linewidth+1.8cm}{!}{Tempra +15, Riflessi +8, Volontà +11}}
	\item[\textbf{Comp.:}] Atletica +11, Consapevolezza +6
	\item[\textbf{Linguaggi:}] Gigante
	\item[\textbf{Sfida:}] 9 (5000 PX)\smallskip
\end{description}

\textbf{Azioni}

\emph{\textbf{Multiattacco.}} Il gigante effettua due attacchi con lo spadone.

\emph{\textbf{Spadone.} Attacco con arma da mischia}: +11 a colpire, portata 3 m, un bersaglio.

\emph{Colpisce:} 28 (6d6 + 7) danni taglienti.

\emph{\textbf{Ampio Fendente.} Attacco con arma da mischia}: +11 a colpire, portata 3 metri, con un singolo attacco può colpire due creature in mischia vicine tra loro.

\emph{\textbf{Sasso.} Attacco con arma a Distanza}: +10 a colpire, gittata 18m, un bersaglio.

\emph{Colpisce:} 29 (4d10 + 7) danni contundenti.

\textbf{Reazione: \emph{Attacco d'opportunità}}: il gigante del fuoco effettua un attacco ad una creatura che attraversi o esca dalla sua portata di 3 metri.

\emph{\textbf{Arrabbiato:}} il gigante del fuoco convoglia la sua energia sull'arma, questa causa +2d6 danni da fuoco fino al termine del combattimento.

\textbf{Ecologia}
Ambiente: Montagne calde\\
Organizzazione: Solitario, gruppo (2-5), banda (6-12 più un 35\% non combattenti e 1 adepto o Devoto di 1°-2° livello), gruppo di razziatori (6-12 più 1 adepto o mago di 3°-5° livello, 2-5 Segugi Infernali e 2-3 Troll o Ettin) o tribù (20-30 più 1 adepto, mago o Devoto di 6°-7° livello; 1 re Guerriero o guardiaboschi di 8°-9° livello; e 17-38 Segugi Infernali, 12-22 Troll, 7-12 Ettin e 1-2 Draghi Rossi Giovani)\\
\textbf{Categoria Tesoro}: Mezza Armatura, Spadone, P\\
\textbf{Descrizione}\\
I giganti del fuoco sono i giganti più rigidi e marziali, sempre pronti alla guerra e a trattare brutalmente chiunque incontrino. La loro rigida struttura di comando prevede soldati, ufficiali e persino generali, e che tutti obbediscano agli ordini del loro re senza discutere.

I giganti del fuoco hanno capelli arancione brillante che splendono e scintillano come se fossero in fiamme. Un maschio adulto è alto tra i 3,6 e i 4,8 metri, con una cassa toracica di circa 2,7 metri, e pesa circa 3.500 kg. Le femmine sono leggermente più basse e snelle. I giganti del fuoco possono vivere fino a 350 anni.

I giganti del fuoco indossano abiti di tessuti robusti o di pelle di color arancione, giallo, nero o rosso. I guerrieri indossano elmi e mezze armature di acciaio brunito e impugnano grandi spadoni che mulinano per il campo di battaglia. In gruppi numerosi, i giganti del fuoco combattono con tattiche di gruppo brutali ed efficienti, e non esitano a sacrificare qualche compagno per tendere un'imboscata al nemico.

I giganti del fuoco preferiscono i luoghi caldi: più caldi sono meglio è. Si possono trovare nei deserti, nei vulcani, nelle fonti termali e nelle profondità della terra nei pressi di camini lavici. Vivono in castelli, insediamenti fortificati o grandi caverne, e l'architettura di questi luoghi riflette il loro stile di vita rigido e militaristico, con gli ufficiali che abitano in alloggi migliori di quelli dei loro sottoposti.

\mostro{Gigante del Gelo}
\noindent
\begin{description}[noitemsep, topsep=0pt, parsep=0pt, partopsep=0pt, leftmargin=0cm, labelwidth=2.2cm]
	\item[\textbf{Taglia/Tipo:}] Enorme gigante, malvagio
	\item[\textbf{Caratt.:}] \resizebox{0.5\linewidth+1.8cm}{!}{For 6 Des -1 Cos 5 Int -1 Sag 0 Car 1}
	\item[\textbf{Punti Ferita:}] 167,  \textbf{Difesa:} 21,  \textbf{Iniziativa:} -1
	\item[\textbf{Movimento:}] 12 m
	\item[\textbf{Tiri Salvez.:}] \resizebox{0.5\linewidth+1.8cm}{!}{\resizebox{0.5\linewidth+1.8cm}{!}{Tempra +13, Riflessi +7, Volontà +8}}
	\item[\textbf{Comp.:}] Atletica +9
	\item[\textbf{Linguaggi:}] Gigante
	\item[\textbf{Sfida:}] 8 (3900 PX)\smallskip
\end{description}

\textbf{Azioni}

\emph{\textbf{Multiattacco.}} Il gigante effettua due attacchi con l'ascia bipenne.

\emph{\textbf{Ascia Bipenne.} Attacco con arma da mischia}: +10 a colpire, portata 3 m, un bersaglio.

\emph{Colpisce:} 25 (3d12 + 6) danni taglienti.

\emph{\textbf{Ampio Fendente.} Attacco con arma da mischia}: +10 a colpire, portata 3 metri, con un singolo attacco può colpire due creature in mischia vicine tra loro.

\emph{\textbf{Sasso.} Attacco con arma a Distanza}: +9 a colpire, gittata 18m, un bersaglio.

\emph{Colpisce:} 28 (4d10 + 6) danni contundenti.

\textbf{Reazione: \emph{Attacco d'opportunità}}: il gigante del freddo effettua un attacco ad una creatura che attraversi o esca dalla sua portata di 3 metri.

\emph{\textbf{Arrabbiato:}} il Gigante del Gelo canalizza le sue energie attraverso l'arma. L'arma causa un 2d6 di danni aggiuntivi da freddo fino alla fine del combattimento.

\textbf{Ecologia}\\
Ambiente: Montagne fredde\\
Organizzazione: Solitario, banda (3-5), gruppo (6-12 più 35\% non combattenti e 1 mago o Devoto di 1°-2° livello), gruppo di razziatori (6-12 più 35\% non combattenti, 1 Devoto o mago di 3°-5° livello, 1-4 Lupi Invernali e 2-3 Ogre) o tribù (21-30 più 1 adepto, mago o Devoto di 6°-7° livello; 1 jarl Barbaro o guardiaboschi 7°-9° livello; e 15-36 Lupi Invernali, 13-22 Ogre e 1-2 Draghi Bianchi Giovani)\\
\textbf{Categoria Tesoro}: Giaco di Maglia, Ascia Bipenne, R\\
\textbf{Descrizione}\\
Un gigante del gelo ha capelli azzurri o giallo sporco, e occhi in genere dello stesso colore. Si vestono con pelli e pellicce, adornandosi con qualsiasi gioiello possiedano. I giganti del gelo combattenti indossano anche giachi di maglia ed elmi di metallo decorati con corna e piume. Un maschio adulto è alto 5 metri e pesa circa 1.400 kg. Le femmine sono leggermente più basse e snelle, ma per il resto sono identiche ai maschi. I giganti del gelo possono vivere fino a 250 anni.

I giganti del gelo sono molto temuti, poiché la brama di distruzione e guerra ed il loro comportamento sprezzante li spingono a manifestazioni di brutalità sempre maggiori. I giganti del gelo iniziano attaccando a distanza, scagliando rocce finché finiscono le munizioni o l'avversario si avvicina, poi lo affrontano con le loro enormi asce. Una delle tattiche preferite è tendere un'imboscata nascondendosi sotto la neve al di sopra di un pendio ghiacciato o innevato, dove gli avversari avranno difficoltà a raggiungerli, e poi iniziano causando una valanga prima di scendere in battaglia. I giganti del gelo possono nascondersi molto bene negli ambienti nevosi e sono dei maestri nella furtività nel loro dominio.

I giganti del gelo sopravvivono cacciando e razziando da soli, dato che vivono in ambienti freddi e desolati. I gruppi di giganti del gelo sono divisi quasi equamente tra quelli che vivono in insediamenti di fortuna o castelli abbandonati e quelli che vagabondano per il gelido nord, come nomadi in cerca di bottino e provviste. I capi dei giganti del gelo si chiamano jarl e richiedono obbedienza assoluta ai loro seguaci. In ogni momento uno jarl può essere sfidato in combattimento per il comando della tribù. Queste sfide tipicamente finiscono con la morte di uno dei contendenti. Un singolo jarl può spesso contare su una dozzina o più di tribù più piccole di giganti del gelo come estensione della sua. In questi casi, i capi delle tribù minori sono noti come capitani o signori della guerra.

I giganti del gelo amano prendere prigionieri e li usano sia come schiavi che come materia prima. Di solito ogni gruppo di giganti del gelo tiene 1-2 schiavi umanoidi incatenati ad un addestratore di schiavi: il più meschino e crudele del gruppo dopo lo jarl. Hanno anche una certa passione per gli animali domestici mostruosi: Draghi Bianchi e Lupi Invernali sono scelte popolari, ma nella tana di un gigante del gelo si possono trovare anche Remorhaz e Yeti.

\mostro{Gigante delle Nuvole}
\noindent
\begin{description}[noitemsep, topsep=0pt, parsep=0pt, partopsep=0pt, leftmargin=0cm, labelwidth=2.2cm]
	\item[\textbf{Taglia/Tipo:}] Enorme gigante, buono (50\%) o malvagio (50\%)
	\item[\textbf{Caratt.:}] \resizebox{0.5\linewidth+1.8cm}{!}{For 8 Des 0 Cos 6 Int 1 Sag 3 Car 3}
	\item[\textbf{Punti Ferita:}] 187,  \textbf{Difesa:} 24,  \textbf{Iniziativa:} +1
	\item[\textbf{Movimento:}] 12 m
	\item[\textbf{Tiri Salvez.:}] \resizebox{0.5\linewidth+1.8cm}{!}{\resizebox{0.5\linewidth+1.8cm}{!}{Tempra +15, Riflessi +9, Volontà +12}}
	\item[\textbf{Comp.:}] Percepire Emozioni +7
	\item[\textbf{Linguaggi:}] Comune, Gigante
	\item[\textbf{Sfida:}] 9 (5000 PX)\smallskip
\end{description}

\emph{\textbf{Incantesimi Innati.}} La caratteristica da incantatore del gigante è il Carisma. Il gigante può lanciare questi incantesimi in maniera innata, senza bisogno di componenti materiali:

A volontà: \emph{\hyperlink{Individuazione del Magico}{Individuazione del Magico}, \hyperlink{Luce}{Luce}, \hyperlink{Nube di Nebbia}{Nube di Nebbia}}

3/giorno ciascuno: \emph{Caduta Piuma, \hyperlink{Passo Velato}{Passo Velato}, \hyperlink{Telecinesi}{Telecinesi}}

1/giorno ciascuno: \emph{\hyperlink{Controllare Tempo Atmosferico}{Controllare Tempo Atmosferico}, \hyperlink{Forma Gassosa}{Forma Gassosa}}

\emph{\textbf{Olfatto Affinato.}} Il gigante ha +1d6 alle prove di Consapevolezza basate sull'olfatto.

\textbf{Azioni}

\emph{\textbf{Multiattacco.}} Il gigante effettua due attacchi con la Mazza chiodata.

\emph{\textbf{Mazza chiodata.} Attacco con arma da mischia}: +11 a colpire, portata 3 m, un bersaglio.

\emph{Colpisce:} 21 (3d8 + 8) danni perforanti.

\emph{\textbf{Ampio Fendente.} Attacco con arma da mischia}: +11 a colpire, portata 3 metri, con un singolo attacco può colpire due creature in mischia vicine tra loro.

\emph{\textbf{Sasso.} Attacco con arma a Distanza}: +11 a colpire, gittata 18m, un bersaglio.

\emph{Colpisce:} 30 (4d10 + 8) danni contundenti.

\textbf{Reazione: \emph{Attacco d'opportunità}}: il gigante delle nubi effettua un attacco ad una creatura che attraversi o esca dalla sua portata di 3 metri.

\emph{\textbf{Arrabbiato:}} il Gigante delle Nubi agita l'arma sopra sulla testa evocando nubi tempestose e lanciando l'incantesimo \hyperlink{Invocare il Fulmine}{Invocare il Fulmine}. Costa 2 Azioni.

\textbf{Ecologia}\\
Ambiente: Montagne Temperate\\
Organizzazione: Solitario, gruppo (2-5), famiglia (2-5 più 35\% non combattenti più 1 mago o Devoto di 4°-7° livello e 2-5 Grifoni) o tribù (6-20 più 1 oracolo mago o Devoto di 7°-12° livello e 2-5 Grifoni)\\
\textbf{Categoria Tesoro}: Giaco di Maglia, Mazza chiodata, U\\
\textbf{Descrizione}\\
Il colore pelle dei giganti delle nuvole varia dal bianco latte al blu polvere. I maschi adulti sono alti circa 5,3 metri e pesano approssimativamente 2.500 kg. Le femmine sono leggermente più basse e snelle. I giganti delle nuvole possono vivere fino a 400 anni, vestono con abiti preziosi e gioielli. Per molti l'aspetto indica lo status. Migliori sono i vestiti e più raffinati i gioielli, più importante è chi li indossa. Inoltre apprezzano la musica, e la maggioranza suona uno o più strumenti (l'arpa è uno dei preferiti).

I giganti delle nuvole possono avere Tratti insolitamente vari; circa metà sono buoni e metà malvagi. I giganti delle nuvole buoni costruiscono strade che collegano i loro insediamenti con le strade degli umani per promuovere il commercio. Non è insolito vedere un gigante delle nuvole buono camminare tra gli uomini, ad esempio, in una città umana nei pressi di un'alta catena montuosa. I giganti delle nuvole malvagi tendono a non creare insediamenti stabili e anzi preferiscono vivere in rozzi rifugi su alti picchi, da cui scendono solo per depredare i villaggi di quello di cui potrebbero aver bisogno. Queste due filosofie portano spesso allo scoppio di guerre violente e durature tra tribù vicine.

Sono molte le leggende che parlano di magiche città dei giganti delle nuvole situate tra le nuvole stesse, che fluttuano sui venti e circumnavigano il mondo. Mentre i giganti delle nuvole riconoscono che si tratta per lo più di fantasie, alcuni sostengono di averle viste e hanno dedicato la loro intera esistenza a ritrovarle.

\mostro{Gigante di Pietra}
\noindent
\begin{description}[noitemsep, topsep=0pt, parsep=0pt, partopsep=0pt, leftmargin=0cm, labelwidth=2.2cm]
	\item[\textbf{Taglia/Tipo:}] Enorme gigante, neutrale
	\item[\textbf{Caratt.:}] \resizebox{0.5\linewidth+1.8cm}{!}{For 6 Des 2 Cos 5 Int 0 Sag 1 Car -1}
	\item[\textbf{Punti Ferita:}] 148,  \textbf{Difesa:} 23,  \textbf{Iniziativa:} +2
	\item[\textbf{Movimento:}] 12 m
	\item[\textbf{Tiri Salvez.:}] \resizebox{0.5\linewidth+1.8cm}{!}{\resizebox{0.5\linewidth+1.8cm}{!}{Tempra +12, Riflessi +9, Volontà +8}}
	\item[\textbf{Comp.:}] Atletica +12
	\item[\textbf{Sensi:}] Scurovisione 18 m
	\item[\textbf{Linguaggi:}] Gigante
	\item[\textbf{Sfida:}] 7 (2900 PX)\smallskip
\end{description}

\emph{\textbf{Mimetismo di Pietra.}} Il gigante ha +1d6 alle prove di Furtività (Nascondersi) effettuate per nascondersi su terreni rocciosi.

\textbf{Azioni}

\emph{\textbf{Multiattacco.}} Il gigante effettua due attacchi con il randello pesante.

\emph{\textbf{Randello Pesante.} Attacco con arma da mischia}: +9 a colpire, portata 5 metri, un bersaglio.

\emph{Colpisce:} 19 (3d8 + 6) danni contundenti.

\emph{\textbf{Ampio Fendente.} Attacco con arma da mischia}: +9 a colpire, portata 3 metri, con un singolo attacco può colpire due creature in mischia vicine tra loro.

\emph{\textbf{Sasso.} Attacco con arma a Distanza}: +8 a colpire, gittata 18m, un bersaglio.

\emph{Colpisce:} 28 (4d10 + 6) danni contundenti. Se il bersaglio è una creatura, deve riuscire un Tiro Salvezza di Tempra DC 19 o cadere prona.

\textbf{Reazione: \emph{Attacco d'opportunità}}: il gigante di pietra effettua un attacco ad una creatura che attraversi o esca dalla sua portata di 3 metri.

\textbf{Reazione: \emph{Afferrare Sassi.}} Se un sasso o un simile oggetto viene scagliato al gigante, il gigante può, riuscendo un Tiro Salvezza su Riflessi DC 10, afferrare il proiettile e non subire danni contundenti da esso.

\emph{\textbf{Arrabbiato:}} il Gigante di Pietra concentra le sue energie rendendo la pelle dura come pietra. Fino alla fine del round successivo acquisisce una Riduzione del danno pari a 13. Costa 2 Azioni

\textbf{Ecologia}
Ambiente: Montagne temperate\\
Organizzazione: Solitario, gruppo (2-5), banda (4-8), gruppo di caccia (9-12 più 1 Anziano) o tribù (13-30 più 35\% non combattenti, 1-3 Anziani e 4-6 Orsi Crudeli)\\
\textbf{Categoria Tesoro}: Randello Pesante Gigante, P\\
\textbf{Descrizione}\\
I giganti di Pietra preferiscono spessi indumenti di cuoio, tinti con tonalità di marrone e grigio per confondersi con la pietra che li circonda. Gli adulti sono alti circa 3,6 metri, pesano circa 750 kg e possono vivere fino a 800 anni.

I giganti di Pietra, se possibile, combattono a distanza, ma se non possono evitare la mischia usano giganteschi randelli di pietra. Una delle tattiche favorite dai giganti di Pietra è di stare immobili, mimetizzandosi con il paesaggio, per poi avanzare scagliando rocce e sorprendere i nemici.

I giganti di Pietra preferiscono vivere in enormi caverne sulle cime rocciose. Raramente vivono a più di qualche giorno di viaggio da altre bande di giganti di Pietra e allevano greggi condivisi di capre e altro bestiame.

I giganti di Pietra più vecchi tendono ad allontanarsi dalla tribù per molto tempo, per vivere in solitudine da qualche parte o tentando di inserirsi in altre civiltà umanoidi. Dopo decadi di esilio auto imposto, chi fa ritorno è noto come Gigante delle Rocce Anziano.

\mostro{Gigante delle Tempeste}
\noindent
\begin{description}[noitemsep, topsep=0pt, parsep=0pt, partopsep=0pt, leftmargin=0cm, labelwidth=2.2cm]
	\item[\textbf{Taglia/Tipo:}] Enorme gigante, buono
	\item[\textbf{Caratt.:}] \resizebox{0.5\linewidth+1.8cm}{!}{For 9 Des 2 Cos 5 Int 3 Sag 4 Car 4}
	\item[\textbf{Punti Ferita:}] 262,  \textbf{Difesa:} 31,  \textbf{Iniziativa:} +3
	\item[\textbf{Movimento:}] 15 m, nuoto 15 m
	\item[\textbf{Tiri Salvez.:}] \resizebox{0.5\linewidth+1.8cm}{!}{\resizebox{0.5\linewidth+1.8cm}{!}{Tempra +18, Riflessi +15, Volontà +17}}
	\item[\textbf{Comp.:}] Arcana +8, Atletica +14, Storia +8
	\item[\textbf{Res. Danni:}] Freddo
	\item[\textbf{Imm. Danni:}] Elettricità, Suono
	\item[\textbf{Linguaggi:}] Comune, Gigante
	\item[\textbf{Sfida:}] 13 (10000 PX)\smallskip
\end{description}

\emph{\textbf{Anfibio.}} Il gigante può respirare aria e acqua.

\emph{\textbf{Incantesimi Innati.}} La caratteristica da incantatore del gigante è il Carisma. Il gigante può lanciare questi incantesimi in maniera innata, senza bisogno di componenti materiali:

A volontà: \emph{\hyperlink{Caduta Piuma}{Caduta Piuma}, individuazione del magico,} \emph{levitazione, \hyperlink{Luce}{Luce}}

3/giorno ciascuno: \emph{\hyperlink{Controllare Tempo Atmosferico}{Controllare Tempo Atmosferico}, \hyperlink{Respirare Sott'Acqua}{Respirare Sott'Acqua}}

\textbf{Azioni}

\emph{\textbf{Multiattacco.}} Il gigante effettua due attacchi con lo spadone.

\emph{\textbf{Spadone.} Attacco con arma da mischia}: +12 a colpire, portata 3 m, un bersaglio.

\emph{Colpisce:} 30 (6d6 + 9) danni taglienti.

\emph{\textbf{Ampio Fendente.} Attacco con arma da mischia}: +12 a colpire, portata 3 metri, con un singolo attacco può colpire due creature in mischia vicine tra loro.

\emph{\textbf{Sasso.} Attacco con arma a Distanza}: +11 a colpire, gittata 18m, un bersaglio.

\emph{Colpisce:} 35 (4d12 + 9) danni contundenti.

\textbf{Reazione: \emph{Attacco d'opportunità}}: il gigante delle tempeste effettua un attacco ad una creatura che attraversi o esca dalla sua portata di 3 metri.

\emph{\textbf{Colpo Fulminante (Ricarica 5-6).}} Il gigante scaglia una folgore magica ad un punto visibile entro 150 metri da sé. Ogni creatura entro 3 metri da quel punto deve effettuare un Tiro Salvezza su Riflessi DC 25, subendo 54 (12d8) danni da elettricità se lo fallisce, o la metà se lo supera.

\emph{\textbf{Arrabbiato:}} il gigante delle tempeste carica di elettricità tutta l'area intorno a se fino alla fine del combattimento. Una creatura che termini il round entro 6 metri da gigante subisce 13 (3d8) danni da elettricità. Costa 1 Azione.

\textbf{Ecologia}\\
Ambiente: Qualsiasi caldo\\
Organizzazione: Solitario o famiglia (2-5 più 1 mago o Devoto di livello 7°-10°, 1-2 Roc, 2-6 Grifoni e 2-8 Squali)\\
\textbf{Categoria Tesoro}: Corazza di Piastre Perfetta, Arco Lungo Composito Perfetto [Forza +9] con 20 Frecce, Spadone Perfetto, H\\
\textbf{Descrizione}\\
I giganti delle tempeste tendono ad avere carnagione abbronzata, anche se rari esemplari hanno pelle viola, capelli viola o blu scuri e occhi grigio argento o porpora. Il colore viola è considerato di buon auspicio tra i giganti delle tempeste, e coloro che lo posseggono tendono a diventare capi tra la loro gente. Gli adulti sono normalmente alti 6,3 metri e pesano 6000 kg. I giganti delle tempeste possono vivere fino a 600 anni.

Quando sono a riposo, preferiscono indossare tuniche corte e ampie cinte ai fianchi, sandali o piedi nudi e una fascia per capelli. Indossano pochi gioielli di semplice ma ottima fattura, i più comuni sono cavigliere (preferite dai giganti a piedi scalzi), anelli o diademi. Ma quando si equipaggiano per la battaglia, indossano corazze di piastre scintillanti e impugnano enormi spadoni e archi.

Come suggerisce il loro nome, sono inclini a violenti sbalzi di umore. I giganti delle tempeste sono facili all'ira di fronte al male e possono essere nemici brutali e pericolosi quando vengono insultati. In battaglia, preferiscono scagliare una pioggia di frecce sui loro nemici, estraendo gli spadoni solo dopo che gli avversari si sono avvicinati.

I giganti delle tempeste vivono in belle torri, castelli o in insediamenti cinti da mura e amano coltivare la terra. Possiedono enormi giardini ben curati e gestiscono centinaia di acri di coltivazioni per gruppo. Spesso impiegano altri umanoidi, come Elfi o Umani, come supporto per condurre le loro immense fattorie. Una enclave di giganti delle tempeste spesso si assume la responsabilità della sicurezza di un'intera isola o linea di costa.

\mostro{Gnoll}
\noindent
\begin{description}[noitemsep, topsep=0pt, parsep=0pt, partopsep=0pt, leftmargin=0cm, labelwidth=2.2cm]
	\item[\textbf{Taglia/Tipo:}] Media umanoide (gnoll), malvagio
	\item[\textbf{Caratt.:}] \resizebox{0.5\linewidth+1.8cm}{!}{For 2 Des 1 Cos 0 Int -2 Sag 0 Car -2}
	\item[\textbf{Punti Ferita:}] 24,  \textbf{Difesa:} 13,  \textbf{Iniziativa:} +1
	\item[\textbf{Movimento:}] 9 m
	\item[\textbf{Tiri Salvez.:}] \resizebox{0.5\linewidth+1.8cm}{!}{Tempra +3, Riflessi +3, Volontà +3}
	\item[\textbf{Sensi:}] Scurovisione 18 m
	\item[\textbf{Linguaggi:}] Gnoll
	\item[\textbf{Sfida:}] 1/2 (100 PX)\smallskip
\end{description}

\emph{\textbf{Rabbia.}} Quando lo gnoll riduce una creatura a 0 Punti Ferita con un attacco da mischia durante il proprio round, può svolgere una Reazione per muoversi fino a metà del suo movimento ed effettuare un attacco di morso.

\textbf{Azioni}

\emph{\textbf{Morso.} Attacco con arma da mischia}: +4 a colpire, portata 1 m, una creatura.

\emph{Colpisce:} 4 (1d4 + 2) danni perforanti, Malattia Rabbia Gnoll

\emph{Rabbia Gnoll:} 1 giorno, TS Tempra DC 13, 12 ore, 1 successo, -2 Saggezza

\emph{\textbf{Lancia.} Attacco con arma da mischia o a Distanza}: +4 a colpire, portata 1 m o gittata 6 m, un bersaglio.

\emph{Colpisce:} 5 (1d6 + 2) danni perforanti o 6 (1d8 + 2) danni perforanti se usata con due mani per effettuare un attacco da mischia.

\emph{\textbf{Arco Lungo.} Attacco con arma a Distanza}: +4 a colpire, gittata 45m, un bersaglio.

\emph{Colpisce:} 5 (1d8 + 1) danni perforanti.

\emph{\textbf{Risata beffarda.}} lo gnoll ride sguaiatamente ad un avversario. La creatura bersaglio deve effettuare un Tiro Salvezza su Volontà DC 13 o essere intimorito ed avere -1 al Tiro per Colpire fino alla fine del round successivo dello gnoll

\textbf{Ecologia}\\
Ambiente: Pianure calde, deserti\\
Organizzazione: Solitario, coppia, gruppo di caccia (2-5 e 1-2 Iene), banda (10-100 adulti più 50\% piccoli non combattenti, 1 sergente di 3° livello ogni 20 adulti, 1 capo di 4°-6° livello e 5-8 Iene) o tribù (20-200 più 1 sergente di 3° livello ogni 20 adulti, 1 o 2 luogotenenti di 4° o 5° livello, 1 capo di 6°-8° livello, 7-12 Iene e 4-7 ienodonti)\\
\textbf{Categoria Tesoro}: equipaggiamento da PNG (Armatura di Cuoio, Scudo Pesante di Legno, Lancia, K)\\
\textbf{Descrizione}\\
Gli gnoll sono umanoidi grandi e massicci, simili alle iene non solo nell'aspetto, ma anche nei comportamenti. Spesso tengono le iene come animali da compagnia e riflettono molti dei loro comportamenti. Pur essendo abili cacciatori, preferiscono trafugare o ripulire carcasse piuttosto che cacciare prede.

Questa pigrizia li porta a procurarsi schiavi di ogni specie per scavare tane, raccogliere provviste e acqua e persino cacciare per loro conto. Le creature non gnoll o iene diventano pasti o schiavi, a seconda del temperamento della tribù. Anche i compagni caduti possono diventare cibo, a meno che non siano onorati con una breve preghiera o cucinati interamente se morti per malattia.

Gli gnoll più civilizzati non mangiano i prigionieri, ma li tengono come schiavi per difendere o migliorare la tana o scambiarli con altre tribù. Gli gnoll apprezzano il combattimento solo quando sono in superiorità numerica. Evitano il combattimento a meno che non sia per ottenere una carcassa o imboscarsi per un lauto pasto, preferendo fuggire quando la vittoria sembra irraggiungibile.

Durante il combattimento, gli gnoll usano tattiche di branco e strategie individuali. Se sicuri di vincere, attaccano l'avversario più debole piuttosto che aiutare i compagni. Se in difficoltà, si coalizzano contro un avversario potente per costringerne alla fuga gli alleati.

I capi gnoll hanno competenze da guardiaboschi, e alcuni sono devoti a famelici Patroni. Difficilmente padroneggiano la magia in modo efficace.

\mostro{Gnomo delle Profondità}
\noindent
\begin{description}[noitemsep, topsep=0pt, parsep=0pt, partopsep=0pt, leftmargin=0cm, labelwidth=2.2cm]
	\item[\textbf{Taglia/Tipo:}] Piccola umanoide (gnomo), buono
	\item[\textbf{Caratt.:}] \resizebox{0.5\linewidth+1.8cm}{!}{For 2 Des 2 Cos 2 Int 1 Sag 0 Car -1}
	\item[\textbf{Punti Ferita:}] 24,  \textbf{Difesa:} 14,  \textbf{Iniziativa:} +2
	\item[\textbf{Movimento:}] 6 m
	\item[\textbf{Tiri Salvez.:}] \resizebox{0.5\linewidth+1.8cm}{!}{Tempra +3, Riflessi +3, Volontà +3}
	\item[\textbf{Comp.:}] Furtività +4, Consapevolezza +2
	\item[\textbf{Sensi:}] Scurovisione 36 m
	\item[\textbf{Linguaggi:}] Gnomica, Linguaggio delle Profondità, Tremun
	\item[\textbf{Sfida:}] 1/2 (100 PX)\smallskip
\end{description}

\emph{\textbf{Astuzia Gnomesca.}} Lo gnomo ha +1d6 ai Tiri Salvezza contro la magia.

\emph{\textbf{Camuffamento di Pietra.}} Lo gnomo ha +1d6 alle prove di Furtività (Nascondersi) effettuate per nascondersi su terreni rocciosi.

\emph{\textbf{Incantesimi Innati.}} La caratteristica da incantatore innato dello gnomo è l'Intelligenza. Lo gnomo può lanciare questi incantesimi in maniera innata, senza bisogno di componenti:

A volontà: \emph{\hyperlink{Anti-Individuazione}{Anti-Individuazione}} (personale)

1/giorno ciascuno: \emph{\hyperlink{Camuffare Sé Stesso}{Camuffare Sé Stesso}, cecità/sordità, \hyperlink{Sfocatura}{Sfocatura}}

\textbf{Azioni}

\emph{\textbf{Piccone da Guerra.} Attacco con arma da mischia}: +4 a colpire, portata 1 m, un bersaglio.

\emph{Colpisce:} 6 (1d8 + 2) danni perforanti.

\emph{\textbf{Dardo Avvelenato.} Attacco con arma a Distanza}: +4 a colpire, gittata 9m, un bersaglio.

\emph{Colpisce:} 4 (1d4 + 2) danni perforanti, e il bersaglio deve riuscire un Tiro Salvezza di Tempra DC 12 o restare avvelenato, -1 Forza e Destrezza, per 1 minuto. Il bersaglio può ripetere il Tiro Salvezza al termine di ciascun suo round, terminando l'effetto su di sé in caso di successo.

\textbf{Ecologia}
Ambiente: Qualsiasi sotterraneo\\
Organizzazione: Solitario, compagnia (2-4), squadra (5-20 più 1 capo 3°-6° e due sergenti di 3° livello), o banda (30-50 più 1 sergente di 3° livello ogni 20 adulti, 5 tenenti di 5° livello, 3 capitani di 7° livello, e 2-5 Elementali della Terra Medi)\\
\textbf{Categoria Tesoro}: Equipaggiamento da PNG (Piccone Pesante, Balestra Leggera con 10 Quadrelli, M)\\
\textbf{Descrizione}\\
I gnomi delle profondità, sono una branca della razza gnomesca. Dimorano nel sottosuolo, in città nascoste, al sicuro dagli elfi scuri e da altre razze sotterranee. La loro pelle è del colore della roccia, di solito grigia o marrone. I maschi sono calvi e le femmine hanno radi capelli grigi.

\mostro{Globulo}
\noindent
\begin{description}[noitemsep, topsep=0pt, parsep=0pt, partopsep=0pt, leftmargin=0cm, labelwidth=2.2cm]
	\item[\textbf{Taglia/Tipo:}] Piccola aberrazione, malvagio
	\item[\textbf{Caratt.:}] \resizebox{0.5\linewidth+1.8cm}{!}{For -2 Des 2 Cos 0 Int 3 Sag 1 Car 3}
	\item[\textbf{Punti Ferita:}] 33,  \textbf{Difesa:} 15,  \textbf{Iniziativa:} +3
	\item[\textbf{Movimento:}] volare 18 m
	\item[\textbf{Tiri Salvez.:}] \resizebox{0.5\linewidth+1.8cm}{!}{Tempra +3, Riflessi +3, Volontà +3}
	\item[\textbf{Imm. Danni:}] da Vuoto, Freddo, Veleno\\
	\item[\textbf{Immunità:}] prono
	\item[\textbf{Sensi:}] Scurovisione 36 m
	\item[\textbf{Linguaggi:}] comprende il Comune ma non lo parla
	\item[\textbf{Sfida:}] 1 (200 PX)\smallskip
\end{description}

\textbf{Odio i volatili} il Globulo ha +1d6 al Tiro per Colpire contro gli uccelli. Attacca prima gli uccelli e creature volanti

\textbf{Natura inusuale} il Globulo non respira

\textbf{Odio l'acqua} il Globulo detesta bagnarsi e ogni 5 litri di acqua spruzzata su lui subisce 1d4 di danno

\textbf{Azioni}

\emph{\textbf{Tentacolo}}. Attacco in mischia, +5 al colpire, portata 3 metri, un obiettivo

\emph{\textbf{Colpisce}} 5 (1d6+2) di danno da Vuoto. Il bersaglio deve fare un Tiro Salvezza su Tempra a DC 11 o aumentare il grado di Affaticamento di 1.

\textbf{\emph{Brillio}} una volta al giorno il Globulo diventa estremamente luminoso, le creature nel raggio di 6 metri attorno a lui devono fare un Tiro Salvezza su Tempra a DC 13 o diventare accecati per 3 round.

\textbf{Ecologia}
Ambiente: Qualsiasi, desertico, notturno\\
Organizzazione: Solitario, gruppi 2d4\\
\textbf{Categoria Tesoro}: Nessuno\\
\textbf{Descrizione}\\
I Globuli sono aberrazioni magiche provenienti da qualche portale aperto verso l'Oltre. Creature di freddo e vuoto sembrano delle piccole stelle che anelano solo di risucchiare la vita della creature incontrate.
Intelligenti e furbe preferiscono attaccare rimanendo in volo e fiaccando l'avversario finché questo è mortalmente affaticato. Una volta ucciso di un Globulo non rimane che una piccola creatura a forma di stella con un grosso occhio centrale, completamente bianco.

\mostro{Goblin}
\noindent
\begin{description}[noitemsep, topsep=0pt, parsep=0pt, partopsep=0pt, leftmargin=0cm, labelwidth=2.2cm]
	\item[\textbf{Taglia/Tipo:}] Piccola umanoide (goblinoide), malvagio
	\item[\textbf{Caratt.:}] \resizebox{0.5\linewidth+1.8cm}{!}{For 0 Des 0 Cos 1 Int -1 Sag -2 Car -1}
	\item[\textbf{Punti Ferita:}] 19,  \textbf{Difesa:} 12,  \textbf{Iniziativa:} +0
	\item[\textbf{Movimento:}] 9 m
	\item[\textbf{Tiri Salvez.:}] \resizebox{0.5\linewidth+1.8cm}{!}{Tempra +3, Riflessi +3, Volontà +3}
	\item[\textbf{Sensi:}] Scurovisione 18 m
	\item[\textbf{Linguaggi:}] Comune, Goblin
	\item[\textbf{Sfida:}] 1/4 (50 PX)\smallskip
\end{description}

\textbf{Azioni}

\emph{\textbf{Spada Corta.} Attacco con arma da mischia}: +4 a colpire, portata 1 m, un bersaglio.

\emph{Colpisce:} 4 (1d6 + 1) danni taglienti

\emph{\textbf{Arco Corto.} Attacco con arma a Distanza}: +3 a colpire, gittata 15m, un bersaglio.

\emph{Colpisce:} 3 (1d6) danni perforanti.

\textbf{Ecologia}\\
Ambiente: Qualsiasi Temperate\\
Organizzazione: Gruppo (4-9), banda da guerra (10-24) o tribù (50+ più 50\% non combattenti\\
\textbf{Categoria Tesoro}: K\\
\textbf{Descrizione}\\
I goblin sono selvaggi, imprevedibili, rumorosi.
I goblin preferiscono vivere nelle caverne, nel fitto delle foreste e quando ne hanno a disposizione nelle strutture antiche abbandonate. I goblin non amano costruire quanto piuttosto distruggere per poi lamentarsi che non c'è nulla di utile.

I goblin sono molto superstiziosi, e vedono la magia con un misto di timore reverenziale e paura. Ogni cosa che non comprendono è per loro magia e questo li porta a essere estremamente sospettosi di tutto e a distruggere tutto, visto che ciò che non capiscono va distrutto.

I goblin sono famelici e possono mangiare enormi quantità di cibo. un goblin non rinuncia a mangiare nulla tranne forse l'insalata..

\mostro{Golem di Argilla}
\noindent
\begin{description}[noitemsep, topsep=0pt, parsep=0pt, partopsep=0pt, leftmargin=0cm, labelwidth=2.2cm]
	\item[\textbf{Taglia/Tipo:}] Grande costrutto, disallineato
	\item[\textbf{Caratt.:}] \resizebox{0.5\linewidth+1.8cm}{!}{For 5 Des -1 Cos 4 Int -4 Sag -1 Car -5}
	\item[\textbf{Punti Ferita:}] 184,  \textbf{Difesa:} 23,  \textbf{Iniziativa:} -1
	\item[\textbf{Movimento:}] 6 m
	\item[\textbf{Tiri Salvez.:}] \resizebox{0.5\linewidth+1.8cm}{!}{\resizebox{0.5\linewidth+1.8cm}{!}{Tempra +13, Riflessi +8, Volontà +8}}
	\item[\textbf{Imm. Danni:}] Acido, Veleno
	\item[\textbf{Immunità:}] affascinato, paralizzato, pietrificato, affaticato, spaventato
	\item[\textbf{Sensi:}] Scurovisione 18 m
	\item[\textbf{Linguaggi:}] comprende le lingue del suo creatore ma non può parlare
	\item[\textbf{Sfida:}] 9 (5000 PX)\smallskip
\end{description}

\emph{\textbf{Riduzione del Danno.}} Il golem d'argilla ha durezza 8/- contro armi non magiche.

\emph{\textbf{Berserk.}} Ogni volta che il golem inizia il suo round con 60 Punti Ferita o meno, tira un d6. Se ottieni 6, il golem va in berserk. Durante ogni suo round mentre è in berserk, guadagna una Azione per quel round. Il golem attacca la creatura più vicina che può vedere. Se non c'è nessuna creatura abbastanza vicina da muoversi e attaccarla, il golem attacca un oggetto, con preferenza per gli oggetti più piccoli di lui. Una volta che il golem è andato in berserk, continuerà ad esserlo finché non viene distrutto o recupera tutti i suoi Punti Ferita.

\emph{\textbf{Armi Magiche.}} Gli attacchi con armi del golem sono magici.

\emph{\textbf{Assorbimento dell'Acido.}} Ogni volta che il golem è vittima di danni da acido, non subisce danni ma invece recupera un pari numero di Punti Ferita.

\emph{\textbf{Forma Immutabile.}} Il golem è immune a qualsiasi incantesimo o effetto che altererebbe la sua forma.

\emph{\textbf{Natura di Costrutto.}} Un golem non ha bisogno di aria, cibo, bevande o sonno.

\emph{\textbf{Resistenza alla Magia.}} Il golem ha +1d6 ai Tiri Salvezza contro incantesimi e altri effetti magici.

\textbf{Azioni}

\emph{\textbf{Multiattacco.}} Il golem effettua due attacchi di schianto oppure un solo attacco di pugno maledetto

\emph{\textbf{Schianto.} Attacco con arma da mischia}: +10 a colpire, portata 1 m, un bersaglio.

\emph{Colpisce:} 16 (2d10 + 5) danni contundenti.

\emph{\textbf{Pugno Maledetto.}: Attacco con arma naturale}: + 11 a colpire, portata 1 m, un bersaglio

\emph{Colpisce:} 16 (2d6 + 5) danni contundenti. Le ferite da pugno maledetto guariscono al ritmo di 1 Punto ferita a giorno. Le cure magiche, incantesimi o pozioni, curano 1 Punto Ferita per dado di cura + tutto l'eventuale fisso (es. una cura di 3d6+4 cura 7 PF)

\emph{\textbf{Velocità (Ricarica 5-6).}} Fino al termine del suo prossimo round, il golem ottiene un bonus magico di +2 alla Difesa, ha +1d6 ai Tiri Salvezza su Riflessi, e può usare gli attacchi di schianto come Azione Immediata.

\textbf{Ecologia}\\
Ambiente: Qualsiasi\\
Organizzazione: Solitario o gruppo (2-4)\\
\textbf{Categoria Tesoro}: Nessuno\\
\textbf{Descrizione}\\
I golem di argilla non indossano abiti, eccezion fatta per un indumento di cuoio trattato o metallo attorno ai fianchi. Mediamente sono alti più di 2,3 metri e pesano 300 chili.

\textbf{Costruzione}
Un golem d'argilla può essere scolpito a partire da un unico blocco d'argilla del peso minimo di 500 chili, trattato con polveri e oli rari per il valore di 1,500 mo.

\mostro{Golem di Carne}
\noindent
\begin{description}[noitemsep, topsep=0pt, parsep=0pt, partopsep=0pt, leftmargin=0cm, labelwidth=2.2cm]
	\item[\textbf{Taglia/Tipo:}] Media costrutto, neutrale
	\item[\textbf{Caratt.:}] \resizebox{0.5\linewidth+1.8cm}{!}{For 4 Des -1 Cos 4 Int -2 Sag 0 Car -3}
	\item[\textbf{Punti Ferita:}] 109,  \textbf{Difesa:} 17,  \textbf{Iniziativa:} -1
	\item[\textbf{Movimento:}] 9 m
	\item[\textbf{Tiri Salvez.:}] \resizebox{0.5\linewidth+1.8cm}{!}{Tempra +9, Riflessi +4, Volontà +5}
	\item[\textbf{Imm. Danni:}] Elettricità, Veleno
	\item[\textbf{Immunità:}] affascinato, paralizzato, pietrificato, affaticato, spaventato
	\item[\textbf{Sensi:}] Scurovisione 18 m
	\item[\textbf{Linguaggi:}] comprende le lingue del suo creatore ma non può
	\item[\textbf{Sfida:}] 5 (1800 PX)\smallskip
\end{description}

\emph{\textbf{Riduzione del Danno.}} Il golem d'argilla ha durezza 6/- contro armi non magiche.

\emph{\textbf{Berserk.}} Ogni volta che il golem inizia il suo round con 40 Punti Ferita o meno, tira un d6. Se ottieni 6, il golem va in berserk. Durante ogni suo round mentre è in berserk guadagna una Azione, il golem attacca la creatura più vicina che possa vedere. Se non c'è nessuna creatura abbastanza vicina da muoversi e attaccarla, il golem attacca un oggetto, con preferenza per gli oggetti più piccoli di lui. Una volta che il golem è andato in berserk, continuerà ad esserlo finché non viene distrutto o recupera tutti i suoi Punti Ferita.

\emph{\textbf{Armi Magiche.}} Gli attacchi con armi del golem sono magici.

\emph{\textbf{Assorbimento dei Fulmini.}} Ogni volta che il golem sia vittima di un danno da elettricità, non subisce danni ma invece recupera un pari numero di Punti Ferita.

\emph{\textbf{Avversione al Fuoco.}} Se il golem subisce danni da fuoco, ha -1d6 ai tiri di attacco e le prove di competenza di Base fino alla fine del suo prossimo round.

\emph{\textbf{Forma Immutabile.}} Il golem è immune a qualsiasi incantesimo o effetto che altererebbe la sua forma.

\emph{\textbf{Natura di Costrutto.}} Un golem non ha bisogno di aria, cibo, bevande o sonno.

\emph{\textbf{Resistenza alla Magia.}} Il golem ha +1d6 ai Tiri Salvezza contro incantesimi e altri effetti magici.

\textbf{Azioni}

\emph{\textbf{Multiattacco.}} Il golem effettua due attacchi di schianto.

\emph{\textbf{Schianto.} Attacco con arma da mischia}: +6 a colpire, portata 1 m, un bersaglio.

\emph{Colpisce:} 13 (2d8 + 4) danni contundenti. La creatura colpita deve effettuare un Tiro Salvezza su Tempra a DC 17 o ammalarsi. Ogni volta che fallisce il Tiro Salvezza esegue una Azione in meno il round successivo. Se arriva a perdere 3 Azioni, ovvero fallisce per 3 volte il Tiro Salvezza di fila, la creatura muore. Appena il Tiro Salvezza riesce si debella la malattia.

\emph{\textbf{Arrabbiato:}} il golem di carne si sovraccarica. Per 2d4 round può eseguire una Azione di in più di Movimento o di Attacco. Costa 1 Azione.

\textbf{Ecologia}\\
Ambiente: Qualsiasi\\
Organizzazione: Solitario o gruppo (2-4)\\
\textbf{Categoria Tesoro}: Nessuno\\
\textbf{Descrizione}\\
Un golem di carne è una mostruosa collezione di parti anatomiche umanoidi trafugate e cucite insieme. La sua carne cadaverica ha tonalità verde pallido o giallognola. Un golem di carne indossa qualsiasi tipo di vestito che il suo creatore desideri, normalmente solo un logoro paio di pantaloni. Non ha Equipaggiamento né armi. Un golem di carne è alto più di 2,3 metri e pesa 250 kg.

Un golem di carne non parla, anche se può emettere una specie di ringhio rauco. Cammina e si muove con un'andatura a scatti, come se non avesse il pieno controllo del proprio corpo.

Anche se molti golem di carne sono privi di ragione, si narra di golem eccezionali che in qualche modo hanno mantenuto i ricordi della vita precedente. La testa (e quindi il cervello) di questi golem di carne deve essere la giusta combinazione di freschezza e (nella vita precedente) decisione, ma di assoluta importanza sembrano essere anche la fortuna e il caso affinché durante la loro creazione si conservi l'intelletto. Certamente quelli che costruiscono golem di carne preferiscono avere schiavi privi di intelletto piuttosto che dotati di una propria volontà, di conseguenza i golem di carne intelligenti sono rari.

\mostro{Golem di Ferro}
\noindent
\begin{description}[noitemsep, topsep=0pt, parsep=0pt, partopsep=0pt, leftmargin=0cm, labelwidth=2.2cm]
	\item[\textbf{Taglia/Tipo:}] Grande costrutto, disallineato
	\item[\textbf{Caratt.:}] \resizebox{0.5\linewidth+1.8cm}{!}{For 7 Des -1 Cos 5 Int -4 Sag 0 Car -5}
	\item[\textbf{Punti Ferita:}] 319,  \textbf{Difesa:} 32,  \textbf{Iniziativa:} -1
	\item[\textbf{Movimento:}] 9 m
	\item[\textbf{Tiri Salvez.:}] \resizebox{0.5\linewidth+1.8cm}{!}{\resizebox{0.5\linewidth+1.8cm}{!}{Tempra +21, Riflessi +15, Volontà +16}}
	\item[\textbf{Imm. Danni:}] Fuoco, Veleno
	\item[\textbf{Immunità:}] affascinato, paralizzato, pietrificato, affaticato, spaventato
	\item[\textbf{Sensi:}] Scurovisione 36 m
	\item[\textbf{Linguaggi:}] comprende le lingue del suo creatore ma non può parlare
	\item[\textbf{Sfida:}] 16 (15000 PX)\smallskip
\end{description}

\emph{\textbf{Riduzione del Danno.}} Il golem d'argilla ha durezza 12/- contro armi non magiche.

\emph{\textbf{Armi Magiche.}} Gli attacchi con armi del golem sono magici.

\emph{\textbf{Assorbimento del Fuoco.}} Ogni volta che il golem sia vittima di un danno da fuoco, non subisce danni ma invece recupera un pari numero di Punti Ferita.

\emph{\textbf{Forma Immutabile.}} Il golem è immune a qualsiasi incantesimo o effetto che altererebbe la sua forma.

\emph{\textbf{Natura di Costrutto.}} Un golem non ha bisogno di aria, cibo, bevande o sonno.

\emph{\textbf{Resistenza alla Magia.}} Il golem ha +1d6 ai Tiri Salvezza contro incantesimi e altri effetti magici.

\textbf{Azioni}

\emph{\textbf{Multiattacco.}} Il golem effettua due attacchi da mischia.

\emph{\textbf{Schianto.} Attacco con arma da mischia}: +14 a colpire, portata 1 m, un bersaglio.

\emph{Colpisce:} 20 (3d8 + 7) danni contundenti.

\emph{\textbf{Spada.} Attacco con arma da mischia}: +14 a colpire, portata 3 m, un bersaglio.

\emph{Colpisce:} 23 (3d10 + 7) danni taglienti.

\textbf{Reazione: \emph{Attacco d'opportunità}}: il golem effettua un attacco ad una creatura che attraversi o esca dalla sua portata di 1 metro.

\emph{\textbf{Soffio Velenoso (Ricarica 6).}} Il golem esala un gas velenoso in un cono di 5 metri. Ogni creatura in quell'area deve effettuare un Tiro Salvezza di Tempra DC 29, subendo 45 (10d8) danni da veleno se fallisce il Tiro Salvezza, o la metà di questi danni se lo riesce.

\emph{\textbf{Arrabbiato:}} il golem di ferro esala un soffio rovente in un cono di 3 metri. Il soffio causa 3d10 danni da fuoco o la metà se il Tiro Salvezza si Riflessi DC 26 riesce. Il golem recupera l'intero ammontare in Punti Ferita ed è Accelerato 1 per 2d4 round. Costa 2 Azioni.

\textbf{Ecologia}\\
Ambiente: Qualsiasi\\
Organizzazione: Solitario o gruppo (2-4)\\
\textbf{Categoria Tesoro}: Nessuno\\
\textbf{Descrizione}\\
Un golem di ferro ha un corpo di forma umanoide in ferro. Il creatore può dargli qualsiasi forma desideri, ma presenta quasi sempre un'armatura di qualche tipo, sia essa cerimoniale e preziosa o semplice e d'uso. Rispetto ad un golem di pietra ha sembianze molto più definite. I golem di ferro, talvolta, portano con sé un'arma, anche se il più delle volte tendono a preferirle i loro attacchi schianto.

Un golem di ferro è alto 3,6m e pesa circa 2.500 chili. Un golem di ferro non può parlare né emettere voce. Inoltre, non emette nessun odore riconoscibile.

Anche se la pratica della costruzione di golem di ferro è gradualmente caduta in disuso, i membri venerabili di alcune grandi civiltà del passato consideravano la capacità di forgiare golem di ferro dalla forza e dalle dimensioni sconcertanti un motivo di vanto. Questi golem (di taglia maggiore o uguale a Enorme), in alcuni angoli remoti del mondo, esistono ancora, e ancora eseguono meccanicamente ordini impartiti loro da imperi ormai scomparsi.

\textbf{Costruzione}
Per costruire un golem di ferro occorrono 2.500 kg di ferro, fuso con tinture rare del valore minimo di 10000 mo.

\mostro{Golem di Pietra}
\noindent
\begin{description}[noitemsep, topsep=0pt, parsep=0pt, partopsep=0pt, leftmargin=0cm, labelwidth=2.2cm]
	\item[\textbf{Taglia/Tipo:}] Grande costrutto, disallineato
	\item[\textbf{Caratt.:}] \resizebox{0.5\linewidth+1.8cm}{!}{For 6 Des -1 Cos 5 Int -4 Sag 0 Car -5}
	\item[\textbf{Punti Ferita:}] 205,  \textbf{Difesa:} 24,  \textbf{Iniziativa:} -1
	\item[\textbf{Movimento:}] 9 m
	\item[\textbf{Tiri Salvez.:}] \resizebox{0.5\linewidth+1.8cm}{!}{\resizebox{0.5\linewidth+1.8cm}{!}{Tempra +15, Riflessi +9, Volontà +10}}
	\item[\textbf{Imm. Danni:}] Veleno
	\item[\textbf{Immunità:}] affascinato, paralizzato, pietrificato, affaticato, spaventato
	\item[\textbf{Sensi:}] Scurovisione 36 m
	\item[\textbf{Linguaggi:}] comprende le lingue del suo creatore ma non può parlare
	\item[\textbf{Sfida:}] 10 (5900 PX)\smallskip
\end{description}

\emph{\textbf{Riduzione del Danno.}} Il golem d'argilla ha durezza 10/- contro armi non magiche.

\emph{\textbf{Armi Magiche.}} Gli attacchi con armi del golem sono magici.

\emph{\textbf{Forma Immutabile.}} Il golem è immune a qualsiasi incantesimo o effetto che altererebbe la sua forma.

\emph{\textbf{Natura di Costrutto.}} Un golem non ha bisogno di aria, cibo, bevande o sonno.

\emph{\textbf{Resistenza alla Magia.}} Il golem ha +1d6 ai Tiri Salvezza contro incantesimi e altri effetti magici.

\textbf{Azioni}

\emph{\textbf{Multiattacco.}} Il golem effettua due attacchi di schianto.

\emph{\textbf{Schianto.} Attacco con arma da mischia}: +11 a colpire, portata 1 m, un bersaglio.

\emph{Colpisce:} 19 (3d8 + 6) danni contundenti.

\textbf{Reazione: \emph{Sasso affilato}}: il golem reagisce ad un attacco subito guadagnando 1 danno bonus al suo attacco di schianto.

\emph{\textbf{Lentezza (Ricarica 5-6).}} Il golem prende a bersaglio una o più creature entro 3 metri da lui e che possa vedere. Ciascun bersaglio deve effettuare un Tiro Salvezza di Volontà DC 24 contro questa magia. Se fallisce il Tiro Salvezza il bersaglio è Rallentato 2/1 minuto. Il bersaglio può ripetere il Tiro Salvezza al termine di ciascun suo round, terminando l'effetto per sé in caso di successo.

\textbf{Ecologia}\\
Ambiente: Qualsiasi\\
Organizzazione: Solitario o gruppo (2-4)\\
\textbf{Categoria Tesoro}: Nessuno\\
\textbf{Descrizione}\\
Un golem di pietra ha un corpo umanoide fatto di pietra, spesso stilizzato per soddisfare il suo creatore. Ad esempio, può essere scolpito in modo da indossare un'armatura, con particolari simboli scolpiti sulla corazza, o avere dei disegni intarsiati nella pietra dei suoi arti. La testa è spesso scolpita per sembrare un elmo o la testa di qualche bestia. Sebbene possa essere scolpito con uno scudo o un'arma di pietra come una spada, queste scelte estetiche non influenzano le sue capacità in combattimento.

Come per la maggior parte dei golem, un golem di pietra non può parlare e non emette altro suono se non quello della pietra che sfrega sulla pietra quando si muove. Un golem di pietra è alto 2,7 metri e pesa circa 1000 kg.

Esistono numerose varianti dei Golem di Pietra, a seconda del materiali di cui sono fatti ma anche come espressioni di spiriti elementali, ovvero é possibile che uno spirito elementale abiti una roccia (o gemma) e ne definisca l'aspetto e lo animi come proprio corpo.

\textbf{Costruzione}
Il corpo di un golem di pietra viene scolpito da un unico blocco di pietra dura, come il granito, del peso di almeno 1.500 kg. La pietra deve essere di qualità eccezionale, e costare 5000 mo.

\mostro{Gorgone}
\noindent
\begin{description}[noitemsep, topsep=0pt, parsep=0pt, partopsep=0pt, leftmargin=0cm, labelwidth=2.2cm]
	\item[\textbf{Taglia/Tipo:}] Grande mostruosità, disallineato
	\item[\textbf{Caratt.:}] \resizebox{0.5\linewidth+1.8cm}{!}{For 5 Des 0 Cos 4 Int -4 Sag 1 Car -2}
	\item[\textbf{Punti Ferita:}] 109,  \textbf{Difesa:} 18,  \textbf{Iniziativa:} +0
	\item[\textbf{Movimento:}] 12 m
	\item[\textbf{Tiri Salvez.:}] \resizebox{0.5\linewidth+1.8cm}{!}{Tempra +9, Riflessi +5, Volontà +6}
	\item[\textbf{Comp.:}] Consapevolezza +4
	\item[\textbf{Immunità:}] Pietrificato
	\item[\textbf{Sensi:}] Scurovisione 18 m
	\item[\textbf{Sfida:}] 5 (1800 PX)\smallskip
\end{description}

\emph{\textbf{Carica Travolgente.}} La gorgone carica un bersaglio. 2 Azioni. Se il bersaglio, entro 18 metri, viene colpito da Incornata, deve anche riuscire un Tiro Salvezza su Tempra DC 18 o cadere prono. Se il bersaglio cade prono la gorgone può effettuare un attacco di zoccoli contro di lui come Azione Immediata.

\textbf{Azioni}

\emph{\textbf{Incornata.} Attacco con arma da mischia}: +6 a colpire, portata 1 m, un bersaglio.

\emph{Colpisce:} 18 (2d12 + 5) danni perforanti.

\emph{\textbf{Zoccoli.} Attacco con arma da mischia}: +6 a colpire, portata 1 m, un bersaglio.

\emph{Colpisce:} 16 (2d10 + 5) danni contundenti.

\emph{\textbf{Soffio Pietrificante (Ricarica 4-6).}} La gorgone esala un gas pietrificante in un cono di 9 metri. Ogni creatura in quell'area deve riuscire un Tiro Salvezza di Tempra DC 16. Se il Tiro Salvezza fallisce la creatura è Rallentata 1/1 minuto. Se successivi soffi portano il bersaglio a non avere più Azioni allora diviene pietrificato finché non viene liberato dall'incantesimo \hyperlink{Pietra in Carne}{Pietra in Carne}.

\emph{\textbf{Arrabbiato:}} la Gorgone concentra un potente soffio pietrificante. Costa 2 azioni. Una creatura a distanza di mischia deve effettuare un Tiro Salvezza su Tempra a DC 16 o diventare di pietra per 24 ore.

\textbf{Ecologia}\\
Ambiente: Pianure Temperate, Colline Rocciose e Sotterranei\\
Organizzazione: Solitario, coppia, branco (3-4) o mandria (5-12)\\
\textbf{Categoria Tesoro}: Nessuno\\
\textbf{Descrizione}\\
Le gorgoni sono creature magiche e irascibili: sebbene a prima vista possano sembrare dei costrutti, sotto le piastre metalliche dall'aspetto artificiale sono fatte di carne e ossa. Come tori aggressivi, sfidano qualsiasi creatura sconosciuta che incontrano, spesso travolgendo il cadavere del loro avversario o frantumando i suoi resti pietrificati finché la creatura non è più riconoscibile. Le femmine sono pericolose quanto i maschi, e i due sessi hanno l'identico aspetto. Una tipica gorgone è alta 1,8 metri e lunga 2,3 metri. Pesa circa 2000 kg.

Le gorgoni ricavano il loro nutrimento consumando minerali, in particolare la pietra delle loro vittime pietrificate, e ogni statua da loro creata viene completamente divorata. Non possono digerire metallo o gemme, così il loro sterco (che assomiglia a polvere grigia dall'odore acre) spesso contiene piccoli cristalli grezzi e pepite d'oro. La loro aggressività verso tutte le altre creature fa sì che nei loro pascoli siano pochi, se non nessuno, i predatori e le prede. Ogni mandria è guidata da un toro dominante; le gorgoni solitarie sono generalmente tori adolescenti allontanati dalla mandria del toro dominante.

La loro carne è dura e muscolosa (una volta che viene rimossa l'armatura), e per coloro che la assaggiano è abbastanza nutriente. Molte tribù di giganti della pietra credono che mangiare la carne di gorgone aumenti la loro armatura naturale. Le corna di gorgone polverizzate valgono 250 mo come componente materiale alternativo per gli oggetti magici ed incantesimi che agiscono sulla Forza o Pietra.

\mostro{Grick}
\noindent
\begin{description}[noitemsep, topsep=0pt, parsep=0pt, partopsep=0pt, leftmargin=0cm, labelwidth=2.2cm]
	\item[\textbf{Taglia/Tipo:}] Media mostruosità, neutrale
	\item[\textbf{Caratt.:}] \resizebox{0.5\linewidth+1.8cm}{!}{For 2 Des 2 Cos 0 Int -4 Sag 2 Car -3}
	\item[\textbf{Punti Ferita:}] 51,  \textbf{Difesa:} 16,  \textbf{Iniziativa:} +2
	\item[\textbf{Movimento:}] 9 m, scalata 9 m
	\item[\textbf{Tiri Salvez.:}] \resizebox{0.5\linewidth+1.8cm}{!}{Tempra +3, Riflessi +4, Volontà +4}
	\item[\textbf{Sensi:}] Scurovisione 18 m
	\item[\textbf{Sfida:}] 2 (450 PX)\smallskip
\end{description}

\emph{\textbf{Camuffamento di Pietra.}} Il grick ha +1d6 alle prove di Furtività (Nascondersi) effettuate per nascondersi su terreno roccioso.

\textbf{Azioni}

\emph{\textbf{Multiattacco.}} Il grick effettua un attacco con i suoi tentacoli. Se l'attacco colpisce, il grick può effettuare un attacco di becco contro lo stesso bersaglio.

\emph{\textbf{Tentacoli.} Attacco con arma da mischia}: +4 a colpire, portata 1 m, un bersaglio.

\emph{Colpisce:} 9 (2d6 + 2) danni taglienti.

\emph{\textbf{Becco.} Attacco con arma da mischia}: +4 a colpire, portata 1 m, un bersaglio.

\emph{Colpisce:} 5 (1d6 + 2) danni perforanti.

\textbf{Ecologia}: \\
Ambiente: Qualsiasi Sotterraneo\\
Organizzazione: Solitario o ammasso (2-5)\\
\textbf{Categoria Tesoro}: Accidentale\\
\textbf{Descrizione}\\
Il grick, una creatura vermiforme, è il terrore delle caverne e dei cunicoli in cui risiede. In agguato nei pressi di tunnel trafficati o città sotterranee, balza fuori dal buio per catturare le sue prede. Solitamente non consuma le prede sul posto, ma le porta nella sua tana, un cunicolo stretto o una sporgenza di una caverna, dove può mangiarle tranquillamente.

Le origini del grick sono ignote. Sebbene possieda una rudimentale intelligenza, non ha una vera e propria società e si incontra solitamente da solo. Nei rari casi in cui sono presenti più grick, non sembrano collaborare tra loro: ognuno attacca un obiettivo individuale e si ritira con il bottino una volta abbattuto l'avversario.

I grick sono predatori capaci con una pelle resistente alle armi, rendendoli particolarmente pericolosi. Molti avventurieri inesperti sono periti sotto il loro attacco poiché non riuscivano a danneggiarli con armi non magiche. Chi conosce i grick (soprattutto Nani, Morlock e Trogloditi) sa che la migliore strategia è ritirarsi e attendere rinforzi più potenti o magici.

I grick si mimetizzano grazie al loro colore scuro e alla capacità di scalare i muri, rimanendo nascosti fino al momento di attaccare. Quando il cibo scarseggia, possono dirigersi verso la superficie in cerca di prede, ma preferiscono le tenebre e la sicurezza di un tetto sopra la testa, evitando il cielo aperto e cercando rifugio sotto alberi, nuvole basse o edifici.

\mostro{Grifone}
\noindent
\begin{description}[noitemsep, topsep=0pt, parsep=0pt, partopsep=0pt, leftmargin=0cm, labelwidth=2.2cm]
	\item[\textbf{Taglia/Tipo:}] Grande mostruosità, disallineato
	\item[\textbf{Caratt.:}] \resizebox{0.5\linewidth+1.8cm}{!}{For 4 Des 2 Cos 3 Int -3 Sag 1 Car 0}
	\item[\textbf{Punti Ferita:}] 52,  \textbf{Difesa:} 16,  \textbf{Iniziativa:} +2
	\item[\textbf{Movimento:}] 9 m, volo 24 m
	\item[\textbf{Tiri Salvez.:}] \resizebox{0.5\linewidth+1.8cm}{!}{Tempra +5, Riflessi +4, Volontà +3}
	\item[\textbf{Comp.:}] Consapevolezza +5
	\item[\textbf{Sensi:}] Scurovisione 18 m
	\item[\textbf{Sfida:}] 2 (450 PX)\smallskip
\end{description}

\emph{\textbf{Vista Affinata.}} Il grifone ha +1d6 nelle prove di Consapevolezza basate sulla vista.

\textbf{Azioni}

\emph{\textbf{Multiattacco.}} Il grifone effettua due attacchi: uno con il becco e uno con gli artigli.

\emph{\textbf{Artigli.} Attacco con arma da mischia}: +6 a colpire, portata 1 m, un bersaglio.

\emph{Colpisce:} 11 (2d6 + 4) danni taglienti, 1 danno da Sanguinamento.

\emph{\textbf{Becco.} Attacco con arma da mischia}: +5 a colpire, portata 1 m, un bersaglio.

\emph{Colpisce:} 8 (1d8 + 4) danni perforanti.

\textbf{Reazione: \emph{Attacco d'opportunità}}: il grifone attacca se sta volando ed una creatura esce o attraversa la sua portata di 3 m.

\textbf{Ecologia}\\
Ambiente: Colline Temperate\\
Organizzazione: Solitario, coppia o branco (6-10)\\
\textbf{Categoria Tesoro}: Accidentale\\
\textbf{Descrizione}\\
I grifoni sono potenti predatori aerei che piombano dai loro nidi per afferrare le prede con becco e artigli. Aggressivi e territoriali, sono combattenti astuti e leali verso chi guadagna il loro rispetto, proteggendoli fino alla morte. Pesano oltre 250 kg e sono lunghi 2,3 metri, con un imponente profilo spesso usato in araldica come simbolo di potenza, autorità e giustizia. Nonostante ciò, sono più interessati a cacciare e difendersi.

Sebbene possano essere addestrati come cavalcature, i grifoni non hanno un'innata affinità con gli umanoidi e spesso entrano in conflitto con razze civilizzate per procurarsi carne di saurovallo. I cittadini possono meravigliarsi alla vista di un grifone addestrato con un'apertura alare di 7 metri, ma i contadini sono ben consapevoli del pericolo che rappresentano.

I grifoni si accoppiano per la vita e cercano vendetta per anni se un compagno o un figlio viene ucciso. Questa lealtà li rende cavalcature e guardiani di tesori ideali, nonostante il pericolo insito nel commercio di grifoni catturati e uova rubate. Le uova valgono fino a 2000 mo l'una e i giovani vivi fino a 3000. Tuttavia, comprare o addomesticare con la violenza queste creature è considerato schiavitù dalle divinità buone. Guadagnarsi la loro lealtà spontanea è un compito difficile ma più sicuro.

Prima di poter cavalcare un grifone in combattimento, la creatura deve fare pratica nel portare il peso del suo cavaliere. Un grifone deve avere un atteggiamento amichevole verso l'addestratore (con una prova di Gestire Animali, Diplomazia o Intimidire), e 6 settimane di pratica con una prova riuscita di Gestire Animali con DC 20 sono necessarie per abituarlo al carico. I grifoni addestrati possono conoscere trucchi e imparare nuovi comandi.

I grifoni possono portare fino a 25 di Ingombro come carico leggero, 50 come carico medio e 70 come carico pesante. È necessaria una sella esotica per cavalcarli.

\mostro{Grimlock}
\noindent
\begin{description}[noitemsep, topsep=0pt, parsep=0pt, partopsep=0pt, leftmargin=0cm, labelwidth=2.2cm]
	\item[\textbf{Taglia/Tipo:}] Media umanoide (grimlock), malvagio
	\item[\textbf{Caratt.:}] \resizebox{0.5\linewidth+1.8cm}{!}{For 3 Des 1 Cos 1 Int -1 Sag -1 Car -2}
	\item[\textbf{Punti Ferita:}] 19,  \textbf{Difesa:} 13,  \textbf{Iniziativa:} +1
	\item[\textbf{Movimento:}] 9 m
	\item[\textbf{Tiri Salvez.:}] \resizebox{0.5\linewidth+1.8cm}{!}{Tempra +3, Riflessi +3, Volontà +3}
	\item[\textbf{Comp.:}] Atletica +5, Furtività +3, Consapevolezza +3
	\item[\textbf{Immunità:}] accecato
	\item[\textbf{Sensi:}] Vista Cieca 9 m o 3 m se assordato (cieco oltre questo raggio)
	\item[\textbf{Linguaggi:}] Linguaggio delle Profondità
	\item[\textbf{Sfida:}] 1/4 (50 PX)\smallskip
\end{description}

\emph{\textbf{Camuffamento di Pietra.}} Il grimlock ha +1d6 alle prove di Furtività (Nascondersi) effettuate per nascondere su terreni rocciosi.

\emph{\textbf{Sensi Ciechi.}} Il grimlock non può usare la vista cieca mentre è assordato e non più fiutare.

\emph{\textbf{Olfatto e Udito Affinati.}} Il grimlock ha +1d6 alle prove di Consapevolezza basate su udito o olfatto.

\textbf{Azioni}

\emph{\textbf{Randello d'Osso Appuntito.} Attacco con arma da mischia}: +5 a colpire, portata 1 m, un bersaglio.

\emph{Colpisce:} 5 (1d4 + 3) danni contundenti più 2 (1d4) danni perforanti.

\emph{\textbf{Arco Lungo.} Attacco con arma a Distanza}: +3 a colpire, gittata 45m, un bersaglio.

\emph{Colpisce:} 5 (1d8 + 1) danni perforanti.

\textbf{Ecologia}\\
I Grimlock abitano gli insediamenti abbandonati di altre Razze e sono spesso trovati come schiavi di altre creature più organizzate, come i nani ed Elfi. Si ritiene che si trattino di una propaggine ancora più degenerata dei Morlock, che viaggiano da Sekamina per cacciare i Grimlock per il cibo e considerano la loro carne una delicatezza.\\
\textbf{Descrizione}\\
I Grimlock sono creature umane cieche e selvagge che abitano nel regno delle terre oscure di profondità, dove si organizzano in piccoli gruppi tribali.

\mostro{Guardiano Protettore}
\noindent
\begin{description}[noitemsep, topsep=0pt, parsep=0pt, partopsep=0pt, leftmargin=0cm, labelwidth=2.2cm]
	\item[\textbf{Taglia/Tipo:}] Grande costrutto, disallineato
	\item[\textbf{Caratt.:}] \resizebox{0.5\linewidth+1.8cm}{!}{For 4 Des -1 Cos 4 Int -2 Sag 0 Car -4}
	\item[\textbf{Punti Ferita:}] 146,  \textbf{Difesa:} 20,  \textbf{Iniziativa:} -1
	\item[\textbf{Movimento:}] 9 m
	\item[\textbf{Tiri Salvez.:}] \resizebox{0.5\linewidth+1.8cm}{!}{Tempra +11, Riflessi +6, Volontà +7}
	\item[\textbf{Imm. Danni:}] Veleno
	\item[\textbf{Immunità:}] affascinato, paralizzato, affaticato, spaventato
	\item[\textbf{Sensi:}] Scurovisione 18 m, Vista Cieca 3 m
	\item[\textbf{Linguaggi:}] comprende i comandi forniti in qualsiasi lingua ma non può parlare
	\item[\textbf{Sfida:}] 7 (2900 PX)\smallskip
\end{description}

\emph{\textbf{Accumulare Incantesimi.}} Un incantatore che indossi l'amuleto del guardiano protettore può far sì che il guardiano accumuli un incantesimo di livello 4 o più basso. Per farlo, l'incantatore deve lanciare l'incantesimo sul guardiano. L'incantesimo non ha effetto ma viene accumulato all'interno del guardiano. Quando gli viene comandato di farlo da chi indossa l'amuleto o si presenta una situazione predeterminata dall'incantatore, il guardiano lancia l'incantesimo accumulato con tutti i parametri predisposti dall'incantatore originale, senza bisogno di componenti. Quando l'incantesimo viene lanciato o qualsiasi nuovo incantesimo viene accumulato, tutti gli incantesimi precedentemente accumulati vengono persi.

\emph{\textbf{Natura di Costrutto.}} Il guardiano non ha bisogno di aria, cibo, bevande o sonno.

\emph{\textbf{Rigenerazione.}} Il guardiano protettore recupera 10 Punti Ferita all'inizio del proprio round se ne possiede ancora almeno 1.

\emph{\textbf{Vincolato.}} Il guardiano protettore è vincolato magicamente ad un amuleto. Finché il guardiano e l'amuleto sono sullo stesso piano di esistenza, chi indossa l'amuleto può richiamare telepaticamente il guardiano perché lo raggiunga, e il guardiano saprà la distanza e la direzione in cui si trova l'amuleto. Se il guardiano si trova entro 18 metri da chi indossa l'amuleto, metà dei danni subiti da chi lo indossa (arrotondati per difetto) vengono trasferiti al guardiano. Se l'amuleto viene distrutto, il guardiano è inabile finché non viene creato un amuleto di rimpiazzo. L'amuleto del guardiano può essere soggetto ad un attacco diretto qualora non sia indossato o trasportato da nessuno. Ha Difesa 10, 10 Punti Ferita e immunità ai danni da veleno. Costruire un amuleto richiede 1 settimana e costa 10000 mo in componenti.

\textbf{Azioni}

\emph{\textbf{Multiattacco.}} Il golem effettua due attacchi di pugno.

\emph{\textbf{Pugno.} Attacco con arma da mischia}: +8 a colpire, portata 1 m, un bersaglio.

\emph{Colpisce:} 14 (3d6 + 4) danni contundenti.

\textbf{Reazione: \emph{Scudo.}} Quando una creatura attacca chi indossa l'amuleto del guardiano, il guardiano conferisce un bonus di +2 alla sua Difesa, se entro 1 metro dal suo controllore.

%\addcontentsline{toc}{subsubsection}{H}
\pdfbookmark[3]{H}{H}

\mostro{Hobgoblin}
\noindent
\begin{description}[noitemsep, topsep=0pt, parsep=0pt, partopsep=0pt, leftmargin=0cm, labelwidth=2.2cm]
	\item[\textbf{Taglia/Tipo:}] Media umanoide (goblinoide), malvagio
	\item[\textbf{Caratt.:}] \resizebox{0.5\linewidth+1.8cm}{!}{For 1 Des 1 Cos 1 Int 0 Sag 0 Car -1}
	\item[\textbf{Punti Ferita:}] 24,  \textbf{Difesa:} 13,  \textbf{Iniziativa:} +1
	\item[\textbf{Movimento:}] 9 m
	\item[\textbf{Tiri Salvez.:}] \resizebox{0.5\linewidth+1.8cm}{!}{Tempra +3, Riflessi +3, Volontà +3}
	\item[\textbf{Sensi:}] Scurovisione 18 m
	\item[\textbf{Linguaggi:}] Comune, Goblin
	\item[\textbf{Sfida:}] 1/2 (100 PX)\smallskip
\end{description}

\emph{\textbf{Marziale.}} Una volta per round, come Reazione, l'hobgoblin può infliggere 7 (2d6) danni aggiuntivi ad una creatura che colpisce con un attacco con arma, se quella creatura si trova entro 1 metro da un alleato dell'hobgoblin che non sia inabile.

\textbf{Azioni}

\emph{\textbf{Spada Lunga.} Attacco con arma da mischia}: +5 a colpire, portata 1 m, un bersaglio.

\emph{Colpisce:} 5 (1d8 + 1) danni taglienti o 6 (1d10 + 1) danni taglienti se usata con due mani.

\emph{\textbf{Arco Lungo.} Attacco con arma a Distanza}: +3 a colpire, gittata 45m, un bersaglio.

\emph{Colpisce:} 5 (1d8 + 1) danni perforanti.

\textbf{Ecologia}\\
Ambiente: Colline Temperate\\
Organizzazione: Gruppo (4-9), banda da guerra (10-24) o tribù (25+ più 50\% non combattenti, 1 sergente di 3° livello per 20 adulti, 1 o 2 luogotenenti di 4° o 5° livello, 1 capo di 6°-8° livello, 6-12 Leopardi e 1-4 Ogre o 1-2 Troll)\\
\textbf{Categoria Tesoro}: Equipaggiamento da PNG (Corazza di Cuoio Borchiato, Scudo Leggero di Metallo, Spada Lunga, Arco Lungo con 20 Frecce, O)\\
\textbf{Descrizione}\\
Gli Hobgoblin sono una razza militarista e prolifica, rendendoli molto pericolosi in alcune regioni. Procreano rapidamente, sostituendo i membri caduti con nuovi soldati, mantenendo il loro numero costante nonostante le guerre. Dichiarano guerra facilmente, spesso per catturare nuovi schiavi, la cui vita è brutale e breve. Gli schiavi sono necessari per rimpiazzare quelli che muoiono o vengono mangiati.

Tra le razze goblinoidi, gli Hobgoblin sono i più civilizzati. Vedono i Bugbear come strumenti utili per missioni specifiche come omicidi e furti, mentre guardano i Goblin con vergogna e frustrazione, nonostante ammirino la loro tenacia. La maggior parte delle tribù Hobgoblin include comunque un piccolo gruppo di Goblin, relegati agli angoli peggiori dell'insediamento.

Molte tribù Hobgoblin combinano l'amore per la guerra con l'intelletto acuto. Sono affascinati dalle macchine d'assedio, dall'alchimia e dall'ingegneria complessa. Gli Hobgoblin particolarmente dotati vengono trattati da eroi e ottengono posizioni di alto rango nella tribù. Gli schiavi con menti raffinate sono apprezzati, rendendo comuni le incursioni nelle città naniche.

Gli Hobgoblin disprezzano la magia e diffidano dei maghi. I loro sciamani, temuti e rispettati, vivono ai margini del covo della tribù. Gli Hobgoblin sono alti circa 1.7 metri e pesano 80 kg.

\mostro{Idra}
\noindent
\begin{description}[noitemsep, topsep=0pt, parsep=0pt, partopsep=0pt, leftmargin=0cm, labelwidth=2.2cm]
	\item[\textbf{Taglia/Tipo:}] Enorme mostruosità, disallineato
	\item[\textbf{Caratt.:}] \resizebox{0.5\linewidth+1.8cm}{!}{For 5 Des 1 Cos 5 Int -4 Sag 0 Car -2}
	\item[\textbf{Punti Ferita:}] 167,  \textbf{Difesa:} 23,  \textbf{Iniziativa:} +1
	\item[\textbf{Movimento:}] 9 m, nuoto 9 m
	\item[\textbf{Tiri Salvez.:}] \resizebox{0.5\linewidth+1.8cm}{!}{\resizebox{0.5\linewidth+1.8cm}{!}{Tempra +13, Riflessi +9, Volontà +8}}
	\item[\textbf{Comp.:}] Consapevolezza +6
	\item[\textbf{Sensi:}] Scurovisione 18 m
	\item[\textbf{Sfida:}] 8 (3900 PX)\smallskip
\end{description}

\emph{\textbf{Teste Multiple.}} L'idra ha cinque teste. Finché ha più di una testa, l'idra ha +1d6 ai Tiri Salvezza contro le condizioni accecata, affascinata, assordata, spaventata, stordita o svenuta.

Ogni volta che l'idra subisce 25 o più danni in un singolo round, una delle sue teste muore. Se tutte le teste muoiono anche l'idra muore.

Al termine del suo round, l'idra ricresce due teste per ciascuna delle sue teste uccise dal suo ultimo round, a meno che non abbia subito danno da fuoco dal suo ultimo round. L'idra recupera 10 Punti Ferita per ogni testa ricresciuta in questo modo.

\emph{\textbf{Teste Reattive.}} Per ogni testa posseduta oltre la prima, l'idra riceve una Azione di Reazione extra che può essere usata solo per compiere prove di Consapevolezza.

\emph{\textbf{Trattenere il Fiato.}} L'idra può trattenere il fiato per 1 ora.

\emph{\textbf{Veglia.}} Mentre l'idra dorme, almeno una delle sue teste resta sveglia.

\textbf{Azioni}

\emph{\textbf{Multiattacco.}} L'idra effettua tanti attacchi di morso quante sono le sue teste.

\emph{\textbf{Morso.} Attacco con arma da mischia}: +9 a colpire, portata 3 m, un bersaglio.

\emph{Colpisce:} 10 (1d10 + 5) danni perforanti.

\emph{\textbf{Gas ripugnanti.}} l'idra emette gas dall'odore ripugnante. Tutte le creature nel raggio di 3 metri dall'idra devono fare un Tiro Salvezza su Tempra DC 21 o subire -2 al Tiro per Colpire per i successivi 2d4 round.

\textbf{Ecologia}\\
Ambiente: Paludi Temperate\\
Organizzazione: Solitario\\
\textbf{Categoria Tesoro}: E\\
\textbf{Descrizione}\\
L'idra è un drago a più teste, ma stupido e con grossi problemi di digestione.

%\addcontentsline{toc}{subsubsection}{I}
\pdfbookmark[3]{I}{I}

\mostro{Ippogrifo}
\noindent
\begin{description}[noitemsep, topsep=0pt, parsep=0pt, partopsep=0pt, leftmargin=0cm, labelwidth=2.2cm]
	\item[\textbf{Taglia/Tipo:}] Grande bestia, disallineato
	\item[\textbf{Caratt.:}] \resizebox{0.5\linewidth+1.8cm}{!}{For 3 Des 1 Cos 1 Int -4 Sag 1 Car -1}
	\item[\textbf{Punti Ferita:}] 33,  \textbf{Difesa:} 14,  \textbf{Iniziativa:} +1
	\item[\textbf{Movimento:}] 12 m, volo 18 m
	\item[\textbf{Tiri Salvez.:}] \resizebox{0.5\linewidth+1.8cm}{!}{Tempra +3, Riflessi +3, Volontà +3}
	\item[\textbf{Comp.:}] Consapevolezza +5
	\item[\textbf{Sfida:}] 1 (200 PX)\smallskip
\end{description}

\emph{\textbf{Vista Affinata.}} L'ippogrifo ha +1d6 nelle prove di Consapevolezza basate sulla vista.

\textbf{Azioni}

\emph{\textbf{Multiattacco.}} L'ippogrifo effettua due attacchi: uno con il becco e uno con gli artigli.

\emph{\textbf{Artigli.} Attacco con arma da mischia}: +5 a colpire, portata 1 m, un bersaglio.

\emph{Colpisce:} 10 (2d6 + 3) danni taglienti.

\emph{\textbf{Becco.} Attacco con arma da mischia}: +5 a colpire, portata 1 m, un bersaglio.

\emph{Colpisce:} 8 (1d10 + 3) danni perforanti.

\textbf{Reazione: \emph{Attacco d'opportunità}}: l'ippogrifo attacca se sta volando ed una creatura esce o attraversa la sua portata di 2 m.

\textbf{Ecologia}\\
Ambiente: Colline Temperate o Pianure\\
Organizzazione: Solitario, coppia o stormo (7-12)\\
\textbf{Categoria Tesoro}: Nessuno\\
\textbf{Descrizione}\\
L'ippogrifo è una creatura affascinante con ali, zampe anteriori e testa di un grande rapace, e corpo e coda di un magnifico cavallo.

Le piume dell'ippogrifo variano nei colori del falco o dell'aquila, con alcuni allevatori che hanno prodotto esemplari con piume bianche o color carbone. Il corpo è spesso baio, nocciola o grigio, con manti pezzati o palomino. Gli ippogrifi misurano 3,3 metri di lunghezza e pesano fino a 680 kg.

Territoriali e feroci nel proteggere il loro dominio, devono anche sorvegliare i cieli poiché sono prede per grifoni, viverne e giovani draghi. Nidificano in praterie erbose, colline e canyon, prediligendo mammiferi e brucando erba per aiutare la digestione.

Le comunità di allevatori offrono spesso ricompense per catturarli poiché possono rappresentare un pericolo per le mandrie. Di gran lunga più facili da addestrare rispetto ai grifoni, gli ippogrifi vengono usati come animali da monta da compagnie scelte di soldati a cavallo. Se catturati giovani, possono essere addestrati come animali normali, ma gli adulti richiedono un addestramento speciale.

Gli ippogrifi sono ovipari e il loro nido contiene solitamente un solo uovo, che vale 200 mo. Un giovane ippogrifo in salute vale 500 mo, mentre un ippogrifo completamente addestrato come cavalcatura può valere fino a 5000 mo. Possono trasportare 90 kg come carico leggero, 180 kg come carico medio e 270 kg come carico pesante, e necessitano di una sella esotica per essere cavalcati.

%\addcontentsline{toc}{subsubsection}{K}
\pdfbookmark[3]{K}{K}

\mostro{Kraken}
\noindent
\begin{description}[noitemsep, topsep=0pt, parsep=0pt, partopsep=0pt, leftmargin=0cm, labelwidth=2.2cm]
	\item[\textbf{Taglia/Tipo:}] Mastodontica mostruosità (titano), malvagio
	\item[\textbf{Caratt.:}] \resizebox{0.5\linewidth+1.8cm}{!}{For 10 Des 0 Cos 7 Int 6 Sag 4 Car 5}
	\item[\textbf{Punti Ferita:}] 461,  \textbf{Difesa:} 42,  \textbf{Iniziativa:} +6
	\item[\textbf{Movimento:}] 6 m, nuoto 18 m
	\item[\textbf{Tiri Salvez.:}] \resizebox{0.5\linewidth+1.8cm}{!}{\resizebox{0.5\linewidth+1.8cm}{!}{Tempra +30, Riflessi +23, Volontà +27}}
	\item[\textbf{Imm. Danni:}] Elettricità, armi +1
	\item[\textbf{Immunità:}] paralizzato, spaventato
	\item[\textbf{Sensi:}] visione del vero 36 m
	\item[\textbf{Linguaggi:}] comprende Abissale, Celestiale, Infernale e Druidico ma non può parlare, telepatia 36 m
	\item[\textbf{Sfida:}] 23 (50000 PX)\smallskip
\end{description}

\emph{\textbf{Anfibio.}} Il kraken può respirare aria e acqua.

\emph{\textbf{Libertà di Movimento.}} Il kraken ignora i terreni difficili, e gli effetti magici non possono ridurne la velocità o far sì che diventi intralciato. Può spendere 1 Azione per liberarsi dalle restrizioni non magiche o dall'essere afferrato.

\emph{\textbf{Mostro d'Assedio.}} Il kraken infligge danni doppi agli oggetti e le strutture.

\textbf{Azioni}

\emph{\textbf{Multiattacco.}} Il kraken effettua tre attacchi di tentacolo, ciascuno dei quali può essere rimpiazzato da un uso di Fiondare.

\emph{\textbf{Morso.} Attacco con arma da mischia}: +17 a colpire, portata 6 m, un bersaglio.

\emph{Colpisce:} 23 (3d8 + 10) danni perforanti. Se il bersaglio è una creatura di taglia Grande o inferiore afferrato dal kraken, quella creatura viene inghiottita, e l'afferrare ha termine. Mentre è inghiottita, la creatura è accecata e intralciata, ha copertura completa contro gli attacchi e altri effetti provenienti dall'esterno del kraken, e subisce 42 (12d6) danni da acido all'inizio di ciascun round del kraken.

Se il kraken subisce 50 o più danni in un singolo round da una creatura al suo interno, il kraken deve riuscire un Tiro Salvezza di Tempra DC 35 o vomitare tutte le creature inghiottite, che cadono prone in uno spazio entro 3 metri dal kraken. Se il kraken muore, una creatura inghiottita non risulta più intralciata da esso e può fuggire dal cadavere usando 2 Azioni e uscendo prona.

\emph{\textbf{Tentacolo.} Attacco con arma da mischia}: +17 a colpire, portata 9 m, un bersaglio.

\emph{Colpisce:} 20 (3d6 + 10) danni contundenti, e il bersaglio è afferrato (DC 18 per fuggire). Il kraken ha dieci tentacoli, ciascuno dei quali può afferrare un bersaglio.

\emph{\textbf{Fiondare.}} Un oggetto impugnato o una creatura afferrata dal kraken, di taglia Grande o inferiore viene lanciato di 18 metri in una direzione casuale e gettata prona. Se il bersaglio lanciato colpisce una superficie solida, subisce 3 (1d6) danni contundenti per ogni 3 metri percorsi. Se il bersaglio viene lanciato contro un'altra creatura, quella creatura deve riuscire un Tiro Salvezza di Riflessi DC 34 o subire lo stesso danno e cadere prona.

\emph{\textbf{Tempesta di Fulmini.}} Il kraken crea magicamente tre saette di energia, ciascuna delle quali può colpire un bersaglio entro 36 metri e che il kraken possa vedere. Il bersaglio deve effettuare un Tiro Salvezza di Riflessi DC 35, e subire 22 (4d10) danni da elettricità se fallisce il Tiro Salvezza, o la metà se lo riesce.

\textbf{Azioni Aggiuntive}

Il kraken può effettuare 3 Azioni aggiuntive, scelte tra le opzioni seguenti. Può usare solo un'opzione Aggiuntiva alla volta e solo al termine del round di un'altra creatura. Il kraken recupera le Azioni aggiuntive spese all'inizio del proprio round.

\textbf{Attacco di Tentacolo o Fiondare.} Il kraken effettua un attacco di tentacolo o usa Fiondare.

\textbf{Nube di Inchiostro (Costa 3 Azioni).} Mentre si trova sott'acqua, il kraken espelle una nube di inchiostro con un raggio di 18 metri. La nube si propaga intorno agli angoli, e quell'area è oscurata pesantemente per tutte le creature tranne il kraken. Ciascuna creatura a parte il kraken che termini il suo round nell'area deve riuscire un Tiro Salvezza su Tempra 34, subendo 16 (3d10) danni da veleno se fallisce il Tiro Salvezza, o la metà se lo riesce. Una forte corrente disperde la nube, che altrimenti svanisce al termine del prossimo round del kraken. \textbf{Tempesta di Fulmini (Costa 2 Azioni).} Il kraken usa Tempesta di Fulmini.

\textbf{Ecologia}\\
Ambiente: Qualsiasi Oceano\\
Organizzazione: Solitario\\
\textbf{Categoria Tesoro}: 2 H\\
\textbf{Descrizione}\\
Il leggendario kraken è una delle più grandi paure dei marinai, perché è una creatura della taglia di una balena, può colpire delle profondità senza esser visto, può comandare i venti e le condizioni meteorologiche necessarie alla nave per muoversi, e possiede il crudele intelletto della maggior parte dei più spietati e creativi criminali del mondo. Alcuni credono che i kraken siano una punizione divina, mentre altri li ritengono i veri signori delle profondità, che considerano le razze che respirano aria nient'altro che bestiame.

Molte leggende sono sorte in merito al fatto che comprenda il linguaggio druidico.

Un kraken è lungo quasi 30 metri e pesa 20000 kg.

%\addcontentsline{toc}{subsubsection}{L}
\pdfbookmark[3]{L}{L}

\mostro{Lamia}
\noindent
\begin{description}[noitemsep, topsep=0pt, parsep=0pt, partopsep=0pt, leftmargin=0cm, labelwidth=2.2cm]
	\item[\textbf{Taglia/Tipo:}] Grande mostruosità, malvagio
	\item[\textbf{Caratt.:}] \resizebox{0.5\linewidth+1.8cm}{!}{For 3 Des 1 Cos 2 Int 2 Sag 2 Car 3}
	\item[\textbf{Punti Ferita:}] 88,  \textbf{Difesa:} 18,  \textbf{Iniziativa:} +2
	\item[\textbf{Movimento:}] 9 m
	\item[\textbf{Tiri Salvez.:}] \resizebox{0.5\linewidth+1.8cm}{!}{Tempra +6, Riflessi +5, Volontà +6}
	\item[\textbf{Comp.:}] Furtività +3, Ingannare +7, Percepire Emozioni +4,
	\item[\textbf{Sensi:}] Scurovisione 18 m
	\item[\textbf{Linguaggi:}] Abissale, Comune
	\item[\textbf{Sfida:}] 4 (1100 PX)\smallskip
\end{description}

\emph{\textbf{Incantesimi Innati.}} La caratteristica da incantatore innato della lamia è il Carisma. La lamia può lanciare in maniera innata i seguenti incantesimi, senza bisogno di componenti materiali:

A volontà: \emph{\hyperlink{Camuffare Sé Stesso}{Camuffare Sé Stesso}} (qualsiasi forma umanoide), \emph{\hyperlink{Immagine Maggiore}{Immagine Maggiore}}

3/Giorno ciascuno: \emph{\hyperlink{Charme su Persone}{Charme su Persone}, \hyperlink{Immagine Speculare}{Immagine Speculare}, \hyperlink{Scrutare}{Scrutare}, \hyperlink{Suggestione}{Suggestione}}

1/Giorno: \emph{\hyperlink{Costrizione}{Costrizione}}

\textbf{Azioni}

\emph{\textbf{Multiattacco.}} La lamia effettua due attacchi: uno con gli artigli e uno con il pugnale o il Tocco Intossicante.

\emph{\textbf{Artigli.} Attacco con arma da mischia}: +6 a colpire, portata 1 m, un bersaglio.

\emph{Colpisce:} 14 (2d10 + 3) danni taglienti, 1 danno da Sanguinamento.

\emph{\textbf{Pugnale.} Attacco con arma da mischia}: +6 a colpire, portata 1 m, un bersaglio.

\emph{Colpisce:} 6 (1d4 + 4) danni perforanti.

\emph{\textbf{Tocco Intossicante.} Attacco con incantesimo in mischia}: +5 a colpire, portata 1 m, una creatura.

\emph{Colpisce:} Il bersaglio è maledetto per 1 ora da questa magia. Fino al termine della maledizione, il bersaglio ha -1d6 ai Tiri Salvezza su Volontà e a tutte le prove di competenza di Base.

\textbf{Ecologia}\\
Ambiente: Deserti Temperati\\
Organizzazione: Solitario, coppia o setta (3-12)\\
\textbf{Categoria Tesoro}: Pugnale+1, D\\

\textbf{Descrizione}\\
Le lamie, eredi di un'antica maledizione, hanno l'aspetto di donne snelle e attraenti dalla vita in su, mentre la parte inferiore del corpo è simile a quella di un possente leone. Le loro caratteristiche umanoidi presentano tratti felini: occhi stretti e ferini e denti simili a zanne. Una lamia tipica è alta 1,8 metri, lunga 2,3 metri e pesa oltre 325 kg.

Queste creature sono attratte da torrioni in rovina, città abbandonate e monumenti dimenticati, specialmente in zone aride. Tuttavia, prediligono templi decrepiti, trovando gioia nel vederli in rovina e cercando di danneggiare i luoghi sacri alle divinità buone.

Le lamie venerano le femmine anziane del loro gruppo, considerandole capi, madri e sciamane, e si legano a loro con fanatica reverenza. Anche se rifuggono la maggior parte delle religioni, vedendole come la fonte della maledizione che le affligge, le lamie anziane affermano di udire i sussurri del vento del deserto e di conoscere i capricci delle stelle, guidando così il loro popolo.

\mostro{Lich}
\noindent
\begin{description}[noitemsep, topsep=0pt, parsep=0pt, partopsep=0pt, leftmargin=0cm, labelwidth=2.2cm]
	\item[\textbf{Taglia/Tipo:}] Media non morto, tratti malvagi
	\item[\textbf{Caratt.:}] \resizebox{0.5\linewidth+1.8cm}{!}{For 0 Des 3 Cos 3 Int 5 Sag 2 Car 3}
	\item[\textbf{Punti Ferita:}] 405,  \textbf{Difesa:} 43,  \textbf{Iniziativa:} +5
	\item[\textbf{Movimento:}] 9 m
	\item[\textbf{Tiri Salvez.:}] \resizebox{0.5\linewidth+1.8cm}{!}{\resizebox{0.5\linewidth+1.8cm}{!}{Tempra +24, Riflessi +24, Volontà +23}}
	\item[\textbf{Res. Danni:}] Freddo, Elettricità, da Vuoto
	\item[\textbf{Imm. Danni:}] Veleno; da arma non magica
	\item[\textbf{Immunità:}] affascinato, paralizzato, affaticato, spaventato, sanguinamento
	\item[\textbf{Sensi:}] visione del vero 36 m
	\item[\textbf{Linguaggi:}] Comune più altre cinque lingue, Expiran
	\item[\textbf{Sfida:}] 21 (33000 PX)\smallskip
\end{description}

\emph{\textbf{Incantesimi.}} Il lich ha CM 18. La sua caratteristica da incantatore è l'Intelligenza. Il lich conosce i seguenti incantesimi:

Trucchetti (a volontà): \emph{\hyperlink{Mano Magica}{Mano Magica}, \hyperlink{Prestidigitazione}{Prestidigitazione}, \hyperlink{Raggio di Gelo}{Raggio di Gelo}}

livello 1 (4 slot): \emph{\hyperlink{Dardo arcano}{Dardo arcano}, \hyperlink{Individuazione del Magico}{Individuazione del Magico}, \hyperlink{Onda Tonante}{Onda Tonante}, \hyperlink{Scudo}{Scudo}}

livello 2 (3 slot): \emph{\hyperlink{Freccia Acida di Restser}{Freccia Acida di Restser}, \hyperlink{Immagine Speculare}{Immagine Speculare}, \hyperlink{Individuazione dei Pensieri}{Individuazione dei Pensieri}, \hyperlink{Invisibilità}{Invisibilità}}

livello 3 (3 slot): \emph{\hyperlink{Animare Morti}{Animare Morti}, \hyperlink{Controincantesimo}{Controincantesimo}, \hyperlink{Dissolvi Magie}{Dissolvi Magie}, \hyperlink{Palla di Fuoco}{Palla di Fuoco}}

livello 4 (3 slot): \emph{\hyperlink{Inaridire}{Inaridire}, \hyperlink{Porta Dimensionale}{Porta Dimensionale}}

livello 5 (3 slot): \emph{\hyperlink{Nebbia mortale}{Nebbia mortale}, \hyperlink{Scrutare}{Scrutare}}

livello 6 (1 slot): \emph{\hyperlink{Disintegrazione}{Disintegrazione}, \hyperlink{Globo di Invulnerabilità}{Globo di Invulnerabilità}}

livello 7 (1 slot): \emph{\hyperlink{Dito della Morte}{Dito della Morte}}

livello 8 (1 slot): \emph{\hyperlink{Dominare Mostri}{Dominare Mostri}, \hyperlink{Parola del Potere Stordire}{Parola del Potere Stordire}}

livello 9 (1 slot): \emph{\hyperlink{Parola del Potere Uccidere}{Parola del Potere Uccidere}}

\emph{\textbf{Natura Non Morta.}} Il lich non ha bisogno di aria, cibo, bevande o sonno.

\emph{\textbf{Resistenza Leggendaria (3/Giorno).}} Se il lich fallisce un Tiro Salvezza, può scegliere invece di riuscirvi.

\emph{\textbf{Resistenza allo Scacciare.}} Il lich ha +1d6 ai Tiri Salvezza contro gli effetti che scacciano i non morti.

\emph{\textbf{Rinvigorimento.}} Se possiede un filatterio, il lich distrutto ottiene un nuovo corpo in 1d10 giorni, recuperando tutti i suoi Punti Ferita e ritornando in attività. Il nuovo corpo compare entro 1 metro dal filatterio.

\emph{\textbf{Sacrifici di Anime.}} Un lich deve ogni periodicamente nutrire di anime il suo filatterio per sostenere la magia che mantiene il suo corpo e la sua coscienza. Per farlo usa l'incantesimo \emph{\hyperlink{Imprigionare}{Imprigionare}}. Invece di scegliere una delle normali opzioni dell'incantesimo, il lich lo impiega per intrappolare magicamente il corpo e l'anima del bersaglio all'interno del filatterio. Il filatterio deve trovarsi sullo stesso piano del lich, perché questo incantesimo funzioni. Il filatterio di un lich può contenere solo una creatura alla volta, e \emph{\hyperlink{Dissolvi Magie Avanzato}{Dissolvi Magie Avanzato}} lanciato con 4 critici magici sul filatterio libera qualsiasi creatura imprigionata al suo interno. Il filatterio divora un anima a settimana. Una creatura imprigionata nel filatterio oltre questo periodo viene consumata e distrutta, dopodiché nulla salvo un intervento di un Patrono potrà riportarla in vita.

Un lich che dimentichi o non riesca a mantenere il suo corpo con le anime sacrificate inizia a cascare a pezzi, e potrebbe infine trasformarsi in un semilich.

\textbf{Azioni}

\emph{\textbf{Tocco Paralizzante.} Attacco con incantesimo in mischia}: +14 a colpire, portata 1 m, una creatura.

\emph{Colpisce:} 10 (3d6) danni da freddo. Il bersaglio deve riuscire un Tiro Salvezza di Tempra DC 32 o restare paralizzato per 1 minuto. Il bersaglio può ripetere il Tiro Salvezza al termine di ciascun suo round, terminando l'effetto su di sé in caso di successo.

\textbf{Reazione: \emph{Incantesimi rapidi}} il lich risponde ad un attacco subito lanciando un incantesimo a sua scelta fino al 3 livello, come Reazione.

\textbf{Azioni Aggiuntive}

Il lich può effettuare 3 Azioni aggiuntive, scelte tra le opzioni seguenti. Può usare solo un'opzione Aggiuntiva alla volta e solo al termine del round di un'altra creatura. Il lich recupera le Azioni aggiuntive spese all'inizio del proprio round.

\emph{\textbf{Distruggere Vita (Costa 3 Azioni).}} Ogni creatura ad eccezione dei non morti entro 6 metri dal lich deve effettuare un Tiro Salvezza su Tempra DC 31 contro questa magia, subendo 21 (6d6) danni da Vuoto se fallisce il Tiro Salvezza, o la metà di questi danni se lo riesce. Le creature diventano Affaticate.

\emph{\textbf{Sguardo Spaventoso (Costa 2 Azioni).}} Il lich fissa il suo sguardo su di una creatura visibile entro 3 metri da esso. Il bersaglio deve riuscire un Tiro Salvezza di Volontà DC 31 contro questa magia o restare spaventato per 1 minuto. Il bersaglio spaventato può ripetere il Tiro Salvezza al termine di ciascun suo round, terminando l'effetto su di sé in caso di successo. Se il Tiro Salvezza del bersaglio è riuscito o l'effetto per lui ha termine, il bersaglio è immune allo sguardo del lich per le successive 24 ore.

\emph{\textbf{Tocco Paralizzante (Costa 2 Azioni).}} Il lich usa il suo Tocco Paralizzante.

\emph{\textbf{Trucchetto.}} Il lich lancia un trucchetto.

\textbf{Ecologia}\\
Ambiente: Qualsiasi\\
Organizzazione: Solitario\\
\textbf{Categoria Tesoro}: Equipaggiamento da PNG (Anello di Protezione +2, Fascia della Sapienza +2 (Consapevolezza), Stivali della Levitazione, pergamena di \hyperlink{Dominare Persone}{Dominare Persone}, pergamena di Teletrasporto, pozione di Invisibilità)\\
\textbf{Descrizione}\\
Poche creature sono più temute dei lich. Apice delle arti Necromantiche, il lich è un incantatore che ha scelto di rinunciare alla vita ed ingannare la morte diventando non morto. Anche se molti di coloro che raggiungono simili vette di potenza farebbero di tutto per raggiungere l'immortalità, l'idea di diventare un lich è aborrita da molte creature. Il processo prevede di estrarre la forza vitale dell'incantatore e imprigionarla in un filatterio preparato in modo speciale; l'incantatore cede la sua vita, ma rimane intrappolato tra la vita e la morte, e fintanto che il suo filatterio rimane intatto può continuare le sue ricerche e il suo lavoro senza temere il passare del tempo.

Esistono anche rarissimi Lich buoni, ma come dice il detto sono più rari di un dente di Roc.

\mostro{Lucertoloide}
\noindent
\begin{description}[noitemsep, topsep=0pt, parsep=0pt, partopsep=0pt, leftmargin=0cm, labelwidth=2.2cm]
	\item[\textbf{Taglia/Tipo:}] Media umanoide (lucertoloide), neutrale
	\item[\textbf{Caratt.:}] \resizebox{0.5\linewidth+1.8cm}{!}{For 2 Des 0 Cos 1 Int -2 Sag 1 Car -2}
	\item[\textbf{Punti Ferita:}] 24,  \textbf{Difesa:} 12,  \textbf{Iniziativa:} +0
	\item[\textbf{Movimento:}] 9 m, nuoto 9 m
	\item[\textbf{Tiri Salvez.:}] \resizebox{0.5\linewidth+1.8cm}{!}{Tempra +3, Riflessi +3, Volontà +3}
	\item[\textbf{Comp.:}] Furtività +4, Consapevolezza +3, Sopravvivenza +5
	\item[\textbf{Linguaggi:}] Draconico
	\item[\textbf{Sfida:}] 1/2 (100 PX)\smallskip
\end{description}

\emph{\textbf{Trattenere il Fiato.}} Il lucertoloide può trattenere il fiato per 15 minuti.

\textbf{Azioni}

\emph{\textbf{Multiattacco.}} Il lucertoloide effettua due attacchi in mischia, ciascuno con un'arma diversa.

\emph{\textbf{Giavellotto.} Attacco con arma da mischia o a Distanza}: +4 a colpire, portata 1 m o gittata 12m, un bersaglio.

\emph{Colpisce:} 5 (1d6 + 2) danni perforanti.

\emph{\textbf{Morso.} Attacco con arma da mischia}: +4 a colpire, portata 1 m, un bersaglio.

\emph{Colpisce:} 5 (1d6 + 2) danni perforanti.

\emph{\textbf{Randello Pesante.} Attacco con arma da mischia}: +4 a colpire, portata 1 m, un bersaglio.

\emph{Colpisce:} 5 (1d6 + 2) danni contundenti.

\emph{\textbf{Scudo Appuntito.} Attacco con arma da mischia}: +4 a colpire, portata 1 m, un bersaglio.

\emph{Colpisce:} 5 (1d6 + 2) danni perforanti.

\textbf{Ecologia}\\
Ambiente: paludi temperate\\
Organizzazione: solitario, coppia, banda (3-12) o tribù (13-60)\\
\textbf{Categoria Tesoro}: Equipaggiamento da PNG (Scudo Pesante di Legno, Mazza chiodata, 3 Giavellotti)\\

\textbf{Descrizione}\\
I lucertoloidi sono rettili predatori orgogliosi e potenti che fanno le loro case comuni in sparuti villaggi nei recessi di paludi e acquitrini. Privi di interesse verso la colonizzazione delle terre aride e soddisfatti delle loro semplici armi e dei rituali che li hanno serviti bene per millenni, i lucertoloidi sono visti da molte delle altre razze come selvaggi retrogradi, ma all'interno delle loro isolate comunità sono in realtà un popolo vitale ricco di tradizioni e con una storia orale che risale a prima che l'uomo camminasse in posizione eretta.

La maggior parte dei lucertoloidi è alta dagli 1,8 ai 2,1 metro e pesa dai 100 ai 125 kg, ed ha i possenti muscoli coperti da scaglie grigie, verdi o marroni. Alcune razze hanno piccole creste dorsali o collari dai colori brillanti, e tutte nuotano bene spostandosi con rapidi movimenti della loro possente coda lunga 1,2 metri. Anche se sono pienamente a loro agio in acqua, trattengono il fiato e tornano alle loro abitazioni poste su colline artificiali per riprodursi e dormire. Poiché il loro sangue da rettile li rende lenti al freddo, molti lucertoloidi cacciano e lavorano durante il giorno e si ritirano nelle loro dimore di notte per rannicchiarsi con gli altri della loro tribù a condividere il calore di grandi fuochi di torba.

Anche se generalmente sono neutrali, il comportamento scostante dei lucertoloidi, il loro strenuo rifiuto dei \emph{doni} della civilizzazione, e la leggendaria ferocia in battaglia li fa mal giudicare dalla maggioranza degli umanoidi. Questi tratti derivano da buone ragioni, tuttavia, poiché il loro basso tasso di riproduzione non ha eguali tra gli umanoidi a sangue caldo, e se le tribù non difendessero i loro territori paludosi fino all'ultimo respiro si troverebbero presto sopraffatte da orde di mammiferi. Per quanto riguarda la loro propensione a mangiare i corpi dei morti sia amici che nemici, i pratici lucertoloidi sono lesti a sottolineare che la vita è dura nella palude, e nulla deve andare sprecato.

I lucertoloidi presentati qui vivono in ambienti paludosi. Le tribù lucertoloidi possono vivere altrettanto bene in altri ambienti, ma come velocità ottengono Scalare 5 metri al posto di Nuotare.

%\addcontentsline{toc}{subsubsection}{M}
\pdfbookmark[3]{M}{M}

\mostro{Maledetto immortale}
\noindent
\begin{description}[noitemsep, topsep=0pt, parsep=0pt, partopsep=0pt, leftmargin=0cm, labelwidth=2.2cm]
	\item[\textbf{Taglia/Tipo:}] Media aberrazione (umano), tendenzialmente folle
	\item[\textbf{Caratt.:}] \resizebox{0.5\linewidth+1.8cm}{!}{For 3 Des 1 Cos 2 Int -1 Sag -2 Car -2}
	\item[\textbf{Punti Ferita:}] 88,  \textbf{Difesa:} 18,  \textbf{Iniziativa:} +1
	\item[\textbf{Movimento:}] 9 m
	\item[\textbf{Tiri Salvez.:}] \resizebox{0.5\linewidth+1.8cm}{!}{Tempra +6, Riflessi +5, Volontà +3}
	\item[\textbf{Comp.:}] Consapevolezza +3, professione che aveva in vita
	\item[\textbf{Imm. Danni:}] Freddo, Veleno, Fuoco, da Vuoto
	\item[\textbf{Immunità:}] affascinato, pietrificato, spaventato
	\item[\textbf{Linguaggi:}] Comune, Nanico, Elfico
	\item[\textbf{Sfida:}] 4 (1100 PX)\smallskip
\end{description}

\emph{\textbf{Immortale}} Il Maledetto immortale rigenera 1 Punto Ferita a round, tranne se ha subito danni da acido. Ciò gli permette di rigenerare arti e tornare in vita. \hyperlink{Rimuovi Maledizione}{Rimuovi Maledizione} a DC 30 lo uccide istantaneamente.

\emph{\textbf{Natura diversa}} Il Maledetto immortale non mangia, beve, dorme, invecchia. Non è un non morto

\textbf{Azioni}

\emph{\textbf{Multiattacco.}} Il Maledetto immortale fa tre attacchi con la spada lunga.

\emph{\textbf{Spada.} Attacco con arma da mischia}: +6 a colpire, portata 1 m, un bersaglio.

\emph{Colpisce:} 12 (1d10 + 7) danni da taglio.

\textbf{Ecologia}\\
Ambiente: Qualsiasi\\
Organizzazione: Solitario\\
\textbf{Categoria Tesoro}: Equipaggiamento da PNG (Armatura di Cuoio Borchiato, 2 Pugnali, Spada, J)\\
\textbf{Descrizione}\\
Il Maledetto immortale è una persona maledetta spesso da un Patrono o da una potente incantatore con la maledizione della folle vita immortale. La maledizione rompe l'equilibro della persona e questa si ritrova a girovagare senza una meta od un obiettivo. Ogni tanto si ricordano chi erano ed allora proseguono nella ricerca di chi li ha maledetti.
Con lo scopo di farsi definitivamente uccidere si getta in ogni scontro sperando che l'avversario sia in grado di ucciderlo una volta per tutte.

\mostro{Monete affamate}\label{moneteaffamate}
\noindent
\begin{description}[noitemsep, topsep=0pt, parsep=0pt, partopsep=0pt, leftmargin=0cm, labelwidth=2.2cm]
	\item[\textbf{Taglia/Tipo:}] Minuscola aberrazione, fortemente malvagia
	\item[\textbf{Caratt.:}] \resizebox{0.5\linewidth+1.8cm}{!}{For -3 Des 1 Cos 3 Int -2 Sag 0 Car -1}
	\item[\textbf{Punti Ferita:}] 24,  \textbf{Difesa:} 13,  \textbf{Iniziativa:} +1
	\item[\textbf{Movimento:}] 3 m
	\item[\textbf{Tiri Salvez.:}] \resizebox{0.5\linewidth+1.8cm}{!}{Tempra +3, Riflessi +3, Volontà +3}
	\item[\textbf{Comp.:}] Consapevolezza +3
	\item[\textbf{Imm. Danni:}] Freddo, Veleno, Elettricità, da Vuoto
	\item[\textbf{Immunità:}] affascinato, pietrificato, spaventato
	\item[\textbf{Sfida:}] 1/2 (100 PX)\smallskip
\end{description}

Le Monete affamate attaccano sempre in gruppi da almeno 8 monete, quelle tenute in mano per contare...

\textbf{Azioni}

\emph{\textbf{Multiattacco.}} La Moneta affamata effettua due attacchi. Può usare il Morso due volte oppure usare il Morso e usare il Pungiglione.

\emph{\textbf{Morso.} Attacco con arma da mischia}: +4 a colpire, portata 0 m, un bersaglio.

\emph{Colpisce:} 2 danni perforanti.

\emph{\textbf{Pungiglione.} Attacco con arma da mischia}: +4 a colpire, portata 1 m, un bersaglio.

\emph{Colpisce:} 3 danni da Veleno. Tiro Salvezza su Tempra DC 13 o essere Rallentati 1/1r.

\textbf{Ecologia}\\
Ambiente: Qualsiasi\\
Organizzazione: Gruppi (3d12)\\
\textbf{Categoria Tesoro}: M, N, O\\
\textbf{Descrizione}\\
La Moneta affamata non è distinguibile da una normale moneta finché non osservata molto attentamente.
Voraci ed affamate amano nascondersi nelle pile di monete di cui si nutrono per assorbire i metalli che gli conferiscono poi il \emph{guscio} e l'aspetto di ordinarie monete. Attaccano sempre in gruppo, solitamente aspettando che qualcuno le tenga in mano per contarle. Ogni 10 Monete affamate, se \emph{svuotate e fuse}, è possibile ricavare abbastanza metallo per una vera moneta.
Monete affamate di Oro o Platino sono solitamente più robuste ed ancora più affamate. Dice la leggenda che una Moneta affamata non attaccherà un Devoto di Rezh.

\mostro{Cinghiale Mannaro}
\noindent
\begin{description}[noitemsep, topsep=0pt, parsep=0pt, partopsep=0pt, leftmargin=0cm, labelwidth=2.2cm]
	\item[\textbf{Taglia/Tipo:}] Media umanoide , mutaforma, malvagio
	\item[\textbf{Caratt.:}] \resizebox{0.5\linewidth+1.8cm}{!}{For 3 Des 0 Cos 2 Int 0 Sag 0 Car -1}
	\item[\textbf{Punti Ferita:}] 88,  \textbf{Difesa:} 17,  \textbf{Iniziativa:} +0
	\item[\textbf{Movimento:}] 9 m (12 m in forma di cinghiale)
	\item[\textbf{Tiri Salvez.:}] \resizebox{0.5\linewidth+1.8cm}{!}{Tempra +6, Riflessi +4, Volontà +4}
	\item[\textbf{Imm. Danni:}] da arma non magica o che non sia argentata
	\item[\textbf{Linguaggi:}] Comune (non può parlare in forma di cinghiale)
	\item[\textbf{Sfida:}] 4 (1100 PX)\smallskip
\end{description}

\emph{\textbf{Carica (Solo Forma di Cinghiale o Ibrida).}} Se il cinghiale mannaro si muove in linea retta di almeno 5 metri verso un bersaglio e poi lo colpisce con le zanne durante lo stesso round, il bersaglio subisce 7 (2d6) danni taglienti aggiuntivi. Se il bersaglio è una creatura, deve riuscire un Tiro Salvezza di Tempra DC 15 o cadere prono. 1 Azione.

\emph{\textbf{Implacabile (Ricarica dopo un 1 ora).}} Se il cinghiale mannaro subisce 14 danni o meno che lo ridurrebbero a 0 Punti Ferita, scende invece a 1 punto ferita.

\emph{\textbf{Mutaforma.}} Il cinghiale mannaro può usare 2 Azioni per trasformarsi in un ibrido cinghiale-umanoide o in un cinghiale, o per tornare alla sua vera forma, che è umanoide. Le sue statistiche, a parte la Difesa, sono le stesse in tutte le forme. Qualsiasi equipaggiamento stia indossando o trasportando non viene trasformato. Alla morte ritorna alla sua vera forma.

\textbf{Azioni}

\emph{\textbf{Multiattacco (Solo in Forma Umanoide o Ibrida).}} Il cinghiale mannaro effettua due attacchi, di cui solo uno può essere con le zanne.

\emph{\textbf{Maglio (Soltanto in Forma Umanoide o Ibrida).} Attacco con arma da mischia}: +6 a colpire, portata 1 m, un bersaglio.

\emph{Colpisce:} 10 (2d6 + 3) danni contundenti.

\emph{\textbf{Zanne (Soltanto in Forma di Cinghiale o Ibrida).} Attacco con arma da mischia}: +6 a colpire, portata 1 m, un bersaglio.

\emph{Colpisce:} 10 (2d6 + 3) danni taglienti. Se il bersaglio è un umanoide, deve riuscire un Tiro Salvezza di Tempra DC 15 o venire maledetto dalla licantropia del cinghiale mannaro.

\textbf{Ecologia}\\
Ambiente: Qualsiasi Foresta o Pianura\\
Organizzazione: Solitario, coppia, famiglia (3-8) o truppa (3-8 più 1-4 Cinghiali)\\
\textbf{Categoria Tesoro}: Equipaggiamento da PNG (Armatura di Cuoio Borchiato, 2 Pugnali, K)\\
\textbf{Descrizione}\\
Nella loro forma umanoide, i cinghiali mannari tendono a essere tozzi, con nasi all'insù, pelo ispido e incisivi prominenti. Hanno capelli rossi, castani o neri ma alcuni sono anche biondi, canuti o calvi. Hanno di norma peluria sul labbro superiore e i maschi di solito non riescono a far crescere la barba. Poiché sono testardi e aggressivi hanno piccole comunità di loro simili e non si mischiano ai non licantropi: di solito vivono in piccole fattorie dall'aspetto assolutamente normale. Tendono ad avere grandi famiglie e molti figli.

\mostro{Lupo Mannaro}
\noindent
\begin{description}[noitemsep, topsep=0pt, parsep=0pt, partopsep=0pt, leftmargin=0cm, labelwidth=2.2cm]
	\item[\textbf{Taglia/Tipo:}] Media umanoide (umano, mutaforma), malvagio
	\item[\textbf{Caratt.:}] \resizebox{0.5\linewidth+1.8cm}{!}{For 2 Des 1 Cos 2 Int 0 Sag 0 Car 0}
	\item[\textbf{Punti Ferita:}] 70,  \textbf{Difesa:} 17,  \textbf{Iniziativa:} +1
	\item[\textbf{Movimento:}] 9 m (12 m in forma di lupo)
	\item[\textbf{Tiri Salvez.:}] \resizebox{0.5\linewidth+1.8cm}{!}{Tempra +5, Riflessi +4, Volontà +3}
	\item[\textbf{Comp.:}] Furtività +3, Consapevolezza +4
	\item[\textbf{Imm. Danni:}] da arma non magica o che non sia argentata
	\item[\textbf{Linguaggi:}] Comune (non può parlare in forma di lupo)
	\item[\textbf{Sfida:}] 3 (700 PX)\smallskip
\end{description}

\emph{\textbf{Mutaforma.}} Il lupo mannaro può usare una Azione per trasformarsi in un ibrido lupo-umanoide o in un lupo, o per tornare alla sua vera forma, che è umanoide. Le sue statistiche, a parte la Difesa, sono le stesse in tutte le forme. Qualsiasi equipaggiamento stia indossando o trasportando non viene trasformato. Alla morte ritorna alla sua vera forma.

\emph{\textbf{Udito e Olfatto Affinato.}} Il lupo mannaro ha +1d6 nelle prove di Consapevolezza basate su udito o olfatto.

\textbf{Azioni}

\emph{\textbf{Multiattacco (Soltanto in Forma Umanoide o Ibrida).}} Il lupo mannaro effettua due attacchi: uno con il morso e uno con gli artigli o la lancia.

\emph{\textbf{Artigli (Soltanto in Forma Ibrida).} Attacco con arma da mischia}: +5 a colpire, portata 1 m, una creatura.

\emph{Colpisce:} 7 (2d4 + 2) danni taglienti.

\emph{\textbf{Lancia (Soltanto in Forma Umanoide).} Attacco con arma da mischia o a Distanza}: +4 a colpire, portata 1 m o gittata 6m, una creatura.

\emph{Colpisce:} 5 (1d6 + 2) danni perforanti o 6 (1d8 + 2) danni perforanti se usata con due mani in un attacco di mischia.

\emph{\textbf{Morso (Soltanto in Forma di Lupo o Ibrida).} Attacco con arma da mischia}: +5 a colpire, portata 1 m, un bersaglio.

\emph{Colpisce:} 6 (1d8 + 2) danni perforanti. Se il bersaglio è un umanoide, deve riuscire un Tiro Salvezza di Tempra DC 15 o venir maledetto dalla licantropia del lupo mannaro.

\textbf{Ecologia}\\
Ambiente: Qualsiasi Terreno\\
Organizzazione: Solitario, coppia o branco (3-6)\\
\textbf{Categoria Tesoro}: Equipaggiamento da PNG (Cotta di Maglia, Spada Lunga, Balestra Leggera con 20 Quadrelli, K)\\
\textbf{Descrizione}\\
Nella forma umana i lupi mannari somigliano a persone normali, anche se alcuni tendono ad avere un aspetto ferino e capelli ribelli. Sopracciglia che crescono unendosi, dito indice più lungo del medio e strane voglie sul palmo della mano sono tutti segni comunemente accettati che una persona sia in realtà un lupo mannaro. Naturalmente, questi segni rivelatori non sono sempre accurati, perché questi tratti fisici esistono anche nelle persone normali, ma nelle zone dove i lupi mannari sono un problema comune, questi tratti possono essere considerati schiaccianti a prescindere.

\mostro{Orso Mannaro}
\noindent
\begin{description}[noitemsep, topsep=0pt, parsep=0pt, partopsep=0pt, leftmargin=0cm, labelwidth=2.2cm]
	\item[\textbf{Taglia/Tipo:}] Media umanoide (umano, mutaforma), buono
	\item[\textbf{Caratt.:}] \resizebox{0.5\linewidth+1.8cm}{!}{For 4 Des 0 Cos 3 Int 0 Sag 1 Car 1}
	\item[\textbf{Punti Ferita:}] 108,  \textbf{Difesa:} 18,  \textbf{Iniziativa:} +0
	\item[\textbf{Movimento:}] 9 m (12 m, scalata 9 m in forma di orso o forma ibrida)
	\item[\textbf{Tiri Salvez.:}] \resizebox{0.5\linewidth+1.8cm}{!}{Tempra +8, Riflessi +5, Volontà +6}
	\item[\textbf{Comp.:}] Consapevolezza +7
	\item[\textbf{Imm. Danni:}] da arma non magica o che non sia argentata
	\item[\textbf{Linguaggi:}] Comune (non può parlare in forma di orso)
	\item[\textbf{Sfida:}] 5 (1800 PX)\smallskip
\end{description}

\emph{\textbf{Mutaforma.}} L'orso mannaro può usare una Azione per trasformarsi in un ibrido orso-umanoide o in un orso, o per tornare alla sua vera forma, che è umanoide. Le sue statistiche, a parte la Difesa, sono le stesse in tutte le forme. Qualsiasi equipaggiamento stia indossando o trasportando non viene trasformato. Alla morte ritorna alla sua vera forma.

\emph{\textbf{Olfatto Affinato.}} L'orso mannaro ha +1d6 nelle prove di Consapevolezza basate sull'olfatto.

\textbf{Azioni}

\emph{\textbf{Multiattacco.}} In forma di orso, l'orso mannaro effettua due attacchi di artiglio. In forma umanoide, effettua due attacchi di ascia bipenne. In forma ibrida, può attaccare come un orso o un umanoide.

\emph{\textbf{Artiglio (Soltanto in Forma di Orso o Ibrida).} Attacco con arma da mischia}: +7 a colpire, portata 1 m, un bersaglio.

\emph{Colpisce:} 13 (2d8 + 2) danni taglienti.

\emph{\textbf{Ascia Bipenne (Soltanto in Forma Umanoide o Ibrida).} Attacco con arma da mischia}: +6 a colpire, portata 1 m, un bersaglio.

\emph{Colpisce:} 10 (1d12 + 4) danni taglienti.

\emph{\textbf{Morso (Soltanto in Forma di Orso o Ibrida).} Attacco con arma da mischia}: +6 a colpire, portata 1 m, un bersaglio.

\emph{Colpisce:} 15 (2d10 + 4) danni perforanti. Se il bersaglio è un umanoide, deve riuscire un Tiro Salvezza di Tempra DC 16 o venir maledetto dalla licantropia dell'orso mannaro.

\textbf{Ecologia}\\
Ambiente: Qualsiasi Foresta\\
Organizzazione: Solitario, coppia, famiglia (3-6) o truppa (3-6 più 1-4 orsi Neri o Grigi)\\
\textbf{Categoria Tesoro}: Equipaggiamento da PNG (Giaco di Maglia, Ascia da Battaglia Perfetta, 2 Asce da Lancio Perfette, K)\\
\textbf{Descrizione}\\
Nelle loro forme umanoidi, gli orsi mannari tendono a essere muscolosi e con spalle larghe, tratti aspri e occhi scuri. Hanno capelli rossi, castani o neri e sembrano abituati a una vita di duro lavoro. Anche se i più benigni fra i licantropi, sono evitati dalla maggior parte delle persone normali, che temono la loro trasformazione animalesca. Per la maggior parte vivono in zone boschive isolate o in piccole unità familiari della loro stessa specie. Evitano di affrontare gli stranieri, ma non esitano se devono scacciare umanoidi malvagi dai loro territori.

\mostro{Ratto Mannaro}
\noindent
\begin{description}[noitemsep, topsep=0pt, parsep=0pt, partopsep=0pt, leftmargin=0cm, labelwidth=2.2cm]
	\item[\textbf{Taglia/Tipo:}] Media umanoide (umano, mutaforma), malvagio
	\item[\textbf{Caratt.:}] \resizebox{0.5\linewidth+1.8cm}{!}{For 0 Des 2 Cos 1 Int 0 Sag 0 Car -1}
	\item[\textbf{Punti Ferita:}] 51,  \textbf{Difesa:} 16,  \textbf{Iniziativa:} +2
	\item[\textbf{Movimento:}] 9 m
	\item[\textbf{Tiri Salvez.:}] \resizebox{0.5\linewidth+1.8cm}{!}{Tempra +3, Riflessi +4, Volontà +3}
	\item[\textbf{Comp.:}] Furtività +4, Consapevolezza +2
	\item[\textbf{Imm. Danni:}] da arma non magica o che non sia argentata
	\item[\textbf{Sensi:}] Scurovisione 18 m (solo in forma di ratto)
	\item[\textbf{Linguaggi:}] Comune (non può parlare in forma di ratto)
	\item[\textbf{Sfida:}] 2 (450 PX)\smallskip
\end{description}

\emph{\textbf{Mutaforma.}} Il ratto mannaro può usare una Azione per trasformarsi in un ibrido ratto-umanoide o in un ratto, o per tornare alla sua vera forma, che è umanoide. Le sue statistiche, a parte la Difesa, sono le stesse in tutte le forme. Qualsiasi equipaggiamento stia indossando o trasportando non viene trasformato. Alla morte ritorna alla sua vera forma.

\emph{\textbf{Olfatto Affinato.}} Il ratto mannaro ha +1d6 nelle prove di Consapevolezza basate sull'olfatto.

\textbf{Azioni}

\emph{\textbf{Multiattacco (Solo in Forma Umanoide o Ibrida).}} Il ratto mannaro effettua due attacchi, di cui solo uno può essere con il morso.

\emph{\textbf{Spada Corta (Soltanto in Forma Umanoide o Ibrida).} Attacco con arma da mischia}: +5 a colpire, portata 1 m, un bersaglio.

\emph{Colpisce:} 5 (1d6 + 2) danni perforanti.

\emph{\textbf{Balestra a mano (Soltanto in Forma Umanoide o Ibrida).} Attacco con arma a Distanza}: +6 a colpire, gittata 9m, un bersaglio.

\emph{Colpisce:} 5 (1d6 + 2) danni perforanti.

\emph{\textbf{Morso (Soltanto in Forma di Ratto o Ibrida).} Attacco con arma da mischia}: +5 a colpire, portata 1 m, un bersaglio.

\emph{Colpisce:} 4 (1d4 + 2) danni perforanti. Se il bersaglio è un umanoide, deve riuscire un Tiro Salvezza di Tempra DC 13 o venir maledetto dalla licantropia del ratto mannaro.

\textbf{Ecologia}\\
Ambiente: Qualsiasi Urbano\\
Organizzazione: Solitario, coppia, branco (5-10) o gilda (11-30 più 5-12 Ratti Crudeli)\\
\textbf{Categoria Tesoro}: Equipaggiamento da PNG (Armatura di Cuoio Borchiato Perfetta, Spada Corta, Balestra Leggera con 20 Quadrelli, K)\\
\textbf{Descrizione}\\
I ratti mannari naturali sono bassi, asciutti e muscolosi, con occhi attenti e vispi, e hanno movimenti nervosi. I maschi spesso hanno sottili baffi striminziti.

\mostro{Tigre Mannara}
\noindent
\begin{description}[noitemsep, topsep=0pt, parsep=0pt, partopsep=0pt, leftmargin=0cm, labelwidth=2.2cm]
	\item[\textbf{Taglia/Tipo:}] Media umanoide (umano, mutaforma), neutrale
	\item[\textbf{Caratt.:}] \resizebox{0.5\linewidth+1.8cm}{!}{For 3 Des 2 Cos 3 Int 0 Sag 1 Car 0}
	\item[\textbf{Punti Ferita:}] 89,  \textbf{Difesa:} 19,  \textbf{Iniziativa:} +2
	\item[\textbf{Movimento:}] 9 m (12 m in forma di tigre)
	\item[\textbf{Tiri Salvez.:}] \resizebox{0.5\linewidth+1.8cm}{!}{Tempra +7, Riflessi +6, Volontà +5}
	\item[\textbf{Comp.:}] Furtività +4, Consapevolezza +5
	\item[\textbf{Imm. Danni:}] da arma non magica che non siano argentati
	\item[\textbf{Sensi:}] Scurovisione 18 m
	\item[\textbf{Linguaggi:}] Comune (non può parlare in forma di tigre)
	\item[\textbf{Sfida:}] 4 (1100 PX)\smallskip
\end{description}

\emph{\textbf{Balzo.}} Se la tigre mannara si muove di almeno 5 metri in linea retta verso una creatura e la colpisce con un attacco di artiglio durante lo stesso round, il bersaglio deve riuscire un Tiro Salvezza su Tempra DC 16 o cadere prono. Se il bersaglio è prono, la tigre mannara può effettuare un attacco di morso contro di esso come Azione Immediata.

\emph{\textbf{Mutaforma.}} La tigre mannara può usare una Azione per trasformarsi in un ibrido tigre-umanoide o in una tigre, o per tornare alla sua vera forma, che è umanoide. Le sue statistiche, a parte la Difesa, sono le stesse in tutte le forme. Qualsiasi equipaggiamento stia indossando o trasportando non viene trasformato. Alla morte ritorna alla sua vera forma.

\emph{\textbf{Olfatto e Udito Affinato.}} La tigre mannara ha +1d6 nelle prove di Consapevolezza basate su olfatto e udito.

\textbf{Azioni}

\emph{\textbf{Multiattacco (Solo in Forma Umanoide o Ibrida).}} In forma umanoide, la tigre mannara effettua due attacchi di scimitarra o due attacchi di arco lungo. In forma ibrida, può attaccare come un umanoide o effettuare due attacchi di artiglio.

\emph{\textbf{Artiglio (Soltanto in Forma di Tigre o Ibrida).} Attacco con arma da mischia}: +5 a colpire, portata 1 m, un bersaglio.

\emph{Colpisce:} 7 (1d8 + 3) danni taglienti, 1 danno da Sanguinamento.

\emph{\textbf{Morso (Soltanto in Forma di Tigre o Ibrida).} Attacco con arma da mischia}: +6 a colpire, portata 1 m, un bersaglio.

\emph{Colpisce:} 8 (1d10 + 3) danni perforanti. Se il bersaglio è un umanoide, deve riuscire un Tiro Salvezza di Tempra DC 16 o venir maledetto dalla licantropia della tigre mannara.

\emph{\textbf{Scimitarra (Soltanto in Forma Umanoide o Ibrida).} Attacco con arma da mischia}: +5 a colpire, portata 1 m, un bersaglio.

\emph{Colpisce:} 6 (1d6 + 3) danni taglienti.

\emph{\textbf{Arco Lungo (Soltanto in Forma Umanoide o Ibrida).} Attacco con arma a Distanza}: +6 a colpire, gittata 45m, un bersaglio.

\emph{Colpisce:} 6 (1d8 + 2) danni perforanti.

\textbf{Reazione: \emph{Attacco d'opportunità}}: la tigre mannara effettua un attacco ad una creatura che attraversi o esca dalla sua portata di 1 metro.

\textbf{Ecologia}
Ambiente: Qualsiasi Pianura o Palude\\
Organizzazione: Solitario o coppia\\
\textbf{Categoria Tesoro}: Equipaggiamento da PNG (Armatura di Cuoio Borchiato, Spada Corta, 2 Pugnali, K)\\
\textbf{Descrizione}\\
Le tigri mannare in forma umanoide hanno grandi occhi, nasi allungati, zigomi sporgenti e capelli castani o rossi, oppure bianchi, neri o grigio-blu. I loro movimenti sono attenti e aggraziati, e chi li guarda potrebbe scambiarli per un ottimo tagliaborse, un danzatore aggraziato o un'abile cortigiana.

\mostro{Manticora}
\noindent
\begin{description}[noitemsep, topsep=0pt, parsep=0pt, partopsep=0pt, leftmargin=0cm, labelwidth=2.2cm]
	\item[\textbf{Taglia/Tipo:}] Grande mostruosità, malvagio
	\item[\textbf{Caratt.:}] \resizebox{0.5\linewidth+1.8cm}{!}{For 3 Des 3 Cos 3 Int -2 Sag 1 Car -1}
	\item[\textbf{Punti Ferita:}] 70,  \textbf{Difesa:} 19,  \textbf{Iniziativa:} +3
	\item[\textbf{Movimento:}] 9 m, volo 15 m
	\item[\textbf{Tiri Salvez.:}] \resizebox{0.5\linewidth+1.8cm}{!}{Tempra +6, Riflessi +6, Volontà +4}
	\item[\textbf{Sensi:}] Scurovisione 18 m
	\item[\textbf{Linguaggi:}] Comune
	\item[\textbf{Sfida:}] 3 (700 PX)\smallskip
\end{description}

\emph{\textbf{Ricrescere Spine della Coda.}} La manticora possiede ventiquattro spine nella coda. Le spine usate ricrescono all'alba.

\textbf{Azioni}

\emph{\textbf{Multiattacco.}} La manticora effettua tre attacchi: uno con il morso e due con gli artigli o tre con le spine della coda.

\emph{\textbf{Artiglio.} Attacco con arma da mischia}: +6 a colpire, portata 1 m, un bersaglio.

\emph{Colpisce:} 6 (1d6 + 3) danni taglienti, 1 danno da Sanguinamento.

\emph{\textbf{Morso.} Attacco con arma da mischia}: +6 a colpire, portata 1 m, un bersaglio.

\emph{Colpisce:} 7 (1d8 + 3) danni perforanti.

\emph{\textbf{Spine della Coda.} Attacco con arma a Distanza}: +6 a colpire, gittata 30m, un bersaglio.

\emph{Colpisce:} 7 (1d8 + 3) danni perforanti.

\textbf{Ecologia}
Ambiente: Colline e Paludi Calde\\
Organizzazione: Solitario, coppia o branco (3-6)\\
\textbf{Categoria Tesoro}: C\\
\textbf{Descrizione}\\
Le manticore sono feroci predatori che controllano vaste aree in cerca di carne fresca. Una tipica manticora è lunga circa 3 metri e pesa circa 500 kg. Alcune hanno la testa simile a quella di un umano, in genere barbuto. Maschi e femmine sono molto simili.

Le manticore mangiano qualsiasi tipo di carne, anche quella delle carogne, ma preferiscono quella umana e raramente si lasciano sfuggire un'occasione di gustare questa delizia. Sono abbastanza furbe e sociali da stringere patti con umanoidi malvagi per formare alleanze o da costringerli ad offre tributi, e molte creature potenti le incaricano di sorvegliare o controllare un posto o una zona. Prediligono fare le loro tane in posti alti, come le sommità delle colline e le caverne tra le rupi.

Anche se le manticore sono simili a delle creazioni magiche, sono da tempo annoverate tra le specie naturali. Curiosamente, le manticore sembrano stranamente feconde e possono incrociarsi con numerose altre specie dalla forma simile, inclusi Leoni, Tigri, Lamie, Sfingi e Chimere.

\mostro{Manto Assassino}
\noindent
\begin{description}[noitemsep, topsep=0pt, parsep=0pt, partopsep=0pt, leftmargin=0cm, labelwidth=2.2cm]
	\item[\textbf{Taglia/Tipo:}] Grande aberrazione, caotico
	\item[\textbf{Caratt.:}] \resizebox{0.5\linewidth+1.8cm}{!}{For 3 Des 2 Cos 1 Int 1 Sag 1 Car 2}
	\item[\textbf{Punti Ferita:}] 160,  \textbf{Difesa:} 24,  \textbf{Iniziativa:} +2
	\item[\textbf{Movimento:}] 3 m, volo 12 m
	\item[\textbf{Tiri Salvez.:}] \resizebox{0.5\linewidth+1.8cm}{!}{\resizebox{0.5\linewidth+1.8cm}{!}{Tempra +9, Riflessi +10, Volontà +9}}
	\item[\textbf{Comp.:}] Furtività +5
	\item[\textbf{Sensi:}] Scurovisione 18 m
	\item[\textbf{Linguaggi:}] Linguaggio delle Profondità
	\item[\textbf{Sfida:}] 8 (3900 PX)\smallskip
\end{description}

\emph{\textbf{Falso Aspetto.}} Mentre il manto assassino resta immobile senza esporre la parte inferiore del corpo, è indistinguibile da un manto in pelle nera.

\emph{\textbf{Sensibilità alla Luce}}. Mentre è alla luce del sole, il manto assassino ha -1d6 ai tiri per colpire, oltre che alle prove di Consapevolezza basate sulla vista.

\emph{\textbf{Trasferimento di Danno.}} Mentre è appiccicato ad una creatura il manto assassino subisce solo la metà dei danni che gli sono inferti (arrotondare per difetto), e la creatura vittima del manto assassino subisce l'altra metà.

\textbf{Azioni}

\emph{\textbf{Multiattacco.}} Il manto assassino effettua due attacchi: uno con il morso e uno con la coda.

\emph{\textbf{Morso.} Attacco con arma da mischia}: +9 a colpire, portata 1 m, una creatura.

\emph{Colpisce:} 10 (2d6 + 3) danni perforanti, e se il bersaglio è di taglia Grande o inferiore, il manto assassino vi si appiccica.
Finché il manto assassino è appiccicato ha un +1d6 ai Tiri per Colpire. Quando effettua un Tiro per Colpire ed ha il bonus di +1d6 e va a segno il bersaglio è accecato e impossibilitato a respirare. Il manto assassino può staccarsi spendendo 1 Azione di Movimento. Una creatura, compreso il bersaglio, può effettuare una Azione per staccare il manto assassino riuscendo una Tiro Salvezza su Tempra con modificatore Forza DC 21.

\emph{\textbf{Coda.} Attacco con arma da mischia}: +9 a colpire, portata 3 m, una creatura.

\emph{Colpisce:} 7 (1d8 + 3) danni taglienti.

\emph{\textbf{Apparizioni (Ricarica dopo un 1 ora).}} Qualora non si trovi sotto luce intensa, il manto assassino crea tre duplicati illusori di sé stesso, che si muovono assieme ad esso e ne imitano le azioni, scambiandosi di posizione per rendere impossibile capire quale sia il reale manto assassino. Se l'originale si trova in un'area di luce intensa, i duplicati svaniscono.

Ogniqualvolta una creatura prenda a bersaglio il manto assassino con un attacco o un incantesimo nocivo mentre sono ancora presenti dei duplicati, quella creatura determina casualmente se prende a bersaglio il manto assassino o uno dei duplicati. Una creatura che non possa vedere o che si affida a sensi diversi dalla vista ignora questo effetto magico.

Un duplicato possiede la Difesa e usa i Tiri Salvezza del manto assassino. Se un attacco colpisce un duplicato, o se un duplicato fallisce un Tiro Salvezza contro un effetto che infligge danni, svanisce.

\emph{\textbf{Gemito.}} Ogni creatura entro 18 metri dal manto assassino, che possa udire il suo gemito e che non sia un'aberrazione, deve riuscire un Tiro Salvezza su Volontà DC 21 o essere spaventata fino al termine del prossimo round del manto assassino. Se il Tiro Salvezza della creatura riesce, la creatura è immune al gemito del manto assassino per le successive 24 ore.

\emph{\textbf{Arrabbiato:}} il Manto assassino ricarica l'abilità Apparizioni. Costa 1 Azione.

\textbf{Ecologia}
Ambiente: Sotterranei\\
Organizzazione: Solitario, coppia, schiera (3-6) o stormo (7-12)\\
\textbf{Categoria Tesoro}: R\\
\textbf{Descrizione}\\
Simili a mante volanti orribilmente malvagie, i manti assassini sono creature misteriose e paranoiche. Un tipico esemplare ha un'apertura alare di 2,3 metri e pesa 50 kg.

Le loro motivazioni sono misteriose e confuse, e diffidano perfino dei loro simili. La strana forma permette loro di essere scambiati per mantelli, arazzi o altri oggetti comuni, e alcune storie narrano di manti assassini che si alleano con altre creature, facendosi trasportare sulla loro schiena e contribuendo alla protezione dei loro alleati per ragioni imperscrutabili.

\mostro{Mantoscuro}
\noindent
\begin{description}[noitemsep, topsep=0pt, parsep=0pt, partopsep=0pt, leftmargin=0cm, labelwidth=2.2cm]
	\item[\textbf{Taglia/Tipo:}] Piccola mostruosità, disallineato
	\item[\textbf{Caratt.:}] \resizebox{0.5\linewidth+1.8cm}{!}{For 3 Des 1 Cos 1 Int -4 Sag 0 Car -3}
	\item[\textbf{Punti Ferita:}] 24,  \textbf{Difesa:} 13,  \textbf{Iniziativa:} +1
	\item[\textbf{Movimento:}] 3 m, volo 9 m
	\item[\textbf{Tiri Salvez.:}] \resizebox{0.5\linewidth+1.8cm}{!}{Tempra +3, Riflessi +3, Volontà +3}
	\item[\textbf{Comp.:}] Furtività +3
	\item[\textbf{Sensi:}] Vista Cieca 18 m
	\item[\textbf{Sfida:}] 1/2 (100 PX)\smallskip
\end{description}

\emph{\textbf{Ecolocazione.}} Il mantoscuro non può usare la vista cieca se assordato.

\emph{\textbf{Falso Aspetto.}} Mentre il mantoscuro rimane immobile, è indistinguibile da una formazione rocciosa come una stalattite o una stalagmite.

\textbf{Azioni}

\emph{\textbf{Spaccare.} Attacco con arma da mischia}: +4 a colpire, portata 1 m, una creatura.

\emph{Colpisce:} 6 (1d6 + 3) danni contundenti e il mantoscuro si appiccica alla creatura. Se il bersaglio è di taglia Media o inferiore il mantoscuro ha +1d6 al Tiro per Colpire, si appiccica avvolgendo la testa del bersaglio, che è accecato e impossibilitato a respirare finché il mantoscuro resta appiccicato in questo modo.

Mentre è appiccicato al bersaglio, il mantoscuro non può attaccare nessun'altra creatura salvo il bersaglio, ma ha +1d6 ai suoi tiri per colpire. La velocità del mantoscuro diventa 0 e non può trarre beneficio da nessun bonus alla velocità, muovendosi assieme al bersaglio.

Una creatura può staccare il mantoscuro con un'Azione e riuscendo un Tiro Salvezza Tempra con Forza DC 13. Durante il suo round, il mantoscuro può staccarsi dal bersaglio da solo usando 1 Azione di Movimento.

\emph{\textbf{Aura di Oscurità (1/Giorno).}} Un'oscurità magica con 5 metri di raggio si estende dal mantoscuro, muovendosi con esso, e propagandosi oltre gli angoli. L'oscurità permane finché il mantoscuro mantiene la concentrazione, massimo 10 minuti (come se si stesse concentrando su di un incantesimo). La Scurovisione non può penetrare questa oscurità, né essa può essere rischiarata da alcuna luce naturale. Se qualsiasi parte dell'oscurità si sovrappone ad un'area di luce generata da un incantesimo di livello 2 o inferiore, l'incantesimo che sta creando la luce viene dissolto.

\textbf{Ecologia}
Ambiente: Qualsiasi (sotterraneo)\\
Organizzazione: Solitario, coppia o nidiata (3-12)\\
\textbf{Categoria Tesoro}: O\\
\textbf{Descrizione}\\
l'apertura tentacolare di un mantoscuro ha un'ampiezza di poco inferiore agli 1 m; quando è appeso alla volta di una caverna, mascherato da stalattite, la sua lunghezza varia tra i 60 ed i 90 cm. Un esemplare tipico di mantoscuro pesa 20 kg. La testa ed il corpo della creatura sono solitamente del colore del basalto o del granito scuro, ma i suoi tentacoli membranosi possono cambiare colore per adattarsi all'ambiente circostante.

I mantoscuro non sono scalatori particolarmente abili, ma sono in grado di appendersi alla volta di una caverna come i pipistrelli, agganciati per mezzo degli uncini posti in fondo ai loro tentacoli, così che il loro corpo penzolante risulti quasi indistinguibile da una stalattite. Da questa postazione nascosta la creatura attende che la preda passi sotto di lei e, a questo punto, si stacca lanciandosi verso di essa, sbattendo contro il bersaglio e tentando di avvolgervi attorno i suoi membranosi tentacoli. Se il mantoscuro manca la preda, risale e si lancia nuovamente contro la preda, fino a quando quest'ultima non viene sconfitta o il mantoscuro è gravemente ferito (nel qual caso svolazza sul soffitto per nascondersi, sperando che la sua preda lo lasci perdere). La capacità innata di questa creatura di celare la zona circostante per mezzo dell'oscurità magica le offre un ulteriore vantaggio contro gli avversari che necessitano della luce per vedere.

I mantoscuro preferiscono vivere e cacciare nelle caverne e nei cunicoli più vicini alla superficie, dal momento che questi offrono un più frequente passaggio di prede che questi mostri possono cacciare. Non si limitano però a queste caverne buie e talvolta possono essere incontrati in fortezze abbandonate o persino nelle fogne delle città affollate. Qualsiasi luogo dove abbondi il cibo e ci sia un soffitto a cui appendersi è un possibile covo per un mantoscuro.

Mantooscuro e Manto Assassino per quanto simili non appartengono alla stessa specie ma leggende narrano di una origine magica comune dovuta, come spesso capita, alla volontà di due maghi di trasformasi per primi in cappe... L'odio tra le due mostruosità è totale e prevarica ogni altro avversario presente.

\mostro{Medusa}
\noindent
\begin{description}[noitemsep, topsep=0pt, parsep=0pt, partopsep=0pt, leftmargin=0cm, labelwidth=2.2cm]
	\item[\textbf{Taglia/Tipo:}] Media mostruosità, malvagio
	\item[\textbf{Caratt.:}] \resizebox{0.5\linewidth+1.8cm}{!}{For 0 Des 2 Cos 3 Int 1 Sag 1 Car 2}
	\item[\textbf{Punti Ferita:}] 126,  \textbf{Difesa:} 22,  \textbf{Iniziativa:} +2
	\item[\textbf{Movimento:}] 9 m
	\item[\textbf{Tiri Salvez.:}] \resizebox{0.5\linewidth+1.8cm}{!}{Tempra +9, Riflessi +8, Volontà +7}
	\item[\textbf{Comp.:}] Furtività +5, Ingannare +5, Percepire Emozioni +4
	\item[\textbf{Sensi:}] Scurovisione 18 m
	\item[\textbf{Linguaggi:}] Comune
	\item[\textbf{Sfida:}] 6 (2300 PX)\smallskip
\end{description}

\emph{\textbf{Sguardo Pietrificante.}} Se una creatura comincia il suo round entro 9 metri da una medusa di cui possa vedere gli occhi, la medusa, qualora non sia inabile e possa vedere a sua volta la creatura, può obbligarla ad effettuare un Tiro Salvezza di Tempra DC 19. Se la creatura fallisce in maniera critica il Tiro Salvezza, viene pietrificata all'istante, altrimenti è Rallentata 1/1 minuto. Successivi sguardi e Tiri Salvezza falliti portano ad aumentare le condizione di Rallentato. Quando la creatura diventa Rallentata 3 si trasforma in pietra. La creatura può tornare di carne se viene lanciato l'incantesimo \hyperlink{Pietra in Carne}{Pietra in Carne} entro 1 mese dalla pietrificazione.

Una creatura che combatte la Medusa cercando di evitare il suo sguardo ha -1d6 al Tiro per Colpire.

Se la medusa vede il suo riflesso su di una superficie riflettente entro 9 metri da lei in un'area di luce intensa, a causa della propria maledizione subirà gli effetti del suo stesso sguardo.

\textbf{Azioni}

\emph{\textbf{Multiattacco.}} La medusa effettua tre attacchi, uno con i capelli serpentini e due con la spada corta oppure due attacchi a distanza con l'arco lungo.

\emph{\textbf{Capelli Serpentini.} Attacco con arma da mischia}: +6 a colpire, portata 1 m, un bersaglio.

\emph{Colpisce:} 4 (1d4 + 2) danni perforanti più 14 (4d6) danni da veleno.

\emph{\textbf{Spada Corta.} Attacco con arma da mischia}: +7 a colpire, portata 1 m, un bersaglio.

\emph{Colpisce:} 5 (1d6 + 2) danni perforanti.

\emph{\textbf{Arco Lungo.} Attacco con arma a Distanza}: +8 a colpire, gittata 45m, un bersaglio.

\emph{Colpisce:} 6 (1d8 + 2) danni perforanti più 7 (2d6) danni da veleno.

\textbf{Reazione: \emph{Attacco d'opportunità}}: la medusa effettua un attacco con i capelli serpentini ad una creatura che attraversi o esca dalla sua portata di 1 metro. Questo attacco non consuma Azioni o Reazioni.

\textbf{Ecologia}\\
Ambiente: Paludi temperate e sotterranei\\
Organizzazione: Solitario\\
\textbf{Categoria Tesoro}: Pugnale, Arco Lungo Perfetto con 20 Frecce, F\\
\textbf{Descrizione}\\
Le meduse sono creature simili agli umani con serpenti al posto dei capelli. Dalla distanza di 9 metri o più, una medusa può passare facilmente per una bella donna se indossa qualcosa che copre la sua chioma serpentina; quando indossa un abbigliamento che ne cela la testa e il volto può essere scambiata per un'umana anche a distanza ravvicinata. Le meduse usano bugie e travestimenti per celare il loro volto fino a che gli avversari non sono abbastanza vicini da usare il loro sguardo pietrificante, anche se gli piace giocare con la loro preda e possono usare delle frecce fiammeggianti per intrappolare i nemici a distanza. Alcune si divertono a creare intricate decorazioni con le loro vittime, usando la pietrificazione per dare un certo tocco ai loro nascondigli paludosi, ma molte meduse hanno cura di nascondere le prove dei loro scontri precedenti così che i loro nuovi nemici non si accorgano della loro pericolosa presenza.

Avvezze a nascondersi, le meduse cittadine generalmente sono ladre, mentre quelle delle zone selvagge spesso finiscono per essere guardiaboschi. Le meduse delle leggende più note, tuttavia, sono quelle che prendono livelli da incantatore. Carismatiche ed intelligenti, le meduse urbane sono spesso coinvolte in gilde di ladri ed altri aspetti del mondo criminale. Le meduse possono formare alleanze con creature cieche o non morti intelligenti, entrambi immuni al loro sguardo pietrificante. Le meduse incantatrici fungono spesso da oracoli o profetesse, vivendo generalmente in remote zone di leggendaria potenza o dalla storia infausta. Queste meduse oracoli traggono grande diletto dal loro ruolo, e se ci si presenta con i giusti doni e adulazioni, i segreti che offrono possono essere veramente utili. Naturalmente, i nascondigli di queste potenti creature sono decorati con le statue di coloro che le hanno offese, come monito ad usare le dovute cautele durante gli incontri.

Tutte le meduse sono femmine. Raramente, una medusa decide di prendere un maschio umanoide come compagno, generalmente grazie all'aiuto di una Elisir d'Amore o qualche magia simile, ed hanno sempre cura di non pietrificare il loro prigioniero, a meno che non si siano annoiate della sua compagnia.

\mostro{Mefito di Ghiaccio}
\noindent
\begin{description}[noitemsep, topsep=0pt, parsep=0pt, partopsep=0pt, leftmargin=0cm, labelwidth=2.2cm]
	\item[\textbf{Taglia/Tipo:}] Piccola elementale, malvagio
	\item[\textbf{Caratt.:}] \resizebox{0.5\linewidth+1.8cm}{!}{For -2 Des 1 Cos 0 Int -1 Sag 0 Car 1}
	\item[\textbf{Punti Ferita:}] 24,  \textbf{Difesa:} 13,  \textbf{Iniziativa:} +1
	\item[\textbf{Movimento:}] 9 m, volo 9 m
	\item[\textbf{Tiri Salvez.:}] \resizebox{0.5\linewidth+1.8cm}{!}{Tempra +3, Riflessi +3, Volontà +3}
	\item[\textbf{Comp.:}] Furtività +3, Consapevolezza +2
	\item[\textbf{Imm. Danni:}] Veleno
	\item[\textbf{Sensi:}] Scurovisione 18 m
	\item[\textbf{Linguaggi:}] Aquan, Ictun
	\item[\textbf{Sfida:}] 1/2 (100 PX)\smallskip
\end{description}

\emph{\textbf{Falso Aspetto.}} Mentre il mefito rimane immobile, è indistinguibile da un ordinario frammento di ghiaccio.

\emph{\textbf{Incantesimi Innati (1/Giorno).}} Il mefito può lanciare in maniera innata \emph{\hyperlink{Nube di Nebbia}{Nube di Nebbia}}, senza bisogno di componenti materiali. La sua caratteristica da incantatore innato è il Carisma.

\emph{\textbf{Natura Elementale.}} Un mefito non ha bisogno di cibo, bevande o sonno.

\emph{\textbf{Scoppio Mortale.}} Quando il mefito muore, esplode in uno scoppio di frammenti di ghiaccio. Ogni creatura entro 1 metro da esso deve effettuare un Tiro Salvezza di Riflessi DC 11 o subire 4 (1d8) danni taglienti in caso di fallimento, o la metà di questi danni in caso
di successo.

\textbf{Azioni}

\emph{\textbf{Artigli.} Attacco con arma da mischia}: +4 a colpire, portata 1 m, una creatura.

\emph{Colpisce:} 3 (1d4 + 1) danni taglienti più 2 (1d4) danni da freddo.

\emph{\textbf{Soffio Gelido (Ricarica 6).}} Il mefito esala un cono di 5 metri di aria fredda. Ogni creatura nell'area deve effettuare un Tiro Salvezza di Riflessi DC 11, subendo 5 (2d4) danni da freddo in caso di fallimento, o la metà di questi danni in caso di successo.

\textbf{Ecologia}\\
Ambiente: Qualsiasi (piano elementale dell'aria)\\
Organizzazione: Solitario, coppia, gruppo (3-6) o stormo (7-12)\\
\textbf{Categoria Tesoro}: J\\
\textbf{Descrizione}\\
I mephit sono i servitori di potenti creature elementali. I siti e le locazioni chiave dei piani elementali sono pieni di mephit che si affannano per svolgere un importante dovere o incarico.

I mephit del ghiaccio comunemente si trovano sul Piano dell'Aria. Questi mephit sono distanti e crudeli.

\mostro{Mefito di Magma}
\noindent
\begin{description}[noitemsep, topsep=0pt, parsep=0pt, partopsep=0pt, leftmargin=0cm, labelwidth=2.2cm]
	\item[\textbf{Taglia/Tipo:}] Piccola elementale, malvagio
	\item[\textbf{Caratt.:}] \resizebox{0.5\linewidth+1.8cm}{!}{For -1 Des 1 Cos 1 Int -2 Sag 0 Car 0}
	\item[\textbf{Punti Ferita:}] 24,  \textbf{Difesa:} 13,  \textbf{Iniziativa:} +1
	\item[\textbf{Movimento:}] 9 m, volo 9 m
	\item[\textbf{Tiri Salvez.:}] \resizebox{0.5\linewidth+1.8cm}{!}{Tempra +3, Riflessi +3, Volontà +3}
	\item[\textbf{Comp.:}] Furtività +3
	\item[\textbf{Imm. Danni:}] Veleno
	\item[\textbf{Sensi:}] Scurovisione 18 m
	\item[\textbf{Linguaggi:}] Ignan, Tremun
	\item[\textbf{Sfida:}] 1/2 (100 PX)\smallskip
\end{description}

\emph{\textbf{Falso Aspetto.}} Mentre il mefito rimane immobile, è indistinguibile da un'ordinaria pozza di magma.

\emph{\textbf{Incantesimi Innati (1/Giorno).}} Il mefito può lanciare in maniera innata \emph{riscaldare metallo} (DC del Tiro Salvezza dell'incantesimo 10), senza bisogno di componenti materiali. La sua caratteristica da incantatore innato è il Carisma.

\emph{\textbf{Natura Elementale.}} Un mefito non ha bisogno di cibo, bevande o sonno.

\emph{\textbf{Scoppio Mortale.}} Quando il mefito muore, esplode in uno scoppio di lava. Ogni creatura entro 1 metro da esso deve effettuare un Tiro Salvezza di Riflessi DC 11 o subire 7 (2d6) danni da fuoco in caso di fallimento, o la metà di questi danni in caso di successo.

\textbf{Azioni}

\emph{\textbf{Artigli.} Attacco con arma da mischia}: +4 a colpire, portata 1 m, una creatura.

\emph{Colpisce:} 3 (1d4 + 1) danni taglienti più 2 (1d4) danni da fuoco.

\emph{\textbf{Soffio Infuocato (Ricarica 6).}} Il mefito esala un cono di 5 metri di fuoco. Ogni creatura nell'area deve effettuare un Tiro Salvezza su Riflessi DC 11, subendo 7 (2d6) danni da fuoco in caso di fallimento, o la metà di questi danni in caso di successo.

\textbf{Ecologia}\\
Ambiente: Qualsiasi (piano elementale del fuoco)\\
Organizzazione: Solitario, coppia, gruppo (3-6) o stormo (7-12)\\
\textbf{Categoria Tesoro}: J\\
\textbf{Descrizione}\\
I mephit sono i servitori di potenti creature elementali. I siti e le locazioni chiave dei piani elementali sono pieni di mephit che si affannano per svolgere un importante dovere o incarico.

I mephit del magma comunemente si trovano sul Piano del Fuoco. Questi mephit sono stupidi bruti.

\mostro{Mefito di Polvere}
\noindent
\begin{description}[noitemsep, topsep=0pt, parsep=0pt, partopsep=0pt, leftmargin=0cm, labelwidth=2.2cm]
	\item[\textbf{Taglia/Tipo:}] Piccola elementale, malvagio
	\item[\textbf{Caratt.:}] \resizebox{0.5\linewidth+1.8cm}{!}{For -3 Des 2 Cos 0 Int -1 Sag 0 Car 0}
	\item[\textbf{Punti Ferita:}] 24,  \textbf{Difesa:} 14,  \textbf{Iniziativa:} +2
	\item[\textbf{Movimento:}] 9 m, volo 9 m
	\item[\textbf{Tiri Salvez.:}] \resizebox{0.5\linewidth+1.8cm}{!}{Tempra +3, Riflessi +3, Volontà +3}
	\item[\textbf{Comp.:}] Furtività +4, Consapevolezza +2
	\item[\textbf{Imm. Danni:}] Veleno
	\item[\textbf{Sensi:}] Scurovisione 18 m
	\item[\textbf{Linguaggi:}] Ictun, Tremun
	\item[\textbf{Sfida:}] 1/2 (100 PX)\smallskip
\end{description}

\emph{\textbf{Incantesimi Innati (1/Giorno).}} Il mefito può eseguire in maniera innata \emph{sonno} (DC del Tiro Salvezza dell'incantesimo 11), senza bisogno di componenti materiali. La sua abilità da incantatore innato è il Carisma.

\emph{\textbf{Natura Elementale.}} Un mefito non ha bisogno di cibo, bevande o sonno.

\emph{\textbf{Scoppio Mortale.}} Quando il mefito muore, esplode in uno scoppio di polvere. Ogni creatura entro 1 metro da esso deve riuscire un Tiro Salvezza di Tempra DC 11 o restare accecata per 1 minuto. Una creatura accecata può ripetere il Tiro Salvezza durante ciascun suo round, terminando l'effetto su di sé in caso di successo.

\textbf{Azioni}

\emph{\textbf{Artigli.} Attacco con arma da mischia}: +4 a colpire, portata 1 m, una creatura.

\emph{Colpisce:} 4 (1d4 + 2) danni taglienti.

\emph{\textbf{Soffio Accecante (Ricarica 6).}} Il mefito esala un cono di 5 metri di polvere accecante. Ogni creatura nell'area deve riuscire un Tiro Salvezza di Riflessi DC 11 o restare accecata per 1 minuto. Una creatura accecata può ripetere il Tiro Salvezza durante ciascun suo round, terminando l'effetto su di sé in caso di successo.

\textbf{Ecologia}\\
Ambiente: Qualsiasi (piano elementale dell'aria)\\
Organizzazione: Solitario, coppia, gruppo (3-6) o stormo (7-12)\\
\textbf{Categoria Tesoro}: J\\
\textbf{Descrizione}\\
I mephit sono i servitori di potenti creature elementali. I siti e le locazioni chiave dei piani elementali sono pieni di mephit che si affannano per svolgere un importante dovere o incarico.

I mephit della polvere comunemente si trovano sul Piano dell'Aria. Questi mephit sono irritanti ed insistenti.

\mostro{Mefito di Vapore}
\noindent
\begin{description}[noitemsep, topsep=0pt, parsep=0pt, partopsep=0pt, leftmargin=0cm, labelwidth=2.2cm]
	\item[\textbf{Taglia/Tipo:}] Piccola elementale, malvagio
	\item[\textbf{Caratt.:}] \resizebox{0.5\linewidth+1.8cm}{!}{For -3 Des 0 Cos 0 Int 0 Sag 0 Car 1}
	\item[\textbf{Punti Ferita:}] 19,  \textbf{Difesa:} 12,  \textbf{Iniziativa:} +0
	\item[\textbf{Movimento:}] 9 m, volo 9 m
	\item[\textbf{Tiri Salvez.:}] \resizebox{0.5\linewidth+1.8cm}{!}{Tempra +3, Riflessi +3, Volontà +3}
	\item[\textbf{Imm. Danni:}] Veleno
	\item[\textbf{Sensi:}] Scurovisione 18 m
	\item[\textbf{Linguaggi:}] Aquan, Ignan
	\item[\textbf{Sfida:}] 1/4 (50 PX)\smallskip
\end{description}

\emph{\textbf{Incantesimi Innati (1/Giorno).}} Il mefito può eseguire in maniera innata \emph{\hyperlink{Sfocatura}{Sfocatura}}, senza bisogno di componenti materiali. La sua abilità da incantatore innato è il Carisma.

\emph{\textbf{Natura Elementale.}} Un mefito non ha bisogno di cibo, bevande o sonno.

\emph{\textbf{Scoppio Mortale.}} Quando il mefito muore, esplode in nube di vapore. Ogni creatura entro 1 metro da esso deve riuscire un Tiro Salvezza su Riflessi DC 10 o subire 4 (1d8) danni da fuoco.

\textbf{Azioni}

\emph{\textbf{Artigli.} Attacco con arma da mischia}: +4 a colpire, portata 1 m, una creatura.

\emph{Colpisce:} 2 (1d4) danni taglienti più 2 (1d4) danni da fuoco.

\emph{\textbf{Soffio Vaporoso (Ricarica 6).}} Il mefito esala un cono di 5 metri di vapore caldo. Ogni creatura nell'area deve effettuare un Tiro Salvezza di Riflessi DC 10, subendo 4 (1d8) danni da fuoco in caso di fallimento, o la metà di questi danni in caso di successo.

\textbf{Ecologia}\\
Ambiente: Qualsiasi (piano elementale del fuoco)\\
Organizzazione: Solitario, coppia, gruppo (3-6) o stormo (7-12)\\
\textbf{Categoria Tesoro}: J\\
\textbf{Descrizione}\\
I mephit sono i servitori di potenti creature elementali. I siti e le locazioni chiave dei piani elementali sono pieni di mephit che si affannano per svolgere un importante dovere o incarico.

I mephit del vapore comunemente si trovano sul Piano del Fuoco. Questi mephit sono insolenti e sprezzanti.

\mostro{Megera Marina}
\noindent
\begin{description}[noitemsep, topsep=0pt, parsep=0pt, partopsep=0pt, leftmargin=0cm, labelwidth=2.2cm]
	\item[\textbf{Taglia/Tipo:}] Media fatato, malvagio
	\item[\textbf{Caratt.:}] \resizebox{0.5\linewidth+1.8cm}{!}{For 3 Des 1 Cos 3 Int 1 Sag 1 Car 1}
	\item[\textbf{Punti Ferita:}] 52,  \textbf{Difesa:} 15,  \textbf{Iniziativa:} +1
	\item[\textbf{Movimento:}] 9 m, nuoto 12 m
	\item[\textbf{Tiri Salvez.:}] \resizebox{0.5\linewidth+1.8cm}{!}{Tempra +5, Riflessi +3, Volontà +3}
	\item[\textbf{Sensi:}] Scurovisione 18 m
	\item[\textbf{Linguaggi:}] Aquan, Comune, Gigante
	\item[\textbf{Sfida:}] 2 (450 PX)\smallskip
\end{description}

\emph{\textbf{Anfibio.}} La megera può respirare aria e acqua.

\emph{\textbf{Aspetto Orripilante.}} Qualsiasi umanoide che inizi il suo round entro 9 metri dalla megera e ne può vedere la vera forma deve effettuare un Tiro Salvezza di Volontà DC 13. Se fallisce il Tiro Salvezza, la creatura resta spaventata per 1 minuto. Una creatura può ripetere il Tiro Salvezza al termine di ciascun suo round, con -1d6 se la megera è in linea di visuale, e terminando l'effetto se riesce il Tiro Salvezza. Se il Tiro Salvezza della creatura riesce o l'effetto ha termine su di essa, la creatura è immune all'Aspetto Orripilante per le successive 24 ore.

A meno che il bersaglio non sia sorpreso o la rivelazione della vera forma della megera non sia improvvisa, il bersaglio può distogliere lo sguardo e evitare di effettuare il Tiro Salvezza iniziale. Fino all'inizio del suo prossimo round, una creatura che distolga lo sguardo ha -1d6 ai tiri di attacco contro la megera.

\textbf{Azioni}

\emph{\textbf{Artigli.} Attacco in mischia con arma}: +5 a colpire, portata 1 m, un bersaglio.

\emph{Colpisce:} 10 (2d6 + 3) danni taglienti, 1 danno da Sanguinamento.

\emph{\textbf{Aspetto Illusorio.}} La megera ricopre se stessa e tutto quello che sta indossando o trasportando in un'illusione magica che le dona l'aspetto di una creatura ripugnante all'incirca della stessa taglia e forma umanoide. L'illusione termina se la megera effettua una Reazione per terminarla o se muore.

I cambiamenti apportati da questo effetto non sono in grado di superare le ispezioni fisiche. Ad esempio, la megera potrebbe apparire come una creatura priva di artigli, ma una persona in contatto con le sue mani li avvertirebbe. Altrimenti, una creatura deve effettuare un'Azione per ispezionare visivamente l'illusione e riuscire una prova di Consapevolezza DC 16 per comprendere che la megera si è camuffata.

\emph{\textbf{Occhiata Mortale.}} La megera prende a bersaglio una creatura spaventata visibile entro 9 metri da lei. Se il bersaglio può vedere la megera, deve riuscire un Tiro Salvezza di Volontà DC 13 contro questa magia o scendere a 0 Punti Ferita.

\textbf{Ecologia}\\
Ambiente: qualsiasi acquatico\\
Organizzazione: solitario o congrega (3 megere di qualsiasi specie)\\
\textbf{Categoria Tesoro}: R (C)\\
\textbf{Descrizione}\\
Queste perfide e mostruose megere possiedono dei tratti terrificanti che pochi osano fissare, traggono piacere dalla discordia e dalla morte dei marinai, e tormentano la gente di mare con ineluttabili sciagure. Le megere marine hanno sempre un aspetto terribile e, malgrado la loro natura famelica, in genere sono creature emaciate che sembrano sul punto di morir di fame. Sono alte 1,8 metri e pesano 75 kg.

Le megere marine preferiscono vivere vicino alla riva dove i pescherecci e i mercantili sono più comuni, e comunque lontano dalle aree urbane di modo che le loro azioni non attraggano troppo l'attenzione di possibili nemici, anche se non è insolito che una megera marina coraggiosa o avida si stabilisca in una città portuale o alla foce di un fiume profondo.

Le megere marine formano congreghe simili a quelle delle altre megere, ma la loro natura acquatica generalmente le spinge ad astenersi dal formare congreghe miste. Nel caso in cui una Megera Verde abiti lungo la costa (spesso in una palude salmastra o in una palude costiera), una congrega è formata da due megere marine che rispettano la Megera Verde come madre e capo. Molto comunemente, una congrega di megere marine consiste in un gruppo di megere marine particolarmente amiche e vicine.

\mostro{Megera Notturna}
\noindent
\begin{description}[noitemsep, topsep=0pt, parsep=0pt, partopsep=0pt, leftmargin=0cm, labelwidth=2.2cm]
	\item[\textbf{Taglia/Tipo:}] Media immondo, malvagio
	\item[\textbf{Caratt.:}] \resizebox{0.5\linewidth+1.8cm}{!}{For 4 Des 2 Cos 3 Int 3 Sag 2 Car 3}
	\item[\textbf{Punti Ferita:}] 108,  \textbf{Difesa:} 20,  \textbf{Iniziativa:} +3
	\item[\textbf{Movimento:}] 9 m
	\item[\textbf{Tiri Salvez.:}] \resizebox{0.5\linewidth+1.8cm}{!}{Tempra +8, Riflessi +7, Volontà +7}
	\item[\textbf{Comp.:}] Furtività +6, Ingannare +7, Percepire Emozioni +6
	\item[\textbf{Res. Danni:}] Freddo, Fuoco; da arma non magica o non siano argentati
	\item[\textbf{Sensi:}] Scurovisione 36 m
	\item[\textbf{Linguaggi:}] Abissale, Comune, Infernale, Druidico
	\item[\textbf{Sfida:}] 5 (1800 PX)\smallskip
\end{description}

\emph{\textbf{Incantesimi Innati.}} La caratteristica da incantatore innato della megera è il Carisma (DC 14 per i Tiri Salvezza degli incantesimi. La megera può lanciare in maniera innata i seguenti incantesimi, senza aver bisogno di componenti materiali.

A volontà: \emph{\hyperlink{Dardo arcano}{Dardo arcano}, \hyperlink{Individuazione del Magico}{Individuazione del Magico}} 2/giorno ciascuno: \emph{\hyperlink{Raggio di Indebolimento}{Raggio di Indebolimento}, \hyperlink{Sonno}{Sonno}}

\emph{\textbf{Resistenza alla Magia.}} La megera ha +1d6 ai tiri salvezza contro incantesimi e altri effetti magici.

\textbf{Azioni}

\emph{\textbf{Artigli (Solo in Forma di Megera).} Attacco con arma da mischia}: +7 a colpire, portata 1 m, un bersaglio.

\emph{Colpisce:} 13 (2d8 + 4) danni taglienti, 1 danno da Sanguinamento.

\emph{\textbf{Forma Eterea.}} La megera entra magicamente nel Piano Etereo dal Piano Materiale, e viceversa.

\textbf{Reazione: \emph{Attacco d'opportunità}}: la megera effettua un attacco ad una creatura che attraversi o esca dalla sua portata di 1 metro.

\emph{\textbf{Infestare Incubi (1/Giorno).}} Mentre si trova sul Piano Etereo, la megera entra magicamente in contatto con un umanoide addormentato che si trova sul Piano Materiale. L'incantesimo \emph{\hyperlink{Cerchio Magico}{Cerchio Magico}} lanciato sul bersaglio previene questo contatto. Finché il contatto persiste, il bersaglio soffre di orribili visioni. Se queste visioni durano per almeno 1 ora, il bersaglio non ottiene benefici dal suo riposo e i suoi Punti Ferita massimi sono ridotti di 5 (1d10). Se questo effetto riduce i Punti Ferita massimi del bersaglio a 0, il bersaglio muore, e se il bersaglio era malvagio, la sua anima resta intrappolata nella \emph{borsa delle anime} della megera. La riduzione dei Punti Ferita massimi del bersaglio rimane finché non viene rimossa dall'incantesimo \emph{\hyperlink{Ristorare Superiore}{Ristorare Superiore}} o simile magia.

\emph{\textbf{Mutare Forma.}} La megera può trasformarsi magicamente in una femmina umanoide di taglia Piccola o Media, o tornare alla sua vera forma. Le sue statistiche sono le stesse in qualsiasi forma. Tutto l'equipaggiamento che stava trasportando o indossando non viene trasformato. Alla morte ritorna alla sua vera forma.

\mostro{Megera Verde}
\noindent
\begin{description}[noitemsep, topsep=0pt, parsep=0pt, partopsep=0pt, leftmargin=0cm, labelwidth=2.2cm]
	\item[\textbf{Taglia/Tipo:}] Media fatato, malvagio
	\item[\textbf{Caratt.:}] \resizebox{0.5\linewidth+1.8cm}{!}{For 4 Des 1 Cos 3 Int 1 Sag 2 Car 2}
	\item[\textbf{Punti Ferita:}] 70,  \textbf{Difesa:} 17,  \textbf{Iniziativa:} +1
	\item[\textbf{Movimento:}] 9 m
	\item[\textbf{Tiri Salvez.:}] \resizebox{0.5\linewidth+1.8cm}{!}{Tempra +6, Riflessi +4, Volontà +5}
	\item[\textbf{Comp.:}] Arcana +3, Furtività +3, Ingannare +4
	\item[\textbf{Sensi:}] Scurovisione 18 m
	\item[\textbf{Linguaggi:}] Comune, Draconico, Silvano
	\item[\textbf{Sfida:}] 3 (700 PX)\smallskip
\end{description}

\emph{\textbf{Anfibio.}} La megera può respirare aria e acqua.

\emph{\textbf{Imitazione.}} La megera può imitare suoni animali e voci umanoidi. Una creatura che senta questi rumori può determinare che si tratti di un'imitazione riuscendo una prova di Consapevolezza DC 14.

\emph{\textbf{Incantesimi Innati.}} La caratteristica da incantatore innato della megera è il Carisma (DC 13 per i Tiri Salvezza degli incantesimi). La megera può lanciare in maniera innata i seguenti incantesimi, senza aver bisogno di componenti materiali.

A volontà: \emph{\hyperlink{Illusione Minore}{Illusione Minore}, \hyperlink{Luci Danzanti}{Luci Danzanti}, \hyperlink{Beffa Crudele}{Beffa Crudele}}

\textbf{Azioni}

\emph{\textbf{Artigli.} Attacco con arma da mischia}: +6 a colpire, portata 1 m, un bersaglio.

\emph{Colpisce:} 13 (2d8 + 4) danni taglienti, 1 danno da Sanguinamento.

\emph{\textbf{Aspetto Illusorio.}} La megera ricopre sé stessa e tutto quello che sta indossando o trasportando in un'illusione magica che le dona l'aspetto di un'altra creatura all'incirca della stessa taglia e forma umanoide. L'illusione termina se la megera effettua una Reazione per terminarla o se muore.

I cambiamenti apportati da questo effetto non sono in grado di superare le ispezioni fisiche. Ad esempio, la megera potrebbe apparire come una creatura dalla pelle liscia, ma il contatto rivelerebbe la sua pelle ruvida. Altrimenti, una creatura deve effettuare un'Azione per ispezionare visivamente l'illusione e riuscire una prova di Consapevolezza DC 20 per comprendere che si tratta di una megera camuffata.

\emph{\textbf{Passaggio Invisibile.}} La megera può rendersi invisibile finché non attacca o lancia un incantesimo, o finché non termina la concentrazione (come se si stesse concentrando su di un incantesimo). Mentre è invisibile, non lascia traccia fisica del suo passaggio, quindi le sue tracce possono essere seguite solo dalla magia. Tutto l'equipaggiamento che sta trasportando o indossando diventa invisibile assieme a lei.

\textbf{Ecologia}
Ambiente: Paludi temperate\\
Organizzazione: Solitario o congrega (3 megere di qualsiasi tipo)\\
\textbf{Categoria Tesoro}: R (C)\\
\textbf{Descrizione}\\
Terrificanti vecchie rugose che frequentano ripugnanti paludi e foreste intricate, le megere verdi nutrono un odio intenso per tutto ciò che è bello e puro. Facendo uso delle loro svariate capacità illusorie, queste vegliarde si dilettano nell'uccidere gli innocenti, nello sconvolgere gli animi nobili e nell'avvilire i cuori puri. Amano utilizzare Camuffare Se Stesso per assumere le forme di giovani e attraenti ragazze così da sedurre e strappare giovani uomini ai loro affetti e parenti, e corrompere nobili e onesti cittadini con ogni sorta di depravazione e scandalo. Alcune megere verdi preferiscono rivelare la loro reale natura ai loro amati in un momento attentamente architettato per spingere l'uomo alla pazzia per l'orrore e la vergogna. Altre prolungano il loro amoreggiamento e fanno di tutto per rovinare completamente la vita degli uomini da loro sedotti prima di mostrare loro la verità. Infine, i più fortunati di questi sventurati finiscono per essere divorati dalla megera verde loro amante: per gli sfortunati, il destino finale può essere molto peggiore, dato che la crudele fantasia della megera verde è immensa. Una tipica megera verde è alta tra 1,5 e 1,8 metri e pesa poco meno di 80 kg.

\mostro{Ameba Paglierina}
\noindent
\begin{description}[noitemsep, topsep=0pt, parsep=0pt, partopsep=0pt, leftmargin=0cm, labelwidth=2.2cm]
	\item[\textbf{Taglia/Tipo:}] Grande melma, disallineato
	\item[\textbf{Caratt.:}] \resizebox{0.5\linewidth+1.8cm}{!}{For 2 Des -2 Cos 2 Int -4 Sag -2 Car -5}
	\item[\textbf{Punti Ferita:}] 51,  \textbf{Difesa:} 12,  \textbf{Iniziativa:} -2
	\item[\textbf{Movimento:}] 3 m, scalata 3 m
	\item[\textbf{Tiri Salvez.:}] \resizebox{0.5\linewidth+1.8cm}{!}{Tempra +4, Riflessi +3, Volontà +3}
	\item[\textbf{Res. Danni:}] Acido
	\item[\textbf{Imm. Danni:}] Elettricità, tagliente
	\item[\textbf{Immunità:}] accecato, affascinato, assordato, prono, affaticato, spaventato
	\item[\textbf{Sensi:}] Vista Cieca 18 m (cieca oltre questo raggio)
	\item[\textbf{Sfida:}] 2 (450 PX)\smallskip
\end{description}

\emph{\textbf{Amorfo.}} L'ameba può muoversi attraverso uno spazio fino a 3 centimetri di larghezza senza doversi stringere.

\emph{\textbf{Natura di Melma.}} L'ameba non necessita di dormire.

\emph{\textbf{Scalare come Ragno.}} L'ameba può scalare superfici difficili, compreso lo stare a testa in giù sul soffitto, senza bisogno di effettuare una prova di Competenza.

\textbf{Azioni}

\emph{\textbf{Pseudopodo.} Attacco con arma da mischia}: +5 a colpire, portata 1 m, un bersaglio.

\emph{Colpisce:} 9 (2d6 + 2) danni contundenti più 3 (1d6) danni da acido.

\textbf{Reazioni}

\emph{\textbf{Divisione.}} Quando un'ameba Media o più grande subisce danni da elettricità o taglienti, si divide in due nuove amebe che hanno almeno 10 Punti Ferita. Ogni nuova ameba ha un numero di Punti Ferita pari alla metà dell'ameba originale, arrotondati per difetto. Le nuove amebe sono di una taglia più piccola di quella originale.


\begin{center}
	\includegraphics[width=0.43\textwidth]{immagini/Amoeba_proteus.png}
\end{center}

\textbf{Ecologia}
Ambiente: Sotterranei o Paludi Temperati\\
Organizzazione: Solitario\\
\textbf{Categoria Tesoro}: Nessuno\\
\textbf{Descrizione}\\
Le Ameba Paglierina sono masse animate di protoplasma di colore simile ad un repellente amalgama di giallo, arancio e marrone. Quando a riposo, il loro corpo piatto e pulsante è alto circa 15 centimetri e si estende tutto intorno; in movimento, si raccolgono in una forma vagamente sferica e sembrano quasi spostarsi rotolando. I loro corpi malleabili permettono loro di attraversare fessure e buchi molto più piccoli dello spazio che occupano. Le creature che vivono sottoterra spesso sigillano tutte le aperture per difendersi dalle Ameba Paglierina.

L'acido altamente specializzato dell'Ameba Paglierina dissolve solo la carne. Questa scoperta ha portato molti maestri avvelenatori ed alchimisti a cercarne esemplari per studiarli. Da questi esperimenti sono nate diverse armi specifiche ideate per distruggere i corpi. Si racconta dell'esistenza di un veleno ad azione lenta che distrugge ad una ad una le cellule delle creature viventi, il cui segreto è ben conservato dal suo creatore.

Un'antica e dimenticata raccolta di appunti descrive un singolare rituale funebre praticato in terre lontane. Anziché cremare i defunti, i corpi venivano racchiusi in sarcofagi di pietra insieme a un'Ameba Paglierina che ne dissolveva lentamente la carne. Successivamente la gelatina risultante veniva trasferita in un'urna accompagnata da una targa di bronzo recante il nome del defunto. Questo metodo preservava gli oggetti sepolti con il corpo, ridotto in breve tempo a uno scheletro lucente, e si credeva che l'essenza vitale del defunto continuasse ad abitare nella gelatina.

L'Ameba Paglierina sono alte circa 15 centimetri, hanno un diametro che può arrivare a 3 metri e pesano circa 1.300 chili. In combattimento, si raccolgono su loro stesse e producono lunghi pseudopodi umidi per colpire ed afferrare qualunque cosa si muova.

Anche se la tipica Ameba Paglierina ha le statistiche qui presentate, nelle profondità della terra questi predatori possono raggiungere dimensioni mostruose.


\mostro{Cubo Gelatinoso}
\noindent
\begin{description}[noitemsep, topsep=0pt, parsep=0pt, partopsep=0pt, leftmargin=0cm, labelwidth=2.2cm]
	\item[\textbf{Taglia/Tipo:}] Grande melma, disallineato
	\item[\textbf{Caratt.:}] \resizebox{0.5\linewidth+1.8cm}{!}{For 2 Des -4 Cos 5 Int -5 Sag -2 Car -5}
	\item[\textbf{Punti Ferita:}] 53,  \textbf{Difesa:} 10,  \textbf{Iniziativa:} -4
	\item[\textbf{Movimento:}] 5 metri
	\item[\textbf{Tiri Salvez.:}] \resizebox{0.5\linewidth+1.8cm}{!}{Tempra +7, Riflessi +3, Volontà +3}
	\item[\textbf{Imm. Danni:}] armi da taglio non magiche, da danno
	\item[\textbf{Immunità:}] accecato, affascinato, assordato, prono, affaticato, spaventato
	\item[\textbf{Sensi:}] Vista Cieca 18 m (cieco oltre questo raggio)
	\item[\textbf{Sfida:}] 2 (450 PX)\smallskip
\end{description}

\emph{\textbf{Cubo di Melma.}} Il cubo occupa il suo intero spazio. Le altre creature possono entrare nello spazio, ma rimangono vittima del Sommergere del cubo e hanno -1d6 al Tiro Salvezza.

Le creature all'interno del cubo sono visibili ma godono di copertura completa.

Una creatura entro 1 metro dal cubo può effettuare un'Azione per tirare una creatura od oggetto fuori dal cubo. Farlo richiede la riuscita di una prova di Atletica DC 14 e la creatura che effettua il tentativo subisce 10 (3d6) danni da acido.

Il cubo può contenere solo una creatura Grande o un massimo di quattro creature Medie o più piccole alla volta.

\emph{\textbf{Natura di Melma.}} Il cubo non necessita di dormire.

\emph{\textbf{Trasparente.}} Anche quando è in piena vista, è necessario riuscire una prova di Consapevolezza DC 15 per notare un cubo che non si è mosso o non ha attaccato. Una creatura che cerchi di entrare nello spazio del cubo mentre è inconsapevole della sua presenza resta sorpresa dal cubo.

\textbf{Azioni}

\emph{\textbf{Pseudopodo.} Attacco con arma da mischia}: +5 a colpire, portata 1 m, un bersaglio.

\emph{Colpisce:} 10 (3d6) danni da acido.

\emph{\textbf{Sommergere.}} Il cubo si muove fino al massimo del suo movimento. Nel farlo, può entrare nello spazio di una creatura di taglia Grande o più piccola. Ogni volta che il cubo entra nello spazio di una creatura, la creatura deve effettuare un Tiro Salvezza di Riflessi DC 13.

Se il Tiro Salvezza riesce, la creatura può scegliere di essere spinta indietro o di lato di 1 metro. Una creatura che decida di non farsi spingere subisce le conseguenze di un Tiro Salvezza fallito.

Se il Tiro Salvezza fallisce, il cubo entra nello spazio della creatura, che subisce 10 (3d6) danni da acido ed è sommersa. La creatura sommersa non può respirare, è intralciata e subisce 21 (6d6) danni da acido all'inizio del round del cubo. Quando il cubo si muove, la creatura sommersa si muove con esso.

Una creatura sommersa può tentare di fuggire effettuando un'Azione per compiere un Tiro Salvezza Tempra con Forza DC 14. Se la riesce, la creatura sfugge e esce entro uno spazio di sua scelta entro 1 metro dal cubo.

\textbf{Ecologia}
Ambiente: Qualsiasi sotterraneo\\
Organizzazione: Solitario\\
\textbf{Categoria Tesoro}: Accidentale\\
\textbf{Descrizione}\\
Tra i predatori più insoliti e peculiari dei dungeon, i cubi gelatinosi trascorrono la loro esistenza vagabondando senza meta per i cunicoli sotterranei e le oscure caverne, inglobando materiali organici come piante, rifiuti, carogne e anche creature viventi. La materia che il cubo non può digerire, come metalli e pietra, riempie di detriti il volume della creatura, e a volte questa può espellerne una parte dal suo corpo. Spesso il tesoro e gli averi delle vittime passate restano dentro il cubo gelatinoso: immagine spettrale dei loro resti materiali.

Alcuni saggi credono che queste creature si siano evolute delle Melme Grigie. Alcuni esseri usano i cubi gelatinosi come guardiani di dungeon e fortificazioni sotterranee, intrappolando queste immense creature in casse di metallo massiccio e trasportandole con poteri o magie fino al loro posto di guardia finale. Sono dei meccanismi di smaltimento rifiuti particolarmente efficaci; una tribù può intrappolare un cubo gelatinoso in una fossa o un'altra area che non possa scalare usandolo come letamaio o anche trappola mortale, a seconda dell'ingegnosità delle creature che l'hanno catturato.

I cubi gelatinosi in genere hanno uno spigolo di 3 metri e pesano più di 7.500 kg, sebbene alcuni esploratori sotterranei affermino che nel sottosuolo esistano esemplari più grandi. In zone in cui il cibo abbonda, i cubi gelatinosi possono vivere per centinaia, se non migliaia, di anni. Tuttavia, se viene a mancare la materia organica per più di 6 mesi, un cubo gelatinoso comincia a deperire, e le sue pareti iniziano a colare, disfacendosi rapidamente in muco liquido finché l'intero corpo non collassa e scompare completamente.

\mostro{Melma Grigia}
\noindent
\begin{description}[noitemsep, topsep=0pt, parsep=0pt, partopsep=0pt, leftmargin=0cm, labelwidth=2.2cm]
	\item[\textbf{Taglia/Tipo:}] Media melma, disallineato
	\item[\textbf{Caratt.:}] \resizebox{0.5\linewidth+1.8cm}{!}{For 1 Des -2 Cos 3 Int -5 Sag -2 Car -4}
	\item[\textbf{Punti Ferita:}] 24,  \textbf{Difesa:} 10,  \textbf{Iniziativa:} -2
	\item[\textbf{Movimento:}] 3 m, scalata 3 m
	\item[\textbf{Tiri Salvez.:}] \resizebox{0.5\linewidth+1.8cm}{!}{Tempra +3, Riflessi +3, Volontà +3}
	\item[\textbf{Res. Danni:}] Acido, Freddo, Fuoco
	\item[\textbf{Immunità:}] accecato, affascinato, assordato, prono, affaticato, spaventato
	\item[\textbf{Sensi:}] Vista Cieca 18 m (cieca oltre questo raggio)
	\item[\textbf{Sfida:}] 1/2 (100 PX)\smallskip
\end{description}

\emph{\textbf{Amorfo.}} La melma può muoversi attraverso uno spazio fino a centimetri di larghezza senza doversi stringere.

\emph{\textbf{Corrodere Metallo.}} Qualsiasi arma non magica fatta di metallo che colpisca la melma si corrode. Dopo aver inflitto il danno, l'arma subisce una penalità permanente e cumulativa di -1 ai tiri di danno. Se la penalità arriva a -5, l'arma è distrutta. Le munizioni non magiche fatte di metallo che colpiscano la melma si distruggono dopo aver inflitto il danno.

La melma può divorare metallo non magico dello spessore di 5 centimetri in un 1 round.

\emph{\textbf{Falso Aspetto.}} Quando la melma rimane immobile, è indistinguibile da una pozza d'olio o una pietra bagnata.

\emph{\textbf{Natura di Melma.}} La melma non necessita di dormire.

\textbf{Azioni}

\emph{\textbf{Pseudopodo.} Attacco con arma da mischia}: +4 a colpire, portata 1 m, un bersaglio.

\emph{Colpisce:} 4 (1d6 + 1) danni contundenti più 7 (2d6) danni da acido. Se il bersaglio sta indossando un'armatura di metallo questa viene parzialmente dissolta e subisce una penalità permanente e cumulativa di -1 alla Difesa che offre. L'armatura è distrutta se la penalità arriva a -6.

\textbf{Ecologia}\\
Ambiente: Paludi fredde e sotterranei\\
Organizzazione: Solitario\\
\textbf{Categoria Tesoro}: Nessuno\\
\textbf{Descrizione}\\
Strisciando attraverso le fredde paludi e gli acquitrini nebbiosi o, a volte in sotterranei e caverne, le melme grigie consumano ogni sostanza organica che incontrano. Sebbene priva di intelligenza, la melma grigia è una delle creature che dà non pochi problemi per la sua trasparenza. Anche se non può arrampicarsi facilmente sui muri o nuotare, la sua abitudine di nascondersi nel fango spesso lungo le rive paludose o di rimanere immobile in pozze dall'aspetto innocuo sul pavimento grigio di un sotterraneo, la rendono molto difficile da notare e da evitare.

Alcuni saggi credono che le melme grigie siano il risultato di un esperimento alchemico fallito, mentre altri teorizzano che le prime melme grigie siano nate spontaneamente da un pozzo di detriti magici. Naturalmente, queste teorie che non le considerano organismi viventi, bensì il risultato di una sfortunata mistura di fluidi caustici e residui magici, sono derisi da chi vive nelle zone infestate da queste creature, che non hanno una storia di inquinamento magico.

\mostro{Protoplasma Nero}
\noindent
\begin{description}[noitemsep, topsep=0pt, parsep=0pt, partopsep=0pt, leftmargin=0cm, labelwidth=2.2cm]
	\item[\textbf{Taglia/Tipo:}] Grande melma, disallineato
	\item[\textbf{Caratt.:}] \resizebox{0.5\linewidth+1.8cm}{!}{For 3 Des -3 Cos 3 Int -5 Sag -2 Car -5}
	\item[\textbf{Punti Ferita:}] 89,  \textbf{Difesa:} 14,  \textbf{Iniziativa:} -3
	\item[\textbf{Movimento:}] 6 m, scalata 6 m
	\item[\textbf{Tiri Salvez.:}] \resizebox{0.5\linewidth+1.8cm}{!}{Tempra +7, Riflessi +3, Volontà +3}
	\item[\textbf{Imm. Danni:}] Acido, Freddo, Elettricità, tagliente, da critico
	\item[\textbf{Immunità:}] accecato, affascinato, assordato, prono, affaticato, spaventato
	\item[\textbf{Sensi:}] Vista Cieca 18 m (cieco oltre questo raggio)
	\item[\textbf{Sfida:}] 4 (1100 PX)\smallskip
\end{description}

\emph{\textbf{Amorfo.}} Il protoplasma nero può muoversi attraverso uno spazio fino a 3 centimetri di larghezza senza doversi stringere.

\emph{\textbf{Forma Corrosiva.}} Una creatura che entri a contatto col protoplasma nero o lo colpisca con un attacco da mischia mentre si trova entro 1 metro da esso subisce 4 (1d8) danni da acido. Qualsiasi arma non magica fatta di metallo o legno che colpisca il protoplasma nero si corrode. Dopo aver inflitto il danno, l'arma subisce una penalità permanente e cumulativa di -1 ai tiri di danno. Se la penalità arriva a -5, l'arma è distrutta. Le munizioni non magiche fatte di metallo o legno che colpiscano il protoplasma nero si distruggono dopo aver inflitto il danno.

Il protoplasma nero può divorare legno o metallo non magico dello spessore di 5 centimetri in un 1 round.

\emph{\textbf{Natura di Melma.}} Il protoplasma nero non necessita di dormire.

\emph{\textbf{Scalare come Ragno.}} Il protoplasma nero può scalare superfici difficili, compreso lo stare a testa in giù sul soffitto, senza bisogno di effettuare una prova di competenza.

\textbf{Azioni}

\emph{\textbf{Pseudopodo.} Attacco con arma da mischia}: +6 a colpire, portata 1 m, un bersaglio.

\emph{Colpisce:} 6 (1d6 + 3) danni contundenti più 18 (4d8) danni da acido. Inoltre, un'armatura non magica indossata dal bersaglio viene parzialmente dissolta e subisce una penalità permanente e cumulativa di -1 alla Difesa che offre. L'armatura è distrutta se la penalità arriva a -5.

\textbf{Reazioni}

\emph{\textbf{Divisione.}} Quando un protoplasma nero di taglia Media o più grande subisce danni da elettricità o taglienti, si divide in due nuovi protoplasma neri di almeno 10 Punti Ferita ciascuno. Ogni nuovo protoplasma nero ha un numero di Punti Ferita pari alla metà del protoplasma nero originale, arrotondati per difetto. I nuovi protoplasmi neri sono di una taglia più piccola di quella originale.

\textbf{Ecologia}\\
Ambiente: Qualsiasi sotterraneo\\
Organizzazione: Solitario\\
\textbf{Categoria Tesoro}: Nessuno\\
\textbf{Descrizione}\\
I protoplasmi neri sono gli spazzini del mondo sotterraneo, costantemente alla ricerca di cibo. Possono percepire corpi organici o metallici nel raggio di 18 metri e attaccano in modo istintivo tali oggetti o esseri finché non li dissolvono, o finché la melma non viene uccisa. Un protoplasma nero si riproduce staccando un pezzo del proprio corpo e formando un nuovo protoplasma più piccolo che raggiunge l'età adulta nel giro di un mese. Alcune tra le creature più intelligenti nel mondo sotterraneo usano i protoplasmi neri per smaltire in modo naturale la spazzatura, creando cave di pietra atte ad ospitare il protoplasma, per poi gettarvi i rifiuti organici o i nemici.

\mostro{Mimic}
\noindent
\begin{description}[noitemsep, topsep=0pt, parsep=0pt, partopsep=0pt, leftmargin=0cm, labelwidth=2.2cm]
	\item[\textbf{Taglia/Tipo:}] Media mostruosità (mutaforma), neutrale
	\item[\textbf{Caratt.:}] \resizebox{0.5\linewidth+1.8cm}{!}{For 3 Des 1 Cos 2 Int -2 Sag 1 Car -1}
	\item[\textbf{Punti Ferita:}] 51,  \textbf{Difesa:} 15,  \textbf{Iniziativa:} +1
	\item[\textbf{Movimento:}] 5 metri
	\item[\textbf{Tiri Salvez.:}] \resizebox{0.5\linewidth+1.8cm}{!}{Tempra +4, Riflessi +3, Volontà +3}
	\item[\textbf{Comp.:}] Furtività +5
	\item[\textbf{Imm. Danni:}] Acido
	\item[\textbf{Immunità:}] prono
	\item[\textbf{Sensi:}] Scurovisione 18 m
	\item[\textbf{Sfida:}] 2 (450 PX)\smallskip
\end{description}

\emph{\textbf{Aderente (Solo Forma di Oggetto).}} Il mimic aderisce a qualsiasi cosa con cui entri in contatto. Una creatura di taglia Enorme o inferiore a cui il mimic aderisce è considerata afferrata da esso (DC 18 per fuggire). Il mimic non si considera Afferrato quando afferra qualcosa.

\emph{\textbf{Afferratore.}} Il mimic ha +1d6 ai tiri per colpire contro una creatura da esso afferrata.

\emph{\textbf{Falso Aspetto (Solo Forma di Oggetto).}} Mentre il mimic rimane immobile, è indistinguibile da un comune oggetto.

\emph{\textbf{Mutaforma.}} Il mimic può usare una Azione per trasformarsi in un oggetto, o per tornare alla sua vera forma amorfa. Le sue statistiche sono le stesse in qualsiasi forma. Qualsiasi equipaggiamento stia indossando o trasportando non si trasforma. Alla morte ritorna al suo vero aspetto.

\medskip

\begin{center}
	\includegraphics[width=0.9\linewidth]{immagini/mimic_grayscale.png}

	\emph{Mimic, la curiosità può fare male...}
\end{center}

\medskip

\textbf{Azioni}

\emph{\textbf{Morso.} Attacco con arma da mischia}: +6 a colpire, portata 1 m, un bersaglio.

\emph{Colpisce:} 7 (1d8 + 3) danni perforanti più 4 (1d8) danni da acido.

\emph{\textbf{Pseudopodo.} Attacco con arma da mischia}: +5 a colpire, portata 1 m, un bersaglio.

\emph{Colpisce:} 7 (1d8 + 3) danni contundenti. Se il mimic è in forma di oggetto, il bersaglio è vittima del tratto Aderente.

\textbf{Ecologia}
Ambiente: Qualsiasi\\
Organizzazione: Solitario\\
\textbf{Categoria Tesoro}: Accidentale (A)\\
\textbf{Descrizione}\\
Si ritiene che i mimic siano il risultato del tentativo di un alchimista di dar vita ad un oggetto inanimato attraverso l'applicazione di un reagente mistico, la cui formula è andata perduta. Nel corso degli anni, queste creature strane ma intelligenti hanno appreso la capacità di trasformarsi in simulacri degli oggetti manufatti, in particolare nei luoghi frequentati poco da un ristretto numero di creature, dove aumentano le loro probabilità di successo con un attacco alle loro vittime.

Anche se i mimic non sono intrinsecamente malvagi, alcuni saggi suggeriscono che attacchino gli uomini e le altre creature intelligenti più per passatempo che per sfamarsi. Il desiderio di ingannare gli altri è parte del loro essere e i loro attacchi a sorpresa rappresentano il culmine di questo desiderio.

Un tipico mimic ha un volume di 2 metri cubi (1 m per 1 m per 2 m) e pesa circa 450 kg. Leggende e storie parlano di mimic di taglie maggiori, con la capacità di assumere la forma di case, navi o interi complessi sotterranei che guarniscono con dei tesori (sia veri che falsi) per attirare al loro interno il loro ignaro cibo.

\mostro{Minotauro}
\noindent
\begin{description}[noitemsep, topsep=0pt, parsep=0pt, partopsep=0pt, leftmargin=0cm, labelwidth=2.2cm]
	\item[\textbf{Taglia/Tipo:}] Grande mostruosità, malvagio
	\item[\textbf{Caratt.:}] \resizebox{0.5\linewidth+1.8cm}{!}{For 4 Des 0 Cos 3 Int -2 Sag 3 Car -1}
	\item[\textbf{Punti Ferita:}] 70,  \textbf{Difesa:} 16,  \textbf{Iniziativa:} +0
	\item[\textbf{Movimento:}] 12 m
	\item[\textbf{Tiri Salvez.:}] \resizebox{0.5\linewidth+1.8cm}{!}{Tempra +6, Riflessi +3, Volontà +6}
	\item[\textbf{Comp.:}] Consapevolezza +7
	\item[\textbf{Sensi:}] Scurovisione 18 m
	\item[\textbf{Linguaggi:}] Abissale
	\item[\textbf{Sfida:}] 3 (700 PX)\smallskip
\end{description}

\emph{\textbf{Carica.}} Se il minotauro si muove di almeno 3 metri diretto verso un bersaglio e lo colpisce con un attacco di incornata durante lo stesso round, il bersaglio subisce 9 (2d8) danni perforanti aggiuntivi. Se il bersaglio è una creatura, deve riuscire un Tiro Salvezza su Tempra DC 15 o venire spinto via fino a 3 metri di distanza e cadere prono. 1 Azione.

\emph{\textbf{Incauto.}} All'inizio del suo round, il minotauro può ottenere +1d6 su tutti i tiri per colpire con armi da mischia effettuati durante quel round, ma i tiri per colpire contro di esso hanno +1d6 fino all'inizio del suo prossimo round.

\emph{\textbf{Ricordare Labirinto.}} Il minotauro può ricordare perfettamente qualsiasi tragitto abbia percorso.

\textbf{Azioni}

\emph{\textbf{Ascia Bipenne.} Attacco con arma da mischia}: +6 a colpire, portata 1 m, un bersaglio.

\emph{Colpisce:} 17 (2d12 + 4) danni taglienti.

\emph{\textbf{Incornata.} Attacco con arma da mischia}: +6 a colpire, portata 1 m, un bersaglio.

\emph{Colpisce:} 13 (2d8 + 4) danni perforanti.

\textbf{Ecologia}\\
Ambiente: Rovine Temperate e Sotterranei\\
Organizzazione: Solitario, coppia o gruppo (3-4)\\
\textbf{Categoria Tesoro}: Ascia Bipenne, O +1 pozione\\
\textbf{Descrizione}\\
Disprezzati dalle razze civilizzate e creati secoli fa da una maledizione divina, i minotauri cacciano, uccidono e divorano gli umanoidi per punire offese vere o presunte, da tempi immemorabili. La maggior parte delle culture ha leggende su come furono creati da divinità vendicative che punirono gli umani deformando le loro sembianze, togliendo loro bellezza e intelligenza, e dotandoli di teste di toro. Tuttavia, i minotauri moderni disprezzano queste leggende, ritenendosi modelli di perfezione divina creati dal signore dei demoni Baphomet.

I nascondigli tradizionali dei minotauri sono i labirinti, sia costruiti sia naturali. Usano la loro astuzia per scoraggiare i nemici incauti che si perdono nei loro nascondigli. Solo quando la disperazione ha preso il sopravvento, il minotauro colpisce le sue vittime. Spesso lasciano scappare una creatura per diffondere il terrore e attirare altri nel loro labirinto, che considerano deliziosi pasti.

I minotauri possono servire mostri o creature malvagie più potenti, cacciando e mangiando a loro piacimento. Possono fare la guardia a potenti oggetti o preziose locazioni, o lavorare come mercenari, cacciando i nemici del padrone.

I minotauri sono combattenti diretti, usando le loro corna per incornare orribilmente le creature vicine all'inizio del combattimento.

\mostro{Mummia}
\noindent
\begin{description}[noitemsep, topsep=0pt, parsep=0pt, partopsep=0pt, leftmargin=0cm, labelwidth=2.2cm]
	\item[\textbf{Taglia/Tipo:}] Media non morto, malvagio
	\item[\textbf{Caratt.:}] \resizebox{0.5\linewidth+1.8cm}{!}{For 3 Des -1 Cos 2 Int -2 Sag 0 Car 1}
	\item[\textbf{Punti Ferita:}] 70,  \textbf{Difesa:} 15,  \textbf{Iniziativa:} -1
	\item[\textbf{Movimento:}] 6 m
	\item[\textbf{Tiri Salvez.:}] \resizebox{0.5\linewidth+1.8cm}{!}{Tempra +5, Riflessi +3, Volontà +3}
	\item[\textbf{Res. Danni:}] da arma non magica
	\item[\textbf{Imm. Danni:}] da Vuoto, Veleno
	\item[\textbf{Immunità:}] affascinato, paralizzato, affaticato, spaventato, sanguinamento
	\item[\textbf{Sensi:}] Scurovisione 18 m
	\item[\textbf{Linguaggi:}] le lingue che conosceva in vita
	\item[\textbf{Sfida:}] 3 (700 PX)\smallskip
\end{description}

\emph{\textbf{Natura Non Morta.}} Un mummia non ha bisogno di aria, cibo, bevande o sonno.

\textbf{Azioni}

\emph{\textbf{Multiattacco.}} La mummia può usare la sua Occhiata Temibile ed effettuare un attacco con il pugno putrefacente.

\emph{\textbf{Pugno Putrefacente.} Attacco con arma da mischia}: +6 a colpire, portata 1 m, un bersaglio.

\emph{Colpisce:} 10 (2d6 + 3) danni contundenti più 10 (3d6) danni da Vuoto. Se il bersaglio è una creatura deve riuscire un Tiro Salvezza su Tempra DC 15 o venire maledetto dalla putrefazione della mummia. Il bersaglio maledetto non può recuperare Punti Ferita e i suoi Punti Ferita massimi diminuiscono di 10 (3d6) ogni 24 ore di durata della maledizione. Se la maledizione riduce i Punti Ferita massimi del bersaglio a 0, il bersaglio muore e il suo corpo si tramuta in polvere. La maledizione dura finché non viene rimossa dall'incantesimo \emph{\hyperlink{Rimuovi Maledizione}{Rimuovi Maledizione}} o altra magia.

\emph{\textbf{Occhiata Temibile.}} La mummia prende a bersaglio una creatura che possa vedere e si trovi entro 18 metri da lei. Se il bersaglio può vedere la mummia deve riuscire un Tiro Salvezza su Volontà DC 15 contro questa magia o restare spaventato fino al termine del prossimo round della mummia. Se il bersaglio fallisce il Tiro Salvezza in maniera critica è anche paralizzato per la stessa durata. Un bersaglio che riesca il Tiro Salvezza è immune all'Occhiata Terribile di tutte le mummie (ma non delle mummie sovrane) per le successive 24 ore.

\mostro{Mummia Sovrana}
\noindent
\begin{description}[noitemsep, topsep=0pt, parsep=0pt, partopsep=0pt, leftmargin=0cm, labelwidth=2.2cm]
	\item[\textbf{Taglia/Tipo:}] Media non morto, malvagio
	\item[\textbf{Caratt.:}] \resizebox{0.5\linewidth+1.8cm}{!}{For 4 Des 0 Cos 3 Int 0 Sag 4 Car 3}
	\item[\textbf{Punti Ferita:}] 294,  \textbf{Difesa:} 32,  \textbf{Iniziativa:} +0
	\item[\textbf{Movimento:}] 6 m
	\item[\textbf{Tiri Salvez.:}] \resizebox{0.5\linewidth+1.8cm}{!}{\resizebox{0.5\linewidth+1.8cm}{!}{Tempra +18, Riflessi +15, Volontà +19}}
	\item[\textbf{Comp.:}] Religione +5, Storia +5
	\item[\textbf{Imm. Danni:}] da Vuoto, Veleno; armi +1
	\item[\textbf{Immunità:}] affascinato, paralizzato, affaticato, spaventato
	\item[\textbf{Sensi:}] Scurovisione 18 m
	\item[\textbf{Linguaggi:}] le lingue che conosceva in vita
	\item[\textbf{Sfida:}] 15 (13000 PX)\smallskip
\end{description}

\emph{\textbf{Cuore della Mummia Sovrana.}} Come parte del rituale che crea una mummia sovrana, il cuore e le viscere della creatura vengono rimossi dal cadavere e piazzati all'interno di contenitori sigillati. Questi contenitori sono di solito fatti in pietra o ceramica, incisi o dipinti con geroglifici religiosi.

Finché il suo cuore avvizzito rimane intatto, la mummia sovrana non può essere permanentemente distrutta. Quando scende a 0 Punti Ferita, la mummia sovrana si riduce in polvere e si riforma a piena forza 24 ore più tardi riemergendo dalla polvere in prossimità della giara sigillata che contiene il suo cuore. Per impedire che una mummia sovrana si riformi e distruggerla una volta per tutte bisogna ridurne il cuore in cenere. Per questo motivo, la mummia sovrana di solito tiene il cuore e le viscere nascoste all'interno di una tomba nascosta.

Il cuore della mummia sovrana ha Difesa 5, 25 Punti Ferita e immunità a tutti i danni eccetto Luce.

\emph{\textbf{Incantesimi.}} La mummia ha CM 10. La sua caratteristica da incantatore è la Saggezza. La mummia ha preparati i seguenti incantesimi: Trucchetti (a volontà): \emph{\hyperlink{Fiamma Sacra}{Fiamma Sacra}, \hyperlink{Taumaturgia}{Taumaturgia}}

livello 1 (4 slot): \emph{\hyperlink{Comando}{Comando}, \hyperlink{Dardo Tracciante}{Dardo Tracciante}}

livello 2 (3 slot): \emph{\hyperlink{Arma Spirituale}{Arma Spirituale}, \hyperlink{Blocca Persona}{Blocca Persona}, \hyperlink{Silenzio}{Silenzio}}

livello 3 (3 slot): \emph{\hyperlink{Animare Morti}{Animare Morti}, \hyperlink{Dissolvi Magie}{Dissolvi Magie}}

livello 4 (3 slot): \emph{\hyperlink{Divinazione}{Divinazione}}

livello 5 (2 slot): \emph{\hyperlink{Contagio}{Contagio}, \hyperlink{Piaga degli Insetti}{Piaga degli Insetti}}

livello 6 (1 slot): \emph{\hyperlink{Ferire}{Ferire}}

\emph{\textbf{Natura Non Morta.}} Un mummia non ha bisogno di aria, cibo, bevande o sonno.

\emph{\textbf{Resistenza alla Magia.}} La mummia sovrana ha +1d6 ai Tiri Salvezza contro incantesimi o altri effetti magici.

\emph{\textbf{Rinvigorimento.}} Una mummia sovrana forma un nuovo corpo entro 24 ore se il suo cuore resta intatto, recuperando tutti i Punti Ferita e potendo agire nuovamente. Il nuovo corpo compare entro 1 metro dal cuore della mummia sovrana.

\textbf{Azioni}

\emph{\textbf{Multiattacco.}} La mummia può usare la sua Occhiata Temibile ed effettuare un attacco con il pugno putrefacente, oppure 2 Pugni Putrefacente.

\emph{\textbf{Pugno Putrefacente.} Attacco con arma da mischia}: +13 a colpire, portata 1 m, un bersaglio.

\emph{Colpisce:} 14 (3d6 + 4) danni contundenti più 21 (6d6) danni da Vuoto. Se il bersaglio è una creatura deve riuscire un Tiro Salvezza su Tempra 28 o venire maledetto dalla putrefazione della mummia. Il bersaglio maledetto non può recuperare Punti Ferita, e i suoi Punti Ferita massimi diminuiscono di 10 (3d6) ogni 24 ore di durata della maledizione. Se la maledizione riduce i Punti Ferita massimi del bersaglio a 0, il bersaglio muore, e il suo corpo si tramuta in polvere. La maledizione dura finché non viene rimossa dall'incantesimo \emph{\hyperlink{Rimuovi Maledizione}{Rimuovi Maledizione}} o altra magia.

\emph{\textbf{Occhiata Temibile.}} La mummia prende a bersaglio una creatura che possa vedere e si trovi entro 18 metri da lei. Se il bersaglio può vedere la mummia, deve riuscire un Tiro Salvezza su Volontà DC 28 contro questa magia o restare spaventato fino al termine del prossimo round della mummia. Se il bersaglio fallisce il Tiro Salvezza in maniera critica è anche paralizzato per la stessa durata. Un bersaglio che riesca il Tiro Salvezza è immune all'Occhiata Terribile di tutte le mummie (ma non delle mummie sovrane) per le successive 24 ore.

\textbf{Reazione: \emph{Attacco d'opportunità}}: la mummia sovrana effettua un pugno putrefacente ad una creatura che attraversi o esca dalla sua portata di 1 metro.

\textbf{Azioni Aggiuntive}

La mummia sovrana può effettuare 3 Azioni aggiuntive, scelte tra le opzioni seguenti. Può usare solo un'opzione Aggiuntiva alla volta e solo al termine del round di un'altra creatura. La mummia sovrana recupera le Azioni aggiuntive spese all'inizio del proprio round.

\emph{\textbf{Attaccare.}} La mummia sovrana effettua un attacco con il pugno putrefacente o usa la sua Occhiata Temibile.

\emph{\textbf{Incanalare Energia Negativa (Costa 2 Azioni).}} La mummia sovrana può scatenare magicamente l'energia negativa. Le creature entro 18 metri dalla mummia sovrana, comprese quelle dietro barriere o angoli, non possono recuperare Punti Ferita fino al termine del prossimo round della mummia sovrana.

\emph{\textbf{Parola Blasfema (Costa 2 Azioni).}} La mummia sovrana pronuncia una parola blasfema. Ciascuna creatura, esclusi i non morti, entro 3 metri dalla mummia sovrana e che possa udire questa frase magica deve riuscire un Tiro Salvezza di Tempra DC 28 o restare stordita fino al termine del prossimo round della mummia sovrana.

\emph{\textbf{Polvere Accecante.}} Polvere e sabbia accecanti turbinano magicamente intorno alla mummia sovrana. Ogni creatura entro 1 metro dalla mummia sovrana deve riuscire un Tiro Salvezza di Tempra DC 28 o restare accecata fino al termine del prossimo round della creatura.

\emph{\textbf{Turbine di Sabbia (Costa 2 Azioni).}} La mummia sovrana può trasformarsi magicamente in un turbine di sabbia, muovendosi di massimo 18 metri, e tornando poi alla sua forma normale. Mentre è in forma di turbine, la mummia sovrana è immune a tutti i danni, e non può essere afferrata, pietrificata, gettata prona, intralciata o stordita. L'equipaggiamento indossato o trasportato dalla mummia sovrana rimane in suo possesso.

\emph{\textbf{Arrabbiato:}} La Mummia sovrana ha fame di vita. Incanala l'energia della morte e distruzione in un raggio di 12 metri intorno a se. Ogni creatura deve superare un Tiro Salvezza su Tempra a DC 26 per dimezzare o subire 22 di danno. La Mummia recupera tutti i Punti Ferita persi dalle altre creature.

\textbf{Ecologia}
Ambiente: Qualsiasi\\
Organizzazione: Solitario, Gruppo (3-6) o Mausoleo (7-12)\\
\textbf{Categoria Tesoro}: T + U\\
\textbf{Descrizione}\\
Molte culture praticano l'arte sacra della mummificazione, anche se le sinistre tecniche magiche utilizzate per infondere ai cadaveri la vitalità dei non morti sono molto meno comuni. In alcune terre antiche, tali tecniche blasfeme sono state affinate attraverso secoli di cerimonie e innumerevoli morti, risultando in mummie di terribile potere. In rare occasioni, se il defunto era di rango elevato ed eccessiva malvagità, poteva sottoporsi a rituali così elaborati, risorgendo dalla tomba come un temibile signore mummia. Allo stesso modo, un sovrano noto per la sua malizia o morto in un momento di grande rabbia potrebbe presentarsi spontaneamente come un despota vendicativo. Indipendentemente dalle circostanze esatte della sua risurrezione, una mummia sovrana conserva le capacità che aveva in vita, diventando una creatura consumata dal desiderio di ripristinare il suo dominio e governare sia i vivi che i morti.

%\addcontentsline{toc}{subsubsection}{N}
\pdfbookmark[3]{N}{N}

\mostro{Naga Guardiano}
\noindent
\begin{description}[noitemsep, topsep=0pt, parsep=0pt, partopsep=0pt, leftmargin=0cm, labelwidth=2.2cm]
	\item[\textbf{Taglia/Tipo:}] Grande mostruosità, buono
	\item[\textbf{Caratt.:}] \resizebox{0.5\linewidth+1.8cm}{!}{For 4 Des 4 Cos 3 Int 3 Sag 4 Car 4}
	\item[\textbf{Punti Ferita:}] 201,  \textbf{Difesa:} 29,  \textbf{Iniziativa:} +4
	\item[\textbf{Movimento:}] 12 m
	\item[\textbf{Tiri Salvez.:}] \resizebox{0.5\linewidth+1.8cm}{!}{\resizebox{0.5\linewidth+1.8cm}{!}{Tempra +13, Riflessi +14, Volontà +14}}
	\item[\textbf{Imm. Danni:}] Veleno
	\item[\textbf{Immunità:}] affascinato
	\item[\textbf{Sensi:}] Scurovisione 18 m
	\item[\textbf{Linguaggi:}] Celestiale, Comune
	\item[\textbf{Sfida:}] 10 (5900 PX)\smallskip
\end{description}

\emph{\textbf{Incantesimi.}} Il naga ha CM 11. La sua caratteristica da incantatore è la Saggezza (+8 a colpire con attacchi con incantesimo), e ha bisogno solo delle componenti verbali per lanciare i suoi incantesimi. Il naga prepara i seguenti incantesimi:

Trucchetti (a volontà): \emph{\hyperlink{Fiamma Sacra}{Fiamma Sacra}, \hyperlink{Riparare}{Riparare}, \hyperlink{Taumaturgia}{Taumaturgia}}

livello 1 (4 slot): \emph{\hyperlink{Comando}{Comando}, \hyperlink{Cura Ferite}{Cura Ferite}}

livello 2 (3 slot): \emph{\hyperlink{Blocca Persona}{Blocca Persona}, \hyperlink{Calmare Emozioni}{Calmare Emozioni}}

livello 3 (3 slot): \emph{\hyperlink{Chiaroveggenza}{Chiaroveggenza}, \hyperlink{Scagliare Maledizione}{Scagliare Maledizione}}

livello 4 (3 slot): \emph{\hyperlink{Esilio}{Esilio}, \hyperlink{Libertà di Movimento}{Libertà di Movimento}}

livello 5 (2 slot): \emph{\hyperlink{Colpo Infuocato}{Colpo Infuocato}, \hyperlink{Costrizione}{Costrizione}}

livello 6 (1 slot): \emph{\hyperlink{Visione del Vero}{Visione del Vero}}

\emph{\textbf{Rinvigorimento.}} Se muore il naga ritorna in vita in 1d6 giorni e recupera tutti i suoi Punti Ferita. Solo l'incantesimo \emph{\hyperlink{Desiderio}{Desiderio}} può impedire a questo tratto di funzionare.

\textbf{Azioni}

\emph{\textbf{Morso.} Attacco con arma da mischia}: +11 a colpire, portata 3 m, una creatura.

\emph{Colpisce:} 8 (1d8 + 4) danni perforanti, e il bersaglio deve effettuare un Tiro Salvezza di Tempra DC 23, subendo 45 (10d8) danni da veleno se fallisce il Tiro Salvezza, o la metà di questi danni se lo riesce.

\emph{\textbf{Sputare Veleno.} Attacco con arma a Distanza}: +11 a colpire, gittata 5m, una creatura.

\emph{Colpisce:} Il bersaglio deve effettuare un Tiro Salvezza su Tempra DC 23, subendo 45 (10d8) danni da veleno se fallisce il Tiro Salvezza, o la metà di questi danni se lo riesce.

\textbf{Reazione: \emph{Attacco d'opportunità}}: il naga effettua un attacco di sputo ad una creatura che attraversi o esca dalla sua portata di 3 metri.

\textbf{Ecologia}\\
Ambiente: Pianure Temperate\\
Organizzazione: Solitario, coppia o nido (3-6)\\
\textbf{Categoria Tesoro}: R\\
\textbf{Descrizione}\\
Sebbene abbiano un aspetto feroce, con scaglie brillanti, cappucci simili a quelli dei cobra e potenti corpi serpentini, i naga guardiani fungono da coscienziosi protettori di luoghi di eccezionale potere e sacralità. Spesso le loro scaglie sfoggiano disegni elaborati simili a quelli degli esotici serpenti della giungla. Un tipico naga guardiano raggiunge la lunghezza di 4,2 metri e un peso approssimativo di 175 kg.

Mentre alcuni naga guardiani aderiscono a pratiche esotiche di divinità antiche o dimenticate, altri sono semplicemente attratti da siti dalla spiccata bellezza naturale, quali templi su imponenti cascate, pinnacoli naturali e cime di montagne, custodendoli con il massimo della reverenza e del senso del dovere. Spesso questi naga si uniscono a fedi ancora attive, servendo come protettori di santuari o antichi tesori. Una coppia di naga può stabilirsi nei pressi di un sito che ritengono meritevole di protezione, covandovi una nidiata e crescendovi la prole. Quando i giovani raggiungono l'età adulta, possono scegliere di partire per cercare la propria casa o rimanere a proteggere la zona sorvegliata dai loro genitori. A volte, un naga guardiano che custodisce delle rovine od un tempio è solo l'ultimo di una successione di sentinelle che si sono avvicendate nel corso dei secoli. Queste sentinelle spesso prendono lo stesso nome dei loro predecessori sembrando un unico individuo eccezionalmente longevo.

\mostro{Naga Spirituale}
\noindent
\begin{description}[noitemsep, topsep=0pt, parsep=0pt, partopsep=0pt, leftmargin=0cm, labelwidth=2.2cm]
	\item[\textbf{Taglia/Tipo:}] Grande mostruosità, malvagio
	\item[\textbf{Caratt.:}] \resizebox{0.5\linewidth+1.8cm}{!}{For 4 Des 3 Cos 2 Int 3 Sag 2 Car 3}
	\item[\textbf{Punti Ferita:}] 162,  \textbf{Difesa:} 25,  \textbf{Iniziativa:} +3
	\item[\textbf{Movimento:}] 12 m
	\item[\textbf{Tiri Salvez.:}] \resizebox{0.5\linewidth+1.8cm}{!}{\resizebox{0.5\linewidth+1.8cm}{!}{Tempra +10, Riflessi +11, Volontà +10}}
	\item[\textbf{Imm. Danni:}] Veleno
	\item[\textbf{Immunità:}] affascinato
	\item[\textbf{Sensi:}] Scurovisione 18 m
	\item[\textbf{Linguaggi:}] Abissale, Comune
	\item[\textbf{Sfida:}] 8 (3900 PX)\smallskip
\end{description}

\emph{\textbf{Incantesimi.}} Il naga ha CM 10. La sua abilità da incantatore è l'Intelligenza (+6 a colpire con attacchi con incantesimo), e ha bisogno solo delle componenti verbali per eseguire i suoi incantesimi. Il naga prepara i seguenti incantesimi:

Trucchetti (a volontà): \emph{\hyperlink{Illusione Minore}{Illusione Minore}, \hyperlink{Mano Magica}{Mano Magica}, \hyperlink{Raggio di Gelo}{Raggio di Gelo}}

livello 1 (4 slot): \emph{\hyperlink{Charme su Persone}{Charme su Persone}, \hyperlink{Individuazione del Magico}{Individuazione del Magico}, \hyperlink{Sonno}{Sonno}}

livello 2 (3 slot): \emph{\hyperlink{Blocca Persona}{Blocca Persona}, \hyperlink{Individuazione dei Pensieri}{Individuazione dei Pensieri}}

livello 3 (3 slot): \emph{\hyperlink{Fulmine}{Fulmine}, \hyperlink{Respirare Sott'Acqua}{Respirare Sott'Acqua}}

livello 4 (3 slot): \emph{\hyperlink{Inaridire}{Inaridire}, \hyperlink{Porta Dimensionale}{Porta Dimensionale}}

livello 5 (2 slot): \emph{\hyperlink{Dominare Persone}{Dominare Persone}}

\emph{\textbf{Rinvigorimento.}} Se muore, il naga ritorna in vita in 1d6 giorni e recupera tutti i suoi Punti Ferita. Solo l'incantesimo \emph{\hyperlink{Desiderio}{Desiderio}} può impedire a questo tratto di funzionare.

\textbf{Azioni}

\emph{\textbf{Morso.} Attacco con arma da mischia}: +9 a colpire, portata 3 m, una creatura.

\emph{Colpisce:} 7 (1d8 + 4) danni perforanti, e il bersaglio deve effettuare un Tiro Salvezza di Tempra DC 20, subendo 31 (7d8) danni da veleno se fallisce il Tiro Salvezza, o la metà di questi danni se lo riesce.

\textbf{Reazione: \emph{Attacco d'opportunità}}: il naga effettua un attacco di sputo ad una creatura che attraversi o esca dalla sua portata di 3 metri.

\mostro{Nano Oscuro}
\noindent
\begin{description}[noitemsep, topsep=0pt, parsep=0pt, partopsep=0pt, leftmargin=0cm, labelwidth=2.2cm]
	\item[\textbf{Taglia/Tipo:}] Media umanoide (nano), malvagio
	\item[\textbf{Caratt.:}] \resizebox{0.5\linewidth+1.8cm}{!}{For 2 Des 0 Cos 2 Int 0 Sag 0 Car -1}
	\item[\textbf{Punti Ferita:}] 33,  \textbf{Difesa:} 13,  \textbf{Iniziativa:} +0
	\item[\textbf{Movimento:}] 8 m
	\item[\textbf{Tiri Salvez.:}] \resizebox{0.5\linewidth+1.8cm}{!}{Tempra +3, Riflessi +3, Volontà +3}
	\item[\textbf{Sensi:}] Scurovisione 36 m
	\item[\textbf{Linguaggi:}] Nanico, Linguaggio delle Profondità
	\item[\textbf{Sfida:}] 1 (200 PX)\smallskip
\end{description}

\emph{\textbf{Resilienza Oscura.}} Il Nano Oscuro ha +1d6 ai Tiri Salvezza contro veleni, incantesimi e illusioni, oltre al resistere al restare affascinato o paralizzato.

\emph{\textbf{Sensibilità alla Luce}}. Mentre è alla luce del sole, il Nano Oscuro ha -1d6 ai tiri di attacco, oltre che alle prove di Consapevolezza basate sulla vista.

\textbf{Azioni}

\emph{\textbf{Ingrandire (Ricarica dopo un 1 ora).}} Per 1 minuto, il Nano Oscuro aumenta magicamente di taglia, insieme a tutto ciò che sta trasportando o indossando. Mentre è ingrandito, il Nano Oscuro è di taglia Grande, raddoppia i dadi di danno degli attacchi con armi basate sulla Forza (già incluso negli attacchi), e ha +1d6 alle prove di Forza e ai Tiri Salvezza di Forza. Se il Nano Oscuro non ha sufficiente spazio per diventare Grande, ottiene la massima taglia concessa dallo spazio a disposizione.

\emph{\textbf{Piccone da Guerra.} Attacco con arma da mischia}: +5 a colpire, portata 1 m, un bersaglio.

\emph{Colpisce:} 6 (1d8 + 2) danni perforanti, o 11 (2d8 + 2) danni perforanti quando ingrandito.

\emph{\textbf{Giavellotto.} Attacco con arma da mischia o a Distanza}: +5 a colpire, portata 1 m o gittata 12m, un bersaglio.

\emph{Colpisce:} 5 (1d6 + 2) danni perforanti o 9 (2d6 + 2) danni
perforanti quando ingrandito.

\emph{\textbf{Invisibilità (Ricarica dopo un 1 ora).}} Il Nano Oscuro diventa magicamente invisibile al massimo per un'ora o finché non attacca, lancia un incantesimo, usa Ingrandire o la sua concentrazione viene spezzata. Tutto l'equipaggiamento che il Nano Oscuro indossa o trasporta diventa invisibile assieme a lui.

\textbf{Ecologia}\\
Ambiente: Qualsiasi sotterraneo\\
Organizzazione: solitario, gruppo (2-5), squadra (6-12 più 3 sergenti di 3° livello e 1 capo di 3°-8° livello), o clan (13-80 più 25\% di bambini non combattenti più 1 sergente di 3° livello ogni 5 adulti, 3-6 tenenti di 3°-6° livello, e 1-4 capitani di 9° livello)\\
\textbf{Categoria Tesoro}: equipaggiamento da PNG (Cotta di Maglia, Scudo Pesante di Metallo, Martello da Guerra, Balestra Leggera con 20 Quadrelli, 3d6 mo)\\
\textbf{Descrizione}\\
Lontani parenti dei Nani, più cupi e deformi, i Nani Oscuro sono creature dal pessimo carattere che odiano gli intrusi nei loro reami sotterranei, ma mai più dei Nani. Vivono in comunità nelle profondità del sottosuolo. Hanno pelle grigio opaco, come fosse sporca di polvere o cenere, ma questa tonalità naturale permette di mimetizzarsi meglio nelle zone sotterranee. Sono una Razza di schiavisti, ma mentre costringono i prigionieri non Nani a lavori massacranti, uccidono senza remore i Nani catturati. In combattimento, i Nani Oscuro tirano di balestra, e poi passano al Martello da Guerra qualche round dopo. Se in inferiorità numerica, o se c'è un pericolo (e spazio) adeguato, un Nano Oscuro userà la sua capacità Ingrandire ed attaccherà.

%\addcontentsline{toc}{subsubsection}{O}
\pdfbookmark[3]{O}{O}

\mostro{Armatura Animata}
\noindent
\begin{description}[noitemsep, topsep=0pt, parsep=0pt, partopsep=0pt, leftmargin=0cm, labelwidth=2.2cm]
	\item[\textbf{Taglia/Tipo:}] Media costrutto, disallineato
	\item[\textbf{Caratt.:}] \resizebox{0.5\linewidth+1.8cm}{!}{For 2 Des 0 Cos 1 Int -5 Sag -4 Car -5}
	\item[\textbf{Punti Ferita:}] 33,  \textbf{Difesa:} 13,  \textbf{Iniziativa:} +0
	\item[\textbf{Movimento:}] 7 m
	\item[\textbf{Tiri Salvez.:}] \resizebox{0.5\linewidth+1.8cm}{!}{Tempra +3, Riflessi +3, Volontà +3}
	\item[\textbf{Imm. Danni:}] Veleno
	\item[\textbf{Immunità:}] accecato, affascinato, assordato, paralizzato, pietrificato, affaticato, spaventato
	\item[\textbf{Sensi:}] Vista Cieca 18 m (cieco oltre questo raggio)
	\item[\textbf{Sfida:}] 1 (200 PX)\smallskip
\end{description}

\emph{\textbf{Falso Aspetto.}} Mentre l'armatura rimane immobile, è indistinguibile da una normale armatura.

\emph{\textbf{Suscettibilità all'Anti Magia.}} L'armatura è inabile se si trova nell'area di un \emph{campo anti-magia}. Se è bersaglio di \emph{\hyperlink{Dissolvi Magie}{Dissolvi Magie}}, l'armatura deve riuscire un Tiro Salvezza su Tempra contro la DC del Tiro Salvezza dell'incantesimo o restare svenuta per 1 minuto.

\textbf{Azioni}

\emph{\textbf{Multiattacco.}} L'armatura effettua due attacchi da mischia.

\emph{\textbf{Schianto.} Attacco con arma da mischia}: +5 a colpire, portata 1 m, un bersaglio.

\emph{Colpisce:} 5 (1d6 + 2) danni contundenti.

\mostro{Spada Volante}
\noindent
\begin{description}[noitemsep, topsep=0pt, parsep=0pt, partopsep=0pt, leftmargin=0cm, labelwidth=2.2cm]
	\item[\textbf{Taglia/Tipo:}] Piccola costrutto, disallineato
	\item[\textbf{Caratt.:}] \resizebox{0.5\linewidth+1.8cm}{!}{For 1 Des 2 Cos 0 Int -5 Sag -3 Car -5}
	\item[\textbf{Punti Ferita:}] 19,  \textbf{Difesa:} 14,  \textbf{Iniziativa:} +2
	\item[\textbf{Movimento:}] 0 m, volo 15 m, Fluttuare
	\item[\textbf{Tiri Salvez.:}] \resizebox{0.5\linewidth+1.8cm}{!}{Tempra +3, Riflessi +3, Volontà +3}
	\item[\textbf{Imm. Danni:}] Veleno
	\item[\textbf{Immunità:}] accecato, affascinato, assordato, paralizzato, pietrificato, spaventato
	\item[\textbf{Sensi:}] Vista Cieca 18 m (cieco oltre questo raggio)
	\item[\textbf{Sfida:}] 1/4 (50 PX)\smallskip
\end{description}

\emph{\textbf{Falso Aspetto.}} Mentre l'arma rimane immobile e non sta volando, è indistinguibile da una normale spada.

\emph{\textbf{Suscettibilità all'Anti Magia.}} La spada è inabile se si trova nell'area di un \emph{campo anti-magia}. Se è bersaglio di \emph{\hyperlink{Dissolvi Magie}{Dissolvi Magie}}, la spada deve riuscire un Tiro Salvezza su Tempra contro la DC del Tiro Salvezza dell'incantesimo o restare svenuta per 1 minuto.

\textbf{Azioni}

\emph{\textbf{Spada Lunga.} Attacco con arma da mischia}: +4 a colpire, portata 1 m, un bersaglio.

\emph{Colpisce:} 5 (1d8 + 1) danni taglienti.

\mostro{Tappeto del Soffocamento}
\noindent
\begin{description}[noitemsep, topsep=0pt, parsep=0pt, partopsep=0pt, leftmargin=0cm, labelwidth=2.2cm]
	\item[\textbf{Taglia/Tipo:}] Grande costrutto, disallineato
	\item[\textbf{Caratt.:}] \resizebox{0.5\linewidth+1.8cm}{!}{For 3 Des 2 Cos 0 Int -5 Sag -4 Car -5}
	\item[\textbf{Punti Ferita:}] 51,  \textbf{Difesa:} 16,  \textbf{Iniziativa:} +2
	\item[\textbf{Movimento:}] 3 m
	\item[\textbf{Tiri Salvez.:}] \resizebox{0.5\linewidth+1.8cm}{!}{Tempra +3, Riflessi +4, Volontà +3}
	\item[\textbf{Imm. Danni:}] Veleno
	\item[\textbf{Immunità:}] accecato, affascinato, assordato, paralizzato, pietrificato, spaventato
	\item[\textbf{Sensi:}] Vista Cieca 18 m (cieco oltre questo raggio)
	\item[\textbf{Sfida:}] 2 (450 PX)\smallskip
\end{description}

\emph{\textbf{Falso Aspetto.}} Mentre il tappeto resta immobile, è indistinguibile da un normale tappeto.

\emph{\textbf{Suscettibilità all'Anti Magia.}} Il tappeto è inabile mentre si trova nell'area di un \emph{campo anti-magia}. Se è il bersaglio di \emph{\hyperlink{Dissolvi Magie}{Dissolvi Magie}}, il tappeto deve riuscire un Tiro Salvezza di Tempra contro la DC del Tiro Salvezza dell'incantatore o cadere privo di sensi per 1 minuto.

\emph{\textbf{Trasferimento di Danno.}} Mentre afferra una creatura, il tappeto subisce solo la metà dei danni che gli sono inferti, e la creatura afferrata dal tappeto subisce l'altra metà.

\textbf{Azioni}

\emph{\textbf{Soffocare.} Attacco con arma da mischia}: +6 a colpire, portata 1 m, una creatura di taglia Media o inferiore.

\emph{Colpisce:} La creatura è afferrata (DC 14 per fuggire). Fino al termine dell'afferrare, il bersaglio è accecato e rischia di soffocare, ma il tappeto non può soffocare un altro bersaglio. Inoltre, all'inizio di ciascun round del bersaglio, il bersaglio subisce 10 (2d6 + 3) danni contundenti.

\mostro{Ogre}
\noindent
\begin{description}[noitemsep, topsep=0pt, parsep=0pt, partopsep=0pt, leftmargin=0cm, labelwidth=2.2cm]
	\item[\textbf{Taglia/Tipo:}] Grande gigante, sadico malvagio
	\item[\textbf{Caratt.:}] \resizebox{0.5\linewidth+1.8cm}{!}{For 4 Des -1 Cos 3 Int -3 Sag -2 Car -2}
	\item[\textbf{Punti Ferita:}] 52,  \textbf{Difesa:} 13,  \textbf{Iniziativa:} -1
	\item[\textbf{Movimento:}] 12 m
	\item[\textbf{Tiri Salvez.:}] \resizebox{0.5\linewidth+1.8cm}{!}{Tempra +5, Riflessi +3, Volontà +3}
	\item[\textbf{Sensi:}] Scurovisione 18 m
	\item[\textbf{Linguaggi:}] Comune, Gigante
	\item[\textbf{Sfida:}] 2 (450 PX)\smallskip
\end{description}

\textbf{Azioni}

\emph{\textbf{Randello Pesante.} Attacco con arma da mischia}: +7 a colpire, portata 1 m, un bersaglio.

\emph{Colpisce:} 13 (2d8 + 4) danni contundenti.

\emph{\textbf{Giavellotto.} Attacco con arma da mischia o a Distanza}: +6 a colpire, portata 1 m o gittata 12m, un bersaglio.

\emph{Colpisce:} 11 (2d6 + 4) danni perforanti.

\textbf{Ecologia}\\
Ambiente: Colline fredde o temperate\\
Organizzazione: Solitario, coppia, gruppo (3-4) o famiglia (5-16)\\
\textbf{Categoria Tesoro}: Armatura di Pelle, Randello Pesante, 4 Giavellotti, J\\
\textbf{Descrizione}\\
Nelle storie riguardanti gli ogre ci sono elementi orrendi: brutalità e ferocia, cannibalismo e tortura. Poi stupri, smembramenti, necrofilia, incesto, mutilazioni e altri esempi di crudeltà. Coloro che non hanno mai incontrato gli ogre ritengono queste storie un avvertimento. Chi è sopravvissuto ad un simile incontro sa che le storie sono niente in confronto alla realtà.

Gli ogre godono della sofferenza altrui.

Se non hanno a disposizione le razze più piccole da schiacciare fra le loro grasse mani o da violare in amplessi violenti, si divertono fra loro. Per gli ogre non esiste tabù.

Si potrebbe pensare che, lasciata a sé stessa, una tribù di ogre si farebbe a pezzi da sola e che soltanto i più forti sopravvivrebbero: se c'è una cosa che gli ogre rispettano, però, è la famiglia.

Le tribù ogre sono conosciute come famiglie, e molte delle loro deformità sono causate dalla pratica comune dell'incesto.
Il capo della tribù è spesso il padre, ma in alcuni casi un ogre femmina è in grado di reclamare il titolo di madre. Le tribù ogre litigano fra loro, cosa che li tiene impegnati ed impedisce loro di tormentare i loro vicini. Di quando in quando, però, emerge un patriarca particolarmente violento o temuto, capace di unire più famiglie sotto il suo comando.

\begin{center}
	\includegraphics[width=0.8\linewidth]{immagini/The_Grey_Fairy_Book_-_Page_345.png}

	\emph{Henry Justice Ford}
\end{center}

Le regioni abitate degli ogre sono luoghi tristi e degradati, dato che questi giganti vivono nello squallore e non sentono il bisogno di essere in armonia con quanto li circonda.

I giochi degli ogre sono violenti e crudeli: le vittime utilizzate come giocattolo sono fortunate a morire il primo giorno. Il crudele senso dell'umorismo degli ogre è il solo caso in cui mostrano di possedere creatività: i metodi e gli strumenti di tortura ogre sembrano usciti dagli incubi.

La grande forza e la mancanza di immaginazione li rendono particolarmente adatti ai lavori pesanti, nelle miniere, come fabbri o nel disboscamento. I giganti più potenti (soprattutto quelli delle Colline e delle Rocce) spesso soggiogano le famiglie ogre perché diventino loro servitori.

Un ogre adulto è alto sui 3 metri e pesa circa 325 kg.

\mostro{Ombra}
\noindent
\begin{description}[noitemsep, topsep=0pt, parsep=0pt, partopsep=0pt, leftmargin=0cm, labelwidth=2.2cm]
	\item[\textbf{Taglia/Tipo:}] Media non morto, malvagio
	\item[\textbf{Caratt.:}] \resizebox{0.5\linewidth+1.8cm}{!}{For -2 Des 2 Cos 1 Int -2 Sag 0 Car -1}
	\item[\textbf{Punti Ferita:}] 24,  \textbf{Difesa:} 14,  \textbf{Iniziativa:} +2
	\item[\textbf{Movimento:}] 12 m
	\item[\textbf{Tiri Salvez.:}] \resizebox{0.5\linewidth+1.8cm}{!}{Tempra +3, Riflessi +3, Volontà +3}
	\item[\textbf{Comp.:}] Furtività +4 (+6 a luce fioca o oscurità)
	\item[\textbf{Res. Danni:}] Acido, Freddo, Elettricità, Fuoco, Suono; da arma non magica
	\item[\textbf{Imm. Danni:}] da Vuoto, Veleno
	\item[\textbf{Immunità:}] afferrato, intralciato, paralizzato, pietrificato, prono, affaticato, spaventato, sanguinamento
	\item[\textbf{Sensi:}] Scurovisione 18 m
	\item[\textbf{Sfida:}] 1/2 (100 PX)\smallskip
\end{description}

\emph{\textbf{Amorfo.}} L'ombra può muoversi attraverso uno spazio stretto fino a 3 centimetri senza stringersi.

\emph{\textbf{Debolezza alla Luce del Sole.}} Mentre si trova alla luce del sole, l'ombra ha -1d6 ai tiri per colpire, le prove di competenza di Base e i Tiri Salvezza.

\emph{\textbf{Spirito dell'Ombra.}} Mentre si trova in una zona di luce fioca l'Ombra rigenera 5 Punti Ferita all'inizio del suo round, se si trova in una zona di oscurità rigenera 10 Punti Ferita all'inizio del suo round e può diventare invisibile usando 1 Azione. Spirito dell'Ombra aumenta il Grado di Sfida dell'Ombra di 1.

\emph{\textbf{Furtività d'Ombra.}} Quando si trova a luce fioca o all'oscurità, l'ombra può usare una Azione per prendere +2 alla Difesa.

\emph{\textbf{Natura Non Morta.}} Un'ombra non necessita aria, cibo, bevande o sonno.

\textbf{Azioni}

\emph{\textbf{Risucchio di Forza.} Attacco con arma da mischia}: +4 a colpire, portata 1 m, una creatura.

\emph{Colpisce:} 9 (2d6 + 2) danni da Vuoto, e il punteggio di Forza del bersaglio viene ridotto di 1. Il bersaglio muore se ciò riduce la sua Forza a -5. Altrimenti, la riduzione resta finché il bersaglio non riposa 8 ore.

Se un umanoide non malvagio muore a causa di questo attacco, entro 1d4 ore dal suo cadavere si animerà una nuova ombra.

\emph{\textbf{Rubare l'ombra.}} Se l'ombra ha già colpito due volte con Risucchio di Forza usando una Azione ruba l'ombra dell'avversario. Rubare l'ombra concede 10 Punti Ferita Temporanei all'ombra. La creatura recupera l'ombra all'alba successiva.

\textbf{Ecologia}\\
Ambiente: Qualsiasi\\
Organizzazione: Solitario, coppia, gruppo (3-6) o sciame (7-12)\\
\textbf{Categoria Tesoro}: Nessuno\\
\textbf{Descrizione}\\
La malvagia ombra si muove lungo il confine tra il buio delle tenebre e la dura verità della luce. L'ombra preferisce infestare le rovine che la civiltà si lascia alle spalle, dove dà la caccia alle creature viventi tanto sciocche da incappare nel suo territorio. L'ombra è un orribile non morto, e come tale non ha scopi o motivazioni apparenti oltre a risucchiare forza vitale e vitalità dagli esseri viventi.

\mostro{Omuncolo}
\noindent
\begin{description}[noitemsep, topsep=0pt, parsep=0pt, partopsep=0pt, leftmargin=0cm, labelwidth=2.2cm]
	\item[\textbf{Taglia/Tipo:}] Minuscola costrutto, neutrale
	\item[\textbf{Caratt.:}] \resizebox{0.5\linewidth+1.8cm}{!}{For -3 Des 2 Cos 0 Int 0 Sag 0 Car -2}
	\item[\textbf{Punti Ferita:}] 15,  \textbf{Difesa:} 14,  \textbf{Iniziativa:} +2
	\item[\textbf{Movimento:}] 6 m, volo 12 m
	\item[\textbf{Tiri Salvez.:}] \resizebox{0.5\linewidth+1.8cm}{!}{Tempra +3, Riflessi +3, Volontà +3}
	\item[\textbf{Imm. Danni:}] Veleno
	\item[\textbf{Immunità:}] affascinato
	\item[\textbf{Sensi:}] Scurovisione 18 m, Vista Cieca 3 m
	\item[\textbf{Linguaggi:}] comprende le lingue del suo creatore ma non può parlare
	\item[\textbf{Sfida:}] 0 (10 PX)\smallskip
\end{description}

\emph{\textbf{Legame Telepatico.}} Mentre l'omuncolo si trova sullo stesso piano di esistenza del suo padrone, può comunicare magicamente al suo padrone quello che percepisce, e i due possono comunicare telepaticamente.

\textbf{Azioni}

\emph{\textbf{Morso.} Attacco con arma da mischia}: +4 a colpire, portata 1 m, una creatura.

\emph{Colpisce:} 1 danno perforante, e il bersaglio deve riuscire un Tiro Salvezza di Tempra DC 10 o restare avvelenato, -1 Forza e Destrezza, per 1 minuto. Se il Tiro Salvezza viene fallito in maniera critica il bersaglio resta invece avvelenato per 5 (1d10) minuti e mentre è avvelenato in questo modo è anche privo di sensi.

\emph{\textbf{Tramite del Padrone}}: usando 3 Azioni l'omuncolo diventa il tramite del lancio di un incantesimo del padrone.

\mostro{Oni}
\noindent
\begin{description}[noitemsep, topsep=0pt, parsep=0pt, partopsep=0pt, leftmargin=0cm, labelwidth=2.2cm]
	\item[\textbf{Taglia/Tipo:}] Grande gigante, malvagio
	\item[\textbf{Caratt.:}] \resizebox{0.5\linewidth+1.8cm}{!}{For 4 Des 0 Cos 3 Int 2 Sag 1 Car 2}
	\item[\textbf{Punti Ferita:}] 145,  \textbf{Difesa:} 21,  \textbf{Iniziativa:} +2
	\item[\textbf{Movimento:}] 9 m, volo 9 m
	\item[\textbf{Tiri Salvez.:}] \resizebox{0.5\linewidth+1.8cm}{!}{Tempra +10, Riflessi +7, Volontà +8}
	\item[\textbf{Comp.:}] Arcana +5, Ingannare +8
	\item[\textbf{Sensi:}] Scurovisione 18 m
	\item[\textbf{Linguaggi:}] Comune, Gigante
	\item[\textbf{Sfida:}] 7 (2900 PX)\smallskip
\end{description}

\emph{\textbf{Armi Magiche.}} Gli attacchi con armi dell'oni sono magici.

\emph{\textbf{Incantesimi Innati.}} La caratteristica da incantatore dell'oni è il Carisma. L'oni può lanciare questi incantesimi in maniera innata, senza bisogno di componenti materiali:

A volontà: \emph{\hyperlink{Invisibilità}{Invisibilità}, \hyperlink{Oscurità}{Oscurità}}

1/giorno: \emph{\hyperlink{Charme su Persone}{Charme su Persone}, \hyperlink{Cono di Freddo}{Cono di Freddo}, \hyperlink{Forma Gassosa}{Forma Gassosa}, \hyperlink{Sonno}{Sonno}}

\emph{\textbf{Rigenerazione.}} Se ha almeno 1 punto ferita, l'oni recupera 10 Punti Ferita all'inizio del suo round.

\textbf{Azioni}

\emph{\textbf{Multiattacco.}} L'oni effettua due attacchi, con gli artigli o con il falcione.

\emph{\textbf{Artiglio (Solo Forma di Oni).} Attacco con arma da mischia}: +8 a colpire, portata 1 m, un bersaglio.

\emph{Colpisce:} 8 (1d8 + 4) danni taglienti.

\emph{\textbf{Falcione.} Attacco con arma da mischia}: +9 a colpire, portata 3 m, un bersaglio.

\emph{Colpisce:} 15 (2d10 + 4) danni taglienti, o 9 (1d10 + 4) danni taglienti in forma Piccola o Media.

\textbf{Reazione: \emph{Attacco d'opportunità}}: l'Oni effettua un attacco ad una creatura che attraversi o esca dalla sua portata di 1/3 metri.

\emph{\textbf{Mutare Forma.}} L'oni può trasformarsi magicamente in un umanoide Piccolo o Medio, in un gigante Grande, o tornare alla sua vera forma. A parte la taglia, le sue statistiche sono le stesse in ciascuna forma. L'unico equipaggiamento che viene trasformato è il falcione, che rimpicciolisce in modo da essere impugnato anche in forma umanoide. Se l'oni muore, ritorna alla sua vera forma e il falcione ritorna alla sua taglia originale.

\emph{\textbf{Arrabbiato:}} l'Oni viene pervaso da una furia assassina, fino alla fine del combattimento i suoi attacchi con Artiglio causano Sanguinamento 2/10.

\mostro{Orchetto}
\noindent
\begin{description}[noitemsep, topsep=0pt, parsep=0pt, partopsep=0pt, leftmargin=0cm, labelwidth=2.2cm]
	\item[\textbf{Taglia/Tipo:}] Media umanoide (orco), caotico
	\item[\textbf{Caratt.:}] \resizebox{0.5\linewidth+1.8cm}{!}{For 2 Des 1 Cos 2 Int 0 Sag 0 Car 0}
	\item[\textbf{Punti Ferita:}] 24,  \textbf{Difesa:} 13,  \textbf{Iniziativa:} +1
	\item[\textbf{Movimento:}] 9 m
	\item[\textbf{Tiri Salvez.:}] \resizebox{0.5\linewidth+1.8cm}{!}{Tempra +3, Riflessi +3, Volontà +3}
	\item[\textbf{Comp.:}] Intimidire +1
	\item[\textbf{Sensi:}] Scurovisione 18 m
	\item[\textbf{Linguaggi:}] Comune, Goblinoide
	\item[\textbf{Sfida:}] 1/2 (100 PX)\smallskip
\end{description}

\textbf{Azioni}

\emph{\textbf{Spada.} Attacco con arma da mischia}: +4 a colpire, portata 1 m, un bersaglio.

\emph{Colpisce:} 6 (1d8 + 2) danni taglienti.

\emph{\textbf{Giavellotto.} Attacco con arma da mischia o a Distanza}: +5 a colpire, portata 1 m o gittata 12m, un bersaglio.

\emph{Colpisce:} 5 (1d6 + 2) danni perforanti.

\textbf{Ecologia}\\
Ambiente: Colline e montagne temperate o sotterranei\\
Organizzazione: solitario, gruppo (2-4), squadra (11-20 più 2 sergenti di 3° livello e 1 capo di 3°-6° livello) o banda \\
\textbf{Categoria Tesoro}: Equipaggiamento da PNG (Armatura di Cuoio Borchiato, Spada, 4 Giavellotti, M)\\
\textbf{Descrizione}\\
Gli orchetti sono una razza creata da Cattalm come esperimento con lo scopo di verificare se una creatura più intelligente ma altrettanto feroce degli orchi avrebbe potuto essere dominante.
L'esperimento è stato un discreto successo con gli orchetti che hanno fondato regni e conquistato diverse regioni. La spinta caotica con il passare del tempo, l'acculturamento, il diventare stanziali e l'evoluzione della società ha portato gli orchetti sempre più fuori dalle spire di Cattalm, anche se non toglie che molti aspetti barbari sono rimasti nella cultura tradizionale.
Un orchetti maschio adulto è alto 1,6 metri e pesa circa 60 kg. Caratteristica peculiare è il volto ed il naso in particolar modo da maiale. Gli orchetti e gli umani possono generare figli.

\mostro{Orco}
\noindent
\begin{description}[noitemsep, topsep=0pt, parsep=0pt, partopsep=0pt, leftmargin=0cm, labelwidth=2.2cm]
	\item[\textbf{Taglia/Tipo:}] Media umanoide (orco), malvagio
	\item[\textbf{Caratt.:}] \resizebox{0.5\linewidth+1.8cm}{!}{For 3 Des 1 Cos 3 Int -2 Sag 0 Car 0}
	\item[\textbf{Punti Ferita:}] 33,  \textbf{Difesa:} 14,  \textbf{Iniziativa:} +1
	\item[\textbf{Movimento:}] 9 m
	\item[\textbf{Tiri Salvez.:}] \resizebox{0.5\linewidth+1.8cm}{!}{Tempra +4, Riflessi +3, Volontà +3}
	\item[\textbf{Comp.:}] Intimidire +2
	\item[\textbf{Sensi:}] Scurovisione 18 m
	\item[\textbf{Linguaggi:}] Comune, Goblinoide
	\item[\textbf{Sfida:}] 1 (100 PX)\smallskip
\end{description}

\emph{\textbf{Aggressivo.}} Come Azione Immediata, l'orco può muoversi fino a metà del suo movimento verso una creatura ostile che possa vedere.

\emph{\textbf{Feroce.}} Come Azione l'orco affonda ancora più il colpo andato a segno causando 1d6 danni aggiuntivi.

\textbf{Azioni}

\emph{\textbf{Ascia Bipenne.} Attacco con arma da mischia}: +6 a colpire, portata 1 m, un bersaglio.

\emph{Colpisce:} 9 (1d12 + 3) danni taglienti.

\emph{\textbf{Giavellotto.} Attacco con arma da mischia o a Distanza}: +5 a colpire, portata 1 m o gittata 12m, un bersaglio.

\emph{Colpisce:} 6 (1d6 + 3) danni perforanti.

\textbf{Ecologia}\\
Ambiente: Colline e montagne temperate o sotterranei\\
Organizzazione: solitario, gruppo (2-4), squadra (11-20 più 2 sergenti di 3° livello e 1 capo di 3°-6° livello) o banda \\
\textbf{Categoria Tesoro}: Equipaggiamento da PNG (Armatura di Cuoio Borchiato, Falcione, 4 Giavellotti, K)\\
\textbf{Descrizione}\\
La differenza principale fra gli orchi e gli umanoidi civilizzati, oltre alla loro forza bruta ed all'intelligenza inferiore, è il loro carattere. Come cultura, gli orchi sono violenti ed aggressivi, ed il forte domina il debole attraverso paura e brutalità. Prendono ciò che vogliono con la forza e non si fanno scrupoli a prendere interi villaggi come schiavi se ne hanno la possibilità. Non si curano delle comodità, ed i loro villaggi e campi tendono ad essere luoghi sporchi e precari, pieni di risse fra ubriachi, arene per i combattimenti ed altri divertimenti sadici. Privi della pazienza necessaria a coltivare e capaci di allevare solo gli animali più robusti ed autosufficienti, gli orchi ritengono più semplice prendere agli altri il frutto del loro lavoro. Sono arroganti e lesti ad infuriarsi quando sfidati, ma si preoccupano dell'onore solo finché farlo porta loro beneficio.

Un orco maschio adulto è alto 2 metri e pesa circa 115 kg. Gli orchi e gli umani possono accoppiarsi, anche se di solito ciò avviene durante le razzie, e non come unione consensuale. Molte tribù orchesche allevano i mezzorchi di proposito, dato che sono ottimi strateghi e capitribù.

Per quanto la vulgata dica che gli orchi siano stati creati da Cattalm per distruggere e portare caos è anche vero che molto spesso sono vittima di pregiudizi e giudizi sommari. Non tutti gli orchi sono uguali e non solo fisicamente, singoli orchi se non intere tribù vivono in maniera normale, civilizzata la loro esistenza eppure in nessun stato del mondo sono previste pene per chi uccide un orco.

\mostro{Orrore Arrampicamuri}
\noindent
\begin{description}[noitemsep, topsep=0pt, parsep=0pt, partopsep=0pt, leftmargin=0cm, labelwidth=2.2cm]
	\item[\textbf{Taglia/Tipo:}] Grande mostruosità, disallineato
	\item[\textbf{Caratt.:}] \resizebox{0.5\linewidth+1.8cm}{!}{For 4 Des 0 Cos 2 Int -2 Sag 1 Car -2}
	\item[\textbf{Punti Ferita:}] 70,  \textbf{Difesa:} 16,  \textbf{Iniziativa:} +0
	\item[\textbf{Movimento:}] 9 m, scalare 9 m
	\item[\textbf{Tiri Salvez.:}] \resizebox{0.5\linewidth+1.8cm}{!}{Tempra +5, Riflessi +3, Volontà +4}
	\item[\textbf{Sensi:}] Scurovisione 3 m, Vista Cieca 18 m
	\item[\textbf{Linguaggi:}] Orrore Arrampicamuri
	\item[\textbf{Sfida:}] 3 (700 PX)\smallskip
\end{description}

\emph{\textbf{Senso Radar.}} l'Orrore Arrampicamuri non può usare vista cieca se è assordato.

\textbf{Azioni}

\emph{\textbf{Multiattacco.}} L'Orrore Arrampicamuri effettua due attacchi con gli artigli uncinati.

\emph{\textbf{Artigli.} Attacco con arma da mischia}: +7 a colpire, portata 1 m, un bersaglio.

\emph{Colpisce:} 10 (2d6 + 4) danni perforanti, 1 danno da Sanguinamento.

\textbf{Ecologia}\\
\textbf{Ambiente: Sottosuolo}
Organizzazione: Solitario, coppia o branco (3-8)\\
\textbf{Categoria Tesoro}: Accidentale\\
\textbf{Descrizione}\\
L'Orrore Arrampicamuri è un feroce predatore del sottosuolo, difende aggressivamente i suoi territori di caccia. Le caverne sotterranee in cui queste creature risiedono rimbombano dei colpi e dei fruscii dei loro uncini quando queste creature si arrampicano sulle rupi rocciose o sulle pareti delle caverne.

Un Orrore Arrampicamuri è una creatura mostruosa apparentemente umanoide dalla testa simile a quella di un avvoltoio e dal torace di un enorme scarabeo, protetto da un esoscheletro tempestato di protuberanze ossee aguzze. Trae il suo nome oltre che dall'orrendo aspetto dal fatto che usando gli arti lunghi e muscolosi che terminano con dei micidiali artigli uncinati ricurvi stia arrampicato sulle pareti.

Gli Orrore Arrampicamuri comunicano colpendo il loro esoscheletro o le superfici rocciose circostanti con i loro uncini

\emph{Branco di Predatori}. Gli Orrore Arrampicamuri sono creature onnivore: si nutrono di funghi, licheni, vegetali e di qualsiasi creatura riescano a catturare. Grazie agli arti uncinati, gli orrori beneficiano di un'ottima presa sulle superfici rocciose e usano le loro abilità da scalatori per tendere imboscate alle prede dall'alto. Vanno a caccia in branco e collaborano per affrontare gli avversari più grossi e pericolosi. Se una battaglia va male, un Orrore Arrampicamuri si arrampica rapidamente lungo la parete di una caverna per fuggire.

\emph{Clan Solidali}. Gli orrori uncinati vivono in vasti gruppi familiari o clan. Ogni clan è retto dalla femmina più anziana, che solitamente pone il suo compagno a capo dei cacciatori del clan. Gli Orrore Arrampicamuri depongono le uova in un'area centrale e ben difesa delle caverne usate come tana.

\mostro{Orsogufo}
\noindent
\begin{description}[noitemsep, topsep=0pt, parsep=0pt, partopsep=0pt, leftmargin=0cm, labelwidth=2.2cm]
	\item[\textbf{Taglia/Tipo:}] Grande bestia, disallineato
	\item[\textbf{Caratt.:}] \resizebox{0.5\linewidth+1.8cm}{!}{For 5 Des 1 Cos 3 Int -4 Sag 1 Car -2}
	\item[\textbf{Punti Ferita:}] 70,  \textbf{Difesa:} 17,  \textbf{Iniziativa:} +1
	\item[\textbf{Movimento:}] 12 m
	\item[\textbf{Tiri Salvez.:}] \resizebox{0.5\linewidth+1.8cm}{!}{Tempra +6, Riflessi +4, Volontà +4}
	\item[\textbf{Sensi:}] Scurovisione 18 m
	\item[\textbf{Sfida:}] 3 (700 PX)\smallskip
\end{description}

\emph{\textbf{Olfatto e Vista Affinati.}} L'Orsogufo ha +1d6 nelle prove di Consapevolezza basate su olfatto o vista.

\textbf{Azioni}

\emph{\textbf{Multiattacco.}} L'Orsogufo effettua due attacchi: uno con il becco e uno con gli artigli.

\emph{\textbf{Artigli.} Attacco con arma da mischia}: +6 a colpire, portata 1 m, un bersaglio.

\emph{Colpisce:} 14 (2d8 + 5) danni taglienti.

\emph{\textbf{Becco.} Attacco con arma da mischia}: +6 a colpire, portata 1 m, una creatura.

\emph{Colpisce:} 10 (1d10 + 5) danni perforanti.

\textbf{Ecologia}\\
\textbf{Ambiente: Foreste Temperate}
Organizzazione: Solitario, coppia o branco (3-8)\\
\textbf{Categoria Tesoro}: Accidentale\\
\textbf{Descrizione}\\
Le origini dell'Orsogufo sono oggetto di dibattito fra gli studiosi delle creature mostruose. La maggior parte di essi concorda che fu un Mago, in passato, a crearne il primo esemplare unendo un orso con un gufo gigante; forse come esperimento su qualche folle concetto della natura della vita, ma più probabilmente a causa della sua totale pazzia. Quale che fosse lo scopo originale di una creazione tanto folle come l'Orsogufo, la creatura ha iniziato a riprodursi, ed è divenuta uno dei predatori più conosciuto delle zone boschive.

Gli Orsogufo sono selvaggi predatori, noti per il loro pessimo temperamento, la loro aggressività e la loro ferocia. Tendono ad attaccare tutto ciò che si muove loro davanti, anche se questo non mostra intenzioni bellicose. Molti studiosi che hanno incontrato queste creature nelle terre selvagge hanno notato che hanno sempre occhi iniettati di sangue che ruotano tutto attorno poco prima di un attacco. Questo è generalmente visto come segno di follia, che suggerisce che tutti gli Orsogufo nascano con un bisogno patologico di combattere ed uccidere, ma i ricercatori più realisti ritengono sia dovuto alla struttura dei loro occhi acuti.

Gli Orsogufo abitano le zone più interne e nascoste dei boschi, e preparano le loro tane all'interno di foreste intricate o di buie e profonde caverne. Possono cacciare sia di giorno che di notte, a seconda delle abitudini delle prede che popolano i territori circostanti alla loro tana.

Gli Orsogufo adulti vivono in coppia e cacciano le prede in branco, lasciando i cuccioli nelle tane. In una tana si possono trovare di solito 1d6 cuccioli, che possono valere fino a 750 mo nei mercati cittadini.

Anche se è pressoché impossibile addomesticarli a causa della loro natura selvaggia, gli Orsogufo possono essere sfruttati come guardiani di un territorio specifico, sempre che vengano lasciati liberi di spostarsi al suo interno per cacciare. Gli addestratori professionisti chiedono fino a 2000 mo per addestrare un Orsogufo perché diventi un guardiano che obbedisca a comandi semplici (DC 23 per un cucciolo di Orsogufo, DC 30 per un Orsogufo adulto).

\emph{\textbf{Variante}}: \textbf{Orsogufo Polare}\index{Orsogufo Polare}\\
Questo Orsogufo è presente nelle regioni artiche o montane innevate. A differenza del normale Orsogufo è più robusto e forte. Ha 85 Punti Ferita, +10 al colpire, 21 di danno ad artiglio +1 da Sanguinamento, 15 di danno con becco. GS 4

\mostro{Orsogufo Saggio}
\noindent
\begin{description}[noitemsep, topsep=0pt, parsep=0pt, partopsep=0pt, leftmargin=0cm, labelwidth=2.2cm]
	\item[\textbf{Taglia/Tipo:}] Grande mostruosità, neutrale
	\item[\textbf{Caratt.:}] \resizebox{0.5\linewidth+1.8cm}{!}{For 3 Des 1 Cos 2 Int 3 Sag 3 Car 1}
	\item[\textbf{Punti Ferita:}] 70,  \textbf{Difesa:} 17,  \textbf{Iniziativa:} +3
	\item[\textbf{Movimento:}] 12 m
	\item[\textbf{Tiri Salvez.:}] \resizebox{0.5\linewidth+1.8cm}{!}{Tempra +5, Riflessi +4, Volontà +6}
	\item[\textbf{Comp.:}] Consapevolezza +9
	\item[\textbf{Sensi:}] Scurovisione 18 m
	\item[\textbf{Linguaggi:}] comprende e legge i seguenti: Comune, Druidico,Celestiale, Infernale, Nanico, Elfico, Orchesco, Gigante, Expiran, lingue Elementali
	\item[\textbf{Sfida:}] 3 (700 PX)\smallskip
\end{description}

\emph{\textbf{Olfatto e Vista Affinati.}} L'Orsogufo saggio ha +1d6 nelle prove di Consapevolezza basate su olfatto o vista.

\emph{\textbf{Incantesimi Innati.}} La caratteristica da incantatore dell'Orsogufo saggio è l'Intelligenza. L'Orsogufo saggio può lanciare in maniera innata i seguenti incantesimi, senza aver bisogno di componenti materiali:

A volontà: \emph{\hyperlink{Mano Magica}{Mano Magica}, \hyperlink{Comprensione degli Scritti}{Comprensione degli Scritti}}

\textbf{Azioni}

\emph{\textbf{Multiattacco.}} L'Orsogufo saggio effettua due attacchi: uno con il becco e uno con gli artigli.

\emph{\textbf{Artigli.} Attacco con arma da mischia}: +6 a colpire, portata 1 m, un bersaglio.

\emph{Colpisce:} 14 (2d8 + 5) danni taglienti.

\emph{\textbf{Becco.} Attacco con arma da mischia}: +6 a colpire, portata 1 m, una creatura.

\emph{Colpisce:} 10 (1d10 + 5) danni perforanti.

\textbf{Ecologia}\\
\textbf{Ambiente: Foreste Temperate}
Organizzazione: Solitario, coppia o branco (3-8)\\
\textbf{Categoria Tesoro}: E + T\\
\textbf{Descrizione}\\
Le origini dell'Orsogufo saggio sono misteriose quanto quelli del suo parente non saggio ma gli appassionati di queste creature li fanno discendere direttamente da Nethergal come variante dell'Orsogufo originale.
Solitamente l'Orsogufo saggio ama circondarsi di libri ed adora la compagnia di altri saggi ma non disdegna i racconti di avventurieri e le avvincenti ballate dei cantastorie. L'Orsogufo saggio ha un vero talento per le lingue e pur non potendo parlare in maniera comprensibile ad un uomo riesce a comprendere tantissime lingue parlate e scritte. L'Orsogufo saggio è in grado di leggere qualsiasi lingua o codice se ha modo di studiarlo per 3 giorni.
Solitamente più deboli e fragili del parente stretto sono comunque esseri temibili in combattimento.
Di preferenza un Orsogufo saggio non attacca se non per difesa e cerca un approccio il più tattico e utile possibile. Un tratto caratteristico degli Orsogufo saggi è una sciarpa rossa portata intorno all'assente collo. Uccidere un Orsogufo saggio è un affronto ai Devoti e Seguaci di Nethergal, è anche capitato che il Patrono stesso togliesse la capacità di comunicare a coloro si sono macchiati di efferatezze con le sue creature preferite.

Addestrare un Orsogufo saggio è molto più facile di un Orsogufo ma l'alta intelligenza della creatura lo spingerà ad avere un rapporto alla pari o come famiglio piuttosto.

L'incantesimo \hyperlink{Mano Magica}{Mano Magica} è solitamente usato per sfogliare i tomi più delicati e per scrivere, anche se con estrema lentezza.

\mostro{Otyugh}
\noindent
\begin{description}[noitemsep, topsep=0pt, parsep=0pt, partopsep=0pt, leftmargin=0cm, labelwidth=2.2cm]
	\item[\textbf{Taglia/Tipo:}] Grande aberrazione, neutrale
	\item[\textbf{Caratt.:}] \resizebox{0.5\linewidth+1.8cm}{!}{For 3 Des 0 Cos 4 Int 3 Sag 1 Car -2}
	\item[\textbf{Punti Ferita:}] 109,  \textbf{Difesa:} 18,  \textbf{Iniziativa:} +3
	\item[\textbf{Movimento:}] 9 m
	\item[\textbf{Tiri Salvez.:}] \resizebox{0.5\linewidth+1.8cm}{!}{Tempra +9, Riflessi +5, Volontà +6}
	\item[\textbf{Imm. Danni:}] Veleno
	\item[\textbf{Immunità:}] malattie
	\item[\textbf{Sensi:}] Scurovisione 36 m
	\item[\textbf{Linguaggi:}] Otyugh, Elfico, Nanico
	\item[\textbf{Sfida:}] 5 (1800 PX)\smallskip
\end{description}

\emph{\textbf{Telepatia Limitata.}} L'otyugh può trasmettere magicamente dei semplici messaggi e immagini a qualsiasi creatura entro 36 metri da esso e che possa comprendere una lingua. Questa forma di telepatia non permette alla creatura ricevente di rispondere telepaticamente.

\textbf{Azioni}

\emph{\textbf{Multiattacco.}} L'otyugh effettua tre attacchi: uno con il morso e due con i tentacoli.

\emph{\textbf{Morso.} Attacco con arma da mischia}: +7 a colpire, portata 1 m, un bersaglio.

\emph{Colpisce:} 12 (2d8 + 3) danni perforanti. Se il bersaglio è una creatura, deve riuscire un Tiro Salvezza di Tempra DC 17 contro malattia o restare malato finché la malattia non viene curata. Ogni 24 ore successive, il bersaglio deve ripetere il Tiro Salvezza, riducendo il suo massimo di Punti Ferita di 5 (1d10) se lo fallisce. Se il Tiro Salvezza riesce, la malattia è passata. Il bersaglio muore se la malattia riduce i suoi Punti Ferita massimi a 0.

Questa riduzione dei Punti Ferita massimi del personaggio perdura finché la malattia non viene curata.

\emph{\textbf{Tentacolo.} Attacco con arma da mischia}: +7 a colpire, portata 3 m, un bersaglio.

\emph{Colpisce:} 7 (1d8 + 3) danni contundenti più 4 (1d8) danni perforanti. Se il bersaglio è di taglia Media o inferiore, è afferrato (DC 13 per fuggire). L'otyugh ha due tentacoli, ciascun dei quali può afferrare un bersaglio diverso.

\emph{\textbf{Schianto di Tentacolo.}} L'otyugh schianta le creature afferrate dai suoi tentacoli, l'una contro l'altra o sul pavimento. Ogni creatura deve riuscire un Tiro Salvezza di Tempra DC 17 o subire 10 (2d6 + 3) danni contundenti e restare stordita fino al termine del prossimo round dell'otyugh. Se il Tiro Salvezza riesce, il bersaglio subisce la metà dei danni contundenti e non è stordito.

\emph{\textbf{Arrabbiato:}} l'otyugh emette un profumo che inebria i sensi. Tutte le creature nel raggio di 6 metri devono fare un Tiro Salvezza su Volontà DC 18 oppure agire in maniera casuale, come incantesimo \hyperlink{incconfusione}{Confusione} (pag. \pageref{incconfusione}), fino alla fine del prossimo round. Costa 2 Azioni.

\textbf{Ecologia}\\
Ambiente: Qualsiasi Sotterraneo\\
Organizzazione: Solitario, coppia o gruppo (3-4)\\
\textbf{Categoria Tesoro}: I\\
\textbf{Descrizione}\\
Gli otyugh sono creature particolarmente luride ed orride che vivono in luoghi che le persone sane di mente tendono ad evitare. Le loro tane si trovano nelle fogne, nei pozzi neri, nelle discariche e nelle paludi più mefitiche: più un luogo è sporco, più attira gli otyugh. Amano il ruolo dello spazzino e vagano per le caverne sotterranee in cerca di nuovi bocconcini in mezzo ai rifiuti. Una volta trovati si ingozzano e riportano alla loro tana quello che non riescono a consumare in una volta sola. Gli otyugh passano parecchio tempo nelle loro luride tane, che riempiono di carogne e letame, che rilasciano effluvi mefitici.

Le creature intelligenti che vivono nelle zone sotterranee vicino agli otyugh a volte formano alleanze di convenienza con essi. Forniscono loro rifiuti e carne cruda agli otyugh, rendendoli un vero e proprio mezzo di smaltimento. In cambio, gli otyugh lasciano in pace i loro benefattori, non li attaccano e possono anche fare da guardiani.

La cosa che la maggior parte delle razze trova terrificante degli otyugh non è la loro dieta o l'odore delle loro tane, ma il fatto che creature con i loro gusti non siano solo spazzini senza cervello. Gli otyugh si mostrano infatti sorprendentemente intelligenti ed amano formare alleanze con coloro che li riforniscono di cibi più raffinati di letame e sporcizia. La maggior parte degli otyugh si rende conto che le altre creature li trovano rivoltanti, ma sono pochi quelli a cui importa davvero.

Un otyugh mangiando gli escrementi o parte di una creatura può capire quale malattia o veleno la affligge.

%\addcontentsline{toc}{subsubsection}{P}
\pdfbookmark[3]{P}{P}

\mostro{Panoptikhan}
\noindent
\begin{description}[noitemsep, topsep=0pt, parsep=0pt, partopsep=0pt, leftmargin=0cm, labelwidth=2.2cm]
	\item[\textbf{Taglia/Tipo:}] Grande aberrazione, malvagio
	\item[\textbf{Caratt.:}] \resizebox{0.5\linewidth+1.8cm}{!}{For 0 Des 1 Cos 2 Int 3 Sag 2 Car 2}
	\item[\textbf{Punti Ferita:}] 235,  \textbf{Difesa:} 29,  \textbf{Iniziativa:} +3
	\item[\textbf{Movimento:}] 1 m, volo 10 metri, fluttuare
	\item[\textbf{Tiri Salvez.:}] \resizebox{0.5\linewidth+1.8cm}{!}{\resizebox{0.5\linewidth+1.8cm}{!}{Tempra +14, Riflessi +13, Volontà +14}}
	\item[\textbf{Sensi:}] Scurovisione 36 m, visione del vero 18m
	\item[\textbf{Linguaggi:}] telepatia 50 m
	\item[\textbf{Sfida:}] 12 (8400 PX)\smallskip
\end{description}

\emph{\textbf{Resistenza alla Magia.}} Il panoptikhan ha +1d6 ai Tiri Salvezza contro incantesimi e altri effetti magici.

\textbf{Azioni}

\emph{\textbf{Multiattacco.}} Il Panoptikhan può attaccare con due corti tentacoli.

\emph{\textbf{Tentacolo.} Attacco con arma da mischia}: +10 a colpire, portata 1 m, un bersaglio.

\emph{Colpisce:} 6 (1d6 + 3) danni da taglio perforanti.

\emph{\textbf{Colui che tutto vede}}. Il Panoptikhan può attivare uno dei suoi tentacoli occhiuti (2 Azioni). Il Panoptikhan ha CM 14.

\emph{Quello che gela}: l'occhio punta un bersaglio entro 18 metri, su questo viene attivato un \hyperlink{Raggio di Gelo}{Raggio di Gelo}. 8d8 di danno da freddo, Tiro Salvezza Riflessi DC 25 per evitare completamente il colpo.

\emph{Quello che scioglie}: l'occhio punta un bersaglio entro 9 metri, su questo viene attivato un raggio che ha effetti di acido. 4d8 di danno da acido, Tiro Salvezza Riflessi DC 25 per dimezzare il danno.

\emph{Quello che brucia}: l'occhio punta un bersaglio entro 18 metri, su questo viene attivato un raggio infuocato. 8d8 di danno da Fuoco, Tiro Salvezza Riflessi DC 25 per evitare completamente il colpo.

\emph{Quello che paralizza}: l'occhio punta un bersaglio entro 9 metri, su questo viene attivato un raggio che paralizza la creatura. Tiro Salvezza su Volontà DC 25 per evitare completamente gli effetti.

\emph{Quello che rallenta}: l'occhio punta in un cono di 9 metri. Sulle creature interessate viene proiettato un raggio che rallenta. Tiro Salvezza su Volontà DC 25 per evitare completamente gli effetti. Durata 1 minuto.

\emph{Quello che confonde}: l'occhio punta in un cono di 18 metri. Sulle creature interessate viene proiettato un raggio che causa confusione. Tiro Salvezza su Volontà DC 25 per evitare completamente gli effetti. Durata 1 minuto, ogni round è possibile effettuare un nuovo Tiro Salvezza per riprendersi dagli effetti.

\emph{Quello che addormenta}: l'occhio punta un bersaglio entro 36 metri, su questo viene attivato un raggio che addormenta la creatura. Tiro Salvezza su Volontà DC 25 per evitare completamente gli effetti.

\emph{Quello che muove}; questo occhio può manifestare l'incantesimo \hyperlink{Mano Magica}{Mano Magica} oppure Telecinesi.

\emph{\textbf{Un solo sguardo.}} Il Panoptikhan attiva l'occhio centrare. L'occhio centrale può essere usato come Azione di Reazione per lanciare Controincantesimo su un incantesimo che ha visto lanciare. DC 25.

\emph{\textbf{Arrabbiato:}} il Panoptikhan in preda alla furia più cieca attiva 1d6 occhi a caso su bersagli a caso. Costa 3 Azioni.

\textbf{Ecologia}\\
Ambiente: Qualsiasi Sotterraneo\\
Organizzazione: Solitario, coppia\\
\textbf{Categoria Tesoro}: H\\
\textbf{Descrizione}\\
I Panoptikhan sono aberrazioni xenofobe, palle di dura carne volante dotate di un grosso occhio centrale, una grande bocca e 7 tentacoli lunghi circa 1 metro ognuno dotato di un occhio (di circa 10 cm di diametro) di colore diverso.

Poco si sa dell'origine dei Panoptikhan, si pensa che siano un esperimento evoluzionario di Calicante, nel tentativo di creare una razza senziente e dominante.

Purtroppo l'arroganza, la superbia, il desiderio di essere al centro dell'attenzione hanno fatto naufragare questi tentativi di società ed i Panoptikhan si sono dispersi nel sottosuolo.

I Panoptikhan hanno una lunghissima vita, nell'ordine dei mille anni ma risultano anche creature che hanno più che raddoppiato questo limite. I Panoptikhan aumentano di taglia con l'età e così il numero di occhi. Le statistiche qui riportate sono riferite ad un esemplare di età adulta di circa 300 anni.

\mostro{Pegaso}
\noindent
\begin{description}[noitemsep, topsep=0pt, parsep=0pt, partopsep=0pt, leftmargin=0cm, labelwidth=2.2cm]
	\item[\textbf{Taglia/Tipo:}] Grande celestiale, buono
	\item[\textbf{Caratt.:}] \resizebox{0.5\linewidth+1.8cm}{!}{For 4 Des 2 Cos 3 Int 0 Sag 2 Car 1}
	\item[\textbf{Punti Ferita:}] 52,  \textbf{Difesa:} 16,  \textbf{Iniziativa:} +2
	\item[\textbf{Movimento:}] 18 m, volo 27 m
	\item[\textbf{Tiri Salvez.:}] \resizebox{0.5\linewidth+1.8cm}{!}{Tempra +5, Riflessi +4, Volontà +4}
	\item[\textbf{Comp.:}] Consapevolezza +6
	\item[\textbf{Linguaggi:}] comprende Celestiale, Comune, Elfico e Silvano ma non può parlare
	\item[\textbf{Sfida:}] 2 (450 PX)\smallskip
\end{description}

\textbf{Azioni}

\emph{\textbf{Zoccoli.} Attacco con arma da mischia}: +5 a colpire, portata 1 m, un bersaglio.

\emph{Colpisce:} 11 (2d6 + 4) danni contundenti.

\textbf{Ecologia}
Ambiente: Pianure Temperate e Calde\\
Organizzazione: Solitario, coppia o branco (6-10)\\
\textbf{Categoria Tesoro}: Nessuno\\
\textbf{Descrizione}\\
Il pegaso è un magnifico cavallo alato che a volte serve la causa del bene. Seppur molto apprezzati come cavalcature volanti, i pegasi sono creature timide che difficilmente stringono amicizie. Un tipico pegaso è alto 1,8 metri al garrese, pesa 750 kg ed ha un'apertura alare di 6 metri. La maggior parte dei pegasi è bianca, ma a volte alcuni esemplari hanno colori diversi.

Il pegaso, nonostante le apparenze, è intelligente quanto un umano. Chi cerca di addestrarne uno a fare da cavalcatura, scoprirà che il pegaso è ricalcitrante e perfino violento. Un pegaso non può parlare, ma capisce il Comune e preferisce la compagnia di creature buone. Il metodo corretto per convincere un pegaso a fare da cavalcatura è farselo amico con Diplomazia, favori e buone azioni. Un pegaso ha di norma atteggiamento indifferente verso le creature buone, maldisposto verso quelle neutrali ed ostile verso quelle malvagie. Prima che possa servire come cavalcatura, un pegaso deve essere reso amichevole tramite una prova di Diplomazia o in altro modo. Cavalcare un pegaso richiede una sella esotica o Cavalcare a pelo, dato che una sella normale interferisce con le sue ali. Un pegaso può combattere portando un cavaliere, ma il cavaliere non può attaccare a sua volta se non supera una prova di Cavalcare. I pegasi addestrati non temono il combattimento ed il cavaliere non deve effettuare una prova di Cavalcare per controllarlo.

I pegasi depongono uova che sul mercato valgono 1000 mo l'una, mentre i piccoli arrivano alle 2000 mo a testa. Essendo creature intelligenti e buone, vendere uova e piccoli è essenzialmente schiavismo: nelle società buone chi lo fa è disprezzato o punito dalla legge.

I pegasi maturano come i cavalli. Gli addestratori professionisti chiedono 1000 per addestrare un pegaso, che servirà un cavaliere buono o neutrale fedelmente per tutta la vita.

Un carico leggero per un pegaso è fino a 150 kg; un carico medio è 150,5-300 kg; un carico pesante è 300,5-450 kg.

In alcuni pegasi il sangue di un antenato che era un eroico stallone è ancora forte. Questi campioni hanno la durata della vita di un umano, manovrabilità perfetta, resistenza al fuoco 10, un bonus razziale di +4 ai Tiri Salvezza contro i Veleni e immunità alla Pietrificazione, un ulteriore +4 al Tiro per Colpire, +4 a Difesa, +25 PF, +4 a tutti i TS ed infliggono +1d6 di danno aggiuntivo. Alcuni riescono a dire poche parole in Celestiale o Comune. Si rendono conto della loro superiorità sui saurovalli, non devono essere addestrati a volare con un cavaliere, ma permettono solo ai più grandi eroi di cavalcarli.

Pegasi ed Unicorni sono stati salvati dalla furia di Calicante verso i \emph{cavalli} solo per espresse intercessione di Ljust.

\mostro{Persecutore Invisibile}
\noindent
\begin{description}[noitemsep, topsep=0pt, parsep=0pt, partopsep=0pt, leftmargin=0cm, labelwidth=2.2cm]
	\item[\textbf{Taglia/Tipo:}] Media elementale, neutrale
	\item[\textbf{Caratt.:}] \resizebox{0.5\linewidth+1.8cm}{!}{For 3 Des 4 Cos 2 Int 0 Sag 2 Car 0}
	\item[\textbf{Punti Ferita:}] 125,  \textbf{Difesa:} 24,  \textbf{Iniziativa:} +4
	\item[\textbf{Movimento:}] 15 m, volo 15 m, Fluttuare
	\item[\textbf{Tiri Salvez.:}] \resizebox{0.5\linewidth+1.8cm}{!}{\resizebox{0.5\linewidth+1.8cm}{!}{Tempra +8, Riflessi +10, Volontà +8}}
	\item[\textbf{Comp.:}] Furtività +10, Consapevolezza +8
	\item[\textbf{Res. Danni:}] da arma non magica
	\item[\textbf{Imm. Danni:}] Veleno
	\item[\textbf{Immunità:}] afferrato, intralciato, paralizzato, pietrificato, privo di sensi, prono, affaticato
	\item[\textbf{Sensi:}] Scurovisione 18 m
	\item[\textbf{Linguaggi:}] Ictun, comprende il Comune ma non lo parla
	\item[\textbf{Sfida:}] 6 (2300 PX)\smallskip
\end{description}

\emph{\textbf{Cacciatore Infallibile.}} Il convocatore assegna una preda al persecutore. Il persecutore sa la direzione e la distanza a cui si trova la preda finché entrambi si trovano sullo stesso piano di esistenza. Il persecutore conosce anche la posizione del suo convocatore.

\emph{\textbf{Invisibilità.}} Il persecutore è invisibile anche quando attacca e dopo che ha attaccato.

\emph{\textbf{Natura Elementale.}} Un persecutore invisibile non ha bisogno di aria, cibo, bevande o sonno.

\textbf{Azioni}

\emph{\textbf{Multiattacco.}} La persecutore effettua due attacchi di schianto.

\emph{\textbf{Schianto.} Attacco con arma da mischia}: +7 a colpire, portata 1 m, un bersaglio.

\emph{Colpisce:} 10 (2d6 + 3) danni contundenti.

\textbf{Reazione: \emph{Attacco d'opportunità}}: il persecutore invisibile effettua un attacco ad una creatura che attraversi o esca dalla sua portata di 1 metro.

\emph{\textbf{Arrabbiato:}} il Persecutore Invisibile rompe il patto e torna nel piano elementale dell'aria.

\textbf{Ecologia}\\
Ambiente: Qualsiasi\\
Organizzazione: Solitario\\
\textbf{Categoria Tesoro}: Nessuno\\
\textbf{Descrizione}\\
Originarie del Piano dell'Aria, queste creature si muovono nel mondo seguendo gli incarichi per coloro che le evocano. I cacciatori invisibili agiscono solitamente come guardiani e assassini. L'invisibilità naturale e la furtività permettono loro di seguire la preda senza essere visti e li avvantaggiano quando decidono di eliminare un bersaglio.

Molti cacciatori invisibili però considerano questi compiti come misere richieste da parte dei mortali. Se viene assegnato loro un compito particolarmente complesso o sgradito, un cacciatore invisibile cercherà di trovare una scappatoia se l'istruzione è formulata in modo scarno. Per esempio, i Maghi che richiamano un cacciatore invisibile con l'istruzione "proteggimi dal pericolo" potrebbero venire scortati in un lontano luogo nascosto, o addirittura portati sul Piano dell'Aria.

A causa delle continue evocazioni, molti cacciatori invisibili avversano gli abitanti del Piano Materiale. Quelli appena evocati nel mondo mortale conoscono solo le storie dei loro simili e mantengono un atteggiamento aperto nei riguardi di chi li richiama. Col tempo, o se servono un padrone malvagio, iniziano a farsi un'opinione negativa di queste creature mortali, che li porta a sviare le istruzioni e a danneggiare i loro padroni. Per i cacciatori invisibili più vecchi e con più esperienza, l'unica cosa che protegge chi li ha evocati è la magia che li lega. Queste creature tentano sempre di usare le incoerenze nella formulazione dei loro compiti e le distorsioni letterali nelle intenzioni per trovare un modo per infastidire, ferire o addirittura uccidere chi li ha portati su questo piano.

\mostro{Pseudodrago}
\noindent
\begin{description}[noitemsep, topsep=0pt, parsep=0pt, partopsep=0pt, leftmargin=0cm, labelwidth=2.2cm]
	\item[\textbf{Taglia/Tipo:}] Minuscola drago, buono
	\item[\textbf{Caratt.:}] \resizebox{0.5\linewidth+1.8cm}{!}{For -2 Des 2 Cos 1 Int 0 Sag 1 Car 0}
	\item[\textbf{Punti Ferita:}] 19,  \textbf{Difesa:} 14,  \textbf{Iniziativa:} +2
	\item[\textbf{Movimento:}] 5 metri, volo 18 m
	\item[\textbf{Tiri Salvez.:}] \resizebox{0.5\linewidth+1.8cm}{!}{Tempra +3, Riflessi +3, Volontà +3}
	\item[\textbf{Comp.:}] Furtività +4, Consapevolezza +3
	\item[\textbf{Sensi:}] Scurovisione 18 m, Vista Cieca 3 m
	\item[\textbf{Linguaggi:}] comprende il Comune e il Draconico ma non parla
	\item[\textbf{Sfida:}] 1/4 (50 PX)\smallskip
\end{description}

\emph{\textbf{Resistenza alla Magia.}} Lo pseudodrago ha +1d6 ai Tiri Salvezza contro incantesimi e altri effetti magici.

\emph{\textbf{Sensi Affinati.}} Lo pseudodrago ha +1d6 alle prove di Consapevolezza basate su vista, udito e olfatto.

\emph{\textbf{Telepatia Limitata.}} Lo pseudodrago può comunicare semplici idee, emozioni e immagini telepaticamente con qualsiasi creatura entro 30 metri da esso che può comprendere una lingua.

\textbf{Azioni}

\emph{\textbf{Morso.} Attacco con arma da mischia}: +4 a colpire, portata 1 m, un bersaglio.

\emph{Colpisce:} 4 (1d4 + 2) danni perforanti.

\emph{\textbf{Pungiglione.} Attacco con arma da mischia}: +4 a colpire, portata 1 m, una creatura.

\emph{Colpisce:} 4 (1d4 + 2) danni perforanti e il bersaglio deve riuscire un Tiro Salvezza di Tempra DC 11 o essere Confuso per 1 round. Se il Tiro Salvezza fallisce criticamente la creatura cade addormentata finché non risvegliata.

\textbf{Ecologia}\\
Ambiente: Foreste temperate\\
Organizzazione: Solitario, coppia o nido (3-5)\\
\textbf{Categoria Tesoro}: R\\
\textbf{Descrizione}\\
Gli pseudodraghi sono piccoli parenti dei veri draghi, giocosi e timidi. Parlano cinguettando, sibilando, ringhiando e facendo le fusa, ma possono comunicare telepaticamente con qualsiasi creatura intelligente. Se avvicinati pacificamente con offerte di cibo, sono disposti a condividere informazioni su quanto si trova nel loro territorio, ma minacce e violenza li fanno fuggire.

Gli pseudodraghi sono carnivori e mangiano insetti, roditori, uccellini e serpenti, anche se mangiano uova ed amano burro, formaggio e pesce. A volte cacciano a terra come le lucertole o volando come gli uccelli predatori. Intelligenti come la maggior parte degli umanoidi, non amano essere trattati come animali domestici, e preferiscono essere considerati amici. Diffidano delle creature malvagie, possono unirsi a incantatori e Devoti come Famigli e alcuni hanno stretto amicizia con Druidi e guardiaboschi o collaborano con i draghi buoni come sentinelle. Gli pseudodraghi diventano Famigli solo se apprezzano la personalità dell'incantatore (e se questi ha l'Abilità Famiglio e Carisma almeno 1), ma possono anche legarsi a persone delle quali apprezzano la compagnia. Uno pseudodrago potrebbe seguire in questo modo un personaggio per giorni, settimane, anni o perfino per tutta la vita, posto che siano ben nutriti e trattati con affetto.

Raggiunta l'età adulta, il corpo di uno pseudodrago è lungo 30 centimetri con una coda di 60 centimetri, e pesa circa 3,5 kg. Le uova di uno pseudodrago sono grandi come quelle di gallina, ma di consistenza simile al cuoio e macchiate di marrone, e le femmine le depongono in gruppi di 2-5 ogni primavera. Un nido di pseudodraghi (che costituiscono un gruppo familiare, e non sono nati dallo stesso gruppo di uova) di solito consiste di una coppia di adulti e diversi cuccioli quasi adulti.

%\addcontentsline{toc}{subsubsection}{R}
\pdfbookmark[3]{R}{R}

\mostro{Rakshasa}
\noindent
\begin{description}[noitemsep, topsep=0pt, parsep=0pt, partopsep=0pt, leftmargin=0cm, labelwidth=2.2cm]
	\item[\textbf{Taglia/Tipo:}] Media immondo, malvagio
	\item[\textbf{Caratt.:}] \resizebox{0.5\linewidth+1.8cm}{!}{For 2 Des 3 Cos 4 Int 1 Sag 3 Car 5}
	\item[\textbf{Punti Ferita:}] 259,  \textbf{Difesa:} 32,  \textbf{Iniziativa:} +3
	\item[\textbf{Movimento:}] 12 m
	\item[\textbf{Tiri Salvez.:}] \resizebox{0.5\linewidth+1.8cm}{!}{\resizebox{0.5\linewidth+1.8cm}{!}{Tempra +17, Riflessi +16, Volontà +16}}
	\item[\textbf{Comp.:}] Ingannare +10, Percepire Emozioni +8
	\item[\textbf{Imm. Danni:}] contundenti, armi +1
	\item[\textbf{Sensi:}] Scurovisione 18 m
	\item[\textbf{Linguaggi:}] Comune, Infernale
	\item[\textbf{Sfida:}] 13 (10000 PX)\smallskip
\end{description}

\emph{\textbf{Immunità alla Magia Limitata.}} Il rakshasa è immune agli affetti o all'individuazione tramite incantesimi di livello 6 o più basso a meno che non desideri esserne soggetto. Ha +1d6 ai Tiri Salvezza contro tutti gli altri incantesimi ed effetti magici.

\emph{\textbf{Incantesimi Innati.}} La caratteristica da incantatore del rakshasa il Carisma (+10 a colpire con attacchi con incantesimi). Il rakshasa può lanciare in maniera innata i seguenti incantesimi senza aver bisogno di componenti materiali:

A volontà: \emph{\hyperlink{Camuffare Sé Stesso}{Camuffare Sé Stesso}, \hyperlink{Illusione Minore}{Illusione Minore}, \hyperlink{Individuazione dei Pensieri}{Individuazione dei Pensieri}, \hyperlink{Mano Magica}{Mano Magica}}

3/Giorno ciascuno: \emph{\hyperlink{Charme su Persone}{Charme su Persone}, \hyperlink{Immagine Maggiore}{Immagine Maggiore}, \hyperlink{Individuazione del Magico}{Individuazione del Magico}, \hyperlink{Invisibilità}{Invisibilità}, \hyperlink{Suggestione}{Suggestione}} 1/Giorno: \emph{\hyperlink{Dominare Persone}{Dominare Persone}, \hyperlink{Visione del Vero}{Visione del Vero}, volare}

\textbf{Azioni}

\emph{\textbf{Multiattacco.}} Il rakshasa può effettuare due attacchi di artiglio.

\emph{\textbf{Artiglio.} Attacco con arma da mischia}: +10 a colpire, portata 1 m, un bersaglio.

\emph{Colpisce:} 9 (2d6 + 2) danni taglienti, e se il bersaglio è una creatura rimane maledetto. La maledizione magica ha effetto ogni qualvolta il bersaglio riposa, riempiendo i pensieri del bersaglio di immagini e sogni orribili. Il bersaglio maledetto non riceve beneficio dall'aver terminato un riposo. La maledizione perdura finché non viene rimossa dall'incantesimo \emph{\hyperlink{Rimuovi Maledizione}{Rimuovi Maledizione}} o simile magia.

\textbf{Reazione: \emph{Attacco d'opportunità}}: il Rakshasa effettua un attacco ad una creatura che attraversi o esca dalla sua portata di 1 metro.

\textbf{Ecologia}
Ambiente: Qualsiasi\\
Organizzazione: Solitario, coppia o culto (3-12)\\
\textbf{Categoria Tesoro}: Pugnale+1, I\\
\textbf{Descrizione}\\
Il rakshasa è uno spirito maligno che si traveste da creatura umanoide così da poter seguire la sua preda in incognito. Personificazione dei tabù della maggioranza delle società e capace di assumere l'aspetto di quelli che cerca di corrompere, un rakshasa compie moltissime azioni orribili. Se fossero umani, la loro blasfemia, il cannibalismo e gli atti ancora peggiori che compiono li marchierebbero come criminali meritevoli del più crudele degli inferni.

Quando non ha un altro aspetto, il rakshasa appare come un umanoide con la testa di un animale. Spesso ha il capo di un grosso felino (come tigri o pantere) o serpente (quali cobra o vipere) e, seppur sia più raro, può avere testa di gorilla, sciacallo, avvoltoio, elefante, mantide, lucertola, rinoceronte, cinghiale e molte altre ancora. In molti casi, il tipo di testa posseduta da un rakshasa dice qualcosa della sua personalità: un rakshasa dalla testa di tigre è furtivo e famelico, mentre uno con la testa di cinghiale può essere ghiotto e crudele. Queste differenze raramente incidono sulle statistiche base del rakshasa, anche se esistono varianti più potenti della standard con molteplici teste, poteri magici più potenti, e strane e letali capacità speciali aggiuntive.

I rakshasa disprezzano le religioni; riconoscono il potere degli dei, ma si vedono come i soli esseri degni di venerazione da parte delle razze mortali. I Devoti rakshasa sono quindi piuttosto rari. Sebbene i rakshasa siano esterni, sono anche creature del Piano Materiale, e alcuni credono che i primi rakshasa scelsero questo esilio al posto di qualche altro ruolo offertogli da un dio da tempo dimenticato. Anche se in genere sono solitari, non è raro trovare grandi famiglie di rakshasa che lavorano insieme per provocare la caduta di una civiltà mortale dall'interno, attraverso il succedersi di molte generazioni.

Un rakshasa è alto 1,8 metri e pesa 90 kg.

\mostro{Razziamorti}
\noindent
\begin{description}[noitemsep, topsep=0pt, parsep=0pt, partopsep=0pt, leftmargin=0cm, labelwidth=2.2cm]
	\item[\textbf{Taglia/Tipo:}] Grande costrutto, non morto, non allineato
	\item[\textbf{Caratt.:}] \resizebox{0.5\linewidth+1.8cm}{!}{For 5 Des 0 Cos 4 Int -4 Sag -2 Car -5}
	\item[\textbf{Punti Ferita:}] 127,  \textbf{Difesa:} 20,  \textbf{Iniziativa:} +0
	\item[\textbf{Movimento:}] 9 m
	\item[\textbf{Tiri Salvez.:}] \resizebox{0.5\linewidth+1.8cm}{!}{Tempra +10, Riflessi +6, Volontà +4}
	\item[\textbf{Imm. Danni:}] Veleno
	\item[\textbf{Immunità:}] affascinato, affaticato, paralizzato, pietrificato, sanguinamento, malattie
	\item[\textbf{Sensi:}] Scurovisione 30 m
	\item[\textbf{Linguaggi:}] comprende tutte le lingue del creatore ma non può parlare
	\item[\textbf{Sfida:}] 6 (2300 PX)\smallskip
\end{description}

\emph{\textbf{Riduzione del Danno.}} Il Razziamorti ha durezza 6/- contro armi non magiche.

\emph{\textbf{Natura Non Morta.}} Il Razziamorti non ha bisogno di aria, cibo, bevande o sonno.

\emph{\textbf{Forma Immutabile.}} Come costrutto non può essere influenzato da magie od effetti che ne cambino la forma.

\emph{\textbf{Contenitore.}} Il Razziamorti ha un comparto apribile con uno sportello sul dorso metallico che può contenere fino a 100kg di oggetti, grandi fino a taglia piccola.

\emph{\textbf{Resistenza all'Aria.}} Il Razziamorti ha una resistenza innata agli incantesimi della Lista di Magia Aria.

\emph{\textbf{Sensibile al Fuoco.}} Il Razziamorti se subisce danni da fuoco esegue una Azione in meno il round dopo.

\textbf{Azioni}

\emph{\textbf{Multiattacco.}} Il Razziamorti attacca con due chele o attacca con una chela e usa l'Occhio Paralizzante.

\textbf{\emph{Chela.}} +8 al colpire, portata 1 metro

\emph{Colpire}: 16 (2d10 + 5) di danni contundente

\emph{\textbf{Occhio Paralizzante}}: la creatura interessata, entro 18 metri, deve fare un Tiro Salvezza su Tempra a DC 18 o rimanere paralizzato per 2d4 round.

\emph{\textbf{Tartaruga triste}}: con una prova di Atletica DC 24 è possibile ribaltare sottosopra il razziamorti che non è più in grado di ribaltarsi da solo. In questa circostanza il razziamorti ha -1d6 a tutti i Tiri per Colpire.

\textbf{Ecologia}\\
Ambiente: Qualsiasi, caverne\\
Organizzazione: 1-2 Razziamorti, 1d4+1 guardiani\\
\textbf{Categoria Tesoro}: Quanto raccolto (C + R)\\
\textbf{Descrizione}\\
I Razziamorti sono dei particolari non morti costruiti da pezzi di vario cadavere e pezzi di ferro perché assomiglino a delle specie di grossi granchi corazzati.
Il dorso, completamente metallico, funge da contenitore per i tesori che il Razziamorti trova, le chele, in numero variabile tra le 6 ed 8 sono lunghe poco più di un metro ed hanno la caratteristica di lasciare ognuna una impronta diversa essendo assemblate da pezzi di metallo e corpi diversi.

Il grosso occhio centrale, forse una volta appartenuto ad un umanoide permette al controllore e costruttore del Razziamorti di vedere e comandarlo. Lo scopo di un Razziamorti é esplorare, solitamente un sistema di caverne o percorsi, alla ricerca dei resti di passati razziatori e avventurieri per carpirne gli oggetti magici e tesori.

Solitamente un Razziamorto è sempre accompagnato da diversi guardiani (altre creature al comando del controllore) che lo aiutano nel \emph{sistemare} eventuali \emph{resistenze} ancora attive.

\mostro{Remorhaz}
\noindent
\begin{description}[noitemsep, topsep=0pt, parsep=0pt, partopsep=0pt, leftmargin=0cm, labelwidth=2.2cm]
	\item[\textbf{Taglia/Tipo:}] Enorme mostruosità, disallineato
	\item[\textbf{Caratt.:}] \resizebox{0.5\linewidth+1.8cm}{!}{For 7 Des 1 Cos 5 Int -3 Sag 0 Car -3}
	\item[\textbf{Punti Ferita:}] 224,  \textbf{Difesa:} 27,  \textbf{Iniziativa:} +1
	\item[\textbf{Movimento:}] 9 m, scavo 6 m
	\item[\textbf{Tiri Salvez.:}] \resizebox{0.5\linewidth+1.8cm}{!}{\resizebox{0.5\linewidth+1.8cm}{!}{Tempra +16, Riflessi +12, Volontà +11}}
	\item[\textbf{Sensi:}] Scurovisione 18 m, senso tellurico 18 m
	\item[\textbf{Sfida:}] 11 (7200 PX)\smallskip
\end{description}

\emph{\textbf{Corpo Riscaldato.}} Una creatura che entri a contatto con il remorhaz o lo colpisca con un attacco da mischia mentre si trova entro 1 metro da esso, subisce 10 (3d6) danni da fuoco.

\textbf{Azioni}

\emph{\textbf{Morso.} Attacco in mischia con arma}: +11 a colpire, portata 3 m, un bersaglio.

\emph{Colpisce:} 40 (6d10 + 7) danni perforanti più 10 (3d6) danni da fuoco. Se il bersaglio è una creatura, è afferrato (DC 17 per fuggire). Fino al termine dell'afferrare  il remorhaz non può attaccare con il morso un altro bersaglio.

\emph{\textbf{Inghiottire.}} Il remorhaz effettua una attacco di morso contro un bersaglio di taglia Media o inferiore che sta afferrando. Se l'attacco colpisce, la creatura subisce il danno da morso ed è inghiottita, e l'afferrare ha termine. Il bersaglio inghiottito è accecato e intralciato, ha copertura completa contro gli attacchi e altri effetti all'esterno del remorhaz, e subisce 21 (6d6) danni da acido all'inizio di ciascun round del remorhaz.

Se il remorhaz subisce 30 o più danni in un singolo round da una creatura al suo interno, il remorhaz deve riuscire un Tiro Salvezza su Tempra DC 24 al termine di quel round o vomitare tutte le creature inghiottite, che cadono prone in uno spazio entro 3 metri dal remorhaz. Se il remorhaz muore, una creatura inghiottita non più intralciata da esso e può uscire dal cadavere utilizzando 2 Azioni e uscendo prona.

\emph{\textbf{Feroce.}} Come Azione il remorhaz affonda ancora più il Morso andato a segno causando 3d6 danni perforanti aggiuntivi. 1 Azione.

\emph{\textbf{Arrabbiato:}} il Remorhaz scalda ancora di più il suo corpo fino alla fine del combattimento portanto a 18 (6d6) il danno da fuoco per chi è entro 1 metro.

\textbf{Ecologia}\\
Ambiente: Deserti Freddi e Ghiacciai\\
Organizzazione: Solitario\\
\textbf{Categoria Tesoro}: Nessuno\\
\textbf{Descrizione}\\
In un mondo di ghiaccio e neve, i remorhaz sono particolarmente temuti per il terribile fuoco che brucia dentro i loro corpi. Questo fuoco interiore fa sì che le piastre lungo il suo dorso divengano roventi quando la creatura è particolarmente arrabbiata, eccitata o nel panico. Le creature che si sono adattate alle regioni artiche spesso sono particolarmente vulnerabili al fuoco, il che rende la principale difesa del remorhaz incredibilmente potente e gli assicura il ruolo di pericoloso predatore delle zone ghiacciate. I remorhaz vivono in estesi labirinti scavati nel cuore dei ghiacciai. Queste bestie usano il loro calore per scavare tunnel nel ghiaccio, tunnel le cui lisce pareti vitree si ricongelano rapidamente lungo la loro scia creando numerosi dedali incredibilmente stabili.

Intelligenti nonostante l'apparenza, i remorhaz capiscono il linguaggio dei Giganti e spesso formano alleanze con loro. I Giganti del Gelo li usano come armi contro i nemici, mentre altri giganti li sfruttano come forge viventi. Un remorhaz misura 7 metri di lunghezza e pesa 5000 kg.

\mostro{Rugginofago}
\noindent
\begin{description}[noitemsep, topsep=0pt, parsep=0pt, partopsep=0pt, leftmargin=0cm, labelwidth=2.2cm]
	\item[\textbf{Taglia/Tipo:}] Media Mostruosità, disallineato
	\item[\textbf{Caratt.:}] \resizebox{0.5\linewidth+1.8cm}{!}{For 1 Des 1 Cos 1 Int -4 Sag 1 Car -2}
	\item[\textbf{Punti Ferita:}] 24,  \textbf{Difesa:} 13,  \textbf{Iniziativa:} +1
	\item[\textbf{Movimento:}] 12 m
	\item[\textbf{Tiri Salvez.:}] \resizebox{0.5\linewidth+1.8cm}{!}{Tempra +3, Riflessi +3, Volontà +3}
	\item[\textbf{Sensi:}] Scurovisione 18 m
	\item[\textbf{Sfida:}] 1/2 (100 PX)\smallskip
\end{description}

\emph{\textbf{Fiuto del Ferro.}} Il rugginofago può individuare, con l'olfatto, l'esatta posizione di metalli ferrosi entro 36 metri.

\emph{\textbf{Arrugginire Metallo.}} Qualsiasi arma non magica fatta di metallo che colpisca il rugginofago si corrode dopo aver applicato il danno. Le munizioni non magiche fatte di metallo e che colpiscono il rugginofago, sono considerate distrutte dopo aver inflitto il danno.

\textbf{Azioni}

\emph{\textbf{Morso.} Attacco con arma da mischia}: +4 a colpire, portata 1 m, un bersaglio.

\emph{Colpisce:} 5 (1d8 + 1) danni perforanti.

\emph{\textbf{Antenne.}} Il rugginofago corrode gli oggetti di metallo ferroso non magici che può vedere e si trovano entro 1 metro. Se l'oggetto non è indossato o trasportato, il contatto col rugginofago ne distrugge un cubo di 30 centimetri di spigolo. Se l'oggetto è indossato o trasportato da una creatura, la creatura può effettuare un Tiro Salvezza su Riflessi DC 13 per evitare il contatto con il rugginofago.

Se l'oggetto con cui entra in contatto è un'armatura o scudo di metallo indossati o trasportati, questi subiscono una penalità permanente e cumulativa di -2 alla Difesa che forniscono. Le armature ridotte a Difesa 0 o gli scudi che scendono ad un bonus di +0 sono distrutti. Se l'oggetto con cui entra in contatto è un'arma di metallo impugnata da qualcuno, la arrugginisce come descritto nel tratto Arrugginire Metallo.

\textbf{Ecologia}
Ambiente: Qualsiasi Sotterraneo\\
Organizzazione: Solitario, coppia o nido (3-10)\\
\textbf{Categoria Tesoro}: Accidentale (nessun tesoro di metallo)\\
\textbf{Descrizione}\\
Di tutte le bestie terrificanti che un esploratore può incontrare nel sottosuolo, il rugginofago è l'unico a mirare al tesoro dell'avventuriero. Lungo circa un metro e pesante almeno 100 kg, il rugginofago somiglia a un crostaceo, e il suo processo nutritivo alieno lo rende ancora più spaventoso.

I rugginofagi divorano oggetti di metallo, preferendo ferro e acciaio, ma consumano anche mithral, adamantio e metalli incantati con facilità. Qualsiasi metallo toccato dalle loro antenne o dalla pelle corazzata si corrode e si riduce in polvere in pochi secondi, rendendoli temuti dagli avventurieri e dai nani minatori.

Sebbene non siano intrinsecamente violenti, la loro insaziabile fame li spinge a caricare chiunque porti abbastanza metallo, rispondendo con feroce aggressività a qualsiasi resistenza. In zone povere di metallo, possono seguire le vittime in fuga per giorni, fiutando i metalli intatti.

Fortunatamente, è spesso possibile sfuggire alle attenzioni di un rugginofago lanciandogli un oggetto di metallo denso, come uno scudo, e correndo nella direzione opposta. Chi frequenta aree infestate dai rugginofagi impara presto a portare con sé armi di legno o pietra.

%\addcontentsline{toc}{subsubsection}{S}
\pdfbookmark[3]{S}{S}

\mostro{Sahuagin}
\noindent
\begin{description}[noitemsep, topsep=0pt, parsep=0pt, partopsep=0pt, leftmargin=0cm, labelwidth=2.2cm]
	\item[\textbf{Taglia/Tipo:}] Media umanoide (sahuagin), malvagio
	\item[\textbf{Caratt.:}] \resizebox{0.5\linewidth+1.8cm}{!}{For 1 Des 0 Cos 1 Int 1 Sag 1 Car -1}
	\item[\textbf{Punti Ferita:}] 24,  \textbf{Difesa:} 12,  \textbf{Iniziativa:} +1
	\item[\textbf{Movimento:}] 9 m, nuoto 12 m
	\item[\textbf{Tiri Salvez.:}] \resizebox{0.5\linewidth+1.8cm}{!}{Tempra +3, Riflessi +3, Volontà +3}
	\item[\textbf{Comp.:}] Consapevolezza +5
	\item[\textbf{Sensi:}] Scurovisione 36 m
	\item[\textbf{Linguaggi:}] Sahuagin
	\item[\textbf{Sfida:}] 1/2 (100 PX)\smallskip
\end{description}

\emph{\textbf{Anfibio Limitato.}} Il sahuagin può respirare aria e acqua, ma deve restare sommerso almeno una volta ogni 4 ore per evitare di soffocare.

\emph{\textbf{Frenesia Sanguinaria.}} Il sahuagin ha +1d6 ai tiri per colpire in mischia contro qualsiasi creatura che non sia al massimo dei suoi Punti Ferita.

\emph{\textbf{Telepatia con gli Squali}}. Il sahuagin può comandare magicamente qualsiasi squalo entro 36 metri da sé, usando una forma limitata di telepatia.

\textbf{Azioni}

\emph{\textbf{Multiattacco.}} Il sahuagin può effettuare due attacchi da mischia: uno con il morso e uno con gli artigli o la lancia.

\emph{\textbf{Artigli.} Attacco con arma da mischia}: +4 a colpire, portata 1 m, un bersaglio.

\emph{Colpisce:} 3 (1d4 + 1) danni taglienti.

\emph{\textbf{Lancia.} Attacco con arma da mischia o a Distanza}: +4 a colpire, portata 1 m o gittata 6m, un bersaglio.

\emph{Colpisce:} 4 (1d6 + 1) danni perforanti, o 5 (1d8 + 1) danni perforanti se usata con due mani per effettuare un attacco da mischia.

\emph{\textbf{Morso.} Attacco con arma da mischia}: +4 a colpire, portata 1 m, un bersaglio.

\emph{Colpisce:} 3 (1d4 + 1) danni perforanti.

\textbf{Ecologia}\\
Ambiente: Oceani Temperati o Caldi\\
Organizzazione: Solitario, coppia, squadra (5-8), pattuglia (11-20 più 1 tenente di 3° livello e 1-2 Squali), banda (20-80 più 100\% non combattenti, 1 tenente di 3° livello e 1 capitano di 4° livello ogni 20 adulti, e 1-2 Squali) o tribù (70-160 più 100\% non combattenti, 1 tenente di 3° livello ogni 20 adulti, 1 capitano di 4° livello ogni 40 adulti, 9 guardie di 4° livello, 1-4 novizie di 3°-6° livello, 1 sacerdotessa di 7° livello, 1 barone di 6°-8° livello, e 5-8 Squali)
\textbf{Categoria Tesoro}: Equipaggiamento da PNG (Tridente, Balestra Pesante con 10 Quadrelli, L)\\
\textbf{Descrizione}\\
Famelici e crudeli, i sahuagin sono, sfortunatamente, tra le razze oceaniche più prosperose. Grandi città sono state costruite da questa razza nelle buie profondità delle fosse oceaniche, e alcune fortezze sorgono nei pressi delle coste da dove lanciano assalti continui contro i nemici che respirano aria che vivono vicino alla riva. Orgogliosi e bellicosi, i sahuagin si alleano raramente con altri, e vedono le altre razze acquatiche, come aboleth, marinidi e simili come concorrenti. Le sole creature che sembrano rispettare oltre ai loro simili sono gli squali; in questi implacabili predatori, infatti, i sahuagin rivedono molto di loro stessi. Un sahuagin è alto 2,1 metro e pesa circa 125 kg.

I sahuagin sono soggetti a mutazioni genetiche e quando nasce un mutante assurge quasi sempre ai ranghi nobiliari o di comando nella società. La mutazione sahuagin più comune consiste in un paio di braccia extra (che concedono due attacchi addizionali con gli artigli o la possibilità di maneggiare più armi). Alcuni parlano dei rari malenti sahuagin che non sembrano uomini squalo ma elfi acquatici, malgrado condividano la sete di sangue e la natura crudele dei loro simili. I malenti spesso servono come spie o assassini i governanti sahuagin, ma si narra di intere tribù composte di malenti in remote zone del mare.

\mostro{Salamandra}
\noindent
\begin{description}[noitemsep, topsep=0pt, parsep=0pt, partopsep=0pt, leftmargin=0cm, labelwidth=2.2cm]
	\item[\textbf{Taglia/Tipo:}] Grande elementale, malvagio
	\item[\textbf{Caratt.:}] \resizebox{0.5\linewidth+1.8cm}{!}{For 4 Des 2 Cos 2 Int 0 Sag 0 Car 1}
	\item[\textbf{Punti Ferita:}] 107,  \textbf{Difesa:} 20,  \textbf{Iniziativa:} +2
	\item[\textbf{Movimento:}] 9 m
	\item[\textbf{Tiri Salvez.:}] \resizebox{0.5\linewidth+1.8cm}{!}{Tempra +7, Riflessi +7, Volontà +5}
	\item[\textbf{Res. Danni:}] da arma non magica
	\item[\textbf{Sensi:}] Scurovisione 18 m
	\item[\textbf{Linguaggi:}] Ignan
	\item[\textbf{Sfida:}] 5 (1800 PX)\smallskip
\end{description}

\emph{\textbf{Armi Riscaldate.}} Qualsiasi arma da mischia metallica che la salamandra impugni infligge 3 (1d6) danni da fuoco aggiuntivi per colpo (già incluso nell'attacco).

\emph{\textbf{Corpo Riscaldato.}} Una creatura che entri a contatto con la salamandra o la colpisce con un attacco da mischia mentre si trova entro 1 metro da essa subisce 7 (2d6) danni da fuoco.

\textbf{Azioni}

\emph{\textbf{Multiattacco.}} La salamandra effettua due attacchi: uno con la lancia e uno con la coda.

\emph{\textbf{Coda.} Attacco con arma da mischia}: +6 a colpire, portata 3 m, un bersaglio.

\emph{Colpisce:} 11 (2d6 + 4) danni contundenti più 7 (2d6) danni da fuoco, e il bersaglio è afferrato (DC 14 per fuggire). Fino al termine dell'afferrare la salamandra può colpire automaticamente il bersaglio con la coda e non può effettuare attacchi di coda contro altri bersagli.

\emph{\textbf{Lancia.} Attacco con arma da mischia o a Distanza}: +5 a colpire, portata 1 m, gittata 6m, un bersaglio.

\emph{Colpisce:} 11 (2d6 + 4) danni perforanti, o 13 (2d8 +4) danni perforanti se usata con due mani per effettuare un attacco da mischia, più 3 (1d6) danni da fuoco.

\textbf{Reazione: \emph{Attacco d'opportunità}}: la salamandra effettua un attacco ad una creatura che attraversi o esca dalla sua portata di 1 metro.

\emph{\textbf{Arrabbiato:}} la Salamadra concentra le sue fiamme in un attacco a distanza. Una creatura entro 9 metri deve effettuare un Tiro Salvezza su Riflessi DC 18 per dimezzare il danno. La creatura viene colpita da un globo di fiamme che causa 4d6 di danno da fuoco. Costa 2 Azioni.

\textbf{Ecologia}
Ambiente: Qualsiasi (Piano del Fuoco)\\
Organizzazione: Solitario, coppia o gruppo (3-5)\\
\textbf{Categoria Tesoro}: Standard (Lancia, P)\\
\textbf{Descrizione}\\
Le Salamandre sono native del Piano del Fuoco, dove le loro legioni di fieri combattenti sono molto temute dagli altri abitanti del Piano. Poiché molte delle più forti Razze Elementali del Fuoco Schiavizzano le Salamandre per la loro Abilità nella metallurgia e capacità combattiva, le Salamandre odiano gli Efreet e gli altri con fervore.

Anche se i loro nascondigli superano i 250 gradi C di temperatura, le Salamandre possono tollerare temperature più basse. Generalmente lo fanno se costrette, e sono anche più burbere e irascibili del normale in questi ambienti. Sebbene provenga dal Piano del Fuoco, la Razza delle Salamandre si identifica di più con l'Abisso, e ha un grande rispetto per i Demoni (in particolare quelli associati col fuoco, come i Balor e certi Signori dei Demoni legati alle fiamme). Per questo non è insolito incontrare un grosso gruppo di Salamandre nell'Abisso.

Le Salamandre sono spesso evocate nel Piano Materiale per servire come guardiani o, più comunemente, come fabbricanti di Armature, Armi e altri oggetti metallurgici, dato che la loro Abilità in questo campo è leggendaria. Le Salamandre infestano anche quelle aree del Piano Materiale dove il confine tra questo mondo e il Piano del Fuoco si è fatto labile, come vicino e dentro i Vulcani.

Abitando zone così estreme, le Salamandre posseggono solo tesori che resistono alle alte temperature, come Spade, Armature, gioielli, Verghe e altri oggetti che hanno un alto punto di fusione. La società delle Salamandre è crudele e basata sul potere e sulla capacità di soggiogare chi è inferiore a loro. Gli esseri inferiori alle Salamandre che causano problemi affrontano una morte lenta e dolorosa.

\mostro{Satiro}
\noindent
\begin{description}[noitemsep, topsep=0pt, parsep=0pt, partopsep=0pt, leftmargin=0cm, labelwidth=2.2cm]
	\item[\textbf{Taglia/Tipo:}] Media fatato, caotico
	\item[\textbf{Caratt.:}] \resizebox{0.5\linewidth+1.8cm}{!}{For 1 Des 3 Cos 0 Int 1 Sag 0 Car 2}
	\item[\textbf{Punti Ferita:}] 24,  \textbf{Difesa:} 15,  \textbf{Iniziativa:} +3
	\item[\textbf{Movimento:}] 12 m
	\item[\textbf{Tiri Salvez.:}] \resizebox{0.5\linewidth+1.8cm}{!}{Tempra +3, Riflessi +3, Volontà +3}
	\item[\textbf{Comp.:}] Furtività +5, Intrattenere +6, Consapevolezza +2
	\item[\textbf{Linguaggi:}] Comune, Elfico, Silvano
	\item[\textbf{Sfida:}] 1/2 (100 PX)\smallskip
\end{description}

\emph{\textbf{Resistenza alla Magia.}} Il satiro ha +1d6 ai Tiri Salvezza contro incantesimi e altri effetti magici.

\textbf{Azioni}

\emph{\textbf{Incornata.} Attacco con arma da mischia}: +4 a colpire, portata 1 m, un bersaglio.

\emph{Colpisce:} 6 (2d4 + 1) danni contundenti.

\emph{\textbf{Spada Corta.} Attacco con arma da mischia}: +4 a colpire, portata 1 m, un bersaglio.

\emph{Colpisce:} 6 (1d6 + 3) danni perforanti.

\emph{\textbf{Arco Corto.} Attacco con arma a Distanza}: +3 a colpire, gittata 24m, un bersaglio.

\emph{Colpisce:} 6 (1d6 + 3) danni perforanti.

\textbf{Ecologia}\\
Ambiente: Foreste Temperate\\
Organizzazione: Solitario, coppia, banda (3-6) o festino (7-11)\\
\textbf{Categoria Tesoro}: Pugnale, Arco Corto più 20 Frecce, flauto di pan perfetto, S\\
\textbf{Descrizione}\\
I satiri, conosciuti in molte regioni come fauni, sono creature debosciate ed edoniste delle parti più profonde e primordiali delle foreste. Adorano il vino, la musica e i piaceri della carne, sono rinomati come libertini e bellimbusti che corteggiano le fanciulle sprovvedute e i pastorelli e si lasciano dietro una scia di spiegazioni imbarazzanti e gravidanze indesiderate.

Anche se i loro corpi sono quasi sempre quelli di uomini attraenti e ben proporzionati, le capacità seduttive dei satiri risiedono nel loro talento musicale. Con l'aiuto del suo flauto, un satiro è capace di tessere una vasta gamma di incantesimi melodici ideati per affascinare gli altri e farli accondiscendere ai suoi capricciosi desideri.

Oltre ad amoreggiare costantemente, i satiri spesso fungono da guardiani delle loro foreste, e quanti riescono a trasformare la lussuria del fauno in ira probabilmente si troveranno di fronte i più pericolosi tra gli animali che circondano il fauno. Inoltre, anche se i satiri tendono a mettere il loro divertimento al di sopra dei diritti altrui, non covano alcun risentimento contro quelli che seducono.

I bambini nati da questi incontri sono sempre satiri di sangue puro e vengono generalmente portati via dai loro sfrenati padri subito dopo la nascita.

\mostro{Scheletro}
\noindent
\begin{description}[noitemsep, topsep=0pt, parsep=0pt, partopsep=0pt, leftmargin=0cm, labelwidth=2.2cm]
	\item[\textbf{Taglia/Tipo:}] Media non morto, malvagio
	\item[\textbf{Caratt.:}] \resizebox{0.5\linewidth+1.8cm}{!}{For 0 Des 2 Cos 2 Int -2 Sag -1 Car -3}
	\item[\textbf{Punti Ferita:}] 19,  \textbf{Difesa:} 14,  \textbf{Iniziativa:} +2
	\item[\textbf{Movimento:}] 9 m
	\item[\textbf{Tiri Salvez.:}] \resizebox{0.5\linewidth+1.8cm}{!}{Tempra +3, Riflessi +3, Volontà +3}
	\item[\textbf{Res. Danni:}] perforante, tagliente
	\item[\textbf{Imm. Danni:}] Veleno
	\item[\textbf{Immunità:}] affaticato, sanguinamento
	\item[\textbf{Sensi:}] Scurovisione 18 m
	\item[\textbf{Linguaggi:}] comprende tutte le lingue che parlava in vita ma non può parlare
	\item[\textbf{Sfida:}] 1/4 (50 PX)\smallskip
\end{description}

\emph{\textbf{Natura Non Morta.}} Lo scheletro non necessita aria, cibo, bevande o sonno.

\emph{\textbf{Spada Corta.} Attacco con arma da mischia}: +4 a colpire, portata 1 m, un bersaglio.

\emph{Colpisce:} 5 (1d6 + 2) danni perforanti.

\emph{\textbf{Arco Corto.} Attacco con arma a Distanza}: +4 a colpire, gittata 24m, un bersaglio.\emph{Colpisce:} 5 (1d6 + 2) danni perforanti.

\textbf{Ecologia}\\
Ambiente: Qualsiasi\\
Organizzazione: Qualsiasi\\
\textbf{Categoria Tesoro}: Nessuno (Giaco di Maglia Rotto, Scimitarra Rotta)\\
\textbf{Descrizione}\\
Gli scheletri sono ossa di morti animate, portate alla non vita da magie sacrileghe. Per la maggior parte, gli scheletri sono automi privi di volontà.

\mostro{Scheletro Campione}
\noindent
\begin{description}[noitemsep, topsep=0pt, parsep=0pt, partopsep=0pt, leftmargin=0cm, labelwidth=2.2cm]
	\item[\textbf{Taglia/Tipo:}] Media non morto, malvagio
	\item[\textbf{Caratt.:}] \resizebox{0.5\linewidth+1.8cm}{!}{For 4 Des 1 Cos 3 Int -2 Sag -1 Car -3}
	\item[\textbf{Punti Ferita:}] 70,  \textbf{Difesa:} 17,  \textbf{Iniziativa:} +1
	\item[\textbf{Movimento:}] 12 m
	\item[\textbf{Tiri Salvez.:}] \resizebox{0.5\linewidth+1.8cm}{!}{Tempra +6, Riflessi +4, Volontà +3}
	\item[\textbf{Imm. Danni:}] Veleno
	\item[\textbf{Res. Danni:}] perforante, tagliente, Elettricità, Fuoco
	\item[\textbf{Immunità:}] affaticato, sanguinamento
	\item[\textbf{Sensi:}] Scurovisione 18 m
	\item[\textbf{Linguaggi:}] comprende l'Expiran, ma non può parlare
	\item[\textbf{Sfida:}] 3 (700 PX)\smallskip
\end{description}

\emph{\textbf{Natura Non Morta.}} Lo scheletro non necessita aria, cibo, bevande o sonno.

\textbf{Azioni}

\emph{\textbf{Ascia Bipenne.} Attacco con arma da mischia}: +6 a colpire, portata 1 m, un bersaglio.

\emph{Colpisce:} 12 (1d12 + 4) danni taglienti.

\mostro{Scheletro di Saurovallo da Guerra}
\noindent
\begin{description}[noitemsep, topsep=0pt, parsep=0pt, partopsep=0pt, leftmargin=0cm, labelwidth=2.2cm]
	\item[\textbf{Taglia/Tipo:}] Grande non morto, malvagio
	\item[\textbf{Caratt.:}] \resizebox{0.5\linewidth+1.8cm}{!}{For 4 Des 1 Cos 2 Int -4 Sag -1 Car -3}
	\item[\textbf{Punti Ferita:}] 24,  \textbf{Difesa:} 13,  \textbf{Iniziativa:} +1
	\item[\textbf{Movimento:}] 18 m
	\item[\textbf{Tiri Salvez.:}] \resizebox{0.5\linewidth+1.8cm}{!}{Tempra +3, Riflessi +3, Volontà +3}
	\item[\textbf{Res. Danni:}] perforante, tagliente
	\item[\textbf{Imm. Danni:}] Veleno
	\item[\textbf{Immunità:}] affaticato, sanguinamento
	\item[\textbf{Sensi:}] Scurovisione 18 m
	\item[\textbf{Sfida:}] 1/2 (100 PX)\smallskip
\end{description}

\emph{\textbf{Natura Non Morta.}} Lo scheletro non necessita aria, cibo, bevande o sonno.

\textbf{Azioni}

\emph{\textbf{Zoccoli.} Attacco con arma da mischia}: +5 a colpire, portata 1 m, un bersaglio.

\emph{Colpisce:} 11 (2d6 + 4) danni contundenti.

\mostro{Segugio Infernale}
\noindent
\begin{description}[noitemsep, topsep=0pt, parsep=0pt, partopsep=0pt, leftmargin=0cm, labelwidth=2.2cm]
	\item[\textbf{Taglia/Tipo:}] Media immondo, malvagio
	\item[\textbf{Caratt.:}] \resizebox{0.5\linewidth+1.8cm}{!}{For 3 Des 1 Cos 2 Int -2 Sag 1 Car -2}
	\item[\textbf{Punti Ferita:}] 70,  \textbf{Difesa:} 17,  \textbf{Iniziativa:} +1
	\item[\textbf{Movimento:}] 15 m
	\item[\textbf{Tiri Salvez.:}] \resizebox{0.5\linewidth+1.8cm}{!}{Tempra +5, Riflessi +4, Volontà +4}
	\item[\textbf{Comp.:}] Consapevolezza +5
	\item[\textbf{Imm. Danni:}] Fuoco
	\item[\textbf{Sensi:}] Scurovisione 18 m
	\item[\textbf{Linguaggi:}] comprende l'Infernale ma non può parlare
	\item[\textbf{Sfida:}] 3 (700 PX)\smallskip
\end{description}

\emph{\textbf{Udito e Olfatto Affinato.}} Il segugio ha +1d6 nelle prove di Consapevolezza basate su udito od olfatto.

\emph{\textbf{Tattiche di Branco.}} Il segugio ha +1d6 ai tiri per colpire contro una creatura se almeno uno degli alleati del segugio si trova entro 1 metro dalla creatura e quell'alleato non è inabile.

\textbf{Azioni}

\emph{\textbf{Morso.} Attacco con arma da mischia}: +6 a colpire, portata 1 m, un bersaglio.

\emph{Colpisce:} 7 (1d6 + 3) danni perforanti più 7 (2d6) danni da fuoco.

\emph{\textbf{Soffio Infuocato (Ricarica 5-6).}} Il segugio esala fuoco in un cono di 5 metri. Ogni creatura in quell'area deve effettuare un Tiro Salvezza di Riflessi DC 14, e subire 21 (6d6) danni da fuoco se fallisce il Tiro Salvezza, o la metà di questi danni se lo riesce.

\mostro{Androsfinge}
\noindent
\begin{description}[noitemsep, topsep=0pt, parsep=0pt, partopsep=0pt, leftmargin=0cm, labelwidth=2.2cm]
	\item[\textbf{Taglia/Tipo:}] Grande mostruosità, legale
	\item[\textbf{Caratt.:}] \resizebox{0.5\linewidth+1.8cm}{!}{For 6 Des 0 Cos 5 Int 3 Sag 4 Car 6}
	\item[\textbf{Punti Ferita:}] 338,  \textbf{Difesa:} 34,  \textbf{Iniziativa:} +3
	\item[\textbf{Movimento:}] 12 m, volo 18 m
	\item[\textbf{Tiri Salvez.:}] \resizebox{0.5\linewidth+1.8cm}{!}{\resizebox{0.5\linewidth+1.8cm}{!}{Tempra +22, Riflessi +17, Volontà +21}}
	\item[\textbf{Comp.:}] Arcana +9, Religione +15
	\item[\textbf{Imm. Danni:}] da arma non magica
	\item[\textbf{Immunità:}] affascinato, spaventato
	\item[\textbf{Sensi:}] visione del vero 36 m
	\item[\textbf{Linguaggi:}] Comune, Sfinge
	\item[\textbf{Sfida:}] 17 (18000 PX)\smallskip
\end{description}

\emph{\textbf{Armi Magiche.}} Gli attacchi con armi della sfinge sono magici.

\emph{\textbf{Imperscrutabile.}} La sfinge è immune a qualsiasi effetto in grado di percepirne le emozioni o leggerne i pensieri oltre che a qualsiasi incantesimo di divinazione che rifiuti. Le prove Percepire Emozioni per discernere le intenzioni o la sincerità della sfinge hanno -1d6.


\begin{center}
	\includegraphics[width=0.5\textwidth]{immagini/ginosfinge.png}
\end{center}


\emph{\textbf{Incantesimi.}} La sfinge ha CM 12.
La sua caratteristica da incantatore è la Saggezza (DC del Tiro Salvezza degli incantesimi 30, +10 a colpire con attacchi con incantesimo). Non ha bisogno di componenti materiali per lanciare i suoi incantesimi. La sfinge tiene preparati i seguenti incantesimi:

Trucchetti (a volontà): \emph{\hyperlink{Fiamma Sacra}{Fiamma Sacra}, \hyperlink{Taumaturgia}{Taumaturgia}}

livello 1 (4 slot): \emph{\hyperlink{Comando}{Comando}, \hyperlink{Individuazione del Magico}{Individuazione del Magico}, \hyperlink{Conoscere i Tratti}{Conoscere i Tratti}}

livello 2 (3 slot): \emph{\hyperlink{Ristorare Inferiore}{Ristorare Inferiore}, \hyperlink{Zona di Verità}{Zona di Verità}}

livello 3 (3 slot): \emph{\hyperlink{Dissolvi Magie}{Dissolvi Magie}, \hyperlink{Lingue}{Lingue}}

livello 4 (3 slot): \emph{\hyperlink{Esilio}{Esilio}, \hyperlink{Libertà di Movimento}{Libertà di Movimento}}

livello 5 (2 slot): \emph{\hyperlink{Colpo Infuocato}{Colpo Infuocato}, \hyperlink{Ristorare Superiore}{Ristorare Superiore}}

livello 6 (1 slot): \emph{\hyperlink{Banchetto degli Eroi}{Banchetto degli Eroi}}

\textbf{Azioni}

\emph{\textbf{Multiattacco.}} La sfinge può effettuare due attacchi di artiglio.

\emph{\textbf{Artiglio.} Attacco con arma da mischia}: +13 a colpire, portata 1 m, un bersaglio.

\emph{Colpisce:} 17 (2d6 + 10) danni taglienti, 1 danno da Sanguinamento.

\emph{\textbf{Ruggito (3/Giorno).}} La sfinge emette un ruggito magico. Ogni volta che ruggisce prima di una nuova alba il ruggito è più forte e l'effetto è diverso, come dettagliato di seguito. Ogni creatura entro 150 metri dalla sfinge e capace di udirne il ruggito deve effettuare un Tiro Salvezza.

\textbf{Primo Ruggito.} Ogni creatura che fallisce un Tiro Salvezza su Volontà DC 30 resta spaventata per 1 minuto. Una creatura spaventata può ripetere il Tiro Salvezza al termine di ciascun suo round, terminandone l'effetto per sé, se lo riesce.

\textbf{Secondo Ruggito.} Ogni creatura che fallisce un Tiro Salvezza su Volontà DC 30 resta assordata e spaventata per 1 minuto. Una creatura spaventata è paralizzata e può ripetere il Tiro Salvezza al termine di ciascun suo round, terminandone l'effetto per sé, se lo riesce.

\textbf{Terzo Ruggito.} Ogni creatura effettua un Tiro Salvezza su Tempra DC 30. Chi fallisce il Tiro Salvezza subisce 44 (8d10) danni da suono ed è gettato prono. Se il Tiro Salvezza riesce, la creatura subisce la metà di questi danni e non viene gettata prona.

\textbf{Reazione: \emph{Attacco d'opportunità}}: la sfinge nero effettua un attacco con Artiglio ad una creatura che attraversi o esca dalla sua portata di 1 metro.

\textbf{Azioni Aggiuntive}

La sfinge può effettuare 3 Azioni aggiuntive, scelte tra le opzioni seguenti. Può usare solo un'opzione Aggiuntiva alla volta e solo al termine del round di un'altra creatura. La sfinge recupera le Azioni aggiuntive spese all'inizio del proprio round.

\textbf{Attacco di Artiglio.} La sfinge effettua un attacco di artiglio.

\textbf{Eseguire un Incantesimo (Costa 3 Azioni).} La sfinge lancia un incantesimo dalla lista degli incantesimi preparati, utilizzando uno slot incantesimo come di norma.

\textbf{Teletrasporto (Costa 2 Azioni).} La sfinge si teletrasporta magicamente, insieme a tutto l'equipaggiamento che sta indossando o trasportando, in uno spazio non occupato che possa vedere, fino a 36 metri di distanza.

\emph{\textbf{Arrabbiato:}} la Sfinge pone un indovinello. La creatura deve rispondere, usando tutte le sue azioni ed una risposta a round, entro 6 round, se sbaglia o non risponde deve effettuare un Tiro Salvezza su Volontà a DC 31 oppure rimanere paralizzata. Ogni round può tentare di nuovo il Tiro Salvezza nel tentativo di dare una risposta. Costa 1 Azione.

\textbf{Ecologia}\\
Ambiente: Colline o Deserti Caldi\\
Organizzazione: Solitario\\
\textbf{Categoria Tesoro}: C\\
\textbf{Descrizione}\\
Le androsfingi, le più potenti tra le sfingi comuni, ritengono di rappresentare tutto ciò che c'è di degno e nobile nella loro specie e si atteggiano come se il peso del mondo intero poggiasse sul loro buon esempio. Guardano le Criosfingi con sufficienza paternalistica, le Ieracosfingi con malcelato disgusto e le Ginosfingi come le uniche altre sfingi degne del loro tempo.

Le androsfingi ostentano una facciata scorbutica e astiosa nei confronti degli stranieri. Non si sforzano in alcun modo di celare il loro fastidio quando sono irritate. Tendono inoltre a essere gelose del loro territorio, anche se meno delle altre sfingi. Quasi inevitabilmente lanciano avvertimenti e proclami roboanti prima di attaccare, e rispettano quasi sempre un appello a trattare. Le androsfingi barattano informazioni e conversazioni, e non tesori, in cambio di un passaggio sicuro.

Le androsfingi sono alte 3,6 metri e pesano 500 kg.

\mostro{Ginosfinge}
\noindent
\begin{description}[noitemsep, topsep=0pt, parsep=0pt, partopsep=0pt, leftmargin=0cm, labelwidth=2.2cm]
	\item[\textbf{Taglia/Tipo:}] Grande mostruosità, legale
	\item[\textbf{Caratt.:}] \resizebox{0.5\linewidth+1.8cm}{!}{For 4 Des 2 Cos 3 Int 4 Sag 4 Car 4}
	\item[\textbf{Punti Ferita:}] 219,  \textbf{Difesa:} 28,  \textbf{Iniziativa:} +4
	\item[\textbf{Movimento:}] 12 m, volo 18 m
	\item[\textbf{Tiri Salvez.:}] \resizebox{0.5\linewidth+1.8cm}{!}{\resizebox{0.5\linewidth+1.8cm}{!}{Tempra +14, Riflessi +13, Volontà +15}}
	\item[\textbf{Comp.:}] Arcana +14, Religione +9, Storia +14
	\item[\textbf{Res. Danni:}] da arma non magica
	\item[\textbf{Immunità:}] affascinato, spaventato
	\item[\textbf{Sensi:}] visione del vero 36 m
	\item[\textbf{Linguaggi:}] Comune, Sfinge
	\item[\textbf{Sfida:}] 11 (7200 PX)\smallskip
\end{description}

\emph{\textbf{Armi Magiche.}} Gli attacchi con armi della sfinge sono magici.

\emph{\textbf{Imperscrutabile.}} La sfinge è immune a qualsiasi effetto in grado di percepirne le emozioni o leggerne i pensieri, oltre che a qualsiasi incantesimo di divinazione che rifiuti. Le prove di Saggezza (Percepire Inganni) per discernere le intenzioni o la sincerità della sfinge hanno -1d6.

\emph{\textbf{Incantesimi.}} La sfinge ha CM 9. La sua abilità da incantatore è l'Intelligenza (DC del Tiro Salvezza degli incantesimi 24. Non ha bisogno di componenti materiali per eseguire i suoi incantesimi. La sfinge tiene preparati i seguenti incantesimi: Trucchetti (a volontà): \emph{\hyperlink{Illusione Minore}{Illusione Minore}, \hyperlink{Mano Magica}{Mano Magica},} \emph{\hyperlink{Prestidigitazione}{Prestidigitazione}}

livello 1 (4 slot): \emph{\hyperlink{Identificare}{Identificare}, \hyperlink{Individuazione del Magico}{Individuazione del Magico}, \hyperlink{Scudo}{Scudo}}

livello 2 (3 slot): \emph{\hyperlink{Localizza Oggetto}{Localizza Oggetto}, \hyperlink{Oscurità}{Oscurità}, \hyperlink{Suggestione}{Suggestione}}

livello 3 (3 slot): \emph{\hyperlink{Dissolvi Magie}{Dissolvi Magie}, \hyperlink{Lingue}{Lingue}, \hyperlink{Rimuovi Maledizione}{Rimuovi Maledizione}}

livello 4 (3 slot): \emph{\hyperlink{Esilio}{Esilio}, \hyperlink{Invisibilità Superiore}{Invisibilità Superiore}}

livello 5 (2 slot): \emph{\hyperlink{Conoscenza delle Leggende}{Conoscenza delle Leggende}}

\textbf{Azioni}

\emph{\textbf{Multiattacco.}} La sfinge può effettuare due attacchi di artiglio.

\emph{\textbf{Artiglio.} Attacco con arma da mischia}: +10 a colpire, portata 1 m, un bersaglio.

\emph{Colpisce:} 13 (2d8 + 4) danni taglienti, 1 danno da Sanguinamento.

\textbf{Reazione: \emph{Attacco d'opportunità}}: la sfinge nero effettua un attacco con Artiglio ad una creatura che attraversi o esca dalla sua portata di 1 metro.

\textbf{Azioni Aggiuntive}

La sfinge può effettuare 3 Azioni aggiuntive, scelte tra le opzioni seguenti. Può usare solo un'opzione Aggiuntiva alla volta e solo al termine del round di un'altra creatura. La sfinge recupera le Azioni aggiuntive spese all'inizio del proprio round.

\textbf{Attacco di Artiglio.} La sfinge effettua un attacco di artiglio.

\textbf{Eseguire un Incantesimo (Costa 3 Azioni).} La sfinge esegue un incantesimo dalla lista degli incantesimi preparati, utilizzando uno slot incantesimo come di norma.

\textbf{Teletrasporto (Costa 2 Azioni).} La sfinge si teletrasporta magicamente, insieme a tutto l'equipaggiamento che sta indossando o trasportando, in uno spazio non occupato che possa vedere, fino a 36 metri di distanza.

\textbf{Ecologia}
Ambiente: Deserti e colline caldi\\
Organizzazione: Solitario, coppia o culto (3-6)\\
\textbf{Categoria Tesoro}: C\\
\textbf{Descrizione}\\
Anche se esistono diversi tipi di sfinge, quella alla quale gli studiosi si riferiscono come Ginosfinge (un nome che molte sfingi trovano offensivo) è una creatura saggia e maestosa ma al contempo terrificante se arrabbiata. Meno moraliste delle loro controparti maschili (le Androsfingi, creature totalmente differenti da quella presentata qui), le sfingi sono prudenti e metodiche quando prendono delle decisioni, e sono orgogliose della loro fredda logica e della loro imparzialità.

Hanno poca pazienza con le varianti inferiori di sfingi, considerandole poco più che animali.

Le sfingi amano gli enigmi e gli indovinelli complicati, e fanno tesoro di fatti insoliti e dilemmi arcani molto più che di oro o gemme.

Pur non essendo grandi studiose in senso tradizionale, il grande apprezzamento delle sfingi per gli enigmi le porta a compiere ricerche in una grande varietà di materie, rendendole spesso una preziosa fonte di informazioni, specialmente quando fanno uso delle loro capacità magiche. Di solito sono felici di avere contatti con altre razze, ed offrono regolarmente beni materiali in cambio di informazioni o di indovinelli nuovi ed interessanti. Sono eccellenti guardiane di templi, tombe ed altri luoghi importanti, fintanto che vengono intrattenute in maniera adeguata. Le sfingi danno grande importanza alla gentilezza, ma possono essere capricciose: possono decidere altruisticamente di dividere i loro ultimi enigmi con dei viaggiatori ma non ci pensano due volte a divorarli se non vi prestano abbastanza attenzione o non forniscono alcun indizio utile alla loro risoluzione.

Una tipica sfinge è lunga 3 metri e pesa circa 400 kg. Anche se le loro ali possono tenerle in aria per lunghi periodi di tempo, sono delle volatrici scarse, e preferiscono atterrare prima di iniziare a combattere, attaccando con i loro poderosi artigli. Nonostante siano estremamente territoriali, le sfingi tendono ad avvisare gli intrusi varie volte prima di attaccare.

\mostro{Sibilante}
\noindent
\begin{description}[noitemsep, topsep=0pt, parsep=0pt, partopsep=0pt, leftmargin=0cm, labelwidth=2.2cm]
	\item[\textbf{Taglia/Tipo:}] Grande mostruosità, caotico
	\item[\textbf{Caratt.:}] \resizebox{0.5\linewidth+1.8cm}{!}{For 2 Des 1 Cos 1 Int -3 Sag 0 Car -2}
	\item[\textbf{Punti Ferita:}] 51,  \textbf{Difesa:} 15,  \textbf{Iniziativa:} +1
	\item[\textbf{Movimento:}] 6 m, arrampicarsi 6 m
	\item[\textbf{Tiri Salvez.:}] \resizebox{0.5\linewidth+1.8cm}{!}{Tempra +3, Riflessi +3, Volontà +3}
	\item[\textbf{Comp.:}] Furtività +4, Consapevolezza +3
	\item[\textbf{Sensi:}] Scurovisione 18 m
	\item[\textbf{Sfida:}] 2 (450 PX)\smallskip
\end{description}

\textbf{Azioni}

\emph{\textbf{Multiattacco.}} Il Sibilante può eseguire due attacchi con gli artigli oppure un colpo con la coda.

\emph{\textbf{Artiglio.} Attacco con arma da mischia}: +5 a colpire, portata 1 m, un bersaglio.

\emph{Colpisce:} 6 (1d8+2) danno tagliente.

\emph{\textbf{Frustata di Coda}}: il Sibilante agita la lunga coda, +5 al colpire, portata 3 metri, un bersaglio.

\emph{Colpisce:} 11 (2d8+2) danni contundenti e 7 (2d6) da taglio. L'eventuale armatura o scudo viene danneggiata abbassando di 1 la Difesa dell'avversario. Il danno all'armatura non si considera permanente.

\textbf{Reazioni}

\textbf{Reazione: \emph{Atterrare}}: quando il Sibilante è attaccato da una creatura nella portata della sua coda questa viene sferzata obbligando l'attaccante, dopo la risoluzione del suo attacco, ad effettuare un Tiro Salvezza su Tempra o Riflessi a DC 14 o subire 7 (2d6) di danni contundenti e cadere prono. Se il Tiro Salvezza riesce subisco solo metà danno e non è prona.

\textbf{Ecologia}\\
Ambiente: Caverne\\
Organizzazione: Solitario, coppia o nido (2-4)\\
\textbf{Categoria Tesoro}: Accidentale\\
\textbf{Descrizione}\\
I Sibilanti, chiamati così per via del rumore che fa la loro coda agitandosi è una creatura molto particolare. Assomiglia a prima vista ad un coccodrillo, lungo circa 5 metri di cui 4 di coda ma ha 8 zampe ed il muso corto e appiattito. La coda estremamente robusta finisce con una specie di uncino che il Sibilante usa per colpire, uccidere ed afferrare i nemici quasi fosse una zampa aggiuntiva.

Di colore grigio scuro, marrone, preferiscono nascondersi nell'oscurità ed attaccare quando affamati o per difendere il loro territorio. Cercano di tenere le distanze in combattimento e se gravemente feriti scappano arrampicandosi sulle pareti.

\mostro{Spettro}
\noindent
\begin{description}[noitemsep, topsep=0pt, parsep=0pt, partopsep=0pt, leftmargin=0cm, labelwidth=2.2cm]
	\item[\textbf{Taglia/Tipo:}] Media non morto, malvagio
	\item[\textbf{Caratt.:}] \resizebox{0.5\linewidth+1.8cm}{!}{For -5 Des 2 Cos 0 Int 0 Sag 0 Car 2}
	\item[\textbf{Punti Ferita:}] 33,  \textbf{Difesa:} 15,  \textbf{Iniziativa:} +2
	\item[\textbf{Movimento:}] 0 m, volo 16 m, fluttuare
	\item[\textbf{Tiri Salvez.:}] \resizebox{0.5\linewidth+1.8cm}{!}{Tempra +3, Riflessi +3, Volontà +3}
	\item[\textbf{Comp.:}] Furtività +8,Consapevolezza +3
	\item[\textbf{Res. Danni:}] Acido, Freddo, Fuoco, Elettricità, Suono, da Vuoto, da arma non magica
	\item[\textbf{Immunità:}] affascinato, spaventato, affaticato, afferrato, paralizzato, pietrificato, veleno, prono, ristretto
	\item[\textbf{Sensi:}] Scurovisione 18 m
	\item[\textbf{Lingue:}] Expiran
	\item[\textbf{Sfida:}] 1 (200 PX)\smallskip
\end{description}

\textbf{Movimento Incorporeo}. Lo spettro può muoversi attraverso creature ed oggetti come se fosse terreno difficile. Subisce 5 (1d10) danni se termina il suo turno all'interno di un oggetto.

\textbf{Sensibilità alla luce solare}. Mentre è illuminato a luce solare lo spettro ha -1d6 ai Tiri per Colpire e alle prove di Consapevolezza.

\textbf{Azioni}

\emph{\textbf{Tocco gelido.}} Attacco a contatto: +4 a colpire, portata 1 m, un bersaglio.

\emph{Colpisce:} 10 danni (3d6) da Vuoto. La creatura perde il medesimo ammontare dai Punti Ferita Massimi. Lo spettro recupera 2 Punti Ferita.

\textbf{Ecologia}\\
Ambiente: Qualsiasi\\
Organizzazione: Solitario\\
\textbf{Categoria Tesoro}: nessuno\\
\textbf{Descrizione}\\
Gli spettri sono non morti malvagi che odiano la luce del sole e gli esseri viventi. La loro genesi è spesso dovuta alla morte violenta di assassini e malvagi. Come i fantasmi, gli spettri infestano i posti dove sono morti e cercano di portare altri con loro.

Uno spettro assomiglia molto a come era in vita e può essere facilmente riconosciuto da coloro che conoscevano l’individuo o ne avevano visto il volto nei dipinti o nei disegni.

\mostro{Spiritello}
\noindent
\begin{description}[noitemsep, topsep=0pt, parsep=0pt, partopsep=0pt, leftmargin=0cm, labelwidth=2.2cm]
	\item[\textbf{Taglia/Tipo:}] Minuscola fatato, buono
	\item[\textbf{Caratt.:}] \resizebox{0.5\linewidth+1.8cm}{!}{For -4 Des 4 Cos 0 Int 2 Sag 1 Car 0}
	\item[\textbf{Punti Ferita:}] 19,  \textbf{Difesa:} 16,  \textbf{Iniziativa:} +4
	\item[\textbf{Movimento:}] 3 m, volo 12 m
	\item[\textbf{Tiri Salvez.:}] \resizebox{0.5\linewidth+1.8cm}{!}{Tempra +3, Riflessi +4, Volontà +3}
	\item[\textbf{Comp.:}] Furtività +8 (la prova è fatta con -1d6 se lo spiritello sta volando),Consapevolezza +3
	\item[\textbf{Linguaggi:}] Comune, Elfico, Silvano
	\item[\textbf{Sfida:}] 1/4 (50 PX)\smallskip
\end{description}

\textbf{Azioni}

\emph{\textbf{Spada Lunga.} Attacco con arma da mischia}: +3 a colpire,

portata 1 m, un bersaglio.

\emph{Colpisce:} 1 danno tagliente.

\emph{\textbf{Arco Corto.} Attacco con arma a Distanza}: +5 a colpire, gittata 12 m, un bersaglio.

\emph{Colpisce:} 1 danno perforante. Se il bersaglio è una creatura, deve riuscire un Tiro Salvezza di Tempra DC 10 o restare avvelenata, -1 Forza e Destrezza, per 1 minuto. Se il risultato di questo Tiro Salvezza è 5 o meno, il bersaglio cade privo di sensi per la stessa durata, o finché subisce danni o un'altra creatura usa un'Azione per risvegliarlo.

\emph{\textbf{Invisibilità.}} Lo spiritello resta invisibile finché non attacca o termina la sua concentrazione. Qualsiasi cosa che lo spiritello stia trasportando o indossando resta invisibile finché rimane in contatto con lo spiritello.

\emph{\textbf{Vista del Cuore.}} Lo spiritello entra in contatto con una creatura e ne apprende l'attuale stato emotivo. Se il bersaglio fallisce un Tiro Salvezza di Tempra DC 10, lo spiritello apprende anche i Tratti della creatura. Celestiali, immondi e non morti falliscono automaticamente questo Tiro Salvezza.
\textbf{Descrizione}\\
Gli spiritelli si riuniscono in gruppi nelle profondità di regioni boschive, uniti nella causa per proteggere la natura. Intere tribù di spiritelli si sono dichiarate protettrici di una determinata persona, di un luogo o di una creatura di particolare rilievo nelle loro terre, anche nel caso in cui l'essere non desideri o non necessiti di alcuna protezione.

Il corpo di uno spiritello è luminoso per natura, sebbene la creatura possa variare il colore e l'intensità della luce emessa dal suo corpo a piacimento. Subito dopo la sua morte, il corpo di uno spiritello si dissolve in una nebbia luccicante. Gli spiritelli sono i più piccoli tra i folletti, alti poco più di 22 centimetri e di un peso che raramente supera 1 kg.

Sotto molti aspetti gli spiritelli sono più primitivi della maggior parte dei folletti. Apprezzano la compagnia dei propri simili, ma tendono a diffidare degli altri folletti e presumono che qualsiasi umanoide o creatura che non hanno espressamente scelto di proteggere voglia far loro del male. Persino gli animali vengono da loro solitamente considerati pericolosi. La ragione di questa diffidenza è per buona parte dovuta alla taglia minuscola di queste creature, che le rende facili prede per i predatori. Pertanto la reazione iniziale di uno spiritello di fronte a un pericolo è darsi alla fuga: in genere utilizza le sue capacità magiche per rallentare o distrarre gli inseguitori, e in seguito si affida alla sua velocità di volare e alla sua taglia per riuscire a fuggire.

Sebbene gli spiritelli di per sé abbiano una natura incolta e selvaggia, hanno una sana curiosità per tutte le cose dotate di magia innata. Sono particolarmente attratti dai luoghi di grande potere magico latente, quali le rovine di antichi templi. Questa curiosità li rende anche insolitamente adatti al ruolo di famigli. Un incantatore caotico di 5° livello può ottenere uno spiritello come famiglio se ha l'Abilità Famiglio.

\mostro{Strige (Uccello Stigeo)}
\noindent
\begin{description}[noitemsep, topsep=0pt, parsep=0pt, partopsep=0pt, leftmargin=0cm, labelwidth=2.2cm]
	\item[\textbf{Taglia/Tipo:}] Minuscola bestia, disallineato
	\item[\textbf{Caratt.:}] \resizebox{0.5\linewidth+1.8cm}{!}{For -3 Des 3 Cos 0 Int -4 Sag -1 Car -2}
	\item[\textbf{Punti Ferita:}] 17,  \textbf{Difesa:} 15,  \textbf{Iniziativa:} +3
	\item[\textbf{Movimento:}] 3 m, volo 12 m
	\item[\textbf{Tiri Salvez.:}] \resizebox{0.5\linewidth+1.8cm}{!}{Tempra +3, Riflessi +3, Volontà +3}
	\item[\textbf{Sensi:}] Scurovisione 18 m
	\item[\textbf{Sfida:}] 1/8 (25 PX)\smallskip
\end{description}

\textbf{Azioni}

\emph{\textbf{Risucchio di Sangue.} Attacco con arma da mischia}: +3 a colpire, portata 1 m, una creatura.

\emph{Colpisce:} 5 (1d4 + 3) danni perforanti e lo strige si attacca al bersaglio. Mentre è attaccato, lo strige non attacca. Invece, all'inizio di ciascun round dello strige, il bersaglio perde 5 (1d4 + 3) Punti Ferita a causa della perdita di sangue.

Lo strige può staccarsi spendendo 1 Azione. Lo fa automaticamente dopo aver risucchiato 10 Punti Ferita dal bersaglio o alla morte del bersaglio. Una creatura, compreso il bersaglio, può usare una Azione per staccare lo strige.

\textbf{Ecologia}
Ambiente: Paludi temperate e calde\\
Organizzazione: Solitario, colonia (2-4), stormo (5-8), nugolo (9-14) o sciame (15-40)\\
\textbf{Categoria Tesoro}: Nessuno\\
\textbf{Descrizione}\\
Gli strige sono pericolosi succhiasangue che infestano le paludi e predano animali selvatici, bestiame ed ignari viaggiatori. Pur essendo deboli individualmente, sciami di queste creature sono capaci di prosciugare un uomo in pochi minuti, lasciando dietro a loro solo un cadavere essiccato.

più simili ai mammiferi che agli insetti, gli strige si alzano in volo con le loro quattro ali di carne, cercando prede a sangue caldo. Spesso si nascondono vicino a pozze di acqua bevibile aspettando che i viaggiatori abbassino la guardia per poi attaccarli e bere a sazietà, conficcando le loro proboscidi nelle vene scoperte. Dopo essersi nutriti, volano via a nascondersi tra la fanghiglia e tra i canneti per deporre le loro uova e riposare finché la fame non li spinge a cacciare di nuovo.

Di solito gli strige sono lunghi circa 30 centimetri, con un'apertura alare di circa il doppio, e pesano meno di 0,5 kg. Sono color rosso ruggine o marrone rossastro, ed hanno il ventre color giallo sporco, ma quelli che non si sono nutriti adeguatamente sono di colore rosa pallido.

\pdfbookmark[3]{T}{T}

\mostro{Tarrasque}
\noindent
\begin{description}[noitemsep, topsep=0pt, parsep=0pt, partopsep=0pt, leftmargin=0cm, labelwidth=2.2cm]
	\item[\textbf{Taglia/Tipo:}] Colossale mostruosità, disallineato
	\item[\textbf{Caratt.:}] \resizebox{0.5\linewidth+1.8cm}{!}{For 10 Des 0 Cos 10 Int -2 Sag 0 Car 0}
	\item[\textbf{Punti Ferita:}] 615,  \textbf{Difesa:} 52,  \textbf{Iniziativa:} +0
	\item[\textbf{Movimento:}] 24 m
	\item[\textbf{Tiri Salvez.:}] \resizebox{0.5\linewidth+1.8cm}{!}{\resizebox{0.5\linewidth+1.8cm}{!}{Tempra +40, Riflessi +30, Volontà +30}}
	\item[\textbf{Imm. Danni:}] Fuoco, Veleno, Elettricità; armi +2
	\item[\textbf{Immunità:}] affascinato, paralizzato, spaventato, affaticato
	\item[\textbf{Sensi:}] Vista Cieca 36 m
	\item[\textbf{Sfida:}] 30 (155000 PX)\smallskip
\end{description}

\emph{\textbf{Carapace Riflettente.}} Ogni volta che il Tarrasque è il bersaglio di un incantesimo \emph{\hyperlink{Dardo arcano}{Dardo arcano} o \hyperlink{Fulmine}{Fulmine}} questo viene ignorato e riflesso sull'origine. Per altri incantesimi a linea, o un incantesimi che richiedono un tiro per colpire a gittata, tira un d6. Da 1 a 5, il Tarrasque lo ignora. Con 6, il Tarrasque lo ignora e l'effetto viene riflesso contro l'incantatore come se fosse originato dal Tarrasque, trasformando l'incantatore nel bersaglio.

\emph{\textbf{Mostro d'Assedio.}} Il Tarrasque infligge danni doppi agli oggetti e le strutture.

\emph{\textbf{Resistenza Leggendaria (3/Giorno).}} Se il Tarrasque fallisce un Tiro Salvezza, può scegliere invece di riuscire.

\emph{\textbf{Resistenza alla Magia.}} Il Tarrasque ha +1d6 ai Tiri Salvezza contro incantesimi o altri effetti magici.

\emph{\textbf{Rigenerazione.}} Il Tarrasque rigenera 10 Punti Ferita all'inizio del suo round.

\textbf{Azioni}

\emph{\textbf{Multiattacco.}} Il Tarrasque può usare la sua Presenza Spaventosa. Poi effettua cinque attacchi: uno con il morso, due con gli artigli, uno con le corna, e uno con la coda. Al posto del morso può usare Inghiottire. Gli attacchi del Tarrasque sono considerati magici +4.

\emph{\textbf{Artiglio.} Attacco con arma da mischia}: +20 a colpire, portata 5 metri, un bersaglio.

\emph{Colpisce:} 28 (4d8 + 10) danni taglienti, 3 danno da Sanguinamento.

\emph{\textbf{Coda.} Attacco con arma da mischia}: +20 a colpire, portata 6 m, un bersaglio.

\emph{Colpisce:} 24 (4d6 + 10) danni contundenti. Se il bersaglio è una creatura, deve riuscire un Tiro Salvezza di Tempra DC 40 o cadere prona.

\emph{\textbf{Corna.} Attacco con arma da mischia}: +20 a colpire, portata 3 m, un bersaglio.

\emph{Colpisce:} 32 (4d10 + 10) danni perforanti.

\emph{\textbf{Morso.} Attacco con arma da mischia}: +20 a colpire, portata 3 m, un bersaglio.

\emph{Colpisce:} 36 (4d12 + 10) danni perforanti. Se il bersaglio è una creatura, è afferrata (DC 20 per fuggire). Fino al termine dell'afferrare il Tarrasque non può usare il morso contro un altro bersaglio.

\emph{\textbf{Inghiottire.}} Il Tarrasque effettua una attacco di morso contro un bersaglio di taglia Grande o inferiore che sta afferrando. Se l'attacco colpisce, il bersaglio è inghiottito, e l'afferrare ha termine. Il bersaglio inghiottito è accecato e intralciato, ha copertura completa contro gli attacchi e altri effetti all'esterno del Tarrasque, e subisce 56 (16d6) danni da acido all'inizio di ciascun round del Tarrasque.

Se il Tarrasque subisce 60 o più danni in un singolo round da una creatura al suo interno, il Tarrasque deve riuscire un Tiro Salvezza su Tempra DC 30 al termine di quel round o vomitare tutte le creature inghiottite, che cadono prone in uno spazio entro 3 metri dal Tarrasque. Se il Tarrasque muore, una creatura inghiottita non più intralciata da esso e può uscire dal cadavere utilizzando 2 Azioni e uscendo prona.

\emph{\textbf{Presenza Spaventosa.}} Ogni creatura scelta dal Tarrasque, che si trovi entro 36 metri da esso e consapevole della sua presenza, deve riuscire un Tiro Salvezza di Volontà DC 40 o restare spaventata per 1 minuto. Una creatura può ripetere il Tiro Salvezza al termine di ciascun suo round, con -1d6 se il Tarrasque è in linea di visuale, terminando l'effetto per sé, se lo riesce. Se il Tiro Salvezza della creatura ha successo o l'effetto ha termine per essa, la creatura è immune alla Presenza Spaventosa del Tarrasque per le successive 24 ore.

\textbf{Azioni Aggiuntive}

Il Tarrasque può effettuare 3 Azioni aggiuntive, scelte tra le opzioni seguenti. Può usare solo un'opzione Aggiuntiva alla volta e solo al termine del round di un'altra creatura. Il tarrasque recupera le azioni aggiuntive spese all'inizio del proprio round.

\textbf{Attacco.} Il Tarrasque effettua un attacco di artiglio o di coda.

\textbf{Masticare (Costa 2 Azioni).} Il Tarrasque effettua un attacco di morso o usa Inghiottire.

\textbf{Muoversi.} Il Tarrasque si muove fino a metà del suo movimento.

\textbf{Ecologia}\\
Ambiente: Qualsiasi\\
Organizzazione: Solitario\\
\textbf{Categoria Tesoro}: Nessuno\\
\textbf{Descrizione}\\
Il leggendario Tarrasque è fra i mostri più distruttivi del mondo. Fortunatamente, passa la maggior parte del suo tempo in una specie di profondo letargo in una sconosciuta caverna in un remoto angolo del mondo. Quando si risveglia, però, muoiono interi regni.

Pur non particolarmente intelligente, il Tarrasque è abbastanza intelligente da capire alcune parole nel linguaggio dei Patroni (pur non potendo parlare). Allo stesso modo, la furia non è incontrollata: si concentra sulla creatura che l'ha danneggiato maggiormente ed è difficile distrarlo con l'inganno.

La leggenda dice che il Tarrasque sia l'animale da compagnia di Cattalm.

\mostro{Teschio Fiammeggiante}
\noindent
\begin{description}[noitemsep, topsep=0pt, parsep=0pt, partopsep=0pt, leftmargin=0cm, labelwidth=2.2cm]
	\item[\textbf{Taglia/Tipo:}] Piccolo non morto, Tratti malvagi
	\item[\textbf{Caratt.:}] \resizebox{0.5\linewidth+1.8cm}{!}{For 0 Des 1 Cos 1 Int 1 Sag 0 Car 0}
	\item[\textbf{Punti Ferita:}] 51,  \textbf{Difesa:} 15,  \textbf{Iniziativa:} +1
	\item[\textbf{Movimento:}] volo 10 m
	\item[\textbf{Tiri Salvez.:}] \resizebox{0.5\linewidth+1.8cm}{!}{Tempra +3, Riflessi +3, Volontà +3}
	\item[\textbf{Res. Danni:}] da Vuoto
	\item[\textbf{Imm. Danni:}] Fuoco, Veleno, da arma non magica
	\item[\textbf{Immunità:}] affascinato, paralizzato, affaticato, spaventato, sanguinamento
	\item[\textbf{Sensi:}] Scurovisione 18 m
	\item[\textbf{Sfida:}] 2 (200 PX)\smallskip
\end{description}

\emph{\textbf{Incantesimi.}} Un Teschio Fiammeggiante può eseguire i seguenti incantesimi in maniera innata.

a Volontà: \emph{\hyperlink{Produrre Fiamma}{Produrre Fiamma}}

1 volta al giorno: \emph{Gragnola di Ghiande Infuocate di Kyrin}

\emph{\textbf{Natura Non Morta.}} Il Teschio Fiammeggiante non ha bisogno di aria, cibo, bevande o sonno.

\emph{\textbf{Esperto del fuoco.}} Costa 1 Azione lanciare il trucchetto \hyperlink{Produrre Fiamma}{Produrre Fiamma}.

\textbf{Ecologia}\\
Ambiente: Qualsiasi\\
Organizzazione: Solitario, paio, pattuglia (2d4)\\
\textbf{Categoria Tesoro}: nessuno\\
\textbf{Descrizione}:\\
I Teschi Fiammeggianti sono creati dai cadaveri degli incantatori specializzati nella Lista di magia del Fuoco e della necromanzia.

Usati come custodi e torce rappresentano spesso una prima linea di difesa nei dungeon.

\mostro{Testuggine Dragona}
\noindent
\begin{description}[noitemsep, topsep=0pt, parsep=0pt, partopsep=0pt, leftmargin=0cm, labelwidth=2.2cm]
	\item[\textbf{Taglia/Tipo:}] Mastodontica drago, neutrale
	\item[\textbf{Caratt.:}] \resizebox{0.5\linewidth+1.8cm}{!}{For 7 Des 0 Cos 5 Int 0 Sag 1 Car 1}
	\item[\textbf{Punti Ferita:}] 338,  \textbf{Difesa:} 34,  \textbf{Iniziativa:} +0
	\item[\textbf{Movimento:}] 6 m, nuoto 12 m
	\item[\textbf{Tiri Salvez.:}] \resizebox{0.5\linewidth+1.8cm}{!}{\resizebox{0.5\linewidth+1.8cm}{!}{Tempra +22, Riflessi +17, Volontà +18}}
	\item[\textbf{Sensi:}] Scurovisione 18 m
	\item[\textbf{Linguaggi:}] Aquan, Draconico
	\item[\textbf{Sfida:}] 17 (18000 PX)\smallskip
\end{description}

\emph{\textbf{Anfibio.}} La testuggine dragona può respirare aria e acqua.

\textbf{Azioni}

\emph{\textbf{Multiattacco.}} Il drago può effettuare tre attacchi: uno con il morso e due con gli artigli. Può effettuare un attacco di coda al posto di due attacchi di artiglio.

\emph{\textbf{Artiglio.} Attacco con arma da mischia}: +13 a colpire, portata 3 m, un bersaglio.

\emph{Colpisce:} 16 (2d8 + 7) danni taglienti.

\emph{\textbf{Coda.} Attacco con arma da mischia}: +13 a colpire, portata 5 metri, un bersaglio.
\emph{Colpisce:} 26 (3d12 + 7) danni contundenti. Se il bersaglio è una creatura, deve riuscire un Tiro Salvezza di Tempra DC 31 o venire spinta di 3 metri lontano dalla testuggine dragona e cadere prona.

\emph{\textbf{Morso.} Attacco con arma da mischia}: +13 a colpire, portata 5 metri, un bersaglio.

\emph{Colpisce:} 26 (3d12 + 7) danni perforanti.

\emph{\textbf{Salto e Schiaccio.} Attacco con arma da mischia}: +12 a colpire, portata 9 metri, fino a 6 creature in 6x6m di area. 2 Azioni.

\emph{Colpisce:} 40 (6d12 + 4) danni contundenti

\emph{\textbf{Soffio di Vapore (Ricarica 5-6).}} La testuggine dragona esala un vapore caldo in un cono di 18 metri. Ogni creatura in quell'area deve effettuare un Tiro Salvezza di Tempra DC 31 e subire 52 (15d6) danni da fuoco se fallisce il Tiro Salvezza, o la metà di questi danni se lo riesce. Trovarsi sott'acqua non dà resistenza contro questo tipo di danno.

\textbf{Ecologia}
Ambiente: Acquatico temperato\\
Organizzazione: Solitario\\
\textbf{Categoria Tesoro}: A\\
\textbf{Descrizione}\\
Le testuggini dragone sono creature delle acque dolci e salate, molto temute dai marinai. Sono noti per aspettarsi offerte in oro e magia dai marinai per un passaggio sicuro. Ignorare una testuggine dragona può renderla molto pericolosa.

Il loro guscio varia di colore, da marrone e rosso ruggine a verde-blu con riflessi argentei. Le testuggini dragone capovolgono le navi che violano il loro territorio, accumulando ricchezze nei loro nascondigli subacquei. Vivono in caverne profonde e difendono aggressivamente il loro territorio, spesso in conflitto con altre razze sottomarine.

Si nutrono di grandi pesci e alghe marine, e non disdegnano i passeggeri delle navi affondate. I loro gusci possono raggiungere i 5 metri di diametro, con una lunghezza totale di 7 metri

\mostro{Topi, La}
\noindent
\begin{description}[noitemsep, topsep=0pt, parsep=0pt, partopsep=0pt, leftmargin=0cm, labelwidth=2.2cm]
	\item[\textbf{Taglia/Tipo:}] Minuscola fatata, indifferente. Patrono
	\item[\textbf{Caratt.:}] \resizebox{0.5\linewidth+1.8cm}{!}{For -1 Des 4 Cos 0 Int 6 Sag 1 Car 10}
	\item[\textbf{Punti Ferita:}] 15,  \textbf{Difesa:} 16,  \textbf{Iniziativa:} +6
	\item[\textbf{Movimento:}] 6 m, volare 18 m, fluttuare
	\item[\textbf{Tiri Salvez.:}] \resizebox{0.5\linewidth+1.8cm}{!}{\resizebox{0.5\linewidth+1.8cm}{!}{Tempra +30, Riflessi +34, Volontà +30}}
	\item[\textbf{Comp.:}] tutte +20
	\item[\textbf{Immunità:}] al danno delle armi con bonus magico inferiore a +6
	\item[\textbf{Immunità:}] a qualsiasi effetto, danno, condizione non faccia piacere alla Topi
	\item[\textbf{Immunità:}] a qualsiasi magia la Topi non voglia essere influenzata
	\item[\textbf{Immunità:}] a subire a qualsiasi tipo di tiro critico
	\item[\textbf{Sensi:}] Senso tellurico 60, Scurovisione 60 m, Visione del Vero 60 m
	\item[\textbf{Linguaggi:}] tutti
	\item[\textbf{Sfida:}] 0 (10 PX)\smallskip
\end{description}


\begin{center}
	\includegraphics[width=0.4\textwidth]{immagini/mice.png}

	\centering
	\emph{La Topi}
\end{center}


\emph{\textbf{E' La Topi}} La Topi ha +3d6 (oppure +18) ogni volta che deve tirare dei dadi o contare un valore.

\emph{\textbf{Ha ragione!}} Qualsiasi attacco effettuato dalla Topi è considerato magico +6 e non è resistibile o rigenerabile o curabile nelle prime 24 ore.

\smallskip

\textbf{Azioni}\\
\emph{\textbf{Musetto}} ogni bersaglio a scelta di Topi, entro 30 metri, subisce un Musetto. Il bersaglio viene allontanato di 2d6 metri e subisce 3d6 danni\\
\emph{\textbf{Morso topetto} Attacco con Arma da Mischia}: +26 al colpire, portata 0 m, un bersaglio.\\
\emph{Colpisce:} 6 danno perforante.\\
\emph{\textbf{Graffiotto} fino a 4 Attacchi con Arma da Mischia}: colpisce  automaticamente, portata 0 m.\\
\emph{Colpisce:} 1 danno perforante.\\
\emph{\textbf{Appoggia nasino} una creatura}. La Topi appoggia il nasino sulla creatura scelta e questa viene guarita da ogni malattia, condizione o ferita in essere.\\
\emph{\textbf{Arrabbiato:}} la Topi fa quello che vuole (Desiderio illimitato). Costo 1 Reazione.\\
\textbf{Azioni Aggiuntive}\\
La Topi, come Patrono, può fare quante Azioni aggiuntive vuole tra tutte quelle segnate. Può usare Desiderio illimitato una volta a round.\\
\textbf{Ecologia}\\
Ambiente: Ovunque, Mercati\\
Organizzazione: Solitario\\
\textbf{Categoria Tesoro}: Speciale\\
\textbf{Descrizione}\\
Potrebbe essere scambiata per una piccola topina bianca, ma La Topi è molto di più. Furba, intelligente, bellissima adora andare al mercato e comprare borsette.

\mostro{Torciascura}
\noindent
\begin{description}[noitemsep, topsep=0pt, parsep=0pt, partopsep=0pt, leftmargin=0cm, labelwidth=2.2cm]
	\item[\textbf{Taglia/Tipo:}] Media non morto, malvagio
	\item[\textbf{Caratt.:}] \resizebox{0.5\linewidth+1.8cm}{!}{For 3 Des 1 Cos 2 Int 0 Sag -1 Car -2}
	\item[\textbf{Punti Ferita:}] 88,  \textbf{Difesa:} 18,  \textbf{Iniziativa:} +1
	\item[\textbf{Movimento:}] 6 m
	\item[\textbf{Tiri Salvez.:}] \resizebox{0.5\linewidth+1.8cm}{!}{Tempra +6, Riflessi +5, Volontà +3}
	\item[\textbf{Res. Danni:}] da Vuoto; da arma non magica o che non sia argentata
	\item[\textbf{Imm. Danni:}] Veleno
	\item[\textbf{Immunità:}] affaticato, sanguinamento
	\item[\textbf{Sensi:}] Scurovisione, vede nell'oscurità magica
	\item[\textbf{Linguaggi:}] Comprende il Comune, ma non parla
	\item[\textbf{Sfida:}] 4 (1100 PX)\smallskip
\end{description}

\emph{\textbf{Invisibile al buio.}} Un Torciascura è completamente invisibile finché è nell'oscurità\\
\emph{\textbf{Natura Non Morta.}} Torciascura non ha bisogno di aria, cibo, bevande o sonno.\\
\emph{\textbf{Sensibilità alla Luce}}. Mentre è alla luce del sole, Torciascura ha -1d6 ai tiri di attacco\\
\textbf{Multiattacco}\\
\emph{\textbf{Attacco}} Torciascura attacca due volte con la sua torcia oppure esegue Urlo di Tristezza\\
\emph{\textbf{Torcia}} Attacco di mischia, +6 al colpire\\
\emph{\textbf{Colpisce}} 7 (1d6+3) di danni contundenti, lancia l'incantesimo Oscurità sull'obiettivo colpito, durata fino alla distruzione del Torciascura\\
\emph{\textbf{Urlo di Tristezza}} cono di 6 metri. Le creature colpite devono effettuare un Tiro Salvezza su Volontà DC 16 o cadere in una triste disperazione che conferisce -2 al Tiro per Colpire, -2 al danno in mischia.\\
\textbf{Ecologia}\\
Ambiente: Dungeon\\
Organizzazione: Solitario, gruppo 2d4\\
\textbf{Categoria Tesoro}: Speciale\\
\textbf{Descrizione}\\
Un Torciascura era un avventuriero, come voi, morto in preda al terrore dopo che l'ultima torcia si spense. Un Torciascura è un non morto, solitamente umanoide, dall'aspetto vagamente indefinito, che brandisce una torcia che emana pura oscurità. Il suo scopo è uccidere nuovi avventurieri avvolgendoli nell'oscurità eterna.

Solitamente il Torciascura si nasconde nell'oscurità aspettando di toccare l'avversario ed avvolgerlo nella sua maledizione. Una creatura uccisa da un Torciascura torna in vita come Torciascura dopo 1d3 giorni.

Un Torciascura quando viene distrutto lascia a terra la sua torcia. Questa torcia, di pura oscurità può lanciare l'incantesimo Oscurità a tocco tre volte al giorno, fuori dalle mani di un Torciascura se esposta alla luce del sole si distrugge in 2d4 round.

\mostro{Troll}
\noindent
\begin{description}[noitemsep, topsep=0pt, parsep=0pt, partopsep=0pt, leftmargin=0cm, labelwidth=2.2cm]
	\item[\textbf{Taglia/Tipo:}] Grande gigante, malvagio
	\item[\textbf{Caratt.:}] \resizebox{0.5\linewidth+1.8cm}{!}{For 5 Des 1 Cos 5 Int -2 Sag -1 Car -2}
	\item[\textbf{Punti Ferita:}] 110,  \textbf{Difesa:} 19,  \textbf{Iniziativa:} +1
	\item[\textbf{Movimento:}] 9 m
	\item[\textbf{Tiri Salvez.:}] \resizebox{0.5\linewidth+1.8cm}{!}{Tempra +10, Riflessi +6, Volontà +4}
	\item[\textbf{Sensi:}] Scurovisione 18 m
	\item[\textbf{Linguaggi:}] Gigante
	\item[\textbf{Sfida:}] 5 (1800 PX)\smallskip
\end{description}

\emph{\textbf{Olfatto Affinato.}} Il troll ha +1d6 alle prove di Consapevolezza basate sull'olfatto.

\emph{\textbf{Rigenerazione.}} Il troll recupera 10 Punti Ferita all'inizio del suo round. Se il troll subisce danno da acido o da fuoco, questo tratto non funziona all'inizio del prossimo round del troll. Il troll muore solo se inizia il suo round ha meno di -5 Punti Ferita e non può rigenerarsi.

\textbf{Azioni}

\emph{\textbf{Multiattacco.}} Il troll può effettuare tre attacchi: uno con il morso e due con gli artigli.

\emph{\textbf{Artiglio.} Attacco con arma da mischia}: +8 a colpire, portata 1 m, un bersaglio.

\emph{Colpisce:} 12 (2d6 + 5) danni taglienti, 1 danno da Sanguinamento.

\emph{\textbf{Morso.} Attacco con arma da mischia}: +7 a colpire, portata 1 m, un bersaglio.

\emph{Colpisce:} 8 (1d6 + 8) danni perforanti.

\textbf{Ecologia}\\
Ambiente: Montagne Fredde\\
Organizzazione: Solitario o banda (2-4)\\
\textbf{Categoria Tesoro}: B\\
\textbf{Descrizione}\\
I troll possiedono artigli affilati ed incredibili capacità rigenerative che permettono loro di guarire quasi tutte le ferite. Sono gobbi, brutti ma fortissimi: combinata con i loro artigli, la loro forza gli permette di lacerare la carne a mani nude. I troll sono alti circa 3 metri, ma la loro postura li fa apparire più bassi. Un troll adulto pesa circa 500 kg.

L'appetito di un troll e le sue capacità rigenerative lo rendono un combattente indomito, che carica a testa bassa la creatura vivente più vicina ed attacca con tutta la sua furia. Solo il fuoco fa esitare un troll, ma perfino quello che per lui è un pericolo mortale non ferma la sua avanzata. Chi affronta i troll sa di dover localizzare e bruciare qualsiasi sua parte dopo un combattimento, perché perfino dal brandello più piccolo del suo corpo, con il tempo può rinascere un troll completo. Fortunatamente, solo le parti più grandi di un troll, come gli arti, ricrescono in questo modo.

Nonostante la loro ferocia, i troll sono straordinariamente teneri e gentili verso i loro piccoli. I troll femmina lavorano in gruppo, passando molto tempo ad insegnare ai cuccioli come cacciare e difendersi prima di mandarli a cercare un proprio territorio. Un troll maschio vive un'esistenza solitaria, incontrando brevemente le femmine solo per accoppiarsi. Tutti i troll trascorrono il loro tempo a cercare cibo, dato che devono consumarne enormi quantità ogni giorno o muoiono di fame. Per questo, la maggior parte dei troll si crea un proprio territorio di caccia che viene spesso difeso combattendo con i rivali. Simili scontri sono di solito non letali, ma i troll conoscono bene le proprie debolezze, sfruttandole per uccidere l'avversario nei periodi di magra.

E' universalmente conosciuto che i troll possono naturalmente mutare acquisendo per brevi periodi le caratteristiche più peculiari delle creature di cui si nutrono. Non avete idea di quanto può essere buffo un Pegasutroll...

%\addcontentsline{toc}{subsubsection}{U}
\pdfbookmark[3]{U}{U}

\mostro{Uomo Acquatico}
\noindent
\begin{description}[noitemsep, topsep=0pt, parsep=0pt, partopsep=0pt, leftmargin=0cm, labelwidth=2.2cm]
	\item[\textbf{Taglia/Tipo:}] Media umanoide (uomo acquatico), neutrale
	\item[\textbf{Caratt.:}] \resizebox{0.5\linewidth+1.8cm}{!}{For 0 Des 1 Cos 1 Int 0 Sag 0 Car 1}
	\item[\textbf{Punti Ferita:}] 17,  \textbf{Difesa:} 13,  \textbf{Iniziativa:} +1
	\item[\textbf{Movimento:}] 3 m, nuoto 12 m
	\item[\textbf{Tiri Salvez.:}] \resizebox{0.5\linewidth+1.8cm}{!}{Tempra +3, Riflessi +3, Volontà +3}
	\item[\textbf{Comp.:}] Consapevolezza +2
	\item[\textbf{Linguaggi:}] Aquan, Comune
	\item[\textbf{Sfida:}] 1/8 (25 PX)\smallskip
\end{description}

\emph{\textbf{Anfibio.}} L'uomo acquatico può respirare aria e acqua.

\textbf{Azioni}

\emph{\textbf{Lancia.} Attacco con arma da mischia o a Distanza}: +3 a colpire, portata 1 m o gittata 6m, un bersaglio.

\emph{Colpisce:} 3 (1d6) danni perforanti, o 4 (1d8) danni perforanti se usata con due mani per effettuare un attacco da mischia.

\textbf{Ecologia}\\
Ambiente: Oceani temperati\\
Organizzazione: Solitario, pattuglia (2-6), banda (6-10 più un tenete di 3° livello, compagnia (11-60 più 3 tenenti di 3° livello, 2 comandanti di 5° livello, 1 commodoro di 7° livello e 3-12 Calamari\\
\textbf{Categoria Tesoro}: Equipaggiamento da PNG (Tridente, Balestra Leggera con 10 Quadrelli, N)\\
\textbf{Descrizione}\\
Fisicamente, gli Uomini Pesce somigliano ai loro antenati, con fronti espressive, pelle pallida, capelli scuri e occhi porpora. Hanno tre sottili branchie sul collo, ma possono passare per Umani per brevi periodi, se lo desiderano.

\mostro{Uomo Albero (Arborom)}
\noindent
\begin{description}[noitemsep, topsep=0pt, parsep=0pt, partopsep=0pt, leftmargin=0cm, labelwidth=2.2cm]
	\item[\textbf{Taglia/Tipo:}] Enorme pianta, buono
	\item[\textbf{Caratt.:}] \resizebox{0.5\linewidth+1.8cm}{!}{For 6 Des -1 Cos 5 Int 1 Sag 3 Car 1}
	\item[\textbf{Punti Ferita:}] 186,  \textbf{Difesa:} 23,  \textbf{Iniziativa:} +1
	\item[\textbf{Movimento:}] 9 m
	\item[\textbf{Tiri Salvez.:}] \resizebox{0.5\linewidth+1.8cm}{!}{\resizebox{0.5\linewidth+1.8cm}{!}{Tempra +14, Riflessi +8, Volontà +12}}
	\item[\textbf{Res. Danni:}] contundente, perforante
	\item[\textbf{Linguaggi:}] Comune, Druidico, Elfico, Silvano
	\item[\textbf{Sfida:}] 9 (5000 PX)\smallskip
\end{description}

\emph{\textbf{Falso Aspetto.}} Mentre l'uomo albero rimane immobile, è indistinguibile da un normale albero.

\emph{\textbf{Mostro d'Assedio.}} L'uomo albero infligge danni doppi agli oggetti e le strutture.

\textbf{Azioni}

\emph{\textbf{Multiattacco.}} L'uomo albero effettua due attacchi di schianto.

\emph{\textbf{Schianto.} Attacco con arma da mischia}: +11 a colpire, portata 2 m, un bersaglio.

\emph{Colpisce:} 16 (3d6 + 6) danni contundenti.

\emph{\textbf{Sasso.} Attacco con arma a Distanza}: +10 a colpire, gittata 18m, un bersaglio.

\emph{Colpisce:} 28 (4d10 + 6) danni contundenti.

\textbf{Reazione: \emph{Attacco d'opportunità}}: l'uomo albero effettua un attacco di schianto ad una creatura che attraversi o esca dalla sua portata di 2 metri.

\emph{\textbf{Animare Alberi (1/Giorno).}} L'uomo albero anima magicamente uno o due alberi visibili entro 18 metri da lui. Questi alberi hanno le stesse statistiche dell'Arborom, eccetto che hanno punteggio di Intelligenza e Carisma -3, non possono parlare, e hanno solo l'opzione di attacco Schianto. Un albero animato agisce come alleato dell'uomo albero. L'albero resta per 1 giorno o finché muore; finché l'uomo albero muore o si trova più di 36 metri lontano dall'albero, o finché l'uomo albero non effettua una Reazione per ritrasformarlo in un albero inanimato. Poi l'albero prenderà radici, se possibile.

\textbf{Ecologia}\\
Ambiente: Qualsiasi foresta\\
Organizzazione: Solitario o macchia (2-7)\\
\textbf{Categoria Tesoro}: J\\
\textbf{Descrizione}\\
I Arborom sono guardiani delle foreste ed ambasciatori degli alberi. Antichi quanto le foreste stesse, si vedono come genitori e pastori piuttosto che giardinieri: sono lenti e metodici, ma terrificanti quando costretti a combattere per difendere il loro gregge. Anche se raramente cercano la compagnia delle razze dalla vita breve ed hanno un'innata sfiducia verso i cambiamenti, mostrano tolleranza verso chi desidera imparare dai loro lunghi, lenti monologhi, specialmente coloro nei cui occhi leggono il desiderio di proteggere le regioni selvagge. Contro coloro che minacciano le loro foreste, specialmente i boscaioli che raccolgono legna o coloro che vorrebbero disboscare una foresta per costruire una strada o un forte, la rabbia dei Arborom si scatena rapida e devastante. Sono in grado di demolire ciò che gli altri costruiscono: un tratto che li aiuta durante i loro eccessi di furia.

I Arborom sono principalmente creature solitarie, ed un singolo individuo è spesso responsabile di un'intera foresta, ma a volte si raccolgono in gruppi detti boschetti per scambiarsi le ultime notizie e riprodursi.

In tempi di grave pericolo, tutti i boschetti di una regione si uniscono per una riunione della durata di mesi detta concilio, ma simili eventi sono molto rari, e fra i concili passano anche millenni.

Un tipico Arborom è alto 9 metri, con un tronco del diametro di 60 centimetri, e pesa circa 2.250 kg. I Arborom somigliano agli alberi più comuni dei territori dove vivono.

Gli Arborom si dice che siano creati per volere di Efrem.

\mostro{Uomo Magma}
\noindent
\begin{description}[noitemsep, topsep=0pt, parsep=0pt, partopsep=0pt, leftmargin=0cm, labelwidth=2.2cm]
	\item[\textbf{Taglia/Tipo:}] Piccola elementale, caotico
	\item[\textbf{Caratt.:}] \resizebox{0.5\linewidth+1.8cm}{!}{For -2 Des 2 Cos 1 Int -1 Sag 0 Car 0}
	\item[\textbf{Punti Ferita:}] 24,  \textbf{Difesa:} 14,  \textbf{Iniziativa:} +2
	\item[\textbf{Movimento:}] 9 m
	\item[\textbf{Tiri Salvez.:}] \resizebox{0.5\linewidth+1.8cm}{!}{Tempra +3, Riflessi +3, Volontà +3}
	\item[\textbf{Res. Danni:}] da arma non magica\\
	\item[\textbf{Imm. Danni:}] Fuoco
	\item[\textbf{Sensi:}] Scurovisione 18 m
	\item[\textbf{Linguaggi:}] Ignan
	\item[\textbf{Sfida:}] 1/2 (100 PX)\smallskip
\end{description}

\emph{\textbf{Illuminazione Incendiaria.}} Come Azione Immediata, l'uomo magma può accendere o spegnere le sue fiamme. Mentre la fiamma è accesa, l'uomo magma irradia luce intensa in un raggio di 3 metri e luce fioca per 6 metri.

\emph{\textbf{Scoppio Mortale.}} Quando l'uomo magma muore, esplode in uno scoppio di fuoco e magma. Ogni creatura entro 3 metri da esso deve effettuare un Tiro Salvezza di Riflessi DC 12, subendo 7 (2d6) danni da fuoco se fallisce il Tiro Salvezza o la metà di questi danni se lo riesce. Gli oggetti infiammabili che non siano indossati o trasportati e che si trovino nell'area, prendono fuoco.

\textbf{Azioni}

\emph{\textbf{Tocco.} Attacco con arma da mischia}: +4 a colpire, portata 1 m, un bersaglio.

\emph{Colpisce:} 7 (2d6) danni da fuoco. Se il bersaglio è una creatura o un oggetto infiammabile, questi prende fuoco. Fino a che una creatura non usa un'Azione per estinguere la fiamma subisce 3 (1d6) danni da fuoco al termine di ciascun suo round.

\textbf{Ecologia}\\
Ambiente: Qualsiasi terreno (Piano del Fuoco)\\
Organizzazione: Solitario o banda (2-8)\\
\textbf{Categoria Tesoro}: L\\
\textbf{Descrizione}\\
Gli uomini di lava, noti come Ignim, abitano il Piano del Fuoco ma a volte scivolano nel Piano Materiale attraverso crepe elementali. Queste crepe si formano in luoghi di forte calore, come vulcani o fiumi sotterranei di magma, o in aree di intensa magia. Spesso, appiccano involontariamente fuoco agli oggetti infiammabili vicini.

Nonostante non siano coraggiosi, gli Ignim sono temibili nemici per chi non ha resistenza al loro calore intenso. Il loro tocco incenerisce gli abiti e le armi di acciaio rischiano di diventare scorie al contatto. Nel Piano del Fuoco, gli Ignim trovano forza nel numero, popolando insediamenti costellati di laghi di magma e geyser di roccia fusa.

Paranoici e diffidenti, gli Ignim temono gli abitanti più grandi del Piano del Fuoco e sommergono gli intrusi con domande. Se le risposte non soddisfano, cercano di sbarazzarsi delle creature il più rapidamente possibile, anche gettandole in laghi di roccia liquida.

Orgogliosi dei loro laghi di magma, ogni lago ha un diverso scopo: bagni, cottura o relax. Gli Ignim aggiungono minerali e sali ai laghi per adeguarli ai loro scopi. I laghi per cucinare, a volte chiamati "laghi assassini", sono più caldi, mentre quelli per il relax sono di solito più scuri.

Alla maturità, gli Ignim sono alti 1,2 metri e pesano 150 kg grazie alla loro densa composizione.

\mostro{Unicorno}
\noindent
\begin{description}[noitemsep, topsep=0pt, parsep=0pt, partopsep=0pt, leftmargin=0cm, labelwidth=2.2cm]
	\item[\textbf{Taglia/Tipo:}] Grande celestiale, buono
	\item[\textbf{Caratt.:}] \resizebox{0.5\linewidth+1.8cm}{!}{For 4 Des 2 Cos 2 Int 0 Sag 3 Car 3}
	\item[\textbf{Punti Ferita:}] 107,  \textbf{Difesa:} 20,  \textbf{Iniziativa:} +2
	\item[\textbf{Movimento:}] 15 m
	\item[\textbf{Tiri Salvez.:}] \resizebox{0.5\linewidth+1.8cm}{!}{Tempra +7, Riflessi +7, Volontà +8}
	\item[\textbf{Imm. Danni:}] Veleno
	\item[\textbf{Immunità:}] affascinato, paralizzato
	\item[\textbf{Sensi:}] Scurovisione 18 m
	\item[\textbf{Linguaggi:}] Celestiale, Elfico, Silvano, telepatia 18 m
	\item[\textbf{Sfida:}] 5 (1800 PX)\smallskip
\end{description}

\emph{\textbf{Armi Magiche.}} Gli attacchi con armi dell'unicorno sono magici.

\emph{\textbf{Carica.}} Se l'unicorno si muove di almeno 6 metri in linea retta verso il bersaglio e lo colpisce con un attacco di corno durante lo stesso round, il bersaglio subisce 9 (2d8) danni perforanti aggiuntivi. Se il bersaglio è una creatura, deve riuscire un T10iro Salvezza su Tempra DC 15 o cadere prono.

\emph{\textbf{Incantesimi Innati.}} La caratteristica da incantatore innato dell'unicorno è il Carisma (DC 14 per i Tiri Salvezza degli incantesimi). L'unicorno può lanciare in maniera innata i seguenti incantesimi, senza bisogno di componenti:

A volontà: \emph{\hyperlink{Artificio Druidico}{Artificio Druidico}}, \emph{\hyperlink{Passare Senza Tracce}{Passare Senza Tracce}}

1/giorno ciascuno: \emph{\hyperlink{Calmare Emozioni}{Calmare Emozioni}} \emph{\hyperlink{Intralciare}{Intralciare}}

\emph{\textbf{Resistenza alla Magia.}} L'unicorno ha +1d6 ai Tiri Salvezza contro incantesimi e altri effetti magici.

\textbf{Azioni}

\emph{\textbf{Multiattacco.}} L'unicorno effettua due attacchi: uno con gli zoccoli e uno con il corno.

\emph{\textbf{Corno.} Attacco con arma da mischia}: +6 a colpire, portata 1 m, un bersaglio.

\emph{Colpisce:} 8 (1d8 + 4) danni perforanti.

\emph{\textbf{Zoccoli.} Attacco con arma da mischia}: +6 a colpire, portata 1 m, un bersaglio.

\emph{Colpisce:} 11 (2d6 + 4) danni contundenti.

\emph{\textbf{Teletrasporto (1/Giorno).}} L'unicorno può teletrasportare magicamente sé stesso e fino a tre altre creature consenzienti visibili entro 1 metro da esso, insieme a tutto l'equipaggiamento che stanno indossando o trasportando, in un luogo familiare all'unicorno, che si trova ad un massimo di 1,5 chilometri di distanza.

\emph{\textbf{Tocco Guaritore (3/Giorno).}} L'unicorno entra a contatto tramite il corno con un'altra creatura. Il bersaglio recupera magicamente 11 (2d8 + 2) Punti Ferita. Inoltre, il contatto rimuove tutte le malattie e neutralizza tutti i veleni che affliggono il bersaglio.

\textbf{Azioni Aggiuntive}

L'unicorno può effettuare 3 Azioni aggiuntive, scelte tra le opzioni seguenti. Può usare solo un'opzione Aggiuntiva alla volta e solo al termine del round di un'altra creatura. L'unicorno recupera le azioni aggiuntive spese all'inizio del proprio round.

\textbf{Autoguarigione (Costa 3 Azioni).} L'unicorno recupera magicamente 11 (2d8 + 2) Punti Ferita.

\textbf{Scudo Scintillante (Costa 2 Azioni).} L'unicorno crea un campo magico scintillante che circonda lui o un'altra creatura visibile a lui entro 18 metri. Il bersaglio ottiene un bonus di +2 alla Difesa fino al termine del prossimo round dell'unicorno.

\textbf{Zoccoli.} L'unicorno effettua un attacco con gli zoccoli.

\textbf{Ecologia}\\
Ambiente: Foreste Temperate\\
Organizzazione: Solitario, coppia o benedizione (3-6)\\
\textbf{Categoria Tesoro}: Nessuno\\
\textbf{Descrizione}\\
Certo! Ecco un breve riassunto del testo sugli unicorni:

Gli unicorni sono creature intelligenti e solitarie che abitano le foreste, apparendo solo per difendere le loro dimore dal male. Evitano tutte le creature tranne i folletti buoni, le donne umanoidi buone e gli animali nativi. Le coppie di unicorni rimangono insieme per tutta la vita e proteggono le loro foreste, permettendo solo alle creature buone e neutrali di attraversarle.

Il corno dell'unicorno è la fonte dei suoi poteri magici, e le creature malvagie danno grande valore a questi corni per i loro riti oscuri e pozioni di guarigione. In rare occasioni, gli unicorni il cui partner è stato ucciso scelgono giovani donne virtuose come sostituti, permettendo loro di cavalcarli e diventare loro guardiani per tutta la vita.

%\begin{center}
%\includegraphics[width=0.45\textwidth]{immagini/Unicorn.png}
%\end{center}

%\addcontentsline{toc}{subsubsection}{V}
\pdfbookmark[3]{V}{V}

\mostro{Vampiro}
\noindent
\begin{description}[noitemsep, topsep=0pt, parsep=0pt, partopsep=0pt, leftmargin=0cm, labelwidth=2.2cm]
	\item[\textbf{Taglia/Tipo:}] Media non morto (mutaforma), malvagio
	\item[\textbf{Caratt.:}] \resizebox{0.5\linewidth+1.8cm}{!}{For 4 Des 4 Cos 4 Int 3 Sag 2 Car 4}
	\item[\textbf{Punti Ferita:}] 259,  \textbf{Difesa:} 33,  \textbf{Iniziativa:} +4
	\item[\textbf{Movimento:}] 9 m
	\item[\textbf{Tiri Salvez.:}] \resizebox{0.5\linewidth+1.8cm}{!}{\resizebox{0.5\linewidth+1.8cm}{!}{Tempra +17, Riflessi +17, Volontà +15}}
	\item[\textbf{Comp.:}] Furtività +9, Consapevolezza +17
	\item[\textbf{Imm. Danni:}] da Vuoto; Veleno, da arma non magica
	\item[\textbf{Immunità:}] affascinato, assordato, sanguinamento
	\item[\textbf{Sensi:}] Scurovisione 36 m
	\item[\textbf{Linguaggi:}] le lingue che conosceva in vita, Expiran
	\item[\textbf{Sfida:}] 13 (10000 PX)\smallskip
\end{description}

\emph{\textbf{Mutaforma.}} Se il vampiro non è sotto la luce del sole o immerso in acqua corrente, può usare una Azione per trasformarsi in un Minuscolo pipistrello, una nube di foschia Media, o per tornare alla sua vera forma.

Mentre è in forma di pipistrello, il vampiro non può parlare, la sua velocità di movimento è 1 metro e ha velocità di volo 9 metri. Le sue statistiche, a parte la taglia e la velocità, sono immutate. Qualsiasi equipaggiamento stia indossando si trasforma con esso, ma quello che stava trasportando viene fatto cadere a terra. Alla morte ritorna alla sua vera forma.

Mentre è in forma di foschia, il vampiro non può effettuare azioni, parlare o manipolare oggetti. È privo di peso, ha velocità di volo 6 metri, può fluttuare, e può entrare nello spazio di una creatura ostile e fermarsi lì. Inoltre, se in uno spazio vi passa dell'aria, la foschia può fare altrettanto senza stringersi, ma non può attraversare l'acqua. Ha +1d6 ai Tiri Salvezza su Tempra e Riflessi ed è immune a tutti i danni non magici, eccetto i danni subiti dalla luce del
sole.

\emph{\textbf{Debolezze del Vampiro.}} Il vampiro ha i seguenti difetti:

\emph{Danneggiato dall'Acqua Corrente.} Il vampiro subisce 20 danni da acido se termina il suo round all'interno dell'acqua corrente.

\emph{Ipersensibilità alla Luce.} Il vampiro subisce 20 danni da Luce quando inizia il suo round alla luce del sole. Mentre è alla luce del sole, ha -1d6 ai tiri di attacco e le prove di competenza di Base.

\emph{Paletto nel Cuore.} Se un'arma perforante fatta di legno viene conficcata nel cuore del vampiro mentre il vampiro è inabile nel suo luogo di riposo, il vampiro resta paralizzato finché il paletto non viene rimosso.

\emph{Proibizione.} Il vampiro non può entrare in un'abitazione senza invito da parte dei suoi occupanti.

\emph{\textbf{Fuga nella Foschia.}} Quando scende a 0 Punti Ferita al di fuori del suo luogo di riposo, il vampiro si trasforma in una nube di foschia (come per il tratto Mutaforma) invece di cadere privo di sensi, purché non sia esposto alla luce del sole o all'acqua corrente. Se non può trasformarsi, viene distrutto.

Mentre si trova a 0 Punti Ferita in questa forma, non può tornare alla sua forma di vampiro, e deve raggiungere il suo luogo di riposo entro 2 ore o venire distrutto. Una volta raggiunto il suo luogo di riposo, ritorna alla sua forma di vampiro. Resterà quindi paralizzato finché non avrà recuperato almeno 1 punto ferita. Dopo aver trascorso almeno 1 ora nel suo luogo di riposo a 0 Punti Ferita, il vampiro recupererà 1 punto ferita.

\emph{\textbf{Natura Non Morta.}} Il vampiro non ha bisogno di aria.

\emph{\textbf{Resistenza Leggendaria (3/Giorno).}} Se il vampiro fallisce un Tiro Salvezza, può scegliere invece di riuscire.

\emph{\textbf{Rigenerazione.}} Il vampiro recupera 20 Punti Ferita all'inizio del suo round se possiede almeno 1 punto ferita e non è esposto alla luce del sole o l'acqua corrente. Se il vampiro subisce danno da Luce o danno dall'Acqua santa, questo tratto non funziona all'inizio del prossimo round del vampiro.

\emph{\textbf{Scalare come Ragno.}} Il vampiro può scalare superfici difficili, compreso lo stare a testa in giù sul soffitto, senza bisogno di effettuare una prova di competenza.

\textbf{Azioni}

\emph{\textbf{Multiattacco.}} Il vampiro può effettuare due attacchi, ma solo uno di essi può essere un attacco con morso.

\emph{\textbf{Colpo Disarmato (Solo in Forma di Vampiro).} Attacco con arma da mischia}: +12 a colpire, portata 1 m, una creatura.

\emph{Colpisce:} 8 (1d8 + 4) danni contundenti. Invece di infliggere danno, il vampiro può afferrare il bersaglio (DC per fuggire 18).

\emph{\textbf{Morso (Solo in Forma di Pipistrello o Vampiro).} Attacco con arma da mischia}: +11 a colpire, portata 1 m, una creatura consenziente o una creatura afferrata dal vampiro, inabile o intralciata.

\emph{Colpisce:} 7 (1d6 + 4) danni perforanti più 10 (3d6) danni da Vuoto. I Punti Ferita massimi del bersaglio sono ridotti di un ammontare pari al danno da Vuoto subito, e il vampiro recupera un numero di Punti Ferita pari a quell'ammontare, TS Tempra DC 23 per resistere alla perdita di Punti Ferita Massimi. Il bersaglio diviene Affaticato. Il bersaglio muore se questo effetto riduce i suoi Punti Ferita massimi a 0. Un umanoide ucciso in questo modo e poi sepolto nel terreno si rianima la notte seguente come progenie vampirica sotto il controllo del vampiro.

\emph{\textbf{Affascinare.}} Il vampiro prende a bersaglio un umanoide entro 9 metri che può vedere. Se il bersaglio può vedere il vampiro, deve effettuare un Tiro Salvezza di Volontà DC 25 contro questa magia o esserne affascinato. Il bersaglio affascinato considera il vampiro un amico fidato da ascoltare e proteggere. Sebbene il bersaglio non sia sotto il controllo del vampiro, prende le richieste e le azioni del vampiro nel modo più favorevole possibile, ed è un bersaglio consenziente dell'attacco con morso del vampiro.

Ogni volta che il vampiro o i compagni del vampiro fanno qualcosa di nocivo al bersaglio, questi può ripetere il Tiro Salvezza, terminando l'effetto su di sé in caso di successo. Altrimenti, l'effetto persiste 24 ore o finché il vampiro non viene distrutto, si trova su di un piano di esistenza diverso dal bersaglio, o effettua una Reazione per terminare l'effetto.

\emph{\textbf{Figli della Notte (1/Giorno).}} Il vampiro richiama magicamente 2d4 sciami di pipistrelli o ratti, purché il sole non sia sorto. Mentre è all'esterno, il vampiro può richiamare invece 3d6 lupi. Le creature richiamate arrivano in 1d4 round, agendo da alleati del vampiro e obbedendo ai suoi comandi. Le bestie restano per 1 ora, finché il vampiro non muore, o finché non le congeda con un'Azione Immediata.

\textbf{Reazione: \emph{Attacco d'opportunità}}: il vampiro effettua un attacco ad una creatura che attraversi o esca dalla sua portata di 1 metro.

\textbf{Azioni Aggiuntive}

Il vampiro può effettuare 3 Azioni aggiuntive, scelte tra le opzioni seguenti. Può usare solo un'opzione Aggiuntiva alla volta e solo al termine del round di un'altra creatura. Il vampiro recupera all'inizio del proprio round le Azioni aggiuntive che ha speso.

\textbf{Colpo Disarmato.} Il vampiro effettua un colpo disarmato.

\textbf{Morso (Costa 2 Azioni).} Il vampiro effettua un attacco con morso.

\textbf{Muoversi.} Il vampiro si muove del suo movimento senza provocare attacchi di opportunità.

\textbf{Ecologia}
Ambiente: Qualsiasi\\
Organizzazione: Solitario o famiglia (vampiro più 2-8 Progenie)\\
\textbf{Categoria Tesoro}: Equipaggiamento da PNG (Anello della Protezione +2, Fascia della Seduzione +4, Mantello della Resistenza +3)\\
\textbf{Descrizione}\\
I vampiri sono creature umanoidi non morte che si nutrono del sangue dei viventi. Hanno un aspetto molto simile a quando erano in vita, diventando spesso più attraenti, sebbene alcuni appaiano invece duri e ferini.

\mostro{Progenie Vampirica}
\noindent
\begin{description}[noitemsep, topsep=0pt, parsep=0pt, partopsep=0pt, leftmargin=0cm, labelwidth=2.2cm]
	\item[\textbf{Taglia/Tipo:}] Media non morto, malvagio
	\item[\textbf{Movimento:}] 9 m
	\item[\textbf{Tiri Salvez.:}] \resizebox{0.5\linewidth+1.8cm}{!}{Tempra +9, Riflessi +9, Volontà +6}
	\item[\textbf{Caratt.:}] \resizebox{0.5\linewidth+1.8cm}{!}{For 3 Des 3 Cos 3 Int 0 Sag 0 Car 1}
	\item[\textbf{Punti Ferita:}] 126,  \textbf{Difesa:} 23,  \textbf{Iniziativa:} +3
	\item[\textbf{Comp.:}] Furtività +6
	\item[\textbf{Res. Danni:}] da Vuoto; da arma non magica
	\item[\textbf{Sensi:}] Scurovisione 18 m
	\item[\textbf{Linguaggi:}] le lingue che conosceva in vita
	\item[\textbf{Sfida:}] 6 (2300 PX)\smallskip
\end{description}

\emph{\textbf{Debolezze della Progenie Vampirica.}} La Progenie Vampirica ha i seguenti difetti:

\emph{Danneggiato dall'Acqua Corrente.} La Progenie Vampirica subisce 20 danni da acido se termina il suo round all'interno dell'acqua corrente.

\emph{Ipersensibilità alla Luce.} La Progenie Vampirica subisce 20 danni da Luce quando inizia il suo round alla luce del sole. Mentre è alla luce del sole, ha -1d6 ai tiri di attacco e le prove di competenza di Base.

\emph{Paletto nel Cuore.} La Progenie Vampirica è distrutto se un'arma perforante di legno gli viene conficcata nel cuore mentre è inabile all'interno del suo luogo di riposo.

\emph{Proibizione.} La Progenie Vampirica non può entrare in un'abitazione senza invito da parte dei suoi occupanti.

\emph{\textbf{Natura Non Morta.}} La Progenie Vampirica non ha bisogno di aria.

\emph{\textbf{Rigenerazione.}} La Progenie Vampirica recupera 10 Punti Ferita all'inizio del suo round se possiede almeno 1 punto ferita e non è esposto alla luce del sole o l'acqua corrente. Se la Progenie Vampirica subisce danno da Luce o danno dall'Acqua santa, questo tratto non funziona all'inizio del prossimo round del vampiro.

\emph{\textbf{Scalare come Ragno.}} La Progenie Vampirica può scalare superfici difficili, compreso lo stare a testa in giù sul soffitto, senza bisogno di effettuare una prova di competenza.

\textbf{Azioni}

\emph{\textbf{Multiattacco.}} La progenie vampirica può effettuare due attacchi, ma solo uno di essi può essere un attacco con morso.

\emph{\textbf{Artigli.} Attacco con arma da mischia}: +8 a colpire, portata 1 m, una creatura.

\emph{Colpisce:} 8 (2d4 + 3) danni taglienti. Invece di infliggere danno, il vampiro può afferrare il bersaglio (DC per fuggire 13).

\emph{\textbf{Morso.} Attacco con arma da mischia}: +8 a colpire, portata 1 m, una creatura afferrata dal vampiro, inabile o intralciata.

\emph{Colpisce:} 6 (1d6 + 3) danni perforanti più 7 (2d6) danni da Vuoto. I Punti Ferita massimi del bersaglio sono ridotti di un ammontare pari al danno da Vuoto subito, e il vampiro recupera un numero di Punti Ferita pari a quell'ammontare, TS Tempra DC 16 per resistere alla perdita di Punti Ferita massimi. Il bersaglio muore se questo effetto riduce i suoi Punti Ferita massimi a 0. La creatura diventa Affaticata.

\textbf{Reazione: \emph{Attacco d'opportunità}}: la progenie vampirica effettua un attacco ad una creatura che attraversi o esca dalla sua portata di 1 metro.

\textbf{Ecologia}\\
Ambiente: Qualsiasi\\
Organizzazione: Solitario, coppia, gruppo (3-6) o turba (7-12)\\
\textbf{Categoria Tesoro}: M\\
\textbf{Descrizione}\\
Un Vampiro può decidere di creare da una vittima una progenie vampirica anziché farne un vampiro completo solo quando usa la sua capacità creare progenie su una creatura umanoide. Questa decisione deve essere presa appena un vampiro uccide una creatura appropriata usando il morso.

\mostro{Vermi delle carne}
\noindent
\begin{description}[noitemsep, topsep=0pt, parsep=0pt, partopsep=0pt, leftmargin=0cm, labelwidth=2.2cm]
	\item[\textbf{Taglia/Tipo:}] minuscola mostruosità, disallineato
	\item[\textbf{Caratt.:}] \resizebox{0.5\linewidth+1.8cm}{!}{For -4 Des 0 Cos -2 Int -4 Sag 0 Car -4}
	\item[\textbf{Punti Ferita:}] 32,  \textbf{Difesa:} 13,  \textbf{Iniziativa:} +0
	\item[\textbf{Movimento:}] 1 m
	\item[\textbf{Tiri Salvez.:}] \resizebox{0.5\linewidth+1.8cm}{!}{Tempra +3, Riflessi +3, Volontà +3}
	\item[\textbf{Sensi:}] vista tellurica 3 m
	\item[\textbf{Sfida:}] 1 (200 PX)\smallskip
\end{description}

\textbf{Azioni}

\emph{\textbf{Infestare la carne.}} Queste minuscole creature penetrano nella carne esposta senza effettuare Tiro per Colpire purché la carne sia esposta a contatto con loro.

\emph{\textbf{Colpisce.}} Entro 2d4 round i vermi (3d6 creature) della carne scavano nel tessuto dirigendosi verso il cuore. L'infestazione dei vermi causa 1 Punto Ferita di danno a round mentre scavano. Una volta arrivati al cuore ogni round il personaggio deve fare un Tiro Salvezza su Tempra DC 14, con penalità cumulativa di -1 per round. Una volta che il Tiro Salvezza fallisce il personaggio muore.

\emph{\textbf{Debellare i Vermi della carne.}} L'unico modo è usare una fiamma viva (una torcia causa 1d6 di danno ad applicazione od un incantesimo tipo Onda rovente) sulla parte dove i vermi stanno scavando. Ogni applicazione di fuoco può eliminare 3d6 vermi. Una prova di Pronto Soccorso a DC 15 rimuove 1d4 parassiti ma causa 1d4 danni nell'estrazione. Passati i 2d4 round i vermi sono troppo in profondità ed è inutile applicare il fuoco, solo un incantesimo di Cura Malattie, o Guarigione, può debellare completamente l'infestazione.

\textbf{Ecologia}\\
Ambiente: alberi marci, carne putrefatta\\
Organizzazione: gruppi 3d6\\
\textbf{Categoria Tesoro}: Nessuno\\
\textbf{Descrizione}\\
I vermi della carne sono tra i più temuti parassiti dagli avventurieri. Si trovano nei cumuli umidi di foglie o tronchi marci, nei cadaveri in putrefazione, nell acque torbide. Pallidi, viscidi, dotati di affilatissimi denti, lunghi poco più di 4 millimetri penetrano nella carne esposta in maniera facilissima e percepiscono il battito del cuore dove si dirigono. Mentre scavano nelle carni si possono percepire ed anche vedere strisciare sottopelle.

\mostro{Verme Purpureo}
\noindent
\begin{description}[noitemsep, topsep=0pt, parsep=0pt, partopsep=0pt, leftmargin=0cm, labelwidth=2.2cm]
	\item[\textbf{Taglia/Tipo:}] Mastodontica mostruosità, disallineato
	\item[\textbf{Caratt.:}] \resizebox{0.5\linewidth+1.8cm}{!}{For 9 Des -2 Cos 6 Int -5 Sag -1 Car -3}
	\item[\textbf{Punti Ferita:}] 303,  \textbf{Difesa:} 30,  \textbf{Iniziativa:} -2
	\item[\textbf{Movimento:}] 15 m, scavo 9 m
	\item[\textbf{Tiri Salvez.:}] \resizebox{0.5\linewidth+1.8cm}{!}{\resizebox{0.5\linewidth+1.8cm}{!}{Tempra +21, Riflessi +13, Volontà +14}}
	\item[\textbf{Sensi:}] Vista Cieca 9 m, senso tellurico 18 m
	\item[\textbf{Sfida:}] 15 (13000 PX)\smallskip
\end{description}

\emph{\textbf{Scavatore di Tunnel.}} Il verme può scavare attraverso la roccia solida a metà della velocità di scavare e lascia un tunnel di 3 metri di diametro dietro di se.

\textbf{Azioni}

\emph{\textbf{Multiattacco.}} Il verme effettua due attacchi: uno con il morso e uno con il pungiglione.

\emph{\textbf{Morso.} Attacco con arma da mischia}: +13 a colpire, portata 3 m, un bersaglio.

\emph{Colpisce:} 22 (3d8 + 9) danni perforanti. Se il bersaglio è una creatura di taglia Grande, deve riuscire un Tiro Salvezza di Riflessi DC 28 o venire inghiottita dal verme. Mentre è inghiottita, la creatura è accecata e intralciata, ha copertura completa contro gli attacchi e altri effetti provenienti dall'esterno del verme, e subisce 21 (6d6) danni da acido all'inizio di ciascun round del verme.

Se il verme subisce 30 o più danni in un singolo round da una creatura al suo interno, il verme deve riuscire un Tiro Salvezza di Tempra DC 25 al termine del suo round o vomitare tutte le creature inghiottite, che cadono prone in uno spazio entro 3 metri dal verme. Se il verme muore, una creatura inghiottita non risulta più intralciata da esso e può fuggire dal cadavere usando 2 Azioni e uscendo prona.

\emph{\textbf{Pungiglione.} Attacco con arma da mischia}: +13 a colpire, portata 3 m, una creatura.

\emph{Colpisce:} 19 (3d6 + 9) danni perforanti e il bersaglio deve effettuare un Tiro Salvezza di Tempra DC 28, subendo 42 (12d6) danni da veleno o la metà di questi danni se lo riesce.

\emph{\textbf{Avviluppare.} Attacco con arma da mischia}: +12 a colpire, portata 3 m, una creatura. Il verme purpureo si stringe attorno alla creatura. 2 Azioni

\emph{Colpisce:} 30 (8d6 + 9) danni contundenti ed afferrato. Ad ogni round la creatura subisce 15 di danno da stritolamento, TS Tempra DC 24 per liberarsi.

\textbf{Ecologia}\\
Ambiente: Qualsiasi sotterraneo\\
Organizzazione: Solitario\\
\textbf{Categoria Tesoro}: Accidentale\\
\textbf{Descrizione}\\
I vermi purpurei sono giganteschi necrofagi che abitano nelle regioni più profonde del mondo, mangiando qualsiasi materiale organico incontrino. Sono noti per inghiottire le loro prede intere. Non è insolito sentire di un gruppo di avventurieri scomparso all'interno delle fameliche fauci di un verme purpureo, gridando di terrore mentre i suoi membri sparivano uno alla volta.

Mentre vanno in cerca di creature viventi per divorarle, i vermi purpurei ingoiano anche un'enorme quantità di terra e minerali scavando nel sottosuolo. Le interiora di un verme purpureo possono contenere un considerevole numero di gemme e altri oggetti in grado di resistere all'acido corrosivo all'interno del suo esofago. In zone ricche di minerali preziosi, come quelle vicine alle miniere naniche, i tunnel naturali creati dagli scavi dei vermi purpurei sono spesso pieni di un notevole numero di pepite d'oro grezzo.

Un verme purpureo generalmente reclama una grande caverna sotterranea come sua tana, e anche se vi torna per riposare e digerire il cibo, passa la maggior parte del suo tempo in cerca di preda, scavando attraverso l'oscurità senza fine o scivolando lungo tunnel preesistenti alla costante ricerca di cibo per saziare la sua immensa fame. Sebbene quasi privi di intelletto, i vermi purpurei raramente sono stupidi. Sono diffusi come guardiani fra chi riesce a controllarli magicamente o hanno nel loro covo una stanza abbastanza grande da ospitarli.

\mostro{Verme Strisciante Tentacolato}
\noindent
\begin{description}[noitemsep, topsep=0pt, parsep=0pt, partopsep=0pt, leftmargin=0cm, labelwidth=2.2cm]
	\item[\textbf{Taglia/Tipo:}] Grande mostruosità, disallineato
	\item[\textbf{Caratt.:}] \resizebox{0.5\linewidth+1.8cm}{!}{For 4 Des 1 Cos 3 Int -4 Sag 1 Car -3}
	\item[\textbf{Punti Ferita:}] 89,  \textbf{Difesa:} 18,  \textbf{Iniziativa:} +1
	\item[\textbf{Movimento:}] 9 m, scalare 9 m
	\item[\textbf{Tiri Salvez.:}] \resizebox{0.5\linewidth+1.8cm}{!}{Tempra +7, Riflessi +5, Volontà +5}
	\item[\textbf{Sensi:}] Scurovisione 18 m
	\item[\textbf{Sfida:}] 4 (1000 PX)\smallskip
\end{description}

\emph{\textbf{Scalare come Ragno.}} Il Verme Strisciante Tentacolato può scalare superfici difficili, compreso lo stare a testa in giù sul soffitto, senza bisogno di effettuare una prova di competenza.

\textbf{Azioni}

\emph{\textbf{Multiattacco.}} Il Verme Strisciante Tentacolato effettua 3 attacchi, uno con il morso e due con i tentacoli.

\emph{\textbf{Morso.} Attacco con arma da mischia}: +7 a colpire, portata 1 m, un bersaglio.

\emph{Colpisce:} 10 (2d8 + 6) danni perforanti.

\emph{\textbf{Tentacolo.} Attacco con arma da mischia}: +7 a colpire, portata 3 m, una creatura.

\emph{Colpisce:} 1 danno contundente. Il bersaglio deve effettuare un Tiro Salvezza di Tempra DC 18 o rimanere paralizzato fino alla fine del round successivo.

\textbf{Ecologia}\\
Ambiente: Qualsiasi sotterraneo\\
Organizzazione: Solitario, paio, tribù (8-12 +3d6 piccoli)\\
\textbf{Categoria Tesoro}: Accidentale\\
\textbf{Descrizione}\\
Un tipico Verme Strisciante Tentacolato è un anellide lungo quasi 3 metri e pesa sui 400 kilogrammi. Di colore scuro (di varie gradazioni dal blu al verde al marrone) è un largo verme dotato di una possente bocca e lunghi e leggeri tentacoli lungo tutta la testa.

Il Verme Strisciante Tentacolato pur se dotato di corte zampe non cammina ma striscia secernendo un muco appiccicoso che gli permette di arrampicarsi anche su superfici in qualsiasi orientamento.

Sono creature fameliche che non perdono occasione per cacciare e divorare o conservare i cadaveri dove seminare le loro uova. Amano la carne di Nibali e si nutrono di qualsiasi creatura vivente (spesso ratti dato il tipico ambiente delle fogne).

Le origini dei Vermi Striscianti Tentacolato sono piuttosto speculative, alcuni ipotizzano che un incantatore abbia provato, come al solito, fallendo criticamente, a trasformarsi in un Verme Purpureo, altri credono fermamente che i giardini di Shayalia avessero bisogni di maggiore concimazione e così la Patrona trasformò dei normali lombrichi in queste terrificanti creature perché divorassero e digerissero i cadaveri seppelliti.

\mostro{Viverna}
\noindent
\begin{description}[noitemsep, topsep=0pt, parsep=0pt, partopsep=0pt, leftmargin=0cm, labelwidth=2.2cm]
	\item[\textbf{Taglia/Tipo:}] Grande drago, disallineato
	\item[\textbf{Caratt.:}] \resizebox{0.5\linewidth+1.8cm}{!}{For 4 Des 0 Cos 3 Int -3 Sag 1 Car -2}
	\item[\textbf{Punti Ferita:}] 126,  \textbf{Difesa:} 20,  \textbf{Iniziativa:} +0
	\item[\textbf{Movimento:}] 6 m, volo 24 m
	\item[\textbf{Tiri Salvez.:}] \resizebox{0.5\linewidth+1.8cm}{!}{Tempra +9, Riflessi +6, Volontà +7}
	\item[\textbf{Sensi:}] Scurovisione 18 m
	\item[\textbf{Sfida:}] 6 (2300 PX)\smallskip
\end{description}

\textbf{Azioni}

\emph{\textbf{Multiattacco.}} La viverna può effettuare due attacchi: uno con il morso e uno con il pungiglione. Mentre vola, può usare i suoi artigli al posto di uno degli altri attacchi.

\emph{\textbf{Artigli.} Attacco con arma da mischia}: +8 a colpire, portata 1 m, un bersaglio.

\emph{Colpisce:} 13 (2d8 + 4) danni taglienti, 1 danno da Sanguinamento.

\emph{\textbf{Morso.} Attacco con arma da mischia}: +8 a colpire, portata 3 m, una creatura.

\emph{Colpisce:} 11 (2d6 + 4) danni perforanti.

\emph{\textbf{Pungiglione.} Attacco con arma da mischia}: +8 a colpire, portata 3 m, una creatura.

\emph{Colpisce:} 11 (2d6 + 4) danni perforanti. Il bersaglio deve effettuare un Tiro Salvezza di Tempra DC 18, e subire 24 (7d6) danni da veleno se lo fallisce, o la metà di questi danni se lo riesce.

\textbf{Reazione: \emph{Attacco d'opportunità}}: la viverna nero effettua un attacco con Artiglio ad una creatura che attraversi o esca dalla sua portata di 3 metri.

\emph{\textbf{Arrabbiato:}} la Viverna punta la coda in direzione del nemico e genera un cono di 3 metri di veleno. E' possibile eseguire un Tiro Salvezza su Riflessi DC 21 per dimezzare i 7d8 di danno da veleno.

\textbf{Ecologia}\\
Ambiente: Colline temperate o calde\\
Organizzazione: Solitario, coppia o stormo (3-6)\\
\textbf{Categoria Tesoro}: D\\
\textbf{Descrizione}\\
Le viverne sono rettili brutali e violenti imparentati con i draghi. Sono sempre aggressive ed impazienti e preferiscono raggiungere i loro scopi utilizzando la forza. Per questa ragione, i draghi guardano alle viverne con superiorità, considerando questi loro lontani parenti come selvaggi primitivi privi di stile ed intelligenza.

Nella maggior parte dei casi, questa generalizzazione è azzeccata. Anche se non certo di intelletto animale e capace di parola, la maggior parte delle viverne non si cura della diplomazia, preferendo combattere prima e discutere poi, solo se si trovano davanti ad un avversario che non possono sconfiggere o da cui non possono fuggire.

Le viverne sono creature territoriali. Pur cacciando occasionalmente prede più grandi in gruppi più estesi, sono creature solitarie il cui territorio di caccia si estende dai 160 ai 320 km quadrati. È noto che le viverne combattono spesso fra loro fino alla morte per le contese su un territorio ricco di prede.

Seppur costantemente affamate ed inclini ad attaccare, una viverna può essere resa amichevole attraverso un'attenta combinazione di lusinghe, intimidazione, cibo e tesoro, per farne un potente alleato. Spesso servono Giganti e Umanoidi Mostruosi come guardiani come guardiani, ed alcune tribù di Boggard e Lucertoloidi le usano come cavalcature, anche se tali accordi spesso risultano parecchio costosi in termini di cibo ed oro, poiché sono poche le viverne che accettano di servire a lungo creature simili come cavalcature.

Una viverna è lunga circa 4,8 metri e la coda rappresenta da sola circa metà della lunghezza. Una viverna pesa in media 1000 kg.

%\addcontentsline{toc}{subsubsection}{W}
\pdfbookmark[3]{W}{W}

\mostro{Wight}
\noindent
\begin{description}[noitemsep, topsep=0pt, parsep=0pt, partopsep=0pt, leftmargin=0cm, labelwidth=2.2cm]
	\item[\textbf{Taglia/Tipo:}] Media non morto, malvagio
	\item[\textbf{Caratt.:}] \resizebox{0.5\linewidth+1.8cm}{!}{For 2 Des 2 Cos 3 Int 0 Sag 1 Car 2}
	\item[\textbf{Punti Ferita:}] 70,  \textbf{Difesa:} 18,  \textbf{Iniziativa:} +2
	\item[\textbf{Movimento:}] 9 m
	\item[\textbf{Tiri Salvez.:}] \resizebox{0.5\linewidth+1.8cm}{!}{Tempra +6, Riflessi +5, Volontà +4}
	\item[\textbf{Comp.:}] Furtività +4, Consapevolezza +3
	\item[\textbf{Res. Danni:}] da Vuoto; da arma non magica o che non sia argentata
	\item[\textbf{Imm. Danni:}] Veleno
	\item[\textbf{Immunità:}] affaticato, sanguinamento
	\item[\textbf{Sensi:}] Scurovisione 18 m
	\item[\textbf{Linguaggi:}] le lingue che conosceva in vita, Expiran
	\item[\textbf{Sfida:}] 3 (700 PX)\smallskip
\end{description}

\emph{\textbf{Natura Non Morta.}} Il wight non ha bisogno di aria, cibo, bevande o sonno.

\emph{\textbf{Sensibilità alla Luce}}. Mentre è alla luce del sole, il wight ha -1d6 ai tiri di attacco, oltre che alle prove di Consapevolezza basate sulla vista.

\textbf{Azioni}

\emph{\textbf{Multiattacco.}} Il wight può effettuare due attacchi con la spada lunga o due attacchi con l'arco lungo. Può usare Risucchiare Vita al posto di uno dei suoi attacchi con la spada lunga.

\emph{\textbf{Risucchiare Vita.} Attacco con arma da mischia}: +6 a colpire, portata 1 m, una creatura.

\emph{Colpisce:} 5 (1d6 + 2) danni da Vuoto. Il bersaglio deve riuscire un Tiro Salvezza di Tempra DC 14 o vedere i suoi Punti Ferita massimi ridotti di un ammontare pari al danno subito. Il bersaglio diviene Affaticato. Il bersaglio muore se l'effetto riduce i suoi Punti Ferita massimi a 0.

Un umanoide ucciso da questo attacco si rianima 24 ore più tardi come zombi sotto il controllo del wight, a meno che l'umanoide non venga prima riportato in vita o il corpo sia distrutto. Il wight non può controllare più di dodici zombi alla volta.

\emph{\textbf{Spada Lunga.} Attacco con arma da mischia}: +5 a colpire, portata 1 m, un bersaglio.

\emph{Colpisce:} 6 (1d8 + 2) danni taglienti o 7 (1d10 + 2) danni taglienti se usata con due mani.

\emph{\textbf{Arco Lungo.} Attacco con arma a Distanza}: +5 a colpire, gittata 45m, un bersaglio.

\emph{Colpisce:} 6 (1d8 + 2) danni perforanti.

\textbf{Ecologia}\\
Ambiente: qualsiasi\\
Organizzazione: Solitario, coppia, gruppo (3-6) o branco (7-12)\\
\textbf{Categoria Tesoro}: Q\\
\textbf{Descrizione}\\
I wight sono umanoidi risorti come non morti a causa della necromanzia, di una morte violenta o di una personalità estremamente malevola. In alcuni casi, un wight sorge quando uno spirito non morto si lega permanentemente ad un cadavere, spesso quello di un guerriero. Sono appena riconoscibili da chi li conosceva in vita: le loro carni sono corrotte dalla malvagità e dalla non morte, gli occhi ardono d'odio ed i denti divengono quelli di una bestia. In un certo senso, un wight è l'anello di congiunzione tra ghoul e spettri: un cadavere deforme che risucchia energia vitale col tocco.

Essendo non morti, i wight non hanno bisogno di respirare, così a volte si possono trovare sott'acqua, sebbene non siano nuotatori particolarmente abili a meno che non siano originati da creature nuotatrici quali elfi acquatici e marinidi. Sott'acqua i wight preferiscono le caverne dal soffitto basso dove le loro scarse capacità di nuoto non sono una limitazione.

\mostro{Wraith}
\noindent
\begin{description}[noitemsep, topsep=0pt, parsep=0pt, partopsep=0pt, leftmargin=0cm, labelwidth=2.2cm]
	\item[\textbf{Taglia/Tipo:}] Media non morto, malvagio
	\item[\textbf{Caratt.:}] \resizebox{0.5\linewidth+1.8cm}{!}{For -2 Des 3 Cos 3 Int 1 Sag 2 Car 2}
	\item[\textbf{Punti Ferita:}] 108,  \textbf{Difesa:} 21,  \textbf{Iniziativa:} +3
	\item[\textbf{Movimento:}] 0 m, volo 18 m, Fluttuare
	\item[\textbf{Tiri Salvez.:}] \resizebox{0.5\linewidth+1.8cm}{!}{Tempra +8, Riflessi +8, Volontà +7}
	\item[\textbf{Res. Danni:}] Acido, Freddo, Elettricità, Fuoco, Suono; da arma non magica o che non sia argentata
	\item[\textbf{Imm. Danni:}] da Vuoto, Veleno
	\item[\textbf{Immunità:}] affascinato, afferrato, intralciato, paralizzato, pietrificato, prono, affaticato, sanguinamento
	\item[\textbf{Sensi:}] Scurovisione 18 m
	\item[\textbf{Linguaggi:}] le lingue che conosceva in vita, Expiran
	\item[\textbf{Sfida:}] 5 (1800 PX)\smallskip
\end{description}

\emph{\textbf{Movimento Incorporeo.}} Il wraith può attraversare creature e oggetti come fossero terreno difficile. Subisce 5 (1d10) danni da forza se termina il proprio round all'interno di un oggetto.

\emph{\textbf{Natura Non Morta.}} Il wraith non ha bisogno di aria, cibo, bevande o sonno.

\emph{\textbf{Sensibilità alla Luce}}. Mentre è alla luce del sole, il wraith ha -1d6 ai tiri di attacco, oltre che alle prove di Consapevolezza basate sulla vista.

\textbf{Azioni}

\emph{\textbf{Risucchiare Vita.} Attacco con arma da mischia}: +7 a colpire, portata 1 m, una creatura.

\emph{Colpisce:} 21 (4d8 + 3) danni da Vuoto. Il bersaglio deve riuscire un Tiro Salvezza di Tempra DC 16 o vedere i suoi Punti Ferita massimi ridotti di un ammontare pari al danno subito. Il bersaglio diviene Affaticato. Il bersaglio muore se l'effetto riduce i suoi Punti Ferita massimi a 0.

\emph{\textbf{Creare Spettro.}} Il wraith prende a bersaglio un umanoide entro 3 metri da esso e che sia morto da non più di 1 minuto e per cause violente. Lo spirito del bersaglio si anima come spettro nello spazio del suo cadavere e nello spazio più vicino non occupato. Lo spettro è sotto il controllo del wraith. Il wraith non può tenere più di sette spettri alla volta sotto il suo controllo.

\textbf{Reazione: \emph{Attacco d'opportunità}}: il Wraith effettua un attacco di Risucchiare Vita ad una creatura che attraversi o esca dalla sua portata di 1 metro.

\emph{\textbf{Arrabbiato:}} il Wraith canalizza le sue energie negative in una esplosione di Vuoto attorno a se nel raggio di 6 metri. Tutte le creature devono effettuare un Tiro Salvezza su Tempra DC 16 o subito 3d6 di danno da Vuoto, se il Tiro Salvezza riesce sono Rallentate 1/3r.

\textbf{Ecologia}\\
Ambiente: Qualsiasi\\
Organizzazione: Solitario, coppia, gruppo (3-6) o branco (7-12)\\
\textbf{Categoria Tesoro}: Nessuno\\
\textbf{Descrizione}\\
I wraith sono creature nate dal male e dall'oscurità. Detestano la luce e le creature viventi, avendo perduto la maggior parte del legame con la loro vita precedente.

%\addcontentsline{toc}{subsubsection}{X}
\pdfbookmark[3]{X}{X}

\mostro{Xorn}
\noindent
\begin{description}[noitemsep, topsep=0pt, parsep=0pt, partopsep=0pt, leftmargin=0cm, labelwidth=2.2cm]
	\item[\textbf{Taglia/Tipo:}] Media elementale, neutrale
	\item[\textbf{Caratt.:}] \resizebox{0.5\linewidth+1.8cm}{!}{For 3 Des 0 Cos 6 Int 0 Sag 0 Car 0}
	\item[\textbf{Punti Ferita:}] 111,  \textbf{Difesa:} 18,  \textbf{Iniziativa:} +0
	\item[\textbf{Movimento:}] 6 m, scavo 6 m
	\item[\textbf{Tiri Salvez.:}] \resizebox{0.5\linewidth+1.8cm}{!}{Tempra +11, Riflessi +5, Volontà +5}
	\item[\textbf{Comp.:}] Furtività +3, Consapevolezza +6
	\item[\textbf{Res. Danni:}] perforante e tagliente di armi non magiche o che non siano di adamantio
	\item[\textbf{Sensi:}] Scurovisione 18 m, senso tellurico 18 m
	\item[\textbf{Linguaggi:}] Tremun
	\item[\textbf{Sfida:}] 5 (1800 PX)\smallskip
\end{description}

\emph{\textbf{Mimetismo di Pietra.}} Lo xorn ha +1d6 alle prove di Furtività (Nascondersi) effettuate per nascondersi su terreno roccioso.

\emph{\textbf{Scorrere sulla Terra.}} Lo xorn può scavare attraversa la terra e la pietra non magiche e non lavorate. Quando lo fa, lo xorn non disturba il materiale che sposta.

\emph{\textbf{Senso del Tesoro.}} Lo xorn può individuare precisamente, con l'olfatto, la posizione di metalli e pietre preziose, come monete e gemme, entro 18 metri da esso.

\textbf{Azioni}

\emph{\textbf{Multiattacco.}} Lo xorn effettua tre attacchi di artiglio e un attacco di morso.

\emph{\textbf{Artiglio.} Attacco con arma da mischia}: +7 a colpire, portata 1 m, un bersaglio.

\emph{Colpisce:} 6 (1d6 + 3) danni taglienti, 1 danno da Sanguinamento.

\emph{\textbf{Morso.} Attacco con arma da mischia}: +6 a colpire, portata 1 m, un bersaglio.

\emph{Colpisce:} 13 (3d6 + 3) danni perforanti.

\emph{\textbf{Arrabbiato:}} lo Xorn erutta le ultime gemme e pepite mangiate. In un cono di 6 metri tutte le creature devono fare un Tiro Salvezza su Riflessi DC 18 per dimezzare il danno, 3d8 contundente, delle gemme e minerali scagliati (che però non hanno più valore).

\textbf{Ecologia}\\
Ambiente: Qualsiasi (Piano della Terra)\\
Organizzazione: Solitario, coppia o gruppo (3-6)\\
\textbf{Categoria Tesoro}: solo metalli preziosi, gemme e gioielli e gemme magiche\\
\textbf{Descrizione}\\
Strane creature larghe quanto alte, gli xorn hanno poco interesse verso i nativi del Piano Materiale, non fosse per le gemme ed i metalli preziosi che potrebbero avere con sé. Nascosti sotto la superficie del terreno per un tempo che ad un umano potrebbe sembrare lunghissimo, uno xorn può attendere mesi, perfino anni, per la preda ideale, per poi assalire chi porta con sé il suo cibo preferito, come una gemma particolare o un determinato tipo di argento. Gli avventurieri che si addentrano nelle regioni abitate dagli xorn portano spesso con sé piccole pepite di minerali o gemme e cristalli di scarso valore da utilizzare come tributo. Anche se il suo valore è solitamente direttamente proporzionale al suo sapore e all'appetibilità che esso può avere, la maggior parte degli xorn è piuttosto ingorda, e preferisce la quantità alla qualità.

Il tesoro che uno xorn porta con sé o nasconde nella sua tana consiste in uno spuntino che ha conservato per il giorno successivo. Offrire un gioiello o un metallo preziosi particolarmente deliziosi (e costosi) ad uno xorn può cementare un'alleanza temporanea. Dato che gli xorn possono attraversare la roccia con facilità sono ottime guide nelle regioni sotterranee.

Gli xorn non sono molto religiosi, ma quelli fra loro che trovano la fede sono solitamente devoti a Efrem (anche se è raro, se non improbabile, che gli xorn abbiano Compagni Animali, dato che non possono seguirli nella roccia, e scelgono invece il dominio della Terra). Bardi e Devoti xorn non sono sconosciuti: i Bardi scelgono di solito Intrattenere (canto), e gli Devoti hanno invariabilmente la Stirpe Elementale (terra).

%\addcontentsline{toc}{subsubsection}{Z}
\pdfbookmark[3]{Z}{Z}

\mostro{Zombi}
\noindent
\begin{description}[noitemsep, topsep=0pt, parsep=0pt, partopsep=0pt, leftmargin=0cm, labelwidth=2.2cm]
	\item[\textbf{Taglia/Tipo:}] Media non morto, malvagio
	\item[\textbf{Caratt.:}] \resizebox{0.5\linewidth+1.8cm}{!}{For 1 Des -2 Cos 3 Int -4 Sag -2 Car -3}
	\item[\textbf{Punti Ferita:}] 19,  \textbf{Difesa:} 10,  \textbf{Iniziativa:} -2
	\item[\textbf{Movimento:}] 6 m
	\item[\textbf{Tiri Salvez.:}] \resizebox{0.5\linewidth+1.8cm}{!}{Tempra +3, Riflessi +3, Volontà +3}
	\item[\textbf{Imm. Danni:}] Veleno
	\item[\textbf{Immunità:}] sanguinamento
	\item[\textbf{Sensi:}] Scurovisione 18 m
	\item[\textbf{Linguaggi:}] comprende tutte le lingue che parlava in vita ma non può parlare
	\item[\textbf{Sfida:}] 1/4 (50 PX)\smallskip
\end{description}

\emph{\textbf{Natura Non Morta.}} Lo zombi non ha bisogno di aria, cibo, bevande o sonno.

\emph{\textbf{Tempra dei Non Morti.}} Se il danno riduce lo zombi a 0 Punti Ferita, lo zombi deve effettuare un Tiro Salvezza di Tempra DC 5 + il danno subito, a meno che il danno non sia da Luce o un colpo critico. Se riesce, lo zombi scende invece a 1 punto ferita.

\emph{\textbf{Lento come uno Zombi.}} Lo zombie esegue solo due Azioni a round.

\textbf{Azioni}

\emph{\textbf{Schianto.} Attacco con arma da mischia}: +4 a colpire, portata 1 m, un bersaglio.

\emph{Colpisce:} 4 (1d6 + 1) danni contundenti.

\textbf{Ecologia}\\
Ambiente: Qualsiasi\\
Organizzazione: Qualsiasi\\
\textbf{Categoria Tesoro}: Nessuno\\
\textbf{Descrizione}\\
Gli zombi sono i cadaveri animati di creature morte, costretti a muoversi da magie necromantiche come Animare Morti. Anche se gli zombi incontrati di norma sono lenti e robusti, altri possiedono tratti differenti, che permettono loro di diffondere una malattia o di muoversi più rapidi.

Gli zombi sono automi senza mente e non possono fare altro che seguire gli ordini. Se lasciati a loro stessi, attendono immobili o si spostano alla ricerca di creature viventi da massacrare e divorare. Gli zombi attaccano fino alla distruzione, senza curarsi della loro sicurezza.

Sebbene siano in grado di seguire gli ordini, gli zombi vengono spesso lasciati liberi con l'ordine di uccidere tutte le creature viventi. Spesso vengono incontrati in branchi che infestano le terre frequentate dai viventi, in cerca di preda. La maggior parte degli zombi viene creata attraverso \hyperlink{Animare Morti}{Animare Morti}. Simili zombi sono sempre standard, a meno che il creatore lanci anche Velocità o Rimuovi Paralisi per creare Zombi Rapidi o Contagio per creare Zombi Infetti.

\mostro{Zombi Ogre}
\noindent
\begin{description}[noitemsep, topsep=0pt, parsep=0pt, partopsep=0pt, leftmargin=0cm, labelwidth=2.2cm]
	\item[\textbf{Taglia/Tipo:}] Grande non morto, malvagio
	\item[\textbf{Caratt.:}] \resizebox{0.5\linewidth+1.8cm}{!}{For 4 Des -2 Cos 4 Int -4 Sag -2 Car -3}
	\item[\textbf{Punti Ferita:}] 52,  \textbf{Difesa:} 12,  \textbf{Iniziativa:} -2
	\item[\textbf{Movimento:}] 9 m
	\item[\textbf{Tiri Salvez.:}] \resizebox{0.5\linewidth+1.8cm}{!}{Tempra +6, Riflessi +3, Volontà +3}
	\item[\textbf{Imm. Danni:}] Veleno
	\item[\textbf{Immunità:}] sanguinamento
	\item[\textbf{Sensi:}] Scurovisione 18 m
	\item[\textbf{Linguaggi:}] comprende Comune e Gigante ma non può parlare
	\item[\textbf{Sfida:}] 2 (450 PX)\smallskip
\end{description}

\emph{\textbf{Natura Non Morta.}} Lo zombi non ha bisogno di aria, cibo, bevande o sonno.

\emph{\textbf{Tempra dei Non Morti.}} Se il danno riduce lo zombi a 0 Punti Ferita, lo zombi deve effettuare un Tiro Salvezza di Tempra DC 5 + il danno subito, a meno che il danno non sia da Luce o un colpo critico. Se riesce, lo zombi scende invece a 1 punto ferita.

\textbf{Azioni}

\emph{\textbf{Mazza Chiodata.} Attacco con arma da mischia}: +6 a colpire, portata 1 m, un bersaglio.

\emph{Colpisce:} 13 (2d8 + 4) danni contundenti.

\textbf{Categoria Tesoro}: Nessuno

\subsection{Appendice A: Creature Varie}

Questa appendice contiene le statistiche di vari animali, parassiti e
altre creature. Le statistiche sono organizzate in ordine alfabetico.

\mostro{Albero Risvegliato}
\begin{description}[noitemsep, topsep=0pt, parsep=0pt, partopsep=0pt, leftmargin=0cm, labelwidth=2.2cm]
	\item[\textbf{Taglia/Tipo:}] Enorme pianta, disallineato
	\item[\textbf{Caratt.:}] \resizebox{0.5\linewidth+1.8cm}{!}{For 4 Des -2 Cos 2 Int 0 Sag 0 Car -2}
  \item[\textbf{Punti Ferita:}] 51,  \textbf{Difesa:} 12,  \textbf{Iniziativa:} +0
	\item[\textbf{Movimento:}] 6 m
	\item[\textbf{Tiri Salvez.:}] \resizebox{0.5\linewidth+1.8cm}{!}{Tempra +4, Riflessi +3, Volontà +3}
	\item[\textbf{Vul. al Danno:}] Fuoco
	\item[\textbf{Res. al Danno:}] contundente, perforante
	\item[\textbf{Comp.:}] Storia +2
	\item[\textbf{Sensi:}] Scurovisione 18 m
	\item[\textbf{Linguaggi:}] una lingua conosciuta dal suo creatore
	\item[\textbf{Sfida:}] 2 (450 PX)\smallskip
\end{description}

\emph{\textbf{Falso Aspetto.}} Mentre l'albero rimane immobile, è indistinguibile da un normale albero.

\textbf{Azioni}

\emph{\textbf{Schianto.} Attacco con Arma da Mischia}: +6 a colpire, portata 3 m, un bersaglio.

\emph{Colpisce:} 14 (3d6 + 4) danni contundenti.

\mostro{Alce}
\begin{description}[noitemsep, topsep=0pt, parsep=0pt, partopsep=0pt, leftmargin=0cm, labelwidth=2.2cm]
	\item[\textbf{Taglia/Tipo:}] Grande bestia, disallineato
	\item[\textbf{Caratt.:}] \resizebox{0.5\linewidth+1.8cm}{!}{For 3 Des 0 Cos 1 Int -4 Sag 0 Car -2}
	\item[\textbf{Punti Ferita:}] 19,  \textbf{Difesa:} 12,  \textbf{Iniziativa:} +0
	\item[\textbf{Tiri Salvez.:}] \resizebox{0.5\linewidth+1.8cm}{!}{Tempra +3, Riflessi +3, Volontà +3}
	\item[\textbf{Movimento:}] 15 m
	\item[\textbf{Sfida:}] 1/4 (50 PX)\smallskip
\end{description}

\emph{\textbf{Carica.}} Se l'alce si muove di almeno 6 metri diretto verso il bersaglio e lo colpisce con un attacco di rostro durante lo stesso round, il bersaglio subisce 7 (2d6) danni contundenti aggiuntivi. Se il bersaglio è una creatura, deve riuscire un Tiro Salvezza di Tempra DC 13 o cadere prono.

\textbf{Azioni}

\emph{\textbf{Rostro.} Attacco con Arma da Mischia}: +5 a colpire, portata 1 m, un bersaglio.

\emph{Colpisce:} 6 (1d6 + 3) danni contundenti.

\emph{\textbf{Zoccoli.} Attacco con Arma da Mischia}: +5 a colpire, portata 1 m, una creatura prona.

\emph{Colpisce:} 8 (2d4 + 3) danni contundenti.

\mostro{Alce Gigante}
\begin{description}[noitemsep, topsep=0pt, parsep=0pt, partopsep=0pt, leftmargin=0cm, labelwidth=2.2cm]
	\item[\textbf{Taglia/Tipo:}] Enorme bestia, disallineato
	\item[\textbf{Caratt.:}] \resizebox{0.5\linewidth+1.8cm}{!}{For 4 Des 3 Cos 2 Int -2 Sag 2 Car 0}
    \item[\textbf{Tiri Salvez.:}] \resizebox{0.5\linewidth+1.8cm}{!}{Tempra +4, Riflessi +5, Volontà +4}
	\item[\textbf{Punti Ferita:}] 51,  \textbf{Difesa:} 17,  \textbf{Iniziativa:} +3
	\item[\textbf{Movimento:}] 18 m
	\item[\textbf{Sfida:}] 2 (450 PX)\smallskip
\end{description}

\emph{\textbf{Carica.}} Se l'alce si muove di almeno 6 metri diretto verso il bersaglio e lo colpisce con un attacco di rostro durante lo stesso round, il bersaglio subisce 7 (2d6) danni contundenti aggiuntivi. Se il bersaglio è una creatura, deve riuscire un Tiro Salvezza di Tempra DC 14 o cadere prono.

\textbf{Azioni}

\emph{\textbf{Rostro.} Attacco con Arma da Mischia}: +6 a colpire, portata 3 m, un bersaglio.

\emph{Colpisce:} 11 (2d6 + 4) danni perforanti.

\emph{\textbf{Zoccoli.} Attacco con Arma da Mischia}: +6 a colpire, portata 1 m, una creatura prona.

\emph{Colpisce:} 22 (4d4 + 4) danni contundenti.

\mostro{Aquila}
\begin{description}[noitemsep, topsep=0pt, parsep=0pt, partopsep=0pt, leftmargin=0cm, labelwidth=2.2cm]
	\item[\textbf{Taglia/Tipo:}] Piccola bestia, disallineato
	\item[\textbf{Caratt.:}] \resizebox{0.5\linewidth+1.8cm}{!}{For -2 Des 2 Cos 0 Int -4 Sag 2 Car -2}
    \item[\textbf{Tiri Salvez.:}] \resizebox{0.5\linewidth+1.8cm}{!}{Tempra +3, Riflessi +3, Volontà +3}
	\item[\textbf{Punti Ferita:}] 15,  \textbf{Difesa:} 14,  \textbf{Iniziativa:} +2
	\item[\textbf{Movimento:}] 3 m, volo 18 m
	\item[\textbf{Sfida:}] 0 (10 PX)\smallskip
\end{description}

\emph{\textbf{Vista Affinata.}} L'aquila ha +1d6 nelle prove di Consapevolezza basate sulla vista.

\textbf{Azioni}

\emph{\textbf{Speroni.} Attacco con Arma da Mischia}: +4 a colpire, portata 1 m, un bersaglio.

\emph{Colpisce:} 4 (1d4 + 2) danni taglienti.

\mostro{Aquila Gigante}
\begin{description}[noitemsep, topsep=0pt, parsep=0pt, partopsep=0pt, leftmargin=0cm, labelwidth=2.2cm]
	\item[\textbf{Taglia/Tipo:}] Grande bestia, disallineato
	\item[\textbf{Caratt.:}] \resizebox{0.5\linewidth+1.8cm}{!}{For 3 Des 3 Cos 1 Int -1 Sag 2 Car 0}
    \item[\textbf{Tiri Salvez.:}] \resizebox{0.5\linewidth+1.8cm}{!}{Tempra +3, Riflessi +4, Volontà +3}
	\item[\textbf{Punti Ferita:}] 33,  \textbf{Difesa:} 16,  \textbf{Iniziativa:} +3
	\item[\textbf{Movimento:}] 3 m, volo 24 m
	\item[\textbf{Linguaggi:}] Aquila Gigante, comprende il Comune e l'Ictun ma non può parlarli
	\item[\textbf{Sfida:}] 1 (200 PX)\smallskip
\end{description}

\emph{\textbf{Vista Affinata.}} L'aquila ha +1d6 nelle prove di Consapevolezza basate sulla vista.

\textbf{Azioni}

\emph{\textbf{Multiattacco.}} L'aquila effettua due attacchi: uno con il becco e uno con gli speroni.

\emph{\textbf{Becco.} Attacco con Arma da Mischia}: +5 a colpire, portata 1 m, un bersaglio.

\emph{Colpisce:} 6 (1d6 + 3) danni perforanti.

\emph{\textbf{Speroni.} Attacco con Arma da Mischia}: +5 a colpire, portata 1 m, un bersaglio.

\emph{Colpisce:} 10 (2d6 + 3) danni taglienti.

\mostro{Avvoltoio}
\begin{description}[noitemsep, topsep=0pt, parsep=0pt, partopsep=0pt, leftmargin=0cm, labelwidth=2.2cm]
	\item[\textbf{Taglia/Tipo:}] Media bestia, disallineato
	\item[\textbf{Caratt.:}] \resizebox{0.5\linewidth+1.8cm}{!}{For -2 Des 0 Cos 1 Int -4 Sag 1 Car -3}
	\item[\textbf{Punti Ferita:}] 15,  \textbf{Difesa:} 12,  \textbf{Iniziativa:} +0
	\item[\textbf{Tiri Salvez.:}] \resizebox{0.5\linewidth+1.8cm}{!}{Tempra +3, Riflessi +3, Volontà +3}
	\item[\textbf{Movimento:}] 3 m, volo 15 m
	\item[\textbf{Sfida:}] 0 (10 PX)\smallskip
\end{description}

\emph{\textbf{Olfatto e Vista Affinati.}} L'avvoltoio ha +1d6 nelle prove di Consapevolezza basate su olfatto o vista.

\emph{\textbf{Tattiche di Branco.}} L'avvoltoio ha +1d6 al tiro di attacco contro una creatura se almeno uno degli alleati dell'avvoltoio si trova entro 1 metro dalla creatura e quell'alleato non è inabile.

\textbf{Azioni}

\emph{\textbf{Becco.} Attacco con Arma da Mischia}: +3 a colpire, portata 1 m, un bersaglio.

\emph{Colpisce:} 2 (1d4) danni perforanti.

\mostro{Avvoltoio Gigante}
\begin{description}[noitemsep, topsep=0pt, parsep=0pt, partopsep=0pt, leftmargin=0cm, labelwidth=2.2cm]
	\item[\textbf{Taglia/Tipo:}] Grande bestia, disallineato
	\item[\textbf{Caratt.:}] \resizebox{0.5\linewidth+1.8cm}{!}{For 2 Des 0 Cos 2 Int -2 Sag 1 Car -2}
	\item[\textbf{Punti Ferita:}] 15,  \textbf{Difesa:} 12,  \textbf{Iniziativa:} +0
	\item[\textbf{Tiri Salvez.:}] \resizebox{0.5\linewidth+1.8cm}{!}{Tempra +3, Riflessi +3, Volontà +3}
	\item[\textbf{Movimento:}] 3 m, volo 18 m
	\item[\textbf{Sfida:}] 0 (10 PX)\smallskip
\end{description}

\emph{\textbf{Olfatto e Vista Affinati.}} L'avvoltoio ha +1d6 nelle prove di Consapevolezza basate su olfatto o vista.

\emph{\textbf{Tattiche di Branco.}} L'avvoltoio ha +1d6 al tiro di attacco contro una creatura se almeno uno degli alleati dell'avvoltoio si trova entro 1 metro dalla creatura e quell'alleato non è inabile.

\textbf{Azioni}

\emph{\textbf{Multiattacco.}} L'avvoltoio effettua due attacchi: uno con il becco e uno con gli speroni.

\emph{\textbf{Becco.} Attacco con Arma da Mischia}: +4 a colpire, portata 1 m, un bersaglio.

\emph{Colpisce:} 7 (2d4 + 2) danni perforanti.

\emph{\textbf{Speroni.} Attacco con Arma da Mischia}: +4 a colpire, portata 1 m, un bersaglio.

\emph{Colpisce:} 9 (2d6 + 2) danni taglienti.

\mostro{Babbuino}
\begin{description}[noitemsep, topsep=0pt, parsep=0pt, partopsep=0pt, leftmargin=0cm, labelwidth=2.2cm]
	\item[\textbf{Taglia/Tipo:}] Piccola bestia, disallineato
	\item[\textbf{Caratt.:}] \resizebox{0.5\linewidth+1.8cm}{!}{For -1 Des 2 Cos 0 Int -3 Sag 1 Car -2}
	\item[\textbf{Punti Ferita:}] 15,  \textbf{Difesa:} 14,  \textbf{Iniziativa:} +2
	\item[\textbf{Tiri Salvez.:}] \resizebox{0.5\linewidth+1.8cm}{!}{Tempra +3, Riflessi +3, Volontà +3}
	\item[\textbf{Movimento:}] 6 m
	\item[\textbf{Sfida:}] 0 (10 PX)\smallskip
\end{description}

\emph{\textbf{Tattiche di Branco.}} Il babbuino ha +1d6 al tiro di attacco contro una creatura se almeno uno degli alleati del babbuino si trova entro 1 metro dalla creatura e quell'alleato non è inabile.

\textbf{Azioni}

\emph{\textbf{Morso.} Attacco con Arma da Mischia}: +1 a colpire, portata 1 m, un bersaglio.

\emph{Colpisce:} 1 (1d4 - 1) danni perforanti.

\mostro{Balena Assassina (Orca)}
\begin{description}[noitemsep, topsep=0pt, parsep=0pt, partopsep=0pt, leftmargin=0cm, labelwidth=2.2cm]
	\item[\textbf{Taglia/Tipo:}] Enorme bestia, disallineato
	\item[\textbf{Caratt.:}] \resizebox{0.5\linewidth+1.8cm}{!}{For 4 Des 0 Cos 1 Int -4 Sag 1 Car -2}
	\item[\textbf{Punti Ferita:}] 69,  \textbf{Difesa:} 16,  \textbf{Iniziativa:} +0
	\item[\textbf{Tiri Salvez.:}] \resizebox{0.5\linewidth+1.8cm}{!}{Tempra +4, Riflessi +3, Volontà +4}
	\item[\textbf{Movimento:}] 0 m, nuoto 18 m
	\item[\textbf{Sfida:}] 3 (700 PX)\smallskip
\end{description}

\emph{\textbf{Ecolocazione.}} La balena non può usare la vista cieca se assordata.

\emph{\textbf{Trattenere il Fiato.}} La balena può trattenere il fiato per 30 minuti

\emph{\textbf{Udito Affinato.}} La balena ha +1d6 alle prove di Consapevolezza basate sull'udito.

\textbf{Azioni}

\emph{\textbf{Morso.} Attacco con Arma da Mischia}: +6 a colpire, portata 1 m, un bersaglio.

\emph{Colpisce:} 21 (5d6 + 4) danni perforanti.

\mostro{Becco d'Ascia}
\begin{description}[noitemsep, topsep=0pt, parsep=0pt, partopsep=0pt, leftmargin=0cm, labelwidth=2.2cm]
	\item[\textbf{Taglia/Tipo:}] Grande bestia, disallineato
	\item[\textbf{Caratt.:}] \resizebox{0.5\linewidth+1.8cm}{!}{For 2 Des 1 Cos 1 Int -4 Sag 0 Car -3}
	\item[\textbf{Punti Ferita:}] 19,  \textbf{Difesa:} 13,  \textbf{Iniziativa:} +1
	\item[\textbf{Tiri Salvez.:}] \resizebox{0.5\linewidth+1.8cm}{!}{Tempra +3, Riflessi +3, Volontà +3}
	\item[\textbf{Movimento:}] 15 m
	\item[\textbf{Sfida:}] 1/4 (50 PX)\smallskip
\end{description}

\textbf{Azioni}

\emph{\textbf{Becco.} Attacco con Arma da Mischia}: +4 a colpire, portata 1 m, un bersaglio.

\emph{Colpisce:} 6 (1d8 + 2) danni taglienti.

\mostro{Cane della Morte}
\begin{description}[noitemsep, topsep=0pt, parsep=0pt, partopsep=0pt, leftmargin=0cm, labelwidth=2.2cm]
	\item[\textbf{Taglia/Tipo:}] Media mostruosità, malvagio
	\item[\textbf{Caratt.:}] \resizebox{0.5\linewidth+1.8cm}{!}{For 2 Des 2 Cos 2 Int -4 Sag 1 Car -2}
	\item[\textbf{Punti Ferita:}] 33,  \textbf{Difesa:} 15,  \textbf{Iniziativa:} +2
	\item[\textbf{Tiri Salvez.:}] \resizebox{0.5\linewidth+1.8cm}{!}{Tempra +3, Riflessi +3, Volontà +3}
	\item[\textbf{Movimento:}] 12 m
	\item[\textbf{Sfida:}] 1 (200 PX)\smallskip
\end{description}

\emph{\textbf{Bicefalo.}} Il cane ha +1d6 nelle prove di Consapevolezza e nei Tiri Salvezza contro le condizioni accecato, affascinato, assordato, spaventato, stordito o svenuto.

\textbf{Azioni}

\emph{\textbf{Multiattacco.}} Il cane effettua due attacchi di morso.

\emph{\textbf{Morso.} Attacco con Arma da Mischia}: +4 a colpire, portata 1 m, un bersaglio.

\emph{Colpisce:} 5 (1d6 + 2) danni perforanti. Se il bersaglio è una creatura, deve riuscire un Tiro Salvezza di Tempra DC 12 contro la malattia o restare malato finché la malattia non viene curata. Dopo ogni 24 ore, la creatura deve ripetere il Tiro Salvezza, riducendo i suoi Punti Ferita massimi di 5 (1d10) in caso di fallimento. Questa riduzione perdura finché la malattia non viene curata. La creatura muore se la malattia riduce i suoi Punti Ferita massimi a 0.

\mostro{Cane Intermittente}
\begin{description}[noitemsep, topsep=0pt, parsep=0pt, partopsep=0pt, leftmargin=0cm, labelwidth=2.2cm]
	\item[\textbf{Taglia/Tipo:}] Media mostruosità, malvagio
	\item[\textbf{Caratt.:}] \resizebox{0.5\linewidth+1.8cm}{!}{For 1 Des 3 Cos 1 Int 0 Sag 1 Car 0}
	\item[\textbf{Punti Ferita:}] 19,  \textbf{Difesa:} 15,  \textbf{Iniziativa:} +3
	\item[\textbf{Vul. al Danno:}] ferro freddo
	\item[\textbf{Tiri Salvez.:}] \resizebox{0.5\linewidth+1.8cm}{!}{Tempra +3, Riflessi +3, Volontà +3}
	\item[\textbf{Movimento:}] 12 m
	\item[\textbf{Sfida:}] 1/4 (50 PX)\smallskip
\end{description}

\emph{\textbf{Udito e Olfatto Affinato.}} Il cane ha +1d6 nelle prove di Consapevolezza basate su udito o olfatto.

\textbf{Azioni}

\emph{\textbf{Morso.} Attacco con Arma da Mischia}: +3 a colpire, portata 1 m, un bersaglio.

\emph{Colpisce:} 4 (1d6 + 1) danni perforanti.

\emph{\textbf{Teletrasporto (Ricarica 4-6).}} Il cane si teletrasporta magicamente, insieme a qualsiasi cosa stia indossando o trasportando, fino a 12 metri in uno spazio non occupato che possa vedere. Prima o dopo il teletrasporto, il cane può effettuare un attacco di morso.

\mostro{Caprone}
\begin{description}[noitemsep, topsep=0pt, parsep=0pt, partopsep=0pt, leftmargin=0cm, labelwidth=2.2cm]
	\item[\textbf{Taglia/Tipo:}] Media bestia, disallineato
	\item[\textbf{Caratt.:}] \resizebox{0.5\linewidth+1.8cm}{!}{For 1 Des 0 Cos 0 Int -4 Sag 0 Car -3}
	\item[\textbf{Punti Ferita:}] 15,  \textbf{Difesa:} 12,  \textbf{Iniziativa:} +0
	\item[\textbf{Tiri Salvez.:}] \resizebox{0.5\linewidth+1.8cm}{!}{Tempra +3, Riflessi +3, Volontà +3}
	\item[\textbf{Movimento:}] 9 m
	\item[\textbf{Sfida:}] 0 (10 PX)\smallskip
\end{description}

\emph{\textbf{Carica.}} Se il caprone si muove di almeno 6 metri diretto verso il bersaglio e colpisce con un attacco di rostro durante lo stesso round, il bersaglio subisce 2 (1d4) danni contundenti aggiuntivi. Se il bersaglio è una creatura, deve riuscire un Tiro Salvezza di Tempra DC 10 o cadere prona.

\emph{\textbf{Piedi Saldi.}} Il caprone ha +1d6 ai Tiri Salvezza su Tempra e Riflessi effettuati contro effetti che lo farebbero cadere prono.

\textbf{Azioni}

\emph{\textbf{Rostro.} Attacco con Arma da Mischia}: +3 a colpire, portata 1 m, un bersaglio.

\emph{Colpisce:} 3 (1d4 + 1) danni contundenti.

\mostro{Caprone Gigante}
\begin{description}[noitemsep, topsep=0pt, parsep=0pt, partopsep=0pt, leftmargin=0cm, labelwidth=2.2cm]
	\item[\textbf{Taglia/Tipo:}] Media bestia, disallineato
	\item[\textbf{Caratt.:}] \resizebox{0.5\linewidth+1.8cm}{!}{For 3 Des 0 Cos 1 Int -4 Sag 1 Car -2}
	\item[\textbf{Punti Ferita:}] 24,  \textbf{Difesa:} 12,  \textbf{Iniziativa:} +0
	\item[\textbf{Tiri Salvez.:}] \resizebox{0.5\linewidth+1.8cm}{!}{Tempra +3, Riflessi +3, Volontà +3}
	\item[\textbf{Movimento:}] 12 m
	\item[\textbf{Sfida:}] 1/2 (100 PX)\smallskip
\end{description}

\emph{\textbf{Carica.}} Se il caprone si muove di almeno 6 metri diretto verso il bersaglio e colpisce con un attacco di rostro durante lo stesso round, il bersaglio subisce 5 (2d4) danni contundenti aggiuntivi. Se il bersaglio è una creatura, deve riuscire un Tiro Salvezza di Tempra DC 13 o cadere prona.

\emph{\textbf{Piedi Saldi.}} Il caprone ha +1d6 ai Tiri Salvezza su Tempra e Riflessi effettuati contro effetti che lo farebbero cadere prono.

\textbf{Azioni}

\emph{\textbf{Rostro.} Attacco con Arma da Mischia}: +5 a colpire, portata 1 m, un bersaglio.

\emph{Colpisce:} 8 (2d4 + 3) danni contundenti.

\mostro{Cavallo Marino Gigante}
\begin{description}[noitemsep, topsep=0pt, parsep=0pt, partopsep=0pt, leftmargin=0cm, labelwidth=2.2cm]
	\item[\textbf{Taglia/Tipo:}] Grande bestia, disallineato
	\item[\textbf{Caratt.:}] \resizebox{0.5\linewidth+1.8cm}{!}{For 1 Des 2 Cos 0 Int -4 Sag 1 Car -3}
	\item[\textbf{Punti Ferita:}] 24,  \textbf{Difesa:} 14,  \textbf{Iniziativa:} +2
	\item[\textbf{Tiri Salvez.:}] \resizebox{0.5\linewidth+1.8cm}{!}{Tempra +3, Riflessi +3, Volontà +3}
	\item[\textbf{Movimento:}] 0 m, nuoto 12 m
	\item[\textbf{Sfida:}] 1/2 (100 PX)\smallskip
\end{description}

\emph{\textbf{Carica.}} Se il cavallo marino si muove di almeno 6 metri diretto verso il bersaglio e colpisce con un attacco di rostro durante lo stesso round, il bersaglio subisce 7 (2d6) danni contundenti aggiuntivi. Se il bersaglio è una creatura, deve riuscire un Tiro Salvezza su Tempra DC 11 o cadere prona.

\emph{\textbf{Respirare Acqua.}} Il cavallo marino può respirare solo sott'acqua.

\textbf{Azioni}

\emph{\textbf{Rostro.} Attacco con Arma da Mischia}: +3 a colpire, portata 1 m, un bersaglio.

\emph{Colpisce:} 4 (1d6 + 1) danni contundenti.

\mostro{Cervo}
\begin{description}[noitemsep, topsep=0pt, parsep=0pt, partopsep=0pt, leftmargin=0cm, labelwidth=2.2cm]
	\item[\textbf{Taglia/Tipo:}] Media bestia, disallineato
	\item[\textbf{Caratt.:}] \resizebox{0.5\linewidth+1.8cm}{!}{For 0 Des 3 Cos 0 Int -4 Sag 2 Car -3}
	\item[\textbf{Punti Ferita:}] 15,  \textbf{Difesa:} 15,  \textbf{Iniziativa:} +3
	\item[\textbf{Tiri Salvez.:}] \resizebox{0.5\linewidth+1.8cm}{!}{Tempra +3, Riflessi +3, Volontà +3}
	\item[\textbf{Movimento:}] 12 m
	\item[\textbf{Sfida:}] 0 (10 PX)\smallskip
\end{description}

\textbf{Azioni}

\emph{\textbf{Morso.} Attacco con Arma da Mischia}: +2 a colpire, portata 1 m, un bersaglio.

\emph{Colpisce:} 2 (1d4) danni perforanti.

\mostro{Cinghiale}
\begin{description}[noitemsep, topsep=0pt, parsep=0pt, partopsep=0pt, leftmargin=0cm, labelwidth=2.2cm]
	\item[\textbf{Taglia/Tipo:}] Media bestia, disallineato
	\item[\textbf{Caratt.:}] \resizebox{0.5\linewidth+1.8cm}{!}{For 1 Des 0 Cos 1 Int -4 Sag -1 Car -3}
	\item[\textbf{Punti Ferita:}] 19,  \textbf{Difesa:} 12,  \textbf{Iniziativa:} +0
	\item[\textbf{Tiri Salvez.:}] \resizebox{0.5\linewidth+1.8cm}{!}{Tempra +3, Riflessi +3, Volontà +3}
	\item[\textbf{Movimento:}] 12 m
	\item[\textbf{Sfida:}] 1/4 (50 PX)\smallskip
\end{description}

\emph{\textbf{Carica.}} Se il cinghiale si muove di almeno 6 metri diretto verso il bersaglio e colpisce con un attacco di zanna durante lo stesso round, il bersaglio subisce 3 (1d6) danni taglienti aggiuntivi. Se il bersaglio è una creatura, deve riuscire un Tiro Salvezza di Tempra DC 11 o cadere prono.

\emph{\textbf{Implacabile (Ricarica dopo un 1 ora).}} Se il cinghiale subisce 7 danni o meno che lo ridurrebbero a 0 Punti Ferita, scende invece a 1 punto ferita.

\textbf{Azioni}

\emph{\textbf{Zanna.} Attacco con Arma da Mischia}: +3 a colpire, portata 1 m, un bersaglio.

\emph{Colpisce:} 4 (1d6 + 1) danni taglienti.

\mostro{Cinghiale Gigante}
\begin{description}[noitemsep, topsep=0pt, parsep=0pt, partopsep=0pt, leftmargin=0cm, labelwidth=2.2cm]
    \item[\textbf{Taglia/Tipo:}] Grande bestia, disallineato
    \item[\textbf{Caratt.:}] \resizebox{0.5\linewidth+1.8cm}{!}{For 3 Des 0 Cos 3 Int -4 Sag -2 Car -3}
    \item[\textbf{Punti Ferita:}] 52,  \textbf{Difesa:} 14,  \textbf{Iniziativa:} +0
    \item[\textbf{Tiri Salvez.:}] \resizebox{0.5\linewidth+1.8cm}{!}{Tempra +5, Riflessi +3, Volontà +3}
    \item[\textbf{Movimento:}] 12 m
    \item[\textbf{Sfida:}] 2 (450 PX)\smallskip
\end{description}

\emph{\textbf{Carica.}} Se il cinghiale si muove di almeno 6 metri diretto verso il bersaglio e colpisce con un attacco di zanna durante lo stesso round, il bersaglio subisce 7 (2d6) danni taglienti aggiuntivi. Se il bersaglio è una creatura, deve riuscire un Tiro Salvezza di Tempra DC 13 o cadere prono.

\emph{\textbf{Implacabile (Ricarica dopo un 1 ora).}} Se il cinghiale subisce 10 danni o meno che lo ridurrebbero a 0 Punti Ferita, scende invece a 1 punto ferita.

\textbf{Azioni}

\emph{\textbf{Zanna.} Attacco con Arma da Mischia}: +5 a colpire, portata 1 m, un bersaglio.

\emph{Colpisce:} 10 (2d6 + 3) danni taglienti.

\mostro{Coccodrillo}
\begin{description}[noitemsep, topsep=0pt, parsep=0pt, partopsep=0pt, leftmargin=0cm, labelwidth=2.2cm]
    \item[\textbf{Taglia/Tipo:}] Grande bestia, disallineato
    \item[\textbf{Caratt.:}] \resizebox{0.5\linewidth+1.8cm}{!}{For 2 Des 0 Cos 1 Int -4 Sag 0 Car -3}
    \item[\textbf{Punti Ferita:}] 24,  \textbf{Difesa:} 12,  \textbf{Iniziativa:} +0
    \item[\textbf{Tiri Salvez.:}] \resizebox{0.5\linewidth+1.8cm}{!}{Tempra +3, Riflessi +3, Volontà +3}
    \item[\textbf{Movimento:}] 6 m, nuoto 9 m
    \item[\textbf{Sfida:}] 1/2 (100 PX)\smallskip
\end{description}

\emph{\textbf{Trattenere il Fiato.}} Il coccodrillo può trattenere il fiato per 15 minuti.

\textbf{Azioni}

\emph{\textbf{Morso.} Attacco con Arma da Mischia}: +4 a colpire, portata 1 m, una creatura.

\emph{Colpisce:} 7 (1d10 + 2) danni perforanti, e il bersaglio è afferrato (DC 12 per fuggire). Fino al termine dell'afferrare il coccodrillo non può usare il morso contro un altro bersaglio.

\mostro{Coccodrillo Gigante}
\begin{description}[noitemsep, topsep=0pt, parsep=0pt, partopsep=0pt, leftmargin=0cm, labelwidth=2.2cm]
    \item[\textbf{Taglia/Tipo:}] Enorme bestia, disallineato
    \item[\textbf{Caratt.:}] \resizebox{0.5\linewidth+1.8cm}{!}{For 5 Des -1 Cos 3 Int -4 Sag 0 Car -2}
    \item[\textbf{Punti Ferita:}] 108,  \textbf{Difesa:} 17,  \textbf{Iniziativa:} -1
    \item[\textbf{Tiri Salvez.:}] \resizebox{0.5\linewidth+1.8cm}{!}{Tempra +8, Riflessi +4, Volontà +5}
    \item[\textbf{Movimento:}] 9 m, nuoto 15 m
    \item[\textbf{Sfida:}] 5 (1800 PX)\smallskip
\end{description}

\emph{\textbf{Trattenere il Fiato.}} Il coccodrillo può trattenere il fiato per 30 minuti.

\textbf{Azioni}

\emph{\textbf{Multiattacco.}} Il coccodrillo effettua due attacchi: uno con il morso e uno con la coda.

\emph{\textbf{Coda.} Attacco con Arma da Mischia}: +8 a colpire, portata 3 m, un bersaglio non afferrato dal coccodrillo.

\emph{Colpisce:} 14 (2d8 + 5) danni contundenti. Se il bersaglio è una creatura, deve riuscire un Tiro Salvezza di Tempra DC 16 o cadere prono.

\emph{\textbf{Morso.} Attacco con Arma da Mischia}: +8 a colpire, portata 1 m, un bersaglio.

\emph{Colpisce:} 21 (3d10 + 5) danni perforanti, e il bersaglio è afferrato (DC 16 per fuggire). Fino al termine dell'afferrare il coccodrillo non può usare il morso contro un altro bersaglio.

\mostro{Corvo}
\begin{description}[noitemsep, topsep=0pt, parsep=0pt, partopsep=0pt, leftmargin=0cm, labelwidth=2.2cm]
    \item[\textbf{Taglia/Tipo:}] Minuscola bestia, disallineato
    \item[\textbf{Caratt.:}] \resizebox{0.5\linewidth+1.8cm}{!}{For -4 Des 2 Cos -1 Int -4 Sag 1 Car -2}
    \item[\textbf{Punti Ferita:}] 15,  \textbf{Difesa:} 14,  \textbf{Iniziativa:} +2
    \item[\textbf{Tiri Salvez.:}] \resizebox{0.5\linewidth+1.8cm}{!}{Tempra +3, Riflessi +3, Volontà +3}
    \item[\textbf{Movimento:}] 3 m, volo 15 m
    \item[\textbf{Sfida:}] 0(10 PX)\smallskip
\end{description}

\emph{\textbf{Imitazione.}} Il corvo può imitare dei semplici suoni che ha udito, come il sussurro di una persona, il pianto di un bambino o il verso di un animale. Una creatura che ode il suono può identificarlo come imitazione riuscendo una prova Sopravvivenza DC 10.

\textbf{Azioni}

\emph{\textbf{Becco.} Attacco con Arma da Mischia}: +4 a colpire, portata 1 m, un bersaglio.

\emph{Colpisce:} 1 danno perforante.

\mostro{Donnola Gigante}
\begin{description}[noitemsep, topsep=0pt, parsep=0pt, partopsep=0pt, leftmargin=0cm, labelwidth=2.2cm]
    \item[\textbf{Taglia/Tipo:}] Media bestia, disallineato
    \item[\textbf{Caratt.:}] \resizebox{0.5\linewidth+1.8cm}{!}{For 0 Des 3 Cos 0 Int -3 Sag 1 Car -3}
    \item[\textbf{Punti Ferita:}] 17,  \textbf{Difesa:} 15,  \textbf{Iniziativa:} +3
    \item[\textbf{Tiri Salvez.:}] \resizebox{0.5\linewidth+1.8cm}{!}{Tempra +3, Riflessi +3, Volontà +3}
    \item[\textbf{Movimento:}] 12 m
    \item[\textbf{Sfida:}] 1/8 (25 PX)\smallskip
\end{description}

\emph{\textbf{Udito e Olfatto Affinati.}} La donnola ha +1d6 nelle prove di Consapevolezza basate su udito o olfatto.

\textbf{Azioni}

\emph{\textbf{Morso.} Attacco con Arma da Mischia}: +5 a colpire, portata 1 m, un bersaglio.

\emph{Colpisce:} 5 (1d4 + 3) danni perforanti.

\mostro{Elefante}
\begin{description}[noitemsep, topsep=0pt, parsep=0pt, partopsep=0pt, leftmargin=0cm, labelwidth=2.2cm]
    \item[\textbf{Taglia/Tipo:}] Enorme bestia, disallineato
    \item[\textbf{Caratt.:}] \resizebox{0.5\linewidth+1.8cm}{!}{For 6 Des -1 Cos 3 Int -4 Sag 0 Car -2}
    \item[\textbf{Punti Ferita:}] 89,  \textbf{Difesa:} 16,  \textbf{Iniziativa:} -1
    \item[\textbf{Tiri Salvez.:}] \resizebox{0.5\linewidth+1.8cm}{!}{Tempra +7, Riflessi +3, Volontà +4}
    \item[\textbf{Movimento:}] 12 m
    \item[\textbf{Sfida:}] 4 (1000 PX)\smallskip
\end{description}

\emph{\textbf{Carica Travolgente.}} Se l'elefante si muove di almeno 6 metri diretto verso una creatura e la colpisce con un attacco di incornata durante lo stesso round, il bersaglio deve riuscire un Tiro Salvezza su Tempra DC 16 o cadere prono. Se il bersaglio è prono, l'elefante può effettuare un attacco di pestone contro di lui come Azione Immediata.

\textbf{Azioni}

\emph{\textbf{Incornata.} Attacco con Arma da Mischia}: +6 a colpire, portata 1 m, un bersaglio.

\emph{Colpisce:} 19 (3d8 + 6) danni perforanti.

\emph{\textbf{Pestone.} Attacco con Arma da Mischia}: +6 a colpire, portata 1 m, un bersaglio prono.

\emph{Colpisce:} 22 (3d10 + 6) danni contundenti.

\mostro{Falco}
\begin{description}[noitemsep, topsep=0pt, parsep=0pt, partopsep=0pt, leftmargin=0cm, labelwidth=2.2cm]
    \item[\textbf{Taglia/Tipo:}] Minuscola bestia, disallineato
    \item[\textbf{Caratt.:}] \resizebox{0.5\linewidth+1.8cm}{!}{For -3 Des 3 Cos -1 Int -4 Sag 2 Car -2}
    \item[\textbf{Punti Ferita:}] 15,  \textbf{Difesa:} 15,  \textbf{Iniziativa:} +3
    \item[\textbf{Tiri Salvez.:}] \resizebox{0.5\linewidth+1.8cm}{!}{Tempra +3, Riflessi +3, Volontà +3}
    \item[\textbf{Movimento:}] 3 m, volo 18 m
    \item[\textbf{Sfida:}] 0(10 PX)\smallskip
\end{description}

\emph{\textbf{Vista Affinata.}} Il falco ha +1d6 alle prove di Consapevolezza basate sulla vista.

\textbf{Azioni}

\emph{\textbf{Speroni.} Attacco con Arma da Mischia}: +5 a colpire, portata 1 m, un bersaglio.

\emph{Colpisce:} 1 danno tagliente.

\mostro{Falco di Sangue}
\begin{description}[noitemsep, topsep=0pt, parsep=0pt, partopsep=0pt, leftmargin=0cm, labelwidth=2.2cm]
    \item[\textbf{Taglia/Tipo:}] Piccola bestia, disallineato
    \item[\textbf{Caratt.:}] \resizebox{0.5\linewidth+1.8cm}{!}{For -2 Des 2 Cos 0 Int -4 Sag 2 Car -3}
    \item[\textbf{Punti Ferita:}] 17,  \textbf{Difesa:} 14,  \textbf{Iniziativa:} +2
    \item[\textbf{Tiri Salvez.:}] \resizebox{0.5\linewidth+1.8cm}{!}{Tempra +3, Riflessi +3, Volontà +3}
    \item[\textbf{Movimento:}] 3 m, volo 18 m
    \item[\textbf{Sfida:}] 1/8 (25 PX)\smallskip
\end{description}

\emph{\textbf{Tattiche di Branco.}} Il falco ha +1d6 ai tiri di attacco contro una creatura se almeno uno degli alleati del falco si trova entro 1 metro dalla creatura e quell'alleato non è inabile.

\emph{\textbf{Vista Affinata.}} Il falco ha +1d6 alle prove di Consapevolezza basate sulla vista.

\textbf{Azioni}

\emph{\textbf{Becco.} Attacco con Arma da Mischia}: +3 a colpire, portata 1 m, un bersaglio.

\emph{Colpisce:} 4 (1d4 + 2) danni perforanti.

\mostro{Pirana}
\begin{description}[noitemsep, topsep=0pt, parsep=0pt, partopsep=0pt, leftmargin=0cm, labelwidth=2.2cm]
    \item[\textbf{Taglia/Tipo:}] Minuscola bestia, disallineato
    \item[\textbf{Caratt.:}] \resizebox{0.5\linewidth+1.8cm}{!}{For -4 Des 3 Cos -1 Int -5 Sag -2 Car -4}
    \item[\textbf{Punti Ferita:}] 15,  \textbf{Difesa:} 15,  \textbf{Iniziativa:} +3
    \item[\textbf{Tiri Salvez.:}] \resizebox{0.5\linewidth+1.8cm}{!}{Tempra +3, Riflessi +3, Volontà +3}
    \item[\textbf{Movimento:}] 0 m, nuoto 12 m
    \item[\textbf{Sfida:}] 0(10 PX)\smallskip
\end{description}

\emph{\textbf{Frenesia Sanguinaria.}} Il pirana ha +1d6 ai tiri di attacco in mischia contro qualsiasi creatura che non sia al massimo dei Punti Ferita.

\emph{\textbf{Respirare Acqua.}} Il pirana può respirare solo sott'acqua.

\textbf{Azioni}

\emph{\textbf{Morso.} Attacco con Arma da Mischia}: +5 a colpire, portata 1 m, un bersaglio.

\emph{Colpisce:} 1 danno perforante.

\mostro{Gatto}
\begin{description}[noitemsep, topsep=0pt, parsep=0pt, partopsep=0pt, leftmargin=0cm, labelwidth=2.2cm]
    \item[\textbf{Taglia/Tipo:}] Minuscola bestia, disallineato
    \item[\textbf{Caratt.:}] \resizebox{0.5\linewidth+1.8cm}{!}{For -4 Des 2 Cos 0 Int -4 Sag 1 Car -2}
    \item[\textbf{Punti Ferita:}] 15,  \textbf{Difesa:} 14,  \textbf{Iniziativa:} +2
    \item[\textbf{Tiri Salvez.:}] \resizebox{0.5\linewidth+1.8cm}{!}{Tempra +3, Riflessi +3, Volontà +3}
    \item[\textbf{Movimento:}] 12 m, scalata 9 m
    \item[\textbf{Sfida:}] 0(10 PX)\smallskip
\end{description}

\emph{\textbf{Olfatto Affinato.}} Il gatto ha +1d6 alle prove di Consapevolezza basate sull'olfatto.

\textbf{Azioni}

\emph{\textbf{Artigli.} Attacco con Arma da Mischia}: +2 a colpire, portata 1 m, un bersaglio.

\emph{Colpisce:} 1 danno tagliente.

\mostro{Granchio Gigante}
\begin{description}[noitemsep, topsep=0pt, parsep=0pt, partopsep=0pt, leftmargin=0cm, labelwidth=2.2cm]
    \item[\textbf{Taglia/Tipo:}] Media bestia, disallineato
    \item[\textbf{Caratt.:}] \resizebox{0.5\linewidth+1.8cm}{!}{For 1 Des 2 Cos 0 Int -5 Sag -1 Car -4}
    \item[\textbf{Punti Ferita:}] 17,  \textbf{Difesa:} 14,  \textbf{Iniziativa:} +2
    \item[\textbf{Tiri Salvez.:}] \resizebox{0.5\linewidth+1.8cm}{!}{Tempra +3, Riflessi +3, Volontà +3}
    \item[\textbf{Movimento:}] 9 m, nuoto 9 m
    \item[\textbf{Sfida:}] 1/8 (25 PX)\smallskip
\end{description}

\emph{\textbf{Anfibio.}} Il granchio può respirare aria e acqua.

\textbf{Azioni}

\emph{\textbf{Artiglio (Chela).} Attacco con Arma da Mischia}: +3 a colpire, portata 1 m, un bersaglio.

\emph{Colpisce:} 4 (1d6 + 1) danni contundenti e il bersaglio è afferrato (DC 11 per fuggire). Il granchio ha due chele, ciascuna delle quali può afferrare un solo bersaglio.

\mostro{Gufo}
\begin{description}[noitemsep, topsep=0pt, parsep=0pt, partopsep=0pt, leftmargin=0cm, labelwidth=2.2cm]
    \item[\textbf{Taglia/Tipo:}] Minuscola bestia, disallineato
    \item[\textbf{Caratt.:}] \resizebox{0.5\linewidth+1.8cm}{!}{For -4 Des 1 Cos -1 Int -4 Sag 1 Car -2}
    \item[\textbf{Punti Ferita:}] 15,  \textbf{Difesa:} 13,  \textbf{Iniziativa:} +1
    \item[\textbf{Tiri Salvez.:}] \resizebox{0.5\linewidth+1.8cm}{!}{Tempra +3, Riflessi +3, Volontà +3}
    \item[\textbf{Movimento:}] 1 m, volo 18 m
    \item[\textbf{Sfida:}] 0(10 PX)\smallskip
\end{description}

\emph{\textbf{Sorvolare.}} Il gufo non provoca attacchi di opportunità quando vola via dalla portata di un nemico.

\emph{\textbf{Udito e Vista Affinati.}} Il gufo ha +1d6 nelle prove di Consapevolezza basate su udito o vista.

\textbf{Azioni}

\emph{\textbf{Speroni.} Attacco con Arma da Mischia}: +3 a colpire, portata 1 m, un bersaglio.

\emph{Colpisce:} 1 danno tagliente.

\mostro{Gufo Gigante}
\begin{description}[noitemsep, topsep=0pt, parsep=0pt, partopsep=0pt, leftmargin=0cm, labelwidth=2.2cm]
    \item[\textbf{Taglia/Tipo:}] Grande bestia, neutrale
    \item[\textbf{Caratt.:}] \resizebox{0.5\linewidth+1.8cm}{!}{For 1 Des 2 Cos 1 Int -1 Sag 1 Car 0}
    \item[\textbf{Punti Ferita:}] 19,  \textbf{Difesa:} 14,  \textbf{Iniziativa:} +2
    \item[\textbf{Tiri Salvez.:}] \resizebox{0.5\linewidth+1.8cm}{!}{Tempra +3, Riflessi +3, Volontà +3}
    \item[\textbf{Movimento:}] 1 m, volo 18 m
    \item[\textbf{Sfida:}] 1/4 (50 PX)\smallskip
\end{description}

\emph{\textbf{Sorvolare.}} Il gufo non provoca attacchi di opportunità quando vola via dalla portata di un nemico.

\emph{\textbf{Udito e Vista Affinati.}} Il gufo ha +1d6 nelle prove di Consapevolezza basate su udito o vista.

\textbf{Azioni}

\emph{\textbf{Speroni.} Attacco con Arma da Mischia}: +4 a colpire, portata 1 m, un bersaglio.

\emph{Colpisce:} 8 (2d6 + 1) danni perforanti.

\mostro{Iena}
\begin{description}[noitemsep, topsep=0pt, parsep=0pt, partopsep=0pt, leftmargin=0cm, labelwidth=2.2cm]
    \item[\textbf{Taglia/Tipo:}] Media bestia, disallineato
    \item[\textbf{Caratt.:}] \resizebox{0.5\linewidth+1.8cm}{!}{For 0 Des 1 Cos 1 Int -4 Sag 1 Car -3}
    \item[\textbf{Punti Ferita:}] 15,  \textbf{Difesa:} 13,  \textbf{Iniziativa:} +1
    \item[\textbf{Tiri Salvez.:}] \resizebox{0.5\linewidth+1.8cm}{!}{Tempra +3, Riflessi +3, Volontà +3}
    \item[\textbf{Movimento:}] 15 m
    \item[\textbf{Sfida:}] 0(10 PX)\smallskip
\end{description}

\emph{\textbf{Tattiche di Branco.}} La iena ha +1d6 ai tiri di attacco contro una creatura se almeno uno degli alleati della iena si trova entro 1 metro dalla creatura e quell'alleato non è inabile.

\textbf{Azioni}

\emph{\textbf{Morso.} Attacco con Arma da Mischia}: +3 a colpire, portata 1 m, un bersaglio.

\emph{Colpisce:} 3 (1d6) danni perforanti.

\mostro{Iena Gigante}
\begin{description}[noitemsep, topsep=0pt, parsep=0pt, partopsep=0pt, leftmargin=0cm, labelwidth=2.2cm]
    \item[\textbf{Taglia/Tipo:}] Grande bestia, disallineato
    \item[\textbf{Caratt.:}] \resizebox{0.5\linewidth+1.8cm}{!}{For 3 Des 2 Cos 2 Int -4 Sag 1 Car -2}
    \item[\textbf{Punti Ferita:}] 33,  \textbf{Difesa:} 15,  \textbf{Iniziativa:} +2
    \item[\textbf{Tiri Salvez.:}] \resizebox{0.5\linewidth+1.8cm}{!}{Tempra +3, Riflessi +3, Volontà +3}
    \item[\textbf{Movimento:}] 15 m
    \item[\textbf{Sfida:}] 1 (200 PX)\smallskip
\end{description}

\emph{\textbf{Rabbia.}} Quando la iena riduce una creatura a 0 Punti Ferita con un attacco di mischia durante il proprio round, la iena può svolgere una Azione Immediata muoversi fino a metà del suo movimento ed effettuare un attacco di morso.

\textbf{Azioni}

\emph{\textbf{Morso.} Attacco con Arma da Mischia}: +5 a colpire, portata 1 m, un bersaglio.

\emph{Colpisce:} 10 (2d6 + 3) danni perforanti.

\mostro{Leone}
\begin{description}[noitemsep, topsep=0pt, parsep=0pt, partopsep=0pt, leftmargin=0cm, labelwidth=2.2cm]
    \item[\textbf{Taglia/Tipo:}] Grande bestia, disallineato
    \item[\textbf{Caratt.:}] \resizebox{0.5\linewidth+1.8cm}{!}{For 3 Des 2 Cos 1 Int -4 Sag 1 Car -1}
    \item[\textbf{Punti Ferita:}] 33,  \textbf{Difesa:} 15,  \textbf{Iniziativa:} +2
    \item[\textbf{Tiri Salvez.:}] \resizebox{0.5\linewidth+1.8cm}{!}{Tempra +3, Riflessi +3, Volontà +3}
    \item[\textbf{Movimento:}] 15 m
    \item[\textbf{Sfida:}] 1 (200 PX)\smallskip
\end{description}

\emph{\textbf{Balzo.}} Se il leone si muove di almeno 6 metri diretto verso una creatura e la colpisce con un attacco di artiglio durante lo stesso round, il bersaglio deve riuscire un Tiro Salvezza di Tempra DC 13 o cadere prono. Se il bersaglio è prono, il leone può effettuare un attacco di morso come Azione Immediata.

\emph{\textbf{Olfatto Affinato.}} Il leone ha +1d6 alle prove di Consapevolezza basate sull'olfatto.

\emph{\textbf{Salto con Rincorsa.}} Con 3 metri di rincorsa, il leone può saltare in lungo fino a 7 metri.

\emph{\textbf{Tattiche di Branco.}} Il leone ha +1d6 ai tiri di attacco contro una creatura se almeno uno degli alleati del leone si trova entro 1 metro dalla creatura e quell'alleato non è inabile.

\textbf{Azioni}

\emph{\textbf{Artiglio.} Attacco con Arma da Mischia}: +5 a colpire, portata 1 m, un bersaglio.

\emph{Colpisce:} 6 (1d6 + 3) danni taglienti, 1 danno da Sanguinamento.

\emph{\textbf{Morso.} Attacco con Arma da Mischia}: +5 a colpire, portata 1 m, un bersaglio.

\emph{Colpisce:} 7 (1d8 + 3) danni perforanti.

\mostro{Lucertola Gigante}
\begin{description}[noitemsep, topsep=0pt, parsep=0pt, partopsep=0pt, leftmargin=0cm, labelwidth=2.2cm]
    \item[\textbf{Taglia/Tipo:}] Grande bestia, disallineato
    \item[\textbf{Caratt.:}] \resizebox{0.5\linewidth+1.8cm}{!}{For 2 Des 1 Cos 1 Int -4 Sag 0 Car -3}
    \item[\textbf{Punti Ferita:}] 19,  \textbf{Difesa:} 13,  \textbf{Iniziativa:} +1
    \item[\textbf{Tiri Salvez.:}] \resizebox{0.5\linewidth+1.8cm}{!}{Tempra +3, Riflessi +3, Volontà +3}
    \item[\textbf{Movimento:}] 9 m, scalata 9 m
    \item[\textbf{Sfida:}] 1/4 (50 PX)\smallskip
\end{description}

\textbf{Azioni}

\emph{\textbf{Morso.} Attacco con Arma da Mischia}: +4 a colpire, portata 1 m, un bersaglio.

\emph{Colpisce:} 6 (1d8 + 2) danni perforanti.

\textbf{VARIANTE}

Alcune lucertole giganti possiedono uno o entrambi i seguenti tratti.

\emph{\textbf{Scalare come Ragno.}} La lucertola può scalare superfici difficili, compreso lo stare a testa in giù sul soffitto, senza bisogno di effettuare una prova di competenza.

\emph{\textbf{Trattenere il Fiato.}} La lucertola può trattenere il fiato per 15 minuti. (Una lucertola con questo tratto possiede anche velocità di nuoto 9 metri).

\mostro{Lupo}
\begin{description}[noitemsep, topsep=0pt, parsep=0pt, partopsep=0pt, leftmargin=0cm, labelwidth=2.2cm]
    \item[\textbf{Taglia/Tipo:}] Media bestia, disallineato
    \item[\textbf{Caratt.:}] \resizebox{0.5\linewidth+1.8cm}{!}{For 1 Des 2 Cos 1 Int -4 Sag 1 Car -2}
    \item[\textbf{Punti Ferita:}] 19,  \textbf{Difesa:} 14,  \textbf{Iniziativa:} +2
    \item[\textbf{Tiri Salvez.:}] \resizebox{0.5\linewidth+1.8cm}{!}{Tempra +3, Riflessi +3, Volontà +3}
    \item[\textbf{Movimento:}] 12 m
    \item[\textbf{Sfida:}] 1/4 (50 PX)\smallskip
\end{description}

\emph{\textbf{Udito e Olfatto Affinato.}} Il lupo ha +1d6 nelle prove di Consapevolezza basate su udito o olfatto.

\emph{\textbf{Tattiche di Branco.}} Il lupo ha +1d6 ai tiri di attacco contro una creatura se almeno uno degli alleati del lupo si trova entro 1 metro dalla creatura e quell'alleato non è inabile.

\textbf{Azioni}

\emph{\textbf{Morso.} Attacco con Arma da Mischia}: +4 a colpire, portata 1 m, un bersaglio.

\emph{Colpisce:} 7 (2d4 + 2) danni perforanti. Se il bersaglio è una creatura, deve riuscire un Tiro Salvezza di Tempra DC 11 o cadere prona.

\mostro{Dinolupo (Metalupo)}
\begin{description}[noitemsep, topsep=0pt, parsep=0pt, partopsep=0pt, leftmargin=0cm, labelwidth=2.2cm]
    \item[\textbf{Taglia/Tipo:}] Grande bestia, disallineato
    \item[\textbf{Caratt.:}] \resizebox{0.5\linewidth+1.8cm}{!}{For 3 Des 2 Cos 2 Int -2 Sag 1 Car -2}
    \item[\textbf{Punti Ferita:}] 33,  \textbf{Difesa:} 15,  \textbf{Iniziativa:} +2
    \item[\textbf{Tiri Salvez.:}] \resizebox{0.5\linewidth+1.8cm}{!}{Tempra +3, Riflessi +3, Volontà +3}
    \item[\textbf{Movimento:}] 15 m
    \item[\textbf{Sfida:}] 1 (200 PX)\smallskip
\end{description}

\emph{\textbf{Udito e Olfatto Affinato.}} Il lupo ha +1d6 nelle prove di Consapevolezza basate su udito o olfatto.

\emph{\textbf{Tattiche di Branco.}} Il lupo ha +1d6 ai tiri di attacco contro una creatura se almeno uno degli alleati del lupo si trova entro 1 metro dalla creatura e quell'alleato non è inabile.

\textbf{Azioni}

\emph{\textbf{Morso.} Attacco con Arma da Mischia}: +5 a colpire, portata 1 m, un bersaglio.

\emph{Colpisce:} 10 (2d6 + 3) danni perforanti. Se il bersaglio è una creatura, deve riuscire un Tiro Salvezza di Tempra DC 13 o cadere prona.

\mostro{Lupo Invernale}
\begin{description}[noitemsep, topsep=0pt, parsep=0pt, partopsep=0pt, leftmargin=0cm, labelwidth=2.2cm]
    \item[\textbf{Taglia/Tipo:}] Grande mostruosità, malvagio
    \item[\textbf{Caratt.:}] \resizebox{0.5\linewidth+1.8cm}{!}{For 4 Des 1 Cos 2 Int -2 Sag 1 Car -1}
    \item[\textbf{Punti Ferita:}] 70,  \textbf{Difesa:} 17,  \textbf{Iniziativa:} +1
    \item[\textbf{Tiri Salvez.:}] \resizebox{0.5\linewidth+1.8cm}{!}{Tempra +5, Riflessi +4, Volontà +4}
    \item[\textbf{Movimento:}] 15 m
    \item[\textbf{Sfida:}] 3 (700 PX)\smallskip
\end{description}

\emph{\textbf{Camuffamento di Neve.}} Il lupo ha +1d6 alle prove di Furtività (Nascondersi) effettuate per nascondersi su terreno innevato.

\emph{\textbf{Udito e Olfatto Affinato.}} Il lupo ha +1d6 nelle prove di Consapevolezza basate su udito o olfatto.

\emph{\textbf{Tattiche di Branco.}} Il lupo ha +1d6 ai tiri di attacco contro una creatura se almeno uno degli alleati del lupo si trova entro 1 metro dalla creatura e quell'alleato non è inabile.

\textbf{Azioni}

\emph{\textbf{Morso.} Attacco con Arma da Mischia}: +6 a colpire, portata 1 m, un bersaglio.

\emph{Colpisce:} 11 (2d6 + 4) danni perforanti. Se il bersaglio è una creatura, deve riuscire un Tiro Salvezza di Tempra DC 14 o cadere prona.

\emph{\textbf{Soffio Gelido (Ricarica 5-6).}} Il lupo esala un'esplosione di vento gelido in un cono di 5 metri. Ogni creatura in quell'area deve effettuare un Tiro Salvezza di Riflessi DC 15, e subire 18 (4d8) danni da freddo se fallisce il Tiro Salvezza, o la metà di questi danni se lo riesce.

\mostro{Mammut}
\begin{description}[noitemsep, topsep=0pt, parsep=0pt, partopsep=0pt, leftmargin=0cm, labelwidth=2.2cm]
    \item[\textbf{Taglia/Tipo:}] Enorme bestia, disallineato
    \item[\textbf{Caratt.:}] \resizebox{0.5\linewidth+1.8cm}{!}{For 7 Des -1 Cos 5 Int -4 Sag 0 Car -2}
    \item[\textbf{Punti Ferita:}] 129,  \textbf{Difesa:} 19,  \textbf{Iniziativa:} -1
    \item[\textbf{Tiri Salvez.:}] \resizebox{0.5\linewidth+1.8cm}{!}{Tempra +11, Riflessi +5, Volontà +6}
    \item[\textbf{Movimento:}] 12 m
    \item[\textbf{Sfida:}] 6 (2300 PX)\smallskip
\end{description}

\emph{\textbf{Carica Travolgente.}} Se il mammut si muove di almeno 6 metri diretto verso una creatura e la colpisce con un attacco di incornata durante lo stesso round, il bersaglio deve riuscire un Tiro Salvezza su Tempra DC 18 o cadere prono. Se il bersaglio è prono, il mammut può effettuare un attacco di pestone contro di lui come Azione Immediata.

\textbf{Azioni}

\emph{\textbf{Incornata.} Attacco con Arma da Mischia}: +8 a colpire, portata 3 m, un bersaglio.

\emph{Colpisce:} 25 (4d8 + 7) danni perforanti.

\emph{\textbf{Pestone.} Attacco con Arma da Mischia}: +8 a colpire, portata 1 m, una creatura prona.

\emph{Colpisce:} 29 (4d10 + 7) danni contundenti.

\mostro{Mastino}
\begin{description}[noitemsep, topsep=0pt, parsep=0pt, partopsep=0pt, leftmargin=0cm, labelwidth=2.2cm]
    \item[\textbf{Taglia/Tipo:}] Media bestia, disallineato
    \item[\textbf{Caratt.:}] \resizebox{0.5\linewidth+1.8cm}{!}{For 1 Des 2 Cos 1 Int -4 Sag 1 Car -2}
    \item[\textbf{Punti Ferita:}] 17,  \textbf{Difesa:} 14,  \textbf{Iniziativa:} +2
    \item[\textbf{Tiri Salvez.:}] \resizebox{0.5\linewidth+1.8cm}{!}{Tempra +3, Riflessi +3, Volontà +3}
    \item[\textbf{Movimento:}] 12 m
    \item[\textbf{Sfida:}] 1/8 (25 PX)\smallskip
\end{description}

\emph{\textbf{Udito e Olfatto Affinato.}} Il mastino ha +1d6 nelle prove di Consapevolezza basate su udito o olfatto.

\textbf{Azioni}

\emph{\textbf{Morso.} Attacco con Arma da Mischia}: +3 a colpire, portata 1 m, un bersaglio.

\emph{Colpisce:} 4 (1d6 + 1) danni perforanti. Se il bersaglio è una creatura, deve riuscire un Tiro Salvezza di Tempra DC 11 o cadere prono.

\mostro{Orso Bruno}
\begin{description}[noitemsep, topsep=0pt, parsep=0pt, partopsep=0pt, leftmargin=0cm, labelwidth=2.2cm]
    \item[\textbf{Taglia/Tipo:}] Grande bestia, disallineato
    \item[\textbf{Caratt.:}] \resizebox{0.5\linewidth+1.8cm}{!}{For 4 Des 0 Cos 3 Int -4 Sag 1 Car -2}
    \item[\textbf{Punti Ferita:}] 33,  \textbf{Difesa:} 13,  \textbf{Iniziativa:} +0
    \item[\textbf{Tiri Salvez.:}] \resizebox{0.5\linewidth+1.8cm}{!}{Tempra +4, Riflessi +3, Volontà +3}
    \item[\textbf{Movimento:}] 12 m, scalata 9 m
    \item[\textbf{Sfida:}] 1 (200 PX)\smallskip
\end{description}

\emph{\textbf{Olfatto Affinato.}} L'orso ha +1d6 alle prove di Consapevolezza basate sull'olfatto.

\textbf{Azioni}

\emph{\textbf{Multiattacco.}} L'orso effettua due attacchi: uno con il morso e uno con gli artigli.

\emph{\textbf{Artigli.} Attacco con Arma da Mischia}: +5 a colpire, portata 1 m, un bersaglio.

\emph{Colpisce:} 11 (2d6 + 4) danni taglienti.

\emph{\textbf{Morso.} Attacco con Arma da Mischia}: +5 a colpire, portata 1 m, un bersaglio.

\emph{Colpisce:} 8 (1d8 + 4) danni perforanti.

\mostro{Orso Nero}
\begin{description}[noitemsep, topsep=0pt, parsep=0pt, partopsep=0pt, leftmargin=0cm, labelwidth=2.2cm]
    \item[\textbf{Taglia/Tipo:}] Media bestia, disallineato
    \item[\textbf{Caratt.:}] \resizebox{0.5\linewidth+1.8cm}{!}{For 3 Des 0 Cos 2 Int -4 Sag 1 Car -2}
    \item[\textbf{Punti Ferita:}] 24,  \textbf{Difesa:} 12,  \textbf{Iniziativa:} +0
    \item[\textbf{Tiri Salvez.:}] \resizebox{0.5\linewidth+1.8cm}{!}{Tempra +3, Riflessi +3, Volontà +3}
    \item[\textbf{Movimento:}] 12 m, scalata 9 m
    \item[\textbf{Sfida:}] 1/2 (100 PX)\smallskip
\end{description}

\emph{\textbf{Olfatto Affinato.}} L'orso ha +1d6 alle prove di Consapevolezza basate sull'olfatto.

\textbf{Azioni}

\emph{\textbf{Multiattacco.}} L'orso nero effettua due attacchi: uno con il morso e uno con gli artigli.

\emph{\textbf{Artigli.} Attacco con Arma da Mischia}: +4 a colpire, portata 1 m, un bersaglio.

\emph{Colpisce:} 7 (2d4 + 3) danni taglienti, 1 danno da Sanguinamento.

\emph{\textbf{Morso.} Attacco con Arma da Mischia}: +5 a colpire, portata 1 m, un bersaglio.

\emph{Colpisce:} 6 (1d6 + 3) danni perforanti.

\mostro{Orso Polare}
\begin{description}[noitemsep, topsep=0pt, parsep=0pt, partopsep=0pt, leftmargin=0cm, labelwidth=2.2cm]
    \item[\textbf{Taglia/Tipo:}] Grande bestia, disallineato
    \item[\textbf{Caratt.:}] \resizebox{0.5\linewidth+1.8cm}{!}{For 5 Des 0 Cos 3 Int -4 Sag 1 Car -2}
    \item[\textbf{Punti Ferita:}] 52,  \textbf{Difesa:} 14,  \textbf{Iniziativa:} +0
    \item[\textbf{Tiri Salvez.:}] \resizebox{0.5\linewidth+1.8cm}{!}{Tempra +5, Riflessi +3, Volontà +3}
    \item[\textbf{Movimento:}] 12 m, nuoto 9 m
    \item[\textbf{Sfida:}] 2 (450 PX)\smallskip
\end{description}

\emph{\textbf{Olfatto Affinato.}} L'orso ha +1d6 alle prove di Consapevolezza basate sull'olfatto.

\textbf{Azioni}

\emph{\textbf{Multiattacco.}} L'orso effettua due attacchi: uno con il morso e uno con gli artigli.

\emph{\textbf{Artigli.} Attacco con Arma da Mischia}: +5 a colpire, portata 1 m, un bersaglio.

\emph{Colpisce:} 12 (2d6 + 5) danni taglienti.

\emph{\textbf{Morso.} Attacco con Arma da Mischia}: +5 a colpire, portata 1 m, un bersaglio.

\emph{Colpisce:} 9 (1d8 + 5) danni perforanti.

\mostro{Orso Corazzato}
\begin{description}[noitemsep, topsep=0pt, parsep=0pt, partopsep=0pt, leftmargin=0cm, labelwidth=2.2cm]
	\item[\textbf{Taglia/Tipo:}] Enorme bestia, corrotta da Cattalm
	\item[\textbf{Caratt.:}] \resizebox{0.5\linewidth+1.8cm}{!}{For 7 Des 2 Cos 4 Int 1 Sag 1 Car 1}
	\item[\textbf{Punti Ferita:}] 90,  \textbf{Difesa:} 19,  \textbf{Iniziativa:} +2
	\item[\textbf{Tiri Salvez.:}] \resizebox{0.5\linewidth+1.8cm}{!}{Tempra +8, Riflessi +6, Volontà +5}
	\item[\textbf{Movimento:}] 12 m, nuoto 9 m
	\item[\textbf{Sfida:}] 4 (450 PX)\smallskip
\end{description}

\emph{\textbf{Olfatto Affinato.}} L'orso ha +1d6 alle prove di Consapevolezza basate sull'olfatto.

\textbf{Azioni}

\emph{\textbf{Multiattacco.}} L'orso effettua due attacchi: uno con il morso e uno con gli artigli.

\emph{\textbf{Artigli.} Attacco con Arma da Mischia}: +9 a colpire, portata 2 m, un bersaglio.

\emph{Colpisce:} 17 (3d8 + 7) danni taglienti.

\emph{\textbf{Morso.} Attacco con Arma da Mischia}: +8 a colpire, portata 2 m, un bersaglio.

\emph{Colpisce:} 20 (3d8 + 10) danni perforanti.

\mostro{Pantera}
\begin{description}[noitemsep, topsep=0pt, parsep=0pt, partopsep=0pt, leftmargin=0cm, labelwidth=2.2cm]
    \item[\textbf{Taglia/Tipo:}] Media bestia, disallineato
    \item[\textbf{Caratt.:}] \resizebox{0.5\linewidth+1.8cm}{!}{For 2 Des 2 Cos 0 Int -4 Sag 2 Car -2}
    \item[\textbf{Punti Ferita:}] 19,  \textbf{Difesa:} 14,  \textbf{Iniziativa:} +2
    \item[\textbf{Tiri Salvez.:}] \resizebox{0.5\linewidth+1.8cm}{!}{Tempra +3, Riflessi +3, Volontà +3}
    \item[\textbf{Movimento:}] 15 m, scalata 12 m
    \item[\textbf{Sfida:}] 1/4 (50 PX)\smallskip
\end{description}

\emph{\textbf{Balzo.}} Se la pantera si muove di almeno 6 metri diretta verso una creatura e la colpisce con un attacco di artiglio durante lo stesso round, il bersaglio deve riuscire un Tiro Salvezza di Tempra DC 12 o cadere prono. Se il bersaglio è prono, la pantera può effettuare un attacco di morso contro di esso come Azione Immediata.

\emph{\textbf{Olfatto Affinato.}} La pantera ha +1d6 alle prove di Consapevolezza basate sull'olfatto.

\textbf{Azioni}

\emph{\textbf{Artiglio.} Attacco con Arma da Mischia}: +4 a colpire, portata 1 m, un bersaglio.

\emph{Colpisce:} 4 (1d4 + 2) danni taglienti, 1 danno da Sanguinamento.

\emph{\textbf{Morso.} Attacco con Arma da Mischia}: +4 a colpire, portata 1 m, un bersaglio.

\emph{Colpisce:} 5 (1d6 + 2) danni perforanti.

\mostro{Ragno}
\begin{description}[noitemsep, topsep=0pt, parsep=0pt, partopsep=0pt, leftmargin=0cm, labelwidth=2.2cm]
    \item[\textbf{Taglia/Tipo:}] Minuscola bestia, disallineato
    \item[\textbf{Caratt.:}] \resizebox{0.5\linewidth+1.8cm}{!}{For 2 (-5) Des 2 Cos -1 Int -5 Sag 0 Car -4}
    \item[\textbf{Punti Ferita:}] 15,  \textbf{Difesa:} 14,  \textbf{Iniziativa:} +2
    \item[\textbf{Tiri Salvez.:}] \resizebox{0.5\linewidth+1.8cm}{!}{Tempra +3, Riflessi +3, Volontà +3}
    \item[\textbf{Movimento:}] 6 m, scalata 6 m
    \item[\textbf{Sfida:}] 0(10 PX)\smallskip
\end{description}

\emph{\textbf{Camminare sulla Tela.}} Il ragno ignora le restrizioni al movimento provocate dalle ragnatele.

\emph{\textbf{Scalare come Ragno.}} Il ragno può scalare superfici difficili, compreso lo stare a testa in giù sul soffitto, senza bisogno di effettuare una prova di competenza.

\emph{\textbf{Senso della Tela.}} Mentre è in contatto con una ragnatela, il ragno sa l'esatta posizione di qualsiasi altra creatura in contatto con la stessa ragnatela.

\textbf{Azioni}

\emph{\textbf{Morso.} Attacco con Arma da Mischia}: +4 a colpire, portata 1 m, una creatura.

\emph{Colpisce:} 1 danno perforante e il bersaglio deve riuscire un Tiro Salvezza su Tempra 9 o subire 2 (1d4) danni da veleno.

\mostro{Ragno Fase}
\begin{description}[noitemsep, topsep=0pt, parsep=0pt, partopsep=0pt, leftmargin=0cm, labelwidth=2.2cm]
    \item[\textbf{Taglia/Tipo:}] Grande mostruosità, disallineato
    \item[\textbf{Caratt.:}] \resizebox{0.5\linewidth+1.8cm}{!}{For 2 Des 2 Cos 1 Int -2 Sag 0 Car -2}
    \item[\textbf{Punti Ferita:}] 69,  \textbf{Difesa:} 18,  \textbf{Iniziativa:} +2
    \item[\textbf{Tiri Salvez.:}] \resizebox{0.5\linewidth+1.8cm}{!}{Tempra +4, Riflessi +5, Volontà +3}
    \item[\textbf{Movimento:}] 9 m, scalata 9 m
    \item[\textbf{Sfida:}] 3 (700 PX)\smallskip
\end{description}

\emph{\textbf{Camminare sulla Tela.}} Il ragno ignora le restrizioni al movimento provocate dalle ragnatele.

\emph{\textbf{Passo Etereo.}} Come Reazione, il ragno può magicamente spostarsi dal Piano Materiale al Piano Etereo, o viceversa.

\emph{\textbf{Scalare come Ragno.}} Il ragno può scalare superfici difficili, compreso lo stare a testa in giù sul soffitto, senza bisogno di effettuare una prova di competenza.

\textbf{Azioni}

\emph{\textbf{Morso.} Attacco con Arma da Mischia}: +5 a colpire, portata 1 m, una creatura.

\emph{Colpisce:} 7 (1d10 + 2) danni perforanti e il bersaglio deve effettuare un Tiro Salvezza di Tempra DC 14, e subire 18 (4d8) danni da veleno se fallisce il Tiro Salvezza, o la metà di questo danno se lo riesce. Se il danno da veleno riduce il bersaglio a 0 Punti Ferita, il bersaglio è stabile ma avvelenato per 1 ora, anche dopo aver recuperato i Punti Ferita, e mentre è avvelenato in questo modo resta paralizzato.

\mostro{Ragno Gigante}
\begin{description}[noitemsep, topsep=0pt, parsep=0pt, partopsep=0pt, leftmargin=0cm, labelwidth=2.2cm]
    \item[\textbf{Taglia/Tipo:}] Grande bestia, disallineato
    \item[\textbf{Caratt.:}] \resizebox{0.5\linewidth+1.8cm}{!}{For 2 Des 3 Cos 1 Int -4 Sag 0 Car -3}
    \item[\textbf{Punti Ferita:}] 33,  \textbf{Difesa:} 16,  \textbf{Iniziativa:} +3
    \item[\textbf{Tiri Salvez.:}] \resizebox{0.5\linewidth+1.8cm}{!}{Tempra +3, Riflessi +4, Volontà +3}
    \item[\textbf{Movimento:}] 9 m, scalata 9 m
    \item[\textbf{Sfida:}] 1 (200 PX)\smallskip
\end{description}

\emph{\textbf{Camminare sulla Tela.}} Il ragno ignora le restrizioni al movimento provocate dalle ragnatele.

\emph{\textbf{Scalare come Ragno.}} Il ragno può scalare superfici difficili, compreso lo stare a testa in giù sul soffitto, senza bisogno di effettuare una prova di competenza.

\emph{\textbf{Senso della Tela.}} Mentre è in contatto con una ragnatela, il ragno sa l'esatta posizione di qualsiasi altra creatura in contatto con la stessa ragnatela.

\textbf{Azioni}

\emph{\textbf{Morso.} Attacco con Arma da Mischia}: +5 a colpire, portata 1 m, una creatura.

\emph{Colpisce:} 7 (1d8 + 3) danni perforanti e il bersaglio deve effettuare un Tiro Salvezza di Tempra DC 11, e subire 9

(2d8) danni da veleno se fallisce il Tiro Salvezza, o la metà di questi danni se lo riesce. Se il danno da veleno riduce il bersaglio a 0 Punti Ferita, il bersaglio è stabile ma avvelenato per 1 ora, anche dopo aver recuperato i Punti Ferita, e mentre è avvelenato in questo modo resta paralizzato.

\emph{\textbf{Ragnatela (Ricarica 5-6).} Attacco con Arma a Gittata}: +5 a colpire, gittata 9m, una creatura.

\emph{Colpisce:} Il bersaglio è intralciato dalla ragnatela. Con un'Azione, il bersaglio intralciato può effettuare un Tiro Salvezza Tempra con Forza DC 12 e, in caso di successo, spezzare la tela. La ragnatela può essere anche attaccata e distrutta (CA 10; Punti Ferita 5; vulnerabilità al danno da fuoco; immunità ai danni contundenti e da veleno).

\mostro{Ragno Lupo Gigante}
\begin{description}[noitemsep, topsep=0pt, parsep=0pt, partopsep=0pt, leftmargin=0cm, labelwidth=2.2cm]
    \item[\textbf{Taglia/Tipo:}] Media bestia, disallineato
    \item[\textbf{Caratt.:}] \resizebox{0.5\linewidth+1.8cm}{!}{For 1 Des 3 Cos 1 Int -4 Sag 1 Car -3}
    \item[\textbf{Punti Ferita:}] 19,  \textbf{Difesa:} 15,  \textbf{Iniziativa:} +3
    \item[\textbf{Tiri Salvez.:}] \resizebox{0.5\linewidth+1.8cm}{!}{Tempra +3, Riflessi +3, Volontà +3}
    \item[\textbf{Movimento:}] 12 m, scalata 12 m
    \item[\textbf{Sfida:}] 1/4 (50 PX)\smallskip
\end{description}

\emph{\textbf{Camminare sulla Tela.}} Il ragno ignora le restrizioni al movimento provocate dalle ragnatele.

\emph{\textbf{Scalare come Ragno.}} Il ragno può scalare superfici difficili, compreso lo stare a testa in giù sul soffitto, senza bisogno di effettuare una prova di competenza.

\emph{\textbf{Senso della Tela.}} Mentre è in contatto con una ragnatela, il ragno sa l'esatta posizione di qualsiasi altra creatura in contatto con la stessa ragnatela.

\textbf{Azioni}

\emph{\textbf{Morso.} Attacco con Arma da Mischia}: +4 a colpire, portata 1 m, una creatura.

\emph{Colpisce:} 4 (1d6 + 1) danni perforanti e il bersaglio deve effettuare un Tiro Salvezza di Tempra DC 11, e subire 7 (2d6) danni da veleno se fallisce il Tiro Salvezza, o la metà di questi danni se lo riesce. Se il danno da veleno riduce il bersaglio a 0 Punti Ferita, il bersaglio è stabile ma avvelenato per 1 ora, anche dopo aver recuperato i Punti Ferita, e mentre è avvelenato in questo modo resta paralizzato.

\mostro{Rana}
\begin{description}[noitemsep, topsep=0pt, parsep=0pt, partopsep=0pt, leftmargin=0cm, labelwidth=2.2cm]
    \item[\textbf{Taglia/Tipo:}] Minuscola bestia, disallineato
    \item[\textbf{Caratt.:}] \resizebox{0.5\linewidth+1.8cm}{!}{For -5 Des 1 Cos -1 Int -5 Sag -1 Car -4}
    \item[\textbf{Punti Ferita:}] 15,  \textbf{Difesa:} 13,  \textbf{Iniziativa:} +1
    \item[\textbf{Tiri Salvez.:}] \resizebox{0.5\linewidth+1.8cm}{!}{Tempra +3, Riflessi +3, Volontà +3}
    \item[\textbf{Movimento:}] 6 m, nuoto 6 m
    \item[\textbf{Sfida:}] 0(0 PX)\smallskip
\end{description}

\emph{\textbf{Anfibio.}} La rana può respirare aria e acqua.

\emph{\textbf{Salto da Fermo.}} Una rana può saltare in lungo fino a 3 metri e in alto fino a 1 metro, con o senza la rincorsa.

Una \textbf{rana} è sprovvista di attacchi. Si nutre di piccoli insetti e di solito vive in prossimità di acquitrini, dentro gli alberi o sottoterra.

\mostro{Rana Gigante}
\begin{description}[noitemsep, topsep=0pt, parsep=0pt, partopsep=0pt, leftmargin=0cm, labelwidth=2.2cm]
    \item[\textbf{Taglia/Tipo:}] Media bestia, disallineato
    \item[\textbf{Caratt.:}] \resizebox{0.5\linewidth+1.8cm}{!}{For 1 Des 1 Cos 0 Int -4 Sag 0 Car -4}
    \item[\textbf{Punti Ferita:}] 19,  \textbf{Difesa:} 13,  \textbf{Iniziativa:} +1
    \item[\textbf{Tiri Salvez.:}] \resizebox{0.5\linewidth+1.8cm}{!}{Tempra +3, Riflessi +3, Volontà +3}
    \item[\textbf{Movimento:}] 9 m, nuoto 9 m
    \item[\textbf{Sfida:}] 1/4 (50 PX)\smallskip
\end{description}

\emph{\textbf{Anfibio.}} La rana può respirare aria e acqua.

\emph{\textbf{Salto da Fermo.}} Una rana può saltare in lungo fino a 6 metri e in alto fino a 3 metri, con o senza la rincorsa.

\textbf{Azioni}

\emph{\textbf{Morso.} Attacco con Arma da Mischia}: +3 a colpire, portata 1 m, un bersaglio.

\emph{Colpisce:} 4 (1d6 + 1) danni perforanti e il bersaglio è afferrato (DC 11 per fuggire). Fino al termine dell'afferrare la rana non può usare il morso contro un altro bersaglio.

\emph{\textbf{Inghiottire.}} La rana effettua una attacco di morso contro un bersaglio di taglia Piccola o inferiore che sta afferrando. Se l'attacco colpisce, il bersaglio è inghiottito, e l'afferrare ha termine. Il bersaglio inghiottito è accecato e intralciato, ha copertura completa contro gli attacchi e altri effetti all'esterno della rana, e subisce 5 (2d4) danni da acido all'inizio di ciascun round della rana. La rana può inghiottire solo un bersaglio alla volta. Se la rana muore, una creatura inghiottita non è più intralciata da essa e può uscire dal cadavere utilizzando 1 metro di movimento, uscendo prona.

\mostro{Ratto}
\begin{description}[noitemsep, topsep=0pt, parsep=0pt, partopsep=0pt, leftmargin=0cm, labelwidth=2.2cm]
    \item[\textbf{Taglia/Tipo:}] Minuscola bestia, disallineato
    \item[\textbf{Caratt.:}] \resizebox{0.5\linewidth+1.8cm}{!}{For -4 Des 0 Cos -1 Int -4 Sag 0 Car -3}
    \item[\textbf{Punti Ferita:}] 15,  \textbf{Difesa:} 12,  \textbf{Iniziativa:} +0
    \item[\textbf{Tiri Salvez.:}] \resizebox{0.5\linewidth+1.8cm}{!}{Tempra +3, Riflessi +3, Volontà +3}
    \item[\textbf{Movimento:}] 6 m
    \item[\textbf{Sfida:}] 0(10 PX)\smallskip
\end{description}

\emph{\textbf{Olfatto Affinato.}} Il ratto ha +1d6 alle prove di Consapevolezza basate sull'olfatto.

\textbf{Azioni}

\emph{\textbf{Morso.} Attacco con Arma da Mischia}: +2 a colpire, portata 1 m, un bersaglio.

\emph{Colpisce:} 1 danno perforante.

\mostro{Ratto Gigante}
\begin{description}[noitemsep, topsep=0pt, parsep=0pt, partopsep=0pt, leftmargin=0cm, labelwidth=2.2cm]
    \item[\textbf{Taglia/Tipo:}] Piccola bestia, disallineato
    \item[\textbf{Caratt.:}] \resizebox{0.5\linewidth+1.8cm}{!}{For -2 Des 2 Cos 0 Int -4 Sag 0 Car -3}
    \item[\textbf{Punti Ferita:}] 17,  \textbf{Difesa:} 14,  \textbf{Iniziativa:} +2
    \item[\textbf{Tiri Salvez.:}] \resizebox{0.5\linewidth+1.8cm}{!}{Tempra +3, Riflessi +3, Volontà +3}
    \item[\textbf{Movimento:}] 9 m
    \item[\textbf{Sfida:}] 1/8 (25 PX)\smallskip
\end{description}

\emph{\textbf{Olfatto Affinato.}} Il ratto ha +1d6 alle prove di Consapevolezza basate sull'olfatto.

\emph{\textbf{Tattiche di Branco.}} Il ratto ha +1d6 al tiro di attacco contro una creatura se almeno uno degli alleati del ratto si trova entro 1 metro dalla creatura e quell'alleato non è inabile.

\textbf{Azioni}

\emph{\textbf{Morso.} Attacco con Arma da Mischia}: +4 a colpire, portata 1 m, un bersaglio.

\emph{Colpisce:} 4 (1d4 + 2) danni perforanti.

\medskip

\textbf{VARIANTE: RATTO GIGANTE AMMALATO}
\index[Mostruario]{Ratto Gigante ammalato}\hypertarget{Ratto Gigante ammalato}{}

Alcuni ratti giganti recano una terribile malattia che diffondono tramite il morso. Un ratto gigante ammalato ha grado di sfida 1/8 (25 PX) e la seguente azione invece del suo normale attacco di morso.

\emph{\textbf{Morso.} Attacco con Arma da Mischia}: +4 a colpire, portata 1 m, un bersaglio.

\emph{Colpisce:} 4 (1d4 + 2) danni perforanti. Se il bersaglio è una creatura, deve riuscire un Tiro Salvezza di Tempra DC 10 o contrarre una malattia. Fino a che la malattia non viene curata, TS Tempra DC 12 ogni 24 ore, il bersaglio non può recuperare Punti Ferita eccetto tramite metodi magici e i Punti Ferita massimi del bersaglio diminuiscono di 3 (1d6) ogni 24 ore. Se i Punti Ferita massimi del bersaglio scendono a 0 come risultato della malattia, il bersaglio muore.

\mostro{Rinoceronte lanoso}
\begin{description}[noitemsep, topsep=0pt, parsep=0pt, partopsep=0pt, leftmargin=0cm, labelwidth=2.2cm]
    \item[\textbf{Taglia/Tipo:}] Grande bestia, disallineato
    \item[\textbf{Caratt.:}] \resizebox{0.5\linewidth+1.8cm}{!}{For 5 Des -1 Cos 2 Int -4 Sag 1 Car -2}
    \item[\textbf{Punti Ferita:}] 51,  \textbf{Difesa:} 13,  \textbf{Iniziativa:} -1
    \item[\textbf{Tiri Salvez.:}] \resizebox{0.5\linewidth+1.8cm}{!}{Tempra +4, Riflessi +3, Volontà +3}
    \item[\textbf{Movimento:}] 12 m
    \item[\textbf{Sfida:}] 2 (450 PX)\smallskip
\end{description}

\emph{\textbf{Carica.}} Se il rinoceronte si muove di almeno 6 metri diretto verso un bersaglio e lo colpisce con un attacco di incornata durante lo stesso round, il bersaglio subisce 9 (2d8) danni contundenti aggiuntivi. Se il bersaglio è una creatura, deve riuscire un Tiro Salvezza su Tempra DC 15 o cadere prono.

\textbf{Azioni}

\emph{\textbf{Incornata.} Attacco con Arma da Mischia}: +4 a colpire, portata 1 m, un bersaglio.

\emph{Colpisce:} 14 (2d8 + 5) danni contundenti.

\mostro{Rospo Gigante}
\begin{description}[noitemsep, topsep=0pt, parsep=0pt, partopsep=0pt, leftmargin=0cm, labelwidth=2.2cm]
    \item[\textbf{Taglia/Tipo:}] Grande bestia, disallineato
    \item[\textbf{Caratt.:}] \resizebox{0.5\linewidth+1.8cm}{!}{For 2 Des 1 Cos 1 Int -4 Sag 0 Car -4}
    \item[\textbf{Punti Ferita:}] 33,  \textbf{Difesa:} 14,  \textbf{Iniziativa:} +1
    \item[\textbf{Tiri Salvez.:}] \resizebox{0.5\linewidth+1.8cm}{!}{Tempra +3, Riflessi +3, Volontà +3}
    \item[\textbf{Movimento:}] 6 m, nuoto 12 m
    \item[\textbf{Sfida:}] 1 (200 PX)\smallskip
\end{description}

\emph{\textbf{Anfibio.}} Il rospo può respirare aria e acqua.

\emph{\textbf{Salto da Fermo.}} Un rospo può saltare in lungo fino a 6 metri e in alto fino a 3 metri, con o senza la rincorsa.

\textbf{Azioni}

\emph{\textbf{Morso.} Attacco con Arma da Mischia}: +4 a colpire, portata 1 m, un bersaglio.

\emph{Colpisce:} 7 (1d10 + 2) danni perforanti più 5 (1d10) danni da veleno, e il bersaglio è afferrato (DC 13 per fuggire). Fino al termine dell'afferrare il rospo non può usare il morso contro un altro bersaglio.

\emph{\textbf{Inghiottire.}} Il rospo effettua una attacco di morso contro un bersaglio di taglia Media o inferiore che sta afferrando. Se l'attacco colpisce, il bersaglio è inghiottito, e l'afferrare ha termine. Il bersaglio inghiottito è accecato e intralciato, ha copertura completa contro gli attacchi e altri effetti all'esterno della rana, e subisce 10 (3d6) danni da acido all'inizio di ciascun round del rospo. Il rospo può inghiottire solo un bersaglio alla volta.

Se il rospo muore, una creatura inghiottita non è più intralciata da esso e può uscire dal cadavere utilizzando 1 metro di movimento, uscendo prono.

\emph{Colpisce:} 8 (2d4 + 3) danni contundenti.

\begin{center}
	\includegraphics[width=0.85\linewidth]{immagini/saurovallo2-ai.png}

	\emph{Saurovallo (B.I.C.)}
\end{center}

\mostro{Saurovallo da Galoppo}
\begin{description}[noitemsep, topsep=0pt, parsep=0pt, partopsep=0pt, leftmargin=0cm, labelwidth=2.2cm]
    \item[\textbf{Taglia/Tipo:}] Grande bestia, disallineato
    \item[\textbf{Caratt.:}] \resizebox{0.5\linewidth+1.8cm}{!}{For 3 Des 0 Cos 1 Int -3 Sag 0 Car -2}
    \item[\textbf{Punti Ferita:}] 19,  \textbf{Difesa:} 12,  \textbf{Iniziativa:} +0
    \item[\textbf{Tiri Salvez.:}] \resizebox{0.5\linewidth+1.8cm}{!}{Tempra +3, Riflessi +3, Volontà +3}
    \item[\textbf{Movimento:}] 18 m
    \item[\textbf{Sfida:}] 1/4 (50 PX)\smallskip
\end{description}

\textbf{Azioni}

\emph{\textbf{Zoccoli.} Attacco con Arma da Mischia}: +4 a colpire, portata 1 m, un bersaglio.

\mostro{Saurovallo da Guerra}
\begin{description}[noitemsep, topsep=0pt, parsep=0pt, partopsep=0pt, leftmargin=0cm, labelwidth=2.2cm]
    \item[\textbf{Taglia/Tipo:}] Grande bestia, disallineato
    \item[\textbf{Caratt.:}] \resizebox{0.5\linewidth+1.8cm}{!}{For 4 Des 1 Cos 1 Int -2 Sag 1 Car -2}
    \item[\textbf{Punti Ferita:}] 24,  \textbf{Difesa:} 13,  \textbf{Iniziativa:} +1
    \item[\textbf{Tiri Salvez.:}] \resizebox{0.5\linewidth+1.8cm}{!}{Tempra +3, Riflessi +3, Volontà +3}
    \item[\textbf{Movimento:}] 18 m
    \item[\textbf{Sfida:}] 1/2 (100 PX)\smallskip
\end{description}

\emph{\textbf{Carica Travolgente.}} Se il Saurovallo si muove di almeno 6 metri diretto verso il bersaglio e lo colpisce con un attacco di zoccoli durante lo stesso round, il bersaglio deve riuscire un Tiro Salvezza su Tempra DC 14 o cadere prono. Se il bersaglio è prono, il saurovallo può effettuare un altro attacco di zoccoli contro di lui come Azione Immediata.

\textbf{Azioni}

\emph{\textbf{Zoccoli.} Attacco con Arma da Mischia}: +4 a colpire, portata 1 m, un bersaglio.

\emph{Colpisce:} 11 (2d6 + 4) danni contundenti.


\mostro{Saurovallo da Tiro}
\begin{description}[noitemsep, topsep=0pt, parsep=0pt, partopsep=0pt, leftmargin=0cm, labelwidth=2.2cm]
    \item[\textbf{Taglia/Tipo:}] Grande bestia, disallineato
    \item[\textbf{Caratt.:}] \resizebox{0.5\linewidth+1.8cm}{!}{For 4 Des 0 Cos 1 Int -3 Sag 0 Car -2}
    \item[\textbf{Punti Ferita:}] 19,  \textbf{Difesa:} 12,  \textbf{Iniziativa:} +0
    \item[\textbf{Tiri Salvez.:}] \resizebox{0.5\linewidth+1.8cm}{!}{Tempra +3, Riflessi +3, Volontà +3}
    \item[\textbf{Movimento:}] 12 m
    \item[\textbf{Sfida:}] 1/4 (50 PX)\smallskip
\end{description}

\textbf{Azioni}

\emph{\textbf{Zoccoli.} Attacco con Arma da Mischia}: +4 a colpire, portata 1 m, un bersaglio.

\emph{Colpisce:} 9 (2d4 + 4) danni contundenti.


\begin{giocatore}[Il Saurovallo]\index{Saurovallo}
		La leggenda narra che Calicante appena scese sulla Terra vide i \emph{cavalli} e provò un disgusto incredibile per questi orrendi esseri e con il semplice volere li fece esplodere tutti. Non contento pochi attimi dopo tutti gli \emph{equini} fecero la stessa fine.

		Asini, muli, cavalli, zebre.. solo il cammello ed il dromedario non essendo propriamente equini si salvarono, anche se molti pensano che Calicante semplicemente li stia ignorando...

		Nethergal piuttosto scossa dal fatto che si era perso un utile animale per portare messaggi e cavalcabile per ampie distanze e non avendo il potere per creare una nuova creatura dal nulla, si rivolse ad Efrem ed Orlaith. Chiese ad Efrem di individuare un animale che potesse essere robusto, veloce ed adatto a essere cavalcato, mentre ad Orlaith chiese di inculcargli obbedienza ed il coraggio.

		Efrem sapendo che Torbiorn aveva portato sul pianeta milioni dei suoi amati dinosauri scelse il Parasaurolophus e, con il supporto di Orlaith, lo rese più compatto, piccolo, mansueto, erbivoro: perfetto per essere cavalcato.

		Creò anche poi una versione ridotta, nana nelle misure, che potesse adattarsi a portare le creature di taglia piccola.

		Purtroppo zanzare, cimici e mosche sono rimasti con massimo dispiacere di tutti!
\end{giocatore}

\mostro{Saurovallo nano}
\begin{description}[noitemsep, topsep=0pt, parsep=0pt, partopsep=0pt, leftmargin=0cm, labelwidth=2.2cm]
    \item[\textbf{Taglia/Tipo:}] Media bestia, disallineato
    \item[\textbf{Caratt.:}] \resizebox{0.5\linewidth+1.8cm}{!}{For 2 Des 0 Cos 1 Int -3 Sag 0 Car -2}
    \item[\textbf{Punti Ferita:}] 17,  \textbf{Difesa:} 12,  \textbf{Iniziativa:} +0
    \item[\textbf{Tiri Salvez.:}] \resizebox{0.5\linewidth+1.8cm}{!}{Tempra +3, Riflessi +3, Volontà +3}
    \item[\textbf{Movimento:}] 12 m
    \item[\textbf{Sfida:}] 1/8 (25 PX)\smallskip
\end{description}

\textbf{Azioni}

\emph{\textbf{Zoccoli.} Attacco con Arma da Mischia}: +4 a colpire, portata 1 m, un bersaglio.

\emph{Colpisce:} 7 (2d4 + 2) danni contundenti.

\mostro{Scarabeo di Fuoco Gigante}
\begin{description}[noitemsep, topsep=0pt, parsep=0pt, partopsep=0pt, leftmargin=0cm, labelwidth=2.2cm]
    \item[\textbf{Taglia/Tipo:}] Piccola bestia, disallineato
    \item[\textbf{Caratt.:}] \resizebox{0.5\linewidth+1.8cm}{!}{For -1 Des 0 Cos 1 Int -5 Sag -2 Car -4}
    \item[\textbf{Punti Ferita:}] 15,  \textbf{Difesa:} 12,  \textbf{Iniziativa:} +0
    \item[\textbf{Tiri Salvez.:}] \resizebox{0.5\linewidth+1.8cm}{!}{Tempra +3, Riflessi +3, Volontà +3}
    \item[\textbf{Movimento:}] 9 m
    \item[\textbf{Sfida:}] 0(10 PX)\smallskip
\end{description}

\emph{\textbf{Illuminazione.}} Lo scarabeo irradia luce intensa in un raggio di 3 metri e luce fioca per 6 metri.

\textbf{Azioni}

\emph{\textbf{Morso.} Attacco con Arma da Mischia}: +2 a colpire, portata 1 m, un bersaglio.

\emph{Colpisce:} 2 (1d6 - 1) danni taglienti.

\mostro{Sciacallo}
\begin{description}[noitemsep, topsep=0pt, parsep=0pt, partopsep=0pt, leftmargin=0cm, labelwidth=2.2cm]
    \item[\textbf{Taglia/Tipo:}] Piccola bestia, disallineato
    \item[\textbf{Caratt.:}] \resizebox{0.5\linewidth+1.8cm}{!}{For -1 Des 2 Cos 0 Int -4 Sag 1 Car -2}
    \item[\textbf{Punti Ferita:}] 15,  \textbf{Difesa:} 14,  \textbf{Iniziativa:} +2
    \item[\textbf{Tiri Salvez.:}] \resizebox{0.5\linewidth+1.8cm}{!}{Tempra +3, Riflessi +3, Volontà +3}
    \item[\textbf{Movimento:}] 12 m
    \item[\textbf{Sfida:}] 0 (10 PX)\smallskip
\end{description}

\emph{\textbf{Tattiche di Branco.}} Lo sciacallo ha +1d6 ai tiri di attacco contro una creatura se almeno uno degli alleati dello sciacallo si trova entro 1 metro dalla creatura e quell'alleato non è inabile.

\emph{\textbf{Udito e Olfatto Affinato.}} Lo sciacallo ha +1d6 nelle prove di Consapevolezza basate su udito o olfatto.

\textbf{Azioni}

\emph{\textbf{Morso.} Attacco con Arma da Mischia}: +2 a colpire, portata 1 m, un bersaglio.

\emph{Colpisce:} 1 (1d4 - 1) danni perforanti.

\mostro{Sciame di Centopiedi}
\begin{description}[noitemsep, topsep=0pt, parsep=0pt, partopsep=0pt, leftmargin=0cm, labelwidth=2.2cm]
    \item[\textbf{Taglia/Tipo:}] Media sciame di Minuscole bestie, disallineato
    \item[\textbf{Caratt.:}] \resizebox{0.5\linewidth+1.8cm}{!}{For -4 Des 1 Cos 0 Int -5 Sag -2 Car -5}
    \item[\textbf{Punti Ferita:}] 24,  \textbf{Difesa:} 13,  \textbf{Iniziativa:} +1
    \item[\textbf{Resistenze al danno:}] contundente, perforante, tagliente
    \item[\textbf{Tiri Salvez.:}] \resizebox{0.5\linewidth+1.8cm}{!}{Tempra +3, Riflessi +3, Volontà +3}
    \item[\textbf{Movimento:}] 6 m, scalata 6 m
    \item[\textbf{Sfida:}] 1/2 (100 PX)\smallskip
\end{description}

\emph{\textbf{Sciame.}} Lo sciame può occupare lo spazio di un'altra creatura e viceversa, lo sciame può muoversi attraverso qualsiasi apertura grande abbastanza per un Minuscolo insetto. Lo sciame non può recuperare Punti Ferita né ottenere Punti Ferita temporanei.

\textbf{Azioni}

\emph{\textbf{Morsi.} Attacco con Arma da Mischia}: +4 a colpire, portata 0 m, un bersaglio nello spazio dello sciame.

\emph{Colpisce:} 10 (4d4) danni perforanti, o 5 (2d4) danni perforanti se lo sciame è ha metà o meno dei suoi Punti Ferita. Una creatura ridotta a 0 Punti Ferita da uno sciame di centopiedi ma stabile resta avvelenata per 1 ora, anche dopo aver recuperato i Punti Ferita, e rimane paralizzata dal veleno durante questo periodo.

\mostro{Sciame di Corvi}
\begin{description}[noitemsep, topsep=0pt, parsep=0pt, partopsep=0pt, leftmargin=0cm, labelwidth=2.2cm]
    \item[\textbf{Taglia/Tipo:}] Media sciame di Minuscole bestie, disallineato
    \item[\textbf{Caratt.:}] \resizebox{0.5\linewidth+1.8cm}{!}{For -2 Des 2 Cos -1 Int -4 Sag 1 Car -2}
    \item[\textbf{Punti Ferita:}] 19,  \textbf{Difesa:} 14,  \textbf{Iniziativa:} +2
    \item[\textbf{Resistenze al danno:}] contundente, perforante, tagliente
    \item[\textbf{Tiri Salvez.:}] \resizebox{0.5\linewidth+1.8cm}{!}{Tempra +3, Riflessi +3, Volontà +3}
    \item[\textbf{Movimento:}] 3 m, volo 15 m
    \item[\textbf{Sfida:}] 1/4 (50 PX)\smallskip
\end{description}

\emph{\textbf{Sciame.}} Lo sciame può occupare lo spazio di un'altra creatura e viceversa, lo sciame può muoversi attraverso qualsiasi apertura grande abbastanza per un Minuscolo corvo. Lo sciame non può recuperare Punti Ferita né ottenere Punti Ferita temporanei.

\textbf{Azioni}

\emph{\textbf{Becchi.} Attacco con Arma da Mischia}: +4 a colpire, portata 1 m, un bersaglio nello spazio dello sciame.

\emph{Colpisce:} 7 (2d6) danni perforanti, o 3 (1d6) danni perforanti se lo sciame è ha metà o meno dei suoi Punti Ferita.

\mostro{Sciame di Pirana}
\begin{description}[noitemsep, topsep=0pt, parsep=0pt, partopsep=0pt, leftmargin=0cm, labelwidth=2.2cm]
    \item[\textbf{Taglia/Tipo:}] Media sciame di Minuscole bestie, disallineato
    \item[\textbf{Caratt.:}] \resizebox{0.5\linewidth+1.8cm}{!}{For 1 Des 3 Cos -1 Int -5 Sag -2 Car -4}
    \item[\textbf{Punti Ferita:}] 32,  \textbf{Difesa:} 16,  \textbf{Iniziativa:} +3
    \item[\textbf{Resistenze al danno:}] contundente, perforante, tagliente
    \item[\textbf{Tiri Salvez.:}] \resizebox{0.5\linewidth+1.8cm}{!}{Tempra +3, Riflessi +4, Volontà +3}
    \item[\textbf{Movimento:}] 0 m, nuoto 12 m
    \item[\textbf{Sfida:}] 1 (200 PX)\smallskip
\end{description}

\emph{\textbf{Frenesia Sanguinaria.}} Lo sciame ha +1d6 ai tiri di attacco in mischia contro qualsiasi creatura che non sia al massimo dei Punti Ferita.

\emph{\textbf{Respirare Acqua.}} Lo sciame può respirare solo sott'acqua.

\emph{\textbf{Sciame.}} Lo sciame può occupare lo spazio di un'altra creatura e viceversa, lo sciame può muoversi attraverso qualsiasi apertura grande abbastanza per un Minuscolo pirana. Lo sciame non può recuperare Punti Ferita né ottenere Punti Ferita temporanei.

\textbf{Azioni}

\emph{\textbf{Morsi.} Attacco con Arma da Mischia}: +5 a colpire, portata 0 m, una creatura nello spazio dello sciame.

\emph{Colpisce:} 14 (4d6) danni perforanti, o 7 (2d6) danni perforanti se lo sciame è ha metà o meno dei suoi Punti Ferita.

\mostro{Sciame di Insetti}
\begin{description}[noitemsep, topsep=0pt, parsep=0pt, partopsep=0pt, leftmargin=0cm, labelwidth=2.2cm]
    \item[\textbf{Taglia/Tipo:}] Media sciame di Minuscole bestie, disallineato
    \item[\textbf{Caratt.:}] \resizebox{0.5\linewidth+1.8cm}{!}{For -4 Des 1 Cos 0 Int -5 Sag -2 Car -5}
    \item[\textbf{Punti Ferita:}] 24,  \textbf{Difesa:} 13,  \textbf{Iniziativa:} +1
    \item[\textbf{Resistenze al danno:}] contundente, perforante, tagliente
    \item[\textbf{Tiri Salvez.:}] \resizebox{0.5\linewidth+1.8cm}{!}{Tempra +3, Riflessi +3, Volontà +3}
    \item[\textbf{Movimento:}] 6 m, scalata 6 m
    \item[\textbf{Sfida:}] 1/2 (100 PX)\smallskip
\end{description}

\emph{\textbf{Sciame.}} Lo sciame può occupare lo spazio di un'altra creatura e viceversa, lo sciame può muoversi attraverso qualsiasi apertura grande abbastanza per un Minuscolo insetto. Lo sciame non può recuperare Punti Ferita né ottenere Punti Ferita temporanei.

\textbf{Azioni}

\emph{\textbf{Morsi.} Attacco con Arma da Mischia}: +3 a colpire, portata 0 m, un bersaglio nello spazio dello sciame.

\emph{Colpisce:} 10 (4d4) danni perforanti, o 5 (2d4) danni perforanti se lo sciame è ha metà o meno dei suoi Punti Ferita.

\mostro{Sciame di Pipistrelli}
\begin{description}[noitemsep, topsep=0pt, parsep=0pt, partopsep=0pt, leftmargin=0cm, labelwidth=2.2cm]
    \item[\textbf{Taglia/Tipo:}] Media sciame di Minuscole bestie, disallineato
    \item[\textbf{Caratt.:}] \resizebox{0.5\linewidth+1.8cm}{!}{For -3 Des 2 Cos 0 Int -4 Sag 1 Car -3}
    \item[\textbf{Punti Ferita:}] 19,  \textbf{Difesa:} 14,  \textbf{Iniziativa:} +2
    \item[\textbf{Resistenze al danno:}] contundente, perforante, tagliente
    \item[\textbf{Tiri Salvez.:}] \resizebox{0.5\linewidth+1.8cm}{!}{Tempra +3, Riflessi +3, Volontà +3}
    \item[\textbf{Movimento:}] 0 m, volo 9 m
    \item[\textbf{Sfida:}] 1/4 (50 PX)\smallskip
\end{description}

\emph{\textbf{Ecolocazione.}} Lo sciame non può usare la vista cieca se assordato.

\emph{\textbf{Sciame.}} Lo sciame può occupare lo spazio di un'altra creatura e viceversa, lo sciame può muoversi attraverso qualsiasi apertura grande abbastanza per un Minuscolo pipistrello. Lo sciame non può recuperare Punti Ferita né ottenere Punti Ferita temporanei.

\emph{\textbf{Udito Affinato.}} Lo sciame ha +1d6 alle prove di Consapevolezza basate sull'udito.

\textbf{Azioni}

\emph{\textbf{Morsi.} Attacco con Arma da Mischia}: +4 a colpire, portata 0 m, una creatura nello spazio dello sciame.

\emph{Colpisce:} 5 (2d4) danni perforanti, o 2 (1d4) danni perforanti se lo sciame è ha metà o meno dei suoi Punti Ferita.

\mostro{Sciame di Ragni}
\begin{description}[noitemsep, topsep=0pt, parsep=0pt, partopsep=0pt, leftmargin=0cm, labelwidth=2.2cm]
    \item[\textbf{Taglia/Tipo:}] Media sciame di Minuscole bestie, disallineato
    \item[\textbf{Caratt.:}] \resizebox{0.5\linewidth+1.8cm}{!}{For -4 Des 1 Cos 0 Int -5 Sag -2 Car -5}
    \item[\textbf{Punti Ferita:}] 24,  \textbf{Difesa:} 13,  \textbf{Iniziativa:} +1
    \item[\textbf{Resistenze al danno:}] contundente, perforante, tagliente
    \item[\textbf{Tiri Salvez.:}] \resizebox{0.5\linewidth+1.8cm}{!}{Tempra +3, Riflessi +3, Volontà +3}
    \item[\textbf{Movimento:}] 6 m, scalata 6 m
    \item[\textbf{Sfida:}] 1/2 (100 PX)\smallskip
\end{description}

\emph{\textbf{Camminare sulla Tela.}} Lo sciame ignora le restrizioni al movimento provocate dalle ragnatele.

\emph{\textbf{Scalare come Ragno.}} Lo sciame può scalare superfici difficili, compreso lo stare a testa in giù sul soffitto, senza bisogno di effettuare una prova di competenza.

\emph{\textbf{Senso della Tela.}} Mentre è in contatto con una ragnatela, lo sciame sa l'esatta posizione di qualsiasi altra creatura in contatto con la stessa ragnatela.

\emph{\textbf{Sciame.}} Lo sciame può occupare lo spazio di un'altra creatura e viceversa, lo sciame può muoversi attraverso qualsiasi apertura grande abbastanza per un Minuscolo insetto. Lo sciame non può recuperare Punti Ferita né ottenere Punti Ferita temporanei.

\textbf{Azioni}

\emph{\textbf{Morsi.} Attacco con Arma da Mischia}: +3 a colpire, portata 0 m, un bersaglio nello spazio dello sciame.

\emph{Colpisce:} 10 (4d4) danni perforanti, o 5 (2d4) danni perforanti se lo sciame è ha metà o meno dei suoi Punti Ferita.

\mostro{Sciame di Ratti}
\begin{description}[noitemsep, topsep=0pt, parsep=0pt, partopsep=0pt, leftmargin=0cm, labelwidth=2.2cm]
    \item[\textbf{Taglia/Tipo:}] Media sciame di Minuscole bestie, disallineato
    \item[\textbf{Caratt.:}] \resizebox{0.5\linewidth+1.8cm}{!}{For -1 Des 0 Cos -1 Int -4 Sag 0 Car -4}
    \item[\textbf{Punti Ferita:}] 19,  \textbf{Difesa:} 12,  \textbf{Iniziativa:} +0
    \item[\textbf{Resistenze al danno:}] contundente, perforante, tagliente
    \item[\textbf{Tiri Salvez.:}] \resizebox{0.5\linewidth+1.8cm}{!}{Tempra +3, Riflessi +3, Volontà +3}
    \item[\textbf{Movimento:}] 9 m
    \item[\textbf{Sfida:}] 1/4 (50 PX)\smallskip
\end{description}

\emph{\textbf{Olfatto Affinato.}} Lo sciame ha +1d6 alle prove di Consapevolezza basate sull'olfatto.

\emph{\textbf{Sciame.}} Lo sciame può occupare lo spazio di un'altra creatura e viceversa, lo sciame può muoversi attraverso qualsiasi apertura grande abbastanza per un Minuscolo ratto. Lo sciame non può recuperare Punti Ferita né ottenere Punti Ferita temporanei.

\textbf{Azioni}

\emph{\textbf{Morsi.} Attacco con Arma da Mischia}: +4 a colpire, portata 0 m, un bersaglio nello spazio dello sciame.

\emph{Colpisce:} 7 (2d6) danni perforanti, o 3 (1d6) danni perforanti se lo sciame è ha metà o meno dei suoi Punti Ferita.

\emph{\textbf{Sciame.}} Lo sciame può occupare lo spazio di un'altra creatura e viceversa, lo sciame può muoversi attraverso qualsiasi apertura grande abbastanza per un Minuscolo insetto. Lo sciame non può recuperare Punti Ferita né ottenere Punti Ferita temporanei.

\textbf{Azioni}

\emph{\textbf{Morsi.} Attacco con Arma da Mischia}: +3 a colpire, portata 0 m, un bersaglio nello spazio dello sciame.

\emph{Colpisce:} 10 (4d4) danni perforanti, o 5 (2d4) danni perforanti se lo sciame è ha metà o meno dei suoi Punti Ferita.

\mostro{Sciame di Serpenti Velenosi}
\begin{description}[noitemsep, topsep=0pt, parsep=0pt, partopsep=0pt, leftmargin=0cm, labelwidth=2.2cm]
    \item[\textbf{Taglia/Tipo:}] Media sciame di Minuscole bestie, disallineato
    \item[\textbf{Caratt.:}] \resizebox{0.5\linewidth+1.8cm}{!}{For -1 Des 4 Cos 0 Int -5 Sag 0 Car -4}
    \item[\textbf{Punti Ferita:}] 19,  \textbf{Difesa:} 16,  \textbf{Iniziativa:} +4
    \item[\textbf{Resistenze al danno:}] contundente, perforante, tagliente
    \item[\textbf{Tiri Salvez.:}] \resizebox{0.5\linewidth+1.8cm}{!}{Tempra +3, Riflessi +4, Volontà +3}
    \item[\textbf{Movimento:}] 9 m
    \item[\textbf{Sfida:}] 1/4 (50 PX)\smallskip
\end{description}

\emph{\textbf{Sciame.}} Lo sciame può occupare lo spazio di un'altra creatura e viceversa, lo sciame può muoversi attraverso qualsiasi apertura grande abbastanza per un Minuscolo serpente. Lo sciame non può recuperare Punti Ferita né ottenere Punti Ferita temporanei.

\textbf{Azioni}

\emph{\textbf{Morsi.} Attacco con Arma da Mischia}: +4 a colpire, portata 0 m, una creatura nello spazio dello sciame.

\emph{Colpisce:} 7 (2d6) danni perforanti, o 3 (1d6) danni perforanti se lo sciame è ha metà o meno dei suoi Punti Ferita, e il bersaglio deve effettuare un Tiro Salvezza di Tempra DC 10, e subire 14 (4d6) danni da veleno se fallisce il Tiro Salvezza, o la metà di questi danni se lo riesce.

\mostro{Sciame di Vespe}
\begin{description}[noitemsep, topsep=0pt, parsep=0pt, partopsep=0pt, leftmargin=0cm, labelwidth=2.2cm]
	\item[\textbf{Taglia/Tipo:}] Media sciame di Minuscole bestie, disallineato
	\item[\textbf{Caratt.:}] \resizebox{0.5\linewidth+1.8cm}{!}{For -1 Des 1 Cos 0 Int -5 Sag -2 Car -5}
	\item[\textbf{Punti Ferita:}] 24,  \textbf{Difesa:} 13,  \textbf{Iniziativa:} +1
	\item[\textbf{Tiri Salvez.:}] \resizebox{0.5\linewidth+1.8cm}{!}{Tempra +3, Riflessi +3, Volontà +3}
	\item[\textbf{Movimento:}] 1m, volo 9 m
	\item[\textbf{Sfida:}] 1/2 (100 PX)\smallskip
\end{description}

\emph{\textbf{Sciame.}} Lo sciame può occupare lo spazio di un'altra creatura e viceversa, e lo sciame può muoversi attraverso qualsiasi apertura grande abbastanza per un Minuscolo insetto. Lo sciame non può recuperare Punti Ferita né ottenere Punti Ferita temporanei.

\textbf{Azioni}

\emph{\textbf{Morsi.} Attacco con Arma da Mischia}: +4 a colpire, portata 0 m, un bersaglio nello spazio dello sciame.

\emph{Colpisce:} 10 (4d4) danni perforanti, o 5 (2d4) danni perforanti se lo sciame è ha metà o meno dei suoi Punti Ferita.

\mostro{Scimmia}
\begin{description}[noitemsep, topsep=0pt, parsep=0pt, partopsep=0pt, leftmargin=0cm, labelwidth=2.2cm]
    \item[\textbf{Taglia/Tipo:}] Piccola bestia, disallineato
    \item[\textbf{Caratt.:}] \resizebox{0.5\linewidth+1.8cm}{!}{For -3 Des 2 Cos 0 Int -3 Sag 1 Car -2}
    \item[\textbf{Punti Ferita:}] 19,  \textbf{Difesa:} 14,  \textbf{Iniziativa:} +2
    \item[\textbf{Comp.:}] Acrobatica +6, Consapevolezza +3
    \item[\textbf{Tiri Salvez.:}] \resizebox{0.5\linewidth+1.8cm}{!}{Tempra +3, Riflessi +3, Volontà +3}
    \item[\textbf{Movimento:}] 9 m, scalata 9 m
    \item[\textbf{Sfida:}] 1/4 (50 PX)\smallskip
\end{description}

\textbf{Azioni}

\emph{\textbf{Graffio.} Attacco con arma da Mischia}: +3 a colpire, portata 1 m, un bersaglio.

\emph{Colpisce:} 1 (1d4 - 1) danni da taglio.

\emph{\textbf{Morso.} Attacco con Arma da Mischia}: +2 a colpire, portata 1 metro, un bersaglio.

\emph{Colpisce:} 2 (1d4) danni perforanti.

\mostro{Scimmione}
\begin{description}[noitemsep, topsep=0pt, parsep=0pt, partopsep=0pt, leftmargin=0cm, labelwidth=2.2cm]
    \item[\textbf{Taglia/Tipo:}] Media bestia, disallineato
    \item[\textbf{Caratt.:}] \resizebox{0.5\linewidth+1.8cm}{!}{For 3 Des 2 Cos 2 Int -2 Sag 1 Car -2}
    \item[\textbf{Punti Ferita:}] 24,  \textbf{Difesa:} 14,  \textbf{Iniziativa:} +2
    \item[\textbf{Tiri Salvez.:}] \resizebox{0.5\linewidth+1.8cm}{!}{Tempra +3, Riflessi +3, Volontà +3}
    \item[\textbf{Movimento:}] 9 m, scalata 9 m
    \item[\textbf{Sfida:}] 1/2 (100 PX)\smallskip
\end{description}

\textbf{Azioni}

\emph{\textbf{Multiattacco.}} Lo scimmione effettua due attacchi di pugno.

\emph{\textbf{Pugno.} Attacco con Arma da Mischia}: +5 a colpire, portata 1 m, un bersaglio.

\emph{Colpisce:} 6 (1d6 + 3) danni contundenti.

\emph{\textbf{Sasso.} Attacco con Arma a Gittata}: +5 a colpire, gittata 8m, un bersaglio.

\emph{Colpisce:} 6 (1d6 + 3) danni contundenti.

\mostro{Scimmione Gigante}
\begin{description}[noitemsep, topsep=0pt, parsep=0pt, partopsep=0pt, leftmargin=0cm, labelwidth=2.2cm]
    \item[\textbf{Taglia/Tipo:}] Enorme bestia, disallineato
    \item[\textbf{Caratt.:}] \resizebox{0.5\linewidth+1.8cm}{!}{For 6 Des 2 Cos 4 Int -2 Sag 1 Car -2}
    \item[\textbf{Punti Ferita:}] 146,  \textbf{Difesa:} 23,  \textbf{Iniziativa:} +2
    \item[\textbf{Tiri Salvez.:}] \resizebox{0.5\linewidth+1.8cm}{!}{Tempra +11, Riflessi +9, Volontà +8}
    \item[\textbf{Movimento:}] 12 m, scalata 12 m
    \item[\textbf{Sfida:}] 7 (2900 PX)\smallskip
\end{description}

\textbf{Azioni}

\emph{\textbf{Multiattacco.}} Lo scimmione effettua due attacchi di pugno.

\emph{\textbf{Pugno.} Attacco con Arma da Mischia}: +9 a colpire, portata 3 m, un bersaglio.

\emph{Colpisce:} 22 (3d10 + 6) danni contundenti.

\emph{\textbf{Sasso.} Attacco con Arma a Gittata}: +9 a colpire, gittata 15m, un bersaglio.

\emph{Colpisce:} 30 (7d6 + 6) danni contundenti.

\mostro{Scorpione}
\begin{description}[noitemsep, topsep=0pt, parsep=0pt, partopsep=0pt, leftmargin=0cm, labelwidth=2.2cm]
    \item[\textbf{Taglia/Tipo:}] Minuscola bestia, disallineato
    \item[\textbf{Caratt.:}] \resizebox{0.5\linewidth+1.8cm}{!}{For -4 Des 0 Cos -1 Int -5 Sag -1 Car -4}
    \item[\textbf{Punti Ferita:}] 15,  \textbf{Difesa:} 12,  \textbf{Iniziativa:} +0
    \item[\textbf{Tiri Salvez.:}] \resizebox{0.5\linewidth+1.8cm}{!}{Tempra +3, Riflessi +3, Volontà +3}
    \item[\textbf{Movimento:}] 3 m
    \item[\textbf{Sfida:}] 0(10 PX)\smallskip
\end{description}

\textbf{Azioni}

\emph{\textbf{Pungiglione.} Attacco con Arma da Mischia}: +2 a colpire, portata 1 m, una creatura.

\emph{Colpisce:} 1 danno perforante e il bersaglio deve effettuare un Tiro Salvezza di Tempra DC 9, e subire 4 (1d8) danni da veleno se fallisce il Tiro Salvezza, o la metà di questi danni se lo riesce.

\mostro{Scorpione Gigante}
\begin{description}[noitemsep, topsep=0pt, parsep=0pt, partopsep=0pt, leftmargin=0cm, labelwidth=2.2cm]
    \item[\textbf{Taglia/Tipo:}] Grande bestia, disallineato
    \item[\textbf{Caratt.:}] \resizebox{0.5\linewidth+1.8cm}{!}{For 2 Des 1 Cos 2 Int -5 Sag -1 Car -4}
    \item[\textbf{Punti Ferita:}] 70,  \textbf{Difesa:} 17,  \textbf{Iniziativa:} +1
    \item[\textbf{Tiri Salvez.:}] \resizebox{0.5\linewidth+1.8cm}{!}{Tempra +5, Riflessi +4, Volontà +3}
    \item[\textbf{Movimento:}] 12 m
    \item[\textbf{Sfida:}] 3 (700 PX)\smallskip
\end{description}

\textbf{Azioni}

\emph{\textbf{Multiattacco.}} Lo scorpione effettua tre attacchi: due con gli artigli e uno con il pungiglione.

\emph{\textbf{Artiglio.} Attacco con Arma da Mischia}: +5 a colpire, portata 1 m, un bersaglio.

\emph{Colpisce:} 6 (1d8 + 2) danni contundenti e il bersaglio è afferrato (DC 12 per fuggire). Lo scorpione ha due artigli, ciascuno dei quali può afferrare solo un bersaglio.

\emph{\textbf{Pungiglione.} Attacco con Arma da Mischia}: +5 a colpire, portata 1 m, una creatura.

\emph{Colpisce:} 7 (1d10 + 2) danni perforanti e il bersaglio deve effettuare un Tiro Salvezza di Tempra DC 14, e subire 22 (4d10) danni da veleno se fallisce il Tiro Salvezza, o la metà di questi danni se lo riesce.

\mostro{Serpente Costrittore}
\begin{description}[noitemsep, topsep=0pt, parsep=0pt, partopsep=0pt, leftmargin=0cm, labelwidth=2.2cm]
    \item[\textbf{Taglia/Tipo:}] Grande bestia, disallineato
    \item[\textbf{Caratt.:}] \resizebox{0.5\linewidth+1.8cm}{!}{For 2 Des 2 Cos 1 Int -5 Sag 0 Car -4}
    \item[\textbf{Punti Ferita:}] 19,  \textbf{Difesa:} 14,  \textbf{Iniziativa:} +2
    \item[\textbf{Tiri Salvez.:}] \resizebox{0.5\linewidth+1.8cm}{!}{Tempra +3, Riflessi +3, Volontà +3}
    \item[\textbf{Movimento:}] 9 m, nuoto 9 m
    \item[\textbf{Sfida:}] 1/4 (50 PX)\smallskip
\end{description}

\textbf{Azioni}

\emph{\textbf{Morso.} Attacco con Arma da Mischia}: +4 a colpire, portata 1 m, una creatura.

\emph{Colpisce:} 5 (1d6 + 2) danni perforanti.

\emph{\textbf{Stritolare.} Attacco con Arma da Mischia}: +4 a colpire, portata 1 m, una creatura.

\emph{Colpisce:} 6 (1d8 + 2) danni contundenti, e il bersaglio è afferrato (DC 14 per fuggire). Fino al termine dell'afferrare, la creatura è intralciata, e il serpente non può stritolare un altro bersaglio.

\mostro{Serpente Costrittore Gigante}
\begin{description}[noitemsep, topsep=0pt, parsep=0pt, partopsep=0pt, leftmargin=0cm, labelwidth=2.2cm]
    \item[\textbf{Taglia/Tipo:}] Enorme bestia, disallineato
    \item[\textbf{Caratt.:}] \resizebox{0.5\linewidth+1.8cm}{!}{For 4 Des 2 Cos 1 Int -5 Sag 0 Car -4}
    \item[\textbf{Punti Ferita:}] 51,  \textbf{Difesa:} 16,  \textbf{Iniziativa:} +2
    \item[\textbf{Tiri Salvez.:}] \resizebox{0.5\linewidth+1.8cm}{!}{Tempra +3, Riflessi +4, Volontà +3}
    \item[\textbf{Movimento:}] 9 m, nuoto 9 m
    \item[\textbf{Sfida:}] 2 (450 PX)\smallskip
\end{description}

\textbf{Azioni}

\emph{\textbf{Morso.} Attacco con Arma da Mischia}: +5 a colpire, portata 3 m, una creatura.

\emph{Colpisce:} 11 (2d6 + 4) danni perforanti.

\emph{\textbf{Stritolare.} Attacco con Arma da Mischia}: +5 a colpire, portata 1 m, una creatura.

\emph{Colpisce:} 13 (2d8 + 4) danni contundenti, e il bersaglio è afferrato (DC 16 per fuggire). Fino al termine dell'afferrare, la creatura è intralciata, e il serpente non può stritolare un altro bersaglio.

\mostro{Serpente Velenoso}
\begin{description}[noitemsep, topsep=0pt, parsep=0pt, partopsep=0pt, leftmargin=0cm, labelwidth=2.2cm]
    \item[\textbf{Taglia/Tipo:}] Minuscola bestia, disallineato
    \item[\textbf{Caratt.:}] \resizebox{0.5\linewidth+1.8cm}{!}{For -4 Des 3 Cos 0 Int -5 Sag 0 Car -4}
    \item[\textbf{Punti Ferita:}] 17,  \textbf{Difesa:} 15,  \textbf{Iniziativa:} +3
    \item[\textbf{Tiri Salvez.:}] \resizebox{0.5\linewidth+1.8cm}{!}{Tempra +3, Riflessi +3, Volontà +3}
    \item[\textbf{Movimento:}] 9 m, nuoto 9 m
    \item[\textbf{Sfida:}] 1/8 (25 PX)\smallskip
\end{description}

\textbf{Azioni}

\emph{\textbf{Morso.} Attacco con Arma da Mischia}: +4 a colpire, portata 1 m, un bersaglio.

\emph{Colpisce:} 1 danno perforante e il bersaglio deve effettuare un Tiro Salvezza di Tempra DC 10, e subire 5 (2d4) danni da veleno se fallisce il Tiro Salvezza, o la metà di questi danni se lo riesce.

\mostro{Serpente Velenoso Gigante}
\begin{description}[noitemsep, topsep=0pt, parsep=0pt, partopsep=0pt, leftmargin=0cm, labelwidth=2.2cm]
    \item[\textbf{Taglia/Tipo:}] Media bestia, disallineato
    \item[\textbf{Caratt.:}] \resizebox{0.5\linewidth+1.8cm}{!}{For 0 Des 4 Cos 1 Int -4 Sag 0 Car -4}
    \item[\textbf{Punti Ferita:}] 19,  \textbf{Difesa:} 16,  \textbf{Iniziativa:} +4
    \item[\textbf{Tiri Salvez.:}] \resizebox{0.5\linewidth+1.8cm}{!}{Tempra +3, Riflessi +4, Volontà +3}
    \item[\textbf{Movimento:}] 9 m, nuoto 9 m
    \item[\textbf{Sfida:}] 1/4 (50 PX)\smallskip
\end{description}

\textbf{Azioni}

\emph{\textbf{Morso.} Attacco con Arma da Mischia}: +4 a colpire, portata 3 m, un bersaglio.

\emph{Colpisce:} 6 (1d4 + 4) danni perforanti e il bersaglio deve effettuare un Tiro Salvezza di Tempra DC 11, e subire 10 (3d6) danni da veleno se fallisce il Tiro Salvezza, o la metà di questi danni se lo riesce.

\mostro{Serpente Volante}
\begin{description}[noitemsep, topsep=0pt, parsep=0pt, partopsep=0pt, leftmargin=0cm, labelwidth=2.2cm]
    \item[\textbf{Taglia/Tipo:}] Minuscola bestia, disallineato
    \item[\textbf{Caratt.:}] \resizebox{0.5\linewidth+1.8cm}{!}{For -3 Des 4 Cos 0 Int -4 Sag 1 Car -3}
    \item[\textbf{Punti Ferita:}] 17,  \textbf{Difesa:} 16,  \textbf{Iniziativa:} +4
    \item[\textbf{Tiri Salvez.:}] \resizebox{0.5\linewidth+1.8cm}{!}{Tempra +3, Riflessi +4, Volontà +3}
    \item[\textbf{Movimento:}] 9 m, nuoto 9 m, volo 18 m
    \item[\textbf{Sfida:}] 1/8 (25 PX)\smallskip
\end{description}

\emph{\textbf{Sorvolare.}} Il serpente non provoca attacchi di opportunità quando vola via dalla portata di un nemico.

\textbf{Azioni}

\emph{\textbf{Morso.} Attacco con Arma da Mischia}: +4 a colpire, portata 1 m, un bersaglio.

\emph{Colpisce:} 1 danno perforante più 7 (3d4) danni da veleno.

\mostro{Squalo Cacciatore}
\begin{description}[noitemsep, topsep=0pt, parsep=0pt, partopsep=0pt, leftmargin=0cm, labelwidth=2.2cm]
    \item[\textbf{Taglia/Tipo:}] Grande bestia, disallineato
    \item[\textbf{Caratt.:}] \resizebox{0.5\linewidth+1.8cm}{!}{For 4 Des 1 Cos 2 Int -5 Sag 0 Car -3}
    \item[\textbf{Punti Ferita:}] 51,  \textbf{Difesa:} 15,  \textbf{Iniziativa:} +1
    \item[\textbf{Tiri Salvez.:}] \resizebox{0.5\linewidth+1.8cm}{!}{Tempra +4, Riflessi +3, Volontà +3}
    \item[\textbf{Movimento:}] 0 m, nuoto 12 m
    \item[\textbf{Sfida:}] 2 (450 PX)\smallskip
\end{description}

\emph{\textbf{Frenesia Sanguinaria.}} Lo squalo ha +1d6 ai tiri di attacco in mischia contro qualsiasi creatura che non sia al massimo dei Punti Ferita.

\emph{\textbf{Respirare Acqua.}} Lo squalo può respirare solo sott'acqua.

\textbf{Azioni}

\emph{\textbf{Morso.} Attacco con Arma da Mischia}: +4 a colpire, portata 1 m, un bersaglio.

\emph{Colpisce:} 13 (2d8 + 4) danni perforanti.

\mostro{Squalo Corallino}
\begin{description}[noitemsep, topsep=0pt, parsep=0pt, partopsep=0pt, leftmargin=0cm, labelwidth=2.2cm]
    \item[\textbf{Taglia/Tipo:}] Media bestia, disallineato
    \item[\textbf{Caratt.:}] \resizebox{0.5\linewidth+1.8cm}{!}{For 2 Des 1 Cos 1 Int -5 Sag 0 Car -3}
    \item[\textbf{Punti Ferita:}] 24,  \textbf{Difesa:} 13,  \textbf{Iniziativa:} +1
    \item[\textbf{Tiri Salvez.:}] \resizebox{0.5\linewidth+1.8cm}{!}{Tempra +3, Riflessi +3, Volontà +3}
    \item[\textbf{Movimento:}] 0 m, nuoto 12 m
    \item[\textbf{Sfida:}] 1/2 (100 PX)\smallskip
\end{description}

\emph{\textbf{Respirare Acqua.}} Lo squalo può respirare solo sott'acqua.

\emph{\textbf{Tattiche di Branco.}} Lo squalo ha +1d6 al tiro di attacco contro una creatura se almeno uno degli alleati dello squalo si trova entro 1 metro dalla creatura e quell'alleato non è inabile.

\textbf{Azioni}

\emph{\textbf{Morso.} Attacco con Arma da Mischia}: +4 a colpire, portata 1 m, un bersaglio.

\emph{Colpisce:} 6 (1d8 + 2) danni perforanti.

\mostro{Squalo Gigante}
\begin{description}[noitemsep, topsep=0pt, parsep=0pt, partopsep=0pt, leftmargin=0cm, labelwidth=2.2cm]
    \item[\textbf{Taglia/Tipo:}] Enorme bestia, disallineato
    \item[\textbf{Caratt.:}] \resizebox{0.5\linewidth+1.8cm}{!}{For 6 Des 0 Cos 5 Int -5 Sag 0 Car -3}
    \item[\textbf{Punti Ferita:}] 110,  \textbf{Difesa:} 18,  \textbf{Iniziativa:} +0
    \item[\textbf{Tiri Salvez.:}] \resizebox{0.5\linewidth+1.8cm}{!}{Tempra +10, Riflessi +5, Volontà +5}
    \item[\textbf{Movimento:}] 0 m, nuoto 15 m
    \item[\textbf{Sfida:}] 5 (1800 PX)\smallskip
\end{description}

\emph{\textbf{Frenesia Sanguinaria.}} Lo squalo ha +1d6 ai tiri di attacco

in mischia contro qualsiasi creatura che non sia al massimo dei Punti Ferita.

\emph{\textbf{Respirare Acqua.}} Lo squalo può respirare solo sott'acqua.

\textbf{Azioni}

\emph{\textbf{Morso.} Attacco con Arma da Mischia}: +7 a colpire, portata 1 m, un bersaglio.

\emph{Colpisce:} 22 (3d10 + 6) danni perforanti.

\mostro{Strige}
\begin{description}[noitemsep, topsep=0pt, parsep=0pt, partopsep=0pt, leftmargin=0cm, labelwidth=2.2cm]
    \item[\textbf{Taglia/Tipo:}] Minuscola bestia, disallineato
    \item[\textbf{Caratt.:}] \resizebox{0.5\linewidth+1.8cm}{!}{For -3 Des 3 Cos 0 Int -4 Sag -1 Car -2}
    \item[\textbf{Punti Ferita:}] 17,  \textbf{Difesa:} 15,  \textbf{Iniziativa:} +3
    \item[\textbf{Tiri Salvez.:}] \resizebox{0.5\linewidth+1.8cm}{!}{Tempra +3, Riflessi +3, Volontà +3}
    \item[\textbf{Movimento:}] 3 m, volo 12 m
    \item[\textbf{Sfida:}] 1/8 (25 PX)\smallskip
\end{description}

\textbf{Azioni}

\emph{\textbf{Risucchio di Sangue.} Attacco con Arma da Mischia}: +4 a colpire, portata 1 m, una creatura.

\emph{Colpisce:} 5 (1d4 + 3) danni perforanti e lo strige si attacca al bersaglio. Mentre è attaccato, lo strige non attacca. Invece, all'inizio di ciascun round dello strige, il bersaglio perde 5 (1d4 + 3) Punti Ferita a causa della perdita di sangue.

Lo strige può staccarsi spendendo 1 Azione. Lo fa automaticamente dopo aver risucchiato 10 Punti Ferita dal bersaglio o alla morte del bersaglio. Una creatura, compreso il bersaglio, può usare una Azione per staccare lo strige.

\mostro{Tasso}
\begin{description}[noitemsep, topsep=0pt, parsep=0pt, partopsep=0pt, leftmargin=0cm, labelwidth=2.2cm]
    \item[\textbf{Taglia/Tipo:}] Minuscola bestia, disallineato
    \item[\textbf{Caratt.:}] \resizebox{0.5\linewidth+1.8cm}{!}{For -3 Des 0 Cos 1 Int -4 Sag 1 Car -3}
    \item[\textbf{Punti Ferita:}] 15,  \textbf{Difesa:} 12,  \textbf{Iniziativa:} +0
    \item[\textbf{Tiri Salvez.:}] \resizebox{0.5\linewidth+1.8cm}{!}{Tempra +3, Riflessi +3, Volontà +3}
    \item[\textbf{Movimento:}] 6 m, scavo 1 m
    \item[\textbf{Sfida:}] 0(10 PX)\smallskip
\end{description}

\emph{\textbf{Olfatto Affinato.}} Il tasso ha +1d6 alle prove di Consapevolezza basate sull'olfatto.

\textbf{Azioni}

\emph{\textbf{Morso.} Attacco con Arma da Mischia}: +3 a colpire, portata 1 m, un bersaglio.

\emph{Colpisce:} 1 danno perforante.

\mostro{Tasso Gigante}
\begin{description}[noitemsep, topsep=0pt, parsep=0pt, partopsep=0pt, leftmargin=0cm, labelwidth=2.2cm]
    \item[\textbf{Taglia/Tipo:}] Media bestia, disallineato
    \item[\textbf{Caratt.:}] \resizebox{0.5\linewidth+1.8cm}{!}{For 1 Des 0 Cos 2 Int -4 Sag 1 Car -3}
    \item[\textbf{Punti Ferita:}] 19,  \textbf{Difesa:} 12,  \textbf{Iniziativa:} +0
    \item[\textbf{Tiri Salvez.:}] \resizebox{0.5\linewidth+1.8cm}{!}{Tempra +3, Riflessi +3, Volontà +3}
    \item[\textbf{Movimento:}] 9 m, scavo 3 m
    \item[\textbf{Sfida:}] 1/4 (50 PX)\smallskip
\end{description}

\emph{\textbf{Olfatto Affinato.}} Il tasso ha +1d6 alle prove di Consapevolezza basate sull'olfatto.

\textbf{Azioni}

\emph{\textbf{Multiattacco.}} Il tasso effettua due attacchi: uno con il morso e uno con gli artigli.

\emph{\textbf{Artigli.} Attacco con Arma da Mischia}: +3 a colpire, portata 1 m, un bersaglio.

\emph{Colpisce:} 6 (2d4 + 1) danni taglienti.

\emph{\textbf{Morso.} Attacco con Arma da Mischia}: +3 a colpire, portata 1 m, un bersaglio.

\emph{Colpisce:} 4 (1d6 + 1) danni perforanti.

\mostro{Tigre}
\begin{description}[noitemsep, topsep=0pt, parsep=0pt, partopsep=0pt, leftmargin=0cm, labelwidth=2.2cm]
    \item[\textbf{Taglia/Tipo:}] Grande bestia, disallineato
    \item[\textbf{Caratt.:}] \resizebox{0.5\linewidth+1.8cm}{!}{For 3 Des 2 Cos 2 Int -4 Sag 1 Car -1}
    \item[\textbf{Punti Ferita:}] 33,  \textbf{Difesa:} 15,  \textbf{Iniziativa:} +2
    \item[\textbf{Tiri Salvez.:}] \resizebox{0.5\linewidth+1.8cm}{!}{Tempra +3, Riflessi +3, Volontà +3}
    \item[\textbf{Movimento:}] 12 m
    \item[\textbf{Sfida:}] 1 (200 PX)\smallskip
\end{description}

\emph{\textbf{Balzo.}} Se la tigre si muove di almeno 6 metri diretta verso una creatura e la colpisce con un attacco di artiglio durante lo stesso round, il bersaglio deve riuscire un Tiro Salvezza di Tempra DC 13 o cadere prono. Se il bersaglio è prono, la tigre può effettuare un attacco di morso contro di esso come Azione Immediata.

\emph{\textbf{Olfatto Affinato.}} La tigre ha +1d6 alle prove di Consapevolezza basate sull'olfatto.

\textbf{Azioni}

\emph{\textbf{Artiglio.} Attacco con Arma da Mischia}: +5 a colpire, portata 1 m, un bersaglio.

\emph{Colpisce:} 7 (1d8 + 3) danni taglienti, 1 danno da Sanguinamento.

\emph{\textbf{Morso.} Attacco con Arma da Mischia}: +5 a colpire, portata 1 m, un bersaglio.

\emph{Colpisce:} 8 (1d10 + 3) danni perforanti.

\mostro{Tigre dai Denti a Sciabola}\label{tigrilla}\hypertarget{Smilodonte}{}\hypertarget{Trigrilla}{}\index[Mostruario]{Trigrilla}
\begin{description}[noitemsep, topsep=0pt, parsep=0pt, partopsep=0pt, leftmargin=0cm, labelwidth=2.2cm]
    \item[\textbf{Taglia/Tipo:}] Grande bestia, disallineato
    \item[\textbf{Caratt.:}] \resizebox{0.5\linewidth+1.8cm}{!}{For 4 Des 2 Cos 2 Int -3 Sag 1 Car 0}
    \item[\textbf{Punti Ferita:}] 51,  \textbf{Difesa:} 16,  \textbf{Iniziativa:} +2
    \item[\textbf{Tiri Salvez.:}] \resizebox{0.5\linewidth+1.8cm}{!}{Tempra +4, Riflessi +4, Volontà +3}
    \item[\textbf{Movimento:}] 12 m
    \item[\textbf{Sfida:}] 2 (450 PX)\smallskip
\end{description}

\emph{\textbf{Balzo.}} Se la tigre si muove di almeno 6 metri diretta verso una creatura e la colpisce con un attacco di artiglio durante lo stesso round, il bersaglio deve riuscire un Tiro Salvezza di Tempra DC 16 o cadere prono. Se il bersaglio è prono, la tigre può effettuare un attacco di morso contro di esso come Azione Immediata.

\emph{\textbf{Olfatto Affinato.}} La tigre ha +1d6 alle prove di Consapevolezza basate sull'olfatto.

\textbf{Azioni}

\emph{\textbf{Artiglio.} Attacco con Arma da Mischia}: +6 a colpire, portata 1 m, un bersaglio.

\emph{Colpisce:} 12 (2d6 + 5) danni taglienti, 1 danno da Sanguinamento.

\emph{\textbf{Morso.} Attacco con Arma da Mischia}: +6 a colpire, portata 1 m, un bersaglio.

\emph{Colpisce:} 10 (1d10 + 5) danni perforanti.

\mostro{Vespa Gigante}
\begin{description}[noitemsep, topsep=0pt, parsep=0pt, partopsep=0pt, leftmargin=0cm, labelwidth=2.2cm]
    \item[\textbf{Taglia/Tipo:}] Media bestia, disallineato
    \item[\textbf{Caratt.:}] \resizebox{0.5\linewidth+1.8cm}{!}{For 0 Des 2 Cos 0 Int -5 Sag 0 Car -4}
    \item[\textbf{Punti Ferita:}] 24,  \textbf{Difesa:} 14,  \textbf{Iniziativa:} +2
    \item[\textbf{Tiri Salvez.:}] \resizebox{0.5\linewidth+1.8cm}{!}{Tempra +3, Riflessi +3, Volontà +3}
    \item[\textbf{Movimento:}] 3 m, volo 15 m
    \item[\textbf{Sfida:}] 1/2 (100 PX)\smallskip
\end{description}

\textbf{Azioni}

\emph{\textbf{Pungiglione.} Attacco con Arma da Mischia}: +4 a colpire, portata 1 m, una creatura.

\emph{Colpisce:} 5 (1d6 + 2) danni perforanti e il bersaglio deve effettuare un Tiro Salvezza di Tempra DC 11, e subire 10 (3d6) danni da veleno se fallisce il Tiro Salvezza, o la metà di questi danni se lo riesce. Se il danno da veleno riduce il bersaglio a 0 Punti Ferita, il bersaglio è stabile ma avvelenato per 1 ora, anche dopo aver recuperato i Punti Ferita, e mentre è avvelenato in questo modo resta paralizzato.

\mostro{Worg}
\begin{description}[noitemsep, topsep=0pt, parsep=0pt, partopsep=0pt, leftmargin=0cm, labelwidth=2.2cm]
    \item[\textbf{Taglia/Tipo:}] Grande mostruosità, malvagio
    \item[\textbf{Caratt.:}] \resizebox{0.5\linewidth+1.8cm}{!}{For 3 Des 1 Cos 1 Int -2 Sag 0 Car -1}
    \item[\textbf{Punti Ferita:}] 24,  \textbf{Difesa:} 13,  \textbf{Iniziativa:} +1
    \item[\textbf{Tiri Salvez.:}] \resizebox{0.5\linewidth+1.8cm}{!}{Tempra +3, Riflessi +3, Volontà +3}
    \item[\textbf{Movimento:}] 15 m
    \item[\textbf{Sfida:}] 1/2 (100 PX)\smallskip
\end{description}

\emph{\textbf{Udito e Olfatto Affinato.}} Il worg ha +1d6 nelle prove di Consapevolezza basate su udito o olfatto.

\textbf{Azioni}

\emph{\textbf{Morso.} Attacco con Arma da Mischia}: +5 a colpire, portata 1 m, un bersaglio.

\emph{Colpisce:} 10 (2d6 + 3) danni perforanti. Se il bersaglio è una creatura, deve riuscire un Tiro Salvezza di Tempra DC 13 o cadere prona.

\subsection{Appendice B: Personaggi Non Giocanti}\noindent\rule{\linewidth}{2pt} \index[Mostruario]{Personaggi Non Giocanti}

Questa appendice contiene le statistiche di vari personaggi non giocanti (PNG) umanoidi che gli avventurieri possono incontrare nel corso di una campagna, da infimi popolani a potenti arcimaghi. Queste statistiche possono essere utilizzate per rappresentare PNG umani e non.

Personalizzare i PNG

Esistono molti semplici modi di personalizzare i PNG di questa appendice per l'uso nella tua campagna casalinga.

\emph{\textbf{Cambiare Incantesimi.}} Un modo per personalizzare un PNG incantatore è quello di rimpiazzare uno o più dei suoi incantesimi. Puoi sostituire qualsiasi incantesimo della lista di
incantesimi del PNG con un diverso incantesimo dello stesso livello. Cambiare incantesimi in questo modo non modifica il grado di sfida del PNG.

\textbf{\emph{Cambiare Armi e Armatura}.} Puoi migliorare o peggiorare l'armatura del PNG o aggiungere o cambiare armi. Le modifiche alla Difesa e ai danni possono modificare il grado di sfida del PNG.

\emph{\textbf{Oggetti Magici}}. più potente è un PNG, maggiori le probabilità che possieda uno o più oggetti magici. Un mago, ad esempio, potrebbe avere una bacchetta o un bastone magico, oltre ad una o più pozioni e pergamene. Fornire un PNG di un potente oggetto magico capace di infliggere danni potrebbe modificarne il grado di sfida.

Alcuni oggetti magici di esempio sono descritti più avanti in questo documento.

\textbf{Combattenti}

I combattenti sono individui che si guadagnano da vivere mettendo la loro spada al servizio di un individuo o un ideale.

\mostro{Guardia}
\begin{description}[noitemsep, topsep=0pt, parsep=0pt, partopsep=0pt, leftmargin=0cm, labelwidth=2.2cm]
    \item[\textbf{Taglia/Tipo:}] Media umanoide, qualsiasi Tratto
    \item[\textbf{Caratt.:}] \resizebox{0.5\linewidth+1.8cm}{!}{For 1 Des 1 Cos 1 Int 0 Sag 0 Car 0}
    \item[\textbf{Punti Ferita:}] 24,  \textbf{Difesa:} 13,  \textbf{Iniziativa:} +1
    \item[\textbf{Tiri Salvez.:}] \resizebox{0.5\linewidth+1.8cm}{!}{Tempra +3, Riflessi +3, Volontà +3}
    \item[\textbf{Movimento:}] 9 m
    \item[\textbf{Sfida:}] 1/2 (100 PX)\smallskip
\end{description}

Le guardie comprendono membri della ronda cittadina, sentinelle di una cittadella o città fortificata e le guardie del corpo di nobili e mercanti.

\textbf{Azioni}

\emph{\textbf{Lancia.} Attacco con Arma da Mischia o a Gittata}: +3 a colpire, portata 1 m o gittata 6m, un bersaglio.

\emph{Colpisce:} 4 (1d6 + 1) danni perforanti o 5 (1d8 + 1) danni perforanti se impiegata con due mani per effettuare un attacco da mischia.

\mostro{Veterano}
\begin{description}[noitemsep, topsep=0pt, parsep=0pt, partopsep=0pt, leftmargin=0cm, labelwidth=2.2cm]
    \item[\textbf{Taglia/Tipo:}] Media umanoide, qualsiasi Tratto
    \item[\textbf{Caratt.:}] \resizebox{0.5\linewidth+1.8cm}{!}{For 3 Des 1 Cos 2 Int 0 Sag 0 Car 0}
    \item[\textbf{Punti Ferita:}] 70,  \textbf{Difesa:} 17,  \textbf{Iniziativa:} +1
    \item[\textbf{Comp.:}] Atletica +5
    \item[\textbf{Tiri Salvez.:}] \resizebox{0.5\linewidth+1.8cm}{!}{Tempra +5, Riflessi +4, Volontà +3}
    \item[\textbf{Movimento:}] 9 m
    \item[\textbf{Linguaggi:}] Comune
    \item[\textbf{Sfida:}] 3 (700 PX)\smallskip
\end{description}

Guerrieri sopravvissuti a lungo, guadagnandosi una grande fama di esperti e abili combattenti.

\textbf{Azioni}

\emph{\textbf{Multiattacco.}} Il veterano effettua due attacchi con la spada lunga. Se ha estratto una spada corta, può effettuare anche un attacco con la spada corta.

\emph{\textbf{Spada Lunga.} Attacco con Arma da Mischia}: +5 a colpire, portata 1 m, un bersaglio.

\emph{Colpisce:} 7 (1d8 + 3) danni taglienti, o 8 (1d10 + 3) danni taglienti se usata con due mani.

\emph{\textbf{Spada Corta.} Attacco con Arma da Mischia}: +5 a colpire, portata 1 m, un bersaglio.

\emph{Colpisce:} 6 (1d6 + 3) danni perforanti.

\emph{\textbf{Balestra Pesante.} Attacco con Arma a Gittata}: +3 a colpire, gittata 30m, un bersaglio.

\emph{Colpisce:} 6 (1d10 + 1) danni perforanti.

\mostro{Cavaliere}
\begin{description}[noitemsep, topsep=0pt, parsep=0pt, partopsep=0pt, leftmargin=0cm, labelwidth=2.2cm]
    \item[\textbf{Taglia/Tipo:}] Media umanoide, qualsiasi Tratto
    \item[\textbf{Caratt.:}] \resizebox{0.5\linewidth+1.8cm}{!}{For 3 Des 0 Cos 2 Int 0 Sag 0 Car 2}
    \item[\textbf{Punti Ferita:}] 70,  \textbf{Difesa:} 16,  \textbf{Iniziativa:} +0
    \item[\textbf{Tiri Salvez.:}] \resizebox{0.5\linewidth+1.8cm}{!}{Tempra +5, Riflessi +3, Volontà +3}
    \item[\textbf{Movimento:}] 9 m
    \item[\textbf{Sfida:}] 3 (700 PX)\smallskip
\end{description}

I cavalieri sono combattenti che giurano fedeltà a sovrani, ordini religiosi, e nobili cause. I Tratti del cavaliere determinano fino a che punto è disposto ad onorare il suo giuramento.

\emph{\textbf{Coraggioso.}} Il cavaliere ha +1d6 ai Tiri Salvezza contro l'essere spaventato.

\textbf{Azioni}

\emph{\textbf{Multiattacco.}} Il cavaliere effettua due attacchi da mischia.

\emph{\textbf{Spada Grossa.} Attacco con Arma da Mischia}: +5 a colpire, portata 1 m, un bersaglio.

\emph{Colpisce:} 10 (2d6 + 3) danni taglienti.

\emph{\textbf{Balestra Pesante.} Attacco con Arma a Gittata}: +2 a colpire, gittata 30m, un bersaglio.

\emph{Colpisce:} 5 (1d10) perforanti.

\emph{\textbf{Autorità (Ricarica dopo un 1 ora)}}. Per 1 minuto, il cavaliere può pronunciare un comando speciale o avvertimento ogni qualvolta una creatura non ostile entro 9 metri da lui, e che possa vedere, effettua un tiro di attacco o Tiro Salvezza. La creatura può sommare un d4 al suo tiro purché possa udire e comprendere il cavaliere. Una creatura può beneficiare di un solo dado Autorità alla volta. Questo effetto termina se il cavaliere è inabile.

\textbf{Reazioni}

\emph{\textbf{Parata.}} Il cavaliere può aggiungere 2 alla sua Difesa contro un attacco da mischia che lo colpirebbe. Per farlo, il cavaliere deve vedere l'attaccante e star impugnando un'arma da mischia.

%\begin{center}
%	\includegraphics[width=0.3\textwidth]{immagini/Knight_Death_and_the_Devil_MET_DP159047.png}

%	\emph{Cavaliere, Morte e Diavolo. Albrecht Durer}

%\end{center}

\noindent{\large\textbf{Cittadini}}

\noindent\rule{\linewidth}{2pt} \index[Mostruario]{Cittadini}\hypertarget{Cittadini}{}

In questa categoria rientrano quegli individui che si occupano di mandare avanti il mondo, svolgendo le mansioni necessarie affinché i campi vengano coltivati, le città amministrate, il cibo coltivato e
nuovi territori esplorati.

\mostro{Nobile}
\begin{description}[noitemsep, topsep=0pt, parsep=0pt, partopsep=0pt, leftmargin=0cm, labelwidth=2.2cm]
    \item[\textbf{Taglia/Tipo:}] Media umanoide, qualsiasi Tratto
    \item[\textbf{Caratt.:}] \resizebox{0.5\linewidth+1.8cm}{!}{For 0 Des 1 Cos 0 Int 1 Sag 2 Car 3}
    \item[\textbf{Punti Ferita:}] 17,  \textbf{Difesa:} 13,  \textbf{Iniziativa:} +1
    \item[\textbf{Comp.:}] Percepire Emozioni +4, Ingannare +5
    \item[\textbf{Tiri Salvez.:}] \resizebox{0.5\linewidth+1.8cm}{!}{Tempra +3, Riflessi +3, Volontà +3}
    \item[\textbf{Movimento:}] 9 m
    \item[\textbf{Linguaggi:}] due lingue qualsiasi
    \item[\textbf{Sfida:}] 1/8 (25 PX)\smallskip
\end{description}

I nobili comandano sulla popolazione, in virtù di un diritto di nascita o per le ricchezze accumulate. Tra costoro si annoverano anche i cortigiani che affollano le corti dei ricchi e dei potenti.

\textbf{Azioni}

\emph{\textbf{Stocco.} Attacco con Arma da Mischia}: +3 a colpire, portata 1 m, un bersaglio.

\emph{Colpisce:} 5 (1d8 + 1) danni perforanti.

\textbf{Reazioni}

\emph{\textbf{Parata.}} Il nobile somma 2 alla sua Difesa contro un attacco da mischia che lo colpirebbe. Per farlo, il nobile deve vedere l'attaccante e impugnare un'arma da mischia.

\mostro{Popolano}
\begin{description}[noitemsep, topsep=0pt, parsep=0pt, partopsep=0pt, leftmargin=0cm, labelwidth=2.2cm]
    \item[\textbf{Taglia/Tipo:}] Media umanoide, qualsiasi Tratto
    \item[\textbf{Caratt.:}] \resizebox{0.5\linewidth+1.8cm}{!}{For 0 Des 0 Cos 0 Int 0 Sag 0 Car 0}
    \item[\textbf{Punti Ferita:}] 17,  \textbf{Difesa:} 12,  \textbf{Iniziativa:} +0
    \item[\textbf{Tiri Salvez.:}] \resizebox{0.5\linewidth+1.8cm}{!}{Tempra +3, Riflessi +3, Volontà +3}
    \item[\textbf{Movimento:}] 9 m
    \item[\textbf{Linguaggi:}] Comune
    \item[\textbf{Sfida:}] 1/8 (25 PX)\smallskip
\end{description}

I popolani comprendono contadini, servi, schiavi, servitori, pellegrini, mercanti, artigiani ed eremiti.

\textbf{Azioni}

\emph{\textbf{Randello.} Attacco con Arma da Mischia}: +2 a colpire, portata 1 m, un bersaglio.

\emph{Colpisce:} 2 (1d4) danni contundenti.

\medskip\textbf{Criminali}

I criminali sono individui che vivono al margine della legalità, procurandosi il pane svolgendo attività spesso considerate illecite e immorali.

\mostro{Bandito/Pirata}
\begin{description}[noitemsep, topsep=0pt, parsep=0pt, partopsep=0pt, leftmargin=0cm, labelwidth=2.2cm]
    \item[\textbf{Taglia/Tipo:}] Media umanoide, qualsiasi Tratto
    \item[\textbf{Caratt.:}] \resizebox{0.5\linewidth+1.8cm}{!}{For 0 Des 1 Cos 1 Int 0 Sag 0 Car 0}
    \item[\textbf{Punti Ferita:}] 17,  \textbf{Difesa:} 13,  \textbf{Iniziativa:} +1
    \item[\textbf{Tiri Salvez.:}] \resizebox{0.5\linewidth+1.8cm}{!}{Tempra +3, Riflessi +3, Volontà +3}
    \item[\textbf{Movimento:}] 9 m
    \item[\textbf{Linguaggi:}] Comune
    \item[\textbf{Sfida:}] 1/8 (25 PX)\smallskip
\end{description}

Che siano uomini di strada o di mare (pirati) costoro guadagnano da vivere depredando il prossimo.

\textbf{Azioni}

\emph{\textbf{Scimitarra.} Attacco con Arma da Mischia}: +3 a colpire, portata 1 m, un bersaglio.

\emph{Colpisce:} 4 (1d6 + 1) danni taglienti.

\emph{\textbf{Balestra Leggera.} Attacco con Arma a Gittata}: +3 a colpire, gittata 24m, un bersaglio.

\emph{Colpisce:} 5 (1d8 + 1) danni taglienti.

\mostro{Spia}
\begin{description}[noitemsep, topsep=0pt, parsep=0pt, partopsep=0pt, leftmargin=0cm, labelwidth=2.2cm]
    \item[\textbf{Taglia/Tipo:}] Media umanoide, qualsiasi Tratto
    \item[\textbf{Caratt.:}] \resizebox{0.5\linewidth+1.8cm}{!}{For 0 Des 2 Cos 0 Int 1 Sag 2 Car 3}
    \item[\textbf{Punti Ferita:}] 33,  \textbf{Difesa:} 15,  \textbf{Iniziativa:} +2
    \item[\textbf{Tiri Salvez.:}] \resizebox{0.5\linewidth+1.8cm}{!}{Tempra +3, Riflessi +3, Volontà +3}
    \item[\textbf{Movimento:}] 9 m
    \item[\textbf{Linguaggi:}] due lingue qualsiasi
    \item[\textbf{Sfida:}] 1 (200 PX)\smallskip
\end{description}

Una spia è un individuo addestramento nel reperire segreti per conto di qualcuno, o a volte per rivenderli al miglior offerente.

\emph{\textbf{Attacco Furtivo (1/Turno).}} La spia infligge 7 (2d6) danni aggiuntivi quando colpisce un bersaglio con un attacco con arma e ha +1d6 al tiro di attacco, o quando il bersaglio è entro 1 metro da un alleato dell'assassino che non è inabile e l'assassino non ha -1d6 al tiro di attacco.

\emph{\textbf{Azione Astuta.}} Durante ciascun suo round, la spia può usare una Azione Immediata per effettuare l'azione Disingaggiare, nascondersi o Scattare.

\textbf{Azioni}

\emph{\textbf{Multiattacco.}} La spia effettua due attacchi da mischia.

\emph{\textbf{Spada Corta.} Attacco con Arma da Mischia}: +4 a colpire, portata 1 m, un bersaglio.

\emph{Colpisce:} 5 (1d6 + 2) danni perforanti.

\emph{\textbf{Balestrino.} Attacco con Arma a Gittata}: +4 a colpire, gittata 9m, un bersaglio.

\emph{Colpisce:} 5 (1d6 + 2) danni perforanti.

\mostro{Capitano dei Banditi o Pirata}
\begin{description}[noitemsep, topsep=0pt, parsep=0pt, partopsep=0pt, leftmargin=0cm, labelwidth=2.2cm]
    \item[\textbf{Taglia/Tipo:}] Media umanoide, qualsiasi Tratto
    \item[\textbf{Caratt.:}] \resizebox{0.5\linewidth+1.8cm}{!}{For 2 Des 3 Cos 2 Int 2 Sag 0 Car 2}
    \item[\textbf{Punti Ferita:}] 51,  \textbf{Difesa:} 17,  \textbf{Iniziativa:} +3
    \item[\textbf{Comp.:}] Atletica +4, Ingannare +4
    \item[\textbf{Tiri Salvez.:}] \resizebox{0.5\linewidth+1.8cm}{!}{Tempra +4, Riflessi +5, Volontà +3}
    \item[\textbf{Movimento:}] 9 m
    \item[\textbf{Linguaggi:}]  due lingue qualsiasi
    \item[\textbf{Sfida:}] 2 (450 PX)\smallskip
\end{description}

Che viva in terra o in mare, è un individuo munito di una grande personalità che riesce a tenere in riga la marmaglia che risponde ai suoi ordini.

\textbf{Azioni}

\emph{\textbf{Multiattacco.}} Il capitano effettua tre attacchi da mischia: due con la scimitarra e uno con il pugnale. Oppure il capitano effettua due attacchi a gittata con i pugnali.

\emph{\textbf{Scimitarra.} Attacco con Arma da Mischia}: +5 a colpire, portata 1 m, un bersaglio.

\emph{Colpisce:} 6 (1d6 + 3) danni taglienti.

\emph{\textbf{Pugnale.} Attacco con Arma da Mischia o a Gittata}: +5 a colpire, portata 1 m o gittata 6m, un bersaglio.

\emph{Colpisce:} 5 (1d4 + 3) danni perforanti.

\textbf{Reazioni}

\emph{\textbf{Parata.}} Il capitano somma 2 alla sua Difesa contro un attacco da mischia che lo colpirebbe. Per farlo, il capitano deve vedere l'attaccante e impugnare un'arma da mischia.

\mostro{Assassino}
\begin{description}[noitemsep, topsep=0pt, parsep=0pt, partopsep=0pt, leftmargin=0cm, labelwidth=2.2cm]
    \item[\textbf{Taglia/Tipo:}] Media umanoide, qualsiasi Tratto
    \item[\textbf{Caratt.:}] \resizebox{0.5\linewidth+1.8cm}{!}{For 0 Des 3 Cos 2 Int 1 Sag 0 Car 0}
    \item[\textbf{Punti Ferita:}] 162,  \textbf{Difesa:} 25,  \textbf{Iniziativa:} +3
    \item[\textbf{Comp.:}] Acrobazia +6, Furtività +9, Ingannare +3
    \item[\textbf{Tiri Salvez.:}] \resizebox{0.5\linewidth+1.8cm}{!}{Tempra +10, Riflessi +11, Volontà +8}
    \item[\textbf{Movimento:}] 9 m
    \item[\textbf{Linguaggi:}] Comune più due altre lingue
    \item[\textbf{Sfida:}] 8 (3900 PX)\smallskip
\end{description}

Solitari o membri di una gilda, gli assassini sono pagati per eliminare, spesso in modo silenzioso e discreto, rivali e nemici dei loro datori di lavoro.

\emph{\textbf{Assassinare.}} Durante il suo primo round, l'assassino ha +1d6 ai tiri di attacco contro le creature che non hanno ancora svolto nessun round. Qualsiasi colpo che l'assassino mandi a segno contro una creatura sorpresa, è un colpo critico.

\emph{\textbf{Attacco Furtivo (1/Turno).}} L'assassino infligge 14 (4d6) danni aggiuntivi quando colpisce un bersaglio con un attacco con arma e ha +1d6 al tiro di attacco, o quando il bersaglio è entro 1 metro da un alleato dell'assassino che non è inabile e l'assassino non ha -1d6 al tiro di attacco.

\emph{\textbf{Evasione.}} Se l'assassino è vittima di un effetto che permette di effettuare un Tiro Salvezza di Riflessi per dimezzare i danni, l'assassino non prende danni se riesce il Tiro Salvezza, e solo la metà se lo fallisce.

\textbf{Azioni}

\emph{\textbf{Multiattacco.}} L'assassino effettua due attacchi con le spade corte.

\emph{\textbf{Spada Corta.} Attacco con Arma da Mischia}: +6 a colpire, portata 1 m, un bersaglio.

\emph{Colpisce:} 6 (1d6 + 3) danni perforanti, e il bersaglio deve effettuare un Tiro Salvezza di Tempra DC 19, subendo 24 (7d6) danni da veleno se fallisce il Tiro Salvezza, o la metà di questi danni se lo riesce.

\emph{\textbf{Balestra Leggera.} Attacco con Arma a Gittata}: +6 a colpire, gittata 24m, un bersaglio.

\emph{Colpisce:} 7 (1d8 + 3) danni perforanti, e il bersaglio deve effettuare un Tiro Salvezza di Tempra DC 19, subendo 24 (7d6) danni da veleno se fallisce il Tiro Salvezza, o la metà di questi danni se lo riesce.

\textbf{Reazione: \emph{Attacco d'opportunità}}: l'assassino effettua un attacco con spada corta ad una creatura che attraversi o esca dalla sua portata di 1 metro.

\medskip\textbf{Mago}

Il mago trascorre la vita nello studio e la pratica della magia.

\mostro{Mago Avventuriero}
\begin{description}[noitemsep, topsep=0pt, parsep=0pt, partopsep=0pt, leftmargin=0cm, labelwidth=2.2cm]
    \item[\textbf{Taglia/Tipo:}] Media umanoide, qualsiasi Tratto
    \item[\textbf{Caratt.:}] \resizebox{0.5\linewidth+1.8cm}{!}{For -1 Des 2 Cos 0 Int 2 Sag 1 Car 0}
    \item[\textbf{Punti Ferita:}] 33,  \textbf{Difesa:} 15,  \textbf{Iniziativa:} +3
    \item[\textbf{Comp.:}] Arcana +5, Storia +5
    \item[\textbf{Tiri Salvez.:}] \resizebox{0.5\linewidth+1.8cm}{!}{Tempra +3, Riflessi +3, Volontà +3}
    \item[\textbf{Movimento:}] 9 m
    \item[\textbf{Linguaggi:}] quattro lingue qualsiasi
    \item[\textbf{Sfida:}] 1 (200 PX)\smallskip
\end{description}

Un Mago novizio, che ha superato con successo le sue prime avventure e ha iniziato a stabilire una reputazione come nobile o famigerato avventuriero.

\emph{\textbf{Incantesimi.}} Il mago ha CM 4. La sua abilità da incantatore è l'Intelligenza (+4 al colpire con attacchi con incantesimo). Il Mago ha preparato i seguenti incantesimi:

Trucchetti (a volontà): \emph{\hyperlink{Luce}{Luce}, \hyperlink{Mano Magica}{Mano Magica}, \hyperlink{Stretta Folgorante}{Stretta Folgorante}}

livello 1 (4 slot): \emph{\hyperlink{Charme su Persone}{Charme su Persone},  \hyperlink{Dardo arcano}{Dardo arcano}}

livello 2 (3 slot): \emph{\hyperlink{Blocca Persona}{Blocca Persona}, \hyperlink{Passo Velato}{Passo Velato}}

\textbf{Azioni}

\emph{\textbf{Bastone.} Attacco con Arma da Mischia}: +1 a colpire, portata 1 m, un bersaglio.

\emph{Colpisce:} 3 (1d8 - 1) danni contundenti.

\mostro{Grande Mago}
\begin{description}[noitemsep, topsep=0pt, parsep=0pt, partopsep=0pt, leftmargin=0cm, labelwidth=2.2cm]
    \item[\textbf{Taglia/Tipo:}] Media umanoide, qualsiasi Tratto
    \item[\textbf{Caratt.:}] \resizebox{0.5\linewidth+1.8cm}{!}{For -1 Des 2 Cos 0 Int 3 Sag 1 Car 0}
    \item[\textbf{Punti Ferita:}] 122,  \textbf{Difesa:} 22,  \textbf{Iniziativa:} +3
    \item[\textbf{Comp.:}] Arcana +6, Storia +6
    \item[\textbf{Tiri Salvez.:}] \resizebox{0.5\linewidth+1.8cm}{!}{Tempra +6, Riflessi +8, Volontà +7}
    \item[\textbf{Movimento:}] 9 m
    \item[\textbf{Linguaggi:}] quattro lingue qualsiasi
    \item[\textbf{Sfida:}] 6 (2300 PX)\smallskip
\end{description}

Un Mago che ha stabilito una discreta fama nel territorio e che attira intorno a sé studenti da ogni dove.

\emph{\textbf{Incantesimi.}} Il mago ha CM 9. La sua abilità da incantatore è l'Intelligenza (+8 al colpire con attacchi con incantesimo). Il Mago ha preparato i seguenti incantesimi:

Trucchetti (a volontà): \emph{\hyperlink{Dardo di Fuoco}{Dardo di Fuoco}, \hyperlink{Luce}{Luce}, \hyperlink{Mano Magica}{Mano Magica}, \hyperlink{Prestidigitazione}{Prestidigitazione}}

livello 1 (4 slot): \emph{\hyperlink{Armatura Magica}{Armatura Magica}, \hyperlink{Dardo arcano}{Dardo arcano}, \hyperlink{Individuazione del Magico}{Individuazione del Magico}, \hyperlink{Scudo}{Scudo}}

livello 2 (3 slot): \emph{\hyperlink{Passo Velato}{Passo Velato}, \hyperlink{Suggestione}{Suggestione}}

livello 3 (3 slot): \emph{\hyperlink{Controincantesimo}{Controincantesimo}, \hyperlink{Palla di Fuoco}{Palla di Fuoco}, volare}

livello 4 (3 slot): \emph{\hyperlink{Invisibilità Superiore}{Invisibilità Superiore}, \hyperlink{Tempesta di Ghiaccio}{Tempesta di Ghiaccio}}

livello 5 (1 slot): \emph{\hyperlink{Cono di Freddo}{Cono di Freddo}}

\textbf{Azioni}

\emph{\textbf{Pugnale.} Attacco con Arma da Mischia o a Gittata}: +5 a colpire, portata 1 m o gittata 6m, un bersaglio.

\emph{Colpisce:} 4 (1d4 + 2) danni perforanti.

\textbf{Reazione: \emph{Incantesimo opportunistico}}: il mago reagisce ad un attacco subito lanciando un trucchetto.

\mostro{Arcimago}
\begin{description}[noitemsep, topsep=0pt, parsep=0pt, partopsep=0pt, leftmargin=0cm, labelwidth=2.2cm]
    \item[\textbf{Taglia/Tipo:}] Media umanoide, qualsiasi Tratto
    \item[\textbf{Caratt.:}] \resizebox{0.5\linewidth+1.8cm}{!}{For 0 Des 2 Cos 1 Int 5 Sag 2 Car 3}
    \item[\textbf{Punti Ferita:}] 233,  \textbf{Difesa:} 30,  \textbf{Iniziativa:} +5
    \item[\textbf{Comp.:}] Arcana +13, Storia +13
    \item[\textbf{Tiri Salvez.:}] \resizebox{0.5\linewidth+1.8cm}{!}{\resizebox{0.5\linewidth+1.8cm}{!}{Tempra +13, Riflessi +14, Volontà +14}}
    \item[\textbf{Movimento:}] 9 m
    \item[\textbf{Linguaggi:}] sei lingue qualsiasi
    \item[\textbf{Sfida:}] 12 (8400 PX)\smallskip
\end{description}

Un mago molto potente (e anche molto anziano) che studia i segreti del multiverso.

\emph{\textbf{Incantesimi.}} Il mago ha CM 18. La sua abilità da incantatore è l'Intelligenza (+15 al colpire con attacchi con incantesimo).

L'arcimago può eseguire \emph{\hyperlink{Camuffare Sé Stesso}{Camuffare Sé Stesso}} e \emph{\hyperlink{Invisibilità}{Invisibilità}} a volontà e ha preparato i seguenti incantesimi:

Trucchetti (a volontà): \emph{\hyperlink{Dardo di Fuoco}{Dardo di Fuoco}, \hyperlink{Luce}{Luce}, \hyperlink{Mano Magica}{Mano Magica}. \hyperlink{Prestidigitazione}{Prestidigitazione}, \hyperlink{Stretta Folgorante}{Stretta Folgorante}}

livello 1 (4 slot): \emph{\hyperlink{Armatura Magica}{Armatura Magica}*, \hyperlink{Dardo arcano}{Dardo arcano}, \hyperlink{Identificare}{Identificare}, \hyperlink{Individuazione del Magico}{Individuazione del Magico}}

livello 2 (3 slot): \emph{\hyperlink{Immagine Speculare}{Immagine Speculare}, \hyperlink{Individuazione dei Pensieri}{Individuazione dei Pensieri}, \hyperlink{Passo Velato}{Passo Velato}}

livello 3 (3 slot): \emph{\hyperlink{Controincantesimo}{Controincantesimo}, \hyperlink{Fulmine}{Fulmine}}

livello 4 (3 slot): \emph{\hyperlink{Esilio}{Esilio}, \hyperlink{Pelle di Pietra}{Pelle di Pietra}*, \hyperlink{Scudo di Fuoco}{Scudo di Fuoco}}

livello 5 (3 slot): \emph{\hyperlink{Cono di Freddo}{Cono di Freddo}, \hyperlink{Muro di Forza}{Muro di Forza}, \hyperlink{Scrutare}{Scrutare}}

livello 6 (1 slot): \emph{\hyperlink{Globo di Invulnerabilità}{Globo di Invulnerabilità}}

livello 7 (1 slot): \emph{\hyperlink{Teletrasporto}{Teletrasporto}}

livello 8 (1 slot): \emph{\hyperlink{Scudo Mentale}{Scudo Mentale}*}

livello 9 (1 slot): \emph{\hyperlink{Fermare il Tempo}{Fermare il Tempo}}

L'arcimago esegue questi {*} incantesimi su di sé prima del combattimento.

\textbf{Azioni}

\emph{\textbf{Pugnale.} Attacco con Arma da Mischia o a Gittata}: +6 a colpire, portata 1 m o gittata 6m, un bersaglio.

\emph{Colpisce:} 4 (1d4 + 2) danni perforanti.

\textbf{Reazione: \emph{Incantesimo opportunistico}}: il mago reagisce ad un attacco subito lanciando un incantesimo di 2 livello o meno.

\medskip\textbf{Sacerdoti}

I sacerdoti sono devoti di una divinità o una fede che si prendono cura di impartire gli insegnamenti divini al loro gregge.

\mostro{Cultista}
\begin{description}[noitemsep, topsep=0pt, parsep=0pt, partopsep=0pt, leftmargin=0cm, labelwidth=2.2cm]
    \item[\textbf{Taglia/Tipo:}] Media umanoide, qualsiasi Tratto
    \item[\textbf{Caratt.:}] \resizebox{0.5\linewidth+1.8cm}{!}{For 0 Des 1 Cos 0 Int 0 Sag 0 Car 0}
    \item[\textbf{Punti Ferita:}] 17,  \textbf{Difesa:} 13,  \textbf{Iniziativa:} +1
    \item[\textbf{Comp.:}] Ingannare +2, Religione +2
    \item[\textbf{Tiri Salvez.:}] \resizebox{0.5\linewidth+1.8cm}{!}{Tempra +3, Riflessi +3, Volontà +3}
    \item[\textbf{Movimento:}] 9 m
    \item[\textbf{Linguaggi:}] Comune
    \item[\textbf{Sfida:}] 1/8 (25 PX)\smallskip
\end{description}

I cultisti giurano fedeltà ai poteri oscuri, e nelle loro credenze e pratiche mostrano spesso segni di follia.

\emph{\textbf{Oscura Devozione.}} Il cultista ha +1d6 sui Tiri Salvezza contro l'essere affascinato o spaventato.

\textbf{Azioni}

\emph{\textbf{Scimitarra.} Attacco con Arma da Mischia}: +3 a colpire, portata 1 m, una creatura.

\emph{Colpisce:} 4 (1d6 + 1) danni taglienti.

\mostro{Accolito}
\begin{description}[noitemsep, topsep=0pt, parsep=0pt, partopsep=0pt, leftmargin=0cm, labelwidth=2.2cm]
    \item[\textbf{Taglia/Tipo:}] Media umanoide, qualsiasi Tratto
    \item[\textbf{Caratt.:}] \resizebox{0.5\linewidth+1.8cm}{!}{For 0 Des 0 Cos 0 Int 0 Sag 2 Car 0}
    \item[\textbf{Punti Ferita:}] 19,  \textbf{Difesa:} 12,  \textbf{Iniziativa:} +0
    \item[\textbf{Comp.:}] Pronto Soccorso +4, Religione +2
    \item[\textbf{Tiri Salvez.:}] \resizebox{0.5\linewidth+1.8cm}{!}{Tempra +3, Riflessi +3, Volontà +3}
    \item[\textbf{Movimento:}] 9 m
    \item[\textbf{Linguaggi:}] Comune
    \item[\textbf{Sfida:}] 1/4 (50 PX)\smallskip
\end{description}

Gli accoliti sono membri di grado minore del clero, e di solito rispondono ad un sacerdote di rango superiore. Svolgono diverse funzioni in un tempio e gli viene conferita dalla loro divinità l'abilità di eseguire incantesimi minori.

\emph{\textbf{Incantesimi.}} L'accolito ha CM 1. La sua abilità da incantatore è la Saggezza (+4 al colpire con attacchi con incantesimo). L'accolito ha preparato i seguenti incantesimi:

Trucchetti (a volontà): \emph{\hyperlink{Fiamma Sacra}{Fiamma Sacra}, \hyperlink{Luce}{Luce}, \hyperlink{Taumaturgia}{Taumaturgia}}

livello 1 (3 slot): \emph{\hyperlink{Benedizione}{Benedizione}, \hyperlink{Cura Ferite}{Cura Ferite}, \hyperlink{Santuario}{Santuario}}

\medskip\textbf{Azioni}

\emph{\textbf{Randello.} Attacco con Arma da Mischia}: +2 a colpire, portata 1 m, un bersaglio.

\emph{Colpisce:} 2 (1d4) danni contundenti.

\mostro{Cultista capo}
\begin{description}[noitemsep, topsep=0pt, parsep=0pt, partopsep=0pt, leftmargin=0cm, labelwidth=2.2cm]
    \item[\textbf{Taglia/Tipo:}] Media umanoide, qualsiasi Tratto
    \item[\textbf{Caratt.:}] \resizebox{0.5\linewidth+1.8cm}{!}{For 0 Des 2 Cos 1 Int 0 Sag 1 Car 2}
    \item[\textbf{Punti Ferita:}] 33,  \textbf{Difesa:} 15,  \textbf{Iniziativa:} +2
    \item[\textbf{Comp.:}] Ingannare +4, Religione +2
    \item[\textbf{Tiri Salvez.:}] \resizebox{0.5\linewidth+1.8cm}{!}{Tempra +3, Riflessi +3, Volontà +3}
    \item[\textbf{Movimento:}] 9 m
    \item[\textbf{Linguaggi:}] Comune ed un altra lingua
    \item[\textbf{Sfida:}] 1 (200 PX)\smallskip
\end{description}

Sono i capi di un culto, che usano il proprio carisma e i propri dogmi per influenzare i deboli di volontà.

\emph{\textbf{Incantesimi.}} Il sacerdote ha CM 4. La sua abilità da incantatore è la Saggezza (+3 al colpire con attacchi con incantesimo). Il sacerdote ha preparato i seguenti incantesimi:

Trucchetti (a volontà): \emph{\hyperlink{Fiamma Sacra}{Fiamma Sacra}, \hyperlink{Luce}{Luce}, \hyperlink{Taumaturgia}{Taumaturgia}}

livello 1 (4 slot): \emph{\hyperlink{Comando}{Comando}, \hyperlink{Infliggi Ferite}{Infliggi Ferite}}

livello 2 (3 slot): \emph{\hyperlink{Arma Spirituale}{Arma Spirituale}, \hyperlink{Blocca Persona}{Blocca Persona}}

\emph{\textbf{Oscura Devozione.}} Il cultista ha +1d6 sui Tiri Salvezza contro l'essere affascinato o spaventato.

\textbf{Azioni}

\emph{\textbf{Multiattacco.}} Il fanatico effettua due attacchi da mischia.

\emph{\textbf{Pugnale.} Attacco con Arma da Mischia o a Gittata}: +4 a colpire, portata 1 m o gittata 6m, una creatura.

\emph{Colpisce:} 4 (1d4 + 2) danni perforanti.

\mostro{Gran Sacerdote}
\begin{description}[noitemsep, topsep=0pt, parsep=0pt, partopsep=0pt, leftmargin=0cm, labelwidth=2.2cm]
    \item[\textbf{Taglia/Tipo:}] Media umanoide, qualsiasi Tratto
    \item[\textbf{Caratt.:}] \resizebox{0.5\linewidth+1.8cm}{!}{For 0 Des 0 Cos 1 Int 1 Sag 3 Car 1}
    \item[\textbf{Punti Ferita:}] 51,  \textbf{Difesa:} 14,  \textbf{Iniziativa:} +1
    \item[\textbf{Tiri Salvez.:}] \resizebox{0.5\linewidth+1.8cm}{!}{Tempra +3, Riflessi +3, Volontà +5}
    \item[\textbf{Movimento:}] 7 m
    \item[\textbf{Linguaggi:}] due lingue qualsiasi
    \item[\textbf{Sfida:}] 2 (450 PX)\smallskip
\end{description}

Sono individui al comando di un tempio o altro luogo sacro e che hanno a loro disposizione diversi accoliti.

\emph{\textbf{Eminenza Divina.}} Come Azione Immediata, il sacerdote può spendere uno slot incantesimo per far sì che il suo attacco con arma da mischia infligge 10 (3d6) danni da Luce aggiuntivi. Il beneficio dura fino al termine del round.

\emph{\textbf{Incantesimi.}} Il sacerdote ha CM 6. La sua abilità da incantatore è la Saggezza (+5 al colpire con attacchi con incantesimo). Il sacerdote ha preparato i seguenti incantesimi:

Trucchetti (a volontà): \emph{\hyperlink{Fiamma Sacra}{Fiamma Sacra}, \hyperlink{Luce}{Luce}, \hyperlink{Taumaturgia}{Taumaturgia}}

livello 1 (4 slot): \emph{\hyperlink{Cura Ferite}{Cura Ferite}, \hyperlink{Dardo Tracciante}{Dardo Tracciante}, \hyperlink{Santuario}{Santuario}}

livello 2 (3 slot): \emph{\hyperlink{Arma Spirituale}{Arma Spirituale}, \hyperlink{Ristorare Inferiore}{Ristorare Inferiore}}

livello 3 (2 slot): \emph{\hyperlink{Dissolvi Magie}{Dissolvi Magie}}

\textbf{Azioni}

\emph{\textbf{Mazza.} Attacco con Arma da Mischia}: +4 a colpire, portata 1 m, un bersaglio.

\emph{Colpisce:} 3 (1d6) danni contundenti.

\medskip\textbf{Selvaggi}

Questi individui vivono ai margini della civiltà, a volte entrandovi raramente in contatto. A disagio tra le mura e nelle terre civilizzate, si trovano nel loro ambiente quando possono muoversi tra le terre selvagge.

\mostro{Berserker}
\begin{description}[noitemsep, topsep=0pt, parsep=0pt, partopsep=0pt, leftmargin=0cm, labelwidth=2.2cm]
    \item[\textbf{Taglia/Tipo:}] Media umanoide, qualsiasi Tratto
    \item[\textbf{Caratt.:}] \resizebox{0.5\linewidth+1.8cm}{!}{For 3 Des 1 Cos 3 Int -1 Sag 0 Car -1}
    \item[\textbf{Punti Ferita:}] 52,  \textbf{Difesa:} 15,  \textbf{Iniziativa:} +1
    \item[\textbf{Tiri Salvez.:}] \resizebox{0.5\linewidth+1.8cm}{!}{Tempra +5, Riflessi +3, Volontà +3}
    \item[\textbf{Movimento:}] 9 m
    \item[\textbf{Linguaggi:}] Comune
    \item[\textbf{Sfida:}] 2 (450 PX)\smallskip
\end{description}

Provenienti da terre selvagge, gli imprevedibili berserker si radunano in compagnie di guerra e sono sempre alla ricerca di conflitti in cui combattere.

\emph{\textbf{Incauto.}} All'inizio del suo round, il berserker può ottenere +1d6 su tutti i tiri di attacco con armi da mischia effettuati durante quel round, ma i tiri di attacco contro di esso hanno +1d6 fino all'inizio del suo prossimo round.

\textbf{Azioni}

\emph{\textbf{Ascia Grossa.} Attacco con Arma da Mischia}: +5 a colpire, portata 1 m, un bersaglio.

\emph{Colpisce:} 9 (1d12 + 3) danni taglienti.

\mostro{Combattente Tribale}
\begin{description}[noitemsep, topsep=0pt, parsep=0pt, partopsep=0pt, leftmargin=0cm, labelwidth=2.2cm]
    \item[\textbf{Taglia/Tipo:}] Media umanoide, qualsiasi Tratto
    \item[\textbf{Caratt.:}] \resizebox{0.5\linewidth+1.8cm}{!}{For 1 Des 0 Cos 1 Int -1 Sag 0 Car -1}
    \item[\textbf{Punti Ferita:}] 17,  \textbf{Difesa:} 12,  \textbf{Iniziativa:} +0
    \item[\textbf{Tiri Salvez.:}] \resizebox{0.5\linewidth+1.8cm}{!}{Tempra +3, Riflessi +3, Volontà +3}
    \item[\textbf{Movimento:}] 9 m
    \item[\textbf{Linguaggi:}] Comune
    \item[\textbf{Sfida:}] 1/8 (25 PX)\smallskip
\end{description}

Sono i difensori delle tribù che vivono ai margini della civiltà.

\emph{\textbf{Tattiche di Branco.}} Il combattente tribale ha +1d6 ai tiri di attacco contro una creatura se almeno uno degli alleati del picchiatore si trova entro 1 metro dalla creatura e quell'alleato non è inabile.

\textbf{Azioni}

\emph{\textbf{Lancia.} Attacco con Arma da Mischia o a Gittata}: +3 a colpire, portata 1 m o gittata 6m, un bersaglio.

\emph{Colpisce:} 4 (1d6 + 1) danni perforanti

\mostro{Druido}
\begin{description}[noitemsep, topsep=0pt, parsep=0pt, partopsep=0pt, leftmargin=0cm, labelwidth=2.2cm]
    \item[\textbf{Taglia/Tipo:}] Media umanoide, qualsiasi Tratto
    \item[\textbf{Caratt.:}] \resizebox{0.5\linewidth+1.8cm}{!}{For 0 Des 1 Cos 1 Int 1 Sag 2 Car 0}
    \item[\textbf{Punti Ferita:}] 51,  \textbf{Difesa:} 15,  \textbf{Iniziativa:} +1
    \item[\textbf{Comp.:}] Pronto Soccorso +4, Natura +3, Consapevolezza +4
    \item[\textbf{Tiri Salvez.:}] \resizebox{0.5\linewidth+1.8cm}{!}{Tempra +3, Riflessi +3, Volontà +4}
    \item[\textbf{Movimento:}] 9 m
    \item[\textbf{Linguaggi:}] Druidico più due altre lingue
    \item[\textbf{Sfida:}] 2 (450 PX)\smallskip
\end{description}

I druidi proteggono il mondo naturale dai mostri e dall'avanzare della civiltà. Alcuni sono sciamani tribali che curano i malati, pregano agli spiriti animali e forniscono consigli spirituali. Solitamente sono devoti di Efrem o Shayalia.

\emph{\textbf{Incantesimi.}} Il Druido ha CM 4. La sua abilità da incantatore è la Saggezza (+4 al colpire con attacchi con incantesimo). Il Druido ha preparato i seguenti incantesimi:

Trucchetti (a volontà): \emph{\hyperlink{Artificio Druidico}{Artificio Druidico}, \hyperlink{Randello Incantato}{Randello Incantato}, \hyperlink{Produrre Fiamma}{Produrre Fiamma}}

livello 1 (4 slot): \emph{\hyperlink{Intralciare}{Intralciare}, \hyperlink{Onda Tonante}{Onda Tonante}, \hyperlink{Parlare con gli Animali}{Parlare con gli Animali}, \hyperlink{Passo Veloce}{Passo Veloce}}

livello 2 (3 slot): \emph{\hyperlink{Animale Messaggero}{Animale Messaggero}, \hyperlink{Pelle di Corteccia}{Pelle di Corteccia}}

\textbf{Azioni}

\emph{\textbf{Bastone da Combattimento.} Attacco con Arma da Mischia}: +3 a colpire (+5 a colpire con \emph{\hyperlink{Randello Incantato}{Randello Incantato}}), portata 1 m o gittata 6m, un bersaglio.

\emph{Colpisce:} 3 (1d6) danni contundenti, o 6 (1d8 + 2) danni contundenti con \emph{\hyperlink{Randello Incantato}{Randello Incantato}} o se impugnato con due mani.

\mostro{Esploratore}
\begin{description}[noitemsep, topsep=0pt, parsep=0pt, partopsep=0pt, leftmargin=0cm, labelwidth=2.2cm]
    \item[\textbf{Taglia/Tipo:}] Media umanoide, qualsiasi Tratto
    \item[\textbf{Caratt.:}] \resizebox{0.5\linewidth+1.8cm}{!}{For 0 Des 2 Cos 1 Int 0 Sag 1 Car 0}
    \item[\textbf{Punti Ferita:}] 24,  \textbf{Difesa:} 14,  \textbf{Iniziativa:} +2
    \item[\textbf{Comp.:}] Furtività +6, Natura +4, Consapevolezza +5, Sopravvivenza +5
    \item[\textbf{Tiri Salvez.:}] \resizebox{0.5\linewidth+1.8cm}{!}{Tempra +3, Riflessi +3, Volontà +3}
    \item[\textbf{Movimento:}] 9 m
    \item[\textbf{Linguaggi:}] Comune
    \item[\textbf{Sfida:}] 1/2 (100 PX)\smallskip
\end{description}

Abili cacciatori e battitori di piste.

\emph{\textbf{Olfatto e Vista Affinati.}} L'esploratore ha +1d6 nelle prove di Consapevolezza basate su olfatto o vista.

\textbf{Azioni}

\emph{\textbf{Multiattacco.}} L'esploratore effettua due attacchi da mischia o due attacchi a gittata.

\emph{\textbf{Spada Corta.} Attacco con Arma da Mischia}: +4 a colpire, portata 1 m, un bersaglio.

\emph{Colpisce:} 5 (1d6 + 2) danni perforanti.

\emph{\textbf{Arco Lungo.} Attacco con Arma da Mischia}: +4 a colpire, gittata 45m, un bersaglio.

\emph{Colpisce:} 6 (1d8 + 2) danni perforanti.

}  %chiude \setlength{\parietlength{\parindent}{0cm}{

%} % chiude \small{

\rule{\linewidth}{2pt}

\subsection{Conversioni dalla 5e}\index{Convertire i mostri dalla 5e}

Per convertire i mostri del famoso Gioco di Ruolo:

\begin{itemize}[leftmargin=*] \setlength{\itemsep}{0pt}
\item \textbf{Difesa} uguale a 12+GS+(GS/3)+Destrezza $\pm4$\
\item \textbf{Tiri Salvezza} da GS+Caratteristica $\pm2$\
\item \textbf{Punti Ferita} da (GS+1+(GS/5))*15 + COS*(GS/5) $\pm GS*2$\
\item \textbf{Tiro per Colpire} da 3-18 (oppure GS*0.7+Forza / Destrezza $\pm3$\ )
\item \textbf{DC per Tiri Salvezza} da abilità da 10+GS+(GS/5) + Caratteristica correlata  $\pm2$\
\item \textbf{Iniziativa} uguale a Destrezza o Intelligenza  $\pm2$\
\item \textbf{Danno x Round} circa 5*GS
\end{itemize}

Arrotondate il GS per eccesso nei calcoli.

\end{multicols}

\vfill

\begin{enfasi}

Timeo Danaos et dona ferentes (Temo i Greci anche quando portano doni). (Eneide, Virgilio)

\medskip

Non è morto ciò che può giacere in eterno, e in strani eoni anche la morte può morire. (Il Richiamo di Cthulhu, H.P. Lovecraft)

\end{enfasi}

\pagebreak

\subsection{Template e suggerimenti per i Mostri}

\begin{multicols}{2}

Sono qui proposti dei suggerimenti su come differenziare i mostri e renderli più specifici alla situazione. Un mostro con uno di questi template prenderà alcuni tratti d'aspetto, di capacità e di gioco tipici del template stesso. \\

\textbf{Spettro}

Il template Spettro conferisce alla creatura un aspetto \emph{da spettro}

\textbf{Punti Ferita}: aumenta di 4 per GS

\textbf{Difesa}: aumenta di 4

\textbf{Sensi}: Scurovisione 18 m

\textbf{Resistenza ai danni}: Acido, Freddo, Fuoco, Elettricità, Suono

\textbf{Immunità}: affascinato, spaventato, affaticato, afferrato, paralizzato, pietrificato, veleno, prono, ristretto

\textbf{Movimento Incorporeo}. La creatura può muoversi attraverso creature ed oggetti come se fosse terreno difficile. Subisce 5 (1d10) danni se termina il suo turno
all’interno di un oggetto.

\textbf{Sensibilità alla luce solare}. Mentre è illuminato a luce solare la creatura ha -1d6 ai Tiri per Colpire e alle prove di Consapevolezza.

\textbf{Attacco}: +1 al Tiro per Colpire

\textbf{Danno}: il danno inferto dalla creatura diventa da Vuoto\\

\textbf{Maledetto}

Il template Maledetto conferisce alla creatura un aspetto \emph{corrotto ed oscuro}

\textbf{Punti Ferita}: aumenta di 6 per GS

\textbf{Difesa}: aumenta di 2

\textbf{Resistenza ai danni}: Vuoto, Elettricità

\textbf{Immunità}: affascinato, spaventato

\textbf{Sensi}: Visione Crepuscolare 18 m

\textbf{Attacco}: +2 al Tiro per Colpire

\textbf{Danno}: +1d4 danno da Vuoto\\

\textbf{Bestia Selvaggia}

Il template Bestia Selvaggia conferisce alla creatura un aspetto \emph{più grosso ed aggressivo}

\textbf{Punti Ferita}: aumenta di 6 per GS

\textbf{Difesa}: aumenta di 3

\textbf{Movimento}: +3 m

\textbf{Attacco}: +3 al Tiro per Colpire

\textbf{Danno}: +1d6 danni\\

\textbf{Demoniaco}

Il template Demoniaco conferisce alla creatura un aspetto \emph{da demone}

\textbf{Punti Ferita}: aumenta di 8 per GS

\textbf{Difesa}: aumenta di 6

\textbf{Movimento}: +3 m

\textbf{Sensi}: Scurovisione 18 m

\textbf{Resistenza ai danni}: Freddo, da armi non magiche

\textbf{Resistenza alla Magia}: +2 ai Tiri Salvezza contro Incantesimi

\textbf{Scurovisione}: 18 m

\textbf{Attacco}: +4 al Tiro per Colpire

\textbf{Danno}: +1d6 Sanguinamento\\

\textbf{Scheletro}

Il template Scheletro conferisce alla creatura un aspetto \emph{scheletrico}

\textbf{Punti Ferita}: aumenta di 4 per GS

\textbf{Difesa}: aumenta di 1

\textbf{Sensi}: Scurovisione 18 m

\textbf{Resistenza ai danni}:  perforante, tagliente, Elettricità, Fuoco

\textbf{Immunità ai danni}: Veleno

\textbf{Immunità}: affaticato, sanguinamento

\textbf{Vulnerabilità}: danni contundenti

\textbf{Attacco}: +1 al Tiro per Colpire

\textbf{Danno}: +2\\

\textbf{Infuso di Magia}

Il template Infuso di Magia conferisce alla creatura un aspetto \emph{brillante}

\textbf{Punti Ferita}: aumenta di 4 per GS

\textbf{Difesa}: aumenta di 3

\textbf{Sensi}: Scurovisione 18 m

\textbf{Resistenza ai danni}:  Elettricità, Fuoco, Luce, Vuoto

\textbf{Resistenza alla Magia}: +1d6 ai Tiri Salvezza contro Incantesimi

\textbf{Attacco}: +4 al Tiro per Colpire contro creature con incantesimi attivi o con CM >1

\textbf{Danno}: al posto di fare danno la creatura può provare un \hyperlink{contrastareincantesimi}{contrastare} incantesimi con DC pari al GS+Intelligenza per annullare un incantesimo attivo sull'avversario\\

\textbf{Oozekin}

Il template Oozekin conferisce alla creatura un aspetto \emph{gelatinoso} e fluido

\textbf{Punti Ferita}: aumenta di 6 per GS

\textbf{Difesa}: aumenta di 4

\textbf{Resistenza ai danni}:  perforante, tagliente

\textbf{Movimento}: -1 m

\textbf{Sensi}: Vista Cieca 18 m (cieco oltre questo raggio)

\textbf{Immunità}: accecato, affascinato, assordato, prono

\textbf{Imm. Danni}: Acido, Elettricità, tagliente, da critico

\textbf{Resistenza alla Magia}: +1 ai Tiri Salvezza contro Incantesimi

\textbf{Attacco}: +3

\textbf{Danno}: +1d8 da Acido

\end{multicols}

%\bigskip

%\begin{center}
%\begin{tikzpicture}
%\draw[thick] (0cm, 3cm) rectangle ++(16cm,20cm); % Sposta la cornice di 0 cm a destra e 3 cm in alto
%\node[anchor=north west, inner sep=5pt] at (0.3cm, 23cm) {\textbf{Appunti:}}; % Aggiunge il testo in alto a sinistra della cornice
%\end{tikzpicture}
%\end{center}

%{\scriptsize
%\printindex}
%\end{document}

\pagebreak

\subsection{Lista Mostri per Grado di Sfida}

\begin{multicols}{3}
{\small
\noindent\hyperlink{Aquila}{Aquila}, GS 0 (10 PX)\\
\hyperlink{Avvoltoio}{Avvoltoio}, GS 0 (10 PX)\\
\hyperlink{Babbuino}{Babbuino}, GS 0 (10 PX)\\
\hyperlink{Caprone}{Caprone}, GS 0 (10 PX)\\
\hyperlink{Cervo}{Cervo}, GS 0 (10 PX)\\
\hyperlink{Corvo}{Corvo}, GS 0 (10 PX)\\
%\hyperlink{Donnola}{Donnola}, GS 0 (10 PX)\\
\hyperlink{Falco}{Falco}, GS 0 (10 PX)\\
\hyperlink{Fungo Stridente}{Fungo Stridente}, GS 0 (10 PX)\\
\hyperlink{Gatto}{Gatto}, GS 0 (10 PX)\\
\hyperlink{Gufo}{Gufo}, GS 0 (10 PX)\\
\hyperlink{Iena}{Iena}, GS 0 (10 PX)\\
\hyperlink{Lemure}{Lemure}, GS 0 (10 PX)\\
%\hyperlink{Lucertola}{Lucertola}, GS 0 (10 PX)\\
\hyperlink{Omuncolo}{Omuncolo}, GS 0 (10 PX)\\
\hyperlink{Pirana}{Pirana}, GS 0 (10 PX)\\
\hyperlink{Popolano}{Popolano}, GS 0 (10 PX)\\
\hyperlink{Ragno}{Ragno}, GS 0 (10 PX)\\
\hyperlink{Rana}{Rana}, GS 0 (10 PX)\\
\hyperlink{Ratto}{Ratto}, GS 0 (10 PX)\\
\hyperlink{Scarabeo di Fuoco Gigante}{Scarabeo di Fuoco Gigante}, GS 0 (10 PX)\\
\hyperlink{Sciacallo}{Sciacallo}, GS 0 (10 PX)\\
\hyperlink{Scorpione}{Scorpione}, GS 0 (10 PX)\\
\hyperlink{Tasso}{Tasso}, GS 0 (10 PX)\\
\hyperlink{Topi, La}{Topi, La}, GS 0 (10 PX)\\
\hyperlink{Bandito/Pirata}{Bandito/Pirata}, GS 1/8 (25 PX)\\
\hyperlink{Coboldo}{Coboldo}, GS 1/8 (25 PX)\\
\hyperlink{Cultista}{Cultista}, GS 1/8 (25 PX)\\
\hyperlink{Donnola Gigante}{Donnola Gigante}, GS 1/8 (25 PX)\\
\hyperlink{Falco di Sangue}{Falco di Sangue}, GS 1/8 (25 PX)\\
\hyperlink{Granchio Gigante}{Granchio Gigante}, GS 1/8 (25 PX)\\
\hyperlink{Guardia}{Guardia}, GS 1/8 (25 PX)\\
\hyperlink{Mastino}{Mastino}, GS 1/8 (25 PX)\\
%\hyperlink{Mulo}{Mulo}, GS 1/8 (25 PX)\\andres
\hyperlink{Nobile}{Nobile}, GS 1/8 (25 PX)\\
\hyperlink{Saurovallo nano}{Saurovallo nano}, GS 1/8 (25 PX)\\
\hyperlink{Ratto Gigante}{Ratto Gigante}, GS 1/8 (25 PX)\\
\hyperlink{Serpente Velenoso}{Serpente Velenoso}, GS 1/8 (25 PX)\\
\hyperlink{Serpente Volante}{Serpente Volante}, GS 1/8 (25 PX)\\
\hyperlink{Strige}{Strige}, GS 1/8 (25 PX)\\
\hyperlink{Strige (Uccello Stigeo)}{Strige (Uccello Stigeo)}, GS 1/8 (25 PX)\\
\hyperlink{Uomo Acquatico}{Uomo Acquatico}, GS 1/8 (25 PX)\\
\hyperlink{Accolito}{Accolito}, GS 1/4 (50 PX)\\
\hyperlink{Alce}{Alce}, GS 1/4 (50 PX)\\
\hyperlink{Becco d'Ascia}{Becco d'Ascia}, GS 1/4 (50 PX)\\
\hyperlink{Cane Intermittente}{Cane Intermittente}, GS 1/4 (50 PX)\\
\hyperlink{Saurovallo da Galoppo}{Saurovallo da Galoppo}, GS 1/4 (50 PX)\\
\hyperlink{Saurovallo da Tiro}{Saurovallo da Tiro}, GS 1/4 (50 PX)\\
%\hyperlink{Centopiedi Gigante}{Centopiedi Gigante}, GS 1/4 (50 PX)\\
\hyperlink{Cinghiale}{Cinghiale}, GS 1/4 (50 PX)\\
\hyperlink{Dretch}{Dretch}, GS 1/4 (50 PX)\\
\hyperlink{Fungo Violetto}{Fungo Violetto}, GS 1/4 (50 PX)\\
\hyperlink{Gablin}{Gablin}, GS 1/4 (50 PX)\\
\hyperlink{Grimlock}{Grimlock}, GS 1/4 (50 PX)\\
\hyperlink{Gufo Gigante}{Gufo Gigante}, GS 1/4 (50 PX)\\
\hyperlink{Lucertola Gigante}{Lucertola Gigante}, GS 1/4 (50 PX)\\
\hyperlink{Lupo}{Lupo}, GS 1/4 (50 PX)\\
\hyperlink{Mefito di Vapore}{Mefito di Vapore}, GS 1/4 (50 PX)\\
\hyperlink{Pantera}{Pantera}, GS 1/4 (50 PX)\\
\hyperlink{Pseudodrago}{Pseudodrago}, GS 1/4 (50 PX)\\
\hyperlink{Ragno Lupo Gigante}{Ragno Lupo Gigante}, GS 1/4 (50 PX)\\
\hyperlink{Rana Gigante}{Rana Gigante}, GS 1/4 (50 PX)\\
\hyperlink{Scheletro}{Scheletro}, GS 1/4 (50 PX)\\
\hyperlink{Sciame di Corvi}{Sciame di Corvi}, GS 1/4 (50 PX)\\
\hyperlink{Sciame di Pipistrelli}{Sciame di Pipistrelli}, GS 1/4 (50 PX)\\
\hyperlink{Sciame di Ratti}{Sciame di Ratti}, GS 1/4 (50 PX)\\
\hyperlink{Serpente Costrittore}{Serpente Costrittore}, GS 1/4 (50 PX)\\
\hyperlink{Serpente Velenoso Gigante}{Serpente Velenoso Gigante}, GS 1/4 (50 PX)\\
\hyperlink{Spada Volante}{Spada Volante}, GS 1/4 (50 PX)\\
\hyperlink{Spiritello}{Spiritello}, GS 1/4 (50 PX)\\
\hyperlink{Tasso Gigante}{Tasso Gigante}, GS 1/4 (50 PX)\\
\hyperlink{Zombi}{Zombi}, GS 1/4 (50 PX)\\
\hyperlink{Caprone Gigante}{Caprone Gigante}, GS 1/2 (100 PX)\\
\hyperlink{Saurovallo da Guerra}{Saurovallo da Guerra}, GS 1/2 (100 PX)\\
\hyperlink{Cavallo Marino Gigante}{Cavallo Marino Gigante}, GS 1/2 (100 PX)\\
\hyperlink{Coccodrillo}{Coccodrillo}, GS 1/2 (100 PX)\\
\hyperlink{Cockatrice}{Cockatrice}, GS 1/2 (100 PX)\\
\hyperlink{Esploratore}{Esploratore}, GS 1/2 (100 PX)\\
\hyperlink{Gnoll}{Gnoll}, GS 1/2 (100 PX)\\
\hyperlink{Gnomo delle Profondità}{Gnomo delle Profondità}, GS 1/2 (100 PX)\\
\hyperlink{Hobgoblin}{Hobgoblin}, GS 1/2 (100 PX)\\
\hyperlink{Lucertoloide}{Lucertoloide}, GS 1/2 (100 PX)\\
\hyperlink{Mantoscuro}{Mantoscuro}, GS 1/2 (100 PX)\\
\hyperlink{Mefito di Ghiaccio}{Mefito di Ghiaccio}, GS 1/2 (100 PX)\\
\hyperlink{Mefito di Magma}{Mefito di Magma}, GS 1/2 (100 PX)\\
\hyperlink{Mefito di Polvere}{Mefito di Polvere}, GS 1/2 (100 PX)\\
\hyperlink{Melma Grigia}{Melma Grigia}, GS 1/2 (100 PX)\\
\hyperlink{Monete affamate}{Monete affamate}, GS 1/2 (100 PX)\\
\hyperlink{Ombra}{Ombra}, GS 1/2 (100 PX)\\
\hyperlink{Orchetto}{Orchetto}, GS 1/2 (100 PX)\\
\hyperlink{Orso Nero}{Orso Nero}, GS 1/2 (100 PX)\\
\hyperlink{Rugginofago}{Rugginofago}, GS 1/2 (100 PX)\\
\hyperlink{Sahuagin}{Sahuagin}, GS 1/2 (100 PX)\\
\hyperlink{Satiro}{Satiro}, GS 1/2 (100 PX)\\
\hyperlink{Scheletro di Saurovallo da Guerra}{Scheletro di Saurovallo da Guerra}, GS 1/2 (100 PX)\\
\hyperlink{Sciame di Insetti}{Sciame di Insetti}, GS 1/2 (100 PX)\\
\hyperlink{Sciame di Ragni}{Sciame di Ragni}, GS 1/2 (100 PX)\\
%\hyperlink{Sciame di Scarabei}{Sciame di Scarabei}, GS 1/2 (100 PX)\\
\hyperlink{Sciame di Vespe}{Sciame di Vespe}, GS 1/2 (100 PX)\\
%\hyperlink{Sciami}{Sciami}, GS 1/2 (100 PX)\\
\hyperlink{Scimmione}{Scimmione}, GS 1/2 (100 PX)\\
\hyperlink{Squalo Corallino}{Squalo Corallino}, GS 1/2 (100 PX)\\
\hyperlink{Uomo Magma}{Uomo Magma}, GS 1/2 (100 PX)\\
\hyperlink{Vespa Gigante}{Vespa Gigante}, GS 1/2 (100 PX)\\
\hyperlink{Worg}{Worg}, GS 1/2 (100 PX)\\
\hyperlink{Aquila Gigante}{Aquila Gigante}, GS 1 (200 PX)\\
\hyperlink{Armatura Animata}{Armatura Animata}, GS 1 (200 PX)\\
\hyperlink{Arpia}{Arpia}, GS 1 (200 PX)\\
\hyperlink{Avvoltoio Gigante}{Avvoltoio Gigante}, GS 1 (200 PX)\\
\hyperlink{Bugbear}{Bugbear}, GS 1 (200 PX)\\
\hyperlink{Cane della Morte}{Cane della Morte}, GS 1 (200 PX)\\
\hyperlink{Dinolupo (Metalupo)}{Dinolupo (Metalupo)}, GS 1 (200 PX)\\
\hyperlink{Drago di Ottone Cucciolo}{Drago di Ottone Cucciolo}, GS 1 (200 PX)\\
\hyperlink{Drago di Rame Cucciolo}{Drago di Rame Cucciolo}, GS 1 (200 PX)\\
\hyperlink{Driade}{Driade}, GS 1 (200 PX)\\
\hyperlink{Ghoul}{Ghoul}, GS 1 (200 PX)\\
\hyperlink{Globulo}{Globulo}, GS 1 (200 PX)\\
\hyperlink{Iena Gigante}{Iena Gigante}, GS 1 (200 PX)\\
\hyperlink{Imp}{Imp}, GS 1 (200 PX)\\
\hyperlink{Ippogrifo}{Ippogrifo}, GS 1 (200 PX)\\
\hyperlink{Leone}{Leone}, GS 1 (200 PX)\\
\hyperlink{Mago Avventuriero}{Mago Avventuriero}, GS 1 (200 PX)\\
\hyperlink{Nano Oscuro}{Nano Oscuro}, GS 1 (200 PX)\\
\hyperlink{Orco}{Orco}, GS 1 (100 PX)\\
\hyperlink{Orso Bruno}{Orso Bruno}, GS 1 (200 PX)\\
\hyperlink{Quasit}{Quasit}, GS 1 (200 PX)\\
\hyperlink{Ragno Gigante}{Ragno Gigante}, GS 1 (200 PX)\\
\hyperlink{Rospo Gigante}{Rospo Gigante}, GS 1 (200 PX)\\
\hyperlink{Sciame di Pirana}{Sciame di Pirana}, GS 1 (200 PX)\\
\hyperlink{Spettro}{Spettro}, GS 1 (200 PX)\\
\hyperlink{Spia}{Spia}, GS 1 (200 PX)\\
\hyperlink{Tigre}{Tigre}, GS 1 (200 PX)\\
\hyperlink{Albero Risvegliato}{Albero Risvegliato}, GS 2 (450 PX)\\
\hyperlink{Alce Gigante}{Alce Gigante}, GS 2 (450 PX)\\
\hyperlink{Ameba Paglierina}{Ameba Paglierina}, GS 2 (450 PX)\\
\hyperlink{Ankheg}{Ankheg}, GS 2 (450 PX)\\
\hyperlink{Azer}{Azer}, GS 2 (450 PX)\\
\hyperlink{Berserker}{Berserker}, GS 2 (450 PX)\\
\hyperlink{Blatta Esplosiva}{Blatta Esplosiva}, GS 2 (450 PX)\\
\hyperlink{Capitano dei Banditi o Pirata}{Capitano dei Banditi o Pirata}, GS 2 (450 PX)\\
\hyperlink{Centauro}{Centauro}, GS 2 (450 PX)\\
\hyperlink{Cinghiale Gigante}{Cinghiale Gigante}, GS 2 (450 PX)\\
\hyperlink{Cubo Gelatinoso}{Cubo Gelatinoso}, GS 2 (450 PX)\\
\hyperlink{Diavolo Spinoso}{Diavolo Spinoso}, GS 2 (450 PX)\\
\hyperlink{Drago Bianco Cucciolo}{Drago Bianco Cucciolo}, GS 2 (450 PX)\\
\hyperlink{Drago di Argento Cucciolo}{Drago di Argento Cucciolo}, GS 2 (450 PX)\\
\hyperlink{Drago di Bronzo Cucciolo}{Drago di Bronzo Cucciolo}, GS 2 (450 PX)\\
\hyperlink{Drago Nero Cucciolo}{Drago Nero Cucciolo}, GS 2 (450 PX)\\
\hyperlink{Drago Verde Cucciolo}{Drago Verde Cucciolo}, GS 2 (450 PX)\\
\hyperlink{Druido}{Druido}, GS 2 (450 PX)\\
\hyperlink{Ettercap}{Ettercap}, GS 2 (450 PX)\\
\hyperlink{Fauci Gorgoglianti}{Fauci Gorgoglianti}, GS 2 (450 PX)\\
\hyperlink{Fuoco Fatuo}{Fuoco Fatuo}, GS 2 (450 PX)\\
\hyperlink{Gargoyle}{Gargoyle}, GS 2 (450 PX)\\
\hyperlink{Ghast}{Ghast}, GS 2 (450 PX)\\
\hyperlink{Grick}{Grick}, GS 2 (450 PX)\\
\hyperlink{Grifone}{Grifone}, GS 2 (450 PX)\\
\hyperlink{Megera Marina}{Megera Marina}, GS 2 (450 PX)\\
\hyperlink{Mimic}{Mimic}, GS 2 (450 PX)\\
\hyperlink{Ogre}{Ogre}, GS 2 (450 PX)\\
\hyperlink{Orso Polare}{Orso Polare}, GS 2 (450 PX)\\
\hyperlink{Pegaso}{Pegaso}, GS 2 (450 PX)\\
\hyperlink{Plesiosauro}{Plesiosauro}, GS 2 (450 PX)\\
\hyperlink{Ratto Mannaro}{Ratto Mannaro}, GS 2 (450 PX)\\
\hyperlink{Rinoceronte lanoso}{Rinoceronte lanoso}, GS 2 (450 PX)\\
\hyperlink{Gran Sacerdote}{Gran Sacerdote}, GS 2 (450 PX)\\
\hyperlink{Sciame di Serpenti Velenosi}{Sciame di Serpenti Velenosi}, GS 2 (450 PX)\\
\hyperlink{Serpente Costrittore Gigante}{Serpente Costrittore Gigante}, GS 2 (450 PX)\\
\hyperlink{Sibilante}{Sibilante}, GS 2 (450 PX)\\
\hyperlink{Silku}{Silku}, GS 2 (450 PX)\\
\hyperlink{Squalo Cacciatore}{Squalo Cacciatore}, GS 2 (450 PX)\\
\hyperlink{Tappeto del Soffocamento}{Tappeto del Soffocamento}, GS 2 (450 PX)\\
\hyperlink{Teschio Fiammeggiante}{Teschio Fiammeggiante}, GS 2 (200 PX)\\
\hyperlink{Tigre dai Denti a Sciabola}{Tigre dai Denti a Sciabola}, GS 2 (450 PX)\\
\hyperlink{Zombi Ogre}{Zombi Ogre}, GS 2 (450 PX)\\
\hyperlink{Balena Assassina (Orca)}{Balena Assassina (Orca)}, GS 3 (700 PX)\\
\hyperlink{Basilisco}{Basilisco}, GS 3 (700 PX)\\
\hyperlink{Campione Gablin}{Campione Gablin}, GS 3 (700)\\
\hyperlink{Cavaliere}{Cavaliere}, GS 3 (700 PX)\\
\hyperlink{Destriero dell'Incubo}{Destriero dell'Incubo}, GS 3 (700 PX)\\
\hyperlink{Diavolo Barbuto}{Diavolo Barbuto}, GS 3 (700 PX)\\
\hyperlink{Doppelganger}{Doppelganger}, GS 3 (700 PX)\\
\hyperlink{Drago Blu Cucciolo}{Drago Blu Cucciolo}, GS 3 (700 PX)\\
\hyperlink{Drago d'Oro Cucciolo}{Drago d'Oro Cucciolo}, GS 3 (700 PX)\\
\hyperlink{Lupo Invernale}{Lupo Invernale}, GS 3 (700 PX)\\
\hyperlink{Lupo Mannaro}{Lupo Mannaro}, GS 3 (700 PX)\\
\hyperlink{Manticora}{Manticora}, GS 3 (700 PX)\\
\hyperlink{Megera Verde}{Megera Verde}, GS 3 (700 PX)\\
\hyperlink{Minotauro}{Minotauro}, GS 3 (700 PX)\\
\hyperlink{Mummia}{Mummia}, GS 3 (700 PX)\\
\hyperlink{Orrore Arrampicamuri}{Orrore Arrampicamuri}, GS 3 (700 PX)\\
\hyperlink{Orsogufo}{Orsogufo}, GS 3 (700 PX)\\
\hyperlink{Orsogufo Saggio}{Orsogufo Saggio}, GS 3 (700 PX)\\
\hyperlink{Ragno Fase}{Ragno Fase}, GS 3 (700 PX)\\
\hyperlink{Scheletro Campione}{Scheletro Campione}, GS 3 (700 PX)\\
\hyperlink{Scorpione Gigante}{Scorpione Gigante}, GS 3 (700 PX)\\
\hyperlink{Segugio Infernale}{Segugio Infernale}, GS 3 (700 PX)\\
\hyperlink{Veterano}{Veterano}, GS 3 (700 PX)\\
\hyperlink{Wight}{Wight}, GS 3 (700 PX)\\
\hyperlink{B.O.C.}{B.O.C.}, GS 4 (1100 PX)\\
\hyperlink{Banshee}{Banshee}, GS 4 (1100 PX)\\
\hyperlink{Chuul}{Chuul}, GS 4 (1100 PX)\\
\hyperlink{Cinghiale Mannaro}{Cinghiale Mannaro}, GS 4 (1100 PX)\\
\hyperlink{Couatl}{Couatl}, GS 4 (1100 PX)\\
\hyperlink{Drago Rosso Cucciolo}{Drago Rosso Cucciolo}, GS 4 (1100 PX)\\
\hyperlink{Elefante}{Elefante}, GS 4 (1100 PX)\\
\hyperlink{Ettin}{Ettin}, GS 4 (1100 PX)\\
\hyperlink{Fantasma}{Fantasma}, GS 4 (1100 PX)\\
\hyperlink{Ghoul, putrescente}{Ghoul, putrescente}, GS 4 (1100 PX)\\
\hyperlink{Lamia}{Lamia}, GS 4 (1100 PX)\\
\hyperlink{Maledetto immortale}{Maledetto immortale}, GS 4 (1100 PX)\\
\hyperlink{Protoplasma Nero}{Protoplasma Nero}, GS 4 (1100 PX)\\
\hyperlink{Succube}{Succube}, GS 4 (1100 PX)\\
\hyperlink{Tigre Mannara}{Tigre Mannara}, GS 4 (1100 PX)\\
\hyperlink{Torciascura}{Torciascura}, GS 4 (1100 PX)\\
\hyperlink{Verme Strisciante Tentacolato}{Verme Strisciante Tentacolato}, GS 4 (1100 PX)\\
\hyperlink{Bulette}{Bulette}, GS 5 (1800 PX)\\
\hyperlink{Coccodrillo Gigante}{Coccodrillo Gigante}, GS 5 (1800 PX)\\
\hyperlink{Cumulo Strisciante}{Cumulo Strisciante}, GS 5 (1800 PX)\\
\hyperlink{Elementale del Fuoco Generico}{Elementale del Fuoco Generico}, GS 5 (1800 PX)\\
\hyperlink{Elementale dell'Acqua Generico}{Elementale dell'Acqua Generico}, GS 5 (1800 PX)\\
\hyperlink{Elementale dell'Aria Generico}{Elementale dell'Aria Generico}, GS 5 (1800 PX)\\
\hyperlink{Elementale della Terra Generico}{Elementale della Terra Generico}, GS 5 (1800 PX)\\
\hyperlink{Fustigatore}{Fustigatore}, GS 5 (1800 PX)\\
\hyperlink{Ghoul, Madre}{Ghoul, Madre}, GS 5 (1800 PX)\\
\hyperlink{Gigante delle Colline}{Gigante delle Colline}, GS 5 (1800 PX)\\
\hyperlink{Golem di Carne}{Golem di Carne}, GS 5 (1800 PX)\\
\hyperlink{Gorgone}{Gorgone}, GS 5 (1800 PX)\\
\hyperlink{Megera Notturna}{Megera Notturna}, GS 5 (1800 PX)\\
\hyperlink{Orso Mannaro}{Orso Mannaro}, GS 5 (1800 PX)\\
\hyperlink{Otyugh}{Otyugh}, GS 5 (1800 PX)\\
\hyperlink{Salamandra}{Salamandra}, GS 5 (1800 PX)\\
\hyperlink{Squalo Gigante}{Squalo Gigante}, GS 5 (1800 PX)\\
\hyperlink{Triceratopo}{Triceratopo}, GS 5 (1800 PX)\\
\hyperlink{Troll}{Troll}, GS 5 (1800 PX)\\
\hyperlink{Unicorno}{Unicorno}, GS 5 (1800 PX)\\
\hyperlink{Wraith}{Wraith}, GS 5 (1800 PX)\\
\hyperlink{Xorn}{Xorn}, GS 5 (1800 PX)\\
\hyperlink{Chimera}{Chimera}, GS 6 (2300 PX)\\
\hyperlink{Drago Bianco Giovane}{Drago Bianco Giovane}, GS 6 (2300 PX)\\
\hyperlink{Drago d'Ottone Giovane}{Drago d'Ottone Giovane}, GS 6 (2300 PX)\\
\hyperlink{Drider}{Drider}, GS 6 (2300 PX)\\
\hyperlink{Ghoul, Nero}{Ghoul, Nero}, GS 6 (2300 PX)\\
\hyperlink{Grande Mago}{Grande Mago}, GS 6 (2300 PX)\\
\hyperlink{Mammut}{Mammut}, GS 6 (2300 PX)\\
\hyperlink{Medusa}{Medusa}, GS 6 (2300 PX)\\
\hyperlink{Paladino Gablin}{Paladino Gablin}, GS 6 (2300 PX)\\
\hyperlink{Persecutore Invisibile}{Persecutore Invisibile}, GS 6 (2300 PX)\\
\hyperlink{Progenie Vampirica}{Progenie Vampirica}, GS 6 (1800 PX)\\
\hyperlink{Viverna}{Viverna}, GS 6 (2300 PX)\\
\hyperlink{Vrock}{Vrock}, GS 6 (2300 PX)\\
\hyperlink{Drago di Rame Giovane}{Drago di Rame Giovane}, GS 7 (2900 PX)\\
\hyperlink{Drago Nero Giovane}{Drago Nero Giovane}, GS 7 (2900 PX)\\
\hyperlink{Gigante di Pietra}{Gigante di Pietra}, GS 7 (2900 PX)\\
\hyperlink{Guardiano Protettore}{Guardiano Protettore}, GS 7 (2900 PX)\\
\hyperlink{Oni}{Oni}, GS 7 (2900 PX)\\
\hyperlink{Scimmione Gigante}{Scimmione Gigante}, GS 7 (2900 PX)\\
\hyperlink{Assassino}{Assassino}, GS 8 (3900 PX)\\
\hyperlink{Diavolo delle Catene}{Diavolo delle Catene}, GS 8 (3900 PX)\\
\hyperlink{Drago di Bronzo Giovane}{Drago di Bronzo Giovane}, GS 8 (3900 PX)\\
\hyperlink{Drago Verde Giovane}{Drago Verde Giovane}, GS 8 (3900 PX)\\
\hyperlink{Gigante del Gelo}{Gigante del Gelo}, GS 8 (3900 PX)\\
\hyperlink{Hezrou}{Hezrou}, GS 8 (3900 PX)\\
\hyperlink{Idra}{Idra}, GS 8 (3900 PX)\\
\hyperlink{Manto Assassino}{Manto Assassino}, GS 8 (3900 PX)\\
\hyperlink{Naga Spirituale}{Naga Spirituale}, GS 8 (3900 PX)\\
\hyperlink{Tirannosauro}{Tirannosauro}, GS 8 (3900 PX)\\
\hyperlink{Diavolo d'Ossa}{Diavolo d'Ossa}, GS 9 (5000 PX)\\
\hyperlink{Divora Cervelli}{Divora Cervelli}, GS 9 (5000 PX)\\
\hyperlink{Drago Blu Giovane}{Drago Blu Giovane}, GS 9 (5000 PX)\\
\hyperlink{Drago di Argento Giovane}{Drago di Argento Giovane}, GS 9 (5000 PX)\\
%\hyperlink{Elementale dell'Acqua Maggiore}{Elementale dell'Acqua Maggiore}, GS 9 (5000 PX)\\
\hyperlink{Gigante del Fuoco}{Gigante del Fuoco}, GS 9 (5000 PX)\\
\hyperlink{Gigante delle Nuvole}{Gigante delle Nuvole}, GS 9 (5000 PX)\\
\hyperlink{Glabrezu}{Glabrezu}, GS 9 (5000 PX)\\
\hyperlink{Golem di Argilla}{Golem di Argilla}, GS 9 (5000 PX)\\
\hyperlink{Uomo Albero (Arborom)}{Uomo Albero (Arborom)}, GS 9 (5000 PX)\\
\hyperlink{Aboleth}{Aboleth}, GS 10 (5900 PX)\\
\hyperlink{Angelo Deva}{Angelo Deva}, GS 10 (5900 PX)\\
\hyperlink{Drago d'Oro Giovane}{Drago d'Oro Giovane}, GS 10 (5900 PX)\\
\hyperlink{Drago Rosso Giovane}{Drago Rosso Giovane}, GS 10 (5900 PX)\\
\hyperlink{G.E.C.}{G.E.C.}, GS 10 (5900 PX)\\
\hyperlink{Golem di Pietra}{Golem di Pietra}, GS 10 (5900 PX)\\
\hyperlink{Naga Guardiano}{Naga Guardiano}, GS 10 (5900 PX)\\
\hyperlink{Behir}{Behir}, GS 11 (7200 PX)\\
\hyperlink{Diavolo Cornuto}{Diavolo Cornuto}, GS 11 (7200 PX)\\
\hyperlink{Djinni}{Djinni}, GS 11 (7200 PX)\\
\hyperlink{Efreeti}{Efreeti}, GS 11 (7200 PX)\\
\hyperlink{Ginosfinge}{Ginosfinge}, GS 11 (7200 PX)\\
\hyperlink{Remorhaz}{Remorhaz}, GS 11 (7200 PX)\\
\hyperlink{Arcimago}{Arcimago}, GS 12 (8400 PX)\\
\hyperlink{Erinni}{Erinni}, GS 12 (8400 PX)\\
\hyperlink{Panoptikhan}{Panoptikhan}, GS 12 (8400 PX)\\
\hyperlink{Drago Bianco Adulto}{Drago Bianco Adulto}, GS 13 (10000 PX)\\
\hyperlink{Drago d'Ottone Adulto}{Drago d'Ottone Adulto}, GS 13 (10000 PX)\\
\hyperlink{Gigante delle Tempeste}{Gigante delle Tempeste}, GS 13 (10000 PX)\\
\hyperlink{Nalfeshnee}{Nalfeshnee}, GS 13 (10000 PX)\\
\hyperlink{Rakshasa}{Rakshasa}, GS 13 (10000 PX)\\
\hyperlink{Vampiro}{Vampiro}, GS 13 (10000 PX)\\
\hyperlink{Diavolo del Ghiaccio}{Diavolo del Ghiaccio}, GS 14 (11.500 PX)\\
\hyperlink{Drago di Rame Adulto}{Drago di Rame Adulto}, GS 14 (11.500 PX)\\
\hyperlink{Drago di Bronzo Adulto}{Drago di Bronzo Adulto}, GS 15 (13000 PX)\\
\hyperlink{Drago Verde Adulto}{Drago Verde Adulto}, GS 15 (13000 PX)\\
\hyperlink{Fenice}{Fenice}, GS 15 (13000 PX)\\
\hyperlink{Mummia Sovrana}{Mummia Sovrana}, GS 15 (13000 PX)\\
\hyperlink{Verme Purpureo}{Verme Purpureo}, GS 15 (13000 PX)\\
\hyperlink{Angelo Planetar}{Angelo Planetar}, GS 16 (15000 PX)\\
\hyperlink{Drago Blu Adulto}{Drago Blu Adulto}, GS 16 (15000 PX)\\
\hyperlink{Drago di Argento Adulto}{Drago di Argento Adulto}, GS 16 (15000 PX)\\
\hyperlink{Golem di Ferro}{Golem di Ferro}, GS 16 (15000 PX)\\
\hyperlink{Marilith}{Marilith}, GS 16 (15000 PX)\\
\hyperlink{Androsfinge}{Androsfinge}, GS 17 (18000 PX)\\
\hyperlink{Drago d'Oro Adulto}{Drago d'Oro Adulto}, GS 17 (18000 PX)\\
\hyperlink{Drago Nero Adulto}{Drago Nero Adulto}, GS 17 (18000 PX)\\
\hyperlink{Drago Rosso Adulto}{Drago Rosso Adulto}, GS 17 (18000 PX)\\
\hyperlink{Testuggine Dragona}{Testuggine Dragona}, GS 17 (18000 PX)\\
\hyperlink{Cavaliere Nero}{Cavaliere Nero}, GS 18 (20000 PX)\\
\hyperlink{Balor}{Balor}, GS 19 (22000 PX)\\
\hyperlink{Diavolo della Fossa}{Diavolo della Fossa}, GS 20 (25000 PX)\\
\hyperlink{Drago Bianco Antico}{Drago Bianco Antico}, GS 20 (25000 PX)\\
\hyperlink{Drago di Ottone Antico}{Drago di Ottone Antico}, GS 20 (25000 PX)\\
\hyperlink{Angelo Solar}{Angelo Solar}, GS 21 (33000 PX)\\
\hyperlink{Drago di Rame Antico}{Drago di Rame Antico}, GS 21 (33000 PX)\\
\hyperlink{Drago Nero Antico}{Drago Nero Antico}, GS 21 (33000 PX)\\
\hyperlink{Lich}{Lich}, GS 21 (33000 PX)\\
\hyperlink{Drago di Bronzo Antico}{Drago di Bronzo Antico}, GS 22 (41000 PX)\\
\hyperlink{Drago Verde Antico}{Drago Verde Antico}, GS 22 (41000 PX)\\
\hyperlink{Drago Blu Antico}{Drago Blu Antico}, GS 23 (50000 PX)\\
\hyperlink{Drago di Argento Antico}{Drago di Argento Antico}, GS 23 (50000 PX)\\
\hyperlink{Drago Giallo Antico}{Drago Giallo Antico}, GS 23 (50000 PX)\\
\hyperlink{Kraken}{Kraken}, GS 23 (50000 PX)\\
\hyperlink{Drago d'Oro Antico}{Drago d'Oro Antico}, GS 24 (62000 PX)\\
\hyperlink{Drago Rosso Antico}{Drago Rosso Antico}, GS 24 (62000 PX)\\
\hyperlink{Demogorgone}{Demogorgone}, GS 26 (90000 PX)\\
\hyperlink{Orcus}{Orcus}, GS 26 (90000 PX)\\
\hyperlink{Tàhil}{Tàhil}, GS 30 (155000 PX)\\
\hyperlink{Tarrasque}{Tarrasque}, GS 30 (155000 PX)\\

}

\end{multicols}

\pagebreak

\section{Condizioni}\index{Condizioni}

\begin{enfasi}
La grandezza dell'uomo è nella decisione di essere più forte della sua condizione. (Albert Camus)
\end{enfasi}

%{\small
\begin{multicols}{2}

\label{condizioni}

\textbf{Accelerato}\index{Accelerato}\hypertarget{Accelerato}{}\label{Accelerato}: una creatura Accelerata ottiene un numero di Azioni bonus per round pari al valore di Accelerato. La durata è indicata dopo la \emph{"\"}. Quando è indicata la durata si intente fino alla fine del round di scadenza. Es. Accelerato 1/3r, Accelerato 2/-.

\textbf{Aumento Categoria di danno}\index{Aumento Dadi}\index{Taglia dei dadi}: quando la regola vi dice di aumentare la taglia o categoria di un dado seguite questo schema.

1d4 -> 1d6 -> 1d8 -> 1d10 (1d12)-> 2d6 -> 2d8 -> 2d10 -> 3d6 -> 3d8 -> 3d10.

\textbf{Accecato}:\index{Accecato} Il personaggio non riesce a vedere nulla. subisce penalità -2 alla maggior parte delle Competenze basate su Forza e Destrezza.

Tutte le prove o le attività basate sulla visione (come ad esempio leggere, o eventuali prove di Consapevolezza basate sulla vista) falliscono automaticamente. Tutti gli avversari vengono considerati dotati di invisibilità nei confronti del personaggio accecato.

I personaggi accecati considerano il terreno sempre come difficile e devono effettuare una prova di Acrobatica con DC 12 per muoversi più veloci della propria velocità dimezzata. Le creature che falliscono questa prova cadono a terra Prone. I personaggi che rimangono per lungo tempo accecati possono abituarsi ad alcune di queste penalità e iniziare a superarne alcune, a discrezione del Narratore.

Chi attacca una creatura per lei invisibile ha un -1d6 al Tiro per Colpire, la creatura invisibile che attacca una creatura che non la vede ha +1d6 al Tiro per Colpire

\textbf{Affascinato}:\index{Affascinato} Una creatura affascinata non può attaccare o bersagliare chi l'ha affascinata con attacchi, capacità speciali o effetti magici dannosi.

Qualsiasi potenziale minaccia causata da chi l'ha affascinata, come ad esempio una creatura ostile in avvicinamento, consente alla creatura affascinata un nuovo Tiro Salvezza contro l'effetto del fascino. Qualsiasi minaccia palese, come ad esempio qualcuno che estrae un'arma, lancia un incantesimo o punta un'arma a distanza verso la creatura affascinata, interrompe automaticamente l'effetto.

Un alleato della creatura affascinata può scuoterla per permettergli un nuovo Tiro Salvezza spendendo 2 Azioni.

L'affascinatore ha +1d6 su qualsiasi prova di Competenza per interagire socialmente con la creatura.

L'effetto di Affascinato termina se la creatura diventa morente.

\textbf{Affaticato}\index{Affaticato}\hypertarget{affaticato}{}\index{Esausto}\label{affaticato}: Un personaggio affaticato non può correre o Caricare e subisce una penalità -1 a Difesa e Prove (Tiro per Colpire, Competenze, Magia, Tiri Salvezza) e Movimento. Se compie qualsiasi cosa normalmente affaticante aumenta il suo grado di Affaticato e prende penalità anche al movimento ed alle prove di competenza di Base.

Se un personaggio non dorme almeno 8 ore (TS Tempra DC 17) nell'arco di 24 ore o dorme con un armatura media o pesante alla mattina è affaticato.

Dopo 1 ora di completo riposo (o \hyperlink{Ristorare Inferiore}{Ristorare Inferiore}), un personaggio Affaticato 2 diventa Affaticato.

\medskip

\textbf{Tabella: Livelli Affaticamento}\index[Tabelle]{Tabella Livelli Affaticamento}

\medskip

\noindent\begin{tabularx}{\linewidth}{lcl}
	\toprule
\rowcolor{gray!20}\textbf{Condizioni}& \textbf{Penalità}&\textbf{Recupero}\\
\toprule
Affaticato 		&	1 		&	1h\\
\rowcolor{gray!20}Affaticato 2	&	2		&	1h\\
Affaticato 3	&	3		&	8h\\
\rowcolor{gray!20}Affaticato 4	&	4		&	12h\\
Affaticato 5	&	Svenuto	&	6h\\
\rowcolor{gray!20}Affaticato 6	&	Morte	&	--
\end{tabularx}

%\begin{tabularx}{\linewidth}{|c|X|l|}
%
%\textbf{Liv.} & \textbf{Effetto} & \textbf{Rec.}\\
%
%1 & -1d6 alle Prove di Competenza&1h\\
%2 & Velocità di movimento dimezzata &1h\\
%3 & -1d6 ai Tiri per Colpire e Tiri Salvezza &8h\\
%4 & Punti ferita massimi dimezzati &12h\\
%5 & Svenuto &6h\\
%6 & Morte &-\\
%
%\end{tabularx}

\medskip

Dopo 8 ore di riposo una creatura passa da Affaticato 3 ad Affaticato 2 e dopo un altra ora passa ad Affaticato, purché non subisca ulteriori affaticamenti.

\medskip

\textbf{Afferrato}\index{Afferrato}: Una creatura afferrata non può muoversi ma può provare a Spingere. Deve usare due Azioni per liberarsi, vedi sezione combattimento \hyperlink{afferrareunavversario}{Afferrare un avversario}.

Può attaccare con armi in mischia se piccole o a mani nude. Ha -2 alla Difesa ed è Distratto.

La condizione vale su chi è afferrato e su chi afferra.

\textbf{Amichevole}:\index{Amichevole} Una creatura amichevole non attaccherà il personaggio se non minacciata esplicitamente.

\textbf{Annegare/Trattenere il fiato}: \index{Annegare}\index{Trattenere il fiato}\index{Soffocare} Qualsiasi personaggio può trattenere il fiato per un numero di round pari 6 round per il suo punteggio di Costituzione, con un minimo di 3 round. Per ogni Azione compiuta la durata restante diminuisce di 1 round. Trascorso questo periodo di tempo, il personaggio deve effettuare un Tiro Salvezza su Tempra con DC 12 ogni round per continuare a trattenere il fiato. Ogni round, la DC aumenta di 2. Vedi pag. \pageref{trattenereilfiato}

\textbf{Assordato}:\index{Assordato}\index{Sordo} Un personaggio assordato non può ascoltare. Fallisce automaticamente tutte le prove di Consapevolezza basate sul suono e si considera Distratto nel lancio degli incantesimi con componenti almeno verbali.

I personaggi che rimangono assordati per lunghi periodi di tempo, possono abituarsi a queste penalità e superarne alcune, a discrezione del Narratore.

\textbf{Avvelenato}\index{Avvelenato}: si considera avvelenato qualsiasi soggetto sotto l'influenza di un veleno o pozione, indipendentemente che questa stia già producendo gli effetti o li debba ancora produrre dato il tempo dell'insorgenza. Una creatura immune al danno da Veleno non può avere la condizione Avvelenato.

\textbf{Bloccato}\index{Bloccato}\hypertarget{bloccato}{}: una creatura bloccata ha entrambe le braccia bloccate. Può spostarsi provando a Spingere, deve usare due Azioni per liberarsi (vedi \hyperlink{afferrareunavversario}{Afferrare un avversario}). Prende -4 alla Difesa ed ai Tiri Salvezza su Riflessi.

Un incantatore Bloccato è Distratto e deve effettuare una Prova di Magia con Successo Critico Magico o non riesce a formulare magie. -1d6 al Tiro per Colpire.

Chi ha Bloccato una creatura si considera Afferrato.

\textbf{Colpo di Grazia}:\index{Colpo di Grazia} Come unica Azione nel round, una creatura può utilizzare un'arma da mischia per infliggere un colpo di grazia ad un personaggio inabile o indifeso. Può anche usare un arco o una balestra, l'importante è che sia adiacente al bersaglio.

L'attaccante colpisce automaticamente ed infligge tre colpi critici. Le creature immuni ai colpi critici non possono subire un Colpo di Grazia.

\textbf{Confuso}: \index{Confuso}\index{Confusione}\hypertarget{confusionecondizione}{}Una creatura confusa è mentalmente ottenebrata e non può agire normalmente. Una creatura confusa non riesce a distinguere un alleato da un nemico e considera tutti come nemici.

Se una creatura confusa è attaccata, attacca sempre l'ultima creatura che la ha attaccata, finché quella creatura non muore o esce dalla sua visuale.

Tirate un dado sulla tabella seguente all'inizio di ogni round della creatura confusa per vedere quello fa in quel round.

\medskip

\noindent\begin{tabularx}{\linewidth}{l|X}
	\toprule
\rowcolor{gray!20}\textbf{d10} & \textbf{Comportamento} \\
\toprule
1 & La creatura usa tutte le sue Azioni per per muoversi in una direzione casuale. Per determinare la direzione tira un d8\\
\rowcolor{gray!20}2-5 & La creatura non fa nulla per tutto il round\\
6 &  La creatura effettua un attacco contro se stessa e finisce il round\\
\rowcolor{gray!20}7-8 & La creatura effettua un attacco contro una creatura determinata a caso entro 1 Azione di Movimento. Se è stata colpita il round precedente attaccherà la creatura che l'ha colpito. Fatto l'attacco il round termina.\\
9-10 & La creatura può agire e muoversi normalmente.
\end{tabularx}

\medskip

Una creatura confusa che non è in grado di eseguire l'azione indicata non farà altro che balbettare in modo incoerente. Gli aggressori non hanno alcun vantaggio speciale quando attaccano una creatura confusa. Qualsiasi creatura confusa che venga attaccata, attacca automaticamente a sua volta il suo aggressore.

\textbf{Distratto}\index{Distratto}: Se l'incantatore è severamente distratto, impedito, disturbato, sanguinante, afferrato, cerca di nascondere il lancio della magia, è sotto attacco mentre cerca di lanciare un incantesimo deve effettuare una Prova di Magia.

\textbf{Dominato}:\index{Dominato} Se si ha un linguaggio in comune, si può generalmente costringere il soggetto ad eseguire i comandi entro i limiti delle sue capacità. Se non si condivide nessun linguaggio, si possono impartire solo comandi di base come \emph{vieni qui}, \emph{vai lì}, \emph{combatti} o \emph{stai fermo}. Si è a conoscenza di ciò che il soggetto sta provando ma non si ricevono percezioni sensoriali dirette da lui, né si può comunicare con lui telepaticamente.

Una volta impartito un ordine alla creatura dominata, questa continua a tentare di eseguirlo con l'esclusione di tutte le altre attività ad eccezione di quelle necessarie per la sopravvivenza quotidiana (come mangiare, dormire e così via). Grazie a questo limitato spettro di attività, una prova di Consapevolezza con DC 15 (invece che DC 25) può determinare se il comportamento del soggetto è stato influenzato da un effetto di ammaliamento.

Concentrandosi completamente sull'incantesimo (2 Azioni), si possono ricevere percezioni sensoriali come vengono interpretate dalla mente del soggetto, anche se questo non può comunque comunicarle. Non si può in realtà vedere attraverso gli occhi del soggetto, quindi non è come se si fosse presenti, ma ci si può rendere conto di cosa sta succedendo.

Ovviamente ordini palesemente autodistruttivi non vengono eseguiti. Una volta stabilito il controllo, il raggio di azione entro il quale può essere mantenuto è illimitato purché entrambi i soggetti rimangano sullo stesso piano. Non c'è bisogno di vedere il soggetto per controllarlo. Se ogni giorno non si trascorre almeno 1 minuto a concentrarsi sull'incantesimo, il soggetto riceve un nuovo Tiro Salvezza per liberarsi dal controllo.

\textbf{Dormire}\index{Dormire}: Ogni volta che un personaggio termina un periodo di 24 ore senza dormire almeno 8 ore, deve superare un Tiro Salvezza su Tempra con DC 17, altrimenti diventa Affaticato. Ogni riposo mancato ulteriore lo renderà ancora più Affaticato cumulando le penalità relative.
Se il personaggio resta sveglio per più giorni, lottare contro il sonno diventa più difficile. Dopo le prime 24 ore, la DC aumenta di 4 per ogni periodo consecutivo di 24 ore trascorso senza aver dormito 8 ore. La DC torna a 17 quando il personaggio completa un riposo di almeno 8 ore.

\textbf{Fiancheggiare}\index{Fiancheggiare}\index{Fiancheggiato}: una creatura è fiancheggiata se ha due avversari non affiancatati attorno a se ed una ipotetica linea che unisce gli avversari attraversa il quadretto della creatura completamente. Le due creature prendono +2 al Tiro per Colpire o alla Difesa.

\textbf{Impreparato / Sorpreso}\index{Impreparato}\index{Surpreso}\label{impreparato}:
Una creatura sorpresa / impreparata ha una penalità di -2 a Difesa ed ai Tiri Salvezza su Riflessi. Non potrà reagire, non userà Azioni o Reazioni se non esplicitamente permesse; dal round successivo potrà dichiarare l'iniziativa ed agire normalmente.

\textbf{Inabile}\index{Inabile}: Una creatura inabile non può effettuare azioni o reazioni. E' Impreparata (-4 a Difesa ed ai Tiri Salvezza su Riflessi)

\textbf{In Lotta}:\index{Lotta} Una creatura in lotta è trattenuta da una creatura, da una trappola o da un effetto. Vedi \textbf{Afferrato}.

\textbf{Indifeso}:\index{Indifeso}\index{Addormentato}\index{Privo di Sensi}\index{Morente}\hypertarget{indifeso}{}\hypertarget{morente}{}\label{morente} Un personaggio addormentato, Privo di Sensi, Morente o per qualche altro motivo completamente alla mercé dei suoi avversari, è considerato Indifeso.

Una creatura Indifesa non può compiere Azioni o Reazioni ne parlare, gli attacchi in mischia contro di lei hanno +1d6 di bonus. Non è consapevole di ciò che gli accade intorno. La creatura lascia cadere qualsiasi cosa impugni e cade prona.

La creatura fallisce automaticamente i Tiri Salvezza su Tempra e Riflessi.

La creatura perde il bonus di Destrezza alla Difesa.

\textbf{Intralciato}:\index{Intralciato}\hypertarget{intralciato}{}\label{intralciato} Un personaggio intralciato ha difficoltà di movimento, ma può comunque provare a muoversi, a meno che i legami che lo intralciano non siano ancorati a un oggetto immobile o impugnati da una forza contrapposta.

Una creatura intralciata tratta il terreno come Difficile, non può Correre o Caricare, subisce penalità -2 alla Difesa ed ai Tiri per Colpire.

Un personaggio intralciato che cerca di lanciare un incantesimo si considera Distratto.

\textbf{Invisibile}:\index{Invisibile} Le creature invisibili non sono percepibili dalla vista.
Chi attacca una creatura per lei invisibile ha un -1d6 al Tiro per Colpire, la creatura invisibile che attacca una creatura che non la vede ha +1d6 al Tiro per Colpire.

\textbf{Morente} \index{Morente}: Un personaggio morente ha -1 Punti Ferita. E' indifeso. Ogni round perde 1 punto ferita finché muore o viene curato. Vedi \textbf{Indifeso}

\textbf{Nauseato}\index{Nauseato}: Se la penalità non è già espressa una creatura nauseata ha -1d6 ai Tiri per Colpire, Tiri Salvezza e Prove Competenze.

\textbf{Morto}:\index{Morto}\hypertarget{morto}{} L'anima del personaggio abbandona permanentemente il suo corpo. I personaggi morti non possono beneficiare delle cure normali o magiche, e non possono essere riportati in vita da un incantesimo. Solo un Patrono ha sufficiente potere per riportare l'anima nel corpo e riportare in vita la creatura. La Scuola di Necromanzia ha incantesimi per rianimare un corpo come non morto.

\textbf{Paralizzato}: \index{Paralizzato} Un personaggio paralizzato non può compiere Azioni o Reazioni ne parlare, gli attacchi in mischia contro di lei hanno +1d6 di bonus e perde il bonus alla Difesa dato dalla Destrezza. La creatura è consapevole di ciò che ha intorno, non lascia cadere gli oggetti. La creatura fallisce automaticamente i Tiri Salvezza su Riflessi.

Una creatura alata in volo, nel momento in cui viene paralizzata non può più battere le ali e precipita. Un nuotatore paralizzato non può più Nuotare e potrebbe annegare.

Il terreno occupato da una creatura paralizzata (o morta) si considera come terreno difficile.

\textbf{Perdita di punti Caratteristica}\index{Perdita di punti Caratteristica}\index{Caratteristica perdita punti}: quando si i punteggi Caratteristica diminuiscono ricordarsi di togliere eventuali Punti Ferita 1 per punto di Costituzione perso per livello, abbassare Tiri Salvezza (Destrezza, Costituzione, Saggezza), Tiri per Colpire (Forza e Destrezza), \textbf{Difesa:} (Difesa). Se non indicato come permanente si recupera 1 punto in tutto di Caratteristica al giorno di riposo.

\textbf{Pietrificato}: \index{Pietrificato}Un personaggio pietrificato è stato trasformato in pietra ed è privo di sensi ed \textbf{Indifeso}, viene considerato un oggetto.

Se un personaggio pietrificato si incrina o si rompe, ma i pezzi rotti sono uniti al corpo quando ritorna di carne, il personaggio non viene ferito o danneggiato. Se il corpo pietrificato del personaggio è incompleto quando viene ritrasformato in carne, il corpo rimane incompleto e potrebbe avere una qualche perdita permanente di Punti Ferita e/o altre menomazioni.

La creatura dispone di resistenza a tutti i danni. La creatura è immune ai veleni e le malattie, ma gli eventuali i veleni e le malattie già presenti nel suo sistema vengono solo sospesi, non neutralizzati. La creatura non ha percezione dell'ambiente ne capacità cognitive.

\textbf{Paura, Spaventato}:\index{Paura}\index{Spaventato}\hypertarget{condizionepaura}{}
Una creatura spaventata ha -1d6 ai Tiri per Colpire, Tiri Salvezza e Prove Competenza finché la sorgente della sua paura è visibile. Una creatura spaventata non può avvicinarsi volontariamente alla sorgente della sua paura.

\textbf{Privo di sensi}\index{Privo di sensi}: si considera che sia \textbf{Indifeso}.

\textbf{Prono}\index{Prono}: chi è prono ha un -4 ad attaccare ed un -4 alla Difesa. Alzarsi da prono costa 1 Azione. Non si può diventare proni se si vola.

Il giocatore può eseguire una prova di Acrobatica se fa 13 o più costa 1 Azione immediata. Se fai un Fallimento Critico nella prova non puoi fare altre azioni quel round e rimani prono.

Quando sei prono puoi strisciare\index{Strisciare}\index{Carponi} o muoverti a carponi. Il terreno si considera difficile e sei comunque considerato ancora prono finché non ti alzi.

\textbf{Punti Ferita Massimi}\index{Punti Ferita Massimi}: una creatura che subisca un attacco che abbassa i Punti Ferita Massimi deve calare prima i Punti Ferita Massimi attuali e poi diminuisce dello stesso ammontare i Punti Ferita attuali se non già tolti. Se i Punti Ferita Massimi arrivano a 0 la creatura è morta. I Punti Ferita massimi si recuperano nella misura di 1 per valore di Costituzione per 8 ore di riposo.

\textbf{Resistenza al Danno}\index{Resistenza al Danno}: una creatura che abbia Resistenza al Danno si considera che dimezzi automaticamente il danno dalla fonte specificata, es. Resistenza al Danno: Suono. La Resistenza al Danno può essere indicata anche con un valore numerico, es. Resistenza al Danno: Fuoco 10. In questo caso la protezione funziona sui primi 10 danni subiti, in caso di effetto che concede un Tiro Salvezza per dimezzare, prima si toglie l'ammontare della protezione al totale, poi si esegue il Tiro Salvezza per dimezzare il danno residuo.

\textbf{Stordito/Svenuto}:\index{Stordito}\index{Svenuto} si considera che sia \textbf{Indifeso}. Può parlare a fatica.

\textbf{Trattenere il fiato}: leggi \textbf{Annegare/Trattenere il fiato}

\textbf{Rallentato}\index{Rallentato}: una creatura rallentata non è in grado di eseguire tutte le sue possibili Azioni nel round. Rallentato viene sempre indicato con due valori, il primo indica quante Azioni in meno si fanno a round, il secondo la durata dell'effetto, se segnato con un - allora non ha una fine indicata. Quando è indicata la durata si intente fino alla fine del round di scadenza. Es. Rallentato 1/3r, Rallentato 2/-.

\textbf{Ristretto}\index{Ristretto}\hypertarget{ristretto}{} : due creature medie o piccole che condividono lo stesso quadretto di mappa si considerano ristretti. Entrambe le creature prendono -1d6 al Tiro per Colpire ed alla Difesa (-4) finché condividono lo spazio. Una creature può condividere il quadretto con una creatura di taglia almeno tre volte più grande senza penalità.

\textbf{Rotto}\index{Rotto}: La condizione rotto ha i seguenti effetti, a seconda dell'oggetto:

- Se l'oggetto è un'arma, tutti gli attacchi effettuati con l'oggetto subiscono penalità -2 al Tiro per Colpire ed ai danni. Tali armi ottengono un Colpo Critico soltanto con un 3 volte 6 ed infliggono al massimo solo 1 volta il danno in aggiunta.

- Se l'oggetto è un'armatura o uno scudo, il bonus che concede alla Difesa è dimezzato, arrotondando per eccesso. L'armatura rotta raddoppia la penalità di armatura alla Prova sulle Competenze.

- Se l'oggetto è un attrezzo necessario per una Competenza, tutte le prove di Competenza di Base effettuate con esso subiscono penalità -2.

- Se l'oggetto è una Bacchetta o un Bastone, utilizzate il doppio delle cariche necessarie ogni volta che viene usato.

- Se l'oggetto non rientra in nessuna delle precedenti categorie, la condizione rotto non ha effetto sul suo uso. Gli oggetti con condizione rotto, a prescindere dal tipo, valgono il 25\% del loro costo normale. Se l'oggetto è magico, può essere riparato soltanto con l'incantesimo \hyperlink{Fabbricare}{Fabbricare} utilizzata da un incantatore di livello uguale o superiore a quello che ha creato dell'oggetto.

\textbf{Sanguinante}\index{Sanguinante}\index{Sanguinamento}\hypertarget{sanguinamento}{}: Una creatura che sta subendo danni da sanguinamento subisce la quantità di danno indicata all'inizio del suo round. La Riduzione del danno (DR) non funziona sul danno da sanguinamento. Il sanguinamento può essere ridotto di un punto superando una prova di Pronto Soccorso con DC 12, 2 Azioni.
Per ogni valore di Sanguinamento sopra 1 la difficoltà aumenta di 2.

Un trattamento di 1 minuto garantisce 1 successo, senza prova. Ogni Successo Critico riduce il sanguinamento di un punto ulteriore. Alcuni effetti di sanguinamento causano un danno di caratteristica o persino un risucchio di caratteristica.

Il sanguinamento si riduce di 1 per ogni dado di cura della Pozione o Incantesimi. Se i Punti Ferita del soggetto vengono riportati al valore massimo il sanguinamento termina. Il sanguinamento prosegue anche se la creatura è \hyperlink{morente}{morente}.\index{Sanguinamento e cure}

Se non indicato diversamente il danno da sanguinamento si cumula fino ad un massimo di 10 Punti Ferita a round. Il danno da sanguinamento viene indicato con Sanguinamento valore/valore massimo, dove valore è il punteggio di sanguinamento causato dall'attacco e valore massimo è il punteggio di sanguinamento massimo che si può raggiungere.

Se la creatura diventa morente, va a Punti Feriti negativi, e poi viene riportata in vita perde gli effetti del Sanguinamento.

\textbf{Vulnerabilità}\index{Vulnerabilità}: funziona al contrario della Resistenza. Il danno viene raddoppiato prima dell'eventuale Tiro Salvezza.

\end{multicols}

\vfill

\begin{enfasi}
	Facilis descensus Averno (Facile è la discesa agli inferi). Eneide, Virgilio
\end{enfasi}

\pagebreak

\noindent\rule{\textwidth}{0.4pt}

\subsection{Tabelle per tiri casuali}

{\small \setlength{\parindent}{0cm}{

\begin{multicols}{2}

\textbf{Tabella Fonti di Energia}\index{Tabella Casuale - Fonti di Energia}\label{fontienergia}\hypertarget{fontienergia}{}

\noindent\begin{tabularx}{\linewidth}{ll|ll}
	\toprule
\rowcolor{gray!20}3d6 &Energia&&\\
3-5&Energia Neg.& 6-8&Energia Pos.\\
\rowcolor{gray!20}9-11&Fuoco & 12-13&Freddo\\
14&Elettricità & 15&Suono\\
\rowcolor{gray!20}16&Luce & 17&Vuoto\\
18&Forza&&\\
\end{tabularx}

\medskip

\begin{tikzpicture}[scale=0.4]

	\fill[orange] (1,17) rectangle (4,20);
	\fill[blue] (2.5,18.5) circle (0.3);
	\node at (3,16) [above] {raggio 1 metro};

	\fill[orange] (6,15) rectangle (11,20);
	\fill[white] (5,20) rectangle (6,21);
	\fill[white] (6,19) rectangle (7,20);
	\fill[white] (10,15) rectangle (11,16);
	\fill[white] (6,15) rectangle (7,16);
	\fill[white] (10,19) rectangle (11,20);
	\fill[blue] (8.5,17.5) circle (0.3);
	\node at (8.5,14) [above] {raggio 2 metri};

	\fill[orange] (12,13) rectangle (19,20);
	\fill[white] (12,19) rectangle (13,20);
	\fill[white] (18,13) rectangle (19,14);
	\fill[white] (12,13) rectangle (13,14);
	\fill[white] (18,19) rectangle (19,20);
	\fill[blue] (15.5,16.5) circle (0.3);
	\node at (16,12) [above] {raggio 3 metri};

	\fill[orange] (2,8) -- (7.5,10.5) -- (2,13) -- cycle;
	\fill[blue] (7.5,10.5) circle (0.3);
	\node at (4.5,7) [above] {cono di 5 metri};

	\fill[orange] (17,8) -- (20.5,9.5) -- (17,11) -- cycle;
	\fill[blue] (20.5,9.5) circle (0.3);
	\node at (18,7) [above] {cono di 3 metri};

	%\fill[orange] (9,7) -- (15.5,10.5) -- (9,13) -- cycle;
	%\fill[blue] (15.5,10.5) circle (0.3);
	%\node at (12,6.5) [above] {cono di 6 metri};

	\fill[orange] (9,7) -- (15.5,10) -- (9,13) -- cycle;
	\fill[blue] (15.5,10.5) circle (0.3); % Punto blu sulla punta destra
	\node at (12,6.5) [above] {cono di 6 metri};

	% Cubo di 3 quadretti di lato
	%\fill[orange] (2,2) -- (5,2) -- (5,5) -- (2,5) -- cycle; % Faccia frontale
	%\fill[orange] (5,2) -- (6,3) -- (6,6) -- (5,5) -- cycle; % Faccia destra
	%\fill[orange] (2,5) -- (5,5) -- (6,6) -- (3,6) -- cycle; % Faccia superiore

	% Cubo di 6 quadretti di lato
	%\fill[orange] (8,2) -- (14,2) -- (14,8) -- (8,8) -- cycle; % Faccia frontale
	%\fill[orange] (14,2) -- (16,4) -- (16,10) -- (14,8) -- cycle; % Faccia destra
	%\fill[orange] (8,8) -- (14,8) -- (16,10) -- (10,10) -- cycle; % Faccia superiore

	% Draw the base grid (20x15)
	\foreach \x in {0,...,20}
	\foreach \y in {5,...,20}
	\draw[gray!30] (\x,\y) grid (\x+1,\y+1);

\end{tikzpicture}

\medskip

Il punto blu determina l'origine dell'incantesimo

\bigskip

\textbf{Esempi portata dei nemici}

\medskip

\textbf{Tabella: forme degli incantesimi - Sfera e Cono}\index[Tabelle]{Tabella forme degli incantesimi - Sfera e Cono}

\medskip


\begin{tikzpicture}[scale=0.4]

	\fill[orange] (1,18) rectangle (2,19);
	\node at (1.5,18.5) {1};

	\foreach \x in {-1,...,1} {
		\node at (\x+1.5,17.5) {A};
		\node at (\x+1.5,19.5) {A};
	}

	\foreach \y in {18} { \node at (+0.5,\y+0.5) {A};
		\node at (2.5,\y+0.5) {A}; }

	\node at (1.5,16) [above] {Portata 1m};

	% Quadrato centrale 2x2 con il numero 2 dentro
	\fill[orange] (5,17) rectangle (7,19);
	\node at (5.5,17.5) {2};
	\node at (5.5,18.5) {2};
	\node at (6.5,17.5) {2};
	\node at (6.5,18.5) {2};

	% Lettere A intorno al quadrato centrale 2x2
	\foreach \x in {3,...,6} {
		\node at (\x+1.5,19.5) {A}; % Sopra
		\node at (\x+1.5,16.5) {A}; % Sotto
	}
	\foreach \y in {9,...,10} {
		\node at (4.5,\y+8.5) {A}; % Sinistra
		\node at (7.5,\y+8.5) {A}; % Destra
	}

	\node at (6,15) [above] {Portata 1m};

	% Quadrato 3x3 con il numero 3 dentro, spostato a sinistra di 1 quadretto e alzato di 1 quadretto
	\fill[orange] (11,15) rectangle (14,18);
	\node at (11.5,17.5) {3};
	\node at (12.5,17.5) {3};
	\node at (13.5,17.5) {3};

	\node at (11.5,16.5) {3};
	\node at (12.5,16.5) {3};
	\node at (13.5,16.5) {3};

	\node at (11.5,15.5) {3};
	\node at (12.5,15.5) {3};
	\node at (13.5,15.5) {3};

	% Prima cornice di quadretti con la lettera A
	% Sopra e sotto il quadrato 3x3
	\foreach \x in {10,...,14} {
		\node at (\x+0.5,14.5) {A};
		\node at (\x+0.5,18.5) {A};
	}

	% Ai lati del quadrato 3x3
	\foreach \y in {13,...,17} {
		\node at (10.5,\y+1.5) {A};
		\node at (14.5,\y+1.5) {A};
	}

	% Seconda cornice di quadretti con la lettera B
	% Sopra e sotto la prima cornice
	\foreach \x in {9,...,15} {
		\node at (\x+0.5,19.5) {B};
		\node at (\x+0.5,13.5) {B};
	}

	% Ai lati della prima cornice
	\foreach \y in {13,...,18} {
		\node at (9.5,\y+0.5) {B};
		\node at (15.5,\y+0.5) {B};
	}
	\node at (12.5,12) [above] {Portata 2m};

	\foreach \x in {-1,...,19}
	\foreach \y in {12,...,19}
	\draw[gray!30] (\x,\y) grid (\x+1,\y+1);

\end{tikzpicture}

\textbf{Generatori Effetti Fallimento Critico in Attacco}\index{Tabella Casuale - Fallimento critico con armi}\label{tabellafallimentiarmi}\hypertarget{tabellafallimentiarmi}{}

\noindent\begin{tabularx}{\linewidth}{l|X}
	\toprule
\rowcolor{gray!20}\textbf{3d6} & \textbf{Effetto}\\
\toprule
3& Sei imbarazzato del tuo colpo, ma non succede nulla di particolare\\
\rowcolor{gray!20}4& Ti sbilanci. Fino a all'inizio del prossimo round hai -2 alla Difesa\\
5& Metti male il piede. Fino alla fine del prossimo round tratti il terreno come difficile\\
\rowcolor{gray!20}6& Perdi il fiato. Fino all'inizio del prossimo round ha -1 Forza\\
7& Piroetta. Ti sposti in una direzione casuale di 1 metro\\
\rowcolor{gray!20}8& Goffo. Fino a all'inizio del prossimo round hai -4 alla Difesa\\
9& Occhio pesto. Fino alla fine del round prossimo ogni avversario gode di copertura leggera\\
\rowcolor{gray!20}10 & Mani di burro. Ti cade l'arma\\
11 & Strappo muscolare. Il prossimo attacco non aggiunge la Forza al danno\\
\rowcolor{gray!20}12 & Caviglia fragile. Portando il colpo inciampi. Cadi prono\\
13 & Ti cade l'arma a 3 metri in una direzione casuale\\
\rowcolor{gray!20}14 & Perdi fiducia in te stesso. Il prossimo attacco lo esegui con un -4 aggiuntivo\\
15 & Bersaglio sbagliato. Colpisci una creatura a caso a portata.\\
\rowcolor{gray!20}16 & Confuso. Ti dai una botta sulla testa. Fino alla fine del round sei sotto l'incantesimo \hyperlink{Confusione}{Confusione}\\
17 & Ti colpisci con forza. Tira il danno regolare verso te stesso. Il round finisce\\
\rowcolor{gray!20}18 & Ti colpisci con forza alla testa. Come 17 e chi è in mischia contro di te può effettuare un attacco d'opportunità usando una Reazione. Il round finisce\\
19+& Ti colpisci con forza. Tira il danno e applica due colpi critici verso te stesso. Il round finisce
\end{tabularx}

\medskip

\end{multicols}

\textbf{Tabella: generazione casuali armi}\index[Tabelle]{Tabella generazione casuali armi}

\medskip

\resizebox{\linewidth}{!}{\begin{tabular}{ll|ll|ll|ll}
		\toprule
  \rowcolor{gray!20}\textbf{1d100} & \textbf{Arma} & \textbf{1d100} & \textbf{Arma} & \textbf{1d100} & \textbf{Arma} & \textbf{1d100} & \textbf{Arma} \\
\toprule
		1-2   & Arma rotta & 26-27 & Alabarda & 51-52 & Spada corta & 76-77 & Spada lunga \\
  \rowcolor{gray!20}3-4   & Arco lungo & 28-29 & Arco corto & 53-54 & Spada a due lame & 78-79 & Spada bastarda \\
		5-6   & Ascia da battaglia & 30-31 & Ascia ad una mano & 55-56 & Picca pesante & 80-81 & Spada larga \\
  \rowcolor{gray!20}7-8   & Balestra ad una mano & 32-33 & Bastone & 57-58 & Pugnale & 82-83 & Spadone a due mani \\
		9-10  & Balestra leggera & 34-35 & Falce & 59-60 & Scimitarra & 84-85 & Stocco \\
  \rowcolor{gray!20}11-12 & Balestra pesante & 36-37 & Flagello doppio & 61-62 & Mazza leggera & 86-87 & Tridente \\
		13-14 & Catena chiodata & 38-39 & Frusta & 63-64 & Mazza flangiata & 88-89 & Urgrosh \\
  \rowcolor{gray!20}15-16 & Estoc & 40-41 & Giavellotto & 65-66 & Mazza chiodata & 90-91 & Katana \\
		17-18 & Falcetto & 42-43 & Machete & 67-68 & Picca leggera & 92-93 & Manganello \\
  \rowcolor{gray!20}19-20 & Falcione in asta & 44-45 & Martello da guerra & 69-70 & Lancia & 94-95 & Guanto chiodato \\
		21-22 & Arco lungo composito & 46-47 & Flagello pesante & 71-72 & Flagello & 96-97 & Ascia martello \\
  \rowcolor{gray!20}23-24 & Arco corto composito & 48-49 & Lancia da fante & 73-74 & Grande ascia doppia & 98-99 & Maglio da guerra \\
		25    & Falcione & 50    & Randello & 75    & Lancia corta & 100   & Arma speciale \\

\end{tabular}}

}}





%\bigskip

%\textbf{Esempio di Fiancheggiamento}

%\begin{tikzpicture}[scale=0.7]
% Disegna il reticolato 4x4
%\foreach \i in {0,...,4} {
%	\draw[thick] (\i,0) -- (\i,4);
%	\draw[thick] (0,\i) -- (4,\i);
%}

% Colora i quadrati 2, 3, 14 e 15 di arancione
%\fill[orange!60, opacity=0.7] (1,3) rectangle (2,4); % quadrato 2
%\fill[orange!60, opacity=0.7] (2,3) rectangle (3,4); % quadrato 3
%\fill[orange!60, opacity=0.7] (1,0) rectangle (2,1); % quadrato 14
%\fill[orange!60, opacity=0.7] (2,0) rectangle (3,1); % quadrato 15

% Traccia linee dritte che attraversano i quadrati
% Linea dal 3 al 14
%\draw[red, ultra thick] (2.5,3.5) -- (1.5,0.5);

% Linea dal 2 al 15
%\draw[blue, ultra thick] (1.5,3.5) -- (2.5,0.5);

% Nuova linea da 4 a 13 (nera)
%\draw[black, ultra thick] (3.5,3.5) -- (0.5,0.5);

% Nuova linea da 1 a 16 (gialla)
%\draw[yellow!80!black, ultra thick] (0.5,3.5) -- (3.5,0.5);

% Numera i quadrati da 1 a 16
%\foreach \i in {0,...,3} {
%	\foreach \j in {0,...,3} {
%		\pgfmathtruncatemacro{\num}{4*(3-\j)+\i+1}
%		\node at (\i+0.5,\j+0.5) {\Large \num};
%	}
%}

%\end{tikzpicture}

%\smallskip

%In questo esempi i personaggi nei quadretti 1-16,4-15,2-15 e 3-4 stanno fiancheggiando.

\pagebreak

\section{Generatore Nomi Casuali}

\begin{multicols}{2}

Per creare un nome fantasy, tira 3d6 sulla Tabella Sillabe Iniziali, 2d6 sulle Centrali e 3d6 sulla Finale.

{\small

\subsection*{Tabelle per Nomi Maschili}

\subsubsection*{Sillabe Iniziali Maschili}
\noindent\begin{tabularx}{\linewidth}{X|l|X|l}
	\toprule
\rowcolor{gray!20}\textbf{3d6} & \textbf{Sillaba} & \textbf{3d6} & \textbf{Sillaba} \\
\toprule
3 & Zeph & 11 & Mer \\
\rowcolor{gray!20}4 & Xar & 12 & Kor \\
5 & Bra &    13 & Lun      \\
\rowcolor{gray!20}6 & Cae & 14 & Nor     \\
7 & Dar & 15 & Oth      \\
\rowcolor{gray!20}8 & Eld &  16 & Quin \\
9 & Gar  & 17 & Thaan \\
\rowcolor{gray!20}10 & Hal &  18 & Zarak  \\
\end{tabularx}


\subsubsection*{Sillabe Centrali (2d6)}
\noindent\begin{tabularx}{\linewidth}{X|l|X|l}
	\toprule
\rowcolor{gray!20}\textbf{2d6} & \textbf{Sillaba} & \textbf{2d6} & \textbf{Sillaba} \\
\toprule
2 & - (salta) & 8 & ren \\
\rowcolor{gray!20}3 & - (salta) & 9 & el \\
4 & - (salta) & 10 & an \\
\rowcolor{gray!20}5 & dor & 11 & en \\
6 & gal & 12 & thil \\
\rowcolor{gray!20}7 & ar & & \\
\end{tabularx}


\subsubsection*{Sillabe Finali Maschili (3d6)}

\noindent\begin{tabularx}{\linewidth}{X|l|X|l|X|l}
	\toprule
\rowcolor{gray!20}\textbf{3d6} & \textbf{Sillaba} & \textbf{3d6} & \textbf{Sillaba} & \textbf{3d6} & \textbf{Sillaba} \\
\toprule
3 & grim & 9 & dan & 15 & reth \\
\rowcolor{gray!20}4 & thus & 10 & ion & 16 & thor \\
5 & dus & 11 & dor & 17 & valos \\
\rowcolor{gray!20}6 & kar & 12 & nor & 18 & zeth \\
7 & las & 13 & on & & \\
\rowcolor{gray!20}8 & mel & 14 & mir & & \\
\end{tabularx}


\subsection*{Tabelle per Nomi Femminili}

\subsubsection*{Sillabe Iniziali Femminili (3d6)}

\noindent\begin{tabularx}{\linewidth}{X|l|X|l|X|l}
	\toprule
\rowcolor{gray!20}\textbf{3d6} & \textbf{Sillaba} & \textbf{3d6} & \textbf{Sillaba} & \textbf{3d6} & \textbf{Sillaba} \\
\toprule
3 & Zara & 9 & Gwen & 15 & Ora \\
\rowcolor{gray!20}4 & Xyla & 10 & Hel & 16 & Syl \\
5 & Ara & 11 & Mira & 17 & Thea \\
\rowcolor{gray!20}6 & Bel & 12 & Isa & 18 & Zyra \\
7 & Cel & 13 & Lil & & \\
\rowcolor{gray!20}8 & Ela & 14 & Nym & & \\
\end{tabularx}

\subsubsection*{Sillabe Finali Femminili (3d6)}
\noindent\begin{tabularx}{\linewidth}{X|l|X|l|X|l}
	\toprule
\rowcolor{gray!20}\textbf{3d6} & \textbf{Sillaba} & \textbf{3d6} & \textbf{Sillaba} & \textbf{3d6} & \textbf{Sillaba} \\
\toprule
3 & neth & 9 & ana & 15 & riel \\
\rowcolor{gray!20}4 & essa & 10 & ara & 16 & tha \\
5 & beth & 11 & elle & 17 & saal \\
\rowcolor{gray!20}6 & dra & 12 & lynn & 18 & vyth \\
7 & ena & 13 & ina & & \\
\rowcolor{gray!20}8 & iel & 14 & mira & & \\
\end{tabularx}


\subsection*{Nomi di Luoghi Fantasy}

\subsubsection*{Prefissi per Luoghi (2d10)}
\noindent\begin{tabularx}{\linewidth}{X|l|X|l|X|l}
	\toprule
 \rowcolor{gray!20}\textbf{2d10} & \textbf{Prefisso} & \textbf{2d10} & \textbf{Prefisso} & \textbf{2d10} & \textbf{Prefisso} \\
\toprule
	2 & Fonda & 9 & Forte & 16 & Terra \\
 \rowcolor{gray!20}3 & Alto & 10 & Lago & 17 & Torre \\
	4 & Basso & 11 & Monte & 18 & Valle \\
 \rowcolor{gray!20}5 & Bosco & 12 & Pietra & 19 & Vento \\
	6 & Campo & 13 & Quercia & 20 & Libera \\
 \rowcolor{gray!20}7 & Casa & 14 & Rocca & & \\
	8 & Colle & 15 & Sole & & \\
\end{tabularx}

\subsubsection*{Suffissi per Luoghi (2d10)}

\noindent\begin{tabularx}{\linewidth}{X|l|X|l|X|l}
\toprule
 \rowcolor{gray!20}\textbf{2d10} & \textbf{Suffisso} & \textbf{2d10} & \textbf{Suffisso} & \textbf{2d10} & \textbf{Suffisso} \\
\toprule
	2 & abisso & 9 & lande & 16 & rocca \\
 \rowcolor{gray!20}3 & borgo & 10 & marca & 17 & sponda \\
	4 & maso & 11 & mare & 18 & valle \\
 \rowcolor{gray!20}5 & campo & 12 & monte & 19 & varco \\
	6 & forte & 13 & nova & 20 & vetta \\
 \rowcolor{gray!20}7 & gard & 14 & piana & & \\
	8 & haven & 15 & porto & & \\
\end{tabularx}


\subsection*{Tabelle Speciali per Culture}

\subsubsection*{Nomi Elfici (2d12)}
\textbf{Tira 2d12 per iniziali e finali elfiche:}

\noindent\begin{tabularx}{\linewidth}{X|l|X|l}
	\toprule
\rowcolor{gray!20}\textbf{2d12} & \textbf{Iniziale} & \textbf{2d12} & \textbf{Finale} \\
\toprule
2 & Ael & 2 & adir \\
\rowcolor{gray!20}3 & Cel & 3 & anduil \\
4 & Elen & 4 & aron \\
\rowcolor{gray!20}5 & Gal & 5 & dir \\
6 & Gil & 6 & galadriel \\
\rowcolor{gray!20}7 & Hal & 7 & las \\
8 & Leg & 8 & moth \\
\rowcolor{gray!20}9 & Mith & 9 & ond \\
10 & Thran & 10 & riel \\
\rowcolor{gray!20}11 & Elr & 11 & thil \\
12 & Galar & 12 & wyn \\
\end{tabularx}


\subsubsection*{Nomi Nanici (2d12)}
\textbf{Tira 2d12 per iniziali e finali naniche:}

\noindent\begin{tabularx}{\linewidth}{X|l|X|l}
	\toprule
\rowcolor{gray!20}\textbf{2d12} & \textbf{Iniziale} & \textbf{2d12} & \textbf{Finale} \\
\toprule
2 & Bal & 2 & dan \\
\rowcolor{gray!20}3 & Bor & 3 & din \\
4 & Dain & 4 & dur \\
\rowcolor{gray!20}5 & Dur & 5 & grim \\
6 & Gim & 6 & gar \\
\rowcolor{gray!20}7 & Grim & 7 & kan \\
8 & Kaz & 8 & mor \\
\rowcolor{gray!20}9 & Mor & 9 & mur \\
10 & Thar & 10 & rik \\
\rowcolor{gray!20}11 & Thor & 11 & thor \\
12 & Ur & 12 & zak \\
\end{tabularx}


%\subsection*{Titoli e Epiteti}

%\subsubsection*{Titoli (2d6)}
%\noindent\begin{tabular}{|c|l||c|l|}
%\hline
%\textbf{2d6} & \textbf{Titolo} & \textbf{2d6} & \textbf{Titolo} \\
%
%2 & Ser & 8 & Capitano \\
%3 & Lord & 9 & Maestro \\
%4 & Lady & 10 & Devoto \\
%5 & Duca & 11 & Principe \\
%6 & Conte & 12 & Re/Regina \\
%7 & Barone & & \\
%
%\end{tabular}

%\subsubsection*{Epiteti (2d8)}
%\noindent\begin{tabular}{|c|l||c|l|}
%\hline
%\textbf{2d8} & \textbf{Epiteto} & \textbf{2d8} & \textbf{Epiteto} \\
%
%2 & il Saggio & 10 & il Coraggioso \\
%3 & la Forte & 11 & l'Ombra \\
%4 & il Giusto & 12 & la Lama \\
%5 & la Bella & 13 & il Drago \\
%6 & il Nero & 14 & la Tempesta \\
%7 & la Bianca & 15 & il Fulmine \\
%8 & il Rosso & 16 & la Morte \\
%9 & la Verde & & \\
%
%\end{tabular}


%\subsection*{Esempi di Nomi Generati}

%\textbf{Nomi Semplici (3d6):}
%\begin{itemize}
%    \item \textbf{Risultato 11+7+12:} Mer + en + nor = \textbf{Merenor}
%    \item \textbf{Risultato 8+3+10:} Eld + (salta) + ion = \textbf{Eldion}
%    \item \textbf{Risultato 10+9+11:} Hel + el + elle = \textbf{Helelle}
%\end{itemize}

%\textbf{Nomi Elfici (2d12):}
%\begin{itemize}
%%    \item \textbf{7+10:} Hal + riel = \textbf{Halriel}
%    \item \textbf{9+7:} Mith + las = \textbf{Mithlas}
%\end{itemize}

} %chiusi small

\end{multicols}



\pagebreak

\section{Autore e Contributi}\index{Autore}

\textbf{Autore}: Andres Zanzani - azanzani@gmail.com

\medskip

\noindent\textbf{Contributi}: Federica Angeli

\section{Materiale per giocare}\index{Scheda}\index{Manuale}\index{Schermo}\index{Informazioni sui personaggi}

\label{scheda-e-manuale}

Siete invitati a scaricare da GitHub \textbf{Old Bell School System} liberamente e senza vincoli se non quelli espressi dalla licenza.
Il sito principale è \href{https://github.com/buzzqw/TUS}{https://github.com/buzzqw/TUS}

\medskip

* \textbf{Manuale di OBSS}:
\href{https://github.com/buzzqw/TUS/blob/master/OBSS/OBSSv2.pdf}{OBSSv2.pdf}

https://github.com/buzzqw/TUS/blob/master/OBSS/OBSSv2.pdf

\smallskip

* \textbf{Scheda}:
\href{https://github.com/buzzqw/TUS/blob/master/OBSS/OBSSv2-scheda.pdf}{OBSSv2-scheda.pdf}

https://github.com/buzzqw/TUS/blob/master/OBSS/OBSSv2-scheda.pdf

\smallskip

Versione su 3 pagine
\href{https://github.com/buzzqw/TUS/blob/master/OBSS/OBSSv2-scheda-v3.pdf}{OBSSv2-scheda-v3.pdf}

https://github.com/buzzqw/TUS/blob/master/OBSS/OBSSv2-scheda-v3.pdf

\smallskip

* \textbf{Schermo del Narratore}:
\href{https://github.com/buzzqw/TUS/blob/master/OBSS/screenv2.pdf}{screenv2.pdf}

https://github.com/buzzqw/TUS/blob/master/OBSS/screenv2.pdf

\smallskip

* \textbf{Info sui personaggi}:
\href{https://github.com/buzzqw/TUS/blob/master/OBSS/OBSS-schema-narratore-personaggi.pdf}{OBSS-schema-narratore-personaggi.pdf}

https://github.com/buzzqw/TUS/blob/master/OBSS/OBSS-schema-narratore-personaggi.pdf

\smallskip

* \textbf{Changelog} \href{https://github.com/buzzqw/TUS/blob/master/OBSS/changelog.md}{changelog.md}

https://github.com/buzzqw/TUS/blob/master/OBSS/changelog.md

\smallskip

* \textbf{Segnalazioni}

Per qualsiasi segnalazione o consiglio aprite una issue su GitHub, oppure mandatemi una mail all'indirizzo azanzani@gmail.com

\section{Ringraziamenti}\index{Ringraziamenti}\index{DnD SRD}

\begin{enfasi}{
			\begin{center}
Ai miei amici, ai miei compagni di avventura, a tutti i giocatori che chiudono gli occhi e si lasciano guidare dalla fantasia in meravigliose avventure. (Andres Zanzani)
			\end{center}
}\end{enfasi}

\medskip

Questa opera include materiale tratto dal System Reference Document 5.1 ("SRD 5.1") della Wizards of the Coast LLC e disponibile all'indirizzo https://dnd.wizards.com/resources/systems-reference-document. L'SRD 5.1 ha licenza Creative Commons Attribuzione 4.0 Internazionale, disponibile all'indirizzo https://creativecommons.org/licenses/by/4.0/legalcode.


\medskip

%Ringrazio \href{https://github.com/ThomasJockin/readexpro}{Readex} ( https://github.com/ThomasJockin/readexpro ) per il font usato. Readex è un font dall'alta leggibilità anche per persone con difficoltà di lettura.

Ringrazio \href{https://www.brailleinstitute.org/freefont/}{Braille Institute} per il font usato. Atkinson Hyperlegible Font è un font dall'alta leggibilità anche per persone con difficoltà di lettura.

\section{Licenza}\hypertarget{Licenza}{}\label{Licenza}

\textbf{Old Bell School System (OBSS)} © 2021 by Andres Zanzani is licensed under \hyperref{https://creativecommons.org/licenses/by-sa/4.0/legalcode}{}{}{CC BY-SA 4.0}

Le immagine presenti nel manuali sono senza licenza o in pubblico dominio. Le immagini segnate con \emph{B.I.C.} e la copertina sono state create con Bing Image Creator. In caso di inclusione di immagini protette da diritto d'autore si prega di comunicarle per poter procedere alla rimozione.

Prima di qualsiasi utilizzo di OBSS o di sue parti siete pregati di contattarmi.

\vspace{0.8cm}

\noindent\begin{minipage}{0.3\textwidth}
\includegraphics[keepaspectratio,width=\linewidth]{immagini/CC_BY-SA_icon.svg.png}
\end{minipage}%
\hfill
\begin{minipage}{0.2\textwidth}\centering
\large{Powered by} \Huge\LaTeX\ {\normalsize {\&}} \Huge\textbf{GitHub}
\end{minipage}%
\hfill
\begin{minipage}{0.4\textwidth}\raggedleft
\begin{enfasi}
And enjoy the game. \\
{\small (Players' Guide to Immortals. Frank Mentzer)}
\end{enfasi}
\end{minipage}

%\vspace{0.5cm}

\thispagestyle{plain}
\begin{center}
\includepdf[pages={1,2},addtotoc={1,section,0,Scheda del Personaggio,incl:first},scale=0.9]{OBSSv2-scheda.pdf}
%\includepdf[pages={1,2},scale=0.85]{OBSS-scheda.pdf}
\end{center}

%\thispagestyle{plain}
%\begin{center}
%\begin{tikzpicture}[remember picture,overlay]
%\node[anchor=south west,inner sep=0pt] at ($(current page.south west)+(1cm,1cm)$) {
%\includegraphics[scale=0.93]{OBSS-scheda-0.png}
%};
%\end{tikzpicture}
%\end{center}
%\pagebreak
%\thispagestyle{plain}
%\begin{center}
%\begin{tikzpicture}[remember picture,overlay]
%\node[anchor=south west,inner sep=0pt] at ($(current page.south west)+(1cm,1cm)$) {
%\includegraphics[scale=0.93]{OBSS-scheda-1.png}
%};
%\end{tikzpicture}
%\end{center}

\pagebreak

\section{Le mie Opzioni}\index{Le mie Opzioni}

\normalsize

Anche io sono un Narratore e per quanto abbia costruito OBSS in base alle mie preferenze ci sono alcune Opzioni, che rendono il gioco più \emph{unico} che mi piace rendere disponibili ai personaggi.

Al mio tavolo da gioco solitamente propongo queste Opzioni, da decidere in Sessione Zero:

\begin{itemize}[leftmargin=*] \setlength{\itemsep}{0pt}

\item
\hyperlink{successoparziale}{Successo Parziale} pag. \pageref{successoparziale}

\item \hyperlink{elencotalentiarmi}{Opzionale - Elenco Manovre d'Arme} (pag. \pageref{elencotalentiarmi}) per rendere meno \emph{noiso} il fumbolare...

%\item
%\hyperlink{lunicaregola}{L'Unica Regola} la uso se sono Narratore di un gruppo di principianti. Pag. \pageref{lunicaregola}

\item
\hyperlink{Un solo credo}{Un solo credo} o \hyperlink{abilitadilista}{Abilità di Lista}, a scelta del giocatore.

\item
\hyperlink{abilitaiconiche}{Abilità Iconiche} in caso di lunghe campagne. Pag. \pageref{abilitaiconiche}

\item
\hyperlink{droghe}{Droghe} \textbf{NO}. Solo in caso di gruppi composti da persone mature ed adulte di testa. Pag. \pageref{droghe}

\item
\textbf{No all'utilizzo del cronometro sulle Luci}: se le vostre avventure non sono ambientate in dungeon o volete una gestione più snella non gestite la durata delle luci in tempo reale.

\end{itemize}

\vfill


\begin{multicols}{2}

\subsubsection*{Note}

{\small

Per me OBSS va giocato in maniera schietta, senza troppi pensieri e cervellotici progetti. OBSS non è fatto per uccidere i personaggi ma allo stesso modo non ne agevola la sopravvivenza, tutto sta al Narratore a decidere come si gioca. E' nel Narratore, nello stile dei giocatori e l'interesse del gruppo la chiave di gioco, OBSS vuole offrire il framework, gli strumenti, per giocare l'avventura.

Cercate di enfatizzare le scene, siate anche teatrali nelle descrizioni, togliete la patina al gioco pulito e politically correct. Rimane sempre il vostro mondo, il vostro tavolo ed il vostro gioco, cercate di dare quell'immersività che spesso nei sistemi più moderni si è un pò persa.
Quando c'è un combattimento fate che sia tale! Deve sentirsi il clangore delle armi, il cozzare sulle armature, l'ozono nell'aria causato dal fulmine, le bruciature crepitanti delle palle di fuoco. Fate che i giocatori apprezzino le possibilità offerte dal sistema e si possano divertire a cercare come effettuare la prova migliore.

Scegliete voi se i personaggi sono canaglie che cercano solo di sopravvivere e accumulare tesori oppure dare un taglio più classico o epico all'avventura. OBSS si sposa che entrambe le scelte, specialmente utilizzando qualche Opzione rispetto ad un altra.

Create il gruppo, e non intendo solo come insieme di personaggi, ma come anche insieme di giocatori. Un gruppo dove le persone si rispettano e fidano (possibilmente...). Costruite avventure che coinvolgano tutti, dove tutti possano dare il loro contributo. Ci potranno essere avventure più \emph{cucite} intorno ad un personaggio ma questo non \textbf{deve} escludere gli altri nel partecipare, nel più ampio termine della parola, non fate che la sessione sia un monologo tra voi ed il singolo giocatore.
Approfittate di ogni avventura per fare conoscere i personaggi tra loro, nulla unisce di più che la paura di morire!

Una volta fatto il gruppo, e potrebbe volerci anche tempo, allora sfruttate le storie personali, gli indizi ed ipotesi create dai giocatori per plasmare situazioni e accadimenti. Come un pesante volano che ruota questa continuerà a creare situazioni, avventure e nuovi plot da seguire.

Potrebbero esserci delle difficoltà nel creare il gruppo, capita purtroppo. Cercate di parlare con il giocatore che da problemi. Cercate di capire se è il suo personaggio che non \emph{funziona} con il gruppo oppure è il giocatore che non ha ben compreso le meccaniche del gruppo.

Per questo vi suggerisco sempre di fare la così detta \hyperlink{sessionezero}{Sessione Zero} (pag. \pageref{sessionezero}), dove come Narratore andrete a delineare a grandi linee quali sono i cardini dell'avventura, cosa vi aspettate dai personaggi, quali sono le regole di base e di morale da seguire. Non c'è nulla di peggio di un gruppo di personaggi slegati dove ognuno vuole fare qualcosa di diverso e non gli interessa l'\emph{obiettivo comune}.

E' molto importante capire cosa piace ai giocatori, ogni persona e gruppo vuole un certo stile di gioco ed è corretto cercare di accontentarli. Se il gruppo vuole avventure politiche, drama romantici cercate di fargli trovare soddisfazione nel mentre dell'avventura. Se invece preferiscono più combattere allora non lesinate scontri purché coerenti con l'avventura stessa.

Fate capire che dovete funzionare come insieme di giocatori e personaggi per poter giocare al meglio e divertirvi tutti ed avere maggiori possibilità di sopravvivenza. Nessun giocatore deve essere sopra gli altri, solo il Narratore ha l'ultima parola.

Infine siate sempre corretti, nel bene e nel male. Ci saranno sessioni più sfortunate ed altre dove i dadi troveranno la strada giusta, dove l'idea brillante salverà il gruppo. Non fate il Narratore che salva \textbf{sempre e comunque} i personaggi, un aiuto ogni tanto ci può stare specialmente nella sessione più sfortunata, ma rispettate le scelte dei personaggi e l'esito dei dadi. Ricordate che i giocatori hanno i Punti Fato da poter usare a differenza dei poveri mostri!.

Personalmente io tiro tutti i dadi davanti ai giocatori.

Il \textbf{Narratore non deve rubare la scena} ai personaggi, bensì costruire l'ambiente, i drammi e le passioni intorno a loro. Non c'è nulla di più fastidioso di un Narratore con manie di protagonismo, vuoi nella tenuta dei PNG vuoi nell'imporre scelte e scene.

Ed infine un ovvietà: \textbf{divertitevi}, sforzatevi \textbf{tutti} affinché la sessione abbia quel misto di tensione, divertimento e soddisfazione. Siete persone che vogliono giocare, divertirsi e stare insieme, non dimenticatelo mai.

} %small

\end{multicols}


\pagebreak

\begin{multicols}{4}
	{\small\printindex}
\end{multicols}

\TotalBox{OBSSv2}\pagebreak

\begin{multicols}{3}
{\small\printindex[Tabelle]}
\end{multicols}

\vfill

\TotalBox{Tabelle}\pagebreak

\begin{multicols}{4}
{\small\printindex[Incantesimi]}
\end{multicols}

%\immediate\write18{./contaspell.sh > contaspell.txt}
%\immediate\openin\myscriptresult=./contaspell.txt
%\read\myscriptresult to \ScriptResult
%\immediate\closein\myscriptresult

\vfill

%Totale elementi in questo indice \ScriptResult

\TotalBox{Incantesimi}\pagebreak

\begin{multicols}{3}
	{\small\printindex[OggettiMagici]}
\end{multicols}

%{\scriptsize\printindex[OggettiMagici]}

%\immediate\write18{./contaomagici.sh > contaomagici.txt}
%\immediate\openin\myscriptresult=./contaomagici.txt
%\read\myscriptresult to \ScriptResult
%\immediate\closein\myscriptresult

\vfill

%Totale elementi in questo indice \ScriptResult

\TotalBox{OggettiMagici}

\begin{multicols}{3}
	{\small\printindex[Abilita]}
\end{multicols}

\vfill

\TotalBox{Abilita}\pagebreak

\begin{multicols}{3}
	{\small\printindex[Mostruario]}
\end{multicols}

\vfill

\TotalBox{Mostruario}

%Totale elementi in questo indice \ScriptResult

%\immediate\write18{./contamostri.sh > contamostri.txt}
%\immediate\openin\myscriptresult=./contamostri.txt
%\read\myscriptresult to \ScriptResult
%\immediate\closein\myscriptresult


\pagebreak

\section*{Appendice N}

\begin{multicols}{2}
		{\small
		\nocite{*}
		\setlength{\columnsep}{1cm} % spazio tra colonne
		\printbibliography
		}
\end{multicols}

\end{document}

% da mettere a riga 1
% arara: xelatex
% arara: biber
% arara: xelatex
% arara: xelatex: { synctex: on }

%livelli alternativi
%copper
%iron
%silver
%gold
%platinum
%mithril
%oricalcum
%adamantium
