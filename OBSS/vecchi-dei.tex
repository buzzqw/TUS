\subsubsection{Le divinità Antiche}\index{divinità Antiche}

I pochi dogmi presentati sono quanto e' stato ricordato dalla lettura de \textit{Il libro degli Dei} dal primo, e unico, devoto che lo lesse milenni addietro.

Essendo basati su ricordi e tradizioni orali e non su documentazione oggettiva non si ha la certezza che siano tutti corretti o completi, ma la fede, come la terra, e' eterna e quindi finché l'arcano tomo non verra' ritrovato e i nomi di tutti gli dei conosciuti, fino ad allora queste saranno le uniche e vere tavole della fede.

\bigskip

\begin{tabularx}{0.95\textwidth}{XXXX}
\textbf{Bene} & \textbf{Male} & \textbf{Chaos} & \textbf{Legge}\\
Koira - La Carità & Daraka - L'Oscurita' & 	Sefryn - Gli Elementi&	Talo - La Conoscenza\\
Daern - Il Creatore & Zarkor - La Sofferenza & Zxcvbnm - I'Inganno & Moraim - La Giustizia\\
Solimi - La Luce &	Eramide - I Morti & Jaskara - La Magia & Ediath - La Morte\\
Seeyek - L'Avventura & 	Averim - La Bellezza & Xabax - Il Destino \\
&& Isar - La Fortuna \\
\textbf{Generico Buono} & \textbf{Generico Malvagio} & \textbf{Generico Chaotico} & \textbf{Generico Legge}\\
\textbf{Divinità Neutrale}: &Kyriel \\

\end{tabularx}

\pagebreak

\subsubsection{Koira}

Aspetto: La Vita\index{Koira}
\bigskip


I fedeli della dea Koira sono i curatori per eccellenza, padroni della vita e portatrici di salute.

I fedeli di Koira sono non violenti e curatori di tutti gli ammalati e bisognosi, l'unica arma loro concessa e' il bastone, usato solo come autodifesa.

Questi i loro dogmi:

\begin{itemize}
	\item Mai rifiutare una cura a chi ne' ha bisogno
	\item Mai curare se stessi se gli altri ne hanno maggiore bisogno
	\item Dare morte pietosa a chi non puo' essere curato
	\item Colpire solo in autodifesa e mai uccidere se non necessario
	\item Impedire uccisioni inutili e indiscriminate
	\item La vita e' sacra ed inviolabile
\end{itemize}


I fedeli di Koira sono spesso curatori vagabondi che girano per villaggi a portare aiuto e assistenza.\\


\textbf{Vantaggi}: Cure Efficaci e Controllo del Metabolismo oppure Imposizione delle Mani \\
\textbf{Essenza Favorita}: Cura\\
\textbf{Essenza Sfavorita}: Distruzione\\

\medskip

Accesso alla Scuole di Magia:\\
Scuole Privilegiate: Universale, Necromanzia (+2 alle prove di CM)\\
Scuole Normali: Illusione, Abiurazione, Invocazione, Trasmutazione, Divinazione, Evocazione, Ammaliamento (+0 alle prove di CM)\\


\bigskip

\subsubsection{Daern}

Aspetto: La Creazione\index{Daern}
\bigskip


I fedeli di Daern sono i custodi della creazione e della purezza originaria.

Essi sostengono di essere i seguaci del dio generatore di tutto e tutti.

Spesso artisti o ingegneri, i maggiori prelati si trovano tra i nani che si reputano la prima razza dal dio creata.

\begin{itemize}
	\item Tutto il creato e' sacro, ogni cosa ha un costo nell'universo, ogni atto di distruzione non necessario e' un sacrilegio.
	\item Daern e' il primo ed ultimo
	\item Colui che plasma e' in grazia al dio
\end{itemize}


I fedeli di Daern hanno una organizzazione abbastanza capillare in tutto il territorio, e' il culto con piu' seguaci.\\

\textbf{Vantaggi}: Radici Magiche\\
\textbf{Essenza Favorita}: Creazione\\
\textbf{Essenza Sfavorita}: Illusione\\

\medskip

Accesso alla Scuole di Magia:\\
Scuole Privilegiate: Universale, Evocazione (+2 alle prove di CM)\\
Scuole Normali: Illusione, Abiurazione, Invocazione, Necromanzia, Trasmutazione, Divinazione, Ammaliamento (+0 alle prove di CM)\\


\bigskip
\subsubsection{Moraim}

Aspetto: La Giustizia\index{Moraim}
\bigskip


I fedeli del dio Moraim sono giudici e giustizieri per scelta e vocazione. essi si considerano gli unici portatori della giustizia divina in terra.


Culto molto guerriero e' diffuso presso gli alti ranghi guerriero-aristocratici

\begin{itemize}
	\item Lotta eterna contro il male e il chaos: tutto cio' che e' male o puo' divenirlo deve essere distrutto
	\item Chi combatte per la verita' non muore, chi la disprezza e' maledetto e condannato in eterno
\end{itemize}

Meno diffusi di Daern, i fedeli di Moraim appartengono spesso a piccole elite guerriere, spesso partecipano a guerre sacre o in opere di colonizzazione\\

\textbf{Vantaggi}: Colpi Poderosi\\
\textbf{Essenza Favorita}: Protezione\\
\textbf{Essenza Sfavorita}: Rivelazione\\

\medskip

Accesso alla Scuole di Magia:\\
Scuole Privilegiate: Universale, Abiurazione (+2 alle prove di CM)\\
Scuole Normali: Illusione, Abiurazione, Invocazione, Necromanzia, Trasmutazione, Divinazione, Evocazione, Ammaliamento (+0 alle prove di CM)\\


\bigskip
\subsubsection{Solimi}

Aspetto: Il Sole\index{Solimi}
\bigskip

I fedeli della dea Solimi sono portatori di luce e verita', spesso fungono da oracoli. Sempre ben abbronzati, potrebbero fissare il sole tutto il tempo.

Di morale positiva sono tra i fedeli piu' diffusi

\begin{itemize}
	\item La luce e' vita e conoscenza
	\item Il sole sorge dal grembo di Solimi ed e' portatore di vita
	\item Mai spegnere una fonte di luce
\end{itemize}

Fedeli tranquilli e pacifici sanno pero' usare fuoco e fiamme per difendere i loro protetti.\\

\textbf{Vantaggi}: Illuminato oppure Direzione Assoluta\\
\textbf{Essenza Favorita}: Attacco\\
\textbf{Essenza Sfavorita}: Trasformazione\\

\medskip

Accesso alla Scuole di Magia:\\
Scuole Privilegiate: Universale, Invocazione, (+2 alle prove di CM)\\
Scuole Normali: Illusione, Abiurazione, Necromanzia, Trasmutazione, Divinazione, Evocazione, Ammaliamento (+0 alle prove di CM)\\

\bigskip

\subsubsection{Isar}

Aspetto: La Fortuna\index{Isar}
\bigskip

Com'e' il bicchiere ? mezzo pieno?, no quasi colmo. questi sono i fedeli del dio Isar ,ottimisti e spensierati, ma non per questo ingenui o sciocchi. Sanno di essere fortunati e di cio' ne approfittano

\begin{itemize}
	\item Mai disperarsi, non siamo soli 
	\item Se qualcosa puo' andare male, il fedele si riterrà fortunato perche' sà che poteva andare peggio
	\item Se c'e' un problema c'e' anche una soluzione
\end{itemize}


Culto diffusissimo, noto ovunque e discretamente praticato.\\


\textbf{Vantaggi}: Fortunato\\
\textbf{Essenza Favorita}: Movimento\\
\textbf{Essenza Sfavorita}: Convocazione\\

\medskip

Accesso alla Scuole di Magia:\\
Scuole Privilegiate: Universale, Trasmutazione (+2 alle prove di CM)\\
Scuole Normali: Illusione, Abiurazione, Necromanzia, Invocazione, Divinazione, Evocazione, Ammaliamento (+0 alle prove di CM)\\

\bigskip

\subsubsection{Talo}

Aspetto: La Conoscenza\index{Talo}
\bigskip

I fedeli di Talo sono gli studiosi e i custodi della conoscenza universale. Spesso stantii nelle grandi biblioteche i fedeli di Talo possono sapere qualsiasi cosa


\begin{itemize}
	\item Scopo nella vita e' accumulare sapienza. non e' importante l'uso che di essa e ne fa'
	\item La conoscenza va raccolta e diffusa, chi mantiene la conoscenza solo per se' compie peccato contro Talo
	\item Chi fa si che la conoscenza venga distrutta compie un sacrilegio
	\item La conoscenza pratica e quella teorica sono di eguale rispetto
\end{itemize}

I fedeli di Talo sono tra i principali creatori di magie e molte delle piu' interessanti creazioni e combinazioni sono loro.\\


\textbf{Vantaggi}: Lingua Universale\\
\textbf{Essenza Favorita}: Rivelazione\\
\textbf{Essenza Sfavorita}: Attacco\\


\medskip

Accesso alla Scuole di Magia:\\
Scuole Privilegiate: Universale, Divinazione (+2 alle prove di CM)\\
Scuole Normali: Illusione, Abiurazione, Necromanzia, Invocazione, Trasmutazione, Evocazione, Ammaliamento (+0 alle prove di CM)\\


\bigskip


\subsubsection{Sefryn}

Aspetto: Gli Elementi\index{Sefryn}
\bigskip


I fedeli della dea Sefryn sono credenti nei quattro elementi generatori del mondo. Spesso di atteggiamento imprevedibile sanno ottimamente manipolare l'ambiente intorno a loro.


\begin{itemize}
	\item Tutto e' nato da Sefryn e tutto a lei torna
	\item Gli elementi sono purificatori ed ognuno e' sacro
\end{itemize}

Purtroppo solo questi due dogmi sono stati tramandati dal racconto del sacro libro, questo fa si che i fedeli di Sefryn siano tra i piu' liberi da vincoli etici.\\


\textbf{Vantaggi}: Resistenza ad un elemento, sceglilo. Ignori i primi 3 punti ferita di danno per round

\textbf{Essenza Favorita}: Difesa\\
\textbf{Essenza Sfavorita:} Illusione\\

\medskip

Accesso alla Scuole di Magia:\\
Scuole Privilegiate: Universale, Evocazione (+2 alle prove di CM)\\
Scuole Normali: Illusione, Abiurazione, Necromanzia, Invocazione, Trasmutazione, Divinazione, Ammaliamento (+0 alle prove di CM)\\


\bigskip

\subsubsection{Jaskara}

Aspetto: La Magia\index{Jaskara}
\bigskip


I fedeli della dea Jaskara si considerano gli unici veri maghi perche' considerano la loro dea la creatrice di tutte le Essenze

\begin{itemize}
	\item La forma piu' alta di potere e' la magia
	\item La creazione magica e' il sacrificio che Jaskara predilige
	\item Il mago ringrazi la Dea al plenilunio 
	\item La ricerca e' il cammino verso la magica jaskara
\end{itemize}


\textbf{Vantaggi}: Rilevare il magico\\
\textbf{Essenza Favorita}: una a scelta\\
\textbf{Essenza Sfavorita}: una a scelta\\


\medskip

Accesso alla Scuole di Magia:\\
Scuole Privilegiate: Universale, una a scelta (+2 alle prove di CM)\\
Scuole Normali: le rimanenti scuole di magia (+0 alle prove di CM)\\


\bigskip

\subsubsection{Ediath}

Aspetto: La Morte\index{Ediath}
\bigskip


I fedeli della dea della morte sono persone che e' meglio non incontrare se si e' anziani o se si hanno bevuto troppe pozioni di giovinezza. Dotati di un particolare codice di comportamento sono a volte spietati assasini o santi curatori, mah..., valli a capire...

\begin{itemize}
	\item La morte e' eterna e senza riposo
	\item Ediath coglie tutti
	\item La morte giunge al suo momento
	\item I non morti sono aberrazioni
\end{itemize}


Culto poco diffuso e poco pubblicizzato, presente, comunque in ogni maggiore citta'.\\


\textbf{Vantaggi}: Tocco gelido\\
\textbf{Essenza Favorita}: Distruzione\\
\textbf{Essenza Sfavorita:} Illusione\\

\medskip

Accesso alla Scuole di Magia:\\
Scuole Privilegiate: Universale, Necromanzia (+2 alle prove di CM)\\
Scuole Normali: Illusione, Abiurazione, Evocazione, Invocazione, Trasmutazione, Divinazione, Ammaliamento (+0 alle prove di CM)\\

\bigskip


\subsubsection{Xabax}

Aspetto: Il Chaos\index{Xabax}
\bigskip

I fedeli del dio Xabax sono persone molto particolari e difficilmente capibili.

Forse pessimisti o ottimisti per natura, predicano l'agnosticismo universale, correndo in contro, se cosi' gli va', a qualsiasi destino.


\begin{itemize}
	\item Nulla e' eterno ed immutabile
	\item La vita e' affidata alla sorte, di cui Xabax e' padrone, Il fedele confida nel Dio
	\item Xabax e' mutevole, cosi' il suo favore, il vero fedele sa di non sapere niente e prende la vita cosi' come viene
	\item Non puoi guardare oltre , perche' li' e' Xabax
\end{itemize}

Culto molto piccolo ma molto conosciuto per le sue particolari idee della vita.

Le feste degli accoliti di Xabax sono le piu' famose e gettonate e si protraggono per giorni e giorni.\\


\textbf{Vantaggi}: Senza paura\\
\textbf{Essenza Favorita}: Convocazione\\
\textbf{Essenza Sfavorita:} Rivelazione\\

\medskip

Accesso alla Scuole di Magia:\\
Scuole Privilegiate: Universale, Evocazione (+2 alle prove di CM)\\
Scuole Normali: Illusione, Abiurazione, Necromanzia, Invocazione, Trasmutazione, Divinazione, Ammaliamento (+0 alle prove di CM)\\


\bigskip

\subsubsection{Seeyek}

Aspetto: L'Avventura\index{Seeyek}
\bigskip

I fedeli di Seeyek sono avventurieri, cercatori di gloria e fama, amiconi e protettori del gruppo.

Ed al gruppo molto e' legato il suo credo e la loro forza.


\begin{itemize}
	\item Il vero fedele ama l'avventura e rispetta i compagni, poiche' grazie a loro potra' aumentare la sua esperienza
	\item Una torcia, una mazza e la fede, solo grazie a Seeyek si aggiunge esperienza
	\item improvvisare adattarsi e raggiungere lo scopo, cosi' il dio impone
\end{itemize}

Autentici self-made man. i fedeli di Seeyek sono fondamentalmente buoni e sanno essere validi compagni, fedeli ed onesti.\\


\textbf{Vantaggi}: Scudo mentale\\
\textbf{Essenza Favorita}: una a selta\\
\textbf{Essenza Sfavorita}: una a scelta\\

\medskip

Accesso alla Scuole di Magia:\\
Scuole Privilegiate: Universale, una a scelta (+2 alle prove di CM)\\
Scuole Normali: le rimanenti scuole di magia (+0 alle prove di CM)\\

\bigskip



\subsubsection{Daraka}

Aspetto: L'Oscurita\index{Daraka}
\bigskip

se Daern e' il culto positivo principale, quello della dea Daraka e' il malvagio piu' noto.

I fedeli di Daraka non fanno nascondimento del loro credo e con le loro scure vesti girano per le strade a cercare accoliti o ad eseguire le loro oscure trame.


I fedeli della Dea Daraka bramano il potere, con qualsiasi mezzo possa essere ottenuto

\begin{itemize}
	\item Diffondere il culto imponendo il terrore del Dio
	\item Potenziare il culto assurgendo a cariche di potere
	\item Solo nell'oscurita' la verita' esiste
	\item Nella notte senza stelle il sorriso di Daraka benedice i suoi discepoli
	\item Nella confusione Daraka appare, e tutto ora e' chiaro
\end{itemize}


E' l'unico culto, malvagio, che abbia pubbliche chiese nelle maggiori citta'.\\


\textbf{Vantaggi}: La mia ombra è mia amica\\
\textbf{Essenza Favorita}: Convocazione\\
\textbf{Essenza Sfavorita}: Trasformazione\\

\medskip

Accesso alla Scuole di Magia:\\
Scuole Privilegiate: Universale, Evocazione (+2 alle prove di CM)\\
Scuole Normali: Illusione, Abiurazione, Necromanzia, Invocazione, Trasmutazione, Divinazione, Ammaliamento (+0 alle prove di CM)\\

\bigskip

\subsubsection{Zxcvbnm}

Aspetto: L'Inganno\index{Zxcvbnm}
\bigskip


I fedeli della dea dell'inganno, non appaiono mai come tali: travestitismo, sotterfugio, inganno sono la loro vita e passione. abilissimi ladri e dotati di limitate capacita' illusionistiche, i fedeli di sua Signora degli Inganni sono i principali componenti e fondatori delle gilde dei ladri. Culto non strettamente malvagio e' presente nei bassifondi di tutte le citta'.

\begin{itemize}
	\item La realta' e illusione, l'illusione e' la realta', chi e' padrone delle illusioni e' padrone della realta'
	\item Un furto spettacolare o una truffa sono l'offerta' piu' gradita alla Dea
	\item Inganno, diplomazia, negoziato, spada: solo in quest'ordine agisce il fedele
\end{itemize}

Quindi quando incontrate un ladruncolo, state attenti potrebbe essere un fedele di SSDI, e potreste attirare l'attenzione di tutti gli altri devoti.\\

\textbf{Vantaggi}: Sensi protetti\\
\textbf{Essenza Favorita}: Illusione\\
\textbf{Essenza Sfavorita}: Difesa\\

\medskip

Accesso alla Scuole di Magia:\\
Scuole Privilegiate: Universale, Illusione (+2 alle prove di CM)\\
Scuole Normali: Evocazione, Abiurazione, Necromanzia, Invocazione, Trasmutazione, Divinazione, Ammaliamento (+0 alle prove di CM)\\

\bigskip

\bigskip


\subsubsection{Zarkor}

Aspetto: La Sofferenza\index{Zarkor}
\bigskip


I fedeli della dea Zarkor, sono dei veri e propri untori, si divertono a martoriare la gente, o diffondere malattie e sofferenze.

\begin{itemize}
	\item La dea gioisce della sofferenza di tutti gli esseri: infliggendo dolore il devoto si magnifica di fronte alla dea
	\item La corruzione della carne e della mente e' il normale stato della natura: portare corruzione, malattia e degrado e' la missione del fedele
	\item Nella malattia e nel dolore la mente si libera e si avvicina alla dea, infliggersi dolore e' grande offerta
\end{itemize}

L'unica cosa divertente di questo culto e vedere come siano cacciati da tutti gli altri fedeli, compresi i Koirani.\\


\textbf{Vantaggi}: Denti oppure Artigli\\
\textbf{Essenza Favorita}: Alterazione\\
\textbf{Essenza Sfavorita} Cura\\

\medskip

Accesso alla Scuole di Magia:\\
Scuole Privilegiate: Universale, Necromanzia (+2 alle prove di CM)\\
Scuole Normali: Evocazione, Abiurazione,  Illusione, Invocazione, Trasmutazione, Divinazione, Ammaliamento (+0 alle prove di CM)\\


\bigskip

\subsubsection{Eramide}

Aspetto: I Non Morti\index{Eramide}
\bigskip

I fedeli del dio Eramide sono i cultori della non vita, artisti nel riportare dall'oltretomba alla loro vita. dio molto amato da alcuni necromanti e' relativamente poco conosciuto e pregato.


\begin{itemize}
	\item Solo Eramide custodisce la chiave per l'ultimo volo
	\item Lui e' padrone della non vita, e abbraccia chi lui ama
	\item I non-morti sono le braccia del Dio, attraverso essi si raggiungera' la pace totale
	\item Lui e' portatore della vera vita, il fedele la anela
\end{itemize}

Fedeli dal tocco non-mortale ( che bella battuta...) sono pericolosi perche' spesso accompagnati dalle loro creature. \\


\textbf{Vantaggi}: Scudo Mentale\\
\textbf{Essenza Favorita}: Distruzione\\
\textbf{Essenza Sfavorita}: Creazione\\

\medskip

Accesso alla Scuole di Magia:\\
Scuole Privilegiate: Universale, Necromanzia (+2 alle prove di CM)\\
Scuole Normali: Evocazione, Abiurazione,  Illusione, Invocazione, Trasmutazione, Divinazione, Ammaliamento (+0 alle prove di CM)\\


\bigskip

\subsubsection{Averim}

Aspetto: La Bellezza\index{Averim}
\bigskip


I fedeli della dea Averim sono i belli, felici, puliti, puri...culto potente perché ammalia le menti delle persone, per fortuna poco diffuso e noto.

\begin{itemize}
	\item La bellezza e' potere
	\item Il fedele sa usare il potere per piegare al proprio volere chi dal potere e' soggiogato
	\item Amore e bellezza sono facce opposte di due medaglie diverse
	\item Il vero fedele sa usare ogni potere lui concesso per il piacere della Dea
\end{itemize}

Veri incantatori, meglio non guardarli negli occhi se non si vuole rimanere stregati e ridotti in schiavi.\\


\textbf{Vantaggi}: Senso della moda, Voce suadente\\
\textbf{Essenza Favorita}: Charme\\
\textbf{Essenza Sfavorita}: Convocazione\\

\medskip

Accesso alla Scuole di Magia:\\
Scuole Privilegiate: Universale, Ammaliamento (+2 alle prove di CM)\\
Scuole Normali: Evocazione, Abiurazione,  Illusione, Invocazione, Trasmutazione, Divinazione, Necromanzia (+0 alle prove di CM)\\


\bigskip

\subsubsection{Kyriel} 

Aspetto: La Natura\index{Kyriel}
\bigskip

I fedeli della dea Kyriel sono i protettori della Natura e di cio' che e' di Yeru. Molto del suo lavoro viene fatto direttamente nell'ambiente proteggendo la flora e la fauna.

\begin{itemize}
	\item La natura non e' crudele e' indifferente
	\item La natura non fa niente senza scopo
	\item Proteggi la natura e proteggerai te stesso
	\item Custodisci e proteggi la flora e la fauna
\end{itemize}

I fedeli sono per qualcuno selvaggi, per altri maestri di vita, per altri protettori di cio' che va protetto. I devoti di Kyriel possono essere stanziali e dedicati ad un luogo o girovaghi e dedicare la propria vita alla protezione della Natura.\\


\textbf{Vantaggi}: Animalia\\
\textbf{Essenza Favorita}: Trasformazione\\
\textbf{Essenza Sfavorita}: Distruzione\\

\medskip

Accesso alla Scuole di Magia:\\
Scuole Privilegiate: Universale, Evocazione (+2 alle prove di CM)\\
Scuole Normali: , Abiurazione, Illusione, Invocazione, Trasmutazione, Divinazione, Ammaliamento, Necromanzia (+0 alle prove di CM)\\

\bigskip

\subsubsection{Gli aspetti generici}\index{Divita' generici}

\medskip

Vengono anche presentati le divinità generiche, ovvero la possibilita' di dichiarare la propria fedelta' ad un aspetto generico, quale il Bene, il Male, la Legge, il Chaos.

Come per i Patroni anche gli aspetti generici concedono dei vantaggi e svantaggi alle Essenze, ogni Aspetto concede una Essenza privilegiata (+2 CM alle prove) e nega una Essenza (non utilizzabile).

Non e' necessario specificare la divinità adorata bensì solo se si e' fedele al Bene, Male, Chaos o Legge.

\medskip

\begin{tabular}{llll}
\toprule
\textbf{Divinta' Generica}	& \textbf{Vantaggio/Abilita' Concessa} & \textbf{Essenza Favorita} & \textbf{Essenza Negata} \\
Buona	& Incanalare Energia & Cura & Distruzione \\
Malvagia	&  Forgiato nella Furia &Distruzione  & Cura \\
Chaotica	&  Colpo Furtivo&Movimento  & Creazione \\
Legale & Colpi Poderosi & Creazione &  Movimento\\
\end{tabular}

\bigskip

In accordo con il Narratore, ed adeguatamente motivato, e' possibile cambiare Abilita' ed Essenze.

\medskip

Accesso alla Scuole di Magia:\\
Scuole Privilegiate: Universale, una a scelta (+2 alle prove di CM)\\
Scuole Normali: le rimanenti scuole di magia (+0 alle prove di CM)\\


\includepdf[pages={1},scale=0.95]{simboli-dei-antichi1.pdf}

\includepdf[pages={1},scale=0.95]{simboli-dei-antichi2.pdf}

\pagebreak
