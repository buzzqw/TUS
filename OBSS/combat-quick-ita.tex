\documentclass[10pt,a4paper,landscape]{article}
\usepackage[margin=0.5cm]{geometry}
\usepackage[utf8]{inputenc}
\usepackage[T1]{fontenc}
\usepackage[italian]{babel}
\usepackage{multicol}
\usepackage{booktabs}
\usepackage{tabularx}
\usepackage{array}
\usepackage{xcolor}
\usepackage{tcolorbox}
\usepackage{enumitem}
\usepackage{tikz}
\usetikzlibrary{positioning,shapes}
\usepackage{fancyhdr}
\usepackage{amsmath}

% Definizione colori
\definecolor{darkblue}{RGB}{44,62,80}
\definecolor{orange}{RGB}{243,156,18}
\definecolor{red}{RGB}{231,76,60}
\definecolor{blue}{RGB}{52,152,219}
\definecolor{gray}{RGB}{149,165,166}

% Configurazione tcolorbox
\tcbuselibrary{skins,breakable}

% Stile per le sezioni principali
\newtcolorbox{mainsection}[1]{
	enhanced,
	colback=darkblue!10,
	colframe=red,
	fonttitle=\bfseries\small,
	title=#1,
	boxrule=1pt,
	left=2pt,
	right=2pt,
	top=2pt,
	bottom=2pt
}

% Stile per le formule
\newtcolorbox{formula}{
	enhanced,
	colback=red!20,
	colframe=red,
	boxrule=1pt,
	left=3pt,
	right=3pt,
	top=3pt,
	bottom=3pt
}

% Comando per testo evidenziato
\newcommand{\highlight}[1]{\textcolor{orange}{\textbf{#1}}}
\newcommand{\critical}[1]{\textcolor{red}{\textbf{#1}}}
\newcommand{\info}[1]{\textcolor{blue}{\textbf{#1}}}

% Configurazione pagina
\pagestyle{fancy}
\fancyhf{}
\fancyhead[C]{\Large\textbf{\textcolor{red}{🗡️ SCHEDA RIASSUNTIVA COMBATTIMENTO - OBSS 🗡️}}}
\renewcommand{\headrulewidth}{2pt}
\renewcommand{\headrule}{\hbox to\headwidth{\color{red}\leaders\hrule height \headrulewidth\hfill}}

\setlength{\parindent}{0pt}
\setlength{\parskip}{2pt}
\setlength{\columnsep}{8pt}

\begin{document}

	\begin{multicols}{4}

		% SEQUENZA DI COMBATTIMENTO
		\begin{mainsection}{📋 SEQUENZA COMBATTIMENTO}
			\begin{center}
				\begin{tikzpicture}[node distance=1.3cm, auto]
					\node[draw, rectangle, fill=red!20, text width=5cm, text centered, font=\normalsize] (init) {1. Iniziativa\\3d6+Des/Int};
					\node[draw, rectangle, fill=orange!20, text width=5cm, text centered, font=\normalsize, below of=init] (decl) {2. Dichiarazione\\Dal più veloce};
					\node[draw, rectangle, fill=blue!20, text width=5cm, text centered, font=\normalsize, below of=decl] (res) {3. Risoluzione\\Azioni};
					\node[draw, rectangle, fill=gray!20, text width=5cm, text centered, font=\normalsize, below of=res] (next) {4. Round\\Successivo};

					\draw[->] (init) -- (decl);
					\draw[->] (decl) -- (res);
					\draw[->] (res) -- (next);
				\end{tikzpicture}
			\end{center}

			\textbf{Iniziativa Critica:}
			\begin{itemize}[noitemsep,leftmargin=8pt]
				\item Se +8: +1 Reazione/Immediata
				\item Se +16: +1 Azione totale
			\end{itemize}
		\end{mainsection}

		% AZIONI PER ROUND
		\begin{mainsection}{⚡ AZIONI PER ROUND}
			\begin{formula}
				\textbf{Disponibili ogni round:}\\
				• 3 Azioni Normali\\
				• 1 Azione Immediata\\
				• 1 Azione di Reazione\\
				• Azioni Gratuite illimitate
			\end{formula}

			\textbf{Ordine Azioni:} Qualsiasi, logicamente coerente

			\textbf{Interruzioni:} Solo Reazioni e Immediate possono interrompere
		\end{mainsection}

		% TIRO PER COLPIRE
		\begin{mainsection}{🎯 TIRO PER COLPIRE}
			\begin{formula}
				\textbf{Mischia:}\\
				3d6 + CA + FOR + Lista + Abilità + Magia + Circostanze
			\end{formula}

			\begin{formula}
				\textbf{Distanza:}\\
				3d6 + CA + DES + Lista + Abilità + Magia + Circostanze
			\end{formula}

			\textbf{Golden Rules:}
			\begin{itemize}[noitemsep,leftmargin=8pt]
				\item \highlight{6 = Esplode} (tira ancora)
				\item \critical{1 = Vale 0}
				\item \highlight{Affidarsi alla sorte:} -4 per +1d6
				\item \critical{3 volte 1 = Mancato automatico}
				\item \highlight{3 volte 6 = Colpisce sempre}
			\end{itemize}
		\end{mainsection}

		% DIFESA
		\begin{mainsection}{🛡️ DIFESA}
			\begin{formula}
				10 + DES + Scudo + Armatura + Magia + Abilità + Circostanze
			\end{formula}

			\begin{tabular}{@{}ll@{}}
				\toprule
				\textbf{Situazione} & \textbf{Modifica} \\
				\midrule
				Sorpreso & -2 Difesa \\
				Prono & -4 Difesa \\
				Affaticato (1/2/3) & -1/-2/-4 \\
				Afferrato & -2 Difesa \\
				Intralciato & -2 Difesa \\
				Bloccato & -4 Difesa \\
				Stordito & -4 Difesa \\
				Copertura L/M/C & +2/+4/+8 \\
				\bottomrule
			\end{tabular}
		\end{mainsection}

		\columnbreak

		% DANNI E CRITICI
		\begin{mainsection}{💥 DANNI E CRITICI}
			\begin{formula}
				\textbf{Danno Base:}\\
				Dado Arma + FOR + Lista + Abilità + Magia
			\end{formula}

			\textbf{Tiro Critico:}
			\begin{itemize}[noitemsep,leftmargin=8pt]
				\item \critical{Ogni +8 oltre Difesa} = +1 Critico
				\item Tira dado arma aggiuntivo (solo dado base)
				\item Non si cumula con esplosione danno
			\end{itemize}

			\textbf{Esplosione del Danno:}
			\begin{itemize}[noitemsep,leftmargin=8pt]
				\item Valore massimo dado: Ritira e somma
				\item Non esplode su: Critici, dadi <=d6
				\item EDX: Esplode su X o più
				\item Non esplode ricorsivamente
			\end{itemize}

			\textbf{Danno Minimo:} Sempre almeno 1 (dopo riduzioni)
		\end{mainsection}

		% ATTACCHI MULTIPLI
		\begin{mainsection}{⚔️ ATTACCHI MULTIPLI}
			\begin{tabular}{@{}cc@{}}
				\toprule
				\textbf{Attacco} & \textbf{Penalità} \\
				\midrule
				1 & +0 \\
				2 & -5 \\
				3 & -10 \\
				4 & -15 \\
				\bottomrule
			\end{tabular}

			\textbf{Due Armi:}
			\begin{itemize}[noitemsep,leftmargin=8pt]
				\item Arma secondaria = Attacco multiplo
				\item FOR dimezzata su secondaria
				\item Se non Leggera: \critical{-3 aggiuntivo}
				\item Può usare per +1 Difesa (no attacchi)
			\end{itemize}

			\textbf{Opzione Basso Livello:}\\
			CA < 6: -4 su entrambi invece di progressione
		\end{mainsection}

		% MOVIMENTO
		\begin{mainsection}{🏃 MOVIMENTO E DISTANZE}
			\begin{tabular}{@{}p{2cm}p{1cm}p{2.5cm}@{}}
				\toprule
				\textbf{Tipo} & \textbf{Costo} & \textbf{Distanza} \\
				\midrule
				Normale & 1 Az. & Mov. base \\
				Scatto & 1 Az. & 2x Mov. \\
				Terr. Diff. & - & 1/2 Mov. \\
				Diagonale & - & 1m/quad. \\
				\bottomrule
			\end{tabular}

			\textbf{Penalità Scatto:}
			\begin{itemize}[noitemsep,leftmargin=8pt]
				\item \critical{-1d6 Tiro per Colpire}
				\item \critical{-4 Difesa} (fino prossimo round)
				\item Distratto per incantesimi
			\end{itemize}

			\textbf{Distanze:}
			\begin{itemize}[noitemsep,leftmargin=8pt]
				\item \textbf{Tocco:} ≤ 1m (senza armi lunghe)
				\item \textbf{Mischia:} ≤ 1m (2m con armi lunghe)
				\item \textbf{Portata:} Metà Taglia occupato
			\end{itemize}
		\end{mainsection}

		\columnbreak

		% AZIONI PRINCIPALI
		\begin{mainsection}{📜 AZIONI PRINCIPALI}
			{\small
				\begin{tabular}{@{}p{2cm}cp{3cm}@{}}
					\toprule
					\textbf{Azione} & \textbf{Az.} & \textbf{Note} \\
					\midrule
					Attacco singolo & 1 & Un Tiro per Colpire \\
					Due attacchi & 2 & Secondo a -5 \\
					Tre+ attacchi & 3 & Penalità cumulative \\
					\midrule
					Movimento & 1 & Fino al massimo \\
					Scatto & 1 & 2x mov., penalità \\
					Carica & 2 & Mov.+att., +1d6 TC, -4 Dif \\
					\midrule
					Incantesimo & 2* & Varia per incantesimo \\
					Prep. Difesa & 1 & +1 Difesa \\
					Difesa Totale & 2 & +4 Dif., terr. difficile \\
					Disingaggiare & 1 & 1m senza provocare \\
					Colpo Preciso & 2 & Un attacco +1d4 TC \\
					\midrule
					Alzarsi Prono & 1 & Acrobatica DC13 (Imm.) \\
					Estrarre/Rinf. & 1 & Gratis con movimento \\
					Cercare Zaino & 2 & - \\
					Prend. Cintura & 1 & - \\
					\midrule
					Bere Pozione & I. & Se in mano \\
					Fare Bere & 2 & Ad un altro \\
					Salire/Scend. & 2 & Da cavalcatura \\
					\bottomrule
				\end{tabular}
			}
		\end{mainsection}

		% MANOVRE SPECIALI
		\begin{mainsection}{🤺 MANOVRE SPECIALI}
			\begin{tabular}{@{}p{1.7cm}p{0.8cm}p{3cm}@{}}
				\toprule
				\textbf{Manovra} & \textbf{Costo} & \textbf{Prova Opposta} \\
				\midrule
				Disarmare & 2 & CA+FOR/DES vs CA+FOR/DES \\
				Finta & 1 & CA+Ingan. vs CA+Perc \\
				Spingere & 1 & Atletica vs TS Tempra+FOR \\
				Afferrare & 2 & Atletica vs TS Tempra+FOR \\
				Far Cadere & 1 & Atletica vs TS Tempra+FOR \\
				Attraversare & 1 & Atleti/Acrob. vs TS Rifl \\
				\bottomrule
			\end{tabular}

			\textbf{Modificatori Taglia:}
			\begin{itemize}[noitemsep,leftmargin=8pt]
				\item +1d6 per taglia di vantaggio
				\item -1d6 per taglia di svantaggio
			\end{itemize}

			\textbf{Fallimento Critico:} Subisci l'effetto tu
		\end{mainsection}

		% ARMI A DISTANZA
		\begin{mainsection}{🏹 ARMI A DISTANZA}
			\begin{tabular}{@{}lc@{}}
				\toprule
				\textbf{Incremento} & \textbf{Penalità TC} \\
				\midrule
				1° (entro gittata) & +0 \\
				2° (gittata × 2) & -6 \\
				3° (gittata × 3) & -12 \\
				\bottomrule
			\end{tabular}
			\medskip
			\textbf{Sotto Minaccia:} -1d6 TC per armi a distanza

			\textbf{Contro Bersaglio in Combattimento:}
			\begin{itemize}[noitemsep,leftmargin=8pt]
				\item -2 TC aggiuntivo
				\item Copertura da altre creature
				\item Fallimento Critico: colpisci casualmente
			\end{itemize}

			\textbf{Forza al Danno:}
			\begin{itemize}[noitemsep,leftmargin=8pt]
				\item Archi compositi: Sì
				\item Archi normali: No
				\item Balestre: No
				\item Armi scagliate: Sì
			\end{itemize}
		\end{mainsection}

		\columnbreak

		% MODIFICATORI SITUAZIONALI
		\begin{mainsection}{🎲 MODIFICATORI SITUAZIONALI}
			\textbf{ATTACCANTE}
			\begin{tabular}{@{}p{3.5cm}c@{}}
				\toprule
				\textbf{Situazione} & \textbf{Mod} \\
				\midrule
				Fiancheggia & +2 \\
				Posizione Sopraelevata & +2 \\
				Attacco alle Spalle & +2 \\
				Invisibile & +1d6 \\
				Carica & +1d6 \\
				Avversario Indifeso & +1d6 \\
				Attacco a Tocco & +1d6 \\
				\midrule
				Prono & -4 \\
				Affaticato (1/2/3) & -1/-2/-3 \\
				Luce Fioca & -1 \\
				Ristretto & -1d6 \\
				Spaventato & -1d6 \\
				Arma Sconosciuta & -1d6 \\
				Bersaglio Invisibile & -1d6 \\
				Arma Lunga a <2m & -4 \\
				Attacco non letale & -4 \\
				\bottomrule
			\end{tabular}

			\vspace{2mm}

			\textbf{DIFENSORE}
			\begin{tabular}{@{}p{3.5cm}c@{}}
				\toprule
				\textbf{Situazione} & \textbf{Mod} \\
				\midrule
				Copertura Leggera & +2 \\
				Copertura Media & +4 \\
				Copertura Completa & +8 \\
				\midrule
				Sorpreso & -2 \\
				Prono & -4 \\
				Afferrato & -2 \\
				Intralciato & -2 \\
				Bloccato & -4 \\
				Stordito & -4 \\
				Affaticato (1/2/3) & -1/-2/-3 \\
				\bottomrule
			\end{tabular}
		\end{mainsection}

		% CONDIZIONI E STATI
		\begin{mainsection}{😵 CONDIZIONI COMUNI}
			\begin{tabular}{@{}p{2.2cm}p{3.4cm}@{}}
				\toprule
				\textbf{Condizione} & \textbf{Effetti} \\
				\midrule
				Prono & -4 TC e Dif in mischia \\
				Affaticato & -1/-2/-4 a TC e Dif \\
				Distratto & Prova Magia penaliz. \\
				Spaventato & -1d6 alle azioni \\
				Confuso & Azioni casuali \\
				Paralizzato & Indifeso, immobile \\
				Svenuto & Indifeso, incapace \\
				Morente & PF neg., -1 PF/round \\
				Accecato & Miss chance 50\% \\
				Assordato & -4 Iniziativa \\
				Nauseato & 1 azione max \\
				Intralciato & -2 TC, Dif, Des \\
				Afferrato & -2 Dif, Distratto \\
				Bloccato & -4 Dif, no movimento \\
				\bottomrule
			\end{tabular}
		\end{mainsection}

		% VITA E MORTE
		\begin{mainsection}{💀 VITA E MORTE}
			\begin{formula}
				\textbf{Stati di Salute:}\\
				• \highlight{PF > 0}: Normale\\
				• \highlight{PF = 0}: Svenuto\\
				• \critical{PF < 0}: Morente (-1 PF/round)\\
				• \critical{PF <= -(10+COS×2)}: Morto
			\end{formula}

			\textbf{Recupero da 0 PF:}
			\begin{itemize}[noitemsep,leftmargin=8pt]
				\item Cura magica = Cura PF
				\item Pronto Soccorso DC 12 = 1 PF
				\item Dopo 1h: TS Tempra DC 15 = 1 PF o -1 PF
			\end{itemize}

			\textbf{Recupero da Morente:}
			\begin{itemize}[noitemsep,leftmargin=8pt]
				\item Pronto Soccorso DC (12+PF neg.) = 0 PF
				\item Difficoltà +2 per volta successiva
				\item Cura magica = 1 PF
			\end{itemize}

			\textbf{Recupero Naturale:}
			\begin{itemize}[noitemsep,leftmargin=8pt]
				\item 8h riposo: COSxCA o COSxCM PF
				\item PF non letali: COS PF/ora
				\item PF Massimi: 1d4+COS per riposo
			\end{itemize}
		\end{mainsection}

		% REGOLE RAPIDE
		\begin{mainsection}{⚡ REGOLE RAPIDE}
			\textbf{Armi Lunghe:} Portata 2m, -4 TC sotto 2m

			\textbf{Armi Versatili:} DES invece FOR al TC

			\textbf{Controcarica:} Prepara vs carica (Reazione), poi att. gratuito con -1d6

			\medskip

			\textbf{Tempo:}
			\begin{itemize}[noitemsep,leftmargin=8pt]
				\item 1 Round = 10 secondi
				\item 1 Minuto = 6 round
				\item 1 Turno = 10 minuti
			\end{itemize}

			\textbf{Riattivazione:} Oggetti/Abilità "1/giorno" si ricaricano all'alba

			\medskip

			\textbf{Cavalcature:}
			\begin{itemize}[noitemsep,leftmargin=8pt]
				\item 2 Azioni, usa la tua iniziativa
				\item Se colpita: Cavalcare DC 15 o disarcionato
				\item +2 TC da posizione sopraelevata
			\end{itemize}
		\end{mainsection}

	\end{multicols}

	% FLOWCHART ATTACCO (parte inferiore)
	\begin{center}
		\begin{tcolorbox}[enhanced,colback=gray!10,colframe=red,title=🎯 FLOWCHART RISOLUZIONE ATTACCO,fonttitle=\bfseries]
			\begin{tikzpicture}[node distance=1.5cm, font=\footnotesize]
				\node[draw, rectangle, fill=blue!20, text width=2cm, align=center] (start) {Dichiarare\\Attacco};
				\node[draw, rectangle, fill=orange!20, text width=2cm, align=center, right=of start] (roll) {Tirare\\3d6 + Mod};
				\node[draw, diamond, fill=yellow!20, text width=2cm, align=center, right=of roll] (compare) {TC $\geq$ Difesa?};
				\node[draw, rectangle, fill=green!20, text width=2cm, align=center, below right=of compare] (hit) {Colpito!\\Tira Danno};
				\node[draw, rectangle, fill=red!20, text width=2cm, align=center, above right=of compare] (miss) {Mancato\\Fine};
				\node[draw, rectangle, fill=green!30, text width=2cm, align=center, right=of hit] (damage) {Applica\\Danno};
				\node[draw, diamond, fill=orange!30, text width=2cm, align=center, right=of damage] (crit) {Margine\\$\geq$ 8?};
				\node[draw, rectangle, fill=red!30, text width=2cm, align=center, below right=of crit] (critdmg) {Danno\\Critico!};

				\draw[->] (start) -- (roll);
				\draw[->] (roll) -- (compare);
				\draw[->] (compare) -- node[above] {No} (miss);
				\draw[->] (compare) -- node[below] {Sì} (hit);
				\draw[->] (hit) -- (damage);
				\draw[->] (damage) -- (crit);
				\draw[->] (crit) -- node[above] {No} ++(2,0) -- ++(0,2) -- (miss);
				\draw[->] (crit) -- node[below] {Sì} (critdmg);
				\draw[->] (critdmg) -- ++(2,0) -- ++(0,2) -- ++(0,2) -- ++(-8,0) -- (miss);
			\end{tikzpicture}
		\end{tcolorbox}
	\end{center}

\end{document}