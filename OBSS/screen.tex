\documentclass[a4paper,12 pt,openany]{book}
\usepackage[utf8]{inputenc}
\usepackage[T1]{fontenc}
\usepackage{amsmath}
\usepackage{amsfonts}
\usepackage{amssymb}
\usepackage{multicol}
\usepackage{graphicx}
\usepackage{xltabular}
\usepackage[a4paper]{geometry}
\geometry{verbose,tmargin=2cm,bmargin=2.5cm,lmargin=1.5cm,rmargin=2cm}
%\usepackage[absolute,overlay]{textpos}
\usepackage[absolute,overlay,showboxes]{textpos}
%\usepackage[width=457.86mm, height=303.28mm]{geometry}
\pagenumbering{gobble}
\begin{document}
	%	 \thispagestyle{empty}
	\center

	\begin{textblock*}{19 cm}(1cm,1cm) % larghezza box, coord X, coord Y
		\flushleft
		\textbf{Condizioni}\\
		\footnotesize

		\begin{multicols}{2}

\textbf{Accecato}:\index{Accecato} -2 prove su Forza e Destrezza.
Si muove a metà della velocita' . Prova di Acrobatica con DC 12 per muoversi più veloci, se fallisci sei Prono.
Chi attacca una creatura per lei invisibile ha un -2d6 al Tiro per Colpire, la creatura invisibile che attacca una creatura che non la vede ha +1d6 al Tiro per Colpire


\textbf{Affaticato}\index{Affaticato}: Non pui correre o Caricare e subisce una penalità -1 a Costituzione e Destrezza. Se compie qualsiasi cosa normalmente affaticante diventa Esausto.
Recuperi con 8 ore di riposo. Se non dormi almeno 6 ore alla mattina è affaticato.

\textbf{Afferrato}\index{Afferrato}: Un personaggio afferrato non può muoversi, deve usare due Azioni per liberarsi (TS Tempra contrapposto). Può attaccare con armi in mischia se adeguate (difficilmente potrà usare uno spadone, alabarda.. un pugnale o spada corta è più probabile).

\textbf{Annegare/Trattenere il fiato}: \index{Annegare/Trattenere il fiato} Qualsiasi personaggio può trattenere il fiato per un numero di round pari 6 round per il suo punteggio di Costituzione, con un minimo di 3 round. Per ogni Azione compiuta la durata restante diminuisce di 1 round. Trascorso questo periodo di tempo, il personaggio deve effettuare un Tiro Salvezza su Tempra con DC 12 ogni round per continuare a trattenere il fiato. Ogni round, la DC aumenta di 1.

\textbf{Assordato}:\index{Assordato}\index{Sordo} Subisce penalità -2 alle prove di Iniziativa, fallisce automaticamente tutte le prove di Consapevolezza basate sul suono e si considera Distratto nel lancio degli incantesimi con componenti almeno verbali.

\textbf{Charmato}:\index{Charmato} una creatura charmata tratta il giocatore con un fidato amico ed alleato. Se la creatura viene minacciata o attaccata può fare un nuovo Tiro Salvezza su Volontà con un +2.

\textbf{Confuso}: \index{Confuso} Tirate un dado sulla tabella seguente all'inizio di ogni round della creatura confusa ad ogni round per vedere quello che la creatura fa in quel round.

\textbf{d100 Comportamento:}

01-25 Agisce normalmente

26-50 Balbetta

51-75 Si infligge 1d8 + Forza con l'arma che tiene in mano

76-100 Attacca la creatura più vicina

\textbf{Esausto}:\index{Esausto} Un personaggio esausto si muove a velocità dimezzata e subisce penalità -2 a Costituzione e Destrezza. Dopo 1 ora di completo riposo (o Ristorare Inferiore), un personaggio esausto diventa solo Affaticato. Un personaggio Affaticato diventa Esausto compiendo una azione che normalmente lo affaticherebbe.
Affaticamenti successivi portano a : -1d6 ai TC e TS, Punti Ferita Massimi dimezzati, Morte\\

\textbf{Colpo di Grazia}:\index{Colpo di Grazia} Unica azione nel round. L'attaccante colpisce automaticamente ed infligge un colpo critico. Se il difensore sopravvive, deve superare un Tiro Salvezza su Tempra (DC 10 + danni inflitti) o muore.

\textbf{Intralciato}:\index{Intralciato} Un personaggio intralciato ha difficoltà di movimento, ma può comunque provare a muoversi, a meno che i legami che lo intralciano non siano ancorati a un oggetto immobile o impugnati da una forza contrapposta.
Una creatura intralciata può muoversi a velocità dimezzata ma non può Correre o Caricare, e subisce penalità -2 ai Tiri per Colpire e penalità -2 alle prove di Destrezza e Difesa.
Un personaggio intralciato che cerca di lanciare un incantesimo si considera Distratto.

\textbf{Invisibile}:\index{Invisibile} Le creature invisibili non sono percepibili dalla vista.Chi attacca una creatura per lei invisibile ha un -2d6 al Tiro per Colpire, la creatura invisibile che attacca una creatura che non la vede ha +1d6 al Tiro per Colpire

\textbf{Paralizzato}: \index{Paralizzato}Un personaggio paralizzato è bloccato sul posto ed è incapace di muoversi od agire. Ha punteggi effettivi di Forza e Destrezza pari a -4, è Indifeso e può compiere azioni esclusivamente mentali.
Una creatura può attraversare una zona occupata da una creatura paralizzata (o morta), che sia un alleato o meno e si considera come terreno difficile.

\textbf{Prono}\index{Prono}: chi è prono ha un -1d6 ad attaccare ed un -4 alla Difesa. Alzarsi da prono costa 2 Azioni.
Il personaggio può eseguire una prova di Acrobatica, se è pari o superiore a 15 ed entro 20 ti permette di dimezzare questi malus e costa una Azione alzarsi, se fai 20 o più annulli i malus e costa 1 Azione alzarsi. <10 rimani prono

\textbf{Paura}:\index{Paura} Incantesimi, Oggetti Magici e certe creature possono influenzare i personaggi con paura. In molti casi, il personaggio deve effettuare un Tiro Salvezza su Volontà per resistere agli effetti, e un tiro fallito indica che il personaggio è scosso, spaventato o in preda al panico.

\textbf{Scosso}:\index{Scosso}I personaggi che sono scossi subiscono penalità -2 ai Tiri per Colpire, ai Tiri Salvezza e alle prove.

\textbf{Spaventato}:\index{Spaventato} I personaggi spaventati sono anche scossi, e inoltre fuggono dalla fonte della loro paura il più velocemente possibile, anche se possono scegliere la direzione di fuga.
I personaggi che non sono in grado di fuggire possono combattere (anche se continuano ad essere scossi).

\textbf{In Preda al Panico}:\index{In Preda al Panico} I personaggi in preda al panico sono scossi e spaventati, inoltre, hanno una probabilità del 50\% di far cadere a terra qualsiasi cosa stanno tenendo in mano e di fuggire dalla fonte del loro terrore il più in fretta possibile seguendo un percorso di fuga completamente casuale.
I personaggi in preda al panico prendono anche la condizione Accovacciato se non possono fuggire.

\textbf{Sanguinante}\index{Sanguinante} Il sanguinamento può essere interrotto superando una prova di Pronto Soccorso con DC 15 o con l'uso di un incantesimo che curi ferite.
Se non indicato diversamente il danno massimo da sanguinamento, anche cumulato, è di 5 PF a round.
\end{multicols}

	\end{textblock*}


	\begin{textblock*}{19 cm}(1cm,24.3cm)
\textbf{Leggere una Pergamena}

\textbf{in caso di pergamene ISY SCROLL}: costo produzione livello*livello*160mo

- per comprendere il contenuto è sufficiente una prova di Arcana a difficoltà DC 10

- per poter lanciare l'incantesimo della pergamena è necessaria una prova di Intelligenza (o Arcana se conosciuta) a difficoltà 12.


\textbf{in caso di pergamene normali}: costo produzione livello*livello*80mo

- per comprenderne il contenuto è necessaria una prova di Arcana a difficoltà 15

- per poter lanciare l'incantesimo della pergamena è necessaria una prova di Arcana a difficoltà 20.
	\end{textblock*}

~\newpage

\begin{textblock*}{4cm}(1cm,1cm) % larghezza box, coord X, coord Y
	{\textbf{Punti Fato}\\
	(20-Livello)/5}
\end{textblock*}

	\begin{textblock*}{4cm}(1cm,2.2cm) % larghezza box, coord X, coord Y
		{\textbf{Morte}\\
			PF=-10-(COS*2)}
	\end{textblock*}

\begin{textblock*}{4cm}(1cm,3.3cm) % larghezza box, coord X, coord Y
\textbf{Copertura - Difesa}
Leggera +2 (>50\%)
Media +4 (<50\%)
Completa +8 (5\%)
	\end{textblock*}

\begin{textblock*}{4cm}(1cm,5.5cm) % larghezza box, coord X, coord Y

\textbf{Colpi Potenti}\\
+1 al danno - 1 TC. MAX CA/4

	\end{textblock*}

\begin{textblock*}{4cm}(1cm,7.2cm) % larghezza box, coord X, coord Y
\textbf{Maestria del combattimento}\\
+4 Difesa -1d6 al Tiro per Colpire\\
-4 Difesa +1d6 il Tiro per Colpire \\
Non è possibile assegnare in questa maniera più di +-2d6.
	\end{textblock*}


\begin{textblock*}{4cm}(1cm,12cm) % larghezza box, coord X, coord Y
\textbf{Carica}\\
3 Azioni. +1d6 a Tiro per Colpire, -4 alla Difesa,
\end{textblock*}

\begin{textblock*}{4cm}(1cm,14.2cm) % larghezza box, coord X, coord Y
	\textbf{Attacco di Opportunità}\\
In movimento esce o attraversa la zona di mischia.
Questo attacco è una Reazione che costa una Azione.
\end{textblock*}


\begin{textblock*}{8.5cm}(5.2cm,1cm) % larghezza box, coord X, coord Y
	\textbf{Rompere Oggetti - DC Forza}\\
	\begin{tabular}{ll|ll}
		Corda   		   & 23&	Porta semplic          & 13\\
		Porta legno        & 15&	Porta robusta          & 18\\
		Porta di ferro     & 28&	Catena                 & 26 \\
	\end{tabular}

\end{textblock*}


\begin{textblock*}{8.5cm}(5.2cm,3cm) % larghezza box, coord X, coord Y
	\footnotesize
	\begin{tabular}{lll}
\textbf{Difficoltà} & \textbf{Descrizione} & \textbf{Competenza} \\
DC 5           & Estremamente facile              & Mediocre                        \\
DC 10          & Facile                           & Normale                         \\
DC 15          & Normale                          & Buona                           \\
DC 20          & Difficile                        & Ottimo                          \\
DC 25          & Molto difficile                  & Eccellente                      \\
DC 30          & Estremamente difficile           & Stupefacente                    \\
DC 35          & Quasi impossibile                & Leggendaria                     \\
DC 40          & Leggendaria                      & Oltre l'umano                   \\
		\end{tabular}
	\end{textblock*}



\begin{textblock*}{6cm}(14cm,1cm) % larghezza box, coord X, coord Y
\textbf{Mod. al combattimento}

{\footnotesize \textbf{TC} +2: flanking, da sopra, alla schiena, arma lunga

+1d6: invisibile, carica

-2: abbagliato, intralciato

-1d6: prono, ristretto, spaventato, arma da lancio/lunga in mischia

\textbf{Difesa} +2/4/: copertura leggera, media, completa

-2: intralciato

-4: accecato, intrappolato, in ginocchio o seduto, prono, ristretto, stordito, lanci un incantesimo, afferrato}
\end{textblock*}


\begin{textblock*}{10.5cm}(5.2cm,7cm) % larghezza box, coord X, coord Y
\textbf{Azioni per Round}

	\begin{tabular}{ll}
		Eseguire un unico attacco con armi in mischia      & 1\\
		Eseguire due con armi in mischia			       & 2\\
		Eseguire più di due attacchi con armi in mischia  & 3\\
		Scoccare una freccia/dardo                         & 1\\
		Scoccare due frecce/dardo                          & 2\\
		Scoccare più di due frecce/dardo                  & 3\\
		Lanciare un'Incantesimo                            & 2\\
		Eseguire una Azione di Movimento*                  & 1\\
		Scatto   						                   & 1\\
		Alzarsi da prono                                   & 2\\
		Aiutare qualcuno                                   & 2\\
		Scambiare un discorso con qualcuno                 & 2\\
		Scambiare poche battute con qualcuno               & 0\\
		Cercare qualcosa nello zaino di pronto             & 2\\
		Usare qualcosa di appena preso dallo zaino/cintura & 1\\
		Bere una pozione tenuta alla cintura               & 1\\
		Estrarre l'arma (poi rimane estratta)              & 1\\
		Imbracciare lo scudo (poi rimane imbracciato)      & 1\\
		Usare un anello/bacchetta/verga/bastone magico     & 2\\
		Eseguire una prova su una competenza               & 2\\
		Nascondersi										   & 2\\
		Mantenere la concentrazione su un Incantesimo      & 1\\
		Salire o scendere dalla cavalcatura				   & 1\\
		Azione Immediata                                   & {*}\\
		Azione Reazione                                    & {*}\\
		Bere una pozione tenuta in mano     	           & I\\
		Fare cadere l'arma o lo scudo					   & R\\
		Gettarsi a terra prono							   & R\\
		Riconoscere un Incantesimo						   & R\\
	\end{tabular}

	\end{textblock*}

	\begin{textblock*}{4cm}(16cm,7cm) % larghezza box, coord X, coord Y
\textbf{Riposare 8 ore} \\fa recuperare COS+CA PF, minimo 1.
		\end{textblock*}


		\begin{textblock*}{4cm}(16cm,9cm) % larghezza box, coord X, coord Y
\textbf{Danni temporanei}\\ Ogni ora si recupera, con un minimo di 1 PF, il proprio valore di Costituzione in PF non letali (danni da stordimento) persi.
	\end{textblock*}


\begin{textblock*}{4cm}(16cm,12.6cm) % larghezza box, coord X, coord Y
\textbf{Difesa Sorpresi}\\NO Scudo, NO Destrezza
\end{textblock*}

\begin{textblock*}{4cm}(16cm,14.2cm) % larghezza box, coord X, coord Y
\textbf{Difesa Tocco}\\ NO Scudo, NO Armatura
\end{textblock*}



\begin{textblock*}{4cm}(16cm,14.2cm) % larghezza box, coord X, coord Y
	\textbf{Difesa Tocco}\\ NO Scudo, NO Armatura
\end{textblock*}


\begin{textblock*}{4cm}(16cm,15.7cm) % larghezza box, coord X, coord Y
\textbf{Tiro Critico}\\
Ogni qual volta hai colpito, tiri un dado arma aggiuntivo e non sommi altro per ogni due volte che hai tirato 6 nel Tiro per Colpire.
\end{textblock*}

\begin{textblock*}{4cm}(16cm,19.8cm) % larghezza box, coord X, coord Y
\textbf{Esplosione del Danno}\\
Se tiro dado e' massimo valore (min 8) ritiri il dado e sommi ancora il valore (del solo dado).
\end{textblock*}


\begin{textblock*}{4cm}(16cm,23.4cm) % larghezza box, coord X, coord Y
\textbf{Combattere in difensiva}

se esegui almeno una Azione di Attacco, +4 alla Difesa, -1d6 TC, costa 1 Azione.
\end{textblock*}


\begin{textblock*}{4cm}(16cm,26.5cm) % larghezza box, coord X, coord Y
	\textbf{Mettersi in difensiva}\\
usi una azioni, +2 Difesa fino a inizio round dopo.
\end{textblock*}

\begin{textblock*}{4cm}(1cm,18.4cm) % larghezza box, coord X, coord Y
	\textbf{Attacchi Multipli}\\
La prima azione di attacco non ha malus mentre la seconda azione di attacco ha -5 al colpire cumulativo per attacco
\end{textblock*}


\begin{textblock*}{4cm}(1cm,18.4cm) % larghezza box, coord X, coord Y
	\textbf{Attacchi Multipli}\\
	La prima azione di attacco non ha malus mentre la seconda azione di attacco ha -5 al colpire cumulativo per attacco
\end{textblock*}



%\begin{textblock*}{8cm}(1cm,22.3cm) % larghezza box, coord X, coord Y
%\textbf{Attacchi con armi a spargimento}\\
%1 2 3\\
%4 \textbf{X} 5\\
%6 7 8\\
%X si considera il bersaglio del tiro.\\
%Se il tiro manca di 5 tirate 2d6 ed un d8. 2d6 per determinare lungo la direzione indicata dal d8 a quanti metri è caduto distante dal bersaglio, ovvero contate i metri dal target.
%\end{textblock*}

\begin{textblock*}{8cm}(1cm,22.4cm) % larghezza box, coord X, coord Y
\textbf{Difesa totale}\\
2 Azioni. No Attacco, NO Incantesimi, puoi fare solo una Azione e guadagni un +4 in Difesa. Non causi Attacchi di Opportunità se attraversi la zona di mischia di un avversario.
\end{textblock*}

\begin{textblock*}{8cm}(1cm,25.4cm) % larghezza box, coord X, coord Y
\textbf{Disingaggiare}\\
2 Azioni. Sposti 3 metri. Non causi Attacchi di Opportunità se attraversi la zona di mischia di un avversario
\end{textblock*}



\begin{textblock*}{6.7cm}(9.1cm,22.4cm) % larghezza box, coord X, coord Y
\textbf{Azione di Scatto}\\
x2 Movimento. -1d6 nel Tiro per Colpire, -4 Difesa, Distratto
\end{textblock*}

\begin{textblock*}{6.7cm}(9.1cm,23.9cm) % larghezza box, coord X, coord Y
\textbf{Alzarsi da prono}\\
2 Azioni. -4 Difesa, -4 Iniziativa. Acrobatica , se è pari o superiore a 15 ed entro 20 ti permette di dimezzare questi malus e costa una Azione alzarsi,
se fai 20 o più annulli i malus e costa 1 Azione alzarsi.
\end{textblock*}



	~\newpage


	\begin{textblock*}{19.5cm}(1cm,1cm) % larghezza box, coord X, coord Y
		\footnotesize
		\begin{tabular}{lllll}
\textbf{Arma}&\textbf{Costo}&\textbf{Taglia/Danno} & \textbf{Gittata, Lista, Speciale} & Peso (kg)\\

Alabarda& 10 & G/1d10 P/T& \textbf{Lance}, \textbf{Aste}, Controcarica, Arma lunga, ED9 & 4\\
Arco Corto& 30 & M/1d6 P& 15 metri, \textbf{Arco}, da tiro& 1\\
Arco Corto Composito& note*& M/Frecce& 20 metri, \textbf{Arco}, da tiro& 1.5\\
Arco Lungo& 75 & G/Frecce& 20 metri, \textbf{Arco}, da tiro& 2\\
Arco Lungo Composito& note*& G/Frecce& 36 metri, \textbf{Arco}, da tiro& 2.5\\
Ascia ad una mano& 6  & M/1d6 T& 6 metri, \textbf{Asce}, \textbf{Armi da Tiro}, Versatile& 1\\
Ascia da battaglia& 10 & M/1d10 T&\textbf{Asce}& 3\\
Ascia Martello& 16 & M/1d6 T/B& \textbf{Asce}& 3\\
Balestra ad una mano& 100& M/Dardi& 6 metri, \textbf{Balestre}, da tiro& 1\\
Balestra leggera& 35 & P/Dardi& 15 metri, \textbf{Balestre}, \textbf{Armi Semplici}, da tiro& 0.5\\
Balestra pesante& 50 & G/Dardi& 20 metri \textbf{Balestre}, da tiro& 3\\
Bastone& 3& M/1d6 B& \textbf{Armi Semplici}, Arma lunga, Versatile& 2\\
Brandistocco& 10 & M/2d4 P/T& \textbf{Lance}, Controcarica, Arma lunga& 3\\
Catena chiodata& 25 & G/2d4 P& \textbf{Palle rotanti}, Arma lunga& 4\\
Falce& 18 & G/2d4 P/T& \textbf{Armi della Morte}, Arma lunga& 3\\
Falcetto& 6& P/1d6 T& \textbf{Armi della Morte} & 1\\
Falcione& 75 & M/2d4 T& \textbf{Armi Aggraziate}, \textbf{Lance}, ED7& 2\\
Falcione in asta& 12 & G/1d10 P/T& \textbf{Lance}, Controcarica, Arma lunga, ED9& 3\\
Fionda& -& P/1d4 B& 10 metri, \textbf{Archi}, da tiro& 0.5\\
Flagello& 8& M/1d8 B& \textbf{Palle Rotanti}, \textbf{Rompi Cranio}& 3\\
Flagello Doppio& 90 & M/1d10 B& \textbf{Palle Rotanti}, \textbf{Armi doppie} & 4\\
Flagello Pesante& 15 & M/1d10 B& \textbf{Palle Rotanti}, \textbf{Armi doppie}& 3\\
Frusta& 1& M/1d3 T& \textbf{Palle Rotanti}, Arma lunga& 2\\
Giavellotto& 1& P/1d6P& 12 metri, \textbf{Aste}, \textbf{Armi da tiro} \textbf{Armi Semplici}& 1.5\\
Grande Ascia Doppia& 25 & G/1d12 T& \textbf{Asce}, \textbf{Armi doppie}, Arma lunga& 4\\
Grosso randello& 2& M/1d8 B&\textbf{Rompi Cranio}& 2\\
Guanto chiodato& 5& P/1d4 P&\textbf{Armi da Stordimento}& 1\\
Katana& 300& M/1d10 T& \textbf{Spade}, \textbf{Armi letali}, ED9& 1.5\\
Lancia& 10 & G/1d8 P&\textbf{Lance}, Arma lunga, Controcarica& 3\\
Lancia corta da fante& 1& M/1d6 P& 6 metri, \textbf{Armi da tiro}, \textbf{Armi Semplici}, \textbf{Aste} & 1.5\\
Lancia da fante& 2& M/1d8 P&6 metri, \textbf{Lance}, Arma lunga, Controcarica& 2 \\
Machete& 10 & M/1d6 T&\textbf{Armi letali} & 1\\
Manganello& 1& P/1d6 B& \textbf{Armi da stordimento}, non letale& 0.5\\
Martello da guerra& 5& M/1d8 B/P& 6 metri, \textbf{Rompi Cranio}& 1.5\\
Mazza Leggera& 3& P/1d6 B/T& \textbf{Armi Leggere}, \textbf{Rompi Cranio}, \textbf{Armi Semplici}&1\\
Mazza Pesante& 5& M/1d8 B/T& \textbf{Rompi Cranio}& 2\\
Morningstar& 6& M 1d8 B/P&\textbf{Rompi Cranio},\textbf{ Armi Semplici}& 1\\
Naginata& 8& G/1d10 T&\textbf{Lance}, Arma lunga, ED9& 2\\
Picca Leggera& 4& M/1d4 P&\textbf{Armi della morte}& 1\\
Picca Pesante& 8& G/1d6 P&\textbf{Armi della morte}, Arma lunga& 3\\
Pugnale& 2& P/1d4 P& 6 metri, \textbf{Armi leggere}, \textbf{Armi da tiro}, \textbf{Armi Semplici}& 0.5\\
Pugno/Calcio nudo& note*& P/1d4 B&Versatile& -\\
Randello& 1& P/1d6 B&\textbf{Rompi Cranio}, \textbf{Armi Semplici}& 0.5\\
Scimitarra& 15 & M/1d6 T&\textbf{Armi Leggere}, \textbf{Armi Aggraziate}, Versatile& 1.5\\
Spada a due lame& 100& G/1d8 T& \textbf{Armi doppie}, \textbf{Spade}& 3\\
Spada bastarda& 35 & M/1d10 T&\textbf{Spade}& 2\\
Spada Corta& 10 & P/1d6 P&\textbf{Armi Leggere}, \textbf{Spade}, Versatile& 1\\
Spada Lunga& 15 & M/1d8 T&\textbf{Spade}& 1.5\\
Spadone a due mani& 50 & G/2d6 T&\textbf{Spade}& 3\\
Stocco& 20 & P/1d6 P& \textbf{Armi Leggere}, \textbf{Armi Aggraziate}, Versatile& 1\\
Tridente& 15 & M/1d6 P/T& 3 metri, \textbf{Aste}, \textbf{Armi da tiro}, Arma Lunga, Controcarica& 2\\
Urgrosh& 18 & M/1d6 T/P& \textbf{Lance}, \textbf{Armi doppie} & 3\\
\end{tabular}


	\end{textblock*}

	\begin{textblock*}{13cm}(1cm,24.5cm) % larghezza box, coord X, coord Y

		\footnotesize

\begin{tabular}{llll}
\textbf{Nome Proiettile}   & \textbf{Numero/Costo (mo)} & \textbf{Danno/Tipo} & Peso(kg) \\
Biglie di Marmo (fionde)   & 15/1 mo                    & 1d4 B               & 0.2      \\
Dardi da balestra, leggeri & 10/1 mo                    & 1d6 P               & 0.1      \\
Dardi per balestra pesante & 3/1 mo                     & 1d10 P              & 0.3      \\
Frecce da caccia           & 20/1 mo                    & 1d6 P               & 0.1      \\
Frecce da guerra           & 10/1 mo                    & 1d8 P               & 0.2      \\
Sasso (fionde)             & -                          & 1d2 B               & 0.2      \\
\end{tabular}

	\end{textblock*}

		~\newpage


	\begin{textblock*}{19cm}(1cm,1cm) % larghezza box, coord X, coord Y

	\begin{tabular}{llllllll}
		%\begin{xltabular}{0.95\textwidth}{lXXXXXXX}
		\textbf{Armatura} & \textbf{Costo (mo)} & \textbf{Difesa} & \textbf{Prove DES} &  \textbf{Tipo} & \textbf{Mov.} & \textbf{Prova Magia}&\textbf{Peso (kg)}\\
\hline
		Imbottita   & 5    & 1   & 0  &  L   & 0   & No&4\\
		Cuoio   & 10   & 2   & 0   & L   & 0   & SI&5\\
		Cuoio rinforzato   & 25  &3  & 0   &    L   & 0 &SI  & 6.5\\
		Giaco di Maglia   & 15   & 4  & -1  &  M   & 0  &+1d6 & 10\\
		Scaglie& 50   & 5  & -1  &  M   & 0 &+1d6  & 22.5\\
		Anelli & 150  & 6  & -1  &  M   & 0  &+1d6 & 20\\
		Pettorale    & 200  & 6  & -2  &  M  &  0 &+1d6  & 10\\
		Bande   & 250  & 7  & -2  &  P   & 0  &+2d6 & 30\\
		Mezza armatura   & 1200 & 8  & -2  &  P   & 1 &+2d6  & 20\\
		da Campo& 1400 & 9 & -3  &   P   & 2  &+2d6 & 25\\
		Completa& 1500 & 10  & -4  & P   & 3  &+2d6 & 32.5\\
	\end{tabular}
\end{textblock*}

		\begin{textblock*}{19cm}(1cm,7.3cm) % larghezza box, coord X, coord Y
\begin{tabular}{lcccccc}
	\textbf{Scudi} & \textbf{Costo} & \textbf{Bonus Difesa} & \textbf{Malus TC} & \textbf{Prova magia} & \textbf{Peso (kg)} & \textbf{Tipo}\\
\hline
	Brocchiero& 5 mo  &  0& 1& SI&1  & L\\
	Scudo leggero di legno   & 3 mo  &  0& 2& SI&2  & L\\
	Scudo leggero di metallo & 9  mo  &  0& 3&SI& 3  & L\\
	Scudo medio legno   & 5 mo   &  -1& 4& +1d6&3  & M\\
	Scudo medio metallo & 12 mo  & -1  & 5& +1d6&5  & M\\
	Scudo pesante di legno   & 7  mo  &  -2    & 6& +2d6&5  & P\\
	Scudo pesante di metallo & 20 mo  &  -2    & 7& +2d6&7  & P\\
\end{tabular}
\end{textblock*}


	\begin{textblock*}{19cm}(1cm,11.7cm) % larghezza box, coord X, coord Y
\textbf{Tempi per indossare e togliere l'armatura}\index{Tabella Tempi per indossare e togliere l'armatura}\\

\begin{tabular}{llll}
	\textbf{Tipo di Armatura}& \textbf{Indossare} & \textbf{Indossare in fretta} & \textbf{Togliere}\\
	Scudo								& 1 azione 	& -     	& 1 azione\\
	Imbottita, Cuoio, Cuoio rinforzata  & 1 minuto	& 3 round  	& - \\
	Giaco di Maglia						& 1 minuto	& 5 round  & 5 round\\
	Scaglie, Anelli, Pettorale, Bande   & 4 minuti 	& 1 minuto{*}  & 1 minuto\\
	Mezza armatura, da Campo, Completa  & 4 minuti{*}{*}& 4 minuti{*}& 1d4+1 minuti\\
\end{tabular}

\end{textblock*}

	\begin{textblock*}{6cm}(1cm,15.5cm) % larghezza box, coord X, coord Y

\begin{tabular}{ll}
	\textbf{Oggetto}&\textbf{Costo}\\
	\textbf{Birra}&\\
	Boccale&4 mr\\
	Caraffa (4 litri)&2 ma\\
	\textbf{Pietanze} &\\
	Banchetto (a persona)&10 mo\\
	Carne, 1 pezzo&3 ma\\
	Formaggio, 1 pezzo&1 ma\\
	Pane (a pagnotta)&2 mr\\
	\textbf{Locanda (al giorno})&\\
	Squallida&7 mr\\
	Povera&1 ma\\
	Modesta&5 ma\\
	Agiata&8 ma\\
	Ricca&2 mo\\
	Aristocratica&4 mo\\
	\textbf{Pasto (al giorno)}&\\
	Squallido&3 mr\\
	Povero&6 mr\\
	Modesto&3 ma\\
	Agiato&5 ma\\
	Ricco&8 ma\\
	Aristocratico&2 mo\\
	\textbf{Vino}&\\
	Buono (bottiglia)&10 mo\\
	Comune (caraffa)&2 ma\\
\end{tabular}

\end{textblock*}

	\begin{textblock*}{12.5cm}(7.5cm,15.5cm) % larghezza box, coord X, coord Y

\begin{tabular}{lllll}
\hline
	\textbf{Cavalcatura}&\textbf{Costo}&\textbf{Mov.}&\textbf{Carico}&Km/h\\
	&(\textbf{mo})&&&\\
	Asino o Mulo&8&12 m&210 kg&6km\\
	Cammello&50&15 m&240 kg&8km\\
	Cavallo da Galoppo&75&18 m&240 kg&12km\\
	Cavallo da Guerra&400&18 m&270 kg&9km\\
	Cavallo da Tiro&50&12 m&270 kg&6km\\
	Elefante&200&12 m&660 kg&6km\\
	Mastino&25&12 m&97,5 kg&6km\\
	Pony&30&12 m&112,5 kg&6km\\
	Carretto/Carro    & 15/30 mo & 9/12 m   &150/600kg    & 3/6km              \\
\end{tabular}

\end{textblock*}

	\begin{textblock*}{12.5cm}(7.5cm,21.3cm) % larghezza box, coord X, coord Y

\begin{tabular}{ll}

	\textbf{Contenitore}&\textbf{Capienza}\\
	Ampolla o Boccale&0,5 litri liquidi\\
	Barile&			160 litri liquidi, 4 cubi di 30 cm\\
	Borsa&			1 cubo di 10 cm/3 kg di oggetti\\
	Bottiglia&		1 litro di liquido\\
	Brocca o Caraffa&4 litri liquidi\\
	Canestro&		2 cubi di 30 cm/20 kg di oggetti\\
	Fiala&			120 ml di liquidi\\
	Forziere&		12 cubi di 30 cm/150 kg di oggetti\\
	Otre&			2 litri liquidi\\
	Sacco&			1 cubo di 30 cm/15 kg di oggetti\\
	Secchio&		12 litri liquidi, 1 cubo di 25 cm\\
	Vaso di Ferro&	4 litri liquidi\\
	Zaino*&			1 cubo di 30 cm/15 kg di oggetti\\
\end{tabular}
\end{textblock*}



	~\newpage

		\begin{textblock*}{4cm}(1cm,1cm) % larghezza box, coord X, coord Y
		{\textbf{Competenze}\\
			\footnotesize
			\textbf{Forza}\\
			Arrampicarsi\\
			Intimidire\\
			Nuotare\\
			Saltare	\\
			\textbf{Destrezza}\\
			Acrobatica\\
			Artista della fuga\\
			Giocoliere\\
			Mani di fata\\
			Muoversi silenziosamente\\
			Nascondersi nelle ombre\\
			Usare corda	\\
			\textbf{Intelligenza}\\
			Arcana\\
			Conoscenza*\\
			Dungeon\\
			Erboristeria\\
			Disattivare congegni\\
			Falsificare\\
			Lingue\\
			Natura magica\\
			Valutare\\
			\textbf{Saggezza}\\
			Cavalcare\\
			Consapevolezza\\
			Gestire animali\\
			Natura\\
			Orientamento\\
			Percepire Emozioni\\
			Pronto soccorso\\
			Religione\\
			Seguire tracce\\
			Sopravvivenza\\
			\textbf{Carisma}\\
			Diplomazia\\
			Intrattenere\\
			Ingannare\\
			Suonare\\
			Tradizioni locali
		}

	\end{textblock*}

	\begin{textblock*}{4cm}(1cm,18.5cm) % larghezza box, coord X, coord Y
\textbf{Riconoscere un incantesimo}\\ Arcana DC 10 + livello dell'incantesimo

	\end{textblock*}

	\begin{textblock*}{14.5cm}(5.5cm,1cm) % larghezza box, coord X, coord Y
	\textbf{Golen Rules}\\

	{\textbf{I 6 esplodono}} - se fai 6, sommi e ritiri\\
	Gli \textbf{1 portano male}, se fai 1 con il dado vale zero\\
	\textbf{Affidarsi alla sorte}. -4 punti di competenza/caratteristica = +1d6\\
\end{textblock*}

	\begin{textblock*}{7cm}(5.5cm,3.2cm) % larghezza box, coord X, coord Y
	\textbf{Pronto Soccorso}\\
3 Azioni: DC 15 recuperi 1d4 PF\\
+2 TS Tempra Veleno\\
DC 15+3xSanguinamento -1 Sanguinamento
\end{textblock*}

\begin{textblock*}{7cm}(13cm,3.2cm) % larghezza box, coord X, coord Y
\textbf{Identificare  Pozioni}\\
Erboristeria a DC 12 + fattore di rarità della pianta
\end{textblock*}

\begin{textblock*}{7cm}(13cm,5cm) % larghezza box, coord X, coord Y
\textbf{Riconoscere oggetto magico}\\
Una prova di Arcana a Difficoltà 25 può dare indicazioni di massima sui poteri e ambiti di utilizzo.
\end{textblock*}

{\small
	\begin{textblock*}{7cm}(5.5cm,5.8cm) % larghezza box, coord X, coord Y
\textbf{Intimidire}\\
2 Azioni e fa la prova su
Intimidire contrapposta al Tiro Salvezza su
Volontà con bonus dato dal Carisma
dell’avversario. Se il Tiro Salvezza fallisce,
l’avversario fino alla fine del round successivo ha
-1d6 al Tiro per Colpire e -2 alla Difesa contro
quell’avversario soltanto.\\
Se la prova di Intimidire fallisce in maniera
critica (l’avversario riesce di 10 o più il Tiro
Salvezza chi ha fatto la prova deve effettuare un
Tiro Salvezza su Volontà con modificatore
Carisma a DC 10+Grado di Sfida dell’avversario
o subire le medesime penalita’ come se fosse
stato intimidito. Se il tiro contrapposto riesce in
maniera critica (l’avversario fallisce di più di 10 il
Tiro Salvezza) la durata dell’effetto permane fino
a fine combattimento.
\end{textblock*}}


\begin{textblock*}{7cm}(13cm,7.3cm) % larghezza box, coord X, coord Y
\textbf{Saltare}\\
\textit{Si hanno penalità dovuta all'Armatura}

\begin{tabular}{ll}
	\textbf{Salto in Lungo (Distanza)} & DC\\
	1.5 m                              & 5  \\
	3 m                                & 10\\
	5 m                                & 15 \\
	7 m                                & 20 \\
	+1,5 m                             & +5	\\
\end{tabular}

\begin{tabular}{ll}
\textbf{Salto in Alto (Altezza)} & DC\\
	0.02 m                           & 4\\
	0.5 m                            & 8\\
	1 m                              & 12\\
	1.5 m                            & 16\\
	+0.5 m                           & +4\\
\end{tabular}

In un \textbf{salto in lungo} la punta più alta del salto è pari ad un 1/4 della lunghezza saltata. Se esegui un salto in lungo di 4 metri a metà salto sei in alto di 1 metro.

\end{textblock*}

\begin{textblock*}{7cm}(13cm,17.1cm) % larghezza box, coord X, coord Y
\textbf{Nuotare}\\
\textit{Penalità dovuta all'Armatura su Forza}\\
Acqua calme DC 10.\\
Acque mosse ha DC 15\\
Acque tempestose DC 20
\end{textblock*}

\begin{textblock*}{7cm}(13cm,19.8cm) % larghezza box, coord X, coord Y

\begin{tabular}{lll}
\textbf{Fonti di Luce} & Durata&Raggio\\
Torcia& 1 ora & 6m\\
Lanterna & 6 ore & 9m\\
\end{tabular}

\end{textblock*}

\begin{textblock*}{9cm}(1cm,22.3cm) % larghezza box, coord X, coord Y

\textbf{Arrampicarsi}\\
\textit{Si hanno penalità dovuta all'Armatura}

\begin{tabular}{ll}
	\textbf{Esempio di Superficie} & \textbf{DC}\\
	Grezza con appigli, mattoni sporgenti&10\\
	Albero, una corda senza nodi&15\\
	Parete liscia con appigli &20\\
	Muro perimetrale pochissimi appigli&25\\
	Parete naturale senza appigli&30\\
	Appoggiare a 2 pareti opposte&-10\\
	Appoggiare a 2 parete angolari&-5\\
	Superficie scivolosa&+5\\
\end{tabular}

\end{textblock*}


	\begin{textblock*}{8cm}(10.5cm,22.3cm) % larghezza box, coord X, coord Y
\textbf{Sopravvivenza}
\begin{tabular}{ll}
	Difficoltà base  & DC 10\\
	Se il terreno è molto morbido& DC +5\\
	Se il terreno è morbido& DC +10\\
	Se il terreno è stabile& DC +15\\
	Se il terreno è duro& DC +20\\
	Ogni 3 creature inseguite& DC -1\\
	A seconda della taglia& DC +-8\\
	Ogni 24 ore passate&DC +2\\
	Ogni ora di pioggia&DC +4\\
	Visibilità scarsa&DC +2\\
	Cerca di occultare le traccie&DC +5\\
\end{tabular}\\
\end{textblock*}


\begin{textblock*}{7cm}(5.5cm,16cm) % larghezza box, coord X, coord Y
	\textbf{Riconoscere un mostro}\\
	Arcana: Giganti, Costrutti, Spiriti, Mostruosità\\
	Aberrazioni, Draghi\\
	Piani: Elementali\\
	Occulto: Immondi, Spiriti, Non Morti\\
	Religione: Spiriti, Non Morti, Celestiali\\
	Dungeon: Aberrazioni, Mostruosità, Melme, creature sotterranee\\
	Natura: Bestie, Piante, Fatati\\
	DC = Grado di Sfida + 10
\end{textblock*}


	~\newpage

\begin{textblock*}{5cm}(1cm,1cm) % larghezza box, coord X, coord Y
		\textbf{Prova di Magia}\\
3d6 + 1d6*(1/4 Comp. Magica)\\
Fallimento Critico: due volte 1, un 1 e due 2\\
	\end{textblock*}

\begin{textblock*}{5cm}(1cm,3.6cm) % larghezza box, coord X, coord Y
	\textbf{Distratto}\\
Sei Distratto se: occulti il lancio di incantesimo, Impedito, disturbato, sanguinante, sotto attacco (-4 Difesa)-
Se fallimento perdi metà Punti Magia. No effetti negativi.\\
\end{textblock*}

\begin{textblock*}{13.5cm}(6.5cm,1cm) % larghezza box, coord X, coord Y
	\textbf{Seguace}\\
2 Tratti comuni con Patrono. Se sei un Seguace ottieni +1d6 alle Prove di Magia nella scuola preferita dal Patrono. Puoi usare l'energia preferita del Patrono nei tuoi incantesimi.\\
\end{textblock*}

\begin{textblock*}{13.5cm}(6.5cm,3.2cm) % larghezza box, coord X, coord Y
	\textbf{Devoto}\\
3 Tratti in comune con Patrono. Un Devoto aggiunge +1d6 alla Prova di Magia nelle scuole preferite dal Patrono e ignora un 1 tirato nella Prova di Magia. Devi usare l'energia preferita del Patrono nei tuoi incantesimi.
\end{textblock*}

\begin{textblock*}{13.5cm}(6.5cm,5.5cm) % larghezza box, coord X, coord Y
{\footnotesize
\textbf{Fallimento Critico Prova di Magia - 3d6 -1d6 x Fallimento Crit. Min. 1d6}
\begin{tabularx}{0.95\textwidth}{lX}
1 & Per tre giorni non sei piu' in grado di canalizzare energie magiche. Non puoi lanciare incantesimi se non facendo un successo magico critico nella Prova di Magia\\
2 & Subisci 30 Punti Ferita di danno e dimezzi i Punti Magia rimanenti.\\
3 & Vieni investito da una roboante colonna di fuoco. In un raggio di 3 metri intorno a te compreso, chiunque deve fare un Tiro Salvezza su Riflessi DC 18 per dimezzare o subire 1d4 di danni per livello di incantesimo\\
4 & Per un minuto sei sotto l'influenza dell'incantesimo Confusione\\
5 & Sei paralizzato per 1 minuto\\
6 & Non riesci a parlare bene, sei balbuziente. Ogni lancio di incantesimi ti costringe ad effettuare una Prova di Magia. Durata 1 minuto.\\
7 & Diventi Invisibile ed incapace di parlare per 1 minuto\\
8 & Vieni avvolto solo tu da una cortina di oscurità magica per 1 minuto\\
9 & Il battito del tuo cuore è come il battito di un tamburo, si puo sentire entro 50 metri\\
10 & Il prossimo incantesimo che lanci ha effetti se possibile minimizzati\\
11 & Perdi 10 Punti Magia e 10 Punti Ferita\\
12 & Emetti una rumorosa e pestilenziale flatulenza. Una insegna luminosa di 1m x 50cm sopra la tua testa ti indica e ti sbeffeggia.\\
13 & Ti cadono tutti i peli del corpo, per fortuna possono ricrescere\\
14 & Ogni oggetto che tieni in mano ti cade a terra\\
15 & I tuoi vestiti non magici scompaiono\\
16 & Guadagni 10 Punti Magia\\
17 & Una incudine cade, 3d6 di danno Tiro Salvezza su Riflessi DC 15 per dimezzare, su una creatura a caso, escluso te, entro sei metri\\
18 & Tutte le creature, escluso te, nel raggio di 3 metri da te subiscono 1d10 danni da Forza\\
\end{tabularx}}
\end{textblock*}
{\footnotesize
\begin{textblock*}{5cm}(1cm,7.8cm) % larghezza box, coord X, coord Y
\textbf{Punti Magia}\\
Mod. Caratteristica + \\
\begin{tabular}{ll|ll}
\textbf{CM} & \textbf{P.M}&	\textbf{CM} & \textbf{P.M}\\
1 &5 &2&8 \\
3&11&4&14\\
5&17&6&21\\
7&26&8&34\\
9&42&10&51\\
11&61&12&72\\
13&84&14&97\\
15&111&16&116\\
17&132&18&149\\
19&167&20&186\\
20+&+19&&\\
	\hline
\end{tabular}
\end{textblock*} }

\begin{textblock*}{5cm}(1cm,13.8cm) % larghezza box, coord X, coord Y
\textbf{Tiro Salvezza Incantesimo}\\
DC = 10 + livello incantesimo + modificatore caratteristica per incantesimo + 2 x Abilità prese in quella Liste di Magia +2 x Critico nella Prova di Magia
\end{textblock*}

\begin{textblock*}{5cm}(1cm,18.4cm) % larghezza box, coord X, coord Y
\textbf{Tiro Salvezza Magia da Oggetto}\\
DC = 10 + 2 x livello incantesimo manifestato
\end{textblock*}

\begin{textblock*}{5cm}(1cm,20.5cm) % larghezza box, coord X, coord Y
	\textbf{Tiro Salvezza Incantesimo Mostro}\\
DC è 10 + 2 x livello incantesimo + Intelligenza
\end{textblock*}

\begin{textblock*}{5cm}(1cm,22.7cm) % larghezza box, coord X, coord Y
{\small
\textbf{Quando si hanno < 50\& Punti Magia} ogni incantesimo deve essere fatto con una Prova di Magia.}
\end{textblock*}

\begin{textblock*}{5cm}(1cm,24.8cm) % larghezza box, coord X, coord Y
{\small
\textbf{Successo Critico Automatico}:  x2 costo Punti Magia cumulativo. Es. 4,8,16,32..}
\end{textblock*}

\begin{textblock*}{5cm}(1cm,26.9cm) % larghezza box, coord X, coord Y
{\small\textbf{Lanciare più volte il medesimo incantesimo} il costo aumento di se stesso. 4,8,12.. Non con Trucchetti}
\end{textblock*}

\begin{textblock*}{13.5cm}(6.5cm,17.4cm) % larghezza box, coord X, coord Y

\textbf{Livello massimo incantesimi}
{\small
		\begin{tabular}{l|lllll|ll}

			CM&\multicolumn{5}{c}{Adepto della magia}&Valore&Livello Inc.\\
			&0&1&2&3&4&Caratt.&Max\\
			\hline
			1		&  \checkmark	  & \checkmark 	& \checkmark & \checkmark & \checkmark& 0&0\\
			2-3     &  &  	 \checkmark 	& \checkmark & \checkmark & \checkmark & 1 &1-2\\
			4-6 	&  & &	 \checkmark 	& \checkmark & \checkmark & 2 &3-4\\
			7-10 	&  & & & \checkmark & \checkmark & 3 &5-6\\
			11-14 	&  & & &   & \checkmark & 4 &7-8\\
			15+ 	&  & & & & \checkmark& 5 &9\\
	\end{tabular}
}
\end{textblock*}
{\small
\begin{textblock*}{13.5cm}(6.5cm,21.7cm) % larghezza box, coord X, coord Y
\textbf{Alterare la Magia}

	\textbf{Magia efficace}: Incantatore o compagno Ignora un 1,2 tirato nella Prova di Magia ogni 4 punti ferita massimi sacrificati.

	\textbf{Magia eterea}: aumentando di 3 i Punti Magia spesi nell'incantesimo le proprie magie hanno pieno effetto su creature eteree o incorporee

	\textbf{Magia pietosa}: aumentando di 3 i Punti Magia spesi le magie infliggono danni temporanei.

	\textbf{Aumentare il tempo} di lancio da 2 Azioni a 1 round -1 in Punti Magia

	\textbf{Magie collaborative}: un altro mago costo metà Punti Magia concede +1d6 alla Prova di Magia del compagno.

	\textbf{Circolo del Potere}: tutti Devoti o Seguaci dello stesso Patrono
	Per mago costo la metà dei Punti Magia dell'incantesimo lanciato dal compagno +1d6 alla Prova di Magia del compagno, max +7d6. Tempo lancio almeno 1 turno
\end{textblock*}}

	 	~\newpage



\end{document}