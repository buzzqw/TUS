\documentclass[a4paper,12 pt,openany]{book}
\usepackage[utf8]{inputenc}
\usepackage[T1]{fontenc}
\usepackage{amsmath}
\usepackage{amsfonts}
\usepackage{amssymb}
\usepackage{multicol}
\usepackage{graphicx}
\usepackage{xltabular}
\usepackage[a4paper]{geometry}
\geometry{verbose,tmargin=2cm,bmargin=2.5cm,lmargin=1.5cm,rmargin=2cm}
%\usepackage[absolute,overlay]{textpos}
\usepackage[absolute,overlay,showboxes]{textpos}
%\usepackage[width=457.86mm, height=303.28mm]{geometry}
\pagenumbering{gobble}

\newcommand{\riga}{\rule{\textwidth}{0.4pt}}

\begin{document}


\center

\footnotesize

\begin{textblock*}{19 cm}(1cm,1cm) % larghezza box, coord X, coord Y
\flushleft

\begin{multicols}{2}

\textbf{Accecato}:\index{Accecato} Il personaggio non riesce a vedere nulla. -2 Competenze basate su Forza e Destrezza. Le prove o le attività basate sulla visione falliscono automaticamente. Tutti gli avversari vengono considerati dotati di invisibilità nei confronti del personaggio accecato.  Chi attacca una creatura per lei invisibile ha un -1d6 al Tiro per Colpire, la creatura invisibile che attacca una creatura che non la vede ha +1d6 al Tiro per Colpire

\textbf{Affascinato}:\index{Affascinato}: Una creatura affascinata non può attaccare o bersagliare chi l’ha affascinata. Qualsiasi minaccia nuovo Tiro Salvezza, se minaccia palese e concreta interrompe automaticamente l'effetto. Un alleato della creatura affascinata può scuoterla per permettergli un nuovo Tiro Salvezza spendendo 2 Azioni. L'affascinatore ha +1d6 su qualsiasi prova di caratteristica per interagire socialmente con la creatura.

\textbf{Affaticato}\index{Affaticato}: Non può correre o Caricare e subisce una penalità -2 Tiro per Colpire e Difesa.

\textbf{Afferrato}\index{Afferrato}: Un personaggio afferrato non può muoversi ma può provare a Spingere. Deve usare due Azioni per liberarsi (TS Tempra contrapposto su Forza +1d6 per Taglia). Perde il bonus della Destrezza alla Difesa ed ai Tiri Salvezza su Riflessi. Può attaccare con armi in mischia se adeguate.

\textbf{Annegare/Trattenere il fiato}: \index{Annegare}\index{Trattenere il fiato} Trattieni l fiato pari 6 round per Costituzione, minimo 3 round. Per ogni Azione compiuta la -1 round alla  . Finta aria Tiro Salvezza su Tempra con DC 12 ogni round per continuare a trattenere il fiato. Ogni round, la DC aumenta di 1.

\textbf{Assordato}:\index{Assordato}\index{Sordo} Fallisce automaticamente tutte le prove di Consapevolezza basate sul suono e si considera Distratto nel lancio degli incantesimi che abbiano componenti verbali.

\textbf{Bloccato}\index{Bloccato}: una creatura bloccata ha la braccia bloccate. Può spostarsi provando a Spingere, deve usare due Azioni per liberarsi (TS Tempra contrapposto). Perde il bonus della Destrezza alla Difesa ed ai Tiri Salvezza su Riflessi. Un incantatore Bloccato deve effettuare una Prova di Magia con successo critico magico o non riesce a formulare magie. considera Distratto.

\textbf{Confuso}: \index{Confuso}

\textbf{d100 Comportamento:}

01-25 Agisce normalmente

26-50 Non fa altro che balbettare in modo incoerente

51-75 Si infligge 1d8 + Forza con l'arma che tiene in mano

76-100 Attacca la creatura più vicina (a tale scopo, un Famiglio conta come parte del soggetto stesso)

Qualsiasi creatura confusa che venga attaccata, attacca automaticamente a sua volta il suo aggressore al suo round successivo.

\textbf{Esausto}:\index{Esausto} Un personaggio esausto si muove a velocità dimezzata e subisce penalità -4 Tiri per Colpire e Difesa. Dopo 1 ora di completo riposo (o Ristorare Inferiore), un personaggio Esausto diventa Affaticato. Un personaggio Affaticato diventa Esausto compiendo una azione che normalmente lo affaticherebbe.

\medskip

\textbf{Tabella: Livelli Affaticamento}\index{Tabella Livelli Affaticamento}

\medskip

\begin{tabularx}{0.45\textwidth}{lcll}
\textbf{Condizioni}& \textbf{Penalità}&\textbf{Rec.}&\textbf{Penalità}\\
&\textbf{TC-Dif.-TS}&&\\
\hline
vedi \textbf{Affaticato} &2&1h&-2 Comp\\
vedi \textbf{Esausto} (1)&4&1h&-2m/-4 Comp\\
Esausto (2) &6&8h&-3m/-6 Comp\\
Esausto (3) &8&12h&-6m/-8 Comp\\
Esausto (4) &Svenuto&12h&\\
Esausto (5) &Morte&--&\\
\end{tabularx}

\textbf{Impreparato}\index{Impreparato}: una creatura Impreparata ha -4 Difesa e TS Riflessi. Non potrai reagire, non userai Azioni o Reazioni se non esplicitamente permesse.

\textbf{Inabile}\index{Inabile}: Una creatura inabile non può effettuare azioni o reazioni. Gli attacchi in mischia contro una creatura inabile hanno +1d6 di bonus. E' Impreparata (-4 a Difesa ed ai Tiri Salvezza su Riflessi)

\textbf{Indifeso}:\index{Indifeso} Un personaggio addormentato, Privo di Sensi, Morente o per qualche altro motivo completamente alla mercé dei suoi avversari, è considerato Indifeso. Una creatura Indifesa non può compiere Azioni o Reazioni ne parlare, gli attacchi contro di lei hanno +1d6 di bonus. Ha una penalità di -4 a Difesa. Non e' consapevole di ciò che gli accade intorno. La creatura lascia cadere qualsiasi cosa impugni e cade prona. La creatura fallisce automaticamente i Tiri Salvezza su Tempra e Riflessi.

\textbf{Intralciato}:\index{Intralciato} Una creatura intralciata può muoversi a velocità dimezzata e non può Correre o Caricare, subisce penalità -2 ai Tiri per Colpire e penalità -2 alle prove di Destrezza. Un personaggio intralciato che cerca di lanciare un incantesimo si considera Distratto.

\textbf{Paralizzato}: \index{Paralizzato} Incapace di muoversi od agire, è \textbf{Indifeso} e può compiere azioni esclusivamente mentali.

\textbf{Paura, Spaventato}:\index{Paura}\index{Spaventato} Una creatura spaventata ha -1d6 ai Tiri per Colpire, Tiri Salvezza e Prove Competenza finché la sorgente della sua paura è visibile. Una creatura spaventata non può avvicinarsi volontariamente alla sorgente della sua paura.

\textbf{Prono}\index{Prono}: chi è prono ha un -1d6 ad attaccare ed un -4 alla Difesa. Alzarsi da prono costa 2 Azioni. Non si può diventare proni se si vola.

\textbf{Rallentato}\index{Rallentato}: viene sempre indicato con due valori, il primo indica quante Azioni in meno si fanno a round, il secondo la durata dell'effetto, se segnato con un - allora non ha una fine indicata. Es. Rallentato 1/3r, Rallentato 2/-.

\textbf{Impreparato / Sorpreso}\index{Impreparato}\index{Surpreso}: 
Una creatura sorpresa / impreparata ha una penalità di -4 a Difesa ed ai Tiri Salvezza su Riflessi. No Reazioni e Azioni per quel round.

\textbf{Stordito/Svenuto}:\index{Stordito}\index{Svenuto} si considera che sia Indifeso. Non può muoversi

\end{multicols}

\end{textblock*}


\begin{textblock*}{14 cm}(1cm,24.2cm)
\textbf{Leggere una Pergamena}

\textbf{in caso di pergamene ISY SCROLL}: costo produzione livello*livello*160mo

- Comprendere: Intelligenza od Arcana DC 10

- Lanciare: Intelligenza od Arcana DC 12.

\textbf{in caso di pergamene normali}: costo produzione livello*livello*80mo

- Comprenderne: Arcana a difficoltà 15

- Lanciare: Arcana DC 20 ed avere accesso alla Lista di Magia
\end{textblock*}



~\newpage

\begin{textblock*}{4cm}(1cm,1cm) % larghezza box, coord X, coord Y
{\textbf{Punti Fato}\\
(20-Livello)/5}

\riga

{\textbf{Morte}\\
PF=-10-(COS*2)}

\riga

\textbf{Copertura - Difesa}
Leggera +2 (>50\%)\\
Media +4 (<50\%)\\
Completa +8 (5\%)\\
Meta' a TS Riflessi

\riga

\textbf{Colpi Potenti}\\
+1 al danno - 2 TC. MAX CA/4

\riga

\textbf{Maestria del combattimento}\\
+2 Difesa -1d6 al Tiro per Colpire\\
-4 Difesa +2 il Tiro per Colpire \\
Non è possibile assegnare in questa maniera più di +-2d6.

\riga

\textbf{Carica}\\
3 Azioni. +1d6 a Tiro per Colpire, -4 alla Difesa, -10 attacchi oltre

\riga

\textbf{Attacco di Opportunità}\\
In movimento esce o attraversa la zona di mischia. Questo attacco è una Reazione che costa una Azione.

\riga

\textbf{Attacchi Multipli}\\
La prima azione di attacco non ha penalità mentre la seconda azione di attacco ha -5 al colpire cumulativo per attacco

\riga

\textbf{Difesa totale}\\
2 Azioni. No Attacco, NO Incantesimi, puoi fare solo una Azione e guadagni un +4 in Difesa. Non causi Attacchi di Opportunità se attraversi la zona di mischia di un avversario.

\riga


\end{textblock*}

\begin{textblock*}{8.3cm}(5.3cm,1cm)

\textbf{Rompere Oggetti - DC Forza}\\
\begin{tabular}{ll|ll}
Corda   		   & 23&	Porta semplice         & 13\\
Porta legno        & 15&	Porta robusta          & 18\\
Porta di ferro     & 28&	Catena                 & 26 \\
\end{tabular}

\riga

\begin{tabular}{lll}
\textbf{Difficoltà} & \textbf{Descrizione} & \textbf{Competenza} \\
5 & Estremamente facile  & Nulla\\
10  & Facile & Scarsa\\
15  & Normale  & Normale\\
20  & Difficile  & Buono\\
25  & Molto difficile  & Ottimo\\
30  & Eroica  	 & Eccellente\\
35  & Quasi impossibile & Stupefacente\\
40  & Impossibile  & Epica\\
\end{tabular}


\riga

\textbf{Azioni per Round} 

\begin{tabular}{ll}
Eseguire un attacco				& 1\\
Eseguire due attacchi				& 2\\
Eseguire più di due attacchi			& 3\\
Lanciare un'Incantesimo*			& 2\\
Eseguire una Azione di  Movimento*		& 1\\
Scatto						& 1\\
Alzarsi da prono				& 2\\
Aiutare qualcuno				& 2\\
Scambiare un dialogo con qualcuno*		& 3\\
\small{Scambiare poche battute con qualcuno*}	& 0\\
Cercare qualcosa nello zaino			& 2\\
{\small Prendere dalla cintura o di pronto} & 1\\
Usare un oggetto tenuto in mano			& 1\\
Bere una pozione tenuta alla cintura		& 1\\
Estrarre/Rinfoderare l'arma					& 1\\
Imbracciare lo scudo				& 1\\
Usare un oggetto magico				& 2\\
Eseguire prova su una competenza*		& 1\\
Sfondare una porta a spallate/calci			&1\\
Forzare una porta con un piede di porco		&1\\
Nascondersi					& 1\\
Concentrarsi su un Incantesimo			& 1\\
Salire o scendere dalla cavalcatura		& 1\\
Azione \textbf{I}mmediata - Azione \textbf{R}eazione		& I - R\\
Bere una pozione tenuta in mano			& I\\
Gettare un oggetto tenuto in mano		& R\\
Gettarsi a terra prono				& R\\
Riconoscere un Incantesimo			& R\\
\end{tabular}

\riga

\textbf{Alzarsi da prono}\\
2 Azioni. -4 Difesa, -4 Iniziativa. Acrobatica DC 13 1 Azione alzarsi. Tre 1 perdi il round. Acrobatica (6p) 1 Azione, Acrobatica (8p) Azione Immediata.

\riga

\textbf{Taglia e Portata standard}\\
\begin{tabular}{lll}
\textbf{Taglia}& \textbf{Spazio} &\textbf{Portata}\\
Minuscola & 25 x 25 cm&0m\\
Piccola & 0,5 x 0,5 m &0m\\
Media & 1 x 1 m & 1m\\
Grande & 3 x 3 m& 2m\\
Enorme & 5 x 5 m &3m\\
Mastodontico & 6 x 6 m&4m\\
Colossale & 12 x 12 m&6m\\
\end{tabular}

\riga

\textbf{Visione}\\

Una creatura Accecata \index{Accecata}subisce penalità -1d6 alle prove di Consapevolezza e -2 alle prove basate su Forza e Destrezza e fallisce automaticamente qualsiasi prova di Consapevolezza dipendente dalla vista.

Usando Scurovisione/Crepuscolare: Prova di Sopravvivenza per cercare trappole o di Consapevolezza solo visiva prende un -2 di penalità.

Combattere con Scarsa luminosita': -2 Tiro per Colpire


\end{textblock*}

\begin{textblock*}{6.2cm}(13.8cm,1cm) % larghezza box, coord X, coord Y
\textbf{Mod. al combattimento}\\
\textbf{Attacco}:
\textbf{+2}: fiancheggiare

\textbf{+1d6}: sei invisibile, carica

\textbf{-2}: abbagliato, intralciato, afferrato

\textbf{-1d6}: prono, ristretto, spaventato, scosso, armi da lancio/lunga in mischia, arma non conosciuta, creatura inv. ma individuata, arrampicando

\textbf{Difesa}: +2/4/8: copertura leggera(30\%)/ media(50\%)/ completa(80\%)

\textbf{-2}: intralciato \textbf{-4}: intrappolato, in ginocchio, seduto, prono, ristretto, stordito, lanci un incantesimo, arrampicando
\riga

\textbf{Riposare 8 ore} \\fa recuperare COS+2xCA+CM PF, minimo 1.

\riga

\textbf{Danni temporanei}\\ Ogni ora si recupera, con un minimo di 1 PF, il proprio valore di Costituzione in PF non letali (danni da stordimento) persi.

\riga

\textbf{Difesa Sorpresi}\\ -4 Difesa, -4 TS Riflessi

\riga

\textbf{Difesa Tocco}\\ NO Scudo, NO Armatura

\riga

\textbf{Tiro Critico}\\
Ogni qual volta hai colpito, tiri un dado arma aggiuntivo e non sommi altro per ogni due volte che hai tirato 6 nel Tiro per Colpire.

\riga

\textbf{Esplosione del Danno}\\
Se tiro dado e' massimo valore (min 8) ritiri il dado e sommi ancora il valore (del solo dado).

\riga

\textbf{Mettersi in difensiva}\\
usi una azione, +1 Difesa fino a inizio round dopo.


\riga

\textbf{Attacchi con armi a spargimento}\\
		1 2 3\\
		4 \textbf{X} 5\\
		6 7 8\\
		0\\
		X bersaglio, 0 origine. gittata 6 metri. 1d8 per direzione, 2d6 per metri.

\riga

\textbf{Azione di Scatto}\\
x2 Movimento. -1d6 nel Tiro per Colpire, -4 Difesa, Distratto

\riga

\textbf{Disingaggiare}\\
costa 1 Azione, ti sposti di 1 metro e non causi attacchi di opportunità

\end{textblock*}



~\newpage

\begin{textblock*}{19.5cm}(1cm,1cm) % larghezza box, coord X, coord Y

\begin{tabularx}{0.95\textwidth}{llll}
\textbf{Arma}&\textbf{Costo}&\textbf{Taglia/Danno} & \textbf{Gittata, Lista, Speciale}\\
Alabarda& 10 & G/1d10 P/T& \textbf{Lance}, \textbf{Aste}, Controcarica, Arma lunga, ED9 \\
Arco Corto Composito& note*& M/Frecce& 20 metri, \textbf{Archi}\\
Arco Corto& 30 & M/1d6 P& 15 metri, \textbf{Archi}\\
Arco Lungo Composito& note*& G/Frecce& 36 metri, \textbf{Archi}\\
Arco Lungo& 75 & G/Frecce& 20 metri, \textbf{Archi}\\
Ascia Martello& 16 & M/1d6 T/B& \textbf{Asce}\\
Ascia ad una mano& 6  & M/1d6 T& 6 metri, \textbf{Asce}, \textbf{Armi da Lancio}, Versatile\\
Ascia da battaglia& 10 & M/1d10 T&\textbf{Asce}\\
Balestra ad una mano& 100& M/Dardi& 6 metri, \textbf{Balestre}\\
Balestra leggera& 35 & P/Dardi& 15 metri, \textbf{Armi Semplici}, \textbf{Balestre}\\
Balestra pesante& 50 & G/Dardi& 30 metri \textbf{Balestre}\\
Bastone& 3& M/1d6 B& \textbf{Armi Semplici}, Arma lunga, Versatile\\
Brandistocco& 10 & M/2d4 P/T& \textbf{Lance}, Controcarica, Arma lunga\\
Catena chiodata& 25 & G/2d4 P& 3 metri, \textbf{Palle rotanti}, Arma lunga\\
Estoc& 25& G/1d8 P& \textbf{Aste}, Arma lunga\\
Falce& 18 & G/2d4 P/T& \textbf{Armi della Morte}, Arma lunga\\
Falcetto& 6& P/1d6 T& \textbf{Armi della Morte}\\
Falcione in asta& 12 & G/1d10 P/T& \textbf{Lance}, Controcarica, Arma lunga, ED9\\
Falcione& 75 & M/2d4 T& \textbf{Armi Aggraziate}, ED7\\
Fionda& -& P/1d4 B& 10 metri, \textbf{Archi}\\
Flagello Doppio& 90 & M/1d10 B& \textbf{Palle Rotanti}, \textbf{Armi doppie}\\
Flagello Pesante& 15 & M/1d10 B& \textbf{Palle Rotanti}\\
Flagello& 8& M/1d8 B& \textbf{Palle Rotanti}, \textbf{Rompi Cranio}\\
Frusta& 1& M/1d3 T& \textbf{Palle Rotanti}, Arma lunga\\
Giavellotto& 1& P/1d6P& 12 metri,  \textbf{Armi Semplici}, \textbf{Aste}, \textbf{Armi da Lancio}\\
Grande Ascia Doppia& 25 & G/1d12 T& \textbf{Asce}, \textbf{Armi doppie}, Arma lunga\\
Grosso randello& 2& M/1d8 B&\textbf{Rompi Cranio}\\
Guanto chiodato& 5& P/1d4 P&\textbf{Armi da Stordimento}\\
Katana& 300& M/1d10 T& \textbf{Spade}, \textbf{Armi letali}, ED9\\
Lancia da fante& 2& M/1d8 P&3 metri, \textbf{Lance}, Arma lunga, Controcarica\\
Lancia& 10 & G/1d8 P&\textbf{Lance}, Arma lunga, Controcarica\\
Machete& 10 & M/1d6 T&\textbf{Armi letali}\\
Maglio da guerra& 7& G/1d10 B& \textbf{Rompi Cranio}\\
Manganello& 1& P/1d6 B& \textbf{Armi da stordimento}, non letale\\
Martello da guerra& 5& M/1d8 B/P& 6 metri, \textbf{Rompi Cranio}\\
Mazza Leggera& 3& P/1d6 B/T& \textbf{Armi Semplici}, \textbf{Armi Leggere}, \textbf{Rompi Cranio} \\
Mazza Pesante& 5& M/1d8 B/T& \textbf{Rompi Cranio}\\
Mazza chiodata& 6& M 1d8 B/P& \textbf{Armi Semplici}, \textbf{Rompi Cranio}\\
Naginata& 8& G/1d12 T&\textbf{Lance}, Arma lunga, ED9\\
Picca Leggera& 4& M/1d4 P&\textbf{Armi della morte}\\
Picca Pesante& 8& G/1d6 P&\textbf{Armi della morte}, Arma lunga\\
Pugnale& 2& P/1d4 P& 6 metri, \textbf{Armi Semplici}, \textbf{Armi leggere}, \textbf{Armi da Lancio}\\
Pugno/Calcio nudo& note*& P/1d4 B&Versatile\\
Randello& 1& P/1d6 B& \textbf{Armi Semplici}, \textbf{Rompi Cranio}\\
Scimitarra& 15 & M/1d6 T&\textbf{Armi Leggere}, \textbf{Armi Aggraziate}, Versatile\\
Spada Corta& 10 & P/1d6 P&\textbf{Armi Leggere}, \textbf{Spade}, Versatile\\
Spada Lunga& 15 & M/1d8 T&\textbf{Spade}\\
Spada a due lame& 100& G/1d8 T& \textbf{Armi doppie}, \textbf{Spade}\\
Spada bastarda& 35 & M/1d8-1d10 T&\textbf{Spade}, 1d8 ad una mano, 1d10 a 2 mani\\
Spada larga& 12 & M/2d4 T&\textbf{Spade}\\
Spadone a due mani& 50 & G/2d6 T&\textbf{Spade}\\
Stocco& 20 & P/1d6 P& \textbf{Armi Leggere}, \textbf{Armi Aggraziate}, Versatile\\
Tridente& 15 & M/1d6 P/T& 3 metri, \textbf{Aste}, \textbf{Armi da Lancio}, Arma Lunga, Controcarica\\
Urgrosh& 18 & M/1d6 T/P& \textbf{Lance}, \textbf{Armi doppie}\\
\end{tabularx}

\end{textblock*}

\begin{textblock*}{11.5cm}(1cm,24.5cm) % larghezza box, coord X, coord Y

\begin{tabular}{llll}
\textbf{Nome Proiettile}   & \textbf{Num./MO} & \textbf{Danno/Tipo} & Peso(kg) \\
Biglie di Marmo (fionde)   & 15/1 mo                    & 1d4 B               & 0.2      \\
Dardi da balestra, leggeri & 10/1 mo                    & 1d6 P               & 0.1      \\
Dardi per balestra pesante & 3/1 mo                     & 1d10 P              & 0.3      \\
Frecce da caccia           & 20/1 mo                    & 1d6 P               & 0.1      \\
Frecce da guerra           & 10/1 mo                    & 1d8 P               & 0.2      \\
Sasso (fionde)             & -                          & 1d2 B               & 0.2      \\
\end{tabular}

\end{textblock*}

\begin{textblock*}{7.8cm}(12.7cm,24.5cm) % larghezza box, coord X, coord Y
\textbf{Capacità di Carico e Armature}\\
La CdC è pari a 9 (P), 16 (M), 25 (G) + Forza + Costituzione.\\
Un Arma Leggera ha Ingombro 1, Media ha 2, Grande ha 4.
\end{textblock*}

~\newpage

\begin{textblock*}{15cm}(1cm,1cm) % larghezza box, coord X, coord Y

\begin{tabular}{lllllll}
\textbf{Armatura} & \textbf{Costo (mo)} & \textbf{Difesa} & \textbf{-Comp.} &  \textbf{Tipo} & \textbf{Mov.} & \textbf{Prova Magia}\\
\hline
Imbottita   & 5    & 1   & 0  &  L   & 0   & No\\
Cuoio   & 10   & 2   & 0   & L   & 0   & SI\\
Cuoio rinforzato   & 25  &3  & 0   &    L   & 0 &SI \\
Giaco di Maglia   & 15   & 4  & -1  &  M   & 0  &+1d6\\
Scaglie& 50   & 5  & -1  &  M   & 0 &+1d6 \\
Anelli & 150  & 6  & -1  &  M   & 0  &+1d6\\
Pettorale    & 200  & 6  & -2  &  M  &  0 &+1d6 \\
Bande   & 250  & 7  & -2  &  P   & 0  &+2d6 \\
Mezza armatura   & 1200 & 8  & -2  &  P   & 1 &+2d6  \\
da Campo& 1400 & 9 & -3  &   P   & 2  &+2d6 \\
Completa& 1500 & 10  & -4  & P   & 3  &+2d6 \\
\end{tabular}

\riga

\begin{tabular}{lccccc}
\textbf{Scudi} & \textbf{Costo} & \textbf{Difesa} & \textbf{Penalità TC} & \textbf{Prova magia} & \textbf{Tipo}\\
\hline
Brocchiero& 5 mo  &  0& 1& SI& L\\
Scudo leggero di legno   & 3 mo  &  0& 2& SI  & L\\
Scudo leggero di metallo & 9  mo  &  0& 3&SI  & L\\
Scudo medio legno   & 5 mo   &  -1& 4& +1d6  & M\\
Scudo medio metallo & 12 mo  & -1  & 5& +1d6  & M\\
Scudo pesante di legno   & 7  mo  &  -2    & +2d6&5  & P\\
Scudo pesante di metallo & 20 mo  &  -2    & +2d6&7  & P\\
\end{tabular}

\riga

\textbf{Ingombri Armature e Scudi}\\
Una Armatura Leggera ha Ingombro 2, Media 4, Pesante 8.\\
Uno Scudo Leggero ha Ingombro 1, Media 2, Pesante 4.

\riga

\textbf{Tempi per indossare e togliere l'armatura}\index{Tabella Tempi per indossare e togliere l'armatura}\\

\begin{tabular}{llll}
\textbf{Tipo di Armatura}& \textbf{Indossare} & \textbf{in fretta} & \textbf{Togliere}\\
Scudo								& 1 azione 	& -     	& 1 azione\\
Imbottita, Cuoio, Cuoio rinforzata  & 1 minuto	& 3 round  	& - \\
Giaco di Maglia						& 1 minuto	& 5 round  & 5 round\\
Scaglie, Anelli, Pettorale, Bande   & 4 minuti 	& 1 minuto{*}  & 1 minuto\\
Mezza armatura, da Campo, Completa  & 4 minuti{*}{*}& 4 minuti{*}& 1d4+1 minuti\\
\end{tabular}

\end{textblock*}

\begin{textblock*}{12cm}(1cm,14.7cm) % larghezza box, coord X, coord Y

\begin{tabular}{lllll}
\hline
\textbf{Cavalcatura}&\textbf{Costo}&\textbf{Mov.}&\textbf{Carico}&Km/h\\
&(\textbf{mo})&&&\\
Asino o Mulo&8&12 m&210 kg&6km\\
Cammello&50&15 m&240 kg&8km\\
Cavallo da Galoppo&75&18 m&240 kg&12km\\
Cavallo da Guerra&400&18 m&270 kg&9km\\
Cavallo da Tiro&50&12 m&270 kg&6km\\
Elefante&200&12 m&660 kg&6km\\
Mastino&25&12 m&97,5 kg&6km\\
Pony&30&12 m&112,5 kg&6km\\
Carretto/Carro    & 15/30 mo & 9/12 m   &150/600kg    & 3/6km              \\
\end{tabular}

\riga

\begin{tabular}{ll}

\textbf{Contenitore}&\textbf{Capienza}\\
Ampolla o Boccale&0,5 litri liquidi\\
Barile&			160 litri liquidi, 4 cubi di 30 cm\\
Borsa&			1 cubo di 10 cm/3 kg di oggetti\\
Bottiglia&		1 litro di liquido\\
Brocca o Caraffa&4 litri liquidi\\
Canestro&		2 cubi di 30 cm/20 kg di oggetti\\
Fiala&			120 ml di liquidi\\
Forziere&		12 cubi di 30 cm/150 kg di oggetti\\
Otre&			2 litri liquidi\\
Sacco&			1 cubo di 30 cm/15 kg di oggetti\\
Secchio&		12 litri liquidi, 1 cubo di 25 cm\\
Vaso di Ferro&	4 litri liquidi\\
Zaino*&			1 cubo di 30 cm/15 kg di oggetti\\
\end{tabular}


\riga


\begin{tabular}{l|cc|c}
\textbf{Fonte di} &\multicolumn{2}{c}{\textbf{Raggio in metri}}& \textbf{Durata}  \\
\textbf{Luce}& \textbf{Luce} & \textbf{Luce Fioca} &\\
Candela  & 1 metro & -  & 1 ora\\
Torcia & 3 metri & 6 metri  & 1 ora\\
Lanterna & 6 metri & 6 metri  & 3 ore \\
\multicolumn{4}{c}{\textbf{incantesimi}}\\
Luce 		 & 3 metri & 6 metri  &30 min \\
Luce Diurna  & 6 metri & 12 metri & 1 ora \\
\end{tabular}


\riga

\end{textblock*}

\begin{textblock*}{6cm}(13.4cm,14.7cm) % larghezza box, coord X, coord Y

\begin{tabular}{ll}
\textbf{Oggetto}&\textbf{Costo}\\
\textbf{Birra}&\\
Boccale&4 mr\\
Caraffa (4 litri)&2 ma\\
\textbf{Pietanze} &\\
Banchetto (a persona)&10 mo\\
Carne, 1 pezzo&3 ma\\
Formaggio, 1 pezzo&1 ma\\
Pane (a pagnotta)&2 mr\\
\textbf{Locanda (al giorno})&\\
Squallida&7 mr\\
Povera&1 ma\\
Modesta&5 ma\\
Agiata&8 ma\\
Ricca&2 mo\\
Aristocratica&4 mo\\
\textbf{Pasto (al giorno)}&\\
Squallido&3 mr\\
Povero&6 mr\\
Modesto&3 ma\\
Agiato&5 ma\\
Ricco&8 ma\\
Aristocratico&2 mo\\
\textbf{Vino}&\\
Buono (bottiglia)&10 mo\\
Comune (caraffa)&2 ma\\
\end{tabular}

\end{textblock*}


~\newpage

\begin{textblock*}{4cm}(1cm,1cm) % larghezza box, coord X, coord Y
{\textbf{Competenze}\\
\textbf{Forza}\\
Arrampicarsi\\
Intimidire\\
Nuotare\\
Saltare	\\
\textbf{Destrezza}\\
Acrobatica\\
Artista della fuga\\
Giocoliere\\
Mani di fata\\
Muoversi silenziosamente\\
Nascondersi nelle ombre\\
Usare corda	\\
\textbf{Intelligenza}\\
Arcana\\
Conoscenza*\\
Dungeon\\
Erboristeria\\
Disattivare congegni\\
Falsificare\\
Lingue\\
Natura magica\\
Valutare\\
\textbf{Saggezza}\\
Cavalcare\\
Consapevolezza\\
Gestire animali\\
Natura\\
Orientamento\\
Percepire Emozioni\\
Pronto soccorso\\
Religione\\
Seguire tracce\\
Sopravvivenza\\
\textbf{Carisma}\\
Diplomazia\\
Intrattenere\\
Ingannare\\
Tradizioni locali
}

\riga

\textbf{Riconoscere un incantesimo}\\ Arcana DC 11 + livello dell'incantesimo. 1 Reazione

\riga

\textbf{Valutare} 3 Azioni : DC 12 + rarità dell'oggetto, + 2 comune, 4 non comune, 6 raro, 12 molto raro, 16 leggendario. \\
Con punteggio 6 costa 2 Azioni, con 12 costa 1 Azione.

\end{textblock*}


\begin{textblock*}{14.5cm}(5.5cm,1cm) % larghezza box, coord X, coord Y
\textbf{Golden Rules}\\

{\textbf{I 6 esplodono}} - se fai 6, sommi e ritiri\\
Gli \textbf{1 portano male}, se fai 1 con il dado vale zero\\
\textbf{Affidarsi alla sorte}. -4 punti di competenza/caratteristica = +1d6\\
\end{textblock*}


\begin{textblock*}{7.4cm}(5.3cm,3.2cm) % larghezza box, coord X, coord Y
\textbf{Pronto Soccorso}\\
DC 12 + INT(-PF) per stabilizzare a 0 PF\\
2 minuti/1 p6: DC 15 recuperi 1d4 PF\\
+2 TS Tempra Veleno\\
DC 12+2xSanguinamento -1 Sanguinamento

\riga

\textbf{Intimidire}\\
2 Azioni. p12 1 Azione. Intimidire è contrapposta al TS Volontà (CAR). Se il Tiro Salvezza fallisce, l’avversario fino alla fine del round successivo ha -1 al Tiro per Colpire e -1 alla Difesa contro quell’avversario soltanto.\\
Se chi tenta la prova di Intimidire non riesce con un successo fallimento critico allora deve effettuare un Tiro Salvezza su Volontà con modificatore Carisma a DC 10+Grado di Sfida dell'avversario (o Livello) o subire le medesime penalità come se fosse stato intimidito.
Se il tiro contrapposto riesce con un successo critico e l'avversario fallisce il Tiro Salvezza la durata dell'effetto permane fino a fine combattimento.

\riga

\textbf{Arrampicarsi - Scalare}\\
\textit{Si hanno penalità dovuta all'Armatura}

\begin{tabular}{ll}
\textbf{Esempio di Superficie} & \textbf{DC}\\
Movimento solo dimezzato & -2d6\\
Superficie scivolosa&+5\\
Grezza con appigli, mattoni sporgenti&10\\
Albero, una corda senza nodi&15\\
Parete liscia con appigli &20\\
Muro perimetrale pochissimi appigli&25\\
Parete naturale senza appigli&30\\
Appoggiare a 2 pareti opposte&-10\\
Appoggiare a 2 parete angolari&-5\\
Puoi usare una corda&-8\\
\end{tabular}

\textbf{Terreno doppio difficile}. Se fallimento 10+ cadi, TS Riflessi stessa DC per afferrarsi.

\riga


\textbf{Riconoscere Mostri} 1 Azione, 	DC=Sfida della creatura + 10 + rarita'
	
\textit{Arcana}: Giganti, Costrutti, Spiriti, Mostruosità, Aberrazioni, Draghi\\
\textit{Piani}: Elementali\\
\textit{Occulto}: Immondi, Spiriti, Non Morti\\
\textit{Religione}: Spiriti, Non Morti, Celestiali\\
\textit{Dungeon}: Aberrazioni, Mostruosità, Melme, e creature sotterranee\\
\textit{Natura}: Bestie, Piante, Fatati\\
	
- \textit{entro \textbf{2}}: nome, tipo, la caratt. principale|
- \textit{oltre \textbf{7}}: migliore Tiro Salvezza, 1 resistenza/immunità a Condizioni, 1 vulnerabilità a Condizioni, attacco tipico|
- \textit{oltre \textbf{12}}: peggiore Tiro Salvezza, 1 resistenze/immunità a Condizioni, 1 immunità a Danni, 1 vulnerabilità a Condizioni, 1 vulnerabilità a tipo di Danno|
- \textit{oltre \textbf{15}}: 2 immunità a Condizioni, 1 immunità a Danni, 1 vulnerabilità a Condizioni, 1 vulnerabilità a tipo di Danno|
- \textit{oltre \textbf{17}}: grado di sfida relativo |
- \textit{oltre \textbf{20}}: attacco e difese speciali|
	

\end{textblock*}


\begin{textblock*}{7cm}(13cm,3.2cm) % larghezza box, coord X, coord Y

\textbf{Prove Contrapposte}\\
Chi esegue la Prova deve fare almeno 10 + Competenza/Tiro Salvezza + Statistica + Abilità

\riga

\textbf{Identificare  Pozioni}\\
Erboristeria a DC 12 + fattore di rarità della pianta.  1 Azione ogni 10 DC, 6p ogni 15 DC, 12p ogni 20 DC

\riga

\textbf{Riconoscere oggetto magico}\\
1 minuto DC 30. Arcana 6p costa 5 min., 12p costa 1 min., 18p costa 1 Round.
\riga

\textbf{Saltare} 1 Azione\\
\textit{Si hanno penalità dovuta all'Armatura}

\textbf{Distanza saltata in lungo}: 30cm per risultato\\

\textbf{Distanza saltata in alto}: 10cm per risultato\\

Rincorsa 3 metri altrimenti metà.

\riga

\textbf{Danno Caduta}: H(m)/3xD6. Ogni 3 dadi oltre i 20 aggiungete 6 di danno (X/3)d6+(X/3-20)*6. Proni. Prova Acrobatica DC 15 1/2 danno entro 9m.  Cadute su superfici morbide (terreno morbido, fango ecc.) -1d6 danni.

\riga

\textbf{Nuotare}\\
\textit{Penalità dovuta all'Armatura su Forza}\\
Acqua calme DC 10.\\
Acque mosse ha DC 15\\
Acque tempestose DC 20

\riga


\textbf{Sopravvivenza}
\begin{tabular}{ll}
Difficoltà base  & DC 10\\
Se il terreno è molto morbido& DC +5\\
Se il terreno è morbido& DC +10\\
Se il terreno è stabile& DC +15\\
Se il terreno è duro& DC +20\\
Ogni 3 creature inseguite& DC -1\\
A seconda della taglia& DC +-8\\
Ogni 24 ore passate&DC +2\\
Ogni ora di pioggia&DC +4\\
Visibilità scarsa&DC +2\\
Cerca di occultare le traccie&DC +5\\
\end{tabular}\\

\riga

\textbf{Artista della Fuga}\\
1 Azione ogni 10 di DC. 6p 1 Azione 15 di DC, 12p 1 Azione 20 DC.

\end{textblock*}

~\newpage

\begin{textblock*}{5cm}(1cm,1cm) % larghezza box, coord X, coord Y
\textbf{Prova di Magia}\\
3d6 + 1d6*(1/4 Comp. Magica)\\
Ignori un dado per ogni Adepto della Magia oltre il primo nella lista\\
Fallimento Critico: due volte 1, un 1 e due 2\\


\riga

\textbf{Distratto}\\
Sei Distratto se: occulti il lancio di incantesimo, Impedito, disturbato, sanguinante, sotto attacco.\\

\riga

\textbf{Punti Magia}\\
Mod. Caratteristica + \\

\begin{tabular}{ll|ll}
\textbf{CM} & \textbf{P.M}&	\textbf{CM} & \textbf{P.M}\\
	1&	2 	&11&43\\
2&	4	&12&47\\
3&	8	&13&50\\
4&	10	&14&54\\
5&	16	&15&58\\
6&	19	&16&62\\
7&	23	&17&71\\
8&	27	&18&76\\
9&	36	&19&82\\
10&	41	&20&89\\
20+&prec.+ 4&&\\
\end{tabular}

\riga

\textbf{Tiro Salvezza Incantesimo}\\
DC = 10 + Competenza Magica/x + modificatore caratteristica per incantesimo + 2 x Abilità prese in quella Liste di Magia +2 x Critico nella Prova di Magia

\riga

\textbf{Tiro Salvezza Magia da Oggetto}\\
DC = 10 + 3 x livello incantesimo manifestato

\riga

\textbf{Tiro Salvezza Incantesimo Mostro}\\
DC è 10 + 3 x livello incantesimo + Intelligenza

\riga

\textbf{Quando si hanno < 50\& Punti Magia} ogni incantesimo deve essere fatto con una Prova di Magia.

\riga

\textbf{Successo Critico Automatico}:  x2 costo Punti Magia cumulativo. Es. 4,8,16,32..

\riga

\end{textblock*}


\begin{textblock*}{13.5cm}(6.5cm,1cm) % larghezza box, coord X, coord Y
\textbf{Seguace}\\
2 Tratti comuni con Patrono. Se sei un Seguace ottieni +1d6 alle Prove di Magia nella scuola preferita dal Patrono. Puoi usare l'energia preferita del Patrono nei tuoi incantesimi.\\

\riga

\textbf{Devoto}\\
3 Tratti in comune con Patrono. Un Devoto aggiunge +1d6 alla Prova di Magia nelle scuole preferite dal Patrono e ignora un dado tirato nella Prova di Magia. Devi usare l'energia preferita del Patrono nei tuoi incantesimi.

\riga

\textbf{Fallimento Critico Prova di Magia - 3d6 -1d6 x Fallimento Crit. Min. 1d6}
\begin{tabularx}{0.95\textwidth}{lX}
1 & Aumenti la condizione di Affaticato di 2 gradi\\
2 & Per 1 giorno non sei più in grado di canalizzare energie magiche. Non puoi lanciare incantesimi se non facendo un successo magico critico nella Prova di Magia\\
3 & Manifesti una modifica corporea minore\\
4 & Vieni investito da una roboante colonna di Luce e Vuoto. In un raggio di 3 metri intorno a te compreso, chiunque deve fare un Tiro Salvezza su Riflessi DC 15 per dimezzare o subire 1d6 di danni per livello di incantesimo\\
5 & Per 3 round sei sotto l'influenza dell'incantesimo Confusione\\
6 & Sei paralizzato per 3 round\\
7 & Vieni teletrasportato entro 3d10 metri in una direzione casuale\\
8 & Diventi Invisibile ed incapace di parlare per 6 round\\
9 &  Solo tu vieni avvolto da una cortina di oscurità magica impenetrabile per 6 round\\
10 & Non riesci a parlare bene, sei balbuziente. Ogni lancio di incantesimi ti costringe a superare una Prova di Magia. Durata 3 round\\
11 & Il prossimo incantesimo che lanci ha effetti se possibile minimizzati\\
12 & Il battito del tuo cuore è come il battito di un tamburo, si può sentire entro 50 metri\\
13 & Ti cadono tutti i peli del corpo, per fortuna possono ricrescere\\
14 & Emetti una rumorosa e pestilenziale flatulenza. Una insegna luminosa di 1m x 50cm sopra la tua testa ti indica e ti sbeffeggia\\
15 & Ogni oggetto che tieni in mano ti cade a terra\\
16 & Guadagni 2d6 Punti Magia\\
17 & Una incudine cade, 3d6 di danno Tiro Salvezza su Riflessi DC 15 per dimezzare, su una creatura a caso, escluso te, entro sei metri\\
18 & Tutte le creature, escluso te, nel raggio di 6 metri da te subiscono 1d10 danni non resistibili
\end{tabularx}

\riga

\textbf{Massimo Livello Incantesimo lanciabile}

\begin{itemize}
	\item
	\textbf{Adepto della Magia} (Regola 1) \\
	- presa 1 volta puoi lanciare solo incantesimi entro il livello 4 compreso\\
	- presa 2 volte puoi lanciare solo incantesimi entro il livello 6 compreso\\
	- presa 3 volte puoi lanciare ogni livello di incantesimo

	\item
	\textbf{Competenza Magica} (Regola 2)\\
	- puoi lanciare incantesimi entro la metà +1 del valore di Competenza Magica, es CM 7 = (7/2)+1 = 3+1 = 4lv incantesimi

\end{itemize}

\riga

\textbf{Alterare la Magia}

\textbf{Magia efficace}: l'Incantatore 1,2 tirato nella Prova di Magia pagando x2,x3,x4. Anche compagna

\textbf{Magia eterea}: aumentando di 3 i Punti Magia spesi nell'incantesimo le proprie magie hanno pieno effetto su creature eteree o incorporee

\textbf{Magia pietosa}: aumentando di 3 i Punti Magia spesi le magie infliggono danni temporanei.

\textbf{Aumentare il tempo} di lancio da 2 Azioni a 1 round -1 in Punti Magia

\textbf{Magie collaborative}: un altro mago costo metà Punti Magia concede +1d6 alla Prova di Magia del compagno.

\textbf{Circolo del Potere}: tutti Devoti o Seguaci dello stesso Patrono
Per mago costo la metà dei Punti Magia dell'incantesimo lanciato dal compagno +1d6 alla Prova di Magia del compagno, max +7d6. Tempo lancio almeno 1 turno
\end{textblock*}

~\newpage

\end{document}
