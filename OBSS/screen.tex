\documentclass[a4paper,12 pt,openany]{book}
\usepackage[utf8]{inputenc}
\usepackage[T1]{fontenc}
\usepackage{amsmath}
\usepackage{amsfonts}
\usepackage{amssymb}
\usepackage{multicol}
\usepackage{multirow}
\usepackage{tabularray}
\usepackage{booktabs}
\usepackage{tabularx}
\usepackage{graphicx}
\usepackage{xltabular}

\usepackage[a4paper]{geometry}
\geometry{verbose,tmargin=2cm,bmargin=2.5cm,lmargin=1.5cm,rmargin=2cm}
%\usepackage[absolute,overlay]{textpos}
\usepackage[absolute,overlay,showboxes]{textpos}
%\usepackage[width=457.86mm, height=303.28mm]{geometry}
\pagenumbering{gobble}

\newcommand{\riga}{\rule{\textwidth}{0.4pt}}

\begin{document}


\center

\footnotesize

\begin{textblock*}{19 cm}(1cm,1cm) % larghezza box, coord X, coord Y
\flushleft

\begin{multicols}{2}
	
\textbf{Aumento Categoria di danno} 1d4 > 1d6 > 1d8 > 1d10/d12 > 2d6 > 2d8 > 2d10 > 3d6 > 3d8 > 3d10.

\textbf{Accecato}:\index{Accecato}. Non vedi. -2  Competenze basate su Forza e Destrezza. Tutti sono invisibili. Terreno sempre difficile (Acrobatica DC 12 per muoversi normalmente). -1d6 a colpire, +1d6 a essere colpito.

\textbf{Affaticato}: No Corsa, no Carica. Malus a TC, Difesa, TS

\begin{tabularx}{0.45\textwidth}{lcl}
	\textbf{Condizioni}& \textbf{Pen./Mov/Comp.}&\textbf{Rec.}\\
	\hline
Affaticato  &1/-/-&1h\\
Affaticato 2&2/2m/-4&1h\\
Affaticato 3&4/3m/-6&8h\\
Affaticato 4&6/6m/-8&12h\\
Affaticato 5&Svenuto&6h\\
Affaticato 6&Morte&--
\end{tabularx}

\textbf{Afferrato}\index{Afferrato}: No movimento. Puo' Spingere. 2 Azioni per liberarsi (TS Tempra + Forza contro prova Atletica, +1d6 per Taglia di differenza). -2 Difesa, Distratto. Solo armi piccole o naturali.

\textbf{Annegare/Trattenere il fiato}: Round 10+10 x Cos. 1 Azione = -1 Round. Dopo TS Tempra DC 12 ogni round per continuare a trattenere il fiato. Ogni round, la DC aumenta di 2. Vedi pag. 243

\textbf{Assordato}: Distratto nel lancio degli incantesimi con componenti almeno verbali.

\textbf{Bloccato}: vedi Afferrato. -4 a Difesa e TS Riflessi. Prova di Magia con successo critico per incantesimi. -1d6 TC

\textbf{Colpo di Grazia}:: 3 azioni, 3 critici. Bersaglio Indifeso.

\textbf{Confuso}: se attaccato attacca ultima creatura.
Ogni round tira

\begin{tabularx}{0.45\textwidth}{lX}
	\hline
	d10 & Comportamento\\
	1 & La creatura usa tutte le sue Azioni per per muoversi in una direzione casuale. Per determinare la direzione tira un d8\\
	2-5 & La creatura non fa nulla per tutto il round\\
	6 & La creatura effettua un attacco contro se stessa e finisce il round\\
	7-8 & La creatura effettua un attacco contro una creatura determinata a caso entro 1 Azione di Movimento. Se è stata colpita il round precedente attaccherà la creatura che l'ha colpito. Fatto l'attacco il round termina.\\
	9-10 & La creatura può agire e muoversi normalmente.
\end{tabularx}


\textbf{Distratto}: distratto, impedito, disturbato, sanguinante, è sotto attacco, devi fare Prova di Magia

\textbf{Correre}: -4 Difesa, -1d6 TC fino a inizio prossimo round

\textbf{Fiancheggiare}: +2 al Tiro per Colpire o alla Difesa.

\textbf{Impreparato / Sorpreso} -2 a Difesa ed TS Riflessi. No Azioni, Reazioni

\textbf{Inabile}: No azioni o reazioni. E' Impreparata

\textbf{In Lotta}: vedi Afferrato

\textbf{Indifeso} addormentato, Privo di Sensi, Morente, alla mercé degli avversari. No Azioni o Reazioni, +1d6 se attaccano. Non è consapevole, lascia cadere oggetti e cade prona. TS Riflessi e Tempra falliscono. No Destrezza alla Difesa.

\textbf{Intralciato}: terreno difficile, no corsa, no carica , -2 TC, -2 Difesa, Distratto.

\textbf{Invisibile}: -1d6 al TC, +1d6 al TC avversario.

\textbf{Morente}: -1 pf a round

\textbf{Nauseato}: -1d6 TS, TC, Prove

\textbf{Paralizzato}: No Azioni o Reazioni. +1d6 al TC avversari. Fallimento TS Riflessi. NO destrezza alla Difesa. E' terreno difficile.

\textbf{Paura, Spaventato}: -1d6 ai TC e TS contro chi la impaurisce.

\textbf{Privo di sensi}: vedi Indifeso

\textbf{Prono}\index{Prono}: -4 TC, -4 Difesa. 1 Azione

\textbf{Stordito/Svenuto}: vedi Indifeso

\textbf{Trattenere il fiato}: vedi \textbf{Annegare/Trattenere il fiato}

\textbf{Ristretto} : -1d6 TC, -4 Difesa.

\textbf{Sanguinante}: ad inizio round. Pronto Soccorso con DC 12, 2 Azioni, +2 per +1 Sanguinamento.

\end{multicols}

\riga


\noindent\begin{tabularx}{0.98\textwidth}{l|X|X}
	\multicolumn{2}{c}{\textbf{Attaccante}}&\multicolumn{1}{c}{\textbf{Difensore}}\\
	\textbf{Mod}.&\multicolumn{1}{c}{\emph{Situazione}}&\multicolumn{1}{c}{\emph{Situazione}}\\
	\textbf{-1}& Affaticato (1), Luce fioca&Affaticato (1)\\
	\hline
	\textbf{-2}& Affaticato (2), Intralciato & Affaticato (2), Afferrato, Intralciato, Sorpreso\\
	\hline
	\textbf{-4}& Affaticato (3), Prono, Arma Lunga a corta distanza, attacco non letale con arma letale& Affaticato (3), Prono, In ginocchio, Seduto, Ristretto, Stordito, Afferrato ad una parete, Bloccato\\
	\hline
	\textbf{-1d6}& Ristretto, Spaventato, Arma da Lancio contro avversario in mischia, Arma non conosciuta, Bersaglio invisibile ma Individuato, Afferrato ad una parete, Bloccato&\\
	\hline
	%\textbf{+1}& & \\
	%\hline
	\textbf{+2}& Fiancheggia, Posizione Sopraelevata, Attacca al spalle& Copertura leggera\\
	\hline
	\textbf{+4}&& Copertura media\\
	\hline
	\textbf{+1d6}& Invisibile, Carica, avversario Indifeso& \\
	\hline
	\textbf{+8}&& Copertura completa
\end{tabularx}



\end{textblock*}

~\newpage

\begin{textblock*}{4cm}(1cm,1cm) % larghezza box, coord X, coord Y
{\textbf{Punti Fato}\\
(20-Livello)/5}

\riga

{\textbf{Morte}\\
PF=-10-(COS*2)}

\riga

\textbf{Copertura - Difesa}
Leggera +2 (>50\%)\\
Media +4 (<50\%)\\
Completa +8 (5\%)\\
Meta' a TS Riflessi

\riga

\textbf{Colpi Potenti}\\
+1 al danno - 2 TC. MAX CA/4

\riga

\textbf{Maestria del combattimento}\\
+1 Difesa -2 TC\\
+1 TC -2 Difesa\\
Max CA/4

\riga

\textbf{Preparare la Difesa}\\
1 Azione = +1 Difesa\\
Tratto Parata= +1 Difesa

\riga

\textbf{Carica}\\
3 Azioni. +1d6 a Tiro per Colpire, -4 alla Difesa, -10 attacchi oltre

\riga

\textbf{Attacco di Opportunità}\\
In movimento esce o attraversa la zona di mischia o lancia un incantesimo. Questo attacco è una Reazione.

\riga

\textbf{Attacchi Multipli}\\
La prima azione di attacco non ha penalità mentre la seconda azione di attacco ha -5 al colpire cumulativo per attacco

\riga

\textbf{Esplosione del Danno}\\
Se tiro dado e' massimo valore (min 8) ritiri il dado e sommi ancora il valore (del solo dado).

\riga

\textbf{Armi a spargimento}\\
8 1 2\\
7 \textbf{X} 3\\
6 5 4\\
0\\
X bersaglio, 0 origine, 1d8 per direzione, 2d6 metri.

\riga



\end{textblock*}

\begin{textblock*}{8.3cm}(5.3cm,1cm)

\textbf{Rompere Oggetti - DC Forza}\\
\begin{tabular}{ll|ll}
Corda   		   & 23&	Porta semplice         & 14\\
Porta legno buono  & 18&	Porta robusta          & 25\\
Porta di ferro     & 30&	Catena                 & 26 \\
\end{tabular}

\riga

\begin{tabular}{lll}
\textbf{Difficoltà} & \textbf{Descrizione} & \textbf{Competenza} \\
5 & Estremamente facile  & Nulla\\
10  & Facile & Scarsa\\
15  & Normale  & Normale\\
20  & Difficile  & Buono\\
25  & Molto difficile  & Ottimo\\
30  & Eroica  	 & Eccellente\\
35  & Quasi impossibile & Stupefacente\\
40  & Impossibile  & Epica\\
\end{tabular}

\riga

\textbf{Azioni per Round} 

\noindent\begin{tabular}{ll}
	\textbf{Cosa si fa} & \textbf{Azioni}\\
	\hline
Eseguire un attacco& 1\\
Eseguire due attacchi& 2\\
Eseguire più di due attacchi& 3\\
Estrarre o Rinfoderare l'arma o scudo& 1\\
\hline
Eseguire una Azione di Movimento &1*\\
Scatto & 1\\
Alzarsi da prono& 1\\
\hline
Aiutare qualcuno& R\\
Eseguire prova su una competenza& 1*\\
Riconoscere una creatura& 1\\
Nascondersi& 1\\
\hline
Salire o scendere dalla cavalcatura& 2\\
Sfondare una porta a spallate/calci& 1\\
Forzare porta con piede di porco& 2\\
\hline
Cercare qualcosa nello zaino& 2\\
Prendere qualcosa dalla cintura o di pronto & 1\\
Usare un oggetto tenuto in mano& 1\\
\hline
Bere una pozione tenuta in mano& I\\
Fare bere una pozione ad un altro & 2\\
\hline
Gettare un oggetto tenuto in mano& R\\
Gettarsi a terra prono& R\\
\hline
Lanciare un Incantesimo*& 2\\
Concentrarsi su un Incantesimo& 1\\
Interrompere un proprio incantesimo & I\\
Riconoscere un Incantesimo& R\\
Usare un oggetto magico& 2\\
\hline
Scambiare un dialogo con qualcuno& 3*\\
Scambiare poche battute con qualcuno& 0*\\
\hline
Preparare la Difesa & 1\\
Difesa Totale & 2\\
Disingaggiare & 1\\
Colpo preciso & 2\\
\hline
Disarmare & 2\\
Finta & 1\\
Spingere & 2\\
Afferrare l'avversario & 2\\
Fare cadere l'avversario & 2
\end{tabular}

\riga

\textbf{Alzarsi da prono}\\
1 Azioni. -4 Difesa, Acrobatica DC 13 1 Azione Immediata. Tre 1 perdi il round. 

\riga


\textbf{Azione di Scatto}\\
x2 Movimento. -1d6 nel Tiro per Colpire, -4 Difesa, Distratto



\end{textblock*}

\begin{textblock*}{6.2cm}(13.8cm,1cm) % larghezza box, coord X, coord Y

\riga

\textbf{Taglia e Portata standard}\\
\begin{tabular}{lll}
	\textbf{Taglia}& \textbf{Spazio} &\textbf{Portata}\\
	Minuscola & 25 x 25 cm&0m\\
	Piccola & 0,5 x 0,5 m &1m\\
	Media & 1 x 1 m & 1m\\
	Grande & 2 x 2 m& 1m\\
	Enorme & 3 x 3 m &2m\\
	Mastodontico & 4 x 4 m&2m\\
	Colossale & 12 x 12 m&6m\\	
\end{tabular}

\riga

\textbf{Visione}\\

Una creatura Accecata \index{Accecata}subisce penalità -1d6 alle prove di Consapevolezza e -2 alle prove basate su Forza e Destrezza e fallisce automaticamente qualsiasi prova di Consapevolezza dipendente dalla vista.

Usando Scurovisione/Crepuscolare: Prova di Sopravvivenza per cercare trappole o di Consapevolezza solo visiva prende un -2 di penalità.

Combattere con Scarsa luminosita': -1 Tiro per Colpire

\riga

\textbf{Riposare 8 ore} \\fa recuperare CA o CM x COS, minimo CA+CM

\riga

\textbf{Danni temporanei}\\ Ogni ora si recupera, con un minimo di 1 PF, il proprio valore di Costituzione in PF non letali (danni da stordimento) persi.

\riga

\textbf{Difesa Sorpresi}\\ -4 Difesa, -4 TS Riflessi

\riga

\textbf{Attacco a Tocco}\\ +1d6 al TC

\riga

\textbf{Tiro Critico}\\
Ogni qual volta hai colpito, tiri un dado arma aggiuntivo e non sommi altro per ogni due volte che hai tirato 6 nel Tiro per Colpire.

\riga


\textbf{Difesa totale}\\
2 Azioni, NO Attacco, NO Incantesimi. Terreno Difficile. No attacchi d'opportunità. +4 in Difesa.

\riga


\textbf{Disingaggiare}\\
costa 1 Azione, ti sposti di 1 metro e non causi attacchi di opportunità

\end{textblock*}



~\newpage

\begin{textblock*}{19.5cm}(1cm,1cm) % larghezza box, coord X, coord Y

\begin{tabularx}{0.95\textwidth}{llll}
\textbf{Arma}&\textbf{Costo}&\textbf{Taglia/Danno} & \textbf{Gittata, Lista, Speciale}\\
Alabarda& 10 & G/1d10 P/T& \textbf{Lance}, \textbf{Aste}, Controcarica, Arma lunga, ED9 \\
Arco Corto Composito& note*& M/Frecce& 20 metri, \textbf{Archi}\\
Arco Corto& 30 & M/1d6 P& 15 metri, \textbf{Archi}\\
Arco Lungo Composito& note*& G/Frecce& 36 metri, \textbf{Archi}\\
Arco Lungo& 75 & G/Frecce& 20 metri, \textbf{Archi}\\
Ascia Martello& 16 & M/1d6 T/B& \textbf{Asce}\\
Ascia ad una mano& 6  & M/1d6 T& 6 metri, \textbf{Asce}, \textbf{Armi da Lancio}, Versatile\\
Ascia da battaglia& 10 & M/1d10 T&\textbf{Asce}\\
Balestra ad una mano& 100& M/Dardi& 6 metri, \textbf{Balestre}\\
Balestra leggera& 35 & P/Dardi& 15 metri, \textbf{Armi Semplici}, \textbf{Balestre}\\
Balestra pesante& 50 & G/Dardi& 30 metri \textbf{Balestre}\\
Bastone& 3& M/1d6 B& \textbf{Armi Semplici}, Arma lunga, Versatile\\
Brandistocco& 10 & M/2d4 P/T& \textbf{Lance}, Controcarica, Arma lunga\\
Catena chiodata& 25 & G/2d4 P& 3 metri, \textbf{Palle rotanti}, Arma lunga\\
Estoc& 25& G/1d8 P& \textbf{Aste}, Arma lunga\\
Falce& 18 & G/2d4 P/T& \textbf{Armi della Morte}, Arma lunga\\
Falcetto& 6& P/1d6 T& \textbf{Armi della Morte}\\
Falcione in asta& 12 & G/1d10 P/T& \textbf{Lance}, Controcarica, Arma lunga, ED9\\
Falcione& 75 & M/2d4 T& \textbf{Armi Aggraziate}, ED7\\
Fionda& -& P/1d4 B& 10 metri, \textbf{Archi}\\
Flagello Doppio& 90 & M/1d10 B& \textbf{Palle Rotanti}, \textbf{Armi doppie}\\
Flagello Pesante& 15 & M/1d10 B& \textbf{Palle Rotanti}\\
Flagello& 8& M/1d8 B& \textbf{Palle Rotanti}, \textbf{Rompi Cranio}\\
Frusta& 1& M/1d3 T& \textbf{Palle Rotanti}, Arma lunga\\
Giavellotto& 1& P/1d6P& 12 metri,  \textbf{Armi Semplici}, \textbf{Aste}, \textbf{Armi da Lancio}\\
Grande Ascia Doppia& 25 & G/1d12 T& \textbf{Asce}, \textbf{Armi doppie}, Arma lunga\\
Grosso randello& 2& M/1d8 B&\textbf{Rompi Cranio}\\
Guanto chiodato& 5& P/1d4 P&\textbf{Armi da Stordimento}\\
Katana& 300& M/1d10 T& \textbf{Spade}, \textbf{Armi letali}, ED9\\
Lancia da fante& 2& M/1d8 P&3 metri, \textbf{Lance}, Arma lunga, Controcarica\\
Lancia& 10 & G/1d8 P&\textbf{Lance}, Arma lunga, Controcarica\\
Machete& 10 & M/1d6 T&\textbf{Armi letali}\\
Maglio da guerra& 7& G/1d10 B& \textbf{Rompi Cranio}\\
Manganello& 1& P/1d6 B& \textbf{Armi da stordimento}, non letale\\
Martello da guerra& 5& M/1d8 B/P& 6 metri, \textbf{Rompi Cranio}\\
Mazza Leggera& 3& P/1d6 B/T& \textbf{Armi Semplici}, \textbf{Armi Leggere}, \textbf{Rompi Cranio} \\
Mazza Pesante& 5& M/1d8 B/T& \textbf{Rompi Cranio}\\
Mazza chiodata& 6& M 1d8 B/P& \textbf{Armi Semplici}, \textbf{Rompi Cranio}\\
Naginata& 8& G/1d12 T&\textbf{Lance}, Arma lunga, ED9\\
Picca Leggera& 4& M/1d4 P&\textbf{Armi della morte}\\
Picca Pesante& 8& G/1d6 P&\textbf{Armi della morte}, Arma lunga\\
Pugnale& 2& P/1d4 P& 6 metri, \textbf{Armi Semplici}, \textbf{Armi leggere}, \textbf{Armi da Lancio}\\
Pugno/Calcio nudo& note*& P/1d4 B&Versatile\\
Randello& 1& P/1d6 B& \textbf{Armi Semplici}, \textbf{Rompi Cranio}\\
Scimitarra& 15 & M/1d6 T&\textbf{Armi Leggere}, \textbf{Armi Aggraziate}, Versatile\\
Spada Corta& 10 & P/1d6 P&\textbf{Armi Leggere}, \textbf{Spade}, Versatile\\
Spada Lunga& 15 & M/1d8 T&\textbf{Spade}\\
Spada a due lame& 100& G/1d8 T& \textbf{Armi doppie}, \textbf{Spade}\\
Spada bastarda& 35 & M/1d8-1d10 T&\textbf{Spade}, 1d8 ad una mano, 1d10 a 2 mani\\
Spada larga& 12 & M/2d4 T&\textbf{Spade}\\
Spadone a due mani& 50 & G/2d6 T&\textbf{Spade}\\
Stocco& 20 & P/1d6 P& \textbf{Armi Leggere}, \textbf{Armi Aggraziate}, Versatile\\
Tridente& 15 & M/1d6 P/T& 3 metri, \textbf{Aste}, \textbf{Armi da Lancio}, Arma Lunga, Controcarica\\
Urgrosh& 18 & M/1d6 T/P& \textbf{Lance}, \textbf{Armi doppie}\\
\end{tabularx}

\end{textblock*}

\begin{textblock*}{11.5cm}(1cm,24.5cm) % larghezza box, coord X, coord Y

\begin{tabular}{llll}
\textbf{Nome Proiettile}   & \textbf{Num./MO} & \textbf{Danno/Tipo} & Peso(kg) \\
Biglie di Marmo (fionde)   & 15/1 mo                    & 1d4 B               & 0.2      \\
Dardi da balestra, leggeri & 10/1 mo                    & 1d6 P               & 0.1      \\
Dardi per balestra pesante & 3/1 mo                     & 1d10 P              & 0.3      \\
Frecce da caccia           & 20/1 mo                    & 1d6 P               & 0.1      \\
Frecce da guerra           & 10/1 mo                    & 1d8 P               & 0.2      \\
Sasso (fionde)             & -                          & 1d2 B               & 0.2      \\
\end{tabular}

\end{textblock*}

\begin{textblock*}{7.8cm}(12.7cm,24.5cm) % larghezza box, coord X, coord Y
\textbf{Capacità di Carico e Armature}\\
La CdC è pari a 9 (P), 16 (M), 25 (G) + Forza + Costituzione.
\end{textblock*}

~\newpage

\begin{textblock*}{18cm}(1cm,1cm) % larghezza box, coord X, coord Y

\begin{tabular}{cccccccc}
\textbf{Armatura} & \textbf{Costo (mo)} & \textbf{Difesa} & \textbf{-Comp.} &  \textbf{Tipo} & \textbf{Mov.} & \textbf{Prova Magia}&\textbf{Ingombro (indossato)}\\
\hline
Imbottita & 5 mo & 1 & 0 & L & 0 & NO&2\\
Cuoio & 10 mo & 2 & 0  & L & 0 & SI&2\\
Cuoio rinforzato& 25 mo  & 3  & 0   & L & 0 & SI&2\\
Giaco di Maglia & 15 mo & 4  & -1  & M & 0 &+2&4\\
Scaglie& 50 mo & 5  & -1  & M & 0 &+2&4\\
Anelli & 150 mo & 6  & -1  & M & 0 &+2&4\\
Pettorale  & 200 mo & 6  & -2  & M & 0 &+2&4\\
Bande & 250 mo & 7  & -2  & P & 0 &+1&8\\
Mezza armatura  & 1200 mo& 8  & -2  & P & 1 &+1,2&8\\
da Campo& 1350 mo& 9  & -3  & P & 2 &+1,2&8\\
Completa& 1500 mo& 10 & -4  & P & 3 &+1,1&8\\
\end{tabular}

\riga

\begin{tabular}{lccccc}
\textbf{Scudi} & \textbf{Costo} & \textbf{Difesa} & \textbf{Penalità TC} & \textbf{Prova magia} &  \textbf{Ingombro (indossato)}\\
\hline
Scudo leggero di legno 	 & 	3 mo  	&  1	& 0& SI	  	& L\\
Scudo leggero di metallo & 	9 mo  	&  1	& 0& SI	  	& L\\
Scudo medio legno		 &	5 mo 	&  2	& 0& +2		& M\\
Scudo medio metallo	 	 &	12 mo  	&  2  	& 0& +2  	& M\\
Scudo pesante di legno   & 	9  mo  	&  3 	& 1& +1,2  	& P\\
Scudo pesante di metallo & 	20 mo  	&  3	& 1& +1,2  	& P\\
\end{tabular}

\riga

%\textbf{Tempi per indossare e togliere l'armatura}\index{Tabella Tempi per indossare e togliere l'armatura}\\

%\begin{tabular}{lccc}
%\textbf{Tipo di Armatura}& \textbf{Indossare} & \textbf{in fretta} & \textbf{Togliere}\\
%Scudo								& 1 azione 	& -     	& 1 azione\\
%Imbottita, Cuoio, Cuoio rinforzata  & 1 minuto	& 3 round  	& - \\
%Giaco di Maglia						& 1 minuto	& 5 round  & 5 round\\
%Scaglie, Anelli, Pettorale, Bande   & 4 minuti 	& 1 minuto{*}  & 1 minuto\\
%Mezza armatura, da Campo, Completa  & 4 minuti{*}{*}& 4 minuti{*}& 1d4+1 minuti\\
%\end{tabular}
%\medskip

%** è necessario qualcuno che aiuti ad indossarla

\end{textblock*}

\begin{textblock*}{12cm}(1cm,9.6cm) % larghezza box, coord X, coord Y

\begin{tabular}{ccccc}
\hline
\textbf{Cavalcatura}&\textbf{Costo (mo)}&\textbf{Mov.}&\textbf{Carico}&Km/h\\
Asino o Mulo&8&12 m&210 kg&6km\\
Cammello&50&15 m&240 kg&8km\\
Saurovallo da Galoppo&75&18 m&240 kg&12km\\
Saurovallo da Guerra&400&18 m&270 kg&9km\\
Saurovallo da Tiro&50&12 m&270 kg&6km\\
Elefante&500&12 m&660 kg&6km\\
Mastino&25&12 m&97,5 kg&6km\\
Saurovallo Nano&30&12 m&112,5 kg&6km\\
Carretto/Carro    & 15/30 mo & 9/12 m   &150/600kg    & 3/6km              \\
\end{tabular}

\riga

%\begin{tabular}{cc}
%\textbf{Contenitore}&\textbf{Capienza}\\
%Ampolla o Boccale&0,5 litri liquidi\\
%Barile&			160 litri liquidi, 4 cubi di 30 cm\\
%Borsa&			1 cubo di 10 cm/3 kg di oggetti\\
%Bottiglia&		1 litro di liquido\\
%Brocca o Caraffa&4 litri liquidi\\
%Canestro&		2 cubi di 30 cm/20 kg di oggetti\\
%Fiala&			120 ml di liquidi\\
%Forziere&		12 cubi di 30 cm/150 kg di oggetti\\
%Otre&			2 litri liquidi\\
%Sacco&			1 cubo di 30 cm/15 kg di oggetti\\
%Secchio&		12 litri liquidi, 1 cubo di 25 cm\\
%Vaso di Ferro&	4 litri liquidi\\
%Zaino*&			1 cubo di 30 cm/15 kg di oggetti\\
%\end{tabular}
%\riga



\begin{tabular}{l|cc|c}
\textbf{Fonte di} &\multicolumn{2}{c}{\textbf{Raggio in metri}}& \textbf{Durata}  \\
\textbf{Luce}& \textbf{Luce} & \textbf{Luce Fioca} &\\
Candela & - & 1 metro & 1 ora\\
Torcia & 3 metri & 6 metri & 1 ora\\
Lanterna & 6 metri & 12 metri & 3 ore \\
\multicolumn{4}{c}{\textbf{Incantesimi}}\\
Lacrima di Ljust & 1 & - & 10 round\\
Luce  & 3 metri & 6 metri &30 min. \\
Luce Diurna & 6 metri & 12 metri & 1 ora 
\end{tabular}

\riga
	
\begin{tabular}{lccc}
	\multirow{2}*{Tipo di movimento} &
	\multicolumn{3}{c}{Movimento}\\
	\cmidrule(lr){2-4} & 6m& 9m & 12m\\
	\hline
	\multicolumn{4}{c}{\textbf{Movimento (Tattico)}}\\
	Camminare& 6m & 9m & 12m\\
	Correre (x2) & 12m& 18m& 24m\\
	\multicolumn{4}{c}{\textbf{Un minuto (Locale)}} \\
	Camminare & 36m& 54m& 72m \\
	Correre (x3) & 108m & 162m & 216m \\
	\multicolumn{4}{c}{\textbf{Un'ora (Via Terra)}} \\
	Camminare& 3km& 4km& 6km\\
	Correre (x3) & 9km& 12km & 18km \\
	\multicolumn{4}{c}{\textbf{Un giorno (Via Terra)}}\\
	Camminare& 24km & 32km & 54km
\end{tabular}

\riga

\end{textblock*}

\begin{textblock*}{6cm}(13.4cm,9.6cm) % larghezza box, coord X, coord Y

\begin{tabular}{ll}
\textbf{Oggetto}&\textbf{Costo}\\
\textbf{Birra}&\\
Boccale&5 mr\\
Caraffa (4 litri)&2 ma\\
\textbf{Pietanze} &\\
Banchetto (a persona)&10 mo\\
Carne, 1 pezzo&3 ma\\
Formaggio, 1 pezzo&1 ma\\
Pane (a pagnotta)&2 mr\\
\textbf{Locanda (al giorno})&\\
Squallida&7 mr\\
Povera&1 ma\\
Modesta&5 ma\\
Agiata&8 ma\\
Ricca&2 mo\\
Aristocratica&4 mo\\
\textbf{Pasto (al giorno)}&\\
Squallido&3 mr\\
Povero&6 mr\\
Modesto&3 ma\\
Agiato&5 ma\\
Ricco&8 ma\\
Aristocratico&2 mo\\
\textbf{Vino}&\\
Buono (bottiglia)&10 mo\\
Comune (caraffa)&2 ma\\
\end{tabular}





\end{textblock*}


~\newpage

\begin{textblock*}{4cm}(1cm,1cm) % larghezza box, coord X, coord Y
{\textbf{Competenze}\\
\textbf{Forza}\\
Arrampicarsi\\
Intimidire\\
Nuotare\\
Saltare	\\
\textbf{Destrezza}\\
Acrobatica\\
Artista della fuga\\
Giocoliere\\
Mani di fata\\
Furtività\\
Usare corda	\\
\textbf{Intelligenza}\\
Arcana\\
Artigianato\\
Conoscenza*\\
Disattivare congegni\\
Erboristeria\\
Falsificare\\
Lingue\\
Valutare\\
\textbf{Saggezza}\\
Cavalcare\\
Consapevolezza\\
Gestire animali\\
Natura\\
Percepire Emozioni\\
Pronto soccorso\\
Seguire tracce\\
Sopravvivenza\\
\textbf{Carisma}\\
Diplomazia\\
Intrattenere\\
Ingannare\\
Tradizioni locali
}

\riga

\textbf{Riconoscere un incantesimo}\\ Arcana DC 10 + livello dell'incantesimo. 1 Reazione

\riga

\textbf{Valutare} 3 Azioni : DC 12 + rarità dell'oggetto, + 2 comune, 4 non comune, 6 raro, 12 molto raro, 16 leggendario. \\
Con punteggio 6 costa 2 Azioni, con 12 costa 1 Azione.

\riga

\textbf{Pronto Soccorso}\\
DC 12 + INT(-PF) per stabilizzare a 0 PF\\
2 minuti/1 p6: DC 15 recuperi 1d4 PF\\
+2 TS Tempra Veleno\\
DC 12+2xSanguinamento -1 Sanguinamento
	
\riga

\end{textblock*}


\begin{textblock*}{14.5cm}(5.5cm,1cm) % larghezza box, coord X, coord Y
\textbf{Golden Rules}\\

{\textbf{I 6 esplodono}} - se fai 6, sommi e ritiri\\
Gli \textbf{1 portano male}, se fai 1 con il dado vale zero\\
\textbf{Affidarsi alla sorte}. Ogni 4 punti tra Competenza Base o Attiva o Caratteristica = +1d6\\
\end{textblock*}


\begin{textblock*}{7.4cm}(5.3cm,3.2cm) % larghezza box, coord X, coord Y

\textbf{Intimidire}\\
1 Azione: Prova Contrapposta al Tiro Salvezza su Volontà con bonus Carisma.
Se il Tiro Salvezza fallisce, l'avversario fino alla fine del suo round successivo ha -1 al Tiro per Colpire contro colui che l'ha intimidito. Int >=-3. Il Tiro Salvezza prende un modificatore di $\pm2$ per taglia di differenza. In caso di successo critico il modificatore diventa -2.  Se chi tenta la prova di Intimidire esegue un fallimento critico subisce le medesime penalità come se fosse stato intimidito.

\riga

\textbf{Arrampicarsi - Scalare}\\
\textit{Si hanno penalità dovuta all'Armatura}

\noindent\begin{tabularx}{1\textwidth}{Xl}
	\textbf{Esempio di Superficie} & \textbf{DC}\\
\hline
	Movimento solo dimezzato & -2d6\\
	Superficie scivolosa&+4\\
	{\small Parete grezza con appigli, mattoni sporgenti}&+12\\
	Un albero, una corda senza nodi&+15\\
	Un muro con pochi mattoni sporgenti &+20\\
	Un muro con pochissimi appigli&+25\\
	Una parete naturale liscia senza appigli&+30\\
	Ti puoi appoggiare a 2 pareti opposte&-8\\
	Ti puoi appoggiare a 2 pareti angolari&-4\\
	Puoi usare una corda&-8\\
	\hline
	Usare una corda per calarsi&12\\
	Usare una corda per arrampicarsi&15\\
	La corda ha nodi & -3\\
\end{tabularx}

Se fallimento non ci si sposta. Se fallimento critico cadi e puoi fare un Tiro Salvezza su Riflessi alla stessa difficoltà per afferrarti, se fallisci il TS cadi fino in fondo. 

\riga


\textbf{Riconoscere Mostri} 1 Azione, 	DC=Sfida della creatura + 10 + rarita'
	
\textit{Arcana}: Giganti, Costrutti, Spiriti, Mostruosità, Aberrazioni, Draghi\\
\textit{Piani}: Elementali\\
\textit{Occulto}: Immondi, Spiriti, Non Morti\\
\textit{Religione}: Spiriti, Non Morti, Celestiali\\
\textit{Dungeon}: Aberrazioni, Mostruosità, Melme, e creature sotterranee\\
\textit{Natura}: Bestie, Piante, Fatati\\
	
- \textit{entro \textbf{2}}: nome, tipo, la caratt. principale|
- \textit{oltre \textbf{7}}: migliore Tiro Salvezza, 1 resistenza/immunità a Condizioni, 1 vulnerabilità a Condizioni, attacco tipico|
- \textit{oltre \textbf{12}}: peggiore Tiro Salvezza, 1 resistenze/immunità a Condizioni, 1 immunità a Danni, 1 vulnerabilità a Condizioni, 1 vulnerabilità a tipo di Danno|
- \textit{oltre \textbf{15}}: 2 immunità a Condizioni, 1 immunità a Danni, 1 vulnerabilità a Condizioni, 1 vulnerabilità a tipo di Danno|
- \textit{oltre \textbf{17}}: grado di sfida relativo |
- \textit{oltre \textbf{20}}: attacco e difese speciali|
	

\end{textblock*}


\begin{textblock*}{7cm}(13cm,3.2cm) % larghezza box, coord X, coord Y

\textbf{Prove Contrapposte}\\
Chi esegue la Prova deve fare almeno 10 + Competenza/Tiro Salvezza + Statistica + Abilità

\riga

\textbf{Identificare  Pozioni}\\
Erboristeria a DC = fattore di rarità della pianta.  1 Azione ogni 10 DC, 6p ogni 15 DC, 12p ogni 20 DC

\riga

\textbf{Riconoscere oggetto magico}\\
1 minuto DC 30. Arcana 6p costa 5 min., 12p costa 1 min., 18p costa 1 Round.
\riga

\textbf{Saltare} 1 Azione\\
\textit{Si hanno penalità dovuta all'Armatura}

\textbf{Distanza saltata in lungo}: 30cm per risultato\\

\textbf{Distanza saltata in alto}: 10cm per risultato\\

Rincorsa 3 metri altrimenti metà.

\riga

\textbf{Danno Caduta}: H(m)/3xD6. Ogni 3 dadi oltre i 20 aggiungete 6 di danno (X/3)d6+(X/3-20)*6. Proni. Prova Acrobatica DC 15 1/2 danno entro 9m.  Cadute su superfici morbide (terreno morbido, fango ecc.) -1d6 danni.

\riga

\textbf{Nuotare}\\
\textit{Penalità dovuta all'Armatura su Forza}\\
Acqua calme DC 10.\\
Acque mosse ha DC 15\\
Acque molto mosse ha DC 20\\
Acque tempestose DC 25

\riga


\textbf{Seguire Tracce}
\begin{tabular}{ll}
Se il terreno è molto morbido& DC -4\\
Se il terreno è stabile/duro& DC +15/20\\
Se il terreno è duro& DC +20\\
A seconda della taglia& DC $\pm4$\\
Ogni 3 creature inseguite& DC -2\\
Ogni 24 ore passate& DC +4\\
Ogni ora di pioggia& DC +4\\
Visibilità scarsa& DC +2\\
Cerca di occultare le tracce& DC +4\\
\end{tabular}\\

\riga

\textbf{Artista della Fuga}\\
1 Azione ogni 10 di DC. 6p 1 Azione 15 di DC, 12p 1 Azione 20 DC.

\end{textblock*}

~\newpage

\begin{textblock*}{5cm}(1cm,1cm) % larghezza box, coord X, coord Y
\textbf{Prova di Magia}\\
3d6 + Aggiungi 1d6 ogni 2 Liste di magia conosciuta, ignori un 1 per ogni 2 Adepto della Magia\\
Fallimento Critico: due volte 1, un 1 e due 2\\


\riga

\textbf{Distratto}\\
Sei Distratto se: occulti il lancio di incantesimo, severamente Distratto, impedito, sanguinante, afferrato, è sotto attacco\\

\riga

\textbf{Punti Magia}\\
Mod. Caratteristica + \\

\begin{tabular}{ll|ll}
\textbf{CM} & \textbf{P.M}&	\textbf{CM} & \textbf{P.M}\\
	1&	2 	&11&43\\
2&	4	&12&47\\
3&	8	&13&50\\
4&	10	&14&54\\
5&	16	&15&58\\
6&	19	&16&62\\
7&	23	&17&71\\
8&	27	&18&76\\
9&	36	&19&82\\
10&	41	&20&89\\
20+&prec.+ 4&&\\
\end{tabular}

\riga

\textbf{Tiro Salvezza Incantesimo}\\
DC = 10 + Competenza Magica + modificatore caratteristica per incantesimo + 1 x Adepto della Magia +1 x Critico nella Prova di Magia

\riga

\textbf{Tiro Salvezza Magia da Oggetto}\\
DC = 10 + 2 x livello incantesimo manifestato

\riga

\textbf{Tiro Salvezza Incantesimo Mostro}\\
DC è 10 + 2 x livello incantesimo + Intelligenza

\riga

\textbf{Opzionale Successo Critico Auto matico}:  x2 costo Punti Magia cumulativo. Es. 4,8,16,32..

\riga

\textbf{Leggere una Pergamena}
	
	ISY SCROLL: Comprendere: Intelligenza od Arcana DC 10
	
	Lanciare: Intelligenza od Arcana DC 12.
	
	normali: Comprenderne: Arcana a difficoltà 15
	
	- Lanciare: Arcana DC 20 ed avere accesso alla Lista di Magia

\end{textblock*}


\begin{textblock*}{13.5cm}(6.5cm,1cm) % larghezza box, coord X, coord Y
\textbf{Seguace}\\
1 Tratti comuni con Patrono. Se sei un Seguace ottieni +1d6 alle Prove di Magia nella scuola preferita dal Patrono. Puoi usare l'energia preferita del Patrono nei tuoi incantesimi.\\

\riga

\textbf{Devoto}\\
2 Tratti in comune con Patrono. Un Devoto aggiunge +1d6 alla Prova di Magia nelle scuole preferite dal Patrono e ignora un dado tirato nella Prova di Magia. Devi usare l'energia preferita del Patrono nei tuoi incantesimi.

\riga

\textbf{Fallimento Critico Prova di Magia - 3d6 -1d6 x Fallimento Crit. Min. 1d6}
\begin{tabularx}{0.95\textwidth}{lX}
1 & Per 1 giorno non sei più in grado di canalizzare energie magiche. Non puoi lanciare incantesimi se non facendo un successo magico critico nella Prova di Magia\\
2 & Aumenti la condizione di Affaticato di 2 gradi\\
3 & Manifesti una modifica corporea minore\\
4 & Vieni investito da una roboante colonna di Luce e Vuoto. In un raggio di 3 metri centrato su di te, chiunque deve fare un Tiro Salvezza su Riflessi DC 15 per dimezzare o subire 1d6 di danni per livello dell'incantesimo\\
5 & Per 3 round sei sotto l'influenza dell'incantesimo Confusione\\
6 & Perdi la concentrazione su qualsiasi incantesimo e per un minuto parli in rima\\
7 & Vieni teletrasportato di 3d10 metri in una direzione casuale\\
8 & Solo tu vieni avvolto da una cortina di oscurità magica impenetrabile per 6 round\\
9 & Diventi Invisibile ed incapace di parlare per 6 round\\
10 & Non riesci a parlare bene, sei balbuziente. Ogni lancio di incantesimi ti costringe a superare una Prova di Magia. Durata 3 round\\
11 & Manifesti l'incantesimo Unto sotto i tuoi piedi\\
12 & Il prossimo incantesimo che lanci ha effetti se possibile minimizzati\\
13 & Il battito del tuo cuore è come il battito di un tamburo, si può sentire entro 36 metri\\
14 & Tutte le creature nel raggio di 36 metri sanno esattamente dove sei e cosa tentavi di fare.\\
15 & Tutte le creature in una semisfera di 9 metri di raggio centrata su di te subiscono 1d10 danni da Vuoto.\\
16 & Guadagni 2d6 Punti Magia\\
17 & Una incudine cade, 3d6 di danno Tiro Salvezza su Riflessi DC 15 per dimezzare, su una creatura a caso, escluso te, entro sei metri\\
18 & Le creature, te escluso, nel raggio di 6 metri da te subiscono 3d10 danni da forza non resistibili
\end{tabularx}

\riga

\textbf{Massimo Livello Incantesimo lanciabile}

Sommate a CM le volte che avete preso Adepto della Magia e dividete il risultato per 2. Es. CM=8, Adepto della Magia preso 4 volte, (8+4)/2=6, oppure CM=13 e Adepto della Magia 1 volta, (13+1)/2=7 livello di incantesimi.

\riga

\textbf{Alterare la Magia}

- \textbf{Magie Punitive}: 2 volte costo magia, +1 dado nella Prova di Magia. Reazione

- \textbf{Magie efficaci}: 3 volte costo magia, -1 dado nella Prova di Magia. Reazione

- \textbf{Magia eterea}\index{Magia eterea}: +3 Punti Magia, colpisci creature eteree o incorporee. Azione Immediata.

- \textbf{Sacrificio Magico}\index{Sacrificio magico}: -4 PF +1 Punto Magia. Non puoi sacrificare più di metà dei Punti Ferita attuali alla volta. Azione Immediata.

- \textbf{Magia pietosa}\index{Magia pietosa}: +3 Punti Magia e danni temporanei. 1 Azione.

- \textbf{Magia mirata}\index{Magia mirata}: ogni due volte Adepto della Magia escludi 1 target da incantesimo. Costo 2 Punti Magia per creatura esclusa. 1 Azione.

- \textbf{Magia lontana}\index{Magia lontana}: +1 Punti Magia aumenti fino a 9 metri la distanza di lancio. 1 Azione.

- \textbf{Aumentare il tempo}\index{Aumentare il tempo} +1 Azione,  -1 Punto Magia

- \textbf{Circolo del Potere}\index{Circolo del Potere}: vedi descrizione

\riga

\end{textblock*}




~\newpage

\end{document}
