\documentclass{article}
\usepackage[utf8]{inputenc}
\usepackage[italian]{babel}
\usepackage{xltabular,tabularx}
\usepackage{array}
\usepackage{booktabs}
\usepackage{multicol}



\begin{document}

\begin{multicols}{2}

\section{Come Tirare per il Tesoro}

Il Signore del Labirinto sceglie la riga appropriata nella tabella delle Classi di Tesoro, e per ogni colonna tira i dadi appropriati per determinare se certi tesori si verificano, e se sì in che quantità. Quando sono indicati oggetti magici, il Signore del Labirinto tira quindi sulla tabella del tesoro appropriata per determinare quali oggetti magici specifici vengono trovati. Se è indicata la presenza di oggetti magici, ma non è indicato alcun tipo specifico, il tipo è determinato tirando sulla tabella Tipo di Magia Casuale.

\section{Gemme}

Quando vengono trovate gemme, il Signore del Labirinto tirerà per determinare il loro valore in monete d'oro. Tutte le gemme possono essere assegnate allo stesso valore, possono ricevere valori individuali, o possono essere divise in gruppi e ricevere valori diversi.

\begin{tabular}{|c|c|}
\hline
\textbf{Tiro d\%} & \textbf{Valore Gemma (mo)} \\
\hline
01-15 & 10 \\
16-30 & 25 \\
31-45 & 50 \\
46-60 & 75 \\
61-75 & 100 \\
76-85 & 250 \\
86-90 & 500 \\
91-95 & 750 \\
96-00 & 1.000 \\
\hline
\end{tabular}

\section{Gioielli}

I gioielli possono variare in valore in modo simile alle gemme. La tabella sottostante può essere usata per determinare il valore di ogni singolo pezzo di gioielleria.

\begin{table}[h]
\centering
\begin{tabular}{|c|c|}
\hline
\textbf{Tiro d\%} & \textbf{Valore Gioielli (mo)} \\
\hline
01-20 & 1d4×10 \\
21-30 & 2d4×10 \\
31-40 & 1d4×100 \\
41-50 & 2d4×100 \\
51-60 & 2d6×100 \\
61-00 & 3d6×100 \\
\hline
\end{tabular}

\end{table}

\section{Trovare e Usare Oggetti Magici}

La maggior parte degli oggetti magici non sono etichettati, quindi i personaggi non conosceranno le proprietà esatte degli oggetti magici se non attraverso tentativi ed errori. Utilizzatori di magia potenti (sopra il 15° livello) possono identificare gli oggetti magici, ma può richiedere molte settimane per farlo. Le pozioni possono essere identificate assaggiandole, o consultando un alchimista. Per usare un oggetto magico, un personaggio deve seguire qualsiasi procedura indicata nella descrizione dell'oggetto. Alcuni oggetti magici sono sempre in effetto, ma altri possono richiedere azioni speciali o concentrazione. Alcuni oggetti magici hanno usi limitati, chiamati "cariche". Quando gli oggetti hanno cariche, ogni carica può essere spesa per un'istanza di effetto magico. Un personaggio non saprà quante cariche ha un oggetto, e quando le cariche sono tutte spese l'oggetto diventa inutile e non magico.

Gli oggetti contrassegnati "Base" indicano che è apparso per la prima volta nelle regole base fondamentali del Signore del Labirinto, e a discrezione del referee potrebbero essere trattati come più rari.

\section{Tabelle del Tesoro}

\subsection{Tipo di Magia Casuale}

\begin{table}[h]
\centering
\begin{tabular}{|c|l|}
\hline
\textbf{Tiro d00} & \textbf{Tipo di Magia} \\
\hline
01-20 & Pozioni \\
21-25 & Anelli \\
26-56 & Pergamene \\
57-61 & Bastoni, Verghe e Bacchette \\
62-66 & Magia Varia \\
67-87 & Spade \\
88-92 & Armi Varie \\
93-00 & Armature \\
\hline
\end{tabular}

\end{table}

\subsection{Pozioni}

\begin{tabular}{|c|l|}
\hline
\textbf{Tiro d00} & \textbf{Pozione} \\
\hline
01-03 & Controllo Animali \\
04-06 & Chiaroudienza \\
07-09 & Chiaroveggenza \\
10-12 & Scalata \\
13-17 & Delusione \\
18-20 & Diminuzione \\
21-23 & Controllo Draghi \\
24-26 & ESP \\
27-28 & Cura Extra \\
29-31 & Resistenza al Fuoco \\
32-36 & Volo \\
37-40 & Forma Gassosa \\
41-43 & Controllo Giganti \\
44-47 & Forza del Gigante \\
48-50 & Crescita \\
51-54 & Cura \\
55-58 & Eroismo \\
59-61 & Controllo Umani \\
62-64 & Invisibilità \\
65-66 & Invulnerabilità \\
67-69 & Levitazione \\
70-71 & Longevità \\
72-73 & Olio di Etereità \\
74-75 & Olio di Scivolosità \\
76-78 & Filtro d'Amore \\
79-81 & Controllo Piante \\
82-83 & Veleno \\
84-85 & Polimorfismo \\
86-88 & Velocità \\
89-90 & Super-eroismo \\
91-93 & Acqua Dolce \\
94-95 & Ricerca Tesori \\
96-97 & Controllo Non Morti \\
98-00 & Respirare sott'Acqua \\
\hline

\end{tabular}



\section{Anelli}

\begin{tabular}{|c|l|}
\hline
\textbf{Tiro d00} & \textbf{Anello} \\
\hline
01-04 & Comando Animali \\
05-09 & Comando Umani \\
10-15 & Comando Piante \\
16-25 & Delusione \\
26-27 & Evocazione Djinn \\
28-38 & Resistenza al Fuoco \\
39-49 & Invisibilità \\
50-70 & Protezione \\
71-72 & Rigenerazione \\
73-74 & Immagazzinamento Incantesimi \\
75-79 & Deviazione Incantesimi \\
80-81 & Telecinesi \\
82-87 & Camminare sull'Acqua \\
88-94 & Debolezza \\
95-97 & Desideri \\
98-00 & Visione a Raggi X \\
\hline

\end{tabular}

\section{Pergamene}

\begin{tabular}{|c|l|}
\hline
\textbf{Tiro d00} & \textbf{Pergamena} \\
\hline
01-05 & Maledetta \\
06-15 & Protezione contro Elementali \\
16-25 & Protezione contro Licantropi \\
26-30 & Protezione contro Magia \\
31-40 & Protezione contro Non Morti \\
41-55 & Incantesimi (1)* \\
56-66 & Incantesimi (2)* \\
67-69 & Incantesimi (3)* \\
70-72 & Incantesimi (4)* \\
73-74 & Incantesimi (5)* \\
75 & Incantesimi (6)* \\
76 & Incantesimi (7)* \\
77-80 & Mappa del Tesoro (Valore 1d4×1000 mo) \\
81-85 & Mappa del Tesoro (Valore 5d6×1000 mo) \\
86-87 & Mappa del Tesoro (Valore 6d6×1000 mo) \\
88-89 & Mappa del Tesoro (Valore 5d6×1000 mo, 5d6 gemme) \\
90-91 & Mappa del Tesoro (Valore 1d6 gemme, 2d10 gioielli) \\
92-93 & Mappa del Tesoro (Valore 1 oggetto magico) \\
94-95 & Mappa del Tesoro (Valore 2 oggetti magici) \\
96 & Mappa del Tesoro (Valore 3 oggetti magici, nessuna arma) \\
97 & Mappa del Tesoro (Valore 3 oggetti magici, +1 pozione) \\
98 & Mappa del Tesoro (Valore 3 oggetti magici, +1 pozione, +1 Pergamena) \\
99 & Mappa del Tesoro (Valore 5d6×1000 mo, 1 oggetto magico) \\
00 & Mappa del Tesoro (Valore 5d6 gemme, 2 oggetti magici) \\
\hline
\multicolumn{2}{l}{*Tira 1d4; 1-3, Mago/Elfo; 4, Chierico. Il numero tra parentesi} \\
\multicolumn{2}{l}{è il numero di incantesimi sulla pergamena. Determina il livello} \\
\multicolumn{2}{l}{dell'incantesimo e gli incantesimi specifici casualmente.} \\
\end{tabular}

\section{Bastoni, Verghe e Bacchette}

\begin{tabular}{|c|l|}
\hline
\textbf{Tiro d00} & \textbf{Tipo} \\
\hline
01-04 & Verga di Assorbimento \\
05-08 & Verga di Cancellazione (Base) \\
09-10 & Verga di Fascinazione \\
11-13 & Verga del Potere Signorile \\
14-16 & Verga di Resurrezione \\
17-18 & Verga del Comando \\
19-21 & Verga di Percussione \\
22-23 & Bastone del Comando [C] (Base) \\
24-27 & Bastone di Cura [C] (Base) \\
28 & Bastone dei Magi \\
29-31 & Bastone del Potere [M/E] (Base) \\
32-34 & Bastone di Percussione [C] (Base) \\
35-36 & Bastone dell'Appassimento [C] (Base) \\
37-38 & Bastone della Stregoneria [M/E] (Base) \\
39-43 & Bastone del Serpente [C] (Base) \\
44-45 & Bacchetta del Freddo (Base) \\
46-48 & Bacchetta di Individuazione Nemici (Base) \\
49-51 & Bacchetta di Individuazione Magia (Base) \\
52-54 & Bacchetta di Individuazione Metalli (Base) \\
55-57 & Bacchetta di Individuazione Porte Segrete (Base) \\
58-60 & Bacchetta di Individuazione Trappole (Base) \\
61-62 & Bacchetta di Negazione Dispositivi (Base) \\
63-64 & Bacchetta della Paura (Base) \\
65-66 & Bacchetta del Fuoco \\
67-69 & Bacchetta delle Palle di Fuoco (Base) \\
70-72 & Bacchetta del Ghiaccio \\
73-75 & Bacchetta dell'Illusione (Base) \\
76-78 & Bacchetta della Luce \\
79-80 & Bacchetta del Fulmine \\
81-83 & Bacchetta dei Fulmini (Base) \\
84-87 & Bacchetta dei Dardi Magici (Base) \\
88-89 & Bacchetta di Negazione \\
90-92 & Bacchetta di Paralizzazione (Base) \\
93-95 & Bacchetta di Polimorfismo (Base) \\
96-97 & Bacchetta di Evocazione \\
98-00 & Bacchetta delle Meraviglie \\
\hline

\end{tabular}

\section{Determinazione delle Tabelle degli Oggetti Magici Vari}

Quando il tesoro indica un oggetto magico vario, tira sotto per determinare quale tabella di Oggetti Magici Vari usare.

\begin{table}[h]
\centering
\begin{tabular}{|c|l|}
\hline
\textbf{Tiro d20} & \textbf{Tabella} \\
\hline
1-4 & Oggetti Magici Vari A-B \\
5-8 & Oggetti Magici Vari C-F \\
9-12 & Oggetti Magici Vari G-J \\
13-16 & Oggetti Magici Vari L-P \\
17-20 & Oggetti Magici Vari R-W \\
\hline
\end{tabular}

\end{table}


\section{Oggetti Magici Vari R-W}

\begin{tabular}{|c|l|}
\hline
\textbf{Tiro d00} & \textbf{Oggetto} \\
\hline
01-02 & Veste dell'Arcimago \\
03-06 & Veste di Mescolamento \\
07-10 & Veste degli Occhi \\
11-14 & Veste di Impotenza \\
15-18 & Veste di Colori Scintillanti \\
19-22 & Veste di Oggetti Utili \\
23-25 & Corda Rampicante (Base) \\
26-28 & Corda di Intrappolamento \\
29-31 & Corda di Strangolamento \\
32-34 & Scarabeo della Morte \\
35-37 & Scarabeo di Protezione (Base) \\
38-41 & Pantofole di Scalata del Ragno \\
42-43 & Sfera di Annichilimento \\
44-46 & Pietra di Controllo Elementali della Terra (Base) \\
47-49 & Pietra della Buona Fortuna (Pietra Fortunata) \\
50-53 & Pietra del Peso (Pietra di Carico) \\
54-57 & Talismano del Puro Bene \\
58-60 & Talismano della Sfera \\
61-64 & Talismano del Male Supremo \\
65-68 & Tomo del Pensiero Chiaro \\
70-73 & Tomo della Conoscenza \\
74-77 & Tomo di Comando e Influenza \\
78-81 & Tomo di Conoscenza Marziale \\
82-85 & Tomo di Furtività \\
86-89 & Tomo di Comprensione \\
90-93 & Pozzo di Molti Mondi \\
94-00 & Ali del Volo \\
\hline

\end{tabular}

\section{Spade}

\begin{tabular}{|c|l|}
\hline
\textbf{Tiro d00} & \textbf{Oggetto} \\
\hline
01-14 & Spada +1 \\
15-20 & Spada +1, +2 contro licantropi (Base) \\
21-26 & Spada +1, +2 contro incantatori (Base) \\
27-32 & Spada +1, +3 contro non morti (Base) \\
33-38 & Spada +1, +3 contro draghi (Base) \\
39-44 & Spada +1, +3 contro mostri rigeneranti (Base) \\
45-50 & Spada +1, +3 contro mostri magici (Base) \\
51-52 & Spada +1, Danzante \\
53 & Spada +1, Smembrante \\
54-55 & Spada +1, Ferente \\
56-58 & Spada +1, luce raggio 9 metri \\
59 & Spada +1, Lingua di Fiamma (Base) \\
60 & Spada +1, Succhia Vita (Base) \\
61-62 & Spada +1, localizza oggetti \\
63 & Spada +1, Lama della Fortuna (Base) \\
64 & Spada +1, Lama dei Desideri (Base) \\
65-74 & Spada +2 \\
75-80 & Spada +2, +3 contro giganti \\
81 & Spada +2, Berserker \\
82 & Spada +2, Vendicatrice Sacra \\
83 & Spada +2, Ruba Nove Vite \\
84-85 & Spada +2, ammalia persone \\
86-87 & Spada +3 \\
88-89 & Spada +3, Marchio del Gelo (Base) \\
90 & Spada +4, Difensiva \\
91-95 & Spada –1 (maledetta) \\
96-99 & Spada –2 (maledetta) \\
00 & Spada Senziente* (Base) \\
\hline
\multicolumn{2}{l}{*Queste spade dovrebbero essere molto rare e usate con discrezione.} \\

\end{tabular}


\section{Armi Varie}

\begin{tabular}{|c|l|}
\hline
\textbf{Tiro d00} & \textbf{Arma} \\
\hline
01-05 & Frecce +1 (quantità 2d6) \\
06-09 & Frecce +1 (quantità 3d10) \\
10-12 & Frecce +2 (quantità 1d6) \\
13-14 & Frecce +3 (quantità 1d4) \\
15-16 & Freccia +3, Freccia Mortale (Base) \\
17-21 & Ascia +1 \\
22-24 & Ascia +2 \\
25-28 & Arco +1 \\
29-33 & Quadrelli da Balestra +1 (quantità 2d6) \\
34-37 & Quadrelli da Balestra +1 (quantità 3d10) \\
38-40 & Quadrelli da Balestra +2 (quantità 1d6) \\
41-42 & Quadrelli da Balestra +3 (quantità 1d4) \\
43-45 & Pugnale -1, maledetto \\
46-53 & Pugnale +1 \\
54 & Pugnale +1, Velenoso \\
55-56 & Pugnale +2, +3 contro goblin, coboldi e orchi \\
57 & Pugnale +2, Assassino \\
58-62 & Mazza +1 \\
63-64 & Mazza +1, Distruzione \\
65-67 & Mazza +2 \\
68-72 & Fionda +1 \\
73-76 & Lancia +1 \\
77-79 & Lancia +2 \\
80-81 & Lancia +3 \\
82-84 & Tridente +1, Comando Pesci \\
85-87 & Tridente +2, Avvertimento \\
88-94 & Martello da Guerra +1 \\
95-98 & Martello da Guerra +2 \\
99-00 & Martello da Guerra +2, Lanciatore Nanico (Base) \\
\hline

\end{tabular}

\section{Armature}

\begin{tabular}{|c|l|}
\hline
\textbf{Tiro d00} & \textbf{Tipo} \\
\hline
01-15 & Armatura +1 \\
16-25 & Armatura +1 e Scudo +1 \\
26-27 & Armatura +1 e Scudo +2 \\
28 & Armatura +1 e Scudo +3 \\
29-32 & Armatura +2 \\
33-35 & Armatura +2 e Scudo +1 \\
36-38 & Armatura +2 e Scudo +2 \\
39 & Armatura +2 e Scudo +3 \\
40 & Armatura +3 \\
41 & Armatura +3 e Scudo +1 \\
42 & Armatura +3 e Scudo +2 \\
43 & Armatura +3 e Scudo +3 \\
44-63 & Scudo +1 \\
64-73 & Scudo +2 \\
74-79 & Scudo +3 \\
80-82 & Armatura –1 (maledetta) \\
83-85 & Armatura –2 (maledetta) \\
86 & Armatura –1 (maledetta) e Scudo +1 \\
87 & Armatura –2 (maledetta) e Scudo +1 \\
88-90 & Armatura CA 9 (maledetta) \\
91-94 & Scudo –1 (maledetto) \\
95-97 & Scudo –2 (maledetto) \\
98-00 & Scudo CA 9 (maledetto) \\
\hline

\end{tabular}

\section{Tabelle per Determinare il Tipo di Armatura Magica}

Quando viene trovata un'armatura magica, usa la tabella sottostante per determinare il tipo di armatura e le sue proprietà.

\begin{table}[h]
\centering
\begin{tabular}{|c|l|c|c|}
\hline
\textbf{Tiro d00} & \textbf{Tipo di Armatura} & \textbf{CA Non Modificata} & \textbf{Peso Magico (kg)} \\
\hline
01-10 & Cotta di maglia a bande & 4 & 7 \\
11-30 & Cotta di maglia & 5 & 9 \\
31-60 & Cuoio & 8 & 4,5 \\
61-67 & Imbottita & 8 & 2 \\
68-85 & Corazza a piastre & 3 & 11 \\
86-90 & Cotta di scaglie & 6 & 7 \\
91-95 & Cotta di piastre & 4 & 9 \\
96-00 & Cuoio borchiato & 7 & 7 \\
\hline
\end{tabular}

\end{table}

\section{Descrizioni degli Oggetti Magici}

\subsection{Pozioni}

Sebbene le pozioni possano essere trovate in una varietà di tipi di contenitori, inclusi vasi di vetro, ceramica o metallo, la maggior parte contiene solo una dose che conferisce gli effetti particolari della pozione per un individuo. La maggior parte delle pozioni non porta etichette e richiede che una piccola quantità venga assaggiata per tentare di identificare il tipo di pozione. Questo non è privo di errori, tuttavia, poiché pozioni dello stesso tipo possono differire nel loro aroma o sapore a seconda di come sono state create.

Come regola standard, le pozioni hanno effetto nello stesso round del loro consumo, e durano per 1d6+6 turni. Questo principio generale viene sostituito dove la descrizione specifica della pozione indica altrimenti. Le pozioni generalmente possono essere consumate a metà dose, così che metà della pozione viene consumata e opera per metà della durata.

La creazione di pozioni richiede gli sforzi congiunti di utilizzatori di magia e alchimisti. Un campione della pozione da creare deve essere ottenuto a un certo punto per imparare la formula per quella particolare pozione.

\textbf{Controllo Animali}: Chiunque prenda questa pozione guadagna la capacità di relazionarsi, comprendere e manipolare le emozioni di un tipo particolare di animale. Il tipo di animale è determinato da un tiro di dado (vedi sotto), e il numero di animali influenzati dipende dalla taglia dell'animale. Si applicano le seguenti taglie e quantità generali: taglia lupo o inferiore, 5d4; fino a taglia umana, 3d4; animali fino a 450 kg, 1d4.

\begin{table}[h]
\centering
\begin{tabular}{|c|l|}
\hline
\textbf{Tiro 1d20} & \textbf{Tipo di Animale} \\
\hline
1-4 & Aviari \\
5-7 & Pesci \\
8-11 & Mammiferi, inclusi marsupiali \\
12-13 & Qualsiasi mammifero e aviario \\
14-17 & Anfibi e rettili \\
18-19 & Anfibi, rettili e pesci \\
20 & Tutti quelli elencati sopra \\
\hline
\end{tabular}

\end{table}

Nota che a meno che il bevitore di questa pozione non abbia qualche altro mezzo di comunicare direttamente con gli animali influenzati dalla pozione, solo emozioni generali o inclinazioni possono essere manipolate. Tutti gli umanoidi non sono influenzati da questa pozione, e qualsiasi creatura intelligente può fare un tiro salvezza per resistere ai suoi effetti.

\textbf{Chiaroudienza}: Questa pozione garantisce al bevitore la capacità di sentire fino a 18 metri per mezzo delle orecchie di un animale. Un animale deve essere in relativa prossimità. Tuttavia, una barriera di piombo ostacola questo effetto.

\textbf{Chiaroveggenza}: Questa pozione garantisce al bevitore la capacità di vedere fino a 18 metri per mezzo degli occhi di un animale. Un animale deve essere in relativa prossimità. Tuttavia, una barriera di piombo ostacola questo effetto.

\textbf{Scalata}: Questa pozione dura per 1 turno + 5d5 round, durante i quali il bevitore guadagna la capacità di scalare come un ladro con abilità del 99\% (un tiro di 00 significa fallimento). Tuttavia, le probabilità di fallimento aumentano del 5\% se il personaggio trasporta 45 kg o più. Inoltre, il tipo di armatura indossata influenzerà l'abilità di scalata diminuendo la percentuale di possibilità di successo come segue:

\begin{table}[h]
\centering
\begin{tabular}{|l|c|}
\hline
\textbf{Tipo di Armatura} & \textbf{Penalità} \\
\hline
Qualsiasi tipo di armatura magica & 1\% \\
Cuoio borchiato & 1\% \\
Cotta di maglia ad anelli & 2\% \\
Cotta di scaglie & 4\% \\
Cotta di maglia & 7\% \\
Armatura a bande e a piastre & 8\% \\
Corazza a piastre & 10\% \\
\hline
\end{tabular}

\end{table}

\textbf{Delusione}: Questa pozione è appropriatamente chiamata, poiché convince il bevitore che la pozione sia di un altro tipo. Se più di una persona assaggia questa pozione, c'è una probabilità del 90\% che tutti crederanno che la pozione sia dello stesso tipo. Per esempio, una pozione di chiaroudienza potrebbe convincere il bevitore che ci sono suoni in lontananza che non esistono veramente.

\textbf{Diminuzione}: Quando bevuta, il bevitore e tutto ciò che trasporta si rimpicciolisce a 15 centimetri di altezza. Il personaggio è così piccolo che se rimane immobile c'è solo una probabilità del 10\% di essere individuato dalle creature vicine. Se solo metà della pozione viene consumata, rimpicciolirà il bevitore al 50\% della sua taglia originale.

\textbf{Controllo Draghi}: Bere questa pozione garantisce al bevitore un potere equivalente ad ammalia mostri su un drago di un tipo determinato dalla tabella sottostante. Ogni pozione influenza solo un tipo di drago. Il bevitore è in grado di controllare un drago entro 18 metri e per la durata di 5d4 round. Tira sulla tabella sottostante per il tipo specifico di pozione di controllo draghi.

\begin{table}[h]
\centering
\begin{tabular}{|c|l|}
\hline
\textbf{Tiro 1d10} & \textbf{Tipo di Drago} \\
\hline
1-2 & Nero \\
3 & Blu \\
4-5 & Verde \\
6 & Rosso \\
7-9 & Bianco \\
0 & Dorato \\
\hline
\end{tabular}

\end{table}

\textbf{ESP}: Questa pozione garantisce un'abilità simile a un incantesimo equivalente all'incantesimo ESP di mago ed elfo per la durata di 5d8 round.

\textbf{Cura Extra}: Bere la dose completa di questa pozione ripristina i danni fino a 3d6+3 punti ferita. A differenza della maggior parte delle altre pozioni, questa pozione può essere bevuta in tre porzioni separate e uguali per il beneficio di 1d6 punti ferita di cura per ogni terzo della pozione.

\textbf{Resistenza al Fuoco}: Il bevitore di questa pozione è impermeabile a tutte le forme di fiamma ordinaria, che sia piccola come una torcia o grande come un rogo furioso, per 1 turno. Inoltre, questa pozione riduce i danni da altri tipi di fuoco di –2 per dado di danno. Questi tipi di fuoco includono palla di fuoco, muro di fuoco, e il calore intenso della roccia fusa. Se l'esposizione a queste fiamme richiede un tiro salvezza, viene fatto con +2 al tiro di dado. Metà della pozione può essere bevuta per resistenza che dura 5 round, e altri bonus forniti sono dimezzati (-1 al danno e +1 ai tiri salvezza).

\textbf{Volo}: Questa pozione garantisce l'abilità simile a un incantesimo equivalente all'incantesimo volo di mago ed elfo.

\textbf{Forma Gassosa}: La persona che beve questa pozione, oltre a tutti gli oggetti sulla sua persona, assume una consistenza gassosa traslucida e fluttua a 9 metri per round. Questa velocità può essere diversa a seconda della velocità naturale del vento nell'ambiente o a causa degli effetti di incantesimi che alterano il vento. Mentre in forma gassosa, la persona influenzata può fluire sotto le porte e altri piccoli spazi che non sono sigillati ermeticamente. Sebbene il fulmine e il fuoco magici facciano alla forma gassosa il danno completo, mentre in forma gassosa il bevitore è altrimenti impermeabile ad altri attacchi. Questa pozione deve essere bevuta completamente per avere effetto.

\subsection{Pozioni (Continuazione)}

\textbf{Controllo Giganti}: Quando bevuta, il bevitore è in grado di controllare fino a due giganti nello stesso modo dell'incantesimo ammalia mostri per 5d6 round. È permesso un tiro salvezza, e se solo un gigante è il bersaglio riceve –4 a questo tiro. Se due giganti sono bersaglio, ricevono +2 a questo tiro. Ogni pozione di controllo giganti influenza solo un tipo di gigante. Consulta la tabella sottostante.

\begin{table}[h]
\centering
\begin{tabular}{|c|l|}
\hline
\textbf{Tiro 1d20} & \textbf{Tipo di Gigante} \\
\hline
1-2 & Delle Nuvole \\
3-6 & Del Fuoco \\
7-10 & Del Gelo \\
11-15 & Delle Colline \\
16-19 & Di Pietra \\
20 & Delle Tempeste \\
\hline
\end{tabular}

\end{table}

\textbf{Forza del Gigante}: Il bevitore di questa pozione diventa temporaneamente forte come un gigante del gelo. Questo bonus in forza è accompagnato dalla capacità gigante di lanciare rocce agli avversari, a una distanza di 60 metri per 3d6 punti ferita di danno. Inoltre, il personaggio fa danno doppio con gli attacchi delle armi. I bonus di forza di questa pozione non possono essere combinati con altri effetti magici che influenzano la forza.

\textbf{Crescita}: Il bevitore di questa pozione raddoppia di taglia. La forza aumenta anche, così che tutto il danno inflitto è raddoppiato.

\textbf{Cura}: Il bevitore di questa pozione riguadagna danni pari a 1d6+1 punti ferita. Questa pozione cura anche la paralisi. Questa pozione può essere bevuta solo completamente per avere effetto.

\textbf{Eroismo}: Solo un nano, halfling o guerriero può usare questa pozione. Livelli extra e i loro benefici accompagnatori al combattimento sono temporaneamente garantiti al bevitore, determinati dal suo livello di esperienza come mostrato nella tabella sottostante. Nota che i punti ferita extra garantiti a causa dell'aumento di livello vengono sottratti per primi quando il personaggio viene ferito.

\begin{table}[h]
\centering
\begin{tabular}{|c|c|}
\hline
\textbf{Livello Bevitore} & \textbf{Livelli Garantiti} \\
\hline
0 & 4 (Guerriero) \\
1-3 & 3 \\
4-7 & 2 \\
8-10 & 1 \\
\hline
\end{tabular}

\end{table}

\textbf{Controllo Umani}: Una volta bevuta, questa pozione garantisce l'abilità simile a un incantesimo di ammalia persone al bevitore per 5d6 round. Molti tipi di umanoidi, semi-umani e umani possono essere influenzati da questa pozione (vedi la tabella sottostante), e 32 dadi vita/livelli di questi esseri sono influenzati. Solo dadi vita completi sono considerati quando si calcola quanti individui sono influenzati, e qualsiasi bonus viene scartato (3 + 1, 4 + 2 sono trattati come 3, 4). Il tipo specifico di essere simile a umano influenzato per ogni pozione è determinato sulla tabella sottostante.

\begin{table}[h]
\centering
\begin{tabular}{|c|l|}
\hline
\textbf{Tiro 1d12} & \textbf{Umanoidi Influenzati} \\
\hline
1-2 & Nani \\
3-4 & Elfi \\
5 & Elfi e Umani \\
6-7 & Gnomi \\
8-9 & Halfling \\
10-11 & Umani \\
12 & Altri umanoidi (orchi, gnoll, goblin, ecc.) \\
\hline
\end{tabular}

\end{table}

\textbf{Invisibilità}: Quando questa pozione è bevuta, il bevitore riceve l'abilità simile a un incantesimo di invisibilità. Questa pozione può essere consumata in incrementi di 1/8, nel qual caso l'invisibilità garantita dura 1d4+2 turni per dose. Qualsiasi azione di combattimento rimuove l'invisibilità, così che una nuova dose deve essere consumata.

\textbf{Invulnerabilità}: Una pozione di invulnerabilità dà al bevitore +2 a tutti i tiri salvezza e garantisce una riduzione nella classe di armatura di due ranghi.

\textbf{Levitazione}: Quando questa pozione è bevuta, il bevitore riceve l'abilità simile a un incantesimo di levitazione.

\textbf{Longevità}: Questa pozione rende il bevitore 1d12 anni più giovane. Questa giovinezza restaurata è possibile non solo per l'invecchiamento naturale, ma anche per l'invecchiamento da magia o effetti di creature. C'è qualche piccolo pericolo tuttavia, poiché ogni volta che una pozione di longevità viene consumata c'è una probabilità cumulativa dell'1\% che tutti i precedenti ringiovanimenti da pozioni di questo tipo vengano negati, aumentando l'età del personaggio all'età che avrebbe senza gli effetti delle pozioni. Non è possibile bere questa pozione a incrementi.

\textbf{Olio di Etereità}: Questa pozione non è bevuta, ma questo olio sottile è applicato al personaggio e a tutti i suoi averi per raggiungere uno stato etereo per 4+1d4 turni. Ci vogliono 3 round perché la pozione produca effetto, e può essere negata prima della durata applicando un liquido leggermente acido. Quando etereo, un personaggio è invisibile e può passare attraverso qualsiasi oggetto che non sia anche etereo.

\textbf{Olio di Scivolosità}: Questo olio è applicato al personaggio nello stesso modo dell'olio di etereità. Qualsiasi personaggio così rivestito non può essere trattenuto o afferrato, e nemmeno avvolto nella presa di serpenti costrittori o altri attacchi di presa, incluse corde che legano, catene o manette, magiche o altrimenti. Semplicemente, niente può ottenere una presa su un personaggio rivestito in questo olio. Inoltre, oggetti possono essere rivestiti con l'olio, e se un pavimento è rivestito qualsiasi individuo anche in piedi sul pavimento avrà una probabilità del 95\% ogni round di cadere, a causa dello scivolamento. Gli effetti dell'olio durano 8 ore, ma l'olio può essere pulito presto con liquido contenente alcol, come whiskey, vino o birra forte.

\textbf{Filtro d'Amore}: Il bevitore di questa pozione diventa ammaliato dalla prossima persona o creatura che vede. Tuttavia, il bevitore diventerà effettivamente ammaliato e infatuato dalla persona o creatura se è del sesso preferito e di stirpe razziale simile. L'aspetto di ammaliamento di questa pozione dura per 4+1d4 turni, ma solo dissolvi magia farà cessare al bevitore di essere incantato da un membro di un sesso preferito.

\textbf{Controllo Piante}: Il bevitore di una pozione di controllo piante è in grado di controllare piante o creature simili a piante (inclusi funghi e muffe) entro un'area di 6 metri quadrati, a una distanza di 27 metri. Questa abilità dura per 5d4 round. Piante e creature simili a piante possono obbedire a comandi al meglio della loro capacità. Per esempio, viti possono essere controllate per avvolgersi attorno ai bersagli, e piante intelligenti possono ricevere ordini. Tuttavia, esseri vegetali intelligenti ricevono un tiro salvezza contro incantesimi. Simile ad altre abilità simili all'ammaliamento, non si può controllare direttamente una creatura vegetale intelligente per infliggere danno a se stessa.

\textbf{Veleno}: Questa pozione è altamente variabile nella sua potenza, ed è solitamente un veleno inodore di colore variabile. Il veleno può richiedere ingestione, contatto cutaneo o applicazione a ferite aperte. La potenza determinerà la facilità con cui un tiro salvezza contro veleno può essere compiuto. Veleni estremamente potenti possono richiedere una penalità da –1 a –4, o veleni più deboli possono fornire un bonus da 1 a 4. Un tiro salvezza fallito risulta in morte.

\textbf{Polimorfismo}: Questa pozione garantisce l'abilità simile a un incantesimo di polimorfismo su sé stessi, come l'incantesimo di quarto livello di mago ed elfo.

\textbf{Velocità}: Questa pozione raddoppia la capacità di combattimento e movimento per 5d4 round. Così, se il bevitore può normalmente muoversi a 36 metri, per la durata dell'effetto di questa pozione il bevitore può muoversi a 72 metri. Il numero di attacchi disponibili raddoppia anche, ma questa pozione non diminuisce il tempo di lancio degli incantesimi. Questa capacità intensificata non viene senza costo, poiché lo stress che mette sul corpo del bevitore lo invecchia di 1 anno permanentemente.

\textbf{Super-eroismo}: Solo nani, halfling e guerrieri possono usare questa pozione. A differenza della pozione di eroismo, questa pozione dura 5d5 round. Livelli extra e i loro benefici accompagnatori al combattimento sono temporaneamente garantiti al bevitore, determinati dal suo livello di esperienza come mostrato nella tabella sottostante. In tutti gli altri aspetti questa pozione è identica all'eroismo.

\begin{table}[h]
\centering
\begin{tabular}{|c|c|}
\hline
\textbf{Livello Bevitore} & \textbf{Livelli Garantiti} \\
\hline
0 & 6 (Guerriero) \\
1-3 & 5 \\
4-7 & 4 \\
8-10 & 3 \\
11-12 & 2 \\
\hline
\end{tabular}

\end{table}

\textbf{Acqua Dolce}: Questo liquido dal sapore dolce può essere usato per pulire l'acqua (incluso trasformare l'acqua salata in acqua dolce) o altrimenti trasformare veleni, acido, ecc. in liquido potabile. Inoltre, l'acqua dolce distruggerà altre pozioni. Per la maggior parte dei liquidi, questa pozione influenzerà fino a 2.800 metri cubi. Tuttavia, solo 28 metri cubi di acido possono essere neutralizzati. Gli effetti dell'acqua dolce sono permanenti, e una volta trattato, il liquido resisterà al deterioramento o alla contaminazione per 5d4 round. Dopo questo tempo può essere contaminato di nuovo.

\textbf{Ricerca Tesori}: Per 5d4 round, il bevitore di questa pozione può percepire qualsiasi tesoro entro 72 metri contenente metalli preziosi o gemme. Per essere rilevato, il valore totale del tesoro deve raggiungere o superare qualsiasi combinazione di 50 monete d'oro o 100 gemme. Qualsiasi metallo prezioso che raggiunge questo valore in quantità è rilevabile, e qualsiasi gemma, incluse quelle nei gioielli, è rilevabile. Sebbene la direzione del tesoro possa essere "percepita", la distanza precisa non può. Nessuna barriera fisica impedirà il rilevamento, con l'eccezione di alcune protezioni magiche o piombo.

\textbf{Controllo Non Morti}: Normalmente, i non morti sono immuni all'ammaliamento. Tuttavia, quando bevuta questa pozione garantisce al bevitore la capacità di ammaliare 3d6 DV di non morti (intelligenti o altrimenti) come l'incantesimo ammalia persone. Gli effetti di questa pozione durano 5d4 round.

\textbf{Respirare sott'Acqua}: Il bevitore di questa pozione riceve la capacità di respirare quando sommerso in qualsiasi liquido che contenga ossigeno disciolto (fiumi, laghi, oceani, ecc.). La durata è 1 ora + 1d10 round per dose. C'è una probabilità del 75\% che una pozione contenga 4 dosi, e una probabilità del 25\% che contenga 2 dosi.

\section{Anelli}

Tutti gli anelli magici sono utilizzabili da qualsiasi classe di personaggio. Devono essere indossati su un dito delle mani solamente (dita o pollice). È possibile indossare solo due anelli magici; se ne vengono indossati più di due, tutti gli anelli non funzionano.

\textbf{Comando Animali}: Una volta per turno, questo anello permette al portatore di controllare 1 animale gigante o 1d6 animali di taglia normale. Gli animali magici o intelligenti non sono influenzati. L'effetto dura finché la concentrazione è mantenuta, e il portatore non può intraprendere altre azioni. Una volta che il controllo finisce, gli animali non saranno ben disposti verso il portatore dell'anello, e qualsiasi tiro di reazione subisce una penalità di 1.

\textbf{Comando Umani}: Questo anello garantisce al portatore la capacità di ammaliare come l'incantesimo ammalia persone. Umani per un totale di 6 DV possono essere ammaliati, e gli umani di livello 0 sono trattati come metà di un DV per questo calcolo. Un tiro salvezza può essere tentato con una penalità di –2. Il portatore dell'anello può annullare l'effetto in qualsiasi momento, o dissolvi magia può essere usato.

\textbf{Comando Piante}: Il portatore dell'anello può controllare piante entro un'area di 3 metri quadrati fino a 18 metri di distanza. Questo controllo si estende alle creature vegetali, e anche se la pianta non è normalmente mobile, questo anello garantisce la capacità di far muovere le piante. L'effetto dura finché la concentrazione è mantenuta, e il portatore non può intraprendere altre azioni.

\textbf{Delusione}: Questo anello maledetto convince il portatore che l'anello sia di un altro tipo. Il Signore del Labirinto potrebbe decidere casualmente che tipo di anello il portatore crede che questo anello sia, o uno potrebbe essere scelto.

\textbf{Evocazione Djinn}: Questo anello potente può essere usato una volta al giorno per evocare un djinn che farà gli ordini del portatore dell'anello per un massimo di 24 ore.

\textbf{Resistenza al Fuoco}: Il portatore dell'anello è impermeabile a tutte le forme di fiamma ordinaria, che sia piccola come una torcia o grande come un rogo furioso. Inoltre, questo anello riduce i danni da altri tipi di fuoco di –1 per dado di danno (minimo di 1 pf di danno per dado di danno). Questi tipi di fuoco includono palla di fuoco, muro di fuoco, soffio di fuoco, e il calore intenso della roccia fusa. Se l'esposizione a queste fiamme richiede un tiro salvezza, viene fatto con +2 al tiro di dado.

\textbf{Invisibilità}: Una volta ogni turno, questo anello garantisce al portatore la capacità di diventare invisibile come l'incantesimo invisibilità.

\textbf{Protezione}: Questo anello ha diversi livelli di potere diversi. Per ogni "+", l'anello abbasserà la CA del portatore di questa quantità, e garantirà al portatore questo bonus a tutti i tiri salvezza. Per esempio, se un personaggio con una CA di 9 sta indossando un anello di protezione +2, la sua CA diventa 7 e tutti i tiri salvezza vengono tirati con un bonus +2. Quando viene trovato un anello di protezione, tira sulla tabella sottostante per determinare di che tipo.

\begin{table}[h]
\centering
\begin{tabular}{|c|c|}
\hline
\textbf{Tiro d00} & \textbf{Bonus} \\
\hline
01-80 & +1 \\
81-91 & +2 \\
92 & +2, raggio 1,5 metri \\
93-99 & +3 \\
00 & +3, raggio 1,5 metri \\
\hline
\end{tabular}

\end{table}

Se è dato un raggio, il potere dell'anello, come si applica solo ai tiri salvezza, si estende a tutte le creature entro il raggio.

\textbf{Rigenerazione}: Questo anello garantisce al portatore la capacità di rigenerare 1 pf per round. Tuttavia, l'anello è incapace di rigenerare danni causati da acido o fuoco, e se i pf del portatore raggiungono zero l'anello non riporta in vita i morti. Anche parti intere del corpo possono essere rigenerate. Pezzi piccoli, come dita, impiegano 1 giorno per ricrescere. Pezzi più grandi, come un arto, possono impiegare 1 settimana per ricrescere.

\textbf{Immagazzinamento Incantesimi}: Un anello di immagazzinamento incantesimi può immagazzinare fino a 6 incantesimi, che possono essere incantesimi di chierico o incantesimi di mago/elfo. Al momento in cui l'anello viene trovato, conterrà già 1d6 incantesimi, da essere determinati casualmente dal Signore del Labirinto. Quando un personaggio mette l'anello addosso, guadagna automaticamente la conoscenza di quali incantesimi sono già immagazzinati. Qualsiasi personaggio può rilasciare gli incantesimi dall'anello. Qualsiasi lanciatore di incantesimi può mettere nuovi incantesimi nell'anello lanciando l'incantesimo e dirigendolo verso l'anello. Un incantesimo lanciato dall'anello è lanciato come se il lanciatore fosse il livello minimo richiesto per usare l'incantesimo.

\textbf{Deviazione Incantesimi}: Quando indossa questo anello, 2d6 incantesimi non influenzano il portatore e vengono invece rimandati indietro all'essere che ha lanciato l'incantesimo.

\textbf{Telecinesi}: Questo anello garantisce al portatore la capacità di muovere oggetti con la sua mente, come l'incantesimo telecinesi. Tuttavia, non c'è durata limitata quando si usa l'anello.

\textbf{Camminare sull'Acqua}: Qualsiasi personaggio che indossa questo anello può camminare sull'acqua come se fosse terra solida e asciutta.

\textbf{Debolezza}: Questo è un anello maledetto, e una volta indossato può essere rimosso solo con un incantesimo rimuovi maledizione. Nel corso di 6 round, la FOR del portatore scende a 3 e tutti gli attacchi e i danni vengono tirati con una penalità di –3 (minimo di 1 pf di danno viene inflitto).

\textbf{Desideri}: Un numero variabile di desideri (1d4) è garantito al portatore di questo anello. I desideri funzionano come l'incantesimo dello stesso nome, e possono essere usati in qualsiasi momento. Una volta che i desideri sono usati, l'anello diventa non magico.

\textbf{Visione a Raggi X}: Una volta per turno, il portatore di questo anello può vedere attraverso un muro di pietra fino a 9 metri. Il portatore può vedere 18 metri se guarda attraverso legno e altro materiale a bassa densità. Un'area di 3 metri quadrati (9 metri quadrati) può essere esaminata visualmente ogni turno, e qualsiasi porta segreta, nicchia nascosta o trappola sarà evidente. Questa attività richiede concentrazione completa. Piombo o oro bloccheranno la visione a raggi X.

\section{Pergamene}

La maggior parte delle pergamene sono pezzi di pergamena, impregnati con gli scritti magici di un incantesimo o altro effetto magico. Questi scritti sono potenti in quanto richiedono semplicemente la pronuncia delle loro parole per rilasciare il loro potere. Alcune pergamene possono essere decifrate e lette da qualsiasi classe, mentre altre hanno restrizioni. Queste saranno discusse sotto.

\subsection{Pergamena di Incantesimi}

Una pergamena di incantesimi sarà trovata con da 1 a 7 incantesimi scritti su di essa. Circa 3/4 di tutte le pergamene di incantesimi contengono incantesimi di mago/elfo, e il rimanente contiene incantesimi di chierico. Pergamene che contengono incantesimi di mago/elfo possono essere lette solo impiegando l'incantesimo leggi magia, e gli incantesimi sono utilizzabili solo da maghi ed elfi. Pergamene con incantesimi di chierico possono essere lette senza decifratura speciale, ma sono utilizzabili solo da chierici. Un incantesimo può essere lanciato anche se non è normalmente utilizzabile da un lanciatore di incantesimi del livello del lettore. Questi incantesimi sono lanciati come se da un lanciatore di incantesimi del livello minimo richiesto per lanciare l'incantesimo. Una volta che un incantesimo è lanciato da una pergamena, la scrittura magica per quell'incantesimo scompare.

Quando si determinano i contenuti di una pergamena di incantesimi, tira prima per determinare il tipo di incantesimi per classe, poi tira per determinare il livello dell'incantesimo di ogni incantesimo.

\begin{table}[h]
\centering
\begin{tabular}{|c|c|c|c|}
\hline
\textbf{Tipo Pergamena} & \textbf{Mago/Elfo} & \textbf{Chierico} & \\
\hline
\textbf{Tiro d4} & \textbf{Classe} & \textbf{Tiro d00} & \textbf{Livello Incantesimo} \\
\hline
1-3 & Mago/Elfo & 01-25 & 1 \\
4 & Chierico & 26-50 & 2 \\
& & 51-70 & 3 \\
& & 71-85 & 4 \\
& & 86-95 & 5 \\
& & 96-97 & 6 \\
& & 98 & 7 \\
& & 99 & 8 \\
& & 00 & 9 \\
\hline
\end{tabular}

\end{table}

\begin{table}[h]
\centering
\begin{tabular}{|c|c|}
\hline
\textbf{Tiro d00} & \textbf{Livello Incantesimo Chierico} \\
\hline
01-25 & 1 \\
26-50 & 2 \\
51-70 & 3 \\
71-85 & 4 \\
86-95 & 5 \\
96-98 & 6 \\
99-00 & 7 \\
\hline
\end{tabular}

\end{table}

\subsection{Pergamena Maledetta}

Una pergamena maledetta infligge una terribile maledizione al lettore. Il Signore del Labirinto ha considerevole flessibilità nel determinare gli effetti della maledizione. Una maledizione può essere rimossa solo con l'incantesimo rimuovi maledizione. Il Signore del Labirinto potrebbe anche permettere che la maledizione sia rimossa se il personaggio compie una missione speciale. Alcune possibili maledizioni sono fornite sotto, ma qualsiasi maledizione simile potrebbe essere usata invece.

\begin{table}[h]
\centering
\begin{tabular}{|c|l|}
\hline
\textbf{Tiro d6} & \textbf{Effetto} \\
\hline
1 & La vittima perde un oggetto magico casuale. \\
2 & Un punteggio di abilità casuale subisce una penalità –4. \\
3 & La vittima non può guadagnare nuova esperienza. \\
4 & Il livello della vittima è ridotto di 1. \\
5 & La vittima è polimorfizzata come polimorfismo su altri, in un piccolo animale. \\
6 & La vittima è resa cieca. \\
\hline
\end{tabular}

\end{table}

\subsection{Pergamene di Protezione}

Queste pergamene sono utilizzabili da tutte le classi. Quando le parole magiche di protezione vengono lette ad alta voce, le parole scompaiono dalla pagina e il lettore è circondato da un'area di protezione di 3 metri di raggio contro il tipo di creatura indicato dalla pergamena. Quest'area di protezione è centrata sul lettore, e si muove dovunque si muova. Questa barriera protettiva impedisce al tipo di creatura di entrare, ma non dall'attaccare con armi a distanza o incantesimi. Il cerchio di protezione durerà fino a quando il lettore lo annulla, o se qualcuno entro il cerchio tenta di attaccare una creatura del tipo protetto con un'arma da mischia.

\textbf{Protezione contro Elementali}: Una pergamena di protezione contro elementali protegge contro tutti gli elementali per 2 turni, soggetta alle regole che governano le pergamene di protezione.

\textbf{Protezione contro Licantropi}: Per 6 turni, una pergamena di protezione contro licantropi protegge contro tutte le forme di licantropi. La barriera protettiva può respingere un certo numero di licantropi, basato sul loro numero di DV. Se i licantropi hanno dadi vita di 3 o meno, 1d10 del loro numero saranno respinti. Se hanno 4 o 5 DV, 1d8 del loro numero saranno respinti. Se i licantropi hanno 6 DV o sopra, allora 1d4 del loro numero sono respinti.

\textbf{Protezione contro Magia}: Una barriera è creata contro tutti gli incantesimi e gli effetti simili a incantesimi da dispositivi o mostri. Questa barriera rimane per 1d4 turni. Questo effetto non può essere dissolto o altrimenti rimosso eccetto attraverso un desiderio.

\textbf{Protezione contro Non Morti}: Per 6 turni, una pergamena di protezione contro non morti protegge contro tutte le forme di non morti. La barriera protettiva può respingere un certo numero di non morti, basato sul loro numero di DV. Se hanno dadi vita di 3 o meno, 2d12 del loro numero saranno respinti. Se hanno 4 o 5 DV, 2d6 del loro numero saranno respinti. Se i non morti hanno 6 DV o sopra, allora 1d6 del loro numero sono respinti.

\subsection{Mappe del Tesoro}

Le mappe del tesoro variano considerevolmente nel valore del tesoro a cui conducono. In tutti i casi, il Signore del Labirinto costruirà la mappa e il tesoro a cui conduce in anticipo. La mappa è probabile che conduca a un tesoro entro il labirinto in cui i personaggi trovano la mappa, o la mappa può condurre a un'altra, a volte remota, località. La difficoltà nell'ottenere il tesoro dovrebbe riflettere il suo valore. Potrebbero esserci trappole, enigmi o altre sfide. La mappa stessa può essere incantata così che richiede leggi magia per decifrarla.

\section{Bastoni, Verghe e Bacchette}

Bastoni e verghe sono collettivamente chiamati "dispositivi". Le verghe a volte sono utilizzabili da qualsiasi classe, ma molte sono limitate all'uso solo da certe classi. Le bacchette possono essere usate solo da maghi ed elfi. Un bastone può essere utilizzabile da maghi/elfi o chierici, a seconda del tipo di bastone. Questi oggetti magici generalmente usano una "carica" quando il loro effetto è attivato, e ogni oggetto ha un numero limitato di cariche. Quando trovata, una verga conterrà 2d6 cariche, un bastone conterrà 3d10 cariche, e una bacchetta conterrà 2d10 cariche. Le eccezioni saranno notate nelle descrizioni specifiche degli oggetti. Fisicamente, questi tre tipi di oggetti magici differiscono principalmente nelle dimensioni. Le bacchette sono piccole e sottili, essendo lunghe circa 45 centimetri. Un bastone è molto più grande, essendo lungo 1,8 metri e generalmente ha un diametro di 60 centimetri. Le verghe sono da qualche parte nel mezzo tra questi due tipi di oggetti, essendo lunghe circa 90 centimetri. Quando un bastone è descritto, il nome dell'oggetto sarà seguito da "C" se è utilizzabile da un chierico, o "M/E" se è utilizzabile da elfi e maghi.

\textbf{Verga di Assorbimento}: Questa verga agisce come un magnete, attirando incantesimi in se stessa. La magia assorbita deve essere un incantesimo a bersaglio singolo o un raggio diretto al personaggio che possiede la verga o al suo equipaggiamento. La verga quindi annulla l'effetto dell'incantesimo e immagazzina il suo potenziale fino a quando il portatore rilascia questa energia sotto forma di incantesimi suoi. Il portatore può istantaneamente rilevare il livello di un incantesimo mentre la verga assorbe l'energia di quell'incantesimo. L'assorbimento non richiede azione da parte dell'utilizzatore se la verga è in mano al momento.

Un totale corrente di livelli di incantesimo assorbiti (e usati) dovrebbe essere tenuto. Il portatore della verga può usare l'energia di incantesimo catturata per lanciare qualsiasi incantesimo che ha preparato in 1 round senza spendere la preparazione stessa (ha ancora l'incantesimo in memoria). Le uniche restrizioni sono che i livelli di energia di incantesimo immagazzinati nella verga devono essere uguali o maggiori del livello dell'incantesimo che il portatore vuole lanciare, e che la verga sia in mano quando lancia. Per lanciatori come i chierici che non preparano incantesimi, l'energia della verga può essere usata per lanciare qualsiasi incantesimo del livello o livelli appropriati che conoscono.

Una verga di assorbimento assorbe un massimo di 50 livelli di incantesimo e può successivamente solo scaricare qualsiasi potenziale rimanente che potrebbe avere. La verga non può essere ricaricata. Il portatore conosce il potenziale di assorbimento rimanente della verga e la quantità attuale di energia immagazzinata.

\textbf{Verga di Cancellazione}: Questo oggetto è molto temuto da coloro che apprezzano i loro oggetti magici, poiché con un solo tocco di questa verga, un oggetto magico perde permanentemente tutto il suo potere e diventa un oggetto ordinario. Quando si tenta di colpire un oggetto su un avversario, tratta l'attacco come se dovesse colpire una CA di 9. Il Signore del Labirinto, a seconda delle circostanze, può aggiustare questo valore. Questa verga è utilizzabile una volta e non può essere ricaricata.

\textbf{Verga di Fascinazione}: Con l'impiego di 1 carica, tutti i mostri e personaggi entro un raggio di 6 metri sono ammaliati dal portatore per 1 turno fintanto che sono almeno minimamente intelligenti. Tutti gli esseri ammaliati non proveranno altro che rispetto e ammirazione per il portatore, e cercheranno di compiacerlo facendo quasi qualsiasi cosa eccetto danneggiarsi o violare il loro allineamento.

\textbf{Verga del Potere Signorile}: Questa verga ha funzioni che sono simili a incantesimi, e può anche essere usata come arma magica di vari tipi. Ha anche diversi usi più mondani. La verga del potere signorile è di metallo, più spessa di altre verghe, con una palla flangiata a un'estremità e sei bottoni simili a borchie lungo la sua lunghezza. Pesa 4,5 kg, e una forza di 16 o maggiore è necessaria per brandire quest'arma. Qualsiasi personaggio con meno di 16 di forza subisce una penalità di attacco di –1 per punto sotto 16.

La verga ha le seguenti funzioni simili a incantesimi, e ognuna costa 1 carica:
\begin{itemize}
\item Paura su tutti i nemici che la vedono, se il portatore lo desidera (gittata massima 18 metri). Al bersaglio è permesso un salvare contro incantesimi.
\item Paralizzare al tocco, se il portatore così comanda. Il portatore deve scegliere di usare questo potere e poi riuscire in un attacco di tocco in mischia per attivare il potere. Se l'attacco fallisce, l'effetto è perso. Al bersaglio è permesso un salvare contro incantesimi.
\item Infliggere 2d4 punti ferita di danno a un avversario con un attacco di tocco riuscito (nessun salvare) e curare il portatore di una quantità simile di danno.
\end{itemize}

Le seguenti funzioni d'arma della verga non richiedono l'uso di cariche:

Nella sua forma normale, la verga può essere usata come una mazza +2.
\begin{itemize}
\item Quando il bottone 1 è premuto, la verga diventa una spada lingua di fiamma +1. Una lama spunta dalla palla, con la palla stessa che diventa l'elsa della spada. L'arma si allunga a una lunghezza totale di 90 centimetri.
\item Quando il bottone 2 è premuto, la verga diventa un'ascia da battaglia +4. Una lama larga spunta alla palla, e l'intero si allunga a 1,2 metri.
\item Quando il bottone 3 è premuto, la verga diventa una lancia +3. La lama della lancia spunta, e il manico può essere allungato fino a 3,6 metri (scelta del portatore), per una lunghezza totale da 1,8 metri a 4,5 metri. Alla sua lunghezza di 4,5 metri, la verga è adatta per l'uso come lancia da cavaliere.
\end{itemize}

Le seguenti altre funzioni della verga non impiegano cariche:
\begin{itemize}
\item Palo/scala per scalare. Quando il bottone 4 è premuto, una punta che può ancorarsi nel granito è estrusa dalla palla, mentre l'altra estremità germoglia tre ganci affilati. La verga si allunga da 1,5 a 15 metri in un singolo round, fermandosi quando il bottone 4 è premuto di nuovo. Sbarre orizzontali lunghe 7,5 centimetri si piegano fuori dai lati, a 30 centimetri di distanza, in progressione sfalsata. La verga è tenuta saldamente dalla punta e dai ganci e può sopportare fino a 1.800 kg. Il portatore può ritrarre il palo premendo il bottone 5.
\item La funzione scala può essere usata per forzare porte aperte. Il portatore pianta la base della verga a 9 metri o meno dal portale da forzare e in linea con esso, poi preme il bottone 4. La forza esercitata garantisce +4 al forzare porte.
\item Quando il bottone 6 è premuto, la verga indica il nord magnetico e dà al portatore conoscenza della sua profondità approssimativa sotto la superficie o altezza sopra di essa.
\end{itemize}

Nota che questa verga non può mai essere ricaricata. Quando le cariche sono esaurite, le funzioni che richiedono cariche non possono essere usate di nuovo, e nemmeno la verga può essere impiegata come una spada lingua di fiamma +1 né un'ascia da battaglia +4. Questi attributi sono persi.

\textbf{Verga di Resurrezione [C]}: Un chierico di qualsiasi livello può usare questa verga una volta al giorno per riportare in vita esseri dai morti come l'incantesimo resurrezione. Un chierico che usa questa verga non ha bisogno di riposare dopo aver speso cariche dalla verga. Diversi tipi di personaggi possono essere resuscitati, e ogni tipo richiede un numero diverso di cariche. Quando tutte le cariche dalla verga sono usate, si sbriciola in polvere.

\begin{table}[h]
\centering
\begin{tabular}{|l|c|l|c|}
\hline
\textbf{Razza} & \textbf{Cariche} & \textbf{Classe} & \textbf{Cariche} \\
\hline
Umano & 1 & Guerriero & 2 \\
Mezzelfo & 2 & Paladino & 1 \\
Mezzorco & 4 & Ranger & 2 \\
Halfling & 2 & Mago & 3 \\
Elfo & 4 & Illusionista & 3 \\
Nano & 3 & Ladro & 3 \\
Gnomo & 3 & Assassino & 4 \\
& & Chierico & 1 \\
& & Druido & 2 \\
& & Monaco & 3 \\
\hline
\end{tabular}

\end{table}

Nota che una verga di resurrezione non è ricaricabile.

\textbf{Verga del Comando}: Questa verga sembra uno scettro regale. Il portatore può comandare l'obbedienza e la fedeltà di creature entro 36 metri quando attiva il dispositivo. Creature per un totale di 100 + 1d4×100 Dadi Vita possono essere comandate, ma creature con punteggi di INT di 15 o superiore e livelli o DV uguali o maggiori di 12 hanno diritto a un salvare contro incantesimi per negare l'effetto. Creature comandate obbediscono al portatore come se fosse il loro sovrano assoluto. Tuttavia, se il portatore dà un comando che è contrario alla natura delle creature comandate, la magia è spezzata. La verga può essere usata dopo 1 round dall'attivazione, e ogni carica spesa permette alla verga di essere usata per 10 round. La verga si sbriciola in polvere una volta che tutte le cariche sono spese, e non può essere ricaricata.

\textbf{Verga di Percussione}: Una verga di percussione infligge 1d8+3 pf di danno, e funziona come un'arma +3 di natura magica. Quando questa verga è usata contro golem, costa sempre 1 carica per colpo riuscito in combattimento, e infligge 2d8+6 pf di danno. Nota che quando impiegata in questo modo contro un golem, un tiro riuscito di 20 annichilirà istantaneamente il golem. Inoltre, questa verga può infliggere danno aggiuntivo a demoni, non morti estremamente potenti, e altri esseri infernali da altri piani. Quando attacca questi mostri, un tiro di attacco riuscito di 20 causa l'impiego di una carica, e la verga infligge danno normale triplicato.


\section{Bastoni (Continuazione)}

\textbf{Bastone del Comando [C]}: Questo bastone può essere usato per comandare piante, animali e umani nello stesso modo degli anelli comando umani, comando animali e comando piante. Ogni uso richiede una carica.

\textbf{Bastone di Cura [C]}: Questo bastone non impiega cariche. Può curare 1d6+1 punti ferita di danno, ma può essere usato solo una volta per creatura al giorno. Questo bastone può curare un numero illimitato di creature in un giorno.

\textbf{Bastone dei Magi}: Il bastone dei magi è una versione molto più potente del bastone della stregoneria. Garantisce diversi incantesimi a disposizione del portatore. Il bastone può essere usato per causare le seguenti abilità simili a incantesimi: individua magia, ingrandimento, blocca portale, luce e protezione dal bene (male). Queste abilità non richiedono l'impiego di cariche.

Inoltre, il bastone ha le seguenti abilità che costano 1 carica per uso: dissolvi magia, palla di fuoco, tempesta di ghiaccio, invisibilità, scassinare, fulmine, passapareti, pirotecnica, muro di fuoco e ragnatela. Le seguenti abilità potenti costano 2 cariche per uso: evoca elementali, spostamento planare, telecinesi e turbine (come un djinn). Il portatore del bastone riceve un bonus di +2 per i tiri salvezza contro magia.

Questo bastone può essere ricaricato, ma solo assorbendo energie magiche che sono lanciate al portatore. Quando un incantesimo è lanciato al portatore, può scegliere di assorbire queste energie, a un tasso di 1 carica per livello di incantesimo dell'incantesimo diretto a lui. Nota che prendere questa manovra è l'azione unica del bastone per quel round, e non può essere usato per altri effetti nello stesso round in cui assorbe energia. Ogni bastone ha un numero massimo di cariche possibili, e assorbirà solo con sicurezza cariche fino a questo limite. Il portatore non conoscerà il limite, o quante cariche sono state usate, a meno che qualche mezzo magico non sia impiegato per scoprire questo. Il bastone non rivela questa informazione. Se il bastone raccoglie energia in eccesso del suo limite, detona nello stesso modo di un attacco di rappresaglia, come descritto sotto.

Un bastone dei magi può essere usato per un attacco di rappresaglia, richiedendo che sia spezzato dal suo portatore. La rottura del bastone deve essere intenzionale e dichiarata dal portatore. Tutte le cariche attualmente nel bastone sono istantaneamente rilasciate in un raggio di 9 metri. Tutti entro 3 metri del bastone spezzato subiscono punti di danno pari a 8 × il numero di cariche nel bastone, quelli da 3 a 6 metri di distanza subiscono 6 × il numero di cariche di danno, e quelli da 6 a 9 metri di distanza subiscono 4 × il numero di cariche di danno. Tutti quelli colpiti possono fare tiri salvezza contro incantesimi per ridurre il danno a metà.

Il personaggio che spezza il bastone ha una probabilità del 50\% di viaggiare a un altro piano di esistenza, ma se non lo fa, il rilascio esplosivo di energia di incantesimo lo distrugge. Dopo che tutte le cariche sono usate dal bastone, rimane un bastone +2. Una volta vuoto di cariche, non può essere usato per un attacco di rappresaglia.

\textbf{Bastone del Potere [M/E]}: Questo bastone potente ha diverse abilità. Prima, può essere usato per lanciare gli incantesimi cono di freddo, fulmine e palla di fuoco (ognuno che infligge 8d6 pf di danno). Inoltre, il bastone può essere usato per lanciare luce continua e telecinesi (con un limite di peso di 113 kg). Infine, questo bastone può anche essere usato con lo stesso effetto di un bastone di percussione.

\textbf{Bastone di Percussione [C]}: Con l'impiego di una carica e un tiro di attacco riuscito, questo bastone può essere usato per colpire un avversario per 2d6 punti ferita di danno.

\textbf{Bastone dell'Appassimento [C]}: Questo bastone funziona come un bastone +1 che infligge 2d4+1 punti ferita di danno quando viene usata una carica. Usando 2 cariche e colpendo con successo un avversario, il bastone invecchia una vittima di 10 anni. Se tre cariche vengono spese in questo attacco, uno degli arti della vittima si rattrappirà in un membro mummificato e inutile (è permesso un tiro salvezza contro dispositivi simili a incantesimi). L'effetto di invecchiamento ucciderà automaticamente la maggior parte delle creature che hanno una durata di vita breve. Nota anche che gli effetti delle cariche spese sono cumulativi, così che se 3 cariche vengono usate, la vittima non solo riceverà danno, ma sarà anche invecchiata e avrà un arto rattrappito.

\textbf{Bastone della Stregoneria [M/E]}: Questo bastone funziona come un bastone +1. Inoltre, il bastone può essere usato per lanciare gli incantesimi evoca elementali, invisibilità, passapareti e ragnatela. Il bastone ha l'effetto simile a incantesimo di un djinn per creare un turbine e può essere usato come una bacchetta di paralizzazione. Ognuna di queste abilità richiede una carica. Il bastone può essere spezzato per un colpo finale. I risultati di un colpo finale dipendono dal numero di cariche nel bastone. Per ogni carica, 8 punti ferita di danno sono inflitti in una grande palla di fuoco a tutti i mostri e personaggi (anche il proprietario del bastone) entro 9 metri. Il bastone è quindi spezzato e inutile.

\textbf{Bastone del Serpente [C]}: Questo bastone non impiega cariche. Colpisce come un bastone +1. L'utilizzatore può comandare al bastone di crescere per diventare un serpente costrittore gigante e stringersi attorno a una vittima (CA 5, DV 3, pf 20, MV 6 metri). Il comando perché il bastone diventi un serpente è pronunciato mentre colpisce una vittima. La vittima deve riuscire in un tiro salvezza contro dispositivi simili a incantesimi o essere tenuta immobile dal serpente costrittore per 1d4 turni, o fino a quando il proprietario comanda al serpente di rilasciarla. Il serpente ritorna al proprietario e ritorna in forma di bastone dopo essersi stretto attorno a un avversario. Se la forma di serpente è uccisa, non ritornerà in forma di bastone e il bastone è distrutto. Quando il serpente ritorna in forma di bastone, tutto il danno che ha subito in combattimento è automaticamente curato.

\section{Bacchette}

\textbf{Bacchetta del Freddo}: Un cono agghiacciante lungo 18 metri e largo 9 metri all'estremità terminale viene scaricato da questa bacchetta. Qualsiasi essere entro il cono di freddo subirà 3d6 pf di danno a meno che non riesca in un tiro salvezza contro bacchette, che riduce il danno a metà. Una carica viene impiegata per uso.

\textbf{Bacchetta di Individuazione Nemici}: Questa bacchetta rende qualsiasi nemico del portatore che si trovi entro 18 metri, sia invisibile che nascosto, circondato da un'aura fiammeggiante luminosa. Questo effetto richiede una carica.

\textbf{Bacchetta di Individuazione Magia}: Questa bacchetta rende qualsiasi oggetto magico entro 6 metri circondato da un'aura di luce blu luminosa. Questo effetto richiede una carica.

\textbf{Bacchetta di Individuazione Metalli}: Dopo aver impiegato una carica, la bacchetta punterà nella direzione di qualsiasi concentrazione di metallo che pesi 45 kg o più se si trova entro 6 metri. Il portatore della bacchetta è intuitivamente consapevole del tipo di metallo rilevato.

\textbf{Bacchetta di Individuazione Porte Segrete}: Questa bacchetta punterà a tutte le porte segrete entro 6 metri. Una carica viene impiegata per uso.

\textbf{Bacchetta di Individuazione Trappole}: Questa bacchetta punterà a tutte le trappole entro 6 metri. Una carica viene impiegata per uso.

\textbf{Bacchetta di Negazione Dispositivi}: Il portatore di questa bacchetta può scegliere una bacchetta, verga o bastone da un avversario, e renderlo impotente per 1 round. L'oggetto è impotente nello stesso round in cui la bacchetta di negazione viene usata. Quindi, l'azione per usare questa bacchetta deve essere annunciata prima di determinare l'iniziativa. Una carica viene impiegata per uso.

\textbf{Bacchetta della Paura}: Un cono lungo 18 metri e largo 9 metri all'estremità terminale viene scaricato da questa bacchetta. Qualsiasi essere entro il cono diventerà pauroso e fuggirà per 30 round con un MV uguale a tre volte il loro tasso normale per round. Una carica viene impiegata per uso.

\textbf{Bacchetta del Fuoco}: Una bacchetta del fuoco produce diversi effetti simili a incantesimi, e può produrre solo un effetto per round. I seguenti effetti richiedono l'impiego di 1 carica: mani brucianti spara un raggio triangolare di fuoco lungo 3,6 metri con una larghezza finale di 3 metri. Qualsiasi essere entro quest'area di effetto subisce 6 pf di danno; pirotecnica può essere prodotta dalla bacchetta, e imita l'incantesimo dello stesso nome.

La bacchetta del fuoco può produrre i seguenti effetti con l'impiego di 2 cariche: una palla di fuoco può essere sparata dalla bacchetta, funzionando esattamente come l'incantesimo dello stesso nome, come se lanciato da un personaggio di 6° livello. Infligge 6d6 pf di danno a tutti entro l'area di effetto (salvare contro bacchette per metà danno). Qualsiasi risultato di 1 su un dado di danno è trattato come un 2. Un muro di fuoco può essere prodotto, come l'incantesimo dello stesso nome lanciato da un personaggio di 6° livello. Il muro può essere formato in un cerchio di 6,6 metri di diametro che circonda il portatore della bacchetta.

Questa bacchetta è ricaricabile.

\textbf{Bacchetta delle Palle di Fuoco}: Una bacchetta delle palle di fuoco può essere usata per scaricare una palla di fuoco come l'incantesimo. Infligge 6d6 pf di danno a meno che la/le vittima/e non riescano in un tiro salvezza contro bacchette, che riduce il danno a metà. Una carica viene impiegata per uso.

\textbf{Bacchetta del Ghiaccio}: Una bacchetta del ghiaccio produce diversi effetti simili a incantesimi, e può produrre solo un effetto per round. I seguenti effetti richiedono l'impiego di 1 carica: una tempesta di ghiaccio (come l'incantesimo di mago) devasta il freddo a una distanza di 18 metri; un muro di ghiaccio può essere portato in esistenza. Ha sempre uno spessore di 15 centimetri, ma può avere qualsiasi area superficiale come comandato dal portatore della bacchetta fino a un massimo di 180 metri quadrati (per esempio, 7,5×24 metri o 3×60 metri).

La bacchetta del ghiaccio può produrre un cono di freddo con l'impiego di 2 cariche. Il cono è di 6 metri di diametro alla sua lunghezza massima di 18 metri. Infligge 6d6 pf di danno a tutti entro l'area di effetto (salvare contro bacchette per metà danno). Qualsiasi risultato di 1 su un dado di danno è trattato come un 2.

La bacchetta del ghiaccio è ricaricabile.

\textbf{Bacchetta dell'Illusione}: Il portatore di questa bacchetta può creare gli effetti dell'incantesimo forza fantasmagorica. Riferisciti a questo incantesimo per gli effetti e i requisiti di concentrazione. Mentre si concentra su un effetto illusorio, il portatore può muoversi a metà movimento, ma se viene colpito con successo in combattimento tutta la concentrazione è persa e l'illusione svanisce istantaneamente.

\textbf{Bacchetta della Luce}: Una bacchetta della luce produce diversi effetti simili a incantesimi, e può produrre solo un effetto per round. I seguenti effetti richiedono l'impiego di 1 carica: luci danzanti come l'incantesimo di mago, e luce come l'incantesimo di mago.

Il portatore può spendere due cariche per creare luce continua.

Infine, spendendo 3 cariche, il portatore può creare un raggio di intensa luce solare. La luce dorata brillante dura solo un momento, ha una gittata di 36 metri, e forma una sfera di luce di 12 metri di diametro. Qualsiasi essere entro l'area deve salvare contro bacchette o essere accecato e stordito per 1 round. La sfera dorata di luce ha un effetto devastante su tutti i non morti, infliggendo 6d6 pf di danno, senza tiro salvezza permesso.

\textbf{Bacchetta del Fulmine}: Se il portatore di una bacchetta del fulmine colpisce un avversario con la bacchetta e spende 1 carica, può consegnare una carica elettrica alla sua vittima, che infligge 1d10 pf di danno. Nessun tiro salvezza è permesso, e per scopi di CA una vittima che indossa armatura di metallo ha una CA di 9, indipendentemente dai bonus di armatura magica. Armatura non metallica e oggetti come un anello di protezione si applicano, tuttavia.

Spendendo due cariche, il portatore di questa bacchetta può produrre un fulmine, con lo stesso effetto dell'incantesimo di mago, per infliggere 6d6 pf di danno. Qualsiasi risultato di 1 su un dado di danno è trattato come un 2. È permesso un salvare contro bacchette per metà danno.

Una bacchetta del fulmine può essere ricaricata.

\textbf{Bacchetta dei Fulmini}: Una bacchetta dei fulmini può essere usata per scaricare un fulmine come l'incantesimo. Infligge 6d6 pf di danno a meno che la vittima non riesca in un tiro salvezza contro bacchette, che riduce il danno a metà. Una carica viene impiegata per uso.

\textbf{Bacchetta dei Dardi Magici}: Questa bacchetta spara uno o due dardi magici (scelta dell'utilizzatore) per round come l'incantesimo di mago/elfo dello stesso nome. I dardi infliggono 1d6+1 punti ferita di danno ciascuno, e colpiscono sempre. Ogni singolo dardo sparato impiega una carica.

\textbf{Bacchetta di Negazione}: Questo dispositivo nega l'incantesimo o la funzione simile a incantesimo di oggetti magici. Il portatore punta la bacchetta all'oggetto magico, e un raggio grigio pallido esce per toccare il dispositivo bersaglio o essere. Il raggio nega qualsiasi funzione di tutte le bacchette, e qualsiasi altro oggetto bersaglio o effetto magico (inclusi incantesimi lanciati dal bersaglio) ha una probabilità del 25\% di resistere alla negazione. Ogni uso della bacchetta costa 1 carica, e può essere usata una volta per round. Una bacchetta di negazione non può mai essere ricaricata.

\textbf{Bacchetta di Paralizzazione}: Un cono lungo 18 metri e largo 9 metri all'estremità terminale viene scaricato da questa bacchetta. Qualsiasi essere entro il cono diventerà paralizzato per 6 turni a meno che non riesca in un tiro salvezza contro bacchette. Una carica viene impiegata per uso.

\textbf{Bacchetta di Polimorfismo}: Una bacchetta di polimorfismo può produrre gli effetti degli incantesimi polimorfismo su altri o polimorfismo su sé stessi, che è determinato dal portatore appena prima di ogni uso. Al destinatario è garantito un tiro salvezza contro bacchette, e il successo nega l'effetto. Un bersaglio consenziente può rinunciare a un tiro salvezza. Una carica viene impiegata per uso.

\textbf{Bacchetta di Evocazione}: Il portatore di questa bacchetta, quando tiene la bacchetta in mano (non quando è riposta) è in grado di identificare istantaneamente incantesimi di natura evocativa, sia quando visti scritti che attivamente lanciati. Il portatore può spendere una carica per lanciare gli incantesimi servo invisibile e evoca mostri. Per lanciare evoca mostri, il portatore deve essere di livello sufficiente per lanciare la versione impiegata (I, II, III, IV, V o VI), e richiede 1 round. Quando lancia evoca mostri, il portatore può spendere fino a 6 cariche, per accumulare gli effetti come se più di un incantesimo fosse lanciato. Per esempio, le 6 cariche potrebbero essere spese per lanciare evoca mostri VI, o potrebbero essere usate per lanciare evoca mostri I sei volte, evoca mostri II e IV, o qualsiasi altra combinazione che totalizzi 6.

Una bacchetta di evocazione può produrre gli effetti di una sfera prismatica (può essere formata in un muro). Tuttavia, a differenza dell'incantesimo, la bacchetta può produrre solo un singolo tipo di colore alla volta, per uso ogni round. Ogni uso costa 1 carica, e impiega 1 round per apparire.

La bacchetta è anche capace di portare in esistenza un velo di oscurità, spendendo 2 cariche. Questo effetto impiega 5 segmenti per lanciare. Questo velo può occupare uno spazio uguale a 540 metri quadrati, e può prendere qualsiasi dimensione che uguagli questa quantità (6×9 metri, 4,5×12 metri, ecc.) Nessuna luce può passare attraverso il velo, ma altri effetti magici e oggetti fisici possono passarci attraverso normalmente.

Una bacchetta di evocazione è ricaricabile.

\textbf{Bacchetta delle Meraviglie}: Una bacchetta delle meraviglie è un dispositivo strano e imprevedibile che genera casualmente qualsiasi numero di effetti strani ogni volta che viene usata. Ogni uso costa 1 carica. In alcuni casi un tiro salvezza è appropriato. Gli effetti dovrebbero essere considerati tipici di quelli possibili, ma il referee può aggiustare questi o creare nuovi effetti, a seconda della situazione. Poteri tipici della bacchetta includono i seguenti:

\begin{table}[h]
\centering
\begin{tabular}{|c|l|}
\hline
\textbf{d\%} & \textbf{Effetto Meraviglioso} \\
\hline
01–05 & Rallenta creatura puntata per 1 turno. \\
06–10 & Fuoco fatato circonda il bersaglio. \\
11–15 & Illude il portatore per 1 round nel credere che la bacchetta \\
& funzioni come indicato da un secondo tiro di dado (nessun salvare). \\
16–20 & Raffica di vento, ma a doppia potenza. \\
21–25 & Il portatore impara i pensieri superficiali del bersaglio \\
& (come con ESP) per 1d4 round. \\
26–30 & Nuvola maleodorante a 9 metri di gittata. \\
31–33 & Pioggia pesante cade per 1 round in un raggio di 18 metri \\
& centrato sul portatore della bacchetta. \\
34–36 & Evoca un animale—un rinoceronte (01–25 su d\%), \\
& elefante (26–50), o topo (51–100). \\
37–46 & Fulmine (21 metri di lunghezza, 1,5 metri di larghezza), 6d6 danni. \\
47–49 & Flusso di 600 grandi farfalle si versa e svolazza \\
& per 2 round, accecando tutti (incluso il portatore). \\
50–53 & Ingrandisce vittima se entro 18 metri dalla bacchetta. \\
54–58 & Oscurità, emisfero di 9 metri di diametro, centrato \\
& a 9 metri dalla bacchetta. \\
59–62 & L'erba cresce in un'area quadrata di 48 metri davanti \\
& alla bacchetta, o l'erba esistente cresce dieci volte \\
& la dimensione normale. \\
63–65 & Fa svanire qualsiasi oggetto non vivente fino a 450 kg \\
& di massa e fino a 27 metri cubi di dimensione. \\
66–69 & Riduce il portatore a 1/12 dell'altezza. \\
70–79 & Palla di fuoco, 6d6 danni, come bacchetta. \\
80–84 & Invisibilità copre il portatore della bacchetta. \\
85–87 & Foglie crescono dal bersaglio se entro 18 metri dalla bacchetta. \\
88–90 & 10–40 gemme, valore 1 mo ciascuna, sparano in un \\
& flusso lungo 9 metri. Ogni gemma infligge 1 punto di \\
& danno a qualsiasi creatura sul suo percorso: Tira 5d4 \\
& per il numero di colpi. \\
91–95 & Colori scintillanti danzano e giocano su un'area \\
& di 12×9 metri davanti alla bacchetta. Le creature \\
& al suo interno sono accecate per 1d6 round. \\
96–97 & Il portatore (50\% di possibilità) o il bersaglio \\
& (50\% di possibilità) diventa permanentemente \\
& blu, verde o viola. \\
98-100 & Carne in pietra (o pietra in carne se il bersaglio \\
& è già pietra) se il bersaglio è entro 18 metri. \\
\hline
\end{tabular}

\end{table}

\section{Oggetti Magici Vari}

\textbf{Amuleto contro il Possesso}: Il portatore di questo amuleto di rame è reso immune a vaso magico, possessione e altri effetti di natura simile, inclusa la possessione demoniaca.

\textbf{Amuleto dei Piani}: Questo dispositivo solitamente appare come un amuleto circolare nero, sebbene qualsiasi personaggio che lo guardi da vicino veda un turbinio scuro e mobile di colore. L'amuleto permette al suo portatore di utilizzare spostamento planare. Tuttavia, questo è un oggetto difficile da padroneggiare. C'è una probabilità del 20\% per i primi 1d6 usi che il portatore venga trasportato in un piano di esistenza casuale.

\textbf{Amuleto di Prova contro Individuazione e Localizzazione}: Questo amuleto d'argento protegge il portatore da scrutamento e localizzazione magica. Di conseguenza, il portatore non può essere localizzato con una sfera di cristallo, né visto attraverso incantesimi come ESP, chiaroudienza, chiaroveggenza o altri incantesimi usati per predire azioni, intenzioni o per rivelare l'allineamento.

\textbf{Amuleto di Localizzazione Inevitabile}: Questo amuleto maledetto sembra essere un amuleto di prova contro individuazione e localizzazione. Tuttavia, in realtà rende il portatore più vulnerabile a queste magie. La probabilità di osservare il portatore è raddoppiata e la durata è doppia di qualsiasi incantesimo usato per osservare il portatore.

\textbf{Amuleto contro Sfere di Cristallo e ESP}: Questo amuleto garantisce al portatore immunità all'individuazione da una sfera di cristallo e agli effetti di ESP.

\textbf{Freccia di Localizzazione}: Questa freccia può essere usata fino a 8 volte nel corso di 8 turni. Viene lanciata casualmente in aria, e atterra puntando in una direzione o verso la caratteristica più vicina desiderata. Caratteristiche possibili includono l'uscita o entrata più vicina, scale, passaggi, caverne e aree simili.

\textbf{Apparato del Granchio}: Questo oggetto appare come un grande barile di ferro sigillato, ma ha una presa segreta che apre un portello a un'estremità. Chiunque strisci dentro trova dieci leve. Il dispositivo ha le seguenti caratteristiche: pf 200; MV 9 metri in avanti, 18 metri indietro; CA 0; Danno 2d6, 2 pinze. Quando attacca con le pinze, il valore di attacco è lo stesso dell'operatore, e se viene segnato un colpo, c'è una possibilità che entrambe le pinze colpiscano, per un totale di 4d6 punti ferita di danno (tiro di 1-5 su 1d20). Questo attacco ignora il contributo dell'armatura dell'avversario alla CA, ma i modificatori di DES si applicano.

\begin{table}[h]
\centering
\begin{tabular}{|c|l|}
\hline
\textbf{Leva} & \textbf{Funzione della Leva} \\
\hline
1 & Estendi/ritrai gambe e coda \\
2 & Scopri/copri oblò anteriore \\
3 & Scopri/copri oblò laterali \\
4 & Estendi/ritrai pinze e antenne \\
5 & Chiudi pinze \\
6 & Muovi avanti/indietro \\
7 & Gira sinistra/destra \\
8 & Apri "occhi" con luce continua dentro/chiudi "occhi" \\
9 & Sali/scendi in acqua (levita) \\
10 & Apri/chiudi portello \\
\hline
\end{tabular}

\end{table}

Due personaggi di taglia umana possono stare dentro. Il dispositivo può funzionare in acqua fino a 270 metri di profondità. Contiene abbastanza aria perché un equipaggio di due sopravviva 1d4+1 ore (due volte tanto per un singolo occupante). Quando attivato, l'apparato sembra qualcosa come un grande aragosta.

\textbf{Borsa Divorante}: Questa borsa magica è della dimensione di un piccolo sacco. Dopo 6+1d4 turni, tutti gli oggetti messi in questa borsa svaniscono e sono permanentemente persi. La borsa deve essere completamente chiusa perché questo effetto abbia luogo.

\textbf{Borsa Contenitrice}: Questa appare come un piccolo sacco comune. La borsa contenitrice si apre in uno spazio non dimensionale. Il suo interno è più grande delle sue dimensioni esterne. È abbastanza grande per contenere un oggetto di 3×1,5×0,9 metri. Indipendentemente da cosa viene messo nella borsa, pesa un massimo di 27 kg ma contiene fino a 450 kg.

\textbf{Borsa di Trasformazione}: Questa borsa magica funziona proprio come una borsa contenitrice per una durata di 1d6 giorni. Dopo quel tempo, tutto il materiale al suo interno o nuovo materiale aggiunto è soggetto a trasformazione a seconda della sua natura. Gemme preziose diventano pietre senza valore, e metalli preziosi diventano metalli di minor valore, come piombo. Gli oggetti magici perdono tutti il loro potere, senza tiro salvezza, per diventare oggetti ordinari del loro tipo. Solo oggetti magici estremamente potenti potrebbero essere immuni a questo effetto.

\textbf{Borsa dei Trucchi}: Questo piccolo sacco appare normale e vuoto. Tuttavia, chiunque metta la mano nella borsa sente una piccola palla pelosa. Se la palla viene rimossa e lanciata fino a 6 metri di distanza, si trasforma in un animale. L'animale serve il personaggio che l'ha estratto dalla borsa per 1 turno, fino a quando non viene ucciso, o fino a quando non viene ordinato di tornare nella borsa. Usa la tabella seguente per determinare che animale viene estratto.

\begin{table}[h]
\centering
\begin{tabular}{|c|l|}
\hline
\textbf{Tiro 2d8} & \textbf{Tipo di Animale} \\
\hline
2 & Orso nero \\
3 & Orso delle caverne \\
4 & Cinghiale \\
5 & Cammello \\
6 & Felino grande, leone \\
7 & Felino grande, tigre \\
8 & Animale da branco, antilope \\
9 & Animale da branco, capra \\
10 & Animale da branco, bufalo \\
11 & Cavallo da guerra \\
12 & Mulo \\
13 & Topo ordinario \\
14 & Rinoceronte \\
15 & Toporagno gigante \\
16 & Rospo gigante \\
\hline
\end{tabular}

\end{table}

Il cavallo da guerra appare con finimenti e bardatura e accetta il personaggio che l'ha estratto dalla borsa come cavaliere. Gli animali prodotti sono sempre casuali, e solo uno può esistere alla volta. Fino a dieci animali possono essere estratti dalla borsa ogni settimana. Il referee decide casualmente quale borsa viene trovata.

\textbf{Barca Pieghevole}: Una barca pieghevole sembra una piccola scatola di legno—circa 30 centimetri di lunghezza, 15 centimetri di larghezza e 15 centimetri di profondità. Può essere usata per immagazzinare oggetti come qualsiasi altra scatola. Se viene detta una parola di comando, tuttavia, la scatola si dispiega per formare una barca lunga 3 metri, larga 1,2 metri e profonda 0,6 metri. Una seconda parola di comando la fa dispiegare in una nave lunga 7,2 metri, larga 2,4 metri e profonda 1,8 metri. Qualsiasi oggetto precedentemente immagazzinato nella scatola ora riposa dentro la barca o nave.

Nella sua forma più piccola, la barca ha un paio di remi, un'ancora, un albero e una vela latina. Nella sua forma più grande, la barca ha un ponte, posti a sedere singoli per remare, cinque set di remi, un remo di governo, un'ancora, una cabina di ponte e un albero con una vela quadrata. La barca può contenere quattro persone comodamente, mentre la nave trasporta quindici con facilità.

Una terza parola di comando fa sì che la barca o nave si pieghi di nuovo in una scatola. Le parole di comando necessarie potrebbero essere presenti, sia visibili che invisibili, incise nella scatola. In alternativa, le parole di comando potrebbero dover essere cercate attraverso un PNG o una piccola missione.

\textbf{Libro di Saggezza Caotica}: Questo libro è la controparte del libro di saggezza legale. Gli effetti sono opposti in termini di chi ne beneficia. Oltre agli effetti normali per un libro di questo tipo, i personaggi neutrali che maneggiano il libro devono salvare contro incantesimi o diventare caotici. Gli assassini che maneggiano questo libro guadagnano l'uso una tantum di qualsiasi incantesimo di chierico di 1° livello, ma non sono mai in grado di guadagnare questa abilità dallo stesso libro di nuovo.

\textbf{Libro di Incantesimi Infiniti}: Questo tomo è davvero una collezione di pergamene magiche rilegate. Se toccato da un personaggio che è incapace di lanciare incantesimi, subisce 5d4 pf di danno ed è stordito per lo stesso numero di turni. Questo è un evento una tantum per quel particolare personaggio. Successivamente, un personaggio, indipendentemente dall'abilità di lancio incantesimi, è in grado di usare incantesimi sulle pagine del tomo come se lanciati da una pergamena. I contenuti specifici della pagina sono determinati casualmente per pagina. Ci sono 20+1d10 pagine in qualsiasi libro trovato. Riferisciti alla tabella sotto:

\begin{table}[h]
\centering
\begin{tabular}{|c|l|}
\hline
\textbf{Tiro d20} & \textbf{Contenuti Pagina} \\
\hline
1-5 & Incantesimo di chierico \\
6-9 & Incantesimo di druido \\
10-11 & Incantesimo di illusionista \\
12-17 & Incantesimo di mago \\
18-20 & vuota \\
\hline
\end{tabular}

\end{table}

Ogni volta che una pagina viene girata, la pagina precedente diventa vuota. Altrimenti, un personaggio può lanciare l'incantesimo su una pagina aperta 1 volta al giorno. Se l'incantesimo è normalmente utilizzabile dal personaggio (per classe e livello), può lanciarlo 4 volte al giorno. Il proprietario del libro può immagazzinarlo in un'altra località, aperto a una pagina, e beneficiare ancora dei poteri del libro.

Tuttavia, c'è una probabilità base del 10\% ogni volta che un incantesimo viene usato che la pagina si giri da sola. Niente può impedire a una pagina di girarsi. Questa probabilità è modificata da +10\% se l'incantesimo non è di un livello o classe normalmente disponibile al proprietario, o +20\% se il personaggio non è di una classe che usa incantesimi.

\textbf{Libro di Saggezza Legale}: Questo libro magico può essere letto da chierici legali. Lo studio richiede 1 settimana, dopo di che il chierico guadagna permanentemente 1 punto di SAG e abbastanza punti esperienza per portarlo al punto medio del livello successivo. I chierici neutrali che leggono o maneggiano il libro perdono 2d4 × 10.000 PE. I chierici caotici perdono esperienza per ridurli all'inizio del loro livello precedente. Maghi o illusionisti devono salvare contro incantesimi o perdere 1 punto di INT permanentemente. Coloro che riescono nel salvare perdono 2d10 × 10.000 PE. Tutte le altre classi non sono influenzate dal maneggiare il libro, eccetto gli assassini che devono riuscire in un tiro salvezza contro morte. Se il salvare fallisce, l'assassino viene istantaneamente ucciso.

\subsection{Stivali di Danza}
Questi stivali maledetti funzionano come qualsiasi altro stivale magico, determinato casualmente. Tuttavia, quando un personaggio entra in combattimento o tenta di fuggire da un potenziale combattimento, cade sotto gli effetti identici all'incantesimo \textit{danza irresistibile}, senza tiro salvezza consentito.

\subsection{Stivali di Levitazione}
A comando, questi stivali di cuoio permettono al portatore di levitare come se avesse lanciato l'incantesimo \textit{levitazione} su se stesso. La durata è indefinita.

\subsection{Stivali della Velocità}
Questi stivali permettono al portatore di muoversi a 240' per turno per un massimo di 12 ore. Il portatore è esausto dopo questa attività ed è obbligato a riposare per 24 ore.

\subsection{Stivali del Viaggio e del Salto}
Mentre si indossano questi stivali, il portatore non ha bisogno di riposare se impegnato in movimento ordinario. Inoltre, può saltare fino a 10' in alto e a una distanza di 30'.

\subsection{Ciotola del Comando degli Elementali dell'Acqua}
Questa ciotola può essere usata per evocare e controllare un elementale dell'acqua al giorno come l'incantesimo \textit{evocare elementale}. L'utilizzatore deve preparare l'oggetto magico e condurre rituali che richiedono 1 turno prima dell'evocazione. L'evocazione stessa richiede solo 1 round. Una volta evocato un elementale, l'evocatore deve continuare a concentrarsi per dare comandi.

\subsection{Ciotola dell'Annegamento}
Questa ciotola maledetta sembra essere una ciotola del comando degli elementali dell'acqua. Tuttavia, invece di evocare un elementale, una sfera trasparente d'acqua si alza e avvolge la testa dell'utilizzatore. Annega in 2d4 round a meno che un tiro salvezza contro incantesimi non riesca. L'acqua è "appiccicosa" e non può essere scossa via. L'incantesimo \textit{distruggi acqua} (opposto di \textit{crea acqua}) rimuoverà l'acqua.

\subsection{Bracciali dell'Armatura}
Questi oggetti sembrano essere protezioni per polsi o braccia. Garantiscono al portatore una CA come se indossasse un'armatura. Entrambi i bracciali devono essere indossati perché la magia sia efficace, e nessun'altra armatura può essere indossata con i bracciali (magica o non magica). I modificatori di Destrezza si applicano. La protezione offerta dai bracciali può essere combinata con altri effetti magici che alterano la CA, come un anello di protezione o mantello di protezione. Tira sulla tabella sottostante per determinare quale tipo di bracciali viene trovato.

\begin{table}[h]
\centering
\begin{tabular}{|c|c|}
\hline
\textbf{Tiro d00} & \textbf{CA Garantita} \\
\hline
01-06 & 8 \\
07-16 & 7 \\
17-36 & 6 \\
37-51 & 5 \\
52-71 & 4 \\
72-86 & 3 \\
87-00 & 2 \\
\hline
\end{tabular}

\end{table}

Alcuni di questi (5\%) saranno maledetti, abbassando effettivamente la CA del portatore a 9, indipendentemente dai modificatori di DES o mezzi magici per abbassare la CA. Non si realizzerà che i bracciali sono maledetti finché il portatore non entra in combattimento. Questi bracciali possono essere rimossi solo con l'incantesimo \textit{rimuovi maledizione}.

\subsection{Braciere del Comando degli Elementali del Fuoco}
Questo braciere può essere usato per evocare e controllare un elementale del fuoco al giorno come l'incantesimo \textit{evocare elementale}. L'utilizzatore deve preparare l'oggetto magico e condurre rituali che richiedono 1 turno prima dell'evocazione. L'evocazione stessa richiede solo 1 round. Una volta evocato un elementale, l'evocatore deve continuare a concentrarsi per dare comandi.

\subsection{Braciere del Sonno Maledetto}
Questo braciere appare e funziona esattamente come un braciere del comando degli elementali del fuoco. Tuttavia, quando attivato il fumo riempie un raggio di 10' attorno al braciere, causando a tutti gli esseri nell'area di cadere in un sonno maledetto a meno che non venga effettuato con successo un tiro salvezza contro incantesimi. Un elementale del fuoco arriva normalmente, ma è ostile e attacca tutti gli esseri vicini. Le creature che cadono sotto gli effetti della maledizione del sonno dormono indefinitamente, finché non vengono uccise, a meno che non vengano usati \textit{rimuovi maledizione} o \textit{dissolvi magie}.

\subsection{Spilla della Protezione}
Questo appare essere un pezzo di gioielleria d'argento o oro usato per fissare un mantello o una cappa. Oltre a questo compito mondano, può assorbire i missili magici del tipo generato da incantesimo, dispositivo o abilità simile agli incantesimi. Una spilla può assorbire fino a 101 punti ferita di danno dai missili magici prima di fondersi e diventare inutile.

\subsection{Scopa dell'Attacco Animato}
Questo oggetto è indistinguibile nell'aspetto da una scopa normale. È identico a una scopa volante per tutti i test tranne l'uso tentato. Quando usata, la scopa fa un giro della morte con il suo speranzoso cavaliere, scaraventandolo a testa in giù da 1d4+5 piedi da terra (nessun danno da caduta, dato che la caduta è inferiore a 10 piedi). La scopa poi attacca la vittima, colpendo la faccia con l'estremità di paglia o ramoscelli e picchiandolo con l'estremità del manico. La scopa ottiene due attacchi per round con ogni estremità (due colpi con la paglia e due con il manico, per un totale di quattro attacchi per round). L'estremità di paglia causa alla vittima di essere accecata per 1 round quando colpisce. Il manico infligge 1d3 punti di danno quando colpisce. La scopa ha CA 7, 18 punti ferita, e attacca come una creatura da 4 DV.

\subsection{Scopa del Volo Maledetto}
Questa scopa magica sembrerà essere una scopa volante. Tuttavia, quando attivata volerà fino a 50' nell'aria, o all'altezza del soffitto, qualunque sia inferiore, e smetterà di funzionare. Il personaggio e la scopa precipitano al suolo, con il personaggio che subisce il danno da caduta appropriato. La scopa perde tutto l'incantamento.

\subsection{Scopa Volante}
Questa scopa magica leggendaria può volare con un cavaliere fino a 240' per turno. Due cavalieri possono essere trasportati, ma la velocità massima è ridotta a 180' per turno.

\subsection{Incensiere del Controllo degli Elementali dell'Aria}
Questo incensiere può essere usato per evocare e controllare un elementale dell'aria al giorno come l'incantesimo \textit{evocare elementale}. L'utilizzatore deve preparare l'oggetto magico e condurre rituali che richiedono 1 turno prima dell'evocazione. L'evocazione stessa richiede solo 1 round. Una volta evocato un elementale, l'evocatore deve continuare a concentrarsi per dare comandi.

\subsection{Incensiere dell'Evocazione Maledetta}
Questo incensiere sembra e sembra funzionare esattamente come un incensiere del controllo degli elementali dell'aria. Tuttavia, una volta acceso l'incensiere è impossibile da spegnere per 1d4 round. Per ognuno di questi round, emerge un elementale dell'aria che attaccherà tutti gli esseri vicini.

\subsection{Carillon del Cannibalismo}
Questo oggetto appare essere un carillon dell'apertura. Funzionerà come questo oggetto per il primo round del suo uso (e ha 1d4×10 cariche per questo scopo). Tuttavia, al secondo suono tutti gli esseri entro 60' devono salvare contro incantesimi o diventare rabbiosamente affamati, attaccando l'umanoide più vicino per ucciderlo e mangiarlo. Un nuovo tiro salvezza è permesso ogni altro round. Se nessun umanoide è presente, le creature colpite attaccheranno qualsiasi altra creatura presente.

\subsection{Carillon dell'Apertura}
Un carillon dell'apertura è un tubo cavo di mithral lungo circa 1 piede. Quando colpito, emette vibrazioni magiche che causano l'apertura di serrature, coperchi, porte, valvole e portali. Il dispositivo funziona contro sbarre normali, ceppi, catene, chiavistelli e così via. Un carillon dell'apertura dissolve automaticamente anche un incantesimo \textit{blocca portale} o persino una \textit{serratura arcana} lanciata da un mago di livello inferiore al 15°. Il carillon deve essere puntato verso l'oggetto o cancello da liberare o aprire (che deve essere visibile e noto all'utilizzatore). Il carillon viene poi colpito, risuona un tono chiaro, e in 1 round la serratura bersaglio viene sbloccata, il ceppo viene allentato, la porta segreta viene aperta, o il coperchio del forziere viene sollevato. Ogni suono apre solo una forma di chiusura, quindi se un forziere è incatenato, chiuso con lucchetto, bloccato e chiuso con serratura arcana, servono quattro usi di un carillon dell'apertura per aprirlo. Un incantesimo \textit{silenzio} annulla il potere del dispositivo. Ogni uso richiede una carica, e un carillon contiene 2d4×10 cariche prima di incrinarsi e diventare inutile.

\subsection{Mantello dell'Aracnide}
Questo mantello nero, ricamato con un motivo simile a una ragnatela in seta, conferisce al portatore la capacità di arrampicarsi come se un incantesimo \textit{camminata del ragno} fosse stato posto su di lui. Inoltre, il mantello garantisce immunità all'intrappolamento da incantesimi \textit{ragnatela} o ragnatele di qualsiasi tipo—può effettivamente muoversi nelle ragnatele a metà della sua velocità normale. Una volta al giorno, il portatore di questo mantello può lanciare \textit{ragnatela}. Inoltre, ottiene un bonus di +2 su tutti i tiri salvezza contro veleni dai ragni.

\subsection{Mantello della Velenosità}
Questo mantello è solitamente fatto di materiale di lana, anche se può essere fatto di cuoio. Il capo può essere maneggiato senza danno, ma non appena viene effettivamente indossato il portatore viene ucciso istantaneamente senza tiro salvezza. Una volta indossato, un mantello della velenosità può essere rimosso solo con un incantesimo \textit{rimuovi maledizione}; farlo distrugge la proprietà magica del mantello. Se viene poi usato un incantesimo \textit{neutralizza veleno}, è possibile rianimare la vittima con un incantesimo \textit{resurrezione} o \textit{resurrezione}, ma non prima.

\subsection{Mantello di Protezione}
Questo mantello magico appare essere un mantello di stoffa marrone ordinario o di cuoio. Il mantello funziona molto come un anello di protezione, offrendo un bonus alla CA del portatore e a tutti i tiri salvezza. Questi bonus sono cumulativi se il mantello viene indossato con un anello di protezione.

\begin{table}[h]
\centering
\begin{tabular}{|c|c|}
\hline
\textbf{Tiro d00} & \textbf{Bonus} \\
\hline
01-80 & +1 \\
81-91 & +2 \\
92-00 & +3 \\
\hline
\end{tabular}

\end{table}

\subsection{Sfera di Cristallo}
Un mago-utilizzatore o elfo può usare questo oggetto magico ambito per vedere luoghi, persone o oggetti distanti. Qualsiasi cosa può essere vista per un massimo di 1 turno, 3 volte al giorno. L'utilizzatore della sfera di cristallo non può comunicare con coloro che osserva, e non può influenzarli magicamente o altrimenti attraverso la sfera di cristallo.

\subsection{Sfera di Cristallo con Chiaroudienza}
Questo tipo di sfera di cristallo ha tutte le proprietà di quella ordinaria, ma garantisce anche all'osservatore la capacità di sentire qualsiasi rumore o conversazione nel luogo osservato.

\subsection{Sfera di Cristallo con ESP}
Questo tipo di sfera di cristallo ha tutte le proprietà di quella ordinaria, ma garantisce anche all'osservatore la capacità di sentire i pensieri di un essere osservato, proprio come l'incantesimo ESP.

\subsection{Sfera d'Ipnosi di Cristallo}
Questo oggetto maledetto è indistinguibile da una sfera di cristallo normale (determina il tipo specifico casualmente). Tuttavia, chiunque tenti di usare il dispositivo di scrutinio diventa affascinato per 1d6 turni, e un suggerimento telepatico viene impiantato nella sua mente se fallisce un tiro salvezza contro incantesimi. L'utilizzatore del dispositivo crede che la creatura o scena desiderata sia stata vista, ma in realtà è caduto sotto l'influenza di un mago potente, o persino di qualche potere o essere di un altro piano. Ogni uso successivo porta l'osservatore della sfera d'ipnosi di cristallo più profondamente sotto l'influenza del controllore, sia come servitore che come strumento. Nota che durante tutto questo tempo, l'utilizzatore rimane inconsapevole della sua soggiogazione.

\subsection{Cubo di Forza}
Questo dispositivo è largo circa 3/4 di pollice e può essere fatto di avorio, osso o qualsiasi minerale duro. Permette al suo possessore di erigere un muro di forza speciale di 10 piedi per lato attorno alla sua persona. Questo schermo cubico si muove con il personaggio ed è impermeabile alle forme di attacco menzionate nella tabella sottostante. Il cubo ha 36 cariche, che vengono rinnovate ogni giorno. Il possessore preme una faccia del cubo per attivare un particolare tipo di schermo o per disattivare il dispositivo. Ogni effetto costa un certo numero di cariche da mantenere per ogni turno (o porzione di minuto) in cui è in funzione. Inoltre, quando un effetto è attivo, la velocità del possessore è limitata al valore massimo dato nella tabella. Gli incantesimi che influenzano l'integrità dello schermo drenano anche cariche extra. Questi incantesimi (elencati di seguito) non possono essere lanciati dentro o fuori dal cubo:

\begin{table}[h]
\centering
\begin{tabular}{|c|c|c|l|}
\hline
\textbf{Faccia Cubo} & \textbf{Cariche per Turno} & \textbf{Velocità Massima} & \textbf{Effetto} \\
\hline
1 & 1 & 10' & Tiene fuori gas, vento, ecc. \\
2 & 2 & 80' & Tiene fuori materia non vivente \\
3 & 3 & 60' & Tiene fuori materia vivente \\
4 & 4 & 40' & Tiene fuori la magia \\
5 & 6 & 30' & Tiene fuori tutte le cose \\
6 & 0 & Come normale & Disattiva \\
\hline
\end{tabular}

\end{table}

\begin{table}[h]
\centering
\begin{tabular}{|l|c|l|c|}
\hline
\textbf{Forma di Attacco} & \textbf{Cariche Extra} & \textbf{Forma di Attacco} & \textbf{Cariche Extra} \\
\hline
Corno del fragore & 6 & Dardo di fulmine & 4 \\
Muro di fuoco & 2 & Lava, altri fuochi caldi & 2 \\
Passa pareti & 3 & Palla di fuoco ritardata & 3 \\
Disintegrazione & 6 & Sciame di meteore & 8 \\
Porta dimensionale & 5 & Palla di fuoco & 3 \\
\hline
\end{tabular}

\end{table}

\subsection{Cubo di Resistenza al Freddo}
Questo cubo viene attivato o disattivato premendo un lato. Quando attivato, crea un'area a forma di cubo di 10 piedi per lato centrata sul possessore (o sul cubo stesso, se l'oggetto viene successivamente posto su una superficie). La temperatura all'interno di quest'area è sempre di almeno 18°C. Il campo assorbe tutti gli attacchi basati sul freddo. Tuttavia, se il campo viene sottoposto a più di 50 punti di danno da freddo in 1 turno (da uno o più attacchi), collassa nella sua forma portatile e non può essere riattivato per 1 ora. Se il campo assorbe più di 100 punti di danno da freddo in un turno, il cubo viene distrutto.

\subsection{Portale Cubico}
Questo oggetto è realizzato in cornalina. Ognuno dei sei lati del cubo è sintonizzato su un piano, uno dei quali è il piano materiale. Il Signore del Labirinto dovrebbe scegliere i piani a cui sono sintonizzati gli altri cinque lati. Se un lato del portale cubico viene premuto una volta, apre un portale verso un punto casuale sul piano sintonizzato a quel lato. C'è una probabilità del 10\% per turno che un essere da quel piano (determina casualmente) venga attraverso cercando cibo, divertimento o guai. Premere il lato una seconda volta chiude il portale. È impossibile aprire più di un portale alla volta. Se un lato viene premuto due volte in rapida successione, il personaggio che lo fa viene trasportato a un punto casuale sull'altro piano, insieme a tutte le creature entro un raggio di 5'.

\subsection{Caraffa dell'Acqua Infinita}
Se il tappo viene rimosso da questo contenitore dall'aspetto ordinario e viene pronunciata una parola di comando, ne esce una quantità di acqua dolce o salata. Parole di comando separate determinano il tipo, così come il volume e la velocità.

"Ruscello" versa 1 gallone per round.

"Fontana" produce un getto lungo 5' a 5 galloni per round.

"Geyser" produce un getto lungo 20', largo 1' a 30 galloni per round.

L'effetto geyser causa una considerevole contropressione, richiedendo al detentore di essere su terreno stabile e puntellato per evitare di essere buttato giù. La forza del geyser uccide le creature piccole. La parola di comando deve essere pronunciata per fermarlo.

\subsection{Ali del Volo}
Una coppia di queste ali potrebbe apparire come nient'altro che un mantello semplice di vecchia stoffa nera, o potrebbero essere eleganti come una lunga cappa di piume blu. Quando il portatore pronuncia la parola di comando, il mantello si trasforma in un paio di ali di pipistrello o uccello che lo autorizzano a volare. L'attività è estenuante per l'utilizzatore, così che può volare solo per 2 turni con un movimento di 315' (105'), o 4 turni a 180' (60'), o 8 turni a 120' (40'). Dopo il tempo indicato, l'utilizzatore deve riposare per 6 turni. Nota che le ali possono essere usate una volta al giorno per qualsiasi durata, ma una volta che le ali vengono ripiegate in un mantello sono finite per la giornata.

\section{Armi, Armature e Spade}

\subsection{Armi e Armature}
Le armi e armature magiche seguono le stesse restrizioni di classe di tutte le armi e armature normali. Gli oggetti magici avranno un valore "+", o se maledetti avranno un valore negativo "-". Quando un oggetto ha un più, come un pugnale +1, questo significa che i tiri per colpire e i tiri per il danno ricevono un bonus di +1. Le armature con un più ridurranno la CA della quantità specificata. Per esempio, un'armatura di cuoio +1 ridurrebbe la CA a 7 anziché 8.

Gli oggetti maledetti hanno l'effetto opposto, incorrendo in penalità basate sul valore negativo fornito. Gli oggetti maledetti, una volta posseduti da un personaggio, possono essere eliminati solo con un incantesimo \textit{dissolvi il male} o \textit{rimuovi maledizione}. Il proprietario di un oggetto maledetto non crederà che l'oggetto sia maledetto, e resisterà agli sforzi per liberarsene finché uno di questi incantesimi non viene lanciato. Inoltre, il possessore di un'arma maledetta preferirà usare quest'arma in combattimento sopra qualsiasi altra arma.

Le armature magiche della varietà non maledetta sono più leggere e meno ingombranti delle altre armature. Fai riferimento alla tabella sottostante per i pesi delle armature magiche e per tirare il tipo di armatura trovata da un gruppo.

\begin{table}[h]
\centering
\begin{tabular}{|c|l|c|c|}
\hline
\textbf{Tiro d00} & \textbf{Tipo di Armatura} & \textbf{CA Non Modificata} & \textbf{Peso Magico (kg)} \\
\hline
01-10 & Cotta di maglia a bande & 4 & 7 \\
11-30 & Cotta di maglia & 5 & 9 \\
31-60 & Cuoio & 8 & 4,5 \\
61-67 & Imbottita & 8 & 2 \\
68-85 & Corazza a piastre & 3 & 11 \\
86-90 & Cotta di scaglie & 6 & 7 \\
91-95 & Cotta di piastre & 4 & 9 \\
96-00 & Cuoio borchiato & 7 & 7 \\
\hline
\end{tabular}

\end{table}

\subsection{Spade}
Altri poteri oltre ad avere un "+" ai loro attacchi e danni si applicano talvolta alle spade e ad altre armi. Possono anche avere più di un bonus elencato, dove il primo bonus si applica a tutti gli attacchi e danni, e il secondo si applica solo a un gruppo esclusivo di creature. Alcune di queste sono dettagliate di seguito, e altre elencate nelle tabelle del tesoro sono auto-esplicative. Altre armi hanno poteri che l'impugnatore è in grado di comandare. Queste armi sono dettagliate di seguito.

\subsection{Spada +1, Danzante}
Una spada danzante può essere liberata per attaccare da sola. L'impugnatore combatte normalmente per 4 round, ogni round aggiungendo +1 al bonus magico della spada, finché non si raggiunge +4 al quarto round. La spada poi combatte da sola per 4 round a +4. Una volta che la spada inizia a combattere da sola, è considerata impugnata dalla creatura per tutti gli scopi di attacco e inflizione di danno. Mentre danza, occupa lo stesso spazio del personaggio attivante e può attaccare nemici adiacenti. L'arma danzante tornerà automaticamente all'impugnatore originale dopo 4 round di attacco indipendente, purché l'impugnatore sia entro 30' dalla spada.

\subsection{Spada +1, Smembrante}
Questa spada è trattata come +3 ai fini di colpire creature che possono essere colpite solo da armi +3 o superiori. Tuttavia, la spada ha l'abitudine di troncare parti del corpo casuali dagli avversari. Su un tiro di attacco modificato di 20 o superiore, un'appendice casuale viene tranciata, possibilmente anche la testa. Inoltre, a comando la spada può illuminare un'area come una lanterna.

\subsection{Spada +1, Lingua di Fiamma}
Questa spada è +2 contro mostri che si rigenerano, +3 contro mostri aviari o mostri che hanno un attacco basato sul freddo, o sono immuni al fuoco, e questa spada è +4 contro tutti i non morti. Quando l'impugnatore pronuncia un comando, la spada diventa avvolta dalle fiamme. Le fiamme forniscono la stessa quantità di luce di una torcia, e possono essere usate per accendere qualsiasi cosa infiammabile.

\subsection{Spada +1, Succhiatrice di Vita}
Questa spada drena un dado vita o un livello di vita da qualsiasi bersaglio colpito se l'impugnatore pronuncia un comando. Questa spada ha 1d4+4 cariche, e ogni uso di questa abilità drena una carica. Una volta che le cariche sono state usate, la spada funziona come una normale spada +1.

\subsection{Spada +1, Localizza Oggetti}
L'impugnatore può localizzare oggetti come l'incantesimo mago-utilizzatore/elfo una volta al giorno, fino a una distanza di 120'.

\subsection{Spada +1, Lama della Fortuna}
Questa spada garantisce al suo possessore un bonus di +1 su tutti i tiri salvezza. Inoltre, una lama della fortuna conterrà 1d4+1 desideri. Quando l'ultimo desiderio viene usato, la spada rimane una spada +1 e garantisce ancora il bonus di +1 ai tiri salvezza.

\subsection{Spada +1, Lama del Desiderio}
Oltre a funzionare come una spada +1, questa spada contiene 1d4 desideri. Fai riferimento all'incantesimo da mago-utilizzatore \textit{desiderio} per le linee guida su come concedere i desideri. Una volta che i desideri sono stati usati, la spada funziona come una normale spada +1.

\subsection{Spada +1, Ferente}
Questa spada infligge ferite che non possono essere guarite da rigenerazione, incantesimi, o altri mezzi magici tranne che da un desiderio. Le ferite possono essere guarite solo naturalmente. Inoltre, per ogni attacco riuscito, 1 pf aggiuntivo di danno viene subito per 10 round consecutivi o finché la ferita non viene bendata.

\subsection{Spada +2, Berserker}
Questo oggetto maledetto sembra avere le caratteristiche di una spada lunga +2. Tuttavia, ogni volta che la spada viene usata in battaglia, il suo impugnatore va in berserk. Attacca la creatura più vicina e continua a combattere finché non è privo di sensi o morto o finché non rimane nessun essere vivente entro 60'.

\subsection{Spada +2, Affascina Persona}
Oltre a funzionare come una spada +1, questa spada garantisce all'impugnatore la capacità di \textit{affascina persona}, come l'incantesimo mago-utilizzatore/elfo, 3 volte in una settimana.

\subsection{Spada +2, Santo Vendicatore}
Questa spada lunga +2 diventa un santo vendicatore +5 nelle mani di un paladino. Inoltre, quando impugnata da un paladino fornisce una protezione contro la magia in un raggio di 5' equivalente a \textit{dissolvi magie} al livello del paladino. Quando usata contro creature caotiche e "malvagie" la spada fornisce +10 al danno.

\subsection{Spada +2, Ladra di Nove Vite}
Questa spada lunga funziona sempre come una spada lunga +2, ma ha anche il potere di trarre la forza vitale da un avversario. Può farlo nove volte prima che l'abilità venga persa. A quel punto, la spada diventa una semplice spada lunga +2. Deve risultare un tiro di 20 (non modificato) su un tiro di attacco, e la vittima deve salvare con successo contro incantesimi o morire istantaneamente. Se il tiro salvezza ha successo, l'abilità mortale della spada non funziona, nessun uso dell'abilità viene speso, e viene determinato il danno normale.

\subsection{Spada +3, Marca di Gelo}
Questa spada è una spada +6 contro mostri che vivono in un ambiente caldo o usano un attacco basato sul fuoco. La spada emette luce come una torcia quando la temperatura scende sotto 0°F. In tali momenti non può essere nascosta quando sguainata, né la sua luce può essere spenta. Il suo impugnatore è protetto dal fuoco nello stesso modo di chi indossa un anello di resistenza al fuoco. Una marca di gelo spegne tutti i fuochi non magici in un'area di 10' quando toccata da una fiamma.

\subsection{Spada +4, Difensiva}
Su base round per round l'impugnatore di questa spada può distribuire il bonus della spada sia per attacco e danno, sia per la CA. Il bonus alla CA si applica solo agli attacchi in mischia, non agli attacchi a distanza. Per esempio, in un round l'impugnatore può usare +1 per danno e colpire, ma abbassare la CA di 3. Il round successivo questo può essere cambiato.

\subsection{Altre Armi Magiche}

\subsubsection{Freccia +3, Freccia Mortale}
Questa freccia +3 è sintonizzata su un particolare tipo di creatura. Se colpisce tale creatura, il bersaglio muore istantaneamente, senza tiro salvezza. Contro qualsiasi altro bersaglio la freccia funziona come una freccia +3. Queste frecce sono spesso adornate con decorazioni che implicano la creatura che possono uccidere. Per determinare il tipo di creatura a cui la freccia è sintonizzata, tira sulla tabella sottostante. Il Signore del Labirinto può aggiungere a questa lista, o scegliere un tipo di creatura appropriato per la situazione.

\begin{table}[h]
\centering
\begin{tabular}{|c|l|}
\hline
\textbf{Tiro d20} & \textbf{Tipo di Creatura} \\
\hline
1-4 & Aviario \\
5-7 & Pesce \\
8-11 & Mammifero, incluso marsupiale \\
12-13 & Qualsiasi mammifero e aviario \\
14-17 & Anfibi e rettili \\
18-19 & Anfibi, rettili e pesci \\
20 & Tutti quelli elencati sopra \\
\hline
\end{tabular}

\end{table}

\subsubsection{Pugnale +2, Assassino}
Questo pugnale funziona come un pugnale +2 normale contro la maggior parte degli avversari. Tuttavia, su un tiro di attacco naturale di 20, la vittima deve effettuare un tiro salvezza contro veleno o morte, oppure morire istantaneamente.

\subsubsection{Mazza +1, Distruzione}
Questa mazza funziona come una mazza +1 contro la maggior parte degli avversari. Contro non morti di qualsiasi tipo, è una mazza +3, e causa la distruzione completa del non morto su qualsiasi tiro di attacco naturale di 20.

\subsubsection{Tridente +1, Comando Pesci}
Questo tridente garantisce al suo impugnatore il controllo su tutti i pesci entro 60' come l'incantesimo \textit{affascina mostro}. Funziona su pesci normali e giganti, ma non su creature senzienti dei mari o creature planari dell'acqua.

\subsubsection{Tridente +2, Avvertimento}
Questo tridente avverte sempre il suo impugnatore di attacchi a sorpresa o imboscate, dando un bonus di +2 alla sorpresa. Inoltre, il tridente inizia a vibrare quando non morti o creature malvagie si trovano entro 120'.

\subsubsection{Martello da Guerra +2, Lanciatore Nanico}
Questo martello da guerra può essere lanciato come se fosse un'arma da lancio, con una gittata di 60'. Quando lanciato, ritorna automaticamente alla mano dell'impugnatore alla fine del round. Se lanciato da un nano, il martello infligge il doppio del danno normale. Non nani possono usare il martello, ma non ottengono il bonus di danno raddoppiato.


\section{Descrizioni degli Oggetti Magici}

\subsection{Pozioni}
Sebbene le pozioni possano essere trovate in una varietà di tipi di contenitori, inclusi fiaschetti di vetro, ceramica o metallo, la maggior parte contiene solo una dose che conferisce i particolari effetti della pozione per un individuo. La maggior parte delle pozioni non porta etichette e richiede che una piccola quantità venga assaggiata per tentare di identificare il tipo di pozione. Questo non è privo di errori, tuttavia, poiché le pozioni dello stesso tipo possono differire nel loro aroma o sapore a seconda di come sono state create.

Come regola standard, le pozioni hanno effetto nello stesso round del loro consumo, e durano per 1d6+6 turni. Questo principio generale viene sostituito dove la descrizione specifica della pozione indica diversamente. Le pozioni possono generalmente essere consumate a metà dose, così che metà della pozione viene consumata e opera per metà durata.

La creazione di pozioni richiede gli sforzi congiunti di mago-utilizzatori e alchimisti. Un campione della pozione da creare deve essere ottenuto a un certo punto per imparare la formula per la particolare pozione.

\subsubsection{Controllo Animali}
Chiunque prenda questa pozione ottiene la capacità di relazionarsi, comprendere e manipolare le emozioni di un particolare tipo di animale. Il tipo di animale è determinato da un tiro di dado (vedi sotto), e il numero di animali colpiti dipende dalla taglia dell'animale. Si applicano le seguenti taglie generali e quantità: taglia lupo o inferiore, 5d4; fino a taglia umana, 3d4; animali fino a 1.000 libbre, 1d4.

\begin{table}[h]
\centering
\begin{tabular}{|c|l|}
\hline
\textbf{Tiro 1d20} & \textbf{Tipo di Animale} \\
\hline
1-4 & Aviario \\
5-7 & Pesce \\
8-11 & Mammifero, incluso marsupiale \\
12-13 & Qualsiasi mammifero e aviario \\
14-17 & Anfibi e rettili \\
18-19 & Anfibi, rettili e pesci \\
20 & Tutti quelli elencati sopra \\
\hline
\end{tabular}

\end{table}

Nota che a meno che il bevitore di questa pozione non abbia qualche altro mezzo per comunicare direttamente con gli animali influenzati dalla pozione, solo emozioni generali o inclinazioni possono essere manipolate. Tutti gli umanoidi non sono colpiti da questa pozione, e qualsiasi creatura intelligente può effettuare un tiro salvezza per resistere ai suoi effetti.

\subsubsection{Chiaroudienza}
Questa pozione garantisce al bevitore la capacità di sentire fino a 60' tramite le orecchie di un animale. Un animale deve essere in relativa vicinanza. Tuttavia, una barriera di piombo ostacola questo effetto.

\subsubsection{Chiaroveggenza}
Questa pozione garantisce al bevitore la capacità di vedere fino a 60' tramite gli occhi di un animale. Un animale deve essere in relativa vicinanza. Tuttavia, una barriera di piombo ostacola questo effetto.

\subsubsection{Arrampicata}
Questa pozione dura per 1 turno + 5d5 round, durante i quali il bevitore ottiene la capacità di arrampicarsi come un ladro con abilità del 99\% (un tiro di 00 significa fallimento). Tuttavia, le probabilità di fallimento aumentano del 5\% se il personaggio trasporta 100 libbre o più. Inoltre, il tipo di armatura indossata influenzerà l'abilità di arrampicata diminuendo la percentuale di successo come segue:

\begin{table}[h]
\centering
\begin{tabular}{|c|l|}
\hline
\textbf{Penalità} & \textbf{Tipo di Armatura} \\
\hline
1\% & Qualsiasi tipo di armatura magica \\
1\% & Cuoio borchiato \\
2\% & Cotta di maglia \\
4\% & Cotta di scaglie \\
7\% & Cotta di maglia \\
8\% & Armatura a bande e rinforzata \\
10\% & Corazza a piastre \\
\hline
\end{tabular}

\end{table}

\subsubsection{Illusione}
Questa pozione è giustamente chiamata, poiché convince il bevitore che la pozione sia di un altro tipo. Se più di una persona assaggia questa pozione, c'è una probabilità del 90\% che tutti crederanno che la pozione sia dello stesso tipo. Per esempio, una pozione di chiaroudienza potrebbe convincere il bevitore che ci sono suoni in lontananza che non esistono veramente.

\subsubsection{Diminuzione}
Quando bevuta, chi la beve e tutto ciò che trasporta si riduce a 6 pollici di altezza. Il personaggio è così piccolo che se rimane immobile c'è solo una probabilità del 10\% di essere individuato dalle creature vicine. Se viene consumata solo metà della pozione, ridurrà chi la beve al 50\% della sua taglia originale.

\subsubsection{Controllo Draghi}
Bere questa pozione garantisce a chi la beve un potere equivalente a \textit{affascina mostro} su un drago di un tipo determinato dalla tabella sottostante. Ogni pozione colpisce solo un tipo di drago. Chi la beve è in grado di controllare un drago entro 60 piedi e per la durata di 5d4 round. Tira sulla tabella sottostante per il tipo specifico di pozione di controllo draghi.

\begin{table}[h]
\centering
\begin{tabular}{|c|l|}
\hline
\textbf{Tiro 1d10} & \textbf{Tipo di Drago} \\
\hline
1-2 & Nero \\
3 & Blu \\
4-5 & Verde \\
6 & Rosso \\
7-9 & Bianco \\
0 & Oro \\
\hline
\end{tabular}

\end{table}

\subsubsection{ESP}
Questa pozione garantisce un'abilità simile a un incantesimo equivalente all'incantesimo da mago-utilizzatore ed elfo ESP per la durata di 5d8 round.

\subsubsection{Guarigione Extra}
Bere la dose completa di questa pozione ripristina i danni fino a 3d6+3 punti ferita. A differenza della maggior parte delle altre pozioni, questa pozione può essere bevuta in tre porzioni separate e uguali per il beneficio di 1d6 punti ferita di guarigione per ogni terzo della pozione.

\subsubsection{Resistenza al Fuoco}
Chi beve questa pozione è impermeabile a tutte le forme di fiamma ordinaria, che sia piccola come una torcia o grande come un falò infuriato, per 1 turno. Inoltre, questa pozione riduce i danni da altri tipi di fuoco di -2 per dado di danno. Questi tipi di fuoco includono \textit{palla di fuoco}, \textit{muro di fuoco}, e il calore intenso della roccia fusa. Se l'esposizione a queste fiamme richiede un tiro salvezza, viene effettuato con +2 al tiro di dado. Metà della pozione può essere bevuta per una resistenza che dura 5 round, e altri bonus forniti sono dimezzati (-1 al danno e +1 ai tiri salvezza).

\subsubsection{Volo}
Questa pozione garantisce l'abilità simile a un incantesimo equivalente all'incantesimo da mago-utilizzatore ed elfo dello stesso nome.

\subsubsection{Forma Gassosa}
La persona che beve questa pozione, oltre a tutti gli oggetti sulla sua persona, assume una consistenza gassosa traslucida e fluttua a 30 piedi per round. Questa velocità può essere diversa a seconda della velocità del vento naturale nell'ambiente o a causa degli effetti di incantesimi che alterano il vento. Mentre in forma gassosa, la persona colpita può fluire sotto le porte e altri piccoli spazi che non sono sigillati ermeticamente. Sebbene i fulmini magici e il fuoco infliggano alla forma gassosa il danno completo, mentre in forma gassosa chi beve è altrimenti impermeabile ad altri attacchi. Questa pozione deve essere bevuta completamente per avere effetto.

\subsubsection{Controllo Giganti}
Quando bevuta, il bevitore è in grado di controllare fino a due giganti nello stesso modo dell'incantesimo \textit{affascina mostro} per 5d6 round. È permesso un tiro salvezza, e se viene preso di mira solo un gigante riceve -4 a questo tiro. Se vengono presi di mira due giganti, ricevono +2 a questo tiro. Ogni pozione di controllo giganti colpisce solo un tipo di gigante. Consulta la tabella sottostante.

\begin{table}[h]
\centering
\begin{tabular}{|c|l|}
\hline
\textbf{Tiro 1d20} & \textbf{Tipo di Gigante} \\
\hline
1-2 & Delle nuvole \\
3-6 & Del fuoco \\
7-10 & Del gelo \\
11-15 & Delle colline \\
16-19 & Della pietra \\
20 & Della tempesta \\
\hline
\end{tabular}

\end{table}

\subsubsection{Forza del Gigante}
Chi beve questa pozione diventa temporaneamente forte come un gigante del gelo. Questo bonus di forza è accompagnato dall'abilità del gigante di lanciare rocce contro gli avversari, a una distanza di 200' per 3d6 punti ferita di danno. Inoltre, il personaggio infligge il doppio del danno con gli attacchi delle armi. I bonus di forza di questa pozione non possono essere combinati con altri effetti magici che influenzano la forza.

\subsubsection{Crescita}
Chi beve questa pozione raddoppia in dimensioni. Anche la forza aumenta, così che tutto il danno inflitto viene raddoppiato.

\subsubsection{Guarigione}
Chi beve questa pozione recupera danni pari a 1d6+1 punti ferita. Questa pozione cura anche la paralisi. Questa pozione può essere bevuta solo totalmente per avere effetto.

\subsubsection{Eroismo}
Solo un nano, halfling o guerriero può usare questa pozione. Livelli extra e i loro benefici accompagnatori al combattimento vengono temporaneamente garantiti a chi la beve, determinati dal suo livello di esperienza come mostrato nella tabella sottostante. Nota che i punti ferita extra garantiti a causa dell'aumento di livello vengono sottratti per primi quando il personaggio viene ferito.

\begin{table}[h]
\centering
\begin{tabular}{|c|c|}
\hline
\textbf{Livello di chi beve} & \textbf{Livelli Garantiti} \\
\hline
0 & 4 (Guerriero) \\
1-3 & 3 \\
4-7 & 2 \\
8-10 & 1 \\
\hline
\end{tabular}

\end{table}

\subsubsection{Controllo Umani}
Una volta bevuta, questa pozione garantisce al bevitore l'abilità simile a un incantesimo di \textit{affascina persona} per 5d6 round. Molti tipi di umanoidi, semi-umani e umani possono essere colpiti da questa pozione (vedi la tabella sottostante), e vengono colpiti 32 dadi vita/livelli di questi esseri. Solo dadi vita interi vengono considerati quando si calcola quanti individui sono colpiti, e qualsiasi bonus viene eliminato (3+1, 4+2 sono trattati come 3, 4). Il tipo specifico di essere simile all'umano colpito per ogni pozione è determinato sulla tabella sottostante.

\begin{table}[h]
\centering
\begin{tabular}{|c|l|}
\hline
\textbf{Tiro 1d12} & \textbf{Umanoidi Colpiti} \\
\hline
1-2 & Nani \\
3-4 & Elfi \\
5 & Elfi e Umani \\
6-7 & Gnomi \\
8-9 & Halfling \\
10-11 & Umani \\
12 & Altri umanoidi (orchi, gnoll, goblin, ecc.) \\
\hline
\end{tabular}

\end{table}

\subsubsection{Invisibilità}
Quando questa pozione viene bevuta, il bevitore riceve l'abilità simile a un incantesimo di \textit{invisibilità}. Questa pozione può essere consumata in incrementi di 1/8, nel qual caso l'invisibilità garantita dura 1d4+2 turni per dose. Qualsiasi azione di combattimento rimuove l'invisibilità, così che deve essere consumata una nuova dose.

\subsubsection{Invulnerabilità}
Una pozione di invulnerabilità dà al bevitore +2 a tutti i tiri salvezza e garantisce una riduzione della classe di armatura di due gradi.

\subsubsection{Levitazione}
Quando questa pozione viene bevuta, il bevitore riceve l'abilità simile a un incantesimo di \textit{levitazione}.

\subsubsection{Longevità}
Questa pozione rende il bevitore più giovane di 1d12 anni. Questa giovinezza restaurata è possibile non solo per l'invecchiamento naturale, ma anche per l'invecchiamento da effetti magici o di creature. C'è però qualche piccolo pericolo, poiché ogni volta che viene consumata una pozione di longevità c'è una probabilità cumulativa dell'1\% che tutti i precedenti ringiovanimenti da pozioni di questo tipo vengano annullati, portando l'età del personaggio all'età che avrebbe senza gli effetti delle pozioni. Non è possibile bere questa pozione a incrementi.

\subsubsection{Olio di Eterealità}
Questa pozione non viene bevuta, ma questo olio sottile viene applicato al personaggio e a tutti i suoi averi per raggiungere uno stato etereo per 4+1d4 turni. Richiede 3 round perché la pozione produca effetto, e può essere annullata prima della durata applicando un liquido leggermente acido. Quando etereo, un personaggio è invisibile e può passare attraverso qualsiasi oggetto che non sia anch'esso etereo.

\subsubsection{Olio della Scivolosità}
Questo olio viene applicato al personaggio nello stesso modo dell'olio di eterealità. Qualsiasi personaggio così ricoperto non può essere trattenuto o afferrato, e nemmeno avvolto nella presa di serpenti costrittori o altri attacchi di presa, incluse corde, catene o manette che legano, magiche o meno. Semplicemente, niente può ottenere una presa su un personaggio ricoperto da questo olio. Inoltre, gli oggetti possono essere ricoperti con l'olio, e se un pavimento viene ricoperto qualsiasi individuo anche solo in piedi sul pavimento avrà una probabilità del 95\% ogni round di cadere, a causa dello scivolamento. Gli effetti dell'olio durano 8 ore, ma l'olio può essere pulito via precocemente con un liquido contenente alcol, come whiskey, vino o birra robusta.

\subsubsection{Filtro d'Amore}
Chi beve questa pozione diventa affascinato dalla prossima persona o creatura su cui posa gli occhi. Tuttavia, il bevitore diventerà effettivamente affascinato e infatuato dalla persona o creatura se è del sesso preferito e di stirpe razziale simile. L'aspetto di fascino di questa pozione dura per 4+1d4 turni, ma solo \textit{dissolvi magie} farà cessare al bevitore di essere coinvolto da un membro di un sesso preferito.

\subsubsection{Controllo Piante}
Chi beve una pozione di controllo piante è in grado di controllare piante o creature simili a piante (inclusi funghi e muffe) entro un'area di 20 piedi quadrati, a una distanza di 90 piedi. Questa abilità dura per 5d4 round. Le piante e le creature simili a piante possono obbedire ai comandi al meglio delle loro capacità. Per esempio, le viti possono essere controllate per avvolgersi attorno ai bersagli, e le piante intelligenti possono ricevere ordini. Tuttavia, gli esseri vegetali intelligenti ricevono un tiro salvezza contro incantesimi. Simile ad altre abilità simili al fascino, non si può controllare direttamente una creatura vegetale intelligente per infliggere danno a se stessa.

\subsubsection{Veleno}
Questa pozione è altamente variabile nella sua potenza, ed è solitamente un veleno inodore di colore variabile. Il veleno può richiedere ingestione, contatto con la pelle, o applicazione a ferite aperte. La potenza determinerà la facilità con cui un tiro salvezza contro veleno può essere compiuto. Veleni estremamente potenti possono richiedere una penalità da -1 a -4, o veleni più deboli possono fornire un bonus da 1 a 4. Un tiro salvezza fallito risulta in morte.

\subsubsection{Polimorfo (se stesso)}
Questa pozione garantisce l'abilità simile a un incantesimo di \textit{polimorfo se stesso}, come l'incantesimo da mago-utilizzatore ed elfo di quarto livello.

\subsubsection{Velocità}
Questa pozione raddoppia l'abilità di combattimento e movimento per 5d4 round. Così, se chi la beve può normalmente muoversi a 120 piedi, per la durata dell'effetto di questa pozione può muoversi a 240 piedi. Anche il numero di attacchi disponibili raddoppia, ma questa pozione non diminuisce il tempo di lancio degli incantesimi. Questa abilità intensificata non viene senza costo, poiché lo sforzo che mette sul corpo di chi la beve lo invecchia di 1 anno permanentemente.

\end{multicols}

\end{document}