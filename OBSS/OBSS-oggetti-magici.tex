\section{Creare Oggetti Magici}\index{Creare Oggetti Magici}

\begin{changemargin}{0.3cm}{0.3cm}\begin{enfasi}{
Creare è vivere due volte. (Albert Camus)}\end{enfasi}\end{changemargin}\medskip

\begin{multicols}{2}

\label{creare-oggetti-magici}

\lettrine[lines=2, lhang=0.33, loversize=0.25, findent=1.5em]{P}{er} Creare Oggetti Magici è necessario avere le Abilità Creazione oggetti magici.

I costi qui elencati sono quelli di produzione, il ricavo si può attestare almeno attorno al 20\% del prezzo di produzione.

Conoscere l'incantesimo (o averlo a disposizione tramite Pergamena) che si applica all'oggetto è un requisito di ogni oggetto magico che si crea.

\begin{changemargin}{0.3cm}{0.3cm}\begin{narratore}
La creazione di oggetti magici può rompere gli equilibri del gioco. Un personaggio con risorse abbondanti e tempo può arrivare a creare oggetti che spezzano gli equilibri dell'avventura. Suggerisco siano gli NPC, i personaggi non giocanti in gestione al Narratore, a creare gli oggetti più meravigliosi. Allo stesso tempo la vendita di oggetti di valore sopra le 2000mo dovrebbe essere il più limitata possibile.
\end{narratore}\end{changemargin}

\subsubsection{Creare Anelli Magici}\index{Anelli Magici}

Per creare un anello magico, un personaggio ha bisogno di una fonte di calore. Ha anche bisogno di una provvista di materiali, di cui il più ovvio è un anello o pezzi di anello da assemblare. Il costo dei materiali è compreso nel costo della creazione dell'anello.

Il costo di produzione dell'anello è pari a livello*livello*4000, un Anello con Invisibilità costa 2*2*4000=16000 mo

\begin{center}
\includegraphics[width=0.5\linewidth]{immagini/onering2.png}

\textit{Non c'è bisogno di dire che Anello sia...}
\end{center}

Un anello permette di fissare un incantesimo per rendere l'effetto sempre attivo.
L'anello deve avere un valore intrinseco pari almeno a 500mo*livello dell'incantesimo che deve ospitare.

Un anello può ospitare un incantesimo di livello 9 o se più incantesimi il massimo livello è 7.

E' anche possibile inserire un incantesimo ad attivazione, in questo caso consultare i costi delle Verghe.

Forgiare un anello richiede 1 giorno per ogni 1000 mo del prezzo base. In caso di più incantesimi i costi e tempi si sommano.

\medskip

\textbf{Talento di creazione oggetto richiesto}: Creare Oggetti Magici Superiori.

\subsection{Creare Armature e Scudi Magici}\index{Creare Armature Magiche}

Per creare un'armatura o scudo magico, un personaggio ha bisogno di una fonte di calore e di alcuni attrezzi per lavorare il ferro, il legno o il cuoio. Ha anche bisogno di una provvista di materiali, di cui il più ovvio è l'armatura/scudo stessa o i pezzi di armatura da assemblare. Un'armatura/scudo che va incantata deve essere di qualità.

\begin{center}
\includegraphics[width=0.5\linewidth]{immagini/Rustning_Gustav_Vasa.png}

\textit{Armour for Gustav I of Sweden by Kunz Lochner, c. 1540 (Livrustkammaren)}
\end{center}

Se i prerequisiti per la creazione dell'armatura comprendono degli incantesimi, l'incantatore deve conoscere detti incantesimi.

Il costo di produzione di un'armatura magica +1 costa 2050 mo, +2 7500 mo, +3 12000 mo, +4 25000 mo, +5 45000 mo più il prezzo dell'armatura stessa.

Infondere un incantesimo in una armatura ha un costo come se si andasse a creare un anello con quell'incantesimo.

Creare armature/scudi magiche richiede un giorno per ogni 1000 mo del valore del prezzo base.

\medskip

\textbf{Talento di creazione oggetto richiesto}: Creare Oggetti Magici.

\subsection{Creare Armi Magiche}\index{Creare Armi Magiche}

Per creare un'arma magica, un personaggio ha bisogno di una fonte di calore e alcuni attrezzi per lavorare il ferro od il materiale con cui è fatta l'arma. Ha anche bisogno di una provvista di materiali, di cui il più ovvio è l'arma stessa o i pezzi di arma da assemblare. Solo un'arma di qualità può essere incantata per diventare un'arma magica, e il suo costo va aggiunto al costo totale di incantamento per determinare il valore finale di mercato.

Un'arma magica deve avere almeno bonus di +1 per avere una qualsiasi capacità speciale o incantesimo.

\medskip

\begin{center}
\includegraphics[width=0.6\linewidth]{immagini/exacaliburfuori.png}

\textit{The drawing of the sword from the stone, Henrietta Elizabeth Marshall's Our Island Story (1906)}
\end{center}

\medskip

Se i prerequisiti per la creazione dell'arma comprendono degli incantesimi, l'incantatore deve conoscere detti incantesimi.

Nel momento della creazione, l'incantatore deve decidere se l'arma emana luce o meno, come effetto secondario della magia infusa nell'arma. Questa decisione non influenza il prezzo o il tempo di creazione, ma una volta che l'oggetto è completato, la decisione è definitiva.

Creare armi doppie viene considerato analogo a creare due armi per quanto riguarda il costo, il tempo e le Capacità Speciali.

Il costo di produzione di un Arma +1 è 1200 mo, +2 4000 mo, +3 11000 mo, +4 25000 mo, +5 45000 mo più il prezzo dell'arma (influente solo se è di un qualche materiale raro o prezioso).

Il costo di produzione di una Freccia +1 è 20 mo, +2 75 mo, +3 325 mo. Incantamenti più potenti sono estremamente rari.

Infondere un incantesimo in un arma ha un costo come se si andasse a creare un anello con quell'incantesimo, se continuativo, altrimenti se mono uso come una pozione.

Creare un'arma magica richiede una giornata per ogni 1000 mo del valore del prezzo base.

\medskip

\textbf{Talento di creazione oggetto richiesto}: Creare Oggetti Magici.

\subsection{Creare Bacchette}\index{Creare Bacchette}

Il costo di produzione della Bacchetta è pari a livello*livello*800, una Bacchetta con Invisibilità costa 2*2*800=4200 mo

Una bacchetta è un oggetto magico che conserva in se un incantesimo caricato in precedenza.

Per ricaricare una bacchetta un incantatore deve infondere lo stesso incantesimo e avere l'Abilità Creare oggetti magici. La bacchetta recupera una carica ma l'incantatore oltre ad avere usato Punti Magia spende l'equivalente di 100*livello monete d'oro in componenti.

Una Bacchetta può contenere come massimo livello di incantesimo 5.

Per creare una bacchetta, un personaggio ha bisogno di una provvista di materiali, di cui il più ovvio è una bacchetta o i pezzi di una bacchetta da assemblare. Le bacchette sono sempre pienamente cariche (20 cariche) all'atto della creazione.

L'incantatore deve conoscere l'incantesimo che inserisce nella Bacchetta.

\begin{center}
\includegraphics[width=0.5\linewidth]{immagini/wand.png}
\end{center}

Creare una bacchetta richiede 1 giorno per ogni 1000 mo del valore del prezzo base.

\medskip

\textbf{Talento di creazione oggetto richiesto}: Creare Oggetti Magici.

\subsection{Creare Bastoni}\index{Creare Bastoni}

\textbf{Costi Base dei Bastoni}

\bigskip

Il costo di produzione del Bastone è pari a livello*livello*1200, un Bastone con Invisibilità costa 2*2*1200=4800 mo

\bigskip

Un Bastone è un oggetto magico dove si caricano una o più incantesimi.

Quando un bastone viene attivato è possibile usare un incantesimo alla volta.

Per creare un bastone, un personaggio ha bisogno di una provvista di materiali, di cui il più ovvio è un bastone o i pezzi di un bastone da assemblare.

I bastoni sono sempre pienamente carichi, 10 cariche, all'atto della creazione.

\begin{center}
\includegraphics[width=0.3\linewidth]{immagini/staff2.png}
\end{center}

Una Bastone può contenere come massimo livello di incantesimo 8, o in caso di diversi incantesimi il massimo livello è il 6.

Creare un bastone richiede 1 giorno per ogni 1000 mo del prezzo base.

\medskip

\textbf{Talento di creazione oggetto richiesto}: Creare Oggetti Magici Superiori.

\subsection{Creare Pergamene}\index{Creare Pergamene}\index{Pergamene}\index{Isy Scroll}
\index{Pergamene facili}\index{Comprare incantesimi}
\medskip

Esistono due tipologie di Pergamene magiche, quelle eseguibili da tutti (dette ISY SCROLL, o Facili) e quelle invece che richiedono la capacità magica di lanciare incantesimi, ovvero Competenza Magica maggiore o uguale a 1.

Le pergamene facili hanno un costo di produzione pari a livello*livello*160 mo.

Le pergamene normali, non facili, hanno un costo di produzione pari a livello*livello*80 mo.

Il costo della pergamena va valutato in base alla rarità e livello dell'incantesimo. Un incantesimo molto raro o di alto livello può facilmente costare livello*livello*livello*80.

\begin{center}
\includegraphics[width=0.4\linewidth]{immagini/scroll3.png}
\end{center}

Se una pergamena include più incantesimi il costo è pari alla somma dei vari incantesimi. Su una pergamena ISY SCROLL non possono esserci incantesimi da pergamena normali e vice versa.

L'incantatore deve conoscere gli incantesimi che inserisce nella pergamena. Per preparare una pergamena è necessario 30 minuti di lavoro per livello di incantesimo presente.

Una pergamena ISY può contenere incantesimi di livello 3 come massimo, mentre una pergamena normale può contenere come massimo livello di incantesimo 9, in caso di più incantesimi il massimo livello è 8.

\medskip

Per leggere una pergamena è necessario:\\

\textbf{in caso di pergamene ISY SCROLL}:

- per comprendere il contenuto è sufficiente una prova di Intelligenza (o Arcana se conosciuta) a difficoltà DC 10

- per poter leggere e lanciare l'incantesimo della pergamena è necessaria una prova di Intelligenza (o Arcana se conosciuta) a difficoltà 12.\\

\medskip

\textbf{in caso di pergamene normali}:\\

- per comprenderne il contenuto è necessaria una prova di Arcana a difficoltà 15

- per poter leggere e lanciare l'incantesimo della pergamena è necessaria una prova di Arcana a difficoltà 11+Livello dell'incantesimo ed avere accesso alla Lista di Magia dell'incantesimo contenuto e che questo sia di livello pari al massimo lanciabile +1.

\medskip

Il \textbf{tempo di lancio} di un incantesimo da una pergamena è pari al tempo di lancio dell'incantesimo presente.

\textbf{Talento di creazione dell'oggetto richiesto}: Creare Oggetti Magici

Competenza usata nella creazione: Arcana o Professione (scrivano).

\textbf{Nota}: un Tomo della Magia è equivalente ad un insieme di pergamene normali. Un personaggio in situazione disperate può leggere la pagina dell'incantesimo dal Tomo della Magia e manifestare la magia come se fosse da una pergamena. Le pagine contenenti l'incantesimo si polverizzeranno e l'incantatore dovrà trovare una sorgente da dove copiare nuovamente l'incantesimo su Tomo.\index{Tomo di Magia come pergamena}

\begin{center}
\includegraphics[width=0.6\linewidth]{immagini/potion2.png}

\textit{A witch, raising her arm above a flaming cauldron, recites a spell; a young woman kneels in front of the cauldron. Mezzotint by J. Dixon after J.H. Mortimer, 1773}
\end{center}

\subsection{Creare Pozioni}\index{Creare Pozioni}\index{Pozioni}

Una pozione contiene l'infuso di un solo incantesimo, ogni pozione è quindi monouso.

\medskip

Il costo di produzione della Pozione è pari a livello*livello*80, una Pozione con Invisibilità costa 2*2*80=320 mo

\bigskip

Per creare una pozione, un personaggio ha bisogno di un piano di lavoro orizzontale e alcuni contenitori per mescere i liquidi, oltre a una fonte di calore per bollire l'infuso.

Una Pozione può contenere come massimo livello di incantesimo 3.

Tutti gli ingredienti e i materiali per mescere una pozione devono essere freschi e mai usati.

L'incantatore deve conoscere l'incantesimo che inserisce nella pozione. Il tempo di preparazione di una pozione è pari al doppio del livello dell'incantesimo contenuto in ore.

\medskip

\textbf{Talento di creazione dell'oggetto richiesto}: Distillare Pozioni.

\subsection{Creare Verghe}\index{Creare Verghe}\index{Verghe}

Una verga è una bacchetta speciale che è capace di rigenerare le proprie cariche. Sono oggetti preziosi e molto costosi.

Per creare una verga, un personaggio ha bisogno di una provvista di materiali, di cui il più ovvio è una verga o i pezzi di una verga da assemblare.

\medskip

Il costo di produzione della Verga è pari a livello*livello*3200, una Verga con Invisibilità costa 2*2*3200=12800 mo

\bigskip

Una verga è in grado di lanciare 1 volta al giorno il proprio incantesimo.

Moltiplicare il costo per 4 se è in grado di lanciarla 2 volte, moltiplicare per 8 se è in grado di lanciarla 3 volte al giorno.

Si può anche lanciare una volta in più nel giorno l'incantesimo contenuta nella verga, dopo di che la verga si distrugge.

Una Verga può contenere come massimo livello di incantesimo 3.

L'incantatore deve conoscere l'incantesimo che inserisce nella Verga.

Creare una verga richiede 1 giorno per ogni 1000 mo del prezzo base.

\textbf{Talento di creazione oggetto richiesto}: Creare Oggetti Magici Superiori.

\subsection{Aggiungere Nuove Capacità}\index{Aggiungere Nuove Capacita}

A volte, la mancanza di fondi o tempo rende impossibile realizzare l'oggetto magico voluto, ma fortunatamente è possibile potenziare o modificare un oggetto magico creato. Solo il tempo, l'oro ed i vari prerequisiti richiesti dalla nuova capacità che si vuole aggiungere all'oggetto magico pongono delle restrizione sul tipo di poteri addizionali che uno può infondere.

Il costo per aggiungere capacità addizionali ad un oggetto è lo stesso che se l'oggetto non fosse magico, meno il valore dell'oggetto originale. Quindi, una spada lunga +1 può diventare una spada lunga vorpal +2, e il costo della creazione è uguale a quello di una spada lunga vorpal +2 meno il costo di una spada lunga +1.

Quando si determina il prezzo di un oggetto magico inventato bisogna considerare molti fattori. Il modo più semplice per decidere il prezzo è confrontare il nuovo oggetto a un oggetto che ha già un prezzo, e usare tale prezzo come guida.

\end{multicols}

\vfill

\begin{center}
\includegraphics[width=0.2\linewidth]{immagini/Rod_of_asclepius.png}

\textit{Il bastone di Asclepio è un antico simbolo greco associato alla medicina. Consiste in un serpente attorcigliato intorno ad una verga.}
\end{center}


\pagebreak

\section{Regole su Oggetti Magici}\index{Regole su Oggetti Magici}\hypertarget{identificareom}{}

\begin{multicols}{2}

\lettrine[lines=2, lhang=0.33, loversize=0.25, findent=1.5em]{Q}{ueste} sono le indicazioni su l'utilizzo degli oggetti magici.

\label{oggetti-magici}
\begin{itemize}
\item
Un personaggio può \textbf{portare numerosi (fino a 12) oggetti magici} su di sé ma per determinare il bonus alla Difesa non si possono sommare più di 2 oggetti (es. 1 anello magico ed un braccialetto). Armatura e Scudo non si considerano in questo conteggio.
\item
Lo stesso principio vale per il bonus ai \textbf{Tiri Salvezza}, puoi sommare solo i bonus provenienti da due oggetti.
\item
Se il bonus è alle \textbf{Caratteristiche} si conta solo quello con il bonus maggiore.
\item
Un personaggio \textbf{non può portare più di due anelli magici} altrimenti entrano in risonanza causando 1d6 di danno (non riducibile o curabile magicamente) a round per ogni anello oltre il secondo.
\item
Per \textbf{riconoscere un oggetto magico} e le sue capacità è necessario l'incantesimo Identificare. DC 25 1 Minuto. Con punteggio Arcana 6 costa 3 round, 12 costa 1 round, 18 costa 1 Azione.
\item
Un \textbf{oggetto magico che manifesta incantesimi} non esegue alcuna Prova di Magia. Il \textbf{Tiro Salvezza} che impone, se non specificato, è pari a 10 + livello*2 dell'incantesimo che manifesta.\index{Tiro Salvezza per incantesimi da oggetto}
\item
\textbf{Attivare capacità magiche}: se non indicato diversamente attivare un abilità magica di un oggetto costa 2 Azioni.
\item
Un oggetto magico che fornisce un \textbf{bonus (o malus) statico} applica il suo valore anche se l'oggetto non è stato identificato, sarà il Narratore ad applicare silenziosamente questo bonus alla Difesa, Tiro per Colpire, Tiri Salvezza... informando il giocatore che percepisce come l'oggetto interagisca con la situazione.

\end{itemize}

\subsubsection{Armi}

\textbf{Armi}: un'arma con una capacità speciale deve avere almeno bonus di +1. Le armi non possono avere la stessa capacità speciale più di una volta.

Il bonus magico di un \textbf{arma può essere identificato} a seguito di due critici in un Tiro per Colpire oppure dedicando 1 ora di allenamento, eventuali talenti o abilità magiche rimangono celate.

\subsubsection{Armature e Scudi}

\textbf{Armi}: un'armatura o scudo con una capacità speciale deve avere almeno bonus di +1. Armature e Scudi non possono avere la stessa capacità speciale più di una volta.

\textbf{Armature}: ogni +2 magico si abbassa di 1 la penalità di Malus Competenze e di un dado la penalità alle Prove di Magia.

\textbf{Scudi}: ogni +2 magico si toglie un dado alla Prova di Magia

\textbf{Il costo di Armi e Armature:} di dimensioni superiori alle Medie è almeno il doppio (o quadruplo in base alla taglia). Armature piccole o Armi piccole pur richiedendo meno materiale costano la medesima cifra delle armi e armature medie.

\subsection{Taglia e Oggetti Magici}

\label{taglia-e-oggetti-magici}

Quando un capo di vestiario o un gioiello magici vengono scoperti, il più delle volte la taglia non è un problema: molti vestiti magici sono di facile utilizzo per tutti oppure si adattano magicamente a chi li indossa. Di regola, la taglia non dovrebbe impedire ai personaggi di varia tipologia fisica l'utilizzo di un oggetto magico.

Ci possono essere delle rare eccezioni, specie con gli oggetti realizzati per una razza specifica.

Le armi e le armature rinvenute casualmente hanno una probabilità del 30\% di essere Piccole (01--30), del 60\% di essere Medie (31--90), e del 10\% di essere di un'altra taglia.

\subsection{Oggetti Magici sul Corpo}\index{Oggetti Magici sul Corpo}

\label{oggetti-magici-sul-corpo}

Molti oggetti magici devono essere indossati da un personaggio che voglia usarli o beneficiare delle loro capacità. Per una creatura di forma umanoide è possibile indossare fino a 12 oggetti magici alla volta. Ognuno di questi oggetti deve essere indossato sopra una parte specifica del corpo denominata "slot".

Un corpo di forma umanoide può portare addosso Equipaggiamento magico consistente di un oggetto per ognuno dei gruppi seguenti, legato alla parte del corpo sulla quale viene indossato l'oggetto.

\textbf{Anello} (due al massimo): anelli.

\textbf{Vesti}: corazze, armature, tuniche e vesti

\textbf{Cintura}: cinture.

\textbf{Collo}: amuleti, collane, medaglioni, scarabei, spille, talismani e sciarpe

\textbf{Mani}: guanti e guanti d'arme.

\textbf{Occhi}: occhi, occhiali e lenti.

\textbf{Piedi}: scarpe, stivali e pantofole.

\textbf{Polso}: braccialetti e bracciali.

\textbf{Scudo}: scudi.

\textbf{Spalle}: cappe e mantelli.

\textbf{Testa}: cappelli, diademi, elmi, maschere, corone, fasce e filatteri

\textbf{Torace}: camicie, giubbe, maglie e manti.

\medskip

Naturalmente, un personaggio può possedere quanti oggetti vuole di uno stesso tipo. Ma oggetti magici dello stesso tipo addizionali, oltre a quelli previsti negli slot, non funzioneranno.

\subsection{Tiri Salvezza Contro i Poteri degli Oggetti Magici}\index{Tiri Salvezza}

\label{tiri-salvezza-contro-i-poteri-degli-oggetti-magici}

Gli oggetti magici normalmente riproducono incantesimi o altri effetti magici. Per un Tiro Salvezza contro la magia o un effetto magico generato da un oggetto magico, la DC è 10 + livello dell'incantesimo manifestato x2 se non specificato diversamente.

\subsection{Danneggiare gli Oggetti Magici}\index{Danneggiare gli Oggetti Magici}

\label{danneggiare-gli-oggetti-magici}

Un oggetto magico non deve compiere un Tiro Salvezza a meno che non sia incustodito, sia il bersaglio specifico dell'effetto, o il suo possessore ottenga un 3 naturale al suo Tiro Salvezza.

Gli oggetti magici hanno sempre diritto a un Tiro Salvezza contro qualcosa che potrebbe danneggiarli, anche quando un oggetto normale dello stesso tipo non avrebbe alcuna possibilità di effettuare un Tiro Salvezza. Gli oggetti magici usano sempre lo stesso bonus ai Tiri Salvezza, indipendentemente dal tipo (Tempra, Riflessi o Volontà). Il bonus ai Tiri Salvezza di un oggetto magico è pari a 2 + 2xlivello dell'incantesimo più potente che ospitano (oppure un +4 per ogni +1 che hanno). Le sole eccezioni a questa regola sono gli oggetti magici intelligenti, che effettuano i Tiri Salvezza su Volontà basandosi sul loro punteggio di Saggezza.

\subsection{Riparare gli Oggetti Magici}\index{Riparare gli Oggetti Magici}
\label{riparare-gli-oggetti-magici}

Per riparare un oggetto magico occorrono materiali e tempo, pari alla metà del tempo e del costo per crearlo.

\subsection{Cariche, Dosi e Usi Multipli}\index{Cariche}\index{Dosi}\index{Usi Multipli}

\label{cariche-dosi-e-usi-multipli}

Molti oggetti, e in modo particolare le bacchette e i bastoni, hanno un potere limitato al numero di cariche che contengono. Normalmente gli oggetti dotati di cariche non superano mai il massimo di 20 cariche (10 per i bastoni). Se oggetti simili vengono trovati come parte casuale di un tesoro, si tira un 5d6 e si divide per 2 per determinare il numero delle cariche rimaste (arrotondando per difetto, minimo 1). Se un oggetto ha un numero massimo di cariche diverso da 20, si tira casualmente per stabilire quante cariche sono rimaste.

I prezzi indicati si riferiscono agli oggetti al massimo delle loro cariche (quando un oggetto viene creato, ha sempre il massimo delle cariche). Il valore di un oggetto dipende dal numero di cariche residue, in caso di oggetti che possono avere un uso anche con poche o senza cariche, il valore rimane più alto.

\end{multicols}

\subsection{Acquisire Oggetti Magici}\index{Acquisire Oggetti Magici}\index{Tabella Acquisire Oggetti Magici}

\label{acquisire-oggetti-magici}

\bigskip

\begin{tabular}{lllll}
\textbf{Dimensioni Comunità} & \textbf{Valore Base} & \textbf{Comune} & \textbf{Non Comune} & \textbf{Raro}\\
\toprule
Insediamento  & 50mo & 1d2 oggetti     && \\
Borgo         & 200mo& 1d4 oggetti     && \\
Villaggio     & 500mo& 1d6 oggetti     & 1d2 oggetti    & \\
Piccolo paese & 1000mo               & 1d4 oggetti     & 1d2 oggetti    & \\
Grande paese  & 2000mo               & 1d6 oggetti     & 1d4 oggetti    & 1d2 oggetti\\
Piccola città & 4000mo               & 2d4 oggetti     & 1d6 oggetti    & 1d4 oggetti\\
Grande città  & 8000mo               & 3d4 oggetti      & 2d4 oggetti    & 1d6 oggetti\\
Metropoli     & 16000mo              & {*}             & 3d4 oggetti    & 2d4 oggetti\\
\end{tabular}

{*} In una metropoli si trovano quasi tutti gli oggetti magici minori.

\begin{multicols}{2}

\bigskip

Gli oggetti magici sono preziosi e la maggior parte delle grandi città ha almeno uno o due fornitori di oggetti magici, dal semplice venditore di pozioni ad un fabbro specializzato nel forgiare spade magiche. Naturalmente, non ogni oggetto in questo manuale è disponibile in ogni città.

Le linee guida seguenti aiutano i Narratori a determinare quali oggetti sono disponibili in una specifica comunità. Esse presuppongono una campagna con un livello medio di magia. Alcune città potrebbero deviare di molto da questa linea di base a discrezione del Narratore. Il Narratore dovrebbe tenere una lista degli oggetti disponibili da ogni mercante e dovrebbe rimpinguare occasionalmente le scorte con nuove acquisizioni.

Il numero ed i tipi di oggetti magici disponibili in una comunità dipendono dalla sua dimensione. Ogni comunità ha un valore base legato ad essa (vedi Tabella: Oggetti Magici Disponibili).

c'è una probabilità del 75\% che qualsiasi oggetto di quel valore o inferiore si possa trovare in vendita facilmente in quella comunità. Inoltre, la comunità ha un certo numero di altri oggetti in vendita. Questi oggetti sono determinati a caso e sono ripartiti in categorie (minore, medio o maggiore).

Dopo aver determinato il numero di oggetti disponibili in ogni categoria, consultate il capitolo Generazione casuale degli Oggetti Magici per determinare il tipo di ogni oggetto (pozione, pergamena, anello, arma,ecc.) prima di passare alle tabelle specifiche per stabilire l'oggetto esatto. Ritirate ogni volta che gli oggetti non si adeguano al valore base della comunità.

Se l'uso della magia nella campagna in cui si gioca è raro, occorre dimezzare il valore base e il numero di oggetti in ogni comunità. Nelle campagne con magia estremamente rara o senza magia potrebbero non esserci affatto oggetti magici in vendita I Narratori che conducono questo tipo di campagne dovrebbe prevedere delle modifiche alle sfide affrontate dai personaggi data la mancanza di oggetti magici.

Le campagne con abbondanti oggetti magici potrebbero avere comunità con il doppio del valore base stabilito e degli oggetti magici casuali disponibili. In alternativa, si potrebbe stabilire che tutte le comunità siano di una categoria di dimensione maggiore allo scopo di stabilire gli oggetti magici disponibili. In una campagna con magia molto comune, tutti gli oggetti magici si possono acquistare in una metropoli.

Oggetti e attrezzi non magici sono in genere disponibili in una comunità di qualsiasi dimensione a meno che l'oggetto non sia molto costoso, come un'armatura completa, o fatto di un materiale insolito, come una spada lunga in adamantio. Questi oggetti dovrebbero seguire la linea guida del valore base per determinare la loro disponibilità, a discrezione del Narratore.

\end{multicols}

\vfill

\begin{center}
\includegraphics[keepaspectratio,width=0.90\textwidth]{immagini/Alchemical_laboratory_Wellcome_M0005193.png}

\textit{Alchemical laboratory}
\end{center}

\pagebreak


\section{Generazione casuale degli Oggetti Magici}\index{Generazione casuale degli Oggetti Magici}\label{generazionetesorimagici}\hypertarget{generazionetesorimagici}{}

\begin{changemargin}{0.3cm}{0.3cm}\begin{enfasi}
{Come ogni amore non corrisposto, anche quello per le cose alla lunga si paga. (Adolfo Bioy Casares)}
\end{enfasi}\end{changemargin}\medskip

\begin{multicols}{2}

Il Narratore nella preparazione dell'avventura può piazzare gli oggetti magici che preferisce, che ce ne sia bisogno, oppure in puro stile OSR affidarsi ad una generazione casuale.

Questo approccio non è sempre suggerito, i risultati potrebbero stravolgere l'avventura se non tutta la campagna! Solitamente se un "nemico" ha un oggetto magico c'è un motivo e questo oggetto avrà uno scopo. Rimane il fatto che ogni tanto, tirare dadi sulle tabelle di generazione causale dei tesori magici da molta soddisfazione ed è divertente!

\begin{changemargin}{0.3cm}{0.3cm}\begin{narratore}   %box narratore
Yeru è un mondo a basso profilo magico, gli oggetti magici esistono ma sono rari ed ancor di più quelli più potenti. Mentre pozioni naturali e piccoli ninnoli possono essere trovati ovunque quello che ha impedito la creazione di tanti oggetti è il costo per creare gli stessi. Forse un tempo erano più accessibili ma ora la creazione degli oggetti più potenti, ed intendo spade +3 non oggetti meravigliosi o quasi unici, richiede risorse che quasi nessuno possiede o ha interessi a spendere.

Questo fa si che se si vuole avere qualche speranza di ottenere qualche oggetto magico è necessario esplorare le zone più antiche, meno conosciute... e più si scende in profondità più c'è possibilità di trovare qualcosa.
\end{narratore}\end{changemargin}

Come prima cosa è necessario stabilire che tipologia di oggetto si andrà a generare.

\medskip

\textbf{Tabella: Tipologia di Oggetto magico}\index{Tabella Tipologia di Oggetto magico}

\medskip

\begin{tabular}{lc}
\textbf{Tipologia di oggetto magico}&\textbf{3d6}\\
Amuleti, Collane, Gioielli&3-4\\
Cinture, Elmi, Stivali e Guanti&5-6\\
Armature e Scudi&7-8\\
Armi Magiche&9-10\\
Pozioni, Filtri e Olii&11-13\\
Bacchette, Bastoni e Verghe&14\\
Anelli&15\\
Cappelli, Mantelli, Occhiali, Tuniche&16\\
Manuali e Tomi&17\\
Oggetti Magici vari&18\\
\end{tabular}

\subsubsection{Armi}

\textbf{Tabella: Generazione Armi}\index{Tabella Generazione Armi}

\medskip

\begin{tabularx}{0.45\textwidth}{lX}
\textbf{1d100} & \textbf{Bonus magico}\\
1-50 &  +1\\
51-65 & -1 Maledetta\\
66-72 & +2\\
73-76 & +3\\
77-79 & +4\\
80 &    +5\\
81-87 & ritira + Capacità Speciale Armi Tipo 3\\
88-91 & ritira + Capacità Speciale Armi Tipo 2\\
92-94 & ritira + Capacità Speciale Armi Tipo 1\\
95-100 &-2 Maledetta\\
\end{tabularx}

\medskip

Quando nel \textit{Bonus Magico} c'è scritto \textit{ritira + Capacità Speciale Armi Tipo...} significa che devi ritirare il 1d100, ignorando altri risultato sopra 80 e tenere il bonus magico ottenuto, poi potrai tirare sulla \textit{Tabella Capacità Speciale Armi Tipo...} risultante.

\medskip

\textbf{Tabella: Capacità Speciale Armi Tipo 1}\index{Tabella Capacità Speciale Armi Tipo 1}

\medskip

\begin{tabular}{ll}
\textbf{1d100} & \textbf{Capacità Speciale Armi Tipo 1}\\
1-8 &Accumula Incantesimi\\
9-16 &Anatema\\
17-21& Danzante\\
22-27& Difensiva\\
28-34& Distruttrice dei Giganti\\
35-41& Distruzione\\
42-47& Energia Luminosa\\
48-54& Gloriosa\\
55-60& Guardiana\\
61-63& Fortunata\\
65-70& Ladra delle Nove Vite\\
71-73& Sacra\\
74-80& Tocco Fantasma\\
81-86& Vampira\\
87-92& Velocità\\
93-99& Arma Maledetta\\
100 &Vorpal\\
\end{tabular}

%\medskip
%
%\begin{center}
%\includegraphics[width=0.55\linewidth]{immagini/armatura-med.png}
%\end{center}
%
%\medskip

\textbf{Tabella: Capacità Speciale Armi Tipo 2}\index{Tabella Capacità Speciale Armi Tipo 2}

\medskip

\begin{tabular}{ll}
\textbf{1d100} & \textbf{Capacità Speciale Armi Tipo 2}\\
1-8& Conduttiva\\
9-16&Coraggiosa\\
17-23&Crudele\\
24-30&Duello\\
31-36&Furia Innata\\
37-43&Impulso Vitale\\
44-58&Immorale\\
59-60&Letale\\
61-65&Perfida\\
66-69&Pietosa\\
70-74&Punitiva\\
75-79&Maledetta\\
80-85&Sprezzante\\
87-95&Terrore\\
95-100 &Titanica\\
\end{tabular}

\bigskip

\begin{center}
\includegraphics[width=0.7\linewidth]{immagini/shield1.png}
\end{center}

\textbf{Tabella: Capacità Speciale Armi Tipo 3}\index{Tabella Capacità Speciale Armi Tipo 3}

\medskip

\begin{tabular}{ll}
\textbf{1d100} & \textbf{Capacità Speciale Armi Tipo 3}\\
1-4& Adattiva\\
5-8 &Affilata\\
9-12& Ammazza Draghi\\
13-16& Ammazza Giganti\\
17-20 &Cacciatore\\
21-24 &Corrosiva\\
25-28& Designante\\
29-32& Distanza\\
33-36& Estingui Fuoco\\
37-40 &Fanatizzante\\
41-44 &Ferimento\\
45-48 &Folgorante\\
49-52 &Gelida\\
53-56 &Infuocata\\
57-60 &Marina\\
61-65 &Mascheramento\\
65-69 &Munizione Fantasma\\
70-72 &Munizioni Infinite\\
73-76 &Planare\\
77-80 &Prensile\\
81-82 &Ricercante\\
83-84 &Ritornante\\
85-88 &Tonante\\
89-91 &Trasformante\\
92-95& Trovacose\\
96-100 & Arma Maledetta\\
\end{tabular}

\medskip

\subsubsection{Armature e Scudi}

\textbf{Tabella: Generazione Armature/Scudi}\index{Tabella Generazione Armature/Scudi}

\medskip

\begin{tabularx}{0.45\textwidth}{lX}
\textbf{1d100} & \textbf{Bonus magico}\\
1-50 &  +1\\
51-65 & -1 Maledetta\\
66-72 & +2\\
73-76 & +3\\
77-79 & +4\\
80 &    +5\\
81-85 & ritira + Capacità Speciale Armature / Scudi Tipo 2\\
86-90 & ritira + Capacità Speciale Armature / Scudi Tipo 1\\
91-100 &-2 Maledetta\\
\end{tabularx}

\medskip

Quando nel \textit{Bonus Magico} c'è scritto \textit{ritira + Capacità Speciale Armature/Scudi Tipo...} significa che devi ritirare il 1d100, ignorando altri risultato sopra 80 e tenere il bonus magico ottenuto, poi potrai tirare sulla \textit{Tabella Capacità Speciale Armature/Scudi Tipo...} risultante.

\textbf{Tabella: Capacità Speciale Armature/Scudi Tipo 1}\index{Tabella Capacità Speciale Armature/Scudi Tipo 1}

\begin{tabularx}{0.45\textwidth}{lX}
\textbf{1d100} & \textbf{Capacità Speciale Armature/Scudi Tipo 1}\\
1-5 & Ariete\\
6-10 &Bilanciata\\
11-15& Bracciali dell'Arciere\\
16-20& Bracciali della Difesa\\
21-25& Bracciali della Difesa Maggiore\\
26-30& Brillante\\
31-35& Determinazione\\
36-40& Difesa dagli Incantesimi\\
41-45& Elegante\\
46-50& Ospitale\\
51-55& Resistenza al Veleno\\
56-60& Resistenza all'Energia\\
61-65& Resistenza all'Energia Superiore\\
66-70& Selvatica\\
71-75& Scaglie di Drago\\
76-80& Scudo Animato\\
81-85& Scudo dell'Attrazione dei Proiettili\\
86-90& Soffio del Dragone\\
91-95& Tocco Fantasma\\
95-100 & Armatura/Scudo Maledetto\\
\end{tabularx}

\medskip

\textbf{Tabella: Capacità Speciale Armature/Scudi Tipo 2}\index{Tabella Capacità Speciale Armature/Scudi Tipo 2}

\begin{tabularx}{0.45\textwidth}{lX}
\textbf{1d100} & \textbf{Capacità Speciale Armature/Scudi Tipo 2}\\
1-5 &Accecante\\
6-10 &Adamantio\\
11-15& Amorfa\\
16-20& Antiemorragica\\
21-25& Attaccabrighe\\
26-30& Carico\\
31-35& Armatura Demoniaca\\
36-40& Denegante\\
41-45& Felpa\\
46-50& Forma Eterea\\
51-55& Invulnerabilità\\
56-60& Irrintracciabile\\
61-65 &Mascheramento\\
66-70 &Mithral\\
71-75 &Ombra\\
76-80 &Percettiva\\
81-85 &Titanica\\
86-90 &Vulnerabilità\\
91-100& Armatura/Scudo Maledetto\\
\end{tabularx}

\medskip

Quando viene indicata la capacità speciale Maledetta, dovete ritirare ed invertire i bonus magici dell'arma, quindi un'armatura o scudo +2 diviene un'armatura o scudo -2.

\subsubsection{Amuleti, Collane e Gioielli}\index{Tabella Generazione Amuleti, Collane e Gioielli}


\begin{tabular}{ll}
\textbf{Tipo Oggetto}&\textbf{1d8}\\
Amuleti, Collane e Gioielli Tipo 1&1-6\\
Amuleti, Collane e Gioielli Tipo 2&7-8\\
\end{tabular}

\medskip

\begin{tabularx}{0.45\textwidth}{lX}
\textbf{1d100} & \textbf{Amuleti, Collane e Gioielli Tipo 1}\\
1-8   & Amuleto Antiveleno\\
8-12  & Amuleto della Cancrena\\
12-18 & Amuleto Cicatrizzante\\
19-26 & Amuleto Contro la Possessione\\
27-34 & Amuleto della Localizzazione inevitabile\\
35    & Amuleto dei Piani\\
36-42 & Amuleto di Protezione dalla Individuazione e Localizzazione\\
42-46 & Amuleto della Resistenza Fisica\\
47-53 & Cerchietto dell'Esplosione\\
53-60 & Collana dell'Adattamento\\
61-70 & Collana dello Strangolamento\\
71-77 & Collana delle Palle di Fuoco\\
78-83 & Collana del Rosario\\
84-90 & Scarabeo della Morte\\
91-100 & Scarabeo di Protezione\\
\end{tabularx}

\medskip

\begin{tabularx}{0.45\textwidth}{lX}
\textbf{1d100} & \textbf{Amuleti, Collane e Gioielli Tipo 1}\\
1-7 & Gemma Elementale\\
8-13& Gemma della Luminosità\\
9-16& Gemma della Vista\\
17-26& Gioiello Attiramostri\\
27-33& Medaglione dei Pensieri\\
34-41& Medaglione della Caduta piuma\\
42-49& Perla della Saggezza\\
50-57& Spilla della Difesa\\
58-60& Talismano del Bene puro\\
61-62& Talismano del Male estremo\\
63-70& Talismano di Protezione dal Veleno\\
71-78& Talismano della Salute\\
79-85& Talismano della Sfera\\
86-100& Gioiello senza valore
\end{tabularx}


\begin{center}
\includegraphics[width=0.8\linewidth]{immagini/gauntlet.png}\\
\end{center}


\subsubsection{Cinture, Elmi, Stivali e Guanti}\index{Tabella Generazione Cinture, Elmi, Stivali e Guanti}

\begin{tabularx}{0.45\textwidth}{lX}
\textbf{1d100} & \textbf{Cinture, Elmi, Stivali e Guanti}\\
1-3 &Cintura dei Giganti\\
3-6 &Cintura dei Nani\\
6-11 &Elmo della Comprensione dei Linguaggi\\
12 &Elmo della Lucentezza\\
13-17 &Elmo del Movimento subacqueo\\
18-22 &Elmo della Telepatia\\
23-26 &Elmo del Teletrasporto\\
27-31 &Guanti Afferra Proiettili\\
31-35 &Guanti del Potere orchesco\\
36-41 &Guanti del Nuoto e della Scalata\\
41-46 &Guanti della Destrezza\\
47-52 &Guanti Maldestri\\
53-58 &Pantofole del Ragno\\
59-63 &Stivali Alati\\
64-66 &Stivali della Corsa e del Salto\\
67-77 &Stivali degli Elfi\\
78-83 &Stivali dell'Inverno\\
84-90 &Stivali della Levitazione\\
91-95 &Stivali della Velocità\\
96-100 &Stivali Danzanti\\
\end{tabularx}

\subsubsection{Bacchette, Bastoni e Verghe}\index{Tabella Generazione Bacchette, Bastoni e Verghe}

Tirare 1d8 per determinare se si trova una Bacchetta o Bastone o Verga.

\medskip

\begin{tabular}{ll}\\
\textbf{Tipo Oggetto}&\textbf{1d8}\\
Bacchette&1-4\\
Bastoni&5-7\\
Verghe&8\\
\end{tabular}

\medskip

\textbf{Tabella: Generazione Bacchette}\index{Tabella Generazione Bacchette}

\medskip

\begin{tabularx}{0.45\textwidth}{lX}
\textbf{1d100} & \textbf{Bacchetta}\\
1-5& Bacchetta Cerca metalli\\
6-10 &Bacchetta dei Dardi Incantati\\
11-15 &Bacchetta delle Comodità\\
16-20 &Bacchetta dei Fulmini\\
21-25& Bacchetta del Fuoco\\
26-30& Bacchetta del Ghiaccio\\
31-35& Bacchetta di Individ. del Magico\\
36-38& Bacchetta di Individ. dei Nemici\\
39-44& Bacchetta dell'Illusioni\\
45-48& Bacchetta dell'Individuazione delle porte segrete\\
46-50& Bacchetta della Luce\\
51 &Bacchetta del Mago da Guerra\\
52 &Bacchetta della Metamorfosi\\
53 &Bacchetta delle Meraviglie\\
54 &Bacchetta della Negazione\\
55-60& Bacchetta delle Palle di Fuoco\\
61-65 &Bacchetta della Paralisi\\
66-70& Bacchetta della Paura\\
71-75 &Bacchetta Scopri trappole\\
76-80& Bacchetta dei Segreti\\
81-85& Bacchetta della Ragnatela\\
86-90& Bacchetta del Vincolo\\
91-95& Bacchetta della Fuga Assistita\\
96-100&Bacchetta Maledetta\\
\end{tabularx}

\medskip

\textbf{Tabella: Generazione Bastoni}\index{Tabella Generazione Bastoni}

\medskip

\begin{tabularx}{0.45\textwidth}{lX}
\textbf{1d100} & \textbf{Bastone}\\
62 &Bastone dell'Arcimago\\
63-65& Bastone di Avvizzimento\\
66-67& Bastone dei Boschi\\
68-70& Bastone dello Charme\\
71-72& Bastone del Colpire\\
73-74& Bastone del Fuoco\\
75-76& Bastone del Gelo\\
77-78& Bastone di Guarigione\\
79-80& Bastone degli Insetti Sciamanti\\
81-82& Bastone del Pitone\\
83 &Bastone del Potere\\
84-86& Bastone dei Tuoni e Fulmini\\
87 &Bastone della Stregoneria\\

\end{tabularx}

\medskip

\textbf{Tabella: Generazione Verghe}\index{Tabella Generazione Verghe}

\medskip

\begin{tabularx}{0.45\textwidth}{lX}
\textbf{1d100} & \textbf{Verga}\\
1-10&Verga dell'Ammaliamento\\
11-20&Verga dell'Assorbimento\\
21-30&Verga Inamovibile\\
31-41&Verga del Colpo possente\\
42-50&Verga della Forza Sovrana\\
51-60&Verga della Prontezza\\
61-70&Verga della Sicurezza\\
71-80&Verga della Sovranità\\
81-90& Verga Tentacolare\\
91-100& Verga Maledetta\\
\end{tabularx}

\begin{center}
\includegraphics[width=0.8\linewidth]{immagini/cupdrinking.png}\\

\textit{Drinking cup depicting scenes from the Odyssey, Athens 550–525 B.C.}
\end{center}

\subsubsection{Pozioni, Filtri e Olii}\index{Tabella Generazione Pozioni, Filtri e Olii}

\begin{tabular}{ll}
\textbf{Pozione}&\textbf{1d8}\\
Pozione Tipo 1 &1-4\\
Pozione Tipo 2 &5-7\\
Pozione Tipo 3 &8\\
\end{tabular}

\medskip

\begin{tabular}{ll}
\textbf{1d100} & \textbf{Pozione Tipo 1}\\
1-8   &Pozione di Arrampicata\\
9-15  &Pozione di Crescita\\
16-23 &Pozione di Eroismo\\
24-29 &Pozione di Forma Gassosa\\
30-35 &Pozione di Forza dei Giganti\\
36-46 &Pozione di Guarigione\\
47-53 &Pozione dell'Inganno\\
54-64 &Pozione di Invisibilità\\
65-74 &Pozione della Levitazione\\
77-78 &Pozione di Resistenza\\
79-84 &Pozione di Respirare Sott'Acqua\\
84-90 &Pozione di Rimpicciolimento\\
91-95 &Pozione di Velocità\\
96-100 &Pozione di Volo\\
\end{tabular}

\medskip

\begin{tabular}{ll}
\textbf{1d100} & \textbf{Pozione Tipo 2}\\
1-10   &Pozione della Chiaraudienza animale\\
11-20  &Pozione della Chiaroveggenza animale\\
21-28  &Pozione di Controllo degli animali\\
29-33  &Pozione di Controllo dei draghi\\
34-38  &Pozione di Controllo dei non morti\\
39-49  &Pozione di Controllo delle persone\\
50-55  &Pozione di Controllo delle piante\\
56-66  &Pozione dell’invulnerabilità\\
67-77  &Pozione di Lettura del Pensiero\\
78-85  &Pozione di Veleno\\
86-95  &Pozione di Guarigione Maggiore\\
96-100 &Pozione di Veleno Maggiore\\
\end{tabular}

\medskip

\begin{tabular}{ll}
\textbf{1d100} & \textbf{Pozione Tipo 3}\\
1-13  & Filtro d'Amore\\
14-27 & Filtro Scopritesori\\
28-40 & Olio di Affilatezza\\
41-53 & Olio di Forma Eterea\\
54-66 & Olio di Scivolosità\\
67-79 & Pozione di Amicizia con gli Animali\\
80-85 & Pozione della Longevità\\
86-95 & Pozione della Metamorfosi\\
96-100& Pozione di Veleno Maggiore\\
\end{tabular}

\subsubsection{Anelli}\index{Tabella Generazione Anelli}

\begin{tabular}{ll}
\textbf{Anello}&\textbf{3d6}\\
Anello Tipo 1 &3-16\\
Anello Tipo 2 &17-18\\
\end{tabular}

\medskip

\begin{tabular}{ll}
\textbf{1d100} & \textbf{Anelli Tipo 1}\\
1-5   & Anello Accumula Incantesimi\\
6-13  & Anello dell'Ariete\\
14-21 & Anello di caduta piuma\\
22-28 & Anello di Camminare sull'Acqua\\
29-35 & Anello del Calore\\
36-41 & Anello della Debolezza\\
42-47 & Anello di Elusione\\
48-50 & Anello di Influenza sugli Animali\\
51-55 & Anello dell’Inganno\\
56-61 & Anello di Libertà di Azione\\
61-67 & Anello del Nuoto\\
68-77 & Anello di Protezione\\
76-84 & Anello di Resistenza\\
85-93 & Anello del Salto\\
93-100 & Anello di Telecinesi\\
\end{tabular}

\medskip

\begin{center}
\includegraphics[width=0.8\linewidth]{immagini/romanring.png}
\end{center}

\begin{tabular}{ll}
\textbf{1d100} & \textbf{Anelli Tipo 2}\\
1-8 &  Anello del Controllo delle persone\\
9-17&  Anello del Controllo delle piante\\
18-23& Anello degli Elementali dell'Acqua\\
24-29& Anello degli Elementali dell'Aria\\
31-36& Anello degli Elementali del Fuoco\\
37-42& Anello degli Elementali della Terra\\
43-48& Anello dell'Evocazione dello Djinni\\
49-56& Anello Respingi Incantesimi\\
57-65& Anello di Invisibilità\\
66-75& Anello di Rigenerazione\\
76-83& Anello dello Scudo Mentale\\
84-90& Anello delle Stelle Cadenti\\
91-92& Anello dei Tre Desideri\\
92-96& Anello dei Tre Desideri esaurito\\
97-100 & Anello della Vista ai Raggi X\\

\end{tabular}


\subsubsection{Cappelli, Mantelli, Occhiali, Tuniche}\index{Tabella Generazione Cappelli, Mantelli, Occhiali, Tuniche}

\begin{tabularx}{0.45\textwidth}{lX}
\textbf{1d100} & \textbf{Cappelli, Mantelli, Occhiali, Tuniche}\\

1-3 &Bandana dell'Intelligenza\\
4-10 &Cappello del Camuffamento\\
11-17& Mantello dell'Aracnide\\
18-23& Mantella del Ciarlatano\\
24-29& Mantello di Distorsione\\
30-40& Mantello degli Elfi\\
41-45& Mantello della Manta\\
46-50& Mantello del Pipistrello\\
51-57& Mantello di Protezione\\
58-62& Mantello della Resistenza agli Incantesimi\\
63-68& Mantello della velenosità\\
69-72& Occhi della pietrificazione\\
73-75& Occhi Affascinanti\\
76-77& Occhi dell'Aquila\\
78-80& Occhi della Vista Dettagliata\\
80-82& Occhiali da Notte\\
83-86 &Tunica del Mimetismo\\
87 &Tunica dell’Arcimago\\
88 &Tunica dei Colori Scintillanti\\
89-91& Tunica dell’Indebolimento\\
92-94 &Tunica degli Occhi\\
95-99 &Tunica degli Oggetti Utili\\
100 &Tunica delle Stelle\\
\end{tabularx}


\subsubsection{Manuali e Tomi}\index{Tabella Generazione Manuali e Tomi}

\begin{tabularx}{0.45\textwidth}{lX}
\textbf{1d100} & \textbf{Manuali e Tomi}\\
1-9 & Manuale dei Golem\\
10-24 & Manuale della Buona salute\\
25-40 &Manuale della Velocità di azione\\
40-54 &Manuale dell'Esercizio fisico\\
55-69 &Tomo dell'Autorità e dell'Influenza\\
70-84 &Tomo della Comprensione\\
85-100& Tomo del Pensiero Limpido\\
\end{tabularx}

\subsubsection{Oggetti Magici vari}\index{Tabella Generazione Oggetti Magici vari}

Tirare 1d10 per determinare se si trova un oggetto magico raro o leggendario oppure dalle liste degli oggetti magici vari

\medskip

\begin{tabular}{ll}
\textbf{Tipo Oggetto} & \textbf{1d12}\\
Oggetti Magici Vari 1&1-3\\
Oggetti Magici Vari 2&4-5\\
Oggetti Magici Vari 3&6-7\\
Oggetti Magici Vari 4&8-9\\
Oggetti Magici Vari 5&10-12\\
Rari e Leggendari&10\\
\end{tabular}

\medskip


\subsubsection{Rari e Leggendari}\index{Tabella Generazione Rari e Leggendari}

\medskip

\begin{tabularx}{0.45\textwidth}{lX}
\textbf{1d100} & \textbf{Oggetto Magico}\\
1-3 &Ali del Volo\\
4-6 &Ampolla di Ferro\\
7-10 &Anfora elementale dell’acqua\\
11-12& Apparato del Granchio\\
13-15& Barca Pieghevole\\
17-20& Borsa Conservante Tipo III\\
21-24& Borsa Conservante Tipo IV\\
25-28& Borsa dei Fagioli\\
29-30& Bottiglia dell'Efreeti\\
31 &Brocca delle Pozioni\\
32-33& Candela di Invocazione\\
34-35 &Corno del Valhalla\\
36-39 &Filatterio della giovinezza\\
40-42 &Fortezza Istantanea\\
43-45 &Mazzo delle Meraviglie\\
46-49 &Miniatura dal Potere Meraviglioso\\
50-53 &Munizione dell'Uccisione\\
54-58 &Palla di Cristallo\\
59-62 &Pergamena contro gli elementali\\
63-65 &Pergamena contro i non morti\\
66-70 &Piffero delle Fogne\\
71-75 &Pigmenti delle Meraviglie\\
76-83 &Portale Cubico\\
84-85 &Pozzo dei Molti Mondi\\
86-87 &Specchio dell’Abilità mentale\\
88-89 &Specchio Intrappola Vita\\
90-91 &Sfera dell'Annientamento\\
92-94 &Turibolo Elementale dell’aria\\
95-96 &Vano Portatile\\
97-98 &Zoccoli della Velocità\\
99-100 &Zoccoli dello Zefiro\\
\end{tabularx}

\medskip

\subsubsection{Oggetti magici vari 1}\index{Tabella Generazione Oggetti magici vari 1}

\begin{tabularx}{0.45\textwidth}{lX}
\textbf{1d100} & \textbf{Oggetti magici vari 1}\\
1-8& Acqua purificatrice\\
9-17&Battaglio dell'Apertura\\
18-27&Borsa Conservante Tipo I\\
28-34& Corda da Arrampicata\\
35-43&Faretra Efficiente\\
44-52&Freccia localizzante\\
53-60&Lanterna della Rivelazione\\
61-70&Perla del Potere\\
71-80&Pietra della Buona Sorte\\
81-83&Solvente Universale\\
84-94&Unguento Ristorativo\\
95-100&Zainetto Pratico\\
\end{tabularx}

\subsubsection{Oggetti magici vari 2}\index{Tabella Generazione Oggetti magici vari 2}


\begin{tabularx}{0.45\textwidth}{lX}
\textbf{1d100} & \textbf{Oggetti magici vari 2}\\
1-8 &Braciere degli Elementali del Fuoco\\
9-17 &Braciere del Sonno maledetto\\
18-27& Cubo di protezione dal freddo\\
28-34& Incensiere degli Elementali dell'Aria\\
35-43& Rete Intralciante\\
44-52& Rete Intrappolante\\
53-60& Scopa dell’Attacco animato\\
61-70& Scopa Volante\\
71-80& Scopa del Volo maledetto\\
81-88& Specchio della Duplicazione\\
89-90& Tappeto Volante\\
99-100& Zappa dei Titani\\
\end{tabularx}

\medskip
\subsubsection{Oggetti magici vari 3}\index{Tabella Generazione Oggetti magici vari 3}

\begin{tabularx}{0.45\textwidth}{lX}
\textbf{1d100} & \textbf{Oggetti magici vari 3}\\
1-8 &Borsa dell'Annullamento\\
9-18 &Brocca dell'Acqua Infinita\\
19-26& Ceppi Dimensionali\\
27-35& Colla Suprema\\
36-40& Incenso della meditazione\\
41-51& Pergamena protettiva contro la magia\\
52-60& Polvere Rivelatrice\\
61-70& Polvere della Sparizione\\
71-82& Polvere dello Starnuto\\
83-90& Pietra Arcana\\
91-96& Pietra del Peso\\
97-100& Ventaglio Arcano\\
\end{tabularx}

\begin{center}
\includegraphics[width=0.8\linewidth]{immagini/ancientdrum.png}
\end{center}


\subsubsection{Oggetti magici vari 4}\index{Tabella Generazione Oggetti magici vari 4}

\begin{tabularx}{0.45\textwidth}{lX}
\textbf{1d100} & \textbf{Oggetti magici vari 4}\\
1-5 &Ampolla delle maledizioni\\
6-10 &Battaglio del Cannibalismo\\
11-16& Borsa Conservante Tipo II\\
17-20& Borsa Divorante\\
21-25& Bottiglia Fumante\\
26-31& Buco Portatile\\
32-37& Collana dell’Aria Salubre\\
38-43& Corda dell'Intralciamento\\
44-48& Corda Strozzatrice\\
49-50& Corno di Distruzione\\
51-52& Cubo di Forza\\
53-58& Fasce di Ferro del Vincolo\\
59-64& Filatterio contro i non moti\\
65-69& Incenso dell’Ossessione\\
70-71& Mazzo delle Illusioni\\
72-76& Palla di Cristallo ipnotica\\
77-82& Pergamena contro i licantropi\\
83-84& Pietra degli Elementali della Terra\\
85-89& Piffero dello Spavento\\
90-92& Piuma Arcana\\
93-94& Polvere dell'Aridità\\
95-96& Tamburi del Panico\\
97-98& Tamburi dello Stordimento\\
99-100& Turibolo dell’Evocazione maledetta\\
\end{tabularx}


\begin{center}
\includegraphics[width=0.6\linewidth]{immagini/ancientbraziers2.png}

\textit{Teotihuacano Old God vessels: Top - stone brazier in Natural History Museum of Los Angeles County}
\end{center}




\pagebreak

\section{Descrizione degli Oggetti Magici}\index{Descrizione degli Oggetti Magici}


Gli oggetti magici sono presentati in ordine alfabetico per categorie di raggruppamento. La descrizione di un oggetto magico fornisce il nome dell'oggetto, la sua categoria, rarità e le sue proprietà magiche.

Benché i costi siano riportati è sempre bene concedere gli oggetti magici come premi, tesoro, a seguito di missione.

In linea di massima un oggetto Comune, l'unico che potrebbe trovarsi facilmente in una grande città, puoi costare dai 50 ai 100 mo, uno Non Comune tra i 150 ed i 500 mo, uno Raro tra i 500 e i 5000 mo, uno Molto Raro fino a 50000 mo.

Oggetti con un bonus oltre il +2, o Leggendari, non si comprano mai, deve essere un epica avventura a farli trovare.



\medskip

Anche gli incantesimi sono oggetti magici e come tali, se il Narratore permette, possono essere acquistati (orrore! non c'è nulla di più bello trovare un nuovo incantesimo tra i tesori di un'avventura).

Un incantesimo costa in monete d'oro livello*livello*livello*80

\bigskip

\subsection{Capacità Speciali delle Armi Magiche}

\lettrine[lines=2, lhang=0.33, loversize=0.25, findent=1.5em]{U}{n'} arma con una capacità speciale deve avere almeno bonus magico +1.

Vengono qui elencati le capacità magiche che un'armatura, scudo o arma può avere oltre al generico bonus magico (+1,+2....). Usate questo elenco come linee guida ed esempi, stessa cosa per i prezzi, usateli come indicazione di rarità.

\index[OggettiMagici]{Armi Magiche!Armi Magiche}\subsubsection*{Arma magica}

\textit{Arma (qualsiasi)} +1 1800 mo, +2 6000 mo, +3 17000 mo, +4 45000 mo, +5 80000 mo

Hai un bonus ai Tiri per Colpire e ai tiri di danno effettuati con quest'arma. Il bonus è determinato dalla rarità dell'arma. Alcune armi magiche possiedono delle ulteriori proprietà, come l'emettere luce.

\subsubsection*{Accumula Incantesimi}\index[OggettiMagici]{Armi Magiche!Accumula Incantesimi}

Un'arma Accumula Incantesimi permette a un incantatore di immagazzinare un incantesimo con bersaglio fino livello 3 nell'arma. L'incantesimo deve avere un tempo di lancio standard di 2 Azioni. Ogni volta che l'arma colpisce una creatura e quest'ultima subisce dei danni, chi impugna l'arma con una azione immediata può liberare l'incantesimo.

Una volta che l'incantesimo viene lanciato, un incantatore può immagazzinarvi all'interno qualsiasi altro incantesimo con bersaglio, sempre fino a livello 3.

L'arma rivela magicamente a chi la impugna il nome dell'incantesimo attualmente contenuto. Un'arma Accumula Incantesimi creata casualmente ha una probabilità del 50\% di avere già un incantesimo contenuto al suo interno. Questa capacità speciale può essere aggiunta solo ad armi da mischia.

Un'arma Accumula Incantesimi emette una forte aura della scuola Invocazione, più l'aura dell'incantesimo contenuto.

\textbf{Dettagli}: Aura Invocazione forte e variabile; Requisiti di Creare Oggetti Magici Superiori, Costo +3000 mo.

\subsubsection*{Adattiva}\index[OggettiMagici]{Armi Magiche!Adattiva}

Questa capacità può essere aggiunta solo agli archi compositi. Un arco Adattivo reagisce alla forza di chi lo impugna, agendo come un arco con un bonus di Forza pari a quello di chi lo sta impugnando. Chi lo impugna può scoccare con un bonus di Forza inferiore (e causare meno danni) se lo desidera.

\textbf{Dettagli}: Aura Trasmutazione debole; Requisiti di Creare Oggetti Magici, Lista Animali e Piante; Costo +1500 mo.

\subsubsection*{Affilata}\index[OggettiMagici]{Armi Magiche!Affilata}

Questa capacità in caso di critico consente di contare il numero di 6 tirati aumentandolo di 1. Solo le armi da mischia taglienti o perforanti possono essere affilate.

\textbf{Dettagli}: Aura Trasmutazione moderata; Requisiti di Creare Oggetti Magici Superiori, Lista della Terra; Costo +5000 mo.

\index[OggettiMagici]{Armi Magiche!Ammazza Draghi}\subsubsection*{Ammazza Draghi}

Quando colpisci un drago con quest'arma, il drago subisce 3d6 danni aggiuntivi del tipo dell'arma. Ai fini di quest'arma, "drago" è qualsiasi creatura del tipo drago.

\textbf{Dettagli}: Aura Invocazione moderata; Requisiti di Creare Oggetti Magici Superiori; Costo +8000 mo.

\index[OggettiMagici]{Armi Magiche!Ammazza Giganti}\subsubsection*{Ammazza Giganti}

Quando colpisci un gigante con quest'arma, il gigante subisce 2d6 danni aggiuntivi del tipo dell'arma e deve superare un Tiro Salvezza su Tempra con DC 18 o cadere prono. Ai fini di quest'arma, "gigante" qualsiasi creatura del tipo gigante.

\textbf{Dettagli}: Aura Invocazione moderata; Requisiti di Creare Oggetti Magici Superiori; Costo +8000 mo.

\subsubsection*{Distruttrice dei Giganti}\index[OggettiMagici]{Armi Magiche!Distruttrice dei Giganti}

Devi indossare una \textit{cintura dei giganti} (qualsiasi varietà) e i \textit{guanti del potere orchesco} per poter usare quest'arma.

Mentre usi il martello il tuo punteggio di Forza aumenta di 2 (fino ad un massimo di 7).

Quando ottieni un critico sul Tiro per Colpire effettuato con quest'arma contro un gigante, il gigante deve superare un tiro Salvezza su Tempra con DC 21 o morire.

Puoi spendere 1 carica ed effettuare un attacco con arma a distanza scagliandolo come se avesse gittata di 6 metri. Se l'attacco colpisce, il martello produce un tuono udibile fino a 90 metri di distanza. Il bersaglio e tutte le creature entro 9 metri da esso devono superare un tiro Salvezza su Tempra con DC 21 o restare stordite fino al termine del tuo prossimo round.

Il martello ha 5 cariche, e recupera 1 carica spesa ogni giorno all'alba.

\subsubsection*{Anatema}\index[OggettiMagici]{Armi Magiche!Anatema}

Un'arma Anatema eccelle nell'attaccare certe creature. Contro il nemico prescelto, il suo bonus effettivo diventa di +2. L'arma, inoltre, infligge +2d6 danni addizionali contro tale nemico. Per determinare casualmente il nemico prescelto dell'arma si usa la tabella seguente:

\medskip

\begin{tabular}{ll}
d\% &Nemico prescelto\\
01-05 &Aberrazioni\\
06-09 &Bestie\\
10-16 &Costrutti\\
17-22 &Draghi\\
23-27 &Fatati\\
28-60 &Umanoidi (scegliere sottotipo)\\
61-70 &Creature Magiche\\
71-72 &Melme\\
73-88 &Immondi\\
89-90 &Piante\\
91-98 &Non Morti\\
99-100 &Insetti\\
\end{tabular}

\medskip

\textbf{Dettagli}: Aura Evocazione moderata; Requisiti di Creare Oggetti Magici Superiori, Lista Abiurazione; Costo +3000 mo.

\subsubsection*{Cacciatore}\index[OggettiMagici]{Armi Magiche!Cacciatore}

Un'arma del Cacciatore aiuta chi la impugna a localizzare e a catturare la preda. Quando l'arma viene tenuta in mano, chi la impugna ottiene il bonus dell'arma alle prove di Sopravvivenza effettuate per seguire le tracce di qualsiasi creatura che l'arma ha danneggiato nel corso del giorno precedente. Infligge +1d6 danni alle creature di cui sono state seguite le tracce con Sopravvivenza da chi la impugna nel corso del giorno precedente.

\textbf{Dettagli}: Aura Divinazione moderata; Requisiti di Creare Oggetti Magici Superiori, Localizza Animali e Piante; Costo +3000 mo.

\subsubsection*{Conduttiva}\index[OggettiMagici]{Armi Magiche!Conduttiva}

Un'arma Conduttiva è in grado di incanalare l'energia di una Abilità magica che richieda un attacco di contatto in mischia o a distanza per colpire il suo bersaglio.

Quando chi la impugna effettua con successo un attacco del tipo appropriato, può scegliere di spendere due utilizzi della sua capacità magica per incanalarla attraverso l'arma, al fine di colpire l'avversario, che subisce gli effetti dell'attacco dell'arma e quelli della capacità speciale (come incanalare energia, imposizione delle mani...).

Questa capacità speciale dell'arma può essere utilizzata solamente una volta per round (anche se possiede più armi Conduttive).

\textbf{Dettagli}: Aura Necromanzia moderata; Requisiti di Creare Oggetti Magici Superiori, Mano Magica; Costo +3000 mo.

\subsubsection*{Coraggiosa}\index[OggettiMagici]{Armi Magiche!Coraggiosa}

Questa capacità speciale può essere aggiunta solo a un'arma da mischia. Un'arma Coraggiosa fortifica il coraggio e il morale in battaglia di chi la indossa. Chi la impugna ottiene un Bonus ai Tiri Salvezza contro Paura pari al bonus dell'arma.

\textbf{Dettagli}: Aura Ammaliamento debole; Requisiti di Creare Oggetti Magici, Eroismo, Paura; Costo +3000 mo.

\subsubsection*{Corrosiva}\index[OggettiMagici]{Armi Magiche!Corrosiva}

A comando, un'arma Corrosiva si ricopre di uno strato di acido che infligge 1d6 danni aggiuntivi da acido quando colpisce il bersaglio. L'acido non danneggia chi la impugna. L'effetto permane fino a quando non viene impartito un nuovo comando.

\textbf{Dettagli}: Aura Invocazione moderata; Requisiti di Creare Oggetti Magici, Freccia Acida; Costo +3000 mo.

\subsubsection*{Crudele}\index[OggettiMagici]{Armi Magiche!Crudele}

Un'arma Crudele si alimenta di paura e sofferenza. Quando chi la impugna colpisce una creatura Spaventata con un'arma crudele, quest'ultima diventa Inferma per 1 round. Quando chi la impugna usa l'arma per rendere Priva di Sensi o uccidere una creatura, ottiene 5 Punti Ferita Temporanei che durano per 10 minuti.

\textbf{Dettagli}: Aura Necromanzia debole; Requisiti di Creare Oggetti Magici, Paura, Costo +3000 mo.

\subsubsection*{Danzante}\index[OggettiMagici]{Armi Magiche!Danzante}

Come azione standard, un'arma Danzante può essere lasciata libera in modo che combatta da sola. L'arma combatte per 4 round usando il Difesa di colui che l'ha lasciata libera e poi cade a terra.

Rimane sempre accanto alla persona che l'ha liberata, anche se si sposta con mezzi fisici o magici. Se colui che l'ha lasciata libera ha una mano libera può riprendere l'arma che sta attaccando da sola, come azione immediata, ma una volta ripresa, la spada non potrà più danzare (attaccare da sola) prima di 4 round.

Questa capacità può essere aggiunta solo ad armi da mischia.

\textbf{Dettagli}: Aura Trasmutazione forte; Requisiti di Creare Oggetti Magici Superiori, Animare Oggetti; Costo +25000 mo.

\subsubsection*{Designante}\index[OggettiMagici]{Armi Magiche!Designante}

Questa capacità speciale può essere aggiunta solo ad armi a distanza o munizioni. Ogni volta che un'arma a distanza o una munizione con questa capacità colpisce una creatura, designa magicamente il bersaglio. Tutti gli alleati ottengono Bonus +2 ai Tiri per Colpire per 1 round. Più colpi andati a segno sullo stesso bersaglio non incrementano i bonus o la loro durata.

\textbf{Dettagli}: Aura Ammaliamento moderato; Requisiti di Creare Oggetti Magici Superiori, Luce; Costo +6000 mo.

\subsubsection*{Difensiva}\index[OggettiMagici]{Armi Magiche!Difensiva}

Un'arma Difensiva permette a chi la impugna di trasferire una parte o tutto il bonus dell'arma alla propria Difesa come un bonus cumulabile con eventuali altri bonus. Come azione immediata, chi la impugna può scegliere come disporre del bonus dell'arma all'inizio del round, prima di usarla, e il bonus alla Difesa dura fino al turno successivo.

\textbf{Dettagli}: Aura Abiurazione moderata; Requisiti di Creare Oggetti Magici Superiori, Scudo; Costo +3000 mo.

\subsubsection*{Distanza}\index[OggettiMagici]{Armi Magiche!Distanza}

Questa capacità speciale può essere aggiunta solo a proiettili. Un proiettile della Distanza ha il doppio della gittata data dall'arma che lancia.

\textbf{Dettagli}: Aura Divinazione moderata; Requisiti di Creare Oggetti Magici, Chiaroveggenza; Costo +3000 mo.

\subsubsection*{Distruzione}\index[OggettiMagici]{Armi Magiche!Distruzione}

Un'arma della Distruzione è la rovina di tutti i Non Morti. Ogni creatura Non Morta colpita in combattimento deve superare un Tiro Salvezza su Volontà con DC 14 o viene distrutta o subire 2d8 di danni aggiuntivi da Luce. Un'arma della Distruzione deve essere un'arma da mischia da botta.

\textbf{Dettagli}: Aura Evocazione forte; Requisiti di Creare Oggetti Magici Superiori, Guarigione; Costo +6000 mo.

\subsubsection*{Duello}\index[OggettiMagici]{Armi Magiche!Duello}

Questa capacità può essere conferita solo ad un'arma da mischia. Un'arma del Duello (che deve essere un'arma che può essere utilizzata con il talento Arma Accurata) garantisce a chi la impugna bonus +1d6 alle prove di Iniziativa, purché l'arma sia stata estratta e impugnata quando viene effettuata la prova di Iniziativa.

\textbf{Dettagli}: Aura Trasmutazione debole; Requisiti di Creare Oggetti Magici, Lista Animali e Piante; Costo +7000 mo.

\subsubsection*{Energia Luminosa}\index[OggettiMagici]{Armi Magiche!Energia Luminosa}

Quest'oggetto sembra l'impugnatura di una spada lunga, ma senza lama. Quando ne afferri l'impugnatura, puoi usare due azioni per far sì che una lama di pura luminescenza si formi, o faccia sparire la lama inserita nell'impugnatura.

Finché la spada esiste, questa spada lunga magica ha la proprietà Versatile. Se sei competente con le spade corte o le spade lunghe, sei competente anche con la lama del sole.

Ottieni un bonus di +2 ai Tiri per Colpire e danno effettuati con quest'arma, che infligge danni da Luce anziché danni taglienti. Quando colpisci con essa una creatura non morta, il bersaglio subisce 1d8 danni da Luce aggiuntivi.

La lama luminosa della spada emette luce intensa in un raggio di 4,5 metri e luce fioca per ulteriori 4,5 metri. La luce è luce solare. Finché la lama è attiva, puoi usare due azioni per espandere o ridurre il raggio della luce intensa e fioca di 1,5 metri ciascuno, fino a un massimo di 9 metri o un minimo di 3 metri ciascuno.

\textbf{Dettagli}: Aura Trasmutazione forte; Requisiti di Creare Oggetti Magici Meravigliosi, Fiamma Perenne, Esplosione Solare; Costo +45000 mo.

\subsubsection*{Estingui Fuoco}\index[OggettiMagici]{Armi Magiche!Estingui Fuoco}

Questa capacità speciale può essere aggiunta solo ad armi da mischia. Un'arma Estingui Fuoco è in grado di estinguere un fuoco non magico di taglia Media o inferiore. Quando usata contro una creatura del Fuoco, infligge 1d6 danni addizionali. Chi impugna un'arma Estingui Fuoco ottiene Bonus di Competenza +2 ai Tiri Salvezza contro gli effetti basati sul fuoco e l'arma stessa è immune ai danni da fuoco.

\textbf{Dettagli}: Aura Trasmutazione debole; Requisiti di Creare Oggetti Magici, Lista dell'Acqua; Costo +3000 mo.

\subsubsection*{Fanatizzante}\index[OggettiMagici]{Armi Magiche!Fanatizzante}

Arma Maledetta. Questa capacità conferisce un bonus +2 agli attacchi, tuttavia, all’inizio della battaglia, fa sì che il portatore venga preso da una collera incontenibile. Il personaggio attaccherà la creatura più vicina, nemica o amica che sia, finché non ne resterà nessuna in vita entro 18 m.

\subsubsection*{Ferimento}\index[OggettiMagici]{Armi Magiche!Ferimento}

Questa capacità può essere aggiunta solo ad armi da mischia. Un'arma da Ferimento infligge 1 danno da Sanguinamento quando colpisce una creatura. Danni multipli di quest'arma aumentano il danno da Sanguinamento fino ad un massimo di 10.
Le creature sanguinanti subiscono il danno da Sanguinamento all'inizio del loro round.

Le creature immuni ai Colpi Critici sono immuni ai danni da Sanguinamento inflitti da quest'arma.

\textbf{Dettagli}: Aura Necromantica moderata; Requisiti di Creare Oggetti Magici Superiori, Contagio; Costo +6000 mo.

\subsubsection*{Folgorante}\index[OggettiMagici]{Armi Magiche!Folgorante}

A comando, un'arma Folgorante viene avvolta da elettricità crepitante che infligge 1d6 danni addizionali da elettricità per ogni colpo andato a segno. Quest'elettricità non danneggia chi impugna l'arma. L'effetto rimane sempre attivo finché l'arma è sguainata.

\textbf{Dettagli}: Aura Invocazione moderata; Requisiti di Creare Oggetti Magici Superiori, Fulmine; Costo +3000 mo.

\subsubsection*{Furia Innata}\index[OggettiMagici]{Armi Magiche!Furia Innata}

Questa capacità speciale può essere aggiunta solo ad armi da mischia. Un'arma della Furia Innata trae potere dalla rabbia e dalla frustrazione che prova chi la impugna quando combatte dei nemici che si rifiutano di morire. Ogni volta che chi la impugna infligge danni ad un avversario con l'arma, il suo bonus aumenta di +1 quando effettua attacchi contro quel nemico (fino a un bonus massimo totale di +5). Questo bonus aggiuntivo svanisce se l'avversario muore, o se chi impugna l'arma la usa per attaccare una creatura diversa, manca con il tiro per colpire o passa 1 ora.

\textbf{Dettagli}: Aura Ammaliamento moderato; Requisiti di Creare Oggetti Magici Superiori, Eroismo; Costo +4000 mo.

\subsubsection*{Fortunata}\index[OggettiMagici]{Armi Magiche!Fortunata}

Finché hai addosso la spada ricevi anche un bonus di +1 ai Tiri Salvezza.

- \textit{Fortuna}. Se hai addosso la spada, puoi affidarti alla sua fortuna (non richiede azioni) per ripetere un Tiro per Colpire, prova di caratteristica o Tiro Salvezza il cui risultato non ti soddisfa. Sei obbligato a usare il secondo risultato del dado. Questa proprietà non può essere usata di nuovo fino alla prossima alba.

- \textit{Desiderio}. Mentre la impugni, puoi usare due azioni per spendere 1 carica e lanciare tramite essa l'incantesimo desiderio. Questa proprietà non può essere usata di nuovo fino alla prossima alba. La spada ha 1d4-1 cariche, e perde questa proprietà se finisce le cariche.

\textbf{Dettagli}: Aura Invocazione molto forte; Requisiti di Creare Oggetti Magici Mitici, Desiderio; Costo +30000 mo.

\subsubsection*{Gelida}\index[OggettiMagici]{Armi Magiche!Gelida}

A comando, un'arma Gelida viene avvolta da un gelo terribile che infligge 1d6 danni da freddo per ogni colpo andato a segno. Questo freddo non danneggia chi impugna l'arma. L'effetto rimane sempre attivo finché l'arma è sguainata.

\textbf{Dettagli}: Aura Invocazione moderata; Requisiti di Creare Oggetti Magici Superiori, Lista dell'Acqua; Costo +3000 mo.

\subsubsection*{Gloriosa}\index[OggettiMagici]{Armi Magiche!Gloriosa}

Un'arma Gloriosa illumina con una luce abbagliante pari a quella un incantesimo Luce Diurna quando viene estratta. Chi la impugna non può sopprimere questa luce, anche se può essere soppressa temporaneamente da qualsiasi effetto che può sopprimere Luce Diurna.

Quando un'arma Gloriosa effettua un Colpo Critico, il bersaglio è Accecato fino all'inizio del round successivo del possessore (Tempra DC 14 nega). Solo un'arma da mischia può avere la capacità Gloriosa.

\textbf{Dettagli}: Aura Invocazione moderata; Requisiti di Creare Oggetti Magici, Cecità/Sordità, Luce Diurna; Costo +6000 mo.

\subsubsection*{Guardiana}\index[OggettiMagici]{Armi Magiche!Guardiana}

Questa capacità può essere aggiunta solo ad armi da mischia. Un'arma Guardiana permette a chi la impugna di trasferire una parte o tutto il bonus dell'arma ai suoi Tiri Salvezza come bonus che si cumula con tutti gli altri. Come azione immediata, chi impugna l'arma sceglie come distribuire il bonus dell'arma all'inizio del suo round prima di usare l'arma. Il bonus a tutti i Tiri Salvezza dura fino al suo round successivo. Solo il bonus proprio dell'arma può essere sacrificato, non si può usare nessun altro bonus derivante da altri effetti.

Se un'arma ha sia la capacità Difensivo che Guardiana, sacrificare un singolo punto del bonus migliora la Difesa o i Tiri Salvezza, ma non entrambi.

\textbf{Dettagli}: Aura Abiurazione moderata; Requisiti di Creare Oggetti Magici Superiori, Resistenza; Costo +3000 mo.

\subsubsection*{Immorale}\index[OggettiMagici]{Armi Magiche!Immorale}

Questa capacità può essere aggiunta solo ad armi da mischia. Quando un'arma Immorale colpisce un avversario, produce un lampo di Vuoto che riecheggia tra chi la impugna e il suo bersaglio. L'energia infligge 2d6 danni addizionali all'avversario e 1d6 danni a chi la impugna.

\textbf{Dettagli}: Aura Invocazione moderata; Requisiti di Creare Oggetti Magici Superiori, Debilitazione; Costo +3000 mo.

\subsubsection*{Impulso Vitale}\index[OggettiMagici]{Armi Magiche!Impulso Vitale}

Questa capacità speciale può essere aggiunta solo ad armi da mischia. Un'arma dell'Impulso Vitale aumenta e sostiene l'energia vitale di chi la impugna, mentre è nel mezzo del combattimento. Chi la impugna guadagna un bonus ai Tiri Salvezza contro gli effetti di Necromanzia (inclusi danni alle caratteristiche, risucchi di caratteristica e riduzioni Punti Ferita massimi dovuti ai poteri dei Non Morti) pari al bonus dell'arma. Inoltre, ogni volta che chi la impugna ottiene Punti Ferita Temporanei da qualsiasi fonte, vi aggiunge il bonus dell'arma.

\textbf{Dettagli}: Aura Evocazione moderata; Requisiti di Creare Oggetti Magici Superiori, Cura Ferite Gravi, Ristorare superiore; Costo +6000 mo.

\subsubsection*{Infuocata}\index[OggettiMagici]{Armi Magiche!Infuocata}

A comando, un'arma Infuocata viene avvolta da fiamme che infliggono 1d6 danni da fuoco per ogni colpo andato a segno. Questo fuoco non danneggia chi impugna l'arma. L'effetto rimane attivo finché non viene disattivato con un altro comando.

\textbf{Dettagli}: Aura Invocazione moderata; Requisiti di Creare Oggetti Magici Superiori, Palla di Fuoco; Costo +3000 mo.

\index[OggettiMagici]{Armi Magiche!Ladra delle Nove Vite}\subsubsection*{Ladra delle Nove Vite}

Ottieni un bonus di +2 ai Tiri per Colpire e danno effettuati con quest'arma magica. Se ottieni un colpo critico contro una creatura che ha meno di 100 Punti Ferita, questa deve superare un tiro Salvezza su Tempra con DC 17 o venire immediatamente uccisa, mentre la spada ne risucchia la forza vitale dal corpo (i costrutti e i non morti sono immuni a questa proprietà).

La spada ha 1d8 + 1 cariche, e perde 1 carica quando una creatura viene uccisa. Quando la spada non ha più cariche, perde questa proprietà.

\textbf{Dettagli}: Aura Necromantica forte; Requisiti di Creare Oggetti Magici Superiori, Palla di Fuoco; Costo +25000 mo.

\subsubsection*{Letale}\index[OggettiMagici]{Armi Magiche!Letale}

Questa capacità speciale può essere aggiunta solo ad armi da mischia che normalmente infliggono Danni Non Letali, da stordimento Tutti i danni di un'arma Letale sono normali (letali). A comando, azione immediata, l'arma sopprime questa capacità finché chi la impugna non gli ordina di riattivarla.

\textbf{Dettagli}: Aura Necromanzia debole; Requisiti di Creare Oggetti Magici Superiori, Cura Ferite Leggere (invertito); Costo +3000 mo.



\subsubsection*{Marina}\index[OggettiMagici]{Armi Magiche!Marina}

Questa capacità speciale può essere aggiunta solo ad armi da mischia. Un'arma Marina funziona tranquillamente negli ambienti acquatici. Con l'arma in mano, chi la impugna ottiene un bonus alle prove di Nuotare pari al doppio del bonus dell'arma.

Inoltre, chi la impugna non subisce le normali penalità ai Tiri per Colpire e per i danni dovuti al trovarsi sott'acqua, come se fosse soggetto a un incantesimo Libertà di Movimento.

\textbf{Dettagli}: Aura Necromanzia moderata; Requisiti di Creare Oggetti Magici Superiori, Libertà di Movimento,Costo +3000 mo.

\subsubsection*{Mascheramento}\index[OggettiMagici]{Armi Magiche!Mascheramento}

A un'arma del Mascheramento può essere comandato di mutare la sua forma e apparire come un altro oggetto di taglia simile. L'arma conserva tutte le sue proprietà (compreso il peso) anche quando è mascherata, ma non irradia magia. Solo Visione del Vero o altre magie simili rivelano la reale natura dell'arma trasformata. Dopo che un'arma del Mascheramento è stata usata per attaccare, questa capacità speciale viene soppressa per 1 minuto.

\textbf{Dettagli}: Aura Illusione moderata; Requisiti di Creare Oggetti Magici Superiori, Arma Magica, Camuffare Se Stesso; Costo +2000 mo.

\subsubsection*{Munizione Fantasma}\index[OggettiMagici]{Armi Magiche!Munizione Fantasma}

Questa capacità può essere conferita solo alle munizioni. Una munizione con questa capacità speciale delle armi si dissolve 1 round dopo essere stata scagliata. In aggiunta, se il Proiettile colpisce un bersaglio, la ferita causata si richiude non appena la munizione si disintegra. Il Proiettile infligge danni normalmente, ma non lascia alcuna traccia visibile di violenza.

Il prezzo si riferisce a 50 Munizioni Fantasma.

\textbf{Dettagli}: Aura Trasmutazione moderata; Requisiti di Creare Oggetti Magici Superiori, Disintegrazione, Riparare; Costo +1000.

\subsubsection*{Munizioni Infinite}\index[OggettiMagici]{Armi Magiche!Munizioni Infinite}

Solo archi e balestre possono essere rese armi dalle Munizioni Infinite. Ogni volta che un'arma dalle Munizioni Infinite viene incoccata, una singola freccia o quadrello non magico viene creato spontaneamente dalla sua magia, quindi chi lo impugna non ha mai bisogno di caricare l'arma con delle munizioni.

Se chi lo impugna tenta di caricare l'arma con altre munizioni, la freccia o il quadrello creato svanisce immediatamente e si può caricare l'arma come di norma. Questa capacità non riduce l'ammontare di tempo necessario per caricare o scoccare con l'arma. La freccia o il quadrello creato svanisce se rimosso dall'arma; persiste solo se scagliato. A differenza di una normale munizione per arco o balestra, queste frecce e quadrelli vengono sempre distrutti quando scagliati.

\textbf{Dettagli}: Aura Evocazione moderata; Requisiti di Creare Oggetti Magici Superiori, Creazione; Costo +6000 mo.

\index[OggettiMagici]{Armi Magiche!Perfida}\subsubsection*{Perfida}

Quando ottieni 17 o 18 al Tiro per Colpire con quest'arma magica, il bersaglio subisce 7 danni aggiuntivi del tipo dell'arma.

\textbf{Dettagli}: Aura Evocazione debole; Requisiti di Creare Oggetti Magici, Causa Ferite Leggere; Costo +3000 mo.

\subsubsection*{Pietosa}\index[OggettiMagici]{Armi Magiche!Pietosa}

Tutto il danno inflitto dall'arma è temporaneo.

A comando, l'arma sopprime questa capacità fino a quando non le viene ordinato di riattivarla (permettendole di infliggere danni letali).

\textbf{Dettagli}: Aura Evocazione debole; Requisiti di Creare Oggetti Magici, Cura Ferite Leggere; Costo +3000 mo.

\subsubsection*{Planare}\index[OggettiMagici]{Armi Magiche!Planare}

Un'arma Planare è efficace contro tutti i tipi di esseri extradimensionali, essendo in grado di superare la loro resistenza ai danni fisici. Quando usata per attaccare gli Esterni, un'arma Planare ignora 5 punti della loro Riduzione del Danno o Resistenze.

\textbf{Dettagli}: Aura Evocazione moderata; Requisiti di Creare Oggetti Magici Superiori, Spostamento Planare; Costo +3000 mo.

\subsubsection*{Prensile}\index[OggettiMagici]{Armi Magiche!Prensile}

Questa capacità può essere conferita solo alle fruste. Una Frusta Prensile può, come azione di movimento, aggrapparsi a un oggetto come se fosse un Rampino. La Frusta può poi essere usata per scalare superfici o dondolare attraverso una stanza o qualsiasi area all'aperto.

\textbf{Dettagli}: Aura Ammaliamento moderato; Requisiti di Creare Oggetti Magici Superiori, Trucco della Corda; Costo +2.500.

\index[OggettiMagici]{Armi Magiche!Mazza della Punizione}\subsubsection*{Mazza della Punizione}

Ottieni un ulteriore +3 al colpire e danno quando usi quest'arma per attaccare un costrutto.

Quando ottieni un critico al Tiro per Colpire con quest'arma, il bersaglio subisce 7 danni da botta aggiuntivi, o 14 danni da botta aggiuntivi se è un costrutto. Se, dopo aver subito questi danni, a un costrutto restano 25 Punti Ferita o meno, viene distrutto.

\textbf{Dettagli}: Aura Invocazione forte; Requisiti di Creare Oggetti Magici Superiori; Costo +7000 mo.

\subsubsection*{Ricercante}\index[OggettiMagici]{Armi Magiche!Ricercante}

Questa capacità può essere aggiunta solo ad armi a distanza. Un'arma Ricercante vira verso il suo bersaglio, negando qualsiasi probabilità di mancarlo che si potrebbe applicare, come quella dovuta all'Occultamento. Chi la impugna deve comunque mirare l'arma nel quadretto giusto. Le Frecce scoccate per errore in uno spazio vuoto, per esempio, non virano per colpire gli avversari Invisibili, se ce n'è qualcuno nelle vicinanze.

\textbf{Dettagli}: Aura Divinazione forte; Requisiti di Creare Oggetti Magici Superiori, Visione del Vero; Costo +3000 mo.

\subsubsection*{Ritornante}\index[OggettiMagici]{Armi Magiche!Ritornante}

Un'arma Ritornante può teletrasportarsi nelle mani del suo possessore come azione immediata, anche se si trova in possesso di un'altra creatura. Questa capacità ha un raggio massimo di 30 metri e gli effetti che bloccano il teletrasporto impediscono il ritorno di un'arma Ritornante. Un'arma Ritornante deve essere in possesso di una creatura per almeno 24 ore perché questa capacità funzioni.

\textbf{Dettagli}: Aura Evocazione moderata; Requisiti di Creare Oggetti Magici Superiori, Teletrasporto; Costo +3000 mo.

\subsubsection*{Sacra}\index[OggettiMagici]{Armi Magiche!Sacra}

Ottieni un bonus di +3 ai Tiri per Colpire e danno effettuati con quest'arma magica. Quando con essa colpisci un immondo o un non morto, quella creatura subisce 2d10 danni da Luce aggiuntivi.

Mentre impugni la spada sguainata, essa crea un'aura di 3 metri di raggio attorno a te. Tu e tutte le creature a te amichevoli all'interno dell'aura ottenete +1d6 ai Tiri Salvezza contro incantesimi e altri effetti magici generati da Seguaci o Devoti di altri Patroni. Se hai Tratti in comune con il Patrono 13 o più, il raggio dell'aura aumenta a 9 metri.

\textbf{Dettagli}: Aura Invocazione moderata; Tratti comuni 12; Requisiti di Creare Oggetti Magici Superiori; Costo +6000 mo.



\subsubsection*{Sprezzante}\index[OggettiMagici]{Armi Magiche!Sprezzante}

Questa capacità speciale può essere aggiunta solo ad armi da mischia. Un'arma Sprezzante aiuta chi la impugna a sopravvivere in condizioni disperate. Rimane nelle mani di chi la impugna anche se quest'ultimo è Spaventato, Stordito o Privo di Sensi. Chi la impugna aggiunge il suo bonus come bonus alle prove per Pronto Soccorso quando è svenuto o morente ed aggiunge lo stesso anche ai TS contro incantesimi che causano morte istantanea.

\textbf{Dettagli}: Aura Abiurazione forte; Requisiti di Creare Oggetti Magici Superiori, Stabilizzare; Costo +6000 mo.

\index[OggettiMagici]{Armi Magiche!Terrore}\subsubsection*{Terrore}

Mentre la impugni, puoi usare due azioni e spendere 1 carica per scatenare un'ondata di terrore.
Ogni creatura di tua scelta, in un raggio di 9 metri a partire da te, deve superare un Tiro Salvezza su Volontà con DC 17 o restare spaventata da te per 1 minuto. Mentre è spaventata a questo modo, una creatura deve impiegare i suoi turni a cercare di muoversi più lontano possibile da te, e non può consapevolmente muoversi in uno spazio che sia entro 9 metri da te. Inoltre non può effettuare reazioni. Come sua azione, può usare solo l'azione di Movimento per Disingaggiare. Se non può muoversi da nessuna parte, la creatura può usare l'Azione Difesa Totale.

Al termine di ciascun suo round, la creatura può ripetere il Tiro Salvezza, terminando l'effetto per sé in caso lo superi. Quest'arma magica ha 3 cariche, e recupera 1d3 cariche ogni giorno all'alba.

\textbf{Dettagli}: Aura Ammaliamento moderata; Requisiti di Creare Oggetti Magici Superiori, Paura; Costo +8000 mo.

\subsubsection*{Titanica}\index[OggettiMagici]{Armi Magiche!Titanica}

Quest'arma è lunga 3 m e pesa quasi 40 kg (8 Ingombro), può essere usato solo da un gigante (o da un personaggio ingrandito). Se usato come un’arma ha un bonus +2 al colpire e infligge 1d4x10 ferite. Può essere usato anche per piantare rapidamente pali grossi come tronchi d'albero e per divellere con pochi colpi porte e cancelli.

\textbf{Dettagli}: Aura Trasmutazione moderata; Requisiti di Creare Oggetti Magici Superiori, Ingrandire/Ridurre; Costo +3000 mo.

\subsubsection*{Tocco Fantasma}\index[OggettiMagici]{Armi Magiche!Tocco Fantasma}

Un'arma del Tocco Fantasma infligge danni alle creature Incorporee normalmente, indipendentemente dal suo bonus magico e dalle immunità della creatura.

\textbf{Dettagli}: Aura Evocazione moderata; Requisiti di Creare Oggetti Magici Superiori, Spostamento Planare; Costo +3000 mo.

\subsubsection*{Tonante}\index[OggettiMagici]{Armi Magiche!Tonante}

Un'arma Tonante crea un tremendo frastuono simile a quello di un tuono, quando mette a segno un Colpo Critico. L'energia sonora non danneggia chi tiene in mano l'arma e infligge 1d8 danni sonori addizionali per ogni Colpo Critico andato a segno.Chi è soggetto a un Colpo Critico da un'arma Tonante deve effettuare un Tiro Salvezza su Tempra con DC 14 o resta Sordo in modo permanente.

\textbf{Dettagli}: Aura Necromanzia debole; Requisiti di Creare Oggetti Magici, Cecità/Sordità; Costo +3000 mo.

\subsubsection*{Trasformante}\index[OggettiMagici]{Armi Magiche!Trasformante}

Questa capacità si può aggiungere solo ad armi da mischia. Un'arma Trasformante altera la sua forma a comando di chi la impugna, diventando una qualsiasi altra arma da mischia con dimensioni simili Ad esempio, una Spada Lunga trasformante può assumere la forma di una qualsiasi altra arma da mischia a una mano Media, come una Scimitarra, un Mazzafrusto Leggero od un Tridente, ma non un'arma da mischia leggera o a due mani Media (come una Spada Corta Media o uno Spadone a due mani).

L'arma conserva tutte le sue capacità, compresi bonus e capacità speciali dell'arma, ad eccezione di quelle proibite dalla sua nuova forma attuale. Se lasciata incustodita, l'arma ritorna alla sua forma originaria.

\textbf{Dettagli}: Aura Trasmutazione moderata; Requisiti di Creare Oggetti Magici Superiori, Creazione Maggiore; Costo +5000 mo.

\subsubsection*{Trovacose}\index[OggettiMagici]{Armi Magiche!Trovacose}
Questa capacità concede a chi impugna quest'arma di lanciare l'incantesimo Localizza oggetto una volta al giorno

\textbf{Dettagli}: Aura Divinazione leggera; Requisiti di Creare Oggetti Magici, Localizza oggetto; Costo +1000 mo.

\index[OggettiMagici]{Armi Magiche!Vampira}\subsubsection*{Vampira}

Quando attacchi una creatura con quest'arma magica e ottieni un critico al Tiro per Colpire, il bersaglio, a parte i costrutti e i non morti, subisce 10 danni da Vuoto aggiuntivi e tu guadagni 10 Punti Ferita temporanei.

\textbf{Dettagli}: Aura Necromantica moderata; Requisiti di Creare Oggetti Magici Superiori, Tocco Vampiro; Costo +8000 mo.

\subsubsection*{Velocità}\index[OggettiMagici]{Armi Magiche!Velocità}

Quando compie più attacchi (2 Azioni), chi impugna un'arma di Velocità può compiere un attacco addizionale con l'arma. L'attacco aggiuntivo non ha le penalità degli attacchi multipli. Questa capacità non è cumulabile con incantesimi o effetti simili.

\textbf{Dettagli}: Aura Trasmutazione moderata; Requisiti di Creare Oggetti Magici Superiori, Velocità; Costo +15000 mo.

\index[OggettiMagici]{Armi Magiche!Vorpal}\subsubsection*{Vorpal}

Pur essendo un arma magica +1 viene considerata un arma magica +5 per valutare immunità e bonus al Tiro per Colpire e danno. Inoltre, l'arma ignora la resistenza ai danni taglienti. Quando attacchi una creatura che abbia almeno una testa con quest'arma e ottieni un critico al Tiro per Colpire, tagli una delle teste della creatura. La creatura muore se non può sopravvivere senza la perdita della testa.

Una creatura è immune a questo effetto se è immune ai danni taglienti, non possiede o non ha bisogno di una testa o il Narratore decide che la creatura è troppo grossa perché la sua testa sia recisa da quest'arma.

Una creatura del genere subisce invece 6d8 danni taglienti aggiuntivi dal colpo subito.

\textbf{Dettagli}: Aura Invocazione molto forte; Requisiti di Creare Oggetti Magici Mitici; Costo +150000 mo, leggendaria.

\subsection{Capacità Speciali delle Armature e Scudi Magici}

Gran parte delle armature e degli scudi magici hanno solo bonus, ma certi possiedono alcune delle capacità speciali descritte qui sotto. Un'armatura o uno scudo con capacità speciali devono avere almeno bonus +1.

\index{Armatura / Scudo Magico}\subsubsection*{Armatura / Scudo Magico}

\textit{Armatura (qualsiasi)} +1 2500 mo, +2 10000 mo, +3 18000 mo, +4 35000 mo, +5 80000 mo

\textit{Scudi (piccoli, medi, pesanti)}: +1 1500 mo, +2 4000 mo, +3 9000 mo, +4 20000 mo, +5 35000 mo

Mentre impugni questo scudo/armatura, hai un bonus alla Difesa determinato dal bonus magico dello scudo/armatura. Questo bonus è in aggiunta al normale bonus alla Difesa fornito dallo scudo/armatura.

\subsubsection*{Accecante}\index[OggettiMagici]{Armature e Scudi!Accecante}

Uno scudo dotato di questo incantamento emana una luce accecante per un massimo di due volte al giorno su comando di chi lo impugna. Tutti coloro che si trovano entro 6 metri dallo scudo, eccetto chi lo impugna, devono superare un Tiro Salvezza su Riflessi con DC 14 o restano Accecati per 1d4 round.

\textbf{Dettagli}: Aura Invocazione moderata; Requisiti di Costruzione Creare Oggetti Magici Superiori, Luce diurna; Costo +3000 mo.

\index[OggettiMagici]{Armature e Scudi!Adamantio}\subsubsection*{Adamantio}

Armatura (media o pesante, ma non di pelle), non comune +700 mo oltre il prezzo base dell'armatura. Mentre la indossi, qualsiasi colpo critico che subisci diventa un colpo normale (ma non protegge dall'esplosione del danno).


\subsubsection*{Amorfa}\index[OggettiMagici]{Armature e Scudi!Amorfa}

Una volta al giorno a comando, chi indossa l'armatura (insieme a qualsiasi equipaggiamento indossi) può assumere la forma di un liquido viscoso che è in grado di passare attraverso qualsiasi spazio nel quale potrebbe ragionevolmente scorrere del fango denso. Mentre si usa questa capacità, la propria velocità viene ridotta a 3 metri e si possono effettuare solo azioni di movimento. Si può assumere questa forma per 1 minuto o finché non si spende un'azione di movimento per tornare alla propria forma naturale. Un'armatura Amorfa deve essere fatta principalmente di cuoio, stoffa o altro materiale organico e flessibile.

\textbf{Dettagli}: Aura Trasmutazione moderata; Requisiti di Costruzione Creare Oggetti Magici Superiori, Metamorfosi, Costo +2.250 mo.


\subsubsection*{Antiemorragica}\index[OggettiMagici]{Armature e Scudi!Antiemorragica}

Un'armatura Antiemorragica aiuta a fermare la perdita di sangue dalle ferite di chi la indossa, stringendo automaticamente come un laccio emostatico nei punti appropriati mentre riduce anche magicamente l'entità della ferita.

Un'armatura Antiemorragica riduce i danni ai Punti Ferita di 1 per colpo subito e si è non si può subire danni da sanguinamento.

\textbf{Dettagli}: Aura Trasmutazione moderata; Requisiti di Costruzione Creare Oggetti Magici Superiori, Cura Ferite Critiche, Ristorare Inferiore o Stabilizzare; Costo +3000 mo.

\subsubsection*{Ariete}\index[OggettiMagici]{Armature e Scudi!Ariete}

Questi scudi sono molto solidi e spesso recano l'emblema di un ariete o un toro. Quando chi indossa uno scudo ariete effettua un attacco con lo scudo come parte di una Carica, il bonus alla Difesa dello scudo si applica ai Tiri per Colpire e per i danni. Questo non si cumula con nessun altro potenziamento che possegga lo scudo. Questa capacità non è applicabile agli scudi di tipo leggero.

\textbf{Dettagli}: Aura Invocazione debole; Requisiti di Costruzione Creare Oggetti Magici Superiori, Costo +3000 mo.


\subsubsection*{Attaccabrighe}\index[OggettiMagici]{Armature e Scudi!Attaccabrighe}

Chi indossa un'armatura Attaccabrighe ottiene bonus +2 ai Tiri per Colpire e per i danni degli attacchi senz'armi. I suoi colpi senz'armi contano come armi magiche al fine di superare la Riduzione del Danno. La capacità Attaccabrighe può essere applicata solo alle armature leggere.

\textbf{Dettagli}: Aura Trasmutazione debole; Requisiti di Costruzione Creare Oggetti Magici, Forza del Toro; Costo +15000 mo

\subsubsection*{Bilanciata}\index[OggettiMagici]{Armature e Scudi!Bilanciata}

Questa armatura respinge tutto ciò che rischia di buttare a terra chi la indossa. Il portatore ottiene bonus +1d6 contro chi prova a spingerlo o buttarlo a terra.

Buttarsi a terra mentre si indossa un'armatura Bilanciata è un'azione di movimento invece di una azione immediata. La capacità Bilanciata può essere applicata ad armature leggere o medie, ma non a quelle pesanti o agli scudi.

\textbf{Dettagli}: Aura Trasmutazione debole; Requisiti di Costruzione Creare Oggetti Magici, Costo +3000 mo.

\index[OggettiMagici]{Armature e Scudi!Bracciali dell'Arciere}\subsubsection*{Bracciali dell'Arciere}

Mentre indossi questi bracciali, hai competenza con l'arco lungo e l'arco corto, e ottieni un bonus di +2 ai tiri di danno degli attacchi a distanza effettuati con queste armi.

\textbf{Dettagli}: Aura Trasmutazione debole; Requisiti di Costruzione Creare Oggetti Magici, Costo +3000 mo.

\index[OggettiMagici]{Armature e Scudi!Bracciali della Difesa}\subsubsection*{Bracciali della Difesa}
\textit{Oggetto meraviglioso, raro}

Mentre indossi questi bracciali, hai un bonus di +1, +2, +3, +4+, +5 alla tua Difesa se non indossi nessuna armatura e non usi nessuno scudo.

\textbf{Dettagli}: Aura Abiurazione; Requisiti di Costruzione Creare Oggetti Magici Superiori, Costo +6000 mo, 15000 mo, 30000 mo, 45000 mo, 60000 mo.

\index[OggettiMagici]{Armature e Scudi!Bracciali della Difesa Maggiore}\subsubsection*{Bracciali della Difesa Maggiore}
\textit{Oggetto meraviglioso, leggendario}

Questi bracciali funzionano come un armatura pur non essendo tali. Vieni avvolto in uno scudo magico invisibile che ti concede Difesa 15, 17, 19, 21, 23. La Difesa può essere aumentata con oggetti magici che migliorano la Difesa, tranne armature e scudi.

\textbf{Dettagli}: Aura Abiurazione; Requisiti di Costruzione Creare Oggetti Magici Superiori, Costo +12000 mo, 24000 mo, 36000 mo, 50000 mo, 75000 mo


\subsubsection*{Brillante}\index[OggettiMagici]{Armature e Scudi!Brillante}

Armature e scudi con la capacità speciale Brillante irradiano luce come una torcia quando indossati, che può essere soppressa o riattivata a comando. L'aspetto dell'oggetto di solito è caratterizzato da colori vivaci e una brillante lucentezza anche quando non illuminato. Una volta al giorno, il portatore può comandare all'armatura o allo scudo di brillare con l'intensità di un incantesimo Luce Diurna per 10 minuti o finché non gli viene comandato di affievolirla.

Questa armatura va pulita almeno 1 volta a settimana o perde i poteri per una settimana.

\textbf{Dettagli}: Aura Invocazione moderata; Requisiti di Costruzione Creare Oggetti Magici, Luce Diurna; Costo +3.750 mo.

\subsubsection*{Carico}\index[OggettiMagici]{Armature e Scudi!Carico}

Un'armatura del Carico distribuisce il peso trasportato da chi la indossa più efficacemente, permettendogli di trasportarne di più senza subire gli effetti dell'Ingombro. La capacità di ingombro di chi la indossa viene aumentata del 50\%.

\textbf{Dettagli}: Aura Trasmutazione debole; Requisiti di Costruzione Creare Oggetti Magici, Armatura Passiva; Costo +2000 mo.

\index[OggettiMagici]{Armature e Scudi!Armatura Demoniaca}\subsubsection*{Armatura Demoniaca}

Mentre indossi l'armatura puoi comprendere e parlare l'Abissale. Inoltre, le manopole artigliate dell'armatura trasformano i colpi disarmati effettuati con le tue mani in armi magiche che infliggono danni taglienti, con un bonus di +1 ai Tiri per Colpire e ai tiri di danno e il d8 come dado di danno.

\textbf{Dettagli}: Aura Evocazione forte; Requisiti di Costruzione Creare Oggetti Magici Superiori; Costo +5000 mo.

\subsubsection*{Denegante}\index[OggettiMagici]{Armature e Scudi!Denegante}

Quando colui che indossa l'armatura è bersaglio di un Colpo Critico o Esplosione del danno effettuato con un'arma da mischia, può automaticamente negare questo Critico e renderlo un attacco normale. Questa capacità può essere applicata solo alle armature pesanti. L'abilità è usabile un numero di volte al giorno pari al bonus magico dell'arma.

\textbf{Dettagli}: Aura Abiurazione forte; Requisiti di Costruzione Creare Oggetti Magici Superiori; Costo +25000 mo.

\subsubsection*{Determinazione}\index[OggettiMagici]{Armature e Scudi!Determinazione}

Uno scudo o un'armatura concede la capacità di combattere in circostanze apparentemente impossibili. Una volta al giorno, quando il possessore raggiungere 0 o meno Punti Ferita, l'oggetto attiva automaticamente l'incantesimo Cure ferite serie.

\textbf{Dettagli}: Aura Evocazione moderata; Requisiti di Costruzione Creare Oggetti Magici Superiori, Cura ferite serie; Costo +15000 mo.

\index[OggettiMagici]{Armature e Scudi!Difesa dagli Incantesimi}\subsubsection*{Difesa dagli Incantesimi}

Hai +1d6 ai Tiri Salvezza contro incantesimi e altri effetti magici.

\textbf{Dettagli}: Aura Abiurazione forte; Requisiti di Costruzione Creare Oggetti Magici Superiori; Costo +5000 mo.

\index[OggettiMagici]{Armature e Scudi!Elegante}\subsubsection*{Elegante}

Puoi usare due azioni per pronunciare la parola di comando per ottenere che l'armatura assuma l'aspetto di un comune abito o qualche altro tipo di armatura. Decidi tu l'aspetto, compreso il colore, lo stile e gli accessori, ma l'armatura / scudo mantiene il suo normale Ingombro e peso. L'aspetto illusorio dura finché non usi di nuovo questa proprietà o ti togli l'armatura.

\textbf{Dettagli}: Aura Illusione moderata; Requisiti di Costruzione Creare Oggetti Magici Superiori; Costo +3000 mo.


\subsubsection*{Felpa}\index[OggettiMagici]{Armature e Scudi!Felpa}

Un armatura con la capacità Felpa conta per quanto concerne le penalità di indossare un armatura ad una armatura leggera. Il personaggio riesce a muoversi quasi senza difficoltà con questa armatura.

\textbf{Dettagli}: Aura Trasmutazione forte; Requisiti di Costruzione Creare Oggetti Magici Superiori; Costo +6000 mo.

\subsubsection*{Forma Eterea}\index[OggettiMagici]{Armature e Scudi!Forma Eterea}

A comando, questa proprietà permette a chi indossa l'armatura di diventare Etereo (come per l'incantesimo Forma Eterea) una volta al giorno. Il personaggio può rimanere Etereo per quanto tempo desidera ma, una volta tornato alla normalità, per quel giorno non può più diventare Etereo.

\textbf{Dettagli}: Aura Trasmutazione forte; Requisiti di Costruzione Creare Oggetti Magici Meravigliosi, Forma Eterea; Costo +24.500 mo.

\subsubsection*{Invulnerabilità}\index[OggettiMagici]{Armature e Scudi!Invulnerabilità}

Questa armatura garantisce a chi la indossa una Riduzione del Danno di 5/magia. Un armatura con Invulnerabilità emette un'aura di Abiurazione forte.

\textbf{Dettagli}: Aura Abiurazione forte; Requisiti di Costruzione Creare Oggetti Magici Mitici, Desiderio; Costo +15000 mo.

\subsubsection*{Irrintracciabile}\index[OggettiMagici]{Armature e Scudi!Irrintracciabile}

Un'armatura Irrintracciabile alleggerisce i passi di chi la indossa e ne camuffa l'aspetto. Le prove di Sopravvivenza per seguire le tracce del portatore subiscono penalità -5, e chi indossa l'armatura ottiene Bonus di +5 alle prove di Furtività. Soltanto le armature di cuoio o di pelle possono essere Irrintracciabili.

\textbf{Dettagli}: Aura Trasmutazione debole; Requisiti di Costruzione Creare Oggetti Magici, Passare Senza Tracce; Costo +3.750 mo.

\subsubsection*{Mascheramento}\index[OggettiMagici]{Armature e Scudi!Mascheramento}

A comando, un'armatura di questo tipo muta la sua forma e appare come un normale set di vestiti. L'armatura conserva tutte le sue proprietà (compreso il peso) anche quando è mascherata. Solo Visione del Vero o altre magie simili rivelano la reale natura dell'armatura trasformata.

\textbf{Dettagli}: Aura Illusione moderata; Requisiti di Costruzione Creare Oggetti Magici Superiori, Camuffare Se Stesso; Costo +1.350 mo.

\index[OggettiMagici]{Armature e Scudi!Mithral}\subsubsection*{Mithral}

Armatura media o pesante, ma non di pelle, non comune +800 mo oltre il prezzo base dell'armatura. Il mithral è un metallo leggero e flessibile. Un giaco di maglia o un pettorale di mithral possono essere indossati sotto abiti normali. Riduce di 1 la categoria di peso per determinare malus alle prove di competenza e Magia.

\subsubsection*{Ombra}\index[OggettiMagici]{Armature e Scudi!Ombra}

Quest'armatura rende chi la indossa sfocato ogni volta che tenta di nascondersi, fornendo Bonus di +5 alle sue prove di Nascondersi nelle ombre. La penalità di armatura alla prova si applica normalmente.

\textbf{Dettagli}: Aura Illusione debole; Requisiti di Costruzione Creare Oggetti Magici, Invisibilità, Silenzio; Costo +1.875 mo.

\subsubsection*{Ospitale}\index[OggettiMagici]{Armature e Scudi!Ospitale}

Un'armatura o uno scudo con questa capacità speciale nasconde animali vivi all'interno della sua iconografia per tenerli al sicuro. Il portatore con una parola di comando immagazzina magicamente un animale a cui è legato, come un Famiglio o una Cavalcatura. L'animale immagazzinato appare come simbolo sull'armatura o sullo scudo, che si tratti di un'imitazione dell'aspetto dell'animale o di una rappresentazione più simbolica e astratta.

Mentre è immagazzinato, l'animale dorme e non dà alcun beneficio (come il bonus alle abilità di un Famiglio) a chi la indossa. La taglia degli animali immagazzinabili dipende dal tipo di armatura o scudo. Le armature leggere o medie e gli scudi leggeri o pesanti possono immagazzinare un animale di taglia massima pari a quella di chi li indossa. Un'armatura pesante o uno scudo torre possono immagazzinare un animale fino a una categoria taglia superiore rispetto a chi li indossa. Una seconda parola di comando rilascia l'animale immagazzinato nell'armatura o nello scudo ospitale. Un animale liberato si risveglia immediatamente, appare in uno spazio adiacente al portatore e può intraprendere azioni nel round in cui appare.

Dato che l'animale immagazzinato dorme anziché essere in animazione sospesa (o persino in letargo), invecchia e ha fame al ritmo normale mentre è immagazzinato. Un'armatura o uno scudo Ospitale rilascia automaticamente un animale immagazzinato 24 ore dopo che vi è stato immagazzinato all'interno.

\textbf{Dettagli}: Aura Evocazione moderata; Requisiti di Costruzione Creare Oggetti Magici Superiori, Scrigno Segreto; Costo +3.750 mo.


\subsubsection*{Percettiva}\index[OggettiMagici]{Armature e Scudi!Percettiva}

Un'armatura Percettiva viene in soccorso quando chi la indossa è stato Accecato, si trova nel buio totale (se il portatore non ha Scurovisione o la capacità Vedere al Buio), o si trova in un'oscurità magica. Quando una di queste condizioni influenza chi indossa l'armatura, un'armatura percettiva gli concede immediatamente Vista Cieca in un raggio di 1,5 metri. Non appena il portatore torna a vedere, i sensi addizionali cessano. Chi indossa l'armatura non può ottenere queste capacità chiudendo gli occhi.

\textbf{Dettagli}: Aura Divinazione forte; Requisiti di Costruzione Creare Oggetti Magici Meravigliosi, Visione del Vero; Costo +15000 mo.

\subsubsection*{Resistenza al Veleno}\index[OggettiMagici]{Armature e Scudi!Resistenza al Veleno}

Un'armatura o uno scudo con questa capacità speciale conferisce a chi lo indossa Bonus di +3 ai Tiri Salvezza contro il veleno.

\textbf{Dettagli}: Aura Trasmutazione debole; Requisiti di Costruzione Creare Oggetti Magici Superiori, Rimuovi Veleno; Costo +1.125 mo.

\subsubsection*{Resistenza all'Energia}\index[OggettiMagici]{Armature e Scudi!Resistenza all'Energia}

Questo tipo di armatura o scudo protegge contro un tipo di energia (Fuoco, Luce, Suono, Elettricità, Energia Positiva, Energia Negativa, Freddo, Vuoto) ed è decorata da disegni che raffigurano l'elemento dal quale protegge. L'armatura o lo scudo assorbono i primi 10 danni di energia per attacco che verrebbero subiti normalmente da chi li indossa.

\textbf{Dettagli}: Aura Abiurazione debole; Requisiti di Costruzione Creare Oggetti Magici, Protezione dall'Energia; Costo +9000 mo.

\subsubsection*{Resistenza all'Energia Superiore}\index[OggettiMagici]{Armature e Scudi!Resistenza all'Energia Superiore}

Questo tipo di armatura o scudo protegge contro un tipo di energia (Fuoco, Luce, Suono, Elettricità, Energia Positiva, Energia Negativa, Freddo, Vuoto) ed è decorata da disegni che raffigurano l'elemento dal quale protegge. L'armatura o lo scudo concedono Resistenza all'energia indicata.

\textbf{Dettagli}: Aura Abiurazione moderata; Requisiti di Costruzione Creare Oggetti Magici Superiori, Protezione all'Energia; Costo +21000 mo.

\subsubsection*{Selvatica}\index[OggettiMagici]{Armature e Scudi!Selvatica}

Un'armatura con questa capacità speciale generalmente sembra fatta di pelle animale magicamente indurita. Chi indossa un'armatura o uno scudo con questa capacità conserva la Difesa anche mentre è trasformato in un animale (vuoi per Incantesimo o Abilità).

Le armature e gli scudi con questa capacità di solito recano motivi di foglie. Mentre chi la indossa è in Forma Selvatica, l'armatura non è visibile.

\textbf{Dettagli}: Aura Trasmutazione moderata; Requisiti di Costruzione Creare Oggetti Magici Superiori, Metamorfosi; Costo +15000 mo.

\index[OggettiMagici]{Armature e Scudi!Scaglie di Drago}\subsubsection*{Scaglie di Drago}

Questa armatura o scudo è fatta con le scaglie di una specie di drago.

Mentre la indossi hai +1d6 ai Tiri Salvezza contro la Presenza Spaventosa e le armi a soffio dei draghi ed hai resistenza a un tipo di danno determinato dalla specie di drago che ha fornito le scaglie.

Inoltre, con due azioni puoi focalizzare i tuoi sensi per determinare magicamente la distanza e la direzione in cui si trovi il drago più vicino entro 45 chilometri che sia della stessa specie dell'armatura. Quest'azione speciale non può essere più usata fino alla prossima alba.

\textbf{Dettagli}: Aura Abiurazione moderata; Requisiti di Costruzione Creare Oggetti Magici Superiori; Costo +8000 mo.

\subsubsection*{Scudo Animato}\index[OggettiMagici]{Armature e Scudi!Scudo Animato}

Mentre impugni questo scudo, con due azioni puoi pronunciare una parola di comando e farlo animare. Lo scudo fluttuerà nell'aria all'interno del tuo spazio per proteggerti come se lo stessi impugnando, lasciandoti libera la mano.

Lo scudo resta animato per 1 minuto, finché non usi due azioni per terminarne l'effetto, sei inabile o muori: a quel punto lo scudo cadrà a terra o tornerà nella tua mano se ne hai una libera.

\textbf{Dettagli}: Aura Trasmutazione forte; Requisiti di Costruzione Creare Oggetti Magici Superiori, Animare Oggetti; Costo +6000 mo.

\index[OggettiMagici]{Armature e Scudi!Scudo dell'Attrazione dei Proiettili}\subsubsection*{Scudo dell'Attrazione dei Proiettili}

Mentre impugni questo scudo apparentemente hai resistenza ai danni da parte degli attacchi con arma a distanza.

\textit{Versione maledetta}.

Togliersi lo scudo non pone fine alla maledizione. Ogni qualvolta un attacco con arma a distanza viene effettuato contro un bersaglio entro 3 metri da te, la maledizione fa sì che diventi tu il bersaglio dell'attacco.

\textbf{Dettagli}: Aura Trasmutazione forte; Requisiti di Costruzione Creare Oggetti Magici Superiori, Animare Oggetti; Costo +2000 mo.


\subsubsection*{Soffio del Dragone}\index[OggettiMagici]{Armature e Scudi!Soffio del Dragone}

Uno scudo con questa capacità speciale di solito è realizzato con le fauci di un drago spalancate sulla parte anteriore. Uno scudo con la capacità speciale Soffio del Dragone è legato a un tipo di energia (veleno, elettricità, freddo o fuoco).Lo scudo recupera 1d4 cariche ad ogni alba e ne può tenere fino a 10.

A comando, 2 Azioni, chi lo indossa può consumare da 1 a 5 cariche dello scudo per fargli emettere un Soffio in un cono di 4,5 metri che infligge 1d4 danni da energia per carica consumata (Riflessi DC 11 dimezza). Questo danno è dello stesso tipo di energia legato allo scudo. Uno scudo non può avere più di una capacità Soffio del Dragone.

\textbf{Dettagli}: Aura Invocazione debole; Requisiti di Costruzione Creare Oggetti Magici, Mani Brucianti; Costo +2.500 mo.

\subsubsection*{Titanica}\index[OggettiMagici]{Armature e Scudi!Titanica}

Un'armatura con la proprietà Titanica è quasi comicamente fuori misura, anche se l'effetto è solo esteriore e l'interno accoglie una creatura come di norma, senza necessitare modifiche. Una creatura che indossa un'armatura Titanica è considerata di una categoria di taglia superiore, questo anche al fine dell'uso di oggetti ed armi o dell'essere influenzati da attacchi speciali che dipendono dalla taglia, come Inghiottire e Travolgere.

\textbf{Dettagli}: Aura Trasmutazione moderata; Requisiti di Costruzione Creare Oggetti Magici Superiori, Ingrandire; Costo +15000 mo.

\subsubsection*{Tocco Fantasma}\index[OggettiMagici]{Armature e Scudi!Tocco Fantasma}

Quest'armatura o scudo sembra quasi trasparente. Il valore di Difesa dato dall'armatura viene conteggiato contro gli attacchi delle creature corporee e Incorporee. Inoltre l'armatura o lo scudo possono essere raccolti, spostati e indossati in qualsiasi momento dalle creature corporee e Incorporee. Le creature Incorporee ottengono il bonus dell'oggetto contro attacchi corporei e incorporei, e mantengono comunque la capacità di passare attraverso gli oggetti solidi.

\textbf{Dettagli}: Aura Trasmutazione forte; Requisiti di Costruzione Creare Oggetti Magici Meravigliosi, Forma Eterea; Costo +15000 mo.

\index[OggettiMagici]{Armature e Scudi!Vulnerabilità}\subsubsection*{Vulnerabilità}

Mentre la indossi, hai resistenza a uno dei seguenti tipi di danno: contundente, perforante o tagliente. Il Narratore sceglie il tipo. L'armatura è maledetta, mentre sei maledetto, hai vulnerabilità a due dei tre tipi di danno associati con l'armatura (che non siano quello a cui hai resistenza).

\textbf{Dettagli}: Aura Necromantica moderata; Requisiti di Costruzione Creare Oggetti Magici, Scagliare Maledizione; Costo +3000 mo.


\subsection{Amuleti, Collane e Gioielli}

\index[OggettiMagici]{Oggetti Magici!Amuleto Antiveleno}\subsubsection*{Amuleto Antiveleno}
3000 mo, non comune, questa gemma appesa a una catenella d’argento è nera e lucida. Chi la indossa ha una +1d6 al Tiro Salvezza contro veleno.

\index[OggettiMagici]{Oggetti Magici!Amuleto della Cancrena}\subsubsection*{Amuleto della Cancrena}
questa gemma incisa appesa a una catenella sembra essere di scarso valore. Se un personaggio la tiene con se per più di 1 giorno, viene colpito da una terribile cancrena che gli fa perdere permanentemente 1 punto di Destrezza, Costituzione e Carisma alla settimana. La gemma (e la cancrena) possono essere neutralizzate solo da Rimuovi Maledizione e cura malattia, seguiti da guarigione o desiderio. La cancrena può anche essere sconfitta macinando un amuleto della salute e spargendone la polvere sul personaggio afflitto

\index[OggettiMagici]{Amuleto Cicatrizzante}\subsubsection*{Amuleto Cicatrizzante}
25000 mo, molto raro, questa gemma appesa a una catenella d’oro è rossa e brillante. Chi la indossa recupera i Punti Ferita due volte più rapidamente del normale (anche Punti Ferita Massimi). L’amuleto impedisce di subire danni da Sanguinamento.

\index[OggettiMagici]{Amuleto Contro la Possessione}\subsubsection*{Amuleto Contro la Possessione}
32000 mo, molto raro, il possessore di questo amuleto di rame diviene immune agli incantesimi di possessione e dominazione.

\index[OggettiMagici]{Amuleto della Localizzazione inevitabile}\subsubsection*{Amuleto della Localizzazione inevitabile}
questo amuleto maledetto ha l'apparenza di un amuleto dell'introvabilità. Al contrario, rende il possessore vulnerabile a questo tipo di magia. La probabilità di osservare il possessore e la durata di incantesimi usati per tale scopo raddoppiano.

\index[OggettiMagici]{Amuleto dei Piani}\subsubsection*{Amuleto dei Piani}
160000 mo, leggendario, mentre indossi questo amuleto, puoi usare due azioni per nominare un luogo con cui sei familiare e che si trovi su di un altro piano di esistenza. Effettua una prova di Intelligenza con DC 18. Se la prova riesce, lanci l'incantesimo spostamento planare tramite l'amuleto. Se la prova fallisce, tu e ciascuna creatura e oggetto entro 5 metri da te venite trasportati in una destinazione casuale. Tira un 1d8. Da 1 a 4, raggiungi una destinazione casuale sul piano che hai nominato. Da 5 a 8, raggiungi un piano dell'esistenza determinato casualmente.

\index[OggettiMagici]{Amuleto di Protezione dalla Individuazione e Localizzazione}\subsubsection*{Amuleto di Protezione dalla Individuazione e Localizzazione}
20000 mo, raro, mentre indossi questo amuleto sei celato alla magia di divinazione. Non puoi essere preso come bersaglio da queste magie o percepito tramite sensori magici di scrutamento.

\index[OggettiMagici]{Amuleto della Resistenza Fisica}\subsubsection*{Amuleto della Resistenza Fisica}
8000 mo, raro, non mentre indossi questo amuleto hai un +2 ai Tiri Salvezza su Tempra.

\index[OggettiMagici]{Cerchietto dell'Esplosione}\subsubsection*{Cerchietto dell'Esplosione}
1500 mo, non comune, mentre indossi questo cerchietto, puoi usare due azioni per lanciare tramite esso l'incantesimo raggio rovente. Il cerchietto non potrà essere usato di nuovo a questo modo fino alla prossima alba.


\index[OggettiMagici]{Collana dell'Adattamento}\subsubsection*{Collana dell'Adattamento}
1500 mo, non comune, mentre indossi questa collana, puoi respirare normalmente in qualsiasi ambiente che abbia aria, e hai +1d6 ai Tiri Salvezza effettuati contro gas e vapori nocivi.

\index[OggettiMagici]{Collana dello Strangolamento}\subsubsection*{Collana dello Strangolamento}
questa collana sembra un gioiello di grande valore. Appena indossata, si stringe fulmineamente intorno al collo, infliggendo 6 danni a round. Non può essere rimossa in alcun modo se non con un desiderio o Rimuovi Maledizione, rimanendo stretta al collo della sua vittima anche dopo la morte. La collana si allenterà solo quando la vittima sarà diventata uno scheletro, pronta per essere raccolta da un ignaro cacciatore di tesori.

\index[OggettiMagici]{Collana delle Palle di Fuoco}\subsubsection*{Collana delle Palle di Fuoco}
a seconda delle sfere presenti: 500 mo, 1000 mo, 1600 mo, 2300 mo, 3100 mo, 4000 mo, 4500 mo, 5000 mo, 5500 mo, 6000 mo, non comune/raro/molto raro: da questa collana pendono 1d6 + 3 sfere. Puoi usare due azioni per staccare una sfera e lanciarla fino a 18 metri di distanza. Quando essa raggiunge il termine della sua traiettoria, la sfera detona come un incantesimo palla di fuoco (DC 18).

\index[OggettiMagici]{Collana del Rosario}\subsubsection*{Collana del Rosario}
3000 mo + variabile, raro, questa collana possiede 1d4 + 2 sfere magiche fatte di acquamarina, perla nera o topazio. Possiede anche diverse sfere non magiche. Se una sfera magica venisse rimossa dalla collana, quella sfera perderebbe la sua magia.

Esistono sei tipi di sfere magiche. Il Narratore decide il tipo di ciascuna sfera facente parte della collana. Una collana può avere più di una sfera dello stesso tipo. Per usarla, devi indossare la collana. Ogni sfera contiene un incantesimo che puoi lanciare con due azioni, con DC dell'Incantesimo pari a 10+2xLivello in caso di Tiro Salvezza. Una volta che l'incantesimo di una sfera magica è stato lanciato, non potrai usare di nuovo quella sfera fino all'alba successiva.

\medskip

\begin{tabularx}{0.45\textwidth}{llX}
\textbf{3d6} &\textbf{Sfera di...} &\textbf{Incantesimo}\\
\hline
3-5 &Benedizione &benedizione\\
6-11& Cura &cura ferite serie o ristorare inferiore\\
12-14 &Favore& ristorare superiore\\
15-16& Punire &punizione marchiante\\
17 &Vento& camminare nel vento\\
18 &Convocare &alleato planare\\
\end{tabularx}


\index[OggettiMagici]{Gemma Elementale}\subsubsection*{Gemma Elementale}
1200 mo, non comune, questa gemma contiene una scintilla di energia elementale. Quando usi due azioni per infrangere la gemma, questa evoca un elementale come se tu avessi lanciato l'incantesimo evoca elementali, e la magia della gemma svanisce. Il tipo di gemma determina l'elementale evocato dall'incantesimo.

\medskip

\begin{tabular}{ll}
\textbf{Gemma} &\textbf{Elementale evocato}\\
\hline
Corindone rosso& Elementale del fuoco\\
Diamante giallo& Elementale della terra\\
Smeraldo &Elementale dell'acqua\\
Zaffiro blu&Elementale dell'aria\\
\end{tabular}

\medskip

\index[OggettiMagici]{Gemma della Luminosità}\subsubsection*{Gemma della Luminosità}
5000 mo, raro, questo prisma ha 50 cariche. Mentre lo impugni, puoi usare due azioni per pronunciare una delle tre parole di comando per provocare uno dei seguenti effetti:

\begin{itemize}
\item
La prima parola di comando fa sì che la gemma produca una luce intensa nel raggio di 9 metri e luce fioca per ulteriori 9 metri. L'effetto non consuma cariche. Dura finché non userai due azioni per ripetere la parola di comando o finché non impiegherai un'altra funzione della gemma.

\item
La seconda parola di comando spende 1 carica e fa sì che la gemma proietti una fascio di luce luminoso contro una creatura visibile entro 18 metri da te. La creatura deve superare un Tiro Salvezza su Tempra con DC 17 o restare accecata per 1 minuto.

\item
La terza parola di comando spende 5 cariche e fa sì che la gemma irradi una luce accecante in un cono di 9 metri originante da te. Ogni creatura all'interno del cono deve effettuare un Tiro Salvezza come se fosse stata colpita dal fascio creato dalla seconda parola di comando.

\end{itemize}

\medskip

Quando tutte le cariche della gemma sono state spese, la gemma diventa un comune gioiello del valore di 50 mo.

\index[OggettiMagici]{Gemma della Vista}\subsubsection*{Gemma della Vista}
32000 mo, molto raro, con due azioni, puoi pronunciare la parola di comando della gemma e spendere 1 carica. Per i successivi 10 minuti, quando guardi attraverso la gemma possiedi la visione del vero fino a 36 metri di distanza. La gemma ha 3 cariche, e recupera 1 carica spese ogni giorno all'alba.

\index[OggettiMagici]{Gioiello Attiramostri}\subsubsection*{Gioiello Attiramostri}
questo gioiello magico è maledetto, il possessore attrae i mostri vaganti con il doppio della probabilità. I mostri, inoltre, lo inseguiranno al doppio della probabilità qualora egli fugga. Il gioiello non può essere abbandonato e riapparirà immediatamente sulla persona del possessore ogni volta che questi proverà a liberarsene. Solo Rimuovi Maledizione permetterà al possessore di lasciarsi indietro il gioiello.

\index[OggettiMagici]{Medaglione della Caduta piuma}\subsubsection*{Medaglione della Caduta piuma}
400 mo, non comune, questo medaglione attiva in automatico l'incantesimo Caduta Piuma quando il possessore cade da una altezza di 2 metri o più.

\index[OggettiMagici]{Medaglione dei Pensieri}\subsubsection*{Medaglione dei Pensieri}
3000 mo, non comune, mentre indossi questo medaglione, puoi usare due azioni e spendere 1 carica per lanciare tramite esso l'incantesimo individuazione dei pensieri (DC del Tiro Salvezza 15). Il medaglione ha 3 cariche, e recupera 1 carica spese ogni giorno all'alba.

\index[OggettiMagici]{Perla della Saggezza}\subsubsection*{Perla della Saggezza}
20000 mo, raro, questa perla magica dona un punto di Saggezza extra che la tiene con sé per 4 settimane. Trascorso questo tempo la perla dovrà essere sempre portata per non perderne i benefici. C'è un 5\% di probabilità che una perla sia maledetta e sortisca l’effetto opposto. In questo caso, dopo 4 settimane, l’effetto negativo è permanente e cancellabile solo da desiderio.

\index[OggettiMagici]{Scarabeo della Morte}\subsubsection*{Scarabeo della Morte}
questa spilla a forma di scarabeo sembra un semplice portafortuna. Tuttavia, se tenuto in mano per 1 round o portato per 1 turno, si trasforma in un orrendo insetto carnivoro. Dotata di potenti mandibole, la famelica creatura penetra attraverso il cuoio e il tessuto, affondando nella carme e raggiungendo il cuore in 1 round. Dopo aver ucciso la sua vittima, la creatura riassume la forma di spilla. Solo il calore che viene dal contatto con un essere vivente può animare l’insetto mostruoso, quindi mettere la spilla in una scatola o in una teca è una precauzione sufficiente per evitare ogni pericolo.

\index[OggettiMagici]{Scarabeo di Protezione}\subsubsection*{Scarabeo di Protezione}
36000 mo, leggendario, se tieni questo medaglione a forma di scarabeo tra le tue mani per 1 round, su di esso compare un'iscrizione che ne rivela la natura magica. Mentre è addosso a te, fornisce due benefici

- Hai +2 ai Tiri Salvezza contro incantesimi.

- Lo scarabeo ha 12 cariche. Se fallisci un Tiro Salvezza contro un incantesimo di necromanzia o un effetto nocivo originante da una creatura non morta, puoi usare una Azione di Reazione per spendere 1 carica e trasformare il Tiro Salvezza fallito in un successo. Lo scarabeo si riduce in polvere ed è distrutto quando viene spesa la sua ultima carica.

\index[OggettiMagici]{Spilla della Difesa}\subsubsection*{Spilla della Difesa}

7500 mo, non comune, la spilla può assorbire 101 danni da incantesimi di Forza, poi perde le sue proprietà magiche.

\index[OggettiMagici]{Talismano del Bene puro}\subsubsection*{Talismano del Bene puro}
50000 mo, leggendario, un Devoto di Gradh o Sumkjr in possesso di questo oggetto può far sì che una voragine di fiamme appaia ai piedi di un Devoto di Calicante o Shayalia entro 30 m. La vittima viene inghiottita dal fuoco e precipita urlando verso il centro della Terra. Un talismano del bene puro ha 6 cariche e non può essere ricaricato. Se un Devoto di Calicante o Shayalia lo tocca subisce 6d6 ferite. Qualsiasi altro Devoto o Seguace non subisce alcun effetto. Il Talismano pulsa di luce entro 36 metri da un Devoto o Seguace di Calicante o Shayalia.

\index[OggettiMagici]{Talismano del Male estremo}\subsubsection*{Talismano del Male estremo}
50000 mo, leggendario, questo talismano funziona esattamente come il talismano del bene puro ma con i Patroni invertiti.

\index[OggettiMagici]{Talismano di Protezione dal Veleno}\subsubsection*{Talismano della Protezione dal Veleno}
5000 mo, raro, mentre indossi questo pendente i veleni non hanno alcun effetto su di te. Sei immune alla condizione avvelenato e hai immunità ai danni da veleno.

\index[OggettiMagici]{Talismano della Salute}\subsubsection*{Talismano della Salute}
5000 mo, raro, mentre indossi questo pendente sei immune alla possibilità di contrarre qualsiasi malattia. Se sei già infetto da una malattia, i suoi effetti vengono sospesi finché indossi questo pendente.

\index[OggettiMagici]{Talismano della Sfera}\subsubsection*{Talismano della Sfera}
75000 mo, leggendario, quando effettui una prova di Arcana per controllare una sfera dell'annientamento mentre stai impugnando questo talismano hai un bonus di 5. Inoltre, quando inizi il round con il controllo di una sfera dell'annientamento, puoi usare due azioni per farla levitare di 3 metri più un numero di metri aggiuntivi pari a 3 x il tuo valore di Intelligenza.

\subsection{Cinture, Elmi, Stivali e Guanti}

\index[OggettiMagici]{Cintura dei Giganti}\subsubsection*{Cintura dei Giganti}

10000/15000/20000/30000/45000 mo, rarità varia, mentre indossi questa cinta, il tuo punteggio raggiunge il punteggio conferito dalla cinta. Se il tuo punteggio di Forza è già pari o superiore al punteggio della cinta, l'oggetto non ha effetto su di te.

Esistono quattro varianti di questa cinta, corrispondenti ciascuna a una specie di veri giganti. La cinta del gigante di pietra e la cinta del gigante del gelo appaiono diverse, ma hanno lo stesso effetto.

\medskip

\begin{tabular}{lll}
\textbf{Tipo di Gigante}& \textbf{Forza} &\textbf{Rarità}\\
\hline
\textbf{Collina} &5& Raro\\
\textbf{Pietra/del Gelo}& 6 &Molto raro\\
\textbf{Fuoco} &7& Molto raro\\
\textbf{Nuvole} &8& Leggendario\\
\textbf{Tempeste}& 9& Leggendario\\
\end{tabular}

\index[OggettiMagici]{Cintura dei Nani}\subsubsection*{Cintura dei Nani}
86000 mo, raro, mentre indossi questa cinta, ottieni i seguenti benefici:

- il tuo punteggio di Costituzione aumenta di 1, fino a un massimo di 5.

- hai +2 alle prove di Carisma effettuate per interagire con i nani.

Inoltre, mentre sei indossi la cintura hai il 50\% di probabilità ogni giorno all'alba di vederti spuntare una folta barba, se può crescerti, oppure di vedere la tua ancora più folta, se già la hai.

Se non sei un nano, ottieni i seguenti benefici aggiuntivi quando indossi questa cintura:

- hai +2 ai Tiri Salvezza contro veleno e hai resistenza ai danni da veleno. Hai scurovisione con una gittata di 18 metri. Puoi parlare, leggere e scrivere in Nanico.

\index[OggettiMagici]{Elmo della Comprensione dei Linguaggi}\subsubsection*{Elmo della Comprensione dei Linguaggi}
600 mo, comune, mentre indossi questo elmo, puoi usare due azioni per lanciare a volontà tramite esso l'incantesimo comprendere linguaggi.

\index[OggettiMagici]{Elmo della Lucentezza}\subsubsection*{Elmo della Lucentezza}
75000 mo, leggendario, questo elmo luminoso è incastonato con 1d10 diamanti, 2d10 rubini, 3d10 opali di fuoco e 4d10 opali. Qualsiasi gemma estratta dall'elmo si riduce in polvere. Quando tutte le gemme sono rimosse o distrutte, l'elmo perde la sua magia. Mentre lo indossi ottieni i seguenti benefici:

\medskip

\begin{itemize}
\item
Puoi usare due azioni per lanciare uno dei seguenti incantesimi, usando una delle gemme dell'elmo del tipo specificato come componente: luce diurna (opale), muro di fuoco (rubino), palla di fuoco (opale di fuoco) o spruzzo prismatico (diamante). Quando l'incantesimo viene lanciato la gemma è distrutta e scompare dall'elmo.

\item
Finché possiede almeno un diamante, l'elmo emette luce in un raggio di 9 metri quando almeno un non morto si trova entro quest'area. Qualsiasi non morto che inizi il suo round all'interno dell'area subisce 1d6 danni da Luce.

\item
Finché l'elmo possiede almeno un rubino, hai resistenza ai danni da fuoco.
\end{itemize}

\medskip

Finché l'elmo possiede almeno un opale di fuoco, puoi usare due azioni e pronunciare una parola di comando per far sì che un'arma che stai impugnando venga avvolta dalle fiamme. Le fiamme emettono luce in un raggio di 3 metri e luce fioca per ulteriori 3 metri. Le fiamme sono innocue per te e per l'arma. Quando colpisci con un attacco sferrato con l'arma infiammata, il bersaglio subisce 1d6 danni da fuoco aggiuntivi. Le fiamme perdurano fino a quando non userai due azioni per pronunciare la parola di comando di nuovo o fino a quando non lascerai cadere o rinfodererai l'arma.

Se stai indossando l'elmo e subisci danni da fuoco in seguito al fallimento critico di un Tiro Salvezza contro un incantesimo, l'elmo emette un fascio di luce tramite le gemme rimanenti. Ogni creatura entro 18 metri dall'elmo, a parte te, deve superare un Tiro Salvezza su Riflessi con DC 21 o venire colpita dal fascio, subendo danni di Luce uguali al numero di gemme nell'elmo x 5. Poi, le gemme e l'elmo vengono distrutti.

\index[OggettiMagici]{Elmo del Movimento subacqueo}\subsubsection*{Elmo del Movimento subacqueo}
4000 mo, raro, questo elmo, solitamente in pelle di pesce, conferisce la capacità di respirare sott'acqua, movimento Nuotare 18 metri, ecolocalizzazione 18 metri. Il potere è usabile per 6 ore al giorno e si ricarica all'alba.

\index[OggettiMagici]{Elmo della Telepatia}\subsubsection*{Elmo della Telepatia}
12000 mo, raro, mentre indossi questo elmo, puoi usare due azioni per lanciare tramite esso l'incantesimo individuazione dei pensieri (DC del Tiro Salvezza 13). Finché mantieni la concentrazione sull'incantesimo, puoi usare due azioni per inviare un messaggio telepatico alla creatura su cui sei concentrato. Essa può replicare (usando due azioni per farlo) fino a quando continui a concentrarti su di lei.

Mentre ti concentri su di una creatura con individuazione dei pensieri, puoi usare due azioni per lanciare tramite l'elmo l'incantesimo suggestione (DC del Tiro Salvezza 13) su quella creatura. Una volta usata, la proprietà suggestione non potrà essere usata di nuovo fino alla prossima alba.

\index[OggettiMagici]{Elmo del Teletrasporto}\subsubsection*{Elmo del Teletrasporto}
64000 mo, raro, mentre indossi questo elmo, puoi usare due azioni e spendere 1 carica per lanciare l'incantesimo teletrasporto tramite esso. L'elmo ha 3 cariche, e ne recupera 1 ogni mattina all'alba.

\index[OggettiMagici]{Guanti Afferra Proiettili}\subsubsection*{Guanti Afferra Proiettili}
3000 mo, non comune, questi quanti sembrano quasi fondersi con la tua pelle quando li indossi. Quando un attacco con arma a distanza ti colpisce mentre li indossi, puoi usare una Azione di Reazione per ridurre il danno di 1d10 + Destrezza, purché tu abbia una mano libera. Se riduci il danno a 0 ed il proiettile è piccolo a sufficienza da essere tenuto in mano, puoi afferrarlo.

\index[OggettiMagici]{Guanti del Potere orchesco}\subsubsection*{Guanti del Potere orchesco}
9000 mo, raro, mentre indossi queste manopole la tua Forza è 4. I guanti non hanno effetto se la tua Forza è già 4 o più.

\index[OggettiMagici]{Guanti del Nuoto e della Scalata}\subsubsection*{Guanti del Nuoto e della Scalata}
2000 mo, non comune, mentre indossi entrambi questi guanti, la scalata e il nuoto non ti costano movimento aggiuntivo. Inoltre, hai un bonus di +1d6 alle prove di Costituzione e Saggezza effettuate mentre scali o nuoti.

\index[OggettiMagici]{Guanti della Destrezza}\subsubsection*{Guanti della Destrezza}
12000 mo, rari, questi guanti impartiscono al possessore una Destrezza minima di +2 e nel caso abbia già un punteggio di +2 questa aumenta di 1 (fino ad un massimo +4). Inoltre il possessore acquisisce +1d6 nella Competenza Mani di Fata

\index[OggettiMagici]{Guanti Maldestri}\subsubsection*{Guanti Maldestri}
questi guanti possono essere di morbido cuoio o pesante materiale protettivo adatto per l’uso con armature. Nel primo caso sembrano essere guanti della destrezza. Nel secondo caso essi sembrano essere guanti del potere orchesco. Ad ogni prova i guanti sembrano avere le funzioni di cui sopra fino a quando chi li indossa non è sotto attacco o in una situazione di vita o di morte. In quel momento la maledizione si attiva. Il personaggio diviene maldestro, con una probabilità del 50\% ad ogni round di lasciar cadere un oggetto che tiene nelle mani. I guanti riducono la Destrezza di 2 punti. Una volta che la maledizione è attiva, i guanti possono essere rimossi soltanto con un incantesimo Rimuovi Maledizione od un desiderio.

\index[OggettiMagici]{Pantofole del Ragno}\subsubsection*{Pantofole del Ragno}
5000 mo, non comune, mentre indossi queste scarpe leggere, puoi muoverti verso l'alto, il basso, e lungo superfici verticali e a testa in giù sul soffitto, lasciando libere le mani. Hai una velocità di scalata pari alla velocità di passeggio. Tuttavia, le pantofole non ti permettono di muoverti a questo modo su terreno difficile, come pareti coperte da ghiaccio, da olio, macerie...

\index[OggettiMagici]{Stivali Alati}\subsubsection*{Stivali Alati}
15000 mo, raro, mentre indossi questi stivali, hai una velocità di volo pari alla tua velocità di passeggio. Puoi usare questi stivali per volare per un massimo di 4 ore, tutte insieme o divise in brevi voli, ciascuno dei quali impiega un minimo di 1 minuto di durata. Se la durata termina mentre stai volando, scendi alla velocità di 9 metri per round finché non atterri. Gli stivali recuperano 2 ore di capacità di volo ogni alba.

\index[OggettiMagici]{Stivali della Corsa e del Salto}\subsubsection*{Stivali della Corsa e del Salto}
5000 mo, non comune, mentre indossi questi stivali, la tua velocità di passeggio diventa 9 metri, a meno che non sia superiore, e la tua velocità non viene ridotta qualora tu sia ingombrato o stia indossando un'armatura pesante. Inoltre, salti tre volte la normale distanza, fino ad un massimo di 9 metri.

\index[OggettiMagici]{Stivali degli Elfi}\subsubsection*{Stivali degli Elfi}
3000 mo, non comune, mentre indossi questi stivali, i tuoi passi non emettono suoni, quale che sia la superficie che stai attraversando. Hai +1d6 alle prove di Muoversi silenziosamente.

\index[OggettiMagici]{Stivali dell'Inverno}\subsubsection*{Stivali dell'Inverno}
10000 mo, rari, mentre indossi questi stivali hai resistenza ai danni da freddo, ignori il terreno difficile prodotto da neve o ghiaccio. Puoi tollerare le temperature fino ai -45° C senza bisogno di ulteriori protezioni. Se indossi abiti pesanti, puoi tollerare temperature fino a -75° C.

\index[OggettiMagici]{Stivali della Levitazione}\subsubsection*{Stivali della Levitazione}
5000 mo, raro, mentre indossi questi stivali, puoi usare a volontà due azioni per lanciare l'incantesimo levitazione su di te.

\index[OggettiMagici]{Stivali della Velocità}\subsubsection*{Stivali della Velocità}
5000 mo, raro, mentre indossi questi stivali, puoi usare una Azione bonus da usare solo per muoversi. Puoi terminare l'effetto quando vuoi. L'effetto dura fino a quando terminato, per un massimo di 10 minuti al giorno. La capacità si ricarica all'alba.

\index[OggettiMagici]{Stivali Danzanti}\subsubsection*{Stivali Danzanti}
questi stivali maledetti funzionano come altri stivali magici. Tuttavia, quando il personaggio entra in combattimento o tenta di fuggire da un potenziale combattimento, egli viene affetto da un incantesimo danza irresistibile, senza possibilità di Tiro Salvezza. E' possibile rimuovere gli Stivali Danzanti con l'incantesimo Rimuovi Maledizione o Desiderio.

\subsection{Bacchette, Verghe e Bastoni}

\index[OggettiMagici]{Bacchetta Cerca metalli}\subsubsection*{Bacchetta Cerca metalli}
500 mo, non comune, quando viene spesa una carica, la bacchetta punta nella direzione di qualunque massa metallica di almeno 100 kg che si trovi entro 6 metri. Chi impugna la bacchetta ha una percezione intuitiva del tipo di metallo individuato.

\index[OggettiMagici]{Bacchetta dei Dardi Incantati}\subsubsection*{Bacchetta dei Dardi Incantati}
8000 mo, raro, mentre impugni questa bacchetta, puoi usare due azioni per spendere 1 o più delle sue cariche per lanciare tramite essa un dardo incantato, come l'omonimo incantesimo. Ogni carica genera 1 dardo. La bacchetta ha 7 cariche. La bacchetta recupera 1d3+ 1 cariche spese all'alba di ciascun giorno. Tuttavia, se spendi l'ultima carica della bacchetta, tira 1d6 se ottieni 1 la bacchetta si riduce in polvere ed è distrutta.

\index[OggettiMagici]{Bacchetta delle Comodità}\subsubsection*{Bacchetta delle Comodità}
300 mo, comune, Il possessore può spendere 1 carica per lanciare gli incantesimi servitore invisibile o cuoco invisibile o disco fluttuante. La bacchetta ha 7 cariche che sono recuperate all'alba.

\index[OggettiMagici]{Bacchetta dei Fulmini}\subsubsection*{Bacchetta dei Fulmini}
32000 mo, raro, mentre impugni questa bacchetta, puoi usare due azioni per spendere 1 carica lanciare tramite essa l'incantesimo fulmine (DC del Tiro Salvezza 18).
Questa bacchetta ha 7 cariche. La bacchetta recupera 1d3 + 1 cariche spese all'alba di ciascun giorno. Tuttavia, se spendi l'ultima carica della bacchetta, tira 1d6 se ottieni 1 la bacchetta si riduce in polvere ed è distrutta.

\index[OggettiMagici]{Bacchetta del Fuoco}\subsubsection*{Bacchetta del Fuoco}
18000 mo, molto raro, una bacchetta del fuoco produce diversi incantesimi e consuma 1 carica + livello dell'incantesimo manifestato. Gli incantesimi manifestabili sono: mani brucianti, piroesperto, palla di fuoco, muro di fuoco. Finché la bacchetta è tenuta in mano ogni 1 sui dadi per danno da fuoco che infligge viene considerato come 2. La bacchetta ha 7 cariche e ne recupera 1 all'alba.

\index[OggettiMagici]{Bacchetta del Ghiaccio}\subsubsection*{Bacchetta del Ghiaccio}
15000 mo, molto raro, una bacchetta del fuoco produce diversi incantesimi e consuma 1 carica + livello dell'incantesimo manifestato. Gli incantesimi manifestabili sono: raggio di gelo, tempesta di nevischio, tempesta di ghiaccio, cono di freddo. Finché la bacchetta è tenuta in mano ogni 1 sui dadi per danno da freddo che infligge viene considerato come 2. La bacchetta ha 7 cariche e ne recupera 1 all'alba.

\index[OggettiMagici]{Bacchetta di Individuazione del Magico}\subsubsection*{Bacchetta dell'Individuazione del Magico}
1500 mo, non comune, mentre impugni questa bacchetta, con due azioni puoi spendere 1 carica per lanciare tramite essa l'incantesimo individuazione del magico. Questa bacchetta ha 7 cariche, e recupera 1d3 cariche spese ogni mattina all'alba.

\index[OggettiMagici]{Bacchetta di Individuazione dei Nemici}\subsubsection*{Bacchetta del l'Individuazione dei Nemici}
4000 mo, raro, mentre impugni questa bacchetta, puoi usare due azioni e spendere 1 carica per pronunciarne la parola di comando. Per il minuto successivo, conosci in che direzione si trovi la creatura ostile più vicina entro 18 metri da te, ma non la distanza che vi separa. La bacchetta può percepire la presenza di creature ostili che siano eteree, invisibili, camuffate, o nascoste, oltre che di quelle in piena vista. L'effetto termina se smetti di impugnare la bacchetta. Questa bacchetta ha 7 cariche. La bacchetta recupera 2 cariche spese all'alba di ciascun giorno. Tuttavia, se spendi l'ultima carica della bacchetta, tira 1d6 se ottieni 1 la bacchetta si riduce in polvere ed è distrutta.

\index[OggettiMagici]{Bacchetta dell'Illusioni}\subsubsection*{Bacchetta dell'Illusioni}
3000 mo, raro, chi impugna questa bacchetta può lanciare Immagine Maggiore (3), Immagine Silenziosa (1), Immagine Speculare (2). Ogni incantesimo costo un numero di cariche pari al livello +1. Mentre si concentra sull’effetto, il personaggio può solo muoversi a velocità dimezzata. Se viene colpito deve riuscire con successo in una Prova di Magia o l’illusione svanisce immediatamente.

\index[OggettiMagici]{Bacchetta dell'Individuazione delle porte segrete}\subsubsection*{Bacchetta dell'Individuazione delle porte segrete}
300 mo, non comune, questa bacchetta punta al passaggio segreto più vicino entro 6 metri. L'effetto consuma una carica delle 7 disponibili, ogni giorno all'alba si recuperano tutte le cariche.

\index[OggettiMagici]{Bacchetta della Luce}\subsubsection*{Bacchetta della Luce}
3500 mo, raro, una bacchetta della luce manifesta diversi incantesimi e consuma 1 carica + livello dell'incantesimo manifestato. Gli incantesimi manifestabili sono: luci danzanti, luce, fiamma perenne, luce diurna. Infine, spendendo 5 cariche, il possessore può creare un raggio di intensa luce solare. La luce di colore giallo-dorato intenso, ha una gittata di 36 m, e forma una sfera di luce del diametro di 12 m. Chiunque si trovi nell'area di effetto viene accecato e stordito per 1 round se fallisce un Tiro Salvezza su Tempra DC 17. La sfera dorata ha un effetto devastante sui non morti, infliggendo 6d6 ferite da Luce senza possibilità di Tiro Salvezza. Questa bacchetta ha 7 cariche. La bacchetta recupera 2 cariche spese all'alba di ciascun giorno. Tuttavia, se spendi l'ultima carica della bacchetta, tira 1d6 se ottieni 1 la bacchetta si riduce in polvere ed è distrutta.

\index[OggettiMagici]{Bacchetta del Mago da Guerra}\subsubsection*{Bacchetta del Mago da Guerra}
1500/5500/25000 mo, non comune (+1), raro (+2), o molto raro (+3), mentre impugni questa bacchetta, ottieni un bonus ai Tiri per Colpire con gli incantesimi determinato dalla rarità della bacchetta. Inoltre, ignori la copertura leggera quando effettui un attacco con incantesimo.

\index[OggettiMagici]{Bacchetta della Metamorfosi}\subsubsection*{Bacchetta della Metamorfosi}
32000 mo, molto raro, mentre impugni questa bacchetta, puoi usare due azioni per spendere 1 carica per lanciare tramite essa l'incantesimo metamorfosi (DC 18, Tiro Salvezza su Volontà). Questa bacchetta ha 3 cariche. La bacchetta recupera 1 carica spesa all'alba di ciascun giorno. Tuttavia, se spendi l'ultima carica della bacchetta, tira 1d6 se ottieni 1, la bacchetta si riduce in polvere ed è distrutta.

\index[OggettiMagici]{Bacchetta delle Meraviglie}\subsubsection*{Bacchetta delle Meraviglie}
25000 mo, molto raro, mentre impugni questa bacchetta, puoi spendere 1 carica con due azioni e scegliere un bersaglio entro 36 metri da te. Il bersaglio può essere una creatura, un oggetto o un punto nello spazio. Il Narratore decide o determina casualmente cosa accadrà quando fai uso della bacchetta. Gli incantesimi lanciati tramite la bacchetta hanno DC del Tiro Salvezza 18. Se l'incantesimo normalmente ha una gittata espressa in metri, la gittata diventa 36 metri qualora non lo sia già. Se un effetto copre un'area, devi centrare l'incantesimo sul bersaglio e includervelo. Se un effetto più agire su più soggetti possibili, il Narratore determina casualmente chi sia affetto.

Questa bacchetta ha 7 cariche. La bacchetta recupera 1 carica ogni giorno all'alba. Se spendi l'ultima carica della bacchetta, tira 1d6 se ottieni 1 la bacchetta si riduce in polvere ed è distrutta.

Ogni volta che fai uso della bacchetta delle meraviglie tira un d100 e consulta questa tabella.

\end{multicols}

\begin{center}
\includegraphics[width=0.3\linewidth]{immagini/bacchette.png}

\end{center}

\medskip

\begin{tabularx}{0.95\textwidth}{lX}
\textbf{d100}& \textbf{Contenuti}\\
\hline
01-05 &Lanci lentezza.\\
06-10 &Lanci fuoco delle fate.\\
11-15 &Sei stordito fino all'inizio del tuo prossimo round, e ritieni che sia accaduto qualcosa di stupefacente.\\
16-20 &Lanci folata di vento.\\
21-25 &Lanci individuazione dei pensieri sul bersaglio da te scelto. Se il tuo a bersaglio non è una creatura, subisci invece 1d6 danni.\\
26-30 &Lanci nube maleodorante.\\
31-33 &Pioggia abbondante precipita in un raggio di 18 metri centrato sul bersaglio. L'area diventa oscurata leggermente. La pioggia continua a cadere fino all'inizio del tuo prossimo round.\\
34-36 &Compare un animale nello spazio non occupato più vicino al bersaglio. L'animale non è sotto il tuo controllo e agisce come di norma. Tira un d100 per determinare che specie di animale compaia.01-25, un rinoceronte; 26-50, un elefante; 51-100, un ratto.\\
37-46 &Lanci fulmine.\\
47-49 &Una nube di 600 enormi farfalle riempie un raggio di 9 metri intorno al bersaglio. L'area diventa oscurata pesantemente. Le farfalle restano per 10 minuti.\\
50-53 &Ingrandisci il bersaglio come se avessi lanciato l'incantesimo ingrandire/ridurre. Se il bersaglio non può essere soggetto all'incantesimo, o se non è una creatura, tu diventi il bersaglio.\\
54-58 &Lanci oscurità.\\
59-62 &Erba folta spunta in un raggio di 18 metri intorno al bersaglio.Se vi è già dell'erba, questa cresce di dieci volte e resta così per 1 minuto.\\
63-65 &Un oggetto a scelta del Narratore scompare sul Piano Etereo.L'oggetto non deve essere né indossato né trasportato, deve essere entro 36 metri dal bersaglio, e non più grande di 3 metri in ciascuna dimensione.\\
66-69 &Ti rimpicciolisci come se avessi lanciato su di te l'incantesimo ingrandire/ridurre.\\
70-79 &Lanci palla di fuoco.\\
80-84 &Lanci invisibilità su di te.\\
85-87 &Sul bersaglio crescono delle foglie. Se hai scelto un punto nello spazio come bersaglio, le foglie spunteranno sulla creatura più vicina a quel punto. A meno che non vengano strappate, le foglie diventeranno marroni e cadranno dopo 24 ore.\\
88-90& Un flusso di 1d4 x 10 gemme del valore di 1 mo ciascuna scaturisce dalla punta della bacchetta in una linea lunga 9 metri e larga 1,5 metri. Ogni gemma infligge 1 danno contundente, e il loro danno totale è diviso equamente tra tutte le creature sulla linea.\\
91-95 &Una raffica di luci scintillanti e colorate si estende da te in un raggio di 9 metri. Tu e tutte le creature nell'area dovete superare un Tiro Salvezza su Tempra con DC 15 o restare accecati per 1 minuto. Una creatura può ripetere il Tiro Salvezza al termine di ciascun suo round, terminando l'effetto su di sé in caso lo superi.\\
96-97 &La pelle del bersaglio assume un colorito blu intenso per 1d10 giorni. Se hai scelto un punto nello spazio, il soggetto sarà la creatura più vicina a quel punto.\\
98-00 &Se il bersaglio è una creatura, deve effettuare un Tiro Salvezza di Tempra con DC 18. Se il bersaglio non è una creatura, il bersaglio diventi tu e sarai tu a effettuare il Tiro Salvezza. Se il Tiro Salvezza fallisce di 5 o più, il bersaglio è pietrificato. Se il Tiro Salvezza fallisce di meno, il bersaglio è intralciato e inizia a trasformarsi in pietra. Mentre è intralciato a questo modo, il bersaglio deve ripetere il Tiro Salvezza al termine di ciascun suo round, diventando pietrificato in caso di fallimento o terminando l'effetto in caso di successo. Il bersaglio resta pietrificato finché non sarà liberato dall'incantesimo pietra in carne o simili magie.\\
\end{tabularx}

\medskip

\begin{multicols}{2}

\index[OggettiMagici]{Bacchetta della Negazione}\subsubsection*{Bacchetta della Negazione}
35000 mo, molto raro, questa bacchetta nega gli incantesimi o effetti simili prodotti da oggetti magici. Il possessore punta la bacchetta verso l'oggetto entro 36 metri, ed essa emana un raggio di colore grigio chiaro che colpisce il bersaglio. Il raggio annulla automaticamente le manifestazione di incantesimi o effetti simili di livello 3 o meno. Ciascun uso della bacchetta costa 1 carica ed essa può essere usata solo una volta per round. Questa bacchetta ha 3 cariche. La bacchetta recupera 1 carica ogni giorno all'alba. Se spendi l'ultima carica della bacchetta, tira 1d6 se ottieni 1 la bacchetta si riduce in polvere ed è distrutta.

\index[OggettiMagici]{Bacchetta delle Palle di Fuoco}\subsubsection*{Bacchetta delle Palle di Fuoco}
32000 mo, raro, mentre impugni questa bacchetta, puoi usare due azioni per spendere 1 carica per lanciare tramite essa l'incantesimo palla di fuoco (DC del Tiro Salvezza 18). Questa bacchetta ha 7 cariche. La bacchetta recupera 1 carica spesa all'alba di ciascun giorno. Tuttavia, se spendi l'ultima carica della bacchetta, tira 1d6 se ottieni 1 la bacchetta si riduce in polvere ed è distrutta.

\index[OggettiMagici]{Bacchetta della Paralisi}\subsubsection*{Bacchetta della Paralisi}
16000 mo, raro, mentre impugni questa bacchetta, puoi usare due azioni per spendere 1 carica per far sì che un sottile raggio parta dalla sua punta verso una creatura visibile entro 18 metri da te. Il bersaglio deve superare un tiro Salvezza su Tempra con DC 17 o restare paralizzato per 1 minuto. Al termine di ciascun round del bersaglio, questi può effettuare un tiro Salvezza su Tempra DC 15, terminando l'effetto su di sé in caso lo superi. Questa bacchetta ha 7 cariche. La bacchetta recupera 1 carica spese all'alba di ciascun giorno. Tuttavia, se spendi l'ultima carica della bacchetta, tira 1d6 se ottieni 1 la bacchetta si riduce in polvere ed è distrutta.

\index[OggettiMagici]{Bacchetta della Paura}\subsubsection*{Bacchetta della Paura}
13000 mo, raro, questa bacchetta ha 7 cariche per le seguenti proprietà. La bacchetta recupera 1 carica spese all'alba di ciascun giorno. Tuttavia, se spendi l'ultima carica della bacchetta, tira 1 se ottieni 1 la bacchetta si riduce in polvere ed è distrutta.

\textbf{Comando} mentre impugni questa bacchetta, puoi usare due azioni per spendere 1 carica e comandare a un'altra creatura di scappare o strisciare, come per l'incantesimo comando (DC del Tiro Salvezza 18)

\textbf{Cono di Paura} mentre impugni questa bacchetta, puoi usare due azioni per spendere 2 cariche, facendo sì che la punta della bacchetta emetta luce in un cono di 18 metri. Ogni creatura nel cono deve superare un Tiro Salvezza su Volontà con DC 18 o restare spaventata da te per 1 minuto. Mentre è spaventata in questo modo, una creatura deve spendere i suoi turni cercando di muoversi più lontano possibile da te, e non può muoversi volontariamente entro 9 metri da te.

Inoltre non può effettuare reazioni. Come sua azione, la creatura può usare solo l'azione Scattare o cercare di liberarsi da un effetto che le impedisca di muoversi. Se non può muoversi da nessuna parte, la creatura può usare l'azione Difesa Totale. Al termine di ciascun suo round, la creatura può ripetere il Tiro Salvezza, terminando l'effetto su di sé in caso lo superi. Questa bacchetta ha 7 cariche. La bacchetta recupera 1 carica spese all'alba di ciascun giorno. Tuttavia, se spendi l'ultima carica della bacchetta, tira 1d6 se ottieni 1 la bacchetta si riduce in polvere ed è distrutta.

\index[OggettiMagici]{Bacchetta Scopri trappole}\subsubsection*{Bacchetta Scopri trappole}
400 mo, non comune, questa bacchetta punta alla trappola più vicina entro 6 m. L'effetto consuma una carica. Questa bacchetta ha 7 cariche. La bacchetta recupera tutte le cariche spese all'alba di ciascun giorno.

\index[OggettiMagici]{Bacchetta dei Segreti}\subsubsection*{Bacchetta dei Segreti}
500 mo, non comune, mentre impugni questa bacchetta, puoi usare due azioni per spendere 1 carica e rilevare se porta segreta o trappola si trova entro 9 metri da te, la bacchetta pulsa e punta a quella più vicina a te. La bacchetta ha 3 cariche. La bacchetta recupera tutte le cariche spese all'alba di ciascun giorno.


\index[OggettiMagici]{Bacchetta della Ragnatela}\subsubsection*{Bacchetta della Ragnatela}
8000 mo, non comune, mentre la impugni, puoi usare due azioni per spendere 1 carica per lanciare tramite essa l'incantesimo ragnatela (DC del Tiro Salvezza 18). Questa bacchetta ha 7 cariche. La bacchetta recupera 1 carica spesa all'alba di ciascun giorno. Tuttavia, se spendi l'ultima carica della bacchetta, tira 1d6 se ottieni 1 la bacchetta si riduce in polvere ed è distrutta.

\index[OggettiMagici]{Bacchetta del Vincolo}\subsubsection*{Bacchetta del Vincolo}
10000 mo, raro, questa bacchetta ha 7 cariche per le seguenti proprietà. La bacchetta recupera 1 carica spese all'alba di ciascun giorno. Tuttavia, se spendi l'ultima carica della bacchetta, tira 1d6 se ottieni 1 la bacchetta si riduce in polvere ed è distrutta. Mentre impugni questa bacchetta, puoi usare due azioni e spendere alcune delle sue cariche per lanciare uno dei seguenti incantesimi (DC del Tiro Salvezza 21):

\textbf{blocca mostri} (5 cariche) o \textbf{blocca persone} (2 cariche).

\index[OggettiMagici]{Bacchetta della Fuga Assistita}\subsubsection*{Bacchetta della Fuga Assistita}
2000 mo, raro, mentre impugni questa bacchetta, puoi usare la Azione di Reazione e spendere 1 carica per ottenere +1d6 ai Tiri Salvezza che effettui per evitare di restare paralizzato o intralciato, o puoi spendere 1 carica per ottenere +1d6 su qualsiasi prova effettuata per sfuggire un tentativo di afferrare.

\index[OggettiMagici]{Bastone dell'Arcimago}\subsubsection*{Bastone dell'Arcimago}
125000 mo, leggendario, il bastone dell'arcimago è una versione molto potente del bastone della stregoneria. Esso mette a disposizione del possessore diversi incantesimi. Il bastone può essere usato per manifestare incantesimi: serratura arcana, individuazione del magico, ingrandire/ridurre e luce. Queste capacità non richiedono il consumo di cariche. In aggiunta, il bastone possiede le seguenti capacità che costano 1 carica per uso: dissolvi magie, fulmine, invisibilità, muro di fuoco, palla di fuoco, passapareti, piroesperto, ragnatela, scassinare e tempesta di ghiaccio. Le seguenti, potenti capacità costano 2 cariche per uso: evoca elementale, spostamento planare, telecinesi. Il possessore del bastone riceve un bonus +2 ai tiri salvezza contro incantesimi. Il bastone può essere ricaricato, ma soltanto assorbendo le energie magiche lanciate contro il possessore, il quale può assorbirle in quantità pari ad 1 carica per livello dell'incantesimo. Questa operazione è la sola azione possibile in un round, ed il bastone non può essere usato per altri effetti nello stesso round in cui esso assorbe energia. Ciascun bastone ha un numero massimo di cariche possibili, ed esso assorbirà cariche solo fino al suo limite senza incorrere in effetti deleteri. Il possessore non ha modo di conoscere tale limite, o quante cariche sono state usate, a meno di non usare qualche metodo magico. Se il bastone assorbe energia in eccesso, esso esplode come nel caso di un colpo definitivo, descritto di seguito. Un bastone dell'arcimago può essere usato per un colpo definitivo, il che richiede che esso venga spezzato dal suo possessore. La rottura non deve essere accidentale e deve essere dichiarata. Tutte le cariche immagazzinate nel bastone vengono rilasciate istantaneamente nel raggio di 9 m. Tutte le creature entro 3 m subiscono ferite pari a 10 volte il numero di cariche nel bastone; tra i 3 m ed i 6 m le ferite sono 6 volte il numero di cariche; e tra i 6 m ed i 9 m le ferite sono 4 volte il numero di cariche. Un Tiro Salvezza su Tempra a DC 25 riduce il danno a metà. Il personaggio che spezza il bastone ha il 50\% di probabilità di andare su un altro piano di esistenza, altrimenti il rilascio esplosivo di energia magica lo distrugge. Quando tutte le cariche sono state consumate, il bastone diviene un bastone +2. Se le cariche sono esaurite non può essere usato per un colpo definitivo.

\index[OggettiMagici]{Bastone di Avvizzimento}\subsubsection*{Bastone dell'Avvizzimento}
3000 mo, raro, il bastone può essere impugnato come un bastone da combattimento magico. Se colpisci, infligge danni come un normale bastone da combattimento, e puoi spendere 1 carica per infliggere 2d10 danni da Vuoto aggiuntivi al bersaglio. Inoltre, il bersaglio deve superare un Tiro Salvezza su Tempra con DC 18 o avere -1d6 per 1 ora a qualsiasi prova di caratteristica o Tiro Salvezza che richieda Costituzione Questo bastone ha 3 cariche e recupera 1d3 cariche spese a mezzanotte.

\index[OggettiMagici]{Bastone dei Boschi}\subsubsection*{Bastone dei Boschi}
44000 mo, raro, il bastone può essere impugnato come un bastone da combattimento magico che conferisce un bonus di +2 ai Tiri per Colpire e danno effettuati con esso. Quando lo impugni hai anche un bonus di +2 ai Tiri per Colpire con incantesimi.
Questo bastone ha 10 cariche per le seguenti proprietà. Recupera 2 cariche spese ogni giorno all'alba. Se spendi l'ultima carica del bastone, tira 1d6 se ottieni 1 il bastone si annerisce, si trasforma in cenere, ed è distrutto.

- \textit{Incantesimi}. Puoi usare due azioni per spendere 1 o più cariche del bastone per lanciare tramite esso uno dei seguenti incantesimi, utilizzando la tua DC del Tiro Salvezza degli incantesimi: amicizia con gli animali (1 carica), localizza animali e piante (1 carica), muro di spine (6 cariche), parlare con gli animali (3 cariche), pelle coriacea (2 cariche) o risveglio (5 cariche). Puoi inoltre usare due azioni per lanciare tramite il bastone l'incantesimo passare senza tracce senza
spendere cariche.

- \textit{Forma d'Albero}. Puoi usare due azioni per piantare un'estremità del bastone su terreno fertile e spendere 1 carica per trasformare il bastone in un albero da frutto vigoroso. L'albero è alto 18 metri, con un tronco di 1,5 metri di diametro; in cima i suoi rami si estendono per 6 metri. L'albero sembra un albero normale ma irradia una debole aura di magia di trasmutazione, qualora sia bersaglio dell'incantesimo individuazione del magico. Mentre sei in contatto con l'albero e usi un'altra azione per pronunciarne la parola di comando, riporti il bastone alla sua forma normale. Qualsiasi creatura sull'albero, cade quando questo si ritrasforma in bastone.

\index[OggettiMagici]{Bastone dello Charme}\subsubsection*{Bastone dello Charme}
12000 mo, raro, mentre impugni questo bastone, puoi usare due azioni per spendere 1 carica per lanciare tramite esso charme su persone, comando o comprendere linguaggi, utilizzando la tua DC dei Tiri Salvezza degli incantesimi. Il bastone può essere usato come bastone da combattimento magico.

Se stai impugnando il bastone e fallisci un Tiro Salvezza contro un incantesimo di ammaliamento che prende come bersaglio solo te e non un'area, puoi trasformare il Tiro Salvezza fallito in un successo. Non potrai più usare questa proprietà del bastone fino all'alba del giorno successivo.

Se riesci in un Tiro Salvezza contro un incantesimo di ammaliamento che prende come bersaglio solo te, con o senza l'intervento del bastone, puoi usare una Azione di Reazione per spendere 3 carica dal bastone e rivolgere l'incantesimo contro chi lo ha lanciato, come se l'incantesimo fosse stato lanciato da te.

Il bastone ha 7 cariche, e recupera 1 carica spesa ogni giorno all'alba. Se spendi l'ultima carica, tira 1d6 se ottieni 1 il bastone diventa un bastone da combattimento normale.

\index[OggettiMagici]{Bastone del Colpire}\subsubsection*{Bastone del Colpire}
25000 mo, molto raro, questo bastone può essere impugnato come un bastone da combattimento magico che conferisce un bonus di +3 ai Tiri per Colpire e di danno effettuati con esso. Quando colpisci con un attacco da mischia facendo uso del bastone, puoi spendere fino a 3 delle sue cariche. Per ogni carica spesa, il bersaglio subisce 1d6 danni da forza aggiuntivi. Il bastone ha 10 cariche, e recupera 2 cariche spese ogni giorno all'alba. Se spendi l'ultima carica, tira 1d6 se ottieni 1 il bastone diventa un bastone da combattimento normale.

\index[OggettiMagici]{Bastone del Fuoco}\subsubsection*{Bastone del Fuoco}
16000 mo, molto raro, mentre impugni questo bastone, hai resistenza al danno da fuoco.
Inoltre, puoi usare due azioni per spendere 1 o più delle sue cariche per lanciare tramite esso uno dei seguenti incantesimi: mani brucianti (1 carica, DC 13), muro di fuoco (4 cariche, DC 19) o palla di fuoco (3 cariche, DC 17).

Il bastone ha 10 cariche, e recupera 2 carica spesa ogni giorno all'alba. Se spendi l'ultima carica del bastone, tira 1d6 se ottieni 1 il bastone si annerisce, si trasforma in cenere, ed è distrutto.

\index[OggettiMagici]{Bastone del Gelo}\subsubsection*{Bastone del Gelo}
26000 mo, molto raro, mentre impugni questo bastone, hai resistenza ai danni da freddo.
Inoltre, puoi usare due azioni per spendere 1 o più delle sue cariche per lanciare tramite esso uno dei seguenti incantesimi.

- \textit{Incantesimi}: cono di freddo (5 cariche, DC 21), muro di ghiaccio (4 cariche, DC 19), nube di nebbia (1 carica, DC 13) o tempesta di ghiaccio (4 cariche, DC 19).

Il bastone ha 10 cariche, e recupera 2 carica spesa ogni giorno all'alba. Se spendi l'ultima carica del bastone, tira 1d6 se ottieni 1 il bastone si trasforma in acqua e si distrugge.

\index[OggettiMagici]{Bastone di Guarigione}\subsubsection*{Bastone della Guarigione}
13000 mo, raro, mentre lo impugni, puoi usare due azioni per spendere 1 o più delle sue cariche per lanciare tramite esso uno dei seguenti incantesimi: cura ferite leggere (1 carica), ristorare inferiore (2 cariche), rimuovi malattie (3 cariche). Questo bastone ha 10 cariche, e recupera 1 carica spesa ogni giorno all'alba. Se spendi l'ultima carica del bastone, tira 1d6 se ottieni 1 il bastone svanisce in un lampo di luce, perso per sempre.

\index[OggettiMagici]{Bastone degli Insetti Sciamanti}\subsubsection*{Bastone degli Insetti Sciamanti}
160000 mo, raro, questo bastone ha 10 cariche che puoi impiegare per usare le proprietà sotto descritte e recupera 1 carica ogni giorno all'alba. Se spendi l'ultima carica del bastone, tira 1d6 se ottieni 1 uno sciame di insetti consuma e distrugge il bastone, e poi si disperde.

- \textit{Incantesimi}. Mentre impugni questo bastone, puoi usare due azioni per spendere le sue cariche ed lanciare uno dei seguenti incantesimi: insetto gigante (4 cariche, DC 19) o piaga degli insetti (5 cariche, DC 21).

- \textit{Nube di Insetti}. Mentre impugni questo bastone, puoi usare due azioni e spendere 1 carica per fa sì che uno sciame di insetti innocui si diffonda in un raggio di 9 metri intorno a te. Gli insetti rimangono per 10 minuti, rendendo l'area oscurata pesantemente per tutti tranne te. Lo sciame si muove assieme a te, rimanendo centrato su di te. Un vento di almeno 15 chilometri all'ora disperde lo sciame e termina l'effetto.

\index[OggettiMagici]{Bastone del Pitone}\subsubsection*{Bastone del Pitone}
2000 mo, non comune, puoi usare due azioni per pronunciare la parola di comando del bastone e scagliarlo sul terreno fino a 3 metri di distanza. Il bastone diventa un serpente costrittore gigante sotto il tuo controllo e agisce al proprio conteggio di iniziativa. Utilizzando due azioni per pronunciare di nuovo la parola di comando, riporti il bastone alla sua forma normale nello spazio precedentemente occupato dal serpente.

Durante il tuo round puoi impartire ordini mentali al serpente finché si trova entro 18 metri da te e non sei inabile. Decidi tu quali azioni effettuerà il serpente e dove si muoverà durante il suo prossimo round, oppure puoi impartirgli un comando generico, come quello di attaccare i tuoi nemici o difendere un luogo. Se il serpente viene ridotto a 0 Punti Ferita, muore e ritorna alla sua forma di bastone. Poi, il bastone si frantuma ed è distrutto. Se il serpente si ritrasforma in forma di bastone prima di perdere tutti i suoi Punti Ferita, recupera tutti quelli persi.

\index[OggettiMagici]{Bastone del Potere}\subsubsection*{Bastone del Potere}
150000 mo, leggendario, questo bastone può essere impugnato come un bastone da combattimento magico che conferisce un bonus di +2 ai Tiri per Colpire e danno effettuati con esso. Mentre lo impugni, ricevi un bonus di +2 alla Difesa, ai Tiri Salvezza, e ai Tiri per Colpire con incantesimi. Questo bastone ha 20 cariche per le seguenti proprietà. Recupera 1d8 + 1 cariche spese ogni giorno all'alba. Se spendi l'ultima carica del bastone, tira 1d6 se ottieni 1 o meno il bastone mantiene il suo bonus di +2 ai Tiri per Colpire e danno ma perde tutte le altre proprietà.

- \textit{Colpo di Potere}. Quando colpisci con un attacco in mischia usando questo bastone, puoi spendere 1 carica per infliggere 1d6 danni da forza aggiuntivi al bersaglio.

- \textit{Incantesimi}. Mentre impugni questo bastone, puoi usare due azioni per spendere 1 o più delle sue cariche per lanciare tramite esso uno dei seguenti incantesimi: blocca mostri (5 cariche DC 21), cono di freddo (5 cariche, DC 21), globo di invulnerabilità (6 cariche, DC 22), levitazione (2 cariche DC 15), muro di forza (5 cariche, DC 21), palla di fuoco (3 cariche DC 17), dardo incantato (1 carica), raggio di indebolimento (1 carica DC 11) o fulmine (3 cariche DC 17).

- \textit{Colpo di Vendetta}. Puoi usare due azioni per spezzare il bastone sul tuo ginocchio o contro una superficie solida, eseguendo un colpo di vendetta. Il bastone viene distrutto e libera la sua magia rimanente in un'esplosione che si espande fino a riempire una sfera di 9 metri di raggio centrata su di esso.

Hai il 50\% di probabilità di viaggiare istantaneamente in un piano di esistenza a caso, evitando così l'esplosione. Se non riesci a evitare l'effetto, subisci danni da forza pari a 16 x il numero di cariche nel bastone. Ogni altra creatura nell'area deve effettuare un Tiro Salvezza su Riflessi con DC 27. Se il Tiro Salvezza fallisce, la creatura subisce un ammontare di danno basato sulla distanza dal punto di origine dell'esplosione, come mostrato sulla tabella seguente.

Se il Tiro Salvezza riesce, la creatura subisce la metà di questi danni.

\medskip

\begin{tabularx}{0.45\textwidth}{Xl}
\hline
\textbf{Distanza dall'origine} &\textbf{Danno}\\
3 metri o meno &8 x cariche nel bastone\\
Fino a 6 metri& 6 x cariche nel bastone\\
Fino a 9 metri& 4 x cariche nel bastone\\
\end{tabularx}

\medskip

Nota: il Bastone dell'Archimago e del Potere sono simili, questo perché preparati da due acerrimi nemici che volevano creare il Bastone più potente.

\index[OggettiMagici]{Bastone dei Tuoni e Fulmini}\subsubsection*{Bastone dei Tuoni e Fulmini}
10000 mo, molto raro, il bastone può essere impugnato come un bastone da combattimento magico che conferisce un bonus di +2 ai Tiri per Colpire e danno effettuati con esso. Inoltre ha le seguenti proprietà. Quando viene usata una di queste proprietà, non se ne potrà più far uso fino all'alba successiva.

- \textit{Fulmine}. Quando colpisci con un attacco in mischia usando il bastone, puoi far sì che il bersaglio subisca 2d6 danni da fulmine aggiuntivi.

- \textit{Tuono}. Quando colpisci con un attacco in mischia usando il bastone, puoi far sì che il bastone emetta il suono di un tuono, udibile fino a 90 metri di distanza. Il bersaglio colpito deve superare un Tiro Salvezza su Tempra con DC 21 o restare stordito fino al termine del tuo prossimo round.

- \textit{Colpo Fulminante}. Puoi usare due azioni per far sì che una fulmine balzi dalla punta del bastone in una linea larga 1,5 metri e lunga 36 metri. Ogni creatura sulla linea deve effettuare un Tiro Salvezza su Riflessi con DC 21, subendo 9d6 danni da fulmine se lo fallisce, o la metà di questi danni se lo supera.

- \textit{Rombo di Tuono}. Puoi usare due azioni per far sì che il bastone produca un rombo di tuono assordante, udibile fino a 180 metri di distanza. Ogni creatura entro 18 metri da te (te escluso) deve effettuare un Tiro Salvezza su Tempra con DC 21. Se fallisce il Tiro Salvezza, la creatura subisce 2d6 danni da suono e resta assordata per 1 minuto. Se supera il Tiro Salvezza, subisce la metà dei danni e non è assordata.

- \textit{Tuoni e Fulmini}. Puoi usare due Azioni per usare le proprietà Colpo Fulminante e Rombo di Tuono assieme. Farlo non consuma l'uso giornaliero di quelle proprietà, ma solo l'uso di questa.

\index[OggettiMagici]{Bastone della Stregoneria}\subsubsection*{Bastone della Stregoneria}
85000 mo, molto raro, in combattimento, questo bastone funziona come un bastone +1. Può essere usato per lanciare evoca elementale, invisibilità, passaparete e ragnatela. Il bastone può essere usato come una bacchetta della paralisi. Ciascuno di questi poteri richiede una carica. E' possibile spezzare il bastone per produrre un “colpo finale”, il cui effetto dipende dal numero di cariche residue. Il bastone esplode in una grande sfera di fiamme, colpendo tutte le creature entro 9 m (compreso il proprietario del bastone) e infliggendo 8 ferite per carica rimasta, Tiro Salvezza su Tempra DC 27 per dimezzare.

\index[OggettiMagici]{Verga dell'Ammaliamento}\subsubsection*{Verga dell'Ammaliamento}
28000 mo, raro, spendendo 1 carica, il possessore può lanciare dominare bestie, con 2 cariche dominare persone e con 3 cariche dominare mostri.

\index[OggettiMagici]{Verga dell'Assorbimento}\subsubsection*{Verga dell'Assorbimento}
50000 mo, molto raro, mentre impugni questa verga, puoi usare una Azione per assorbire un incantesimo che prenda come bersaglio solo te e privo di un'area di effetto. L'effetto dell'incantesimo assorbito è cancellato, e l'energia dell'incantesimo (non l'incantesimo stesso) viene assorbita dalla verga. Nel corso della sua esistenza la verga può assorbire e contenere fino ad una somma di 31 Livelli di incantesimi. Una volta che la verga ha assorbito 8 incantesimi (max livello 4), non ne potrà più assorbire. Se sei il bersaglio di un incantesimo che la verga non può contenere, la verga non ha alcun effetto sull'incantesimo. Quando prendi in mano la verga, sai quanti incantesimi la verga ha assorbito finora. Se sei un incantatore e impugni la verga, puoi convertire tutta l'energia contenuta per avere 10 Punti Magia in più.

\index[OggettiMagici]{Verga Inamovibile}\subsubsection*{Verga Inamovibile}
5000 mo, non comune, questa verga di ferro piatta ha un pulsante a un'estremità. Puoi usare due azioni per premere il pulsante, che fa sì che la verga resti magicamente fissata sul posto. Fino a quando tu o un'altra creatura userete due azioni per premere di nuovo il pulsante, la verga non si muoverà, anche se dovesse sfidare la gravità. La verga può sostenere fino a 4000 chili di peso. Un peso maggiore fa sì che la verga si disattivi e cada. Una creatura può usare due azioni per effettuare una prova di Forza con DC 30, spostando la verga di 3 metri in caso di successo.

\index[OggettiMagici]{Verga del Colpo possente}\subsubsection*{Verga del Colpo possente}
30000 mo, molto rara, una verga del colpo possente infligge 1d8+3 ferite, e funziona come una mazza leggera magica +3. Quando la verga è usata contro i golem, consuma 1 carica per colpo inflitto, ed infligge 2d8+6 ferite. Si noti che quando la verga è usata come arma contro un golem un Tiro per Colpire Critico lo annienta istantaneamente. In aggiunta, questa verga infligge ferite addizionali a immondi e non morti. Quando si attaccano questi mostri, un Tiro per Colpire Critico causa il consumo di 1 carica, e la verga infligge il triplo delle ferite.

\index[OggettiMagici]{Verga della Forza Sovrana}\subsubsection*{Verga della Forza Sovrana}
50000 mo, leggendaria, questa verga ha una testa flangiata, e funziona come una mazza magica che conferisce un bonus di +3 ai Tiri per Colpire e danno effettuati con essa. La verga ha delle proprietà associate ai sei diversi pulsanti che sono disposti lungo il manico. Possiede anche altre tre proprietà descritte di seguito.

\textbf{Sei Pulsanti}. Puoi premere uno dei sei pulsanti della verga con due azioni. L'effetto del pulsante dura finché non premi un pulsante differente o premi di nuovo lo stesso pulsante, facendo tornare la verga alla sua forma normale.

- Se premi il \textit{pulsante 1}, la verga diventa un'arma lingua di fuoco, e una lama infuocata fuoriesce dall'estremità opposta alla testa flangiata.

- Se premi il \textit{pulsante 2}, la testa flangiata della verga si ripiega e fuoriescono due lame a mezzaluna, che trasformano la verga in un'ascia da battaglia magica che conferisce un bonus di +3 ai Tiri per Colpire e danno effettuati con essa.

- Se premi il \textit{pulsante 3}, la testa flangiata della verga si ripiega, e una punta di lancia esce fuori dall'estremità della verga, mentre il manico si allunga fino a 1,8 metri, trasformando la verga in una lancia magica che conferisce un bonus di +3 ai Tiri per Colpire e danno effettuati con essa.

- Se premi il \textit{pulsante 4}, la verga si trasforma in un'asta per scalare lunga fino a 15 metri, come richiesto da te. Sulle superfici dure come il granito, uno spuntone sul fondo e tre in cima tengono l'asta fissa sul posto. Sbarre orizzontali lunghe 7,5 centimetri si dipanano lungo i lati della verga, a 30 centimetri di distanza l'uno dall'altro, per formare una scala. L'asta può sostenere 2000 chili. Un peso superiore o la mancanza di un ancoraggio solido fa sì che la verga torni alla sua forma normale.

- Se premi il \textit{pulsante 5}, la verga si trasforma in un ariete da sfondamento e conferisce a chi lo usa un bonus di +10 alle prove di Forza effettuate per sfondare porte, barricate o altre barriere.

- Se premi il \textit{pulsante 6}, la verga assume o rimane nella sua forma normale e indica il nord magnetico (non accade nulla se questa funzione della verga viene impiegata in zone prive di un nord magnetico). La verga ti fornisce anche un'approssimativa conoscenza della profondità sottoterra e della tua altezza sul livello del mare.

\textit{Risucchiare Vita}. Quando colpisci una creatura con un attacco in mischia utilizzando la verga, puoi obbligare il bersaglio a effettuare un tiro Salvezza su Tempra con DC 21. Se lo fallisce, il bersaglio subisce 4d6 danni da Vuoto aggiuntivi e vengono tolti dal massimo dei Punti Ferita, e tu recuperi un numero di Punti Ferita pari alla metà del danno da Vuoto inflitto. Una volta usata, questa proprietà non più essere usata fino all'alba del giorno successivo.

\textbf{Paralizzare}. Quando colpisci una creatura con un attacco da mischia utilizzando la verga, puoi obbligare il bersaglio a effettuare un Tiro Salvezza su Tempra con DC 21. Se lo fallisce, il bersaglio è paralizzato per 1 minuto. Il bersaglio può ripetere il Tiro Salvezza al termine di ciascun suo round, terminando l'effetto su di sé in caso lo superi. Una volta usata, questa proprietà non può più essere usata fino all'alba del giorno successivo.

\textit{Terrorizzare}. Mentre impugni questa verga, puoi obbligare ogni creatura che vedi entro 9 metri da te a effettuare un Tiri Salvezza su Volontà con DC 21. Se lo fallisce, il bersaglio è spaventato da te per 1 minuto. Il bersaglio spaventato può ripetere il Tiro Salvezza al termine di ciascun suo round, terminando l'effetto su di sé in caso lo superi. Una volta usata, questa proprietà non può più essere usata fino all'alba del giorno successivo.

Questa verga non può essere ricaricata. Quando le cariche finiscono rimane una

\index[OggettiMagici]{Verga della Prontezza}\subsubsection*{Verga della Prontezza}
25000 mo, molto raro, questa verga dalla testa flangiata ha le seguenti proprietà.

\textit{Prontezza}. Mentre impugni questa verga, hai +2 alle prove di Saggezza e ai tiri di iniziativa.

\textit{Incantesimi}. Mentre impugni questa verga, puoi usare due azioni per lanciare tramite essa uno dei seguenti incantesimi: individuazione del bene e del male, individuazione del magico, individuazione del veleno e delle malattie o vedere invisibilità.

\textit{Aura Protettiva}. Con due azioni, puoi piantare l'estremità appuntita della verga nel terreno. A quel punto la testa della verga irradierà luce intensa in un raggio di 18 metri e luce fioca per ulteriori 18 metri. All'interno di questa luce intensa, tu e qualsiasi creatura a te amichevole otterrete un bonus di +1 alla Difesa e ai Tiri Salvezza e potrete percepire la posizione di qualsiasi creatura invisibile ostile che si trovi anch'essa all'interno della luce intensa. La testa della verga smette di emettere luce e termina l'effetto dopo 10 minuti, o quando una creatura usa due azioni per estrarre la verga dal terreno. Questa proprietà non può essere usata di nuovo fino all'alba del giorno successivo.

\index[OggettiMagici]{Verga della Sicurezza}\subsubsection*{Verga della Sicurezza}
90000 mo, molto raro, mentre impugni questa verga, puoi usare due azioni per attivarla. Di conseguenza la verga trasporta te e fino ad altre 199 altre creature consenzienti visibili in un paradiso collocato in uno spazio extraplanare. Sarai tu a scegliere la forma di questo paradiso. Potrebbe essere un placido giardino, una gradevole radura, un'allegra taverna, un immenso palazzo, un'isola tropicale, o una fantastica fiera o qualsiasi altra cosa tu riesca a immaginare. Quale che sia la sua natura, il paradiso contiene cibo e bevande sufficienti ad alimentare i suoi visitatori. Tutto ciò con cui si può interagire nello spazio extraplanare può esistere solo al suo interno.

Per ogni ora trascorsa in questo paradiso, un visitatore recupera Punti Ferita come se avesse avesse riposato una notte. Inoltre, finché le creature restano nel paradiso non invecchiano, sebbene il tempo trascorra normalmente. I visitatori possono restare nel paradiso per un massimo di 200 giorni diviso il numero di creature presenti (arrotondare per difetto).

Quando il tempo termina o usi due azioni per farlo terminare, tutti i visitatori ricompaiono nel luogo da loro occupato quando hai attivato la verga, o nello spazio non occupato più vicino a quello. La verga non potrà essere usata di nuovo prima che siano passati dieci giorni.

\index[OggettiMagici]{Verga della Sovranità}\subsubsection*{Verga della Sovranità}
16000 mo, raro, puoi usare due azioni e presentare la verga e richiedere obbedienza a ciascuna creatura visibile entro 36 metri da te di tua scelta. Ogni bersaglio deve superare un Tiri Salvezza su Volontà con DC 17 o restare Affascinato da te per 8 ore. Mentre è affascinata in questa maniera, la creatura ti considera un capo fidato. Se le viene recato danno da te o dai tuoi compagni, o le viene ordinato di fare qualcosa contrario alla sua natura, il bersaglio smetterà di essere affascinato in questa maniera. La verga non può essere usata di nuovo prima della prossima alba.

\index[OggettiMagici]{Verga Tentacolare}\subsubsection*{Verga Tentacolare}
5000 mo, raro, questa verga è un'arma magica che termina in tre tentacoli di cuoio. Mentre impugni la verga, puoi usare due azioni per dirigere ciascun tentacolo per attaccare una creatura visibile entro 4,5 metri da te. Ogni tentacolo effettua un Tiro per Colpire da mischia con un bonus di +9. Se colpisci, il tentacolo infligge 1d6 danni da botta. Se colpisci un bersaglio con tutti e tre i tentacoli, esso deve effettuare un Tiro Salvezza su Tempra con DC 15. Se lo fallisce, la velocità della creatura è dimezzata, ha -1d6 ai Tiri Salvezza di Riflessi, e per 1 minuto non può usare le sue reazioni. Inoltre, durante ciascun suo round, egli può effettuare due azioni o due azioni ma non entrambe. Il bersaglio può ripetere il Tiro Salvezza al termine di ciascun suo round, terminando l'effetto su di sé in caso lo superi.


\subsection{Pozioni - Olii}

\index[OggettiMagici]{Pozione di Amicizia con gli Animali}\subsubsection*{Pozione di Amicizia con gli Animali}

non comune, 200 mo, quando bevi questa pozione, per 1 ora puoi lanciare a volontà l'incantesimo Amicizia con gli Animali (DC del Tiro Salvezza 15).

\index[OggettiMagici]{Pozione di Arrampicata}\subsubsection*{Pozione di Arrampicata}

comune,250 mo, quando bevi questa pozione, per 1 ora ottieni velocità di scalata pari alla tua velocità di passeggio. Durante questo periodo hai +1d6 alle prove di Resistenza che compi per effettuare una scalata.

\subsubsection*{Pozione della Chiaraudienza animale}\index[OggettiMagici]{Pozione della Chiaraudienza animale}

non comune, 500 mo, questa pozione conferisce a chi la beve la facoltà di percepire i suoni attraverso le orecchie di un animale che si trovi in un raggio di 18 metri. Una barriera di piombo blocca questo effetto.

\subsubsection*{Pozione della Chiaroveggenza animale}\index[OggettiMagici]{Pozione della chiaroveggenza animale}
non comune, 500 mo,questa pozione conferisce a chi la beve la facoltà di vedere attraverso gli occhi di un animale che si trovi in un raggio di 18 metri. Una barriera di piombo blocca questo effetto.

\subsubsection*{Pozione di Controllo degli animali}\index[OggettiMagici]{Pozione di Controllo degli animali}
rara 1500 mo, chiunque beva questa posizione è come avesse lanciato Dominare Bestie


\subsubsection*{Pozione di Controllo dei draghi}\index[OggettiMagici]{Pozione di Controllo dei draghi}
leggendaria, 5000 mo, questa pozione conferisce un potere equivalente all’incantesimo dominare mostri su un singolo tipo di drago. E' possibile controllare un drago entro 18 metri per 5d4 round.

\subsubsection*{Pozione di Controllo dei non morti}\index[OggettiMagici]{Pozione di Controllo dei non morti}
2500 mo, rara, anche se normalmente i non morti sono immuni a questo tipo di effetti, questa pozione permette a chi la beve di influenzare 3d6 DV di non morti (intelligenti o no) come se usasse l'incantesimo charme. La durata dell’effetto è di 5d4 round.

\subsubsection*{Pozione di Controllo delle persone}\index[OggettiMagici]{Pozione di Controllo delle persone}
500 mo, non comune, una volta ingerita, questa pozione conferisce a chi la beve un potere analogo all'incantesimo charme.

\subsubsection*{Pozione di Controllo delle piante}\index[OggettiMagici]{Pozione di Controllo delle piante}
1500 mo, rara, chi beve questa pozione è in grado di controllare tutte le piante e le creature vegetali (compresi i funghi) in un’area quadrata di 6x6 m ed entro una distanza di 27 metri. L'effetto dura 5d4 round. Le piante obbediscono secondo le loro possibilità (ad esempio, le liane possono attorcigliarsi e infittirsi, causando lentezza o impedimento alla vista). E' possibile dare ordini a creature vegetali senzienti, ma queste hanno diritto a un Tiro Salvezza su Volontà DC 19. Come per altri tipi di ammaliamento, non si può ordinare a una creatura controllata di farsi male da sola.

\index[OggettiMagici]{Pozione di Crescita}\subsubsection*{Pozione di Crescita}
300 mo, non comune, quando bevi questa pozione, per 1d4 ore ottieni l'effetto "ingrandire" dell'incantesimo ingrandire/ridurre (non richiede concentrazione).

\index[OggettiMagici]{Pozione di Eroismo}\subsubsection*{Pozione di Eroismo}
200 mo, raro, quando bevi questa pozione, ottieni 10 Punti Ferita temporanei che durano 1 ora. Per la stessa durata sei sotto l'effetto dell'incantesimo benedizione (non richiede concentrazione).

\index[OggettiMagici]{Pozione di Forma Gassosa}\subsubsection*{Pozione di Forma Gassosa}

1500 mo, raro, quando bevi questa pozione, per 1 ora o finché non terminerai l'effetto con due azioni ottieni l'effetto dell'incantesimo forma gassosa (non richiede concentrazione).

\index[OggettiMagici]{Pozione di Forza dei Giganti}\subsubsection*{Pozione di Forza dei Giganti}
rarità varia, costo vario, quando bevi questa pozione, per 1 ora il tuo punteggio di Forza cambia. Il tipo di gigante determina il punteggio (vedi la tabella seguente). La pozione non ha effetto se il tuo punteggio di Forza è pari o superiore al nuovo punteggio. La pozione della forza del gigante del gelo e la pozione della forza del gigante di pietra hanno lo stesso effetto.

- delle colline, Forza 5, Non comune 500 mo

- di pietra o del gelo, Forza 6, Raro 1000 mo

- del fuoco, Forza 7, Raro 2000 mo

- delle nuvole, Forza 8, Molto raro 5000 mo

- delle tempeste, Forza 9, Leggendario 10000 mo

\index[OggettiMagici]{Pozione di Guarigione}\subsubsection*{Pozione di Guarigione}
rarità varia, costo vario, quando bevi da questa pozione, recuperi un numero di Punti Ferita che varia a seconda della rarità della pozione di guarigione.

- Comune, Punti Ferita 2d4 + 2, 75 mo

- Maggiore, Punti Ferita 4d4 + 4, 150 mo

\index[OggettiMagici]{Pozione di Guarigione}\subsubsection*{Pozione di Guarigione Maggiore}
rarità varia, costo vario, quando bevi da questa pozione, recuperi un numero di Punti Ferita che varia a seconda della rarità della pozione di guarigione.

- Superiore, Punti Ferita 8d4 + 8, 350 mo

- Suprema, Punti Ferita 10d4 + 20, 1500 mo

\subsubsection*{Pozione dell'Inganno}\index[OggettiMagici]{Pozione dell’Inganno}

500 mo, rara, questa pozione ha un nome quanto mai appropriato, poiché convince chi la beve di aver ingerito una pozione di un altro tipo. Per esempio, una finta “pozione di chiaraudienza” potrebbe far sentire a chi l’ha bevuta suoni che in realtà non esistono. Se più persone assaggiano questa pozione, c'è una probabilità del 90\% che concordino nel ritenerla dello stesso tipo.

\index[OggettiMagici]{Pozione di Invisibilità}\subsubsection*{Pozione di Invisibilità}
200 mo, molto raro, quando bevi questa pozione, per 1 ora diventi invisibile. Mentre sei invisibile, tutto ciò che trasporti o indossi resta anch'esso invisibile assieme a te. L'effetto ha termine qualora tu attacchi o lanci un incantesimo.

\subsubsection*{Pozione dell’invulnerabilità}\index[OggettiMagici]{Pozione dell’invulnerabilità}
800 mo, rara, una pozione di invulnerabilità conferisce a chi la beve un bonus +2 ai Tiri Salvezza e un miglioramento di 2 punti la Difesa.

\index[OggettiMagici]{Pozione di Lettura del Pensiero}\subsubsection*{Pozione di Lettura del Pensiero}
200 mo, raro, quando bevi questa pozione, ottieni l'effetto dell'incantesimo individuazione dei pensieri (DC del Tiro Salvezza 15).

\subsubsection*{Pozione della Levitazione}\index[OggettiMagici]{Pozione della Levitazione}
200 mo, non comune, questa pozione ha lo stesso effetto dell’incantesimo levitazione.

\subsubsection*{Pozione della Longevità}\index[OggettiMagici]{Pozione della Longevità}
15000 mo, leggendaria, questa pozione fa ringiovanire di 1d12 anni. La giovinezza riguadagnata non annulla soltanto l'invecchiamento naturale, ma anche l’invecchiamento causato da effetti magici o creature. Esiste un pericolo nell'usare questa pozione, poiché ogni volta che si beve una pozione di longevità, c'è una probabilità cumulativa dell’1\% che tutti i benefici precedentemente guadagnati con pozioni di questo tipo siano annullati. Non è possibile consumare una dose parziale di questa pozione.

\subsubsection*{Pozione della Metamorfosi}\index[OggettiMagici]{Pozione della Metamorfosi}
2500 mo, rara, questa pozione conferisce un potere analogo all’incantesimo metamorfosi.

\index[OggettiMagici]{Pozione di Resistenza}\subsubsection*{Pozione di Resistenza}
300 mo, non comune, quando bevi questa pozione, per 1 ora ottieni resistenza a un tipo di danno. Il Narratore sceglie il tipo di danno o lo determina casualmente (Acido, Freddo, Fuoco, Forza, Fulmine, Vuoto, Veleno, Luce, Suono)

\index[OggettiMagici]{Pozione di Respirare Sott'Acqua}\subsubsection*{Pozione di Respirare Sott'Acqua} \textit{Pozione, non comune} 200 mo

Dopo aver bevuto questa pozione, puoi respirare sott'acqua per 1 ora.

\index[OggettiMagici]{Pozione di Rimpicciolimento}\subsubsection*{Pozione di Rimpicciolimento}
300 mo, raro, quando bevi questa pozione, per 1d4 ore ottieni l'effetto "ridurre" dell'incantesimo ingrandire/ridurre (non richiede concentrazione).

\index[OggettiMagici]{Pozione di Veleno}\subsubsection*{Pozione di Veleno}
250 mo, non comune, questo distillato assomiglia, odora e ha il sapore di una pozione di guarigione o di un'altra pozione benefica. Tuttavia è in realtà un veleno mascherato da magie di illusione. L'incantesimo identificare ne rivela la vera natura.

Se lo bevi, subisci 3d6 danni da veleno, e devi superare un tiro Salvezza su Tempra con DC 13 o restare avvelenato un ulteriore round e subire 1d6 di danno all'inizio del round successivo.

\index[OggettiMagici]{Pozione di Veleno}\subsubsection*{Pozione di Veleno Maggiore}
450 mo, non comune, questo distillato assomiglia, odora e ha il sapore di una pozione di guarigione o di un'altra pozione benefica. Tuttavia è in realtà un veleno mascherato da magie di illusione. Se identificato si comprende la vera natura.

Se lo bevi, subisci 5d6 danni da veleno, e devi superare un tiro Salvezza su Tempra con DC 18 o restare avvelenato. All'inizio di ciascun tuo round, finché resti avvelenato a questo modo, subisci 2d6 danni da veleno. Puoi ripetere il Tiro Salvezza al termine di ciascun tuo round. Se il Tiro Salvezza riesce, il danno da veleno subito nei turni successivi diminuisce di 1d6. Il veleno cessa i suoi effetti quando il danno scende a 0d6.

\index[OggettiMagici]{Pozione di Velocità}\subsubsection*{Pozione di Velocità}
400 mo, molto raro, quando bevi questa pozione, ottieni l'effetto dell'incantesimo velocità per 1 minuto (non richiede concentrazione).

\index[OggettiMagici]{Pozione di Volo}\subsubsection*{Pozione di Volo}
500 mo, molto raro, quando bevi questa pozione, per 1 ora ottieni velocità di volo pari alla tua normale velocità di passeggio e puoi fluttuare. Se la pozione ha termine mentre stai volando, cadi a meno che non possiedi qualche altro metodo per restare in aria.

\index[OggettiMagici]{Filtro d'Amore}\subsubsection*{Filtro d'Amore}
120 mo, non comune, resterai Affascinato per 1 ora dalla prima creatura che vedrai entro 10 minuti da quando avrai bevuto questo filtro. Se la creatura è di una specie o genere da cui sei normalmente attratto, finché sei Affascinato la considererai il tuo unico e grande amore.

\subsubsection*{Filtro Scopritesori}\index[OggettiMagici]{Filtro Scopritesori}
500 mo, raro, chi beve questa pozione può percepire entro 72 metri i tesori che contengono metalli preziosi o gemme, purché abbiano un valore di almeno 50 monete d’oro. Si può percepire la direzione del tesoro, ma non la sua esatta distanza. Nessuna barriera non magica può impedire di percepire i tesori, tranne una lastra di piombo.

\index[OggettiMagici]{Olio di Affilatezza}\subsubsection*{Olio di Affilatezza}
3200 mo, molto raro, quest'olio può ricoprire un'arma tagliente o perforante o fino a 5 munizioni taglienti o perforanti. Applicare l'olio richiede 1 minuto. Per 1 ora, l'arma ricoperta dall'olio è magica e ha un bonus di +3 ai Tiri per Colpire e danno.

\index[OggettiMagici]{Olio di Forma Eterea}\subsubsection*{Olio di Forma Eterea}
2000 mo, raro, una dose di olio è sufficiente a ricoprire una creatura di taglia Media o inferiore, e l'equipaggiamento che indossa e trasporta (è necessaria un'ulteriore fiala per ogni categoria di taglia sopra la Media). Applicare l'olio richiede 10 minuti. Dopodiché la creatura ottiene l'effetto dell'incantesimo forma eterea per 1 ora.

\index[OggettiMagici]{Olio di Scivolosità}\subsubsection*{Olio di Scivolosità}
500 mo, non comune, l'olio può coprire una creatura di taglia Media o inferiore, insieme a tutto l'equipaggiamento che indossa o trasporta (è necessaria un'ulteriore fiala per ogni categoria di taglia sopra la Media). Applicare l'olio richiede 10 minuti. La creatura ottiene poi il beneficio dell'incantesimo libertà di movimento per 8 ore. In alternativa, con due azioni si può versare l'olio sul terreno, duplicando per 8 ore l'effetto dell'incantesimo unto su quell'area.


\subsection{Anelli}

\index[OggettiMagici]{Anello Accumula Incantesimi}\subsubsection*{Anello Accumula Incantesimi}
24000 mo, raro, questo anello immagazzina gli incantesimi lanciati su di esso, conservandoli fino a che chi lo indossa non ne faccia uso. L'anello può accumulare fino a 3 Incantesimi per un totale di 17 Punti Magia con un massimo di 6 punti magia singolo.

Qualsiasi creatura può lanciare un incantesimo accumulato di livello da 1 a 5 sull'anello toccandolo. L'incantesimo ha una DC pari a 10 + 2 x Livello incantesimo, l'eventuale Tiro per Colpire viene effettuato da chi lancia l'incantesimo.

L'incantatore lanciato deve mirare all'anello per farlo assorbire. Se l'anello non può contenere l'incantesimo, l'incantesimo si manifesta normalmente. Un incantesimo lanciato tramite questo anello non è più contenuto al suo interno, e libera spazio per altri incantesimi.

\index[OggettiMagici]{Anello dell'Ariete}\subsubsection*{Anello dell'Ariete}
5000 mo, raro, mentre indossi questo anello, puoi usare due azioni per spendere da 1 a 3 cariche per attaccare una creatura visibile entro 18 metri da te.

L'anello produce una testa di ariete spettrale ed effettua il suo Tiro per Colpire con un bonus di +7. Se colpisce, per ogni carica spesa, il bersaglio subisce 2d10 danni da forza e viene spinto di 1,5 metri lontano da te.

In alternativa, puoi spendere da 1 a 3 cariche dell'anello con due azioni per tentare di rompere un oggetto visibile entro 18 metri da te che non sia indossato o trasportato. L'anello effettua una prova con Forza +5 ogni carica spesa.

Questo anello ha 3 cariche, e recupera 1d3 cariche spese ogni mattina all'alba.

\index[OggettiMagici]{Anello di Caduta Piuma}\subsubsection*{Anello di Caduta Piuma}
2000 mo, raro, se cadi da più di 1 metro e indossi questo anello si attiva l'incantesimo Caduta Piuma

\index[OggettiMagici]{Anello di Camminare sull'Acqua}\subsubsection*{Anello di Camminare sull'Acqua}
1500 mo, non comune, mentre indossi questo anello, puoi stare in piedi o muoverti su qualsiasi superficie liquida come se fosse terreno solido.

\index[OggettiMagici]{Anello del Calore}\subsubsection*{Anello del Calore}
5000 mo, non comune, mentre indossi questo anello, hai resistenza ai danni da freddo. Inoltre, tu e tutto quello che indossi e trasporti siete immuni agli effetti delle temperature basse fino a -45° C.

\index[OggettiMagici]{Anello degli Elementali dell'Acqua}\subsubsection*{Anello degli Elementali dell'Acqua}
250000 mo, leggendario. questo anello è collegato al Piano Elementale dell'Acqua. Mentre lo indossi, hai +1d6 ai Tiri per Colpire contro gli elementali del Piano Elementale dell'Acqua, ed essi hanno -1d6 ai Tiri per Colpire effettuati contro di te.

Puoi spendere 2 cariche dell'anello per lanciare dominare mostri su di un elementale dell'acqua. Inoltre, puoi stare in piedi e camminare sulle superfici liquide come se fossero terreno solido. Puoi parlare e comprendere l'Aquan.

Se aiuti a uccidere un elementale dell'acqua mentre indossi l'anello, ottieni accesso alle seguenti proprietà aggiuntive:

\medskip

\begin{itemize}
\item
Puoi respirare sott'acqua e hai velocità di nuovo pari alla tua velocità di passeggio.
\item
Puoi lanciare tramite l'anello i seguenti incantesimi, spendendo il numero di cariche richieste: creare o distruggere acqua (1 carica), controllare tempo atmosferico (3 cariche), muro di ghiaccio (3 cariche) o tempesta di ghiaccio (2 cariche).
L'anello ha 5 cariche. Recupera 1d4 + 1 cariche ogni giorno all'alba. Gli incantesimi lanciati tramite l'anello hanno DC del Tiro Salvezza 21.

\end{itemize}

\index[OggettiMagici]{Anello degli Elementali dell'Aria}\subsubsection*{Anello degli Elementali dell'Aria}
250000 mo, leggendario, questo anello è collegato al Piano Elementale dell'Aria. Mentre lo indossi, hai +1d6 ai Tiri per Colpire contro gli elementali del Piano Elementale dell'Aria, ed essi hanno -1d6 ai Tiri per Colpire effettuati contro di te.

Puoi spendere 2 cariche dell'anello per lanciare dominare mostri su di un elementale dell'aria. Inoltre, quando cadi, scendi di 18 metri per round e non subisci danni dalla caduta. Puoi parlare e comprendere l'Auran.

Se aiuti a uccidere un elementale dell'aria mentre indossi l'anello, ottieni accesso alle seguenti proprietà aggiuntive:

\medskip

\begin{itemize}
\item
Hai resistenza ai danni da fulmine.
\item
Hai velocità di volo pari alla tua velocità di passeggio e puoi fluttuare.
\item
Puoi lanciare tramite l'anello i seguenti incantesimi, spendendo il numero di cariche richieste: catena di fulmini (3 cariche), folata di vento (2 cariche) o muro di vento (1 carica).
\end{itemize}
\medskip

L'anello ha 5 cariche. Recupera 1d4 + 1 cariche ogni giorno all'alba.

Gli incantesimi lanciati tramite l'anello hanno DC del Tiro Salvezza 21.


\index[OggettiMagici]{Anello degli Elementali del Fuoco}\subsubsection*{Anello degli Elementali del Fuoco}
250000 mo, leggendario, questo anello è collegato al Piano Elementale del Fuoco. Mentre lo indossi, hai +1d6 ai Tiri per Colpire contro gli elementali del Piano Elementale del Fuoco, ed essi hanno -1d6 ai Tiri per Colpire effettuati contro di te.

Puoi spendere 2 cariche dell'anello per lanciare dominare mostri su di un elementale del fuoco. Inoltre, hai resistenza ai danni da fuoco. Puoi parlare e comprendere l'Ignan.

Se aiuti a uccidere un elementale del fuoco mentre indossi l'anello, ottieni accesso alle seguenti proprietà aggiuntive:

\medskip

\begin{itemize}
\item
Hai immunità ai danni da fuoco.
\item
Puoi lanciare tramite l'anello i seguenti incantesimi, spendendo il numero di cariche richieste: mani brucianti (1 carica), muro di fuoco (3 cariche) o palla di fuoco (2 cariche).
\end{itemize}

\medskip

L'anello ha 5 cariche. Recupera 1d4 + 1 cariche ogni giorno all'alba.

Gli incantesimi lanciati tramite l'anello hanno DC del Tiro Salvezza 21.

\index[OggettiMagici]{Anello degli Elementali della Terra}\subsubsection*{Anello degli Elementali della Terra}
250000 mo, leggendario, questo anello è collegato al Piano Elementale della Terra. Mentre lo indossi, hai +1d6 ai Tiri per Colpire contro gli elementali del Piano Elementale della Terra, ed essi hanno -1d6 ai Tiri per Colpire effettuati contro di te.

Puoi spendere 2 cariche dell'anello per lanciare dominare mostri su di un elementale della terra. Inoltre, puoi muoverti su terreno difficile composto da macerie, pietre o terra come se fosse terreno normale. Puoi parlare e comprendere il Terran.

Se aiuti a uccidere un elementale della terra mentre indossi l'anello, ottieni accesso alle seguenti proprietà aggiuntive:

\medskip

\begin{itemize}
\item
Hai resistenza ai danni da acido.
\item
Puoi muoverti attraverso la terra o la roccia solida come se fossero terreno difficile. Se vi termini il tuo round, vieni proiettato fuori nello spazio non occupato più vicino che hai occupato per ultimo.
\item
Puoi lanciare tramite l'anello i seguenti incantesimi, spendendo il numero di cariche richieste: scolpire pietra (2 cariche), muro di pietra (3 cariche) o pelle di pietra (1 carica).
\end{itemize}

\medskip

L'anello ha 5 cariche. Recupera 1d4 + 1 cariche ogni giorno all'alba.

Gli incantesimi lanciati tramite l'anello hanno DC del Tiro Salvezza 21.

\subsubsection*{Anello del Controllo delle persone}\index[OggettiMagici]{Anello del Controllo delle persone}
2500 mo, raro, questo anello conferisce a chi lo indossa la capacità di usare l'incantesimo charme una volta la giorno. L'effetto dura finché chi esercita il controllo non vi pone termine, passa 1 ora o non viene usato dispersione della magia.

\subsubsection*{Anello del Controllo delle piante}\index[OggettiMagici]{Anello del Controllo delle piante}
5000 mo, molto raro, chi indossa questo anello può controllare le piante e le creature vegetali in un’area quadrata di 3x3 m entro una distanza di 18 metri. Anche se una pianta è immobile, essa si può spostare mentre è sotto l’effetto di questo anello. Il controllo dura fintanto che chi lo esercita mantiene una concentrazione totale, che impedisce ogni altra azione.

\subsubsection*{Anello della Debolezza}\index[OggettiMagici]{Anello della Debolezza}
raro, una volta indossato, questo anello può essere rimosso solo da rimuovere maledizione. Nel corso di 6 round, la forza di chi indossa l’anello si riduce a -3.

\index[OggettiMagici]{Anello dei Tre Desideri}\subsubsection*{Anello dei Tre Desideri}
75000 mo, leggendario, mentre indossi quest'anello, puoi usare due azioni per spendere 1 delle sue 1d3 cariche per lanciare tramite esso l'incantesimo desiderio. L'anello perde la sua magia quando usi l'ultima carica.

\index[OggettiMagici]{Anello di Elusione}\subsubsection*{Anello di Elusione}
5000 mo, raro, mentre indossi questo anello e fallisci un Tiro Salvezza su Riflessi, puoi usare la tua Azione di Reazione per spendere 1 carica per riuscire il Tiro Salvezza che hai appena fallito. Questo anello ha 3 cariche, e recupera 1d3 cariche spese ogni mattina all'alba.

\index[OggettiMagici]{Anello dell'Evocazione dello Djinni}\subsubsection*{Anello dell'Evocazione dello Djinni}
35000 mo, leggendario, mentre indossi quest'anello, puoi pronunciarne la parola di comando con due azioni per evocare uno specifico djinni del Piano Elementale dell'Aria. Lo djinni compare in uno spazio non occupato a tua scelta, entro 36 metri da te. Resta finché rimani concentrato (come se ti concentrassi su di un incantesimo), per un massimo di 1 ora, o finché non scende a 0 Punti Ferita. Poi ritorna al suo piano natio.

Finché resta evocato, lo djinni è amichevole verso di te e i tuoi compagni. Obbedisce a qualsiasi comando gli dai, non importa la lingua usata. Se non gli impartisci ordini, lo djinni si difenderà dagli attacchi ma non effettuerà nessun'altra azione.

Dopo la partenza dello djinni, esso non potrà più essere evocato prima che siano passate 24 ore, e se lo djinni muore l'anello perde la sua magia.

\index[OggettiMagici]{Anello di Influenza sugli Animali}\subsubsection*{Anello di Influenza sugli Animali} \textit{Anello, raro} 4000 mo

Mentre indossi questo anello, puoi usare due azioni per spendere 1 delle sue cariche per lanciare tramite esso uno dei seguenti incantesimi: amicizia con gli animali (DC del Tiro Salvezza 15), parlare con gli animali, paura (DC del Tiro Salvezza 15, prende come bersaglio solo bestie che hanno Intelligenza -2 o meno).

Questo anello ha 3 cariche, e recupera 1d3 cariche spese ogni giorno all'alba.

\subsubsection*{Anello dell’Inganno}\index[OggettiMagici]{Anello dell’Inganno}
raro, chi indossa questo anello maledetto è convinto che abbia un potere scelto dal Narratore o determinato casualmente.

\index[OggettiMagici]{Anello di Invisibilità}\subsubsection*{Anello di Invisibilità}
10000 mo, molto raro, mentre indossi quest'anello, puoi renderti invisibile con due azioni. Tutto ciò che indossi o trasporti diventa invisibile assieme a te. Resti invisibile finché l'anello non viene rimosso, attacchi o lanci un incantesimo, o finché non usi due azioni per tornare visibile.

\index[OggettiMagici]{Anello di Libertà di Azione}\subsubsection*{Anello di Libertà di Azione}
20000 mo, raro, mentre indossi questo anello, il terreno difficile non ti costa movimento aggiuntivo. Inoltre, la magia non può né ridurre la tua velocità né renderti paralizzato o intralciato.

\index[OggettiMagici]{Anello del Nuoto}\subsubsection*{Anello del Nuoto}
3000 mo, non comune, mentre indossi questo anello, hai velocità di nuoto 12 metri.

\index[OggettiMagici]{Anello di Protezione}\subsubsection*{Anello di Protezione}
costo vario, rarità varia, mentre indossi questo anello, hai un bonus da +1 (5000 mo, raro), +2 (7500 mo, raro), +3 (12000mo, molto raro) alla Difesa e ai Tiri Salvezza.

\index[OggettiMagici]{Anello Respingi Incantesimi}\subsubsection*{Anello Respingi Incantesimi}
35000 mo, leggendario, mentre indossi quest'anello, hai +1d6 ai Tiri Salvezza contro qualsiasi incantesimo che prende come bersaglio solo te e non un'area di effetto. Inoltre, se fai un successo critico salvezza e l'incantesimo ha livello 6 o più basso, l'incantesimo non ha effetto su di te e invece prende come bersaglio l'incantatore che ha lanciato l'incantesimo

\index[OggettiMagici]{Anello di Rigenerazione}\subsubsection*{Anello di Rigenerazione}
12000 mo, molto raro, mentre indossi questo anello, recuperi 1d6 Punti Ferita ogni 10 minuti, purché ti rimanga almeno 1 punto ferita. Se perdi una parte del corpo, l'anello fa sì che la parte mancante ricresca e ritorni alla sua completa funzionalità in 1d6 + 1 giorni, purché per tutto il periodo ti rimanga sempre almeno 1 punto ferita.

\index[OggettiMagici]{Anello di Resistenza}\subsubsection*{Anello di Resistenza}
6000 mo, raro, mentre indossi questo anello, hai resistenza a un tipo di danno. La gemma incastonata nell'anello indica il tipo di danno, che viene scelto o determinato casualmente dal Narratore.

\medskip

\begin{tabular}{lll}
\textbf{d10} & \textbf{Tipo di Danno} & \textbf{Gemma}\\

\hline
1 &Acido &Perla\\
2& Forza &Zaffiro\\
3& Freddo &Tormalina\\
4& Fulmine &Citrino\\
5& Fuoco &Granato\\
6& Vuoto& Giaietto\\
7& Energia Positiva &Giada\\
8& Luce &Topazio\\
9& Suono &Spinello\\
10& Energia Negativa &Ametista\\
\end{tabular}

\medskip

\index[OggettiMagici]{Anello del Salto}\subsubsection*{Anello del Salto}
2500 mo, non comune, mentre indossi questo anello, con due azioni puoi lanciare tramite esso l'incantesimo saltare a volontà, ma il bersaglio puoi essere solo tu.

\index[OggettiMagici]{Anello dello Scudo Mentale}\subsubsection*{Anello dello Scudo Mentale}
16000 mo, non comune, mentre indossi questo anello, sei immune alla magia che permette alle altre creature di leggere i tuoi pensieri, determinare se stai mentendo, conoscere i tuoi Tratti, o apprendere che tipo di creatura sei. Le creature possono comunicare telepaticamente con te solo se glielo concedi.

Puoi usare due azioni per far diventare invisibile l'anello fino a che un'altra azione non lo renderà di nuovo visibile, finché non lo rimuovi o muori. Se muori mentre indossi questo anello, la tua anima vi viene catturata, a meno che non ospiti già un'altra anima. Puoi decidere di rimanere nell'anello o raggiungere la vita ultraterrena. Finché la tua anima resta nell'anello, puoi comunicare telepaticamente con qualsiasi creatura lo indossi. Chi lo indossa non può impedire questa forma di comunicazione telepatica.

\index[OggettiMagici]{Anello delle Stelle Cadenti}\subsubsection*{Anello delle Stelle Cadenti}
14000 mo, molto raro, mentre indossi questo anello a luce fioca o all'oscurità, puoi lanciare tramite esso luci danzanti e luce a volontà. Lanciare uno dei due incantesimi tramite l'anello richiede due azioni. L'anello ha 6 cariche per le seguenti altre proprietà.

L'anello recupera 1d6 cariche spese ogni giorno all'alba.

\textit{Luminescenza}. Spendi 1 carica con due azioni per lanciare tramite l'anello l'incantesimo luminescenza.

\textit{Palla di fulmini}. Puoi spendere 2 cariche con due azioni per creare da una a quattro sfere di fulmini di 1 metro di diametro. Più sfere crei, meno potente sarà ciascuna sfera individualmente.
Ogni sfera compare in uno spazio non occupato visibile entro 36 metri da te. La sfera dura finché ti concentri su di essa (come se ti concentrassi su di un incantesimo), fino a un massimo di 1 minuto. Ogni sfera irradia luce fioca in un raggio di 9 metri. Con due azioni puoi muovere ciascuna sfera di massimo 9 metri, ma senza superare i 36 metri di distanza da te. Quando una creatura, a parte te, si trova entro 1,5 metri da una sfera, la sfera scarica i fulmini contro quella creatura e poi scompare. Quella creatura deve effettuare un Tiro Salvezza su Riflessi con DC 18. Se fallisce il Tiro Salvezza, la creatura subisce danni da fulmine in base al numero di sfere da te creato (4 sfere, 2d4 danni; 3 sfere, 2d6 danni; 2 sfere, 5d4 danni; 1 sfera, 4d12 danni).

\textit{Stelle Cadenti}. Puoi spendere da 1 a 3 cariche con due azioni. Per ogni carica spesa, scagli una scintilla di luce dall'anello in un punto visibile entro 18 metri da te. Ogni creatura, in cubo di 4,5 metri di lato originante da quel punto, viene ricoperta di scintille e deve effettuare un Tiro Salvezza di Destrezza DC 15, subendo 5d4 danni da fuoco se lo fallisce, o la metà di questi danni se lo supera.

\index[OggettiMagici]{Anello di Telecinesi}\subsubsection*{Anello di Telecinesi}
80000 mo, molto raro, mentre indossi questo anello, puoi lanciare a volontà l'incantesimo telecinesi, ma puoi prendere come bersaglio solo oggetti che non siano indossati o trasportati.

\index[OggettiMagici]{Anello della Vista ai Raggi X}\subsubsection*{Anello della Vista ai Raggi X}
6000 mo, raro, mentre indossi questo anello, puoi usare due azioni per pronunciarne la parola di comando. Quando lo fai, puoi vedere attraverso la materia solida per 1 minuto. Questa vista ha un raggio di 9 metri. Per te, gli oggetti solidi all'interno del raggio appaiono trasparenti e non impediscono alla luce di attraversarli.

Questa vista può penetrare 30 centimetri di pietra, 2,5 centimetri di metallo comune o fino a 90 centimetri di legno o terra. Le sostanze più dense bloccano la vista, così come un sottile foglio di piombo. Ogni qualvolta usi di nuovo l'anello prima di aver terminato una notte di riposo devi superare un Tiro Salvezza su Tempra con DC 18 o diventare affaticato.

\subsection{Cappelli, Mantelli, Occhiali, Tuniche}


\index[OggettiMagici]{Bandana dell'Intelligenza}\subsubsection*{Bandana dell'Intelligenza}
8000 mo, raro, mentre indossi questa bandana il tuo Intelligenza è +4. La fascetta non ha effetto se hai già Intelligenza è già +4 o più alta.

\index[OggettiMagici]{Cappello del Camuffamento}\subsubsection*{Cappello del Camuffamento}
5000 mo, non comune, mentre indossi questo cappello, puoi usare due azioni per lanciare a volontà l'incantesimo camuffare sé stesso. L'incantesimo termina quando il cappello viene rimosso.

\index[OggettiMagici]{Mantello dell'Aracnide}\subsubsection*{Mantello dell'Aracnide}
8000 mo, molto raro, mentre indossi questo elegante abito di seta nera intessuto con fili d'argento, ottieni i seguenti benefici:

\medskip

\begin{itemize}
\item
Hai resistenza ai danni da veleno.
\item
Hai velocità di scalata pari alla tua velocità di passeggio.
\item
Puoi muoverti verso l'alto, il basso e lungo superfici verticali e a testa in giù sui soffitti, tenendo le mani libere.
\item
Non puoi essere catturato da alcuna sorta di ragnatela e ti muovi attraverso le ragnatele come fossero terreno difficile.
\item
Puoi usare due azioni per lanciare l'incantesimo ragnatela (DC del Tiro Salvezza 15). La ragnatela creata dall'incantesimo riempie il doppio della sua normale area. Una volta usata, questa proprietà della Mantello non può essere usata di nuovo fino alla prossima alba.
\end{itemize}

\index[OggettiMagici]{Mantella del Ciarlatano}\subsubsection*{Mantella del Ciarlatano}
8000 mo, raro, mentre indossi questa mantella che odora lievemente di zolfo, puoi usarla per lanciare l'incantesimo porta dimensionale con due azioni. La proprietà di questa mantella non può essere usata di nuovo fino all'alba. Quando scompari, ti lasci alle spalle una nube di fumo, e riappari alla tua destinazione all'interno di una simile nube di fumo. Questo fumo oscura leggermente lo spazio che hai lasciato e quello dove riappari, e si dissipa alla fine del tuo prossimo round. Un vento leggero o più forte disperde il fumo.

\index[OggettiMagici]{Mantello di Distorsione}\subsubsection*{Mantello di Distorsione}
60000 mo, raro, mentre indossi questa Mantello, essa proietta un'illusione che ti fa apparire come se stessi in un punto vicino alla tua reale posizione, facendo sì che tutte le creature abbiano -1d6 ai Tiri per Colpire contro di te. Se subisci danni, la proprietà cessa di funzionare fino all'inizio del tuo prossimo round. Questa proprietà è soppressa mentre sei inabile, intralciato o altrimenti impossibilitato a muoverti.

\index[OggettiMagici]{Mantello degli Elfi}\subsubsection*{Mantello degli Elfi}
5000 mo, non comune, mentre indossi questo Mantello tirando su il cappuccio, le prove di Consapevolezza effettuate per notarti hanno -1d6, e hai +1d6 alle prove di Destrezza effettuate per nasconderti. Tirare su o giù il cappuccio richiede due azioni.

\index[OggettiMagici]{Mantello della Manta}\subsubsection*{Mantello della Manta}
6000 mo, non comune, mentre indossi questa Mantello con il cappuccio tirato su, puoi respirare sott'acqua e hai velocità di nuoto 18 metri. Tirare su o giù il cappuccio richiede 1 azione.

\index[OggettiMagici]{Mantello del Pipistrello}\subsubsection*{Mantello del Pipistrello}
6000 mo, raro, mentre indossi questa Mantello, hai +1d6 alle prove di Destrezza. In aree di luce fioca o oscurità, puoi afferrare i bordi della Mantello con entrambe le mani e usarla per muoverti a velocità di volo 12 metri. Se dovessi smettere di tenere i bordi della Mantello mentre voli a questo modo, perdi la tua velocità di volo. Mentre indossi la Mantello in un'area di luce fioca o oscurità, puoi usare la tua azione per lanciare metamorfosi su di te, trasformandoti in un pipistrello. Quando sei in forma di pipistrello, mantieni i tuoi punteggi di Intelligenza, Saggezza e Carisma. La Mantello non può essere impiegata di nuovo in questo modo fino alla prossima alba.

\index[OggettiMagici]{Mantello di Protezione}\subsubsection*{Mantello di Protezione}
rarità varia, costo vario, mentre indossi questa Mantello, ottieni un bonus di +1 (non comune, 3500 mo),+2 (raro, 6000 mo),+3 (molto raro, 15000 mo) alla Difesa e ai Tiri Salvezza.

\index[OggettiMagici]{Mantello della Resistenza agli Incantesimi}\subsubsection*{Mantello della Resistenza agli Incantesimi}
non comune, 3000 mo, mentre indossi questa Mantello, hai +2 ai Tiri Salvezza contro incantesimi.


\subsubsection*{Mantello della velenosità}\index[OggettiMagici]{Mantello della velenosità}
raro, 4000 mo, questo mantello è di solito fatto di lana, sebbene possa essere anche di pelle. L’indumento può essere manipolato senza pericolo, ma appena viene indossato causa 5d6 di danno da veleno. Ogni round successivo può essere fatto un Tiro Salvezza su Tempra DC 21 per ridurre di 1d6 il danno fino ad un minimo di 1d6 di danno rimanente. Il mantello può essere rimosso soltanto con un incantesimo Rimuovi Maledizione o desiderio.

\subsubsection*{Occhi della pietrificazione}\index[OggettiMagici]{Occhi della pietrificazione}
queste due lenti di cristallo magico si sovrappongono alle pupille degli occhi. Quando una creatura mette queste lenti, viene immediatamente pietrificata senza Tiro Salvezza. Circa un quarto di questi oggetti (probabilità del 25\%) consentono invece a chi le mette di pietrificare con lo sguardo, ma in questo caso le vittime hanno diritto a un Tiro Salvezza. Non è possibile combinare due tipi di lenti magiche.

\index[OggettiMagici]{Occhi Affascinanti}\subsubsection*{Occhi Affascinanti}
3000 mo, non comune, mentre indossi queste lenti di cristallo davanti agli occhi, puoi spendere 1 carica con due azioni per lanciare l'incantesimo charme su persone (DC del Tiro Salvezza 15) su di un umanoide entro 9 metri da te, purché tu e il bersaglio vi possiate vedere. Le lenti hanno 3 cariche e recuperano 1 carica di quelle spese ogni giorno all'alba.

\index[OggettiMagici]{Occhi dell'Aquila}\subsubsection*{Occhi dell'Aquila}
4500 mo, non comune, mentre indossi queste lenti di cristallo davanti agli occhi, hai +1d6 alle prove di Consapevolezza basate sulla vista. In condizioni di visibilità limpida, puoi distinguere i dettagli anche di creature e oggetti molto distanti delle dimensioni di 50 centimetri.

\index[OggettiMagici]{Occhi della Vista Dettagliata}\subsubsection*{Occhi della Vista Dettagliata}
2500 mo, non comune, mentre indossi queste lenti di cristallo davanti agli occhi, puoi vedere molto meglio del normale fino a una distanza di 30 centimetri. Hai +1d6 alle prove di Consapevolezza basata su vista mentre perlustri un'area o studi un oggetto a distanza ravvicinata.

\index[OggettiMagici]{Occhiali da Notte}\subsubsection*{Occhiali da Notte}
1500 mo, non comune, mentre indossi queste lenti scure, possiedi la scurovisione, con una gittata di 18 metri. Se già possiedi la scurovisione, indossare questi occhiali ne aumenta la gittata di 18 metri.

\subsubsection*{Tunica del Mimetismo}\index[OggettiMagici]{Tunica del Mimetismo}
1500 mo, raro, quando indossa questa tunica, un personaggio capisce immediatamente il suo potere. Una tunica del mimetismo permette al personaggio di confondersi con l’ambiente circostante, qualunque esso sia, e di nascondersi. Ha +1d6 nelle prove di Nascondersi nelle ombre. Il possessore può assumere a volontà l'aspetto di un altro umanoide, come per l'incantesimo Alterare se stesso (Cambio Aspetto). In questo caso, gli amici del possessore e chi lo conosce molto bene sono istintivamente consci della sua vera identità.

\subsubsection*{Tunica dell’Arcimago}\index[OggettiMagici]{Tunica dell’Arcimago}
8000 mo, leggendario, questo abito apparentemente normale può essere giallo (01-45 su 1d100), grigio (46-75) o nero (76-00). Può essere indossato solo da un incantatore con Competenza Magica 2 o superiore. Conferisce le seguenti bonus:

- Difesa 15

- +2 ai Tiri Salvezza contro incantesimi e oggetti magici

\index[OggettiMagici]{Tunica dei Colori Scintillanti}\subsubsection*{Tunica dei Colori Scintillanti}
6000 mo, molto raro, questa vestaglia ha 3 cariche, e recupera 1d3 cariche spese ogni giorno all'alba. Quando la indossi, puoi usare due azioni e spendere 1 carica per far sì che l'indumento produca una trama mutevole di colori abbaglianti fino al termine del tuo prossimo round. Durante questo periodo, la vestaglia emana luce intensa in un raggio di 9 metri e luce fioca per ulteriori 9 metri. Le creature che ti vedono hanno -1d6 ai Tiri per Colpire contro di te. Inoltre, qualsiasi creatura sotto la luce intensa e che ti veda quando il potere della vestaglia viene attivato, deve superare un Tiri Salvezza su Volontà con DC 17 o restare stordita fino al termine dell'effetto.

\subsubsection*{Tunica dell’Indebolimento}\index[OggettiMagici]{Tunica dell’Indebolimento}
5000 mo, raro, una tunica dell’indebolimento sembra un abito magico di un altro tipo. Appena un personaggio la indossa, la sua Forza e la sua Intelligenza scendono a -3 ed egli perde la capacità di lanciare incantesimi. La tunica può essere rimossa facilmente, ma per ripristinare gli attributi occorre Rimuovi Maledizione seguito da guarigione.

\index[OggettiMagici]{Tunica degli Occhi}\subsubsection*{Tunica degli Occhi}
30000 mo, raro, questa vestaglia è adornata da un disegno di occhi. Mentre la indossi, ottieni i seguenti benefici:

- La vestaglia ti permette di vedere in tutte le direzioni e hai +1d6 alle prove di Consapevolezza basate sulla vista.

- Hai scurovisione con una portata di 36 metri.

- Puoi vedere creature e oggetti invisibili, oltre che nel Piano Etereo, fino a una gittata di 36 metri.

Gli occhi della vestaglia non possono essere chiusi o distolti, e mentre indossi questa vestaglia non viene mai considerato a occhi chiusi o distolti.

L'incantesimo luce lanciato sulla vestaglia o l'incantesimo luce diurna lanciato entro 1,5 metri dalla vestaglia ti rendono accecato per 1 minuto. Al termine di ciascun tuo round, puoi effettuare un tiro Salvezza su Tempra (DC 13 per luce o DC 17 per luce diurna), ponendo fine alla condizione accecato in caso lo superi.

\index[OggettiMagici]{Tunica degli Oggetti Utili}\subsubsection*{Tunica degli Oggetti Utili}
300 mo, non comune, mentre indossi questa vestaglia ricoperta da toppe di varie forme e colori, puoi usare due azioni per staccare una delle toppe, facendola diventare l'oggetto o la creatura che rappresenta. Quando l'ultima toppa viene rimossa, la vestaglia diventa un indumento normale. La vestaglia possiede due di ciascuna delle seguenti toppe:

Asta di 3 metri, Corda di canapa (15 metri,arrotolata), Lanterna a lente sporgente (piena e accesa), Pugnale, Sacco, Specchio d'acciaio.

Inoltre, la vestaglia ha 4d4 altre toppe. Il Narratore sceglie le toppe o le determina a caso, scegliendo tra proprietà totalmente diverse da quelle già presenti.

Tira un d100 sulla tabella seguente per scoprire le proprietà delle altre 4d4 toppe della vestaglia degli oggetti utili.

\end{multicols}

\medskip
\begin{tabularx}{0.95\textwidth}{lX}
\textbf{d100} & \textbf{Effetto}\\
\hline
01-08 &Borsello con 100 mo.\\
09-15& Forziere d'argento (lungo 30 cm, largo e profondo 15 cm) del valore di 500 mo.\\
16-22& Porta di ferro (larga e alta massimo 3 metri, sbarrata dal lato di tua scelta), che puoi piazzare su qualsiasi apertura a portata; si adatta per entrare nell'apertura, fissandosi e creando dei cardini.\\
23-30 &10 gemme del valore di 100 mo ciascuna.\\
31-44 &Una scala di legno (7,5 metri).\\
45-51 &Un cavallo da corsa con sacche da sella 52-59 Fossa (un cubo di 3 metri di spigolo), che puoi piazzare sul terreno entro 3 metri da te.\\
60-68 &4 pozioni di guarigione. \\
69-75 &Barca a remi (lunga 3,5 metri).\\
76-83& Pergamena degli incantesimi contenente un incantesimo di livello dal 1° al 3°.\\
84-90& Due mastini.\\
91-96 &Finestra (60 x 120 cm, profonda massimo 60 cm), che puoi piazzare su qualsiasi superficie verticale a portata.\\
97-100 &Ariete portatile.\\
\end{tabularx}

\begin{multicols}{2}

\medskip

\index[OggettiMagici]{Tunica delle Stelle}\subsubsection*{Tunica delle Stelle}
60000 mo, raro, mentre indossi questa vestaglia, ottieni un bonus di +1 ai Tiri Salvezza. Sei stelle, posizionate sulla parte superiore frontale della vestaglia, sono più grosse delle altre. Mentre indossi questa vestaglia, puoi usare due azioni per estrarre una delle stelle e usarla per lanciare dardo incantato. Ogni giorno al tramonto, la stella rimossa ricompare sulla vestaglia. Mentre indossi la vestaglia, puoi usare due azioni per entrare nel Piano Astrale assieme a tutto ciò che indossi o trasporti. Resterai lì fino a quando userai due azioni per ritornare al piano in cui ti trovavi prima. Ricompari nell'ultimo spazio da te occupato, o se quello spazio è occupato, nello spazio non occupato più vicino.

\subsection{Manuali, Tomi, Libri}


\index[OggettiMagici]{Manuale dei Golem}\subsubsection*{Manuale dei Golem}
10000 mo, molto raro, questo tomo contiene le informazioni e incantamenti necessari a costruire un tipo particolare di golem. Il Narratore sceglie il tipo di golem che è possibile costruire o lo determina casualmente. Per decifrare e usare il manuale devi avere almeno Competenza Magica 10. Una creatura che non possa usare il manuale dei golem e provi a leggerlo, subisce 6d6 danni da forza.

\medskip

\begin{tabular}{llll}
3d6 &Golem &Tempo &Costo\\
\hline
3-4 &Argilla &30 giorni &65000 mo\\
5-16 &Carne &60 giorni& 50000 mo\\
17 &Ferro &120 giorni &100000 mo\\
18 &Pietra& 90 giorni &80000 mo\\
\end{tabular}


\medskip

Per creare un golem, devi trascorrere il tempo sopra indicato, lavorando senza interruzione con il manuale a disposizione e riposando per non più di 8 ore al giorno. Devi anche pagare il costo specificato per acquistare i materiali necessari.

Una volta finito di creare il golem, il libro viene consumato da fiamme arcane. Il golem si anima quando le ceneri del manuale saranno sparse su di esso. Sarà sotto il tuo controllo e comprende e obbedisce gli ordini pronunciati da te.

\subsubsection*{Manuale della Buona salute}\index[OggettiMagici]{Manuale della buona salute}
15000 mo, molto raro, questo tomo contiene istruzioni per rafforzare il corpo e la salute. Per leggere il libro occorrono 24 ore in un minimo di 3 giorni. Le sue istruzioni andranno seguite per 4 settimane, al termine delle quali il lettore guadagnerà permanentemente un punto di Costituzione. Una volta letto, il libro perde il suo potere magico e il lettore non potrà mai più usarne uno simile.

\subsubsection*{Manuale della Velocità di azione}\index[OggettiMagici]{Manuale della velocità di azione}
15000 mo, molto raro, questo tomo contiene esercizi per l'equilibrio e la coordinazione. Funziona come un manuale della buona salute, ma fa ottenere un punto di Destrezza.

\subsubsection*{Manuale dell'Esercizio fisico}\index[OggettiMagici]{Manuale dell'esercizio fisico}
15000 mo, molto raro, questo tomo funziona esattamente come il manuale della salute, ma conferisce al lettore un punto di Forza.

\index[OggettiMagici]{Tomo dell'Autorità e dell'Influenza}\subsubsection*{Tomo dell'Autorità e dell'Influenza}
15000 mo, molto raro, questo libro contiene indicazioni su come influenzare e affascinare il prossimo, e le sue parole sono soffuse di magia. Se trascorri 48 ore in un periodo di 6 giorni o meno a studiare i contenuti del libro e praticarne le indicazioni, il tuo punteggio di Carisma aumenta di 1. Poi il manuale perde la sua magia, per recuperarla dopo un secolo.

\index[OggettiMagici]{Tomo della Comprensione}\subsubsection*{Tomo della Comprensione}
15000 mo, molto raro, questo libro contiene esercizi di intuizione e discernimento, e le sue parole sono soffuse di magia. Se trascorri 48 ore in un periodo di 6 giorni o meno a studiare i contenuti del libro e praticarne le indicazioni, il tuo punteggio di Saggezza aumenta di 1, e così fa il tuo punteggio massimo per quella caratteristica. Poi il manuale perde la sua magia, per recuperarla dopo un secolo.

\index[OggettiMagici]{Tomo del Pensiero Limpido}\subsubsection*{Tomo del Pensiero Limpido}
15000 mo, molto raro, questo libro contiene esercizi di memoria e logica, e le sue parole sono soffuse di magia. Se trascorri 48 ore in un periodo di 6 giorni o meno a studiare i contenuti del libro e praticarne le indicazioni, il tuo punteggio di Intelligenza aumenta di 1. Poi il manuale perde la sua magia, per recuperarla dopo un secolo.

\subsection{Oggetti Magici Vari}

\subsubsection*{Acqua purificatrice}\index[OggettiMagici]{Acqua purificatrice}
500 mo, raro, questo liquido dolce può essere usato per purificare l’acqua (anche per desalinizzare quella di mare) e per trasformare veleni, acidi e altri liquidi nocivi in una bevanda potabile. Inoltre, l’acqua purificatrice neutralizza l'efficacia di ogni altra pozione. Questa pozione può trasformare fino a 1000 metri cubi di quasi tutti i liquidi a base d’acqua, ma solo 10 metri cubi di acido. Gli effetti sono permanenti e un liquido purificato non può essere deteriorato o contaminato di nuovo per un periodo di 5d4 round.

\index[OggettiMagici]{Ali del Volo}\subsubsection*{Ali del Volo}
54000 mo, leggendario, mentre indossi questo Mantello, puoi usare due azioni per pronunciare la sua parola di comando, trasformandola in un paio di ali da pipistrello o da uccello che spuntano dalla tua schiena per 1 ora o finché non ripeti la parola di comando con un'azione. Le ali ti forniscono velocità di volo 18 metri. Quando scompaiono, non potrai più usarle fino all'alba del giorno dopo ore.

\index[OggettiMagici]{Ampolla di Ferro}\subsubsection*{Ampolla di Ferro}
35000 mo, leggendario, questa bottiglia di ferro ha un tappo di ottone. Puoi usare due azioni per pronunciare la parola di comando dell'ampolla, prendendo come bersaglio una creatura visibile entro 18 metri da te. Se il bersaglio è nativo di un piano di esistenza diverso da quello in cui ti trovi, deve superare un Tiro Salvezza su Volontà con DC 21 o venir intrappolato nell'ampolla. Se il bersaglio è già stato intrappolato nell'ampolla, riceve +1d6 al Tiro Salvezza. Una volta intrappolata, la creatura rimarrà nell'ampolla finché non sarà liberata. L'ampolla può contenere solo una creatura alla volta. Una creatura intrappolata nell'ampolla non ha bisogno di respirare, mangiare o dormire e non invecchia. Puoi usare due azioni per rimuovere il tappo dell'ampolla e liberare la creatura che contiene. La creatura sarà amichevole verso di te e i tuoi compagni per 1 ora e obbedirà ai vostri comandi per quella durata. Se non le impartisci comandi o gliene dai uno che provocherebbe la sua morte, si difenderà ma non compirà altre azioni. Al termine della durata, la creatura agirà in base al suo normale comportamento

L'incantesimo identificare rivela che una creatura si trova all'interno dell'ampolla, ma l'unico modo per determinare che sorta di creatura sia è di aprire l'ampolla. Un'ampolla di ferro appena scoperta potrebbe già contenere una creatura scelta dal Narratore o determinata casualmente.

\medskip

\begin{tabular}{ll}
\hline
d100 &Contiene\\
1-50 &Vuota\\
51-66 &Demone \\
67 &Angelo Deva\\
68-69 &Diavolo (superiore)\\
70-73 &Diavolo (inferiore)\\
74-75 &Genio Djinni\\
76-77 &Genio Efreeti\\
78-83 &Elementale (qualsiasi)\\
84-86 &Persecutore invisibile\\
87-90 &Megera notturna\\
91 &Angelo Planetar\\
92-95 &Salamandra\\
96 &Angelo Solar\\
97-99 &Succube/Incubo\\
100 &Xorn\\
\end{tabular}
\medskip


\subsubsection*{Anfora elementale dell’acqua}\index[OggettiMagici]{Anfora elementale dell’acqua}
2500 mo, raro, quest'anfora può essere usata per evocare e controllare un elementale dell’acqua in modo analogo all’incantesimo evoca elementale. È necessario preparare l'oggetto magico e condurre un rituale per un turno prima dell’evocazione vera e propria, che richiede un round. Dopo che l’elementale è stato evocato, occorre mantenere la concentrazione per potergli impartire gli ordini. L'anfora è usabile una volta al giorno.

\index[OggettiMagici]{Apparato del Granchio}\subsubsection*{Apparato del Granchio}
15000 mo, leggendario, quest'oggetto appare come un barile di ferro sigillato di taglia Grande e del peso di 250 chili. Il barile nasconde un fermo, che può essere trovato superando una prova di Intelligenza con DC 25. Rimuovere il fermo apre uno scomparto a una delle estremità dell'apparato, che permette a due creature di taglia Media o inferiore di entrarvi dentro. All'estremità opposta sono disposte dieci leve, ciascuna in posizione neutrale, in grado di muoversi verso l'alto o il basso. Quando vengono impiegate determinate leve, l'apparato si trasforma e assomiglia a un'aragosta gigante.

L'apparato è un oggetto Grande con le seguenti statistiche.

Difesa: 20, Punti Ferita: 200, Velocità: 9 m, nuoto 9 m (o 0 m entrambi se le gambe e la coda non vengono estese)

Immunità ai Danni: veleno

Per essere usato come veicolo, l'apparato necessita un pilota. Quando lo sportello dell'apparato viene chiuso, il compartimento è a tenuta stagna, e non fa filtrare aria o acqua. I compartimenti conservano aria sufficiente per 10 ore, divise per il numero di creature all'interno. L'apparato galleggia in acqua e può anche spingersi sott'acqua fino a una profondità di 270 metri. Al di sotto di questa soglia, l'apparato subisce 2d6 danni da botta al minuto a causa della pressione. Una creatura all'interno del compartimento può usare due azioni per muovere verso l'alto o il basso fino a due leve. Dopo ciascun uso, la leva torna alla sua posizione neutrale. Ogni leva, da sinistra a destra, funziona come mostrato sulla tabella seguente.

1: Estende gambe e coda, permettendo all'apparato di camminare e nuotare. Ritrae gambe e coda, riducendo la velocità dell'apparato a 0 e rendendolo incapace di beneficiare di bonus alla velocità.

2: Apre l'oblò frontale. Chiude l'oblò frontale.

3: Apre gli oblò laterali (due per lato). Chiude gli oblò laterali (due per lato).

4: Estende due chele dal lato frontale dell'apparato. Ritrae le chele.

5: Effettua un attacco con arma da mischia con ciascuna chela estesa: +8 al Tiro per Colpire, portata 1,5 m, un bersaglio. Colpisce: 7 (2d6) danni da botta. Effettua un attacco con arma da mischia con ciascuna chela estesa: +8 al Tiro per Colpire, portata 1,5 m, un bersaglio. Colpisce: Il bersaglio è afferrato (DC 18 per fuggire).

6: L'apparato cammina o nuota in avanti. L'apparato cammina o nuota indietro.

7: L'apparato svolta di 90 gradi a sinistra. L'apparato svolta di 90 gradi a destra.

8: Delle fessure frontali emettono luce intensa in un raggio di 9 metri e luce fioca per ulteriori metri. Spegne le luci.

9: L'apparato affonda di 6 metri nei liquidi. L'apparato risale di 6 metri dai liquidi.

10: Sblocca e apre il portellone posteriore. Chiude e sigilla il portellone posteriore.


\subsubsection*{Ampolla delle maledizioni}\index[OggettiMagici]{Ampolla delle maledizioni}
800 mo, raro, questo oggetto ha l’aspetto di una ampolla, bottiglia, caraffa, contenitore, fiasco, o brocca. Può contenere un liquido o emanare fumo. Quando l’ampolla viene stappata per la prima volta, tutte le creature entro 9 m vengono maledette.

\index[OggettiMagici]{Barca Pieghevole}\subsubsection*{Barca Pieghevole}
12000 mo, raro, questo oggetto sembra una scatola di legno che misura 30 centimetri di lunghezza, 15 centimetri di larghezza e 15 centimetri di profondità. Pesa 2 chili e galleggia. Può essere aperta per porvi oggetti all'interno. Quest'oggetto possiede tre parole di comando, ciascuna delle quali richiede due azioni per essere pronunciata. Una parola di comando fa sì che la scatola si dispieghi in una barca lunga 3 metri, larga 1,2 metri e profonda 50 centimetri. La barca ha un paio di remi, un'ancora, un albero e una vela. La barca può contenere fino a quattro creature di taglia Media.

La seconda parola di comando fa sì che la scatola si dispieghi in una nave lunga 7,2 metri, larga 2,5 metri e profonda 2 metri. La nave ha un ponte, file di voga, cinque serie di remi, un timone direzionale, un'ancora, una cabina e un albero con la vela quadrata. La nave può contenere quindici creature di taglia Media.

La terza parola di comando fa sì che la barca pieghevole ritorni a piegarsi nella scatola, purché nessuna creatura sia a bordo. Qualsiasi oggetto a bordo che non possa entrare nella scatola resta fuori della scatola mentre questa si piega. Qualsiasi oggetto a bordo che possa entrare nella scatola, vi entra.

\subsubsection*{Bacinella dell’annegamento}: questa bacinella maledetta ha l'apparenza di un'anfora elementale dell'acqua. Tuttavia, invece di evocare un elementale, sprigiona un globo d’acqua che avvolge la testa del personaggio. Questi affoga in 2d4 round a meno che non riesca in un Tiro Salvezza contro incantesimi. L'acqua è “appiccicosa” e può essere rimossa solo con la magia (dispersione della magia o distruggere acqua).

\index[OggettiMagici]{Battaglio dell'Apertura}\subsubsection*{Battaglio dell'Apertura}
1500 mo, raro, questo tubo metallico cavo misura circa 30 centimetri di lunghezza e pesa 0,5 chili. Puoi batterlo con due azioni, puntandolo verso un oggetto entro 36 metri che può essere aperto, come una porta o una serratura. Il battaglio emette un suono limpido, e una serratura o laccio dell'oggetto si apre a meno che il suono sia impedito dal raggiungere l'oggetto. Se non rimangono serrature o lacci da aprire, l'oggetto si apre da sé.

Il battaglio può essere usato dieci volte. Dopo la decima, si spacca e diventa inutilizzabile.

\subsubsection*{Battaglio del Cannibalismo}\index[OggettiMagici]{Battaglio del Cannibalismo}
questo oggetto sembra una Battaglio dell'apertura. Funziona come tale per il primo round di uso (ed ha 1d4x10 cariche per questo scopo). Tuttavia, al secondo tintinnio tutte le creature entro 18 m devono riuscire in un Tiro Salvezza su Volontà DC 21 o cadere preda di una fame vorace, attaccando il più vicino umanoide per ucciderlo e divorarlo. A round alterni è permesso un nuovo Tiro Salvezza. Se non sono presenti umanoidi, le creature affette attaccheranno le altre creature presenti.

\index[OggettiMagici]{Borsa Conservante}\subsubsection*{Borsa Conservante}

Sussistono diverse tipologie di Borse Conservanti e tutte hanno in comune la capacità di poter contenere molto di più di quello che dovrebbero date le loro dimensioni.

Le Borse Conservanti si suddividono in 4 tipi (Tipo I, II, III; IV) a seconda dalla capacità di conservazione che hanno.

Se la borsa è sovraccarica, perforata o strappata, la borsa si rompe ed è distrutta e il suo contenuto sparpagliato per il Piano Astrale. Se la borsa viene rivoltata, i suoi contenuti vengono espulsi, illesi, ma la borsa dev'essere rimessa nel verso giusto prima che possa essere riutilizzata. Le creature che respirano, piazzate nella borsa, possono sopravvivervi per un numero di minuti pari a 10 diviso il numero di creature (minimo 1 minuto), dopodiché inizieranno a soffocare.

Piazzare una borsa conservante all'interno dello spazio extradimensionale generato da uno zainetto pratico, un buco portatile o simile oggetto, distrugge entrambi gli oggetti e apre un portale verso il Piano Astrale. Il portale origina nel punto in cui un oggetto è stato posto all'interno dell'altro. Qualsiasi creatura entro 3 metri dal portale viene risucchiata al suo interno e ricompare in un posto a caso sul Piano Astrale, poi il portale si richiude. Il portale è a senso unico e non può essere riaperto.


Alcuni incantatori preferiscono creare Bauli Conservanti, che funzionano nell'identico modo delle borse conservanti.

\index[OggettiMagici]{Borsa Conservante Tipo I}\subsubsection*{Borsa Conservante Tipo I}
500 mo, non comune, questo è il modello più piccolo delle borse conservanti. All'apparenza è un piccolo sacchetto di 20 cm di diametro con una bocca larga circa altrettanto.
Non è possibile fare entrare oggetti che abbiano una larghezza superiore ai 20 cm ed una lunghezza superiore ai 50cm.
La capacità massima è di 20 kg/Ingombro 7.

\index[OggettiMagici]{Borsa Conservante Tipo II}\subsubsection*{Borsa Conservante Tipo II}
1000 mo, non comune, questo è il modello medio delle borse conservanti. All'apparenza è un sacchetto di 40 cm di diametro con una bocca larga circa altrettanto.
Non è possibile fare entrare oggetti che abbiano una larghezza superiore ai 40 cm ed una lunghezza superiore ai 100cm.
La capacità massima è di 100 kg/Ingombro 25.

\index[OggettiMagici]{Borsa Conservante Tipo III}\subsubsection*{Borsa Conservante Tipo III}
1500 mo, raro, all'apparenza è un sacco di 80 cm di diametro con una bocca larga circa altrettanto.
Non è possibile fare entrare oggetti che abbiano una larghezza superiore ai 80 cm ed una lunghezza superiore ai 150cm. La capacità massima è di 200 kg/Ingombro 50.

\index[OggettiMagici]{Borsa Conservante Tipo IV}\subsubsection*{Borsa Conservante Tipo IV}
5000 mo, molto raro, all'apparenza è un saccone di 120 cm di diametro con una bocca larga circa altrettanto.
Non è possibile fare entrare oggetti che abbiano una larghezza superiore ai 120 cm ed una lunghezza superiore ai 200cm. La capacità massima è di 300 kg/Ingombro 75.

\index[OggettiMagici]{Borsa Divorante}\subsubsection*{Borsa Divorante}
2000 mo, raro, la borsa appare come una borsa conservante. Se la borsa viene rivolta le sue proprietà smettono di funzionare. Una creatura extradimensionale attaccata alla borsa può percepire qualsiasi cosa vi venga posto all'interno. La materia animale o vegetale posta interamente dentro la borsa viene divorata ed è persa per sempre. Quando una parte di una creatura vivente viene posta nella borsa, c'è una probabilità del 50\% che la creatura venga trascinata dentro la borsa. Una creatura all'interno della borsa può usare due azioni per cercare di fuggirne superando una prova di Forza con DC 18.

Un'altra creatura può usare due azioni per afferrare la creatura all'interno della borsa e tirarla fuori, superando una prova di Forza con DC 20 (e sempre che non venga a sua volta trascinata dentro la borsa). Qualsiasi creatura che inizi il proprio round all'interno della borsa viene divorata, il suo corpo distrutto.

All'interno della borsa possono essere posti oggetti inanimati, fino a 27 dm3 di materiale. Tuttavia, una volta al giorno, la borsa inghiotte qualsiasi oggetto posto al suo interno e lo risputa fuori in un altro piano di esistenza. Il Narratore determina il momento e il piano. Se la borsa venisse fatta a pezzi o strappata, è distrutta, e qualsiasi cosa contenga verrebbe trasportata in un luogo casuale del Piano Astrale.

\index[OggettiMagici]{Bottiglia dell'Efreeti}\subsubsection*{Bottiglia dell'Efreeti}
15000 mo, molto raro, questa bottiglia di ottone dipinta pesa 500 grammi. Quando usi due azioni per rimuoverne il tappo, una nube di denso fumo fuoriesce dalla bottiglia. Al termine del tuo round, il fumo si dissipa in un lampo di fuoco innocuo, e un efreeti compare in uno spazio non occupato entro 9 metri da te. La prima volta che la bottiglia viene aperta, il Narratore determina casualmente cosa accade.

\medskip

\begin{tabularx}{0.45\textwidth}{lX}
\textbf{3d6} &\textbf{Effetto}\\
\hline
3-5 & L'efreeti ti attacca. Dopo aver combattuto per 5 round, l'efreeti scompare e la bottiglia perde la sua magia.\\
6-16 &L'efreeti ti obbedisce per 1 ora, agendo ai tuoi comandi. Poi torna nella bottiglia, e un nuovo tappo lo può contenere. Il tappo non potrà essere rimosso prima che siano passate 24 ore. Le prossime due volte che la bottiglia viene aperta, si ripresenta lo stesso effetto. Se la bottiglia viene aperta una quarta volta, l'efreeti scappa e scompare, e la bottiglia perde la sua magia.\\
17-18 & L'efreeti può lanciare l'incantesimo desiderio a tuo favore per tre volte. Scompare quando conferisce il desiderio finale o dopo 1 ora, allorché la bottiglia perde la sua magia.
\end{tabularx}


\index[OggettiMagici]{Borsa dei Fagioli}\subsubsection*{Borsa dei Fagioli}
5000 mo, raro, all'interno di questa borsa si trovano 3d4 fagioli secchi. La borsa pesa 250 grammi più 125 grammi per ogni fagiolo che contiene.

Se riversi il contenuto della borsa sul terreno, i fagioli esplodono in un raggio di 3 metri. Ogni creatura nell'area, te compreso, deve effettuare un Tiro Salvezza di Riflessi con DC 18, subendo 5d4 danni da fuoco se lo fallisce, o la metà di questi danni se lo supera.

Il fuoco incendia gli oggetti infiammabili nell'area che non siano indossati o trasportati. Se rimuovi il fagiolo dalla borsa, lo pianti nel terreno o la sabbia, e lo innaffi, il fagiolo produrrà un effetto 1 minuto dopo, a partire dal punto del terreno in cui è stato piantato. Il Narratore sceglie l'effetto o lo determina casualmente.

\end{multicols}

\begin{center}
\includegraphics[width=0.6\linewidth]{immagini/borsetta.png}

\textit{Borsetta conservante, modello classico, Tipo II}
\end{center}


\medskip

\begin{tabularx}{0.95\textwidth}{lX}
\textbf{d100} & \textbf{Effetto}\\
\hline
01 &Spuntano 5d4 funghi. Se una creatura mangia un fungo, tira un dado. Se il risultato è dispari, costui deve superare un tiro Salvezza su Tempra con DC 15 o subire 5d6 danni da veleno e restare avvelenato per 1 ora. Se il risultato è pari, costui ottiene 5d6 Punti Ferita temporanei per 1 ora.\\
02-10 &Erutta un geyser che sputa acqua, birra, succo di frutta, tè, aceto, vino od olio (a discrezione del Narratore) 9 metri in aria per 1d12 round.\\
11-20 &Spunta un uomo albero. C'è una probabilità del 50\% che l'uomo albero sia caotico malvagio e ti attacchi.\\
21-30 &Una statua di pietra animata con le tue fattezze si leva dal terreno. Essa comincerà a minacciarti verbalmente. Se dovessi andartene e altre persone giungessero sul posto, la statua ti descriverebbe come il più pericoloso dei criminali, e li esorterebbe ad cercarti e attaccarti. Se ti trovi sullo stesso piano di esistenza della statua, essa saprà sempre dove sei. Dopo 24 ore la statua diventerà inanimata.\\
31-40 &Un fuoco da campo che produce fiamme blu spunta dal terreno e brucia per 24 ore (o finché non viene spento).\\
41-50 &Sputano 1d6 + 6 funghi urlatori.\\
51-60 &Compaiono 1d4 + 8 rospi fucsia. Ogniqualvolta un rospo viene toccato, si trasforma in un mostro di taglia Grande o inferiore a scelta del Narratore. Il mostro resta per 1 minuto e poi scompare in un sbuffo di fumo fucsia. 61-70 Un bulette esce dal terreno e attacca.\\
71-80 &Cresce un albero da frutta. Possiede 1d10 + 20 frutti. 1d8 di questi funzionano come una pozione magica determinata a caso, mentre uno di loro funge da veleno ingerito del tipo determinato dal Narratore. L'albero svanisce dopo 1 ora. I frutti raccolti invece rimangono, e mantengono la propria magia per 30 giorni. \\
81-90 &Compare un nido con 1d4 + 3 uova. Qualsiasi creatura che mangi un uovo deve effettuare un Tiro Salvezza su Tempra con DC 28. Se il Tiro Salvezza riesce, la
creatura aumenta permanentemente il suo punteggio di caratteristica più basso di 1, scegliendo casualmente in caso di parità. Se il Tiro Salvezza fallisce, la creatura subisce 10d6 danni da forza a causa di un'esplosione magica al suo interno.\\
91-99 &Spunta dal terreno una piramide dalla base quadrata di 18 metri. All'interno c'è un sarcofago che contiene una mummia sovrana. La piramide è considerata come la tana della mummia sovrana, e il suo sarcofago contiene un tesoro a scelta del Narratore.\\
100 &Un enorme pianta di fagioli cresce sul posto, fino a un'altezza a scelta del Narratore. La cima conduce dovunque voglia il Narratore, che sia il castello di un gigante delle nuvole o un altro piano di esistenza.
\end{tabularx}

\begin{multicols}{2}

\medskip

\index[OggettiMagici]{Bottiglia Fumante}\subsubsection*{Bottiglia Fumante}
1200 mo, non comune, dalla bocca di questa bottiglia di ottone fuoriesce continuamente del fumo, trattenuto dal suo tappo di piombo. La bottiglia pesa 500 grammi. Quando usi due azioni per rimuovere il tappo, una nube di denso fumo si sparge in un raggio di 18 metri intorno alla bottiglia. L'area della nube è oscurata pesantemente. Per ciascun minuto in cui la bottiglia resta aperta e all'interno della nube, il raggio aumenta di 3 metri finché non raggiunge il raggio massimo di 36 metri.

La nube persiste fino a quando la bottiglia resta aperta. Chiudere la bottiglia richiede che tu pronunci la sua parola di comando con due azioni. Una volta chiusa la bottiglia, la nube si disperde dopo 10 minuti. Un vento moderato (dai 15 ai 30 km/h) può disperdere il fumo in1 minuto, e un vento forte (più di 30 km/h) può disperderlo in 1 round.

\subsubsection*{Borsa dell'Annullamento}\index[OggettiMagici]{Borsa dell'Annullamento}
9000 mo, raro, questa borsa magica funziona come una borsa conservante per 1d6 giorni. Trascorso questo periodo, tutto il materiale al suo interno o nuovo materiale aggiunto è soggetto ad una trasformazione dipendente dalla sua natura. Pietre preziose diventano inutili sassi, e metalli preziosi si trasformano in metalli di minor valore tipo piombo. Gli oggetti magici perdono il loro potere senza alcun Tiro Salvezza, e si trasformano in oggetti mondani del loro tipo. Solo oggetti magici estremamente potenti sono possibilmente immuni a questo effetto.

\index[OggettiMagici]{Braciere degli Elementali del Fuoco}\subsubsection*{Braciere del Comando degli Elementali del Fuoco}
8000 mo, raro, mentre il fuoco arde all'interno di questo braciere di ottone, puoi usare due azioni per pronunciare la parola di comando del braciere ed evocare un elementale del fuoco, come se avessi lanciato l'incantesimo evoca elementali. Il braciere non può di nuovo essere usato a questo modo, fino alla prossima alba.

Il braciere pesa 2,5 chili.

\subsubsection*{Braciere del Sonno maledetto}\index[OggettiMagici]{Braciere del sonno maledetto}
questo braciere ha l'apparenza di, e funziona come, un braciere del comando degli elementi del fuoco. Tuttavia, quando viene attivato, il fumo si addensa per 3 m di raggio intorno al braciere, inducendo a un sonno maledetto chiunque si trovi nell'area, a meno che non riesca in un Tiro Salvezza su Volontà DC 21. Un elementale del fuoco compare normalmente, ma è ostile ed attacca tutte le creature presenti. Creature soggette al sonno maledetto dormono indefinitamente fino a che non vengono uccise, a meno che non venga usato Rimuovi Maledizione.

\index[OggettiMagici]{Brocca dell'Acqua Infinita}\subsubsection*{Brocca dell'Acqua Infinita} 12000 mo 12000 mo, non comune, quest'ampolla tappata emette un suono di liquido quando viene smossa, come se contenesse acqua. La brocca pesa 1 chilo. Puoi usare due azioni per rimuovere il tappo e pronunciare una delle tre parole di comando, e a quel punto un ammontare di acqua fresca o acqua salata (a tua scelta) si riverserà fuori dell'ampolla, fino all'inizio del tuo prossimo round. Scegli una delle opzioni seguenti:

\medskip

\begin{itemize}
\item
"Ruscello" produce 4 litri d'acqua.
\item
"Fontana" produce 20 litri d'acqua.
\item
"Geyser" produce 150 litri d'acqua che vengono proiettati da un geyser lungo 9 metri e largo 30 centimetri. Con due azioni, mentre impugni la brocca, puoi prendere come bersaglio del geyser una creatura visibile entro 9 metri da te.

Il bersaglio deve superare un Tiro Salvezza su Tempra con DC 15 o subire 1d4 danni da botta e cadere prono. Invece di una creatura, puoi prendere come bersaglio un oggetto che non sia indossato o trasportato e che non pesi più di 100 chili. L'oggetto viene ribaltato o spinto 4,5 metri lontano da te.
\end{itemize}

\subsubsection*{Brocca delle Pozioni}\index[OggettiMagici]{Brocca delle Pozioni}
18000 mo, leggendaria, questa brocca di ceramica azzurra ha un tappo d’oro massiccio. La brocca contiene 1d4+1 pozioni magiche, ognuna delle quali può essere versata ogni 2 giorni. Le specifiche pozioni sono determinate a caso, rimangono le stesse nel tempo e devono essere versate sempre nello stesso ordine. Non tutte sono necessariamente benefiche.

\index[OggettiMagici]{Buco Portatile}\subsubsection*{Buco Portatile}
10000 mo, raro, questo elegante tessuto nero, soffice come la seta, si piega fino alle dimensioni di un fazzoletto. Si dispiega in uno strato circolare di 1 metri di diametro. Puoi usare 1 round per dispiegare un buco portatile e piazzarlo sopra o contro una superficie solida, sulla quale il Buco portatile crea un foro profondo 3 metri. Qualsiasi creatura abbastanza piccola può usare il Buco Portatile per attraversare la parete o superficie su cui è appoggiato purché sia profonda meno di 3 metri.

Puoi usare 1 round per chiudere un Buco portatile prendendo i margini del tessuto e ripiegandolo. Piegare il tessuto chiude il buco, e qualsiasi creatura od oggetto al suo interno viene espulso con una probabilità del 50\% di uscire da una parte o dall'altra.

Piazzare un buco portatile all'interno dello spazio extradimensionale creato da una borsa conservante, Vano portatile, uno zainetto pratico o simile oggetto distrugge istantaneamente entrambi gli oggetti e apre un portale verso il Piano Astrale. Il portale origina dal punto in cui un oggetto è stato piazzato all'interno dell'altro. Qualsiasi creatura entro 3 metri dal portale viene risucchiata al suo interno e depositata in un luogo casuale del Piano Astrale. Poi il portale si chiude. Il portale è a senso unico e non può essere riaperto.

\index[OggettiMagici]{Candela di Invocazione}\subsubsection*{Candela di Invocazione}
8000 mo, molto raro, questa lunga e sottile candela è dedicata a un Patrono e ne condivide i Tratti. I Tratti della candela possono essere individuati tramite un rituale di 1 ora di affiancamento alla candela.

Il Narratore sceglie il Patrono e i Tratti associato a esso o lo determina casualmente.

La magia della candela si attiva quando la candela viene accesa con due azioni. Dopo aver bruciato per 4 ore, la candela è distrutta. Puoi decidere di spegnerla anticipatamente per riutilizzarla più tardi. Dedurre il tempo che rimane alla candela prima di estinguersi a incrementi di 1 minuto, per determinare per quanto abbia bruciato la candela.

Quando è accesa, la candela irradia luce fioca in un raggio di 9 metri. Qualsiasi creatura all'interno della luce Devota o Seguace a quello della candela effettua Tiri per Colpire, Tiri Salvezza e prove di abilità con +1d6.

In alternativa, quando accendi la candela per la prima volta, puoi lanciare l'incantesimo portale. Farlo distrugge la candela.

\index[OggettiMagici]{Ceppi Dimensionali}\subsubsection*{Ceppi Dimensionali}
4000 mo, raro, puoi usare 2 Azioni per piazzare queste manette su di una creatura inabile. Le manette si adattano a qualsiasi creatura da taglia Piccola a Grande. Oltre a servire da comuni manette, i ceppi impediscono a una creatura legata con essi dall'usare qualsiasi metodo di movimento extradimensionale, compreso il teletrasporto o il viaggio verso piani diversi dell'esistenza. Tuttavia non impediscono a una creatura di attraversare un portale interdimensionale.

Tu e qualsiasi creatura da te indicata quando fai uso dei ceppi potete usare due azioni per rimuoverli. Una volta ogni 30 giorni, la creatura legata può effettuare una prova di Forza con DC 40. Se la supera, la creatura si libera e distrugge i ceppi.

\index[OggettiMagici]{Colla Suprema}\subsubsection*{Colla Suprema}
400 mo, non comune, questa sostanza bianco lattea e viscosa può formare un legame adesivo permanente tra qualsiasi due oggetti. Deve essere contenuto in una giara o ampolla che è stata ricoperta all'interno di olio di scivolosità. Quando viene trovata, il suo contenitore ne tiene 1d6 + 1 per 30 grammi. 30 grammi di colla possono coprire una superficie quadrata di 30 centimetri di lato. La colla ci mette 1 minuto per fissarsi. Una volta fissata la colla, il legame creato può essere spezzato solo dal solvente universale o l'olio della forma eterea, o tramite l'incantesimo desiderio.

\subsubsection*{Collana dell’Aria Salubre}\index[OggettiMagici]{Collana dell’Aria Salubre}
2500 mo, non comune, questa collana è una catena con un medaglione di platino. La magia della collana circonda chi la indossa con una bolla di aria pura, rendendolo immune agli effetti dei vapori e dei gas. La bolla consente di sopravvivere in un ambiente senz’aria per una settimana.

\index[OggettiMagici]{Corda da Arrampicata}\subsubsection*{Corda da Arrampicata}
2000 mo, non comune, questa corda di seta lunga 18 metri, pesa 1,5 chili e può sostenere fino a 1.500 chili. Se impugni un'estremità della corda e usi due azioni per pronunciare la parola di comando, la corda si anima. Con due azioni puoi comandare all'altra estremità di muoversi verso una destinazione di tua scelta. Quell'estremità si muove di 3 metri durante il tuo round quando riceve il tuo primo comando, e di 3 metri durante ciascun round successivo finché non raggiunge la sua destinazione, fino alla sua lunghezza massima, o finché non le dici di fermarsi. Puoi anche dire alla corda di stringersi o sganciarsi da un oggetto, annodarsi o snodarsi, o riavvolgersi per essere trasportata. Se dici alla corda di compiere un nodo, grossi nodi compariranno a intervalli di 30 centimetri lungo la corda. Mentre è annodata, la corda diminuisce fino a un lunghezza di 15 metri e conferisce +1d6 alle prove effettuate per arrampicarvisi.

La corda ha Difesa 20, Durezza 3 e 20 Punti Ferita. Recupera 1 punto ferita ogni 5 minuti finché ha almeno 1 punto ferita. Se la corda scende a 0 Punti Ferita, è distrutta.

\index[OggettiMagici]{Corda dell'Intralciamento}\subsubsection*{Corda dell'Intralciamento}
4000 mo, raro, questa corda è lunga 9 metri e pesa 1,5 chili. Se tieni un'estremità della corda e usi due azioni per pronunciare la sua parola di comando, l'altra estremità scatterà in avanti per impigliare una creatura visibile entro 6 metri da te. Il bersaglio deve superare un Tiro Salvezza su Riflessi con DC 18 o restare intralciato. Puoi rilasciare la creatura usando due azioni per pronunciare una seconda parola di comando. Un bersaglio intralciato dalla corda può usare due azioni per effettuare una prova di Forza o Artista della Fuga con DC 18 (a scelta del bersaglio). Se la supera, la creatura non è più intralciata dalla corda.

La corda ha Difesa 20 e 20 Punti Ferita. Recupera 1 punto ferita ogni 5 minuti finché ha almeno 1 punto ferita. se la corda scende a 0 Punti Ferita, è distrutta.

\subsubsection*{Corda Strozzatrice}\index[OggettiMagici]{Corda Strozzatrice}
rara, questa corda magica, benché normale all'apparenza, può animarsi e aggredire chi cerca di usarla, stringendosi intorno al collo e cercando di strangolare la sua vittima. La corda strozzatrice è lunga abbastanza da poter strangolare fimo a 1d4 creature in un raggio di 3 m, infliggendo 2d6 ferite a round a ognuna di loro. E' necessario superare un Tiro Salvezza su Riflessi DC 19 per evitare di essere presi. La corda ha Difesa 22 e 25 Punti Ferita, ma solo chi non viene strangolato può attaccarla. Le vittime non possono liberarsi da sole in alcun modo, né lanciare incantesimi.

\index[OggettiMagici]{Corno di Distruzione}\subsubsection*{Corno di Distruzione}
750 mo, raro, puoi usare due azioni per pronunciare la parola di comando del corno e poi suonarlo, emettendo uno scoppio tonante in un cono di 9 metri e udibile fino a 180 metri di distanza. Ogni creatura all'interno del cono deve effettuare un tiro Salvezza su Tempra con DC 18. Se il Tiro Salvezza fallisce, la creatura subisce 5d6 danni da suono e resta assordata per 1 minuto. Se il Tiro Salvezza riesce, la creatura subisce la metà dei danni e non è assordata. Le creature e gli oggetti fatti di vetro o cristallo hanno -1d6 al Tiro Salvezza e subiscono 10d6 danni da suono anziché 5d6.

Ogni uso della magia del corno ha il 20\% di probabilità di farlo esplodere. L'esplosione infligge 10d6 danni da fuoco a chi lo suona e distrugge il corno.

\index[OggettiMagici]{Corno del Valhalla}\subsubsection*{Corno del Valhalla}
6000 mo, raro, puoi usare due Azioni per suonare questo corno. Come risposta, entro 18 metri da te appaiono gli spiriti guerrieri di Asgard. Questi spiriti usano le statistiche dei berserker. Essi ritornano ad Asgard dopo 1 ora o quando scendono a 0 Punti Ferita. Una volta usato, il corno non potrà essere usato di nuovo prima che siano passati 7 giorni.


\index[OggettiMagici]{Cubo di Forza}\subsubsection*{Cubo di Forza}
16000 mo, raro, questo cubo ha 2,5 centimetri di spigolo. Ogni faccia ha un marchio unico che può essere premuto. Il cubo inizia con 36 cariche, e recupera 3d6 cariche spese ogni giorno all'alba. Puoi usare due Azioni per premere una delle facce del cubo, spendendo un numero di cariche basate sulla faccia del cubo.

Ogni faccia ha un effetto diverso. Se al cubo non rimangono più cariche, non succede nulla. Altrimenti, si erge una barriera di forza invisibile, che forma un cubo di 4,5 metri di spigolo. La barriera è centrata su di te, si muove con te, e dura per 1 minuto, fino a che non usi due azioni per premere la sesta faccia del cubo, o il cubo esaurisce le cariche. Puoi cambiare l'effetto della barriera premendo una faccia diversa del cubo e spendendo il numero di cariche richiesto, resettandone la durata.

Se il tuo movimento fa sì che la barriera entri a contatto con un oggetto solido che non può attraversare il cubo, finché rimane la barriera non potrai avvicinarti all'oggetto.

\medskip

Il cubo perde cariche quando la barriera viene presa come bersaglio da certi incantesimi o entra a contatto con certi incantesimi o effetti di oggetti magici, come indicato nella tabella seguente.

\medskip

\begin{tabular}{ll}
\textbf{Incantesimo o Oggetto} &\textbf{Cariche Perse}\\
\hline
Dardo Incantato (5 colpi) &1\\
Disintegrazione &1d12\\
Muro di fuoco &1d4\\
Passapareti& 1d6\\
Spruzzo prismatico &3d6\\
\end{tabular}

\medskip

\begin{tabularx}{0.45\textwidth}{llX}
\textbf{Faccia} & \textbf{Cariche}& \textbf{Effetto}\\
\hline
1& 1& Gas, vento e nebbia non possono penetrare la barriera\\
2& 2 &La materia non vivente non può attraversare la barriera. Muri, pavimenti e soffitti possono attraversarla a tua discrezione.\\
3 &3 &La materia vivente non può attraversare la barriera.\\
4 &4 &Gli effetti dell'incantesimo non possono attraversare la barriera.\\
5 &5 &Nulla può attraversare la barriera. Muri,pavimenti e soffitti possono attraversarla a tua discrezione.\\
6 &0& La barriera si disattiva. \\
\end{tabularx}


\subsubsection*{Cubo di protezione dal freddo}\index[OggettiMagici]{Cubo di protezione dal freddo}
2500 mo, raro, questo ciondolo cubico si attiva e si disattiva premendone una faccia (azione immediata). Quando è attivato, emana un campo protettivo cubico con lo spigolo di 3 m (simile a quello di un cubo di forza ma dall'effetto diverso). La temperatura all’interno del campo protettivo si mantiene a 21 °C. Il campo assorbe tutti gli attacchi di freddo, negandoli completamente. Se nega più di 50 danni da freddo in un round (sia da un singolo attacco che da attacchi multipli), però, il campo magico collassa e non può essere riattivato per un'ora. Se il campo nega più di 100 ferite da freddo in un round, il cubo viene distrutto.

\index[OggettiMagici]{Fasce di Ferro del Vincolo}\subsubsection*{Fasce di Ferro del Vincolo}
5000 mo, raro, questa sfera di ferro arrugginita misura 7,5 centimetri di diametro e pesa 500 grammi. Puoi usare due azioni per pronunciare una parola di comando e scagliare la sfera contro una creatura visibile di taglia Enorme o inferiore entro 18 metri da te. La sfera si muove nell'aria, aprendosi in un reticolato di fasce metalliche. Effettua un Tiro per Colpire a distanza, se colpisci, il bersaglio è intralciato fino a quando non effettuerai due azioni per pronunciare una parola di comando e liberarlo. Farlo, o mancare l'attacco, fa sì che le fasce si contraggano e ritornino a essere una sfera.

Una creatura, compresa quella intralciata, può usare due azioni per effettuare una prova di Forza con DC 25 per spezzare le fasce di ferro. Se la riesce, l'oggetto viene distrutto, e la creatura intralciata è libera. Se la prova fallisce, qualsiasi ulteriore tentativo effettuato dalla creatura fallisce automaticamente fino a quando non saranno trascorse 24 ore. Una volta che le fasce sono state usate non potranno più esserlo fino alla prossima alba.

\index[OggettiMagici]{Faretra Efficiente}\subsubsection*{Faretra Efficiente}
2500 mo, raro, ciascuno dei tre compartimenti della faretra è collegato a uno spazio extradimensionale che le permetta di trasportare numerosi oggetti non pesando mai più di 1 chilo.

Il compartimento più piccolo può contenere fino a 60 frecce, saette od oggetti simili. Il compartimento mediano può contenere fino a 18 giavellotti od oggetti simili. Il compartimento più lungo può contenere fino a 6 oggetti lunghi, come archi, bastoni da combattimento o lance. Puoi estrarre qualsiasi oggetto contenuto nella faretra come se lo stessi prendendo da una normale faretra o fodero.

\subsubsection*{Filatterio contro i non morti}\index[OggettiMagici]{Filatterio contro i non morti}
1000 mo, raro, questo oggetto sacro permette di usare l'Abilità Scacciare non morti con un bonus di +2 alla somma dei Tratti in comune con il Patrono.

\subsubsection*{Filatterio della giovinezza}\index[OggettiMagici]{Filatterio della giovinezza}
10000 mo, leggendario, la striscia di pergamena di questo filatterio è di solito rinchiusa in un tubetto metallico da portare appeso al collo. Quando un personaggio lo indossa, la sua naturale velocità di invecchiamento scende al 75\%, mentre un eventuale invecchiamento magico è ridotto della metà.

\index[OggettiMagici]{Fortezza Istantanea}\subsubsection*{Fortezza Istantanea}
75000 mo, molto raro, puoi usare due azioni per porre questo cubo di metallo di 2,5 centimetri di spigolo sul terreno e pronunciarne la parola di comando. Il cubo cresce rapidamente fino a diventare una fortezza che resterà fino a quando userai due azioni per pronunciare la parola di comando che la congeda, la quale funziona solo quando la fortezza è vuota.

La fortezza è una torre quadrata, 6 metri per lato e alta 9 metri, con feritoie su tutti i lati e spalti in cima. Il suo interno è diviso in due piani, con una scala che corre lungo una parete a congiungerli. La scala termina con una botola che si apre sul tetto. Quando viene attivata, la torre presenta una piccola porta sul lato rivolto verso di te. La porta si apre solo al tuo comando, che puoi pronunciare con due azioni. È immune all'incantesimo scassinare e magie simili, come quella del battaglio dell'apertura.

Ogni creatura nell'area in cui la fortezza compare deve effettuare un Tiro Salvezza su Riflessi con DC 17, subendo 10d10 danni da botta se lo fallisce, o la metà di questi danni se lo riesce. In entrambi i casi, la creatura viene spinta in uno spazio fuori della fortezza ma in sua prossimità. Gli oggetti nell'area che non sono indossati o trasportati subiscono gli stessi danni e vengono spinti automaticamente.

La torre è fatta di adamantio, e la sua magia le impedisce di venir ribaltata. Il tetto, la porta e le mura hanno 100 Punti Ferita ognuno, immunità ai danni dalle armi non magiche a eccezione delle armi da assedio, e resistenza a tutti gli altri danni.

Solo l'incantesimo desiderio può riparare la fortezza. Ciascun lancio di desiderio fa sì che il tetto, la porta o una delle pareti recuperi 50 Punti Ferita.


\subsubsection*{Freccia localizzante}\index[OggettiMagici]{Freccia localizzante}
400 mo, non comune, questa freccia può essere usata fino ad 8 volte al giorno. Essa viene lanciata per aria, e quando atterra indica una direzione o un luogo desiderati. Possibili indicazioni includono l'uscita o l'ingresso più vicini, scale, passaggi, caverne e simili aree.

\index[OggettiMagici]{Incensiere del Comando degli Elementali dell'Aria}\subsubsection*{Incensiere del Comando degli Elementali dell'Aria}
8000 mo, raro, mentre l'incenso brucia all'interno di questo incensiere, puoi usare due azioni per pronunciare la parola di comando del braciere ed evocare un elementale dell'aria, come se avessi lanciato l'incantesimo evoca elementali. L'incensiere non può di nuovo essere usato a questo modo fino alla prossima alba. Questo incensiere largo 15 centimetri e alto 30 centimetri assomiglia a un calice dalla copertura decorata. Pesa 0,5 chili.

\subsubsection*{Incenso della meditazione}\index[OggettiMagici]{Incenso della meditazione}
5000 mo, raro, questo blocchetto d’incenso dal profumo dolce è indistinguibile dal normale incenso finché non viene acceso. Quando brucia, la sua particolare fragranza e il suo fumo perlaceo sono riconoscibili con una prova di Arcana a DC 13. Dopo che un incantatore avrà trascorso 8 ore ripassando sul Tomo e meditando nei pressi di un blocchetto acceso, acquisirà la capacità di lanciare i suoi incantesimi con il massimo effetto e la massima durata possibile, gli incantesimi che richiedono un Tiro Salvezza imporranno una penalità -1 aggiuntiva. Ogni blocchetto d’incenso brucia per 8 ore e l’effetto persiste per altre 8 ore. Di solito si trovano 2d4 blocchetti di incenso nella stessa custodia.

\index[OggettiMagici]{Lanterna della Rivelazione}\subsubsection*{Lanterna della Rivelazione}
5000 mo, non comune, mentre è accesa, questa lanterna brucia per 6 ore con 1 fiasca d'olio, irradiando luce intensa in un raggio di 9 metri e luce fioca per ulteriori 9 metri. Le creature e gli oggetti invisibili sono resi visibili mentre si trovano sotto la luce intensa della lanterna.

Puo usare due azioni per abbassare la copertura, riducendo la luce a fioca con un raggio di 1,5 metri.

\subsubsection*{Incenso dell’Ossessione}\index[OggettiMagici]{Incenso dell’Ossessione}
raro, del tutto simile all’incenso della meditazione, questo incenso dà a chi lo usa anche l'impressione del suo effetto, ma sarà sotto Confusione per 24 ore se fallisce un Tiro Salvezza su Volontà DC 23.

\index[OggettiMagici]{Mazzo delle Illusioni}\subsubsection*{Mazzo delle Illusioni}
6500 mo, non comune, questa scatola contiene un set di carte di pergamena. Un mazzo completo contiene 34 carte, ognuna raffigurante una creatura diversa. Le creature rappresentate vengono lasciate alla discrezionalità del Narratore. Di solito i mazzi trovati in giro sono privi di 3d6-3 carte.

La magia del mazzo funziona solo se le carte vengono pescate a caso (potete usare un mazzo di normali carte da gioco modificato per simulare il mazzo delle illusioni). Puoi usare due azioni per pescare una carta dal mazzo e scagliarla in un punto sul terreno a 9 metri da te.

L'illusione di una o più creature si forma sopra la carta lanciata e rimane finché non viene dissolta. La creatura illusoria sembra reale, della taglia appropriata, e si comporta come fosse una vera creatura, eccetto che non può recare danni. Finché resti entro 36 metri dalla creatura illusoria e puoi vederla, puoi usare due azioni per muoverla magicamente in qualsiasi punto entro 9 metri dalla carta. Qualsiasi interazione fisica con la creatura illusoria la rivela come illusione, dato che gli oggetti le passano attraverso. Qualcuno che usi due azioni per ispezionare visivamente la creatura, la identifica come illusoria superando una prova di Intelligenza con DC 17. La creatura le apparirà quindi trasparente.
L'illusione permane finché la carta non viene mossa o l'illusione dissolta. Quando l'illusione termina, l'immagine sulla carta scompare, e quella carta non potrà più essere usata.

\end{multicols}

\medskip

\begin{center}
\includegraphics[width=0.55\linewidth]{immagini/Incenso.png}

\end{center}

\begin{tabular}{ll|ll}
\textbf{Carta da Gioco}& \textbf{Illusione}&\textbf{Carta da Gioco}& \textbf{Illusione}\\
\hline
Asso di cuori &Drago rosso&Asso di quadri& Beholder\\
Re di cuori &Cavaliere e quattro guardie&Re di quadri & Arcimago e magio apprendista\\
Regina di cuori &Succube o incubo&Regina di quadri &Megera notturna\\
Fante di cuori &Druido&Fante di quadri &Assassino\\
Dieci di cuore &Gigante delle nuvole&Dieci di quadri &Gigante del fuoco\\
Nove di cuori &Ettin&Nove di quadri &Oni\\
Otto di cuori& Bugbear&Otto di quadri &Gnoll\\
Due di cuori &Goblin&Due di quadri &Coboldo\\
Asso di picche &Lich&Asso di fiori& Golem di ferro\\
Re di picche &Sacerdote e due accoliti&Re di fiori &Capitano bandito e tre banditi\\
Regina di picche& Medusa&Regina di fiori &Erinni\\
Fante di picche &Veterano&Fante di fiori &Berserker\\
Dieci di picche &Gigante del gelo&Dieci di fiori &Gigante di collina\\
Nove di picche &Troll&Nove di fiori &Ogre\\
Otto di picche &Hobgoblin&Otto di fiori& Orco\\
Due di picche &Goblin&Due di fiori &Coboldo\\
Jolly (2) &Tu (il proprietario del mazzo)&&\\
\end{tabular}

\begin{multicols}{2}

\medskip

\index[OggettiMagici]{Mazzo delle Meraviglie}\subsubsection*{Mazzo delle Meraviglie}
100000 mo, leggendario, di solito lo si trova in un borsello o una scatola, che contiene delle carte fatte d'avorio o vello. La maggior parte di questi mazzi (il 75\%) ha solo tredici carte, mentre i restanti mazzi ne hanno ventidue.

Prima di pescare una carta, devi dichiarare quante carte intendi pescare e poi pescarle casualmente (puoi usare un mazzo di carte da gioco modificato per simulare il mazzo). Qualsiasi carta pescata in eccesso di questo numero non ha effetto. Altrimenti, appena peschi una carta dal mazzo, la sua magia ha effetto.

Devi pescare ciascuna carta entro 1 ora dalla pescata precedente. Se non peschi il numero scelto di carte, il numero di carte rimanenti uscirà fuori dal mazzo spontaneamente e avrà effetto in contemporanea. Una volta estratta una carta, questa svanirà dall'esistenza. A meno che la carta non sia il Matto o il Buffone, la carta ricompare nel mazzo, rendendo possibile pescare due volte la stessa carta.

\medskip

\end{multicols}

\begin{tabularx}{0.95\textwidth}{lX|lX}
\textbf{Carta da Gioco}& \textbf{Carta}&\textbf{Carta da Gioco}& \textbf{Carta}\\
\hline
Asso di quadri& Visir*&Asso di cuori &Fato*\\
Re di quadri &Sole&Re di cuori &Trono\\
Regina di quadri& Luna&Regina di cuori& Chiave\\
Fante di quadri &Stella&Fante di cuori& Cavaliere\\
Due di quadri &Cometa*&Due di cuori &Gemma*\\
Asso di fiori &Speroni*&Asso di picche& Dongione*\\
Re di fiori &Il Vuoto&Re di picche& Rovina\\
Regina di fiori& Fiamme&Regina di picche &Euriale\\
Fante di fiori &Teschio&Fante di cuori& Furfante\\
Due di fiori &Idiota&Due di picche &Appeso*\\
Jolly &Matto*&Jolly &Buffone\\
\end{tabularx}

\begin{multicols}{2}

\medskip

* Solo in mazzo da 22 carte

\textit{Appeso} (solo in mazzo da 22). La tua mente è sconvolta, e cambi 2 Tratti

\textit{Buffone}. Ottieni 35 PX o puoi pescare due carte aggiuntive oltre alle tue pescate dichiarate.

\textit{Cavaliere}. Ottieni i servigi di un guerriero con CA 4 livello che compare in uno spazio a tua scelta entro 9 metri da te. Il guerriero è della tua stessa razza e ti servirà lealmente fino alla morte, credendo che sia stato il fato a portarlo al tuo servizio. Il personaggio è controllato da te.

\textit{Chiave}. Un'arma magica rara, molto rara o leggendaria con la quale sei competente compare tra le tue mani. Il Narratore determina di che tipo di arma si tratta.

\textit{Cometa} (solo in mazzo da 22). Se sconfiggi da solo il prossimo mostro o gruppo ostile che incontrerai, otterrai abbastanza punti esperienza da guadagnare un livello. Altrimenti, questa carta non avrà effetto.

\textit{Euriale}. Sei maledetto dalla carta e subisci una penalità di -2 a tutti i Tiri Salvezza finché resterai maledetto a questo modo. Solo un Patrono o la magia della carta del Fato può porre fine a questa maledizione.

\textit{Fato} (solo in mazzo da 22). La struttura della realtà si dissolve e riforma, permettendoti di evitare o cancellare un evento come se non fosse mai accaduto. Puoi usare la magia di questa carta non appena l'hai pescata o aspettare un qualsiasi altro momento fino alla tua morte.

\textit{Fiamme}. Un potente diavolo diventa tuo nemico. Il diavolo cercherà di rovinare e infestare la tua esistenza, assaporando le tue sofferenze fino al momento in cui cercherà di ucciderti. Questa inimicizia durerà fino alla morte tua o del diavolo.

\textit{Furfante}. Un personaggio non dei giocatori a scelta del Narratore diventa ostile nei tuoi confronti. L'identità del nuovo nemico è ignota fino a quando il PNG o qualcun altro la rivelerà. Nulla a meno di un desiderio o intervento divino potrà porre fine all'ostilità del PNG nei tuoi confronti.

\textit{Gemma} (solo in mazzo da 22). Davanti ai tuoi piedi compaiono venticinque gioielli del valore di 2000 mo ciascuno o cinquanta gemme del valore di 1000 mo ciascuna.

\textit{Idiota} (solo in mazzo da 22). Riduci permanentemente il tuo punteggio di Intelligenza di 2 (fino a un punteggio minimo di 3). Puoi pescare un'ulteriore carta prima delle tue altre pescate dichiarate.

\textit{Luna}. Ricevi la capacità di lanciare l'incantesimo desiderio 1d3 volte.

\textit{Matto} (solo in mazzo da 22). Perdi 10000 PX, scarti questa carta, e peschi di nuovo dal mazzo, contando entrambe le pescate come solo una delle tue pescate. Se perdere quel numero di PX ti farebbe perdere un livello, rimarrai invece con il numero di PE appena sufficienti per mantenere il tuo livello.

\textit{Rovina}. Perdi tutte le ricchezze che hai con te, a parte gli altri oggetti magici. Attività, edifici e le terre che possiedi vengono perse nel modo che altera di meno la realtà. Qualsiasi documento che provi che tu sia il proprietario di qualcosa che hai perso a causa di questa carta, scompare.

\textit{Sole}. Ottieni 35 PX, e un oggetto meraviglioso (determinato dal Narratore) compare tra le tue mani.

\textit{Sotterraneo} (solo in mazzo da 22). Scompari e vieni sepolto in uno stato di animazione sospesa all'interno di una sfera extradimensionale. Tutto ciò che stavi indossando o trasportando rimane nello spazio da te occupato quando sei scomparso. Rimarrai imprigionato finché non sarai ritrovato e rimosso dalla sfera. Non puoi essere localizzato tramite nessuna magia di divinazione, ma l'incantesimo desiderio può rivelare la posizione della tua prigione. Non si pescano ulteriori carte.

\textit{Speroni} (solo in mazzo da 22). Ogni oggetto magico che indossi o trasporti viene disintegrato. Gli artefatti in tuo possesso non vengono disintegrati, ma svaniscono.

\textit{Stella}. Aumenta un tuo punteggio di caratteristica di 1. Il punteggio può superare il 5, ma non può superare 7.
\textit{Teschio}. Evochi un avatara della morte (uno spettrale scheletro umanoide avvolto in una vestaglia nera e sbrindellata, il quale impugna una falce spettrale). Esso compare in uno spazio a scelta del Narratore entro 3 metri da te e ti attacca, avvisando tutti gli altri che devi vincere la battaglia da solo. L'avatara combatte fino alla tua morte o finché non scende a 0 Punti Ferita, al che svanisce. Se qualcuno cerca di aiutarti, costui evocherà il proprio avatara della morte. Una creatura uccisa da un avatara della morte non può essere riportata in vita.

\textit{Trono}. Ottieni +1d6 in Diplomazia. Inoltre, ottieni il diritto di proprietà su di una piccola rocca da qualche parte nel mondo. Tuttavia, la rocca è attualmente occupata da mostri, che dovrai cacciare prima di poterla rivendicare come tua.

\textit{Visir} (solo in mazzo da 22). In qualsiasi momento di tua scelta, entro un anno da quando hai pescato questa carta, puoi chiedere, meditando, risposta a una tua domanda e ricevere una risposta veritiera a essa. A parte fornire informazioni, la risposta può aiutarti a risolvere un problema complesso o un dilemma. In altre parole, la conoscenza è fornita assieme alla saggezza su come impiegarla.

\textit{Vuoto}. Questa carta nera è indice di disastro. La tua anima viene rapita dal corpo e imprigionata all'interno di un oggetto in un luogo a scelta del Narratore. Una o più potenti creature proteggono questo luogo. Finché la tua anima è così intrappolata, il tuo corpo è inabile. L'incantesimo desiderio non è in grado di ripristinare la tua anima, ma può rivelare il luogo in cui si trova l'oggetto che la contiene. Non si pescano più carte.

\textit{Avatara della Morte}

Non morto media, neutrale malvagio

\textbf{Forza} +3

\textbf{Destrezza}' +3

\textbf{Intelligenza} +3

\textbf{Saggezza} +3

\textbf{Carisma} +3

\textbf{Difesa} 20

\textbf{Punti Ferita} metà dei Punti Ferita del suo evocatore

\textbf{Movimento}: Velocità 18 m, volo 18 m (fluttua)

\textbf{Immunità ai Danni}: Vuoto, veleno

\textbf{Immunità alle Condizioni}: Affascinato, avvelenato, paralizzato, pietrificato, spaventato, svenuto

\textbf{Sensi}: scurovisione 18 m, visione del vero 18 m

\textbf{Linguaggi}: tutti i linguaggi conosciuti dal suo evocatore

\textbf{Sfida} (0 PX)

\textbf{Movimento Incorporeo}. L'avatara può attraversare creature e oggetti come fossero terreno difficile. Subisce 5 (1d10) danni da forza se termina il proprio round all'interno di un oggetto.

\textbf{Immunità allo Scacciare}. L'avatara è immune agli effetti che scacciano i non morti.

\textbf{Azioni}

\textbf{Falce Mietitrice}. L'avatara affonda la sua falce spettrale in una creatura entro 1,5 metri da esso, infliggendo 7 (1d8 + 3) danni perforanti più 4 (1d8) danni da Energia Negativa.

\index[OggettiMagici]{Miniatura dal Potere Meraviglioso}\subsubsection*{Miniatura dal Potere Meraviglioso}
rarità variabile, costo variabile, una miniatura dal potere meraviglioso è una statuetta di una bestia, piccola a sufficienza da entrare in tasca. Se usi due azioni per pronunciare una parola di comando e lanciare la miniatura in un punto del terreno entro 18 metri da te, la miniatura diventa una creatura vivente. Se lo spazio in cui la creatura dovesse apparire è occupato da un'altra creatura od oggetto, o se non c'è spazio sufficiente per la creatura, la miniatura non si trasforma.

La creatura è amichevole nei confronti tuoi e dei tuoi compagni. Comprende i tuoi linguaggi e obbedisce agli ordini impartitele. Se non le impartisci ordini, la creatura si difende ma non effettua altre azioni. Vedi il Bestiario per le altre statistiche della creatura.

La creatura resta per la durata specificata per ciascuna miniatura. Al termine della durata, la creatura ritorna alla sua forma di miniatura. Si trasforma anticipatamente se scende a 0 Punti Ferita o se usi due azioni per pronunciare la parola di comando di nuovo mentre la tocchi. Dopo che la creatura è tornata a essere una miniatura, le sue proprietà non possono più essere usate fino a quando non sarà trascorso un certo ammontare di tempo, come specificato nella descrizione della miniatura.

\textit{Cane di Onice} (Raro, 500 mo). Questa statuetta di onice raffigura un cane. Può diventare un mastino per un massimo di 6 ore. Il mastino ha Intelligenza -2 e può parlare Comune. Inoltre ha scurovisione con una gittata di 18 metri e può vedere le creature e gli oggetti invisibili entro quella gittata. Una volta usata, non può essere usata di nuovo prima che siano passati 7 giorni.

\textit{Caprone d'Avorio (Raro. 1000 mo)}. Queste statuette d'avorio di caproni sono sempre create in set da tre. Ogni caprone ha un aspetto unico e funziona in modo diverso dagli altri. Le loro proprietà sono le seguenti:

Il caprone del terrore può diventare un caprone gigante per un massimo di 3 ore. Il caprone non può attaccare, ma puoi rimuoverne i corni e usarli come armi. Un corno diventa una lancia da cavaliere +1 mentre l'altro diventa una spada lunga +2.

Rimuovere un corno richiede due azioni, e le armi scompaiono e i corni ricompaiono quando il caprone torna alla sua forma di miniatura. Inoltre, il caprone irradia un'aura di terrore con raggio 9 metri finché lo cavalchi. Qualsiasi creatura a te ostile che inizi il proprio round all'interno dell'aura deve superare un Tiro Salvezza su Volontà con DC 17 o restare
spaventata dal caprone per 1 minuto, o finché il caprone non torna alla forma di miniatura. La creatura spaventata può ripetere il Tiro Salvezza al termine di ciascun suo round, terminando l'effetto se lo supera. Una volta che ha riuscito il Tiro Salvezza contro questo effetto, una creatura è immune all'aura del caprone per le successive 24 ore. Una volta usata, la miniatura non può essere usato di nuovo prima che siano passati 15 giorni.

Il caprone del travaglio può diventare un caprone gigante per un massimo di 3 ore. Una volta usato, non può essere usato di nuovo prima che siano passati 30 giorni.
Il caprone del viaggio può diventare un caprone Grande con le stesse statistiche di un cavallo da corsa. Ha 24 cariche, e ciascuna ora o porzione di essa che trascorre in forma di bestia costa 1 carica. Finché ha cariche, lo puoi usare quanto ti pare. Una volta terminate le cariche, ritorna a essere una miniatura e non può essere usato di nuovo prima che siano passati 7 giorni, allorché avrà recuperato tutte le sue cariche.

\textit{Corvo d'Argento} (Non Comune, 300 mo). Questa statuetta d'argento raffigura un corvo. Può diventare un corvo per un massimo di 6 ore. Una volta usata, non può essere usata di nuovo prima che siano passati 2 giorni. Mentre è in forma di corvo, la miniatura ti permette di lanciare a volontà l'incantesimo messaggero animale su di essa.

\textit{Destriero di Ossidiana} (Molto Raro, 1000 mo). Questa statuetta di ossidiana liscia diventa un incubo per un massimo di 24 ore. L'incubo combatte solo per difendersi. Una volta usata, non può essere usata di nuovo prima che siano passati 5 giorni.

\textit{Elefante di Marmo} (Raro, 1500 mo). Questa statuetta di marmo è larga e alta circa 10 centimetri. Può diventare un elefante per un massimo di 24 ore. Una volta usata, non può essere usata di nuovo prima che siano passati 7 giorni.

\textit{Grifone di Bronzo} (Raro, 1250 mo). Questa statuetta di bronzo raffigura un grifone rampante. Può diventare un grifone per un massimo di 6 ore. Una volta usata, non può essere usata di nuovo prima che siano passati 5 giorni.

\textit{Gufo Serpentino} (Raro, 400 mo). Questa statuetta serpentina di un gufo può diventare un gufo gigante per un massimo di 8 ore. Una volta usata, non può essere usata di nuovo prima che siano passati 2 giorni. Se vi trovate sullo stesso piano di esistenza, il gufo può comunicare telepaticamente con te a qualsiasi distanza.

\textit{Leoni d'Oro} (Raro, 800 mo). Queste statuette d'oro di leoni sono sempre create a coppie. Puoi usare una o entrambe le miniature contemporaneamente. Ciascuna può diventare un leone per un massimo di 1 ora. Una volta usato uno dei leoni, questi non può essere usato di nuovo prima che siano passati 7 giorni.

cd \index[OggettiMagici]{Munizione dell'Uccisione}\subsubsection*{Munizione dell'Uccisione}
700 mo, molto raro, se una creatura appartenente al tipo, razza o gruppo a cui la freccia dell'uccisione è associata subisce danni dalla freccia, la creatura deve effettuare un tiro Salvezza su Tempra con DC 21, subendo 6d10 danni perforanti aggiuntivi se lo fallisce, o la metà di questi danni se lo riesce.

Una volta che la freccia dell'uccisione ha inflitto danni aggiuntivi alla creatura, diventa una freccia non magica.

\index[OggettiMagici]{Palla di Cristallo}\subsubsection*{Palla di Cristallo}
50000 mo, molto raro o leggendario, una tipica palla di cristallo ha il diametro di circa 15 centimetri. Mentre la tocchi, puoi lanciare tramite essa l'incantesimo scrutare (DC del Tiro Salvezza 21). Le seguenti palle di cristallo varianti sono oggetti leggendari e hanno proprietà aggiuntive.

\textit{Palla di Cristallo di Lettura del Pensiero}. Questa palla di cristallo è di circa 12 centimetri di diametro. Mentre la tocchi, puoi lanciare tramite di essa l'incantesimo scrutare (DC del Tiro Salvezza 21). Puoi usare due azioni per lanciare l'incantesimo individuazione dei pensieri (DC del Tiro Salvezza 21) mentre stai scrutando tramite questa palla di cristallo, prendendo come bersaglio le creature che puoi vedere e si trovano entro 9 metri dal sensore dell'incantesimo. Non devi concentrarti su questo individuazione dei pensieri per mantenerlo per la sua durata, che termina quando termina scrutare.

\textit{Palla di Cristallo di Telepatia}. Questa palla di cristallo è di circa 12 centimetri di diametro. Mentre la tocchi, puoi lanciare tramite di essa l'incantesimo scrutare (DC del Tiro Salvezza 21). Mentre scruti tramite questa palla di cristallo, puoi comunicare telepaticamente con le creature che puoi vedere e si trovano entro 9 metri dal sensore dell'incantesimo. Puoi anche usare due azioni per lanciare l'incantesimo suggestione (DC del Tiro Salvezza 21) su una di queste creature tramite il sensore. Non devi concentrarti su questa suggestione per mantenerla per la sua durata, che termina se termina scrutare. Una volta usato, il potere suggestione della palla di cristallo non può essere usato di nuovo fino alla prossima alba.

\textit{Palla di Cristallo di Visione del Vero}. Questa palla di cristallo è di circa 12 centimetri di diametro. Mentre la tocchi, puoi lanciare tramite di essa l'incantesimo scrutare (DC del Tiro Salvezza 21). Mentre scruti con questa palla di cristallo, hai visione del vero con un raggio di 36 metri centrato sul sensore dell'incantesimo.

\subsubsection*{Palla di Cristallo ipnotica}\index[OggettiMagici]{Palla di Cristallo ipnotica}
raro, questo oggetto maledetto è indistinguibile da una normale Palla di cristallo. Tuttavia chiunque tenti di usare il dispositivo rimane affascinato per 1d6 turni, ed una suggestione telepatica viene impiantata nella sua mente se fallisce un Tiro Salvezza su Volontà DC 27. L'utilizzatore del dispositivo crede di aver visto la creatura o scena desiderata, ma in realtà è sotto l'influenza di un potente incantatore, o addirittura una potenza o essere da un altro piano di esistenza. Ad ogni uso ulteriore l'utilizzatore cade sempre più sotto l’influenza del controllore, come servo o come strumento. L'utilizzatore è sempre ignaro di essere soggiogato.

\index[OggettiMagici]{Pergamena degli Incantesimi}\subsubsection*{Pergamena degli Incantesimi}
rarità variabile, vedi costi creazione pergamena, una pergamena degli incantesimi riporta le parole di un singolo incantesimo, scritte in un codice mistico.

Per leggere una pergamena è necessario:

\textbf{in caso di pergamene ISY SCROLL}:

per comprendere il contenuto è sufficiente una prova di Arcana a difficoltà DC 10

per poter leggere e lanciare l'incantesimo della pergamena è necessaria una prova di Intelligenza (o Arcana se conosciuta) a difficoltà 12.

\textbf{in caso di pergamene normali}:

per comprenderne il contenuto è necessaria una prova di Arcana a difficoltà 15

per poter leggere e lanciare l'incantesimo della pergamena è necessaria una prova di Arcana a difficoltà 20.

Lanciare l'incantesimo leggendolo da una pergamena richiede il normale tempo di lancio dell'incantesimo. Una volta che l'incantesimo è stato lanciato, le parole sulla pergamena svaniscono, e la pergamena viene ridotta in polvere. Se il lancio viene interrotto, la pergamena non si dissolve.

\subsubsection*{Pergamena protettiva contro gli elementali}\index[OggettiMagici]{Pergamena protettiva contro gli elementali}
800 mo, raro, protegge da tutti gli elementali per 20 round, concedendo +4 alla Difesa e Tiri Salvezza contro attacchi o effetti prodotti dagli elementali.

\subsubsection*{Pergamena contro i licantropi}\index[OggettiMagici]{Pergamena protettiva contro i licantropi}
700 mo, non comune, protegge da tutti i licantropi per 20 round, concedendo +4 alla Difesa e Tiri Salvezza contro attacchi o effetti prodotti dai licantropi.

\subsubsection*{Pergamena contro i non morti}\index[OggettiMagici]{Pergamena protettiva contro i non morti}
900 mo, non comune, protegge da tutti i non morti per 20 round, concedendo +4 alla Difesa e Tiri Salvezza contro attacchi o effetti prodotti dai non morti.

\subsubsection*{Pergamena contro la magia}\index[OggettiMagici]{Pergamena protettiva contro la magia}
1500 mo, raro, la pergamena lancia un incantesimo di Campo Anti-Magia.

\index[OggettiMagici]{Perla del Potere}\subsubsection*{Perla del Potere}
6000 mo, non comune, mentre hai la perla con te, puoi usare due azioni per recuperare 2d4 Punti Magia. Una volta usata, la perla non potrà essere usata di nuovo fino alla prossima alba. Esistono varianti più potenti che fanno recuperare più punti.

\subsubsection*{Pietra del Peso}\index[OggettiMagici]{Pietra del Peso}
questo oggetto sembra un sasso nero liscio e lucido. Quando chi la porta è coinvolto in un combattimento o in una fuga, subisce improvvisamente gli effetti dell’incantesimo lentezza. Una volta presa, la pietra non può essere buttata via normalmente, poiché dopo poco tempo riappare magicamente sulla persona del possessore. Per liberarsi definitivamente della pietra occorre l'incantesimo Rimuovi Maledizione.

\index[OggettiMagici]{Pietra Arcana}\subsubsection*{Pietra Arcana}\index{Ioun Stone}
costo variabile, rarità variabile, esistono diversi tipi di pietra arcana, ogni tipo una specifica combinazione di forme e colori.

Quando usi due azioni per lanciare una di queste pietre in aria, la pietra inizia a orbitare intorno alla tua testa alla distanza di 1d3 x 30 centimetri e ti conferisce un beneficio.
Dopodiché, un'altra creatura dovrà usare due azioni per afferrare o imbrigliare la pietra e separarla da te, riuscendo in un Tiro per Colpire contro Difesa 24 o superando una prova di Destrezza con DC 31. Puoi usare due azioni per afferrare e mettere da parte la pietra, terminandone l'effetto.

Una pietra ha Difesa 24, 10 Punti Ferita e resistenza a tutti i danni. Mentre orbita intorno alla tua testa è considerata un oggetto indossato.

\textit{Destrezza} (molto raro, 3000 mo). Mentre orbita intorno alla tua testa il tuo punteggio di Destrezza aumenta di 1, fino a un massimo di 5.

\textit{Assorbimento} (molto raro, 6000 mo). Mentre orbita intorno alla tua testa, puoi usare una tua Azione per cancellare un incantesimo di livello 4 o inferiore lanciato da una creatura visibile e che prende a bersaglio solo te. Una volta che la pietra ha cancellato 5 Incantesimi, si esaurisce e diventa grigia opaca, perdendo la sua magia.

\textit{Autorità} (molto raro, 3000 mo). Mentre orbita intorno alla tua testa il tuo punteggio di Carisma aumenta di 1, fino a un massimo di 5.

\textit{Consapevolezza} (raro, 12000 mo). Mentre orbita intorno alla tua testa non puoi essere sorpreso.

\textit{Forza} (molto raro, 3000 mo). Mentre orbita intorno alla tua testa il tuo punteggio di Forza aumenta di 1, fino a un massimo di 5.

\textit{Intelligenza} (molto raro, 3000 mo). Mentre orbita intorno alla tua testa il tuo punteggio di Intelligenza aumenta di 1, fino a un massimo di 5.

\textit{Intuizione} (molto raro, 3000 mo). Mentre orbita intorno alla tua testa il tuo punteggio di Saggezza aumenta di 2, fino a un massimo di 5.

\textit{Protezione} (raro, 10000 mo). Mentre orbita intorno alla tua testa ottieni un bonus di +1 alla Difesa.

\textit{Sostentamento} (raro, 3500 mo). Mentre orbita intorno alla tua testa non hai bisogno di mangiare né di bere.

\index[OggettiMagici]{Pietra della Buona Sorte}\subsubsection*{Pietra della Buona Sorte}
4500 mo, non comune, finché la pietra è con te, ottieni un bonus di +1 alle prove di caratteristica e ai Tiri Salvezza.

\index[OggettiMagici]{Pietra del Controllo degli Elementali della Terra}\subsubsection*{Pietra del Controllo degli Elementali della Terra}
8000 mo, raro, se la pietra tocca terra, puoi usare due azioni per pronunciare la parola di comando ed evocare un elementale della terra, come se avessi lanciato l'incantesimo evocare elementali. La pietra non può di nuovo essere usata a questo modo, fino alla prossima alba. La pietra pesa 2,5 chili.

\index[OggettiMagici]{Piffero delle Fogne}\subsubsection*{Piffero delle Fogne}
2000 mo, non comune, devi essere competente con gli strumenti a fiato per usare questo piffero. Mentre usi questo piffero, i ratti normali e i ratti giganti sono indifferenti nei tuoi confronti e non ti attaccheranno a meno che non li minacci o li danneggi. Se con due azioni suoni il piffero, puoi usare due azioni per spendere da 1 a 3 cariche, richiamando uno sciame di ratti per ogni carica spesa, purché ci siano abbastanza ratti entro 750 metri da te da richiamare in questa maniera (a discrezione del Narratore). Se non ci sono abbastanza ratti da formare uno sciame, la carica è sprecata. Gli sciami richiamati si muovono verso la musica tramite la rotta più breve possibile, ma non sono in alcun altro modo sotto il tuo controllo. Il piffero ha 3 cariche e recupera 1d3 cariche spese ogni giorno all'alba.

Ogni qualvolta uno sciame di ratti che non sia sotto il controllo di un'altra creatura si avvicina entro 9 metri da te mentre stai suonando il piffero, puoi effettuare una prova di Carisma contesa dalla prova di Saggezza dello sciame. Se perdi la contesa, lo sciame si comporta come di norma e non può essere di nuovo distratto dalla musica del piffero per le successive 24 ore. Se vinci la contesa, lo sciame è attratto dalla musica del piffero e diventa amichevole nei confronti tuoi e dei tuoi compagni fino a che continui a suonare il piffero con due azioni ogni round. Uno sciame amichevole obbedisce ai tuoi comandi. Se non impartisci ordini a uno sciame amichevole, questo si difenderà ma non compirà altre azioni.

Se uno sciame amichevole all'inizio del round non può udire la musica del piffero, il tuo controllo su quello sciame termina, e lo sciame si comporta come farebbe normalmente e non può essere attirato nuovamente dalla musica del piffero per le successive 24 ore.

\index[OggettiMagici]{Piffero dello Spavento}\subsubsection*{Piffero dello Spavento}
6000 mo, non comune, devi essere competente con gli strumenti a fiato per usare questo piffero. Puoi usare due azioni per suonarlo e spendere 1 carica per creare un suono incantevole e spettrale. Ogni creatura entro 9 metri da te e che ti oda suonare deve superare un Tiro Salvezza su Volontà con DC 17 o restare spaventata da te per 1 minuto. Se lo desideri, tutte le creature nell'area che non ti siano ostili possono superare automaticamente il loro Tiro Salvezza. Una creatura che fallisca il Tiro Salvezza può ripeterlo alla fine del suo round, terminando l'effetto su di sé in caso lo superi. Una creatura che superi il Tiro Salvezza è immune all'effetto di questo piffero per 24 ore. Il piffero ha 3 cariche e recupera 1d3 cariche spese ogni giorno all'alba.

\index[OggettiMagici]{Pigmenti delle Meraviglie}\subsubsection*{Pigmenti delle Meraviglie}
400 mo, molto raro, trovati solitamente in 1d4 vasetti all'interno di eleganti scatole di legno assieme a un pennello (del peso totale di 500 grammi), questi pigmenti ti permettono di creare oggetti tridimensionali, dipingendoli a due dimensioni. La pittura fluisce dal pennello per formare l'oggetto desiderato mentre ti concentri sull'immagine

Ogni vasetto di pittura è sufficiente a coprire 90 m quadri di una superficie, permettendoti di creare oggetti inanimati e caratteristiche del terreno (porte, fosse, fiori, alberi, celle, stanze o armi) che occupino un totale di 270 metri cubi. Ci vogliono 10 minuti per coprire 90 quadri.

Quando completi il dipinto, l'oggetto o la caratteristica del terreno dipinta diventa un oggetto reale, non magico. Quindi, dipingere una porta su di una parete crea una vera porta che può essere aperta per accedere a ciò che si trova oltre di essa. Dipingere una fossa sul pavimento crea una vera fossa, la cui profondità è conteggiata nell'area totale degli oggetti che puoi creare.

Nulla di ciò che viene creato dai pigmenti può avere un valore superiore ai 25 mo. Se dipingi un oggetto di valore superiore (un diamante o una pila d'oro), l'oggetto sembrerà autentico, ma un attento esame rivelerà che è fatto di gesso, ossa o qualche altro materiale privo di valore.

Se dipingi una forma di energia, come fuoco o fulmine, l'energia compare ma si dissipa non appena completi il dipinto, senza recare danni a niente.

\index[OggettiMagici]{Piuma Arcana}\subsubsection*{Piuma Arcana}
costo variabile, rarità variabile, questo minuscolo oggetto assomiglia a una piuma. Esistono diversi tipi di piume arcane, ciascuno dotato di un singolo effetto monouso. Il Narratore sceglie il tipo di piuma arcana.

\textit{Albero}. Devi trovarti all'aperto per poter usare questa piuma arcana. Puoi usare due azioni per appoggiarla a uno spazio non occupato sul terreno. La piuma svanisce e al suo posto spunta un albero di quercia non magico. L'albero è alto 18 metri e ha un tronco di 1,5 metri di diametro. In cima, i suoi rami si estendono per un massimo di 6 metri. 50 mo

\textit{Ancora}. Puoi usare due azioni per appoggiare la piuma arcana a una barca o nave. Per le successive 24 ore, il vascello non potrà essere mosso in alcun modo. Toccare di nuovo il vascello con la piuma arcana termina questo effetto. Quando l'effetto termina, la piuma svanisce. 50 mo

\textit{Frusta}. Puoi usare due azioni per lanciare la piuma arcana verso un punto entro 3 metri da te. La piuma svanisce e al suo posto compare una frusta fluttuante. Puoi poi usare due azioni per effettuare un attacco con incantesimo in mischia contro una creatura entro 3 metri dalla frusta, con un bonus di attacco +9. Se colpisci, il bersaglio subisce 1d6 + 5 danni da forza. Durante il tuo round, con due azioni puoi dirigere la frusta affinché voli per un massimo di 6 metri e ripeta l'attacco contro una creatura entro 3 metri da essa. La frusta svanisce dopo 1 ora, quando usi due azioni per congedarla, o quando sei inabile o muori. 250 mo

\textit{Nave Cigno}. Puoi usare due azioni per appoggiare la piuma arcana su di una massa d'acqua di almeno 18 metri di diametro. La piuma svanisce e al suo posto compare una barca lunga 15 metri e larga 6 metri dalla forma di cigno. La barca si sposta da sola e si muove in acqua alla velocità di 9 chilometri all'ora. Puoi usare due azioni, mentre sei a bordo per comandarle di muoversi o voltare di 90 gradi. La barca può trasportare fino a trentadue creature di taglia Media o inferiore. Una creatura Grande conta come quattro creature Medie, mentre una creatura Enorme conta come nove creature Medie. La barca svanisce dopo 24 ore. Puoi congedare la barca con due azioni. 3000 mo

\textit{Uccello}. Puoi usare due azioni per lanciare la piuma arcana 1,5 metri nell'aria. La piuma svanisce e un enorme uccello multicolore ne prende il posto. L'uccello ha le statistiche di un Roc, ma obbedisce a comandi semplici e non può attaccare. Può trasportare fino a 250 chili mentre vola alla sua velocità massima (24 chilometri all'ora per un massimo di 216 chilometri al giorno, con un'ora di riposo ogni 3 ore di volo), o 500 chili di peso a metà velocità. L'uccello svanisce dopo aver volato per la distanza massima possibile in un giorno o se scende a 0 Punti Ferita. Puoi congedare l'uccello con due azioni. 3000 mo

\textit{Ventaglio}. Se ti trovi su di una barca o una nave, puoi usare due azioni per lanciare la piuma arcana fino a 3 metri in aria. La piuma svanisce e un gigantesco ventaglio compare al suo posto. Il ventaglio galleggia e crea un vento forte abbastanza da gonfiare le vele della nave, aumentandone la velocità di 7,5 chilometri all'ora per 8 ore. Puoi congedare il ventaglio con due azioni. 250 mo

\index[OggettiMagici]{Polvere dell'Aridità}\subsubsection*{Polvere dell'Aridità}
120 mo, raro, questa piccola confezione contiene 1d6 + 4 pizzichi di polvere. Puoi usare due azioni per spargere un pizzico di polvere sull'acqua La polvere trasforma un cubo d'acqua di 4,5 metri di spigolo in una pallina di polvere delle dimensioni di una biglia, che fluttua o si deposita nel punto in cui è stata gettata la polvere. Il peso della pallina è trascurabile.

Chiunque può usare due azioni per spaccare la pallina contro una superficie dura, facendo sì che la pallina si rompa e liberi l'acqua assorbita dalla polvere. Farlo esaurisce la magia della pallina.

Un elementale composto principalmente d'acqua e che venga esposto a un pizzico di questa polvere, deve effettuare un tiro Salvezza su Tempra con DC 15, subendo 10d6 danni da Vuoto se lo fallisce, o la metà di questi danni se lo riesce.

\subsubsection*{Polvere Rivelatrice}\index[OggettiMagici]{Polvere Rivelatrice}
500 mo, non comune, questa polverina sottile sembra un pulviscolo metallico molto leggero. Una manciata di questa sostanza spruzzata in aria ricopre tutti gli oggetti in un raggio di 3 metri, rendendo ogni cosa visibile. Se viene spruzzata attraverso una cerbottana, la polvere riempie un cono lungo 6 metri e largo 1,5 metri all’estremità. La polvere annulla gli effetti di potere illusorio, del mantello distorcente, del mantello elfico e le capacità speciali di creature come i molossi instabili e le pantere distorcenti; l’effetto dura 2d10 turni. La polvere rivelatrice di solito viene conservata in piccoli sacchetti di seta o in tubetti cavi fatti d’osso; normalmente si trovano 5d10 dosi di polvere.

\index[OggettiMagici]{Polvere della Sparizione}\subsubsection*{Polvere della Sparizione}
700 mo, raro, rinvenuta in piccoli sacchetti, questa polverina sembra sabbia molto sottile. In un sacchetto ce n'è a sufficienza per un uso. Quando usi due azioni per lanciare la polvere in aria, tu e ciascuna creatura e oggetto entro 3 metri da te diventate invisibili per 2d4 minuti. La durata è la stessa per tutti i soggetti, e quando la magia prende effetto la polvere si consuma. Se una creatura sotto l'effetto della polvere attacca o lancia un incantesimo, l'invisibilità ha fine solo per quella creatura.

\index[OggettiMagici]{Polvere dello Starnuto e del Soffocamento}\subsubsection*{Polvere dello Starnuto e del Soffocamento}
480 mo, non comune, trovata in piccoli contenitori, questa polverina sembra sabbia sottile. Appare simile alla polvere della sparizione, e l'incantesimo identificare la rivela come tale. Ce n'è a sufficienza per un uso. Quando usi due azioni per lanciare una manciata di polvere in aria, tu e tutte le creature che necessitano di respirare e si trovino entro 9 metri da te dovete superare un tiro Salvezza su Tempra con DC 17 o smettere di respirare, e iniziare a starnutire in maniera incontrollabile. Una creatura afflitta a questo modo è inabile e soffoca. Finché è cosciente, la creatura può ripetere il Tiro Salvezza alla fine di ciascun suo round, terminando l'effetto in caso lo superi. Anche l'incantesimo ristorare inferiore può terminare l'effetto che affligge la creatura.

\index[OggettiMagici]{Portale Cubico}\subsubsection*{Portale Cubico}
40000 mo, leggendario, questo cubo di 7,5 centimetri di spigolo irradia una palpabile energia magica. Le sei facce del cubo sono ciascuna collegata a un diverso piano di esistenza, uno dei quali è il Piano Materiale. Le altre facce sono collegate a piani determinati dal Narratore.

Puoi usare due azioni per premere una faccia del cubo per lanciare tramite esso l'incantesimo portale, aprendo un passaggio verso il piano collegato a quella faccia. In alternativa, se usi due azioni per premere una faccia due volte, puoi lanciare l'incantesimo spostamento planare (DC del Tiro Salvezza 17) tramite il cubo e trasportarne i bersagli al piano collegato a quella faccia. Il cubo ha 3 cariche. Ogni uso del cubo spende 1 carica. Il cubo recupera 1 carica spesa ogni giorno all'alba.

\index[OggettiMagici]{Pozzo dei Molti Mondi}\subsubsection*{Pozzo dei Molti Mondi}
75000 mo, leggendario, questo elegante tessuto nero, soffice come la seta, è avvolto fino alle dimensioni di un fazzoletto. Si dispiega in un foglio circolare di 1,8 metri di diametro. Puoi usare due azioni per dispiegare e piazzare il pozzo dei molti mondi su di una superficie solida, su cui crea un portale bidirezionale verso un altro mondo o piano di esistenza. Ogni volta che l'oggetto apre un portale, il Narratore decide il posto a cui conduce. Puoi usare due azioni per chiudere un portale aperto afferrando i margini del tessuto e ripiegandoli. Una volta che un pozzo dei molti mondi ha aperto un portale, non potrà farlo di nuovo prima che siano passate 1d8 ore.

\subsubsection*{Rete Intralciante}\index[OggettiMagici]{Rete Intralciante}
800 mo, raro, questa rete quadrata di 3 m di lato può essere lanciata a un avversario per intralciarlo. La rete è molto resistente e occorre la forza di un gigante (For 5) per strapparla a mani nude. La rete resiste anche ai tagli, e deve essere colpita con estrema precisione (Difesa 25, PF 30) affinché ceda. La rete può anche essere appesa o messa sul terreno come una trappola, che si attiverà magicamente al comando del possessore.

\subsubsection*{Rete Intrappolante}\index[OggettiMagici]{Rete intrappolante}
900 mo, raro, questa rete può essere usata solo sott'acqua, ma funziona esattamente come una rete intralciante in superficie, fluttuando se occorre fino a 9 m di distanza per avvinghiare un avversario.

\subsubsection*{Scopa dell’Attacco animato}\index[OggettiMagici]{Scopa dell’Attacco animato}
questo oggetto è indistinguibile in apparenza da una scopa normale. A tutti i test risulta identica ad una scopa volante, fino vola a 6 metri d'altezza. Quando ciò avviene la scopa esegue una piroetta e fa cadere il suo pilota sulla testa da un'altezza di 1d4+5 x 30 cm (non viene inflitto danno da caduta poiché la distanza è inferiore a 3 m). La scopa quindi attacca la vittima, colpendola in viso con la spazzola e battendola con il manico. La scopa effettua due attacchi per round con ciascuna estremità (due attacchi con la spazzola e due col manico per un totale di quattro attacchi). La spazzola acceca la vittima per 1 round quando colpisce. Il manico infligge 1d3 ferite. La scopa ha Difesa 13, 18 Punti Ferita, e ha +4 al Tiro per Colpire.

\index[OggettiMagici]{Scopa Volante}\subsubsection*{Scopa Volante}
8000 mo, non comune, questa scopa di legno, del peso di circa 1,5 chili, funziona come una normale scopa fino a quando non vi siedi sopra e ne pronunci la parola di comando. Essa inizia così a fluttuare sotto di te e può essere cavalcata in aria. Ha velocità di volo 15 metri. Può trasportare fino a 200 chili, ma la sua velocità di volo diventa 9 metri se dovesse trasportare più di 100 chili. Quando atterri, la scopa smette di fluttuare.

Pronunciando la parola di comando, nominando il posto e se vi sei familiare, puoi inviare la scopa da sola in un posto fino a 1,5 chilometri da te. La scopa tornerà da te quando pronuncerai un'altra parola di comando, purché si trovi ancora entro 1,5 chilometri da te.

\subsubsection*{Scopa del Volo maledetto}\index[OggettiMagici]{Scopa del Volo maledetto}
questa scopa magica sembra una scopa volante. Tuttavia, quando viene attivata, vola fino a 15 m di altezza o fino al soffitto (se più basso) e poi smette di funzionare, facendo precipitare chi la cavalca. Dopodiché la scopa cade al suolo e perde il suo potere magico.

\index[OggettiMagici]{Sfera dell'Annientamento}\subsubsection*{Sfera dell'Annientamento}
250000 mo, leggendario, questa sfera nera di 50 centimetri di diametro è in realtà un foro nella struttura del multiverso, che fluttua nello spazio ed è stabilizzata dal campo magico che la circonda.

La sfera annienta tutta la materia che attraversa e tutta la materia che l'attraversa. L'unica eccezione sono gli artefatti. A meno che l'artefatto non sia suscettibile ai danni della sfera dell'annientamento, esso può attraversare la sfera senza problemi. Qualsiasi altra cosa tocchi la sfera e non ne sia completamente avvolta e annientata da essa, subisce 4d10 danni da forza a round.

La sfera resta immobile fino a quando qualcuno non la controlla. Se ti trovi entro 18 metri da una sfera incontrollata, puoi impiegare due azioni per effettuare una prova di Arcana con DC 30. Se la superi, la sfera levita in una direzione a tua scelta, per un numero di metri pari a 1,5 x il Intelligenza (minimo 1,5 metri). Se fallisci, la sfera si muove di 3 metri verso di te. Una creatura nel cui spazio entri la sfera, deve superare un Tiro Salvezza di Riflessi con DC 15 o venire toccata da essa, subendo 4d10 danni da forza.

Se tenti di controllare una sfera che si trova sotto il controllo di un'altra creatura, effettui una prova contesa di arcana contro arcana dell'altra creatura. Il vincitore della contesa ottiene il controllo della sfera e può farla levitare come di norma.

Se la sfera entra in contatto con un portale planare, come quello creato dall'incantesimo portale, o uno spazio extradimensionale, come quello all'interno di un buco portatile, il Narratore determina casualmente ciò che accade, utilizzando la tabella seguente.

\medskip

\begin{tabularx}{0.45\textwidth}{lX}
\textbf{3d6}& \textbf{Risultato}\\
\hline
3-10 &La sfera è distrutta\\
12-16& La sfera si muove attraverso il portale o all'interno dello spazio extradimensionale.\\
17-18 &Un squarcio spaziale spedisce ogni creatura e oggetto entro 54 metri dalla sfera, sfera inclusa, in un piano dell'esistenza casuale.\\
\end{tabularx}

\medskip

\index[OggettiMagici]{Solvente Universale}\subsubsection*{Solvente Universale}
300 mo, leggendario, questo tubetto contiene un liquido bianco con un forte odore di alcool. Puoi usare due azioni per versarne i contenuti su di una superficie a portata. Il liquido dissolve istantaneamente 1000 cm x cm di adesivo con cui entra in contatto, compresa la colla suprema.

\subsubsection*{Specchio dell’Abilita' mentale}\index[OggettiMagici]{Specchio dell’Abilità mentale}
15000 mo, molto raro, questo oggetto sembra un normale specchio alto un metro e mezzo e largo 60 cm. A comando, il possessore può usarlo nei seguenti modi:

- Leggerei pensieri di una persona riflessa sulla sua superficie con la telepatia (senza bisogno di capire una lingua sconosciuta).

- Vedere altri luoghi come con una palla di cristallo, con la possibilità di vedere in altri piani, purché siano sufficientemente familiari all’osservatore.

- Creare un portale per visitare altri luoghi. Il possessore dovrà prima visualizzare il luogo, poi entrare fisicamente nello specchio, da solo o con gli accompagnatori che desidera. Lo specchio creerà un portale invisibile dall’altra parte, attraverso cui il possessore, o chiunque riesca a individuarlo, potrà attraversarlo.

- Una volta alla settimana, lo specchio può rispondere con precisione a una domanda riguardante una persona riflessa sulla sua superficie (un effetto simile all’incantesimo conoscenza delle leggende.

\subsubsection*{Specchio della Duplicazione}\index[OggettiMagici]{Specchio della Duplicazione}
leggendario, questo specchio è alto un po’ più di un metro e largo un po’ di meno. Quando una creatura si riflette sulla superficie dello specchio, la sua immagine riflessa (un duplicato identico in tutto e per tutto) esce per attaccare l'originale. Il duplicato ha tutto l’equipaggiamento e i poteri dell'originale, compresa la magia. Il duplicato sparisce immediatamente, assieme a tutti i suoi oggetti, alla morte sua o dell'originale.

\index[OggettiMagici]{Specchio Intrappola Vita}\subsubsection*{Specchio Intrappola Vita}
18000 mo, raro, quando questo specchio alto 120 centimetri viene guardato in maniera indiretta, la sua superficie mostra una vaga immagine della creatura. Lo specchio pesa 25 chili, ha Difesa 11, 10 Punti Ferita e vulnerabilità ai danni da botta. Si frantuma ed è distrutto quando viene ridotto a 0 Punti Ferita.

Se lo specchio è appeso a una superficie verticale e ti trovi entro 1,5 metri da esso, puoi usare due azioni per pronunciare la sua parola di comando e attivarlo. Rimarrà attivo fino a quando non pronuncerai di nuovo la parola di comando.

Qualsiasi creatura, a parte te, che veda il suo riflesso nello specchio attivato mentre si trova entro 9 metri da esso deve superare un Tiri Salvezza su Volontà con DC 17 o finire intrappolata, insieme a tutto ciò che indossa o trasporta, in una delle dodici celle extradimensionali dello specchio. Questo Tiro Salvezza riceve +1d6 se la creatura conosce la natura dello specchio ed i costrutti riescono automaticamente il Tiro Salvezza.

Una cella extradimensionale è uno spazio infinito colmo di una densa foschia che riduce la visibilità a 3 metri. Le creature intrappolate nelle celle dello specchio non invecchiano, e non hanno bisogno di mangiare, bere o dormire. Una creatura intrappolata all'interno di una cella può fuggirne usando la magia che permette di viaggiare tra i piani. Altrimenti, la creatura è confinata nella cella fino a quando non sarà liberata.

Se lo specchio intrappola una creatura ma le sue dodici celle extradimensionali sono già occupate, lo specchio libera una delle creature intrappolate a caso per alloggiare il nuovo prigioniero. La creatura liberata compare in uno spazio non occupato in vista dello specchio ma rivolta dalla parte opposta. Se lo specchio viene infranto, tutte le creature che contiene sono liberate e ricompaiono in uno spazio non occupato in sua prossimità.

Mentre ti trovi entro 1,5 metri dallo specchio, puoi usare due azioni per pronunciare il nome di una delle creature intrappolate al suo interno o richiamare un particolare numero di cella. La creatura nominata o contenuta nella cella nominata appare come immagine sulla superficie dello specchio. Dopodiché tu e la creatura nominata potete comunicare normalmente.

In un modo simile, puoi usare due azioni per pronunciare una seconda parola di comando e liberare una delle creature intrappolate nello specchio. La creatura liberata compare, insieme a tutte le sue proprietà, nello spazio non occupato più vicino allo specchio e rivolta nella direzione opposta a esso.

\subsubsection*{Tamburi del Panico}\index[OggettiMagici]{Tamburi del Panico}
1500 mo, non comune, questi tamburi sono simili a timpani (piccoli strumenti a percussione facilmente trasportabili). Si trovano a coppie e hanno un aspetto poco appariscente. Se vengono suonati entrambi, tutte le creature entro 72 m (tranne quelle all’interno di un cerchio di 3 m centrato sui tamburi) vengono assalite da Paura e fuggono per 30 round alla massima velocità. È consentito un Tiro Salvezza su Volontà a DC 21 per salvarsi dagli effetti.

\subsubsection*{Tamburi dello Stordimento}\index[OggettiMagici]{Tamburi dello Stordimento}
raro, questi due tamburi accoppiati somigliano ai tamburi del panico; quando vengono suonati entrambi, tutte le creature entro 3 m devono riuscire in un Tiro Salvezza su Tempra DC 21 essere stordite per 2d4 round. Tutte le creature entro 21 m sono immediatamente assordate. Gli incantesimi ristorare superiore, guarigione, rigenerazione o effetti simili possono curare la sordità.

\index[OggettiMagici]{Tappeto Volante}\subsubsection*{Tappeto Volante}
15000 mo, molto raro, puoi pronunciare la parola di comando del tappeto con due azioni per far fluttuare e volare il tappeto. Esso si muove in base alle direzioni indicategli a voce, purché ti trovi entro 9 metri da esso.

Esistono quattro taglie di tappeto volante. Il Narratore sceglie la taglia del tappeto o la determina casualmente.

\medskip

\begin{tabular}{llll}
d100 &Taglia (cm)&Capacità &Vel. di Volo\\
01-20& 90 x 150 &100 kg/25&24 metri\\
21-55& 120 x 180 &200 kg/50&18 metri\\
56-80& 150 x 210 &300 kg/75&12 metri\\
81-100& 180  x 270 & 400 kg/100& 9 metri\\
\end{tabular}

\medskip
Il valore di Capacità indica sia il peso trasportato che l'Ingombro. Il tappeto può trasportare fino al doppio del carico indicato sulla tabella, ma vola a velocità dimezzata se trasporta di più.

\subsubsection*{Turibolo Elementale dell’aria}\index[OggettiMagici]{Turibolo Elementale dell’aria}
1500 mo, raro, questo turibolo può essere usato per evocare e controllare un elementale dell’aria in modo analogo all’incantesimo evoca elementale. È necessario preparare l'oggetto magico e condurre un rituale per un turno prima dell’evocazione vera e propria, che richiede un round. Dopo che l’elementale è stato evocato, occorre mantenere la concentrazione per potergli impartire gli ordini.

\subsubsection*{Turibolo dell’Evocazione maledetta}\index[OggettiMagici]{Turibolo dell’Evocazione maledetta}
raro, questo turibolo ha l'aspetto di, e sembra funzionare come, un turibolo elementale dell’aria. Tuttavia, una volta acceso è impossibile spegnerlo per 1d4 round. In ciascun round un elementale dell’aria emerge ed attacca tutte le creature vicine.

\index[OggettiMagici]{Unguento Ristorativo}\subsubsection*{Unguento Ristorativo}
5000 mo, non comune, questa giara di vetro, 7,5 centimetri di diametro, contiene 1d4 + 1 dosi di una densa mistura. La giara e i suoi contenuti pesano 250 grammi. Con due azioni, si può inghiottire o applicare sulla pelle una dose di unguento. La creatura che lo riceve recupera 2d8 + 2 Punti Ferita, smette di essere avvelenata e viene curata da qualsiasi malattia.

\index[OggettiMagici]{Vano Portatile}\subsubsection*{Vano Portatile}
10000 mo, raro, questo elegante tessuto nero, soffice come la seta, si piega fino alle dimensioni di un fazzoletto e si dispiega fino ad cerchio di 1,8 metri di diametro. Puoi usare due azioni per dispiegare un Vano portatile e piazzarlo sopra o contro una superficie solida, sulla quale il Vano portatile crea un foro extradimensionale profondo 3 metri. Lo spazio cilindrico all'interno del foro si trova su di un piano diverso, e quindi non può essere usato per aprire dei passaggi. Qualsiasi creatura all'interno di un Vano portatile aperto può uscirne fuori arrampicandosi fuori di esso.

Puoi usare due azioni per chiudere un Vano portatile prendendo i margini del tessuto e ripiegandolo. Piegare il tessuto chiude il Vano, e qualsiasi creatura od oggetto al suo interno rimane nello spazio extradimensionale. Non importa quello che contiene, il Vano non pesa nulla.

Se il Vano viene ripiegato, una creatura all'interno dello spazio dimensionale del Vano può usare due azioni per effettuare una prova di Forza con DC 10. Se la prova riesce, la creatura riesce a liberarsi e ricompare entro 1,5 metri dal Vano portatile o della creatura che lo trasporta. Una creatura che respira può sopravvivere all'interno di un buco portatile chiuso per un massimo di 10 minuti, dopodiché iniziare a soffocare.

Piazzare un Vano portatile all'interno dello spazio extradimensionale creato da una borsa conservante, uno zainetto pratico o simile oggetto distrugge istantaneamente entrambi gli oggetti e apre un portale verso il Piano Astrale. Qualsiasi creatura entro 3 metri dal portale viene risucchiata al suo interno e depositata in un luogo casuale del Piano Astrale. Poi il portale scompare.

\index[OggettiMagici]{Ventaglio Arcano}\subsubsection*{Ventaglio Arcano}
1500 mo, non comune, mentre impugni questo ventaglio, puoi usare due azioni per lanciare tramite esso l'incantesimo folata di vento (DC del Tiro Salvezza 15). Una volta usato, il ventaglio
non dovrebbe essere usato di nuovo fino alla prossima alba. Ogni volta che venga usato prima di allora, c'è una probabilità cumulativa del 20\% che non funzioni e si rompa in inutili brandelli privi di magia.


\index[OggettiMagici]{Zainetto Pratico}\subsubsection*{Zainetto Pratico}
7000 mo, raro, questo zaino ha una sacca centrale e due laterali, ciascuna delle quali è in realtà uno spazio extradimensionale. Ogni sacca laterale può contenere 10 chili di materiale, che non ecceda un volume di 60 dm3

La grande sacca centrale può contenere fino a 240 dm3 o 40 chili di materiale. Lo zaino pesa sempre 2,5 chili, quali che siano i suoi contenuti.

Piazzare un oggetto all'interno dello zainetto segue le normali regole di interazione con gli oggetti. Recuperare un oggetto dallo zainetto richiede l'uso di due azioni. Quando cerchi un oggetto nello zainetto, questo magicamente si troverà sempre in cima alla pila degli oggetti che questo contiene.

Lo zainetto ha alcune limitazioni. Se sovraccarico, o un oggetto affilato lo taglia o si strappa, lo zainetto si spacca e viene distrutto. Se lo zainetto è distrutto, ciò che conteneva è perso per sempre, sebbene un artefatto ricomparirà sempre da qualche parte nel multiverso. Se lo zainetto viene rivoltato, ciò che contiene viene espulso, senza recargli danno, e lo zainetto deve essere rimesso al verso giusto prima che possa essere usato di nuovo. Se una creatura che respira viene posta all'interno dello zainetto, vi può sopravvivere per al massimo 10 minuti, prima di cominciare a soffocare.

Piazzare lo zainetto all'interno dello spazio extradimensionale creato da una borsa conservante, un buco portatile o un oggetto simile distrugge immediatamente entrambi gli oggetti e apre un portale verso il Piano Astrale. Il portale origina dal punto in cui gli oggetti sono stati posti l'uno dentro l'altro. Qualsiasi creatura entro 3 metri dal portale viene risucchiata attraverso di esso e trascinata in un luogo casuale del Piano Astrale. Poi il portale si chiude. Il portale è a senso unico e non può essere riaperto.

\subsubsection*{Zappa dei Titani}\index[OggettiMagici]{Zappa dei Titani}
2000 mo, non comune, questo strumento sovradimensionato è lungo 3 m e pesante 120 kg (30 Ingombro), e può essere usato solo da un gigante (o da un personaggio ingrandito) per spostare grandi quantità di terriccio e costruire terrapieni (un cubo di 3 m per turno). La zappa può anche essere usata per spaccare la pietra con grande rapidità. Se usata come un'arma ha un bonus +3 al colpire e infligge 5d6 ferite.

\index[OggettiMagici]{Zoccoli della Velocità}\subsubsection*{Zoccoli della Velocità}
5000 mo, raro, questi zoccoli di ferro si trovano in set da quattro. Quando tutti e quattro gli zoccoli sono fissati a un cavallo o creatura simile, aumentano la velocità di passeggio di quella creatura di 9 metri.

\index[OggettiMagici]{Zoccoli dello Zefiro}\subsubsection*{Zoccoli dello Zefiro}
1500 mo, molto raro, questi zoccoli di ferro si trovano in set da quattro. Quando tutti e quattro gli zoccoli sono fissati a un cavallo o creatura simile, permettono a quella creatura di muoversi normalmente, mentre fluttua a circa 10 centimetri dal terreno. Questo effetto vuol dire che la creatura può attraversare o passare sopra superfici non solide o instabili, come l'acqua o la lava. La creatura non lascia tracce e ignora il terreno difficile. Inoltre, la creatura può muoversi alla sua normale velocità per un massimo di 12 ore al giorno senza subire l'affaticamento a causa della marcia forzata.

\end{multicols}

\pagebreak

\section{Oggetti Maledetti}\index{Oggetti Maledetti}

\begin{changemargin}{0cm}{0.5cm}\begin{enfasi}{Quando un empio maledice l'avversario, maledice se stesso. (Siracide)

\medskip

Se maledici una persona ci saranno due fosse. (Proverbio giapponese)}
\end{enfasi}\end{changemargin}\medskip

\begin{multicols}{2}

\label{oggetti-maledetti}

\lettrine[lines=2, lhang=0.33, loversize=0.25, findent=1.5em]{G}{li} oggetti maledetti sono oggetti magici dotati di un'influenza potenzialmente negativa sul personaggio. A volte tendono a confondere il male con il bene, costringendo il loro possessore a fare scelte difficili.

Gli oggetti maledetti non sono quasi mai realizzati intenzionalmente, ma piuttosto sono il risultato di un lavoro mal riuscito, di artigiani con poca esperienza o della mancanza di componenti adeguati o patti non rispettati con qualche Patrono.

Il Narratore può chiedere una prova di Arcana con una DC pari a 10+giorni impiegati per costruire l'oggetto magico in caso di oggetti particolarmente complessi o ci siano state situazioni problematiche nella creazione e quando la prova fallisce di 10 o più o si sia stato un fallimento critico (due 1, due 2 ed un 1) tirate sulla tabella per determinare il tipo di maledizione che l'oggetto possiede.

Una maledizione può manifestarsi a seguito dalle influenze negative od emozionali estreme che influenzano un oggetto.

\medskip

\textbf{Maledizioni Comuni degli Oggetti}

\medskip

\begin{tabular}{ll}
\textbf{\%} & \textbf{Maledizione}\\
\toprule
01-15     & Inganno\\
16-40       & Effetto o Bersaglio Opposto\\
41-50       & Funzionamento Discontinuo\\
51-65       & Requisito\\
66-90       & Inconveniente\\
91-100      & Effetto completamente diverso\\
\end{tabular}

\medskip

Gli oggetti maledetti sono \hypertarget{oggettimaledettiid}{identificati} come qualsiasi altro oggetto magico con una sola eccezione: a meno che non la prova di Arcana per identificare l'oggetto non superi 30 o l'incantesimo Identificare sia lanciato con una Prova di Magia ed ottenga un critico magico (2 volte 6) la maledizione non viene individuata. Se la prova è sotto 30 o senza critico magico tutto quello che viene rivelato è l'originale scopo dell'oggetto magico.

Se si sa che l'oggetto è maledetto, la natura della maledizione può essere determinata usando la DC \hyperlink{identificareom}{standard} per identificare l'oggetto.

\begin{center}
\includegraphics[width=0.75\linewidth]{immagini/vasobasano.png}

\textit{Vaso di Basano. Questo vaso è stato realizzato nella seconda metà del XV secolo ed è realizzato in argento.}
\end{center}


\begin{changemargin}{0.3cm}{0.3cm}\begin{narratore}
Una maledizione è sempre un \textit{inconveniente} particolare, che non si usa a caso. Ragionate attentamente sugli oggetti maledetti che farete trovare ai personaggi perché vi chiederanno molte informazioni e dovrete essere pronti.

Non c'è bisogno che la maledizione sia eccessiva e limitante può essere benissimo ridicola o particolare, fate in modo che sia caratterizzante. Il personaggio non deve sentirsi (tranne se lo volete) condannato in eterno, sfruttate l'occasione per costruire nuove avventure e spirito di gruppo.
\end{narratore}\end{changemargin}


\subsection{Rimuovere Oggetti Maledetti}\index{Rimuovere Oggetti Maledetti}

Mentre alcuni oggetti maledetti possono essere semplicemente posati, altri esercitano una forte compulsione sul possessore a tenerli con sé, a qualsiasi costo. Altri riappaiono anche se abbandonati o è impossibile gettarli via.

Questi oggetti possono essere rimossi solo dopo che sul personaggio o l'oggetto viene lanciato l'incantesimo Rimuovi Maledizione.

Se l'oggetto è stato maledetto tramite l'incantesimo Scagliare Maledizione, o comunque il Narratore decide che l'oggetto ha una maledizione particolare allora la Competenza Magica necessaria per rimuovere la maledizione deve essere superiore o pari a quella di colui che l'ha lanciato.
Un incantatore può lanciare Rimuovi Maledizione con una Prova di Magia e per ogni successo critico aggiunge 2 alla propria Competenza Magica per capire se è in grado di rimuovere la maledizione.

Se la prova ha successo, l'oggetto può essere rimosso nel round successivo, ma la maledizione rimane e colpisce nuovamente se l'oggetto viene usato/indossato un'altra volta.

Ogni oggetto maledetto ha un proprio metodo per essere distrutto, dall'essere gettato in un vulcano attivo, ad essere colpito dal martello del dio del Tuono (o Patrono...) oppure divorato da un Verme colossale delle sabbie se non colpito dal soffio di un drago rosso e un drago bianco contemporaneamente, ma anche usando un Scacciare Maledizioni con 3 critici se l'oggetto maledetto non è un artefatto...

\subsection{Effetti Comuni degli Oggetti Maledetti}

Gli effetti più comuni degli oggetti maledetti sono i seguenti, il Narratore può inventare nuovi effetti particolari per specifici oggetti maledetti.

\subsubsection{Inganno}

Chi utilizza l'oggetto continua a credere che sia ciò che sembra a prima vista, ma in realtà non ha alcun potere, a parte quello di ingannare. Chi lo usa è mentalmente spinto a credere che funzioni, e non può essere convinto del contrario se non con l'uso di Rimuovi maledizione

\begin{center}
\includegraphics[width=0.70\linewidth]{immagini/mirror.png}

\textit{The mirror in The Myrtles Plantation.}
\end{center}

\subsubsection{Effetto o Bersaglio Opposto}

Questi oggetti maledetti tendono ad avere dei difetti di funzionamento che in alcuni casi generano effetti diametralmente opposti a quelli desiderati dal loro creatore, mentre in altri casi tendono a colpire chi li utilizza invece di qualcun altro.

Ma la cosa più interessante è che questi oggetti potrebbero anche non essere uno svantaggio per chi li possiede. La categoria degli oggetti magici dagli effetti opposti include anche le armi che infliggono penalità ai Tiri per Colpire e per i danni, invece che bonus.

Visto che un personaggio non dovrebbe sapere immediatamente quale sia il bonus di un oggetto magico, non dovrebbe venire a conoscenza nemmeno della natura della sua maledizione. Una volta che lo verrà a sapere per liberarsi dall'oggetto sarà necessario l'Incantesimo Rimuovi maledizione.

Alcune maledizione particolarmente forti, a discrezione del Narratore, possono essere rimosse da Rimuovi Maledizioni lanciato da un incantatore molto esperto (controllo valore della Competenza Magica).

\subsection{Funzionamento Discontinuo}

Gli oggetti discontinui funzionano esattamente come dovrebbero, quando funzionano. Stabilite se l'oggetto è Inaffidabile, Condizionato oppure Incontrollabile.

\medskip
\subsubsection{Inaffidabile}

Ogni volta che l'oggetto viene attivato, c'è una probabilità del 5\% che non funzioni.

\subsubsection{Condizionato}

Questo oggetto funziona solo in determinate situazioni. Per determinare quali siano, scegliete una condizione di attivazione o consultate la tabella poco sotto.

\subsubsection{Incontrollabile}

Un oggetto incontrollabile tende ad attivarsi casualmente. Tirare un d\% ogni giorno. Con un risultato di 01--05 l'oggetto si attiva spontaneamente in un certo momento del giorno.

\medskip

\begin{tabularx}{0.45\textwidth}{lX}
\textbf{\%} & \textbf{Situazione}\\
\toprule
01-03       & Temperatura sotto lo zero\\
04-05       & Temperatura sopra lo zero\\
06-10       & Durante il giorno\\
11-15       & Durante la notte\\
16-20       & Esposto alla luce solare\\
21-25       & In assenza di luce solare\\
26-34       & Sott'acqua\\
35-37       & Fuori dall'acqua\\
38-45       & Sottoterra\\
46-55       & In superficie\\
56-60       & Entro 3 metri da un tipo di creatura\\
61-64       & Entro 3 metri da una razza o tipo di creatura\\
65-72       & Entro 3 metri da un incantatore\\
73-80       & Entro 3 metri da un Seguace o Devoto di un Patrono specifico\\
81-85       & Nelle mani di un personaggio non incantatore\\
86-90       & Nelle mani di un personaggio incantatore\\
91-95       & Nelle mani di una creatura con particolare Tratto\\
96          & Nelle mani di una creatura di un particolare genere\\
97-99       & Nei giorni non sacri o durante particolari ricorrenze astronomiche\\
100         & A più di 150 km da un determinato luogo\\
\end{tabularx}

\subsection{Requisito}

Alcuni oggetti hanno requisiti molto più difficili da soddisfare perché funzionino. Per far funzionare l'oggetto in questione, potrebbe essere necessario soddisfare una delle seguenti condizioni:

\begin{itemize}
\item Il personaggio deve mangiare il doppio del normale.
\item Il personaggio deve dormire il doppio del normale.
\item Il personaggio deve compiere almeno una missione specifica.
\item Il personaggio deve sacrificare (distruggere) un valore pari a 100 mo di oggetti o materiali preziosi al giorno.
\item Il personaggio deve giurare lealtà ad un nobile in particolare o alla sua famiglia.
\item Il personaggio deve abbandonare tutti gli altri oggetti magici.
\item Il personaggio deve essere un Seguace o Devoto di uno specifico Patrono
\item Il personaggio deve avere un numero minimo di gradi in una particolare competenza.
\item Il personaggio deve sacrificare parte della propria energia vitale (1 punto di Costituzione permanente) la prima volta che usa l'oggetto.
\item L'oggetto deve essere purificato con l'acqua sacra di uno specifico Patrono ogni giorno.
\item L'oggetto deve essere bagnato in almeno mezzo litro di sangue (animale o umanoide) al giorno.
\item L'oggetto deve essere usato per uccidere una creatura vivente al giorno.
\item L'oggetto deve essere usato almeno una volta al giorno, o smette di funzionare per il suo attuale possessore.
\item Quando viene brandito, l'oggetto deve spillare sangue (solo armi). Non può essere messo da parte o cambiato con un altro oggetto finché non ha messo a segno un colpo.
\end{itemize}

\medskip

\begin{center}
\includegraphics[width=0.8\linewidth]{immagini/donnalemb.png}

\textit{Donna di Lemb o Statua della Dea della Morte, 3500 AC}
\end{center}

\medskip

I requisiti dipendono dalla convenienza dell'oggetto che non dovrebbero mai essere determinati a caso. Un oggetto intelligente con un requisito spesso impone il proprio requisito grazie alla sua personalità.

Se il requisito non viene soddisfatto, l'oggetto smette di funzionare. Se invece viene soddisfatto, di solito l'oggetto funziona per un giorno intero prima di dover di nuovo soddisfare il requisito (anche se alcuni requisiti vanno soddisfatti una volta sola, altri una volta al mese e altri ancora in continuazione).

\subsection{Inconveniente}

Gli oggetti che hanno degli inconvenienti hanno solitamente degli effetti positivi su chi li usa, ma hanno anche degli aspetti negativi. Anche se a volte gli inconvenienti vengono alla luce solo quando gli oggetti sono utilizzati (o tenuti in mano, nel caso di oggetti come le armi), di solito rimangono presenti fino a quando il personaggio non si libera dell'oggetto in questione.

A meno che non sia indicato diversamente, gli inconvenienti rimangono attivi per tutto il tempo in cui l'oggetto rimane in possesso del personaggio. La DC dei Tiro Salvezza per evitare questi effetti è pari a 10 + DC della maledizione (se non avete stabilito la difficoltà impostate il Tiro Salvezza, solitamente su Volontà, a DC 25)

\end{multicols}

\medskip

\begin{changemargin}{0.3cm}{0.3cm}\begin{narratore}L'elenco è di esempio per poter generare casualmente degli effetti sul possessore dell'oggetto. Prendete spunto e siate creativi!  Non fate però che una maledizione renda impossibile giocare il personaggio piuttosto deve essere vissuta come l'occasione per provare, fare, qualcosa di diverso. Non gettate mai un oggetto maledetto a caso nel mucchio dei tesori, pensate sempre cosa potrà accadere e quali conseguenze si genereranno. Un oggetto maledetto richiede sempre un alto livello di attenzione e pianificazione da parte del Narratore\end{narratore}\end{changemargin}

\bigskip

\textbf{Tabella: Effetti degli oggetti magici maledetti}\index{Tabella Effetti degli oggetti magici maledetti}

\medskip

{\small
\begin{tabularx}{0.95\textwidth}{lX}
\textbf{\%} & \textbf{Inconveniente}\\
\toprule
01-02& I capelli del personaggio crescono di 2,5 cm all'ora.\\
02-04& Le unghie del personaggio crescono di 1 cm ogni 8 ore\\
05-06   & L'altezza del personaggio diminuisce di 5d10 cm \\
07-09   & L'altezza del personaggio aumenta di 5d10 cm \\
10-11   & La temperatura intorno all'oggetto è di 5° C più fredda del normale.\\
12-13   & La temperatura intorno all'oggetto è di 20° C più fredda del normale.\\
14-15   & La temperatura intorno all'oggetto è di 5° C più calda del normale.\\
16-17   & La temperatura intorno all'oggetto è di 20° C più calda del normale.\\
18-20   & Il colore dei capelli del personaggio cambia.\\
21-23   & II colore della pelle del personaggio cambia.\\
24& Il colore dei capelli del personaggio cambia ogni ora\\
25& Il colore della pelle del personaggio cambia ogni ora\\
26      & Delle corna come un montone crescono sulla testa del personaggio\\
27      & Un palco di corna come un alce crescono sulla testa del personaggio\\
28-29   & II personaggio ora porta un segno distintivo (un tatuaggio, una strana luminescenza ecc.).\\
30-32   & II sesso del Personaggio cambia ogni giorno all'alba.\\
33-34   & La razza o la specie del Personaggio cambiano.\\
35      & II PG viene colpito da una Malattia determinata casualmente, che non può essere curata.\\
36-39   & L'oggetto emette costantemente suoni sgradevoli (lamenti, maledizioni, insulti...).\\
40      & L'oggetto ha un aspetto ridicolo (colori sgargianti, forma, brilla di un alone rosa ecc.).\\
41      & Un unicorno blu, visibile solo con la magia, di dimensioni piccole vola sempre attorno al Personaggio dando consigli inutili e facendo battute stupide.\\
42& Ogni giorno ti prende una improvvisa voglia e capacità di fare l'uncinetto per almeno 1 ora.\\
43-45   & II personaggio diventa estremamente possessivo nei confronti dell'oggetto.\\
46-49   & II personaggio ha una paura incontrollabile di perdere l'oggetto o che venga danneggiato.\\
50      & Un Tratto viene sostituito\\
51& Il metabolismo del personaggio cambia e diventa esclusivamente carnivoro\\
52& Il metabolismo del personaggio cambia e diventa esclusivamente vegetariano\\
53-54   & II personaggio deve attaccare la creatura a lui più vicina (probabilità del 5\% ogni giorno).\\
55-57   & II personaggio rimane Stordito per 1d4 round ogni volta che l'oggetto è servito al suo scopo\\
58-60   & Il personaggio diventa sordo\\
61-64   & I Punti Ferita massimi calano di 10 permanentemente (rimanendo con un minimo di 1).\\
65      & I Punti Ferita massimi calano di 20 permanentemente (rimanendo con un minimo di 1).\\
66-68   & Il PG acquisisce una Fobia a caso.\\
69-71   & TS su Volontà ogni giorno all'alba con mod. Intelligenza o subisce 1 danno a Intelligenza permanente.\\
72-74   & TS su Volontà ogni giorno all'alba o subisce 1 danno a Saggezza permanente.\\
75-77   & TS su Volontà ogni giorno all'alba con mod. Carisma o subisce 1 danno a Carisma permanente.\\
78-80   & TS su Tempra ogni giorno all'alba con mod. Forza o subisce 1 danno a Forza permanente.\\
81-83   & TS su Tempra ogni giorno all'alba con mod. Destrezza o subisce 1 danno a Destrezza permanente.\\
84-85   & TS su Tempra ogni giorno all'alba o subisce 1 danno a Costituzione permanente.\\
86-89& Il PG incomincia a parlare di se in terza persona.\\
90-92& Cavalli, cani e gatti domestici diventano ostili.\\
93& Un Patrono farà di tutto per ucciderti.\\
94      & Il PG viene teletrasportato a 10d100 Km di distanza ogni giorno all'alba.\\
95      & II personaggio viene trasformato in una creatura a caso di una specie specifica (probabilità del 5\% ogni giorno).\\
96      & II personaggio viene trasformato in una creatura specifica (probabilità del 5\% ogni giorno).\\
97      & II personaggio non può più usare oggetti magici o Incantesimi con livello oltre 5\\
98      & II personaggio non può più usare oggetti magici o Incantesimi con livello oltre 3\\
99      & II personaggio non può più usare Incantesimi\\
100     & Tira due volte\\
\end{tabularx}}
