\section{La storia fino ad adesso...}

\begin{multicols}{2}

Il mondo come lo conoscevamo è un ricordo sbiadito, una tela lacerata da cataclismi e dalla furia degli dei. Leggende, miti e fantasia si sono intrecciati in un guazzabuglio cacofonico con la realtà dei fatti.

Da qualche parte nel terzo millennio del vecchio calendario, avvenne l’impensabile: ciò che mai si sarebbe potuto immaginare o desiderare. Da un giorno all’altro, la Terra si trovò coinvolta in una guerra tra entità di potenza divina, che, con la complicità delle varie nazioni, non fecero altro che distruggere il nostro povero mondo.

La \emph{Freten} era un'azienda che sviluppava sistemi energetici alternativi, basati sulla possibilità di attingere energia da altrove o, come dicevano loro, dal vuoto cosmico.
Non è mai stato chiarito quali furono le origini dei loro esperimenti; molto probabilmente avevano effettivamente trovato qualcosa (\emph{qualcuno?}), che potesse funzionare da portale per attingere a questa forma di energia pressoché illimitata.

Nel giorno dell'inaugurazione del loro primo reattore alimentato da ciò che chiamavano  \textbf{Omniessenza}, una \emph{parte} della loro \emph{invenzione}, avvenne l'impossibile.

I racconti si fanno molto confusi a questo punto, di fatto l'\emph{Omniessenza} era effettivamente qualcosa di vero e di \emph{vivo}, una parte di una energia più grande. All'attivazione del reattore questo esplose con un'energia e forza mai viste sulla Terra, buona parte di quelli che erano gli stati centrali degli USA vennero vaporizzati all'istante.

Nel punto dove una volta sorgeva la sede della Freten si aprì una breccia simile a un portale: una colossale fiamma divisa in due lingue di fuoco di colore diverso.

Da questa fiamma uscì un mastodontico \emph{drago rosso} le cui scaglie erano impenetrabili a qualsiasi arma umana. Tàhil, questo il suo nome, nelle prime 24 ore distrusse quello ciò che restava della costa orientale degli Stati Uniti.

All'alba del secondo giorno una mistica energia avvolse Tàhil e questo mutò in un drago a molte teste di colori diversi. Nel suo petto apparì un'altra breccia ed a ogni passo una moltitudine di altri draghi, per fortuna \emph{leggermente} più piccoli incominciò ad uscire.

All'alba del terzo giorno Tàhil pronunciò l'Editto della Dimenticanza, un'onda magica che fuse i componenti di ogni apparato elettrico e cancellò ogni dato. Il sapere stesso divenne un enigma, le parole dei libri si mescolarono in un caos incomprensibile.

Tàhil si stabilì in piazza Medan Merdeka a Giacarta, un posto sufficientemente ampio per permettergli di sdraiarsi e comandare i suoi eserciti di draghi.

All’alba del quarto giorno, Tàhil scatenò l’Editto della Guerra: per sette giorni, gli umani si combatterono tra loro, distruggendo ciò che restava della loro civiltà.

All’alba del dodicesimo giorno, Tàhil proclamò l’Editto dell’Anarchia, e i popoli abbatterono le forme di governo costituite. Il mondo sprofondò nel caos, senza leggi né ordine.

All’alba del diciannovesimo giorno, l’Editto del Sacrificio uccise un terzo della popolazione, in una dimostrazione spietata di potere.

All’alba del trentesimo giorno, Tàhil proclamò l’Editto della Rifondazione. Nuove brecce si aprirono, e altri esseri, altri poteri si manifestarono. Il mondo fu trasformato e nuove regole vennero scritte.

Intanto e per 1 anno intero i draghi distrussero e uccisero qualsiasi cosa, ogni persona. Nessun esercito sopravvisse, nessun governo rimase in carica, nessuna nazione si poteva ancora chiamare tale.

I Terrestri erano stati puniti per il loro affronto, solo il 10\% della popolazione era sopravvissuta.

Questi nuovi esseri facevano scomparire, sprofondare, ribaltare; distruggevano intere città, mutavano ambienti e creature, facevano comparire nuove specie. Dal nulla apparivano orde di mostri, come quelli descritti nei libri di gioco dei bambini. La realtà, per loro, era un capriccio da plasmare secondo gusti eccentrici.

Le nazioni, come le conoscevamo, non esistevano più. Anche la natura si era trasformata, assumendo forme tra le più aliene immaginabili. Molte zone erano divenute deserti nucleari, inospitali e letali per chiunque... o quasi.

Poi, tutte le entità, tranne Tàhil e i draghi, sparirono nel nulla per sei mesi.
Trascorso quel tempo, i sogni dei pochi esseri rimasti cominciarono a essere invasi da visioni di altri \emph{esseri}, altre entità.

E arrivò così la seconda ondata dei \emph{Patroni} come collettivamente si facevano chiamare. Per fortuna questi esseri si rivelarono, tutto sommato, più gentili e \emph{umani}, o almeno qualcuno lo era. Bonificarono buona parte delle zone radioattive ed insegnarono a chi accettava i loro Tratti ad attingere alla loro energia per poter formulare delle vere, reali, concrete \textbf{magie}!
Alcune entità crearono o richiamarono altre razze; vuoi per poter dominare gli umani, vuoi per poterli guidare, vuoi per aggiungere caos ed entropia al mondo.

Sono passati poco più di cento anni dalla seconda venuta eppure tanto è bastato perché La nostra Terra tornasse ad un medioevo di fantastiche origini.

Molti dei Patroni più oscuri hanno aperto portali verso regni dell'incubo se non demoniaci, altri hanno attinto dal folklore locale per divertirsi con la nostra sofferenza e morte. Come divinità i Patroni camminano sulla Terra con il solo scopo di avere più persone che le adorino, che seguano i loro insegnamenti, che abbiano i loro Tratti.

\medskip

\begin{enfasi}{Si può scoprire di più su una persona in un'ora di gioco che in un anno di conversazione. (Platone)}\end{enfasi}

\subsection{Introduzione}

Benvenuti in \textbf{OBSS}, un mondo dove l'ordinario si fonde con l'incredibile, dove la magia antica coesiste con la tecnologia avanzata e le leggende nascono dai gesti di individui comuni. In questo universo narrativo, i vostri personaggi non sono eroi predestinati, ma persone comuni, con i loro sogni, paure e ambizioni, catapultate in un vortice di eventi che cambieranno per sempre le loro vite.

Preparatevi a confrontarvi con sfide inaspettate, a stringere alleanze fragili e a lottare per la sopravvivenza in un mondo caotico e pericoloso.

Il Narratore è l’architetto di questo mondo: colui che plasma la realtà e tesse le fila della storia.
Spetta a lui presentarvi le sfide, descrivere i luoghi, dare vita ai personaggi non giocanti e interpretare le conseguenze delle vostre azioni.

In OBSS, la collaborazione tra giocatori e Narratore è fondamentale per creare un’esperienza di gioco coinvolgente e indimenticabile.
La vera regola che il Narratore deve ricordare è che qualsiasi regola sia usata finché fa divertire tutti è quella giusta!.

La sopravvivenza è la legge fondamentale di questo mondo. In OBSS, non ci sono garanzie di successo e ogni passo potrebbe essere l'ultimo. Ma è proprio in questa lotta per la sopravvivenza che si forgiano i veri eroi. Attraverso le vostre azioni, la vostra astuzia e il vostro coraggio potrete reclamare la Legge del Premio, guadagnando esperienza, ricchezze e la possibilità di influenzare il corso degli eventi.

Ogni personaggio è definito da sei Caratteristiche fondamentali che rappresentano i suoi attributi innati: Forza, Destrezza, Costituzione, Intelligenza, Saggezza e Carisma. Inoltre ogni personaggio possiede cinque Tratti, qualità uniche che lo caratterizzano a livello personale e che influenzano il suo stile di gioco e le sue interazioni con il mondo. I Tratti non sono semplici elementi di background, ma vere e proprie guide alle vostre azioni: plasmano le decisioni e determinano il legame con i Patroni, entità misteriose che offrono poteri speciali in cambio di fedeltà.

In questo manuale troverete le istruzioni per creare i vostri personaggi, per interagire con il mondo di gioco, per combattere, per usare la magia e per affrontare le sfide che incontrerete lungo il vostro cammino.

Ma prima di iniziare le vostre avventure dovrete partecipare alla Sessione Zero, un momento cruciale per gettare le basi della vostra esperienza. Durante la Sessione Zero, stabilirete le regole ed opzioni di gioco, definirete le aspettative del gruppo, creerete i vostri personaggi e darete vita alle loro storie.

In OBSS, la vostra fantasia è l'unico limite. Non abbiate paura di sperimentare, di mettere alla prova le vostre idee e di costruire personaggi che siano unici e indimenticabili. Siete liberi di plasmare il vostro destino e di lasciare il segno in questo mondo, un tiro di dado alla volta.

Le azioni sono misurate in base a delle Azioni e il successo si basa sui tiri di dado, la vostra competenza, Abilità e le vostre scelte tattiche.

Ricordate: le prove possono essere evitate con intelligenza e strategia. L'esplorazione, la capacità di risolvere enigmi e l'immaginazione sono componenti cruciali di questo gioco. Non cercate per forza la soluzione nella scheda, usate l'ingegno!

Ora che siete pronti, è tempo di abbracciare il vostro destino e di scrivere la vostra leggenda. Siate coraggiosi, siate creativi e siate pronti a tutto. Che la vostra avventura abbia inizio!

\begin{center}
Buona lettura e Buon Divertimento!
\end{center}

\begin{flushright}
Andres Zanzani
\end{flushright}

\end{multicols}

\vfill

\begin{enfasi}
D\&D ha nelle proprie origini tratti misogini e razzisti che con il tempo sono stati rimossi grazie alle tantissime persone di tutti i generi e tipi che ci hanno giocato.
OBSS vuole continuare nel solco di un gioco inclusivo e libero. Ogni gruppo è libero di approcciare argomenti controversi come meglio crede ma sempre nel rispetto di ogni giocatore e sensibilità. Non fate che OBSS sia motivo di litigio ma di unione e spirito fraterno, un gioco che unisca e mai divida. (Andres Zanzani)
\end{enfasi}

\pagebreak

\subsection{Termini Comuni}\label{Termini Comuni}

\begin{multicols}{2}

Ti elenco un pò di termini\index{Termini comuni} e concetti che troverai ripetuti più volte nel libro.

\medskip

\textbf{Arrotondamenti}: \index{Arrotondamenti}sempre per difetto se non esplicitato diversamente ma con un minimo di 1. Es. 7/2 = 3, 9/4=2, 1/2=1

\textbf{Abilità}: \index{Abilità}sono capacità particolari che il personaggio ha imparato ad usare. Spesso simili a capacità magiche, permettono azioni particolari, di sovvertire le regole e concedono dei bonus ai Tiri Salvezza che si cumulano tra loro. Si prendono ai passaggi di livello (vedi \hyperlink{abilita}{Abilità}, pag. \pageref{abilita})

\textbf{Azione}: \index{Azione} è ciò che si fa in un intervallo di tempo. Ogni cosa che viene fatta dal personaggio si misura in Azioni. Combattere, lanciare Incantesimi, scassinare, bere pozioni, lo spostarsi... in ogni round si possono fare 3 Azioni. Un'Azione dura circa 3 secondi.

\textbf{Bonus}: \index{Bonus}qualsiasi modificatore dovuto a fattori esterni, ambientali, magici, di circostanza o che decida il Narratore è un bonus o penalità da applicare al tiro di dado o difficoltà nella prova.

\textbf{Check/Prova}: \index{Check}\index{Prova}un check (o prova) è il tiro di 3d6 più il punteggio indicato dalla Caratteristica e Competenza coinvolta, potrebbero essere applicati modificatori dati da Abilità e circostanze. Se non si possiede la Competenza si tirano 2d6 + il modificatore della Caratteristica.

\textbf{Classe}: In OBSS non ci sono classi. Ogni personaggio è costruito in base a ciò che sa fare, non troverete la parola Classe nel manuale. Ogni personaggio è unico e definito dalle sue scelte.

\textbf{Colpo Critico}\index{Colpo Critico}: quando il Tiro per Colpire supera di almeno 8 la Difesa dell'avversario applichi il solo dado dell'arma in più al danno causato.

\textbf{Prova di Magia}\index{Prova di Magia}: la Prova di Magia può essere dovuta a particolari situazioni, ad esempio quando il personaggio è ferito o distratto, ma può essere richiesta anche dal giocatore.

La Prova di Magia permette al personaggio di spingersi oltre nel lancio dell'incantesimo e provare ad attingere e sfruttare più magia.

A seconda dei risultati potrebbe ottenere vantaggi o svantaggi.

\textbf{Lanciare Incantesimi sotto attacco, minaccia, distrazione..}:\index{Prova di Concentrazione}\index{Lanciare Incantesimi sotto attacco, minaccia, distrazione..}\index{Distratto} quando un incantatore vuole usare una Magia ma è disturbato, attaccato, ferito o comunque distratto durante il lancio di un incantesimo allora dovrà effettuare una Prova di Magia.

\textbf{Classe di Difficoltà (DC)}:\index{Classe di Difficoltà} \index{DC}indica quanto è difficile riuscire in una prova. Può essere usato per le competenze (nuotare..) come le conoscenze (veleni..). Negli incantesimi è la difficoltà a resistere agli incantesimi. Indica a che valore arrivare per superare e riuscire nella prova.

\textbf{Competenza} \index{Competenza}(skill)\index{Skill}: la competenza ci dice ciò che sappiamo ed il suo valore indica il grado di conoscenza della stessa. Possa essere lo studio di una lingua, l'arrampicarsi, il notare piccole cose.

\textbf{Competenza con le Armi (CA) (da mischia o distanza)} \index{Competenza con le Armi} è la tua capacità di saper colpire l'avversario con armi da mischia (spade, mazze, pugni..) o da tiro/distanza (pugnali da lancio, archi, balestre..)

\textbf{Competenza Magica (CM)}: \index{Competenza Magica}\index{CM}è la tua capacità di usare le magie, più è alto questo valore più le magie saranno efficaci, più ne avrai a disposizione, più ne potrai lanciare.

\begin{center}
\includegraphics[keepaspectratio,width=0.3\textwidth]{immagini/spiritomagia2.png}
\end{center}

\textbf{Difesa}: \index{Difesa}per Difesa si intende il valore totale ottenuto da 10 + Scudo + Armatura + Destrezza + vari ed eventuali bonus. Rappresenta la capacità di non farsi colpire e non essere ferito. Un nemico con alta Difesa potrà essere estremamente agile ed avere una \emph{pellaccia} estremamente resistente al ferimento.

\textbf{+1d6 oppure -1d6}: è un bonus o penalità ad una prova. Aggiungi o sottrai un tiro di dado a 6 alla prova. Il massimo della penalità porta i numero dei dadi lanciati a 0 ed il bonus massimo a +3d6.\index{Massimo valore del bonus}

\textbf{Distanza}:\index{Distanza} la distanza, per quando riguarda il combattimento è misurato in quadretti da 1 metro.

\textbf{Devoto}\index{Devoto}: un personaggio che si é legato ad un Patrono ed ha almeno 2 Tratti in comune.

\textbf{Seguace}: un personaggio che si è legato ad un Patrono con 1 Tratti in comune

\textbf{Esplosione del 6}:\index{Esplosione del 6} quando, esegui il Tiro per Colpire, Tiro Salvezza, Prova di Competenza, Prova di Magia, Iniziativa (leggi le specifiche nel capitolo dedicato) o comunque ogni volta che viene indicato che vale l'esplosione del 6 significa che per ogni dado tirato che ha fatto 6 va segnato e ritirato il dado. Il risultato del nuovo tiro va anche lui sommato e se si fa un 6 si continua a ritirare finché si continua a fare 6.

\textbf{Iniziativa}: \index{Iniziativa}è una prova di Destrezza oppure Intelligenza. Stabilisce l'ordine delle azioni in combattimento. Chi ha il punteggi più alto nella prova agisce per primo.

\textbf{Livello}:\index{Livello} il Livello indica la competenza e potere raggiunto dal personaggio. Può indicare quando è \emph{forte} il nemico.

\textbf{Livello dell'incantesimo}: indica la scala (da 1 a 9) della potenza magica dell'incantesimo.

\textbf{Incantatore, Mago:} \index{Incantatore}indica un qualsiasi usufruitore di magia a qualsiasi titolo.

\textbf{Mischia}: \index{Mischia}con mischia si intende il combattimento di contatto, corpo a corpo, spada a spada, ovvero quando il tuo personaggio combatte con un arma che abbia non abbia gittata (arco, balestre, fionde...) contro un avversario.
Si considera in mischia qualsiasi creatura che il personaggio possa raggiungere con la sua arma non da tiro. Una creatura di grandi dimensioni (o con un arma lunga) potrebbe essere in mischia con il personaggio ma non viceversa.

\textbf{Movimento}: \index{Movimento}il movimento rappresenta la capacità di spostarsi. Una Azione di Movimento rappresenta lo spostarsi del personaggio, più è alto il valore di Movimento più metri una creatura può muoversi.

\begin{wrapfigure}[24]{r}[.5\width+.5\columnsep]{7cm}%\itshape

\centering
\includegraphics[width=6.5cm]{immagini/merlin.png}

\emph{Merlin dictating his prophecies to his scribe. Robert de Boron's Merlin en prose (written ca 1200)}
\end{wrapfigure}

%\begin{center}
%\includegraphics[width=0.35\textwidth]{immagini/merlin.png}
%
%\emph{Merlin dictating his prophecies to his scribe. Robert de Boron's Merlin en prose (written ca 1200)}
%\end{center}

\textbf{Narratore}:\index{Narratore} è la persona che conduce l'avventura, stabilisce le regole e controlla gli elementi della storia. Il dovere di ogni Narratore è fare divertire, essere corretto ed usare il buon senso. Il Narratore ha l'ultima parola in ogni questione.

\textbf{Opzionale}:\index{Opzionale} in OBSS sono presenti diverse regole Opzionali per diversificare e personalizzare il gioco. Parlatene durante la Sessione Zero e decidete che stile dare al vostro OBSS.

\textbf{Prova di Caratteristica}\index{Prova di caratteristica}: è una prova di Competenza che usa come bonus il valore di una Caratteristica, quale Forza, Carisma..

\textbf{Patrono}:\index{Patrono} o divinità. Il Patrono è un essere superiore che può concedere poteri e garantire vantaggi.

\textbf{Penalità/Malus} \index{Penalità}: come il bonus le penalità o malus sono valori, numeri, che indicano le circostanze sfavorevoli, magie penalizzanti o quant'altro renda più difficile la prova. Purtroppo a differenza dei Bonus le penalità, se non specificato diversamente, si sommano sempre fra loro.

\textbf{PG, Personaggio}: \index{Personaggio}è la creatura che viene guidata, gestita, \emph{ruolata} dal giocatore.

\textbf{PNG}: \index{PNG}personaggio non giocante. Sono personaggi particolari, importanti o meno che il Narratore tiene per condurre l'avventura.

\textbf{Punti Esperienza/PX}: \index{Punti Esperienza} \index{PX} ogni qual volta si risolvano difficoltà, indovinelli, si affrontino mostri o si trovino dei tesori, si giochi bene il personaggio e ci si diverta si guadagna esperienza. Questi punti accumulati nel tempo stabiliscono il livello e quindi le capacità del personaggio.

\textbf{Punteggi caratteristica}: \index{Punteggi caratteristica} \index{Statistiche} abbreviati anche in caratteristica o statistiche. Ogni personaggio ha 6 Caratteristiche: Forza (FOR), Destrezza (DES), Intelligenza (INT), Saggezza (SAG) e Carisma (CAR). più è alto il punteggio maggiore è la valenza o capacità del personaggio in quello specifico ambito.

\textbf{Punti Fato}:\index{Punti Fato} \index{Fortuna del Principiante}o Fortuna del Principiante sono dei punti a disposizione che il giocatore può trasformare in d6 da aggiungere ai Tiri Salvezza o Tiri per Colpire o Tiri Competenze. Vengono chiamati Fortuna dei Principianti perché il loro numero diminuisce all'aumentare di livello del personaggio.

\textbf{Punti ferita (Punti Ferita)}:\index{Punti Ferita} \index{Punti Ferita}indicano l'energia vitale, la resistenza, la fortuna nel resistere alle ferite della creatura. Finché la creatura ha 1 punto ferita combatterà al suo meglio, senza problemi (ma potrebbe anche decidere di scappare piuttosto che morire!).

\begin{wrapfigure}[24]{l}[.5\width+.5\columnsep]{7cm}

	\centering
\end{wrapfigure}

Ad ogni passaggio di livello si guadagna un certo numero di Punti Ferita, stabilito dalle regole. Ogni ferita si sottrae da questa cumulo di energie e quando si raggiungono gli 0 (zero) Punti Ferita si sviene, incapaci di agire.

%\medskip
%\begin{center}
%\includegraphics[width=0.35\textwidth]{immagini/Sakramentarz_tyniecki_02.png}
%
%\emph{Sakramentarz Tyniecki: Majuskuła "V".}
%\end{center}
%\medskip

Se si viene ulteriormente feriti ed i Punti Ferita scendono fino 10 + il doppio del valore della Costituzione allora si muore.

\textbf{Riduzione del Danno (DR)}: \index{Riduzione del Danno} \index{DR} alcune creature hanno una resistenza innata al danno e ferite. Questa resistenza si denota come DR. La Riduzione si applica dopo la Resistenza ed i Tiri Salvezza.

\textbf{Resistenza al Danno (RD)}, \textbf{Resistenza}: \index{Resistenza al Danno}\index{RD}: una creatura potrebbe avere una resistenza ad una tipologia di danno. In questo caso si considera che dimezzi automaticamente il danno subito prima di applicare eventuali Tiri Salvezza.

\textbf{Round}:\index{Round} il combattimento o azioni sono divise in round. Un round rappresenta una unità temporale di circa 10 secondi. Durante il round ogni creatura ha la possibilità di agire in base alla sua iniziativa ed eseguire fino a 3 Azioni.

\textbf{Successo Critico/Fallimento Critico Magico}\index{Successo Critico Magico} \index{Fallimento Critico Magico}: nel caso in cui il giocatore passi la Prova di Magia con dei critici. Il Successo Critico Magico porta a spettacolose modifiche nell'incantesimo, viceversa potrebbero succedere brutte cose all'incantatore.

\textbf{Tiro Salvezza (TS)}:\index{Tiro Salvezza} \index{TS}quando una creatura è sottoposta ad un effetto particolare spesso viene concesso un Tiro Salvezza per mitigare o annullare gli effetti. Il Tiro Salvezza è un'azione che non occupa tempo o Azioni.

I Tiri Salvezza riguardano i riflessi e lo schivare (Riflessi), resistere a veleni/malattie o cambiamenti del corpo (Tempra) oppure per resistere ad attacchi mentali ed effetti che agiscano sull'arbitrio e volontà (Volontà).

\textbf{Successo Critico/Fallimento Critico nel Tiro Salvezza}\index{Sucesso Critico Magico nel Tiro Salvezza} \index{Fallimento Critico nel Tiro Salvezza}: a seconda dell'incantesimo in caso di Successo Critico nel Tiro Salvezza si dimezzano ulteriormente gli effetti mentre in caso di Fallimento Critico si subisce ancora di più il danno.

\begin{center}
\includegraphics[width=0.5\textwidth]{immagini/Jan_Steen2.png}

\emph{Jan Havicksz. Steen}
\end{center}

\textbf{Tiro per Colpire (TC)}:\index{Tiro per Colpire} \index{TC}è una prova di Attacco (Competenza Armi + Forza/Destrezza + Abilità + capacità dati da lista armi...) contro Difesa (armatura + scudo + Abilità + magia...). Il Tiro per Colpire può essere da mischia (ovvero per le creature prossime alla tua arma, a distanza di mischia) oppure da distanza (per archi, balestre, ma anche pugnali lanciati..).. Leggi bene il capitolo del combattimento.

\textbf{Tratto}: \index{Tratti}indica una componente del carattere. Ogni personaggio sceglie 5 Tratti per comporre e costruire la sua personalità.

\textbf{Turno}: \index{Turno}sono 10 minuti, ovvero 60 round

\textbf{Uno porta male}: \index{Uno porta male}se tiri un 1 con il dato togli 1 dal risultato totale. Non per questo un 6 tirato diventa un 5, l'esplosione del 6 rimane.. solo che togli 1 al risultato finale. Detta diversamente 1 vale 0.

\begin{enfasi}
Il gioco di D\&D non ha né vinti né vincitori, ha solo giocatori che amano esercitare la propria immaginazione. I giocatori ed il DM condividono la creazione di avventure in terre fantastiche dove abbondano gli eroi e la magia funziona davvero. In un certo senso, il gioco di D\&D non ha regole, solo suggerimenti di regole. Nessuna regola è inviolata, in particolare se una regola nuova o modificata incoraggerà la creatività e l'immaginazione. L'importante è godersi l'avventura. (Tom Moldvay, 03/12/1980. E tutto quanto detto vale anche per OBSS! NdA)
\end{enfasi}

\medskip

Nel Manuale troverete diverse tipologie di box, ognuno ha un significato preciso:

\medskip

\begin{enfasi}{Esempio di box contenente una citazione o frase motivazionale}\end{enfasi}

\begin{giocatore}[Informazioni per il Giocatore]Box contenente indicazioni e chiarimenti per il Giocatore.\end{giocatore}

\begin{narratore}[Informazioni per il Narratore]Box contenente indicazioni e suggerimenti per il Narratore.\end{narratore}

\end{multicols}

%\vfill

%\begin{center}
%\includegraphics[width=0.3\textwidth]{immagini/cavaliere2.png}
%\end{center}

\vfill

\begin{center}
\includegraphics[keepaspectratio,width=0.95\linewidth]{immagini/dice.png}

\medskip

\emph{Tipico set di dadi da gioco di ruolo}
\end{center}

\pagebreak

