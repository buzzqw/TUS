\section{Opzionale - Archetipi del Carattere}

\begin{multicols}{2}

Questa opzione presenta un sistema che integra gli archetipi junghiani \autocite{jung1971} con il framework esistente di Tratti del carattere e Patroni. Ispirandosi agli archetipi di Carl Jung e all'indicatore tipologico Myers-Briggs (MBTI) \autocite{myers1995}, vengono presentati 21 distinti pattern archetipici che possono essere utilizzati per la creazione del personaggio, lo sviluppo e la narrazione.

Ogni archetipo è presentato con un insieme di Tratti raccomandati che si allineano naturalmente con quell'energia archetipica. Insieme ai Tratti sono anche riportati quelli  generalmente incompatibili o contraddittori (e per questo stimolanti) alla natura fondamentale dell'archetipo. Inoltre vengono elencati quali Patroni condividono almeno due Tratti (e quindi permettono di essere Devoti) con ogni archetipo, suggerendo affinità spirituali naturali.

\begin{itemize}
\item Scegli un archetipo che ti attrae o si adatta alla tua idea del personaggio
\item Considera di adottare almeno 2-3 dei Tratti raccomandati per quell'archetipo
\item Evita i Tratti sconsigliati a meno che tu non stia specificamente mirando a creare conflitto interno
\item Guarda ai Patroni allineati per una guida su quali poteri spirituali potrebbero essere affini naturalmente con il tuo personaggio
\end{itemize}

Gli archetipi possono anche evolversi durante le avventure di un personaggio. Un personaggio potrebbe iniziare come un archetipo (L'Innocente) e trasformarsi in un altro (L'Eroe) attraverso le sue esperienze. Questa evoluzione può essere riflessa nel cambiamento graduale dei Tratti e delle affinità con i Patroni.

\subsection*{I 21 Archetipi}

\subsection*{L'Eroe}
Il personaggio coraggioso che supera gli ostacoli per raggiungere un obiettivo, spesso trasformando se stesso nel processo.

\noindent\begin{itemize}[leftmargin=*] \setlength{\itemsep}{0pt}
\item \textbf{Tratti Raccomandati:} Coraggioso, Testardo, Ambizioso, Leale
\item \textbf{Tratti Sconsigliati:} Codardo, Indeciso, Disonesto
\item \textbf{Tratti Contraddittori:} Cinico, Crudele, Dissoluto
\item \textbf{Patroni Allineati:} Gradh, Sumkjr, Nedraf, Ljust, Lynx, Orlaith
\end{itemize}

\subsection*{Il Mentore}
La guida saggia che fornisce conoscenza, intuizione e supporto agli altri, spesso all'Eroe.

\noindent\begin{itemize}[leftmargin=*] \setlength{\itemsep}{0pt}
\item \textbf{Tratti Raccomandati:} Paziente, Gentile, Prudente, Compassionevole
\item \textbf{Tratti Sconsigliati:} Impulsivo, Arrogante, Vendicativo
\item \textbf{Tratti Contraddittori:} Crudele, Disonesto, Avaro
\item \textbf{Patroni Allineati:} Ljust, Sumkjr, Thaft, Ledyal, Gaya
\end{itemize}

\subsection*{Il Guardiano della Soglia}
Il personaggio che mette alla prova i giocatori, presentando sfide che devono superare per procedere.

\noindent\begin{itemize}[leftmargin=*] \setlength{\itemsep}{0pt}
\item \textbf{Tratti Raccomandati:} Intransigente, Sospettoso, Prudente, Paziente
\item \textbf{Tratti Sconsigliati:} Impulsivo, Compassionevole, Altruista
\item \textbf{Tratti Contraddittori:} Codardo, Indeciso, Disonesto
\item \textbf{Patroni Allineati:} Atmos, Orlaith, Lynx, Krondal, Sixiser
\end{itemize}

\subsection*{L'Araldo}
Il personaggio che annuncia la chiamata all'avventura e segnala la necessità del cambiamento.

\noindent\begin{itemize}[leftmargin=*] \setlength{\itemsep}{0pt}
\item \textbf{Tratti Raccomandati:} Entusiasta, Estroverso, Curioso, Coraggioso
\item \textbf{Tratti Sconsigliati:} Prudente, Indeciso, Sospettoso
\item \textbf{Tratti Contraddittori:} Codardo, Cinico, Disonesto
\item \textbf{Patroni Allineati:} Nethergal, Sumkjr, Lynx, Nedraf
\end{itemize}

\subsection*{Il Mutaforma}
Il personaggio la cui lealtà e vera natura sono costantemente in discussione.

\noindent\begin{itemize}[leftmargin=*] \setlength{\itemsep}{0pt}
\item \textbf{Tratti Raccomandati:} Disonesto, Arrogante, Impulsivo, Ambizioso
\item \textbf{Tratti Sconsigliati:} Leale, Altruista, Gentile
\item \textbf{Tratti Contraddittori:} Intransigente, Prudente, Paziente
\item \textbf{Patroni Allineati:} Calicante, Orudjs, Tazher, Tàhil, Torbiorn
\end{itemize}

\subsection*{L'Ombra}
Il riflesso oscuro del personaggio, che rappresenta aspetti rifiutati o temuti del sé.

\noindent\begin{itemize}[leftmargin=*] \setlength{\itemsep}{0pt}
\item \textbf{Tratti Raccomandati:} Crudele, Vendicativo, Cinico, Arrogante
\item \textbf{Tratti Sconsigliati:} Compassionevole, Gentile, Altruista
\item \textbf{Tratti Contraddittori:} Leale, Paziente, Entusiasta
\item \textbf{Patroni Allineati:} Calicante, Cattalm, Tàhil, Shayalia, Torbiorn
\end{itemize}

\subsection*{L'Imbroglione}
Il personaggio malizioso che disturba lo status quo e porta trasformazione attraverso il caos.

\noindent\begin{itemize}[leftmargin=*] \setlength{\itemsep}{0pt}
\item \textbf{Tratti Raccomandati:} Impulsivo, Curioso, Disonesto, Entusiasta
\item \textbf{Tratti Sconsigliati:} Prudente, Paziente, Intransigente
\item \textbf{Tratti Contraddittori:} Leale, Altruista, Compassionevole
\item \textbf{Patroni Allineati:} Orudjs, Belevon, Nihar
\end{itemize}

\subsection*{L'Alleato}
Il personaggio fedele che supporta il gruppo nel suo viaggio.

\noindent\begin{itemize}[leftmargin=*] \setlength{\itemsep}{0pt}
\item \textbf{Tratti Raccomandati:} Leale, Coraggioso, Altruista, Gentile
\item \textbf{Tratti Sconsigliati:} Disonesto, Invidioso, Crudele
\item \textbf{Tratti Contraddittori:} Vendicativo, Arrogante, Avaro
\item \textbf{Patroni Allineati:} Ljust, Sumkjr, Efrem, Gradh, Thaft
\end{itemize}

\subsection*{L'Innocente}
Il personaggio puro e ingenuo che vede il mondo con meraviglia e ottimismo.

\noindent\begin{itemize}[leftmargin=*] \setlength{\itemsep}{0pt}
\item \textbf{Tratti Raccomandati:} Gentile, Entusiasta, Altruista, Curioso
\item \textbf{Tratti Sconsigliati:} Cinico, Sospettoso, Vendicativo
\item \textbf{Tratti Contraddittori:} Crudele, Disonesto, Dissoluto
\item \textbf{Patroni Allineati:} Ljust, Ledyal, Sumkjr, Gaya
\end{itemize}

\subsection*{Il Saggio}
Il personaggio custode della conoscenza che ha accumulato saggezza attraverso lo studio o l'esperienza.

\noindent\begin{itemize}[leftmargin=*] \setlength{\itemsep}{0pt}
\item \textbf{Tratti Raccomandati:} Prudente, Paziente, Curioso, Intransigente
\item \textbf{Tratti Sconsigliati:} Impulsivo, Disonesto, Dissoluto
\item \textbf{Tratti Contraddittori:} Indeciso, Codardo, Arrogante
\item \textbf{Patroni Allineati:} Atmos, Nethergal, Sixiser
\end{itemize}

\subsection*{Il Sovrano}
Il personaggio che cerca di stabilire ordine e controllo su suo dominio.

\noindent\begin{itemize}[leftmargin=*] \setlength{\itemsep}{0pt}
\item \textbf{Tratti Raccomandati:} Ambizioso, Intransigente, Arrogante, Testardo
\item \textbf{Tratti Sconsigliati:} Indeciso, Codardo, Impulsivo
\item \textbf{Tratti Contraddittori:} Altruista, Compassionevole, Gentile
\item \textbf{Patroni Allineati:} Calicante, Erondil, Cattalm, Krondal, Tàhil
\end{itemize}

\subsection*{Il Creatore}
Il personaggio costruttore, innovativo, che porta nuove cose all'esistenza.

\noindent\begin{itemize}[leftmargin=*] \setlength{\itemsep}{0pt}
\item \textbf{Tratti Raccomandati:} Curioso, Entusiasta, Ambizioso, Paziente
\item \textbf{Tratti Sconsigliati:} Indeciso, Codardo, Cinico
\item \textbf{Tratti Contraddittori:} Dissoluto, Intransigente, Vendicativo
\item \textbf{Patroni Allineati:} Erondil, Efrem, Gaya, Nethergal
\end{itemize}

\subsection*{Il Custode}
Il personaggio protettore, premuroso, che si prende cura e difende gli altri.

\noindent\begin{itemize}[leftmargin=*] \setlength{\itemsep}{0pt}
\item \textbf{Tratti Raccomandati:} Compassionevole, Altruista, Gentile, Paziente
\item \textbf{Tratti Sconsigliati:} Crudele, Avaro, Vendicativo
\item \textbf{Tratti Contraddittori:} Arrogante, Disonesto, Dissoluto
\item \textbf{Patroni Allineati:} Ljust, Ledyal, Atherim, Belevon, Sumkjr
\end{itemize}

\subsection*{Il Mago}
Il personaggio che sfrutta conoscenze uniche per alterare la realtà.

\noindent\begin{itemize}[leftmargin=*] \setlength{\itemsep}{0pt}
\item \textbf{Tratti Raccomandati:} Curioso, Ambizioso, Arrogante, Intransigente
\item \textbf{Tratti Sconsigliati:} Codardo, Indeciso, Impulsivo
\item \textbf{Tratti Contraddittori:} Altruista, Compassionevole, Leale
\item \textbf{Patroni Allineati:} Erondil, Orudjs, Nethergal, Nihar, Krondal
\end{itemize}

\subsection*{Il Fuorilegge}
Il personaggio ribelle che sfida le norme stabilite e combatte contro i vincoli.

\noindent\begin{itemize}[leftmargin=*] \setlength{\itemsep}{0pt}
\item \textbf{Tratti Raccomandati:} Coraggioso, Impulsivo, Arrogante, Vendicativo
\item \textbf{Tratti Sconsigliati:} Prudente, Paziente, Leale
\item \textbf{Tratti Contraddittori:} Altruista, Compassionevole, Gentile
\item \textbf{Patroni Allineati:} Lynx, Tàhil, Gradh, Tazher, Calicante
\end{itemize}

\subsection*{L'Amante}
Il personaggio cercatore appassionato di connessione, intimità e piacere sensuale.

\noindent\begin{itemize}[leftmargin=*] \setlength{\itemsep}{0pt}
\item \textbf{Tratti Raccomandati:} Compassionevole, Entusiasta, Impulsivo, Dissoluto
\item \textbf{Tratti Sconsigliati:} Sospettoso, Cinico, Arrogante
\item \textbf{Tratti Contraddittori:} Prudente, Intransigente, Avaro
\item \textbf{Patroni Allineati:} Shayalia, Ledyal, Orudjs
\end{itemize}

\subsection*{Il Giullare}
Il personaggio intrattenitore giocoso che porta gioia e leggerezza in situazioni difficili.

\noindent\begin{itemize}[leftmargin=*] \setlength{\itemsep}{0pt}
\item \textbf{Tratti Raccomandati:} Entusiasta, Estroverso, Impulsivo, Curioso
\item \textbf{Tratti Sconsigliati:} Prudente, Sospettoso, Intransigente
\item \textbf{Tratti Contraddittori:} Crudele, Vendicativo, Arrogante
\item \textbf{Patroni Allineati:} Nihar, Belevon, Nethergal, Orudjs
\end{itemize}

\subsection*{L'Uomo Comune}
Il personaggio ordinario che cerca appartenenza e connessione.

\noindent\begin{itemize}[leftmargin=*] \setlength{\itemsep}{0pt}
\item \textbf{Tratti Raccomandati:} Leale, Gentile, Prudente, Indeciso
\item \textbf{Tratti Sconsigliati:} Arrogante, Ambizioso, Dissoluto
\item \textbf{Tratti Contraddittori:} Crudele, Vendicativo, Disonesto
\item \textbf{Patroni Allineati:} Efrem, Thaft, Atherim
\end{itemize}

\subsection*{L'Esploratore}
Il personaggio avventuriero che cerca nuove esperienze e scoperte.

\noindent\begin{itemize}[leftmargin=*] \setlength{\itemsep}{0pt}
\item \textbf{Tratti Raccomandati:} Curioso, Coraggioso, Impulsivo, Entusiasta
\item \textbf{Tratti Sconsigliati:} Prudente, Indeciso, Sospettoso
\item \textbf{Tratti Contraddittori:} Codardo, Cinico, Avaro
\item \textbf{Patroni Allineati:} Lynx, Nihar, Nethergal, Sumkjr
\end{itemize}

\subsection*{Il Martire}
Il personaggio che si sacrifica e dà tutto per una causa o per gli altri.

\noindent\begin{itemize}[leftmargin=*] \setlength{\itemsep}{0pt}
\item \textbf{Tratti Raccomandati:} Altruista, Coraggioso, Intransigente, Testardo
\item \textbf{Tratti Sconsigliati:} Arrogante, Avaro, Ambizioso
\item \textbf{Tratti Contraddittori:} Cinico, Crudele, Disonesto
\item \textbf{Patroni Allineati:} Ljust, Sumkjr, Atherim, Lynx
\end{itemize}

\subsection*{Il Tiranno}
Il personaggio oppressore, controllore, che governa attraverso la paura e la dominazione.

\noindent\begin{itemize}[leftmargin=*] \setlength{\itemsep}{0pt}
\item \textbf{Tratti Raccomandati:} Crudele, Arrogante, Vendicativo, Avaro
\item \textbf{Tratti Sconsigliati:} Compassionevole, Altruista, Gentile
\item \textbf{Tratti Contraddittori:} Indeciso, Codardo, Leale
\item \textbf{Patroni Allineati:} Calicante, Tàhil, Cattalm, Torbiorn, Rezh
\end{itemize}

\subsection*{L'Eremita}
Il personaggio cercatore solitario che si ritira dalla società per trovare la verità interiore.

\noindent\begin{itemize}[leftmargin=*] \setlength{\itemsep}{0pt}
\item \textbf{Tratti Raccomandati:} Prudente, Paziente, Intransigente, Sospettoso
\item \textbf{Tratti Sconsigliati:} Estroverso, Impulsivo, Entusiasta
\item \textbf{Tratti Contraddittori:} Dissoluto, Arrogante, Avaro
\item \textbf{Patroni Allineati:} Atmos, Sixiser, Efrem
\end{itemize}

\subsection*{Cicli Archetipici e Sviluppo del Personaggio}

I personaggi raramente incarnano un singolo archetipo durante l'intero loro viaggio. Piuttosto, possono evolversi attraverso una serie di archetipi mentre crescono e si sviluppano. Di seguito sono riportate alcune progressioni archetipiche comuni che possono servire come modelli per lo sviluppo del personaggio:

\noindent\begin{itemize}[leftmargin=*] \setlength{\itemsep}{0pt}
\item \textbf{Il Viaggio dell'Eroe:} Innocente > Esploratore > Eroe > Saggio > Sovrano
\item \textbf{La Caduta e la Redenzione:} Innocente > Fuorilegge > Ombra > Martire > Eroe
\item \textbf{L'Arco della Corruzione:} Alleato > Mutaforma > Ombra > Tiranno
\item \textbf{Il Sentiero della Saggezza:} Esploratore > Guardiano della Soglia > Eremita > Saggio > Mentore
\item \textbf{L'Evoluzione della Leadership:} Alleato > Eroe > Mentore > Sovrano
\item \textbf{Il Risveglio Spirituale:} Uomo Comune > Guardiano della Soglia > Eremita > Mago > Creatore
\end{itemize}

Ogni transizione tra archetipi rappresenta un momento significativo di sviluppo del personaggio, spesso accompagnato dall'acquisizione o perdita di Tratti specifici e potenzialmente da un cambiamento nell'affinità con i Patroni. Narratori e giocatori possono utilizzare questi cicli archetipici per pianificare archi narrativi significativi del personaggio.

\subsection*{Affinità con i Patroni e Risonanze Archetipiche}

Alcuni Patroni incarnano naturalmente o risuonano con archetipi specifici più fortemente di altri. Queste connessioni possono informare sia lo sviluppo del personaggio che la narrazione:

\noindent\begin{tabularx}{\columnwidth}{lX}
\toprule
\textbf{Patrono} & \textbf{Risonanze Archetipiche} \\
\midrule
Ljust & L'Eroe, Il Custode, Il Martire, Il Mentore \\
\hline
Calicante & L'Ombra, Il Tiranno, Il Mutaforma, Il Sovrano \\
\hline
Atmos & Il Saggio, L'Eremita, Il Guardiano della Soglia \\
\hline
Lynx & L'Esploratore, Il Fuorilegge, L'Eroe \\
\hline
Gradh & L'Eroe, L'Alleato, Il Sovrano, Il Fuorilegge \\
\hline
Atherim & Il Custode, Il Guardiano, Il Martire \\
\hline
Belevon & L'Imbroglione, Il Giullare, Il Mutaforma \\
\hline
Cattalm & L'Ombra, Il Tiranno, Il Distruttore \\
\hline
Efrem & L'Uomo Comune, Il Creatore, L'Eremita \\
\hline
Erondil & Il Creatore, Il Mago, Il Sovrano \\
\hline
Gaya & Il Creatore, L'Innocente, Il Custode \\
\hline
Krondal & Il Sovrano, Il Guardiano della Soglia, Il Mago \\
\hline
Ledyal & Il Custode, L'Innocente, L'Amante \\
\hline
Laydel & L'Ombra, Il Mutaforma, Il Fuorilegge \\
\hline
Nethergal & L'Araldo, L'Esploratore, Il Mago \\
\hline
Nedraf & L'Eroe, L'Esploratore, L'Alleato \\
\hline
Nihar & Il Giullare, L'Esploratore, Il Giullare \\
\hline
Orudjs & Il Mutaforma, Il Giullare, L'Amante \\
\hline
Orlaith & Il Guardiano della Soglia, Il Sovrano, L'Eroe \\
\hline
Rezh & Il Tiranno, L'Ombra, Il Mutaforma \\
\hline
Shayalia & L'Ombra, L'Amante, Il Mutaforma \\
\hline
Sixiser & L'Eremita, Il Saggio, L'Ombra \\
\hline
Sumkjr & L'Eroe, L'Alleato, Il Mentore, Il Martire \\
\hline
Tàhil & L'Ombra, Il Tiranno, Il Fuorilegge \\
\hline
Tazher & L'Ombra, Il Fuorilegge, Il Mutaforma \\
\hline
Thaft & L'Uomo Comune, L'Alleato, Il Custode \\
\hline
Torbiorn & Il Tiranno, L'Ombra, Il Sovrano \\
\bottomrule
\end{tabularx}

\subsection*{Utilizzare gli Archetipi nel Gioco}

\subsection*{Per i Giocatori}
\noindent\begin{itemize}[leftmargin=*] \setlength{\itemsep}{0pt}
\item Usa gli archetipi come punto di partenza per la creazione del personaggio, selezionando Tratti che si allineano con il tuo archetipo scelto
\item Considera il potenziale viaggio archetipico del tuo personaggio, come potrebbe evolversi nel tempo?
\item Guarda ai Patroni allineati del tuo archetipo quando consideri affiliazioni spirituali
\item Usa archetipi contrastanti all'interno del tuo gruppo per creare dinamiche inter-personaggio interessanti
\end{itemize}

\subsection*{Per il Narratore}
\noindent\begin{itemize}[leftmargin=*] \setlength{\itemsep}{0pt}
\item Popola il tuo mondo con PNG che incarnano pattern archetipici chiari
\item Progetta sfide che testano specificamente i personaggi contro le loro debolezze archetipiche
\item Crea antagonisti che servono come riflessi oscuri degli archetipi degli eroi
\item Usa interventi dei Patroni per evidenziare o sfidare la natura archetipica di un personaggio
\item Progetta archi di campagna che seguono cicli archetipici classici
\end{itemize}

\begin{changemargin}{0.3cm}{0.3cm}\begin{narratore}\index{Archetipi Caratteriali}
Questo sistema archetipico fornisce un altro livello di profondità alla creazione e sviluppo del personaggio, collegando la teoria psicologica con i sistemi esistenti di Tratti e Patroni del gioco. Comprendendo questi pattern archetipici, giocatori e Narratore possono creare personaggi e storie più coerenti e psicologicamente consistenti.

\textbf{Piuttosto che limitare la creatività, questi archetipi servono come framework e linee guida utili che possono essere abbracciati, sovvertiti o trasformati durante il viaggio di un personaggio}. L'integrazione di Tratti e affinità con i Patroni con pattern archetipici crea un ricco arazzo di possibilità per lo sviluppo del personaggio e la narrazione.
\end{narratore}\end{changemargin}

\end{multicols}

\pagebreak

