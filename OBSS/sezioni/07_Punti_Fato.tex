\section{Punti Fato}\index{Punti Fato}\index{Fortuna del Principiante}\label{puntifato}

\begin{enfasi}{Se il destino è contro di noi, peggio per lui. (motto del 1º Reggimento Carabinieri Paracadutisti "Tuscania")}\end{enfasi}
\medskip

%In un mondo non facile la Fortuna del Principiante aiuta chi non ha esperienza.
Ogni personaggio ha un numero di Punti Fato pari a (20 - Livello)/5, arrotondato all'intero più vicino, con un minimo di 1. I Punti Fato si azzerano e conteggiano per sessione di gioco.
Si recupera un Punto Fato ogni volta che si tirano almeno tre 1 in una prova.\index{Recuperare Punti Fato}

Non costa Azioni usare un Punto Fato e può essere usato per:

\begin{description}[labelwidth=1cm]
	\item[\FatePoint] o più, aggiungere 1d6 ad un Tiro Salvezza, Tiro per Colpire, Prova di Competenza. Da dichiarasi prima del tiro dei dadi. Il dado aggiunto può esplodere secondo le Golden Rules.
	\item[\FatePoint\FatePoint] ritirare 1d6 nelle prove
	\item[\FatePoint] negare un Tiro Critico d'arma subito
	\item[\FatePoint\FatePoint] ritirare completamente una prova
	\item[\FatePoint] trasformare prova fallita criticamente in fallita semplicemente
	\item[\FatePoint] far ritirare un Tiro Salvezza ad un obiettivo
	\item[\FatePoints{3}] tornare a 0 Punti Ferita (tutti i punti disponibili)
	\item[\FatePoint] o più, diminuire di 3 i danni subiti
\end{description}

\subsection*{Opzionale - Punti Caos}\index{Opzionale - Punti Caos}

Un sistema per aggiungere tensione è gestire un insieme di Punti Fato condivisi tra personaggi ed avversari al posto di quelli del singolo giocatore.

Si mette al centro del tavolo un contenitore, una piccola ciotola, con dentro un numero di d6 pari al numero dei personaggi. Ogni giocatore è libero di prendere un dado alla volta ed usarlo come se fossero Punti Fato.

I dadi usati dai giocatori vengono poi spostati in un altro contenitore che il Narratore, sempre massimo uno alla volta per avversario, userà a \emph{suo beneficio}. Una volta che il Narratore ha usato il dado lo rimette nel contenitore dei giocatori.

\end{multicols}

\pagebreak
