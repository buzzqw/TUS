\section{Regole su Oggetti Magici}\index{Regole su Oggetti Magici}\hypertarget{identificareom}{}\label{regoleoggettimagici}\hypertarget{regoleoggettimagici}{}

\begin{multicols}{2}

Queste sono le indicazioni su l'utilizzo degli oggetti magici.

\label{oggetti-magici}
\begin{itemize}[leftmargin=*] \setlength{\itemsep}{0pt}
\item
Un personaggio può \textbf{tenere attivi numerosi (fino a 10) oggetti magici} su di sé.

\item Per determinare il bonus alla \textbf{Difesa} non si possono sommare più di 2 oggetti (es. 1 anello magico ed un braccialetto). Armatura e Scudo non si considerano in questo conteggio.
\item
Se hai più oggetti magici che concedono bonus allo stesso \textbf{Tiro Salvezza} si applicano solo i due con il bonus maggiore.
\item
Se hai più oggetti magici che concedono bonus alla stessa \textbf{Caratteristica} allora si applica solo il bonus maggiore.
\item
Un personaggio \textbf{non può indossare più di due anelli magici} altrimenti entrano in risonanza abbassando i Punti Ferita massimi di 1d6 (non riducibile o curabile) a round per ogni anello oltre il secondo.
\item
Per \textbf{riconoscere un oggetto magico} vedi \hyperlink{rinoscereoggettomagico}{Riconoscere un oggetto magico} (pag. \pageref{rinoscereoggettomagico}) e \hyperlink{oggettimaledettiid}{Oggetti Maledetti} (pag. \pageref{oggettimaledettiid}).
\item
Un \textbf{oggetto magico che manifesta incantesimi} non esegue alcuna Prova di Magia. Il \textbf{Tiro Salvezza} che impone, se non specificato, è pari a 12 + livello*2 dell'incantesimo che manifesta.\index{Tiro Salvezza per incantesimi da oggetto}\label{tirosalvezzaincoggetto}
\item
Per \textbf{Attivare delle capacità magiche} di un oggetto se non indicato diversamente  costa 2 Azioni.
\item
Un oggetto magico che fornisce un \textbf{bonus (o penalità) statico} applica il suo valore anche se l'oggetto non è stato identificato, sarà il Narratore ad applicare silenziosamente questo bonus alla Difesa, Tiro per Colpire, Tiri Salvezza... informando il giocatore che percepisce come l'oggetto interagisca con la situazione.
\item
Un oggetto magico che ha degli usi giornalieri si ricarica all'alba del giorno successivo all'uso.
\item
Bacchette, Bastoni, Pergamene (non Isy), Verghe sono usabili solo da personaggi che abbiano il punteggio di CM pari al livello di incantesimo più alto formulabile dall'oggetto.
\item
Se spendi l'ultima carica della bacchetta, tira 1d6 se ottieni 1 la bacchetta si riduce in polvere ed è distrutta
\item
Un oggetto magico impiega 10 minuti di tempo una volta indossato prima di permettere di usare i suoi modificatori, bonus o talenti.

\end{itemize}

\subsubsection{Armi}

\textbf{Abilità Speciali}: un'arma con una capacità speciale deve avere almeno bonus di +1. Le armi non possono avere la stessa capacità speciale più di una volta.

Il bonus magico di un \textbf{arma può essere compreso} tirando di due critici in un Tiro per Colpire oppure dedicando 1 ora di allenamento, eventuali talenti o abilità magiche rimangono celate.

\subsubsection{Armature e Scudi}

\textbf{Abilità Speciali}: un'armatura o scudo con una capacità speciale deve avere almeno bonus di +1. Armature e Scudi non possono avere la stessa capacità speciale più di una volta. Vedi anche sezione \hyperlink{armaturaescudimagici}{Armatura, Scudi e Magia} (pag. \pageref{armaturaescudimagici}).

Un'\emph{armatura} +1 abbassa di 1 la penalità di Competenza e di 1 metro la quella al movimento

Una armatura o \emph{scudo} +2 riduce di 2 la penalità data dall'armatura alla Prova di Magia.

Una armatura +3 ulteriormente toglie 1 alla penalità di Competenza, riduce di 1m la penalità al Movimento e riduce di ulteriori due la penalità data dall'armatura alla Prova di Magia.

\textbf{Il costo di Armi e Armature}: di dimensioni superiori alle Medie è almeno il doppio (o quadruplo in base alla taglia). Armature piccole o Armi piccole pur richiedendo meno materiale costano la medesima cifra delle armi e armature medie.

\subsection{Taglia e Oggetti Magici}\label{tagliaoggettimagici}

\label{taglia-e-oggetti-magici}

Quando un capo di vestiario o un gioiello magici vengono scoperti, il più delle volte la taglia non è un problema: molti vestiti magici sono di facile utilizzo per tutti oppure si adattano magicamente a chi li indossa. Di regola, la taglia non dovrebbe impedire ai personaggi di varia tipologia fisica l'utilizzo di un oggetto magico.

Ci possono essere delle rare eccezioni, specie con gli oggetti realizzati per una razza specifica.

Le armi e le armature rinvenute casualmente hanno una probabilità del 30\% di essere Piccole (01--30), del 60\% di essere Medie (31--90), e del 10\% di essere di un'altra taglia. Le armature non si adattano alla taglia del possessore se non indicato diversamente.

\subsection{Oggetti Magici sul Corpo}\index{Oggetti Magici sul Corpo}

\label{oggetti-magici-sul-corpo}

Molti oggetti magici devono essere indossati da un personaggio che voglia usarli o beneficiare delle loro capacità. Per una creatura di forma umanoide è possibile indossare fino a 10 oggetti magici alla volta. Ognuno di questi oggetti deve essere indossato sopra una parte specifica del corpo denominata \textbf{slot}.

Un corpo di forma umanoide può indossare l'equipaggiamento magico in queste parti del corpo:

\textbf{Dita}: anelli (due al massimo).

\textbf{Vesti}: corazze, armature, tuniche e vesti

\textbf{Cintura}: cinture.

\textbf{Collo}: amuleti, collane, medaglioni, scarabei, spille, talismani e sciarpe

\textbf{Mani}: armi, guanti e guanti d'arme.

\textbf{Occhi}: occhi, occhiali e lenti.

\textbf{Piedi}: scarpe, stivali e pantofole.

\textbf{Polso}: braccialetti e bracciali.

\textbf{Braccia}: scudi.

\textbf{Spalle}: cappe e mantelli.

\textbf{Testa}: cappelli, diademi, elmi, maschere, corone, fasce e filatteri.

\textbf{Torace}: camicie, giubbe, maglie e manti.

\subsection{Tiri Salvezza Contro i Poteri degli Oggetti Magici}\index{Tiri Salvezza}

\label{tiri-salvezza-contro-i-poteri-degli-oggetti-magici}

Gli oggetti magici normalmente riproducono incantesimi o altri effetti magici. Per un Tiro Salvezza contro la magia o un effetto magico generato da un oggetto magico, la DC è 12 + livello dell'incantesimo manifestato x2 se non specificato diversamente.

\subsection{Danneggiare gli Oggetti Magici}\index{Danneggiare gli Oggetti Magici}

\label{danneggiare-gli-oggetti-magici}

Un oggetto magico non deve compiere un Tiro Salvezza a meno che non sia incustodito, sia il bersaglio specifico dell'effetto, o il suo possessore ottenga un 0 (tre volte 1) naturale al suo Tiro Salvezza.

Gli oggetti magici hanno sempre diritto a un Tiro Salvezza contro qualcosa che potrebbe danneggiarli, anche quando un oggetto normale dello stesso tipo non avrebbe alcuna possibilità di effettuare un Tiro Salvezza. Gli oggetti magici usano sempre lo stesso bonus ai Tiri Salvezza, indipendentemente dal tipo (Tempra, Riflessi o Volontà). Il bonus ai Tiri Salvezza di un oggetto magico è pari a 2 + 2x livello dell'incantesimo più potente che ospitano (oppure un +6 per ogni +1 che hanno). Le sole eccezioni a questa regola sono gli oggetti magici intelligenti, che effettuano i Tiri Salvezza su Volontà basandosi sul loro punteggio di Saggezza.

\subsection{Riparare gli Oggetti Magici}\index{Riparare gli Oggetti Magici}
\label{riparare-gli-oggetti-magici}

Per riparare un oggetto magico occorrono materiali e tempo, pari alla metà del tempo e del costo per crearlo.

\subsection{Cariche, Dosi e Usi Multipli}\index{Cariche}\index{Dosi}\index{Usi Multipli}

\label{cariche-dosi-e-usi-multipli}

Molti oggetti, ed in modo particolare le bacchette e i bastoni, hanno un potere limitato al numero di cariche che contengono. Normalmente gli oggetti dotati di cariche non superano mai il massimo di 20 cariche (10 per i bastoni). Se oggetti simili vengono trovati come parte casuale di un tesoro, si tira un 1d10+10 per determinare il numero delle cariche rimaste. Se un oggetto ha un numero massimo di cariche diverso da 20, si tira casualmente per stabilire quante cariche sono rimaste.

I prezzi indicati si riferiscono agli oggetti al massimo delle loro cariche (quando un oggetto viene creato, ha sempre il massimo delle cariche). Il valore di un oggetto dipende dal numero di cariche residue, in caso di oggetti che possono avere un uso anche con poche o senza cariche, il valore rimane più alto.

\subsection{Ricaricare gli oggetti magici}\index{Ricaricare gli oggetti magici}\label{Ricaricare gli oggetti magici}

Oggetti magici \emph{a carica} come i Bacchette e Bastoni hanno un numero di usi, cariche, ovvero ogni volta che si attinge al suo potere si usa una carica.

Per ricaricare una bacchetta od un bastone un incantatore deve infondere lo stesso incantesimo che vuole ricaricare spendendo il doppio dei Punti Magia del costo dell'incantesimo e superare una Prova di Magia.

\end{multicols}

\subsection{Acquisire Oggetti Magici}\index{Acquisire Oggetti Magici}\index[Tabelle]{Tabella Acquisire Oggetti Magici}

\label{acquisire-oggetti-magici}

\bigskip

\noindent\begin{tabularx}{\linewidth}{lXXXX}
	\toprule
\rowcolor{gray!20}\textbf{Dimensioni Comunità} & \textbf{Valore Base} & \textbf{Comune} & \textbf{Non Comune} & \textbf{Raro}\\
\toprule
Insediamento& 20 mo & 1d2 oggetti && \\
\rowcolor{gray!20}Borgo & 150 mo& 1d4 oggetti && \\
Villaggio & 300 mo& 1d6 oggetti & 1d2 oggetti& \\
\rowcolor{gray!20}Piccolo paese & 700 mo & 1d4 oggetti & 1d2 oggetti& \\
Grande paese& 1500 mo & 1d6 oggetti & 1d4 oggetti& 1d2 oggetti\\
\rowcolor{gray!20}Piccola città & 2500 mo & 2d4 oggetti & 1d6 oggetti& 1d4 oggetti\\
Grande città& 6000 mo & 3d4 oggetti& 2d4 oggetti& 1d6 oggetti\\
\rowcolor{gray!20}Metropoli & 12000 mo& {*} & 3d4 oggetti& 2d4 oggetti
\end{tabularx}

{*} In una metropoli si trovano quasi tutti gli oggetti magici minori.

\begin{multicols}{2}

\bigskip

Gli oggetti magici sono preziosi e la maggior parte delle grandi città ha almeno uno o due fornitori di oggetti magici, dal semplice venditore di pozioni ad un fabbro specializzato nel forgiare spade magiche. Ovviamente non ogni oggetto in questo manuale è disponibile in ogni città.

Le linee guida seguenti aiutano i Narratori a determinare quali oggetti sono disponibili in una specifica comunità. Esse presuppongono una campagna con un livello medio di magia. Alcune città potrebbero deviare di molto da questa linea di base a discrezione del Narratore. Il Narratore dovrebbe tenere una lista degli oggetti disponibili da ogni mercante e dovrebbe rimpinguare occasionalmente le scorte con nuove acquisizioni.

Il numero ed i tipi di oggetti magici disponibili in una comunità dipendono dalla sua dimensione. Ogni comunità ha un valore base legato ad essa (vedi Tabella: Oggetti Magici Disponibili).

c'è una probabilità del 75\% che qualsiasi oggetto di quel valore o inferiore si possa trovare in vendita facilmente in quella comunità. Inoltre, la comunità ha un certo numero di altri oggetti in vendita. Questi oggetti sono determinati a caso e sono ripartiti in categorie (minore, medio o maggiore).

Dopo aver determinato il numero di oggetti disponibili in ogni categoria, consultate il capitolo Generazione casuale degli Oggetti Magici per determinare il tipo di ogni oggetto (pozione, pergamena, anello, arma,ecc.) prima di passare alle tabelle specifiche per stabilire l'oggetto esatto. Ritirate ogni volta che gli oggetti non si adeguano al valore base della comunità.

Se l'uso della magia nella campagna in cui si gioca è raro, occorre dimezzare il valore base e il numero di oggetti in ogni comunità. Nelle campagne con magia estremamente rara o senza magia potrebbero non esserci affatto oggetti magici in vendita I Narratori che conducono questo tipo di campagne dovrebbe prevedere delle modifiche alle sfide affrontate dai personaggi data la mancanza di oggetti magici.

Le campagne con abbondanti oggetti magici potrebbero avere comunità con il doppio del valore base stabilito e degli oggetti magici casuali disponibili. In alternativa, si potrebbe stabilire che tutte le comunità siano di una categoria di dimensione maggiore allo scopo di stabilire gli oggetti magici disponibili. In una campagna con magia molto comune, tutti gli oggetti magici si possono acquistare in una metropoli.

Oggetti e attrezzi non magici sono in genere disponibili in una comunità di qualsiasi dimensione a meno che l'oggetto non sia molto costoso, come un'armatura completa, o fatto di un materiale insolito, come una spada lunga in adamantio. Questi oggetti dovrebbero seguire la linea guida del valore base per determinare la loro disponibilità, a discrezione del Narratore.

\subsection{Gli artefatti del vecchio mondo}

Nel corso delle avventure i giocatori troveranno degli oggetti del passato dimenticato. Potranno essere chincaglieria senza utilità se non come reperto storico di un era che non tornerà più. Potranno spesso essere apparati elettronici che senza una fonte di energia non funzioneranno mai..

Potranno essere altrimenti strumenti creati negli ultimi giorni della prima era quando appresi i rudimenti della magia qualche genio e Devoto riuscì a sfruttare la tecnologia con la magia, riuscì ad attivare \emph{magicamente} un apparato tecnologico.

Non stupiamoci allora di trovare oggetti che possano essere ricaricati, anche se per poco tempo, con \emph{Stretta Folgorante}, o veicoli funzionanti se colpiti da un \emph{Fulmine}.

Questi oggetti saranno rari e preziosi, quasi come una dose di antibiotico ancora attiva. Sbizzarritevi nel creare oggetti \emph{tecnomagici}, prendete ispirazione dal mondo moderno e dalla fantascienza per creare strabilianti oggetti utili alla vostra avventura.

\end{multicols}

\vfill

\begin{center}
\includegraphics[keepaspectratio,width=0.8\textwidth]{immagini/Alchemical_laboratory_Wellcome_M0005193.png}

\emph{Alchemical laboratory}
\end{center}

\pagebreak

