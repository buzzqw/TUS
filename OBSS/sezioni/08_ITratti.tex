\section{I Tratti}\index{Tratti}\hypertarget{tratti}{}\label{tratti}

\begin{enfasi}{Chi dunque sa fare il bene e non lo compie, commette peccato. (Giacomo il Giusto 4.17, Lettera di Giacomo)
\smallskip

E' un diritto naturale saziarsi l'anima con la vendetta. (Attila)
\smallskip

Est Sularus Oth Mithas. ("Il mio onore è la mia vita", Giuramento dei Cavalieri di Solamnia)
}\end{enfasi}

\begin{multicols}{2}

\index{Tratti}
In OBSS non c'è una netta distinzione tra bene e male, legge e caos, tra ciò che è giusto e ciò che è sbagliato.

In OBSS esistono i Tratti, aspetti e sfumature caratteriali che \textbf{contribuiscono} al background del personaggio, aiutano il giocatore a ruolare meglio e gli possono fornire quelle linee guida per interpretare in maniera più corretta il personaggio che si è voluto creare.

Un Tratto è un dettaglio che aiuta meglio a inquadrare il personaggio, ne delinea i caratteri principali concedendogli sfumature diverse.

\textbf{Ogni giocatore sceglie 5 Tratti per il proprio personaggio alla creazione del personaggio.} Questi che suggeriranno il personaggio nell'agire e nelle scelte.

\begin{giocatore}[Scegliere i Tratti] %box giocatore
I Tratti non sono il personaggio, non lo bloccano ne lo fissano eterno nel tempo. Un personaggio è sempre in costante evoluzione e così il suo carattere, morale, comportamento e desideri. Non essere rigido ma usa i Tratti per darti dei suggerimenti da cui farti ispirare.
\end{giocatore}

I Tratti non hanno accezione positiva o negativa, servono solo ad inquadrare il personaggio e capire quale Patrono è più interessato al personaggio. Non vogliono definire se sei buono o cattivo, ognuno ha la propria morale indipendentemente dai Tratti posseduti.

\textbf{Al primo livello scegli un Tratto maggiormente caratteristico per il personaggio, questo avrà valore 1, gli altri 4 Tratti avranno valore 0.}

Col passare del tempo e delle avventure i Tratti aumenteranno valore o potranno essere sostituiti, in concerto tra Narratore e giocatore in base a come giocato, da altri Tratti. \textbf{più è alto un valore di Tratto più questo è presente e permeante nelle scelte del personaggio}.

Durante le avventure il Narratore a seguito di particolari scene e recitazione potrà fare aumentare di un punto, o di una frazione di punto, un Tratto del personaggio.

Ad esempio a seguito di una particolare situazione e climax di avventura il Narratore potrebbe concedere a tutti o qualcuno il Tratto Coraggio o dare un +1 a Coraggio a chi ha già questo Tratto. Per i Tratti non presi si considera il valore base in punti di -1, ovvero il primo punto serve per prendere il Tratto ed i successivi per enfatizzarli.

Mentre è \emph{relativamente} facile acquisire nuovi Tratti è complicato cambiare quelli già presenti. Parlane con il Narratore, saprà preparare situazioni ed avventure che ti aiuteranno a comprendere come evolvere il personaggio ed eventualmente ad evolvere i Tratti scelti.

Nella scheda troverai dei \textbf{check} da mettere vicino ai Tratti, questi vengono segnati a seguito di azioni idonee ad accrescere il valore del Tratto; raggiunti i 10 check il Tratto aumenterà di 1 punto e si ricomincerà a segnare una nuova decina.

Sarà il Narratore durante l'avventura a dirti quando segnare, o cancellare, dei punti parziali. \textbf{In linea di massima si presume che un personaggio acquisisca almeno un punto Tratto a livello.}

Ogni azione particolarmente importante dove il personaggio abbia seguito un Tratto porta il personaggio ad avvicinarsi al \textbf{Patrono} competente per quel Tratto.

All'aumentare del valore della somma dei Tratti comuni con il Patrono il personaggio potrà acquisire dei poteri, indipendentemente sia un credente (Seguace o Devoto) o meno di quel Patrono.

\noindent\begin{itemize}[leftmargin=*] \setlength{\itemsep}{0pt}
\item A \textbf{'5'} punti si può incominciare a sentire la presenza di un Patrono

\item A \textbf{'10'} punti si sente la vicinanza di un Patrono

\item A \textbf{'15'} punti si è legati ad Patrono

\item A \textbf{'20'} punti si è un campione del Patrono
\end{itemize}

Non è necessario credere in un Patrono per sentirne la vicinanza, esserne legati o campione, semplicemente è la propria natura (i propri Tratti) che è affine al Patrono, che lo si voglia o meno. I poteri si prendono solo dal Patrono che ha somma tratti più alta rispetto agli altri.

Dato che lo scopo di un Patrono è fare che i propri Tratti siano dominanti sugli altri, avere persone di alto livello e potere che siano così affini a lui tornerà utile nel giudizio dei 100 anni. Usate a vostro vantaggio i Tratti ed il legame che il Patrono instaurerà con voi.

Per individuare il Patrono più affine, quello che vi darà i poteri, verificate il vostro Tratto a maggior valore sulla \hyperlink{tabellacollegamentopatronotratto}{Tabella Collegamento Patrono - Tratto} (pag. \pageref{tabellacollegamentopatronotratto}) ed individuate il Patrono che quel Tratto maggiormente caratterizzante, in caso il Tratto fosse condiviso tra più Patroni verificate gli altri Tratti ed in base alla somiglianza scegliete il Patrono.
Verificate poi in \hyperlink{cosmologia}{Cosmologia} (pag. \pageref{patroni}) i poteri concessi dal Patrono. Questo controllo è opportuno farlo ad ogni aumento di valore di Tratto.

Si è Devoti con almeno 2 Tratti e Seguaci con almeno 1 Tratto in comune con il Patrono. Non si può essere contemporaneamente Seguace o Devoti di più Patroni.

Il Narratore è libero di inserire nuovi Tratti a suo piacere o richiesti dai giocatori, si suggerisce di attribuire questi nuovi Tratti anche ai Patroni.

\medskip

\textbf{Lista dei Tratti}\index[Tabelle]{Tabella dei Tratti}

Ogni Tratto è brevemente descritto nel suo significato generico. Il personaggio è libero di interpretare il Tratto come più lo sente proprio.

\medskip

\noindent\begin{itemize}[leftmargin=*] \setlength{\itemsep}{0pt}
	\item \textbf{Altruista}: Persona che mette gli altri al primo posto, anche sacrificando i propri bisogni.
	\item \textbf{Ambizioso}: Pensa solo ai propri interessi e bisogni, senza considerare quelli degli altri
	\item \textbf{Arrogante}: Ha un'opinione esagerata di sé stesso e tende a sminuire gli altri.
	\item \textbf{Avaro}: Eccessivamente attaccato ai propri beni materiali e riluttante a condividere.
	\item \textbf{Cinico}: Tende a vedere il peggio nelle persone e nelle situazioni, spesso con un atteggiamento sprezzante.
	\item \textbf{Codardo}: Che manca di coraggio e tende a evitare situazioni di pericolo.
	\item \textbf{Compassionevole}: Mostra empatia e comprensione verso le sofferenze degli altri.
	\item \textbf{Coraggioso}: Affronta le paure e le sfide con determinazione.
	\item \textbf{Crudele}: Senza pietà e compassione, provoca sofferenza intenzionalmente.
	\item \textbf{Curioso}: Ha un forte desiderio di conoscere e imparare cose nuove.
	\item \textbf{Disonesto}: Non dice la verità e inganna gli altri per il proprio vantaggio.
	\item \textbf{Dissoluto}: Vive in modo sregolato e senza considerare le conseguenze morali delle proprie azioni.
	\item \textbf{Entusiasta}: Mostra grande energia e passione per ciò che fa.
	\item \textbf{Estroverso}: Socievole e a proprio agio in situazioni sociali.
	\item \textbf{Gentile}: Tratta gli altri con rispetto e considerazione.
	\item \textbf{Impulsivo}: Tende ad agire e reagire senza pensare troppo alle conseguenze.
	\item \textbf{Indeciso}: Non riesce rapidamente a prendere decisione soffermandosi troppo nel ponderare le scelte.
	\item \textbf{Intransigente}: Non è disposto a scendere a compromessi o a considerare punti di vista diversi.
	\item \textbf{Invidioso}: Prova risentimento verso chi possiede qualcosa che lui desidera.
	\item \textbf{Leale}: Fedele e affidabile nei confronti degli amici e delle persone care.
	\item \textbf{Paziente}: Capace di aspettare senza irritarsi o perdere la calma.
	\item \textbf{Prudente}: Pondera attentamente le situazioni difficili o pericolose.
	\item \textbf{Sospettoso}: Sei convito che tutti abbiano interesse a danneggiarti.
	\item \textbf{Testardo}: Determinato e persistente nel raggiungere i propri obiettivi, nonostante le difficoltà.
	\item \textbf{Vanitoso}: Sei certo delle tue eccezionali qualità, capacità ed aspetto.
	\item \textbf{Vendicativo}: Cerca di punire chi gli ha fatto un torto, spesso in modo sproporzionato.

\end{itemize}

\end{multicols}

%valutare le motivazioni, una tabella delle motivazioni

\smallskip

Se il personaggio è completamente difforme ai suoi Tratti non acquisirà punti esperienza.

\vfill

\begin{center}
\includegraphics[height=0.3\linewidth]{immagini/troll.png}
\end{center}

\medskip

\begin{enfasi}{Se un viaggiatore non riporta qualcosa da condividere, non è un \emph{Eroe} ma un impostore, un egoista privo di saggezza. (Il viaggio dell'Eroe, Christopher Vogler)}\end{enfasi}

\pagebreak

