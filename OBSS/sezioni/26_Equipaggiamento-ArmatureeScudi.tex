\section{Equipaggiamento - Armature e Scudi} \index{Armature}\index{Scudi}\hypertarget{equipaggiamento.armature.scudi}{}\label{equipaggiamentoarmature}

\label{equipaggiamento---armature-e-scudi}

\begin{changemargin}{0.3cm}{0.3cm}\begin{enfasi}{
Armatura (s.f.). Abito che si indossa se il proprio sarto è un fabbro. (Ambrose Bierce)

\medskip

Armatura Fantozzi: banderuola 4 venti in funzione di pennacchio, pauroso elmo vichingo con visibilità azzerata, sospensorio in bronzo sottratto alla statua di Pipino il Breve e, ai piedi, ferroni da stiro a carbonella di piombo fuso. Peso complessivo armatura Fantozzi: 4 quintali, 32 chili e 7 etti e mezzo. (Superfantozzi, Film)} \end{enfasi}\end{changemargin}\medskip

Le armature aiutano ad essere non colpiti (alzano la Difesa) e penalizzano la Prova di Magia e le prove di competenza di Base.

La Penalità Competenze è la penalità che si applica alle prove di competenza di Base influenzate dal peso ed Ingombro dell'armatura. Armature diverse, specifiche o magiche hanno punteggio diversi, questa tabella serve come linea guida per il Narratore.\index{Penalità armatura}

\subsubsection{Tabella Armature}\index[Tabelle]{Tabella Armature}

\label{tabella-armature}
\noindent\begin{tabular}{llllllll}
\textbf{Armatura} & \textbf{Costo} & \textbf{Difesa} & \textbf{Pen. Comp.} & \textbf{Tipo} & \textbf{Mov.} & \textbf{Prova Magia}&\textbf{Ingombro}\\
\toprule
Imbottita & 5 mo & 1 & 0 & L & 0 & NO&2\\
Cuoio & 10 mo & 2 & 0 & L & 0 & SI&2\\
Cuoio rinforzato& 25 mo& 3 & 0& L & 0 & SI&2\\
Giaco di Maglia & 15 mo & 4 & -1 & M & 0 &+2&4\\
Scaglie& 50 mo & 5& -1& M & 0 &+2&4\\
Anelli & 150 mo & 6& -1& M & 0 &+2&4\\
Pettorale& 200 mo & 6& -2& M & 0 &+2&4\\
Bande & 250 mo & 7& -2& P & 0 &+1&8\\
Mezza armatura& 1200 mo& 8& -2& P & 1 &+1,2&8\\
da Campo& 1350 mo& 9& -3& P & 2 &+1,2&8\\
Completa& 1500 mo& 10 & -4& P & 3 &+1,1&8
\end{tabular}

\begin{multicols}{2}

\medskip

\textbf{Costo}: è per un armatura di taglia media.

\textbf{Difesa}: è il bonus data alla Difesa

\textbf{Penalità Comp.}: è la penalità dato alle prove di Competenza di Base dato dal peso ed Ingombro dell'armatura.

\textbf{Tipo}: indica se l'armatura è \textbf{L}eggera, \textbf{M}edia oppure \textbf{P}esante.

\textbf{Ingombro}: indica l'ingombro quando l'armatura viene trasportata e non indossata.

\begin{changemargin}{0.3cm}{0.3cm}\begin{narratore}
Quando conteggiate l'Ingombro dato dall'armatura e scudo \textbf{indossato} dovete dividerlo per due.

L'Ingombro di armatura e scudi è da intendersi quando è \emph{caricata nello zaino}, ovvero trasportata ma non indossata.
\end{narratore}\end{changemargin}

\textbf{Mov. (movimento)}: è la riduzione in metri di movimento da applicare per Azione di Movimento.

\textbf{Prova Magia}: indica se la prova è da fare (SI) o meno (NO). I numeri indicati sono da aggiungere alla Prova di Magia.

\textbf{Costo}: il costo di un'armatura o scudo +1 è di 2250mo, +2 10000mo. Non è praticamente possibile acquistare armature o scudi od armi con incantamenti superiore a +2, devono essere \emph{trovate}.

\textbf{Controllate i requisiti} per indossare un \hyperlink{indossarearmature}{Armatura o Scudo} a pag. \pageref{indossarearmature}.

\medskip

\begin{center}
	\includegraphics[width=0.75\linewidth]{immagini/donnacavalierecavallo.png}

	\textit{Rara immagine di un cavallo, ora estinti}
\end{center}

\subsubsection{Armature, Scudi e Magia}\index{Armature e Magia}\index{Scudi e Magia}\hypertarget{armatureemagie}{}\label{armatureemagie}

Tutte le Armature, ad esclusione dell'armatura Imbottita forzano chi lancia incantesimi ad superare una Prova di Magia non considerando eventuali successi critici.\index{Armature e Prova di Magia} Sono anche segnati dei numeri (+1,+2) questi sono i valori che si considerano tirati nella Prova di Magia in aggiunta a quelli ottenuti dai dadi.

Es. Tups indossa una armatura Pettorale e lancia un incantesimo. E' obbligato dal portare l'armatura ad effettuare la Prova di Magia. Tira 3d6 +4 dadi (perché ha 9 punti in CM), ignora 2 dadi (perché ha preso 4 volte Adepto della Magia) ma aggiunge un dado che è come avesse già fatto 2 (per l'armatura Pettorale).

Nella prova esce 4,5,2 / 3,4,\st{1},\st{1}. Toglie i due 1 per Adepto della Magia ma aggiunge un 2 per l'armatura. Nel complesso non ha ottenuto fallimenti critici, grazie ai dadi tolti da Adepto della Magia, ne può ottenere successi critici. Riesce quindi a formulare l'incantesimo (ha ottenuto un 2 e nessun altro 2 o 1).

Se Tups avesse indossato una Mezza armatura la prova sarebbe stata 4,5,2 / 3,4,\st{1},\st{1} / 1,2 (dall'armatura). Anche togliendo i due 2 rimane un 1 e due 2, abbastanza per fare fallire la Prova di Magia e quindi fallire il lancio dell'incantesimo.

Anche se è il giocatore a richiedere una Prova di Magia, si eseguirà la prova con i valori dovuti ad armatura e/o scudi. \textbf{Il portare l'armatura o scudo negherà qualsiasi successo magico critico ottenuto ma non impedirà i fallimenti magici critici}.

\subsubsection{Descrizione delle Armature}

\textbf{Armature Leggere}

Fatte di materiali leggeri e flessibili, le armature leggere favoriscono gli avventurieri agili dato che offrono protezione senza sacrificare la mobilità.

\emph{Imbottita}. Le armature imbottite consistono di strati di tessuto e imbottitura cuciti insieme.

\emph{Cuoio}. Il corpetto e le protezioni delle spalle di questa armatura sono fatte di cuoio indurito dopo essere stato bollito nell'olio. Il resto dell'armatura è composto di
materiali più morbidi e flessibili.

\emph{Cuoio Rinforzato}. Fatta di cuoio duro ma flessibile, l'armatura di cuoio rinforzato è arricchita da rivetti o spuntoni.

\textbf{Armature Medie}

Le armature medie offrono più protezione di quelle leggere, ma limitano i movimenti.

\emph{Giaco di Maglia}. Composto di anelli metallici intrecciati tra di loro, un giaco di maglia viene indossato sopra strati di abiti o cuoio. Questo tipo di armatura offre una protezione modesta alla parte superiore del corpo, mentre il rumore degli anelli che strusciano fra di loro viene attutito dagli altri strati.

\emph{Scaglie}. Quest'armatura consiste in una cotta e gambali (a volte anche di una gonna separata) di cuoio coperti da pezzi di metallo sovrapposti, in maniera simile alle scaglie di un pesce. L'armatura è completa di guanti.

\emph{Anelli}. Quest'armatura è un'armatura di cuoio con dei pesanti anelli cuciti sopra. Gli anelli servono a rinforzare l'armatura contro i colpi di spada e d'ascia. L'armatura è completa di guanti.

\emph{Pettorale}. Questa armatura consiste di un corpetto di metallo indossato su uno strato di cuoio. Sebbene lasci braccia e gambe relativamente scoperte, l'armatura fornisce una buona protezione agli organi vitali del personaggio, senza procurargli grande ingombro.

\textbf{Armature Pesanti}

\emph{Bande}. Questa armatura e fatta di strisce di metallo cucite ad un robusto schienale di cuoio e maglia di ferro. Le dimensioni delle piastre metalliche, interconnesse alle bande di metallo e gli strati di armatura sottostanti la rendono una delle più protettive tra le armature.

\emph{Mezza Armatura}. La mezza armatura consiste in piastre di metallo sagomate che coprono gran parte del corpo del personaggio. Non comprende protezioni per le gambe oltre a dei semplici schinieri legati con lacci di cuoio.

\emph{da Campo}. Molto simile all'armatura completa ma più leggera in costruzione sacrificando un poco di protezione a favore di una maggiore flessibilità e mobilità.

\emph{Completa}. Quest'armatura consiste di piastre di metallo sagomate a incastro che coprono l'intero corpo. Un'armatura completa comprende guanti, stivali di cuoio pesanti, un elmo con visiera, e uno spesso strato di imbottitura sotto l'armatura. Fibbie e lacci distribuiscono il peso dell'armatura su tutto il corpo.

\subsubsection{Regole base per l'utilizzo dell'armatura}

\textbf{Usare un'Armatura senza l'adeguata competenza} impedisce di usare il bonus di Destrezza e diminuisce il bonus alla Difesa fornito di 1.

\textbf{Usare uno Scudo senza l'adeguata competenza} peggiora il Tiro per Colpire di 1 e diminuisce di 1 il Bonus Difesa concesso.

\textbf{Dormire in Armatura}: se si dorme in un'armatura media o pesante, il giorno seguente si è automaticamente \hyperlink{affaticato}{Affaticati}.

Dormire in un'armatura leggera non provoca Affaticamento.

La \textbf{capacità di movimento} del personaggio rimarrà la medesima fino all'armatura a bande poi calerà progressivamente. Il valore indicato nella colonna Mov. sono i metri in meno che il personaggio fa per Azione di Movimento.

Ad esempio un umano in armatura completa ha movimento 6 metri, un nano 3 metri.

\textbf{Peso}: il peso indicato si riferisce alla versione per personaggi di taglia Media. Le armature adattate per personaggi di taglia Piccola pesano la metà, mentre per quelli di taglia Grande pesano il doppio.

\textbf{Armature Perfette}\index{Armature Perfette}

Un'armatura perfetta è un armatura creata da un abilissimo fabbro che pur non essendo magica, grazie al suo perfetto bilanciamento, ha un +1 alla Difesa. Un fabbro per creare un armatura perfetta deve superare con un successo critico la DC impostata per la creazione dell'armatura. Un Armatura perfetta costa il doppio di un armatura normale.

\subsubsection{Armature e Scudi magici}\index{Armature magiche}\index{Scudi magici}

Un armatura magica o scudo magico non solo protegge meglio ma è anche più leggera e affine alla magia.

Una armatura +1 abbassa di 1 la penalità di Competenza e di 1 metro la quella al movimento.
Una armatura o scudo +2 inoltre toglie il dado alla Prova di Magia con il valore più alto. Una armatura +3 ulteriormente toglie 1 alla penalità di Competenza, riduce di 1m la penalità Movimento e rimuove 1 dado dalla Prova di Magia.


%\vfill
%%
\begin{center}
\includegraphics[width=1\linewidth]{immagini/buckler.png}

\emph{Buckler, fronte e retro}
\end{center}

\subsubsection{Gli Scudi}

Gli \textbf{Scudi} \index{Scudi}permettono di aumentare la propria Difesa, più lo scudo è imponente e pesante più protegge, più aumentano le penalità alle prove di competenza magica e meno rende facile combattere (penalità Tiro per Colpire).

Gli Scudi possono essere di tipo Leggero, Medio, Pesante.

\textbf{Bonus Difesa}: è il bonus che si applica alla Difesa quando lo scudo è indossato.

\textbf{Penalità TC}: è la penalità al Tiro per Colpire che si ha quando lo scudo è indossato e non si ha Forza almeno 3.

\textbf{Tipo}: indica la taglia dello scudo. \textbf{L}eggero, \textbf{M}edio, \textbf{P}esante.

Uno \textbf{Scudo} Leggero ha \textbf{Ingombro} 1, uno Scudo Medio ha Ingombro 2, uno Scudo Pesante ha Ingombro 4.\index{Imgombro per Scudi}

La penalità alla \textbf{Prova di Magia} si somma con quella eventualmente dovuta dall'armatura e si applica quando lo scudo è indossato.\index{Penalità Magia Scudo ed Armatura}

Uno scudo può essere usato come \hyperlink{armaimprovvisata}{arma improvvisata}. Uno scudo piccolo fa 1d4 di danno (B/T), uno scudo medio fa 1d6 di danno (B/T), uno scudo pesante fa 1d8 di danno (B/T).

Usare lo scudo come arma improvvisata non fa applicare il suo bonus alla Difesa se non si usa una Reazione per reimpostarlo alla Difesa dopo aver attaccato.

Imbracciare uno scudo occupa la mano ed il braccio.

\end{multicols}


\subsubsection{Tabella Scudi}\index[Tabelle]{Tabella Scudi}

\label{tabella-scudi}

\noindent\begin{tabular}{lcccccc}
\textbf{Scudi} & \textbf{Costo} & \textbf{Bonus Difesa} & \textbf{penalità TC} & \textbf{Prova magia} & \textbf{Tipo} & \textbf{Ingombro}\\
\toprule
Scudo leggero di legno& 3 mo&1& 0& SI& L & 1\\
Scudo leggero di metallo & 9 mo&1& 0& SI& L& 1\\
Scudo medio legno &5 mo &2& 0& +2& M& 2\\
Scudo medio metallo&12 mo&2& 0& +2& M& 2\\
Scudo pesante di legno & 9mo&3 & 1& +1,2& P& 4\\
Scudo pesante di metallo & 20 mo&3& 1& +1,2& P& 4\\
\end{tabular}

\begin{multicols}{2}

\subsubsection{Indossare e Togliere Armature}\index{Indossare e Togliere Armature}

Indossare e togliere armature è una operazione che richiede tempo ed attenzione, farlo in fretta non aiuta ed anzi tende a peggiorare la protezione fornita.

\end{multicols}

\textbf{Tabella: Tempi per indossare e togliere l'armatura}\index[Tabelle]{Tabella Tempi per indossare e togliere l'armatura}

\medskip

\noindent\begin{tabular}{llll}
\textbf{Tipo di Armatura}& \textbf{Indossare} & \textbf{Indossare in fretta} & \textbf{Togliere}\\
\toprule
Scudo& 1 Azione & - & 1 Azione\\
Imbottita, Cuoio, Cuoio rinforzata& 1 minuto& 3 round& - \\
Giaco di Maglia& 1 minuto& 5 round& 5 round\\
Scaglie, Anelli, Pettorale, Bande & 4 minuti & 1 minuto{*}& 1 minuto\\
Mezza armatura, da Campo, Completa& 4 minuti{*}{*}& 4 minuti{*}& 1d4+1 minuti
\end{tabular}

\bigskip

\begin{multicols}{2}

{*} Se qualcuno aiuta, il tempo si dimezza. Un singolo personaggio che non sta facendo altro può aiutare uno o due personaggi adiacenti a lui. Due personaggi non possono aiutarsi l'un l'altro a indossare un'armatura contemporaneamente.

{*}{*} Bisogna essere aiutati per indossare questa armatura. Senza aiuto è possibile indossarla solo in fretta.

\textbf{Indossare un'armatura in fretta} implica una penalità di -1 alla Difesa fornita dall'Armatura ed una penalità aggiuntiva di +1 alle prove di Competenza di Base.

\end{multicols}



\pagebreak


