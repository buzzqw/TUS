\section{Mostruario di OBSS}\index{Mostruario}

\begin{changemargin}{0.3cm}{0.3cm}\begin{enfasi}{Chi lotta con i mostri deve guardarsi di non diventare, così facendo, un mostro. E se tu scruterai a lungo in un abisso, anche l'abisso scruterà dentro di te. (Friedrich Nietzsche)

\medskip

I mostri possono essere sconfitti soltanto dai loro simili. (Claymore)

\medskip

La tragedia dei mostri è di essere troppo grandi e potenti per essere accettati dal genere umano. (Ishiro Honda)

\medskip

Per aspera ad astra! ("attraverso le asperità sino alle stelle")

}\end{enfasi}\end{changemargin}\medskip

\begin{multicols}{2}

Benvenuti in un universo ricco di avversari, spesso cattivi, altre volte violenti, pure subdoli, anche intelligenti, forse meschini e quasi sempre giganteschi.. e quant'altro tu vorrai. I mostri sono il caposaldo di qualsiasi gioco di ruolo fantasy.

Vengono qui spiegati e presentati dei mostri, non certo tutti ne tanto meno esaustivi, usateli per popolare di incubi le avventure dei vostri compagni.

\medskip

\begin{center}

\includegraphics[width=0.9\linewidth]{immagini/sangiorgioedrago.png}

\emph{San Giorgio e il drago (1460 circa) di Paolo Uccello. National Gallery di Londra}
\end{center}

\subsection{Introduzione}

Un avventura non è solo un insieme di avversari ma di situazioni, di luoghi, di sorprese, insomma di tutto ciò che può affascinare, coinvolgere stupire, impegnare i personaggi. Ma anche i mostri servono. Picchiare ha un aspetto catartico, liberatorio.

Inserite nell'avventura mostri difficili e letali dove serve ma ogni tanto, raramente, fate sentire i personaggi potenti, fategli affrontare mostri che in pochissimi round possono risolvere. Descrivete il combattimento enfatizzando i colpi, i critici, il dolore ed il sangue dei mostri. Fate capire quanto possano essere potenti i personaggi.

Altre volte fate che i mostri incutano timore perché' sono grossi, affamati, magici e cattivi, è necessario che i giocatori abbiano paura per i loro personaggi, che non diano mai per scontato la vittoria.

La forza dell'avversario è nella sicurezza nel descrivere la situazione, in poche battute, il fissare negli occhi i giocatori. Coinvolgete i giocatori ed una volta che avrete la loro attenzione anche i personaggi saranno più attenti. Cercate di mettere mostri coerenti all'ambiente, all'avventura, alla situazione. Non tirate a caso su tabelle, uno scontro ben organizzato da molta più soddisfazione che mostri a caso che \emph{spawnano}.

Non riducete tutto a un MMORG dove l'obiettivo è solo uccidere tutto e tutti, ci possono essere sempre tante scelte se ti impegni un pò.

\begin{changemargin}{0.3cm}{0.3cm}\begin{tcolorbox}[title = Affrontare i mostri]
{
Lascia che questo vecchio ti dia un paio di consigli giovane avventuriero!

- Non tutti i nemici si sconfiggono con la spada, molte volte serve anche una mazza!

- A volte le armi e la forza bruta non bastano. Se non hai compagni che possono lanciare incantesimi assicurati di avere sempre la possibilità di appiccare un fuoco.

- Scappa. E' sempre una opzione valida se hai modo e vedi che la situazione non promette niente di buono.

- Organizzati! non entrare nel dungeon a testa bassa senza mai fermarti tranne quando sei morto! Riposati, esplora, controlla l'ambiente e quando sei sicuro e stai meglio prosegui! anche i tuoi nemici si organizzano e si riposano intanto, stai attento!

- A volte si può anche parlare con i nemici, anche loro non vogliono morire sempre.

- Se devi uccidere fallo con cattiveria e velocità. Non perdere tempo e ottimizza i colpi, risparmia le energie e preparati immediatamente ad un altro scontro.

}\end{tcolorbox}\end{changemargin}

\subsection{Modificare le Creature}

Nonostante la variopinta collezione di incontri presente in questo manuale, potresti comunque trovarti in imbarazzo quando si tratta di trovare la creatura perfetta per una tua avventura. Sentiti libro di modificare le creature esistenti e trasformarle in qualcosa che ti sia più utile, magari prendendo in prestito uno o due caratteristiche da un mostro diverso.

Tieni a mente che modificare un avversario potrebbe cambiarne il grado di sfida.

\subsection{Taglia e Dimensioni}

Un mostro può essere di taglia Minuscola, Piccola, Media, Grande, Enorme o Mastodontica e Colossale. La tabella Categorie di Taglia mostra la grandezza media di una creatura e quanto spazio occupi sulla griglia.

Se non indicata la portata di una creatura dipende dalla taglia e dall'arma usata (pensate ad un gigantesco spadone brandito da un titano..)

\end{multicols}

\textbf{Tabella: Categorie di Taglia, Quadretti occupati e Portata}\index[Tabelle]{Tabella Categorie di Taglia, Quadretti occupati e Portata}\index{Portata per creature}\index{Quadretti per creature}\index{Taglia e quadretti}\index{Creature per quadretto}\label{tagliaedimensioni}\hypertarget{tagliaedimensioni}{}

\medskip

\begin{tabularx}{0.95\textwidth}{lllll}
\toprule
\textbf{Taglia}& \textbf{Dimensione} & \textbf{Esempio}&\textbf{Quadretti}&\textbf{Portata}\\
Minuscola & 25 x 25 cm&Gatto, spiritello& 1/4&0m\\
Piccola & 0,5 x 0,5 m & Goblin, cane, Gnomo&1/2&1m\\
Media & 1 x 1 m & Orco, Umano, Elfo, Nano, Nibali &1&1m\\
Grande & 2 x 2 m& Ogre&2x2&1m\\
Enorme & 3 x 3 m & Gigante, Ent&3x3&2m\\
Mastodontico & 4 x 4 m&Kraken, Drago&4x4&2m\\
Colossale & 12 x 12 m&Drago anziano, Tarrasque&6x6&6m
\end{tabularx}

\medskip

I più avvezzi avranno notato che le dimensioni delle creature sono inferiori alle solite, questo perché le miniature in commercio sono fatte per scala 1 quadretto=1.5 metri, mentre in OBSS 1 quadretto=1 metro.

\begin{multicols}{2}

\subsection{Tipo}

Il tipo di un mostro si riferisce alla sua natura basilare. Certi incantesimi, oggetti magici, Abilità e altri effetti del gioco interagiscono in modi speciali con le creature di un tipo specifico. Ad esempio, una \emph{freccia ammazza draghi} infligge danni extra non solo ai draghi ma anche a tutte le altre creature del tipo drago, come i draghi tartaruga e le viverne.

Il gioco comprende i seguenti tipi di mostri:

\smallskip\textbf{Aberrazioni}, creature totalmente aliene. Molte di esse possiedono innate abilità magiche che attingono alla mente aliena della creatura anziché dalle forze mistiche del mondo. Esempi classici di aberrazioni sono aboleti, divora cervelli ed i fustigatori.

\smallskip\textbf{Bestie}, creature non umanoidi che sono una componente naturale di un mondo fantasy. Alcune possiedono poteri magici, ma la maggior parte è priva di Intelligenza e non ha alcuna forma di società o linguaggio. Esempi classici di bestie sono tutte le specie di animali comuni, i dinosauri e le versioni giganti degli animali.

\smallskip\textbf{Celestiali}, creature native dei Piani Superiori. Molti di loro sono servitori delle divinità, impiegati come messaggeri o agenti nel mondo dei mortali e per i piani.

\medskip

I celestiali sono di natura buona, esempi classici di celestiali sono angeli, couatl e pegasi.

\smallskip\textbf{Costrutti}, sono creati e non partoriti. Alcuni sono programmati dai loro creatori per seguire una semplice serie di istruzioni, mentre altri sono senzienti e capaci di pensare per proprio conto. I golem sono i costrutti più rappresentativi.

\smallskip\textbf{Draghi}, sono grandi creature rettili di antica origine ed enorme potere. I veri draghi, compresi i buoni draghi di Ljust e i malvagi draghi di Tàhil, sono molto intelligenti e possiedono doti magiche innate. In questa categoria si collocano anche creature lontanamente imparentate con i veri draghi, ma meno potenti, meno intelligenti e meno magiche, come le viverne e gli pseudodraghi.

\smallskip\textbf{Elementali}, sono creature native dei piani elementali. Alcune creature di questo tipo sono poco più che masse animate del rispettivo elemento, e includono le creature chiamate semplicemente elementali. Altre creature possiedono forme biologiche infuse di energia elementale. Le razze dei geni, compresi djinn ed efreet, formano le civiltà più importanti dei piani elementali. Altre creature elementali sono gli azer, i persecutori invisibili e le bizzarrie d'acqua.

\smallskip\textbf{Fatati}, sono creature magiche strettamente legate alle forze della natura. Vivono in radure nascoste e foreste nebbiose. Esempi di fatati sono driadi, pixie, fate e satiri e La Topi.

\smallskip\textbf{Giganti}, troneggiano sugli umani e i loro simili. Sono di forma umana, sebbene alcuni abbiano più teste (ettin) o deformità. Le sei varianti dei veri giganti sono Gigante delle Colline, gigante di pietra, gigante del gelo, gigante del fuoco, gigante delle nuvole, gigante delle tempeste. Oltre questi, anche ogri e troll sono giganti.

\smallskip\textbf{Immondi}, vengono genericamente chiamati immondi le creature malvagie provenienti da altri piani. A volte sacerdoti e incantatori malvagi evocano gli immondi nel mondo materiale perché eseguano le loro volontà. Se un celestiale malvagio è una rarità, un immondo buono è praticamente inconcepibile. Gli immondi includono demoni, diavoli, segugi infernali, rakshasa, gablin...

\smallskip\textbf{Melme}, sono creature gelatinose che difficilmente hanno una forma fissa. Vivono principalmente sottoterra, stabilendosi in grotte e sotterranei, nutrendosi di rifiuti, carcasse o creature tanto sfortunate da incapparvi. I protoplasmi neri e i cubi gelatinosi sono tra le melme più riconoscibili.

\smallskip\textbf{Mostruosità}, sono mostri nel senso più stretto del termine creature spaventose che non sono comuni, né davvero naturali, e quasi mai benigne. Alcune sono il risultato di esperimenti magici andati male, mentre altri sono il prodotto di terribili maledizioni (tra cui ricordiamo il minotauro). Sfuggono a qualsiasi categorizzazione, e in qualche modo servono da categoria onnicomprensiva per quelle creature che non corrispondono a nessun altro tipo di mostro.

\smallskip\textbf{Non Morti}, sono creature un tempo vive condotte ad un orribile stato di non morte tramite la pratica della magia negromantica o qualche blasfema maledizione. Tra i non morti si annoverano cadaveri ambulanti, come vampiri e zombi, oppure spiriti incorporei, come fantasmi e spettri. Alcuni non morti più intelligenti parlano Expiran, una lingua fatti di oscuri sussurri.

\smallskip\textbf{Piante}, in questo contesto si tratta di creature vegetali, non della normale flora. La maggior parte di esse sono mobili e alcune sono carnivore. L'esempio più classico di piante sono i Cumulo Strisciante e gli Uomini Albero. Anche le creature fungoidi e i miconidi rientrano in questa categoria.

\begin{center}
\includegraphics[width=0.7\linewidth]{immagini/sanmichelesatana.png}\\
\emph{San Michele sconfigge Satana. Raffaello ed aiuti (1518). Museo del Louvre}
\end{center}

\smallskip\textbf{Umanoidi}, sono la popolazione principale dei mondi di gioco, civilizzati e selvaggi, comprendono gli umani e un'ampia gamma di altre specie. Possiedono una lingua e una cultura, poche o nessuna abilità magica innata (sebbene molti umanoidi possano apprendere gli incantesimi), ed una forma bipede. Le razze più comuni di umanoide sono quelle più adatte come personaggi del giocatore: umani, nani, elfi e nibali, diversi. Quasi altrettanto numerose, ma più brutali e selvagge, e quasi tutte malvagie, sono le razze goblinoidi (goblin, hobgoblin e bugbear), orchi, gnoll, lucertoloidi e coboldi.

\medskip

Queste categorie possono essere a loro volta raggruppate in tipologie di Creature:
\smallskip
\begin{itemize}[leftmargin=*] \setlength{\itemsep}{0pt}
\item
Le \textbf{Creature Naturali}: sono Insetti, Rettili, Bestie, Umanoidi, Piante, Creature acquatiche, Mostrusità, Melme
\item
Le \textbf{Creature Magiche} sono: Immondi, Demoni, Diavoli, Fatati, Spiriti, Non morti, Giganti, Celestiali, Costrutti, Aberrazioni (tutto ciò che è alieno o innaturale) e Draghi.

Se una Creatura Naturale ha poteri magici allora si considera anche come Creatura Magica.
\end{itemize}

\medskip\textbf{Etichette}

Un mostro può presentare una o più etichette indicate tra parentesi, a seguire il suo tipo. Ad esempio un orco ha il tipo \emph{umanoide (orco)}. Le etichette tra parentesi forniscono ulteriori categorizzazioni per determinate creature. Le etichette non hanno delle proprie regole specifiche, ma alcuni elementi del gioco, come gli oggetti magici, vi possono fare riferimento. Ad esempio, una lancia particolarmente efficace contro i demoni, funzionerebbe contro qualsiasi mostro che abbia l'etichetta demone.

\subsection{Tratti}

I mostri non presentano l'elenco dettagliato dei Tratti, troverete solo l'indicazione sugli assi del Caos, Legge, Bene e Male. Ricordatevi che sono indicazioni, le eccezioni possono capitare specialmente nelle specie più intelligenti.
Determinate creature sono \textbf{disallineate}, ovvero non hanno una condotta morale o etica.

\subsection{Difesa}

Un mostro che indossa un'armatura o trasporta uno scudo ha una Difesa che tiene conto dell'armatura, dello scudo e della Destrezza. Altrimenti, la Difesa di un mostro è basata sul suo valore di Destrezza e l'armatura naturale se la possiede (la "\emph{pellaccia}"). Se un mostro possiede un'armatura naturale, indossa armature o trasporta uno scudo, viene indicato tra parentesi dopo il valore della sua Difesa.

Qualora il mostro fosse \textbf{colto di sorpresa} sottraete alla Difesa -4.

\subsection{Punti Ferita}

Di solito quando scende a 0 Punti Ferita, un mostro muore o viene considerato morto.

I Punti Ferita di un mostro sono presentati con il suo valore.

Anche il valore di Costituzione di un mostro influenza il numero di Punti Ferita che possiede. Il suo valore di Costituzione viene moltiplicato per il Grado di Sfida che possiede e il risultato viene sommato ai suoi Punti Ferita. Ad esempio, un mostro che ha Costituzione 1 e Grado di Sfida 2 avrà, \emph{mediamente} (GS+1)*15+GS*Cos = 47 Punti Ferita.

Capiterà che i giocatori vi chiedano \textbf{\emph{come sta il mostro}}, vi suggerisco di non scendere mai nei dettagli dicendo quanti Punti Ferita ha in tutto o ne ha persi, bensì rimanere in questi gradi: Non ferito (Punti Ferita pieni), Ferito (30\% Punti Ferita subiti), Gravemente ferito (almeno 50\% Punti Ferita subiti), ovvero dare una descrizione generica dello stato. \index{Come sta il mostro}\index{Chiedere Punti Ferita del Mostro}

\subsubsection{Arrabbiato}\index{Arrabbiato}\index{Bloodied}\label{mostroarrabbiato}

A discrezione del Narratore una creatura che abbia perso almeno i 50\% dei Punti Ferita totali innesca una furia che gli permette azioni particolari.
I mostri con Grado di Sfida 5 o più possono avere una scheda \textbf{Arrabbiato}. L'abilità Arrabbiato si può usare una volta per scontro al costo, se non segnato diversamente, di 1 Azione.

Creature particolarmente feroci e potenti potrebbero avere più note di Arrabbiato ed entrambe, rispettando le eventuali condizioni segnate, sono attivabili.

%\subsubsection{Opzionale - Tutti Arrabbiati}\index{Opzionale - Tutti Arrabbiati}
%Per una maggiore aggressività potete fare che il mostro quando scende sotto la metà dei Punti Ferita prende +1d6 al Tiro per Colpire oppure al Danno oppure ai Tiri Salvezza a seconda del tipo di creatura.\index{Sanguinante}\index{Bloodied} \index{Arrabbiato}

\medskip

Potete anche decidere che la creatura annulla una condizione che ha su di se.

\subsection{Movimento}

Il Movimento di un mostro ti dice di quanto si possa muovere durante il suo round per Azione di Movimento

Tutte le creature possiedono un movimento di passeggio, detto semplicemente movimento del mostro. Le creature che non possiedono alcuna forma di spostamento terreno hanno velocità di movimento 0 metri.

Alcune creature possiedono uno o più dei seguenti modi di movimento aggiuntivi.

%\begin{center}
%\includegraphics[width=0.65\linewidth]{immagini/roc.png}\\
%\emph{Henry Justice Ford}
%\end{center}

\smallskip\textbf{Nuoto}

Un mostro che possieda una velocità di nuoto non deve spendere movimento extra per nuotare (non è terreno difficile)

\smallskip\textbf{Scalata}

Un mostro che possieda una velocità di scalata può usare tutto o solo parte del suo movimento per muoversi su superfici verticali. Il mostro non deve spendere movimento extra (x4) per scalare.

\smallskip\textbf{Scavo}

Un mostro che possieda una velocità di scavo può usare la sua velocità per attraversare sabbia, terra, fango, ecc. Un mostro non può scavare attraverso la roccia solida a meno che non possieda un tratto speciale che glielo permetta.

\smallskip\textbf{Volo}

Un mostro che possieda una velocità di volo può usare tutto o solo parte del suo movimento per volare. Alcuni mostri hanno l'abilità di \hyperlink{Fluttuare}{\textbf{Fluttuare}} (pag. \pageref{Fluttuare}), che li rende difficili da abbattere. Il mostro smette di fluttuare quando muore.

\subsection{Punteggi di Caratteristica}

Ogni mostro possiede sei punteggi di caratteristica (Forza, Destrezza, Costituzione, Intelligenza, Saggezza, Carisma)

\subsection{Competenze}\index{Competenza Armi dei mostri}

La voce Competenze è riservata a quei mostri che sono capaci in una o più competenze peculiari diverse da quelle che normalmente userebbe per vivere. Ad esempio, un mostro che è molto attento e furtivo potrebbe avere bonus alle prove di Consapevolezza e Destrezza.

Si possono applicare anche altri modificatori, ad esempio, un mostro potrebbe avere un bonus maggiore del previsto per tenere conto della sua grande perizia.

Se non indicata, ma necessaria per le prove (non al Tiro per Colpire, dove si usa il valore già segnato) la Competenza Armi di un Mostro è pari al suo Grado di Sfida.

\medskip

\textbf{Tabella: Equivalenza Armi Magiche}\index[Tabelle]{Tabella Equivalenza Armi Magiche}\label{equivalenzaarmimagiche}\hypertarget{equivalenzaarmimagiche}{}

\medskip

\begin{tabular}{lp{0.055\textwidth}p{0.06\textwidth}p{0.07\textwidth}}
	\toprule
	\textbf{Immunità} & \textbf{Magia Arma} & \textbf{Attacco Nat. (CA)}& \textbf{Pugno Vuoto}\\
	+1 & +1 & 3& 2\\
	+2 & +2 & 6& 4\\
	Ferro Freddo & +1 & 4& 2\\
	Argento & +1 & 4& 2\\
	Adamantio & +2 & 6& 4\\
	+3 & +3 & 12& 8\\
	+4 & +4 & 16& 12\\
	+5 & +5 & - & 16
\end{tabular}

\subsection{Vulnerabilità, Resistenze e Immunità}\index{Equivalenze armi}\index{Pugni magici}\label{vulnerabilitaresistenze}
Alcune creature possiedono vulnerabilità, resistenze o immunità ad un certo tipo di danno. Creature particolari sono addirittura resistenti o immuni agli attacchi non magici (un attacco magico è un attacco sferrato tramite un incantesimo, un oggetto magico o arma, o un'altra fonte di magia).

Quando è indicata una immunità alle armi magiche (es. +1 oppure +2) significa che bisogna usare un arma con un incantamento maggiore per poter danneggiare la creatura. In caso di creature immune ai critici questo vale sia per incantesimi che per armi, rimane efficace l'esplosione del danno. \index{Critico sui mostri}.

Una creatura immune alle armi non magiche o +1 ma vulnerabile al ferro freddo o all'argento applica prima le sue immunità poi se passate applica le vulnerabilità all'attacco subito, e quindi un arma d'argento non farà danno, ma se d'argento +1 farà il doppio del danno.

Inoltre, certe creature sono immuni a determinate condizioni. Se un mostro è immune ad un effetto di gioco che non viene considerato danno o condizione, possiede invece un tratto speciale.

Nella tabella sottostante viene indicato quale incantamento magico dell'arma è necessario per superare l'immunità indicata. E' anche indicato il punteggio minimo di Competenza Armi nel caso si colpisca con calci e pugni.

In caso di personaggio con Lista d'Armi \textbf{Pugno Vuoto} si controlla quanto volte si è presa la lista.

\subsection{Consapevolezza}

Tutti i mostri, quando non segnato, hanno un valore di Consapevolezza pari a \textbf{Grado di Sfida/2 + Saggezza}.

\subsection{Sensi}

La voce Sensi elenca qualsiasi senso speciale di cui il mostro sia in possesso. I sensi speciali sono descritti di seguito. Se non è presente la voce Sensi, la creatura ha dei sensi standard (visione, olfatto, gusto, tatto...) non particolarmente evoluti.

\begin{center}
	\includegraphics[width=0.7\linewidth]{immagini/ciclope.png}

	\emph{Henry Justice Ford}
\end{center}

\subsubsection{Percezione Tellurica}

Un mostro con percezione tellurica può individuare e trovare le origini delle vibrazioni entro uno specifico raggio, purché il mostro e la fonte della vibrazione siano in contatto con lo stesso terreno o sostanza. La percezione tellurica non può essere impiegata per individuare creature volanti o incorporee. Molte creature scavatrici, come gli ankheg e i colossi di terra, possiedono questo senso speciale.

\subsubsection{Visione Crepuscolare o Scurovisione}

Una creatura con Visione Crepuscolare può vedere nella più tenue delle luci, ma non nell'oscurità completa a differenza di quelle che possiedono scurovisione. Molte creature che vivono sottoterra possiedono questo senso speciale. Vedi capitolo \hyperlink{visioneeluce}{Caratteristiche Speciali}.

\subsubsection{Visione del Vero}

Un mostro con la visione del vero può, fino ad una specifica gittata, vedere attraverso l'oscurità normale e magica, vedere creature e oggetti invisibili, automaticamente individuare le illusioni e riuscire i Tiri Salvezza contro di loro, percepire la forma originale di un mutaforma o di una creatura trasformata dalla magia. Inoltre, la creatura può vedere nel Piano Etereo fino alla stessa gittata.

\subsubsection{Vista Cieca}

Una creatura con vista cieca può percepire l'ambiente circostante, senza fare affidamento alla vista, fino ad una specifica gittata.

Le creature senza occhi come i grimlock e le melme e le creature con ecolocazione o sensi potenziati, come i pipistrelli ed i draghi, possiedono questo senso.

Se un mostro è cieco di natura, la cosa viene annotata tra parentesi, in questo caso la portata della sua vista cieca definisce anche la portata massima della sua percezione.

\subsection{Linguaggi}

Le lingue che un mostro può parlare sono riportate in ordine alfabetico. Se un mostro può capire una lingua ma non parlarla la cosa viene indicata in questa voce. Se un mostro non ha la nota \emph{Linguaggi} significa che non conosce linguaggi diversi dalla propria lingua (se applicabile).

\subsection{Telepatia}

La telepatia è un'abilità che permette ad un mostro di comunicare mentalmente con un'altra creatura nel raggio di azione specificato. La creatura contattata non è necessario che parli la stessa lingua del mostro per comunicare in questo modo. Una creatura senza telepatia può ricevere e rispondere a messaggi telepatici ma non può iniziare o terminare una conversazione telepatica.

Un mostro telepatico non ha bisogno di vedere la creatura contattata e può terminare il contatto telepatico in qualsiasi momento. Il contatto è infranto non appena le due creature non si trovano più entro il raggio di azione o se il mostro telepatico contatta un'altra creatura a gittata. Un mostro telepatico può iniziare o terminare una conversazione telepatica senza dover usare un'azione, ma mentre il mostro è inabile non può dare inizio ad un contatto telepatico, e qualsiasi contatto in corso viene terminato. Per avviare una comunicazione telepatica l'obiettivo deve essere stato almeno individuato.

Una creatura nell'area di un \emph{campo anti-magia} o in qualsiasi altro posto in cui la magia non funziona può inviare o ricevere messaggi telepatici.

\subsection{Sfida}

Il \textbf{grado di sfida} (GS) di un mostro vi dice quanto sia grande la minaccia che pone. Una compagnia di quattro avventurieri equipaggiata in maniera appropriata e riposata deve essere in grado di sconfiggere un mostro dal grado di sfida pari al proprio livello medio senza subire perdite. Ad esempio, una compagnia di quattro personaggi di 3° livello dovrebbe ritenere un mostro di grado di sfida 3 una sfida normale e non pericolosa.

I mostri che sono significativamente più deboli dei personaggi di 1° livello hanno un grado di sfida inferiore ad 1. I mostri con un grado di sfida 0 non presentano problemi eccetto in grandi numeri, quelli privi di reali attacchi non valgono punti esperienza.

\subsection{Riconoscere i Mostri}\label{riconoscereimostri}\hypertarget{riconoscereimostri}{} \index{Riconoscere i Mostri}

Sapere riconoscere un mostro può essere estremamente utile ed è qualcosa che non andrebbe mai sottovalutato.

Per \textbf{riconoscere un mostro} si effettua una prova di Conoscenza. (\textbf{1 Azione}) su:

\medskip

\noindent\begin{itemize}[leftmargin=*] \setlength{\itemsep}{0pt}
\item \textbf{Arcana}: Giganti, Costrutti, Spiriti, Mostruosità, Aberrazioni, Draghi
\item \textbf{Piani}: Elementali, Giganti
\item \textbf{Occulto}: Immondi (Diavoli e Demoni), Spiriti, Non Morti
\item \textbf{Religione}: Spiriti, Non Morti, Celestiali
\item \textbf{Dungeon}: Aberrazioni, Mostruosità, Melme e creature sotterranee
\item \textbf{Natura}: Bestie, Piante, Fatati
\end{itemize}

\medskip

La DC delle prova è pari \textbf{10 + grado di Sfida} della creatura + \textbf{fattore di rarità}/notorietà (comune (0), non comune (+1), raro (+2), molto raro (+4), leggendario +(10)).

Le informazioni ottenibili dipendono dal margine di successo ottenuto.

\noindent\begin{itemize} \setlength{\itemsep}{0pt}
\item \textbf{entro 2}: nome, tipo, la Caratteristica principale
\item \textbf{fino 7}: quale è il migliore Tiro Salvezza, una immunità a Condizioni, una vulnerabilità a Danni, attacco tipico
\item \textbf{fino 12}: quale è il peggiore Tiro Salvezza, una immunità a Condizioni, una immunità a Danni, una vulnerabilità a Condizioni, una vulnerabilità a tipo di Danno
\item \textbf{fino 15}: due immunità a Condizioni, una immunità a Danni, una vulnerabilità a Condizioni, una vulnerabilità a tipo di Danno
\item \textbf{fino 17}: grado di sfida relativo ovvero se è una scontro facile, medio, alto, straordinario, mortale o epico
\item \textbf{oltre 17}: attacco e difese speciali
\end{itemize}

\medskip

Le informazioni ottenute sono cumulative, ovvero se la prova riesce di 15 ottieni le informazioni entro 2, 7 e 12.

\subsection{Tratti Speciali}

I tratti speciali (che compaiono dopo il grado di sfida di un mostro ma prima di qualsiasi Azione o reazione) sono peculiarità che avranno probabilmente un ruolo in un incontro di combattimento e che richiedono delle spiegazioni.

\subsection{Incantesimi}

Un mostro con il privilegio Incantesimi o Incantesimi Innati è in grado di lanciare Incantesimi.

La \textbf{DC è 10 + livello incantesimo x2 + Intelligenza o Saggezza a seconda della caratteristica migliore oppure indicata}. Un mostro non necessita di eseguire Prove di Magia ma può farle se ha un valore di Competenza Magica (es. Lich, Mummia, Naga...).

Il Tiro per Colpire con Incantesimi è pari al valore di Competenza Magica se segnata, se non è segnata è  pari a metà del GS + Intelligenza o modificatore di caratteristica indicato. Se è necessario calcolare la Competenza Armi per l'uso di incantesimi e questa non è specificata, allora è pari alla metà del punteggio di Competenza Magica.

Un mostro con incantesimi può lanciare un numero di incantesimi segnati di quel livello pari a \emph{slot}, scegliendoli tra quelli indicati.

%\begin{center}
%	\includegraphics[width=0.65\linewidth]{immagini/lich2.png}

%	\emph{Lich - Battle of Wesnoth}

%\end{center}

\subsection{Azioni}

Anche i mostri agiscono secondo lo schema delle 3 Azioni disponibili per round. Possono essere segnate abilità e capacità che gli permettono di eseguire un numero più elevato di Azioni.

Quando un mostro svolge le sue azioni, può scegliere tra le opzioni della sezione Azioni del suo blocco statistiche o impiegare una delle Azioni disponibili a tutte le creature, come Scattare, Nascondersi, \hyperlink{preparareladifesa}{Preparare la difesa}, se non indicato diversamente usare una Azione (non facente parte del Multiattacco o segnata come \emph{Attacco}) costa 2 Azioni.

\subsubsection{Attacchi da Mischia e a Distanza}\index{Mostri e Attacchi}

L'azione più comune che un mostro effettuerà in combattimento sarà un attacco in mischia o a distanza. Possono essere attacchi con incantesimi o attacchi con armi, dove l'arma può essere un manufatto o un'arma naturale, come gli artigli o la coda chiodata.

\emph{\textbf{Creatura contro Bersaglio}.} Il bersaglio di un attacco da mischia o a distanza è di solito una creatura o un bersaglio.

\textbf{Portata}: la portata indicata è la distanza \textbf{entro} quanti metri la creatura può colpire l'avversario. Una creatura con portata 0 deve esserti addosso per colpirti, solitamente hanno portata 0 le creature estremamente piccole.

\emph{\textbf{Colpisce.}} Qualsiasi danno inflitto o altro effetto che avviene come risultato di un attacco che colpisce il bersaglio viene descritto nell'annotazione \emph{Colpisce}. Puoi scegliere se prendere il danno medio o tirare i dadi; per questo motivo vengono presentati sia il danno medio che una formula di dadi.

Anche nel Tiro per Colpire per i Mostri valgono le Golden Rules.

Il Tiro per Colpire del mostro \textbf{non ha applica danno critico ne esplosione del danno}, ma \textbf{non subisce penalità per il multiattacco}. Ogni attacco del mostro, quindi anche 3 attacchi a round, viene effettuato con il Tiro per Colpire senza penalità del Multiattacco.\index{Attacco dei mostri}

\textbf{\emph{Manca}.} Se un attacco ha un effetto prodotto da un colpo a vuoto, quell'informazione viene fornita dall'annotazione \emph{Manca}.

\emph{\textbf{Danni.}} Se un mostro impugna armi manufatte, infligge danni appropriati all'arma. I mostri più grossi di solito impugnano armi di dimensioni superiori che infliggono danni extra quando colpiscono. Se usano questo tipo di armi il danno è già segnato, altrimenti se raccolgono o usano un arma non prevista raddoppiare i dadi dell'arma se la creatura è Grande, triplicarli se Enorme e quadruplicarli se Mastodontica qualora usino armi della loro taglia.

%\begin{changemargin}{0.3cm}{0.3cm}\begin{narratore}
%Una creatura che abbia almeno Grado di Sfida 6 a discrezione del Narratore può sferrare un attacco di Opportunità al costo di una Reazione (vedi \hyperlink{opportunista}{Opportunista} pag. \pageref{opportunista}).

%\medskip
%Per valorizzare i mostri e renderli più incisivi potete decidere che ogni mostro abbia una \hyperlink{riduzionedeldanno}{Riduzione del Danno} pari a metà del suo Grado di Sfida ( $\frac{GS}{2}$/- )

%\end{narratore}\end{changemargin}

\subsubsection{Multiattacco e Attacco}\index{Multiattacco dei mostri}

L'\textbf{Azione Multiattacco consuma 2 Azioni} anche se porta più di 2 attacchi.

Come per l'Attacco normale i mostri non cumulano le penalità del Multiattacco e non hanno il Tiro Critico o l'Esplosione del Danno. Ogni attacco è portato con il bonus al Tiro per Colpire segnato.

Altre \emph{Azioni di Attacco} elencate sotto Multiattacco ma non facente parte di quelle elencate nella descrizione del Multiattacco costano 1 Azione se non descritto diversamente e non cumulano le penalità del Multiattacco.\index{Multiattacco dei Mostri}

Ad esempio il lancio del Sasso per il \hyperlink{Gigante delle Colline}{Gigante delle Colline} costa 1 Azione perché e' un \emph{Attacco} e non fa parte delle Azioni elencate nel Multiattacco.

Il Soffio infuocato di una \hyperlink{Chimera}{Chimera} non ha l'attributo \emph{Attacco} e quindi costa 2 Azioni.

Usare \textbf{1 solo Attacco} della catena del Multiattacco \textbf{costa 1 Azione}.

\subsubsection{Regole dell'Afferrare per i Mostri}

Molti mostri possiedono un attacco speciale che gli permette di afferrare rapidamente la preda. Quando un mostro colpisce con un simile attacco, non deve effettuare un'ulteriore prova per determinare se l'afferrare riesce a meno che l'attacco non dica altrimenti.

Una creatura afferrata dal mostro segue le indicazioni di \hyperlink{afferrareunavversario}{Afferrare un avversario} (pag. \pageref{afferrareunavversario}).

Se non viene fornita una \textbf{DC di fuga} assumere che sia ia uguale a 10 + (Tiro Salvezza su Tempra  + Forza) del mostro +1d6 per Taglia di differenza.

\subsubsection{Munizioni}

Un mostro porta con sé munizioni sufficienti per effettuare i suoi attacchi a distanza. Puoi presumere che un mostro abbia 2d4 proiettili per un attacco con armi da lancio (giavellotti, macigni..), e 2d10 proiettili per un'arma a proiettili come un arco o una balestra.

\begin{center}
	\includegraphics[width=0.65\linewidth]{immagini/polpo.png}

	\emph{Alphonse de Neuville - Hetzel edition of 20000 Lieues Sous les Mers}
\end{center}

\subsubsection{Reazioni}

Se un mostro può compiere qualcosa di speciale con le sue reazioni, è riportato qui. Se una creatura non ha reazioni speciali, questa sezione è assente.

\subsubsection{Uso Limitato}

Alcune abilità speciali hanno restrizioni sul numero di volte che possono essere usate.

\textbf{\emph{X/Giorno}.} L'annotazione "X/Giorno" indica un'abilità speciale che può essere usata X volte prima che il sorga l'alba per recuperare gli usi consumati. Ad esempio, \emph{1/Giorno} indica un'abilità speciale che può essere usata una volta prima che il mostro debba aspettare la nuova alba.

\emph{\textbf{Ricarica X-Y.}} L'annotazione "Ricarica X-Y" indica che il mostro può usare un'abilità speciale una volta e che l'abilità ha una probabilità casuale di ricaricarsi ogni round seguente di combattimento. All'inizio di ciascun round del mostro, tira un d6. Se il risultato è uno dei numeri dell'annotazione di ricarica, il mostro recupera l'uso dell'abilità speciale. L'abilità si ricarica anche all'alba di un nuovo giorno.

Ad esempio, \emph{Ricarica 5-6} indica che un mostro può usare la sua abilità speciale una volta. Poi, all'inizio del round del mostro, recupera l'uso dell'abilità se tira 5 o 6 su di un d6.

\subsection{Azioni Aggiuntive}

Certe creature possono possono eseguire azioni speciali al di fuori del proprio round, ed alcune possono estendere il proprio potere all'ambiente, provocando la manifestazione di effetti magici straordinari nelle loro vicinanze.

Una creatura con azioni aggiuntive può effettuare un certo numero di azioni speciali, dette \emph{azioni aggiuntive}, al di fuori del suo round. Solo un'azione aggiuntiva può essere usata alla volta e solo al termine del round di un'altra creatura. Non costa Azioni o Reazioni usare una Azione Aggiuntiva. Una creatura con azioni aggiuntive recupera all'inizio del suo round le azioni aggiuntive che ha usato. Non è obbligata ad usare le sue azioni aggiuntive e non può usare le azioni aggiuntive mentre è inabile o altrimenti incapace di effettuare Reazioni. Se sorpresa, non può usarle fin dopo il suo primo round di combattimento.

Se una creatura assume la forma di una creatura con azioni aggiuntive, magari tramite un incantesimo, non ne ottiene però le azioni aggiuntive o le azioni da tana.

\subsubsection{La Tana di una Creatura}\index{Tana di una Creatura}\index{Azioni da tana}

Una creatura con Azioni aggiuntive può presentare una sezione che ne descrive la tana e gli effetti speciali che vi può creare mentre si trova lì, o per propria volontà o semplicemente grazie alla sua presenza. Questa sezione si applica solo alle creature leggendarie che trascorrono molto tempo nelle loro tane dove è altamente probabile che li vengano incontrate.

Se una creatura con azioni aggiuntive ha un' \textbf{Azione da tana}, può usarla per imbrigliare la magia ambientale della sua tana. Al conteggio di iniziativa 10, perdendo i pareggi, la creatura può usare una delle sue opzioni di azioni da tana. Non può farlo mentre è inabile o altrimenti incapace di effettuare azioni. Se sorpresa, non può farne uso fino a dopo il suo primo round di combattimento.

%\medskip

%\begin{center}
%\includegraphics[width=0.7\linewidth]{immagini/cupido.png}

%\emph{Eros con il suo arco. Musei Capitolini}
%\end{center}

\subsection{Equipaggiamento}

Il blocco statistiche si riferisce all'equipaggiamento, oltre le armi o le armature utilizzate dal mostro. Una creatura che normalmente indossa abiti, come un umanoide, si assume sia vestito in maniera appropriata.

Puoi equipaggiare i mostri con ulteriore equipaggiamento come preferisci, utilizzando il capitolo \hyperlink{equipaggiamento}{Equipaggiamento} come fonte di ispirazione, sei tu a decidere quanto dell'equipaggiamento del mostro è recuperabile dopo che la creatura è stata uccisa o se qualsiasi parte del suo equipaggiamento sia ancora utilizzabile. Ad esempio, un'armatura ammaccata fatta per un mostro difficilmente sarà utilizzabile da qualcun altro. Se un mostro incantatore necessita di componenti materiali per lanciare i suoi incantesimi, dai per scontato che abbia le componenti materiali per lanciare gli incantesimi indicati nella sua scheda.

\subsection{Punti Esperienza per GS}

Ogni mostro se \emph{sconfitto} concede un certo ammontare di Punti Esperienza da suddividere tra tutti i partecipanti allo scontro.

Questa tabella indica per GS i Punti Esperienza relativi.

\medskip

\textbf{Tabella: Grado di Sfida e Punti Esperienza}\index[Tabelle]{Tabella Punti Esperienza per Grado di Sfida}

\medskip

\begin{tabularx}{0.42\textwidth}{ll|ll|ll}

\textbf{GS} & \textbf{PX} &\textbf{GS} & \textbf{PX} &\textbf{GS} & \textbf{PX}\\
0& 10 &9& 5000& 21&33000\\
1/8& 25 &10& 5900&22&41000\\
1/4& 50 &11& 7200&23&50000\\
1/2& 100 &12& 8400&24&62000\\
1& 200 &13& 10000&25&75000\\
2& 450&14& 11500&26&90000\\
3& 700&15& 13000&27&105000\\
4& 1100&16& 15000&28&120000\\
5& 1800&17& 18000&29&135000\\
6& 2300&18& 20000&30&155000\\
7& 2900&19& 22000&&\\
8& 3900&20& 25000&&
\end{tabularx}

\subsection{Tipologie di Tesoro}

Ogni tipologia di creatura può preferire un tipo di tesoro (inteso come oggetti, monete, gemme...) diverso. Questi sono solo suggerimenti su come costruire il tesoro del mostro.

Vedi anche \hyperlink{valoretesoroincontro}{Tabella: Valori del Tesoro per Incontro} (pag. \pageref{valoretesoroincontro}).

\medskip

\begin{itemize}[leftmargin=*] \setlength{\itemsep}{0pt}

	\item \textbf{Aberrazione}
	Molte aberrazioni hanno scarsa considerazione per i tesori, possedendo solo quel che prendono dai resti delle loro vittime. Alcune sono nemici intelligenti che usano vari oggetti magici e tesori per potenziare le loro capacità.

	\item \textbf{Animale - Bestia Magica} Gli animali non prestano alcuna attenzione ai tesori, lasciando monete e oggetti tra i resti dei loro pasti. Per quelli che possiedono un tesoro, questo si trova solitamente nelle loro tane, sparso tra ossa e altri rifiuti.

	\item \textbf{Costrutto}
	Spesso è il costrutto stesso il tesoro più di valore. I costrutti sono solitamente usati per sorvegliare tesori o oggetti magici di grande valore.

	\item \textbf{Drago}
	Noti per i loro preziosi tesori, i draghi spesso riposano su pile di monete, gemme, oggetti magici e costosi.

	\item \textbf{Esterno}
	Gli esterni sono tra le creature più diverse e, di conseguenza, possono possedere qualsiasi tipo di tesoro, sia su di loro che nascosto nei loro rifugi. Il Narratore dovrebbe valutare ogni singola creatura per determinare il tipo di tesoro più adatto a ciascun esterno.

	\item \textbf{Folletto}
	I folletti danno valore agli oggetti belli e magici. Hanno scarsa considerazione per monete e merci.

	\item \textbf{Melma - Parassita - Vegetale}
	Le melme non sanno cosa è un tesoro e lasciano dove trovano tutto ciò che non è digeribile. Qualsiasi tesoro possano trasportare è completamente accidentale.

	\item \textbf{Non Morto}
	I tesori posseduti dai non morti dipendono dalla loro intelligenza. I non morti senza intelletto di solito hanno solo i pochi beni che avevano in vita, raramente utili come tesori. Al contrario, i non morti intelligenti utilizzano una varietà di oggetti magici per annientare i viventi.

	\item \textbf{Umanoide}
	Queste creature sono molto diverse tra loro, ma anche gli umanoidi più primitivi utilizzano equipaggiamenti e oggetti magici in qualche misura. In gruppi più grandi, come le comunità, gli umanoidi spesso possiedono una notevole quantità di tesori che custodiscono collettivamente.

\end{itemize}

\subsubsection{Opzionale - Esperienza per Sfida}\index{Opzionale - Esperienza per Sfida}

Con questo sistema i Punti Esperienza sono dati in base alla difficoltà relativa della Sfida dato il livello dei personaggi. Uno scontro con 5 Troll non darà (1800 x 5) Punti Esperienza, ma a seconda della sfida relativa concederà un ammontare diverso.

Il gruppo di Troll (Sfida 5, 1800 PX) non da sempre 1800 PX a troll sconfitto; se viene affrontato da un gruppo di basso livello, ovvero per una sfida di difficoltà Straordinaria, ne darà di più mentre affrontato da un gruppo di alto livello, dove 5 troll sono una sfida Alta, ne darà di meno.

Con questo sistema ogni 1000 Punti Esperienza si passa di livello. Valgono tutte le considerazione del capitolo Masterizzare per preparare gli scontri.

\medskip

\textbf{Tabella: Punti Esperienza per Grado di Sfida}\index[Tabelle]{Tabella Punti Esperienza per Grado di Sfida}

\begin{tabular}{ll|ll}
\textbf{Grado di Sfida} & \textbf{PX}&\textbf{Grado di Sfida} & \textbf{PX}\\
\toprule
Facile& 20& Media& 30\\
Alta& 50& Straordinaria& 80\\
Mortale& 120& Epica& 160
\end{tabular}

\medskip

Anche per trappole o sfide superate si usa questo sistema per calcolare i PX guadagnati. I Punti Esperienza premio per ogni obiettivo personale o di gruppo raggiunto sono 10.

\subsubsection{Opzionale - Metodo alternativo per costruire gli Incontri}\index{Opzionale - Sistema alternativo per costruire gli Incontri}

\begin{enumerate}[leftmargin=*] \setlength{\itemsep}{0pt}

\item \textbf{Definire l'APL (Average Party Level):} Calcolare il livello medio del gruppo. Sommare i livelli di tutti i personaggi e dividere per il numero di personaggi, come già spiegato.

%Questa guida presuppone un gruppo di quattro o cinque personaggi. Se il vostro gruppo ha sei o più giocatori, aggiungete uno al loro livello medio. Se il vostro gruppo contiene tre o meno giocatori, sottraete uno dal loro livello medio. Per esempio, se il vostro gruppo consiste di sei giocatori, due di 5° livello e quattro di 7° livello, il APL è il 7° (38 livelli totali, diviso per sei giocatori, arrotondando all'intero più vicino, ed aggiungendo uno al risultato finale).

\item \textbf{Stabilire la Difficoltà Desiderata:} Decidere quale livello di sfida si vuole presentare al gruppo.

%\begin{itemize}
%\setlength{\itemsep}{-1pt} % Set item separation to zero
%\setlength{\itemindent}{-1pt}
%\item \makebox[2.5cm][l]{Facile:} 75\% - 125\% \\
%\item \makebox[2.5cm][l]{Media:} 126\% - 175\% \\
%\item \makebox[2.5cm][l]{Impegnativa:} 176\% - 225\% \\
%\end{itemize}

\begin{itemize}[leftmargin=*]
\setlength{\itemsep}{0pt}    % Riduce distanza tra gli item
\setlength{\parsep}{0pt}     % Riduce spazio extra nei paragrafi interni
\setlength{\topsep}{0pt}     % Riduce spazio prima della lista
\setlength{\partopsep}{0pt}  % Elimina spazio extra sopra la lista
\item \makebox[2.5cm][l]{Facile:} 75\% - 105\%
\item \makebox[2.5cm][l]{Media:} 106\% - 145\%
\item \makebox[2.5cm][l]{Impegnativa:} 146\% - 195\%
\item \makebox[2.5cm][l]{Alta:} 196\% - 255\%
\item \makebox[2.5cm][l]{Straordinaria:} 256\% - 325\%
\item \makebox[2.5cm][l]{Mortale:} 326\% - 405\%
\item \makebox[2.5cm][l]{Epica:} 406\% e oltre
\end{itemize}

%\begin{tabular}{@{}ll@{}} % @{} removes extra padding
%Facile: & 75\% - 105\% \\
%Media: & 106\% - 145\% \\
%Impegnativa: & 146\% - 195\% \\
%Alta: & 196\% - 255\% \\
%Straordinaria: & 256\% - 325\% \\
%Mortale: & 326\% - 405\% \\
%Epica: & 406\%+
%\end{tabular}

\item \textbf{Assegnare un valore Percentuale ai Mostri:} Utilizzare la tabella seguente per determinare il \emph{peso} (percentuale) di ciascun Avversario (\emph{Avv.}) in base alla differenza tra il suo Grado di Sfida (GS) e all'APL del gruppo (\emph{Rapp.})

\noindent\begin{tabularx}{0.95\linewidth}{c|c|c|c}
\textbf{Rapp.} & \textbf{\% per Avv.} &\textbf{Rapp.} & \textbf{\% per Avv.}\\
\toprule
-6 & 3\% &  0 & 70\% \\
-5 & 5\% & +1 & 105\% \\
-4 & 10\% & +2 & 160\% \\
-3 & 15\% & +3 & 240\% \\
-2 & 25\% & +4 & 360\% \\
-1 & 45\% & +5 & 480\% \\
\end{tabularx}

Partite dai mostri con Grado di Sfida più alto e poi aggiungere mostri con GS più basso per raggiungere la percentuale desiderata.

\end{enumerate}

\end{multicols}

\vfill

\begin{center}
\includegraphics[width=0.6\linewidth]{immagini/trex.png}

\emph{Tyrannosaurus Rex skeleton (the specimen RTMP 81.6.1), California Academy of Sciences, San Francisco}

\emph{(Tirannosauro, scheletro, GS 12!)}
\end{center}

%\vfill

%\begin{changemargin}{0.3cm}{0.3cm}\begin{enfasi}{
%Io sono il mostro che gli uomini che respirano bramerebbero uccidere. Io sono Dracula. (Dracula, Bram Stoker)}\end{enfasi}\end{changemargin}

\pagebreak

