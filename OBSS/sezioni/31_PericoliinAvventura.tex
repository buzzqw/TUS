\section{Pericoli in Avventura}\index{Pericoli in Avventura}


\begin{changemargin}{0.3cm}{0.3cm}\begin{enfasi}{
Un'avventura è un risultato ragionevole. Due sono meglio, tre meritano di essere tramandate, e quattro... nessuno potrà mai contestare quattro avventure. (John Steinbeck)

\medskip


Corre meno pericoli colui che, anche se è al sicuro, sta in guardia. (Publilio Siro)} \end{enfasi}\end{changemargin}\medskip

\label{pericoli-in-avventura}

\begin{multicols}{2}

Il mondo è pieno dì pericoli oltre che di draghi ed immondi famelici. I pericoli sono minacce presenti nell'ambiente e che hanno molto in comune con le trappole, ma che di solito fanno parte del posto anziché venir costruite. I pericoli si dividono in tre categorie principali: ambientali, viventi e magici.

I pericoli ambientali includono frane, incendi e simili. I pericoli viventi includono creature che pur non essendo considerate mostri, rappresentano una minaccia per gli avventurieri \emph{incauti}, come fanghiglie, funghi e muschi. I pericoli magici sono i più imprevedibili e possono essere residui di esperimenti arcani, strane radiazioni sotterraneo o antichi incantesimi falliti.

\medskip

\begin{center}
\includegraphics[width=0.75\linewidth]{immagini/boscopericoli.png}
\end{center}

\textbf{Zona di Antimagia (grado di Sfida 6)}\index{Zone di Antimagia}\index{Antimagia}

Zona di entropia magica che distruggono le magie, le zone di Antimagia si formano sui siti di grandi duelli magici, attraverso la distruzione di potenti artefatti o da vortici di energia mistica ai margini delle zone di antimagia. Le dimensioni variano da piccole bolle di appena pochi metri fino a grandi aree delle dimensioni di una città.

Una prova riuscita di Arcana con DC 20 rivela la vicinanza di una Zona di Antimagia con un formicolio nell'aria. Una magia attiva portata in una zona antimagica potrebbe venir dissolta, qualsiasi incantesimo lanciato al suo interno è soggetta ad un contro incantesimo immediato. Se si ottiene un critico nella Prova di Magia questo riesce a passare il contro incantesimo ma non genera ulteriori effetti.

Se l'incantesimo fallisce il rilascio di energia magica infligge 2d6 danni da forza in un'esplosione in un raggio di 3 metri centrata su chi ha tentato l'incantesimo; un Tiro Salvezza su Riflessi a DC 15 permetti di dimezzare questo danno.

Una magia manifestata da un oggetto, che non sia un Artefatto, fallisce sempre.

Se più scoppi sovrapposti colpiscono lo stesso bersaglio, si applica solo quello più dannoso. Una magia che ha resistito ad un tentativo di dissoluzione, non viene influenzato nuovamente a meno che non esca e rientri dalla zona.

Le zone antimagiche più potenti sono ancora più distruttive. Ogni +1 di incremento del grado di Sfida aumenta di 1d6 il danno e la DC del Tiro Salvezza di 1.

\medskip
\textbf{Aria Viziata (grado di Sfida 1 o 4)}\index{Aria Viziata}

Le sacche di gas sono un rischio per minatori, speleologi e avventurieri che investigano nelle caverne. I gas ininfiammabili hanno grado di Sfida 1 e richiedono una prova di Sopravvivenza con DC 25 per essere notati. Le creature che respirano quell'aria devono superare un Tiro Salvezza su Tempra (DC 15 +1 per ogni tiro precedente) ogni ora o diventano Affaticati. Le creature che trattengono il fiato possono evitare questi effetti.

I vapori infiammabili sono molto più pericolosi (grado di Sfida 4). Questo gas sostituisce l'aria respirabile nei polmoni, provocando affaticamento: inoltre, qualsiasi fiamma aperta o scintilla causa un'esplosione che infligge 6d6 danni (TS su Riflessi con DC 15 dimezza) a chi è nella caverna o entro 3 metri da un ingresso. Il fuoco brucia l'ossigeno nell'aria, rendendola irrespirabile per 2d4 minuti. Dopo un'esplosione, il gas infiammabile generalmente impiega molti giorni per ritornare a livelli pericolosi.

\medskip
\textbf{Parassiti}\index{Parassiti}

Parassiti come \emph{cercaorecchie} o \emph{larve necrofaghe} provocano parassitosi, un tipo particolare di Malattia. Le parassitosi possono essere guarite solo attraverso trattamenti specifici; indipendentemente da quanti Tiri Salvezza si effettuano, la parassitosi continua ad affliggere il bersaglio. Anche se un Rimuovi Malattia (o un effetto simile) uccide immediatamente una parassitosi, l'immunità alle Malattie non offre protezione, dato che è causata da parassiti.

\medskip
\noindent\emph{Cercaorecchie (grado di Sfida 5)}\index{Cercaorecchie}

I cercaorecchie sono minuscoli vermi bianchi che vivono nel legno marcio o altri detriti organici. Si possono notare con una prova di Consapevolezza (DC 15). Altrimenti, una creatura vivente che frughi nella loro tana, si trasferisce inavvertitamente addosso uno o più cercaorecchie, i quali poi cercano una zona calda sul corpo della creatura, prediligendo il condotto uditivo, e li depongono 2d8 uova prima di morire.

Le uova si schiudono 4d6 ore dopo e le larve divorano la carne intorno. Alla morte del loro ospite, i vermetti strisciano fuori e ne cercano uno nuovo.

Rimuovi Malattia uccide tutti i cercaorecchie o le uova non ancora schiuse su un ospite. Alcuni cercaorecchie preferiscono vivere nel legno corrotto, spesso nascondendosi nelle porte dei sotterranei. I piccoli fori lasciati da questa variante sono molto difficili da notare (Consapevolezza DC 20).

\medskip
\textbf{Cercaorecchie}

Tipo: Parassitosi

TS: Tempra DC 15

Insorgenza: 4d6 ore

Frequenza Tiro Salvezza: 1 ogni ora

Effetti: 1d3 a Costituzione se fallisce il Tiro Salvezza

\medskip
\noindent\emph{Larve Necrofaghe (grado di Sfida 4)}\index{Larve Necrofaghe}

Una volta occupato un corpo vivente, le larve scavano verso il cuore, il cervello e altri organi interni chiave dell'ospite, provocandone infine la morte.

Nel primo round di parassitosi, applicando del fuoco nel foro di ingresso si possono uccidere le larve e salvare l'ospite, ma questo subisce 1d6 danni da fuoco.

Anche estrarle funziona, ma più a lungo le larve restano nell'ospite, più danni provoca questo metodo. Per estrarre le larve occorre un'arma tagliente ed una prova di Pronto Soccorso con DC 20, infliggendo 1d6 danni per ogni round che l'ospite è stato afflitto da parassitosi. Se la prova di Pronto Soccorso riesce una larva viene rimossa. Rimuovi Malattia uccide tutte le larve necrofaghe presenti in un ospite.

\medskip
\textbf{Larve Necrofaghe}

Tipo: Parassitosi

TS: Tempra DC 17

Insorgenza: immediata

Frequenza: 1/round

Effetti: 1 danno a Costituzione per larva

\medskip
\textbf{Cristalli magici (grado di Sfida 3)}\index{Cristalli magici}

I cristalli magici sono grandi (3-12 metri d'altezza) grappoli di cristalli di quarzo viola che irradiano un'aura di alterazione forte. Per identificarli occorre una prova di Arcana con DC 25.

I cristalli magici accumulano energia magica per crescere e difendersi. Un cristallo magici assorbe gli incantesimi lanciati nei 3 metri intorno a lui. L'incantatore deve effettuare una Prova di Magia con un Successo Critico per evitare l'effetto.

Danneggiando o rompendo i cristalli le magie assorbite vengono espulsi con un'esplosione di energia magica che infligge 1d4 danni per Punti Magia assorbiti (solitamente 10d4) a tutti coloro che si trovano entro 6 metri di raggio.

I cristalli magici sono molto fragili (Durezza 0, 4 Punti Ferita).
In aree ricche di cristalli, le creature che vi passano attraverso devono superare una prova di Acrobatica con DC 10 per evitare di camminarci sopra o sfiorarli rompendoli.

\medskip
\textbf{Magnete (grado di Sfida 2)}\index{Magnete}

Le strane energie del mondo sotterraneo possono caricare pietre e vene di minerali con potenti campi magnetici, creando un pericolo per chi porta o indossa metalli ferrosi. Tutti gli oggetti di ferro o acciaio portate entro raggio di 3 metri dal minerale sono trascinate verso di esso.

%\medskip

%\begin{center}
%\includegraphics[width=0.65\linewidth]{immagini/neodimio.png}

%\emph{Neodimio}
%\end{center}

Ogni creatura che abbia più di 4 di ingombro in metallo viene inesorabilmente attirato verso il minerale magnetico. E' concesso un Tiro Salvezza su Tempra con modificatore Forza a DC 25 per non avvicinarsi o riuscire a staccarsi dalla grossa calamita.

\medskip
\textbf{Pozzo Maledetto (grado di Sfida 3)}\index{Pozzo Maledetto}

Un pozzo maledetto attira gli avventurieri nelle sue profondità attraverso un illusione (TS su Volontà con DC 16 per non crederci) di uno meraviglioso tesoro sul fondo profondo solo 3 metri. Qualsiasi creatura che giunga al \emph{tesoro} attiva la maledizione.

Una creatura all'interno del pozzo deve superare un Tiro Salvezza su Volontà con DC 18 o è colpita dalla maledizione, che distorce la sua percezione del pozzo. L'acqua sembra addensarsi in un viscosa melma che spinge la creatura verso il fondo a 12 metri.

E richiesta una prova di Nuotare a DC 16 ogni round, il fallimento indica che si incomincia ad affogare.

Un pozzo maledetto irradia una forte magia, e può essere distrutta da Dissolvi Magie o da Rimuovi Maledizione.

\medskip
\textbf{Quercia Velenosa (grado di Sfida 1 o 3)}\index{Quercia Velenosa}

Il contatto con una quercia velenosa (grado di Sfida 1) causa una dolorosa, 1d4 Punti Ferita di danno, eruzione cutanea che rende la vittima Affaticata finché i danni non guariscono. Un pieno contatto col corpo o l'inalazione del fumo di una quercia velenosa che brucia potrebbero essere fatali (grado di Sfida 3) causando 2 gradi di Affaticato ed 1d8 di danno.
Una prova di Natura (o Erboristeria) con DC 15 rivela i pericoli insiti nella pianta. Questo pericolo può essere usato anche per piante nocive simili (edera velenosa, sommaco velenoso od ortiche pungenti..)

\textbf{Quercia Velenosa}

Tipo: Veleno, contatto

TS: Tempra DC 13

Insorgenza: 1 ora

Effetti: 1d4 danni, la creatura è affaticata finché i danni non guariscono

Cura: 1 TS


\subsection{Prepararsi per il riposo}\index{Prepararsi per il riposo}\index{Dormire}\index{Turni di guardia}

Ogni avventuriero deve riposarsi ogni tanto, lo deve fare con attenzione e stando attento a non incorrere in brutte e pericolose sorprese.

Ogni volta che un personaggio termina un periodo di 24 ore senza dormire almeno 8 ore, deve superare un Tiro Salvezza su Tempra con DC 17, altrimenti diventa Affaticato.

Ogni riposo mancato ulteriore lo renderà ancora più Affaticato cumulando le penalità relative. Se il personaggio resta sveglio per più giorni, lottare contro il sonno diventa più difficile. Dopo le prime 24 ore, la DC aumenta di 4 per ogni periodo consecutivo di 24 ore trascorso senza aver dormito 8 ore. La DC torna a 17 quando il personaggio completa un riposo di almeno 8 ore.

Dormire in armatura media o pesanti rende Affaticati, tranne se hai l'Abilità \hyperlink{secondapelle}{Seconda pelle}.

Non si riesce a dormire le 8 ore ad intervalli minori di 16 ore.\index{Dormire più volte al giorno}

Se il personaggio viene svegliato e coinvolto in una attività impegnativa come combattere, lanciare incantesimi, cavalcare... se questa si protrae per più di 10 minuti obbliga il personaggio a riprendere completamente il riposo.

\subsubsection{Organizzare i Turni di Guardia}

Se il gruppo è numeroso i turni di guardia per vegliare e controllare l'ambiente diventano più corti.

\medskip{}

\textbf{Tabella: Durata turni di guardia}\index[Tabelle]{Tabella Durata turni di guardia}

In questa tabella vengono indicate la durata dei turni di guardia ed il tempo totale di riposo del gruppo, nell'ipotesi di riposare almeno 8 ore.

\medskip{}

\noindent\begin{tabularx}{0.5\textwidth}{XXX}
\textbf{Membri} &\textbf{Durata}&\textbf{Durata}\\
\textbf{del gruppo}&\textbf{del Turno}&\textbf{Totale}\\
2& 8 h& 16 h\\
3& 4 h & 12 h\\
4& 2 h e 30 min. & 10 h e 30 min.\\
5& 2 h& 10 h\\
6& 1 h e 30 min. & 9 h e 30 min.
\end{tabularx}

\medskip{}

Un \textbf{rumore brusco} concede una prova di Consapevolezza a DC 15, oppure pari alla prova di Furtività +8 dell'avversario, per svegliarsi.\index{Svegliarsi per rumore}\index{Rumore nella notte}

\end{multicols}

\vfill

\begin{center}
\includegraphics[width=0.9\linewidth]{immagini/mappaparigi.png}

\emph{Antica mappa di Parigi}
\end{center}

\pagebreak

\subsection{Avventure e Trappole}\index{Trappole}\label{trappole}


\begin{changemargin}{0.3cm}{0.3cm}\begin{enfasi}{
Chi pone la trappola sempre allo stesso posto non prenderà alcun'iguana. (Proverbio Africano)}\end{enfasi}\end{changemargin}\medskip


\begin{multicols}{2}

Quasi ovunque si può incontrare una trappola. Le trappole possono essere di natura magica o meccanica. Le trappole meccaniche comprendono fosse, frecce, massi che cadono, stanze piene d'acqua, lame rotanti e qualsiasi altra cosa che dipenda da un meccanismo per operare. Le trappole magiche sono congegni magici trappola o incantesimi trappola. I congegni magici trappola quando attivati generano gli effetti di un incantesimo, mentre gli incantesimi trappola sono incantesimi come glifo di interdizione e simbolo che funzionano come trappole.

Quando gli avventurieri si imbattono in una trappola, dovreste sapere come la trappola si attiva e cosa faccia, oltre ad avere un'idea di come i personaggi possano individuare la trappola e riuscire a disarmarla o evitarla.

\subsubsection{Attivare una Trappola}
La maggior parte delle trappole si attivano quando una creatura giunge in un punto o tocca qualcosa che il creatore della trappola voleva proteggere. Normali sistemi di attivazione sono pedane a pressione o false sezioni di pavimento, tirare un cavo, girare una maniglia e usare la chiave sbagliata nella serratura. Le trappole magiche spesso si attivano quando una creatura entra in un'area o tocca un oggetto. 

Alcune trappole magiche (come l'incantesimo glifo di interdizione) possiedono delle condizioni di attivazione più complesse, tra cui l'impiego di parole d'ordine per impedire l'attivazione della trappola.

Una volta attivata la trappola esegue l'effetto indicato: il personaggio cade nella fossa, parte il dardo avvelenato, viene attivato l'incantesimo collegato. Al personaggio che subisce l'attivazione della trappola è concesso un Tiro Salvezza o ulteriore prova solo se specificato nella descrizione della trappola stessa.

\subsubsection{Individuare e Disabilitare una Trappola}
Di solito, alcuni elementi di una trappola sono ben visibili a un'attenta ispezione.

La descrizione della trappola specifica le prove e le DC necessarie per individuarla, disabilitarla o entrambe. Un personaggio che cerchi attivamente una trappola può tentare una prova di \textbf{Sopravvivenza} contro la DC della trappola.

Il Narratore può anche comparare la DC per individuare la trappola contro il punteggio di Sopravvivenza (a tiro dadi 8) dei personaggi al fine di determinare se un membro del gruppo noti la trappola. Se gli avventurieri notano la trappola prima di attivarla, potrebbero tentare di disarmarla, in maniera permanente o abbastanza a lungo da permettergli il passaggio.

Il Narratore potrebbe richiedere una prova di Disattivare Congegni. Se non si hanno \textbf{attrezzi da scasso}\index{Attrezzi da scasso} o adeguati la prova la fai con un -1d6 di penalità. \index{Disattivare congegni senza attrezzi}Può essere usata anche la competenza Sopravvivenza seppure con un -1d6 per disattivare una trappola, lucchetto..., in questo caso la durata dell'operazione è pari ad 1 Azione per DC della trappola.

Se si vuole disattivare temporaneamente\index{Trappola disattivare temporaneamente} una trappola aggiungete 6 alla difficoltà. Questo disattiverà la trappola per 2d4 minuti.

Una \textbf{trappola magica può essere disattivata} con una prova di Disattivare Congegni purché il valore di Arcana sia almeno 1/5 della DC della trappola in aggiunta a qualsiasi altra prova indicata nella descrizione della trappola. La prova di Arcana può essere fatta anche da chi non ha Disattivare Congegni ma con la difficoltà segnata nella trappola ed insieme a chi fa la prova di Disattivare Congegni. L'incantesimo Dissolvi Magie ha una probabilità di annullare la maggior parte delle trappole magiche.\index{Disattivare trapple magiche}

Se la prova per disattivare o disabilitare la trappola fallisce\index{Fallimento disattivare trappola} e si ottiene un fallimento critico la trappola scatta.

Di solito, la descrizione della trappola è abbastanza chiara da permettere al Narratore di giudicare se le azioni di un personaggio riescano a individuare o disattivare la trappola.

Usate il buon senso e basatevi sulla descrizione della trappola per determinare cosa accade. Nessun progetto di trappola può prevedere ogni possibile azione che i personaggi potrebbero tentare.

Il Narratore dovrebbe consentire a un personaggio di scoprire una trappola senza dover effettuare prove di competenza, se le sue azioni o la descrizione di ciò che fa rivelano chiaramente la presenza della trappola.

Disattivare le trappole può essere un pò più complicato. Prendiamo, ad esempio, un forziere protetto da una trappola. Se il forziere viene aperto senza tirare le due maniglie laterali, un meccanismo interno spara una raffica di aghi avvelenati verso chiunque si trovi di fronte.

Dopo aver ispezionato il forziere e fatto qualche prova, i personaggi non sono ancora sicuri che sia trappolato. Invece di aprirlo direttamente, puntano uno scudo davanti al forziere e lo aprono a distanza con un'asta di ferro. In questo caso, la trappola si attiva, ma la raffica di aghi colpisce lo scudo senza ferire nessuno.

Le trappole sono spesso progettate con meccanismi che permettono di disattivarle o aggirarle.


\medskip

\begin{center}
\includegraphics[width=0.7\linewidth]{immagini/medusa.png}
\end{center}

\subsubsection{Effetti delle Trappole}
Gli effetti delle trappole possono essere da semplici inconvenienti a letali. La descrizione di una trappola specifica cosa accade quando viene attivata.
Il bonus di attacco di una trappola, la DC del Tiro Salvezza per resistere ai suoi effetti, e il danno che infligge possono variare in base alla pericolosità della trappola.

Usare la tabella DC dei Tiri Salvezza e Bonus di Attacco delle Trappole e la tabella Gravità del Danno per Livello come suggerimenti sui tre livelli di gravità delle trappole.

\medskip

\textbf{Tabella: DC dei Tiri Salvezza e Bonus di Attacco delle Trappole}\index[Tabelle]{Tabella DC dei Tiri Salvezza e Bonus di Attacco delle Trappole}

\medskip

\noindent\begin{tabularx}{0.48\textwidth}{X|X|X}
Pericolosità della Trappola&DC Tiro Salvezza& Bonus di Attacco\\
\toprule
Minima&13-14&+4 a +6\\
Pericolosa&16-20&+8 a +10\\
Mortale&21-26&+12 a +15
\end{tabularx}

\medskip

\textbf{Tabella: Gravità del Danno per Livello}\index[Tabelle]{Tabella Gravità del Danno per Livello}

\medskip

\noindent\begin{tabularx}{0.48\textwidth}{X|X|X|X}
Livello PG&Minima&Pericolosa&Mortale\\
\toprule
1°-4°&1d10&2d10&4d10\\
5°-10°&2d10&4d10&10d10\\
11°-16°&4d10&10d10&18d10\\
17°-20°&10d10&18d10&24d10
\end{tabularx}

\medskip

\subsubsection{Trappole di Esempio}\hypertarget{trappoleesempio}{}\label{trappoleesempio}
\textbf{Ago Avvelenato}

Trappola meccanica

Un ago avvelenato è nascosto all'interno della serratura di un forziere, o altro oggetto che si possa aprire. Aprire il forziere senza la chiave adeguata farebbe scattare l'ago, che dispensa una dose di veleno.

Quando la trappola viene attivata, l'ago si estende per 7,5 centimetri dalla serratura. Una creatura a gittata subisce 1 danno perforante e 11 (2d10) danni da veleno, e deve superare un Tiro Salvezza su Tempra con DC 20 o prendere -1d6 al Tiro per Colpire e -1d6 alle prove di Competenza di Base per 1 ora.

Il personaggio che superi una prova di Sopravvivenza con DC 22, può dedurre la presenza della trappola dalle modifiche apportate alla serratura per ospitare l'ago. Una prova superata di Disattivare Congegni disarma la trappola rimuovendo l'ago dalla serratura. \textbf{Una prova fallita per scassinare la serratura fa scattare la trappola}\index{Trappola, fallire prova}. Dichiarare di incastrare un bastone nella serratura è altrettanto efficace nel disattivare la trappola.

\medskip

\textbf{Dardi Avvelenati}

Trappola meccanica

Quando una creatura calpesta una pedana a pressione nascosta, dei dardi avvelenati vengono sparati da un meccanismo a molla o da tubi pressurizzati astutamente nascosti all'interno delle pareti circostanti. Un'area potrebbe presentare più pedane a pressione, ciascuna collegata alla propria serie di dardi.

I minuscoli fori nelle pareti sono celati da polvere e ragnatele, oppure astutamente celati tra i bassorilievi, murali o affreschi che adornano la stanza. La DC della prova per notarli (Sopravvivenza) è 18.

Il personaggio che superi una prova di Sopravvivenza con DC 18, può dedurre la presenza della pedana a pressione nascosta dalle differenze nella pavimentazione di cui è composta rispetto al resto del pavimento.

Incuneare una punta di ferro o altro oggetto sotto la pedana a pressione previene l'attivazione della trappola. Riempire i fori di tessuto o cera impedisce la fuoriuscita dei dardi contenuti all'interno.

La trappola si attiva quando più di 10 chili di peso vengono posti sulla pedana a pressione, facendo così sparare quattro dardi. Ogni dardo effettua un attacco a distanza con un bonus di attacco +10 contro un bersaglio casuale entro 3 metri dalla pedana a pressione (la visuale non ha alcun impatto su questo tiro di attacco).

Se non ci sono bersagli nell'area, il dardo non colpisce nulla. Un bersaglio colpito subisce 2 (1d4) danni perforanti e deve effettuare un Tiro Salvezza su Tempra con DC 18 e subire 11 (2d10) danni da veleno se lo fallisce, o la metà di questi danni se lo supera.


\medskip

\textbf{Fosse}

Trappola meccanica

Presentiamo di seguito quattro tipi base di fosse.

\medskip

\emph{Fossa Semplice}

La fossa semplice è un buco scavato nel terreno. Il buco è coperto da un grosso tessuto ancorato ai margini della fossa e mimetizzato con terra e detriti.
La DC per notare la fossa è 14. Chiunque metta piede sul tessuto cade all'interno della buca e si tira dietro il tessuto, subendo danni in base alla profondità della fossa (di solito 3 metri, ma alcune fosse sono più profonde).

\medskip

\emph{Fossa Nascosta}

Questa fossa possiede una copertura fatta di materiale identico a quello del pavimento circostante.
Superando una prova di Consapevolezza con DC 18 si nota l'assenza di tracce nella sezione di pavimento che forma la copertura della fossa.

È necessario superare una prova di Sopravvivenza con DC 18 per confermare che quella sezione di pavimento copra in realtà una fossa.

Quando una creatura mette piede sulla copertura, questa si spalanca come una botola, facendo precipitare l'intruso nella fossa sottostante. La fossa è profonda solitamente tra i 3 e i 6 metri, ma può esserlo anche di più.

Una volta che la fossa è stata individuata, uno spuntone di ferro o simile oggetto può essere conficcato tra la copertura della fossa e il terreno circostante per impedire che la copertura si apra, rendendo sicuro il passaggio. La copertura può anche essere tenuta chiusa magicamente tramite l'incantesimo Serratura Magica o magie simili.

\medskip
\emph{Fossa a Scatto}

Questa fossa è identica alla trappola fossa nascosta, con un'eccezione fondamentale: la botola che copre la fossa nasconde un meccanismo a molla. Dopo che una creatura è caduta nella fossa, la copertura si richiude di scatto per intrappolare la vittima al suo interno.

È necessario superare un Tiro Salvezza Tempra con Forza DC 20 per aprire a forza la copertura. La copertura può essere anche distrutta. Un personaggio all'interno della fossa può anche tentare di disabilitare il meccanismo a molla dall'interno superando una prova di Disattivare Congegni con DC 18 purché possa raggiungere e vedere il meccanismo in questione. In alcuni casi, un altro meccanismo fa riaprire la fossa.

\medskip

\emph{Fossa con Spuntoni}

La fossa è una fossa semplice, nascosta o a scatto, sul cui fondo si trovano delle punte di legno o degli spuntoni di ferro. Una creatura che caschi nella fossa subisce 11 (2d10) danni perforanti dagli spuntoni, oltre al danno da caduta.

Versioni più crudeli di questa trappola sono munite di veleno cosparso sulle punte collocate in fondo alla fossa. In quel caso, chiunque subisca danni perforanti dagli spuntoni deve anche effettuare un Tiro Salvezza su Tempra con DC 16 e subire 22 (4d10) danni da veleno se lo fallisce, o la metà di questi danni se lo supera.


\medskip

\textbf{Rete che Casca}

Trappola meccanica

Questa trappola usa un cavetto per liberare una rete appesa al soffitto.

Il cavetto è collocato a 7 centimetri dal terreno e si estende tra due colonne o alberi. La rete è nascosta da ragnatele o fogliame. La DC (Sopravvivenza) per notare il cavetto e la rete è 15. Una prova superata di Disattivare Congegni con DC 20 disabilita il cavetto.

Un personaggio privo degli attrezzi da scasso può tentare comunque la prova con -1d6 usando un'arma o un attrezzo affilati. Se la prova fallisce, la trappola si attiva.

Quando la trappola viene attivata, la rete viene rilasciata coprendo un'area quadrata di 3 metri di lato. Tutte le creature nell'area vengono intrappolate dalla rete e sono intralciate, mentre quelle che falliscono un Tiro Salvezza su Tempra, con modificatore Forza, con DC 13 cadono anche prone.

Una creatura può usare 2 Azioni per effettuare un Tiro Salvezza Tempra con Forza DC 13, liberando se stessa o un'altra creatura a portata se la supera.

La rete ha Difesa 10 e 20 Punti Ferita. infliggere 5 danni taglienti alla rete ne distrugge una sezione quadrata di 1 metro di lato, liberando qualsiasi creatura intrappolata in quella sezione.

\medskip

\textbf{Sfera Rotolante}

Trappola meccanica

Quando 10 o più chili vengono posti sulla pedana a pressione della trappola, una botola nascosta nel soffitto si apre, rilasciando una sfera di 3 metri di diametro interamente fatta di pietra.

Superando una prova di Sopravvivenza con DC 20 un personaggio può notare la botola e la pedana a pressione. Se un esame del pavimento è accompagnato da una prova superata di Sopravvivenza con DC 20, rivelerà la presenza della pedana a pressione tramite la differenza di struttura della pavimentazione che la accomoda. La stessa prova effettuata mentre si controlla il soffitto, rivelerà la presenza di una botola. Incuneare uno spuntone di ferro o un altro oggetto sotto la pedana a pressione impedirà l'attivazione della trappola.

L'attivazione della sfera fa sì che tutte le creature presenti tirino per l'iniziativa. La sfera tira l'iniziativa con un bonus di +8.

Durante il suo round, la sfera si muove di 18 metri in linea retta. La sfera può muoversi attraverso lo spazio di una creatura, e le creature possono muoversi attraverso lo spazio che occupa, considerandolo terreno difficile.

Ogni qualvolta la sfera entri nello spazio di una creatura o una creatura entri nel suo spazio mentre la sfera sta rotolando, la creatura deve superare un Tiro Salvezza su Riflessi con DC 15 o subire 55 (10d10) danni contundenti e cadere prona.

La sfera si ferma quando colpisce un muro o una barriera simile. Non può girare gli angoli, ma gli abili costruttori di sotterranei incorporano lievi curve e svolte curvilinee nei passaggi limitrofi che permettono alla sfera di continuare a muoversi.

Con 2 Azioni, una creatura entro 1 metro dalla sfera può tentare di rallentarla superando un Tiro Salvezza Tempra con Forza DC 20. Se la prova viene superata, la velocità della sfera viene ridotta di 3 metri. Se la velocità della sfera scende a 0, arresta il movimento e non è più una minaccia.

\medskip

\textbf{Soffitto che Crolla}

Trappola meccanica

Questa trappola usa un cavetto per fare crollare i sostegni che sorreggono una sezione instabile di soffitto.

Il cavetto è collocato a 7 centimetri dal terreno e si estende tra i due sostegni. La DC (Sopravvivenza) per notare il cavetto è 13. Una prova superata di Disattivare Congegni con DC 20 disabilita il cavetto.

Un personaggio privo degli attrezzi da scasso può tentare comunque la prova con -1d6 usando un'arma o un attrezzo affilati. Se la prova fallisce, la trappola si attiva.

Chiunque ispezioni i sostegni può facilmente dedurre che sono solo appoggiati. Con un'Azione, il personaggio può far cadere un sostegno e attivare la trappola.

Il soffitto sopra il cavetto è in cattivo stato, e chiunque possa vederlo può capire che rischia di crollare. Quando la trappola viene attivata, il soffitto instabile crolla. Tutte le creature nell'area sotto la sezione instabile devono effettuare un Tiro Salvezza su Riflessi con DC 20, subendo 22 (4d10) danni contundenti se lo falliscono o la metà di questi danni se lo superano. Una volta attivata la trappola, il pavimento dell'area è pieno di macerie e diventa terreno difficile.


\medskip

\textbf{Statua Soffia Fuoco}

Trappola magica

Questa trappola si attiva quando un intruso calpesta una pedana a pressione nascosta, liberando una vampata di fiamme magiche da una statua vicina.

La DC (Sopravvivenza) per notare la pedana a pressione o segni di bruciature sul pavimento e le pareti è 20. Un incantesimo o altro effetto che può percepire la presenza di magia, come individuazione del magico, rivela un'aura magica di invocazione intorno alla statua.

La trappola si attiva quando più di 10 chili di peso vengono posti sulla pedana a pressione, facendo sì che dalla statua scaturisca un cono di fuoco di 9 metri. Tutte le creature nel cono devono effettuare un Tiro Salvezza su Riflessi con DC 17, subendo 22 (4d10) danni da fuoco se lo falliscono o la metà di questi danni se lo superano.

Infilare uno spuntone di ferro o altro oggetto sotto la pedana a pressione impedisce alla trappola di attivarsi. Una prova di Disattivare Congegni a DC 20 (ed è necessario avere 3 in Arcana) disattiva la trappola. Un dissolvi magie (DC 17) lanciato sulla statua distrugge la trappola.


\medskip

\textbf{Trappole ad incantesimo e Dissolvi Magia}\index{Trappole ad incantesimo e Dissolvi Magia}

Le trappole di cui sopra possono essere dotate di un incantesimo che si attiva con la trappola.
I Tiri Salvezza per resistere all'incantesimo sono i medesimi dell'incantesimo lanciato da oggetto o come indicato nella descrizione della trappola.

Un Dissolvi Magia cancella l'incantesimo sulla trappola se questa ha Grado di Sfida 2 o meno e ne disabilita l'effetto magico per 10 minuti se di Grado di Sfida 3.
Un Dissolvi Magie Avanzato cancella l'incantesimo sulla trappola se questa ha GS 4 o meno e ne disabilita l'effetto magico per 10 minuti se di GS 5. In caso di Critico Magico nel lancio dell'incantesimo si agisce su un grado maggiore di trappola.


\subsubsection{Altri esempi di trappole}

Sono qui presentate ulteriori trappole per la vostra gioia.


\medskip

\textbf{Piccola legenda}:

\textbf{Grado di Sfida (GS)}: indica quale è il grado di sfida della trappola

\textbf{Tipo}: se la trappola è di tipo Meccanico (Mec.) o Magico (Mag.)

\textbf{DC Sopravvivenza (SOP)}: quale è la prova e difficoltà per rivelare la trappola

\textbf{DC Disattivare Congegni (DIS)}: quale è la prova e difficoltà per disattivare la trappola. Il punteggio dopo la sbarra (es DC 26/6) indica il requisito minimo di conoscenza Arcana per disattivarla.

\textbf{Attivatore}: se si attiva a contatto o distanza o tramite un incantesimo come Allarme (per GS < 3), Occhio arcano (per GS tra 3 e 10) o Visione del Vero (per GS oltre 10).

\textbf{Ripristino (Ripr.)}: se è possibile ripristinare la trappola una volta scattata

\textbf{Effetto}: quale è l'effetto della trappola

\bigskip

\textbf{Dardo Avvelenato}

\begin{tabularx}{0.48\textwidth}{lX}
	\textbf{GS:} & 1 \\
	\textbf{Tipo:} & Meccanico \\
	\textbf{Sopravviv.:} & DC 13 \\
	\textbf{Dis. Cong.:} & DC 15 \\
	\textbf{Attivatore:} & Contatto \\
	\textbf{Ripristino:} & Nessuno \\
	\textbf{Effetto:} & Attacco a distanza 12 metri +10 (1d3 di danno più \hyperlink{bavadilucos}{Bava fermentata di Lucos}, pag \pageref{bavadilucos}) 
\end{tabularx}\\

\textbf{Freccia}

\begin{tabularx}{0.48\textwidth}{lX}
	\textbf{GS:} & 1 \\
	\textbf{Tipo:} & Meccanico \\
	\textbf{Sopravviv.:} & DC 13 \\
	\textbf{Dis. Cong.:} & DC 15 \\
	\textbf{Attivatore:} & Contatto \\
	\textbf{Ripristino:} & Nessuno \\
	\textbf{Effetto:} & Attacco a distanza 12 metri +15 (1d8+3) 
\end{tabularx}\\

\textbf{Fossa}

\begin{tabularx}{0.48\textwidth}{lX}
	\textbf{GS:} & 1 \\
	\textbf{Tipo:} & meccanico \\
	\textbf{Sopravviv.:} & DC 14 \\
	\textbf{Dis. Cong.:} & DC 16 \\
	\textbf{Attivatore:} & posizione \\
	\textbf{Ripristino:} & manuale \\
	\textbf{Effetto:} & fossa profonda 3 metri (2d6 danni da caduta) 
\end{tabularx}\\

%\textbf{Trancia Dita}

%\begin{tabularx}{0.48\textwidth}{lX}
%	\textbf{GS:} & 1 \\
%	\textbf{Tipo:} & meccanico \\
%	\textbf{Sopravviv.:} & DC 17 \\
%	\textbf{Dis. Cong.:} & DC 14 \\
%	\textbf{Attivatore:} & posizione \\
%	\textbf{Ripristino:} & manuale \\
%	\textbf{Effetto:} & trancia la prima falange (1d8+1) 
%\end{tabularx}\\

\textbf{Lama Falciante}

\begin{tabularx}{0.48\textwidth}{lX}
	\textbf{GS:} & 1 \\
	\textbf{Tipo:} & meccanico \\
	\textbf{Sopravviv.:} & DC 13 \\
	\textbf{Dis. Cong.:} & DC 15 \\
	\textbf{Attivatore:} & posizione \\
	\textbf{Ripristino:} & manuale \\
	\textbf{Effetto:} & 2 attacchi in mischia +10 (1d8+1) 
\end{tabularx}\\

\textbf{Fossa con Spuntoni}

\begin{tabularx}{0.48\textwidth}{lX}
	\textbf{GS:} & 2 \\
	\textbf{Tipo:} & meccanico \\
	\textbf{Sopravviv.:} & DC 16 \\
	\textbf{Dis. Cong.:} & DC 18 \\
	\textbf{Attivatore:} & posizione \\
	\textbf{Ripristino:} & manuale \\
	\textbf{Effetto:} & fossa profonda 3 metri + spuntoni (Attacco in mischia +10, 1d4 spuntoni per bersaglio per 1d4+2 danni ciascuno) 
\end{tabularx}\\

\textbf{Onda rovente}

\begin{tabularx}{0.48\textwidth}{lX}
	\textbf{GS:} & 2 \\
	\textbf{Tipo:} & magico \\
	\textbf{Sopravviv.:} & DC 16 \\
	\textbf{Dis. Cong.:} & DC 18/4 \\
	\textbf{Attivatore:} & prossimità (Allarme) \\
	\textbf{Ripristino:} & nessuno \\
	\textbf{Effetto:} & 5d4 danni da fuoco 
\end{tabularx}\\

\textbf{Giavellotto}

\begin{tabularx}{0.48\textwidth}{lX}
	\textbf{GS:} & 2 \\
	\textbf{Tipo:} & meccanico \\
	\textbf{Sopravviv.:} & DC 15 \\
	\textbf{Dis. Cong.:} & DC 17 \\
	\textbf{Attivatore:} & posizione \\
	\textbf{Ripristino:} & nessuno \\
	\textbf{Effetto:} & Attacco a distanza 12 metri +15 (2d6+6), entro raggio 6 metri 
\end{tabularx}\\

\textbf{Fossa con non morti}

\begin{tabularx}{0.48\textwidth}{lX}
	\textbf{GS:} & 2 \\
	\textbf{Tipo:} & meccanico \\
	\textbf{Sopravviv.:} & DC 14 \\
	\textbf{Dis. Cong.:} & DC 16 \\
	\textbf{Attivatore:} & posizione \\
	\textbf{Ripristino:} & nessuno \\
	\textbf{Effetto:} & Sul fondo della fossa (2d6 di danno da caduta) sono presenti due zombi 
\end{tabularx}\\

\textbf{Freccia Acida}

\begin{tabularx}{0.48\textwidth}{lX}
	\textbf{GS:} & 3 \\
	\textbf{Tipo:} & magico \\
	\textbf{Sopravviv.:} & DC 18 \\
	\textbf{Dis. Cong.:} & DC 20/4 \\
	\textbf{Attivatore:} & prossimità (Allarme) \\
	\textbf{Ripristino:} & nessuno \\
	\textbf{Effetto:} & Attacco distanza di 16 metri (4d4 danni da acido per 4 round) 
\end{tabularx}\\

\textbf{Fossa Celata}

\begin{tabularx}{0.48\textwidth}{lX}
	\textbf{GS:} & 3 \\
	\textbf{Tipo:} & meccanico \\
	\textbf{Sopravviv.:} & DC 19 \\
	\textbf{Dis. Cong.:} & DC 20 \\
	\textbf{Attivatore:} & posizione \\
	\textbf{Ripristino:} & manuale \\
	\textbf{Effetto:} & 3d6 danni da caduta
\end{tabularx}\\

\textbf{Arco Elettrico}

\begin{tabularx}{0.48\textwidth}{lX}
	\textbf{GS:} & 4 \\
	\textbf{Tipo:} & magico \\
	\textbf{Sopravviv.:} & DC 21 \\
	\textbf{Dis. Cong.:} & DC 20/4 \\
	\textbf{Attivatore:} & contatto \\
	\textbf{Ripristino:} & nessuno \\
	\textbf{Effetto:} & Arco elettrico. 2 bersagli entro 3 metri tra loro, 5d6 danni da elettricità 
\end{tabularx}\\

\textbf{Falce a Parete}

\begin{tabularx}{0.48\textwidth}{lX}
	\textbf{GS:} & 4 \\
	\textbf{Tipo:} & meccanico \\
	\textbf{Sopravviv.:} & DC 20 \\
	\textbf{Dis. Cong.:} & DC 18 \\
	\textbf{Attivatore:} & posizione \\
	\textbf{Ripristino:} & automatico \\
	\textbf{Effetto:} & tre attacchi in mischia +10 (2d8+3) 
\end{tabularx}\\

\textbf{Blocco in Caduta}

\begin{tabularx}{0.48\textwidth}{lX}
	\textbf{GS:} & 5 \\
	\textbf{Tipo:} & meccanico \\
	\textbf{Sopravviv.:} & DC 23 \\
	\textbf{Dis. Cong.:} & DC 22 \\
	\textbf{Attivatore:} & posizione \\
	\textbf{Ripristino:} & manuale \\
	\textbf{Effetto:} & Attacco in mischia +15 (6d6) 
\end{tabularx}\\

\textbf{Colpo Infuocato}

\begin{tabularx}{0.48\textwidth}{lX}
	\textbf{GS:} & 6 \\
	\textbf{Tipo:} & magico \\
	\textbf{Sopravviv.:} & DC 25 \\
	\textbf{Dis. Cong.:} & DC 24/4 \\
	\textbf{Attivatore:} & prossimità (Allarme) \\
	\textbf{Ripristino:} & nessuno \\
	\textbf{Effetto:} & cono 3 metri, 8d6 danni da fuoco
\end{tabularx}\\

%\textbf{Freccia Avvelenata}
%
%\begin{tabularx}{0.48\textwidth}{lX}
%	\textbf{GS:} & 6 \\
%	\textbf{Tipo:} & meccanico \\
%	\textbf{Sopravviv.:} & DC 25 \\
%	\textbf{Dis. Cong.:} & DC 20 \\
%	\textbf{Attivatore:} & posizione \\
%	\textbf{Ripristino:} & nessuno \\
%	\textbf{Effetto:} & 6 frecce, attacco a distanza 18 metri +10 (1d6 più 1d6 Veleno) 
%\end{tabularx}\\

\textbf{Acque bollenti}

\begin{tabularx}{0.48\textwidth}{lX}
	\textbf{GS:} & 7 \\
	\textbf{Tipo:} & meccanico \\
	\textbf{Sopravviv.:} & DC 25 \\
	\textbf{Dis. Cong.:} & DC 20/4 \\
	\textbf{Attivatore:} & posizione \\
	\textbf{Ripristino:} & nessuno \\
	\textbf{Effetto:} & cono di 6 metri (spruzzo di acqua bollente, 8d6 danni da fuoco). 
\end{tabularx}\\

\textbf{Trappola a Gas}

\begin{tabularx}{0.48\textwidth}{lX}
	\textbf{GS:} & 8 \\
	\textbf{Tipo:} & meccanico \\
	\textbf{Sopravviv.:} & DC 28 \\
	\textbf{Dis. Cong.:} & DC 26 \\
	\textbf{Attivatore:} & posizione \\
	\textbf{Ripristino:} & riparabile \\
	\textbf{Effetto:} & Gas velenoso, cubo 6 metri di spigolo. TS Tempra DC 17 oppure rallentati 1/1 minuto.
\end{tabularx}\\

\textbf{Raffica di Frecce}

\begin{tabularx}{0.48\textwidth}{lX}
	\textbf{GS:} & 9 \\
	\textbf{Tipo:} & meccanico \\
	\textbf{Sopravviv.:} & DC 30 \\
	\textbf{Dis. Cong.:} & DC 27 \\
	\textbf{Attivatore:} & visivo ( Occhio Arcano) \\
	\textbf{Ripristino:} & riparabile \\
	\textbf{Effetto:} & 6 frecce. Attacco a distanza +14 (1d8+1) 
\end{tabularx}\\

\textbf{Fossa Celata con Spuntoni}

\begin{tabularx}{0.48\textwidth}{lX}
	\textbf{GS:} & 8 \\
	\textbf{Tipo:} & meccanico \\
	\textbf{Sopravviv.:} & DC 29 \\
	\textbf{Dis. Cong.:} & DC 20 \\
	\textbf{Attivatore:} & posizione \\
	\textbf{Ripristino:} & manuale \\
	\textbf{Effetto:} & Fossa profonda 15 m + spuntoni (Attacco in mischia +9, 3 spuntoni per bersaglio, 1d6+5 danni ciascuno) 
\end{tabularx}\\

\textbf{Pavimento Elettrico}

\begin{tabularx}{0.48\textwidth}{lX}
	\textbf{GS:} & 9 \\
	\textbf{Tipo:} & magico \\
	\textbf{Sopravviv.:} & DC 30 \\
	\textbf{Dis. Cong.:} & DC 26/5 \\
	\textbf{Attivatore:} & prossimità (Allarme) \\
	\textbf{Ripristino:} & nessuno \\
	\textbf{Effetto:} & 6mx9m. 8d6 danni da Elettricità.
\end{tabularx}\\

\textbf{Risucchio di Energia}

\begin{tabularx}{0.48\textwidth}{lX}
	\textbf{GS:} & 10 \\
	\textbf{Tipo:} & magico \\
	\textbf{Sopravviv.:} & DC 34 \\
	\textbf{Dis. Cong.:} & DC 34/6 \\
	\textbf{Attivatore:} & visivo (Visione del Vero) \\
	\textbf{Ripristino:} & nessuno \\
	\textbf{Effetto:} & Attacco di contatto a distanza 18 metri +14 da Vuoto, Punti Ferita max calano di 10d4 + Affaticato.
\end{tabularx}\\

\textbf{Stanza delle Lame}

\begin{tabularx}{0.48\textwidth}{lX}
	\textbf{GS:} & 10 \\
	\textbf{Tipo:} & meccanico \\
	\textbf{Sopravviv.:} & DC 25 \\
	\textbf{Dis. Cong.:} & DC 20 \\
	\textbf{Attivatore:} & posizione \\
	\textbf{Ripristino:} & riparabile \\
	\textbf{Effetto:} & Attacco in mischia +15 (a tutti tre attacchi 3d8+3) 
\end{tabularx}\\

\textbf{Cono di Ghiaccio}

\begin{tabularx}{0.48\textwidth}{lX}
	\textbf{GS:} & 11 \\
	\textbf{Tipo:} & magico \\
	\textbf{Sopravviv.:} & DC 30 \\
	\textbf{Dis. Cong.:} & DC 30/6 \\
	\textbf{Attivatore:} & prossimità (Allarme) \\
	\textbf{Ripristino:} & nessuno \\
	\textbf{Effetto:} & come \hyperlink{Cono di Freddo}{Cono di Freddo} da 9d6 danni. TS Riflessi DC 22 per dimezzare
\end{tabularx}\\


\textbf{Fossa Avvelenata}

\begin{tabularx}{0.48\textwidth}{lX}
	\textbf{GS:} & 11 \\
	\textbf{Tipo:} & meccanico \\
	\textbf{Sopravviv.:} & DC 25 \\
	\textbf{Dis. Cong.:} & DC 20 \\
	\textbf{Attivatore:} & posizione \\
	\textbf{Ripristino:} & manuale \\
	\textbf{Effetto:} & Fossa 6mx3m, 15 m profonda + spuntoni (3 attacchi in mischia +15 per bersaglio. 1d6+5 danni + veleno 2d6 danni) 
\end{tabularx}\\

\textbf{Galleria dei Fulmini}

\begin{tabularx}{0.48\textwidth}{lX}
	\textbf{GS:} & 13 \\
	\textbf{Tipo:} & magico \\
	\textbf{Sopravviv.:} & DC 29 \\
	\textbf{Dis. Cong.:} & DC 29/5 \\
	\textbf{Attivatore:} & prossimità (Allarme) \\
	\textbf{Ripristino:} & nessuno \\
	\textbf{Effetto:} & come incantesimo \hyperlink{Fulmine a catena}{Fulmine a catena}. DC 22.
\end{tabularx}\\


\textbf{Inferno di fuoco}

\begin{tabularx}{0.48\textwidth}{lX}
	\textbf{GS:} & 15 \\
	\textbf{Tipo:} & magico \\
	\textbf{Sopravviv.:} & DC 31 \\
	\textbf{Dis. Cong.:} & DC 31/6 \\
	\textbf{Attivatore:} & prossimità (Allarme) \\
	\textbf{Ripristino:} & nessuno \\
	\textbf{Effetto:} & 60 danni da fuoco. TS Riflessi DC 23 per dimezzare. 
\end{tabularx}\\

\textbf{Distruzione}

\begin{tabularx}{0.48\textwidth}{lX}
	\textbf{GS:} & 16 \\
	\textbf{Tipo:} & N/A \\
	\textbf{Sopravviv.:} & DC 34 \\
	\textbf{Dis. Cong.:} & DC 34/6 \\
	\textbf{Attivatore:} & prossimità (Allarme) \\
	\textbf{Ripristino:} & nessuno \\
	\textbf{Effetto:} & come incantesimo \hyperlink{Disintegrazione}{Disintegrazione} con 1 Critico Magico. DC 24
\end{tabularx}

\textbf{Pioggia di Meteore}

\begin{tabularx}{0.48\textwidth}{lX}
	\textbf{GS:} & 19 \\
	\textbf{Tipo:} & magico \\
	\textbf{Sopravviv.:} & DC 34 \\
	\textbf{Dis. Cong.:} & DC 34/6 \\
	\textbf{Attivatore:} & visivo \\
	\textbf{Ripristino:} & nessuno \\
	\textbf{Effetto:} & come incantesimo \hyperlink{Pioggia di Meteore}{Pioggia di Meteore}. DC 28
\end{tabularx}\\

\end{multicols}


\begin{changemargin}{0.3cm}{0.3cm}\begin{tcolorbox}[title = Tups e la trappola]{\small
In questo esempio vi porto l'approccio vecchia scuola quando si presumeva che ci fossero delle trappole. Nulla vieta al Narratore di permettere prove di Sopravvivenza o Disattivare Congegni. Posso solo dire che questo approccio è però più coinvolgente.

\bigskip

\emph{Narratore}: un corridoio largo 3 metri porta a nord, nell'oscurità.

\emph{Tups}: Avanziamo tastando il pavimento con la nostra pertica di 3 metri.

\emph{Narratore}: la pertica è stata lasciata incastrata nello scontro con l'idolo di pietra.
[\emph{Se avesse usato la pertica la trappola sarebbe stata scoperta facilmente}.]
Prosegui nel corridoio?

\emph{Tups}: No, sono sospettoso. Posso vedere qualche crepa nel pavimento, magari di forma quadrata ?

\emph{Narratore}: No, ci sono milioni di crepe, non riesci a vedere una fossa così chiaramente [\emph{il Narratore valuta che la fossa è ben mimetizzata e Tups ha scarsa illuminazione per vedere bene}]

\emph{Tups}: Ok, prendo la mia fiasca d'acqua dallo zaino. Vado a versare un pò d'acqua sul pavimento. Sembra infilarsi nel pavimento in qualche punto o rivelare qualche forma di trama ?

\emph{Narratore}: Si, l'acqua sembra convogliare intorno ad una forma quadrata, leggermente rialzata sul pavimento.

\emph{Tups}: Sembra una fossa coperta ?

\emph{Narratore}: potrebbe essere

\emph{Tups}: posso disattivarla ?

\emph{Narratore}: come ? [\emph{Il Narratore volutamente non fa fare una prova, ma coinvolge il giocatore}]

\emph{Tups}: ci incastro il piede di porco così ché il meccanismo non faccia aprire la botola [\emph{Tups non chiede di tirare un dado per capire come disarmarla o disarmarla direttamente, spiega al Narratore come lo fa e basta}]

\emph{Narratore}: attraversi la zona adesso in sicurezza e vedi che si apre su una piccola stanza con due porte di legno rinforzato... }

\medskip

Liberamente ispirato da \href{https://friendorfoe.com/d/Old%20School%20Primer.pdf}{ \textbf{Quick Primer for Old School Gaming}}

\end{tcolorbox}\end{changemargin}

\bigskip

\begin{changemargin}{0.3cm}{0.3cm}\begin{narratore}
Una trappola visibile/ovvia obbliga i giocatori ad interagire con essa, a sforzarsi per capirne il funzionamento ed ingegnarsi per evitarla o disattivarla. Evitate quando potete risoluzioni solo basate sul tiro di dado (Cerco trappole/Disattivo trappole), piuttosto premiate l'ingegnosità anche semplice ma creativa del giocatore per evitare il pericolo... e magari prima o poi si ricorderanno di recuperare il piede di porco...!
\end{narratore}\end{changemargin}

\vfill

\begin{center}
	\includegraphics[width=0.7\linewidth]{immagini/Bear_trap.png}
	
	\emph{Non tutte le trappole sono cosi' segnalate...}
\end{center}


\pagebreak



