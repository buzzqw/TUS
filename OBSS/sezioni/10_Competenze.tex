\section{Competenze}\index{Competenze}

\begin{changemargin}{0.3cm}{0.3cm}\begin{enfasi}{
Chi dice che una cosa è impossibile, non dovrebbe disturbare chi la sta facendo. (Albert Einstein)

\medskip
Non hai veramente capito qualcosa fino a quando non sei in grado di spiegarlo a tua nonna. (Albert Einstein)}\end{enfasi}\end{changemargin}

\begin{multicols}{2}

Le Competenze rappresentano il cosa si conosce ed il cosa si sa fare. I punteggi delle stesse rappresentano quanto bene è conosciuta la competenza e quindi più è alto il valore più si è esperti.

\subsection{Competenze di Base}\index{Competenze di Base}\label{competenzebase}

\begin{changemargin}{0.3cm}{0.3cm}\begin{enfasi}{
Lo studio è per i perdenti! (Lobo) }\end{enfasi}
\end{changemargin}

Ogni personaggio ha una Professione iniziale, un percorso di vita e lavoro che l'ha portato ad apprendere determinate competenze.

Sono elencate alcune Professioni e le loro competenze relative, il personaggio acquisisce queste competenze con il punteggio indicato nella tabella.

Nella scheda va segnata la Professione iniziale e le competenze acquisite, in accordo con il Narratore è possibile selezionare delle competenze diverse e anche creare professioni diverse!

\end{multicols}

\textbf{Tabella: Elenco delle Professioni e relative Competenze}\index[Tabelle]{Tabella Elenco delle Professioni e relative Competenze}\index{Professioni}

\medskip

\noindent\begin{tabularx}{0.95\textwidth}{lllll}
\textbf{Professione}& \textbf{1 punto} & \textbf{2 punti} & \textbf{2 punti} & \textbf{3 punti}\\
\toprule
\textbf{Accolito}& Occulto& Storia o Geografia& Arcana& Religione\\
\textbf{Agricoltore}& Sopravvivenza &Erboristeria& Gestire Animali & Natura\\
\textbf{Alchimista}& Valutare&Natura& Erboristeria& Arcana\\
\textbf{Allevatore}& Sopravvivenza&Seguire tracce& Gestire animali&Natura \\
\textbf{Apprendista mago}& Storia e Geografia&Occulto&Miti e Leggende&Arcana\\
\textbf{Araldo} & Tradizioni Locali & Araldica & Lingue & Diplomazia \\
\textbf{Azzeccagarbugli}& Valutare&Ingannare&Percepire Emozioni&Diplomazia\\
\textbf{Bibliotecario}& Natura e Geografia&Tradizioni locali&Religione e Arcana&Storia\\
\textbf{Borseggiatore} & Disattivare congegni&Artista della fuga&Furtività&Mani di fata\\
\textbf{Boscaiolo}& Usare corda&Seguire tracce & Natura& Sopravvivenza\\
\textbf{Cacciatore}& Furtività&Seguire tracce&Sopravvivenza& Natura\\
\textbf{Carovaniere}&Tradizioni locali &Gestire animali&Sopravvivenza&Cavalcare\\
\textbf{Contadino} & Usare Corda & Erboristeria & Gestire Animali& Natura\\
\textbf{Delinquente}& Sopravvivenza&Cavalcare&Valutare&Furtività\\
\textbf{Erborista}& Miti e Leggende&Geografia&Natura&Erboristeria\\
\textbf{Fabbro} & Pronto Soccorso & Valutare&Atletica & Artigianato \\
\textbf{Giocatore di carte}& Percepire Emozioni&Valutare&Intrattenere&Ingannare\\
\textbf{Guardia boschi}& Miti e Leggende&Erboristeria&Cavalcare & Natura\\
\textbf{Guardia}& Percepire Emozioni&Conoscenza Legge&Cavalcare&Intimidire\\
\textbf{Guida}& Miti e Leggende&Dungeon&Natura&Geografia\\
\textbf{Locandiere}& Pronto soccorso&Valutare&Percepire Emozioni&Diplomazia\\
\textbf{Mercante}& Lingue&Tradizioni Locali&Ingannare&Valutare\\
\textbf{Mercenario} & Percepire Emozioni & Acrobatica & Intimidire & Atletica\\
\textbf{Minatore}& Usare corde&Pronto soccorso&Valutare&Dungeon\\
\textbf{Monaco} & Intrattenere &Pronto Soccorso & Natura & Religione\\
\textbf{Orafo} & Tradizioni locali & Mani di fata & Valutare & Falsificare\\
\textbf{Pescatore}& Usare corde&Nuotare&Sopravvivenza&Natura\\
\textbf{Scriba} &Tradizioni Locali& Falsificare & Miti e Leggende & Lingue\\
\textbf{Sensale} & Intrattenere & Tradizioni locali & Percepire Emozioni & Diplomazia\\
\textbf{Soldato}& Nuotare&Gestire animali&Atletica&Cavalcare\\
\textbf{Teatrante}& Percepire emozioni& Lingue&Acrobatica&Intrattenere\\
\textbf{Uomo di medicina}& Miti e Leggende&Natura&Erboristeria&Pronto soccorso
\end{tabularx}

\vfill

\begin{changemargin}{0.3cm}{0.3cm}\begin{enfasi}{
Anche se indubbiamente il desiderio di conoscere è naturale per tutti gli uomini, la voglia di imparare non è cosa da tutti...(Richard de Bury)
}\end{enfasi}\end{changemargin}

\begin{multicols}{2}

Una professione non si esplicita in sole 4 competenze ma queste sono quelle che più entreranno in uso durante le avventure, il Narratore sarà aiutato dalla vostra professione a capire come il vostro personaggio potrà risolvere le situazioni e come interagirà con gli altri personaggi.

Qui sotto c'è la \textbf{Tabella elenco competenze} da cui scegliere per eventuali nuove professioni o personalizzazioni delle stesse.

\subsubsection{Personalizzare Competenza e Professione}\index{Personalizzare Competenza e Professione}

Ad ogni nuova professione che andrete a creare associate 4 competenze prese dalla \textbf{Tabella: Elenco Competenze e Relativa Caratteristica d'uso}. Una competenza partirà con un punteggio di 1, due competenze partiranno con il punteggio di 2 e quella più specifica e professionale partirà con il punteggio di 3.

In accordo con il Narratore è anche possibile cambiare l'ordine delle Competenze per le Professioni già elencate rendendo più capace il personaggio in alcune competenze piuttosto che altre.

\subsubsection{Competenze, Intelligenza e Background del personaggio}\label{quintacompetenza}\index{Il Competenze e Background del personaggio}

Il giocatore alla creazione del personaggio può decidere di prendere un +1 ad una Competenza già conosciuta oppure prendere una nuova Competenza, legata alla storia del personaggio, a punteggio 1.

Il personaggio acquisisce una Competenza a punteggio 1 per ogni punto di Intelligenza superiore a 2, e perde 1 punto in una Competenza per ogni punto di Intelligenza inferiore a 0.

\medskip

Il giocatore \textbf{aumenta di 1 il punteggio di una Caratteristica che si colleghi alla Professione od al background} fino al valore massimo di 4. Potrebbe essere Intelligenza per un Apprendista mago, ma se questo fa il culturista per hobby potrebbe essere anche Forza.

\begin{changemargin}{0.3cm}{0.3cm}\begin{tcolorbox}[title = Professione ???]
Non sottovalutate la scelta della Professione! Non tutto può risolversi con asciate o magia. Sapere districare nodi, seguire tracce, riconoscere erbe o malattie fanno del personaggio un esperto, creano una professione. Non dovete definire il personaggio solo in base alle Abilità che ha ma in base a cosa e quanto bene sa farlo. Un personaggio di basso livello ma esperto di sopravvivenza sarà sempre più utile di un combattente esperto se si tratta di attraversare un deserto.\end{tcolorbox}\end{changemargin}

\end{multicols}

\medskip

\textbf{Tabella: Elenco Competenze e relativa Caratteristica d'uso}\index[Tabelle]{Tabella Elenco Competenze e relativa Caratteristica d'uso}

\medskip

\begin{tabular}{lllll}\label{competenzeelenco}\hypertarget{competenzeelenco}{}
\textbf{Forza} & \textbf{Destrezza} & \textbf{Intelligenza} & \textbf{Saggezza} & \textbf{Carisma}\\
\toprule
Arrampicarsi	& Acrobatica		& Arcana		&Cavalcare				& Diplomazia \\
Atletica		& Artista della fuga& Artigianato 	&\emph{Consapevolezza} 	& Intrattenere \\
Intimidire		& Furtività 		& Conoscenza	&Gestire animali		& Ingannare \\
Nuotare 		& Mani di fata		& Disattivare congegni &Natura			& Tradizioni locali \\
- & Usare corda	& Erboristeria		& Percepire Emozioni & - \\
- & -				& Falsificare		& Pronto soccorso	 & - \\
- & -				& Valutare			& Seguire tracce	 & - \\
- & -		        & -        			& Sopravvivenza      & - \\

\end{tabular}

\medskip

La \textbf{Conoscenza} va esplicitata su quale argomento verte: Architettura ed Ingegneria, Dungeon, Geografia, Legge, Lingue (terrestri o meno), Miti e Leggende, Nobiltà ed Araldica, Occulto, Piani, Religione, Storia, Tecnologia Antica ...

\begin{multicols}{2}

Ad ogni \textbf{livello successivo al primo} distribuisci un numero di punti pari la metà del punteggio di Intelligenza +1, $[(Int/2)+1]$, con un minimo di 1 punto, tra le competenze già conosciute o perfezionate nell'avventura od apprese ex novo.\index{Competenze per livello}

\textbf{Nessuna competenza di Base o Attiva può avere più del livello +3 punti assegnati.}\index{Massimo punteggio Competenza}

\subsubsection{Consapevolezza}\label{consapevolezza}\index{Consapevolezza}

Una competenza che hanno tutti i personaggi è \textbf{Consapevolezza}, ovvero la capacità di percepire l'ambiente intorno a loro. Questa competenza ha un punteggio fisso pari a 1/3 del livello del personaggio (arrotondato per eccesso) più Saggezza. Su questa competenza il personaggio non può assegnare punti, ma può scegliere certe Abilità per alzarne il punteggio.

I giocatori più che usare Consapevolezza per ricercare informazioni dovrebbero fare domande, indagare, curiosare, arguire ipotesi e confrontarsi e non limitarsi a chiedere un tiro di Consapevolezza per trovare qualcosa.

\subsubsection{Apprendere nuove competenze, professioni}\label{apprenderenuovecompetenze}

Un personaggio può apprendere una nuova competenza o migliorarla con un studio/pratica di almeno 4 ore al giorno per almeno 4 mesi con un insegnante che abbia un punteggio di competenza superiore a quello che mira il personaggio. Dopo questo lasso di tempo il giocatore può assegnare un punto alla competenza di base per cui si è applicato.

Per apprendere una nuova professione deve passare almeno 6 mesi per 6 ore al giorno con chi pratica quella professione. Passati i 6 mesi il personaggio acquisisce le 4 competenze della professione. Eventuali Competenze già conosciute aumenteranno di 1 punto.

\subsubsection{Competenze ed i loro ambiti di utilizzo}\label{competenzeambitidiutilizzo}

Sono descritte sommariamente le Competenze ed i loro utilizzi soliti. Viene anche indicato il numero di Azioni necessarie per svolgere la prova tipica, usi più complessi richiedono più tempo ed Azioni.

Le Azioni necessarie alla prova possono variare a seconda della capacità del personaggio e della complessità della prova.

In ogni caso ricordate sempre di valutare con attenzione come il giocatore dichiara di svolgere le azioni per capirne la durata ed effetti.

La Competenze con un \textbf{*} subiscono le penalità dovute all'\hyperlink{equipaggiamento.armature.scudi}{armatura} indossata (pag. \pageref{equipaggiamentoarmature}).

\medskip

\textbf{Acrobatica* (DES)}: Questa competenza serve per mantenere l'equilibrio su superfici strette o precarie, per tuffarsi, rotolare, fare capriole, salti mortali, superare degli ostacoli nonché cadere e non farsi male. 1 Azione.

\textbf{Arcana (INT)}: Con questa competenza si è esperti di magia e di incantesimi, di oggetti magici è si è grado di identificare gli incantesimi che vengono lanciati. 1 Azione.

\textbf{Arrampicarsi* (FOR)}: Con questa competenza si possono scalare superfici verticali, dalle mura cittadine alle pareti rocciose. E' legata all'Azione di Movimento. Con 8 punti il movimento di Scalare è solo dimezzato.

\textbf{Artigianato (INT)}: si esplicita su una capacità costruttiva, permette costruire l'oggetto dell'artigianato e di giudicare e valutare un lavoro nell'ambito della competenza.

\textbf{Artista della fuga (DES)}: Con questa competenza ci si può liberare da legacci (contrapposta all'Usare Corde) e manette. 1 Azione ogni 10 di DC. Con 6 punti il tempo è 1 Azione ogni 15 di DC, con 12 è 1 Azione ogni 20 DC.\label{artistadellafuga}

\textbf{Atletica* (FOR)}: Con questa competenza si è esperti atleti, capaci di prodigiosi salti ed eccezionali prove di Forza. 1 Azione.

\textbf{Cavalcare (SAG)}: Con questa competenza è possibile cavalcare in maniera professionale e dare comandi alla propria cavalcatura. Vedi Capitolo \hyperlink{cavalcare}{Cavalcare} (pag. \pageref{cavalcare}) 1 Azione.

\textbf{Consapevolezza (SAG)}: per cercare, accorgersi, notare. E' un qualcosa di attivo. 2 Azioni. \textbf{Usare 1 Azione impone una penalità -1d6 alla prova}.

\textbf{Conoscenza di Architettura ed Ingegneria (INT)}: Sei un esperto costruttore e sai valutare la struttura delle costruzioni. Sai anche riconoscere stili architettonici e creare progetti d'interno e d'arredo. 1 Azione.

\textbf{Conoscenza dei Dungeon (INT)}: Con questa competenza si hanno conoscenze di Aberrazioni, caverne, esplorazioni sotterranee, Melme. 1 Azione.

\textbf{Conoscenze di Geografia (INT)}: Con questa competenza si hanno conoscenze sul clima, popolazione, terreni, territori, nazioni e confini. 1 Azione.

\textbf{Conoscenza delle Legge (INT)}: Con questa competenza si conosce la Legge di una regione. Si è esperti nel conoscere le norme ed i cavilli. Si sanno citare casi e si conoscono altri azzeccagarbugli e giudici. 2 Azioni.

\textbf{Conoscenza Lingue (INT)}: Ogni punto in questa competenza permette di apprendere un nuovo linguaggio scritto e parlato. Un buon punteggio di Lingue aiuta a comprendere lingue non note ed a farsi comprendere. Viene usata anche per comprendere testi complessi. Costo variabile.

\textbf{Conoscenza dei Miti e Leggende (INT)}: Si ha una e vera propria passione per i miti e leggende tradizionali e più remoti. Conosci località, storia e creature leggendarie. 1 Azione.

\textbf{Conoscenza di Nobiltà e Araldica (INT)}: Conosci linee nobiliari, casate, dicerie, stemmi araldici, personalità ed i maggiori possedimenti e tesori. Si applica anche a personaggi famosi ed importanti. 1 Azione.

\textbf{Conoscenze dei Piani (INT)}: Con questa competenza si è esperti di Piani e relativi abitanti. 1 Azione.

\textbf{Conoscenze Occulte (INT)}: Con questa competenza si è esperti di occulto, creature immondi. 1 Azione.

\textbf{Conoscenze Religione (INT)}: Con questa competenza si hanno conoscenze su Patroni, mitologia, Celestiali, Non Morti, simboli sacri, tradizione ecclesiastica, feste e ricorrenze liturgiche. 1 Azione.

\textbf{Conoscenze di Storia (INT)}: Con questa competenza si hanno conoscenze di Storia quali guerre, migrazioni, colonie, fondazioni di città, accadimenti importanti.. 1 Azione.

\textbf{Diplomazia (CAR)}: Con questa competenza si possono risolvere diverbi e raccogliere preziose informazioni e dicerie dalle persone. La competenza è anche usata per negoziare in modo efficace con la giusta etichetta e condotta adatta alla situazione controversa. Costo variabile.

\textbf{Disattivare congegni (INT)}: Con questa competenza si possono disarmare Trappole e aprire serrature, sabotare congegni meccanici semplici, come le catapulte, le ruote di un carro o le porte. 1 Azione ogni 10 di DC. Con 6 punti il tempo è 1 Azione ogni 15 di DC, con 12 punti è 1 Azione ogni 20 DC.

\textbf{Erboristeria (INT)}: Con questa competenza si hanno conoscenze di come riconoscere e preparare pozioni e veleni naturali. Il punteggio si applica alle prove per distillare pozioni. Riconoscere Pozioni naturali 1 Azione ogni 10 di DC. Con 6 punti il tempo è 1 Azione ogni 15 di DC, con 12 punti è 1 Azione ogni 20 DC.

\textbf{Falsificare (INT)}: Con questa competenza si sa falsificare e riconoscere come falsi oggetti d'arte, mappe, firme... Costo variabile.

\textbf{Gestire animali (SAG)}: Con questa competenza è possibile addestrare e ammansire animali. 1 minuto ogni 5 di DC. Con 6 punti il tempo è 1 minuto ogni 10 di DC, con 12 è 1 minuto ogni 15 DC.

\textbf{Intimidire (FOR)}: Intimidire si basa sull'approccio fisico per convincere l'interessato. 1 Azione.

\textbf{Ingannare (CAR)}: La competenza Ingannare può essere usata per Raggirare (dicendo quindi fandonie) o Persuadere (adattando la verità) al fine di convincere delle proprie parole l'interessato. Costo variabile.

\textbf{Intrattenere (CAR)}: Con questa competenza si è esperti in una espressione artistica, dal canto alla recitazione, dal ballo a suonare strumenti musicali. E' necessario specificare la forma di intrattenimento. Costo variabile.

\textbf{Mani di fata* (DES)}: Con questa competenza si può borseggiare, estrarre un'arma nascosta, oppure compiere altre azioni senza essere notati(ad esempio barare a carte). 1 Azione.

\textbf{Furtività* (DES)}: Con questa competenza si è in grado di muoversi senza causare rumore o nascondersi nelle ombre. 1 Azione.

\textbf{Natura (SAG)}: Con questa competenza si hanno conoscenze di Animali, Fatati, stagioni e cicli, tempo atmosferico, vegetali. 1 Azione.

\textbf{Nuotare* (FOR)}: Con questa competenza si è in grado di nuotare, anche in acque tempestose. Senza competenza si sa stare a galla in acqua placide. Legata all'Azione di Movimento.

\textbf{Percepire Emozioni (SAG)}: Con questa competenza si può capire se qualcuno sta mentendo o si possono intuire le sue vere intenzioni. 1 Azione.

\textbf{Pronto soccorso (SAG)}: Con questa competenza si possono curare le ferite e le malattie. Costo variabile.

\textbf{Seguire tracce (SAG)}: Con questa competenza si sa seguire le tracce lasciate nell'ambiente. 1 Azione ogni 10 di DC. Con 6 punti il tempo è 1 Azione ogni 15 di DC, con 12 punti è 1 Azione ogni 20 DC.

\textbf{Sopravvivenza (SAG)}: Con questa competenza si può sopravvivere e orientarsi nelle terre selvagge. La competenza è usata anche per cercare attivamente trappole e fosse. 1 minuto per cercare trappole in 3x3 metri, con punteggio 6 costa 3 round, con punteggio di 12 costa 1 round, con punteggio 18 costa 1 Azione.

\textbf{Tradizioni locali (CAR)}: Con questa competenza si hanno conoscenze degli abitanti (più noti), costumi, leggende, leggi, personalità, tradizioni. E' necessario specificare una regione geografica dove è applicabile la conoscenza. 1 Azione.

\textbf{Usare corda (DES)}: Con questa competenza si è esperti in legacci e nodi per fissare e bloccare oggetti o persone. 2 Azioni.

\textbf{Valutare (INT)}: Con questa competenza si sa stimare il valore monetario di un oggetto. La difficoltà è in base alla rarità dell'oggetto, DC 12 + 2 comune, 4 non comune, 8 raro, 12 molto raro, 16 leggendario. 1 Azione ogni 5 di DC. Con 6 punti il tempo è 1 Azione ogni 10 di DC, con punteggio 12 è 1 Azione ogni 20 DC. \label{valutare}

\medskip

\subsubsection{Opzionale - Non usare le Competenze di Base}\index{Opzionale - Non usare le Competenze di Base}\label{nonusarecompetenze}

Fate scegliere la professione ai giocatori e non fate segnare alcun punteggio o valore di Competenza di base.
Ragionate a mente aperta e comprendete, sia tu Narratore che tu Giocatore, per ogni situazione chi ha la professione ed Abilità che meglio si adattano alla prova.
La prova se pertinente alla professione si risolve con un 3d6+Saggezza+1/2LV, se non è pertinente il Narratore ridurrà il bonus dato dal livello, usando la Caratteristica più appropriata. Meglio ancora in base alla descrizione di come viene effettuata la prova deciderà l'esito.

\subsection{Competenze Attive}\index{Competenze Attive}\label{competenzeattive}

\textbf{Il personaggio prende 1 punto, ad ogni livello, da distribuire tra le Competenze Attive od attribuirlo alle Competenze di Base}.

\medskip

Le \textbf{Competenze Attive} sono: Competenza Magica, Competenza Armi, Tiri Salvezza (Riflessi, Tempra, Volontà).

\noindent\begin{itemize}[leftmargin=*] \setlength{\itemsep}{0pt}

\item \textbf{Competenza Magica (CM)}: \index{CM}\index{Competenza Magica} indica la capacità e competenza nel lanciare un incantesimo.

\item \textbf{Competenza Armi (CA)}: \index{CA}\index{Competenza Armi} è la capacità e bravura di combattere con un'arma da mischia o da tiro/distanza.

\item I \textbf{Tiri Salvezza} rappresentano la resistenza, la capacità fisica e psichica del personaggio.
\end{itemize}

Attribuire il punto di Competenze Attive alle \textbf{Competenza di Base} significa distribuire 4 punti aggiuntivi su almeno 3 Competenze di Base a piacere. Attribuire il punto ai Tiri Salvezza significa aumentare di 1 punto un Tiro Salvezza.\index{Aumentare Competenze di Base}

\medskip

\begin{changemargin}{0.3cm}{0.3cm}\begin{enfasi}{C'è solo un modo per allenarsi: quello giusto. (Carl Lewis)
\medskip

Wang Chi: Sei pronto?

Jack Burton: Io sono nato pronto! (Grosso guaio a Chinatown, Film 1986)
}\end{enfasi}\end{changemargin}

\subsubsection{Tiri Salvezza}\index{Tiri Salvezza}\label{tirisavellza}

I \textbf{Tiri Salvezza} (abbreviati in TS) sono usati quando il personaggio é sottoposto ad uno sforzo, vuoi di resistenza fisica, mentale o agilità eccezionale. Il punteggio dei Tiri Salvezza si base sulle Abilità scelte. Abilità più fisiche tenderanno a migliorare l'aspetto di resistenza del personaggio, Abilità più atletiche o di attenzione aumenteranno i riflessi, le Abilità prettamente mentali rafforzeranno la volontà del personaggio.

Il \textbf{Tiro Salvezza su Tempra} indica quanto si è in grado di sopportare le sofferenze fisiche o attacchi contro la propria vitalità e salute. Al valore dei Tiri Salvezza su Tempra si aggiunge il punteggio della \textbf{Costituzione}.

Il \textbf{Tiro Salvezza su Volontà} indica la resistenza contro l'influenza mentale ed altri effetti magici, ciò che vuole modificare il tuo libero arbitrio nelle scelte e nell'agire. Al valore dei Tiri Salvezza su Volontà si aggiunge il punteggio di \textbf{Saggezza}.

Il \textbf{Tiro Salvezza su Riflessi} indica quanto si è agili e pronti per evitare ostacoli o magie. Al valore dei Tiri Salvezza su Riflessi si aggiunge il punteggio di \textbf{Destrezza}.

Quando viene chiesto un Tiro Salvezza significa fare una prova sulla Competenza Attiva richiesta, possa essere Volontà, Tempra o Riflessi.
La prova si andrà ad eseguire tirando 3d6 + valore della Competenza Attiva richiesta ovvero il punteggio nel Tiro Salvezza su Volontà, Riflessi o Tempra + il valore della Caratteristica collegata alla tipologia Competenza Attiva (Saggezza, Destrezza o Costituzione) + Abilità + bonus magici (oggetti che influenzano il Tiro Salvezza) e modificatori vari presenti.

\medskip

\begin{changemargin}{0.3cm}{0.3cm}\begin{tcolorbox}[title = Tiri Salvezza non standard]
E' possibile che vengano richiesti dei Tiri Salvezza con modificatori diversi, ovvero un Tiro Salvezza su Tempra con modificatore Forza oppure un Tiro Salvezza su Volontà con modificatore Carisma. Sarà il Narratore a dirvi quando si applica un modificatore diverso.
\end{tcolorbox}\end{changemargin}

\subsubsection{Competenza Armi}\label{competenzaarmi}

La \textbf{Competenza Armi} (abbreviata in \textbf{CA}) indica la capacità e bravura nell'usare un arma. La competenza si riflette direttamente nelle prove per colpire l'avversario con armi.

Il \textbf{Tiro per Colpire per le armi da mischia}\index{Armi da mischia} si risolve con una prova di Competenza Armi (\textbf{CA}) + \textbf{Forza} + eventuali Abilità + bonus da Lista d'Armi + bonus magici e modificatori contro la Difesa dell'avversario (Destrezza + armatura + scudo + modificatori).

Il \textbf{Tiro per Colpire con armi da distanza} \index{Armi da distanza}(archi, balestre, pugnali da lancio, giavellotti, sassi..) si risolve con una prova di Competenza Armi (\textbf{CA}) + \textbf{Destrezza} + bonus da Lista d'Armi + eventuali capacità, bonus magici e modificatori contro la Difesa dell'avversario (Destrezza + armatura + scudo + modificatori).

Quando si assegna un punto ad \textbf{CA} è necessario precisare su quale gruppo di arma si prende, se non si dichiara allora è come averlo preso nel gruppo Armi Semplici.
Controllare l'elenco \hyperlink{lista.armi}{Armi per Tipologia Omogenea} (pag. \pageref{lista.armi}).\index{Tipologia Omogenea}

Il personaggio può decidere di assegnare il suo punto ad una tipologia di armi che già conosce, migliorando così la sua capacità ed talento nell'uso od apprendere un altra tipologia di armi.

più è alto il punteggio in una tipologia d'armi più facilmente può usufruire di vantaggi con le armi della stessa, ma conoscerà meno armi.

Se il giocatore non ha assegnato alcun punto nella \textbf{CA} può utilizzare senza penalità al colpire solo le armi raggruppate come Armi Semplici.

Le \textbf{Armi Semplici} sono: Pugnale, Mazza Leggera, Mazza chiodata, Bastone, Balestra (Leggera), Giavellotto\index{Armi Semplici}

Usare un'\textbf{Arma senza conoscere la Lista d'Armi di appartenenza}, o che non sia un \textbf{Arma Semplice}, impone un -1d6 al Tiro per Colpire.\index{Arma senza competenza}

\begin{changemargin}{0.3cm}{0.3cm}\begin{tcolorbox}[title = Specializzarsi o meno in un arma]
I vantaggi di specializzarsi in una Lista d'Armi sono concreti e tangibili ma si portano dietro il limite di non sapere usare bene le altre armi.

Valutate il tipo di avventura, se avrete sempre a disposizione o potrete facilmente ottenere le vostre armi \emph{preferite}.

Valutate anche se la tipologia di nemici che affronterete possa avere delle resistenze o meno alla tipologia di danno che causate. Creare un ottimo arciere quando si affrontano quasi sempre scheletri non è una buona idea.
\end{tcolorbox}\end{changemargin}

Per poter utilizzare \textbf{Armature Leggere} e \textbf{Scudi Leggeri} è necessario avere Forza almeno -1.\index{Armature Leggere}\index{Scudi Leggeri}\label{indossarearmature}\hypertarget{indossarearmature}{}

Per poter utilizzare \textbf{Armature Medie} e \textbf{Scudi Medi} è necessario avere almeno 2 punti in Competenza Armi.\index{Armature Medie}

Con almeno 3 punti in Competenza Armi ed 1 in Forza si possono usare senza penalità \textbf{Armature Pesanti} e \textbf{Scudi Pesanti}.\index{Armature Pesanti}\index{Scudi Pesanti}

Usare un'\textbf{Armatura senza l'adeguata competenza} impedisce di usare il valore di Destrezza in Difesa ed il bonus conferito dall'armatura alla Difesa si riduce di 1.\index{Armatura senza competenza}

Usare uno \textbf{Scudo senza l'adeguata competenza} peggiora il Tiro per Colpire di 1 e lo scudo conferisce un bonus massimo a Difesa di 1.\index{Scudo senza competenza}

\subsubsection{Competenza Magica}\label{competenzamagica}

La \textbf{Competenza Magica} (abbreviata in \textbf{CM}) permette al personaggio di poter conoscere più incantesimi, più potenti, più efficaci e più facilmente lanciabili.

Un personaggio con alta \textbf{Competenza Magica} sa manipolare più incantesimi e con risultati migliori.

Il valore di Competenza Magica stabilisce insieme all'Abilità Adepto della Magia ed al modificatore di caratteristica per incantesimi il livello massimo di incantesimi lanciabile.

Il punto di Competenza Magica, a differenza di quello di Competenza Armi, non è necessario dichiararlo su una Lista di Magia, Patrono o altro. Il punto si \emph{trasforma} in grezzo potere magico.

E' necessario avere almeno 1 punto in Competenza Magica se si vuole essere degli usufruitori di magia, se si vuole conoscere il come funziona la magia e la connessione di questa con i Patroni.

Non è strettamente necessario mettere sempre il punto di Competenza Attiva in Competenza Magia se si vuole fare un incantatore, anzi, qualche punto in Competenza Armi serve anche al mago più bravo se vuole sapere \emph{mirare e colpire} un avversario con qualcosa di diverso da un incantesimo.

Un punteggio di Competenza Magica 14 e aver preso 4 volte Adepto della Magia, avendo 4 nel modificatore di caratteristica per incantesimi, è sufficiente per poter lanciare incantesimi di massimo livello. CM 9 e Adepto della Magia preso tre volte vi possono garantire il sesto livello di incantesimi.

\subsubsection{Opzionale - Abilità come Competenze Attive}\index{Opzionale - Abilità come Competenze Attive}

Il Narratore può concedere su richiesta del giocatore di poter usare il punto di Competenza Attiva non per aumentare la Competenza Magica o delle Armi, bensì per selezionare una nuova Abilità, rispettando i requisiti.

\subsubsection{I punteggi delle Competenze Base e Attive}\label{punteggicompetenzebaseattive}\index{I punteggi delle Competenze Base e Attive}

Ogni punto attribuito nella Competenze di Base o Competenza Armi o Magica permette di usufruire di +1 nella prova relativa (Prove, Tiro per Colpire, Competenza Magica)

\begin{changemargin}{0.3cm}{0.3cm}\begin{tcolorbox}[title = Tups arriva al 4' livello!]
Tups è arrivato al 4' livello! Ecco come ha distribuito i punti delle Competenze Attive.

\textbf{1 livello}: +1 Competenza Armi, Abilità: Armatura del Devoto (+2 Volontà, +1 Riflessi, \textbf{Costituzione}), Il Patrono è la mia Arma (+1 Riflessi, +1 Volontà, \textbf{Costituzione}), Conoscenza istintiva (+2 Volontà, +1 Tempra, \textbf{Saggezza})

\textbf{2 livello}: +1 Competenza Magica, Abilità: Colpi Poderosi (+2 Tempra, \textbf{Costituzione})

\textbf{3 livello}: +1 Competenza Magica, Abilità: Fedele (+2 Volontà, +1 Tempra, \textbf{Costituzione}). Il punteggio di Costituzione aumenta di 1.

\textbf{4 livello}: +1 Competenza Armi, Abilità: Incantatore Prudente (+2 Riflessi, +1 Tempra, \textbf{Intelligenza})

\textbf{\emph{Totale}}: +2 CA, +2 CM, +4 Tiro Salvezza Riflessi, +6 Tiro Salvezza Tempra, +6 Tiro Salvezza Volontà

\end{tcolorbox}\end{changemargin}

\end{multicols}

\vfill

\begin{center}
\includegraphics[width=0.45\linewidth]{immagini/attaccoallespalle.png}

\emph{Prova fallita di furtività...}
\end{center}

\pagebreak
