\section{I Tesori}\index{Disporre Tesori}\label{disporretesori}

\begin{multicols}{2}

Mentre i personaggi avanzano di livello anche la quantità di tesori che trasportano ed usano aumenta. In OBSS si suppone che tutti i personaggi di pari livello abbiano più o meno la stessa quantità di tesoro e oggetti magici. Poiché il reddito primario per un personaggio deriva dai tesori e dai bottini ricavati dalle avventure è importante stare attenti alla ricchezza e i tesori delle avventure.

\subsection{Dove sono i Tesori}

Alcuni tesori li avranno i mostri (vedi sotto), altri saranno dispersi e nascosti nel dungeon ed altri ancora saranno sul fondo di trappole e tunnel nascosti.

Come distribuire un tesoro è una faccenda importante. I tesori non devono essere sbattuti in faccia ai personaggi, ne tanto meno nascosti che non è possibile trovarli.

Un consiglio è fare in modo che i tesori (oggetti e monete) trovati nei dungeon siano distribuiti secondo questo criterio:

\begin{itemize}[leftmargin=*] \setlength{\itemsep}{0pt}
\item \textbf{un terzo li avranno i mostri addosso}
\item \textbf{un terzo saranno nascosti dietro passaggi segreti o trappole}
\item \textbf{un terzo sarà sparso in giro}
\end{itemize}

Questo stimolerà i giocatori a continuare l'esplorazione, affrontare i mostri e cercare attivamente nel dungeon.

Animali, Vegetali, Costrutti, Non Morti non intelligenti, Melme e trappole sono ottimi \emph{incontri con poco tesoro}.
In alternativa, se i personaggi affrontano un certo numero di creature con poco o nessun tesoro, dovrebbero avere l'occasione di ottenere un certo numero di oggetti di valore più significativo nell'immediato futuro per compensare lo squilibrio.

Come regola generale, i personaggi non dovrebbero possedere alcun oggetto magico di valore superiore alla metà della ricchezza totale del personaggio, pertanto controllate bene prima di ricompensare i personaggi con oggetti molti costosi.

\subsection{Costruire un Bottino}\index{Costruire un Bottino}\label{costruireunbottino}

Spesso è sufficiente dire ai vostri giocatori che hanno trovato 5000 mo in gemme e 10000 mo in gioielli, ma è più interessante fornire dei particolari. Dare a un tesoro una personalità può non solo aiutare la verosimiglianza del gioco, ma può a volte innescare nuove avventure.

Nelle pagine seguenti troverete le regole e tabelle per attribuire i tesori ai nemici così da poter generare casualmente cosa trovano i personaggi.

\subsubsection{Monete e Gemme}\index{Elenco Gemme}\index{Gemme}\label{gemme}

\textbf{Monete}: Le monete in un tesoro possono essere di rame, argento, oro e platino: quelle d'argento e d'oro sono le più comuni, ma potete decidere diversamente. Per le monete ed il loro valore di cambio andate all'Equipaggiamento.

Le monete in possesso di mostri e creature selvagge non saranno certo fior di conio e saranno probabilmente segnate da morsi o bave appiccicose. Monete invece trovate nei tesori o in fondo a qualche tana potrebbero essere di altri regni, se non mondi.. ed in quel caso quello che li fa valere è lo stretto valore metallurgico. 10 grammi di oro sono sempre 10 grammi di oro anche se su una faccia della moneta c'è un fiore ed in un altra un castello.

Usate la Tabella \hyperlink{valoredellegemme}{Valore delle gemme} (pag. \pageref{valoredellegemme}) per determinare il valore delle gemme trovate. Qui sono elencate le gemme per valore.

\textbf{Gemme}: Anche se potete assegnare qualsiasi valore ad una gemma, alcune possono valere di più delle altre. Utilizzate le categorie di valore qui sotto (e le pietre preziose associate) come guida di riferimento quando assegnate i valori alle pietre preziose. Solitamente le gemme vengono vendute ed acquistate a valore pieno.

\textbf{Gemme di Bassa Qualità} (10 mo): agata; azzurrite; quarzo blu; ematite; lapislazzuli; malachite; ossidiana; rodocrosite; occhio di tigre; turchese; perla di fiume (irregolare).

\textbf{Gemme Semi Preziose} (50 mo): eliotropio, corniola; calcedonio; crisoprasio; citrino; diaspro; lunaria; onice; crisolito; cristallo di roccia (quarzo chiaro); sardonica; sardonice; quarzo rosato, affumicato o rosa di stella; zircone.

\textbf{Pietre Preziose di Media Qualità} (100 mo): ambra; ametista; crisoberillo; corallo; granato rosso o verde-marrone; giada; giaietto; perla bianca, dorata, rosa o argentata; spinello rosso, marrone-rosso o verde scuro; tormalina.

\textbf{Pietre Preziose di Alta Qualità} (500 mo): alessandrite; acquamarina; granato viola; perla nera; spinello blu scuro; topazio giallo oro.

\textbf{Gioielli} (1000 mo): smeraldo; opale bianco, nero, o di fuoco; zaffiro blu; corindone giallo fuoco o vermiglio; zaffiro a stella blu o nero; tanzanite.

\textbf{Gioielli Eccezionali} (5000 mo o più): smeraldo verde brillante, diamante, giacinto, rubino, miele topetto cristallino.

\textbf{Tesori non Magici} Questa categoria include monili, abiti raffinati, merci, oggetti alchemici, oggetti perfetti e altri.

Diversamente delle gemme, molti di questi oggetti hanno valori stabiliti, ma potete sempre aumentare il valore dell'oggetto decorandolo con pietre preziose o con fatture particolarmente artistiche.

\textbf{Oggetti d'Arte Raffinati} (100 mo o più): Anche se alcuni oggetti d'arte sono composti di materiali preziosi, il valore della maggior parte di pitture, sculture, opere letterarie, abiti raffinati, e simili consiste nella fattura con cui sono realizzati e nella bravura di chi li ha realizzati. Gli oggetti d'arte sono spesso ingombranti o difficili da spostare, e fragili, rendendone il recupero ed il trasporto un'avventura a sé.

\textbf{Monili Minori} (50 mo): Questa categoria comprende monili realizzati con materiali come ottone, bronzo, rame, avorio, o legni esotici, a volte impreziositi con gemme di bassa qualità molto piccole o difettate. I monili minori includono anelli, braccialetti e orecchini.

\textbf{Monili Normali} (100-500 mo): La maggior parte dei monili è realizzata con argento, oro, giada, o corallo, e decorata spesso con gemme semi preziose o pietre preziose di qualità media. I monili normali comprendono tutti i tipi di monili minori più bracciali, collane e spille.

\medskip

\begin{changemargin}{0.3cm}{0.3cm}\begin{narratore}
Non esagerate mai con i tesori, specialmente quelli magici. Un tesoro non deve diventare un abitudine. Un conto possono essere le monete, gemme e consumabili un conto sono i veri tesori, quelli magici, particolari, unici.

Rispettare la Legge del Premio non significa riempire le tasche ai personaggi, altrimenti gli verrà a noia il rischiare la vita per nuovi tesori ed oggetti. Quando fate trovare un oggetto magico ragionate sempre in prospettiva. E' vero che può essere bello vedere i giocatori felici per quello che hanno trovato ma poi sarete costretti l'avventura successiva a dare qualcosa di ancora più potente.
\end{narratore}\end{changemargin}

\medskip

\textbf{Monili Preziosi} (500 mo o più): I monili preziosi sono realizzati in oro, mithral, platino, o simili metalli rari. Tali oggetti comprendono i tipi di monili normali più scettri, pendenti ed altri grandi oggetti.

\textbf{Attrezzi fatti ottimamente} (100-300 mo): Questa categoria include attrezzi per le Professioni o Competenze: vedi Equipaggiamento per i dettagli e i costi di questi oggetti.

\textbf{Oggetti Comuni} (fino a 1000 mo): Ci sono molti oggetti di valore di natura alchemica o comune che possono essere utilizzati come tesoro. La maggior parte degli oggetti alchemici sono oggetti portabili e stimabili, ma anche altri come serrature, simboli sacri, cannocchiali, vini prelibati o abiti raffinati possono costituire parti interessanti di un tesoro. Anche le merci commerciali possono servire da tesoro: 5 kg di zafferano, per esempio, valgono 150 mo.

\textbf{Mappe del Tesoro e Oggetti d'Informazione} (variabili): Gli oggetti come mappe del tesoro, documenti legali di navi e case, liste di informatori o dei turni di guardia, parole d'accesso, e simili possono essere divertenti oggetti da trovare in un tesoro: potete stabilire il valore di questi oggetti come volete e possono essere di doppia utilità in quanto possono generare idee per nuove avventure.

\textbf{Tesori Accidentali}: sono i tesori che la creatura ha con se o nella tana per puro caso o perché non voluti. Possono essere i resti \emph{non digeriti} di un lauto pasto o qualcosa che ha attirato l'attenzione della creatura. Un tesoro accidentale \index{Tesoro Accidentale} va valutato caso per caso a seconda dell'ambiente e creatura che lo possiede.


\subsection{Oggetti Magici}

Naturalmente, la scoperta di un Oggetto Magico è il vero premio per qualsiasi avventuriero. Fate attenzione a collocare gli Oggetti Magici in un tesoro: è molto più soddisfacente per molti giocatori trovare un oggetto magico piuttosto che comprarlo.

Anche se in genere dovreste collocare gli oggetti con attenta riflessione sui loro probabili effetti sulla vostra campagna, può essere divertente generare gli oggetti magici in un tesoro a caso. Fate attenzione, comunque! è facile, con un pò di fortuna (o sfortuna) dei dadi gonfiare il vostro gioco con troppo tesoro o privarlo dello stesso. Il collocamento di oggetti magici casuali dovrebbe essere temperato sempre dal buon senso del Narratore.

Anche gli incantesimi sono veri e propri tesori e premi al pari di oggetti magici. Valutate con attenzione quali possono essere trovati. Ricordate che una abilità magica non è un incantesimo copiabile, solo quelli presenti nei tomi, pergamene e quant'altro appositamente creato per essere un ricettacolo di incantesimi è idoneo alla copia.


\subsubsection{Tesori Magici}\index{Tesori Magici}\index{Trovare tesori magici}\label{tesorimagici}

I Tesori magici possono essere trovati dai personaggi in tre modi:

\medskip

- \textbf{addosso ai nemici o nelle loro tane}

- \textbf{dispersi o nascosti nel dungeon}

- \textbf{comprati} (!!!)

\medskip


Quale sia la situazione il Narratore deve sempre prestare attenzione agli oggetti magici che i personaggi \emph{troveranno}.

I Tesori magici vanno inseriti, se addosso ai nemici o dungeon, con parsimonia e ragionando, cercate di resistere alla tentazione di essere generosi con i personaggi perché facilmente si abitueranno e difficilmente potrete recuperare la situazione.

Ancora di più è necessario che gli oggetti magici, specialmente quelli più potenti, non possano essere comprati come \emph{vili} oggetti comuni. Non lesinate su Pozioni di Cura o piccoli ninnoli magici che hanno una loro utilità, eppure gli oggetti più meravigliosi (dalla spada +2 in su..) devono essere trovati, deve essere affrontato colui che attualmente possiede quell'oggetto, altrimenti lo scopo dell'avventura e del pericolo va a scemare.

Nel caso preferite una distribuzione stabilita e bilanciata seguite le indicazioni sottostanti.\index{Oggetti magici per livello}

Per quanto riguarda gli oggetti magici \textbf{permanenti} come armi, armature, altri oggetti senza cariche o con cariche giornaliere, potete distribuire gli oggetti secondo questo schema, cumulativo per ogni livello:

\medskip

- livelli 1-4: un oggetto non comune

- livelli 5-7: un secondo oggetto non comune

- livelli 8-10: un oggetto raro

- livelli 11-13: un secondo oggetto raro

- livelli 14-16: un oggetto molto raro

- oltre il 17: oggetto molto raro o leggendario.

\medskip

Per quanto riguarda i \textbf{consumabili} come pozioni, pergamene od oggetti con un uso a scalare di cariche, potete distribuire gli oggetti secondo questo schema, cumulativo per ogni livello:

\medskip

- livelli 1-5: un consumabile comune

- livelli 6-10: un consumabile non comune

- livelli 11-15: un consumabile raro

- livelli 16-19: un consumabile molto raro

- oltre il 20: un consumabile leggendario

\medskip

Tutto questo chiaramente dipende dal livello di magia che si vuole dare all'avventura.

In questa maniera piloterete gli oggetti trovati in base alle necessità del gruppo ed al bilanciamento dell'avventura.


\subsection*{Ma perché dungeon senza fine ?}

I primi Patroni detestavano questo nuovo pianeta ed ancora di più il Sole come manifestazione di Ljust.
Distruggevano la Terra per ordine divino e per poter depredare e costruire un \emph{nuovo mondo nel sottosuolo}, il più lontano possibile dalla superficie. Se un anno può sembrare poco ricordatevi che la loro volontà era assoluta e la realtà si piegava a ciò che volevano.

Furono costruite innumerevoli sistemi, ambienti, strutture, città sotterranee la cui dimensione rivaleggiava con interi stati e qui cumularono i \emph{tesori} e qualunque oggetto attirasse la loro attenzione. Riempirono questi luoghi di creature \emph{aliene} e infinite mostruosità e oggetti moderni e creazioni prese dall'infinita fantasia della cultura di innumerevoli pianeti.

Per renderle infine più sicure, se mai fosse stato necessario, nella zona più profonda ed inaccessibile crearono quello che viene chiamato il \emph{cuore del dungeon} una frattura, un portale costruito specificatamente per rimpinguare le orde di mostri che ci abitano. Questi portali sono da estirpare e chiudere se si vuole che i mostri terminino e si possa veramente ricominciare a costruire un nuovo mondo.

Se i dungeon più piccoli (quelli entro 5, 6 piani) hanno un solo \emph{cuore} quelli più ampi e profondi ne hanno molti di più, ed ogni volta protetti da creature più forti e difficili da sconfiggere.

\medskip

\begin{changemargin}{0.3cm}{0.3cm}\begin{enfasi}
%Come sarebbe bello dire \emph{per caso}? .. "Tu credi davvero che ci sia qualcosa che succede \emph{per caso}?" (Alessandro Baricco)
%\medskip

Tesoro è qualunque cosa mobile di pregio, nascosta o sotterrata, di cui nessuno può provare d'essere proprietario. (Codice civile italiano)

\end{enfasi}\end{changemargin}

\subsection*{Ma come fanno ad esistere oggetti magici sulla Terra ?}

I primi Patroni non hanno solo distrutto, stravolto, modificata tutta la geografia della Terra ma hanno anche portato nuove razze, creature, demoni e mostri!

Queste creature sono state portate con il loro bagaglio di conoscenze ed oggetti. Non solo, i Patroni hanno creato e fornito a questi le armi ed \emph{attrezzature} necessarie al loro scopo.

Hanno poi attinto alla tradizione, folklore, letteratura ed incubi, non solo nostri, per creare gli oggetti più straordinari e unici e seminarli nelle profondità del pianeta in quelle che dovevano diventare le loro dimore.

Questi oggetti sono nelle profondità e custoditi dalle aberrazioni evocati dai Patroni, come sempre più si scende in profondità più c'è possibilità di trovare qualcosa, perché quella che è per noi la Legge del Premio, per i primi Patroni è il modo di nascondere i preziosi, nelle profondità delle loro \emph{case}.

Le nuove razze, a differenza degli umani, possedevano già le conoscenze per creare questi oggetti unici e hanno continuato a produrli.

Nel secolo che è passato parte della conoscenza per creare gli oggetti magici è stata appresa da Elfi, Nani e Gnomi ed in parte dai Patroni stessi. Gli umani hanno incominciato a costruire loro stessi oggetti fantastici anche se alcuni dei tesori più preziosi rimangono nelle terrorizzanti gotiche cattedrali e città elfiche o nelle miniere infinite che sono le case dei nani.

Come Narratore approfittate della conoscenza che può avere un anziano mago elfo, per coinvolgere i personaggi in avventure per recuperare rari ingredienti, leggendari incantesimi, oggetti mitici.. ed imparare una coltura così antica e diversa.

Usate anche gli oggetti mitici della cultura terrestre, sicuramente un Patrono si è divertito a crearli per poi annoiarsene qualche attimo dopo ed averlo gettato nelle viscere di qualche caverna.

\subsection{Tesori e Mostri}\index{Tesori e Mostri}

Ogni mostro ha assegnato una \textbf{Categoria Tesoro} (CT), questa categoria indica in linea di massima che tipo di tesoro, monete ed eventuali oggetti magici, la creatura porta con se o sono disponibili nella sua tana. Potete sempre decidere autonomamente o modificare le percentuali indicate per favorire un certo tipo di tesoro.

Quando è indicata una percentuale significa che è necessario fare uguale o meno con il d100 per trovare il tesoro indicato. Quando è indicato +1 \emph{oggetto} allora significa che indipendentemente dal tiro percentuale fatto ci sarà almeno 1 oggetto, pozione o pergamena, del tipo indicato.

Es. una creatura è segnata come \emph{Tesoro CT} \textbf{F} significa che nella sua tana, nascondiglio, ci sarà il 10\% di trovare 3d6 monete l'argento, il 40\% di trovare 1d6 monete d'oro ed 5 volte il 30\% di trovare un oggetto magici che non siano armi.

%Non tutti i tesori pecuniari devono necessariamente essere costituiti da monete, gemme, gioielli, potete distribuire i tesori sotto forma di opere d'arte, arazzi, sculture e altri oggetti. Gli stessi oggetti magici possono avere funzioni aggiunte dal Narratore per arricchire l'avventura ed introdurre nuove situazioni.

Consultate poi la \hyperlink{tipologiaoggettomagico}{Tabella Tipologia Oggetto magico} (pag. \pageref{tipologiaoggettomagico}) per tirare e scoprire quale oggetti magici aveva la creatura.

\end{multicols}

%\subsubsection{Composizione dei tesori}

\textbf{Tabella: Composizione dei tesori}\index[Tabelle]{Tabella Composizione dei tesori}\label{valoretesoroincontro}\hypertarget{valoretesoroincontro}{}\index{Lettere dei Tesori}

\medskip

\noindent\begin{tabularx}{\textwidth}{>{\bfseries}l|>{\small}c|>{\small}c|>{\small}c|>{\small}c|>{\small}c|>{\small}c|>{\small}X}
	\toprule
	\multicolumn{8}{c}{\textbf{Tesori da tana o nascondigli di creature}} \\
	\midrule
	CT & \textbf{mr} & \textbf{ma} & \textbf{mo} & \textbf{mp} & \textbf{Gemme} & \textbf{Gioielli} & \textbf{Oggetti magici} \\
	& \multicolumn{1}{c}{x 1000} & \multicolumn{1}{c}{x 1000} & \multicolumn{1}{c}{x 1000} & \multicolumn{1}{c}{x 100} & & & \\
	\midrule
	A & 1d3, 25\% & 2d10, 30\% & 1d6, 40\% & 3d6, 35\% & 1d4, 60\% & 2d6, 50\% & 3 qualsiasi, 30\% \\
	\hline
	B & 1d6, 50\% & 1d3, 25\% & 2d10, 25\% & 1d10, 25\% & 1d8, 30\% & 1d4, 20\% & Armature o armi, 10\% \\
	\hline
	C & 1d10, 20\% & 1d6, 30\% & - & 1d6 10\% & 1d6, 25\% & 1d3, 20\% & 2 qualsiasi, 10\% \\
	\hline
	D & 1d6, 10\% & 1d10, 15\% & 1d3, 50\% & 1d6, 15\% & 1d10, 30\% & 1d6, 25\% & 2 qualsiasi, 15\%, + 1 pozione\\
	\hline
	E & 1d6, 5\% & 1d10, 25\% & 1d4, 25\% & 3d6, 25\% & 1d12, 15\% & 1d6, 10\% & 3 qualsiasi, 25\%, +1 pergamena \\
	\hline
	F & - & 3d6, 10\% & 1d6, 40\% & 1d4, 15\% & 2d10, 20\% & 1d8, 10\% & 5 qualsiasi, non armi 30\% \\
	\hline
	G & - & - & 2d10, 50\% & 1d10, 50\% & 3d6, 30\% & 1d6, 25\% & 5 qualsiasi 35\% \\
	\hline
	H & 3d6, 25\% & 2d10, 40\% & 2d10, 55\% & 1d8, 40\% & 3d10, 50\% & 2d10, 50\% & 6 qualsiasi 15\% \\
	\hline
	I & - & - & - & 1d6, 30\% & 2d6, 55\% & 2d4, 50\% & 1 qualsiasi 15\% \\
	\bottomrule
\end{tabularx}




\medskip

\begin{center}
	\includegraphics[width=0.5\linewidth]{immagini/Hoxne_Hoard_1.png}

	\emph{Riproduzione del tesoro di Hoxne}
\end{center}


\bigskip

\noindent\begin{tabularx}{\textwidth}{>{\bfseries}l|>{\small}c|>{\small}c|>{\small}c|>{\small}c|>{\small}c|>{\small}c|>{\small}X}
\toprule
\multicolumn{8}{c}{\textbf{Tesori Individuali, piccole tane, zaini e borse}} \\
\midrule
CT & \textbf{mr} & \textbf{ma} & \textbf{mo} & \textbf{mp} & \textbf{Gemme} & \textbf{Gioielli} & \textbf{Oggetti magici} \\
\midrule
J & 3d8 & - & - & - & - & - & - \\
\hline
K & - & 3d6 & - & - & - & - & - \\
\hline
L & - & - & - & 2d6 & - & - & - \\
\hline
M & - & - & 2d4 & - & - & - & - \\
\hline
N & - & - & - & 1d6 & - & - & - \\
\hline
O & 1d4*10 & 1d3*10 & - & - & - & - & - \\
\hline
P & - & 1d6*10 & - & 3d6 & - & - & - \\
\hline
Q & - & - & - & - & 1d4 & - & - \\
\hline
R & - & - & 2d10 & 1d6*10 & 2d4 & 1d3 & - \\
\hline
S & - & - & - & - & - & - & 1d8 pozioni \\
\hline
T & - & - & - & - & - & - & 1d4 pergamene \\
\hline
U & - & - & - & - & 2d8, 90\% & 1d6, 80\% & 1 qualsiasi, 70\% \\
\hline
V & - & - & - & - & - & - & 2 qualsiasi \\
\hline
W & - & - & 5d6 & 1d8 & 2d8, 60\% & 1d8, 50\% & 2 qualsiasi, 60\% \\
\hline
X & - & - & - & - & - & - & 2 pozioni \\
\hline
Y & - & - & 1d6*100 & - & - & - & - \\
\hline
Z & 1d3*100 & 1d4*100 & 1d6*100 & 1d4*100 & 1d6, 50\% & 2d6, 50\% & 3 qualsiasi, 50\% \\
\bottomrule
\end{tabularx}

\bigskip

Quando il tesoro è indicato da più lettere la creatura possiede entrambi i tesori indicati.

Alcune creature particolarmente \emph{ricche} potrebbero avere più volte lo stesso tesoro (2 H ovvero 2 volte il tesoro H).

\pagebreak

