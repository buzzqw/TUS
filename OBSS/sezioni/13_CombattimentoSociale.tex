\section{Combattimento Sociale}\index{Combattimento Sociale}

\begin{changemargin}{0.3cm}{0.3cm}\begin{enfasi}{

Per formulare la dialettica in modo limpido bisogna considerarla, senza badare alla verità oggettiva (che è oggetto della logica), semplicemente come l'arte di ottenere ragione, la qual cosa sarà certo tanto più facile se si ha oggettivamente ragione. (Arthur Schopenhauer)

}\end{enfasi}\end{changemargin}


\begin{multicols}{2}

Per Combattimento Sociale si intende il tentativo da parte dei personaggi di convincere, forzare o raggirare i PNG o comunque creature tenute dal Narratore a fare o dire cose che non vorrebbero.

Può capitare che i giocatori tentino di corrompere una guardia, di ottenere informazioni in maniera diplomatica oppure intimidatoria, di ottenere una paga più alta, di raggirare un mercante o più semplicemente ogni qual volta lo \emph{scontro} o \emph{confronto} non è tramite armi ma a parole.

Per quanto il combattimento sociale possa riguardare una moltitudine di situazioni quello che accomuna tutte le prove è il metodo con cui si vuole ottenere il risultato finale

\begin{wrapfigure}[16]{r}[.5\width+.5\columnsep]{7.5cm}%\itshape

\centering
\includegraphics[width=6.5cm]{immagini/Greuter_Socrates.png}

\emph{Socrates and His Students. Johann Friedrich Greuter, 17th century.}
\end{wrapfigure}

, non tramite armi ma cercando di \emph{convincere} l'avversario.

In queste situazioni si possono seguire due approcci distinti, da una parte il Narratore valuta il risultato in base a quanto dicono i giocatori, dall'altra questo sistema imposta le regole come fosse un combattimento per stabilire chi vince nella prova finale.

Ogni Narratore sceglie l'approccio che preferisce, diciamo che in base all'esperienza con il sistema ed il gioco di ruolo in generale potrebbe preferire un sistema o l'altro. Per un approccio neutro usare delle regole può essere più indicato.

A seconda che il giocatore usi metodi più o meno coercitivi l'avversario resisterà di conseguenza.
Il giocatore eseguirà una Prova Contrapposta di Intimidire, Diplomazia od Ingannare e l'avversario cercherà di resistere con un Tiro Salvezza su Volontà con bonus Carisma.
Se si deve resistere ad una coercizione basata su minacce contrapponete un Tiro Salvezza su Volontà con bonus di Forza.

Il Narratore in base al livello del PNG stabilirà quanti successi consecutivi sono necessari per convincerlo. In linea di massima è necessario 1 + 1 successo ogni due livelli del PNG.\index{Successi necessari per convincere} Il numero di successi può essere modificato in base alle convinzioni, promesse, patti, rapporti interpersonali che l'avversario ha riguardo alla situazione.

Se si vincono tutte le prove si vincerà il \emph{combattimento} e si otterrà l'informazione o quanto richiesto. In caso di successo critico, ovvero oltre superare la prova si sono tirati almeno due 6 si conteranno due successi.

In caso di fallimento della prova questa può essere riprovata con un -1 di penalità se le conseguenze del fallimento non portano ad una scena successiva.

\begin{wrapfigure}[16]{l}[.5\width+.5\columnsep]{7.5cm}

\centering
\end{wrapfigure}

Se il fallimento è critico allora non solo la prova è fallita ma non sarà possibile effettuare ulteriori tentativi e l'avversario diverrà ancora meno amichevole. Molto probabilmente il Narratore deciderà l'evoluzione della situazione in base alla richiesta e scena originale.

In caso di Intimidazione molto probabilmente l'obiettivo del giocatore diventerà ostile, in caso di Inganno è possibile che sentendosi ingannata menta o che non dica nulla. In caso di Diplomazia è più probabile un silenzio o un cortese diniego.

Il Narratore deve usare queste prove, che siano risultate positive o negative, per fare evolvere la scena ed arricchire l'avventura.

Se c'è una informazione che non volete dare impostate una difficoltà più alta.
Ricordate che i personaggi si fidano del risultato ottenuto e se incominciate a dare informazioni false non potranno più fidarsi delle prove fatte.

Non dovete pensare che dare l'informazione sia un problema, alla fine i giocatori se la sono guadagnata e per voi è una nuova possibilità per arricchire l'avventura.

\end{multicols}

\vfill

\begin{changemargin}{0.3cm}{0.3cm}\begin{enfasi}{
Colui che non vuole ragionare è un fanatico, colui che non sa ragionare è un pazzo e colui che non osa ragionare è uno schiavo. (William Drummond di Hawthornden)
}\end{enfasi}\end{changemargin}



\pagebreak

