\section{La Terra}\index{Terra}\index{Atilantis}

\begin{changemargin}{0.3cm}{0.3cm}\begin{enfasi}{
Così la Terra è davvero tonda. Però non immaginavo che fosse azzurra. Perché gli uomini che vivono su un pianeta tanto bello non fanno altro che combattere tra loro? (Nadia - Il mistero della pietra azzurra)

\medskip

Il pianeta non ci appartiene, siamo noi ad appartenergli. Noi siamo di passaggio, lui rimane. (Pierre Rabhi)}\end{enfasi}\end{changemargin}\medskip

\begin{multicols}{2}

\label{terra}

La Terra, inteso proprio come pianeta, per come lo abbiamo conosciuto è un ricordo oramai sbiadito.

Tàhil fu strumento diretto di Calicante per sfogare la rabbia dell'uccisione di un suo figlio.

Nessuna città fu risparmiata, ogni insediamento fu vittima di terremoti, inondazioni, epidemie. Da un giorno all'altro le città vennero letteralmente ribaltate sottosopra.

Passati 30 giorni Calicante decise che non bastava distruggere tutto e tutti ma era necessario cambiare una volta e per sempre la Terra, era necessario che nuovi Patroni intervenissero, era necessario una Rifondazione.

Questi attinsero alla cultura e tradizione, alla paure più recondite agli incubi più terrorizzanti e da migliaia di portali arrivarono orde demoniache, orchi, draghi e abomini come generati dalla mente fervida di qualche pazzo. I morti si sollevarono dalle tombe e presero a cacciare i vivi.

Anche se durarono solo 1 anno le manifestazioni dirette di potere dei Patroni l'umanità perse oltre il 90\% della sua popolazione nel tentativo di difendersi, di sopravvivere.

L'industria, la conoscenze vennero distrutte e nel secolo successivo la barbarie ed ignoranza non ci ha certo aiutato a riprenderci.
Fu l'Editto della Dimenticanza che distrusse tutto. L'Editto emise una potente onda magica che fuse i componenti di ogni apparato elettrico e allo stesso tempo cancellò qualsiasi dato potesse essere li custodito, ma non fu questo il peggio: l'Editto confuse le parole di ogni libro scritto.

Molti degli apparati sono ancora li, dove erano in origine, la maggior parte vandalizzata per recuperare materiali, altri invece chiusi chissà in quale segreto posto. Qualche speranza c'è ancora di trovare un apparato funzionante, pur se remota o magari un libro ancora leggibile.

Le vecchie rovine delle città sono spesso ricettacolo di orde di nefaste creature che aspettano solo di potersi nutrire. Molte zone non sono state rimappate e pur sapendo cosa c'era non si sa più cosa c'è adesso.

Poche sono le città che superano i 50000 abitanti. Ogni stato ha una capitale che per il beffardo destino della Terra molto spesso viene distrutta o scompare. La legge è spesso assente e solo quella del più forte vige.

Ampie lande inesplorate si dipanano dove antichi resti di civiltà scomparse sono rifugio di nuovi abitanti. Ci sono strati su strati di civiltà sepolte sotto i propri piedi con tesori, segreti, caverne e protettori.


\subsection{Società}\index{Società sulla Terra}\label{societasullaterra}

Le dinastie raramente sono destinati a regnare per più di qualche generazione, guerre fratricide, attacchi dall'esterno, voleri di Patroni, fanno che le forme societarie più rigide facciano fatica a prosperare e lo spirito \emph{democratico} non è sempre così sviluppato da permettere di creare società evolute che possano adattarsi alla situazione.

Le nazioni hanno così confini molto labili, spesso definiti dalla geografia più che dalle conquiste. Non sempre gli eserciti possono difenderli da attacchi esterni e ancora più spesso le milizie devono concentrarsi a difendere la città principale da attacchi interni, ribellioni o improvvise orde di mostri usciti da chissà quale impronta di Cattalm.

In tutto il mondo la forma di governo e società più diffusa sono le Città Stato, roccaforti e terreni raccolti attorno ad una città in grado di difenderli e proteggerli dagli assalti esterni, governate da un leader forte con l'appoggio di un Patrono.

Piccoli e grandi villaggi sorgono ovunque nel territorio, attorno a fonti d'acqua e risorse naturali, spesso sono in balia di bande di disperati se non di gablin.
Spesso è qui che i nostri eroi hanno la prima formazione nel tentativo di difendere prima la loro casa, poi il villaggio, dall'assalto di qualche astuto e sanguinario nemico.

Ancora si ergono i resti di magnifiche città del passato e spesso queste tornano ad essere popolate anche se sugli abitanti spesso aleggia la maledizione che ha condannato la Prima Era.

Il sottosuolo possa essere caverne, catacombe o infiniti cunicoli se non vere e proprie città sotterranee sono per tutta la Terra, memoria imperitura, stratificata e ristratificata, della sua storia. Non c'è mai una fine a quanto si può andare in profondità, c'è sempre qualcos'altro sotto di ancora più magnifico e pericoloso.

Le leggende parlano di intere regioni inghiottite sottoterra, città che dal giorno alla notte sono scomparse in una nube di polvere. Ovunque sono presenti accessi alle profondità dove si favoleggiano tesori e ricchezze, dove la Legge del Premio aspetta chi osa raccogliere la sfida.

\subsection{Le Religioni}\index{Le altre Religioni}

Le \emph{vecchie} religioni esistono ancora. I Patroni non ti fanno guerra perché adori un altro Patrono o Dio, non è il loro scopo o interesse. La religione ha acquisito un ruolo molto più concreto e reale, con numerosi estremisti che vanno profetizzando la venuta di altri patroni, dei veri Dei. E' sempre molto pericoloso manifestare il proprio credo, non sai mai come potrebbero reagire gli altri.

Le stessi Devoti o Seguaci se pubblicamente declamano i loro Patroni rischiano a volte il linciaggio, quasi tutti gli usufruitori di magia vengono visti come persone che hanno venduto l'anima al \emph{diavolo}. La fiducia è un bene prezioso che va guadagnato.

\subsection{Avventure}\index{Avventure}\label{avventure}

In un'ambientazione così complessa, le possibilità per le avventure sono praticamente infinite. Ecco alcune idee che potrebbero ispirarti:

\begin{itemize}[leftmargin=*] \setlength{\itemsep}{0pt}
\item I personaggi sono inviati a esplorare le zone devastate e mutate, alla ricerca di risorse, sopravvissuti o antiche tecnologie
\item Le diverse fazioni umane e non umane sono in in guerra per il controllo delle risorse o dei territori. I personaggi potrebbero scegliere una fazione o cercare di mediare la pace
\item Proteggere un villaggio dagli attacchi di entità ostili o mostri evocati dai Portali.
\item I Patroni vogliono manipolare i leader dei paesi per obbligare il loro culto. I personaggi devono sventare complotti e tradimenti.
\item Formare alleanze con i Patroni per ottenere poteri magici o protezione, navigando tra le diverse richieste e aspettative di questi \emph{esseri}.
\item Indagare sull'origine e la natura dell'Omniessenza, cercando di capire come sfruttarne i residui senza causare ulteriori catastrofi.
\item I personaggi sono apprendisti maghi che devono recuperare un qualche incantesimo fondamentale per salvare il loro villaggio/città.
\item Aiutare a ricostruire le comunità devastate, trovando risorse, costruendo infrastrutture e difendendole dalle minacce.
\item Sopravvivere in ambienti mutati e pericolosi, come deserti nucleari o foreste infestate da creature mostruose.
\item Esplorare miti e leggende per scoprire verità nascoste e artefatti potenti, scoprire cosa è rimasto mito e leggenda e cosa è stato creato
\item Utilizzare i Portali per esplorare altri mondi o dimensioni, affrontando le sfide e scoprendo i segreti di questi luoghi.
\item Missioni per salvare gruppi di sopravvissuti intrappolati in zone pericolose o sotto l'influenza di creature maligne.
\item Esplorare eventi inspiegabili e fenomeni paranormali causati dalle \emph{entità} (uscite da portali ?), cercando di scoprire la verità dietro di essi.
\item Rappresentare una comunità o una fazione in missioni diplomatiche presso altri popoli, cercando di ottenere supporto o risorse.
\item Attraverso i Portali è possibile viaggiare nel tempo. Cercare di prevenire catastrofi o correggere errori del passato, affrontando i rischi dei paradossi temporali.
\item Organizzare e guidare una ribellione contro leader/nemici oppressivi, cercando di liberare territori occupati.
\item Cercare di comprendere i Patroni, cercando di ottenere il loro favore.
\item Supportare l'espansione della propria comunità nel territorio, conquistando nuove terre e difendendole dagli attacchi nemici.
\item Proteggere avamposti strategici da assalti nemici, utilizzando tattiche e risorse limitate.
\item Alcune regioni e città hanno subito uno sfasamento temporale, tornando ad epoche antiche.
\item Strani accadimenti stanno succedendo nel territorio, tutte situazioni allineate lungo le linee di Ley.
\end{itemize}

Il \emph{problema} per gli avventurieri ed esploratori è l'estrema diversificazione e mutevolezza non solo dell'ambiente ma anche delle culture e civiltà che si possono incontrare.

La Terra non si potrà più dire esplorata, la stessa zona può cambiare da un giorno all'altro perché un Patrono ha deciso così. Curiosi, ineffabili, volubili sono capaci di costruire in un battito di mani l'avventura della vita solo per godersi lo spettacolo.

Saranno orde di gablin affamati nel dedalo delle profondità della città, saranno orde barbariche devote a Cattalm ad uccidere e rapire la prole, saranno regni di ghoul che spuntati dal nulla vorranno mangiare tutto e tutti, saranno carestie e pestilenze risolvibili sono ritrovando antichi artefatti, potranno essere antiche città spuntate da una impronta di Cattalm, putride paludi in piena espansione cariche di mostri...

I Patroni faranno di tutto per sconfiggerti ed umiliarti ma ricorda bene la Legge del Premio è superiore anche a loro!


\subsection{Nuovi luoghi notevoli sulla Terra}\label{luoghinotevoli}

\subsubsection{Deserto di Kranguran}

In questo immenso deserto si celano giganteschi mostri. Alcuni nascosti sotto la sabbia usano il senso tellurico per cacciare le loro prede, altri giganteschi dinosauri cacciano qualsiasi preda possa capitare.

Ogni creatura in questo deserto è gigantesca, mostruosa e sproporzionata nell'aspetto, quasi fosse nato dall'incubo di qualcuno.

La stessa vegetazione nelle poche oasi presenti è enorme ed ipertrofica. Nessuna creatura che qui nasce può uscire, nessuna creature che qui muore lascia veramente il posto. La leggenda narra che ogni \emph{mostro} in questo deserto sia in realtà una creatura qui morta. A seconda del potere della creatura questa è rinata come più gigantesco mostro affamato.

Secondo i più il Deserto di Kranguran è un luogo di gioco di Cattalm.

\subsubsection{Città di Knandir}

Questa ricca, prosperosa e popolosa città antica fu distrutta nell'arco di una notte da un gigantesco cataclisma.

Si dice che il volere del Patrono fu talmente pervasivo che tutti gli edifici vennero distrutti o severamente danneggiati. Non soddisfatto dell'opera condanno la città a essere sfasata rispetto alla realtà, facendola scomparire agli occhi di tutti gli altri.

I pochi abitanti sopravvissuti perirono tra atroci sofferenze, condannati a non poter uscire, a non aver da magiare o bere.

La città venne maledetta e nei pochi giorni dell'anno in cui è possibile raggiungerla ogni persona che ci mette piede per saccheggiare gli immensi tesori contenuti sembra condannato a non uscire più, vittima della maledizione o dei numerosi fantasmi, spiriti e non morti dei precedenti abitanti.

La città oltretutto non appare mai nello stesso posto ma si sposta seguendo uno schema non ben compreso. Si dice che la Pergamena perduta di Knandir ne spieghi gli spostamenti, un peculiare oggetto vetroso con una mappa che ne indica la posizione con una luce che appare e scompare.

%\subsubsection{Il Mare silente}

%C'è una particolare zona di mare, compresa tra tre isole maggiori e contenente diverse isole minori, dove qualsiasi suono viene silenziato. Un suono che venga generato in quelle acque, e non sulla terra ferma, viene zittito.

\subsubsection{La torre dei gorilla blu}

L'origine di questo antico e magico edificio è ormai dimenticata, si dice che fosse stata creata per sfidare un Patrono, probabilmente Gradh. La torre, a base quadrata di 20 metri di lato è apparentemente alta 7 piani. In ogni piano, la cui mappa sembra essere costantemente mutevole, appaiono dei gorilla blu, assolutamente brutali e con l'intenzione di uccidere chiunque sia nella torre. Una volta sconfitto l'ultimo gorilla del piano la porta che conduce alle scale per il piano successivo si apre e i personaggi possono salire. Ad ogni piano i gorilla diventano più forti, resistenti e più intelligenti.

E' noto che già al 4 piano acquisiscano anche poteri magici. I personaggi entrati possono uscire quando vogliono, se dovessero morire all'interno della torre verranno automaticamente teletrasporati fuori, ma vivi ad 1 Punto Ferita ed estremamente affaticati, senza l'oggetto più prezioso che avevano addosso al momento della morte. Non ci sono oggetti magici all'interno della torre, almeno nei piani noti, l'unica cosa che i personaggi guadagnano è esperienza per i combattimenti fatti. L'attuale record è stato raggiungere il 7 piano. Riusciranno dei nuovi eroi ad arrivare alla fine (???) della torre, e che premi ci saranno per chi sopravvive?

\subsubsection{e la Terra ?}\index{La Terra}

Il nostro mondo esiste e persiste, a discapito di ogni distruzione e tragedia. Per quanto la descrizione fatta fin'ora possa sembrare apocalittica, alcuni luoghi sono ancora quasi integri, piccoli angoli di un paradiso dimenticato.

Non troverete una industria o fabbrica funzionante ma piuttosto una piccola impresa, un valente artigiano, qualcuno che porta avanti le tradizioni familiari arrabattandosi come può.

Molti luoghi d'altronde sono stati svuotati, depredati, parzialmente distrutti, città abitate non più da umani ma da \emph{altro}, metropolitane che sono diventate i nuovi dungeon, catacombe dimenticate che sono tornate ad essere luogo di rifugio, città seppellite da terremoti eppure ancora \emph{abitate}. I massicci rifugi antiatomici, i poderosi condomini sottoterra, hanno resistito eppure hanno dovuto dovuto dividere gli spazi vitali con chi è stato mutato, cambiato.

Si dice che da qualche parte, dove la Freten è esplosa ci sia il portale che conduce agli stessi Patroni della Genesi, quel che è certo è che ovunque ci sono Portali che portano in mondi fantastici, stupendi o mortalmente terrorizzanti.

Molti cercano il nuovo eden saltando da un portale all'altro sperando di trovare il mondo giusto sempre alla ricerca di una avventura o della tranquillità.

Ricordate che la natura è stata mutata, nuove specie ispirate ai manuali dei giochi sono state create per diletto, mostri di ogni genere forma e morale sono presenti solo per farli combattere con i sopravvissuti. Foreste gigantesche, lussureggianti, piene di vita raramente amichevole, sono spuntate in mezzo ai deserti. La grande jungla del Sahara è tra le più temute ed evitate anche se all'interno ci sono immensi giacimenti di minerali preziosi.

E ovunque, Draghi! Innumerevoli, affamati, cattivi.


\subsubsection{I vecchi Stati}

E' impossibile in queste poche righe descrivervi come tutto il pianeta sia stato \emph{riscritto}. La magia dei Patroni è assoluta ed il loro volere Legge, non stupiamoci se quello che era il Deserto del Sahara adesso è la più fitta e lussureggiante jungla del pianeta, conosciuta come Giardino di Shayalia.
Buona parte della zona est della Russia, quella ai confini con gli ex stati dell'est Europa è diventa l'Impero dei Ghoul, uno dei luoghi più terribili dove vivere, se non si è devoti di Sixiser.

Molto del Nord America è un deserto nucleare con le poche popolazioni che si sono rifugiate nelle coste est ed ovest, cacciati da bande di predoni cannibali e mutati sputa acido.

Quello che era il Brasile è diventato una zone totalmente selvaggia in mano a dinosauri e popolazioni folli che adorano Orudjs.

La parte dell'Italia centrale è sotto la teocrazia di Rezh mentre numerosissime contee, baroniee e semplici città si sono auto proclamate sotto la protezione di un qualche signorotto locale.

La Francia è comandata direttamente dal nuovo Re Sole, pardon, Re Torbion XXIII che invaghitosi della storia e cultura ha voluto riproporre, con volere questa volta veramente divino, gli sfarzi ed atteggiamenti di quella corte e periodo, rendendo il tutto tremendamente più pericoloso ed infido.

La Germania, quella che era il motore della vecchia Europa, ha subito tra i danni maggiori, ritornando ad uno stato barbarico, con una involuzione culturale e naturale forzata da Efrem.

Buona parte delle terre tra Francia e Germania sono tornate ad uno spirito più primitivo ed ancestrale, qui Gaya ed Erondil hanno creato i loro culti maggiori ispirati a quella che era la tradizione celtica.

Le fredde terre del nord Europa si sono isolate dopo che i loro morti sono risorti. Questa volta per volontà delle persone è stato chiesto aiuto a Krondal e Nedraf perché li potessero salvare. Nedraf gli diede le armi e l'esperienza per usarle, Krondal, da vero folle fece tornare gli ancestrali ricordi di un passato guerriero fatto di miti e Dei dimenticati, o meglio ignorati, dai più.
Così Krondal ha ricreato come suoi servitori Aegir, Alfadur, Hel, Idhunn, Norne per non citare i più noti Thor, Loki, Valchirie...

\begin{changemargin}{0.3cm}{0.3cm}\begin{narratore} %box narratore
Usate le mappe geografiche fisiche reali terrestri per aiutarvi con l'ambiente. Cercate online le mappe delle antiche città. Avete a disposizione il più grande setting mai creato, si tratta solo di popolarlo con i miti, leggende, storie, fantasia che già sono intorno a voi.

Ogni città ha le sue leggende storiche, scopritele e giocatele insieme!
\end{narratore}\end{changemargin}


\end{multicols}

\subsection{I Portali}\index{Portali}

\begin{changemargin}{0.3cm}{0.3cm}\begin{enfasi}{
Non aprire mai le porte a coloro che le aprono anche senza il tuo permesso. (Stanislaw Jerzy Lec)}\end{enfasi}\end{changemargin}\medskip

\begin{multicols}{2}


Questo proliferare di piccoli, grandi, duraturi o istantanei Portali ha causato uno squarcio nel tessuto dimensionale della Terra generando a sua volta un proliferare di tunnel spontanei più o meno grandi e duraturi.

Questi occasionali Portali saranno spesso la causa di situazioni da affrontare e sconfiggere. Ogni Portale ha una propria DC da superare per poter essere chiuso con l'incantesimo \hyperlink{Chiudi Portale}{Chiudi Portale} (pag. \pageref{Chiudi Portale}), solitamente questa DC va dai 20 per i più facili ai 40 per quelli permanenti.

Ci sono portali conosciuti e stabili, fino ad ora, che collegano continenti. I Portali più interessanti per il commercio sono quasi tutti sotto il controllo, per non dire dentro la roccaforte, di reali e potenti.

Non tutti i Portali sono pericolosi, molti fungono semplicemente da varchi di trasporto da un paese all'altro, possa essere anche a centinaia di kilometri. Altri Portali porteranno in altri mondi, da esplorare e forse vivere, altri invece porteranno in luoghi che è bene evitare e che dovranno essere sigillati e chiusi per evitare ulteriori invasioni di \emph{xenomorfi alieni}.

Potete inventare mille ed una avventura dietro ai Portali, ognuno è una possibilità di un mondo diverso e di una moltitudine di avventure.


\begin{changemargin}{0.3cm}{0.3cm}\begin{narratore} %box narratore
Dovete intendere i portali come chiave per mille ed una avventura. Ogni portale vi condurrà in un posto diverso, fantastico come voi lo intendete. Volete un avventura in un mondo primitivo, ambientata nella società moderna, in un pianeta chissà dove? Usate i portale per spalancare le porte della vostra immaginazione.

Gli stessi personaggi potrebbero essere non \emph{terrestri} e cercare un modo per tornare a casa...
\end{narratore}\end{changemargin}


\begin{changemargin}{0.3cm}{0.3cm}\begin{narratore}
Usate l'ambientazione che più preferite! Questo mondo è un esempio di un mondo caotico e leggermente anarchico dominato dai continui cambiamenti di umori di divinità capricciose.
Scegliete voi l'ambientazione, usate Greyhawk, Dark Sun, Mystara quello che preferite. Siete voi il Narratore, siete voi il mondo, siete voi a proiettare luce ed oscurità, OBSS vi fornirà gli strumenti per condurre le vostre campagne!

%Il primo suggerimento che vi do è di conoscere bene l'ambientazione, maggiore sarà la vostra conoscenza, più facilmente saprete adattarvi alle situazione che vi andranno a capitare.
\end{narratore}\end{changemargin}

%\begin{center}
%	\includegraphics[width=0.7\linewidth]{immagini/ancientwell.png}
%
%	\emph{Antico pozzo e portale}
%\end{center}

\end{multicols}




%\vfill



%\begin{changemargin}{0.3cm}{0.3cm}\begin{enfasi}{
%Una volta eliminato l'impossibile ciò che rimane, per quanto improbabile, dev'essere la verità. (Sir Arthur Conan Doyle)
%} \end{enfasi}\end{changemargin}


%Tàhil rosso
%Elysan argento
%Curyan vita
%Tiya scuro


\pagebreak

\subsection{Il Calendario}\index{Calendario}

\begin{changemargin}{0.3cm}{0.3cm}\begin{enfasi}{
Mi è capitato spesso di finire su un calendario. Ma mai per una data precisa. (Marilyn Monroe)

\medskip

Tutto ebbe inizio la tredicesima ora del tredicesimo giorno del tredicesimo mese... Eravamo lì per discutere degli errori di stampa dei calendari acquistati dalla scuola. (I Simpson)} \end{enfasi}\end{changemargin}

\medskip


\begin{multicols}{2}

\label{il-calendario}

Basato sul ciclo lunare presenta 12 mesi da 28 giorni.

Questi i nomi dei mesi a partire da quello che si definisce inizio anno:

1°) Ianas (stagione: primavera)

2°) Prineva (stagione: primavera)

3°) Marc (stagione: estate)

4°) Epral (stagione: estate)

5°) Meea (stagione: estate)

6°) Vernam (stagione: autunno)

7°) Ilai (stagione: autunno)

8°) Arkast (stagione: autunno)

9°) Cester (stagione: inverno)

10°) Koper (stagione: inverno)

11°) Narava (stagione: inverno)

0°) Raanant* (speciale)

12°) Kartan (stagione: primavera)

\medskip

\emph{Raanant} è il mese che si festeggia alla fine del Ciclo secolare, ogni cento anni. E' un mese di libertà dai Patroni, dalle Leggi, è il mese della catarsi e della violenza, della libertà e della rinascita.

\medskip

Un ciclo di sette giorni, settimana, è composta da giorni di nome:

\medskip

1°) Kalint (solitamente festivo)

2°) Iratam

3°) Munrat

4°) Arai

5°) Venran

6°) Kittam

7°) Viltar

\medskip

Il giorno è diviso in 24 ore. L'anno corrente è il 125 del nuovo calendario.

\subsection{Oltre la morte}

Gli abitanti hanno una visione abbastanza pessimistica di ciò che succede dopo la morte. Per i più dopo la morte non c'è nulla se non la dissoluzione del corpo.

I Devoti e Seguaci credono che il loro spirito si riunirà con il Patrono, rendendolo più forte.

Altri ancora credono ancora che ogni spirito si incarni per 4 volte per essere poi giudicato dai Patroni della Genesi e mandato nel piano a lui assegnato.

Quale sia la verità non è dato saperlo.

\end{multicols}

\vfill

\begin{center}
\includegraphics[width=0.5\linewidth]{immagini/Laminas_8_y_9_del_Codice_de_Dresden2.png}

\medskip

\emph{Codice di Dresda, pagine 10 e 11}
\end{center}


\pagebreak



\subsection{I Cicli Secolari}\index{I cicli secolari}

\begin{changemargin}{0.3cm}{0.3cm}\begin{enfasi}{
Vidi poi un angelo che scendeva dal cielo con la chiave dell'Abisso e una gran catena in mano.

Afferrò il dragone, il serpente antico - cioè il diavolo, satana - e lo incatenò per mille anni; lo gettò nell'Abisso, ve lo rinchiuse e ne sigillò la porta sopra di lui, perché non seducesse più le nazioni, fino al compimento dei mille anni. Dopo questi dovrà essere sciolto per un pò di tempo. (Apocalisse 20,1-3, apostolo Giovanni)
}\end{enfasi}\end{changemargin}\medskip

\begin{multicols}{2}

Dice il mito che ogni cento anni la Terra muoia per rinascere nuovamente, più bella di prima. Non è proprio così ma ci si avvicina molto.

E' noto a pochi eruditi di Atmos che ogni secolo i Patroni riconosciuti, e da cui molti traggono i poteri, scompaiano e lascino il posto, dopo esattamente 1 anno a nuovi Patroni.

Improvvisamente gli incantesimi cessano di funzionare, solo gli oggetti magici che possono assorbire e conservare la magia funzionano (come ad esempio una Pozione, una Armatura o Arma se non un Anello od un Bastone che abbia delle cariche, ma non oggetti che si ricaricano automaticamente come le Verghe), neanche i Devoti o Seguaci hanno più accesso a nessun incantesimo.

\begin{wrapfigure}[20]{r}[.5\width+.5\columnsep]{7cm}

\centering
\includegraphics[width=6cm]{immagini/Aztec_calendar.png}

\medskip

\emph{Antico calendario Azteco}
\end{wrapfigure}

Con qualche eccezione. I Patroni della Genesi, Atmos e Lynx ed il Patrono Vincitore sono gli unici a rimanere costanti e non cambiare. Solo i loro Devoti e Seguaci possono continuare ad usare gli incantesimi a disposizione nell'anno di intermezzo.

A partire dal sesto mese i Seguaci e Devoti dei precedenti patroni incominciano a sentire delle voci, a sognare nuovi volti e nomi di nuovi Patroni.

Questo è quello che successe alla fine della prima venuta, con la sola differenza che Ljust concesse la vittoria a Calicante per poter interrompere immediatamente il ciclo e salvare dalla distruzione il nostro mondo dalla vendetta per aver ucciso il Primo Patrono.\index{Il primo ciclo}\index{I primi Patroni}

Ogni nuovo Patrono, in base ai Tratti che comanda, avvicina un Seguace o Devoto e cerca di convincerlo ad accettarlo come nuovo Patrono. Gli incantatori solo al termine dell'anno potranno usare gli incantesimi, indipendentemente che seguano un Patrono o meno.

E' un periodo estremamente turbolento ed agitato dove scoppiano guerre e vendette approfittando dell'assenza della magia. Per molti è un periodi di puro odio e violenza dove vengono sfogati gli istinti più bassi sapendo poi che non si sarà giudicati da nessun Patrono.

La verità è che ogni cento anni i Patroni delle Genesi giudicano i loro figli, i Patroni, valutando chi ha fatto meglio e chi peggio. E' una sfida tra Calicante ed Ljust a chi ha, tramite i Patroni, ottenuto più Seguaci e Devoti.

Il Patrono che più di tutti si è dimostrato capace di conquistare più persone rimarrà anche nel secolo successivo, questo sarà il Vincitore ed i suoi credenti ne canteranno per altri cento anni la gloria e la potenza.

Inebriato dalla vittoria il Patrono della Genesi esprimerà un desiderio che l'altro dovrà cercare di rispettare il più possibile.

Ovvio che il Patrono stesso potrebbe soddisfarlo ma la gioia di obbligare l'altro a fare qualcosa

\begin{wrapfigure}[20]{l}[.5\width+.5\columnsep]{7cm}

	\centering
\end{wrapfigure}

che detesta è superiore a ogni cosa. Ed è per questo che ogni cento anni succede l'impossibile, oltre alla nascita di nuovi Patroni.

Può essere un nuovo continente, un mare che si apre tra le terre, nuove razze, animali... qualcosa di imponente cambia per tutti i terrestri. E' un periodo di sconvolgimenti globali.

Solo i sommi Devoti di Atmos conoscono questa verità come conoscono che i Patroni della Genesi dopo la vittoria giacciono insieme per sei mesi generando i nuovi Patroni.

Un altra verità sconosciuta purtroppo è che in realtà il nostro pianeta è sotto il gioco dei Patroni da molto più di un secolo e che è solo per desiderio di Ljust che non si ha memoria di tutti i cicli precedenti. La Patrona della Luce per non fare perdere la speranza all'umanità ha ottenuto che ci dimenticassimo dei secoli di soprusi causati dalle vittorie continue dei Patroni di Calicante, dalle distruzioni perpetrate dai draghi e mantenessimo una flebile e vitale speranza in un mondo che possa essere più gentile e amorevole verso tutte le sue creature.

Traccia di questi cicli passati si possono trovare negli innumerevoli ed altrimenti ingiustificati pericoli, dungeon, mostri, draghi, città sotterrane che riempiono ogni angolo della Terra.

\end{multicols}


\pagebreak

