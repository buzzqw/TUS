\section{Punti Ferita}\index{Punti Ferita}\index{Punti Ferita}\index{PF}\label{puntiferita}

\begin{enfasi}{Chi non stima la vita, non la merita. (Leonardo da Vinci)}\end{enfasi}

\medskip

I Punti Ferita rappresentano l'energia vitale del personaggio ma anche l'abilità, fortuna, la capacità del personaggio di resistere e combattere. Finché il personaggio/avversario ha almeno 1 Punto Ferita (PF) combatterà e lotterà al meglio delle sue capacità.

\begin{itemize}[leftmargin=*] \setlength{\itemsep}{0pt}
\item Ogni personaggio parte con 8 Punti Ferita al primo livello + il punteggio della Costituzione.

\item Ad ogni livello, oltre il primo, guadagna 1d6 Punti Ferita + il punteggio della Costituzione. Se il tiro di dado è inferiore a Costituzione, può prendere come risultato il valore di Costituzione.\index{Punti Ferita minimi}

\item Ogni punto preso in Competenza Armi aumenta i Punti Ferita presi di 3. Ulteriori Abilità possono aumentare i Punti Ferita.
\end{itemize}

Segna nella scheda i Punti Ferita massimi che hai ed indica il valore attuale di volta in volta che ne perdi o recuperi. Segna sulla scheda sempre qual é l'ammontare dei Punti Ferita attuale, dopo ogni danno subito. I Punti Ferita Massimi sono l'ammontare di Punti Ferita quando il personaggio è \emph{perfettamente sano}.

%\begin{tcolorbox}[title = Sto per morire!]\index{Sto per morire!}
%SCAPPA! Ritirati, nasconditi, esci dal combattimento. Non c'è gloria nell'essere morto. Meglio una ritirata che un TPK (Total Party Kill ovvero morte di tutto il gruppo).
%\end{tcolorbox}\end{changemargin}

\medskip

I \textbf{Punti Ferita si recuperano} in diversi modi:\index{Recupero Punti Ferita}\label{recuperarepf}

\begin{itemize}[leftmargin=*] \setlength{\itemsep}{0pt}

\item per ogni notte di riposo (almeno 8 ore) recuperi in Punti Ferita il valore di Costituzione*Livello, con un minimo di PF pari a Livello. \index{Recupero PF dormendo}

\item tramite magie curative (incantesimi, pozioni.. o altri effetti magici)

\item competenza \hyperlink{prontosoccorso}{Pronto Soccorso} (pag. \pageref{prontosoccorso}), tramite trattamenti più o meno lunghi

\end{itemize}

I \textbf{Punti Ferita} possono essere anche \textbf{temporanei}\index{Punti Ferita Temporanei} ovvero aggiunti o tolti temporaneamente ai tuoi attuali.

\noindent\begin{itemize}[leftmargin=*] \setlength{\itemsep}{0pt}

\item Una magia che conceda +10 Punti Ferita temporanei alzerà i Punti Ferita attuali di 10, se subisci 8 di danno ti rimarranno 2 Punti Ferita temporanei. Se invece subisci 13 di danno oltre a perdere tutti i Punti Ferita temporanei subirai anche 3 Punti Ferita \emph{normali}.

\item Quando ottieni Punti Ferita temporanei devi scegliere se l'effetto sostituisce il precedente. I Punti Ferita temporanei non si cumulano e non possono essere superiori alla metà dei Punti Ferita massimi. Una magia di cura fa recuperare i PF normali, non i PF temporanei persi.

\item Al termine dell'effetto che concede Punti Ferita temporanei questi scompaiono facendo rimanere la creatura ai suoi Punti Ferita.

\item Se non esplicitato diversamente i Punti Ferita Temporanei scompaiono dopo un ora da quando si sono aggiunti.

\item I Punti Ferita Temporanei vanno tolti per primi quando si viene feriti.

\end{itemize}

Un'arma o effetto che causa danni non letali vuole dire che causa \hyperlink{recuperopuntiferitanonletali}{ferite temporanee}\label{feritetemporanee}.
