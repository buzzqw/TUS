\section{Equipaggiamento - Armi}\index{Equipaggiamento}\index{Armi}\label{equipaggiamentoarmi}
\hypertarget{equipaggiamento.armi}{}

\label{equipaggiamento---armi}
\begin{enfasi}{
Questo è il mio fucile. Ce ne sono tanti come lui, ma questo è il mio. Il mio fucile è il mio migliore amico, è la mia vita. Io debbo dominarlo come domino la mia vita. Senza di me il mio fucile non è niente; senza il mio fucile io sono niente. Debbo saper colpire il bersaglio, debbo sparare meglio del mio nemico che cerca di ammazzare me, debbo sparare io prima che lui spari a me e lo farò. Al cospetto di Dio giuro su questo credo: il mio fucile e me stesso siamo i difensori della patria, siamo i dominatori dei nostri nemici, siamo i salvatori della nostra vita e così sia, finché non ci sarà più nemico ma solo pace, amen. (Full Metal Jacket, Film, 1987)

\medskip

La spada davvero buona è quella che rimane nel suo fodero. (Sanjuro)}\end{enfasi}

\medskip

Usare un'Arma senza l'adeguata competenza impone un -1d6 al colpire

La tabella presenta il nome dell'arma, il suo costo in monete d'oro, il danno ed il tipo di danno (se da Taglio, Contundente o Perforante), la gittata, la Lista d'Arma appartenente e le caratteristiche speciali che può avere. Vedi anche \hyperref[sec:capacita-di-carico-e-trasporto-ingombro]{Capacità di Carico e Trasporto.}

\medskip

\textbf{Tabella: Lista della Armi}\index[Tabelle]{Tabella Lista della Armi}

\noindent\begin{xltabular}{\linewidth}{lllX}
	\toprule
\rowcolor{gray!20}\textbf{Arma}&\textbf{Costo}&\textbf{Dim./Danno} & \textbf{Gittata, Lista, Speciale}\\
\toprule
Alabarda& 10 & G/1d10 P/T& \textbf{Lance}, \textbf{Aste}, Controcarica, Arma lunga, ED9 \\
\rowcolor{gray!20}Arco corto composito& note*& M/Frecce& 20 metri, \textbf{Archi}\\
Arco corto& 30 & M/1d6 P& 15 metri, \textbf{Archi}\\
\rowcolor{gray!20}Arco lungo composito& note*& G/Frecce& 36 metri, \textbf{Archi}\\
Arco lungo& 75 & G/Frecce& 20 metri, \textbf{Archi}\\
\rowcolor{gray!20}Ascia martello& 16 & M/1d6 T/C& \textbf{Scuri e Accette}\\
Ascia ad una mano& 6 & M/1d6 T& 6 metri, \textbf{Scuri e Accette}, \textbf{Armi da Lancio}, Versatile\\
\rowcolor{gray!20}Ascia da battaglia& 10 & G/1d10 T&\textbf{Scuri e Accette}\\
Balestra ad una mano& 100& M/Dardi& 6 metri, \textbf{Balestre}\\
\rowcolor{gray!20}Balestra leggera& 35 & P/Dardi& 15 metri, \textbf{Armi Semplici}, \textbf{Balestre}\\
Balestra pesante& 50 & G/Dardi& 30 metri, \textbf{Balestre}\\
\rowcolor{gray!20}Bastone& 3& M/1d6 C& \textbf{Armi Semplici}, Arma lunga, Versatile, Parata\\
%Brandistocco& 10 & M/2d4 P/T& \textbf{Lance}, Controcarica, Arma lunga\\
Catena chiodata& 25 & G/2d4 P& 3 metri, \textbf{Palle rotanti}, Arma lunga\\
\rowcolor{gray!20}Estoc& 25& G/1d8 P& \textbf{Spade}, Arma lunga, Parata\\
Falce& 18 & G/2d4 P/T& \textbf{Armi della Morte}, Arma lunga\\
\rowcolor{gray!20}Falcetto& 6& P/1d6 T& \textbf{Armi della Morte}\\
Falcione in asta& 12 & G/1d10 P/T& \textbf{Lance}, Controcarica, Arma lunga, ED9\\
\rowcolor{gray!20}Falcione& 75 & M/2d4 T& \textbf{Armi Aggraziate}, ED7\\
Fionda& -& P/1d4 B& 10 metri, \textbf{Armi da lancio}\\
\rowcolor{gray!20}Flagello doppio& 90 & G/1d10 C& \textbf{Palle Rotanti}, \textbf{Armi doppie}\\
Flagello pesante& 15 & G/1d10 C& \textbf{Palle Rotanti}\\
\rowcolor{gray!20}Flagello& 8& M/1d8 C& \textbf{Palle Rotanti}, \textbf{Rompi Cranio}\\
Frusta& 1& M/1d3 T& \textbf{Palle Rotanti}, Arma lunga\\
\rowcolor{gray!20}Giavellotto& 1& P/1d6 P& 12 metri, \textbf{Armi Semplici}, \textbf{Aste}, \textbf{Armi da Lancio}\\
Grande ascia doppia& 25 & G/1d10 T& \textbf{Scuri e Accette}, \textbf{Armi doppie}\\
%Grosso randello& 2& M/1d8 C&\textbf{Rompi Cranio}\\
\rowcolor{gray!20}Guanto chiodato& 5& P/1d4 P&\textbf{Armi da Stordimento}, non letale\\
Katana& 300& M/1d10 T& \textbf{Armi letali}, ED9\\
\rowcolor{gray!20}Lancia da fante& 2& M/1d8 P&3 metri, \textbf{Lance}, Arma lunga, Controcarica\\
Lancia& 10 & G/1d10 P&\textbf{Lance}, Arma lunga, Controcarica\\
\rowcolor{gray!20}Machete& 10 & M/1d6 T&\textbf{Armi letali}\\
Maglio da guerra& 7& G/1d10 C& \textbf{Rompi Cranio}\\
\rowcolor{gray!20}Manganello& 1& P/1d6 C& \textbf{Armi da stordimento}, non letale\\
Martello da guerra& 5& M/1d8 C/P& 6 metri, \textbf{Rompi Cranio}\\
\rowcolor{gray!20}Mazza leggera& 3& P/1d6 C/T& \textbf{Armi Semplici}, \textbf{Armi Leggere}, \textbf{Rompi Cranio} \\
Mazza flangiata& 5& M/1d8 C/T& \textbf{Rompi Cranio}\\
\rowcolor{gray!20}Mazza chiodata& 6& M 1d8 C/P& \textbf{Armi Semplici}, \textbf{Rompi Cranio}\\
%Naginata& 8& G/1d10 T&\textbf{Lance}, Arma lunga, ED9\\
Picca leggera& 4& M/1d4 P&\textbf{Armi della morte}\\
\rowcolor{gray!20}Picca pesante& 8& G/1d6 P&\textbf{Armi della morte}, Arma lunga\\
Pugnale& 2& P/1d4 P& 6 metri, \textbf{Armi Semplici}, \textbf{Armi leggere}, \textbf{Armi da Lancio}\\
\rowcolor{gray!20}Pugno/Calcio & note*& P/1d4 C&Versatile\\
%Randello& 1& P/1d6 C& \textbf{Armi Semplici}, \textbf{Rompi Cranio}\\
Scimitarra& 15 & M/1d6 T&\textbf{Armi Leggere}, \textbf{Armi Aggraziate}, Versatile\\
\rowcolor{gray!20}Spada corta& 10 & P/1d6 P&\textbf{Armi Leggere}, \textbf{Spade}, Versatile, Parata\\
Spada lunga& 15 & M/1d8 T&\textbf{Spade}, Parata\\
\rowcolor{gray!20}Spada a due lame& 100& G/1d8 T& \textbf{Armi doppie}, \textbf{Spade}, Parata\\
Spada bastarda& 35 & M/1d8 T&\textbf{Spade}, Parata, 1d8 ad una mano, 2d6 a 2 mani\\
\rowcolor{gray!20}Spada larga& 12 & M/2d4 T&\textbf{Spade}, Parata, 2d4 ad una mano, 1d10 a 2 mani\\
Spadone a due mani& 50 & G/2d6 T&\textbf{Spade}, Parata\\
\rowcolor{gray!20}Stocco& 20 & P/1d6 P& \textbf{Armi Leggere}, \textbf{Armi Aggraziate}, Versatile\\
Tridente& 15 & M/1d6 P/T& 3 metri, \textbf{Aste}, \textbf{Armi da Lancio}, Arma Lunga, Controcarica\\
\rowcolor{gray!20}Urgrosh& 18 & M/1d6 T/P& \textbf{Lance}, \textbf{Armi doppie}\\
\end{xltabular}

\medskip

Un \textbf{Arma} Piccola ha \textbf{Ingombro} 1, una Arma Media ha Ingombro 2, un Arma Grande ha Ingombro 4, un Arma Enorme ha Ingombro 8.\index{Ingombro Armi}\index{Ingombro Armi}

\medskip

\textbf{Tabella: Lista dei proiettili - Archi - Balestre - Fionde}\index[Tabelle]{Tabella Lista dei proiettili - Archi - Balestre - Fionde}\label{proiettili}

\noindent\begin{tabular}{lcc}
	\toprule
\rowcolor{gray!20}\textbf{Nome Proiettile}& \textbf{Numero di colpi/Costo (mo)} & \textbf{Danno / Tipo}\\
\toprule
Dardi da balestra (una mano, leggera) & 10/1 mo & 1d6 P\\
\rowcolor{gray!20}Dardi per balestra (pesante) & 3/1 mo & 1d10 P\\
Frecce da caccia (Arco Corto, Arco Lungo)& 20/1 mo & 1d6 P\\
\rowcolor{gray!20}Frecce da guerra (Arco Lungo)& 10/1 mo & 1d8 P\\
Biglie di Marmo (fionde)& 15/1 mo & 1d4 B\\
\rowcolor{gray!20}Sasso (fionde)& -& 1d2 B
\end{tabular}

\medskip

Una \textbf{Faretra} (piena o vuota) di Proiettili (Frecce o Dardi) ha \textbf{Ingombro} 2.\index{Ingombro Proiettili}

Un \textbf{dardo pesante} per Balestra penetra più facilmente le armature di metallo causando +2 danni aggiuntivi.\index{Quadrello da Balestra pesante}\index{Balestra pesante}

\begin{multicols}{2}

Un Arma +1 costa 1500 mo, +2 5000 mo. Non è possibile acquistare armi con incantamenti superiore a +2, devono essere trovate.

Una Freccia/Dardo/Sasso magico con un bonus +1 costa 25 mo, se +2 costa 100 mo. Proiettili con bonus magico superiori a +2 sono quasi impossibili da trovare.

\textbf{Un proiettile non acquisisce proprietà magiche perché il suo lanciatore è magico.}

\medskip

\textbf{Pugno Vuoto}: \hyperlink{pugnovuoto}{vedi Lista d'Armi}

\medskip

\textbf{Arco Composito}\index{Arco Composito}
Un arco composito è un arco particolarmente robusto e rigido che richiede un certo minimo di Forza per essere usato efficacemente.
Un \textbf{arco composito} lungo ha un modificatore fisso, da +1 a +5, il bonus si applica solo al danno e non al Tiro per Colpire. Un arco composito applica al danno un bonus pari al \textbf{minimo valore tra Forza ed il suo bonus}.

Un arco composito +3 usato da un personaggio con Forza 2 non può essere tirato completamente e quindi la freccia che parte avrà un modificatore al danno di +2.
Un arco composito +1 usato da un personaggio con Forza 4 può essere tirato completamente e quindi la freccia che parte avrà un modificatore al danno di +1.

Il costo di un arco composito dipende dal sul modificatore.
Un arco composito con modificatore di +1 costa 75 mo, +2 150 mo, +3 300 mo, +4 600 mo, +5 1500 mo. Non è possibile acquistare archi compositi con bonus superiori a +3, devono essere \emph{trovati}.

Un arco corto composito ha come massimo modificatore di Forza +3.

\textbf{Balestra}\index{Balestre}\index{Ricarica Balestra}
Una balestra pesante richiede due Azioni per essere ricaricata. Una balestra leggera od a una mano richiede 1 Azione per essere ricaricata.

\textbf{Gittata}\index{Gittata}\index{Tirare lontano}
La distanza indicata è quello a pieno Tiro per Colpire. Ogni arma a distanza può colpire entro tre volte la distanza indicata.

Se il target è entro la distanza indicata non si hanno penalità al colpire, se il target è tra il primo e secondo incremento la penalità al colpire è -1d6. Se il target è tra il secondo è terzo incremento la penalità al colpire è di -2d6.

Un giavellotto tirato entro 12 metri non ha penalità, ma tirato entro 24 metri ha un -6 al colpire, a distanza tra 24 e 36 metri un -12 al colpire, oltre non può essere tirato.

Un \textbf{Proiettile che colpisce si considera distrutto}, se manca ha un 50\% (4-5-6 su un d6) di probabilità che sia ancora integra.

Un Proiettile magico somma i suoi bonus a quelli del lanciatore per determinare il Tiro per Colpire ed il Danno.

La \textbf{Dimensione dell'Arma}\label{dimensionediunarma}\hypertarget{dimensionediunarma}{} è indicata come P (piccola), M (media), G (grande) ed è riferito ad una creatura media. \hyperref[armatroppogrande]{Vedi sezione Arma troppo grande}

Una \textbf{arma di dimensione superiore} \index{Arma di dimensione superiore} come ad esempio una Spada Lunga forgiata per un Ogre aumenta di una categoria il suo dado di danno.

Le Armi hanno indicato una \textbf{Tipologia di danno}\index{Tipologia di danno}, ovvero T/C/P.

Queste lettere stanno ad indicare se il danno è di tipo Taglio, Contundente o da Perforazione. Questa caratteristica può essere importante perché determinate creature possono essere immuni o subire meno danno da un particolare tipo di ferita (es uno scheletro contro un'arma da perforazione o un cubo gelatinoso contro un arma da taglio..).

Un arma può essere usata per causare un tipo di danno diverso (da taglio a perforazione o contundente) riducendo di una categoria il dado di danno (es. Spada Lunga per fare danno contundente causa 1d6).

\medskip

\begin{center}
	\includegraphics[width=0.7\linewidth]{immagini/bow2.png}
\end{center}

\medskip

\textbf{Armi Perfette}\index{Armi Perfette}

Un arma perfetta è un arma creata da un abilissimo armaiolo che pur non essendo magica, grazie al suo perfetto bilanciamento ed affilatura, ha un +1 al Tiro per Colpire.

Un armaiolo per creare un arma perfetta deve superare con un Successo Critico la DC impostata per la creazione dell'arma (DC = 12 + Ingombro dell'arma).\index{Armi Perfette}

Un arma perfetta costa il doppio di un arma normale.

\textbf{Armi Improvvisate}\index{Armi Improvvisate}\label{armaimprovvisata}\hypertarget{armaimprovvisata}{}

Talvolta oggetti che non sono stati creati per essere armi possono avere una certa efficacia in combattimento. Dal momento che non si tratta di oggetti pensati per questo utilizzo, la creatura che attacca con uno di essi subisce una penalità -1d6 al Tiro per Colpire. Un'arma improvvisata di piccole dimensioni (bottiglia) fa 1d3 di danno, di medie dimensioni (la gamba di una sedia) da 1d6, di grandi dimensioni (la gamba di un tavolo) fa 1d8 di danno.

Un'arma da lancio improvvisata ha una gittata 3 metri.

\medskip

\textbf{Lanciare armi}\index{Lanciare armi}

Una spada o comunque un arma non fatta per essere lanciata può comunque essere scagliata contro l'avversario. Il Tiro per Colpire prende un -1d6 e l'arma fa una categoria di danno inferiore (la spada lunga fa 1d6, una spada corta 1d4..). La gittata è 3 metri.

\medskip

\textbf{Usare un'Arma senza l'adeguata competenza se non è un Arma Semplice} Impone un -1d6 al Tiro per Colpire.

\textbf{Esempio}: Una creatura piccola che usa un alabarda in combattimento ravvicinato ha -1d6 perché l'arma è grande, -1d6 perché non è competente, -1d6 perché usa l'arma in mischia.

In questo caso essendo le penalità superiori ai 3d6 il personaggio non tira dadi ma usa solo la sua Competenza Armi e Forza come valore per colpire.

\subsubsection{Le armi antiche}\index{Armi Antiche}\index{Revolver}\index{Shotgun}\index{Fucile}\index{Armi da fuoco}

E' possibile trovare ancora delle armi antiche funzionanti, armi che dopo 100 anni ancora possono essere usate.

La maggior parte delle armi da fuoco dopo un lasso di tempo così lungo richiedono pezzi di ricambio ed una continua manutenzione. Questi pezzi di ricambio sono molto rari da trovare integri ed ancora più difficile è trovare un artigiano che sappia farli.

Le armi che potrete trovare funzionanti sono i revolver, gli shotgun, i fucili semi automatici ed i fucili automatici.

\medskip

\begin{description}[noitemsep, topsep=0pt, parsep=0pt, partopsep=0pt, leftmargin=0cm, labelwidth=2cm]
\item[\textbf{Revolver}]
\item[\textbf{Azioni:}] 1 Azione per un singolo colpo sparato
\item[\textbf{Caricatore:}] 6 proiettili
\item[\textbf{Gittata:}] 12 metri
\item[\textbf{Danno:}] 1d10 (P) danni a proiettile
\item[\textbf{Regole:}] è necessario un Tiro per Colpire con armi a distanza.
\end{description}

\medskip

\begin{description}[noitemsep, topsep=0pt, parsep=0pt, partopsep=0pt, leftmargin=0cm, labelwidth=2cm]
	\item[\textbf{Shotgun}]
	\item[\textbf{Azioni:}] 1 Azione per un singolo colpo sparato
	\item[\textbf{Caricatore:}] 4 proiettili
	\item[\textbf{Gittata:}] cono 6 metri
	\item[\textbf{Danno:}] 2d8 (P) danni
	\item[\textbf{Regole:}] le creature nel cono possono effettuare un Tiro Salvezza Riflessi contro il tuo Tiro per Colpire. In caso di Successo o Successo Critico il danno è dimezzato, in caso di Fallimento Critico il danno è raddoppiato. In caso di fallimento il danno è normale.
\end{description}

\medskip

\begin{description}[noitemsep, topsep=0pt, parsep=0pt, partopsep=0pt, leftmargin=0cm, labelwidth=2cm]
	\item[\textbf{Fucile Semi-automatico}]
	\item[\textbf{Azioni:}] 1 Azione per 3 colpi sparati
	\item[\textbf{Caricatore:}] 21 proiettili
	\item[\textbf{Gittata:}] 18 metri
	\item[\textbf{Danno:}] 1d8 (P) danni a proiettile
	\item[\textbf{Regole:}] è necessario un Tiro per Colpire con armi a distanza ogni 3 colpi sparati. +1 al Tiro per Colpire per Azione in cui si spara sempre al medesimo obiettivo.
\end{description}

\medskip

\begin{description}[noitemsep, topsep=0pt, parsep=0pt, partopsep=0pt, leftmargin=0cm, labelwidth=2cm]
	\item[\textbf{Fucile Automatico}]
	\item[\textbf{Azioni:}] 1 Azione per 6 colpi sparati
	\item[\textbf{Caricatore:}] 30 proiettili
	\item[\textbf{Gittata:}] 12 metri
	\item[\textbf{Danno:}] 1d6 (P) danni a proiettile
	\item[\textbf{Regole:}] è necessario un Tiro per Colpire con armi a distanza ogni 6 colpi sparati. -1 al Tiro per Colpire per Azione in cui si spara sempre al medesimo obiettivo.
\end{description}

\medskip

\textbf{Regola generale per i fucili} semi-automatici ed automatici: va a segno 1 proiettile per per differenza tra Tiro per Colpire e Difesa avversario. Es. Tiro per Colpire 16 e Difesa 14, la differenza è 2; in quella scarica di proiettili sono andati a segno 2 colpi. Non possono andare a segno più proiettili dei colpi sparati per Azione.

\textbf{Regola sui proiettili}: ogni arma usa dei proiettili diversi. Non puoi utilizzare i proiettili del revolver su un fucile semi automatico o quelli del fucile automatico su uno shotgun o fucile semi automatico.

\index{Proiettili} Inserire un nuovo caricatore o caricare un arma usa 2 Azioni.\index{Ricaricare armi da fuoco}

\textbf{Tiro Critico}: per ogni Tiro Critico ottenuto si considera un danno da singolo proiettile in più.

\textbf{Lista d'Armi}: le armi antiche sono armi improvvisate a meno di creare una Lista d'Armi da Fuoco ed assegnargli almeno 1 punto.

\subsubsection*{Proiettili}\index{Proiettili}

I proiettili sono la cosa in assoluto più difficile da trovarsi. Nessun proiettile veniva costruito con l'idea di essere sparato 100 anni dopo la sua creazione.
La polvere da sparo si è inumidita, ha perso la carica, la camicia di metallo si è corrosa con il tempo.. ci sono tantissimi fattori che rendono i proiettili estremamente rari, quasi e più delle armi magiche.

Un eventuale costo non sarebbe inferiore alle 30 mo a proiettile.

\subsubsection*{Problemi di fuoco}\index{Problemi di fuoco}\index{Inceppamento armi da fuoco}

Ogni qual volta il Tiro per Colpire sia un Fallimento Critico c'è stato un problema con l'arma e non ha sparato con successo.

\medskip

\textbf{Tira e somma 2d10, consulta la tabella}

\medskip

\noindent\begin{tabularx}{\linewidth}{lX}
	\toprule
\rowcolor{gray!20}\textbf{\#}& \textbf{Effetto}\\
\toprule
2 & Il proiettile è difettoso, per fortuna non ci sono altri problemi. Costa una Azione togliere il proiettile inceppato.\\
\rowcolor{gray!20}3 & Il proiettile si è incastrato. Costa due Azioni togliere il proiettile inceppato.\\
4 & L'aver mancato ti lascia esitante, perdi 1 Azione.\\
\end{tabularx}
\noindent\begin{tabularx}{\linewidth}{lX}
\rowcolor{gray!20}\textbf{\#}& \textbf{Effetto}\\
\toprule
5 & Il mirino è impreciso. Il prossimo colpo ha -2 al Tiro per Colpire.\\
\rowcolor{gray!20}6 & Spari inavvertitamente due colpi/raffiche. Il secondo non colpisce nessuno e fa solo consumare proiettili.\\
7 & L'arma non è adeguatamente lubrificata. Inserire il prossimo caricatore costa 1 round intero.\\
\rowcolor{gray!20}8 & Il caricatore si sgancia/cade. Devi caricare un nuovo caricatore (2 Azioni) o recuperare e rimettere il precedente caricatore.\\
9 & Il rinculo è così forte che il personaggio cade a terra prono.\\
\rowcolor{gray!20}10 & Proiettile molto incastrato. Per liberare il colpo devi eseguire una prova di Artigianato DC 15, 1 round.\\
11 & L'arma si surriscalda e non può essere usata fino alla fine del prossimo round.\\
\rowcolor{gray!20}12 & L'arma emette una fitta coltre di fumo attorno a te che fornisce copertura leggera verso te e da te verso gli altri.\\
13 & Parziale ostruzione della canna. Il prossimo colpo/raffica sparato fa metà danno.\\
\rowcolor{gray!20}14 & Il grilletto si incastra. Costa 3 Azioni sparare il prossimo colpo/raffica.\\
15 & Il rinculo è tale che ti cade l'arma per terra entro 1d4 metri di distanza.\\
\rowcolor{gray!20}16 & Il proiettile è esploso nella canna. Per liberare il colpo devi eseguire una prova di Artigianato DC 17, 1d4 round.\\
17 & Colpisci un altro. Il colpo prende una traiettoria non voluta e colpisci una creatura a caso nella direzione del colpo.\\
\rowcolor{gray!20}18 & Il rumore è talmente forte che il personaggio è assordato per 1 minuto.\\
19 & Il proiettile è esploso nella canna danneggiandola. È necessaria una prova di Artigianato DC 21 per ripristinare l'arma, 1 turno.\\
\rowcolor{gray!20}20 & L'intero caricatore è esploso. Subisci danno come se metà dei proiettili rimasti nell'arma ti colpissero. L'arma e proiettili sono distrutti.
\end{tabularx}

%Agile: usare l'arma senza scudo conferisce bonus +1
%Devastante: se crit fai danno da sanguinamento
%Fendente: +1 critico a soggetti proni
%Potente: x1.5 danno da forza
%Impatto: +2 tc contro armature pesanti
%Rompicostole: +2 tc contro armature leggere
%Precisa: solo 1 arma in mano, +1 tc
%Intralcio: -3 m movimento per colpo a segno

\end{multicols}

\vfill

\begin{center}
%\includegraphics[width=0.4\linewidth]{immagini/armiriempitivo3.png}
\includegraphics[width=0.8\linewidth]{immagini/Double-barreled_Shotgun.png}

\emph{Shotgun}
\end{center}

\pagebreak

