\section{Costruiamo il Personaggio}\index{Personaggio}

\begin{enfasi}{
Mai dimenticare chi sei, perché di certo il mondo non lo dimenticherà. Trasforma chi sei nella tua forza, così non potrà mai essere la tua debolezza. Fanne un'armatura, e non potrà mai essere usata contro di te. (Tyrion Lannister)
}\end{enfasi}

\begin{multicols}{2}

OBSS è un sistema duro, pericoloso, mortale ma anche ricco di soddisfazioni. I tuoi personaggi non sono eroi, non sono prescelti. Sono sfortunati che si trovano in imprese dove forse sopravviveranno e sarà a discapito di qualche compagno. Non sei tu a scegliere l'avventura ma è lei a trascinarti impetuosamente dentro. Sii forte, coraggioso, arguto ma non avventato.

Sopravvivi e reclama la Legge del Premio e vedrai che con il passare dei livelli acquisirai competenze ed abilità fuori dal comune!. \emph{Spes ultima dea}!

Come prima cosa prepara davanti a te la scheda ed un foglio dove prendere note ed appunti.

Per creare un personaggio prova rispondere a queste domande, potranno aiutarti ad immaginarlo e plasmarlo:

- Immagina che aspetto abbia

- Quale è il Tratto principale del carattere

- Quali sono i suoi tic, modi di fare, abitudini

- Quali sono i suoi obiettivi primari

- Una cosa curiosa, una buffa, una imbarazzante ed una espressione tipica del personaggio

- In cosa è bravo, in cosa si impegna, in cosa è negato

- I tre difetti ed i tre pregi principali del personaggio

\begin{center}

 \includegraphics[width=0.7\linewidth]{immagini/Leonidas_I_of_Sparta.png}

\emph{Leonida di Sparta}
\end{center}

E' cresciuto in famiglia, in un clan, vagabondo, per strada.. cosa l'ha portato e che scelte ha fatto per arrivare fino ad adesso ?

Quale è il suo stile di combattimento e strategia tipica ? Magia, Spada, dalle retrovie.. incitare i compagni.. scappare...

E non meno importante: quale è il suo scopo ? cosa lo ha fatto uscire di casa, dalle sue sicurezze.. da una vita normale e intrapreso quella di avventuriero ?

Ricorda sempre che questo è un mondo crudele, pieno di rischi, trappole e mostri, ma anche occasioni che possono renderti potente e ricchissimo.

Per incominciare leggi il capitolo sulle Razze ed individua quella del tuo personaggio.

Recupera un pò di d6 e tira!

Consulta il capitolo delle \hyperlink{assegnazione.punteggi.caratteristica}{Caratteristiche} per capire quanto sei stato fortunato (pag. \pageref{assegnazionepunteggicaratteristica}).

E se i valori delle Caratteristiche non sono venuti come che ti aspettavi allora lasciati guidare dal caos e crea qualcosa di diverso ma ugualmente divertente e magnifico.

Se hai Intelligenza pari o superiore a 2 scegli un altra \hyperlink{linguaggi}{lingua} (pag. \pageref{linguaggi}) parlata/scritta oltre al Comune, se hai 3 puoi sceglierne 2 di lingue in più.

Scegli al Professione del personaggio, le Competenze Base vengono assegnata in base a questa. Sceglila con attenzione e cura, oltre alle competenze previste dalla Professione scelta ne puoi prendere una quinta data dal tuo background oppure aumentare di uno il punteggio in una già presa.
In base al background e professione scelta aumenti una caratteristica di 1, fino ad un massimo di 4 + modificatore razziale.

Passa alle Competenze Attive: qui hai 1 punto  da distribuire tra Competenza Armi e Competenza Magica.

La Competenza Armi ti aiuta nel colpire meglio. La Competenza Magica è l'unica cosa che ti permette di usare la magia. Ricorda anche che i punti in Competenze Armi vanno dichiarati a quale \hyperlink{lista.armi}{Lista Armi} (pag. \pageref{lista.armi}) sono stati applicati.

Se non hai punti in Competenza Armi puoi usare solo le \hyperlink{armi.semplici}{armi semplici} (pag. \pageref{listaarmisemplice}) senza incorrere in penalità al Tiro per Colpire e non potrai usare armature medie o pesanti.

I Punti Ferita sono pari a 8 + Costituzione, aggiungi 3 se hai messo 1 punto in Competenza Armi (CA).

A questo punto scegli i \hyperlink{tratti}{Tratti} (pag. \pageref{tratti}). Fallo con attenzione, stai costruendo il tuo personaggio ed i Tratti delineano a forti pennellate il carattere. Ricordati che saranno fondamentali per la scelta del \hyperlink{patroni}{Patrono} (pag. \pageref{patroni}).

Nella scheda, nello specchietto dei Tratti, dove c'è la colonna Patrono scrivi il Patrono che ti collega a quel Tratto, indipendentemente che tu lo abbia scelto o meno.

Ricorda infine che un personaggio \emph{Dissoluto} e \emph{Leale} suona bene un in racconto dove è il solo protagonista ma qui si gioca in \textbf{gruppo}. Non prendere Tratti in ovvia opposizione agli altri o comunque non giocare da \emph{stronzo}, altrimenti il personaggio verrà naturalmente allontanato dagli altri personaggi e dal Narratore.

Se hai messo dei punti in Competenza Magica valuta anche di prendere l'Abilità Adepto della Magia per potenziare la tua magia (e \emph{apprendere} un incantesimo in più!).

\begin{center}
\includegraphics[width=0.85\linewidth]{immagini/Alexander_and_Bucephalus_-_Battle_of_Issus_mosaic.png}

\emph{Alessandro Magno}
\end{center}

Consulta il \hyperlink{tomocm1}{Tomo della Magia}, pag. \pageref{tomocm1}, per capire quanti incantesimi devi scrivere nel tuo Tomo.

Scelti gli incantesimi del Tomo devi decidere quali hai appreso e quindi puoi lanciare, vedi \hyperlink{incantesimicm1}{Regole della Magia} a pag. \pageref{incantesimicm1}.

Passa alle \hyperlink{abilita}{Abilità} (pag. \pageref{abilita}), al primo livello ne scegli due, stai attento ai prerequisiti ed anche ad eventuali Abilità che ti concede la tua razza.

Sono le Abilità che scegli ad aumentare il punteggio dei Tiri Salvezza. Ricorda che i Tiri Salvezza determinano la tua capacità di resistere a traumi e magie. Nella scheda indica la singola Caratteristica che vuoi che quella Abilità migliori (quando ne avrai quattro uguali).

Scegli l'\hyperlink{equipaggiamento}{equipaggiamento} (pag. \pageref{equipaggiamento}), \hyperlink{equipaggiamento.armature.scudi}{armatura} (pag. \pageref{equipaggiamentoarmature}), \hyperlink{equipaggiamento.armi}{armi} (pag. \pageref{equipaggiamentoarmi}), zaino, due torce, qualche razione di cibo.. un peluche.. quello che ti sembra indispensabile per l'avventura.
Aggiorna poi la parte di scheda relativa alla Difesa segnando che bonus ti da l'armatura e scudo indossata. Ricorda che parti con 100 mo, spendile in maniera accurata!

Entra nella parte, concediti di giocare questo straordinario personaggio. Se mai ti stufassi di giocarlo e volessi provare qualcosa di diverso parlane con il Narratore, saprà consigliarti e suggerirti la strada migliore.
Hai il vantaggio che in OBSS le classi non esistono, il personaggio cresce evolve ed impara in base a ciò che fai e sperimenti. Puoi prepararti la \emph{build} a tavolino ma non avrai mai la certezza che il tuo personaggio si evolva come hai pensato. Lascialo vivere e crescere!

In ultimo ricordati della Legge del Premio\index{Legge del Premio}. Questo mondo è feroce, spesso malvagio, ancor di più vorrà ucciderti, eppure per chi sopravvive c'è la Legge del Premio, una legge che neanche i Patroni possono violare. La Legge è piuttosto semplice nel suo concetto base \emph{A chi sopravvive andranno i tesori e la gloria}.

\begin{narratore}[Personaggi disfunzionali]\index{Personaggi disfunzionali}
Il Narratore può concedere in base alla tipologia di campagna che se il personaggio creato ha tutti i punteggi di Caratteristica a 0 o meno può essere creato nuovamente.
\end{narratore}

\subsection{Avanziamo di Livello}\index{Livello}\index{Avanzamento di Livello}\label{avanzamentodilivello}\index{Livellare}

\begin{enfasi}{
Ma ci sono cose che non si possono capire con la riflessione, bisogna viverle. (La storia infinita, Michael Ende)
}\end{enfasi}

Ogni qual volta il Narratore ti conferma il passaggio di livello sono da compiere diverse operazioni per aggiornare la scheda del personaggio.

\begin{itemize}[leftmargin=*] \setlength{\itemsep}{0pt}
\item Innanzitutto prendete la scheda, matita e gomma ed i dadi (almeno il d6)
\item Aggiornate i Punti Esperienza
\item Aggiornate il Livello aumentandolo di 1
\item Distribuite 1 punto tra Competenza Armi e Competenza Magica
\item Aumentate i Punti Ferita massimi ed attuali di 1d6+Costituzione ed aggiungetene altri 3 se avete dato 1 punto in Competenza Armi. Se il tiro di dado è inferiore a Costituzione, potete prendere come risultato il valore di Costituzione
\item Se avete assegnato un punto in Competenza Armi stabilite se prendete una nuova \hyperlink{lista.armi}{Lista d'Armi} (pag. \pageref{lista.armi}) o approfondite la conoscenza di una lista già appresa
\item Controllate se acquisite una nuova Abilità. Potete prenderne una nuova oppure migliorare una Abilità già appresa, state attenti ai prerequisiti. Vedi \hyperlink{abilita}{Abilità} (pag. \pageref{abilita}).
\item Aggiornate il punteggio dei Tiri Salvezza in base alle nuove Abilità prese.
\item Aggiornate il punteggio dei Tiri per Colpire in base al nuovo valore della Competenza Armi, Abilità, bonus dati dalla Lista d'Armi
\item Distribuite (Int/2)+1, con un minimo di 1 punto, tra le \hyperlink{competenzeelenco}{Competenze Base} (pag. \pageref{competenzeelenco}) conosciute o apprese durante le avventure. Verificate il punteggio di Consapevolezza.
\item Aggiornate il punteggio dei Punti Fato $(20-livello)/5$ , all'intero più vicino
\item Aumentate il punteggio dei Tratti come vi dirà il Narratore. Verificate se avete raggiunto un punteggio sufficiente per acquisire poteri legati ai Tratti
\item Verificate in base al nuovo punteggio di Competenza Magica ed all'Abilità Adepto della Magia il \hyperlink{scuoleelivelli}{livello massimo di incantesimo} (pag. \pageref{scuoleelivelli}) lanciabile ed i \hyperlink{magiepuntimagia}{Punti Magia disponibili} (pag. \pageref{magiepuntimagia})
\item Se avete aumentato la Competenza Magica apprendete 1 nuovo incantesimo dal Tomo di Magia oppure potete apprendere due Trucchetti (Incantesimi di livello 0)
\item Aggiornate la seconda parte della scheda in base al nuovo punteggio di Competenza Magica
\end{itemize}

Come avrete notato i punteggi delle Competenze sono ridotti, si prendono pochi punti da distribuire alla volta.
Come giocatori avete l'opportunità di prediligere un approccio specializzato ovvero \emph{puntare} su poche e specifiche Competenze oppure diluire i punti su più competenze per sapere un pò di tutto e non avere penalità nelle prove (la prova si fa solo con 2d6 + Caratteristica se non avete punti nella Competenza).

Un suggerimento è anche di usare le Abilità, ed in particolare Esperto, che vi concede un bonus di +2 alle prove di Competenze.

\begin{giocatore}[Potere, percepito]
Il livello di potere \textbf{percepito} dei personaggi in OBSS è inferiore a quello di altri GDR. La debolezza del personaggio è solo una percezione ed anzi vi accorgerete presto della vera potenza del personaggio. Giocate di gruppo e sopravviverete perché ricordate che questo è un mondo cattivo, dispettoso e mortale con gli \textbf{egoisti}.
\end{giocatore}

%\subsection{Suggerimenti per divertirsi e sopravvivere nelle avventure di OBSS}\index{Linee guida per i giocatori}\label{suggerimentigiocatori}

\subsection{Come Sopravvivere e Divertirsi}\index{Linee guida per i giocatori}\label{suggerimentigiocatori}

\begin{enfasi}{
-\noindent Ci vuole un piano.

-\noindent Da quando gli eroi hanno bisogno di piani? (Final Fantasy XIII)

\medskip

Vado matto per i piani ben riusciti! (Colonnello John \emph{Hannibal} Smith, A-Team)}
\end{enfasi}\medskip

\begin{itemize}[leftmargin=*] \setlength{\itemsep}{0pt}

\item
Ogni combattimento è potenzialmente letale. Decidi con raziocinio e approccialo con attenzione. Impara a scappare, non aver paura di sopravvivere.

\item
Non c'è tutto nella scheda. La scheda di un personaggio è il perimetro dello stesso ma non definisce ciò che può o non può fare. Spremiti le meningi e sii creativo, alternativo, curioso ma non suicida o avventato.

\item
Non si risolve tutto con un tiro di dado. Fai le domande giuste, parla con i compagni e descrivi con attenzione cosa intendi fare. Il Narratore premia le descrizioni accurate. Descrivere come e cosa si fa può evitare di fare la prova!.

\item
Delle basse Caratteristiche sono solo delle basse Caratteristiche e non il personaggio. Sfrutta le competenze, le Abilità, fai in modo di dover tirare meno dadi possibili per risolvere i problemi.

\item
Improvvisare, adattarsi e raggiungere lo scopo! (Tom Highway - Gunny, Film). Oppure come alcune mie giocatrici preferivano \emph{Improvvisare, \textbf{Ingannare} e raggiungere lo scopo}.

\item
Vivi appieno il tuo personaggio. Amplifica la sua storia porta nel presente il suo passato. Aiuta i compagni a conoscerti ed il Narratore a imbastire storie migliori intorno alle vostre storie.

\item
Una cosa che non potrà mai portarti via nessuno è l'essere eroico, intelligente, risoluto, caparbio, cocciuto ma non stupido.

\item
Descrivi in maniera realistica ciò che fai, aiuterai il Narratore e i compagni intorno a te. E' sicuramente meglio che dire \emph{faccio una prova di Consapevolezza}. Esaltati nel descrivere le azioni più importanti, il Narratore ne terrà conto.

%\item
%E finché non potrai dire "\emph{Io sono cattivo, incazzato e stanco. Sono uno che mangia filo spinato, piscia napalm e riesce a mettere una palla in culo ad una pulce a 200 metri}." (Tom Highway - Gunny, Film) allora stai al tuo posto e non fare lo sbruffone, c'è sempre qualcuno più grosso ed arrabbiato di te.

\item
Ricorda sempre che maggiore è il pericolo maggiore è l'esperienza maturata. più è profondo il dungeon maggiore saranno i tesori e l'esperienza acquisita!

\item
Lo scopo è divertirsi, fare divertire ed assaporare la sfida. Non creare un personaggio che sia contro gli altri personaggi o dia sempre fastidio e problemi. Media il tuo desiderio con le necessità del gruppo, perché sempre e \textbf{solo come gruppo} sopravviverete e mai solo come singolo.

\item
Pensa prima di agire, ma non farti aspettare dagli altri. Usa il tempo tra i tuoi round per pianificare come agire al meglio.

\item
Se hai difficoltà a capire o immaginare qualcosa, chiedi al Narratore maggiori informazioni e chiarimenti, gli farà solo piacere.

\item
Abbraccia il fallimento. Fallire con stile è molto meglio di una noiosa vittoria.

\item
Fai in modo che il tuo personaggio si preoccupi sempre di qualcosa di più che della sua vita.

\item
Non aver paura di discutere con gli altri personaggi, ma assicurati sempre di non andare sul personale con i giocatori.

\end{itemize}

\end{multicols}

\vfill

\begin{enfasi}{
La candela accesa da entrambe le parti dura la metà. (Anonimo)
}\end{enfasi}

%\vfill

%\begin{center}

%\includegraphics[width=0.9\linewidth]{immagini/Granblue.Fantasy.full.2108782.png}

%\filltopageendgraphics[width=0.7\linewidth]{immagini/Granblue.Fantasy.full.2108782.png}

%\emph{Attorno al fuoco, raccontando la giornata trascorsa.}
%\end{center}

%\begin{center}
%\includegraphics[width=0.45\linewidth]{immagini/threasure2.png}
%\end{center}

\pagebreak

