\section{Le mie Opzioni}\index{Le mie Opzioni}

\normalsize

Anche io sono un Narratore e per quanto abbia costruito OBSS in base alle mie preferenze ci sono alcune Opzioni, che rendono il gioco più \emph{unico} che mi piace rendere disponibili ai personaggi.

Al mio tavolo da gioco solitamente propongo queste Opzioni, da decidere in Sessione Zero:

\begin{itemize}[leftmargin=*] \setlength{\itemsep}{0pt}

\item
\hyperlink{successoparziale}{Successo Parziale} pag. \pageref{successoparziale}

\item
\hyperlink{varianteiniziativa}{Variante Iniziativa}, solo se ho giocatori esperti. Pag. \pageref{varianteiniziativa}

\item
Partendo con \hyperlink{variantetiricritici}{Variante Tiro Critico}, \hyperlink{tirocriticovariante}{Tiro Critico Variante}, \hyperlink{OpzionaleAzioniTiroCritico}{Azioni Tiro Critico}, \hyperlink{varianteattacchimultipli}{Variante Attacchi multipli} sono a scelta del giocatore se usarli o meno. Pag. \pageref{tirocriticovariante}, pag. \pageref{OpzionaleAzioniTiroCritico}, pag. \pageref{variantetiricritici} e pag. \pageref{varianteattacchimultipli}. Sono Opzioni suggerite per giocatori già con esperienza e consapevoli del proprio personaggio.

\item \hyperlink{elencotalentiarmi}{Opzionale - Elenco Manovre d'Arme} (pag. \pageref{elencotalentiarmi}) per rendere meno \emph{noiso} il fumbolare...

%\item
%\hyperlink{lunicaregola}{L'Unica Regola} la uso se sono Narratore di un gruppo di principianti. Pag. \pageref{lunicaregola}

\item
\hyperlink{Un solo credo}{Un solo credo} o \hyperlink{abilitadilista}{Abilità di Lista}, a scelta del giocatore.

\item
\hyperlink{componenticomeofferta}{Componenti come offerta} per una magia personale, diversa e unica, sempre legata al personaggio. Pag. \pageref{componenticomeofferta}

\item
\hyperlink{abilitaiconiche}{Abilità Iconiche} in caso di lunghe campagne. Pag. \pageref{abilitaiconiche}

\item
\hyperlink{droghe}{Droghe} \textbf{NO}. Solo in caso di gruppi composti da persone mature ed adulte di testa. Pag. \pageref{droghe}

\item
\textbf{No all'utilizzo del cronometro sulle Luci}: se le vostre avventure non sono ambientate in dungeon o volete una gestione più snella non gestite la durata delle luci in tempo reale.

\end{itemize}

\vfill
{\small

\begin{multicols}{2}

\subsubsection*{Note}

Per me OBSS va giocato in maniera schietta, senza troppi pensieri e cervellotici progetti. OBSS non è fatto per uccidere i personaggi ma allo stesso modo non ne agevola la sopravvivenza, tutto sta al Narratore a decidere come si gioca. E' nel Narratore, nello stile dei giocatori e l'interesse del gruppo la chiave di gioco, OBSS vuole offrire il framework, gli strumenti, per giocare l'avventura.

Cercate di enfatizzare le scene, siate anche teatrali nelle descrizioni, togliete la patina al gioco pulito e politically correct. Rimane sempre il vostro mondo, il vostro tavolo ed il vostro gioco, cercate di dare quell'immersività che spesso nei sistemi più moderni si è un pò persa.
Quando c'è un combattimento fate che sia tale! Deve sentirsi il clangore delle armi, il cozzare sulle armature, l'ozono nell'aria causato dal fulmine, le bruciature crepitanti delle palle di fuoco. Fate che i giocatori apprezzino le possibilità offerte dal sistema e si possano divertire a cercare come effettuare la prova migliore.

Scegliete voi se i personaggi sono canaglie che cercano solo di sopravvivere e accumulare tesori oppure dare un taglio più classico o epico all'avventura. OBSS si sposa che entrambe le scelte, specialmente utilizzando qualche Opzione rispetto ad un altra.

Create il gruppo, e non intendo solo come insieme di personaggi, ma come anche insieme di giocatori. Un gruppo dove le persone si rispettano e fidano (possibilmente...). Costruite avventure che coinvolgano tutti, dove tutti possano dare il loro contributo. Ci potranno essere avventure più \emph{cucite} intorno ad un personaggio ma questo non \textbf{deve} escludere gli altri nel partecipare, nel più ampio termine della parola, non fate che la sessione sia un monologo tra voi ed il singolo giocatore.
Approfittate di ogni avventura per fare conoscere i personaggi tra loro, nulla unisce di più che la paura di morire!

Una volta fatto il gruppo, e potrebbe volerci anche tempo, allora sfruttate le storie personali, gli indizi ed ipotesi create dai giocatori per plasmare situazioni e accadimenti. Come un pesante volano che ruota questa continuerà a creare situazioni, avventure e nuovi plot da seguire.

Potrebbero esserci delle difficoltà nel creare il gruppo, capita purtroppo. Cercate di parlare con il giocatore che da problemi. Cercate di capire se è il suo personaggio che non \emph{funziona} con il gruppo oppure è il giocatore che non ha ben compreso le meccaniche del gruppo.

Per questo vi suggerisco sempre di fare la così detta \hyperlink{sessionezero}{Sessione Zero} (pag. \pageref{sessionezero}), dove come Narratore andrete a delineare a grandi linee quali sono i cardini dell'avventura, cosa vi aspettate dai personaggi, quali sono le regole di base e di morale da seguire. Non c'è nulla di peggio di un gruppo di personaggi slegati dove ognuno vuole fare qualcosa di diverso e non gli interessa l'\emph{obiettivo comune}.

E' molto importante capire cosa piace ai giocatori, ogni persona e gruppo vuole un certo stile di gioco ed è corretto cercare di accontentarli. Se il gruppo vuole avventure politiche, drama romantici cercate di fargli trovare soddisfazione nel mentre dell'avventura. Se invece preferiscono più combattere allora non lesinate scontri purché coerenti con l'avventura stessa.

Fate capire che dovete funzionare come insieme di giocatori e personaggi per poter giocare al meglio e divertirvi tutti ed avere maggiori possibilità di sopravvivenza. Nessun giocatore deve essere sopra gli altri, solo il Narratore ha l'ultima parola.

Infine siate sempre corretti, nel bene e nel male. Ci saranno sessioni più sfortunate ed altre dove i dadi troveranno la strada giusta, dove l'idea brillante salverà il gruppo. Non fate il Narratore che salva \textbf{sempre e comunque} i personaggi, un aiuto ogni tanto ci può stare specialmente nella sessione più sfortunata, ma rispettate le scelte dei personaggi e l'esito dei dadi. Ricordate che i giocatori hanno i Punti Fato da poter usare a differenza dei poveri mostri!.

Personalmente io tiro tutti i dadi davanti ai giocatori.

Il \textbf{Narratore non deve rubare la scena} ai personaggi, bensì costruire l'ambiente, i drammi e le passioni intorno a loro. Non c'è nulla di più fastidioso di un Narratore con manie di protagonismo, vuoi nella tenuta dei PNG vuoi nell'imporre scelte e scene.

Ed infine un ovvietà: \textbf{divertitevi}, sforzatevi \textbf{tutti} affinché la sessione abbia quel misto di tensione, divertimento e soddisfazione. Siete persone che vogliono giocare, divertirsi e stare insieme, non dimenticatelo mai.
\end{multicols}}

\pagebreak

\begin{multicols}{4}
{\small\printindex}
\end{multicols}

\TotalBox{OBSSv2}\pagebreak

\begin{multicols}{3}
{\small\printindex[Tabelle]}
\end{multicols}

\vfill

\TotalBox{Tabelle}\pagebreak

\begin{multicols}{3}
{\small\printindex[Incantesimi]}
\end{multicols}

%\immediate\write18{./contaspell.sh > contaspell.txt}
%\immediate\openin\myscriptresult=./contaspell.txt
%\read\myscriptresult to \ScriptResult
%\immediate\closein\myscriptresult

\vfill

%Totale elementi in questo indice \ScriptResult

\TotalBox{Incantesimi}\pagebreak

\begin{multicols}{3}
{\small\printindex[OggettiMagici]}
\end{multicols}

%{\scriptsize\printindex[OggettiMagici]}

%\immediate\write18{./contaomagici.sh > contaomagici.txt}
%\immediate\openin\myscriptresult=./contaomagici.txt
%\read\myscriptresult to \ScriptResult
%\immediate\closein\myscriptresult

\vfill

%Totale elementi in questo indice \ScriptResult

\TotalBox{OggettiMagici}\pagebreak

\begin{multicols}{3}
{\small\printindex[Abilita]}
\end{multicols}

\vfill

\TotalBox{Abilita}\pagebreak

\begin{multicols}{3}
{\small\printindex[Mostruario]}
\end{multicols}

\vfill

\TotalBox{Mostruario}\pagebreak

%Totale elementi in questo indice \ScriptResult

%\immediate\write18{./contamostri.sh > contamostri.txt}
%\immediate\openin\myscriptresult=./contamostri.txt
%\read\myscriptresult to \ScriptResult
%\immediate\closein\myscriptresult

\section*{Appendice N}

\begin{multicols}{2}
{\small
\nocite{*}
\setlength{\columnsep}{1cm} % spazio tra colonne
\printbibliography
}
\end{multicols}

\end{document}

% da mettere a riga 1
% arara: xelatex
% arara: biber
% arara: xelatex
% arara: xelatex: { synctex: on }

%livelli alternativi
%copper
%iron
%silver
%gold
%platinum
%mithril
%oricalcum
%adamantium
