\section{Opzionale - Capacità Iconiche}\index{Opzionale - Abilità Iconiche}\hypertarget{abilitaiconiche}{}\label{abilitaiconiche}

\begin{enfasi}{
La vita è diventata incommensurabilmente migliore da quanto sono stato forzato a smettere di prenderla seriamente. (Daniel Day Lewis)
}\end{enfasi}

\begin{multicols}{2}

%{\small

Queste capacità rappresentano l'apice di un personaggio, non inteso come le abilità finali del 20esimo livello, ma come capacità legate al modo di ruolare, al tipo di personaggio che si è andato a creare e crescere. Queste capacità andrebbero date solo a personaggi che sono stati portati dal primo ad almeno il 15esimo livello, è un riconoscimento al giocatore.

Ogni personaggio può avere una sola capacità iconica, una capacità che contraddistingue gli eroi, capaci di azioni al limite e oltre l'umano. I giocatori sono invitati a creare nuove Abilità Iconiche in base allo sviluppo del personaggio.

\medskip

\subsubsection{Una Luce contro le tenebre}\index{Una Luce contro le tenebre}

\textbf{Requisiti suggeriti}: Patrono Ljust, Sumkjr

Una volta al giorno emetti per 60 minuti luce sacra intorno a te che ti conferisce +1d6 ai Tiri Salvezza e Tiri per Colpire contro i Devoti e Seguaci non del tuo Patrono. Puoi convogliare 1 volta al giorno la luce e tutte le creature Seguaci o Devoti di altri Patroni in raggio di 10 metri da te devono effettuare un TS Tempra a DC 10 + somma Tratti in comune con il Patrono + Saggezza o essere storditi per 2d6 round.

\subsubsection{Il Fabbro}\index{Il Fabbro}

\textbf{Requisiti suggeriti}: abilità nel lavorare il metallo

Le tue capacità di lavorare con le armi ed armature sono leggendarie.
Ogni armatura da te fatta ingombra e pesa come una categoria inferiore, le armi fanno un danno di una categoria superiore di dado.

\subsubsection{L'Oracolo della Guerra}\index{L'Oracolo della Guerra}

\textbf{Requisiti suggeriti}: maestro combattente di mischia

Ogni arma nelle tue mani è letale. Il dado dell'arma raddoppia come raddoppia il danno causato dalla Forza. Es. una spada lunga fa 2d8 di danno e se hai Forza +3 il danno totale diventa 2d8+6

\subsubsection{L'Eroe senza paura}\index{L'Eroe senza paura}

\textbf{Requisiti suggeriti}: coraggioso e risoluto

Una volta per avversario puoi ignorare (per 1d4 round) le condizioni che ti affliggono come Reazione.

\subsubsection{Mindmaster}\index{Mindmaster}

\textbf{Requisiti suggeriti}: una vita a avventurosa gestita con intelligenza e sangue freddo

Puoi usare il punteggio in una Caratteristica mentale (Intelligenza, Saggezza o Carisma) al posto di una fisica (Forza, Destrezza, Costituzione) per quanto riguarda tutte le prove.

\subsubsection*{Su un saurovallo pallido}\index{Su un saurovallo pallido}

\textbf{Requisiti suggeriti}: non temere al morte, aver ucciso tantissimi avversari

Sei la cosa più simile alla morte che i tuoi nemici vedranno mai.
Quando uccidi un nemico tutti gli avversari (che possono aver visto la scena) in 10m di raggio devono fare un TS Volontà con DC pari al Tiro per Colpire, costo una Reazione, od essere influenzati come dall'incantesimo Paura. La capacità è usabile 3 volte al giorno.

\subsubsection{La Furia Magica}\index{La Furia Magica}

\textbf{Requisiti suggeriti}: una vita dedicata alla magia esplosiva

Sei capace di scatenare l'inferno con la magia. La difficoltà (DC) di ogni tuo incantesimo aumenta di 2, quando fai una Prova di Magia tiri 3d6 in più ed ignori 2 dadi tirati.

\subsubsection{L'Ombra}\index{L'Ombra}

\textbf{Requisiti suggeriti}: una vita dedicata a nascondersi e sorprendere i nemici

Tre volte al giorno puoi teletrasportati sull'ombra di un'altra creatura che sia entro 30 metri. Costo 1 Reazione.

\subsubsection{La Madre}\index{La Madre}

\textbf{Requisiti suggeriti}: passato più tempo in forma animale che propria

Ha la capacità innata di lasciare le orme di qualsiasi animale, compatibile con la tua taglia, anche se non sei trasformato. Puoi parlare con qualsiasi animale come se fossi sempre sotto effetto dell'incantesimo Parlare con gli Animali.

\subsubsection{Il Morto}\index{Il Morto}

\textbf{Requisiti suggeriti}: una vita di sempre sul baratro della morte

Tre volte al giorno quando i tuoi Punti Ferita scendono sotto 1, con una Azione di Reazione recuperi 3d12 Punti Ferita. Questa capacità può essere usata anche quando i Punti Ferita sono negativi o si dovrebbe essere direttamente morti.

\subsubsection{Il Cacciatore}\index{Il Cacciatore}

\textbf{Requisiti suggeriti}: una vita dedicata a cacciare ed inseguire

Le tue prove di Sopravvivenza hanno un +2d6 di bonus. Il primo colpo che va a segno contro un avversario ottiene automaticamente 2 critici.

\end{multicols}

\pagebreak

