\section{Nascondigli e coperture} \index{Nascondigli}

\begin{enfasi}
Dove c'è molta luce, l'ombra è più nera. (Johann Wolfgang von Goethe)
\end{enfasi}

\begin{multicols}{2}

Non sempre l'avversario si palesa davanti a noi, spesso può essere nascosto se non addirittura invisibile.

Potrebbe essere nascosto dietro un muretto o dei barili, se non coperto da un muscoloso e gigantesco famiglio.
E se fosse alle nostre spalle e neanche l'abbiamo notato ?

\subsection{La Copertura}\index{Copertura} \label{copertura}\hypertarget{copertura}{}

Se l'obiettivo è noto che ci sia ma è occultato in qualche maniera allora si dice che ha \textbf{copertura}.

\begin{itemize}[leftmargin=*] \setlength{\itemsep}{0pt}
\item
Se l'obiettivo \textbf{ha più della metà} (ma non totale) della superficie \textbf{visibile} allora la copertura si definisce \textbf{leggera}, ovvero ha +2 alla Difesa. Può essere il caso di una creatura dietro un altra creatura della medesima taglia o di 1 taglia più grande.

Può essere il caso di un arciere in piedi dietro un muretto di 1 metro.

\begin{center}
\includegraphics[width=0.9\linewidth]{immagini/hide.png}

\emph{British Soldiers Hiding From Boer Fire At The Battle Of Majuba Hill.}
\end{center}

\item
Se l'obiettivo ha \textbf{meno della metà} (ma almeno un terzo) della superficie \textbf{visibile} allora la copertura si definisce \textbf{media}, ovvero ha +4 alla Difesa. Può essere il caso di una creatura dietro un altra creatura di 2 taglie più grande.

Può essere il caso di un nemico armato di balestra che si sporge quel tanto per tenere appoggiata la balestra al muretto e sparare (petto, spalle, braccia e testa visibili).

\item
Se l'obiettivo si sa dove è ma \textbf{si nasconde completamente} affacciandosi solo per controllare o tirare una freccia ogni tanto, dietro ad un muro, finestra, porta, tavolo, una creatura più grande di lui (almeno 3 taglie).. allora la copertura si definisce \textbf{completa}, ovvero ha +8 alla Difesa.

\end{itemize}

Metà del bonus di copertura si applica anche ai \textbf{Tiri Salvezza} contro Incantesimi che abbiano un \textbf{effetto ad area} (es. Palle di Fuoco che esplodano intorno..).\index{Copertura nei Tiri Salvezza}

\subsubsection{Combattimento con armi da tiro in caso di Copertura}\index{Esempi di Combattimento con armi da tiro con Copertura}\label{esempicopertura}\hypertarget{esempicopertura}{}

Quando si effettuano attacchi da lancio (arco, balestre, pugnali, giavellotti...) contro avversari con copertura è necessario verificare bene la linea di tiro e controllare quante creature ci sono all'interno.

Ogni creatura di taglia uguale all'avversario in linea, che \emph{copre} l'obiettivo aumenta di un grado la copertura fornita.

\textbf{\textit{Esempio}}. la quarta creatura sulla linea di tiro equivale, in caso di creature tutte delle stessa taglia, beneficia di una Copertura Completa. La prima creatura fornirà copertura leggera (+2 Difesa), la seconda copertura media (+4 Difesa) e la terza di copertura completa (+8 Difesa). Ogni ulteriore creatura della medesima taglia somma un ulteriore +6 alla Difesa di copertura e +6 ulteriori per taglia di differenza.

\textbf{\textit{Esempio}}. se la terza creatura sulla linea di tiro è coperta da una creatura media e poi da creatura piccola, godrà di una Copertura Leggera. La creatura piccola non fornisce bonus di copertura, se l'obiettivo è di taglia media.

Se chi si deve colpire è di taglia maggiore tra le creatura coinvolte nella copertura questa non beneficerà di alcuna copertura.

La copertura fornita da una creatura di taglia maggiore rispetto alla taglia dell'obiettivo si \emph{conta una creatura in più} per taglia di differenza per la protezione data da copertura.

\textbf{\textit{Esempio.}} se la la creatura da colpire è di taglia piccola ed è coperta da una creatura grande avrà bonus di copertura pari a Completa +8 (+2 per una copertura, +4 per prima taglia di differenza, +8 per seconda taglia di differenza).

\textbf{\textit{Esempio.}} se la la creatura da colpire è di taglia media ed è coperta da una creatura di taglia grande la copertura fornita sarà +4. Normalmente sarebbe copertura leggera perché è una sola creatura a coprire, ma essendo di una taglia maggiore conta come 2 creature a fornire copertura.

Vedi anche Abilità \hyperlink{Precisino}{Precisino} (pag. \pageref{Precisino}).

\subsection{Invisibilita'}\index{Invisibilità} \hypertarget{invisibilita}{}\label{invisibilita}

Se un avversario è invisibile o non si sa dove è si seguono le regole della Invisibilità.

Anche se si è invisibili non è detto che non si possa essere percepiti diversamente attraverso altri sensi, come l'olfatto, l'udito od il tatto. L'invisibilità rende una creatura non individuabile tramite la vista ma non rende di per sé una creatura non percepibile o immune ai Tiri Critici o Esplosioni del Danno.

Una creatura accecata o che combatta contro una creatura invisibile o che combatta nell'oscurità più completa senza scurovisione, può effettuare una prova di Consapevolezza, 1 Azione, a difficoltà 20, oppure 2 Azioni a Difficoltà 15, per \textbf{individuare} una creatura se questa è entro 6 metri da lei.\index{Individuare bersagli}

La prova di Consapevolezza può essere fatta contestualmente all'Azione di Movimento per avvicinarsi alla creatura a difficoltà base di 25.

A seconda della distanza della creatura invisibile o di ciò che questa ha fatto nel round precedente sono presenti diversi modificatori alla prova di Consapevolezza per individuarla.

\medskip

\begin{narratore}[Invisibilità]
La prova di Consapevolezza ha una difficoltà alta per un personaggio di basso livello. State ben attenti a considerare tutti i modificatori del caso altrimenti i personaggi difficilmente potranno individuarli ed attaccheranno quadretti a caso...
\end{narratore}

\bigskip

\textbf{Tabella: Modificatori alla DC di Consapevolezza per Rilevare Creature Invisibili}\index[Tabelle]{Tabella Modificatori Consapevolezza per Rilevare Creature Invisibili}

\medskip

\noindent\begin{tabularx}{\linewidth}{Xl}
	\toprule
\rowcolor{gray!20}\textbf{La Creatura Invisibile...} & \textbf{Mod.}\\
\toprule
Si è mossa& -4\\
\rowcolor{gray!20}Ha scagliato un proiettile & -4\\
Un compagno che lo vede ti indirizza & -4\\
\rowcolor{gray!20}Ha corso o caricato& -8\\
Usa Furtività & prova+10\\
\rowcolor{gray!20}E' ferma e non fa rumori & +4\\
Per ogni metro oltre i 6 metri & +2\\
\rowcolor{gray!20}Ha copertura Legg./Media/Compl. & +4/8/12
\end{tabularx}

\medskip

Questi modificatori sono cumulativi tra loro.

Se la creatura invisibile ha attaccato in mischia e non si è spostata si considera \textbf{automaticamente individuata}.

Se la prova per individuare riesce l'osservatore ha la sensazione che \emph{ci sia qualcosa} ma non può vederlo o prenderlo di mira in modo accurato con un attacco.

Chi attacca una creatura per lei \textbf{invisibile ma individuata} ha un -1d6 al Tiro per Colpire, la creatura che attacca colui che non la vede ha +1d6 al Tiro per Colpire.

Una creatura Accecata \index{Accecata}subisce penalità di -2 alle Prove di Competenza Base basate su Forza e Destrezza e fallisce automaticamente qualsiasi prova di Consapevolezza dipenda dalla vista.

Attaccare un bersaglio non individuato significa attaccare un \emph{quadretto} a caso della mappa. Permettete sempre il Tiro per Colpire, che ci sia un avversario o meno in quel quadretto. Se il bersaglio è in quel quadretto modificate la sua Difesa di +8, se il \emph{quadretto} è vuoto il Tiro per Colpire non colpirà nessuno ed informerete il personaggio che non si è colpito nulla.

\medskip
\begin{center}

	\includegraphics[width=0.8\linewidth]{immagini/DnD_Invisible_stalker.png}

	\emph{Persecutore Invisibile!, LadyofHats}

\end{center}

\medskip

%\begin{center}
%	\includegraphics[width=0.9\linewidth]{immagini/brickwall.png}
%
%	\emph{c'è qualcuno davanti a questo muro ?}
%\end{center}

\subsubsection{Note su invisibilità}

Se un personaggio invisibile raccoglie un oggetto visibile, l'oggetto resta visibile. Una creatura invisibile può raccogliere un piccolo oggetto visibile e nasconderselo addosso (mettendolo in una tasca o sotto il mantello, chiudendolo nel pugno) e renderlo effettivamente invisibile.

Qualcuno potrebbe spargere su un oggetto invisibile della farina per tenere traccia almeno della sua posizione (finché la farina non cade del tutto o viene soffiata via).

Le creature invisibili lasciano impronte. Le loro tracce possono essere seguite senza problemi. Impronte su sabbia, fango o altre superfici soffici possono dare ai nemici indicazioni sulla posizione della creatura invisibile rendendola individuata.

Una creatura invisibile nell'acqua muove il liquido, rivelando la propria posizione. La creatura invisibile rimane comunque difficile da colpire e gode dei benefici di una copertura media (+4 alla Difesa).

Una torcia accesa invisibile emana comunque luce (così come un oggetto invisibile soggetto ad una magia di luce).

Le creature invisibili non possono utilizzare gli attacchi con lo sguardo. L'invisibilità non influisce sull'essere obiettivo di un incantesimo di Divinazione.

\end{multicols}

%\vspace{4cm}

%\vfill
%
%\begin{center}
%\includegraphics[keepaspectratio,width=0.4\textwidth]{immagini/impronteneve.png}
%\emph{Può aiutare a trovare un lupo invisibile...}

%\end{center}

\pagebreak
