\section{Abilità}\index{Abilità}\hypertarget{abilita}{}\label{abilita}

\begin{changemargin}{0.3cm}{0.3cm}\begin{enfasi}{Il martirio è l'unica maniera per un uomo di diventare famoso se non ha abilità (George Bernard Shaw, The Devil's Disciple)} \end{enfasi}\end{changemargin}\medskip

\begin{multicols}{2}

Le Abilità sono capacità peculiari, frutto di allenamento o doti particolari. Le Abilità hanno sempre un effetto pratico.

Le Abilità costituiscono una buona parte di ciò che può fare il personaggio, vanno scelte con attenzione e cura. E' scegliendo le Abilità che si stabilisce lo stile e capacità del personaggio, se lo si vuole più guerriero o mago o curatore... o qualsivoglia combinazione e \emph{unicità}.

\textbf{Al primo livello si prendono due Abilità}. Successivamente si prende una Abilità ai livelli 2, 3, 4, 5, 6, 7, 9, 10, 12, 13, 15, 16, 18, 20. Questa può essere un'Abilità già conosciuta oppure una nuova Abilità appresa durante le avventure.\index{Abilita' al primo livello}\index{Abilità nei livelli}

E' possibile che siano indicati dei Requisiti sotto il nome dell'Abilità, in questo caso vanno rispettati per prendere l'Abilità in oggetto.
Eventuali requisiti successivi vengono indicati volta per volta.

Non prendete le Abilità in base al potere, forza, combinazione con altre ma perché in linea con la storia del personaggio.
Scegliere un'accozzaglia di Abilità solo perché forti non rende un personaggio potente ma sbilanciato, non fate il power-player ad ogni costo.

\medskip

\textbf{Le Abilità devono essere prese in base al percorso evolutivo del personaggio, in base a quanto vissuto ed appreso durante le avventure.}

\medskip

E' possibile cambiare una Abilità scelta, rispettando comunque i requisiti, riaddestrandosi per almeno una settimana per 4 ore al giorno con qualcuno che abbia la nuova Abilità.\index{Riaddestrarsi}\index{Cambiare le Abilita'}

Le capacità fornite dalle Abilità se non descritto diversamente sono cumulative o se si tratta dello stesso bonus si applica quello maggiore. Se non esplicitato una Abilità non può essere presa più volte.

\subsection{Tiri Salvezza ed Abilita'}\label{tirisalvezzaedabilita}

Ogni Abilità, tranne se diversamente indicato, concede dei bonus ai Tiri Salvezza che si cumulano tra loro, anche quando si prende più volte la stessa Abilità.

Quando scegliete un'Abilità fate anche caso a quali Tiri Salvezza aumenta!

\subsection{Caratteristiche ed Abilita'}\label{caratteristicheedabilita}\index{Caratteristiche ed Abilita'}

Ogni Abilità ha segnato una nota tipo \emph{\textbf{Caratteristica}} con a fianco l'indicazione di una o più Caratteristiche. Nella scheda segnate che Caratteristica, 1 sola, decidete di potenziare con quell'Abilità così che ogni 4 potenziamenti alla medesima Caratteristica potete aumentare di 1, entro il limite di 4 + modificatori razziali, il punteggio della Caratteristica in questione.

\subsection{Aggiungere nuove Abilita'}\label{aggiungereabilita}

Questo elenco non potrà mai essere esaustivo data la fantasia dei giocatori! Cercate però di capire se quello che il giocatore vuole è una Abilità o Competenza, l'avere una capacità o sapere fare qualcosa di particolare.
Valutate bene i prerequisiti ed i vantaggi che concede, cercate sempre di essere bilanciati, piuttosto concedete dei vantaggi a scalare, ovvero prendendo più volte l'Abilità.

Ricordatevi anche di segnare i bonus relativi ai Tiri Salvezza. Solitamente una Abilità concreta e pratica concede un bonus di +3 divisi tra 2 Tiri Salvezza, una Abilità più generica concede un 2 punti da dividere tra un solo Tiro Salvezza o due.

\subsection{Elenco Abilità}

\smallskip\noindent\rule{\linewidth}{2pt} \index[Abilita]{Adepto della Magia}\hypertarget{Adepto della Magia}{}\medskip\noindent{\textbf{Adepto della Magia}}\pdfbookmark[3]{Adepto della Magia}{Adepto della Magia}
\noindent
\begin{description}[noitemsep, topsep=0pt, parsep=0pt, partopsep=0pt, leftmargin=0cm, labelwidth=2.5cm]
    \item[\textbf{Requisito}]: Competenza Magica 1
    \item[\textbf{Tiri Salvezza}]: +1 a due Tiri Salvezza a propria scelta.
    \item[\textbf{Caratteristica}]: Modificatore di caratteristica per incantesimi
\end{description}

Tramite questa Abilità si approfondisce la capacità di lanciare incantesimi.

L'Abilità Adepto della Magia permette di lanciare incantesimi di più alto livello, di rendere più difficile resistere ai propri incantesimi, di fallire meno nella Prova di Magia.

Ogni volta che prendi questa Abilità \emph{apprendi} un incantesimo in più presente nel Tomo di Magia.

L'Abilità è prendibile più volte purché sia inferiore a CM/2.

\smallskip\noindent\rule{\linewidth}{2pt} \index[Abilita]{Ali della Fenice}\hypertarget{Ali della Fenice}{}\medskip\noindent{\textbf{Ali della Fenice}}\pdfbookmark[3]{Ali della Fenice}{Ali della Fenice}
\noindent
\begin{description}[noitemsep, topsep=0pt, parsep=0pt, partopsep=0pt, leftmargin=0cm, labelwidth=2.5cm]
    \item[\textbf{Requisito}]: Lista Pugno Vuoto 3, Gru d'Argento 1
    \item[\textbf{Tiri Salvezza}]: +2 Riflessi, +1 Tempra
    \item[\textbf{Caratteristica}]: Destrezza o Forza
\end{description}

Il tuo stile di combattimento enfatizza i colpi portati da lontano come pugni e calci volanti.

La \textbf{prima volta} che prendi questa Abilità la tua distanza di mischia con la Lista Pugno Vuoto diventa di 2 metri.

Le \textbf{seconda volta} che prendi questa Abilità, requisito Lista Pugno Vuoto 6, Gru d'Argento 3, Pugno di Ferro 1, la tua distanza di mischia diventa di 3 metri.

Le \textbf{terza volta} che prendi questa Abilità, requisito Lista Pugno Vuoto 12, Gru d'Argento 4, Pugno di Ferro 2, la tua distanza di mischia diventa di 4 metri.

\smallskip\noindent\rule{\linewidth}{2pt} \index[Abilita]{Allungo}\hypertarget{Allungo}{}\medskip\noindent{\textbf{Allungo}}\pdfbookmark[3]{Allungo}{Allungo}
\noindent
\begin{description}[noitemsep, topsep=0pt, parsep=0pt, partopsep=0pt, leftmargin=0cm, labelwidth=2.5cm]
    \item[\textbf{Requisito}]: Competenza Armi 2
    \item[\textbf{Tiri Salvezza}]: +1 Volontà, +2 Tempra
    \item[\textbf{Caratteristica}]: Destrezza o Forza
\end{description}

Usi una Azione in concomitanza alla tua Azione di Attacco in mischia per aumentare la portata di 1 metro con il tuo colpo.

\smallskip\noindent\rule{\linewidth}{2pt} \index[Abilita]{Animalia}\hypertarget{Animalia}{}\medskip\noindent{\textbf{Animalia}}\pdfbookmark[3]{Animalia}{Animalia}
\noindent
\begin{description}[noitemsep, topsep=0pt, parsep=0pt, partopsep=0pt, leftmargin=0cm, labelwidth=2.5cm]
    \item[\textbf{Requisito}]: Seguace o Devoto di Efrem oppure Shayalia, Competenza Magica 2.
    \item[\textbf{Tiri Salvezza}]: +2 Volontà, +1 Tempra
    \item[\textbf{Caratteristica}]: Modificatore di caratteristica per incantesimi
\end{description}

Si acquisisce la capacità di trasformarsi in una creatura conosciuta. Costo 2 Azioni.

I propri incantesimi di cura funzionano anche su Animali e Piante, normali e magiche.

\medskip

La \textbf{prima volta} che prendi questa Abilità puoi trasformarti in una creatura con queste caratteristiche:

\medskip

\textbf{Tipologia Creature}: Bestie

\textbf{Caratteristiche}: quelle fisiche, Difesa, Tiri Salvezza e forme di attacco sono dell'animale.

\textbf{Incantesimi}: non puoi lanciare incantesimi nella nuova forma.

\textbf{Equipaggiamento}: vedi Regole base per la trasformazione.

\medskip

La \textbf{seconda volta} che prendi questa Abilità puoi trasformarti anche in una creatura con queste caratteristiche:

\medskip

\textbf{Requisito}: Competenza Magica 4

\textbf{Tipologia Creature}: Piante e Melme

\textbf{Caratteristiche}: il Personaggio sceglie se le Caratteristiche fisiche, Difesa, Tiri Salvezza sono proprie o dell'animale.

\textbf{Incantesimi}: non puoi lanciare incantesimi nella nuova forma

\textbf{Equipaggiamento}: quello magico non ha effetto. Armature e Scudi applicano il bonus magico alla Difesa della creatura. Le capacità magiche degli oggetti non possono essere attivate.

\medskip

La \textbf{terza volta} che prendi questa Abilità puoi trasformarti anche in una creatura con queste caratteristiche:

\medskip

\textbf{Requisito}: Competenza Magica 10

\textbf{Tipologia Creature}: Elementali

\textbf{Caratteristiche}: il Personaggio sceglie se le Caratteristiche fisiche, Difesa, Tiri Salvezza sono proprie o dell'animale.

\textbf{Incantesimi}: puoi lanciare incantesimi nella nuova forma purché abbiano componenti solo Verbali

\textbf{Equipaggiamento}: quello magico continua ad avere effetto se possibile. Armature e Scudi applicano il bonus magico alla Difesa della creatura e le capacità magiche degli oggetti non possono essere attivate.

\medskip

La \textbf{quarta volta} che prendi questa Abilità puoi trasformarti anche in una creatura con queste caratteristiche:

\medskip

\textbf{Requisito}: Competenza Magica 16

\textbf{Tipologia Creature}: Mostruosità

\textbf{Caratteristiche}: il Personaggio sceglie se le Caratteristiche fisiche, Difesa, Tiri Salvezza sono proprie o dell'animale. I Punti Ferita rimangono quelli del personaggio.

\textbf{Incantesimi}: puoi lanciare incantesimi nella nuova forma purché abbiano componenti solo Verbali e Somatici

\textbf{Equipaggiamento}: quello magico continua ad avere effetto se possibile. Armature e Scudi applicano il bonus magico alla Difesa della creatura ed eventuali capacità magiche possono essere attivate.

\medskip

\textbf{Regole base per la trasformazione}

\medskip

L'\textbf{equipaggiamento} viene assorbito nella nuova forma e non può essere usato. Quello magico può funzionare ed essere usato come indicato.

La creatura in cui ti trasformi deve avere un \textbf{Grado di Sfida} entro un terzo del tuo punteggio di Competenza Magica + le volte che hai preso l'Abilità Animalia.

I \textbf{Punti Ferita} rimangono quelli attuali del personaggio. La \textbf{taglia} della creatura in cui ti trasformi deve essere entro la tua $\pm 1$ ogni 2 volte che hai preso questa Abilità.

Puoi \textbf{rimanere trasformato} per 1 minuto per somma Tratto in comune con il Patrono  o per punto in Competenza Magica, con utilizzo minimo di 1 minuto.

Costa 2 Azioni cambiare forma e prima di passare da una forma all'altra è necessario tornare in forma normale.

Il personaggio conserva i propri Tratti, personalità, Abilità (ma non è detto che la nuova forma gli permetta di usarle) e caratteristiche mentali.

Se la creatura possiede una competenza che anche il personaggio possiede ed il bonus della creatura è superiore a quello del personaggio allora usa il bonus della creatura anziché il proprio. Se la creatura possiede delle azioni aggiuntive o di tana, il personaggio non può usarle.

Qualsiasi azione che richieda le mani è limitata alle capacità della sua nuova forma. La trasformazione non interrompe la concentrazione del personaggio su un incantesimo che egli ha già lanciato e non gli impedisce di effettuare azioni che fanno parte di un incantesimo già lanciato, come per esempio Invocare il Fulmine.

Le forme di attacco sono sempre quelle delle creatura.

Della nuova forma acquisisce le caratteristiche e capacità, come sensi, movimento, lingue (ma non è detto che che possa parlare altre lingue oltre quella dell'animale).

Quando sei trasformato puoi canalizzare i tuoi Punti Magia per migliorare la trasformazione, per ogni Punto Magia usato prendi un +1 al Tiro per Colpire, al danno con gli attacchi, Difesa e Tiri Salvezza. La capacità va dichiarata all'inizio del round come Azione Immediata che dura fino all'inizio del tuo round successivo. Non puoi usare più Punti Magia alla volta di quante volte tu abbia preso l'Abilità Animalia.

\begin{center}
\includegraphics[width=0.9\linewidth]{immagini/animalia3.png}
\emph{Henry Justice Ford}
\end{center}

\smallskip\noindent\rule{\linewidth}{2pt} \index[Abilita]{Animaletto / Famiglio}\hypertarget{Animaletto / Famiglio}{}\medskip\noindent{\textbf{Animaletto / Famiglio}}\pdfbookmark[3]{Animaletto / Famiglio}{Animaletto / Famiglio}
\noindent
\begin{description}[noitemsep, topsep=0pt, parsep=0pt, partopsep=0pt, leftmargin=0cm, labelwidth=2.5cm]
    \item[\textbf{Requisito}]: Competenza Magica 1
    \item[\textbf{Tiri Salvezza}]: +1 Volontà, +1 Tempra
    \item[\textbf{Caratteristica}]: Intelligenza o Modificatore di caratteristica per incantesimi
\end{description}

La \textbf{prima volta} che prendi questa Abilità guadagni un animale naturale. Questo animaletto ha un Grado di Sfida pari ad un quarto della tua Saggezza, con un minimo di 1/4. Puoi insegnare azioni di base al tuo animale e fargli fare dei compiti semplici.

La \textbf{seconda volta} che prendi questa Abilità guadagni un \hyperlink{famiglio}{Famiglio} (pag. \pageref{famiglio}).

\smallskip\noindent\rule{\linewidth}{2pt} \index[Abilita]{Armato}\hypertarget{Armato}{}\medskip\noindent{\textbf{Armato}}\pdfbookmark[3]{Armato}{Armato}
\noindent
\begin{description}[noitemsep, topsep=0pt, parsep=0pt, partopsep=0pt, leftmargin=0cm, labelwidth=2.5cm]
    \item[\textbf{Requisito}]: Forza 3, Competenza Armi 1
    \item[\textbf{Tiri Salvezza}]: +2 Tempra
    \item[\textbf{Caratteristica}]: Forza o Costituzione
\end{description}

La \textbf{prima volta} che prendi questa Abilità quando usi un arma di una taglia troppo grande la penalità al colpire diventa -2.

La \textbf{seconda volta} che prendi questa Abilità, requisito Competenza Armi 6, non hai penalità nell'usare un arma di una taglia superiore.

\smallskip\noindent\rule{\linewidth}{2pt} \index[Abilita]{Armatura del Devoto}\hypertarget{Armatura del Devoto}{}\medskip\noindent{\textbf{Armatura del Devoto}}\pdfbookmark[3]{Armatura del Devoto}{Armatura del Devoto}
\noindent
\begin{description}[noitemsep, topsep=0pt, parsep=0pt, partopsep=0pt, leftmargin=0cm, labelwidth=2.5cm]
    \item[\textbf{Requisito}]: Valore singolo Tratto in comune con il Patrono 4, essere Devoto o Seguace
    \item[\textbf{Tiri Salvezza}]: +2 Volontà, +1 Riflessi
    \item[\textbf{Caratteristica}]: Costituzione o Modificatore di caratteristica per incantesimi
\end{description}

La \textbf{prima volta} che prendi questa Abilità il costante allenamento con la tua armatura riduce di 2 la penalità alla Prova di Magia quando indossi armature leggere.

La \textbf{seconda volta} che si prende questa Abilità, requisito singolo Tratto 6, riduci di 4 la penalità alla Prova di Magia quando indossi armature medie.

La \textbf{terza volta} che si prende questa Abilità, requisito singolo Tratto 8, il portare armature leggere non ti obbliga a fare Prove di Magia, riduci la penalità per armature medie di 6 e riduci la penalità per le armature pesanti di 8.

La \textbf{quarta volta} che si prende l'Abilità, requisito singolo Tratto 12, la Prova di Magia per indossare armature è obbligatoria solo se indossi armature pesanti e riduci la penalità di 12.

\smallskip\noindent\rule{\linewidth}{2pt} \index[Abilita]{Armatura della Montagna Incantata}\hypertarget{Armatura della Montagna Incantata}{}\medskip\noindent{\textbf{Armatura della Montagna Incantata}}\pdfbookmark[3]{Armatura della Montagna Incantata}{Armatura della Montagna Incantata}
\noindent
\begin{description}[noitemsep, topsep=0pt, parsep=0pt, partopsep=0pt, leftmargin=0cm, labelwidth=2.5cm]
    \item[\textbf{Requisito}]: Lista armi Pugno Vuoto 1, Competenza Armi 1, Costituzione 2, Saggezza 1
    \item[\textbf{Tiri Salvezza}]: +2 Tempra, +1 Volontà
    \item[\textbf{Caratteristica}]: Costituzione o Saggezza
\end{description}

Il constante allenamento nello spirito e corpo di permette di indurire la tua pelle e renderla più difficile da ferire. Per usufruire di questi bonus non devi portare armature o scudi od oggetti che migliorino la Difesa. Le capacità elencate non sono cumulabili con l'Abilità Gru d'Argento.

La \textbf{prima volta} che prendi questa Abilità la tua Difesa è 10 + Costituzione + 1/3 dei punti in Pugno Vuoto + varie ed eventuali.

La \textbf{seconda volta} che prendi questa Abilità, requisito Pugno Vuoto 5, acquisisci una riduzione al danno (DR) di 1/-

La \textbf{terza volta} che prendi questa Abilità, requisito Pugno Vuoto 7, riduci automaticamente il Sanguinamento di 1 a fine round.

La \textbf{quarta volta} che prendi questa Abilità, requisito Pugno Vuoto 8, acquisisci una riduzione al danno (DR) di 3/-

La \textbf{quinta volta} che prendi questa Abilità, requisito Pugno Vuoto 13, acquisisci una riduzione al danno (DR) di 5/-

\smallskip\noindent\rule{\linewidth}{2pt} \index[Abilita]{Arciere su saurovallo}\hypertarget{Arciere su saurovallo}{}\medskip\noindent{\textbf{Arciere su saurovallo}}\pdfbookmark[3]{Arciere su saurovallo}{Arciere su saurovallo}
\noindent
\begin{description}[noitemsep, topsep=0pt, parsep=0pt, partopsep=0pt, leftmargin=0cm, labelwidth=2.5cm]
    \item[\textbf{Requisito}]: Competenza Armi 1
    \item[\textbf{Tiri Salvezza}]: +1 Riflessi, +1 Tempra
    \item[\textbf{Caratteristica}]: Destrezza o Saggezza
\end{description}

le penalità di tirare frecce da saurovallo diminuisce di 2 ogni volta che prendi questa Abilità.

Le penalità standard sono -4 e -6 a seconda che si trotti (movimento x2) o galoppi (movimento x3)

\medskip

\begin{center}
\includegraphics[width=0.9\linewidth]{immagini/horsearcher.png}

\emph{Arcere Assiro}
\end{center}

\smallskip\noindent\rule{\linewidth}{2pt} \index[Abilita]{Arma Focalizzata}\hypertarget{Arma Focalizzata}{}\medskip\noindent{\textbf{Arma Focalizzata}}\pdfbookmark[3]{Arma Focalizzata}{Arma Focalizzata}
\noindent
\begin{description}[noitemsep, topsep=0pt, parsep=0pt, partopsep=0pt, leftmargin=0cm, labelwidth=2.5cm]
    \item[\textbf{Requisito}]: Competenza Armi 1
    \item[\textbf{Tiri Salvezza}]: +1 Riflessi, +1 Tempra
    \item[\textbf{Caratteristica}]: Forza o Destrezza
\end{description}

Scegli un arma in una Lista d'Armi che conosci. Ottieni un +1 a Iniziativa e Tiro per Colpire quando usi questa arma.

\smallskip\noindent\rule{\linewidth}{2pt} \index[Abilita]{Artista dell'Arma}\hypertarget{Artista dell'Arma}{}\medskip\noindent{\textbf{Artista dell'Arma}}\pdfbookmark[3]{Artista dell'Arma}{Artista dell'Arma}
\noindent
\begin{description}[noitemsep, topsep=0pt, parsep=0pt, partopsep=0pt, leftmargin=0cm, labelwidth=2.5cm]
    \item[\textbf{Requisito}]: Competenza Armi 2
    \item[\textbf{Tiri Salvezza}]: +1 Volontà, +1 Tempra
    \item[\textbf{Caratteristica}]: Destrezza o Forza
\end{description}

Scegli una Lista d'Armi, su queste armi ottieni un +1 al colpire.

L'Abilità può essere presa più volte, con almeno CA 5,9,13,17.

Se prendi \textbf{4 volte} questa Abilità sulla stessa Lista d'Armi i bonus al colpire si riducono a +1, invece che +4, ma effettui due Tiri per Colpire per i primi due attacchi del round e scegli il tiro da tenere.

\smallskip\noindent\rule{\linewidth}{2pt} \index[Abilita]{Attacco Turbinante}\hypertarget{Attacco Turbinante}{}\medskip\noindent{\textbf{Attacco Turbinante}}\pdfbookmark[3]{Attacco Turbinante}{Attacco Turbinante}
\noindent
\begin{description}[noitemsep, topsep=0pt, parsep=0pt, partopsep=0pt, leftmargin=0cm, labelwidth=2.5cm]
    \item[\textbf{Requisito}]: Competenza Armi 12, Intrattenere 3
    \item[\textbf{Tiri Salvezza}]: +2 Riflessi, +1 Tempra
    \item[\textbf{Caratteristica}]: Destrezza o Carisma
\end{description}

La \textbf{prima volta} che prendi questa Abilità usando 3 Azioni puoi eseguire un singolo attacco (con penalità di 1d6 al Tiro per Colpire) contro tutti gli avversari in mischia attorno a te.

La \textbf{seconda volta} che prendi questa Abilità, Competenza Armi 15, Intrattenere 5, non hai la penalità al Tiro per Colpire.

\smallskip\noindent\rule{\linewidth}{2pt} \index[Abilita]{Batteria Magica}\hypertarget{Batteria Magica}{}\medskip\noindent{\textbf{Batteria Magica}}\pdfbookmark[3]{Batteria Magica}{Batteria Magica}
\noindent
\begin{description}[noitemsep, topsep=0pt, parsep=0pt, partopsep=0pt, leftmargin=0cm, labelwidth=2.5cm]
    \item[\textbf{Requisito}]: Competenza Magica 3
    \item[\textbf{Tiri Salvezza}]: +2 Volontà, +1 Tempra
    \item[\textbf{Caratteristica}]: Modificatore di caratteristica per incantesimi
\end{description}

Hai una particolare connessione con la magia che permane la Terra.

La prima volta che prendi questa Abilità aumenti di 3 i punti Magia a disposizione.

L'Abilità può essere presa più volte ed il totale deve essere pari o inferiore a CM/3.

\smallskip\noindent\rule{\linewidth}{2pt} \index[Abilita]{Batteria Estesa}\hypertarget{Batteria Estesa}{}\medskip\noindent{\textbf{Batteria Estesa}}\pdfbookmark[3]{Batteria Estesa}{Batteria Estesa}
\noindent
\begin{description}[noitemsep, topsep=0pt, parsep=0pt, partopsep=0pt, leftmargin=0cm, labelwidth=2.5cm]
    \item[\textbf{Requisito}]: Competenza Magica 1, Adepto della Magia
    \item[\textbf{Tiri Salvezza}]: + 1 Tempra, +1 Volontà
    \item[\textbf{Caratteristica}]: Modificatore di caratteristica per incantesimi
\end{description}

Riesci a sopportare meglio lo stress di lanciare incantesimi.

Quando effettui una Prova di Magia e riesci in almeno un Successo Critico Magico il costo dell'incantesimo diminuisce di un punto, con un costo minimo di 1.

\smallskip\noindent\rule{\linewidth}{2pt} \index[Abilita]{Colpo Furtivo}\hypertarget{Colpo Furtivo}{}\medskip\noindent{\textbf{Colpo Furtivo}}\pdfbookmark[3]{Colpo Furtivo}{Colpo Furtivo}
\noindent
\begin{description}[noitemsep, topsep=0pt, parsep=0pt, partopsep=0pt, leftmargin=0cm, labelwidth=2.5cm]
    \item[\textbf{Requisito}]: Competenza Armi 3
    \item[\textbf{Tiri Salvezza}]: +2 Riflessi, +1 Volontà
    \item[\textbf{Caratteristica}]: Destrezza o Intelligenza
\end{description}

La \textbf{prima volta} che prendi questa Abilità quando l'avversario viene \hyperlink{sorpresa}{sorpreso} (vedi pag. \pageref{coltidisorpresa}) con un arma da mischia se il primo attacco del combattimento colpisce causa un danno critico aggiuntivo.

La \textbf{seconda volta} che si prende questa Abilità, requisito Competenza Armi 6, causi 2 danni critici aggiuntivi.

La \textbf{terza volta} che si prende questa Abilità, requisito Competenza Armi 10, causi 3 danni critici aggiuntivi.

La \textbf{quarta} che si prende questa Abilità, requisito Competenza Armi 12, causi 4 danni critici aggiuntivi.

\begin{center}
	\includegraphics[width=0.7\linewidth]{immagini/teseo.png}

	\emph{Henry Justice Ford - Colpo furtivo!}
\end{center}

\smallskip\noindent\rule{\linewidth}{2pt} \index[Abilita]{Colpo Indebolente}\hypertarget{Colpo Indebolente}{}\medskip\noindent{\textbf{Colpo Indebolente}}\pdfbookmark[3]{Colpo Indebolente}{Colpo Indebolente}
\noindent
\begin{description}[noitemsep, topsep=0pt, parsep=0pt, partopsep=0pt, leftmargin=0cm, labelwidth=2.5cm]
    \item[\textbf{Requisito}]: Colpo furtivo 3, Competenza Armi 12
    \item[\textbf{Tiri Salvezza}]: +2 Riflessi, +1 Volontà
    \item[\textbf{Caratteristica}]: Intelligenza o Destrezza
\end{description}

Colpo Indebolente è una forma avanzata di colpo furtivo. Ogni Colpo Indebolente abbassa Forza o Destrezza (scelta giocatore) di quante volte si è preso Colpo Furtivo.

All'avversario è concesso un Tiro Salvezza Tempra con DC pari al Tiro per Colpire. Si causa il danno aggiuntivo del Colpo Furtivo o la perdita di punti caratteristica.

\smallskip\noindent\rule{\linewidth}{2pt} \index[Abilita]{Colpo Mortale}\hypertarget{Colpo Mortale}{}\medskip\noindent{\textbf{Colpo Mortale}}\pdfbookmark[3]{Colpo Mortale}{Colpo Mortale}
\noindent
\begin{description}[noitemsep, topsep=0pt, parsep=0pt, partopsep=0pt, leftmargin=0cm, labelwidth=2.5cm]
    \item[\textbf{Requisito}]: Competenza Armi 5
    \item[\textbf{Tiri Salvezza}]: +2 Riflessi, +1 Volontà
    \item[\textbf{Caratteristica}]: Saggezza o Forza
\end{description}

Esegui il Tiro per Colpire con penalità di -1d6, se colpisci causi 2 danni critici. I Tiri per Colpire successivi partono da -10 al colpire.

\smallskip\noindent\rule{\linewidth}{2pt} \index[Abilita]{Colpo Paralizzante}\hypertarget{Colpo Paralizzante}{}\medskip\noindent{\textbf{Colpo Paralizzante}}\pdfbookmark[3]{Colpo Paralizzante}{Colpo Paralizzante}
\noindent
\begin{description}[noitemsep, topsep=0pt, parsep=0pt, partopsep=0pt, leftmargin=0cm, labelwidth=2.5cm]
    \item[\textbf{Requisito}]: Colpo Indebolente, Colpo furtivo 4, Competenza Armi 18
    \item[\textbf{Tiri Salvezza}]: +2 Riflessi, +1 Tempra
    \item[\textbf{Caratteristica}]: Forza o Destrezza
\end{description}

Dedichi 2 Azioni a Round, per 5 round, a studiare un avversario che puoi minacciare. Nel sesto round utilizzando 2 Azioni porti un attacco in mischia o da distanza. L'avversario deve effettuare un Tiro Salvezza su Tempra con DC pari al Tiro per Compire o rimanere paralizzato per 3d6 round. La creatura non deve essere di 2 taglie più grande della tua.

\smallskip\noindent\rule{\linewidth}{2pt} \index[Abilita]{Colpi Poderosi}\hypertarget{Colpi Poderosi}{}\medskip\noindent{\textbf{Colpi Poderosi}}\pdfbookmark[3]{Colpi Poderosi}{Colpi Poderosi}
\noindent
\begin{description}[noitemsep, topsep=0pt, parsep=0pt, partopsep=0pt, leftmargin=0cm, labelwidth=2.5cm]
    \item[\textbf{Requisito}]: Competenza Armi 1
    \item[\textbf{Tiri Salvezza}]: +2 Tempra
    \item[\textbf{Caratteristica}]: Forza o Costituzione
\end{description}

Il tuo stile enfatizza colpi poderosi.

Guadagni un +1 al danno con una Lista d'Arma.

\smallskip\noindent\rule{\linewidth}{2pt} \index[Abilita]{Combattere alla Cieca}\hypertarget{Combattere alla Cieca}{}\medskip\noindent{\textbf{Combattere alla Cieca}}\pdfbookmark[3]{Combattere alla Cieca}{Combattere alla Cieca}
\noindent
\begin{description}[noitemsep, topsep=0pt, parsep=0pt, partopsep=0pt, leftmargin=0cm, labelwidth=2.5cm]
    \item[\textbf{Requisito}]: Consapevolezza 2
    \item[\textbf{Tiri Salvezza}]: +2 Riflessi, +1 Volontà
    \item[\textbf{Caratteristica}]: Destrezza o Saggezza
\end{description}

La \textbf{prima volta} che prendi questa Abilità un avversario con copertura leggera non ottiene bonus alla Difesa, con copertura media ha un +2 alla Difesa, con copertura completa ha un +6 alla Difesa.

Un attaccante invisibile in mischia non ottiene alcun vantaggio al colpire il personaggio in mischia.

La \textbf{seconda volta} che prendi l'Abilità, requisito Consapevolezza a 3, riduci di ulteriori due il bonus alla Difesa da creature con copertura completa.

Non c'è bisogno di effettuare prove di Acrobatica per muoversi a piena velocità mentre si è Accecati.

La penalità al Tiro per Colpire contro creature invisibili è -2.

\emph{Livello Zatoichi}, la \textbf{terza volta} che prendi l'Abilità, requisito Consapevolezza a 5, in mischia una creatura invisibile non ha alcun vantaggio contro di te ne tu hai penalità contro di lei.

\smallskip\noindent\rule{\linewidth}{2pt} \index[Abilita]{Combattimento con due armi}\hypertarget{Combattimento con due armi}{}\medskip\noindent{\textbf{Combattimento con due armi}}\pdfbookmark[3]{Combattimento con due armi}{Combattimento con due armi}
\noindent
\begin{description}[noitemsep, topsep=0pt, parsep=0pt, partopsep=0pt, leftmargin=0cm, labelwidth=2.5cm]
    \item[\textbf{Requisito}]: Destrezza 2, Forza 1, Competenza Armi 2
    \item[\textbf{Tiri Salvezza}]: +2 Riflessi, +1 Tempra
    \item[\textbf{Caratteristica}]: Destrezza o Forza
\end{description}

La \textbf{prima volta} che prendi questa Abilità il constante è continuo allenamento ti permette di ridurre la penalità del multiattacco dato dall'attacco con l'arma secondaria. Quando attacchi con l'arma secondaria cumuli penalità al colpire di -4 al posto di -5 se l'arma è leggera.

\textbf{Requisito} Destrezza 3, Competenza Armi 12

La \textbf{seconda volta} se l'arma secondaria non è leggera non cumuli l'ulteriore -3 al colpire.

\textbf{Requisito} Competenza Armi 18

La \textbf{terza volta} il primo attacco effettuato con l'arma secondaria non cumula la penalità degli attacchi multipli.

\smallskip\noindent\rule{\linewidth}{2pt} \index[Abilita]{Concentrato}\hypertarget{Concentrato}{}\medskip\noindent{\textbf{Concentrato}}\pdfbookmark[3]{Concentrato}{Concentrato}
\noindent
\begin{description}[noitemsep, topsep=0pt, parsep=0pt, partopsep=0pt, leftmargin=0cm, labelwidth=2.5cm]
    \item[\textbf{Requisito}]: Competenza Magica 2
    \item[\textbf{Tiri Salvezza}]: +1 Tempra, +1 Volontà
    \item[\textbf{Caratteristica}]: Modificatore di caratteristica per incantesimi
\end{description}

Scegli una Lista di Magia, la DC dei Tiri Salvezza dei tuoi incantesimi in quella lista aumenta di 1.

L'Abilità può essere presa più volte sulla stessa Lista di Magia o su altre liste ed il totale deve essere pari o inferiore a CM/4.

\smallskip\noindent\rule{\linewidth}{2pt} \index[Abilita]{Conoscenza istintiva}\hypertarget{Conoscenza istintiva}{}\medskip\noindent{\textbf{Conoscenza istintiva}}\pdfbookmark[3]{Conoscenza istintiva}{Conoscenza istintiva}
\noindent
\begin{description}[noitemsep, topsep=0pt, parsep=0pt, partopsep=0pt, leftmargin=0cm, labelwidth=2.5cm]
    \item[\textbf{Requisito}]: Conoscenza 1
    \item[\textbf{Tiri Salvezza}]: +2 Volontà, +1 Tempra
    \item[\textbf{Caratteristica}]: Saggezza o Intelligenza
\end{description}

Non dimentichi mai un nemico.

Hai una istintiva capacità nel ricordare e valutare un nemico. Quando prendi questa Abilità puoi effettuare una prova di \hyperlink{riconoscereimostri}{Riconoscere un Mostro} (pag. \pageref{riconoscereimostri}) utilizzando una Reazione.

\begin{center}
	\includegraphics[width=0.8\linewidth]{immagini/oggettimagiciuomo.png}

	\emph{Henry Purcell - King Arthur}
\end{center}

\smallskip\noindent\rule{\linewidth}{2pt} \index[Abilita]{Creare Oggetti Magici}\hypertarget{Creare Oggetti Magici}{}\medskip\noindent{\textbf{Creare Oggetti Magici}}\pdfbookmark[3]{Creare Oggetti Magici}{Creare Oggetti Magici}
\noindent
\begin{description}[noitemsep, topsep=0pt, parsep=0pt, partopsep=0pt, leftmargin=0cm, labelwidth=2.5cm]
    \item[\textbf{Requisito}]: Competenza Magica 6
    \item[\textbf{Tiri Salvezza}]: +1 Tempra, +1 Volontà
    \item[\textbf{Caratteristica}]: Modificatore di caratteristica per incantesimi o a scelta
\end{description}

La \textbf{prima volta} che prendi questa Abilità tramite questa Abilità l'incantatore è in grado di infondere un incantesimo fino a livello 3 in un oggetto magico.

Le \textbf{seconda volta} che prendi questa Abilità, requisito Competenza Magica 12, l'incantatore è in grado di infondere un incantesimo fino a livello 5 in un oggetto magico.

Le \textbf{terza volta} che prendi questa Abilità, requisito Competenza Magica 16, l'incantatore è in grado di infondere un incantesimo fino a livello 8 in un oggetto magico.

Le \textbf{quarta volta} che prendi questa Abilità, requisito Competenza Magica 18, l'incantatore è in grado di infondere un incantesimo fino a livello 9 in un oggetto magico.

\smallskip\noindent\rule{\linewidth}{2pt} \index[Abilita]{Dadi Truccati}\hypertarget{Dadi Truccati}{}\medskip\noindent{\textbf{Dadi Truccati}}\pdfbookmark[3]{Dadi Truccati}{Dadi Truccati}
\noindent
\begin{description}[noitemsep, topsep=0pt, parsep=0pt, partopsep=0pt, leftmargin=0cm, labelwidth=2.5cm]
    \item[\textbf{Requisito}]: Competenza Magica 6
    \item[\textbf{Tiri Salvezza}]: +1 Tempra, +1 Riflessi
    \item[\textbf{Caratteristica}]: Saggezza o Carisma
\end{description}

La \textbf{prima volta} che prendi questa Abilità puoi aumentare di 1, entro il valore di 6, un dado nella Prova di Magia.

La \textbf{seconda volta} che prendi questa Abilità, requisito Competenza Magica 12, puoi aumentare di 1, entro il valore di 6, un ulteriore dado nella Prova di Magia.

\smallskip\noindent\rule{\linewidth}{2pt} \index[Abilita]{Danno Coordinato}\hypertarget{Danno Coordinato}{}\medskip\noindent{\textbf{Danno Coordinato}}\pdfbookmark[3]{Danno Coordinato}{Danno Coordinato}
\noindent
\begin{description}[noitemsep, topsep=0pt, parsep=0pt, partopsep=0pt, leftmargin=0cm, labelwidth=2.5cm]
    \item[\textbf{Requisito}]: Competenza Armi 6, Saggezza 2
    \item[\textbf{Tiri Salvezza}]: +2 Volontà
    \item[\textbf{Caratteristica}]: Carisma o Intelligenza
\end{description}

La tua esperienza nel gestire gli alleati ti permette di massimizzare l'efficacia degli attacchi.

La \textbf{prima volta} che prendi questa Abilità puoi coordinare gli attacchi di due tuoi alleati, che siano a distanza di mischia tra loro, affinché il danno causato da uno colpisca il nemico dell'altro e vice versa. Costa 2 Azioni eseguire questo coordinamento.

La \textbf{seconda volta} che prendi questa Abilità, requisito Competenza Armi 8, Intelligenza 2, puoi coordinare e scambiare il danno di tre alleati purché in distanza di mischia tra loro. Costo 2 Azioni.

E' necessario che i Tiri per Colpire vadano a segno per poter applicare il danno all'altro avversario.

\smallskip\noindent\rule{\linewidth}{2pt} \index[Abilita]{Danza della Lama}\hypertarget{Danza della Lama}{}\medskip\noindent{\textbf{Danza della Lama}}\pdfbookmark[3]{Danza della Lama}{Danza della Lama}
\noindent
\begin{description}[noitemsep, topsep=0pt, parsep=0pt, partopsep=0pt, leftmargin=0cm, labelwidth=2.5cm]
    \item[\textbf{Requisito}]: Lista d'Armi: Armi Aggraziate a 2, Destrezza o Carisma 1, Intrattenere 1
    \item[\textbf{Tiri Salvezza}]: +2 Riflessi, +1 Tempra
    \item[\textbf{Caratteristica}]: Carisma o Destrezza
\end{description}

La \textbf{prima volta} che prendi questa Abilità quando usi Armi Aggraziate puoi sostituire il solo danno dato dalla Forza negli attacchi di mischia con metà del valore del Carisma o Destrezza.

La \textbf{seconda volta}, requisito Armi Aggraziate 4, Intrattenere 3, che prendi l'Abilità puoi usare il Carisma come modificatore al danno dell'arma, ignorando il danno dato dalla Forza.

La \textbf{terza volta}, requisito Armi Aggraziate 7, Intrattenere 5, che prendi l'Abilità puoi usare la Destrezza od il Carisma come modificatore al danno dell'arma, ignorando il danno dato dalla Forza.

Il secondo e terzo vantaggio non sono cumulativi.

\smallskip\noindent\rule{\linewidth}{2pt} \index[Abilita]{Daredevil}\hypertarget{Daredevil}{}\medskip\noindent{\textbf{Daredevil}}\pdfbookmark[3]{Daredevil}{Daredevil}
\noindent
\begin{description}[noitemsep, topsep=0pt, parsep=0pt, partopsep=0pt, leftmargin=0cm, labelwidth=2.5cm]
    \item[\textbf{Requisito}]: Competenza Armi +2, Destrezza 1
    \item[\textbf{Tiri Salvezza}]: +2 Riflessi, +1 Tempra
    \item[\textbf{Caratteristica}]: Destrezza o Costituzione
\end{description}

Ti piace buttarti nella mischia, specialmente se si corrono pericoli! I bonus sono cumulativi.

La \textbf{prima volta} che prendi questa Abilità hai un +1 al Tiro per Colpire in mischia ed alla Difesa sei in mischia con 3 o più avversari.

La \textbf{seconda volta} che prendi questa Abilità hai un +2 al Tiro per Colpire in mischia ed alla Difesa sei in mischia con 2 o più avversari.

\smallskip\noindent\rule{\linewidth}{2pt} \index[Abilita]{Dattilografo}\hypertarget{Dattilografo}{}\medskip\noindent{\textbf{Dattilografo}}\pdfbookmark[3]{Dattilografo}{Dattilografo}
\noindent
\begin{description}[noitemsep, topsep=0pt, parsep=0pt, partopsep=0pt, leftmargin=0cm, labelwidth=2.5cm]
    \item[\textbf{Requisito}]: Competenza Magica 1
    \item[\textbf{Tiri Salvezza}]: +1 Tempra, +1 Volontà
    \item[\textbf{Caratteristica}]: Modificatore di caratteristica per incantesimi o Destrezza
\end{description}

Sei estremamente rapido nel copiare nuovi incantesimi sul tuo Tomo della Magia. Il tempo per copiare un incantesimo passa da 1 ora a 30 minuti a pagina (un incantesimo occupa un numero di pagine pari al proprio livello). Il costo in inchiostri passa da 10 mo a pagina a 5 mo a pagina.

\smallskip\noindent\rule{\linewidth}{2pt} \index[Abilita]{Decifrare scritti magici}\hypertarget{Decifrare scritti magici}{}\medskip\noindent{\textbf{Decifrare scritti magici}}\pdfbookmark[3]{Decifrare scritti magici}{Decifrare scritti magici}
\noindent
\begin{description}[noitemsep, topsep=0pt, parsep=0pt, partopsep=0pt, leftmargin=0cm, labelwidth=2.5cm]
    \item[\textbf{Requisito}]: Competenza Magica 1
    \item[\textbf{Tiri Salvezza}]: +1 Tempra, +1 Volontà
    \item[\textbf{Caratteristica}]: Modificatore di caratteristica per incantesimi o Saggezza
\end{description}

Ha un bonus di +1d6 nel comprendere il contenuto di una pergamena e nel lanciare l'incantesimo contenuto. Il bonus si applica anche alla prova per copiare un incantesimo sul proprio Tomo della magia.

\smallskip\noindent\rule{\linewidth}{2pt} \index[Abilita]{Difendere Cavalcatura}\hypertarget{Difendere Cavalcatura}{}\medskip\noindent{\textbf{Difendere Cavalcatura}}\pdfbookmark[3]{Difendere Cavalcatura}{Difendere Cavalcatura}
\noindent
\begin{description}[noitemsep, topsep=0pt, parsep=0pt, partopsep=0pt, leftmargin=0cm, labelwidth=2.5cm]
    \item[\textbf{Requisito}]: Cavalcare 1
    \item[\textbf{Tiri Salvezza}]: +1 Tempra, +1 Riflessi
    \item[\textbf{Caratteristica}]: Destrezza o Saggezza
\end{description}

Ogni qual volta la cavalcatura viene colpita, puoi effettuare una prova di Cavalcare per negare il colpo.

La tua prova di Cavalcare deve essere maggiore del Tiro per Colpire dell'avversario

L'Abilità è utilizzabile solo una volta per round, per un solo attacco, costa la Reazione.

\smallskip\noindent\rule{\linewidth}{2pt} \index[Abilita]{Difesa pronta}\hypertarget{Difesa pronta}{}\medskip\noindent{\textbf{Difesa pronta}}\pdfbookmark[3]{Difesa pronta}{Difesa pronta}
\noindent
\begin{description}[noitemsep, topsep=0pt, parsep=0pt, partopsep=0pt, leftmargin=0cm, labelwidth=2.5cm]
    \item[\textbf{Requisito}]: Competenza Armi 2
    \item[\textbf{Tiri Salvezza}]: +2 Riflessi
    \item[\textbf{Caratteristica}]: Destrezza o Intelligenza
\end{description}

Sei sempre vigile ed attento quando rischi la vita.

Hai un +2 alla Difesa contro gli attacchi di opportunità, alle spalle, o da fiancheggiato.

\smallskip\noindent\rule{\linewidth}{2pt} \index[Abilita]{Distillare pozioni}\hypertarget{Distillare pozioni}{}\medskip\noindent{\textbf{Distillare pozioni}}\pdfbookmark[3]{Distillare pozioni}{Distillare pozioni}
\noindent
\begin{description}[noitemsep, topsep=0pt, parsep=0pt, partopsep=0pt, leftmargin=0cm, labelwidth=2.5cm]
    \item[\textbf{Requisito}]: Competenza Magica 1
    \item[\textbf{Tiri Salvezza}]: +1 Tempra, +1 Volontà
    \item[\textbf{Caratteristica}]: Saggezza o Intelligenza
\end{description}

Sei più che competente nel distillare pozioni.

La \textbf{prima volta} che prendi questa Abilità acquisisci un bonus di +1d6 su Conoscenze Erboristeria, distillare creare pozioni e veleni naturali.

La \textbf{seconda volta} che prendi l'Abilità il tempo per preparare le pozioni/veleni viene dimezzato ed in caso di Fallimento Critico non ci si espone al prodotto. Dedicando un ora al giorno puoi creare una Pozione generica di Cura od una Indebolente con le erbe che trovi li intorno. Questa pozione \emph{scade} all'alba del giorno dopo la creazione.

\smallskip\noindent\rule{\linewidth}{2pt} \index[Abilita]{Doppia porzione}\hypertarget{Doppia porzione}{}\medskip\noindent{\textbf{Doppia porzione}}\pdfbookmark[3]{Doppia porzione}{Doppia porzione}
\noindent
\begin{description}[noitemsep, topsep=0pt, parsep=0pt, partopsep=0pt, leftmargin=0cm, labelwidth=2.5cm]
    \item[\textbf{Requisito}]: Combattimento con due armi, Competenza Armi 4
    \item[\textbf{Tiri Salvezza}]: +2 Tempra, +1 Riflessi
    \item[\textbf{Caratteristica}]: Forza o Costituzione
\end{description}

Il costante allenamento con due armi ti permette di applicare il bonus al danno dovuto alla Forza in maniera piena anche all'arma secondaria.

\smallskip\noindent\rule{\linewidth}{2pt} \index[Abilita]{Duro a morire}\hypertarget{Duro a morire}{}\medskip\noindent{\textbf{Duro a morire}}\pdfbookmark[3]{Duro a morire}{Duro a morire}\label{Duro a morire}
\noindent
\begin{description}[noitemsep, topsep=0pt, parsep=0pt, partopsep=0pt, leftmargin=0cm, labelwidth=2.5cm]
    \item[\textbf{Requisito}]: -
    \item[\textbf{Tiri Salvezza}]: +1 Tempra, +1 Volontà
    \item[\textbf{Caratteristica}]: Costituzione o Saggezza
\end{description}

Sei particolarmente ostinato nel non volere morire. Il personaggio aumenta di 3 Punti Ferita la tolleranza prima di morire, ovvero muore a 13+COS*2.

\smallskip\noindent\rule{\linewidth}{2pt} \index[Abilita]{Energia Psichica}\hypertarget{Energia Psichica}{}\medskip\noindent{\textbf{Energia Psichica}}\pdfbookmark[3]{Energia Psichica}{Energia Psichica}
\noindent
\begin{description}[noitemsep, topsep=0pt, parsep=0pt, partopsep=0pt, leftmargin=0cm, labelwidth=2.5cm]
    \item[\textbf{Requisito}]: Forza 1, Saggezza 2, Competenza Armi 1, Competenza Magica 1
    \item[\textbf{Tiri Salvezza}]: +2 Volontà, +1 Tempra
    \item[\textbf{Caratteristica}]: Saggezza o Carisma
\end{description}

Dopo anni di allenamento, meditazione e stage a Panda Barbat sei in grado di raccogliere la tua Energia Chi.

La \textbf{prima volta} che prendi questa Abilità ogni giorno, dopo almeno 6 ore di riposo e 2 ore di meditazione/allenamento, riempi il tuo corpo di energia Chi pari a Competenza Armi+Competenza Magica+Saggezza/2

La \textbf{seconda volta} che prendi questa Abilità, requisito Forza 1, Saggezza 2, Competenza Armi 4, Competenza Magica 4

Recuperi 1 punto Chi ogni 10 minuti in cui il personaggio non effettua attività impegnative.

\smallskip\noindent\rule{\linewidth}{2pt} \index[Abilita]{Colpo Psichico}\hypertarget{Colpo Psichico}{}\medskip\noindent{\textbf{Colpo Psichico}}\pdfbookmark[3]{Colpo Psichico}{Colpo Psichico}
\noindent
\begin{description}[noitemsep, topsep=0pt, parsep=0pt, partopsep=0pt, leftmargin=0cm, labelwidth=2.5cm]
    \item[\textbf{Requisito}]: Energia Psichica, Destrezza 1
    \item[\textbf{Tiri Salvezza}]: +2 Volontà, +1 Tempra
    \item[\textbf{Caratteristica}]: Saggezza o Forza
\end{description}

La \textbf{prima volta} che prendi questa Abilità concentri il tuo Chi nelle tue mani. Puoi concentrare un numero di punti Chi pari alla Saggezza.

Con un Attacco a Tocco andato a segno, nel round scarichi l'energia che causa 1d6 danni da forza per punto Chi usato, fino ad un massimo di punti Chi pari al punteggio di Saggezza.

Il colpo si considera come portato da un arma magica con un bonus pari ai punti Chi usati.

La \textbf{seconda volta} che prendi questa Abilità, requisito Colpo Psichico, Saggezza 3, Competenza Armi 2, se il Tiro per Colpire va a segno consumi un punto Chi in meno.

La \textbf{terza volta} che prendi questa Abilità, Competenza Armi 3, se il Tiro per Colpire va a segno consumi due punti Chi in meno. Puoi usare un numero di punti Chi contemporaneo pari ad una volta e mezza il valore della Saggezza.

La \textbf{quarta volta} che prendi questa Abilità, Competenza Armi 7, Saggezza 4, se il Tiro per Colpire va a segno consumi tre punti Chi in meno. Puoi usare un numero di punti Chi contemporaneo pari al doppio del valore della Saggezza.

\smallskip\noindent\rule{\linewidth}{2pt} \index[Abilita]{Raggio Psichico}\hypertarget{Raggio Psichico}{}\medskip\noindent{\textbf{Raggio Psichico}}\pdfbookmark[3]{Raggio Psichico}{Raggio Psichico}
\noindent
\begin{description}[noitemsep, topsep=0pt, parsep=0pt, partopsep=0pt, leftmargin=0cm, labelwidth=2.5cm]
    \item[\textbf{Requisito}]: Colpo Psichico, Saggezza 3, Competenza Armi 5
    \item[\textbf{Tiri Salvezza}]: +2 Riflessi, +1 Volontà
    \item[\textbf{Caratteristica}]: Saggezza o Destrezza
\end{description}

La \textbf{prima volta} che prendi questa Abilità puoi effettuare un attacco a distanza entro 9 metri usando l'Energia Psichica. L'Attacco, a Tocco, causa 1d6 di danno da forza per punto Psichico speso focalizzato sul danno.

E' possibile focalizzare uno o più punti Psichici per aumentare la distanza ogni volta di 9 metri. Non puoi usare un numero di punti Chi totali (per distanza e e danno) superiore alla Saggezza.

La \textbf{seconda volta} che prendi questa Abilità requisito Saggezza 3, Competenza Armi 9, puoi utilizzare fino a doppio del tuo punteggio in Saggezza per potenziare il Raggio Psichico.

\smallskip\noindent\rule{\linewidth}{2pt} \index[Abilita]{Elementalista}\hypertarget{Elementalista}{}\medskip\noindent{\textbf{Elementalista}}\pdfbookmark[3]{Elementalista}{Elementalista}
\noindent
\begin{description}[noitemsep, topsep=0pt, parsep=0pt, partopsep=0pt, leftmargin=0cm, labelwidth=2.5cm]
    \item[\textbf{Requisito}]: Competenza Magia 3, Almeno 4 incantesimi da Lista di Magia Elementale
    \item[\textbf{Tiri Salvezza}]: +1 Volontà, +1 Tempra
    \item[\textbf{Caratteristica}]: Modificatore di caratteristica per incantesimi o Costituzione
\end{description}

La \textbf{prima volta} che prendi questa Abilità scegli un tipo di Energia Elementale: Fuoco, Elettricità, Freddo, Suono.

Sei capace di scambiare gli elementi presenti nei tuoi incantesimi. Puoi sostituire un tipo di danno di energia elementale con il danno scelto quando ha preso questa Abilità.
Il tempo di lancio dell'incantesimo aumenta di 1 Azione, se il tempo totale di lancio supera le 3 Azioni non è possibile usare questa Abilità sull'incantesimo.

La \textbf{seconda volta} che prendi questa Abilità, requisito Competenza Magica 6, scegli un nuovo tipo di Energia. Quando effettui la sostituzione puoi scegliere tra i tipi di energia a disposizione. Non hai più la penalità al tempo di lancio dell'incantesimo.

\smallskip\noindent\rule{\linewidth}{2pt} \index[Abilita]{Esperto}\hypertarget{Esperto}{}\medskip\noindent{\textbf{Esperto}}\pdfbookmark[3]{Esperto}{Esperto}
\noindent
\begin{description}[noitemsep, topsep=0pt, parsep=0pt, partopsep=0pt, leftmargin=0cm, labelwidth=2.5cm]
    \item[\textbf{Requisito}]: Caratteristica collegata almeno a -1
    \item[\textbf{Tiri Salvezza}]: +1 a due Tiri Salvezza a scelta.
    \item[\textbf{Caratteristica}]: a scelta
\end{description}

Sei un esperto in un argomento.

La \textbf{prima volta} che prendi questa Abilità guadagni un +1 alle prove su una Competenza base a tua scelta.

La \textbf{seconda volta} che prendi questa Abilità aggiungi +2 alla prova. Puoi prendere 10 alla prova impiegando 5 round anziché 10 (vedi pag. \pageref{prendere10}).

La \textbf{terza volta} che prendi questa Abilità aggiungi 1d6 alla prova. Puoi prendere 14 alla prova impiegando 5 minuti anziché 10.

La \textbf{quarta volta} che prendi questa Abilità consideri il totale dei dadi tirati come 10 se ha tirato da 4-9.

Il bonus sono cumulativi se riferiti sempre la stessa Competenza.

Non è usabile su Consapevolezza (vedi \hyperlink{Percettivo}{Percettivo}, pag. \pageref{Percettivo}).

\smallskip\noindent\rule{\linewidth}{2pt} \index[Abilita]{Estrazione rapida}\hypertarget{Estrazione rapida}{}\medskip\noindent{\textbf{Estrazione rapida}}\pdfbookmark[3]{Estrazione rapida}{Estrazione rapida}
\noindent
\begin{description}[noitemsep, topsep=0pt, parsep=0pt, partopsep=0pt, leftmargin=0cm, labelwidth=2.5cm]
    \item[\textbf{Requisito}]: Competenza Armi 1
    \item[\textbf{Tiri Salvezza}]: +1 Riflessi, +1 Volontà
    \item[\textbf{Caratteristica}]: Destrezza o Intelligenza
\end{description}

La \textbf{prima volta} che prendi questa Abilità puoi estrarre un arma per te non grande con il costo di una Azione immediata.

La \textbf{seconda volta} che prendi questa Abilità puoi riporre l'arma attuale ed estrarne un'altra come Azione di Movimento.

La \textbf{terza volta} che prendi questa Abilità puoi riporre l'arma attuale ed estrarne un'altra come Azione Immediata.

\smallskip\noindent\rule{\linewidth}{2pt} \index[Abilita]{Fare Infuriare}\hypertarget{Fare Infuriare}{}\medskip\noindent{\textbf{Fare Infuriare}}\pdfbookmark[3]{Fare Infuriare}{Fare Infuriare}
\noindent
\begin{description}[noitemsep, topsep=0pt, parsep=0pt, partopsep=0pt, leftmargin=0cm, labelwidth=2.5cm]
    \item[\textbf{Requisito}]: Competenza Armi 2 e Carisma o Forza 2
    \item[\textbf{Tiri Salvezza}]: +2 Volontà, +1 Tempra
    \item[\textbf{Caratteristica}]: Forza o Carisma
\end{description}

Impieghi 2 Azioni ad infamare ed inveire contro un avversario.

Il target deve fare una prova contrapposta di Tiro Salvezza Volontà contro la tua prova di competenza Intrattenere od Intimidire oppure perdere il bonus di Destrezza (Tiri Salvezza, Tiro Colpire e Difesa) fino alla fine del tuo round successivo.

L'avversario può non comprendere la tua lingua ma deve avere Intelligenza pari a -2 o più.

\smallskip\noindent\rule{\linewidth}{2pt} \index[Abilita]{Fedele}\hypertarget{Fedele}{}\medskip\noindent{\textbf{Fedele}}\pdfbookmark[3]{Fedele}{Fedele}
\noindent
\begin{description}[noitemsep, topsep=0pt, parsep=0pt, partopsep=0pt, leftmargin=0cm, labelwidth=2.5cm]
    \item[\textbf{Requisito}]: Competenza Magica 1, Somma valore Tratti in comune 2, essere Devoto
    \item[\textbf{Tiri Salvezza}]: +2 Volontà, +1 Tempra
    \item[\textbf{Caratteristica}]: Modificatore di caratteristica per incantesimi o Saggezza
\end{description}

La tua connessione con il Patrono è forte ed energetica. Aumenti i tuoi Punti Magia di 3 punti.

L'Abilità può essere presa più volte ed il totale deve essere pari o inferiore alla somma dei Tratti comune con il Patrono/3.

Questa Abilità non si cumula con l'Abilità Batteria Magica.

\smallskip\noindent\rule{\linewidth}{2pt} \index[Abilita]{Ferocia}\hypertarget{Ferocia}{}\medskip\noindent{\textbf{Ferocia}}\pdfbookmark[3]{Ferocia}{Ferocia}
\noindent
\begin{description}[noitemsep, topsep=0pt, parsep=0pt, partopsep=0pt, leftmargin=0cm, labelwidth=2.5cm]
    \item[\textbf{Requisito}]: Competenza Armi 1
    \item[\textbf{Tiri Salvezza}]: +2 Tempra, +1 Volontà
    \item[\textbf{Caratteristica}]: Costituzione o Forza
\end{description}

La tua rabbia è tale da sconfiggere, temporaneamente, la morte.

La \textbf{prima volta} che prendi questa Abilità quando scendi sotto lo 0 Punti Ferita non svieni ed incominci a perdere 1 punto ferita a round.

Una creatura dotata di Ferocia sviene quando ha un punteggio di Punti Ferita negativo pari al doppio dei punti di Costituzione e muore quando i suoi Punti Ferita scendono al punteggio negativo pari al suo triplo del punteggio di Costituzione+5 (COS*3+5)

La \textbf{seconda volta} che prendi questa Abilità, requisito Competenza Armi 4, puoi fare che la tua Forza aumenti di 2 e acquisisci 6 Punti Ferita temporanei per 1 minuto. A fine scontro il tuo livello di affaticamento aumenta di 1 per 10 minuti.

La \textbf{terza volta} che prendi questa Abilità, requisito Competenza Armi 7, puoi fare che la tua Forza aumenti di 3 e acquisisci 12 Punti Ferita temporanei per 3 minuti. A fine scontro il tuo livello di affaticamento aumenta di 2 per 20 minuti.

La \textbf{quarta volta} che prendi questa Abilità, requisito Competenza Armi 11, puoi fare che la tua Forza aumenti di 4 e acquisisci 24 Punti Ferita temporanei per 15 minuti. A fine scontro il tuo livello di affaticamento aumenta di 3 per 30 minuti.

Il giocatore può scegliere un solo grado di Ferocia da usare nello scontro (2, 3, 4).

\smallskip\noindent\rule{\linewidth}{2pt} \index[Abilita]{Figlia di Shayalia}\hypertarget{Figlia di Shayalia}{}\medskip\noindent{\textbf{Figlia di Shayalia}}\pdfbookmark[3]{Figlia di Shayalia}{Figlia di Shayalia}
\noindent
\begin{description}[noitemsep, topsep=0pt, parsep=0pt, partopsep=0pt, leftmargin=0cm, labelwidth=2.5cm]
    \item[\textbf{Requisito}]: Devoto o Seguace di Shayalia
    \item[\textbf{Tiri Salvezza}]: +1 Tempra, +2 Volontà
    \item[\textbf{Caratteristica}]: Modificatore di caratteristica per incantesimi o Carisma
\end{description}

Hai una profonda ed istintiva connessione con il mondo naturale.

La \textbf{prima volta} che prendi questa Abilità ottieni un +2 alla prove di Natura ed un +2 ai Tiri Salvezza contro veleni naturali.

La \textbf{seconda volta} che prendi questa Abilità, requisito somma Tratti in comune 6, ottieni un +4 alla prove di Natura ed un +4 ai Tiri Salvezza contro effetti, anche magici, causati da Animali o Piante.

La \textbf{terza volta} che prendi questa Abilità, requisito somma Tratti in comune 12, sei sempre sotto l'effetto dell'incantesimo Santuario verso qualsiasi animale non magico.

La \textbf{quarta volta} che prendi questa Abilità, requisito Animalia preso 4 volte, puoi trasformati usando Animalia, in qualsiasi creatura purché non sia un immondo o drago.

\smallskip\noindent\rule{\linewidth}{2pt} \index[Abilita]{Figlio di Tazher}\hypertarget{Figlio di Tazher}{}\medskip\noindent{\textbf{Figlio di Tazher}}\pdfbookmark[3]{Figlio di Tazher}{Figlio di Tazher}
\noindent
\begin{description}[noitemsep, topsep=0pt, parsep=0pt, partopsep=0pt, leftmargin=0cm, labelwidth=2.5cm]
    \item[\textbf{Requisito}]: Devoto o Seguace di Tazher, somma tratti comuni 10
    \item[\textbf{Tiri Salvezza}]: +2 Riflessi, +1 Volontà
    \item[\textbf{Caratteristica}]: Destrezza o Saggezza
\end{description}

Quando prendi questa Abilità la tua ombra diventa l'equivalente di un \hyperlink{Servitore Invisibile}{Servitore Invisibile}.
L'ombra ha Difesa pari a 10 + valore Tratto in comune con Tazher a più alto punteggio, i Tiri Salvezza sono pari ai tuoi, i Punti Ferita sono pari alla somma dei tratti in comune con Tazher.

Usando 2 Azioni puoi evocarla nuovamente se dovesse essere uccisa.

La tua ombra non può allontanarsi da te più della somma dei tratti in comune con Tazher in metri. Può essere usata per \hyperlink{Condividere Incantesimi}{Condividere Incantesimi}, \hyperlink{Trasmettere Incantesimi a Contatto}{Trasmettere Incantesimi a Contatto} (ma usi la tua capacità di attacco), vedi anche \hyperlink{famiglio}{Famiglio} (pag. \pageref{famiglio}).

Può essere usato come punto di lancio degli incantesimi.

Non puoi avere Famigli. Non puoi interagire con la tua Ombra se non ci sono le condizione per esserci un ombra.

\smallskip\noindent\rule{\linewidth}{2pt} \index[Abilita]{Figlio Unico}\hypertarget{Figlio Unico}{}\medskip\noindent{\textbf{Figlio Unico}}\pdfbookmark[3]{Figlio Unico}{Figlio Unico}
\noindent
\begin{description}[noitemsep, topsep=0pt, parsep=0pt, partopsep=0pt, leftmargin=0cm, labelwidth=2.5cm]
    \item[\textbf{Requisito}]: Costituzione 0
    \item[\textbf{Tiri Salvezza}]: +1 Tempra, +1 Volontà
    \item[\textbf{Caratteristica}]: Modificatore di caratteristica per incantesimi o Saggezza
\end{description}

La \textbf{prima volta} che prendi questa Abilità scegli una Lista Magica e due Trucchetti da questa lista. Puoi lanciare questi due Trucchetti senza Prova di Magia anche se sei essere Distratto o con armatura.

Le volte successive che prendi questa Abilità puoi individuare un Patrono che abbia nelle Liste Privilegiate la Lista di Magia che hai individuato precedentemente.

La \textbf{seconda volta} che prendi questa Abilità individua altri due Trucchetti, o se sei un Seguace o Devoto scegli un incantesimo di primo livello dalla Lista Privilegiata del Patrono. Requisito Tratto in comune a punteggio 3.

La \textbf{terza volta} che prendi questa Abilità individua altri due Trucchetti, o se sei un Devoto scegli un incantesimo di secondo livello od inferiore dalla Lista Privilegiata del Patrono. Requisito Tratto in comune a punteggio 5.

La \textbf{quarta volta} che prendi questa Abilità individua altri due Trucchetti, o se sei un Devoto scegli un incantesimo di terzo livello od inferiore dalla Lista Privilegiata del Patrono. Requisito Tratto in comune a punteggio 9.

Le Abilità 2, 3, 4 possono essere prese più volte. Gli incantesimi di primo livello e successivo si lanciano pagando Punti Magia altrimenti si possono lanciare solo una volta al giorno per volta che si è preso l'Abilità 4.

\smallskip\noindent\rule{\linewidth}{2pt} \index[Abilita]{Finta Morte}\hypertarget{Finta Morte}{}\medskip\noindent{\textbf{Finta Morte}}\pdfbookmark[3]{Finta Morte}{Finta Morte}
\noindent
\begin{description}[noitemsep, topsep=0pt, parsep=0pt, partopsep=0pt, leftmargin=0cm, labelwidth=2.5cm]
    \item[\textbf{Requisito}]: Costituzione 0
    \item[\textbf{Tiri Salvezza}]: +1 Volontà, +2 Tempra
    \item[\textbf{Caratteristica}]: Costituzione o Saggezza
\end{description}

Come Azione di Reazione sei in grado di cadere a terra (stramazzare!) morto. Solo una prova di Pronto Soccorso DC 20 può rivelare che sei vivo.

L'effetto dura al massimo 2 minuti. La finta morte non è ripetibile in intervalli inferiori ai 10 minuti l'una dall'altra.

\smallskip\noindent\rule{\linewidth}{2pt} \index[Abilita]{Flagello Danzante}\hypertarget{Flagello Danzante}{}\medskip\noindent{\textbf{Flagello Danzante}}\pdfbookmark[3]{Flagello Danzante}{Flagello Danzante}
\noindent
\begin{description}[noitemsep, topsep=0pt, parsep=0pt, partopsep=0pt, leftmargin=0cm, labelwidth=2.5cm]
    \item[\textbf{Requisito}]: Competenza Armi 1, usare un arma della Lista Palle rotanti
    \item[\textbf{Tiri Salvezza}]: +1 Tempra, +1 Volontà
    \item[\textbf{Caratteristica}]: Forza o Carisma
\end{description}

Quando usi la tua arma della lista Palle Rotanti hai un bonus di +1 al Tiro per Colpire e +1 alla Difesa

\smallskip\noindent\rule{\linewidth}{2pt} \index[Abilita]{Forgiato nella furia}\hypertarget{Forgiato nella furia}{}\medskip\noindent{\textbf{Forgiato nella furia}}\pdfbookmark[3]{Forgiato nella furia}{Forgiato nella furia}
\noindent
\begin{description}[noitemsep, topsep=0pt, parsep=0pt, partopsep=0pt, leftmargin=0cm, labelwidth=2.5cm]
    \item[\textbf{Requisito}]: Competenza Armi 5
    \item[\textbf{Tiri Salvezza}]: +1 Tempra, +1 Riflessi
    \item[\textbf{Caratteristica}]: Forza o Costituzione
\end{description}

Quando effettui un tiro critico con un attacco in mischia considera di aver colpito con un margine ulteriore di +2, per la verifica di ulteriori critici.

\smallskip\noindent\rule{\linewidth}{2pt} \index[Abilita]{Fortunato}\hypertarget{Fortunato}{}\medskip\noindent{\textbf{Fortunato}}\pdfbookmark[3]{Fortunato}{Fortunato}
\noindent
\begin{description}[noitemsep, topsep=0pt, parsep=0pt, partopsep=0pt, leftmargin=0cm, labelwidth=2.5cm]
    \item[\textbf{Requisito}]: nessuno
    \item[\textbf{Tiri Salvezza}]: +1 Tempra, +1 Riflessi
    \item[\textbf{Caratteristica}]: Carisma o Destrezza
\end{description}

Una volta al giorno puoi fare ritirare 1d6 di una prova (Tiri per Colpire, Prove Competenze, Tiri Salvezza) al Narratore e prendere il valore più basso tra i due tiri.

L'Abilità può essere dichiarata anche dopo il tiro dei dadi.

\smallskip\noindent\rule{\linewidth}{2pt} \index[Abilita]{Forma Elementale}\hypertarget{Forma Elementale}{}\medskip\noindent{\textbf{Forma Elementale}}\pdfbookmark[3]{Forma Elementale}{Forma Elementale}
\noindent
\begin{description}[noitemsep, topsep=0pt, parsep=0pt, partopsep=0pt, leftmargin=0cm, labelwidth=2.5cm]
    \item[\textbf{Requisito}]: Seguace o Devoto di Erondil, Gaya, Efrem oppure Shayalia. Almeno 3 incantesimi da 2 Liste di Magia Elementare diverse, Competenza Magica 6
    \item[\textbf{Tiri Salvezza}]: +1 Tempra, +1 Volontà
    \item[\textbf{Caratteristica}]: Modificatore di caratteristica per incantesimi o Costituzione
\end{description}

Quando ti trasformi con l'Abilità Animalia scegli un elemento tra gli incantesimi imparati presenti in una Lista di Magia Elementale.

La \textbf{prima volta} che prendi questa Abilità quando ti trasformi con Animalia i tuoi attacchi in mischia fanno il tipo di danno elementale scelto.

La \textbf{seconda volta} che prendi questa Abilità, Competenza Magica 11, quando ti trasformi con Animalia sei resistente al medesimo tipo di danno elementale che causi.

La \textbf{terza volta} che prendi questa Abilità, requisito Competenza Magica 14, i tuoi attacchi quando ti trasformi con Animalia causano 2d6 di danno in più del tipo elementale scelto.

Se sei un Devoto o Seguace di Gaya o Erondil non è necessario trasformarsi in animale, il tipo di danno si applica ai tuoi attacchi in mischia.

\smallskip\noindent\rule{\linewidth}{2pt} \index[Abilita]{Freccia chiamata, freccia consegnata}\hypertarget{Freccia chiamata, freccia consegnata}{}\medskip\noindent{\textbf{Freccia chiamata, freccia consegnata}}\pdfbookmark[3]{Freccia chiamata, freccia consegnata}{Freccia chiamata, freccia consegnata}
\noindent
\begin{description}[noitemsep, topsep=0pt, parsep=0pt, partopsep=0pt, leftmargin=0cm, labelwidth=2.5cm]
    \item[\textbf{Requisito}]: Competenza Armi 2
    \item[\textbf{Tiri Salvezza}]: +2 Riflessi
    \item[\textbf{Caratteristica}]: Destrezza o Intelligenza
\end{description}

E' questione di un attimo perché ti accenda!

Puoi tirare 1 freccia/dardo, una volta al giorno, come Reazione, senza penalità al colpire date dal multiattacco.

L'arco/balestra deve già essere in mano.

\smallskip\noindent\rule{\linewidth}{2pt} \index[Abilita]{Furia}\hypertarget{Furia}{}\medskip\noindent{\textbf{Furia}}\pdfbookmark[3]{Furia}{Furia}
\noindent
\begin{description}[noitemsep, topsep=0pt, parsep=0pt, partopsep=0pt, leftmargin=0cm, labelwidth=2.5cm]
    \item[\textbf{Requisito}]: Competenza Armi 1
    \item[\textbf{Tiri Salvezza}]: +2 Tempra, +1 Volontà
    \item[\textbf{Caratteristica}]: Forza o Costituzione
\end{description}

Il tuo stile di combattimento è rappresentato dalla cieca furia omicida.

Aggiungi +1d6 al danno ad ogni attacco andato a segno in mischia ed i tuoi avversari guadagnano +1d6 al colpire verso di te.

Puoi decidere di attivare questa Abilità round per round. Costa 1 Azione Immediata e dura fino all'inizio del tuo round successivo.

\begin{center}
	\includegraphics[width=0.9\linewidth]{immagini/Early_Egyptian_juggling_art.png}

	\emph{This ancient wall painting appears to depict jugglers.}
\end{center}

\smallskip\noindent\rule{\linewidth}{2pt} \index[Abilita]{Giocoliere}\hypertarget{Giocoliere}{}\medskip\noindent{\textbf{Giocoliere}}\pdfbookmark[3]{Giocoliere}{Giocoliere}
\noindent
\begin{description}[noitemsep, topsep=0pt, parsep=0pt, partopsep=0pt, leftmargin=0cm, labelwidth=2.5cm]
    \item[\textbf{Requisito}]: Destrezza 2
    \item[\textbf{Tiri Salvezza}]: +2 Riflessi
    \item[\textbf{Caratteristica}]: Destrezza o Carisma
\end{description}

Hai un talento naturale per maneggiare gli oggetti.

Qualsiasi prova di Acrobatica che coinvolga il maneggiare oggetti o l'equilibrio ha un +2 di Bonus.

Puoi lanciare un secondo pugnale come Azione Immediata in seguito all'Azione di attacco di lancio di un pugnale, questo pugnale ha un -3 al Tiro per Colpire e non cumula penalità al multiattacco.

\smallskip\noindent\rule{\linewidth}{2pt} \index[Abilita]{Guerriero della Magia}\hypertarget{Guerriero della Magia}{}\medskip\noindent{\textbf{Guerriero della Magia}}\pdfbookmark[3]{Guerriero della Magia}{Guerriero della Magia}
\noindent
\begin{description}[noitemsep, topsep=0pt, parsep=0pt, partopsep=0pt, leftmargin=0cm, labelwidth=2.5cm]
    \item[\textbf{Requisito}]: Competenza Armi 2, Competenza Magica 2
    \item[\textbf{Tiri Salvezza}]: +1 Volontà, +1 Riflessi
    \item[\textbf{Caratteristica}]: Modificatore di caratteristica per incantesimi o Forza
\end{description}

Non segui solo la via della magie e neanche quella della spada, il tuo stile abbraccia entrambi in un fendente di pura magia.

La \textbf{prima volta} che prendi questa Abilità sei in grado di scaricare un incantesimo con distanza di mischia con la tua arma. Effettui il Tiro per Colpire e se colpisci oltre al danno dell'attacco scarichi anche l'incantesimo. Devi riuscire in una Prova di Magia. In questa maniera puoi eseguire un solo attacco con l'arma. Costa 3 Azioni.

La \textbf{seconda volta} che prendi questa Abilità, requisito Competenza Armi 3, Competenza Magica 3, consumando 3 Azioni sei in grado di scaricare un incantesimo che non sia personale o a tocco con un arma a gittata. Devi riuscire in una Prova di Magia.

La \textbf{terza volta} che prendi questa Abilità, requisito Competenza Armi 6, Competenza Magica 9, puoi convogliare incantesimi fino al 3 livello tramite l'arma.

La \textbf{quarta volta} che prendi questa Abilità non è necessario più effettuare la Prova di Magia per scaricare l'incantesimo con l'arma.

Non puoi scaricare incantesimi di livello superiore a 3 con questa Abilità ed il tempo di lancio dell'incantesimo non può essere superiore alle 2 Azioni.

\smallskip\noindent\rule{\linewidth}{2pt} \index[Abilita]{Gru d'Argento}\hypertarget{Gru d'Argento}{}\medskip\noindent{\textbf{Gru d'Argento}}\pdfbookmark[3]{Gru d'Argento}{Gru d'Argento}
\noindent
\begin{description}[noitemsep, topsep=0pt, parsep=0pt, partopsep=0pt, leftmargin=0cm, labelwidth=2.5cm]
    \item[\textbf{Requisito}]: Lista Pugno Vuoto 2, Destrezza 1
    \item[\textbf{Tiri Salvezza}]: +2 Riflessi, +1 Volontà
    \item[\textbf{Caratteristica}]: Destrezza o Intelligenza
\end{description}

Per usufruire di questi bonus non devi portare armature o scudi od oggetti anche magici che migliorino la Difesa. Le capacità elencate non sono cumulabili con l'Abilità Armatura della Montagna Incantata.

La \textbf{prima volta} che prendi questa Abilità la tua Difesa naturale aumenta di 1 + 1/3 dei punti in Pugno Vuoto + Destrezza + eventuali modificatori.

La \textbf{seconda volta} che prendi questa Abilità, requisito Lista Pugno Vuoto 4, la tua Iniziativa aumenta di 2 (solo con attacchi disarmati).

La \textbf{terza volta} che prendi questa Abilità, requisito Lista Pugno Vuoto 9 e Destrezza 2, hai un bonus nei Tiri salvezza su Volontà 2 (cumulativo).

La \textbf{quarta volta} che prendi questa Abilità, requisito Lista Pugno Vuoto 11, la tua Difesa naturale ed Iniziativa aumentano di 2 (cumulativo).

La \textbf{quinta volta} che prendi questa Abilità, requisito Lista Pugno Vuoto 13 e Destrezza 3, hai un bonus nei Tiri salvezza su Riflessi di 2 (cumulativo).

I bonus sono attivi anche se non stai combattendo.

\smallskip\noindent\rule{\linewidth}{2pt} \index[Abilita]{Guarigione accelerata}\hypertarget{Guarigione accelerata}{}\medskip\noindent{\textbf{Guarigione accelerata}}\pdfbookmark[3]{Guarigione accelerata}{Guarigione accelerata}
\noindent
\begin{description}[noitemsep, topsep=0pt, parsep=0pt, partopsep=0pt, leftmargin=0cm, labelwidth=2.5cm]
    \item[\textbf{Requisito}]: Costituzione 0
    \item[\textbf{Tiri Salvezza}]: +2 Tempra, +1 Volontà
    \item[\textbf{Caratteristica}]: Costituzione
\end{description}

I tuoi naturali processi curativi sono estremamente efficienti. Le capacità si cumulano.

La \textbf{prima volta} che prendi questa Abilità dopo una notte di riposo recuperi 1d6 Punti Ferita in più.

La \textbf{seconda volta}, requisito Costituzione 1,  che prendi questa Abilità alla fine di ogni tuo round diminuisci di 1 la condizione di Sanguinamento.

La \textbf{terza volta}, requisito Costituzione 2, che prendi questa Abilità dopo una notte di riposo recuperi il doppio dei Punti Ferita.

\smallskip\noindent\rule{\linewidth}{2pt} \index[Abilita]{Guaritore}\hypertarget{Guaritore}{}\medskip\noindent{\textbf{Guaritore}}\pdfbookmark[3]{Guaritore}{Guaritore}
\noindent
\begin{description}[noitemsep, topsep=0pt, parsep=0pt, partopsep=0pt, leftmargin=0cm, labelwidth=2.5cm]
    \item[\textbf{Requisito}]: Saggezza 1
    \item[\textbf{Tiri Salvezza}]: +2 Volontà, +1 Tempra
    \item[\textbf{Caratteristica}]: Saggezza
\end{description}

Hai un naturale talento nel curare le persone.

La \textbf{prima volta} che prendi questa Abilità hai +4 alle prove di Pronto soccorso.

La \textbf{seconda volta} che prendi questa Abilità, requisito somma tratti comuni con il Patrono Ledyal 8, ogni volta che usi un incantesimo di cura tu recuperi 1 Punte Ferita e la creatura curata +1d6 Punti Ferita in più.

\smallskip\noindent\rule{\linewidth}{2pt} \index[Abilita]{Ho detto CADI!}\hypertarget{Ho detto CADI!}{}\medskip\noindent{\textbf{Ho detto CADI!}}\pdfbookmark[3]{Ho detto CADI!}{Ho detto CADI!}
\noindent
\begin{description}[noitemsep, topsep=0pt, parsep=0pt, partopsep=0pt, leftmargin=0cm, labelwidth=2.5cm]
    \item[\textbf{Requisito}]: Competenza Armi 4
    \item[\textbf{Tiri Salvezza}]: +2 Tempra, +1 Volontà
    \item[\textbf{Caratteristica}]: Forza o Costituzione
\end{description}

Se colpisci 3 volte entro 2 round un avversario questo deve fare una Tiro Salvezza su Tempra con DC pari al Tiro per Colpire dell'ultimo attacco o cadere prono. Il Tiro Salvezza ha 1d6 di modificatore per taglia di differenza.

\smallskip\noindent\rule{\linewidth}{2pt} \index[Abilita]{Il Patrono è con me}\hypertarget{Il Patrono è con me}{}\medskip\noindent{\textbf{Il Patrono è con me}}\pdfbookmark[3]{Il Patrono è con me}{Il Patrono è con me}
\noindent
\begin{description}[noitemsep, topsep=0pt, parsep=0pt, partopsep=0pt, leftmargin=0cm, labelwidth=2.5cm]
    \item[\textbf{Requisito}]: Devoto, Somma Tratti comuni con il Patrono 2
    \item[\textbf{Tiri Salvezza}]: +1 Volontà, +1 Riflessi
    \item[\textbf{Caratteristica}]: Modificatore di caratteristica per incantesimi o a scelta
\end{description}

La \textbf{prima volta} che prendi questa Abilità per 1 volta al giorno puoi ritirare un dado tirato nella Prova di Magia.

La \textbf{seconda volta} che prendi questa Abilità, requisito somma Tratti comuni con il Patrono 6, per 2 volte al giorno puoi ritirare fino a 2 dadi tirati nella Prova di Magia per il lancio di incantesimo.

La \textbf{terza volta} che prendi questa Abilità, requisito somma Tratti comuni con il Patrono 12, per 3 volte al giorno puoi ritirare fino a 3 dadi tirati nella Prova di Magia per il lancio di incantesimo.

L'Abilità può essere dichiarata anche dopo il lancio dei dadi. Qualsiasi nuovo valore ottenuto con il nuovo tiro va tenuto o si usa nuovamente questa Abilità.

\smallskip\noindent\rule{\linewidth}{2pt} \index[Abilita]{Il Patrono è la mia Arma}\hypertarget{Il Patrono è la mia Arma}{}\medskip\noindent{\textbf{Il Patrono è la mia Arma}}\pdfbookmark[3]{Il Patrono è la mia Arma}{Il Patrono è la mia Arma}\label{Il Patrono è la mia Arma}
\noindent
\begin{description}[noitemsep, topsep=0pt, parsep=0pt, partopsep=0pt, leftmargin=0cm, labelwidth=2.5cm]
    \item[\textbf{Requisito}]: Somma Tratti comuni con il Patrono 1, essere Seguaci
    \item[\textbf{Tiri Salvezza}]: +1 Volontà, +1 Riflessi
    \item[\textbf{Caratteristica}]: Modificatore di caratteristica per incantesimi o a scelta
\end{description}

Non hai penalità al colpire con l'arma del Patrono se non sei competente nella sua Lista d'Armi.

La \textbf{prima volta} che prendi questa Abilità hai un +1 di al Tiro per Colpire ed al Danno quando usi l'Arma preferita del tuo Patrono.

La \textbf{seconda volta} che prendi questa Abilità, requisito somma Tratti comune 5, Competenza Armi 1, la penalità per gli attacchi multipli con l'arma preferita del Patrono diviene -4.

\begin{center}
	\includegraphics[width=0.56\linewidth]{immagini/streghegoya.png}

	\emph{Sabba delle Streghe (Goya, 1798)}
\end{center}

La \textbf{terza volta} che prendi questa Abilità, requisito somma Tratti comuni con il Patrono 10, Competenza Armi 4, essere Devoti, puoi usare il tuo Modificatore di caratteristica per incantesimi per stabilire il Tiro per Colpire con l'arma del Patrono, al posto di Forza o Destrezza.

La \textbf{quarta volta} che prendi questa Abilità, requisito somma Tratti comuni con il Patrono 10, Competenza Armi 5, aggiungi +1d6 al Tiro per Colpire quando effettui il terzo attacco nel round con l'arma del Patrono.

La \textbf{quinta volta} che prendi questa Abilità, requisito somma Tratti comuni con il Patrono 13, Competenza Armi 6, aumenti di un grado il dado di danno della tua arma del Patrono.

La \textbf{sesta volta} che prendi questa Abilità, requisito somma Tratti comuni con il Patrono 16, ottieni un ulteriore +1 al Tiro per Colpire e +1 al Danno. Il primo attacco andato a segno nel round con l'arma del Patrono causa sempre danno critico.

\smallskip\noindent\rule{\linewidth}{2pt} \index[Abilita]{Iaijutsu}\hypertarget{Iaijutsu}{}\medskip\noindent{\textbf{Iaijutsu}}\pdfbookmark[3]{Iaijutsu}{Iaijutsu}
\noindent
\begin{description}[noitemsep, topsep=0pt, parsep=0pt, partopsep=0pt, leftmargin=0cm, labelwidth=2.5cm]
    \item[\textbf{Requisito}]: Competenza Armi 2
    \item[\textbf{Tiri Salvezza}]: +2 Riflessi, +1 Volontà
    \item[\textbf{Caratteristica}]: Intelligenza o Destrezza
\end{description}

La \textbf{prima volta} che prendi questa Abilità fai un passo di un metro, attacchi una volta e torni dove eri prima, il tutto in meno di un battito di ciglia.

La \textbf{seconda volta}, requisito CA 6, che prendi questa Abilità puoi muoverti di 2 metri prima di attaccare e tornare dove eri.

La \textbf{terza volta}, requisito CA 12, che prendi questa Abilità puoi muoverti di il tuo movimento prima di attaccare e tornare dove eri, tratti il terreno come difficile.

Consumi due Azioni.

\smallskip\noindent\rule{\linewidth}{2pt} \index[Abilita]{Improvvisare}\hypertarget{Improvvisare}{}\medskip\noindent{\textbf{Improvvisare}}\pdfbookmark[3]{Improvvisare}{Improvvisare}
\noindent
\begin{description}[noitemsep, topsep=0pt, parsep=0pt, partopsep=0pt, leftmargin=0cm, labelwidth=2.5cm]
    \item[\textbf{Requisito}]: Competenza Armi 1
    \item[\textbf{Tiri Salvezza}]: +1 Tempra, +1 Riflessi
    \item[\textbf{Caratteristica}]: Destrezza o Saggezza
\end{description}

Qualsiasi oggetto che possa definirsi un arma improvvisata per te non è improvvisata.
Non soffri di penalità al colpire quando usi un \hyperlink{armaimprovvisata}{arma improvvisata}. Ottieni un +1 al danno quando usi un arma improvvisata.

\smallskip\noindent\rule{\linewidth}{2pt} \index[Abilita]{Incantatore da Combattimento}\hypertarget{Incantatore da Combattimento}{}\medskip\noindent{\textbf{Incantatore da Combattimento}}\pdfbookmark[3]{Incantatore da Combattimento}{Incantatore da Combattimento}
\noindent
\begin{description}[noitemsep, topsep=0pt, parsep=0pt, partopsep=0pt, leftmargin=0cm, labelwidth=2.5cm]
    \item[\textbf{Requisito}]: Competenza Magica 1
    \item[\textbf{Tiri Salvezza}]: +1 Tempra, +1 Volontà
    \item[\textbf{Caratteristica}]: Modificatore di caratteristica per incantesimi o a scelta
\end{description}

La \textbf{prima volta} che prendi questa Abilità quando sei Distratto puoi ignorare un dado nella Prova di Magia.

La \textbf{seconda volta}, requisito Competenza Magica 6, che prendi questa Abilità, quando sei Distratto, puoi ignorare un ulteriore dado nella Prova di Magia.

La \textbf{terza volta}, requisito Competenza Magica 12, che prendi questa Abilità, quando sei Distratto, puoi ignorare un ulteriore dado nella Prova di Magia.

Questa Abilità si può usare nella Prova di Magia richiesta da \hyperlink{guerrierodellamagia}{Guerriero della Magia}. Le capacità indicate si cumulano.

\smallskip\noindent\rule{\linewidth}{2pt} \index[Abilita]{Incantatore Prudente}\hypertarget{Incantatore Prudente}{}\medskip\noindent{\textbf{Incantatore Prudente}}\pdfbookmark[3]{Incantatore Prudente}{Incantatore Prudente}
\noindent
\begin{description}[noitemsep, topsep=0pt, parsep=0pt, partopsep=0pt, leftmargin=0cm, labelwidth=2.5cm]
    \item[\textbf{Requisito}]: Competenza Magica 8
    \item[\textbf{Tiri Salvezza}]: +2 Riflessi, +1 Tempra
    \item[\textbf{Caratteristica}]: Modificatore di caratteristica per incantesimi o a scelta
\end{description}

Quando una creatura ostile entra per la prima volta in uno spazio entro 1 metro da te puoi usare una Reazione per lanciare un trucchetto, entro 2 Azioni, senza potenziamenti o Prova di Magia.

Questa Abilità non influisce su fatto che si è comunque Distratti nel lancio di un successivo incantesimo.

\smallskip\noindent\rule{\linewidth}{2pt} \index[Abilita]{Immunita' ai veleni}\hypertarget{Immunita' ai veleni}{}\medskip\noindent{\textbf{Immunita' ai veleni}}\pdfbookmark[3]{Immunita' ai veleni}{Immunita' ai veleni}
\noindent
\begin{description}[noitemsep, topsep=0pt, parsep=0pt, partopsep=0pt, leftmargin=0cm, labelwidth=2.5cm]
    \item[\textbf{Requisito}]: Costituzione 1
    \item[\textbf{Tiri Salvezza}]: +2 Tempra, +1 Volontà
    \item[\textbf{Caratteristica}]: Costituzione o Saggezza
\end{description}

\index{Mitridatismo}

La \textbf{prima volta} che prendi questa Abilità il corpo si abitua ai veleni, il personaggio guadagna un +2 Tiro Salvezza sui veleni.

La \textbf{seconda volta} che prendi l'Abilità divieni immune ai veleni naturali. Non riesci più ad ubriacarti normalmente.

La \textbf{terza volta} hai un +1d6 ai Tiro Salvezza ai veleni magici e subire gli effetti di fumi tossici (ma puoi sempre soffocare).

\medskip

\begin{center}
	\includegraphics[width=0.75\linewidth]{immagini/Portrait_of_V_Greatrakesv2.png}

	\emph{Portrait of V. Greatrakes laying on his hands, window, in right-hand corner showing several successful cures, possibly. By W. Faithorne }
\end{center}

\smallskip\noindent\rule{\linewidth}{2pt} \index[Abilita]{Imposizione delle mani}\hypertarget{Imposizione delle mani}{}\medskip\noindent{\textbf{Imposizione delle mani}}\pdfbookmark[3]{Imposizione delle mani}{Imposizione delle mani}
\noindent
\begin{description}[noitemsep, topsep=0pt, parsep=0pt, partopsep=0pt, leftmargin=0cm, labelwidth=2.5cm]
    \item[\textbf{Requisito}]: Competenza Magica 1, Tratti comuni 3, essere Devoto o Seguace
    \item[\textbf{Tiri Salvezza}]: +2 Volontà, +1 Tempra
    \item[\textbf{Caratteristica}]: Carisma o Saggezza
\end{description}

Se i tuoi Tratti sono in comune con un Patrono positivo puoi convogliare energia curativa (effetto curativo/dannoso su non morti), se sono in comune con un Patrono neutrale o malvagio puoi convogliare energia negativa (effetto dannoso/curativo sui non morti).

Usabile un numero di volte al giorno pari alla (somma dei Tratti in comune con il Patrono)/2.

La \textbf{prima volta} che prendi questa Abilità attraverso l'imposizione delle mani puoi curare/ferire 5 Punti Ferita ad una creatura. Puoi applicare più usi con il singolo tocco.

L'Abilità presa più volte permette di togliere specifiche condizioni che affliggono la creatura facendo consumare più usi.

La \textbf{seconda volta} che prendi questa Abilità, somma Tratti in comune 4, puoi togliere la condizione Affaticato (3 usi).

La \textbf{terza volta} che prendi questa Abilità, somma Tratti in comune 6, puoi togliere le condizioni: Spaventato, Nauseato, Svenuto, Affaticato 2 (3 usi).

La \textbf{quarta volta} che prendi questa Abilità, somma Tratti in comune 8, puoi togliere le condizioni: Accecato, Assordato, Avvelenato, Malato* (3 usi).

La \textbf{quinta volta} che prendi questa Abilità, somma Tratti in comune 10, puoi togliere le condizioni: Confuso, Paralizzato, Affascinato (4 usi).

La \textbf{sesta volta} che prendi questa Abilità, somma Tratti in comune 12, puoi togliere le condizioni: Dominato, Pietrificato, Affaticato 3 (5 usi).

\textbf{Altri usi}:

\smallskip

Ogni uso dell'Abilità Imposizione delle mani per curare riduce il valore di Sanguinamento di 2 e permette di recuperare 3 Punti Ferita Massimi.

In caso di malattie magiche una eventuale prova di contrasto è fatta con 3d6 + somma punti Tratto in comune + Saggezza.

L'energia proviene dalle mani (non conta se ci sono guanti) e si applica con un Attacco a Tocco.

Tiro Salvezza su Tempra DC 10 + somma Tratti in comune con il Patrono + Saggezza per evitare l'effetto. 2 Azioni.

\smallskip\noindent\rule{\linewidth}{2pt} \index[Abilita]{Incanalare Energia}\hypertarget{Incanalare Energia}{}\medskip\noindent{\textbf{Incanalare Energia}}\pdfbookmark[3]{Incanalare Energia}{Incanalare Energia}
\noindent
\begin{description}[noitemsep, topsep=0pt, parsep=0pt, partopsep=0pt, leftmargin=0cm, labelwidth=2.5cm]
    \item[\textbf{Requisito}]: Imposizione delle mani, Competenza Magica 1, Tratti comuni 4
    \item[\textbf{Tiri Salvezza}]: +2 Volontà, +1 Tempra
    \item[\textbf{Caratteristica}]: Saggezza o Costituzione
\end{description}

Sei in grado di usare l'energia di Imposizione delle Mani per creare un'aura energetica intorno a te.
Tramite l'Imposizione delle mani crei un'aura istantanea nel raggio di 3 metri attorno a te che cura o ferisce 5 Punti Ferita a tutte le creature presenti ogni 2 usi consumati.

Ogni volta che prendi questa Abilità, oltre la prima, aumenti il raggio di 1 metro e puoi escludere una creatura dall'effetto dell'aura.

L'energia proviene dal tuo corpo ed influenza te stesso e le creature intorno a te. Tiro Salvezza su Riflessi DC 10 + somma Tratti in comune con il Patrono + Saggezza per evitare l'effetto. \index{Incanalare energia su non morti} 2 Azioni.

%\begin{center}
%\includegraphics[width=0.8\linewidth]{immagini/kameame.png}
%
%\emph{Kamehameha!}
%\end{center}

\smallskip\noindent\rule{\linewidth}{2pt} \index[Abilita]{Infondere Coraggio}\hypertarget{Infondere Coraggio}{}\medskip\noindent{\textbf{Infondere Coraggio}}\pdfbookmark[3]{Infondere Coraggio}{Infondere Coraggio}
\noindent
\begin{description}[noitemsep, topsep=0pt, parsep=0pt, partopsep=0pt, leftmargin=0cm, labelwidth=2.5cm]
    \item[\textbf{Requisito}]: Carisma 2, Intrattenere 1
    \item[\textbf{Tiri Salvezza}]: +2 Volontà, +1 Tempra
    \item[\textbf{Caratteristica}]: Carisma o Saggezza
\end{description}

Tramite la tua esibizione, canora, di balletto, oratoria od artistica in generale, sei in grado di infondere coraggio nei compagni in grado di sentirti o vederti, nel raggio di 6 metri.

La \textbf{prima volta} che prendi questa Abilità i tuoi compagni hanno un bonus di +1 al Tiro per Colpire ed al Danno in combattimento.

La \textbf{seconda volta} che prendi questa Abilità, requisito Intrattenere 4, puoi decidere di infondere fino a 2 di questi bonus. +2 Tiro per Colpire, +2 Difesa, +2 Danno, +2 Tiro Salvezza su Volontà. I tuoi compagni devono essere entro 12 metri di raggio.

La \textbf{terza volta} che prendi questa Abilità, requisito Intrattenere 12, puoi decidere di infondere fino a 2 di questi bonus. +1d6 Tiro per Colpire, +4 Difesa, +4 Danno, +1d6 TS. I tuoi compagni devono essere entro 24 metri di raggio.

Attivare, mantenere o cambiare effetto dell'Abilità richiede 2 Azioni.

Puoi mantenere l'Abilità un numero di round, anche non consecutivi, pari a punteggio Intrattenere x 3 al giorno.

Le creature per rimanere influenzate devono continuare a vedere/sentire la tua esibizione.

\smallskip\noindent\rule{\linewidth}{2pt} \index[Abilita]{Infondere Energia Magica}\hypertarget{Infondere Energia Magica}{}\medskip\noindent{\textbf{Infondere Energia Magica}}\pdfbookmark[3]{Infondere Energia Magica}{Infondere Energia Magica}
\noindent
\begin{description}[noitemsep, topsep=0pt, parsep=0pt, partopsep=0pt, leftmargin=0cm, labelwidth=2.5cm]
    \item[\textbf{Requisito}]: Competenza Armi 1, Competenza Magica 2
    \item[\textbf{Tiri Salvezza}]: +1 Riflessi, +1 Tempra
    \item[\textbf{Caratteristica}]: Modificatore di caratteristica per incantesimi o a scelta
\end{description}

Sai manipolare le energie magiche in maniera istintiva ed infonderle nelle armi. Costa 1 Azione infondere la magia nell'arma.

La \textbf{prima volta} che prendi questa Abilità puoi usare due Punti Magia e canalizzarli nella tua arma.

Per la durata di 6 round la tua arma diviene un arma magica +1, se possiede già capacità magiche l'effetto non funziona.

La \textbf{seconda volta} che prendi questa Abilità, requisito Competenza Magica 4, puoi usare quattro Punti Magia ed un arma con cui vieni a contatto diventa un arma +2 per 6 round, se è già incantata acquisisce un bonus ulteriore di +1 fino ad un massimo di +3.

La \textbf{terza volta} che prendi questa Abilità, requisito Competenza Magica 8, puoi usare sei Punti Magia ed un arma con cui vieni a contatto diventa un arma +3 per 6 round, se è già incantata acquisisce un bonus ulteriore di +2 fino ad un massimo di +4.

\smallskip\noindent\rule{\linewidth}{2pt} \index[Abilita]{Infondere Energia Magica Superiore}\hypertarget{Infondere Energia Magica Superiore}{}\medskip\noindent{\textbf{Infondere Energia Magica Superiore}}\pdfbookmark[3]{Infondere Energia Magica Superiore}{Infondere Energia Magica Superiore}
\noindent
\begin{description}[noitemsep, topsep=0pt, parsep=0pt, partopsep=0pt, leftmargin=0cm, labelwidth=2.5cm]
    \item[\textbf{Requisito}]: Competenza Armi 4, Competenza Magica 6
    \item[\textbf{Tiri Salvezza}]: +1 Riflessi, +1 Tempra
    \item[\textbf{Caratteristica}]: Modificatore di caratteristica per incantesimi o a scelta
\end{description}

Sai infondere l'arma di energia magica per fargli acquisire capacità fantastiche.

Costa 1 Azione attivare l'infusione della magia nell'arma. L'arma deve essere magica.

La \textbf{prima volta} che prendi questa Abilità usando un Punto Magia a round puoi rendere la tua arma fiammeggiante o elettrificata o cambiare la forma.

Ogni colpo portato a segna causa 1d6 di danni da fuoco o elettricità aggiuntivi. Se smetti di pagare il Punto Magia torna alla forma precedente e smette di causare danni aggiuntivi.

La \textbf{seconda volta} che prendi questa Abilità usando due Punti Magia a round puoi rendere un arma con cui vieni a contatto estremamente pericolosa. Ogni colpo portato a segno causa 1 danno critico aggiuntivo. Requisito Competenza Magica 7.

La \textbf{terza volta} che prendi questa Abilità usando tre Punti Magia a round puoi concedere ad un arma con cui vieni a contatto entrambi le Abilità precedenti.

Le Abilità non sono cumulative, devi scegliere quale applicare round per round.

\smallskip\noindent\rule{\linewidth}{2pt} \index[Abilita]{Infondere Paura}\hypertarget{Infondere Paura}{}\medskip\noindent{\textbf{Infondere Paura}}\pdfbookmark[3]{Infondere Paura}{Infondere Paura}
\noindent
\begin{description}[noitemsep, topsep=0pt, parsep=0pt, partopsep=0pt, leftmargin=0cm, labelwidth=2.5cm]
    \item[\textbf{Requisito}]: Carisma 2
    \item[\textbf{Tiri Salvezza}]: +2 Volontà, +1 Tempra
    \item[\textbf{Caratteristica}]: Carisma o Costituzione
\end{description}

Tramite la tua esibizione, canora, di balletto, oratoria.. sei in grado di infondere paura negli avversari in grado di sentirti, nel raggio di 6 metri.

La \textbf{prima volta} che prendi questa Abilità i tuoi nemici hanno penalità di -1 al Tiro per Colpire ed al Danno in combattimento.

La \textbf{seconda volta} che prendi questa Abilità, requisito Intrattenere 4, la forza della tua arte aggredisce i nemici e puoi selezionare due effetti tra: -2 Tiro per Colpire, -2 Danno in combattimento, -2 Difesa, -2 al Tiri Salvezza su Volontà. I tuoi nemici devono essere entro 12 metri di raggio.

La \textbf{terza volta} che prendi questa Abilità, requisito Intrattenere 12, la forza della tua arte aggredisce i nemici e puoi selezionare due effetti tra: -1d6 Tiro per Colpire, -4 Difesa, -4 Danno, -1d6 TS. I tuoi nemici devono essere entro 24 metri di raggio.

All'avversario è concesso un Tiro Salvezza Volontà DC pari 10+CAR+punteggio Intrattenere. Una creatura che riesce nel Tiro Salvezza è immune per quel giorno a nuove manifestazioni di questo tuo potere.

Attivare, mantenere o cambiare effetto dell'Abilità richiede 2 Azioni e dura fino all'inizio del tuo round successivo. Puoi mantenere l'Abilità un numero di round, anche non consecutivi, pari a punteggio Intrattenere x 3 al giorno. Le creature per rimanere influenzate devono continuare a vedere/sentire la performance.

\smallskip\noindent\rule{\linewidth}{2pt} \index[Abilita]{Iniziativa migliorata}\hypertarget{Iniziativa migliorata}{}\medskip\noindent{\textbf{Iniziativa migliorata}}\pdfbookmark[3]{Iniziativa migliorata}{Iniziativa migliorata}
\noindent
\begin{description}[noitemsep, topsep=0pt, parsep=0pt, partopsep=0pt, leftmargin=0cm, labelwidth=2.5cm]
    \item[\textbf{Requisito}]: Intelligenza o Destrezza 1
    \item[\textbf{Tiri Salvezza}]: +2 Riflessi
    \item[\textbf{Caratteristica}]: Destrezza o Intelligenza
\end{description}

Aumenti l'iniziativa di +1. L'Abilità può essere presa fino a 2 volte ed il bonus si cumula.

\smallskip\noindent\rule{\linewidth}{2pt} \index[Abilita]{La mia pelle}\hypertarget{La mia pelle}{}\medskip\noindent{\textbf{La mia pelle}}\pdfbookmark[3]{La mia pelle}{La mia pelle}
\noindent
\begin{description}[noitemsep, topsep=0pt, parsep=0pt, partopsep=0pt, leftmargin=0cm, labelwidth=2.5cm]
    \item[\textbf{Requisito}]: Competenza Armi 1
    \item[\textbf{Tiri Salvezza}]: +3 Tempra
    \item[\textbf{Caratteristica}]: Costituzione o Forza
\end{description}

Hai un rapporto quasi simbiotico con la tua armatura.

La \textbf{prima volta} che prendi questa Abilità la Difesa che ti concede l'armatura che porti aumenta di 1.

La \textbf{seconda volta} che prendi questa Abilità, requisito Competenza Armi 6, la Difesa che ti concede l'armatura che porti aumenta di 2.

\smallskip\noindent\rule{\linewidth}{2pt} \index[Abilita]{La mia morte la tua morte}\hypertarget{La mia morte la tua morte}{}\medskip\noindent{\textbf{La mia morte la tua morte}}\pdfbookmark[3]{La mia morte la tua morte}{La mia morte la tua morte}
\noindent
\begin{description}[noitemsep, topsep=0pt, parsep=0pt, partopsep=0pt, leftmargin=0cm, labelwidth=2.5cm]
    \item[\textbf{Requisito}]: Competenza Armi 1, Forza 1

    \item[\textbf{Tiri Salvezza}]: +2 Tempra, +1 Volontà
    \item[\textbf{Caratteristica}]: Carisma o Forza
\end{description}

Per ogni singolo avversario in combattimento puoi fare che il primo colpo a segno dello scontro causi un danno aggiuntivo pari al doppio di Competenza Armi. L'avversario guadagna un bonus al Tiro per Colpire ed al danno pari al valore della tua Competenza Armi al primo Tiro per Colpire effettuato entro la fine del round successivo.

L'Abilità va dichiarato prima del Tiro per Colpire e dura fino all'inizio del prossimo round.

\smallskip\noindent\rule{\linewidth}{2pt} \index[Abilita]{La mia Testa è più Dura}\hypertarget{La mia Testa è più Dura}{}\medskip\noindent{\textbf{La mia Testa è più Dura}}\pdfbookmark[3]{La mia Testa è più Dura}{La mia Testa è più Dura}
\noindent
\begin{description}[noitemsep, topsep=0pt, parsep=0pt, partopsep=0pt, leftmargin=0cm, labelwidth=2.5cm]
    \item[\textbf{Requisito}]: Competenza Armi 1
    \item[\textbf{Tiri Salvezza}]: +1 Tempra, +1 Volontà
    \item[\textbf{Caratteristica}]: Forza o Costituzione
\end{description}

La tua arma, presente nella Lista Armi Rompi Cranio, fa +2 danni

\smallskip\noindent\rule{\linewidth}{2pt} \index[Abilita]{Lesto}\hypertarget{Lesto}{}\medskip\noindent{\textbf{Lesto}}\pdfbookmark[3]{Lesto}{Lesto}
\noindent
\begin{description}[noitemsep, topsep=0pt, parsep=0pt, partopsep=0pt, leftmargin=0cm, labelwidth=2.5cm]
    \item[\textbf{Requisito}]: Competenza Armi 3
    \item[\textbf{Tiri Salvezza}]: +1 Riflessi, +1 Volontà
    \item[\textbf{Caratteristica}]: Destrezza o Forza
\end{description}

La \textbf{prima volta} che prendi questa Abilità quando un alleato esegue un tiro critico puoi, usando una reazione, effettuare un Tiro per Colpire, con penalità di -1d6, contro il medesimo avversario purché in mischia anche con te.

La \textbf{seconda volta} che prendi questa Abilità, Competenza Armi 6, guadagni una reazione da poter usare solo per l'Abilità Lesto.

\smallskip\noindent\rule{\linewidth}{2pt} \index[Abilita]{Litania versatile}\hypertarget{Litania versatile}{}\medskip\noindent{\textbf{Litania versatile}}\pdfbookmark[3]{Litania versatile}{Litania versatile}
\noindent
\begin{description}[noitemsep, topsep=0pt, parsep=0pt, partopsep=0pt, leftmargin=0cm, labelwidth=2.5cm]
    \item[\textbf{Requisito}]: Competenza Intrattenere 6
    \item[\textbf{Tiri Salvezza}]: +1 Volontà, +1 Riflessi
    \item[\textbf{Caratteristica}]: Carisma o Destrezza
\end{description}

Tramite la tua esibizione puoi decidere di infondere coraggio o paura alle creature entro 9 metri da te.

La \textbf{prima volta} che prendi questa Abilità ogni round, per tutte le creature influenzate, puoi decidere di applicare fino a 2 modificatori tra: bonus di +4 al Tiro per Colpire oppure +4 alla Difesa oppure -4 al Tiro per Colpire oppure -4 alla Difesa.

All'avversario è concesso un Tiro Salvezza Volontà DC pari 10+CAR+punteggio Intrattenere. Una creatura che riesce nel Tiro Salvezza è immune per quel giorno a nuove manifestazioni di questo tuo potere.

Attivare e mantenere l'Abilità richiede 2 Azioni. Puoi mantenere l'Abilità un numero di round, anche non consecutivi, pari a punteggio Intrattenere al giorno.

Le creature per rimanere influenzate devono continuare a vedere/sentire la performance.

La \textbf{seconda volta} che prendi questa Abilità puoi escludere dall'effetto un numero di creature fino al punteggio di Carisma

\smallskip\noindent\rule{\linewidth}{2pt} \index[Abilita]{Lo scudo è mio amico}\hypertarget{Lo scudo è mio amico}{}\medskip\noindent{\textbf{Lo scudo è mio amico}}\pdfbookmark[3]{Lo scudo è mio amico}{Lo scudo è mio amico}
\noindent
\begin{description}[noitemsep, topsep=0pt, parsep=0pt, partopsep=0pt, leftmargin=0cm, labelwidth=2.5cm]
    \item[\textbf{Requisito}]: Competenza Armi 1, Competenza Magica 1
    \item[\textbf{Tiri Salvezza}]: +1 Tempra, +1 Riflessi
    \item[\textbf{Caratteristica}]: Destrezza o Costituzione
\end{description}

Il tuo \emph{amico} è sempre al tuo fianco.

La \textbf{prima volta} che prendi questa Abilità il costante allenamento con lo scudo ti permette di usare scudi leggeri senza penalità alla Prova di Magia.

La \textbf{seconda volta} che prendi questa Abilità il costante allenamento con lo scudo ti permette di usare scudi medi riducendo la penalità di 2 alla Prova di Magia.

La \textbf{terza volta} che prendi questa Abilità il costante allenamento con lo scudo ti permette di usare scudi medi senza penalità alla Prova di Magia e riducendola di 2 quando usi gli scudi pesanti.

\smallskip\noindent\rule{\linewidth}{2pt} \index[Abilita]{Magie Potenti}\hypertarget{Magie Potenti}{}\medskip\noindent{\textbf{Magie Potenti}}\pdfbookmark[3]{Magie Potenti}{Magie Potenti}
\noindent
\begin{description}[noitemsep, topsep=0pt, parsep=0pt, partopsep=0pt, leftmargin=0cm, labelwidth=2.5cm]
    \item[\textbf{Requisito}]: Competenza Magica 5
    \item[\textbf{Tiri Salvezza}]: +2 Volontà
    \item[\textbf{Caratteristica}]: Modificatore di caratteristica per incantesimi o a scelta
\end{description}

Le tue magie sono straordinariamente efficaci.

Scegli una Liste di Magia, ottieni un +1d6 alla Prova di Magia nel lancio di incantesimi da questa Lista. L'Abilità può essere presa più volte ma il totale deve essere inferiore od uguale a CM/4.

\smallskip\noindent\rule{\linewidth}{2pt} \index[Abilita]{Molla}\hypertarget{Molla}{}\medskip\noindent{\textbf{Molla}}\pdfbookmark[3]{Molla}{Molla}
\noindent
\begin{description}[noitemsep, topsep=0pt, parsep=0pt, partopsep=0pt, leftmargin=0cm, labelwidth=2.5cm]
    \item[\textbf{Requisito}]: Forza 0
    \item[\textbf{Tiri Salvezza}]: +1 Riflessi, +1 Tempra
    \item[\textbf{Caratteristica}]: Destrezza o Forza
\end{description}

La \textbf{prima volta} che prendi questa Abilità puoi ignorare il requisito dei 3 metri di rincorsa prima di un salto.

La \textbf{seconda volta} che prendi questa Abilità quando effettui una prova per saltare in lungo od in alto tiri 1d6 in più.

\begin{center}
	\includegraphics[width=0.65\linewidth]{immagini/elcolosso.png}

	\emph{The Colossus (also known as The Giant), is known in Spanish as El Coloso.}
\end{center}

\smallskip\noindent\rule{\linewidth}{2pt} \index[Abilita]{Montagna umana}\hypertarget{Montagna umana}{}\medskip\noindent{\textbf{Montagna umana}}\pdfbookmark[3]{Montagna umana}{Montagna umana}
\noindent
\begin{description}[noitemsep, topsep=0pt, parsep=0pt, partopsep=0pt, leftmargin=0cm, labelwidth=2.5cm]
    \item[\textbf{Requisito}]: Costituzione 1
    \item[\textbf{Tiri Salvezza}]: +3 Tempra
    \item[\textbf{Caratteristica}]: Costituzione o Forza
\end{description}

Forse una volta eri gracile e debole, adesso sei una montagna di muscoli.

La \textbf{prima volta} che prendi questa Abilità quando prendi questa Abilità aumenti di 1d6 i Punti Ferita.

La \textbf{seconda volta} che prendi questa Abilità aumenti di 1 i Punti Ferita presi per livello.

La \textbf{terza volta} che prendi questa Abilità aumenti la taglia del dado per tirare i Punti Ferita.

I bonus sono cumulativi e retroattivi ai livelli precedenti, tranne che l'aumento del dado per determinare i PF.

La \textbf{quarta volta}, requisito Costituzione 3, che prendi questa Abilità aumenti di una taglia (P > M > G).

\smallskip\noindent\rule{\linewidth}{2pt} \index[Abilita]{Muro mentale}\hypertarget{Muro mentale}{}\medskip\noindent{\textbf{Muro mentale}}\pdfbookmark[3]{Muro mentale}{Muro mentale}
\noindent
\begin{description}[noitemsep, topsep=0pt, parsep=0pt, partopsep=0pt, leftmargin=0cm, labelwidth=2.5cm]
    \item[\textbf{Requisito}]: Saggezza +1
    \item[\textbf{Tiri Salvezza}]: +2 Volontà, +1 Tempra
    \item[\textbf{Caratteristica}]: Modificatore di caratteristica per incantesimi o Saggezza
\end{description}

La tua mente è addestrata contro chi vuole influenzarla. Ogni volta che prendi questa Abilità guadagni +1 ai Tiri Salvezza contro gli incantesimi della Lista di Magia di Ammaliamento.

\smallskip\noindent\rule{\linewidth}{2pt} \index[Abilita]{Occhi della magia}\hypertarget{Occhi della magia}{}\medskip\noindent{\textbf{Occhi della magia}}\pdfbookmark[3]{Occhi della magia}{Occhi della magia}
\noindent
\begin{description}[noitemsep, topsep=0pt, parsep=0pt, partopsep=0pt, leftmargin=0cm, labelwidth=2.5cm]
    \item[\textbf{Requisito}]: Competenza Magica 1
    \item[\textbf{Tiri Salvezza}]: +1 Volontà, +1 Tempra
    \item[\textbf{Caratteristica}]: Modificatore di caratteristica per incantesimi o Carisma
\end{description}

La \textbf{prima volta} che prendi questa Abilità se lo puoi vedere sai anche se è magico. Costa una Azione attivare la vista magica e dura un round.

La \textbf{seconda volta}, requisito Competenza Magica 1, che prendi l'Abilità attivare la vista magica costa una Reazione.
\smallskip\noindent\rule{\linewidth}{2pt} \index[Abilita]{Occhio Clinico}\hypertarget{Occhio Clinico}{}\medskip\noindent{\textbf{Occhio Clinico}}\pdfbookmark[3]{Occhio Clinico}{Occhio Clinico}
\noindent
\begin{description}[noitemsep, topsep=0pt, parsep=0pt, partopsep=0pt, leftmargin=0cm, labelwidth=2.5cm]
    \item[\textbf{Requisito}]: Competenza Armi 3
    \item[\textbf{Tiri Salvezza}]: +2 Riflessi
    \item[\textbf{Caratteristica}]: Destrezza o Forza
\end{description}

Sei in grado di fare danno critico a creature normalmente immuni ai critici.

\smallskip\noindent\rule{\linewidth}{2pt} \index[Abilita]{Occhio di Falco}\hypertarget{Occhio di Falco}{}\medskip\noindent{\textbf{Occhio di Falco}}\pdfbookmark[3]{Occhio di Falco}{Occhio di Falco}
\noindent
\begin{description}[noitemsep, topsep=0pt, parsep=0pt, partopsep=0pt, leftmargin=0cm, labelwidth=2.5cm]
    \item[\textbf{Requisito}]: Competenza Armi 3
    \item[\textbf{Tiri Salvezza}]: +2 Riflessi, +1 Volontà
    \item[\textbf{Caratteristica}]: Destrezza o Intelligenza
\end{description}

La \textbf{prima volta} che prendi questa Abilità i proiettili, frecce o dardi, lanciati tra il primo ed il secondo incremento di gittata non hanno penalità al Tiro per Colpire.

La \textbf{seconda volta} che prendi questa Abilità, la penalità per i tiri entro il terzo incremento di portata è di 6.

La \textbf{terza volta} che prendi questa Abilità sei in grado di estendere ancora di più il tuo tiro e portarlo ad un quinto incremento con un -12 di penalità al colpire. Non hai penalità entro i primi 3 incrementi mentre hai -6 a colpire tra il terzo e quarto incremento.

\smallskip\noindent\rule{\linewidth}{2pt} \index[Abilita]{Opportunista}\hypertarget{Opportunista}{}\medskip\noindent{\textbf{Opportunista}}\pdfbookmark[3]{Opportunista}{Opportunista}
\noindent
\begin{description}[noitemsep, topsep=0pt, parsep=0pt, partopsep=0pt, leftmargin=0cm, labelwidth=2.5cm]
    \item[\textbf{Requisito}]: Competenza Armi 2
    \item[\textbf{Tiri Salvezza}]: +2 Riflessi, +1 Volontà
    \item[\textbf{Caratteristica}]: Intelligenza o Destrezza
\end{description}

\label{attaccoopportunita}

Puoi tentare di colpire in mischia un avversario che \textbf{esca} o \textbf{attraversi} un area di mischia che tu minacci oppure che \textbf{usi un arma da lancio} nella tua area di mischia o \textbf{formuli un incantesimo}. L'Abilità è usabile come Reazione. Questo attacco viene anche chiamato attacco di opportunità.

\smallskip\noindent\rule{\linewidth}{2pt} \index[Abilita]{Parata}\hypertarget{Parata}{}\medskip\noindent{\textbf{Parata}}\pdfbookmark[3]{Parata}{Parata}
\noindent
\begin{description}[noitemsep, topsep=0pt, parsep=0pt, partopsep=0pt, leftmargin=0cm, labelwidth=2.5cm]
    \item[\textbf{Requisito}]: Competenza Armi 3 oppure Pugno Vuoto 2
    \item[\textbf{Tiri Salvezza}]: +1 Riflessi, +1 Volontà
    \item[\textbf{Caratteristica}]: Forza o Destrezza
\end{description}

La \textbf{prima volta} che prendi questa Abilità quando usi l'Azione per \hyperlink{preparareladifesa}{Preparare la Difesa} (pag. \pageref{preparareladifesa}) questa aumenta di 1.

La \textbf{seconda volta} che prendi questa Abilità, requisito Competenza Armi 6 oppure Pugno Vuoto 4, usi una Azione Immediata per Preparare la Difesa.

La \textbf{terza volta} che prendi questa Abilità, requisito Competenza Armi 9 oppure Pugno Vuoto 6, puoi usare Preparare la Difesa su una creatura a distanza di mischia da te.

Usare l'Abilità Parata deve essere dichiarata nel proprio round e rimane attiva fino all'inizio del tuo round successivo.

%\medskip

%\begin{center}
%		\includegraphics[width=0.7\linewidth]{immagini/distillare.png}
%
%		\emph{The Alchemist Discovering Phosphorus. Joseph Wright of Derby (1771-1795)}
%\end{center}

\smallskip\noindent\rule{\linewidth}{2pt} \index[Abilita]{Passo Felpato}\hypertarget{Passo Felpato}{}\medskip\noindent{\textbf{Passo Felpato}}\pdfbookmark[3]{Passo Felpato}{Passo Felpato}
\noindent
\begin{description}[noitemsep, topsep=0pt, parsep=0pt, partopsep=0pt, leftmargin=0cm, labelwidth=2.5cm]
    \item[\textbf{Requisito}]: Furtività 1
    \item[\textbf{Tiri Salvezza}]: +1 Riflessi, +1 Tempra
    \item[\textbf{Caratteristica}]: Destrezza o Saggezza
\end{description}

Il tuo passo è naturalmente silenzioso.

La \textbf{prima volta} che prendi questa Abilità la penalità per muoversi a piena velocità usando Furtività diviene di -1d6.

La \textbf{seconda volta} che prendi questa Abilità, requisito Destrezza 3, Furtività 8, non hai penalità a muoverti a piena velocità.

\smallskip\noindent\rule{\linewidth}{2pt} \index[Abilita]{Passo Veloce}\hypertarget{abPasso Veloce}{}\medskip\noindent{\textbf{Passo Veloce}}\pdfbookmark[3]{Passo Veloce}{abPasso Veloce}
\noindent
\begin{description}[noitemsep, topsep=0pt, parsep=0pt, partopsep=0pt, leftmargin=0cm, labelwidth=2.5cm]
    \item[\textbf{Requisito}]: Destrezza 1
    \item[\textbf{Tiri Salvezza}]: +2 Riflessi, +1 Tempra
    \item[\textbf{Caratteristica}]: Destrezza o Costituzione
\end{description}

Il tuo passo è naturalmente rapido.
Se hai movimento 6m passi a movimento 7m, se hai movimento 9m passi a movimento 10m.

Ogni ulteriori \textbf{due volte} che prendi l'Abilità il tuo movimento aumenta di 1 metro per Azione di Movimento, fino ad un massimo di +3 metri a round.

\smallskip\noindent\rule{\linewidth}{2pt} \index[Abilita]{Passo Sicuro}\hypertarget{Passo Sicuro}{}\medskip\noindent{\textbf{Passo Sicuro}}\pdfbookmark[3]{Passo Sicuro}{Passo Sicuro}
\noindent
\begin{description}[noitemsep, topsep=0pt, parsep=0pt, partopsep=0pt, leftmargin=0cm, labelwidth=2.5cm]
    \item[\textbf{Requisito}]: Saggezza 1
    \item[\textbf{Tiri Salvezza}]: +2 Tempra, +1 Riflessi
    \item[\textbf{Caratteristica}]: Destrezza o Costituzione
\end{description}

E' la capacità di non essere rallentati in un ambiente ostile. E' necessario dichiarare su quale ambiente si prende l'Abilità. In questi ambienti il terreno naturale non è difficile. Finché ci si muove nell'ambiente scelto si ha un +1 alle prove di Sopravvivenza.

\medskip

\begin{tabular}{l|l}
\textbf{Ambiente} & \textbf{Ambiente}\\
\toprule
Giungla & Aquatico \& Costale\\
Palude & Collina \& Foresta \\
Pianura & Desertico \\
Montagna & Ghiacciai \& Tundra \\
Urbano& Sotterraneo
\end{tabular}

\medskip

Ogni qual volta si prende nuovamente questa Abilità si sceglie un ambiente diverso e si aggiunge al precedente o ci si specializza sullo stesso.

Le \textbf{seconda volta} che prendi questa Abilità sul medesimo terreno, specializzandoti, acquisisci una capacità a seconda del terreno.

\begin{itemize}[leftmargin=*] \setlength{\itemsep}{0pt}

\item \emph{Giungla / Foresta / Collina / Pianura}: il tuo movimento aumenta di 1 metro su questo terreno

\item \emph{Costale / Aquatico}: velocità di nuoto pari al tuo movimento

\item \emph{Palude}: +2 ai Tiri Salvezza contro Veleno

\item \emph{Desertico}: Riduzione danni da Fuoco pari al livello

\item \emph{Montagna / Ghiacciai / Tundra}: Riduzione danni da Freddo pari al livello

\item \emph{Sotterraneo}: Visione crepuscolare 9 metri

\item \emph{Urbano}: +1 Linguaggio, +1 a scelta in due Conoscenze

\end{itemize}

\smallskip\noindent\rule{\linewidth}{2pt} \index[Abilita]{Pelle Coriacea}\hypertarget{Pelle Coriacea}{}\medskip\noindent{\textbf{Pelle Coriacea}}\pdfbookmark[3]{Pelle Coriacea}{Pelle Coriacea}
\noindent
\begin{description}[noitemsep, topsep=0pt, parsep=0pt, partopsep=0pt, leftmargin=0cm, labelwidth=2.5cm]
    \item[\textbf{Requisito}]: Costituzione 2
    \item[\textbf{Tiri Salvezza}]: +2 Tempra
    \item[\textbf{Caratteristica}]: Costituzione o Saggezza
\end{description}

La \textbf{prima volta} che prendi questa Abilità la tua pelle è estremamente resistente. Subisci 1 danno in meno quando colpito da armi taglienti.

La \textbf{seconda volta} che prendi questa Abilità, requisito Competenza Armi 6, subisci 1 danno in meno quando colpito da armi taglienti, perforanti, contundenti. Riduci di 1 la condizione di Sanguinamento quando acquisita.

La \textbf{terza volta} che prendi questa Abilità, requisito Competenza Armi 12 e Costituzione 3, subisci 1 danno in meno quando colpito da armi taglienti, perforanti o contundenti. Subisci 1 danno in meno quando colpito da magia. Riduci di 1 la condizione di Sanguinamento quando acquisita.

La \textbf{quarta volta} che prendi questa Abilità, requisito Competenza Armi 16, ignori 1 tiro critico quando colpito da armi taglienti, perforanti o contundenti e subisci 1 danno in meno quando colpito da magia. Riduci di 1 la condizione di Sanguinamento quando acquisita.

I bonus sono cumulativi.

\smallskip\noindent\rule{\linewidth}{2pt} \index[Abilita]{Percettivo}\hypertarget{Percettivo}{}\medskip\noindent{\textbf{Percettivo}}\pdfbookmark[3]{Percettivo}{Percettivo}\label{Percettivo}
\noindent
\begin{description}[noitemsep, topsep=0pt, parsep=0pt, partopsep=0pt, leftmargin=0cm, labelwidth=2.5cm]
    \item[\textbf{Requisito}]: Saggezza 0
    \item[\textbf{Tiri Salvezza}]: +1 Riflessi, +1 Volontà
    \item[\textbf{Caratteristica}]: Saggezza o Intelligenza
\end{description}

La tua Consapevolezza e attenzione ai particolari è sopra la media.

Prendi un bonus di +1 alla prove di Consapevolezza. L'Abilità può essere presa massimo 3 volte.

\smallskip\noindent\rule{\linewidth}{2pt} \index[Abilita]{Persona veramente malvagia}\hypertarget{Persona veramente malvagia}{}\medskip\noindent{\textbf{Persona veramente malvagia}}\pdfbookmark[3]{Persona veramente malvagia}{Persona veramente malvagia}
\noindent
\begin{description}[noitemsep, topsep=0pt, parsep=0pt, partopsep=0pt, leftmargin=0cm, labelwidth=2.5cm]
    \item[\textbf{Requisito}]: Competenza Armi 1
    \item[\textbf{Tiri Salvezza}]: +1 Riflessi, +1 Volontà
    \item[\textbf{Caratteristica}]: Forza o Carisma
\end{description}

Quando vuoi sai essere cattivo.

CA/4 volte al giorno aggiungi il tuo valore di Competenza Armi al danno di un singolo attacco in mischia ad un singolo tuo avversario.

L'Abilità deve essere dichiarata prima di sapere l'esito del Tiro per Colpire. Costa una Azione.

\smallskip\noindent\rule{\linewidth}{2pt} \index[Abilita]{più sono grossi più fanno rumore quando cadono}\hypertarget{più sono grossi più fanno rumore quando cadono}{}\medskip\noindent{\textbf{più sono grossi più fanno rumore quando cadono}}\pdfbookmark[3]{più sono grossi più fanno rumore quando cadono}{più sono grossi più fanno rumore quando cadono}
\noindent
\begin{description}[noitemsep, topsep=0pt, parsep=0pt, partopsep=0pt, leftmargin=0cm, labelwidth=2.5cm]
    \item[\textbf{Requisito}]: Competenza Armi 1
    \item[\textbf{Tiri Salvezza}]: +2 Tempra, +1 Volontà
    \item[\textbf{Caratteristica}]: Forza o Carisma
\end{description}

Quando attacchi una creatura di almeno 2 taglie più grosse di te fai +1 danno aggiuntivo ogni 2 punti Competenza Armi. Se è solo una taglia superiore aggiungi 1 danno in più ogni 3 punti Competenza Armi.

\smallskip\noindent\rule{\linewidth}{2pt} \index[Abilita]{Poliglotta}\hypertarget{Poliglotta}{}\medskip\noindent{\textbf{Poliglotta}}\pdfbookmark[3]{Poliglotta}{Poliglotta}
\noindent
\begin{description}[noitemsep, topsep=0pt, parsep=0pt, partopsep=0pt, leftmargin=0cm, labelwidth=2.5cm]
    \item[\textbf{Requisito}]: almeno Intelligenza -1, alla creazione del personaggio
    \item[\textbf{Tiri Salvezza}]: +2 Volontà
    \item[\textbf{Caratteristica}]: Intelligenza o Carisma
\end{description}

Hai una straordinaria capacità di imparare le lingue. Per ogni punto attribuito a Conoscenza Lingue conosci due lingue in più.

\smallskip\noindent\rule{\linewidth}{2pt} \index[Abilita]{Potere del Patrono}\hypertarget{Potere del Patrono}{}\medskip\noindent{\textbf{Potere del Patrono}}\pdfbookmark[3]{Potere del Patrono}{Potere del Patrono}
\noindent
\begin{description}[noitemsep, topsep=0pt, parsep=0pt, partopsep=0pt, leftmargin=0cm, labelwidth=2.5cm]
    \item[\textbf{Requisito}]: Somma Tratti comuni con il Patrono 1, essere Devoto
    \item[\textbf{Tiri Salvezza}]: +1 Tempra, +2 Volontà
    \item[\textbf{Caratteristica}]: Modificatore di caratteristica per incantesimi o a scelta
\end{description}

La tua fede nel Patrono non conosce limiti ne crolli di fiducia.

1 volta al giorno per singola prova, come Reazione prima di effettuare la prova, usi come modificatore positivo unico la somma dei Tratti comune con il Patrono. Puoi usare questa Abilità su Tiri Salvezza, Tiri per Colpire e prove di competenze.

Se riescono tutte e tre le prove è probabile che si sia una Manifestazione del Patrono.

\smallskip\noindent\rule{\linewidth}{2pt} \index[Abilita]{Primo Sangue}\hypertarget{Primo Sangue}{}\medskip\noindent{\textbf{Primo Sangue}}\pdfbookmark[3]{Primo Sangue}{Primo Sangue}
\noindent
\begin{description}[noitemsep, topsep=0pt, parsep=0pt, partopsep=0pt, leftmargin=0cm, labelwidth=2.5cm]
    \item[\textbf{Requisito}]: Competenza Armi 1
    \item[\textbf{Tiri Salvezza}]: +1 Tempra, +1 Volontà
    \item[\textbf{Caratteristica}]: Intelligenza o Carisma
\end{description}

Il primo Tiro per Colpire nel giorno ha un bonus di +1d6 e causa un tiro critico se colpisce.

\smallskip\noindent\rule{\linewidth}{2pt} \index[Abilita]{Precisino}\hypertarget{Precisino}{}\medskip\noindent{\textbf{Precisino}}\pdfbookmark[3]{Precisino}{Precisino}\label{Precisino}
\noindent
\begin{description}[noitemsep, topsep=0pt, parsep=0pt, partopsep=0pt, leftmargin=0cm, labelwidth=2.5cm]
    \item[\textbf{Requisito}]: Competenza Armi 2
    \item[\textbf{Tiri Salvezza}]: +1 Riflessi, +1 Volontà
    \item[\textbf{Caratteristica}]: Destrezza o Charisma
\end{description}

La \textbf{prima volta} che prendi questa Abilità quando scagli un proiettile diminuisci di 2 la penalità data da Copertura.

La \textbf{seconda volta} che prendi questa Abilità, Competenza Armi 4, riduci la penalità data da Copertura di 4.

\smallskip\noindent\rule{\linewidth}{2pt} \index[Abilita]{Prodigioso}\hypertarget{Prodigioso}{}\medskip\noindent{\textbf{Prodigioso}}\pdfbookmark[3]{Prodigioso}{Prodigioso}
\noindent
\begin{description}[noitemsep, topsep=0pt, parsep=0pt, partopsep=0pt, leftmargin=0cm, labelwidth=2.5cm]
    \item[\textbf{Requisito}]: Competenza Magica 3
    \item[\textbf{Tiri Salvezza}]: +2 Volontà
    \item[\textbf{Caratteristica}]: Modificatore di caratteristica per incantesimi
\end{description}

La tua mente non ha confini. Puoi apprendere due incantesimi presenti sul tuo Tomo di Magia, sempre rispettando i limiti del massimo livello di incantesimi lanciabile.

L'Abilità può essere presa più volte ed il totale deve essere pari o inferiore a CM/4.

\smallskip\noindent\rule{\linewidth}{2pt} \index[Abilita]{Proseguire}\hypertarget{Proseguire}{}\medskip\noindent{\textbf{Proseguire}}\pdfbookmark[3]{Proseguire}{Proseguire}
\noindent
\begin{description}[noitemsep, topsep=0pt, parsep=0pt, partopsep=0pt, leftmargin=0cm, labelwidth=2.5cm]
    \item[\textbf{Requisito}]: Competenza Armi 1
    \item[\textbf{Tiri Salvezza}]: +1 Tempra, +1 Volontà
    \item[\textbf{Caratteristica}]: Forza o Destrezza
\end{description}

La \textbf{prima volta} che prendi questa Abilità se elimini un avversario con la tua ultima Azione di Attacco in mischia, puoi effettuare un'azione di attacco bonus.

L'attacco bonus utilizza gli stessi modificatori dell'ultima Azione di Attacco ed è diretto a un altro nemico entro la distanza di mischia.

Se elimini questa seconda creatura, non puoi effettuare ulteriori attacchi.

La \textbf{seconda volta}, requisiti Proseguire, Competenza Armi 6, se con l'attacco bonus di Proseguire elimini un avversario puoi effettuare una ulteriore azione di attacco bonus, utilizzando gli stessi modificatori dell'ultima Azione di Attacco. Se elimini questa creatura puoi continuare, spostandoti di massimo 1 metro, ad attaccare la creatura successiva.

Ogni attacco bonus oltre il primo subisce una penalità cumulativa: -2 al colpire e -1 al danno.

\smallskip\noindent\rule{\linewidth}{2pt} \index[Abilita]{Pugno di Ferro}\hypertarget{Pugno di Ferro}{}\medskip\noindent{\textbf{Pugno di Ferro}}\pdfbookmark[3]{Pugno di Ferro}{Pugno di Ferro}
\noindent
\begin{description}[noitemsep, topsep=0pt, parsep=0pt, partopsep=0pt, leftmargin=0cm, labelwidth=2.5cm]
    \item[\textbf{Requisito}]: Lista Pugno Vuoto 3
    \item[\textbf{Tiri Salvezza}]: +2 Tempra, +1 Volontà
    \item[\textbf{Caratteristica}]: Forza o Destrezza
\end{description}

La tua tecnica di combattimento senza armi è estremamente precisa e potente.

La \textbf{prima volta} che prendi questa Abilità il danno causato dai tuoi colpi ed il Tiro per Colpire aumenta di 1.

La \textbf{seconda volta} che prendi questa Abilità, requisito Pugno Vuoto 6. Il danno aumenta di +2, il Tiro per Colpire +1.

La \textbf{terza volta} che prendi questa Abilità, requisito Pugno Vuoto 9. Danno +1, Tiro per Colpire +2.

La \textbf{quarta volta} che prendi questa Abilità, requisito Pugno Vuoto 12. Danno +2, Tiro per Colpire +1.

La \textbf{quinta volta} che prendi questa Abilità, requisito Pugno Vuoto 15. Danno +1, Tiro per Colpire +2.

La \textbf{sesta volta} che prendi questa Abilità, Requisito Pugno Vuoto 18. Danno +2, Tiro per Colpire +1.

I bonus indicati sono cumulativi.

\smallskip\noindent\rule{\linewidth}{2pt} \index[Abilita]{Pugno Potente}\hypertarget{Pugno Potente}{}\medskip\noindent{\textbf{Pugno Potente}}\pdfbookmark[3]{Pugno Potente}{Pugno Potente}
\noindent
\begin{description}[noitemsep, topsep=0pt, parsep=0pt, partopsep=0pt, leftmargin=0cm, labelwidth=2.5cm]
    \item[\textbf{Requisito}]: Lista Pugno Vuoto 3
    \item[\textbf{Tiri Salvezza}]: +1 Tempra, +2 Volontà
    \item[\textbf{Caratteristica}]: Forza o Costituzione
\end{description}

Consumi 2 Azioni. Effettui un unico Tiro per Colpire con -5 di penalità.
Se colpisci, oltre al danno ed un danno critico, l'avversario che deve essere massimo di due taglie superiore alla tua deve effettuare un Tiro Salvezza su Tempra con DC pari al tuo Tiro per Colpire oppure essere spinto di 3 metri in una direzione a tua scelta.

Se fallisce il Tiro Salvezza in maniera critica subisce un ulteriore danno critico.

\smallskip\noindent\rule{\linewidth}{2pt} \index[Abilita]{Questo è il mio pugnale}\hypertarget{Questo è il mio pugnale}{}\medskip\noindent{\textbf{Questo è il mio pugnale}}\pdfbookmark[3]{Questo è il mio pugnale}{Questo è il mio pugnale}
\noindent
\begin{description}[noitemsep, topsep=0pt, parsep=0pt, partopsep=0pt, leftmargin=0cm, labelwidth=2.5cm]
    \item[\textbf{Requisito}]: Competenza Armi 1
    \item[\textbf{Tiri Salvezza}]: +2 Tempra, +1 Riflessi
    \item[\textbf{Caratteristica}]: Destrezza o Carisma
\end{description}

Quando fai un danno critico con il tuo pugnale sommi un ulteriore danno critico. L'Abilità è usabile 1 volta per avversario e si applica automaticamente al primo danno critico effettuato.

\smallskip\noindent\rule{\linewidth}{2pt} \index[Abilita]{Questa è la mia arma!}\hypertarget{Questa è la mia arma!}{}\medskip\noindent{\textbf{Questa è la mia arma!}}\pdfbookmark[3]{Questa è la mia arma!}{Questa è la mia arma!}
\noindent
\begin{description}[noitemsep, topsep=0pt, parsep=0pt, partopsep=0pt, leftmargin=0cm, labelwidth=2.5cm]
    \item[\textbf{Requisito}]: Competenza Armi 1
    \item[\textbf{Tiri Salvezza}]: +2 Tempra, +1 Volontà
    \item[\textbf{Caratteristica}]: Forza o Carisma
\end{description}

La \textbf{prima volta} che prendi questa Abilità ogni volta che colpisci il medesimo avversario, a partire dal secondo round, fai un danno aggiuntivo (Max +1 per round di combattimento, anche se lo colpisci più volte nel round) fino ad un massimo +5. La prima volta che non colpisci nel round l'avversario il bonus torna a +0. Il bonus si può mantenere su un solo avversario alla volta.

La \textbf{seconda volta} che prendi questa Abilità, Competenza Armi 5, puoi mancare l'avversario con un colpo e non perdere i benefici.

\smallskip\noindent\rule{\linewidth}{2pt} \index[Abilita]{Radici magiche}\hypertarget{Radici magiche}{}\medskip\noindent{\textbf{Radici magiche}}\pdfbookmark[3]{Radici magiche}{Radici magiche}
\noindent
\begin{description}[noitemsep, topsep=0pt, parsep=0pt, partopsep=0pt, leftmargin=0cm, labelwidth=2.5cm]
    \item[\textbf{Requisito}]: Competenza Magica 1
    \item[\textbf{Tiri Salvezza}]: +2 Volontà, +1 Tempra
    \item[\textbf{Caratteristica}]: Modificatore di caratteristica per incantesimi o a scelta
\end{description}

Finché sei influenzato da un tuo incantesimo, utilizzando un'Azione la tua arma guadagna un +1 al colpire ed al danno e si considera un'arma magica fino alla fine del round.


\begin{center}
	%	\includegraphics[width=0.7\linewidth]{immagini/distillare.png}
	%
	%	\emph{The Alchemist Discovering Phosphorus. Joseph Wright of Derby (1771-1795)}

	\includegraphics[width=0.7\linewidth]{immagini/Theatrum_Chemicum_Britannicum.png}

	\emph{Theatrum Chemicum Britannicum, 1652}
\end{center}

\smallskip\noindent\rule{\linewidth}{2pt} \index[Abilita]{Rappresaglia}\hypertarget{Rappresaglia}{}\medskip\noindent{\textbf{Rappresaglia}}\pdfbookmark[3]{Rappresaglia}{Rappresaglia}
\noindent
\begin{description}[noitemsep, topsep=0pt, parsep=0pt, partopsep=0pt, leftmargin=0cm, labelwidth=2.5cm]
    \item[\textbf{Requisito}]: Competenza Armi 1, almeno Seguace di Gradh o Nedraf o Orlaith o Sumkjr
    \item[\textbf{Tiri Salvezza}]: +2 Volontà, +1 Tempra
    \item[\textbf{Caratteristica}]: Carisma o Saggezza
\end{description}

Vedere i tuoi amici feriti ti riempie di rabbia.

Quanto un compagno (o te stesso) scende sotto metà dei Punti Ferita guadagni un +1 al Tiro per Colpire e Tiri Salvezza.

La durata massima dell'effetto é 1 minuto (6 round) al giorno e deve essere consecutiva. Il giocatore sceglie se attivare o meno l'Abilità ed il compagno ferito deve essere entro 9 metri.

Puoi prendere questa Abilità \textbf{fino a 3 volte}, ogni volta il bonus al Tiro per Colpire e Tiro Salvezza aumentano di 1.

\smallskip\noindent\rule{\linewidth}{2pt} \index[Abilita]{Recupero}\hypertarget{Recupero}{}\medskip\noindent{\textbf{Recupero}}\pdfbookmark[3]{Recupero}{Recupero}
\noindent
\begin{description}[noitemsep, topsep=0pt, parsep=0pt, partopsep=0pt, leftmargin=0cm, labelwidth=2.5cm]
    \item[\textbf{Requisito}]: Costituzione 0
    \item[\textbf{Tiri Salvezza}]: +2 Tempra
    \item[\textbf{Caratteristica}]: Costituzione
\end{description}

Il tuo corpo produce spontaneamente caffeina!

Impieghi la metà del tempo per recuperare dalle condizioni Affaticato.

\smallskip\noindent\rule{\linewidth}{2pt} \index[Abilita]{Resistenza della pietra}\hypertarget{Resistenza della pietra}{}\medskip\noindent{\textbf{Resistenza della pietra}}\pdfbookmark[3]{Resistenza della pietra}{Resistenza della pietra}
\noindent
\begin{description}[noitemsep, topsep=0pt, parsep=0pt, partopsep=0pt, leftmargin=0cm, labelwidth=2.5cm]
    \item[\textbf{Requisito}]: Costituzione 0
    \item[\textbf{Tiri Salvezza}]: nessuno
    \item[\textbf{Caratteristica}]: Costituzione o Saggezza
\end{description}

Nel tempo hai allenato la tua Costituzione a reggere gli urti, trasformazioni, veleni e quant'altro volesse modificare il tuo corpo.

La \textbf{prima volta} che prendi questa Abilità ottieni un bonus di +2 al Tiro Salvezza su Tempra. Il bonus è cumulativo, +2 la prima volta, +1 la \textbf{seconda}, +1 la \textbf{terza}.

La \textbf{quarta volta} che prendi questa Abilità puoi decidere di riuscire automaticamente in un Tiro Salvezza su Tempra una volta al giorno come Reazione. Deve essere dichiarata e non fa tirare il Tiro Salvezza.

\begin{center}
	\includegraphics[width=0.9\linewidth]{immagini/Historia_Mundi_Naturalis.png}

	\emph{Woodcut illustration from an edition of Pliny the Elder's Naturalis Historia (1582)}
\end{center}

\smallskip\noindent\rule{\linewidth}{2pt} \index[Abilita]{Riflessi fulminei}\hypertarget{Riflessi fulminei}{}\medskip\noindent{\textbf{Riflessi fulminei}}\pdfbookmark[3]{Riflessi fulminei}{Riflessi fulminei}
\noindent
\begin{description}[noitemsep, topsep=0pt, parsep=0pt, partopsep=0pt, leftmargin=0cm, labelwidth=2.5cm]
    \item[\textbf{Requisito}]: Destrezza 1
    \item[\textbf{Tiri Salvezza}]: nessuno
    \item[\textbf{Caratteristica}]: Destrezza o Intelligenza
\end{description}

Nel tempo hai allenato i tuoi riflessi a schivare e prevedere qualsiasi ostacolo. Il bonus è cumulativo, +2 la \textbf{prima volta}, +1 la \textbf{seconda}, +1 la \textbf{terza} ai Tiri Salvezza su Riflessi.

La \textbf{quarta volta} che prendi questa Abilità puoi decidere di riuscire automaticamente in un Tiro Salvezza su Riflessi una volta al giorno come Reazione. Deve essere dichiarata e non fa tirare il Tiro Salvezza.

\smallskip\noindent\rule{\linewidth}{2pt} \index[Abilita]{Rinoceronte}\hypertarget{abRinoceronte}{}\medskip\noindent{\textbf{Rinoceronte}}\pdfbookmark[3]{Rinoceronte}{abRinoceronte}\label{Rinoceronte}
\noindent
\begin{description}[noitemsep, topsep=0pt, parsep=0pt, partopsep=0pt, leftmargin=0cm, labelwidth=2.5cm]
    \item[\textbf{Requisito}]: Costituzione 1
    \item[\textbf{Tiri Salvezza}]: +2 Tempra
    \item[\textbf{Caratteristica}]: Costituzione o Saggezza
\end{description}

La tua carica è distruttiva! Dietro di te lasci solo una scia di macerie e non ti lasci fermare da nulla.

La \textbf{prima volta} che prendi questa Abilità puoi effettuare una Carica anche se il terreno è difficile.

La \textbf{seconda volta} che prendi questa Abilità non consideri il terreno come difficile quando carichi.

La \textbf{terza volta} che prendi questa Abilità hai indurito la tua pelle talmente tanto che la tua Difesa naturale aumenta di +1.

\smallskip\noindent\rule{\linewidth}{2pt} \index[Abilita]{Robusto}\hypertarget{Robusto}{}\medskip\noindent{\textbf{Robusto}}\pdfbookmark[3]{Robusto}{Robusto}
\noindent
\begin{description}[noitemsep, topsep=0pt, parsep=0pt, partopsep=0pt, leftmargin=0cm, labelwidth=2.5cm]
    \item[\textbf{Requisito}]: Costituzione -1
    \item[\textbf{Tiri Salvezza}]: +2 Tempra
    \item[\textbf{Caratteristica}]: Costituzione o Saggezza
\end{description}

La \textbf{prima volta} che prendi questa Abilità aumenti i Punti Ferita di 3.

La \textbf{seconda volta} che prendi questa Abilità aumenti di 1 i Punti Ferita presi per livello.

I bonus sono cumulativi e retroattivi ai livelli precedenti.

\smallskip\noindent\rule{\linewidth}{2pt} \index[Abilita]{Sangue Puro}\hypertarget{Sangue Puro}{}\medskip\noindent{\textbf{Sangue Puro}}\pdfbookmark[3]{Sangue Puro}{Sangue Puro}
\noindent
\begin{description}[noitemsep, topsep=0pt, parsep=0pt, partopsep=0pt, leftmargin=0cm, labelwidth=2.5cm]
    \item[\textbf{Requisito}]: Animalia, Devoto di Efrem oppure Shayalia
    \item[\textbf{Tiri Salvezza}]: +1 Volontà , +2 Tempra
    \item[\textbf{Caratteristica}]: Modificatore di caratteristica per incantesimi o Costituzione
\end{description}

La \textbf{prima volta} che prendi questa Abilità con questa Abilità ogni tuo attacco quando ti trasformi con Animalia causa 1 danno aggiuntivo ed è considerato un attacco magico +1. Concentrandoti sul tuo passo puoi lasciare le impronte di un animale in cui ti puoi trasformare ed il terreno si considera doppiamente difficile.

Le \textbf{seconda volta} che prendi questa Abilità, Competenza Magica 8, quando usi l'Abilità di Animalia puoi eseguire una trasformazione parziale ovvero prendere il tipo di Movimento oppure Sensi della creatura in cui ti trasformi. Quando usi l'Abilità Animalia puoi selezionare una creatura con un Grado di Sfida aumentato di 1. Lasciare impronte diverse è considerato terreno difficile.

Le \textbf{terza volta} che prendi questa Abilità, Competenza Magica 12, quando ti trasformi in un animale, puoi usare la tua Competenza Magica al posto della Competenza Armi negli attacchi naturali. Quando usi l'Abilità Animalia puoi selezionare una creatura con un Grado di Sfida aumentato di 1. Lasciare impronte diverse non è considerato terreno difficile.

Le Abilità due e tre sono cumulative.

\smallskip\noindent\rule{\linewidth}{2pt} \index[Abilita]{Sapiente}\hypertarget{Sapiente}{}\medskip\noindent{\textbf{Sapiente}}\pdfbookmark[3]{Sapiente}{Sapiente}
\noindent
\begin{description}[noitemsep, topsep=0pt, parsep=0pt, partopsep=0pt, leftmargin=0cm, labelwidth=2.5cm]
    \item[\textbf{Requisito}]: Competenza Magica 1, solo alla creazione del personaggio
    \item[\textbf{Tiri Salvezza}]: +2 Volontà
    \item[\textbf{Caratteristica}]: Modificatore di caratteristica per incantesimi o a scelta
\end{description}

Il tuo interesse e connessione con la magia non ha eguali. Puoi conoscere due incantesimi in più (pur rispettando i vincoli di massimo livello copiabile sul Tomo).

L'Abilità può essere presa più volte ed il totale deve essere pari o inferiore a CM/4.

\smallskip\noindent\rule{\linewidth}{2pt} \index[Abilita]{Scacciare i non morti}\hypertarget{Scacciare i non morti}{}\medskip\noindent{\textbf{Scacciare i non morti}}\pdfbookmark[3]{Scacciare i non morti}{Scacciare i non morti}
\noindent
\begin{description}[noitemsep, topsep=0pt, parsep=0pt, partopsep=0pt, leftmargin=0cm, labelwidth=2.5cm]
    \item[\textbf{Requisito}]: Somma Tratti comune 2, essere Devoto o Seguace
    \item[\textbf{Tiri Salvezza}]: +2 Volontà, +1 Tempra
    \item[\textbf{Caratteristica}]: Carisma o Saggezza
\end{description}

Concentrandoti sulla potenza del tuo Patrono convogli l'energia positiva ed allontani o distruggi i non-morti.

Tira 1d6 + somma dei Tratti in comune con il Patrono, questo totale è il tuo Potere Divino.

Partendo dai non morti più deboli intorno a te, nel raggio di 9 metri, controlla il punteggio del Potere Divino ed il Grado di Sfida del non morto.

Se il Potere Divino è almeno il doppio del Grado di Sfida, il non-morto viene distrutto e si sottrae il doppio del Grado di Sfida dal valore del Potere Divino.

Se il non morto non viene distrutto allora esegue un Tiro Salvezza su Volontà a DC pari a 10 + Potere Divino per resistere allo scaccio. Se il Tiro Salvezza fallisce il non morto è scacciato, se riesce non viene influenzato. Sia che riesca il Tiro Salvezza o meno sottrai il Grado di Sfida dal punteggio del Potere Divino prima di passare a verificare un nuovo non morto.

L'Abilità è usabile un numero di volte al giorno pari alla Saggezza ma un non morto può essere influenzato solo una volta al giorno dal tuo effetto.

Un non-morto che viene scacciato è sotto \hyperlink{condizionepaura}{Paura} per 1d4 round, un non-morto distrutto viene ridotto in polvere ed energia divina.

Un Devoto di Sixiser al posto di scacciare può dominare il non morto per 2d4 round, oppure per oppure per 1 ora reale al posto di distruggerlo.

Un Devoto di Thaft ottiene +1d6 al Potere Divino.

\begin{center}
	\includegraphics[width=0.6\linewidth]{immagini/turning-undead-six.png}
\end{center}

\smallskip\noindent\rule{\linewidth}{2pt} \index[Abilita]{Schivare trappole}\hypertarget{Schivare trappole}{}\medskip\noindent{\textbf{Schivare trappole}}\pdfbookmark[3]{Schivare trappole}{Schivare trappole}
\noindent
\begin{description}[noitemsep, topsep=0pt, parsep=0pt, partopsep=0pt, leftmargin=0cm, labelwidth=2.5cm]
    \item[\textbf{Requisito}]: Destrezza 2
    \item[\textbf{Tiri Salvezza}]: +2 Riflessi, +1 Tempra
    \item[\textbf{Caratteristica}]: Destrezza o Intelligenza
\end{description}

La \textbf{prima volta} che prendi l'Abilità ottieni un bonus di +1d6 al Tiro Salvezza per evitare l'effetto delle trappole.

La \textbf{seconda volta} che prendi l'Abilità, requisito Competenza Armi 5, anche se la trappola non concede Tiro Salvezza la tua naturale propensione ad evitare i danni ti concede un Tiro Salvezza su Riflessi per dimezzare i danni.

E' anche possibile usare questa Abilità, usa una Reazione, per evitare Attacco furtivo (TS Riflessi superiore a Tiro Colpire avversario).

%La \textbf{terza volta} che prendi l'Abilità requisiti Competenza Armi 9, il Tiro Salvezza se riuscito ti permette di evitare qualsiasi effetto della trappola, se fisicamente possibile.

\smallskip\noindent\rule{\linewidth}{2pt} \index[Abilita]{Schivata prodigiosa}\hypertarget{Schivata prodigiosa}{}\medskip\noindent{\textbf{Schivata prodigiosa}}\pdfbookmark[3]{Schivata prodigiosa}{Schivata prodigiosa}
\noindent
\begin{description}[noitemsep, topsep=0pt, parsep=0pt, partopsep=0pt, leftmargin=0cm, labelwidth=2.5cm]
    \item[\textbf{Requisito}]: Destrezza 3
    \item[\textbf{Tiri Salvezza}]: +2 Riflessi
    \item[\textbf{Caratteristica}]: Destrezza o Saggezza
\end{description}

La \textbf{prima volta} che prendi questa Abilità come Reazione ad una Azione di attacco avversaria puoi aggiungere +1 alla tua Difesa. Puoi usare l'Abilità fino a 3 volte al giorno.

La \textbf{seconda volta} che prendi l'Abilità, requisito Competenza Armi 4, un avversario non prende il bonus al colpire da fiancheggiamento contro di te.

La \textbf{terza volta} che prendi l'Abilità, requisito Competenza Armi 8, Destrezza 4, fino a due avversari non prendono il bonus al colpire dato fa fiancheggiamento.

Puoi usare l'Abilità anche dopo che si è saputo di quanto si è stati colpiti.

\smallskip\noindent\rule{\linewidth}{2pt} \index[Abilita]{Seconda pelle}\hypertarget{Seconda pelle}{}\medskip\noindent{\textbf{Seconda pelle}}\pdfbookmark[3]{Seconda pelle}{Seconda pelle}
\noindent
\begin{description}[noitemsep, topsep=0pt, parsep=0pt, partopsep=0pt, leftmargin=0cm, labelwidth=2.5cm]
    \item[\textbf{Requisito}]: Competenza Armi 1
    \item[\textbf{Tiri Salvezza}]: +2 Tempra
    \item[\textbf{Caratteristica}]: Costituzione o Forza
\end{description}

Il costante utilizzo dell'armatura ti permette di indossarle senza particolari penalità.

La \textbf{prima volta} che prendi questa Abilità le penalità alle prove di competenza di Base dato dall'armatura diminuisce di 1.

La \textbf{seconda volta} che si prende questa Abilità, requisito Competenza Armi 6, la penalità alle prove di competenza diminuisce di ulteriori 1. La penalità al movimento diminuisce di 1 metro. Puoi dormire in armature medie senza essere affaticato.

La \textbf{terza volta} che si prende questa Abilità, requisito Competenza Armi 11, la penalità alle prove di competenza diminuisce di ulteriori 1. La penalità al movimento diminuisce di un ulteriore 1 metro. Puoi dormire in armature pesanti senza essere affaticato.

\smallskip\noindent\rule{\linewidth}{2pt} \index[Abilita]{Segugio}\hypertarget{Segugio}{}\medskip\noindent{\textbf{Segugio}}\pdfbookmark[3]{Segugio}{Segugio}
\noindent
\begin{description}[noitemsep, topsep=0pt, parsep=0pt, partopsep=0pt, leftmargin=0cm, labelwidth=2.5cm]
    \item[\textbf{Requisito}]: Intelligenza 1, Saggezza 1, Competenza Armi 1
    \item[\textbf{Tiri Salvezza}]: +1 Riflessi, +1 Volontà
    \item[\textbf{Caratteristica}]: Saggezza o Intelligenza
\end{description}

Hai un talento naturale per seguire le persone

La \textbf{prima volta} che prendi questa Abilità con due Azioni ti focalizzi su un target che puoi vedere e finché lo vedi rimani focalizzato. Le tue Azioni che coinvolgono quel target hanno un +1 di bonus. Mantenere la focalizzazione costa 1 Azione per round.

La \textbf{seconda} volta che prendi questa Abilità, requisito Competenza Armi 10, Saggezza 2, il bonus sale a +2.

La \textbf{terza} volta che prendi questa Abilità, requisito Competenza Armi 16, Saggezza 3, il bonus sale a +3.

Il bonus può essere usato al Tiro per Colpire, Tiri Salvezza causati dall'avversario e prove di competenza di Base, non al danno.

\smallskip\noindent\rule{\linewidth}{2pt} \index[Abilita]{Senza Traccia}\hypertarget{Senza Traccia}{}\medskip\noindent{\textbf{Senza Traccia}}\pdfbookmark[3]{Senza Traccia}{Senza Traccia}
\noindent
\begin{description}[noitemsep, topsep=0pt, parsep=0pt, partopsep=0pt, leftmargin=0cm, labelwidth=2.5cm]
    \item[\textbf{Requisito}]: Passo Sicuro
    \item[\textbf{Tiri Salvezza}]: +2 Volontà, +1 Riflessi
    \item[\textbf{Caratteristica}]: Saggezza o Destrezza
\end{description}

La capacità di non lasciare impronte nell'ambiente scelto. Ogni volta che prendi questa Abilità puoi scegliere un ambiente diverso (vedi Abilità \hyperlink{passosicuro}{Passo Sicuro} di cui hai preso l'Abilità. La difficoltà della prova di Seguire Tracce per inseguirti aumentata di 10.

\smallskip\noindent\rule{\linewidth}{2pt} \index[Abilita]{Sifone Nero}\hypertarget{Sifone Nero}{}\medskip\noindent{\textbf{Sifone Nero}}\pdfbookmark[3]{Sifone Nero}{Sifone Nero}
\noindent
\begin{description}[noitemsep, topsep=0pt, parsep=0pt, partopsep=0pt, leftmargin=0cm, labelwidth=2.5cm]
    \item[\textbf{Requisito}]: Competenza magia 6, Seguace o Devoto di Tazher, punti Tratto in comune 6
    \item[\textbf{Tiri Salvezza}]: +1 Tempra, +2 Volontà
    \item[\textbf{Caratteristica}]: Modificatore di caratteristica per incantesimi o a scelta
\end{description}

Quando lanci un incantesimo che abbia durata istantanea e che causi danno ai Punti Ferita ad uno o più soggetti, aumentando di metà, arrotondato per eccesso, i Punti Magia usati nell'incantesimo, recuperi un ammontare di Punti Ferita pari a metà della creatura che ne ha persi di più.

Il tempo di lancio dell'incantesimo aumenta a 3 Azioni.

\smallskip\noindent\rule{\linewidth}{2pt} \index[Abilita]{Sfortunato}\hypertarget{Sfortunato}{}\medskip\noindent{\textbf{Sfortunato}}\pdfbookmark[3]{Sfortunato}{Sfortunato}
\noindent
\begin{description}[noitemsep, topsep=0pt, parsep=0pt, partopsep=0pt, leftmargin=0cm, labelwidth=2.5cm]
    \item[\textbf{Requisito}]: Fortunato, almeno 6 punti nella somma dei Tratti
    \item[\textbf{Tiri Salvezza}]: +1 Tempra, +1 Volontà
    \item[\textbf{Caratteristica}]: una caratteristica a scelta
\end{description}

Una volta al giorno puoi trasformare un 6 tirato dal Narratore (Tiri per Colpire, Prove Competenze, Tiri Salvezza) in un 1. L'Abilità si può dichiarare una volta venuto a sapere del tiro effettuato.

\smallskip\noindent\rule{\linewidth}{2pt} \index[Abilita]{Spara e Scappa}\hypertarget{Spara e Scappa}{}\medskip\noindent{\textbf{Spara e Scappa}}\pdfbookmark[3]{Spara e Scappa}{Spara e Scappa}
\noindent
\begin{description}[noitemsep, topsep=0pt, parsep=0pt, partopsep=0pt, leftmargin=0cm, labelwidth=2.5cm]
    \item[\textbf{Requisito}]: Lista Balestre 3
    \item[\textbf{Tiri Salvezza}]: +1 Tempra, +1 Riflessi
    \item[\textbf{Caratteristica}]: Destrezza o Carisma
\end{description}

Mentre esegui una Azione di Movimento puoi ridurre di 1 Azione il tempo di caricamento della tua balestra. In caso di Balestre Leggere od a una mano puoi puoi quindi ricaricarla mentre ti muovi, in caso di Balestre pesanti riduci di 1 Azione il tempo di caricamento.

\smallskip\noindent\rule{\linewidth}{2pt} \index[Abilita]{Specialista}\hypertarget{Specialista}{}\medskip\noindent{\textbf{Specialista}}\pdfbookmark[3]{Specialista}{Specialista}
\noindent
\begin{description}[noitemsep, topsep=0pt, parsep=0pt, partopsep=0pt, leftmargin=0cm, labelwidth=2.5cm]
    \item[\textbf{Requisito}]: Competenza Magica 3
    \item[\textbf{Tiri Salvezza}]: +2 Tempra
    \item[\textbf{Caratteristica}]: Modificatore di caratteristica per incantesimi o a scelta
\end{description}

Scegli un incantesimo che conosci, i Punti Magia spesi per lanciare questo incantesimo diminuiscono di 1.

L'Abilità può essere presa più volte su incantesimi ogni volta diversi.

\smallskip\noindent\rule{\linewidth}{2pt} \index[Abilita]{Stai giù!}\hypertarget{Stai giù!}{}\medskip\noindent{\textbf{Stai giù!}}\pdfbookmark[3]{Stai giù!}{Stai giù!}
\noindent
\begin{description}[noitemsep, topsep=0pt, parsep=0pt, partopsep=0pt, leftmargin=0cm, labelwidth=2.5cm]
    \item[\textbf{Requisito}]: Competenza Armi 3
    \item[\textbf{Tiri Salvezza}]: +2 Tempra, +1 Volontà
    \item[\textbf{Caratteristica}]: Forza o Carisma
\end{description}

La \textbf{prima volta} che prendi questa Abilità quando un tuo attacco causa due tiri critici su un avversario, la forza del colpo è tale da metterlo prono. L'avversario deve fare un Tiro Salvezza Tempra (DC pari all'ultimo Tiro per Colpire con l'arma che ha causato l'ultimo tiro critico) o cadere prono. L'Abilità funziona su creature di taglia pari o inferiore a quella del personaggio.

La \textbf{seconda volta} che prendi l'Abilità puoi influenzare anche creature di una taglia superiore.

La \textbf{terza volta} che prendi l'Abilità puoi influenzare anche creature di due taglie superiori. Il grado 3 non è cumulabile con il grado 2.

\smallskip\noindent\rule{\linewidth}{2pt} \index[Abilita]{Tattico}\hypertarget{Tattico}{}\medskip\noindent{\textbf{Tattico}}\pdfbookmark[3]{Tattico}{Tattico}
\noindent
\begin{description}[noitemsep, topsep=0pt, parsep=0pt, partopsep=0pt, leftmargin=0cm, labelwidth=2.5cm]
    \item[\textbf{Requisito}]: Competenza Armi 1, Intelligenza 1
    \item[\textbf{Tiri Salvezza}]: +1 Tempra , +1 Volontà
    \item[\textbf{Caratteristica}]: Intelligenza o Carisma
\end{description}

Hai una capacità quasi istintiva di gestire e prevedere l'esito dei combattimenti. Usare l'Abilità costa 1 Azione.

Scambi l'esito della iniziativa tra due creature di cui almeno una sia una tua alleata.

Se nessuno dei due bersagli ha ancora agito allora l'effetto della tua Abilità si attiva subito. Se almeno una delle due ha già agito allora la tua Abilità prende effetto il round successivo.

L'effetto dura solo un round, poi i bersagli tornano alle loro iniziative precedenti.

La \textbf{prima volta} che prendi questa Abilità puoi influire su una coppia di bersagli che siano in mischia tra loro.

La \textbf{seconda volta} che prendi questa Abilità, requisito Intelligenza 2, Competenza Armi 6, puoi scambiare l'iniziativa su due coppie che siano reciprocamente in mischia tra loro.

\smallskip\noindent\rule{\linewidth}{2pt} \index[Abilita]{Tempesta di Furia}\hypertarget{Tempesta di Furia}{}\medskip\noindent{\textbf{Tempesta di Furia}}\pdfbookmark[3]{Tempesta di Furia}{Tempesta di Furia}
\noindent
\begin{description}[noitemsep, topsep=0pt, parsep=0pt, partopsep=0pt, leftmargin=0cm, labelwidth=2.5cm]
    \item[\textbf{Requisito}]: Lista Pugno Vuoto 7, Destrezza 1, Forza 1
    \item[\textbf{Tiri Salvezza}]: +2 Riflessi, +1 Volontà
    \item[\textbf{Caratteristica}]: Destrezza o Forza
\end{description}

Quando usi questa Abilità puoi dichiarare di usare la Tempesta di Furia come tua unica Azione (3 Azioni).

Fai un unico Tiro per Colpire con -1d6 e se colpisci con il tuo attacco naturale causi un numero di danno critico pari a Lista Pugno Vuoto/4.

\smallskip\noindent\rule{\linewidth}{2pt} \index[Abilita]{Testa cava}\hypertarget{Testa cava}{}\medskip\noindent{\textbf{Testa cava}}\pdfbookmark[3]{Testa cava}{Testa cava}
\noindent
\begin{description}[noitemsep, topsep=0pt, parsep=0pt, partopsep=0pt, leftmargin=0cm, labelwidth=2.5cm]
    \item[\textbf{Requisito}]: Lista Balestre 4
    \item[\textbf{Tiri Salvezza}]: +2 Tempra
    \item[\textbf{Caratteristica}]: Destrezza o Intelligenza
\end{description}

Riesci a imprimere un mortale effetto ai tuoi proiettili.

Il tuo dardo da balestra aumenta di una taglia di danno.

\smallskip\noindent\rule{\linewidth}{2pt} \index[Abilita]{Tiro Preciso}\hypertarget{Tiro Preciso}{}\medskip\noindent{\textbf{Tiro Preciso}}\pdfbookmark[3]{Tiro Preciso}{Tiro Preciso}
\noindent
\begin{description}[noitemsep, topsep=0pt, parsep=0pt, partopsep=0pt, leftmargin=0cm, labelwidth=2.5cm]
    \item[\textbf{Requisito}]: Destrezza 3, Competenza Armi 1
    \item[\textbf{Tiri Salvezza}]: +2 Riflessi
    \item[\textbf{Caratteristica}]: Destrezza o Saggezza
\end{description}

Da vicino sai colpire dove fa male.

Guadagni +1 al Tiro per Colpire ed al danno, con armi da lancio quando il bersaglio è entro 9 metri.

%\begin{center}
%\includegraphics[width=0.7\linewidth]{immagini/kenilguerriero.png}
%\end{center}

\smallskip\noindent\rule{\linewidth}{2pt} \index[Abilita]{Tiro Rapido}\hypertarget{Tiro Rapido}{}\medskip\noindent{\textbf{Tiro Rapido}}\pdfbookmark[3]{Tiro Rapido}{Tiro Rapido}
\noindent\label{Tiro Rapido}
\begin{description}[noitemsep, topsep=0pt, parsep=0pt, partopsep=0pt, leftmargin=0cm, labelwidth=2.5cm]
    \item[\textbf{Requisito}]: Destrezza 3, Tiro Preciso, Competenza Armi 2
    \item[\textbf{Tiri Salvezza}]: +2 Riflessi
    \item[\textbf{Caratteristica}]: Destrezza o Intelligenza
\end{description}

Non hai rivale nella precisione con cui scagli i tuoi proiettili.

Quando usi arco, balestre o lanci un arma le penalità per l'attacco multiplo sono inferiori.

Ogni proiettile lanciato oltre il primo prende un -4 al Tiro per Colpire cumulativo (e non il -5).

Il primo colpo ha un Tiro per Colpire normale, il secondo ha un -4, il terzo un -8 ...

\smallskip\noindent\rule{\linewidth}{2pt} \index[Abilita]{Toccata e fuga}\hypertarget{Toccata e fuga}{}\medskip\noindent{\textbf{Toccata e fuga}}\pdfbookmark[3]{Toccata e fuga}{Toccata e fuga}
\noindent
\begin{description}[noitemsep, topsep=0pt, parsep=0pt, partopsep=0pt, leftmargin=0cm, labelwidth=2.5cm]
    \item[\textbf{Requisito}]: Destrezza 1, Competenza Armi 1
    \item[\textbf{Tiri Salvezza}]: +2 Riflessi
    \item[\textbf{Caratteristica}]: Destrezza o Intelligenza
\end{description}

Se nel round esegui almeno un attacco questi hanno una penalità base di -5 e puoi effettuare una Azione di movimento in più. Non è possibile eseguire in questa maniera più di una Azione di Movimento bonus. Costa una Azione Immediata.

\smallskip\noindent\rule{\linewidth}{2pt} \index[Abilita]{Tutt'uno con la magia}\hypertarget{Tutt'uno con la magia}{}\medskip\noindent{\textbf{Tutt'uno con la magia}}\pdfbookmark[3]{Tutt'uno con la magia}{Tutt'uno con la magia}\label{Tutt'uno con la magia}
\noindent
\begin{description}[noitemsep, topsep=0pt, parsep=0pt, partopsep=0pt, leftmargin=0cm, labelwidth=2.5cm]
    \item[\textbf{Requisito}]: Adepto della Magia
    \item[\textbf{Tiri Salvezza}]: +1 in due Tiri Salvezza a propria scelta
    \item[\textbf{Caratteristica}]: Modificatore di caratteristica per incantesimi o a scelta
\end{description}

Ogni volta che prendi l'Abilità Tutt'uno con la magia devi stabilire a che Caratteristica si collega.
La tua Caratteristica ha un +1 al valore per determinare gli effetti dell'incantesimo, Punti Magia e livello massimo di incantesimo lanciabile.

\smallskip\noindent\rule{\linewidth}{2pt} \index[Abilita]{Un braccio, un arma}\hypertarget{Un braccio, un arma}{}\medskip\noindent{\textbf{Un braccio, un arma}}\pdfbookmark[3]{Un braccio, un arma}{Un braccio, un arma}
\noindent
\begin{description}[noitemsep, topsep=0pt, parsep=0pt, partopsep=0pt, leftmargin=0cm, labelwidth=2.5cm]
    \item[\textbf{Requisito}]: Competenza Armi 2
    \item[\textbf{Tiri Salvezza}]: +1 Tempra, +1 Volontà
    \item[\textbf{Caratteristica}]: Forza o Costituzione
\end{description}

Scegli una Lista d'Armi. Il danno da Forza applicato dalle armi di quella lista aumenta di 1.

L’Abilità può essere presa più volte, con almeno CA 5,9,13.

Se prendi \textbf{4 volte} questa Abilità sulla stessa Lista d'Armi il bonus al danno si riduce a +2 ma tiri due volte il danno e scegli il risultato migliore. Non si applica sull'esplosione del danno o sul danno critico.

\smallskip\noindent\rule{\linewidth}{2pt} \index[Abilita]{Un colpo un morto}\hypertarget{Un colpo un morto}{}\medskip\noindent{\textbf{Un colpo un morto}}\pdfbookmark[3]{Un colpo un morto}{Un colpo un morto}
\noindent
\begin{description}[noitemsep, topsep=0pt, parsep=0pt, partopsep=0pt, leftmargin=0cm, labelwidth=2.5cm]
    \item[\textbf{Requisito}]: Competenza Magica 1, Adepto della Magia 1
    \item[\textbf{Tiri Salvezza}]: +2 Riflessi
    \item[\textbf{Caratteristica}]: Modificatore di caratteristica per incantesimi o a scelta
\end{description}

La \textbf{prima volta} che prendi questa Abilità ottieni un +1 ai Tiri per Colpire agli incantesimi che richiedono un Tiro per Colpire.

La \textbf{seconda volta} il bonus al Tiro per Colpire per Incantesimi diventa +1 per ogni volta che hai preso l'Abilità Adepto della Magia. Non si cumula con il bonus preso la volta precedente.

\smallskip\noindent\rule{\linewidth}{2pt} \index[Abilita]{Un solo corpo, una sola mente, un solo spirito}\hypertarget{Un solo corpo, una sola mente, un solo spirito}{}\medskip\noindent{\textbf{Un solo corpo, una sola mente, un solo spirito}}\pdfbookmark[3]{Un solo corpo, una sola mente, un solo spirito}{Un solo corpo, una sola mente, un solo spirito}
\noindent
\begin{description}[noitemsep, topsep=0pt, parsep=0pt, partopsep=0pt, leftmargin=0cm, labelwidth=2.5cm]
    \item[\textbf{Requisito}]: nessuno
    \item[\textbf{Tiri Salvezza}]:  +1 a scelta
    \item[\textbf{Caratteristica}]: nessuna
\end{description}

Assegnate un punto a Competenza Armi oppure Competenza Magica. Questa Abilità può essere presa al massimo 2 volte.

\smallskip\noindent\rule{\linewidth}{2pt} \index[Abilita]{Uno con l'arco}\hypertarget{Uno con l'arco}{}\medskip\noindent{\textbf{Uno con l'arco}}\pdfbookmark[3]{Uno con l'arco}{Uno con l'arco}\label{Uno con l'arco}
\noindent
\begin{description}[noitemsep, topsep=0pt, parsep=0pt, partopsep=0pt, leftmargin=0cm, labelwidth=2.5cm]
    \item[\textbf{Requisito}]: Competenza Armi 4
    \item[\textbf{Tiri Salvezza}]: +1 Tempra, +1 Riflessi
    \item[\textbf{Caratteristica}]: Destrezza o Costituzione
\end{description}

A discapito del nome questa Abilità si applica a tutte le armi che lanciano proiettili.

La \textbf{prima volta} che prendi questa Abilità la penalità per scagliare proiettili mentre si usa un Arma da Lancio sotto minaccia diventa -2.

La \textbf{seconda volta} che prendi questa Abilità, Destrezza 3 e Competenza Armi 7, le penalità vengono annullate.

\smallskip\noindent\rule{\linewidth}{2pt} \index[Abilita]{Un solo credo}\hypertarget{Un solo credo}{}\medskip\noindent{\textbf{Un solo credo}}\pdfbookmark[3]{Un solo credo}{Un solo credo}\label{Un solo credo}
\noindent
\begin{description}[noitemsep, topsep=0pt, parsep=0pt, partopsep=0pt, leftmargin=0cm, labelwidth=2.5cm]
    \item[\textbf{Requisito}]: Competenza Magica 2
    \item[\textbf{Tiri Salvezza}]: +1 Volontà, +1 Tempra
    \item[\textbf{Caratteristica}]: Modificatore di caratteristica per incantesimi
\end{description}

Il personaggio dedica la propria vita allo studio e perfezionamento di una sola Lista di Magia.

La \textbf{prima volta} che prendi questa Abilità devi scegliere una Lista di Magia \emph{preferita} e 3 \emph{opposte}.

Nella lista \emph{preferita} il costo di lancio degli incantesimi diminuisce di 1 rimanendo un costo minimo di 1. Per le 3 Liste di Magia \emph{opposte} il costo di lancio degli incantesimi aumenta di 1.

La \textbf{seconda volta} che prendi questa Abilità, requisito Competenza Magica 6, nella Lista di Magia \emph{preferita} le Prove di Magia vengono effettuate con 1d6 aggiuntivo e puoi scartare 1 dado dalla stessa.

La \textbf{terza volta} che prendi questa Abilità, requisito Competenza Magica 11, nella Lista di Magia \emph{preferita} quando fai una Prova di Magia conti un 6 in più rispetto a quanti tirati.

La \textbf{quarta volta} che prendi questa Abilità, requisito Competenza Magica 14, nella Lista di Magia \emph{preferita} puoi ritirare una volta la Prova di Magia in caso di fallimento critico.

La \textbf{quinta volta} che prendi questa Abilità, requisito Competenza Magica 17, nella Lista di Magia \emph{preferita} ogni qual volta devi tirare una Prova di Magia puoi non tirare e considerare di aver fatto un Successo Critico Magico.

La \textbf{sesta volta} che prendi questa Abilità, requisito Competenza Magica 20, nella Lista di Magia \emph{preferita} gli incantesimi inferiori al 4 livello non costano Punti Magia nella formulazione base.

\medskip

\textbf{Regole}:

\smallskip

\begin{itemize}[leftmargin=*] \setlength{\itemsep}{0pt}
\item Ogni volta che l'Abilità viene presa, oltre la prima, si devono selezionare due nuove Lista di Magia \emph{opposta} e il livello massimo lanciabile di incantesimi per tutte le Liste di magia \emph{opposte} diminuisce di 2.  La Lista di Magia Universale non è sceglibile tra le \emph{opposte}.

\item L'Abilità \emph{Un solo credo} non può essere presa assieme a: \hyperlink{figliounico}{Figlio Unico}, \hyperlink{magiepotenti}{Magie Potenti}, \hyperlink{specialista}{Specialista}.

\item Se usi l'Abilità \emph{Un solo credo} non puoi usare usare le \hyperlink{abilitadilista}{Abilità di Lista} (pag. \pageref{abilitadilista}).
\end{itemize}

\smallskip\noindent\rule{\linewidth}{2pt} \index[Abilita]{Volonta' Ferrea}\hypertarget{Volonta' Ferrea}{}\medskip\noindent{\textbf{Volonta' Ferrea}}\pdfbookmark[3]{Volonta' Ferrea}{Volonta' Ferrea}
\noindent
\begin{description}[noitemsep, topsep=0pt, parsep=0pt, partopsep=0pt, leftmargin=0cm, labelwidth=2.5cm]
    \item[\textbf{Requisito}]: Saggezza 0
    \item[\textbf{Tiri Salvezza}]: nessuno
    \item[\textbf{Caratteristica}]: Saggezza o Carisma
\end{description}

Nel tempo hai allenato la tua volontà per resistere a qualsiasi debolezza e paura.

La \textbf{prima volta} prendi questa Abilità ottieni un bonus di +2 ai Tiri Salvezza su Volontà. Il bonus è cumulativo, +2 la prima volta, +1 la \textbf{seconda volta}, +1 la \textbf{terza} volta.

La \textbf{quarta volta} che prendi questa Abilità puoi decidere di riuscire automaticamente in un Tiro Salvezza su Volontà una volta al giorno prima di aver tirato i dadi.

%Superumano: una serie di abilita' a scelta, quando poi prendi due volte sbocchi altre, quando 3 volte sblocchi altre..
% 1 Duro da soggiogare : +2 Tiro Salvezza su Volontà su incantesimi della Lista Ammaliamento.
% 1 Arcobaleno : sei un artista. Le tue ditaspontaneamente producono colore
% 1 Consumi ridotti : bevi e mangi la metà di unuomo normale. Sei sotto peso
% 1 Super piastrine : Riduci il danno da Sanguinamento di 1 a fine di ogni round
% 1 Direzione Assoluta : sai sempre dove è il nord magnetico. Hai un +1d6 alle prove di orientamento.
% 1 Il fegato non si conta 10: puoi bere tanto e non ti ubriachi
% 1 Mano Piede palmata 5: +1d6 alle prove di nuotare
% 2 Lento e Fermo 5: Sei eccezionalmente stabile sui tuoi piedi. Non puoi essere mosso o sollevato se non da una creatura di 2 taglie superiori
% 2 Empatia Animale 10: +1d6 alle prove per gestire gli animali (anche selvaggi).
% 2 Polmoni di ferro 5: puoi trattenere il respiro 2*Costituzione minuti (minimo 2 minuti)
% 2 Magnetico 5-10: sprigioni luce quando vuoi. per fortuna non letteralmente. ±1/2 alle prove basate sul Carisma.
% 4 Vedere l’invisibile 15: Meglio la vista a raggi X
% 5 Rigenerazione 30: +1 Punti Ferita per Turno (non rigeneri arti)
% 5 Senso delle vibrazioni (Senso Tellurico) 30: tutto fa tremare un poco la terra, o quasi, raggio di 18 metri intorno a te.
% 5 Anfibio : puoi respirare sia sott’acqua che l’aria
% 6 Parlare con gli animali 20: scegli una famiglia (ovini, marsupiali, caviette..)
% 7 Parlare con le piante 25: ho sempre voluto parlare con le zucchine
% 8 Rigenerazione veloce 40: +1 Punti Ferita per round (non rigeneri arti). Muori se distruggono il tuo corpo (o non rimane che cenere
% 6 Tocco gelido 10: Toccando un morto (entro 1 giorno per livello) puoi vedere e sentire cosa è successo nel suo ultimo round di vita.

\end{multicols}

\vfill

\begin{center}

\includegraphics[width=0.9\linewidth]{immagini/Granblue.Fantasy.full.2108782.png}

%%\filltopageendgraphics[width=0.7\linewidth]{immagini/Granblue.Fantasy.full.2108782.png}

\emph{Attorno al fuoco, raccontando la giornata trascorsa.}
\end{center}

\pagebreak

\subsection{Raggruppamento Abilita' per Stile}

Per facilitare la transizione da chi viene da altri giochi di ruolo con classi sono qui suddivise le Abilità per le classi più canoniche. Sono chiaramente solo indicazioni, in OBSS il personaggio può essere costruito come meglio si preferisce e come la storia che vive lo sta istruendo.

\begin{multicols}{3}

{\small

\begin{flushleft}

\titlespacing*{\subsubsection}{0pt}{0.5em}{0.5em}\subsubsection*{Guerriero}

\hyperlink{Allungo}{Allungo}\\
\hyperlink{Arma Focalizzata}{Arma Focalizzata}\\
\hyperlink{Artista dell'Arma}{Artista dell'Arma}\\
\hyperlink{Attacco Turbinante}{Attacco Turbinante}\\
\hyperlink{Colpi Poderosi}{Colpi Poderosi}\\
\hyperlink{Combattere alla Cieca}{Combattere alla Cieca}\\
\hyperlink{Combattimento con due armi}{Combattimento con due armi}\\
\hyperlink{Difesa pronta}{Difesa pronta}\\
\hyperlink{Flagello Danzante}{Flagello Danzante}\\
\hyperlink{Iaijutsu}{Iaijutsu}\\
\hyperlink{Iniziativa migliorata}{Iniziativa migliorata}\\
\hyperlink{La mia pelle}{La mia pelle}\\
\hyperlink{Lesto}{Lesto}\\
\hyperlink{Parata}{Parata}\\
\hyperlink{Pelle Coriacea}{Pelle Coriacea}\\
\hyperlink{Primo Sangue}{Primo Sangue}\\
\hyperlink{Proseguire}{Proseguire}\\
\hyperlink{Questa è la mia arma!}{Questa è la mia arma!}\\
\hyperlink{Resistenza della pietra}{Resistenza della pietra}\\
\hyperlink{Riflessi fulminei}{Riflessi fulminei}\\
\hyperlink{Robusto}{Robusto}\\
\hyperlink{Seconda pelle}{Seconda pelle}\\
\hyperlink{Stai giù!}{Stai giù!}\\
\hyperlink{Un braccio, un arma}{Un braccio, un arma}

\hyperlink{Armato}{Armato}\\
\hyperlink{Colpo Mortale}{Colpo Mortale}\\
\hyperlink{Conoscenza istintiva}{Conoscenza istintiva}\\
\hyperlink{Daredevil}{Daredevil}\\
\hyperlink{Duro a morire}{Duro a morire}\\
\hyperlink{Ferocia}{Ferocia}\\
\hyperlink{Forgiato nella furia}{Forgiato nella furia}\\
\hyperlink{Furia}{Furia}\\
\hyperlink{Ho detto CADI!}{Ho detto CADI!}\\
\hyperlink{La mia morte la tua morte}{La mia morte la tua morte}\\
\hyperlink{La mia Testa è più Dura}{La mia Testa è più Dura}\\
\hyperlink{Montagna umana}{Montagna umana}\\
\hyperlink{Pelle Coriacea}{Pelle Coriacea}\\
\hyperlink{Persona veramente malvagia}{Persona veramente malvagia}\\
\hyperlink{più sono grossi più fanno rumore quando cadono}{più sono grossi più fanno rumore quando cadono}\\
\hyperlink{Primo Sangue}{Primo Sangue}\\
\hyperlink{Recupero}{Recupero}\\
\hyperlink{abRinoceronte}{Rinoceronte}\\
\hyperlink{Un braccio, un arma}{Un braccio, un arma}\\
\hyperlink{Volonta' Ferrea}{Volonta' Ferrea}

\titlespacing*{\subsubsection}{0pt}{0.5em}{0.5em}\subsubsection*{Ladro}

\hyperlink{Colpo Furtivo}{Colpo Furtivo}\\
\hyperlink{Colpo Indebolente}{Colpo Indebolente}\\
\hyperlink{Colpo Paralizzante}{Colpo Paralizzante}\\
\hyperlink{Estrazione rapida}{Estrazione rapida}\\
\hyperlink{Difesa pronta}{Difesa pronta}\\
\hyperlink{Fare Infuriare}{Fare Infuriare}\\
\hyperlink{Freccia chiamata, freccia consegnata}{Freccia chiamata, freccia consegnata}\\
\hyperlink{Giocoliere}{Giocoliere}\\
\hyperlink{Improvvisare}{Improvvisare}\\
\hyperlink{Lesto}{Lesto}\\
\hyperlink{Occhio Clinico}{Occhio Clinico}\\
\hyperlink{Opportunista}{Opportunista}\\
\hyperlink{Passo Felpato}{Passo Felpato}\\
\hyperlink{Percettivo}{Percettivo}\\
\hyperlink{Questo è il mio pugnale}{Questo è il mio pugnale}\\
\hyperlink{Schivare trappole}{Schivare trappole}\\
\hyperlink{Schivata prodigiosa}{Schivata prodigiosa}\\
\hyperlink{Toccata e fuga}{Toccata e fuga}

\titlespacing*{\subsubsection}{0pt}{0.5em}{0.5em}\subsubsection*{Paladino}

\hyperlink{Armatura del Devoto}{Armatura del Devoto}\\
\hyperlink{Il Patrono è la mia Arma}{Il Patrono è la mia Arma}\\
\hyperlink{Imposizione delle mani}{Imposizione delle mani}\\
\hyperlink{Incanalare Energia}{Incanalare Energia}\\
\hyperlink{Lo scudo è mio amico}{Lo scudo è mio amico}\\
\hyperlink{Muro mentale}{Muro mentale}\\
\hyperlink{Rappresaglia}{Rappresaglia}\\

\titlespacing*{\subsubsection}{0pt}{0.5em}{0.5em}\subsubsection*{Bardo}

\hyperlink{Dadi Truccati}{Dadi Truccati}\\
\hyperlink{Danno Coordinato}{Danno Coordinato}\\
\hyperlink{Danza della Lama}{Danza della Lama}\\
\hyperlink{Esperto}{Esperto}\\
\hyperlink{Fare Infuriare}{Fare Infuriare}\\
\hyperlink{Figlio Unico}{Figlio Unico}\\
\hyperlink{Fortunato}{Fortunato}\\
\hyperlink{Guerriero della Magia}{Guerriero della Magia}\\
\hyperlink{Incantatore da Combattimento}{Incantatore da Combattimento}\\
\hyperlink{Infondere Coraggio}{Infondere Coraggio}\\
\hyperlink{Infondere Energia Magica}{Infondere Energia Magica}\\
\hyperlink{Infondere Energia Magica Superiore}{Infondere Energia Magica Superiore}\\
\hyperlink{Infondere Paura}{Infondere Paura}\\
\hyperlink{Litania versatile}{Litania versatile}\\
\hyperlink{Poliglotta}{Poliglotta}\\
\hyperlink{Radici magiche}{Radici magiche}\\
\hyperlink{Tattico}{Tattico}\\
\hyperlink{Sfortunato}{Sfortunato}

\titlespacing*{\subsubsection}{0pt}{0.5em}{0.5em}\subsubsection*{Ranger}

\hyperlink{Combattimento con due armi}{Combattimento con due armi}\\
\hyperlink{Conoscenza istintiva}{Conoscenza istintiva}\\
\hyperlink{Difendere Cavalcatura}{Difendere Cavalcatura}\\
\hyperlink{Doppia porzione}{Doppia porzione}\\
\hyperlink{Freccia chiamata, freccia consegnata}{Freccia chiamata, freccia consegnata}\\
\hyperlink{Occhio di Falco}{Occhio di Falco}\\
\hyperlink{Passo Sicuro}{Passo Sicuro}\\
\hyperlink{abPasso Veloce}{Passo Veloce}\\
\hyperlink{Precisino}{Precisino}\\
\hyperlink{più sono grossi più fanno rumore quando cadono}{più sono grossi più fanno rumore quando cadono}\\
\hyperlink{Segugio}{Segugio}\\
\hyperlink{Senza Traccia}{Senza Traccia}\\
\hyperlink{Spara e Scappa}{Spara e Scappa}\\
\hyperlink{Testa cava}{Testa cava}\\
\hyperlink{Tiro Preciso}{Tiro Preciso}\\
\hyperlink{Tiro Rapido}{Tiro Rapido}\\
\hyperlink{Uno con l'arco}{Uno con l'arco}

\titlespacing*{\subsubsection}{0pt}{0.5em}{0.5em}\subsubsection*{Druido}

\hyperlink{Adepto della Magia}{Adepto della Magia}\\
\hyperlink{Animalia}{Animalia}\\
\hyperlink{Distillare pozioni}{Distillare pozioni}\\
\hyperlink{Elementalista}{Elementalista}\\
\hyperlink{Figlia di Shayalia}{Figlia di Shayalia}\\
\hyperlink{Forma Elementale}{Forma Elementale}\\
\hyperlink{Il Patrono è con me}{Il Patrono è con me}\\
\hyperlink{Sangue Puro}{Sangue Puro}

\titlespacing*{\subsubsection}{0pt}{0.5em}{0.5em}\subsubsection*{Chierico}

\hyperlink{Adepto della Magia}{Adepto della Magia}\\
\hyperlink{Armatura del Devoto}{Armatura del Devoto}\\
\hyperlink{Dattilografo}{Dattilografo}\\
\hyperlink{Fedele}{Fedele}\\
\hyperlink{Guaritore}{Guaritore}\\
\hyperlink{Il Patrono è con me}{Il Patrono è con me}\\
\hyperlink{Il Patrono è la mia Arma}{Il Patrono è la mia Arma}\\
\hyperlink{Imposizione delle mani}{Imposizione delle mani}\\
\hyperlink{Incanalare Energia}{Incanalare Energia}\\
\hyperlink{Potere del Patrono}{Potere del Patrono}\\
\hyperlink{Scacciare i non morti}{Scacciare i non morti}\\
\hyperlink{Specialista}{Specialista}\\
\hyperlink{Tutt'uno con la magia}{Tutt'uno con la magia}

\titlespacing*{\subsubsection}{0pt}{0.5em}{0.5em}\subsubsection*{Mago/Stregone}

\hyperlink{Adepto della Magia}{Adepto della Magia}\\
\hyperlink{Animaletto / Famiglio}{Animaletto / Famiglio}\\
\hyperlink{Batteria Magica}{Batteria Magica}\\
\hyperlink{Batteria Estesa}{Batteria Estesa}\\
\hyperlink{Concentrato}{Concentrato}\\
\hyperlink{Creare Oggetti Magici}{Creare Oggetti Magici}\\
\hyperlink{Dadi Truccati}{Dadi Truccati}\\
\hyperlink{Dattilografo}{Dattilografo}\\
\hyperlink{Decifrare scritti magici}{Decifrare scritti magici}\\
\hyperlink{Elementalista}{Elementalista}\\
\hyperlink{Figlio di Tazher}{Figlio di Tazher}\\
\hyperlink{Incantatore Prudente}{Incantatore Prudente}\\
\hyperlink{Il Patrono è con me}{Il Patrono è con me}\\
\hyperlink{Magie Potenti}{Magie Potenti}\\
\hyperlink{Occhi della magia}{Occhi della magia}\\
\hyperlink{Prodigioso}{Prodigioso}\\
\hyperlink{Robusto}{Robusto}\\
\hyperlink{Sapiente}{Sapiente}\\
\hyperlink{Sifone Nero}{Sifone Nero}\\
\hyperlink{Specialista}{Specialista}\\
\hyperlink{Tutt'uno con la magia}{Tutt'uno con la magia}\\
\hyperlink{Un colpo un morto}{Un colpo un morto}\\
\hyperlink{Un solo credo}{Un solo credo}

\titlespacing*{\subsubsection}{0pt}{0.5em}{0.5em}\subsubsection*{Monaco}

\hyperlink{Ali della Fenice}{Ali della Fenice}\\
\hyperlink{Armatura della Montagna Incantata}{Armatura della Montagna Incantata}\\
\hyperlink{Colpo Psichico}{Colpo Psichico}\\
\hyperlink{Energia Psichica}{Energia Psichica}\\
\hyperlink{Finta Morte}{Finta Morte}\\
\hyperlink{Gru d'Argento}{Gru d'Argento}\\
\hyperlink{Immunita' ai veleni}{Immunita' ai veleni}\\
\hyperlink{Molla}{Molla}\\
\hyperlink{Montagna umana}{Montagna umana}\\
\hyperlink{Muro mentale}{Muro mentale}\\
\hyperlink{abPasso Veloce}{Passo Veloce}\\
\hyperlink{Pugno di Ferro}{Pugno di Ferro}\\
\hyperlink{Pugno Potente}{Pugno Potente}\\
\hyperlink{Raggio Psichico}{Raggio Psichico}\\
\hyperlink{Tempesta di Furia}{Tempesta di Furia}\\
\hyperlink{Volonta' Ferrea}{Volonta' Ferrea}

\end{flushleft}
}

\end{multicols}

\subsection{Esempi di costruzione del personaggio}

Sono qui presentati alcuni esempi di personaggi secondo i canoni standard fantasy, prendeteli come uno spunto per costruire i vostri personaggi. Ricordate di aggiungere il valore della Costituzione per livello ai Punti Ferita.

{\small

	%\begin{tabularx}{\linewidth}{>{\hsize=0.05\hsize}X>{\hsize=0.05\hsize}X>{\hsize=0.13\hsize}X>{\hsize=0.50\hsize}X|>{\hsize=0.05\hsize}X|>{\hsize=0.05\hsize}X>{\hsize=0.05\hsize}X>{\hsize=0.13\hsize}X>{\hsize=0.50\hsize}X}
	%	& \multicolumn{3}{c}{\textbf{Guerriero}} & & \multicolumn{3}{c}{\textbf{Ladro}} & \\
	%	\cmidrule(lr){1-4} \cmidrule(lr){6-9}
	%	\textbf{CA} & \textbf{CM} & \textbf{PF} & \textbf{Abilità} & \textbf{Lv} & \textbf{CA} & \textbf{CM} & \textbf{PF} & \textbf{Abilità} \\
	%	\toprule
	%	1 & 0 & 8+3 & Arma Focalizzata - Colpi poderosi & 1 & 1 & 0 & 8+3 & Persona veramente malvagia - Estrazione rapida \\
	%	2 & 0 & 2d6+6 & Estrazione rapida & 2 & 2 & 0 & 2d6+6 & Opportunista \\
	%	3 & 0 & 3d6+9 & Opportunista & 3 & 3 & 0 & 3d6+9 & Colpo furtivo \\
	%	4 & 0 & 4d6+12 & Pelle Coriacea & 4 & 4 & 0 & 4d6+12 & Passo felpato \\
	%	5 & 0 & 5d6+15 & Proseguire & 5 & 5 & 0 & 5d6+15 & Occhio Clinico \\
	%	6 & 0 & 6d6+18 & Iniziativa migliorata & 6 & 6 & 0 & 6d6+18 & Schivare trappole \\
	%	7 & 0 & 7d6+21 & Questa è la mia arma! & 7 & 7 & 0 & 7d6+21 & Percettivo \\
	%	8 & 0 & 8d6+24 & & 8 & 8 & 0 & 8d6+24 & \\
	%	9 & 0 & 9d6+27 & Proseguire & 9 & 9 & 0 & 9d6+27 & Colpo furtivo \\
	%	10 & 0 & 10d6+30 & Riflessi fulminei & 10 & 10 & 0 & 10d6+30 & Esperto \\
	%	11 & 0 & 11d6+33 & & 11 & 11 & 0 & 11d6+33 & \\
	%	12 & 0 & 12d6+36 & Pelle Coriacea & 12 & 12 & 0 & 12d6+36 & Colpo furtivo \\
	%	13 & 0 & 13d6+39 & Seconda pelle & 13 & 13 & 0 & 13d6+39 & Schivata prodigiosa \\
	%	14 & 0 & 14d6+42 & & 14 & 14 & 0 & 14d6+42 & \\
	%	15 & 0 & 15d6+45 & Seconda pelle & 15 & 15 & 0 & 15d6+45 & Colpo furtivo \\
	%	16 & 0 & 16d6+48 & Combattere alla Cieca & 16 & 16 & 0 & 16d6+48 & Improvvisare \\
	%	17 & 0 & 17d6+51 & & 17 & 17 & 0 & 17d6+51 & \\
	%	18 & 0 & 18d6+54 & A scelta & 18 & 18 & 0 & 18d6+54 & A scelta \\
	%	19 & 0 & 19d6+57 & & 19 & 19 & 0 & 19d6+57 & \\
	%	20 & 0 & 20d6+60 & A scelta & 20 & 20 & 0 & 20d6+60 & A scelta \\
	%	\bottomrule
	%\end{tabularx}

\begin{multicols}{2}

\subsubsection*{Guerriero}

\begin{tabularx}{\linewidth}{c|>{\hsize=0.08\hsize}X>{\hsize=0.08\hsize}X>{\hsize=0.33\hsize}X|X|}
	\textbf{Lv} & \multicolumn{3}{c|}{\textbf{Guerriero}} & \textbf{Abilità} \\
	& \centering\arraybackslash \textbf{CA} & \centering\arraybackslash \textbf{CM} & \centering\arraybackslash \textbf{PF} & \\
	\toprule
	1 &1	& 0	&	8+3	&\hyperlink{Arma Focalizzata}{Arma Focalizzata} - \hyperlink{Colpi Poderosi}{Colpi Poderosi}\\
	2	&	2	& 0	&	2d6+6	&\hyperlink{Estrazione rapida}{Estrazione rapida}\\
	3	&	3	& 0	&	3d6+9	&\hyperlink{Opportunista}{Opportunista}\\
	4	&	4	& 0	&	4d6+12	&\hyperlink{Pelle Coriacea}{Pelle Coriacea}\\
	5	&	5	& 0	&	5d6+15	&\hyperlink{Proseguire}{Proseguire}\\
	6	&	6	& 0	&	6d6+18	&\hyperlink{Iniziativa migliorata}{Iniziativa migliorata}\\
	7	&	7	& 0	&	7d6+21	&\hyperlink{Questa è la mia arma!}{Questa è la mia arma!}\\
	8	&	8	& 0	&	8d6+24	&\\
	9	&	9	& 0	&	9d6+27	&\hyperlink{Proseguire}{Proseguire}\\
	10	&	10	& 0	&	10d6+30	&\hyperlink{Riflessi fulminei}{Riflessi fulminei}\\
	11	&	11	& 0	&	11d6+33	&\\
	12	&	12	& 0	&	12d6+36	&\hyperlink{Pelle Coriacea}{Pelle Coriacea}\\
	13	&	13	& 0	&	13d6+39	&\hyperlink{Seconda pelle}{Seconda pelle}\\
	14	&	14	& 0	&	14d6+42	&\\
	15	&	15	& 0	&	15d6+45	&\hyperlink{Seconda pelle}{Seconda pelle}\\
	16	&	16	& 0	&	16d6+48	&\hyperlink{Combattere alla Cieca}{Combattere alla Cieca}\\
	17	&	17	& 0	&	17d6+51	&\\
	18	&	18	& 0	&	18d6+54	& a scelta\\
	19	&	19	& 0	&	19d6+57	&\\
	20	&	20	& 0	&	20d6+60	& a scelta\\
	\bottomrule
\end{tabularx}

\subsubsection*{Ladro}

\begin{tabularx}{\linewidth}{c|>{\hsize=0.08\hsize}X>{\hsize=0.08\hsize}X>{\hsize=0.33\hsize}X|X|}
	\textbf{Lv} & \multicolumn{3}{c|}{\textbf{Ladro}} & \textbf{Abilità} \\
	& \centering\arraybackslash \textbf{CA} & \centering\arraybackslash \textbf{CM} & \centering\arraybackslash \textbf{PF} & \\
	\toprule
	1 	&	1	& 0	&	8+3	&\hyperlink{Persona veramente malvagia}{Persona veramente malvagia} - \hyperlink{Estrazione rapida}{Estrazione rapida}\\
	2	&	2	& 0	&	2d6+6	&\hyperlink{Opportunista}{Opportunista}\\
	3	&	3	& 0	&	3d6+9	&\hyperlink{Colpo Furtivo}{Colpo Furtivo}\\
	4	&	4	& 0	&	4d6+12	&\hyperlink{Passo Felpato}{Passo Felpato}\\
	5	&	5	& 0	&	5d6+15	&\hyperlink{Occhio Clinico}{Occhio Clinico}\\
	6	&	6	& 0	&	6d6+18	&\hyperlink{Schivare trappole}{Schivare trappole}\\
	7	&	7	& 0	&	7d6+21	&\hyperlink{Percettivo}{Percettivo}\\
	8	&	8	& 0	&	8d6+24	&\\
	9	&	9	& 0	&	9d6+27	&\hyperlink{Colpo Furtivo}{Colpo Furtivo}\\
	10	&	10	& 0	&	10d6+30	&\hyperlink{Esperto}{Esperto}\\
	11	&	11	& 0	&	11d6+33	&\\
	12	&	12	& 0	&	12d6+36	&\hyperlink{Colpo Furtivo}{Colpo Furtivo}\\
	13	&	13	& 0	&	13d6+39	&\hyperlink{Schivata prodigiosa}{Schivata prodigiosa}\\
	14	&	14	& 0	&	14d6+42	&\\
	15	&	15	& 0	&	15d6+45	&\hyperlink{Colpo Furtivo}{Colpo Furtivo}\\
	16	&	16	& 0	&	16d6+48	&\hyperlink{Improvvisare}{Improvvisare}\\
	17	&	17	& 0	&	17d6+51	&\\
	18	&	18	& 0	&	18d6+54	& a scelta\\
	19	&	19	& 0	&	19d6+57	&\\
	20	&	20	& 0	&	20d6+60	& a scelta\\
	\bottomrule
\end{tabularx}

\subsubsection*{Mago}

\begin{tabularx}{\linewidth}{c|>{\hsize=0.08\hsize}X>{\hsize=0.08\hsize}X>{\hsize=0.33\hsize}X|X|}
	\textbf{Lv} & \multicolumn{3}{c|}{\textbf{Mago}} & \textbf{Abilità} \\
	& \centering\arraybackslash \textbf{CA} & \centering\arraybackslash \textbf{CM} & \centering\arraybackslash \textbf{PF} & \\
	\toprule
	1 	&	0	& 1	&	8	&	\hyperlink{Adepto della Magia}{Adepto della Magia} - \hyperlink{Tutt'uno con la magia}{Tutt'uno con la magia}\\
	2	&	0	& 2	&	2d6	&	\hyperlink{Un colpo un morto}{Un colpo un morto}\\
	3	&	0	& 3	&	3d6	&	\hyperlink{Specialista}{Specialista}\\
	4	&	0	& 4	&	4d6	&	\hyperlink{Adepto della Magia}{Adepto della Magia}\\
	5	&	1	& 4	&	5d6+3	&\hyperlink{Concentrato}{Concentrato}\\
	6	&	1	& 5	&	6d6+3	&\hyperlink{Magie Potenti}{Magie Potenti}\\
	7	&	1	& 6	&	7d6+3	&\hyperlink{Incantatore Prudente}{Incantatore Prudente}\\
	8	&	1	& 7	&	8d6+3	&\\
	9	&	2	& 7	&	9d6+6	&\hyperlink{Prodigioso}{Prodigioso}\\
	10	&	2	& 8	&	10d6+6	&\hyperlink{Dattilografo}{Dattilografo}\\
	11	&	2	& 9	&	11d6+6	&\\
	12	&	2	& 10&	12d6+6	&\hyperlink{Creare Oggetti Magici}{Creare Oggetti Magici}\\
	13	&	3	& 10&	13d6+9	&\hyperlink{Concentrato}{Concentrato}\\
	14	&	3	& 11&	14d6+9	&\\
	15	&	3	& 12&	15d6+9	&\hyperlink{Adepto della Magia}{Adepto della Magia}\\
	16	&	3	& 13&	16d6+9	&\hyperlink{Adepto della Magia}{Adepto della Magia}\\
	17	&	4	& 13&	17d6+12	&\\
	18	&	4	& 14&	18d6+12	& a scelta\\
	19	&	4	& 15&	19d6+12	&\\
	20	&	5	& 15&	20d6+15	& a scelta\\
	\bottomrule
\end{tabularx}

\subsubsection*{Chierico}

\begin{tabularx}{\linewidth}{c|>{\hsize=0.08\hsize}X>{\hsize=0.08\hsize}X>{\hsize=0.33\hsize}X|X|}
	\textbf{Lv} & \multicolumn{3}{c|}{\textbf{Chierico}} & \textbf{Abilità}\\
	& \centering\arraybackslash \textbf{CA} & \centering\arraybackslash \textbf{CM} & \centering\arraybackslash \textbf{PF} & \\
	\toprule
	1 	&	0	& 1	&	8	&\hyperlink{Adepto della Magia}{Adepto della Magia} - \hyperlink{Tutt'uno con la magia}{Tutt'uno con la magia}\\
	2	&	0	& 2	&	2d6	&\hyperlink{Il Patrono è la mia Arma}{Il Patrono è la mia Arma}\\
	3	&	0	& 3	&	3d6	&\hyperlink{Imposizione delle mani}{Imposizione delle mani}\\
	4	&	1	& 3	&	4d6+3	&\hyperlink{Scacciare i non morti}{Scacciare i non morti}\\
	5	&	1	& 4	&	5d6+3	&\hyperlink{Potere del Patrono}{Potere del Patrono}\\
	6	&	1	& 5	&	6d6+3	&\hyperlink{Armatura del Devoto}{Armatura del Devoto}\\
	7	&	1	& 6	&	7d6+3	&\hyperlink{Incanalare Energia}{Incanalare Energia}\\
	8	&	2	& 6	&	8d6+6	&\\
	9	&	2	& 7	&	9d6+6	&\hyperlink{Armatura del Devoto}{Armatura del Devoto}\\
	10	&	2	& 8	&	10d6+6	&\hyperlink{Il Patrono è con me}{Il Patrono è con me}\\
	11	&	2	& 9	&	11d6+6	&\\
	12	&	3	& 9	&	12d6+9	&\hyperlink{Il Patrono è la mia Arma}{Il Patrono è la mia Arma}\\
	13	&	3	& 10&	13d6+9	&\hyperlink{Adepto della Magia}{Adepto della Magia}\\
	14	&	3	& 11&	14d6+9	&\\
	15	&	3	& 12&	15d6+9	&\hyperlink{Armatura del Devoto}{Armatura del Devoto}\\
	16	&	4	& 12&	16d6+12	&\hyperlink{Il Patrono è con me}{Il Patrono è con me}\\
	17	&	4	& 13&	17d6+12	&\\
	18	&	4	& 14&	18d6+12	& a scelta\\
	19	&	5	& 14&	19d6+15	&\\
	20	&	5	& 15&	20d6+15	& a scelta\\
	\bottomrule
\end{tabularx}

\subsubsection*{Barbaro}

\begin{tabularx}{\linewidth}{c|>{\hsize=0.08\hsize}X>{\hsize=0.08\hsize}X>{\hsize=0.33\hsize}X|X|}
	\textbf{Lv} & \multicolumn{3}{c|}{\textbf{Barbaro}} & \textbf{Abilità} \\
	& \centering\arraybackslash \textbf{CA} & \centering\arraybackslash \textbf{CM} & \centering\arraybackslash \textbf{PF} & \\
	\toprule
	1 	&	1	& 0	&	8+3	&\hyperlink{Ferocia}{Ferocia} - \hyperlink{Furia}{Furia}\\
	2	&	2	& 0	&	2d6+6	&\hyperlink{La mia morte la tua morte}{La mia morte la tua morte}\\
	3	&	3	& 0	&	3d6+9	&\hyperlink{abRinoceronte}{Rinoceronte}\\
	4	&	4	& 0	&	4d6+12	&\hyperlink{Ho detto CADI!}{Ho detto CADI!}\\
	5	&	5	& 0	&	5d6+16	&\hyperlink{Duro a morire}{Duro a morire}\\
	6	&	6	& 0	&	6d6+20	&\hyperlink{Pelle Coriacea}{Pelle Coriacea}\\
	7	&	7	& 0	&	7d6+24	&\hyperlink{Colpo Mortale}{Colpo Mortale}\\
	8	&	8	& 0	&	8d6+28	&\\
	9	&	9	& 0	&	9d6+32	&\hyperlink{Primo Sangue}{Primo Sangue}\\
	10	&	10	& 0	&	10d6+36	&\hyperlink{Montagna umana}{Montagna umana}\\
	11	&	11	& 0	&	11d6+41	&\\
	12	&	12	& 0	&	12d6+43	&\hyperlink{più sono grossi più fanno rumore quando cadono}{più sono grossi più fanno rumore quando cadono}\\
	13	&	13	& 0	&	13d6+48	&\hyperlink{Un braccio, un arma}{Un braccio, un arma}\\
	14	&	14	& 0	&	14d6+53	&\\
	15	&	15	& 0	&	15d6+58	&\hyperlink{Pelle Coriacea}{Pelle Coriacea}\\
	16	&	16	& 0	&	16d6+63	&\hyperlink{Volonta' Ferrea}{Volonta' Ferrea}\\
	17	&	17	& 0	&	17d6+68	&\\
	18	&	18	& 0	&	18d6+73	& a scelta\\
	19	&	19	& 0	&	19d6+78	&\\
	20	&	20	& 0	&	20d6+83	& a scelta\\
	\bottomrule
\end{tabularx}

\bigskip

\subsubsection*{Paladino}

\begin{tabularx}{\linewidth}{c|>{\hsize=0.08\hsize}X>{\hsize=0.08\hsize}X>{\hsize=0.33\hsize}X|X|}
	\textbf{Lv} & \multicolumn{3}{c|}{\textbf{Paladino}} & \textbf{Abilità} \\
	& \centering\arraybackslash \textbf{CA} & \centering\arraybackslash \textbf{CM} & \centering\arraybackslash \textbf{PF} & \\
	\toprule
	1 	&	1	& 0	&	8+3	&\hyperlink{Il Patrono è la mia Arma}{Il Patrono è la mia Arma} - \hyperlink{Rappresaglia}{Rappresaglia}\\
	2	&	2	& 0	&	2d6+6	&\hyperlink{Questa è la mia arma!}{Questa è la mia arma!}\\
	3	&	3	& 0	&	3d6+9	&\hyperlink{Imposizione delle mani}{Imposizione delle mani}\\
	4	&	3	& 1	&	4d6+9	&\hyperlink{Armatura del Devoto}{Armatura del Devoto}\\
	5	&	4	& 1	&	5d6+12	&\hyperlink{Il Patrono è la mia Arma}{Il Patrono è la mia Arma}\\
	6	&	5	& 1	&	6d6+15	&\hyperlink{Lo scudo è mio amico}{Lo scudo è mio amico}\\
	7	&	6	& 1	&	7d6+18	&\hyperlink{Imposizione delle mani}{Imposizione delle mani}\\
	8	&	6	& 2	&	8d6+18	&\\
	9	&	7	& 2	&	9d6+21	&\hyperlink{Armatura del Devoto}{Armatura del Devoto}\\
	10	&	8	& 2	&	10d6+24	&\hyperlink{Il Patrono è la mia Arma}{Il Patrono è la mia Arma}\\
	11	&	9	& 2	&	11d6+27	&\\
	12	&	9	& 3	&	12d6+27	&\hyperlink{Il Patrono è la mia Arma}{Il Patrono è la mia Arma}\\
	13	&	10	& 3	&	13d6+30	&\hyperlink{Imposizione delle mani}{Imposizione delle mani}\\
	14	&	11	& 3	&	14d6+33	&\\
	15	&	12	& 3	&	15d6+36	&\hyperlink{Il Patrono è la mia Arma}{Il Patrono è la mia Arma}\\
	16	&	12	& 4	&	16d6+36	&\hyperlink{Armatura del Devoto}{Armatura del Devoto}\\
	17	&	13	& 4	&	17d6+39	&\\
	18	&	14	& 4	&	18d6+42	& a scelta\\
	19	&	15	& 4	&	19d6+45	&\\
	20	&	15	& 5	&	20d6+45	& a scelta\\
	\bottomrule
\end{tabularx}

\subsubsection*{Bardo}

\begin{tabularx}{\linewidth}{c|>{\hsize=0.08\hsize}X>{\hsize=0.08\hsize}X>{\hsize=0.33\hsize}X|X|}
	\textbf{Lv} & \multicolumn{3}{c|}{\textbf{Bardo}} & \textbf{Abilità} \\
	& \centering\arraybackslash \textbf{CA} & \centering\arraybackslash \textbf{CM} & \centering\arraybackslash \textbf{PF} & \\
	\toprule
	1 	&	0	& 1	&	8	&\hyperlink{Incantatore da Combattimento}{Incantatore da Combattimento} - \hyperlink{Infondere Coraggio}{Infondere Coraggio}\\
	2	&	1	& 1	&	2d6+3	&\hyperlink{Fortunato}{Fortunato}\\
	3	&	2	& 1	&	3d6+6	&\hyperlink{Infondere Paura}{Infondere Paura}\\
	4	&	2	& 2	&	4d6+6	&\hyperlink{Fare Infuriare}{Fare Infuriare}\\
	5	&	3	& 2	&	5d6+9	&\hyperlink{Infondere Coraggio}{Infondere Coraggio}\\
	6	&	3	& 3	&	6d6+9	&\hyperlink{Tattico}{Tattico}\\
	7	&	4	& 3	&	7d6+12	&\hyperlink{Litania versatile}{Litania versatile}\\
	8	&	4	& 4	&	8d6+12	&\\
	9	&	5	& 4	&	9d6+15	&\hyperlink{Danno Coordinato}{Danno Coordinato}\\
	10	&	5	& 5	&	10d6+15	&\hyperlink{Incantatore da Combattimento}{Incantatore da Combattimento}\\
	11	&	6	& 5	&	11d6+18	&\\
	12	&	6	& 6	&	12d6+18	&\hyperlink{Infondere Coraggio}{Infondere Coraggio}\\
	13	&	7	& 6	&	13d6+21	&\hyperlink{Infondere Energia Magica}{Infondere Energia Magica}\\
	14	&	7	& 7	&	14d6+21	&\\
	15	&	8	& 7	&	15d6+24	&\hyperlink{Danno Coordinato}{Danno Coordinato}\\
	16	&	8	& 8	&	16d6+24	&\hyperlink{Infondere Energia Magica Superiore}{Infondere Energia Magica Superiore}\\
	17	&	9	& 8	&	17d6+27	&\\
	18	&	9	& 9	&	18d6+27	& a scelta\\
	19	&	10	& 9	&	19d6+30	&\\
	20	&	10	& 10&	20d6+30	& a scelta\\
	\bottomrule
\end{tabularx}

\subsubsection*{Druido}

\begin{tabularx}{\linewidth}{c|>{\hsize=0.08\hsize}X>{\hsize=0.08\hsize}X>{\hsize=0.33\hsize}X|X|}
	\textbf{Lv} & \multicolumn{3}{c|}{\textbf{Druido}} & \textbf{Abilità} \\
	& \centering\arraybackslash \textbf{CA} & \centering\arraybackslash \textbf{CM} & \centering\arraybackslash \textbf{PF} & \\
	\toprule
	1 	&	0	& 1	&	8		&\hyperlink{Adepto della Magia}{Adepto della Magia} - \hyperlink{Distillare pozioni}{Distillare pozioni}\\
	2	&	0	& 2	&	2d6		&\hyperlink{Animalia}{Animalia}\\
	3	&	0	& 3	&	3d6		&\hyperlink{Sangue Puro}{Sangue Puro}\\
	4	&	1	& 3	&	4d6+3	&\hyperlink{Armatura del Devoto}{Armatura del Devoto}\\
	5	&	1	& 4	&	5d6+3	&\hyperlink{Animalia}{Animalia}\\
	6	&	1	& 5	&	6d6+3	&\hyperlink{Il Patrono è con me}{Il Patrono è con me}\\
	7	&	1	& 6	&	7d6+3	&\hyperlink{Forma Elementale}{Forma Elementale}\\
	8	&	2	& 6	&	8d6+6	&\\
	9	&	2	& 7	&	9d6+6	&\hyperlink{Il Patrono è con me}{Il Patrono è con me}\\
	10	&	2	& 8	&	10d6+6	&\hyperlink{Sangue Puro}{Sangue Puro}\\
	11	&	2	& 9	&	11d6+6	&\\
	12	&	3	& 9	&	12d6+9	&\hyperlink{Il Patrono è con me}{Il Patrono è con me}\\
	13	&	3	& 10&	13d6+9	&\hyperlink{Animalia}{Animalia}\\
	14	&	3	& 11&	14d6+9	&\\
	15	&	3	& 12&	15d6+9	&\hyperlink{Sangue Puro}{Sangue Puro}\\
	16	&	4	& 12&	16d6+12	&\hyperlink{Adepto della Magia}{Adepto della Magia}\\
	17	&	4	& 13&	17d6+12	&\\
	18	&	4	& 14&	18d6+12	& a scelta\\
	19	&	5	& 14&	19d6+15	&\\
	20	&	5	& 15&	20d6+15	& a scelta\\
	\bottomrule
\end{tabularx}

\end{multicols}

}

\pagebreak

