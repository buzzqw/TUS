\section{Caratteristiche Speciali}

\begin{enfasi}{
Non basta avere gli occhi per vedere (anonimo)
}\end{enfasi}

\begin{multicols}{2}

Ogni creatura è speciale ed unica eppure ci sono esseri ancora più unici e speciali per le loro caratteristiche. Queste sono le peculiarità di alcune di queste.

\subsection{Visione Crepuscolare}\index{Visione Crepuscolare}\label{visionecrepuscolare}

Quello che per molti é oscurità per chi ha \hypertarget{visionecrepuscolare}{visione crepuscolare} é vedere bene purché ci sia una fonte minima di luce.

La visione crepuscolare è una visione a colori.
Un incantatore dotato di visione crepuscolare può leggere una Pergamena fino a quando ha accanto come fonte di luce anche la più smorta delle candele.

I personaggi dotati di visione crepuscolare possono vedere all'esterno nelle notti illuminate dalla luna come se si trovassero alla luce del giorno.
Nella assoluta mancanza di luce la visione crepuscolare non aiuta, rimane buio pesto impenetrabile.

\subsection{Scurovisione}\index{Scurovisione}\label{scurovisione}

La Scurovisione è la capacità straordinaria di vedere senza fonti di luce, fino ad una distanza massima indicata per ogni creatura.

La Scurovisione è solo in bianco e nero (non consente alla creatura di distinguere i colori). Non permette ai personaggi di vedere nulla che non possano altrimenti vedere: gli oggetti Invisibili sono ancora Invisibili, e le Illusioni sono ancora visibili per quello che sembrano essere.

Alla stessa maniera, la Scurovisione rende una creatura soggetta agli attacchi con lo sguardo normalmente. La presenza di luce non altera la Scurovisione.
Effettuare una prova di Sopravvivenza per cercare trappole o di Consapevolezza solo visiva prende un -2 di penalità.

\subsection{Fiuto}\index{Fiuto}\label{fiuto}

Questa qualità speciale permette ad una creatura di sfruttare l'olfatto per individuare i nemici nascosti o in avvicinamento e di seguire le tracce. Le creature dotate di fiuto possono identificare con l'olfatto gli odori familiari come gli umani fanno con quello che vedono.

La creatura può individuare le creature entro 6 metri di distanza con l'olfatto. Se l'avversario è sottovento, il raggio aumenta a 18 metri; se è sopravento, il raggio diminuisce a distanza di 3 metri.
Gli odori più forti, come il fumo, spazzatura o corpi in decomposizione, possono essere individuati al doppio del raggio sopra indicato.

Quando una creatura individua un odore, non viene rivelata l'esatta posizione della sua fonte, ma solo la sua presenza entro il raggio d'azione. La creatura può utilizzare un'Azione per individuare la direzione da cui proviene l'odore. Quando si trova a distanza di mischia dalla fonte, ne individua la posizione.

Una creatura dotata di fiuto può seguire tracce utilizzando l'olfatto, effettuando una prova di Seguire Tracce per trovare e seguire una traccia. La tipica DC di una traccia fresca è 10 (a prescindere dalla superficie su cui si trova la traccia). La DC aumenta o diminuisce a seconda dell'intensità della traccia, del numero di creature che la lasciano e del tempo trascorso da quando è stata lasciata. Per ogni ora trascorsa la DC aumenta di 2.

\begin{center}
\includegraphics[width=0.7\linewidth]{immagini/mostro.png}

\emph{John D. Batten}
\end{center}

Per il resto, questa capacità segue le regole della competenza Sopravvivenza. Le creature che seguono tracce con il fiuto ignorano gli effetti delle superfici su cui si trova la traccia e della scarsa visibilità.

L'acqua, e in particolare l'acqua corrente, nega la capacità di seguire tracce delle creature.

Alcuni forti odori possono facilmente mascherarne altri. La presenza di un odore simile rende impossibile individuare o identificare esattamente una creatura mediante il Fiuto; la DC base della competenza Sopravvivenza per seguire tracce in presenza di odori coprenti passa da 10 a 20.

\subsection{Vista Cieca}\index{Vista Cieca}\label{vistacieca}

Utilizzando un insieme di sensi diversi dalla vista, come la percezione delle vibrazioni, un fiuto sensibile, un udito acuto o un sonar, una creatura dotata di vista cieca si muove e combatte bene quanto una creatura dotata della vista.

Invisibilità e buio sono indifferenti anche se la creatura dotata di vista cieca deve avere una linea di effetto per notare una determinata creatura o oggetto.

Una creatura con copertura continua comunque ad avere il suo vantaggio alla Difesa.

Il raggio della capacità è indicato nella descrizione della creatura. La creatura, in genere, non deve effettuare prove di Consapevolezza per notare creature entro il raggio della sua vista cieca.

A meno che non sia diversamente indicato, la vista cieca è sempre attiva e la creatura non deve compiere azioni per attivarla. Alcune forme di vista cieca devono essere attivate come Azione Immediata. In questo caso, viene indicato nella descrizione della creatura.

Una creatura eterea non è visibile alla vista cieca.

\subsection{Visione del Vero}\index{Visione del Vero}\label{cap Visione del Vero}\hypertarget{cap Visione del Vero}{}

Una creatura con Visione del Vero può, entro il raggio indicato, vedere nell'oscurità normale e magica, vedere creature e oggetti invisibili, rilevare automaticamente le illusioni visive e superare i Tiri Salvezza contro di esse. Percepisce la forma originale di un mutaforma o di una creatura trasformata dalla magia. La creatura con Visione del Vero può vedere nel Piano Eterico.

\medskip

\begin{center}
\includegraphics[height=0.65\linewidth]{immagini/grabroid.png}

\medskip

\emph{Grabroid. Conosciuti anche come Agguantatori. Tremors (Film)}
\end{center}

\subsection{Senso Tellurico}\index{Senso Tellurico}\label{sensotellurico}
Una creatura dotata di Senso Tellurico è sensibile alle vibrazioni del suolo e può automaticamente individuare qualsiasi cosa sia in contatto con il terreno entro il raggio specificato dal Senso Tellurico.

Le Creature Acquatiche dotate di Senso Tellurico (ecolocalizzazione) possono percepire la posizione di creature in contatto con l'acqua.

Il raggio della capacità è specificato nel testo descrittivo della creatura.

\end{multicols}

\vfill

\begin{center}
	\includegraphics[width=0.75\linewidth]{immagini/argus2.png}

	\emph{Argus Panoptes Guarding the Heifer (Io), Red Figure pitcher, c. 460 BC Museum of Fine Arts, Boston}
\end{center}

\pagebreak

