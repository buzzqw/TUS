\section{Movimento e Trasporto}\index{Trasporto}\index{Movimento}\label{movimentocap}

\label{movimento-e-trasporto}

\begin{changemargin}{0.3cm}{0.3cm}\begin{enfasi}{
Il mio piede sinistro funziona benissimo, ma non riuscirei comunque a camminare se non ci fosse il destro! (Madagascar 3 - Ricercati in Europa, Film )

\medskip

Quando non puoi più correre, cammina veloce; quando non puoi più camminare veloce, cammina; quando non puoi più camminare, usa il bastone; però non trattenerti mai. (Madre Teresa di Calcutta)}
\end{enfasi}\end{changemargin}\medskip

\begin{multicols}{2}

Il movimento si può distinguere in base a quale situazione si applica.

\medskip

\begin{itemize}[leftmargin=*] \setlength{\itemsep}{0pt}
\item Tattico, quando si combatte, si usano le distanze precise, mappa ed i quadretti di 1 metro di lato
\item Locale, per esplorare una zona, misurato in metri al minuto.
\item Via Terra, per muoversi da un posto all'altro, misurato in km all'ora o al giorno.
\end{itemize}

\subsection{Tipi di Movimento}\label{tipodimovimento}\hypertarget{tipodimovimento}{}

Quando si muovono nelle differenti situazioni di movimento (Tattico, Locale, Via Terra), le creature generalmente camminano o corrono.

\textbf{Camminare}:\index{Camminare} Camminare rappresenta un movimento non affrettato ma deciso di circa 4 km all'ora per un umano non ingombrato. Per Azione di Movimento la creatura percorre la distanza indicata in Movimento.

\textbf{Correre}\index{Correre}: Significa muoversi di circa 12 km all'ora per un umano.

Correre come Azione di Movimento raddoppia la velocità di movimento.
Il personaggio che corre ha penalità di 1d6 nel Tiro per Colpire e di 4 nella Difesa fino all'inizio del suo round successivo.
Solo in situazioni di non combattimento la corsa triplica il movimento, ovvero quando si usa il Movimento locale o Via Terra.

\textbf{Tabella: Movimento e Distanza e Velocità: a Piedi} \index{Movimento a Piedi}\index[Tabelle]{Tabella Movimento e Distanza e Velocità : a Piedi}

Questa tabella mostra i valori base di movimento a terra in situazioni di non combattimento.

\medskip

\noindent\begin{tabularx}{0.48\textwidth}{lccc}
\multirow{2}*{\textbf{Tipo di movimento}} &
\multicolumn{3}{c}{\textbf{Movimento}}\\
\cmidrule(lr){2-4} & \textbf{6m}& \textbf{9m} & \textbf{12m}\\
\midrule
\multicolumn{4}{c}{\textbf{Movimento (Tattico)}}\\
Camminare& 6m & 9m & 12m\\
Correre (x2) & 12m& 18m& 24m\\
\multicolumn{4}{c}{\textbf{Un minuto (Locale)}} \\
Camminare & 36m& 54m& 72m \\
Correre (x3) & 108m & 162m & 216m \\
\multicolumn{4}{c}{\textbf{Un'ora (Via Terra)}} \\
Camminare& 3km& 4km& 6km\\
Correre (x3) & 9km& 12km & 18km \\
\multicolumn{4}{c}{\textbf{Un giorno (Via Terra)}}\\
Camminare& 24km & 32km & 54km
\end{tabularx}

\subsection{Movimento Tattico}\index{Movimento Tattico}\label{movimentotattico}

Durante un combattimento si utilizza il Movimento tattico.
Le distanze vengono misurate in quadretti da un metro, il movimento è gestito tramite le Azioni di Movimento.

Un personaggio può usare 1 Azione (di Movimento) per muoversi fino a tutto il proprio movimento. Può effettuare più volte nel round, fino a 3 volte, l'Azione Movimento, spostandosi quindi del triplo del suo movimento.

Può anche effettuare una Azione di Scatto\index{Scatto} ovvero \textbf{Correre} e quindi muoversi del doppio del suo Movimento in una sola Azione. Incappa così però nelle penalità per chi corre (-1d6 al Tiro per Colpire, -4 Difesa).

Un personaggio può effettuare fino a 3 Azioni di Scatto, ovvero corre per tutto il round percorrendo quindi il suo movimento * 6.

\subsubsection{Movimento Ostacolato - Terreno Difficile}\index{Terreno difficile}\label{terrenodifficile}

Terreno difficile, innevato, ghiacciato, con rapide salite e discese, pieno di macerie o con ostacoli o scarsa visibilità possono impedire i movimenti. Quando il movimento è ostacolato ci si muove a metà della velocità, sono necessarie 2 Azioni per coprire la propria distanza di 9 metri (se si è umano senza ingombro..), oppure con una Azione di Movimento si copre solo 4 metri.

Se esiste più di una condizione particolare, aggiungere tra loro tutti i costi aggiuntivi applicabili, ovvero se un terreno è difficile e ci si muove a carponi significa muoversi di un quarto del proprio movimento. \index{Terreno doppiamente difficile}

In alcune situazioni il movimento è talmente ostacolato\index{Movimento quasi impossibile} che la distanza percorribile per Azione è minima, in tal caso si possono utilizzare tutte e 3 le Azioni per muoversi di solo 1 metro in qualsiasi direzione.

Non applicate questa regola per attraversare terreni impraticabili o per muoversi quando non è possibile farlo in alcun modo.

Non si può \textbf{Scattare} (Correre) su un terreno difficile, il giocatore può tentare una prova di Acrobatica a DC 20 per riuscire a correre ma non a caricare, se fallisce tratta il terreno come Difficile, se Fallisce Criticamente cade prono sul posto. \index{Scatto su terreno difficile}

Non si può \hyperlink{carica}{\textbf{Caricare}}\index{Carica su Terreno difficile} attraverso un terreno difficile a meno di avere l'Abilità \hyperlink{Rinoceronte}{Rinoceronte} (pag. \pageref{Rinoceronte}).

Muoversi da prono\index{Muoversi da prono}\index{Muoversi a carponi}, Nuotare o Strisciare\index{Strisciare} è considerato terreno difficile, Arrampicarsi è doppiamente difficile.

Il terreno dove sono presenti dei corpi di creature si considera difficile. \index{Muoversi su corpi}

\begin{changemargin}{0.3cm}{0.3cm}\begin{tcolorbox}[title = Tups nel cunicolo] %box giocatore
Tups è con i suoi compagni in uno stretto cunicolo in fila indiana. E' alla quarta posizione.

Improvvisamente un nemico si para davanti e Tups è il più veloce a reagire, usando una Azione di Movimento \emph{\textbf{attraversa}} i 3 compagni che ha davanti rimanendo \textbf{ristretto} con il primo della fila.

Potrebbe decidere di (tra le varie possibilità):

\begin{itemize}[leftmargin=*] \setlength{\itemsep}{-1pt}
	\item stare fermo senza scattare in avanti, lasciare agire prima i compagni davanti a lui.
	\item rimanere ristretto ed attaccare. Rimangono 2 Azioni.
	\item spingere il compagno (1 Azione) nel quadretto precedente, facendolo stringere con un altro compagno. Rimane 1 Azione.
	\item spingere il compagno (1 Azione) nel quadretto successivo! facendo attraversare a lui il quadretto del nemico. Rimane 1 Azione.
	\item tornare indietro (1 Azione) al suo quadretto iniziale. Rimane 1 Azione.
	\item provare ad attraversare l'avversario (1 Azione), ma se fallisce sarebbe ristretto con il compagno, danneggiando entrambi e gli rimarrebbe soltanto 1 altra Azione
\end{itemize}

\end{tcolorbox}\end{changemargin}

\subsubsection{Condividere gli Spazi}\index{Condividere il quadretto}\index{più creature nello stesso spazio}\label{condividereglispazi}\hypertarget{condividereglispazi}{}

Una creatura di taglia media o più piccola può condividere lo stesso quadretto con una creatura di taglia piccola.

Una creatura di taglia superiore a media può condividere i propri quadretti solo se l'altra creatura è di almeno 2 taglie inferiore.

Es. un mostro di taglia Grande può condividere il suo spazio solo con una creatura di taglia Piccola o inferiore, se fosse Enorme potrebbe condividerlo con una creatura di taglia Media o inferiore.

\subsubsection{Scambiarsi di posto}\index{Scambiarsi di posto}
Un personaggio a contatto con un altra creatura può usare \textbf{una Azione} per \textbf{scambiarsi di posto} con questa. Se la creatura è ostile è necessaria una Prova Atletica contrapposta ad un Tiro Salvezza su Tempra per riuscire a scambiarsi. Per ogni taglia di differenza chi ha quella maggiore prende +1d6 di bonus alla prova. Costa una Reazione alla creatura amichevole.

\begin{changemargin}{0.3cm}{0.3cm}\begin{narratore} %box narratore
Se volete un crudo realismo allora è terreno difficile attraversare anche zone dove ci sono creature amichevoli. \end{narratore}\end{changemargin}

\subsubsection{Essere ristretti con qualcuno}\index{Essere ristretti con qualcuno}
Due creature ristrette, ovvero che condividono lo stesso quadretto e non rispettano le regole di \hyperlink{condividereglispazi}{Condividere gli Spazi} subiscono un -1d6 al Tiro per Colpire ed un -4 alla Difesa finché ristretti.\index{Ristretti}\hypertarget{ristretti}{}\label{ristretti}

\subsubsection{Passare per strettoie o restringimenti}\index{Passare per strettoie o restringimenti}\index{Strettoie}\index{Restringimenti}

Passare attraverso uno spazio di una taglia più piccola equivale a muoversi in terreno difficile. Es. una creatura media, che occupa 1 quadretto, che deve passare per una strettoia da mezzo quadretto (mezzo metro) tratta quel percorso come terreno difficile.

Non è possibile attraversare restringimenti più stretti di una taglia.

\subsection{Movimento Locale}\index{Movimento Locale}\label{movimentolocale}

I personaggi che esplorano una zona usano il movimento locale, misurato in metri al minuto.

In queste situazione non è fondamentale misurare la distanza in maniera precisa ma appena la situazione diventa \emph{problematica} o richiede attenzione la mappa si converte in movimento tattico, quadrettata e misurata.

\begin{itemize}[leftmargin=*] \setlength{\itemsep}{0pt}
\item
Camminare: Un personaggio può camminare senza problemi in Movimento locale per 8 ore al giorno.
\item
Correre: Un personaggio può Correre per un numero di minuti pari suo valore di Tiro Salvezza Tempra senza bisogno di riposarsi (minimo 1 round).
\end{itemize}

\subsection{Movimento Via Terra}\index{Movimento Via Terra}\label{movimentoviaterra}

I personaggi che percorrono lunghe distanze usano il movimento Via terra. Il movimento Via terra è misurato in ore o giorni. Un giorno rappresenta 8 ore di tempo di viaggio reale. Per imbarcazioni a remi, un giorno significa remare per 10 ore. Per navi a vela rappresenta 24 ore di movimento.

Camminare più a lungo può sfinire (vedi Marcia forzata, sotto).

\textbf{Andare Veloci}\index{Andare Veloci}\label{andareveloci}

Si può andare veloci (movimento*2) per 1 ora senza problemi. Andare veloci per una seconda ora compresa tra due cicli di sonno provoca 1 Danno Non Letale e ogni ora aggiuntiva provoca il doppio dei danni subiti nell'ora precedente. Un personaggio che subisce Danni Non Letali da andatura veloce è considerato Affaticato per quel giorno.

Un personaggio Affaticato non può Correre o Caricare.

\textbf{Correre}\index{Correre}\label{correre}

Non è possibile Correre per un lungo periodo di tempo. Tentativi di Correre e riposarsi a cicli funzionano come Andare Veloci.

\textbf{Marcia Forzata}\index{Marcia Forzata}\label{marciaforzata}

In un giorno di cammino normale, si può camminare per 8 ore. Il resto del giorno viene sfruttato per fare e disfare il campo, riposarsi e mangiare.

Se si cammina di più di 8 ore è necessario effettuare un Tiro Salvezza su Tempra a difficoltà 11 +1 per ogni giorno consecutivo di marcia forzata o si diventa Affaticati. Il Tiro Salvezza viene effettuato ogni 2 ore oltre le 8 di cammino altrimenti aumenta il livello di Affaticato.

La marcia forzata può essere tenuta per un numero di giorni pari al valore di Costituzione+1 prima di incorrere nell'Affaticamento indipendentemente dall'esito del Tiro Salvezza.

\textbf{Terreno}\index{Terreno}\label{terreno}

Il terreno su cui si viaggia influenza quanta distanza viene percorsa in un'ora o in un giorno. A seconda dell'ambiente, clima, qualità della strada il Narratore può valutare che il movimento può essere normale, ridotto di un terzo, ridotto di metà oppure talmente impervio e difficile da ridurlo ad un quarto del movimento totale possibile.

\textbf{Movimento in sella}\index{Movimento in sella}\label{movimentoacavallo}

Una cavalcatura che porta un cavaliere può muoversi con andatura veloce. Tuttavia, i danni che subisce sono danni normali invece che non letali. Può anche essere costretta a una marcia forzata, ma le sue prove di Costituzione falliscono automaticamente ed i danni che subisce sono danni normali. Anche le cavalcature sono considerate Affaticate quando subiscono danni da andatura veloce o marcia forzata.

\textbf{Bardature da Cavalcatura}\index{Bardature da Cavalcatura}\index{Armature da Saurovallo}\label{ArmaturedaCavallo}\hypertarget{ArmaturedaCavallo}{}

Una cavalcatura può essere bardata con un armatura. Un armatura leggera conferirà un bonus alla Difesa di +2, una armatura Media concederà un bonus di +4 alla Difesa riducendo il movimento del 25\%, una armatura Pesante darà un +6 alla Difesa abbassando il movimento del 33\%.

\end{multicols}

%\medskip
%\begin{center}
%\includegraphics[height=0.3\linewidth]{immagini/carretto.png}
%\end{center}

\subsection{Tabella: Cavalcature e Veicoli}\index{Cavalcature}\index{Veicoli}\index[Tabelle]{Tabella Cavalcature e Veicoli}\index{Saurovallo movimento}\index{Movimento al giorno su saurovallo}

\medskip

\label{tabella-cavalcature-e-veicoli}\index{Cane}\index{Saurovallonano}\index{Carretto}\index{Zattera}\index{Barca}\index{Nave}\hypertarget{tabella-cavalcature-e-veicoli}{}

\begin{tabularx}{0.95\textwidth}{llXX}
\multirow{2}*{\textbf{Cavalcatura o Veicolo}} & \textbf{Ingombro trasportato} & \textbf{Movimento} & \textbf{Movimento}\\
&\textbf{(CdC)}&\textbf{All'ora} & \textbf{Al giorno}\\
\toprule
Cane da Galoppo & 30 & 6km & 36km \\
Saurovallo da Galoppo& 60& 8km & 48km \\
Saurovallo da Guerra & 80& 7km & 42km \\
Saurovallo Nano& 30& 5km & 30km \\
Saurovallo da Tiro& 70& 5km & 30km \\
Cammello& 50& 8km & 48km \\
Elefante& 160& 6km & 36km \\
\toprule
\textbf{Imbarcazione} &&& \\
\toprule
Zattera o Chiatta (pertica o rimorchio)& 225 & 0.75km & 7.5km \\
Barcone a Remi**& 425 & 1.5km& 15km\\
Barca a Remi**& 200 & 2.25km & 22.5km\\
Nave a Vela& 800 & 3km& 72km\\
Nave da Guerra (vele e remi) & 2200 & 3.5km & 90km\\
Nave Lunga (vele e remi)& 600& 5km& 108km \\
Galea (remi e vele) & 3300 & 6km& 144km
\end{tabularx}

\begin{multicols}{2}

\bigskip

Una cavalcatura può portare in groppa una creatura solo se di taglia inferiore alla sua. Il movimento al giorno si intende per 6 ore di cavalcata, oltre queste ore la cavalcatura si sfianca richiedendo un giorno intero di riposo.\index{Ore di cavalcata al giorno}

Zattere, chiatte e barconi sono usati su laghi e fiumi. Se seguono la corrente, sommare la velocità della corrente (di solito 4,5 km/h) alla velocità dell'imbarcazione. Oltre a essere spinta con pertiche o remi per 10 ore, l'imbarcazione può anche essere trasportata dalla corrente per altre 14 ore, se qualcuno è in grado di guidarla, e quindi si aggiungono altri 100 km nelle 24 ore. Queste imbarcazioni non possono essere spinte a remi contro una corrente molto forte, ma possono essere tirate controcorrente da animali da soma sulla riva.

Le Zattere e Chiatte attrezzate per il trasporto sono delle piccole locande che permettono un pasto frugale del pescato giornaliero ed un pò di frutta e verdura portata da riva. Non ci sono stanze per dormire. A chi ne fa richiesta, dietro un piccolo compenso, vengono stese delle stuoie e srotolati vissuti materassi e se il clima lo rende necessario vengono fornite coperte.

La guida della Zattera o Chiatta avviene su turni di 8 ore giornalieri, per permettere anche la continua navigazione. Quando è notte la navigazione si ferma o prosegue con la sola forza della corrente se non impetuosa e non ci sono pericoli noti. Pagando un sovrapprezzo è possibile navigare anche sulle 24 ore.

Se il viaggio dura più giorni diventa una occasione di conoscenza tra i personaggi quando nelle lunghe sere si raccolgono assieme agli altri ospiti e marinai per consumare il pasto e raccontarsi storie.

\subsection{Fuga ed Inseguimento}\index{Fuga}\index{Inseguimento}\label{fugainseguimento}

Nel movimento round per round è impossibile per un personaggio lento sfuggire ad un personaggio veloce senza qualche tipo di aiuto. Allo stesso modo non è un problema per un personaggio veloce sfuggire ad uno più lento.

Quando l'inseguimento avviene in città o comunque in un ambiente che permette di nascondersi o fare perdere le proprie tracce, se la velocità dei due personaggi coinvolti è uguale occorre che inseguitore ed inseguito effettuino 3 Tiri Salvezza consecutivi su Riflessi contrapposti. Chi vince la sfida riesce a fare perdere le proprie tracce od agguantare il fuggitivo.

Se l'inseguimento avviene all'aperto dove non c'è modo di nascondersi o fare perdere le proprie tracce eseguite 3 Tiri Salvezza su Tempra contrapposti per determinare quale delle due parti può mantenere più a lungo il ritmo. Chi vince la sfida riesce a seminare l'inseguitore od agguantare il fuggitivo.

\subsection{Capacita' di Carico e Trasporto: Ingombro}\index{Capacità di Carico}\index{Ingombro}\hypertarget{ingombro}{}\label{ingombro}

\label{sec:capacita-di-carico-e-trasporto-ingombro}

\subsubsection{Peso e Ingombro}\index{Ingombro}\index{Peso}

Portare tesori, pezzi di drago, armature complete per non parlare di armi sproporzionate o arieti da sfondamento, carrucole e paranco, rendono difficile il movimento.

Quando valutate il peso trasportato ragionate anche sull'ingombro!
Portare un rotolo di 12 metri x 6 metri di seta non è una attività fisica impegnativa, saranno pochi chili, ma l'ingombro è tale da non poter permettere ulteriore carico.

Ci possono essere oggetti leggeri ma estremamente ingombranti (tronchi cavi, tappeti di seta appunto..) oppure piccoli ma pesantissimi (sfere di mercurio, vestiti intessuti d'oro), per tutti questi oggetti il valore del peso deve essere ragionato anche in funzione dell'ingombro.

Ogni oggetto ha un proprio valore di Ingombro, in linea di massima \textbf{ogni 3 kg si ha 1 come fattore di Ingombro}. Questo valore può diventare anche 5Kg se l'oggetto è facilmente trasportabile. I valori di Ingombro degli oggetti si sommano tra di loro per dare il carico totale portato che si confronta con la Capacità di Carico della creatura.\index{Kili ed Ingombro}

Gli oggetti con scarso peso e volume hanno ingombro \textbf{Leggero} (L). Questi oggetti contano come 1 di Ingombro ogni 10 oggetti. Ogni 500 monete si ha 1 di Ingombro.\index{Ingombro delle monete}

\subsubsection{Capacita' di Carico}\label{capacitadicarico}\index{Capacita' di Carico}

La Capacità di Carico di una creatura è data dalla \textbf{somma di Taglia, Forza e la Costituzione}.

La Taglia di una creatura concede un bonus alla \textbf{CdC} (Capacità di Carico) pari a 9 se Piccola, 16 se Media, 25 se Grande. L'Ingombro di una creatura se trascinata\index{Trascinare un corpo} di peso è pari alla metà della sua Capacità di Carico, data dalla taglia, più il suo ingombro.\index{Ingombro Creature trasportate} Se trasportata di peso la CdC sarà pari all'Ingombro che ha la creatura.

Quando la CdC totale viene superata allora muoversi ed effettuare prove di competenza basate sulla Destrezza diventa problematico. Si diviene appesantiti e la capacità di movimento scende della metà e le prove di Competenza basate su Destrezza hanno un -3 di penalità.

Se la CdC viene doppiata allora non ci si può più muovere per l'ingombro dei pesi portati.

\emph{Ricordate che l'armatura e scudo quando indossati hanno un Ingombro dimezzato rispetto a quando trasportate.}

Es. Tups ha indosso una Armatura ad Anelli (ingombro 2 essendo indossata), una spada lunga (arma media, ingombro 2), una mazza chiodata (ing. 2), 18 oggetti leggeri (ing. 1), uno zaino (ing. 1), una tenda (ing. 2), una lanterna (ing 1). Totale Ingombro = 11.

Tups è una creatura Media con Forza -1 e Costituzione -1 (è un pò gracile e debole..) questo gli concede una Capacità di Carico di 16-1-1=14.

La CdC di Tups è superiore al suo ingombro ma deve stare attento, forse è meglio se lascia la tenda sul suo saurovallo...

Se il carico viene appoggiato su un carro puoi spingerlo a movimento pieno se entro la tua CdC, a metà movimento se entro il doppio della CdC ed ad un quarto del movimento se entro il quadruplo della CdC.

In caso più creature spingano o trainino un carro considerate come CdC quella più alta ed aggiungete metà delle altre creature. Un carro può essere spinto da 2 creature +1 per taglia del carro superiore a media.

\subsubsection{Creature più Grandi e più Piccole}\label{tagliaeportata}

Nella \textbf{Tabella: CdC trasportato in base alla taglia}\index{Ingombro trasportato in base alla taglia} viene riportata la Capacità di Carico in base alla taglia. Al valore dato dalla taglia vanno sommati i valori di Forza e Costituzione.

\medskip

\begin{tabularx}{0.45\textwidth}{ll|ll}
\textbf{Taglia}& \textbf{Ing.}&\textbf{Taglia} & \textbf{Ing.}\\
\toprule
Piccolissima &1/4& Grande & 25\\
Minuta & 1 & Enorme& 36\\
Minuscola & 4& Mastodontica&49\\
Piccola & 9 & Colossale&64\\
Media & 16&&
\end{tabularx}

\medskip

Creature con 4 zampe o più possono trasportare carichi maggiori.

%\begin{center}
%\includegraphics[height=0.5\linewidth]{immagini/cavallo.png}
%\end{center}

\textbf{Tabella: modificatori trasporto per creature con più zampe}\index[Tabelle]{Tabella modificatori trasporto per creature con più zampe}

\medskip

\begin{tabularx}{0.45\textwidth}{ll}
\textbf{Zampe Creatura}&\textbf{CdC}\\
\toprule
4 zampe & x2\\
6 zampe & x2.5\\
8 zampe & x3\\
12 zampe & x4\\
ogni altre 2 zampe & +0.5
\end{tabularx}

\medskip

Queste Tabelle sono da usare per gli animali insoliti non indicati o assimilabili a quelli nella Tabella: Cavalcature e Veicoli.

\subsection{Altri Tipi di Movimento}

\begin{changemargin}{0.3cm}{0.3cm}\begin{enfasi}{
Uno dei problemi riguarda la velocità della luce e le difficoltà che comporta il tentare di superarla. Non la si può superare. Niente viaggia più in fretta della velocità della luce, con la possibile eccezione delle cattive notizie, che seguono proprie leggi specifiche. (Douglas Adams)
}\end{enfasi}\end{changemargin}

\label{altri-tipi-di-movimento}

\subsubsection{Nuotare}\index{Nuotare}\label{nuotare}

Vedi Capito Ambiente per \hyperlink{pericoli-dellacqua}{prove nuotare} (pag. \pageref{pericoli-dellacqua}) e \hyperlink{combatteresottacqua}{combattimento sott'acqua} (pag. \pageref{combatteresottacqua}).

\subsubsection{Scalare}\index{Scalare}\label{scalare}

Una creatura con una velocità di Scalare ha un bonus di +2d6 su tutte le prove di Arrampicarsi, quando necessario. La creatura se deve fare una prova di Arrampicarsi per arrampicarsi su qualsiasi parete o pendenza può sempre scegliere di prendere 10, anche se di fretta o minacciata durante la salita.

Se una creatura con una velocità di Scalare tenta una scalata rapida (vedi sopra), è come se eseguisse una Azione di Scatto e fa una singola prova di Arrampicarsi a DC 13. Se la creatura non ha un punteggio di Scalare indicato si considera il valore pari al suo GS + Movimento in metri per Scalare.

Una creatura con Velocità di Scalare non ha penalità alla Difesa durante la salita e non ha penalità ai Tiri per Colpire mentre attacca.

Se non si ha il tipo di \textbf{movimento Scalare} si considera come \textbf{terreno doppiamente difficile}, e quindi ci si muove ad un quarto del del Movimento.

\subsubsection{Scavare}\index{Scavare}\label{scavare}

Una creatura con una velocità di Scavare può scavare tunnel attraverso la terra, ma non attraverso la roccia a meno che il testo descrittivo non dica il contrario. Le creature non possono caricare o correre mentre scavano.

La maggior parte delle creature scavatrici non lascia tunnel che altre creature possono utilizzare (sia perché il materiale attraverso cui scavano riempie il tunnel dietro di loro o perché in realtà non spostano materiale quando scavano), vedere la descrizione della singola creatura per i dettagli.

\subsubsection{Camminare - Velocità Su Terreno}

La Velocità sul terreno é la normale velocità per personaggi che non scalano, nuotano o volano.

\subsubsection{Volare}\index{Volare}\label{volare}

Volare per una creatura dotata di questa abilità è come camminare per una creatura terrestre. Una creatura dotata di volo usa le sua azioni per muoversi ma difficilmente sarà influenzata dal terreno difficile.

Una creatura volante che viene danneggiata in un singolo colpo della metà dei suoi Punti Ferita massimi deve fare un Tiro Salvezza su Tempra a DC 17 o cadere a terra.

\subsubsection{Fluttuare}\index{Fluttuare}\label{Fluttuare}\hypertarget{Fluttuare}{}

Fluttuare è la capacità che consente di rimanere fluttuante nell'aria, all'altezza voluta, anche se non ci si muove o si è privi di sensi.

\end{multicols}

\vfill

\begin{center}
\includegraphics[width=0.5\linewidth]{immagini/grifonicastello.png}
\end{center}

\pagebreak

