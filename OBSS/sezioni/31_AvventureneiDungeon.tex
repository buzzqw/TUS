\section{Avventure nei Dungeon}\index{Dungeon}

\begin{enfasi}{

\st{Linux} Dungeon is user friendly. It's just very picky about who its friends are. (anonimo)

\medskip

Il dungeon è inclinato. Le creature sono infuriate perché non riescono a giocare a biglie (Dungeon Keeper 2, Videogioco, 1999)

}\end{enfasi}

\label{avventure-nei-dungeon}
\begin{multicols}{2}

Di tutti i luoghi strani che un avventuriero può esplorare, nessuno è più letale di un dungeon. Questi labirinti, pieni di trappole mortali, mostri affamati e tesori meravigliosi, provano ogni abilità e capacità dei personaggi. Queste regole si possono applicare a qualsiasi tipo di dungeon, dal relitto di una nave ad un vasto complesso di grotte sotterranee.

\begin{narratore}[Dungeon!!!]
Il dungeon, caverna, catacomba, spelonca, ecosistema sotterraneo... chiamatelo come preferite è un cardine dell'avventura!

Un dungeon è una ricetta fatta di umidità, fetore, aria stantia, sporcizia, fango, resti di creature, trappole, melme, trappole (abbondate...), mostri, nemici, mostri (abbondare!), oscurità, rumori sinistri, funghi, scricchiolii, guaiti, urla, gemiti.. ma anche di paura, tensione, brivido terrore \& raccapriccio, enfasi, rabbia, dolore, delusione e tesori!!!

Il vostro dungeon non è mai solo una caverna. MAI!
\end{narratore}


Che siano caverne, antri, cave, grotte, tane, spelonche i \emph{Dungeon} rappresentano spesso il centro focale dell'avventura, dell'esplorazione e sopravvivenza.

I personaggi passeranno molto tempo in questi ambienti ed il Narratore deve essere preparato e pronto sull'ambiente che incontreranno.

Quando si prepara una caverna è necessario ragionare in maniera intelligente sul tipo di caverna e sulle creature che si andranno ad incontrare, ogni caverna è un complesso ecosistema.
Mettere un gruppo di lucertoloidi senza pensare a cosa mangiano, dove dormono, che tipo di organizzazione hanno è pericoloso, per non parlare di inserire una chimera.
Avrà le ali atrofizzate perché la caverna è alta 3 metri e larga 3 e fa fatica a muoversi ? Di cosa si è nutrita in questo periodo ? Piuttosto meglio usare una gorgone che si nutre di minerali...

Se progettato con attenzione e cura una caverna può diventare un'ottima esperienza di incontri, situazioni ed avventura.

\subsection{Il sottosuolo}\index{Il sottosuolo}

Le condizione naturali del sottosuolo dipendono da vari fattori ma ci sono sicuramente dei punti comuni a tutti.

\begin{itemize}[leftmargin=*] \setlength{\itemsep}{0pt}
\item Nessuna luce per illuminare gli spazi. Possono esserci sporadici funghi fluorescenti, che irradiano luce fioca nelle vicinanze, ma nulla che possa illuminare tutto l'ambiente
\item Ambiente umido
\item Temperatura ambiente solitamente fresca, raramente ci sono caverne con temperature estreme sia nel caldo che nel freddo.
\end{itemize}

\subsubsection{Illuminazione}\index{Illuminazione}

In una caverna non ci sono fonti di luci artificiali o naturali se non quelle introdotte da creature senzienti. Possono esserci gruppi di funghi, licheni, che illuminano debolmente il terreno dove crescono, entro 1 metro, ma null'altro intorno.
Oltretutto se strappati dal terreno perdono la bioluminescenza dopo 2d4 Turni.

Le creature che vivono nelle caverne si abituano all'oscurità sviluppando qualche forma di visione alternativa, quale scurovisione, senso tellurico o vista cieca.

Anche la stessa torcia può dare un sollievo limitato dato che il suo raggio di luce è di 3 metri più 3 metri di luce fioca e dura un ora prima di spegnersi.

Controlla le sezioni su \hyperlink{copertura}{Coperture} ed \hyperlink{invisibilita}{Invisibilità} (pag. \pageref{invisibilita}) per maggiori informazioni.

\subsubsection{Movimento}\index{Movimento}

Se non si hanno mezzi per vedere il terreno si considera difficile e buche, precipizi ed ostacoli vari possono essere molto pericolosi.\index{Terreno difficile in oscurità}

In caso di totale buio ed in un ambiente naturale va fatta una prova di Acrobatica a DC 12 ogni Azione di Movimento o inciampare e subire 1 di danno temporaneo.

\subsection{Tipologie di caverne}

Si possono individuare diverse tipologie di caverne:

\begin{itemize}[leftmargin=*] \setlength{\itemsep}{0pt}

\item \textbf{creati dallo scorrere dell'acqua}. In questo caso il tunnel può essere parecchio caotico nel suo dipanarsi a causa del tipo di rocce che l'acqua ha incontrato. Ci possono essere ancora fiumi e laghi sotterranei.

\item \textbf{creati dalla erosione}. In questo caso l'acqua probabilmente non c'è più se non in minima parte, le caverne risultanti possono essere anche molto grandi con sale di decine se non centinaia di metri di ampiezza.

\item possono essere stati \textbf{creati da un vulcano} con lo scorrere della lava. In questo caso il tunnel scavato dalla roccia è spesso lineare e in qualche maniera levigato, la lava una volta rappresa si è poi sbriciolata nei millenni.

\item possono essere \textbf{caverne artiche}, scavate nel ghiaccio dall'acqua. In questo caso valutate bene l'ambiente circostante e la temperatura gelida.

\item possono essere \textbf{caverne artificiali}, costruite da creature di diverso tipo.

\end{itemize}

\subsubsection{Le quattro tipologie di Dungeon}\index{Le quattro tipologie di Dungeon}

I quattro tipi base di dungeon sono definiti dal loro stato attuale. Molti dungeon sono varianti di questi tipi base o combinazioni di più tipi. Occasionalmente, antichi dungeon vengono usati da nuovi abitanti per scopi diversi.

\textbf{Struttura in Rovina}: Un tempo abitato, questo luogo è ora abbandonato (completamente o in parte) dai suoi creatori originari ed è occupato da altre creature. Molte creature sotterranee vanno alla ricerca di costruzioni sotterrane ed abbandonate in cui stabilire le loro tane. Qualsiasi trappola che possa essere esistita è stata probabilmente già rimossa o attivata, è possibile trovare bestie erranti.

\medskip
\begin{center}
\includegraphics[width=0.9\linewidth]{immagini/avventure_dungeon.png}

\emph{The Red Romance Book, Henry Justice Ford}
\end{center}
\medskip

\textbf{Struttura Occupata}: Questo dungeon viene ancora utilizzato. Delle creature (di solito intelligenti) ancora lo abitano, anche se potrebbero non essere i creatori del dungeon. Una struttura occupata potrebbe essere una casa, una fortezza, un tempio, una miniera attiva, una prigione, un quartier generale...

Questo tipo di dungeon è meno probabile che abbia trappole o bestie erranti, e più probabilmente dispone di guardie organizzate, sia di guardia che di pattuglia. Le trappole e le bestie erranti che si possono incontrare sono spesso sotto il controllo degli occupanti. Le strutture occupate dispongono di arredo adatto agli abitanti, così come decorazioni, riserve di cibo, e la possibilità per gli abitanti di muoversi.

Gli abitanti possono disporre anche di un sistema di comunicazione, e quasi sempre controllano almeno un accesso verso l'esterno.

Alcuni dungeon sono parzialmente occupati e parzialmente vuoti o in rovina. In questi casi, gli occupanti di solito non sono gli originari costruttori del luogo, ma bensì un gruppo di creature intelligenti che hanno stabilito la loro base, tana o fortificazione all'interno del dungeon abbandonato.

\textbf{Riparo Sicuro}: Quando qualcuno vuole proteggere una cosa, spesso la seppellisce sottoterra. Che l'oggetto che vuole proteggere sia un favoloso tesoro, un artefatto proibito o il cadavere di un uomo importante, questi oggetti di valore vengono posti all'interno di un dungeon e circondati da barriere, trappole e guardiani.

Il dungeon del tipo riparo sicuro è quello che avrà più trappole e meno bestie erranti. E' normalmente costruito in base alla funzionalità piuttosto che all'aspetto, anche se a volte viene decorato con statue e pareti dipinte, specie per le tombe di personaggi importanti.

%\begin{center}
%\includegraphics[width=0.7\linewidth]{immagini/dungeon.png}
%\end{center}

A volte, però, una sala del tesoro o una cripta vengono costruite in modo da ospitare guardiani viventi. Il problema con questa strategia è che occorre tenere in vita le creature tra un tentativo di intrusione e un altro. La magia è di solito la soluzione migliore per rifornire di cibo e acqua queste creature. I costruttori di tombe e sepolcri, di solito, pongono non morti e costrutti, che non hanno bisogno di sostentamento o di riposo, a protezione dei loro dungeon. Le trappole magiche possono attaccare gli intrusi convocando mostri nel dungeon che scompaiono quando terminano il loro compito.

\textbf{Complesso di Caverne Naturali}: Le caverne sotterranee offrono riparo a qualsiasi tipo di creatura delle profondità. Create naturalmente e collegate da un sistema di passaggi labirintici, queste caverne mancano di qualsiasi parvenza di ordine, logica o decorazioni. Senza alcuna potenza intelligente che lo abbia costruito, questo tipo di dungeon è quello che ha minori probabilità di presentare trappole o porte.

Molteplici varietà di funghi vivono nelle caverne, a volte crescendo fino a formare enormi foreste di funghi e vesce, dove si aggirano predatori sotterranei a caccia di chi si nutre di questi vegetali. Alcune varietà di funghi producono un bagliore fosforescente in grado di fornire al complesso di caverne naturali una propria limitata fonte di illuminazione. In altre zone, l'uso di incantesimi di Luce Diurna può garantire luce sufficiente per la crescita di piante verdi.

Spesso, un complesso di caverne naturali è collegato ad altri tipi di dungeon, essendo stato scoperto quando è stato costruito il dungeon artificiale. Un complesso di caverne può collegare due dungeon indipendenti, producendo a volte uno strano ambiente misto. Un complesso di caverne naturali unito a un altro dungeon spesso offre un percorso che le creature sotterranee possono usare per raggiungere un dungeon artificiale e popolarlo.

\subsection{Esplorazione}\index{Esplorazione}\index{Muoversi con attenzione}

Muoversi all'interno di un dungeon richiede attenzione e sangue freddo. Pavimenti accidentati, rumori sinistri, botole e trappole, luci che appaiono e scompaiono rendono non facile avventurarsi in sicurezza in questi ambienti pericolosi.

I personaggi dovranno stare attenti, cercare attivamente trappole, osservare in lontananza e tenere un atteggiamento prudente. Tutto questo significa che il movimento è dimezzato se i personaggi \emph{mettono in essere precauzioni} per evitare problemi, ovvero avere un minimo bonus alle prove di Consapevolezza.

Descrivere ciò che il personaggio fa per cercare trappole, passaggi.. \emph{problemi} o richiedere una prova (Sopravvivenza oppure Consapevolezza) a DC 13 può dare indicazioni generiche sulla \emph{sensazione} che ci sia qualcosa che non va.

\subsection{Terreno del Dungeon}\index{Terreno del Dungeon}

Le regole seguenti riguardano i terreni di base che si possono trovare in un dungeon.

\subsubsection{Pareti}\index{Pareti}\label{pareti}\hypertarget{pareti}{}

A volte pareti in mattoni (pietre accatastate una sopra tenute insieme con la calce) dividono i dungeon in corridoi e stanze. Le pareti dei dungeon possono anche essere scolpite nella nuda roccia, ottenendo così un aspetto scalpellato, oppure possono essere composte di pietra liscia e semplice come si trova nelle caverne naturali. Le pareti dei dungeon sono difficili da danneggiare o da sfondare, ma di solito sono facilmente scalabili.

\end{multicols}
\textbf{Tabella: Pareti}\index[Tabelle]{Tabella Pareti}
\medskip

\noindent\begin{tabularx}{0.98\textwidth}{lccccc}
	\toprule
\textbf{Tipo di Parete} & \textbf{Spessore} & \textbf{Sfondare} & \textbf{Durezza} & \textbf{Punti Ferita} & \textbf{DC Scalare}\\
\toprule
Mattoni di pietra & 30 cm& 35 & 8 & 90& 20\\
Mattoni di pietra superiori & 30 cm& 35 & 8 & 120 & 25\\
Mattoni di pietra rinforzati & 30 & 45 & 8 & 180 & 20\\
Pietra Scolpita & 90 & 50 & 8 & 540 & 25\\
Pietra grezza & 150 cm & 65 & 8 & 900 & 25\\
Ferro & 7.5 cm & 30 & 10& 90& 25\\
Carta & variabile & 1 & --& 1 & 30\\
Legno & 15 cm& 20 & 5 & 60& 21
\end{tabularx}

\medskip

\textbf{Tabella: Scavare un tunnel}\index[Tabelle]{Tabella Scavare un tunnel}

\medskip

\noindent\begin{tabular}{lccc}
	\toprule
\textbf{Minatore}&\multicolumn{3}{c}{\textbf{Materiale da Scavare (1 minuto)}}\\
&\textbf{Terreno}&\textbf{Pietra} \textbf{morbida}&\textbf{Pietra dura}\\
\toprule
Umano&50 cm&15 cm&7 cm\\
Gnomo &45 cm&30 cm&15 cm\\
Nano/Orco & 55 cm&45 cm&20 cm\\
Gigante della Pietra& 3 m& 1.5 m& 75 cm\\
Xorn &6 m&6 m& 6m\\
Elementale della Terra & 9 m&9 m&9 m
\end{tabular}

\medskip

Le distanze scavate indicate si presume che siano ottenute con strumenti idonei come vanghe o picconi, altrimenti ridurre ad un terzo.

\begin{multicols}{2}

\textbf{Pareti in Mattoni di pietra}: Il tipo più comune di parete per un dungeon, le pareti in pietre di solito sono spesse almeno 30 centimetri. Spesso queste antiche pareti presentano fori e fessure, all'interno dei quali possono annidarsi fanghiglie e piccole creature, che aspettano lì le loro prede. Le pareti di mattone di pietra sono in grado di bloccare tutti i rumori, tranne quelli più forti. E' necessaria una prova di Arrampicarsi con DC 20 per muoversi lungo una parete in mattoni.

\textbf{Pareti in Mattoni in Pietra di Qualità Superiore}: Queste pareti sono spesso abbellite da dipinti, bassorilievi o altre decorazioni. Le pareti in mattoni di qualità superiore non sono più difficili da danneggiare delle normali pareti in mattoni, ma sono più difficili da Arrampicarsi (DC 25).

\textbf{Pareti rinforzate} Queste sono pareti in mattoni con sbarre di ferro su uno o entrambi i lati, o inserite all'interno della parete stessa per rinforzarla. La Durezza della parete rinforzata resta la stessa, ma i Punti Ferita vengono raddoppiati e la DC per sfondarla viene incrementata di 10.

\textbf{Pareti di Pietra Scolpita}: Queste pareti generalmente si trovano in stanze o passaggi scavati nella nuda roccia. La ruvida superficie di una parete scolpita presenta minuscole sporgenze su cui possono crescere funghi e crepe all'interno delle quali possono vivere parassiti, pipistrelli o serpenti sotterranei.

%Quando una parete di questo tipo ha un altro lato (la parete separa due stanze in un dungeon), la parete è spessa almeno 90 centimetri; se fosse più sottile rischierebbe di far crollare tutto perché non sarebbe in grado di sostenere il peso della volta di pietra. E' necessaria una prova di Arrampicarsi con DC 25 per scalare una parete di pietra scolpita.

\textbf{Pareti di Pietra Grezza}: Queste superfici sono irregolari e raramente piatte. Di solito sono bagnate o perlomeno umide, in quanto le caverne naturali sono in genere il prodotto di infiltrazioni d'acqua. Quando una parete di questo tipo da un altro lato, la parete è di solito spessa almeno 150 centimetri.

E' necessaria una prova di Arrampicarsi con DC 15 per muoversi lungo una parete di pietra grezza.

\textbf{Pareti di Ferro}: Queste pareti sono poste all'interno dei dungeon intorno a luoghi importanti come le sale del tesoro.

%\textbf{Pareti di Stoffa}: Le pareti di stoffa sono l'opposto di quelle di ferro, utilizzate come schermi per impedire la vista ma nulla più.

\textbf{Pareti di Legno}: Le pareti di legno si trovano spesso come recenti aggiunte a dungeon più antichi, utilizzate per creare recinti per animali, depositi, o anche solo per dividere in una serie di stanze più piccole una più grande.

\textbf{Pareti Trattate Magicamente}: Queste pareti sono più forti della media, con una Durezza maggiore, con più Punti Ferita e per sfondarle bisogna superare una DC maggiore. La magia può di solito raddoppiare la Durezza e i Punti Ferita della parete e aggiungere fino a +20 alla sua DC per sfondarla. Una parete trattata magicamente ottiene anche un Tiro Salvezza contro Incantesimi che potrebbero avere effetto su di essa, con il bonus al Tiro Salvezza pari a 2 + metà del livello dell'incantatore della magia che rinforza la parete. Creare una parete magica richiede il talento Creare Oggetti Meravigliosi e la spesa di 1.500 mo per ogni sezione di 3 per 3 metri.

\textbf{Pareti con Feritoie}: Le pareti con feritoie possono essere costruite con qualsiasi materiale resistente, ma sono di solito fatte in mattoni, pietra scolpita o legno. Permettono ai difensori di scagliare frecce o quadrelli da balestra contro gli intrusi restando dietro la relativa protezione di un muro. Gli arcieri dietro alle feritoie godono di una Copertura superiore che fornisce loro bonus +8 alla Difesa, bonus +1d6 ai Tiri Salvezza su Riflessi.\index{Feritoie e frecce}

\subsubsection{Pavimenti}\index{Pavimenti}

Così come per le pareti, esistono molti tipi di pavimenti per dungeon.

\textbf{Lastricato}: Come le pareti in mattoni, i pavimenti possono essere composti da pietre incastrate tra loro. Sono di solito piene di fessure e solitamente appena livellate. Fanghiglie e muffe crescono all'interno di queste fessure. In certi casi l'acqua scorre in piccoli scoli attraverso le pietre o forma pozze stagnanti. Il lastricato è il tipo di pavimento più comune nei dungeon.

\textbf{Lastricato Irregolare}: Col passare del tempo, alcuni pavimenti possono diventare talmente irregolari da richiedere una prova di Acrobatica con DC 10 per correre o Caricare sulla loro superficie. Coloro che falliscono la prova non possono muoversi durante quel round. Pavimenti così pericolosi dovrebbero essere in realtà l'eccezione e non la regola.

\textbf{Pavimento di Pietra Scolpita}: Ruvidi e irregolari, i pavimenti scolpiti nella pietra sono di solito coperti da pietre smosse, ghiaia, polvere e altri detriti. Una prova di Acrobatica con DC 10 è necessaria per correre o Caricare su un simile pavimento. Un fallimento significa che il personaggio può ancora agire, ma non può correre o Caricare in quel round.

\textbf{Pietrisco Scarso}: Piccoli e sparuti detriti sono presenti a terra. Un pavimento su cui sia presente del pietrisco scarso aggiunge 2 alla DC delle prove di Acrobatica.

\textbf{Pietrisco fitto}: Il terreno è ricoperto di detriti di tutte le dimensioni. Il pietrisco si considera terreno difficile. Un pavimento cosparso di pietrisco fitto aggiunge 5 alla DC delle prove di Acrobatica, e aggiunge 2 alla DC delle prove di Consapevolezza contro Furtività.

\textbf{Pavimento di Pietra Liscia}: Pavimenti lisci, perfetti e a volte anche levigati si trovano solo nei dungeon creati da costruttori capaci e attenti.

\medskip

\begin{center}
	\includegraphics[width=0.9\linewidth]{immagini/pavimento_grey.png}
\end{center}

\medskip

\textbf{Pavimento di Pietra Naturale}: Il pavimento di una caverna naturale è irregolare quanto le pareti. E' difficile che queste caverne presentino ampie superfici piane; è più probabile che i loro pavimenti siano disposti su più livelli.

Alcune superfici potrebbero variare in elevazione di appena 30 centimetri, cosicché lo spostamento da un punto all'altro non sia più difficile del salire un gradino di una scala, ma in certi punti il pavimento potrebbe scendere o salire di oltre 1.5 metri, obbligando il personaggio a una prova di Arrampicarsi (pag \pageref{arrampicarsi}) per spostarsi da una superficie a un'altra.

A meno che non ci sia un percorso scavato dal tempo o ben battuto il terreno è considerato difficile e quindi il movimento è dimezzato, per praticità gradoni sotto i 50cm considerateli terreno difficile e quelli entro 1.5m terreno doppiamente difficile.\index{Gradini} La Carica e la corsa in questi ambienti sono impossibili, tranne che sui percorsi in questione.

\textbf{Scivoloso}: Acqua, ghiaccio, melma o sangue possono rendere qualunque pavimento descritto in questa sezione più insidioso. I pavimenti scivolosi aumentano la DC delle prove di Acrobatica di 5.

\textbf{Grata}: Una grata spesso copre una fossa o una zona al di sotto del pavimento principale. Le grate sono di solito costruite in ferro, ma quelle più grosse potrebbero essere anche fatte di tronchi d'albero rinforzati. Molte grate hanno cardini che permettono l'accesso alla zona sottostante (queste grate possono essere chiuse a chiave come una porta), mentre altre sono fisse e create per non poter essere spostate. Una tipica grata di ferro spessa 3 centimetri ha 25 Punti Ferita, Durezza 10, e DC 27 per sfondarla o smuoverla.

\textbf{Sporgenze}: Le sporgenze permettono alle creature di camminare al di sopra di un'area sottostante. Spesso sono disposte intorno a fosse, lungo il corso di fiumi sotterranei, come balconate che circondano un'ampia stanza oppure forniscono una posizione dalla quale gli arcieri possono appostarsi per attaccare i nemici dall'alto.

Le sporgenze strette (di ampiezza inferiore a 30 centimetri) richiedono a coloro che vi si muovono sopra, 3 Azioni di Movimento, delle prove di Acrobatica (DC 15). Un fallimento implica che il personaggio che si stava muovendo cade dalla sporgenza.

A volte le sporgenze hanno una ringhiera. In questi casi i personaggi ottengono bonus +1d6 alle prove di Acrobatica per muoversi lungo la sporgenza. Un personaggio vicino alla ringhiera ha Bonus +2 alla propria Prova Contrapposta di Forza per evitare di essere spinto giù dalla sporgenza.

\textbf{Pavimenti Trasparenti}: I pavimenti trasparenti, fatti di vetro rinforzato o di materiali magici permettono di osservare un ambiente pericoloso dall'alto. I pavimenti trasparenti sono di solito posti al di sopra di pozze di lava, arene, tane di mostri e stanze di tortura. Possono essere usati dai difensori per sorvegliare un'area.

\textbf{Pavimenti Scorrevoli}: Un pavimento scorrevole è un tipo di botola, creato per essere spostato e rivelare qualcosa che si trova al di sotto. In genere un pavimento scorrevole si muove tanto lentamente che chiunque vi si trovi sopra può evitare di cadere nell'apertura, purché abbia spazio per spostarsi. Se un pavimento di questo tipo scorre velocemente che c'è la possibilità che un personaggio cada in quello che si trova sotto di esso (lance acuminate, una vasca con olio bollente, o una pozza infestata da squali, acido...) allora si tratta come una trappola.

\textbf{Pavimenti Trappola}: Questi pavimenti sono stati progettati per diventare di colpo pericolosi. Con l'applicazione della giusta quantità di peso o l'azionamento di una leva nelle vicinanze, spuntoni sbucano dal pavimento, fiammate o sbuffi di vapore partono da fori nascosti, o l'intero pavimento si muove. Questi strani pavimenti si trovano di solito dentro alle arene, progettati per rendere i combattimenti più appassionanti e letali. Questo tipo di pavimento si gestisce come una trappola.

\subsection{Le porte}\index{Porte}

\noindent\begin{itemize}[leftmargin=*] \setlength{\itemsep}{0pt}
\item \textbf{Bloccata / Incastrata}: DC per Sfondare (TS Tempra con Forza, +1d6 se viene usato un piede di porco). Sfondare una porta a spallate/calci costa 1 Azione, 2 Azioni se si usa un piede di porco.\index{Azione Sfondare porte}\index{Azione Forzare porte}
\item \textbf{Chiusa a Chiave}: DC per Scassinare (prova di Disattivare Congegni).
\item \textbf{Non bloccata}: una porta non chiusa a chiave o bloccata richiede 1 Azione per aprirla oppure la si può aprire con l'Azione di Movimento usata per attraversarla.\index{Aprire una porta}
\end{itemize}

\medskip

Il \textbf{fallimento critico} in una prova di Forza (TS Tempra con Forza) significa essersi fatti male nella manovra di sfondamento. Finché non passano almeno 10 minuti non è più possibile sfondare una porta.\index{Fallire prova di forza sfondare porte}\index{Sfondare porte}

\index{Porte}Le porte all'interno dei dungeon sono ben più che semplici entrate o uscite. Spesso possono essere dei veri e propri incontri. Le porte dei dungeon si presentano in tre tipi basilari: di legno, di pietra e di ferro.

\end{multicols}

%\textbf{Tabella: Porte}
\index[Tabelle]{Tabella Porte}\index{Scassinare una porta}\label{tabellaporte}\hypertarget{tabellaporte}{}

\medskip

\noindent\begin{tabular}{llllll}
	\toprule
\textbf{Tipo di porta} & \textbf{Spessore tipico} & \textbf{Durezza} & \textbf{Punti Ferita} & \multicolumn{2}{c}{\textbf{DC per sfondare}} \\
&\textbf{(cm)}&&& \textbf{Bloccata} & \textbf{Chiusa a chiave}\\
\toprule
Legno semplice & 2.5& 5 & 10& 15 & 18\\
Legno buono& 3.75 & 5 & 15& 18 & 21\\
Legno robusto& 5& 5 & 20& 25 & 28\\
Pietra& 10 & 8 & 60& 31 & 34\\
Ferro & 5& 10& 60& 30 & 33\\
Saracinesca di legno & 7.5& 5 & 30& 27& 30\\
Saracinesca di ferro & 5& 10& 60& 28& 31\\
Serratura& -& 15& 30& -& -\\
Cardini & -& 10& 30& -& -
\end{tabular}

\medskip

\begin{multicols}{2}

\medskip

\textbf{Porte di Legno}\index{Porte di Legno}: Costruite con spesse assi inchiodate, a volte rinforzate con sbarre di ferro (poste anche per impedire le deformazioni prodotte dall'umidità dei dungeon), quelle di legno sono il tipo più comune di porta. Le porte di legno variano per durezza: possono essere semplici, buone o robuste. Le porte semplici (DC 15 per sfondarle) non sono progettate per tenere alla larga assalitori motivati.

Le porte di buona fattura (DC 18 per sfondarle), sebbene forti e resistenti, non sono comunque progettate per subire una grande quantità di danni. Le porte robuste (DC 25 per sfondarle) sono rivestite in ferro e sono delle barriere discretamente resistenti contro coloro che cerchino di oltrepassarle. Cardini di ferro sorreggono la porta e di solito un anello circolare posto al centro serve ad aprirla. A volte, al posto di un anello, una porta dispone di una sbarra di ferro su uno o entrambi i lati che funziona come maniglia.

Nei dungeon abitati queste porte sono di solito ben tenute (non bloccate) e non chiuse a chiave, anche se le zone importanti probabilmente saranno chiuse a chiave.

\textbf{Porte di Pietra}\index{Porte di Pietra}: Costruite da blocchi di pietra solida, queste porte pesanti e poco maneggevoli sono spesso pensate in modo da ruotare su se stesse quando vengono aperte, anche se i nani e altri abili artigiani sono in grado di costruire cardini forti abbastanza da sostenere il peso di una porta di pietra.

\begin{center}
	\includegraphics[width=0.9\linewidth]{immagini/porta_grey.png}
\end{center}

Le porte segrete nascoste lungo una parete di pietra sono solitamente di pietra. Altrimenti, le porte di questo tipo sono studiate per diventare resistenti barriere che proteggono qualsiasi cosa si trovi al di là di esse. Di conseguenza si trovano spesso chiuse a chiave o sbarrate.

\textbf{Porte di Ferro}\index{Porte di Ferro}: Arrugginite ma resistenti, le porte di ferro in un dungeon sono dotate di cardini come quelle di legno. Queste porte sono le porte più resistenti del tipo non magico. Sono di solito chiuse a chiave o sbarrate.

\textbf{Sfondare}\index{Sfondare porte}: Le porte dei dungeon possono essere chiuse a chiave, munite di trappole, rinforzate, sbarrate, sigillate magicamente o, a volte, semplicemente bloccate.

Tutti, ad eccezione dei personaggi più deboli, riusciranno a buttar giù una porta con un pesante attrezzo come un maglio, numerosi incantesimi ed oggetti magici possono offrire ai personaggi un modo facile per superare una porta chiusa.

\textbf{DC 13 o inferiore}: Una porta che chiunque può sfondare.

\textbf{DC 13--15}: Una porta che una persona forte dovrebbe sfondare con un solo tentativo, e che una persona di forza media potrebbe avere qualche speranza di abbattere in un solo colpo.

\textbf{DC 16--20}: Una porta che praticamente chiunque potrebbe sfondare, avendo a disposizione il tempo necessario.

\textbf{DC 21--25}: Una porta che solo una persona forte o molto forte ha una speranza di sfondare, e probabilmente non al primo tentativo.

\textbf{DC 26 o superiore}: Una porta che solo una persona dotata di una forza eccezionale può avere una qualche speranza di sfondare.

\textbf{Serrature}\index{Serratura}: Le porte dei dungeon sono spesso chiuse a chiave e così torna utile la competenza Disattivare Congegni. Le serrature sono incassate sul bordo opposto ai cardini o dritte nel centro della porta. Le serrature di solito controllano una sbarra di ferro o legno che si estende dalla porta dentro il muro che la sostiene.

I lucchetti fissano tra due anelli, uno sulla porta e uno sul muro. Serrature più complesse, come quelle a combinazione o quelle ad enigma, sono di solito costruite dentro la porta stessa.

La DC per scassinare una serratura con una prova di Disattivare Congegni spesso ricade tra 15 e 30, anche se esistono serrature con DC maggiori o inferiori. Una porta può disporre di più di una serratura, ognuna delle quali da aprire separatamente.\index{Scassinare una porta}. Scassinare serratura senza attrezzi da scasso comporta una penalità di -1d6 alla prova.\index{Scassinare senza attrezzi}\hypertarget{Attrezzi da scasso}{}

Un Fallimento Critico nell'apertura di un porta o lucchetto causa la rottura degli attrezzi da scasso.\index{Rompere gli attrezzi da scasso}\index{Fallire l'apertura di un lucchetto}

Le serrature sono spesso dotate di trappole, di solito aghi avvelenati che scattano all'infuori per pungere le dita del ladro.

\subsubsection{Porte, passaggi ed aperture}\index{Porte, passaggi ed aperture}

Una porta speciale potrebbe avere una serratura senza chiave, ma che richiede che venga indovinata la giusta combinazione delle leve vicine o vengano premuti nell'ordine corretto i simboli su un pannello per riuscire ad aprirla.

\textbf{Porte Bloccate}: I dungeon sono spesso luoghi umidi, e in alcuni casi le porte rimangono bloccate, in modo particolare se sono fatte di legno. Di solito si suppone che all'incirca 1 su 6 delle porte di legno e il 1 su 10 delle altre porte siano bloccate. Questi valori possono essere raddoppiati (al 2 su 6 e 2 su 10 rispettivamente) nel caso di dungeon da tempo abbandonati o trascurati.

\textbf{Porte Sbarrate}: Quando un personaggio cerca di sfondare una porta sbarrata, è la qualità della sbarra che fa la differenza, non il materiale della porta in sé. Sfondare una porta chiusa da una sbarra di legno richiede un Tiro Salvezza Tempra con Forza con DC 25, e la DC sale a 30 nel caso di una sbarra metallica.

I personaggi possono attaccare la porta e distruggerla, lasciando la sbarra appesa nel passaggio sgombro. Usare un piede di porco per forzare una porta incastrata/bloccata concede un +1d6 alla prova.\index{Piede di Porco su porta}

\textbf{Sigilli Magici}: Incantesimi messi su una porta possono rendere ostico l'attraversamento di una porta.

Una porta su cui è stato lanciato un blocco magico si considera chiusa anche se non ha fisicamente una serratura. E' necessario un incantesimo che scassina o dissolve le magie oppure una prova riuscita di sfondare per oltrepassare una porta chiusa in questo modo.

\textbf{Cardini}: La maggior parte delle porte è dotata di cardini. Ovviamente le porte scorrevoli non lo sono (queste sono piuttosto dotate di solchi sul pavimento, che permettono loro di scorrere a lato con facilità).

Gli avventurieri possono rimuovere i cardini uno alla volta superando varie prove di Disattivare Congegni (solo se, naturalmente, sono davanti al lato della porta su cui si trovano i cardini). Una simile azione ha una DC di 20, in quanto molti dei cardini sono arrugginiti o bloccati.

Spaccare un cardine è difficile. La maggior parte ha Durezza 10 e 30 Punti Ferita. La DC per spaccare un cardine è la stessa che serve per abbattere la porta

%\begin{center}
%\includegraphics[width=0.8\linewidth]{immagini/cardini.png}
%\end{center}

\textbf{Cardini a Inserimento}: Questi cardini sono molto più complessi e si trovano solo in zone di eccellente costruzione. Questi cardini sono costruiti dentro la parete e permettono alla porta di aprirsi in entrambe le direzioni. I personaggi non possono raggiungere i cardini per rimuoverli a meno che non sfondino il sostegno della porta o la parete. I cardini a inserimento si trovano di solito sulle porte di pietra, ma a volte si vedono anche su porte di legno o di ferro.

\textbf{Perni}: I perni non sono veri cardini, ma semplici pioli che si protendono dal lato superiore e inferiore della porta e si infilano dentro i buchi nel suo sostegno, permettendole di girare. I vantaggi dei perni è che non possono essere rimossi come i cardini e che sono facili da realizzare. Lo svantaggio è che siccome la porta gira sul suo centro di gravità (di solito nel mezzo), nulla più grosso di metà dell'ampiezza della porta vi può passare attraverso.

Le porte dotate di perni sono di solito di pietra e spesso anche abbastanza larghe per ovviare allo svantaggio. Un'altra soluzione è quella di piazzare il perno verso un'estremità e fare la porta più spessa da quella parte e più sottile dall'altra, in modo che si apra più o meno come una porta normale.

Le porte segrete all'interno di muri spesso ruotano, in quanto la mancanza di cardini rende più facile occultare la presenza della porta. I perni permettono anche a oggetti come una libreria di essere usati come porte segrete.

\textbf{Porte Segrete}: Camuffata da comune porzione di muro (o di pavimento o di soffitto), da libreria, da focolare, da fontana, una porta segreta porta ad un passaggio segreto oppure ad una stanza.

Qualcuno che stia esaminando la zona può trovare una porta segreta (se ne esiste una) con una prova riuscita di Consapevolezza (con DC 20 per una porta segreta comune e DC 30 per una porta molto ben nascosta).

Molte porte segrete richiedono un metodo speciale per essere aperte, come un bottone nascosto o una piastra a pressione. Le porte segrete possono aprirsi come porte comuni, girare su un perno, scorrere, sprofondare, sollevarsi o anche calare come un ponte levatoio.

Un costruttore potrebbe piazzare una porta segreta molto bassa vicino al pavimento oppure molto in alto su un muro, in modo da rendere più difficile sia il rinvenimento che l'utilizzo della porta.

\textbf{Porte Magiche} Incantata dal costruttore originario, una porta può apostrofare gli esploratori invitandoli a non proseguire. Potrebbe essere protetta dai danni, con una Durezza maggiore o un numero maggiore di Punti Ferita, oltre che un bonus al Tiro Salvezza migliorato. Una porta magica potrebbe non condurre allo spazio che si trova dietro di essa, ma essere in realtà un portale verso un luogo molto distante o addirittura verso un altro piano di esistenza. Altre porte magiche potrebbero aver bisogno di una parola d'ordine o di chiavi speciali per aprirsi.
Le porte magiche sono apribili solo tramite comando specifico o annullando la magia che le pervade, pochissime hanno una serratura.
In tal caso il Narratore potrebbe decidere di aumentare la prova di Disattivare Congegni di 10, portandola a 30 o più e potrebbe essere necessario avere qualche punto in Arcana.

\begin{center}
	\includegraphics[width=0.8\linewidth]{immagini/arcoserpenti.png}

	\emph{Henry Justice Ford}
\end{center}

\textbf{Saracinesche}: Queste porte speciali sono fatte con aste di ferro o di spesso legno rinforzato che calano da un recesso nella parte superiore di un arco. A volte una saracinesca dispone di barre orizzontali a formare una griglia, altre volte no. Sollevate di solito con un argano o simile macchinario, le saracinesche possono esser fatte scendere in fretta, e le sbarre terminano in punte per scoraggiare chiunque dal passarci sotto (o dal tentare di attraversarle in corsa mentre calano). Una volta scesa, una saracinesca si chiude, a meno che non sia così grande che nessuna persona normale sarebbe in grado di sollevarla. In ogni caso, sollevare una tipica saracinesca richiede un Tiro Salvezza Tempra con Forza con DC 25.

\textbf{Pareti, Porte ed azioni di Individuazione}

Le pareti di pietra, di ferro e le porte di ferro sono generalmente sufficientemente spessi da bloccare la maggior parte delle Divinazioni. Le pareti di legno, le porte di legno e di pietra in genere non sono sufficientemente spesse da fare altrettanto. Tuttavia, una porta segreta di pietra costruita in un muro e spessa come il muro stesso (almeno 30 centimetri) bloccherà la maggior parte di queste Azioni.

\subsection{Pericoli nei Dungeon}\index{Pericoli nei Dungeon}

Nei dungeon e nelle caverne oltre ai mostri ci sono anche altri pericoli tra crolli, muffe, funghi e altro.

\subsubsection{Crolli e Cedimenti (grado di Sfida 8)}\index{Crolli e Cedimenti}

I crolli e i cedimenti nei tunnel sono estremamente pericolosi. Non solo gli esploratori di dungeon corrono il rischio di essere schiacciati da tonnellate di pietra ma anche qualora dovessero sopravvivere, di rimanere bloccati sotto un mucchio di detriti o di essere impossibilitati a raggiungere un'uscita.

Un crollo seppellisce chiunque si trovi nel mezzo della zona sepolta e i detriti che rotolano via infliggeranno danni a tutti coloro che si trovano nelle zone attorno alla zona sepolta. Un tipico corridoio soggetto a un crollo potrebbe avere una zona sepolta con raggio 3 metri e una zona di scorrimento detriti con raggio di 1 metro all'estremità di quella sepolta.

Un soffitto pericolante può essere identificato con una prova di Conoscenza Ingegneria con DC 20 o Professione Muratore con DC 20.

Un soffitto pericolante può crollare sotto l'impatto di una grossa forza. Un personaggio può provocare un crollo distruggendo la metà dei pilastri che reggono il soffitto.

I personaggi che si trovano nella zona sepolta subiscono 8d6 danni o danni dimezzati se superano un Tiro Salvezza su Riflessi con DC 15 e sono sepolti. I personaggi nella zona ai bordi subiscono 3d6 danni o nessun danno se superano un Tiro Salvezza su Riflessi con DC 15. I personaggi che si trovano nelle zone ai bordi sono anch'essi sepolti per 1 quadretto se falliscono il Tiro Salvezza.

I personaggi sepolti subiscono 1d6 danni non letali per ogni minuto che rimangono sotto le macerie. Se un personaggio in queste condizioni cade privo di sensi, deve effettuare un Tiro Salvezza su Tempra con DC 15, se fallisce la prova, inizia a subire 1d6 danni letali al minuto fino a quando non viene liberato o muore.\index{Sepolti vivi}

I personaggi che non sono stati sepolti possono estrarre i loro compagni da sotto le macerie. In 1 minuto una creatura, usando solo le mani, libera un quarto di un quadretto di macerie, se usa degli strumenti adatti, come un piccone, vanga o una pala può scavare mezzo quadretto al minuto. Un personaggio sepolto può tentare di liberarsi da solo superando un Tiro Salvezza Tempra con Forza con DC 30, una volta al minuto.

\subsubsection{Fanghiglie, Muffe e Funghi}\index{Fanghiglie, Muffe e Funghi}

Negli umidi e oscuri recessi dei dungeon, le muffe e i funghi prosperano, temete le colonne di muffa! Per quanto riguarda gli incantesimi ed altri effetti speciali, tutte le fanghiglie, le muffe e i funghi sono considerati vegetali. Come le trappole, le fanghiglie e le muffe pericolose sono dotate di un grado di Sfida e i personaggi guadagnano Punti Esperienza per averle incontrate.

Una lucida melma organica ricopre qualsiasi cosa che rimanga per troppo tempo immersa nell'oscurità e nell'umidità dei dungeon. Questo tipo di fanghiglia, benché possa essere repellente, non è pericolosa. Le muffe e i funghi abbondano nei luoghi bui, freddi e umidi. Sebbene alcuni siano innocui quanto le normali fanghiglie dei dungeon, altri sono alquanto pericolosi. Funghi commestibili, vesce, lieviti, muffe e altri tipi di funghi fibrosi, bulbosi o intere distese di spore fungine possono essere rinvenuti nella maggior parte dei dungeon. Di solito sono innocui e spesso sono anche commestibili (anche se la maggior parte è poco invitante o ha uno strano sapore).

\begin{center}
	\includegraphics[width=0.95\linewidth]{immagini/funghi.png}

	\emph{Sono luminosi al buio, credetemi! e fritti sono ancora meglio!}
\end{center}

\textbf{Boleto Stridente}\index{Boleto Stridente}: Questi funghi viola di grandezza umana emettono un suono penetrante che dura 1d3 round ogni volta che c'è un movimento o una sorgente di luce entro raggio 3 metri. Questo grido rende impossibile sentire altri suoni o rumori entro raggio di mischia. Il suono attira le creature nelle vicinanze che sono disposte ad investigare. Alcune creature che vivono vicino ai boleti stridenti hanno imparato che il rumore significa molto spesso cibo.

\textbf{Fanghiglia Verde}\index{Fanghiglia Verde} (grado di Sfida 4): Questo pericolo dei dungeon è una varietà insidiosa della normale fanghiglia.
La fanghiglia verde divora la carne e i materiali organici che vi entrano in contatto ed è addirittura capace di dissolvere i metalli. Di un verde splendente, bagnata ed appiccicosa, si distribuisce a chiazze su pareti, pavimenti e soffitti e si riproduce consumando materiale organico. Si lascia cadere dalle pareti e dai soffitti quando individua del movimento (e possibile nutrimento) sotto di sé.

La fanghiglia verde infligge 1 danno alla Costituzione per ogni round in cui divora la carne. Al primo round di contatto, la fanghiglia può essere asportata da una creatura (con la probabile distruzione dell'oggetto utilizzato per asportarla), ma dopo il primo round deve essere congelata, bruciata o tagliata (infliggendo danni anche alla sua vittima) per essere rimossa. Tutto ciò che infligge danni da fuoco o da freddo, la luce solare o un incantesimo di rimuovi malattia distruggono una chiazza di fanghiglia verde. Nel caso di legno o metallo, la fanghiglia verde infligge 2d6 danni per round, ignorando la Durezza del metallo ma non quella del legno. Non danneggia la pietra. Difesa 10, Punti Ferita 30, Tiri Salvezza T 3, R 0, V 1.

\textbf{Fungo Fosforescente}\index{Fungo Fosforescente}: Questo strano fungo sotterraneo emana una debole luminescenza violacea che illumina le caverne e i passaggi sotterranei

come una candela. Rare macchie di questo fungo illuminano come una torcia. Strappato dal suo ambiente si spegne in un 1d4 turni.

\textbf{Muffa Gialla} \index{Muffa Gialla}(grado di Sfida 6): Se disturbata nel raggio di 3 metri rilascia una nube di spore velenose. Tutti coloro entro raggio di 3 metri dalla muffa devono superare un Tiro Salvezza su Tempra con DC 15 o subiscono 1d3 danni a Costituzione. Un altro Tiro Salvezza su Tempra con DC 15 è necessario una volta per round per i successivi 5 round o per evitare di subire altri 1d3 danni a Costituzione. Un Tiro Salvezza riuscito blocca questo effetto. Il fuoco distrugge la muffa gialla, mentre la luce solare la rende inerte. Difesa 10, Punti Ferita 25, Tiri Salvezza T 3, R 0, V 1, Vulnerabilità al Fuoco.

\textbf{Muffa Marrone} \index{Muffa Marrone}(grado di Sfida 2): La muffa marrone si nutre di calore, estraendolo da tutto ciò che la circonda. Di solito si presenta in chiazze con diametro di dimensione di 1 metro e la temperatura attorno alla muffa risulta sempre fredda in un raggio di 3 metri. Le creature viventi entro 1 metro da essa subiscono 3d6 danni non letali da freddo. Se viene portata una fonte di fuoco entro 1 metro dalla muffa questa raddoppia immediatamente le proprie dimensioni. I danni da freddo, come quelli inflitti da un cono di freddo, la distruggono all'istante. Difesa 10, Punti Ferita 12, Tiri Salvezza T 3, R 0, V 1, Vulnerabilità Freddo, converte i danni da fuoco subiti in Punti Ferita.

\subsubsection{Esempio di Trappole da dungeon}\index{Esempio di Trappole da dungeon}

Viene indicato il nome della trappola, la DC per la prova di Sopravvivenza per trovare la trappola e le indicazioni d'uso della stessa. Vedi anche \hyperlink{trappoleesempio}{Trappole} (pag. \pageref{trappoleesempio}).

\medskip

\textbf{Stanza allagata, DC 17}: se i personaggi non notano la piastra a pressione sul pavimento questa farà sigillare la porta di ingresso e la stanza incomincerà a riempirsi d'acqua.
La stanza si riempie d'acqua in 10 round. Una prova di Sopravvivenza a DC 15, combinata con una prova di Nuotare DC 13, fa rilevare la piastra che attiva la fuoriuscita d'acqua.

\textbf{Stanza stritolante, DC 15}: se i personaggi non notano la piastra a pressione sul pavimento questa farà sigillare la porta di ingresso e fortissimi rumori di stridii ed ingranaggi riempiranno la stanza. Le pareti incominceranno ad avvicinarsi tra loro come il soffitto al pavimento. Se i personaggi non trovano la mattonella nascosta (DC 17) subiranno 10d6 di danno da stritolamento. La trappola è più facile da rilevare di altre perché le pareti sono più spesse rendendo la stanza più piccola.

\textbf{Soffitto schiacciante, DC 18}: se i personaggi non notano il sistema di attivazione ( piastra a pressione, cavo, raggio di luce interrotto..) una sezione di soffitto di 3m x 3m cadrà sui i personaggi con un danno di 3d6.

\textbf{Tunnel di ragnatele, DC 12}: questo tunnel è evidentemente pieno di ragnatele fitte, dense, robuste. Se i personaggi entrano si considerano Intralciati. Dopo 1d4 round di permanenza un attivatore genererà una scintilla dando fuoco alle ragnatele per 1d4 round. Ogni round all'interno del tunnel si subiscono 2d4 di danno da fuoco.

\textbf{Fossa, DC 15}: il personaggio disattento farà crollare una sezione di 3m x 3m di pavimento su una fossa. Questa può essere una semplice fossa (1d6 di danno da caduta), con spuntoni (1d6+2d4), con acido (1d6 per round), con non morti...

\textbf{Garrotte, DC 14}: questa trappola può essere molto insidiosa. Un filo affilato magicamente è a 1 metro da terra, tra una parete e quella opposta e scorre verso i giocatori.
E' necessario un Tiro Salvezza su Riflessi DC 14 oppure subire 2d6 di danno da taglio.

\textbf{Porta schiacciante, DC 16}: questa porta appena toccata rotea su dei cardini centrali e roteando picchia il personaggio (o personaggi se un grande portone). Causa 1d6 di danni contundenti e continua a roteare per 1d6 round.

\textbf{Trincia Dito, DC 14}: questa trappola è molto subdola. Si presenta con un foro di circa 1 cm di diametro e profondo 7 cm. Qualsiasi cosa che ne tocchi il fondo farà scattare la trappola, causando 2d4 di danno al dito/oggetto inserito. La lama potrebbe anche essere avvelenata.

\end{multicols}

\vfill

\begin{center}
\includegraphics[width=0.35\linewidth]{immagini/GBP14.png}

\emph{Carceri d'invenzione, XIV / The Gothic Arch, Giovanni Battista Piranesi}
\end{center}

\pagebreak

