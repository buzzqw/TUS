\section{Masterizzare}\index{Masterizzare}\index{Narratore}

\label{masterizzare}

\begin{changemargin}{0.3cm}{0.3cm}\begin{enfasi}{
Chi comanda al racconto non è la voce: è l'orecchio. (Italo Calvino)

\medskip

Per fare ciò che si vuole bisogna nascere re o stupidi. (\emph{o Narratore}, NdA) (Lucio Anneo Seneca)

\medskip

%"Il gioco di ruolo di Dungeons\& Dragons (\emph{ed anche OBSS}) è sulla narrazione in mondi di spade e stregoneria. Condivide elementi con i giochi di finzione dell'infanzia. Come quei giochi, D\&D è guidato dall'immaginazione. Si tratta di immaginare l'imponente castello sotto il tempestoso cielo notturno e immaginare come un avventuriero fantasy potrebbe reagire alle sfide che la scena presenta." (DnD 5e Basic Rules)

%\medskip

"Non è compito del solo DM (\emph{Narratore}) intrattenere i giocatori e assicurarsi che si divertano. Ogni persona che gioca è responsabile del divertimento del gioco di tutti. Tutti accelerano il gioco, aumentano il dramma, aiutano a stabilire quanto il gruppo si sente a suo agio nel gioco di ruolo e danno vita al mondo di gioco con la loro immaginazione. Tutti dovrebbero anche trattarsi reciprocamente con rispetto e considerazione: i litigi personali e le liti tra i personaggi ostacolano il divertimento.

Persone diverse hanno idee diverse su ciò che è divertente in D\&D. Ricorda che il \emph{modo giusto} di giocare a D\&D è il modo in cui tu e i tuoi giocatori siete d'accordo e vi divertite. Se tutti si mettono al tavolo pronti a contribuire al gioco, tutti si divertono". (Dungeon Master Guide, 4ed)

}\end{enfasi}\end{changemargin}\medskip

\begin{multicols}{2}

\subsection{Il Narratore}

\label{il-narratore}

Mentre il giocatore interpreta un personaggio in un'avventura, il Narratore è colui che la gestisce. Ha certamente molto più lavoro, ma creare un mondo intero affinché i propri amici lo esplorino, può dare molte soddisfazioni.

Il ruolo del Narratore non è facile ma concede enormi privilegi. Vedere i propri amici giocare, divertirsi, ammattirsi dietro dubbi, indovinelli e situazioni da te create dà tantissimo divertimento e momenti di vera convivialità.

Il tuo ruolo è quello del grande orchestratore, pianificatore o anche paesaggista se preferisci, con poche semplici pennellate delinei la struttura e saranno poi i giocatori ad aggiungere dettagli e situazioni.

\begin{changemargin}{0.3cm}{0.3cm}\begin{narratore}
OBSS vuole aiutare te e gli altri giocatori a divertirsi. Usa sempre il buon senso quando devi applicare una regola. Il tuo scopo non è ammazzare i personaggi ma creare mondi e campagne che si evolvono attorno ai personaggi ed al mondo che crei, alle loro azioni e decisioni. Incorpora le cose che interessano i giocatori, tienili partecipi, fagli comprendere che il mondo è vivo e ne fanno parte. Se sei bravo le tue avventure, le situazioni riecheggeranno in altre sessioni e fuori dal tavolo.
\end{narratore}\end{changemargin}

Il tuo \emph{lavoro e divertimento} è fondamentale ed importantissimo, la bontà della sessione di gioco dipende anche da te. Il tuo scopo è innanzitutto divertirti, essere creativo, improvvisare, recitare, creare ingegnose situazioni. Finché tu ti diverti è estremamente probabile che anche i giocatori si stiano divertendo!

\textbf{Ricorda che non sei tu il protagonista ne l'avventura, ma i personaggi}, non rubare la scena ma come un gran ballo sii il direttore d'orchestra dove gli strumenti sono le possibilità offerte dal OBSS, la musica è l'avventura ed i ballerini i personaggi.

\subsection{Punti Esperienza}\index{Punti Esperienza}\index{PX}

\label{punti-esperienza}

In OBSS i Punti Esperienza che prendono i personaggi servono a determinare il livello e quindi le capacità ed abilità a loro disposizione.

I personaggi prenderanno Punti Esperienza in base ai mostri sconfitti ma anche ad altri fattori quali obiettivi, idee, azioni particolari, difficoltà superate.. ma anche tesori recuperati!

Il suggerimento principale è premiare i personaggi che più si sono impegnati per il gruppo, quelli che maggiormente hanno contribuito al buon esito dell'avventura e della sessione. I Punti Esperienza non misurano solo il successo ma anche la partecipazione al gioco.
E' quindi possibile avere personaggi con Punti Esperienza diversi e potenzialmente anche livelli diversi.

I Punti Esperienza che assegna la sconfitta di un mostro sono indicati nel Mostruario es. Sfida 13 (10000 PX). Questi Punti Esperienza vanno divisi tra tutti i personaggi che hanno partecipato allo scontro in qualsiasi maniera.

La Tabella Punti Esperienza per Livello indica i Punti Esperienza necessari per passare da un livello a successivo.

Non esagerate mai nell'assegnazione dei Punti Esperienza altrimenti rischierete di sbilanciare il gioco e dover modificare anche in maniera significativa l'avventura. Siate chiari anche con i giocatori all'inizio della campagna, nella Sessione Zero, come i Punti Esperienza saranno calcolati, distribuiti e cosa è possibile fare per poterne avere di più.



%\begin{tabularx}{0.45\textwidth}{lX|lX}
%\textbf{Livello} & \textbf{Punti Esperienza}&\textbf{Livello} & \textbf{Punti Esperienza}\\
%\toprule
%1&0&11&122970\\
%2&1610&12&198960\\
%3&2620&13&321920\\
%4&4240&14&600000\\
%5&6850&15&520860\\
%6&11090&16&842750\\
%7&17940&17&1363570\\
%8&29030&18&2206260\\
%9&46970&19&3569730\\
%10&76000&20&5775820\\
%&+prec*1.680&&\\
%\end{tabularx}

\medskip

Non dovete però tenere conto solo dei Punti Esperienza concessi dalle sfide ma dovete valutare i personaggi e gruppo durante la sessione.

Ogni qual volta il personaggio od il gruppo:\index{Bonus Punti Esperienza}\label{puntiesperienzabonus}\index{Punti Esperienza Bonus}
\begin{itemize}[leftmargin=*] \setlength{\itemsep}{0pt}
\item \textbf{Raggiunga gli obiettivi prefissati} (premio al gruppo od al personaggio);
\item \textbf{Sfrutti a pieno ed anzi sia alternativo nell'uso delle proprie Abilità e capacità} (premio al personaggio);
\item \textbf{Risolva i problemi in maniera creativa, fantasiosa e funzionale} (premio al personaggio);
\item \textbf{Proponga piani e azioni e funzionanti ed alternative a quanto previsto} (premio al personaggio);
\item \textbf{Scopra o avvii indizi di avventura e creazione di nuovi plot} (premio al personaggio);
\item \textbf{Usi in maniera intelligente e furba una competenza od oggetto} (premio al personaggio);
\item \textbf{Usi in maniera geniale (ed alternativo) un incantesimo} (premio al personaggio);
\item \textbf{Compia una azione che mette a repentaglio la propria vita per il gruppo} (premio al personaggio);
\item \textbf{Compia azioni seguendo il credo del proprio Patrono (per Devoti). Queste dovrebbero dare punti Tratto} (premio al personaggio);
\item \textbf{Converta un PNG, di livello equivalente, al suo Patrono (solo per Devoti)} (premio al personaggio);
\item \textbf{Raccolga almeno 500*Livello in monete d'oro (o tesoro equivalente)} (premio al gruppo, una volta per sessione massimo);
\end{itemize}

\medskip

\textbf{Tabella: Punti Esperienza per Livello}\index[Tabelle]{Tabella Punti Esperienza per Livello}\label{tabellapuntiesperienza}

\begin{tabularx}{0.45\textwidth}{lX|lX}
	\textbf{Livello} & \textbf{PX}&\textbf{Livello} & \textbf{PX}\\
	\toprule
	1&0&11&300000\\
	2&2000&12&390000\\
	3&8000&13&490000\\
	4&15000&14&600000\\
	5&35000&15&740000\\
	6&60000&16&890000\\
	7&90000&17&1050000\\
	8&120000&18&1250000\\
	9&170000&19&1470000\\
	10&220000&20&1730000\\
	&+prec*0.2&&
\end{tabularx}

\medskip

Vi suggerisco anche di valutare queste azioni per premiare l'impegno del giocatore
\begin{itemize}[leftmargin=*] \setlength{\itemsep}{0pt}
\item \textbf{Sia collaborativo con gli altri giocatori} (premio al gruppo od al personaggio);
\item \textbf{Aiuti un giocatore in difficoltà} (premio al personaggio od al gruppo);
\end{itemize}

\medskip

\textbf{Concedete 200 Punti Esperienza * Livello del personaggio o Livello medio del gruppo.}

\medskip

%Concedete in Punti Esperienza 1\% dei Punti Esperienza del livello successivo (segnatevi su un foglio le volte che il premio viene guadagnato per poi sommare i Punti Esperienza una sola volta).\index{Premio Punti Esperienza}

Un ulteriore approccio, ma da considerare solo nei gruppi più affiatati e maturi, è a fine sessione chiedere ai giocatori di scegliere chi tra loro ha giocato meglio, in un insieme di ruolo, ispirazione, incisività e collaborazione. Premiate il suo personaggio con 200 PX per Livello del Personaggio.

Questi Punti Esperienza potranno essere assegnati al gruppo e quindi a tutti i personaggi od al singolo personaggio.
Non c'è bisogno di dare i Punti Esperienza a fine sessione di gioco, tenetene traccia e informate i giocatori quando c'è un momento di pausa, di riflessione su quanto accaduto e fatto.
In questo sistema sono necessarie circa 8/12 sessioni per passare di livello, potenzialmente anche molte meno se i giocatori si dimostrano bravi ed affrontano le situazioni in maniera efficace.

Costruite la sessione perché tutti i personaggi possano essere partecipi e nessuno si senta escluso.

%\begin{center}
%\includegraphics[width=0.7\linewidth]{immagini/deathbeowulf.png}

%\emph{Henry Justice Ford}
%\end{center}

Quando dico \emph{incontro} non pensate al semplice scontro con i mostri, per incontro si intende qualsiasi evento di ruolo che sfidi e metta alla prova i personaggi. Questa sfida può essere una arguta discussione con il nobile che non li vuole pagare al termine di una missione, alla sfida di un indovinello, rebus, delle trappole ben piazzate. In base alla difficoltà della sfide ricavate i Punti Esperienza.

Un mostro non deve essere per forza ucciso per averne i Punti Esperienza, è sufficiente sconfiggerlo, catturarlo, vincere in maniera diversa. In caso di ritirata da parte dei personaggi o nemico accordate la metà dei Punti Esperienza previsti per lo scontro se c'è almeno stato il tentativo di sfida.

Nel limite del possibile ogni sessione dovrebbe includere una parte di ruolo, una parte di esplorazione, tre parti di combattimento (anche molte più di tre), una parte di riposo.

\medskip

\begin{changemargin}{0.3cm}{0.3cm}\begin{narratore}
Sia chiaro che nulla vi vieta di approntare un passaggio di livello basato su punti fissi (milestone) durante l'avventura. Vostro il tavolo, vostre le regole!
\end{narratore}\end{changemargin}

\medskip

\begin{changemargin}{0.3cm}{0.3cm}\begin{narratore}
Può sembrare anacronistico quando c'è già in sviluppo la sesta edizione del più famoso gioco di ruolo tornare a premiare i personaggi in base all'oro preso ai mostri.

Posso però garantirvi che qualora il vostro gruppo sia particolarmente \emph{povero} di giocate di ruolo o semplicemente preferisca uno stile più combattivo, sapere che l'oro raccolto equivale ad esperienza può rendere molto più dinamico ed avvincente l'andare in avventura.

OBSS si rifà ai principi dell'OSR e come tale la fase di esplorazione e combattimento ha un proprio peso importante e vitale.
\end{narratore}\end{changemargin}

\subsection{Incontri}\index{Incontri}

\label{incontri}

\begin{changemargin}{0.3cm}{0.3cm}\begin{enfasi}{Che è la vita senza speranza? Una gittata di dadi fra le tenebre, fra i deliri. (Ambrogio Bazzero)}
\end{enfasi}\end{changemargin}

\medskip

Un incontro è un momento di tensione e speranza, paura e sfida. E' l'occasione di mostrare e manifestare le proprie capacità e di lavorare come gruppo.

Un incontro non è l'occasione per fare sfoggio del proprio potere assoluto, sia come Narratore, che come Giocatore. Il Narratore saprà \st{punire} educare il giocatore che vuole essere oltre il gruppo e non parte di esso.

Troverete nelle pagine seguenti le istruzioni per creare delle sfide facili, medie, alte, straordinarie, mortali ed epiche.

Attraverso gli strumenti forniti dal manuale e dalla vostra esperienza con il gruppo saprete quale livello la sfida propone e ne valuterete sia l'impatto come punti esperienza che come ricompense.

Un incontro è un evento che mette i personaggi di fronte ad un problema specifico che devono risolvere. Molti sono combattimenti con i mostri o i PNG ostili, ma ce ne sono altri tipi: un corridoio irto di trappole, un'interazione politica con un re sospettoso, un passaggio pericoloso sopra un ponticello di corda traballante, un argomento scomodo con un PNG amichevole che ritiene che un personaggio lo abbia tradito, o qualsiasi cosa che aggiunga un pò di drammaticità al gioco.

Rompicapi, sfide interpretative e prove di competenza sono i metodi classici per la risoluzione degli incontri. Gli incontri più complessi da costruire e bilanciare saranno gli incontri di combattimento. Fidatevi del vostro istinto e dei suggerimenti forniti in OBSS.

Uno scontro può anche nascere palesemente sbilanciato, sarà l'accortezza dei giocatori a capire quando scappare!

Nel progettare un incontro di combattimento in primo luogo decidete che livello di sfida volete far fronteggiare ai PG, quindi seguite i punti descritti qui di seguito.

\textbf{Determinare APL}: \index{APL}Determinate il livello medio dei personaggi: questo è il Livello Medio del Gruppo (APL in breve, Average Party Level). Dovreste arrotondate questo valore al numero intero più vicino (questa è una delle poche eccezioni alla regola dell'arrotondamento per difetto).

Si noti che questa guida di riferimento alla creazione di un incontro presuppone un gruppo di quattro o cinque personaggi. Se il vostro gruppo ha sei o più giocatori, aggiungete uno al loro livello medio. Se il vostro gruppo contiene tre o meno giocatori, sottraete uno dal loro livello medio. Per esempio, se il vostro gruppo consiste di sei giocatori, due di 5° livello e quattro di 7° livello, il APL è il 7° (38 livelli totali, diviso per sei giocatori, arrotondando all'intero più vicino, ed aggiungendo uno al risultato finale).

\medskip

\textbf{Tabella: Determinare gli Incontri}\index[Tabelle]{Tabella Determinare gli Incontri}

\medskip

\begin{tabular}{@{}ll@{}} % @{} removes extra padding
\textbf{difficoltà} & \textbf{Grado di Sfida}\\
\toprule
Facile& APL\\
Media& APL +1\\
Alta& APL +2\\
Straordinaria& APL +3\\
Mortale& APL +4\\
Epica& APL +6
\end{tabular}


\medskip


%\begin{center}
%\includegraphics[width=0.7\linewidth]{immagini/impegnativa.png}
%
%\emph{Henry Justice Ford}
%\end{center}

\textbf{Determinare il grado di Sfida}: Il Grado di Sfida (GS) è un numero di convenienza usato per indicare i rischi relativi presentati da un mostro, una trappola, un pericolo o un altro incontro: più il grado di Sfida è alto, più pericoloso è l'incontro. Riferitevi alla Tabella: Determinare gli Incontri per determinare il Grado di Sfida che il vostro gruppo dovrebbe affrontare, in base alla difficoltà della sfida che volete e al APL.

\subsubsection{Quanti scontri affrontare}\index{Quanti scontri affrontare}\label{quantiincontri}

Non c'è una risposta unica. E' a vostra scelta, il sistema trova un suo equilibrio tra i 3 ed i 5 scontri al giorno. Ovvio che non devono essere tutti a difficoltà Alta!.

Gli scontri sono alla fine una gestione di risorse da usare contro un nemico. Queste risorse sono i Punti Ferita, gli incantesimi, le pozioni, pergamene ed oggetti consumabili posseduti. 

Se piazzate una sfida Straordinaria come primo incontro è probabile che i giocatori poi decidano di riposarsi per recuperare le energie, diversamente potreste optare per stancarli pian piano con incontri medi e poi provarli con una difficoltà maggiore. Ricorda infine che uno \emph{scontro} non deve essere per forza fisico, ma anche trappole, puzzle/indovinelli, sfide alternative.. qualsiasi cosa che faccia consumare risorse e ragionare.

Valutate sempre dove si muovono e cosa c'è intorno, verrà naturale trovare il giusto numero e tipi di scontri e nemici.

\subsubsection{Costruire l'Incontro}\index{Costruire l'Incontro}\label{costruireincontro}

Per costruire un incontro come prima cosa calcolate il valore dell' APL (il livello medio del vostro gruppo).

Per sviluppare il vostro incontro, aggiungete le creature, le trappole ed i pericoli finché non arrivate a vostro APL programmato.

Parti calcolando le sfide con grado di Sfida più alto dell'incontro, completando il resto con sfide minori.

Per esempio, volete che il vostro gruppo di sei personaggi di 7° livello abbia una sfida Media ed affronti alcuni Gargoyle (grado di Sfida 2 ciascuno), degli Xorn (grado di Sfida 5) e il loro capo, un Gigante delle Pietre (grado di Sfida 7). I personaggi hanno APL 8 e la Tabella: Determinare gli Incontri stabilisce che una sfida Media per un APL 8 è un incontro di grado di Sfida 9 (Difficoltà Media = APL+1).

Partendo da un grado di Sfida stabilito (9) seguite questa tabella per stabilire quanti mostri inserire nello scontro.

\medskip

\textbf{Tabella: Peso grado di Sfida per calcolo incontro}\index[Tabelle]{Tabella Peso grado di Sfida per calcolo incontro}

\medskip

\noindent\begin{tabular}{l|l||l|l|}
\textbf{Fattore} & \textbf{\% Peso} &\textbf{Fattore} & \textbf{\% Peso}\\
\toprule
-7/8& 5 & -2 & 55\\
-6& 10 & -1 & 60\\
-5& 15 &0 & 75\\
-4& 25 &+1& 85\\
-3& 47 &+2& 100\\
\end{tabular}

\medskip

Per \textbf{Fattore} si intende la differenza tra il GS del mostro rispetto al Grado di Sfida scelto. Il Peso è la \% relativa che il mostro apporta per raggiungere l'obiettivo del 100\%.

\textbf{Per raggiungere l'obiettivo dobbiamo sommare \emph{le percentuali} di ogni singolo avversario fino a raggiungere 100, ovvero il 100\% della sfida.}

Nel nostro esempio un Gigante delle Pietre ha grado di Sfida 7, ovvero un grado di Sfida -2 rispetto al nostro obiettivo di difficoltà grado di Sfida 9, lo Xorn ha grado di Sfida 5 ovvero -4 rispetto al grado di Sfida 9, i Gargoyle hanno grado di Sfida 2 ovvero -7 rispetto al grado di Sfida 9.

Un nemico con grado di Sfida -2 ha peso 45, un grado di Sfida -4 ha peso 20, un grado di Sfida -7 ha un peso di 5.

Per raggiungere l'obiettivo di un grado di Sfida 9 metterò 1 grado di Sfida -2 ( ovvero un gigante delle pietre), 2 grado Sfida -4 (ovvero 2 Xorn) e 3 grado di Sfida -7 (ovvero 2 gargoyle). Il Totale sarà 45 (un Gigante di Pietra) + 2*20 (due Xorn) + 3*5 (tre gargoyle) = 45+40+15 = 100. Obiettivo raggiunto!

Il totale dei Punti Esperienza sarà : 2900+2*1800+3*450 = 7850 Punti Esperienza / 6 Personaggi = 1235 Punti Esperienza a personaggio!

Avversari con grado di Sfida inferiore a 8 rispetto al APL si contano, pesano, solo se sono superiori a 20 come unità.

Avversari con GS pari a 1/2 considerateli come di GS 1, con GS inferiore ad 1/2 considerateli di GS 0.


\begin{changemargin}{0.3cm}{0.3cm}\begin{narratore}
Ricordatevi al termine di ogni \emph{incontro} o \emph{sfida} di segnarvi i Tratti che hanno caratterizzato le azioni dei personaggi. Questi punti parziali li potete concedere alla prima occasione di riposo dei personaggi.
\end{narratore}\end{changemargin}


\subsubsection{Scontri troppo veloci}

Un problema a cui potreste andare incontro è che lo scontro si risolve troppo velocemente. Possono esserci diversi motivi ed altrettante soluzioni.

Se i giocatori si aspettano pochi incontri è probabile che useranno le loro migliori risorse ed opzioni subito ad inizio combattimento andando così a sconfiggere velocemente i nemici, in questo caso prendeteli di sorpresa con ondate successivi di nemici.

E' possibile che ci siano troppi pochi nemici e quindi anche se questi sono \emph{forti} canalizzando su di loro tutti gli attacchi risultano facile preda dei personaggi, in questo caso dei gregari o l'impedire di riposare e quindi recuperare Punti Magie ed Punti Ferita sarà di aiuto.

E' ovviamente possibile che lo scontro non sia tarato bene ed effettivamente abbiate bilanciato l'incontro perché sia troppo facile, questo è il caso più facile da risolvere, l'esperienza vi insegnerà a meglio costruire gli incontri vuoi aggiungendo o sostituendo gli avversari.

Ricordate che i \emph{mostri} possono anche loro eseguire Azioni come \hyperlink{spingereavversario}{Spingere}, \hyperlink{afferrareunavversario}{Afferrare}, \hyperlink{farecadereavversario}{Buttare a terra}, \hyperlink{fiancheggiare}{Fiancheggiare}, non limitatevi nelle scelte.

\begin{changemargin}{0.3cm}{0.3cm}\begin{enfasi}{
L'essenza del mondo è il gioco ... noi giochiamo il serio, giochiamo l'autentico, giochiamo la realtà, il lavoro e la lotta, giochiamo l'amore e la morte e giochiamo perfino il gioco. (Eugen Fink)
}\end{enfasi}\end{changemargin}


\subsubsection{Lo scontro con il Boss}\index{Lo scontro con il Boss}\index{BBEG}

Quando preparate uno scontro il boss, ovvero con quello che potete definire un nemico significativo che ha un certo peso nello svolgimento della campagna dovete preoccuparvi di rendere interessante la sfida!

Se lo scontro deve essere memorabile non basta piazzare il cattivo, organizzate il tutto perché tutti gli avvenimenti risultino coinvolgenti ed emozionanti.

Organizzate i nemici affinché:

\begin{itemize}[leftmargin=*] \setlength{\itemsep}{-1pt}

\item arrivino in più ondate così che ci sia un senso di falso successo
\item che i nemici arrivino da più parti per non fare concentrare le forze solo da un lato
\item che siano intervallati nemici più o meno ostici così che ci sia un senso di falsa sicurezza
\item che l'ambiente sia significativo e giochi un ruolo importante nel combattimento
\item dividi i personaggi su più fronti
\item fai che l'attacco non sembri un attacco
\item gioca con arguzia e non farti demoralizzare.
\end{itemize}

Ed in ogni caso ricorda sempre: non è uno scontro tra Narratore e Giocatori! L'obiettivo è creare momenti memorabili!!!

\subsubsection{Aggiungere i PNG}\index{Aggiungere i PNG}

Una creatura che possiede livelli, Abilità, competenze, che potrebbe essere un personaggio ma viene gestito dal Narratore si considera un PNG. Queste creature possono svolgere un ruolo molto importante e non vanno usate come semplici mostri. Dategli uno spessore e creerete delle figure indimenticabili.

\subsubsection{Modifiche ad Hoc del grado di Sfida}\index{Modifiche ad Hoc del grado di Sfida}

Mentre potete modificare il grado di Sfida specifico del mostro avanzandolo, applicando modifiche o livelli, potete anche aggiustare la difficoltà dell'incontro applicando modifiche ad hoc all'incontro o alla creatura in sé.

Qui descritti ci sono tre modi aggiuntivi con cui potete alterare la difficoltà dell'incontro.

\medskip

\textbf{Terreno Favorevole ai PG}\index{Terreno Favorevole ai PG}

Un incontro contro un mostro che non è nel suo elemento preferito (come uno Yeti incontrato in una caverna piena di lava, o un Drago enorme incontrato in una stanza molto piccola) da ai personaggi un vantaggio. Considerate l'incontro come se avesse un grado di Sfida più basso del suo grado di Sfida reale.

\medskip

\textbf{Terreno Sfavorevole ai PG}\index{Terreno Sfavorevole ai PG}

I mostri sono progettati con il presupposto che siano incontrati nel loro terreno preferito: incontrare un Aboleth sott'acqua non aumenta il grado di Sfida dell'incontro, anche se nessun personaggio è in grado di respirare sott'acqua.

Se, d'altra parte, il terreno ha un impatto più significativo sull'incontro (come un incontro contro una creatura con Vista Cieca in una zona che sopprime ogni fonte di luce), si possono, aumentare il grado di Sfida dell'incontro fosse di un grado più alto.

\medskip

\textbf{Modifiche all'Equipaggiamento dei PNG}\index{Modifiche all'Equipaggiamento dei PNG}

Potete aumentare o diminuire la difficoltà data dai PNG modificandone l'Equipaggiamento. Un PNG incontrato senza equipaggiamento dovrebbe avere un grado di Sfida ridotto di 1 (a condizione che la perdita di equipaggiamento sia realmente controproducente per il PNG), mentre un PNG che ha un equipaggiamento equivalente a quello di un personaggio (come indicato sulla Tabella: Ricchezza dei Personaggi per Livello) ha un grado di Sfida superiore di 1 al suo grado di Sfida reale.

Occorre prestare attenzione ad assegnare ai PNG questo equipaggiamento supplementare, specie ai livelli più alti, in cui potete consumare l'intero tesoro della vostra avventura in un colpo solo!

\subsubsection{Assegnare i PX}\index{Assegnare i PX}\label{assegnarepuntiesperienza}

I personaggi avanzano di livello sconfiggendo mostri, superando sfide, divertendosi,completando l'avventura ed arraffando tesori: nel farlo guadagnano i Punti Esperienza (PX in breve). Potete assegnare Punti Esperienza non appena una sfida viene superata, ma ciò potrebbero interrompere il flusso del gioco. E' più facile assegnare i punti esperienza alla fine di una sessione di gioco (o più sessioni) che permetta ai personaggi di riflettere su quanto accaduto. Il giocatore può usare il tempo a disposizione fra le sessioni di gioco per aggiornare la scheda.

%\begin{center}
%\includegraphics[width=0.7\linewidth]{immagini/tesoro2.png}
%\end{center}


\subsection{Ricchezza dei Personaggi per Livello}\index{Richezza per Livello}

\textbf{Tabella: Ricchezza dei Personaggi per Livello}\index[Tabelle]{Tabella Ricchezza dei Personaggi per Livello}

\medskip

\noindent\begin{tabularx}{0.50\textwidth}{XX|XX}
\textbf{Livello} & \textbf{Ricchezza (mo)} & \textbf{Livello} & \textbf{Ricchezza (mo)}\\
\toprule
1 & 100 & 11 & 13900\\
2 & 160 & 12 & 19900\\
3 & 220 & 13 & 25900\\
4 & 340 & 14 & 37900\\
5 & 530 & 15 & 49800\\
6 & 2030 & 16 & 67700\\
7 & 3660 & 17 & 85700\\
8 & 5780 & 18 & 142000\\
9 & 8100 & 19 & 253000\\
10 & 11000 & 20 & 365000
\end{tabularx}

\medskip


La \textbf{Tabella: Ricchezza dei Personaggi per Livello} per Livello indica la quantità di monete d'oro equivalenti in tesori ed oggetti che ogni personaggio dovrebbe avere ad un livello specifico. Si noti che questa tabella si basa su un modello standard di gioco.

Le avventure con magia rara potrebbero assegnare soltanto la metà di questo valore, mentre avventure più epiche potrebbero raddoppiarlo. Si presume che parte del tesoro sia consumato nel corso di un'avventura (come pozioni e pergamene) e che alcuni degli oggetti meno utilizzati siano venduti per metà del loro valore per acquistare un equipaggiamento più utile.

La Tabella: Ricchezza dei Personaggi per Livello può anche essere usata per stabilire l'equipaggiamento per i personaggi che cominciano dopo il 1° livello, come un nuovo personaggio creato per sostituirne uno morto. I personaggi non dovrebbero spendere più di un terzo della loro ricchezza totale su un singolo oggetto.

Per un metodo equilibrato, i personaggi che vengono creati dopo il 1° livello dovrebbero spendere il 25\% della loro ricchezza per le armi, il 25\% per armatura e oggetti di protezione, il 25\% per altri oggetti magici, il 15\% per oggetti che si consumano come bacchette, pergamene e pozioni e il 10\% per un equipaggiamento normale e monete. Tipi di personaggio differenti potrebbero spendere diversamente la loro ricchezza rispetto a come suggerito; ad esempio, gli incantatori arcani potrebbero spendere di più per oggetti magici e a consumo che per le armi.

\subsection{Io conosco un tizio...}\index{Io conosco un tizio}\label{ioconoscountizio}\hypertarget{ioconoscountizio}{}

Per agevolare lo spirito di avventura e non lasciare i personaggi incapaci o indecisi nell'agire, permettetegli di conoscere un certo numero di PNG pari al loro punteggio di Carisma +1. Il giocatore in qualsiasi momento potrà dichiarare di conoscere questo PNG e dovrà tenerne traccia. Questi PNG potranno essere \emph{sfruttati} quando i personaggi si trovano in situazioni difficili, di pericolo o semplicemente bisognosi si supporto. Il personaggio che si appella al \emph{io conosco un tizio...} deve descrivere adeguatamente il soggetto ed il rapporto che c'è tra loro. Il Narratore adatterà la situazione per includere questo personaggio al meglio delle possibilità.

Il tizio potrebbe essere un commerciante che gli deve un favore, se non un ladro od un burocrate. I personaggi sono invitati a non inventarsi amicizie o favori da personaggi troppo importanti.


\subsection{Recitare}\index{Recitare}\label{ruolare}

\label{recitare}

Un gioco di ruolo non è un semplice tirare dadi, è un incontro di pensieri, opinioni, sfide, lotte. E' un gioco catartico, liberatorio, evolutivo ed istruttivo.

E' giusto che ci sia combattimento, lotta, sangue paura ed azione, allo stesso modo deve esserci la possibilità di giocare i propri personaggi con i loro svantaggi, vantaggi, poteri e storie e anche drammi personali.

Il giocatore deve sempre impersonare il personaggio, immedesimarsi e partecipare attivamente.

Ci possono essere situazioni di contorno, gestite velocemente, che vengono fatte in terza persona, eppure ogni volta che si renda necessario giocare allora deve essere vero, fatto dal giocatore calandosi appieno nel personaggio.

\medskip

\textbf{Quando un giocatore interpreta bene e descrive l'azione che va a svolgere in maniera \textbf{partecipativa}, \textbf{coinvolgente}, \textbf{ispirata}, dategli un premio, concedete un bonus di +1 all'azione che sta svolgendo}

\medskip

Fatelo presente al giocatore che grazie alla sua interpretazione ha quel bonus.

Allo stesso tempo potrebbero esserci situazioni che si rivelano sgradevoli da gestire e giocare per qualche giocatore. Fate molta attenzione in questo caso, andare contro la sensibilità di un giocatore, di un amico, non è come andare contro l'etica o morale di un personaggio. Se percepite un senso di disagio ed imbarazzo fermate subito il gioco e chiarite la situazione con i giocatori e riprendete solo quando vi sarete accordati su come modificare la situazione per evitare che accada di nuovo.\index{Veli e Divieti}\index{Giocare e non spaventare}


\begin{changemargin}{0.3cm}{0.3cm}\begin{enfasi}
{Concentratevi sulle persone, non sulle regole. Spingete per uno stile di gioco di gruppo; l'interpretazione è divertente ma non deve ostacolare il piano; sostenete i vostri compagni. (Frank Mentzer)}\end{enfasi}\end{changemargin}

\subsection{Cambiare Personaggio}\index{Cambiare personaggio}

Per quanto il sistema favorisca la libertà di costruzione e sviluppo del personaggio se un giocatore è in difficoltà con il personaggio creato permettetegli, entro il 4 livello di cambiare personaggio e crearne uno nuovo. Ricordate che l'obiettivo è divertirsi tutti.


\subsection{Circa OBSS ed i tiro di dadi}\index{Circa OBSS ed i tiro di dadi}\label{obssedadi}

OBSS usa un sistema di tiro di dadi peculiare andando a mescolare una distribuzione 3d6 ad il potenziale dei 6 che esplodono. Questo sistema riesce a garantire una buona varianza e pur se concentrando i risultati intorno ai valori centrali della distribuzione lascia aperto il limite superiore a tiri particolarmente fortunati.

Se volete divertirvi a studiare la curva corrispondente vi consiglio il sito\href{https://anydice.com/}{Anydice}. Questo lo pseudo codice da inserire (o cliccate \href{https://anydice.com/program/2610e}{qui} per il codice già inserito):

\medskip

\noindent{

function: explode ROLLED:n \{

if ROLLED = 6 \{ result: 6 + [explode d6] \}

if ROLLED = 1 \{ result: 0 \}

result: ROLLED\}

output 3d[explode d6]}

\medskip

oppure cliccate \href{https://anydice.com/program/2610e}{qui} per il codice già inserito.

\subsubsection{Opzionale - Variante Tiri Critici}\index{Opzionale - Variante Tiri Critici}\label{variantetiricritici}\hypertarget{variantetiricritici}{}

In OBSS vale la regola che se un successo viene fatto ottenendo almeno due 6 con i dadi si definisce successo critico, allo stesso modo se fallisco la prova ottenendo nel tiro di dadi due 1, oppure un 1 e due 2, allora si definisce un fallimento critico.

Questi tiri critici molto legati all'aleatorietà dei dadi potrebbero non piacere a tutti i giocatori e come per la regola \hyperlink{tirocriticovariante}{Opzionale Tiro Critico Variante} (pag. \pageref{tirocriticovariante}) questa regola opzionale stabilisce che in caso di Prova di Competenza, Tiro per Colpire e Tiro Salvezza:

\begin{itemize}[leftmargin=*] \setlength{\itemsep}{0pt}
\item considerare come un \textbf{fallimento critico} una prova \textbf{fallita di 8} o più.
\item considerare come un \textbf{successo critico} una prova \textbf{riuscita di 8} o più.
\item per \textbf{ogni multiplo di 8} per il quale la prova è riuscita o fallita si \textbf{cumula} rispettivamente un \textbf{successo o fallimento critico addizionale}. 
\end{itemize}

Lo stesso principio vale per la Prova di Magia che si esegue tirando 3d6 + Competenza Magica + modificatore di caratteristica per incantesimi + 1*Adepto della Magia ed il risultato deve superare una difficoltà pari 10 + Livello di Magia manifestata*3. \index{Prova di magia alternativa} 

Questa Opzione rende molto più efficaci i personaggi nelle sfide contro avversari di basso rango.\index{Valorizzare i livelli}

\subsubsection{Opzionale - Variante Consumo Risorse}\index{Opzionale - Variante Consumo Risorse}\label{varianteconsumorisorse}\hypertarget{varianteconsumorisorse}{}

Ogni qual volta il personaggio usi delle risorse \emph{contate}, quali Frecce, Razioni di cibo, Torce, se non si ha pressione di fare consumare gli oggetti si può optare per questa regola opzionale.

Al termine di un combattimento, dopo una giornata di avventura, il giocatore tira 1d12 per ogni tipo di risorsa che ha consumato. Se fa 1 o 2 con il dado ha diminuito la sua scorta.
La volta dopo tirerà invece che 1d12 un 1d8 e poi 1d6 e poi 1d4. Quando arriva a tirare il d4 e fa 1 o 2 ha finito completamente la risorsa e deve ricomprare 20 frecce, 7 giorni di cibo, 6 torce... 

Nella scheda a fianco a quelle risorse segna il dado da usare per il successivo tiro.

Il Narratore potrebbe decidere di tracciare in questa maniera i soli consumabili comuni e non di quelli di pregio.


\subsection{Le avventure in OBSS} \hypertarget{OSR}{} \index{OSR}\index{Avventura in OBSS}\label{avventureinobss}

Suggerisco la lettura integrale dell'articolo: \href{https://lithyscaphe.blogspot.com/p/principia-apocrypha.html} {Principia Apocrypha}

https://lithyscaphe.blogspot.com/p/principia-apocrypha.html quello che segue è un sunto da me adattato e modificato delle linee guida che seguo quando masterizzo OBSS.

OBSS segue i principi dell' \href{https://it.wikipedia.org/wiki/Old_School_Renaissance}{OSR} (wikipedia). Le avventure in OBSS mirano ad essere letali, avere un mondo liberamente esplorabile, una trama abbozzata, spingere sul problem-solving ed avere un sistema di ricompense incentrato sull'esplorazione, sui tesori e sulla partecipazione al gruppo. OBSS non si cura troppo del bilanciamento degli incontri e apprezza l'intraprendenza dei giocatori e cattura le loro idee mettendole nell'avventura.

Per me l'OSR non sono tabelle di incontri casuali e randomizzazione caotica ne un regolamento specifico, è piuttosto lo spirito di avventura, meraviglia, paura, gloria, stupore e sfida che si sviluppa nelle avventure. Non siate troppo lineari, troppo prevedibili, aggiungete nelle avventure quel giusto mix che le rendono sempre uniche.

Se il metodo può non piacere usate quello che più vi aggrada, personalmente nel corso dei decenni ho imparato ad apprezzare e vedere apprezzato la spontaneità e naturalezza che i cardini dell'OSR portano nel gioco.

\bigskip

\textbf{Queste sono regole di base per il Narratore che suggerisco per condurre le avventure.}\index{Linee guida per i Narratori}\index{Principi OSR}

\medskip

\begin{itemize}[leftmargin=*] \setlength{\itemsep}{0pt}

\item
Tu sei il Narratore, tue le Regole, tuo il Mondo.

Non farti limitare dall'avventura, dal sistema, dall'elenco dei mostri, sentiti sempre libero di modificare e adattare in base alle necessità dell'avventura e del gruppo

\item
Ricordati di esser giusto e corretto. Improvvisa, adatta quanto vuoi ma sii coerente. Se stabilisci una regola (od una modifica ad una regola) seguila fino in fondo.

Allo stesso tempo se ti serve una regola e non la trovi usa il buon senso, è sicuramente la scelta giusta in quel momento.

Rispetta i dadi ed i risultati ottenuti, come capiteranno ai giocatori capiteranno risultati particolari anche a te. E' giusto così.

\item
Non devi salvare il \emph{culo} ai personaggi. Non sei il loro amico ne il loro nemico. Il tuo ruolo è di raccontare storie che nascono dalle storie dei personaggi, dalle loro azioni ed inazioni.

\item
Abbozza la storia, scrivi le parti centrali o da leggere ai giocatori ma non farti dominare o vincolare da quello che ti aspetti. Spesso e volentieri i giocatori ti stupiranno, meglio sapere dove si muovono e cosa hanno intorno per poter reagire sempre puntualmente.

Sono i giocatori a dare la direzione all'avventura e tu a dipanarla.

\item
Apprezza il caso e crea situazioni diverse dove i giocatori possono scegliere strade diverse o intrecciarne di nuove. E' la tua fortuna avere dei giocatori creativi che sanno sorprenderti.

\item
Non costringere nessuno a fare qualcosa, lascia sbagliare i giocatori, lascia che paghino le loro scelte. Non devi ostacolarli ne devi imbeccarli per una direzioni. Richiede da parte tua una immaginazione e capacità di adattamento non indifferente, ma sicuramente l'avventura ed il divertimento ne gioverà.

\item
I personaggi sono esploratori, per definizione. Focalizza sull'esplorazione, più si esplora più si creano situazioni, più si creano agganci nell'avventura, più si conoscono altri png più ci sono zone da esplorare.

Fa capire che i tesori sono esperienza, in senso letterale e pratico. Non dovrai mai spingerli tu in un dungeon ma la loro brama di esperienza e tesoro.

\item
Fai risolvere i problemi ai giocatori e non ai personaggi. Lascia ruolare le scene, sono sempre meglio di un tiro di dado. Incoraggia il giocatori ad interagire e chiedere una prova solo come ultima chance. Proponi problemi che non debbano essere risolti per forza con un tiro di dado bensì piuttosto tramite più azioni, anche complesse.

Premia le azioni creative e le scelte coraggiose prima più di tutto l'arguzia e il volere trovare situazioni alternative e creative.

\item
Fa che i giocatori ti chiedano informazioni, si confrontino con l'ambiente e tra di loro. Incoraggia l'interazione con il mondo esterno e solo come ultima possibilità concedi un tiro di dado.

\item
Grandi sfide e rischi danno sempre grandi ricompense. Non deludere i giocatori (se non per scopo di avventura) negandogli il giusto tesoro o esperienza, più si addentreranno in profondità più i pericoli saranno letali maggiore sarà la ricompensa (Legge del Premio).

\item
Non devono esistere abitudini o consuetudini. Non creare uno standard.
Cerca sempre di sorprendere i giocatori con mostri fuori luogo (ma che abbiano un senso), trappole anomale, ambienti alternativi. Situazioni diverse stimoleranno i giocatori a risolvere in maniera diversa ogni problema.

Prepara soluzioni diverse e accetta soluzioni diverse. Metti nell'avventura problemi e situazioni che nell'insieme permettano la soluzione, ogni stanza non dovrà essere un asettico ambiente ma contenere indizi e soluzioni per altri problemi anche senza una diretta soluzione.

\item
Accetta la morte. Un combattimento se tale è sempre letale, non avere paura di ferire o uccidere i personaggi. Falli ragionare, studiare il nemico, capire quale è il migliore approccio; ed infine stupiscili. I personaggi devono prima battere i nemici in astuzia e pianificazione, se vogliono sopravvivere.
Se proteggi i personaggi la partita mancherà di tensione e i giocatori risolveranno tutti i problemi con la forza bruta.
%I dungeon non devono essere ambienti per forza da svuotare dai mostri. Lo scopo dei mostri e limitare e orientare le azioni, consumare le opzioni.

Se i giocatori cercano sempre e comunque lo scontro frontale allora daglielo, come richiedono.

\item
Mantieni l'attenzione alta. Fa in modo che il passare del tempo abbia conseguenze, se i giocatori temono lo scorrere del tempo faranno scelte più ardite o forse sbagliate. Mantieni la tensione fra il desiderio di esplorare e fare bottino e il terrore di restare fermi troppo a lungo.

\item
Tu sei la sorgente delle informazioni, i giocatori le elaborano, i personaggi le usano.

Non nascondere informazioni che i personaggi devono sapere o sanno già, non dovrai fare il professore ma allo stesso modo fa in modo che siano consapevoli di ciò che hanno intorno.
Allo stesso tempo non devi rivelare tutto subito, falli indagare, curiosare. Come una cipolla le informazioni che otterranno saranno nascoste sotto strati di altre informazioni magari di minore importanza.

\item
Gli indizi creano situazioni. Lascia che i tuoi indizi, specifici e curiosi, attirino l'attenzione dei giocatori.% Come un esca su un amo attira i giocatori in situazioni di dubbio, dove indagare e capire cosa succede.

Non infarcire l'avventura di dettagli inutili, lascia spazio alla creatività e immaginazione dei giocatori. %, i dettagli che però fornirai dovranno non solo avere un senso ma essere necessari all'avventura.

\item
Se i giocatori tendono a dimenticare le informazioni utili date cerca di sfruttare un PNG che abbia memoria o invitali a prendere appunti, non c'è nulla di male nell'essere preparati.

\item
L'avventura non è mai statica ne tanto meno il mondo dove si muovono i personaggi.
Il mondo ha la stessa importanza se non di più dell'avventura stessa. Azioni dei giocatori possono scatenare accadimenti a livello globale. Pensate sempre alle conseguenze dei gesti.

\item
Se usi i PNG non farli essere delle semplici macchiette, fa in modo che i personaggi si possano affezionare e considerare il PNG uno del gruppo alla pari di tutti gli altri.

\item
I mostri non devono essere stupidi per forza. Falli parlare, ragionare, scappare.. anche loro vogliono vivere!

\item
Ricordati la Legge del Premio. Premia gli audaci, premia che si spinge più in profondità nelle caverne. Premia chi sopravvive.

\end{itemize}

\subsection{Sessione Zero}\index{Sessione Zero}\label{sessionezero}\hypertarget{sessionezero}{}

La Sessione Zero, la prima sessione di gioco, ha una valenza ed importanza particolare. Può essere la sessione in cui ci si conosce per la prima volta, spesso è la sessione in cui si creano i personaggi che si andranno a giocare, sempre è la sessione in cui si vanno a stabilire le regole ed aspettative comuni.

La Sessione Zero serve a stabilire cosa e come si andrà a giocare, quali saranno le principali caratteristiche della campagna e del gruppo che si va a creare.

Per partire bene come gruppo di giocatori è importante conoscersi personalmente e avere fiducia e rispetto negli altri. Non devi dire tutto di te ma almeno le passioni, interessi, curiosità, ciò che almeno all'inizio serve a creare fiducia.

Suggerisco ai Narratori di stabilire delle regole chiare per il buon gioco. Purtroppo l'esperienza insegna che siamo tutte persone diverse con stili, prospettive ed aspettative diverse. Conoscersi serve anche a questo, a capire se il proprio personaggio può stare bene insieme agli altri e capire se la propria persona e personalità è in qualche maniera affine o meno alle altre persone.

\textbf{Il Narratore prima di incominciare è opportuno che chiarisca quali sono le regole essenziali al suo tavolo}. Un esempio di regole possono essere:

\begin{itemize}[leftmargin=*] \setlength{\itemsep}{0pt}

\item Che ogni giocatore \textbf{conosca} la parte del regolamento del manuale che maggiormente andrà ad usare (combattimento, magia, patroni...).
\item Rispetti i \textbf{limiti} degli altri. Ogni persona ha una diversa sensibilità a certi argomenti (stupri, schiavitù, razzismo, violenza...) è fondamentale che si chiarisca insieme quali sono i limiti da non superare mai.
\item \textbf{Rispetta} ogni persona con cui giochi. Ciò include essere puntuali e non annullare senza un motivo importante. 
\item I giocatori devono creare un gruppo \textbf{coeso} fatto da individualità che collaborano.
\end{itemize}

\textbf{Vanno condivise e stabilite le informazioni base dell'avventura.}

\medskip

\begin{itemize}[leftmargin=*] \setlength{\itemsep}{0pt}
\item Introduci in linea di massima la campagna o avventure che si andranno a svolgere. Indica la tipologia (eroica, dark, gothic, horror, politica, caverne infinite, esplorazione, sopravvivenza..) e grado di difficoltà.
\item Introduci le informazioni necessarie relative all'ambientazione o fornisci dispense e manuali sull'argomento. Indica se ci sono delle Abilità suggerite.
\item Stabilite le regole opzionali e che siano chiare a tutti.
\item Indica la lista o la tipologia di Tratti accettati e se ci sono dei limiti nella scelta dei Patroni.
\item Incentiva la creazione di personaggi con background condivisi così che il gruppo sia già coeso alla formazione.
\item Come Narratore devi comprendere il grado di conoscenza del sistema da parte dei giocatori e se necessario quando serve soffermarti a spiegare le regole.
\end{itemize}

\textbf{Altre indicazioni utili riguardano}:

\medskip

\begin{itemize}[leftmargin=*] \setlength{\itemsep}{0pt}
\item Cosa è permesso portare ed usare al tavolo e cosa no (bibite, mangiare, cellulari, alcolici, fumare..). Sapere se ci sono animali in casa.
\item Stabilite il numero minimo di giocatori per fare la sessione, giorno di gioco ed orari.
\end{itemize}

In definitiva, la Sessione Zero è fondamentale per stabilire una solida base per il buon gioco di ruolo. Aiuta a creare un ambiente collaborativo in cui tutti si sentono partecipi e contribuisce a evitare problemi e disaccordi durante il corso della campagna.

Anche nel miglior gruppo già affiatato è sempre bene ricordare e condividere questi suggerimenti ad ogni inizio di campagna.


\end{multicols}

\vfill

\begin{center}
\includegraphics[width=0.95\linewidth]{immagini/fognelondra.png}

\emph{Mappa delle fogne di Londra, 1880}

\emph{fondamentale per tutti i cacciatori di ratti...}
\end{center}

\bigskip


\begin{changemargin}{0.3cm}{0.3cm}\begin{enfasi}
{
I problemi più complessi hanno soluzioni semplici e facili da comprendere ma sbagliate (Arthur Bloch).... ma se sono divertenti e piacciono a tutti allora usale! (NdA)
}
\end{enfasi}\end{changemargin}\medskip




%\vfill

%\begin{center}
%\includegraphics[keepaspectratio,width=0.55\textwidth]{immagini/dungeonsample.png}
%
%\emph{Dettaglio di un dungeon}
%\end{center}

\pagebreak

