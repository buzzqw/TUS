\section{Regole per le Competenze}\index{Regole per le Competenze}\index{Competenze}

\begin{enfasi}{
Occorre che la legge sia breve, perché più facilmente i mal pratici la ricordino. (Lucio Anneo Seneca)}\end{enfasi}

\begin{multicols}{2}

Le prove (i check), per le Competenze o Caratteristiche, si eseguono tirando 3d6, al risultato dei dadi si somma il punteggio della Competenza (di base o attiva) e della Caratteristica collegata ed eventuali bonus magici e di circostanza o Abilità, il risultato ottenuto deve essere comunicato al Narratore, il quale lo confronterà con la difficoltà (DC) della prova.

Quando dovete stabilire una difficoltà partite pensando che la prova deve essere rapportata da una persona \emph{normale}. Non pensate \emph{se la dovessi fare io allora la prova sarebbe impossibile}, \emph{se la prova la fa Arsenio Lupin la prova è facilissima}. Partite dal presupposto che la difficoltà deve racchiudere in se tutti gli elementi circostanziali.

Pensate se piove, c'è poca luce, il personaggio sta correndo, è ferito, fa le cose di fretta ed anche alla complessità della cosa che deve fare, saltare un fosso di 3 metri non è come uno di 3 metri al buio, senza scarpe, sotto la pioggia ed inseguiti e con le tasche strapiene di monete...

Decifrare uno scritto antico potrà essere una passeggiata per un linguista esperto, ma per una \emph{persona normale} che non ha idea di cosa può avere davanti la prova è semplicemente impossibile. Questo \emph{impossibile} è la vostra DC, la difficoltà della prova.

E non spaventatevi se i personaggi falliscono le prove, renderà l'avventura più interessante e permetterà al Narratore di introdurre fatti, indizi e nuove avventure.

\begin{narratore}[Non serve sempre una Prova]
Evita di chiedere una prova qualora i giocatori dichiarino \textbf{come} effettuano la prova, come e dove cercano, che dialogo imbastiscono per intimidire l'obiettivo.. Valutate con attenzione come il giocatore descrive ciò che fa perché questa è già la prova. Non è solo per velocizzare il gioco, serve a stimolare i giocatori a pensare in maniera completa ed a calarsi nel personaggio e nell'ambiente.

Renderà il gioco più dinamico e tutti i giocatori parteciperanno alla situazione e collaboreranno dichiarando cosa e come agiscono. Usate sempre il buon senso e risparmiate tiri di dadi! Tirare un dado significa creare la possibilità di fallire!
\end{narratore}

%\medskip
%\begin{center}
%\includegraphics[width=0.8\linewidth]{immagini/master2.png}
%
%\emph{The Master of the Gamblers}
%\end{center}

\medskip

\textbf{Quando devi fare una prova per una Competenza di Base in cui non sei preparato, ovvero non hai punti, devi tirare solo 2d6 + punteggio della Caratteristica collegata}.\index{Prova competenza senza competenze}

Quando si scrive -1d6 significa che si tira un dado in meno (o due se è -2d6), viceversa se c'è scritto +1d6 si tira un dado a 6 in più e si somma.

La tabella qui sotto serve a rapportare la difficoltà alla capacità minima necessaria per riuscire la prova con un tiro medio (un punteggio di 10 lanciando 3d6). Usate queste indicazione per avere una idea delle scale di difficoltà.

Il Narratore non ti dirà fammi una prova a difficoltà 10, ma dirà che la prova non presenta elementi di particolare difficoltà.

%\begin{center}
%\includegraphics[width=0.9\linewidth]{immagini/difficulty.png}
%
%\emph{A City on a Rock, long attributed to Goya, is now thought to have been painted by 19th-century artist Eugenio Lucas Velázquez. Ottima prova di falsificazione}
%\end{center}
%\medskip

\medskip

\textbf{Tabella: Classe di difficoltà}\index[Tabelle]{Tabella Classe di difficoltà}\label{basedifficolta}

\medskip

\noindent\begin{tabularx}{0.45\textwidth}{lll}
	\toprule
\textbf{Diff.} & \textbf{Descrizione} & \textbf{Livello}\\
\textbf{DC}&\textbf{difficoltà}& \textbf{Competenza}\\
\toprule
6 & Estremamente facile & Nulla\\
10 & Facile & Scarsa\\
15 & Normale & Normale\\
20 & Difficile & Buona\\
25 & Molto difficile & Ottima\\
30 & Eroica& Eccellente\\
35 & Quasi impossibile & Stupefacente\\
40 & Impossibile & Epica
\end{tabularx}

\medskip

Se devi fare una prova su una Caratteristica devi tirare 3d6 e sommare il punteggio della Caratteristica e altri modificatori. Comunica questo risultato al Narratore che la confronterà con la difficoltà (DC).

\subsection{Le Golden Rules}\index{Le Golden Rules}\label{goldenrules}

Se non specificato diversamente per tutte le prove di competenza (Base, Attive) valgono tre regole base \index{Regole Base} chiamate \textbf{Golden Rules}:\index{Golden Rules}

\begin{itemize}[leftmargin=*] \setlength{\itemsep}{0pt}
\item
I \textbf{6 esplodono}, ovvero se nella prova dei 3d6 un dado fa sei, somma il risultato e ritira, e se fa 6 nuovamente sommi il risultato e ritiri ancora e ancora..
\item
Gli \textbf{1 portano male}. Quando si tira 1 con un dado, quel dado non contribuisce al risultato. Il valore del dado che mostra 1 viene considerato zero.
\item
\textbf{Affidarsi alla sorte}. Ogni 4 punti tra Competenza (Base o Attiva) e Caratteristica che rinunci a sommare nella prova tiri un dado a 6 in più (Tiro per Colpire, Tiro Salvezza, prove Competenza). Questo valore non può essere tolto dal punteggio dato da Abilità o oggetti magici.

%\columnbreak

\begin{center}\textbf{\emph{Corollario}}\end{center}\index{Corollario Golden Rules}

\item \textbf{Tirare 3 volte 6 con i primi tre dadi è un successo}, sia nelle Prove di Competenza, Tiri Salvezza e Tiri per Colpire indipendentemente dal risultato finale.\index{Tirare tre volte 6}

\item \textbf{Tirare 3 volte 1 con i primi tre dadi è fallimento}, sia nelle Prove di Competenza, Tiri Salvezza e Tiri per Colpire indipendentemente dal risultato finale. \index{Tirare tre volte 1}

\item\textbf{Gli 1 tirati a seguito di un 6} valgono sempre zero

\end{itemize}

Sfruttate le \textbf{Golden Rules} a vostro vantaggio! Osate, tentate, rischiate quando la situazione non permette altre soluzioni!

\begin{giocatore}[Non c'è solo la scheda!]{
Non cercate per forza la soluzione nella scheda. Usate la vostra capacità di immaginare, di risolvere, di intuire per uscire e risolvere situazioni. La scheda rappresenta solo una piccola parte di ciò che il vostro personaggio può fare.
}\end{giocatore}

\subsection{Superare o Fallire la prova}\index{Superare o Fallire la prova di tanto}\label{superareofallirelaprova}\index{Successo critico nelle prove}

La prova è superata quando tirati i 3d6 e sommata la Competenza interessata e la Caratteristica nonché i vari modificatori il risultato è pari o superiore alla DC stabilita dal Narratore.

Se il risultato è inferiore alla difficoltà la prova è fallita.

Una prova può essere ripetuta\index{Ripetere una prova}\index{Rifare una prova} finché non mutano le condizioni che permettono alla prova di essere ripetuta.

\subsubsection{Successo Critico - Fallimento Critico}\index{Successo Critico - Fallimento Critico}

Se la prova viene \textbf{superata almeno di 8} rispetto alla difficoltà stabilita il Narratore allora si considerà come un Successo Critico.
Il Narratore può decidere di dare maggiori informazioni, concedere un bonus alle azioni successive (+1).. qualsiasi cosa possa valorizzare quanto agevolmente la prova è stata superata.\index{Superare la prova con un Critico}.

Viceversa se la prova fallisce \textbf{di almeno 8 punti} il Narratore potrebbe descrivere come miseramente la prova è fallita e come il risultato pessimo influenzi l'Azione e quelle successive.

Per ogni 8 punti superiori od inferiori alla difficoltà stabilita si conta un Successo Critico o un Fallimento Critico.  Quando nel manuale di parla di 2 Successi Critici significa superare la prova di almeno 16 punti.

A discrezione del Narratore una prova fallita criticamente non può essere ripetuta dallo stesso personaggio.

\begin{narratore}[Ripetere le Prove]
Se la prova può essere ripetuta fino all'eventuale successo senza problemi o interruzioni allora non fate fare la prova, descrivete i tentativi, le difficoltà incontrate e dichiarate il successo.
\end{narratore}

Ragionate su quanto è competente un personaggio al fine di evitare qualsiasi prova dal risultato scontato.

\subsection{Consapevolezza}\index{Consapevolezza}\label{consapevolezza2}

La Consapevolezza è una di quelle competenze che entra in gioco molto spesso.

Fate in modo che siano le domande ed i ragionamenti dei personaggi a rivelare gli indizi, una prova di Consapevolezza potrà essere fatta ogni qual volta ci sia da cercare qualcosa di non ovvio, qualcosa che deve essere cercato altrimenti non risulta immediatamente percettibile o intuibile, qualcosa che i giocatori desiderano trovare e che c'è ma non fanno la domanda giusta.

\begin{narratore}[Non sono le prove a comandare]
Non fate che siano le prove a governare il vostro gioco. \textbf{Fate giocare i giocatori}, fateli recitare, fateli partecipare ed in base a quanto dicono stabilite se la prova è passata o meno.

Se vi dicono \emph{convinco la guardia a farci passare} fate fare una prova di Intimidire (o Diplomazia), se invece intavolano un dialogo convincente potete considerare che la prova sia stata fatta con esito positivo (o negativo se non sono riusciti ad argomentare!) Premiate il COME più che il COSA.
\end{narratore}

\subsection{Le Prove}\index{Prove opposte}\label{proveopposte}\index{Prove Contrapposte}

\subsubsection{Prove di Competenza contrapposte ad un avversario}\index{Prove Contrapposte ad un avversario}

Ci sono situazioni in cui il personaggio deve effettuare una Prova Contrapposta ad un avversario ad esempio Furtività per muoversi silenziosamente alle spalle di una guardia, rubare dalle tasche del mercante, intimidire l'orchetto per farsi dare indicazioni, spingere un avversario..

In questo caso il personaggio effettua la prova indicata la cui \textbf{difficoltà (DC) è pari 10} + il punteggio della Caratteristica + Competenza oppure Tiro Salvezza (come indicato dalla prova) + modificatori (bonus/penalità) contingenti.

Chi ottiene il valore più alto vince, in caso di parità vince chi ha il valore più alto nella Competenza, poi nella Caratteristica ed infine l'eventuale \emph{avversario}. \index{DC Statica nelle prove contrapposte}\index{Prova Contrapposte}

\medskip

\textbf{Alcuni esempi di Prova Contrapposte}

\begin{description}[noitemsep, topsep=0pt, parsep=0pt, partopsep=0pt, leftmargin=0cm]
\item - \textbf{Ingannare qualcuno}: Ingannare Vs Percepire Emozioni
\item - \textbf{Travestirsi per sembrare qualcun altro}: Intrattenere Vs Consapevolezza
\item - \textbf{Creare una mappa falsa}: Falsificare Vs Valutare
\item - \textbf{Furtività} : Competenza Vs Consapevolezza, purché non visto
\item - \textbf{Intimidire}: Intimidire Vs Tiro Salvezza su Volontà (con modificatore Carisma)
\item - \textbf{Rubare}: Mani di Fata Vs Consapevolezza, o Mani di Fata se posseduta
\item - \textbf{Slegarsi da delle corde}: Usare Corde Vs Artista della fuga
\item - \textbf{Braccio di ferro}: Tiro Salvezza Tempra (con modificatore Forza)
\end{description}

\subsubsection{Prove di Caratteristica Contrapposte}\index{Prove di Caratteristica Contrapposte}

Ogni qual volta la \textbf{Prova} o \textbf{Prova Contrapposta} riguarda una \textbf{Caratteristica} e non anche una Competenza fate la prova (3d6) sommando alla Caratteristica più adeguata il Tiro Salvezza più adatto.

\smallskip

\textbf{Tabella: Prove Contrapposte}\index[Tabelle]{Tabella Prove Contrapposte e Caratteristiche}\label{Tabella Prove Contrapposte e Caratteristiche}

\smallskip

\noindent\begin{tabularx}{0.45\textwidth}{Xl}
\toprule
\textbf{Prova Contrapposta}& \textbf{TS} \\
\toprule
Forza& Tempra \\
Destrezza&Riflessi\\
Costituzione& Tempra\\
Intelligenza, Saggezza, Carisma& Volontà
\end{tabularx}

\smallskip

E' possibile che siano chieste Prove Contrapposte con indicato modificatori diversi. Quelli riportati nella tabella sopra sono esempi di utilizzo tipici. E' possibile fare un prova contrapposta di Forza, facendo un Tiro Salvezza su Tempra e sommando il punteggio di Forza per capire chi vince in una gara di sollevamento pesi.

\subsubsection{Prove di Caratteristica non Contrapposte}\index{Prove di Caratteristica non Contrapposte}

Alcune prove possono essere indicate come \emph{Esegui prova di Destrezza a DC 20} senza indicare Tiro Salvezza o Competenze. In questo caso è necessario effettuare la prova sommando esclusivamente il punteggio della caratteristica indicata. Es. 3d6 + 1 (il valore della Destrezza).

\subsubsection{Prove contro una DC statica}\index{Prove non contrapposte, statiche}

Qualora la Prova sia contrapposta ad un \textit{avversario statico}, ovvero non una creatura dotata di Caratteristiche ed Competenze, ma ad una serratura, un salto da compiere.. allora si esegue la prova confrontando 3d6 + la Caratteristica interessata + la Competenza Attiva (TS/CM/CA) o Competenza Base (Disattivare Congegni, Atletica...) più idonea contro la difficoltà (\textbf{DC}) stabilita dal Narratore.

\begin{enfasi}{Audentes fortuna iuvat (\emph{La fortuna aiuta gli audaci}, Virgilio) }\end{enfasi}

\subsection{Bonus e Penalità} \index{Vantaggi}\index{Bonus}\index{Malus}\index{Penalità}\index{Svantaggi}\label{vantaggi}

A seconda delle circostanze potranno esserci dei bonus, vantaggio o penalità, svantaggi nella prove.

Il modificatore nelle \textbf{prove dinamiche}\index{Prove dinamiche} è da usarsi quando la prova viene fatta tirando i 3d6, in questo caso si potranno sommare bonus o penalità (-1, +2...) o addirittura tirare dadi in più od in meno (+1d6, -2d6), fino a non tirare dadi (con 3d6 di penalità)!.

Se le penalità accumulate portano i dadi della prova sotto zero si conta solo il valore della Competenza e Caratteristica.

Si intendono \textbf{prove a valore fisso} \index{Prove a valore fisso} quando il valore non dipende dal tiro di dadi (es. Difesa), in questo caso il punteggio si alza/abbassa del valore indicato.

Cercate di rimanere sempre tra questi valori di bonus e penalità, altrimenti potete direttamente dire che la prova è riuscita o fallita.

Il giocatore può richiedere di effettuare la prova anche se il risultato è certo.

\smallskip

\textbf{Tabella: Vantaggi e Svantaggi}:\index[Tabelle]{Tabella Vantaggi e Svantaggi}

\smallskip

%\noindent\begin{tabular}{lll}
%\multirow{2}*{\textbf{Vantaggio / Svantaggio}} & \multicolumn{2}{c}{\textbf{Prove}}\\
%\cmidrule(lr){2-3} & \textbf{Dinamiche} & \textbf{Fisse} \\
%\hline
%Bonus leggero & +1& +1\\
%Bonus normale & +2 & +2\\
%Bonus forte & +1d6 & +4\\
%Bonus molto forte & +2d6 & +8\\
%Svantaggio leggero & -1 & -1\\
%Svantaggio normale & -2 & -2\\
%Svantaggio forte & -1d6 & -4\\
%Svantaggio molto forte & -2d6 & -8
%\end{tabular}

\noindent\begin{tabular}{lcc}
	\toprule
\multirow{2}*{\textbf{Vantaggio / Svantaggio}} & \multicolumn{2}{c}{\textbf{Prove}}\\
\cmidrule(lr){2-3} & \textbf{Dinamiche} & \textbf{Fisse} \\
\toprule
Leggero & $\pm$1 & $\pm$1\\
Normale & $\pm$2 & $\pm$2\\
Forte & $\pm$1d6 & $\pm$4\\
Molto forte & $\pm$2d6 & $\pm$8
\end{tabular}

\begin{narratore}[Il valore dei dadi]
I bonus e penalità nel tiro di 3d6 hanno più \emph{effetto} che nella prova fatta con il d20. Cercate di rimanere entro i $\pm2$ e solo in situazioni particolari di effettivo e forte vantaggio o svantaggio applicate bonus o penalità maggiori.
\end{narratore}

\subsubsection{Fattore tempo}\index{Fattore tempo}\label{fattoretempo}

\textbf{Se un personaggio non è in difficoltà o pressione}\index{Senza problemi di tempo}\index{Prendere il 10} nell'effettuare la prova può prendere il 10 (+ Caratteristica + Competenze + Abilità..), ovvero considerare che abbia tirato 10 con i dadi. L'azione impiega 10 round. \label{prendere10}

\textbf{Se il personaggio non ha impellenti limiti di tempo}, ovvero può dedicare almeno 10 minuti per lavorarci (60 round) può considerare di prendere 14. Ovvero come se avesse fatto la prova e tirato 14 con i 3d6. \label{prendere14}

\textbf{Se il tempo diventa un fattore da non considerare}, ovvero il personaggio ha almeno 1 ora per pensare e lavorare e non ha alcuna penalità o rischio considerare di avere tirato 18 (ma non c'è nessuna esplosione di dadi o Successo Critico anche se il totale è 18).\label{prendere18}

\begin{narratore}[Consiglio di lettura]
Consiglio a tutti di leggere l'ottimo articolo di Lorenzo Bertini \href{https://dietroschermo.wordpress.com/2022/03/10/elogio-del-10-e-del-20}{Elogio del 10 e del 20} per una disamina critica ed intelligente sul successo e fallimento delle prove.
\end{narratore}

Se vuoi prendere questi valori chiedilo al Narratore, sarà lui che ti dirà se in base alla situazione, urgenza, pericolosità di ciò che ti circonda riesci a prendere il punteggio. Mettersi a scassinare una porta in un dungeon chiedendo il 10 richiede un estremo sangue freddo ed incoscienza. Prendere il 10/14/18 non deve essere concesso per le prove di conoscenza.

\subsubsection{Aiutare un Altro nelle Prove}\label{aiutarealtro}\index{Aiutare un altro nelle Prove}

Si può aiutare un amico in una prova dandogli supporto e suggerimenti. L'aiutante deve effettuare la \textbf{medesima prova} con un bonus di +1d6, se ci riesce non ottiene effetti ma concede un +1 alla prova del compagno. Se esegue un Successo Critico allora il bonus è di +2. L'aiutante esegue la prova come Reazione nel round in cui viene effettuata la prova effettiva.

più personaggi possono aiutare lo stesso personaggio; i bonus di questo tipo sono cumulabili fino ad un bonus pari ad un quarto della difficoltà da battere (es +6 nel caso di difficoltà 25).\index{Aiutare un altro}

\textbf{In caso di prove basate su Competenze chi aiuta deve aver assegnato almeno un punto nella Competenza coinvolta}.

Il Narratore valuterà la possibilità che più di un personaggio fornisca aiuto considerando spazi, modi e tempi (non è facile aiutare qualcuno ad infilare un filo nella cruna di un ago).

Se la prova per aiutare fallisce in modo critico chi doveva essere aiutato ha un -1 di penalità nella prova.

\subsection{Prove fatte dal Narratore}\label{provefattedalnarratore}

Evitate di fare voi le prove al posto dei Giocatori. Siate descrittivi ma non andate a dire al Giocatore che \emph{potrebbe} servire una prova di qualcosa. Qualora dovesse essere necessario eseguire delle prove di nascosto dal giocatore non tirate nessun dado ma aggiungete a 10 il valore della Caratteristica ed il punteggio Competenza o il valore del Tiro Salvezza in questione del personaggio e confrontate il risultato con la difficoltà della prova.

\subsection{Tirare o non Tirare dadi}\label{tirarenontiraredadi}

Non fate tirare dadi per prove che non hanno possibilità di fallire, per le prove che non hanno o generano \textbf{problemi} se sono fallite o possono essere ritentate senza problemi. Fate tirare i dadi ogni qual volta la prova può avere un risultato \textbf{spettacoloso}, \textbf{fallimentare} o innesca ulteriori scene. Fate godere il giocatore del successo o temere del fallimento critico.

\subsubsection{Opzionale - Successo Parziale}\index{Successo Parziale}\index{Prova con Rischio}\index{Opzionale - Successo Parziale}\hypertarget{successoparziale}{}\label{successoparziale}

Una \textbf{Prova con Rischio} si chiede in prove di particolare tensione ed urgenza in cui è più importante il risultato finale che il rischio che si corre. Questa richiesta va fatta prima di tirare i dadi.

Se la prova fallisce di 1 potrà considerarsi riuscita anche se con un problema leggero, se è fallita di 2 si porta dietro un problema serio se è fallita di 3 è riuscita con un problema critico, se è fallita di 4 o più la prova non è comunque riuscita. Applicata a competenze come Conoscenza si può decidere di fornire informazioni non complete oppure in parte vere e false, oppure ancora se si tratta di aprire una serratura si potrebbe rompere il grimaldello nella serratura!

\subsection{Prove di Gruppo}\label{provedigruppo}\hypertarget{provedigruppo}{}\index{Prove di Gruppo}

Ci sono situazioni in cui il gruppo deve fare una prova di competenza ma il risultato deve essere unico, in questo caso se almeno metà del gruppo riesce nella prova questa ha successo.

\subsection{Esempi Prove Competenza}\label{esempiprovecompetenza}\hypertarget{esempiprovecompetenze}{}\index{Esempi prove Competenza}

\textbf{Prove atipiche}\index{Prove atipiche}. Il giocatore è invitato a trovare usi, soluzioni, approcci che esulino dalle più ovvie prove. Siate creativi e descrivete al Narratore la meravigliosa azione che volete fare e come farla! Sarà lui a stabilire in base alla vostra descrizione dell'azione cosa provare e quanto potrà essere difficile.

Le Competenza che hanno un \textbf{*} a fianco al nome, come \textbf{Acrobatica*} hanno le penalità alla prova dovuta dall'armatura portata.

\titlespacing*{\subsubsection}{0pt}{0.5em}{0.5em}\subsubsection*{Acrobatica*}\index{Acrobatica} \label{acrobatica}
Una prova di Acrobatica riuscita con DC 15 permette al personaggio di ridurre di 3 il danno quando cade entro 6 metri (\textbf{Reazione}).

\textbf{Scendere o Salire} entro 50 cm è terreno difficile, tra i 50 e 150 cm è terreno doppiamente difficile, oltre è cadere o arrampicarsi. Il danno da caduta è 1d6 danni ogni 3 metri in caduta. \index{Scendere e Salire}

Vedi paragrafo \hyperlink{cadute}{Cadute} (pag. \pageref{cadute}) per i dettagli su come usare Acrobatica quando si cade.

\titlespacing*{\subsubsection}{0pt}{0.5em}{0.5em}\subsubsection*{Arrampicarsi/Scalare*} \index{Arrampicarsi}\index{Scalare}\label{arrampicarsi}

Arrampicarsi, scalari o scendere da una superficie impervia equivale a muoversi in un \textbf{terreno doppiamente difficile}.

\medskip

\noindent\begin{tabularx}{0.47\textwidth}{Xl}
	\toprule
	\textbf{Esempio di Superficie} & \textbf{DC}\\
	\toprule
	Movimento solo dimezzato & -2d6\\
	Superficie scivolosa&+4\\
	{\small Parete grezza con appigli, mattoni sporgenti}&+12\\
	Un albero, una corda senza nodi&+15\\
	Un muro con pochi mattoni sporgenti &+20\\
	Un muro con pochissimi appigli&+25\\
	Una parete naturale liscia senza appigli&+30\\
	Ti puoi appoggiare a 2 pareti opposte&-8\\
	Ti puoi appoggiare a 2 pareti angolari&-4\\
	\midrule
	Esempi di prove con uso della corda&\\
	\midrule
	Usare una corda per calarsi&12\\
	Usare una corda per arrampicarsi&15\\
	La corda ha nodi & -3
\end{tabularx}

\medskip

In caso di fallimento della prova si consuma l'Azione senza spostarsi. Se si ottiene un Fallimento Critico perdi la presa e puoi fare un Tiro Salvezza su Riflessi alla stessa difficoltà per afferrarti a qualcosa, se fallisci anche il TS cadi fino in fondo.

In caso di \textbf{Successo Critico} nella prova scali, ti arrampichi o scendi come fosse terreno difficile. \textbf{Usare una corda} consente di trattare l'arrampicata come terreno difficile.

Vedi anche la \hyperlink{pareti}{Tabella: Pareti}, pag. \pageref{pareti}.

\titlespacing*{\subsubsection}{0pt}{0.5em}{0.5em}\subsubsection*{Arcana - Riconoscere un oggetto magico} \index{Riconoscere oggetto magico}\label{rinoscereoggettomagico}\hypertarget{rinoscereoggettomagico}{}

Per riconoscere un oggetto magico le sue capacità è necessaria una prova di \textbf{Arcana} a difficoltà 20 per avere indicazioni di massima sui poteri e ambiti di utilizzo, solo con un risultato di almeno 25 nella prova, puoi apprenderne i dettagli, bonus magici e cariche. \textbf{10 minuti}. Con punteggio Arcana 6 impiega 5 minuti, con 12 impiega 1 minuto, con Arcana 18 impiega 1 Round eseguire la prova.

\titlespacing*{\subsubsection}{0pt}{0.5em}{0.5em}\subsubsection*{Arcana - Riconoscere un incantesimo} \index{Riconoscere un incantesimo} \label{riconoscereincantesimo}\hypertarget{riconoscereincantesimo}{}
Mentre viene lanciato è una prova di \textbf{Arcana} a DC pari la 10 + livello dell'incantesimo. Costa una \textbf{Reazione}. Se fatto assieme al lancio di un \hyperlink{Controincantesimo}{Controincantesimo} non costa Reazione.

\titlespacing*{\subsubsection}{0pt}{0.5em}{0.5em}\subsubsection*{Atletica*}\index[Tabelle]{Tabella Saltare} \emph{Penalità dovuta all'Armatura.} \textbf{1 Azione}\index{Saltare}\label{atletica}\label{saltare}
%\begin{tabular}{lc|lc}
%\textbf{Salto in Lungo} & \textbf{DC}&\textbf{Salto in Alto} & \textbf{DC}\\
%\multicolumn{2}{c}{Lunghezza} &\multicolumn{2}{c}{Altezza} \\
%\toprule
%1.5 m & 5 & 0.02 m & 4\\
%3 m & 10 &0.5 m & 8\\
%5 m & 15 & 1 m & 12\\
%7 m & 20 & 1.5 m & 16\\
%+1,5 m & +5 &+0.5 m & +4\\
%\end{tabular}

La \textbf{distanza saltata in lungo} è pari a 30cm per risultato ottenuto nella prova, arrotondando all'intero più vicino. Es. se nella prova di saltare faccio 11, il salto sarà lungo 30cm*11=330cm=3 metri, con 16 nella prova è 30cm*16=480cm=5m.

La \textbf{distanza saltata in alto} è pari a 10cm per risultato ottenuto nella prova.

In un \textbf{salto in lungo} la punta più alta del salto è pari ad un 1/3 della lunghezza saltata. Se esegui un salto in lungo di 3 metri a metà salto sei in alto di 1 metro.

Se non si ha almeno 3 metri di rincorsa si salta la metà. In lungo si salta al massimo il proprio movimento ed in alto la metà.

Effettuare un Salto da fermo costa 1 Azione. Un Salto effettuato entro metà del proprio movimento (quindi si salta entro 4 metri percorsi per un umano) usa la stessa Azione del Movimento, altrimenti consumi una Azione per il Movimento ed una Azione per il Salto.

\titlespacing*{\subsubsection}{0pt}{0.5em}{0.5em}\subsubsection*{Conoscenza - Identificare una pozione o veleno naturale}\index{Identificare Veleno}\index{Erboristeria} \index{Identificare Pozione}\label{identificarepozioni}
è possibile con una prova di \textbf{Erboristeria} a DC uguale al fattore di rarità della pianta, oppure il TS che questa concede in caso di Veleni.

Impiega 1 Azione ogni 10 di DC. Con 6 in Erboristeria il tempo è 1 Azione ogni 15 di DC, con 12 punti è 1 Azione ogni 20 DC eseguire la prova. Se si fallisce con un Fallimento Critico si è venuti a contatto/ingerito parte della pozione e se ne subiscono gli effetti.

\titlespacing*{\subsubsection}{0pt}{0.5em}{0.5em}\subsubsection*{Conoscenza - Identificare una creatura} \index{Riconoscere una creatura}\label{riconosceremostro}\hypertarget{riconosceremostro}{}
si effettua una prova di Conoscenza. Controlla il capitolo \hyperlink{riconoscereimostri}{Riconoscere i Mostri} nel Mostruario (pag. \pageref{riconoscereimostri}). Costa 1 Azione.

\titlespacing*{\subsubsection}{0pt}{0.5em}{0.5em}\subsubsection*{Intimidire}\index{Intimidire}\label{intimidire}
Il personaggio usa \textbf{1 Azione} ed esegue una Prova Contrapposta al Tiro Salvezza su Volontà con bonus dato dal Carisma.
Se il Tiro Salvezza fallisce, l'avversario fino alla fine del suo round successivo ha -1 al Tiro per Colpire contro colui che l'ha intimidito. L'avversario deve avere Intelligenza pari o maggiore di -3. Il Tiro Salvezza prende un modificatore di $\pm2$ per taglia di differenza. In caso di Successo Critico il modificatore diventa -2.

Se chi tenta la prova di Intimidire esegue un Fallimento Critico subisce le medesime penalità come se fosse stato intimidito.

%\medskip

%\begin{center}
%	\includegraphics[width=0.7\linewidth]{immagini/Foster_Bible_Pictures.png}
%
%	\emph{Bible Pictures and What They Teach Us}
%\end{center}

\titlespacing*{\subsubsection}{0pt}{0.5em}{0.5em}\subsubsection*{Furtività*} \index{Nascondersi} \index{Muoversi silenziosamente}\index{Furtivita'}\label{furtivita}

Furtività raccoglie le capacità di muoversi silenziosamente, nascondersi nelle ombre, passare non visto e tutte quelle azioni che richiedono di non essere visti o sentiti.
Cercare di muoversi silenziosamente non costa Azioni, è \emph{compresa} nell'Azione di Movimento usata per spostarsi. Il terreno viene però trattato come difficile e se già lo fosse diviene doppiamente difficile. Muoversi a piena velocità cercando di non fare rumore impone alla prova di Furtività una penalità di 2d6.

Usando \textbf{1 Azione} puoi cercare di nasconderti dalla vista degli avversari. Non è possibile nascondersi se l'ambiente non lo permette, per quando la tua prova possa essere alta non puoi nasconderti se non c'è qualcosa che ti può nascondere od occultare. Per nascondersi dietro una creatura questa deve essere almeno di 3 taglie superiori alla tua (altrimenti la creatura fornisce solo copertura).

\titlespacing*{\subsubsection}{0pt}{0.5em}{0.5em}\subsubsection*{Gestire animali - Ammansire un animale}\index{Ammansire animale}\index{Gestire animali}\label{gestireanimali}
è una prova di \textbf{Gestire Animali} a DC 12+2*GS dell'animale. Impiega 1 minuto ogni 3 di DC, con 6 punti il tempo è 1 minuto ogni 6 di DC, con 12 è 1 minuto ogni 10 DC eseguire la prova. La creatura deve avere Intelligenza -3 o superiore.

\titlespacing*{\subsubsection}{0pt}{0.5em}{0.5em}\subsubsection*{Nuotare*}\index{Nuotare}\label{compnuotare}

In acqua calme DC 10, in acque mosse ha DC 15, in acque molto mosse DC 20, tempestose DC 25. La prova è necessaria per stare a galla o nuotare. Nuotare in acqua si considera \textbf{terreno difficile}.

Vedi Capitolo \hyperlink{combatteresottacqua}{Avventure in Acqua} (pag. \pageref{combatteresottacqua}).

\titlespacing*{\subsubsection}{0pt}{0.5em}{0.5em}\subsubsection*{Professione}
Un eventuale prova sulla \textbf{Professione} viene fatta con 3d6+Saggezza+metà del livello.

%\subsubsection*{Artista della Fuga}
%1 Azione ogni 10 di DC. 6p 1 Azione 15 di DC, 12p 1 Azione 20 DC.

\titlespacing*{\subsubsection}{0pt}{0.5em}{0.5em}\subsubsection*{Pronto Soccorso}\hypertarget{prontosoccorso}{}\label{prontosoccorso}\index{Pronto Soccorso}

Se il personaggio ha Punti Ferita negativi, è morente, la prova di Pronto Soccorso, 3 Azioni, a difficoltà 12 più il valore dei Punti Ferita negativi porterà il personaggio a 0 Punti Ferita, ovvero svenuto. Ogni volta successiva che il personaggio torna sotto 0 Punti Ferita la difficoltà della prova di Pronto Soccorso aumenta di 2.

Una prova riuscita (DC 15) fa recuperare 1d4 Punti Ferita \textbf{dopo uno scontro}, se il personaggio non è morente, o concede un +2 ad un Tiro Salvezza su Tempra per resistere ad un veleno. Da fare entro 1 Turno dal termine del combattimento. Costo \textbf{2 minuti}.

Con punteggio 6 costa 1 minuto e recuperi 1d4+4 PF. Con punteggio 12 costa 3 round e recuperi 2d4+8 PF, con punteggio 18 costa 1 round e recuperi 3d4+12 PF.

Una prova riuscita (base DC 12) riduce di 1 i danni da \hyperlink{sanguinamento}{\textbf{Sanguinamento}}. Per ogni valore di Sanguinamento sopra 1 la difficoltà aumenta di 2. Costo \textbf{2 Azioni}. Un trattamento di 1 minuto garantisce 1 successo, senza prova. Per ogni Successo Critico riduci il sanguinamento di un punto ulteriore.

Una prova riuscita (base DC 13) per \textbf{prendersi cura per 8 ore} di un paziente fa recuperare a questo il doppio dei Punti Ferita, con un minimo di 4, e concede un nuovo Tiro Salvezza su Tempra per debellare a Malattie naturali o Veleni già in corso.
Se effettuato durante le ore di riposo chi amministra la cura risulterà Affaticato.

Oggetti come \hyperlink{borsadaguaritore}{Borsa da Guaritore} (pag. \pageref{borsadaguaritore}) e \hyperlink{Fermasangue}{Fermasangue} (pag. \pageref{fermasangue}) possono essere utili nelle prove.

\titlespacing*{\subsubsection}{0pt}{0.5em}{0.5em}\subsubsection*{Seguire Tracce}\index{Seguire Tracce}\label{seguiretracce}

Alla \textbf{Difficoltà base di 15} si applicano i seguenti modificatori:

\medskip

\noindent\begin{tabularx}{0.5\textwidth}{ll}
	Se il terreno è molto morbido& DC -4\\
	Se il terreno è stabile& DC +5\\
	Se il terreno è duro& DC +10\\
	A seconda della taglia& DC $\pm4$\\
	Ogni 3 creature inseguite& DC -2\\
	Ogni 24 ore passate& DC +4\\
	Ogni ora di pioggia& DC +4\\
	Visibilità scarsa& DC +2\\
	Cerca di occultare le tracce& DC +4
	%Se il terreno è molto morbido& DC +4\\
	%Se il terreno è stabile/duro& DC +15/20\\
	%Ogni 3 creature inseguite& DC -2\\
	%A seconda della taglia/visibilità& DC $\pm4$\\
	%Ogni 24 ore passate/ora di pioggia& DC +4\\
	%Cerca di occultare le traccie& DC +4\\
\end{tabularx}

\titlespacing*{\subsubsection}{0pt}{0.5em}{0.5em}\subsubsection*{Valutare}\label{compvalutare}\index{Valutare}
DC 12 + fattore rarità oggetto. Comune +0, Non Comune +2, Raro +6, Molto Raro +10, Leggendario +16. 3 Azioni

\titlespacing*{\subsubsection}{0pt}{0.5em}{0.5em}\subsubsection*{Sopravvivenza}\index{Sopravvivenza}\label{sopravvivenza}

Sopravvivenza può essere usata al posto di \textbf{Disattivare Congegni} con un -1d6 per disattivare trappole o serrature. 1 Azione per DC.

Ogni tre punti ottenuti nella prova di Sopravvivenza oltre la DC (solitamente 13) il personaggio è in grado di \textbf{procacciare cibo} per se stesso ed un altra persona purché si trovi in un ambiente capace di sostenere la vita.

Si può usare per cercare trappole: 1 minuto per cercare trappole in 3x3 metri, con punteggio 6 costa 3 round, con punteggio di 12 costa 1 round, con punteggio 18 costa 1 Azione.

\subsubsection*{Punteggio Competenze}\label{Punteggio Competenze}\index{Punteggio Competenze}

Quando nel manuale si parla di \emph{valore o punteggio Competenza} si intende sempre il valore della competenza compreso di tutti i punteggi e modificatori.

\begin{giocatore}[Prove Prove e Prove!]
	Ad essere cinici un gioco di ruolo è tutta una prova, vuoi per riuscire a fare un salto, per colpire qualcuno, per evitare una trappola od un incantesimo...!
	Devi essere più intelligente e furbo. Le prove possono essere spesso evitate o affrontate con vantaggio. Gioca con arguzia, usa la tua immaginazione, sii creativo!
\end{giocatore}


\begin{narratore}[Il ruolo delle Prove]
	L'esecuzione e la gestione delle prove determina il tipo di gioco. È fondamentale ascoltare i giocatori, percepire il loro entusiasmo e comprendere gli obiettivi delle loro azioni. Un giocatore coinvolto trasmette entusiasmo a tutto il gruppo.

	%Quando i giocatori chiedono semplicemente di fare una \emph{prova Consapevolezza} o di \emph{convincere la guardia}, assecondate le loro intenzioni ma cercate di coinvolgerli maggiormente.

	\textbf{Non c'è una regola per tutto ma divertimento e buon senso non devono mai mancare}!
\end{narratore}


\titlespacing*{\subsubsection}{0pt}{0.5em}{0.5em}\subsubsection*{Linguaggi}\index{Linguaggi}\hypertarget{linguaggi}{}\label{linguaggi}

Nel mondo ci sono le vecchie lingue umane, usate solo negli antichi tomi ed in comunità isolate e c'è la lingua Comune costruita dall'insieme dei vecchi idiomi terrestri e comprensibile più o meno a chiunque. Ogni personaggio che abbia almeno Intelligenza -2 parla il linguaggio della propria cultura, con 0 lo scrive. Per ogni punto pari o superiore a 2 parla e scrive un altra lingua che sarà scelta alla creazione del personaggio. Per ogni punto speso nella Competenza Conoscenza Lingua parla e scrive un altra lingua.

Le lingue segnate con un \textbf{*} possono essere parlate solo da creature appartenenti a quella specie o gruppo culturale.

Le creature extraplanari come Celestiali, Demoni, Diavoli, Draghi, Elfi, Nani, Gnomi ... parlano e scrivono le proprie lingue.

\smallskip

\textbf{Tabella delle Lingue}\index[Tabelle]{Tabella delle Lingue}

\smallskip

{\noindent\begin{tabularx}{0.48\textwidth}{lll}
		\toprule
\textbf{Ambito culturale}& \textbf{Parlato} & \textbf{Scritto}\\
\toprule
%Umano & Comune& Comune\\
%Nanico& Nanico& Nanico\\
%Elfico& Elfico & Elfico \\
%Gnomico& Gnomica & Gnomica \\
Vecchie lingue terrestri& varie & varie\\
%Gnoll & Gnoll & Goblinoide\\
%Giganti& Gigante & Gigante\\
%Orco& Orchesco & Orchesco \\
Creature marine* & Aquan& Elfico\\
%Creature marine senzienti* & Aquan& - \\
Uccelli senzienti& Ictun & -\\
Abitanti dei boschi& Silvano& - \\
%Celestiale& Celestiale & Celestiale\\
%Infernale & Infernale & Infernale\\
%Abissale & Abissale& Abissale\\
%Draghi& Draconico & Draconico\\
Elementali del Fuoco* & Ignan&-\\
Elementali della Terra*& Tremun &-\\
Elementali dell'Acqua* & Aquan & - \\
Elementali dell'Aria*& Ictun &-\\
Non-morti & Expiran & - \\
%Sottosuolo & Profondità& Profondità\\
%Lingua dei Segni*& dei Segni & - \\
\end{tabularx}}

\medskip

La \textbf{Telepatia}\index{Telepatia} è un mezzo per parlare con qualsiasi creatura che abbia Intelligenza almeno -3. Non c'è il vincolo del linguaggio, la telepatia funge da traduttore universale.

%\begin{center}
%\includegraphics[width=0.8\linewidth]{immagini/Pieter_Bruegel_the_Elder-The_Tower_of_Babel.png}
%
%\emph{La Torre di Babele, Pieter Bruegel il Vecchio.}
%\end{center}

%{\small

\end{multicols}

\pagebreak

