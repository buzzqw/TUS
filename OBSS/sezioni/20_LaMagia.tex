\section{La Magia}\index{Magia}\label{lamagia}

\begin{enfasi}{
La magia non è nel pendolino, ma in chi lo usa. (NCIS - Unità anticrimine)

\medskip

Non lascerai vivere colei che pratica la magia. (Libro dell'Esodo)(Sempre a seconda dei propri Tratti...)

\medskip

Uno stregone non è mai in ritardo, Frodo Baggins. Né in anticipo. Arriva precisamente quando intende farlo. (Gandalf, Il Signore degli Anelli - La Compagnia dell'Anello. J.R.R. Tolkien)} \end{enfasi}

\begin{multicols}{2}

La magia permea i mondi di gioco e la sua forma più comune è quella di un incantesimo. Questo capitolo fornisce le regole per lanciare incantesimi.

\medskip

\subsection{Cos'è un Incantesimo?}\index{Cos'è incantesimo}\index{Incantesimo definizione}

Un incantesimo è una manifestazione del potere di un Patrono. Ogni incantesimo è frutto di potere e conoscenza, l'incantatore è un tramite superiore che canalizza la potenza dei Patroni. Nel lanciare un incantesimo, un personaggio compone gesti, parole ed usa oggetti che altro non fanno che collegarlo alla fonte, il Patrono che sovraintende quella lista di magia.

\subsection{Come fare magie! (In sintesi)}\index{Come fare magie! (In sintesi)}

Il tuo personaggio deve avere investito un punto in Competenza Magica.

La Competenza Magica ti permette di avere più Punti Magia, più incantesimi conosciuti e grazie all'Abilità Adepto della Magia anche di rendere i tuoi incantesimi più difficili da resistere. Un punteggio alto di Competenza Magica insieme all'Abilità Adepto della Magia ti permette di accedere a incantesimi di più alto livello.

Non scordarti di cercare antichi tomi e pergamene! Gli incantesimi sono un bene raro e prezioso non perdere l'occasione per trovarne di nuovi e copiarli sul tuo Tomo della Magia.

\subsection{Le caratteristiche degli incantesimi}\index{Le caratteristiche degli incantesimi}\label{caratteristicheincantesimi}

La descrizione di ciascun incantesimo inizia con un blocco di informazioni che comprende il livello, le Liste di Magia a cui appartiene, tempo di lancio, gittata e durata dell'incantesimo. Il resto della descrizione ci informa dell'effetto dell'incantesimo.

Quando un personaggio lancia qualsiasi incantesimo, si usano le seguenti regole base indipendentemente dall'effetto dell'incantesimo.

\subsubsection{Tempo di Lancio}\index{Tempo di Lancio Incantesimi}\label{magietempodilancio}\index{Incantesimi, Azioni per lanciare}\index{Casting Time}\hypertarget{magietempodilancio}{}

La maggior parte degli incantesimi possono essere lanciati con due Azioni. Alcuni incantesimi richiedono un'Azione Immediata, una Azione di Reazione o molto più tempo per essere lanciati. Nella descrizione degli incantesimi è indicata come \emph{T. di Lancio}.

Durante lo stesso round non puoi lanciare un altro incantesimo, a meno che non si tratti di un incantesimo di livello 0 (chiamati Trucchetti).

\smallskip \textbf{Azione Immediata} \smallskip

Un incantesimo lanciato con un'Azione Immediata è particolarmente rapido. Puoi lanciare un solo incantesimo a round come Azione Immediata e non devi averla già usata.

\smallskip \textbf{Reazioni} \smallskip

Alcuni incantesimi possono essere lanciati come Reazioni. Questi incantesimi richiedono una frazione di secondo per essere formulati e possono essere lanciati in risposta a un evento. Se un incantesimo può essere lanciato come Reazione, la descrizione dell'incantesimo ti dice esattamente quando puoi farlo. Devi avere a disposizione una Azione di Reazione e non averla già usata.

\smallskip \textbf{Tempo di Lancio più Lungo} \smallskip

Certi incantesimi richiedono più tempo per essere lanciati: round, minuti o addirittura ore.
Quando lanci un incantesimo con un tempo di lancio di 1 round, significa che passi un intero round a formulare l'incantesimo e il round successivo usando 1 Azione tiri l'incantesimo con l'iniziativa che hai tirato all'inizio della formulazione.

Quando la formulazione dura più di 1 round devi spendere 1 Azione a round per continuare la formulazione. Per quei round è come se dovessi mantenere la Concentrazione per determinare eventuali effetti.

\subsubsection{Le Liste di Magia}\hypertarget{lescuoledimagia}{} \index{Le Liste di Magia}\label{magielistadimagia}\index{Incantesimi, Liste di Magia}

Le Liste qui presentate sono quelle codificate ed insegnate nelle poche scuole di magia.

%Si racconta di ulteriori liste create, curate e diffuse in circoli ristretti o sette. Una di queste liste segrete è quella degli gnomi Devoti di Shayalia, una lista prettamente naturale che mescola la Lista tradizionale degli Animali e Piante con alcuni incantesimi dalle Liste degli elementi.
%Altre liste più oscure sono quelle demoniache o degli Aboleth, alcune altre sono legate all'appartenenza a gruppi di Devoti. Altre liste più nefaste arrivano a corrompere l'anima dei personaggi imponendo anche i Tratti. Al personaggio queste liste saranno normalmente chiuse ma non è detto che con l'aumentare della Competenza Magica non sia lui stesso a creare nuove liste di incantesimi.

Le Liste di Magia aiutano a descrivere gli incantesimi; non hanno delle proprie regole, sebbene alcune regole possano fare riferimento a queste liste.

\begin{itemize}[leftmargin=*] \setlength{\itemsep}{0pt}
\item
\emph{Abiurazione} riguarda incantesimi di natura protettiva, sebbene ne contenga anche alcuni dall'uso aggressivo. Questi incantesimi creano barriere magiche, negano effetti dannosi o bandiscono le creature in altri piani di esistenza.

\item
\emph{Acqua} sono gli incantesimi che agiscono sull'elemento acqua e freddo ed in minima parte anche sulle cure

\item
\emph{Aria} riguarda gli incantesimi che manipolano ed usano l'aria ed anche l'elettricità.

\item
\emph{Ammaliamento} riguarda incantesimi che agiscono sulla mente altrui, influenzandone o controllandone il comportamento. Questi incantesimi possono far sì che i nemici considerino l'incantatore un amico od addirittura controllare un'altra creatura come fosse una marionetta.

\item
\emph{Animali e Piante} questi sono gli incantesimi che agiscono su animali e piante, naturali o magiche.

\item
\emph{Cura} riguarda gli incantesimi che permettono di recuperare le energie fisiche, mentali ed annullare debolezze e veleni.

\item
\emph{Divinazione} riguarda incantesimi che rivelano informazioni perdute nel tempo, dimenticate, visioni del futuro, la posizione di oggetti nascosti, la verità dietro le illusioni od immagini di persone e luoghi lontani.

\item
\emph{Evocazione} riguarda incantesimi che trasportano oggetti e creature da un luogo all'altro. Alcuni incantesimi richiamano creature o oggetti al fianco dell'incantatore, mentre altri permettono all'incantatore di teletrasportarsi da un luogo a un altro. Alcune evocazioni creano oggetti o effetti dal nulla.

\emph{Fuoco} Gli incantesimi più pericolosi sono qua dentro con tutto ciò che serve a bruciare ed incenerire.

\begin{center}
	\includegraphics[width=0.6\linewidth]{immagini/Leonids-1833.png}

	\emph{The most famous depiction of the famous 1833 Leonids \hyperlink{sciamedimeteore}{Meteor Storm} (Pioggia di Meteore!)}
\end{center}

\emph{Illusione} riguarda incantesimi che ingannano i sensi e la mente altrui. Fanno vedere alle persone cose che non esistono, non gli fanno notare le cose che esistono, fanno udire rumori fasulli o ricordare cose che non sono mai accadute. Alcune illusioni creano immagini spettrali che chiunque può vedere.

\item
\emph{Invocazione} riguarda incantesimi che manipolano l'energia magica per produrre un effetto desiderato.

\item
\emph{Necromanzia} riguarda incantesimi che manipolano le energie della vita e della morte. Questi incantesimi possono conferire una riserva aggiuntiva di forza vitale, risucchiare l'energia vitale da un'altra creatura, creare non morti o addirittura riportare in vita i morti (se concesso).

\emph{In OBSS solo un Patrono ha sufficiente potere per poter riportare in vita un morto}.

\item
\emph{Terra} Gli incantesimi che agiscono e muovono la terra

\item
\emph{Trasmutazione} riguarda incantesimi che cambiano le proprietà di una creatura, oggetto o ambiente.

\item
\emph{Universale} alcuni incantesimi sono capisaldi della magia in se e come tali accessibili a tutti gli incantatori. Per accedere agli incantesimi contenuti in questa Lista di Magia è necessario avere almeno un punto in Competenza Magica. Il massimo livello di incantesimi lanciabile in questa lista è pari numero di volte che si è presa l'Abilità Adepto della Magia, con un minimo di 1.

\end{itemize}

\subsubsection{Gittata}\index{Gittata}\label{magiegittata}\index{Incantesimi, Gittata}

Il bersaglio di un incantesimo deve essere nella gittata dell'incantesimo. Per un incantesimo come Dardo arcano, il bersaglio è una creatura. Per un incantesimo come palla di fuoco, il bersaglio è il punto nello spazio da cui la sfera di fuoco esplode. La maggior parte degli incantesimi hanno una gittata espressa in metri. Alcuni incantesimi possono prendere a bersaglio solo una creatura (te compreso) con cui sei in contatto fisico. Altri incantesimi, come l'incantesimo scudo, agiscono solo su di te: questi incantesimi hanno come gittata \emph{personale}. Un incantesimo che ha come area di effetto \emph{un alleato} può essere lanciato anche su se stesso.

Gli incantesimi che creano coni o linee di effetto che originano da te, hanno anch'essi gittata personale\index{Gittata Personale}, a indicare che sei tu il punto di origine dell'effetto dell'incantesimo (vedi \emph{Aree di Effetto} più avanti in questo capitolo).

\subsubsection{Durata}\index{Durata Incantesimi}\label{magiedurata}\index{Incantesimi, Durata}\hypertarget{magiedurata}{}

La durata di un incantesimo è la lunghezza di tempo per cui esso persiste. La durata può essere espressa in round, minuti, ore o addirittura anni. Alcuni incantesimi specificano che i loro effetti durano finché l'incantesimo non viene dissolto o distrutto. Un \textbf{incantesimo può essere interrotto dal proprio incantatore come Azione Immediata}.\index{Interrompere un proprio incantesimo}

Qualora un critico magico raddoppi la durata si intente sempre riferita alla durata iniziale. Es. se la durata è 2 ore dopo il primo raddoppio diventa 4 ore, con il secondo diventa di 6 ore e poi 8 ore..\index{Successo critico magico sulla durata}

\begin{itemize}[leftmargin=*] \setlength{\itemsep}{0pt}

\item
\emph{Istantanea}

Molti incantesimi sono istantanei. L'incantesimo ferisce, cura, crea o altera una creatura o un oggetto in modo che non possa essere dissolto, dato che la sua magia esiste solo per un istante.

\item

\emph{Concentrazione}\index{Concentrazione}\index{Incantesimi, Durata Concentrazione}

Alcuni incantesimi richiedono che tu mantenga la concentrazione per tenerne la magia attiva. Se non puoi mantenere la concentrazione, l'incantesimo avrà fine. Se un incantesimo deve essere mantenuto tramite concentrazione, la cosa è indicata alla voce Durata, l'incantesimo specifica quanto a lungo vi potrai mantenere la concentrazione. Puoi terminare la concentrazione in qualsiasi momento usando una Reazione.

Normali attività, come muoversi e attaccare, non interferiscono con la concentrazione. Mantenere la concentrazione costa 1 Azione a round.
\end{itemize}

\subsubsection{Formulare gli incantesimi}\index{Formulare gli incantesimi}\label{magiecomponenti}\index{Incantesimi, Formulare gli incantesimi}

Ogni incantesimo prevede che l'incantatore abbia le mani libere e possa parlare.

La maggior parte degli incantesimi richiede di intonare parole mistiche e gesticolare in maniera particolare. Le parole ed i gesti, il ritmo, la cadenza e risonanza permettono la sintonia con il Patrono che fornisce la magia.

E' possibile consumare oggetti al momento di lancio dell'incantesimo come offerta al proprio Patrono, o quello che sovraintende la Lista di Magia dell'incantesimo, per ottenere vantaggi. A seconda della \emph{preziosità} e \emph{storia} dell'oggetto offerto, a discrezione del Narratore, la Prova di Magia può prendere $\pm2d6$\ di modificatore.

\subsubsection{Recuperare da morente}\index{Recuperare da morente}\label{magieessereucciso}\index{Incantesimi, Inabile}

Se scendi a zero o sotto gli zero Punti Ferita perdi la metà dei Punti Magia rimanenti, con un minimo di 10 Punti Magia persi. Tutti gli incantesimi su cui stai tenendo la concentrazione vengono interrotti.

\subsubsection{Lanciare Incantesimi in Armatura}\index{Lanciare Incantesimi in Armatura}\label{magielanciareincantesimiinarmatura}\index{Incantesimi, in Armatura}

Data la concentrazione mentale e i gesti precisi richiesti l'armatura distrae e sbilancia i flussi. La Prova di Magia nel lancio dell'incantesimo è obbligatoria e viene modificata come indicato nella sezione delle \hyperlink{armatureemagie}{armature} (pag. \pageref{armatureemagie}).

\subsubsection{Opzionale - Incantesimi in Armatura}\index{Opzionale - Incantesimi in armatura}

E' proposta questa opzione per gestire gli incantatori in armatura:

- Questa opzione prevede tutti gli incantesimi lanciati dall'incantatore diventino con Gittata a Contatto, ovvero scaricabili solo tramite la mano dell'incantatore. Non sono richieste Prove di Magia per il fatto di portare l'armatura.


\subsubsection{Bersagli}\index{Bersagli}\label{magiebersagli}\index{Incantesimi, Bersagli}

Un normale incantesimo richiede che tu scelga uno o più bersagli che siano affetti dalla sua magia. La descrizione dell'incantesimo ti dice se l'incantesimo prende a bersaglio creature, oggetti o un punto di origine per generare un'area di effetto. A meno che l'incantesimo non abbia un effetto percepibile, una creatura potrebbe non capire mai di essere stata bersaglio di un incantesimo. Un effetto come un fulmine crepitante è palese, ma un effetto più subdolo, come il tentativo di leggere i pensieri di una creatura, di solito non viene notato, a meno che l'incantesimo non dica altrimenti.

Lanciare un incantesimo è una azione che non passa inosservata. Una prova di Furtività a difficoltà 15 oppure lanciare l'incantesimo come se si fosse Distratto permettono di celare il lancio, se non avviene proprio davanti all'osservatore.

\subsubsection*{Traiettoria Sgombra Verso il Bersaglio}\index{Incantesimi, vedere bersaglio}

\textbf{Per prendere come bersaglio una creatura od oggetto}, devi vederlo ed avere la traiettoria sgombera verso di essa, e quindi questa \textbf{non può trovarsi dietro una copertura completa}. Se piazzi un'area di effetto in un punto che non puoi vedere e un'ostruzione, come un muro, si trova tra di te e quel punto, il punto di origine si crea dal tuo lato più vicino dell'ostruzione (una Palla di Fuoco dietro una porta chiusa esplode al contatto con la porta dalla tua parte e non si manifesta oltre la porta).\index{Magia vedere il bersaglio}

\subsubsection*{Prendere Te stesso come Bersaglio}\index{Se stesso come bersaglio}\index{Incantesimi, se stesso come bersaglio}

Se un incantesimo prende come bersaglio una creatura a tua scelta od un alleato, puoi scegliere anche te stesso, a meno che la creatura non debba essere ostile o sia specificato che non possa essere tu. Se ti trovi nell'area di effetto di un incantesimo lanciato da te, anche tu ne sarai influenzato.

\subsubsection{Aree di Effetto}\index{Area di Effetto incantesimi}\label{magieareedieffetto}\index{Incantesimi, Area di effetto}

Incantesimi come Onda rovente e cono di freddo coprono un'area, permettendogli di colpire più creature alla volta.

La descrizione di un incantesimo specifica la sua area di effetto, che di solito rientra in una di queste cinque forme: cilindro, cono, cubo, linea o sfera. Ogni area di effetto ha un punto di origine, un luogo da cui si manifesta l'energia dell'incantesimo. Le regole per ciascuna forma specificano come posizionare il suo punto di origine. Di solito il punto di origine è un punto nello spazio, ma alcuni incantesimi hanno un'area la cui origine è una creatura o un oggetto. Il punto di origine deve essere sempre valido.

\begin{itemize}[leftmargin=*] \setlength{\itemsep}{0pt}
\item \emph{\textbf{Cilindro}}: il punto di origine di un cilindro è il centro di un cerchio di specifico raggio come indicato nella descrizione dell'incantesimo. L'energia in un cilindro si espande in linea retta dal punto di origine al perimetro del cerchio, formando la base del cilindro. L'effetto dell'incantesimo parte poi dal basso verso l'alto o dall'alto verso il basso fino a una distanza uguale all'altezza del cilindro. Il punto di origine del cilindro è incluso nella sua area di effetto.

\item \emph{\textbf{Cono}}: un cono si estende in una direzione a tua scelta dal suo punto di origine. La larghezza del un cono in un dato punto della sua lunghezza è uguale alla distanza di quel punto dal punto di origine. L'area di effetto di un cono specifica la sua lunghezza massima. Il punto di origine del cono non è incluso nella sua area di effetto a meno che tu non decida altrimenti.

Es. Un Cono di Freddo di 9 metri è largo al termine 9 metri e si allunga dal punto di origine di 9 metri, a 3 metri di distanza dal punto di origine è largo 3 metri.

\item \emph{\textbf{Cubo}}: selezioni il punto di origine di un angolo del cubo. Le dimensioni del cubo vengono espresse come lunghezza di ciascun suo spigolo. Il punto di origine del cubo non è incluso nella sua area di effetto, a meno che tu non decida altrimenti.

\item \emph{\textbf{Linea}}: una linea si estende dal suo punto di origine in un percorso dritto per tutta la sua lunghezza e copre un'area definita dalla sua larghezza. Il punto di origine della linea non è incluso nella sua area di effetto, a meno che tu non decida altrimenti. Una linea se non specificato diversamente è larga un quadretto.


\begin{center}
	\includegraphics[width=0.6\linewidth]{immagini/3dformev2.png}

	\emph{Cono, Sfera, Cilindro, Cubo. Il punto nero indica l'origine dell'incantesimo. Nella sfera è al centro della stessa.}
\end{center}

\item \emph{\textbf{Sfera}}: selezioni il punto di origine di una sfera, che deve essere valido (vedi Gittata e Bersagli) e la sfera si estenderà da quel punto fino ad incontrare un ostacolo insormontabile o la sua dimensione espressa nel raggio. La misura della sfera è indicata come raggio in metri che si estende da quel punto. Il punto di origine della sfera è incluso nella sua area di effetto.

Una palla di fuoco che viene generata in una stanza di 9x9 m ne prenderà una buona parte e in una stanza di 6x6 m la riempirà tutta. In una stanza di 3x3 m se ha modo di uscire da una porta od una finestra continuerà la sua esplosione fino ad arrivare ai 6 metri di raggio. Una palla di fuoco in un corridoio di 3x3 m lo saturerà per 6 metri avanti e indietro dal punto di origine.

\end{itemize}

\subsubsection{Rarità degli Incantesimi}\index{Rarità degli Incantesimi}\label{magieraritaincantesimi}\index{Incantesimi, Rarita' incantesimi}

Negli incantesimi è indicata la Rarità ovvero quanto è probabile trovare questo incantesimo o quanto può essere conosciuto. La rarità dipende non solo dal livello stesso dell'incantesimo, ovviamente gli incantesimi più potenti sono anche i più rari, ma anche da quanto normalmente sono diffusi e conosciuti. Il Narratore userà questa scala su 3d6 per valutare cosa può essere trovato più facilmente: % Comune (1-75\%) - Non Comune (76-93\%) - Raro (94-97\%) - Molto Raro (98-99\%) - Leggendario (100\%).
%, (1-70,71-93,94-97,98-99,100)
Comune (1-14) - Non Comune (15) - Raro (16) - Molto Raro (17) - Leggendario (18).

\subsubsection{Combinare Effetti Magici}\index{Combinare Effetti Magici}\label{magiecombinareeffettimagici}\index{Incantesimi, Combinare effetti}

Gli effetti di incantesimi diversi si sommano fino a che la loro durata si sovrappone. Gli effetti dello stesso incantesimo o che danno lo stesso bonus lanciato più volte sullo stesso bersaglio non si combinano. Sarà invece l'incantesimo più potente fra quelli lanciati, quello di livello più alto ed a parità quello che ha ottenuto più Critici Magici ad applicarsi finché le durate si sovrappongono.

In caso di incantesimi istantanei gli effetti agiscono singolarmente se agiscono nel medesimo segmento di iniziativa. Es. Se vengo colpito da un fulmine a segmento di iniziativa 4 e poi da un altro fulmine a segmento di iniziativa 8 farò due distinti Tiri Salvezza con relativa gestione del danno, se fossero nel medesimo segmento di iniziativa subirei solo quello più potente (vedi sopra).

%\subsection{Regole di base}\index{Regole Base per la Magia}\label{magieregoledibase}\index{Incantesimi, Regole base}

\subsection{Regole di base}\label{magieregoledibase}

\begin{itemize}[leftmargin=*] \setlength{\itemsep}{0pt}

\item
L'incantatore al lancio del suo primo incantesimo sceglie se utilizzare come modificatore alla Prova di Magia l'Intelligenza oppure se è un Devoto può scegliere la Caratteristica indicata dal Patrono. Una volta fatta la scelta non è più possibile cambiarla.

Questo modificatore viene chiamato \textbf{modificatore di caratteristica per incantesimi}.\index{Modificatore di caratteristica per incantesimi}
\item
Il personaggio quando assegna il primo punto di Competenza Magica \textbf{conosce} (sono presenti) nel suo Tomo della Magia un numero di Trucchetti pari al modificatore di caratteristica per incantesimi +2 (con un minimo di 4 Trucchetti) ed un numero di incantesimi di primo livello pari allo stesso modificatore, con un minimo di 4.
\item
Ogni giorno, dopo il riposo, il personaggio \textbf{apprende} dal sul suo Tomo di Magia un numero di incantesimi pari a Competenza Magica/2 (minimo 1) + modificatore di caratteristica per incantesimi + Adepto della Magia.\label{incantesimicm1}\hypertarget{incantesimicm1}{}
\item
Il numero di incantesimi formulabile al giorno dipende dalla capacità dell'incantatore. Vedi \textbf{Tabella Punti magia e Competenza Magica}. Un incantesimo ha un costo in Punti Magia pari al suo livello.
\item
Un Seguace aggiunge +1d6 alle Prove di Magia negli incantesimi delle liste privilegiate dal Patrono. I tuoi incantesimi possono usare una delle forme energetiche preferite dal Patrono.\index{Liste Privilegiate}\label{listeprivilegiate}\hypertarget{listeprivilegiate}{}
\item
Un Devoto aggiunge +1d6 alle Prova di Magia negli incantesimi delle liste privilegiate dal Patrono e può ignorare un dado tirato nella Prova di Magia. I tuoi incantesimi usano una delle forme energetiche preferite dal Patrono.
\item
Con il termine \textbf{appreso}\index{Incatesimi, Appreso} si intende un incantesimo presente sul Tomo della Magia che si è memorizzato e si può lanciare quando voluto.
\item
Con il termine \textbf{conosciuto}\index{Incantesimi, Conosciuto} si intente un incantesimo presente sul Tomo della Magia che però non si è appreso, ovvero non si è memorizzato e non si può lanciare quando voluto.
\end{itemize}

\subsection{Massimo livello di incantesimo lanciabile}\hypertarget{scuoleelivelli}{}\index{Livello Incantesimi per Abilità}\label{magieaccessoallelistedimagia}\index{Incantesimi, Massimo livello di incantesimo lanciabile}\index{Massimo livello di incantesimi lanciabili}\label{scuoleelivelli}

Mentre la Competenza Magica indica lo studio e dedizione alla Magia nella forma più astratta è l'Abilità Adepto della Magia che permette di capire quanto si è \emph{votati} al formulare gli incantesimi.

Per stabilire il livello massimo lanciabile di incantesimi sommate il punteggio di Competenza Magica ed Adepto della Magia, dividendo per due ed arrotondando per eccesso. Confrontate il risultato con il (doppio del punteggio del modificatore di caratteristica per incantesimi)+1, prendendo il valore minore.

Es. CM=8, Adepto della Magia preso 4 volte, (8+4)/2=6lv.

Es. CM=16 e Adepto della Magia 1 volta, (16+1)/2=9 livello di incantesimi.

Se l'incantatore ha come modificatore di caratteristica per incantesimo 0 non potrà lanciare incantesimi superiori al primo livello (vedi anche Abilità \hyperlink{Tutt'uno con la magia}{Tutt'uno con la magia}, pag. \pageref{Tutt'uno con la magia}).

Negli esempi sopra se il modificatore di caratteristica per incantesimi è 3 il massimo livello lanciabile sarà il 6lv e 7lv rispettivamente.

\subsection{Distratto - Problemi nel lancio dell'incantesimo}\index{Distratto - Problemi nel lancio dell'incantesimo}\index{Distratto}\label{magiedistratto}

Se l'incantatore è \textbf{Distratto}, cerca di nascondere il lancio della magia, è impedito, severamente disturbato, è sanguinante, afferrato, è sotto attacco/minacciato mentre cerca di lanciare un incantesimo, \textbf{che non sia un Trucchetto}, deve effettuare una \textbf{Prova di Magia}.

\subsection{Prova di Magia}\index{Prova di Magia}\index{Successo critico magico}\index{Fallimento critico magico}\label{magieprovadimagia}\index{Incantesimi, Prova di Magia}

Non sempre lanciare un incantesimo è sufficiente, molte volte è necessario che questo funzioni bene ed anzi agisca oltre normali aspettative. L'incantatore può decidere di richiamare più energia nel lancio dell'incantesimo, ovvero effettuare un \emph{\textbf{Prova di Magia}} e confidare nelle sue capacità.

L'incantatore tira \textbf{3d6 + 1d6 ogni tre punti di Competenza Magica} (arrotondato per eccesso) più eventuali bonus, Abilità o penalità (armatura, scudi, critici subiti).

L'incantatore può \textbf{ignorare un dado tirato} nella Prova di Magia per \textbf{ogni due volte} che ha preso \textbf{Adepto della Magia}.\index{Scartare dadi nella Prova di Magia} Questo per evitare di tirare tre volte 1.

La Prova di Magia si considera superata se il tiro è superiore a 10 + Livello dell'Incantesimo*2 + eventuali penalità. I Successi Critici o Fallimenti Magici si raffrontano a questo valore.\index{Fallimento Critico Magico}. In caso di Successo Critico Magico il costo dell'incantesimo diminuisce di 1 con un minimo di costo di 1.

Per ogni tiro critico o critico magico che si è subito nel round la Prova di Magia viene fatta con +4 di difficoltà aggiuntiva.\index{Danno critico se si lancia incantesimo}. Eventuali Fallimenti Critici o Successi Critici vengono presi in considerazione.

Quando viene richiesto di superare o fare una Prova di Magia è sufficiente superare la difficoltà data dall'incantesimo e non fare un Fallimento Critico Magico. Se viene richiesto di ottenere un Successo Critico e la Prova di Magia non lo ottiene allora qualsiasi risultato ottenuto sarà considerato un Fallimento Critico.

La Prova di Magia come tutte le Prove segue le \hyperlink{goldenrules}{Golden Rules}, pag.\pageref{goldenrules}.

\begin{narratore}[Declamare la Magia]
Concedete un +1d6 nella Prova di Magia quando il personaggio declama con perizia e trasporto il lancio dell'incantesimo. Se dice \emph{Lancio una palla di fuoco} non otterrà vantaggi ma se con trasporto declama \emph{Per la Fiamma della Genesi possa Nedraf distruggervi con le sue sacre fiamme. Bruciate indegni. Palla di Fuoco!} allora si!.

\medskip

Fate che la partecipazione e recitazione guidi sempre il personaggio, coinvolgendo anche gli altri giocatori.
\end{narratore}

\medskip

%\begin{center}
%	\includegraphics[width=1\linewidth]{immagini/spellbook.png}
%\end{center}



\begin{giocatore}[Osare la Prova di Magia]
La Prova di Magia è una parte importante e integrante del sistema magico, usatela a vostro vantaggio. Non è solo questione di fortuna! Con le giuste Abilità potete evolvere un personaggio capace di dominare la sorte!\\

Non rinunciate al divertimento per paura di sbagliare. Meglio fallire clamorosamente in una esplosione di colori che rinunciare e basta!
\end{giocatore}


\subsection{Fallimento Critico nella Prova di Magia}\index{Fallimento Critico nella Prova di Magia}\label{magiefallimentocriticonellaprovadimagia}\index{Incantesimi, Fallimento Prova di Magia}\hypertarget{magiefallimentocriticonellaprovadimagia}{}

Se la Prova di Magia ha avuto almeno un Fallimento Critico Magico, tirato tre 1 oppure tirato veramente basso, tira 3d6 e consulta la seguente tabella. Per ogni Fallimento Critico Magico oltre il primo che si è manifestato nel lancio dell'incantesimo e per ogni Tiro Critico subito, tira un 1d6 in meno, fino a tirare un solo 1d6.

\medskip

\textbf{Tabella: Effetti Fallimento Critico magico}\index[Tabelle]{Tabella Effetti Fallimento Critico Prova di Magia}

\medskip

\noindent\begin{tabularx}{\linewidth}{l|X}
	\toprule
\rowcolor{gray!20}\textbf{Dadi} & \textbf{Effetti}\\
\toprule
1 & Per 1 giorno non sei più in grado di canalizzare energie magiche. Non puoi lanciare incantesimi se non facendo un successo magico critico nella Prova di Magia\\
\rowcolor{gray!20}2 & Aumenti la condizione di Affaticato di 2 gradi, fino ad un massimo di Affaticato 5\\
3 & Manifesti una modifica corporea minore\\
\rowcolor{gray!20}4 & Vieni investito da una roboante colonna di Luce e Vuoto. In un raggio di 6 metri centrato su di te, chiunque deve fare un Tiro Salvezza su Riflessi DC 15 per dimezzare o subire 3d10 di danni da forza non resistibili\\
5 & Per 3 round sei sotto l'influenza dell'incantesimo Confusione\\
\rowcolor{gray!20}6 & Per 1 minuto non sei più in grado di concentrarti e parli in rima\\
7 & Vieni teletrasportato di 3d10 metri in una direzione casuale\\
\rowcolor{gray!20}8 & Diventi Invisibile e paralizzato per 6 round\\
9 & Solo tu vieni avvolto da una cortina di oscurità magica impenetrabile per 6 round\\
\rowcolor{gray!20}10 & Non riesci a parlare bene, sei balbuziente. Ogni lancio di incantesimi ti costringe a superare una Prova di Magia. Durata 3 round\\
11 & Manifesti l'incantesimo Unto sotto i tuoi piedi\\
\rowcolor{gray!20}12 & Il prossimo incantesimo che lanci ha effetti se possibile minimizzati\\
13 & Il battito del tuo cuore è come il battito di un tamburo, si può sentire entro 36 metri\\
\rowcolor{gray!20}14 & Tutte le creature nel raggio di 36 metri sanno esattamente dove sei e cosa tentavi di fare\\
15 & Tutte le creature in una sfera di 9 metri di raggio centrata su di te subiscono 1d10 danni da Vuoto\\
\rowcolor{gray!20}16 & Guadagni 2d6 Punti Magia\\
17 & Una incudine cade, 3d6 di danno Tiro Salvezza su Riflessi DC 15 per dimezzare, su una creatura a caso, escluso te, entro sei metri\\
\rowcolor{gray!20}18 & Le creature, te escluso, nel raggio di 6 metri da te subiscono 3d10 danni da forza non resistibili
\end{tabularx}

\subsection{Modificare la Prova di Magia}

\textbf{Prima di effettuare} la Prova di Magia l'incantatore può decidere investire ulteriori Punti Magia per migliorare la sua Prova di Magia.

Per ogni volta, fino ad un massimo di tre volte, che paga il costo dell'incantesimo, può \textbf{aggiungere} 1d6 in più nella Prova di magia. \index{Prova di Magia, più dadi}

\textbf{Dopo aver effettuato} la Prova di Magia, usando una Reazione, per ogni due volte che paga il costo dell'incantesimo (fino ad un massimo di sei volte), può \textbf{ignorare} un dado tirato nella Prova di magia. \index{Prova di Magia, ignorare i dadi}

Un incantatore può anche \textbf{volontariamente fallire la Prova di Magia}.


\subsection{I Punti Magia}\index{I Punti Magia}\label{magiepuntimagia}\index{Incantesimi, Punti Magia}\hypertarget{magiepuntimagia}{}

A seconda del punteggio in Competenza Magica l'incantatore ha a disposizione un certo ammontare di Punti Magia.

\textbf{Gli incantesimi hanno un costo in Punti Magia pari al loro livello}\index{Punti Magia e costo incantesimi}

Ogni qual volta si lanci un incantesimo si sottrae il costo ai Punti Magia a disposizione per il giorno.
In caso di Trucchetti questi non consumano Punti Magia ma è necessario avere almeno 1 Punto Magia residuo.

L'incantatore ha un \textbf{bonus} al punteggio di Punti Magia pari al suo modificatore di caratteristica per incantesimi.

I Punti Magia si recuperano tutti con 8 ore di riposo. \index{Incantesimi, Recupero Ponti Magia}


\medskip

\textbf{Tabella: Competenza Magica (CM) e Punti Magia (PM)}\index[Tabelle]{Tabella Competenza Magica (CM) e Punti Magia (PM)}

\medskip

\noindent\begin{tabularx}{\linewidth}{XX|XX|XX}
	\toprule
 \rowcolor{gray!20}\textbf{CM} & \textbf{PM}&\textbf{CM} & \textbf{PM}&\textbf{CM} & \textbf{PM}\\
	\toprule
	1&	4  &	8&	28&	15&	53\\
 \rowcolor{gray!20}2&	7  &	9&	32&	16&	56\\
	3&	11 &	10&	35&	17&	60\\
 \rowcolor{gray!20}4&	14 &	11&	39&	18&	63\\
	5&	18 &	12&	42&	19&	67\\
 \rowcolor{gray!20}6&	21 &	13&	46&	20&	70\\
	7&	25 & 	14&	49&	20+&	prec.+ 3
\end{tabularx}

PM = (CM × 3) + (CM ÷ 2 arrotondato per eccesso) + Modificatore Caratteristica


\begin{giocatore}[Scegliere gli Incantesimi]
	Ogni incantesimo è un tesoro prezioso che si deve trovare ed imparare.

	Ogni incantesimo è alla stregua di un oggetto magico, un vero tesoro da cercare e ottenere!

	Dovrai intraprendere perigliose avventure, pagare mercenari, cercare i tomi antichi e svelare i segreti più oscuri e dimenticati per poter imparare nuovi incantesimi.
\end{giocatore}


\subsection*{Incantesimi come Rituali}\index{Incantesimi come Rituali}\index{Rituali, Incantesimi}

Specialmente ai primi livelli può essere molto fastidioso non aver appreso un incantesimo pur avendolo a disposizione nel Tomo della Magia.

L'incantatore può lanciare un incantesimo che sia presente sul suo Tomo di Magia e che sia entro il 3 livello ed entro il massimo livello di incantesimo lanciabile, allungandone il tempo di lancio ad 1 ora per costo in Punto Magia. In caso di incantesimo così lanciato non si usano Punti Magia, ma è necessario superare una Prova di Magia al termine della formulazione.


\subsection{Successo Critico Auto Magico}\index{Successo Critico Auto Magico}\index{Nova}\label{magienova}

L'incantatore può decidere di spendere, in aggiunta ai \textbf{dei Punti Magia} dell'incantesimo, un uguale ammontare per avere in automatico un \textbf{Successo Critico Magico}.
Ogni volta che voglio applicare un Successo Critico Magico aggiuntivo oltre il primo il costo in Punti Magia aumenta di 1. La dichiarazione di volere usare il Successo Critico Auto Magico è da dichiarare prima di effettuare, e superare, la Prova di Magia.


%\medskip

%\begin{center}
%	\includegraphics[width=0.7\linewidth]{immagini/Arthur-Pyle_The_Enchanter_Merlin.png}
%
%	\emph{Merlin. Howard Pyle, The Story of King Arthur and His Knights (1903)}
%\end{center}

Il tempo di lancio di un incantesimo potenziato in questa maniera aumenta di 1 Azione.

Es. \hyperlink{Velocità}{Velocità}, voglio che faccia 2 critici magici, pago 3 Punti Magia per lanciarla, più 3 per il primo Successo Critico Magico più 4 per il secondo Successo Critico Magico, ed eventualmente 5 per un terzo Successo Critico Magico. Si pagano sempre tutti i Punti Magia usati indipendentemente dal risultato della Prova di Magia.

Non si possono spendere più di metà dei Punti Magia attuali per potenziare un incantesimo, non si possono fare più Autocritici del modificatore di caratteristica per incantesimo.

%\subsection{L'Essenza della Magia}
%Diversi incantesimi hanno la possibilità di essere potenziati direttamente sfruttando il grezzo potere magico.
%Quando dopo il livello e la Lista di Magia è presente un \textbf{*} in fondo allora valgono queste regole:
%\begin{itemize}[leftmargin=*] \setlength{\itemsep}{0pt}
%\item ogni Punto Magia speso in più il danno aumenta di 1d8
%\item ogni 3 Punti Magia spesi in più la DC del Tiro Salvezza aumenta di 1
%\item il tempo di lancio aumenta di 1 Azione
%\item non è possibile usare più Punti Magia della metà del punteggio di Competenza Magica
%\item non è possibile lanciare nuovamente quell'incantesimo per 10 round
%\item non è possibile combinare l'\emph{Essenza della magia} con il \emph{Successo Critico %Automagico}
%\end{itemize}


\subsection{Opzionale - Il vero costo della Magia}\index{Opzionale - Il vero costo della Magia}

Il sistema magico può diventare sbilanciato abusando sempre degli stessi incantesimi. Per limitare questo sono proposti due approcci, da stabilire nella Sessione Zero:

\begin{itemize}[leftmargin=*] \setlength{\itemsep}{0pt}
\item Il costo in Punti Magia dell'incantesimo aumenta del costo stesso ogni volta che viene rilanciato (\emph{metodo suggerito})
\item Un incantatore può lanciare lo stesso incantesimo al massimo 1 volta al giorno
\end{itemize}

\subsection{Il Tomo della Magia}\index{Tomo della Magia}\index{Il Tomo della Magia}\label{magietomodellamagia}\index{Incantesimi, Tomo della Magia}

Se i Patroni sono la sorgente della magia è solo l'applicazione di antichi riti e formule che permette di manifestare questa energia grezza in una forma ed espressione che chiamiamo incantesimo.

Ogni usufruitore di magia ha uno o più \textbf{Tomo} degli incantesimi, non pensate solo a un grosso Tomo antico rilegato in pelle, le diverse culture hanno sviluppato nel tempo la capacità di iscrivere le rune degli incantesimi in carte, bastoni, lastre di pietra, tatuaggi... fate la vostra scelta quando create il personaggio.
Questa scelta non vi impedirà di copiare incantesimi da \textbf{Tomi} fatti diversamente, per voi sarà sempre facile (prova di Arcana DC 12) capire se si è di fronte ad un Tomo di qualche tipo.

Un nuovo personaggio con Competenza Magica 1, avrà un Tomo di Magia con un certo elenco di incantesimi. In questo Tomo sono presenti un numero di Trucchetti pari al modificatore di caratteristica per incantesimi +2 (con un minimo di 4 Trucchetti) ed un numero di incantesimi di primo livello sempre pari allo stesso modificatore, con un minimo di 4.\index{Incantesimi al primo livello}\label{tomocm1}\hypertarget{tomocm1}{}

Ogni incantesimo occupa un numero di pagine nel Tomo pari al proprio livello, con un minimo di una, \textbf{copiare una pagina di incantesimo} porta via 1 ora di lavoro e 10 mo di preziosi inchiostri.\index{Copiare Incantesimi sul Tomo}

Un Tomo (libro) di incantesimi costa 5 mo per pagina.

Un incantatore può copiare sul suo Tomo incantesimi il cui livello è di uno in più rispetto al suo massimo lanciabile (vedi \hyperlink{scuoleelivelli}{Massimo livello di incantesimi lanciabile}).

Se l'incantesimo è di più di due livelli più alto l'incantatore deve fare una Prova di Magia ed ottenere un Successo Critico Magico. Se il personaggio è un Devoto e l'incantesimo appartiene ad una Lista di Magia preferita del Patrono allora la Prova di Magia si esegue solo se l'incantesimo è di tre o più livelli superiori al massimo lanciabile.\index{Incantesimo non conosciuto}

Se non ottiene almeno un Successo Critico Magico non potrà tentare di copiare quell'incantesimo fino al prossimo punto di Competenza Magica acquisito. Se ottiene un Fallimento Critico Magico accadranno brutte cose al Tomo e 1d4 incantesimi casuali verranno cancellati dal Tomo stesso.

La sorgente di nuovi incantesimi può essere un altro Tomo o pergamena.. insomma qualsiasi cosa che il precedente incantatore usasse per custodire gli incantesimi. Un oggetto magico (bastone magico, anello, verga..bacchetta..) non è idoneo quale fonte da cui copiare l'incantesimo che contiene, si deve copiare dall'equivalente Tomo o pergamena di un altro incantatore. Un incantesimo quando copiato sul nuovo Tomo svanisce dalla sorgente originale.

\begin{narratore}[Magie vero tesoro]
Gli incantesimi diventano oggetti e premi magici a tutti gli effetti. Sfruttate la sete di conoscenza e potere dei personaggi per costruire avventure interessanti che possano ruotare attorno tomi antichi e leggendari incantesimi perduti.
\end{narratore}

\subsection{Studiare gli incantesimi}\index{Studiare gli incantesimi}\label{magiestudiareincantesimi}\index{Incantesimi, Studiare}

Il personaggio che vuole lanciare magie deve ogni giorno ripassare le antiche formule sul suo Tomo. Questa operazione è piuttosto rapida, impiegando solo 3 minuti per Competenza Magica.

Se l'incantatore non ha ripassato gli incantesimi dopo aver riposato almeno 6 ore deve superare una Prova di Magia per ogni incantesimo formulato finché non avrà ripassato.

L'incantatore può studiare gli incantesimi anche da più Tomi..

\subsection{Tiro per Colpire con le Magie}\index{Tiro per Colpire con Incantesimi}\label{magietiropercolpireconlemagie}\index{Incantesimi, Tiro per Colpire}\hypertarget{magietiropercolpireconlemagie}{}

Diversi incantesimi devono essere scagliati e colpire un avversario per funzionare.

Quando l'incantesimo ti dice di fare un \emph{Tiro per colpire con incantesimo} oppure \emph{attacco a distanza con incantesimo} devi effettuare un Tiro per Colpire contro la Difesa dell'avversario.

Questo Tiro per Colpire, che sia in mischia od a distanza, è effettuato con 3d6 + \textbf{Competenza Magica} + \textbf{Modificatore di caratteristica per incantesimi} + \textbf{Abilità} e \textbf{modificatori vari}.

E' anche possibile che sia richiesto un \textbf{Tiro per Colpire con incantesimo a tocco} ovvero l'attacco viene effettuato con un bonus di +1d6, come per Attacco a Tocco.\index{Attacco a Tocco con incantesimo}

Tiro per Colpire con incantesimo o con arma cumulano le penalità dell'attacco multiplo.\index{Penalita' attacco multiplo con incantesimo}

\medskip

Quando la magia è ad area non è necessario effettuare un Tiro per Colpire se non per mirare a difficili e minute specificate aree, ovvero si mira in una area ben circoscritta e si vuole evitare di colpire qualcuno con un incantesimo ad area.

\subsection{Tiro Salvezza - Resistere all'incantesimo}\index{Tiro Salvezza - Resistere all'Incantesimo}\index{Tiro Salvezza Incantesimi}\label{magietirosalvezza}\hypertarget{magietirosalvezza}{}

Il \textbf{Tiro Salvezza} imposto dal personaggio ha difficoltà (DC) pari a \textbf{10} + \textbf{Competenza Magica} + \textbf{modificatore caratteristica per incantesimo} + \textbf{numero di volte che si è presa l'Abilità Adepto della Magia} +\textbf{1 per ogni Successo Critico Magico} nella Prova di Magia.

Questa DC è usata per misurare la \emph{forza ed efficacia} dell'incantesimo quando confrontato con altri effetti.

Nella descrizione dell'incantesimo è scritto se è necessario un Tiro Salvezza e quale eseguire.\index{DC di una magia}

Se è il personaggio a dover resistere ad una magia il Narratore non ti dirà di fare un Tiro Salvezza a difficoltà 18, è lui che confronta il tuo tiro con la difficoltà, potrà dirti che la prova è complessa, difficile o facile...

\begin{itemize}[leftmargin=*] \setlength{\itemsep}{0pt}

\item
Se nel Tiro Salvezza tiri 3 volte 6 sei riuscito a passarlo, indipendentemente dal totale, ed ottieni un \textbf{Successo Critico Salvezza}.

\item
Se il Tiro Salvezza riesce per ogni margine di riuscita di 8 ottieni un \textbf{Successo Critico Salvezza}\index{Successo Critico Incantesimi}.

\item
Se nel Tiro Salvezza tiri 3 volte 1 hai fallito il tiro, indipendentemente dal totale, ed ottieni un \textbf{Fallimento Critico Salvezza}.\index{Tre 1 nei Tiri Salvezza magici}

\item
Se il Tiro Salvezza fallisce ed il margine di fallimento è almeno 8, per ogni margine di fallimento di 8 ottieni un \textbf{Fallimento Critico Salvezza}\index{Fallimento Critico Incantesimi}.

\end{itemize}

\begin{giocatore}[Tups lancia Dardo Tracciante!]
Tups che ha Intelligenza 4, Competenza Magica 6 e ha preso 2 volta Adepto della Magia, lancia l'incantesimo \hyperlink{Dardo Tracciante}{Dardo Tracciante}. La difficoltà (DC) del Tiro Salvezza su Riflessi sarà pari a 10 + 6 (CM) + 4 (modificatore caratteristica per incantesimo, Intelligenza) + 2 (ha preso 2 volte Adepto della Magia) ovvero 10+6+4+2 = 22 per dimezzare i danni. Se avesse fatto una Prova di Magia e questa avesse avuto un Successo Critico magico la DC sarebbe diventata 23.
\end{giocatore}

E' anche possibile che nella descrizione dell'incantesimo sia riportato cosa succede in caso di Successo o Fallimento Critico del Tiro Salvezza.

Per i \textbf{mostri} o comunque per un lancio di incantesimi dato da abilità magiche innate, se non specificato la \textbf{DC del Tiro Salvezza è pari alla 12 + 2 x livello dell'incantesimo + Intelligenza o modificatore di incantesimi indicato}.\index{DC Tiro Salvezza Incantesimo mostri}\index{Difficoltà incantesimi dei mostri}\label{tirosalvezzainccmostro}

\subsection{Contrastare gli Incantesimi}\index{Contrastare gli incantesimi}\label{contrastareincantesimi}\hypertarget{contrastareincantesimi}{}

Diversi incantesimi interagiscono con altri effetti annullandoli o modificandoli. Quando è scritto che un incantesimo \textbf{contrasta} o è \textbf{contrastato} un altro è necessario verificare la DC degli incantesimi o effetti per accertarsi quale effetto domini sull'altro.

Ad esempio l'incantesimo \hyperlink{Lentezza}{Lentezza} contrasta \hyperlink{Velocità}{Velocità}, \hyperlink{Rimuovi Maledizione}{Rimuovi Maledizione} sulle maledizioni, \hyperlink{Rimuovi Veleno}{Rimuovi Veleno} sui veleni...

Il \textbf{proprio valore di contrasto} si computa con una prova di 3d6 + CM + modificatore di caratteristica per incantesimi + volte che si è preso Adepto della Magia.\index{Valore di Contrasto} + 1 per Successo Critico Magico ottenuto nella Prova di Magia.\index{Annullare gli incantesimi}

Il valore di contrasto si \textbf{confronta} con la DC dell'effetto magico da contrastare.

Quando per un incantesimo non è fornita la DC da contrastare allora considerate la stessa pari a 12+Livello incantesimo*2.

\subsection{Concentrazione}\index{Colpito mentre concentrato}\index{Concentrazione}\label{magieconcentrazione}\hypertarget{magieconcentrazione}{}

Perdi la concentrazione su di un incantesimo se lanci un altro incantesimo che richieda concentrazione. Non ti puoi concentrare su due incantesimi alla volta. Interrompere la concentrazione costa una Azione Immediata.

Se vieni colpito\index{Colpito mentre tieni la concentrazione} mentre sei concentrato su un incantesimo devi effettuare una Prova di Magia con difficoltà pari all'incantesimo su cui ti stai concentrando ed ottenere almeno 1 Successo Critico Magico, +1 per ogni critico subito oppure perdere la concentrazione.

%Mentre sei concentrato puoi lanciare solo Trucchetti senza difficoltà altrimenti se riesci in una Prova di Magia con un Successo Critico Magico puoi anche lanciare incantesimi non Trucchetti. Mantenere la Concentrazione costa 1 Azione a round.

Mentre sei concentrato puoi lanciare solo Trucchetti senza difficoltà altrimenti se riesci in una Prova di Magia puoi anche lanciare incantesimi non Trucchetti.

Ogni 6 punti di Competenza Magica si può mantenere la concentrazione su un incantesimo aggiuntivo. Se vieni colpito devi fare una prova per ogni incantesimo su cui stai mantenendo la concentrazione. In caso di fallimento anche su un solo incantesimo tutte le concentrazioni mantenute vengono interrotte.

Per ogni incantesimo su cui si mantiene la concentrazione si usa 1 Azione a round.

\subsection{Conservare la magia}\index{Conservare la magia}\label{magieconservare}

L'incantatore può lanciare l'incantesimo e trattenerlo nel suo pugno, senza manifestarlo. Per poter trattenere un incantesimo l'incantatore formula la magia e spende subito una Azione per rimanere Concentrato e paga 1 Punto Magia addizionale.
L'incantesimo può essere trattenuto fino ad 1 round per punteggio di modificatore di caratteristica da incantesimi +1 round per volte che ha preso Adepto della Magia.

Per trattenere l'incantesimo l'incantatore deve rimanere Concentrato e pagare 1 Punto Magia per round.
Per lanciare l'incantesimo conservato è sufficiente tirare l'iniziativa ed usare 1 Azione. Non è possibile lanciare ulteriori incantesimi che non siano Trucchetti finché si conserva un incantesimo.

\subsection{Regole per le creature evocate}\index{Regole per le creature evocate}\index{Evocazione creature}

Queste regole valgono per tutte le creature evocate magicamente.

La creatura evocata agisce nel tuo round, non deve tirare l'iniziativa, bensì usa la tua ed è amichevole verso di te e i tuoi compagni. Una creatura evocata può già agire nel round in cui è stata evocata.

Una creature evocata ha 2 azioni a round, se non vengono specificati ordini al momento dell'evocazione la creatura si difende e contrattacca chi l'ha attaccata.

La creatura evocata comprende i comandi che gli vengono dati al meglio delle sue capacità mentali. Per cambiare l'ordine si deve usare una Azione.

\subsection{Tentare più Magie nello stesso round}\index{Tentare più Magie nello stesso round}\index{Magie multiple nello stesso round}\hypertarget{piumagieround}{}\label{piumagieround}

E' possibile lanciare più magie nel round purché il tempo totale di lancio non superi le Azioni a disposizione e si superi una Prova di Magia al lancio del secondo incantesimo. più incantesimi che richiedano il Tiro per Colpire o attaccare con un arma e lanciare un incantesimo cumulano le penalità del multiattacco.\index{Incantesimi e multiattaco}

\subsection{Alterare le Magie}\index{Alterare le Magie}\label{magiealteraremagie}

L'incantatore può modificare gli incantesimi in diversi modi. Queste possibilità aggiungono versatilità all'incantatore ed è opportuno che il personaggio le abbia sempre presenti nelle situazioni più critiche.

%- \textbf{Magie Punitive}\index{Magie Punitive}\label{magiepunitive}\hypertarget{magiepunitive}{}: pagando due volte il costo dell'incantesimo in Punti Magia puoi tirare un dado in più nella Prova di Magia ed ignorare un dado tirato. La capacità è usabile fino a 3 dadi in più per incantesimo. Da parte del compagno è una Azione di Reazione da dichiararsi prima della Prova di Magia.

- \textbf{Magia eterea}\index{Magia eterea}: aumentando di 3 i Punti Magia spesi nell'incantesimo le proprie magie hanno pieno effetto su creature eteree o incorporee. Azione Immediata da dichiararsi prima del lancio dell'incantesimo.

- \textbf{Sacrificio Magico}\index{Sacrificio magico}: l'incantatore riducendo i suoi Punti Ferita attuali e Massimi di 4 acquisisce 1 Punto Magia da usare contestualmente al lancio di una magia. Non puoi sacrificare più di metà dei Punti Ferita attuali alla volta. Azione Immediata.

- \textbf{Magia pietosa}\index{Magia pietosa}: aumentando di 3 i Punti Magia spesi le magie infliggono danni temporanei. Le magie che infliggono danni di un tipo particolare (come da fuoco) infliggono danni temporanei dello stesso tipo. 1 Azione.

- \textbf{Magia mirata}\index{Magia mirata}: ogni 2 Punti Magia che paghi in aggiunta al costo dell'incantesimo puoi escludere una persona dall'area di effetto dell'incantesimo. Non puoi escludere più persone di quante volte non abbia preso l'Abilità Adepto della Magia. 1 Azione. %1 punto magia per livello incantesimo per creatura esclusa

- \textbf{Magia lontana}\index{Magia lontana}: aumentando di 1 i Punti Magia usati aumenti fino a 9 metri la distanza di lancio dell'incantesimo. 1 Azione.

- \textbf{Aumentare il tempo}\index{Aumentare il tempo} di lancio da 2 Azioni a 3 Azioni diminuisce di 1 i Punti Magia spesi per il lancio di incantesimo, con un minimo di costo di 1 Punto Magia.

%- \textbf{Circolo del Potere}\index{Circolo del Potere}: più incantatori che siano tutti Devoti o Seguaci dello stesso Patrono possono collaborare affinché uno di loro riesca meglio nel lancio di un incantesimo.
%Ogni incantatore sacrifica metà dei Punti Magia dell'incantesimo lanciato dal compagno e supera una Prova di Magia. Ogni due compagni che superano la Prova di Magia si genera un Successo critico magico, fino ad un massimo di 7 successi critici magici. Il tempo di lancio di un incantesimo tramite Circolo di Potere diviene almeno 1 Turno. Requisito Competenza Magica 5.

\medskip

Le possibilità concesse da Alterare le Magie sono cumulabili tra loro.

\medskip

\textbf{Modifiche lievi} \index{Modifiche lievi agli incantesimi} alla manifestazione dell'incantesimo possono essere concordati con il Narratore per un costo di Punti Magia aggiuntivi o con una Prova di Magia riuscita.

\subsection{Tentare Incantesimi con impedimenti}\index{Tentare Incantesimi con impedimenti} \index{Impedimenti}\label{magieconimpedimenti}\hypertarget{magieconimpedimenti}{}

Il lancio di un incantesimo è vincolato a gesti e parole particolari e unici. Quando il personaggio si trova in una situazione in cui non può gesticolare o parlare allora può tentare di lanciare l'incantesimo comunque anche se diventa molto più difficile.

I Punti Magia richiesti per il lancio di incantesimi se non può gesticolare vengono triplicati e se non può parlare vengono ulteriormente triplicati, è necessario in ogni caso superare una Prova di Magia.

\subsection{Definizioni obiettivi degli incantesimi}\index{Obiettivi degli incantesimi}\label{magiedefinizioniobiettivi}

Negli incantesimi sotto elencati troverete spesso i riferimenti alle tipologie di soggetti ed obiettivi influenzabili nonché a diverse tipologie di energia ed elementi.

- Le \textbf{Creature Naturali} sono Insetti, Rettili, Bestie, Umanoidi, Piante, Creature acquatiche, Mostruosità, Melme.

- Le \textbf{Creature Magiche} sono: Immondi (Diavoli e Demoni), Fatati, Spiriti, Non morti, Giganti, Celestiali, Elementali, Costrutti, Aberrazioni (tutto ciò che è alieno o innaturale) ed i Draghi.

Se una Creatura Naturale ha poteri magici allora si considera anche come Creatura Magica. Una descrizione più completa di queste categorie la trovate nel Capitolo del Mostruario.

- \textbf{Energia} comprende: Forza, Fuoco, Luce, Suono, Elettricità, Energia Positiva, Energia Negativa, Freddo, Vuoto.\label{elencoenergia}\hypertarget{elencoenergia}{}

\subsection{Danni da Energia, Luce e Vuoto}

Il danno causato da \textbf{Luce}\index{Luce} è per metà da fuoco e per metà da energia positiva, ovvero una resistenza al fuoco od all'energia positiva si applica solo su metà del danno causato dall'attacco.

Il danno causato da \textbf{Vuoto}\index{Vuoto} è per metà da freddo e per metà da energia negativa, eventuali protezioni si applicano alle rispettive metà del danno.

Essere Immuni o avere una Resistenza alla Luce o Vuoto non rende immune o resistenti a sua volta ai danni da Fuoco/Energia Positiva o Freddo/Energia Negativa.

La sola \textbf{energia negativa} danneggia\index{Energia Negativa} i viventi e cura i non morti, la sola \textbf{energia positiva}\index{Energia Positiva} danneggia i non morti ma non cura i viventi (a discrezione del Narratore l'esposizione per un round potrebbe equivalere ad un incantesimo di Ristorare Inferiore), vedi anche descrizioni dei Piani. Un obiettivo prende danno pieno da Luce o da Vuoto se non ha resistenze inerenti.

Un caso particolare è l'\textbf{energia positiva Curativa}\index{Energia positiva curativa} che cura i viventi e danneggia i non morti. Questa energia è quella dell'Imposizione delle mani, Incanalare energia e degli incantesimi di Cura.\index{Energia positiva su non morti}

\subsection{Abilità di Lista}\label{abilitadilista}\index{Abilità di Lista}\hypertarget{abilitadilista}{}

%\begin{enfasi}{
%Volli, e volli sempre, e fortissimamente volli (Vittorio Alfieri, 06/09/1783, Lettera a Ranieri dè Calzabigi)
%}\end{enfasi}\end{changemargin}

Lo studio della magia e l'approfondita conoscenza delle Liste di Magia porta l'incantatore ad imparare aspetti non sempre conosciuti della stessa.

Le capacità indicate sono attivabili se con le Azioni precedenti l'incantatore ha lanciato un incantesimo, non Trucchetto, dalla Lista di Magia indicata. L'incantatore può scegliere il potere che vuole purché abbia preso Adepto della Magia almeno il numero indicato di volte.

\emph{\textbf{Lista Abiurazione}}

\textbf{3: Scudo Minore.} Usando una Azione Immediata sei in grado di canalizzare le energie magiche che ti pervadono manifestando una protezione. Fino all'inizio del tuo prossimo round hai un +1 alla Difesa.

\textbf{4: Protezione Maggiore.} Usando una Azione Immediata sei in grado di canalizzare le energie magiche che ti pervadono manifestando una protezione. Scegli fino a 2 creature nel raggio di 6 metri, fino all'inizio del tuo prossimo round prendono +1 Difesa oppure +1 ai Tiri Salvezza, da scegliere al momento dell'utilizzo del potere.

\emph{\textbf{Lista dell'Acqua}}

\textbf{3: Acque profonde.} Usando una Azione Immediata acquisisci resistenza 5 al freddo ed al fuoco fino all'inizio del tuo prossimo round.

\textbf{5: Acque limpide.} Usando una Azione Immediata puoi toccare una creatura e lo aiuti a liberarsi da veleni e tossine. Viene concesso un nuovo Tiro Salvezza (se possibile) per perdere la condizione di avvelenato.

\emph{\textbf{Lista dell'Aria}}

\textbf{3: Fra le nuvole.} Usando una Reazione sei in grado di lanciare su te stesso l'incantesimo Caduta Piuma senza usare Punti Magia.

\textbf{4: Scossa.} La tua mano manifesta un crepitio di elettricità, il prossimo incantesimo che lanci entro la fine del prossimo round che abbia un Tiro per Colpire causa 2d8 danni in più da elettricità. Costa una Azione Immediata.

\textbf{Lista Ammaliamento}

\textbf{3: Distrazione.} Quando una creatura che puoi osservare entro 9 metri da te effettua un attacco con armi o magie, puoi usare una Reazione per distrarlo fino a fine round. La creatura ha -2 al Tiro per Colpire.

\textbf{4: Distrazione maggiore.} Quando una creatura che puoi osservare entro 9 metri da te effettua un attacco con armi o magie, puoi usare una Reazione per distrarlo fino a fine round. Tira 1d6, se il risultato è 3-4-5 la creatura ha -2 al Tiro per Colpire, se il risultato è 6 il bersaglio dell'attacco della creatura diventa casuale tra i bersagli colpibili.

\emph{\textbf{Lista Animali e Piante}}

\textbf{3: Corteccia.} Usando una Azione Immediata rendi la tua pelle più dura e resistente. Hai una riduzione del danno pari a 2 fino all'inizio del tuo prossimo round.

\textbf{4: Artigli.} Usando una Azione Immediata rendi ancora più affilati i tuoi attacchi naturali fino alla fine del prossimo round. Ogni attacco naturale andato a segno causa Sanguinamento 1, cumulabile fino a Sanguinamento 5.

\emph{\textbf{Lista Cura}}

\textbf{3: Mano calda.} Usando una Azione Immediata fai che l'incantesimo di cura che hai lanciato ti curi un numero di Punti Ferita aggiuntivi pari al livello dell'incantesimo stesso.

\textbf{4: Spirito benevolo.} Usando una Azione canalizzi l'energia residua di un tuo incantesimo. Usando una Azione Immediata puoi toccare una creatura e curarla di 1d4 Punti Ferita.

\emph{\textbf{Lista Divinazione}}

\textbf{3: Premonizione.} Usando una Azione Immediata ha una fugace previsione degli accadimenti futuri. Fino all'inizio del tuo prossimo round hai un +1 ai Tiri Salvezza su Riflessi.

\textbf{4: Punto Cieco.} Usando una Azione Immediata puoi toccare una creatura, fino all'inizio del tuo prossimo ha un +2 al Tiro per Colpire.

\emph{\textbf{Lista Evocazione}}

\textbf{3: Mano cava.} Con una Azione Immediata, puoi fare scomparire e riapparire quando vuoi un oggetto di volume L. Non puoi trattenere più di tre oggetti in questa maniera.

\textbf{4: Passo cauto.} Con una Azione Immediata fai che la successiva Azione di Movimento non causi attacchi d'opportunità.

\emph{\textbf{Lista del Fuoco}}

\textbf{3: Gola arrossata.} Con una Azione Immediata sputi un getto di fuoco in un quadretto entro 1 metro di distanza. Il terreno si considera difficile ed attraversarlo o sostare causa 1d6 Punti Ferita da Fuoco. Dura fino all'inizio del tuo round successivo.

\textbf{4: Napalm.} Con una Azione Immediata tocchi un arma. L'arma viene avvolta da fiamme, fino alla fine del tuo round successivo causa 1d8 di danno da Fuoco in più.

\emph{\textbf{Lista Illusione}}

\textbf{2: Prestidigitatore.} Puoi usare l'incantesimo Prestidigitazione con una Reazione.

\textbf{5: Abbondanza} Con una Reazione puoi creare un oggetto inorganico di volume 1 o inferiore del valore di 1 mo o meno. L'oggetto permane per 1 Turno o finché non viene usata nuovamente questa capacità.

\emph{\textbf{Lista Invocazione}}

\textbf{3: Speranza.} Con una Azione Immediata puoi illuminare la tua mano fino all'inizio del tuo round successivo. La mano illumina il solo tuo quadretto ed è luce fioca nel quadretto dopo.

\textbf{4: Augurio.} Con una Azione Immediata tocchi una creatura cedendogli un tuo Punto Fato.

\emph{\textbf{Lista Necromanzia}}

\textbf{3: Sangue Nero.} Usando una Azione Immediata fino all'inizio del tuo round successivo ignori la condizione di affaticato.

\textbf{4: Sangue Morto.} Usando una Azione Immediata puoi toccare una creatura. Questa prende +2 ai Tiri Salvezza su Tempra e -1 ai Tiri Salvezza su Riflessi fino all'inizio del tuo round successivo.

\emph{\textbf{Lista della Terra}}

\textbf{3: Colla.} Sei in grado di lanciare l'incantesimo Riparare come Azione Immediata senza spendere Punti Magia.

\textbf{4: Titano.} Usando una Azione, ogni volta che lanci un incantesimo della Lista della Terra, purché tu sia in contatto con la terra solida recuperi un numero di Punti Ferita pari al livello dell'incantesimo lanciato.

\emph{\textbf{Lista Trasmutazione}}

\textbf{3: Condivisione.} Usando una Azione Immediata tocchi una creatura, la creatura guadagna una Reazione in più da usarsi entro la fine del tuo round prossimo.

\textbf{4: Salto.} Con una Azione Immediata ti sposti nello spazio. Tira un 1d6, appari in un quadretto non occupato a quella distanza.

\textbf{5: Apparenze.} Con una Azione Immediata alteri la tua presenza nello spazio. Tira un 1d6, se fai 6 fino all'inizio del tuo prossimo round diventi invisibile.

\emph{\textbf{Lista Universale}}

\textbf{3: Vista.} Acquisisci l'Abilità Occhi della Magia, come se l'avessi scelta 2 volte.

\textbf{4: Precisione.} Esegui la Prova di Magia con un d6 in più e puoi ignorare un dado nella Prova di Magia.

\textbf{5: Conoscere.} Puoi lanciare l'incantesimo Identificare con una Azione Immediata, senza usare Punti Magia.

\textbf{Queste capacità non sono concesse a chi prende l'Abilità \hyperlink{Un solo credo}{Un solo credo}} (pag. \pageref{Un solo credo}).

%\begin{center}

%\includegraphics[width=0.6\linewidth]{immagini/Hex32.png}

%\emph{The Witchcraft Art of Jacques de Gheyn II}

%\end{center}

\subsection*{Circa OBSS e la dimensione degli incantesimi}

Partiamo da un assunto: in OBSS una creatura media occupa uno spazio di $(1*1)\si{m^{2}}$ mentre nella 5e la stessa creatura occupa $(1.5*1.5)\si{m{^2}}$ (2.25\si{m{^2}}).

Una \hyperlink{Palla di Fuoco}{Palla di Fuoco} agisce in un raggio di 6 metri, ovvero \emph{occupa} ($6*6*\pi$)\si{m^{2}}, in questo spazio ci stanno \emph{circa 110} creature in OBSS e 50 creature, sempre medie, della 5e!

A fare due conti la Palla di Fuoco dovrebbe avere un raggio di 4 metri per influenzare lo stesso numero di creature!

Se questo fattore influisce troppo sul vostro gioco riducete della metà ($r-r/2$) i raggi delle esplosioni e le lunghezze dei coni.

\end{multicols}

\vfill

%\begin{center}
%	\includegraphics[width=0.40\linewidth]{immagini/The_Chariot_of_Hermes.png}
%
%	\emph{Tarocchi - il Carro}
%\end{center}



%\vfill

%\begin{center}
%\includegraphics[width=0.8\linewidth]{immagini/Binsfeld_witches_grayscale.png}
%\medskip
%\emph{Artwork depicting various witch practices, 1592, Peter Binsfeld}
%\end{center}

\begin{center}

\includegraphics[width=0.55\linewidth]{immagini/David_Ryckaert_III_La_Ronde_des_farfadets_grayscale.png}

\medskip

\emph{Peinture à l'huile sur toile de 1651}

\end{center}

%\vfill
%
%\begin{center}
%\includegraphics[width=0.47\linewidth]{immagini/Voynich_Manuscript.png}
%
%\smallskip
%
%\emph{Una pagina dal manoscritto di Voynich, tuttora indecifrato.}
%\end{center}

\pagebreak

