\section{Lista Armi per Tipologia Omogenea}\index{Lista Armi}\index{Tipologia Omogenea}\hypertarget{lista.armi}{}\label{lista.armi}

\begin{changemargin}{0.3cm}{0.3cm}\begin{enfasi}{La forza non risiede in una Spada, ma nelle braccia di un valoroso. (The Legend of Zelda: Twilight Princess)} \end{enfasi}\end{changemargin}\medskip

\begin{multicols}{2}

Ogni qual volta si assegna un punto a Competenza Armi si può decidere se continuare a perfezionarsi in una Lista di Armi già nota o apprendere una nuova, se non si dichiara l'uso questo è assegnato alla Lista delle Armi Semplici.

Nella scheda segnatevi a quale Lista d'Armi assegnate il punto di Competenza Armi.

Per riassegnare un punto di Competenza Armi in un altra lista sono necessari almeno 4 ore di allenamento per 4 mesi.

Usare un arma senza l'adeguata competenza impone un -1d6 al Tiro per Colpire.

\textbf{Tutte le Liste d'Armi} concedono, se non scritto diversamente, questi vantaggi cumulativi quando il punteggio nella Lista d'Armi raggiunge il valore indicato:

\begin{itemize}[leftmargin=*] \setlength{\itemsep}{0pt}\index{Bonus comuni Lista d'armi}

\item 6 punti: se affronti qualcuno che usa un arma in questa lista sei immediatamente in grado di capire la sua capacità di Competenza Armi (o bonus al Tiro per Colpire in caso di mostri).

\item 10 punti: se colpisci con almeno due attacchi il medesimo avversario nel round puoi spostarti da lui di un metro senza usare Azioni oppure il secondo attacco causa 1 danno critico se non ha generati

\item 14 punti: se colpisci con almeno un attacco l'avversario puoi spostarti da lui di un metro senza usare Azioni.

\item 18 punti: quando effettui un Tiro per Colpire consideri anche i 5 per il conteggio dei Critici (ma non ritiri il dado).

\item 20 punti: quando effettui un Tiro per Colpire consideri anche i 5 per il conteggio dei Critici e ritiri il dado.

\end{itemize}

I punti assegnati in una Lista d'Arma non si sommano al Tiro per Colpire! Bisogna verificare il punteggio nella Lista d'Arma con gli eventuali bonus che la stessa lista elenca.

I bonus indicati nelle Liste d'Armi si applicano solo quando combatti con le armi indicate dalla stessa lista.

I vantaggi indicati sono cumulativi se non indicato diversamente.

\subsection{Archi} \index{Archi} Arco Lungo, Arco Corto, Arco Lungo Composito, Arco Corto Composito\label{listaarmiarchi}



\begin{itemize}[leftmargin=*] \setlength{\itemsep}{0pt}

\item 4 punti: aggiungi il valore di Forza al danno, anche se l'arco non è composito. Su un arco corto puoi aggiungere fino a +1 di danno, su un arco lungo fino a +2 di danno.
\item 5 punti: riduci di 6 la penalità per tirare oltre la gittata standard.
\item 7 punti: la tua maestria nell'utilizzo dell'arco in combattimento è tale che non subisci nessuna penalità nel lanciare frecce a nemici con copertura leggera.
\item 9 punti: ti e' possibile con il primo Tiro per Colpire che esegui nel round scagliare due frecce. Il Tiro per Colpire parte da una penalità di -5.
\item 11 punti: riduci di 6 la penalità per tirare oltre la gittata standard.
\item 16 punti: la prima freccia che colpisce nel round aggiunge un danno critico.

\end{itemize}

\begin{center}
\includegraphics[width=0.7\linewidth]{immagini/arma-arco.png}
\end{center}

\subsection{Armature}\index{Lista Armature} \label{listaarmature}

Questa Lista conferisce solo i bonus cumulativi qui elencati quando si indossa una Armatura.

\begin{itemize}[leftmargin=*] \setlength{\itemsep}{0pt}

\item 1 punto: dimezzi il tempo necessario per indossare e togliere un'armatura
\item 2 punti: la Difesa concessa dall'armatura aumenta di 1 punto, dormire in armature medie non causa affaticamento
\item 3 punti: la Penalità alla Competenza diminuisce di 1 punto, dormire in armature pesanti non causa affaticamento
\item 4 punti: la penalità al Movimento diminuisce di 1 metro, il bonus di la Difesa concessa dell'armatura aumenta di 1 punto
\item 5 punti: diminuisci di 1 i Tiri Critici subiti per attacco in mischia, la Penalità alla Competenza diminuiscono di 1 punto, la penalità al Movimento diminuisce di 1 metro
\item 6 punti: la diminuzione del Tiro Critico subito si applica anche agli attacchi a distanza. annulli la Penalità alla Competenza ed al Movimento
\item 7 punti: portare una armatura non obbliga più ad eseguire la Prova di Magia
\end{itemize}

\subsection{Armi Leggere}\index{Armi Leggere} Spada Corta, Mazza leggera, Stocco, Scimitarra, Ascia ad una mano, Pugnale\label{listaarmileggere}

\begin{itemize}[leftmargin=*] \setlength{\itemsep}{0pt}

\item 4 punti: puoi usare la Destrezza al posto della Forza nel Tiro per Colpire.
\item 5 punti: puoi estrarre l'arma come parte dell'Azione di Movimento.
\item 7 punti: puoi estrarre l'arma come Azione Immediata.
\item 9 punti: aumenti di un grado il dado di danno dell'arma. Se il dado di danno diventa 8 o più l'arma acquisisce l'EDX sul massimo valore del dado.
\item 11 punti: aumenti di un grado il dado di danno dell'arma. L'EDX si riduce di 1.
\item 16 punti: usando una Reazione prendi +4 alla Difesa contro attacchi in mischia. Se eviti l'attacco puoi effettuare un attacco in risposta.


\end{itemize}

\subsection{Armi doppie} \index{Armi doppie} Grande Ascia Doppia, Flagello Doppio, Spada a due lame, Urgrosh\label{listaarmidoppie}

\begin{itemize}[leftmargin=*] \setlength{\itemsep}{0pt}
\item 4 punti: la tua competenza nell'uso di queste armi ti rende estremamente versatile dandoti la possibilità a inizio del tuo round di scegliere se essere difensivo o offensivo aumentando di 1 il Tiro per Colpire o la Difesa fino all'inizio del round successivo. Non costa Azioni.
\item 5 punti: prendendo -4 al Tiro per Colpire al primo attacco che esegui nel round prendi +4 alla Difesa fino all'inizio del tuo round successivo.
\item 7 punti: usare un arma doppia non leggera non cumula il -3 aggiuntivo al Tiro per Colpire.
\item 9 punti: la tua tecnica non lascia punti scoperti, per ogni Tiro per Colpire andato a segno nel round prendi +1 alla Difesa fino all'inizio del tuo round successivo.
\item 11 punti: colpisci vorticosamente con la tua arma. Il primo colpo andato a segno equivale a due colpi andati a segno.
\item 16 punti: ogni volta che colpisci con un tiro critico puoi portare, senza usare Azioni, un colpo con l'altra estremità dell'arma. Questo Tiro per Colpire non può a sua volta causare critici e ha -4 al Tiro per Colpire.

\end{itemize}

\subsection{Armi aggraziate}\index{Armi aggraziate} Stocco, Scimitarra, Falcione\label{listaarmiaggraziate}

\medskip

\begin{center}
\includegraphics[width=0.7\linewidth]{immagini/sciabole.png}
\end{center}

\begin{itemize}[leftmargin=*] \setlength{\itemsep}{0pt}
\item 4 punti: il tuo stile assomiglia molto ad una danza. Puoi usare il valore del Carisma o Destrezza al Tiro per Colpire.
\item 5 punti: puoi usare il punteggio di Intrattenere al posto di Competenza Armi nel Tiro per Colpire.
\item 7 punti: sai colpire dove fa veramente male. Il primo Colpo Critico somma un colpo critico aggiuntivo.
\item 9 punti: il dado dell'arma aumenta di una categoria.
\item 11 punti: usando una Azione di Reazione puoi cercare di intercettare gli attacchi dell'avversario, aggiungi +2 alla Difesa fino all'inizio del tuo round successivo.
\item 16 punti: la tua danza blocca l'avversario nell'affrontarti. Costringi l'avversario in mischia con te ad attaccare solo te fino alla fine del tuo prossimo round. 1 Azione.

\end{itemize}


\begin{center}
	\includegraphics[width=0.6\linewidth]{immagini/scythe-types.png}

	\emph{Eric Sloane. A Museum of Early American Tools.}

\end{center}

\subsection{Armi della morte}\index{Armi della morte} Picca Leggera, Picca Pesante, Falce, Falcetto\label{listaarmidelamorte}

\begin{itemize}[leftmargin=*] \setlength{\itemsep}{0pt}
\item 4 punti: puoi eseguire un Colpo di Grazia con il costo di 1 Azione.
\item 5 punti: il primo colpo critico che esegui sull'avversario somma un colpo critico aggiuntivo.
\item 7 punti: aumenti di un grado il dado di danno dell'arma.
\item 9 punti: il primo colpo critico che esegui sull'avversario somma 2 colpo critico aggiuntivi.
\item 11 punti: aumenti di un grado il dado di danno dell'arma.
\item 16 punti: aumenti di un grado il dado di danno dell'arma.
\end{itemize}

\subsection{Armi da stordimento}\index{Armi da stordimento} Manganello, Guanto chiodato\label{listaarmistordimento}

\begin{itemize}[leftmargin=*] \setlength{\itemsep}{0pt}
\item 4 punti: un avversario inconsapevole se colpito con queste armi (durante il round di sorpresa) deve effettuare un Tiro Salvezza Tempra DC 15 oppure essere Rallentato 1/1r.
\item 5 punti: per ogni Tiro Critico l'avversario deve fare un Tiro Salvezza su Tempra a DC 13 o essere allentato 1/1r.
\item 7 punti: raddoppi il tuo bonus di danno dato dalla Forza. Il Tiro Salvezza dell'abilità a 4 punti diventa 19.
\item 9 punti: la difficoltà dell'abilità al punto 4 diventa 19
\item 11 punti: la tua arma da stordimento fa 1d6 di danno non letale in più. Il Tiro Salvezza dell'abilità a 4 punti diventa 23
\item 16 punti: ogni volta che colpisci con un danno critico un avversario, un compagno in mischia con quell'avversario può usare una Reazione per effettuare un attacco contro di lui.

\end{itemize}


\begin{center}
	\includegraphics[width=0.7\linewidth]{immagini/david di Michelangelo.png}

	\emph{David di Michelangelo, Galleria dell'Accademia, Firenze}
\end{center}

\subsection{Armi da Lancio} Ascia ad una mano, Giavellotto, Tridente, Fionda, Pugnale\index{Armi da lancio}\label{listarmitiro}

\begin{itemize}[leftmargin=*] \setlength{\itemsep}{0pt}
\item 4 punti: sei diventato estremamente preciso nel lancio della tua arma hai un +1 al colpire e un +1 ai danni.
\item 5 punti: il primo Tiro Critico che esegui sull'avversario somma un colpo critico aggiuntivo.
\item 7 punti: la tua abilità ti permette di non avere tempi morti dopo il lancio di un arma puoi istantaneamente estrarne un altra senza consumare azioni.
\item 9 punti: il primo Tiro per Colpire scaglia 2 armi.
\item 11 punti: riduci di 6 la penalità alla gittata oltre lo standard.
\item 16 punti: sei diventato estremamente preciso nel lancio della tua arma hai un +4 al colpire e un +4 ai danni.
\end{itemize}



\subsection{Armi letali} Katana, Machete\index{Armi letali}\label{listarmiletali}

\begin{itemize}[leftmargin=*] \setlength{\itemsep}{0pt}

\item 4 punti: contro avversari sorpresi aggiungi al danno la tua Competenza Armi.
\item 5 punti: il primo Tiro Critico che esegui sull'avversario somma un Colpo Critico in aggiunta.
\item 7 punti: aumenti di un grado il dado di danno dell'arma. Se questo porta l'arma ad avere il d8 come dado di danno acquisisce anche EDX pari a 8.  Se l'arma ha già un EDX  questo diminuisce di 1.
\item 9 punti: il primo Colpo Critico che esegui sull'avversario aggiunge due Colpi Critici.
\item 11 punti: migliori EDX.  Se l'arma ha già un EDX  questo diminuisce di 1.
\item 16 punti: aumenti di un grado il dado di danno dell'arma.
\end{itemize}


%\begin{center}
%\includegraphics[width=0.7\linewidth]{immagini/katana3.png}

%\emph{Katana}
%\end{center}


\begin{center}
	\includegraphics[width=0.7\linewidth]{immagini/alabarda2.png}

	\emph{Fauchard, Partigiana, Spetum, Alabarda, Guisarma, Bardica}
\end{center}


\subsection{Aste} \index{Aste}Giavellotto, Tridente, Alabarda\label{listaarmiaste}

\begin{itemize}[leftmargin=*] \setlength{\itemsep}{0pt}

\item 4 punti: se fai almeno un tiro critico con il Tiro per Colpire puoi lasciare l'arma nel corpo dell'avversario, penalizzandolo con un -1 Destrezza. L'arma quando rimossa fa un danno critico.
\item 5 punti: puoi effettuare un attacco di opportunità contro gli avversari che attraversano la tua zona di mischia usando una Reazione.
\item 7 punti: puoi usare l'arma lunga in mischia entro un metro senza penalità. Il danno dell'abilità a 4 punti diventa pari a 2 danni critici.
\item 9 punti: il danno dell'abilità a 4 punti diventa pari a 3 danni critici.
\item 11 punti: la gittata se assente diventa 3 metri, se presente la raddoppi.
\item 16 punti: usando una Reazione puoi seguire l'avversario mantenendo la distanza attuale di mischia. Non puoi spostarti più del tuo Movimento.

\end{itemize}


\subsection{Balestre}\index{Balestre}Balestra leggera, Balestra pesante, Balestra ad una mano\label{listaarmibalestr}

\begin{itemize}[leftmargin=*] \setlength{\itemsep}{0pt}

\item 4 punti: guadagni l'Abilità \hyperlink{Tiro Rapido}{Tiro Rapido} (pag. \pageref{Tiro Rapido}).
\item 5 punti: il primo Tiro Critico che esegui sull'avversario somma un colpo critico aggiuntivo.
\item 7 punti: ogni Azione che dedichi a mirare, fino ad un massimo di 2, ti concede un +2 a colpire.
\item 9 punti: il primo Tiro Critico che esegui sull'avversario somma due colpi critici in aggiunta, non si cumula con il vantaggio al punto 5.
\item 11 punti: riduci di 6 la penalità per tirare oltre la gittata standard.
\item 16 punti: riduci di 6 la penalità per tirare oltre la gittata standard.

\end{itemize}


%\begin{center}
%\includegraphics[width=0.9\linewidth]{immagini/arma-balestra.png}
%\end{center}



\begin{center}
	\includegraphics[width=0.7\linewidth]{immagini/arma-asta.png}

	\emph{1 Spiedo dei lanzichenecchi; 2 Picca; 3 Lancia; 4 Spiedo da caccia; 5 Buttafuoco; 6 Falcione; 7 Partigiana ; 8 Alabarda; 9 Alabarda; 10 Roncone; 11 Mazzapicchio; 12 Berdica}
\end{center}


\subsection{Lance} \index{Lance}Alabarda, Urgrosh, Lancia da fante, Falcione in asta, Lancia

\begin{itemize}[leftmargin=*] \setlength{\itemsep}{0pt}
\item 4 punti: usata contro una carica od in carica, purché abbia l'abilità Controcarica, il danno critico aggiuntivo fa il massimo valore.
\item 5 punti: puoi usarla anche contro avversari a distanza di 1 metro senza penalità.
\item 7 punti: usata contro una carica od in carica, purché abbia l'abilità Controcarica fa un danno critico aggiuntivo.
\item 9 punti: rotei la tua arma. Usando 3 Azioni fai un unico Tiro per Colpire a -5. Confronta il tiro con la Difesa di tutte le creature in mischia con te per valutare se le hai colpite.
\item 11 punti: la portata della tua lancia diventa 3 metri.
\item 16 punti: usi 2 Azioni e fai un unico Tiro per Colpire. Se va a segno causi 3 colpi critici aggiuntivi.
\end{itemize}


\begin{center}
	\includegraphics[width=0.5\linewidth]{immagini/mazzafrusto.png}
\end{center}

\subsection{Palle rotanti} Flagello, Flagello Pesante, Flagello Doppio, Catena chiodata, Frusta\label{listaarmipallerotanti}

\begin{itemize}[leftmargin=*] \setlength{\itemsep}{0pt}
\item 4 punti: se il Tiro per Colpire va a segno puoi effettuare un ulteriore TC (senza consumare Azioni) a -5 contro un avversario in mischia con te che non sia l'avversario già colpito.
\item 5 punti: se colpisci due volte l'avversario nel round, il secondo Tiro per Colpire genera un danno critico aggiuntivo.
\item 7 punti: l'impatto dei tuo colpi è tale da stordire i nemici. Se colpisci l'avversario con un Tiro Critico questo subirà Rallentato 1/1r.
\item 9 punti: puoi usare una Azione Immediata ed usare la tua arma per cercare di deviare un Tiro per Colpire a te indirizzato su una creatura a distanza di mischia dall'avversario. Effettua un Tiro per Colpire, la manovra riesce solo se è superiore al Tiro per Colpire che vuoi deviare.
\item 11 punti: la precisione ed abilità nel roteare la tua arma è tale da confondere la difesa del nemico, ignori la protezione (Difesa) data dallo scudo.
\item 16 punti: puoi usare una Azione di Reazione ed usare la tua arma per cercare di proteggere una creatura in mischia con te fino all'inizio del tuo prossimo round. La creatura prende +4 alla Difesa.
\end{itemize}

\subsection{Pugno Vuoto} Pugni e Calci\index{Pugno Vuoto}\hypertarget{pugnovuoto}{}\label{listarmipugnonudo}

Hai addestrato il tuo corpo a diventare l'arma definitiva. Sei addestrato nell'usare calci e pugni in maniera efficace e letale.

La Lista Pugno Vuoto non beneficia del Colpo Critico, tranne per il vantaggio preso a 9 punti.

\textbf{Pugno Vuoto}: Ogni volta che prendi questa competenza il danno aumenta seguendo questa progressione: 1d6 (lista presa 2 volte), 1d8 (3), 2d6 (5), 2d8 (7), 2d10 (9), 3d6 (11), 3d8 (13), 3d10 (15), 4d6 (17).

Il giocatore può anche decidere di fare danno non letale non incorrendo in alcuna penalità, al danno può applicare a proprio piacere il valore di Forza o Destrezza.

\begin{itemize}[leftmargin=*] \setlength{\itemsep}{0pt}
\item 1 punto: tuoi pugni fanno danno letale (1d4). Puoi usare il valore di Forza o Destrezza al Tiro per Colpire ed al danno.
\item 4 punti: Saggezza della mano vuota. Puoi usare il valore della Saggezza al colpire ed al Danno al posto di Forza o Destrezza. Le penalità dell'attacco multiplo diventano -4 e non -5.
\item 5 punti: il punteggio di Difesa naturale aumenta di 1 punto.
\item 9 punti: colpo solitario. Usi tre Azioni per portare un solo colpo devastante, il colpo se va a segno somma 2 colpi critici aggiuntivi.
\item 11 punti: ottieni un bonus al colpire ed al danno pari al doppio della Caratteristica usata per determinare questo bonus.

\end{itemize}

Consultate \hyperlink{equivalenzaarmimagiche}{Vulnerabilità, Resistenza e Immunità} (pag. \pageref{equivalenzaarmimagiche}) per sapere quanto è magico il vostro colpo.

\subsection{Rompi Cranio} \index{Rompi Cranio}Flagello, Maglio da guerra, Martello da guerra, Mazza Leggera, Mazza flangiata, Mazza chiodata
\label{listaarmirompicranio}

\begin{center}
\includegraphics[width=0.7\linewidth]{immagini/arma-mazza3.png}
\end{center}


\begin{itemize}[leftmargin=*] \setlength{\itemsep}{0pt}
\item 4 punti: sei diventato cosi abile che puoi controllare la forza dei tuo colpi, puoi fare danno non letale senza penalità al colpire.

Puoi scegliere di ridurre di 4 il Tiro per Colpire per aumentare il danno di 8 (non cumulabile con Colpi Potenti).
\item 5 punti: il primo Tiro Critico che esegui sull'avversario somma un colpo critico aggiuntivo.
\item 7 punti: i tuoi colpi frastornano il nemico. Ogni Tiro Critico andato a segno abbassa la Difesa di 1 punto, fino ad un massimo di 3. L'avversario recupera all'inizio del suo round un punto di penalità.
\item 9 punti: aumenti di un grado il dado di danno dell'arma.
\item 11 punti: il vantaggio a 5 punti diventa di due colpi critici.
\item 16 punti: usando una Reazione, ogni volta che colpisci con un Tiro Critico, puoi effettuare un altro Tiro per Colpire con lo stesso punteggio contro diverso avversario purché in distanza di mischia.

\end{itemize}


\begin{center}
	\includegraphics[width=0.9\linewidth]{immagini/scudotorre.png}

	\emph{Henry Justice Ford. Scudo Pesante}
\end{center}


\subsection{Scudi}\index{Scudi} Scudi Leggeri, Medi, Pesanti\label{listaarmiscudi}

Sei un maestro nell'uso degli scudi, anche come arma.

Puoi usare lo scudo come arma, uno scudo piccolo fa 1d4 di danno (B/T), uno scudo medio fa 1d6 di danno (B/T), uno scudo pesante fa 1d8 di danno (B/T).
Non hai penalità al colpire con lo scudo, per te lo scudo non è un arma improvvisata. Questa Lista d'Armi non ha il bonus dei 6 punti e quello dei 18 comuni alle altre Liste d'Armi.

La tua tecnica mescola efficacemente difesa e attacco. Puoi lanciare il tuo scudo con una gittata di 6 metri.


\begin{itemize}[leftmargin=*] \setlength{\itemsep}{0pt}
\item 1 punto: sei competente in tutte le tipologie di scudo. Non hai il vincolo del limite di Forza 1 sugli Scudi Pesanti.
\item 2 punti: il bonus di Difesa quando usi lo scudo aumenta di 1 e ogni 4 volte che prendi questa Lista d'Armi (6,10,14,18..) Non usi Azioni per ripristinare lo scudo in Difesa dopo aver effettuato un attacco con lo stesso.
\item 3 punti: la penalità Competenza Magica data dallo scudo diminuisce di un dado
\item 4 punti: la penalità al Tiro per Colpire diminuisce di 1.
\item 5 punti: aumenta di 1 la categoria di danno dello scudo ed ogni 4 punti ulteriori in lista (9,13,17..).
\item 8 punti: ogni alleato adiacente (entro 1 metro) a te ha un +1 Difesa. Puoi lanciare lo scudo per difendere un compagno garantendogli +2 alla Difesa, da usare come Reazione. Lo scudo cade a terra dove hai difeso il compagno. Puoi lanciare il tuo scudo con una gittata di 9 metri. La penalità Competenza Magica data dallo scudo diminuisce di un dado.
\item 12 punti: puoi lanciare il tuo scudo come fosse un arma con gittata 12 metri. Se colpisci ed ottieni un Tiro Critico nel lancio dello scudo questo torna nelle tue mani a fine round. Ogni alleato adiacente (entro 1 metro) a te ha un +2 Difesa.
\item 16 punti: se un avversario esegue almeno tue tiri per colpire mancandoti entrambi puoi effettuare come Reazione un attacco di scudo contro di lui.
\item 18 punti: lo scudo lanciato ha una gittata di 18 metri e torna nelle tue mani, se non impossibilitato. Questo ti permette di effettuare attacchi multipli anche da lancio con il medesimo scudo. Puoi lanciare lo scudo per difendere un compagno garantendogli +4 alla Difesa, da usare come Reazione. Lo scudo cade a terra dove hai difeso il compagno.

Non è possibile applicare questi bonus se si usa più di uno scudo.

\end{itemize}

\subsection{Scuri e Accette}\index{Scuri e Accette} Ascia ad una mano, Ascia da battaglia, Ascia Martello, Grande Ascia Doppia, attacchi naturali del Sornelian\label{listaasce}

\begin{itemize}[leftmargin=*] \setlength{\itemsep}{0pt}

\item 4 punti: la furia dei tuoi attacchi è tale che guadagni un +2 al danno sul colpo.
\item 5 punti: se uccidi una creatura con un colpo critico il danno in eccesso, se il Tiro per Colpire è sufficiente, lo prende un'altra creatura in mischia con te.
\item 7 punti: le ferite che provochi sono cosi profonde che causi Sanguinamento. Ogni tuo attacco andato a segno aumenta di 1 il sanguinamento fino ad un massimo di Sanguinamento 5.
\item 9 punti: ogni colpo critico che provochi aumenta il Sanguinamento di 2, fino ad un massimo di 10.


\smallskip

\begin{center}
	\includegraphics[width=0.7\linewidth]{immagini/scurieaccette.png}
\end{center}

\item 11 punti: le ferite che provochi sono cosi profonde che causi molto Sanguinamento. Il valore di Sanguinamento massimo sale a 15.
\item 16 punti: consumi 3 Azioni, effettui un singolo Tiro per Colpire che confronti contro tutte le creature in un cono pari al tuo movimento per capire se le hai colpite. Al termine dell'attacco sei in fondo al cono.

\end{itemize}

\subsection{Spade}\index{Spade} Spada Corta, Spada Lunga, Spadone a due mani, Spada bastarda, Spada a due lame, Spada larga, Spada a due lame, Estoc

\begin{center}
	\includegraphics[width=0.8\linewidth]{immagini/arma-tipi-di-spade.png}

	\emph{A Sciabola, B Scimitarra, C Spada ad una mano, D Spada larga, E Stocco, F Spada lunga, G Spada a una mano e mezza o bastarda, H Spadone a due mani}
\end{center}

\medskip

\begin{itemize}[leftmargin=*] \setlength{\itemsep}{0pt}

\item 4 punti: la tua maestria nella tecnica della spada ti conferisce +1 a danno e Tiro per Colpire.
\item 5 punti: il primo Tiro Critico che esegui sull'avversario somma un colpo critico aggiuntivo.
\item 7 punti: la tua maestria nella tecnica della spada ti conferisce +2 a danno e Tiro per Colpire.
\item 9 punti: il primo colpo andato a segno nel round somma un colpo critico.
\item 12 punti: hai raggiunto l'apice della maestria con la spada i tuo colpi sono precisi e difficili da prevedere ottieni +1 a danno, Tiro per Colpire e Difesa. L'EDX della spada se presente si abbassa di 1.
\item 16 punti: il dado di danno della tua spada aumenta di una categoria.

La mano che non tiene la spada deve essere libera o usata sull'arma.

\end{itemize}

\subsection{Spade e Scudi}\index{Spade e Scudi} Spada Corta, Spada Lunga, Spada larga, Scudo Piccolo, Scudo Medio\label{listaarmispadescudi}

\begin{itemize}[leftmargin=*] \setlength{\itemsep}{0pt}

\item 4 punti: la tua maestria nella tecnica della spada e scudo ti conferisce +1 alla Difesa ed al Tiro per Colpire.
\item 5 punti: se vai a segno con due colpi consecutivi con la spada puoi effettuare un Tiro per Colpire, senza cumulo ulteriore di penalità da multiattacco o attacco improvvisato, con lo scudo consumando una Reazione.
\item 7 punti: la tua maestria nella tecnica della spada e scudo ti conferisce +2 alla Difesa ed al Tiro per Colpire.
\item 9 punti: usando una Reazione puoi usare lo scudo per proteggere una creatura a distanza di mischia con te. La sua Difesa aumenta di 2 punti fino all'inizio del round successivo.
\item 11 punti: il dado di danno della tua spada aumenta di una categoria, EDX si abbassa di 1.
\item 16 punti: sommi il valore di Difesa dello Scudo ai Tiri Salvezza su Riflessi.

\end{itemize}

Il personaggio deve tenere in una mano la spada e nell'altra lo scudo.


\subsection{Armi Semplici} Pugnale, Mazza Leggera, Mazza chiodata, Bastone, Balestra (Leggera), Giavellotto.\index{Armi Semplici}\hypertarget{armi.semplici}{}\label{listaarmisemplice}

\medskip

Questa suddivisione è sceglibile anche da chi non ha assegnato punti a Competenza Armi. Questa Lista d'Armi non concede bonus specifici.

\subsection{Armi in più Lista d'Armi}\index{Armi in più Liste d'Armi}\label{listaarmiinpiuliste}

Quando un personaggio usa un arma presente in più Liste d'Armi conosciute può applicare per avversario una sola tecnica (una Lista d'Armi) di combattimento, non cumula i vantaggi anche di eventuali altre liste.

Utilizzando 2 Azioni può concentrarsi e passare ad utilizzare i bonus derivanti dall'applicazione di una diversa Lista D'Armi.

\end{multicols}

\vfill

\begin{center}
\includegraphics[width=0.95\linewidth]{immagini/brancastle.png}

\emph{dettaglio dal Castello di Bran, Transilvania}
\end{center}



\pagebreak

