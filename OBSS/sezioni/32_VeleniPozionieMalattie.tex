\section{Veleni, Pozioni e Malattie}\index{Veleni}\index{Pozioni}\index{Malattie}

\label{veleni-e-pozioni}


\begin{changemargin}{0.3cm}{0.3cm}\begin{enfasi}{
Un giorno, un uomo fu colpito da una freccia avvelenata. Gli amici e i parenti, in ansia, chiamarono un medico. Quando gli si avvicinarono per prendere la freccia, l'uomo disse loro: "Prima di farlo, vorrei sapere chi mi ha trafitto con questa freccia... Era uno schiavo, un re, o un bramino? Era grande? Piccolo? Di che colore era la sua pelle? Dove viveva? E la freccia com'è stata costruita? Quale veleno è stato impiegato? ..."

Mentre si stava ponendo tutte queste domande... il veleno fece il suo effetto e l'uomo ferito finì per morire. (Budda)}\end{enfasi}\end{changemargin}\medskip

\begin{multicols}{2}

\subsection{Tipo di Veleno e Pozione}\label{tipidiveleno}

I veleni e pozioni possono distinguersi in base all'effetto scatenato.
Non tutti i veleni sono tossici se ingeriti o inalati.

Per identificare una pozione naturale è necessario una prova di Erboristeria a DC uguale alla rarità della pianta oppure in caso di Veleni la difficoltà è pari al Tiro Salvezza dello stesso. Costa 1 Azione ogni 10 di DC, con Erboristeria a 6 o più costa 1 Azione ogni 15 DC, con 12 punti costa 1 Azione ogni 20 DC. Le pozioni se non descritto diversamente devono essere bevute (ingestione).

\textbf{Contatto}: sono contratti nel momento in cui qualcuno tocca il veleno con la pelle nuda. I veleni a contatto hanno solitamente un tempo di insorgenza di 1 round. Un veleno a contatto può essere un unguento, balsamo, liquido di qualsiasi densità o anche polvere se specifica per contatto e non inalazione.

\textbf{Ingestione}: si attivano quando una creatura le mangia o le beve. I veleni ad ingestione hanno solitamente un tempo di insorgenza di 10 minuti.

\textbf{Ferimento}: vengono trasferiti soprattutto con gli attacchi di alcune creature e tramite armi cosparse di veleno. I veleni a ferimento hanno solitamente un tempo di insorgenza istantaneo.

\textbf{Inalazione (R)}: si attivano nel momento in cui una creatura entra in un'area che contiene tali veleni. Molti veleni ad inalazione riempiono un volume pari ad un cubo con lato di 3x3x3 metri per dose. Le creature possono tentare di trattenere il fiato mentre si trovano all'interno dell'area per evitare di inalare la tossina.
Vedi regole per trattenere il fiato e soffocare in \hyperlink{trattenereilfiato}{Ambiente} (pag. \pageref{trattenereilfiato}).


\subsection{Insorgenza ed Effetto}\index{Insorgenza veleno}\index{Tempo di attivazione veleno}\label{insorgenzaveleno}\hypertarget{insorgenzaveleno}{}

Per insorgenza si intende quanto tempo ci mette il veleno o pozione a fare effetto. Se il tempo di insorgenza è 1 Turno significa che per gli effetti del veleno/pozione ed il Tiro Salvezza si aspetta 10 minuti. Se nella tabella del veleno/pozione insorgenza non è specificata significa che l'effetto è immediato dopo l'entrata in contatto con il veleno.

L'effetto di un veleno/pozione è immediato dopo l'insorgenza. Verificare la descrizione del veleno per capirne l'effetto. Se il Tiro Salvezza su Tempra riesce il veleno non ha fatto effetto e si può ritenere neutralizzato. 

Ci sono alcuni casi in cui è presente la voce Frequenza, in queste occasioni il Tiro Salvezza va ripetuto ogni volta che passa la Frequenza indicata, in caso di fallimento del Tiro Salvezza gli effetti indicati vengono nuovamente applicati.

Bere una pozione tenuta in mano costa 1 Azione Immediata, farla bere ad un compagno privo di sensi costa 2 Azioni.

Se il personaggio \textbf{dedica 1 minuto} a bere una Pozione di Cura o Naturale questa avrà effetto massimizzato.\index{Pozioni effetto massimizzato}

%\begin{center}
%\includegraphics[height=0.6\linewidth]{immagini/potion.png}
%\end{center}

\begin{changemargin}{0.3cm}{0.3cm}\begin{narratore}
I veleni qui proposti sono alcuni dei tanti presenti e possibili. Usali come linee guida. Se per tua etica e stile non ti piacciono i veleni, specialmente quelli più cattivi, suggerisco di usare le Pozioni Generiche che trovi in fondo al capitolo. Sono veleni più blandi e meno personali, probabilmente più facilmente usabili anche dai giocatori.
\end{narratore}\end{changemargin}

\subsubsection{Avvelenati}\index{Avvelenati}\label{avvelenato}

\textbf{Prima dose}: Quando si viene esposti a un veleno per la prima volta (durante la propria azione o quella di qualcun altro), è necessario effettuare un Tiro Salvezza all'insorgenza per evitare di venire avvelenati.

\textbf{Successo}: Si resiste al veleno. Non si subiscono effetti negativi e non sono necessari ulteriori Tiri Salvezza.

\textbf{Fallimento}: Siete stati avvelenati e si subisce subito l'effetto indicato.

\textbf{Più dosi}: Se si vieni esposti a più dosi dello stesso veleno prima dell'insorgenza la difficoltà del Tiro Salvezza aumenta di 1 per dose aggiuntiva.\index{Veleno più dosi}

\textbf{In tempi diversi}: se si viene esposti al veleno in tempi diversi dopo la prima insorgenza, ogni volta ci sarà un nuovo Tiro Salvezza e si subiranno gli eventuali effetti previsti, se invece l'esposizione avviene prima dell'insorgenza allora il Tiro Salvezza è unico e si conta come \emph{Più Dosi}.

Se si viene esposti a veleni diversi è necessario effettuare un Tiro Salvezza per ogni tipo di veleno assunto.


\begin{changemargin}{0.3cm}{0.3cm}\begin{tcolorbox}[title = Veleno ?]
{Il veleno è un arma a doppio taglio. Finché la usi tu va bene ma se te la usano contro, magari lo stesso, diventa un problema. Ci sono anche degli aspetti etici nell'usare i veleni, valuta se i tuoi Tratti ti permettono di usare dei veleni e di che tipi.}\end{tcolorbox}\end{changemargin}

\subsection{Applicare il Veleno}\index{Applicare il Veleno}\label{applicareveleno}

Applicare il veleno ad un'arma o ad una munizione richiede 1 Azione.

Ogni volta che un personaggio applica o prepara un veleno per l'uso deve effettuare una Prova di Erboristeria (DC 11) e se ottiene un fallimento critico nella prova è entrato in contatto con il veleno e deve effettuare un Tiro Salvezza contro il veleno come di regola. Ciò non consuma la dose di veleno.

Ogni volta che un personaggio attacca con un'arma avvelenata, se esegue un fallimento critico con il Tiro per Colpire esegue un nuovo Tiro per Colpire e se si \emph{colpisce} allora si espone agli effetti del veleno. Ciò consuma il veleno sull'arma.\index{Fallimento critico con arma avvelenata}

Un pozione di veleno è sufficiente per coprire di veleno un arma media oppure 3 frecce. Il veleno viene così consumato e rimane attivo sull'arma finché questa non colpisce.

Una creatura sotto gli effetti di un veleno, anche se non manifestato, ha la condizione di Avvelenato.

\subsection{Rimuovere il veleno}

L'incantesimo \hyperlink{incrimuoviveleno}{Rimuovi Veleno} (pag. \pageref{incrimuoviveleno}) esegue una prova di \hyperlink{contrastareincantesimi}{contrasto} con il veleno, e quindi la condizione avvelenato.

Se la DC del veleno non è espressa allora si considera che sia sufficiente il semplice lancio dell'incantesimo per annullarne gli effetti.

Una prova di Pronto Soccorso\index{Pronto Soccorso e Veleni}\index{Veleni e Pronto Soccorso}, 3 Azioni, che sia almeno la metà della DC del veleno entro il tempo dell'insorgenza, permette di effettuare subito un Tiro Salvezza con un bonus di +2.

Questo Tiro Salvezza se riesce annulla gli effetti del veleno, se fallisce la creatura rimane avvelenata, se fallisce criticamente il TS avrà una penalità di -2. 
Una volta fatta la prova di Pronto Soccorso non è più possibile rifarla se non dopo l'insorgenza.

Un trattamento continuativo di Pronto Soccorso per 8 ore permette di effettuare un nuovo Tiro Salvezza con +1d6 bonus dopo l'attivazione del veleno se questo è ancora attivo.

\medskip

\begin{changemargin}{0.3cm}{0.3cm}\begin{enfasi}{
Io credo che una foglia d'erba non sia meno di una giornata di lavoro compiuto dagli astri. (Walt Whitman)}
\end{enfasi}\end{changemargin}

\subsection{Creare Veleni e Pozioni Naturali}\index{Creare Veleni Naturali}\label{crearevelenonaturale}\index{Creare Pozioni Naturali}

Pozioni e veleni naturali possono essere realizzati usando Erboristeria. La DC per preparare il veleno o pozione è uguale alla DC del Tiro Salvezza -4.

Un Erborista può preparare contemporaneamente fino al suo valore in (Erboristeria/2)+1 pozioni naturali o veleni.

Se gli ingredienti si comprano il costo per preparare il veleno è metà del costo di vendita indicato, se si cercano in natura il costo di produzione scende ad un quarto. Il tempo per preparare queste pozioni/veleni è pari alla DC/3 in ore. 

Ottenendo un fallimento critico con la prova di Erboristeria ci si espone al prodotto durante la sua preparazione. Se la prova di DC Erboristeria ha successo se ne preparano 1d2+1 dosi.

Una Pozione che \emph{\textbf{Rimuove}} una condizione è efficace se la sua DC è superiore a quella della Condizione stessa. \index{Pozioni e rimozioni condizioni}

Gli esempi seguenti rappresentano solo alcuni dei possibili veleni. Tutti i costi sono espressi in Monete d'Oro. I Veleni sono presentati, specialmente nel Mostruario con questa dizione: Nome Veleno, Uso (I/R/F/C), tempo Insorgenza, DC del Tiro Salvezza, Effetto.

\begin{changemargin}{0.3cm}{0.3cm}\begin{narratore}
I veleni fanno parte della lunga tradizione dei problemi ed avversità nei giochi di ruolo. Quando volete usare un veleno pensate innanzitutto perché si trova li, per chi doveva essere usato, per quale scopo. Non è detto che tutti i veleni debbano uccidere, un abile ladro potrebbe anche usare veleni stordenti o che indeboliscono la volontà del suo obiettivo giusto quel tanto che basta a farsi aprire la cassaforte.
\end{narratore}\end{changemargin}

\subsection{Dove trovare le piante}

Ogni Pozione Naturale ha indicato in \textbf{Loc.} una sigla che serve a determinare dove poter trovare le piante.

Es: Gusterbloon CM14. La prima Lettera indica il CLIMA, la Seconda indica l'AMBIENTE, la Terza indica la RARITA'. La rarità indica il valore minimo della prova di Natura o Erboristeria per trovare l'erba/pianta ricercata, se c'è corrispondenza di clima ed ambiente.


\end{multicols}

\vspace{2cm}

\textbf{Tabella: Veleni. (costo in Monete d'Oro)}\index[Tabelle]{Tabella Veleni}\label{tabellaveleni}

\medskip

\noindent\begin{tabularx}{1\textwidth}{m{4.8cm}lllm{6.5cm}c}
	\toprule
	\textbf{Nome Veleno} & \textbf{Uso} & \textbf{Ins.} & \textbf{TS} & \textbf{Effetto (danno)} & \textbf{Costo}\\
	\midrule
	Bacca Viola di Barsar\index{Bacca Viola di Barsar}& I& 1 r& 18& Incapace di violenze per 3d8 ore& 40 \\
	\midrule
	Bacche Azzurre di fosso \index{Bacche Azzurre di fosso}& I& 1 T& 21& -1d3 Intelligenza e Saggezza per 6 ore& 55\\
	\midrule
	Bava fermentata di Lucos \index{Bava fermentata di Lucos}\label{bavadilucos}\hypertarget{bavadilucos}{}& F& - & 15& 1d8 Punti Ferita& 25\\
	\midrule
	Cenere di Corteccia Gialla \index{Cenere di Corteccia Gialla} & F& 6 r& 15& Privo di sensi per 1d3 ore& 25\\
	\midrule
	Concentrato Viola \index{Concentrato Viola} & F& -& 15& 2d6 Punti Ferita & 15\\
	\midrule
	Dita di Daraka\index{Dita di Daraka} & F& - & 17& -1d6 Forza, per 1 ora & 35\\
	\midrule
	Erba puntuta rosa \index{Erba puntuta rosa}& I& 1 r& 22& -1d6 Destrezza, per 1 ora& 60\\
	\midrule
	Fegato di Toporagno Viola \index{Fegato di Toporagno Viola} & I& 1 ora& 25& 2d6 di danno a Saggezza e Intelligenza. Permanente & 75 \\
	\midrule
	Fiocco bianco di Mucot \index{Fiocco bianco di Mucot}& C& - & 20& Dorme per 2d12 ore& 20\\
	\midrule
	Fumi di Curna\index{Fumi di Curna} & R& - & 18& -1d3 Saggezza & 40\\
	\midrule
	Gelo blu \index{Gelo blu} & F& -& 18& 3d6 Punti Ferita da freddo& 25\\
	\midrule
	Grasso di Toporagno Viola \index{Grasso di Toporagno Viola} & C& 1 r& 13& 2d12 Punti Ferita & 15\\
	\midrule
	Lingua di Kreex \index{Lingua di Kreex} & F& - & 20& La ferita sanguina. +1 danno da sanguinamento. 1 uso nelle 24 ore. & 50 \\
	\midrule
	Mistura Rossa \index{Mistura Rossa} & F& -& 13& -1d6 TC/TS per 10 minuti & 10\\
	\midrule
	Muschio Giallo \index{Muschio Giallo}& I& 1 r& 20& la creatura guadagna una taglia. -2 Int e Sag. Durata 10 minuti& 50\\
	\midrule
	Nocciolo di Dennar \index{Nocciolo di Dennar}& I& 1 r& 13& -1d2 Forza, per 3gg& 15\\
	\midrule
	Olio di Nabar \index{Olio di Nabar}& R-F& - & 20& Confuso per 2d6 round& 50\\
	\midrule
	Pelle di Rospo Azzurro \index{Pelle di Rospo Azzurro}& C& 10 r & 22& Paralizzato per 1d6 turni& 60\\
	\midrule
	Polline di Rosa di Omro\index{Polline di Rosa di Omro} & I& - & 15& -1d3 Costituzione e Destrezza, per 1 ora & 25\\
	\midrule
	Profumo di Ragmor \index{Profumo di Ragmor}& R& - & 16& -1d3 Carisma, per 1 giorno & 30\\
	\midrule
	Sangue di Thrun \index{Sangue di Thrun} & C& - & 26& -1d3 Costituzione & 80\\
	\midrule
	Succo di Ythis\index{Succo di Ythis} & I& 1 T& 14& -1d2 Intelligenza, per 1g& 20\\
	\midrule
	Veleno di Ottalm\index{Veleno di Ottalm}& F& - & 20& Morte o -1d2 Costituzione permanente& 50\\
	\midrule
	Veleno di Serpe del Sangue \index{Veleno di Serpe del Sangue} & F& - & 25& Paralisi per 1d6 ore -1d4 punti Forza per 7 giorni & 75\\
	\bottomrule
\end{tabularx}

\medskip

\textbf{Applicazione}: \textbf{I}(ngestione), \textbf{F}(erimento), \textbf{C}(ontatto), \textbf{R}(espirazione).

Il Tiro Salvezza è sempre su Tempra se non specificato diversamente

I punti caratteristica persi si recuperano al ritmo di 1 al giorno se non permanenti o indicato diversamente.

\vfill

\begin{center}
\includegraphics[height=0.3\linewidth]{immagini/poison.png}
\end{center}



%\vfill


%\begin{center}
%\includegraphics[width=0.5\linewidth]{immagini/funeralebarca.png}
%\end{center}


\textbf{Tabella: Pozioni Naturali (costo in monete d'oro)}\index[Tabelle]{Pozioni Naturali}\label{tabellapozioni}

\medskip 

\noindent\begin{xltabular}{1\textwidth}{
		>{\raggedright\arraybackslash}p{3.1cm}
		>{\centering\arraybackslash}p{0.4cm}
		>{\centering\arraybackslash}p{0.8cm}
		>{\centering\arraybackslash}p{0.6cm}
		>{\raggedright\arraybackslash}X
		>{\centering\arraybackslash}p{1cm}
		>{\centering\arraybackslash}p{1cm}
	}
	\textbf{Nome}& \textbf{Uso} & \textbf{Ins.} & \textbf{DC} & \textbf{Effetto}& \textbf{Loc.} & \textbf{Costo} \\
	\toprule
	Arduur\index{Arduur} & I& 1 r & 25& Rimuove Veleni& SZ20 & 75 \\
	\toprule
	Arksun\index{Arksun} & C& 1 T& 25& Cura 1d6 PF a Turno per 3 turni& MT20 & 75 \\
\toprule
Attarna\index{Attarna} & I& 1 T & 20& Concede un nuovo Tiro Salvezza per Malattie con un +1d6& TF17 & 50 \\
\toprule
Bacche di Ljust \index{Bacche di Ljust} & I& 1 r & 16& Preso la sera recuperi il doppio dei PF, minimo 4 & AZ15 & 10 \\
\toprule
Bacio di Ljust\index{Bacio di Ljust}& C& 1 r & 35& Cura 100 Punti Ferita& HO26 & 500\\
\toprule
Barannie\index{Barannie} & I& 10 r& 15& Rimuove nausea & MD13 & 3 \\
\toprule
Callynthine\index{Callynthine} & C& 1 ora& 24& Rinsalda le fratture, cura 2d8+8 PF & CF26 & 200\\
\toprule
Corteccia Dagmather\index{Polvere di corteccia di Dagmather}& R& 1 r & 25& Rimuove un livello di Affaticamento& SS17 & 15 \\
\toprule
Corteccia di Aklent\index{Corteccia di Aklent}& I& 1 T & 10& La corteccia masticata per almeno 10 round concede per le 24 ore successive un +1 Tiro Salvezza vs Veleno& MT15 & 1\\
\toprule
Culcoa\index{Culcoa}& C& 1 r & 16& Recuperi 2d6 da danno da fuoco & TS17 & 8 \\
\toprule
Darsirion\index{Darsirion}& C& 1 r & 25& Cura 1d4 Punti Ferita& CM14 & 5 \\
\toprule
Delrean Plus\index{Delrean Plus} & I& 1 r & 18& Allontana insetti per 3 giorni & CC13 & 5\\
\toprule
Delrean\index{Delrean} & C& 1 r & 15& Allontana insetti per 1 giorno & CC9 & 2\\
\toprule
Draaf \index{Draaf} & C& 1 r & 20& Cura 1d8 Punti Ferita& SO17 & 50 \\
\toprule
Eldrin'tail\index{Eldrin'tail}& I& 1 r& 15& Concede un nuovo Tiro Salvezza su Veleni& FH20 & 18 \\
\toprule
Estratto di Bacca Illa\index{Estratto di Bacca Illa bruciata}& I& 1 r & 15& +2 Iniziativa, +2 Destrezza, -1d6 Tiro Salvezza su Volontà, per 1 minuto& MS17 & 15\\
\toprule
Estratto radice Gisenosa\index{Estratto di radice Gisenosa}& I& 3 T& 15& Cura tosse e raffreddore& MT14 & 3\\
\toprule
Garioe\index{Garioe}& I& 1 r & 25& Cura 2d6 Punti Ferita& AZ20 & 95 \\
\toprule
Gelfnus \index{Geffnull}& I& 5 r & 28& Cura 3d8+3 Punti Ferita & EV27 & 150 \\
\toprule
Gusperboon \index{Gusterbloon}& C& 1 r & 20& La pelle diventa camaleontica concedendo un +1d6 alla prove di Furtività & CM14 & 8\\
\toprule
Gylvert\index{Gylvert} & I& 10 r& 25& Concede respirare sott'acqua per 4 ore & MO23 & 3 \\
\toprule
Harfy \index{Harfy} & C& I & 12& -1 al sanguinamento& SS6 & 3 \\
\toprule
Harfindar\index{Harfindar}& I& 1 T & 15& Fa abortire& SS15 & 3 \\
\toprule
Jojopo\index{Jojopo}& C& 1 r & 15& Resistenza a Freddo per 1 ora & FM17 & 45 \\
\toprule
Kelventare\index{Kelventare}& I& 2 r & 28& Recuperi 2d6 Punti Ferita & TT16 & 100 \\
\toprule
Klagul\index{Klagul}& C& 1 T & 20& Pulisce i denti & SS14 & 2 \\
\toprule
Klandor\index{Klandor} & I& I& 15& Rimuove paralisi. Aumenta di 1 il livello di affaticamento& HB18 & 18 \\
\toprule
Klynkyx\index{Klynkyx} & C& 1 ora & 15& Fa cadere tutti i capelli per 1d6+4 gg & MO16 & 4\\
\toprule
Lievito Muschio Bianco \index{Lievito di Muschio Bianco} & I& 10 r& 12& I prodotti da forno che usano questo lievito causano meteorismo incontrollabile ed incredibilmente puzzolente per 12 ore & CA12 & 1\\
\toprule
Lingua Rossa di Xabax\index{Lingua Rossa di Xabax}& C& 1 T & 20& Cura 2d6 Punti Ferita ma se c'è malattia o veleno tenta la rimozione causando 2d6 PF di danno & TA21 & 13 \\
\toprule
Melandrir\index{Melandrir}& I& 1 r & 15& Concede un nuovo Tiro Salvezza per Malattie con +4& CF20 & 100 \\
\toprule
Mirenna\index{Mirenna} & I& 1 r & 20& Cura 5 Punti Ferita& CM16 & 30 \\
\toprule
Miscela 31\index{Miscela 31}& I& 1 T & 20&La cavalcatura è estremamente resistente. +4 ore di galoppo al giorno & SM18 & 15\\
\toprule
Muschio argentato\index{Muschio Argentato}& I& I& 25& Rimuove Malattie magiche & MU26 & 250\\
\toprule
Nazamuse \index{Nazamuse}& I& I& 30& Rimuove Veleni e Malattie naturali & EW9 & 175\\
\toprule
Nelthalion \index{Nelthalion} & I& I& 15& Fa vomitare& SR9 & 1\\
\toprule
Petali di Lisbeth \index{Petali di Lisbeth}& I& 1 T & 15&+2 Intelligenza, -2 Destrezza per 10 minuti & MC15 & 20 \\
\toprule
Polline di Rosa Verde\index{Polline di Rosa Verde}& R& 3 T & 25& Recuperi 2d4 danni Intelligenza e Saggezza& FA26 & 350 \\
\toprule
Radice di Kathaus\index{Radice secca di Kathaus} & R& 1 r& 20& +2 Forza e Destrezza per 1 ora & FW16 & 100 \\
\toprule
%Rewky\index{Rewky}& I& 1 T& 25& Cura 2d8 Punti Ferita& TD19 & 20\\
%\toprule
Silea\index{Silea}& C& 5 r& 15& Cura 1d6+3 Punti Ferita & TT17 & 50 \\
\toprule
Estratto 100 erbe\index{Estratto alcolico 100 erbe}& I& I& 24& Rimuove Cecità, Sordità, Paralisi, Veleno &FM21&150 \\
\toprule
Uovo di Urk\index{Uovo di Urk}& I& 1 T & 12& 1 giorno di cibo& FH14 & 1\\
\toprule
Uscaboo \index{Uscaboo}& R& 1 T & 25& Rimuove cecità& MO21 & 125 \\
\toprule
Wickalim\index{Wickalim} & I& 1 ora& 15& Cura 2 Punti Ferita& TD7 & 5 \\
\toprule
Yajeth\index{Yajeth}& I& 1 T& 20& Cura 2d8 Punti Ferita& MO16 & 100
\end{xltabular}


\begin{multicols}{2}

\subsubsection{Note sui Veleni e Pozioni}


\textbf{Arduur}: Un'erba sacra, raccolta principalmente nei boschi di Sangzhar. Viene macerata e ridotta in un decotto, bevuto per rimuovere i veleni più persistenti.

\textbf{Arksun}: Pozione dai riflessi dorati, distillata dalle lacrime dei fiori di sole. Ogni druido custodisce gelosamente la ricetta.

\textbf{Attarna}: Estratto dalla corteccia degli alberi di Tarna. La sua assunzione garantisce una possibilità contro malattie debilitanti.

\textbf{Bacca Viola di Barsar}: curiosità, il Toporagno viola è schifato da queste bacche.

\textbf{Bacio di Ljust}: Leggendario rimedio. Si dice che ogni pianta, da cui si ricava questo elisir, cresca al centro di un cerchio di funghetti magici.

\textbf{Barannie}: Un'erba semplice ma efficace, spesso usata dai pastori nomadi per alleviare la nausea.

\textbf{Bava fermentata di Lucos}: Lucos è una lucertola erbivora e pacifica. La bava raccolta va fatta fermentare al buio per 1 settimana prima di essere utilizzabile.

\textbf{Callynthine}: Creato dalle radici della vite di Callyntha, il suo colore azzurro scuro riflette la nobiltà del suo effetto curativo.

\textbf{Cenere di Corteccia Gialla}: la corteccia va prima macerata e battuta in acqua e sale. La poltiglia risultante va seccata e poi fatta scaldare senza bruciarla direttamente.

\textbf{Corteccia Dagmather}: Piccolo arbusto sempre verde.

\textbf{Corteccia di Aklent}: chiamato anche \emph{Cespuglio Puzzola} per il suo pungente e caratteristico odore.

\textbf{Culcoa}: Un unguento rosso lucente usato principalmente per curare le ustioni da fuoco.

\textbf{Darsirion}: Una pozione leggera, fatta con i fiori argentei risplendenti alla luna. Utilizzata dai guaritori per curare piccole ferite.

\textbf{Delrean Plus}: Una formula avanzata che mescola l'estratto di foglie di Delrean con cera d'api. È un repellente infallibile.

\textbf{Delrean}: Estratto dalle foglie di Delrean viene utilizzato come repellente per insetti fastidiosi durante il raccolto.

\textbf{Dita di Daraka}: le Dita di Daraka sono il frutto dell'albero di Daraka. Il baccello di forma allungata e nera ricorda le dita dell'antica dea dell'oscurità.

\textbf{Draaf}: Un preparato basato su antiche ricette alchemiche. La pozione è conosciuta per la sua capacità di rigenerazione rapida.

\textbf{Eldrin'tail}: Il tè preparato con questa erba argentata ha un sapore amaro, ma efficace per contrastare i veleni più forti.

\textbf{Estratto 100 erbe}: Una miscela complessa e poco conosciuta

\textbf{Estratto di Bacca Illa}: Le bacche usate in questa preparazione devono essere bruciate delicatamente, per preservare i nutrienti.

\textbf{Estratto di radice Gisenosa}: pianta tipo carciofo, estremamente spinosa. Tende a crescere circondata dal \emph{Tribulus terrestris} o \emph{baciapiedi}.

\textbf{Estratto radice Gisenosa}: La radice spinosa venga bollita a lungo per creare questo rimedio contro i malanni del freddo.

\textbf{Fegato di Toporagno Viola}: avvelenamento riconoscibile dai tipici occhi iniettati di sangue

\textbf{Fumi di Curna}: la Curna è l'inflorescenza del cardo comune.

\textbf{Garioe}: Un frutto raro delle colline sud-orientali. Viene macerato in acqua e alcool per creare questa pozione curativa potente.

\textbf{Gelfnus}: creato dai fiori che sbocciano solo nella notte della luna piena.

\textbf{Gusperboon}: una piccola margherita, all'apparenza.

\textbf{Gylvert}: Un composto derivato dalle alghe di mare profondo

\textbf{Harfindar}: Polvere polivalente, talvolta usata per scopi medici altre volte no.

\textbf{Harfy}: Una crema composta da muschi, radici e lumache.

\textbf{Jojopo}: Un elisir caldo, zuccherino leggermente alcolico.

\textbf{Kelventare}: Uno sciroppo puro estratto dalla linfa dell’Albero della Vita.

\textbf{Klagul}: Una pasta abrasiva di antica ricette goblin. Usata come dentifricio in molte tribù selvagge.

\textbf{Klandor}: Polvere piccante e salata.

\textbf{Klynkyx}: un intruglio di miele e sterco di toporagno viola.

\textbf{Lingua Rossa di Xabax}: è il petalo lungo della Xabax. Dei 7 petali solo quello lungo ha i sostanze necessarie a preparare l'unguento.

\textbf{Melandrir}: Prodotto con una rara orchidea

\textbf{Mirenna}: Estratto da piante officinali.

\textbf{Miscela 31}: un insieme studiato di droghe per i saurovalli. Terminato l'effetto la creatura deve fare un Tiro Salvezza su Tempra DC 23 o cadere svenuto per 12 ore.

\textbf{Muschio Argentato}: molto simile, per un non esperto, al Muschio Bianco. Si
 raccolgono le bacche.
 
\textbf{Nazamuse}: Creato con tecniche avanzate dai Devoti di Atherim.

\textbf{Nelthalion}: L'estratto rosso-brunastro induce vomito.

\textbf{Olio di Nabar}: le piccole bacche di Nabar sono esclusivamente mangiate dai Toporagni, immuni ai loro malefici effetti. Bollito a lungo diviene un ottimo unguento per la pelle.

\textbf{Petali di Lisbeth}: Bellissimi e nerissimi.

\textbf{Polline di Rosa Verde}: Solo dalle Rose verdi più rare.

\textbf{Radice secca di Kathaus}: piccolo tubero nero estremamente duro e legnoso. Solitamente si lascia seccare al sole prima di macinarla.

%\textbf{Rewky}: Estratto alcolico blu traslucido.

\textbf{Silea}: Un balsamo verde pastoso.

\textbf{Toporagno Viola}: secondo molti il Toporagno è l'animaletto preferito di Cattalm. Aggressivo, violento, pericoloso in ogni sua fibra.

\textbf{Uovo di Urk}: Urk è un grosso coleottero, l'uovo è poco più grande di una nocciola. Solitamente viene prima affumicato con legno di faggio, mangiato crudo il sapore è di muffa e terra.

\textbf{Uscaboo}: Distillato dalle foglie di un tubero rosa

\textbf{Veleno di Ottalm}: l'Ottalm è una variante di Toporagno viola dotato di un pungiglione velenoso.

\textbf{Wickalim}: Usato spesso per piccole ferite, lasciano un sapore salato in bocca.

\textbf{Yajeth}: Estratto trito del fiore.


\end{multicols}

\textbf{Tabella: della corrispondenza Pozioni - Luoghi}\index[Tabelle]{Tabella della corrispondenza Pozioni - Luoghi}

\medskip

\noindent\begin{tabular}{ll|ll|ll}
\textbf{1' lett.} & \textbf{Clima} &\textbf{2' lett.} & \textbf{Ambiente} & \textbf{2' lett.} & \textbf{Ambiente}\\
\toprule
A & Arido& A & Alpino & B & Gole\\
C & Freddo & C & Foresta di Conifere& D & Foresta Decidua\\
E & Ghiacci perenni & F & Argini fiumi e torrenti & G& Campi ghiacciati\\
F & Freddo severo& H & Campi secchi &J & Giungla, Foreste piovose\\
H & Umido e caldo & M & Montagna & N & Oceano, distese salate\\
M & Temperato& S & Erba bassa & T & Erba alta\\
S & Semi arido & U & Caverne e sotterranei & V & Vulcanica\\
T & Temperato fresco & W & Discariche / Rifiuti & Z & Deserto\\
X & Sconosciuto& X & Sconosciuto&&\\
\end{tabular}

\begin{multicols}{2}

\bigskip

\subsection{Pozioni generiche}\index{Pozioni generiche}\index{Pozioni}

Il Narratore è libero di usare tutte le pozioni e veleni già elencate oppure usare delle pozioni generiche pronte all'uso. Nella tabella sono indicati i costi ed effetti di queste pozioni generiche.

L'insorgenza è sempre immediata, la durata per le cure è immediata, per le altre è 10 minuti (quindi la pozione Rimuovi Veleno ti protegge per 1 Turno contro un veleno). 

\end{multicols}

\begin{center}
\includegraphics[width=0.23\linewidth]{immagini/mandragola2.png}

\emph{Pianta di Mandragola}
\end{center}



\textbf{Tabella: delle pozioni generiche. Costo in Monete d'Oro.}\index[Tabelle]{Tabella delle pozioni generiche}\label{pozionigeneriche}\hypertarget{pozionigeneriche}{}

\medskip

\noindent\begin{tabularx}{1\linewidth}{lXcc}
\textbf{Nome Pozione}& \textbf{Effetto}&\textbf{Costo}& \textbf{Appl.}\\
\toprule
Cura& cura 1d8+1 Punti Ferita & 50 & I\\
\toprule
Cura potenziata& cura 3d8+3 Punti Ferita & 125& I\\
\toprule
Cura maggiore& cura 3d10+15 Punti Ferita & 300& I\\
\toprule
Indebolente& -2 TC. TS DC 15 Tempra per annullare gli effetti& 35 & F\\
\toprule
Indebolente potenziata& -1d6 TC. TS DC 18 Tempra per annullare gli effetti& 50 & F/I \\
\toprule
Veleno&2d6+2 di danno. TS DC 15 Tempra per dimezzare& 30 & I/F \\
\toprule
Veleno potenziata& 2d8+4 di danno. TS DC 18 Tempra per dimezzare & 50 & F \\
\toprule
Veleno maggiore& 4d10+8 di danno. TS DC 24 Tempra per dimezzare& 125 & F \\
\toprule
Rimuovi Veleno& concede un nuovo TS con +1d6 & 75 & I\\
\toprule
Pozione Generica* & vedi \hyperlink{crearepozioni}{Creare Pozioni} (pag. \pageref{crearepozioni}) &Lv*Lv*50&I

\end{tabularx}



\subsection{Opzionale - Droghe}\index{Droga}\index{Opzionale - Droghe}\hypertarget{droghe}{}\label{droghe}



\textbf{Tabella: Elenco Droghe}\index[Tabelle]{Tabella Elenco Droghe}

\medskip

\noindent\begin{tabularx}{1\textwidth}{lcccXcc}
\textbf{Nome}& \textbf{Uso} & \textbf{Ins.} & \textbf{DC} & \textbf{Effetto}& \textbf{Loc.} & \textbf{Costo} \\
\toprule
Foglie fermentate di Luside\index{Foglie fermentate di Luside}& I& 1 Turno& 17&+4 Carisma ed Intelligenza per 1d4 ore& SF7 & 5 mo\\
\toprule
Ferpillon \index{Ferpillon}& I& 1 round & 20& Fa dormire per 24 ore& SC5 & 50 mo \\
\toprule
Unto Grigio \index{Unto Grigio} & I& 1 round & 24& Rimuove condizionamenti mentali causati da incantesimi di livello inferiore al 5& AH9 & 80 mo\\
\toprule
Cenere di Arpasur \index{Cenere di Arpasur}& R& 1 round & 20& Rimuove 2 livelli di affaticato& FT6 & 10 mo\\
\toprule
%Carne secca di Toporagno Viola \index{Carne secca di Ragno Viola} & I& 1 round & 24& +4 Forza -4 Intelligenza (minimo -3) per 1 turno& SH7 & 30 \\ 
%\toprule
Estratto alcolico di Melzaa\index{Estratto alcolico di Melzaa}& I& 1 round & 20& +2 For, +2 Des, -2 Sag. Per 1d4 ore & AF6 & 25 mo \\
\toprule
Essenza profumata di Inut\index{Essenza profumata do Inut} & R& I& 15& +4 Destrezza per 1d8 ore& HB6 & 15 mo \\
\toprule
Polline di Julnnaus\index{Polline di Julnnaus} & R& I& 20& +3 Costituzione per 2 ore & FO6 & 25 mo\\
\toprule
Polline del fiore di Erain \index{Polline del fiore di Erain} & R& 1 round& 20& +2 For Int Des, +3d6 PF temporanei per 1 ora& FT7 & 75 mo\\
\end{tabularx}

\begin{multicols}{2}

\medskip

\textbf{L'utilizzo delle droghe è completamente opzionale. E' il Narratore a decidere la loro presenza e disponibilità anche in base alla sensibilità dei giocatori}.

Le droghe danno dipendenza. Terminato l'effetto è necessario effettuare un Tiro Salvezza su Volontà a difficoltà 15 o prenderne un altra dose, il successivo Tiro Salvezza avrà difficoltà +1 e così via.

Se il Tiro Salvezza riesce è comunque necessario effettuarne uno nuovo il giorno dopo con le stesse conseguenze.

Non prendere una dose aumenta il livello di Affaticamento di uno. Sono necessari 7 Tiri Salvezza riusciti di seguito per terminare l'effetto di dipendenza.

La DC indicata è per resistere agli effetti.

\subsection{Opzionale - Bere troppo}\index{Bere troppo}\index{Opzionale - Bere troppo}\hypertarget{alcolismo}{}\label{beretroppo}\index{Ubriachi}

Una creatura può bere un numero di boccali di birra o bicchierini di liquore pari al suo punteggio di Costituzione, con un minimo di 1. Ulteriori boccali costringono la creatura a fare un Tiro Salvezza su Tempra a DC 11, ogni successivo boccale aumenta il Tiro Salvezza di +2. Quando il Tiro Salvezza fallisce la creatura è ubriaca e si considera sotto l'effetto dell'incantesimo Confusione per un numero di Turni pari al modificatore aggiuntivo alla difficoltà.

Birre o liquori a maggiore gradazione impongono un modificatore più alto al Tiro Salvezza.

Il Narratore può decidere di gestire una creatura \emph{alticcia} solo attraverso il \emph{roleplaying}.

\subsection{Malattie}\index{Malattie}\hypertarget{malattie}{}\label{malattie}

Le malattie in linea di principio si gestiscono eseguendo un Tiro Salvezza per verificare se ci si è contagiati ed altri Tiri Salvezza per guarire.
Solitamente il tempo di scatenamento di una malattia non è così immediato come un veleno eppure quelle magiche possono essere dirompenti ed agire in pochi minuti.

Ogni malattia ha indicato il tempo di insorgenza, il Tiro Salvezza iniziale, ogni quanto va rifatto il Tiro Salvezza, quanti successi ai Tiri Salvezza sono necessari per guarire, gli effetti che si subiscono. Se non specificato diversamente le malattie si trasferiscono tramite il ferimento di un infetto.

Es. \emph{Febbre Demoniaca minore}: 1 minuto, TS Tempra DC 18, 6 ore, 3 successi, -1 Costituzione e Saggezza

La Febbre Demoniaca minore costringe al Tiro Salvezza su Tempra a DC 18 dopo un solo minuto che la si è presa. Successivamente ogni 6 ore va rifatto il Tiro Salvezza e la malattia permane finché non si sono fatti almeno 3 successi consecutivi al TS. Ogni 6 ore il malato perde 1 punto di Costituzione e Saggezza.

Per guarire da una malattia, non naturale, come quelle afflitte dai mostri è necessario superare i Tiri Salvezza richiesti oppure usare l'incantesimo di \hyperlink{rimuovimalattie}{Rimuovi Malattie} (pag. \pageref{rimuovimalattie}).

Si effettua una prova di \hyperlink{contrastareincantesimi}{contrasto} (pag. \pageref{contrastareincantesimi}) tra l'incantesimo Rimuovi Malattie e la DC della malattia.

Se la DC della malattia non è espressa allora si considera che sia sufficiente il semplice lancio dell'incantesimo per annullarne gli effetti.

Una prova di \textbf{Pronto Soccorso}\index{Pronto Soccorso e Malattie}\index{Malattie e pronto soccorso}, 10 minuti, con DC pari almeno alla metà della DC della malattia (o 15 se non indicata), effettuata tra un Tiro Salvezza e successivo, permette di avere un +2 al Tiro Salvezza per resistere agli effetti della malattia. Un trattamento per una intera notte concede +1d6 al Tiro Salvezza successivo.

Essere colpiti più volte dalla stessa malattia non ne aumenta la difficoltà di guarigione ne cambia i tempi ed effetti della stessa.

Esempi di Malattie:

\textbf{Influenza Demoniaca}: 1 minuto, TS Tempra DC 16, 12 ore, 2 successi, -1 Costituzione\index{Influenza Demoniaca}

\textbf{Corruzione di Rezh}: 1 giorno, TS Volontà DC 18, 1 ora, 2 successi, -1d6 Punti Ferita Massimi\index{Corruzione di Rezh}

\textbf{Maledizione di Efrem}\index{Maledizione di Efrem}: 8 ore, TS Tempra DC 24, 12 ore, 2 successi, -1 punto a Destrezza e +1 Difesa

\textbf{Torpore Violento}\index{Torpore Violento}: 24 ore, TS Volontà DC 12, 12 ore, 1 successo, +1 al Danno con Armi da Mischia e -1 Saggezza

\textbf{Febbre Demoniaca minore}\index{Febbre Demoniaca minore}: 1 minuto, TS Tempra DC 18, 6 ore, 3 successi, -1 Costituzione e Saggezza

\textbf{Sangue Nero}\index{Sangue Nero}: 10 minuti, TS Tempra 28, 12 ore, 1 successo, perdita metà Punti Ferita rimasti

\textbf{Peste T}\index{Peste T}: 1 minuto, TS Tempra 30, 2 ore, 3 successi, esegui 3 successi consecutivi altrimenti vieni trasformato in uno zombi di pari GS. Si trasmette attraverso il contatto.

\end{multicols}

\vfill




\begin{center}
\includegraphics[width=0.6\linewidth]{immagini/plaguedoctor.png}

\emph{Engraving of the Plague Doctor, Paul Furst, 1656}
\end{center}


\pagebreak

