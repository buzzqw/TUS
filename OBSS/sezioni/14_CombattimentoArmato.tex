\section{Combattimento Armato}\label{combattimento-armato}\index{Combattimento Armato}

\begin{enfasi}{
Si vis pacem, para bellum (\emph{Se vuoi la pace, prepara la guerra}, Vegezio, libro III, Epitoma rei militaris)
\smallskip

Non conta come cadi, ma se e come ti rialzi (anonimo)

\smallskip

Non sono un eroe. No e non lo sarò mai. Sono solo un cattivo che viene pagato per pestare tipi peggiori di lui. (Deadpool)

\smallskip

Occhio per occhio... e il mondo diventa cieco (Mahatma Gandhi, NdA i suoi Tratti aborrivano la violenza!)}\end{enfasi}

\begin{multicols}{2}

Il combattimento è tra le fasi principali di un avventura ed è quando i personaggi cercano, con risultati alterni, di dare sfoggio della loro maestria con le armi o magie.

Il combattimento è diviso in 2 fasi:\index{Combattimento}
\begin{itemize}
\item verifica dell'iniziativa
\item risoluzione delle azioni (movimento, attacco, azioni varie..)
\end{itemize}

\begin{center}
\includegraphics[width=0.65\linewidth]{immagini/Achildbookofwarriors.png}

\emph{A child's book of warriors (1907), William Canton}
\end{center}

\subsection{L'Iniziativa}\index{Iniziativa}\label{iniziativa}

L'iniziativa è una prova (3d6) di Destrezza o Intelligenza ed Abilità inerenti che potete avere.

Il giocatore sceglie la Caratteristica che preferisce. Se viene scelta la Destrezza saranno i riflessi a determinare la reazione del personaggio, mentre l'Intelligenza guiderà la capacità di cogliere le tattiche dell'avversario ed anticiparle.

Chi ha l'iniziativa tra giocatori e nemici più alta incomincia per primo e successivamente agiscono gli altri in ordine decrescente, dichiarando le Azioni ed eseguendole. In caso di Iniziativa di pari punteggio agisce per primo chi ha il punteggio Caratteristica più alto, altrimenti lo scontro sarà in contemporanea. L'iniziativa vale per l'intero scontro e si ritira al cambio dell'avversario.\index{Iniziativa uguale}

\begin{narratore}[Gestire il combattimento] %box narratore
Cercate di fare fluire il combattimento in maniera naturale. Non interrompete il flusso delle azioni, bensì descrivendone gli effetti coinvolgete i giocatori (e nemici) nelle azioni seguenti. Vi consiglio la lettura dell'articolo \href{https://theangrygm.com/manage-combat-like-a-dolphin/}{How to Manage Combat Like a Dolphin} per capire nel dettaglio il metodo.
\end{narratore}

\textbf{Anche nella Prova di Iniziativa valgono le Golden Rules.}\index{Iniziativa e Golden Rules}

\subsubsection{Risoluzione delle Azioni}\index{Risoluzione delle Azioni}\label{risoluzionedelleazioni}

\begin{enfasi}{
Non è vero che abbiamo poco tempo: la verità è che ne perdiamo molto. (Lucio Anneo Seneca)
}
\end{enfasi}

Dal più veloce al più lento c'è la risoluzione delle Azioni.

Il Narratore chiederà al più veloce, quello con l'iniziativa più alta, di dichiarare le sue Azioni ed agire, proseguirà poi a chiedere e fare agire gli altri giocatori e nemici.

In questo modo la scelta dell'azione avviene quando è il round del giocatore che potrà agire anche in base alle Azioni e risoluzioni già avvenute.

%\begin{center}
%\includegraphics[width=0.9\linewidth]{immagini/Arthur-Pyle_Two_Knights.png}
%\emph{Howard Pyle, from the 1903 edition of The Story of King Arthur and His Knights}
%\end{center}

\subsubsection{Il Tempo (Round, Minuti e Turni)}\index{Round}\label{iltempo}

\begin{enfasi}{L'esitazione è la morte del vantaggio. Magic di V.E. Schwab} \end{enfasi}

Un \textbf{round} dura 10 secondi circa, è un lasso di tempo sufficiente per agire, correre, parlare.. combattere. Un Minuto sono quindi 6 round ed un Turno dura 10 Minuti (o 60 round).\index{Round e Turno, durata}

I round si usano nelle scene di combattimento o dove la tensione deve rimanere costantemente alta ed ad ogni Azione corrisponde un evolversi della situazione.

\subsubsection{Tempo di riattivazione Oggetti ed Abilita'}\index{Tempo di attivazione Oggetti ed Abilità}\label{temporiattivazioneoggetti}\index{Ricarica oggetti magici}

Se non specificato diversamente un oggetto o Abilità che prevede un certo numero di usi al giorno \emph{es. una volta al giorno} si ricarica all'alba successiva l'uso.

%\begin{center}
%\includegraphics[width=0.6\linewidth]{immagini/hjford-fight.png}

%\emph{\\Fairy book - Fairytale illustration, Henry Justice Ford}
%\end{center}

\subsection{Azioni nel Round}\index{Azioni nel Round}\index{Azione}\label{azioninelround}

%\begin{enfasi}{
%Un vero uomo d'azione vede subito dinanzi a sé tante cose da fare che il %lavoro non gli mancherà mai e riuscirà. (Fëdor Dostoevskij)
%} \end{enfasi}\end{changemargin}\medskip

Un personaggio può eseguire fino a 3 Azioni, 1 Azione Immediata, ed 1 Azione di Reazione per Round. Può anche usare 1 o più Azioni Gratuite se a disposizione.

Se l'iniziativa tirata è la più veloce ed è di +8 o più punti superiore alla seconda allora in quel round potrà usare una Azione di Reazione od Immediata in più, se il differenziale è di almeno +16 la sua grande reattività gli permette di eseguire una Azione in più.\index{Critico nell'Iniziativa}

Le Azioni possono essere eseguite nell'ordine preferito.

Nella tabella sottostante sono indicate le Azioni principali che un personaggio può fare, sono linee guida da seguire. Nel capitolo dedicato al combattimento e degli esempi d'uso delle competenze vengono elencate altre Azioni ed i loro costi relativi in Azioni.

\textbf{Una Azione non può essere interrotta da un altra Azione, ma può essere seguita da una Azione di Reazione o da una Azione Immediata}. \index{Interrompere Azioni}\index{Azioni, Interrompere}
Se un personaggio vuole fare più attacchi spostandosi nel campo di battaglia può usare una Azione per eseguire un attacco, usare una Azione di Movimento per spostarsi fino a tutto il suo movimento a disposizione, ed usare un ultima Azione di attacco per eseguire un ultimo singolo attacco, questo secondo attacco conta come attacco multiplo con le relative penalità.

E' possibile \textbf{ritardare} una o più Azioni\index{Ritardare Azioni} per aspettare lo svolgersi delle scene. Il personaggio che ritarda una sua Azione agisce per primo tra i soggetti che agiscono in quel valore di iniziativa, nei successivi round continuerà ad agire nel nuovo ordine di iniziativa. In questa maniera il giocatore ritarda volontariamente la sua iniziativa per inserirsi nell'ordine delle iniziative in un altro posto.

Un giocatore che dichiara di aspettare una certa situazione per poter agire equivale ad eseguire una o più \textbf{Azioni Preparate}\index{Azioni Preparate}. In questo caso il personaggio (o nemico) agisce \textbf{dopo} l'Azione scatenante con le sue Azioni ma rimane nel suo ordine di iniziativa al termine del round.

Se il personaggio ha già compiuto tutte le Azioni allora potrà agire nel round solo con un \textbf{Azione Immediata} e fuori dalla sua iniziativa solo tramite una Reazione, se a disposizione. L'\textbf{Azione di Reazione} si attiva sempre dopo l'Azione scatenante.
La \textbf{Azioni Gratuite} possono essere usate in qualsiasi momento.

\textbf{Azione di Attacco}: si intende sia l'uso di armi in mischia che l'uso di armi da lancio o tiro come archi, balestre o pugnali da lancio. Nel caso di armi da lancio ogni lancio/tiro conta come un attacco.

Il personaggio che esegue una Azione di Attacco ed il Lancio di un Incantesimo nel medesimo round si considera Distratto ovvero deve eseguire una Prova di Magia per lanciare l'incantesimo.\index{Attacco ed Incantesimo}

\textbf{Azione di Movimento*}: un Azione di Movimento è una Azione dedicata a spostarsi. Ci si può spostare fino a tutto il proprio movimento (9 metri per umani, 6 metri per nani..) per Azione usata. Ogni movimento consuma una Azione anche se non si sfrutta tutto il proprio movimento a disposizione.\index{Azione di Movimento}

Durante l'Azione di Movimento è possibile \textbf{Estrarre l'Arma} o Scudo o \textbf{Rinfoderare l'Arma} o lo Scudo.

\medskip

\textbf{Tabella: Azioni per Round}\index[Tabelle]{Tabella delle Azioni per Round}

\medskip

%\noindent\begin{tabularx}{1\linewidth}{lc}
\noindent\begin{tabular}{lc}
	\toprule
\rowcolor{gray!20}\textbf{Cosa si fa} & \textbf{Azioni}\\
\toprule
\hyperlink{tiropercolpireedifesa}{Eseguire un attacco}& 1\\
\rowcolor{gray!20}Eseguire due attacchi& 2\\
\hyperlink{attacchimultiplimischia}{Eseguire più di due attacchi}& 3\\
\rowcolor{gray!20}Estrarre o Rinfoderare l'arma o scudo& 1\\
\midrule
\hyperlink{tipodimovimento}{Eseguire una Azione di Movimento} &1*\\
\rowcolor{gray!20}\hyperlink{azionediscatto}{Scatto} & 1\\
\hyperlink{alzarsidaprono}{Alzarsi da prono}& 1\\
\midrule
\rowcolor{gray!20}\hyperlink{aiutare}{Aiutare qualcuno}& R\\
\hyperlink{esempiprovecompetenze}{Eseguire prova su una competenza}& 1*\\
\rowcolor{gray!20}\hyperlink{riconosceremostro}{Riconoscere una creatura}& 1\\
\hyperlink{copertura}{Nascondersi}& 1\\
\midrule
\rowcolor{gray!20}\hyperlink{cavalcare}{Salire o scendere dalla cavalcatura}& 2\\
\hyperlink{sfondare}{Sfondare una porta a spallate/calci}& 1\\
\rowcolor{gray!20}\hyperlink{piedediporco}{Forzare porta con piede di porco}& 2\\
\midrule
Cercare qualcosa nello zaino& 2\\
\rowcolor{gray!20}\resizedown{\linewidth}{Prendere qualcosa dalla cintura o di pronto} & 1\\
Usare un oggetto tenuto in mano& 1\\
\midrule
\rowcolor{gray!20}\hyperlink{insorgenzaveleno}{Bere una pozione tenuta in mano}& Imm.\\
\hyperlink{insorgenzaveleno}{Fare bere una pozione ad un altro} & 2\\
\midrule
\rowcolor{gray!20}Gettare un oggetto tenuto in mano& R\\
Gettarsi a terra prono& R\\
\midrule
\rowcolor{gray!20}\hyperlink{magietempodilancio}{Lanciare un Incantesimo}*& 2\\
\hyperlink{magieconcentrazione}{Concentrarsi su un Incantesimo}& 1\\
\rowcolor{gray!20}\hyperlink{magiedurata}{Interrompere un proprio incantesimo} & Imm.\\
\hyperlink{riconoscereincantesimo}{Riconoscere un Incantesimo}& R\\
\rowcolor{gray!20}\hyperlink{regoleoggettimagici}{Usare un oggetto magico}& 2\\
\midrule
Scambiare un dialogo con qualcuno& 3*\\
\rowcolor{gray!20}Scambiare poche battute con qualcuno& 0*\\
\midrule
\hyperlink{preparareladifesa}{Preparare la Difesa} & 1\\
\rowcolor{gray!20}\hyperlink{difesatotale}{Difesa Totale} & 2\\
\hyperlink{disingaggiare}{Disingaggiare} & 1\\
\rowcolor{gray!20}\hyperlink{colpopreciso}{Colpo preciso} & 2\\
\midrule
\hyperlink{disarmare}{Disarmare} & 2\\
\rowcolor{gray!20}\hyperlink{finta}{Finta} & 1\\
\hyperlink{spingereavversario}{Spingere un avversario} & 2\\
\rowcolor{gray!20}\hyperlink{afferrareunavversario}{Afferrare l'avversario} & 2\\
\hyperlink{farecadereavversario}{Fare cadere l'avversario} & 2
\end{tabular}

\medskip

\textbf{Lanciare un Incantesimo*}: solitamente sono necessarie 2 Azioni. Nella descrizione dell'incantesimo è indicato il numero di Azioni necessarie. Nel capitolo della Magia sono specificate le \hyperlink{piumagieround}{regole} (pag. \pageref{piumagieround}) per lanciare più incantesimi nel round.

\textbf{Scambiare un dialogo con qualcuno*}: Un dialogo può essere di pochi secondi se non di minuti. Il Narratore valuterà quanto questo dura.

\textbf{Scambiare poche battute con qualcuno*}: Finché sono veramente poche battute o uno sguardo non consuma Azioni, se questo diventa più articolato allora utilizza delle Azioni. L'obiettivo è non interrompere il flusso delle Azioni con un fitto dialogo ma comunque permettere l'interazione tra i personaggi.

\textbf{Eseguire prova su una competenza*}: se sfruttano una frazione del round costano 1 Azione, altrimenti 2 o più. Controllate negli \hyperlink{esempiprovecompetenze}{Esempi Prove Competenza} i costi riportati.

Una Azione di \textbf{Reazione (R)} \index{Azione di Reazione}può essere eseguita liberamente anche fuori dal proprio round. Questa Azione è solitamente dovuta ad Abilità o situazioni particolari. Se non indicato diversamente una Azione di Reazione accade immediatamente dopo la causa che la scatena.

Una Azione \textbf{Immediata (Imm.)} \index{Azione Immediata}può essere eseguita liberamente nel proprio round, primo o dopo la propria Azione. Una Azione Immediata è solitamente concessa da particolari Abilità.

E' possibile, se non descritto specificatamente nell'Abilità, eseguire solo una Azione Immediata ed una Azione di Reazione per round.

\smallskip

Questo \textbf{elenco non è completo}, prendetelo come linee guida per stabilire il peso delle decisioni ed azioni dei personaggi. Una Azione dura circa 3 secondi.

L'\textbf{ordine} con cui si eseguono le Azioni non è importante se non per correlazione logica e fisica. L'Azione di Movimento può essere tra altre Azioni (movimento, attacco/incantesimi/altra Azione, movimento).

Un personaggio potrebbe attaccare, muoversi ed ancora attaccare, questo secondo attacco avrebbe le penalità descritte negli attacchi multipli.

\end{multicols}

\vfill


\begin{center}

\includegraphics[width=0.8\linewidth]{immagini/Perseus_Fighting_Phineus_and_his_Companions.png}

	\emph{Luca Giordano: Perseus turning Phineas and his Followers to Stone}

\end{center}

\bigskip

\begin{enfasi}
I tesori non si vincono con cautela e previdenza, ma con uccisione rapida e attacco sconsiderato. (Michael Moorcock)
\end{enfasi}


\pagebreak

\subsection{Movimento}\index{Movimento}\label{movimento}

\begin{enfasi}{Un mobile più lento non può essere raggiunto da uno più rapido; giacché quello che segue deve arrivare al punto che occupava quello che è seguito e dove questo non è più (quando il secondo arriva); in tal modo il primo conserva sempre un vantaggio sul secondo. (Paradosso di Zenone)}
\end{enfasi}

\begin{multicols}{2}

Il movimento di un personaggio è dato dalla sua taglia e razza e da ciò che porta, dai pesi, ingombri ma anche magie ed oggetti magici.

Il Movimento scritto nella razza del personaggio è l'indicazione di quanti metri per Azione (di Movimento) il personaggio può fare.

Una creatura o personaggio potrebbe anche decidere di spostarsi più velocemente del solito ovvero correndo (Azione di Scatto).\label{azionediscatto}\hypertarget{azionediscatto}{}

L'Azione di Scatto è una Azione di Movimento particolare, consiste nel correre per quell'Azione.
Se si esegue un'Azione di \textbf{Scatto} \index{Scatto}si raddoppiano i metri percorsi (2x9 metri per un umano), per un nano (Movimento 6m) significa fare 12 metri, in una Azione.
E' anche possibile fare più Azioni di Scatto, fino a 3 in un round, ovvero correre per 6 volte il proprio movimento.

Il personaggio che fa una Azione di Scatto \index{Azione di Scatto}corre ed ha una penalità di 1d6 nel Tiro per Colpire, la Difesa diminuisce di 4 fino all'inizio del suo round successivo e si considera Distratto per il lancio di incantesimi.\index{Penalita' per correre}

Non è possibile spostarsi anche solo di 1 metro se non si spendono Azioni di Movimento.

Queste precisazioni hanno senso e vanno usate quando si tratta di combattere ed il dislocamento sul territorio, sulla mappa, è fondamentale. Durante gli spostamenti normali, mentre si cavalca o cammina liberi senza pericoli si usa la normale gestione del movimento orario.

Quando si parla di \textbf{\emph{quadretto}} \index{Quadretto}per indicare una distanza od una influenza si intende un quadretto di mappa di 1 metro x 1 metro.

%Nel caso di spostamento diagonale\index{Movimento diagonale}\index{Spostarsi di lato} si conta una distanza di 1,5 metri per quadretto, in caso di arrotondamenti sull'ultimo quadretto si fa per difetto, ovvero si torna indietro all'ultimo attraversato.

Nel caso di \textbf{spostamento diagonale}\index{Movimento diagonale}\index{Spostarsi di lato} per praticità contate un quadretto normalmente.

\textbf{Se ci si sposta su un terreno \emph{difficile}, si percorre la metà del movimento disponibile quindi un umano copre 4 metri per Azione di Movimento (ogni quadretto attraversato conta per due).}

Nel Mostruario sono indicate le dimensioni e relativi spazi occupati dalle creature di \hyperlink{tagliaedimensioni}{taglia} diversa (pag \pageref{tagliaedimensioni}).

\subsection{Distanza}\index{Distanza}\label{distanza}

Per \textbf{distanza di Tocco} \index{Distanza di Tocco} \index{Tocco}si intende una distanza che permette il toccare l'avversario, quindi non più di un metro per creature di taglia media senza armi lunghe o con portata. La distanza di tocco è distanza di mischia qualora non si usino armi lunghe.

Per \textbf{distanza di Mischia} \index{Distanza di Mischia} \index{Mischia}si intende una distanza che permette il combattimento corpo a corpo data la lunghezza della propria arma. L'area che può essere colpita attorno al personaggio viene chiamata anche \textbf{area di minaccia}\index{Area di minaccia} (solitamente entro 1 metro attorno al personaggio, oppure entro 2 metri in caso di arma lunga).

Nei mostri questa distanza è indicata dalla portata, per le armi da lancio è chiamata gittata.

Se non indicata nella scheda del avversario la \textbf{portata} è pari a metà dello spazio occupato arrotondato per eccesso. Un gigante delle colline, taglia enorme (3x3 quadretti sulla mappa), ha portata 2 quadretti, ovvero colpisce creature entro 2 quadretti/metri da lui.\index{Taglia e distanza di mischia}\index{Portata}

\begin{narratore}[Distanza in Combattimento]
Es. per una creatura armata di lancia la portata è 2 ovvero la distanza di mischia è 2 metri perché l'arma è lunga. Per uno gnomo armato di martello, o a mani nude, la distanza di mischia è 1 metro. Per creature particolarmente grandi (Enormi o più) con armi altrettanto grandi la portata viene indicata o si desume dalle dimensioni del mostro e dalla tipologia di arma.

\textbf{La portata indica fino a che distanza puoi colpire in mischia.}
\end{narratore}

\end{multicols}

\vfill

\begin{center}
	\includegraphics[width=0.5\textwidth]{immagini/camminata.png}
\end{center}

\pagebreak

\subsection{Vita e Morte}\index{Morire}\label{morire}

\begin{enfasi}{Chi non conosce la morte, non conosce la vita. (Grand Hotel, film 1932)

\medskip

Il meritevole Game Master non uccide mai volontariamente i personaggi dei giocatori. Lui presenta le opportunità per i giocatori frettolosi e sbadati di fare tutto da soli. (Gary Gygax)}\end{enfasi}

\begin{multicols}{2}

Il danno da arma si calcola come somma del dado dell'arma, Forza (o Destrezza se indicato da Abilità) che sia positiva o negativa, bonus dati da Lista d'Armi, bonus dati Abilità, bonus dati dall'arma e bonus circostanziali.\index{Come calcolare il danno dell'arma}\index{Danno dell'arma}

Quando una creatura raggiunge i 0 (zero) Punti Ferita si considera svenuto\index{Svenuto}, ovvero Indifeso ed Inabile a fare qualsiasi cosa. Una Cura magica (Incantesimo, Pozione..) lo porterà cosciente ed ai Punti Ferita curati. Una prova di \hyperlink{prontosoccorso}{Pronto Soccorso} (pag. \pageref{prontosoccorso}) (DC 12) potrà essere usata per riportarlo cosciente a 1 Punto Ferita.
Se lasciato svenuto dopo un ora se non è successo qualcosa a mutare la situazione il personaggio può fare un Tiro Salvezza su Tempra a DC 15, se riesce torna a 1 Punto Ferita, se fallisce va a -1 e diventa morente.\index{Zero PF, recupero da}\index{0 Punti Ferita}

Un personaggio morente ha Punti Ferita negativi (-1 o meno) ed è svenuto e \hyperlink{morente}{indifeso}. Continuerà a perdere 1 Punto Ferita a round fiche il valore non raggiungerà il doppio della Costituzione +10 ed il personaggio morirà, se non viene curato.

Una magia (incantesimo o pozione) di Cura, di qualsiasi ammontare di cura lo porterà a 1 Punto Ferita, successive cure funzioneranno normalmente.

Una prova di \hyperlink{prontosoccorso}{Pronto Soccorso}, 3 Azioni, a difficoltà 12 più il valore dei Punti Ferita negativi porterà il personaggio a 0 Punti Ferita, ovvero svenuto. Ogni volta successiva che il personaggio torna sotto 0 Punti Ferita la difficoltà della prova di Pronto Soccorso aumenta di 2.

\begin{giocatore}[Tups sta morendo]
Es. Tups é gravemente ferito ed ha attualmente -6 Punti Ferita, Jade decide di provare a curarlo (dopo averlo spostato in un posto più sicuro). Jade tenta una prova di Pronto Soccorso (3 Azioni) per stabilizzare il compagno, la sua difficoltà alla prova é 12+6 ovvero deve superare con Pronto Soccorso DC 18 per riportarlo a 0 Punti Ferita (svenuto)

Una successiva prova di Pronto Soccorso, svolta entro 10 minuti, potrà sanare ulteriori ferite.
\end{giocatore}

Un personaggio morente che subisce ulteriore danno, come nemici che infieriscono sul corpo od incantesimi diretti a lui od ad area, continua a sottrarre Punti Ferita con il rischio di morire.

Le \textbf{Condizioni} \index{Condizioni mentali}\textbf{di tipo mentale} quali Affascinato, Confuso ma non Dominato, terminano quando il personaggio diventa morente.

\begin{center}
	\includegraphics[width=0.8\linewidth]{immagini/Nuremberg_chronicles.png}

	\emph{The Dance of Death (1493) by Michael Wolgemut, Nuremberg Chronicle of Hartmann Schedel}
\end{center}

\medskip

Se un attacco o incantesimo porta il personaggio direttamente a -(10+COS*2), il personaggio muore\index{Morte immediata}\index{Danno Massivo} senza possibilità di essere curato.

Quando un personaggio torna a Punti Ferita positivi dopo che era andato negativo perde la metà dei Punti Magia rimanenti con una riduzione di almeno 10 Punti Magia e diviene ulteriormente \hyperlink{affaticato}{\textbf{Affaticato}} (pag. \pageref{affaticato}).

Quando un personaggio arriva a Punti Ferita negativi pari 10+doppio del suo punteggio di Costituzione é \hyperlink{morto}{\textbf{morto}} [-(10+(COS*2))].

Un personaggio con i Punti Ferita non Letali a 0 o meno sviene finché i Punti Ferita normali non sono tornati ad 1.

Es. Se ha Costituzione 2 morirà a -[10+4]=-14 Punti Ferita, se ha Costituzione 0 morirà a -10 Punti Ferita, se ha Costituzione -2 morirà a -[10-4]=-6 Punti Ferita. In caso di valori di Costituzione pari od inferiore a -3 il personaggio muore a -5 Punti Ferita.

Se il danno non letale di un personaggio arriva a Punti Ferita negativi pari 20+4*Costituzione il personaggio é morto.\hypertarget{puntiferitatemporaneimorte}{}

\begin{narratore}[Recitare]
Descrivete con pathos e trasporto la caduta del personaggio, fate capire la sofferenza provata. Enfatizzate la caduta a terra, il sangue che sgorga, i rantoli. Siate teatrali.
Se avete a che fare con giocatori facilmente impressionabili allora è meglio ridurre il \emph{gore}.
\end{narratore}

Un personaggio morto non può beneficiare delle cure normali o magiche, e non può essere riportato in vita da un incantesimo. Solo un Patrono ha sufficiente potere per riportare l'anima nel corpo e riportare in vita la creatura. L'incantesimo di \hyperlink{Animare Morti}{Animare Morti} può rianimare un corpo, ma come non morto.

\begin{giocatore}[La morte del personaggio]
Cerca di capire perché è morto, quali sono le cause, gli errori commessi. Quali sono le scelte che lo hanno portato fino a li. Ogni personaggio che muore è una ferita personale ma anche esperienza e consapevolezza. Fanne tesoro sia tu ma anche tutto il gruppo. Se qualcosa non ha funzionato cercate di capirlo insieme, senza accusarsi o darsi colpe ma con lo spirito consapevole che si può migliorare, tutti.
\end{giocatore}

\subsubsection{Opzionale - Recupero da 0 Punti Ferita} \index{Recupero} \index{Svenuto}\index{Opzionale - Recupero da 0 Punti Ferita}\label{recuperozeropf}

\begin{enfasi}{
Le notizie sulla mia morte sono molto esagerate. (Samuel Clemens)
}\end{enfasi}

\textbf{Nel caso vogliate un sistema meno letale potete applicare questa regola opzionale.}

Ogni round successivo ad essere andato a 0 Punti Ferita o meno, quindi svenuto o morente, il personaggio deve effettuare un Tiro Salvezza su Tempra a difficoltà 15, se riesce riprende coscienza e va ad 1 punto ferita.

Se fallisce il Tiro Salvezza può effettuarne un altro a DC +1 rispetto alla precedente il round successivo. Quanto la difficoltà raggiunge 18 (ovvero 3 prove fallite di seguito) il personaggio muore.

Appena la prova riesce (entro i 3 fallimenti) il personaggio torna ad 1 punto ferita ed è affaticato. Ogni volta che torna a meno di 0 Punti Ferita la difficoltà iniziale (15) aumenta di 1.

\subsubsection{Recupero punti Caratteristica}\index{Recupero punti caratteristica}\label{recuperopunticcaratteristica}

Eventuali punti Caratteristica persi si recuperano al ritmo di 1 punto al giorno, se non indicati come perdita permanente.

\subsubsection{Recupero Punti Ferita naturale}\index{Recupero Punti Ferita naturale}\label{recuperopuntiferitanaturale}

Per ogni notte di riposo (almeno 8 ore) recuperi in Punti Ferita il valore di Costituzione * CA o CM (a scelta del personaggio, con un minimo di PF pari a CA o CM).

\subsubsection{Recupero Punti Ferita non letali}\index{Recupero Punti Ferita non letali}\index{Punti Ferita non letali}\label{recuperopuntiferitanonletali}\hypertarget{recuperopuntiferitanonletali}{}

Ogni ora si recupera, con un minimo di 1 Punto Ferita, il proprio valore di Costituzione.

\subsubsection{Punti Ferita Massimi}\index{Recupero Punti Ferita massimi}\index{Punti Ferita massimi}\label{puntiferitamassimi}

Se non indicato diversamente ogni qual volta il personaggio subisce un danno che abbassa i Punti Ferita Massimi oltre ad abbassare questi deve anche sottrarli dai Punti Ferita attuali. Un personaggio quando curato non può superare i Punti Ferita Massimi attuali.

Ogni 8 ore di riposo, nelle 24 ore, si recupera 1d4 + Costituzione in Punti Ferita Massimi, con un minimo di 1.

\end{multicols}

\vfill

\begin{center}

%\includegraphics[width=0.7\linewidth]{immagini/caravaggioSalomeLondon.png}

%\emph{Salomè con la testa del Battista è un dipinto di Caravaggio realizzato in olio su tela (91x106 cm) tra il 1607 e il 1610.\\ È conservato nella National Gallery di Londra.}
\includegraphics[width=0.45\linewidth]{immagini/giantdeath.png}

\emph{Henry Justice Ford}

\end{center}

\pagebreak

\subsection{Tiro per Colpire e Difesa}\index{Tiro per Colpire}\index{Difesa}\label{tiropercolpireedifesa}\hypertarget{tiropercolpireedifesa}{}

\begin{enfasi}{Applica sempre la giusta forza, mai troppa mai troppo poca. (Kano Jigoro)}\end{enfasi}

\begin{multicols}{2}

Il \textbf{Tiro per Colpire} è dato dall'insieme delle capacità combattive (Competenza Armi e bonus concessi da Lista d'Armi), Forza, armi magiche e tutto ciò che influisce nel combattimento. Se l'\textbf{attaccante} porta l'attacco con:

\begin{itemize}[leftmargin=*] \setlength{\itemsep}{0pt}
\item \textbf{Armi da Mischia o Contatto}: l'attaccante deve effettuare un \textbf{Tiro per Colpire (TC)}= 3d6 + Competenza Armi + Forza + eventuali bonus dati dalla Lista d'Armi + Abilità + bonus magici dell'arma e fattori circostanziali (ambiente, maledizioni..)

\item
\textbf{Armi da Distanza}: l'attaccante deve effettuare un Tiro per Colpire (TC) = 3d6 + Competenza Armi + Destrezza + eventuali bonus dati dalla Lista d'Armi + Abilità + bonus magici dell'arma e fattori circostanziali (ambiente, maledizioni..). Vale per archi, balestre, pugnali tirati, giavellotti...

\item
\textbf{Incantesimo}: vedi Capitolo sulla \hyperlink{magietiropercolpireconlemagie}{Magia} (pag. \pageref{magietiropercolpireconlemagie})
\end{itemize}

Il giocatore può decidere di rinunciare a parte del bonus dato dalla Competenza Armi per avere un migliore punteggio di Difesa. Questi punti non saranno a disposizione nell'attacco successivo (vedi Altre azioni e situazioni).

%\medskip

%\begin{center}
%\includegraphics[width=0.9\linewidth]{immagini/3fightmen.png}

%\emph{Henry Thomas Alken - Three Men, Yale Center for British Art}
%\end{center}

%\medskip

Chi si \textbf{difende} ha una \textbf{Difesa} pari a: 10 + Destrezza + Scudo + Armatura + bonus magici + Abilità e bonus circostanziali (copertura ad esempio), per i mostri il valore della Difesa è già calcolato dei valori normali, sarà eventualmente da modificare dai valori pertinenti alla situazione.

Si intende Difesa naturale il punteggio di 10 alla base del calcolo della Difesa, alcune eccezionali abilità possono aumentare questo valore.\index{Difesa naturale}

\subsection{La Difesa e l'Attacco}\index{Difesa}\index{Attacco}\label{difesaeattacco}

\begin{enfasi}{La difesa è sempre legittima (anonima vittima)}\end{enfasi}

Ogni Tiro per Colpire si raffronta la Difesa.

Se il \textbf{Tiro per Colpire} è pari o superiore al valore della Difesa l'avversario è stato colpito e si stabilirà il danno della ferita, dato dal dado dell'arma + punteggio Forza ed altri fattori quali bonus magici, Lista d'Armi ed Abilità.

Se il Tiro per Colpire (TC) è più basso della Difesa allora l'avversario avrà parato, schivato, evitato.. La scelta la si lascia al giocatore (o Narratore), evitato l'attacco non si subiscono ferite.

Ci sono situazioni che possono avvantaggiare la Difesa quali coperture, nascondigli, trincee, porte, compagni di taglia molto più grande della propria, invisibilità... Consultate i paragrafi relativi ai \hyperlink{copertura}{Nascondigli e Coperture} per capire il vantaggio che possono dare.

Ci sono occasioni in cui non è importante penetrare la difesa e ferire l'avversario ma semplicemente basta toccarlo.

Altre volte l'avversario è sorpreso e non può difendersi completamente.

Se è \textbf{sufficiente toccare l'avversario} il Tiro per Colpire ha +1d6 di bonus dato che non è necessario portare il colpo quanto solo sfiorarlo.\index{Toccare l'avversario}. Nel manuale è chiamato Attacco a Tocco.\index{Attacco a tocco}\label{attaccoatocco}\hypertarget{attaccoatocco}{}\label{difesaatocco}

Se \textbf{l'avversario è sorpreso} ovvero non si aspetta l'attacco la Difesa ed il Tiro Salvezza su Riflessi avranno una penalità di -2. Questo è il valore della \textbf{Difesa di sorpresa}.\label{difesasorpresi}

\textbf{Anche per il Tiro per Colpire valgono le Golden Rules}. I d6 esplodono in caso si tiri 6 con il dado, fare 1 porta male (vale zero) ed affidarsi alla sorte (ovvero togliere 4 punti tra Competenza Armi e Forza o Destrezza per aggiungere 1d6 al Tiro per Colpire, non dai bonus dati da Lista d'Armi od Abilità od oggetti magici).

Se i modificatori e circostanze portano il danno inflitto ad essere zero o negativo comunque farai 1 di danno.
Questa regola si applica ai modificatori del danno dell'arma che appunto non possono portare il danno totale ad essere inferiore a 1, se ci sono protezioni magiche o riduzioni del danno questo può diventare zero e quindi non ferirai l'avversario (ma se diventa negativo non lo curi!).

Quando effettui il Tiro per Colpire assicurati di aver conteggiato tutti i modificatori a te noti e ricorda, che per ogni 6 tirato (nei 3d6 del Tiro per Colpire) devi tirarne un altro e continuare a tirare finché continui a fare 6 con il dado.

\begin{center}
	\includegraphics[width=0.6\linewidth]{immagini/esplosionedanno.png}

	\emph{Henry Justice Ford}
\end{center}

Puoi \textbf{togliere 4} o multipli al tuo attacco per tirare un d6 in più. La scelta è da fare nelle situazioni più disperate dove solo la fortuna può risolvere il duello. Il valore lo togli dal punteggio di Competenza Armi o di Forza o Destrezza non da punteggi dati da Abilità, Liste d'Armi o bonus magici.

Se colpisci, per ogni margine di 8 superiore alla Difesa dell'avversario , l'arma fa del danno in più ovvero un Tiro Critico. Tira nuovamente solo il dado del danno dell'arma, senza altri modificatori.

Anche per il Tiro per Colpire valgono le regole base delle Competenze. La Difesa è un valore fisso e come tale usa i modificatori per le prove a valore fisso.

\begin{narratore}[Partecipazione]
OBSS vuole essere divertente da giocare, vuole che i giocatori si divertano e vedano i risultati ottenuti dai dadi (e ovviamente dalle loro scelte). Le Golden Rules e l'Esplosione del Danno vogliono proprio togliere la patina di polvere ai dadi e fare divertire. Un giocatore, apprezzerà, ancor più se di esperienza, come i tiri dei dadi non siano solo un numero ma bensì aprano la possibilità di fare la differenza. Chiedete al giocatore di descrivere il colpo critico e fatelo recitare nella sua gloria di potenza!
\end{narratore}

\subsection{Tirare 3 volte 1}\index{Tirare 2 volte 1 oppure 2 volte 2 ed una volta 1}\label{tiraretrevolteuno}\index{Tirare 3 volte 1}

Se hai tirato tre volte 1 hai mancato, indipendentemente dal risultato finale. Il Narratore potrebbe anche decidere che succedono brutte cose... (ad esempio vedi \hyperlink{tabellafallimentiarmi}{Tabella Fallimento Tiri per Colpire}, pag. \pageref{tabellafallimentiarmi})

\begin{center}
	\includegraphics[width=0.9\linewidth]{immagini/critico.png}

	\emph{Henry Justice Ford}
\end{center}


\subsection{Tirare 3 volte 6}\index{Tirare 3 volte 6}\label{tiraretrevoltesei}

Se nei primi 3 Tiri per Colpire fai tre volte 6 prenderai l'avversario indipendentemente dal risultato finale del Tiro per Colpire. Oltre ad avere la certezza di aver fatto un Tiro Critico, il Narratore potrebbe decidere di applicare qualche effetto descrittivo (od effettivo) ulteriore. E ricordati di continuare a tirare quei magnifici dadi nella speranza di fare ancora 6!

\subsection{Tiro Critico}\index{Tiro Critico}\index{Danno critico}\label{tirocritico}

Ogni qual volta hai colpito, tiri un \textbf{danno aggiuntivo della sola arma} per ogni margine di 8 che hai superato la Difesa con il tuto Tiro per Colpire, questo danno viene anche chiamato \textbf{danno critico}. Se hai fatto due Tiri Critici vuole dire che devi tirare 2 dadi di arma in più e che hai colpito con un margine tra +16 e +23!

\begin{giocatore}[Esempio Tiro Critico]
Es tiro 6 4 5, tiro in aggiunta 6, tiro in aggiunta un 6, tiro in aggiunta 4, totale 31. La Difesa dell'avversario è 15. Come danno tiri 3 volte il danno dell'arma, una volta perché ho colpito ed due perché l'hai colpito con un margine di 16!.
\end{giocatore}


\subsection{Esplosione del Danno}\index{Esplosione del Danno}\label{esplosionedeldanno}

Ogni qual volta nel tiro del dado dell'arma ottieni il valore massimo (nel classico d8 per la spada lunga ad esempio fai 8 ed è quindi il valore massimo del dado), ritiri il dado e sommi ancora il valore (del solo dado).

In caso di armi con più dadi (esempio 2d4, il valore massimo deve essere ottenuto come somma dei due dadi, ovvero 8). Non c'è esplosione del danno per le armi con danno massimo inferiore od uguale a 6.

Alcune armi hanno una esplosione del danno diversa. Nella tabella delle armi dove è segnato EDX (es ED9), il valore X sta per il valore minimo sufficiente per tirare un'altra volta il danno, quindi in caso di ED9 puoi fare l'esplosione del danno con 9 o più con il dado dell'arma.

Questa è una caratteristica di poche armi estremamente letali.

L'esplosione del danno non esplode a sua volta, anche se fai il massimo del dado con il dado aggiunto questo non esplode nuovamente.

I tiri di dado aggiunti grazie al Tiro Critico non hanno il vantaggio dell'esplosione del danno. Se il dado dell'arma tirato grazie al Tiro Critico fa il massimo non ritiri il dado. Eventualmente usa dadi diversi tra loro quando tiri i dadi dell'armi.


\subsection{Attacchi multipli}\index{Attacchi multipli}\label{attacchimultiplimischia}\hypertarget{attacchimultiplimischia}{}
Con \textbf{una Azione} il personaggio può eseguire un \textbf{singolo Tiro per Colpire}.
Con \textbf{due Azioni} il personaggio può effettuare fino a \textbf{due Tiri per Colpire}. \textbf{Se vuole fare 3 o più attacchi deve usare 3 Azioni}.

Ogni singola freccia, dardo, pugnale o arma con gittata scagliata conta come un attacco.\index{Attacchi multipli con armi da distanza}
La prima Azione di attacco non ha penalità mentre la seconda Azione di attacco ha -5 al Tiro per Colpire. Successivi Tiri per Colpire cumuleranno -5 al colpire, quindi un terzo attacco avrà -10 ed un quarto attacco -15...
Se la penalità al colpire cumulativa diventa maggiore del Tiro per Colpire non è più possibile fare ulteriori attacchi.

I personaggi con Tiro per Colpire meno di 6 possono scegliere di effettuare 2 attacchi spendendo 2 Azioni ma applicando una penalità di -4 ad entrambi gli attacchi invece della progressione standard. Questo permette anche ai personaggi di livello basso di sfruttare efficacemente le loro Azioni in combattimento anche se con significative penalità.

\begin{giocatore}[Esempio Attacco Multiplo]
Ad esempio se ho Competenza Armi 5, Forza 1, +2 al compire come bonus dalla Lista d'Armi e +1 al colpire dato da una Abilità, +2 perché fiancheggio e +1 per arma magica il primo Tiro per Colpire sarà 3d6+12, il secondo sarà 3d6+7, il terzo 3d6+2. Non è possibile effettuare un quarto attacco in quanto il bonus al colpire diventerebbe negativo.
\end{giocatore}

Eventuali bonus al colpire dinamici e non \emph{fissi}, es. +1d6, si applicano al solo primo Tiro per Colpire e non al computo del bonus per calcolare il numero di attacchi multipli. Nel caso di esempio il Tiro per Colpire diventa 4d6+12, 3d6+7 e poi 3d6+2.

Il giocatore può dichiarare di effettuare gli attacchi su bersagli differenti. Ogni attacco può essere inframmezzato da una Azione di Movimento, purché si abbiano abbastanza Azioni.

I personaggi che non possono effettuare attacchi multipli possono utilizzare le Azioni rimanenti per spostarsi, cercare di fiancheggiare con un compagno, mettersi sulla difensiva o recuperare un oggetto.


%\begin{center}
%\includegraphics[width=0.9\linewidth]{immagini/archer.png}
%
%\emph{Scythian archers in ancient attic vase painting}
%\end{center}

\subsection{Armi da Lancio}\index{Attacchi multipli armi da lancio}\label{armidatiro}\index{Armi da Tiro}\index{Armi da lancio}

Le armi da lancio, o da tiro, sono tutte le armi con una gittata, ovvero che possono essere lanciate o lanciano dei proiettili. Le principali armi da lancio sono gli archi, balestre, fionde ma anche pugnali, giavellotti o lance qualora siano scagliate.

Il bonus al danno dato da Forza si applica in automatico per le fionde, pugnali, giavellotti..ovvero con tutte le armi che vengono scagliate di forza, gli archi applicano questo bonus solo se sono di tipo composito, le balestre non lo applicano mai.

La Destrezza modifica solo il Tiro per Colpire.

\textbf{I proiettili lanciati da Archi, Fionde, Balestre magiche non si considerano magici}.

\textbf{In caso di proiettili magici questi sommano il loro bonus magico al Tiro per Colpire ed al danno}.

In ogni arma da lancio è indicata la gittata ovvero entro che distanza è possibile tirare il proiettile senza penalità. Ogni arma da lancio può colpire entro tre volte la gittata indicata.

Se l'obiettivo è entro la gittata indicata non si hanno penalità al colpire, se il target è tra il primo e secondo incremento la penalità al colpire è -6. Se il target è tra il secondo è terzo incremento la penalità al colpire è di -12.\index{Penalita' distanza}

Un pugnale tirato entro 6 metri non ha penalità, tirato tra i 6 ed i 12 metri ha un -6 al colpire, a distanza tra 12 e 18 metri un -12 al colpire, oltre non può essere tirato.

%\begin{center}
%\includegraphics[width=0.75\linewidth]{immagini/fenice.png}
%
%\emph{Henry Justice Ford}
%\end{center}
\subsection{Arma Lunga} \index{Arma Lunga}\label{armalunga}

l'arma lunga permette di colpire un obiettivo a distanza di 2 metri.

Usare \textbf{Arma lunga a breve distanza} \index{Arma lunga a breve distanza}\label{armalungabrevedistanza}, inferiore ai 2 metri, comporta una penalità al Tiro per Colpire di -4, ad eccezione dell'utilizzo del Bastone.

\begin{giocatore}[Combattimento con Arma Lunga]
		Es. Tups armato di spada lunga affronta uno brigante armato di lancia lunga. Tups ha iniziativa 15, il brigante 12.

		Tups sfruttando la sua agilità arriva sotto il brigante colpendolo potentemente. Il brigante trovandosi in mischia con Tups non riesce a sfruttare la sua arma lunga che anzi lo penalizza.

		Usa una Azione per allontanarsi di due metri e poi attacca.

		Come terza azioni si allontana di altri 9 metri e urla imprechi verso Tups.

		Tups è a questo punto a 11 metri dall'avversario, decide di caricare aprendo così la propria difesa ma ottenendo un bonus al colpire.

		Carica il brigante colpendolo e arrivandogli addosso, con un ultima Azione decide di migliorare la sua Difesa (Preparare la Difesa).

		Il brigante molto ferito prova a colpirlo confidando che la sua difficoltà ad usare una arma lunga così da vicino sia bilanciata dalle penalità date dalla corsa di Tups. Tups viene colpito ed il brigante getta a terra la lancia ed estrae un corto pugnale e si mette sulla difensiva anche lui.

\end{giocatore}

\subsection{Arma Doppia} \index{Arma Doppia}\label{armadippia}

un'arma doppia è un'arma che è pericolosa da entrambe le estremità. Può essere usata come arma singola, oppure, incorrendo nelle penalità del combattimento con due armi, come appunto due armi.

Se non specificato un'arma doppia usata per combattimento con due armi equivale ad usare due armi medie.

\subsection{Armi Versatili} \index{Armi Versatili}\label{armiversatili}

le armi con il talento Versatile possono usare la Destrezza invece della Forza sui Tiri per Colpire. Al danno si usa sempre la Forza.

\subsection{Armi Leggere} \index{Armi Leggere}\label{armileggere}

queste armi sono leggere ed indicate per il \hyperlink{combattimentoaduemani}{combattimento a due armi}.

%\medskip

%\begin{center}
%\includegraphics[width=0.8\linewidth]{immagini/twoweapon.png}
%\end{center}

\subsection{Combattimento con due armi}\index{Combattimento con due armi}\hypertarget{combattimentoaduemani}{}\label{combattimentoduemani}

Gli attacchi fatti con l'arma secondaria si considerano attacchi multipli.
Se attacco una prima volta, indipendente che sia con l'arma primaria o secondaria, questo avrà il Tiro per Colpire a bonus pieno, gli altri attacchi cumuleranno il -5 al colpire.

Il bonus al danno dato dalla Forza sull'arma secondaria viene dimezzato. Se l'arma secondaria non è \textbf{Leggera} il Tiro per Colpire ha un ulteriore -3 al colpire (es. 0,-8,-10,-18..).

E' possibile usare l'arma secondaria per migliorare la Difesa di un punto ma non si possono fare attacchi con quell'arma.

\subsection{Carica} \index{Carica}\label{carica}\hypertarget{carica}{}

l'avversario deve essere entro 2 Azioni di movimento (18 o 12 metri solitamente) ed a non meno di 3 metri, il terreno non deve essere difficile (vedi anche Abilità \hyperlink{Rinoceronte}{Rinoceronte}, pag. \pageref{Rinoceronte}). Si deve correre fino ad essere a distanza di mischia.

Si ottiene un +1d6 a Tiro per Colpire, -4 alla Difesa fino all'inizio del proprio round successivo, l'attacco successivo al primo prende un -10 al colpire ed un eventuale successivo -15, 20...

Il movimento ed attacco costa 2 Azioni. Non si considerano altre penalità per avere corso oltre quelli indicati.

L'Azione di Carica ti porta addosso, in mischia, con l'avversario. L'attacco se fatto con arma lunga viene portato a distanza di 2 metri per poi finire a contatto con l'avversario.

%\begin{center}
%\includegraphics[width=0.9\linewidth]{immagini/carica.png}

%\emph{A Connecticut Yankee in King Arthur's Court / Samuel Clemens. New York : Charles L. Webster \& Co., 1889}
%\end{center}

\subsubsection{Arma da Controcarica}\index{Arma da Controcarica}\label{controcarica}\index{Controcarica}\label{caricaarmadacontrocarica}

se effettui una Carica ed il Tiro per Colpire ha successo, la tua arma con tratto Controcarica infligge un Tiro Critico aggiuntivo.


\subsubsection{Preparare una arma lunga/da controcarica contro una carica} \index{Preparare una arma lunga contro una carica}\label{prepararearmalungacontrocarica}

Solo un arma con il tratto controcarica può essere usata contro una carica. Preparare l'arma contro una carica costa una Reazione.

Se chi carica ha un portata minore dell'avversario allora chi prepara la controcarica può effettuare un attacco con l'arma, come Azione Gratuita (oltre alla Reazione per preparare l'arma) con un Tiro per Colpire con -1d6 di penalità, prima dell'avversario. Se colpisce infligge un Tiro Critico aggiunto.


%\begin{center}
%	\includegraphics[width=0.9\linewidth]{immagini/pilum.png}
%
%	\emph{Soldati romani armati di Pilum, pronti per una controcarica.}
%\end{center}

\subsection{Attacchi con armi a spargimento} \index{Armi a spargimento}\index{Acqua santa}\index{Olio Incediato}\label{attacchiarmidaspargimento}\hypertarget{spargimento}{}

sono armi a spargimenti quelle che \emph{spargono} il loro contenuto dove cadono, ad esempio olio incendiato/Acqua santa... Una arma a spargimento ha una gittata di 6 metri\index{Lanciare Armi a Spargimento}\index{Gittata armi a spargimento}.

In caso l'attacco manchi (di almeno 5) tirare un d8 e consultare questo schema per capire dove la fiala è caduta e tira un 2d6 per determinare lungo la direzione indicata dal d8 precedente a quanti metri è caduto distante dal bersaglio, ovvero contate i metri dal target.

\medskip
\begin{center}

\begin{tabular}{ccc}
 8& 1& 2\\
 \rowcolor{gray!20}	7 &\textbf{X}& 3\\
 6 &5 &4\\
	&\textbf{0}&
\end{tabular}
\end{center}

\smallskip

\textbf{X} si considera il bersaglio dell'oggetto tirato. \textbf{0} il punto di origine del lancio.

Ad esempio con il tiro del d8 faccio 5 e poi tirando 2d6 faccio 4, significa che la boccetta è caduta a destra del bersaglio a 4 metri.

E' anche possibile che ci si sia tirati sui piedi la boccetta (es faccio 7 e poi 6.. potrei averla tirata addosso ad un compagno o dietro di me!).

\subsection{Impreparato -- Colti di Sorpresa}\index{Impreparato}\index{Sorpresa}\label{coltidisorpresa}\hypertarget{sorpresa}{}

se una creatura viene colta di sorpresa, ovvero non si aspetta di essere attaccata, si deve considerare questo primo round come round di sorpresa. Chi è sorpreso ha un -2 alla Difesa ed ai Tiri Salvezza su Riflessi.

Non si possono usare Azioni o Reazioni se non esplicitamente permesse; dal round successivo si potrà dichiarare l'iniziativa ed agire normalmente. Le medesime considerazioni valgono per gli avversari se sorpresi.

Confrontate la prova di Furtività di chi si muove furtivo contro 10+Consapevolezza di chi potrebbe essere sorpreso. Se la prova è superiore allora la creatura è effettivamente sorpresa. Se chi dovrebbe essere sorpreso è sull'attenti e vigile concedete +2 alla prova di Consapevolezza.

Quando entrambe le creature sono colte di sorpresa per valutare chi effettivamente è sorpreso effettuate un Tiro Salvezza su Riflessi, chi ottiene più di 15 non è sorpreso.

\subsection{Magia in combattimento}\index{Magia in combattimento}\label{magiaincombattimento}

l'incantatore che lancia una magia mentre è in combattimento (ha un avversario in mischia o viene bersagliato da distanza) si considera Distratto.

%\begin{center}
%\includegraphics[width=0.45\linewidth]{immagini/shield-milan.png}
%
%\emph{Sconfitta di Johann Friedrich di Sassonia all'imperatore Charles V}
%\end{center}

\subsection{Modificatori in attacco o difesa} \index{Modificatori in Attacco e Difesa}\label{modificatoriattaccodifesaparticolari}

Il migliore suggerimento che si può dare nel gestire le situazioni di combattimento più caotiche è pensare a queste come ad un film, valutate la cinematicità della situazione.

Non è una questione di miniature, spazi, quadretti.. è una questione di divertimento e visualizzazione della scena. Soluzioni non ortodosse per situazioni non ortodosse.

Concedete un bonus o penalità ($\pm 1-2$) se non indicato diversamente) ogni qual volta il giocatore abbia un vantaggio o svantaggio ed allo stesso modo all'avversario.


\subsubsection{Lettura dei modificatori in attacco e difesa} \index{Lettura dei modificatori in attacco e difesa}\label{letturamodificatoriattaccodifesaparticolari}

Quando si scrive -1d6 significa che si tira un dado in meno (o due se è -2d6), parimente se c'è scritto +1d6 si tira un dado a 6 in più e si somma.

Quando la penalità è alla Difesa considerare ogni -1d6 come un -4 alla Difesa.

\medskip

In linea di principio in combattimento un bonus leggero è un +1, medio +2, alto +1d6 (o +4), un bonus molto alto è +2d6 (o +8), viceversa per le penalità.


\medskip

\end{multicols}

\noindent\begin{tabularx}{\linewidth}{l|X|X}
	\toprule
 \rowcolor{gray!20}\multicolumn{2}{c}{\textbf{Attaccante}}&\multicolumn{1}{c}{\textbf{Difensore}}\\
\textbf{Mod}.&\multicolumn{1}{c}{\emph{Situazione}}&\multicolumn{1}{c}{\emph{Situazione}}\\
\toprule
\rowcolor{gray!20}\textbf{-1}& Affaticato (1), Luce fioca&Affaticato (1)\\

\textbf{-2}& Affaticato (2), Intralciato & Affaticato (2), Afferrato, Intralciato, Sorpreso\\

\rowcolor{gray!20}\textbf{-4}& Affaticato (4), Prono, Arma Lunga a corta distanza, attacco non letale con arma letale& Affaticato (4), Prono, In ginocchio, Seduto, Ristretto, Stordito, Afferrato ad una parete, Bloccato\\

\textbf{-1d6}& Ristretto, Spaventato, Arma da Lancio contro avversario in mischia, Arma non conosciuta, Bersaglio invisibile ma Individuato, Afferrato ad una parete, Bloccato&\\

%\textbf{+1}& & \\
%
\rowcolor{gray!20}\textbf{+2}& Fiancheggia, Posizione Sopraelevata, Attacca al spalle& Copertura leggera\\

\textbf{+4}&& Copertura media\\

\rowcolor{gray!20}\textbf{+1d6}& Invisibile, Carica, avversario Indifeso& \\

\textbf{+8}&& Copertura completa
\end{tabularx}

\medskip

\begin{multicols}{2}

\begin{narratore}[Non è tutto Regole]
Cercate di non interrompere il gioco cercando la regola precisa, lasciatelo fluire, dite ai giocatori che per brevità gestite la situazione in una certa maniera; ci sarà poi tempo per ricordare la situazione e trovare la regola giusta. Interrompere il gioco di continuo spezza il \emph{pathos} della situazione.
\end{narratore}

\begin{narratore}[Motivi e Scopi]
	Ricordate sempre che lo scopo è divertirsi, a scapito (per il Narratore) di qualche mostro, non siate rigidi ma dinamici e adattatevi alle situazioni.
\end{narratore}

I \textbf{modificatori positivi indicati} nella \emph{Tabella: Modificatori in attacco o difesa} si sommano a partire da quello maggiore e si aggiunge un +1 per ogni ulteriore bonus presente. Se un avversario è sopra il personaggio, alle spalle, invisibile ed in carica avrà un bonus al colpire di +1d6 (carica o invisibilità) +1 perché sopra, +1 perché alle spalle, +1 perché in carica.\index{Somma bonus al colpire}

Le \textbf{penalità} si sommano per intero tra loro. Se il personaggio è sorpreso e prono ha un -6 alla Difesa.\index{Bonus comulativi}\index{Somma penalità}


\subsection{Altre azioni e situazioni} \label{AltreAzioni}\index{Altre azioni e situazioni}

\subsubsection{Attacco a mani nude} \index{Pugno}\index{Calci} \index{Fare a botte}\label{attaccomaninude}

due armi che non mancheranno mai a nessuno sono i propri pugni e calci, con queste armi si è sempre addestrato e non si considerano attacchi improvvisati.

Se non si ha preso la lista d'armi \emph{Pugno Vuoto} un pugno o calcio farà 1d3 + Forza di danno non letale. Solo con la Lista d'Armi Pugno Vuoto si diventa artisti marziali.

\subsubsection{Aiutare un altro in combattimento}\index{Aiutare in combattimento}\label{aiutare}\hypertarget{aiutare}{}

si può aiutare un compagno ad attaccare o a difendersi negli scontri in mischia, distraendo o interferendo con l'avversario. Si può portare un attacco in mischia (1 Azione) contro un avversario che ha già ingaggiato battaglia con un proprio alleato.

Si effettua un Tiro per Colpire contro la Difesa dell'avversario con 1d6 di bonus. Se l'attacco va a segno non si fa danno ma il compagno ottiene bonus di +1 al Tiro per Colpire verso quell'avversario od un bonus di +1 alla Difesa entro la fine del tuo round successivo contro quell'avversario sul primo attacco. Se l'aiutante ottiene un Tiro Critico allora chi è aiutato avrà un bonus di +2.

più personaggi possono aiutare lo stesso alleato; i bonus di questo tipo sono cumulabili (massimo 4 su taglia media), purché l'avversario sia circondato.

\subsubsection{Alzarsi da prono}\index{Alzarsi da prono}\label{alzarsidaprono}\hypertarget{alzarsidaprono}{}

costa 1 Azione. Con una Azione Immediata il personaggio può eseguire una prova di Acrobatica ed alzarsi se fa 13. Se fa un Fallimento Critico nella prova non può fare altre azioni quel round e rimane prono.

Quando sei prono puoi strisciare\index{Strisciare}\index{Carponi} o muoverti a carponi. Il terreno si considera difficile e sei comunque considerato ancora prono finché non ti alzi.

\subsubsection{Colpo di Grazia} \index{Colpo di Grazia}\label{colpodigrazia}

costa 3 Azioni, si può utilizzare un'arma da mischia per infliggere un colpo di grazia ad un bersaglio inabile o indifeso (svenuto o intrappolato). Si può anche usare un arco o una balestra, l'importante è che si sia adiacente al bersaglio.

L'attaccante colpisce automaticamente ed infligge tre colpi critici.

\subsubsection{Tiri Mirati}\index{Tiri Mirati}\label{tirimirati}\index{Mirare a parti specifiche}

OBSS non prevede la possibilità di effettuare tiri mirati con armi o incantesimi, tranne se questo lo specifica.

Quando si colpisce il bersaglio lo si colpisce genericamente, senza possibilità di specificare se alla testa, gamba o altro, medesimo concetto vale in caso di colpi ad oggetti, es. se miri ad un cardine di una porta colpisci tutta la porta. Questo non impedisce al Narratore di valutare conseguenze adeguate all'azione intrapresa.

\medskip

\begin{center}
	\includegraphics[width=0.6\linewidth]{immagini/Judith_Beheading_Holofernes_MET_MM27116.png}

	\emph{Judith Beheading Holofernes}
\end{center}

\subsubsection{Danno non letale}\index{Danno non letale}\label{dannononletale}

il danno non letale è una forma di danno causato da armi particolari o quando lo scopo è fare svenire l'avversario e non ucciderlo.

Il danno non letale si tratta come il danno normale se non che si recupera più velocemente e scendere sotto zero Punti Ferita causa svenimento e non l'avvicinarsi della morte.

\index{Danno non letale con arma non idonea} \label{dannononletalearmanonidonea}

Se vuoi fare danno non letale con un'arma non predisposta al danno non letale hai un -4 al Tiro per Colpire.

\subsubsection{Senza Competenza}\index{Senza Competenza}\label{senzacompetenza}

usare un'arma senza l'adeguata competenza, ovvero non avere la Lista d'Armi d'appartenenza dell'arma, impone un -1d6 al Tiro per Colpire.

Non puoi usare la capacità Versatile di un arma se non la sai usare. Calci e Pugni oppure un Arma Semplice sono usabili senza penalità anche senza competenze specifiche.

\subsubsection{Lanciare armi} \index{Lanciare armi}\label{lanciarearmi}

una spada o comunque un arma non fatta per essere lanciata, senza Gittata, può comunque essere scagliata contro l'avversario.

Il Tiro per Colpire prende un -1d6 e l'arma fa una categoria di danno inferiore (la spada lunga fa 1d6, una spada corta 1d4..). La gittata di lancio è 3 metri.

\subsubsection{Colpi Potenti}\index{Colpi Potenti}\label{colpipotenti}

Il personaggio al momento dell'attacco può dichiarare di aggiungere un +1 al danno togliendo 2 al Tiro per Colpire con l'arma da mischia (requisito Competenza Armi +1). Non si può togliere più di Competenza Armi/4 al Tiro per Colpire. Da dichiarare prima del Tiro per Colpire.

\subsubsection{Fiancheggiare, attaccare alle spalle}\hypertarget{fiancheggiare}{} \index{Fiancheggiare}\label{fiancheggiare}\index{Attaccare alle spalle}

se due personaggi sono attorno allo stesso bersaglio ma non sono a fianco tra loro prendono +2 al Tiro per Colpire o alla Difesa (a loro scelta quale bonus prendere).

Al massimo ci possono essere 4 personaggi attorno ad una creatura di taglia media che prendono il bonus di fiancheggiare. Il tipo di bonus si sceglie round per round, se non dichiarato vale come +2 al Tiro per Colpire.

Se tirando una ipotetica riga che collega i due personaggi questa attraversa in pieno il quadretto dell'avversario allora c'è la situazione di fiancheggiamento.

Un creatura può attaccare alle spalle se l'avversario non è in grado di fronteggiarlo. Attaccare alle spalle concede un +2 al Tiro per Colpire. Non si cumula con il Fiancheggiare.

\medskip

\textbf{Esempio di fiancheggiamento}\index{Esempi di Fiancheggiamento}

\medskip
\begin{center}

\begin{tabular}{lll}
\hline
\rowcolor{gray!20}A & G & D\\
B & \textbf{X} & E\\
\rowcolor{gray!20}C & H & F
\end{tabular}
\end{center}

\bigskip

In questo schema il fiancheggiamento è preso dalle coppie: A-F, B-E, C-D, G-H

\bigskip

Se la creatura può fronteggiare più creature contemporaneamente queste non godranno del bonus di fiancheggiamento.

\subsubsection{Maestria del combattimento} \index{Maestria del combattimento}\label{maestriacombattimento}

Il personaggio al momento dell'attacco può dichiarare di aggiungere un +1 alla Difesa togliendo 2 al Tiro per Colpire. Viceversa può prendere un -2 alla Difesa per alzare di +1 il Tiro per Colpire e quindi migliorare l'attacco.

Questi modificatori permangono fino all'inizio del proprio round successivo.

Non si può togliere/aggiungere più di Competenza Armi/4 al Tiro per Colpire/Difesa, Maestria del combattimento non consuma Azioni.

\subsubsection{Preparare la Difesa}\index{Preparare la Difesa}\label{preparareladifesa}\hypertarget{preparareladifesa}{}\index{Parata}

Il personaggio può usare una Azione per prepararsi meglio ai successivi attacchi degli avversari. Fino all'inizio del tuo prossimo round hai un +1 alla Difesa.

Se almeno un arma che usi ha il tratto \textbf{Parata} prendi un ulteriore +1 alla Difesa.

\subsubsection{Difesa totale} \index{Difesa totale}\label{difesatotale}\hypertarget{difesatotale}{}

Il personaggio usa 2 Azioni, prende un bonus di +4 alla Difesa e tratta come Difficile il terreno fino all'inizio del suo prossimo round.

\subsubsection{Disingaggiare} \index{Disingaggiare}\label{disingaggiare}\hypertarget{disingaggiare}{}

Il personaggio usando 1 Azione si sposta di 1 metro e non causa attacchi di opportunità.\index{Passo difensivo}

%\subsubsection{Capriola} \index{Capriola}\label{capriola}\hypertarget{capriola}{}

%Il personaggio può tentare di muoversi tra gli avversari non causando attacchi di opportunità se riesce in una prova di Acrobatica per avversario che vuole \emph{evitare}. La DC è pari a 10+GS avversario +2 per avversario già evitato. Il personaggio non consuma Azioni per la prova di Acrobatica ma per le Azioni di Movimento fatte. Se il personaggio fallisce la prova di Acrobatica interrompe la sua Azione di Movimento vicino all'avversario per il quale ha fallito la prova. Il terreno si considera difficile.

\subsubsection{Colpo Preciso} \index{Colpo Preciso}\label{colpopreciso}\hypertarget{colpopreciso}{}

Il personaggio usando 2 Azioni effettua un solo attacco in mischia. Su questo singolo attacco ottiene un bonus di +1d4 al Tiro per Colpire.

\subsubsection{Prendere la Mira (cecchino)} \index{Prendere la Mira (cecchino)}\label{cecchino}\hypertarget{cecchino}{}

Il personaggio che attacca con un arma da lancio può usare 2 Azioni a round, fino ad un massimo di 3 round, per prendere la mira contro un obiettivo. Ha un bonus al Tiro per Colpire pari a +1 il primo round, +2 il secondo round ed infine nel terzo round di Prendere la Mira il bonus arriva a +4.

Non può usare Azioni di Movimento mentre prende la mira.

\subsubsection{Arma da lancio contro obiettivi in combattimento} \index{Arma da lancio contro avversario impegnato in combattimento}\label{usarearmalancioinmischia}

In combattimento non è facile mirare un obiettivo che è in combattimento con un altra creatura.

Oltre le eventuali penalità date della \hyperlink{esempicopertura}{Copertura} (pag. \pageref{esempicopertura}) si ha una penalità aggiuntiva di -2 al Tiro per Colpire.

Es. Se voglio colpire un avversario coperto ed in combattimento con un mio compagno oltre alla penalità di -2 al Tiro per Colpire perché è in combattimento con qualcuno, questo avversario ha \emph{copertura leggera} (+2 Difesa), se ho due creature in fila e poi l'avversario da colpire questo avrà da copertura +4 ed un ulteriore +2 alla Difesa perché combatte con qualcuno.
Se sono 3 creature allora la \emph{copertura sarà completa} (+8 Difesa, +2 perché combatte con un altro).

Se si mira ad un obiettivo che è in combattimento ed entrambi sono frontali al personaggio si ha solo il -2 al Tiro per Colpire senza la penalità della copertura.

In caso di Fallimento Critico nel Tiro per Colpire si colpisce casualmente una creatura che dava copertura o chi era a fianco dell'obbiettivo.

Vedi anche Abilità \hyperlink{Precisino}{Precisino} (pag. \pageref{Precisino}).

\subsubsection{Usare un'arma da lancio sotto minaccia} \index{Usare un'arma da lancio sotto minaccia}\label{usarearmalanciosottominaccia}

usare un'arma da lancio come arco, balestra o pugnale (che si vuole lanciare) mentre si è minacciati in mischia impone una penalità di -1d6 al Tiro per Colpire.

Vedi anche Abilità \hyperlink{Uno con l'arco}{Uno con l'arco} (pag. \pageref{Uno con l'arco}.)

\medskip
\subsubsection{Arma troppo grande}\index{Arma troppo grande} \label{armatroppogrande}

La taglia indicata nelle tabella delle armi (vedi \hyperlink{dimensionediunarma}{Dimensioni delle Armi} ) è riferita ad una creature di taglia media. Per una creatura di taglia piccola la dimensione si deve intendere di una categoria superiore, es. una spada corta che è di dimensioni piccole per una creatura di taglia media, se usata da una creatura di taglia piccola si considera un'arma di dimensioni medie.

Allo stesso modo una arma grande, come uno spadone a due mani, nelle mani di un gigante diventa un'arma di dimensioni medie.

Questo non ne cambia il danno o il tipo di danno causato dall'arma.

Una creatura può usare un arma con dimensione della propria taglia o di un solo grado inferiore con una mano e deve usare due mani per brandire un arma di una taglia superiore alla propria.

Se l'arma è di taglia superiore a quella usabile con 2 mani, esempio un Alabarda (arma grande) per una creatura di taglia piccola la penalità al Tiro per Colpire è -1d6. Lo stesso principio è valido per una spadone a due mani di taglia grande (2d8 di danno) nelle mani di una creatura di taglia media.


\begin{center}
	\includegraphics[width=0.55\linewidth]{immagini/angelospadone.png}
\end{center}


Nella tabella delle armi la dimensione è segnata come P (piccola), M (media), G (grande), E (enorme) ed è riferita ad una creatura di taglia media. Una versione \emph{più grande} di un arma aumenta di una categoria il danno dell'arma (1d4->1d6, 1d6->1d8, 1d8->1d10, 1d10/1d12->2d6, 2d6->2d8, 2d8->2d10, 2d10->3d6...).

Es. uno spada lunga grande (+1 taglia) passa da 1d8 a 1d10 di danno.


\subsubsection{Usare un'arma con due mani} \index{Usare un'arma con due mani}\label{usarearmaconduemani}

un arma ad una mano che può (ma non deve) essere usata a due mani aumenta il dado di danno quando usata a due mani.

Es. Spada Lunga per una creatura media può causare 1d8 ad una mano o 1d10 a due mani. Una Spada Corta non può essere impugnata a due mani da una creatura media ma può essere tenuta a due mani da una creatura piccola.

Se l'arma deve essere tenuta a due mani perché troppo grande per la propria taglia questo modificatore non si considera (es. uno spadone a due mani per una creatura di taglia media, o una spada lunga per una creatura piccola).

Il valore di EDX se diverso dal massimo danno dell'arma aumenta di 2 (la Katana causerà 2d6 di danno ed avrà ED11) quando usata a due mani.

\subsubsection{Combattere al buio}\index{Combattere al Buio}

Combattere in condizioni di luminosità ridotta comporta delle difficoltà riassunte in questo schema.

\medskip

\noindent\begin{tabular}{lll}
	\toprule
\rowcolor{gray!20}\textbf{Visione} & \multicolumn{2}{c}{\textbf{Condizione}}\\
& Luce Fioca & Oscurità\\
\toprule
Normale & -1 TC, -2 Cons. & Invis. (pag. \pageref{invisibilita})\\
\rowcolor{gray!20}Crepuscolare & Normale & Invis. (pag. \pageref{invisibilita})\\
Scurovisione & Normale & -2 Cons.
\end{tabular}

\medskip

Vedi anche Capitolo \hyperlink{visioneeluce}{Visione e Luce} (pag. \pageref{visioneeluce}).

%\subsection{Opzionale - L'Unica Regola}\index{Opzionale - L'Unica Regola}\hypertarget{lunicaregola}{}\label{lunicaregola}
%Questa opzione vuole semplificare la gestione di qualsiasi prova contrapposta, possa essere relativa alle Competenze di Base o Attiva.
%Quando una creatura o personaggio ha un vantaggio o svantaggio tira in aggiunta 1d6 alla prova. Se ha due vantaggi tira 2d6, se ne ha tre di vantaggi tira 3d6...
%Alla prova aggiunge o sottrae, in caso di bonus o penalità, il valore più alto tra i dadi tirati. Per questi dadi non vale l'esplosione del risultato.
%Uno Svantaggio annulla un Vantaggio se presente.
%Se il vantaggio/svantaggio è relativo ad un valore statico (come la Difesa) allora questo aumenta di 2 per ogni vantaggio/svantaggio accumulato.

\subsection{Manovre Opzionali in Combattimento}\label{azioniopzionaliincombattimento}

Queste Azioni di combattimento sono a discrezione del Narratore che può concederle o meno. \textbf{Ogni manovra conta come Azione di Attacco} per quanto riguardo le penalità del multiattacco.\index{Manovre ed Azioni di Attacco}

Quando queste manovre sono fatte dagli avversari e non sono segnati i valori di Tiro per Colpire, Atletica, Ingannare... contrapporre alla prova il Tiro Salvezza indicato dopo il costo in Azioni ed i modificatori suggeriti (Taglia...).

\medskip

\subsubsection{Disarmare*}\index{Disarmare}\label{disarmare}\hypertarget{disarmare}{}

fai una Prova Contrapposta di Competenza Armi + Destrezza o Forza (3d6+CA+For o Des).

Se chi tenta la manovra non riesce e ottiene un Fallimento Critico è lui che perde l'arma. Costa 2 Azioni (Riflessi).

\subsubsection{Finta*} \index{Finta}\label{finta}\hypertarget{finta}{}

fai una Prova Contrapposta di Competenza Armi + Ingannare (chi fa la finta) contro Competenza Armi + Percepire Emozioni (chi subisce la finta). Se la prova riesce l'avversario ha un -2 alla Difesa contro di te fino alla fine del tuo round.

Se chi tenta la manovra non riesce e ottiene un Fallimento Critico è lui che prende -2 alla Difesa fino alla fine del suo round successivo. Costa 1 Azione (Volontà).

%\begin{center}
%	\includegraphics[width=0.95\linewidth]{immagini/alfieri37.png}
%\end{center}

\subsubsection{Spingere/Tirare un avversario*} \index{Spingere un avversario}\label{spingereavversario}\hypertarget{spingereavversario}{}\index{Tirare un avversario}

è una prova di Atletica contrapposta ad un Tiro Salvezza su Tempra con Forza. Chi ha una taglia maggiore guadagna un bonus di +1d6 per taglia di differenza.

Chi vince può spingere o tirare l'avversario fino a 0.5 metri nella direzione che vuole per successo nella prova (fino al massimo del suo movimento). Es. se vinci la prova di 7 sposti l'avversario fino a 3 metri. Chi spinge si può muovere assieme a chi è spinto senza usare altre Azioni.

Se si vuole \emph{lanciare} l'avversario il movimento è di 0.3 metri per successo ottenuto.\index{Lanciare una creatura}

Se chi tenta la manovra non riesce e ottiene un fallimento critico, l'avversario usando una Reazione, può spostarlo secondo le regole di sopra. Costa 2 Azioni (Tempra).

\subsubsection{Afferrare un avversario*}\index{Afferrare un avversario}\label{afferrareunavversario}\hypertarget{afferrareunavversario}{}

è una prova di Atletica contrapposta ad un Tiro Salvezza su Tempra con Forza. Chi ha una taglia maggiore guadagna un bonus di +1d6 per taglia di differenza. Se chi riesce nella manovra ottiene un Successo Critico si considera che abbia \hyperlink{bloccato}{Bloccato} l'avversario.

Costa 2 Azioni (Tempra) fare e mantenere e liberarsi dalla presa. Si considera che chi afferra sia anche Afferrato ed abbia almeno una mano occupata nell'afferrare.
Muovere una creatura afferrata richiede \hyperlink{spingereavversario}{Spingere un avversario}.

Ogni contendente può attaccare l'altro afferrato con un arma piccola o armi naturali, la Difesa ha una penalità di -2 e si è considerati Distratti. Attaccare una creatura diversa da chi si afferra ha un -1d6 al Tiro per Colpire.

\subsubsection{Attraversare i nemici*}\index{Attraverso i nemici}\index{Attraversare quadretti occupati}\label{attraversonemici}\index{Destreggiarsi}\hypertarget{attraversarenemici}{}\hypertarget{destreggiarsi}{}

\index{Attraversare nemici}\index{Movimento attraverso}Un personaggio può \textbf{attraversare} ma non sostare in \textbf{una zona occupata} da una creatura senza essere \hyperlink{ristretto}{\textbf{ristretti}}\index{Ristretto}.

Per attraversare il terreno, dove c'è una creatura ostile o che è in portata di attacco dalla creatura ostile che si passa a fianco, è necessario eseguire una Prova Contrapposta su Atletica o Acrobatica contro Tiro Salvezza su Riflessi della creatura al quale si vuole \textbf{attraversare} il terreno, ogni creatura attraversata oltre la prima cumula una difficoltà di +2.

Costa 1 Azione (Riflessi) la prova per attraversare, indipendentemente dal numero di creature, oltre all'Azione di Movimento. Il terreno occupato o minacciato dalla creatura ostile si considera difficile. Il terreno non si considera difficile solo nel caso in cui la creatura sia di due o più taglie inferiori. In caso di Successo Critico nella prova di Atletica o Acrobatica non si consuma l'Azione usata per Attraversare i nemici ma solo quella di movimento.

Se si fallisce si rimane ne quadretto immediatamente precedente al nemico, con il rischio di essere \hyperlink{ristretti}{ristretti} (pag. \pageref{ristretti}). Si considera terminata sia l'Azione di Movimento che quella di oltrepassare.
Se il nemico ha l'Abilità \hyperlink{opportunista}{Opportunista} oltre ad ostacolare il passaggio può eseguire un attacco, usando una Reazione.

%\begin{center}
%\includegraphics[width=0.7\linewidth]{immagini/vantaggio.png}
%
%\emph{Henry Justice Ford}
%\end{center}

\subsubsection{Fare cadere un avversario*} \index{Fare cadere un avversari}\label{farecadereavversario}\hypertarget{farecadereavversario}{}

è una prova di Atletica contrapposta ad un Tiro Salvezza su Tempra con Forza per mettere Prono l'avversario.

Per ogni gamba/zampa in più il contendente ottiene un bonus di +1 alla prova e ottiene un +1d6 per differenza di Taglia.

Se chi tenta la manovra non riesce e ottiene un Fallimento Critico è lui che cade. Costa 2 Azioni (Tempra).

%\subsubsection{Modificare le proprie dimensioni*}\index{Modificare le proprie dimensioni}\label{modificatedimensioni}

%nel caso il personaggio cambi dimensione \index{Modificare le dimensioni} la sua Difesa cambia di conseguenza.

%\medskip
%\noindent\begin{tabular}{ll|ll}
%\textbf{Taglia} & \textbf{Difesa}&\textbf{Taglia} & \textbf{Difesa}\\
%\toprule
%Piccolissima& +8 & Grande & -1\\
%Minuta & +4 & Enorme & -2\\
%Minuscola & +2 & Mastodontica & -4\\
%Piccola & +1 & Colossale & -8\\
%Media & +0 &&
%\end{tabular}

\subsection{Opzionale - Azioni Tiro Critico}\index{Opzionale - Azioni Tiro Critico}\label{OpzionaleAzioniTiroCritico}\hypertarget{OpzionaleAzioniTiroCritico}{}

Questa Opzione permette un combattimento meno incentrato sul danno ma più sulle manovre e la tattica di utilizzo dei colpi.

Il giocatore tiene conto dei Tiri Critici che tira e che non applica al danno. Entro tre round dal loro tiro vanno usati.

Ogni round può \emph{usare} uno o più Tiri Critici accumulati per eseguire Azioni Critiche. L'uso delle Azioni Critiche deve essere contro l'avversario al quale si sono effettuati i Tiri Critici.

L'elenco propone l'elenco delle Azioni Critiche di Tiri Critici consumati. Non si possono avere più di 6 Tiri Critici accumulati per singolo round.
L'attivazione di queste Azioni Critiche costa una Reazione se non indicato diversamente.

Usate questi esempi come delle linee guida per stimolare il personaggio a creare un suo stile di combattimento. E' importante che il personaggio descriva come attiva l'Azione Critica.

\medskip

\noindent\textbf{Critici} \hskip 0.5cm \textbf{Effetto}

\medskip

\begin{itemize}[leftmargin=*]
	\setlength{\itemsep}{0pt}

	% ATTACCHI AGLI OCCHI
	\item \textbf{Attacchi agli occhi}
	\begin{itemize}[leftmargin=*]
		\setlength{\itemsep}{0pt}
		\item \textbf{1}: Sabbia negli occhi. Entro la fine del tuo prossimo round l'avversario ha -2 al primo Tiro per Colpire

		\item \textbf{2}: Graffio agli occhi. Entro la fine del tuo prossimo round l'avversario ha -4 al Tiro per Colpire

		\item \textbf{3}: Bersaglio abbagliato. Tira 1d6, con 1-2-3 l'avversario ha mancato il suo attacco. Dura fino alla fine del prossimo round.

		\item \textbf{4}: Bersaglio accecato. Per 1d6 round, l'avversario considera tutti come invisibili.

		\item \textbf{5}: Orbo. L'avversario esegue un Tiro Salvezza su Tempra con DC pari al tuo ultimo Tiro per Colpire, se fallisce è accecato permanentemente, altrimenti subisce gli effetti del punto 4.
	\end{itemize}

	% ATTACCHI ALL'ARMA
	\item \textbf{Attacchi all'arma}
	\begin{itemize}[leftmargin=*]
		\setlength{\itemsep}{0pt}
		\item \textbf{1}: Colpisci l'arma. L'avversario esegue un Tiro Salvezza su Tempra DC 15, $ \pm 2 $ per differenza di taglia dell'arma, oppure lascia cadere l'arma

		\item \textbf{2}: Arma danneggiata. L'arma dell'avversario infligge una categoria di danno in meno

		\item \textbf{3}: Colpo alla mano. A causa del dolore entro la fine del tuo prossimo round l'avversario perde i primi due attacchi

		\item \textbf{4}: Disarmi l'avversario. L'avversario lascia cadere l'arma

		\item \textbf{5}: Mano compromessa. L'avversario fino all'alba del giorno successivo ha -4 al Tiro per Colpire
	\end{itemize}

	% SPINTONI E SLANCI
	\item \textbf{Spintoni e Slanci}
	\begin{itemize}[leftmargin=*]
		\setlength{\itemsep}{0pt}
		\item \textbf{1}: Allontani l'avversario di 3 metri, $ \pm 1$ per taglia di differenza

		\item \textbf{1}: Ti puoi spostare di una Azione di Movimento come Reazione. Tratti il terreno come difficile

		\item \textbf{2}: Come punto 1 ma la distanza iniziale è di 6 metri

		\item \textbf{2}: Ti puoi spostare di una Azione di Movimento come Reazione

		\item \textbf{3}: Come punto 1 ma la distanza iniziale è di 9 metri

		\item \textbf{4}: Come punto 3 e puoi spostarti con l'avversario
	\end{itemize}

	% SGAMBETTI
	\item \textbf{Sgambetti}
	\begin{itemize}[leftmargin=*]
		\setlength{\itemsep}{0pt}
		\item \textbf{1}: Spallata. L'avversario esegue un Tiro Salvezza su Tempra DC 15, $ \pm 4 $ per taglia differenza o cadere prono

		\item \textbf{2}: Sgambetto. L'avversario esegue un Tiro Salvezza su Riflessi DC 19, $ \pm 2 $ per taglia differenza o cadere prono

		\item \textbf{3}: Spinta. L'avversario esegue un Tiro Salvezza su Tempra DC 23, $ \pm 2 $ per taglia differenza o cadere prono. Se il Tiro Salvezza riesce viene allontanato di 1d6 metri

		\item \textbf{4}: Urto. L'avversario esegue un Tiro Salvezza su Tempra con DC pari al tuo ultimo Tiro per Colpire, $ \pm 2 $ per taglia differenza o cadere prono. Se il Tiro Salvezza riesce viene allontanato di 1d10 metri
	\end{itemize}

	% PROIETTILI
	\item \textbf{Proiettili}
	\begin{itemize}[leftmargin=*]
		\setlength{\itemsep}{0pt}
		\item \textbf{Vari}: Per ogni Tiro Critico usato aggiungi una ulteriore gittata alla tua arma

		\item \textbf{Vari}: Per ogni due Tiro Critico usato aggiungi +4 al prossimo Tiro per Colpire entro la fine del round successivo

		\item \textbf{5}: \emph{Freccia Kennedy}. Entro la fine del tuo prossimo round il primo proiettile ignora qualsiasi copertura o ostacolo e se fisicamente possibile colpisce l'avversario
	\end{itemize}

	% FURIA
	\item \textbf{Furia}
	\begin{itemize}[leftmargin=*]
		\setlength{\itemsep}{0pt}
		\item \textbf{1}: Incitare i compagni. I tuoi compagni entro 6 metri hanno al loro primo attacco +2 al Tiro per Colpire

		\item \textbf{2}: Berserker. Un tuo compagno entro sei metri può eseguire una Azione di Attacco contro un avversario in mischia al costo di una Reazione

		\item \textbf{3}: Scossa. Un tuo compagno, entro 6 metri, può usare una Reazione per eseguire una Azione di Movimento trattando il terreno come difficile

		\item \textbf{4}: Gloria. I tuoi compagni entro 6 metri hanno +2 al Tiro per Colpire entro la fine del loro prossimo round

		\item \textbf{5}: Gloria!. I tuoi compagni entro 9 metri hanno +4 al Tiro per Colpire entro la fine del loro prossimo round
	\end{itemize}

	% DIFESA!
	\item \textbf{Difesa!}
	\begin{itemize}[leftmargin=*]
		\setlength{\itemsep}{0pt}
		\item \textbf{1}: Fino alla fine del tuo prossimo round hai +2 alla Difesa.

		\item \textbf{2}: Fino alla fine del tuo prossimo round tutti i compagni nel tuo raggio di mischia hanno +2 alla Difesa

		\item \textbf{3}: Fino alla fine del tuo prossimo round un compagno ha +8 alla Difesa.

		\item \textbf{4}: Fino alla fine del tuo prossimo round tutti i compagni nel raggio di 9 metri hanno +4 alla Difesa

		\item \textbf{5}: Per 1d6 round tutti i tuoi compagni hanno +4 alla Difesa
	\end{itemize}

\end{itemize}

\medskip

\begin{narratore}[Azioni Critiche]
Queste Azioni Critiche possono essere descritte come approfittare della distrazione dell'avversario, gettare terra negli occhi, costringere a colpi di arma a spostarsi...
\end{narratore}

\begin{enfasi}{
Onestà e Giustizia, Eroico Coraggio, Compassione, Gentile Cortesia, Completa Sincerità, Onore, Dovere e Lealtà (I sette princìpi del bushido)
}\end{enfasi}


\subsection{Opzionale - Elenco Manovre d'Arme}\hypertarget{elencotalentiarmi}{}\label{elencotalentiarmi}\index{Opzionale - Elenco Manovre d'Arme}

più il personaggio diventa competente con le armi maggiormente è in grado di sfruttare le occasioni d'attacco ed effettuare manovre d'armi. Ogni qual volta il personaggio esegua almeno due attacchi con armi nel round e \textbf{nessuno dei due vada a segno} è possibile consultare la lista Manovre d'Arme per comprendere quale manovra è possibile usare usando una Reazione.

Ogni Manovra ha indicata quale è la situazione che lo attiva (Attiv.) e quale è l'Effetto.

Può essere indicato anche un Effetto Critico, ovvero l'Effetto che si ha quando si ottiene un Fallimento Critico in almeno un Tiro per Colpire. Purché l'Attivatore sia sempre rispettato, il giocatore può scegliere tra l'Effetto e l'Effetto Critico.

L'Attivatore può specificare un valore pari o dispari che è da confrontare con il Tiro per Colpire.

Le Manovre d'Arme sono raggruppate per livello, ovvero il punteggio di Competenza Armi minima per poter usare quelle manovre, il giocatore può scegliere tra tutte le Manovre d'Arme a lui accessibili ed Attivabili.

\begin{giocatore}[Ho mancato!]
		Sfruttalo come e meglio che potete la vostra sfortuna! Fate che la sorte vi possa sorridere grazie alla vostra abilità nel scegliere la manovra da attivare. Valutate sempre l'ambiente, i nemici, i compagni e la situazione in genere prima di decidere cosa attivare.
\end{giocatore}

\begin{itemize}[leftmargin=*]
	\setlength{\itemsep}{0pt}

	% MANOVRE LIVELLO 4
	\item \textbf{Manovre livello 4}
	\begin{itemize}[leftmargin=*]
		\setlength{\itemsep}{0pt}
		\item \textbf{Schivata Acrobatica} - Attiv.: \textbf{\emph{Mancato}} con un \textbf{\emph{dispari}}. \emph{Effetto}: +1 ai Tiri Salvezza sui Riflessi fino al prossimo turno. \emph{Critico}: +2 ai Tiri Salvezza sui Riflessi.

		\item \textbf{Finta e Riposizionamento} - Attiv.: \textbf{\emph{Mancato}} con un \textbf{\emph{pari}}. \emph{Effetto}: Muovi 1 metro senza provocare attacchi di opportunità. \emph{Critico}: Come sopra ma di muovi 2 metri.

		\item \textbf{Colpo Distraente} - Attiv.: \textbf{\emph{Mancato}}. \emph{Effetto}: +1 alla Difesa contro l'avversario fino al fine del prossimo round. \emph{Critico}: L'avversario considera il personaggio come se avesse copertura leggera.

		\item \textbf{Intralcio Rapido} - Attiv.: \textbf{\emph{Mancato}} con un \textbf{\emph{dispari}}. \emph{Effetto}: L'avversario riduce di 1 metro il suo movimento nel prossimo round. \emph{Critico}: il prossimo movimento entro il prossimo round dell'avversario lo considera come terreno difficile.

		\item \textbf{Apertura Tattica} - Attiv.: \textbf{\emph{Mancato}} con un \textbf{\emph{pari}}. \emph{Effetto}: Un alleato ottiene +1 al prossimo Tiro per Colpire contro l'avversario. \emph{Critico}: L'alleato ottiene +2 al Tiro per Colpire.
	\end{itemize}

	% MANOVRE LIVELLO 6
	\item \textbf{Manovre livello 6}
	\begin{itemize}[leftmargin=*]
		\setlength{\itemsep}{0pt}
		\item \textbf{Reazione Rapida} - Attiv.: \textbf{\emph{Mancato}}. \emph{Effetto}: il personaggio può bere una pozione. \emph{Critico}: Il personaggio può somministrare una pozione ad un alleato adiacente.

		\item \textbf{Finta Avanzata} - Attiv.: \textbf{\emph{Pari}}. \emph{Effetto}: Aggiungi il modificatore di Intelligenza o Saggezza alla Difesa contro l'avversario fino all'inizio del prossimo round. \emph{Critico}: Aggiungi il modificatore di Intelligenza e Saggezza al prossimo Tiro per Colpire contro l'avversario fino all'inizio del prossimo round.

		\item \textbf{Supporto Coordinato} - Attiv.: \textbf{\emph{Pari}}. \emph{Effetto}: Un alleato adiacente ottiene un bonus al Tiro per Colpire pari al modificatore di Intelligenza del personaggio. \emph{Critico}: Il bonus si applica a due alleati a te adiacenti.

		\item \textbf{Contrattacco Immediato} - Attiv.: \textbf{\emph{Dispari}}. \emph{Effetto}: Se il Tiro per Colpire del fallimento avrebbe colpito un avversario adiacente, infliggi danni pari al modificatore di Forza. \emph{Critico}: Infliggi danni pari al doppio del modificatore di Forza.

		\item \textbf{Distrazione Sonica} - Attiv.: \textbf{\emph{Dispari}}. \emph{Effetto}: Un alleato afferrato o bloccato può effettuare una prova per liberarsi. \emph{Critico}: Come sopra e può muoversi di 1 metro.
	\end{itemize}

	% MANOVRE LIVELLO 8
	\item \textbf{Manovre livello 8}
	\begin{itemize}[leftmargin=*]
		\setlength{\itemsep}{0pt}
		\item \textbf{Concentrazione guerriera} - Attiv.: \textbf{\emph{Mancato}}. \emph{Effetto}: +2 ai Tiri per Colpire in mischia fino alla fine del prossimo round. \emph{Critico}: Fino alla fine del prossimo round se colpisci causi un Danno Critico in più.

		\item \textbf{Elusione Perfetta} - Attiv.: \textbf{\emph{Dispari}}. \emph{Effetto}: +2 alla Difesa fino al prossimo turno. \emph{Critico}: Il prossimo attacco in mischia da quell'avversario manca automaticamente.

		\item \textbf{Analisi dell'Avversario} - Attiv.: \textbf{\emph{Dispari}}. \emph{Effetto}: Puoi effettuare una Prova di Conoscenze per ottenere informazioni sull'avversario. \emph{Critico}: Un alleato può effettuare la stessa prova.

		\item \textbf{Terreno Instabile} - Attiv.: \textbf{\emph{Pari}}. \emph{Effetto}: Fino alla fine del prossimo round consideri il terreno come difficile ed hai +4 al Tiro per Colpire. \emph{Critico}: Non puoi muoverti entro la fine del prossimo round ma se colpisci causi due danni critici in più.

		\item \textbf{Attacco a Sorpresa} - Attiv.: \textbf{\emph{Pari}}. \emph{Effetto}: Usando una Reazione puoi effettuare un Attacco con il medesimo Tiro per Colpire dell'ultimo attacco. \emph{Critico}: Se sufficiente il tuo Tiro per Colpire colpisce un altro avversario in mischia.
	\end{itemize}

	% MANOVRE LIVELLO 10
	\item \textbf{Manovre livello 10}
	\begin{itemize}[leftmargin=*]
		\setlength{\itemsep}{0pt}
		\item \textbf{Manovra Laterale} - Attiv.: \textbf{\emph{Mancato}}. \emph{Effetto}: Ti muovi di 1 metro. \emph{Critico}: Ti muovi fino a 4 metri, ma il prossimo round esegui una Azione in meno.

		\item \textbf{Apertura Devastante} - Attiv.: \textbf{\emph{Dispari}}. \emph{Effetto}: Un alleato ottiene +4 al Tiro per Colpire contro l'avversario fino alla fine del tuo prossimo round. \emph{Critico}: Due alleati ottengono il bonus, ma il personaggio subisce -4 al Tiro per Colpire fino alla fine del prossimo round.

		\item \textbf{Intimidazione Superiore} - Attiv.: \textbf{\emph{Dispari}}. \emph{Effetto}: L'avversario subisce -4 al primo attacco contro di te entro la fine del prossimo round. \emph{Critico}: Entro la fine del prossimo round l'avversario in mischia non può fare danno critico contro di te.

		\item \textbf{Ferita Sanguinante} - Attiv.: \textbf{\emph{Pari}}. \emph{Effetto}: L'avversario subisce +1 al sanguinamento. \emph{Critico}: +2 al sanguinamento e il personaggio subisce danni pari al modificatore di Forza.

		\item \textbf{Valutazione Strategica} - Attiv.: \textbf{\emph{Pari}}. \emph{Effetto}: Il prossimo attacco a segno entro fine del round prossimo infligge un danno critico in più. \emph{Critico}: Come sopra e due danni critici ma il round successivo esegui una Azione in meno.
	\end{itemize}

	% MANOVRE LIVELLO 12
	\item \textbf{Manovre livello 12}
	\begin{itemize}[leftmargin=*]
		\setlength{\itemsep}{0pt}
		\item \textbf{Assalto Incessante} - Attiv.: \textbf{\emph{Mancato}}. \emph{Effetto}: Il prossimo round hai un +1 cumulativo al Tiro per Colpire per volta che attacchi. \emph{Critico}: Il prossimo round hai solo 1 Azione. Se la usi per attaccare e colpisci causi 2 danni critici in più.

		\item \textbf{Attacco Predittivo} - Attiv.: \textbf{\emph{Dispari}}. \emph{Effetto}: Il prossimo round il primo attacco portato manca, se colpisci con un attacco successivo infliggi 2 danni critici aggiuntivi. \emph{Critico}: Come sopra ma 3 danni critici ed esegui 1 Azione in meno nel round.

		\item \textbf{Grido di Guerra Potente} - Attiv.: \textbf{\emph{Dispari}}. \emph{Effetto}: Gli alleati entro 9 metri ottengono +1d6 al Tiro per Colpire entro la fine del tuo prossimo round. \emph{Critico}: Come sopra ma +2d6, e il personaggio esegue una sola Azione.

		\item \textbf{Intuizione Letale} - Attiv.: \textbf{\emph{Pari}}. \emph{Effetto}: Il prossimo attacco andato a segno entro la fine del round successivo infligge danni critici massimizzati. \emph{Critico}: Come sopra ma computi 2 danni critici massimizzati, ma esegui una Azione in meno.

		\item \textbf{Furia Incontenibile} - Attiv.: \textbf{\emph{Pari}}. \emph{Effetto}: Confronta il Tiro per Colpire con un avversario adiacente per capire se l'hai colpito, se si sommi anche un danno critico. \emph{Critico}: Fino alla fine del prossimo round hai -4 alla Difesa, +1d6 al Tiro per Colpire e ogni attacco andato a segno causa un danno critico aggiuntivo.
	\end{itemize}

\end{itemize}

%\begin{center}
%	\includegraphics[width=0.2\linewidth]{immagini/Bushido_Calligraphy.png}
%
%	\medskip
%
%	\emph{Trascrizione in kanji di} bushido

%\end{center}

\begin{narratore}[Partecipazione nel bene e nel male]
		Invitate il giocatore crei un suo stile di \emph{fallimento}, fatelo gioire di un \emph{fumble}!
\end{narratore}


%\begin{center}
%	\includegraphics[width=0.95\linewidth]{immagini/fauchard.png}
%
%	\emph{Evoluzione delle armi ad asta}
%\end{center}

\subsection{Cavalcature}\index{Combattimento con saurovallo}\index{Saurovallo}\label{cavalcature}\hypertarget{cavalcare}{}\label{cavalcare}

\begin{enfasi}{
- E ti puoi trovare un'altra moglie!

- Ah, questo sì. ma il guaio è che mi ha portato via il fucile e il cavallo! Peccato, era così bella, io mi ci ero affezionato. Le davo qualche frustata, ma lei non ci faceva caso.

- Chi, tua moglie?

- No, la mia cavalla. A trovare un'altra moglie si fa presto, ma una cavalla come quella non la ritrovo più. (Ombre rosse, film 1939)}\end{enfasi}

Per comandare una cavalcatura è necessario avere la competenza Cavalcare, altrimenti è solo possibile dare la direzione del movimento.

Una cavalcatura ha 2 Azioni e di norma sono usate per spostarsi o per reagire ed ubbidire ai tuoi comandi.

Una cavalcatura agisce nel tuo round e sei tu a decidere quando esegue le sue Azioni rispetto alle tue. Non tira l'iniziativa, usa la tua.

Gli attacchi verso un personaggio su un saurovallo (o cavalcatura in genere) se non dichiarati diversamente mirano al cavaliere e non al saurovallo.

\subsubsection{Situazioni e regole}\label{cavallosituazioniregole}

\begin{itemize}[leftmargin=*] \setlength{\itemsep}{0pt}
\item
Ogni qual volta la cavalcatura è colpita il cavaliere deve effettuare una prova di Cavalcare a DC 15 o essere disarcionato.

Se la cavalcatura è da guerra, addestrata al combattimento, la prova di Cavalcare ha difficoltà 12.

\item
Combattere da posizione sopraelevata concede un +2 al Tiro per Colpire se l'avversario non è alla tua altezza.

\item
Salire o Scendere dalla cavalcatura costa 1 Azione se si ha la competenza Cavalcare, altrimenti 2 Azioni.

\item
Se una magia o situazione sposta, bruscamente, la cavalcatura contro la tua volontà devi effettuare un Tiro Salvezza su Riflessi a DC 15 oppure una prova di Cavalcare (DC 15) o venire disarcionato.

\item
Se si è disarcionati si cade a terra proni e si subiscono 1d6 di danno.
\end{itemize}

\subsubsection{Controllare una Cavalcatura}\label{controllocavalcatura}

Mentre sei in sella, hai due scelte: dai ordini alla tua cavalcatura oppure gli permetti di agire da sola.

Cavalcature particolarmente intelligenti tendono a privilegiare l'autonomia di azione piuttosto che essere comandati.

Puoi controllare una cavalcatura solo se questa è stata addestrata ad accettare un cavaliere. Si presume che saurovalli addestrati da guerra creature abbiano ricevuto tale addestramento.

Spendendo 1 tua Azione puoi fare eseguire 2 di queste Azioni alla cavalcatura: Muoversi, Attaccare, Disingaggiare.

Se la cavalcatura è intelligente questa potrebbe agire e muoversi come preferisce a discapito delle indicazioni del cavaliere. Potrebbe fuggire dal combattimento, lanciarsi all'attacco e divorare un nemico ferito gravemente, o agire in qualche altro modo contro la volontà di chi la cavalca.

\end{multicols}

\vfill

\begin{figure}[h]
		\centering
	\begin{minipage}{0.45\textwidth}
		\centering
		\includegraphics[width=\textwidth]{immagini/napoleone.png}  % Sostituisci con il nome del file
\emph{Cavallo bianco, con Napoleone Bonaparte, mentre valica il Gran San Bernardo. (Jacques-Louis David, 1801, Castello di Malmaison)}
	\end{minipage}
	\hspace{0.5cm}
	\begin{minipage}{0.45\textwidth}
		\centering
		\includegraphics[width=\textwidth]{immagini/saurovallo1-ai.png} % Sostituisci con il nome del file
\emph{Saurovallo addestrato, senza Napoleone Bonaparte, con sella e finimenti, da qualche parte nella fu Italia (B.I.C.)}
	\end{minipage}
\end{figure}

\bigskip

\begin{enfasi}
Artax galoppava attraverso la Palude della Tristezza, e ad ogni passo i suoi zoccoli affondavano più profondamente.(La Storia Infinita, Michael Ende)

\medskip

Il cavallo conosce la strada verso casa anche quando il cavaliere ha smarrito la via. (Le Tombe di Atuan, Ursula K. Le Guin)

\end{enfasi}


\pagebreak

