\section{Ambiente}\index{Ambiente}

\label{ambiente}
\begin{changemargin}{0.3cm}{0.3cm}\begin{enfasi}{
La natura non è crudele, è solo spietatamente indifferente. Questa è una delle più dure lezioni che un essere umano debba imparare. (Richard Dawkins)

\medskip

L'antidoto principale contro un cattivo ambiente consiste, naturalmente, nel sostituirlo con uno buono. (Robert Baden-Powell)}\end{enfasi}\end{changemargin}\medskip

\begin{multicols}{2}

Dai deserti senza vita ai dungeon zeppi di trappole, l'ambiente aiuta a definire il mondo, renderlo vivo, dinamico e ricco. Consente di creare un'esperienza di gioco emozionante e coinvolgente.

\subsection{Regole Ambientali}

\label{regole-ambientali}

\subsubsection{Visione e Luce}\index{Visione}\index{Luce}\hypertarget{visioneeluce}{}\label{visioneeluce}

\begin{changemargin}{0.3cm}{0.3cm}\begin{narratore}
Il diverso funzionamento delle fonti di luce vuole rendere più cupo, oscuro e difficile l'esplorazione, specialmente di caverne e zone prive di fonti luminose. Basta gruppi che castano Luce ogni minuto o gridano \emph{Scurovisione!}. L'oscurità aiuta l'immaginazione ed alza il livello di tensione. Enfatizzate il crepitare della fiamma della torcia, l'ondeggiare e a volte quasi spegnersi per le correnti improvvise. Rendete misterioso ciò che c'è intorno ai personaggi!.
\end{narratore}\end{changemargin}

\label{sec:visione-e-luce}

In un ambiente naturale l'illuminazione può assumere diverse gradazioni e queste gradazioni aiutano a capire fino a che distanza una creatura può vedere.

Le gradazione di luce possono essere:
\begin{itemize}[leftmargin=*] \setlength{\itemsep}{0pt}
\item
\textbf{Oscurità}': buio pesto, può essere naturale o magico
\item
\textbf{Luce Fioca / Scarsa / Penombra / Oscurata leggermente}: poca illuminazione, permette di riconoscere le sagome\index{Oscurata leggermente}\index{Luce Fioca}
\item
\textbf{Luce}: luce intensa, luce brillante, coprente, assolata
\end{itemize}

Saranno le fonti di luce, o la loro assenza, a stabilire quanta illuminazione c'è e fino a che distanza. La Tabella Fonti di Luce indica per le più comuni fonti di luce il raggio illuminato a pieno, quello illuminato in maniera inferiore (Luce Fioca) e la durata.

Molti incantesimi ed oggetti usano come durata il \emph{tempo di gioco reale}, ovvero non si contano i round o turni per stabilire la durata bensì si segna sulla scheda l'orario di accensione della torcia, lanterna, incantesimo. Un altro metodo può essere quello di impostare un timer nello smartphone. In questa maniera risulterà più facile la gestione e maggiore l'attenzione alle risorse consumabili.

\medskip

\textbf{Tabella: Fonti di luce}\index[Tabelle]{Tabella delle fonti di luce}\label{fontidiluce}

\medskip

\index{Luce Fioca}

\noindent\begin{tabular}{l|cc|c}
\textbf{Fonte di} &\multicolumn{2}{c}{\textbf{Raggio in metri}}& \textbf{Durata} \\
\textbf{Luce}& \textbf{Luce} & \textbf{Luce Fioca} &\\
\toprule
\hyperlink{Candela}{Candela} & - & 1 m & 1 ora\\
\hyperlink{Torcia}{Torcia} & 3 m & 6 m & 1 ora\\
\hyperlink{Lanterna}{Lanterna} & 6 m & 12 m & 3 ore \\
\multicolumn{4}{c}{\textbf{Incantesimi}}\\
\hyperlink{Lacrima di Ljust}{Lacrima di Ljust} & 1 m & - & 10 round\\
\hyperlink{Luce}{Luce}& 3 m & 6 m &30 min. \\
\hyperlink{Luce Diurna}{Luce Diurna} & 6 m & 12 m & 1 ora
\end{tabular}

\smallskip

La durata indicata è espressa, quando in minuti o ore, come durata di tempo reale di gioco.\index{Durata Luce}

\medskip

\begin{center}
\includegraphics[width=0.8\linewidth]{immagini/oscurita.png}

\emph{Henry Justice Ford}
\end{center}

\medskip

La \textbf{Luce fioca}\index{Luce Fioca} è la luce oltre una fonte di luce. E' il passare in un corridoio di 3 metri se è illuminato solo da leggere candele, è una notte di luna piena, è una zona oscurata leggermente.
In linea di massima una fonte di luce crea luce fioca in un raggio doppio rispetto al raggio di luce normale. \textbf{Una creatura in Luce Fioca ha un -2 alle prove di Consapevolezza ed un -1 ai Tiri per Colpire}.\index{Luce Fioca penalita' TC}

\medskip

\textbf{Oscurità}\index{Oscurità}: è il buio più completo senza alcuna fonte di luce. Per creature con visione normale l'oscurità è ciò che c'è oltre la Luce fioca.
Il \textbf{personaggio cieco}\index{Cieco} o che combatte nell'oscurità (e non può vedere nell'oscurità) ha -1d6 alla Consapevolezza e tutti gli avversari sono \hyperlink{invisibilita}{invisibili} (vedi pag. \pageref{invisibilita}).

\medskip

La \textbf{Luce}\index{Luce} è la luce all'aperto sotto il sole, ma anche se si tiene una torcia in mano o in un corridoio illuminato da lanterne. Se le fonti di luci non si susseguono si creano zone di luce fioca se non oscurità.

\subsubsection{Tipi di Visione e Illuminazione}

\begin{itemize}[leftmargin=*] \setlength{\itemsep}{0pt}
\item
Una creatura con \textbf{Visione Normale} \index{Visione Normale}vede fino alla distanza, come raggio circolare intorno alla fonte di luce, indicato in Luce. Oltre è Luce Fioca e oltre ancora Oscurità.

\item
Una creatura con \textbf{Visione Crepuscolare} \index{Visione Crepuscolare}vede senza problemi fino alla distanza, come raggio circolare intorno alla fonte di luce, indicato in Luce fioca, o indicato dalla razza se minore, oltre è oscurità.

\item
Una creatura con \textbf{Scurovisione} \index{Scurovisione} vede, entro il raggio indicato dalla sua Scurovisione, in condizioni di luce normale e luce fioca senza problemi, mentre nell'oscurità ha -2 Consapevolezza ed a Sopravvivenza per cercare trappole. La Scurovisione è una visione in bianco e nero.
\end{itemize}

\begin{changemargin}{0.3cm}{0.3cm}\begin{tcolorbox}[title = Nota sulle fonti di luce]
Vi sarete accorti o lo farete presto, che le fonti di luce magiche funzionano in maniera diversa, molto spesso durano molto di meno o generano poca luce. Questo è dovuto al volere di un Patrono e come tale solo un Patrono può annullarne gli effetti (o il Narratore!).
\end{tcolorbox}\end{changemargin}

\subsubsection{Buio}\index{Buio}

\label{buio}

Le torce e le lanterne possono essere spente all'improvviso da una folata di vento, le fonti di luce magiche possono essere dissolte o contrastate ed alcune trappole magiche possono creare aree di buio impenetrabile.

In certi casi alcuni personaggi o mostri potrebbero essere in grado di vedere mentre gli altri sono Accecati. Ai fini delle regole che seguono, una creatura Accecata è semplicemente una creatura che non è in grado di vedere ciò che la circonda.

\subsubsection{Accecato}\index{Accecato}\index{Invisibile}

\label{accecato}

Le creature Accecate perdono la loro capacità di infliggere danni extra causati ad esempio dall'Abilità di pugnalare alle spalle (ma non da esplosione del danno o critico al colpire).

Le creature accecate considerano il terreno come difficile \index{Muoversi al buio}. Devono effettuare una prova di Acrobatica con DC 12 per Azione di Movimento per muoversi a velocità normale. Se la prova fallisce cadono a terra prone. Le creature accecate non possono caricare.

Una creatura accecata, o che combatte contro una creatura invisibile,\index{Invisibile} può effettuare una prova di Consapevolezza a difficoltà 20 (oppure 10+ prova di Furtività dell'avversario se questo non vuole farsi trovare) per individuare la creatura purché questa sia entro 6 metri dal personaggio.

Una creatura Accecata \index{Accecata}subisce penalità di -2 alle Prove di Competenza basate su Forza e Destrezza e fallisce automaticamente qualsiasi prova di Consapevolezza dipenda dalla vista.

Inoltre, una creatura accecata non può usare incantesimi che prevedano l'uso dello sguardo ed è immune agli incantesimi che prevedono lo sguardo.

Per i modificatori in dettaglio vedi in \hyperlink{invisibilita}{Invisibilità} (pag \pageref{invisibilita}).

\begin{center}
\includegraphics[width=0.75\linewidth]{immagini/oggetticadenti.png}

\emph{Henry Justice Ford}
\end{center}

\subsubsection{Cadute}\index{Cadute}\index{Cadere}\hypertarget{cadute}{}\label{cadute}

Le creature che cadono si fanno male. Dividi l'altezza di caduta (in metri) per 3, arrotonda per difetto, il numero che risulta sono i d6 di danno subiti. Es 16 metri di caduta sono 16/3=5d6 di danno. Per praticità si suggerisce di applicare 1 danno ogni metro di caduta.\index{Danno da caduta}

Le creature che subiscono danni da una caduta, atterrano in posizione \textbf{prona}.

Una prova, usando una Reazione, di Acrobatica riuscita con DC 15 permette al personaggio di ridurre il danno da caduta di 3. Per ogni punto oltre il 15 nella prova riduce di un ulteriore 1 il danno. La caduta deve essete entro 6 metri.

Quando Acrobatica raggiunge punteggio 6 la riduzione si applica a cadute entro 9 metri, a punteggio 9 a cadute entro 12 metri.

Cadute su superfici morbide (terreno morbido, fango ecc.) riducono di 3 i danni.

Un personaggio termina una Azione di Movimento con una caduta, ma solo se non si è fatto danni può proseguire con la stessa Azione, altrimenti prima deve alzarsi da prono.

In un round di caduta libera si precipita di 150 metri (50d6 oppure 150 di danno), al termine del primo segmento cade a 20 metri, poi a 80m poi a 150m. Un personaggio non può lanciare incantesimi mentre cade, a meno che la caduta non sia superiore o pari a 100 metri. Si è Distratti mentre si prova a lanciare un incantesimo mentre si cade.\index{Lanciare incantesimi mentre si cade}

\medskip

\noindent \textbf{Cadere in Acqua}\index{Cadere in Acqua}

Le cadute in acqua sono gestite in modo leggermente diverso. Fino a quando l'acqua ha una profondità di almeno di 3 metri ed il tuffo è da una altezza entro 12 metri non si subiscono danni.

Per determinare il danno da caduta in acqua sottraete all'altezza di caduta 12 metri, aggiungete 1d6 di danno per ogni 3 metri rimanenti ($((H-12)/3)*1d6)$).

I personaggi che si tuffano volontariamente in acqua non subiscono danni se superano una prova di Nuotare con DC 15 e se l'acqua è profonda almeno 6 metri. La DC della prova aumenta di 1 ogni metro oltre i 12 metri di altezza.

\subsubsection{Effetti dell'Acido}\index{Acido}

\label{effetti-dellacido}

Gli acidi corrosivi infliggono 1d6 danni per round di esposizione, tranne nel caso di totale immersione (come in una vasca d'acido) che infligge 10d6 danni per round. Un attacco con l'acido, come quello di una boccetta lanciata o la saliva/soffio di un mostro, deve essere considerato come un round di esposizione.

I vapori prodotti dalla maggior parte degli acidi sono equivalenti a veleni inalati. Coloro che si avvicinano ad un grosso ammasso di acido devono effettuare un Tiro Salvezza su Tempra con DC 13 o subiranno 1 danno temporaneo alla Costituzione per round di esposizione. Questo veleno non ha frequenza, pertanto una creatura è salva se si allontana dall'acido.

Le creature immuni alle proprietà caustiche dell'acido potrebbero comunque annegare se vi vengono totalmente immerse (vedi Annegamento).

\subsubsection{Effetti del Fumo}\index{Fumo}

\label{effetti-del-fumo}

Un personaggio costretto a respirare del fumo denso deve superare un Tiro Salvezza su Tempra ogni round (DC 15, +1 per ogni prova precedente) oppure passa il round a tossire e soffocare. Un personaggio che continua a soffocare per 2 round consecutivi subisce 1d6 Danni Non Letali per ulteriore round di esposizione. Il fumo oscura la vista, fornendo Copertura leggera (+2 Difesa) ai personaggi che si trovano al suo interno.

\subsubsection{Fame e Sete}\index{Fame}\index{Sete}

\label{fame-e-sete}

I personaggi potrebbero trovarsi senz'acqua o cibo e privo dei mezzi per procurarsene. Nei climi normali, i personaggi di taglia Media hanno bisogno di almeno 2 litri di liquidi e 0.5 kg di cibo decente al giorno per evitare la fame, i personaggi di taglia Piccola necessitano della metà. Nei climi molto caldi i personaggi possono aver bisogno di due o tre volte quella quantità d'acqua per evitare la disidratazione.

Ogni giorno senza cibo è necessario effettuare un Tiro Salvezza su Tempra a difficoltà 11 +1 per giorno senza cibo, se non si ha da bere la difficoltà aumenta di +3.

Se si fallisce il Tiro Salvezza si subiscono 1d4 di danno e si diventa sempre più Affaticati. Le penalità date dall'affaticamento rimangono finché non si mangia e beve abbastanza.

\subsubsection{Oggetti cadenti}\index{Oggetti Cadenti}\index{Caduta oggetti}

\label{oggetti-cadenti}

Proprio come i personaggi subiscono danni dalle cadute superiori ai 3 metri, allo stesso modo subiscono danni se vengono colpiti da oggetti cadenti.

Gli oggetti che cadono addosso ai personaggi infliggono danni a seconda del loro peso e della distanza da cui sono caduti.

La \textbf{Tabella: Danno da Oggetti Cadenti} determina la quantità di danni inflitti da un oggetto in base alla sua taglia. Si presume che l'oggetto sia fatto di un materiale denso e pesante, come la pietra.
Gli oggetti fatti di materiali più leggeri potrebbero infliggere la metà o meno del danno indicato, a discrezione del Narratore. Per esempio un masso Enorme che colpisce un personaggio infligge 6d6 danni, mentre un carro di legno potrebbe infliggerne solo 3d6.

Inoltre, se l'oggetto cade da una distanza inferiore ai 3 metri, infligge la metà dei danni indicati. Se un oggetto cade da una distanza superiore ai 20 metri, infligge danni raddoppiati. L'oggetto che cade subisce la stessa quantità di danni che infligge.

\bigskip

\textbf{Tabella: Danno da Oggetti Cadenti}\index[Tabelle]{Tabella Danno da Oggetti Cadenti}

\medskip

\begin{tabular}{ll}
\textbf{Taglia dell'Oggetto} & \textbf{Danno}\\
\toprule
Minuscola o più Piccola & 1d6\\
Piccola & 2d6\\
Media & 3d6\\
Grande & 4d6\\
Enorme & 6d6\\
Mastodontica & 8d6\\
Colossale & 10d6
\end{tabular}

\bigskip

Lasciar cadere addosso ad una creatura un oggetto richiede un attacco a tocco a distanza (vedi \hyperlink{attaccoatocco}{Attacco a Tocco}, pag. \pageref{attaccoatocco}). Questi attacchi hanno di solito una gittata di 3 metri. Se un oggetto cade su una creatura la creatura deve effettuare, se colpita, un Tiro Salvezza su Riflessi con DC 15 per dimezzare il danno se è consapevole dell'oggetto che sta cadendo. Gli oggetti cadenti che sono parte di una trappola usano le regole relative alle trappole invece che quelle qui descritte.

\subsubsection{Pericoli dell'Acqua}\index{Pericoli dell'Acqua}\index{Acqua}\hypertarget{pericoli-dellacqua}{}\label{pericoli-dellacqua} \index{Nuotare}

Qualsiasi personaggio può attraversare acque relativamente calme che non abbiano una profondità maggiore alla sua altezza, senza bisogno di prove.

Una creatura con una velocità di Nuotare può muoversi attraverso l'acqua alla sua velocità indicata senza fare prove di Nuotare. Ha un bonus di +2d6 su qualsiasi prova di Nuotare per eseguire un'azione particolare o evitare un pericolo.
La creatura può sempre scegliere di prendere 10 su una prova di Nuotare, anche se distratti od in pericolo quando si nuota. Non può prendere il 10 solo in caso di acque tempestose. Una tale creatura può utilizzare l'azione di correre mentre nuota, a condizione che nuoti in linea retta.

Un incantatore si considera Distratto se lancia un incantesimo mentre è in acqua.

Se non si ha il tipo di movimento Nuotare \textbf{muoversi nell'acqua} si considera come \textbf{\emph{terreno} difficile}, e quindi ci si muovo a metà della velocità indicata da movimento.

Se la creatura sa nuotare non sono necessarie prove per muoversi normalmente in acque calme, tranne in cui voglia \emph{correre} (DC 13) o le acque siano mosse (DC 15) o siano tempestose (DC 20).

\medskip

\begin{center}
	\includegraphics[width=0.7\linewidth]{immagini/affogare.png}

	\emph{Henry Justice Ford}\end{center}

\medskip

Se la creatura non sa nuotare allora deve fare una Tiro Salvezza su Tempra a DC 13 ogni round che vuole muoversi, se le acque sono mosse la DC è 19 e se sono tempestose la DC è 24, se si vuole \emph{correre} la DC aumenta di 4.
In caso di fallimento non ci si sposta e si ha un -1 alla prova successiva, in caso di Fallimento Critico la successiva prova prende -4, le penalità si cumulano finché non si riesce nella prova.
Se la prova di nuotare fallisce la creature subisce 1d6 danni letali se le acque scorrono sopra rocce e avvallamenti.

Quando le penalità cumulate sono 9 o più si incomincia ad affondare ed annegare (vedi sotto).

L'acqua molto profonda non è solo nera come la pece ma infligge danni ancora peggiori a causa della pressione nell'ordine di 1d6 danni al minuto ogni 30 metri che separano il personaggio dalla superficie. Un Tiro Salvezza su Tempra superato con successo (DC 15, +1 per ogni prova precedente) indica che il personaggio immerso non subisce danni in quel minuto. L'acqua, oltre i 150 metri di profondità, è molto fredda ed infligge 1d6 Danni Non Letali per minuto di esposizione a causa dell'ipotermia.

\medskip

\textbf{Annegamento / Trattenere il fiato}\index{Annegamento}\index{Affogare}\index{Soffocare}\hypertarget{trattenereilfiato}{}\label{trattenereilfiato}\index{Trattenere il fiato}

\medskip

Qualsiasi personaggio può trattenere il fiato per un numero di round pari 10 + 10 round per il suo punteggio di Costituzione, con un minimo di 10 round. Per ogni Azione compiuta la durata restante diminuisce di 1 round, lanciare un incantesimo con componenti Verbali consuma 3 round di aria in più. Trascorso questo periodo di tempo, il personaggio deve effettuare un Tiro Salvezza su Tempra con DC 12 ogni round per continuare a trattenere il fiato. Ogni round, la DC aumenta di 2.

L'incantatore che lanci incantesimi sott'acqua si considera Distratto.

Se il Tiro Salvezza fallisce il personaggio va immediatamente a 0 Punti Ferita e sviene. Dal round successivo incomincia a perdere 1 Punto Ferita a round fino alla morte (o alla rianimazione!)

Si può annegare in sostanze diverse dall'acqua, come la sabbia, le sabbie mobili, la polvere molto fine o un silos pieno di farro o semplicemente trattenendo il respiro.

\subsubsection{Pericoli del Caldo}\index{Caldo}

\begin{center}
	\includegraphics[height=0.65\linewidth]{immagini/desert.png}
\end{center}

\label{pericoli-del-caldo}

Una creatura sottoposta a temperature molto elevate (sopra i 40° C) deve superare un Tiro Salvezza su Tempra ogni ora (DC 15, +1 per ogni prova precedente) oppure subisce 1d4 danni Non Letali. Se indossa abiti pesanti o qualsiasi tipo di armatura, subisce penalità -1d6 a questi Tiri Salvezza. Un personaggio somma i suoi punti assegnati in Sopravvivenza e può dare un bonus ai compagni pari alla metà del valore per lo stesso Tiro Salvezza. I personaggi Privi di Sensi iniziano a subire danni letali (1d4 danni all'ora).

Un personaggio che subisce Danni Non Letali a causa dell'esposizione al caldo, è soggetto ad un colpo di calore ed è Affaticato. Queste penalità terminano quando il personaggio recupera i Danni Non Letali subiti a causa del caldo.

Il caldo infernale (temperatura dell'aria sopra i 60° C, fuoco, acqua che bolle, lava) infligge danni letali. Respirare l'aria con queste temperature infligge 1d6 danni da fuoco al minuto (senza Tiro Salvezza).

L'acqua bollente infligge 1d6 danni da scottatura, a meno che il personaggio non vi venga completamente immerso, nel qual caso subirebbe 10d6 danni per round di esposizione.

\subsubsection{Prendere Fuoco}\index{Prendere Fuoco}\index{Fuoco}

\label{prendere-fuoco}

I personaggi esposti ad olio bollente, fuochi da campo, fuochi magici non istantanei possono vedere i loro abiti, capelli o equipaggiamento prendere fuoco. Gli incantesimi specificano se sono in grado di appiccare il fuoco.

I personaggi che rischiano di prendere fuoco possono effettuare un Tiro Salvezza su Riflessi con DC 15 per evitare questo fato. Se i vestiti o i capelli di un personaggio prendono fuoco, egli subisce immediatamente 1d6 danni. Per ogni round successivo il personaggio in fiamme deve effettuare un altro Tiro Salvezza su Riflessi. Il fallimento indica che subisce altri 1d6 danni in quel round. Il successo indica che il fuoco si è estinto (una volta che supera il Tiro Salvezza non sta più andando a fuoco).

\begin{center}
	\includegraphics[width=0.7\linewidth]{immagini/fuocopericolo.png}
\end{center}

Un personaggio che va a fuoco può estinguere automaticamente le fiamme saltando dentro a dell'acqua sufficiente a spegnerle. Se non ci sono grosse quantità d'acqua a disposizione, rotolarsi sul terreno o smorzare la fiamma con mantelli o simili può concedere al personaggio +1d6 al Tiro Salvezza.

Il personaggio in fiamme deve fare un Tiro Salvezza su Riflessi a DC 15 per ogni oggetto portato, se fallisce gli oggetti subiscono la stessa quantità di danni del personaggio.\\

\textbf{Effetti della Lava}\index{Lava}\\

La lava o il magma infliggono 2d6 danni per round di esposizione, tranne in caso di totale immersione (come quando un personaggio cade nel cratere di un vulcano attivo), che infligge 20d6 danni per round (più eventuali danni da caduta e magari trova un anello..).

I danni provocati dal magma continuano per 1d3 round dopo il termine dell'esposizione, ma questi danni addizionali sono solo la metà di quelli inflitti durante l'ultimo round di effettivo contatto (20/10/5). Un'Immunità o una Resistenza al fuoco serve anche come resistenza alla lava od al magma. Tuttavia, le creature Immuni o Resistenti al Fuoco potrebbero annegare se immerse nella lava (vedi Annegamento).

\subsubsection{Pericoli del Freddo}\index{Freddo}

\label{pericoli-del-freddo}

I personaggi non ben vestiti in climi freddi (sotto i 5° C) devono superare un Tiro Salvezza su Tempra ogni ora (DC 15, +1 per ogni prova precedente) oppure subiscono 1d6 Danni Non Letali.
In condizioni di freddo estremo o di esposizione sotto i -17° C, un personaggio non adeguatamente vestito deve effettuare un Tiro Salvezza su Tempra ogni 10 minuti (DC 15, +1 per ogni prova precedente), subendo 1d6 Danni Letali per ogni Tiro Salvezza fallito. I personaggi che indossano abiti invernali hanno bisogno di effettuare la prova per il freddo e l'esposizione solo una volta all'ora.

\begin{center}
	%\includegraphics[height=0.7\linewidth]{immagini/snowfall.png}
	\includegraphics[height=0.6\linewidth]{immagini/Forest_road_Slavne_2017_G4_grayscale.png}
\end{center}

Un personaggio somma i punti assegnati in Sopravvivenza ai Tiri Salvezza e può dare un bonus ai compagni pari alla metà del valore per lo stesso Tiro Salvezza.

Un personaggio che subisce Danni Non Letali a causa del freddo o dell'esposizione, è soggetto ai geloni o all'ipotermia (considerarlo come Affaticato). Queste penalità terminano quando il personaggio recupera dai Danni Non Letali subiti a causa del freddo e dell'esposizione.

Le condizioni di freddo intollerabile o di esposizione (sotto i -28° C) infliggono ai personaggi 1d6 danni letali per minuto (senza alcun Tiro Salvezza) se non specificatamente protetti.

\subsubsection{Effetti del Ghiaccio}\index{Ghiaccio}

I personaggi che camminano sul ghiaccio è come se camminassero su terreno difficile. Il movimento è dimezzato, eventuali prove di Acrobatica hanno un aumento di difficoltà +4. I personaggi che sono per lungo tempo a contatto con il ghiaccio potrebbero subire dei danni da freddo estremo.

\subsubsection{Soffocamento Lento}\index{Soffocamento}

Un personaggio di taglia Media può respirare tranquillamente per circa 6 ore in una camera sigillata che misura 3 metri di lato. Dopo questo tempo, subisce 1d6 Danni Non Letali ogni 15 minuti. Ogni ulteriore personaggio di taglia Media oppure ogni fuoco significativo (una torcia, per esempio) riducono proporzionalmente la durata dell'aria respirabile. Una volta privi di sensi per l'accumulo di Danni Non Letali, i personaggi iniziano a subire Danni Letali allo stesso ritmo. I personaggi di taglia Piccola consumano metà dell'aria dei personaggi di taglia Media.

\subsection{Tempo Atmosferico - Meteo}\index{Meteo}

\label{tempo-atmosferico---meteo}

A volte il tempo atmosferico può giocare un ruolo importante nel corso di un'avventura. La Tabella: Tempo Atmosferico Casuale è una tabella generica che può essere utilizzata per stabilire le condizioni atmosferiche locali. I termini della tabella sono definiti qui di seguito:

\end{multicols}

\medskip

\textbf{Tabella: Tempo Atmosferico Casuale}\index[Tabelle]{Tabella Tempo Atmosferico Casuale}

\medskip

\noindent\begin{tabularx}{1\textwidth}{llXXl}
\textbf{d\%} & \textbf{Meteo} & \textbf{Clima Freddo}& \textbf{Clima Temperato {*}} & \textbf{Deserto}\\
\toprule
01-70 & Normale& Freddo, calmo & Normale per la stagione {*}{*} & Torrido, calmo\\
71-80 & Anormale & Ondata di Caldo (01-30) / Ondata di Freddo (31-100)&Ondata di Caldo (01-50) - Ondata di Freddo (51-100)& Torrido, ventilato \\
81-90 & Inclemente & Precipitazioni (neve)& Precipitazioni normali per la stagione& Torrido, ventilato \\
91-99 & Tempesta & Tempesta di neve& Tempesta di fulmini / Tempesta di neve& Tempesta di polvere \\
100& Tempesta violenta& Tormenta & Bufera, tormenta, uragano, tornado & Acquazzone
\end{tabularx}

\medskip

* Temperato comprende foreste, colline, paludi, montagne, pianure e zone marine calde.

** L'inverno è freddo, l'estate è calda, l'autunno e la primavera sono moderati. Le paludi sono sempre leggermente più calde d'inverno.

\begin{multicols}{2}

%\begin{center}

%	\includegraphics[width=0.95\linewidth]{immagini/Paesaggio-pioggia-Auvers.png}

%	\emph{Vincent van Gogh, Paesaggio sotto la pioggia ad Auvers, 1890, olio su tela, cm 50 x 100}
%\end{center}

\textbf{Acquazzone}: Considerarlo come pioggia (vedi Precipitazioni sotto), ma offre copertura come la nebbia. Può provocare inondazioni e dura di solito 2d4 ore.

\textbf{Caldo}: La temperatura è tra 15° e 30° C di giorno, e tra 6 e 11 gradi in meno di notte.

\textbf{Calmo}: Vento leggero (tra 0 e 15 km/h).

\textbf{Freddo}: Temperatura tra -17° e 5° C durante il giorno, e tra 6 a 11 gradi in meno di notte.

\textbf{Moderato}: Temperatura tra i 5° e i 15° C durante il giorno, e tra 6 e 11 gradi in meno di notte.

\textbf{Ondata Caldo}: Fa aumentare la temperatura di 6° C.

\textbf{Ondata Freddo}: Abbassa la temperatura di 6° C.

\textbf{Precipitazioni}: Tirare un d100 per determinare se la precipitazione è nebbia (01-30), pioggia/neve (31-90), o nevischio/ grandine (91-00). La neve e il nevischio si verificano solo quando la temperatura è di 0° C o inferiore. La maggior parte delle precipitazioni dura 2d4 ore. La grandine, invece, dura solo 3d6 minuti ma di solito è accompagnata da 1d4 ore di pioggia.

\textbf{Tempesta} (di Fulmini/di Neve/di Polvere): Il vento è molto forte (da 45 a 75 km/h) e la visibilità è ridotta di tre quarti. Le tempeste durano 2d4-1 ore. Vedi Tempeste, sotto, per ulteriori dettagli.

\textbf{Tempesta} (Bufera/Tormenta/Uragano/Tornado): La velocità del vento è superiore ai 75 km/h (vedi Tabella: Effetti del Vento). Inoltre, le tormente sono accompagnate da pesanti nevicate (1d3 \texttimes{} 30 cm), e gli uragani sono accompagnati da acquazzoni. Le bufere durano 1d6 ore, le tormente 1d3 giorni. Gli uragani possono durare fino a una settimana, ma l'impatto maggiore per i personaggi avverrà in un periodo di tempo tra le 24 e le 48 ore, mentre il centro della tempesta si sposta nella loro zona. I tornado durano molto poco (1d6 \texttimes{} 10 minuti), e di solito si formano come parte di una tempesta di fulmini.

\textbf{Torrido}: Temperatura tra i 30° e i 43° C durante il giorno e tra 6 e 11 gradi in meno di notte.

\textbf{Ventilato}: La velocità del vento va da moderata a forte (da 15 a 45 km/h); vedi Tabella: Effetti del Vento.

\textbf{Pioggia, Neve, Nevischio e Grandine}: La brutta stagione frequentemente rallenta o blocca i trasporti via terra e rende praticamente impossibile la navigazione. acquazzoni torrenziali e bufere oscurano la visuale tanto quanto lo farebbe una nebbia densa.

La maggior parte delle precipitazioni si manifesta come pioggia, ma nei climi freddi possono manifestarsi anche come neve, nevischio o grandine. Le precipitazioni di qualsiasi tipo, seguite da un calo della temperatura da sopra a sotto gli 0° C possono produrre ghiaccio.

\textbf{Pioggia intensa}\index{Pioggia intensa}: La pioggia dimezza la visibilità, e impone penalità -1d6 alle prove di Consapevolezza. Ha lo stesso effetto di un vento molto forte sulle fiamme, sugli attacchi con armi a distanza e sulle prove di Consapevolezza come vento molto forte.

\textbf{Neve}\index{Neve}: Mentre cade, la neve ha gli stessi effetti della pioggia su visibilità, attacchi con armi a distanza e prove di Consapevolezza ed il terreno è considerato difficile. Una nevicata della durata di un giorno lascia al suolo 3d6*2.5 centimetri di neve.

\textbf{Neve Fitta}: Una fitta nevicata ha gli stessi effetti di una nevicata normale, ma oscura la visibilità come la nebbia (vedi Nebbia). Un giorno di neve fitta lascia sul terreno 2d4 x 30 centimetri di neve ed il terreno viene considerato doppiamente difficile (movimento/4). Una fitta nevicata accompagnata da venti forti o molto forti può dare origine a cumuli di neve profondi 1d4 x 1 metro, specialmente sopra e intorno ad oggetti abbastanza grandi da deflettere il vento (una capanna o una grande tenda, per esempio).
C'è una probabilità del 10\% che una nevicata fitta sia accompagnata da fulmini (vedi Tempesta di Fulmini). La neve ha gli stessi effetti del vento moderato sulle fiamme.

\textbf{Nevischio}: Si tratta fondamentalmente di pioggia congelata, che ha gli stessi effetti della pioggia quando cade (eccetto che la probabilità di estinguere fiamme protette è del 75\%) e quelli della neve una volta depositatasi.

\textbf{Grandine}: La grandine non riduce la visibilità, ma il suono della grandine che cade rende più difficili le prove di Consapevolezza basate sull'udito (penalità -1d6). A volte (probabilità del 5\%) la grandine può essere talmente grossa da infliggere 1 danno letale (per tempesta) a qualsiasi cosa si trovi all'aperto. Una volta depositata, la grandine ha lo stesso effetto della neve sul movimento.

\subsubsection{Tempeste}\index{Tempeste}

\label{tempeste}

Gli effetti combinati delle precipitazioni (o della polvere) e del vento, che accompagnano tutte le tempeste, riducono la visibilità di tre quarti, imponendo penalità -8 a tutte le prove di Consapevolezza. Le tempeste rendono impossibili gli attacchi con le armi a distanza, tranne che con le armi da assedio, che subiscono penalità -1d6 i Tiri per Colpire.
Estinguono automaticamente le candele, le torce o simili fiamme non protette. Le fiamme protette, come quelle delle lanterne, vengono agitate violentemente e hanno una probabilità del 50\% di estinguersi. Vedi Tabella: Effetti del Vento per le possibili conseguenze sulle creature sorprese all'esterno senza ripari.

Le tempeste sono di tre tipi.

\textbf{Tempesta di Polvere (grado di Sfida 3)}

queste tempeste desertiche si differenziano dalle altre tempeste in quanto non hanno precipitazioni. Al contrario, le tempeste di polvere trasportano granelli di sabbia che oscurano la vista, soffocano le fiamme non protette e possono addirittura spegnere quelle protette (probabilità del 50\%). Molte tempeste di polvere sono accompagnate da venti molto forti e si lasciano alle spalle un deposito di 1d6 \texttimes{} 2.5 centimetri di sabbia.
Esiste anche una probabilità del 10\% di incontrare grandi tempeste di polvere con bufere di vento (vedi Tabella: Effetti del Vento). Queste violente tempeste di polvere infliggono 1d3 danni non letali per round a chiunque venga sorpreso all'aperto senza riparo e pongono anche il rischio del soffocamento (vedi Annegamento, eccetto che un personaggio con una sciarpa o simile protezione sulla bocca e il naso, non inizia a soffocare se non dopo un numero di round pari 10 \texttimes{} il suo punteggio di Costituzione). Le grandi tempeste di polvere si depositano alle spalle (2d3-1) x 30 centimetri di sabbia.

\textbf{Tempesta di Neve}

oltre ai venti e alle precipitazioni comuni alle altre tempeste, le tempeste di neve depositano 1d6 \texttimes{} 2.5 centimetri di neve sul terreno.

\textbf{Tempesta di Fulmini}

oltre ai venti e alle precipitazioni (di solito pioggia, ma a volte anche grandine), le tempeste di fulmini sono accompagnate da scariche elettriche che rappresentano un pericolo per i personaggi che si trovano all'aperto senza riparo (specialmente se indossano armature metalliche). Come regola generale, si può considerare un fulmine al minuto per un periodo di un'ora nel cuore della tempesta. Ogni fulmine infligge danni da elettricità tra 4d8 e 10d8. Una tempesta di fulmini su dieci viene accompagnata da un tornado.

\textbf{Tempeste Violente}

Venti molto forti e precipitazioni torrenziali riducono la visibilità a zero, e rendono impossibile effettuare prove di Consapevolezza e compiere attacchi con armi a distanza. Le fiamme non protette vengono automaticamente spente, e c'è una probabilità del 75\% che ciò si verifichi anche per quelle protette. Le creature sorprese in queste zone devono effettuare un Tiro Salvezza su Tempra o devono affrontare effetti a seconda della propria taglia (vedi Tabella: Effetti del Vento). Le tempeste violente sono suddivise nei seguenti quattro tipi.

%\begin{center}
%\includegraphics[width=0.95\linewidth]{immagini/Vincent_van_Gogh_tempesta.png}

%\emph{Vincent van Gogh, Campo di grano sotto un cielo tempestoso (Auvers-sur-Oise, luglio 1890)}

%\end{center}

\textbf{Bufera}

Sebbene abbiano poche o nessuna precipitazione, le bufere possono provocare danni ingenti a causa della forza del vento.

\textbf{Tormenta}

La combinazione di forti venti, neve fitta (di solito 1d3 \texttimes{} 30 cm) e freddo intenso rende le tormente letali per chiunque non vi sia preparato.

\textbf{Uragano}

Oltre ai venti molto forti e alla pioggia intensa, gli uragani sono seguiti da inondazioni. Molte attività in un'avventura sono impossibili in queste condizioni.

\textbf{Tornado}

Oltre ai venti molto forti, i tornado possono ferire gravemente ed uccidere quelli che vengono catturati al suo interno.

\subsubsection{Nebbia}\index{Nebbia}

\label{nebbia}

Sia nella forma di una nube a bassa altitudine che di una foschia che sale dal terreno, la nebbia ostacola la visuale oltre la distanza di 3 metri. Le creature più lontane di 3 metri godono di Copertura leggera (+2 Difesa).

La nebbia rende il terreno difficile.

La nebbia potrebbe essere anche molto fitta in quel caso le creature più lontane di 6 metri godono di Copertura completa, entro 4 metri hanno copertura media (+4 alla Difesa) e quelle entro 1 metro hanno comunque copertura leggera.

\subsubsection{Venti}\index{Venti}

\label{venti}

I venti possono creare turbini di sabbia o polvere, alimentare grossi incendi, rovesciare piccole imbarcazioni e disperdere gas o vapori. Se sono forti a sufficienza possono addirittura buttare a terra i personaggi (vedi Tabella: Effetti del Vento), interferire con gli attacchi a distanza, o imporre penalità ad alcune Prove di Competenze.

\medskip

\textbf{Tabella: Effetti del Vento Forza del Vento}\index[Tabelle]{Tabella Effetti del Vento Forza del Vento}

\medskip

\begin{tabularx}{0.45\textwidth}{llX}
\textbf{Intensità} & \textbf{Velocità} & \textbf{Attacchi a Distanza} \\
\toprule
Leggero & 0-15km &\\
Moderato & 16,5-30 km/h& \\
Forte & 31.5-45 & -2 \\
Molto forte & 45.5-75km/h & -4 \\
Bufera & 76.5-111km/h & impossibile \\
Uragano & 12-261km/h & impossibile \\
Tornado & 262-450km/h & impossibile
\end{tabularx}

\bigskip

\textbf{Vento Leggero}

Una brezza gentile, che non ha effetti pratici sul gioco.

\textbf{Vento Moderato}

Un vento sostenuto, che ha una probabilità del 50\% di estinguere qualsiasi piccola fiamma non protetta, come quella di una candela.

\textbf{Vento forte}: Folate che spengono automaticamente le fiamme non protette (candele, torce e simili). Queste folate impongono penalità -2 ai Tiri per Colpire a distanza ed alle prove di Consapevolezza.

\textbf{Vento Molto Forte}

Oltre a spegnere automaticamente le fiamme non protette, i venti di questa intensità agitano violentemente le fiamme protette (come quelle di una lanterna) e hanno una probabilità del 50\% di estinguerle. Gli attacchi con le armi a distanza e le prove di Consapevolezza subiscono penalità -1d6.

\textbf{Bufera}\index{Bufera}

Abbastanza forti da abbattere i rami o addirittura interi alberi, le bufere estinguono automaticamente le fiamme non protette e hanno una probabilità del 75\% di estinguere quelle protette, come quelle delle lanterne. Gli attacchi con le armi a distanza sono impossibili, e anche le armi da assedio subiscono penalità -1d6 ai Tiri per Colpire. Le prove di Consapevolezza basate sull'udito subiscono penalità -2d6 per l'ululare del vento.

\textbf{Uragano}\index{Uragano}

Estingue tutte le fiamme. Gli attacchi a distanza sono impossibili (eccetto con le armi da assedio che subiscono penalità -2d6 ai Tiri per Colpire). Anche le prove di Consapevolezza basate sull'udito sono impossibili e tutto ciò che i personaggi possono udire è l'ululare del vento. Gli uragani spesso sono in grado di abbattere gli alberi.

\textbf{Tornado (grado di Sfida 10)}\index{Tornado}

Estingue tutte le fiamme. Tutti gli attacchi a distanza sono impossibili (compresi quelli con le armi da assedio), così come le prove di Consapevolezza basate sull'udito. Invece di essere portati via (vedi Tabella: Effetti del Vento), i personaggi che si trovano nelle immediate vicinanze di un tornado e che falliscono un Tiro Salvezza su Tempra (DC 20+) vengono risucchiati dentro il tornado.

Coloro che entrano in contatto con il tornado vengono sollevati da terra e sbatacchiati per 1d10 round, subendo 6d6 danni per round, prima di venirne espulsi violentemente. La creatura viene espulsa da un altezza di 1d6 metri per round di permanenza nel tornado.

Sebbene la velocità rotatoria di un tornado possa raggiungere i 450 km/h, il cono stesso si muove in avanti ad una media di 45 km/h (circa 75 metri per ogni round). Un tornado è in grado di sradicare alberi, distruggere edifici e provocare altre forme di simile devastazione.

\end{multicols}

\vfill

\begin{center}
\includegraphics[keepaspectratio,width=0.6\textwidth]{immagini/blizzard.png}

\end{center}

\pagebreak

