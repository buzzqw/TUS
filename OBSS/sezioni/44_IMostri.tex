\section{I Mostri}

\begin{narratore}[Un pò di mostri...]
Le creature qui presentate vogliono essere un esempio, corposo, degli avversari che i tuoi personaggi potrebbero incontrare. Attenzione, non è detto che siano tutti nemici o per forza che abbiano intenzione negative.

Creature più civilizzate avranno una loro condotta etica e morale individuale, anche all'interno di uno stesso gruppo di avversari c'è chi potrebbe essere più nemico o semplicemente indifferente.

Sfrutta le peculiarità e unicità delle creature per creare incontri non scontati e sfidanti dal punto di vista tattico. Non essere scontato ma neanche assurdo nelle scelte, deve sempre esserci della coerenza nel scegliere le creature.
\end{narratore}

\bigskip

\begin{enfasi}{

Amon Goth: Il controllo è potere. Questo è il potere.

Oskar Schindler: è per questo che ci temono?

Amon Goth: Abbiamo il potere di uccidere. Per questo ci temono.

Oskar Schindler: Ci temono perché abbiamo il potere di uccidere arbitrariamente. Un uomo commette un reato, doveva pensarci, lo facciamo uccidere e ci sentiamo in pace. O lo uccidiamo noi stessi e ci sentiamo ancora meglio. Questo non è il potere però! Questa è giustizia, è una cosa diversa dal potere. Il potere è quando abbiamo ogni giustificazione per uccidere e non lo facciamo.

Amon Goth: è questo il potere?

Oskar Schindler: L'avevano gli imperatori questo. Un uomo ruba qualcosa, viene portato davanti all'imperatore e si lascia cadere per terra tremante, implora per avere pietà. E' conscio che sta per andarsene. E l'imperatore lo perdona, invece. Quell'uomo, immeritevole, lo lascia libero.

(Schindler's list - La lista di Schindler, Film, 1993)

\medskip

I mostri sono la poesia della paura. (Stephen King)
}\end{enfasi}

\begin{multicols}{2}

\setlength{\parindent}{0cm}{

%\small{

\pdfbookmark[3]{A}{A}
%\addcontentsline{toc}{subsubsection}{A}

\mostro{Aboleth}
\begin{description}[noitemsep, topsep=0pt, parsep=0pt, partopsep=0pt, leftmargin=0cm, labelwidth=2.2cm]
	\item[\textbf{Taglia/Tipo:}] Grande aberrazione, malvagio
	\item[\textbf{Caratt.:}] \resizebox{0.5\linewidth+1.8cm}{!}{For 5 Des -1 Cos 2 Int 4 Sag 2 Car 4}
	\item[\textbf{Punti Ferita:}] 199,  \textbf{Difesa:} 24,  \textbf{Iniziativa:} +4
	\item[\textbf{Movimento:}] 3 m, nuoto 12 m
	\item[\textbf{Tiri Salvez.:}] \resizebox{0.5\linewidth+1.8cm}{!}{\resizebox{0.5\linewidth+1.8cm}{!}{Tempra +12, Riflessi +9, Volontà +12}}
	\item[\textbf{Incant.:}] immune incantesimi di Illusione inferiori al 2 livello
	\item[\textbf{Comp.:}] Consapevolezza +10, Storia +12
	\item[\textbf{Sensi:}] Scurovisione 36 m
	\item[\textbf{Linguaggi:}] Aquan, telepatia 36 m
	\item[\textbf{Sfida:}] 10 (5900 PX)\smallskip
\end{description}

\emph{\textbf{Anfibio.}} L'aboleth può respirare aria e acqua.

\emph{\textbf{Nube di Muco.}} Mentre è sott'acqua, l'aboleth è avvolto da muco mutante. Una creatura che entri a contatto con l'aboleth, o che lo colpisca con un attacco da mischia mentre si trova entro 1 metro da esso, deve effettuare un Tiro Salvezza di Tempra DC 24. Se lo fallisce, la creatura resta ammalata per 1d4 ore. La creatura ammalata può respirare solo sott'acqua.

\emph{\textbf{Sonda Telepatica.}} Se una creatura comunica telepaticamente con l'aboleth, e l'aboleth può vederla, l'aboleth ne apprende i più grandi desideri.

\textbf{Azioni}

\emph{\textbf{Multiattacco.}} L'aboleth effettua tre attacchi con i tentacoli

\emph{\textbf{Tentacolo.} Attacco con arma da mischia}: +10 a colpire, portata 3 m, un bersaglio.

\emph{Colpisce}: 12 (2d6 + 5) danni contundenti. Se il bersaglio è una creatura, deve riuscire un Tiro Salvezza di Tempra DC 24 o divenire ammalato. La malattia non produce alcun effetto per 1 minuto e può essere rimossa da qualsiasi magia che curi le malattie. Dopo 1 minuto, la pelle della creatura ammalata diventa trasparente e viscida, la creatura non può recuperare Punti Ferita a meno che non sia sott'acqua, e la malattia può essere rimossa solo da \emph{guarire} o un altro incantesimo cura malattie di livello 3 o più. Quando la creatura si trova al di fuori di un corpo d'acqua, subisce 6 (1d12) danni da acido ogni 10 minuti a meno che la sua pelle non venga bagnata prima che siano passati questi 10 minuti.

\emph{\textbf{Coda.} Attacco con arma da mischia}: +9 a colpire, portata 3 m, un bersaglio.

\emph{Colpisce:} 15 (3d6 + 5) danni contundenti.

\emph{\textbf{Schiavizzare (3/Giorno).}} L'aboleth prende a bersaglio una creatura che può vedere entro 9 metri da esso. Il bersaglio deve riuscire un Tiro Salvezza di Volontà DC 24 o restare affascinato magicamente dall'aboleth finché l'aboleth muore o i due si trovano su piani di esistenza differenti. Il bersaglio affascinato è sotto il controllo dell'aboleth e non può effettuare reazioni. L'aboleth e il bersaglio possono comunicare telepaticamente tra di loro a qualsiasi distanza.

Ogniqualvolta il bersaglio affascinato subisce danni, può ripetere il Tiro Salvezza. Se lo riesce, l'effetto termina. Non più di una volta ogni 24 ore, può ripetere il Tiro Salvezza quando si trova almeno a 1,5 chilometri di distanza dall'aboleth.

\textbf{Azioni Aggiuntive}

L'aboleth può effettuare 3 Azioni aggiuntive, scelte tra le opzioni seguenti. Può usare solo un'opzione Aggiuntive alla volta e solo al termine del round di un'altra creatura. L'aboleth recupera le Azioni aggiuntive spese all'inizio del proprio round.

\textbf{Individuare.} L'aboleth effettua una prova Consapevolezza.

\textbf{Risucchio Psichico (Costa 2 Azioni).} Una creatura affascinata dall'aboleth subisce 10 (3d6) danni e l'aboleth recupera un numero di Punti Ferita pari al danno subito dalla creatura.

\textbf{Spazzata di Coda.} L'aboleth effettua un attacco di coda.

\textbf{Ecologia}\\
Ambiente: Qualsiasi Acquatico\\
Organizzazione: Solitario, coppia, nidiata (3-6) o branco (7-19)\\
\textbf{Categoria Tesoro}: F\\
\textbf{Descrizione}\\
Come suggerisce il loro aspetto primitivo gli aboleth sono fra le più antiche forme di vita al mondo. Un aboleth è lungo 7 metri e pesa circa 3,2 tonnellate. Gli aboleth abitano in fondo ai mari nelle loro enormi città, serviti da innumerevoli schiavi.

\mostro{Angelo Deva}
\begin{description}[noitemsep, topsep=0pt, parsep=0pt, partopsep=0pt, leftmargin=0cm, labelwidth=2.2cm]
	\item[\textbf{Taglia/Tipo:}] Media celestiale, buono
	\item[\textbf{Caratt.:}] \resizebox{0.5\linewidth+1.8cm}{!}{For 4 Des 4 Cos 4 Int 3 Sag 5 Car 5}
	\item[\textbf{Punti Ferita:}] 203,  \textbf{Difesa:} 29,  \textbf{Iniziativa:} +4
	\item[\textbf{Movimento:}] 9 m, volo 27 m
	\item[\textbf{Tiri Salvez.:}] \resizebox{0.5\linewidth+1.8cm}{!}{\resizebox{0.5\linewidth+1.8cm}{!}{Tempra +14, Riflessi +14, Volontà +15}}
	\item[\textbf{Comp.:}] Percepire Emozioni +9
	\item[\textbf{Res. Danni:}] da Luce; da arma non magica
	\item[\textbf{Immunità:}] affascinato, affaticato, spaventato
	\item[\textbf{Sensi:}] Scurovisione 36 m
	\item[\textbf{Linguaggi:}] tutte, telepatia 36 m
	\item[\textbf{Sfida:}] 10 (5900 PX)\smallskip
\end{description}

\emph{\textbf{Armi Angeliche.}} Gli attacchi con arma del deva sono magici. Quando il deva colpisce con qualsiasi arma, l'arma infligge 4d8 danni da Luce aggiuntivi (già compresi nell'attacco).

\emph{\textbf{Incantesimi Innati.}} La caratteristica da incantatore innato del deva è il Carisma (DC 17 per i Tiri Salvezza degli incantesimi). Il deva può lanciare in maniera innata i seguenti incantesimi, con l'uso delle sole componenti verbali:

A volontà: \emph{\hyperlink{Conoscere i Tratti}{Conoscere i Tratti}}

1/giorno: \emph{\hyperlink{Comunione}{Comunione}}

\emph{\textbf{Resistenza alla Magia.}} Il deva ha +1d6 ai Tiri Salvezza contro incantesimi e altri effetti magici.

\textbf{Azioni}

\emph{\textbf{Multiattacco.}} Il deva effettua due attacchi da mischia.

\emph{\textbf{Mazza.} Attacco con arma da mischia}: +10 a colpire, portata 1 m, un bersaglio.

\emph{Colpisce:} 7 (1d6 + 4) danni contundenti più 18 (4d8) danni da Luce.

\emph{\textbf{Tocco Guaritore (3/Giorno).}} Il deva entra a contatto con un'altra creatura. Il bersaglio recupera magicamente 20 (4d8 + 2) Punti Ferita ed è libero da qualsiasi cecità, malattia, maledizione, sordità o veleno.

\emph{\textbf{Mutare Forma.}} Il deva può trasformarsi magicamente in un umanoide o bestia il cui grado di sfida sia pari o inferiore al proprio, o tornare alla sua vera forma. Alla morte ritorna alla sua vera forma. Qualsiasi equipaggiamento stia indossando o trasportando viene assorbito o trasportato nella nuova forma (a scelta del deva).

Nella nuova forma, il deva mantiene le sue statistiche di gioco e la facoltà di parlare, ma la sua Difesa, metodi di movimento, Forza, Destrezza e sensi speciali vengono rimpiazzati da quelli della nuova forma, e ottiene qualsiasi statistica o capacità (Azioni aggiuntive e azioni da tana) possedute dalla sua nuova forma e non dalla sua originale.

\textbf{Ecologia}\\
Ambiente: Qualsiasi piano di con Tratti buono\\
Organizzazione: Solitario, coppia, o squadriglia (3-6)\\
\textbf{Categoria Tesoro}: (Spadone Infuocato +1, altro tesoro)\\
\textbf{Descrizione}\\
I deva movanici compongono i ranghi della fanteria delle armate celesti, sebbene trascorrano la maggior parte del loro tempo pattugliando il Piano Positivo, quello Negativo e quello Materiale. Sul Piano Positivo sorvegliano le anime buone erranti. Sul Piano Negativo combattono i non morti e altri strani esseri che cacciano nel famelico vuoto. Le loro rare volte sul Piano Materiale hanno solitamente lo scopo di portare aiuto a potenti mortali, quando un grande pericolo minaccia di far cadere nelle mani del male un intero regno.

\mostro{Angelo Planetar}
\begin{description}[noitemsep, topsep=0pt, parsep=0pt, partopsep=0pt, leftmargin=0cm, labelwidth=2.2cm]
	\item[\textbf{Taglia/Tipo:}] Grande celestiale, buono
	\item[\textbf{Caratt.:}] \resizebox{0.5\linewidth+1.8cm}{!}{For 7 Des 5 Cos 7 Int 4 Sag 6 Car 7}
	\item[\textbf{Punti Ferita:}] 325,  \textbf{Difesa:} 38,  \textbf{Iniziativa:} +5
	\item[\textbf{Movimento:}] 12 m, volo 36 m
	\item[\textbf{Tiri Salvez.:}] \resizebox{0.5\linewidth+1.8cm}{!}{\resizebox{0.5\linewidth+1.8cm}{!}{Tempra +23, Riflessi +21, Volontà +22}}
	\item[\textbf{Comp.:}] Consapevolezza +13
	\item[\textbf{Res. Danni:}] da Luce;
	\item[\textbf{Immunità:}] affascinato, affaticato, spaventato, armi +1
	\item[\textbf{Sensi:}] visione del vero 36 m
	\item[\textbf{Linguaggi:}] tutte, telepatia 36 m
	\item[\textbf{Sfida:}] 16 (15000 PX)\smallskip
\end{description}

\emph{\textbf{Armi Angeliche.}} Gli attacchi con arma del planetar sono magici. Quando colpisce con qualsiasi arma, l'arma infligge 5d8 danni da Luce aggiuntivi (già indicati nell'attacco).

\emph{\textbf{Consapevolezza Divina.}} Il planetar riconosce immediatamente le bugie.

\emph{\textbf{Incantesimi Innati.}} La caratteristica da incantatore innato del planetario è il Carisma (DC 20 per i Tiri Salvezza degli incantesimi). Il planetario può lanciare in maniera innata i seguenti incantesimi, senza bisogno di componenti materiali:

A volontà: \emph{\hyperlink{Conoscere i Tratti}{Conoscere i Tratti}}, \emph{\hyperlink{Invisibilità}{Invisibilità}} (solo personale)

3/giorno: \emph{\hyperlink{Barriera di Lame}{Barriera di Lame}, \hyperlink{Colpo Infuocato}{Colpo Infuocato}}

1/giorno: \emph{\hyperlink{Comunione}{Comunione}, \hyperlink{Controllare Tempo Atmosferico}{Controllare Tempo Atmosferico}, \hyperlink{Piaga degli Insetti}{Piaga degli Insetti}}

\emph{\textbf{Resistenza alla Magia.}} Il planetar ha +1d6 ai Tiri Salvezza contro incantesimi e altri effetti magici.

\textbf{Azioni}

\emph{\textbf{Multiattacco.}} Il planetar effettua due attacchi da mischia.

\emph{\textbf{Spadone.} Attacco con arma da mischia}: +13 a colpire, portata 2 m, un bersaglio.

\emph{Colpisce:} 21 (4d6 + 7) danni taglienti più 22 (5d8) danni da Luce.

\emph{\textbf{Tocco Guaritore (4/Giorno).}} Il planetar entra a contatto con un'altra creatura. Il bersaglio recupera magicamente 30 (6d8 + 3) Punti Ferita ed è libero da qualsiasi cecità, malattia, maledizione, sordità o veleno.

\emph{\textbf{Arrabbiato:}}

- il Planetar evoca le potenze angeliche in suo aiuto. Usando 3 Azioni evoca 1d4 Angeli Deva.

- Il Planetar causa un danno critico (2d6) ogni volta che colpisce fino a fine combattimento. Costa 1 Azione.

\textbf{Ecologia}\\
Ambiente: Qualsiasi piano con Tratti buono\\
Organizzazione: Solitario o coppia\\
\textbf{Categoria Tesoro}: Spadone Sacro +3\\
\textbf{Descrizione}\\
I planetar sono i generali delle armate celestiali volti alla distruzione del male. Un planetar è di norma alto 2,7 metri e pesa circa 250 kg. Sono ottimi diplomatici, ma contro gli immondi preferiscono la guerra piuttosto che negoziare una pace.

\mostro{Angelo Solar}
\begin{description}[noitemsep, topsep=0pt, parsep=0pt, partopsep=0pt, leftmargin=0cm, labelwidth=2.2cm]
	\item[\textbf{Taglia/Tipo:}] Grande celestiale, buono
	\item[\textbf{Caratt.:}] \resizebox{0.5\linewidth+1.8cm}{!}{For 8 Des 6 Cos 8 Int 7 Sag 7 Car 10}
	\item[\textbf{Punti Ferita:}] 426,  \textbf{Difesa:} 46,  \textbf{Iniziativa:} +7
	\item[\textbf{Movimento:}] 15 m, volo 45 m
	\item[\textbf{Tiri Salvez.:}] \resizebox{0.5\linewidth+1.8cm}{!}{\resizebox{0.5\linewidth+1.8cm}{!}{Tempra +29, Riflessi +27, Volontà +28}}
	\item[\textbf{Comp.:}] Consapevolezza +16
	\item[\textbf{Res. Danni:}] da Luce; Fuoco, Freddo, Elettricità
	\item[\textbf{Immunità:}] affascinato, avvelenato, affaticato, spaventato, arma +2
	\item[\textbf{Sensi:}] visione del vero 36 m
	\item[\textbf{Linguaggi:}] tutte, telepatia 36 m
	\item[\textbf{Sfida:}] 21 (33000 PX)\smallskip
\end{description}

\emph{\textbf{Armi Angeliche.}} Gli attacchi con arma del solar sono magici. Quando colpisce con qualsiasi arma, l'arma infligge 6d8 danni da Luce aggiuntivi (già compresi nell'attacco).

\emph{\textbf{Consapevolezza Divina.}} Il solar riconosce immediatamente le bugie.

\emph{\textbf{Incantesimi Innati.}} La caratteristica da incantatore innato del solar è il Carisma (DC 25 per i Tiri Salvezza degli incantesimi). Il solar può lanciare in maniera innata i seguenti incantesimi, senza bisogno di componenti materiali:

A volontà: \emph{\hyperlink{Conoscere i Tratti}{Conoscere i Tratti}}, \emph{\hyperlink{Invisibilità}{Invisibilità}} (solo personale)

3/giorno: \emph{\hyperlink{Barriera di Lame}{Barriera di Lame}, \hyperlink{Colpo Infuocato}{Colpo Infuocato}}

1/giorno: \emph{\hyperlink{Comunione}{Comunione}, \hyperlink{Controllare Tempo Atmosferico}{Controllare Tempo Atmosferico}}

\emph{\textbf{Resistenza alla Magia.}} Il solar ha +1d6 ai Tiri Salvezza contro incantesimi e altri effetti magici.

\textbf{Azioni}

\emph{\textbf{Multiattacco.}} Il solar effettua due attacchi con lo spadone.

\emph{\textbf{Spadone.} Attacco con arma da mischia}: +16 a colpire, portata 1 m, un bersaglio.

\emph{Colpisce:} 22 (4d6 + 8) danni taglienti più 27 (6d8) danni da Luce.

\emph{\textbf{Arco Lungo dell'Uccisione.} Attacco con arma a distanza}: +17 a colpire, gittata 45m, un bersaglio.

\emph{Colpisce:} 15 (2d8 + 6) danni perforanti più 27 (6d8) danni da Luce. Se il bersaglio è una creatura con 100 Punti Ferita o meno, deve riuscire un Tiro Salvezza di Tempra DC 35 o morire.

\emph{\textbf{Spada Volante.}} Il solar libera il suo spadone perché fluttui magicamente in uno spazio non occupato entro 1 metro da lui. Se il solar può vedere la spada, con un'azione gratuita le può ordinare mentalmente di volare per un massimo di 15 metri ed effettuare un attacco contro un bersaglio o ritornare nella mano del solar. Se la spada fluttuante è bersaglio di un effetto, si considera come se fosse impugnata dal solar. Se il solar muore, la spada fluttuante cade a terra.

\emph{\textbf{Tocco Guaritore (4/Giorno).}} Il solar entra a contatto con un'altra creatura. Il bersaglio recupera magicamente 40 (8d8 + 4) Punti Ferita ed è libero da qualsiasi cecità, malattia, maledizione, sordità o veleno.

Il solar può effettuare 3 azioni aggiuntive, scelte tra le opzioni seguenti. Può usare solo un'Azione Aggiuntiva alla volta e solo al termine del round di un'altra creatura. Il solar recupera le azioni aggiuntive spese all'inizio del proprio round.

\textbf{Esplosione Incandescente (Costa 2 Azioni).} Il solar emette energia magica divina. Ogni creatura di sua scelta, in un raggio di 3 metri, deve effettuare un Tiro Salvezza su Riflessi DC 35, subendo 14 (4d6) danni da fuoco più 14 (4d6) danni da Luce se fallisce il Tiro Salvezza, o la metà se lo riesce.

\textbf{Sguardo Accecante (Costa 3 Azioni).} Il solar prende a bersaglio una creatura entro 9 metri e che possa vedere. Se il bersaglio può vedere il solar, il bersaglio deve riuscire un Tiro Salvezza su Tempra DC 35 o restare accecato finché un incantesimo come \emph{\hyperlink{Ristorare Inferiore}{Ristorare Inferiore}} non rimuoverà la cecità.

\textbf{Teletrasporto.} Il solar si teletrasporta magicamente fino a 36 metri di distanza, insieme a tutto l'equipaggiamento che sta indossando o trasportando, in uno spazio non occupato e che può vedere.

\textbf{Ecologia}\\
Ambiente: Qualsiasi piano con Tratti buono\\
Organizzazione: Solitario o coppia\\
\textbf{Categoria Tesoro}: Armatura Completa +5, Spadone Danzante +5, Arco Lungo Composito +5\\
\textbf{Descrizione}\\
I solar sono i più potenti tra gli angeli, solitamente braccio destro delle divinità o campioni di cause importanti. Hanno un aspetto quasi umano e sono alti circa 2,7 metri, con pelle argentata o dorata. Sono benedetti con potenti capacità magiche e sono in grado di uccidere le creature malvagie più forti.

I solar possono seguire tracce lasciate da potenti nemici e sono conosciuti come uccisori di mostri. Alcuni proteggono concetti soprannaturali, oggetti o creature di grande importanza, come i condotti di energia solare o le catene che imprigionano divinità malvagie. Possono diventare profeti e guru, gettando le basi per grandi religioni.

Pur non essendo divinità, il loro potere si avvicina a quello dei semidei e spesso fanno da consiglieri per le giovani divinità. Sono creati come servitori degli dei, amalgamando anime buone ed energia divina pura. Alcuni solar provengono dalla promozione di angeli minori come deva e planetar.

I solar che passano tempo sul Piano Materiale possono influenzare linee di sangue umano, creando discendenti mezzo-celestiali. Tendono a evitare il contatto diretto con la loro progenie per evitare vergogna, ma controllano e aiutano da lontano.

Rispettati da tutti gli angeli, i solar a volte comandano armate contro le legioni dell'Inferno e le orde dell'Abisso.

\mostro{Ankheg}
\begin{description}[noitemsep, topsep=0pt, parsep=0pt, partopsep=0pt, leftmargin=0cm, labelwidth=2.2cm]
	\item[\textbf{Taglia/Tipo:}] Grande mostruosità, disallineato
	\item[\textbf{Caratt.:}] \resizebox{0.5\linewidth+1.8cm}{!}{For 3 Des 0 Cos 1 Int -5 Sag 1 Car -2}
	\item[\textbf{Punti Ferita:}] 51,  \textbf{Difesa:} 14,  \textbf{Iniziativa:} +0
	\item[\textbf{Movimento:}] 9 m, scavo 3 m
	\item[\textbf{Tiri Salvez.:}] \resizebox{0.5\linewidth+1.8cm}{!}{Tempra +3, Riflessi +3, Volontà +3}
	\item[\textbf{Sensi:}] Scurovisione 18 m, percezione tellurica 18 m
	\item[\textbf{Sfida:}] 2 (450 PX)\smallskip
\end{description}

\textbf{Azioni}

\emph{\textbf{Morso.} Attacco con arma da mischia}: +5 a colpire, portata 1 m, un bersaglio.

\emph{Colpisce:} 10 (2d6 + 3) danni taglienti più 3 (1d6) danni da acido. Se il bersaglio è una creatura di taglia Grande o inferiore, è afferrata (DC 13 per fuggire). Fino al termine dell'afferrare, l'ankheg può mordere solo la creatura afferrata e ha +1d6 ai tiri di attacco contro di essa.

\emph{\textbf{Spruzzo Acido}} L'ankheg sputa acido in una linea lunga 9 metri e larga 1 metro, purché non stia afferrando nessuna creatura. Ogni creatura su quella linea deve effettuare un Tiro Salvezza di Riflessi DC 14, e subire 10 (3d6) danni da acido se fallisce il Tiro Salvezza, o la metà di questi danni se lo riesce.

\textbf{Ecologia}\\
Ambiente: Pianure temperate o calde\\
Organizzazione: Solitario, coppia o nido (3-6)\\
\textbf{Categoria Tesoro}: C\\
\textbf{Descrizione}\\
Gli ankheg sono mostri scavatori che prediligono le campagne per via del terreno morbido che facilita i loro spostamenti. Solitamente evitano le aree densamente popolate ma la loro predilezione per la carne del bestiame e degli umani li tiene lontani dalle zone disabitate. Nei deserti remoti, si trovano ankheg più grandi che si nutrono di scorpioni e cammelli.

Alcuni ankheg sono addestrabili e possono diventare animali da carico, sebbene il loro comportamento imprevedibile li renda pericolosi nelle regioni civilizzate.

\mostro{Arpia}
\begin{description}[noitemsep, topsep=0pt, parsep=0pt, partopsep=0pt, leftmargin=0cm, labelwidth=2.2cm]
	\item[\textbf{Taglia/Tipo:}] Media mostruosità, Arrogante, Impulsivo
	\item[\textbf{Caratt.:}] \resizebox{0.5\linewidth+1.8cm}{!}{For 1 Des 1 Cos 1 Int -2 Sag 0 Car 1}
	\item[\textbf{Punti Ferita:}] 33,  \textbf{Difesa:} 14,  \textbf{Iniziativa:} +1
	\item[\textbf{Movimento:}] 6 m, volo 12 m
	\item[\textbf{Tiri Salvez.:}] \resizebox{0.5\linewidth+1.8cm}{!}{Tempra +3, Riflessi +3, Volontà +3}
	\item[\textbf{Linguaggi:}] Comune
	\item[\textbf{Sfida:}] 1 (200 PX)\smallskip
\end{description}

\textbf{Azioni}

\emph{\textbf{Multiattacco.}} L'arpia effettua due attacchi: uno con gli artigli e uno con il randello.

\emph{\textbf{Artigli.} Attacco con arma da mischia}: +4 a colpire, portata 1 m, un bersaglio.

\emph{Colpisce:} 5 (2d4 + 1) danni taglienti, 1 danno da Sanguinamento.

\emph{\textbf{Randello.} Attacco con arma da mischia}: +4 a colpire, portata 1 m, un bersaglio.

\emph{Colpisce:} 3 (1d4 + 1) danni contundenti.

\emph{\textbf{Venti affamati} (2 Azioni)}: L'arpia usa il vento per avvicinare le sue prede. Un bersaglio entro 6 metri deve fare un Tiro Salvezza su Tempra DC 12 o essere tirato a fianco all'arpia. Se il bersaglio è stato alzato da terra e non può volare, poi cade normalmente.

\emph{\textbf{Canto Ammaliatore.}} L'arpia canta una melodia magica. Ogni umanoide e gigante entro 90 metri dall'arpia e che possa udire la canzone deve riuscire un Tiro Salvezza di Volontà DC 13 o restare affascinato fino al termine della canzone. L'arpia deve effettuare un'Azione Immediata durante il suo prossimo round per continuare a cantare. Può smettere di cantare in qualsiasi momento. Il canto ha termine se l'arpia è inabile.

Mentre è affascinato dall'arpia, un bersaglio è inabile e ignora le canzoni di altre arpie. Se il bersaglio affascinato si trova a più di 1 metro dall'arpia, il bersaglio deve muoversi durante il proprio round per dirigersi verso l'arpia usando la via più diretta. Prima di muoversi in un terreno pericoloso, come lava o un pozzo, e prima di subire danno da qualsiasi fonte che non sia l'arpia, il bersaglio potrà ripetere il Tiro Salvezza. Una creatura può ripetere il Tiro Salvezza al termine di ciascun proprio round. Se il Tiro Salvezza ha successo, l'effetto ha termine per quel bersaglio.

Un bersaglio che riesce il Tiro Salvezza è immune al canto di quell'arpia per le successive 24 ore.

\textbf{Ecologia}\\
Ambiente: Paludi Temperate\\
Organizzazione: Solitario, coppia o stormo (3-12)\\
\textbf{Categoria Tesoro}: R (C)\\
\textbf{Descrizione}\\
Spesso viste come creature malvagie e corrotte, le arpie sanno come gli altri pensano e agiscono. Questa capacità percettiva offre loro un vantaggio nel trovare i loro pasti preferiti. Sebbene le creature selvatiche cadano facilmente vittime del canto ammaliatore, queste malvagie donne-uccello preferiscono pasti conditi con complessi pensieri senzienti. Le facili prede rendono il pasto noioso.

Anche se in definitiva selvagge e senza alcun rimorso per le loro azioni, diverse arpie vivono presso le società umanoidi e si divertono a sfruttare le creature che reputano potenziali pasti.

Le arpie tendono ad indossare ninnoli e ciondoli rubati alle loro vittime, perché amano compiacersi dei brillanti ornamenti degli uomini. Da vicino queste creature trasudano del puzzo delle loro vittime divorate e raramente lasciano che le creature non ancora ammaliate si avvicinino troppo, cosicché non sentano l'odore del sangue e della putrefazione sulle loro penne. Per questo motivo, molte arpie si cospargono di profumi e oli aromatici.

Le arpie sono marcatamente differenti a seconda della regione in cui vivono. Alcune assomigliano ad una mescolanza di avvoltoi e donne, mentre altri portano sulle penne i tratti regali di falchi e falconi. Rare nidiate di arpie, in luoghi isolati e tropicali del mondo, hanno anche piume colorate come i pappagalli.

\mostro{Azer}
\begin{description}[noitemsep, topsep=0pt, parsep=0pt, partopsep=0pt, leftmargin=0cm, labelwidth=2.2cm]
	\item[\textbf{Taglia/Tipo:}] Media elementale, legale
	\item[\textbf{Caratt.:}] \resizebox{0.5\linewidth+1.8cm}{!}{For 3 Des 1 Cos 2 Int 1 Sag 1 Car 0}
	\item[\textbf{Punti Ferita:}] 51,  \textbf{Difesa:} 15,  \textbf{Iniziativa:} +1
	\item[\textbf{Movimento:}] 9 m
	\item[\textbf{Tiri Salvez.:}] \resizebox{0.5\linewidth+1.8cm}{!}{Tempra +4, Riflessi +3, Volontà +3}
	\item[\textbf{Imm. Danni:}] Fuoco, Veleno
	\item[\textbf{Linguaggi:}] Ignan
	\item[\textbf{Sfida:}] 2 (450 PX)\smallskip
\end{description}

\emph{\textbf{Armi Riscaldate.}} Quando l'azer colpisce con un'arma da mischia in metallo, infligge 3 (1d6) danni da fuoco aggiuntivi (già inclusi nell'attacco).

\emph{\textbf{Corpo Riscaldato.}} Una creatura che entri a contatto con l'azer o lo colpisca con un attacco da mischia mentre si trova entro 1 metro da lui subisce 5 (1d10) danni da fuoco.

\emph{\textbf{Fuoco Vivente.}} Un azer non ha bisogno di cibo, bevande o di dormire.

\emph{\textbf{Illuminazione.}} L'azer irradia luce intensa in un raggio di 3 metri e luce fioca per 6 metri.

\textbf{Azioni}

\emph{\textbf{Martello da Guerra.} Attacco con arma da mischia}: +6 a colpire, portata 1 m, un bersaglio.

\emph{Colpisce:} 7 (1d8 + 3) danni contundenti, o 8 (1d10 + 3) danni contundenti se usato a due mani per effettuare un attacco da mischia, più 3 (1d6) danni da fuoco.

\textbf{Ecologia}\\
Ambiente: Qualsiasi terreno (Piano del Fuoco)\\
Organizzazione: Solitario, coppia, gruppo (3-6), squadra (11-20 più 2 sergenti di 3° livello e 1 capo di 3°-6° livello) o clan (30-100 più 50\% di non combattenti più 1 sergente di 3° livello ogni 20 adulti, 5 tenenti di 5° livello e 3 capitani di 7° livello)\\
\textbf{Categoria Tesoro}: Armatura a Scaglie Perfetta, Martello da Guerra Perfetto, Martello Leggero, N\\
\textbf{Descrizione}\\
Una Razza orgogliosa e industriosa proveniente dal Piano del Fuoco, gli Azer lavorano nelle loro fortezze di bronzo e d'ottone, sempre pronti a combattere la loro lunga e ribollente guerra contro gli Efreet. Gli Azer vivono in una società in cui ogni membro sa qual è il suo posto. Nati con specifici doveri, solitamente legati alle attività del padre o della madre, gli Azer si dedicano a queste occupazioni per tutta la vita. Un sistema di caste provvede a tenere ulteriormente in riga la società Azer. I nobili, che regnano senza dover rendere conto a nessuno, indossano kilt di ottone decorato come simbolo della loro casta, mentre quelli dei mercanti e dei proprietari di negozi sono in resistente bronzo. I kilt di rame sono indossati dalla casta lavoratrice, composta da servitori, artigiani e braccianti.

Capaci di incanalare calore tramite le Armi e gli attrezzi in metallo, gli Azer non utilizzano quasi mai Armi non metalliche, e prediligono il corpo a corpo agli attacchi a distanza. Sono soliti fare prigionieri, riportandoli alle loro fortezze e obbligandoli a lavorare per loro per un anno e un giorno.

Nella leggendaria Città d'Ottone abitano più di mezzo milione di Azer. La maggior parte di questi sfortunati Azer vive una vita di Schiavitù sotto gli Efreet. Gli Azer soggiogati a questa Schiavitù continuano a eseguire i loro doveri senza porre domande, preferendo aspettare la conclusione dei loro contratti o sperando che i loro padroni muoiano o vengano sconfitti. La dedizione all'ordine arde intensa in questa Razza, al punto che alcuni degli Schiavi Azer fungono da supervisori sulla loro stessa gente. Al di fuori della Città d'Ottone, gli Azer sono liberi di vivere le loro vite, spesso in altre metropoli Planari, creando oggetti, vendendo merci e gestendo taverne.

A un occhio non allenato gli Azer si somigliano tra loro in modo impressionante. Sono alti 1,2 metri ma pesano 100 kg.

%\addcontentsline{toc}{subsubsection}{B}
\pdfbookmark[3]{B}{B}

\mostro{Banshee}
\begin{description}[noitemsep, topsep=0pt, parsep=0pt, partopsep=0pt, leftmargin=0cm, labelwidth=2.2cm]
	\item[\textbf{Taglia/Tipo:}] Media non morto, Arrogante, Vanitoso
	\item[\textbf{Caratt.:}] \resizebox{0.5\linewidth+1.8cm}{!}{For -5 Des 5 Cos 0 Int 1 Sag 1 Car 4}
	\item[\textbf{Punti Ferita:}] 87,  \textbf{Difesa:} 22,  \textbf{Iniziativa:} +5
	\item[\textbf{Movimento:}] 0 m, volo 18 m, Fluttuare
	\item[\textbf{Tiri Salvez.:}] \resizebox{0.5\linewidth+1.8cm}{!}{Tempra +4, Riflessi +9, Volontà +5}
	\item[\textbf{Res. Danni:}] Acido, Elettricità, Fuoco, Suono; arma magica +1
	\item[\textbf{Imm. Danni:}] da Vuoto, Veleno, Freddo, da arma non magica
	\item[\textbf{Immunità:}] affascinato, afferrato, intralciato, paralizzato, pietrificato, prono, affaticato, sanguinamento
	\item[\textbf{Sensi:}] Scurovisione 18 m
	\item[\textbf{Linguaggi:}] Elfico, Comune, Expiran
	\item[\textbf{Sfida:}] 4 (1100 PX)\smallskip
\end{description}

\emph{\textbf{Individuazione della Vita}}. La Banshee percepisce la presenza di creature che non siano non morti e costrutti entro un raggio di 5 chilometri. Conosce la direzione generale in cui si trovano, ma non la loro precisa ubicazione.

\emph{\textbf{Movimento Incorporeo}}. La Banshee può muoversi attraverso altre creature e oggetti come se fossero terreno difficile. Subisce 5 (1d10) danni da forza se termina il suo round all'interno di un oggetto.

\emph{\textbf{Natura Non Morta.}} La Banshee non ha bisogno di aria, cibo, bevande o sonno.

\emph{\textbf{Sensibilità alla Luce}}. Mentre è alla luce del sole, la Banshee è Rallentata 1.

\textbf{Azioni}

\emph{\textbf{Tocco Corruttore}}. Attacco a Tocco: +6 al Tiro per Colpire, portata 1 m, un bersaglio.

\emph{Colpito}: 12 (3d6 +2) danni da Vuoto.

\textbf{Reazione: \emph{Urlo d'odio}}: usando una reazione la Banshee urla contro un avversario che ha fatto un Tiro Critico contro di lei. L'avversario deve fare un Tiro Salvezza su Tempra con bonus Carisma DC 16 per dimezzare 2d8 di danno da forza.

\emph{\textbf{Volto Terrificante}}. Ogni creatura che non sia un non morto situata entro 18 metri dalla Banshee e che sia in grado di vederla deve superare un Tiro Salvezza su Volontà con modificatore Carisma con DC 18, altrimenti è spaventata per 1 minuto. Un bersaglio spaventato può ripetere il Tiro Salvezza alla fine di ogni suo round, subendo -1d6 se la Banshee si trova entro linea di vista; se supera il tiro, l'effetto per lui termina. Se un bersaglio supera il Tiro Salvezza o l'effetto per lui termina, quel bersaglio è immune al Volto Terrificante della Banshee per le 24 ore successive.

\emph{\textbf{Lamento (1/Giorno)}}. La Banshee emette un lamento funesto, purché non sia esposta alla luce del sole. Questo lamento non ha alcun effetto sui costrutti e sui non morti. Ogni altra creatura situata entro 9 metri da lei e in grado di udirla deve effettuare un Tiro Salvezza su Tempra con bonus Carisma DC 17; se lo fallisce, scende a O Punti Ferita, mentre se lo supera, subisce 35 (10d6) danni da forza.

\textbf{Ecologia}\\
Ambiente: Qualsiasi\\
Organizzazione: Solitario\\
\textbf{Categoria Tesoro}: D\\
\textbf{Descrizione}\\
La Banshee è lo spirito infuriato di una donna che ha tradito i propri cari o è stata a sua volta tradita. Impazzita per il dolore, la Banshee riversa la propria vendetta su ogni creatura vivente (innocente o colpevole) con il suo temibile tocco e le sue grida mortali.

\mostro{Basilisco}
\noindent
\begin{description}[noitemsep, topsep=0pt, parsep=0pt, partopsep=0pt, leftmargin=0cm, labelwidth=2.2cm]
	\item[\textbf{Taglia/Tipo:}] Media mostruosità, disallineato
	\item[\textbf{Caratt.:}] \resizebox{0.5\linewidth+1.8cm}{!}{For 3 Des -1 Cos 2 Int -4 Sag -1 Car -2}
	\item[\textbf{Punti Ferita:}] 70,  \textbf{Difesa:} 15,  \textbf{Iniziativa:} -1
	\item[\textbf{Movimento:}] 6 m
	\item[\textbf{Tiri Salvez.:}] \resizebox{0.5\linewidth+1.8cm}{!}{Tempra +5, Riflessi +3, Volontà +3}
	\item[\textbf{Sensi:}] Scurovisione 18 m
	\item[\textbf{Sfida:}] 3 (700 PX)\smallskip
\end{description}

\emph{\textbf{Sguardo Pietrificante.}} Se una creatura comincia il suo round entro 9 metri dal basilisco e i due si possono vedere vicendevolmente, se non inabile il basilisco può obbligare la creatura ad effettuare un Tiro Salvezza di Tempra DC 14. Se la creatura fallisce il Tiro Salvezza diventa Rallentato 1. La creatura deve ripetere il Tiro Salvezza al termine del suo prossimo round. Se lo riesce, l'effetto termina. Se lo fallisce, la creatura è pietrificata finché non viene liberata dall'incantesimo \emph{\hyperlink{Ristorare Superiore}{Ristorare Superiore}} o altra magia.

Una creatura che non sia sorpresa e che voglia attaccare il basilisco senza guardarla direttamente ha -1d6 al Tiro per Colpire.

Se il basilisco si trova entro 9 metri dal suo riflesso a luce intensa e lo vede, lo scambia per un rivale e diventa il bersaglio del proprio sguardo.

\textbf{Azioni}

\emph{\textbf{Morso.} Attacco con arma da mischia}: +6 a colpire, portata 1 m, un bersaglio.

\emph{Colpisce:} 10 (2d6 + 3) danni perforanti più Veleno del Basilisco.

\emph{Veleno:} Veleno del Basilisco, F, istantaneo, TS Tempra 14 oppure Rallentato 1/3r.

\textbf{Ecologia}\\
Ambiente: Qualsiasi\\
Organizzazione: Solitario, coppia o colonia (3-6)\\
\textbf{Categoria Tesoro}: F\\
\textbf{Descrizione}\\
Il basilisco, spesso chiamato \emph{Re dei Serpenti} è un rettile a otto zampe di indole aggressiva che ha la capacità di trasformare le creature in pietra con il suo sguardo. La leggenda narra che, come la Cockatrice, i primi basilischi nacquero da uova deposte da serpenti e covate da galli, ma ben poco nella fisiologia del basilisco lascia spazio a questa teoria.

I basilischi vivono in quasi tutti gli ambienti asciutti, dalla foresta al deserto, e la loro pelle tende a rispecchiare l'ambiente che li circonda

Tendono a usare come rifugio le grotte, le tane o altre zone riparate. Questi rifugi sono spesso segnalati da statue raffiguranti persone e animali in pose naturali, che non sono altro che i resti pietrificati degli sventurati imbattutisi in un basilisco.

I basilischi hanno la capacità di consumare le creature pietrificate. Quando non sono in attesa dei piccoli mammiferi, uccelli o rettili che fanno parte della loro dieta, i basilischi passano il tempo a dormire nelle tane. Coloro che sono abbastanza coraggiosi da catturare i basilischi o da nascondere un tesoro vicino a loro, scoprono che questi esseri possono fare da custodi o da cani da guardia.

Un basilisco adulto è lungo quasi 3 metri, di cui la metà occupata dalla lunga coda, e pesa 135 chili. Alcune razze presentano delle piccole corna ricurve sul naso o piccole creste di pungiglioni ossuti sopra la testa simili a una corona. Sebbene siano creature in genere solitarie che si riuniscono solo per accoppiarsi e deporre le uova, in zone particolarmente pericolose possono riunirsi in piccoli gruppi per proteggersi e attaccare gli intrusi in massa.

Per motivi ignoti, le donnole, furetti e topine sono immuni allo sguardo del basilisco, e a volte si intrufolano nelle tane mentre l'adulto è a caccia per cibarsi dei suoi piccoli.

%\begin{center}
%\includegraphics[width=0.45\textwidth]{immagini/head-of-a-war-hammer.png}
%\end{center}

\mostro{Behir}
\noindent
\begin{description}[noitemsep, topsep=0pt, parsep=0pt, partopsep=0pt, leftmargin=0cm, labelwidth=2.2cm]
	\item[\textbf{Taglia/Tipo:}] Enorme mostruosità, malvagio
	\item[\textbf{Caratt.:}] \resizebox{0.5\linewidth+1.8cm}{!}{For 6 Des 3 Cos 4 Int -2 Sag 2 Car 1}
	\item[\textbf{Punti Ferita:}] 221,  \textbf{Difesa:} 29,  \textbf{Iniziativa:} +3
	\item[\textbf{Movimento:}] 15 m, scalata 12 m
	\item[\textbf{Tiri Salvez.:}] \resizebox{0.5\linewidth+1.8cm}{!}{\resizebox{0.5\linewidth+1.8cm}{!}{Tempra +15, Riflessi +14, Volontà +13}}
	\item[\textbf{Comp.:}] Furtività +7
	\item[\textbf{Imm. Danni:}] Elettricità
	\item[\textbf{Sensi:}] Scurovisione 27 m
	\item[\textbf{Linguaggi:}] Draconico
	\item[\textbf{Sfida:}] 11 (7200 PX)\smallskip
\end{description}

\textbf{Azioni}

\emph{\textbf{Multiattacco.}} Il behir effettua due attacchi: uno con il morso e uno per stritolare.

\emph{\textbf{Morso.} Attacco con arma da mischia}: +13 a colpire, portata 3 m, un bersaglio.

\emph{Colpisce:} 22 (3d10 + 6) danni perforanti.

\emph{\textbf{Stritolare.} Attacco con arma da mischia}: +13 a colpire, portata 1 m, una creatura di taglia Grande o inferiore.

\emph{Colpisce:} 17 (2d10 + 6) danni contundenti più 17 (2d10 + 6) danni taglienti. Il bersaglio è afferrato (DC 16 per fuggire).

\emph{\textbf{Inghiottire.}} Il behir effettua una attacco di morso contro un bersaglio di taglia Media o inferiore che sta afferrando. Se l'attacco colpisce, il bersaglio è inghiottito, e l'afferrare ha termine. Il bersaglio inghiottito è accecato e intralciato, ha copertura completa contro gli attacchi e altri effetti all'esterno del behir, e subisce 21 (6d6) danni da acido all'inizio di ciascun round del behir. Il behir può inghiottire solo una creatura alla volta.

Se il behir subisce 30 o più danni in un singolo round da una creatura che ha inghiottito, deve riuscire un Tiro Salvezza di Tempra DC 19 al termine di quel round o vomitare la creatura, che ricade prona in uno spazio entro 3 metri dal behir. Se il behir muore, una creatura inghiottita non è più intralciata da esso e può uscire dal cadavere utilizzando 2 Azioni e uscendo prona.

\emph{\textbf{Soffio di Fulmine (Ricarica 5-6).}} Il behir esala fulmini in una linea lunga 6 metri e larga 1 metro. Ogni creatura su quella linea deve effettuare un Tiro Salvezza di Riflessi DC 24 e subire 66 (12d10) danni da elettricità se fallisce il Tiro Salvezza, o la metà di questi danni se lo riesce.

\emph{\textbf{Arrabbiato:}} il Behir ricarica il soffio di fulmine. Costa 2 Azioni.

\textbf{Ecologia}\\
Ambiente: Colline e Deserti Caldi\\
Organizzazione: Solitario o coppia\\
\textbf{Categoria Tesoro}: U\\
\textbf{Descrizione}\\
Il behir è una creatura territoriale che striscia per le colline sabbiose e le rocce del deserto, cacciando tutte le creature che osano entrare nel suo territorio. Con una lunghezza media di 12 metri e un peso di circa 1800 kg, il behir è dotato di sei paia di zampe con artigli che utilizza in combattimento per afferrare i nemici, correre o scalare.

Nonostante la sua furia bestiale, il behir non è necessariamente malvagio e può essere convinto da negoziatori intrepidi. I behir sono spesso legati ai draghi blu, ma la natura di questo legame è sconosciuta. Questo legame suscita rancore nei behir, rendendoli pronti ad attaccare qualunque drago entri nel loro territorio.

\mostro{Blatta Esplosiva}
\noindent
\begin{description}[noitemsep, topsep=0pt, parsep=0pt, partopsep=0pt, leftmargin=0cm, labelwidth=2.2cm]
	\item[\textbf{Taglia/Tipo:}] Piccolo Elementale, neutrale
	\item[\textbf{Caratt.:}] \resizebox{0.5\linewidth+1.8cm}{!}{For 1 Des 2 Cos 1 Int -5 Sag -1 Car -2}
	\item[\textbf{Punti Ferita:}] 51,  \textbf{Difesa:} 16,  \textbf{Iniziativa:} +2
	\item[\textbf{Movimento:}] 4 m, salto 9 m, scavare 2 m
	\item[\textbf{Tiri Salvez.:}] \resizebox{0.5\linewidth+1.8cm}{!}{Tempra +3, Riflessi +4, Volontà +3}
	\item[\textbf{Res. Danni:}] contundenti
	\item[\textbf{Imm. Danni:}] Fuoco
	\item[\textbf{Immunità:}] affaticato, spaventato
	\item[\textbf{Sensi:}] Vista Cieca 5 m
	\item[\textbf{Sfida:}] 2 (450 PX)\smallskip
\end{description}

\emph{Individuazione del fuoco}: la Blatta Esplosiva può percepire fuochi entro 100 metri di distanza, purché pari o superiori ad una torcia

\emph{Scavare}: la blatta esplosiva può scavare nel terreno solido a metà del proprio movimento.

\textbf{Azioni}

\emph{\textbf{Multiattacco.}} la Blatta Esplosiva può effettuare 1 attacco di carica oppure emettere una poltiglia di fuoco.

\emph{\textbf{Carica.}} Attacco da mischia: +6 a a colpire, portata 1 metro, un bersaglio.

\emph{Colpisce:} 12 (3d6 + 3) danni contundenti. La creature deve effettuare un Tiro Salvezza su Tempra a DC 12 o cadere prona.

\emph{\textbf{Poltiglia di Fuoco}} Attacco da distanza: +5 al colpire, gittata 3 metri. La Blatta Esplosiva rigurgita un liquido appiccicoso e infiammabile all'aria. Ricarica 3-6.

\emph{Colpisce:} 18 (4d6 + 6) danni da fuoco. Tiro Salvezza su Riflessi DC 14 per dimezzare.

\emph{\textbf{Morte:}} Quando la Blatta Esplosiva muore la gelatina all'interno a contatto con l'aria esplode tutto intorno, nel raggio di 1 metro attorno alla blatta le fiamme causano 12 (4d6) di danno, Tiro Salvezza su Riflessi DC 15 per dimezzare.

\textbf{Ecologia}\\
Ambiente: caverne calde\\
Organizzazione: Solitario, nido (8-64)\\
\textbf{Categoria Tesoro}: Diamante 1d4x1d50mo\\
\textbf{Descrizione}\\
Le Blatta Esplosive sono creature native tra il piano elementale del fuoco e della terra. Solitamente attirati da ambienti ricchi di fiamme, pietra o almeno caldo e terra.
Dalla forma proporzionata a quelli di una comune blatta se non lunga circa 40 cm e pensante circa 4 kg, è una creatura completamente priva di intelletto agendo solo per puro istinto.
Sono ormai comuni nella caverne prossime a vulcani o tane di drago rosso essendosi abituate a vivere sulla Terra.

Nel nido dove dimorano c'è almeno una regina che comanda le blatte, estremamente più grossa e forte. Le Blatte Esplosive si nutrono di carbone, legni bruciati, carcasse bruciate. Sono estremamente golosi di diamanti che una volta bruciati sono delle vere e proprie leccornie.

\mostro{B.O.C.}
\noindent
\begin{description}[noitemsep, topsep=0pt, parsep=0pt, partopsep=0pt, leftmargin=0cm, labelwidth=2.2cm]
	\item[\textbf{Taglia/Tipo:}] grande mostruosità, malvagio
	\item[\textbf{Caratt.:}] \resizebox{0.5\linewidth+1.8cm}{!}{For 4 Des 3 Cos 2 Int -2 Sag 1 Car -1}
	\item[\textbf{Punti Ferita:}] 88,  \textbf{Difesa:} 20,  \textbf{Iniziativa:} +3
	\item[\textbf{Movimento:}] 13 m
	\item[\textbf{Tiri Salvez.:}] \resizebox{0.5\linewidth+1.8cm}{!}{Tempra +6, Riflessi +7, Volontà +5}
	\item[\textbf{Comp.:}] Furtività +8
	\item[\textbf{Sensi:}] Scurovisione 20 m, Visione Crepuscolare 18 m
	\item[\textbf{Linguaggi:}] Comune, può solo comprenderlo
	\item[\textbf{Sfida:}] 4 (1100 PX)\smallskip
\end{description}

\textbf{Azioni}

\emph{\textbf{Multiattacco.}} Il B.O.C effettua due attacchi con artigli ed uno con il morso, oppure effettua due attacchi con i tentacoli

\emph{\textbf{Artigli.} Attacco con arma da mischia}: +7 a colpire, portata 3 m, un bersaglio, 1 danno da Sanguinamento.

\emph{Colpisce:} 7 (1d6 + 4) danni da taglio.

\emph{\textbf{Morso.} Attacco con arma da mischia}: per ogni artiglio che ha colpito il B.O.C ottiene +2 al colpire con il morso. +6 a colpire, portata 3 m, un bersaglio.

\emph{Colpisce:} 10 (1d8 + 6) danni da taglio.

\emph{\textbf{Tentacoli.} Attacco con arma da mischia}: Ogni tentacolo può colpire fino a 6 metri di distanza ed ognuno può colpire un bersaglio diverso, +6 al colpire.

\emph{Colpisce:} 6 (1d4 + 4) danni contundenti

\textbf{Reazione: \emph{Frustata}}: quando colpito da un avversario in mischia il B.O.C. usa una Reazione per effettuare un colpo con i Tentacoli.

\emph{\textbf{Deflettere la luce.}} Il B.O.C. è costantemente influenzato da un effetto che ne altera la posizione, ogni Tiro per Colpire ha -1d6. Questa penalità si elimina se si può attaccare il B.O.C. senza usare la vista per individuarlo.

Il B.O.C. piega costantemente la luce intorno a se apparendo quasi un metro spostato rispetto alla sua reale posizione. Questa abilità non è influenzata da visioni di tipo normali, solo \hyperlink{Visione del Vero}{Visione del Vero}, vista cieca o senso tellurico possono percepire correttamente il B.O.C.

\textbf{Ecologia}\\
Ambiente: Colline e foreste\\
Organizzazione: Solitario, coppia oppure branco (2d4)\\
\textbf{Categoria Tesoro}: I\\
\textbf{Descrizione}\\
Il Black Ops Cat meglio conosciuto con B.O.C. è un grande felino predatore, ovviamente di colore nero. Feroce, insaziabile, uccide per il gusto di cacciare. Agisce solitamente in branco ed è estremamente fedele al gruppo.

\mostro{Bugbear}
\noindent
\begin{description}[noitemsep, topsep=0pt, parsep=0pt, partopsep=0pt, leftmargin=0cm, labelwidth=2.2cm]
	\item[\textbf{Taglia/Tipo:}] Media umanoide (goblinoide), Arrogante, Impulsivo
	\item[\textbf{Caratt.:}] \resizebox{0.5\linewidth+1.8cm}{!}{For 2 Des 2 Cos 1 Int -1 Sag 0 Car -1}
	\item[\textbf{Punti Ferita:}] 33,  \textbf{Difesa:} 15,  \textbf{Iniziativa:} +2
	\item[\textbf{Movimento:}] 9 m
	\item[\textbf{Tiri Salvez.:}] \resizebox{0.5\linewidth+1.8cm}{!}{Tempra +3, Riflessi +3, Volontà +3}
	\item[\textbf{Comp.:}] Furtività +6, Sopravvivenza +2
	\item[\textbf{Sensi:}] Scurovisione 18 m
	\item[\textbf{Linguaggi:}] Comune, Goblin
	\item[\textbf{Sfida:}] 1 (200 PX)\smallskip
\end{description}

\emph{\textbf{Attacco di Sorpresa.}} Se il bugbear sorprende una creatura e la colpisce con un attacco durante il primo round di combattimento, il bersaglio subisce 7 (2d6) danni aggiuntivi dall'attacco.

\emph{\textbf{Bruto.}} Un'arma da mischia infligge un dado aggiuntivo di danno quando il bugbear colpisce con essa (già incluso nell'attacco).

\textbf{Azioni}

\emph{\textbf{Mazza Chiodata.} Attacco con arma da mischia}: +4 a colpire, portata 1 m, un bersaglio.

\emph{Colpisce:} 11 (2d8 + 2) danni perforanti.

\emph{\textbf{Giavellotto.} Attacco con arma da mischia o a Distanza}: +4 a colpire, portata 1 m o gittata 12m, un bersaglio.

\emph{Colpisce:} 9 (2d6 + 2) danni perforanti in mischia o 5 (1d6 + 2) danni perforanti a gittata.

\textbf{Ecologia}\\
Ambiente: Montagne temperate\\
Organizzazione: Solitario, coppia, gruppo (3-6) o banda da guerra (7-12 più 2 Guerrieri di 1° livello e 1 capitano di 3°-5° livello)\\
\textbf{Categoria Tesoro}: Equipaggiamento da PNG (Armatura di Cuoio, Scudo Leggero di Legno, Mazza chiodata, 3 Giavellotti, O)\\
\textbf{Descrizione}\\
Il bugbear è il più grande degli esponenti della razza Goblinoide, un bruto dai movimenti pesanti che supera di almeno una testa la maggior parte degli Umani. Sono solitari che preferiscono vivere ed uccidere da soli piuttosto che in tribù, sebbene non sia insolito trovare una piccola banda di Bugbear che collabora o vive con una tribù di Goblin od Hobgoblin fungendo da guardia d'élite o carnefici.

I bugbear non formano grandi insediamenti come i goblin o nazioni come gli hobgoblin; preferiscono qualcosa di più piccolo e caotico che li lasci liberi di fare quello che preferiscono (uccidere e torturare) a un livello più personale. Gli umani sono le prede preferite dei bugbear, e la maggior parte di essi annovera la carne umana come uno degli alimenti principali della propria dieta. Macabri trofei quali orecchie e dita sono decorazioni comuni tra i bugbear.

I bugbear, quando si rivolgono alla religione, prediligono le divinità dell'omicidio e della violenza, con i vari signori dei demoni tra i preferiti. Un tipico bugbear è alto 2,1 metro e pesa 200 kg.

\mostro{Bulette}
\noindent
\begin{description}[noitemsep, topsep=0pt, parsep=0pt, partopsep=0pt, leftmargin=0cm, labelwidth=2.2cm]
	\item[\textbf{Taglia/Tipo:}] Grande bestia, disallineato
	\item[\textbf{Caratt.:}] \resizebox{0.5\linewidth+1.8cm}{!}{For 4 Des 0 Cos 5 Int -4 Sag 0 Car -3}
	\item[\textbf{Punti Ferita:}] 110,  \textbf{Difesa:} 18,  \textbf{Iniziativa:} +0
	\item[\textbf{Movimento:}] 12 m, scavo 12 m
	\item[\textbf{Tiri Salvez.:}] \resizebox{0.5\linewidth+1.8cm}{!}{Tempra +10, Riflessi +5, Volontà +5}
	\item[\textbf{Comp.:}] Consapevolezza +6
	\item[\textbf{Sensi:}] Scurovisione 18 m, percezione tellurica 18 m
	\item[\textbf{Sfida:}] 5 (1800 PX)\smallskip
\end{description}

\emph{\textbf{Salto da Fermo.}} Un bulette può saltare in lungo fino a 9 metri e in alto fino a 5 m con o senza la rincorsa.

\textbf{Azioni}

\emph{\textbf{Morso.} Attacco con arma da mischia}: +7 a colpire, portata 1 m, un bersaglio.

\emph{Colpisce:} 30 (4d12 + 4) danni perforanti.

\emph{\textbf{Salto Letale.}} Se il bulette può saltare di almeno 3 metri come parte del suo movimento, può usare poi questa azione per atterrare in piedi in uno spazio che contiene una o più creature. Ciascuna di queste creature deve riuscire un Tiro Salvezza di Tempra o Riflessi DC 18 (a scelta del bersaglio) o venire gettata prona e subire 14 (3d6 + 4) danni contundenti più 14 (3d6 + 4) danni taglienti. Se il Tiro Salvezza riesce, la creatura subisce solo la metà dei danni, non è gettata prona, e viene spinta di 1 metro fuori dello spazio del bulette in uno spazio non occupato a scelta della creatura. Se non ci sono spazi non occupati a gittata, la creatura cade prona nello spazio del bulette.

\emph{\textbf{Fiuto del sangue.}} la bulette concentra la sua attenzione su una creatura che ha ferito, 1 Azione, fino alla fine del combattimento o finché la creatura non è totalmente guarita ha +2 al Tiro per Colpire.

\emph{\textbf{Arrabbiato:}} La bulette si ricarica delle ultime energie, recupera tre volte il suo GS in punti ferita. Costa 1 Azione.

\textbf{Ecologia}\\
Ambiente: Colline Temperate\\
Organizzazione: Solitario o coppia\\
\textbf{Categoria Tesoro}: Nessuno\\
\textbf{Descrizione}\\
Creazione di uno sconosciuto mago del passato, il bulette ora è diventato un feroce predatore di collina. Scavando rapidamente sotto il terreno, fende la superficie con la sua pinna dorsale lasciandosi dietro una scia caratteristica. Il bulette balza fuori, liberandosi da pietre e terriccio, per fare a pezzi la sua preda senza rimorsi, dando così origine al suo soprannome di \emph{squalo terrestre}.

I bulette sono noti per il pessimo carattere, e attaccano creature molto più grandi di loro senza alcuna paura. Bestie solitarie tranne per le occasionali coppie in fase riproduttiva, passano la maggior parte del tempo pattugliando i loro territori, che possono superare i 4 km2, cacciando e punendo gli intrusi con una furia in grado di scuotere i pendii delle colline.

I bulette sono macchine perfette per divorare e distruggere ossa, armature e anche oggetti magici con le loro possenti mascelle e l'acido ribollente del loro stomaco. In mancanza d'altro, un bulette potrebbe sgranocchiare oggetti comuni, ma per qualche ragione non mangia volontariamente carne di elfo, segno forse di un coinvolgimento della magia elfica nella loro creazione, o di nani, anche se può far strage dei membri di entrambe le razze. Gli Gnomi, invece, sono tra i cibi preferiti di queste bestie, e non ci sono Gnomi assennati che si avventurino nel territorio di un bulette a cuor leggero.

Il bulette è un combattente astuto, e sorprende i nemici con agilità impressionante. Una delle sue tattiche preferite è lanciarsi alla carica e balzare sulla preda attaccando con i suoi artigli affilati come rasoi. Si dice che la carne dietro la cresta dorsale della bestia sia particolarmente tenera, e che quanti vogliano o riescano ad attendere che la pinna venga sollevata nella concitazione del combattimento o dell'accoppiamento possano tentare di sferrare un colpo mortale in quel punto, anche se la maggior parte di quelli che hanno affrontato uno squalo terrestre concordano sul fatto che il miglior modo per vincere un combattimento con un bulette sia evitarlo del tutto.

%\addcontentsline{toc}{subsubsection}{C}
\pdfbookmark[3]{C}{C}

\mostro{Cavaliere Nero}
\noindent
\begin{description}[noitemsep, topsep=0pt, parsep=0pt, partopsep=0pt, leftmargin=0cm, labelwidth=2.2cm]
	\item[\textbf{Taglia/Tipo:}] Media non morto, Arrogante, Paziente
	\item[\textbf{Caratt.:}] \resizebox{0.5\linewidth+1.8cm}{!}{For 5 Des 1 Cos 5 Int 1 Sag 2 Car 3}
	\item[\textbf{Punti Ferita:}] 357,  \textbf{Difesa:} 37,  \textbf{Iniziativa:} +1
	\item[\textbf{Movimento:}] 9 metri
	\item[\textbf{Tiri Salvez.:}] \resizebox{0.5\linewidth+1.8cm}{!}{\resizebox{0.5\linewidth+1.8cm}{!}{Tempra +23, Riflessi +19, Volontà +20}}
	\item[\textbf{Comp.:}] Intimidire +12, Religione +8, Conoscenza Piani +8, Conoscenza Arcana +5
	\item[\textbf{Res. Danni:}] Freddo, Elettricità
	\item[\textbf{Imm. Danni:}] da Vuoto, Veleno; armi +1
	\item[\textbf{Immunità:}] affascinato, paralizzato, affaticato, spaventato, sanguinamento
	\item[\textbf{Sensi:}] Scurovisione 36 m
	\item[\textbf{Linguaggi:}] Comune, Abissale, Expiran
	\item[\textbf{Sfida:}] 18 (20000 PX)\smallskip
\end{description}

\emph{\textbf{Incantesimi.}} Il Cavaliere Nero ha CM 7. La sua caratteristica da incantatore è il Carisma. Il Cavaliere Nero conosce i seguenti incantesimi:

livello 1 (4 slot): \emph{\hyperlink{Comando}{Comando},  \hyperlink{Dardo arcano}{Dardo arcano}, Onda rovente, \hyperlink{Scudo}{Scudo}}

livello 2 (3 slot): \emph{\hyperlink{Blocca Persona}{Blocca Persona}, arma magica}

livello 3 (3 slot): \emph{\hyperlink{Controincantesimo}{Controincantesimo}, \hyperlink{Dissolvi Magie}{Dissolvi Magie}, \hyperlink{Palla di Fuoco}{Palla di Fuoco}}

livello 4 (3 slot): \emph{\hyperlink{Esilio}{Esilio}, Punizione marchiante (con 1 critico magico automatico, danno da Vuoto)}

\emph{\textbf{Natura Non Morta.}} Il Cavaliere Nero non ha bisogno di aria, cibo, bevande o sonno.

\emph{\textbf{Resistenza Leggendaria (1/Giorno).}} Se il Cavaliere Nero fallisce un Tiro Salvezza, può scegliere invece di riuscirvi.

\emph{\textbf{Resistenza allo Scacciare.}} Il Cavaliere Nero ha +1d6 ai Tiri Salvezza contro gli effetti che scacciano i non morti.

\textbf{Azioni}

\emph{\textbf{Multiattacco.} 3 attacchi con spada lunga +3}: +27 al colpire, portata 1 m, fino a tre creature differenti, oppure 1 colpo di spada con Corruzione

\emph{Colpisce:} 13 (1d10+5+3) danni da taglio + Colpo Fiammeggiante (danno da Vuoto)

\emph{Corruzione:} 15 (1d10+10) danni da taglio. L'obiettivo deve fare un Tiro Salvezza su Volontà DC 30 oppure perdere un 1/10 di un punto Tratto legato ad un Patrono buono se presente.

\textbf{Reazione: \emph{Attacco d'opportunità}}: il Cavaliere nero effettua un attacco ad una creatura che attraversi o esca dalla sua portata di 1 metro.

\textbf{Ecologia}\\
Ambiente: Qualsiasi\\
Organizzazione: Solitario\\
\textbf{Categoria Tesoro}: spada lunga +3 od armatura completa +3, il resto dell'equipaggiamento scompare con la morte del Cavaliere Nero.\\
\textbf{Descrizione}\\
Dannato fin nel profondo della sua anima il Cavaliere Nero è l'antitesi del cavaliere di Sumkjr, anzi spesso nasce dalla corruzione di un cavaliere di Sumkjr. Avversario temibile, furbo, tattico, adora comportarsi e ragionare malignamente, come una persona ancora viva. La sua tattica è di lanciare la \hyperlink{Palla di Fuoco}{Palla di Fuoco} il prima possibile per poi consumare la vittima con Punizione marchiante.

\mostro{Centauro}
\noindent
\begin{description}[noitemsep, topsep=0pt, parsep=0pt, partopsep=0pt, leftmargin=0cm, labelwidth=2.2cm]
	\item[\textbf{Taglia/Tipo:}] Grande mostruosità, buono
	\item[\textbf{Caratt.:}] \resizebox{0.5\linewidth+1.8cm}{!}{For 4 Des 2 Cos 2 Int -1 Sag 1 Car 0}
	\item[\textbf{Punti Ferita:}] 51,  \textbf{Difesa:} 16,  \textbf{Iniziativa:} +2
	\item[\textbf{Movimento:}] 15 m
	\item[\textbf{Tiri Salvez.:}] \resizebox{0.5\linewidth+1.8cm}{!}{Tempra +4, Riflessi +4, Volontà +3}
	\item[\textbf{Comp.:}] Atletica +6, Consapevolezza +3, Sopravvivenza +3
	\item[\textbf{Linguaggi:}] Elfico, Silvano
	\item[\textbf{Sfida:}] 2 (450 PX)\smallskip
\end{description}

\emph{\textbf{Carica.}} Se il centauro si muove di almeno 9 metri diretto verso il bersaglio e colpisce con un attacco di picca durante lo stesso round, il bersaglio subisce 10 (3d6) danni perforanti aggiuntivi.

\textbf{Azioni}

\emph{\textbf{Multiattacco.}} Il centauro effettua due attacchi: uno con la picca e uno con gli zoccoli o due con l'arco lungo.

\emph{\textbf{Picca.} Attacco con arma da mischia}: +5 a colpire, portata 3 m, un bersaglio.

\emph{Colpisce:} 9 (1d10 + 4) danni perforanti.

\emph{\textbf{Zoccoli.} Attacco con arma da mischia}: +6 a colpire, portata 1 m, un bersaglio.

\emph{Colpisce:} 11 (2d6 + 4) danni contundenti.

\emph{\textbf{Arco Lungo.} Attacco con arma a Distanza}: +4 a colpire, gittata 45m, un bersaglio.

\emph{Colpisce:} 6 (1d8 + 2) danni perforanti.

\textbf{Ecologia}\\
Ambiente: Pianure e foreste temperate\\
Organizzazione: Solitario, coppia, banda (3-10), tribù (11-30 più 3 cacciatori di 3° livello e 1 capo di 6° livello)\\
\textbf{Categoria Tesoro}: B\\
\textbf{Descrizione}\\
Leggendari cacciatori e abili guerrieri, i centauri sono in parte uomini e in parte cavalli. Generalmente collocata ai margini della civilizzazione, questa stoica popolazione varia enormemente come aspetto: di solito il colore della pelle è molto abbronzato ma simile a quello degli umani delle regioni limitrofe, mentre la parte inferiore del corpo ha le tonalità degli equini locali. Hanno capelli e occhi di colore scuro e i tratti del volto piuttosto marcati, mentre la loro stazza totale dipende dalla taglia del cavallo di cui hanno la parte inferiore del corpo. Quindi, anche se un centauro medio è alto in piedi 2,1 metro e pesa più di 1000 kg, esistono molteplici varianti regionali, dagli esili corridori delle pianure ai massicci cacciatori di montagna.

I centauri vivono in media circa 60 anni. Distanti dalle altre razze e in conflitto con gli altri della loro specie, i centauri sono una razza antica che lentamente comincia ad accettare il mondo moderno. Anche se la maggioranza dei centauri vive ancora in tribù vagando per vaste pianure o ai margini di mistiche foreste, alcuni hanno abbandonato i modi isolazionisti dei loro antenati per stabilirsi in città cosmopolite. Spesso questi spiriti liberi sono considerati dei reietti e vengono disprezzati dalle loro tribù, e pertanto la decisione di abbandonarle è una scelta pesante. In alcuni casi, comunque, intere tribù guidate da capi progressisti hanno cominciato a commerciare o stringere alleanze con altre comunità di umanoidi, specie Elfi, a volte Gnomi, e più raramente Umani o Nani. Molte razze rimangono caute nei confronti dei centauri, però, per lo più a causa di leggende che li ritraggono come creature territoriali e feroci e dei periodici scontri violenti che essi hanno con i coloni testardi e i paesi in via di espansione.

La leggenda vuole che i Centauri dovessero esplodere come tutti gli equini, per volere di Calicante. Ljust inorridita da tanta morte intercesse su Calicante perché lasciasse stare queste creature. Questo salvataggio ha portato molti tribù di Centauri ad essere devoti della Signora della Luce, anche se altri hanno preferito invece dedicare il loro culto a Calicante nella speranza che non li uccida tutti in una notte.

\mostro{Chimera}
\noindent
\begin{description}[noitemsep, topsep=0pt, parsep=0pt, partopsep=0pt, leftmargin=0cm, labelwidth=2.2cm]
	\item[\textbf{Taglia/Tipo:}] Grande mostruosità, Arrogante, Vanitoso
	\item[\textbf{Caratt.:}] \resizebox{0.5\linewidth+1.8cm}{!}{For 4 Des 0 Cos 4 Int -4 Sag 2 Car 0}
	\item[\textbf{Punti Ferita:}] 127,  \textbf{Difesa:} 20,  \textbf{Iniziativa:} +0
	\item[\textbf{Movimento:}] 9 m, volo 18 m
	\item[\textbf{Tiri Salvez.:}] \resizebox{0.5\linewidth+1.8cm}{!}{Tempra +10, Riflessi +6, Volontà +8}
	\item[\textbf{Comp.:}] Consapevolezza +8
	\item[\textbf{Sensi:}] Scurovisione 18 m
	\item[\textbf{Linguaggi:}] comprende il Draconico ma non può parlare
	\item[\textbf{Sfida:}] 6 (2300 PX)\smallskip
\end{description}

\textbf{Azioni}

\emph{\textbf{Multiattacco.}} La chimera effettua tre attacchi: uno con il morso, uno con le corna e uno con gli artigli. Quando il soffio infuocato è disponibile, può usare il soffio al posto del morso o delle corna.

\emph{\textbf{Artigli.} Attacco con arma da mischia}: +7 a colpire, portata 1 m, un bersaglio.

\emph{Colpisce:} 11 (2d6 + 4) danni taglienti, 1 danno da Sanguinamento.

\emph{\textbf{Corna.} Attacco con arma da mischia}: +7 a colpire, portata 1 m, un bersaglio.

\emph{Colpisce:} 10 (1d12 + 4) danni contundenti.

\emph{\textbf{Morso.} Attacco con arma da mischia}: +6 a colpire, portata 1 m, un bersaglio.

\emph{Colpisce:} 11 (2d6 + 4) danni perforanti.

\emph{\textbf{Soffio Infuocato (Ricarica 5-6).}} La testa di drago esala fuoco in un cono di 5 metri. Ogni creatura in quell'area deve effettuare un Tiro Salvezza di Riflessi DC 21 e subire 31 (7d8) danni da fuoco se fallisce il Tiro Salvezza, o la metà di questi danni se lo riesce.

\textbf{Reazione: \emph{Attacco d'opportunità}}: la Chimera effettua un attacco ad una creatura che attraversi o esca dalla sua portata di 1 metro.

\emph{\textbf{Arrabbiato:}} la Chimera brilla di energia. Ricarica il Soffio Infuocato. Costa 1 Azione.

\textbf{Ecologia}\\
Ambiente: Colline Temperate\\
Organizzazione: Solitario, coppia, branco (3-6) o stormo (7-12)\\
\textbf{Categoria Tesoro}: F\\
\textbf{Descrizione}\\
Le chimere sono mostruose creature nate dal male primordiale. Odiose e fameliche, cacciano sia a terra che in aria. La testa di drago di una chimera può essere di qualunque tipo di drago malvagio, con il soffio corrispondente e le ali generalmente dotate delle stesse scaglie della testa. Le chimere parlano con tre voci che si sovrappongono, ma lo fanno raramente, tipicamente solo per adulare una creatura più potente. Una chimera è alta al garrese 1 metro, raggiungendo la lunghezza di 4 metri e il peso di 350 kg.\\
Le chimere preferiscono la carne, ma possono sopravvivere di vegetali se necessario (anche se quando sono costrette a farlo il loro umore peggiora ulteriormente). Il fatto che volino significa che possono scegliere con attenzione le loro prede, e generalmente cacciano in vaste aree cercando quelle facili. Sono troppo stupide e belligeranti per acquisire seguaci, anche se a volte una tribù di coboldi può far loro delle offerte. Al contrario, sono abbastanza intelligenti e caparbie da essere mediocri animali domestici, e solo una creatura molto più potente di loro può riuscire a sottometterle. Possono formare collaborazioni paritarie con umanoidi rispettosi o creature simili, e acconsentono anche ad essere usate come cavalcature. Un branco di chimere ha una gerarchia simile a quella dei leoni, con un maschio dominante che comanda il gruppo e la maggior parte delle cacce svolte dalle femmine. Una chimera solitaria può essere un giovane maschio solitario o una femmina con i cuccioli nelle vicinanze.

\mostro{Chuul}
\noindent
\begin{description}[noitemsep, topsep=0pt, parsep=0pt, partopsep=0pt, leftmargin=0cm, labelwidth=2.2cm]
	\item[\textbf{Taglia/Tipo:}] Grande aberrazione, malvagio
	\item[\textbf{Caratt.:}] \resizebox{0.5\linewidth+1.8cm}{!}{For 4 Des 0 Cos 3 Int -3 Sag 0 Car -3}
	\item[\textbf{Punti Ferita:}] 89,  \textbf{Difesa:} 17,  \textbf{Iniziativa:} +0
	\item[\textbf{Movimento:}] 9 m, nuoto 9 m
	\item[\textbf{Tiri Salvez.:}] \resizebox{0.5\linewidth+1.8cm}{!}{Tempra +7, Riflessi +4, Volontà +4}
	\item[\textbf{Comp.:}] Consapevolezza +4
	\item[\textbf{Imm. Danni:}] Veleno
	\item[\textbf{Sensi:}] Scurovisione 18 m
	\item[\textbf{Linguaggi:}] comprende la Linguaggio delle Profondità ma non può parlare
	\item[\textbf{Sfida:}] 4 (1100 PX)\smallskip
\end{description}

\emph{\textbf{Anfibio.}} Il chuul può respirare aria e acqua.

\emph{\textbf{Senso della Magia.}} Il chuul percepisce la magia entro 36 metri da sé. Questo tratto funziona come l'incantesimo \emph{individuazione} \emph{del magico} ma di per sé non è magico.

\textbf{Azioni}

\emph{\textbf{Multiattacco.}} Il chuul effettua due attacchi con le chele. Se il chuul sta afferrando una creatura, può anche usare i suoi tentacoli una volta.

\emph{\textbf{Chele.} Attacco con arma da mischia}: +7 a colpire, portata 3 m, un bersaglio.

\emph{Colpisce:} 11 (2d6 + 4) danni contundenti. Un bersaglio è afferrato (DC 14 per fuggire) se è di taglia Grande o inferiore e il chuul non sta già afferrando altre due creature.

\emph{\textbf{Tentacoli.}} Una creatura afferrata dal chuul deve riuscire un Tiro Salvezza di Tempra DC 16 o restare avvelenata per 1 minuto. Fino al termine dell'avvelenamento, il bersaglio è paralizzato. Il bersaglio può ripetere il Tiro Salvezza al termine di ciascun suo round, terminando l'effetto per sé in caso di successo.

\textbf{Reazione: \emph{Attacco d'opportunità}}: il chuul effettua un attacco ad una creatura che attraversi o esca dalla sua portata di 3 metri.

\textbf{Ecologia}\\
Ambiente: Paludi Temperate\\
Organizzazione: Solitario, coppia o branco (3-6)\\
\textbf{Categoria Tesoro}: U (B)\\
\textbf{Descrizione}\\
I chuul sono predatori corazzati simili ai crostacei, sempre in agguato sotto la superficie degli stagni e dei pantani poco profondi, che escono dal loro nascondiglio per afferrare le loro prede con le loro chele e poi paralizzarle con i tentacoli della bocca prima di mangiarle vive.

I chuul sono eccellenti nuotatori, ma preferiscono attaccare le creature terrestri o abituate ad acque poco profonde. Una volta afferrate le loro vittime, i chuul spesso le trascinano nell'acqua profonda. I lucertoloidi sono le prede preferite dei chuul, anche se le pallide specie di chuul che vivono nei sotterranei preferiscono morlock, nani oscuri, incauti elfi e altri sfortunati che si avvicinano troppo ai loro corsi d'acqua sotterranei, ad eccezione dei trogloditi il cui sapore i chuul trovano particolarmente disgustoso.

I chuul sono sorprendentemente intelligenti e molti si impegnano in inutili speculazioni sulle loro origini e motivazioni. Parlano un cinguettante e gorgogliante dialetto del Comune, ma pochi di essi sono inclini a chiacchierare con quanti non siano della loro razza, e se esiste una società chuul al di fuori del frenetico periodo degli amori, nessuno lo ha ancora scoperto. Al contrario, le menti dei chuul sembrano dedite solo alla ricerca del luogo perfetto in cui tendere un'imboscata per attaccare altre creature intelligenti e a come decorare le loro elaborate tane con trofei delle loro vittime. Anche se i chuul sembrano non interessati all'utilizzo di utensili, hanno un bisogno compulsivo di collezionare quelli delle loro vittime. Un tipico chuul è alto 2,3 metri e pesa 325 kg.

\mostro{Coboldo}
\noindent
\begin{description}[noitemsep, topsep=0pt, parsep=0pt, partopsep=0pt, leftmargin=0cm, labelwidth=2.2cm]
	\item[\textbf{Taglia/Tipo:}] Piccola umanoide (coboldo), malvagio
	\item[\textbf{Caratt.:}] \resizebox{0.5\linewidth+1.8cm}{!}{For -2 Des 2 Cos -1 Int -1 Sag -2 Car -1}
	\item[\textbf{Punti Ferita:}] 17,  \textbf{Difesa:} 14,  \textbf{Iniziativa:} +2
	\item[\textbf{Movimento:}] 9 m
	\item[\textbf{Tiri Salvez.:}] \resizebox{0.5\linewidth+1.8cm}{!}{Tempra +3, Riflessi +3, Volontà +3}
	\item[\textbf{Sensi:}] Scurovisione 18 m
	\item[\textbf{Linguaggi:}] Comune, Draconico
	\item[\textbf{Sfida:}] 1/8 (25 PX)\smallskip
\end{description}

\emph{\textbf{Sensibilità alla Luce}}. Mentre è alla luce del sole, il coboldo ha -1d6 ai tiri per colpire, oltre che alle prove di Consapevolezza basate sulla vista.

\emph{\textbf{Tattiche di Branco.}} Il coboldo ha +1d6 ai tiri per colpire contro una creatura se almeno uno degli alleati del coboldo si trova entro 1 metro dalla creatura e quell'alleato non è inabile.

\textbf{Azioni}

\emph{\textbf{Pugnale.} Attacco con arma da mischia}: +3 a colpire, portata 1 m, un bersaglio.

\emph{Colpisce:} 4 (1d4 + 2) danni perforanti.

\emph{\textbf{Fionda.} Attacco con arma a distanza}: +3 a colpire, gittata 9m, un bersaglio.

\emph{Colpisce:} 4 (1d4 + 2) danni contundenti.

\textbf{Ecologia}\\
Ambiente: Foreste temperate o sotterranei\\
Organizzazione: solitario, gruppo (2-4), nido (5-30 più un ugual numero di non combattenti, 1 sergente di 3° livello ogni 20 adulti e 1 capo di 4°-6° livello) o tribù (31-300 più di 35\% di non combattenti, 1 sergente di 3° livello ogni 20 adulti, 2 tenenti di 4° livello, 1 capo di 6°-8° livello e 5-16 Ratti Crudeli)\\
\textbf{Categoria Tesoro}: Equipaggiamento da PNG (Armatura di Cuoio, Lancia, Fionda, altro tesoro), 2d6 monete d'argento\\
\textbf{Descrizione}\\
I coboldi sono creature dell'oscurità, che si incontrano più facilmente in enormi dedali sotterranei o negli angoli bui delle foreste dove il sole non batte mai. A causa della somiglianza fisica i coboldi si proclamano a gran voce discendenti della stirpe draconica e destinati a governare la terra sotto l'ala dei loro grandi cugini divini, ma la maggior parte dei draghi li considera poco più che insetti fastidiosi. Codardi ed intriganti, non lottano mai apertamente se possono evitarlo tendendo invece imboscate e trappole, rintanandosi nelle loro labirintiche costruzioni dietro una coltre di rozzi ma ingegnosi trabocchetti, o rovesciandosi sul nemico in vaste orde ululanti.

La tonalità dei coboldi varia anche tra i fratelli della stessa covata, spaziando tra i colori dei draghi di Tàhil, con una predominanza del rosso e porpora, e più di rado bianco, verde, blu e nero.

I coboldi hanno un debole per l'argento ma essendo pessimi minatori preferiscono predare dalle monete d'argento gli avventurieri e ne mangiano come fossero biscotti al burro. I coboldi possono digerire l'argento piuttosto velocemente e più mangiano più le loro squame sono luminose ed i coboldi sembrano sani.

\mostro{Cockatrice}
\begin{description}[noitemsep, topsep=0pt, parsep=0pt, partopsep=0pt, leftmargin=0cm, labelwidth=2.2cm]
	\item[\textbf{Taglia/Tipo:}] Piccola mostruosità, disallineato
	\item[\textbf{Caratt.:}] \resizebox{0.5\linewidth+1.8cm}{!}{For -2 Des 1 Cos 1 Int -4 Sag 1 Car -3}
	\item[\textbf{Punti Ferita:}] 24,  \textbf{Difesa:} 13,  \textbf{Iniziativa:} +1
	\item[\textbf{Movimento:}] 6 m, volo 12 m
	\item[\textbf{Tiri Salvez.:}] \resizebox{0.5\linewidth+1.8cm}{!}{Tempra +3, Riflessi +3, Volontà +3}
	\item[\textbf{Sensi:}] Scurovisione 18 m
	\item[\textbf{Sfida:}] 1/2 (100 PX)\smallskip
\end{description}

\textbf{Azioni}

\emph{\textbf{Morso.} Attacco con arma da mischia}: +4 a colpire, portata 1 m, una creatura.

\emph{Colpisce:} 3 (1d4 + 1) danni perforanti, e il bersaglio deve riuscire un Tiro Salvezza di Tempra DC 11 o essere Rallentato 1/1r per via della progressiva pietrificazione. Se successivi morsi portano la creatura a non avere più Azioni la creatura è pietrificata per 24 ore.

\textbf{Ecologia}\\
Ambiente: Pianure temperate\\
Organizzazione: Solitario, coppia, squadriglia (3-5) o stormo (6-12)\\
\textbf{Categoria Tesoro}: D\\
\textbf{Descrizione}\\
Stupide, malevole e repellenti, le cockatrici sono evitate dalle altre creature per la loro capacità di trasformare la carne in pietra. I maschi si distinguono solo per barbigli e creste. Una tipica cockatrice è alta poco più di 60 centimetri e pesa 2,5 kg.

Anche se la loro dieta consiste principalmente di semi e insetti pietrificati (che nello stomaco della creatura fungono sia da gastroliti che da nutrimento), le cockatrici difendono ferocemente il loro territorio da tutto ciò che ritengono una minaccia, e i vagabondaggi dei maschi raminghi in cerca di nuovi luoghi dove costruire tane a volte li portano ad involontari contatti con gli umani, con risultati devastanti.

La strana capacità della cockatrice di trasformare le altre creature in pietra è la sua miglior difesa e la sua tana è invariabilmente piena di resti dei nemici pietrificati. Per ironia della sorte, tuttavia, donnole e furetti, le creature che più probabilmente finiscono nei nidi delle cockatrici per mangiarne le uova, sembrano completamente immuni a questo effetto. Per ragioni sconosciute, le cockatrici sono sia terrorizzate che furiose con i galli comuni e c'è la stessa probabilità che fuggano o attacchino quando avviene un confronto.

\mostro{Couatl}
\noindent
\begin{description}[noitemsep, topsep=0pt, parsep=0pt, partopsep=0pt, leftmargin=0cm, labelwidth=2.2cm]
	\item[\textbf{Taglia/Tipo:}] Media celestiale, buono
	\item[\textbf{Caratt.:}] \resizebox{0.5\linewidth+1.8cm}{!}{For 3 Des 5 Cos 3 Int 4 Sag 5 Car 4}
	\item[\textbf{Punti Ferita:}] 89,  \textbf{Difesa:} 22,  \textbf{Iniziativa:} +5
	\item[\textbf{Movimento:}] 9 m, volo 9 m
	\item[\textbf{Tiri Salvez.:}] \resizebox{0.5\linewidth+1.8cm}{!}{Tempra +7, Riflessi +9, Volontà +9}
	\item[\textbf{Res. Danni:}] da Luce
	\item[\textbf{Imm. Danni:}] da arma non magica
	\item[\textbf{Sensi:}] visione del vero 36 m
	\item[\textbf{Linguaggi:}] tutte, telepatia 36 m
	\item[\textbf{Sfida:}] 4 (1100 PX)\smallskip
\end{description}

\emph{\textbf{Armi Magiche.}} Gli attacchi con armi del couatl sono magici.

\emph{\textbf{Incantesimi Innati.}} La caratteristica da incantatore innato del couatl è il Carisma. Il couatl può lanciare questi incantesimi in maniera innata, usando solo componenti verbali:

A volontà: \emph{\hyperlink{Conoscere i Tratti}{Conoscere i Tratti}, \hyperlink{Individuazione del Magico}{Individuazione del Magico}, \hyperlink{Individuazione dei Pensieri}{Individuazione dei Pensieri}}

3/giorno ciascuno: \emph{\hyperlink{Benedizione}{Benedizione}, \hyperlink{Creare cibo e acqua}{Creare cibo e acqua}, \hyperlink{Cura Ferite}{Cura Ferite}, protezione dai veleni, \hyperlink{Ristorare Inferiore}{Ristorare Inferiore}, \hyperlink{Santuario}{Santuario}, \hyperlink{Scudo}{Scudo}} 1/giorno ciascuno: \emph{\hyperlink{Ristorare Superiore}{Ristorare Superiore}, \hyperlink{Scrutare}{Scrutare}, \hyperlink{Sogno}{Sogno}}

\emph{\textbf{Mente Protetta.}} Il couatl è immune allo scrutare e qualsiasi effetto che percepisca le sue emozioni, legga i suoi pensieri o individui la sua posizione.

\textbf{Azioni}

\emph{\textbf{Morso.} Attacco con arma da mischia}: +8 a colpire, portata 1 m, una creatura.

\emph{Colpisce:} 8 (1d6 + 5) danni perforanti, e il bersaglio deve riuscire un Tiro Salvezza di Tempra DC 16 o restare avvelenato per 24 ore. Fino al termine dell'avvelenamento, il bersaglio è privo di sensi. Un'altra creatura può effettuare un'Azione per risvegliare il bersaglio.

\emph{\textbf{Stritolare.} Attacco con arma da mischia}: +7 a colpire, portata 3 m, una creatura di taglia Media o inferiore.

\emph{Colpisce:} 10 (2d6 + 3) danni contundenti, e il bersaglio è afferrato (DC 15 per fuggire).

\emph{\textbf{Mutare Forma.}} Il couatl può trasformarsi magicamente in un umanoide o bestia il cui grado di sfida sia pari o inferiore al proprio, o tornare alla sua vera forma. Alla morte ritorna alla sua vera forma. Qualsiasi equipaggiamento stia indossando o trasportando viene assorbito o trasportato nella nuova forma (a scelta del couatl).

Nella nuova forma, il couatl mantiene le sue statistiche di gioco e la facoltà di parlare, ma la sua Difesa, metodi di movimento, Forza, Destrezza e altre azioni vengono rimpiazzati da quelli della nuova forma, e ottiene qualsiasi statistica o capacità (Azioni aggiuntive e azioni da tana) possedute dalla sua nuova forma e non dalla sua originale. Se la nuova forma ha un attacco di morso, il couatl può usare il proprio morso nella nuova forma.

\textbf{Ecologia}\\
Ambiente: foreste calde\\
Organizzazione: Solitario, coppia o stormo (3-6)\\
\textbf{Categoria Tesoro}: I\\
\textbf{Descrizione}\\
Rispettati ed ammirati per la loro saggezza e bellezza, cercano di portare i mortali sulla retta via e usano i loro poteri per combattere il male, specie quelli noti per viaggiare tra i piani. Un couatl è lungo circa 3,6 metri, con un'apertura alare di circa 5 metri e pesa 900 kg.

Preferiscono gli stessi alimenti dei veri serpenti, come mammiferi e uccelli, anche se è noto che divorano gli umanoidi malvagi. Poiché preferiscono passare il tempo a perseguire i loro intenti anziché cacciare, apprezzano le offerte di cibo, in particolare piccoli cinghiali e volatili. Un couatl talvolta mostra il suo apprezzamento a un avventuriero o a un gruppo che gli ha reso un servizio donandogli 1d4 delle sue brillanti piume colorate.

\mostro{Cumulo Strisciante}
\noindent
\begin{description}[noitemsep, topsep=0pt, parsep=0pt, partopsep=0pt, leftmargin=0cm, labelwidth=2.2cm]
	\item[\textbf{Taglia/Tipo:}] Grande pianta, disallineato
	\item[\textbf{Caratt.:}] \resizebox{0.5\linewidth+1.8cm}{!}{For 4 Des -1 Cos 3 Int -3 Sag 0 Car -3}
	\item[\textbf{Punti Ferita:}] 108,  \textbf{Difesa:} 17,  \textbf{Iniziativa:} -1
	\item[\textbf{Movimento:}] 6 m, nuoto 6 m
	\item[\textbf{Tiri Salvez.:}] \resizebox{0.5\linewidth+1.8cm}{!}{Tempra +8, Riflessi +4, Volontà +5}
	\item[\textbf{Comp.:}] Furtività +2
	\item[\textbf{Res. Danni:}] Freddo, Fuoco
	\item[\textbf{Imm. Danni:}] Elettricità
	\item[\textbf{Immunità:}] accecato, assordato, affaticato
	\item[\textbf{Sensi:}] Vista Cieca 18 m (cieco oltre questo raggio)
	\item[\textbf{Sfida:}] 5 (1800 PX)\smallskip
\end{description}

\emph{\textbf{Assorbimento dei Fulmini.}} Ogni qual volta il cumulo strisciante subisce danni da elettricità, non subisce danni e recupera un numero di Punti Ferita pari al danno da elettricità inferto.

\textbf{Azioni}

\emph{\textbf{Multiattacco.}} Il cumulo strisciante effettua due attacchi di schianto. Se entrambi gli attacchi colpiscono una creatura di taglia Media o inferiore, il bersaglio è afferrato (DC 14 per fuggire) e il cumulo strisciante usa Avvolgere su di esso.

\emph{\textbf{Schianto.} Attacco con arma da mischia}: +7 a colpire, portata 1 m, un bersaglio.

\emph{Colpisce:} 13 (2d8 + 4) danni contundenti.

\emph{\textbf{Avvolgere.}} Il cumulo strisciante avvolge una creatura di taglia Media o inferiore che ha afferrato. Il bersaglio avvolto è accecato e impossibilitato a respirare, e deve riuscire un Tiro Salvezza di Tempra DC 17 all'inizio di ciascun round del tumulo o subire 13 (2d8 + 4) danni contundenti. Se il cumulo si muove, il bersaglio avvolto si muove con esso. Il cumulo può avvolgere solo una creatura alla volta.

\emph{\textbf{Arrabbiato:}} Il Cumulo strisciante rilascia un'onda di elettricità. Tutte le creature entro 3 metri subiscono 3d6 di danno da elettricità. Costa 2 Azioni.

\textbf{Ecologia}\\
Ambiente: Foreste o Paludi Temperate\\
Organizzazione: Solitario\\
\textbf{Categoria Tesoro}: A\\
\textbf{Descrizione}\\
I cumuli striscianti, chiamati anche soltanto striscianti, sembrano masse vegetali in decomposizione. Sono piante carnivore intelligenti, con un debole per la carne elfica. Il cervello e gli organi sensoriali si trovano nella parte superiore del corpo. Di solito i cumuli striscianti hanno una circonferenza di 2,3 metri e sono alti da 1,8 a 2,7 metri. Pesano circa 1.900 kg.

I cumuli striscianti sono strane creature, più simili a un groviglio di rampicanti parassiti che ad una singola pianta dotata di radici. Sono onnivori, capaci di trarre sostentamento da qualsiasi cosa, avvinghiandosi agli alberi per succhiarne la linfa, inserendo le radici nel terreno per assorbire nutrienti semplici o consumando la carne e le ossa dalle prede.

I cumuli striscianti sono incredibilmente furtivi nel loro ambiente naturale. Si confondono con il terreno circostante e possono attendere immobili per giorni l'arrivo di una potenziale preda. Possono essere praticamente ovunque ed attaccare in qualsiasi momento senza alcun preavviso e senza curarsi che ci siano o meno sopravvissuti, fintanto che hanno da mangiare.

Di solito i cumuli striscianti conducono un'esistenza nomade e solitaria in profonde foreste e fetide paludi ma possono essere trovati anche sottoterra, in mezzo a boschetti di funghi. Voci preoccupanti parlano di gruppi di cumuli striscianti che si radunano intorno a grandi tumuli nelle profondità di giungle e paludi, spesso durante violente tempeste di fulmini. Il motivo di questo comportamento è sconosciuto, e molti saggi si chiedono se dietro non ci sia uno scopo oscuro ed alieno.

%\addcontentsline{toc}{subsubsection}{D}
\pdfbookmark[3]{D}{D}

\mostro{Balor}
\noindent
\begin{description}[noitemsep, topsep=0pt, parsep=0pt, partopsep=0pt, leftmargin=0cm, labelwidth=2.2cm]
	\item[\textbf{Taglia/Tipo:}] Enorme demone, malvagio
	\item[\textbf{Caratt.:}] \resizebox{0.5\linewidth+1.8cm}{!}{For 8 Des 2 Cos 6 Int 5 Sag 3 Car 6}
	\item[\textbf{Punti Ferita:}] 379,  \textbf{Difesa:} 39,  \textbf{Iniziativa:} +5
	\item[\textbf{Movimento:}] 12 m, volo 24 m
	\item[\textbf{Tiri Salvez.:}] \resizebox{0.5\linewidth+1.8cm}{!}{\resizebox{0.5\linewidth+1.8cm}{!}{Tempra +25, Riflessi +21, Volontà +22}}
	\item[\textbf{Res. Danni:}] Freddo, Elettricità;
	\item[\textbf{Imm. Danni:}] Fuoco, Veleno, armi +1
	\item[\textbf{Vulnerabilità:}] ferro freddo, Luce
	\item[\textbf{Sensi:}] visione del vero 36 m
	\item[\textbf{Linguaggi:}] Abissale, telepatia 36 m
	\item[\textbf{Sfida:}] 19 (22000 PX)\smallskip
\end{description}

\emph{\textbf{Armi Magiche.}} Gli attacchi con arma del demone sono magici.

\emph{\textbf{Aura di Fuoco.}} All'inizio di ciascun round del demone, ciascuna creatura entro 1 metro da lui subisce 10 (3d6) danni da fuoco, e gli oggetti infiammabili che si trovano nell'aura e che non sono indossati o trasportati prendono fuoco. Una creatura che entri a contatto con il demone o lo colpisca con un attacco da mischia mentre si trova entro 1 metro da esso subisce 10 (3d6) danni da fuoco.

\emph{\textbf{Resistenza alla Magia.}} Il demone ha +1d6 ai Tiri Salvezza contro incantesimi e altri effetti magici.

\emph{\textbf{Spasmo Mortale.}} Quando il demone muore, esplode; ciascuna creatura entro 9 metri da esso deve effettuare un Tiro Salvezza di Riflessi DC 31, subendo 70 (20d6) danni da fuoco se fallisce il Tiro Salvezza, o la metà di questi danni se lo riesce. L'esplosione appicca il fuoco agli oggetti infiammabili che non sono indossati o trasportati, e distrugge le armi del demone.

\textbf{Azioni}

\emph{\textbf{Multiattacco.}} Il demone effettua due attacchi: uno con la spada lunga e uno con la frusta.

\emph{\textbf{Frusta.} Attacco con arma da mischia}: +14 a colpire, portata 9 m, un bersaglio.

\emph{Colpisce:} 15 (2d6 + 8) danni taglienti più 10 (3d6) danni da fuoco, e il bersaglio deve riuscire un Tiro Salvezza di Tempra DC 32 o venire trascinato 7 metri verso il demone.

\emph{\textbf{Spada Lunga.} Attacco con arma da mischia}: +14 a colpire, portata 3 m, un bersaglio.

\emph{Colpisce:} 21 (3d8 + 8) danni taglienti più 13 (3d8) danni da elettricità.

\textbf{Reazione: \emph{Attacco d'opportunità}}: il demone effettua un attacco ad una creatura che attraversi o esca dalla sua portata di 6 metri.

\emph{\textbf{Teletrasporto.}} Il demone si teletrasporta magicamente, insieme a tutto l'equipaggiamento che indossa o trasporta, in uno spazio non occupato e che può vedere entro 36 metri.

\textbf{Ecologia}\\
Ambiente: Qualsiasi (Abisso)\\
Organizzazione: Solitario o banda di guerra (1 Balor e 2-5 Glabrezu)\\
\textbf{Categoria Tesoro}: Standard (Spada Lunga Sacrilega+1, Frusta Infuocata+1, R)\\
\textbf{Descrizione}\\
Quando la gente sussurra terrificanti racconti di creature demoniache, immagina per lo più un'imponente figura di fuoco e carne, un incubo cornuto armato di frusta e spada fiammeggianti, che vola nella notte in cerca delle sue prede. Il demone che queste persone temono è il Balor, e questa paura è pienamente giustificata, dal momento che pochi demoni possono eguagliare il possente Balor in forza o in brutalità.

Nell'Abisso, i Balor sono per lo più al servizio dei signori dei demoni, in qualità di generali o capitani (quando non si tratti di balor estremamente potenti, noti come signori dei balor). Un balor solitamente comanda vaste legioni di demoni e, sebbene spesso consenta a questi servi bramosi e sbavanti di combattere le sue battaglie, è tutt'altro che un codardo. Se si presenta l'opportunità di unirsi ad uno scontro, sono pochi i balor che scelgono di trattenersi.

Un Balor è alto 4,2 metri e pesa 2.250 kg. Solo le anime mortali più crudeli possono alimentare la creazione di un balor: a differenza degli altri demoni, spesso occorrono numerose anime di potenti malvagi per far nascere un nuovo balor.

\mostro{Demogorgone}
\noindent
\begin{description}[noitemsep, topsep=0pt, parsep=0pt, partopsep=0pt, leftmargin=0cm, labelwidth=2.2cm]
	\item[\textbf{Taglia/Tipo:}] Enorme principe demone, malvagio
	\item[\textbf{Caratt.:}] \resizebox{0.5\linewidth+1.8cm}{!}{For 9 Des 2 Cos 8 Int 5 Sag 3 Car 7}
	\item[\textbf{Punti Ferita:}] 524,  \textbf{Difesa:} 48,  \textbf{Iniziativa:} +5
	\item[\textbf{Movimento:}] 15 metri, nuotare 9m
	\item[\textbf{Tiri Salvez.:}] \resizebox{0.5\linewidth+1.8cm}{!}{\resizebox{0.5\linewidth+1.8cm}{!}{Tempra +34, Riflessi +28, Volontà +29}}
	\item[\textbf{Comp.:}] tutte +15
	\item[\textbf{Res. Danni:}] Freddo, Elettricità, Fuoco
	\item[\textbf{Imm. Danni:}] da Vuoto, Veleno; armi +2
	\item[\textbf{Immunità:}] affascinato, paralizzato, affaticato, spaventato
	\item[\textbf{Vulnerabilità:}] ferro freddo, Luce
	\item[\textbf{Sensi:}] Visione del vero 40 m
	\item[\textbf{Linguaggi:}] tutti, telepatia 45 m
	\item[\textbf{Sfida:}] 26 (90000 PX)\smallskip
\end{description}

\emph{\textbf{Incantesimi.}} Il Demogorgone ha CM 20. La sua caratteristica da incantatore è la Forza. Il Demogorgon conosce i seguenti incantesimi:

A volontà: \hyperlink{Individuazione del Magico}{Individuazione del Magico}, \hyperlink{Immagine Maggiore}{Immagine Maggiore}

livello 3 (4 slot): \emph{\hyperlink{Dissolvi Magie}{Dissolvi Magie}, \hyperlink{Paura}{Paura}, \hyperlink{Telecinesi}{Telecinesi}}

livello 4 (1 slot): \emph{\hyperlink{Immagine Proiettata}{Immagine Proiettata}, \hyperlink{Regressione Mentale}{Regressione Mentale}}

\emph{\textbf{Natura Demoniaca.}} Il Demogorgone non ha bisogno di aria, cibo, bevande o sonno.

\emph{\textbf{Resistenza Leggendaria (3/Giorno).}} Se il Demogorgone fallisce un Tiro Salvezza, può scegliere invece di riuscirvi.

\emph{\textbf{Resistenza allo Scacciare.}} Il Demogorgone ha +1d6 ai Tiri Salvezza contro gli effetti che scacciano i non morti.

\emph{\textbf{Due teste.}} Demogorgone ha +1d6 ai Tiri Salvezza contro essere cieco, sordo, svenuto

\textbf{Azioni}

\emph{\textbf{Multiattacco.} 2 attacchi con tentacolo}: +19, portata 3 metri, una creatura. Tutti gli attacchi di Demogorgone sono considerati magici +2.

\emph{Colpisce:} 35 (4d12 +9) danni contundenti. La creatura colpita deve fare un Tiro Salvezza su Tempra a DC 33 od i suoi Punti Ferita massimi scendono dello stesso ammontare.

\textbf{Reazione: \emph{Attacco d'opportunità}}: il Demogorgone effettua un attacco ad una creatura che attraversi o esca dalla sua portata di 3 metri.

\emph{\textbf{Sguardo}} Demogorgone fissa una creatura che può vedere entro 40 metri. Il bersaglio deve fare un Tiro Salvezza su Volontà a DC 33.

\emph{Effetto Sguardo:} Demogorgone sceglie uno di questi effetti oppure è a caso:

1. Sguardo Potente. Il bersaglio è svenuto fino al prossimo round o finché il Demogorgon è fuori dalla linea di vista

2. Sguardo Ipnotico. Il bersaglio è in dominato dal Demogorgone che ne stabilisce ogni azione. Questo sguardo necessità dell'utilizzo di entrambe le teste del Demogorgon.

3. Sguardo della Follia. Il bersaglio è sotto l'influenza dell'incantesimo Confusione che permane, senza Tiro Salvezza ulteriore, finché Demogorgone è in area di vista. Il Demogorgone non deve rimanere concentrato per il perdurare dell'effetto.


\textbf{Azioni Aggiuntive}

Il Demogorgone può effettuare 3 azioni aggiuntive, scelte da quelle sottostanti ed una per round solo al termine del round di un altra creatura.

\textbf{Coda.} Il Demogorgone attacca con la coda. +19 to al colpire, portata 5 metri, un obiettivo. Se colpisce 31 Punti Ferita di danni contundenti più 4d6 danni da Vuoto

\textbf{Sguardo di Follia.} Demogorgone usa o lo sguardo Potente o lo Sguardo della Follia

\textbf{Ecologia}\\
Ambiente: Abisso\\
Organizzazione: Unico\\
\textbf{Tesoro}: R, S, T, V\\
\textbf{Descrizione}\\
Demogorgone è un enorme demone, principe dell'abisso e della follia alto circa 5 metri. Appare come un rettiloide bipede con due teste da babbuino, i colli sono lunghi e serpentini come le braccia tentacolari. Le due teste di Demogorgone sono hanno personalità distinte che si detestano. Spesso tentano di dominarsi a vicenda e molte delle storie che riguardano il Demogorgone trattano proprio su come una o l'altra testa cechi di dominare il tutto. Tra il Demogorgone ed Orcus c'è una forte rivalità.


\begin{center}
	%\includegraphics[width=1\linewidth]{immagini/banner.png}
	\includegraphics[width=0.9\linewidth]{immagini/ercole-cerbero_grayscale.png}

	\emph{Hercules and Cerberus: Hercules grasps the collar of Cerberus. Antonio Tempesta}
\end{center}

\mostro{Dretch}
\noindent
\begin{description}[noitemsep, topsep=0pt, parsep=0pt, partopsep=0pt, leftmargin=0cm, labelwidth=2.2cm]
	\item[\textbf{Taglia/Tipo:}] Piccolo demone, malvagio
	\item[\textbf{Caratt.:}] \resizebox{0.5\linewidth+1.8cm}{!}{For 0 Des 0 Cos 1 Int -3 Sag -1 Car -4}
	\item[\textbf{Punti Ferita:}] 19,  \textbf{Difesa:} 12,  \textbf{Iniziativa:} +0
	\item[\textbf{Movimento:}] 6 m
	\item[\textbf{Tiri Salvez.:}] \resizebox{0.5\linewidth+1.8cm}{!}{Tempra +3, Riflessi +3, Volontà +3}
	\item[\textbf{Res. Danni:}] Freddo, Elettricità, Fuoco
	\item[\textbf{Imm. Danni:}] Veleno
	\item[\textbf{Vulnerabilità:}] ferro freddo, Luce
	\item[\textbf{Sensi:}] Scurovisione 18 m
	\item[\textbf{Linguaggi:}] Abissale, telepatia 18 m (funziona solo con le creature che comprendono l'Abissale)
	\item[\textbf{Sfida:}] 1/4 (50 PX)\smallskip
\end{description}

\textbf{Azioni}

\emph{\textbf{Multiattacco.}} Il demone effettua due attacchi: uno con il morso e uno con gli artigli.

\emph{\textbf{Artigli.} Attacco con arma da mischia}: +3 a colpire, portata 1 m, un bersaglio.

\emph{Colpisce:} 5 (2d4) danni taglienti.

\emph{\textbf{Morso.} Attacco con arma da mischia}: +3 a colpire, portata 1 m, un bersaglio.

\emph{Colpisce:} 3 (1d6) danni perforanti.

\textbf{Reazione: \emph{Anatomia opportunistica}} il dretch riorganizza la propria anatomia demoniaca dimezzando fine alla fine del round ogni danni critico subito.

\emph{\textbf{Nube Fetida (1/Giorno).}} Un disgustoso gas verde si estende in un raggio di 3 metri dal demone. Il gas si propaga intorno agli angoli e la sua area è oscurata leggermente. Rimane per 1 minuto o finché non viene disperso da un forte vento. Qualsiasi creatura che inizi il proprio round in quell'area deve riuscire un Tiro Salvezza di Tempra DC 11 o restare avvelenata fino all'inizio del suo prossimo round. Mentre è avvelenato in questo modo, il bersaglio, durante il suo round, è Rallentato 1.

\textbf{Ecologia}\\
Ambiente: Qualsiasi (Abisso)\\
Organizzazione: Solitario, coppia, banda (3-5), gruppo (6-12) o folla (13+)\\
\textbf{Categoria Tesoro}: Nessuno\\
\textbf{Descrizione}\\
Anche il più infimo demone dell'Abisso è pericoloso e possiede la necessità impellente di spargere rovina e sgomento. Il miserabile dretch è tanto orripilante e fetido quanto crudele, anche se non possiede la forza ed il potere per riuscire a soddisfare la sua voglia di brutalizzare gli altri nel suo reame nativo. Lo scopo dell'esistenza dei dretch è quello di servire demoni più potenti come vittime sacrificabili, e solo pochi fortunati riescono a sopravvivere abbastanza a lungo da evolversi.

I dretch sono i bersagli preferiti dai dilettanti in evocazioni abissali. Relativamente deboli e facili da intimorire, i dretch spesso possono essere obbligati a lunghi periodi di servitù utilizzando vaghe promesse di opportunità di sfogare le loro frustrazioni e la loro rabbia contro avversari più deboli. Eppure il potenziale evocatore di dretch farebbe meglio a ricordarsi che questi demoni sono codardi ed infidi quanto gli altri demoni. Un dretch che si trova di fronte un nemico più potente sarà assai lieto di scambiare qualsiasi informazione di cui disponga in cambio della sua miserevole vita.

A differenza della maggior parte dei demoni, la sciatta personalità del dretch ed il suo disprezzo per il lavoro fisico prolungato raramente danno dei risultati. I dretch avanzati sono rari, ma quelli che riescono a trovare la forza in se stessi per diventare più di quello che erano al momento della loro creazione divengono i sovrani poveri dell'Abisso, crudeli ed amareggiati, che regnano su parassiti, anime spezzate, non morti privi di intelletto e altri dretch. I loro imperi sono limitati a tratti abbandonati di fogne sotto città dimenticate, instabili distese paludose evitate dalle menti più sensate ed altri sgraditi angoli dell'Abisso che persino i demoni considerano scomodi o ripugnanti. Eppure per i signori dei dretch questi regni sono i loro imperi, e li difendono con pietosa tenacia.

Un dretch è alto 1,2 metri e pesa 90 kg. I dretch solitamente si formano dalle anime di mortali malvagi ed indolenti: è sufficiente solo un piccolo frammento di anima per dare origine ad una nascita così orripilante. Una sola anima spesso può causare l'apparizione di una piccola armata di dretch e la vista di un'orda di dretch appena nati che si liberano dalla protomateria pulsante dell'Abisso è al contempo nauseante e terrificante.


\begin{center}
	\includegraphics[width=0.9\linewidth]{immagini/Demone_Alato.png}

	\emph{Tomba dei demoni alati}
\end{center}

\mostro{Glabrezu}
\noindent
\begin{description}[noitemsep, topsep=0pt, parsep=0pt, partopsep=0pt, leftmargin=0cm, labelwidth=2.2cm]
	\item[\textbf{Taglia/Tipo:}] Grande demone, malvagio
	\item[\textbf{Caratt.:}] \resizebox{0.5\linewidth+1.8cm}{!}{For 5 Des 2 Cos 5 Int 4 Sag 3 Car 3}
	\item[\textbf{Punti Ferita:}] 186,  \textbf{Difesa:} 26,  \textbf{Iniziativa:} +4
	\item[\textbf{Movimento:}] 12 m
	\item[\textbf{Tiri Salvez.:}] \resizebox{0.5\linewidth+1.8cm}{!}{\resizebox{0.5\linewidth+1.8cm}{!}{Tempra +14, Riflessi +11, Volontà +12}}
	\item[\textbf{Res. Danni:}] Freddo, Elettricità, Fuoco; da arma non magica
	\item[\textbf{Imm. Danni:}] Veleno
	\item[\textbf{Vulnerabilità:}] ferro freddo, Luce
	\item[\textbf{Sensi:}] visione del vero 36 m
	\item[\textbf{Linguaggi:}] Abissale, telepatia 36 m
	\item[\textbf{Sfida:}] 9 (5000 PX)\smallskip
\end{description}

\emph{\textbf{Incantesimi Innati.}} La caratteristica da incantatore del demone è l'Intelligenza. Il demone può lanciare questi incantesimi in maniera innata, senza bisogno di componenti materiali:

A volontà: \emph{\hyperlink{Dissolvi Magie}{Dissolvi Magie}, \hyperlink{Individuazione del Magico}{Individuazione del Magico}, \hyperlink{Oscurità}{Oscurità}}

1/giorno ciascuno: \emph{\hyperlink{Confusione}{Confusione}, \hyperlink{Parola del Potere Stordire}{Parola del Potere Stordire}, \hyperlink{Volare}{Volare}}

\emph{\textbf{Resistenza alla Magia.}} Il demone ha +1d6 ai Tiri Salvezza contro incantesimi e altri effetti magici.

\textbf{Azioni}

\emph{\textbf{Multiattacco.}} Il demone effettua quattro attacchi: due con le chele e due con i pugni. In alternativa, può effettuare due attacchi con le chele e lanciare un incantesimo.

\emph{\textbf{Chela.} Attacco con arma da mischia}: +9 a colpire, portata 3 m, un bersaglio.

\emph{Colpisce:} 16 (2d10 + 5) danni contundenti. Se il bersaglio è una creatura di taglia Media o inferiore, è afferrato (DC 15 per fuggire). Il glabrezu possiede due chele, ciascuna delle quali può afferrare un bersaglio.

\emph{\textbf{Pugno.} Attacco in mischia con arma}: +9 a colpire, portata 1 m, un bersaglio.

\emph{Colpisce:} 7 (2d4 + 2) danni contundenti.

\textbf{Reazione: \emph{Attacco d'opportunità}}: il demone effettua un attacco ad una creatura che attraversi o esca dalla sua portata di 3 metri.

\emph{\textbf{Arrabbiato:}} il glabrezu crea un duplicato di se stesso dal piano delle ombre. Questo duplicato ha le stesse caratteristiche del glabrezu ma non attacca. Quando si attacca il glabrezu si ha un 50\% di attaccare il duplicato d'ombra.

\textbf{Ecologia}\\
Ambiente: Qualsiasi (Abisso)\\
Organizzazione: Solitario o drappello (1 glabrezu, 1 Succube e 2-5 Vrock)
\textbf{Categoria Tesoro}: U\\
\textbf{Descrizione}\\
Mentre la Succube è un demone che adesca la sua preda sfruttandone i desideri e le necessità carnali, il glabrezu è un tentatore di altro genere. Feroce e dalla forma bestiale, il glabrezu è in realtà un maestro di inganni e bugie. Con la sua abilità di nascondere la sua vera forma dietro piacenti illusioni, usa la sua magia per esaudire i desideri degli umanoidi mortali, come forma di ricompensa per coloro che soccombono ai suoi inganni e raggiri. Un desiderio esaudito da un glabrezu appaga la necessità di chi lo esprime nel modo più rovinoso possibile, sebbene queste conseguenze possano non rivelarsi immediatamente tali. Un fabbro che fatica ad affermarsi potrebbe desiderare fama ed abilità nella professione scelta, solo per scoprire che il suo miglior patrono è un crudele e sadico omicida che usa le armi per promuovere i propri distruttivi desideri. Un uomo solo che esprime il desiderio di avere una compagna, potrebbe vedere il suo desiderio avverarsi con una sua vecchia fiamma ritornata alla vita in forma di vampiro, ed altri esempi di questo tipo. Il glabrezu è assai creativo nel soddisfare i desideri di un mortale.

Un glabrezu è alto 5,3 metri e pesa poco più di 3000 kg. Questi perfidi demoni si originano dalle anime dei traditori, dei falsi e dei sovversivi: anime di mortali che, in vita, giurarono il falso o utilizzarono il tradimento e l'inganno per rovinare le vite altrui.

\mostro{Hezrou}
\noindent
\begin{description}[noitemsep, topsep=0pt, parsep=0pt, partopsep=0pt, leftmargin=0cm, labelwidth=2.2cm]
	\item[\textbf{Taglia/Tipo:}] Grande demone, malvagio
	\item[\textbf{Caratt.:}] \resizebox{0.5\linewidth+1.8cm}{!}{For 4 Des 3 Cos 5 Int 5 Sag 1 Car 1}
	\item[\textbf{Punti Ferita:}] 167,  \textbf{Difesa:} 25,  \textbf{Iniziativa:} +5
	\item[\textbf{Movimento:}] 9 m
	\item[\textbf{Tiri Salvez.:}] \resizebox{0.5\linewidth+1.8cm}{!}{Tempra +13, Riflessi +11, Volontà +9}
	\item[\textbf{Res. Danni:}] Freddo, Elettricità, Fuoco; da arma non magica
	\item[\textbf{Imm. Danni:}] Veleno
	\item[\textbf{Vulnerabilità:}] ferro freddo, Luce
	\item[\textbf{Sensi:}] Scurovisione 36 m
	\item[\textbf{Linguaggi:}] Abissale, telepatia 36 m
	\item[\textbf{Sfida:}] 8 (3900 PX)\smallskip
\end{description}

\emph{\textbf{Fetore.}} Qualsiasi creatura che inizi il suo round entro 3 metri dal demone, deve riuscire un Tiro Salvezza di Tempra DC 21 o restare avvelenata, -1 Forza e Destrezza, fino all'inizio del proprio round. Se riesce il Tiro Salvezza, la creatura è immune al fetore del demone per 24 ore.

\emph{\textbf{Resistenza alla Magia.}} Il demone ha +1d6 ai Tiri Salvezza contro incantesimi e altri effetti magici.

\textbf{Azioni}

\emph{\textbf{Multiattacco.}} Il demone effettua tre attacchi: uno con il morso e due con gli artigli.

\emph{\textbf{Artiglio.} Attacco con arma da mischia}: +8 a colpire, portata 1 m, un bersaglio.

\emph{Colpisce:} 11 (2d6 + 4) danni taglienti, 2 danni da Sanguinamento.

\emph{\textbf{Morso.} Attacco con arma da mischia}: +8 a colpire, portata 1 m, un bersaglio.

\emph{Colpisce:} 15 (2d10 + 4) danni perforanti e malattia Febbre Demoniaca minore.

\emph{Febbre Demoniaca minore}: 1 minuto, TS Tempra DC 18, 6 ore, 3 successi, -1 Costituzione e Saggezza.

\textbf{Reazione: \emph{Pustola esplosiva}} il demone quando colpito da un critico fa esplodere una pustola fetida che infligge alla creatura che ha portato il critico, entro 2 metri, 2d8 danni da acido.

\emph{\textbf{Arrabbiato:}} Hezrou rilascia una nube di fetore incendiaria. Tutte le creature attorno a lui entro 3 metri devono fare un Tiro Salvezza su Riflessi DC 21 per dimezzare i 4d10 di danno da fuoco. Costa 2 Azioni.

\textbf{Ecologia}\\
Ambiente: Qualsiasi (Abisso)\\
Organizzazione: Solitario o banda (2-4)\\
\textbf{Categoria Tesoro}: W\\
\textbf{Descrizione}\\
L'hezrou vive nelle vaste paludi, acquitrini e corsi d'acqua dell'Abisso. E' a suo agio sia nell'acqua che sulla terraferma. La presenza di un hezrou ha un effetto dannoso su flora, causando mutazioni e rendendo maleodoranti e dal sapore salmastro le acque. Spesso intere comunità isolate di mutanti deformi devono il loro aspetto contorto non tanto ai loro depravati costumi quanto alla vicinanza di un hezrou.

Queste mostruose e bestiali creature nascono dalle anime di mortali malvagi che hanno avvelenato se stessi, i loro parenti o il loro ambiente, ad esempio, drogati, assassini ed alchimisti che non si sono preoccupati di come i loro esperimenti avvelenassero il mondo naturale.

\mostro{Marilith}
\noindent
\begin{description}[noitemsep, topsep=0pt, parsep=0pt, partopsep=0pt, leftmargin=0cm, labelwidth=2.2cm]
	\item[\textbf{Taglia/Tipo:}] Grande demone, malvagio
	\item[\textbf{Caratt.:}] \resizebox{0.5\linewidth+1.8cm}{!}{For 4 Des 5 Cos 5 Int 4 Sag 3 Car 5}
	\item[\textbf{Punti Ferita:}] 319,  \textbf{Difesa:} 38,  \textbf{Iniziativa:} +5
	\item[\textbf{Movimento:}] 12 m
	\item[\textbf{Tiri Salvez.:}] \resizebox{0.5\linewidth+1.8cm}{!}{\resizebox{0.5\linewidth+1.8cm}{!}{Tempra +21, Riflessi +21, Volontà +19}}
	\item[\textbf{Res. Danni:}] Freddo, Elettricità, Fuoco
	\item[\textbf{Imm. Danni:}] Veleno, armi +1
	\item[\textbf{Vulnerabilità:}] ferro freddo, Luce
	\item[\textbf{Sensi:}] visione del vero 36 m
	\item[\textbf{Linguaggi:}] Abissale, telepatia 36 m
	\item[\textbf{Sfida:}] 16 (15000 PX)\smallskip
\end{description}

\emph{\textbf{Armi Magiche.}} Gli attacchi con armi del demone sono magici.

\emph{\textbf{Reattivo.}} il marlith può effettuare una Reazione di Parata durante ciscuno round.

\emph{\textbf{Resistenza alla Magia.}} Il demone ha +1d6 ai Tiri Salvezza contro incantesimi e altri effetti magici.

\textbf{Azioni}

\emph{\textbf{Multiattacco.}} Il demone effettua sette attacchi: sei con le spade lunghe e uno con la coda.

\emph{\textbf{Coda.} Attacco con arma da mischia}: +13 a colpire, portata 3 m, una creatura.

\emph{Colpisce:} 15 (2d10 + 4) danni contundenti. Se il bersaglio è di taglia Media o inferiore, è afferrato (DC 19 per fuggire). Fino al termine dell'afferrare il demone può colpire automaticamente il bersaglio con la coda, ma non può effettuare attacchi di coda contro altri bersagli.

\emph{\textbf{Spada Lunga.} Attacco con arma da mischia}: +13 a colpire, portata 1 m, un bersaglio.

\emph{Colpisce:} 13 (2d8 + 4) danni taglienti.

\textbf{Reazione: \emph{Parata.}} Il demone somma 5 alla sua Difesa contro un attacco da mischia che lo colpirebbe. Per farlo il demone deve poter vedere il suo attaccante e impugnare un'arma da mischia.

\textbf{Reazione: \emph{Attacco d'opportunità}}: il demone effettua un attacco ad una creatura che attraversi o esca dalla sua portata di 3 metri.

\emph{\textbf{Arrabbiato:}}

- la marilith affila le sue spade tra loro, ogni attacco con la spada lunga guadagna Sanguinamento 1/20. Costa 2 Azioni, dura fino alla fine del combattimento.

- la marilith condanna l'avversario all'abisso. Costo 2 Azioni. L'avversario deve effettuare un Tiro Salvezza su Volontà DC 28 o essere trasportato nell'abisso.

\textbf{Ecologia}\\
Ambiente: Qualsiasi (Abisso)\\
Organizzazione: Solitario, coppia o plotone (1 marilith, 1-3 Glabrezu e 3-14 Babau)\\
\textbf{Categoria Tesoro}: C\\
\textbf{Descrizione}\\
Sovrane di orde demoniache e regine di nazioni abissali, le temibili marilith servono i signori dei demoni come governanti, consigliere e persino amanti, eppure la loro supremazia come strateghe le rende particolarmente richieste come generali e comandanti d'armate. Le marilith più potenti non sono al servizio di nessuno e comandano invece fameliche legioni demoniache.

Una marilith è alta da 1,8 a 2,7 metri, lunga 6 metri dalla testa alla punta della coda, e pesa 2000 kg. Solo le anime malvagie più arroganti ed orgogliose, solitamente quelle di crudeli sovrani, sadici generali e signori della guerra particolarmente violenti, possono causare la nascita di una marilith.

\mostro{Nalfeshnee}
\noindent
\begin{description}[noitemsep, topsep=0pt, parsep=0pt, partopsep=0pt, leftmargin=0cm, labelwidth=2.2cm]
	\item[\textbf{Taglia/Tipo:}] Grande demone, malvagio
	\item[\textbf{Caratt.:}] \resizebox{0.5\linewidth+1.8cm}{!}{For 5 Des 0 Cos 6 Int 4 Sag 1 Car 2}
	\item[\textbf{Punti Ferita:}] 264,  \textbf{Difesa:} 29,  \textbf{Iniziativa:} +4
	\item[\textbf{Movimento:}] 6 m, volo 9 m
	\item[\textbf{Tiri Salvez.:}] \resizebox{0.5\linewidth+1.8cm}{!}{\resizebox{0.5\linewidth+1.8cm}{!}{Tempra +19, Riflessi +13, Volontà +14}}
	\item[\textbf{Res. Danni:}] Freddo, Elettricità, Fuoco; da arma non magica
	\item[\textbf{Imm. Danni:}] Veleno
	\item[\textbf{Vulnerabilità:}] ferro freddo, Luce
	\item[\textbf{Sensi:}] Scurovisione 36 m
	\item[\textbf{Linguaggi:}] Abissale, telepatia 36 m
	\item[\textbf{Sfida:}] 13 (10000 PX)\smallskip
\end{description}

\emph{\textbf{Resistenza alla Magia.}} Il demone ha +1d6 ai Tiri Salvezza contro incantesimi e altri effetti magici.

\textbf{Azioni}

\emph{\textbf{Multiattacco.}} Il demone usa, se possibile, Aureola di Orrore. Poi effettua tre attacchi: uno con il morso e due con gli artigli.

\emph{\textbf{Artiglio.} Attacco con arma da mischia}: +12 a colpire, portata 3 m, un bersaglio.

\emph{Colpisce:} 15 (3d6 + 5) danni taglienti, 2 danni da Sanguinamento.

\emph{\textbf{Morso.} Attacco con arma da mischia}: +12 a colpire, portata 1 m, un bersaglio.

\emph{Colpisce:} 32 (5d10 + 5) danni perforanti e Febbre Demoniaca.

\emph{Febbre Demoniaca}: 1 minuto, TS Tempra DC 23, 4 ore, 3 successi, -1 Costituzione e Saggezza/4 ore.

\emph{\textbf{Aureola di Orrore (Ricarica 5-6).}} Il demone emette una luce magica multicolore e scintillante. Ogni creatura entro 5 metri dal demone e che possa vedere la luce deve riuscire un Tiro Salvezza su Volontà DC 25 o restare spaventata per 1 minuto.

Una creatura può ripetere il Tiro Salvezza al termine di ciascun suo round, terminando l'effetto per sé se lo riesce.

Se il Tiro Salvezza della creatura riesce o l'effetto ha termine per essa la creatura è immune all'Aureola di Orrore del demone per le successive 24 ore.

\emph{\textbf{Teletrasporto.}} Il demone si teletrasporta, insieme a tutto l'equipaggiamento che sta indossando o trasportando, in uno spazio non occupato che possa vedere fino a 36 metri di distanza. E' una Azione di Movimento.

\textbf{Reazione: \emph{Attacco d'opportunità}}: il demone effettua un attacco ad una creatura che attraversi o esca dalla sua portata di 1 metro.

\emph{\textbf{Arrabbiato:}} il nalfeshnee mima le parole arcane ed in gesti visti entro 3 round precedenti e lancia un incantesimo di cui è stato testimone. Costa 3 Azioni.

\textbf{Ecologia}
Ambiente: Qualsiasi (Abisso)\\
Organizzazione: Solitario o banda di guerra (1 nalfeshnee, 1 Hezrou e 2-5 Vrock)\\
\textbf{Categoria Tesoro}: G\\
\textbf{Descrizione}\\
Sono pochi i demoni che comprendono le meccaniche interne che regolano l'Abisso come i nalfeshnee, e non è raro che questi demoni servano l'Abisso stesso invece che un signore dei demoni. Alcuni sovrintendono i reami organici che generano i nuovi demoni, mentre altri custodiscono luoghi di particolare importanza nei recessi nascosti del piano. Spesso il regno di un nalfeshnee nell'Abisso è superiore per forze e dimensioni al più grande dei regni mortali, in quanto questi demoni hanno una predisposizione naturale a governare ed imporre una sorta di ordine al caos dell'Abisso. Gli evocatori mortali spesso li richiamano per il loro folle ma impareggiabile intelletto, esaminando accuratamente gli accordi presi con questi demoni onde evitare eventuali conseguenze nascoste e risvolti non voluti, in quanto un nalfeshnee raramente accetta qualcosa che, in qualche modo contorto, non gli consenta di soddisfare le necessità ed i desideri dell'Abisso.

I nalfeshnee sono alti 6 metri e pesano 4000 kg. Sono creati dalle anime di malvagi mortali avari o bramosi, in particolare di coloro che hanno regnato su imperi di schiavitù, furto, brigantaggio e altri vizi ancora più violenti.

\mostro{Orcus}
\noindent
\begin{description}[noitemsep, topsep=0pt, parsep=0pt, partopsep=0pt, leftmargin=0cm, labelwidth=2.2cm]
	\item[\textbf{Taglia/Tipo:}] Enorme principe demone, malvagio
	\item[\textbf{Caratt.:}] \resizebox{0.5\linewidth+1.8cm}{!}{For 8 Des 2 Cos 7 Int 5 Sag 5 Car 7}
	\item[\textbf{Punti Ferita:}] 519,  \textbf{Difesa:} 48,  \textbf{Iniziativa:} +5
	\item[\textbf{Movimento:}] 15 metri, volare 15 metri
	\item[\textbf{Tiri Salvez.:}] \resizebox{0.5\linewidth+1.8cm}{!}{\resizebox{0.5\linewidth+1.8cm}{!}{Tempra +33, Riflessi +28, Volontà +31}}
	\item[\textbf{Comp.:}] tutte +13
	\item[\textbf{Res. Danni:}] Freddo, Elettricità, Fuoco
	\item[\textbf{Imm. Danni:}] da Vuoto, Veleno; armi +2
	\item[\textbf{Immunità:}] affascinato, paralizzato, affaticato, spaventato
	\item[\textbf{Vulnerabilità:}] Luce
	\item[\textbf{Sensi:}] Visione del vero 40 m
	\item[\textbf{Linguaggi:}] tutti, telepatia 45 m
	\item[\textbf{Sfida:}] 26 (90000 PX)\smallskip
\end{description}

\emph{\textbf{Incantesimi.}} Orcus ha CM 17, DC 30. La sua caratteristica da incantatore è il Carisma. Orcus conosce i seguenti incantesimi:

A volontà: \hyperlink{Individuazione del Magico}{Individuazione del Magico}, \hyperlink{Tocco Gelido}{Tocco Gelido}

livello 3 (3 slot): \emph{\hyperlink{Dissolvi Magie}{Dissolvi Magie}}

livello 6 (3 slot): \emph{\hyperlink{Creare Non Morti}{Creare Non Morti}}

livello 9 (1 slot): \emph{\hyperlink{Fermare il Tempo}{Fermare il Tempo}}

\emph{\textbf{Natura Demoniaca.}} Orcus non ha bisogno di aria, cibo, bevande o sonno.

\emph{\textbf{Resistenza Leggendaria (3/Giorno).}} Se il Orcus fallisce un Tiro Salvezza, può scegliere invece di riuscirvi.

\emph{\textbf{Signore dei non morti.}} Orcus può sempre decidere il tipo di non morto che crea e questo rimane sotto il suo controllo per tempo indefinito, oltretutto può lanciare l'incantesimo in qualsiasi condizione si trovi.

\textbf{Azioni}

\emph{\textbf{Multiattacco.} 2 attacchi con bacchetta}: +19, portata 3 metri, una creatura. Tutti gli attacchi di Orcus sono considerati magici +3.

\emph{Colpisce:} 21 (3d8 + 8) danni contundenti + 13 (2d12) da Vuoto

\emph{\textbf{Coda}} Orcus colpisce con la sua coda. +19, portata 3 metri, una creatura

\emph{Colpisce:} 21 (3d8 + 8) danni contundenti + 18 (4d8) da Veleno

\textbf{Reazione: \emph{Attacco d'opportunità}}: il demone effettua un attacco ad una creatura che attraversi o esca dalla sua portata di 3 metri.

\textbf{Azioni Aggiuntive}

Il Orcus può effettuare 3 azioni aggiuntive, scelte da quelle sottostanti ed una per round solo al termine del round di un altra creatura.

\textbf{Coda.} Il Orcus attacca con la coda. +19 to al colpire, portata 5 metri, un obiettivo. Se colpisce 21 (3d8 + 8) danni contundenti + 18 (4d8) da Veleno

\textbf{Assaggio di Morte.} Orcus lancia l'incantesimo \hyperlink{Colpo Infuocato}{Colpo Infuocato}, in maniera blasfema, con danni da Vuoto

\textbf{Ecologia}\\
Ambiente: Abisso\\
Organizzazione: Unico\\
\textbf{Categoria Tesoro}: Z\\
\textbf{Descrizione}\\
Orcus è il Principe Demone dei non morti. Predilige la compagnia e servizio dei non morti. Desidera vedere scomparire tutta la vita e questa trasformarsi tutta in una gigantesca necropoli di non morti sotto il suo comando. Orcus ha la testa e le gambe da capra, corna simili a montoni, un corpo gonfio, ali da pipistrello e una lunga coda.

\mostro{Silku}
\noindent
\begin{description}[noitemsep, topsep=0pt, parsep=0pt, partopsep=0pt, leftmargin=0cm, labelwidth=2.2cm]
	\item[\textbf{Taglia/Tipo:}] Media demone, malvagio
	\item[\textbf{Caratt.:}] \resizebox{0.5\linewidth+1.8cm}{!}{For 2 Des 2 Cos 3 Int 1 Sag 0 Car 2}
	\item[\textbf{Punti Ferita:}] 52,  \textbf{Difesa:} 16,  \textbf{Iniziativa:} +2
	\item[\textbf{Movimento:}] 9 m
	\item[\textbf{Tiri Salvez.:}] \resizebox{0.5\linewidth+1.8cm}{!}{Tempra +5, Riflessi +4, Volontà +3}
	\item[\textbf{Comp.:}] Furtività +2, Ingannare +5
	\item[\textbf{Res. Danni:}] Freddo, Elettricità; da arma non magica o non argentata
	\item[\textbf{Imm. Danni:}] Veleno
	\item[\textbf{Vulnerabilità:}] ferro freddo, Luce
	\item[\textbf{Sensi:}] Scurovisione 36 m
	\item[\textbf{Linguaggi:}] Abissale, Comune
	\item[\textbf{Sfida:}] 2 (450 PX)\smallskip
\end{description}

\emph{\textbf{Resistenza alla Magia.}} Il demone ha +1d6 ai Tiri Salvezza contro incantesimi e altri effetti magici.

\textbf{Azioni}

\emph{\textbf{Artigli.} Attacco con arma da mischia}: +4 a colpire, portata 1 m, un bersaglio.

\emph{Colpisce:} 6 (1d6 + 3) danni taglienti. Se il bersaglio è una creatura, deve riuscire un Tiro Salvezza di Tempra DC 14 o subire 5 (2d4) danni da veleno

\emph{\textbf{Cambiare aspetto (a volontà).}} Il Silku può apparire come un umanoide di taglia media a suo piacimento. E' necessario una prova di Consapevolezza DC 16 per percepire il vero aspetto. 2 Azioni

\emph{\textbf{Rigenerazione.}} Il Silku rigenera 2 PF alla fine del suo round a meno che non sia stato ferito con acido o fuoco, l'effetto è attivo anche se il Silku ha Punti Ferita negativi.

\textbf{Ecologia}\\
Ambiente: Qualsiasi (Abisso)\\
Organizzazione: Piccoli gruppi (3-6)\\
\textbf{Categoria Tesoro}: P\\
\textbf{Descrizione}\\
"...I loro volti avevano qualcosa di strano, erano come ... sfocati, era l'unica parte del loro corpo che non riuscivo a mettere a fuoco. Stupita dalla stranezza, sbattei le palpebre più volte e concentrai lo sguardo sui volti di entrambi. Sentii uno strano pizzicore sul volto e poi la vista si schiarì.
Sgranai gli occhi e indietreggiai di un passo, mentre il terrore si impadroniva di me. I loro visi non erano umani. Avevano entrambi la pelle grigiastra e grinzosa, il naso schiacciato e lunghi canini che uscivano dalla bocca, grandi orecchie e occhi piccoli e neri. Sembrava il muso di un pipistrello."

Da \emph{Il Guardiano di Falkonia}, romanzo di Federica Angeli

\mostro{Quasit}
\noindent
\begin{description}[noitemsep, topsep=0pt, parsep=0pt, partopsep=0pt, leftmargin=0cm, labelwidth=2.2cm]
	\item[\textbf{Taglia/Tipo:}] Minuscola demone, mutaforma, malvagio
	\item[\textbf{Caratt.:}] \resizebox{0.5\linewidth+1.8cm}{!}{For -3 Des 3 Cos 0 Int -2 Sag 0 Car 0}
	\item[\textbf{Punti Ferita:}] 33,  \textbf{Difesa:} 16,  \textbf{Iniziativa:} +3
	\item[\textbf{Movimento:}] 12 m (3 m, volo 12 m in forma di pipistrello; 12 m, scalata 12 m in forma di centopiedi; 12 m, nuoto 12 m in forma di rospo)
	\item[\textbf{Tiri Salvez.:}] \resizebox{0.5\linewidth+1.8cm}{!}{Tempra +3, Riflessi +4, Volontà +3}
	\item[\textbf{Comp.:}] Furtività +5
	\item[\textbf{Res. Danni:}] Freddo, Elettricità, Fuoco; da arma non magica
	\item[\textbf{Imm. Danni:}] Veleno
	\item[\textbf{Vulnerabilità:}] ferro freddo, Luce
	\item[\textbf{Sensi:}] Scurovisione 36 m
	\item[\textbf{Linguaggi:}] Abissale, Comune
	\item[\textbf{Sfida:}] 1 (200 PX)\smallskip
\end{description}

\emph{\textbf{Mutaforma.}} Il demone può usare una Azione per trasformarsi in una forma bestiale da pipistrello, centopiedi o rospo, o per tornare alla sua vera forma. Le sue statistiche sono le stesse in tutte le forme, sebbene gli attacchi possano variare per alcune di esse. Qualsiasi equipaggiamento stia indossando o trasportando non viene trasformato. Alla morte ritorna alla sua vera forma.

\emph{\textbf{Resistenza alla Magia.}} Il demone ha +1d6 ai Tiri Salvezza contro incantesimi e altri effetti magici.

\textbf{Azioni}

\emph{\textbf{Artigli (Morso in Forma di Bestia).} Attacco con arma da mischia}: +4 a colpire, portata 1 m, un bersaglio.

\emph{Colpisce:} 5 (1d4 + 3) danni perforanti. Se il bersaglio è una creatura, deve riuscire un Tiro Salvezza di Tempra DC 12 o subire 5 (2d4) danni da veleno e restare avvelenato, -1 Forza e Destrezza, per 1 minuto. La creatura può ripetere il Tiro Salvezza al termine di ciascun suo round, ponendo termine all'effetto se lo riesce.

\emph{\textbf{Invisibilità.}} Il demone resta invisibile finché non attacca o termina la sua concentrazione. Qualsiasi cosa che il demone stia trasportando o indossando resta invisibile finché rimane in contatto con il demone.

\emph{\textbf{Spavento (1/Giorno).}} Una creatura scelta dal demone che si trovi entro 6 metri da lui, deve riuscire un Tiro Salvezza su Volontà DC 12 o restare spaventata per 1 minuto. Il bersaglio può ripetere il Tiro Salvezza al termine di ciascun suo round, con -1d6 se il demone è in linea di visuale, ponendo termine all'effetto prematuramente se riesce il Tiro Salvezza.

\textbf{Ecologia}\\
Ambiente: Qualsiasi (Abisso)\\
Organizzazione: Solitario o stormo (2-12)\\
\textbf{Categoria Tesoro}: Nessuno\\
\textbf{Descrizione}\\
Il quasit è forse il demone meno potente, ma non è tra i meno rispettati: persino i quasit si ritengono superiori alle orde di Dretch e, fedeli alla propria natura, i Dretch mancano del coraggio o degli stimoli necessari a dimostrare loro che si sbagliano. Il ruolo primario in vita di un quasit è quello di famiglio al servizio di un incantatore, ma quei quasit che sfuggono a questa umiliante servitù acquisiscono una volontà propria e sono molto più pericolosi. Un quasit tipico è alto 45 centimetri e pesa solo 4 kg.

Unici tra le orde demoniache, i quasit non nascono dalle anime di malvagi mortali deceduti, ma da anime viventi: quando un incantatore cerca di richiamare a sé un quasit come famiglio, la sua anima sfiora l'Abisso ed esso reagisce, creando dalla sua materia un quasit collegato all'anima dell'incantatore e generando un potente legame tra i due.

I quasit appena creati vengono alla luce direttamente nel Piano Materiale, dove diventano famigli e, finché sono soggetti alla volontà del loro padrone, lo odiano e disprezzano, dal momento che possono percepire il pulsare delle sua anima e sanno che potrebbero aspirare a qualcosa di più. Un quasit serve, eppure osserva e vigila nell'attesa di errori che possano costare la vita al suo signore, o meglio, che gli consentano di rivoltarsi contro il proprio padrone. Alla morte del proprio padrone il quasit spesso decide di rimanere nel Piano Materiale in cerca di altri modi per divertirsi, solitamente insediandosi in un'area urbana dove ci sono molti individui da tormentare.

\mostro{Succube}
\noindent
\begin{description}[noitemsep, topsep=0pt, parsep=0pt, partopsep=0pt, leftmargin=0cm, labelwidth=2.2cm]
	\item[\textbf{Taglia/Tipo:}] Media demone, malvagio
	\item[\textbf{Caratt.:}] \resizebox{0.5\linewidth+1.8cm}{!}{For -1 Des 3 Cos 1 Int 2 Sag 1 Car 5}
	\item[\textbf{Punti Ferita:}] 87,  \textbf{Difesa:} 20,  \textbf{Iniziativa:} +3
	\item[\textbf{Movimento:}] 9 m, volo 18 m
	\item[\textbf{Tiri Salvez.:}] \resizebox{0.5\linewidth+1.8cm}{!}{Tempra +5, Riflessi +7, Volontà +5}
	\item[\textbf{Comp.:}] Furtività +5, Percepire Emozioni +5, Consapevolezza +5, Ingannare +9
	\item[\textbf{Res. Danni:}] Freddo, Elettricità, Fuoco, Veleno; da arma non magica
	\item[\textbf{Sensi:}] Scurovisione 18 m
	\item[\textbf{Vulnerabilità:}] ferro freddo, Luce
	\item[\textbf{Linguaggi:}] Abissale, Comune, Infernale, telepatia 18 m
	\item[\textbf{Sfida:}] 4 (1100 PX)
\end{description}

\noindent\rule{\linewidth}{2pt}

\medskip

\emph{\textbf{Legame Telepatico.}} L'immondo ignora le restrizioni di raggio di azione della sua telepatia quando comunica con una creatura che ha affascinato. I due non sono neppure costretti a trovarsi sullo stesso piano di esistenza.

\emph{\textbf{Mutaforma.}} L'immondo può usare una Azione per trasformarsi in un umanoide di taglia Piccola o Media, o per tornare alla sua vera forma. Senza le ali, l'immondo perde la velocità di volo. A parte la taglia e la velocità, le sue statistiche sono le stesse in tutte le forme. Qualsiasi equipaggiamento stia indossando o trasportando non viene trasformato. Alla morte ritorna alla sua vera forma.

\textbf{Azioni}

\emph{\textbf{Artiglio (Solo Forma Immonda).} Attacco con arma da mischia}: +6 a colpire, portata 1 m, un bersaglio.

\emph{Colpisce:} 6 (1d6 + 3) danni taglienti.

\emph{\textbf{Affascinare.}} Un umanoide visibile all'immondo entro 9 metri da esso deve riuscire un Tiro Salvezza di Volontà DC 16 o restare magicamente affascinato per 1 giorno. Il bersaglio affascinato obbedisce ai comandi verbali o telepatici dell'immondo. Se il bersaglio subisce danni o riceve un comando suicida, può ripetere il Tiro Salvezza, terminando l'effetto se lo riesce. Se il bersaglio riesce il Tiro Salvezza contro l'effetto, o se l'effetto termina, il bersaglio è immune all'Affascinare dell'immondo per le successive 24 ore.

L'immondo può tenere affascinato solo un bersaglio alla volta. Se ne affascina un altro, l'effetto sul bersaglio precedente termina.

\emph{\textbf{Bacio Risucchiante.}} L'immondo bacia una creatura affascinata o una creatura consenziente. Il bersaglio deve effettuare un Tiro Salvezza di Tempra DC 16 contro questa magia, subendo 32 (5d10 + 5) danni se lo fallisce, o la metà di questi danni se lo riesce. L'immondo recupera metà dei Punti Ferita persi dalla creatura. I Punti Ferita massimi del bersaglio vengono ridotti di un ammontare pari ai danni subiti. Questa riduzione perdura finché non sorge l'alba. Il bersaglio muore se questo effetto riduce i suoi Punti Ferita massimi a 0.

\emph{\textbf{Forma Eterea.}} L'immondo entra magicamente nel Piano Etereo dal Piano Materiale, e viceversa.

\textbf{Ecologia}\\
Ambiente: Qualsiasi (Abisso)\\
Organizzazione: Solitario, coppia o harem (3-12)\\
\textbf{Categoria Tesoro}: I\\
\textbf{Descrizione}\\
Tra le orde demoniache una succube spesso può raggiungere altissimi livelli di potere, utilizzando le sue manipolazioni ed il suo fascino sensuale, e molte guerre demoniache imperversano a causa delle subdole macchinazioni di queste creature. Una succube si origina dalle anime di malvagi mortali particolarmente libidinosi ed avidi.

\mostro{Vrock}
\noindent
\begin{description}[noitemsep, topsep=0pt, parsep=0pt, partopsep=0pt, leftmargin=0cm, labelwidth=2.2cm]
	\item[\textbf{Taglia/Tipo:}] Grande demone, malvagio
	\item[\textbf{Caratt.:}] \resizebox{0.5\linewidth+1.8cm}{!}{For 3 Des 2 Cos 4 Int -1 Sag 1 Car -1}
	\item[\textbf{Punti Ferita:}] 127,  \textbf{Difesa:} 22,  \textbf{Iniziativa:} +2
	\item[\textbf{Movimento:}] 12 m, volo 18 m
	\item[\textbf{Tiri Salvez.:}] \resizebox{0.5\linewidth+1.8cm}{!}{Tempra +10, Riflessi +8, Volontà +7}
	\item[\textbf{Res. Danni:}] Freddo, Elettricità, Fuoco; da arma non magica
	\item[\textbf{Imm. Danni:}] Veleno
	\item[\textbf{Vulnerabilità:}] ferro freddo, Luce
	\item[\textbf{Sensi:}] Scurovisione 36 m
	\item[\textbf{Linguaggi:}] Abissale, telepatia 36 m
	\item[\textbf{Sfida:}] 6 (2300 PX)\smallskip
\end{description}

\emph{\textbf{Resistenza alla Magia.}} Il demone ha +1d6 ai Tiri Salvezza contro incantesimi e altri effetti magici.

\textbf{Azioni}

\emph{\textbf{Multiattacco.}} Il demone effettua due attacchi: uno con il becco e uno con gli speroni.

\emph{\textbf{Becco.} Attacco con arma da mischia}: +8 a colpire, portata 1 m, un bersaglio.

\emph{Colpisce:} 10 (2d6 + 3) danni perforanti.

\emph{\textbf{Speroni.} Attacco con arma da mischia}: +8 a colpire, portata 1 m, un bersaglio.

\emph{Colpisce:} 14 (2d10 + 3) danni taglienti.

\emph{\textbf{Spore (Ricarica 6).}} Una nube di spore tossiche si diffonde in un raggio di 5 metri intorno al demone. Ogni creatura in quell'area deve riuscire un Tiro Salvezza su Tempra DC 18 o restare avvelenata. Mentre avvelenato in questo modo, un bersaglio subisce 5 (1d10) danni da veleno all'inizio di ciascun suo round. Il bersaglio può ripetere il Tiro Salvezza al termine di ciascun suo round, ponendo termine all'effetto se lo riesce. Svuotare una fiala di Acqua santa sul bersaglio pone termine all'effetto.

\emph{\textbf{Strillo Stordente (1/Giorno).}} Il demone emette uno strillo orripilante. Ogni creatura entro 6 metri da esso e che lo possa udire, e non sia un demone, deve riuscire un Tiro Salvezza su Tempra DC 18 o restare stordita fino al termine del prossimo round del demone.

\textbf{Reazione: \emph{Attacco d'opportunità}}: il Vrock effettua un attacco ad una creatura che attraversi o esca dalla sua portata di 1 metro.

\emph{\textbf{Arrabbiato:}} Il Vrock striscia il becco con gli speroni rendendoli ancora più affilati. Fino alla fine del combattimento il danno causato da Becco e Speroni causa 1 danno da Sanguinamento fino ad un massimo di 10 danni. 1 Azione.

\textbf{Ecologia}\\
Ambiente: Qualsiasi (Abisso)\\
Organizzazione: Solitario, coppia o banda (3-10)\\
\textbf{Categoria Tesoro}: B\\
\textbf{Descrizione}\\
Profani campioni dell'Abisso, i vrock incarnano tutta la rabbia, l'odio e la violenza di questo reame. Tanto voraci e grottescamente opportunisti quanto il saprofago a cui assomigliano, i vrock si deliziano nello spargimento di sangue, godendo del suono e delle sensazioni derivanti dallo strappare gli intestini ancora pulsanti da una creatura vivente.\\
Un vrock tipico è alto 2,3 metri e pesa 200 kg. Queste creature solitamente si originano dalle anime di malvagi mortali colmi di odio e di collera, in particolare coloro che erano criminali professionisti, mercenari o assassini.

\mostro{Destriero dell'Incubo}
\begin{description}[noitemsep, topsep=0pt, parsep=0pt, partopsep=0pt, leftmargin=0cm, labelwidth=2.2cm]
	\item[\textbf{Taglia/Tipo:}] Grande immondo, malvagio
	\item[\textbf{Caratt.:}] \resizebox{0.5\linewidth+1.8cm}{!}{For 4 Des 2 Cos 3 Int 0 Sag 1 Car 2}
	\item[\textbf{Punti Ferita:}] 70,  \textbf{Difesa:} 18,  \textbf{Iniziativa:} +2
	\item[\textbf{Movimento:}] 18 m, volo 24 m
	\item[\textbf{Tiri Salvez.:}] \resizebox{0.5\linewidth+1.8cm}{!}{Tempra +6, Riflessi +5, Volontà +4}
	\item[\textbf{Imm. Danni:}] Fuoco
	\item[\textbf{Comp.:}] Consapevolezza +6
	\item[\textbf{Vulnerabilità:}] Luce
	\item[\textbf{Sensi:}] Scurovisione 36 m
	\item[\textbf{Linguaggi:}] comprende Abissale, Comune e Infernale ma non può parlare
	\item[\textbf{Sfida:}] 3 (700 PX)\smallskip
\end{description}

\emph{\textbf{Conferire Resistenza al Fuoco.}} Il destriero dell'incubo può conferire resistenza al danno da fuoco a chiunque lo cavalchi.

\emph{\textbf{Illuminazione.}} Il destriero da incubo irradia luce intensa in un raggio di 3 metri e luce fioca per 6 metri.

\textbf{Azioni}

\emph{\textbf{Zoccoli.} Attacco con arma da mischia}: +6 a colpire, portata 3 m, un bersaglio.

\emph{Colpisce:} 13 (2d8 + 4) danni contundenti più 7 (2d6) danni da fuoco.

\emph{\textbf{Passo Etereo.}} Il destriero da incubo e fino a tre creature consenzienti entro 3 metro da esso possono entrare magicamente nel Piano Etereo dal Piano Materiale e viceversa.

\textbf{Ecologia}\\
Ambiente: Qualsiasi\\
Organizzazione: Solitario\\
\textbf{Categoria Tesoro}: Nessuno\\
\textbf{Descrizione}\\
Gli incubi sono fiammeggianti messaggeri di morte. Permettono solo alle creature più malvagie di cavalcarli, e non sono mai soltanto cavalcature, ma collaborano nella distruzione provocata dai loro cavalieri.

\begin{enfasi}{L'inferno è vuoto, tutti i diavoli sono qui. (William Shakespeare, La Tempesta)}\end{enfasi}

\mostro{Diavolo Barbuto}
\noindent
\begin{description}[noitemsep, topsep=0pt, parsep=0pt, partopsep=0pt, leftmargin=0cm, labelwidth=2.2cm]
	\item[\textbf{Taglia/Tipo:}] Media diavolo, malvagio
	\item[\textbf{Caratt.:}] \resizebox{0.5\linewidth+1.8cm}{!}{For 3 Des 2 Cos 2 Int -1 Sag 0 Car 0}
	\item[\textbf{Punti Ferita:}] 70,  \textbf{Difesa:} 18,  \textbf{Iniziativa:} +2
	\item[\textbf{Movimento:}] 9 m
	\item[\textbf{Tiri Salvez.:}] \resizebox{0.5\linewidth+1.8cm}{!}{Tempra +5, Riflessi +5, Volontà +3}
	\item[\textbf{Res. Danni:}] Freddo; da arma non magica o non argentata
	\item[\textbf{Imm. Danni:}] Fuoco, Veleno
	\item[\textbf{Vulnerabilità:}] argento, Luce
	\item[\textbf{Sensi:}] Scurovisione 36 m
	\item[\textbf{Linguaggi:}] Infernale, telepatia 36 m
	\item[\textbf{Sfida:}] 3 (700 PX)\smallskip
\end{description}

\emph{\textbf{Resistenza alla Magia.}} Il diavolo ha +1d6 ai Tiri Salvezza contro incantesimi e altri effetti magici.

\emph{\textbf{Risoluto.}} Il diavolo non può essere spaventato finché riesce a vedere una creatura alleata entro 9 metri da lui.

\emph{\textbf{Vista del Diavolo.}} La Scurovisione del diavolo non è limitata dall'oscurità magica.

\textbf{Azioni}

\emph{\textbf{Multiattacco.}} Il diavolo effettua due attacchi: uno con la barba e uno con il falcione.

\emph{\textbf{Barba.} Attacco con arma da mischia}: +5 a colpire, portata 1 m, una creatura.

\emph{Colpisce:} 6 (1d8 + 2) danni perforanti e il bersaglio deve riuscire un Tiro Salvezza di Tempra DC 14 o restare avvelenato per 1 minuto. Mentre è avvelenato in questo modo, il bersaglio non può recuperare Punti Ferita. Il bersaglio può ripetere il Tiro Salvezza al termine di ciascun suo round, terminando l'effetto se riesce il Tiro Salvezza.

\emph{\textbf{Falcione.} Attacco con arma da mischia}: +6 a colpire, portata 3, un bersaglio.

\emph{Colpisce:} 8 (1d10 + 3) danni taglienti. Se il bersaglio è una creatura, ad esclusione di costrutti e non morti, deve riuscire un Tiro Salvezza su Tempra 15 o perdere 5 (1d10) Punti Ferita all'inizio di ciascun suo round a causa della ferita infernale. Ogni volta che il diavolo colpisce il bersaglio ferito con questo attacco, il danno inflitto dalla ferita aumenta di 5 (1d10). Qualsiasi creatura può effettuare due Azioni per bloccare la ferita con una prova riuscita di Saggezza (Pronto Soccorso) DC 12. La ferita si richiude anche nel caso in cui il bersaglio riceva della magia guaritrice.

\textbf{Reazione: \emph{Attacco d'opportunità}}: il diavolo effettua un attacco ad una creatura che attraversi o esca dalla sua portata di 1 metro.

\textbf{Ecologia}\\
Ambiente: Qualsiasi (Inferno)\\
Organizzazione: Solitario, coppia, squadra (3-10) o truppa (10-40)\\
\textbf{Categoria Tesoro}: Falcione, L\\
\textbf{Descrizione}\\
Guerrieri scelti delle legioni infernali, i diavoli barbuti, o barbazu, combattono selvaggiamente in nome dei loro signori infernali e in battaglia comandano orde brutali di dannati. Si radunano e si addestrano con i loro falcioni forgiati negli inferi, tra le volte del terzo girone dell'Inferno, Erebo, ma ritornano inevitabilmente nel primo girone, Averno, per servire al fianco del temibile signore Barbatos.

I barbazu amano effettuare attacchi di carica con i loro falcioni e cercano di mantenere una distanza di 3 metri tra loro ed i loro avversari, così che possono utilizzare le loro caratteristiche armi ad asta con la massima efficacia. In posizione eretta i diavoli barbuti sono alti più di 1,8 metri (sebbene la posizione accovacciata che tengono in battaglia li faccia spesso sembrare più bassi) e pesano più di 100 kg.


\begin{center}
	\includegraphics[width=0.9\linewidth]{immagini/Diavoli_giotto_2.png}

	\emph{Diavoli di Giotto}
\end{center}


\mostro{Diavolo delle Catene}
\noindent
\begin{description}[noitemsep, topsep=0pt, parsep=0pt, partopsep=0pt, leftmargin=0cm, labelwidth=2.2cm]
	\item[\textbf{Taglia/Tipo:}] Media diavolo, malvagio
	\item[\textbf{Caratt.:}] \resizebox{0.5\linewidth+1.8cm}{!}{For 4 Des 2 Cos 4 Int 0 Sag 1 Car 2}
	\item[\textbf{Punti Ferita:}] 165,  \textbf{Difesa:} 24,  \textbf{Iniziativa:} +2
	\item[\textbf{Movimento:}] 9 m
	\item[\textbf{Tiri Salvez.:}] \resizebox{0.5\linewidth+1.8cm}{!}{\resizebox{0.5\linewidth+1.8cm}{!}{Tempra +12, Riflessi +10, Volontà +9}}
	\item[\textbf{Res. Danni:}] Freddo; da arma non magica o non argentata
	\item[\textbf{Imm. Danni:}] Fuoco, Veleno
	\item[\textbf{Vulnerabilità:}] argento, Luce
	\item[\textbf{Sensi:}] Scurovisione 36 m
	\item[\textbf{Linguaggi:}] Infernale, telepatia 36 m
	\item[\textbf{Sfida:}] 8 (3900 PX)\smallskip
\end{description}

\emph{\textbf{Resistenza alla Magia.}} Il diavolo ha +1d6 ai Tiri Salvezza contro incantesimi e altri effetti magici.

\emph{\textbf{Vista del Diavolo.}} La Scurovisione del diavolo non è limitata dall'oscurità magica.

\textbf{Azioni}

\emph{\textbf{Multiattacco.}} Il diavolo effettua due attacchi con la catena.

\emph{\textbf{Catena.} Attacco con arma da mischia}: +9 a colpire, portata 3 m, un bersaglio.

\emph{Colpisce:} 11 (2d6 + 4) danni taglienti. Il bersaglio è afferrato (DC 14 per fuggire) se il diavolo non sta già afferrando un'altra creatura. Fino al termine dell'afferrare, il bersaglio subisce 7 (2d6) danni perforanti all'inizio di ciascun suo round.

\emph{\textbf{Animare Catene (Ricarica dopo un 1 ora).}} Fino a quattro catene che il diavolo possa vedere e si trovano entro 18 metri da lui producono dei bordi affilati e si animano sotto il controllo del diavolo, purché quelle catene non siano né indossate né trasportate da qualcun altro.

Ogni catena animata è un oggetto con Difesa 20, 20 Punti Ferita, resistenza ai danni perforanti, e immunità ai danni da suono. Quando il diavolo usa Multiattacco durante il suo round, può usare ciascuna catena animata per effettuare un ulteriore attacco di catena. Una catena animata può afferrare una creatura per conto proprio ma non può effettuare attacchi mentre afferra. Una catena animata ritorna al suo stato inanimato se viene ridotta a 0 Punti Ferita o se il diavolo è reso inabile o muore.

\textbf{Reazione: \emph{Maschera Snervante.}} Quando una creatura che il diavolo può vedere inizia il proprio round entro 9 metri dal diavolo, il diavolo può creare un'illusione per assomigliare all'amore perduto o un acerrimo rivale di quella creatura. Se la creatura può vedere il diavolo, deve riuscire un Tiro Salvezza di Volontà DC 21 o rimanere spaventata fino al termine del suo round.

\textbf{Reazione: \emph{Attacco d'opportunità}}: il diavolo effettua un attacco ad una creatura che attraversi o esca dalla sua portata di 3 metri.

\emph{\textbf{Arrabbiato:}} il Diavolo delle Catene agita le catene davanti a se. Fino alla fine del combattimento la Difesa è 27. Costa 1 Azione a round mantenere l'effetto.

\textbf{Ecologia}\\
Ambiente: Qualsiasi\\
Organizzazione: Solitario, coppia, anello (3-6) o catena (7-20)\\
\textbf{Categoria Tesoro}: R\\
\textbf{Descrizione}\\
Spesso classificati dai profani tra le fila dei diavoli infernali, i Diavolo delle Catene non sono veri diavoli. Anche se alcuni sono noti per vivere all'Inferno, essi esistono al di fuori delle gerarchie stabilite dagli dei degli inferi e dai suoi arcidiavoli e a volte si possono trovare su altri piani, in particolare sul Piano delle Ombre. Molti suggeriscono che siano nativi dell'Inferno che esisteva prima dell'avvento della stirpe diabolica, anche se altri ipotizzano che siano stati portati sul piano da qualche sadica potenza. Indipendentemente dalle loro origini vagano per i piani assecondano il loro desiderio di causare e ricevere sofferenza, ricercando il dolore attraverso violenti rapimenti e sadiche depravazioni.

\mostro{Diavolo Cornuto}
\noindent
\begin{description}[noitemsep, topsep=0pt, parsep=0pt, partopsep=0pt, leftmargin=0cm, labelwidth=2.2cm]
	\item[\textbf{Taglia/Tipo:}] Grande diavolo, malvagio
	\item[\textbf{Caratt.:}] \resizebox{0.5\linewidth+1.8cm}{!}{For 6 Des 3 Cos 5 Int 1 Sag 3 Car 3}
	\item[\textbf{Punti Ferita:}] 224,  \textbf{Difesa:} 29,  \textbf{Iniziativa:} +3
	\item[\textbf{Movimento:}] 6 m, volo 18 m
	\item[\textbf{Tiri Salvez.:}] \resizebox{0.5\linewidth+1.8cm}{!}{\resizebox{0.5\linewidth+1.8cm}{!}{Tempra +16, Riflessi +14, Volontà +14}}
	\item[\textbf{Res. Danni:}] Freddo;
	\item[\textbf{Imm. Danni:}] Fuoco, Veleno, armi +1
	\item[\textbf{Vulnerabilità:}] argento, Luce
	\item[\textbf{Sensi:}] Scurovisione 36 m
	\item[\textbf{Linguaggi:}] Infernale, telepatia 36 m
	\item[\textbf{Sfida:}] 11 (7200 PX)\smallskip
\end{description}

\emph{\textbf{Resistenza alla Magia.}} Il diavolo ha +1d6 ai Tiri Salvezza contro incantesimi e altri effetti magici.

\emph{\textbf{Vista del Diavolo.}} La Scurovisione del diavolo non è limitata dall'oscurità magica.

\emph{\textbf{Sguardo del comandante.}} i diavoli a più bassa Sfida entro 9 metri prendono +1 al Tiro per Colpire, Difesa e Tiri Salvezza. Non è cumulabile.

\textbf{Azioni}

\emph{\textbf{Multiattacco.}} Il diavolo effettua tre attacchi da mischia: due con il forcone e uno con il pungiglione. Può usare Scagliare Fiamma al posto di qualsiasi attacco da mischia.

\emph{\textbf{Coda.} Attacco con arma da mischia}: +10 a colpire, portata 3 m, un bersaglio.

\emph{Colpisce:} 10 (1d8 + 6) danni perforanti. Se il bersaglio è una creatura, ad esclusione di costrutti e non morti, deve riuscire un Tiro Salvezza su Tempra 25 o Sanguinare 10 (3d6). Ogni volta che il diavolo ferisce il bersaglio con questo attacco, il danno inflitto dal Sanguinamento aumenta di 10 (3d6).

\emph{\textbf{Forcone.} Attacco con arma da mischia}: +11 colpire, portata 3 m, un bersaglio.

\emph{Colpisce:} 15 (2d8 + 6) danni perforanti.

\emph{\textbf{Pungiglione.} Attacco con arma da mischia}: +9 a colpire, portata 3 m, un bersaglio.

\emph{Colpisce:} 13 (2d8 + 4) danni perforanti più 17 (5d6) danni da veleno, e il bersaglio deve riuscire un Tiro Salvezza di Tempra DC 24, o restare avvelenato, -1 Forza e Destrezza, per 1 minuto. Il bersaglio può ripetere il Tiro Salvezza al termine di ciascun suo round, terminando l'effetto se lo riesce.

\emph{\textbf{Scagliare Fiamma.} Attacco con incantesimo a Distanza}: +10 a colpire, gittata 45 m, un bersaglio.

\emph{Colpisce:} 14 (4d6) danni da fuoco. Se il bersaglio è un oggetto infiammabile che non sia indossato o trasportato, prende fuoco.

\textbf{Reazione: \emph{Attacco d'opportunità}}: il diavolo effettua un attacco ad una creatura che attraversi o esca dalla sua portata di 3 metri.

\emph{\textbf{Arrabbiato:}} il Diavolo Cornuto risucchia la vita che i nemici stanno perdendo. Fino alla fine del round successivo recupera tutti i Punti Ferita persi da Sanguinamento da ferite da lui causate.

\textbf{Ecologia}\\
Ambiente: Qualsiasi (Inferno)\\
Organizzazione: Solitario, coppia o stormo (3-10)\\
\textbf{Categoria Tesoro}: Catena Chiodata Sacrilega +1, P\\
\textbf{Descrizione}\\
Tra i più letali guerrieri degli arcidiavoli ed abili comandanti dei diavoli minori, i diavoli cornuti divulgano le regole dell'Inferno dovunque passano. Questi diavoli maggiori sono addestrati, forgiati e riforgiati per essere tra i più implacabili ed obbedienti guerrieri del multiverso. I diavoli cornuti delle truppe degli eserciti infernali sono noti come cornugon, mentre i più grandi tra loro sono chiamati malebranche.

Un diavolo cornuto tipico raggiunge la ragguardevole altezza di 2,7 metri, è dotato di ali con un'apertura di 4,2 metri, e pesa 350 kg.

\medskip

\begin{enfasi}{Il SIGNORE disse a Satana: "Da dove vieni?" Satana rispose al SIGNORE: "Dal percorrere la terra e dal passeggiare per essa". Giobbe 1,6-12}\end{enfasi}

\mostro{Diavolo della Fossa}
\noindent
\begin{description}[noitemsep, topsep=0pt, parsep=0pt, partopsep=0pt, leftmargin=0cm, labelwidth=2.2cm]
	\item[\textbf{Taglia/Tipo:}] Grande diavolo, malvagio
	\item[\textbf{Caratt.:}] \resizebox{0.5\linewidth+1.8cm}{!}{For 8 Des 2 Cos 7 Int 6 Sag 4 Car 7}
	\item[\textbf{Punti Ferita:}] 403,  \textbf{Difesa:} 40,  \textbf{Iniziativa:} +6
	\item[\textbf{Movimento:}] 9 m, volo 18 m
	\item[\textbf{Tiri Salvez.:}] \resizebox{0.5\linewidth+1.8cm}{!}{\resizebox{0.5\linewidth+1.8cm}{!}{Tempra +27, Riflessi +22, Volontà +24}}
	\item[\textbf{Res. Danni:}] Freddo;
	\item[\textbf{Imm. Danni:}] Fuoco, Veleno, armi +2
	\item[\textbf{Vulnerabilità:}] Luce
	\item[\textbf{Sensi:}] visione del vero 36 m
	\item[\textbf{Linguaggi:}] Infernale, telepatia 36 m
	\item[\textbf{Sfida:}] 20 (25000 PX)\smallskip
\end{description}

\emph{\textbf{Arma Magica.}} Gli attacchi con arma del diavolo della fossa sono magici.

\emph{\textbf{Aura di Paura.}} Qualsiasi creatura ostile al diavolo che inizi il suo round entro 6 metri da esso, deve effettuare un Tiro Salvezza su Volontà DC 35, a meno che il diavolo non sia inabile. Se fallisce il Tiro Salvezza, la creatura è spaventata fino all'inizio del suo prossimo round. Se il Tiro Salvezza della creatura riesce, la creatura è immune all'Aura di Paura del diavolo per le successive 24 ore.

\emph{\textbf{Incantesimi Innati.}} La caratteristica da incantatore diavolo della fossa è il Carisma. Il diavolo della fossa può lanciare questi incantesimi in maniera innata, senza bisogno di componenti materiali:

A volontà: \emph{\hyperlink{Individuazione del Magico}{Individuazione del Magico}, \hyperlink{Palla di Fuoco}{Palla di Fuoco}}

3/giorno ciascuno: \emph{blocca mostri, \hyperlink{Muro di Fuoco}{Muro di Fuoco}}

\emph{\textbf{Resistenza alla Magia.}} Il diavolo ha +1d6 ai Tiri Salvezza contro incantesimi e altri effetti magici.

\textbf{Azioni}

\emph{\textbf{Multiattacco.}} Il diavolo effettua quattro attacchi: uno con il morso, uno con l'artiglio, uno con la mazza e uno con la coda.

\emph{\textbf{Artiglio.} Attacco con arma da mischia}: +16 a colpire, portata 3 m, un bersaglio.

\emph{Colpisce:} 17 (2d8 + 8) danni taglienti, 3/20 danni da Sanguinamento.

\emph{\textbf{Coda.} Attacco con arma da mischia}: +16 a colpire, portata 3 m, un bersaglio.

\emph{Colpisce:} 24 (3d10 + 8) danni contundenti.

\emph{\textbf{Mazza.} Attacco con arma da mischia}: +17 a colpire, portata 3 m, un bersaglio.

\emph{Colpisce:} 15 (2d6 + 8) danni contundenti più 21 (6d6) danni da fuoco.

\emph{\textbf{Morso.} Attacco con arma da mischia}: +15 a colpire, portata 1 m, un bersaglio.

\emph{Colpisce:} 22 (4d6 + 8) danni perforanti. Il bersaglio deve riuscire un Tiro Salvezza di Tempra DC 33 o restare avvelenato. Mentre è avvelenato in questo modo, il bersaglio non può recuperare Punti Ferita, e subisce 21 (6d6) danni da veleno all'inizio di ciascun suo round. Il bersaglio avvelenato può ripetere il Tiro Salvezza al termine di ciascun suo round, terminando l'effetto su di sé.

\textbf{Reazione: \emph{Attacco d'opportunità}}: il diavolo effettua un attacco ad una creatura che attraversi o esca dalla sua portata di 3 metri.

\textbf{Ecologia}\\
Ambiente: Qualsiasi (Inferno)\\
Organizzazione: Solitario, coppia o concilio (3-9)\\
\textbf{Categoria Tesoro}: G\\
\textbf{Descrizione}\\
I diavoli della fossa sono potenti sovrani infernali, generali delle armate dell'Inferno e consiglieri degli arcidiavoli. Con corpi massicci e intelletti malvagi, dominano distese infernali e sottomettono mondi mortali. Alti oltre 4 metri e pesanti più di 500 kg, sono corazzati e dotati di ali imponenti.

I diavoli della fossa radunano eserciti, trasformando i lemure in veri diavoli e puntano a conquistare semipiani e mondi mortali vulnerabili. Obbediscono alla gerarchia infernale ma possono deporre padroni indegni. Solo i più potenti incantatori osano evocarli rischiando la dannazione eterna.

\mostro{Diavolo del Ghiaccio}
\noindent
\begin{description}[noitemsep, topsep=0pt, parsep=0pt, partopsep=0pt, leftmargin=0cm, labelwidth=2.2cm]
	\item[\textbf{Taglia/Tipo:}] Grande diavolo, malvagio
	\item[\textbf{Caratt.:}] \resizebox{0.5\linewidth+1.8cm}{!}{For 5 Des 2 Cos 4 Int 4 Sag 2 Car 4}
	\item[\textbf{Punti Ferita:}] 278,  \textbf{Difesa:} 32,  \textbf{Iniziativa:} +4
	\item[\textbf{Movimento:}] 12 m
	\item[\textbf{Tiri Salvez.:}] \resizebox{0.5\linewidth+1.8cm}{!}{\resizebox{0.5\linewidth+1.8cm}{!}{Tempra +18, Riflessi +16, Volontà +16}}
	\item[\textbf{Imm. Danni:}] Freddo, Fuoco, Veleno, armi +1
	\item[\textbf{Vulnerabilità:}] argento, Luce
	\item[\textbf{Sensi:}] Vista Cieca 18 m, Scurovisione 36 m
	\item[\textbf{Linguaggi:}] Infernale, telepatia 36 m
	\item[\textbf{Sfida:}] 14 (11500 PX)\smallskip
\end{description}

\emph{\textbf{Resistenza alla Magia.}} Il diavolo ha +1d6 ai Tiri Salvezza contro incantesimi e altri effetti magici.

\emph{\textbf{Vista del Diavolo.}} La Scurovisione del diavolo non è limitata dall'oscurità magica.

\textbf{Azioni}

\emph{\textbf{Multiattacco.}} Il diavolo effettua tre attacchi: uno con il morso, uno con gli artigli e uno con la coda. In alternativa effettua due attacchi: uno con la coda e uno con lancia.

\emph{\textbf{Artigli.} Attacco con arma da mischia}: +12 a colpire, portata 1 m, un bersaglio.

\emph{Colpisce:} 10 (2d4 + 5) danni taglienti più 10 (3d6) danni da freddo, 1 danno da Sanguinamento.

\emph{\textbf{Coda.} Attacco con arma da mischia}: +12 a colpire, portata 3 m, un bersaglio.

\emph{Colpisce:} 12 (2d6 + 5) danni contundenti più 10 (3d6) danni da freddo.

\emph{\textbf{Lancia di Ghiaccio.} Attacco con arma da mischia}: +12 a colpire, portata 3 m, un bersaglio.

\emph{Colpisce:} 14 (2d8 + 5) danni perforanti più 10 (3d6) danni da freddo. Se il bersaglio è una creatura, deve riuscire un Tiro Salvezza su Tempra DC 26, o avere per 1 minuto la velocità ridotta di 3 metri; durante ciascun suo round può effettuare solo un'Azione o un'Azione Immediata, ma non entrambe; non può effettuare reazioni. Il bersaglio può ripetere il Tiro Salvezza al termine di ciascun suo round, terminando l'effetto su di sé in caso di successo.

\emph{\textbf{Morso.} Attacco con arma da mischia}: +12 a colpire, portata 1 m, un bersaglio.

\emph{Colpisce:} 12 (2d6 + 5) danni perforanti più 10 (3d6) danni da freddo. TS su Tempra DC 18 o Rallentato 1/1r.

\emph{\textbf{Muro di Ghiaccio (Ricarica 6).}} Il diavolo forma magicamente un muro di ghiaccio opaco su di una superficie solida che possa vedere entro 18 metri da lui. Il muro è spesso 30 centimetri e largo fino a 9 metri per un massimo di 3 metri di altezza, oppure una cupola semisferica di massimo 6 metri di diametro. Quando la parete appare, ogni creatura nel suo spazio viene spinta fuori da esso tramite la via più breve. La creatura sceglie su quale lato del muro finire, a meno che la creatura non sia inabile. La creatura poi effettua un Tiro Salvezza di Riflessi DC 25, subendo 35 (10d6) danni da freddo se lo fallisce, o la metà di questi danni se lo riesce.

Il muro rimane per 1 minuto o finché il diavolo non è reso inabile o muore. Il muro può essere danneggiato e bucato; ogni sezione di 3 metri ha Difesa 5, 30 Punti Ferita, vulnerabilità al danno da fuoco, e Immune al Danno da acido, freddo, da Vuoto e da veleno. Se una sezione viene distrutta, lascia una patina di aria gelida nello spazio che occupava prima il muro. Ogni volta che una creatura finisce per muoversi attraverso quest'aria gelida durante un round, consenziente o meno, deve effettuare un Tiro Salvezza di Tempra DC 25, subendo 17 (5d6)danni da freddo se lo fallisce, o la metà di questi danni se lo riesce. L'aria gelida si dissipa quando il resto del muro svanisce.

\textbf{Reazione: \emph{Attacco d'opportunità}}: il diavolo effettua un attacco ad una creatura che attraversi o esca dalla sua portata di 1 metro.

\emph{\textbf{Arrabbiato:}} il Diavolo di Ghiaccio punta al cuore del nemico e cerca di strapparlo. La creatura, entro 1 metro, deve fare un Tiro Salvezza su Tempra DC 26 od avere il cuore strappato.

\textbf{Ecologia}\\
Ambiente: Qualsiasi (Inferno)\\
Organizzazione: Solitario, squadra (2-3), concilio (4-10) o contingente (1-3 diavoli del ghiaccio, 2-6 diavoli cornuti e 1-4 diavoli d'ossa\\
\textbf{Categoria Tesoro}: Lancia Gelida +1, R\\
\textbf{Descrizione}\\
Strateghi illuminati delle armate dell'Inferno, gli insettoidi diavoli del ghiaccio sono tra le menti più ingegnose e crudeli dell'infermo. Un diavolo del ghiaccio nasconde nel suo petto un cuore ghiacciato trafugato ad un mortale, che gli permette di prendere decisioni libero da emozioni. Nati nel girone ghiacciato di Cocito, il settimo girone infernale, la maggior parte dei diavoli del ghiaccio migra a Caina, l'ottavo girone, dove complotta per dannare il mondo. Sebbene abbiano le sembianze più aliene e mostruose tra tutti i diavoli, a pochi altri viene accordato un maggiore rispetto.

In combattimento manda avanti i suoi sottoposti, così da poter valutare le tattiche, i punti di forza e le debolezze dell'avversario nelle retrovie, e fornire loro supporto con le sue capacità magiche, evitando di coglierli nell'area di effetto dei suoi incantesimi: atteggiamento non dovuto ad un senso di cameratismo, bensì alla fredda e logica verità che i suoi alleati possono sopravvivere più a lungo in uno scontro se non sono esposti a fuoco amico.

I Diavoli del Ghiaccio sono alti 3,6 metri e pesano approssimativamente 350 kg.

\mostro{Diavolo d'Ossa}
\noindent
\begin{description}[noitemsep, topsep=0pt, parsep=0pt, partopsep=0pt, leftmargin=0cm, labelwidth=2.2cm]
	\item[\textbf{Taglia/Tipo:}] Grande diavolo, malvagio
	\item[\textbf{Caratt.:}] \resizebox{0.5\linewidth+1.8cm}{!}{For 4 Des 3 Cos 4 Int 1 Sag 2 Car 3}
	\item[\textbf{Punti Ferita:}] 184,  \textbf{Difesa:} 27,  \textbf{Iniziativa:} +3
	\item[\textbf{Movimento:}] 12 m, volo 12 m
	\item[\textbf{Tiri Salvez.:}] \resizebox{0.5\linewidth+1.8cm}{!}{\resizebox{0.5\linewidth+1.8cm}{!}{Tempra +13, Riflessi +12, Volontà +11}}
	\item[\textbf{Comp.:}] Ingannare +7, Percepire Emozioni +6
	\item[\textbf{Res. Danni:}] Freddo; da arma non magica o non argentata
	\item[\textbf{Imm. Danni:}] Fuoco, Veleno
	\item[\textbf{Vulnerabilità:}] argento, Luce
	\item[\textbf{Sensi:}] Scurovisione 36 m
	\item[\textbf{Linguaggi:}] Infernale, telepatia 36 m
	\item[\textbf{Sfida:}] 9 (5000 PX)\smallskip
\end{description}

\emph{\textbf{Resistenza alla Magia.}} Il diavolo ha +1d6 ai Tiri Salvezza contro incantesimi e altri effetti magici.

\emph{\textbf{Vista del Diavolo.}} La Scurovisione del diavolo non è limitata dall'oscurità magica.

\textbf{Azioni}

\emph{\textbf{Multiattacco.}} Il diavolo effettua tre attacchi: due con gli artigli e uno con il pungiglione oppure uno con la sua arma inastata uncinata e uno con il pungiglione.

\emph{\textbf{Arma Inastata Uncinata.} Attacco con arma da mischia}: +11 a colpire, portata 3 m, un bersaglio.

\emph{Colpisce:} 17 (2d12 + 4) danni perforanti. Se il bersaglio è una creatura di taglia Enorme o inferiore, può essere afferrata (DC 14 per fuggire). Fino al termine dell'afferrare, il diavolo non può usare la sua arma inastata su di un altro bersaglio.

\emph{\textbf{Artiglio.} Attacco con arma da mischia}: +9 a colpire, portata 3 m, un bersaglio.

\emph{Colpisce:} 8 (1d8 + 4) danni taglienti, 1 danno da Sanguinamento.

\emph{\textbf{Pungiglione.} Attacco con arma da mischia}: +9 a colpire, portata 3 m, un bersaglio.

\emph{Colpisce:} 13 (2d8 + 4) danni perforanti più 17 (5d6) danni da veleno, e il bersaglio deve riuscire un Tiro Salvezza di Tempra DC 21, o restare avvelenato, -1 Forza e Destrezza, per 1 minuto. Il bersaglio può ripetere il Tiro Salvezza al termine di ciascun suo round, terminando l'effetto se lo riesce.

\emph{\textbf{Arrabbiato:}} il Diavolo d'Ossa attacca tutte le creature intorno a lui con l'arma inastata. Tutte le creature ne raggio di 3 metri subiscono un attacco di Arma Inastata Uncinata, senza essere afferrati. Costo 2 Azioni. Il Diavolo d'ossa può decidere di diventare invisibile come sotto l'incantesimo di Invisibilità superiore. 2 Azioni.

\textbf{Ecologia}\\
Ambiente: Qualsiasi (Inferno)\\
Organizzazione: Solitario, squadra (2-3), concilio (4-10) o contingente (1-3 diavoli del ghiaccio, 2-6 diavoli cornuti e 1-4 diavoli d'ossa\\
\textbf{Categoria Tesoro}: I\\
\textbf{Descrizione}\\
I diavoli d'ossa sono inquisitori delle razze diaboliche, noti per la loro passione per la tortura di mortali, anime e altri diavoli. Nati nelle paludi di Stigia, nel quinto girone dell'Inferno, fanno rispettare gli ordini degli arcidiavoli con devozione assoluta.

I diavoli d'ossa viaggiano spesso fino al piano mortale per servire malvagi incantatori, recuperando informazioni preziose. Alti 2,7 metri e pesanti oltre 200 kg, con coda e ali terrificanti, sono imponenti e temuti.

\mostro{Diavolo Spinoso}
\noindent
\begin{description}[noitemsep, topsep=0pt, parsep=0pt, partopsep=0pt, leftmargin=0cm, labelwidth=2.2cm]
	\item[\textbf{Taglia/Tipo:}] Piccola diavolo, malvagio
	\item[\textbf{Caratt.:}] \resizebox{0.5\linewidth+1.8cm}{!}{For 0 Des 2 Cos 1 Int 0 Sag 2 Car -1}
	\item[\textbf{Punti Ferita:}] 51,  \textbf{Difesa:} 16,  \textbf{Iniziativa:} +2
	\item[\textbf{Movimento:}] 6 m, volo 12 m
	\item[\textbf{Tiri Salvez.:}] \resizebox{0.5\linewidth+1.8cm}{!}{Tempra +3, Riflessi +4, Volontà +4}
	\item[\textbf{Res. Danni:}] Freddo; da arma non magica o non argentata
	\item[\textbf{Imm. Danni:}] Fuoco, Veleno
	\item[\textbf{Vulnerabilità:}] argento, Luce
	\item[\textbf{Sensi:}] Scurovisione 36 m
	\item[\textbf{Linguaggi:}] Infernale, telepatia 36 m
	\item[\textbf{Sfida:}] 2 (450 PX)\smallskip
\end{description}

\emph{\textbf{Resistenza alla Magia.}} Il diavolo ha +1d6 ai Tiri Salvezza contro incantesimi e altri effetti magici.

\emph{\textbf{Sorvolare.}} Il diavolo non provoca attacchi di opportunità quando vola via dalla portata di un nemico.

\emph{\textbf{Spine Limitate.}} Il diavolo possiede dodici spine caudali. Le spine usate ricrescono a mezzanotte.

\emph{\textbf{Vista del Diavolo.}} La Scurovisione del diavolo non è limitata dall'oscurità magica.

\textbf{Azioni}

\emph{\textbf{Multiattacco.}} Il diavolo effettua due attacchi: uno con il morso e uno con il suo forcone o due con le sue spine caudali.

\emph{\textbf{Forcone.} Attacco con arma da mischia}: +4 a colpire, portata 1 m, un bersaglio.

\emph{Colpisce:} 3 (1d6) danni perforanti.

\emph{\textbf{Morso.} Attacco con arma da mischia}: +4 a colpire, portata 1 m, un bersaglio.

\emph{Colpisce:} 5 (2d4) danni taglienti.

\emph{\textbf{Spina Caudale.} Attacco con arma a Distanza}: +4 a colpire, gittata 6m, un bersaglio.

\emph{Colpisce:} 4 (1d4 + 2) danni perforanti più 3 (1d6) danni da fuoco.

\textbf{Ecologia}\\
Ambiente: Qualsiasi (Inferno)\\
Organizzazione: Solitario, coppia, gruppo (3-5) o plotone (6-11)\\
\textbf{Categoria Tesoro}: J\\
\textbf{Descrizione}\\
Sentinelle delle volte dell'Inferno, carcerieri delle anime più nere e armi viventi delle forge infernali. Un Diavolo Spinoso ama sentire il sangue caldo sulle proprie spine e preferisce gettarsi nella mischia quando gli viene offerta l'opportunità di combattere.
I Diavoli Spinoso sono collezionisti ed organizzatori. Se lasciati agire liberamente, nei nascondigli di questi diavoli spesso fanno mostra i trofei trafitti di vecchie vittime.
La maggior parte dei diavoli spinosi è alta dai 2,1 metri in su e pesa 150 kg, sebbene i loro corpi asciutti e muscolosi sembrino più grossi per via degli spuntoni in continua crescita che fuoriescono dai loro corpi, taglienti come lame.

\mostro{Erinni}
\noindent
\begin{description}[noitemsep, topsep=0pt, parsep=0pt, partopsep=0pt, leftmargin=0cm, labelwidth=2.2cm]
	\item[\textbf{Taglia/Tipo:}] Media diavolo, malvagio
	\item[\textbf{Caratt.:}] \resizebox{0.5\linewidth+1.8cm}{!}{For 4 Des 3 Cos 4 Int 2 Sag 2 Car 4}
	\item[\textbf{Punti Ferita:}] 240,  \textbf{Difesa:} 31,  \textbf{Iniziativa:} +3
	\item[\textbf{Movimento:}] 9 m, volo 18 m
	\item[\textbf{Tiri Salvez.:}] \resizebox{0.5\linewidth+1.8cm}{!}{\resizebox{0.5\linewidth+1.8cm}{!}{Tempra +16, Riflessi +15, Volontà +14}}
	\item[\textbf{Res. Danni:}] Freddo; da arma non magica o non argentata
	\item[\textbf{Imm. Danni:}] Fuoco, Veleno
	\item[\textbf{Vulnerabilità:}] argento, Luce
	\item[\textbf{Sensi:}] visione del vero 36 m
	\item[\textbf{Linguaggi:}] Infernale, telepatia 36 m
	\item[\textbf{Sfida:}] 12 (8400 PX)\smallskip
\end{description}

\emph{\textbf{Armi Diaboliche.}} Gli attacchi con arma dell'erinni sono magici e infliggono 13 (3d8) danni da veleno aggiuntivi quando colpiscono (già incluso negli attacchi).

\emph{\textbf{Resistenza alla Magia.}} L'erinni ha +1d6 ai Tiri Salvezza contro incantesimi e altri effetti magici.

\textbf{Azioni}

\emph{\textbf{Multiattacco.}} L'erinni effettua tre attacchi.

\emph{\textbf{Spada Lunga.} Attacco con arma da mischia}: +11 a colpire, portata 1 m, un bersaglio.

\emph{Colpisce:} 8 (1d8 + 4) danni taglienti, o 9 (1d10 + 4) danni taglienti se usata con due mani, più 13 (3d8) danni da veleno.

\emph{\textbf{Arco Lungo.} Attacco con arma a Distanza}: +11 a colpire, gittata 45m, un bersaglio.

\emph{Colpisce:} 7 (1d8 + 4) danni perforanti più 13 (3d8) danni da veleno, e il bersaglio deve riuscire un Tiro Salvezza di Tempra DC 25 o restare avvelenato, -1 Forza e Destrezza. Il veleno rimane finché non viene rimosso da un incantesimo \emph{ristorazione inferiore} o simile.

\textbf{Reazione: \emph{Attacco d'opportunità}}: il diavolo effettua un attacco ad una creatura che attraversi o esca dalla sua portata di 1 metro.

\textbf{Reazione: \emph{Parata.}} L'erinni somma 4 alla sua Difesa contro un attacco da mischia che lo colpirebbe. Per farlo, l'erinni deve poter vedere il suo attaccante e impugnare un'arma da mischia.

\emph{\textbf{Arrabbiato:}} L'Erinni incanala la sua energia magica in un attacco. Il bersaglio dell'attacco viene colpito da una infernale fiamma che causa 12d6 di danno da Vuoto. Tiro Salvezza DC 25 Riflessi per dimezzare. Costa 2 Azioni.

\textbf{Ecologia}\\
Ambiente: Qualsiasi (Inferno)\\
Organizzazione: Solitario o trio\\
\textbf{Categoria Tesoro}: Arco Lungo Composito Infuocato +1 [Forza +5], corda, Spada Lunga+1\\
\textbf{Descrizione}\\
Le erinni, note anche come Caduti, Ali Cineree e Furie, sono diavoli che insultano la loro forma angelica con la loro sete di vendetta e giustizia sanguinosa. Volteggiano sopra i cornicioni di Dite, il secondo girone dell'Inferno, sempre pronte alla battaglia per difendere l'inferno o per i capricci dei loro signori diabolici.

Questi angeli bellissimi e oscuri accrescono la loro sensualità con cicatrici e lividi, ma preferiscono risolvere i problemi con violenza rapida e atroce. Utilizzano corde viventi fatte dei loro capelli per intralciare e sollevare i nemici, prolungando le loro sofferenze.

Le erinni sono alte circa 1,8 metri, pesano 70 kg e hanno ali nere con un'apertura di oltre 3 metri. Sono abili nel mantenere i nemici in vita per prolungare il tormento, e le più potenti possono far perdurare le sofferenze anche dopo la morte del soggetto.

\mostro{Imp}
\noindent
\begin{description}[noitemsep, topsep=0pt, parsep=0pt, partopsep=0pt, leftmargin=0cm, labelwidth=2.2cm]
	\item[\textbf{Taglia/Tipo:}] Minuscola diavolo, mutaforma, malvagio
	\item[\textbf{Caratt.:}] \resizebox{0.5\linewidth+1.8cm}{!}{For -2 Des 3 Cos 1 Int 0 Sag 1 Car 2}
	\item[\textbf{Punti Ferita:}] 33,  \textbf{Difesa:} 16,  \textbf{Iniziativa:} +3
	\item[\textbf{Movimento:}] 6 m, volo 12 m (6 m in forma di ratto; 6 m, volo 18 m in forma di corvo; 6 m, scalata 6 m in forma di ragno)
	\item[\textbf{Tiri Salvez.:}] \resizebox{0.5\linewidth+1.8cm}{!}{Tempra +3, Riflessi +4, Volontà +3}
	\item[\textbf{Comp.:}] Furtività +5, Ingannare +4, Percepire Emozioni +3
	\item[\textbf{Res. Danni:}] Freddo; da arma non magica o non argentata
	\item[\textbf{Imm. Danni:}] Fuoco, Veleno
	\item[\textbf{Vulnerabilità:}] argento, Luce
	\item[\textbf{Sensi:}] Scurovisione 36 m
	\item[\textbf{Linguaggi:}] Infernale, Comune
	\item[\textbf{Sfida:}] 1 (200 PX)\smallskip
\end{description}

\emph{\textbf{Mutaforma.}} Il diavolo può usare una Azione per trasformarsi in una forma bestiale da ratto, corvo o ragno, o per tornare alla sua vera forma. Le sue statistiche sono le stesse in tutte le forme, sebbene gli attacchi possano variare per alcune di esse. Qualsiasi equipaggiamento stia indossando o trasportando non viene trasformato. Alla morte ritorna alla sua vera forma.

\emph{\textbf{Resistenza alla Magia.}} Il diavolo ha +1d6 ai Tiri Salvezza contro incantesimi e altri effetti magici.

\emph{\textbf{Vista del Diavolo.}} La Scurovisione del diavolo non è limitata dall'oscurità magica.

\textbf{Azioni}

\emph{\textbf{Pungiglione (Morso in Forma di Bestia).} Attacco con arma da mischia}: +4 a colpire, portata 1 m, una creatura.

\emph{Colpisce:} 5 (1d4 + 3) danni perforanti e il bersaglio deve effettuare un Tiro Salvezza di Tempra DC 12, subendo 10 (3d6) danni da veleno se lo fallisce, o la metà di questi danni se lo riesce.

\emph{\textbf{Invisibilità.}} Il diavolo resta invisibile finché non attacca o termina la sua concentrazione. Qualsiasi cosa che il diavolo stia trasportando o indossando, resta invisibile finché rimane in contatto con il diavolo.

\textbf{Ecologia}\\
Ambiente: Qualsiasi (Inferno)\\
Organizzazione: Solitario, coppia o stormo (3-10)\\
\textbf{Categoria Tesoro}: K\\
\textbf{Descrizione}\\
Nati direttamente dalle fosse dell'Inferno, gli imp sono i diavoli meno potenti, anche se queste crudeli ed invadenti creature svolgono un ruolo importante nella corruzione delle anime mortali. Libere dalle gerarchie e dai doveri delle armate infernali, gli imp si dilettano ad ogni opportunità di viaggiare fino al Piano Materiale e di tentare astutamente i mortali, spingendoli a compiere atti sempre più depravati.

Volontariamente al servizio di incantatori nel ruolo di famigli, gli imp recitano la parte dei fedeli servitori, offrendo spesso ai loro padroni astuti consigli ed infernali intuizioni. In realtà, gli imp operano per inviare anime all'Inferno, accertandosi che l'anima del loro padrone, insieme a molte altre, sia destinata alla dannazione dopo la morte.

Gli imp variano molto in aspetto, in un ampio spettro di tratti bestiali e grotteschi, sebbene molti di essi abbiano la forma di un umanoide alato dalla pelle rossiccia, con lineamenti bulbosi. Il tipico imp è alto solamente 60 centimetri, ha un'apertura alare di 90 centimetri e pesa 5 kg.

Diversamente dagli altri diavoli, gli imp si ritrovano spesso liberi e soli nel Piano Materiale, in particolare dopo che sono stati evocati per servire come famigli ed i loro padroni sono morti (spesso, indirettamente, a causa delle macchinazioni dell'imp stesso). Senza alcun mezzo per poter fare ritorno a casa questi imp, liberi da ogni legame con padroni arcani, possono diventare pericolosi seccatori o persino porsi a capo di piccole tribù di sanguinosi umanoidi, quali Gablin o Coboldi.

\mostro{Lemure}
\noindent
\begin{description}[noitemsep, topsep=0pt, parsep=0pt, partopsep=0pt, leftmargin=0cm, labelwidth=2.2cm]
	\item[\textbf{Taglia/Tipo:}] Media diavolo, malvagio
	\item[\textbf{Caratt.:}] \resizebox{0.5\linewidth+1.8cm}{!}{For 0 Des -3 Cos 0 Int -5 Sag 0 Car -4}
	\item[\textbf{Punti Ferita:}] 15,  \textbf{Difesa:} 9,  \textbf{Iniziativa:} -3
	\item[\textbf{Movimento:}] 5 metri
	\item[\textbf{Tiri Salvez.:}] \resizebox{0.5\linewidth+1.8cm}{!}{Tempra +3, Riflessi +3, Volontà +3}
	\item[\textbf{Res. Danni:}] Freddo
	\item[\textbf{Imm. Danni:}] Fuoco, Veleno
	\item[\textbf{Immunità:}] affascinato, spaventato
	\item[\textbf{Vulnerabilità:}] argento, Luce
	\item[\textbf{Sensi:}] Scurovisione 36 m
	\item[\textbf{Linguaggi:}] comprende l'Infernale ma non può parlare
	\item[\textbf{Sfida:}] 0 (10 PX)\smallskip
\end{description}

\emph{\textbf{Rinvigorimento Diabolico.}} Un lemure che muore nei Nove Inferi ritorna in vita con tutti i suoi Punti Ferita in 1d10 giorni a meno che non venga ucciso da una creatura con tratti buoni su cui sia stato eseguito l'incantesimo \emph{benedire} o i suoi resti vengano cosparsi di Acqua santa.

\emph{\textbf{Vista del Diavolo.}} La Scurovisione del diavolo non è limitata dall'oscurità magica.

\textbf{Azioni}

\emph{\textbf{Pugno.} Attacco con arma da mischia}: +3 a colpire, portata 1 m, un bersaglio.

\emph{Colpisce:} 2 (1d4) danni contundenti.

\textbf{Ecologia}\\
Ambiente: Qualsiasi (Inferno)\\
Organizzazione: Solitario, coppia, gruppo (3-5), sciame (6-17) o schiera (10-40 o più)\\
\textbf{Categoria Tesoro}: Nessuno\\
\textbf{Descrizione}\\
I lemure sono i diavoli più infimi, nati dalle anime dannate all'inferno. Sono masse informi di carne tremolante, con tratti grotteschi che imitano i loro torturatori. Alti più di 1,2 metri e pesanti oltre 100 kg, sono creature rivoltanti che distruggono qualsiasi forma di vita non infernale.

Essi rivestono un ruolo vitale nell'ecologia dell'Inferno. Le anime dannate vengono tormentate per secoli, dimenticando le loro vite e diventando automi guidati dall'odio e dalla paura. Alla fine, queste anime vengono trasformate in lemure, la forma di vita più elementare dei diavoli.

I diavoli maggiori possono riconoscere i lemure più corrotti e trasformarli in veri diavoli, pronti a servire nelle legioni dei dannati.

\mostro{Plesiosauro}
\noindent
\begin{description}[noitemsep, topsep=0pt, parsep=0pt, partopsep=0pt, leftmargin=0cm, labelwidth=2.2cm]
	\item[\textbf{Taglia/Tipo:}] Grande bestia, disallineato
	\item[\textbf{Caratt.:}] \resizebox{0.5\linewidth+1.8cm}{!}{For 4 Des 2 Cos 3 Int -4 Sag 1 Car -3}
	\item[\textbf{Punti Ferita:}] 52,  \textbf{Difesa:} 16,  \textbf{Iniziativa:} +2
	\item[\textbf{Movimento:}] 6 m, nuoto 12 m
	\item[\textbf{Tiri Salvez.:}] \resizebox{0.5\linewidth+1.8cm}{!}{Tempra +5, Riflessi +4, Volontà +3}
	\item[\textbf{Comp.:}] Furtività +4, Consapevolezza +3
	\item[\textbf{Sfida:}] 2 (450 PX)\smallskip
\end{description}

\emph{\textbf{Trattenere il Fiato.}} Il plesiosauro può trattenere il fiato per 1 ora.

\textbf{Azioni}

\emph{\textbf{Morso.} Attacco con arma da mischia}: +5 a colpire, portata 3 m, un bersaglio.

\emph{Colpisce:} 14 (3d6 + 4) danni perforanti.

\textbf{Ecologia}\\
Ambiente: Acquatico Caldo\\
Organizzazione: Solitario, coppia o branco (3-6)\\
\textbf{Categoria Tesoro}: Nessuno\\
\textbf{Descrizione}\\
Il plesiosauro è un rettile acquatico dal lungo collo. Sebbene tecnicamente non sia un dinosauro, questa creatura ed i suoi simili si trovano spesso a cacciare in laghi ed oceani nei quali è facile trovare dei dinosauri.

\mostro{Tirannosauro}
\noindent
\begin{description}[noitemsep, topsep=0pt, parsep=0pt, partopsep=0pt, leftmargin=0cm, labelwidth=2.2cm]
	\item[\textbf{Taglia/Tipo:}] Enorme bestia, disallineato
	\item[\textbf{Caratt.:}] \resizebox{0.5\linewidth+1.8cm}{!}{For 7 Des 0 Cos 4 Int -4 Sag 1 Car -1}
	\item[\textbf{Punti Ferita:}] 165,  \textbf{Difesa:} 22,  \textbf{Iniziativa:} +0
	\item[\textbf{Movimento:}] 15 m
	\item[\textbf{Tiri Salvez.:}] \resizebox{0.5\linewidth+1.8cm}{!}{\resizebox{0.5\linewidth+1.8cm}{!}{Tempra +12, Riflessi +8, Volontà +9}}
	\item[\textbf{Sfida:}] 8 (3900 PX)\smallskip
\end{description}

\textbf{Azioni}

\emph{\textbf{Multiattacco.}} Il tirannosauro effettua due attacchi: uno con il morso e uno con la coda. Non può effettuare entrambi gli attacchi contro lo stesso bersaglio.

\emph{\textbf{Coda.} Attacco con arma da mischia}: +10 a colpire, portata 3 m, un bersaglio.

\emph{Colpisce:} 20 (3d8 + 7) danni contundenti.

\emph{\textbf{Morso.} Attacco con arma da mischia}: +10 a colpire, portata 3 m, un bersaglio.

\emph{Colpisce:} 33 (4d12 + 7) danni perforanti. Se il bersaglio è una creatura di taglia Media o inferiore, è afferrato (DC 17 per fuggire). Fino al termine dell'afferrare il tirannosauro non può usare il morso contro un altro bersaglio.

\emph{\textbf{Zampata.} Attacco con arma da mischia}: +10 a colpire, portata 6 m, fino a due bersagli. Il Tirannosauro concentra le azioni e salta su un avversario. 2 Azioni

\emph{Colpisce:} 30 (4d10 + 8) danni contundenti.

\textbf{Reazione: \emph{Attacco d'opportunità}}: il Tirannosauro effettua un attacco ad una creatura che attraversi o esca dalla sua portata di 3 metri.

\emph{\textbf{Arrabbiato:}} il Tirannosauro è pervaso da furia assassina. Attacca qualsiasi creatura amica o nemica. Il Tiro per Colpire guadagna +1d6 ed il morso causa Sanguinamento 2/15.

\textbf{Ecologia}\\
Ambiente: Foreste e Pianure Calde\\
Organizzazione: Solitario, coppia o branco (3-6)\\
\textbf{Categoria Tesoro}: Nessuno\\
\textbf{Descrizione}\\
Il tirannosauro è un predatore primario che misura 12 metri di lunghezza e pesa 7000 kg.

\mostro{Triceratopo}
\noindent
\begin{description}[noitemsep, topsep=0pt, parsep=0pt, partopsep=0pt, leftmargin=0cm, labelwidth=2.2cm]
	\item[\textbf{Taglia/Tipo:}] Enorme bestia, disallineato
	\item[\textbf{Caratt.:}] \resizebox{0.5\linewidth+1.8cm}{!}{For 6 Des -1 Cos 3 Int -4 Sag 0 Car -3}
	\item[\textbf{Punti Ferita:}] 108,  \textbf{Difesa:} 17,  \textbf{Iniziativa:} -1
	\item[\textbf{Movimento:}] 15 m
	\item[\textbf{Tiri Salvez.:}] \resizebox{0.5\linewidth+1.8cm}{!}{Tempra +8, Riflessi +4, Volontà +5}
	\item[\textbf{Sfida:}] 5 (1800 PX)\smallskip
\end{description}

\emph{\textbf{Carica Travolgente.}} Se il triceratopo si muove di almeno 6 metri diretto verso una creatura e la colpisce con un attacco di incornata durante lo stesso round, il bersaglio deve riuscire un Tiro Salvezza su Tempra DC 19 o cadere prono. Se il bersaglio è prono, il triceratopo può effettuare un attacco di pestone contro di lui come Azione Immediata.

\textbf{Azioni}

\emph{\textbf{Incornata.} Attacco con arma da mischia}: +7 a colpire, portata 1 m, un bersaglio.

\emph{Colpisce:} 24 (3d10 + 6) danni perforanti.

\emph{\textbf{Pestone.} Attacco con arma da mischia}: +7 a colpire, portata 1 m, una creatura prona.

\emph{Colpisce:} 22 (3d10 + 6) danni contundenti.

\textbf{Reazione: \emph{Collare protettivo}} il Triceratopo piega la testa sotto il collare e guadagna +4 alla Difesa.

\textbf{Ecologia}\\
Ambiente: Pianure Calde\\
Organizzazione: Solitario, coppia o branco (5-8)\\
\textbf{Categoria Tesoro}: Nessuno\\
\textbf{Descrizione}\\
Il triceratopo è un erbivoro irascibile e caparbio. Un tipico triceratopo è lungo 9 metri e pesa 10000 kg.

\mostro{Divora Cervelli}
\noindent
\begin{description}[noitemsep, topsep=0pt, parsep=0pt, partopsep=0pt, leftmargin=0cm, labelwidth=2.2cm]
	\item[\textbf{Taglia/Tipo:}] Piccola aberrazione, malvagio
	\item[\textbf{Caratt.:}] \resizebox{0.5\linewidth+1.8cm}{!}{For 1 Des 6 Cos 5 Int 3 Sag 0 Car 3}
	\item[\textbf{Punti Ferita:}] 186,  \textbf{Difesa:} 30,  \textbf{Iniziativa:} +6
	\item[\textbf{Movimento:}] 12 m
	\item[\textbf{Tiri Salvez.:}] \resizebox{0.5\linewidth+1.8cm}{!}{\resizebox{0.5\linewidth+1.8cm}{!}{Tempra +14, Riflessi +15, Volontà +9}}
	\item[\textbf{Imm. Danni:}] Fuoco
	\item[\textbf{Immunità:}] incantesimi dalle liste di magia Illusione e Charme
	\item[\textbf{Sensi:}] Vista Cieca 18 m
	\item[\textbf{Linguaggi:}] telepatia 50 m
	\item[\textbf{Sfida:}] 9 (5000 PX)\smallskip
\end{description}

\emph{\textbf{Occhi della Magia.}} Il Divora Cervelli ha \hyperlink{Individuazione del Magico}{Individuazione del Magico} sempre attivo.

\emph{\textbf{Incantesimi Innati.}} La caratteristica da incantatore del Divora Cervelli è il Carisma. Il Divora Cervelli può lanciare in maniera innata i seguenti incantesimi, senza bisogno di componenti materiali:

A volontà: \emph{\hyperlink{Confusione}{Confusione} (un unico bersaglio), \hyperlink{Infliggi Ferite}{Infliggi Ferite} con un critico, \hyperlink{Invisibilità}{Invisibilità}}

3/giorno: \emph{\hyperlink{Cura Ferite}{Cura Ferite} 3, \hyperlink{Globo di Invulnerabilità}{Globo di Invulnerabilità}}

\textbf{Azioni}

\emph{\textbf{Multiattacco.}} Il Divora Cervelli può effettuare 4 attacchi, uno per artiglio

\emph{\textbf{Artiglio.} Attacco con arma da mischia}: +9 al a colpire, portata 1 m, una creatura.

\emph{Colpisce:} 3 danni da taglio (1d4+1), 1 danno da Sanguinamento.

\textbf{Abilità speciali}

\emph{\textbf{Furto del corpo}}

Spendendo 3 Azioni un Divora Cervelli può diventare minuscolo e strisciare nella bocca/naso/orecchie di una creatura indifesa o morta ed arrivare al cervello per nutrirsene. Si tratta di una azione che uccide la creatura.

Il Divora Cervelli assume il controllo del corpo e lo può usare a suo piacimento, come se controllasse la vittima con un incantesimo Dominare Mostri. Il Divora Cervelli ha pieno accesso a tutte le capacità difensive e offensive dell'ospite tranne che per le capacità magiche e gli incantesimi (anche se il Divora Cervelli può comunque usare le proprie capacità magiche).

Un corpo ospite non deve essere morto da più di 1 giorno perché questa capacità funzioni, e anche dopo essere stati occupati con successo i corpi si decompongono diventando inutilizzabili in 7 giorni (a meno che questo periodo venga prolungato con l'incantesimo Riposo Inviolato).

Finché il Divora Cervelli occupa il corpo, conosce (e può parlare) i linguaggi conosciuti dalla vittima e le informazioni sulla sua identità e personalità, ma non può possederne gli specifici ricordi e conoscenze.

Il danno inflitto al corpo, che ha il doppio dei Punti Ferita originali, ospite non danneggia il Divora Cervelli e se il corpo ospite viene distrutto il Divora Cervelli esce ed è Stordito per 1 round.

\textbf{Ecologia}\\
Ambiente: Qualsiasi sotterraneo\\
Organizzazione: Solitario, covata (2-6) o tribù (7-16)\\
\textbf{Categoria Tesoro}: G\\
\textbf{Descrizione}\\
Un Divora Cervelli altro non è che un cervello di circa 50 cm dotato di 4 potenti zampe artigliate.

I Divora Cervelli sono certamente una tra le razze più crudeli del mondo. Incapaci di provare emozioni o di sguazzare nei peccati del proprio piacere fisico, i divora cervelli sono costretti a rubare corpi per soddisfare la loro golosità, lussuria e crudeltà. Esistono storie che narrano di intere città sotterranee di queste creature che indossano corpi come se fossero vestiti per consumare spaventose orge e macabri festini.

I Divora Cervelli solitari spesso vivono in rovine o caverne ai margini delle regioni civilizzate per poter fare periodiche scorrerie in città per "acquistare" un nuovo allettante corpo.

Si dice che il giardino di Shayalia sia pieno di Divora Cervelli. Un Divora Cervelli è lungo 90 cm e pesa circa 30 kg.

\mostro{Dobi}
\noindent
\begin{description}[noitemsep, topsep=0pt, parsep=0pt, partopsep=0pt, leftmargin=0cm, labelwidth=2.2cm]
	\item[\textbf{Taglia/Tipo:}] Minuscola fatata, neutrale
	\item[\textbf{Caratt.:}] \resizebox{0.5\linewidth+1.8cm}{!}{For -3 Des -1 Cos 2 Int -2 Sag 1 Car 3}
	\item[\textbf{Punti Ferita:}] 15,  \textbf{Difesa:} 11,  \textbf{Iniziativa:} -1
	\item[\textbf{Movimento:}] 3 m, Nuotare 9 m
	\item[\textbf{Tiri Salvez.:}] \resizebox{0.5\linewidth+1.8cm}{!}{Tempra +3, Riflessi +3, Volontà +3}
	\item[\textbf{Sensi:}] Visione Crepuscolare 18 m
	\item[\textbf{Linguaggi:}] comprende il Comune, ma non lo parla
	\item[\textbf{Imm. Danni:}] al danno delle armi non magiche
	\item[\textbf{Sfida:}] 0 (10 PX)\smallskip
\end{description}

\emph{\textbf{Dobi}} Il Dobi si appiccica, per spostarlo è necessario essere gentili e chiederglielo.\\
\emph{\textbf{Dobi Dobi Dobi}} Quando il Dobi subisce più di 3 punti ferita di danno con un arma non contundente si divide in due Dobi più piccoli ognuno con lo stesso ammontare di Punti Ferita rimasti al Dobi precedente.\\
\smallskip\textbf{Azioni}\\
\emph{\textbf{Dobi Dobi}} il Dobi proietta un aura di \hyperlink{Calmare Emozioni}{Calmare Emozioni} come l'omonimo incantesimo ma non è concesso il Tiro Salvezza. Il Dobi può influenzare una sola creatura alla volta con il suo potere.\\
\textbf{Ecologia}\\
Ambiente: Paludi\\
Organizzazione: gruppo\\
\textbf{Categoria Tesoro}: Accidentale\\
\textbf{Descrizione}\\
{\small "...Smossi le foglie dell'acquitrino e vidi a terra una strana palla di pelo, di circa dieci centimetri di diametro, di colore chiaro. Incuriosito lo raccolsi, accarezzando il suo pelo soffice e lo scrutai con attenzione. Sembrava non avere arti o segni di possedere un muso con occhi, orecchie, bocca, ma non appena lo accarezzai la palla vibrò, emettendo uno squittio.

	Finalmente scorsi due occhietti neri e vispi aprirsi in tutto quel pelo e poi due orecchiette tonde spuntare, quindi due zampette corte ma robuste, adatte al salto, appoggiate a terra e altre due, sempre corte ma dotate di ben cinque dita ognuna, a mezza altezza.

	- Dobi! - rispose l'animaletto, esprimendo una sorta di gioia ed entusiasmo. - Dobi dobi! -.

	- Che carino! - esclamai, accarezzandolo. Era l'animaletto più tenero che avessi mai visto. - Ora però ti rimetto giù -.

	- Dobi - rispose la palla di pelo.
	Portai a terra la mano, ma l'animale non si mosse. Provai a staccarmelo dalla mano, ma rimase appiccicato all'altra. Lo presi con due dita, tirando forte e lo appoggiai veloce a terra, ma subito mi saltò sul piede e vi rimase attaccato. Dovetti attraversare l'acquitrino con il dobi attaccato al piede, senza contare gli altri quattro che trovai avvinghiati all'armatura."}

Da \emph{Viaggio nel primo mondo.} Romanzo di Federica Angeli

\mostro{Doppelganger}
\noindent
\begin{description}[noitemsep, topsep=0pt, parsep=0pt, partopsep=0pt, leftmargin=0cm, labelwidth=2.2cm]
	\item[\textbf{Taglia/Tipo:}] Media mostruosità (mutaforma), neutrale
	\item[\textbf{Caratt.:}] \resizebox{0.5\linewidth+1.8cm}{!}{For 0 Des 4 Cos 2 Int 0 Sag 1 Car 2}
	\item[\textbf{Punti Ferita:}] 70,  \textbf{Difesa:} 20,  \textbf{Iniziativa:} +4
	\item[\textbf{Movimento:}] 9 m
	\item[\textbf{Tiri Salvez.:}] \resizebox{0.5\linewidth+1.8cm}{!}{Tempra +5, Riflessi +7, Volontà +4}
	\item[\textbf{Comp.:}] Ingannare +6, Percepire Emozioni +3
	\item[\textbf{Immunità:}] affascinato
	\item[\textbf{Sensi:}] Scurovisione 18 m
	\item[\textbf{Linguaggi:}] Comune
	\item[\textbf{Sfida:}] 3 (700 PX)\smallskip
\end{description}

\emph{\textbf{Mutaforma.}} Il doppelganger può usare una Azione per cambiare la propria forma in quella di un umanoide Piccolo o Medio che abbia visto, o per tornare alla sua vera forma. Le sue statistiche, a parte la taglia, sono le stesse in tutte le forme. Qualsiasi equipaggiamento stia indossando o trasportando non viene trasformato. Alla morte ritorna alla sua vera forma.

\emph{\textbf{Appostato.}} Nel primo round di combattimento, il doppelganger ha +1d6 ai tiri di attacco contro qualsiasi creatura abbia preso di sorpresa.

\emph{\textbf{Attacco di Sorpresa.}} Se il doppelganger sorprende una creatura e la colpisce con un attacco durante il primo round di combattimento, il bersaglio subisce 10 (3d6) danni aggiuntivi dall'attacco.

\textbf{Azioni}

\emph{\textbf{Multiattacco.}} Il doppelganger effettua due attacchi da mischia.

\emph{\textbf{Schianto.} Attacco con arma da mischia}: +5 a colpire, portata 1 m, un bersaglio.

\emph{Colpisce:} 7 (1d6 + 4) danni contundenti.

\emph{\textbf{Leggere Pensieri.}} Il doppelganger legge magicamente i pensieri di superficie di una creatura entro 18 metri da lui. L'effetto può penetrare le barriere, ma 1 metro di legno o terra, 50 centimetri di pietra, 5 centimetri di metallo, o un sottile foglio di piombo lo blocca. Mentre il bersaglio è a gittata, il doppelganger può continuare a leggerne i pensieri, purché la concentrazione del doppelganger non venga infranta (come la concentrazione di un incantesimo). Mentre legge la mente di un bersaglio, il doppelganger ha +1d6 alle prove di Saggezza e Carisma contro il bersaglio.

\textbf{Ecologia}\\
Ambiente: Qualsiasi\\
Organizzazione: Solitario, coppia o banda (3-6)\\
\textbf{Categoria Tesoro}: Equipaggiamento da PNG\\
\textbf{Descrizione}\\
I doppelganger sono esseri che possono assumere la forma di chiunque incontrino. Nella loro forma naturale, sembrano umanoidi snelli e fragili, con tratti facciali non del tutto formati e carnagione pallida.

Preferiscono infiltrarsi in società complesse per accumulare ricchezza e potere, usando la loro abilità mimetica per tendere imboscate e trappole. Sebbene non siano necessariamente malvagi, sono egoisti e vedono gli altri come giocattoli da manipolare.

Alcuni doppelganger amano i giochi politici, mentre altri cambiano continuamente razza, sesso e partner amorosi. Sono noti per le loro capacità di cambiare forma e per evitare la cattura. I più potenti possono assumere anche abilità e ricordi delle creature che impersonano.

\medskip


\begin{enfasi}
	Conosci il tuo nemico e conosci te stesso; in cento battaglie non correrai mai pericolo. L'Arte della Guerra, Sun Tzu
\end{enfasi}

\rule{\linewidth}{2pt}

\medskip

\pdfbookmark[3]{Draghi}{Draghi}



I Draghi sono creature temibili, pericolose, antiche; rappresentano il potere stesso.

Ogni Drago ha pieno accesso a tutti gli incantesimi di una specifica lista di magia a seconda del proprio colore.

Questo accesso è garantito da Tàhil o Ljust a seconda che siano draghi fedeli ad uno o all'altro.

Ed è da questa distinzione che i draghi vengono suddivisi tra Draghi di Tàhil e di Ljust. I primi rappresentano a vario titolo e grado Caos, distruzione, violenza e morte, mentre i Draghi di Ljust sono l'emblema del buono, giusto, corretto, protettivo. Mentre i draghi di Tàhil sono solitamente definiti anche cromatici quelli di Ljust sono definiti metallici.

I Draghi di Ljust sono errori di trasporto, magari perché il portale di Tàhil si è aperto mentre un drago malvagio combatteva con un drago buono.

\textbf{Fallire il Tiro Salvezza} contro il soffio di un drago in maniera critica ne raddoppia i danni subiti mentre riuscire in maniera critica non ne dimezza ulteriormente il danno ricevuto.\index{Soffio del drago}

\medskip

\textbf{Draghi e Magia}

\begin{itemize}[leftmargin=*] \setlength{\itemsep}{0pt}
	\item Ogni Drago può lanciare incantesimi sino ad un livello massimo pari ad un quarto del suo Grado si Sfida, con un minimo accesso al primo livello.
	\item Ogni Drago ha un numero di Punti Magia pari 5 volte il suo Grado di Sfida
	\item Ogni Drago ha un punteggio di Competenza Magica pari alla metà del suo Grado di Sfida
\end{itemize}

\medskip

\textbf{Tabella: accesso Lista di Magia per Draghi}\index[Tabelle]{Tabella accesso Lista di Magia per Draghi}

\medskip

\noindent\begin{tabularx}{\linewidth}{ll}
\toprule
 \rowcolor{gray!20}\textbf{Colore} & \textbf{Lista} \\
\toprule
	Bianco & Acqua \\
 \rowcolor{gray!20}Blu & Aria \\
	Giallo & Fuoco, Evocazione \\
 \rowcolor{gray!20}Porpora & Terra \\
	Nero & Acqua, Necromanzia \\
 \rowcolor{gray!20}Rosso & Fuoco \\
	Verde & Animali e Piante \\
 \rowcolor{gray!20}Bronzo & Abiurazione \\
	Ottone & Illusione, Divinazione \\
 \rowcolor{gray!20}Argento & Trasmutazione \\
	Oro & Cura, Evocazione \\
 \rowcolor{gray!20}Rame & Invocazione \\

\end{tabularx}

\medskip

Tutti i Draghi hanno accesso alla lista di magia Universale e prediligono certi incantesimi che sono segnati nella loro descrizione.

\medskip

Nella \textbf{Descrizione di ogni Drago Antico} troverete una breve descrizione del tipo di drago.

\medskip

\textbf{Poteri speciali dei Draghi}\medskip

Ogni Drago a seconda dell'età ha uno o più poteri speciale casuali.
Se è un Drago Cucciolo ha 1 potere casuale, 2 se è Giovane o Adulto e 3 se è Antico. In caso di poteri ripetuti non ripetere il tiro.

\medskip

\noindent\begin{tabularx}{\linewidth}{lX}
	\toprule
 \rowcolor{gray!20}\textbf{3d6}& \textbf{Potere del Drago}\\
\toprule
	3	& Ricarica veloce. Il Drago ricarica il soffio con 4-6. \\
	\end{tabularx}
\noindent\begin{tabularx}{\linewidth}{lX}
 \rowcolor{gray!20}4& Agilità sorprendente. La Difesa del Drago aumenta di un ulteriore +4. \\
	5-7 	& Signore dei Serpenti. La coda ha un pungiglione velenoso che infligge 2xGS PF di danno da veleno. TS Tempra DC 10+GS per dimezzare.\\
 \rowcolor{gray!20}8-10 	& Benedetto di Tàhil. Il Drago ha migliori Tiri Salvezza. +1d6 ad ogni Tiro Salvezza.\\
	11-13 	& Regina Lucertola. Lo sguardo del Drago ha lo stesso effetto di quello del \hyperlink{Basilisco}{Basilisco}.\\
 \rowcolor{gray!20}14-15 	& Potere del Ferro. Il Drago ha \emph{Arrugginire Metallo} come il \hyperlink{Rugginofago}{Rugginofago}.\\
	16 	& Resistenza alla magia. Il Drago è immune agli incantesimi sotto GS/5 livello.\\
 \rowcolor{gray!20}17 	& Immunità aggiuntiva. Il Drago è immune ai danni da una forma di \hyperlink{elencoenergia}{Energia} (pag. \pageref{elencoenergia}) in più a \hyperlink{fontienergia}{caso} (pag. \pageref{fontienergia}).\\
	18 	& Pelle corazzata. Il Drago ha riduzione del danno pari a GS/3 ai danni T/P/C.
\end{tabularx}

\medskip

\begin{enfasi}{
Oh maledetti possa Lynx chiudervi tutti i portali\\
Oh assassini possa Sumkjir sterminarvi\\
Oh devastatori possa Nedraf rompervi le ossa!\\
(imprecazioni popolari contro i Draghi)}\end{enfasi}

\medskip

\rule{\linewidth}{2pt}

\medskip

\textbf{Draghi di Tàhil}

\pdfbookmark[3]{Draghi di Tahil}{Draghi di Tahil}

\mostro{Drago Bianco Antico}
\noindent
\begin{description}[noitemsep, topsep=0pt, parsep=0pt, partopsep=0pt, leftmargin=0cm, labelwidth=2.2cm]
	\item[\textbf{Taglia/Tipo:}] Mastodontica drago, malvagio
	\item[\textbf{Caratt.:}] \resizebox{0.5\linewidth+1.8cm}{!}{For 8 Des 0 Cos 8 Int 3 Sag 1 Car 2}
	\item[\textbf{Punti Ferita:}] 407,  \textbf{Difesa:} 38,  \textbf{Iniziativa:} +3
	\item[\textbf{Movimento:}] 12 m, nuoto 12 m, volo 24 m
	\item[\textbf{Tiri Salvez.:}] \resizebox{0.5\linewidth+1.8cm}{!}{\resizebox{0.5\linewidth+1.8cm}{!}{Tempra +28, Riflessi +20, Volontà +21}}
	\item[\textbf{Comp.:}] Furtività +6, Consapevolezza +13
	\item[\textbf{Imm. Danni:}] Freddo, armi +1
	\item[\textbf{Sensi:}] Scurovisione 36 m, Vista Cieca 18 m
	\item[\textbf{Linguaggi:}] Comune, Draconico
	\item[\textbf{Sfida:}] 20 (25000 PX)\smallskip
\end{description}

\emph{\textbf{Aura di gelo.}} il drago emette nel raggio di 3 metri un gelo magico che causa 2d6 danni da freddo a round.

\emph{\textbf{Camminare sul Ghiaccio.}} Il drago può muoversi e arrampicarsi su superfici ghiacciate senza bisogno di effettuare prove su competenze di base. Inoltre, il terreno difficile composto di ghiaccio o neve non gli costa movimento aggiuntivo.

\emph{\textbf{Resistenza Leggendaria (3/Giorno).}} Se il drago fallisce un Tiro Salvezza, può scegliere invece di riuscire.

\emph{\textbf{Resistenza alla Magia:}} 3lv

\textbf{Azioni}

\emph{\textbf{Multiattacco.}} Il drago può usare la sua Presenza Spaventosa e poi effettuare tre attacchi: uno con il morso e due con gli artigli.

\emph{\textbf{Artiglio.} Attacco con arma da mischia}: +16 a colpire, portata 3 m, un bersaglio.

\emph{Colpisce:} 15 (2d6 + 8) danni taglienti, 3/20 danni da Sanguinamento.

\emph{\textbf{Coda.} Attacco con arma da mischia}: +16 a colpire, portata 6 m, un bersaglio.

\emph{Colpisce:} 17 (2d8 + 8) danni contundenti.

\emph{\textbf{Morso.} Attacco con arma da mischia}: +16 a colpire, portata 5 metri, un bersaglio.

\emph{Colpisce:} 19 (2d10 + 8) danni perforanti più 9 (2d8) danni da freddo.

\emph{\textbf{Presenza Spaventosa.}} Ogni creatura scelta dal drago, che si trovi entro 36 metri da esso e consapevole della sua presenza, deve riuscire un Tiro Salvezza di Volontà DC 35 o restare spaventata per 1 minuto. Una creatura può ripetere il Tiro Salvezza al termine di ciascun suo round, terminando l'effetto se lo riesce. Se il Tiro Salvezza della creatura ha successo o l'effetto ha termine per essa, la creatura è immune alla Presenza Spaventosa del drago per le successive 24 ore.

\emph{\textbf{Soffio Gelido (Ricarica 5-6).}} Il drago esala un'esplosione di ghiaccio in un cono di 27 metri. Ogni creatura in quell'area deve effettuare un Tiro Salvezza di Tempra DC 35 e subire 72 (16d8) danni da freddo se fallisce il Tiro Salvezza, o la metà di questi danni se lo riesce.

\textbf{Azioni Aggiuntive}

Il drago può effettuare 3 Azioni aggiuntive, scelte tra le opzioni seguenti. Può usare solo un'opzione Aggiuntive alla volta e solo al termine del round di un'altra creatura. Il drago recupera le Azioni aggiuntive spese all'inizio del proprio round.

\textbf{Attacco di Ala (Costa 2 Azioni).} Il drago batte le ali. Ogni creatura entro 5 metri dal drago deve riuscire un Tiro Salvezza su Riflessi DC 33 o subire 15 (2d6 + 8) danni contundenti e venir gettato prono. Il drago può poi volare fino a metà del suo movimento di volo.

\textbf{Attacco di Coda.} Il drago effettua un attacco di coda.

\textbf{Individuare.} Il drago effettua una prova Consapevolezza.

\textbf{Ecologia}\\
Ambiente: Montagne Fredde\\
Organizzazione: Solitario\\
\textbf{Categoria Tesoro}: H\\
\textbf{Descrizione}\\
I Draghi Bianchi sono tra i più selvaggi e animali di tutti i draghi.
Amano i posti freddi e ghiacciati, trovando rifugio nelle valli più fredde come i picchi ghiacciati delle montagne e le steppe gelide.

I Draghi Bianchi hanno un aspetto selvaggio quasi sempre mostrano i denti e gli artigli sono estratti per muoversi agilmente sul terreno ghiacciato.
Non hanno penalità di movimento su questi terreni.

Sfruttano il loro naturale camuffamento per aggredire e catturare le prede, sono ottimi cacciatori, molto intelligenti nello sfruttare l'ambiente.

Poco inclini alla magia sanno però soffiare schegge di ghiaccio molto più frequentemente di altri draghi. E' immune gli attacchi basati sul freddo e ghiaccio.

Le loro tane sono caverne ghiacciate nelle montagne o scavate nei ghiacciai più massicci.

I Draghi Bianchi hanno +1d6 nelle prove di magia e possono ignorare un dado tirato nella prova con la Lista dell'Acqua ed è immune al freddo.\\

\textbf{Incantesimi}\index{Incantesimi da Drago Bianco}\\
Gli incantesimi preferiti di questo Drago sono:\\
- \hyperlink{Scudo di Fuoco}{Scudo di Fuoco}\\
- \hyperlink{Tempesta di Ghiaccio}{Tempesta di Ghiaccio}\\
- \hyperlink{Tempesta di Nevischio}{Tempesta di Nevischio}

\mostro{Drago Bianco Adulto}
\noindent
\begin{description}[noitemsep, topsep=0pt, parsep=0pt, partopsep=0pt, leftmargin=0cm, labelwidth=2.2cm]
	\item[\textbf{Taglia/Tipo:}] Enorme drago, malvagio
	\item[\textbf{Caratt.:}] \resizebox{0.5\linewidth+1.8cm}{!}{For 6 Des 0 Cos 6 Int 2 Sag 1 Car 1}
	\item[\textbf{Punti Ferita:}] 264,  \textbf{Difesa:} 29,  \textbf{Iniziativa:} +2
	\item[\textbf{Movimento:}] 12 m, nuoto 12 m, volo 24 m
	\item[\textbf{Tiri Salvez.:}] \resizebox{0.5\linewidth+1.8cm}{!}{\resizebox{0.5\linewidth+1.8cm}{!}{Tempra +19, Riflessi +13, Volontà +14}}
	\item[\textbf{Comp.:}] Furtività +5, Consapevolezza +8
	\item[\textbf{Imm. Danni:}] Freddo
	\item[\textbf{Sensi:}] Scurovisione 36 m, Vista Cieca 18 m
	\item[\textbf{Linguaggi:}] Comune, Draconico
	\item[\textbf{Sfida:}] 13 (10000 PX)\smallskip
\end{description}

\emph{\textbf{Aura di gelo.}} il drago emette nel raggio di 3 metri un gelo magico che causa 1d6 danni da freddo a round.

\emph{\textbf{Camminare sul Ghiaccio.}} Il drago può muoversi e arrampicarsi su superfici ghiacciate senza bisogno di effettuare prove su competenze di base. Inoltre, il terreno difficile composto di ghiaccio o neve non gli costa movimento aggiuntivo.

\emph{\textbf{Resistenza Leggendaria (3/Giorno).}} Se il drago fallisce un Tiro Salvezza, può scegliere invece di riuscire.

\textbf{Azioni}

\emph{\textbf{Multiattacco.}} Il drago può usare la sua Presenza Spaventosa e poi effettuare tre attacchi: uno con il morso e due con gli artigli.

\emph{\textbf{Artiglio.} Attacco con arma da mischia}: +12 a colpire, portata 1 m, un bersaglio, 1 danno da Sanguinamento.

\emph{Colpisce:} 13 (2d6 + 6) danni taglienti.

\emph{\textbf{Coda.} Attacco con arma da mischia}: +12 a colpire, portata 5 metri, un bersaglio.

\emph{Colpisce:} 15 (2d8 + 6) danni contundenti.

\emph{\textbf{Morso.} Attacco con arma da mischia}: +12 a colpire, portata 3 m, un bersaglio.

\emph{Colpisce:} 17 (2d10 + 6) danni perforanti più 4 (1d8) danni da freddo.

\emph{\textbf{Presenza Spaventosa.}} Ogni creatura scelta dal drago, che si trovi entro 36 metri da esso e consapevole della sua presenza, deve riuscire un Tiro Salvezza di Volontà DC 27 o restare spaventata per 1 minuto. Una creatura può ripetere il Tiro Salvezza al termine di ciascun suo round, terminando l'effetto se lo riesce. Se il Tiro Salvezza della creatura ha successo o l'effetto ha termine per essa, la creatura è immune alla Presenza Spaventosa del drago per le successive 24 ore.

\emph{\textbf{Soffio Gelido (Ricarica 5-6).}} Il drago esala un'esplosione di ghiaccio in un cono di 18 metri. Ogni creatura in quell'area deve effettuare un Tiro Salvezza di Tempra DC 27 e subire 54 (12d8) danni da freddo se fallisce il Tiro Salvezza, o la metà di questi danni se lo riesce.

\textbf{Azioni Aggiuntive}

Il drago può effettuare 3 Azioni aggiuntive, scelte tra le opzioni seguenti. Può usare solo un'opzione Aggiuntiva alla volta e solo al termine del round di un'altra creatura. Il drago recupera le Azioni aggiuntive spese all'inizio del proprio round.

\textbf{Attacco di Ala (Costa 2 Azioni).} Il drago batte le ali. Ogni creatura entro 3 metri dal drago deve riuscire un Tiro Salvezza su Riflessi DC 27 o subire 13 (2d6 + 6) danni contundenti e venir gettato prono. Il drago può poi volare fino a metà del suo movimento di volo. \textbf{Attacco di Coda.} Il drago effettua un attacco di coda
.
\textbf{Individuare.} Il drago effettua una prova di Consapevolezza.

\emph{\textbf{Arrabbiato:}} Il Drago Bianco Adulto può eseguire queste azioni a costo 2 Azioni.

\emph{Focalizzare}: la creatura interrompe un effetto mentale su di se in corso

\emph{Brutalità}: la creatura attacca con ferocia inaudita. +1d6 al Tiro per Colpire, 1 danno critico automatico quando colpisce.

\textbf{Ecologia}\\
Ambiente: Montagne Fredde\\
Organizzazione: Solitario\\
\textbf{Categoria Tesoro}: E\\
\textbf{Descrizione}\\
Vedi descrizione Drago Bianco Antico.\\
\textbf{Incantesimi}\index{Incantesimi da Drago Bianco}\\
Gli incantesimi preferiti di questo Drago sono:\\
- \hyperlink{Scudo di Fuoco}{Scudo di Fuoco}\\
- \hyperlink{Tempesta di Ghiaccio}{Tempesta di Ghiaccio}\\
- \hyperlink{Tempesta di Nevischio}{Tempesta di Nevischio}

\mostro{Drago Bianco Giovane}
\noindent
\begin{description}[noitemsep, topsep=0pt, parsep=0pt, partopsep=0pt, leftmargin=0cm, labelwidth=2.2cm]
	\item[\textbf{Taglia/Tipo:}] Grande drago, malvagio
	\item[\textbf{Caratt.:}] \resizebox{0.5\linewidth+1.8cm}{!}{For 4 Des 0 Cos 4 Int -2 Sag 0 Car 1}
	\item[\textbf{Punti Ferita:}] 127,  \textbf{Difesa:} 20,  \textbf{Iniziativa:} +0
	\item[\textbf{Movimento:}] 12 m, nuoto 12 m, volo 24 m
	\item[\textbf{Tiri Salvez.:}] \resizebox{0.5\linewidth+1.8cm}{!}{Tempra +10, Riflessi +6, Volontà +6}
	\item[\textbf{Comp.:}] Furtività +3, Consapevolezza +6
	\item[\textbf{Imm. Danni:}] Freddo
	\item[\textbf{Sensi:}] Scurovisione 36 m, Vista Cieca 9 m
	\item[\textbf{Linguaggi:}] Comune, Draconico
	\item[\textbf{Sfida:}] 6 (2300 PX)\smallskip
\end{description}

\emph{\textbf{Camminare sul Ghiaccio.}} Il drago può muoversi e arrampicarsi su superfici ghiacciate senza bisogno di effettuare prove su competenze di base. Inoltre, il terreno difficile composto di ghiaccio o neve non gli costa movimento aggiuntivo.

\textbf{Azioni}

\emph{\textbf{Multiattacco.}} Il drago può usare la sua Presenza Spaventosa e poi effettuare tre attacchi: uno con il morso e due con gli artigli.

\emph{\textbf{Artiglio.} Attacco con arma da mischia}: +9 a colpire, portata 1 m, un bersaglio.

\emph{Colpisce:} 11 (2d6 + 4) danni taglienti, 1 danno da Sanguinamento.

\emph{\textbf{Morso.} Attacco con arma da mischia}: +9 a colpire, portata 3 m, un bersaglio.

\emph{Colpisce:} 15 (2d10 + 4) danni perforanti più 4 (1d8) danni da freddo.

\emph{\textbf{Soffio Gelido (Ricarica 5-6).}} Il drago esala un'esplosione di ghiaccio in un cono di 9 metri. Ogni creatura in quell'area deve effettuare un Tiro Salvezza di Tempra DC 18 e subire 45 (10d8) danni da freddo se fallisce il Tiro Salvezza, o la metà di questi danni se lo riesce.

\textbf{Ecologia}\\
Ambiente: Montagne Fredde\\
Organizzazione: Solitario\\
\textbf{Categoria Tesoro}: D\\
\textbf{Descrizione}\\
Vedi descrizione Drago Bianco Antico.\\
\textbf{Incantesimi}\index{Incantesimi da Drago Bianco}\\
Gli incantesimi preferiti di questo Drago sono:\\
- \hyperlink{Scudo di Fuoco}{Scudo di Fuoco}\\
- \hyperlink{Tempesta di Ghiaccio}{Tempesta di Ghiaccio}\\
- \hyperlink{Tempesta di Nevischio}{Tempesta di Nevischio}

\mostro{Drago Bianco Cucciolo}
\noindent
\begin{description}[noitemsep, topsep=0pt, parsep=0pt, partopsep=0pt, leftmargin=0cm, labelwidth=2.2cm]
	\item[\textbf{Taglia/Tipo:}] Media drago, malvagio
	\item[\textbf{Caratt.:}] \resizebox{0.5\linewidth+1.8cm}{!}{For 2 Des 0 Cos 2 Int -3 Sag 0 Car 0}
	\item[\textbf{Punti Ferita:}] 51,  \textbf{Difesa:} 14,  \textbf{Iniziativa:} +0
	\item[\textbf{Movimento:}] 9 m, nuoto 9 m, volo 18 m
	\item[\textbf{Tiri Salvez.:}] \resizebox{0.5\linewidth+1.8cm}{!}{Tempra +4, Riflessi +3, Volontà +3}
	\item[\textbf{Comp.:}] Furtività +2, Consapevolezza +4
	\item[\textbf{Imm. Danni:}] Freddo
	\item[\textbf{Sensi:}] Scurovisione 18 m, Vista Cieca 3 m
	\item[\textbf{Linguaggi:}] Draconico
	\item[\textbf{Sfida:}] 2 (450 PX)\smallskip
\end{description}

\textbf{Azioni}

\emph{\textbf{Morso.} Attacco con arma da mischia}: +5 a colpire, portata 3 m, un bersaglio.

\emph{Colpisce:} 15 (2d10 + 4) danni perforanti più 4 (1d8) danni da freddo.

\emph{\textbf{Soffio Gelido (Ricarica 5-6).}} Il drago esala un'esplosione di ghiaccio in un cono di 5 metri. Ogni creatura in quell'area deve effettuare un Tiro Salvezza di Tempra DC 15 e subire 22 (5d8) danni da freddo se fallisce il Tiro Salvezza, o la metà di questi danni se lo riesce.

\textbf{Ecologia}\\
Ambiente: Montagne Fredde\\
Organizzazione: Solitario\\
\textbf{Categoria Tesoro}: C\\
\textbf{Descrizione}\\
Vedi descrizione Drago Bianco Antico.

\mostro{Drago Blu Antico}
\noindent
\begin{description}[noitemsep, topsep=0pt, parsep=0pt, partopsep=0pt, leftmargin=0cm, labelwidth=2.2cm]
	\item[\textbf{Taglia/Tipo:}] Mastodontica drago, malvagio
	\item[\textbf{Caratt.:}] \resizebox{0.5\linewidth+1.8cm}{!}{For 9 Des 0 Cos 8 Int 4 Sag 3 Car 5}
	\item[\textbf{Punti Ferita:}] 465,  \textbf{Difesa:} 42,  \textbf{Iniziativa:} +4
	\item[\textbf{Movimento:}] 12 m, scavo 12 m, volo 24 m
	\item[\textbf{Tiri Salvez.:}] \resizebox{0.5\linewidth+1.8cm}{!}{\resizebox{0.5\linewidth+1.8cm}{!}{Tempra +31, Riflessi +23, Volontà +26}}
	\item[\textbf{Comp.:}] Furtività +7, Consapevolezza +17
	\item[\textbf{Imm. Danni:}] Elettricità, armi +1
	\item[\textbf{Sensi:}] Scurovisione 36 m, Vista Cieca 18 m
	\item[\textbf{Linguaggi:}] Comune, Draconico
	\item[\textbf{Sfida:}] 23 (50000 PX)\smallskip
\end{description}

\emph{\textbf{Scarica elettrica.}} il drago emette nel raggio di 3 metri scariche elettriche magiche che causano 2d6 danni da elettricità a round.

\emph{\textbf{Resistenza Leggendaria (3/Giorno).}} Se il drago fallisce un Tiro Salvezza, può scegliere invece di riuscire.

\emph{\textbf{Resistenza alla Magia:}} 3lv

\textbf{Azioni}

\emph{\textbf{Multiattacco.}} Il drago può usare la sua Presenza Spaventosa e poi effettuare tre attacchi: uno con il morso e due con gli artigli.

\emph{\textbf{Artiglio.} Attacco con arma da mischia}: +16 a colpire,
portata 3 m, un bersaglio.

\emph{Colpisce:} 16 (2d6 + 9) danni taglienti, 3/20 danno da Sanguinamento.

\emph{\textbf{Coda.} Attacco con arma da mischia}: +16 a colpire, portata 6 m, un bersaglio.

\emph{Colpisce:} 18 (2d8 + 9) danni contundenti.

\emph{\textbf{Morso.} Attacco con arma da mischia}: +16 a colpire, portata 5 metri, un bersaglio.

\emph{Colpisce:} 20 (2d10 + 9) danni perforanti più 11 (2d10) danni da elettricità.

\emph{\textbf{Presenza Spaventosa.}} Ogni creatura scelta dal drago, che si trovi entro 36 metri da esso e consapevole della sua presenza, deve riuscire un Tiro Salvezza di Volontà DC 35 o restare spaventata per 1 minuto. Una creatura può ripetere il Tiro Salvezza al termine di ciascun suo round, terminando l'effetto se lo riesce. Se il Tiro Salvezza della creatura ha successo o l'effetto ha termine per essa, la creatura è immune alla Presenza Spaventosa del drago per le successive 24 ore.

\emph{\textbf{Soffio Fulminante (Ricarica 5-6).}} Il drago esala fulmini in una linea lunga 36 metri e larga 3 metri. Ogni creatura su quella linea deve effettuare un Tiro Salvezza di Riflessi DC 35 e subire 88 (16d10) danni da elettricità se fallisce il Tiro Salvezza, o la metà di questi danni se lo riesce.

\textbf{Azioni Aggiuntive}

Il drago può effettuare 3 Azioni aggiuntive, scelte tra le opzioni seguenti. Può usare solo un'opzione Aggiuntiva alla volta e solo al termine del round di un'altra creatura. Il drago recupera le Azioni aggiuntive spese all'inizio del proprio round.

\textbf{Attacco di Ala (Costa 2 Azioni).} Il drago batte le ali. Ogni creatura entro 5 metri dal drago deve riuscire un Tiro Salvezza su Riflessi DC 35 o subire 16 (2d6 + 9) danni contundenti e venir gettato prono. Il drago può poi volare fino a metà del suo movimento di volo.

\textbf{Attacco di Coda.} Il drago effettua un attacco di coda.

\textbf{Individuare.} Il drago effettua una prova di Consapevolezza.\\
\textbf{Ecologia}\\
Ambiente: Picchi montuosi\\
Organizzazione: Solitario\\
\textbf{Categoria Tesoro}: H\\
\textbf{Descrizione}\\
I Draghi Blu abitano tra le nuvole, volando (e levitando) tra le tempeste.

I Draghi Blu hanno un aspetto serpentiforme, allungato ed legante, con corna lunghe all'indietro.

La faccia di un Drago Blu è meno segnata da increspature e rimane liscia.
Sono gli unici draghi a non avere ali pur volando meglio di ogni altro drago.

La loro magica ma naturale capacità di volo unita al fatto di nutrirsi di elettricità ne fa creature prettamente volanti che quasi mai scendono a terra (e mai toccano terra considerandola impura e sporca!), preferiscono rimanere tra le nubi, specialmente tra quelle più scure e cariche di energia per nutrirsi

La tana del Drago Blu solitamente è tra i picchi più alti delle montagne possibilmente tanto alte da arrivare alle nubi. Questa non è mai coperta e spesso assomiglia a giganteschi nidi.

I Draghi Blu possono assimilare carne ma non vegetali, non traggono nutrimenti da ciò che mangiano avendo un metabolismo puramente elettrico.

Sono draghi sociali, che amano stare con i loro simili e sono molto protettivi con la loro prole.
Solitamente non si trova mai un nido da solo, ma interi altopiani dominati da decine di draghi.

Non vanno d'accordo con i draghi viola che disprezzano per la scelta di aver rinunciato al volo per vivere sottoterra.

I Draghi Blu hanno +1d6 nelle prove di magia e possono ignorare un dado tirato nella prova con la Lista dell'Aria ed è immune all'elettricità.\\
\textbf{Incantesimi}\index{Incantesimi da Drago Blu}\\
Gli incantesimi preferiti di questo Drago sono:\\
- \hyperlink{Nebbia mortale}{Nebbia mortale}\\
- \hyperlink{Invocare il Fulmine}{Invocare il Fulmine}\\
- \hyperlink{Tempesta di Ghiaccio}{Tempesta di Ghiaccio}

\mostro{Drago Blu Adulto}
\noindent
\begin{description}[noitemsep, topsep=0pt, parsep=0pt, partopsep=0pt, leftmargin=0cm, labelwidth=2.2cm]
	\item[\textbf{Taglia/Tipo:}] Enorme drago, malvagio
	\item[\textbf{Caratt.:}] \resizebox{0.5\linewidth+1.8cm}{!}{For 7 Des 0 Cos 6 Int 3 Sag 2 Car 4}
	\item[\textbf{Punti Ferita:}] 322,  \textbf{Difesa:} 33,  \textbf{Iniziativa:} +3
	\item[\textbf{Movimento:}] 12 m, scavo 12 m, volo 24 m
	\item[\textbf{Tiri Salvez.:}] \resizebox{0.5\linewidth+1.8cm}{!}{\resizebox{0.5\linewidth+1.8cm}{!}{Tempra +22, Riflessi +16, Volontà +18}}
	\item[\textbf{Comp.:}] Furtività +5, Consapevolezza +13
	\item[\textbf{Imm. Danni:}] Elettricità
	\item[\textbf{Sensi:}] Scurovisione 36 m, Vista Cieca 18 m
	\item[\textbf{Linguaggi:}] Comune, Draconico
	\item[\textbf{Sfida:}] 16 (15000 PX)\smallskip
\end{description}

\emph{\textbf{Scarica elettrica.}} il drago emette nel raggio di 3 metri scariche elettriche magiche che causano 1d6 danni da elettricità a round.

\emph{\textbf{Resistenza Leggendaria (3/Giorno).}} Se il drago fallisce un Tiro Salvezza, può scegliere invece di riuscire.

\textbf{Azioni}

\emph{\textbf{Multiattacco.}} Il drago può usare la sua Presenza Spaventosa e poi effettuare tre attacchi: uno con il morso e due con gli artigli.

\emph{\textbf{Artiglio.} Attacco con arma da mischia}: +14 a colpire, portata 1 m, un bersaglio.

\emph{Colpisce:} 14 (2d6 + 7) danni taglienti, 1 danno da Sanguinamento.

\emph{\textbf{Coda.} Attacco con arma da mischia}: +14 a colpire, portata 5 metri, un bersaglio.

\emph{Colpisce:} 16 (2d8 + 7) danni contundenti.

\emph{\textbf{Morso.} Attacco con arma da mischia}: +14 a colpire, portata 3 m, un bersaglio.

\emph{Colpisce:} 18 (2d10 + 7) danni perforanti più 5 (1d10) danni da elettricità.

\emph{\textbf{Presenza Spaventosa.}} Ogni creatura scelta dal drago, che si trovi entro 36 metri da esso e consapevole della sua presenza, deve riuscire un Tiro Salvezza di Volontà DC 30 o restare spaventata per 1 minuto. Una creatura può ripetere il Tiro Salvezza al termine di ciascun suo round, terminando l'effetto se lo riesce. Se il Tiro Salvezza della creatura ha successo o l'effetto ha termine per essa, la creatura è immune alla Presenza Spaventosa del drago per le successive 24 ore.

\emph{\textbf{Soffio Fulminante (Ricarica 5-6).}} Il drago esala fulmini in una linea lunga 27 metri e larga 1 metro. Ogni creatura su quella linea deve effettuare un Tiro Salvezza di Riflessi DC 30 e subire 66 (12d10) danni da elettricità se fallisce il Tiro Salvezza, o la metà di questi danni se lo riesce.

\textbf{Azioni Aggiuntive}

Il drago può effettuare 3 Azioni aggiuntive, scelte tra le opzioni seguenti. Può usare solo un'opzione Aggiuntiva alla volta e solo al termine del round di un'altra creatura. Il drago recupera le Azioni aggiuntive spese all'inizio del proprio round.

\textbf{Attacco di Ala (Costa 2 Azioni).} Il drago batte le ali. Ogni creatura entro 3 metri dal drago deve riuscire un Tiro Salvezza su Riflessi DC 30 o subire 14 (2d6 + 7) danni contundenti e venir gettato prono. Il drago può poi volare fino a metà della del suo movimento di volo.

\textbf{Attacco di Coda.} Il drago effettua un attacco di coda.

\textbf{Individuare.} Il drago effettua una prova di Consapevolezza.

\emph{\textbf{Arrabbiato:}} Il Drago Blu Adulto può eseguire queste azioni a costo 2 Azioni.

\emph{Focalizzare}: la creatura interrompe un effetto mentale su di se in corso

\emph{Brutalità}: la creatura attacca con ferocia inaudita. +1d6 al Tiro per Colpire, 1 danno critico automatico quando colpisce.

\textbf{Ecologia}\\
Ambiente: Picchi montuosi\\
Organizzazione: Solitario\\
\textbf{Categoria Tesoro}: E\\
\textbf{Descrizione}\\
Vedi Descrizione Drago Blu Antico.\\
\textbf{Incantesimi}\index{Incantesimi da Drago Blu}\\
Gli incantesimi preferiti di questo Drago sono:\\
- \hyperlink{Nebbia mortale}{Nebbia mortale}\\
- \hyperlink{Invocare il Fulmine}{Invocare il Fulmine}\\
- \hyperlink{Tempesta di Ghiaccio}{Tempesta di Ghiaccio}

\mostro{Drago Blu Giovane}
\noindent
\begin{description}[noitemsep, topsep=0pt, parsep=0pt, partopsep=0pt, leftmargin=0cm, labelwidth=2.2cm]
	\item[\textbf{Taglia/Tipo:}] Enorme drago, malvagio
	\item[\textbf{Caratt.:}] \resizebox{0.5\linewidth+1.8cm}{!}{For 5 Des 0 Cos 4 Int 2 Sag 1 Car 3}
	\item[\textbf{Punti Ferita:}] 184,  \textbf{Difesa:} 24,  \textbf{Iniziativa:} +2
	\item[\textbf{Movimento:}] 12 m, scavo 12 m, volo 24 m
	\item[\textbf{Tiri Salvez.:}] \resizebox{0.5\linewidth+1.8cm}{!}{\resizebox{0.5\linewidth+1.8cm}{!}{Tempra +13, Riflessi +9, Volontà +10}}
	\item[\textbf{Comp.:}] Furtività +4, Consapevolezza +9
	\item[\textbf{Imm. Danni:}] Elettricità
	\item[\textbf{Sensi:}] Scurovisione 36 m, Vista Cieca 9 m
	\item[\textbf{Linguaggi:}] Comune, Draconico
	\item[\textbf{Sfida:}] 9 (5000 PX)\smallskip
\end{description}

\textbf{Azioni}

\emph{\textbf{Multiattacco.}} Il drago può effettuare tre attacchi: uno con il morso e due con gli artigli.

\emph{\textbf{Artiglio.} Attacco con arma da mischia}: +12 a colpire, portata 1 m, un bersaglio.

\emph{Colpisce:} 12 (2d6 + 5) danni taglienti, 1 danno da Sanguinamento.

\emph{\textbf{Morso.} Attacco con arma da mischia}: +12 a colpire, portata 3 m, un bersaglio.

\emph{Colpisce:} 16 (2d10 + 5) danni perforanti più 5 (1d10) danni da elettricità.

\emph{\textbf{Soffio Fulminante (Ricarica 5-6).}} Il drago esala fulmini in una linea lunga 18 metri e larga 1 metro. Ogni creatura su quella linea deve effettuare un Tiro Salvezza di Riflessi DC 21 e subire 55 (10d10) danni da elettricità se fallisce il Tiro Salvezza, o la metà di questi danni se lo riesce.

\emph{\textbf{Arrabbiato:}} il Drago Blu Giovane ricarica il suo soffio fulminante.

\textbf{Ecologia}\\
Ambiente: Picchi montuosi\\
Organizzazione: Solitario\\
\textbf{Categoria Tesoro}: D\\
\textbf{Descrizione}\\
Vedi Descrizione Drago Blu Antico.\\
\textbf{Incantesimi}\index{Incantesimi da Drago Blu}\\
Gli incantesimi preferiti di questo Drago sono:\\
- \hyperlink{Nebbia mortale}{Nebbia mortale}\\
- \hyperlink{Invocare il Fulmine}{Invocare il Fulmine}\\
- \hyperlink{Tempesta di Ghiaccio}{Tempesta di Ghiaccio}

\mostro{Drago Blu Cucciolo}
\noindent
\begin{description}[noitemsep, topsep=0pt, parsep=0pt, partopsep=0pt, leftmargin=0cm, labelwidth=2.2cm]
	\item[\textbf{Taglia/Tipo:}] Enorme drago, malvagio
	\item[\textbf{Caratt.:}] \resizebox{0.5\linewidth+1.8cm}{!}{For 3 Des 0 Cos 2 Int 1 Sag 0 Car 2}
	\item[\textbf{Punti Ferita:}] 70,  \textbf{Difesa:} 16,  \textbf{Iniziativa:} +1
	\item[\textbf{Movimento:}] 9 m, scavo 5 metri, volo 18 m
	\item[\textbf{Tiri Salvez.:}] \resizebox{0.5\linewidth+1.8cm}{!}{Tempra +5, Riflessi +3, Volontà +3}
	\item[\textbf{Comp.:}] Furtività +2, Consapevolezza +4
	\item[\textbf{Imm. Danni:}] Elettricità
	\item[\textbf{Sensi:}] Scurovisione 18 m, Vista Cieca 3 m
	\item[\textbf{Linguaggi:}] Draconico
	\item[\textbf{Sfida:}] 3 (700 PX)\smallskip
\end{description}

\textbf{Azioni}

\emph{\textbf{Morso.} Attacco con arma da mischia}: +5 a colpire, portata 1 m, un bersaglio.

\emph{Colpisce:} 8 (1d10 + 3) danni perforanti più 3 (1d6) danni da elettricità.

\emph{\textbf{Soffio Fulminante (Ricarica 5-6).}} Il drago esala fulmini in una linea lunga 9 metri e larga 1 metro. Ogni creatura su quella linea deve effettuare un Tiro Salvezza di Riflessi DC 14 e subire 22 (4d10) danni da elettricità se fallisce il Tiro Salvezza, o la metà di questi danni se lo riesce.

\textbf{Ecologia}\\
Ambiente: Picchi montuosi\\
Organizzazione: Solitario\\
\textbf{Categoria Tesoro}: C\\
\textbf{Descrizione}\\
Vedi Descrizione Drago Blu Antico.

\mostro{Drago Giallo Antico}
\noindent
\begin{description}[noitemsep, topsep=0pt, parsep=0pt, partopsep=0pt, leftmargin=0cm, labelwidth=2.2cm]
	\item[\textbf{Taglia/Tipo:}] Mastodontica drago, malvagio
	\item[\textbf{Caratt.:}] \resizebox{0.5\linewidth+1.8cm}{!}{For 10 Des 1 Cos 8 Int 3 Sag 2 Car 4}
	\item[\textbf{Punti Ferita:}] 465,  \textbf{Difesa:} 43,  \textbf{Iniziativa:} +3
	\item[\textbf{Movimento:}] 12 m, scavo 24 m, scalata 24, volo 12 m
	\item[\textbf{Tiri Salvez.:}] \resizebox{0.5\linewidth+1.8cm}{!}{\resizebox{0.5\linewidth+1.8cm}{!}{Tempra +31, Riflessi +24, Volontà +25}}
	\item[\textbf{Comp.:}] Furtività +7, Consapevolezza +17
	\item[\textbf{Imm. Danni:}] Fuoco, armi +1
	\item[\textbf{Sensi:}] Scurovisione 36 m, Vista Cieca 18 m
	\item[\textbf{Linguaggi:}] Comune, Draconico
	\item[\textbf{Sfida:}] 23 (50000 PX)\smallskip
\end{description}

\emph{\textbf{Calore rovente.}} il drago emette nel raggio di 3 metri calore magico che causa 2d6 danni da fuoco a round.

\emph{\textbf{Resistenza Leggendaria (3/Giorno).}} Se il drago fallisce un Tiro Salvezza, può scegliere invece di riuscire. \\
\emph{\textbf{Resistenza alla Magia:}} 3lv\\
\smallskip\textbf{Azioni}\\
\emph{\textbf{Multiattacco.}} Il drago può usare la sua Presenza Spaventosa e poi effettuare tre attacchi: uno con il morso e due con gli artigli.\\
\emph{\textbf{Artiglio.} Attacco con arma da mischia}: +16 al colpire, portata 3 m, un bersaglio.\\
\emph{Colpisce:} 16 (2d6 + 9) danni taglienti, 3/20 danno da Sanguinamento.\\
\emph{\textbf{Coda.} Attacco con arma da mischia}: +16 al colpire, portata 6 m, un bersaglio.\\
\emph{Colpisce:} 18 (2d8 + 9) danni contundenti.\\
\emph{\textbf{Morso.} Attacco con arma da mischia}: +16 al colpire, portata 5 metri, un bersaglio.\\
\emph{Colpisce:} 20 (2d10 + 9) danni perforanti più 11 (2d10) danni da elettricità.\\
\emph{\textbf{Presenza Spaventosa.}} Ogni creatura scelta dal drago, che si trovi entro 36 metri da esso e consapevole della sua presenza, deve riuscire un Tiro Salvezza su Volontà DC 35 o restare spaventata per 1 minuto. Una creatura può ripetere il Tiro Salvezza al termine di ciascun suo round, terminando l'effetto se lo riesce. Se il Tiro Salvezza della creatura ha successo o l'effetto ha termine per essa, la creatura è immune alla Presenza Spaventosa del drago per le successive 24 ore.\\
\emph{\textbf{Soffio Incendiario (Ricarica 5-6).}} Il drago esala aria rovente in una linea lunga 36 metri e larga 3 metri. Ogni creatura su quella linea deve effettuare un Tiro Salvezza su Riflessi DC 35 e subire 88 (16d10) danni da fuoco se fallisce il Tiro Salvezza, o la metà di questi danni se lo riesce.\\
\textbf{Azioni Aggiuntive}\\
Il drago può effettuare 3 azioni aggiuntive, scelte tra le opzioni seguenti. Può usare solo un'Azione Aggiuntiva alla volta e solo al termine del round di un'altra creatura. Il drago recupera le Azioni Aggiuntive spese all'inizio del proprio round.\\
\textbf{Attacco di Ala (Costa 2 Azioni).} Il drago batte le ali. Ogni creatura entro 5 metri dal drago deve riuscire un Tiro Salvezza su Riflessi DC 35 o subire 16 (2d6 + 9) danni contundenti e venir gettato prono. Il drago può poi volare fino a metà della sua velocità di volo.\\
\textbf{Attacco di Coda.} Il drago effettua un attacco di coda.\\
\textbf{Individuare.} Il drago effettua una prova di Consapevolezza.\\
\textbf{Ecologia}\\
Ambiente: Deserti Caldi\\
Organizzazione: Solitario\\
\textbf{Categoria Tesoro}: H\\
\textbf{Descrizione}\\
I Draghi Gialli hanno scaglie di vari toni di giallo che con la crescita prendono ad assomigliare sempre di più al colore delle sabbie dove dimorano, dal giallo chiaro all'ocra mattone.

Sono molto intelligenti ma essendo per natura solitari non hanno interesse a comunicare con le altre razze.

Vivono nei deserti dove spesso tendono agguati alle loro prede nascondendosi sul fondo di ampie buche scavate nella sabbia.
Appena percepiscono un movimento sopra di loro escono e divorano qualunque creatura.
Hanno una passione per la carne dei nani che trovano saporita anche se asciutta.

Il Drago Giallo pur se intelligente è una macchina di morte e difficilmente scende a patti, solo se si trova in serio pericolo.

Un Drago Giallo hanno +1d6 nelle prove di magia e possono ignorare un dado tirato nella prova con la Lista del Fuoco ed è immune al fuoco.
\\
\textbf{Incantesimi}\index{Incantesimi da Drago Giallo}\\
Gli incantesimi preferiti di questo Drago sono:\\
- \hyperlink{Creare Cibo e Acqua}{Creare Cibo e Acqua}\\
- \hyperlink{Muro di Fuoco}{Muro di Fuoco}\\
- \hyperlink{Scudo di Fuoco}{Scudo di Fuoco}

\mostro{Drago Nero Antico}
\noindent
\begin{description}[noitemsep, topsep=0pt, parsep=0pt, partopsep=0pt, leftmargin=0cm, labelwidth=2.2cm]
	\item[\textbf{Taglia/Tipo:}] Mastodontica drago, malvagio
	\item[\textbf{Caratt.:}] \resizebox{0.5\linewidth+1.8cm}{!}{For 8 Des 2 Cos 7 Int 3 Sag 2 Car 4}
	\item[\textbf{Punti Ferita:}] 422,  \textbf{Difesa:} 42,  \textbf{Iniziativa:} +3
	\item[\textbf{Movimento:}] 12 m, scalata 12 m, volo 24 m
	\item[\textbf{Tiri Salvez.:}] \resizebox{0.5\linewidth+1.8cm}{!}{\resizebox{0.5\linewidth+1.8cm}{!}{Tempra +28, Riflessi +23, Volontà +23}}
	\item[\textbf{Comp.:}] Furtività +9, Consapevolezza +16
	\item[\textbf{Imm. Danni:}] Acido, armi +1
	\item[\textbf{Sensi:}] Scurovisione 36 m, Vista Cieca 18 m
	\item[\textbf{Linguaggi:}] Comune, Draconico
	\item[\textbf{Sfida:}] 21 (33000 PX)\smallskip
\end{description}

\emph{\textbf{Gas corrosivi.}} il drago emette nel raggio di 3 metri gas corrosivi che causano 2d6 danni da acido a round.

\emph{\textbf{Anfibio.}} Il drago può respirare aria e acqua.

\emph{\textbf{Resistenza Leggendaria (3/Giorno).}} Se il drago fallisce un Tiro Salvezza, può scegliere invece di riuscire.

\emph{\textbf{Resistenza alla Magia:}} 3lv

\textbf{Azioni}

\emph{\textbf{Multiattacco.}} Il drago può usare la sua Presenza Spaventosa e poi effettuare tre attacchi: uno con il morso e due con gli artigli.

\emph{\textbf{Artiglio.} Attacco con arma da mischia}: +16 a colpire, portata 3 m, un bersaglio.

\emph{Colpisce:} 15 (2d6 + 8) danni taglienti, 3/20 danno da Sanguinamento.

\emph{\textbf{Coda.} Attacco con arma da mischia}: +16 a colpire, portata 6 m, un bersaglio.

\emph{Colpisce:} 17 (2d8 + 8) danni contundenti.

\emph{\textbf{Morso.} Attacco con arma da mischia} : +16 a colpire, portata 5 metri, un bersaglio.

\emph{Colpisce:} 19 (2d10 + 8) danni perforanti più 9 (4d6) danni da acido.

\emph{\textbf{Presenza Spaventosa.}} Ogni creatura scelta dal drago, che si trovi entro 36 metri da esso e consapevole della sua presenza, deve riuscire un Tiro Salvezza di Volontà DC 33 o restare spaventata per 1 minuto. Una creatura può ripetere il Tiro Salvezza al termine di ciascun suo round, terminando l'effetto se lo riesce. Se il Tiro Salvezza della creatura ha successo o l'effetto ha termine per essa, la creatura è immune alla Presenza Spaventosa del drago per le successive 24 ore.

\emph{\textbf{Soffio Acido (Ricarica 5-6).}} Il drago esala acido in una linea di 27 metri larga 3 metri. Ogni creatura in quell'area deve effettuare un Tiro Salvezza di Riflessi DC 33 e subire 67 (15d8) danni da acido se fallisce il Tiro Salvezza, o la metà di questi danni se lo riesce.

\textbf{Azioni Aggiuntive}

Il drago può effettuare 3 Azioni aggiuntive, scelte tra le opzioni seguenti. Può usare solo un'opzione Aggiuntiva alla volta e solo al termine del round di un'altra creatura. Il drago recupera le Azioni aggiuntive spese all'inizio del proprio round.

\textbf{Attacco di Ala (Costa 2 Azioni).} Il drago batte le ali. Ogni creatura entro 5 metri dal drago deve riuscire un Tiro Salvezza su Riflessi DC 33 o subire 15 (2d6 + 8) danni contundenti e venir gettato prono. Il drago può poi volare fino a metà del suo movimento di volo.

\textbf{Attacco di Coda.} Il drago effettua un attacco di coda.

\textbf{Individuare.} Il drago effettua una prova di Consapevolezza.

\textbf{Ecologia}\\
Ambiente: Paludi Calde\\
Organizzazione: Solitario\\
\textbf{Categoria Tesoro}: D\\
\textbf{Descrizione}\\
I Draghi Neri sono violenti ed aggressivi, vivono in paludi e acquitrini e che generalmente governano come padroni indiscussi.

I Draghi Neri sono creature minacciose che hanno grandi corna curve in avanti.
La testa si collega ad un collo relativamente corto e ad un corpo da lucertola grossa e muscoloso.

Hanno ali piccolissime che si trovano sui lati, ma riescono comunque a volare grazie alla magia.
Hanno le zampe palmate per permettere loro di nuotare con maggiore facilità nelle zone paludose dove vivono.

I Draghi Neri tendono a fare le loro tane al centro della palude o acquitrino.
Considerano quel territorio il loro e nessuno può bagnarsi senza subire la loro ira.

Una tana di drago nero può essere un ammasso gigantesco di tronchi ma anche una caverna sotterranea sommersa d'acqua, se non il fondo di un lago.
Potendo respirare sott'acqua non si fanno preoccupazione su dove costruire la loro dimora.

La loro casa è sempre protetta da trappole e dai loro seguaci malvagi che gli portano cibo, possibilmente vivo.

L'ambiente dove vive un drago nero ne subisce i suoi effetti, vapori acidi, distruzione, corruzione sono immediatamente percepibili.

Il Drago Nero rappresentano i Tratti dell'egoismo e violenza odiando ogni forma di vita, compreso gli stessi draghi neri.

I Draghi neri hanno +1d6 nelle prove di magia e possono ignorare un dado tirato nella prova con la Lista di Necromanzia ed è immune all'acido.\\
\textbf{Incantesimi}\index{Incantesimi da Drago Nero}\\
Gli incantesimi preferiti di questo Drago sono:\\
- \hyperlink{Animare Morti}{Animare Morti}\\
- \hyperlink{Creare Non Morti}{Creare Non Morti}\\
- \hyperlink{Scagliare Maledizione}{Scagliare Maledizione}

Ebbene si, il Drago Nero è l'unica creatura sulla Terra che può portare in vita un morto a discapito di tutti i vincoli imposti dai Patroni.

\mostro{Drago Nero Adulto}
\noindent
\begin{description}[noitemsep, topsep=0pt, parsep=0pt, partopsep=0pt, leftmargin=0cm, labelwidth=2.2cm]
	\item[\textbf{Taglia/Tipo:}] Enorme drago, malvagio
	\item[\textbf{Caratt.:}] \resizebox{0.5\linewidth+1.8cm}{!}{For 6 Des 2 Cos 5 Int 2 Sag 1 Car 3}
	\item[\textbf{Punti Ferita:}] 338,  \textbf{Difesa:} 36,  \textbf{Iniziativa:} +2
	\item[\textbf{Movimento:}] 12 m, scalata 12 m, volo 24 m
	\item[\textbf{Tiri Salvez.:}] \resizebox{0.5\linewidth+1.8cm}{!}{\resizebox{0.5\linewidth+1.8cm}{!}{Tempra +22, Riflessi +19, Volontà +18}}
	\item[\textbf{Comp.:}] Furtività +7, Consapevolezza +11
	\item[\textbf{Imm. Danni:}] Acido
	\item[\textbf{Sensi:}] Scurovisione 36 m, Vista Cieca 18 m
	\item[\textbf{Linguaggi:}] Comune, Draconico
	\item[\textbf{Sfida:}] 17 (18000 PX)\smallskip
\end{description}

\emph{\textbf{Gas corrosivi.}} il drago emette nel raggio di 3 metri gas corrosivi che causano 1d6 danni da acido a round.

\emph{\textbf{Anfibio.}} Il drago può respirare aria e acqua.

\emph{\textbf{Resistenza Leggendaria (3/Giorno).}} Se il drago fallisce un Tiro Salvezza, può scegliere invece di riuscire.

\textbf{Azioni}

\emph{\textbf{Multiattacco.}} Il drago può usare la sua Presenza Spaventosa e poi effettuare tre attacchi: uno con il morso e due con gli artigli.

\emph{\textbf{Artiglio.} Attacco con arma da mischia}: +14 a colpire, portata 1 m, un bersaglio.

\emph{Colpisce:} 13 (2d6 + 6) danni taglienti, 1 danno da Sanguinamento.

\emph{\textbf{Coda.} Attacco con arma da mischia}: +14 a colpire, portata 5 metri, un bersaglio.

\emph{Colpisce:} 15 (2d8 + 6) danni contundenti.

\emph{\textbf{Morso.} Attacco con arma da mischia}: +14 a colpire, portata 3 m, un bersaglio.

\emph{Colpisce:} 17 (2d10 + 6) danni perforanti più 4 (1d8) danni da acido.

\emph{\textbf{Presenza Spaventosa.}} Ogni creatura scelta dal drago, che si trovi entro 36 metri da esso e consapevole della sua presenza, deve riuscire un Tiro Salvezza di Volontà DC 30 o restare spaventata per 1 minuto. Una creatura può ripetere il Tiro Salvezza al termine di ciascun suo round, terminando l'effetto se lo riesce. Se il Tiro Salvezza della creatura ha successo o l'effetto ha termine per essa, la creatura è immune alla Presenza Spaventosa del drago per le successive 24 ore.

\emph{\textbf{Soffio Acido (Ricarica 5-6).}} Il drago esala acido in una linea di 18 metri larga 1 metro. Ogni creatura in quell'area deve effettuare un Tiro Salvezza di Riflessi DC 30 e subire 54 (12d8) danni da acido se fallisce il Tiro Salvezza, o la metà di questi danni se lo riesce.

\textbf{Azioni Aggiuntive}

Il drago può effettuare 3 Azioni aggiuntive, scelte tra le opzioni seguenti. Può usare solo un'opzione Aggiuntiva alla volta e solo al termine del round di un'altra creatura. Il drago recupera le Azioni aggiuntive spese all'inizio del proprio round.

\textbf{Attacco di Ala (Costa 2 Azioni).} Il drago batte le ali. Ogni creatura entro 3 metri dal drago deve riuscire un Tiro Salvezza su Riflessi DC 30 o subire 13 (2d6 + 6) danni contundenti e venir gettato prono. Il drago può poi volare fino a metà della del suo movimento di volo.

\textbf{Attacco di Coda.} Il drago effettua un attacco di coda.

\textbf{Individuare.} Il drago effettua una prova di Consapevolezza.

\emph{\textbf{Arrabbiato:}} Il Drago Nero Adulto può eseguire queste azioni a costo 2 Azioni.

\emph{Focalizzare}: la creatura interrompe un effetto mentale su di se in corso

\emph{Brutalità}: la creatura attacca con ferocia inaudita. +1d6 al Tiro per Colpire, 1 danno critico automatico quando colpisce.

\textbf{Ecologia}\\
Ambiente: Paludi Calde\\
Organizzazione: Solitario\\
\textbf{Categoria Tesoro}: D\\
\textbf{Descrizione}\\
Vedi Descrizione Drago Nero Antico.\\
\textbf{Incantesimi}\index{Incantesimi da Drago Nero}\\
Gli incantesimi preferiti di questo Drago sono:\\
- \hyperlink{Animare Morti}{Animare Morti}\\
- \hyperlink{Creare Non Morti}{Creare Non Morti}\\
- \hyperlink{Scagliare Maledizione}{Scagliare Maledizione}

\mostro{Drago Nero Giovane}
\noindent
\begin{description}[noitemsep, topsep=0pt, parsep=0pt, partopsep=0pt, leftmargin=0cm, labelwidth=2.2cm]
	\item[\textbf{Taglia/Tipo:}] Grande drago, malvagio
	\item[\textbf{Caratt.:}] \resizebox{0.5\linewidth+1.8cm}{!}{For 4 Des 2 Cos 3 Int 1 Sag 0 Car 2}
	\item[\textbf{Punti Ferita:}] 145,  \textbf{Difesa:} 23,  \textbf{Iniziativa:} +2
	\item[\textbf{Movimento:}] 12 m, scalata 12 m, volo 24 m
	\item[\textbf{Tiri Salvez.:}] \resizebox{0.5\linewidth+1.8cm}{!}{Tempra +10, Riflessi +9, Volontà +7}
	\item[\textbf{Comp.:}] Furtività +5, Consapevolezza +6
	\item[\textbf{Imm. Danni:}] Acido
	\item[\textbf{Sensi:}] Scurovisione 36 m, Vista Cieca 9 m
	\item[\textbf{Linguaggi:}] Comune, Draconico
	\item[\textbf{Sfida:}] 7 (2900 PX)\smallskip
\end{description}

\emph{\textbf{Anfibio.}} Il drago può respirare aria e acqua.

\textbf{Azioni}

\emph{\textbf{Multiattacco.}} Il drago può effettuare tre attacchi: uno con il morso e due con gli artigli.

\emph{\textbf{Artiglio.} Attacco con arma da mischia}: +8 a colpire, portata 1 m, un bersaglio.

\emph{Colpisce:} 11 (2d6 + 4) danni taglienti, 1 danno da Sanguinamento.

\emph{\textbf{Morso.} Attacco con arma da mischia}: +8 a colpire, portata 3 m, un bersaglio.

\emph{Colpisce:} 11 (2d10 + 4) danni perforanti più 4 (1d8) danni da acido.

\emph{\textbf{Soffio Acido (Ricarica 5-6).}} Il drago esala acido in una linea di 9 metri larga 1 metro. Ogni creatura in quell'area deve effettuare un Tiro Salvezza di Riflessi DC 19 e subire 49 (11d8) danni da acido se fallisce il Tiro Salvezza, o la metà di questi danni se lo riesce.

\emph{\textbf{Arrabbiato:}} Il Drago Nero Giovane ricarica il soffio acido. Costa 1 Azione.

\textbf{Ecologia}\\
Ambiente: Paludi Calde\\
Organizzazione: Solitario\\
\textbf{Categoria Tesoro}: C\\
\textbf{Descrizione}\\
Vedi Descrizione Drago Nero Antico.\\
\textbf{Incantesimi}\index{Incantesimi da Drago Nero}\\
Gli incantesimi preferiti di questo Drago sono:\\
- \hyperlink{Animare Morti}{Animare Morti}\\
- \hyperlink{Creare Non Morti}{Creare Non Morti}\\
- \hyperlink{Scagliare Maledizione}{Scagliare Maledizione}

\medskip

\begin{center}
	\includegraphics[width=0.9\linewidth]{immagini/Friedrich-Johann-Justin-Bertuch_Mythical-Creature-Dragon_1806.png}

	\emph{Drago, Friedrich Johann Justin Bertuch}
\end{center}

\mostro{Drago Nero Cucciolo}
\noindent
\begin{description}[noitemsep, topsep=0pt, parsep=0pt, partopsep=0pt, leftmargin=0cm, labelwidth=2.2cm]
	\item[\textbf{Taglia/Tipo:}] Media drago, malvagio
	\item[\textbf{Caratt.:}] \resizebox{0.5\linewidth+1.8cm}{!}{For 2 Des 2 Cos 1 Int 0 Sag 0 Car 1}
	\item[\textbf{Punti Ferita:}] 51,  \textbf{Difesa:} 16,  \textbf{Iniziativa:} +2
	\item[\textbf{Movimento:}] 9 m, scalata 9 m, volo 18 m
	\item[\textbf{Tiri Salvez.:}] \resizebox{0.5\linewidth+1.8cm}{!}{Tempra +3, Riflessi +4, Volontà +3}
	\item[\textbf{Comp.:}] Furtività +4, Consapevolezza +4
	\item[\textbf{Imm. Danni:}] Acido
	\item[\textbf{Sensi:}] Scurovisione 18 m, Vista Cieca 3 m
	\item[\textbf{Linguaggi:}] Draconico
	\item[\textbf{Sfida:}] 2 (450 PX)\smallskip
\end{description}

\emph{\textbf{Anfibio.}} Il drago può respirare aria e acqua.

\textbf{Azioni}

\emph{\textbf{Morso.} Attacco con arma da mischia}: +5 a colpire, portata 1 m, un bersaglio.

\emph{Colpisce:} 7 (1d10 + 2) danni perforanti più 2 (1d4) danni da acido.

\emph{\textbf{Soffio Acido (Ricarica 5-6).}} Il drago esala acido in una linea di 5 metri larga 1 metro. Ogni creatura in quell'area deve effettuare un Tiro Salvezza di Riflessi DC 14 e subire 22 (5d8) danni da acido se fallisce il Tiro Salvezza, o la metà di questi danni se lo riesce.

\textbf{Ecologia}\\
Ambiente: Paludi Calde\\
Organizzazione: Solitario\\
\textbf{Categoria Tesoro}: H\\
\textbf{Descrizione}\\
Vedi Descrizione Drago Nero Antico.

\mostro{Drago Porpora Antico}
\noindent
\begin{description}[noitemsep, topsep=0pt, parsep=0pt, partopsep=0pt, leftmargin=0cm, labelwidth=2.2cm]
	\item[\textbf{Taglia/Tipo:}] Mastodontica drago, malvagio
	\item[\textbf{Caratt.:}] \resizebox{0.5\linewidth+1.8cm}{!}{For 8 Des 3 Cos 4 Int 4 Sag 4 Car 4}
	\item[\textbf{Punti Ferita:}] 428,  \textbf{Difesa:} 44,  \textbf{Iniziativa:} +4
	\item[\textbf{Movimento:}] 12 m, scavare 24 m
	\item[\textbf{Tiri Salvez.:}] \resizebox{0.5\linewidth+1.8cm}{!}{\resizebox{0.5\linewidth+1.8cm}{!}{Tempra +26, Riflessi +25, Volontà +26}}
	\item[\textbf{Comp.:}] Conoscenza Dungeon +8, Intimidazione +11, Percepire Emozioni +10, Consapevolezza + 15
	\item[\textbf{Imm. Danni:}] Suono, armi +1
	\item[\textbf{Sensi:}] Scurovisione 36 m, Senso Tellurico 72 m
	\item[\textbf{Linguaggi:}] Comune, Draconico
	\item[\textbf{Sfida:}] 22 (41000 PX)\smallskip
\end{description}

\emph{\textbf{Onde esplosive.}} il drago emette nel raggio di 3 metri vibrazioni sonore che causano 2d6 danni da suono a round.

\emph{\textbf{Terrestre.}} Il drago finché è sotto terra può non respirare ne mangiare.

\emph{\textbf{Resistenza Leggendaria (3/Giorno).}} Se il drago fallisce un Tiro Salvezza, può scegliere invece di riuscire.

\emph{\textbf{Resistenza alla Magia:}} 3lv

\textbf{Azioni}

\emph{\textbf{Multiattacco.}} Il drago può usare la sua Presenza Spaventosa e poi effettuare tre attacchi: uno con il morso e due con gli artigli.

\emph{\textbf{Artiglio.} Attacco con arma da mischia}: +17 a colpire, portata 3 m, un bersaglio.

\emph{Colpisce:} 15 (2d6 + 8) danni taglienti, 3/20 danno da Sanguinamento.

\emph{\textbf{Coda.} Attacco con arma da mischia}: +17 a colpire, portata 6 m, un bersaglio.

\emph{Colpisce:} 17 (2d8 + 8) danni contundenti.

\emph{\textbf{Morso.} Attacco con arma da mischia}: +17 a colpire, portata 5 metri, un bersaglio.

\emph{Colpisce:} 19 (2d10 + 8) danni perforanti più 10 (3d6) danni da veleno.

\emph{\textbf{Presenza Spaventosa.}} Ogni creatura scelta dal drago, che si trovi entro 36 metri da esso e consapevole della sua presenza, deve riuscire un Tiro Salvezza di Volontà DC 35 o restare spaventata per 1 minuto. Una creatura può ripetere il Tiro Salvezza al termine di ciascun suo round, terminando l'effetto se lo riesce. Se il Tiro Salvezza della creatura ha successo o l'effetto ha termine per essa, la creatura è immune alla Presenza Spaventosa del drago per le successive 24 ore.

\emph{\textbf{Soffio Sonico (Ricarica 5-6).}} Il drago emette un cono di 27 metri. Ogni creatura in quell'area deve effettuare un Tiro Salvezza di Tempra DC 35 e subire 77 (22d6) danni da suono se fallisce il Tiro Salvezza, o la metà di questi danni se lo riesce.

\textbf{Azioni Aggiuntive}

Il drago può effettuare 3 Azioni aggiuntive, scelte tra le opzioni seguenti. Può usare solo un'opzione Aggiuntiva alla volta e solo al termine del round di un'altra creatura. Il drago recupera le Azioni aggiuntive spese all'inizio del proprio round.

\textbf{Schianto (Costa 2 Azioni).} Il drago salta sul posto. Ogni creatura entro 5 metri dal drago deve riuscire un Tiro Salvezza su Riflessi DC 35 o subire 15 (2d6 + 8) danni contundenti e venire gettato prono. Il drago può poi spostarsi fino a metà del suo movimento.

\textbf{Attacco di Coda.} Il drago effettua un attacco di coda.

\textbf{Individuare.} Il drago effettua una prova di Consapevolezza.

\textbf{Ecologia}\\
Ambiente: Caverne\\
Organizzazione: Solitario, creature sotterranee\\
\textbf{Categoria Tesoro}: E\\
\textbf{Descrizione}\\
I Draghi Porpora vivono sotto terra e si sono perfettamente adattati alla vita sotterranea.
Capaci di vedere al buio come se fosse pieno giorno, dotati di Senso Tellurico, hanno perso la capacità di volare ma acquisito quella di scavare con la stessa velocità come se corressero.

Un Drago Porpora è molto territoriale e stabilirà un perimetro (di circa 5 km di raggio) dove crea, se non già presente un intricata serie di cunicoli e caverne per i suoi servi.

Un Drago Porpora è molto protettivo con le sue creature, con chi gli porta da mangiare e gli offre tesori.

Dall'aspetto tozzo hanno denti fini e lunghi ed artigli enormi che continuamente crescono. Hanno un potentissimo attacco sonico che spesso crea crolli nelle caverne, crolli che sono completamente indifferente a lui.

E' forte e coraggioso, arrogante ma non sfrontato. Non ha paura di combattere se pensa di vincere. Porta sempre la battaglia sottoterra dove può creare fosse per fare precipitare i nemici o scappare se necessario.

Un Drago Porpora ha +1d6 nelle prove di magia e può ignorare un dado tirato nella prova con la Lista della Terra, è immune al danno ed effetti sonori.\\
\textbf{Incantesimi}\index{Incantesimi da Drago Porpora}\\
Gli incantesimi preferiti di questo Drago sono:\\
- \hyperlink{Freccia Acida di Restser}{Freccia Acida di Restser}\\
- \hyperlink{Passare Senza Tracce}{Passare Senza Tracce}\\
- \hyperlink{Scolpire Pietra}{Scolpire Pietra}

\mostro{Drago Rosso Antico}
\noindent
\begin{description}[noitemsep, topsep=0pt, parsep=0pt, partopsep=0pt, leftmargin=0cm, labelwidth=2.2cm]
	\item[\textbf{Taglia/Tipo:}] Mastodontica drago, malvagio
	\item[\textbf{Caratt.:}] \resizebox{0.5\linewidth+1.8cm}{!}{For 10 Des 0 Cos 9 Int 4 Sag 2 Car 6}
	\item[\textbf{Punti Ferita:}] 490,  \textbf{Difesa:} 44,  \textbf{Iniziativa:} +4
	\item[\textbf{Movimento:}] 12 m, scalata 12 m, volo 24 m
	\item[\textbf{Tiri Salvez.:}] \resizebox{0.5\linewidth+1.8cm}{!}{\resizebox{0.5\linewidth+1.8cm}{!}{Tempra +33, Riflessi +24, Volontà +26}}
	\item[\textbf{Comp.:}] Furtività +7, Consapevolezza +16
	\item[\textbf{Imm. Danni:}] Fuoco, armi +1
	\item[\textbf{Sensi:}] Scurovisione 36 m, Vista Cieca 18 m
	\item[\textbf{Linguaggi:}] Comune, Draconico
	\item[\textbf{Sfida:}] 24 (62000 PX)\smallskip
\end{description}

\emph{\textbf{Aura di fiamma.}} il drago emette nel raggio di 3 metri calore magico che causa 2d6 danni da fuoco a round.

\emph{\textbf{Resistenza Leggendaria (3/Giorno).}} Se il drago fallisce un Tiro Salvezza, può scegliere invece di riuscire.

\emph{\textbf{Resistenza alla Magia:}} 3lv

\textbf{Azioni}

\emph{\textbf{Multiattacco.}} Il drago può usare la sua Presenza Spaventosa e poi effettuare tre attacchi: uno con il morso e due con gli artigli.

\emph{\textbf{Artiglio.} Attacco con arma da mischia}: +18 a colpire, portata 3 m, un bersaglio.

\emph{Colpisce:} 17 (2d6 + 10) danni taglienti, 3/20 danno da Sanguinamento.

\emph{\textbf{Coda.} Attacco con arma da mischia}: +18 a colpire, portata 6 m, un bersaglio.

\emph{Colpisce:} 19 (2d8 + 10) danni contundenti.

\emph{\textbf{Morso.} Attacco con arma da mischia}: +18 a colpire, portata 5 metri, un bersaglio.

\emph{Colpisce:} 21 (2d10 + 10) danni perforanti più 14 (4d6) danni da fuoco.

\emph{\textbf{Presenza Spaventosa.}} Ogni creatura scelta dal drago, che si trovi entro 36 metri da esso e consapevole della sua presenza, deve riuscire un Tiro Salvezza di Volontà DC 38 o restare spaventata per 1 minuto. Una creatura può ripetere il Tiro Salvezza al termine di ciascun suo round, terminando l'effetto se lo riesce. Se il Tiro Salvezza della creatura ha successo o l'effetto ha termine per essa, la creatura è immune alla Presenza Spaventosa del drago per le successive 24 ore.

\emph{\textbf{Soffio Infuocato (Ricarica 5-6).}} Il drago esala fuoco in un cono di 27 metri. Ogni creatura in quell'area deve effettuare un Tiro Salvezza su Riflessi DC 38 e subire 91 (26d6) danni da fuoco se fallisce il Tiro Salvezza, o la metà di questi danni se lo riesce.

\textbf{Azioni Aggiuntive}

Il drago può effettuare 3 Azioni aggiuntive, scelte tra le opzioni seguenti. Può usare solo un'opzione Aggiuntiva alla volta e solo al termine del round di un'altra creatura. Il drago recupera le Azioni aggiuntive spese all'inizio del proprio round.

\textbf{Attacco di Ala (Costa 2 Azioni).} Il drago batte le ali. Ogni creatura entro 5 metri dal drago deve riuscire un Tiro Salvezza su Riflessi DC 38 o subire 17 (2d6 + 10) danni contundenti e venir gettato prono. Il drago può poi volare fino a metà del suo movimento di volo.

\textbf{Attacco di Coda.} Il drago effettua un attacco di coda.

\textbf{Individuare.} Il drago effettua una prova di Consapevolezza.

\emph{\textbf{Arrabbiato}}: Il drago rosso si scuote e ruggisce. 1 volta al giorno, la prima volta che è Arrabbiato, termina tutte le condizioni negative su di se e tutte le abilità si ricaricano. Il soffio si ricarica con 3-6.

\textbf{Ecologia}\\
Ambiente: Montagne calde\\
Organizzazione: Solitario\\
\textbf{Categoria Tesoro}: H\\
\textbf{Descrizione}\\
Il Drago Rosso si crede il Re dei Draghi per via della sua potenza fisica e del soffio capace di sciogliere la pietra.

I Draghi Rossi sono i draghi più grandi sia per corporatura che per apertura alare.
Spesso le scaglie, di un rosso scuro quasi di sangue, hanno bordi affilati ed allungati.

I Draghi Rossi prediligono le montagne calde e se possibile direttamente direttamente dentro un vulcano.

Combattono sfruttando la loro mole, le ali, il morso artigli.. insomma tutto ciò che sono ed hanno a disposizione. Un Drago Rosso combatte sempre fino alla morte non si ritira ne scappa ne rinuncia ad una sfida, l'orgoglio di cui sono tronfi non gli permette di mostrarsi deboli.

Un Drago Rosso hanno +1d6 nelle prove di magia e possono ignorare un dado tirato nella prova con la Lista del Fuoco ed è immune al fuoco.\\
\textbf{Incantesimi}\index{Incantesimi da Drago Rosso}\\
Gli incantesimi preferiti di questo Drago sono:\\
- \hyperlink{Palla di Fuoco}{Palla di Fuoco}\\
- \hyperlink{Nube Incendiaria}{Nube Incendiaria}\\
- \hyperlink{Muro di Fuoco}{Muro di Fuoco}


\begin{center}
	\includegraphics[width=0.9\linewidth]{immagini/Pair_of_winged_dragons.png}

	\emph{China, Pair of winged dragons, 4th-5th century}
\end{center}

\mostro{Drago Rosso Adulto}
\noindent
\begin{description}[noitemsep, topsep=0pt, parsep=0pt, partopsep=0pt, leftmargin=0cm, labelwidth=2.2cm]
	\item[\textbf{Taglia/Tipo:}] Enorme drago, malvagio
	\item[\textbf{Caratt.:}] \resizebox{0.5\linewidth+1.8cm}{!}{For 8 Des 0 Cos 7 Int 3 Sag 1 Car 5}
	\item[\textbf{Punti Ferita:}] 344,  \textbf{Difesa:} 34,  \textbf{Iniziativa:} +3
	\item[\textbf{Movimento:}] 12 m, scalata 12 m, volo 24 m
	\item[\textbf{Tiri Salvez.:}] \resizebox{0.5\linewidth+1.8cm}{!}{\resizebox{0.5\linewidth+1.8cm}{!}{Tempra +24, Riflessi +17, Volontà +18}}
	\item[\textbf{Comp.:}] Furtività +6, Consapevolezza +13
	\item[\textbf{Imm. Danni:}] Fuoco
	\item[\textbf{Sensi:}] Scurovisione 36 m, Vista Cieca 18 m
	\item[\textbf{Linguaggi:}] Comune, Draconico
	\item[\textbf{Sfida:}] 17 (18000 PX)\smallskip
\end{description}

\emph{\textbf{Aura di fiamma.}} il drago emette nel raggio di 3 metri calore magico che causa 1d6 danni da fuoco a round.

\emph{\textbf{Resistenza Leggendaria (3/Giorno).}} Se il drago fallisce un Tiro Salvezza, può scegliere invece di riuscire.

\textbf{Azioni}

\emph{\textbf{Multiattacco.}} Il drago può usare la sua Presenza Spaventosa e poi effettuare tre attacchi: uno con il morso e due con gli artigli.

\emph{\textbf{Artiglio.} Attacco con arma da mischia}: +14 a colpire, portata 1 m, un bersaglio.

\emph{Colpisce:} 15 (2d6 + 8) danni taglienti, 1 danno da Sanguinamento.

\emph{\textbf{Coda.} Attacco con arma da mischia}: +14 a colpire, portata 5 metri, un bersaglio.

\emph{Colpisce:} 17 (2d8 + 8) danni contundenti.

\emph{\textbf{Morso.} Attacco con arma da mischia}: +14 a colpire, portata 3 m, un bersaglio.

\emph{Colpisce:} 19 (2d10 + 8) danni perforanti più 7 (2d6) danni da fuoco.

\emph{\textbf{Presenza Spaventosa.}} Ogni creatura scelta dal drago, che si trovi entro 36 metri da esso e consapevole della sua presenza, deve riuscire un Tiro Salvezza di Volontà DC 30 o restare spaventata per 1 minuto. Una creatura può ripetere il Tiro Salvezza al termine di ciascun suo round, terminando l'effetto se lo riesce. Se il Tiro Salvezza della creatura ha successo o l'effetto ha termine per essa, la creatura è immune alla Presenza Spaventosa del drago per le successive 24 ore.

\emph{\textbf{Soffio Infuocato (Ricarica 5-6).}} Il drago esala fuoco in un cono di 18 metri. Ogni creatura in quell'area deve effettuare un Tiro Salvezza su Riflessi DC 30 e subire 63 (18d6) danni da fuoco se fallisce il Tiro Salvezza, o la metà di questi danni se lo riesce.

\textbf{Azioni Aggiuntive}

Il drago può effettuare 3 Azioni aggiuntive, scelte tra le opzioni seguenti. Può usare solo un'opzione Aggiuntiva alla volta e solo al termine del round di un'altra creatura. Il drago recupera le Azioni aggiuntive spese all'inizio del proprio round.

\textbf{Attacco di Ala (Costa 2 Azioni).} Il drago batte le ali. Ogni creatura entro 3 metri dal drago deve riuscire un Tiro Salvezza su Riflessi DC 30 o subire 15 (2d6 + 8) danni contundenti e venir gettato prono.

Il drago può poi volare fino a metà del suo movimento di volo.

\textbf{Attacco di Coda.} Il drago effettua un attacco di coda.

\textbf{Individuare.} Il drago effettua una prova di Consapevolezza.

\emph{\textbf{Arrabbiato:}} Il Drago Rosso Adulto può eseguire queste azioni a costo 2 Azioni.

\emph{Focalizzare}: la creatura interrompe un effetto mentale su di se in corso

\emph{Brutalità}: la creatura attacca con ferocia inaudita. +1d6 al Tiro per Colpire, 1 danno critico automatico quando colpisce.

\textbf{Ecologia}\\
Ambiente: Montagne calde\\
Organizzazione: Solitario\\
\textbf{Categoria Tesoro}: C\\
\textbf{Descrizione}\\
Vedi Descrizione Drago Rosso Antico.\\
\textbf{Incantesimi}\index{Incantesimi da Drago Rosso}\\
Gli incantesimi preferiti di questo Drago sono:\\
- \hyperlink{Palla di Fuoco}{Palla di Fuoco}\\
- \hyperlink{Nube Incendiaria}{Nube Incendiaria}\\
- \hyperlink{Muro di Fuoco}{Muro di Fuoco}


\mostro{Drago Rosso Giovane}
\noindent
\begin{description}[noitemsep, topsep=0pt, parsep=0pt, partopsep=0pt, leftmargin=0cm, labelwidth=2.2cm]
	\item[\textbf{Taglia/Tipo:}] Grande drago, malvagio
	\item[\textbf{Caratt.:}] \resizebox{0.5\linewidth+1.8cm}{!}{For 6 Des 0 Cos 5 Int 2 Sag 0 Car 4}
	\item[\textbf{Punti Ferita:}] 205,  \textbf{Difesa:} 25,  \textbf{Iniziativa:} +2
	\item[\textbf{Movimento:}] 12 m, scalata 12 m, volo 24 m
	\item[\textbf{Tiri Salvez.:}] \resizebox{0.5\linewidth+1.8cm}{!}{\resizebox{0.5\linewidth+1.8cm}{!}{Tempra +15, Riflessi +10, Volontà +10}}
	\item[\textbf{Comp.:}] Furtività +4, Consapevolezza +8
	\item[\textbf{Imm. Danni:}] Fuoco
	\item[\textbf{Sensi:}] Scurovisione 36 m, Vista Cieca 9 m
	\item[\textbf{Linguaggi:}] Comune, Draconico
	\item[\textbf{Sfida:}] 10 (5900 PX)\smallskip
\end{description}

\textbf{Azioni}

\emph{\textbf{Multiattacco.}} Il drago può effettuare tre attacchi: uno con il morso e due con gli artigli.

\emph{\textbf{Artiglio.} Attacco con arma da mischia}: +11 a colpire, portata 1 m, un bersaglio.

\emph{Colpisce:} 13 (2d6 + 6) danni taglienti, 1 danno da Sanguinamento.

\emph{\textbf{Morso.} Attacco con arma da mischia}: +11 a colpire, portata 3 m, un bersaglio.

\emph{Colpisce:} 17 (2d10 + 6) danni perforanti più 3 (1d6) danni da fuoco.

\emph{\textbf{Soffio Infuocato (Ricarica 5-6).}} Il drago esala fuoco in un cono di 9 metri. Ogni creatura in quell'area deve effettuare un Tiro Salvezza su Riflessi DC 22 e subire 56 (16d6) danni da fuoco se fallisce il Tiro Salvezza, o la metà di questi danni se lo riesce.

\emph{\textbf{Arrabbiato:}} il giovane drago rosso ricarica il soffio infuocato.

Costa 1 Azione.

\textbf{Ecologia}\\
Ambiente: Montagne calde\\
Organizzazione: Solitario\\
\textbf{Categoria Tesoro}: D\\
Vedi Descrizione Drago Rosso Antico.\\
\textbf{Incantesimi}\index{Incantesimi da Drago Rosso}\\
Gli incantesimi preferiti di questo Drago sono:\\
- \hyperlink{Palla di Fuoco}{Palla di Fuoco}\\
- \hyperlink{Nube Incendiaria}{Nube Incendiaria}\\
- \hyperlink{Muro di Fuoco}{Muro di Fuoco}

\mostro{Drago Rosso Cucciolo}
\noindent
\begin{description}[noitemsep, topsep=0pt, parsep=0pt, partopsep=0pt, leftmargin=0cm, labelwidth=2.2cm]
	\item[\textbf{Taglia/Tipo:}] Media drago, malvagio
	\item[\textbf{Caratt.:}] \resizebox{0.5\linewidth+1.8cm}{!}{For 4 Des 0 Cos 3 Int 1 Sag 0 Car 2}
	\item[\textbf{Punti Ferita:}] 89,  \textbf{Difesa:} 17,  \textbf{Iniziativa:} +1
	\item[\textbf{Movimento:}] 9 m, scalata 9 m, volo 18 m
	\item[\textbf{Tiri Salvez.:}] \resizebox{0.5\linewidth+1.8cm}{!}{Tempra +7, Riflessi +4, Volontà +4}
	\item[\textbf{Comp.:}] Furtività +2, Consapevolezza +4
	\item[\textbf{Imm. Danni:}] Fuoco
	\item[\textbf{Sensi:}] Scurovisione 18 m, Vista Cieca 3 m
	\item[\textbf{Linguaggi:}] Draconico
	\item[\textbf{Sfida:}] 4 (1100 PX)\smallskip
\end{description}

\textbf{Azioni}

\emph{\textbf{Morso.} Attacco con arma da mischia}: +6 a colpire, portata 1 m, un bersaglio.

\emph{Colpisce:} 9 (1d10 + 4) danni perforanti più 3 (1d6) danni da fuoco.

\emph{\textbf{Soffio Infuocato (Ricarica 5-6).}} Il drago esala fuoco in un cono di 5 metri. Ogni creatura in quell'area deve effettuare un Tiro Salvezza di Riflessi DC 16 e subire 24 (7d6) danni da fuoco se fallisce il Tiro Salvezza, o la metà di questi danni se lo riesce.

\textbf{Ecologia}\\
Ambiente: Montagne calde\\
Organizzazione: Solitario\\
\textbf{Categoria Tesoro}: C\\
Vedi Descrizione Drago Rosso Antico.

\mostro{Drago Verde Antico}
\noindent
\begin{description}[noitemsep, topsep=0pt, parsep=0pt, partopsep=0pt, leftmargin=0cm, labelwidth=2.2cm]
	\item[\textbf{Taglia/Tipo:}] Mastodontica drago, malvagio
	\item[\textbf{Caratt.:}] \resizebox{0.5\linewidth+1.8cm}{!}{For 8 Des 1 Cos 7 Int 5 Sag 3 Car 4}
	\item[\textbf{Punti Ferita:}] 441,  \textbf{Difesa:} 42,  \textbf{Iniziativa:} +5
	\item[\textbf{Movimento:}] 12 m, nuoto 12 m, volo 24 m
	\item[\textbf{Tiri Salvez.:}] \resizebox{0.5\linewidth+1.8cm}{!}{\resizebox{0.5\linewidth+1.8cm}{!}{Tempra +29, Riflessi +23, Volontà +25}}
	\item[\textbf{Comp.:}] Furtività +8, Ingannare +11, Percepire Emozioni +10, Consapevolezza + 15
	\item[\textbf{Imm. Danni:}] Veleno, armi +1
	\item[\textbf{Sensi:}] Scurovisione 36 m, Vista Cieca 18 m
	\item[\textbf{Linguaggi:}] Comune, Draconico
	\item[\textbf{Sfida:}] 22 (41000 PX)\smallskip
\end{description}

\emph{\textbf{Aria mefitica.}} il drago emette nel raggio di 3 metri gas magici che causano 2d6 danni da veleno a round.

\emph{\textbf{Anfibio.}} Il drago può respirare aria e acqua.

\emph{\textbf{Resistenza Leggendaria (3/Giorno).}} Se il drago fallisce un Tiro Salvezza, può scegliere invece di riuscire.

\emph{\textbf{Resistenza alla Magia:}} 3lv

\textbf{Azioni}

\emph{\textbf{Multiattacco.}} Il drago può usare la sua Presenza Spaventosa e poi effettuare tre attacchi: uno con il morso e due con gli artigli.

\emph{\textbf{Artiglio.} Attacco con arma da mischia}: +17 a colpire, portata 3 m, un bersaglio.

\emph{Colpisce:} 15 (2d6 + 8) danni taglienti, 3/20 danno da Sanguinamento.

\emph{\textbf{Coda.} Attacco con arma da mischia}: +17 a colpire, portata 6 m, un bersaglio.

\emph{Colpisce:} 17 (2d8 + 8) danni contundenti.

\emph{\textbf{Morso.} Attacco con arma da mischia}: +17 a colpire, portata 5 metri, un bersaglio.

\emph{Colpisce:} 19 (2d10 + 8) danni perforanti più 10 (3d6) danni da veleno.

\emph{\textbf{Presenza Spaventosa.}} Ogni creatura scelta dal drago, che si trovi entro 36 metri da esso e consapevole della sua presenza, deve riuscire un Tiro Salvezza di Volontà DC 25 o restare spaventata per 1 minuto. Una creatura può ripetere il Tiro Salvezza al termine di ciascun suo round, terminando l'effetto se lo riesce. Se il Tiro Salvezza della creatura ha successo o l'effetto ha termine per essa, la creatura è immune alla Presenza Spaventosa del drago per le successive 24 ore.

\emph{\textbf{Soffio Velenoso (Ricarica 5-6).}} Il drago esala gas velenosi in un cono di 27 metri. Ogni creatura in quell'area deve effettuare un Tiro Salvezza di Tempra DC 35 e subire 77 (22d6) danni da veleno se fallisce il Tiro Salvezza, o la metà di questi danni se lo riesce.

\textbf{Azioni Aggiuntive}

Il drago può effettuare 3 Azioni aggiuntive, scelte tra le opzioni seguenti. Può usare solo un'opzione Aggiuntiva alla volta e solo al termine del round di un'altra creatura. Il drago recupera le Azioni aggiuntive spese all'inizio del proprio round.

\textbf{Attacco di Ala (Costa 2 Azioni).} Il drago batte le ali. Ogni creatura entro 5 metri dal drago deve riuscire un Tiro Salvezza su Riflessi DC 35 o subire 15 (2d6 + 8) danni contundenti e venire gettato prono. Il drago può poi volare fino a metà del suo movimento di volo.

\textbf{Attacco di Coda.} Il drago effettua un attacco di coda.

\textbf{Individuare.} Il drago effettua una prova di Consapevolezza.

\textbf{Ecologia}\\
Ambiente: Foreste Temperate\\
Organizzazione: Solitario\\
\textbf{Categoria Tesoro}: H\\
\textbf{Descrizione}\\
I Draghi verdi amano le foreste e la natura incontaminata dove si reputano i padroni e re indiscussi.

I potenti draghi verdi hanno la testa tondeggiante e pronunciate orecchie all'indietro, le corna sono corte e non appuntite.
Gli artigli e le fauci sono devastanti, potenti e capace di tranciare qualsiasi cosa.
Il naso è largo e le narici aperte come se dovesse soffiare in qualsiasi momento.

Il soffio dei draghi verde è veleno, così che possa uccidere le creature viventi ma non le piante.

La tana di un drago verde è sempre vicino ad una sorgente d'acqua, possibilmente nella parte più lussureggiante ed incontaminata della foresta.

Un Drago verde non ama volare e preferisce saltare schiacciando con il suo peso e dilaniare con i suoi artigli.

Tra i tanti draghi quello verde è forse quello che farà parlare gli avventurieri se si dimostrano rispettosi ed impauriti dalla sua regalità.

I Draghi Verdi hanno +1d6 nelle prove di magia e possono ignorare un dado tirato nella prova con la Lista di Animali e Piante ed è immune i Veleni sia a quelli magici che naturali.\\
\textbf{Incantesimi}\index{Incantesimi da Drago Verde}\\
Gli incantesimi preferiti di questo Drago sono:\\
- \hyperlink{Guscio Anti-Vita}{Guscio Anti-Vita}\\
- \hyperlink{Localizza Animali e Piante}{Localizza Animali e Piante}\\
- \hyperlink{Rimuovi Veleno}{Rimuovi Veleno}

\mostro{Drago Verde Adulto}
\noindent
\begin{description}[noitemsep, topsep=0pt, parsep=0pt, partopsep=0pt, leftmargin=0cm, labelwidth=2.2cm]
	\item[\textbf{Taglia/Tipo:}] Enorme drago, malvagio
	\item[\textbf{Caratt.:}] \resizebox{0.5\linewidth+1.8cm}{!}{For 6 Des 1 Cos 5 Int 4 Sag 2 Car 3}
	\item[\textbf{Punti Ferita:}] 300,  \textbf{Difesa:} 33,  \textbf{Iniziativa:} +4
	\item[\textbf{Movimento:}] 12 m, nuoto 12 m, volo 24 m
	\item[\textbf{Tiri Salvez.:}] \resizebox{0.5\linewidth+1.8cm}{!}{\resizebox{0.5\linewidth+1.8cm}{!}{Tempra +20, Riflessi +16, Volontà +17}}
	\item[\textbf{Comp.:}] Furtività +6, Ingannare +8, Percepire Emozioni +7, Consapevolezza +12
	\item[\textbf{Imm. Danni:}] Veleno
	\item[\textbf{Sensi:}] Scurovisione 36 m, Vista Cieca 18 m
	\item[\textbf{Linguaggi:}] Comune, Draconico
	\item[\textbf{Sfida:}] 15 (13000 PX)\smallskip
\end{description}

\emph{\textbf{Aria mefitica.}} il drago emette nel raggio di 3 metri gas magici che causano 1d6 danni da veleno a round.

\emph{\textbf{Anfibio.}} Il drago può respirare aria e acqua.

\emph{\textbf{Resistenza Leggendaria (3/Giorno).}} Se il drago fallisce un Tiro Salvezza, può scegliere invece di riuscire.

\textbf{Azioni}

\emph{\textbf{Multiattacco.}} Il drago può usare la sua Presenza Spaventosa e poi effettuare tre attacchi: uno con il morso e due con gli artigli.

\emph{\textbf{Artiglio.} Attacco con arma da mischia}: +13 a colpire, portata 1 m, un bersaglio.

\emph{Colpisce:} 13 (2d6 + 6) danni taglienti, 1 danno da Sanguinamento.

\emph{\textbf{Coda.} Attacco con arma da mischia}: +13 a colpire, portata 5 metri, un bersaglio.

\emph{Colpisce:} 15 (2d8 + 6) danni contundenti.

\emph{\textbf{Morso.} Attacco con arma da mischia}: +13 a colpire, portata 3 m, un bersaglio.

\emph{Colpisce:} 17 (2d10 + 6) danni perforanti più 7 (2d6) danni da veleno.

\emph{\textbf{Presenza Spaventosa.}} Ogni creatura scelta dal drago, che si trovi entro 36 metri da esso e consapevole della sua presenza, deve riuscire un Tiro Salvezza di Volontà DC 28 o restare spaventata per 1 minuto. Una creatura può ripetere il Tiro Salvezza al termine di ciascun suo round, terminando l'effetto se lo riesce. Se il Tiro Salvezza della creatura ha successo o l'effetto ha termine per essa, la creatura è immune alla Presenza Spaventosa del drago per le successive 24 ore.

\emph{\textbf{Soffio Velenoso (Ricarica 5-6).}} Il drago esala gas velenosi in un cono di 18 metri. Ogni creatura in quell'area deve effettuare un Tiro Salvezza di Tempra DC 28 e subire 56 (16d6) danni da veleno se fallisce il Tiro Salvezza, o la metà di questi danni se lo riesce.

\textbf{Azioni Aggiuntive}

Il drago può effettuare 3 Azioni aggiuntive, scelte tra le opzioni seguenti. Può usare solo un'opzione Aggiuntiva alla volta e solo al termine del round di un'altra creatura. Il drago recupera le Azioni aggiuntive spese all'inizio del proprio round.

\textbf{Attacco di Ala (Costa 2 Azioni).} Il drago batte le ali. Ogni creatura entro 3 metri dal drago deve riuscire un Tiro Salvezza su Riflessi DC 28 o subire 13 (2d6 + 6) danni contundenti e venir gettato prono. Il drago può poi volare fino a metà del suo movimento di volo.

\textbf{Attacco di Coda.} Il drago effettua un attacco di coda.

\textbf{Individuare.} Il drago effettua una prova di Consapevolezza.

\emph{\textbf{Arrabbiato:}} Il Drago Verde Adulto può eseguire queste azioni a costo 2 Azioni.

\emph{Focalizzare}: la creatura interrompe un effetto mentale su di se in corso

\emph{Brutalità}: la creatura attacca con ferocia inaudita. +1d6 al Tiro per Colpire, 1 danno critico automatico quando colpisce.

\textbf{Ecologia}\\
Ambiente: Foreste Temperate\\
Organizzazione: Solitario\\
\textbf{Categoria Tesoro}: E\\
\textbf{Descrizione}\\
Vedi Descrizione Drago Verde Antico.\\
\textbf{Incantesimi}\index{Incantesimi da Drago Verde}\\
Gli incantesimi preferiti di questo Drago sono:\\
- \hyperlink{Guscio Anti-Vita}{Guscio Anti-Vita}\\
- \hyperlink{Localizza Animali e Piante}{Localizza Animali e Piante}\\
- \hyperlink{Rimuovi Veleno}{Rimuovi Veleno}

\mostro{Drago Verde Giovane}
\noindent
\begin{description}[noitemsep, topsep=0pt, parsep=0pt, partopsep=0pt, leftmargin=0cm, labelwidth=2.2cm]
	\item[\textbf{Taglia/Tipo:}] Grande drago, malvagio
	\item[\textbf{Caratt.:}] \resizebox{0.5\linewidth+1.8cm}{!}{For 4 Des 1 Cos 3 Int 3 Sag 1 Car 2}
	\item[\textbf{Punti Ferita:}] 163,  \textbf{Difesa:} 23,  \textbf{Iniziativa:} +3
	\item[\textbf{Movimento:}] 12 m, nuoto 12 m, volo 24 m
	\item[\textbf{Tiri Salvez.:}] \resizebox{0.5\linewidth+1.8cm}{!}{\resizebox{0.5\linewidth+1.8cm}{!}{Tempra +11, Riflessi +9, Volontà +9}}
	\item[\textbf{Comp.:}] Furtività +4, Ingannare +5, Consapevolezza +7
	\item[\textbf{Imm. Danni:}] Veleno
	\item[\textbf{Sensi:}] Scurovisione 36 m, Vista Cieca 9 m
	\item[\textbf{Linguaggi:}] Comune, Draconico
	\item[\textbf{Sfida:}] 8 (3900 PX)\smallskip
\end{description}

\emph{\textbf{Anfibio.}} Il drago può respirare aria e acqua.

\textbf{Azioni}

\emph{\textbf{Multiattacco.}} Il drago può effettuare tre attacchi: uno con il morso e due con gli artigli.

\emph{\textbf{Artiglio.} Attacco con arma da mischia}: +9 a colpire, portata 1 m, un bersaglio.

\emph{Colpisce:} 11 (2d6 + 4) danni taglienti, 1 danno da Sanguinamento.

\emph{\textbf{Morso.} Attacco con arma da mischia}: +9 a colpire, portata 3 m, un bersaglio.

\emph{Colpisce:} 15 (2d10 + 4) danni perforanti più 7 (2d6) danni da veleno.

\emph{\textbf{Soffio Velenoso (Ricarica 5-6).}} Il drago esala gas velenosi in un cono di 9 metri. Ogni creatura in quell'area deve effettuare un Tiro Salvezza di Tempra DC 20 e subire 42 (12d6) danni da veleno se fallisce il Tiro Salvezza, o la metà di questi danni se lo riesce.

\emph{\textbf{Arrabbiato:}} il Drago Verde Giovane ricarica il suo soffio Velenoso.

\textbf{Ecologia}\\
Ambiente: Foreste Temperate\\
Organizzazione: Solitario\\
\textbf{Categoria Tesoro}: D\\
\textbf{Descrizione}\\
Vedi Descrizione Drago Verde Antico.\\
\textbf{Incantesimi}\index{Incantesimi da Drago Verde}\\
Gli incantesimi preferiti di questo Drago sono:\\
- \hyperlink{Guscio Anti-Vita}{Guscio Anti-Vita}\\
- \hyperlink{Localizza Animali e Piante}{Localizza Animali e Piante}\\
- \hyperlink{Rimuovi Veleno}{Rimuovi Veleno}

\mostro{Drago Verde Cucciolo}
\noindent
\begin{description}[noitemsep, topsep=0pt, parsep=0pt, partopsep=0pt, leftmargin=0cm, labelwidth=2.2cm]
	\item[\textbf{Taglia/Tipo:}] Media drago, malvagio
	\item[\textbf{Caratt.:}] \resizebox{0.5\linewidth+1.8cm}{!}{For 2 Des 1 Cos 1 Int 2 Sag 0 Car 1}
	\item[\textbf{Punti Ferita:}] 51,  \textbf{Difesa:} 15,  \textbf{Iniziativa:} +2
	\item[\textbf{Movimento:}] 9 m, nuoto 9 m, volo 18 m
	\item[\textbf{Tiri Salvez.:}] \resizebox{0.5\linewidth+1.8cm}{!}{Tempra +3, Riflessi +3, Volontà +3}
	\item[\textbf{Comp.:}] Furtività +3, Consapevolezza +4
	\item[\textbf{Imm. Danni:}] Veleno
	\item[\textbf{Sensi:}] Scurovisione 18 m, Vista Cieca 3 m
	\item[\textbf{Linguaggi:}] Draconico
	\item[\textbf{Sfida:}] 2 (450 PX)\smallskip
\end{description}

\emph{\textbf{Anfibio.}} Il drago può respirare aria e acqua.

\textbf{Azioni}

\emph{\textbf{Morso.} Attacco con arma da mischia}: +5 a colpire, portata 1 m, un bersaglio.

\emph{Colpisce:} 7 (1d10 + 2) danni perforanti più 3 (1d6) danni da veleno.

\emph{\textbf{Soffio Velenoso (Ricarica 5-6).}} Il drago esala gas velenosi in un cono di 5 metri. Ogni creatura in quell'area deve effettuare un Tiro Salvezza di Tempra DC 13 e subire 21 (6d6) danni da veleno se fallisce il Tiro Salvezza, o la metà di questi danni se lo riesce.

\textbf{Ecologia}\\
Ambiente: Foreste Temperate\\
Organizzazione: Solitario\\
\textbf{Categoria Tesoro}: C\\
\textbf{Descrizione}\\
Vedi Descrizione Drago Verde Antico.

\medskip

\rule{\linewidth}{2pt}

\medskip

\textbf{Draghi di Ljust}

\pdfbookmark[3]{Draghi di Ljust}{Draghi di Ljust}

Pochissimi draghi buoni o di Ljust come vengono chiamati, sono presenti nella Terra.
Elysan è probabilmente il più noto e potente, un antico drago d'argento.

\mostro{Drago di Argento Antico}
\noindent
\begin{description}[noitemsep, topsep=0pt, parsep=0pt, partopsep=0pt, leftmargin=0cm, labelwidth=2.2cm]
	\item[\textbf{Taglia/Tipo:}] Mastodontica drago, buono
	\item[\textbf{Caratt.:}] \resizebox{0.5\linewidth+1.8cm}{!}{For 10 Des 0 Cos 9 Int 4 Sag 2 Car 6}
	\item[\textbf{Punti Ferita:}] 470,  \textbf{Difesa:} 42,  \textbf{Iniziativa:} +4
	\item[\textbf{Movimento:}] 12 m, volo 24 m
	\item[\textbf{Tiri Salvez.:}] \resizebox{0.5\linewidth+1.8cm}{!}{\resizebox{0.5\linewidth+1.8cm}{!}{Tempra +32, Riflessi +23, Volontà +25}}
	\item[\textbf{Comp.:}] Arcana +11, Furtività +7, Consapevolezza +16, Storia +11
	\item[\textbf{Imm. Danni:}] Freddo, armi +1
	\item[\textbf{Sensi:}] Scurovisione 36 m, Vista Cieca 18 m
	\item[\textbf{Linguaggi:}] Comune, Draconico
	\item[\textbf{Sfida:}] 23 (50000 PX)\smallskip
\end{description}

\emph{\textbf{Aura rallentante.}} il drago emette nel raggio di 3 metri una aura magica che causa Rallentato 1.

\emph{\textbf{Resistenza Leggendaria (3/Giorno).}} Se il drago fallisce un Tiro Salvezza, può scegliere invece di riuscire.

\emph{\textbf{Resistenza alla Magia:}} 3lv

\textbf{Azioni}

\emph{\textbf{Multiattacco.}} Il drago può usare la sua Presenza Spaventosa e poi effettuare tre attacchi: uno con il morso e due con gli artigli.

\emph{\textbf{Artiglio.} Attacco con arma da mischia}: +17 a colpire, portata 3 m, un bersaglio.

\emph{Colpisce:} 17 (2d6 + 10) danni taglienti, 3/20 danno da Sanguinamento.

\emph{\textbf{Coda.} Attacco con arma da mischia}: +17 a colpire, portata 6 m, un bersaglio.

\emph{Colpisce:} 19 (2d8 + 10) danni contundenti.

\emph{\textbf{Morso.} Attacco con arma da mischia}: +17 a colpire, portata 5 metri, un bersaglio.

\emph{Colpisce:} 21 (2d10 + 10) danni perforanti.

\emph{\textbf{Presenza Spaventosa.}} Ogni creatura scelta dal drago, che si trovi entro 36 metri da esso e consapevole della sua presenza, deve riuscire un Tiro Salvezza di Volontà DC 36 o restare spaventata per 1 minuto. Una creatura può ripetere il Tiro Salvezza al termine di ciascun suo round, terminando l'effetto se lo riesce. Se il Tiro Salvezza della creatura ha successo o l'effetto ha termine per essa, la creatura è immune alla Presenza Spaventosa del drago per le successive 24 ore.

\emph{\textbf{Arma a Soffio (Ricarica 5-6).}} Il drago usa una delle seguenti armi a soffio:

\emph{Soffio Gelido.} Il drago esala un'esplosione ghiacciata in un cono di 27 metri. Ogni creatura nell'area deve effettuare un Tiro Salvezza su Tempra DC 36, subendo 67 (15d8) danni da freddo se fallisce il Tiro Salvezza, o la metà di questi danni se lo riesce.

\emph{Soffio Paralizzante.} Il drago esala un gas paralizzante in un cono di 23 metri. Ogni creatura nell'area deve riuscire un Tiro Salvezza su Tempra 36 o restare paralizzata per 1 minuto. Una creatura può ripetere il Tiro Salvezza al termine di ciascun suo round, terminando l'effetto per sé in caso di successo.

\emph{\textbf{Mutare Forma.}} Il drago può trasformarsi magicamente in un umanoide o bestia il cui grado di sfida sia pari o inferiore al proprio, o tornare alla sua vera forma. Alla morte ritorna alla sua vera forma.

Qualsiasi equipaggiamento stia indossando o trasportando viene assorbito o trasportato nella nuova forma (a scelta del drago).

Nella nuova forma, il drago mantiene i suoi Tratti, Punti Ferita, la facoltà di parlare, le competenze, la Resistenza Leggendaria, le azioni da tana, e i punteggi di Intelligenza, Saggezza e Carisma, oltre a questa azione. Le sue statistiche e capacità vengono altrimenti rimpiazzate da quelle della nuova forma, eccetto Azioni aggiuntive della nuova forma.

\textbf{Azioni Aggiuntive}

Il drago può effettuare 3 Azioni aggiuntive, scelte tra le opzioni seguenti. Può usare solo un'opzione Aggiuntiva alla volta e solo al termine del round di un'altra creatura. Il drago recupera le Azioni aggiuntive spese all'inizio del proprio round.

\textbf{Attacco di Ala (Costa 2 Azioni).} Il drago batte le ali. Ogni creatura entro 5 metri dal drago deve riuscire un Tiro Salvezza su Riflessi DC 36 o subire 17 (2d6 + 10) danni contundenti e venir gettato prono. Il drago può poi volare fino a metà della sua velocità di volo.

\textbf{Attacco di Coda.} Il drago effettua un attacco di coda.

\textbf{Individuare.} Il drago effettua una prova di Consapevolezza.

\textbf{Ecologia}\\
Ambiente: Montagne Temperate\\
Organizzazione: Solitario\\
\textbf{Categoria Tesoro}: H\\
\textbf{Descrizione}\\
Tra tutti i draghi, quelli d'argento sono i più coraggiosi, e si attengono ad un codice cavalleresco che impone loro di aiutare i deboli, sconfiggere il male e comportarsi in modo onorevole.\\
\textbf{Incantesimi}\index{Incantesimi da Drago Argento}\\
Gli incantesimi preferiti di questo Drago sono:\\
- \hyperlink{lentezza}{Lentezza}\\
- \hyperlink{Fabbricare}{Fabbricare}\\
- \hyperlink{Sogno}{Sogno}

%\begin{center}
%\includegraphics[width=0.9\linewidth]{immagini/silver.png}
%
%\textit{Argento, grezzo}
%\end{center}

\mostro{Drago di Argento Adulto}
\noindent
\begin{description}[noitemsep, topsep=0pt, parsep=0pt, partopsep=0pt, leftmargin=0cm, labelwidth=2.2cm]
	\item[\textbf{Taglia/Tipo:}] Enorme drago, buono
	\item[\textbf{Caratt.:}] \resizebox{0.5\linewidth+1.8cm}{!}{For 8 Des 0 Cos 7 Int 3 Sag 1 Car 5}
	\item[\textbf{Punti Ferita:}] 325,  \textbf{Difesa:} 33,  \textbf{Iniziativa:} +3
	\item[\textbf{Movimento:}] 12 m, volo 24 m
	\item[\textbf{Tiri Salvez.:}] \resizebox{0.5\linewidth+1.8cm}{!}{\resizebox{0.5\linewidth+1.8cm}{!}{Tempra +23, Riflessi +16, Volontà +17}}
	\item[\textbf{Comp.:}] Arcana +8, Furtività +5, Consapevolezza +11, Storia +8
	\item[\textbf{Imm. Danni:}] Freddo
	\item[\textbf{Sensi:}] Scurovisione 36 m, Vista Cieca 18 m
	\item[\textbf{Linguaggi:}] Comune, Draconico
	\item[\textbf{Sfida:}] 16 (15000 PX)\smallskip
\end{description}

\emph{\textbf{Resistenza Leggendaria (3/Giorno).}} Se il drago fallisce un Tiro Salvezza, può scegliere invece di riuscire.

\textbf{Azioni}

\emph{\textbf{Multiattacco.}} Il drago può usare la sua Presenza Spaventosa e poi effettuare tre attacchi: uno con il morso e due con gli artigli.

\emph{\textbf{Artiglio.} Attacco con arma da mischia}: +14 a colpire, portata 1 m, un bersaglio.

\emph{Colpisce:} 15 (2d6 + 8) danni taglienti, 1 danno da Sanguinamento.

\emph{\textbf{Coda.} Attacco con arma da mischia}: +14 a colpire, portata 5 metri, un bersaglio.

\emph{Colpisce:} 17 (2d8 + 8) danni contundenti.

\emph{\textbf{Morso.} Attacco con arma da mischia}: +14 a colpire, portata 3 m, un bersaglio.

\emph{Colpisce:} 19 (2d10 + 8) danni perforanti.

\emph{\textbf{Presenza Spaventosa.}} Ogni creatura scelta dal drago, che si trovi entro 36 metri da esso e consapevole della sua presenza, deve riuscire un Tiro Salvezza di Volontà DC 28 o restare spaventata per 1 minuto. Una creatura può ripetere il Tiro Salvezza al termine di ciascun suo round, terminando l'effetto se lo riesce. Se il Tiro Salvezza della creatura ha successo o l'effetto ha termine per essa, la creatura è immune alla Presenza Spaventosa del drago per le successive 24 ore.

\emph{\textbf{Arma a Soffio (Ricarica 5-6).}} Il drago usa una delle seguenti armi a soffio:

\emph{Soffio Gelido.} Il drago esala un'esplosione ghiacciata in un cono di 18 metri. Ogni creatura nell'area deve effettuare un Tiro Salvezza su Tempra DC 28, subendo 58 (13d8) danni da freddo se fallisce il Tiro Salvezza, o la metà di questi danni se lo riesce.

\emph{Soffio Paralizzante.} Il drago esala un gas paralizzante in un cono di 18 metri. Ogni creatura nell'area deve riuscire un Tiro Salvezza su Tempra 28 o restare paralizzata per 1 minuto. Una creatura può ripetere il Tiro Salvezza al termine di ciascun suo round, terminando l'effetto per sé in caso di successo.

\emph{\textbf{Mutare Forma.}} Il drago può trasformarsi magicamente in un umanoide o bestia il cui grado di sfida sia pari o inferiore al proprio, o tornare alla sua vera forma. Alla morte ritorna alla sua vera forma. Qualsiasi equipaggiamento stia indossando o trasportando viene assorbito o trasportato nella nuova forma (a scelta del drago).

Nella nuova forma, il drago mantiene i suoi Tratti, Punti Ferita la facoltà di parlare, le competenze, la Resistenza Leggendaria, le azioni da tana, e i punteggi di Intelligenza, Saggezza e Carisma, oltre a questa azione. Le sue statistiche e capacità vengono altrimenti rimpiazzate da quelle della nuova forma, eccetto Azioni aggiuntive della nuova forma.

\textbf{Azioni Aggiuntive}

Il drago può effettuare 3 Azioni aggiuntive, scelte tra le opzioni seguenti. Può usare solo un'opzione Aggiuntiva alla volta e solo al termine del round di un'altra creatura. Il drago recupera le Azioni aggiuntive spese all'inizio del proprio round.

\textbf{Attacco di Ala (Costa 2 Azioni).} Il drago batte le ali. Ogni creatura entro 3 metri dal drago deve riuscire un Tiro Salvezza su Riflessi DC 28 o subire 15 (2d6 + 8) danni contundenti e venir gettato prono. Il drago può poi volare fino a metà del suo movimento di volo.

\textbf{Attacco di Coda.} Il drago effettua un attacco di coda.

\textbf{Individuare.} Il drago effettua una prova di Consapevolezza.

\emph{\textbf{Arrabbiato:}} Il Drago d'Argento Adulto può eseguire queste azioni a costo 2 Azioni.

\emph{Focalizzare}: la creatura interrompe un effetto mentale su di se in corso

\emph{Brutalità}: la creatura attacca con ferocia inaudita. +1d6 al Tiro per Colpire, 1 danno critico automatico quando colpisce.

\textbf{Ecologia}\\
Ambiente: Montagne Temperate\\
Organizzazione: Solitario\\
\textbf{Categoria Tesoro}: E\\
\textbf{Descrizione}\\
Tra tutti i draghi, quelli d'argento sono i più coraggiosi, e si attengono ad un codice cavalleresco che impone loro di aiutare i deboli, sconfiggere il male e comportarsi in modo onorevole.\\
\textbf{Incantesimi}\index{Incantesimi da Drago Argento}\\
Gli incantesimi preferiti di questo Drago sono:\\
- \hyperlink{lentezza}{Lentezza}\\
- \hyperlink{Fabbricare}{Fabbricare}\\
- \hyperlink{Sogno}{Sogno}

\medskip

\begin{center}
	\includegraphics[width=0.9\linewidth]{immagini/Dragon_Ljubljana.png}

	\emph{Dragon Bridge, Ljubljana}
\end{center}

\mostro{Drago di Argento Giovane}
\noindent
\begin{description}[noitemsep, topsep=0pt, parsep=0pt, partopsep=0pt, leftmargin=0cm, labelwidth=2.2cm]
	\item[\textbf{Taglia/Tipo:}] Grande drago, buono
	\item[\textbf{Caratt.:}] \resizebox{0.5\linewidth+1.8cm}{!}{For 6 Des 0 Cos 5 Int 2 Sag 0 Car 4}
	\item[\textbf{Punti Ferita:}] 186,  \textbf{Difesa:} 24,  \textbf{Iniziativa:} +2
	\item[\textbf{Movimento:}] 12 m, volo 24 m
	\item[\textbf{Tiri Salvez.:}] \resizebox{0.5\linewidth+1.8cm}{!}{\resizebox{0.5\linewidth+1.8cm}{!}{Tempra +14, Riflessi +9, Volontà +9}}
	\item[\textbf{Comp.:}] Arcana +6, Furtività +4, Consapevolezza +8, Storia +6
	\item[\textbf{Imm. Danni:}] Freddo
	\item[\textbf{Sensi:}] Scurovisione 36 m, Vista Cieca 9 m
	\item[\textbf{Linguaggi:}] Comune, Draconico
	\item[\textbf{Sfida:}] 9 (5000 PX)\smallskip
\end{description}

\textbf{Azioni}

\emph{\textbf{Multiattacco.}} Il drago può effettuare tre attacchi: uno con il morso e due con gli artigli.

\emph{\textbf{Artiglio.} Attacco con arma da mischia}: +10 a colpire, portata 1 m, un bersaglio.

\emph{Colpisce:} 13 (2d6 + 6) danni taglienti, 1 danno da Sanguinamento.

\emph{\textbf{Morso.} Attacco con arma da mischia}: +10 a colpire, portata 3 m, un bersaglio.

\emph{Colpisce:} 17 (2d10 + 6) danni perforanti.

\emph{\textbf{Arma a Soffio (Ricarica 5-6).}} Il drago usa una delle seguenti armi a soffio:

\emph{Soffio Gelido.} Il drago esala un'esplosione ghiacciata in un cono di 9 metri. Ogni creatura nell'area deve effettuare un Tiro Salvezza su Tempra DC 21, subendo 54 (12d8) danni da freddo se fallisce il Tiro Salvezza, o la metà di questi danni se lo riesce.

\emph{Soffio Paralizzante.} Il drago esala un gas paralizzante in un cono di 9 metri. Ogni creatura nell'area deve riuscire un Tiro Salvezza su Tempra 21 o restare paralizzata per 1 minuto. Una creatura può ripetere il Tiro Salvezza al termine di ciascun suo round, terminando l'effetto per sé in caso di successo.

\emph{\textbf{Arrabbiato:}} il giovane drago d'argento ricarica uno dei suoi soffi.

\textbf{Ecologia}\\
Ambiente: Montagne Temperate\\
Organizzazione: Solitario\\
\textbf{Categoria Tesoro}: D\\
\textbf{Descrizione}\\
Tra tutti i draghi, quelli d'argento sono i più coraggiosi, e si attengono ad un codice cavalleresco che impone loro di aiutare i deboli, sconfiggere il male e comportarsi in modo onorevole.\\
\textbf{Incantesimi}\index{Incantesimi da Drago Argento}\\
Gli incantesimi preferiti di questo Drago sono:\\
- \hyperlink{lentezza}{Lentezza}\\
- \hyperlink{Fabbricare}{Fabbricare}\\
- \hyperlink{Sogno}{Sogno}

\mostro{Drago di Argento Cucciolo}
\noindent
\begin{description}[noitemsep, topsep=0pt, parsep=0pt, partopsep=0pt, leftmargin=0cm, labelwidth=2.2cm]
	\item[\textbf{Taglia/Tipo:}] Media drago, buono
	\item[\textbf{Caratt.:}] \resizebox{0.5\linewidth+1.8cm}{!}{For 4 Des 0 Cos 3 Int 1 Sag 0 Car 2}
	\item[\textbf{Punti Ferita:}] 52,  \textbf{Difesa:} 14,  \textbf{Iniziativa:} +1
	\item[\textbf{Movimento:}] 9 m, volo 18 m
	\item[\textbf{Tiri Salvez.:}] \resizebox{0.5\linewidth+1.8cm}{!}{Tempra +5, Riflessi +3, Volontà +3}
	\item[\textbf{Comp.:}] Furtività +2, Consapevolezza +4
	\item[\textbf{Imm. Danni:}] Freddo
	\item[\textbf{Sensi:}] Scurovisione 18 m, Vista Cieca 3 m
	\item[\textbf{Linguaggi:}] Draconico
	\item[\textbf{Sfida:}] 2 (450 PX)\smallskip
\end{description}

\textbf{Azioni}

\emph{\textbf{Morso.} Attacco con arma da mischia}: +5 a colpire, portata 1 m, un bersaglio.

\emph{Colpisce:} 9 (1d10 + 4) danni perforanti.

\emph{\textbf{Arma a Soffio (Ricarica 5-6).}} Il drago usa una delle seguenti armi a soffio:

\emph{Soffio Gelido.} Il drago esala un'esplosione ghiacciata in un cono di 5 metri. Ogni creatura nell'area deve effettuare un Tiro Salvezza su Tempra 14, subendo 18 (4d8) danni da freddo se fallisce il Tiro Salvezza, o la metà di questi danni se lo riesce.

\emph{Soffio Paralizzante.} Il drago esala un gas paralizzante in un cono di 5 metri. Ogni creatura nell'area deve riuscire un Tiro Salvezza su Tempra 14 o restare paralizzata per 1 minuto. Una creatura può ripetere il Tiro Salvezza al termine di ciascun suo round, terminando l'effetto per sé in caso di successo.

\textbf{Ecologia}\\
Ambiente: Montagne Temperate\\
Organizzazione: Solitario\\
\textbf{Categoria Tesoro}: C\\
\textbf{Descrizione}\\
Tra tutti i draghi, quelli d'argento sono i più coraggiosi, e si attengono ad un codice cavalleresco che impone loro di aiutare i deboli, sconfiggere il male e comportarsi in modo onorevole.

\mostro{Drago di Bronzo Antico}
\noindent
\begin{description}[noitemsep, topsep=0pt, parsep=0pt, partopsep=0pt, leftmargin=0cm, labelwidth=2.2cm]
	\item[\textbf{Taglia/Tipo:}] Mastodontica drago, buono
	\item[\textbf{Caratt.:}] \resizebox{0.5\linewidth+1.8cm}{!}{For 9 Des 0 Cos 8 Int 4 Sag 3 Car 5}
	\item[\textbf{Punti Ferita:}] 446,  \textbf{Difesa:} 41,  \textbf{Iniziativa:} +4
	\item[\textbf{Movimento:}] 12 m, nuoto 12 m, volo 24 m
	\item[\textbf{Tiri Salvez.:}] \resizebox{0.5\linewidth+1.8cm}{!}{\resizebox{0.5\linewidth+1.8cm}{!}{Tempra +30, Riflessi +22, Volontà +25}}
	\item[\textbf{Comp.:}] Furtività +7, Percepire Emozioni +10, Consapevolezza +17
	\item[\textbf{Imm. Danni:}] Elettricità, armi +1
	\item[\textbf{Sensi:}] Scurovisione 36 m, Vista Cieca 18 m
	\item[\textbf{Linguaggi:}] Comune, Draconico
	\item[\textbf{Sfida:}] 22 (41000 PX)\smallskip
\end{description}

\emph{\textbf{Aura repulsiva.}} il drago emette nel raggio di 3 metri un aura che disturba le creature. Ogni attacco portato subisce una penalità di 3 - distanza attacco.

\emph{\textbf{Anfibio.}} Il drago può respirare aria e acqua.

\emph{\textbf{Resistenza Leggendaria (3/Giorno).}} Se il drago fallisce un Tiro Salvezza, può scegliere invece di riuscire.

\emph{\textbf{Resistenza alla Magia:}} 3lv

\textbf{Azioni}

\emph{\textbf{Multiattacco.}} Il drago può usare la sua Presenza Spaventosa e poi effettuare tre attacchi: uno con il morso e due con gli artigli.

\emph{\textbf{Artiglio.} Attacco con arma da mischia}: +17 a colpire, portata 3 m, un bersaglio.

\emph{Colpisce:} 16 (2d6 + 9) danni taglienti, 3/20 danno da Sanguinamento.

\emph{\textbf{Coda.} Attacco con arma da mischia}: +17 a colpire, portata 6 m, un bersaglio.

\emph{Colpisce:} 18 (2d8 + 9) danni contundenti.

\emph{\textbf{Morso.} Attacco con arma da mischia}: +17 a colpire, portata 5 metri, un bersaglio.

\emph{Colpisce:} 20 (2d10 + 9) danni perforanti.

\emph{\textbf{Presenza Spaventosa.}} Ogni creatura scelta dal drago, che si trovi entro 36 metri da esso e consapevole della sua presenza, deve riuscire un Tiro Salvezza di Volontà DC 35 o restare spaventata per 1 minuto. Una creatura può ripetere il Tiro Salvezza al termine di ciascun suo round, terminando l'effetto se lo riesce. Se il Tiro Salvezza della creatura ha successo o l'effetto ha termine per essa, la creatura è immune alla Presenza Spaventosa del drago per le successive 24 ore.

\emph{\textbf{Arma a Soffio (Ricarica 5-6).}} Il drago usa una delle seguenti armi a soffio:

\emph{Soffio Fulminante.} Il drago esala fulmini in una linea lunga 36 metri e larga 3 metri. Ogni creatura sulla linea deve effettuare un Tiro Salvezza su Riflessi DC 35, subendo 88 (16d10) danni da elettricità se fallisce il Tiro Salvezza, o la metà di questi danni se lo riesce.

\emph{Soffio Repulsivo.} Il drago esala dell'energia repulsiva in un cono di 9 metri. Ogni creatura in quell'area deve riuscire un Tiro Salvezza su Tempra DC 35, altrimenti viene allontana di 18 metri dal drago.

\emph{\textbf{Mutare Forma.}} Il drago può trasformarsi magicamente in un umanoide o bestia il cui grado di sfida sia pari o inferiore al proprio, o tornare alla sua vera forma. Alla morte ritorna alla sua vera forma. Qualsiasi equipaggiamento stia indossando o trasportando viene assorbito o trasportato nella nuova forma (a scelta del drago).

Nella nuova forma, il drago mantiene i suoi Tratti, Punti Ferita la facoltà di parlare, le competenze, la Resistenza Leggendaria, le azioni da tana, e i punteggi di Intelligenza, Saggezza e Carisma, oltre a questa azione. Le sue statistiche e capacità vengono altrimenti rimpiazzate da quelle della nuova forma, eccetto Azioni aggiuntive della nuova forma.

\textbf{Azioni Aggiuntive}

Il drago può effettuare 3 Azioni aggiuntive, scelte tra le opzioni seguenti. Può usare solo un'opzione Aggiuntiva alla volta e solo al termine del round di un'altra creatura. Il drago recupera le Azioni aggiuntive spese all'inizio del proprio round.

\textbf{Attacco di Ala (Costa 2 Azioni).} Il drago batte le ali. Ogni creatura entro 5 metri dal drago deve riuscire un Tiro Salvezza su Riflessi DC 35 o subire 16 (2d6 + 9) danni contundenti e venir gettato prono. Il drago può poi volare fino a metà del suo movimento di volo.

\textbf{Attacco di Coda.} Il drago effettua un attacco di coda.

\textbf{Individuare.} Il drago effettua una prova di Consapevolezza.

\textbf{Ecologia}\\
Ambiente: Zone Costiere Temperate\\
Organizzazione: Solitario\\
\textbf{Categoria Tesoro}: H\\
\textbf{Descrizione}\\
I draghi di bronzo sono noti per allearsi con viaggiatori ed avventurieri se causa e ricompensa sono giuste e adeguate\\
\textbf{Incantesimi}\index{Incantesimi da Drago Bronzo}\\
Gli incantesimi preferiti di questo Drago sono:\\
- \hyperlink{Globo di Invulnerabilità}{Globo di Invulnerabilità}\\
- \hyperlink{Libertà di Movimento}{Libertà di Movimento}

\mostro{Drago di Bronzo Adulto}
\noindent
\begin{description}[noitemsep, topsep=0pt, parsep=0pt, partopsep=0pt, leftmargin=0cm, labelwidth=2.2cm]
	\item[\textbf{Taglia/Tipo:}] Enorme drago, buono
	\item[\textbf{Caratt.:}] \resizebox{0.5\linewidth+1.8cm}{!}{For 7 Des 0 Cos 6 Int 3 Sag 2 Car 4}
	\item[\textbf{Punti Ferita:}] 303,  \textbf{Difesa:} 32,  \textbf{Iniziativa:} +3
	\item[\textbf{Movimento:}] 12 m, nuoto 12 m, volo 24 m
	\item[\textbf{Tiri Salvez.:}] \resizebox{0.5\linewidth+1.8cm}{!}{\resizebox{0.5\linewidth+1.8cm}{!}{Tempra +21, Riflessi +15, Volontà +17}}
	\item[\textbf{Comp.:}] Furtività +5, Percepire Emozioni +7, Consapevolezza +12
	\item[\textbf{Imm. Danni:}] Elettricità
	\item[\textbf{Sensi:}] Scurovisione 36 m, Vista Cieca 18 m
	\item[\textbf{Linguaggi:}] Comune, Draconico
	\item[\textbf{Sfida:}] 15 (13000 PX)\smallskip
\end{description}

\emph{\textbf{Anfibio.}} Il drago può respirare aria e acqua.

\emph{\textbf{Resistenza Leggendaria (3/Giorno).}} Se il drago fallisce un Tiro Salvezza, può scegliere invece di riuscire.

\textbf{Azioni}

\emph{\textbf{Multiattacco.}} Il drago può usare la sua Presenza Spaventosa e poi effettuare tre attacchi: uno con il morso e due con gli artigli.

\emph{\textbf{Artiglio.} Attacco con arma da mischia}: +13 a colpire, portata 1 m, un bersaglio.

\emph{Colpisce:} 14 (2d6 + 7) danni taglienti, 1 danno da Sanguinamento.

\emph{\textbf{Coda.} Attacco con arma da mischia}: +13 a colpire, portata 5 metri, un bersaglio.

\emph{Colpisce:} 16 (2d8 + 7) danni contundenti.

\emph{\textbf{Morso.} Attacco con arma da mischia}: +13 a colpire, portata 3 m, un bersaglio.

\emph{Colpisce:} 18 (2d10 + 7) danni perforanti.

\emph{\textbf{Presenza Spaventosa.}} Ogni creatura scelta dal drago, che si trovi entro 36 metri da esso e consapevole della sua presenza, deve riuscire un Tiro Salvezza di Volontà DC 29 o restare spaventata per 1 minuto. Una creatura può ripetere il Tiro Salvezza al termine di ciascun suo round, terminando l'effetto se lo riesce. Se il Tiro Salvezza della creatura ha successo o l'effetto ha termine per essa, la creatura è immune alla Presenza Spaventosa del drago per le successive 24 ore.

\emph{\textbf{Arma a Soffio (Ricarica 5-6).}} Il drago usa una delle seguenti armi a soffio:

\emph{Soffio Fulminante.} Il drago esala fulmini in una linea lunga 27 metri e larga 1 metro. Ogni creatura sulla linea deve effettuare un Tiro Salvezza di Riflessi DC 29, subendo 66 (12d10) danni da elettricità se fallisce il Tiro Salvezza, o la metà di questi danni se lo riesce.

\emph{Soffio Repulsivo.} Il drago esala dell'energia repulsiva in un cono di 9 metri. Ogni creatura in quell'area deve riuscire un Tiro Salvezza su Tempra DC 29, altrimenti viene allontana di 18 metri dal drago.

\emph{\textbf{Mutare Forma.}} Il drago può trasformarsi magicamente in un umanoide o bestia il cui grado di sfida sia pari o inferiore al proprio, o tornare alla sua vera forma. Alla morte ritorna alla sua vera forma. Qualsiasi equipaggiamento stia indossando o trasportando viene assorbito o trasportato nella nuova forma (a scelta del drago).

Nella nuova forma, il drago mantiene i suoi Tratti, Punti Ferita la facoltà di parlare, le competenze, la Resistenza Leggendaria, le azioni da tana, e i punteggi di Intelligenza, Saggezza e Carisma, oltre a questa azione. Le sue statistiche e capacità vengono altrimenti rimpiazzate da quelle della nuova forma, eccetto Azioni aggiuntive della nuova forma.

\textbf{Azioni Aggiuntive}

Il drago può effettuare 3 Azioni aggiuntive, scelte tra le opzioni seguenti. Può usare solo un'opzione Aggiuntiva alla volta e solo al termine del round di un'altra creatura. Il drago recupera le Azioni aggiuntive spese all'inizio del proprio round.

\textbf{Attacco di Ala (Costa 2 Azioni).} Il drago batte le ali. Ogni creatura entro 3 metri dal drago deve riuscire un Tiro Salvezza su Riflessi DC 29 o subire 14 (2d6 + 7) danni contundenti e venir gettato prono. Il drago può poi volare fino a metà del suo movimento di volo.

\textbf{Attacco di Coda.} Il drago effettua un attacco di coda.

\textbf{Individuare.} Il drago effettua una prova di Consapevolezza.

\emph{\textbf{Arrabbiato:}} Il Drago di Bronzo Adulto può eseguire queste azioni a costo 2 Azioni.

\emph{Focalizzare}: la creatura interrompe un effetto mentale su di se in corso

\emph{Brutalità}: la creatura attacca con ferocia inaudita. +1d6 al Tiro per Colpire, 1 danno critico automatico quando colpisce.

\textbf{Ecologia}\\
Ambiente: Zone Costiere Temperate\\
Organizzazione: Solitario\\
\textbf{Categoria Tesoro}: E\\
\textbf{Descrizione}\\
I draghi di bronzo sono noti per allearsi con viaggiatori ed avventurieri se causa e ricompensa sono giuste e adeguate\\
\textbf{Incantesimi}\index{Incantesimi da Drago Bronzo}\\
Gli incantesimi preferiti di questo Drago sono:\\
- \hyperlink{Globo di Invulnerabilità}{Globo di Invulnerabilità}\\
- \hyperlink{Libertà di Movimento}{Libertà di Movimento}

%\begin{center}
%	\includegraphics[width=0.7\linewidth]{immagini/Ancient_bronze_greek_helmet_-South_Italy.png}
%\end{center}

\mostro{Drago di Bronzo Giovane}
\noindent
\begin{description}[noitemsep, topsep=0pt, parsep=0pt, partopsep=0pt, leftmargin=0cm, labelwidth=2.2cm]
	\item[\textbf{Taglia/Tipo:}] Grande drago, buono
	\item[\textbf{Caratt.:}] \resizebox{0.5\linewidth+1.8cm}{!}{For 5 Des 0 Cos 4 Int 2 Sag 1 Car 3}
	\item[\textbf{Punti Ferita:}] 165,  \textbf{Difesa:} 22,  \textbf{Iniziativa:} +2
	\item[\textbf{Movimento:}] 12 m, nuoto 12 m, volo 24 m
	\item[\textbf{Tiri Salvez.:}] \resizebox{0.5\linewidth+1.8cm}{!}{\resizebox{0.5\linewidth+1.8cm}{!}{Tempra +12, Riflessi +8, Volontà +9}}
	\item[\textbf{Comp.:}] Furtività +3, Percepire Emozioni +4, Consapevolezza +7
	\item[\textbf{Imm. Danni:}] Elettricità
	\item[\textbf{Sensi:}] Scurovisione 36 m, Vista Cieca 9 m
	\item[\textbf{Linguaggi:}] Comune, Draconico
	\item[\textbf{Sfida:}] 8 (3900 PX)\smallskip
\end{description}

\emph{\textbf{Anfibio.}} Il drago può respirare aria e acqua.

\textbf{Azioni}

\emph{\textbf{Multiattacco.}} Il drago può usare effettuare tre attacchi: uno con il morso e due con gli artigli.

\emph{\textbf{Artiglio.} Attacco con arma da mischia}: +10 a colpire, portata 1 m, un bersaglio.

\emph{Colpisce:} 12 (2d6 + 5) danni taglienti, 1 danno da Sanguinamento.

\emph{\textbf{Morso.} Attacco con arma da mischia}: +10 a colpire, portata 3 m, un bersaglio.

\emph{Colpisce:} 16 (2d10 + 5) danni perforanti.

\emph{\textbf{Arma a Soffio (Ricarica 5-6).}} Il drago usa una delle seguenti armi a soffio:

\emph{Soffio Fulminante.} Il drago esala fulmini in una linea lunga 18 metri e larga 1 metro. Ogni creatura sulla linea deve effettuare un Tiro Salvezza di Riflessi DC 20, subendo 55 (10d10) danni da elettricità se fallisce il Tiro Salvezza, o la metà di questi danni se lo riesce.

\emph{Soffio Repulsivo.} Il drago esala dell'energia repulsiva in un cono di 9 metri. Ogni creatura in quell'area deve riuscire un Tiro Salvezza su Tempra DC 20, altrimenti viene allontana di 12 metri dal drago.

\emph{\textbf{Arrabbiato:}} il Drago di Bronzo Giovane ricarica uno dei suoi soffi.

\textbf{Ecologia}\\
Ambiente: Zone Costiere Temperate\\
Organizzazione: Solitario\\
\textbf{Categoria Tesoro}: D\\
\textbf{Descrizione}\\
I draghi di bronzo sono noti per allearsi con viaggiatori ed avventurieri se causa e ricompensa sono giuste e adeguate\\
\textbf{Incantesimi}\index{Incantesimi da Drago Bronzo}\\
Gli incantesimi preferiti di questo Drago sono:\\
- \hyperlink{Globo di Invulnerabilità}{Globo di Invulnerabilità}\\
- \hyperlink{Libertà di Movimento}{Libertà di Movimento}

\mostro{Drago di Bronzo Cucciolo}
\noindent
\begin{description}[noitemsep, topsep=0pt, parsep=0pt, partopsep=0pt, leftmargin=0cm, labelwidth=2.2cm]
	\item[\textbf{Taglia/Tipo:}] Media drago, buono
	\item[\textbf{Caratt.:}] \resizebox{0.5\linewidth+1.8cm}{!}{For 3 Des 0 Cos 2 Int 1 Sag 0 Car 2}
	\item[\textbf{Punti Ferita:}] 51,  \textbf{Difesa:} 14,  \textbf{Iniziativa:} +1
	\item[\textbf{Movimento:}] 9 m, nuoto 9 m, volo 18 m
	\item[\textbf{Tiri Salvez.:}] \resizebox{0.5\linewidth+1.8cm}{!}{Tempra +4, Riflessi +3, Volontà +3}
	\item[\textbf{Comp.:}] Furtività +2, Consapevolezza +4
	\item[\textbf{Imm. Danni:}] Elettricità
	\item[\textbf{Sensi:}] Scurovisione 18 m, Vista Cieca 3 m
	\item[\textbf{Linguaggi:}] Draconico
	\item[\textbf{Sfida:}] 2 (450 PX)\smallskip
\end{description}

\emph{\textbf{Anfibio.}} Il drago può respirare aria e acqua.

\textbf{Azioni}

\emph{\textbf{Morso.} Attacco con arma da mischia}: +5 a colpire,
portata 1 m, un bersaglio.

\emph{Colpisce:} 8 (1d10 + 3) danni perforanti.

\emph{\textbf{Arma a Soffio (Ricarica 5-6).}} Il drago usa una delle seguenti armi a soffio:

\emph{Soffio Fulminante.} Il drago esala fulmini in una linea lunga 12 metri e larga 1 metro. Ogni creatura sulla linea deve effettuare un Tiro Salvezza di Riflessi DC 16, subendo 16 (3d10) danni da elettricità se fallisce il Tiro Salvezza, o la metà di questi danni se lo riesce.

\emph{Soffio Repulsivo.} Il drago esala dell'energia repulsiva in un cono di 9 metri. Ogni creatura in quell'area deve riuscire un Tiro Salvezza su Tempra DC 16, altrimenti viene allontana di 9 metri dal drago.

\textbf{Ecologia}\\
Ambiente: Zone Costiere Temperate\\
Organizzazione: Solitario\\
\textbf{Categoria Tesoro}: C\\
\textbf{Descrizione}\\
I draghi di bronzo sono noti per allearsi con viaggiatori ed avventurieri se causa e ricompensa sono giuste e adeguate.

\mostro{Drago d'Oro Antico}
\noindent
\begin{description}[noitemsep, topsep=0pt, parsep=0pt, partopsep=0pt, leftmargin=0cm, labelwidth=2.2cm]
	\item[\textbf{Taglia/Tipo:}] Mastodontica drago, buono
	\item[\textbf{Caratt.:}] \resizebox{0.5\linewidth+1.8cm}{!}{For 10 Des 2 Cos 9 Int 4 Sag 3 Car 9}
	\item[\textbf{Punti Ferita:}] 490,  \textbf{Difesa:} 46,  \textbf{Iniziativa:} +4
	\item[\textbf{Movimento:}] 12 m, nuoto 12 m, volo 24 m
	\item[\textbf{Tiri Salvez.:}] \resizebox{0.5\linewidth+1.8cm}{!}{\resizebox{0.5\linewidth+1.8cm}{!}{Tempra +33, Riflessi +26, Volontà +27}}
	\item[\textbf{Comp.:}] Furtività +9, Percepire Emozioni +10, Consapevolezza +17, Ingannare +16
	\item[\textbf{Imm. Danni:}] Fuoco, armi +1
	\item[\textbf{Sensi:}] Scurovisione 36 m, Vista Cieca 18 m
	\item[\textbf{Linguaggi:}] Comune, Draconico
	\item[\textbf{Sfida:}] 24 (62000 PX)\smallskip
\end{description}

\emph{\textbf{Aura indebolente.}} il drago emette nel raggio di 3 metri un aura che causa Affaticato 2. Rimanere nell'aura non aumenta il livello di affaticato.

\emph{\textbf{Anfibio.}} Il drago può respirare aria e acqua.

\emph{\textbf{Resistenza Leggendaria (3/Giorno).}} Se il drago fallisce un Tiro Salvezza, può scegliere invece di riuscire.

\emph{\textbf{Resistenza alla Magia:}} 3lv

\textbf{Azioni}

\emph{\textbf{Multiattacco.}} Il drago può usare la sua Presenza Spaventosa e poi effettuare tre attacchi: uno con il morso e due con gli artigli.

\emph{\textbf{Artiglio.} Attacco con arma da mischia}: +18 a colpire, portata 3 m, un bersaglio.

\emph{Colpisce:} 17 (2d6 + 10) danni taglienti, 3/20 danno da Sanguinamento.

\emph{\textbf{Coda.} Attacco con arma da mischia}: +18 a colpire, portata 6 m, un bersaglio.

\emph{Colpisce:} 19 (2d8 + 10) danni contundenti.

\emph{\textbf{Morso.} Attacco con arma da mischia}: +18 a colpire, portata 5 metri, un bersaglio.

\emph{Colpisce:} 21 (2d10 + 10) danni perforanti.

\emph{\textbf{Presenza Spaventosa.}} Ogni creatura scelta dal drago, che si trovi entro 36 metri da esso e consapevole della sua presenza, deve riuscire un Tiro Salvezza di Volontà DC 37 o restare spaventata per 1 minuto. Una creatura può ripetere il Tiro Salvezza al termine di ciascun suo round, terminando l'effetto se lo riesce. Se il Tiro Salvezza della creatura ha successo o l'effetto ha termine per essa, la creatura è immune alla Presenza Spaventosa del drago per le successive 24 ore.

\emph{\textbf{Arma a Soffio (Ricarica 5-6).}} Il drago usa una delle seguenti armi a soffio:

\emph{Soffio Infuocato.} Il drago esala fuoco in un cono di 27 metri. Ogni creatura nell'area deve effettuare un Tiro Salvezza di Riflessi DC 37, subendo 71 (13d10) danni da fuoco se fallisce il Tiro Salvezza, o la metà di questi danni se lo riesce.

\emph{Soffio Indebolente.} Il drago esala del gas in un cono di 27 metri. Ogni creatura in quell'area deve riuscire un Tiro Salvezza su Tempra DC 37 o avere -1d6 ai tiri di attacco basati sulla Forza, prove di Forza, e Tiri Salvezza su Tempra per 1 minuto. Una creatura può ripetere il Tiro Salvezza al termine di ciascun suo round, terminando l'effetto su di sé in caso di successo.

\emph{\textbf{Mutare Forma.}} Il drago può trasformarsi magicamente in un umanoide o bestia il cui grado di sfida sia pari o inferiore al proprio, o tornare alla sua vera forma. Alla morte ritorna alla sua vera forma. Qualsiasi equipaggiamento stia indossando o trasportando viene assorbito o trasportato nella nuova forma (a scelta del drago).

Nella nuova forma, il drago mantiene i suoi Tratti, Punti Ferita la facoltà di parlare, le competenze, la Resistenza Leggendaria, le azioni da tana, e i punteggi di Intelligenza, Saggezza e Carisma, oltre a questa azione. Le sue statistiche e capacità vengono altrimenti rimpiazzate da quelle della nuova forma, eccetto Azioni aggiuntive della nuova forma.

\textbf{Azioni Aggiuntive}

Il drago può effettuare 3 Azioni aggiuntive, scelte tra le opzioni seguenti. Può usare solo un'opzione Aggiuntiva alla volta e solo al termine del round di un'altra creatura. Il drago recupera le Azioni aggiuntive spese all'inizio del proprio round.

\textbf{Attacco di Ala (Costa 2 Azioni).} Il drago batte le ali. Ogni creatura entro 5 metri dal drago deve riuscire un Tiro Salvezza su Riflessi DC 37 o subire 17 (2d6 + 10) danni contundenti e venir gettato prono. Il drago può poi volare fino a metà del suo movimento di volo.

\textbf{Attacco di Coda.} Il drago effettua un attacco di coda.

\textbf{Individuare.} Il drago effettua una prova di Consapevolezza.

\textbf{Ecologia}\\
Ambiente: Pianure calde\\
Organizzazione: Solitario\\
\textbf{Categoria Tesoro}: H\\
\textbf{Descrizione}\\
I draghi d'oro sono l'emblema della virtù. Gli altri draghi di Ljust li riveriscono come agenti delle potenze divine e membri esemplari della razza draconica, e spesso li cercano per consigli o aiuto.\\
\textbf{Incantesimi}\index{Incantesimi da Drago d'Oro}\\
Gli incantesimi preferiti di questo Drago sono:\\
- \hyperlink{Guarigione}{Guarigione}\\
- \hyperlink{Ristorare Superiore}{Ristorare Superiore}\\
- \hyperlink{Tentacoli Neri}{Tentacoli Neri}

\mostro{Drago d'Oro Adulto}
\noindent
\begin{description}[noitemsep, topsep=0pt, parsep=0pt, partopsep=0pt, leftmargin=0cm, labelwidth=2.2cm]
	\item[\textbf{Taglia/Tipo:}] Enorme drago, buono
	\item[\textbf{Caratt.:}] \resizebox{0.5\linewidth+1.8cm}{!}{For 8 Des 2 Cos 7 Int 3 Sag 2 Car 7}
	\item[\textbf{Punti Ferita:}] 344,  \textbf{Difesa:} 36,  \textbf{Iniziativa:} +3
	\item[\textbf{Movimento:}] 12 m, nuoto 12 m, volo 24 m
	\item[\textbf{Tiri Salvez.:}] \resizebox{0.5\linewidth+1.8cm}{!}{\resizebox{0.5\linewidth+1.8cm}{!}{Tempra +24, Riflessi +19, Volontà +19}}
	\item[\textbf{Comp.:}] Furtività +8, Percepire Emozioni +8, Consapevolezza +14, Ingannare +13
	\item[\textbf{Imm. Danni:}] Fuoco
	\item[\textbf{Sensi:}] Scurovisione 36 m, Vista Cieca 18 m
	\item[\textbf{Linguaggi:}] Comune, Draconico
	\item[\textbf{Sfida:}] 17 (18000 PX)\smallskip
\end{description}

\emph{\textbf{Aura indebolente.}} il drago emette nel raggio di 3 metri un aura che causa Affaticato 1. Rimanere nell'aura non aumenta il livello di affaticato.

\emph{\textbf{Anfibio.}} Il drago può respirare aria e acqua.

\emph{\textbf{Resistenza Leggendaria (3/Giorno).}} Se il drago fallisce un Tiro Salvezza, può scegliere invece di riuscire.

\textbf{Azioni}

\emph{\textbf{Multiattacco.}} Il drago può usare la sua Presenza Spaventosa e poi effettuare tre attacchi: uno con il morso e due con gli artigli.

\emph{\textbf{Artiglio.} Attacco con arma da mischia}: +14 a colpire, portata 1 m, un bersaglio.

\emph{Colpisce:} 15 (2d6 + 8) danni taglienti, 1 danno da Sanguinamento.

\emph{\textbf{Coda.} Attacco con arma da mischia}: +14 a colpire, portata 5 metri, un bersaglio.

\emph{Colpisce:} 17 (2d8 + 8) danni contundenti.

\emph{\textbf{Morso.} Attacco con arma da mischia}: +14 a colpire, portata 3 m, un bersaglio.

\emph{Colpisce:} 19 (2d10 + 8) danni perforanti.

\emph{\textbf{Presenza Spaventosa.}} Ogni creatura scelta dal drago, che si trovi entro 36 metri da esso e consapevole della sua presenza, deve riuscire un Tiro Salvezza di Volontà DC 30 o restare spaventata per 1 minuto. Una creatura può ripetere il Tiro Salvezza al termine di ciascun suo round, terminando l'effetto se lo riesce. Se il Tiro Salvezza della creatura ha successo o l'effetto ha termine per essa, la creatura è immune alla Presenza Spaventosa del drago per le successive 24 ore.

\emph{\textbf{Arma a Soffio (Ricarica 5-6).}} Il drago usa una delle seguenti armi a soffio:

\emph{Soffio Infuocato.} Il drago esala fuoco in un cono di 18 metri. Ogni creatura nell'area deve effettuare un Tiro Salvezza di Riflessi DC 30, subendo 66 (12d10) danni da fuoco se fallisce il Tiro Salvezza, o la metà di questi danni se lo riesce.

\emph{Soffio Indebolente.} Il drago esala del gas in un cono di 18 metri. Ogni creatura in quell'area deve riuscire un Tiro Salvezza su Tempra DC 30 o avere -1d6 ai tiri di attacco basati sulla Forza, prove di Forza, e Tiri Salvezza su Tempra per 1 minuto. Una creatura può ripetere il Tiro Salvezza al termine di ciascun suo round, terminando l'effetto su di sé in caso di successo.

\emph{\textbf{Mutare Forma.}} Il drago può trasformarsi magicamente in un umanoide o bestia il cui grado di sfida sia pari o inferiore al proprio, o tornare alla sua vera forma. Alla morte ritorna alla sua vera forma. Qualsiasi equipaggiamento stia indossando o trasportando viene assorbito o trasportato nella nuova forma (a scelta del drago).

Nella nuova forma, il drago mantiene i suoi Tratti, Punti Ferita la facoltà di parlare, le competenze, la Resistenza Leggendaria, le azioni da tana, e i punteggi di Intelligenza, Saggezza e Carisma, oltre a questa azione. Le sue statistiche e capacità vengono altrimenti rimpiazzate da quelle della nuova forma, eccetto Azioni aggiuntive della nuova forma.

\textbf{Azioni Aggiuntive}

Il drago può effettuare 3 Azioni aggiuntive, scelte tra le opzioni seguenti. Può usare solo un'opzione Aggiuntiva alla volta e solo al termine del round di un'altra creatura. Il drago recupera le Azioni aggiuntive spese all'inizio del proprio round.

\textbf{Attacco di Ala (Costa 2 Azioni).} Il drago batte le ali. Ogni creatura entro 3 metri dal drago deve riuscire un Tiro Salvezza su Riflessi DC 30 o subire 15 (2d6 + 8) danni contundenti e venir gettato prono. Il drago può poi volare fino a metà del suo movimento di volo.

\textbf{Attacco di Coda.} Il drago effettua un attacco di coda.

\textbf{Individuare.} Il drago effettua una prova di Consapevolezza.

\emph{\textbf{Arrabbiato:}} Il Drago d'Oro Adulto può eseguire queste azioni a costo 2 Azioni.

\emph{Focalizzare}: la creatura interrompe un effetto mentale su di se in corso

\emph{Brutalità}: la creatura attacca con ferocia inaudita. +1d6 al Tiro per Colpire, 1 danno critico automatico quando colpisce.

\textbf{Ecologia}\\
Ambiente: Pianure calde\\
Organizzazione: Solitario\\
\textbf{Categoria Tesoro}: E\\
\textbf{Descrizione}\\
I draghi d'oro sono l'emblema della virtù. Gli altri draghi di Ljust li riveriscono come agenti delle potenze divine e membri esemplari della razza draconica, e spesso li cercano per consigli o aiuto.\\
\textbf{Incantesimi}\index{Incantesimi da Drago d'Oro}\\
Gli incantesimi preferiti di questo Drago sono:\\
- \hyperlink{Guarigione}{Guarigione}\\
- \hyperlink{Ristorare Superiore}{Ristorare Superiore}\\
- \hyperlink{Tentacoli Neri}{Tentacoli Neri}

\mostro{Drago d'Oro Giovane}
\noindent
\begin{description}[noitemsep, topsep=0pt, parsep=0pt, partopsep=0pt, leftmargin=0cm, labelwidth=2.2cm]
	\item[\textbf{Taglia/Tipo:}] Grande drago, buono
	\item[\textbf{Caratt.:}] \resizebox{0.5\linewidth+1.8cm}{!}{For 6 Des 2 Cos 5 Int 3 Sag 1 Car 5}
	\item[\textbf{Punti Ferita:}] 205,  \textbf{Difesa:} 27,  \textbf{Iniziativa:} +3
	\item[\textbf{Movimento:}] 12 m, nuoto 12 m, volo 24 m
	\item[\textbf{Tiri Salvez.:}] \resizebox{0.5\linewidth+1.8cm}{!}{\resizebox{0.5\linewidth+1.8cm}{!}{Tempra +15, Riflessi +12, Volontà +11}}
	\item[\textbf{Comp.:}] Furtività +6, Percepire Emozioni +5, Consapevolezza +9, Ingannare +9
	\item[\textbf{Imm. Danni:}] Fuoco
	\item[\textbf{Sensi:}] Scurovisione 36 m, Vista Cieca 9 m
	\item[\textbf{Linguaggi:}] Comune, Draconico
	\item[\textbf{Sfida:}] 10 (5900 PX)\smallskip
\end{description}

\emph{\textbf{Anfibio.}} Il drago può respirare aria e acqua.

\textbf{Azioni}

\emph{\textbf{Multiattacco.}} Il drago può effettuare tre attacchi: uno con il morso e due con gli artigli.

\emph{\textbf{Artiglio.} Attacco con arma da mischia}: +12 a colpire, portata 1 m, un bersaglio.

\emph{Colpisce:} 13 (2d6 + 6) danni taglienti, 1 danno da Sanguinamento.

\emph{\textbf{Morso.} Attacco con arma da mischia}: +12 a colpire, portata 3 m, un bersaglio.

\emph{Colpisce:} 17 (2d10 + 6) danni perforanti.

\emph{\textbf{Arma a Soffio (Ricarica 5-6).}} Il drago usa una delle seguenti armi a soffio:

\emph{Soffio Infuocato.} Il drago esala fuoco in un cono di 9 metri. Ogni creatura nell'area deve effettuare un Tiro Salvezza di Riflessi DC 23, subendo 55 (10d10) danni da fuoco se fallisce il Tiro Salvezza, o la metà di questi danni se lo riesce.

\emph{Soffio Indebolente.} Il drago esala del gas in un cono di 9 metri. Ogni creatura in quell'area deve riuscire un Tiro Salvezza di Tempra DC 23 o avere -1d6 ai tiri di attacco basati sulla Forza, prove di Forza, e Tiri Salvezza su Tempra per 1 minuto. Una creatura può ripetere il Tiro Salvezza al termine di ciascun suo round, terminando l'effetto su di sé in caso di successo.

\emph{\textbf{Arrabbiato:}} il giovane drago d'oro ricarica uno dei suoi soffi. Costa 1 Azione.

\textbf{Ecologia}\\
Ambiente: Pianure calde\\
Organizzazione: Solitario\\
\textbf{Categoria Tesoro}: D\\
\textbf{Descrizione}\\
I draghi d'oro sono l'emblema della virtù. Gli altri draghi di Ljust li riveriscono come agenti delle potenze divine e membri esemplari della razza draconica, e spesso li cercano per consigli o aiuto.\\
\textbf{Incantesimi}\index{Incantesimi da Drago d'Oro}\\
Gli incantesimi preferiti di questo Drago sono:\\
- \hyperlink{Guarigione}{Guarigione}\\
- \hyperlink{Ristorare Superiore}{Ristorare Superiore}\\
- \hyperlink{Tentacoli Neri}{Tentacoli Neri}

\mostro{Drago d'Oro Cucciolo}
\noindent
\begin{description}[noitemsep, topsep=0pt, parsep=0pt, partopsep=0pt, leftmargin=0cm, labelwidth=2.2cm]
	\item[\textbf{Taglia/Tipo:}] Media drago, buono
	\item[\textbf{Caratt.:}] \resizebox{0.5\linewidth+1.8cm}{!}{For 4 Des 2 Cos 3 Int 2 Sag 0 Car 3}
	\item[\textbf{Punti Ferita:}] 70,  \textbf{Difesa:} 18,  \textbf{Iniziativa:} +2
	\item[\textbf{Movimento:}] 9 m, nuoto 9 m, volo 18 m
	\item[\textbf{Tiri Salvez.:}] \resizebox{0.5\linewidth+1.8cm}{!}{Tempra +6, Riflessi +5, Volontà +3}
	\item[\textbf{Comp.:}] Furtività +4, Consapevolezza +4
	\item[\textbf{Imm. Danni:}] Fuoco
	\item[\textbf{Sensi:}] Scurovisione 18 m, Vista Cieca 3 m
	\item[\textbf{Linguaggi:}] Draconico
	\item[\textbf{Sfida:}] 3 (700 PX)\smallskip
\end{description}

\emph{\textbf{Anfibio.}} Il drago può respirare aria e acqua.

\textbf{Azioni}

\emph{\textbf{Morso.} Attacco con arma da mischia}: +6 a colpire, portata 1 m, un bersaglio.

\emph{Colpisce:} 9 (1d10 + 4) danni perforanti.

\emph{\textbf{Arma a Soffio (Ricarica 5-6).}} Il drago usa una delle seguenti armi a soffio:

\emph{Soffio Infuocato.} Il drago esala fuoco in un cono di 5 metri. Ogni creatura nell'area deve effettuare un Tiro Salvezza di Riflessi DC 15, subendo 22 (4d10) danni da fuoco se fallisce il Tiro Salvezza, o la metà di questi danni se lo riesce.

\emph{Soffio Indebolente.} Il drago esala del gas in un cono di 5 metri. Ogni creatura in quell'area deve riuscire un Tiro Salvezza su Tempra DC 15 o avere -1d6 ai tiri di attacco basati sulla Forza, prove di Forza, e Tiri Salvezza su Tempra per 1 minuto. Una creatura può ripetere il Tiro Salvezza al termine di ciascun suo round, terminando l'effetto su di sé in caso di successo.

\textbf{Ecologia}\\
Ambiente: Pianure calde\\
Organizzazione: Solitario\\
\textbf{Categoria Tesoro}: C\\
\textbf{Descrizione}\\
I draghi d'oro sono l'emblema della virtù. Gli altri draghi di Ljust li riveriscono come agenti delle potenze divine e membri esemplari della razza draconica, e spesso li cercano per consigli o aiuto.

\mostro{Drago di Ottone Antico}
\begin{description}[noitemsep, topsep=0pt, parsep=0pt, partopsep=0pt, leftmargin=0cm, labelwidth=2.2cm]
	\item[\textbf{Taglia/Tipo:}] Mastodontica drago, buono
	\item[\textbf{Caratt.:}] \resizebox{0.5\linewidth+1.8cm}{!}{For 8 Des 0 Cos 7 Int 3 Sag 2 Car 4}
	\item[\textbf{Punti Ferita:}] 403,  \textbf{Difesa:} 38,  \textbf{Iniziativa:} +3
	\item[\textbf{Movimento:}] 12 m, scavo 12 m, volo 24 m
	\item[\textbf{Tiri Salvez.:}] \resizebox{0.5\linewidth+1.8cm}{!}{\resizebox{0.5\linewidth+1.8cm}{!}{Tempra +27, Riflessi +20, Volontà +22}}
	\item[\textbf{Imm. Danni:}] Fuoco, armi +1
	\item[\textbf{Comp.:}] Consapevolezza +14
	\item[\textbf{Sensi:}] Scurovisione 36 m, Vista Cieca 18 m
	\item[\textbf{Linguaggi:}] Comune, Draconico
	\item[\textbf{Sfida:}] 20 (25000 PX)\smallskip
\end{description}

\emph{\textbf{Aura soporifera.}} il drago emette nel raggio di 3 metri una aura magica che causa Rallentato 1 o Affaticato 1, casualmente per creatura.

\emph{\textbf{Resistenza Leggendaria (3/Giorno).}} Se il drago fallisce un Tiro Salvezza, può scegliere invece di riuscire.

\emph{\textbf{Resistenza alla Magia:}} 3lv

\textbf{Azioni}

\emph{\textbf{Multiattacco.}} Il drago può usare la sua Presenza Spaventosa e poi effettuare tre attacchi: uno con il morso e due con gli artigli.

\emph{\textbf{Artiglio.} Attacco con arma da mischia}: +16 a colpire, portata 3 m, un bersaglio.

\emph{Colpisce:} 15 (2d6 + 8) danni taglienti, 3/20 danno da Sanguinamento.

\emph{\textbf{Coda.} Attacco con arma da mischia}: +16 a colpire, portata 6 m, un bersaglio.

\emph{Colpisce:} 17 (2d8 + 8) danni contundenti.

\emph{\textbf{Morso.} Attacco con arma da mischia}: +16 a colpire, portata 5 metri, un bersaglio.

\emph{Colpisce:} 19 (2d10 + 8) danni perforanti.

\emph{\textbf{Presenza Spaventosa.}} Ogni creatura scelta dal drago, che si trovi entro 36 metri da esso e consapevole della sua presenza, deve riuscire un Tiro Salvezza di Volontà DC 34 o restare spaventata per 1 minuto. Una creatura può ripetere il Tiro Salvezza al termine di ciascun suo round, terminando l'effetto se lo riesce. Se il Tiro Salvezza della creatura ha successo o l'effetto ha termine per essa, la creatura è immune alla Presenza Spaventosa del drago per le successive 24 ore.

\emph{\textbf{Arma a Soffio (Ricarica 5-6).}} Il drago usa una delle seguenti armi a soffio:

\emph{Soffio Infuocato.} Il drago esala fuoco in una linea lunga 27 metri e larga 3 metri. Ogni creatura sulla linea deve effettuare un Tiro Salvezza su Riflessi DC 34, subendo 56 (16d6) danni da fuoco se fallisce il Tiro Salvezza, o la metà di questi danni se lo riesce.

\emph{Soffio Soporifero.} Il drago esala del gas soporifero in un cono di 27 metri. Ogni creatura in quell'area deve riuscire un Tiro Salvezza su Tempra 34 o cadere svenuta per 10 minuti. Questo effetto termina se la creatura svenuta subisce danni o qualcuno impiega un'azione per risvegliarla.

\emph{\textbf{Mutare Forma.}} Il drago può trasformarsi magicamente in un umanoide o bestia il cui grado di sfida sia pari o inferiore al proprio, o tornare alla sua vera forma. Alla morte ritorna alla sua vera forma. Qualsiasi equipaggiamento stia indossando o trasportando viene assorbito o trasportato nella nuova forma (a scelta del drago).

Nella nuova forma, il drago mantiene i suoi Tratti, Punti Ferita la facoltà di parlare, le competenze, la Resistenza Leggendaria, le azioni da tana, e i punteggi di Intelligenza, Saggezza e Carisma, oltre a questa azione. Le sue statistiche e capacità vengono altrimenti rimpiazzate da quelle della nuova forma, eccetto Azioni aggiuntive della nuova forma.

\textbf{Azioni Aggiuntive}

Il drago può effettuare 3 Azioni aggiuntive, scelte tra le opzioni seguenti. Può usare solo un'opzione Aggiuntiva alla volta e solo al termine del round di un'altra creatura. Il drago recupera le Azioni aggiuntive spese all'inizio del proprio round.

\textbf{Attacco di Ala (Costa 2 Azioni).} Il drago batte le ali. Ogni creatura entro 5 metri dal drago deve riuscire un Tiro Salvezza su Riflessi DC 34 o subire 15 (2d6 + 8) danni contundenti e venir gettato prono. Il drago può poi volare fino a metà del suo movimento di volo.

\textbf{Attacco di Coda.} Il drago effettua un attacco di coda.

\textbf{Individuare.} Il drago effettua una prova di Consapevolezza.

\textbf{Ecologia}\\
Ambiente: Deserti Caldi\\
Organizzazione: Solitario\\
\textbf{Categoria Tesoro}: H\\
\textbf{Descrizione}\\
Ottimi conversatori, i draghi d'ottone preferiscono parlare invece che combattere. I draghi d'ottone fanno la tana vicino agli insediamenti umanoidi, dove possono udire le notizie e i pettegolezzi più recenti.\\
\textbf{Incantesimi}\index{Incantesimi da Drago d'Ottone}\\
Gli incantesimi preferiti di questo Drago sono:\\
- \hyperlink{Visione del Vero}{Visione del Vero}\\
- \hyperlink{Conoscenza delle Leggende}{Conoscenza delle Leggende}\\
- \hyperlink{Scrutare}{Scrutare}

\mostro{Drago d'Ottone Adulto}
\begin{description}[noitemsep, topsep=0pt, parsep=0pt, partopsep=0pt, leftmargin=0cm, labelwidth=2.2cm]
	\item[\textbf{Taglia/Tipo:}] Enorme drago, buono
	\item[\textbf{Caratt.:}] \resizebox{0.5\linewidth+1.8cm}{!}{For 6 Des 0 Cos 5 Int 2 Sag 1 Car 3}
	\item[\textbf{Punti Ferita:}] 262,  \textbf{Difesa:} 29,  \textbf{Iniziativa:} +2
	\item[\textbf{Movimento:}] 12 m, scavo 9 m, volo 24 m
	\item[\textbf{Tiri Salvez.:}] \resizebox{0.5\linewidth+1.8cm}{!}{\resizebox{0.5\linewidth+1.8cm}{!}{Tempra +18, Riflessi +13, Volontà +14}}
	\item[\textbf{Imm. Danni:}] Fuoco
	\item[\textbf{Comp.:}] Furtività +5, Consapevolezza +11, Ingannare +8, Storia +7
	\item[\textbf{Sensi:}] Scurovisione 36 m, Vista Cieca 18 m
	\item[\textbf{Linguaggi:}] Comune, Draconico
	\item[\textbf{Sfida:}] 13 (10000 PX)\smallskip
\end{description}

\emph{\textbf{Resistenza Leggendaria (3/Giorno).}} Se il drago fallisce un Tiro Salvezza, può scegliere invece di riuscire.

\textbf{Azioni}

\emph{\textbf{Multiattacco.}} Il drago può usare la sua Presenza Spaventosa e poi effettuare tre attacchi: uno con il morso e due con gli artigli.

\emph{\textbf{Artiglio.} Attacco con arma da mischia}: +12 a colpire, portata 1 m, un bersaglio.

\emph{Colpisce:} 13 (2d6 + 6) danni taglienti, 1 danno da Sanguinamento.

\emph{\textbf{Coda.} Attacco con arma da mischia}: +12 a colpire, portata 5 metri, un bersaglio.

\emph{Colpisce:} 15 (2d8 + 6) danni contundenti.

\emph{\textbf{Morso.} Attacco con arma da mischia}: +12 a colpire, portata 3 m, un bersaglio.

\emph{Colpisce:} 17 (2d10 + 6) danni perforanti.

\emph{\textbf{Presenza Spaventosa.}} Ogni creatura scelta dal drago, che si trovi entro 36 metri da esso e consapevole della sua presenza, deve riuscire un Tiro Salvezza di Volontà DC 26 o restare spaventata per 1 minuto. Una creatura può ripetere il Tiro Salvezza al termine di ciascun suo round, terminando l'effetto se lo riesce. Se il Tiro Salvezza della creatura ha successo o l'effetto ha termine per essa, la creatura è immune alla Presenza Spaventosa del drago per le successive 24 ore.

\emph{\textbf{Arma a Soffio (Ricarica 5-6).}} Il drago usa una delle seguenti armi a soffio:

\emph{Soffio Infuocato.} Il drago esala fuoco in una linea lunga 18 metri e larga 1 metro. Ogni creatura sulla linea deve effettuare un Tiro Salvezza di Riflessi DC 26, subendo 45 (13d6) danni da fuoco se fallisce il Tiro Salvezza, o la metà di questi danni se lo riesce.

\emph{Soffio Soporifero.} Il drago esala del gas soporifero in un cono di 18 metri. Ogni creatura in quell'area deve riuscire un Tiro Salvezza su Tempra 26 o cadere svenuta per 10 minuti. Questo effetto termina se la creatura svenuta subisce danni o qualcuno impiega un'Azione per risvegliarla.

\textbf{Azioni Aggiuntive}

Il drago può effettuare 3 Azioni aggiuntive, scelte tra le opzioni seguenti. Può usare solo un'opzione Aggiuntiva alla volta e solo al termine del round di un'altra creatura. Il drago recupera le Azioni aggiuntive spese all'inizio del proprio round.

\textbf{Attacco di Ala (Costa 2 Azioni).} Il drago batte le ali. Ogni creatura entro 3 metri dal drago deve riuscire un Tiro Salvezza su Riflessi DC 26 o subire 13 (2d6 + 6) danni contundenti e venir gettato prono. Il drago può poi volare fino a metà del suo movimento di volo.

\textbf{Attacco di Coda.} Il drago effettua un attacco di coda.

\textbf{Individuare.} Il drago effettua una prova di Consapevolezza.

\emph{\textbf{Arrabbiato:}} Il Drago d'ottone Adulto può eseguire queste azioni a costo 2 Azioni.

\emph{Focalizzare}: la creatura interrompe un effetto mentale su di se in corso

\emph{Brutalità}: la creatura attacca con ferocia inaudita. +1d6 al Tiro per Colpire, 1 danno critico automatico quando colpisce.

\textbf{Ecologia}\\
Ambiente: Deserti Caldi\\
Organizzazione: Solitario\\
\textbf{Categoria Tesoro}: E\\
\textbf{Descrizione}\\
Ottimi conversatori, i draghi d'ottone preferiscono parlare invece che combattere. I draghi d'ottone fanno la tana vicino agli insediamenti umanoidi, dove possono udire le notizie e i pettegolezzi più recenti.\\
\textbf{Incantesimi}\index{Incantesimi da Drago d'Ottone}\\
Gli incantesimi preferiti di questo Drago sono:\\
- \hyperlink{Visione del Vero}{Visione del Vero}\\
- \hyperlink{Conoscenza delle Leggende}{Conoscenza delle Leggende}\\
- \hyperlink{Scrutare}{Scrutare}

\mostro{Drago d'Ottone Giovane}
\begin{description}[noitemsep, topsep=0pt, parsep=0pt, partopsep=0pt, leftmargin=0cm, labelwidth=2.2cm]
	\item[\textbf{Taglia/Tipo:}] Grande drago, buono
	\item[\textbf{Caratt.:}] \resizebox{0.5\linewidth+1.8cm}{!}{For 4 Des 0 Cos 3 Int 1 Sag 0 Car 2}
	\item[\textbf{Punti Ferita:}] 126,  \textbf{Difesa:} 20,  \textbf{Iniziativa:} +1
	\item[\textbf{Movimento:}] 12 m, scavo 6 m, volo 24 m
	\item[\textbf{Tiri Salvez.:}] \resizebox{0.5\linewidth+1.8cm}{!}{Tempra +9, Riflessi +6, Volontà +6}
	\item[\textbf{Imm. Danni:}] Fuoco
	\item[\textbf{Comp.:}] Furtività +3, Consapevolezza +6, Ingannare +5
	\item[\textbf{Sensi:}] Scurovisione 18 m, Vista Cieca 3 m
	\item[\textbf{Linguaggi:}] Comune, Draconico
	\item[\textbf{Sfida:}] 6 (2300 PX)\smallskip
\end{description}

\textbf{Azioni}

\emph{\textbf{Multiattacco.}} Il drago può effettuare tre attacchi: uno con il morso e due con gli artigli.

\emph{\textbf{Artiglio.} Attacco con arma da mischia}: +7 a colpire, portata 1 m, un bersaglio.

\emph{Colpisce:} 11 (2d6 + 4) danni taglienti, 1 danno da Sanguinamento.

\emph{\textbf{Morso.} Attacco con arma da mischia}: +7 a colpire, portata 3 m, un bersaglio.

\emph{Colpisce:} 15 (2d10 + 4) danni perforanti.

\emph{\textbf{Arma a Soffio (Ricarica 5-6).}} Il drago usa una delle seguenti armi a soffio:

\emph{Soffio Infuocato.} Il drago esala fuoco in una linea lunga 12 metri e larga 1 metro. Ogni creatura sulla linea deve effettuare un Tiro Salvezza di Riflessi DC 18, subendo 42 (12d6) danni da fuoco se fallisce il Tiro Salvezza, o la metà di questi danni se lo riesce.

\emph{Soffio Soporifero.} Il drago esala del gas soporifero in un cono di 9 metri. Ogni creatura in quell'area deve riuscire un Tiro Salvezza su Tempra 18 o cadere svenuta per 5 minuti. Questo effetto termina se la creatura svenuta subisce danni o qualcuno impiega un'Azione per risvegliarla.

\textbf{Ecologia}\\
Ambiente: Deserti Caldi\\
Organizzazione: Solitario\\
\textbf{Categoria Tesoro}: D\\
\textbf{Descrizione}\\
Ottimi conversatori, i draghi d'ottone preferiscono parlare invece che combattere. I draghi d'ottone fanno la tana vicino agli insediamenti umanoidi, dove possono udire le notizie e i pettegolezzi più recenti.\\
\textbf{Incantesimi}\index{Incantesimi da Drago d'Ottone}\\
Gli incantesimi preferiti di questo Drago sono:\\
- \hyperlink{Visione del Vero}{Visione del Vero}\\
- \hyperlink{Conoscenza delle Leggende}{Conoscenza delle Leggende}\\
- \hyperlink{Scrutare}{Scrutare}

\mostro{Drago di Ottone Cucciolo}
\begin{description}[noitemsep, topsep=0pt, parsep=0pt, partopsep=0pt, leftmargin=0cm, labelwidth=2.2cm]
	\item[\textbf{Taglia/Tipo:}] Media drago, buono
	\item[\textbf{Caratt.:}] \resizebox{0.5\linewidth+1.8cm}{!}{For 2 Des 0 Cos 1 Int 0 Sag 0 Car 1}
	\item[\textbf{Punti Ferita:}] 33,  \textbf{Difesa:} 13,  \textbf{Iniziativa:} +0
	\item[\textbf{Movimento:}] 9 m, scavo 5 m, volo 18 m
	\item[\textbf{Tiri Salvez.:}] \resizebox{0.5\linewidth+1.8cm}{!}{Tempra +3, Riflessi +3, Volontà +3}
	\item[\textbf{Imm. Danni:}] Fuoco
	\item[\textbf{Comp.:}] Furtività +2, Consapevolezza +4
	\item[\textbf{Sensi:}] Scurovisione 18 m, Vista Cieca 3 m
	\item[\textbf{Linguaggi:}] Draconico
	\item[\textbf{Sfida:}] 1 (200 PX)\smallskip
\end{description}

\textbf{Azioni}

\emph{\textbf{Morso.} Attacco con arma da mischia}: +4 a colpire, portata 1 m, un bersaglio.

\emph{Colpisce:} 7 (1d10 + 2) danni perforanti.

\emph{\textbf{Arma a Soffio (Ricarica 5-6).}} Il drago usa una delle seguenti armi a soffio:

\emph{Soffio Infuocato.} Il drago esala fuoco in una linea lunga 6 metri e larga 1 metro. Ogni creatura sulla linea deve effettuare un Tiro Salvezza su Riflessi DC 12, subendo 14 (4d6) danni da fuoco se fallisce il Tiro Salvezza, o la metà di questi danni se lo riesce.

\emph{Soffio Soporifero.} Il drago esala del gas soporifero in un cono di 5 metri. Ogni creatura in quell'area deve riuscire un Tiro Salvezza su Tempra 12 o cadere svenuta per 1 minuto. Questo effetto termina se la creatura svenuta subisce danni o qualcuno impiega un'Azione per risvegliarla.

\textbf{Ecologia}\\
Ambiente: Deserti Caldi\\
Organizzazione: Solitario\\
\textbf{Categoria Tesoro}: C\\
\textbf{Descrizione}\\
Ottimi conversatori, i draghi d'ottone preferiscono parlare invece che combattere. I draghi d'ottone fanno la tana vicino agli insediamenti umanoidi, dove possono udire le notizie e i pettegolezzi più recenti.

\mostro{Drago di Rame Antico}
\noindent
\begin{description}[noitemsep, topsep=0pt, parsep=0pt, partopsep=0pt, leftmargin=0cm, labelwidth=2.2cm]
	\item[\textbf{Taglia/Tipo:}] Mastodontica drago, buono
	\item[\textbf{Caratt.:}] \resizebox{0.5\linewidth+1.8cm}{!}{For 8 Des 1 Cos 7 Int 5 Sag 3 Car 4}
	\item[\textbf{Punti Ferita:}] 422,  \textbf{Difesa:} 41,  \textbf{Iniziativa:} +5
	\item[\textbf{Movimento:}] 12 m, scalata 12 m, volo 24 m
	\item[\textbf{Tiri Salvez.:}] \resizebox{0.5\linewidth+1.8cm}{!}{\resizebox{0.5\linewidth+1.8cm}{!}{Tempra +28, Riflessi +22, Volontà +24}}
	\item[\textbf{Comp.:}] Furtività +8, Ingannare +11, Consapevolezza +17
	\item[\textbf{Imm. Danni:}] Acido, armi +1
	\item[\textbf{Sensi:}] Scurovisione 36 m, Vista Cieca 18 m
	\item[\textbf{Linguaggi:}] Comune, Draconico
	\item[\textbf{Sfida:}] 21 (33000 PX)\smallskip
\end{description}

\emph{\textbf{Gas corrosivi.}} il drago emette nel raggio di 3 metri gas corrosivi che causano 2d6 danni da acido a round.

\emph{\textbf{Resistenza Leggendaria (3/Giorno).}} Se il drago fallisce un Tiro Salvezza, può scegliere invece di riuscire.

\emph{\textbf{Resistenza alla Magia:}} 3lv

\textbf{Azioni}

\emph{\textbf{Multiattacco.}} Il drago può usare la sua Presenza Spaventosa e poi effettuare tre attacchi: uno con il morso e due con gli artigli.

\emph{\textbf{Artiglio.} Attacco con arma da mischia}: +16 a colpire, portata 3 m, un bersaglio.

\emph{Colpisce:} 15 (2d6 + 8) danni taglienti, 3/20 danno da Sanguinamento.

\emph{\textbf{Coda.} Attacco con arma da mischia}: +16 a colpire, portata 6 m, un bersaglio.

\emph{Colpisce:} 17 (2d8 + 8) danni contundenti.

\emph{\textbf{Morso.} Attacco con arma da mischia}: +16 a colpire, portata 5 metri, un bersaglio.

\emph{Colpisce:} 19 (2d10 + 8) danni perforanti.

\emph{\textbf{Presenza Spaventosa.}} Ogni creatura scelta dal drago, che si trovi entro 36 metri da esso e consapevole della sua presenza, deve riuscire un Tiro Salvezza di Volontà DC 34 o restare spaventata per 1 minuto. Una creatura può ripetere il Tiro Salvezza al termine di ciascun suo round, terminando l'effetto se lo riesce. Se il Tiro Salvezza della creatura ha successo o l'effetto ha termine per essa, la creatura è immune alla Presenza Spaventosa del drago per le successive 24 ore.

\emph{\textbf{Arma a Soffio (Ricarica 5-6).}} Il drago usa una delle seguenti armi a soffio:

\emph{Soffio Acido.} Il drago esala acido in una linea lunga 27 metri e larga 3 metri. Ogni creatura sulla linea deve effettuare un Tiro Salvezza su Riflessi DC 34, subendo 63 (14d8) danni da acido se fallisce il Tiro Salvezza, o la metà di questi danni se lo riesce.

\emph{Soffio Rallentante.} Il drago esala del gas in un cono di 27 metri. Ogni creatura in quell'area deve riuscire un Tiro Salvezza su Tempra DC 34. Se fallisce il Tiro Salvezza, la creatura ha una Azione in meno a round ed ha la velocità dimezzata. Questi effetti permangono 1 minuto. La creatura può ripetere il Tiro Salvezza al termine di ciascun suo round, terminando l'effetto su di sé in caso di successo.

\emph{\textbf{Mutare Forma.}} Il drago può trasformarsi magicamente in un umanoide o bestia il cui grado di sfida sia pari o inferiore al proprio, o tornare alla sua vera forma. Alla morte ritorna alla sua vera forma. Qualsiasi equipaggiamento stia indossando o trasportando viene assorbito o trasportato nella nuova forma (a scelta del drago).

Nella nuova forma, il drago mantiene i suoi Tratti, Punti Ferita la facoltà di parlare, le competenze, la Resistenza Leggendaria, le azioni da tana, e i punteggi di Intelligenza, Saggezza e Carisma, oltre a questa Azione. Le sue statistiche e capacità

vengono altrimenti rimpiazzate da quelle della nuova forma, eccetto Azioni aggiuntive della nuova forma.

\textbf{Azioni Aggiuntive}

Il drago può effettuare 3 Azioni aggiuntive, scelte tra le opzioni seguenti. Può usare solo un'opzione Aggiuntiva alla volta e solo al termine del round di un'altra creatura. Il drago recupera le Azioni aggiuntive spese all'inizio del proprio round.

\textbf{Attacco di Ala (Costa 2 Azioni).} Il drago batte le ali. Ogni creatura entro 5 metri dal drago deve riuscire un Tiro Salvezza su Riflessi DC 34 o subire 15 (2d6 + 8) danni contundenti e venir gettato prono. Il drago può poi volare fino a metà del suo movimento di volo.

\textbf{Attacco di Coda.} Il drago effettua un attacco di coda.

\textbf{Individuare.} Il drago effettua una prova di Consapevolezza.

\textbf{Ecologia}\\
Ambiente: Colline Calde\\
Organizzazione: Solitario\\
\textbf{Categoria Tesoro}: H\\
\textbf{Descrizione}\\
Questo drago capriccioso durante il combattimento cerca di ostacolare e frustrare i suoi nemici.\\
\textbf{Incantesimi}\index{Incantesimi da Drago di Rame}\\
Gli incantesimi preferiti di questo Drago sono:\\
- \hyperlink{Barriera di Lame}{Barriera di Lame}\\
- \hyperlink{Muro di Forza}{Muro di Forza}\\
- \hyperlink{Scudo di Fuoco}{Scudo di Fuoco}

\mostro{Drago di Rame Adulto}
\noindent
\begin{description}[noitemsep, topsep=0pt, parsep=0pt, partopsep=0pt, leftmargin=0cm, labelwidth=2.2cm]
	\item[\textbf{Taglia/Tipo:}] Enorme drago, buono
	\item[\textbf{Caratt.:}] \resizebox{0.5\linewidth+1.8cm}{!}{For 6 Des 1 Cos 5 Int 4 Sag 2 Car 3}
	\item[\textbf{Punti Ferita:}] 281,  \textbf{Difesa:} 31,  \textbf{Iniziativa:} +4
	\item[\textbf{Movimento:}] 12 m, scalata 12 m, volo 24 m
	\item[\textbf{Tiri Salvez.:}] \resizebox{0.5\linewidth+1.8cm}{!}{\resizebox{0.5\linewidth+1.8cm}{!}{Tempra +19, Riflessi +15, Volontà +16}}
	\item[\textbf{Comp.:}] Furtività +6, Ingannare +8, Consapevolezza +12
	\item[\textbf{Imm. Danni:}] Acido
	\item[\textbf{Sensi:}] Scurovisione 36 m, Vista Cieca 18 m
	\item[\textbf{Linguaggi:}] Comune, Draconico
	\item[\textbf{Sfida:}] 14 (11.500 PX)\smallskip
\end{description}

\emph{\textbf{Resistenza Leggendaria (3/Giorno).}} Se il drago fallisce un Tiro Salvezza, può scegliere invece di riuscire.

\textbf{Azioni}

\emph{\textbf{Multiattacco.}} Il drago può usare la sua Presenza Spaventosa e poi effettuare tre attacchi: uno con il morso e due con gli artigli.

\emph{\textbf{Artiglio.} Attacco con arma da mischia}: +13 a colpire, portata 1 m, un bersaglio.

\emph{Colpisce:} 13 (2d6 + 6) danni taglienti, 1 danno da Sanguinamento.

\emph{\textbf{Coda.} Attacco con arma da mischia}: +13 a colpire, portata 5 metri, un bersaglio.

\emph{Colpisce:} 15 (2d8 + 6) danni contundenti.

\emph{\textbf{Morso.} Attacco con arma da mischia}: +13 a colpire, portata 3 m, un bersaglio.

\emph{Colpisce:} 17 (2d10 + 6) danni perforanti.

\emph{\textbf{Presenza Spaventosa.}} Ogni creatura scelta dal drago, che si trovi entro 36 metri da esso e consapevole della sua presenza, deve riuscire un Tiro Salvezza di Volontà DC 27 o restare spaventata per 1 minuto. Una creatura può ripetere il Tiro Salvezza al termine di ciascun suo round, terminando l'effetto se lo riesce. Se il Tiro Salvezza della creatura ha successo o l'effetto ha termine per essa, la creatura è immune alla Presenza Spaventosa del drago per le successive 24 ore.

\emph{\textbf{Arma a Soffio (Ricarica 5-6).}} Il drago usa una delle seguenti armi a soffio:

\emph{Soffio Acido.} Il drago esala acido in una linea lunga 18 metri e larga 1 metro. Ogni creatura sulla linea deve effettuare un Tiro Salvezza su Riflessi DC 27, subendo 54 (12d8) danni da acido se fallisce il Tiro Salvezza, o la metà di questi danni se lo riesce.

\emph{Soffio Rallentante.} Il drago esala del gas in un cono di 27 metri. Ogni creatura in quell'area deve riuscire un Tiro Salvezza su Tempra DC 27. Se fallisce il Tiro Salvezza, la creatura ha una Azione in meno a round ed ha la velocità dimezzata. Questi effetti permangono 1 minuto. La creatura può ripetere il Tiro Salvezza al termine di ciascun suo round, terminando l'effetto su di sé in caso di successo.

\textbf{Azioni Aggiuntive}

Il drago può effettuare 3 Azioni aggiuntive, scelte tra le opzioni seguenti. Può usare solo un'opzione Aggiuntiva alla volta e solo al termine del round di un'altra creatura. Il drago recupera le Azioni aggiuntive spese all'inizio del proprio round.

\textbf{Attacco di Ala (Costa 2 Azioni).} Il drago batte le ali. Ogni creatura entro 3 metri dal drago deve riuscire un Tiro Salvezza su Riflessi DC 27 o subire 13 (2d6 + 6) danni contundenti e venir gettato prono. Il drago può poi volare fino a metà del suo movimento di volo.

\textbf{Attacco di Coda.} Il drago effettua un attacco di coda.

\textbf{Individuare.} Il drago effettua una prova di Consapevolezza.

\emph{\textbf{Arrabbiato:}} Il Drago di Rame Adulto può eseguire queste azioni a costo 2 Azioni.

\emph{Focalizzare}: la creatura interrompe un effetto mentale su di se in corso

\emph{Brutalità}: la creatura attacca con ferocia inaudita. +1d6 al Tiro per Colpire, 1 danno critico automatico quando colpisce.

\textbf{Ecologia}\\
Ambiente: Colline Calde\\
Organizzazione: Solitario\\
\textbf{Categoria Tesoro}: E\\
\textbf{Descrizione}\\
Questo drago capriccioso durante il combattimento cerca di ostacolare e frustrare i suoi nemici.\\
\textbf{Incantesimi}\index{Incantesimi da Drago di Rame}\\
Gli incantesimi preferiti di questo Drago sono:\\
- \hyperlink{Barriera di Lame}{Barriera di Lame}\\
- \hyperlink{Muro di Forza}{Muro di Forza}\\
- \hyperlink{Scudo di Fuoco}{Scudo di Fuoco}

%\begin{center}
%\includegraphics[width=0.45\textwidth]{immagini/Media_età_del_bronzo,_pugnale_in_bronzo,_02.png}
%\end{center}

\mostro{Drago di Rame Giovane}
\noindent
\begin{description}[noitemsep, topsep=0pt, parsep=0pt, partopsep=0pt, leftmargin=0cm, labelwidth=2.2cm]
	\item[\textbf{Taglia/Tipo:}] Grande drago, buono
	\item[\textbf{Caratt.:}] \resizebox{0.5\linewidth+1.8cm}{!}{For 4 Des 1 Cos 3 Int 3 Sag 1 Car 2}
	\item[\textbf{Punti Ferita:}] 145,  \textbf{Difesa:} 22,  \textbf{Iniziativa:} +3
	\item[\textbf{Movimento:}] 12 m, scalata 12 m, volo 24 m
	\item[\textbf{Tiri Salvez.:}] \resizebox{0.5\linewidth+1.8cm}{!}{Tempra +10, Riflessi +8, Volontà +8}
	\item[\textbf{Comp.:}] Furtività +4, Ingannare +5, Consapevolezza +7
	\item[\textbf{Imm. Danni:}] Acido
	\item[\textbf{Sensi:}] Scurovisione 36 m, Vista Cieca 18 m
	\item[\textbf{Linguaggi:}] Comune, Draconico
	\item[\textbf{Sfida:}] 7 (2900 PX)\smallskip
\end{description}

\textbf{Azioni}

\emph{\textbf{Multiattacco.}} Il drago può effettuare tre attacchi: uno con il morso e due con gli artigli.

\emph{\textbf{Artiglio.} Attacco con arma da mischia}: +8 a colpire, portata 1 m, un bersaglio.

\emph{Colpisce:} 11 (2d6 + 4) danni taglienti, 1 danno da Sanguinamento.

\emph{\textbf{Morso.} Attacco con arma da mischia}: +8 a colpire, portata 3 m, un bersaglio.

\emph{Colpisce:} 15 (2d10 + 4) danni perforanti.

\emph{\textbf{Arma a Soffio (Ricarica 5-6).}} Il drago usa una delle seguenti armi a soffio:

\emph{Soffio Acido.} Il drago esala acido in una linea lunga 12 metri e larga 1 metro. Ogni creatura sulla linea deve effettuare un Tiro Salvezza su Riflessi DC 19, subendo 40 (9d8) danni da acido se fallisce il Tiro Salvezza, o la metà di questi danni se lo riesce.

\emph{Soffio Rallentante.} Il drago esala del gas in un cono di 27 metri. Ogni creatura in quell'area deve riuscire un Tiro Salvezza su Tempra DC 19. Se fallisce il Tiro Salvezza, la creatura ha una Azione in meno a round ed ha la velocità dimezzata. Questi effetti permangono 1 minuto. La creatura può ripetere il Tiro Salvezza al termine di ciascun suo round, terminando l'effetto su di sé in caso di successo.

\emph{\textbf{Arrabbiato:}} il Drago di Rame Giovane ricarica uno dei due soffi. Costo 1 Azione.

\textbf{Ecologia}\\
Ambiente: Colline Calde\\
Organizzazione: Solitario\\
\textbf{Categoria Tesoro}: D\\
\textbf{Descrizione}\\
Questo drago capriccioso durante il combattimento cerca di ostacolare e frustrare i suoi nemici.\\
\textbf{Incantesimi}\index{Incantesimi da Drago di Rame}\\
Gli incantesimi preferiti di questo Drago sono:\\
- \hyperlink{Barriera di Lame}{Barriera di Lame}\\
- \hyperlink{Muro di Forza}{Muro di Forza}\\
- \hyperlink{Scudo di Fuoco}{Scudo di Fuoco}

\mostro{Drago di Rame Cucciolo}
\noindent
\begin{description}[noitemsep, topsep=0pt, parsep=0pt, partopsep=0pt, leftmargin=0cm, labelwidth=2.2cm]
	\item[\textbf{Taglia/Tipo:}] Media drago, buono
	\item[\textbf{Caratt.:}] \resizebox{0.5\linewidth+1.8cm}{!}{For 2 Des 1 Cos 1 Int 2 Sag 0 Car 1}
	\item[\textbf{Punti Ferita:}] 33,  \textbf{Difesa:} 14,  \textbf{Iniziativa:} +2
	\item[\textbf{Movimento:}] 9 m, scalata 9 m, volo 18 m
	\item[\textbf{Tiri Salvez.:}] \resizebox{0.5\linewidth+1.8cm}{!}{Tempra +3, Riflessi +3, Volontà +3}
	\item[\textbf{Comp.:}] Furtività +3, Ingannare +3, Consapevolezza +4
	\item[\textbf{Imm. Danni:}] Acido
	\item[\textbf{Sensi:}] Scurovisione 36 m, Vista Cieca 18 m
	\item[\textbf{Linguaggi:}] Comune, Draconico
	\item[\textbf{Sfida:}] 1 (200 PX)\smallskip
\end{description}

\textbf{Azioni}

\emph{\textbf{Morso.} Attacco con arma da mischia}: +5 a colpire, portata 1 m, un bersaglio.

\emph{Colpisce:} 7 (1d10 + 2) danni perforanti.

\emph{\textbf{Arma a Soffio (Ricarica 5-6).}} Il drago usa una delle seguenti armi a soffio:

\emph{Soffio Acido.} Il drago esala acido in una linea lunga 6 metri e larga 1 metro. Ogni creatura sulla linea deve effettuare un Tiro Salvezza su Riflessi DC 12, subendo 18 (4d8) danni da acido se fallisce il Tiro Salvezza, o la metà di questi danni se lo riesce.

\emph{Soffio Rallentante.} Il drago esala del gas in un cono di 27 metri. Ogni creatura in quell'area deve riuscire un Tiro Salvezza su Tempra DC 12. Se fallisce il Tiro Salvezza, la creatura ha una Azione in meno a round ed ha la velocità dimezzata. Questi effetti permangono 1 minuto. La creatura può ripetere il Tiro Salvezza al termine di ciascun suo round, terminando l'effetto su di sé in caso di successo.

\textbf{Ecologia}\\
Ambiente: Colline Calde\\
Organizzazione: Solitario\\
\textbf{Categoria Tesoro}: C\\
\textbf{Descrizione}\\
Questo drago capriccioso durante il combattimento cerca di ostacolare e frustrare i suoi nemici.

\begin{enfasi}{Il gran dragone, il serpente antico, che è chiamato diavolo e Satana, il seduttore di tutto il mondo, fu gettato giù; fu gettato sulla terra, e con lui furono gettati anche i suoi angeli." Giovanni, Apocalisse 12:9}\end{enfasi}

\mostro{Tàhil}
\noindent
\begin{description}[noitemsep, topsep=0pt, parsep=0pt, partopsep=0pt, leftmargin=0cm, labelwidth=2.2cm]
	\item[\textbf{Taglia/Tipo:}] Colossale drago, Patrono
	\item[\textbf{Caratt.:}] \resizebox{0.5\linewidth+1.8cm}{!}{For 10 Des 0 Cos 10 Int 8 Sag 8 Car 9}
	\item[\textbf{Punti Ferita:}] 615,  \textbf{Difesa:} 52,  \textbf{Iniziativa:} +8
	\item[\textbf{Movimento:}] 20 metri, volare 20 metri
	\item[\textbf{Tiri Salvez.:}] \resizebox{0.5\linewidth+1.8cm}{!}{\resizebox{0.5\linewidth+1.8cm}{!}{Tempra +40, Riflessi +30, Volontà +38}}
	\item[\textbf{Comp.:}] tutte +18
	\item[\textbf{Imm. Danni:}] Freddo, Elettricità, Fuoco, Acido, Veleno, Suono, armi +3
	\item[\textbf{Immunità:}] affascinato, paralizzato, affaticato, spaventato
	\item[\textbf{Sensi:}] Scurovisione 60 m, Visione del vero 40 m
	\item[\textbf{Linguaggi:}] tutti
	\item[\textbf{Sfida:}] 30 (155000 PX)\smallskip
\end{description}

\emph{\textbf{Aura distruttiva.}} il drago emette nel raggio di 6 metri un aura che causa 1 danno da forza cumulativo per round di permanenza. Il danno si azzera dopo 1 ora di allontanamento.

\textbf{Immortale sulla Terra.} Quando il corpo di Tàhil viene ucciso sulla Terra questo si riforma in 3d6 giorni nella tana fatta da Calicante.

\emph{\textbf{Incantesimi.}} Tàhil ha CM 20. La sua caratteristica da incantatore è il Carisma. Tàhil conosce i seguenti incantesimi:

A volontà: Parola Divina

\emph{\textbf{Natura Divina.}} Tàhil non ha bisogno di aria, cibo, bevande o sonno. Gli incantesimi di 5 livello o inferiore non hanno effetto su Tàhil tranne se lo vuole.

\emph{\textbf{Padrone dei Draghi.}} Ogni Drago non di Ljust sulla Terra è fedele ed ubbidiente al volere di Tàhil.

\emph{\textbf{Voce del Padrone.}} Tàhil può dialogare con ogni drago di Tàhil presente in sulla Terra, indipendentemente dalla distanza.

\emph{\textbf{Richiamo del Padrone.}} Tàhil apre un portale ed escono 1d2+1 draghi di Tàhil di età e colore casuale. Il potere è usabile 1 volta al giorno.

\emph{\textbf{Resistenza Leggendaria (5/Giorno).}} Se il Tàhil fallisce un Tiro Salvezza, può scegliere invece di riuscirvi.

\emph{\textbf{più teste.}} Tàhil ha +1d6 ai Tiri Salvezza contro essere cieco, sordo, svenuto. Tàhil può eseguire fino a 6 Reazioni per round.

\emph{\textbf{Rigenerazione.}} Tàhil rigenera 30 Punti Ferita all'inizio del suo round

\textbf{Azioni}

\emph{\textbf{Multiattacco.}} Tàhil può usare la sua Presenza Spaventosa oppure effettuare 3 attacchi (2 con artigli ed uno con la coda) oppure uno solo con il morso. Artiglio +30, portata 5 metri. Coda +19 portata 8 metri. Morso +19, portata 6 metri. Tutti gli attacchi di Tàhil sono considerati magici +5.

\emph{Colpisce:} Artiglio, 24 (4d6 +10, 5/40 danni da sanguinamento) da taglio. Coda, 28 (4d8 +10) contundenti. Morso 48 (8d6 +10) tagliente. Se colpisce con un margine di 10 con il morso mozza il corpo a metà della creatura se non si riesce un TS su Tempra a DC 30.

\emph{\textbf{Presenza Spaventosa}} Ogni creatura che possa vedere Tàhil e sia entro 80 metri deve fare un Tiro Salvezza su Volontà a DC 45 o essere Spaventato per 1 minuto. Ogni round la creatura può effettuare il Tiro Salvezza, se questo riesce è immune alla Presenza Spaventosa di Tàhil per le successive 24 ore.

\textbf{Azioni Aggiuntive}

Il Tàhil può effettuare 3 azioni aggiuntive, scelte da quelle sottostanti ed una per round solo al termine del round di un altra creatura. Tàhil può cambiare il colore della sua testa per accedere ai poteri degli altri tipi di drago. Le azioni dipendono dalla testa scelta.

\textbf{Attacco con Artiglio.}: +19, portata 6 metri, un obiettivo. Se colpisce 32 (4d10 + 10, 3 da Sanguinamento) danno da taglio più 14 (4d6) danni da acido (testa Nera) oppure Elettricità (testa Blu) oppure da Veleno (testa Verde) oppure da Fuoco (testa Rossa) oppure da Freddo (testa Bianca) oppure da Fuoco (testa Gialla) oppure da Suono (testa Viola)

\textbf{Testa Nera.}: Costa 2 azioni Aggiuntive, Tàhil soffia Acido in un cono di 40 metri. Tiro Salvezza su Riflessi DC 40 o prendere 68 (15d8) di danno da acido oppure dimezzare.

\textbf{Testa Blu.}: Costa 2 azioni Aggiuntive, Tàhil soffia Elettricità in un cono di 40 metri. Tiro Salvezza su Riflessi DC 40 o prendere 88 (16d10) di danno da Elettricità oppure dimezzare.

\textbf{Testa Verde.}: Costa 2 azioni Aggiuntive, Tàhil soffia Veleno in un cono di 30 metri. Tiro Salvezza su Riflessi DC 40 o prendere 77 (22d6) di danno da Veleno oppure dimezzare.

\textbf{Testa Rossa.}: Costa 2 azioni Aggiuntive, Tàhil soffia Fuoco in un cono di 30 metri. Tiro Salvezza su Riflessi DC 40 o prendere 91 (26d6) di danno da Fuoco oppure dimezzare.

\textbf{Testa Bianca.}: Costa 2 azioni Aggiuntive, Tàhil soffia Ghiaccio in un cono di 30 metri. Tiro Salvezza su Riflessi DC 40 o prendere 72 (16d8) di danno da Ghiaccio oppure dimezzare.

\textbf{Testa Viola.}: Costa 2 azioni Aggiuntive, Tàhil soffia Suono in un cono di 30 metri. Tiro Salvezza su Riflessi DC 40 o prendere 90 (18d8) di danno da Suono oppure dimezzare.

\textbf{Testa Gialla.}: Costa 2 azioni Aggiuntive, Tàhil soffia sabbia rovente in un cono di 60 metri. Tiro Salvezza su Riflessi DC 40 o prendere 72 (16d8) di danno da Fuoco oppure dimezzare.

\textbf{Ecologia}\\
Ambiente: Sconosciuto\\
Organizzazione: Unico\\
\textbf{Categoria Tesoro}: 6 H\\
\textbf{Descrizione}\\
Tàhil è il Patrono dei Draghi incarnato. Nulla resiste alla sua furia, follia, rabbia e distruzione. Tàhil è una mastodontica creatura con 7 teste di drago, ognuna colorata in modo diverso, ognuna a rappresentare un colore di un Drago. Vedi capitolo sulla Cosmologia per i dettagli della sua storia.

\mostro{Drider}
\noindent
\begin{description}[noitemsep, topsep=0pt, parsep=0pt, partopsep=0pt, leftmargin=0cm, labelwidth=2.2cm]
	\item[\textbf{Taglia/Tipo:}] Grande mostruosità, malvagio
	\item[\textbf{Caratt.:}] \resizebox{0.5\linewidth+1.8cm}{!}{For 3 Des 3 Cos 4 Int 1 Sag 2 Car 1}
	\item[\textbf{Punti Ferita:}] 127,  \textbf{Difesa:} 23,  \textbf{Iniziativa:} +3
	\item[\textbf{Movimento:}] 9 m, scalata 9 m
	\item[\textbf{Tiri Salvez.:}] \resizebox{0.5\linewidth+1.8cm}{!}{\resizebox{0.5\linewidth+1.8cm}{!}{Tempra +10, Riflessi +9, Volontà +8}}
	\item[\textbf{Comp.:}] Furtività +9, Consapevolezza +5
	\item[\textbf{Sensi:}] Scurovisione 36 m
	\item[\textbf{Linguaggi:}] Elfico, Linguaggio delle Profondità
	\item[\textbf{Sfida:}] 6 (2300 PX)\smallskip
\end{description}

\emph{\textbf{Camminare sulla Tela.}} Il drider ignora le restrizioni al movimento provocate dalle ragnatele.

\emph{\textbf{Discendenza Fatata.}} Il drider ha +1d6 ai Tiri Salvezza per non restare affascinato e la magia non può far addormentare un drider.

\emph{\textbf{Incantesimi Innati.}} La caratteristica da incantatore innato del drider è la Saggezza. Il drider può lanciare in maniera innata i seguenti incantesimi, senza bisogno di componenti materiali:

A volontà: \emph{luci danzanti}

1/Giorno: \emph{luce diurna, \hyperlink{Oscurità}{Oscurità}}

\emph{\textbf{Scalare come Ragno.}} Il drider può scalare superfici difficili, compreso lo stare a testa in giù sul soffitto, senza bisogno di effettuare una prova di competenza.

\textbf{Azioni}

\emph{\textbf{Multiattacco.}} Il drider effettua tre attacchi con la spada lunga o con l'arco lungo. Può rimpiazzare uno di questi attacchi con un attacco di morso.

\emph{\textbf{Morso.} Attacco con arma da mischia}: +8 a colpire, portata 1 m, una creatura.

\emph{Colpisce:} 2 (1d4) danni perforanti più 9 (2d8) danni da veleno.

\emph{\textbf{Spada Lunga.} Attacco con arma da mischia}: +7 a colpire, portata 1 m, un bersaglio.

\emph{Colpisce:} 7 (1d8 + 3) danni taglienti, o 8 (1d8 + 3) danni taglienti se usata con due mani.

\emph{\textbf{Arco Lungo.} Attacco con arma a Distanza}: +9 a colpire, gittata 45m, un bersaglio.

\emph{Colpisce:} 7 (1d8 + 3) danni perforanti più 4 (1d8) danni da veleno.

\emph{\textbf{Ragnatela.}} il Drider usando 1 sola Azione lancia l'incantesimo \hyperlink{Ragnatela}{Ragnatela}.

\emph{\textbf{Arrabbiato:}} il Drider raccoglie la saliva velenosa e la sputa sulle sue armi. Fino alla fine del combattimento l'attacco di Spada Lunga causa anche 1d8 di danni da veleno. Costa 1 Azione.

\textbf{Ecologia}\\
Ambiente: Qualsiasi sotterraneo\\
Organizzazione: Solitario, coppia o gruppo (3-8)\\
\textbf{Categoria Tesoro}: Mazza flangiata Perfetta, Arco Lungo Composito Perfetto [Forza +2] con 20 Frecce, Y\\
\textbf{Descrizione}\\
Creato dal corpo di un elfo, alterato e mutato attraverso speciali veleni ed elisir per assumere le caratteristiche di un ragno gigante, il drider è una creatura pericolosa.\\
I drider sono sessualmente dimorfici. La parte inferiore da ragno del corpo di un drider femmina è lucente ed aggraziata, spesso simile al corpo di una vedova nera, mentre il busto superiore di elfo mantiene le sue curve allettanti e il bel viso (con l'eccezione delle venefiche zanne acuminate). La parte inferiore del corpo di un drider maschio è tozza come una tarantola, mentre quella superiore ha un fisico asciutto e supporta un'orrenda faccia più da ragno che da elfo, completa di mandibole zannute.


\begin{center}
	\includegraphics[width=0.9\linewidth]{immagini/James_Paterson_-_Dryade.png}

	\emph{Dryade, James Paterson}
\end{center}

\mostro{Driade}
\noindent
\begin{description}[noitemsep, topsep=0pt, parsep=0pt, partopsep=0pt, leftmargin=0cm, labelwidth=2.2cm]
	\item[\textbf{Taglia/Tipo:}] Media fatato, neutrale
	\item[\textbf{Caratt.:}] \resizebox{0.5\linewidth+1.8cm}{!}{For 0 Des 1 Cos 0 Int 2 Sag 2 Car 4}
	\item[\textbf{Punti Ferita:}] 33,  \textbf{Difesa:} 14,  \textbf{Iniziativa:} +2
	\item[\textbf{Movimento:}] 9 m
	\item[\textbf{Tiri Salvez.:}] \resizebox{0.5\linewidth+1.8cm}{!}{Tempra +3, Riflessi +3, Volontà +3}
	\item[\textbf{Comp.:}] Furtività +5, Consapevolezza +4
	\item[\textbf{Sensi:}] Scurovisione 18
	\item[\textbf{Linguaggi:}] Elfico, Silvano
	\item[\textbf{Sfida:}] 1 (200 PX)\smallskip
\end{description}

\emph{\textbf{Camminata Arborea.}} Uno volta durante il suo round, la driade può usare 1 Azione per entrare magicamente in un albero vivo a sua portata

ed emergere da un altro albero vivo entro 18 metri dal primo albero, ricomparendo in uno spazio non occupato entro 1 metro dal secondo albero. Entrambi gli alberi devono essere di taglia Grande o superiore.

\emph{\textbf{Incantesimi Innati.}} La caratteristica da incantatore innato della driade è il Carisma (DC 14 per i Tiri Salvezza degli incantesimi). La driade può lanciare in maniera innata i seguenti incantesimi, senza aver bisogno di componenti materiali. A volontà:

\emph{\hyperlink{Artificio Druidico}{Artificio Druidico}}

3/giorno ciascuno: \emph{\hyperlink{Bacche Benefiche}{Bacche Benefiche}, \hyperlink{Intralciare}{Intralciare}}

1/giorno: \emph{\hyperlink{Passare Senza Tracce}{Passare Senza Tracce}, \hyperlink{Pelle di Corteccia}{Pelle di Corteccia}, \hyperlink{Randello incantato}{Randello Incantato}}

\emph{\textbf{Parlare con Animali e Piante.}} La driade può comunicare con bestie e piante come se parlassero la stessa lingua.

\emph{\textbf{Resistenza alla Magia.}} La driade ha +1d6 ai Tiri Salvezza contro incantesimi e altri effetti magici.


\textbf{Azioni}

\emph{\textbf{Randello.} Attacco con arma da mischia}: +4 a colpire (+6 a colpire con il randello incantato), portata 1 m, un bersaglio.

\emph{Colpisce:} 2 (1d4) danni contundenti, o 8 (1d8 + 4) danni contundenti con randello incantato

\emph{\textbf{Fascino Fatato.}} La driade può prendere a bersaglio un umanoide o bestia entro 9 metri da lei e che possa vedere. Se il bersaglio può vedere la driade, deve riuscire un Tiro Salvezza su Volontà DC 14 o restare affascinato dalla magia. Le creature affascinate considerano la driade un'amica fidata da ascoltare e proteggere. Sebbene il bersaglio non sia sotto il controllo della driade, interpreterà le richieste o le azioni della driade nel modo più favorevole possibile.

Ogni volta che la driade o i suoi alleati arrecano danno al bersaglio, esso può ripetere il Tiro Salvezza, terminando l'effetto in caso di successo. Altrimenti, l'effetto permane 24 ore o finché la driade muore, si trova su di un piano di esistenza diverso rispetto al bersaglio, o termina l'effetto con un'Azione Immediata.

Se il Tiro Salvezza del bersaglio riesce, il bersaglio sarà immune al Fascino Fatato della driade per le successive 24 ore.

La driade non può tenere affascinati più di un umanoide o tre bestie alla volta.

\textbf{Ecologia}\\
Ambiente: Foreste Temperate\\
Organizzazione: Solitario, coppia o boschetto (3-8)\\
\textbf{Categoria Tesoro}: Arco Lungo Perfetto con 20 Frecce, Pugnale,D\\
\textbf{Descrizione}\\
Le driadi sono spiriti della natura che amano i boschi appartati lontani dagli umanoidi bisognosi di legname. L'interesse principale delle driadi è la propria sopravvivenza e quella delle adorate foreste e sono note per costringere magicamente i viaggiatori ad aiutarle in quei compiti che non possono espletare.
Sono amichevoli con druidi e guardiaboschi non malvagi, dato che riconoscono la loro empatia o il loro rispetto per la natura.
Le driadi sono benevole guardiane degli alberi, e sebbene non siano violente di natura, possono bloccare e sventare le minacce alle loro dimore o trasformare i nemici in alleati.

%\addcontentsline{toc}{subsubsection}{E}
\pdfbookmark[3]{E}{E}

\mostro{Elementale dell'Acqua Generico}
\noindent


\begin{description}[noitemsep, topsep=0pt, parsep=0pt, partopsep=0pt, leftmargin=0cm, labelwidth=2.2cm]
	\item[\textbf{Taglia/Tipo:}] Elementale
	\item[\textbf{Caratt.:}] For 2+GS/3 Des 0+GS/6 Cos 2+GS/3 Int -2+GS/6 Sag 0+GS/6 Car 0+GS/6
	\item[\textbf{Punti Ferita:}] (GS+2)*15, \textbf{Difesa:} GS+Des, \textbf{Iniziativa:} +Des
	\item[\textbf{Movimento:}] 9 m, nuoto GS*4 m
	\item[\textbf{Tiri Salvez.:}] Tempra GS+GS/5+COS, Riflessi GS+DES, Volontà GS+SAG
	\item[\textbf{Res. Danni:}] Acido; da arma non magica
	\item[\textbf{Imm. Danni:}] Veleno
	\item[\textbf{Immunità:}] afferrato, intralciato, paralizzato, pietrificato, privo di sensi, prono, affaticato
	\item[\textbf{Sensi:}] Scurovisione 18 m
	\item[\textbf{Linguaggi:}] Aquan
	\item[\textbf{Sfida:}] GS \\
\end{description}

\emph{\textbf{Congelamento.}} Se l'elementale subisce danno da freddo, gela parzialmente; il suo movimento è ridotto di 6 metri fino al termine del suo prossimo round.\\
\emph{\textbf{Forma d'Acqua.}} L'elementale può entrare nello spazio di una creatura ostile e fermarsi lì. Può muoversi attraverso uno spazio stretto fino a 3 centimetri senza doversi stringere.\\
\emph{\textbf{Natura Elementale.}} Un elementale non ha bisogno di aria,cibo, bevande o sonno.\\
\textbf{Azioni}\\
\emph{\textbf{Multiattacco.}} L'elementale effettua due attacchi di schianto.\\
\emph{\textbf{Schianto.} Attacco con arma da mischia}: +GS/2+FOR a colpire, portata GS/3 metri, un bersaglio.\\
\emph{Colpisce:} GS*1d8 danni contundenti.\\


\begin{center}
	\includegraphics[width=0.9\linewidth]{immagini/geyser.png}
\end{center}

\textbf{Reazione: \emph{Attacco d'opportunità}}: l'elementale effettua un attacco ad una creatura che attraversi o esca dalla sua portata di GS/3 metri.

\emph{\textbf{Sommergere (Ricarica 4-6).}} Ogni creatura nello spazio dell'elementale deve effettuare un Tiro Salvezza di Tempra DC 10+GS+GS/5. Se lo fallisce, il bersaglio subisce (1d8+1)*GS/2 danni contundenti. Se è di taglia GS/3 >=4, il bersaglio è anche afferrato (DC CR*2 per fuggire). Fino al termine dell'afferrare, il bersaglio non può respirare a meno che non sia in grado di respirare acqua. Se il Tiro Salvezza riesce, il bersaglio viene spinto fuori dallo spazio dell'elementale.\\
L'elementale può afferrare una creatura di taglia GS/3 oppure 2 di GS/2 oppure. All'inizio di ciascun round dell'elementale, ogni bersaglio afferrato subisce (1d6)*GS/2 danni contundenti. Una creatura entro 3 metri dall'elementale può trascinare fuori da esso una creatura o oggetto, impiegando un'Azione per tentare di riuscire una prova di Tiro Salvezza Tempra con Forza DC 2+GS*2.\\

\begin{center}
	\includegraphics[width=0.9\linewidth]{immagini/tornado_Elie_Manitoba_2007.png}
\end{center}

\mostro{Elementale dell'Aria Generico}
\noindent
\begin{description}[noitemsep, topsep=0pt, parsep=0pt, partopsep=0pt, leftmargin=0cm, labelwidth=2.2cm]
	\item[\textbf{Taglia/Tipo:}] GS/3 (Piccola, Media, Grande, Enorme, Mastodontico, Colossale)
	\item[\textbf{Caratt.:}] For 0+GS/6 Des 3+GS/3 Cos 0+GS/6 Int -2+GS/6 Sag -1+GS/6 Car 0+GS/6
	\item[\textbf{Punti Ferita:}] (GS+1)*15, \textbf{Difesa:} GS+Des+2, \textbf{Iniziativa:} +Des
	\item[\textbf{Movimento:}] 0 m, volare GS*4 m
	\item[\textbf{Tiri Salvez.:}] Tempra GS+COS, Riflessi GS+GS/5 + DES, Volontà GS+SAG
	\item[\textbf{Res. Danni:}] Elettricità, Suono; da arma non magica
	\item[\textbf{Imm. Danni:}] Veleno
	\item[\textbf{Immunità:}] afferrato, intralciato, paralizzato, pietrificato, privo di sensi, prono, affaticato
	\item[\textbf{Sensi:}] Scurovisione 18 m
	\item[\textbf{Linguaggi:}] Ictun
	\item[\textbf{Sfida:}] GS \\
\end{description}

\emph{\textbf{Forma d'Aria.}} L'elementale può entrare nello spazio di una creatura ostile e fermarsi lì. Può muoversi attraverso uno spazio stretto fino a 3 centimetri senza doversi stringere.\\
\emph{\textbf{Natura Elementale.}} Un elementale non ha bisogno di aria, cibo, bevande o sonno.\\
\textbf{Azioni}\\
\emph{\textbf{Multiattacco.}} L'elementale effettua due attacchi di schianto.\\
\emph{\textbf{Schianto.} Attacco con arma da mischia}: +GS/2+FOR a colpire, portata GS/3 metri, un bersaglio.\\
\emph{Colpisce:} 1d6*GS/3 danni contundenti.\\

\textbf{Reazione: \emph{Attacco d'opportunità}}: l'elementale effettua un attacco ad una creatura che attraversi o esca dalla sua portata di GS/3 metri.

\emph{\textbf{Turbine (Ricarica 4-6).}} Ogni creatura nello spazio dell'elementale deve effettuare un Tiro Salvezza di Tempra DC 10+GS*1.5. Se lo fallisce, il bersaglio subisce 1d8*GS/3 danni contundenti e viene scagliato a GS metri di distanza dall'elementale in una direzione casuale e cadere prono. Se un bersaglio lanciato colpisce un oggetto, come un muro o il pavimento, subisce 3 (1d6) danni contundenti per ogni 3 metri per cui è stato lanciato. Se il bersaglio viene lanciato contro un'altra creatura, quella creatura deve riuscire un Tiro Salvezza di Riflessi DC 13 o subire lo stesso danno e cadere prona.
Se il Tiro Salvezza riesce, il bersaglio subisce la metà del danno contundente e non viene scagliato via né cade prono.

\begin{center}
	%\includegraphics[width=0.9\linewidth]{immagini/elefuoco.png}
	\includegraphics[width=0.7\linewidth]{immagini/wildfire_grayscale.png}
\end{center}

\mostro{Elementale del Fuoco Generico}
\noindent
\begin{description}[noitemsep, topsep=0pt, parsep=0pt, partopsep=0pt, leftmargin=0cm, labelwidth=2.2cm]
	\item[\textbf{Taglia/Tipo:}] GS/3 (Piccola, Media, Grande, Enorme, Mastodontico, Colossale)
	\item[\textbf{Caratt.:}] For 0+GS/3 Des 2+GS/3 Cos 1+GS/6 Int -2+GS/6 Sag -1+GS/6 Car -2+GS/6
	\item[\textbf{Punti Ferita:}] (GS+2)*15, \textbf{Difesa:} GS+1+Des, \textbf{Iniziativa:} +Des
	\item[\textbf{Movimento:}] 15 m
	\item[\textbf{Tiri Salvez.:}] Tempra GS+COS, Riflessi GS+DES, Volontà GS+SAG
	\item[\textbf{Res. Danni:}] da arma non magica
	\item[\textbf{Imm. Danni:}] Fuoco, Veleno
	\item[\textbf{Immunità:}] afferrato, intralciato, paralizzato, pietrificato, privo di sensi, prono, affaticato
	\item[\textbf{Sensi:}] Scurovisione 18 m
	\item[\textbf{Linguaggi:}] Ignan
	\item[\textbf{Sfida:}] GS \\
\end{description}

\emph{\textbf{Forma di Fuoco.}} L'elementale può spostarsi attraverso uno spazio fino a 3 centimetri di larghezza senza stringersi. Una creatura che entri a contatto o colpisca l'elementale con un attacco da mischia mentre si trova entro 1 metro da esso subisce 5 (1d10) danni da fuoco. Inoltre, l'elementale può entrare nello spazio di una creatura ostile e fermarsi lì. La prima volta che entra nello spazio di una creatura in un round, la creatura subisce GS danni da fuoco e prende fuoco; finché qualcuno non impiega un'Azione per spegnere le fiamme, la creatura subirà GS danni da fuoco all'inizio di ciascun proprio round.\\

\emph{\textbf{Illuminazione.}} L'elementale emette luce intensa in un raggio di GS*2 metri e luce fioca per GS*4 metri.\\
\emph{\textbf{Natura Elementale.}} Un elementale non ha bisogno di aria, cibo, bevande o sonno.\\
\emph{\textbf{Suscettibilità all'Acqua.}} L'elementale subisce 1 danno da freddo per ogni 1 metro che si muove in acqua o per ogni 4 litri d'acqua che gli vengono spruzzati addosso.
\textbf{Azioni}\\
\emph{\textbf{Multiattacco.}} L'elementale effettua due attacchi di contatto.\\
\emph{\textbf{Schianto.} Attacco con arma da mischia}: +GS/2+FOR a colpire, portata GS/3 metri, un bersaglio.\\
\emph{Colpisce:} GS*2 danni da fuoco. Se il bersaglio è una creatura o un oggetto infiammabile, prende fuoco. Finché una creatura non impiega un'Azione per spegnere le fiamme, la creatura subirà CR danni da fuoco all'inizio di ciascun proprio round.

\textbf{Reazione: \emph{Attacco d'opportunità}}: l'elementale effettua un attacco ad una creatura che attraversi o esca dalla sua portata di GS/3 metri.


\begin{center}
	\includegraphics[width=0.6\linewidth]{immagini/eleterra.png}
\end{center}

\mostro{Elementale della Terra Generico}
\noindent
\begin{description}[noitemsep, topsep=0pt, parsep=0pt, partopsep=0pt, leftmargin=0cm, labelwidth=2.2cm]
	\item[\textbf{Taglia/Tipo:}] GS/3 (Piccola, Media, Grande, Enorme, Mastodontico, Colossale)
	\item[\textbf{Caratt.:}] For GS Des -2+GS/6 Cos 1+GS/3 Int -3+GS/6 Sag -1+GS/6 Car -3+GS/6
	\item[\textbf{Punti Ferita:}] (GS+3)*15, \textbf{Difesa:} GS+Des, \textbf{Iniziativa:} +Des
	\item[\textbf{Movimento:}] 9 m, arrampicarsi 9 m, scavo 9 m
	\item[\textbf{Tiri Salvez.:}] Tempra GS+COS+GS/5, Riflessi GS+DES, Volontà GS+SAG
	\item[\textbf{Res. Danni:}] da arma non magica
	\item[\textbf{Imm. Danni:}] Veleno, Suono
	\item[\textbf{Immunità:}] afferrato, intralciato, paralizzato, pietrificato, privo di sensi, prono, affaticato
	\item[\textbf{Sensi:}] percezione tellurica 18 m, Scurovisione 18 m
	\item[\textbf{Linguaggi:}] Tremun
	\item[\textbf{Sfida:}] GS \\
\end{description}

\emph{\textbf{Mostro d'Assedio.}} L'elementale infligge danni doppi agli oggetti e le strutture.\\
\emph{\textbf{Natura Elementale.}} Un elementale non ha bisogno di aria, cibo, bevande o sonno.\\
\emph{\textbf{Planata Terrestre.}} L'elementale può scavare attraversa la terra e la pietra non magiche e non lavorate. Quando lo fa, l'elementale non disturba il materiale che sposta.\\
\textbf{Azioni}\\
\emph{\textbf{Multiattacco.}} L'elementale effettua due attacchi di schianto.\\
\emph{\textbf{Schianto.} Attacco con arma da mischia}: +GS a colpire, portata GS/6 metri, un bersaglio.\\
\emph{Colpisce:} GS*3 danni contundenti.

\textbf{Reazione: \emph{Attacco d'opportunità}}: l'elementale effettua un attacco ad una creatura che attraversi o esca dalla sua portata di GS/3 metri.


\mostro{Ettercap}
\noindent
\begin{description}[noitemsep, topsep=0pt, parsep=0pt, partopsep=0pt, leftmargin=0cm, labelwidth=2.2cm]
	\item[\textbf{Taglia/Tipo:}] Media mostruosità, malvagio
	\item[\textbf{Caratt.:}] \resizebox{0.5\linewidth+1.8cm}{!}{For 2 Des 2 Cos 1 Int -2 Sag 1 Car -2}
	\item[\textbf{Punti Ferita:}] 51,  \textbf{Difesa:} 16,  \textbf{Iniziativa:} +2
	\item[\textbf{Movimento:}] 9 m, scalata 9 m
	\item[\textbf{Tiri Salvez.:}] \resizebox{0.5\linewidth+1.8cm}{!}{Tempra +3, Riflessi +4, Volontà +3}
	\item[\textbf{Comp.:}] Furtività +4, Consapevolezza +3, Sopravvivenza +3
	\item[\textbf{Sensi:}] Scurovisione 18 m
	\item[\textbf{Sfida:}] 2 (450 PX)\smallskip
\end{description}

\emph{\textbf{Camminare sulla Tela.}} L'ettercap ignora le restrizioni al movimento provocate dalle ragnatele.

\emph{\textbf{Scalare come Ragno.}} L'ettercap può scalare superfici difficili, compreso lo stare a testa in giù sul soffitto, senza bisogno di effettuare una prova.

\emph{\textbf{Senso della Tela.}} Mentre è in contatto con una ragnatela, l'ettercap sa l'esatta posizione di qualsiasi altra creatura in contatto con la stessa ragnatela.

\textbf{Azioni}

\emph{\textbf{Multiattacco.}} L'ettercap effettua due attacchi: uno con il morso e uno con gli artigli

\emph{\textbf{Artigli.} Attacco con arma da mischia}: +6 a colpire, portata 1 m, un bersaglio.

\emph{Colpisce:} 7 (2d4 + 2) danni taglienti, 1 danno da Sanguinamento.

\emph{\textbf{Morso.} Attacco con arma da mischia}: +6 a colpire, portata 1 m, un bersaglio.

\emph{Colpisce:} 6 (1d8 + 2) danni perforanti più 4 (1d8) danni da veleno. Il bersaglio deve riuscire un Tiro Salvezza di Tempra DC 11 o restare avvelenato, -1 Forza e Destrezza, per 1 minuto. La creatura può ripetere il Tiro Salvezza al termine di ciascun suo round, terminando l'effetto se riesce il Tiro Salvezza.

\emph{\textbf{Ragnatela (Ricarica 5-6).} Attacco con arma a Distanza}: +5 a colpire, gittata 9m, una creatura di taglia Grande o minore.

\emph{Colpisce:} La creatura è intralciata dalla ragnatela. Con un'Azione, la creatura intralciata può effettuare un Tiro Salvezza Tempra con Forza DC 11, liberandosi dalla tela se la riesce. L'effetto termina se la tela è distrutta. La tela ha Difesa 10, 5 Punti Ferita, vulnerabilità ai danni da fuoco, e immunità ai danni contundenti e da veleno.

\textbf{Ecologia}\\
Ambiente: Foreste Temperate\\
Organizzazione: solitario, coppia o nido (3-6 più 2-8 ragni giganti)\\
\textbf{Categoria Tesoro}: C\\
\textbf{Descrizione}\\
Gli ettercap sono umanoidi alti di solito 1,8 metri e pesano circa 100 kg, con braccia allungate fino a terra ed un orrendo volto con elementi ragneschi. Sono solitari e raramente si uniscono ad altri della loro razza, tranne per l'accoppiamento. Quando fanno gruppo, tendono ad attrarre varie specie di ragni, formando uno strano connubio di ettercap e aracnidi.\\
Gli ettercap sono noti per la costruzione di astute trappole fatte di ragnatele e altri materiali naturali, che usano per catturare prede. Costruiscono rifugi di ragnatela, tra i rami più alti gli alberi lontano dagli altri predatori terrestri, e usano ragni mostruosi come vedette e guardiani.\\
Gli ettercap non sono coraggiosi, ma le loro trappole spesso impediscono al nemico di estrarre le armi. Un ettercap attacca con artigli e morsi velenosi. In genere evita la mischia con gli avversari che possono ancora muoversi e fugge se si liberano.

\mostro{Ettin}
\noindent
\begin{description}[noitemsep, topsep=0pt, parsep=0pt, partopsep=0pt, leftmargin=0cm, labelwidth=2.2cm]
	\item[\textbf{Taglia/Tipo:}] Grande gigante, malvagio
	\item[\textbf{Caratt.:}] \resizebox{0.5\linewidth+1.8cm}{!}{For 5 Des -1 Cos 3 Int -2 Sag 0 Car -1}
	\item[\textbf{Punti Ferita:}] 89,  \textbf{Difesa:} 16,  \textbf{Iniziativa:} -1
	\item[\textbf{Movimento:}] 12 m
	\item[\textbf{Tiri Salvez.:}] \resizebox{0.5\linewidth+1.8cm}{!}{Tempra +7, Riflessi +3, Volontà +4}
	\item[\textbf{Comp.:}] Consapevolezza +4
	\item[\textbf{Sensi:}] percezione tellurica 18 m
	\item[\textbf{Linguaggi:}] Gigante, Goblinoide
	\item[\textbf{Sfida:}] 4 (1100 PX)\smallskip
\end{description}

\emph{\textbf{Due Teste.}} L'ettin ha +1d6 alle prove di Consapevolezza e sui Tiri Salvezza contro le condizioni accecato, affascinato, assordato, privo di sensi, spaventato e stordito.

\emph{\textbf{Veglia.}} Quando una delle due teste dell'ettin è addormentata, l'altra è sveglia.

\textbf{Azioni}

\emph{\textbf{Multiattacco.}} L'ettin effettua due attacchi: uno con l'ascia da battaglia e uno con la mazza chiodata.

\emph{\textbf{Ascia da Battaglia.} Attacco con arma da mischia}: +7 a colpire, portata 1 m, un bersaglio.

\emph{Colpisce:} 14 (2d8 + 5) danni taglienti.

\emph{\textbf{Mazza Chiodata.} Attacco con arma da mischia}: +7 a colpire, portata 1 m, un bersaglio.

\emph{Colpisce:} 14 (2d8 + 5) danni perforanti.

\textbf{Ecologia}\\
Ambiente: Colline fredde\\
Organizzazione: Solitario, coppia, gruppo (3-6), truppa (1-2 più 1-2 Orsi Bruni, banda (3-6 più 1-2 Orsi Bruni) o colonia (3-6 più 1-2 Orsi Bruni e 7-12 Orchi, o 9-16 Goblin)\\
\textbf{Categoria Tesoro}: Armatura di Cuoio, 2 Mazzafrusti Leggeri, 4 Giavellotti, P\\
\textbf{Descrizione}\\
Gli ettin, o giganti a due teste, sono cacciatori notturni malevoli e imprevedibili. Le due teste gli concedono impareggiabili poteri di percezione, facendone dei guardiani eccellenti.\\
Gli ettin sembrano Giganti di Collina o Giganti di Pietra, ma il volto zannuto tradisce una discendenza orchesca. Hanno pelle marrone rosata e non si lavano mai se non vi sono costretti, cosa che li rende così sporchi e sudici che la loro pelle sembra spessa e grigia.\\
Gli adulti sono alti 3,9 metri e pesano 2.600 kg. Gli ettin vivono circa 75 anni.\\
Gli ettin non hanno un loro linguaggio ma parlano un gergo misto di Gigante, Goblin e Orchesco. Le creature che parlano uno qualsiasi di questi linguaggi possono comunicare con un ettin effettuando una prova di Intelligenza con DC 15. La prova si effettua una volta per ogni frammento di informazione; se l'altra creatura parla due di questi linguaggi la DC è 10, mentre per qualcuno che li parla tutti e tre è 5.\\
Sebbene gli ettin non siano molto intelligenti, sono guerrieri astuti. Preferiscono tendere imboscate alle loro vittime anziché ingaggiarle in combattimento, ma una volta che la battaglia è cominciata, un ettin combatte furiosamente fino alla morte del nemico.\\
Gli ettin sono creature solitarie, si stabiliscono nella sicurezza di cave rocciose e cavità, spesso circondate da buche e fossi, e tengono a volte degli orsi delle caverne come animali da compagnia o guardiani.\\
Un ettin particolarmente potente può attrarre un gruppo di seguaci, specie Goblin o Orchi. Comunque, questi assembramenti sono più che altro delle eccezioni, e raramente durano a lungo, con gli individualisti ettin che vanno per la loro strada appena le opportunità di saccheggio e rapina diminuiscono o se il capo viene ucciso.\\
In genere formano delle coppie riproduttive per allevare la prole solo per brevi periodi prima di riprendere ognuno la propria strada. I giovani ettin maturano rapidamente, raggiungendo la taglia adulta in un anno, potendo così provvedere a se stessi.

%\addcontentsline{toc}{subsubsection}{F}
\pdfbookmark[3]{F}{F}

\mostro{Fantasma}
\noindent
\begin{description}[noitemsep, topsep=0pt, parsep=0pt, partopsep=0pt, leftmargin=0cm, labelwidth=2.2cm]
	\item[\textbf{Taglia/Tipo:}] Media non morto, qualsiasi tratto
	\item[\textbf{Caratt.:}] \resizebox{0.5\linewidth+1.8cm}{!}{For -2 Des 1 Cos 0 Int 0 Sag 1 Car 3}
	\item[\textbf{Punti Ferita:}] 87,  \textbf{Difesa:} 18,  \textbf{Iniziativa:} +1
	\item[\textbf{Movimento:}] 0 m, volo 12 m, Fluttuare
	\item[\textbf{Tiri Salvez.:}] \resizebox{0.5\linewidth+1.8cm}{!}{Tempra +4, Riflessi +5, Volontà +5}
	\item[\textbf{Res. Danni:}] Acido, Elettricità, Fuoco, Suono, Veleno; da armi non magiche
	\item[\textbf{Immunità:}] affascinato, afferrato, intralciato, paralizzato, pietrificato, prono, affaticato, spaventato, sanguinamento
	\item[\textbf{Sensi:}] Scurovisione 18 m
	\item[\textbf{Linguaggi:}] qualsiasi lingua conosciuta in vita, Expiran
	\item[\textbf{Sfida:}] 4 (1100 PX)\smallskip
\end{description}

\emph{\textbf{Movimento Incorporeo.}} Il fantasma può attraversare altre creature e oggetti come se fossero terreno difficile. Subisce 5 (1d10) danni da forza se termina il suo round all'interno di un oggetto.

\emph{\textbf{Natura Non Morta.}} Il fantasma non ha bisogno di aria, cibo, bevande o di dormire.

\emph{\textbf{Vista Eterea.}} Il fantasma può vedere 18 metri nel Piano Etereo quando si trova sul Piano Materiale, e vice versa.

\textbf{Azioni}

\emph{\textbf{Tocco Avvizzente.} Attacco con arma da mischia}: +5 a colpire, portata 1 m, un bersaglio.

\emph{Colpisce:} 17 (4d6 + 3) danni da Vuoto. Il bersaglio deve fare un Tiro Salvezza su Tempra a DC 15 o divenire Affaticato.

\emph{\textbf{Eterealità.}} Il fantasma entra nel Piano Etereo dal Piano Materiale, o vice versa. È visibile sul Piano Materiale mentre è nel Piano Etereo, e vice versa, ma non può interagire con nulla che si trovi sull'altro piano.

\emph{\textbf{Possessione (Ricarica 6).}} Un umanoide, entro 1 metro e visibile al fantasma, deve riuscire un Tiro Salvezza di Volontà DC 15 o venire posseduto dal fantasma; il fantasma poi scompare, e il bersaglio è inabile e perde il controllo del suo corpo. Il fantasma ora controlla il corpo ma non priva il bersaglio della sua consapevolezza. Il fantasma non può essere bersaglio di attacchi, incantesimi, o altri effetti, eccetto quelli che scacciano i non morti, e mantiene i suoi Tratti, Intelligenza, Saggezza, Carisma e immunità all'essere affascinato e spaventato. Per il resto usa altrimenti le statistiche del bersaglio posseduto, ma non accede al sapere e competenze del bersaglio.

La possessione dura finché il corpo scende a 0 Punti Ferita, il fantasma la termina con un'Azione Immediata, o il fantasma viene scacciato o espulso. Quando la possessione termina, il fantasma riappare in uno spazio non occupato entro 1 metro dal corpo. Il bersaglio è immune alla Possessione di questo fantasma per 24 ore dopo aver riuscito il Tiro Salvezza o al termine della possessione.

\emph{\textbf{Viso Orripilante.}} Ogni creatura che non sia non morta, entro 18 metri dal fantasma e che lo possa vedere, deve riuscire un Tiro Salvezza di Volontà DC 15 o essere spaventata per 1 minuto. Se il Tiro Salvezza fallisce di 5 o più, il bersaglio invecchia anche di 1d4 x 10 anni. Un bersaglio spaventato può ripetere il Tiro Salvezza al termine di ciascun proprio round, terminando l'effetto per sé, qualora riuscisse il Tiro Salvezza. Se il Tiro Salvezza del bersaglio riesce e per lui l'effetto ha fine, il bersaglio è immune al Viso Orripilante del fantasma per le successive 24 ore. Tramite l'incantesimo \hyperlink{Ristorare Superiore}{Ristorare Superiore} si può recuperare 1 anno di invecchiamento, ma solo se eseguito entro 24 dall'effetto di invecchiamento.

\textbf{Ecologia}
Ambiente: qualsiasi\\
Organizzazione: solitario\\
\textbf{Categoria Tesoro}: Nessuno\\
\textbf{Descrizione}\\
Quando ad un'anima non è concesso il riposo a causa di qualche grave ingiustizia, vera o presunta, a volte essa torna come fantasma. Questi esseri sono eternamente angosciati, privi di sostanza e incapaci di rimettere le cose a posto. Sebbene i fantasmi possano avere qualsiasi Tratto, molti si aggrappano al mondo dei viventi con un forte senso di odio e rabbia, e come risultato diventano malvagi; anche una creatura buona dopo morta può diventare un fantasma odioso e crudele.

più di altri mostri, il fantasma deve avere un background ben delineato. Perché questo personaggio è diventato un fantasma? Quali leggende lo circondano? Un incontro con un fantasma non dovrebbe mai avvenire in modo accidentale: ci sono molti altri non morti incorporei, come Wraith e Spettri, per questo. Un incontro adeguato con un fantasma dovrebbe avvenire in una scena al culmine di un lungo periodo di tensione costruito con servitori minori o manifestazioni di spiriti non morti. L'esempio di fantasma sopra rappresenta una principessa umana assassinata da un amante infedele; dopo un confronto, lui la legò con delle catene e la gettò nel pozzo del castello, dove morì annegata. Le capacità del fantasma sono state selezionate in base al background, mostrando come si possa creare un potente antagonista. Applicando l'archetipo a creature con livelli e quindi Abilità proprie o con capacità razziali significative si possono creare fantasmi molto più potenti.

Quando viene creato un fantasma, questi ottiene le copie degli oggetti a cui in vita dava particolare valore (a condizione che gli originali non siano in possesso di altre creature). L'equipaggiamento funziona normalmente per il fantasma ma passa attraverso gli oggetti o le creature materiali. Un'arma +1 o con un potenziamento superiore, tuttavia, può danneggiare le creature materiali. Un fantasma può usare scudi e armature solo se hanno la capacità Tocco Fantasma.

Gli oggetti originali vengono lasciati indietro, proprio come le spoglie fisiche del fantasma. Se un'altra creatura impugna l'originale, la copia incorporea svanisce. Questa perdita fa inevitabilmente infuriare il fantasma, che non si ferma davanti a nulla per riportare l'oggetto nel posto in cui giaceva originariamente (e riguadagnarne l'utilizzo).

\mostro{Fauci Gorgoglianti}
\noindent
\begin{description}[noitemsep, topsep=0pt, parsep=0pt, partopsep=0pt, leftmargin=0cm, labelwidth=2.2cm]
	\item[\textbf{Taglia/Tipo:}] Media aberrazione, neutrale
	\item[\textbf{Caratt.:}] \resizebox{0.5\linewidth+1.8cm}{!}{For 0 Des -1 Cos 3 Int -4 Sag 0 Car -2}
	\item[\textbf{Punti Ferita:}] 52,  \textbf{Difesa:} 13,  \textbf{Iniziativa:} -1
	\item[\textbf{Movimento:}] 3 m, nuoto 3 m
	\item[\textbf{Tiri Salvez.:}] \resizebox{0.5\linewidth+1.8cm}{!}{Tempra +5, Riflessi +3, Volontà +3}
	\item[\textbf{Immunità:}] prono
	\item[\textbf{Sensi:}] Scurovisione 18 m
	\item[\textbf{Sfida:}] 2 (450 PX)\smallskip
\end{description}

\emph{\textbf{Gorgoglio.}} Finché la fauce è in grado di vedere una creatura e non è inabile, pronuncia frasi incoerenti. Ogni creatura che inizi il suo round entro 6 metri dalla fauce e può udire il suo gorgoglio deve effettuare un Tiro Salvezza di Volontà DC 12. Se lo fallisce, la creatura non può effettuare reazioni fino all'inizio del suo prossimo round e tira un d8 per determinare cosa farà durante il proprio round. Da 1 a 4, la creatura non fa nulla. Con 5 o 6, la creatura non svolge nessun'Azione o Reazione e usa tutto il suo movimento per muoversi in una direzione determinata casualmente. Con 7 o 8, la creatura effettua un attacco da mischia contro una creatura determinata a caso entro la sua portata o non fa nulla se non è in grado di effettuare un simile attacco.

\emph{\textbf{Terreno Aberrante.}} Il terreno in un raggio di 3 metri intorno alla fauce è considerato terreno difficile. Ogni creatura che inizi il suo round in quell'area deve riuscire un Tiro Salvezza di Tempra DC 11 o vedere il suo movimento ridotto a 0 fino all'inizio del suo round successivo.

\textbf{Azioni}

\emph{\textbf{Multiattacco.}} La fauce gorgogliante effettua un attacco di morso e, se può, uno Sputo Accecante.

\emph{\textbf{Morso.} Attacco con arma da mischia}: +4 a colpire, portata 1 m, una creatura.

\emph{Colpisce:} 17 (5d6) danni perforanti. Se il bersaglio è di taglia Media o inferiore, deve riuscire un Tiro Salvezza di Tempra DC 11 o venir gettato prono. Se il bersaglio viene ucciso da questo danno, viene assorbito dalla fauce.

\emph{\textbf{Sputo Accecante (Ricarica 5-6).}} La fauce sputa un globo chimico ad un punto visibile entro 5 metri da essa. Il globo esplode all'impatto in un lampo accecante di luce. Ogni creatura entro 1 metro dal lampo deve riuscire un Tiro Salvezza di Riflessi DC 13 o restare accecata fino al termine del prossimo round della fauce.

\textbf{Reazione: \emph{Sputo opportunistico}} la fauce, quando colpita con un danno critico sputa un globo acido alla creatura che l'ha ferita causando 2d6 di danno da acido.

\textbf{Ecologia}\\
Ambiente: Qualsiasi Sotterraneo\\
Organizzazione: Solitario\\
\textbf{Categoria Tesoro}: accidentale (O)\\
\textbf{Descrizione}\\
Disgustosa, nauseante e affamata: queste sono le uniche parole che descrivono in modo appropriato la fauce gorgogliante. Bestie ripugnanti che si nascondono nelle grotte, nelle fogne e negli incubi, le fauci non hanno altro senso sociale, ecologico o religioso diverso dalla loro capacità di far impazzire coloro che le ascoltano. Alcuni studiosi credono che le fauci gorgoglianti siano una variante più piccola del molto più pericoloso shoggoth, mentre altri teorizzano che sia una punizione di Orudjs inflitta a coloro che l'hanno offesa.

\mostro{Fenice}
\noindent
\begin{description}[noitemsep, topsep=0pt, parsep=0pt, partopsep=0pt, leftmargin=0cm, labelwidth=2.2cm]
	\item[\textbf{Taglia/Tipo:}] Mastodontica celestiale, Coraggioso, Protettivo, Buono
	\item[\textbf{Caratt.:}] \resizebox{0.5\linewidth+1.8cm}{!}{For 8 Des 6 Cos 5 Int 5 Sag 6 Car 6}
	\item[\textbf{Punti Ferita:}] 300,  \textbf{Difesa:} 38,  \textbf{Iniziativa:} +6
	\item[\textbf{Movimento:}] 9 m, volare 27 m (buono)
	\item[\textbf{Tiri Salvez.:}] \resizebox{0.5\linewidth+1.8cm}{!}{\resizebox{0.5\linewidth+1.8cm}{!}{Tempra +20, Riflessi +21, Volontà +21}}
	\item[\textbf{Imm. Danni:}] Fuoco, Luce, Veleno, armi +1
	\item[\textbf{Immunità:}] afferrato, intralciato, paralizzato, pietrificato, prono, privo di sensi, affaticato, sanguinamento
	\item[\textbf{Sensi:}] Scurovisione 18 m, Visione Crepuscolare 18 m
	\item[\textbf{Linguaggi:}] Ictun, Celestiale, Comune, Ignan
	\item[\textbf{Sfida:}] 15 (13000 PX)\smallskip
\end{description}

\emph{\textbf{Consapevolezza della Luce.}} La Fenice ha sempre attivi i seguenti incantesimi \emph{\hyperlink{Individuazione del Magico}{Individuazione del Magico}, \hyperlink{Individuazione delle Malattie e dei Veleni}{Individuazione delle Malattie e dei Veleni}, \hyperlink{Vedere l'invisibile}{Vedere l'invisibile}}

\emph{\textbf{Incantesimi Innati.}} La caratteristica da incantatore della Fenice è il Carisma. La Fenice può lanciare in maniera innata i seguenti incantesimi, senza bisogno di componenti materiali:

A volontà: \emph{\hyperlink{Cura Ferite}{Cura Ferite} 1, \hyperlink{Dissolvi Magie}{Dissolvi Magie}, \hyperlink{Fiamma Perenne}{Fiamma Perenne}, \hyperlink{Rimuovi Maledizione}{Rimuovi Maledizione}, \hyperlink{Metamorfosi}{Metamorfosi} (solo in umanoidi)}

3/giorno: \emph{\hyperlink{Cura Ferite}{Cura Ferite} 5 di Massa, \hyperlink{Guarigione}{Guarigione}, \hyperlink{Muro di Fuoco}{Muro di Fuoco}, \hyperlink{Ristorare Superiore}{Ristorare Superiore}, \hyperlink{Tempesta di Fuoco}{Tempesta di Fuoco}}

1 volta: \emph{\hyperlink{Resurrezione}{Resurrezione}} la Fenice sacrificando la sua vita in maniera definitiva può riportare in vita una creatura.

\textbf{Azioni}

\emph{\textbf{Multiattacco.}} La Fenice può attaccare con due artigli ed il morso

\emph{\textbf{Morso.} Attacco con arma da mischia}: +12 al a colpire, portata 6 m, una creatura.

\emph{Colpisce:} 19 danni perforanti (2d8+8 + 1d6 da Luce)

\emph{\textbf{Artiglio.} Attacco con arma da mischia}: +12 al a colpire, portata 6 m, una creatura.

\emph{Colpisce:} 17 danni da taglio (2d6+8 + 1d6 da Luce)

\textbf{Reazione: \emph{Attacco d'opportunità}}: la Fenice effettua un attacco ad una creatura che attraversi o esca dalla sua portata di 6 metri.

\textbf{Abilità speciali}

\emph{\textbf{Rinascita}}

Una Fenice uccisa si riduce ad un falò di 3 metri cubi dove giace al centro un uovo di fenice. Dopo 1d4+4 round questo uovo si schiude e diventa una Fenice perfettamente sana. L'unico modo per evitare la rinascita è togliere l'uovo dal falò (20d6 di danno da Luce) od usare un incantesimo di Disintegrazione sull'uovo.
Una Fenice può resuscitare in questo modo una volta all'anno, se muore prima che sia trascorso questo tempo, la morte è definitiva. Uccidere una Fenice scatena l'ira delle Allieve della Luce e dei cavalieri di Sumkjr.

\emph{\textbf{Ali di fiamma}}

La Fenice può trasformare le sue piume in fiamma senza usare Azioni. Queste piume infliggono 1d6 danni da fuoco + 1d6 danni da Luce a tutte le creature entro 6 metri all'inizio del suo round.

\emph{\textbf{Arrabbiato:}} solo le leggende narrano di una Fenice arrabbiata e si dice che sia intervenuto direttamente un Patrono.

\textbf{Ecologia}\\
Ambiente: Deserti e colline calde\\
Organizzazione: Solitario\\
\textbf{Categoria Tesoro}: Nessuno\\
\textbf{Descrizione}\\
La leggenda narra che le Fenici siano gli uccelli da compagnia di Ljust, sicuramente sono creature maestose e bellissime ed emanano una Luce simile a quella della Patrona della Genesi. Il movimento delle loro ali non produce rumore mentre la loro voce è canto. La fenice è un leggendario uccello di fuoco e luce che vive solitamente nei deserti. Sono creature molto intelligenti e sagge ed a volte usando la loro capacità di metamorfosi si recano nelle città dove aiutano chi combatte contro l'oscurità.

La leggenda racconta che le fenici si generino quando un Cavaliere di Sumkjir o un Allieva della Luce compia l'estremo sacrificio.

\mostro{Fioritura Ossea}
\noindent
\begin{description}[noitemsep, topsep=0pt, parsep=0pt, partopsep=0pt, leftmargin=0cm, labelwidth=2.2cm]
	\item[\textbf{Taglia/Tipo:}] Grande non morto, non allineato
	\item[\textbf{Caratt.:}] \resizebox{0.5\linewidth+1.8cm}{!}{For 3 Des 2 Cos 4 Int -2 Sag -2 Car -3}
	\item[\textbf{Punti Ferita:}] 127,  \textbf{Difesa:} 22,  \textbf{Iniziativa:} +2
	\item[\textbf{Movimento:}] 12 m
	\item[\textbf{Tiri Salvez.:}] \resizebox{0.5\linewidth+1.8cm}{!}{Tempra +10, Riflessi +8, Volontà +4}
	\item[\textbf{Imm. Danni:}] Veleno
	\item[\textbf{Res. Danni:}] perforante, tagliente, Veleno, da Luce
	\item[\textbf{Immunità:}] affaticato, sanguinamento, rallentato, lentezza
	\item[\textbf{Sensi:}] Vista cieca 18 m
	\item[\textbf{Linguaggi:}] comprende il Comune, druidico, silvano ma non può parlare
	\item[\textbf{Sfida:}] 6 (2300 PX)\smallskip
\end{description}

\emph{\textbf{Un piede nella Natura.}} finché Fioritura Ossea è a contatto con la terra rigenera all'inizio del suo round 6 Punti Ferita.

\emph{\textbf{Uno nella Natura.}} finché Fioritura Ossea è in un ambiente naturale e non si muove attacca di sorpresa se non notato. E' richiesta una prova di Consapevolezza 21 per notarlo.

\emph{\textbf{Natura Non Morta.}} Fioritura Ossea non necessita aria, cibo, bevande o sonno.

\textbf{Azioni}

\emph{\textbf{Multiattacco}} Fioritura Ossea può attaccare con il Grande Randello 3 volte oppure usa Soffiare Spore ed esegue un attacco con il Grande Randello

\emph{\textbf{Grande Randello.} Attacco con arma da mischia}: +8 a colpire, portata 2 m, un bersaglio.

\emph{Colpisce:} 17 (2d10 + 6) danni contundenti

\textbf{Reazione: \emph{Attacco d'opportunità}}: la Fioritura ossea effettua un attacco ad una creatura che attraversi o esca dalla sua portata di 1 metro.

\emph{\textbf{Soffio di Spore}}: raggio di 6 metri. Fioritura Ossea emana spore e pollini tutto intorno a se. Qualsiasi creatura che respiri nel raggio di 6 metri dalla Fioritura Ossea deve effettuare un Tiro Salvezza su Tempra a DC 18. Se il Tiro Salvezza fallisce la creatura subisce 3d8 danni da veleno ed è sotto l'influenza dell'incantesimo \hyperlink{lentezza}{Lentezza} per 1 minuto. Se il Tiro Salvezza riesce subisce metà del danno ed è rallentato fino alla fine del round successivo.

\emph{\textbf{Arrabbiato:}} la Fioritura Ossea raccoglie le energie della natura intorno a se avvizzendola. Recupera 50 Punti Ferita. Costa 2 Azioni.

\textbf{Ecologia}\\
Ambiente: Qualsiasi foresta\\
Organizzazione: Solitario, gruppi (2d12)\\
\textbf{Categoria Tesoro}: Accidentale\\
\textbf{Descrizione}\\
Le Fioriture Ossee sono creature morte nel fitto della foresta per i più disparati motivi. La Natura non volendo sprecare nulla anima la creatura per farne suo difensore. A prima vista una Fioritura Ossea non è diverso da un tronco coperto di licheni colorati, piccoli funghi e manto erboso tanto è uno con la natura.

\mostro{Fungo Stridente}
\noindent
\begin{description}[noitemsep, topsep=0pt, parsep=0pt, partopsep=0pt, leftmargin=0cm, labelwidth=2.2cm]
	\item[\textbf{Taglia/Tipo:}] Media pianta, disallineato
	\item[\textbf{Caratt.:}] \resizebox{0.5\linewidth+1.8cm}{!}{For -5 Des -5 Cos 0 Int -5 Sag -4 Car -5}
	\item[\textbf{Punti Ferita:}] 15,  \textbf{Difesa:} 7,  \textbf{Iniziativa:} -5
	\item[\textbf{Movimento:}] 0 m
	\item[\textbf{Tiri Salvez.:}] \resizebox{0.5\linewidth+1.8cm}{!}{Tempra +3, Riflessi +3, Volontà +3}
	\item[\textbf{Immunità:}] accecato, assordato, spaventato
	\item[\textbf{Sensi:}] Vista Cieca 9 m (cieco oltre questo raggio)
	\item[\textbf{Sfida:}] 0 (10 PX)\smallskip
\end{description}

\emph{\textbf{Falso Aspetto.}} Mentre il fungo stridente rimane immobile, è indistinguibile da un normale fungo.

\textbf{Azioni}

\emph{\textbf{Strillo.}} Quando una luce intensa o una creatura si trova entro 9 metri dal fungo stridente, esso emette un strillo udibile fino a 90 metri di distanza. Il fungo stridente continua a strillare finché la fonte del disturbo non si è portata fuori gittata e per altri 1d4 round successivi, ovvero finché non si è sgonfiato il cappello.

\textbf{Ecologia}\\
Ambiente: Qualsiasi sotterraneo\\
Organizzazione: Solitario, coppia o macchia (3-12)\\
\textbf{Categoria Tesoro}: Accidentale\\
\textbf{Descrizione}\\
Un fungo stridente è alto circa 50 cm, dall'ampio cappello marrone. Una volta emesso l'urlo il cappello si rigonfia in 1d3 minuti.

Si racconta di cuochi delle profondità specializzati nel cuocere questi funghi in pietanze sopraffine. I più bravi riescono anche a non fare sgonfiare il cappello.

\mostro{Fungo Violetto}
\noindent
\begin{description}[noitemsep, topsep=0pt, parsep=0pt, partopsep=0pt, leftmargin=0cm, labelwidth=2.2cm]
	\item[\textbf{Taglia/Tipo:}] Media pianta, disallineato
	\item[\textbf{Caratt.:}] \resizebox{0.5\linewidth+1.8cm}{!}{For -4 Des -5 Cos 0 Int -5 Sag -4 Car -5}
	\item[\textbf{Punti Ferita:}] 19,  \textbf{Difesa:} 7,  \textbf{Iniziativa:} -5
	\item[\textbf{Movimento:}] 2 m
	\item[\textbf{Tiri Salvez.:}] \resizebox{0.5\linewidth+1.8cm}{!}{Tempra +3, Riflessi +3, Volontà +3}
	\item[\textbf{Immunità:}] accecato, assordato, spaventato
	\item[\textbf{Sensi:}] Vista Cieca 9 m (cieco oltre questo raggio)
	\item[\textbf{Sfida:}] 1/4 (50 PX)\smallskip
\end{description}

\emph{\textbf{Falso Aspetto.}} Mentre il fungo violetto rimane immobile, è indistinguibile da un normale fungo.

\textbf{Azioni}

\emph{\textbf{Multiattacco.}} Il fungo effettua 1d4 attacchi con Contatto Putrido.

\emph{\textbf{Contatto Putrido.} Attacco con arma da mischia}: +3 a colpire, portata 3 m, un bersaglio.

\emph{Colpisce:} 4 (1d8) danni da Vuoto.

\textbf{Ecologia}\\
Ambiente: Qualsiasi sotterraneo\\
Organizzazione: Solitario, coppia o macchia (3-12)\\
\textbf{Categoria Tesoro}: Accidentale\\
\textbf{Descrizione}\\
I funghi viola sono uno dei più noti e temuti pericoli delle caverne. Un viaggiatore può spesso notare i segni lasciati dal fungo viola su coloro che vivono o cacciano nei luoghi in cui questi funghi carnivori si appostano. Queste profonde e orribili cicatrici sembrano solchi scavati nella carne: i segni di un incontro ravvicinato con un fungo viola.

Un fungo viola si nutre della materia organica putrefatta, ma a differenza della maggioranza dei funghi non è un consumatore passivo. I viticci di un fungo viola possono colpire con inaspettata rapidità e sono ricoperti di un veleno virulento che causa la putrefazione delle carni con nauseante velocità. Questo potente veleno, se trascurato, può far marcire rapidamente un intero braccio o una gamba, lasciandosi dietro solo ossa che presto si corroderanno anch'esse.

Sebbene i funghi viola possano muoversi, lo fanno solo per attaccare o cacciare la preda. Un fungo viola con un flusso regolare di putredine di cui nutrirsi si accontenta di restare in un posto. Molti abitanti del sottosuolo, in particolare Trogloditi e Vegepigmei, sfruttano questo comportamento a loro vantaggio e posizionano molteplici funghi viola in giunzioni ed entrate chiave delle loro caverne come guardiani, assicurandosi di fornire loro cadaveri a sufficienza per evitare che si addentrino nel rifugio in cerca di cibo.

Alcune specie di Boleto Stridente hanno un aspetto piuttosto simile a quello dei funghi viola, sebbene manchino di ramificazioni tentacolari. Non è strano trovare boleti stridenti e funghi viola nello stesso groviglio, specialmente nelle aree dove altre creature coltivano questi funghi come guardiani.

Un fungo viola è alto 1,2 metri e pesa 25 kg.

\mostro{Fuoco Fatuo}
\noindent
\begin{description}[noitemsep, topsep=0pt, parsep=0pt, partopsep=0pt, leftmargin=0cm, labelwidth=2.2cm]
	\item[\textbf{Taglia/Tipo:}] Minuscola non morto, malvagio
	\item[\textbf{Caratt.:}] \resizebox{0.5\linewidth+1.8cm}{!}{For -5 Des 9 Cos 0 Int 1 Sag 2 Car 0}
	\item[\textbf{Punti Ferita:}] 51,  \textbf{Difesa:} 23,  \textbf{Iniziativa:} +9
	\item[\textbf{Movimento:}] 0 m, volo 15 m, Fluttuare
	\item[\textbf{Tiri Salvez.:}] \resizebox{0.5\linewidth+1.8cm}{!}{Tempra +3, Riflessi +11, Volontà +4}
	\item[\textbf{Res. Danni:}] Acido, Veleno, Freddo, Fuoco, da Vuoto, Suono; armi che non siano magiche
	\item[\textbf{Immunità:}] afferrato, intralciato, paralizzato, privo di sensi, prono, affaticato, sanguinamento
	\item[\textbf{Sensi:}] Scurovisione 36 m
	\item[\textbf{Linguaggi:}] le lingue che conosceva in vita
	\item[\textbf{Sfida:}] 2 (450 PX)\smallskip
\end{description}

\emph{\textbf{Consumare Vita.}} Con un'Azione Immediata, il fuoco fatuo può prendere a bersaglio una creatura che può vedere entro 1 metro da esso e che abbia 0 Punti Ferita o meno e sia ancora in vita. Il bersaglio deve riuscire un Tiro Salvezza di Tempra DC 12 contro questa magia o morire. Se il bersaglio muore, il fuoco fatuo recupera 10 (3d6) Punti Ferita.

\emph{\textbf{Effimero.}} Il fuoco fatuo non può indossare né trasportare nulla.

\emph{\textbf{Illuminazione Variabile.}} Il fuoco fatuo promana luce intensa in un raggio da 1 a 6 metri e luce fioca per un numero di metri pari al doppio del raggio scelto. Il fuoco fatuo può modificare questo raggio con una Reazione.

\emph{\textbf{Movimento Incorporeo.}} Il fuoco fatuo può muoversi attraverso altre creature e oggetti come se fossero terreno difficile. Subisce 5 (1d10) danni da forza se termina il suo round all'interno di un oggetto.

\emph{\textbf{Natura Non Morta.}} Il fuoco fatuo non ha bisogno di aria, cibo o bevande.

\textbf{Azioni}

\emph{\textbf{Scossa.} Attacco con incantesimo in mischia}: +6 a colpire, portata 1 m, una creatura.

\emph{Colpisce:} 9 (2d8) danni da elettricità.

\emph{\textbf{Invisibilità.}} Il fuoco fatuo e la sua luce diventano magicamente invisibili finché non attacca o usa Consumare Vita, o finché la sua concentrazione non termina (come se si stesse concentrando su di un incantesimo).

\textbf{Ecologia}
Ambiente: Qualsiasi Palude\\
Organizzazione: Solitario, coppia o sequenza (3-4)\\
\textbf{Categoria Tesoro}: Accidentale\\
\textbf{Descrizione}\\
Malvagie creature che si nutrono delle forti emanazioni psichiche delle creature terrorizzate, i fuochi fatui traggono piacere nel mettere i viaggiatori creduloni in situazioni pericolose. Nelle terre selvagge, dove sono molto comuni, i fuochi fatui preferiscono tattiche semplici come posizionarsi su scogli o sabbie mobili dove possono essere scambiati facilmente per lanterne (specialmente se possono predisporre la trappola nei pressi di vere lanterne di segnalazione), così da attirare i viaggiatori verso il pericolo.

I fuochi fatui possono contare solo sulla loro scossa elettrica in situazioni pericolose, quindi preferiscono lasciare che altre creature o pericoli si occupino delle loro vittime mentre loro fluttuano nelle vicinanze e banchettano.

I fuochi fatui possono brillare di qualunque colore desiderino, ma sono più spesso gialli, bianchi, verdi o blu. Possono anche variare la loro luminosità per creare un disegno: molti fuochi fatui amano creare forme che somigliano vagamente a teschi nella loro luminescenza per aumentare il terrore nelle loro vittime. I loro veri corpi sono globi di materiale spugnoso traslucido appena visibili di circa 30 centimetri che pesano 1,5 kg e possono emettere luce su tutta la loro superficie. La luce dei fuochi fatui brilla approssimativamente come una torcia, e sebbene non sembrino utilizzare suoni per comunicare, sentono perfettamente e possono far vibrare i loro corpi così rapidamente da imitare il linguaggio.

Nonostante siano denigrati dalla maggioranza delle creature senzienti, i fuochi fatui sono in realtà alquanto intelligenti, sebbene ragionino in modo completamente alieno.

I fuochi fatui non hanno età e sono di fatto immortali, a meno che non muoiano di morte violenta; i fuochi fatui più antichi possono essere ottimi depositari di conoscenze del passato, sebbene convincere una di queste crudeli creature a cooperare possa essere piuttosto complicato.

\mostro{Fustigatore}
\noindent
\begin{description}[noitemsep, topsep=0pt, parsep=0pt, partopsep=0pt, leftmargin=0cm, labelwidth=2.2cm]
	\item[\textbf{Taglia/Tipo:}] Grande mostruosità, malvagio
	\item[\textbf{Caratt.:}] \resizebox{0.5\linewidth+1.8cm}{!}{For 4 Des -1 Cos 3 Int 3 Sag 3 Car -2}
	\item[\textbf{Punti Ferita:}] 108,  \textbf{Difesa:} 17,  \textbf{Iniziativa:} +3
	\item[\textbf{Movimento:}] 3 m, scalata 3 m
	\item[\textbf{Tiri Salvez.:}] \resizebox{0.5\linewidth+1.8cm}{!}{Tempra +8, Riflessi +4, Volontà +8}
	\item[\textbf{Comp.:}] Furtività +5, Consapevolezza +6
	\item[\textbf{Sensi:}] Scurovisione 18 m
	\item[\textbf{Linguaggi:}] comune, lingue antiche (latino, greco, celtico..)
	\item[\textbf{Sfida:}] 5 (1800 PX)\smallskip
\end{description}

\emph{\textbf{Falso Aspetto.}} Quando il fustigatore rimane immobile, è indistinguibile da una normale formazione rocciosa, come una stalagmite.

\emph{\textbf{Scalare come Ragno.}} Il fustigatore può scalare superfici difficili, compreso lo stare a testa in giù sul soffitto, senza bisogno di effettuare una prova di competenza.

\emph{\textbf{Viticci Afferranti.}} Il fustigatore può avere fino a sei viticci alla volta. Ogni viticcio può essere attaccato (Difesa 16; 10 Punti Ferita; immunità ai danni da veleno). Distruggere un viticcio non infligge danni al fustigatore, che può produrre un viticcio di rimpiazzo nel suo prossimo round. Un viticcio può essere anche rotto se una creatura effettua un'Azione e riesce un Tiro Salvezza Tempra con Forza DC 17 contro di esso.

\textbf{Azioni}

\emph{\textbf{Multiattacco.}} Il fustigatore può effettuare quattro attacchi con i suoi viticci, usare avvolgere ed effettuare un attacco con il morso.

\emph{\textbf{Morso.} Attacco con arma da mischia}: +7 a colpire, portata 2 m, un bersaglio.

\emph{Colpisce:} 22 (4d8 + 4) danni perforanti e Malattia Necrosi Purulenta

\emph{Necrosi Purulenta:} 1 giorno, TS Tempra DC 15, 12 ore, 1 successo, -1 Costituzione.

\emph{\textbf{Viticcio.} Attacco con arma da mischia}: +7 a colpire, portata 15 m, una creatura.

\emph{Colpisce:} Il bersaglio è afferrato (DC 15 per fuggire). Fino al termine dell'afferrare il fustigatore non può usare lo stesso viticcio contro un altro bersaglio.

\emph{\textbf{Avvolgere.}} Il fustigatore trascina le creature afferrate da lui di 7 metri verso di lui. TS Tempra DC 17 per non farsi spostare.

\textbf{Reazione: \emph{Attacco d'opportunità}}: il fustigatore effettua un attacco con Viticcio ad una creatura che attraversi o esca dalla sua portata di 6 metri.

\emph{\textbf{Arrabbiato:}} il fustigatore emette un onda cacofonica nauseabonda. Tutte le creature nel raggio di 6 metri devono eseguire un Tiro Salvezza su Tempra DC 18 o essere Nauseato fino alla fine del round successivo. Costa 2 Azioni.

\textbf{Ecologia}
Ambiente: Qualsiasi Sotterraneo\\
Organizzazione: Solitario, coppia o gruppo (3-6)\\
\textbf{Categoria Tesoro}: D\\
\textbf{Descrizione}\\
Il fustigatore è un cacciatore da agguato. Capace di modificare la colorazione e la forma del suo corpo, un fustigatore nascosto sembra una stalagmite di pietra o ghiaccio (o in luoghi dal soffitto basso, una colonna di pietra o ghiaccio). Nelle aree prive di questi tratti per nascondersi un fustigatore può comprimere il suo corpo fino a sembrare un masso. Le sferze che può estroflettere non sono di carne ma di uno spesso materiale semiliquido simile a cera parzialmente fusa ma con la resistenza di una catena di ferro e la capacità di intirizzire la carne e indebolire le forze. Il fustigatore può usare queste sferze con grande maestria e farle volare fino a 15 metri per rubare gli oggetti che attraggono la sua attenzione.

Nonostante la sua forma aliena e mostruosa, il fustigatore è uno degli abitanti più intelligenti del sottosuolo. Non formano vaste società (anche se spesso si trovano a vivere insieme ad altre creature del sottosuolo come i Divora Cervelli, con cui a volte si alleano), ma spesso si aggregano in piccoli gruppi. Particolarmente interessato alla filosofia della vita e della morte, e agli aspetti più sottili delle religioni più sinistre e crudeli del mondo, un fustigatore può parlare o discutere per ore con quelli che inizialmente aveva semplicemente cercato di mangiare. Alcune storie parlano di oratori e filosofi particolarmente dotati che sono stati tenuti per giorni o anche anni come animali domestici o compagni di conversazione da gruppi di fustigatori; alla fine, però, se non riescono a scappare, l'appetito dei fustigatori finisce per avere la meglio sulla loro curiosa intelligenza, specialmente nei casi in cui questi animali da compagnia superano costantemente l'arguzia e la pazienza dei loro guardiani.
Un fustigatore è alto 2,7 metri e pesa 1.100 kg.

%\addcontentsline{toc}{subsubsection}{G}
\pdfbookmark[3]{G}{G}

\mostro{Gablin}
\noindent
\begin{description}[noitemsep, topsep=0pt, parsep=0pt, partopsep=0pt, leftmargin=0cm, labelwidth=2.2cm]
	\item[\textbf{Taglia/Tipo:}] Piccolo immondo, malvagio
	\item[\textbf{Caratt.:}] \resizebox{0.5\linewidth+1.8cm}{!}{For 2 Des 1 Cos 1 Int -2 Sag -1 Car -2}
	\item[\textbf{Punti Ferita:}] 19,  \textbf{Difesa:} 13,  \textbf{Iniziativa:} +1
	\item[\textbf{Movimento:}] 9 m
	\item[\textbf{Tiri Salvez.:}] \resizebox{0.5\linewidth+1.8cm}{!}{Tempra +3, Riflessi +3, Volontà +3}
	\item[\textbf{Sensi:}] Scurovisione 18 m
	\item[\textbf{Linguaggi:}] comprendono il Comune ma non lo parlano, Abissale
	\item[\textbf{Sfida:}] 1/4 (50 PX)\smallskip
\end{description}

\emph{\textbf{Sensibilità alla Luce}}. Mentre è alla luce del sole, il gablin ha -1d6 ai tiri per colpire oltre che alle prove di Consapevolezza basate sulla vista.

\textbf{Azioni}

\emph{\textbf{Spada Corta.} Attacco con arma da mischia}: +4 a colpire, portata 1 m, un bersaglio.

\emph{Colpisce:} 5 (1d6 + 2) danni taglienti.

\emph{\textbf{Morso.} Attacco con arma da mischia}: +5 a colpire, contatto, un bersaglio.

\emph{Colpisce:} 3 (1d1 + 2) danni perforanti.

\textbf{Ecologia}\\
Ambiente: Ovunque\\
Organizzazione: Gruppo (8-12), banda da guerra (10-24) o tribù (50+, 1 sergente di 3° livello per 20 adulti, 1 o 2 luogotenenti di 4° o 5° livello, 1 capo di 6°-8° livello, 6-12 lupi selvatici e 1-4 Ogre o 1-2 Campione Gablin)\\
\textbf{Categoria Tesoro}: Accidentale\\
\textbf{Descrizione}\\
I Gablin sono la feccia della feccia, si dice che un Gablin nasce ad ogni pensiero cattivo e sicuramente sono veramente tanti.
I Gablin sono piccoli umanoidi dalla pelle scura, con striature verdi generati inizialmente per volontà di Cattalm con l'unico scopo di portare distruzione, morte e sofferenza.
I Gablin si possono nascondere ovunque purché in prossimità di una fonte di cibo, solitamente prediligono le fogne oppure strutture abbandonate vicino ai villaggi.
Lo scopo unico di un Gablin è uccidere e perpetuare la specie. I Gablin sono tutti maschi e la loro natura immonda li rende capaci di impregnare qualsiasi femmina umanoide.
Solitamente la gestazione dura solo 3 settimane durante le quali le donne vengono torturate per rafforzare gli 1d6+2 piccoli che porta in grembo. Il parto solitamente si conclude con i piccoli di Gablin che sventrano la madre e ne fanno il primo loro pasto.
Questo metodo di procreazione unita alla loro voracia famelica di sangue e carne ne fanno tra le creature più odiate e temute.
Anche se singolarmente non sono particolarmente temibili i Gablin si muovono sempre in gruppo e se questo supera le due dozzine allora c'è quasi sempre un Gablin Incantatore o addirittura un Campione Gablin a guidarli.

\mostro{Campione Gablin}
\noindent
\begin{description}[noitemsep, topsep=0pt, parsep=0pt, partopsep=0pt, leftmargin=0cm, labelwidth=2.2cm]
	\item[\textbf{Taglia/Tipo:}] Media immondo, malvagio
	\item[\textbf{Caratt.:}] \resizebox{0.5\linewidth+1.8cm}{!}{For 4 Des 2 Cos 3 Int 1 Sag 0 Car -1}
	\item[\textbf{Punti Ferita:}] 70,  \textbf{Difesa:} 18,  \textbf{Iniziativa:} +2
	\item[\textbf{Movimento:}] 12 m
	\item[\textbf{Tiri Salvez.:}] \resizebox{0.5\linewidth+1.8cm}{!}{Tempra +6, Riflessi +5, Volontà +3}
	\item[\textbf{Sensi:}] Scurovisione 18 m
	\item[\textbf{Linguaggi:}] Comune, Abissale
	\item[\textbf{Sfida:}] 3 (700 PX)\smallskip
\end{description}

\textbf{Azioni}

\emph{\textbf{Randello Pesante.} Attacco con arma da mischia}: +6 a colpire, portata 2 m, un bersaglio.

\emph{Colpisce:} 11 (2d6 + 4) danni contundenti.

\emph{\textbf{Evocare Gablin}}: 3 Azioni. Il Gablin spilla il suo sangue a terra e a questo sorgono 2d4 Gablin, perde 1 Punto Ferita

\textbf{Ecologia}\\
Ambiente: Qualsiasi\\
Organizzazione: a capo di un gruppo di Gablin\\
\textbf{Categoria Tesoro}: Armatura di Pelle, Randello pesante, B\\
\textbf{Descrizione}\\
I Campioni Gablin vengono generati spontaneamente quando il numero di Gablin presente raggiunge le 20 unità. Enormemente più grossi, più forti ed intelligenti di un Gablin i Campioni sono i leader del gruppo, coloro che pianificano le battaglie e gli scontri.
Non hanno remore a mandare al massacro i Gablin o ad uccidere qualsiasi cosa che respiri. Pervasi dello spirito di Cattalm il loro scopo è sempre e solo distruggere ed uccidere.

\mostro{Paladino Gablin}
\noindent
\begin{description}[noitemsep, topsep=0pt, parsep=0pt, partopsep=0pt, leftmargin=0cm, labelwidth=2.2cm]
	\item[\textbf{Taglia/Tipo:}] Grande immondo, malvagio
	\item[\textbf{Caratt.:}] \resizebox{0.5\linewidth+1.8cm}{!}{For 5 Des 2 Cos 3 Int 2 Sag 3 Car 3}
	\item[\textbf{Punti Ferita:}] 126,  \textbf{Difesa:} 22,  \textbf{Iniziativa:} +2
	\item[\textbf{Movimento:}] 12 m
	\item[\textbf{Tiri Salvez.:}] \resizebox{0.5\linewidth+1.8cm}{!}{\resizebox{0.5\linewidth+1.8cm}{!}{Tempra +9, Riflessi +8, Volontà +9}}
	\item[\textbf{Sensi:}] Scurovisione 18 m
	\item[\textbf{Linguaggi:}] Comune, Abissale
	\item[\textbf{Sfida:}] 6 (2300 PX)\smallskip
\end{description}

\textbf{Azioni}

\emph{\textbf{Multiattacco.}} Il Paladino Gablin attacca con 2 colpi di spada bastarda.

\emph{\textbf{Spada Bastarda.} Attacco con arma da mischia}: +8 a colpire, portata 2 m, un bersaglio.

\emph{Colpisce:} 10 (1d10 + 5) danni contundenti, più 1d6 danno da Vuoto. Se la creatura colpita è un Seguace o Devoto di Gradh il danno aumenta di un ulteriore 1d6.

\emph{\textbf{Evocare Gablin}}: 3 Azioni. Il Gablin spilla il suo sangue a terra e a questo sorgono 3d4 Gablin.

\textbf{Reazione: \emph{Attacco d'opportunità}}: il Paladino Gablin effettua un attacco ad una creatura che attraversi o esca dalla sua portata di 1 metro.

\textbf{Aura immonda}: il Paladino Gablin emana un aura di 6 metri di raggio intorno a lui che conferisce +2 al Tiro per Colpire ed al Danno a tutti gli altri Gablin ed impone -2 al Tiro per Colpire e TS alle altre creature non Devoti o Seguaci di Cattalm.

\textbf{Ecologia}\\
Ambiente: Qualsiasi\\
Organizzazione: a capo di un armata di Gablin\\
\textbf{Categoria Tesoro}: Armatura da campo, Spada Bastarda +1, S\\
\textbf{Descrizione}\\
I Paladini Gablin sono tra i più potenti gablin che si conoscano, i veri eletti di Cattalm. Evocati da più potenti seguaci di Cattalm possono da soli guidare centinaia di Gablin e grazia al loro acume preparare accurati piani e portare scompiglio e distruzione in intere regioni.

\mostro{Gargoyle}
\noindent
\begin{description}[noitemsep, topsep=0pt, parsep=0pt, partopsep=0pt, leftmargin=0cm, labelwidth=2.2cm]
	\item[\textbf{Taglia/Tipo:}] Media elementale, malvagio
	\item[\textbf{Caratt.:}] \resizebox{0.5\linewidth+1.8cm}{!}{For 2 Des 0 Cos 3 Int -2 Sag 0 Car -2}
	\item[\textbf{Punti Ferita:}] 52,  \textbf{Difesa:} 14,  \textbf{Iniziativa:} +0
	\item[\textbf{Movimento:}] 9 m, volo 18 m
	\item[\textbf{Tiri Salvez.:}] \resizebox{0.5\linewidth+1.8cm}{!}{Tempra +5, Riflessi +3, Volontà +3}
	\item[\textbf{Res. Danni:}] Veleno, da arma non magica o che non siano di adamantio
	\item[\textbf{Immunità:}] pietrificato, affaticato
	\item[\textbf{Sensi:}] Scurovisione 18 m
	\item[\textbf{Linguaggi:}] Tremun
	\item[\textbf{Sfida:}] 2 (450 PX)\smallskip
\end{description}

\emph{\textbf{Falso Aspetto.}} Mentre il gargoyle rimane immobile, è indistinguibile da una statua inanimata.

\emph{\textbf{Natura Elementale.}} Una gargoyle non ha bisogno di aria, cibo, bevande o sonno.

\textbf{Azioni}

\emph{\textbf{Multiattacco.}} Il gargoyle effettua due attacchi: uno con il morso e uno con gli artigli.

\emph{\textbf{Artigli.} Attacco con arma da mischia}: +5 a colpire, portata 1 m, un bersaglio.

\emph{Colpisce:} 5 (1d6 + 2) danni taglienti, 1 danno da Sanguinamento.

\emph{\textbf{Morso.} Attacco con arma da mischia}: +5 a colpire, portata 1 m, un bersaglio.

\emph{Colpisce:} 5 (1d6 + 2) danni perforanti.

\textbf{Reazione: \emph{Attacco d'opportunità}}: il gargoyle attacca se sta volando ed una creatura esce o attraversa la sua portata di 1 m.

\textbf{Ecologia}
Ambiente: Qualsiasi\\
Organizzazione: Solitario, coppia o stormo (3-12)\\
\textbf{Categoria Tesoro}: Q\\
\textbf{Descrizione}\\
I gargoyle spesso sembrano essere statue alate di pietra, poiché possono rimanere immobili indefinitamente per poi sorprendere i nemici. I gargoyle tendono a comportamenti ossessivo-compulsivi, tanto diversi quanto abbondante è la loro specie. Libri, ninnoli rubati, armi e trofei raccolti dai nemici caduti sono solo alcuni esempi dei tipi di oggetti che un gargoyle può collezionare per decorare la sua tana e il suo territorio.

I gargoyle tendono ad avere uno stile di vita solitario, anche se a volte formano temibili stormi detti ali per protezione e divertimento. In certe condizioni, una tribù di gargoyle può persino allearsi con altre creature, ma anche la più stabile di queste alleanze può crollare per ragioni infime; i gargoyle sono solo traditori, meschini e vendicativi.

I gargoyle sono noti per abitare nel cuore delle città più grandi, accovacciati tra le decorazioni di pietra delle cattedrali e degli edifici dove si nascondono in bella vista di giorno piombando giù per nutrirsi di vagabondi, mendicanti e altri sfortunati la notte.

più a lungo una tribù di gargoyle dimora in un'area di edifici o rovine, più i suoi membri cominciano ad assomigliare allo stile architettonico della zona. I cambiamenti subiti dall'aspetto di un gargoyle sono lenti e sottili, ma nel corso degli anni possono diventare radicali.

Un'insolita variante del gargoyle non abita tra edifici e rovine ma sotto le onde del mare. Queste creature sono note come kapoacinth; hanno le stesse statistiche base dei gargoyle normali, eccetto che hanno il sottotipo acquatico e le loro ali gli garantiscono una velocità di nuotare di 12 metri (ma sono inutili per volare). I kapoacinth abitano nelle regioni costiere poco profonde dove possono strisciare fuori dalla spuma per dare la caccia ai residenti della zona. È più probabile che formino stormi, poiché i kapoacinth preferiscono la vita di gruppo a quella solitaria.

\mostro{G.E.C.}
\noindent
\begin{description}[noitemsep, topsep=0pt, parsep=0pt, partopsep=0pt, leftmargin=0cm, labelwidth=2.2cm]
	\item[\textbf{Taglia/Tipo:}] Grande aberrazione, malvagio
	\item[\textbf{Caratt.:}] \resizebox{0.5\linewidth+1.8cm}{!}{For 6 Des 1 Cos 5 Int 3 Sag 1 Car -1}
	\item[\textbf{Punti Ferita:}] 205,  \textbf{Difesa:} 26,  \textbf{Iniziativa:} +3
	\item[\textbf{Movimento:}] 9 m, scavare 9 m
	\item[\textbf{Tiri Salvez.:}] \resizebox{0.5\linewidth+1.8cm}{!}{\resizebox{0.5\linewidth+1.8cm}{!}{Tempra +15, Riflessi +11, Volontà +11}}
	\item[\textbf{Comp.:}] Consapevolezza +10
	\item[\textbf{Sensi:}] Scurovisione 18 m, senso tellurico 18 m
	\item[\textbf{Sfida:}] 10 (5900 PX)\smallskip
\end{description}

\textbf{Azioni}

\emph{\textbf{Multiattacco.}} Il G.E.C. può attaccare con due artigli oppure con il morso

\textbf{Artigli}: Attacco con arma naturale da mischia: +11 a colpire, portata 3 m, un bersaglio.

\emph{Colpisce:} 20 (6d6 + 5) danni taglienti, 1 danno da Sanguinamento.

\textbf{Morso}: Attacco con arma naturale da mischia: +11 al colpire, portata 3 m, un bersaglio

\emph{Colpisce:} 22 (6d6 + 8) danni taglienti, 1 danno da Sanguinamento, Visione Offuscata.

\textbf{Visione Offuscata:} è un effetto da Veleno, TS Volontà DC 18 oppure fino alla fine del round successivo il bersaglio ha -1d6 al Tiro per Colpire.

\emph{\textbf{Sguardo.}} E' sufficiente guardare il G.E.C. per essere affetti da \hyperlink{Confusione}{Confusione}, come omonimo incantesimo. Per resistere è necessario effettuare un Tiro Salvezza su Volontà a DC 22. Ogni round è possibile ripetere il Tiro Salvezza per resistere all'effetto.

Combattere senza guardare il G.E.C. impone -1d6 al Tiro per Colpire.

\emph{\textbf{Arrabbiato:}} il G.E.C. emette un ruggito cacofonico. Le creature entro 6 metri da lui devono fare un Tiro Salvezza su Volontà a DC 22 o essere affetti da Confusione per 2 round. Costa 2 Azioni.

\textbf{Ecologia}\\
Ambiente: Sotterraneo\\
Organizzazione: solitario, gruppo (2-4) \\
\textbf{Categoria Tesoro}: Accidentale\\
\textbf{Descrizione}\\
Il Grande Essere Chitinoso, o G.E.C, è un insetto dal vago aspetto umanoide di quasi 3 metri di altezza, possente e dotato di due chele fortissime e resistenti capaci di scavare e tranciare qualsiasi materiale. 4 occhi piccoli, centrali e multi faccettati emanano un fioca luminescenza cangiante che confondono le creature che incrociano il loro sguardo.

Probabilmente frutto di una qualche incantesimo di trasformazione andato a male i G.E.C. sono padroni del sottosuolo. Creature dotate di una reale intelligenza amano la carne di elfo e combattono in maniera tattica ed accorta.

\mostro{Djinni}
\noindent
\begin{description}[noitemsep, topsep=0pt, parsep=0pt, partopsep=0pt, leftmargin=0cm, labelwidth=2.2cm]
	\item[\textbf{Taglia/Tipo:}] Grande elementale, buono
	\item[\textbf{Caratt.:}] \resizebox{0.5\linewidth+1.8cm}{!}{For 5 Des 2 Cos 6 Int 2 Sag 3 Car 5}
	\item[\textbf{Punti Ferita:}] 226,  \textbf{Difesa:} 28,  \textbf{Iniziativa:} +2
	\item[\textbf{Movimento:}] 9 m, volo 27 m
	\item[\textbf{Tiri Salvez.:}] \resizebox{0.5\linewidth+1.8cm}{!}{\resizebox{0.5\linewidth+1.8cm}{!}{Tempra +17, Riflessi +13, Volontà +14}}
	\item[\textbf{Imm. Danni:}] Elettricità, Suono
	\item[\textbf{Sensi:}] Scurovisione 36 m
	\item[\textbf{Linguaggi:}] Ictun
	\item[\textbf{Sfida:}] 11 (7200 PX)\smallskip
\end{description}

\emph{\textbf{Decesso Elementale.}} Se il djinni muore, il suo corpo si disintegra in una brezza calda, lasciando dietro di sé solo l'equipaggiamento che il djinni stava indossando o trasportando.

\emph{\textbf{Incantesimi Innati.}} La caratteristica da incantatore innato del djinni è il Carisma 17. Può lanciare in maniera innata i seguenti incantesimi, senza bisogno di componenti materiali:

A volontà: \emph{\hyperlink{Conoscere i Tratti}{Conoscere i Tratti}, \hyperlink{Individuazione del Magico}{Individuazione del Magico}, \hyperlink{Onda Tonante}{Onda Tonante}}

3/giorno ciascuno: \emph{\hyperlink{Camminare nel Vento}{Camminare nel Vento}, \hyperlink{Creare Cibo e Acqua}{Creare Cibo e Acqua}} (può creare vino al posto dell'acqua), \emph{\hyperlink{Lingue}{Lingue}}

1/giorno ciascuno: \emph{\hyperlink{Creazione}{Creazione}}, \emph{\hyperlink{Evoca Elementale}{Evoca Elementale}} (solo elementale dell'aria), \emph{\hyperlink{Forma Gassosa}{Forma Gassosa}, \hyperlink{Immagine Maggiore}{Immagine Maggiore}, \hyperlink{Invisibilità}{Invisibilità}}

\textbf{Azioni}

\emph{\textbf{Multiattacco.}} Il djinni effettua tre attacchi di scimitarra.

\emph{\textbf{Scimitarra.} Attacco con arma da mischia}: +11 a colpire, portata 1 m, un bersaglio.

\emph{Colpisce:} 12 (2d6 + 5) danni taglienti più 3 (1d6) danni da elettricità o suono (a scelta del gin).

\textbf{Reazione: \emph{Nube improvvisa}} il djinni subisce un colpo critico, diventa immediatamente di vapore fino alla fine del round. Costa 1 Azione tornare in forma solida.

\emph{\textbf{Creare Turbine.}} Un cilindro d'aria turbinante di 1 metro di raggio e alto 9 metri si forma magicamente in un punto visibile al djinni entro 36 metri da esso. Il turbine resta finché il djinni mantiene la concentrazione (come se si stesse concentrando su di un incantesimo). Qualsiasi creatura salvo il djinni che entri nel turbine deve riuscire un Tiro Salvezza di Tempra DC 23 o restare intralciata da esso. Il djinni può muovere il turbine di massimo 18 metri con un'Azione, e le creature intralciate dal turbine si muovono con esso. Il turbine termina se il djinni lo perde di vista.

Una creatura può usare una Azione per liberare una creatura intralciata dal turbine, compresa se stessa, riuscendo un Tiro Salvezza Tempra con Forza DC 22. Se la prova riesce, la creatura non è più intralciata e si sposta nello spazio più vicino all'esterno del turbine.

\textbf{Ecologia}
Ambiente: Qualsiasi (Piano dell'Aria)\\
Organizzazione: Solitario, coppia, compagnia (3-6) o banda (7-10)\\
\textbf{Categoria Tesoro}: Scimitarra Perfetta, U\\
\textbf{Descrizione}\\
I Djinn (singolare djinni) sono Geni provenienti dal Piano Elementale dell'Aria. Si dice che siano fatti di nuvole e abbiano la forza delle tempeste più potenti. Un Djinni è alto circa 3 metri e pesa circa 500 kg.

I Djinn disdegnano il Combattimento fisico, preferendo usare i loro poteri Magici e capacità aeree contro i nemici. Un Djinni sconfitto in Combattimento generalmente prende il volo e diventa un turbine per molestare chi lo insegue. Quando non ha altra scelta che combattere in mischia, la maggioranza dei Djinn preferisce impugnare Scimitarre a Due Mani Perfette.

Verso gli altri Geni, i Djinn vanno d'accordo con gli Janni e i Marid. Sono frequentemente in contrasto con gli Shaitan, e sono nemici giurati degli Efreeti, disprezzando questi Geni feroci più di qualsiasi altra delle Razze di Geni. Il conflitto tra gli Efreeti e i Djinn è così leggendario che molti incantatori tentano (con vari gradi di successo) di assicurarsi il servizio di un Djinni promettendogli aiuto nella causa contro gli odiati nemici.

\mostro{Efreeti}
\noindent
\begin{description}[noitemsep, topsep=0pt, parsep=0pt, partopsep=0pt, leftmargin=0cm, labelwidth=2.2cm]
	\item[\textbf{Taglia/Tipo:}] Grande elementale, malvagio
	\item[\textbf{Caratt.:}] \resizebox{0.5\linewidth+1.8cm}{!}{For 6 Des 1 Cos 7 Int 3 Sag 2 Car 3}
	\item[\textbf{Punti Ferita:}] 228,  \textbf{Difesa:} 27,  \textbf{Iniziativa:} +3
	\item[\textbf{Movimento:}] 12 m, volo 18 m
	\item[\textbf{Tiri Salvez.:}] \resizebox{0.5\linewidth+1.8cm}{!}{\resizebox{0.5\linewidth+1.8cm}{!}{Tempra +18, Riflessi +12, Volontà +13}}
	\item[\textbf{Imm. Danni:}] Fuoco
	\item[\textbf{Sensi:}] Scurovisione 36 m
	\item[\textbf{Linguaggi:}] Ignan
	\item[\textbf{Sfida:}] 11 (7200 PX)\smallskip
\end{description}

\emph{\textbf{Decesso Elementale.}} Se l'efreeti muore, il suo corpo si disintegra in un lampo di fuoco e uno sbuffo di fumo, lasciando dietro di sé solo l'equipaggiamento che l'efreeti stava indossando o trasportando.

\emph{\textbf{Incantesimi Innati.}} La caratteristica da incantatore innato dell'efreeti è il Carisma. Può lanciare in maniera innata i seguenti incantesimi, senza bisogno di componenti materiali:

A volontà: \emph{\hyperlink{Individuazione del Magico}{Individuazione del Magico}}

3/giorno ciascuno: \emph{\hyperlink{Ingrandire/Ridurre}{Ingrandire/Ridurre}, \hyperlink{Lingue}{Lingue}}

1/giorno ciascuno: \emph{\hyperlink{Evoca Elementale}{Evoca Elementale}} (solo elementale del fuoco), \emph{\hyperlink{Forma Gassosa}{Forma Gassosa}, \hyperlink{Immagine Maggiore}{Immagine Maggiore}}, \emph{\hyperlink{Invisibilità}{Invisibilità}, \hyperlink{Muro di Fuoco}{Muro di Fuoco}}

\textbf{Azioni}

\emph{\textbf{Multiattacco.}} L'efreeti effettua due attacchi di scimitarra o usa due volte Scagliare Fiamma.

\emph{\textbf{Scimitarra.} Attacco con arma da mischia}: +11 a colpire, portata 1 m, un bersaglio.

\emph{Colpisce:} 13 (2d6 + 6) danni taglienti più 7 (2d6) danni da fuoco.

\emph{\textbf{Scagliare Fiamma.} Attacco con arma a Distanza}: +12 a colpire, gittata 36 m, un bersaglio.

\emph{Colpisce:} 17 (5d6) danni da fuoco.

\textbf{Reazione: \emph{Attacco d'opportunità}}: efreeti effettua un attacco ad una creatura che attraversi o esca dalla sua portata di 1 metro.

\textbf{Ecologia}
Ambiente: Qualsiasi (Piano del Fuoco)\\
Organizzazione: Solitario, coppia, compagnia (3-6) o banda (7-12)\\
\textbf{Categoria Tesoro}: Falcione Perfetto, U\\
\textbf{Descrizione}\\
Gli Efreet (singolare Efreeti) sono Geni provenienti dal Piano del Fuoco. Sono alti 3,6 metri e pesano circa 1000 kg.

Gli Efreet hanno pochi alleati tra gli altri Geni: odiano i Djinni e li attaccano a vista, non sopportano i Marid e vedono i Janni come deboli e fragili. Gli Efreet spesso cooperano bene con gli Shaitan, eppure anche queste alleanze sono temporanee.

\mostro{Ghast}
\noindent
\begin{description}[noitemsep, topsep=0pt, parsep=0pt, partopsep=0pt, leftmargin=0cm, labelwidth=2.2cm]
	\item[\textbf{Taglia/Tipo:}] Media non morto, malvagio
	\item[\textbf{Caratt.:}] \resizebox{0.5\linewidth+1.8cm}{!}{For 3 Des 3 Cos 0 Int 0 Sag 0 Car -1}
	\item[\textbf{Punti Ferita:}] 51,  \textbf{Difesa:} 17,  \textbf{Iniziativa:} +3
	\item[\textbf{Movimento:}] 9 m
	\item[\textbf{Tiri Salvez.:}] \resizebox{0.5\linewidth+1.8cm}{!}{Tempra +3, Riflessi +5, Volontà +3}
	\item[\textbf{Res. Danni:}] da Vuoto
	\item[\textbf{Imm. Danni:}] Veleno
	\item[\textbf{Immunità:}] affascinato, affaticato
	\item[\textbf{Sensi:}] Scurovisione 18 m
	\item[\textbf{Linguaggi:}] Comune, Expiran
	\item[\textbf{Sfida:}] 2 (450 PX)\smallskip
\end{description}

\emph{\textbf{Fetore.}} Qualsiasi creatura che inizi il suo round entro 1 metro dal ghast deve riuscire un Tiro Salvezza di Tempra DC 14 o restare Nauseata (-1d6 a TC, TS e Prove) fino all'inizio del suo prossimo round. Se riesce il Tiro Salvezza, la creatura è immune al Fetore del ghast per le successive 24 ore.

\emph{\textbf{Ribellione allo Scacciare.}} Il ghast e tutti i ghoul entro 9 metri da esso hanno +1d6 ai Tiri Salvezza contro gli effetti che scacciano i non morti.

\textbf{Azioni}

\emph{\textbf{Artigli.} Attacco con arma da mischia}: +5 a colpire, portata 1 m, un bersaglio.

\emph{Colpisce:} 10 (2d6 + 3) danni taglienti. Se il bersaglio è una creatura, diversa da un non morto, deve riuscire un Tiro Salvezza su Tempra DC 14 o restare paralizzata per 1 minuto. Il bersaglio può ripetere il Tiro Salvezza al termine di ciascun suo round, terminando l'effetto se riesce il Tiro Salvezza.

\emph{\textbf{Morso.} Attacco con arma da mischia}: +5 a colpire, portata 1 m, una creatura.

\emph{Colpisce:} 12 (2d8 + 3) danni perforanti.

\emph{\textbf{Morso affamato.} Attacco con arma da mischia}: +5 a colpire, portata 1 m, una creatura. 2 Azioni

\emph{Colpisce:} 14 (3d6 + 3) danni perforanti, il ghast recupera la metà del danno in Punti Ferita.

\textbf{Ecologia}\\
Ambiente: Qualsiasi terreno\\
Organizzazione: Solitario, gruppo (2-4) o branco (7-12)\\
\textbf{Categoria Tesoro}: B\\
\textbf{Descrizione}\\
I ghast sono Ghoul con un legame più profondo con il Vuoto. La paralisi di un ghast ha effetto anche sugli Elfi. I ghast si aggirano in branchi o comandano gruppi di Ghoul comuni. Il fetore di morte e putrefazione che circonda queste creature è travolgente.

\mostro{Ghoul}
\noindent
\begin{description}[noitemsep, topsep=0pt, parsep=0pt, partopsep=0pt, leftmargin=0cm, labelwidth=2.2cm]
	\item[\textbf{Taglia/Tipo:}] Media non morto, malvagio
	\item[\textbf{Caratt.:}] \resizebox{0.5\linewidth+1.8cm}{!}{For 1 Des 2 Cos 0 Int -2 Sag 0 Car -2}
	\item[\textbf{Punti Ferita:}] 33,  \textbf{Difesa:} 15,  \textbf{Iniziativa:} +2
	\item[\textbf{Movimento:}] 9 m
	\item[\textbf{Tiri Salvez.:}] \resizebox{0.5\linewidth+1.8cm}{!}{\resizebox{0.5\linewidth+1.8cm}{!}{Tempra +3, Riflessi +3, Volontà +3}}
	\item[\textbf{Imm. Danni:}] Veleno
	\item[\textbf{Immunità:}] affascinato, affaticato
	\item[\textbf{Sensi:}] Scurovisione 18 m
	\item[\textbf{Linguaggi:}] Comune
	\item[\textbf{Sfida:}] 1 (200 PX)\smallskip
\end{description}

\textbf{Azioni}

\emph{\textbf{Artigli.} Attacco con arma da mischia}: +4 a colpire, portata 1 m, un bersaglio.

\emph{Colpisce:} 7 (2d4 + 2) danni taglienti, 1 danno da Sanguinamento. Se il bersaglio è una creatura, diversa da un elfo o un non morto, deve riuscire un Tiro Salvezza su Tempra DC 13 o restare paralizzata per 1 minuto. Il bersaglio può ripetere il Tiro Salvezza al termine di ciascun suo round, terminando l'effetto se riesce il Tiro Salvezza.

\emph{\textbf{Morso.} Attacco con arma da mischia}: +4 a colpire, portata 1 m, una creatura.

\emph{Colpisce:} 9 (2d6 + 2) danni perforanti.

\textbf{Ecologia}
Ambiente: Qualsiasi terreno\\
Organizzazione: Solitario, gruppo (2-4) o branco (7-12)\\
\textbf{Categoria Tesoro}: K\\
\textbf{Descrizione}\\
I ghoul sono non morti che frequentano i cimiteri e mangiano i cadaveri. Le leggende sostengono che i primi ghoul fossero umani cannibali che una fame innaturale ha riportato indietro dalla morte, oppure umani che in vita si nutrivano dei resti in decomposizione dei loro simili e che morirono (e poi rinacquero) a causa di un'orrenda malattia; la vera origine di questi non morti necrofagi è incerta.

I ghoul si appostano ai margini della civilizzazione (dentro o nei pressi dei cimiteri o nelle fogne cittadine) dove possono reperire ampie scorte del loro cibo preferito. Sebbene preferiscano i corpi in putrefazione e spesso seppelliscano le loro vittime per migliorarne il sapore, mangiano i morti freschi se hanno abbastanza fame.

Anche se molti ghoul di superficie vivono in modo primitivo, delle voci parlano di città di ghoul nelle profondità del sottosuolo comandate da sacerdoti che adorano antiche divinità crudeli o strani signori dei demoni della fame. Questi ghoul \emph{civilizzati} non sono meno orribili nelle loro abitudini alimentari, e in effetti il loro concetto di tavola ben imbandita per banchetti è forse anche più orrendo dell'idea di un pasto fresco prelevato da una bara.

\mostro{Ghoul, Nero}
\noindent
\begin{description}[noitemsep, topsep=0pt, parsep=0pt, partopsep=0pt, leftmargin=0cm, labelwidth=2.2cm]
	\item[\textbf{Taglia/Tipo:}] Media non morto, malvagio
	\item[\textbf{Caratt.:}] \resizebox{0.5\linewidth+1.8cm}{!}{For 4 Des 2 Cos 2 Int 0 Sag 1 Car -2}
	\item[\textbf{Punti Ferita:}] 125,  \textbf{Difesa:} 22,  \textbf{Iniziativa:} +2
	\item[\textbf{Movimento:}] 12 m
	\item[\textbf{Tiri Salvez.:}] \resizebox{0.5\linewidth+1.8cm}{!}{Tempra +8, Riflessi +8, Volontà +7}
	\item[\textbf{Imm. Danni:}] Veleno, da Vuoto, da critico, sanguinamento,
	\item[\textbf{Immunità:}] affascinato, affaticato,
	\item[\textbf{Res. Danni:}] armi non magiche o d'argento
	\item[\textbf{Sensi:}] Scurovisione 18 m
	\item[\textbf{Linguaggi:}] Comune, Expiran
	\item[\textbf{Sfida:}] 6 (2300 PX)\smallskip
\end{description}

\textbf{\emph{Aura nefasta}}: il Ghoul Nero emana costantemente un aura attorno a se che indebolisce le difese di chiunque tranne che di altri ghoul. Ogni due round di permanenza nell'aura di 12 metri di raggio attorno al Ghoul Nero si cumula un -1 a tutti i TS, quando ci si allontana dal Ghoul Nero si recupera 1 punto a round.

\textbf{Azioni}

\emph{\textbf{Artigli.} Attacco con arma da mischia}: +8 a colpire, portata 1 m, un bersaglio.

\emph{Colpisce:} 15 (2d10 + 4) danni taglienti, 2 danno da Sanguinamento. Se il bersaglio è una creatura, diversa da un elfo o un non morto, deve riuscire un Tiro Salvezza su Tempra DC 17 o restare paralizzata per 1 minuto. Il bersaglio può ripetere il Tiro Salvezza al termine di ciascun suo round, terminando l'effetto se riesce il Tiro Salvezza.

\emph{\textbf{Morso.} Attacco con arma da mischia}: +9 a colpire, portata 1 m, una creatura.

\emph{Colpisce:} 18 (3d8 + 6) danni perforanti, 1 da Sanguinamento, Malattia del Ghoul

\textbf{Reazione: \emph{Attacco d'opportunità}}: il Ghoul Nero effettua un attacco ad una creatura che attraversi o esca dalla sua portata di 1 metro.

\emph{Malattia del Ghoul:} 3 giorni, TS Tempra DC 18, 6 ore, 3 successi, -1 Costituzione, ti trasformi in un Ghoul

\textbf{Ecologia}
Ambiente: Qualsiasi terreno\\
Organizzazione: Gruppo (4-8) o branco (14-24)\\
\textbf{Categoria Tesoro}: B\\
\textbf{Descrizione}\\
Il Ghoul Nero rappresenta una delle elite evolutive dei Ghoul. Solitamente a capo di un gruppo almeno un ghoul putrescente di circa 18 ghoul.

%\begin{center}
%\includegraphics[width=0.45\textwidth]{immagini/Pazin_Burgmuseum_-_Waffen_1.png}
%\end{center}

\mostro{Ghoul, Madre}
\noindent
\begin{description}[noitemsep, topsep=0pt, parsep=0pt, partopsep=0pt, leftmargin=0cm, labelwidth=2.2cm]
	\item[\textbf{Taglia/Tipo:}] Media non morto, malvagio
	\item[\textbf{Caratt.:}] \resizebox{0.5\linewidth+1.8cm}{!}{For 0 Des 3 Cos 2 Int 2 Sag 1 Car 2}
	\item[\textbf{Punti Ferita:}] 107,  \textbf{Difesa:} 21,  \textbf{Iniziativa:} +3
	\item[\textbf{Movimento:}] 9 m
	\item[\textbf{Tiri Salvez.:}] \resizebox{0.5\linewidth+1.8cm}{!}{Tempra +7, Riflessi +8, Volontà +6}
	\item[\textbf{Imm. Danni:}] Veleno, Vuoto, da critico, sanguinamento
	\item[\textbf{Immunità:}] affascinato, affaticato
	\item[\textbf{Res. Danni:}] armi non magiche
	\item[\textbf{Sensi:}] Scurovisione 18 m
	\item[\textbf{Linguaggi:}] Comune, Expiran
	\item[\textbf{Sfida:}] 5 (1800 PX)\smallskip
\end{description}

\textbf{Azioni}

\emph{\textbf{Artigli.} Attacco con arma da mischia}: +6 a colpire, portata 1 m, un bersaglio.

\emph{Colpisce:} 12 (2d6 + 6) danni taglienti, 2 danno da Sanguinamento. Se il bersaglio è una creatura diverso da un non morto, deve riuscire un Tiro Salvezza su Tempra DC 17 o restare paralizzata per 1 minuto. Il bersaglio può ripetere il Tiro Salvezza al termine di ciascun suo round, terminando l'effetto se riesce il Tiro Salvezza. Se la creatura fallisce il TS allora è vittima della maledizione del Ghoul. Entro 1d3+1 giorni si trasformerà in un Ghoul. E' necessario un Scacciare Maledizioni DC 19 entro la trasformazione per evitare la trasformazione.

\emph{\textbf{Morso.} Attacco con arma da mischia}: +6 a colpire, portata 1 m, una creatura.

\emph{Colpisce:} 8 (2d6 + 2) danni perforanti.

\textbf{Ecologia}
Ambiente: Qualsiasi terreno\\
Organizzazione: Clan (7-12+)\\
\textbf{Categoria Tesoro}: I\\
\textbf{Descrizione}\\
La Madre Ghoul è solitamente a capo di un clan di ghoul che può raggiungere anche diverse decine di membri. Rispettata e temuta è solitamente tra i ghoul evoluti più intelligenti e molto apprezzata per la sua capacità di poter trasformare in ghoul i viventi. La loro tattica prevede di ferire e non uccidere diverse persone così che tornate a casa e poi trasformati possano attaccare ed uccidere tutto il villaggio.

%\begin{center}
%\includegraphics[width=0.45\textwidth]{immagini/Mexican_machete.png}
%\end{center}

\mostro{Ghoul, putrescente}
\noindent
\begin{description}[noitemsep, topsep=0pt, parsep=0pt, partopsep=0pt, leftmargin=0cm, labelwidth=2.2cm]
	\item[\textbf{Taglia/Tipo:}] Grande non morto, malvagio
	\item[\textbf{Caratt.:}] \resizebox{0.5\linewidth+1.8cm}{!}{For 1 Des 2 Cos 3 Int -1 Sag 0 Car -2}
	\item[\textbf{Punti Ferita:}] 89,  \textbf{Difesa:} 19,  \textbf{Iniziativa:} +2
	\item[\textbf{Movimento:}] 6 m
	\item[\textbf{Tiri Salvez.:}] \resizebox{0.5\linewidth+1.8cm}{!}{Tempra +7, Riflessi +6, Volontà +4}
	\item[\textbf{Imm. Danni:}] Veleno, sanguinamento, da critico, da Vuoto
	\item[\textbf{Immunità:}] affascinato, affaticato
	\item[\textbf{Res. Danni:}] armi non magiche o d'argento
	\item[\textbf{Sensi:}] Scurovisione 36 m
	\item[\textbf{Linguaggi:}] Comune, Expiran
	\item[\textbf{Sfida:}] 4 (1100 PX)\smallskip
\end{description}

\textbf{\emph{Rigenerazione}}. Il Ghoul Putrescente rigenera 5 Punti Ferita a round tranne se è in piena luce solare o ha subito danni da Luce nel round precedente. Se il Ghoul Putrescente è in un cimitero recupera 10 Punti Ferita a round.

\textbf{Azioni}

\emph{\textbf{Artigli.} Attacco con arma da mischia}: +6 a colpire, portata 2 m, un bersaglio.

\emph{Colpisce:} 12 (2d10 + 2) danni taglienti, 1 danno da Sanguinamento. Se il bersaglio è una creatura, diversa da un non morto, deve riuscire un Tiro Salvezza su Tempra DC 15 o restare paralizzata per 1 minuto.

\emph{\textbf{Morso.} Attacco con arma da mischia}: +6 a colpire, portata 1 m, una creatura.

\emph{Colpisce:} 10 (2d8 + 2) danni perforanti.

\textbf{Reazione: \emph{Attacco d'opportunità}}: il Ghoul Putrescente effettua un attacco ad una creatura che attraversi o esca dalla sua portata di 1 metro.

\emph{\textbf{Aura di Sofferenza.}}: il Ghoul Putrescente emana un aura di 6 metri intorno a lui, ogni attacco di ghoul andato a segno causa automaticamente un danno critico. Attivare questa aura costa 2 Azioni e dura fino all'inizio del round successivo.

\textbf{Ecologia}
Ambiente: Qualsiasi terreno\\
Organizzazione: Gruppo (4-8) o branco (10-18)\\
\textbf{Categoria Tesoro}: Nessuno\\
\textbf{Descrizione}\\
I Ghoul Putrescenti sono una delle tante l'evoluzione dei Ghoul. Il contatto continuo con l'energia negativa ed il nutrirsi per secoli di cadaveri di ogni genere lo hanno reso più grande, forte e capace di infliggere e fare infliggere le ferite più pericolose.

\mostro{Gigante delle Colline}
\noindent
\begin{description}[noitemsep, topsep=0pt, parsep=0pt, partopsep=0pt, leftmargin=0cm, labelwidth=2.2cm]
	\item[\textbf{Taglia/Tipo:}] Enorme gigante, malvagio
	\item[\textbf{Caratt.:}] \resizebox{0.5\linewidth+1.8cm}{!}{For 5 Des -1 Cos 4 Int -3 Sag -1 Car -2}
	\item[\textbf{Punti Ferita:}] 109,  \textbf{Difesa:} 17,  \textbf{Iniziativa:} -1
	\item[\textbf{Movimento:}] 12 m
	\item[\textbf{Tiri Salvez.:}] \resizebox{0.5\linewidth+1.8cm}{!}{Tempra +9, Riflessi +4, Volontà +4}
	\item[\textbf{Linguaggi:}] Gigante
	\item[\textbf{Sfida:}] 5 (1800 PX)\smallskip
\end{description}

\textbf{Azioni}

\emph{\textbf{Multiattacco.}} Il gigante effettua due attacchi con il randello pesante.

\emph{\textbf{Randello Pesante.} Attacco con arma da mischia}: +7 a colpire, portata 3 m, un bersaglio.

\emph{Colpisce:} 18 (3d8 + 5) danni contundenti.

\emph{\textbf{Ampio Fendente.} Attacco con arma da mischia}: +7 a colpire, portata 3 metri, con un singolo attacco può colpire due creature in mischia vicine tra loro.

\emph{\textbf{Sasso.} Attacco con arma a Distanza}: +5 a colpire, gittata 18m, un bersaglio.

\emph{Colpisce:} 21 (3d10 + 5) danni contundenti.

\textbf{Ecologia}\\
Ambiente: Colline Temperate\\
Organizzazione: Solitario, gruppo (2-5), banda (6-8), gruppo di razzia (9-12 più 1d4 Lupi Crudeli) o tribù (13-30 più 35\% non combattente più 1 capo combattente di 4°-6° livello, 11-16 Lupi Crudeli, 1-4 Ogre e 13-20 schiavi orchi)\\
\textbf{Categoria Tesoro}: Armatura di Pelle, Randello Pesante, B\\
\textbf{Descrizione}\\
I giganti di Collina hanno pelle che varia dal marrone chiaro al rossastro, capelli castani o neri, ed occhi dello stesso colore. Indossano strati di pelli rozzamente conciate con ancora il pelo. Raramente lavano o riparano i propri indumenti, e preferiscono semplicemente aggiungere nuovi strati man mano che i vecchi si logorano. Gli adulti sono alti circa 3 metri e pesano più o meno 550 kg. I giganti di Collina possono vivere fino a 200 anni, anche se raramente raggiungono quest'età.

I giganti di Collina preferiscono combattere dall'alto di sporgenze e rupi, da dove possono colpire gli avversari con rocce e massi, limitando così il rischio personale. Amano effettuare attacchi di oltrepassare contro creature più piccole all'inizio del combattimento, e solo dopo prendono posizione e iniziano a roteare i loro massicci randelli.

I giganti di Collina sono per natura nomadi e preferiscono viaggiare da un luogo all'altro per razziare e saccheggiare. Sebbene gradiscano di più i climi temperati, non disdegnano di viaggiare lontano dal loro ambiente favorito, se la razzia è abbondante e prospera. Si tratta, nel complesso, di creature molto egoiste, che raramente affrontano battaglie che non siano sicuri di vincere. I giganti delle colline sono noti per l'abitudine di spingersi l'un l'altro se devono confrontarsi con avversari temibili e non esitano a sacrificare un compagno per salvarsi la pelle. Bande erranti di giganti di Collina sono diffuse sulle colline temperate, e la loro costante aggressività li rende uno dei pericoli più temuti in questo ambiente.

I giganti di Collina solitari e non malvagi sono molto rari, ma li si può trovare qualche volta in altre società umanoidi, anche se non sono quasi mai accettati nelle città principali o nei centri popolati. Si trovano a proprio agio come lavoratori e soldati nelle remote città di frontiera, e spesso fungono da rudimentali diplomatici per negoziare con le bande di giganti di Collina razziatori. Sfortunatamente, i giganti di Collina che abbandonano il proprio stile di vita razziale per la civiltà vengono derisi e spesso uccisi a vista dai loro fratelli nomadi. Tuttavia, questi giganti di Collina \emph{civilizzati} possono trovare il proprio posto nella società e molti sono riusciti a vivere un'esistenza pacifica e tranquilla.

\mostro{Gigante del Fuoco}
\noindent
\begin{description}[noitemsep, topsep=0pt, parsep=0pt, partopsep=0pt, leftmargin=0cm, labelwidth=2.2cm]
	\item[\textbf{Taglia/Tipo:}] Enorme gigante, malvagio
	\item[\textbf{Caratt.:}] \resizebox{0.5\linewidth+1.8cm}{!}{For 7 Des -1 Cos 6 Int 0 Sag 2 Car 1}
	\item[\textbf{Punti Ferita:}] 187,  \textbf{Difesa:} 23,  \textbf{Iniziativa:} +0
	\item[\textbf{Movimento:}] 9 m
	\item[\textbf{Tiri Salvez.:}] \resizebox{0.5\linewidth+1.8cm}{!}{\resizebox{0.5\linewidth+1.8cm}{!}{Tempra +15, Riflessi +8, Volontà +11}}
	\item[\textbf{Comp.:}] Atletica +11, Consapevolezza +6
	\item[\textbf{Linguaggi:}] Gigante
	\item[\textbf{Sfida:}] 9 (5000 PX)\smallskip
\end{description}

\textbf{Azioni}

\emph{\textbf{Multiattacco.}} Il gigante effettua due attacchi con lo spadone.

\emph{\textbf{Spadone.} Attacco con arma da mischia}: +11 a colpire, portata 3 m, un bersaglio.

\emph{Colpisce:} 28 (6d6 + 7) danni taglienti.

\emph{\textbf{Ampio Fendente.} Attacco con arma da mischia}: +11 a colpire, portata 3 metri, con un singolo attacco può colpire due creature in mischia vicine tra loro.

\emph{\textbf{Sasso.} Attacco con arma a Distanza}: +10 a colpire, gittata 18m, un bersaglio.

\emph{Colpisce:} 29 (4d10 + 7) danni contundenti.

\textbf{Reazione: \emph{Attacco d'opportunità}}: il gigante del fuoco effettua un attacco ad una creatura che attraversi o esca dalla sua portata di 3 metri.

\emph{\textbf{Arrabbiato:}} il gigante del fuoco convoglia la sua energia sull'arma, questa causa +2d6 danni da fuoco fino al termine del combattimento.

\textbf{Ecologia}
Ambiente: Montagne calde\\
Organizzazione: Solitario, gruppo (2-5), banda (6-12 più un 35\% non combattenti e 1 adepto o Devoto di 1°-2° livello), gruppo di razziatori (6-12 più 1 adepto o mago di 3°-5° livello, 2-5 Segugi Infernali e 2-3 Troll o Ettin) o tribù (20-30 più 1 adepto, mago o Devoto di 6°-7° livello; 1 re Guerriero o guardiaboschi di 8°-9° livello; e 17-38 Segugi Infernali, 12-22 Troll, 7-12 Ettin e 1-2 Draghi Rossi Giovani)\\
\textbf{Categoria Tesoro}: Mezza Armatura, Spadone, P\\
\textbf{Descrizione}\\
I giganti del fuoco sono i giganti più rigidi e marziali, sempre pronti alla guerra e a trattare brutalmente chiunque incontrino. La loro rigida struttura di comando prevede soldati, ufficiali e persino generali, e che tutti obbediscano agli ordini del loro re senza discutere.

I giganti del fuoco hanno capelli arancione brillante che splendono e scintillano come se fossero in fiamme. Un maschio adulto è alto tra i 3,6 e i 4,8 metri, con una cassa toracica di circa 2,7 metri, e pesa circa 3.500 kg. Le femmine sono leggermente più basse e snelle. I giganti del fuoco possono vivere fino a 350 anni.

I giganti del fuoco indossano abiti di tessuti robusti o di pelle di color arancione, giallo, nero o rosso. I guerrieri indossano elmi e mezze armature di acciaio brunito e impugnano grandi spadoni che mulinano per il campo di battaglia. In gruppi numerosi, i giganti del fuoco combattono con tattiche di gruppo brutali ed efficienti, e non esitano a sacrificare qualche compagno per tendere un'imboscata al nemico.

I giganti del fuoco preferiscono i luoghi caldi: più caldi sono meglio è. Si possono trovare nei deserti, nei vulcani, nelle fonti termali e nelle profondità della terra nei pressi di camini lavici. Vivono in castelli, insediamenti fortificati o grandi caverne, e l'architettura di questi luoghi riflette il loro stile di vita rigido e militaristico, con gli ufficiali che abitano in alloggi migliori di quelli dei loro sottoposti.

\mostro{Gigante del Gelo}
\noindent
\begin{description}[noitemsep, topsep=0pt, parsep=0pt, partopsep=0pt, leftmargin=0cm, labelwidth=2.2cm]
	\item[\textbf{Taglia/Tipo:}] Enorme gigante, malvagio
	\item[\textbf{Caratt.:}] \resizebox{0.5\linewidth+1.8cm}{!}{For 6 Des -1 Cos 5 Int -1 Sag 0 Car 1}
	\item[\textbf{Punti Ferita:}] 167,  \textbf{Difesa:} 21,  \textbf{Iniziativa:} -1
	\item[\textbf{Movimento:}] 12 m
	\item[\textbf{Tiri Salvez.:}] \resizebox{0.5\linewidth+1.8cm}{!}{\resizebox{0.5\linewidth+1.8cm}{!}{Tempra +13, Riflessi +7, Volontà +8}}
	\item[\textbf{Comp.:}] Atletica +9
	\item[\textbf{Linguaggi:}] Gigante
	\item[\textbf{Sfida:}] 8 (3900 PX)\smallskip
\end{description}

\textbf{Azioni}

\emph{\textbf{Multiattacco.}} Il gigante effettua due attacchi con l'ascia bipenne.

\emph{\textbf{Ascia Bipenne.} Attacco con arma da mischia}: +10 a colpire, portata 3 m, un bersaglio.

\emph{Colpisce:} 25 (3d12 + 6) danni taglienti.

\emph{\textbf{Ampio Fendente.} Attacco con arma da mischia}: +10 a colpire, portata 3 metri, con un singolo attacco può colpire due creature in mischia vicine tra loro.

\emph{\textbf{Sasso.} Attacco con arma a Distanza}: +9 a colpire, gittata 18m, un bersaglio.

\emph{Colpisce:} 28 (4d10 + 6) danni contundenti.

\textbf{Reazione: \emph{Attacco d'opportunità}}: il gigante del freddo effettua un attacco ad una creatura che attraversi o esca dalla sua portata di 3 metri.

\emph{\textbf{Arrabbiato:}} il Gigante del Gelo canalizza le sue energie attraverso l'arma. L'arma causa un 2d6 di danni aggiuntivi da freddo fino alla fine del combattimento.

\textbf{Ecologia}\\
Ambiente: Montagne fredde\\
Organizzazione: Solitario, banda (3-5), gruppo (6-12 più 35\% non combattenti e 1 mago o Devoto di 1°-2° livello), gruppo di razziatori (6-12 più 35\% non combattenti, 1 Devoto o mago di 3°-5° livello, 1-4 Lupi Invernali e 2-3 Ogre) o tribù (21-30 più 1 adepto, mago o Devoto di 6°-7° livello; 1 jarl Barbaro o guardiaboschi 7°-9° livello; e 15-36 Lupi Invernali, 13-22 Ogre e 1-2 Draghi Bianchi Giovani)\\
\textbf{Categoria Tesoro}: Giaco di Maglia, Ascia Bipenne, R\\
\textbf{Descrizione}\\
Un gigante del gelo ha capelli azzurri o giallo sporco, e occhi in genere dello stesso colore. Si vestono con pelli e pellicce, adornandosi con qualsiasi gioiello possiedano. I giganti del gelo combattenti indossano anche giachi di maglia ed elmi di metallo decorati con corna e piume. Un maschio adulto è alto 5 metri e pesa circa 1.400 kg. Le femmine sono leggermente più basse e snelle, ma per il resto sono identiche ai maschi. I giganti del gelo possono vivere fino a 250 anni.

I giganti del gelo sono molto temuti, poiché la brama di distruzione e guerra ed il loro comportamento sprezzante li spingono a manifestazioni di brutalità sempre maggiori. I giganti del gelo iniziano attaccando a distanza, scagliando rocce finché finiscono le munizioni o l'avversario si avvicina, poi lo affrontano con le loro enormi asce. Una delle tattiche preferite è tendere un'imboscata nascondendosi sotto la neve al di sopra di un pendio ghiacciato o innevato, dove gli avversari avranno difficoltà a raggiungerli, e poi iniziano causando una valanga prima di scendere in battaglia. I giganti del gelo possono nascondersi molto bene negli ambienti nevosi e sono dei maestri nella furtività nel loro dominio.

I giganti del gelo sopravvivono cacciando e razziando da soli, dato che vivono in ambienti freddi e desolati. I gruppi di giganti del gelo sono divisi quasi equamente tra quelli che vivono in insediamenti di fortuna o castelli abbandonati e quelli che vagabondano per il gelido nord, come nomadi in cerca di bottino e provviste. I capi dei giganti del gelo si chiamano jarl e richiedono obbedienza assoluta ai loro seguaci. In ogni momento uno jarl può essere sfidato in combattimento per il comando della tribù. Queste sfide tipicamente finiscono con la morte di uno dei contendenti. Un singolo jarl può spesso contare su una dozzina o più di tribù più piccole di giganti del gelo come estensione della sua. In questi casi, i capi delle tribù minori sono noti come capitani o signori della guerra.

I giganti del gelo amano prendere prigionieri e li usano sia come schiavi che come materia prima. Di solito ogni gruppo di giganti del gelo tiene 1-2 schiavi umanoidi incatenati ad un addestratore di schiavi: il più meschino e crudele del gruppo dopo lo jarl. Hanno anche una certa passione per gli animali domestici mostruosi: Draghi Bianchi e Lupi Invernali sono scelte popolari, ma nella tana di un gigante del gelo si possono trovare anche Remorhaz e Yeti.

\mostro{Gigante delle Nuvole}
\noindent
\begin{description}[noitemsep, topsep=0pt, parsep=0pt, partopsep=0pt, leftmargin=0cm, labelwidth=2.2cm]
	\item[\textbf{Taglia/Tipo:}] Enorme gigante, buono (50\%) o malvagio (50\%)
	\item[\textbf{Caratt.:}] \resizebox{0.5\linewidth+1.8cm}{!}{For 8 Des 0 Cos 6 Int 1 Sag 3 Car 3}
	\item[\textbf{Punti Ferita:}] 187,  \textbf{Difesa:} 24,  \textbf{Iniziativa:} +1
	\item[\textbf{Movimento:}] 12 m
	\item[\textbf{Tiri Salvez.:}] \resizebox{0.5\linewidth+1.8cm}{!}{\resizebox{0.5\linewidth+1.8cm}{!}{Tempra +15, Riflessi +9, Volontà +12}}
	\item[\textbf{Comp.:}] Percepire Emozioni +7
	\item[\textbf{Linguaggi:}] Comune, Gigante
	\item[\textbf{Sfida:}] 9 (5000 PX)\smallskip
\end{description}

\emph{\textbf{Incantesimi Innati.}} La caratteristica da incantatore del gigante è il Carisma. Il gigante può lanciare questi incantesimi in maniera innata, senza bisogno di componenti materiali:

A volontà: \emph{\hyperlink{Individuazione del Magico}{Individuazione del Magico}, \hyperlink{Luce}{Luce}, \hyperlink{Nube di Nebbia}{Nube di Nebbia}}

3/giorno ciascuno: \emph{Caduta Piuma, \hyperlink{Passo Velato}{Passo Velato}, \hyperlink{Telecinesi}{Telecinesi}}

1/giorno ciascuno: \emph{\hyperlink{Controllare Tempo Atmosferico}{Controllare Tempo Atmosferico}, \hyperlink{Forma Gassosa}{Forma Gassosa}}

\emph{\textbf{Olfatto Affinato.}} Il gigante ha +1d6 alle prove di Consapevolezza basate sull'olfatto.

\textbf{Azioni}

\emph{\textbf{Multiattacco.}} Il gigante effettua due attacchi con la Mazza chiodata.

\emph{\textbf{Mazza chiodata.} Attacco con arma da mischia}: +11 a colpire, portata 3 m, un bersaglio.

\emph{Colpisce:} 21 (3d8 + 8) danni perforanti.

\emph{\textbf{Ampio Fendente.} Attacco con arma da mischia}: +11 a colpire, portata 3 metri, con un singolo attacco può colpire due creature in mischia vicine tra loro.

\emph{\textbf{Sasso.} Attacco con arma a Distanza}: +11 a colpire, gittata 18m, un bersaglio.

\emph{Colpisce:} 30 (4d10 + 8) danni contundenti.

\textbf{Reazione: \emph{Attacco d'opportunità}}: il gigante delle nubi effettua un attacco ad una creatura che attraversi o esca dalla sua portata di 3 metri.

\emph{\textbf{Arrabbiato:}} il Gigante delle Nubi agita l'arma sopra sulla testa evocando nubi tempestose e lanciando l'incantesimo \hyperlink{Invocare il Fulmine}{Invocare il Fulmine}. Costa 2 Azioni.

\textbf{Ecologia}\\
Ambiente: Montagne Temperate\\
Organizzazione: Solitario, gruppo (2-5), famiglia (2-5 più 35\% non combattenti più 1 mago o Devoto di 4°-7° livello e 2-5 Grifoni) o tribù (6-20 più 1 oracolo mago o Devoto di 7°-12° livello e 2-5 Grifoni)\\
\textbf{Categoria Tesoro}: Giaco di Maglia, Mazza chiodata, U\\
\textbf{Descrizione}\\
Il colore pelle dei giganti delle nuvole varia dal bianco latte al blu polvere. I maschi adulti sono alti circa 5,3 metri e pesano approssimativamente 2.500 kg. Le femmine sono leggermente più basse e snelle. I giganti delle nuvole possono vivere fino a 400 anni, vestono con abiti preziosi e gioielli. Per molti l'aspetto indica lo status. Migliori sono i vestiti e più raffinati i gioielli, più importante è chi li indossa. Inoltre apprezzano la musica, e la maggioranza suona uno o più strumenti (l'arpa è uno dei preferiti).

I giganti delle nuvole possono avere Tratti insolitamente vari; circa metà sono buoni e metà malvagi. I giganti delle nuvole buoni costruiscono strade che collegano i loro insediamenti con le strade degli umani per promuovere il commercio. Non è insolito vedere un gigante delle nuvole buono camminare tra gli uomini, ad esempio, in una città umana nei pressi di un'alta catena montuosa. I giganti delle nuvole malvagi tendono a non creare insediamenti stabili e anzi preferiscono vivere in rozzi rifugi su alti picchi, da cui scendono solo per depredare i villaggi di quello di cui potrebbero aver bisogno. Queste due filosofie portano spesso allo scoppio di guerre violente e durature tra tribù vicine.

Sono molte le leggende che parlano di magiche città dei giganti delle nuvole situate tra le nuvole stesse, che fluttuano sui venti e circumnavigano il mondo. Mentre i giganti delle nuvole riconoscono che si tratta per lo più di fantasie, alcuni sostengono di averle viste e hanno dedicato la loro intera esistenza a ritrovarle.

\mostro{Gigante di Pietra}
\noindent
\begin{description}[noitemsep, topsep=0pt, parsep=0pt, partopsep=0pt, leftmargin=0cm, labelwidth=2.2cm]
	\item[\textbf{Taglia/Tipo:}] Enorme gigante, neutrale
	\item[\textbf{Caratt.:}] \resizebox{0.5\linewidth+1.8cm}{!}{For 6 Des 2 Cos 5 Int 0 Sag 1 Car -1}
	\item[\textbf{Punti Ferita:}] 148,  \textbf{Difesa:} 23,  \textbf{Iniziativa:} +2
	\item[\textbf{Movimento:}] 12 m
	\item[\textbf{Tiri Salvez.:}] \resizebox{0.5\linewidth+1.8cm}{!}{\resizebox{0.5\linewidth+1.8cm}{!}{Tempra +12, Riflessi +9, Volontà +8}}
	\item[\textbf{Comp.:}] Atletica +12
	\item[\textbf{Sensi:}] Scurovisione 18 m
	\item[\textbf{Linguaggi:}] Gigante
	\item[\textbf{Sfida:}] 7 (2900 PX)\smallskip
\end{description}

\emph{\textbf{Mimetismo di Pietra.}} Il gigante ha +1d6 alle prove di Furtività (Nascondersi) effettuate per nascondersi su terreni rocciosi.

\textbf{Azioni}

\emph{\textbf{Multiattacco.}} Il gigante effettua due attacchi con il randello pesante.

\emph{\textbf{Randello Pesante.} Attacco con arma da mischia}: +9 a colpire, portata 5 metri, un bersaglio.

\emph{Colpisce:} 19 (3d8 + 6) danni contundenti.

\emph{\textbf{Ampio Fendente.} Attacco con arma da mischia}: +9 a colpire, portata 3 metri, con un singolo attacco può colpire due creature in mischia vicine tra loro.

\emph{\textbf{Sasso.} Attacco con arma a Distanza}: +8 a colpire, gittata 18m, un bersaglio.

\emph{Colpisce:} 28 (4d10 + 6) danni contundenti. Se il bersaglio è una creatura, deve riuscire un Tiro Salvezza di Tempra DC 19 o cadere prona.

\textbf{Reazione: \emph{Attacco d'opportunità}}: il gigante di pietra effettua un attacco ad una creatura che attraversi o esca dalla sua portata di 3 metri.

\textbf{Reazione: \emph{Afferrare Sassi.}} Se un sasso o un simile oggetto viene scagliato al gigante, il gigante può, riuscendo un Tiro Salvezza su Riflessi DC 10, afferrare il proiettile e non subire danni contundenti da esso.

\emph{\textbf{Arrabbiato:}} il Gigante di Pietra concentra le sue energie rendendo la pelle dura come pietra. Fino alla fine del round successivo acquisisce una Riduzione del danno pari a 13. Costa 2 Azioni

\textbf{Ecologia}
Ambiente: Montagne temperate\\
Organizzazione: Solitario, gruppo (2-5), banda (4-8), gruppo di caccia (9-12 più 1 Anziano) o tribù (13-30 più 35\% non combattenti, 1-3 Anziani e 4-6 Orsi Crudeli)\\
\textbf{Categoria Tesoro}: Randello Pesante Gigante, P\\
\textbf{Descrizione}\\
I giganti di Pietra preferiscono spessi indumenti di cuoio, tinti con tonalità di marrone e grigio per confondersi con la pietra che li circonda. Gli adulti sono alti circa 3,6 metri, pesano circa 750 kg e possono vivere fino a 800 anni.

I giganti di Pietra, se possibile, combattono a distanza, ma se non possono evitare la mischia usano giganteschi randelli di pietra. Una delle tattiche favorite dai giganti di Pietra è di stare immobili, mimetizzandosi con il paesaggio, per poi avanzare scagliando rocce e sorprendere i nemici.

I giganti di Pietra preferiscono vivere in enormi caverne sulle cime rocciose. Raramente vivono a più di qualche giorno di viaggio da altre bande di giganti di Pietra e allevano greggi condivisi di capre e altro bestiame.

I giganti di Pietra più vecchi tendono ad allontanarsi dalla tribù per molto tempo, per vivere in solitudine da qualche parte o tentando di inserirsi in altre civiltà umanoidi. Dopo decadi di esilio auto imposto, chi fa ritorno è noto come Gigante delle Rocce Anziano.

\mostro{Gigante delle Tempeste}
\noindent
\begin{description}[noitemsep, topsep=0pt, parsep=0pt, partopsep=0pt, leftmargin=0cm, labelwidth=2.2cm]
	\item[\textbf{Taglia/Tipo:}] Enorme gigante, buono
	\item[\textbf{Caratt.:}] \resizebox{0.5\linewidth+1.8cm}{!}{For 9 Des 2 Cos 5 Int 3 Sag 4 Car 4}
	\item[\textbf{Punti Ferita:}] 262,  \textbf{Difesa:} 31,  \textbf{Iniziativa:} +3
	\item[\textbf{Movimento:}] 15 m, nuoto 15 m
	\item[\textbf{Tiri Salvez.:}] \resizebox{0.5\linewidth+1.8cm}{!}{\resizebox{0.5\linewidth+1.8cm}{!}{Tempra +18, Riflessi +15, Volontà +17}}
	\item[\textbf{Comp.:}] Arcana +8, Atletica +14, Storia +8
	\item[\textbf{Res. Danni:}] Freddo
	\item[\textbf{Imm. Danni:}] Elettricità, Suono
	\item[\textbf{Linguaggi:}] Comune, Gigante
	\item[\textbf{Sfida:}] 13 (10000 PX)\smallskip
\end{description}

\emph{\textbf{Anfibio.}} Il gigante può respirare aria e acqua.

\emph{\textbf{Incantesimi Innati.}} La caratteristica da incantatore del gigante è il Carisma. Il gigante può lanciare questi incantesimi in maniera innata, senza bisogno di componenti materiali:

A volontà: \emph{\hyperlink{Caduta Piuma}{Caduta Piuma}, individuazione del magico,} \emph{levitazione, \hyperlink{Luce}{Luce}}

3/giorno ciascuno: \emph{\hyperlink{Controllare Tempo Atmosferico}{Controllare Tempo Atmosferico}, \hyperlink{Respirare Sott'Acqua}{Respirare Sott'Acqua}}

\textbf{Azioni}

\emph{\textbf{Multiattacco.}} Il gigante effettua due attacchi con lo spadone.

\emph{\textbf{Spadone.} Attacco con arma da mischia}: +12 a colpire, portata 3 m, un bersaglio.

\emph{Colpisce:} 30 (6d6 + 9) danni taglienti.

\emph{\textbf{Ampio Fendente.} Attacco con arma da mischia}: +12 a colpire, portata 3 metri, con un singolo attacco può colpire due creature in mischia vicine tra loro.

\emph{\textbf{Sasso.} Attacco con arma a Distanza}: +11 a colpire, gittata 18m, un bersaglio.

\emph{Colpisce:} 35 (4d12 + 9) danni contundenti.

\textbf{Reazione: \emph{Attacco d'opportunità}}: il gigante delle tempeste effettua un attacco ad una creatura che attraversi o esca dalla sua portata di 3 metri.

\emph{\textbf{Colpo Fulminante (Ricarica 5-6).}} Il gigante scaglia una folgore magica ad un punto visibile entro 150 metri da sé. Ogni creatura entro 3 metri da quel punto deve effettuare un Tiro Salvezza su Riflessi DC 25, subendo 54 (12d8) danni da elettricità se lo fallisce, o la metà se lo supera.

\emph{\textbf{Arrabbiato:}} il gigante delle tempeste carica di elettricità tutta l'area intorno a se fino alla fine del combattimento. Una creatura che termini il round entro 6 metri da gigante subisce 13 (3d8) danni da elettricità. Costa 1 Azione.

\textbf{Ecologia}\\
Ambiente: Qualsiasi caldo\\
Organizzazione: Solitario o famiglia (2-5 più 1 mago o Devoto di livello 7°-10°, 1-2 Roc, 2-6 Grifoni e 2-8 Squali)\\
\textbf{Categoria Tesoro}: Corazza di Piastre Perfetta, Arco Lungo Composito Perfetto [Forza +9] con 20 Frecce, Spadone Perfetto, H\\
\textbf{Descrizione}\\
I giganti delle tempeste tendono ad avere carnagione abbronzata, anche se rari esemplari hanno pelle viola, capelli viola o blu scuri e occhi grigio argento o porpora. Il colore viola è considerato di buon auspicio tra i giganti delle tempeste, e coloro che lo posseggono tendono a diventare capi tra la loro gente. Gli adulti sono normalmente alti 6,3 metri e pesano 6000 kg. I giganti delle tempeste possono vivere fino a 600 anni.

Quando sono a riposo, preferiscono indossare tuniche corte e ampie cinte ai fianchi, sandali o piedi nudi e una fascia per capelli. Indossano pochi gioielli di semplice ma ottima fattura, i più comuni sono cavigliere (preferite dai giganti a piedi scalzi), anelli o diademi. Ma quando si equipaggiano per la battaglia, indossano corazze di piastre scintillanti e impugnano enormi spadoni e archi.

Come suggerisce il loro nome, sono inclini a violenti sbalzi di umore. I giganti delle tempeste sono facili all'ira di fronte al male e possono essere nemici brutali e pericolosi quando vengono insultati. In battaglia, preferiscono scagliare una pioggia di frecce sui loro nemici, estraendo gli spadoni solo dopo che gli avversari si sono avvicinati.

I giganti delle tempeste vivono in belle torri, castelli o in insediamenti cinti da mura e amano coltivare la terra. Possiedono enormi giardini ben curati e gestiscono centinaia di acri di coltivazioni per gruppo. Spesso impiegano altri umanoidi, come Elfi o Umani, come supporto per condurre le loro immense fattorie. Una enclave di giganti delle tempeste spesso si assume la responsabilità della sicurezza di un'intera isola o linea di costa.

\mostro{Gnoll}
\noindent
\begin{description}[noitemsep, topsep=0pt, parsep=0pt, partopsep=0pt, leftmargin=0cm, labelwidth=2.2cm]
	\item[\textbf{Taglia/Tipo:}] Media umanoide (gnoll), malvagio
	\item[\textbf{Caratt.:}] \resizebox{0.5\linewidth+1.8cm}{!}{For 2 Des 1 Cos 0 Int -2 Sag 0 Car -2}
	\item[\textbf{Punti Ferita:}] 24,  \textbf{Difesa:} 13,  \textbf{Iniziativa:} +1
	\item[\textbf{Movimento:}] 9 m
	\item[\textbf{Tiri Salvez.:}] \resizebox{0.5\linewidth+1.8cm}{!}{Tempra +3, Riflessi +3, Volontà +3}
	\item[\textbf{Sensi:}] Scurovisione 18 m
	\item[\textbf{Linguaggi:}] Gnoll
	\item[\textbf{Sfida:}] 1/2 (100 PX)\smallskip
\end{description}

\emph{\textbf{Rabbia.}} Quando lo gnoll riduce una creatura a 0 Punti Ferita con un attacco da mischia durante il proprio round, può svolgere una Reazione per muoversi fino a metà del suo movimento ed effettuare un attacco di morso.

\textbf{Azioni}

\emph{\textbf{Morso.} Attacco con arma da mischia}: +4 a colpire, portata 1 m, una creatura.

\emph{Colpisce:} 4 (1d4 + 2) danni perforanti, Malattia Rabbia Gnoll

\emph{Rabbia Gnoll:} 1 giorno, TS Tempra DC 13, 12 ore, 1 successo, -2 Saggezza

\emph{\textbf{Lancia.} Attacco con arma da mischia o a Distanza}: +4 a colpire, portata 1 m o gittata 6 m, un bersaglio.

\emph{Colpisce:} 5 (1d6 + 2) danni perforanti o 6 (1d8 + 2) danni perforanti se usata con due mani per effettuare un attacco da mischia.

\emph{\textbf{Arco Lungo.} Attacco con arma a Distanza}: +4 a colpire, gittata 45m, un bersaglio.

\emph{Colpisce:} 5 (1d8 + 1) danni perforanti.

\emph{\textbf{Risata beffarda.}} lo gnoll ride sguaiatamente ad un avversario. La creatura bersaglio deve effettuare un Tiro Salvezza su Volontà DC 13 o essere intimorito ed avere -1 al Tiro per Colpire fino alla fine del round successivo dello gnoll

\textbf{Ecologia}\\
Ambiente: Pianure calde, deserti\\
Organizzazione: Solitario, coppia, gruppo di caccia (2-5 e 1-2 Iene), banda (10-100 adulti più 50\% piccoli non combattenti, 1 sergente di 3° livello ogni 20 adulti, 1 capo di 4°-6° livello e 5-8 Iene) o tribù (20-200 più 1 sergente di 3° livello ogni 20 adulti, 1 o 2 luogotenenti di 4° o 5° livello, 1 capo di 6°-8° livello, 7-12 Iene e 4-7 ienodonti)\\
\textbf{Categoria Tesoro}: equipaggiamento da PNG (Armatura di Cuoio, Scudo Pesante di Legno, Lancia, K)\\
\textbf{Descrizione}\\
Gli gnoll sono umanoidi grandi e massicci, simili alle iene non solo nell'aspetto, ma anche nei comportamenti. Spesso tengono le iene come animali da compagnia e riflettono molti dei loro comportamenti. Pur essendo abili cacciatori, preferiscono trafugare o ripulire carcasse piuttosto che cacciare prede.

Questa pigrizia li porta a procurarsi schiavi di ogni specie per scavare tane, raccogliere provviste e acqua e persino cacciare per loro conto. Le creature non gnoll o iene diventano pasti o schiavi, a seconda del temperamento della tribù. Anche i compagni caduti possono diventare cibo, a meno che non siano onorati con una breve preghiera o cucinati interamente se morti per malattia.

Gli gnoll più civilizzati non mangiano i prigionieri, ma li tengono come schiavi per difendere o migliorare la tana o scambiarli con altre tribù. Gli gnoll apprezzano il combattimento solo quando sono in superiorità numerica. Evitano il combattimento a meno che non sia per ottenere una carcassa o imboscarsi per un lauto pasto, preferendo fuggire quando la vittoria sembra irraggiungibile.

Durante il combattimento, gli gnoll usano tattiche di branco e strategie individuali. Se sicuri di vincere, attaccano l'avversario più debole piuttosto che aiutare i compagni. Se in difficoltà, si coalizzano contro un avversario potente per costringerne alla fuga gli alleati.

I capi gnoll hanno competenze da guardiaboschi, e alcuni sono devoti a famelici Patroni. Difficilmente padroneggiano la magia in modo efficace.

\mostro{Gnomo delle Profondità}
\noindent
\begin{description}[noitemsep, topsep=0pt, parsep=0pt, partopsep=0pt, leftmargin=0cm, labelwidth=2.2cm]
	\item[\textbf{Taglia/Tipo:}] Piccola umanoide (gnomo), buono
	\item[\textbf{Caratt.:}] \resizebox{0.5\linewidth+1.8cm}{!}{For 2 Des 2 Cos 2 Int 1 Sag 0 Car -1}
	\item[\textbf{Punti Ferita:}] 24,  \textbf{Difesa:} 14,  \textbf{Iniziativa:} +2
	\item[\textbf{Movimento:}] 6 m
	\item[\textbf{Tiri Salvez.:}] \resizebox{0.5\linewidth+1.8cm}{!}{Tempra +3, Riflessi +3, Volontà +3}
	\item[\textbf{Comp.:}] Furtività +4, Consapevolezza +2
	\item[\textbf{Sensi:}] Scurovisione 36 m
	\item[\textbf{Linguaggi:}] Gnomica, Linguaggio delle Profondità, Tremun
	\item[\textbf{Sfida:}] 1/2 (100 PX)\smallskip
\end{description}

\emph{\textbf{Astuzia Gnomesca.}} Lo gnomo ha +1d6 ai Tiri Salvezza contro la magia.

\emph{\textbf{Camuffamento di Pietra.}} Lo gnomo ha +1d6 alle prove di Furtività (Nascondersi) effettuate per nascondersi su terreni rocciosi.

\emph{\textbf{Incantesimi Innati.}} La caratteristica da incantatore innato dello gnomo è l'Intelligenza. Lo gnomo può lanciare questi incantesimi in maniera innata, senza bisogno di componenti:

A volontà: \emph{\hyperlink{Anti-Individuazione}{Anti-Individuazione}} (personale)

1/giorno ciascuno: \emph{\hyperlink{Camuffare Sé Stesso}{Camuffare Sé Stesso}, cecità/sordità, \hyperlink{Sfocatura}{Sfocatura}}

\textbf{Azioni}

\emph{\textbf{Piccone da Guerra.} Attacco con arma da mischia}: +4 a colpire, portata 1 m, un bersaglio.

\emph{Colpisce:} 6 (1d8 + 2) danni perforanti.

\emph{\textbf{Dardo Avvelenato.} Attacco con arma a Distanza}: +4 a colpire, gittata 9m, un bersaglio.

\emph{Colpisce:} 4 (1d4 + 2) danni perforanti, e il bersaglio deve riuscire un Tiro Salvezza di Tempra DC 12 o restare avvelenato, -1 Forza e Destrezza, per 1 minuto. Il bersaglio può ripetere il Tiro Salvezza al termine di ciascun suo round, terminando l'effetto su di sé in caso di successo.

\textbf{Ecologia}
Ambiente: Qualsiasi sotterraneo\\
Organizzazione: Solitario, compagnia (2-4), squadra (5-20 più 1 capo 3°-6° e due sergenti di 3° livello), o banda (30-50 più 1 sergente di 3° livello ogni 20 adulti, 5 tenenti di 5° livello, 3 capitani di 7° livello, e 2-5 Elementali della Terra Medi)\\
\textbf{Categoria Tesoro}: Equipaggiamento da PNG (Piccone Pesante, Balestra Leggera con 10 Quadrelli, M)\\
\textbf{Descrizione}\\
I gnomi delle profondità, sono una branca della razza gnomesca. Dimorano nel sottosuolo, in città nascoste, al sicuro dagli elfi scuri e da altre razze sotterranee. La loro pelle è del colore della roccia, di solito grigia o marrone. I maschi sono calvi e le femmine hanno radi capelli grigi.

\mostro{Globulo}
\noindent
\begin{description}[noitemsep, topsep=0pt, parsep=0pt, partopsep=0pt, leftmargin=0cm, labelwidth=2.2cm]
	\item[\textbf{Taglia/Tipo:}] Piccola aberrazione, malvagio
	\item[\textbf{Caratt.:}] \resizebox{0.5\linewidth+1.8cm}{!}{For -2 Des 2 Cos 0 Int 3 Sag 1 Car 3}
	\item[\textbf{Punti Ferita:}] 33,  \textbf{Difesa:} 15,  \textbf{Iniziativa:} +3
	\item[\textbf{Movimento:}] volare 18 m
	\item[\textbf{Tiri Salvez.:}] \resizebox{0.5\linewidth+1.8cm}{!}{Tempra +3, Riflessi +3, Volontà +3}
	\item[\textbf{Imm. Danni:}] da Vuoto, Freddo, Veleno\\
	\item[\textbf{Immunità:}] prono
	\item[\textbf{Sensi:}] Scurovisione 36 m
	\item[\textbf{Linguaggi:}] comprende il Comune ma non lo parla
	\item[\textbf{Sfida:}] 1 (200 PX)\smallskip
\end{description}

\textbf{Odio i volatili} il Globulo ha +1d6 al Tiro per Colpire contro gli uccelli. Attacca prima gli uccelli e creature volanti

\textbf{Natura inusuale} il Globulo non respira

\textbf{Odio l'acqua} il Globulo detesta bagnarsi e ogni 5 litri di acqua spruzzata su lui subisce 1d4 di danno

\textbf{Azioni}

\emph{\textbf{Tentacolo}}. Attacco in mischia, +5 al colpire, portata 3 metri, un obiettivo

\emph{\textbf{Colpisce}} 5 (1d6+2) di danno da Vuoto. Il bersaglio deve fare un Tiro Salvezza su Tempra a DC 11 o aumentare il grado di Affaticamento di 1.

\textbf{\emph{Brillio}} una volta al giorno il Globulo diventa estremamente luminoso, le creature nel raggio di 6 metri attorno a lui devono fare un Tiro Salvezza su Tempra a DC 13 o diventare accecati per 3 round.

\textbf{Ecologia}
Ambiente: Qualsiasi, desertico, notturno\\
Organizzazione: Solitario, gruppi 2d4\\
\textbf{Categoria Tesoro}: Nessuno\\
\textbf{Descrizione}\\
I Globuli sono aberrazioni magiche provenienti da qualche portale aperto verso l'Oltre. Creature di freddo e vuoto sembrano delle piccole stelle che anelano solo di risucchiare la vita della creature incontrate.
Intelligenti e furbe preferiscono attaccare rimanendo in volo e fiaccando l'avversario finché questo è mortalmente affaticato. Una volta ucciso di un Globulo non rimane che una piccola creatura a forma di stella con un grosso occhio centrale, completamente bianco.

\mostro{Goblin}
\noindent
\begin{description}[noitemsep, topsep=0pt, parsep=0pt, partopsep=0pt, leftmargin=0cm, labelwidth=2.2cm]
	\item[\textbf{Taglia/Tipo:}] Piccola umanoide (goblinoide), malvagio
	\item[\textbf{Caratt.:}] \resizebox{0.5\linewidth+1.8cm}{!}{For 0 Des 0 Cos 1 Int -1 Sag -2 Car -1}
	\item[\textbf{Punti Ferita:}] 19,  \textbf{Difesa:} 12,  \textbf{Iniziativa:} +0
	\item[\textbf{Movimento:}] 9 m
	\item[\textbf{Tiri Salvez.:}] \resizebox{0.5\linewidth+1.8cm}{!}{Tempra +3, Riflessi +3, Volontà +3}
	\item[\textbf{Sensi:}] Scurovisione 18 m
	\item[\textbf{Linguaggi:}] Comune, Goblin
	\item[\textbf{Sfida:}] 1/4 (50 PX)\smallskip
\end{description}

\textbf{Azioni}

\emph{\textbf{Spada Corta.} Attacco con arma da mischia}: +4 a colpire, portata 1 m, un bersaglio.

\emph{Colpisce:} 4 (1d6 + 1) danni taglienti

\emph{\textbf{Arco Corto.} Attacco con arma a Distanza}: +3 a colpire, gittata 15m, un bersaglio.

\emph{Colpisce:} 3 (1d6) danni perforanti.

\textbf{Ecologia}\\
Ambiente: Qualsiasi Temperate\\
Organizzazione: Gruppo (4-9), banda da guerra (10-24) o tribù (50+ più 50\% non combattenti\\
\textbf{Categoria Tesoro}: K\\
\textbf{Descrizione}\\
I goblin sono selvaggi, imprevedibili, rumorosi.
I goblin preferiscono vivere nelle caverne, nel fitto delle foreste e quando ne hanno a disposizione nelle strutture antiche abbandonate. I goblin non amano costruire quanto piuttosto distruggere per poi lamentarsi che non c'è nulla di utile.

I goblin sono molto superstiziosi, e vedono la magia con un misto di timore reverenziale e paura. Ogni cosa che non comprendono è per loro magia e questo li porta a essere estremamente sospettosi di tutto e a distruggere tutto, visto che ciò che non capiscono va distrutto.

I goblin sono famelici e possono mangiare enormi quantità di cibo. un goblin non rinuncia a mangiare nulla tranne forse l'insalata..

\mostro{Golem di Argilla}
\noindent
\begin{description}[noitemsep, topsep=0pt, parsep=0pt, partopsep=0pt, leftmargin=0cm, labelwidth=2.2cm]
	\item[\textbf{Taglia/Tipo:}] Grande costrutto, disallineato
	\item[\textbf{Caratt.:}] \resizebox{0.5\linewidth+1.8cm}{!}{For 5 Des -1 Cos 4 Int -4 Sag -1 Car -5}
	\item[\textbf{Punti Ferita:}] 184,  \textbf{Difesa:} 23,  \textbf{Iniziativa:} -1
	\item[\textbf{Movimento:}] 6 m
	\item[\textbf{Tiri Salvez.:}] \resizebox{0.5\linewidth+1.8cm}{!}{\resizebox{0.5\linewidth+1.8cm}{!}{Tempra +13, Riflessi +8, Volontà +8}}
	\item[\textbf{Imm. Danni:}] Acido, Veleno
	\item[\textbf{Immunità:}] affascinato, paralizzato, pietrificato, affaticato, spaventato
	\item[\textbf{Sensi:}] Scurovisione 18 m
	\item[\textbf{Linguaggi:}] comprende le lingue del suo creatore ma non può parlare
	\item[\textbf{Sfida:}] 9 (5000 PX)\smallskip
\end{description}

\emph{\textbf{Riduzione del Danno.}} Il golem d'argilla ha durezza 8/- contro armi non magiche.

\emph{\textbf{Berserk.}} Ogni volta che il golem inizia il suo round con 60 Punti Ferita o meno, tira un d6. Se ottieni 6, il golem va in berserk. Durante ogni suo round mentre è in berserk, guadagna una Azione per quel round. Il golem attacca la creatura più vicina che può vedere. Se non c'è nessuna creatura abbastanza vicina da muoversi e attaccarla, il golem attacca un oggetto, con preferenza per gli oggetti più piccoli di lui. Una volta che il golem è andato in berserk, continuerà ad esserlo finché non viene distrutto o recupera tutti i suoi Punti Ferita.

\emph{\textbf{Armi Magiche.}} Gli attacchi con armi del golem sono magici.

\emph{\textbf{Assorbimento dell'Acido.}} Ogni volta che il golem è vittima di danni da acido, non subisce danni ma invece recupera un pari numero di Punti Ferita.

\emph{\textbf{Forma Immutabile.}} Il golem è immune a qualsiasi incantesimo o effetto che altererebbe la sua forma.

\emph{\textbf{Natura di Costrutto.}} Un golem non ha bisogno di aria, cibo, bevande o sonno.

\emph{\textbf{Resistenza alla Magia.}} Il golem ha +1d6 ai Tiri Salvezza contro incantesimi e altri effetti magici.

\textbf{Azioni}

\emph{\textbf{Multiattacco.}} Il golem effettua due attacchi di schianto oppure un solo attacco di pugno maledetto

\emph{\textbf{Schianto.} Attacco con arma da mischia}: +10 a colpire, portata 1 m, un bersaglio.

\emph{Colpisce:} 16 (2d10 + 5) danni contundenti.

\emph{\textbf{Pugno Maledetto.}: Attacco con arma naturale}: + 11 a colpire, portata 1 m, un bersaglio

\emph{Colpisce:} 16 (2d6 + 5) danni contundenti. Le ferite da pugno maledetto guariscono al ritmo di 1 Punto ferita a giorno. Le cure magiche, incantesimi o pozioni, curano 1 Punto Ferita per dado di cura + tutto l'eventuale fisso (es. una cura di 3d6+4 cura 7 PF)

\emph{\textbf{Velocità (Ricarica 5-6).}} Fino al termine del suo prossimo round, il golem ottiene un bonus magico di +2 alla Difesa, ha +1d6 ai Tiri Salvezza su Riflessi, e può usare gli attacchi di schianto come Azione Immediata.

\textbf{Ecologia}\\
Ambiente: Qualsiasi\\
Organizzazione: Solitario o gruppo (2-4)\\
\textbf{Categoria Tesoro}: Nessuno\\
\textbf{Descrizione}\\
I golem di argilla non indossano abiti, eccezion fatta per un indumento di cuoio trattato o metallo attorno ai fianchi. Mediamente sono alti più di 2,3 metri e pesano 300 chili.

\textbf{Costruzione}
Un golem d'argilla può essere scolpito a partire da un unico blocco d'argilla del peso minimo di 500 chili, trattato con polveri e oli rari per il valore di 1,500 mo.

\mostro{Golem di Carne}
\noindent
\begin{description}[noitemsep, topsep=0pt, parsep=0pt, partopsep=0pt, leftmargin=0cm, labelwidth=2.2cm]
	\item[\textbf{Taglia/Tipo:}] Media costrutto, neutrale
	\item[\textbf{Caratt.:}] \resizebox{0.5\linewidth+1.8cm}{!}{For 4 Des -1 Cos 4 Int -2 Sag 0 Car -3}
	\item[\textbf{Punti Ferita:}] 109,  \textbf{Difesa:} 17,  \textbf{Iniziativa:} -1
	\item[\textbf{Movimento:}] 9 m
	\item[\textbf{Tiri Salvez.:}] \resizebox{0.5\linewidth+1.8cm}{!}{Tempra +9, Riflessi +4, Volontà +5}
	\item[\textbf{Imm. Danni:}] Elettricità, Veleno
	\item[\textbf{Immunità:}] affascinato, paralizzato, pietrificato, affaticato, spaventato
	\item[\textbf{Sensi:}] Scurovisione 18 m
	\item[\textbf{Linguaggi:}] comprende le lingue del suo creatore ma non può
	\item[\textbf{Sfida:}] 5 (1800 PX)\smallskip
\end{description}

\emph{\textbf{Riduzione del Danno.}} Il golem d'argilla ha durezza 6/- contro armi non magiche.

\emph{\textbf{Berserk.}} Ogni volta che il golem inizia il suo round con 40 Punti Ferita o meno, tira un d6. Se ottieni 6, il golem va in berserk. Durante ogni suo round mentre è in berserk guadagna una Azione, il golem attacca la creatura più vicina che possa vedere. Se non c'è nessuna creatura abbastanza vicina da muoversi e attaccarla, il golem attacca un oggetto, con preferenza per gli oggetti più piccoli di lui. Una volta che il golem è andato in berserk, continuerà ad esserlo finché non viene distrutto o recupera tutti i suoi Punti Ferita.

\emph{\textbf{Armi Magiche.}} Gli attacchi con armi del golem sono magici.

\emph{\textbf{Assorbimento dei Fulmini.}} Ogni volta che il golem sia vittima di un danno da elettricità, non subisce danni ma invece recupera un pari numero di Punti Ferita.

\emph{\textbf{Avversione al Fuoco.}} Se il golem subisce danni da fuoco, ha -1d6 ai tiri di attacco e le prove di competenza di Base fino alla fine del suo prossimo round.

\emph{\textbf{Forma Immutabile.}} Il golem è immune a qualsiasi incantesimo o effetto che altererebbe la sua forma.

\emph{\textbf{Natura di Costrutto.}} Un golem non ha bisogno di aria, cibo, bevande o sonno.

\emph{\textbf{Resistenza alla Magia.}} Il golem ha +1d6 ai Tiri Salvezza contro incantesimi e altri effetti magici.

\textbf{Azioni}

\emph{\textbf{Multiattacco.}} Il golem effettua due attacchi di schianto.

\emph{\textbf{Schianto.} Attacco con arma da mischia}: +6 a colpire, portata 1 m, un bersaglio.

\emph{Colpisce:} 13 (2d8 + 4) danni contundenti. La creatura colpita deve effettuare un Tiro Salvezza su Tempra a DC 17 o ammalarsi. Ogni volta che fallisce il Tiro Salvezza esegue una Azione in meno il round successivo. Se arriva a perdere 3 Azioni, ovvero fallisce per 3 volte il Tiro Salvezza di fila, la creatura muore. Appena il Tiro Salvezza riesce si debella la malattia.

\emph{\textbf{Arrabbiato:}} il golem di carne si sovraccarica. Per 2d4 round può eseguire una Azione di in più di Movimento o di Attacco. Costa 1 Azione.

\textbf{Ecologia}\\
Ambiente: Qualsiasi\\
Organizzazione: Solitario o gruppo (2-4)\\
\textbf{Categoria Tesoro}: Nessuno\\
\textbf{Descrizione}\\
Un golem di carne è una mostruosa collezione di parti anatomiche umanoidi trafugate e cucite insieme. La sua carne cadaverica ha tonalità verde pallido o giallognola. Un golem di carne indossa qualsiasi tipo di vestito che il suo creatore desideri, normalmente solo un logoro paio di pantaloni. Non ha Equipaggiamento né armi. Un golem di carne è alto più di 2,3 metri e pesa 250 kg.

Un golem di carne non parla, anche se può emettere una specie di ringhio rauco. Cammina e si muove con un'andatura a scatti, come se non avesse il pieno controllo del proprio corpo.

Anche se molti golem di carne sono privi di ragione, si narra di golem eccezionali che in qualche modo hanno mantenuto i ricordi della vita precedente. La testa (e quindi il cervello) di questi golem di carne deve essere la giusta combinazione di freschezza e (nella vita precedente) decisione, ma di assoluta importanza sembrano essere anche la fortuna e il caso affinché durante la loro creazione si conservi l'intelletto. Certamente quelli che costruiscono golem di carne preferiscono avere schiavi privi di intelletto piuttosto che dotati di una propria volontà, di conseguenza i golem di carne intelligenti sono rari.

\mostro{Golem di Ferro}
\noindent
\begin{description}[noitemsep, topsep=0pt, parsep=0pt, partopsep=0pt, leftmargin=0cm, labelwidth=2.2cm]
	\item[\textbf{Taglia/Tipo:}] Grande costrutto, disallineato
	\item[\textbf{Caratt.:}] \resizebox{0.5\linewidth+1.8cm}{!}{For 7 Des -1 Cos 5 Int -4 Sag 0 Car -5}
	\item[\textbf{Punti Ferita:}] 319,  \textbf{Difesa:} 32,  \textbf{Iniziativa:} -1
	\item[\textbf{Movimento:}] 9 m
	\item[\textbf{Tiri Salvez.:}] \resizebox{0.5\linewidth+1.8cm}{!}{\resizebox{0.5\linewidth+1.8cm}{!}{Tempra +21, Riflessi +15, Volontà +16}}
	\item[\textbf{Imm. Danni:}] Fuoco, Veleno
	\item[\textbf{Immunità:}] affascinato, paralizzato, pietrificato, affaticato, spaventato
	\item[\textbf{Sensi:}] Scurovisione 36 m
	\item[\textbf{Linguaggi:}] comprende le lingue del suo creatore ma non può parlare
	\item[\textbf{Sfida:}] 16 (15000 PX)\smallskip
\end{description}

\emph{\textbf{Riduzione del Danno.}} Il golem d'argilla ha durezza 12/- contro armi non magiche.

\emph{\textbf{Armi Magiche.}} Gli attacchi con armi del golem sono magici.

\emph{\textbf{Assorbimento del Fuoco.}} Ogni volta che il golem sia vittima di un danno da fuoco, non subisce danni ma invece recupera un pari numero di Punti Ferita.

\emph{\textbf{Forma Immutabile.}} Il golem è immune a qualsiasi incantesimo o effetto che altererebbe la sua forma.

\emph{\textbf{Natura di Costrutto.}} Un golem non ha bisogno di aria, cibo, bevande o sonno.

\emph{\textbf{Resistenza alla Magia.}} Il golem ha +1d6 ai Tiri Salvezza contro incantesimi e altri effetti magici.

\textbf{Azioni}

\emph{\textbf{Multiattacco.}} Il golem effettua due attacchi da mischia.

\emph{\textbf{Schianto.} Attacco con arma da mischia}: +14 a colpire, portata 1 m, un bersaglio.

\emph{Colpisce:} 20 (3d8 + 7) danni contundenti.

\emph{\textbf{Spada.} Attacco con arma da mischia}: +14 a colpire, portata 3 m, un bersaglio.

\emph{Colpisce:} 23 (3d10 + 7) danni taglienti.

\textbf{Reazione: \emph{Attacco d'opportunità}}: il golem effettua un attacco ad una creatura che attraversi o esca dalla sua portata di 1 metro.

\emph{\textbf{Soffio Velenoso (Ricarica 6).}} Il golem esala un gas velenoso in un cono di 5 metri. Ogni creatura in quell'area deve effettuare un Tiro Salvezza di Tempra DC 29, subendo 45 (10d8) danni da veleno se fallisce il Tiro Salvezza, o la metà di questi danni se lo riesce.

\emph{\textbf{Arrabbiato:}} il golem di ferro esala un soffio rovente in un cono di 3 metri. Il soffio causa 3d10 danni da fuoco o la metà se il Tiro Salvezza si Riflessi DC 26 riesce. Il golem recupera l'intero ammontare in Punti Ferita ed è Accelerato 1 per 2d4 round. Costa 2 Azioni.

\textbf{Ecologia}\\
Ambiente: Qualsiasi\\
Organizzazione: Solitario o gruppo (2-4)\\
\textbf{Categoria Tesoro}: Nessuno\\
\textbf{Descrizione}\\
Un golem di ferro ha un corpo di forma umanoide in ferro. Il creatore può dargli qualsiasi forma desideri, ma presenta quasi sempre un'armatura di qualche tipo, sia essa cerimoniale e preziosa o semplice e d'uso. Rispetto ad un golem di pietra ha sembianze molto più definite. I golem di ferro, talvolta, portano con sé un'arma, anche se il più delle volte tendono a preferirle i loro attacchi schianto.

Un golem di ferro è alto 3,6m e pesa circa 2.500 chili. Un golem di ferro non può parlare né emettere voce. Inoltre, non emette nessun odore riconoscibile.

Anche se la pratica della costruzione di golem di ferro è gradualmente caduta in disuso, i membri venerabili di alcune grandi civiltà del passato consideravano la capacità di forgiare golem di ferro dalla forza e dalle dimensioni sconcertanti un motivo di vanto. Questi golem (di taglia maggiore o uguale a Enorme), in alcuni angoli remoti del mondo, esistono ancora, e ancora eseguono meccanicamente ordini impartiti loro da imperi ormai scomparsi.

\textbf{Costruzione}
Per costruire un golem di ferro occorrono 2.500 kg di ferro, fuso con tinture rare del valore minimo di 10000 mo.

\mostro{Golem di Pietra}
\noindent
\begin{description}[noitemsep, topsep=0pt, parsep=0pt, partopsep=0pt, leftmargin=0cm, labelwidth=2.2cm]
	\item[\textbf{Taglia/Tipo:}] Grande costrutto, disallineato
	\item[\textbf{Caratt.:}] \resizebox{0.5\linewidth+1.8cm}{!}{For 6 Des -1 Cos 5 Int -4 Sag 0 Car -5}
	\item[\textbf{Punti Ferita:}] 205,  \textbf{Difesa:} 24,  \textbf{Iniziativa:} -1
	\item[\textbf{Movimento:}] 9 m
	\item[\textbf{Tiri Salvez.:}] \resizebox{0.5\linewidth+1.8cm}{!}{\resizebox{0.5\linewidth+1.8cm}{!}{Tempra +15, Riflessi +9, Volontà +10}}
	\item[\textbf{Imm. Danni:}] Veleno
	\item[\textbf{Immunità:}] affascinato, paralizzato, pietrificato, affaticato, spaventato
	\item[\textbf{Sensi:}] Scurovisione 36 m
	\item[\textbf{Linguaggi:}] comprende le lingue del suo creatore ma non può parlare
	\item[\textbf{Sfida:}] 10 (5900 PX)\smallskip
\end{description}

\emph{\textbf{Riduzione del Danno.}} Il golem d'argilla ha durezza 10/- contro armi non magiche.

\emph{\textbf{Armi Magiche.}} Gli attacchi con armi del golem sono magici.

\emph{\textbf{Forma Immutabile.}} Il golem è immune a qualsiasi incantesimo o effetto che altererebbe la sua forma.

\emph{\textbf{Natura di Costrutto.}} Un golem non ha bisogno di aria, cibo, bevande o sonno.

\emph{\textbf{Resistenza alla Magia.}} Il golem ha +1d6 ai Tiri Salvezza contro incantesimi e altri effetti magici.

\textbf{Azioni}

\emph{\textbf{Multiattacco.}} Il golem effettua due attacchi di schianto.

\emph{\textbf{Schianto.} Attacco con arma da mischia}: +11 a colpire, portata 1 m, un bersaglio.

\emph{Colpisce:} 19 (3d8 + 6) danni contundenti.

\textbf{Reazione: \emph{Sasso affilato}}: il golem reagisce ad un attacco subito guadagnando 1 danno bonus al suo attacco di schianto.

\emph{\textbf{Lentezza (Ricarica 5-6).}} Il golem prende a bersaglio una o più creature entro 3 metri da lui e che possa vedere. Ciascun bersaglio deve effettuare un Tiro Salvezza di Volontà DC 24 contro questa magia. Se fallisce il Tiro Salvezza il bersaglio è Rallentato 2/1 minuto. Il bersaglio può ripetere il Tiro Salvezza al termine di ciascun suo round, terminando l'effetto per sé in caso di successo.

\textbf{Ecologia}\\
Ambiente: Qualsiasi\\
Organizzazione: Solitario o gruppo (2-4)\\
\textbf{Categoria Tesoro}: Nessuno\\
\textbf{Descrizione}\\
Un golem di pietra ha un corpo umanoide fatto di pietra, spesso stilizzato per soddisfare il suo creatore. Ad esempio, può essere scolpito in modo da indossare un'armatura, con particolari simboli scolpiti sulla corazza, o avere dei disegni intarsiati nella pietra dei suoi arti. La testa è spesso scolpita per sembrare un elmo o la testa di qualche bestia. Sebbene possa essere scolpito con uno scudo o un'arma di pietra come una spada, queste scelte estetiche non influenzano le sue capacità in combattimento.

Come per la maggior parte dei golem, un golem di pietra non può parlare e non emette altro suono se non quello della pietra che sfrega sulla pietra quando si muove. Un golem di pietra è alto 2,7 metri e pesa circa 1000 kg.

Esistono numerose varianti dei Golem di Pietra, a seconda del materiali di cui sono fatti ma anche come espressioni di spiriti elementali, ovvero é possibile che uno spirito elementale abiti una roccia (o gemma) e ne definisca l'aspetto e lo animi come proprio corpo.

\textbf{Costruzione}
Il corpo di un golem di pietra viene scolpito da un unico blocco di pietra dura, come il granito, del peso di almeno 1.500 kg. La pietra deve essere di qualità eccezionale, e costare 5000 mo.

\mostro{Gorgone}
\noindent
\begin{description}[noitemsep, topsep=0pt, parsep=0pt, partopsep=0pt, leftmargin=0cm, labelwidth=2.2cm]
	\item[\textbf{Taglia/Tipo:}] Grande mostruosità, disallineato
	\item[\textbf{Caratt.:}] \resizebox{0.5\linewidth+1.8cm}{!}{For 5 Des 0 Cos 4 Int -4 Sag 1 Car -2}
	\item[\textbf{Punti Ferita:}] 109,  \textbf{Difesa:} 18,  \textbf{Iniziativa:} +0
	\item[\textbf{Movimento:}] 12 m
	\item[\textbf{Tiri Salvez.:}] \resizebox{0.5\linewidth+1.8cm}{!}{Tempra +9, Riflessi +5, Volontà +6}
	\item[\textbf{Comp.:}] Consapevolezza +4
	\item[\textbf{Immunità:}] Pietrificato
	\item[\textbf{Sensi:}] Scurovisione 18 m
	\item[\textbf{Sfida:}] 5 (1800 PX)\smallskip
\end{description}

\emph{\textbf{Carica Travolgente.}} La gorgone carica un bersaglio. 2 Azioni. Se il bersaglio, entro 18 metri, viene colpito da Incornata, deve anche riuscire un Tiro Salvezza su Tempra DC 18 o cadere prono. Se il bersaglio cade prono la gorgone può effettuare un attacco di zoccoli contro di lui come Azione Immediata.

\textbf{Azioni}

\emph{\textbf{Incornata.} Attacco con arma da mischia}: +6 a colpire, portata 1 m, un bersaglio.

\emph{Colpisce:} 18 (2d12 + 5) danni perforanti.

\emph{\textbf{Zoccoli.} Attacco con arma da mischia}: +6 a colpire, portata 1 m, un bersaglio.

\emph{Colpisce:} 16 (2d10 + 5) danni contundenti.

\emph{\textbf{Soffio Pietrificante (Ricarica 4-6).}} La gorgone esala un gas pietrificante in un cono di 9 metri. Ogni creatura in quell'area deve riuscire un Tiro Salvezza di Tempra DC 16. Se il Tiro Salvezza fallisce la creatura è Rallentata 1/1 minuto. Se successivi soffi portano il bersaglio a non avere più Azioni allora diviene pietrificato finché non viene liberato dall'incantesimo \hyperlink{Pietra in Carne}{Pietra in Carne}.

\emph{\textbf{Arrabbiato:}} la Gorgone concentra un potente soffio pietrificante. Costa 2 azioni. Una creatura a distanza di mischia deve effettuare un Tiro Salvezza su Tempra a DC 16 o diventare di pietra per 24 ore.

\textbf{Ecologia}\\
Ambiente: Pianure Temperate, Colline Rocciose e Sotterranei\\
Organizzazione: Solitario, coppia, branco (3-4) o mandria (5-12)\\
\textbf{Categoria Tesoro}: Nessuno\\
\textbf{Descrizione}\\
Le gorgoni sono creature magiche e irascibili: sebbene a prima vista possano sembrare dei costrutti, sotto le piastre metalliche dall'aspetto artificiale sono fatte di carne e ossa. Come tori aggressivi, sfidano qualsiasi creatura sconosciuta che incontrano, spesso travolgendo il cadavere del loro avversario o frantumando i suoi resti pietrificati finché la creatura non è più riconoscibile. Le femmine sono pericolose quanto i maschi, e i due sessi hanno l'identico aspetto. Una tipica gorgone è alta 1,8 metri e lunga 2,3 metri. Pesa circa 2000 kg.

Le gorgoni ricavano il loro nutrimento consumando minerali, in particolare la pietra delle loro vittime pietrificate, e ogni statua da loro creata viene completamente divorata. Non possono digerire metallo o gemme, così il loro sterco (che assomiglia a polvere grigia dall'odore acre) spesso contiene piccoli cristalli grezzi e pepite d'oro. La loro aggressività verso tutte le altre creature fa sì che nei loro pascoli siano pochi, se non nessuno, i predatori e le prede. Ogni mandria è guidata da un toro dominante; le gorgoni solitarie sono generalmente tori adolescenti allontanati dalla mandria del toro dominante.

La loro carne è dura e muscolosa (una volta che viene rimossa l'armatura), e per coloro che la assaggiano è abbastanza nutriente. Molte tribù di giganti della pietra credono che mangiare la carne di gorgone aumenti la loro armatura naturale. Le corna di gorgone polverizzate valgono 250 mo come componente materiale alternativo per gli oggetti magici ed incantesimi che agiscono sulla Forza o Pietra.

\mostro{Grick}
\noindent
\begin{description}[noitemsep, topsep=0pt, parsep=0pt, partopsep=0pt, leftmargin=0cm, labelwidth=2.2cm]
	\item[\textbf{Taglia/Tipo:}] Media mostruosità, neutrale
	\item[\textbf{Caratt.:}] \resizebox{0.5\linewidth+1.8cm}{!}{For 2 Des 2 Cos 0 Int -4 Sag 2 Car -3}
	\item[\textbf{Punti Ferita:}] 51,  \textbf{Difesa:} 16,  \textbf{Iniziativa:} +2
	\item[\textbf{Movimento:}] 9 m, scalata 9 m
	\item[\textbf{Tiri Salvez.:}] \resizebox{0.5\linewidth+1.8cm}{!}{Tempra +3, Riflessi +4, Volontà +4}
	\item[\textbf{Sensi:}] Scurovisione 18 m
	\item[\textbf{Sfida:}] 2 (450 PX)\smallskip
\end{description}

\emph{\textbf{Camuffamento di Pietra.}} Il grick ha +1d6 alle prove di Furtività (Nascondersi) effettuate per nascondersi su terreno roccioso.

\textbf{Azioni}

\emph{\textbf{Multiattacco.}} Il grick effettua un attacco con i suoi tentacoli. Se l'attacco colpisce, il grick può effettuare un attacco di becco contro lo stesso bersaglio.

\emph{\textbf{Tentacoli.} Attacco con arma da mischia}: +4 a colpire, portata 1 m, un bersaglio.

\emph{Colpisce:} 9 (2d6 + 2) danni taglienti.

\emph{\textbf{Becco.} Attacco con arma da mischia}: +4 a colpire, portata 1 m, un bersaglio.

\emph{Colpisce:} 5 (1d6 + 2) danni perforanti.

\textbf{Ecologia}: \\
Ambiente: Qualsiasi Sotterraneo\\
Organizzazione: Solitario o ammasso (2-5)\\
\textbf{Categoria Tesoro}: Accidentale\\
\textbf{Descrizione}\\
Il grick, una creatura vermiforme, è il terrore delle caverne e dei cunicoli in cui risiede. In agguato nei pressi di tunnel trafficati o città sotterranee, balza fuori dal buio per catturare le sue prede. Solitamente non consuma le prede sul posto, ma le porta nella sua tana, un cunicolo stretto o una sporgenza di una caverna, dove può mangiarle tranquillamente.

Le origini del grick sono ignote. Sebbene possieda una rudimentale intelligenza, non ha una vera e propria società e si incontra solitamente da solo. Nei rari casi in cui sono presenti più grick, non sembrano collaborare tra loro: ognuno attacca un obiettivo individuale e si ritira con il bottino una volta abbattuto l'avversario.

I grick sono predatori capaci con una pelle resistente alle armi, rendendoli particolarmente pericolosi. Molti avventurieri inesperti sono periti sotto il loro attacco poiché non riuscivano a danneggiarli con armi non magiche. Chi conosce i grick (soprattutto Nani, Morlock e Trogloditi) sa che la migliore strategia è ritirarsi e attendere rinforzi più potenti o magici.

I grick si mimetizzano grazie al loro colore scuro e alla capacità di scalare i muri, rimanendo nascosti fino al momento di attaccare. Quando il cibo scarseggia, possono dirigersi verso la superficie in cerca di prede, ma preferiscono le tenebre e la sicurezza di un tetto sopra la testa, evitando il cielo aperto e cercando rifugio sotto alberi, nuvole basse o edifici.

\mostro{Grifone}
\noindent
\begin{description}[noitemsep, topsep=0pt, parsep=0pt, partopsep=0pt, leftmargin=0cm, labelwidth=2.2cm]
	\item[\textbf{Taglia/Tipo:}] Grande mostruosità, disallineato
	\item[\textbf{Caratt.:}] \resizebox{0.5\linewidth+1.8cm}{!}{For 4 Des 2 Cos 3 Int -3 Sag 1 Car 0}
	\item[\textbf{Punti Ferita:}] 52,  \textbf{Difesa:} 16,  \textbf{Iniziativa:} +2
	\item[\textbf{Movimento:}] 9 m, volo 24 m
	\item[\textbf{Tiri Salvez.:}] \resizebox{0.5\linewidth+1.8cm}{!}{Tempra +5, Riflessi +4, Volontà +3}
	\item[\textbf{Comp.:}] Consapevolezza +5
	\item[\textbf{Sensi:}] Scurovisione 18 m
	\item[\textbf{Sfida:}] 2 (450 PX)\smallskip
\end{description}

\emph{\textbf{Vista Affinata.}} Il grifone ha +1d6 nelle prove di Consapevolezza basate sulla vista.

\textbf{Azioni}

\emph{\textbf{Multiattacco.}} Il grifone effettua due attacchi: uno con il becco e uno con gli artigli.

\emph{\textbf{Artigli.} Attacco con arma da mischia}: +6 a colpire, portata 1 m, un bersaglio.

\emph{Colpisce:} 11 (2d6 + 4) danni taglienti, 1 danno da Sanguinamento.

\emph{\textbf{Becco.} Attacco con arma da mischia}: +5 a colpire, portata 1 m, un bersaglio.

\emph{Colpisce:} 8 (1d8 + 4) danni perforanti.

\textbf{Reazione: \emph{Attacco d'opportunità}}: il grifone attacca se sta volando ed una creatura esce o attraversa la sua portata di 3 m.

\textbf{Ecologia}\\
Ambiente: Colline Temperate\\
Organizzazione: Solitario, coppia o branco (6-10)\\
\textbf{Categoria Tesoro}: Accidentale\\
\textbf{Descrizione}\\
I grifoni sono potenti predatori aerei che piombano dai loro nidi per afferrare le prede con becco e artigli. Aggressivi e territoriali, sono combattenti astuti e leali verso chi guadagna il loro rispetto, proteggendoli fino alla morte. Pesano oltre 250 kg e sono lunghi 2,3 metri, con un imponente profilo spesso usato in araldica come simbolo di potenza, autorità e giustizia. Nonostante ciò, sono più interessati a cacciare e difendersi.

Sebbene possano essere addestrati come cavalcature, i grifoni non hanno un'innata affinità con gli umanoidi e spesso entrano in conflitto con razze civilizzate per procurarsi carne di saurovallo. I cittadini possono meravigliarsi alla vista di un grifone addestrato con un'apertura alare di 7 metri, ma i contadini sono ben consapevoli del pericolo che rappresentano.

I grifoni si accoppiano per la vita e cercano vendetta per anni se un compagno o un figlio viene ucciso. Questa lealtà li rende cavalcature e guardiani di tesori ideali, nonostante il pericolo insito nel commercio di grifoni catturati e uova rubate. Le uova valgono fino a 2000 mo l'una e i giovani vivi fino a 3000. Tuttavia, comprare o addomesticare con la violenza queste creature è considerato schiavitù dalle divinità buone. Guadagnarsi la loro lealtà spontanea è un compito difficile ma più sicuro.

Prima di poter cavalcare un grifone in combattimento, la creatura deve fare pratica nel portare il peso del suo cavaliere. Un grifone deve avere un atteggiamento amichevole verso l'addestratore (con una prova di Gestire Animali, Diplomazia o Intimidire), e 6 settimane di pratica con una prova riuscita di Gestire Animali con DC 20 sono necessarie per abituarlo al carico. I grifoni addestrati possono conoscere trucchi e imparare nuovi comandi.

I grifoni possono portare fino a 25 di Ingombro come carico leggero, 50 come carico medio e 70 come carico pesante. È necessaria una sella esotica per cavalcarli.

\mostro{Grimlock}
\noindent
\begin{description}[noitemsep, topsep=0pt, parsep=0pt, partopsep=0pt, leftmargin=0cm, labelwidth=2.2cm]
	\item[\textbf{Taglia/Tipo:}] Media umanoide (grimlock), malvagio
	\item[\textbf{Caratt.:}] \resizebox{0.5\linewidth+1.8cm}{!}{For 3 Des 1 Cos 1 Int -1 Sag -1 Car -2}
	\item[\textbf{Punti Ferita:}] 19,  \textbf{Difesa:} 13,  \textbf{Iniziativa:} +1
	\item[\textbf{Movimento:}] 9 m
	\item[\textbf{Tiri Salvez.:}] \resizebox{0.5\linewidth+1.8cm}{!}{Tempra +3, Riflessi +3, Volontà +3}
	\item[\textbf{Comp.:}] Atletica +5, Furtività +3, Consapevolezza +3
	\item[\textbf{Immunità:}] accecato
	\item[\textbf{Sensi:}] Vista Cieca 9 m o 3 m se assordato (cieco oltre questo raggio)
	\item[\textbf{Linguaggi:}] Linguaggio delle Profondità
	\item[\textbf{Sfida:}] 1/4 (50 PX)\smallskip
\end{description}

\emph{\textbf{Camuffamento di Pietra.}} Il grimlock ha +1d6 alle prove di Furtività (Nascondersi) effettuate per nascondere su terreni rocciosi.

\emph{\textbf{Sensi Ciechi.}} Il grimlock non può usare la vista cieca mentre è assordato e non più fiutare.

\emph{\textbf{Olfatto e Udito Affinati.}} Il grimlock ha +1d6 alle prove di Consapevolezza basate su udito o olfatto.

\textbf{Azioni}

\emph{\textbf{Randello d'Osso Appuntito.} Attacco con arma da mischia}: +5 a colpire, portata 1 m, un bersaglio.

\emph{Colpisce:} 5 (1d4 + 3) danni contundenti più 2 (1d4) danni perforanti.

\emph{\textbf{Arco Lungo.} Attacco con arma a Distanza}: +3 a colpire, gittata 45m, un bersaglio.

\emph{Colpisce:} 5 (1d8 + 1) danni perforanti.

\textbf{Ecologia}\\
I Grimlock abitano gli insediamenti abbandonati di altre Razze e sono spesso trovati come schiavi di altre creature più organizzate, come i nani ed Elfi. Si ritiene che si trattino di una propaggine ancora più degenerata dei Morlock, che viaggiano da Sekamina per cacciare i Grimlock per il cibo e considerano la loro carne una delicatezza.\\
\textbf{Descrizione}\\
I Grimlock sono creature umane cieche e selvagge che abitano nel regno delle terre oscure di profondità, dove si organizzano in piccoli gruppi tribali.

\mostro{Guardiano Protettore}
\noindent
\begin{description}[noitemsep, topsep=0pt, parsep=0pt, partopsep=0pt, leftmargin=0cm, labelwidth=2.2cm]
	\item[\textbf{Taglia/Tipo:}] Grande costrutto, disallineato
	\item[\textbf{Caratt.:}] \resizebox{0.5\linewidth+1.8cm}{!}{For 4 Des -1 Cos 4 Int -2 Sag 0 Car -4}
	\item[\textbf{Punti Ferita:}] 146,  \textbf{Difesa:} 20,  \textbf{Iniziativa:} -1
	\item[\textbf{Movimento:}] 9 m
	\item[\textbf{Tiri Salvez.:}] \resizebox{0.5\linewidth+1.8cm}{!}{Tempra +11, Riflessi +6, Volontà +7}
	\item[\textbf{Imm. Danni:}] Veleno
	\item[\textbf{Immunità:}] affascinato, paralizzato, affaticato, spaventato
	\item[\textbf{Sensi:}] Scurovisione 18 m, Vista Cieca 3 m
	\item[\textbf{Linguaggi:}] comprende i comandi forniti in qualsiasi lingua ma non può parlare
	\item[\textbf{Sfida:}] 7 (2900 PX)\smallskip
\end{description}

\emph{\textbf{Accumulare Incantesimi.}} Un incantatore che indossi l'amuleto del guardiano protettore può far sì che il guardiano accumuli un incantesimo di livello 4 o più basso. Per farlo, l'incantatore deve lanciare l'incantesimo sul guardiano. L'incantesimo non ha effetto ma viene accumulato all'interno del guardiano. Quando gli viene comandato di farlo da chi indossa l'amuleto o si presenta una situazione predeterminata dall'incantatore, il guardiano lancia l'incantesimo accumulato con tutti i parametri predisposti dall'incantatore originale, senza bisogno di componenti. Quando l'incantesimo viene lanciato o qualsiasi nuovo incantesimo viene accumulato, tutti gli incantesimi precedentemente accumulati vengono persi.

\emph{\textbf{Natura di Costrutto.}} Il guardiano non ha bisogno di aria, cibo, bevande o sonno.

\emph{\textbf{Rigenerazione.}} Il guardiano protettore recupera 10 Punti Ferita all'inizio del proprio round se ne possiede ancora almeno 1.

\emph{\textbf{Vincolato.}} Il guardiano protettore è vincolato magicamente ad un amuleto. Finché il guardiano e l'amuleto sono sullo stesso piano di esistenza, chi indossa l'amuleto può richiamare telepaticamente il guardiano perché lo raggiunga, e il guardiano saprà la distanza e la direzione in cui si trova l'amuleto. Se il guardiano si trova entro 18 metri da chi indossa l'amuleto, metà dei danni subiti da chi lo indossa (arrotondati per difetto) vengono trasferiti al guardiano. Se l'amuleto viene distrutto, il guardiano è inabile finché non viene creato un amuleto di rimpiazzo. L'amuleto del guardiano può essere soggetto ad un attacco diretto qualora non sia indossato o trasportato da nessuno. Ha Difesa 10, 10 Punti Ferita e immunità ai danni da veleno. Costruire un amuleto richiede 1 settimana e costa 10000 mo in componenti.

\textbf{Azioni}

\emph{\textbf{Multiattacco.}} Il golem effettua due attacchi di pugno.

\emph{\textbf{Pugno.} Attacco con arma da mischia}: +8 a colpire, portata 1 m, un bersaglio.

\emph{Colpisce:} 14 (3d6 + 4) danni contundenti.

\textbf{Reazione: \emph{Scudo.}} Quando una creatura attacca chi indossa l'amuleto del guardiano, il guardiano conferisce un bonus di +2 alla sua Difesa, se entro 1 metro dal suo controllore.

%\addcontentsline{toc}{subsubsection}{H}
\pdfbookmark[3]{H}{H}

\mostro{Hobgoblin}
\noindent
\begin{description}[noitemsep, topsep=0pt, parsep=0pt, partopsep=0pt, leftmargin=0cm, labelwidth=2.2cm]
	\item[\textbf{Taglia/Tipo:}] Media umanoide (goblinoide), malvagio
	\item[\textbf{Caratt.:}] \resizebox{0.5\linewidth+1.8cm}{!}{For 1 Des 1 Cos 1 Int 0 Sag 0 Car -1}
	\item[\textbf{Punti Ferita:}] 24,  \textbf{Difesa:} 13,  \textbf{Iniziativa:} +1
	\item[\textbf{Movimento:}] 9 m
	\item[\textbf{Tiri Salvez.:}] \resizebox{0.5\linewidth+1.8cm}{!}{Tempra +3, Riflessi +3, Volontà +3}
	\item[\textbf{Sensi:}] Scurovisione 18 m
	\item[\textbf{Linguaggi:}] Comune, Goblin
	\item[\textbf{Sfida:}] 1/2 (100 PX)\smallskip
\end{description}

\emph{\textbf{Marziale.}} Una volta per round, come Reazione, l'hobgoblin può infliggere 7 (2d6) danni aggiuntivi ad una creatura che colpisce con un attacco con arma, se quella creatura si trova entro 1 metro da un alleato dell'hobgoblin che non sia inabile.

\textbf{Azioni}

\emph{\textbf{Spada Lunga.} Attacco con arma da mischia}: +5 a colpire, portata 1 m, un bersaglio.

\emph{Colpisce:} 5 (1d8 + 1) danni taglienti o 6 (1d10 + 1) danni taglienti se usata con due mani.

\emph{\textbf{Arco Lungo.} Attacco con arma a Distanza}: +3 a colpire, gittata 45m, un bersaglio.

\emph{Colpisce:} 5 (1d8 + 1) danni perforanti.

\textbf{Ecologia}\\
Ambiente: Colline Temperate\\
Organizzazione: Gruppo (4-9), banda da guerra (10-24) o tribù (25+ più 50\% non combattenti, 1 sergente di 3° livello per 20 adulti, 1 o 2 luogotenenti di 4° o 5° livello, 1 capo di 6°-8° livello, 6-12 Leopardi e 1-4 Ogre o 1-2 Troll)\\
\textbf{Categoria Tesoro}: Equipaggiamento da PNG (Corazza di Cuoio Borchiato, Scudo Leggero di Metallo, Spada Lunga, Arco Lungo con 20 Frecce, O)\\
\textbf{Descrizione}\\
Gli Hobgoblin sono una razza militarista e prolifica, rendendoli molto pericolosi in alcune regioni. Procreano rapidamente, sostituendo i membri caduti con nuovi soldati, mantenendo il loro numero costante nonostante le guerre. Dichiarano guerra facilmente, spesso per catturare nuovi schiavi, la cui vita è brutale e breve. Gli schiavi sono necessari per rimpiazzare quelli che muoiono o vengono mangiati.

Tra le razze goblinoidi, gli Hobgoblin sono i più civilizzati. Vedono i Bugbear come strumenti utili per missioni specifiche come omicidi e furti, mentre guardano i Goblin con vergogna e frustrazione, nonostante ammirino la loro tenacia. La maggior parte delle tribù Hobgoblin include comunque un piccolo gruppo di Goblin, relegati agli angoli peggiori dell'insediamento.

Molte tribù Hobgoblin combinano l'amore per la guerra con l'intelletto acuto. Sono affascinati dalle macchine d'assedio, dall'alchimia e dall'ingegneria complessa. Gli Hobgoblin particolarmente dotati vengono trattati da eroi e ottengono posizioni di alto rango nella tribù. Gli schiavi con menti raffinate sono apprezzati, rendendo comuni le incursioni nelle città naniche.

Gli Hobgoblin disprezzano la magia e diffidano dei maghi. I loro sciamani, temuti e rispettati, vivono ai margini del covo della tribù. Gli Hobgoblin sono alti circa 1.7 metri e pesano 80 kg.

\mostro{Idra}
\noindent
\begin{description}[noitemsep, topsep=0pt, parsep=0pt, partopsep=0pt, leftmargin=0cm, labelwidth=2.2cm]
	\item[\textbf{Taglia/Tipo:}] Enorme mostruosità, disallineato
	\item[\textbf{Caratt.:}] \resizebox{0.5\linewidth+1.8cm}{!}{For 5 Des 1 Cos 5 Int -4 Sag 0 Car -2}
	\item[\textbf{Punti Ferita:}] 167,  \textbf{Difesa:} 23,  \textbf{Iniziativa:} +1
	\item[\textbf{Movimento:}] 9 m, nuoto 9 m
	\item[\textbf{Tiri Salvez.:}] \resizebox{0.5\linewidth+1.8cm}{!}{\resizebox{0.5\linewidth+1.8cm}{!}{Tempra +13, Riflessi +9, Volontà +8}}
	\item[\textbf{Comp.:}] Consapevolezza +6
	\item[\textbf{Sensi:}] Scurovisione 18 m
	\item[\textbf{Sfida:}] 8 (3900 PX)\smallskip
\end{description}

\emph{\textbf{Teste Multiple.}} L'idra ha cinque teste. Finché ha più di una testa, l'idra ha +1d6 ai Tiri Salvezza contro le condizioni accecata, affascinata, assordata, spaventata, stordita o svenuta.

Ogni volta che l'idra subisce 25 o più danni in un singolo round, una delle sue teste muore. Se tutte le teste muoiono anche l'idra muore.

Al termine del suo round, l'idra ricresce due teste per ciascuna delle sue teste uccise dal suo ultimo round, a meno che non abbia subito danno da fuoco dal suo ultimo round. L'idra recupera 10 Punti Ferita per ogni testa ricresciuta in questo modo.

\emph{\textbf{Teste Reattive.}} Per ogni testa posseduta oltre la prima, l'idra riceve una Azione di Reazione extra che può essere usata solo per compiere prove di Consapevolezza.

\emph{\textbf{Trattenere il Fiato.}} L'idra può trattenere il fiato per 1 ora.

\emph{\textbf{Veglia.}} Mentre l'idra dorme, almeno una delle sue teste resta sveglia.

\textbf{Azioni}

\emph{\textbf{Multiattacco.}} L'idra effettua tanti attacchi di morso quante sono le sue teste.

\emph{\textbf{Morso.} Attacco con arma da mischia}: +9 a colpire, portata 3 m, un bersaglio.

\emph{Colpisce:} 10 (1d10 + 5) danni perforanti.

\emph{\textbf{Gas ripugnanti.}} l'idra emette gas dall'odore ripugnante. Tutte le creature nel raggio di 3 metri dall'idra devono fare un Tiro Salvezza su Tempra DC 21 o subire -2 al Tiro per Colpire per i successivi 2d4 round.

\textbf{Ecologia}\\
Ambiente: Paludi Temperate\\
Organizzazione: Solitario\\
\textbf{Categoria Tesoro}: E\\
\textbf{Descrizione}\\
L'idra è un drago a più teste, ma stupido e con grossi problemi di digestione.

%\addcontentsline{toc}{subsubsection}{I}
\pdfbookmark[3]{I}{I}

\mostro{Ippogrifo}
\noindent
\begin{description}[noitemsep, topsep=0pt, parsep=0pt, partopsep=0pt, leftmargin=0cm, labelwidth=2.2cm]
	\item[\textbf{Taglia/Tipo:}] Grande bestia, disallineato
	\item[\textbf{Caratt.:}] \resizebox{0.5\linewidth+1.8cm}{!}{For 3 Des 1 Cos 1 Int -4 Sag 1 Car -1}
	\item[\textbf{Punti Ferita:}] 33,  \textbf{Difesa:} 14,  \textbf{Iniziativa:} +1
	\item[\textbf{Movimento:}] 12 m, volo 18 m
	\item[\textbf{Tiri Salvez.:}] \resizebox{0.5\linewidth+1.8cm}{!}{Tempra +3, Riflessi +3, Volontà +3}
	\item[\textbf{Comp.:}] Consapevolezza +5
	\item[\textbf{Sfida:}] 1 (200 PX)\smallskip
\end{description}

\emph{\textbf{Vista Affinata.}} L'ippogrifo ha +1d6 nelle prove di Consapevolezza basate sulla vista.

\textbf{Azioni}

\emph{\textbf{Multiattacco.}} L'ippogrifo effettua due attacchi: uno con il becco e uno con gli artigli.

\emph{\textbf{Artigli.} Attacco con arma da mischia}: +5 a colpire, portata 1 m, un bersaglio.

\emph{Colpisce:} 10 (2d6 + 3) danni taglienti.

\emph{\textbf{Becco.} Attacco con arma da mischia}: +5 a colpire, portata 1 m, un bersaglio.

\emph{Colpisce:} 8 (1d10 + 3) danni perforanti.

\textbf{Reazione: \emph{Attacco d'opportunità}}: l'ippogrifo attacca se sta volando ed una creatura esce o attraversa la sua portata di 2 m.

\textbf{Ecologia}\\
Ambiente: Colline Temperate o Pianure\\
Organizzazione: Solitario, coppia o stormo (7-12)\\
\textbf{Categoria Tesoro}: Nessuno\\
\textbf{Descrizione}\\
L'ippogrifo è una creatura affascinante con ali, zampe anteriori e testa di un grande rapace, e corpo e coda di un magnifico cavallo.

Le piume dell'ippogrifo variano nei colori del falco o dell'aquila, con alcuni allevatori che hanno prodotto esemplari con piume bianche o color carbone. Il corpo è spesso baio, nocciola o grigio, con manti pezzati o palomino. Gli ippogrifi misurano 3,3 metri di lunghezza e pesano fino a 680 kg.

Territoriali e feroci nel proteggere il loro dominio, devono anche sorvegliare i cieli poiché sono prede per grifoni, viverne e giovani draghi. Nidificano in praterie erbose, colline e canyon, prediligendo mammiferi e brucando erba per aiutare la digestione.

Le comunità di allevatori offrono spesso ricompense per catturarli poiché possono rappresentare un pericolo per le mandrie. Di gran lunga più facili da addestrare rispetto ai grifoni, gli ippogrifi vengono usati come animali da monta da compagnie scelte di soldati a cavallo. Se catturati giovani, possono essere addestrati come animali normali, ma gli adulti richiedono un addestramento speciale.

Gli ippogrifi sono ovipari e il loro nido contiene solitamente un solo uovo, che vale 200 mo. Un giovane ippogrifo in salute vale 500 mo, mentre un ippogrifo completamente addestrato come cavalcatura può valere fino a 5000 mo. Possono trasportare 90 kg come carico leggero, 180 kg come carico medio e 270 kg come carico pesante, e necessitano di una sella esotica per essere cavalcati.

%\addcontentsline{toc}{subsubsection}{K}
\pdfbookmark[3]{K}{K}

\mostro{Kraken}
\noindent
\begin{description}[noitemsep, topsep=0pt, parsep=0pt, partopsep=0pt, leftmargin=0cm, labelwidth=2.2cm]
	\item[\textbf{Taglia/Tipo:}] Mastodontica mostruosità (titano), malvagio
	\item[\textbf{Caratt.:}] \resizebox{0.5\linewidth+1.8cm}{!}{For 10 Des 0 Cos 7 Int 6 Sag 4 Car 5}
	\item[\textbf{Punti Ferita:}] 461,  \textbf{Difesa:} 42,  \textbf{Iniziativa:} +6
	\item[\textbf{Movimento:}] 6 m, nuoto 18 m
	\item[\textbf{Tiri Salvez.:}] \resizebox{0.5\linewidth+1.8cm}{!}{\resizebox{0.5\linewidth+1.8cm}{!}{Tempra +30, Riflessi +23, Volontà +27}}
	\item[\textbf{Imm. Danni:}] Elettricità, armi +1
	\item[\textbf{Immunità:}] paralizzato, spaventato
	\item[\textbf{Sensi:}] visione del vero 36 m
	\item[\textbf{Linguaggi:}] comprende Abissale, Celestiale, Infernale e Druidico ma non può parlare, telepatia 36 m
	\item[\textbf{Sfida:}] 23 (50000 PX)\smallskip
\end{description}

\emph{\textbf{Anfibio.}} Il kraken può respirare aria e acqua.

\emph{\textbf{Libertà di Movimento.}} Il kraken ignora i terreni difficili, e gli effetti magici non possono ridurne la velocità o far sì che diventi intralciato. Può spendere 1 Azione per liberarsi dalle restrizioni non magiche o dall'essere afferrato.

\emph{\textbf{Mostro d'Assedio.}} Il kraken infligge danni doppi agli oggetti e le strutture.

\textbf{Azioni}

\emph{\textbf{Multiattacco.}} Il kraken effettua tre attacchi di tentacolo, ciascuno dei quali può essere rimpiazzato da un uso di Fiondare.

\emph{\textbf{Morso.} Attacco con arma da mischia}: +17 a colpire, portata 6 m, un bersaglio.

\emph{Colpisce:} 23 (3d8 + 10) danni perforanti. Se il bersaglio è una creatura di taglia Grande o inferiore afferrato dal kraken, quella creatura viene inghiottita, e l'afferrare ha termine. Mentre è inghiottita, la creatura è accecata e intralciata, ha copertura completa contro gli attacchi e altri effetti provenienti dall'esterno del kraken, e subisce 42 (12d6) danni da acido all'inizio di ciascun round del kraken.

Se il kraken subisce 50 o più danni in un singolo round da una creatura al suo interno, il kraken deve riuscire un Tiro Salvezza di Tempra DC 35 o vomitare tutte le creature inghiottite, che cadono prone in uno spazio entro 3 metri dal kraken. Se il kraken muore, una creatura inghiottita non risulta più intralciata da esso e può fuggire dal cadavere usando 2 Azioni e uscendo prona.

\emph{\textbf{Tentacolo.} Attacco con arma da mischia}: +17 a colpire, portata 9 m, un bersaglio.

\emph{Colpisce:} 20 (3d6 + 10) danni contundenti, e il bersaglio è afferrato (DC 18 per fuggire). Il kraken ha dieci tentacoli, ciascuno dei quali può afferrare un bersaglio.

\emph{\textbf{Fiondare.}} Un oggetto impugnato o una creatura afferrata dal kraken, di taglia Grande o inferiore viene lanciato di 18 metri in una direzione casuale e gettata prona. Se il bersaglio lanciato colpisce una superficie solida, subisce 3 (1d6) danni contundenti per ogni 3 metri percorsi. Se il bersaglio viene lanciato contro un'altra creatura, quella creatura deve riuscire un Tiro Salvezza di Riflessi DC 34 o subire lo stesso danno e cadere prona.

\emph{\textbf{Tempesta di Fulmini.}} Il kraken crea magicamente tre saette di energia, ciascuna delle quali può colpire un bersaglio entro 36 metri e che il kraken possa vedere. Il bersaglio deve effettuare un Tiro Salvezza di Riflessi DC 35, e subire 22 (4d10) danni da elettricità se fallisce il Tiro Salvezza, o la metà se lo riesce.

\textbf{Azioni Aggiuntive}

Il kraken può effettuare 3 Azioni aggiuntive, scelte tra le opzioni seguenti. Può usare solo un'opzione Aggiuntiva alla volta e solo al termine del round di un'altra creatura. Il kraken recupera le Azioni aggiuntive spese all'inizio del proprio round.

\textbf{Attacco di Tentacolo o Fiondare.} Il kraken effettua un attacco di tentacolo o usa Fiondare.

\textbf{Nube di Inchiostro (Costa 3 Azioni).} Mentre si trova sott'acqua, il kraken espelle una nube di inchiostro con un raggio di 18 metri. La nube si propaga intorno agli angoli, e quell'area è oscurata pesantemente per tutte le creature tranne il kraken. Ciascuna creatura a parte il kraken che termini il suo round nell'area deve riuscire un Tiro Salvezza su Tempra 34, subendo 16 (3d10) danni da veleno se fallisce il Tiro Salvezza, o la metà se lo riesce. Una forte corrente disperde la nube, che altrimenti svanisce al termine del prossimo round del kraken. \textbf{Tempesta di Fulmini (Costa 2 Azioni).} Il kraken usa Tempesta di Fulmini.

\textbf{Ecologia}\\
Ambiente: Qualsiasi Oceano\\
Organizzazione: Solitario\\
\textbf{Categoria Tesoro}: 2 H\\
\textbf{Descrizione}\\
Il leggendario kraken è una delle più grandi paure dei marinai, perché è una creatura della taglia di una balena, può colpire delle profondità senza esser visto, può comandare i venti e le condizioni meteorologiche necessarie alla nave per muoversi, e possiede il crudele intelletto della maggior parte dei più spietati e creativi criminali del mondo. Alcuni credono che i kraken siano una punizione divina, mentre altri li ritengono i veri signori delle profondità, che considerano le razze che respirano aria nient'altro che bestiame.

Molte leggende sono sorte in merito al fatto che comprenda il linguaggio druidico.

Un kraken è lungo quasi 30 metri e pesa 20000 kg.

%\addcontentsline{toc}{subsubsection}{L}
\pdfbookmark[3]{L}{L}

\mostro{Lamia}
\noindent
\begin{description}[noitemsep, topsep=0pt, parsep=0pt, partopsep=0pt, leftmargin=0cm, labelwidth=2.2cm]
	\item[\textbf{Taglia/Tipo:}] Grande mostruosità, malvagio
	\item[\textbf{Caratt.:}] \resizebox{0.5\linewidth+1.8cm}{!}{For 3 Des 1 Cos 2 Int 2 Sag 2 Car 3}
	\item[\textbf{Punti Ferita:}] 88,  \textbf{Difesa:} 18,  \textbf{Iniziativa:} +2
	\item[\textbf{Movimento:}] 9 m
	\item[\textbf{Tiri Salvez.:}] \resizebox{0.5\linewidth+1.8cm}{!}{Tempra +6, Riflessi +5, Volontà +6}
	\item[\textbf{Comp.:}] Furtività +3, Ingannare +7, Percepire Emozioni +4,
	\item[\textbf{Sensi:}] Scurovisione 18 m
	\item[\textbf{Linguaggi:}] Abissale, Comune
	\item[\textbf{Sfida:}] 4 (1100 PX)\smallskip
\end{description}

\emph{\textbf{Incantesimi Innati.}} La caratteristica da incantatore innato della lamia è il Carisma. La lamia può lanciare in maniera innata i seguenti incantesimi, senza bisogno di componenti materiali:

A volontà: \emph{\hyperlink{Camuffare Sé Stesso}{Camuffare Sé Stesso}} (qualsiasi forma umanoide), \emph{\hyperlink{Immagine Maggiore}{Immagine Maggiore}}

3/Giorno ciascuno: \emph{\hyperlink{Charme su Persone}{Charme su Persone}, \hyperlink{Immagine Speculare}{Immagine Speculare}, \hyperlink{Scrutare}{Scrutare}, \hyperlink{Suggestione}{Suggestione}}

1/Giorno: \emph{\hyperlink{Costrizione}{Costrizione}}

\textbf{Azioni}

\emph{\textbf{Multiattacco.}} La lamia effettua due attacchi: uno con gli artigli e uno con il pugnale o il Tocco Intossicante.

\emph{\textbf{Artigli.} Attacco con arma da mischia}: +6 a colpire, portata 1 m, un bersaglio.

\emph{Colpisce:} 14 (2d10 + 3) danni taglienti, 1 danno da Sanguinamento.

\emph{\textbf{Pugnale.} Attacco con arma da mischia}: +6 a colpire, portata 1 m, un bersaglio.

\emph{Colpisce:} 6 (1d4 + 4) danni perforanti.

\emph{\textbf{Tocco Intossicante.} Attacco con incantesimo in mischia}: +5 a colpire, portata 1 m, una creatura.

\emph{Colpisce:} Il bersaglio è maledetto per 1 ora da questa magia. Fino al termine della maledizione, il bersaglio ha -1d6 ai Tiri Salvezza su Volontà e a tutte le prove di competenza di Base.

\textbf{Ecologia}\\
Ambiente: Deserti Temperati\\
Organizzazione: Solitario, coppia o setta (3-12)\\
\textbf{Categoria Tesoro}: Pugnale+1, D\\

\textbf{Descrizione}\\
Le lamie, eredi di un'antica maledizione, hanno l'aspetto di donne snelle e attraenti dalla vita in su, mentre la parte inferiore del corpo è simile a quella di un possente leone. Le loro caratteristiche umanoidi presentano tratti felini: occhi stretti e ferini e denti simili a zanne. Una lamia tipica è alta 1,8 metri, lunga 2,3 metri e pesa oltre 325 kg.

Queste creature sono attratte da torrioni in rovina, città abbandonate e monumenti dimenticati, specialmente in zone aride. Tuttavia, prediligono templi decrepiti, trovando gioia nel vederli in rovina e cercando di danneggiare i luoghi sacri alle divinità buone.

Le lamie venerano le femmine anziane del loro gruppo, considerandole capi, madri e sciamane, e si legano a loro con fanatica reverenza. Anche se rifuggono la maggior parte delle religioni, vedendole come la fonte della maledizione che le affligge, le lamie anziane affermano di udire i sussurri del vento del deserto e di conoscere i capricci delle stelle, guidando così il loro popolo.

\mostro{Lich}
\noindent
\begin{description}[noitemsep, topsep=0pt, parsep=0pt, partopsep=0pt, leftmargin=0cm, labelwidth=2.2cm]
	\item[\textbf{Taglia/Tipo:}] Media non morto, tratti malvagi
	\item[\textbf{Caratt.:}] \resizebox{0.5\linewidth+1.8cm}{!}{For 0 Des 3 Cos 3 Int 5 Sag 2 Car 3}
	\item[\textbf{Punti Ferita:}] 405,  \textbf{Difesa:} 43,  \textbf{Iniziativa:} +5
	\item[\textbf{Movimento:}] 9 m
	\item[\textbf{Tiri Salvez.:}] \resizebox{0.5\linewidth+1.8cm}{!}{\resizebox{0.5\linewidth+1.8cm}{!}{Tempra +24, Riflessi +24, Volontà +23}}
	\item[\textbf{Res. Danni:}] Freddo, Elettricità, da Vuoto
	\item[\textbf{Imm. Danni:}] Veleno; da arma non magica
	\item[\textbf{Immunità:}] affascinato, paralizzato, affaticato, spaventato, sanguinamento
	\item[\textbf{Sensi:}] visione del vero 36 m
	\item[\textbf{Linguaggi:}] Comune più altre cinque lingue, Expiran
	\item[\textbf{Sfida:}] 21 (33000 PX)\smallskip
\end{description}

\emph{\textbf{Incantesimi.}} Il lich ha CM 18. La sua caratteristica da incantatore è l'Intelligenza. Il lich conosce i seguenti incantesimi:

Trucchetti (a volontà): \emph{\hyperlink{Mano Magica}{Mano Magica}, \hyperlink{Prestidigitazione}{Prestidigitazione}, \hyperlink{Raggio di Gelo}{Raggio di Gelo}}

livello 1 (4 slot): \emph{\hyperlink{Dardo arcano}{Dardo arcano}, \hyperlink{Individuazione del Magico}{Individuazione del Magico}, \hyperlink{Onda Tonante}{Onda Tonante}, \hyperlink{Scudo}{Scudo}}

livello 2 (3 slot): \emph{\hyperlink{Freccia Acida di Restser}{Freccia Acida di Restser}, \hyperlink{Immagine Speculare}{Immagine Speculare}, \hyperlink{Individuazione dei Pensieri}{Individuazione dei Pensieri}, \hyperlink{Invisibilità}{Invisibilità}}

livello 3 (3 slot): \emph{\hyperlink{Animare Morti}{Animare Morti}, \hyperlink{Controincantesimo}{Controincantesimo}, \hyperlink{Dissolvi Magie}{Dissolvi Magie}, \hyperlink{Palla di Fuoco}{Palla di Fuoco}}

livello 4 (3 slot): \emph{\hyperlink{Inaridire}{Inaridire}, \hyperlink{Porta Dimensionale}{Porta Dimensionale}}

livello 5 (3 slot): \emph{\hyperlink{Nebbia mortale}{Nebbia mortale}, \hyperlink{Scrutare}{Scrutare}}

livello 6 (1 slot): \emph{\hyperlink{Disintegrazione}{Disintegrazione}, \hyperlink{Globo di Invulnerabilità}{Globo di Invulnerabilità}}

livello 7 (1 slot): \emph{\hyperlink{Dito della Morte}{Dito della Morte}}

livello 8 (1 slot): \emph{\hyperlink{Dominare Mostri}{Dominare Mostri}, \hyperlink{Parola del Potere Stordire}{Parola del Potere Stordire}}

livello 9 (1 slot): \emph{\hyperlink{Parola del Potere Uccidere}{Parola del Potere Uccidere}}

\emph{\textbf{Natura Non Morta.}} Il lich non ha bisogno di aria, cibo, bevande o sonno.

\emph{\textbf{Resistenza Leggendaria (3/Giorno).}} Se il lich fallisce un Tiro Salvezza, può scegliere invece di riuscirvi.

\emph{\textbf{Resistenza allo Scacciare.}} Il lich ha +1d6 ai Tiri Salvezza contro gli effetti che scacciano i non morti.

\emph{\textbf{Rinvigorimento.}} Se possiede un filatterio, il lich distrutto ottiene un nuovo corpo in 1d10 giorni, recuperando tutti i suoi Punti Ferita e ritornando in attività. Il nuovo corpo compare entro 1 metro dal filatterio.

\emph{\textbf{Sacrifici di Anime.}} Un lich deve ogni periodicamente nutrire di anime il suo filatterio per sostenere la magia che mantiene il suo corpo e la sua coscienza. Per farlo usa l'incantesimo \emph{\hyperlink{Imprigionare}{Imprigionare}}. Invece di scegliere una delle normali opzioni dell'incantesimo, il lich lo impiega per intrappolare magicamente il corpo e l'anima del bersaglio all'interno del filatterio. Il filatterio deve trovarsi sullo stesso piano del lich, perché questo incantesimo funzioni. Il filatterio di un lich può contenere solo una creatura alla volta, e \emph{\hyperlink{Dissolvi Magie Avanzato}{Dissolvi Magie Avanzato}} lanciato con 4 critici magici sul filatterio libera qualsiasi creatura imprigionata al suo interno. Il filatterio divora un anima a settimana. Una creatura imprigionata nel filatterio oltre questo periodo viene consumata e distrutta, dopodiché nulla salvo un intervento di un Patrono potrà riportarla in vita.

Un lich che dimentichi o non riesca a mantenere il suo corpo con le anime sacrificate inizia a cascare a pezzi, e potrebbe infine trasformarsi in un semilich.

\textbf{Azioni}

\emph{\textbf{Tocco Paralizzante.} Attacco con incantesimo in mischia}: +14 a colpire, portata 1 m, una creatura.

\emph{Colpisce:} 10 (3d6) danni da freddo. Il bersaglio deve riuscire un Tiro Salvezza di Tempra DC 32 o restare paralizzato per 1 minuto. Il bersaglio può ripetere il Tiro Salvezza al termine di ciascun suo round, terminando l'effetto su di sé in caso di successo.

\textbf{Reazione: \emph{Incantesimi rapidi}} il lich risponde ad un attacco subito lanciando un incantesimo a sua scelta fino al 3 livello, come Reazione.

\textbf{Azioni Aggiuntive}

Il lich può effettuare 3 Azioni aggiuntive, scelte tra le opzioni seguenti. Può usare solo un'opzione Aggiuntiva alla volta e solo al termine del round di un'altra creatura. Il lich recupera le Azioni aggiuntive spese all'inizio del proprio round.

\emph{\textbf{Distruggere Vita (Costa 3 Azioni).}} Ogni creatura ad eccezione dei non morti entro 6 metri dal lich deve effettuare un Tiro Salvezza su Tempra DC 31 contro questa magia, subendo 21 (6d6) danni da Vuoto se fallisce il Tiro Salvezza, o la metà di questi danni se lo riesce. Le creature diventano Affaticate.

\emph{\textbf{Sguardo Spaventoso (Costa 2 Azioni).}} Il lich fissa il suo sguardo su di una creatura visibile entro 3 metri da esso. Il bersaglio deve riuscire un Tiro Salvezza di Volontà DC 31 contro questa magia o restare spaventato per 1 minuto. Il bersaglio spaventato può ripetere il Tiro Salvezza al termine di ciascun suo round, terminando l'effetto su di sé in caso di successo. Se il Tiro Salvezza del bersaglio è riuscito o l'effetto per lui ha termine, il bersaglio è immune allo sguardo del lich per le successive 24 ore.

\emph{\textbf{Tocco Paralizzante (Costa 2 Azioni).}} Il lich usa il suo Tocco Paralizzante.

\emph{\textbf{Trucchetto.}} Il lich lancia un trucchetto.

\textbf{Ecologia}\\
Ambiente: Qualsiasi\\
Organizzazione: Solitario\\
\textbf{Categoria Tesoro}: Equipaggiamento da PNG (Anello di Protezione +2, Fascia della Sapienza +2 (Consapevolezza), Stivali della Levitazione, pergamena di \hyperlink{Dominare Persone}{Dominare Persone}, pergamena di Teletrasporto, pozione di Invisibilità)\\
\textbf{Descrizione}\\
Poche creature sono più temute dei lich. Apice delle arti Necromantiche, il lich è un incantatore che ha scelto di rinunciare alla vita ed ingannare la morte diventando non morto. Anche se molti di coloro che raggiungono simili vette di potenza farebbero di tutto per raggiungere l'immortalità, l'idea di diventare un lich è aborrita da molte creature. Il processo prevede di estrarre la forza vitale dell'incantatore e imprigionarla in un filatterio preparato in modo speciale; l'incantatore cede la sua vita, ma rimane intrappolato tra la vita e la morte, e fintanto che il suo filatterio rimane intatto può continuare le sue ricerche e il suo lavoro senza temere il passare del tempo.

Esistono anche rarissimi Lich buoni, ma come dice il detto sono più rari di un dente di Roc.

\mostro{Lucertoloide}
\noindent
\begin{description}[noitemsep, topsep=0pt, parsep=0pt, partopsep=0pt, leftmargin=0cm, labelwidth=2.2cm]
	\item[\textbf{Taglia/Tipo:}] Media umanoide (lucertoloide), neutrale
	\item[\textbf{Caratt.:}] \resizebox{0.5\linewidth+1.8cm}{!}{For 2 Des 0 Cos 1 Int -2 Sag 1 Car -2}
	\item[\textbf{Punti Ferita:}] 24,  \textbf{Difesa:} 12,  \textbf{Iniziativa:} +0
	\item[\textbf{Movimento:}] 9 m, nuoto 9 m
	\item[\textbf{Tiri Salvez.:}] \resizebox{0.5\linewidth+1.8cm}{!}{Tempra +3, Riflessi +3, Volontà +3}
	\item[\textbf{Comp.:}] Furtività +4, Consapevolezza +3, Sopravvivenza +5
	\item[\textbf{Linguaggi:}] Draconico
	\item[\textbf{Sfida:}] 1/2 (100 PX)\smallskip
\end{description}

\emph{\textbf{Trattenere il Fiato.}} Il lucertoloide può trattenere il fiato per 15 minuti.

\textbf{Azioni}

\emph{\textbf{Multiattacco.}} Il lucertoloide effettua due attacchi in mischia, ciascuno con un'arma diversa.

\emph{\textbf{Giavellotto.} Attacco con arma da mischia o a Distanza}: +4 a colpire, portata 1 m o gittata 12m, un bersaglio.

\emph{Colpisce:} 5 (1d6 + 2) danni perforanti.

\emph{\textbf{Morso.} Attacco con arma da mischia}: +4 a colpire, portata 1 m, un bersaglio.

\emph{Colpisce:} 5 (1d6 + 2) danni perforanti.

\emph{\textbf{Randello Pesante.} Attacco con arma da mischia}: +4 a colpire, portata 1 m, un bersaglio.

\emph{Colpisce:} 5 (1d6 + 2) danni contundenti.

\emph{\textbf{Scudo Appuntito.} Attacco con arma da mischia}: +4 a colpire, portata 1 m, un bersaglio.

\emph{Colpisce:} 5 (1d6 + 2) danni perforanti.

\textbf{Ecologia}\\
Ambiente: paludi temperate\\
Organizzazione: solitario, coppia, banda (3-12) o tribù (13-60)\\
\textbf{Categoria Tesoro}: Equipaggiamento da PNG (Scudo Pesante di Legno, Mazza chiodata, 3 Giavellotti)\\

\textbf{Descrizione}\\
I lucertoloidi sono rettili predatori orgogliosi e potenti che fanno le loro case comuni in sparuti villaggi nei recessi di paludi e acquitrini. Privi di interesse verso la colonizzazione delle terre aride e soddisfatti delle loro semplici armi e dei rituali che li hanno serviti bene per millenni, i lucertoloidi sono visti da molte delle altre razze come selvaggi retrogradi, ma all'interno delle loro isolate comunità sono in realtà un popolo vitale ricco di tradizioni e con una storia orale che risale a prima che l'uomo camminasse in posizione eretta.

La maggior parte dei lucertoloidi è alta dagli 1,8 ai 2,1 metro e pesa dai 100 ai 125 kg, ed ha i possenti muscoli coperti da scaglie grigie, verdi o marroni. Alcune razze hanno piccole creste dorsali o collari dai colori brillanti, e tutte nuotano bene spostandosi con rapidi movimenti della loro possente coda lunga 1,2 metri. Anche se sono pienamente a loro agio in acqua, trattengono il fiato e tornano alle loro abitazioni poste su colline artificiali per riprodursi e dormire. Poiché il loro sangue da rettile li rende lenti al freddo, molti lucertoloidi cacciano e lavorano durante il giorno e si ritirano nelle loro dimore di notte per rannicchiarsi con gli altri della loro tribù a condividere il calore di grandi fuochi di torba.

Anche se generalmente sono neutrali, il comportamento scostante dei lucertoloidi, il loro strenuo rifiuto dei \emph{doni} della civilizzazione, e la leggendaria ferocia in battaglia li fa mal giudicare dalla maggioranza degli umanoidi. Questi tratti derivano da buone ragioni, tuttavia, poiché il loro basso tasso di riproduzione non ha eguali tra gli umanoidi a sangue caldo, e se le tribù non difendessero i loro territori paludosi fino all'ultimo respiro si troverebbero presto sopraffatte da orde di mammiferi. Per quanto riguarda la loro propensione a mangiare i corpi dei morti sia amici che nemici, i pratici lucertoloidi sono lesti a sottolineare che la vita è dura nella palude, e nulla deve andare sprecato.

I lucertoloidi presentati qui vivono in ambienti paludosi. Le tribù lucertoloidi possono vivere altrettanto bene in altri ambienti, ma come velocità ottengono Scalare 5 metri al posto di Nuotare.

%\addcontentsline{toc}{subsubsection}{M}
\pdfbookmark[3]{M}{M}

\mostro{Maledetto immortale}
\noindent
\begin{description}[noitemsep, topsep=0pt, parsep=0pt, partopsep=0pt, leftmargin=0cm, labelwidth=2.2cm]
	\item[\textbf{Taglia/Tipo:}] Media aberrazione (umano), tendenzialmente folle
	\item[\textbf{Caratt.:}] \resizebox{0.5\linewidth+1.8cm}{!}{For 3 Des 1 Cos 2 Int -1 Sag -2 Car -2}
	\item[\textbf{Punti Ferita:}] 88,  \textbf{Difesa:} 18,  \textbf{Iniziativa:} +1
	\item[\textbf{Movimento:}] 9 m
	\item[\textbf{Tiri Salvez.:}] \resizebox{0.5\linewidth+1.8cm}{!}{Tempra +6, Riflessi +5, Volontà +3}
	\item[\textbf{Comp.:}] Consapevolezza +3, professione che aveva in vita
	\item[\textbf{Imm. Danni:}] Freddo, Veleno, Fuoco, da Vuoto
	\item[\textbf{Immunità:}] affascinato, pietrificato, spaventato
	\item[\textbf{Linguaggi:}] Comune, Nanico, Elfico
	\item[\textbf{Sfida:}] 4 (1100 PX)\smallskip
\end{description}

\emph{\textbf{Immortale}} Il Maledetto immortale rigenera 1 Punto Ferita a round, tranne se ha subito danni da acido. Ciò gli permette di rigenerare arti e tornare in vita. \hyperlink{Rimuovi Maledizione}{Rimuovi Maledizione} a DC 30 lo uccide istantaneamente.

\emph{\textbf{Natura diversa}} Il Maledetto immortale non mangia, beve, dorme, invecchia. Non è un non morto

\textbf{Azioni}

\emph{\textbf{Multiattacco.}} Il Maledetto immortale fa tre attacchi con la spada lunga.

\emph{\textbf{Spada.} Attacco con arma da mischia}: +6 a colpire, portata 1 m, un bersaglio.

\emph{Colpisce:} 12 (1d10 + 7) danni da taglio.

\textbf{Ecologia}\\
Ambiente: Qualsiasi\\
Organizzazione: Solitario\\
\textbf{Categoria Tesoro}: Equipaggiamento da PNG (Armatura di Cuoio Borchiato, 2 Pugnali, Spada, J)\\
\textbf{Descrizione}\\
Il Maledetto immortale è una persona maledetta spesso da un Patrono o da una potente incantatore con la maledizione della folle vita immortale. La maledizione rompe l'equilibro della persona e questa si ritrova a girovagare senza una meta od un obiettivo. Ogni tanto si ricordano chi erano ed allora proseguono nella ricerca di chi li ha maledetti.
Con lo scopo di farsi definitivamente uccidere si getta in ogni scontro sperando che l'avversario sia in grado di ucciderlo una volta per tutte.

\mostro{Monete affamate}\label{moneteaffamate}
\noindent
\begin{description}[noitemsep, topsep=0pt, parsep=0pt, partopsep=0pt, leftmargin=0cm, labelwidth=2.2cm]
	\item[\textbf{Taglia/Tipo:}] Minuscola aberrazione, fortemente malvagia
	\item[\textbf{Caratt.:}] \resizebox{0.5\linewidth+1.8cm}{!}{For -3 Des 1 Cos 3 Int -2 Sag 0 Car -1}
	\item[\textbf{Punti Ferita:}] 24,  \textbf{Difesa:} 13,  \textbf{Iniziativa:} +1
	\item[\textbf{Movimento:}] 3 m
	\item[\textbf{Tiri Salvez.:}] \resizebox{0.5\linewidth+1.8cm}{!}{Tempra +3, Riflessi +3, Volontà +3}
	\item[\textbf{Comp.:}] Consapevolezza +3
	\item[\textbf{Imm. Danni:}] Freddo, Veleno, Elettricità, da Vuoto
	\item[\textbf{Immunità:}] affascinato, pietrificato, spaventato
	\item[\textbf{Sfida:}] 1/2 (100 PX)\smallskip
\end{description}

Le Monete affamate attaccano sempre in gruppi da almeno 8 monete, quelle tenute in mano per contare...

\textbf{Azioni}

\emph{\textbf{Multiattacco.}} La Moneta affamata effettua due attacchi. Può usare il Morso due volte oppure usare il Morso e usare il Pungiglione.

\emph{\textbf{Morso.} Attacco con arma da mischia}: +4 a colpire, portata 0 m, un bersaglio.

\emph{Colpisce:} 2 danni perforanti.

\emph{\textbf{Pungiglione.} Attacco con arma da mischia}: +4 a colpire, portata 1 m, un bersaglio.

\emph{Colpisce:} 3 danni da Veleno. Tiro Salvezza su Tempra DC 13 o essere Rallentati 1/1r.

\textbf{Ecologia}\\
Ambiente: Qualsiasi\\
Organizzazione: Gruppi (3d12)\\
\textbf{Categoria Tesoro}: M, N, O\\
\textbf{Descrizione}\\
La Moneta affamata non è distinguibile da una normale moneta finché non osservata molto attentamente.
Voraci ed affamate amano nascondersi nelle pile di monete di cui si nutrono per assorbire i metalli che gli conferiscono poi il \emph{guscio} e l'aspetto di ordinarie monete. Attaccano sempre in gruppo, solitamente aspettando che qualcuno le tenga in mano per contarle. Ogni 10 Monete affamate, se \emph{svuotate e fuse}, è possibile ricavare abbastanza metallo per una vera moneta.
Monete affamate di Oro o Platino sono solitamente più robuste ed ancora più affamate. Dice la leggenda che una Moneta affamata non attaccherà un Devoto di Rezh.

\mostro{Cinghiale Mannaro}
\noindent
\begin{description}[noitemsep, topsep=0pt, parsep=0pt, partopsep=0pt, leftmargin=0cm, labelwidth=2.2cm]
	\item[\textbf{Taglia/Tipo:}] Media umanoide , mutaforma, malvagio
	\item[\textbf{Caratt.:}] \resizebox{0.5\linewidth+1.8cm}{!}{For 3 Des 0 Cos 2 Int 0 Sag 0 Car -1}
	\item[\textbf{Punti Ferita:}] 88,  \textbf{Difesa:} 17,  \textbf{Iniziativa:} +0
	\item[\textbf{Movimento:}] 9 m (12 m in forma di cinghiale)
	\item[\textbf{Tiri Salvez.:}] \resizebox{0.5\linewidth+1.8cm}{!}{Tempra +6, Riflessi +4, Volontà +4}
	\item[\textbf{Imm. Danni:}] da arma non magica o che non sia argentata
	\item[\textbf{Linguaggi:}] Comune (non può parlare in forma di cinghiale)
	\item[\textbf{Sfida:}] 4 (1100 PX)\smallskip
\end{description}

\emph{\textbf{Carica (Solo Forma di Cinghiale o Ibrida).}} Se il cinghiale mannaro si muove in linea retta di almeno 5 metri verso un bersaglio e poi lo colpisce con le zanne durante lo stesso round, il bersaglio subisce 7 (2d6) danni taglienti aggiuntivi. Se il bersaglio è una creatura, deve riuscire un Tiro Salvezza di Tempra DC 15 o cadere prono. 1 Azione.

\emph{\textbf{Implacabile (Ricarica dopo un 1 ora).}} Se il cinghiale mannaro subisce 14 danni o meno che lo ridurrebbero a 0 Punti Ferita, scende invece a 1 punto ferita.

\emph{\textbf{Mutaforma.}} Il cinghiale mannaro può usare 2 Azioni per trasformarsi in un ibrido cinghiale-umanoide o in un cinghiale, o per tornare alla sua vera forma, che è umanoide. Le sue statistiche, a parte la Difesa, sono le stesse in tutte le forme. Qualsiasi equipaggiamento stia indossando o trasportando non viene trasformato. Alla morte ritorna alla sua vera forma.

\textbf{Azioni}

\emph{\textbf{Multiattacco (Solo in Forma Umanoide o Ibrida).}} Il cinghiale mannaro effettua due attacchi, di cui solo uno può essere con le zanne.

\emph{\textbf{Maglio (Soltanto in Forma Umanoide o Ibrida).} Attacco con arma da mischia}: +6 a colpire, portata 1 m, un bersaglio.

\emph{Colpisce:} 10 (2d6 + 3) danni contundenti.

\emph{\textbf{Zanne (Soltanto in Forma di Cinghiale o Ibrida).} Attacco con arma da mischia}: +6 a colpire, portata 1 m, un bersaglio.

\emph{Colpisce:} 10 (2d6 + 3) danni taglienti. Se il bersaglio è un umanoide, deve riuscire un Tiro Salvezza di Tempra DC 15 o venire maledetto dalla licantropia del cinghiale mannaro.

\textbf{Ecologia}\\
Ambiente: Qualsiasi Foresta o Pianura\\
Organizzazione: Solitario, coppia, famiglia (3-8) o truppa (3-8 più 1-4 Cinghiali)\\
\textbf{Categoria Tesoro}: Equipaggiamento da PNG (Armatura di Cuoio Borchiato, 2 Pugnali, K)\\
\textbf{Descrizione}\\
Nella loro forma umanoide, i cinghiali mannari tendono a essere tozzi, con nasi all'insù, pelo ispido e incisivi prominenti. Hanno capelli rossi, castani o neri ma alcuni sono anche biondi, canuti o calvi. Hanno di norma peluria sul labbro superiore e i maschi di solito non riescono a far crescere la barba. Poiché sono testardi e aggressivi hanno piccole comunità di loro simili e non si mischiano ai non licantropi: di solito vivono in piccole fattorie dall'aspetto assolutamente normale. Tendono ad avere grandi famiglie e molti figli.

\mostro{Lupo Mannaro}
\noindent
\begin{description}[noitemsep, topsep=0pt, parsep=0pt, partopsep=0pt, leftmargin=0cm, labelwidth=2.2cm]
	\item[\textbf{Taglia/Tipo:}] Media umanoide (umano, mutaforma), malvagio
	\item[\textbf{Caratt.:}] \resizebox{0.5\linewidth+1.8cm}{!}{For 2 Des 1 Cos 2 Int 0 Sag 0 Car 0}
	\item[\textbf{Punti Ferita:}] 70,  \textbf{Difesa:} 17,  \textbf{Iniziativa:} +1
	\item[\textbf{Movimento:}] 9 m (12 m in forma di lupo)
	\item[\textbf{Tiri Salvez.:}] \resizebox{0.5\linewidth+1.8cm}{!}{Tempra +5, Riflessi +4, Volontà +3}
	\item[\textbf{Comp.:}] Furtività +3, Consapevolezza +4
	\item[\textbf{Imm. Danni:}] da arma non magica o che non sia argentata
	\item[\textbf{Linguaggi:}] Comune (non può parlare in forma di lupo)
	\item[\textbf{Sfida:}] 3 (700 PX)\smallskip
\end{description}

\emph{\textbf{Mutaforma.}} Il lupo mannaro può usare una Azione per trasformarsi in un ibrido lupo-umanoide o in un lupo, o per tornare alla sua vera forma, che è umanoide. Le sue statistiche, a parte la Difesa, sono le stesse in tutte le forme. Qualsiasi equipaggiamento stia indossando o trasportando non viene trasformato. Alla morte ritorna alla sua vera forma.

\emph{\textbf{Udito e Olfatto Affinato.}} Il lupo mannaro ha +1d6 nelle prove di Consapevolezza basate su udito o olfatto.

\textbf{Azioni}

\emph{\textbf{Multiattacco (Soltanto in Forma Umanoide o Ibrida).}} Il lupo mannaro effettua due attacchi: uno con il morso e uno con gli artigli o la lancia.

\emph{\textbf{Artigli (Soltanto in Forma Ibrida).} Attacco con arma da mischia}: +5 a colpire, portata 1 m, una creatura.

\emph{Colpisce:} 7 (2d4 + 2) danni taglienti.

\emph{\textbf{Lancia (Soltanto in Forma Umanoide).} Attacco con arma da mischia o a Distanza}: +4 a colpire, portata 1 m o gittata 6m, una creatura.

\emph{Colpisce:} 5 (1d6 + 2) danni perforanti o 6 (1d8 + 2) danni perforanti se usata con due mani in un attacco di mischia.

\emph{\textbf{Morso (Soltanto in Forma di Lupo o Ibrida).} Attacco con arma da mischia}: +5 a colpire, portata 1 m, un bersaglio.

\emph{Colpisce:} 6 (1d8 + 2) danni perforanti. Se il bersaglio è un umanoide, deve riuscire un Tiro Salvezza di Tempra DC 15 o venir maledetto dalla licantropia del lupo mannaro.

\textbf{Ecologia}\\
Ambiente: Qualsiasi Terreno\\
Organizzazione: Solitario, coppia o branco (3-6)\\
\textbf{Categoria Tesoro}: Equipaggiamento da PNG (Cotta di Maglia, Spada Lunga, Balestra Leggera con 20 Quadrelli, K)\\
\textbf{Descrizione}\\
Nella forma umana i lupi mannari somigliano a persone normali, anche se alcuni tendono ad avere un aspetto ferino e capelli ribelli. Sopracciglia che crescono unendosi, dito indice più lungo del medio e strane voglie sul palmo della mano sono tutti segni comunemente accettati che una persona sia in realtà un lupo mannaro. Naturalmente, questi segni rivelatori non sono sempre accurati, perché questi tratti fisici esistono anche nelle persone normali, ma nelle zone dove i lupi mannari sono un problema comune, questi tratti possono essere considerati schiaccianti a prescindere.

\mostro{Orso Mannaro}
\noindent
\begin{description}[noitemsep, topsep=0pt, parsep=0pt, partopsep=0pt, leftmargin=0cm, labelwidth=2.2cm]
	\item[\textbf{Taglia/Tipo:}] Media umanoide (umano, mutaforma), buono
	\item[\textbf{Caratt.:}] \resizebox{0.5\linewidth+1.8cm}{!}{For 4 Des 0 Cos 3 Int 0 Sag 1 Car 1}
	\item[\textbf{Punti Ferita:}] 108,  \textbf{Difesa:} 18,  \textbf{Iniziativa:} +0
	\item[\textbf{Movimento:}] 9 m (12 m, scalata 9 m in forma di orso o forma ibrida)
	\item[\textbf{Tiri Salvez.:}] \resizebox{0.5\linewidth+1.8cm}{!}{Tempra +8, Riflessi +5, Volontà +6}
	\item[\textbf{Comp.:}] Consapevolezza +7
	\item[\textbf{Imm. Danni:}] da arma non magica o che non sia argentata
	\item[\textbf{Linguaggi:}] Comune (non può parlare in forma di orso)
	\item[\textbf{Sfida:}] 5 (1800 PX)\smallskip
\end{description}

\emph{\textbf{Mutaforma.}} L'orso mannaro può usare una Azione per trasformarsi in un ibrido orso-umanoide o in un orso, o per tornare alla sua vera forma, che è umanoide. Le sue statistiche, a parte la Difesa, sono le stesse in tutte le forme. Qualsiasi equipaggiamento stia indossando o trasportando non viene trasformato. Alla morte ritorna alla sua vera forma.

\emph{\textbf{Olfatto Affinato.}} L'orso mannaro ha +1d6 nelle prove di Consapevolezza basate sull'olfatto.

\textbf{Azioni}

\emph{\textbf{Multiattacco.}} In forma di orso, l'orso mannaro effettua due attacchi di artiglio. In forma umanoide, effettua due attacchi di ascia bipenne. In forma ibrida, può attaccare come un orso o un umanoide.

\emph{\textbf{Artiglio (Soltanto in Forma di Orso o Ibrida).} Attacco con arma da mischia}: +7 a colpire, portata 1 m, un bersaglio.

\emph{Colpisce:} 13 (2d8 + 2) danni taglienti.

\emph{\textbf{Ascia Bipenne (Soltanto in Forma Umanoide o Ibrida).} Attacco con arma da mischia}: +6 a colpire, portata 1 m, un bersaglio.

\emph{Colpisce:} 10 (1d12 + 4) danni taglienti.

\emph{\textbf{Morso (Soltanto in Forma di Orso o Ibrida).} Attacco con arma da mischia}: +6 a colpire, portata 1 m, un bersaglio.

\emph{Colpisce:} 15 (2d10 + 4) danni perforanti. Se il bersaglio è un umanoide, deve riuscire un Tiro Salvezza di Tempra DC 16 o venir maledetto dalla licantropia dell'orso mannaro.

\textbf{Ecologia}\\
Ambiente: Qualsiasi Foresta\\
Organizzazione: Solitario, coppia, famiglia (3-6) o truppa (3-6 più 1-4 orsi Neri o Grigi)\\
\textbf{Categoria Tesoro}: Equipaggiamento da PNG (Giaco di Maglia, Ascia da Battaglia Perfetta, 2 Asce da Lancio Perfette, K)\\
\textbf{Descrizione}\\
Nelle loro forme umanoidi, gli orsi mannari tendono a essere muscolosi e con spalle larghe, tratti aspri e occhi scuri. Hanno capelli rossi, castani o neri e sembrano abituati a una vita di duro lavoro. Anche se i più benigni fra i licantropi, sono evitati dalla maggior parte delle persone normali, che temono la loro trasformazione animalesca. Per la maggior parte vivono in zone boschive isolate o in piccole unità familiari della loro stessa specie. Evitano di affrontare gli stranieri, ma non esitano se devono scacciare umanoidi malvagi dai loro territori.

\mostro{Ratto Mannaro}
\noindent
\begin{description}[noitemsep, topsep=0pt, parsep=0pt, partopsep=0pt, leftmargin=0cm, labelwidth=2.2cm]
	\item[\textbf{Taglia/Tipo:}] Media umanoide (umano, mutaforma), malvagio
	\item[\textbf{Caratt.:}] \resizebox{0.5\linewidth+1.8cm}{!}{For 0 Des 2 Cos 1 Int 0 Sag 0 Car -1}
	\item[\textbf{Punti Ferita:}] 51,  \textbf{Difesa:} 16,  \textbf{Iniziativa:} +2
	\item[\textbf{Movimento:}] 9 m
	\item[\textbf{Tiri Salvez.:}] \resizebox{0.5\linewidth+1.8cm}{!}{Tempra +3, Riflessi +4, Volontà +3}
	\item[\textbf{Comp.:}] Furtività +4, Consapevolezza +2
	\item[\textbf{Imm. Danni:}] da arma non magica o che non sia argentata
	\item[\textbf{Sensi:}] Scurovisione 18 m (solo in forma di ratto)
	\item[\textbf{Linguaggi:}] Comune (non può parlare in forma di ratto)
	\item[\textbf{Sfida:}] 2 (450 PX)\smallskip
\end{description}

\emph{\textbf{Mutaforma.}} Il ratto mannaro può usare una Azione per trasformarsi in un ibrido ratto-umanoide o in un ratto, o per tornare alla sua vera forma, che è umanoide. Le sue statistiche, a parte la Difesa, sono le stesse in tutte le forme. Qualsiasi equipaggiamento stia indossando o trasportando non viene trasformato. Alla morte ritorna alla sua vera forma.

\emph{\textbf{Olfatto Affinato.}} Il ratto mannaro ha +1d6 nelle prove di Consapevolezza basate sull'olfatto.

\textbf{Azioni}

\emph{\textbf{Multiattacco (Solo in Forma Umanoide o Ibrida).}} Il ratto mannaro effettua due attacchi, di cui solo uno può essere con il morso.

\emph{\textbf{Spada Corta (Soltanto in Forma Umanoide o Ibrida).} Attacco con arma da mischia}: +5 a colpire, portata 1 m, un bersaglio.

\emph{Colpisce:} 5 (1d6 + 2) danni perforanti.

\emph{\textbf{Balestra a mano (Soltanto in Forma Umanoide o Ibrida).} Attacco con arma a Distanza}: +6 a colpire, gittata 9m, un bersaglio.

\emph{Colpisce:} 5 (1d6 + 2) danni perforanti.

\emph{\textbf{Morso (Soltanto in Forma di Ratto o Ibrida).} Attacco con arma da mischia}: +5 a colpire, portata 1 m, un bersaglio.

\emph{Colpisce:} 4 (1d4 + 2) danni perforanti. Se il bersaglio è un umanoide, deve riuscire un Tiro Salvezza di Tempra DC 13 o venir maledetto dalla licantropia del ratto mannaro.

\textbf{Ecologia}\\
Ambiente: Qualsiasi Urbano\\
Organizzazione: Solitario, coppia, branco (5-10) o gilda (11-30 più 5-12 Ratti Crudeli)\\
\textbf{Categoria Tesoro}: Equipaggiamento da PNG (Armatura di Cuoio Borchiato Perfetta, Spada Corta, Balestra Leggera con 20 Quadrelli, K)\\
\textbf{Descrizione}\\
I ratti mannari naturali sono bassi, asciutti e muscolosi, con occhi attenti e vispi, e hanno movimenti nervosi. I maschi spesso hanno sottili baffi striminziti.

\mostro{Tigre Mannara}
\noindent
\begin{description}[noitemsep, topsep=0pt, parsep=0pt, partopsep=0pt, leftmargin=0cm, labelwidth=2.2cm]
	\item[\textbf{Taglia/Tipo:}] Media umanoide (umano, mutaforma), neutrale
	\item[\textbf{Caratt.:}] \resizebox{0.5\linewidth+1.8cm}{!}{For 3 Des 2 Cos 3 Int 0 Sag 1 Car 0}
	\item[\textbf{Punti Ferita:}] 89,  \textbf{Difesa:} 19,  \textbf{Iniziativa:} +2
	\item[\textbf{Movimento:}] 9 m (12 m in forma di tigre)
	\item[\textbf{Tiri Salvez.:}] \resizebox{0.5\linewidth+1.8cm}{!}{Tempra +7, Riflessi +6, Volontà +5}
	\item[\textbf{Comp.:}] Furtività +4, Consapevolezza +5
	\item[\textbf{Imm. Danni:}] da arma non magica che non siano argentati
	\item[\textbf{Sensi:}] Scurovisione 18 m
	\item[\textbf{Linguaggi:}] Comune (non può parlare in forma di tigre)
	\item[\textbf{Sfida:}] 4 (1100 PX)\smallskip
\end{description}

\emph{\textbf{Balzo.}} Se la tigre mannara si muove di almeno 5 metri in linea retta verso una creatura e la colpisce con un attacco di artiglio durante lo stesso round, il bersaglio deve riuscire un Tiro Salvezza su Tempra DC 16 o cadere prono. Se il bersaglio è prono, la tigre mannara può effettuare un attacco di morso contro di esso come Azione Immediata.

\emph{\textbf{Mutaforma.}} La tigre mannara può usare una Azione per trasformarsi in un ibrido tigre-umanoide o in una tigre, o per tornare alla sua vera forma, che è umanoide. Le sue statistiche, a parte la Difesa, sono le stesse in tutte le forme. Qualsiasi equipaggiamento stia indossando o trasportando non viene trasformato. Alla morte ritorna alla sua vera forma.

\emph{\textbf{Olfatto e Udito Affinato.}} La tigre mannara ha +1d6 nelle prove di Consapevolezza basate su olfatto e udito.

\textbf{Azioni}

\emph{\textbf{Multiattacco (Solo in Forma Umanoide o Ibrida).}} In forma umanoide, la tigre mannara effettua due attacchi di scimitarra o due attacchi di arco lungo. In forma ibrida, può attaccare come un umanoide o effettuare due attacchi di artiglio.

\emph{\textbf{Artiglio (Soltanto in Forma di Tigre o Ibrida).} Attacco con arma da mischia}: +5 a colpire, portata 1 m, un bersaglio.

\emph{Colpisce:} 7 (1d8 + 3) danni taglienti, 1 danno da Sanguinamento.

\emph{\textbf{Morso (Soltanto in Forma di Tigre o Ibrida).} Attacco con arma da mischia}: +6 a colpire, portata 1 m, un bersaglio.

\emph{Colpisce:} 8 (1d10 + 3) danni perforanti. Se il bersaglio è un umanoide, deve riuscire un Tiro Salvezza di Tempra DC 16 o venir maledetto dalla licantropia della tigre mannara.

\emph{\textbf{Scimitarra (Soltanto in Forma Umanoide o Ibrida).} Attacco con arma da mischia}: +5 a colpire, portata 1 m, un bersaglio.

\emph{Colpisce:} 6 (1d6 + 3) danni taglienti.

\emph{\textbf{Arco Lungo (Soltanto in Forma Umanoide o Ibrida).} Attacco con arma a Distanza}: +6 a colpire, gittata 45m, un bersaglio.

\emph{Colpisce:} 6 (1d8 + 2) danni perforanti.

\textbf{Reazione: \emph{Attacco d'opportunità}}: la tigre mannara effettua un attacco ad una creatura che attraversi o esca dalla sua portata di 1 metro.

\textbf{Ecologia}
Ambiente: Qualsiasi Pianura o Palude\\
Organizzazione: Solitario o coppia\\
\textbf{Categoria Tesoro}: Equipaggiamento da PNG (Armatura di Cuoio Borchiato, Spada Corta, 2 Pugnali, K)\\
\textbf{Descrizione}\\
Le tigri mannare in forma umanoide hanno grandi occhi, nasi allungati, zigomi sporgenti e capelli castani o rossi, oppure bianchi, neri o grigio-blu. I loro movimenti sono attenti e aggraziati, e chi li guarda potrebbe scambiarli per un ottimo tagliaborse, un danzatore aggraziato o un'abile cortigiana.

\mostro{Manticora}
\noindent
\begin{description}[noitemsep, topsep=0pt, parsep=0pt, partopsep=0pt, leftmargin=0cm, labelwidth=2.2cm]
	\item[\textbf{Taglia/Tipo:}] Grande mostruosità, malvagio
	\item[\textbf{Caratt.:}] \resizebox{0.5\linewidth+1.8cm}{!}{For 3 Des 3 Cos 3 Int -2 Sag 1 Car -1}
	\item[\textbf{Punti Ferita:}] 70,  \textbf{Difesa:} 19,  \textbf{Iniziativa:} +3
	\item[\textbf{Movimento:}] 9 m, volo 15 m
	\item[\textbf{Tiri Salvez.:}] \resizebox{0.5\linewidth+1.8cm}{!}{Tempra +6, Riflessi +6, Volontà +4}
	\item[\textbf{Sensi:}] Scurovisione 18 m
	\item[\textbf{Linguaggi:}] Comune
	\item[\textbf{Sfida:}] 3 (700 PX)\smallskip
\end{description}

\emph{\textbf{Ricrescere Spine della Coda.}} La manticora possiede ventiquattro spine nella coda. Le spine usate ricrescono all'alba.

\textbf{Azioni}

\emph{\textbf{Multiattacco.}} La manticora effettua tre attacchi: uno con il morso e due con gli artigli o tre con le spine della coda.

\emph{\textbf{Artiglio.} Attacco con arma da mischia}: +6 a colpire, portata 1 m, un bersaglio.

\emph{Colpisce:} 6 (1d6 + 3) danni taglienti, 1 danno da Sanguinamento.

\emph{\textbf{Morso.} Attacco con arma da mischia}: +6 a colpire, portata 1 m, un bersaglio.

\emph{Colpisce:} 7 (1d8 + 3) danni perforanti.

\emph{\textbf{Spine della Coda.} Attacco con arma a Distanza}: +6 a colpire, gittata 30m, un bersaglio.

\emph{Colpisce:} 7 (1d8 + 3) danni perforanti.

\textbf{Ecologia}
Ambiente: Colline e Paludi Calde\\
Organizzazione: Solitario, coppia o branco (3-6)\\
\textbf{Categoria Tesoro}: C\\
\textbf{Descrizione}\\
Le manticore sono feroci predatori che controllano vaste aree in cerca di carne fresca. Una tipica manticora è lunga circa 3 metri e pesa circa 500 kg. Alcune hanno la testa simile a quella di un umano, in genere barbuto. Maschi e femmine sono molto simili.

Le manticore mangiano qualsiasi tipo di carne, anche quella delle carogne, ma preferiscono quella umana e raramente si lasciano sfuggire un'occasione di gustare questa delizia. Sono abbastanza furbe e sociali da stringere patti con umanoidi malvagi per formare alleanze o da costringerli ad offre tributi, e molte creature potenti le incaricano di sorvegliare o controllare un posto o una zona. Prediligono fare le loro tane in posti alti, come le sommità delle colline e le caverne tra le rupi.

Anche se le manticore sono simili a delle creazioni magiche, sono da tempo annoverate tra le specie naturali. Curiosamente, le manticore sembrano stranamente feconde e possono incrociarsi con numerose altre specie dalla forma simile, inclusi Leoni, Tigri, Lamie, Sfingi e Chimere.

\mostro{Manto Assassino}
\noindent
\begin{description}[noitemsep, topsep=0pt, parsep=0pt, partopsep=0pt, leftmargin=0cm, labelwidth=2.2cm]
	\item[\textbf{Taglia/Tipo:}] Grande aberrazione, caotico
	\item[\textbf{Caratt.:}] \resizebox{0.5\linewidth+1.8cm}{!}{For 3 Des 2 Cos 1 Int 1 Sag 1 Car 2}
	\item[\textbf{Punti Ferita:}] 160,  \textbf{Difesa:} 24,  \textbf{Iniziativa:} +2
	\item[\textbf{Movimento:}] 3 m, volo 12 m
	\item[\textbf{Tiri Salvez.:}] \resizebox{0.5\linewidth+1.8cm}{!}{\resizebox{0.5\linewidth+1.8cm}{!}{Tempra +9, Riflessi +10, Volontà +9}}
	\item[\textbf{Comp.:}] Furtività +5
	\item[\textbf{Sensi:}] Scurovisione 18 m
	\item[\textbf{Linguaggi:}] Linguaggio delle Profondità
	\item[\textbf{Sfida:}] 8 (3900 PX)\smallskip
\end{description}

\emph{\textbf{Falso Aspetto.}} Mentre il manto assassino resta immobile senza esporre la parte inferiore del corpo, è indistinguibile da un manto in pelle nera.

\emph{\textbf{Sensibilità alla Luce}}. Mentre è alla luce del sole, il manto assassino ha -1d6 ai tiri per colpire, oltre che alle prove di Consapevolezza basate sulla vista.

\emph{\textbf{Trasferimento di Danno.}} Mentre è appiccicato ad una creatura il manto assassino subisce solo la metà dei danni che gli sono inferti (arrotondare per difetto), e la creatura vittima del manto assassino subisce l'altra metà.

\textbf{Azioni}

\emph{\textbf{Multiattacco.}} Il manto assassino effettua due attacchi: uno con il morso e uno con la coda.

\emph{\textbf{Morso.} Attacco con arma da mischia}: +9 a colpire, portata 1 m, una creatura.

\emph{Colpisce:} 10 (2d6 + 3) danni perforanti, e se il bersaglio è di taglia Grande o inferiore, il manto assassino vi si appiccica.
Finché il manto assassino è appiccicato ha un +1d6 ai Tiri per Colpire. Quando effettua un Tiro per Colpire ed ha il bonus di +1d6 e va a segno il bersaglio è accecato e impossibilitato a respirare. Il manto assassino può staccarsi spendendo 1 Azione di Movimento. Una creatura, compreso il bersaglio, può effettuare una Azione per staccare il manto assassino riuscendo una Tiro Salvezza su Tempra con modificatore Forza DC 21.

\emph{\textbf{Coda.} Attacco con arma da mischia}: +9 a colpire, portata 3 m, una creatura.

\emph{Colpisce:} 7 (1d8 + 3) danni taglienti.

\emph{\textbf{Apparizioni (Ricarica dopo un 1 ora).}} Qualora non si trovi sotto luce intensa, il manto assassino crea tre duplicati illusori di sé stesso, che si muovono assieme ad esso e ne imitano le azioni, scambiandosi di posizione per rendere impossibile capire quale sia il reale manto assassino. Se l'originale si trova in un'area di luce intensa, i duplicati svaniscono.

Ogniqualvolta una creatura prenda a bersaglio il manto assassino con un attacco o un incantesimo nocivo mentre sono ancora presenti dei duplicati, quella creatura determina casualmente se prende a bersaglio il manto assassino o uno dei duplicati. Una creatura che non possa vedere o che si affida a sensi diversi dalla vista ignora questo effetto magico.

Un duplicato possiede la Difesa e usa i Tiri Salvezza del manto assassino. Se un attacco colpisce un duplicato, o se un duplicato fallisce un Tiro Salvezza contro un effetto che infligge danni, svanisce.

\emph{\textbf{Gemito.}} Ogni creatura entro 18 metri dal manto assassino, che possa udire il suo gemito e che non sia un'aberrazione, deve riuscire un Tiro Salvezza su Volontà DC 21 o essere spaventata fino al termine del prossimo round del manto assassino. Se il Tiro Salvezza della creatura riesce, la creatura è immune al gemito del manto assassino per le successive 24 ore.

\emph{\textbf{Arrabbiato:}} il Manto assassino ricarica l'abilità Apparizioni. Costa 1 Azione.

\textbf{Ecologia}
Ambiente: Sotterranei\\
Organizzazione: Solitario, coppia, schiera (3-6) o stormo (7-12)\\
\textbf{Categoria Tesoro}: R\\
\textbf{Descrizione}\\
Simili a mante volanti orribilmente malvagie, i manti assassini sono creature misteriose e paranoiche. Un tipico esemplare ha un'apertura alare di 2,3 metri e pesa 50 kg.

Le loro motivazioni sono misteriose e confuse, e diffidano perfino dei loro simili. La strana forma permette loro di essere scambiati per mantelli, arazzi o altri oggetti comuni, e alcune storie narrano di manti assassini che si alleano con altre creature, facendosi trasportare sulla loro schiena e contribuendo alla protezione dei loro alleati per ragioni imperscrutabili.

\mostro{Mantoscuro}
\noindent
\begin{description}[noitemsep, topsep=0pt, parsep=0pt, partopsep=0pt, leftmargin=0cm, labelwidth=2.2cm]
	\item[\textbf{Taglia/Tipo:}] Piccola mostruosità, disallineato
	\item[\textbf{Caratt.:}] \resizebox{0.5\linewidth+1.8cm}{!}{For 3 Des 1 Cos 1 Int -4 Sag 0 Car -3}
	\item[\textbf{Punti Ferita:}] 24,  \textbf{Difesa:} 13,  \textbf{Iniziativa:} +1
	\item[\textbf{Movimento:}] 3 m, volo 9 m
	\item[\textbf{Tiri Salvez.:}] \resizebox{0.5\linewidth+1.8cm}{!}{Tempra +3, Riflessi +3, Volontà +3}
	\item[\textbf{Comp.:}] Furtività +3
	\item[\textbf{Sensi:}] Vista Cieca 18 m
	\item[\textbf{Sfida:}] 1/2 (100 PX)\smallskip
\end{description}

\emph{\textbf{Ecolocazione.}} Il mantoscuro non può usare la vista cieca se assordato.

\emph{\textbf{Falso Aspetto.}} Mentre il mantoscuro rimane immobile, è indistinguibile da una formazione rocciosa come una stalattite o una stalagmite.

\textbf{Azioni}

\emph{\textbf{Spaccare.} Attacco con arma da mischia}: +4 a colpire, portata 1 m, una creatura.

\emph{Colpisce:} 6 (1d6 + 3) danni contundenti e il mantoscuro si appiccica alla creatura. Se il bersaglio è di taglia Media o inferiore il mantoscuro ha +1d6 al Tiro per Colpire, si appiccica avvolgendo la testa del bersaglio, che è accecato e impossibilitato a respirare finché il mantoscuro resta appiccicato in questo modo.

Mentre è appiccicato al bersaglio, il mantoscuro non può attaccare nessun'altra creatura salvo il bersaglio, ma ha +1d6 ai suoi tiri per colpire. La velocità del mantoscuro diventa 0 e non può trarre beneficio da nessun bonus alla velocità, muovendosi assieme al bersaglio.

Una creatura può staccare il mantoscuro con un'Azione e riuscendo un Tiro Salvezza Tempra con Forza DC 13. Durante il suo round, il mantoscuro può staccarsi dal bersaglio da solo usando 1 Azione di Movimento.

\emph{\textbf{Aura di Oscurità (1/Giorno).}} Un'oscurità magica con 5 metri di raggio si estende dal mantoscuro, muovendosi con esso, e propagandosi oltre gli angoli. L'oscurità permane finché il mantoscuro mantiene la concentrazione, massimo 10 minuti (come se si stesse concentrando su di un incantesimo). La Scurovisione non può penetrare questa oscurità, né essa può essere rischiarata da alcuna luce naturale. Se qualsiasi parte dell'oscurità si sovrappone ad un'area di luce generata da un incantesimo di livello 2 o inferiore, l'incantesimo che sta creando la luce viene dissolto.

\textbf{Ecologia}
Ambiente: Qualsiasi (sotterraneo)\\
Organizzazione: Solitario, coppia o nidiata (3-12)\\
\textbf{Categoria Tesoro}: O\\
\textbf{Descrizione}\\
l'apertura tentacolare di un mantoscuro ha un'ampiezza di poco inferiore agli 1 m; quando è appeso alla volta di una caverna, mascherato da stalattite, la sua lunghezza varia tra i 60 ed i 90 cm. Un esemplare tipico di mantoscuro pesa 20 kg. La testa ed il corpo della creatura sono solitamente del colore del basalto o del granito scuro, ma i suoi tentacoli membranosi possono cambiare colore per adattarsi all'ambiente circostante.

I mantoscuro non sono scalatori particolarmente abili, ma sono in grado di appendersi alla volta di una caverna come i pipistrelli, agganciati per mezzo degli uncini posti in fondo ai loro tentacoli, così che il loro corpo penzolante risulti quasi indistinguibile da una stalattite. Da questa postazione nascosta la creatura attende che la preda passi sotto di lei e, a questo punto, si stacca lanciandosi verso di essa, sbattendo contro il bersaglio e tentando di avvolgervi attorno i suoi membranosi tentacoli. Se il mantoscuro manca la preda, risale e si lancia nuovamente contro la preda, fino a quando quest'ultima non viene sconfitta o il mantoscuro è gravemente ferito (nel qual caso svolazza sul soffitto per nascondersi, sperando che la sua preda lo lasci perdere). La capacità innata di questa creatura di celare la zona circostante per mezzo dell'oscurità magica le offre un ulteriore vantaggio contro gli avversari che necessitano della luce per vedere.

I mantoscuro preferiscono vivere e cacciare nelle caverne e nei cunicoli più vicini alla superficie, dal momento che questi offrono un più frequente passaggio di prede che questi mostri possono cacciare. Non si limitano però a queste caverne buie e talvolta possono essere incontrati in fortezze abbandonate o persino nelle fogne delle città affollate. Qualsiasi luogo dove abbondi il cibo e ci sia un soffitto a cui appendersi è un possibile covo per un mantoscuro.

Mantooscuro e Manto Assassino per quanto simili non appartengono alla stessa specie ma leggende narrano di una origine magica comune dovuta, come spesso capita, alla volontà di due maghi di trasformasi per primi in cappe... L'odio tra le due mostruosità è totale e prevarica ogni altro avversario presente.

\mostro{Medusa}
\noindent
\begin{description}[noitemsep, topsep=0pt, parsep=0pt, partopsep=0pt, leftmargin=0cm, labelwidth=2.2cm]
	\item[\textbf{Taglia/Tipo:}] Media mostruosità, malvagio
	\item[\textbf{Caratt.:}] \resizebox{0.5\linewidth+1.8cm}{!}{For 0 Des 2 Cos 3 Int 1 Sag 1 Car 2}
	\item[\textbf{Punti Ferita:}] 126,  \textbf{Difesa:} 22,  \textbf{Iniziativa:} +2
	\item[\textbf{Movimento:}] 9 m
	\item[\textbf{Tiri Salvez.:}] \resizebox{0.5\linewidth+1.8cm}{!}{Tempra +9, Riflessi +8, Volontà +7}
	\item[\textbf{Comp.:}] Furtività +5, Ingannare +5, Percepire Emozioni +4
	\item[\textbf{Sensi:}] Scurovisione 18 m
	\item[\textbf{Linguaggi:}] Comune
	\item[\textbf{Sfida:}] 6 (2300 PX)\smallskip
\end{description}

\emph{\textbf{Sguardo Pietrificante.}} Se una creatura comincia il suo round entro 9 metri da una medusa di cui possa vedere gli occhi, la medusa, qualora non sia inabile e possa vedere a sua volta la creatura, può obbligarla ad effettuare un Tiro Salvezza di Tempra DC 19. Se la creatura fallisce in maniera critica il Tiro Salvezza, viene pietrificata all'istante, altrimenti è Rallentata 1/1 minuto. Successivi sguardi e Tiri Salvezza falliti portano ad aumentare le condizione di Rallentato. Quando la creatura diventa Rallentata 3 si trasforma in pietra. La creatura può tornare di carne se viene lanciato l'incantesimo \hyperlink{Pietra in Carne}{Pietra in Carne} entro 1 mese dalla pietrificazione.

Una creatura che combatte la Medusa cercando di evitare il suo sguardo ha -1d6 al Tiro per Colpire.

Se la medusa vede il suo riflesso su di una superficie riflettente entro 9 metri da lei in un'area di luce intensa, a causa della propria maledizione subirà gli effetti del suo stesso sguardo.

\textbf{Azioni}

\emph{\textbf{Multiattacco.}} La medusa effettua tre attacchi, uno con i capelli serpentini e due con la spada corta oppure due attacchi a distanza con l'arco lungo.

\emph{\textbf{Capelli Serpentini.} Attacco con arma da mischia}: +6 a colpire, portata 1 m, un bersaglio.

\emph{Colpisce:} 4 (1d4 + 2) danni perforanti più 14 (4d6) danni da veleno.

\emph{\textbf{Spada Corta.} Attacco con arma da mischia}: +7 a colpire, portata 1 m, un bersaglio.

\emph{Colpisce:} 5 (1d6 + 2) danni perforanti.

\emph{\textbf{Arco Lungo.} Attacco con arma a Distanza}: +8 a colpire, gittata 45m, un bersaglio.

\emph{Colpisce:} 6 (1d8 + 2) danni perforanti più 7 (2d6) danni da veleno.

\textbf{Reazione: \emph{Attacco d'opportunità}}: la medusa effettua un attacco con i capelli serpentini ad una creatura che attraversi o esca dalla sua portata di 1 metro. Questo attacco non consuma Azioni o Reazioni.

\textbf{Ecologia}\\
Ambiente: Paludi temperate e sotterranei\\
Organizzazione: Solitario\\
\textbf{Categoria Tesoro}: Pugnale, Arco Lungo Perfetto con 20 Frecce, F\\
\textbf{Descrizione}\\
Le meduse sono creature simili agli umani con serpenti al posto dei capelli. Dalla distanza di 9 metri o più, una medusa può passare facilmente per una bella donna se indossa qualcosa che copre la sua chioma serpentina; quando indossa un abbigliamento che ne cela la testa e il volto può essere scambiata per un'umana anche a distanza ravvicinata. Le meduse usano bugie e travestimenti per celare il loro volto fino a che gli avversari non sono abbastanza vicini da usare il loro sguardo pietrificante, anche se gli piace giocare con la loro preda e possono usare delle frecce fiammeggianti per intrappolare i nemici a distanza. Alcune si divertono a creare intricate decorazioni con le loro vittime, usando la pietrificazione per dare un certo tocco ai loro nascondigli paludosi, ma molte meduse hanno cura di nascondere le prove dei loro scontri precedenti così che i loro nuovi nemici non si accorgano della loro pericolosa presenza.

Avvezze a nascondersi, le meduse cittadine generalmente sono ladre, mentre quelle delle zone selvagge spesso finiscono per essere guardiaboschi. Le meduse delle leggende più note, tuttavia, sono quelle che prendono livelli da incantatore. Carismatiche ed intelligenti, le meduse urbane sono spesso coinvolte in gilde di ladri ed altri aspetti del mondo criminale. Le meduse possono formare alleanze con creature cieche o non morti intelligenti, entrambi immuni al loro sguardo pietrificante. Le meduse incantatrici fungono spesso da oracoli o profetesse, vivendo generalmente in remote zone di leggendaria potenza o dalla storia infausta. Queste meduse oracoli traggono grande diletto dal loro ruolo, e se ci si presenta con i giusti doni e adulazioni, i segreti che offrono possono essere veramente utili. Naturalmente, i nascondigli di queste potenti creature sono decorati con le statue di coloro che le hanno offese, come monito ad usare le dovute cautele durante gli incontri.

Tutte le meduse sono femmine. Raramente, una medusa decide di prendere un maschio umanoide come compagno, generalmente grazie all'aiuto di una Elisir d'Amore o qualche magia simile, ed hanno sempre cura di non pietrificare il loro prigioniero, a meno che non si siano annoiate della sua compagnia.

\mostro{Mefito di Ghiaccio}
\noindent
\begin{description}[noitemsep, topsep=0pt, parsep=0pt, partopsep=0pt, leftmargin=0cm, labelwidth=2.2cm]
	\item[\textbf{Taglia/Tipo:}] Piccola elementale, malvagio
	\item[\textbf{Caratt.:}] \resizebox{0.5\linewidth+1.8cm}{!}{For -2 Des 1 Cos 0 Int -1 Sag 0 Car 1}
	\item[\textbf{Punti Ferita:}] 24,  \textbf{Difesa:} 13,  \textbf{Iniziativa:} +1
	\item[\textbf{Movimento:}] 9 m, volo 9 m
	\item[\textbf{Tiri Salvez.:}] \resizebox{0.5\linewidth+1.8cm}{!}{Tempra +3, Riflessi +3, Volontà +3}
	\item[\textbf{Comp.:}] Furtività +3, Consapevolezza +2
	\item[\textbf{Imm. Danni:}] Veleno
	\item[\textbf{Sensi:}] Scurovisione 18 m
	\item[\textbf{Linguaggi:}] Aquan, Ictun
	\item[\textbf{Sfida:}] 1/2 (100 PX)\smallskip
\end{description}

\emph{\textbf{Falso Aspetto.}} Mentre il mefito rimane immobile, è indistinguibile da un ordinario frammento di ghiaccio.

\emph{\textbf{Incantesimi Innati (1/Giorno).}} Il mefito può lanciare in maniera innata \emph{\hyperlink{Nube di Nebbia}{Nube di Nebbia}}, senza bisogno di componenti materiali. La sua caratteristica da incantatore innato è il Carisma.

\emph{\textbf{Natura Elementale.}} Un mefito non ha bisogno di cibo, bevande o sonno.

\emph{\textbf{Scoppio Mortale.}} Quando il mefito muore, esplode in uno scoppio di frammenti di ghiaccio. Ogni creatura entro 1 metro da esso deve effettuare un Tiro Salvezza di Riflessi DC 11 o subire 4 (1d8) danni taglienti in caso di fallimento, o la metà di questi danni in caso
di successo.

\textbf{Azioni}

\emph{\textbf{Artigli.} Attacco con arma da mischia}: +4 a colpire, portata 1 m, una creatura.

\emph{Colpisce:} 3 (1d4 + 1) danni taglienti più 2 (1d4) danni da freddo.

\emph{\textbf{Soffio Gelido (Ricarica 6).}} Il mefito esala un cono di 5 metri di aria fredda. Ogni creatura nell'area deve effettuare un Tiro Salvezza di Riflessi DC 11, subendo 5 (2d4) danni da freddo in caso di fallimento, o la metà di questi danni in caso di successo.

\textbf{Ecologia}\\
Ambiente: Qualsiasi (piano elementale dell'aria)\\
Organizzazione: Solitario, coppia, gruppo (3-6) o stormo (7-12)\\
\textbf{Categoria Tesoro}: J\\
\textbf{Descrizione}\\
I mephit sono i servitori di potenti creature elementali. I siti e le locazioni chiave dei piani elementali sono pieni di mephit che si affannano per svolgere un importante dovere o incarico.

I mephit del ghiaccio comunemente si trovano sul Piano dell'Aria. Questi mephit sono distanti e crudeli.

\mostro{Mefito di Magma}
\noindent
\begin{description}[noitemsep, topsep=0pt, parsep=0pt, partopsep=0pt, leftmargin=0cm, labelwidth=2.2cm]
	\item[\textbf{Taglia/Tipo:}] Piccola elementale, malvagio
	\item[\textbf{Caratt.:}] \resizebox{0.5\linewidth+1.8cm}{!}{For -1 Des 1 Cos 1 Int -2 Sag 0 Car 0}
	\item[\textbf{Punti Ferita:}] 24,  \textbf{Difesa:} 13,  \textbf{Iniziativa:} +1
	\item[\textbf{Movimento:}] 9 m, volo 9 m
	\item[\textbf{Tiri Salvez.:}] \resizebox{0.5\linewidth+1.8cm}{!}{Tempra +3, Riflessi +3, Volontà +3}
	\item[\textbf{Comp.:}] Furtività +3
	\item[\textbf{Imm. Danni:}] Veleno
	\item[\textbf{Sensi:}] Scurovisione 18 m
	\item[\textbf{Linguaggi:}] Ignan, Tremun
	\item[\textbf{Sfida:}] 1/2 (100 PX)\smallskip
\end{description}

\emph{\textbf{Falso Aspetto.}} Mentre il mefito rimane immobile, è indistinguibile da un'ordinaria pozza di magma.

\emph{\textbf{Incantesimi Innati (1/Giorno).}} Il mefito può lanciare in maniera innata \emph{riscaldare metallo} (DC del Tiro Salvezza dell'incantesimo 10), senza bisogno di componenti materiali. La sua caratteristica da incantatore innato è il Carisma.

\emph{\textbf{Natura Elementale.}} Un mefito non ha bisogno di cibo, bevande o sonno.

\emph{\textbf{Scoppio Mortale.}} Quando il mefito muore, esplode in uno scoppio di lava. Ogni creatura entro 1 metro da esso deve effettuare un Tiro Salvezza di Riflessi DC 11 o subire 7 (2d6) danni da fuoco in caso di fallimento, o la metà di questi danni in caso di successo.

\textbf{Azioni}

\emph{\textbf{Artigli.} Attacco con arma da mischia}: +4 a colpire, portata 1 m, una creatura.

\emph{Colpisce:} 3 (1d4 + 1) danni taglienti più 2 (1d4) danni da fuoco.

\emph{\textbf{Soffio Infuocato (Ricarica 6).}} Il mefito esala un cono di 5 metri di fuoco. Ogni creatura nell'area deve effettuare un Tiro Salvezza su Riflessi DC 11, subendo 7 (2d6) danni da fuoco in caso di fallimento, o la metà di questi danni in caso di successo.

\textbf{Ecologia}\\
Ambiente: Qualsiasi (piano elementale del fuoco)\\
Organizzazione: Solitario, coppia, gruppo (3-6) o stormo (7-12)\\
\textbf{Categoria Tesoro}: J\\
\textbf{Descrizione}\\
I mephit sono i servitori di potenti creature elementali. I siti e le locazioni chiave dei piani elementali sono pieni di mephit che si affannano per svolgere un importante dovere o incarico.

I mephit del magma comunemente si trovano sul Piano del Fuoco. Questi mephit sono stupidi bruti.

\mostro{Mefito di Polvere}
\noindent
\begin{description}[noitemsep, topsep=0pt, parsep=0pt, partopsep=0pt, leftmargin=0cm, labelwidth=2.2cm]
	\item[\textbf{Taglia/Tipo:}] Piccola elementale, malvagio
	\item[\textbf{Caratt.:}] \resizebox{0.5\linewidth+1.8cm}{!}{For -3 Des 2 Cos 0 Int -1 Sag 0 Car 0}
	\item[\textbf{Punti Ferita:}] 24,  \textbf{Difesa:} 14,  \textbf{Iniziativa:} +2
	\item[\textbf{Movimento:}] 9 m, volo 9 m
	\item[\textbf{Tiri Salvez.:}] \resizebox{0.5\linewidth+1.8cm}{!}{Tempra +3, Riflessi +3, Volontà +3}
	\item[\textbf{Comp.:}] Furtività +4, Consapevolezza +2
	\item[\textbf{Imm. Danni:}] Veleno
	\item[\textbf{Sensi:}] Scurovisione 18 m
	\item[\textbf{Linguaggi:}] Ictun, Tremun
	\item[\textbf{Sfida:}] 1/2 (100 PX)\smallskip
\end{description}

\emph{\textbf{Incantesimi Innati (1/Giorno).}} Il mefito può eseguire in maniera innata \emph{sonno} (DC del Tiro Salvezza dell'incantesimo 11), senza bisogno di componenti materiali. La sua abilità da incantatore innato è il Carisma.

\emph{\textbf{Natura Elementale.}} Un mefito non ha bisogno di cibo, bevande o sonno.

\emph{\textbf{Scoppio Mortale.}} Quando il mefito muore, esplode in uno scoppio di polvere. Ogni creatura entro 1 metro da esso deve riuscire un Tiro Salvezza di Tempra DC 11 o restare accecata per 1 minuto. Una creatura accecata può ripetere il Tiro Salvezza durante ciascun suo round, terminando l'effetto su di sé in caso di successo.

\textbf{Azioni}

\emph{\textbf{Artigli.} Attacco con arma da mischia}: +4 a colpire, portata 1 m, una creatura.

\emph{Colpisce:} 4 (1d4 + 2) danni taglienti.

\emph{\textbf{Soffio Accecante (Ricarica 6).}} Il mefito esala un cono di 5 metri di polvere accecante. Ogni creatura nell'area deve riuscire un Tiro Salvezza di Riflessi DC 11 o restare accecata per 1 minuto. Una creatura accecata può ripetere il Tiro Salvezza durante ciascun suo round, terminando l'effetto su di sé in caso di successo.

\textbf{Ecologia}\\
Ambiente: Qualsiasi (piano elementale dell'aria)\\
Organizzazione: Solitario, coppia, gruppo (3-6) o stormo (7-12)\\
\textbf{Categoria Tesoro}: J\\
\textbf{Descrizione}\\
I mephit sono i servitori di potenti creature elementali. I siti e le locazioni chiave dei piani elementali sono pieni di mephit che si affannano per svolgere un importante dovere o incarico.

I mephit della polvere comunemente si trovano sul Piano dell'Aria. Questi mephit sono irritanti ed insistenti.

\mostro{Mefito di Vapore}
\noindent
\begin{description}[noitemsep, topsep=0pt, parsep=0pt, partopsep=0pt, leftmargin=0cm, labelwidth=2.2cm]
	\item[\textbf{Taglia/Tipo:}] Piccola elementale, malvagio
	\item[\textbf{Caratt.:}] \resizebox{0.5\linewidth+1.8cm}{!}{For -3 Des 0 Cos 0 Int 0 Sag 0 Car 1}
	\item[\textbf{Punti Ferita:}] 19,  \textbf{Difesa:} 12,  \textbf{Iniziativa:} +0
	\item[\textbf{Movimento:}] 9 m, volo 9 m
	\item[\textbf{Tiri Salvez.:}] \resizebox{0.5\linewidth+1.8cm}{!}{Tempra +3, Riflessi +3, Volontà +3}
	\item[\textbf{Imm. Danni:}] Veleno
	\item[\textbf{Sensi:}] Scurovisione 18 m
	\item[\textbf{Linguaggi:}] Aquan, Ignan
	\item[\textbf{Sfida:}] 1/4 (50 PX)\smallskip
\end{description}

\emph{\textbf{Incantesimi Innati (1/Giorno).}} Il mefito può eseguire in maniera innata \emph{\hyperlink{Sfocatura}{Sfocatura}}, senza bisogno di componenti materiali. La sua abilità da incantatore innato è il Carisma.

\emph{\textbf{Natura Elementale.}} Un mefito non ha bisogno di cibo, bevande o sonno.

\emph{\textbf{Scoppio Mortale.}} Quando il mefito muore, esplode in nube di vapore. Ogni creatura entro 1 metro da esso deve riuscire un Tiro Salvezza su Riflessi DC 10 o subire 4 (1d8) danni da fuoco.

\textbf{Azioni}

\emph{\textbf{Artigli.} Attacco con arma da mischia}: +4 a colpire, portata 1 m, una creatura.

\emph{Colpisce:} 2 (1d4) danni taglienti più 2 (1d4) danni da fuoco.

\emph{\textbf{Soffio Vaporoso (Ricarica 6).}} Il mefito esala un cono di 5 metri di vapore caldo. Ogni creatura nell'area deve effettuare un Tiro Salvezza di Riflessi DC 10, subendo 4 (1d8) danni da fuoco in caso di fallimento, o la metà di questi danni in caso di successo.

\textbf{Ecologia}\\
Ambiente: Qualsiasi (piano elementale del fuoco)\\
Organizzazione: Solitario, coppia, gruppo (3-6) o stormo (7-12)\\
\textbf{Categoria Tesoro}: J\\
\textbf{Descrizione}\\
I mephit sono i servitori di potenti creature elementali. I siti e le locazioni chiave dei piani elementali sono pieni di mephit che si affannano per svolgere un importante dovere o incarico.

I mephit del vapore comunemente si trovano sul Piano del Fuoco. Questi mephit sono insolenti e sprezzanti.

\mostro{Megera Marina}
\noindent
\begin{description}[noitemsep, topsep=0pt, parsep=0pt, partopsep=0pt, leftmargin=0cm, labelwidth=2.2cm]
	\item[\textbf{Taglia/Tipo:}] Media fatato, malvagio
	\item[\textbf{Caratt.:}] \resizebox{0.5\linewidth+1.8cm}{!}{For 3 Des 1 Cos 3 Int 1 Sag 1 Car 1}
	\item[\textbf{Punti Ferita:}] 52,  \textbf{Difesa:} 15,  \textbf{Iniziativa:} +1
	\item[\textbf{Movimento:}] 9 m, nuoto 12 m
	\item[\textbf{Tiri Salvez.:}] \resizebox{0.5\linewidth+1.8cm}{!}{Tempra +5, Riflessi +3, Volontà +3}
	\item[\textbf{Sensi:}] Scurovisione 18 m
	\item[\textbf{Linguaggi:}] Aquan, Comune, Gigante
	\item[\textbf{Sfida:}] 2 (450 PX)\smallskip
\end{description}

\emph{\textbf{Anfibio.}} La megera può respirare aria e acqua.

\emph{\textbf{Aspetto Orripilante.}} Qualsiasi umanoide che inizi il suo round entro 9 metri dalla megera e ne può vedere la vera forma deve effettuare un Tiro Salvezza di Volontà DC 13. Se fallisce il Tiro Salvezza, la creatura resta spaventata per 1 minuto. Una creatura può ripetere il Tiro Salvezza al termine di ciascun suo round, con -1d6 se la megera è in linea di visuale, e terminando l'effetto se riesce il Tiro Salvezza. Se il Tiro Salvezza della creatura riesce o l'effetto ha termine su di essa, la creatura è immune all'Aspetto Orripilante per le successive 24 ore.

A meno che il bersaglio non sia sorpreso o la rivelazione della vera forma della megera non sia improvvisa, il bersaglio può distogliere lo sguardo e evitare di effettuare il Tiro Salvezza iniziale. Fino all'inizio del suo prossimo round, una creatura che distolga lo sguardo ha -1d6 ai tiri di attacco contro la megera.

\textbf{Azioni}

\emph{\textbf{Artigli.} Attacco in mischia con arma}: +5 a colpire, portata 1 m, un bersaglio.

\emph{Colpisce:} 10 (2d6 + 3) danni taglienti, 1 danno da Sanguinamento.

\emph{\textbf{Aspetto Illusorio.}} La megera ricopre se stessa e tutto quello che sta indossando o trasportando in un'illusione magica che le dona l'aspetto di una creatura ripugnante all'incirca della stessa taglia e forma umanoide. L'illusione termina se la megera effettua una Reazione per terminarla o se muore.

I cambiamenti apportati da questo effetto non sono in grado di superare le ispezioni fisiche. Ad esempio, la megera potrebbe apparire come una creatura priva di artigli, ma una persona in contatto con le sue mani li avvertirebbe. Altrimenti, una creatura deve effettuare un'Azione per ispezionare visivamente l'illusione e riuscire una prova di Consapevolezza DC 16 per comprendere che la megera si è camuffata.

\emph{\textbf{Occhiata Mortale.}} La megera prende a bersaglio una creatura spaventata visibile entro 9 metri da lei. Se il bersaglio può vedere la megera, deve riuscire un Tiro Salvezza di Volontà DC 13 contro questa magia o scendere a 0 Punti Ferita.

\textbf{Ecologia}\\
Ambiente: qualsiasi acquatico\\
Organizzazione: solitario o congrega (3 megere di qualsiasi specie)\\
\textbf{Categoria Tesoro}: R (C)\\
\textbf{Descrizione}\\
Queste perfide e mostruose megere possiedono dei tratti terrificanti che pochi osano fissare, traggono piacere dalla discordia e dalla morte dei marinai, e tormentano la gente di mare con ineluttabili sciagure. Le megere marine hanno sempre un aspetto terribile e, malgrado la loro natura famelica, in genere sono creature emaciate che sembrano sul punto di morir di fame. Sono alte 1,8 metri e pesano 75 kg.

Le megere marine preferiscono vivere vicino alla riva dove i pescherecci e i mercantili sono più comuni, e comunque lontano dalle aree urbane di modo che le loro azioni non attraggano troppo l'attenzione di possibili nemici, anche se non è insolito che una megera marina coraggiosa o avida si stabilisca in una città portuale o alla foce di un fiume profondo.

Le megere marine formano congreghe simili a quelle delle altre megere, ma la loro natura acquatica generalmente le spinge ad astenersi dal formare congreghe miste. Nel caso in cui una Megera Verde abiti lungo la costa (spesso in una palude salmastra o in una palude costiera), una congrega è formata da due megere marine che rispettano la Megera Verde come madre e capo. Molto comunemente, una congrega di megere marine consiste in un gruppo di megere marine particolarmente amiche e vicine.

\mostro{Megera Notturna}
\noindent
\begin{description}[noitemsep, topsep=0pt, parsep=0pt, partopsep=0pt, leftmargin=0cm, labelwidth=2.2cm]
	\item[\textbf{Taglia/Tipo:}] Media immondo, malvagio
	\item[\textbf{Caratt.:}] \resizebox{0.5\linewidth+1.8cm}{!}{For 4 Des 2 Cos 3 Int 3 Sag 2 Car 3}
	\item[\textbf{Punti Ferita:}] 108,  \textbf{Difesa:} 20,  \textbf{Iniziativa:} +3
	\item[\textbf{Movimento:}] 9 m
	\item[\textbf{Tiri Salvez.:}] \resizebox{0.5\linewidth+1.8cm}{!}{Tempra +8, Riflessi +7, Volontà +7}
	\item[\textbf{Comp.:}] Furtività +6, Ingannare +7, Percepire Emozioni +6
	\item[\textbf{Res. Danni:}] Freddo, Fuoco; da arma non magica o non siano argentati
	\item[\textbf{Sensi:}] Scurovisione 36 m
	\item[\textbf{Linguaggi:}] Abissale, Comune, Infernale, Druidico
	\item[\textbf{Sfida:}] 5 (1800 PX)\smallskip
\end{description}

\emph{\textbf{Incantesimi Innati.}} La caratteristica da incantatore innato della megera è il Carisma (DC 14 per i Tiri Salvezza degli incantesimi. La megera può lanciare in maniera innata i seguenti incantesimi, senza aver bisogno di componenti materiali.

A volontà: \emph{\hyperlink{Dardo arcano}{Dardo arcano}, \hyperlink{Individuazione del Magico}{Individuazione del Magico}} 2/giorno ciascuno: \emph{\hyperlink{Raggio di Indebolimento}{Raggio di Indebolimento}, \hyperlink{Sonno}{Sonno}}

\emph{\textbf{Resistenza alla Magia.}} La megera ha +1d6 ai tiri salvezza contro incantesimi e altri effetti magici.

\textbf{Azioni}

\emph{\textbf{Artigli (Solo in Forma di Megera).} Attacco con arma da mischia}: +7 a colpire, portata 1 m, un bersaglio.

\emph{Colpisce:} 13 (2d8 + 4) danni taglienti, 1 danno da Sanguinamento.

\emph{\textbf{Forma Eterea.}} La megera entra magicamente nel Piano Etereo dal Piano Materiale, e viceversa.

\textbf{Reazione: \emph{Attacco d'opportunità}}: la megera effettua un attacco ad una creatura che attraversi o esca dalla sua portata di 1 metro.

\emph{\textbf{Infestare Incubi (1/Giorno).}} Mentre si trova sul Piano Etereo, la megera entra magicamente in contatto con un umanoide addormentato che si trova sul Piano Materiale. L'incantesimo \emph{\hyperlink{Cerchio Magico}{Cerchio Magico}} lanciato sul bersaglio previene questo contatto. Finché il contatto persiste, il bersaglio soffre di orribili visioni. Se queste visioni durano per almeno 1 ora, il bersaglio non ottiene benefici dal suo riposo e i suoi Punti Ferita massimi sono ridotti di 5 (1d10). Se questo effetto riduce i Punti Ferita massimi del bersaglio a 0, il bersaglio muore, e se il bersaglio era malvagio, la sua anima resta intrappolata nella \emph{borsa delle anime} della megera. La riduzione dei Punti Ferita massimi del bersaglio rimane finché non viene rimossa dall'incantesimo \emph{\hyperlink{Ristorare Superiore}{Ristorare Superiore}} o simile magia.

\emph{\textbf{Mutare Forma.}} La megera può trasformarsi magicamente in una femmina umanoide di taglia Piccola o Media, o tornare alla sua vera forma. Le sue statistiche sono le stesse in qualsiasi forma. Tutto l'equipaggiamento che stava trasportando o indossando non viene trasformato. Alla morte ritorna alla sua vera forma.

\mostro{Megera Verde}
\noindent
\begin{description}[noitemsep, topsep=0pt, parsep=0pt, partopsep=0pt, leftmargin=0cm, labelwidth=2.2cm]
	\item[\textbf{Taglia/Tipo:}] Media fatato, malvagio
	\item[\textbf{Caratt.:}] \resizebox{0.5\linewidth+1.8cm}{!}{For 4 Des 1 Cos 3 Int 1 Sag 2 Car 2}
	\item[\textbf{Punti Ferita:}] 70,  \textbf{Difesa:} 17,  \textbf{Iniziativa:} +1
	\item[\textbf{Movimento:}] 9 m
	\item[\textbf{Tiri Salvez.:}] \resizebox{0.5\linewidth+1.8cm}{!}{Tempra +6, Riflessi +4, Volontà +5}
	\item[\textbf{Comp.:}] Arcana +3, Furtività +3, Ingannare +4
	\item[\textbf{Sensi:}] Scurovisione 18 m
	\item[\textbf{Linguaggi:}] Comune, Draconico, Silvano
	\item[\textbf{Sfida:}] 3 (700 PX)\smallskip
\end{description}

\emph{\textbf{Anfibio.}} La megera può respirare aria e acqua.

\emph{\textbf{Imitazione.}} La megera può imitare suoni animali e voci umanoidi. Una creatura che senta questi rumori può determinare che si tratti di un'imitazione riuscendo una prova di Consapevolezza DC 14.

\emph{\textbf{Incantesimi Innati.}} La caratteristica da incantatore innato della megera è il Carisma (DC 13 per i Tiri Salvezza degli incantesimi). La megera può lanciare in maniera innata i seguenti incantesimi, senza aver bisogno di componenti materiali.

A volontà: \emph{\hyperlink{Illusione Minore}{Illusione Minore}, \hyperlink{Luci Danzanti}{Luci Danzanti}, \hyperlink{Beffa Crudele}{Beffa Crudele}}

\textbf{Azioni}

\emph{\textbf{Artigli.} Attacco con arma da mischia}: +6 a colpire, portata 1 m, un bersaglio.

\emph{Colpisce:} 13 (2d8 + 4) danni taglienti, 1 danno da Sanguinamento.

\emph{\textbf{Aspetto Illusorio.}} La megera ricopre sé stessa e tutto quello che sta indossando o trasportando in un'illusione magica che le dona l'aspetto di un'altra creatura all'incirca della stessa taglia e forma umanoide. L'illusione termina se la megera effettua una Reazione per terminarla o se muore.

I cambiamenti apportati da questo effetto non sono in grado di superare le ispezioni fisiche. Ad esempio, la megera potrebbe apparire come una creatura dalla pelle liscia, ma il contatto rivelerebbe la sua pelle ruvida. Altrimenti, una creatura deve effettuare un'Azione per ispezionare visivamente l'illusione e riuscire una prova di Consapevolezza DC 20 per comprendere che si tratta di una megera camuffata.

\emph{\textbf{Passaggio Invisibile.}} La megera può rendersi invisibile finché non attacca o lancia un incantesimo, o finché non termina la concentrazione (come se si stesse concentrando su di un incantesimo). Mentre è invisibile, non lascia traccia fisica del suo passaggio, quindi le sue tracce possono essere seguite solo dalla magia. Tutto l'equipaggiamento che sta trasportando o indossando diventa invisibile assieme a lei.

\textbf{Ecologia}
Ambiente: Paludi temperate\\
Organizzazione: Solitario o congrega (3 megere di qualsiasi tipo)\\
\textbf{Categoria Tesoro}: R (C)\\
\textbf{Descrizione}\\
Terrificanti vecchie rugose che frequentano ripugnanti paludi e foreste intricate, le megere verdi nutrono un odio intenso per tutto ciò che è bello e puro. Facendo uso delle loro svariate capacità illusorie, queste vegliarde si dilettano nell'uccidere gli innocenti, nello sconvolgere gli animi nobili e nell'avvilire i cuori puri. Amano utilizzare Camuffare Se Stesso per assumere le forme di giovani e attraenti ragazze così da sedurre e strappare giovani uomini ai loro affetti e parenti, e corrompere nobili e onesti cittadini con ogni sorta di depravazione e scandalo. Alcune megere verdi preferiscono rivelare la loro reale natura ai loro amati in un momento attentamente architettato per spingere l'uomo alla pazzia per l'orrore e la vergogna. Altre prolungano il loro amoreggiamento e fanno di tutto per rovinare completamente la vita degli uomini da loro sedotti prima di mostrare loro la verità. Infine, i più fortunati di questi sventurati finiscono per essere divorati dalla megera verde loro amante: per gli sfortunati, il destino finale può essere molto peggiore, dato che la crudele fantasia della megera verde è immensa. Una tipica megera verde è alta tra 1,5 e 1,8 metri e pesa poco meno di 80 kg.

\mostro{Ameba Paglierina}
\noindent
\begin{description}[noitemsep, topsep=0pt, parsep=0pt, partopsep=0pt, leftmargin=0cm, labelwidth=2.2cm]
	\item[\textbf{Taglia/Tipo:}] Grande melma, disallineato
	\item[\textbf{Caratt.:}] \resizebox{0.5\linewidth+1.8cm}{!}{For 2 Des -2 Cos 2 Int -4 Sag -2 Car -5}
	\item[\textbf{Punti Ferita:}] 51,  \textbf{Difesa:} 12,  \textbf{Iniziativa:} -2
	\item[\textbf{Movimento:}] 3 m, scalata 3 m
	\item[\textbf{Tiri Salvez.:}] \resizebox{0.5\linewidth+1.8cm}{!}{Tempra +4, Riflessi +3, Volontà +3}
	\item[\textbf{Res. Danni:}] Acido
	\item[\textbf{Imm. Danni:}] Elettricità, tagliente
	\item[\textbf{Immunità:}] accecato, affascinato, assordato, prono, affaticato, spaventato
	\item[\textbf{Sensi:}] Vista Cieca 18 m (cieca oltre questo raggio)
	\item[\textbf{Sfida:}] 2 (450 PX)\smallskip
\end{description}

\emph{\textbf{Amorfo.}} L'ameba può muoversi attraverso uno spazio fino a 3 centimetri di larghezza senza doversi stringere.

\emph{\textbf{Natura di Melma.}} L'ameba non necessita di dormire.

\emph{\textbf{Scalare come Ragno.}} L'ameba può scalare superfici difficili, compreso lo stare a testa in giù sul soffitto, senza bisogno di effettuare una prova di Competenza.

\textbf{Azioni}

\emph{\textbf{Pseudopodo.} Attacco con arma da mischia}: +5 a colpire, portata 1 m, un bersaglio.

\emph{Colpisce:} 9 (2d6 + 2) danni contundenti più 3 (1d6) danni da acido.

\textbf{Reazioni}

\emph{\textbf{Divisione.}} Quando un'ameba Media o più grande subisce danni da elettricità o taglienti, si divide in due nuove amebe che hanno almeno 10 Punti Ferita. Ogni nuova ameba ha un numero di Punti Ferita pari alla metà dell'ameba originale, arrotondati per difetto. Le nuove amebe sono di una taglia più piccola di quella originale.


\begin{center}
	\includegraphics[width=0.43\textwidth]{immagini/Amoeba_proteus.png}
\end{center}

\textbf{Ecologia}
Ambiente: Sotterranei o Paludi Temperati\\
Organizzazione: Solitario\\
\textbf{Categoria Tesoro}: Nessuno\\
\textbf{Descrizione}\\
Le Ameba Paglierina sono masse animate di protoplasma di colore simile ad un repellente amalgama di giallo, arancio e marrone. Quando a riposo, il loro corpo piatto e pulsante è alto circa 15 centimetri e si estende tutto intorno; in movimento, si raccolgono in una forma vagamente sferica e sembrano quasi spostarsi rotolando. I loro corpi malleabili permettono loro di attraversare fessure e buchi molto più piccoli dello spazio che occupano. Le creature che vivono sottoterra spesso sigillano tutte le aperture per difendersi dalle Ameba Paglierina.

L'acido altamente specializzato dell'Ameba Paglierina dissolve solo la carne. Questa scoperta ha portato molti maestri avvelenatori ed alchimisti a cercarne esemplari per studiarli. Da questi esperimenti sono nate diverse armi specifiche ideate per distruggere i corpi. Si racconta dell'esistenza di un veleno ad azione lenta che distrugge ad una ad una le cellule delle creature viventi, il cui segreto è ben conservato dal suo creatore.

Un'antica e dimenticata raccolta di appunti descrive un singolare rituale funebre praticato in terre lontane. Anziché cremare i defunti, i corpi venivano racchiusi in sarcofagi di pietra insieme a un'Ameba Paglierina che ne dissolveva lentamente la carne. Successivamente la gelatina risultante veniva trasferita in un'urna accompagnata da una targa di bronzo recante il nome del defunto. Questo metodo preservava gli oggetti sepolti con il corpo, ridotto in breve tempo a uno scheletro lucente, e si credeva che l'essenza vitale del defunto continuasse ad abitare nella gelatina.

L'Ameba Paglierina sono alte circa 15 centimetri, hanno un diametro che può arrivare a 3 metri e pesano circa 1.300 chili. In combattimento, si raccolgono su loro stesse e producono lunghi pseudopodi umidi per colpire ed afferrare qualunque cosa si muova.

Anche se la tipica Ameba Paglierina ha le statistiche qui presentate, nelle profondità della terra questi predatori possono raggiungere dimensioni mostruose.


\mostro{Cubo Gelatinoso}
\noindent
\begin{description}[noitemsep, topsep=0pt, parsep=0pt, partopsep=0pt, leftmargin=0cm, labelwidth=2.2cm]
	\item[\textbf{Taglia/Tipo:}] Grande melma, disallineato
	\item[\textbf{Caratt.:}] \resizebox{0.5\linewidth+1.8cm}{!}{For 2 Des -4 Cos 5 Int -5 Sag -2 Car -5}
	\item[\textbf{Punti Ferita:}] 53,  \textbf{Difesa:} 10,  \textbf{Iniziativa:} -4
	\item[\textbf{Movimento:}] 5 metri
	\item[\textbf{Tiri Salvez.:}] \resizebox{0.5\linewidth+1.8cm}{!}{Tempra +7, Riflessi +3, Volontà +3}
	\item[\textbf{Imm. Danni:}] armi da taglio non magiche, da danno
	\item[\textbf{Immunità:}] accecato, affascinato, assordato, prono, affaticato, spaventato
	\item[\textbf{Sensi:}] Vista Cieca 18 m (cieco oltre questo raggio)
	\item[\textbf{Sfida:}] 2 (450 PX)\smallskip
\end{description}

\emph{\textbf{Cubo di Melma.}} Il cubo occupa il suo intero spazio. Le altre creature possono entrare nello spazio, ma rimangono vittima del Sommergere del cubo e hanno -1d6 al Tiro Salvezza.

Le creature all'interno del cubo sono visibili ma godono di copertura completa.

Una creatura entro 1 metro dal cubo può effettuare un'Azione per tirare una creatura od oggetto fuori dal cubo. Farlo richiede la riuscita di una prova di Atletica DC 14 e la creatura che effettua il tentativo subisce 10 (3d6) danni da acido.

Il cubo può contenere solo una creatura Grande o un massimo di quattro creature Medie o più piccole alla volta.

\emph{\textbf{Natura di Melma.}} Il cubo non necessita di dormire.

\emph{\textbf{Trasparente.}} Anche quando è in piena vista, è necessario riuscire una prova di Consapevolezza DC 15 per notare un cubo che non si è mosso o non ha attaccato. Una creatura che cerchi di entrare nello spazio del cubo mentre è inconsapevole della sua presenza resta sorpresa dal cubo.

\textbf{Azioni}

\emph{\textbf{Pseudopodo.} Attacco con arma da mischia}: +5 a colpire, portata 1 m, un bersaglio.

\emph{Colpisce:} 10 (3d6) danni da acido.

\emph{\textbf{Sommergere.}} Il cubo si muove fino al massimo del suo movimento. Nel farlo, può entrare nello spazio di una creatura di taglia Grande o più piccola. Ogni volta che il cubo entra nello spazio di una creatura, la creatura deve effettuare un Tiro Salvezza di Riflessi DC 13.

Se il Tiro Salvezza riesce, la creatura può scegliere di essere spinta indietro o di lato di 1 metro. Una creatura che decida di non farsi spingere subisce le conseguenze di un Tiro Salvezza fallito.

Se il Tiro Salvezza fallisce, il cubo entra nello spazio della creatura, che subisce 10 (3d6) danni da acido ed è sommersa. La creatura sommersa non può respirare, è intralciata e subisce 21 (6d6) danni da acido all'inizio del round del cubo. Quando il cubo si muove, la creatura sommersa si muove con esso.

Una creatura sommersa può tentare di fuggire effettuando un'Azione per compiere un Tiro Salvezza Tempra con Forza DC 14. Se la riesce, la creatura sfugge e esce entro uno spazio di sua scelta entro 1 metro dal cubo.

\textbf{Ecologia}
Ambiente: Qualsiasi sotterraneo\\
Organizzazione: Solitario\\
\textbf{Categoria Tesoro}: Accidentale\\
\textbf{Descrizione}\\
Tra i predatori più insoliti e peculiari dei dungeon, i cubi gelatinosi trascorrono la loro esistenza vagabondando senza meta per i cunicoli sotterranei e le oscure caverne, inglobando materiali organici come piante, rifiuti, carogne e anche creature viventi. La materia che il cubo non può digerire, come metalli e pietra, riempie di detriti il volume della creatura, e a volte questa può espellerne una parte dal suo corpo. Spesso il tesoro e gli averi delle vittime passate restano dentro il cubo gelatinoso: immagine spettrale dei loro resti materiali.

Alcuni saggi credono che queste creature si siano evolute delle Melme Grigie. Alcuni esseri usano i cubi gelatinosi come guardiani di dungeon e fortificazioni sotterranee, intrappolando queste immense creature in casse di metallo massiccio e trasportandole con poteri o magie fino al loro posto di guardia finale. Sono dei meccanismi di smaltimento rifiuti particolarmente efficaci; una tribù può intrappolare un cubo gelatinoso in una fossa o un'altra area che non possa scalare usandolo come letamaio o anche trappola mortale, a seconda dell'ingegnosità delle creature che l'hanno catturato.

I cubi gelatinosi in genere hanno uno spigolo di 3 metri e pesano più di 7.500 kg, sebbene alcuni esploratori sotterranei affermino che nel sottosuolo esistano esemplari più grandi. In zone in cui il cibo abbonda, i cubi gelatinosi possono vivere per centinaia, se non migliaia, di anni. Tuttavia, se viene a mancare la materia organica per più di 6 mesi, un cubo gelatinoso comincia a deperire, e le sue pareti iniziano a colare, disfacendosi rapidamente in muco liquido finché l'intero corpo non collassa e scompare completamente.

\mostro{Melma Grigia}
\noindent
\begin{description}[noitemsep, topsep=0pt, parsep=0pt, partopsep=0pt, leftmargin=0cm, labelwidth=2.2cm]
	\item[\textbf{Taglia/Tipo:}] Media melma, disallineato
	\item[\textbf{Caratt.:}] \resizebox{0.5\linewidth+1.8cm}{!}{For 1 Des -2 Cos 3 Int -5 Sag -2 Car -4}
	\item[\textbf{Punti Ferita:}] 24,  \textbf{Difesa:} 10,  \textbf{Iniziativa:} -2
	\item[\textbf{Movimento:}] 3 m, scalata 3 m
	\item[\textbf{Tiri Salvez.:}] \resizebox{0.5\linewidth+1.8cm}{!}{Tempra +3, Riflessi +3, Volontà +3}
	\item[\textbf{Res. Danni:}] Acido, Freddo, Fuoco
	\item[\textbf{Immunità:}] accecato, affascinato, assordato, prono, affaticato, spaventato
	\item[\textbf{Sensi:}] Vista Cieca 18 m (cieca oltre questo raggio)
	\item[\textbf{Sfida:}] 1/2 (100 PX)\smallskip
\end{description}

\emph{\textbf{Amorfo.}} La melma può muoversi attraverso uno spazio fino a centimetri di larghezza senza doversi stringere.

\emph{\textbf{Corrodere Metallo.}} Qualsiasi arma non magica fatta di metallo che colpisca la melma si corrode. Dopo aver inflitto il danno, l'arma subisce una penalità permanente e cumulativa di -1 ai tiri di danno. Se la penalità arriva a -5, l'arma è distrutta. Le munizioni non magiche fatte di metallo che colpiscano la melma si distruggono dopo aver inflitto il danno.

La melma può divorare metallo non magico dello spessore di 5 centimetri in un 1 round.

\emph{\textbf{Falso Aspetto.}} Quando la melma rimane immobile, è indistinguibile da una pozza d'olio o una pietra bagnata.

\emph{\textbf{Natura di Melma.}} La melma non necessita di dormire.

\textbf{Azioni}

\emph{\textbf{Pseudopodo.} Attacco con arma da mischia}: +4 a colpire, portata 1 m, un bersaglio.

\emph{Colpisce:} 4 (1d6 + 1) danni contundenti più 7 (2d6) danni da acido. Se il bersaglio sta indossando un'armatura di metallo questa viene parzialmente dissolta e subisce una penalità permanente e cumulativa di -1 alla Difesa che offre. L'armatura è distrutta se la penalità arriva a -6.

\textbf{Ecologia}\\
Ambiente: Paludi fredde e sotterranei\\
Organizzazione: Solitario\\
\textbf{Categoria Tesoro}: Nessuno\\
\textbf{Descrizione}\\
Strisciando attraverso le fredde paludi e gli acquitrini nebbiosi o, a volte in sotterranei e caverne, le melme grigie consumano ogni sostanza organica che incontrano. Sebbene priva di intelligenza, la melma grigia è una delle creature che dà non pochi problemi per la sua trasparenza. Anche se non può arrampicarsi facilmente sui muri o nuotare, la sua abitudine di nascondersi nel fango spesso lungo le rive paludose o di rimanere immobile in pozze dall'aspetto innocuo sul pavimento grigio di un sotterraneo, la rendono molto difficile da notare e da evitare.

Alcuni saggi credono che le melme grigie siano il risultato di un esperimento alchemico fallito, mentre altri teorizzano che le prime melme grigie siano nate spontaneamente da un pozzo di detriti magici. Naturalmente, queste teorie che non le considerano organismi viventi, bensì il risultato di una sfortunata mistura di fluidi caustici e residui magici, sono derisi da chi vive nelle zone infestate da queste creature, che non hanno una storia di inquinamento magico.

\mostro{Protoplasma Nero}
\noindent
\begin{description}[noitemsep, topsep=0pt, parsep=0pt, partopsep=0pt, leftmargin=0cm, labelwidth=2.2cm]
	\item[\textbf{Taglia/Tipo:}] Grande melma, disallineato
	\item[\textbf{Caratt.:}] \resizebox{0.5\linewidth+1.8cm}{!}{For 3 Des -3 Cos 3 Int -5 Sag -2 Car -5}
	\item[\textbf{Punti Ferita:}] 89,  \textbf{Difesa:} 14,  \textbf{Iniziativa:} -3
	\item[\textbf{Movimento:}] 6 m, scalata 6 m
	\item[\textbf{Tiri Salvez.:}] \resizebox{0.5\linewidth+1.8cm}{!}{Tempra +7, Riflessi +3, Volontà +3}
	\item[\textbf{Imm. Danni:}] Acido, Freddo, Elettricità, tagliente, da critico
	\item[\textbf{Immunità:}] accecato, affascinato, assordato, prono, affaticato, spaventato
	\item[\textbf{Sensi:}] Vista Cieca 18 m (cieco oltre questo raggio)
	\item[\textbf{Sfida:}] 4 (1100 PX)\smallskip
\end{description}

\emph{\textbf{Amorfo.}} Il protoplasma nero può muoversi attraverso uno spazio fino a 3 centimetri di larghezza senza doversi stringere.

\emph{\textbf{Forma Corrosiva.}} Una creatura che entri a contatto col protoplasma nero o lo colpisca con un attacco da mischia mentre si trova entro 1 metro da esso subisce 4 (1d8) danni da acido. Qualsiasi arma non magica fatta di metallo o legno che colpisca il protoplasma nero si corrode. Dopo aver inflitto il danno, l'arma subisce una penalità permanente e cumulativa di -1 ai tiri di danno. Se la penalità arriva a -5, l'arma è distrutta. Le munizioni non magiche fatte di metallo o legno che colpiscano il protoplasma nero si distruggono dopo aver inflitto il danno.

Il protoplasma nero può divorare legno o metallo non magico dello spessore di 5 centimetri in un 1 round.

\emph{\textbf{Natura di Melma.}} Il protoplasma nero non necessita di dormire.

\emph{\textbf{Scalare come Ragno.}} Il protoplasma nero può scalare superfici difficili, compreso lo stare a testa in giù sul soffitto, senza bisogno di effettuare una prova di competenza.

\textbf{Azioni}

\emph{\textbf{Pseudopodo.} Attacco con arma da mischia}: +6 a colpire, portata 1 m, un bersaglio.

\emph{Colpisce:} 6 (1d6 + 3) danni contundenti più 18 (4d8) danni da acido. Inoltre, un'armatura non magica indossata dal bersaglio viene parzialmente dissolta e subisce una penalità permanente e cumulativa di -1 alla Difesa che offre. L'armatura è distrutta se la penalità arriva a -5.

\textbf{Reazioni}

\emph{\textbf{Divisione.}} Quando un protoplasma nero di taglia Media o più grande subisce danni da elettricità o taglienti, si divide in due nuovi protoplasma neri di almeno 10 Punti Ferita ciascuno. Ogni nuovo protoplasma nero ha un numero di Punti Ferita pari alla metà del protoplasma nero originale, arrotondati per difetto. I nuovi protoplasmi neri sono di una taglia più piccola di quella originale.

\textbf{Ecologia}\\
Ambiente: Qualsiasi sotterraneo\\
Organizzazione: Solitario\\
\textbf{Categoria Tesoro}: Nessuno\\
\textbf{Descrizione}\\
I protoplasmi neri sono gli spazzini del mondo sotterraneo, costantemente alla ricerca di cibo. Possono percepire corpi organici o metallici nel raggio di 18 metri e attaccano in modo istintivo tali oggetti o esseri finché non li dissolvono, o finché la melma non viene uccisa. Un protoplasma nero si riproduce staccando un pezzo del proprio corpo e formando un nuovo protoplasma più piccolo che raggiunge l'età adulta nel giro di un mese. Alcune tra le creature più intelligenti nel mondo sotterraneo usano i protoplasmi neri per smaltire in modo naturale la spazzatura, creando cave di pietra atte ad ospitare il protoplasma, per poi gettarvi i rifiuti organici o i nemici.

\mostro{Mimic}
\noindent
\begin{description}[noitemsep, topsep=0pt, parsep=0pt, partopsep=0pt, leftmargin=0cm, labelwidth=2.2cm]
	\item[\textbf{Taglia/Tipo:}] Media mostruosità (mutaforma), neutrale
	\item[\textbf{Caratt.:}] \resizebox{0.5\linewidth+1.8cm}{!}{For 3 Des 1 Cos 2 Int -2 Sag 1 Car -1}
	\item[\textbf{Punti Ferita:}] 51,  \textbf{Difesa:} 15,  \textbf{Iniziativa:} +1
	\item[\textbf{Movimento:}] 5 metri
	\item[\textbf{Tiri Salvez.:}] \resizebox{0.5\linewidth+1.8cm}{!}{Tempra +4, Riflessi +3, Volontà +3}
	\item[\textbf{Comp.:}] Furtività +5
	\item[\textbf{Imm. Danni:}] Acido
	\item[\textbf{Immunità:}] prono
	\item[\textbf{Sensi:}] Scurovisione 18 m
	\item[\textbf{Sfida:}] 2 (450 PX)\smallskip
\end{description}

\emph{\textbf{Aderente (Solo Forma di Oggetto).}} Il mimic aderisce a qualsiasi cosa con cui entri in contatto. Una creatura di taglia Enorme o inferiore a cui il mimic aderisce è considerata afferrata da esso (DC 18 per fuggire). Il mimic non si considera Afferrato quando afferra qualcosa.

\emph{\textbf{Afferratore.}} Il mimic ha +1d6 ai tiri per colpire contro una creatura da esso afferrata.

\emph{\textbf{Falso Aspetto (Solo Forma di Oggetto).}} Mentre il mimic rimane immobile, è indistinguibile da un comune oggetto.

\emph{\textbf{Mutaforma.}} Il mimic può usare una Azione per trasformarsi in un oggetto, o per tornare alla sua vera forma amorfa. Le sue statistiche sono le stesse in qualsiasi forma. Qualsiasi equipaggiamento stia indossando o trasportando non si trasforma. Alla morte ritorna al suo vero aspetto.

\medskip

\begin{center}
	\includegraphics[width=0.9\linewidth]{immagini/mimic_grayscale.png}

	\emph{Mimic, la curiosità può fare male...}
\end{center}

\medskip

\textbf{Azioni}

\emph{\textbf{Morso.} Attacco con arma da mischia}: +6 a colpire, portata 1 m, un bersaglio.

\emph{Colpisce:} 7 (1d8 + 3) danni perforanti più 4 (1d8) danni da acido.

\emph{\textbf{Pseudopodo.} Attacco con arma da mischia}: +5 a colpire, portata 1 m, un bersaglio.

\emph{Colpisce:} 7 (1d8 + 3) danni contundenti. Se il mimic è in forma di oggetto, il bersaglio è vittima del tratto Aderente.

\textbf{Ecologia}
Ambiente: Qualsiasi\\
Organizzazione: Solitario\\
\textbf{Categoria Tesoro}: Accidentale (A)\\
\textbf{Descrizione}\\
Si ritiene che i mimic siano il risultato del tentativo di un alchimista di dar vita ad un oggetto inanimato attraverso l'applicazione di un reagente mistico, la cui formula è andata perduta. Nel corso degli anni, queste creature strane ma intelligenti hanno appreso la capacità di trasformarsi in simulacri degli oggetti manufatti, in particolare nei luoghi frequentati poco da un ristretto numero di creature, dove aumentano le loro probabilità di successo con un attacco alle loro vittime.

Anche se i mimic non sono intrinsecamente malvagi, alcuni saggi suggeriscono che attacchino gli uomini e le altre creature intelligenti più per passatempo che per sfamarsi. Il desiderio di ingannare gli altri è parte del loro essere e i loro attacchi a sorpresa rappresentano il culmine di questo desiderio.

Un tipico mimic ha un volume di 2 metri cubi (1 m per 1 m per 2 m) e pesa circa 450 kg. Leggende e storie parlano di mimic di taglie maggiori, con la capacità di assumere la forma di case, navi o interi complessi sotterranei che guarniscono con dei tesori (sia veri che falsi) per attirare al loro interno il loro ignaro cibo.

\mostro{Minotauro}
\noindent
\begin{description}[noitemsep, topsep=0pt, parsep=0pt, partopsep=0pt, leftmargin=0cm, labelwidth=2.2cm]
	\item[\textbf{Taglia/Tipo:}] Grande mostruosità, malvagio
	\item[\textbf{Caratt.:}] \resizebox{0.5\linewidth+1.8cm}{!}{For 4 Des 0 Cos 3 Int -2 Sag 3 Car -1}
	\item[\textbf{Punti Ferita:}] 70,  \textbf{Difesa:} 16,  \textbf{Iniziativa:} +0
	\item[\textbf{Movimento:}] 12 m
	\item[\textbf{Tiri Salvez.:}] \resizebox{0.5\linewidth+1.8cm}{!}{Tempra +6, Riflessi +3, Volontà +6}
	\item[\textbf{Comp.:}] Consapevolezza +7
	\item[\textbf{Sensi:}] Scurovisione 18 m
	\item[\textbf{Linguaggi:}] Abissale
	\item[\textbf{Sfida:}] 3 (700 PX)\smallskip
\end{description}

\emph{\textbf{Carica.}} Se il minotauro si muove di almeno 3 metri diretto verso un bersaglio e lo colpisce con un attacco di incornata durante lo stesso round, il bersaglio subisce 9 (2d8) danni perforanti aggiuntivi. Se il bersaglio è una creatura, deve riuscire un Tiro Salvezza su Tempra DC 15 o venire spinto via fino a 3 metri di distanza e cadere prono. 1 Azione.

\emph{\textbf{Incauto.}} All'inizio del suo round, il minotauro può ottenere +1d6 su tutti i tiri per colpire con armi da mischia effettuati durante quel round, ma i tiri per colpire contro di esso hanno +1d6 fino all'inizio del suo prossimo round.

\emph{\textbf{Ricordare Labirinto.}} Il minotauro può ricordare perfettamente qualsiasi tragitto abbia percorso.

\textbf{Azioni}

\emph{\textbf{Ascia Bipenne.} Attacco con arma da mischia}: +6 a colpire, portata 1 m, un bersaglio.

\emph{Colpisce:} 17 (2d12 + 4) danni taglienti.

\emph{\textbf{Incornata.} Attacco con arma da mischia}: +6 a colpire, portata 1 m, un bersaglio.

\emph{Colpisce:} 13 (2d8 + 4) danni perforanti.

\textbf{Ecologia}\\
Ambiente: Rovine Temperate e Sotterranei\\
Organizzazione: Solitario, coppia o gruppo (3-4)\\
\textbf{Categoria Tesoro}: Ascia Bipenne, O +1 pozione\\
\textbf{Descrizione}\\
Disprezzati dalle razze civilizzate e creati secoli fa da una maledizione divina, i minotauri cacciano, uccidono e divorano gli umanoidi per punire offese vere o presunte, da tempi immemorabili. La maggior parte delle culture ha leggende su come furono creati da divinità vendicative che punirono gli umani deformando le loro sembianze, togliendo loro bellezza e intelligenza, e dotandoli di teste di toro. Tuttavia, i minotauri moderni disprezzano queste leggende, ritenendosi modelli di perfezione divina creati dal signore dei demoni Baphomet.

I nascondigli tradizionali dei minotauri sono i labirinti, sia costruiti sia naturali. Usano la loro astuzia per scoraggiare i nemici incauti che si perdono nei loro nascondigli. Solo quando la disperazione ha preso il sopravvento, il minotauro colpisce le sue vittime. Spesso lasciano scappare una creatura per diffondere il terrore e attirare altri nel loro labirinto, che considerano deliziosi pasti.

I minotauri possono servire mostri o creature malvagie più potenti, cacciando e mangiando a loro piacimento. Possono fare la guardia a potenti oggetti o preziose locazioni, o lavorare come mercenari, cacciando i nemici del padrone.

I minotauri sono combattenti diretti, usando le loro corna per incornare orribilmente le creature vicine all'inizio del combattimento.

\mostro{Mummia}
\noindent
\begin{description}[noitemsep, topsep=0pt, parsep=0pt, partopsep=0pt, leftmargin=0cm, labelwidth=2.2cm]
	\item[\textbf{Taglia/Tipo:}] Media non morto, malvagio
	\item[\textbf{Caratt.:}] \resizebox{0.5\linewidth+1.8cm}{!}{For 3 Des -1 Cos 2 Int -2 Sag 0 Car 1}
	\item[\textbf{Punti Ferita:}] 70,  \textbf{Difesa:} 15,  \textbf{Iniziativa:} -1
	\item[\textbf{Movimento:}] 6 m
	\item[\textbf{Tiri Salvez.:}] \resizebox{0.5\linewidth+1.8cm}{!}{Tempra +5, Riflessi +3, Volontà +3}
	\item[\textbf{Res. Danni:}] da arma non magica
	\item[\textbf{Imm. Danni:}] da Vuoto, Veleno
	\item[\textbf{Immunità:}] affascinato, paralizzato, affaticato, spaventato, sanguinamento
	\item[\textbf{Sensi:}] Scurovisione 18 m
	\item[\textbf{Linguaggi:}] le lingue che conosceva in vita
	\item[\textbf{Sfida:}] 3 (700 PX)\smallskip
\end{description}

\emph{\textbf{Natura Non Morta.}} Un mummia non ha bisogno di aria, cibo, bevande o sonno.

\textbf{Azioni}

\emph{\textbf{Multiattacco.}} La mummia può usare la sua Occhiata Temibile ed effettuare un attacco con il pugno putrefacente.

\emph{\textbf{Pugno Putrefacente.} Attacco con arma da mischia}: +6 a colpire, portata 1 m, un bersaglio.

\emph{Colpisce:} 10 (2d6 + 3) danni contundenti più 10 (3d6) danni da Vuoto. Se il bersaglio è una creatura deve riuscire un Tiro Salvezza su Tempra DC 15 o venire maledetto dalla putrefazione della mummia. Il bersaglio maledetto non può recuperare Punti Ferita e i suoi Punti Ferita massimi diminuiscono di 10 (3d6) ogni 24 ore di durata della maledizione. Se la maledizione riduce i Punti Ferita massimi del bersaglio a 0, il bersaglio muore e il suo corpo si tramuta in polvere. La maledizione dura finché non viene rimossa dall'incantesimo \emph{\hyperlink{Rimuovi Maledizione}{Rimuovi Maledizione}} o altra magia.

\emph{\textbf{Occhiata Temibile.}} La mummia prende a bersaglio una creatura che possa vedere e si trovi entro 18 metri da lei. Se il bersaglio può vedere la mummia deve riuscire un Tiro Salvezza su Volontà DC 15 contro questa magia o restare spaventato fino al termine del prossimo round della mummia. Se il bersaglio fallisce il Tiro Salvezza in maniera critica è anche paralizzato per la stessa durata. Un bersaglio che riesca il Tiro Salvezza è immune all'Occhiata Terribile di tutte le mummie (ma non delle mummie sovrane) per le successive 24 ore.

\mostro{Mummia Sovrana}
\noindent
\begin{description}[noitemsep, topsep=0pt, parsep=0pt, partopsep=0pt, leftmargin=0cm, labelwidth=2.2cm]
	\item[\textbf{Taglia/Tipo:}] Media non morto, malvagio
	\item[\textbf{Caratt.:}] \resizebox{0.5\linewidth+1.8cm}{!}{For 4 Des 0 Cos 3 Int 0 Sag 4 Car 3}
	\item[\textbf{Punti Ferita:}] 294,  \textbf{Difesa:} 32,  \textbf{Iniziativa:} +0
	\item[\textbf{Movimento:}] 6 m
	\item[\textbf{Tiri Salvez.:}] \resizebox{0.5\linewidth+1.8cm}{!}{\resizebox{0.5\linewidth+1.8cm}{!}{Tempra +18, Riflessi +15, Volontà +19}}
	\item[\textbf{Comp.:}] Religione +5, Storia +5
	\item[\textbf{Imm. Danni:}] da Vuoto, Veleno; armi +1
	\item[\textbf{Immunità:}] affascinato, paralizzato, affaticato, spaventato
	\item[\textbf{Sensi:}] Scurovisione 18 m
	\item[\textbf{Linguaggi:}] le lingue che conosceva in vita
	\item[\textbf{Sfida:}] 15 (13000 PX)\smallskip
\end{description}

\emph{\textbf{Cuore della Mummia Sovrana.}} Come parte del rituale che crea una mummia sovrana, il cuore e le viscere della creatura vengono rimossi dal cadavere e piazzati all'interno di contenitori sigillati. Questi contenitori sono di solito fatti in pietra o ceramica, incisi o dipinti con geroglifici religiosi.

Finché il suo cuore avvizzito rimane intatto, la mummia sovrana non può essere permanentemente distrutta. Quando scende a 0 Punti Ferita, la mummia sovrana si riduce in polvere e si riforma a piena forza 24 ore più tardi riemergendo dalla polvere in prossimità della giara sigillata che contiene il suo cuore. Per impedire che una mummia sovrana si riformi e distruggerla una volta per tutte bisogna ridurne il cuore in cenere. Per questo motivo, la mummia sovrana di solito tiene il cuore e le viscere nascoste all'interno di una tomba nascosta.

Il cuore della mummia sovrana ha Difesa 5, 25 Punti Ferita e immunità a tutti i danni eccetto Luce.

\emph{\textbf{Incantesimi.}} La mummia ha CM 10. La sua caratteristica da incantatore è la Saggezza. La mummia ha preparati i seguenti incantesimi: Trucchetti (a volontà): \emph{\hyperlink{Fiamma Sacra}{Fiamma Sacra}, \hyperlink{Taumaturgia}{Taumaturgia}}

livello 1 (4 slot): \emph{\hyperlink{Comando}{Comando}, \hyperlink{Dardo Tracciante}{Dardo Tracciante}}

livello 2 (3 slot): \emph{\hyperlink{Arma Spirituale}{Arma Spirituale}, \hyperlink{Blocca Persona}{Blocca Persona}, \hyperlink{Silenzio}{Silenzio}}

livello 3 (3 slot): \emph{\hyperlink{Animare Morti}{Animare Morti}, \hyperlink{Dissolvi Magie}{Dissolvi Magie}}

livello 4 (3 slot): \emph{\hyperlink{Divinazione}{Divinazione}}

livello 5 (2 slot): \emph{\hyperlink{Contagio}{Contagio}, \hyperlink{Piaga degli Insetti}{Piaga degli Insetti}}

livello 6 (1 slot): \emph{\hyperlink{Ferire}{Ferire}}

\emph{\textbf{Natura Non Morta.}} Un mummia non ha bisogno di aria, cibo, bevande o sonno.

\emph{\textbf{Resistenza alla Magia.}} La mummia sovrana ha +1d6 ai Tiri Salvezza contro incantesimi o altri effetti magici.

\emph{\textbf{Rinvigorimento.}} Una mummia sovrana forma un nuovo corpo entro 24 ore se il suo cuore resta intatto, recuperando tutti i Punti Ferita e potendo agire nuovamente. Il nuovo corpo compare entro 1 metro dal cuore della mummia sovrana.

\textbf{Azioni}

\emph{\textbf{Multiattacco.}} La mummia può usare la sua Occhiata Temibile ed effettuare un attacco con il pugno putrefacente, oppure 2 Pugni Putrefacente.

\emph{\textbf{Pugno Putrefacente.} Attacco con arma da mischia}: +13 a colpire, portata 1 m, un bersaglio.

\emph{Colpisce:} 14 (3d6 + 4) danni contundenti più 21 (6d6) danni da Vuoto. Se il bersaglio è una creatura deve riuscire un Tiro Salvezza su Tempra 28 o venire maledetto dalla putrefazione della mummia. Il bersaglio maledetto non può recuperare Punti Ferita, e i suoi Punti Ferita massimi diminuiscono di 10 (3d6) ogni 24 ore di durata della maledizione. Se la maledizione riduce i Punti Ferita massimi del bersaglio a 0, il bersaglio muore, e il suo corpo si tramuta in polvere. La maledizione dura finché non viene rimossa dall'incantesimo \emph{\hyperlink{Rimuovi Maledizione}{Rimuovi Maledizione}} o altra magia.

\emph{\textbf{Occhiata Temibile.}} La mummia prende a bersaglio una creatura che possa vedere e si trovi entro 18 metri da lei. Se il bersaglio può vedere la mummia, deve riuscire un Tiro Salvezza su Volontà DC 28 contro questa magia o restare spaventato fino al termine del prossimo round della mummia. Se il bersaglio fallisce il Tiro Salvezza in maniera critica è anche paralizzato per la stessa durata. Un bersaglio che riesca il Tiro Salvezza è immune all'Occhiata Terribile di tutte le mummie (ma non delle mummie sovrane) per le successive 24 ore.

\textbf{Reazione: \emph{Attacco d'opportunità}}: la mummia sovrana effettua un pugno putrefacente ad una creatura che attraversi o esca dalla sua portata di 1 metro.

\textbf{Azioni Aggiuntive}

La mummia sovrana può effettuare 3 Azioni aggiuntive, scelte tra le opzioni seguenti. Può usare solo un'opzione Aggiuntiva alla volta e solo al termine del round di un'altra creatura. La mummia sovrana recupera le Azioni aggiuntive spese all'inizio del proprio round.

\emph{\textbf{Attaccare.}} La mummia sovrana effettua un attacco con il pugno putrefacente o usa la sua Occhiata Temibile.

\emph{\textbf{Incanalare Energia Negativa (Costa 2 Azioni).}} La mummia sovrana può scatenare magicamente l'energia negativa. Le creature entro 18 metri dalla mummia sovrana, comprese quelle dietro barriere o angoli, non possono recuperare Punti Ferita fino al termine del prossimo round della mummia sovrana.

\emph{\textbf{Parola Blasfema (Costa 2 Azioni).}} La mummia sovrana pronuncia una parola blasfema. Ciascuna creatura, esclusi i non morti, entro 3 metri dalla mummia sovrana e che possa udire questa frase magica deve riuscire un Tiro Salvezza di Tempra DC 28 o restare stordita fino al termine del prossimo round della mummia sovrana.

\emph{\textbf{Polvere Accecante.}} Polvere e sabbia accecanti turbinano magicamente intorno alla mummia sovrana. Ogni creatura entro 1 metro dalla mummia sovrana deve riuscire un Tiro Salvezza di Tempra DC 28 o restare accecata fino al termine del prossimo round della creatura.

\emph{\textbf{Turbine di Sabbia (Costa 2 Azioni).}} La mummia sovrana può trasformarsi magicamente in un turbine di sabbia, muovendosi di massimo 18 metri, e tornando poi alla sua forma normale. Mentre è in forma di turbine, la mummia sovrana è immune a tutti i danni, e non può essere afferrata, pietrificata, gettata prona, intralciata o stordita. L'equipaggiamento indossato o trasportato dalla mummia sovrana rimane in suo possesso.

\emph{\textbf{Arrabbiato:}} La Mummia sovrana ha fame di vita. Incanala l'energia della morte e distruzione in un raggio di 12 metri intorno a se. Ogni creatura deve superare un Tiro Salvezza su Tempra a DC 26 per dimezzare o subire 22 di danno. La Mummia recupera tutti i Punti Ferita persi dalle altre creature.

\textbf{Ecologia}
Ambiente: Qualsiasi\\
Organizzazione: Solitario, Gruppo (3-6) o Mausoleo (7-12)\\
\textbf{Categoria Tesoro}: T + U\\
\textbf{Descrizione}\\
Molte culture praticano l'arte sacra della mummificazione, anche se le sinistre tecniche magiche utilizzate per infondere ai cadaveri la vitalità dei non morti sono molto meno comuni. In alcune terre antiche, tali tecniche blasfeme sono state affinate attraverso secoli di cerimonie e innumerevoli morti, risultando in mummie di terribile potere. In rare occasioni, se il defunto era di rango elevato ed eccessiva malvagità, poteva sottoporsi a rituali così elaborati, risorgendo dalla tomba come un temibile signore mummia. Allo stesso modo, un sovrano noto per la sua malizia o morto in un momento di grande rabbia potrebbe presentarsi spontaneamente come un despota vendicativo. Indipendentemente dalle circostanze esatte della sua risurrezione, una mummia sovrana conserva le capacità che aveva in vita, diventando una creatura consumata dal desiderio di ripristinare il suo dominio e governare sia i vivi che i morti.

%\addcontentsline{toc}{subsubsection}{N}
\pdfbookmark[3]{N}{N}

\mostro{Naga Guardiano}
\noindent
\begin{description}[noitemsep, topsep=0pt, parsep=0pt, partopsep=0pt, leftmargin=0cm, labelwidth=2.2cm]
	\item[\textbf{Taglia/Tipo:}] Grande mostruosità, buono
	\item[\textbf{Caratt.:}] \resizebox{0.5\linewidth+1.8cm}{!}{For 4 Des 4 Cos 3 Int 3 Sag 4 Car 4}
	\item[\textbf{Punti Ferita:}] 201,  \textbf{Difesa:} 29,  \textbf{Iniziativa:} +4
	\item[\textbf{Movimento:}] 12 m
	\item[\textbf{Tiri Salvez.:}] \resizebox{0.5\linewidth+1.8cm}{!}{\resizebox{0.5\linewidth+1.8cm}{!}{Tempra +13, Riflessi +14, Volontà +14}}
	\item[\textbf{Imm. Danni:}] Veleno
	\item[\textbf{Immunità:}] affascinato
	\item[\textbf{Sensi:}] Scurovisione 18 m
	\item[\textbf{Linguaggi:}] Celestiale, Comune
	\item[\textbf{Sfida:}] 10 (5900 PX)\smallskip
\end{description}

\emph{\textbf{Incantesimi.}} Il naga ha CM 11. La sua caratteristica da incantatore è la Saggezza (+8 a colpire con attacchi con incantesimo), e ha bisogno solo delle componenti verbali per lanciare i suoi incantesimi. Il naga prepara i seguenti incantesimi:

Trucchetti (a volontà): \emph{\hyperlink{Fiamma Sacra}{Fiamma Sacra}, \hyperlink{Riparare}{Riparare}, \hyperlink{Taumaturgia}{Taumaturgia}}

livello 1 (4 slot): \emph{\hyperlink{Comando}{Comando}, \hyperlink{Cura Ferite}{Cura Ferite}}

livello 2 (3 slot): \emph{\hyperlink{Blocca Persona}{Blocca Persona}, \hyperlink{Calmare Emozioni}{Calmare Emozioni}}

livello 3 (3 slot): \emph{\hyperlink{Chiaroveggenza}{Chiaroveggenza}, \hyperlink{Scagliare Maledizione}{Scagliare Maledizione}}

livello 4 (3 slot): \emph{\hyperlink{Esilio}{Esilio}, \hyperlink{Libertà di Movimento}{Libertà di Movimento}}

livello 5 (2 slot): \emph{\hyperlink{Colpo Infuocato}{Colpo Infuocato}, \hyperlink{Costrizione}{Costrizione}}

livello 6 (1 slot): \emph{\hyperlink{Visione del Vero}{Visione del Vero}}

\emph{\textbf{Rinvigorimento.}} Se muore il naga ritorna in vita in 1d6 giorni e recupera tutti i suoi Punti Ferita. Solo l'incantesimo \emph{\hyperlink{Desiderio}{Desiderio}} può impedire a questo tratto di funzionare.

\textbf{Azioni}

\emph{\textbf{Morso.} Attacco con arma da mischia}: +11 a colpire, portata 3 m, una creatura.

\emph{Colpisce:} 8 (1d8 + 4) danni perforanti, e il bersaglio deve effettuare un Tiro Salvezza di Tempra DC 23, subendo 45 (10d8) danni da veleno se fallisce il Tiro Salvezza, o la metà di questi danni se lo riesce.

\emph{\textbf{Sputare Veleno.} Attacco con arma a Distanza}: +11 a colpire, gittata 5m, una creatura.

\emph{Colpisce:} Il bersaglio deve effettuare un Tiro Salvezza su Tempra DC 23, subendo 45 (10d8) danni da veleno se fallisce il Tiro Salvezza, o la metà di questi danni se lo riesce.

\textbf{Reazione: \emph{Attacco d'opportunità}}: il naga effettua un attacco di sputo ad una creatura che attraversi o esca dalla sua portata di 3 metri.

\textbf{Ecologia}\\
Ambiente: Pianure Temperate\\
Organizzazione: Solitario, coppia o nido (3-6)\\
\textbf{Categoria Tesoro}: R\\
\textbf{Descrizione}\\
Sebbene abbiano un aspetto feroce, con scaglie brillanti, cappucci simili a quelli dei cobra e potenti corpi serpentini, i naga guardiani fungono da coscienziosi protettori di luoghi di eccezionale potere e sacralità. Spesso le loro scaglie sfoggiano disegni elaborati simili a quelli degli esotici serpenti della giungla. Un tipico naga guardiano raggiunge la lunghezza di 4,2 metri e un peso approssimativo di 175 kg.

Mentre alcuni naga guardiani aderiscono a pratiche esotiche di divinità antiche o dimenticate, altri sono semplicemente attratti da siti dalla spiccata bellezza naturale, quali templi su imponenti cascate, pinnacoli naturali e cime di montagne, custodendoli con il massimo della reverenza e del senso del dovere. Spesso questi naga si uniscono a fedi ancora attive, servendo come protettori di santuari o antichi tesori. Una coppia di naga può stabilirsi nei pressi di un sito che ritengono meritevole di protezione, covandovi una nidiata e crescendovi la prole. Quando i giovani raggiungono l'età adulta, possono scegliere di partire per cercare la propria casa o rimanere a proteggere la zona sorvegliata dai loro genitori. A volte, un naga guardiano che custodisce delle rovine od un tempio è solo l'ultimo di una successione di sentinelle che si sono avvicendate nel corso dei secoli. Queste sentinelle spesso prendono lo stesso nome dei loro predecessori sembrando un unico individuo eccezionalmente longevo.

\mostro{Naga Spirituale}
\noindent
\begin{description}[noitemsep, topsep=0pt, parsep=0pt, partopsep=0pt, leftmargin=0cm, labelwidth=2.2cm]
	\item[\textbf{Taglia/Tipo:}] Grande mostruosità, malvagio
	\item[\textbf{Caratt.:}] \resizebox{0.5\linewidth+1.8cm}{!}{For 4 Des 3 Cos 2 Int 3 Sag 2 Car 3}
	\item[\textbf{Punti Ferita:}] 162,  \textbf{Difesa:} 25,  \textbf{Iniziativa:} +3
	\item[\textbf{Movimento:}] 12 m
	\item[\textbf{Tiri Salvez.:}] \resizebox{0.5\linewidth+1.8cm}{!}{\resizebox{0.5\linewidth+1.8cm}{!}{Tempra +10, Riflessi +11, Volontà +10}}
	\item[\textbf{Imm. Danni:}] Veleno
	\item[\textbf{Immunità:}] affascinato
	\item[\textbf{Sensi:}] Scurovisione 18 m
	\item[\textbf{Linguaggi:}] Abissale, Comune
	\item[\textbf{Sfida:}] 8 (3900 PX)\smallskip
\end{description}

\emph{\textbf{Incantesimi.}} Il naga ha CM 10. La sua abilità da incantatore è l'Intelligenza (+6 a colpire con attacchi con incantesimo), e ha bisogno solo delle componenti verbali per eseguire i suoi incantesimi. Il naga prepara i seguenti incantesimi:

Trucchetti (a volontà): \emph{\hyperlink{Illusione Minore}{Illusione Minore}, \hyperlink{Mano Magica}{Mano Magica}, \hyperlink{Raggio di Gelo}{Raggio di Gelo}}

livello 1 (4 slot): \emph{\hyperlink{Charme su Persone}{Charme su Persone}, \hyperlink{Individuazione del Magico}{Individuazione del Magico}, \hyperlink{Sonno}{Sonno}}

livello 2 (3 slot): \emph{\hyperlink{Blocca Persona}{Blocca Persona}, \hyperlink{Individuazione dei Pensieri}{Individuazione dei Pensieri}}

livello 3 (3 slot): \emph{\hyperlink{Fulmine}{Fulmine}, \hyperlink{Respirare Sott'Acqua}{Respirare Sott'Acqua}}

livello 4 (3 slot): \emph{\hyperlink{Inaridire}{Inaridire}, \hyperlink{Porta Dimensionale}{Porta Dimensionale}}

livello 5 (2 slot): \emph{\hyperlink{Dominare Persone}{Dominare Persone}}

\emph{\textbf{Rinvigorimento.}} Se muore, il naga ritorna in vita in 1d6 giorni e recupera tutti i suoi Punti Ferita. Solo l'incantesimo \emph{\hyperlink{Desiderio}{Desiderio}} può impedire a questo tratto di funzionare.

\textbf{Azioni}

\emph{\textbf{Morso.} Attacco con arma da mischia}: +9 a colpire, portata 3 m, una creatura.

\emph{Colpisce:} 7 (1d8 + 4) danni perforanti, e il bersaglio deve effettuare un Tiro Salvezza di Tempra DC 20, subendo 31 (7d8) danni da veleno se fallisce il Tiro Salvezza, o la metà di questi danni se lo riesce.

\textbf{Reazione: \emph{Attacco d'opportunità}}: il naga effettua un attacco di sputo ad una creatura che attraversi o esca dalla sua portata di 3 metri.

\mostro{Nano Oscuro}
\noindent
\begin{description}[noitemsep, topsep=0pt, parsep=0pt, partopsep=0pt, leftmargin=0cm, labelwidth=2.2cm]
	\item[\textbf{Taglia/Tipo:}] Media umanoide (nano), malvagio
	\item[\textbf{Caratt.:}] \resizebox{0.5\linewidth+1.8cm}{!}{For 2 Des 0 Cos 2 Int 0 Sag 0 Car -1}
	\item[\textbf{Punti Ferita:}] 33,  \textbf{Difesa:} 13,  \textbf{Iniziativa:} +0
	\item[\textbf{Movimento:}] 8 m
	\item[\textbf{Tiri Salvez.:}] \resizebox{0.5\linewidth+1.8cm}{!}{Tempra +3, Riflessi +3, Volontà +3}
	\item[\textbf{Sensi:}] Scurovisione 36 m
	\item[\textbf{Linguaggi:}] Nanico, Linguaggio delle Profondità
	\item[\textbf{Sfida:}] 1 (200 PX)\smallskip
\end{description}

\emph{\textbf{Resilienza Oscura.}} Il Nano Oscuro ha +1d6 ai Tiri Salvezza contro veleni, incantesimi e illusioni, oltre al resistere al restare affascinato o paralizzato.

\emph{\textbf{Sensibilità alla Luce}}. Mentre è alla luce del sole, il Nano Oscuro ha -1d6 ai tiri di attacco, oltre che alle prove di Consapevolezza basate sulla vista.

\textbf{Azioni}

\emph{\textbf{Ingrandire (Ricarica dopo un 1 ora).}} Per 1 minuto, il Nano Oscuro aumenta magicamente di taglia, insieme a tutto ciò che sta trasportando o indossando. Mentre è ingrandito, il Nano Oscuro è di taglia Grande, raddoppia i dadi di danno degli attacchi con armi basate sulla Forza (già incluso negli attacchi), e ha +1d6 alle prove di Forza e ai Tiri Salvezza di Forza. Se il Nano Oscuro non ha sufficiente spazio per diventare Grande, ottiene la massima taglia concessa dallo spazio a disposizione.

\emph{\textbf{Piccone da Guerra.} Attacco con arma da mischia}: +5 a colpire, portata 1 m, un bersaglio.

\emph{Colpisce:} 6 (1d8 + 2) danni perforanti, o 11 (2d8 + 2) danni perforanti quando ingrandito.

\emph{\textbf{Giavellotto.} Attacco con arma da mischia o a Distanza}: +5 a colpire, portata 1 m o gittata 12m, un bersaglio.

\emph{Colpisce:} 5 (1d6 + 2) danni perforanti o 9 (2d6 + 2) danni
perforanti quando ingrandito.

\emph{\textbf{Invisibilità (Ricarica dopo un 1 ora).}} Il Nano Oscuro diventa magicamente invisibile al massimo per un'ora o finché non attacca, lancia un incantesimo, usa Ingrandire o la sua concentrazione viene spezzata. Tutto l'equipaggiamento che il Nano Oscuro indossa o trasporta diventa invisibile assieme a lui.

\textbf{Ecologia}\\
Ambiente: Qualsiasi sotterraneo\\
Organizzazione: solitario, gruppo (2-5), squadra (6-12 più 3 sergenti di 3° livello e 1 capo di 3°-8° livello), o clan (13-80 più 25\% di bambini non combattenti più 1 sergente di 3° livello ogni 5 adulti, 3-6 tenenti di 3°-6° livello, e 1-4 capitani di 9° livello)\\
\textbf{Categoria Tesoro}: equipaggiamento da PNG (Cotta di Maglia, Scudo Pesante di Metallo, Martello da Guerra, Balestra Leggera con 20 Quadrelli, 3d6 mo)\\
\textbf{Descrizione}\\
Lontani parenti dei Nani, più cupi e deformi, i Nani Oscuro sono creature dal pessimo carattere che odiano gli intrusi nei loro reami sotterranei, ma mai più dei Nani. Vivono in comunità nelle profondità del sottosuolo. Hanno pelle grigio opaco, come fosse sporca di polvere o cenere, ma questa tonalità naturale permette di mimetizzarsi meglio nelle zone sotterranee. Sono una Razza di schiavisti, ma mentre costringono i prigionieri non Nani a lavori massacranti, uccidono senza remore i Nani catturati. In combattimento, i Nani Oscuro tirano di balestra, e poi passano al Martello da Guerra qualche round dopo. Se in inferiorità numerica, o se c'è un pericolo (e spazio) adeguato, un Nano Oscuro userà la sua capacità Ingrandire ed attaccherà.

%\addcontentsline{toc}{subsubsection}{O}
\pdfbookmark[3]{O}{O}

\mostro{Armatura Animata}
\noindent
\begin{description}[noitemsep, topsep=0pt, parsep=0pt, partopsep=0pt, leftmargin=0cm, labelwidth=2.2cm]
	\item[\textbf{Taglia/Tipo:}] Media costrutto, disallineato
	\item[\textbf{Caratt.:}] \resizebox{0.5\linewidth+1.8cm}{!}{For 2 Des 0 Cos 1 Int -5 Sag -4 Car -5}
	\item[\textbf{Punti Ferita:}] 33,  \textbf{Difesa:} 13,  \textbf{Iniziativa:} +0
	\item[\textbf{Movimento:}] 7 m
	\item[\textbf{Tiri Salvez.:}] \resizebox{0.5\linewidth+1.8cm}{!}{Tempra +3, Riflessi +3, Volontà +3}
	\item[\textbf{Imm. Danni:}] Veleno
	\item[\textbf{Immunità:}] accecato, affascinato, assordato, paralizzato, pietrificato, affaticato, spaventato
	\item[\textbf{Sensi:}] Vista Cieca 18 m (cieco oltre questo raggio)
	\item[\textbf{Sfida:}] 1 (200 PX)\smallskip
\end{description}

\emph{\textbf{Falso Aspetto.}} Mentre l'armatura rimane immobile, è indistinguibile da una normale armatura.

\emph{\textbf{Suscettibilità all'Anti Magia.}} L'armatura è inabile se si trova nell'area di un \emph{campo anti-magia}. Se è bersaglio di \emph{\hyperlink{Dissolvi Magie}{Dissolvi Magie}}, l'armatura deve riuscire un Tiro Salvezza su Tempra contro la DC del Tiro Salvezza dell'incantesimo o restare svenuta per 1 minuto.

\textbf{Azioni}

\emph{\textbf{Multiattacco.}} L'armatura effettua due attacchi da mischia.

\emph{\textbf{Schianto.} Attacco con arma da mischia}: +5 a colpire, portata 1 m, un bersaglio.

\emph{Colpisce:} 5 (1d6 + 2) danni contundenti.

\mostro{Spada Volante}
\noindent
\begin{description}[noitemsep, topsep=0pt, parsep=0pt, partopsep=0pt, leftmargin=0cm, labelwidth=2.2cm]
	\item[\textbf{Taglia/Tipo:}] Piccola costrutto, disallineato
	\item[\textbf{Caratt.:}] \resizebox{0.5\linewidth+1.8cm}{!}{For 1 Des 2 Cos 0 Int -5 Sag -3 Car -5}
	\item[\textbf{Punti Ferita:}] 19,  \textbf{Difesa:} 14,  \textbf{Iniziativa:} +2
	\item[\textbf{Movimento:}] 0 m, volo 15 m, Fluttuare
	\item[\textbf{Tiri Salvez.:}] \resizebox{0.5\linewidth+1.8cm}{!}{Tempra +3, Riflessi +3, Volontà +3}
	\item[\textbf{Imm. Danni:}] Veleno
	\item[\textbf{Immunità:}] accecato, affascinato, assordato, paralizzato, pietrificato, spaventato
	\item[\textbf{Sensi:}] Vista Cieca 18 m (cieco oltre questo raggio)
	\item[\textbf{Sfida:}] 1/4 (50 PX)\smallskip
\end{description}

\emph{\textbf{Falso Aspetto.}} Mentre l'arma rimane immobile e non sta volando, è indistinguibile da una normale spada.

\emph{\textbf{Suscettibilità all'Anti Magia.}} La spada è inabile se si trova nell'area di un \emph{campo anti-magia}. Se è bersaglio di \emph{\hyperlink{Dissolvi Magie}{Dissolvi Magie}}, la spada deve riuscire un Tiro Salvezza su Tempra contro la DC del Tiro Salvezza dell'incantesimo o restare svenuta per 1 minuto.

\textbf{Azioni}

\emph{\textbf{Spada Lunga.} Attacco con arma da mischia}: +4 a colpire, portata 1 m, un bersaglio.

\emph{Colpisce:} 5 (1d8 + 1) danni taglienti.

\mostro{Tappeto del Soffocamento}
\noindent
\begin{description}[noitemsep, topsep=0pt, parsep=0pt, partopsep=0pt, leftmargin=0cm, labelwidth=2.2cm]
	\item[\textbf{Taglia/Tipo:}] Grande costrutto, disallineato
	\item[\textbf{Caratt.:}] \resizebox{0.5\linewidth+1.8cm}{!}{For 3 Des 2 Cos 0 Int -5 Sag -4 Car -5}
	\item[\textbf{Punti Ferita:}] 51,  \textbf{Difesa:} 16,  \textbf{Iniziativa:} +2
	\item[\textbf{Movimento:}] 3 m
	\item[\textbf{Tiri Salvez.:}] \resizebox{0.5\linewidth+1.8cm}{!}{Tempra +3, Riflessi +4, Volontà +3}
	\item[\textbf{Imm. Danni:}] Veleno
	\item[\textbf{Immunità:}] accecato, affascinato, assordato, paralizzato, pietrificato, spaventato
	\item[\textbf{Sensi:}] Vista Cieca 18 m (cieco oltre questo raggio)
	\item[\textbf{Sfida:}] 2 (450 PX)\smallskip
\end{description}

\emph{\textbf{Falso Aspetto.}} Mentre il tappeto resta immobile, è indistinguibile da un normale tappeto.

\emph{\textbf{Suscettibilità all'Anti Magia.}} Il tappeto è inabile mentre si trova nell'area di un \emph{campo anti-magia}. Se è il bersaglio di \emph{\hyperlink{Dissolvi Magie}{Dissolvi Magie}}, il tappeto deve riuscire un Tiro Salvezza di Tempra contro la DC del Tiro Salvezza dell'incantatore o cadere privo di sensi per 1 minuto.

\emph{\textbf{Trasferimento di Danno.}} Mentre afferra una creatura, il tappeto subisce solo la metà dei danni che gli sono inferti, e la creatura afferrata dal tappeto subisce l'altra metà.

\textbf{Azioni}

\emph{\textbf{Soffocare.} Attacco con arma da mischia}: +6 a colpire, portata 1 m, una creatura di taglia Media o inferiore.

\emph{Colpisce:} La creatura è afferrata (DC 14 per fuggire). Fino al termine dell'afferrare, il bersaglio è accecato e rischia di soffocare, ma il tappeto non può soffocare un altro bersaglio. Inoltre, all'inizio di ciascun round del bersaglio, il bersaglio subisce 10 (2d6 + 3) danni contundenti.

\mostro{Ogre}
\noindent
\begin{description}[noitemsep, topsep=0pt, parsep=0pt, partopsep=0pt, leftmargin=0cm, labelwidth=2.2cm]
	\item[\textbf{Taglia/Tipo:}] Grande gigante, sadico malvagio
	\item[\textbf{Caratt.:}] \resizebox{0.5\linewidth+1.8cm}{!}{For 4 Des -1 Cos 3 Int -3 Sag -2 Car -2}
	\item[\textbf{Punti Ferita:}] 52,  \textbf{Difesa:} 13,  \textbf{Iniziativa:} -1
	\item[\textbf{Movimento:}] 12 m
	\item[\textbf{Tiri Salvez.:}] \resizebox{0.5\linewidth+1.8cm}{!}{Tempra +5, Riflessi +3, Volontà +3}
	\item[\textbf{Sensi:}] Scurovisione 18 m
	\item[\textbf{Linguaggi:}] Comune, Gigante
	\item[\textbf{Sfida:}] 2 (450 PX)\smallskip
\end{description}

\textbf{Azioni}

\emph{\textbf{Randello Pesante.} Attacco con arma da mischia}: +7 a colpire, portata 1 m, un bersaglio.

\emph{Colpisce:} 13 (2d8 + 4) danni contundenti.

\emph{\textbf{Giavellotto.} Attacco con arma da mischia o a Distanza}: +6 a colpire, portata 1 m o gittata 12m, un bersaglio.

\emph{Colpisce:} 11 (2d6 + 4) danni perforanti.

\textbf{Ecologia}\\
Ambiente: Colline fredde o temperate\\
Organizzazione: Solitario, coppia, gruppo (3-4) o famiglia (5-16)\\
\textbf{Categoria Tesoro}: Armatura di Pelle, Randello Pesante, 4 Giavellotti, J\\
\textbf{Descrizione}\\
Nelle storie riguardanti gli ogre ci sono elementi orrendi: brutalità e ferocia, cannibalismo e tortura. Poi stupri, smembramenti, necrofilia, incesto, mutilazioni e altri esempi di crudeltà. Coloro che non hanno mai incontrato gli ogre ritengono queste storie un avvertimento. Chi è sopravvissuto ad un simile incontro sa che le storie sono niente in confronto alla realtà.

Gli ogre godono della sofferenza altrui.

Se non hanno a disposizione le razze più piccole da schiacciare fra le loro grasse mani o da violare in amplessi violenti, si divertono fra loro. Per gli ogre non esiste tabù.

Si potrebbe pensare che, lasciata a sé stessa, una tribù di ogre si farebbe a pezzi da sola e che soltanto i più forti sopravvivrebbero: se c'è una cosa che gli ogre rispettano, però, è la famiglia.

Le tribù ogre sono conosciute come famiglie, e molte delle loro deformità sono causate dalla pratica comune dell'incesto.
Il capo della tribù è spesso il padre, ma in alcuni casi un ogre femmina è in grado di reclamare il titolo di madre. Le tribù ogre litigano fra loro, cosa che li tiene impegnati ed impedisce loro di tormentare i loro vicini. Di quando in quando, però, emerge un patriarca particolarmente violento o temuto, capace di unire più famiglie sotto il suo comando.

\begin{center}
	\includegraphics[width=0.8\linewidth]{immagini/The_Grey_Fairy_Book_-_Page_345.png}

	\emph{Henry Justice Ford}
\end{center}

Le regioni abitate degli ogre sono luoghi tristi e degradati, dato che questi giganti vivono nello squallore e non sentono il bisogno di essere in armonia con quanto li circonda.

I giochi degli ogre sono violenti e crudeli: le vittime utilizzate come giocattolo sono fortunate a morire il primo giorno. Il crudele senso dell'umorismo degli ogre è il solo caso in cui mostrano di possedere creatività: i metodi e gli strumenti di tortura ogre sembrano usciti dagli incubi.

La grande forza e la mancanza di immaginazione li rendono particolarmente adatti ai lavori pesanti, nelle miniere, come fabbri o nel disboscamento. I giganti più potenti (soprattutto quelli delle Colline e delle Rocce) spesso soggiogano le famiglie ogre perché diventino loro servitori.

Un ogre adulto è alto sui 3 metri e pesa circa 325 kg.

\mostro{Ombra}
\noindent
\begin{description}[noitemsep, topsep=0pt, parsep=0pt, partopsep=0pt, leftmargin=0cm, labelwidth=2.2cm]
	\item[\textbf{Taglia/Tipo:}] Media non morto, malvagio
	\item[\textbf{Caratt.:}] \resizebox{0.5\linewidth+1.8cm}{!}{For -2 Des 2 Cos 1 Int -2 Sag 0 Car -1}
	\item[\textbf{Punti Ferita:}] 24,  \textbf{Difesa:} 14,  \textbf{Iniziativa:} +2
	\item[\textbf{Movimento:}] 12 m
	\item[\textbf{Tiri Salvez.:}] \resizebox{0.5\linewidth+1.8cm}{!}{Tempra +3, Riflessi +3, Volontà +3}
	\item[\textbf{Comp.:}] Furtività +4 (+6 a luce fioca o oscurità)
	\item[\textbf{Res. Danni:}] Acido, Freddo, Elettricità, Fuoco, Suono; da arma non magica
	\item[\textbf{Imm. Danni:}] da Vuoto, Veleno
	\item[\textbf{Immunità:}] afferrato, intralciato, paralizzato, pietrificato, prono, affaticato, spaventato, sanguinamento
	\item[\textbf{Sensi:}] Scurovisione 18 m
	\item[\textbf{Sfida:}] 1/2 (100 PX)\smallskip
\end{description}

\emph{\textbf{Amorfo.}} L'ombra può muoversi attraverso uno spazio stretto fino a 3 centimetri senza stringersi.

\emph{\textbf{Debolezza alla Luce del Sole.}} Mentre si trova alla luce del sole, l'ombra ha -1d6 ai tiri per colpire, le prove di competenza di Base e i Tiri Salvezza.

\emph{\textbf{Spirito dell'Ombra.}} Mentre si trova in una zona di luce fioca l'Ombra rigenera 5 Punti Ferita all'inizio del suo round, se si trova in una zona di oscurità rigenera 10 Punti Ferita all'inizio del suo round e può diventare invisibile usando 1 Azione. Spirito dell'Ombra aumenta il Grado di Sfida dell'Ombra di 1.

\emph{\textbf{Furtività d'Ombra.}} Quando si trova a luce fioca o all'oscurità, l'ombra può usare una Azione per prendere +2 alla Difesa.

\emph{\textbf{Natura Non Morta.}} Un'ombra non necessita aria, cibo, bevande o sonno.

\textbf{Azioni}

\emph{\textbf{Risucchio di Forza.} Attacco con arma da mischia}: +4 a colpire, portata 1 m, una creatura.

\emph{Colpisce:} 9 (2d6 + 2) danni da Vuoto, e il punteggio di Forza del bersaglio viene ridotto di 1. Il bersaglio muore se ciò riduce la sua Forza a -5. Altrimenti, la riduzione resta finché il bersaglio non riposa 8 ore.

Se un umanoide non malvagio muore a causa di questo attacco, entro 1d4 ore dal suo cadavere si animerà una nuova ombra.

\emph{\textbf{Rubare l'ombra.}} Se l'ombra ha già colpito due volte con Risucchio di Forza usando una Azione ruba l'ombra dell'avversario. Rubare l'ombra concede 10 Punti Ferita Temporanei all'ombra. La creatura recupera l'ombra all'alba successiva.

\textbf{Ecologia}\\
Ambiente: Qualsiasi\\
Organizzazione: Solitario, coppia, gruppo (3-6) o sciame (7-12)\\
\textbf{Categoria Tesoro}: Nessuno\\
\textbf{Descrizione}\\
La malvagia ombra si muove lungo il confine tra il buio delle tenebre e la dura verità della luce. L'ombra preferisce infestare le rovine che la civiltà si lascia alle spalle, dove dà la caccia alle creature viventi tanto sciocche da incappare nel suo territorio. L'ombra è un orribile non morto, e come tale non ha scopi o motivazioni apparenti oltre a risucchiare forza vitale e vitalità dagli esseri viventi.

\mostro{Omuncolo}
\noindent
\begin{description}[noitemsep, topsep=0pt, parsep=0pt, partopsep=0pt, leftmargin=0cm, labelwidth=2.2cm]
	\item[\textbf{Taglia/Tipo:}] Minuscola costrutto, neutrale
	\item[\textbf{Caratt.:}] \resizebox{0.5\linewidth+1.8cm}{!}{For -3 Des 2 Cos 0 Int 0 Sag 0 Car -2}
	\item[\textbf{Punti Ferita:}] 15,  \textbf{Difesa:} 14,  \textbf{Iniziativa:} +2
	\item[\textbf{Movimento:}] 6 m, volo 12 m
	\item[\textbf{Tiri Salvez.:}] \resizebox{0.5\linewidth+1.8cm}{!}{Tempra +3, Riflessi +3, Volontà +3}
	\item[\textbf{Imm. Danni:}] Veleno
	\item[\textbf{Immunità:}] affascinato
	\item[\textbf{Sensi:}] Scurovisione 18 m, Vista Cieca 3 m
	\item[\textbf{Linguaggi:}] comprende le lingue del suo creatore ma non può parlare
	\item[\textbf{Sfida:}] 0 (10 PX)\smallskip
\end{description}

\emph{\textbf{Legame Telepatico.}} Mentre l'omuncolo si trova sullo stesso piano di esistenza del suo padrone, può comunicare magicamente al suo padrone quello che percepisce, e i due possono comunicare telepaticamente.

\textbf{Azioni}

\emph{\textbf{Morso.} Attacco con arma da mischia}: +4 a colpire, portata 1 m, una creatura.

\emph{Colpisce:} 1 danno perforante, e il bersaglio deve riuscire un Tiro Salvezza di Tempra DC 10 o restare avvelenato, -1 Forza e Destrezza, per 1 minuto. Se il Tiro Salvezza viene fallito in maniera critica il bersaglio resta invece avvelenato per 5 (1d10) minuti e mentre è avvelenato in questo modo è anche privo di sensi.

\emph{\textbf{Tramite del Padrone}}: usando 3 Azioni l'omuncolo diventa il tramite del lancio di un incantesimo del padrone.

\mostro{Oni}
\noindent
\begin{description}[noitemsep, topsep=0pt, parsep=0pt, partopsep=0pt, leftmargin=0cm, labelwidth=2.2cm]
	\item[\textbf{Taglia/Tipo:}] Grande gigante, malvagio
	\item[\textbf{Caratt.:}] \resizebox{0.5\linewidth+1.8cm}{!}{For 4 Des 0 Cos 3 Int 2 Sag 1 Car 2}
	\item[\textbf{Punti Ferita:}] 145,  \textbf{Difesa:} 21,  \textbf{Iniziativa:} +2
	\item[\textbf{Movimento:}] 9 m, volo 9 m
	\item[\textbf{Tiri Salvez.:}] \resizebox{0.5\linewidth+1.8cm}{!}{Tempra +10, Riflessi +7, Volontà +8}
	\item[\textbf{Comp.:}] Arcana +5, Ingannare +8
	\item[\textbf{Sensi:}] Scurovisione 18 m
	\item[\textbf{Linguaggi:}] Comune, Gigante
	\item[\textbf{Sfida:}] 7 (2900 PX)\smallskip
\end{description}

\emph{\textbf{Armi Magiche.}} Gli attacchi con armi dell'oni sono magici.

\emph{\textbf{Incantesimi Innati.}} La caratteristica da incantatore dell'oni è il Carisma. L'oni può lanciare questi incantesimi in maniera innata, senza bisogno di componenti materiali:

A volontà: \emph{\hyperlink{Invisibilità}{Invisibilità}, \hyperlink{Oscurità}{Oscurità}}

1/giorno: \emph{\hyperlink{Charme su Persone}{Charme su Persone}, \hyperlink{Cono di Freddo}{Cono di Freddo}, \hyperlink{Forma Gassosa}{Forma Gassosa}, \hyperlink{Sonno}{Sonno}}

\emph{\textbf{Rigenerazione.}} Se ha almeno 1 punto ferita, l'oni recupera 10 Punti Ferita all'inizio del suo round.

\textbf{Azioni}

\emph{\textbf{Multiattacco.}} L'oni effettua due attacchi, con gli artigli o con il falcione.

\emph{\textbf{Artiglio (Solo Forma di Oni).} Attacco con arma da mischia}: +8 a colpire, portata 1 m, un bersaglio.

\emph{Colpisce:} 8 (1d8 + 4) danni taglienti.

\emph{\textbf{Falcione.} Attacco con arma da mischia}: +9 a colpire, portata 3 m, un bersaglio.

\emph{Colpisce:} 15 (2d10 + 4) danni taglienti, o 9 (1d10 + 4) danni taglienti in forma Piccola o Media.

\textbf{Reazione: \emph{Attacco d'opportunità}}: l'Oni effettua un attacco ad una creatura che attraversi o esca dalla sua portata di 1/3 metri.

\emph{\textbf{Mutare Forma.}} L'oni può trasformarsi magicamente in un umanoide Piccolo o Medio, in un gigante Grande, o tornare alla sua vera forma. A parte la taglia, le sue statistiche sono le stesse in ciascuna forma. L'unico equipaggiamento che viene trasformato è il falcione, che rimpicciolisce in modo da essere impugnato anche in forma umanoide. Se l'oni muore, ritorna alla sua vera forma e il falcione ritorna alla sua taglia originale.

\emph{\textbf{Arrabbiato:}} l'Oni viene pervaso da una furia assassina, fino alla fine del combattimento i suoi attacchi con Artiglio causano Sanguinamento 2/10.

\mostro{Orchetto}
\noindent
\begin{description}[noitemsep, topsep=0pt, parsep=0pt, partopsep=0pt, leftmargin=0cm, labelwidth=2.2cm]
	\item[\textbf{Taglia/Tipo:}] Media umanoide (orco), caotico
	\item[\textbf{Caratt.:}] \resizebox{0.5\linewidth+1.8cm}{!}{For 2 Des 1 Cos 2 Int 0 Sag 0 Car 0}
	\item[\textbf{Punti Ferita:}] 24,  \textbf{Difesa:} 13,  \textbf{Iniziativa:} +1
	\item[\textbf{Movimento:}] 9 m
	\item[\textbf{Tiri Salvez.:}] \resizebox{0.5\linewidth+1.8cm}{!}{Tempra +3, Riflessi +3, Volontà +3}
	\item[\textbf{Comp.:}] Intimidire +1
	\item[\textbf{Sensi:}] Scurovisione 18 m
	\item[\textbf{Linguaggi:}] Comune, Goblinoide
	\item[\textbf{Sfida:}] 1/2 (100 PX)\smallskip
\end{description}

\textbf{Azioni}

\emph{\textbf{Spada.} Attacco con arma da mischia}: +4 a colpire, portata 1 m, un bersaglio.

\emph{Colpisce:} 6 (1d8 + 2) danni taglienti.

\emph{\textbf{Giavellotto.} Attacco con arma da mischia o a Distanza}: +5 a colpire, portata 1 m o gittata 12m, un bersaglio.

\emph{Colpisce:} 5 (1d6 + 2) danni perforanti.

\textbf{Ecologia}\\
Ambiente: Colline e montagne temperate o sotterranei\\
Organizzazione: solitario, gruppo (2-4), squadra (11-20 più 2 sergenti di 3° livello e 1 capo di 3°-6° livello) o banda \\
\textbf{Categoria Tesoro}: Equipaggiamento da PNG (Armatura di Cuoio Borchiato, Spada, 4 Giavellotti, M)\\
\textbf{Descrizione}\\
Gli orchetti sono una razza creata da Cattalm come esperimento con lo scopo di verificare se una creatura più intelligente ma altrettanto feroce degli orchi avrebbe potuto essere dominante.
L'esperimento è stato un discreto successo con gli orchetti che hanno fondato regni e conquistato diverse regioni. La spinta caotica con il passare del tempo, l'acculturamento, il diventare stanziali e l'evoluzione della società ha portato gli orchetti sempre più fuori dalle spire di Cattalm, anche se non toglie che molti aspetti barbari sono rimasti nella cultura tradizionale.
Un orchetti maschio adulto è alto 1,6 metri e pesa circa 60 kg. Caratteristica peculiare è il volto ed il naso in particolar modo da maiale. Gli orchetti e gli umani possono generare figli.

\mostro{Orco}
\noindent
\begin{description}[noitemsep, topsep=0pt, parsep=0pt, partopsep=0pt, leftmargin=0cm, labelwidth=2.2cm]
	\item[\textbf{Taglia/Tipo:}] Media umanoide (orco), malvagio
	\item[\textbf{Caratt.:}] \resizebox{0.5\linewidth+1.8cm}{!}{For 3 Des 1 Cos 3 Int -2 Sag 0 Car 0}
	\item[\textbf{Punti Ferita:}] 33,  \textbf{Difesa:} 14,  \textbf{Iniziativa:} +1
	\item[\textbf{Movimento:}] 9 m
	\item[\textbf{Tiri Salvez.:}] \resizebox{0.5\linewidth+1.8cm}{!}{Tempra +4, Riflessi +3, Volontà +3}
	\item[\textbf{Comp.:}] Intimidire +2
	\item[\textbf{Sensi:}] Scurovisione 18 m
	\item[\textbf{Linguaggi:}] Comune, Goblinoide
	\item[\textbf{Sfida:}] 1 (100 PX)\smallskip
\end{description}

\emph{\textbf{Aggressivo.}} Come Azione Immediata, l'orco può muoversi fino a metà del suo movimento verso una creatura ostile che possa vedere.

\emph{\textbf{Feroce.}} Come Azione l'orco affonda ancora più il colpo andato a segno causando 1d6 danni aggiuntivi.

\textbf{Azioni}

\emph{\textbf{Ascia Bipenne.} Attacco con arma da mischia}: +6 a colpire, portata 1 m, un bersaglio.

\emph{Colpisce:} 9 (1d12 + 3) danni taglienti.

\emph{\textbf{Giavellotto.} Attacco con arma da mischia o a Distanza}: +5 a colpire, portata 1 m o gittata 12m, un bersaglio.

\emph{Colpisce:} 6 (1d6 + 3) danni perforanti.

\textbf{Ecologia}\\
Ambiente: Colline e montagne temperate o sotterranei\\
Organizzazione: solitario, gruppo (2-4), squadra (11-20 più 2 sergenti di 3° livello e 1 capo di 3°-6° livello) o banda \\
\textbf{Categoria Tesoro}: Equipaggiamento da PNG (Armatura di Cuoio Borchiato, Falcione, 4 Giavellotti, K)\\
\textbf{Descrizione}\\
La differenza principale fra gli orchi e gli umanoidi civilizzati, oltre alla loro forza bruta ed all'intelligenza inferiore, è il loro carattere. Come cultura, gli orchi sono violenti ed aggressivi, ed il forte domina il debole attraverso paura e brutalità. Prendono ciò che vogliono con la forza e non si fanno scrupoli a prendere interi villaggi come schiavi se ne hanno la possibilità. Non si curano delle comodità, ed i loro villaggi e campi tendono ad essere luoghi sporchi e precari, pieni di risse fra ubriachi, arene per i combattimenti ed altri divertimenti sadici. Privi della pazienza necessaria a coltivare e capaci di allevare solo gli animali più robusti ed autosufficienti, gli orchi ritengono più semplice prendere agli altri il frutto del loro lavoro. Sono arroganti e lesti ad infuriarsi quando sfidati, ma si preoccupano dell'onore solo finché farlo porta loro beneficio.

Un orco maschio adulto è alto 2 metri e pesa circa 115 kg. Gli orchi e gli umani possono accoppiarsi, anche se di solito ciò avviene durante le razzie, e non come unione consensuale. Molte tribù orchesche allevano i mezzorchi di proposito, dato che sono ottimi strateghi e capitribù.

Per quanto la vulgata dica che gli orchi siano stati creati da Cattalm per distruggere e portare caos è anche vero che molto spesso sono vittima di pregiudizi e giudizi sommari. Non tutti gli orchi sono uguali e non solo fisicamente, singoli orchi se non intere tribù vivono in maniera normale, civilizzata la loro esistenza eppure in nessun stato del mondo sono previste pene per chi uccide un orco.

\mostro{Orrore Arrampicamuri}
\noindent
\begin{description}[noitemsep, topsep=0pt, parsep=0pt, partopsep=0pt, leftmargin=0cm, labelwidth=2.2cm]
	\item[\textbf{Taglia/Tipo:}] Grande mostruosità, disallineato
	\item[\textbf{Caratt.:}] \resizebox{0.5\linewidth+1.8cm}{!}{For 4 Des 0 Cos 2 Int -2 Sag 1 Car -2}
	\item[\textbf{Punti Ferita:}] 70,  \textbf{Difesa:} 16,  \textbf{Iniziativa:} +0
	\item[\textbf{Movimento:}] 9 m, scalare 9 m
	\item[\textbf{Tiri Salvez.:}] \resizebox{0.5\linewidth+1.8cm}{!}{Tempra +5, Riflessi +3, Volontà +4}
	\item[\textbf{Sensi:}] Scurovisione 3 m, Vista Cieca 18 m
	\item[\textbf{Linguaggi:}] Orrore Arrampicamuri
	\item[\textbf{Sfida:}] 3 (700 PX)\smallskip
\end{description}

\emph{\textbf{Senso Radar.}} l'Orrore Arrampicamuri non può usare vista cieca se è assordato.

\textbf{Azioni}

\emph{\textbf{Multiattacco.}} L'Orrore Arrampicamuri effettua due attacchi con gli artigli uncinati.

\emph{\textbf{Artigli.} Attacco con arma da mischia}: +7 a colpire, portata 1 m, un bersaglio.

\emph{Colpisce:} 10 (2d6 + 4) danni perforanti, 1 danno da Sanguinamento.

\textbf{Ecologia}\\
\textbf{Ambiente: Sottosuolo}
Organizzazione: Solitario, coppia o branco (3-8)\\
\textbf{Categoria Tesoro}: Accidentale\\
\textbf{Descrizione}\\
L'Orrore Arrampicamuri è un feroce predatore del sottosuolo, difende aggressivamente i suoi territori di caccia. Le caverne sotterranee in cui queste creature risiedono rimbombano dei colpi e dei fruscii dei loro uncini quando queste creature si arrampicano sulle rupi rocciose o sulle pareti delle caverne.

Un Orrore Arrampicamuri è una creatura mostruosa apparentemente umanoide dalla testa simile a quella di un avvoltoio e dal torace di un enorme scarabeo, protetto da un esoscheletro tempestato di protuberanze ossee aguzze. Trae il suo nome oltre che dall'orrendo aspetto dal fatto che usando gli arti lunghi e muscolosi che terminano con dei micidiali artigli uncinati ricurvi stia arrampicato sulle pareti.

Gli Orrore Arrampicamuri comunicano colpendo il loro esoscheletro o le superfici rocciose circostanti con i loro uncini

\emph{Branco di Predatori}. Gli Orrore Arrampicamuri sono creature onnivore: si nutrono di funghi, licheni, vegetali e di qualsiasi creatura riescano a catturare. Grazie agli arti uncinati, gli orrori beneficiano di un'ottima presa sulle superfici rocciose e usano le loro abilità da scalatori per tendere imboscate alle prede dall'alto. Vanno a caccia in branco e collaborano per affrontare gli avversari più grossi e pericolosi. Se una battaglia va male, un Orrore Arrampicamuri si arrampica rapidamente lungo la parete di una caverna per fuggire.

\emph{Clan Solidali}. Gli orrori uncinati vivono in vasti gruppi familiari o clan. Ogni clan è retto dalla femmina più anziana, che solitamente pone il suo compagno a capo dei cacciatori del clan. Gli Orrore Arrampicamuri depongono le uova in un'area centrale e ben difesa delle caverne usate come tana.

\mostro{Orsogufo}
\noindent
\begin{description}[noitemsep, topsep=0pt, parsep=0pt, partopsep=0pt, leftmargin=0cm, labelwidth=2.2cm]
	\item[\textbf{Taglia/Tipo:}] Grande bestia, disallineato
	\item[\textbf{Caratt.:}] \resizebox{0.5\linewidth+1.8cm}{!}{For 5 Des 1 Cos 3 Int -4 Sag 1 Car -2}
	\item[\textbf{Punti Ferita:}] 70,  \textbf{Difesa:} 17,  \textbf{Iniziativa:} +1
	\item[\textbf{Movimento:}] 12 m
	\item[\textbf{Tiri Salvez.:}] \resizebox{0.5\linewidth+1.8cm}{!}{Tempra +6, Riflessi +4, Volontà +4}
	\item[\textbf{Sensi:}] Scurovisione 18 m
	\item[\textbf{Sfida:}] 3 (700 PX)\smallskip
\end{description}

\emph{\textbf{Olfatto e Vista Affinati.}} L'Orsogufo ha +1d6 nelle prove di Consapevolezza basate su olfatto o vista.

\textbf{Azioni}

\emph{\textbf{Multiattacco.}} L'Orsogufo effettua due attacchi: uno con il becco e uno con gli artigli.

\emph{\textbf{Artigli.} Attacco con arma da mischia}: +6 a colpire, portata 1 m, un bersaglio.

\emph{Colpisce:} 14 (2d8 + 5) danni taglienti.

\emph{\textbf{Becco.} Attacco con arma da mischia}: +6 a colpire, portata 1 m, una creatura.

\emph{Colpisce:} 10 (1d10 + 5) danni perforanti.

\textbf{Ecologia}\\
\textbf{Ambiente: Foreste Temperate}
Organizzazione: Solitario, coppia o branco (3-8)\\
\textbf{Categoria Tesoro}: Accidentale\\
\textbf{Descrizione}\\
Le origini dell'Orsogufo sono oggetto di dibattito fra gli studiosi delle creature mostruose. La maggior parte di essi concorda che fu un Mago, in passato, a crearne il primo esemplare unendo un orso con un gufo gigante; forse come esperimento su qualche folle concetto della natura della vita, ma più probabilmente a causa della sua totale pazzia. Quale che fosse lo scopo originale di una creazione tanto folle come l'Orsogufo, la creatura ha iniziato a riprodursi, ed è divenuta uno dei predatori più conosciuto delle zone boschive.

Gli Orsogufo sono selvaggi predatori, noti per il loro pessimo temperamento, la loro aggressività e la loro ferocia. Tendono ad attaccare tutto ciò che si muove loro davanti, anche se questo non mostra intenzioni bellicose. Molti studiosi che hanno incontrato queste creature nelle terre selvagge hanno notato che hanno sempre occhi iniettati di sangue che ruotano tutto attorno poco prima di un attacco. Questo è generalmente visto come segno di follia, che suggerisce che tutti gli Orsogufo nascano con un bisogno patologico di combattere ed uccidere, ma i ricercatori più realisti ritengono sia dovuto alla struttura dei loro occhi acuti.

Gli Orsogufo abitano le zone più interne e nascoste dei boschi, e preparano le loro tane all'interno di foreste intricate o di buie e profonde caverne. Possono cacciare sia di giorno che di notte, a seconda delle abitudini delle prede che popolano i territori circostanti alla loro tana.

Gli Orsogufo adulti vivono in coppia e cacciano le prede in branco, lasciando i cuccioli nelle tane. In una tana si possono trovare di solito 1d6 cuccioli, che possono valere fino a 750 mo nei mercati cittadini.

Anche se è pressoché impossibile addomesticarli a causa della loro natura selvaggia, gli Orsogufo possono essere sfruttati come guardiani di un territorio specifico, sempre che vengano lasciati liberi di spostarsi al suo interno per cacciare. Gli addestratori professionisti chiedono fino a 2000 mo per addestrare un Orsogufo perché diventi un guardiano che obbedisca a comandi semplici (DC 23 per un cucciolo di Orsogufo, DC 30 per un Orsogufo adulto).

\emph{\textbf{Variante}}: \textbf{Orsogufo Polare}\index{Orsogufo Polare}\\
Questo Orsogufo è presente nelle regioni artiche o montane innevate. A differenza del normale Orsogufo è più robusto e forte. Ha 85 Punti Ferita, +10 al colpire, 21 di danno ad artiglio +1 da Sanguinamento, 15 di danno con becco. GS 4

\mostro{Orsogufo Saggio}
\noindent
\begin{description}[noitemsep, topsep=0pt, parsep=0pt, partopsep=0pt, leftmargin=0cm, labelwidth=2.2cm]
	\item[\textbf{Taglia/Tipo:}] Grande mostruosità, neutrale
	\item[\textbf{Caratt.:}] \resizebox{0.5\linewidth+1.8cm}{!}{For 3 Des 1 Cos 2 Int 3 Sag 3 Car 1}
	\item[\textbf{Punti Ferita:}] 70,  \textbf{Difesa:} 17,  \textbf{Iniziativa:} +3
	\item[\textbf{Movimento:}] 12 m
	\item[\textbf{Tiri Salvez.:}] \resizebox{0.5\linewidth+1.8cm}{!}{Tempra +5, Riflessi +4, Volontà +6}
	\item[\textbf{Comp.:}] Consapevolezza +9
	\item[\textbf{Sensi:}] Scurovisione 18 m
	\item[\textbf{Linguaggi:}] comprende e legge i seguenti: Comune, Druidico,Celestiale, Infernale, Nanico, Elfico, Orchesco, Gigante, Expiran, lingue Elementali
	\item[\textbf{Sfida:}] 3 (700 PX)\smallskip
\end{description}

\emph{\textbf{Olfatto e Vista Affinati.}} L'Orsogufo saggio ha +1d6 nelle prove di Consapevolezza basate su olfatto o vista.

\emph{\textbf{Incantesimi Innati.}} La caratteristica da incantatore dell'Orsogufo saggio è l'Intelligenza. L'Orsogufo saggio può lanciare in maniera innata i seguenti incantesimi, senza aver bisogno di componenti materiali:

A volontà: \emph{\hyperlink{Mano Magica}{Mano Magica}, \hyperlink{Comprensione degli Scritti}{Comprensione degli Scritti}}

\textbf{Azioni}

\emph{\textbf{Multiattacco.}} L'Orsogufo saggio effettua due attacchi: uno con il becco e uno con gli artigli.

\emph{\textbf{Artigli.} Attacco con arma da mischia}: +6 a colpire, portata 1 m, un bersaglio.

\emph{Colpisce:} 14 (2d8 + 5) danni taglienti.

\emph{\textbf{Becco.} Attacco con arma da mischia}: +6 a colpire, portata 1 m, una creatura.

\emph{Colpisce:} 10 (1d10 + 5) danni perforanti.

\textbf{Ecologia}\\
\textbf{Ambiente: Foreste Temperate}
Organizzazione: Solitario, coppia o branco (3-8)\\
\textbf{Categoria Tesoro}: E + T\\
\textbf{Descrizione}\\
Le origini dell'Orsogufo saggio sono misteriose quanto quelli del suo parente non saggio ma gli appassionati di queste creature li fanno discendere direttamente da Nethergal come variante dell'Orsogufo originale.
Solitamente l'Orsogufo saggio ama circondarsi di libri ed adora la compagnia di altri saggi ma non disdegna i racconti di avventurieri e le avvincenti ballate dei cantastorie. L'Orsogufo saggio ha un vero talento per le lingue e pur non potendo parlare in maniera comprensibile ad un uomo riesce a comprendere tantissime lingue parlate e scritte. L'Orsogufo saggio è in grado di leggere qualsiasi lingua o codice se ha modo di studiarlo per 3 giorni.
Solitamente più deboli e fragili del parente stretto sono comunque esseri temibili in combattimento.
Di preferenza un Orsogufo saggio non attacca se non per difesa e cerca un approccio il più tattico e utile possibile. Un tratto caratteristico degli Orsogufo saggi è una sciarpa rossa portata intorno all'assente collo. Uccidere un Orsogufo saggio è un affronto ai Devoti e Seguaci di Nethergal, è anche capitato che il Patrono stesso togliesse la capacità di comunicare a coloro si sono macchiati di efferatezze con le sue creature preferite.

Addestrare un Orsogufo saggio è molto più facile di un Orsogufo ma l'alta intelligenza della creatura lo spingerà ad avere un rapporto alla pari o come famiglio piuttosto.

L'incantesimo \hyperlink{Mano Magica}{Mano Magica} è solitamente usato per sfogliare i tomi più delicati e per scrivere, anche se con estrema lentezza.

\mostro{Otyugh}
\noindent
\begin{description}[noitemsep, topsep=0pt, parsep=0pt, partopsep=0pt, leftmargin=0cm, labelwidth=2.2cm]
	\item[\textbf{Taglia/Tipo:}] Grande aberrazione, neutrale
	\item[\textbf{Caratt.:}] \resizebox{0.5\linewidth+1.8cm}{!}{For 3 Des 0 Cos 4 Int 3 Sag 1 Car -2}
	\item[\textbf{Punti Ferita:}] 109,  \textbf{Difesa:} 18,  \textbf{Iniziativa:} +3
	\item[\textbf{Movimento:}] 9 m
	\item[\textbf{Tiri Salvez.:}] \resizebox{0.5\linewidth+1.8cm}{!}{Tempra +9, Riflessi +5, Volontà +6}
	\item[\textbf{Imm. Danni:}] Veleno
	\item[\textbf{Immunità:}] malattie
	\item[\textbf{Sensi:}] Scurovisione 36 m
	\item[\textbf{Linguaggi:}] Otyugh, Elfico, Nanico
	\item[\textbf{Sfida:}] 5 (1800 PX)\smallskip
\end{description}

\emph{\textbf{Telepatia Limitata.}} L'otyugh può trasmettere magicamente dei semplici messaggi e immagini a qualsiasi creatura entro 36 metri da esso e che possa comprendere una lingua. Questa forma di telepatia non permette alla creatura ricevente di rispondere telepaticamente.

\textbf{Azioni}

\emph{\textbf{Multiattacco.}} L'otyugh effettua tre attacchi: uno con il morso e due con i tentacoli.

\emph{\textbf{Morso.} Attacco con arma da mischia}: +7 a colpire, portata 1 m, un bersaglio.

\emph{Colpisce:} 12 (2d8 + 3) danni perforanti. Se il bersaglio è una creatura, deve riuscire un Tiro Salvezza di Tempra DC 17 contro malattia o restare malato finché la malattia non viene curata. Ogni 24 ore successive, il bersaglio deve ripetere il Tiro Salvezza, riducendo il suo massimo di Punti Ferita di 5 (1d10) se lo fallisce. Se il Tiro Salvezza riesce, la malattia è passata. Il bersaglio muore se la malattia riduce i suoi Punti Ferita massimi a 0.

Questa riduzione dei Punti Ferita massimi del personaggio perdura finché la malattia non viene curata.

\emph{\textbf{Tentacolo.} Attacco con arma da mischia}: +7 a colpire, portata 3 m, un bersaglio.

\emph{Colpisce:} 7 (1d8 + 3) danni contundenti più 4 (1d8) danni perforanti. Se il bersaglio è di taglia Media o inferiore, è afferrato (DC 13 per fuggire). L'otyugh ha due tentacoli, ciascun dei quali può afferrare un bersaglio diverso.

\emph{\textbf{Schianto di Tentacolo.}} L'otyugh schianta le creature afferrate dai suoi tentacoli, l'una contro l'altra o sul pavimento. Ogni creatura deve riuscire un Tiro Salvezza di Tempra DC 17 o subire 10 (2d6 + 3) danni contundenti e restare stordita fino al termine del prossimo round dell'otyugh. Se il Tiro Salvezza riesce, il bersaglio subisce la metà dei danni contundenti e non è stordito.

\emph{\textbf{Arrabbiato:}} l'otyugh emette un profumo che inebria i sensi. Tutte le creature nel raggio di 6 metri devono fare un Tiro Salvezza su Volontà DC 18 oppure agire in maniera casuale, come incantesimo \hyperlink{incconfusione}{Confusione} (pag. \pageref{incconfusione}), fino alla fine del prossimo round. Costa 2 Azioni.

\textbf{Ecologia}\\
Ambiente: Qualsiasi Sotterraneo\\
Organizzazione: Solitario, coppia o gruppo (3-4)\\
\textbf{Categoria Tesoro}: I\\
\textbf{Descrizione}\\
Gli otyugh sono creature particolarmente luride ed orride che vivono in luoghi che le persone sane di mente tendono ad evitare. Le loro tane si trovano nelle fogne, nei pozzi neri, nelle discariche e nelle paludi più mefitiche: più un luogo è sporco, più attira gli otyugh. Amano il ruolo dello spazzino e vagano per le caverne sotterranee in cerca di nuovi bocconcini in mezzo ai rifiuti. Una volta trovati si ingozzano e riportano alla loro tana quello che non riescono a consumare in una volta sola. Gli otyugh passano parecchio tempo nelle loro luride tane, che riempiono di carogne e letame, che rilasciano effluvi mefitici.

Le creature intelligenti che vivono nelle zone sotterranee vicino agli otyugh a volte formano alleanze di convenienza con essi. Forniscono loro rifiuti e carne cruda agli otyugh, rendendoli un vero e proprio mezzo di smaltimento. In cambio, gli otyugh lasciano in pace i loro benefattori, non li attaccano e possono anche fare da guardiani.

La cosa che la maggior parte delle razze trova terrificante degli otyugh non è la loro dieta o l'odore delle loro tane, ma il fatto che creature con i loro gusti non siano solo spazzini senza cervello. Gli otyugh si mostrano infatti sorprendentemente intelligenti ed amano formare alleanze con coloro che li riforniscono di cibi più raffinati di letame e sporcizia. La maggior parte degli otyugh si rende conto che le altre creature li trovano rivoltanti, ma sono pochi quelli a cui importa davvero.

Un otyugh mangiando gli escrementi o parte di una creatura può capire quale malattia o veleno la affligge.

%\addcontentsline{toc}{subsubsection}{P}
\pdfbookmark[3]{P}{P}

\mostro{Panoptikhan}
\noindent
\begin{description}[noitemsep, topsep=0pt, parsep=0pt, partopsep=0pt, leftmargin=0cm, labelwidth=2.2cm]
	\item[\textbf{Taglia/Tipo:}] Grande aberrazione, malvagio
	\item[\textbf{Caratt.:}] \resizebox{0.5\linewidth+1.8cm}{!}{For 0 Des 1 Cos 2 Int 3 Sag 2 Car 2}
	\item[\textbf{Punti Ferita:}] 235,  \textbf{Difesa:} 29,  \textbf{Iniziativa:} +3
	\item[\textbf{Movimento:}] 1 m, volo 10 metri, fluttuare
	\item[\textbf{Tiri Salvez.:}] \resizebox{0.5\linewidth+1.8cm}{!}{\resizebox{0.5\linewidth+1.8cm}{!}{Tempra +14, Riflessi +13, Volontà +14}}
	\item[\textbf{Sensi:}] Scurovisione 36 m, visione del vero 18m
	\item[\textbf{Linguaggi:}] telepatia 50 m
	\item[\textbf{Sfida:}] 12 (8400 PX)\smallskip
\end{description}

\emph{\textbf{Resistenza alla Magia.}} Il panoptikhan ha +1d6 ai Tiri Salvezza contro incantesimi e altri effetti magici.

\textbf{Azioni}

\emph{\textbf{Multiattacco.}} Il Panoptikhan può attaccare con due corti tentacoli.

\emph{\textbf{Tentacolo.} Attacco con arma da mischia}: +10 a colpire, portata 1 m, un bersaglio.

\emph{Colpisce:} 6 (1d6 + 3) danni da taglio perforanti.

\emph{\textbf{Colui che tutto vede}}. Il Panoptikhan può attivare uno dei suoi tentacoli occhiuti (2 Azioni). Il Panoptikhan ha CM 14.

\emph{Quello che gela}: l'occhio punta un bersaglio entro 18 metri, su questo viene attivato un \hyperlink{Raggio di Gelo}{Raggio di Gelo}. 8d8 di danno da freddo, Tiro Salvezza Riflessi DC 25 per evitare completamente il colpo.

\emph{Quello che scioglie}: l'occhio punta un bersaglio entro 9 metri, su questo viene attivato un raggio che ha effetti di acido. 4d8 di danno da acido, Tiro Salvezza Riflessi DC 25 per dimezzare il danno.

\emph{Quello che brucia}: l'occhio punta un bersaglio entro 18 metri, su questo viene attivato un raggio infuocato. 8d8 di danno da Fuoco, Tiro Salvezza Riflessi DC 25 per evitare completamente il colpo.

\emph{Quello che paralizza}: l'occhio punta un bersaglio entro 9 metri, su questo viene attivato un raggio che paralizza la creatura. Tiro Salvezza su Volontà DC 25 per evitare completamente gli effetti.

\emph{Quello che rallenta}: l'occhio punta in un cono di 9 metri. Sulle creature interessate viene proiettato un raggio che rallenta. Tiro Salvezza su Volontà DC 25 per evitare completamente gli effetti. Durata 1 minuto.

\emph{Quello che confonde}: l'occhio punta in un cono di 18 metri. Sulle creature interessate viene proiettato un raggio che causa confusione. Tiro Salvezza su Volontà DC 25 per evitare completamente gli effetti. Durata 1 minuto, ogni round è possibile effettuare un nuovo Tiro Salvezza per riprendersi dagli effetti.

\emph{Quello che addormenta}: l'occhio punta un bersaglio entro 36 metri, su questo viene attivato un raggio che addormenta la creatura. Tiro Salvezza su Volontà DC 25 per evitare completamente gli effetti.

\emph{Quello che muove}; questo occhio può manifestare l'incantesimo \hyperlink{Mano Magica}{Mano Magica} oppure Telecinesi.

\emph{\textbf{Un solo sguardo.}} Il Panoptikhan attiva l'occhio centrare. L'occhio centrale può essere usato come Azione di Reazione per lanciare Controincantesimo su un incantesimo che ha visto lanciare. DC 25.

\emph{\textbf{Arrabbiato:}} il Panoptikhan in preda alla furia più cieca attiva 1d6 occhi a caso su bersagli a caso. Costa 3 Azioni.

\textbf{Ecologia}\\
Ambiente: Qualsiasi Sotterraneo\\
Organizzazione: Solitario, coppia\\
\textbf{Categoria Tesoro}: H\\
\textbf{Descrizione}\\
I Panoptikhan sono aberrazioni xenofobe, palle di dura carne volante dotate di un grosso occhio centrale, una grande bocca e 7 tentacoli lunghi circa 1 metro ognuno dotato di un occhio (di circa 10 cm di diametro) di colore diverso.

Poco si sa dell'origine dei Panoptikhan, si pensa che siano un esperimento evoluzionario di Calicante, nel tentativo di creare una razza senziente e dominante.

Purtroppo l'arroganza, la superbia, il desiderio di essere al centro dell'attenzione hanno fatto naufragare questi tentativi di società ed i Panoptikhan si sono dispersi nel sottosuolo.

I Panoptikhan hanno una lunghissima vita, nell'ordine dei mille anni ma risultano anche creature che hanno più che raddoppiato questo limite. I Panoptikhan aumentano di taglia con l'età e così il numero di occhi. Le statistiche qui riportate sono riferite ad un esemplare di età adulta di circa 300 anni.

\mostro{Pegaso}
\noindent
\begin{description}[noitemsep, topsep=0pt, parsep=0pt, partopsep=0pt, leftmargin=0cm, labelwidth=2.2cm]
	\item[\textbf{Taglia/Tipo:}] Grande celestiale, buono
	\item[\textbf{Caratt.:}] \resizebox{0.5\linewidth+1.8cm}{!}{For 4 Des 2 Cos 3 Int 0 Sag 2 Car 1}
	\item[\textbf{Punti Ferita:}] 52,  \textbf{Difesa:} 16,  \textbf{Iniziativa:} +2
	\item[\textbf{Movimento:}] 18 m, volo 27 m
	\item[\textbf{Tiri Salvez.:}] \resizebox{0.5\linewidth+1.8cm}{!}{Tempra +5, Riflessi +4, Volontà +4}
	\item[\textbf{Comp.:}] Consapevolezza +6
	\item[\textbf{Linguaggi:}] comprende Celestiale, Comune, Elfico e Silvano ma non può parlare
	\item[\textbf{Sfida:}] 2 (450 PX)\smallskip
\end{description}

\textbf{Azioni}

\emph{\textbf{Zoccoli.} Attacco con arma da mischia}: +5 a colpire, portata 1 m, un bersaglio.

\emph{Colpisce:} 11 (2d6 + 4) danni contundenti.

\textbf{Ecologia}
Ambiente: Pianure Temperate e Calde\\
Organizzazione: Solitario, coppia o branco (6-10)\\
\textbf{Categoria Tesoro}: Nessuno\\
\textbf{Descrizione}\\
Il pegaso è un magnifico cavallo alato che a volte serve la causa del bene. Seppur molto apprezzati come cavalcature volanti, i pegasi sono creature timide che difficilmente stringono amicizie. Un tipico pegaso è alto 1,8 metri al garrese, pesa 750 kg ed ha un'apertura alare di 6 metri. La maggior parte dei pegasi è bianca, ma a volte alcuni esemplari hanno colori diversi.

Il pegaso, nonostante le apparenze, è intelligente quanto un umano. Chi cerca di addestrarne uno a fare da cavalcatura, scoprirà che il pegaso è ricalcitrante e perfino violento. Un pegaso non può parlare, ma capisce il Comune e preferisce la compagnia di creature buone. Il metodo corretto per convincere un pegaso a fare da cavalcatura è farselo amico con Diplomazia, favori e buone azioni. Un pegaso ha di norma atteggiamento indifferente verso le creature buone, maldisposto verso quelle neutrali ed ostile verso quelle malvagie. Prima che possa servire come cavalcatura, un pegaso deve essere reso amichevole tramite una prova di Diplomazia o in altro modo. Cavalcare un pegaso richiede una sella esotica o Cavalcare a pelo, dato che una sella normale interferisce con le sue ali. Un pegaso può combattere portando un cavaliere, ma il cavaliere non può attaccare a sua volta se non supera una prova di Cavalcare. I pegasi addestrati non temono il combattimento ed il cavaliere non deve effettuare una prova di Cavalcare per controllarlo.

I pegasi depongono uova che sul mercato valgono 1000 mo l'una, mentre i piccoli arrivano alle 2000 mo a testa. Essendo creature intelligenti e buone, vendere uova e piccoli è essenzialmente schiavismo: nelle società buone chi lo fa è disprezzato o punito dalla legge.

I pegasi maturano come i cavalli. Gli addestratori professionisti chiedono 1000 per addestrare un pegaso, che servirà un cavaliere buono o neutrale fedelmente per tutta la vita.

Un carico leggero per un pegaso è fino a 150 kg; un carico medio è 150,5-300 kg; un carico pesante è 300,5-450 kg.

In alcuni pegasi il sangue di un antenato che era un eroico stallone è ancora forte. Questi campioni hanno la durata della vita di un umano, manovrabilità perfetta, resistenza al fuoco 10, un bonus razziale di +4 ai Tiri Salvezza contro i Veleni e immunità alla Pietrificazione, un ulteriore +4 al Tiro per Colpire, +4 a Difesa, +25 PF, +4 a tutti i TS ed infliggono +1d6 di danno aggiuntivo. Alcuni riescono a dire poche parole in Celestiale o Comune. Si rendono conto della loro superiorità sui saurovalli, non devono essere addestrati a volare con un cavaliere, ma permettono solo ai più grandi eroi di cavalcarli.

Pegasi ed Unicorni sono stati salvati dalla furia di Calicante verso i \emph{cavalli} solo per espresse intercessione di Ljust.

\mostro{Persecutore Invisibile}
\noindent
\begin{description}[noitemsep, topsep=0pt, parsep=0pt, partopsep=0pt, leftmargin=0cm, labelwidth=2.2cm]
	\item[\textbf{Taglia/Tipo:}] Media elementale, neutrale
	\item[\textbf{Caratt.:}] \resizebox{0.5\linewidth+1.8cm}{!}{For 3 Des 4 Cos 2 Int 0 Sag 2 Car 0}
	\item[\textbf{Punti Ferita:}] 125,  \textbf{Difesa:} 24,  \textbf{Iniziativa:} +4
	\item[\textbf{Movimento:}] 15 m, volo 15 m, Fluttuare
	\item[\textbf{Tiri Salvez.:}] \resizebox{0.5\linewidth+1.8cm}{!}{\resizebox{0.5\linewidth+1.8cm}{!}{Tempra +8, Riflessi +10, Volontà +8}}
	\item[\textbf{Comp.:}] Furtività +10, Consapevolezza +8
	\item[\textbf{Res. Danni:}] da arma non magica
	\item[\textbf{Imm. Danni:}] Veleno
	\item[\textbf{Immunità:}] afferrato, intralciato, paralizzato, pietrificato, privo di sensi, prono, affaticato
	\item[\textbf{Sensi:}] Scurovisione 18 m
	\item[\textbf{Linguaggi:}] Ictun, comprende il Comune ma non lo parla
	\item[\textbf{Sfida:}] 6 (2300 PX)\smallskip
\end{description}

\emph{\textbf{Cacciatore Infallibile.}} Il convocatore assegna una preda al persecutore. Il persecutore sa la direzione e la distanza a cui si trova la preda finché entrambi si trovano sullo stesso piano di esistenza. Il persecutore conosce anche la posizione del suo convocatore.

\emph{\textbf{Invisibilità.}} Il persecutore è invisibile anche quando attacca e dopo che ha attaccato.

\emph{\textbf{Natura Elementale.}} Un persecutore invisibile non ha bisogno di aria, cibo, bevande o sonno.

\textbf{Azioni}

\emph{\textbf{Multiattacco.}} La persecutore effettua due attacchi di schianto.

\emph{\textbf{Schianto.} Attacco con arma da mischia}: +7 a colpire, portata 1 m, un bersaglio.

\emph{Colpisce:} 10 (2d6 + 3) danni contundenti.

\textbf{Reazione: \emph{Attacco d'opportunità}}: il persecutore invisibile effettua un attacco ad una creatura che attraversi o esca dalla sua portata di 1 metro.

\emph{\textbf{Arrabbiato:}} il Persecutore Invisibile rompe il patto e torna nel piano elementale dell'aria.

\textbf{Ecologia}\\
Ambiente: Qualsiasi\\
Organizzazione: Solitario\\
\textbf{Categoria Tesoro}: Nessuno\\
\textbf{Descrizione}\\
Originarie del Piano dell'Aria, queste creature si muovono nel mondo seguendo gli incarichi per coloro che le evocano. I cacciatori invisibili agiscono solitamente come guardiani e assassini. L'invisibilità naturale e la furtività permettono loro di seguire la preda senza essere visti e li avvantaggiano quando decidono di eliminare un bersaglio.

Molti cacciatori invisibili però considerano questi compiti come misere richieste da parte dei mortali. Se viene assegnato loro un compito particolarmente complesso o sgradito, un cacciatore invisibile cercherà di trovare una scappatoia se l'istruzione è formulata in modo scarno. Per esempio, i Maghi che richiamano un cacciatore invisibile con l'istruzione "proteggimi dal pericolo" potrebbero venire scortati in un lontano luogo nascosto, o addirittura portati sul Piano dell'Aria.

A causa delle continue evocazioni, molti cacciatori invisibili avversano gli abitanti del Piano Materiale. Quelli appena evocati nel mondo mortale conoscono solo le storie dei loro simili e mantengono un atteggiamento aperto nei riguardi di chi li richiama. Col tempo, o se servono un padrone malvagio, iniziano a farsi un'opinione negativa di queste creature mortali, che li porta a sviare le istruzioni e a danneggiare i loro padroni. Per i cacciatori invisibili più vecchi e con più esperienza, l'unica cosa che protegge chi li ha evocati è la magia che li lega. Queste creature tentano sempre di usare le incoerenze nella formulazione dei loro compiti e le distorsioni letterali nelle intenzioni per trovare un modo per infastidire, ferire o addirittura uccidere chi li ha portati su questo piano.

\mostro{Pseudodrago}
\noindent
\begin{description}[noitemsep, topsep=0pt, parsep=0pt, partopsep=0pt, leftmargin=0cm, labelwidth=2.2cm]
	\item[\textbf{Taglia/Tipo:}] Minuscola drago, buono
	\item[\textbf{Caratt.:}] \resizebox{0.5\linewidth+1.8cm}{!}{For -2 Des 2 Cos 1 Int 0 Sag 1 Car 0}
	\item[\textbf{Punti Ferita:}] 19,  \textbf{Difesa:} 14,  \textbf{Iniziativa:} +2
	\item[\textbf{Movimento:}] 5 metri, volo 18 m
	\item[\textbf{Tiri Salvez.:}] \resizebox{0.5\linewidth+1.8cm}{!}{Tempra +3, Riflessi +3, Volontà +3}
	\item[\textbf{Comp.:}] Furtività +4, Consapevolezza +3
	\item[\textbf{Sensi:}] Scurovisione 18 m, Vista Cieca 3 m
	\item[\textbf{Linguaggi:}] comprende il Comune e il Draconico ma non parla
	\item[\textbf{Sfida:}] 1/4 (50 PX)\smallskip
\end{description}

\emph{\textbf{Resistenza alla Magia.}} Lo pseudodrago ha +1d6 ai Tiri Salvezza contro incantesimi e altri effetti magici.

\emph{\textbf{Sensi Affinati.}} Lo pseudodrago ha +1d6 alle prove di Consapevolezza basate su vista, udito e olfatto.

\emph{\textbf{Telepatia Limitata.}} Lo pseudodrago può comunicare semplici idee, emozioni e immagini telepaticamente con qualsiasi creatura entro 30 metri da esso che può comprendere una lingua.

\textbf{Azioni}

\emph{\textbf{Morso.} Attacco con arma da mischia}: +4 a colpire, portata 1 m, un bersaglio.

\emph{Colpisce:} 4 (1d4 + 2) danni perforanti.

\emph{\textbf{Pungiglione.} Attacco con arma da mischia}: +4 a colpire, portata 1 m, una creatura.

\emph{Colpisce:} 4 (1d4 + 2) danni perforanti e il bersaglio deve riuscire un Tiro Salvezza di Tempra DC 11 o essere Confuso per 1 round. Se il Tiro Salvezza fallisce criticamente la creatura cade addormentata finché non risvegliata.

\textbf{Ecologia}\\
Ambiente: Foreste temperate\\
Organizzazione: Solitario, coppia o nido (3-5)\\
\textbf{Categoria Tesoro}: R\\
\textbf{Descrizione}\\
Gli pseudodraghi sono piccoli parenti dei veri draghi, giocosi e timidi. Parlano cinguettando, sibilando, ringhiando e facendo le fusa, ma possono comunicare telepaticamente con qualsiasi creatura intelligente. Se avvicinati pacificamente con offerte di cibo, sono disposti a condividere informazioni su quanto si trova nel loro territorio, ma minacce e violenza li fanno fuggire.

Gli pseudodraghi sono carnivori e mangiano insetti, roditori, uccellini e serpenti, anche se mangiano uova ed amano burro, formaggio e pesce. A volte cacciano a terra come le lucertole o volando come gli uccelli predatori. Intelligenti come la maggior parte degli umanoidi, non amano essere trattati come animali domestici, e preferiscono essere considerati amici. Diffidano delle creature malvagie, possono unirsi a incantatori e Devoti come Famigli e alcuni hanno stretto amicizia con Druidi e guardiaboschi o collaborano con i draghi buoni come sentinelle. Gli pseudodraghi diventano Famigli solo se apprezzano la personalità dell'incantatore (e se questi ha l'Abilità Famiglio e Carisma almeno 1), ma possono anche legarsi a persone delle quali apprezzano la compagnia. Uno pseudodrago potrebbe seguire in questo modo un personaggio per giorni, settimane, anni o perfino per tutta la vita, posto che siano ben nutriti e trattati con affetto.

Raggiunta l'età adulta, il corpo di uno pseudodrago è lungo 30 centimetri con una coda di 60 centimetri, e pesa circa 3,5 kg. Le uova di uno pseudodrago sono grandi come quelle di gallina, ma di consistenza simile al cuoio e macchiate di marrone, e le femmine le depongono in gruppi di 2-5 ogni primavera. Un nido di pseudodraghi (che costituiscono un gruppo familiare, e non sono nati dallo stesso gruppo di uova) di solito consiste di una coppia di adulti e diversi cuccioli quasi adulti.

%\addcontentsline{toc}{subsubsection}{R}
\pdfbookmark[3]{R}{R}

\mostro{Rakshasa}
\noindent
\begin{description}[noitemsep, topsep=0pt, parsep=0pt, partopsep=0pt, leftmargin=0cm, labelwidth=2.2cm]
	\item[\textbf{Taglia/Tipo:}] Media immondo, malvagio
	\item[\textbf{Caratt.:}] \resizebox{0.5\linewidth+1.8cm}{!}{For 2 Des 3 Cos 4 Int 1 Sag 3 Car 5}
	\item[\textbf{Punti Ferita:}] 259,  \textbf{Difesa:} 32,  \textbf{Iniziativa:} +3
	\item[\textbf{Movimento:}] 12 m
	\item[\textbf{Tiri Salvez.:}] \resizebox{0.5\linewidth+1.8cm}{!}{\resizebox{0.5\linewidth+1.8cm}{!}{Tempra +17, Riflessi +16, Volontà +16}}
	\item[\textbf{Comp.:}] Ingannare +10, Percepire Emozioni +8
	\item[\textbf{Imm. Danni:}] contundenti, armi +1
	\item[\textbf{Sensi:}] Scurovisione 18 m
	\item[\textbf{Linguaggi:}] Comune, Infernale
	\item[\textbf{Sfida:}] 13 (10000 PX)\smallskip
\end{description}

\emph{\textbf{Immunità alla Magia Limitata.}} Il rakshasa è immune agli affetti o all'individuazione tramite incantesimi di livello 6 o più basso a meno che non desideri esserne soggetto. Ha +1d6 ai Tiri Salvezza contro tutti gli altri incantesimi ed effetti magici.

\emph{\textbf{Incantesimi Innati.}} La caratteristica da incantatore del rakshasa il Carisma (+10 a colpire con attacchi con incantesimi). Il rakshasa può lanciare in maniera innata i seguenti incantesimi senza aver bisogno di componenti materiali:

A volontà: \emph{\hyperlink{Camuffare Sé Stesso}{Camuffare Sé Stesso}, \hyperlink{Illusione Minore}{Illusione Minore}, \hyperlink{Individuazione dei Pensieri}{Individuazione dei Pensieri}, \hyperlink{Mano Magica}{Mano Magica}}

3/Giorno ciascuno: \emph{\hyperlink{Charme su Persone}{Charme su Persone}, \hyperlink{Immagine Maggiore}{Immagine Maggiore}, \hyperlink{Individuazione del Magico}{Individuazione del Magico}, \hyperlink{Invisibilità}{Invisibilità}, \hyperlink{Suggestione}{Suggestione}} 1/Giorno: \emph{\hyperlink{Dominare Persone}{Dominare Persone}, \hyperlink{Visione del Vero}{Visione del Vero}, volare}

\textbf{Azioni}

\emph{\textbf{Multiattacco.}} Il rakshasa può effettuare due attacchi di artiglio.

\emph{\textbf{Artiglio.} Attacco con arma da mischia}: +10 a colpire, portata 1 m, un bersaglio.

\emph{Colpisce:} 9 (2d6 + 2) danni taglienti, e se il bersaglio è una creatura rimane maledetto. La maledizione magica ha effetto ogni qualvolta il bersaglio riposa, riempiendo i pensieri del bersaglio di immagini e sogni orribili. Il bersaglio maledetto non riceve beneficio dall'aver terminato un riposo. La maledizione perdura finché non viene rimossa dall'incantesimo \emph{\hyperlink{Rimuovi Maledizione}{Rimuovi Maledizione}} o simile magia.

\textbf{Reazione: \emph{Attacco d'opportunità}}: il Rakshasa effettua un attacco ad una creatura che attraversi o esca dalla sua portata di 1 metro.

\textbf{Ecologia}
Ambiente: Qualsiasi\\
Organizzazione: Solitario, coppia o culto (3-12)\\
\textbf{Categoria Tesoro}: Pugnale+1, I\\
\textbf{Descrizione}\\
Il rakshasa è uno spirito maligno che si traveste da creatura umanoide così da poter seguire la sua preda in incognito. Personificazione dei tabù della maggioranza delle società e capace di assumere l'aspetto di quelli che cerca di corrompere, un rakshasa compie moltissime azioni orribili. Se fossero umani, la loro blasfemia, il cannibalismo e gli atti ancora peggiori che compiono li marchierebbero come criminali meritevoli del più crudele degli inferni.

Quando non ha un altro aspetto, il rakshasa appare come un umanoide con la testa di un animale. Spesso ha il capo di un grosso felino (come tigri o pantere) o serpente (quali cobra o vipere) e, seppur sia più raro, può avere testa di gorilla, sciacallo, avvoltoio, elefante, mantide, lucertola, rinoceronte, cinghiale e molte altre ancora. In molti casi, il tipo di testa posseduta da un rakshasa dice qualcosa della sua personalità: un rakshasa dalla testa di tigre è furtivo e famelico, mentre uno con la testa di cinghiale può essere ghiotto e crudele. Queste differenze raramente incidono sulle statistiche base del rakshasa, anche se esistono varianti più potenti della standard con molteplici teste, poteri magici più potenti, e strane e letali capacità speciali aggiuntive.

I rakshasa disprezzano le religioni; riconoscono il potere degli dei, ma si vedono come i soli esseri degni di venerazione da parte delle razze mortali. I Devoti rakshasa sono quindi piuttosto rari. Sebbene i rakshasa siano esterni, sono anche creature del Piano Materiale, e alcuni credono che i primi rakshasa scelsero questo esilio al posto di qualche altro ruolo offertogli da un dio da tempo dimenticato. Anche se in genere sono solitari, non è raro trovare grandi famiglie di rakshasa che lavorano insieme per provocare la caduta di una civiltà mortale dall'interno, attraverso il succedersi di molte generazioni.

Un rakshasa è alto 1,8 metri e pesa 90 kg.

\mostro{Razziamorti}
\noindent
\begin{description}[noitemsep, topsep=0pt, parsep=0pt, partopsep=0pt, leftmargin=0cm, labelwidth=2.2cm]
	\item[\textbf{Taglia/Tipo:}] Grande costrutto, non morto, non allineato
	\item[\textbf{Caratt.:}] \resizebox{0.5\linewidth+1.8cm}{!}{For 5 Des 0 Cos 4 Int -4 Sag -2 Car -5}
	\item[\textbf{Punti Ferita:}] 127,  \textbf{Difesa:} 20,  \textbf{Iniziativa:} +0
	\item[\textbf{Movimento:}] 9 m
	\item[\textbf{Tiri Salvez.:}] \resizebox{0.5\linewidth+1.8cm}{!}{Tempra +10, Riflessi +6, Volontà +4}
	\item[\textbf{Imm. Danni:}] Veleno
	\item[\textbf{Immunità:}] affascinato, affaticato, paralizzato, pietrificato, sanguinamento, malattie
	\item[\textbf{Sensi:}] Scurovisione 30 m
	\item[\textbf{Linguaggi:}] comprende tutte le lingue del creatore ma non può parlare
	\item[\textbf{Sfida:}] 6 (2300 PX)\smallskip
\end{description}

\emph{\textbf{Riduzione del Danno.}} Il Razziamorti ha durezza 6/- contro armi non magiche.

\emph{\textbf{Natura Non Morta.}} Il Razziamorti non ha bisogno di aria, cibo, bevande o sonno.

\emph{\textbf{Forma Immutabile.}} Come costrutto non può essere influenzato da magie od effetti che ne cambino la forma.

\emph{\textbf{Contenitore.}} Il Razziamorti ha un comparto apribile con uno sportello sul dorso metallico che può contenere fino a 100kg di oggetti, grandi fino a taglia piccola.

\emph{\textbf{Resistenza all'Aria.}} Il Razziamorti ha una resistenza innata agli incantesimi della Lista di Magia Aria.

\emph{\textbf{Sensibile al Fuoco.}} Il Razziamorti se subisce danni da fuoco esegue una Azione in meno il round dopo.

\textbf{Azioni}

\emph{\textbf{Multiattacco.}} Il Razziamorti attacca con due chele o attacca con una chela e usa l'Occhio Paralizzante.

\textbf{\emph{Chela.}} +8 al colpire, portata 1 metro

\emph{Colpire}: 16 (2d10 + 5) di danni contundente

\emph{\textbf{Occhio Paralizzante}}: la creatura interessata, entro 18 metri, deve fare un Tiro Salvezza su Tempra a DC 18 o rimanere paralizzato per 2d4 round.

\emph{\textbf{Tartaruga triste}}: con una prova di Atletica DC 24 è possibile ribaltare sottosopra il razziamorti che non è più in grado di ribaltarsi da solo. In questa circostanza il razziamorti ha -1d6 a tutti i Tiri per Colpire.

\textbf{Ecologia}\\
Ambiente: Qualsiasi, caverne\\
Organizzazione: 1-2 Razziamorti, 1d4+1 guardiani\\
\textbf{Categoria Tesoro}: Quanto raccolto (C + R)\\
\textbf{Descrizione}\\
I Razziamorti sono dei particolari non morti costruiti da pezzi di vario cadavere e pezzi di ferro perché assomiglino a delle specie di grossi granchi corazzati.
Il dorso, completamente metallico, funge da contenitore per i tesori che il Razziamorti trova, le chele, in numero variabile tra le 6 ed 8 sono lunghe poco più di un metro ed hanno la caratteristica di lasciare ognuna una impronta diversa essendo assemblate da pezzi di metallo e corpi diversi.

Il grosso occhio centrale, forse una volta appartenuto ad un umanoide permette al controllore e costruttore del Razziamorti di vedere e comandarlo. Lo scopo di un Razziamorti é esplorare, solitamente un sistema di caverne o percorsi, alla ricerca dei resti di passati razziatori e avventurieri per carpirne gli oggetti magici e tesori.

Solitamente un Razziamorto è sempre accompagnato da diversi guardiani (altre creature al comando del controllore) che lo aiutano nel \emph{sistemare} eventuali \emph{resistenze} ancora attive.

\mostro{Remorhaz}
\noindent
\begin{description}[noitemsep, topsep=0pt, parsep=0pt, partopsep=0pt, leftmargin=0cm, labelwidth=2.2cm]
	\item[\textbf{Taglia/Tipo:}] Enorme mostruosità, disallineato
	\item[\textbf{Caratt.:}] \resizebox{0.5\linewidth+1.8cm}{!}{For 7 Des 1 Cos 5 Int -3 Sag 0 Car -3}
	\item[\textbf{Punti Ferita:}] 224,  \textbf{Difesa:} 27,  \textbf{Iniziativa:} +1
	\item[\textbf{Movimento:}] 9 m, scavo 6 m
	\item[\textbf{Tiri Salvez.:}] \resizebox{0.5\linewidth+1.8cm}{!}{\resizebox{0.5\linewidth+1.8cm}{!}{Tempra +16, Riflessi +12, Volontà +11}}
	\item[\textbf{Sensi:}] Scurovisione 18 m, senso tellurico 18 m
	\item[\textbf{Sfida:}] 11 (7200 PX)\smallskip
\end{description}

\emph{\textbf{Corpo Riscaldato.}} Una creatura che entri a contatto con il remorhaz o lo colpisca con un attacco da mischia mentre si trova entro 1 metro da esso, subisce 10 (3d6) danni da fuoco.

\textbf{Azioni}

\emph{\textbf{Morso.} Attacco in mischia con arma}: +11 a colpire, portata 3 m, un bersaglio.

\emph{Colpisce:} 40 (6d10 + 7) danni perforanti più 10 (3d6) danni da fuoco. Se il bersaglio è una creatura, è afferrato (DC 17 per fuggire). Fino al termine dell'afferrare  il remorhaz non può attaccare con il morso un altro bersaglio.

\emph{\textbf{Inghiottire.}} Il remorhaz effettua una attacco di morso contro un bersaglio di taglia Media o inferiore che sta afferrando. Se l'attacco colpisce, la creatura subisce il danno da morso ed è inghiottita, e l'afferrare ha termine. Il bersaglio inghiottito è accecato e intralciato, ha copertura completa contro gli attacchi e altri effetti all'esterno del remorhaz, e subisce 21 (6d6) danni da acido all'inizio di ciascun round del remorhaz.

Se il remorhaz subisce 30 o più danni in un singolo round da una creatura al suo interno, il remorhaz deve riuscire un Tiro Salvezza su Tempra DC 24 al termine di quel round o vomitare tutte le creature inghiottite, che cadono prone in uno spazio entro 3 metri dal remorhaz. Se il remorhaz muore, una creatura inghiottita non più intralciata da esso e può uscire dal cadavere utilizzando 2 Azioni e uscendo prona.

\emph{\textbf{Feroce.}} Come Azione il remorhaz affonda ancora più il Morso andato a segno causando 3d6 danni perforanti aggiuntivi. 1 Azione.

\emph{\textbf{Arrabbiato:}} il Remorhaz scalda ancora di più il suo corpo fino alla fine del combattimento portanto a 18 (6d6) il danno da fuoco per chi è entro 1 metro.

\textbf{Ecologia}\\
Ambiente: Deserti Freddi e Ghiacciai\\
Organizzazione: Solitario\\
\textbf{Categoria Tesoro}: Nessuno\\
\textbf{Descrizione}\\
In un mondo di ghiaccio e neve, i remorhaz sono particolarmente temuti per il terribile fuoco che brucia dentro i loro corpi. Questo fuoco interiore fa sì che le piastre lungo il suo dorso divengano roventi quando la creatura è particolarmente arrabbiata, eccitata o nel panico. Le creature che si sono adattate alle regioni artiche spesso sono particolarmente vulnerabili al fuoco, il che rende la principale difesa del remorhaz incredibilmente potente e gli assicura il ruolo di pericoloso predatore delle zone ghiacciate. I remorhaz vivono in estesi labirinti scavati nel cuore dei ghiacciai. Queste bestie usano il loro calore per scavare tunnel nel ghiaccio, tunnel le cui lisce pareti vitree si ricongelano rapidamente lungo la loro scia creando numerosi dedali incredibilmente stabili.

Intelligenti nonostante l'apparenza, i remorhaz capiscono il linguaggio dei Giganti e spesso formano alleanze con loro. I Giganti del Gelo li usano come armi contro i nemici, mentre altri giganti li sfruttano come forge viventi. Un remorhaz misura 7 metri di lunghezza e pesa 5000 kg.

\mostro{Rugginofago}
\noindent
\begin{description}[noitemsep, topsep=0pt, parsep=0pt, partopsep=0pt, leftmargin=0cm, labelwidth=2.2cm]
	\item[\textbf{Taglia/Tipo:}] Media Mostruosità, disallineato
	\item[\textbf{Caratt.:}] \resizebox{0.5\linewidth+1.8cm}{!}{For 1 Des 1 Cos 1 Int -4 Sag 1 Car -2}
	\item[\textbf{Punti Ferita:}] 24,  \textbf{Difesa:} 13,  \textbf{Iniziativa:} +1
	\item[\textbf{Movimento:}] 12 m
	\item[\textbf{Tiri Salvez.:}] \resizebox{0.5\linewidth+1.8cm}{!}{Tempra +3, Riflessi +3, Volontà +3}
	\item[\textbf{Sensi:}] Scurovisione 18 m
	\item[\textbf{Sfida:}] 1/2 (100 PX)\smallskip
\end{description}

\emph{\textbf{Fiuto del Ferro.}} Il rugginofago può individuare, con l'olfatto, l'esatta posizione di metalli ferrosi entro 36 metri.

\emph{\textbf{Arrugginire Metallo.}} Qualsiasi arma non magica fatta di metallo che colpisca il rugginofago si corrode dopo aver applicato il danno. Le munizioni non magiche fatte di metallo e che colpiscono il rugginofago, sono considerate distrutte dopo aver inflitto il danno.

\textbf{Azioni}

\emph{\textbf{Morso.} Attacco con arma da mischia}: +4 a colpire, portata 1 m, un bersaglio.

\emph{Colpisce:} 5 (1d8 + 1) danni perforanti.

\emph{\textbf{Antenne.}} Il rugginofago corrode gli oggetti di metallo ferroso non magici che può vedere e si trovano entro 1 metro. Se l'oggetto non è indossato o trasportato, il contatto col rugginofago ne distrugge un cubo di 30 centimetri di spigolo. Se l'oggetto è indossato o trasportato da una creatura, la creatura può effettuare un Tiro Salvezza su Riflessi DC 13 per evitare il contatto con il rugginofago.

Se l'oggetto con cui entra in contatto è un'armatura o scudo di metallo indossati o trasportati, questi subiscono una penalità permanente e cumulativa di -2 alla Difesa che forniscono. Le armature ridotte a Difesa 0 o gli scudi che scendono ad un bonus di +0 sono distrutti. Se l'oggetto con cui entra in contatto è un'arma di metallo impugnata da qualcuno, la arrugginisce come descritto nel tratto Arrugginire Metallo.

\textbf{Ecologia}
Ambiente: Qualsiasi Sotterraneo\\
Organizzazione: Solitario, coppia o nido (3-10)\\
\textbf{Categoria Tesoro}: Accidentale (nessun tesoro di metallo)\\
\textbf{Descrizione}\\
Di tutte le bestie terrificanti che un esploratore può incontrare nel sottosuolo, il rugginofago è l'unico a mirare al tesoro dell'avventuriero. Lungo circa un metro e pesante almeno 100 kg, il rugginofago somiglia a un crostaceo, e il suo processo nutritivo alieno lo rende ancora più spaventoso.

I rugginofagi divorano oggetti di metallo, preferendo ferro e acciaio, ma consumano anche mithral, adamantio e metalli incantati con facilità. Qualsiasi metallo toccato dalle loro antenne o dalla pelle corazzata si corrode e si riduce in polvere in pochi secondi, rendendoli temuti dagli avventurieri e dai nani minatori.

Sebbene non siano intrinsecamente violenti, la loro insaziabile fame li spinge a caricare chiunque porti abbastanza metallo, rispondendo con feroce aggressività a qualsiasi resistenza. In zone povere di metallo, possono seguire le vittime in fuga per giorni, fiutando i metalli intatti.

Fortunatamente, è spesso possibile sfuggire alle attenzioni di un rugginofago lanciandogli un oggetto di metallo denso, come uno scudo, e correndo nella direzione opposta. Chi frequenta aree infestate dai rugginofagi impara presto a portare con sé armi di legno o pietra.

%\addcontentsline{toc}{subsubsection}{S}
\pdfbookmark[3]{S}{S}

\mostro{Sahuagin}
\noindent
\begin{description}[noitemsep, topsep=0pt, parsep=0pt, partopsep=0pt, leftmargin=0cm, labelwidth=2.2cm]
	\item[\textbf{Taglia/Tipo:}] Media umanoide (sahuagin), malvagio
	\item[\textbf{Caratt.:}] \resizebox{0.5\linewidth+1.8cm}{!}{For 1 Des 0 Cos 1 Int 1 Sag 1 Car -1}
	\item[\textbf{Punti Ferita:}] 24,  \textbf{Difesa:} 12,  \textbf{Iniziativa:} +1
	\item[\textbf{Movimento:}] 9 m, nuoto 12 m
	\item[\textbf{Tiri Salvez.:}] \resizebox{0.5\linewidth+1.8cm}{!}{Tempra +3, Riflessi +3, Volontà +3}
	\item[\textbf{Comp.:}] Consapevolezza +5
	\item[\textbf{Sensi:}] Scurovisione 36 m
	\item[\textbf{Linguaggi:}] Sahuagin
	\item[\textbf{Sfida:}] 1/2 (100 PX)\smallskip
\end{description}

\emph{\textbf{Anfibio Limitato.}} Il sahuagin può respirare aria e acqua, ma deve restare sommerso almeno una volta ogni 4 ore per evitare di soffocare.

\emph{\textbf{Frenesia Sanguinaria.}} Il sahuagin ha +1d6 ai tiri per colpire in mischia contro qualsiasi creatura che non sia al massimo dei suoi Punti Ferita.

\emph{\textbf{Telepatia con gli Squali}}. Il sahuagin può comandare magicamente qualsiasi squalo entro 36 metri da sé, usando una forma limitata di telepatia.

\textbf{Azioni}

\emph{\textbf{Multiattacco.}} Il sahuagin può effettuare due attacchi da mischia: uno con il morso e uno con gli artigli o la lancia.

\emph{\textbf{Artigli.} Attacco con arma da mischia}: +4 a colpire, portata 1 m, un bersaglio.

\emph{Colpisce:} 3 (1d4 + 1) danni taglienti.

\emph{\textbf{Lancia.} Attacco con arma da mischia o a Distanza}: +4 a colpire, portata 1 m o gittata 6m, un bersaglio.

\emph{Colpisce:} 4 (1d6 + 1) danni perforanti, o 5 (1d8 + 1) danni perforanti se usata con due mani per effettuare un attacco da mischia.

\emph{\textbf{Morso.} Attacco con arma da mischia}: +4 a colpire, portata 1 m, un bersaglio.

\emph{Colpisce:} 3 (1d4 + 1) danni perforanti.

\textbf{Ecologia}\\
Ambiente: Oceani Temperati o Caldi\\
Organizzazione: Solitario, coppia, squadra (5-8), pattuglia (11-20 più 1 tenente di 3° livello e 1-2 Squali), banda (20-80 più 100\% non combattenti, 1 tenente di 3° livello e 1 capitano di 4° livello ogni 20 adulti, e 1-2 Squali) o tribù (70-160 più 100\% non combattenti, 1 tenente di 3° livello ogni 20 adulti, 1 capitano di 4° livello ogni 40 adulti, 9 guardie di 4° livello, 1-4 novizie di 3°-6° livello, 1 sacerdotessa di 7° livello, 1 barone di 6°-8° livello, e 5-8 Squali)
\textbf{Categoria Tesoro}: Equipaggiamento da PNG (Tridente, Balestra Pesante con 10 Quadrelli, L)\\
\textbf{Descrizione}\\
Famelici e crudeli, i sahuagin sono, sfortunatamente, tra le razze oceaniche più prosperose. Grandi città sono state costruite da questa razza nelle buie profondità delle fosse oceaniche, e alcune fortezze sorgono nei pressi delle coste da dove lanciano assalti continui contro i nemici che respirano aria che vivono vicino alla riva. Orgogliosi e bellicosi, i sahuagin si alleano raramente con altri, e vedono le altre razze acquatiche, come aboleth, marinidi e simili come concorrenti. Le sole creature che sembrano rispettare oltre ai loro simili sono gli squali; in questi implacabili predatori, infatti, i sahuagin rivedono molto di loro stessi. Un sahuagin è alto 2,1 metro e pesa circa 125 kg.

I sahuagin sono soggetti a mutazioni genetiche e quando nasce un mutante assurge quasi sempre ai ranghi nobiliari o di comando nella società. La mutazione sahuagin più comune consiste in un paio di braccia extra (che concedono due attacchi addizionali con gli artigli o la possibilità di maneggiare più armi). Alcuni parlano dei rari malenti sahuagin che non sembrano uomini squalo ma elfi acquatici, malgrado condividano la sete di sangue e la natura crudele dei loro simili. I malenti spesso servono come spie o assassini i governanti sahuagin, ma si narra di intere tribù composte di malenti in remote zone del mare.

\mostro{Salamandra}
\noindent
\begin{description}[noitemsep, topsep=0pt, parsep=0pt, partopsep=0pt, leftmargin=0cm, labelwidth=2.2cm]
	\item[\textbf{Taglia/Tipo:}] Grande elementale, malvagio
	\item[\textbf{Caratt.:}] \resizebox{0.5\linewidth+1.8cm}{!}{For 4 Des 2 Cos 2 Int 0 Sag 0 Car 1}
	\item[\textbf{Punti Ferita:}] 107,  \textbf{Difesa:} 20,  \textbf{Iniziativa:} +2
	\item[\textbf{Movimento:}] 9 m
	\item[\textbf{Tiri Salvez.:}] \resizebox{0.5\linewidth+1.8cm}{!}{Tempra +7, Riflessi +7, Volontà +5}
	\item[\textbf{Res. Danni:}] da arma non magica
	\item[\textbf{Sensi:}] Scurovisione 18 m
	\item[\textbf{Linguaggi:}] Ignan
	\item[\textbf{Sfida:}] 5 (1800 PX)\smallskip
\end{description}

\emph{\textbf{Armi Riscaldate.}} Qualsiasi arma da mischia metallica che la salamandra impugni infligge 3 (1d6) danni da fuoco aggiuntivi per colpo (già incluso nell'attacco).

\emph{\textbf{Corpo Riscaldato.}} Una creatura che entri a contatto con la salamandra o la colpisce con un attacco da mischia mentre si trova entro 1 metro da essa subisce 7 (2d6) danni da fuoco.

\textbf{Azioni}

\emph{\textbf{Multiattacco.}} La salamandra effettua due attacchi: uno con la lancia e uno con la coda.

\emph{\textbf{Coda.} Attacco con arma da mischia}: +6 a colpire, portata 3 m, un bersaglio.

\emph{Colpisce:} 11 (2d6 + 4) danni contundenti più 7 (2d6) danni da fuoco, e il bersaglio è afferrato (DC 14 per fuggire). Fino al termine dell'afferrare la salamandra può colpire automaticamente il bersaglio con la coda e non può effettuare attacchi di coda contro altri bersagli.

\emph{\textbf{Lancia.} Attacco con arma da mischia o a Distanza}: +5 a colpire, portata 1 m, gittata 6m, un bersaglio.

\emph{Colpisce:} 11 (2d6 + 4) danni perforanti, o 13 (2d8 +4) danni perforanti se usata con due mani per effettuare un attacco da mischia, più 3 (1d6) danni da fuoco.

\textbf{Reazione: \emph{Attacco d'opportunità}}: la salamandra effettua un attacco ad una creatura che attraversi o esca dalla sua portata di 1 metro.

\emph{\textbf{Arrabbiato:}} la Salamadra concentra le sue fiamme in un attacco a distanza. Una creatura entro 9 metri deve effettuare un Tiro Salvezza su Riflessi DC 18 per dimezzare il danno. La creatura viene colpita da un globo di fiamme che causa 4d6 di danno da fuoco. Costa 2 Azioni.

\textbf{Ecologia}
Ambiente: Qualsiasi (Piano del Fuoco)\\
Organizzazione: Solitario, coppia o gruppo (3-5)\\
\textbf{Categoria Tesoro}: Standard (Lancia, P)\\
\textbf{Descrizione}\\
Le Salamandre sono native del Piano del Fuoco, dove le loro legioni di fieri combattenti sono molto temute dagli altri abitanti del Piano. Poiché molte delle più forti Razze Elementali del Fuoco Schiavizzano le Salamandre per la loro Abilità nella metallurgia e capacità combattiva, le Salamandre odiano gli Efreet e gli altri con fervore.

Anche se i loro nascondigli superano i 250 gradi C di temperatura, le Salamandre possono tollerare temperature più basse. Generalmente lo fanno se costrette, e sono anche più burbere e irascibili del normale in questi ambienti. Sebbene provenga dal Piano del Fuoco, la Razza delle Salamandre si identifica di più con l'Abisso, e ha un grande rispetto per i Demoni (in particolare quelli associati col fuoco, come i Balor e certi Signori dei Demoni legati alle fiamme). Per questo non è insolito incontrare un grosso gruppo di Salamandre nell'Abisso.

Le Salamandre sono spesso evocate nel Piano Materiale per servire come guardiani o, più comunemente, come fabbricanti di Armature, Armi e altri oggetti metallurgici, dato che la loro Abilità in questo campo è leggendaria. Le Salamandre infestano anche quelle aree del Piano Materiale dove il confine tra questo mondo e il Piano del Fuoco si è fatto labile, come vicino e dentro i Vulcani.

Abitando zone così estreme, le Salamandre posseggono solo tesori che resistono alle alte temperature, come Spade, Armature, gioielli, Verghe e altri oggetti che hanno un alto punto di fusione. La società delle Salamandre è crudele e basata sul potere e sulla capacità di soggiogare chi è inferiore a loro. Gli esseri inferiori alle Salamandre che causano problemi affrontano una morte lenta e dolorosa.

\mostro{Satiro}
\noindent
\begin{description}[noitemsep, topsep=0pt, parsep=0pt, partopsep=0pt, leftmargin=0cm, labelwidth=2.2cm]
	\item[\textbf{Taglia/Tipo:}] Media fatato, caotico
	\item[\textbf{Caratt.:}] \resizebox{0.5\linewidth+1.8cm}{!}{For 1 Des 3 Cos 0 Int 1 Sag 0 Car 2}
	\item[\textbf{Punti Ferita:}] 24,  \textbf{Difesa:} 15,  \textbf{Iniziativa:} +3
	\item[\textbf{Movimento:}] 12 m
	\item[\textbf{Tiri Salvez.:}] \resizebox{0.5\linewidth+1.8cm}{!}{Tempra +3, Riflessi +3, Volontà +3}
	\item[\textbf{Comp.:}] Furtività +5, Intrattenere +6, Consapevolezza +2
	\item[\textbf{Linguaggi:}] Comune, Elfico, Silvano
	\item[\textbf{Sfida:}] 1/2 (100 PX)\smallskip
\end{description}

\emph{\textbf{Resistenza alla Magia.}} Il satiro ha +1d6 ai Tiri Salvezza contro incantesimi e altri effetti magici.

\textbf{Azioni}

\emph{\textbf{Incornata.} Attacco con arma da mischia}: +4 a colpire, portata 1 m, un bersaglio.

\emph{Colpisce:} 6 (2d4 + 1) danni contundenti.

\emph{\textbf{Spada Corta.} Attacco con arma da mischia}: +4 a colpire, portata 1 m, un bersaglio.

\emph{Colpisce:} 6 (1d6 + 3) danni perforanti.

\emph{\textbf{Arco Corto.} Attacco con arma a Distanza}: +3 a colpire, gittata 24m, un bersaglio.

\emph{Colpisce:} 6 (1d6 + 3) danni perforanti.

\textbf{Ecologia}\\
Ambiente: Foreste Temperate\\
Organizzazione: Solitario, coppia, banda (3-6) o festino (7-11)\\
\textbf{Categoria Tesoro}: Pugnale, Arco Corto più 20 Frecce, flauto di pan perfetto, S\\
\textbf{Descrizione}\\
I satiri, conosciuti in molte regioni come fauni, sono creature debosciate ed edoniste delle parti più profonde e primordiali delle foreste. Adorano il vino, la musica e i piaceri della carne, sono rinomati come libertini e bellimbusti che corteggiano le fanciulle sprovvedute e i pastorelli e si lasciano dietro una scia di spiegazioni imbarazzanti e gravidanze indesiderate.

Anche se i loro corpi sono quasi sempre quelli di uomini attraenti e ben proporzionati, le capacità seduttive dei satiri risiedono nel loro talento musicale. Con l'aiuto del suo flauto, un satiro è capace di tessere una vasta gamma di incantesimi melodici ideati per affascinare gli altri e farli accondiscendere ai suoi capricciosi desideri.

Oltre ad amoreggiare costantemente, i satiri spesso fungono da guardiani delle loro foreste, e quanti riescono a trasformare la lussuria del fauno in ira probabilmente si troveranno di fronte i più pericolosi tra gli animali che circondano il fauno. Inoltre, anche se i satiri tendono a mettere il loro divertimento al di sopra dei diritti altrui, non covano alcun risentimento contro quelli che seducono.

I bambini nati da questi incontri sono sempre satiri di sangue puro e vengono generalmente portati via dai loro sfrenati padri subito dopo la nascita.

\mostro{Scheletro}
\noindent
\begin{description}[noitemsep, topsep=0pt, parsep=0pt, partopsep=0pt, leftmargin=0cm, labelwidth=2.2cm]
	\item[\textbf{Taglia/Tipo:}] Media non morto, malvagio
	\item[\textbf{Caratt.:}] \resizebox{0.5\linewidth+1.8cm}{!}{For 0 Des 2 Cos 2 Int -2 Sag -1 Car -3}
	\item[\textbf{Punti Ferita:}] 19,  \textbf{Difesa:} 14,  \textbf{Iniziativa:} +2
	\item[\textbf{Movimento:}] 9 m
	\item[\textbf{Tiri Salvez.:}] \resizebox{0.5\linewidth+1.8cm}{!}{Tempra +3, Riflessi +3, Volontà +3}
	\item[\textbf{Res. Danni:}] perforante, tagliente
	\item[\textbf{Imm. Danni:}] Veleno
	\item[\textbf{Immunità:}] affaticato, sanguinamento
	\item[\textbf{Sensi:}] Scurovisione 18 m
	\item[\textbf{Linguaggi:}] comprende tutte le lingue che parlava in vita ma non può parlare
	\item[\textbf{Sfida:}] 1/4 (50 PX)\smallskip
\end{description}

\emph{\textbf{Natura Non Morta.}} Lo scheletro non necessita aria, cibo, bevande o sonno.

\emph{\textbf{Spada Corta.} Attacco con arma da mischia}: +4 a colpire, portata 1 m, un bersaglio.

\emph{Colpisce:} 5 (1d6 + 2) danni perforanti.

\emph{\textbf{Arco Corto.} Attacco con arma a Distanza}: +4 a colpire, gittata 24m, un bersaglio.\emph{Colpisce:} 5 (1d6 + 2) danni perforanti.

\textbf{Ecologia}\\
Ambiente: Qualsiasi\\
Organizzazione: Qualsiasi\\
\textbf{Categoria Tesoro}: Nessuno (Giaco di Maglia Rotto, Scimitarra Rotta)\\
\textbf{Descrizione}\\
Gli scheletri sono ossa di morti animate, portate alla non vita da magie sacrileghe. Per la maggior parte, gli scheletri sono automi privi di volontà.

\mostro{Scheletro Campione}
\noindent
\begin{description}[noitemsep, topsep=0pt, parsep=0pt, partopsep=0pt, leftmargin=0cm, labelwidth=2.2cm]
	\item[\textbf{Taglia/Tipo:}] Media non morto, malvagio
	\item[\textbf{Caratt.:}] \resizebox{0.5\linewidth+1.8cm}{!}{For 4 Des 1 Cos 3 Int -2 Sag -1 Car -3}
	\item[\textbf{Punti Ferita:}] 70,  \textbf{Difesa:} 17,  \textbf{Iniziativa:} +1
	\item[\textbf{Movimento:}] 12 m
	\item[\textbf{Tiri Salvez.:}] \resizebox{0.5\linewidth+1.8cm}{!}{Tempra +6, Riflessi +4, Volontà +3}
	\item[\textbf{Imm. Danni:}] Veleno
	\item[\textbf{Res. Danni:}] perforante, tagliente, Elettricità, Fuoco
	\item[\textbf{Immunità:}] affaticato, sanguinamento
	\item[\textbf{Sensi:}] Scurovisione 18 m
	\item[\textbf{Linguaggi:}] comprende l'Expiran, ma non può parlare
	\item[\textbf{Sfida:}] 3 (700 PX)\smallskip
\end{description}

\emph{\textbf{Natura Non Morta.}} Lo scheletro non necessita aria, cibo, bevande o sonno.

\textbf{Azioni}

\emph{\textbf{Ascia Bipenne.} Attacco con arma da mischia}: +6 a colpire, portata 1 m, un bersaglio.

\emph{Colpisce:} 12 (1d12 + 4) danni taglienti.

\mostro{Scheletro di Saurovallo da Guerra}
\noindent
\begin{description}[noitemsep, topsep=0pt, parsep=0pt, partopsep=0pt, leftmargin=0cm, labelwidth=2.2cm]
	\item[\textbf{Taglia/Tipo:}] Grande non morto, malvagio
	\item[\textbf{Caratt.:}] \resizebox{0.5\linewidth+1.8cm}{!}{For 4 Des 1 Cos 2 Int -4 Sag -1 Car -3}
	\item[\textbf{Punti Ferita:}] 24,  \textbf{Difesa:} 13,  \textbf{Iniziativa:} +1
	\item[\textbf{Movimento:}] 18 m
	\item[\textbf{Tiri Salvez.:}] \resizebox{0.5\linewidth+1.8cm}{!}{Tempra +3, Riflessi +3, Volontà +3}
	\item[\textbf{Res. Danni:}] perforante, tagliente
	\item[\textbf{Imm. Danni:}] Veleno
	\item[\textbf{Immunità:}] affaticato, sanguinamento
	\item[\textbf{Sensi:}] Scurovisione 18 m
	\item[\textbf{Sfida:}] 1/2 (100 PX)\smallskip
\end{description}

\emph{\textbf{Natura Non Morta.}} Lo scheletro non necessita aria, cibo, bevande o sonno.

\textbf{Azioni}

\emph{\textbf{Zoccoli.} Attacco con arma da mischia}: +5 a colpire, portata 1 m, un bersaglio.

\emph{Colpisce:} 11 (2d6 + 4) danni contundenti.

\mostro{Segugio Infernale}
\noindent
\begin{description}[noitemsep, topsep=0pt, parsep=0pt, partopsep=0pt, leftmargin=0cm, labelwidth=2.2cm]
	\item[\textbf{Taglia/Tipo:}] Media immondo, malvagio
	\item[\textbf{Caratt.:}] \resizebox{0.5\linewidth+1.8cm}{!}{For 3 Des 1 Cos 2 Int -2 Sag 1 Car -2}
	\item[\textbf{Punti Ferita:}] 70,  \textbf{Difesa:} 17,  \textbf{Iniziativa:} +1
	\item[\textbf{Movimento:}] 15 m
	\item[\textbf{Tiri Salvez.:}] \resizebox{0.5\linewidth+1.8cm}{!}{Tempra +5, Riflessi +4, Volontà +4}
	\item[\textbf{Comp.:}] Consapevolezza +5
	\item[\textbf{Imm. Danni:}] Fuoco
	\item[\textbf{Sensi:}] Scurovisione 18 m
	\item[\textbf{Linguaggi:}] comprende l'Infernale ma non può parlare
	\item[\textbf{Sfida:}] 3 (700 PX)\smallskip
\end{description}

\emph{\textbf{Udito e Olfatto Affinato.}} Il segugio ha +1d6 nelle prove di Consapevolezza basate su udito od olfatto.

\emph{\textbf{Tattiche di Branco.}} Il segugio ha +1d6 ai tiri per colpire contro una creatura se almeno uno degli alleati del segugio si trova entro 1 metro dalla creatura e quell'alleato non è inabile.

\textbf{Azioni}

\emph{\textbf{Morso.} Attacco con arma da mischia}: +6 a colpire, portata 1 m, un bersaglio.

\emph{Colpisce:} 7 (1d6 + 3) danni perforanti più 7 (2d6) danni da fuoco.

\emph{\textbf{Soffio Infuocato (Ricarica 5-6).}} Il segugio esala fuoco in un cono di 5 metri. Ogni creatura in quell'area deve effettuare un Tiro Salvezza di Riflessi DC 14, e subire 21 (6d6) danni da fuoco se fallisce il Tiro Salvezza, o la metà di questi danni se lo riesce.

\mostro{Androsfinge}
\noindent
\begin{description}[noitemsep, topsep=0pt, parsep=0pt, partopsep=0pt, leftmargin=0cm, labelwidth=2.2cm]
	\item[\textbf{Taglia/Tipo:}] Grande mostruosità, legale
	\item[\textbf{Caratt.:}] \resizebox{0.5\linewidth+1.8cm}{!}{For 6 Des 0 Cos 5 Int 3 Sag 4 Car 6}
	\item[\textbf{Punti Ferita:}] 338,  \textbf{Difesa:} 34,  \textbf{Iniziativa:} +3
	\item[\textbf{Movimento:}] 12 m, volo 18 m
	\item[\textbf{Tiri Salvez.:}] \resizebox{0.5\linewidth+1.8cm}{!}{\resizebox{0.5\linewidth+1.8cm}{!}{Tempra +22, Riflessi +17, Volontà +21}}
	\item[\textbf{Comp.:}] Arcana +9, Religione +15
	\item[\textbf{Imm. Danni:}] da arma non magica
	\item[\textbf{Immunità:}] affascinato, spaventato
	\item[\textbf{Sensi:}] visione del vero 36 m
	\item[\textbf{Linguaggi:}] Comune, Sfinge
	\item[\textbf{Sfida:}] 17 (18000 PX)\smallskip
\end{description}

\emph{\textbf{Armi Magiche.}} Gli attacchi con armi della sfinge sono magici.

\emph{\textbf{Imperscrutabile.}} La sfinge è immune a qualsiasi effetto in grado di percepirne le emozioni o leggerne i pensieri oltre che a qualsiasi incantesimo di divinazione che rifiuti. Le prove Percepire Emozioni per discernere le intenzioni o la sincerità della sfinge hanno -1d6.


\begin{center}
	\includegraphics[width=0.5\textwidth]{immagini/ginosfinge.png}
\end{center}


\emph{\textbf{Incantesimi.}} La sfinge ha CM 12.
La sua caratteristica da incantatore è la Saggezza (DC del Tiro Salvezza degli incantesimi 30, +10 a colpire con attacchi con incantesimo). Non ha bisogno di componenti materiali per lanciare i suoi incantesimi. La sfinge tiene preparati i seguenti incantesimi:

Trucchetti (a volontà): \emph{\hyperlink{Fiamma Sacra}{Fiamma Sacra}, \hyperlink{Taumaturgia}{Taumaturgia}}

livello 1 (4 slot): \emph{\hyperlink{Comando}{Comando}, \hyperlink{Individuazione del Magico}{Individuazione del Magico}, \hyperlink{Conoscere i Tratti}{Conoscere i Tratti}}

livello 2 (3 slot): \emph{\hyperlink{Ristorare Inferiore}{Ristorare Inferiore}, \hyperlink{Zona di Verità}{Zona di Verità}}

livello 3 (3 slot): \emph{\hyperlink{Dissolvi Magie}{Dissolvi Magie}, \hyperlink{Lingue}{Lingue}}

livello 4 (3 slot): \emph{\hyperlink{Esilio}{Esilio}, \hyperlink{Libertà di Movimento}{Libertà di Movimento}}

livello 5 (2 slot): \emph{\hyperlink{Colpo Infuocato}{Colpo Infuocato}, \hyperlink{Ristorare Superiore}{Ristorare Superiore}}

livello 6 (1 slot): \emph{\hyperlink{Banchetto degli Eroi}{Banchetto degli Eroi}}

\textbf{Azioni}

\emph{\textbf{Multiattacco.}} La sfinge può effettuare due attacchi di artiglio.

\emph{\textbf{Artiglio.} Attacco con arma da mischia}: +13 a colpire, portata 1 m, un bersaglio.

\emph{Colpisce:} 17 (2d6 + 10) danni taglienti, 1 danno da Sanguinamento.

\emph{\textbf{Ruggito (3/Giorno).}} La sfinge emette un ruggito magico. Ogni volta che ruggisce prima di una nuova alba il ruggito è più forte e l'effetto è diverso, come dettagliato di seguito. Ogni creatura entro 150 metri dalla sfinge e capace di udirne il ruggito deve effettuare un Tiro Salvezza.

\textbf{Primo Ruggito.} Ogni creatura che fallisce un Tiro Salvezza su Volontà DC 30 resta spaventata per 1 minuto. Una creatura spaventata può ripetere il Tiro Salvezza al termine di ciascun suo round, terminandone l'effetto per sé, se lo riesce.

\textbf{Secondo Ruggito.} Ogni creatura che fallisce un Tiro Salvezza su Volontà DC 30 resta assordata e spaventata per 1 minuto. Una creatura spaventata è paralizzata e può ripetere il Tiro Salvezza al termine di ciascun suo round, terminandone l'effetto per sé, se lo riesce.

\textbf{Terzo Ruggito.} Ogni creatura effettua un Tiro Salvezza su Tempra DC 30. Chi fallisce il Tiro Salvezza subisce 44 (8d10) danni da suono ed è gettato prono. Se il Tiro Salvezza riesce, la creatura subisce la metà di questi danni e non viene gettata prona.

\textbf{Reazione: \emph{Attacco d'opportunità}}: la sfinge nero effettua un attacco con Artiglio ad una creatura che attraversi o esca dalla sua portata di 1 metro.

\textbf{Azioni Aggiuntive}

La sfinge può effettuare 3 Azioni aggiuntive, scelte tra le opzioni seguenti. Può usare solo un'opzione Aggiuntiva alla volta e solo al termine del round di un'altra creatura. La sfinge recupera le Azioni aggiuntive spese all'inizio del proprio round.

\textbf{Attacco di Artiglio.} La sfinge effettua un attacco di artiglio.

\textbf{Eseguire un Incantesimo (Costa 3 Azioni).} La sfinge lancia un incantesimo dalla lista degli incantesimi preparati, utilizzando uno slot incantesimo come di norma.

\textbf{Teletrasporto (Costa 2 Azioni).} La sfinge si teletrasporta magicamente, insieme a tutto l'equipaggiamento che sta indossando o trasportando, in uno spazio non occupato che possa vedere, fino a 36 metri di distanza.

\emph{\textbf{Arrabbiato:}} la Sfinge pone un indovinello. La creatura deve rispondere, usando tutte le sue azioni ed una risposta a round, entro 6 round, se sbaglia o non risponde deve effettuare un Tiro Salvezza su Volontà a DC 31 oppure rimanere paralizzata. Ogni round può tentare di nuovo il Tiro Salvezza nel tentativo di dare una risposta. Costa 1 Azione.

\textbf{Ecologia}\\
Ambiente: Colline o Deserti Caldi\\
Organizzazione: Solitario\\
\textbf{Categoria Tesoro}: C\\
\textbf{Descrizione}\\
Le androsfingi, le più potenti tra le sfingi comuni, ritengono di rappresentare tutto ciò che c'è di degno e nobile nella loro specie e si atteggiano come se il peso del mondo intero poggiasse sul loro buon esempio. Guardano le Criosfingi con sufficienza paternalistica, le Ieracosfingi con malcelato disgusto e le Ginosfingi come le uniche altre sfingi degne del loro tempo.

Le androsfingi ostentano una facciata scorbutica e astiosa nei confronti degli stranieri. Non si sforzano in alcun modo di celare il loro fastidio quando sono irritate. Tendono inoltre a essere gelose del loro territorio, anche se meno delle altre sfingi. Quasi inevitabilmente lanciano avvertimenti e proclami roboanti prima di attaccare, e rispettano quasi sempre un appello a trattare. Le androsfingi barattano informazioni e conversazioni, e non tesori, in cambio di un passaggio sicuro.

Le androsfingi sono alte 3,6 metri e pesano 500 kg.

\mostro{Ginosfinge}
\noindent
\begin{description}[noitemsep, topsep=0pt, parsep=0pt, partopsep=0pt, leftmargin=0cm, labelwidth=2.2cm]
	\item[\textbf{Taglia/Tipo:}] Grande mostruosità, legale
	\item[\textbf{Caratt.:}] \resizebox{0.5\linewidth+1.8cm}{!}{For 4 Des 2 Cos 3 Int 4 Sag 4 Car 4}
	\item[\textbf{Punti Ferita:}] 219,  \textbf{Difesa:} 28,  \textbf{Iniziativa:} +4
	\item[\textbf{Movimento:}] 12 m, volo 18 m
	\item[\textbf{Tiri Salvez.:}] \resizebox{0.5\linewidth+1.8cm}{!}{\resizebox{0.5\linewidth+1.8cm}{!}{Tempra +14, Riflessi +13, Volontà +15}}
	\item[\textbf{Comp.:}] Arcana +14, Religione +9, Storia +14
	\item[\textbf{Res. Danni:}] da arma non magica
	\item[\textbf{Immunità:}] affascinato, spaventato
	\item[\textbf{Sensi:}] visione del vero 36 m
	\item[\textbf{Linguaggi:}] Comune, Sfinge
	\item[\textbf{Sfida:}] 11 (7200 PX)\smallskip
\end{description}

\emph{\textbf{Armi Magiche.}} Gli attacchi con armi della sfinge sono magici.

\emph{\textbf{Imperscrutabile.}} La sfinge è immune a qualsiasi effetto in grado di percepirne le emozioni o leggerne i pensieri, oltre che a qualsiasi incantesimo di divinazione che rifiuti. Le prove di Saggezza (Percepire Inganni) per discernere le intenzioni o la sincerità della sfinge hanno -1d6.

\emph{\textbf{Incantesimi.}} La sfinge ha CM 9. La sua abilità da incantatore è l'Intelligenza (DC del Tiro Salvezza degli incantesimi 24. Non ha bisogno di componenti materiali per eseguire i suoi incantesimi. La sfinge tiene preparati i seguenti incantesimi: Trucchetti (a volontà): \emph{\hyperlink{Illusione Minore}{Illusione Minore}, \hyperlink{Mano Magica}{Mano Magica},} \emph{\hyperlink{Prestidigitazione}{Prestidigitazione}}

livello 1 (4 slot): \emph{\hyperlink{Identificare}{Identificare}, \hyperlink{Individuazione del Magico}{Individuazione del Magico}, \hyperlink{Scudo}{Scudo}}

livello 2 (3 slot): \emph{\hyperlink{Localizza Oggetto}{Localizza Oggetto}, \hyperlink{Oscurità}{Oscurità}, \hyperlink{Suggestione}{Suggestione}}

livello 3 (3 slot): \emph{\hyperlink{Dissolvi Magie}{Dissolvi Magie}, \hyperlink{Lingue}{Lingue}, \hyperlink{Rimuovi Maledizione}{Rimuovi Maledizione}}

livello 4 (3 slot): \emph{\hyperlink{Esilio}{Esilio}, \hyperlink{Invisibilità Superiore}{Invisibilità Superiore}}

livello 5 (2 slot): \emph{\hyperlink{Conoscenza delle Leggende}{Conoscenza delle Leggende}}

\textbf{Azioni}

\emph{\textbf{Multiattacco.}} La sfinge può effettuare due attacchi di artiglio.

\emph{\textbf{Artiglio.} Attacco con arma da mischia}: +10 a colpire, portata 1 m, un bersaglio.

\emph{Colpisce:} 13 (2d8 + 4) danni taglienti, 1 danno da Sanguinamento.

\textbf{Reazione: \emph{Attacco d'opportunità}}: la sfinge nero effettua un attacco con Artiglio ad una creatura che attraversi o esca dalla sua portata di 1 metro.

\textbf{Azioni Aggiuntive}

La sfinge può effettuare 3 Azioni aggiuntive, scelte tra le opzioni seguenti. Può usare solo un'opzione Aggiuntiva alla volta e solo al termine del round di un'altra creatura. La sfinge recupera le Azioni aggiuntive spese all'inizio del proprio round.

\textbf{Attacco di Artiglio.} La sfinge effettua un attacco di artiglio.

\textbf{Eseguire un Incantesimo (Costa 3 Azioni).} La sfinge esegue un incantesimo dalla lista degli incantesimi preparati, utilizzando uno slot incantesimo come di norma.

\textbf{Teletrasporto (Costa 2 Azioni).} La sfinge si teletrasporta magicamente, insieme a tutto l'equipaggiamento che sta indossando o trasportando, in uno spazio non occupato che possa vedere, fino a 36 metri di distanza.

\textbf{Ecologia}
Ambiente: Deserti e colline caldi\\
Organizzazione: Solitario, coppia o culto (3-6)\\
\textbf{Categoria Tesoro}: C\\
\textbf{Descrizione}\\
Anche se esistono diversi tipi di sfinge, quella alla quale gli studiosi si riferiscono come Ginosfinge (un nome che molte sfingi trovano offensivo) è una creatura saggia e maestosa ma al contempo terrificante se arrabbiata. Meno moraliste delle loro controparti maschili (le Androsfingi, creature totalmente differenti da quella presentata qui), le sfingi sono prudenti e metodiche quando prendono delle decisioni, e sono orgogliose della loro fredda logica e della loro imparzialità.

Hanno poca pazienza con le varianti inferiori di sfingi, considerandole poco più che animali.

Le sfingi amano gli enigmi e gli indovinelli complicati, e fanno tesoro di fatti insoliti e dilemmi arcani molto più che di oro o gemme.

Pur non essendo grandi studiose in senso tradizionale, il grande apprezzamento delle sfingi per gli enigmi le porta a compiere ricerche in una grande varietà di materie, rendendole spesso una preziosa fonte di informazioni, specialmente quando fanno uso delle loro capacità magiche. Di solito sono felici di avere contatti con altre razze, ed offrono regolarmente beni materiali in cambio di informazioni o di indovinelli nuovi ed interessanti. Sono eccellenti guardiane di templi, tombe ed altri luoghi importanti, fintanto che vengono intrattenute in maniera adeguata. Le sfingi danno grande importanza alla gentilezza, ma possono essere capricciose: possono decidere altruisticamente di dividere i loro ultimi enigmi con dei viaggiatori ma non ci pensano due volte a divorarli se non vi prestano abbastanza attenzione o non forniscono alcun indizio utile alla loro risoluzione.

Una tipica sfinge è lunga 3 metri e pesa circa 400 kg. Anche se le loro ali possono tenerle in aria per lunghi periodi di tempo, sono delle volatrici scarse, e preferiscono atterrare prima di iniziare a combattere, attaccando con i loro poderosi artigli. Nonostante siano estremamente territoriali, le sfingi tendono ad avvisare gli intrusi varie volte prima di attaccare.

\mostro{Sibilante}
\noindent
\begin{description}[noitemsep, topsep=0pt, parsep=0pt, partopsep=0pt, leftmargin=0cm, labelwidth=2.2cm]
	\item[\textbf{Taglia/Tipo:}] Grande mostruosità, caotico
	\item[\textbf{Caratt.:}] \resizebox{0.5\linewidth+1.8cm}{!}{For 2 Des 1 Cos 1 Int -3 Sag 0 Car -2}
	\item[\textbf{Punti Ferita:}] 51,  \textbf{Difesa:} 15,  \textbf{Iniziativa:} +1
	\item[\textbf{Movimento:}] 6 m, arrampicarsi 6 m
	\item[\textbf{Tiri Salvez.:}] \resizebox{0.5\linewidth+1.8cm}{!}{Tempra +3, Riflessi +3, Volontà +3}
	\item[\textbf{Comp.:}] Furtività +4, Consapevolezza +3
	\item[\textbf{Sensi:}] Scurovisione 18 m
	\item[\textbf{Sfida:}] 2 (450 PX)\smallskip
\end{description}

\textbf{Azioni}

\emph{\textbf{Multiattacco.}} Il Sibilante può eseguire due attacchi con gli artigli oppure un colpo con la coda.

\emph{\textbf{Artiglio.} Attacco con arma da mischia}: +5 a colpire, portata 1 m, un bersaglio.

\emph{Colpisce:} 6 (1d8+2) danno tagliente.

\emph{\textbf{Frustata di Coda}}: il Sibilante agita la lunga coda, +5 al colpire, portata 3 metri, un bersaglio.

\emph{Colpisce:} 11 (2d8+2) danni contundenti e 7 (2d6) da taglio. L'eventuale armatura o scudo viene danneggiata abbassando di 1 la Difesa dell'avversario. Il danno all'armatura non si considera permanente.

\textbf{Reazioni}

\textbf{Reazione: \emph{Atterrare}}: quando il Sibilante è attaccato da una creatura nella portata della sua coda questa viene sferzata obbligando l'attaccante, dopo la risoluzione del suo attacco, ad effettuare un Tiro Salvezza su Tempra o Riflessi a DC 14 o subire 7 (2d6) di danni contundenti e cadere prono. Se il Tiro Salvezza riesce subisco solo metà danno e non è prona.

\textbf{Ecologia}\\
Ambiente: Caverne\\
Organizzazione: Solitario, coppia o nido (2-4)\\
\textbf{Categoria Tesoro}: Accidentale\\
\textbf{Descrizione}\\
I Sibilanti, chiamati così per via del rumore che fa la loro coda agitandosi è una creatura molto particolare. Assomiglia a prima vista ad un coccodrillo, lungo circa 5 metri di cui 4 di coda ma ha 8 zampe ed il muso corto e appiattito. La coda estremamente robusta finisce con una specie di uncino che il Sibilante usa per colpire, uccidere ed afferrare i nemici quasi fosse una zampa aggiuntiva.

Di colore grigio scuro, marrone, preferiscono nascondersi nell'oscurità ed attaccare quando affamati o per difendere il loro territorio. Cercano di tenere le distanze in combattimento e se gravemente feriti scappano arrampicandosi sulle pareti.

\mostro{Spettro}
\noindent
\begin{description}[noitemsep, topsep=0pt, parsep=0pt, partopsep=0pt, leftmargin=0cm, labelwidth=2.2cm]
	\item[\textbf{Taglia/Tipo:}] Media non morto, malvagio
	\item[\textbf{Caratt.:}] \resizebox{0.5\linewidth+1.8cm}{!}{For -5 Des 2 Cos 0 Int 0 Sag 0 Car 2}
	\item[\textbf{Punti Ferita:}] 33,  \textbf{Difesa:} 15,  \textbf{Iniziativa:} +2
	\item[\textbf{Movimento:}] 0 m, volo 16 m, fluttuare
	\item[\textbf{Tiri Salvez.:}] \resizebox{0.5\linewidth+1.8cm}{!}{Tempra +3, Riflessi +3, Volontà +3}
	\item[\textbf{Comp.:}] Furtività +8,Consapevolezza +3
	\item[\textbf{Res. Danni:}] Acido, Freddo, Fuoco, Elettricità, Suono, da Vuoto, da arma non magica
	\item[\textbf{Immunità:}] affascinato, spaventato, affaticato, afferrato, paralizzato, pietrificato, veleno, prono, ristretto
	\item[\textbf{Sensi:}] Scurovisione 18 m
	\item[\textbf{Lingue:}] Expiran
	\item[\textbf{Sfida:}] 1 (200 PX)\smallskip
\end{description}

\textbf{Movimento Incorporeo}. Lo spettro può muoversi attraverso creature ed oggetti come se fosse terreno difficile. Subisce 5 (1d10) danni se termina il suo turno all'interno di un oggetto.

\textbf{Sensibilità alla luce solare}. Mentre è illuminato a luce solare lo spettro ha -1d6 ai Tiri per Colpire e alle prove di Consapevolezza.

\textbf{Azioni}

\emph{\textbf{Tocco gelido.}} Attacco a contatto: +4 a colpire, portata 1 m, un bersaglio.

\emph{Colpisce:} 10 danni (3d6) da Vuoto. La creatura perde il medesimo ammontare dai Punti Ferita Massimi. Lo spettro recupera 2 Punti Ferita.

\textbf{Ecologia}\\
Ambiente: Qualsiasi\\
Organizzazione: Solitario\\
\textbf{Categoria Tesoro}: nessuno\\
\textbf{Descrizione}\\
Gli spettri sono non morti malvagi che odiano la luce del sole e gli esseri viventi. La loro genesi è spesso dovuta alla morte violenta di assassini e malvagi. Come i fantasmi, gli spettri infestano i posti dove sono morti e cercano di portare altri con loro.

Uno spettro assomiglia molto a come era in vita e può essere facilmente riconosciuto da coloro che conoscevano l’individuo o ne avevano visto il volto nei dipinti o nei disegni.

\mostro{Spiritello}
\noindent
\begin{description}[noitemsep, topsep=0pt, parsep=0pt, partopsep=0pt, leftmargin=0cm, labelwidth=2.2cm]
	\item[\textbf{Taglia/Tipo:}] Minuscola fatato, buono
	\item[\textbf{Caratt.:}] \resizebox{0.5\linewidth+1.8cm}{!}{For -4 Des 4 Cos 0 Int 2 Sag 1 Car 0}
	\item[\textbf{Punti Ferita:}] 19,  \textbf{Difesa:} 16,  \textbf{Iniziativa:} +4
	\item[\textbf{Movimento:}] 3 m, volo 12 m
	\item[\textbf{Tiri Salvez.:}] \resizebox{0.5\linewidth+1.8cm}{!}{Tempra +3, Riflessi +4, Volontà +3}
	\item[\textbf{Comp.:}] Furtività +8 (la prova è fatta con -1d6 se lo spiritello sta volando),Consapevolezza +3
	\item[\textbf{Linguaggi:}] Comune, Elfico, Silvano
	\item[\textbf{Sfida:}] 1/4 (50 PX)\smallskip
\end{description}

\textbf{Azioni}

\emph{\textbf{Spada Lunga.} Attacco con arma da mischia}: +3 a colpire,

portata 1 m, un bersaglio.

\emph{Colpisce:} 1 danno tagliente.

\emph{\textbf{Arco Corto.} Attacco con arma a Distanza}: +5 a colpire, gittata 12 m, un bersaglio.

\emph{Colpisce:} 1 danno perforante. Se il bersaglio è una creatura, deve riuscire un Tiro Salvezza di Tempra DC 10 o restare avvelenata, -1 Forza e Destrezza, per 1 minuto. Se il risultato di questo Tiro Salvezza è 5 o meno, il bersaglio cade privo di sensi per la stessa durata, o finché subisce danni o un'altra creatura usa un'Azione per risvegliarlo.

\emph{\textbf{Invisibilità.}} Lo spiritello resta invisibile finché non attacca o termina la sua concentrazione. Qualsiasi cosa che lo spiritello stia trasportando o indossando resta invisibile finché rimane in contatto con lo spiritello.

\emph{\textbf{Vista del Cuore.}} Lo spiritello entra in contatto con una creatura e ne apprende l'attuale stato emotivo. Se il bersaglio fallisce un Tiro Salvezza di Tempra DC 10, lo spiritello apprende anche i Tratti della creatura. Celestiali, immondi e non morti falliscono automaticamente questo Tiro Salvezza.
\textbf{Descrizione}\\
Gli spiritelli si riuniscono in gruppi nelle profondità di regioni boschive, uniti nella causa per proteggere la natura. Intere tribù di spiritelli si sono dichiarate protettrici di una determinata persona, di un luogo o di una creatura di particolare rilievo nelle loro terre, anche nel caso in cui l'essere non desideri o non necessiti di alcuna protezione.

Il corpo di uno spiritello è luminoso per natura, sebbene la creatura possa variare il colore e l'intensità della luce emessa dal suo corpo a piacimento. Subito dopo la sua morte, il corpo di uno spiritello si dissolve in una nebbia luccicante. Gli spiritelli sono i più piccoli tra i folletti, alti poco più di 22 centimetri e di un peso che raramente supera 1 kg.

Sotto molti aspetti gli spiritelli sono più primitivi della maggior parte dei folletti. Apprezzano la compagnia dei propri simili, ma tendono a diffidare degli altri folletti e presumono che qualsiasi umanoide o creatura che non hanno espressamente scelto di proteggere voglia far loro del male. Persino gli animali vengono da loro solitamente considerati pericolosi. La ragione di questa diffidenza è per buona parte dovuta alla taglia minuscola di queste creature, che le rende facili prede per i predatori. Pertanto la reazione iniziale di uno spiritello di fronte a un pericolo è darsi alla fuga: in genere utilizza le sue capacità magiche per rallentare o distrarre gli inseguitori, e in seguito si affida alla sua velocità di volare e alla sua taglia per riuscire a fuggire.

Sebbene gli spiritelli di per sé abbiano una natura incolta e selvaggia, hanno una sana curiosità per tutte le cose dotate di magia innata. Sono particolarmente attratti dai luoghi di grande potere magico latente, quali le rovine di antichi templi. Questa curiosità li rende anche insolitamente adatti al ruolo di famigli. Un incantatore caotico di 5° livello può ottenere uno spiritello come famiglio se ha l'Abilità Famiglio.

\mostro{Strige (Uccello Stigeo)}
\noindent
\begin{description}[noitemsep, topsep=0pt, parsep=0pt, partopsep=0pt, leftmargin=0cm, labelwidth=2.2cm]
	\item[\textbf{Taglia/Tipo:}] Minuscola bestia, disallineato
	\item[\textbf{Caratt.:}] \resizebox{0.5\linewidth+1.8cm}{!}{For -3 Des 3 Cos 0 Int -4 Sag -1 Car -2}
	\item[\textbf{Punti Ferita:}] 17,  \textbf{Difesa:} 15,  \textbf{Iniziativa:} +3
	\item[\textbf{Movimento:}] 3 m, volo 12 m
	\item[\textbf{Tiri Salvez.:}] \resizebox{0.5\linewidth+1.8cm}{!}{Tempra +3, Riflessi +3, Volontà +3}
	\item[\textbf{Sensi:}] Scurovisione 18 m
	\item[\textbf{Sfida:}] 1/8 (25 PX)\smallskip
\end{description}

\textbf{Azioni}

\emph{\textbf{Risucchio di Sangue.} Attacco con arma da mischia}: +3 a colpire, portata 1 m, una creatura.

\emph{Colpisce:} 5 (1d4 + 3) danni perforanti e lo strige si attacca al bersaglio. Mentre è attaccato, lo strige non attacca. Invece, all'inizio di ciascun round dello strige, il bersaglio perde 5 (1d4 + 3) Punti Ferita a causa della perdita di sangue.

Lo strige può staccarsi spendendo 1 Azione. Lo fa automaticamente dopo aver risucchiato 10 Punti Ferita dal bersaglio o alla morte del bersaglio. Una creatura, compreso il bersaglio, può usare una Azione per staccare lo strige.

\textbf{Ecologia}
Ambiente: Paludi temperate e calde\\
Organizzazione: Solitario, colonia (2-4), stormo (5-8), nugolo (9-14) o sciame (15-40)\\
\textbf{Categoria Tesoro}: Nessuno\\
\textbf{Descrizione}\\
Gli strige sono pericolosi succhiasangue che infestano le paludi e predano animali selvatici, bestiame ed ignari viaggiatori. Pur essendo deboli individualmente, sciami di queste creature sono capaci di prosciugare un uomo in pochi minuti, lasciando dietro a loro solo un cadavere essiccato.

più simili ai mammiferi che agli insetti, gli strige si alzano in volo con le loro quattro ali di carne, cercando prede a sangue caldo. Spesso si nascondono vicino a pozze di acqua bevibile aspettando che i viaggiatori abbassino la guardia per poi attaccarli e bere a sazietà, conficcando le loro proboscidi nelle vene scoperte. Dopo essersi nutriti, volano via a nascondersi tra la fanghiglia e tra i canneti per deporre le loro uova e riposare finché la fame non li spinge a cacciare di nuovo.

Di solito gli strige sono lunghi circa 30 centimetri, con un'apertura alare di circa il doppio, e pesano meno di 0,5 kg. Sono color rosso ruggine o marrone rossastro, ed hanno il ventre color giallo sporco, ma quelli che non si sono nutriti adeguatamente sono di colore rosa pallido.

\pdfbookmark[3]{T}{T}

\mostro{Tarrasque}
\noindent
\begin{description}[noitemsep, topsep=0pt, parsep=0pt, partopsep=0pt, leftmargin=0cm, labelwidth=2.2cm]
	\item[\textbf{Taglia/Tipo:}] Colossale mostruosità, disallineato
	\item[\textbf{Caratt.:}] \resizebox{0.5\linewidth+1.8cm}{!}{For 10 Des 0 Cos 10 Int -2 Sag 0 Car 0}
	\item[\textbf{Punti Ferita:}] 615,  \textbf{Difesa:} 52,  \textbf{Iniziativa:} +0
	\item[\textbf{Movimento:}] 24 m
	\item[\textbf{Tiri Salvez.:}] \resizebox{0.5\linewidth+1.8cm}{!}{\resizebox{0.5\linewidth+1.8cm}{!}{Tempra +40, Riflessi +30, Volontà +30}}
	\item[\textbf{Imm. Danni:}] Fuoco, Veleno, Elettricità; armi +2
	\item[\textbf{Immunità:}] affascinato, paralizzato, spaventato, affaticato
	\item[\textbf{Sensi:}] Vista Cieca 36 m
	\item[\textbf{Sfida:}] 30 (155000 PX)\smallskip
\end{description}

\emph{\textbf{Carapace Riflettente.}} Ogni volta che il Tarrasque è il bersaglio di un incantesimo \emph{\hyperlink{Dardo arcano}{Dardo arcano} o \hyperlink{Fulmine}{Fulmine}} questo viene ignorato e riflesso sull'origine. Per altri incantesimi a linea, o un incantesimi che richiedono un tiro per colpire a gittata, tira un d6. Da 1 a 5, il Tarrasque lo ignora. Con 6, il Tarrasque lo ignora e l'effetto viene riflesso contro l'incantatore come se fosse originato dal Tarrasque, trasformando l'incantatore nel bersaglio.

\emph{\textbf{Mostro d'Assedio.}} Il Tarrasque infligge danni doppi agli oggetti e le strutture.

\emph{\textbf{Resistenza Leggendaria (3/Giorno).}} Se il Tarrasque fallisce un Tiro Salvezza, può scegliere invece di riuscire.

\emph{\textbf{Resistenza alla Magia.}} Il Tarrasque ha +1d6 ai Tiri Salvezza contro incantesimi o altri effetti magici.

\emph{\textbf{Rigenerazione.}} Il Tarrasque rigenera 10 Punti Ferita all'inizio del suo round.

\textbf{Azioni}

\emph{\textbf{Multiattacco.}} Il Tarrasque può usare la sua Presenza Spaventosa. Poi effettua cinque attacchi: uno con il morso, due con gli artigli, uno con le corna, e uno con la coda. Al posto del morso può usare Inghiottire. Gli attacchi del Tarrasque sono considerati magici +4.

\emph{\textbf{Artiglio.} Attacco con arma da mischia}: +20 a colpire, portata 5 metri, un bersaglio.

\emph{Colpisce:} 28 (4d8 + 10) danni taglienti, 3 danno da Sanguinamento.

\emph{\textbf{Coda.} Attacco con arma da mischia}: +20 a colpire, portata 6 m, un bersaglio.

\emph{Colpisce:} 24 (4d6 + 10) danni contundenti. Se il bersaglio è una creatura, deve riuscire un Tiro Salvezza di Tempra DC 40 o cadere prona.

\emph{\textbf{Corna.} Attacco con arma da mischia}: +20 a colpire, portata 3 m, un bersaglio.

\emph{Colpisce:} 32 (4d10 + 10) danni perforanti.

\emph{\textbf{Morso.} Attacco con arma da mischia}: +20 a colpire, portata 3 m, un bersaglio.

\emph{Colpisce:} 36 (4d12 + 10) danni perforanti. Se il bersaglio è una creatura, è afferrata (DC 20 per fuggire). Fino al termine dell'afferrare il Tarrasque non può usare il morso contro un altro bersaglio.

\emph{\textbf{Inghiottire.}} Il Tarrasque effettua una attacco di morso contro un bersaglio di taglia Grande o inferiore che sta afferrando. Se l'attacco colpisce, il bersaglio è inghiottito, e l'afferrare ha termine. Il bersaglio inghiottito è accecato e intralciato, ha copertura completa contro gli attacchi e altri effetti all'esterno del Tarrasque, e subisce 56 (16d6) danni da acido all'inizio di ciascun round del Tarrasque.

Se il Tarrasque subisce 60 o più danni in un singolo round da una creatura al suo interno, il Tarrasque deve riuscire un Tiro Salvezza su Tempra DC 30 al termine di quel round o vomitare tutte le creature inghiottite, che cadono prone in uno spazio entro 3 metri dal Tarrasque. Se il Tarrasque muore, una creatura inghiottita non più intralciata da esso e può uscire dal cadavere utilizzando 2 Azioni e uscendo prona.

\emph{\textbf{Presenza Spaventosa.}} Ogni creatura scelta dal Tarrasque, che si trovi entro 36 metri da esso e consapevole della sua presenza, deve riuscire un Tiro Salvezza di Volontà DC 40 o restare spaventata per 1 minuto. Una creatura può ripetere il Tiro Salvezza al termine di ciascun suo round, con -1d6 se il Tarrasque è in linea di visuale, terminando l'effetto per sé, se lo riesce. Se il Tiro Salvezza della creatura ha successo o l'effetto ha termine per essa, la creatura è immune alla Presenza Spaventosa del Tarrasque per le successive 24 ore.

\textbf{Azioni Aggiuntive}

Il Tarrasque può effettuare 3 Azioni aggiuntive, scelte tra le opzioni seguenti. Può usare solo un'opzione Aggiuntiva alla volta e solo al termine del round di un'altra creatura. Il tarrasque recupera le azioni aggiuntive spese all'inizio del proprio round.

\textbf{Attacco.} Il Tarrasque effettua un attacco di artiglio o di coda.

\textbf{Masticare (Costa 2 Azioni).} Il Tarrasque effettua un attacco di morso o usa Inghiottire.

\textbf{Muoversi.} Il Tarrasque si muove fino a metà del suo movimento.

\textbf{Ecologia}\\
Ambiente: Qualsiasi\\
Organizzazione: Solitario\\
\textbf{Categoria Tesoro}: Nessuno\\
\textbf{Descrizione}\\
Il leggendario Tarrasque è fra i mostri più distruttivi del mondo. Fortunatamente, passa la maggior parte del suo tempo in una specie di profondo letargo in una sconosciuta caverna in un remoto angolo del mondo. Quando si risveglia, però, muoiono interi regni.

Pur non particolarmente intelligente, il Tarrasque è abbastanza intelligente da capire alcune parole nel linguaggio dei Patroni (pur non potendo parlare). Allo stesso modo, la furia non è incontrollata: si concentra sulla creatura che l'ha danneggiato maggiormente ed è difficile distrarlo con l'inganno.

La leggenda dice che il Tarrasque sia l'animale da compagnia di Cattalm.

\mostro{Teschio Fiammeggiante}
\noindent
\begin{description}[noitemsep, topsep=0pt, parsep=0pt, partopsep=0pt, leftmargin=0cm, labelwidth=2.2cm]
	\item[\textbf{Taglia/Tipo:}] Piccolo non morto, Tratti malvagi
	\item[\textbf{Caratt.:}] \resizebox{0.5\linewidth+1.8cm}{!}{For 0 Des 1 Cos 1 Int 1 Sag 0 Car 0}
	\item[\textbf{Punti Ferita:}] 51,  \textbf{Difesa:} 15,  \textbf{Iniziativa:} +1
	\item[\textbf{Movimento:}] volo 10 m
	\item[\textbf{Tiri Salvez.:}] \resizebox{0.5\linewidth+1.8cm}{!}{Tempra +3, Riflessi +3, Volontà +3}
	\item[\textbf{Res. Danni:}] da Vuoto
	\item[\textbf{Imm. Danni:}] Fuoco, Veleno, da arma non magica
	\item[\textbf{Immunità:}] affascinato, paralizzato, affaticato, spaventato, sanguinamento
	\item[\textbf{Sensi:}] Scurovisione 18 m
	\item[\textbf{Sfida:}] 2 (200 PX)\smallskip
\end{description}

\emph{\textbf{Incantesimi.}} Un Teschio Fiammeggiante può eseguire i seguenti incantesimi in maniera innata.

a Volontà: \emph{\hyperlink{Produrre Fiamma}{Produrre Fiamma}}

1 volta al giorno: \emph{Gragnola di Ghiande Infuocate di Kyrin}

\emph{\textbf{Natura Non Morta.}} Il Teschio Fiammeggiante non ha bisogno di aria, cibo, bevande o sonno.

\emph{\textbf{Esperto del fuoco.}} Costa 1 Azione lanciare il trucchetto \hyperlink{Produrre Fiamma}{Produrre Fiamma}.

\textbf{Ecologia}\\
Ambiente: Qualsiasi\\
Organizzazione: Solitario, paio, pattuglia (2d4)\\
\textbf{Categoria Tesoro}: nessuno\\
\textbf{Descrizione}:\\
I Teschi Fiammeggianti sono creati dai cadaveri degli incantatori specializzati nella Lista di magia del Fuoco e della necromanzia.

Usati come custodi e torce rappresentano spesso una prima linea di difesa nei dungeon.

\mostro{Testuggine Dragona}
\noindent
\begin{description}[noitemsep, topsep=0pt, parsep=0pt, partopsep=0pt, leftmargin=0cm, labelwidth=2.2cm]
	\item[\textbf{Taglia/Tipo:}] Mastodontica drago, neutrale
	\item[\textbf{Caratt.:}] \resizebox{0.5\linewidth+1.8cm}{!}{For 7 Des 0 Cos 5 Int 0 Sag 1 Car 1}
	\item[\textbf{Punti Ferita:}] 338,  \textbf{Difesa:} 34,  \textbf{Iniziativa:} +0
	\item[\textbf{Movimento:}] 6 m, nuoto 12 m
	\item[\textbf{Tiri Salvez.:}] \resizebox{0.5\linewidth+1.8cm}{!}{\resizebox{0.5\linewidth+1.8cm}{!}{Tempra +22, Riflessi +17, Volontà +18}}
	\item[\textbf{Sensi:}] Scurovisione 18 m
	\item[\textbf{Linguaggi:}] Aquan, Draconico
	\item[\textbf{Sfida:}] 17 (18000 PX)\smallskip
\end{description}

\emph{\textbf{Anfibio.}} La testuggine dragona può respirare aria e acqua.

\textbf{Azioni}

\emph{\textbf{Multiattacco.}} Il drago può effettuare tre attacchi: uno con il morso e due con gli artigli. Può effettuare un attacco di coda al posto di due attacchi di artiglio.

\emph{\textbf{Artiglio.} Attacco con arma da mischia}: +13 a colpire, portata 3 m, un bersaglio.

\emph{Colpisce:} 16 (2d8 + 7) danni taglienti.

\emph{\textbf{Coda.} Attacco con arma da mischia}: +13 a colpire, portata 5 metri, un bersaglio.
\emph{Colpisce:} 26 (3d12 + 7) danni contundenti. Se il bersaglio è una creatura, deve riuscire un Tiro Salvezza di Tempra DC 31 o venire spinta di 3 metri lontano dalla testuggine dragona e cadere prona.

\emph{\textbf{Morso.} Attacco con arma da mischia}: +13 a colpire, portata 5 metri, un bersaglio.

\emph{Colpisce:} 26 (3d12 + 7) danni perforanti.

\emph{\textbf{Salto e Schiaccio.} Attacco con arma da mischia}: +12 a colpire, portata 9 metri, fino a 6 creature in 6x6m di area. 2 Azioni.

\emph{Colpisce:} 40 (6d12 + 4) danni contundenti

\emph{\textbf{Soffio di Vapore (Ricarica 5-6).}} La testuggine dragona esala un vapore caldo in un cono di 18 metri. Ogni creatura in quell'area deve effettuare un Tiro Salvezza di Tempra DC 31 e subire 52 (15d6) danni da fuoco se fallisce il Tiro Salvezza, o la metà di questi danni se lo riesce. Trovarsi sott'acqua non dà resistenza contro questo tipo di danno.

\textbf{Ecologia}
Ambiente: Acquatico temperato\\
Organizzazione: Solitario\\
\textbf{Categoria Tesoro}: A\\
\textbf{Descrizione}\\
Le testuggini dragone sono creature delle acque dolci e salate, molto temute dai marinai. Sono noti per aspettarsi offerte in oro e magia dai marinai per un passaggio sicuro. Ignorare una testuggine dragona può renderla molto pericolosa.

Il loro guscio varia di colore, da marrone e rosso ruggine a verde-blu con riflessi argentei. Le testuggini dragone capovolgono le navi che violano il loro territorio, accumulando ricchezze nei loro nascondigli subacquei. Vivono in caverne profonde e difendono aggressivamente il loro territorio, spesso in conflitto con altre razze sottomarine.

Si nutrono di grandi pesci e alghe marine, e non disdegnano i passeggeri delle navi affondate. I loro gusci possono raggiungere i 5 metri di diametro, con una lunghezza totale di 7 metri

\mostro{Topi, La}
\noindent
\begin{description}[noitemsep, topsep=0pt, parsep=0pt, partopsep=0pt, leftmargin=0cm, labelwidth=2.2cm]
	\item[\textbf{Taglia/Tipo:}] Minuscola fatata, indifferente. Patrono
	\item[\textbf{Caratt.:}] \resizebox{0.5\linewidth+1.8cm}{!}{For -1 Des 4 Cos 0 Int 6 Sag 1 Car 10}
	\item[\textbf{Punti Ferita:}] 15,  \textbf{Difesa:} 16,  \textbf{Iniziativa:} +6
	\item[\textbf{Movimento:}] 6 m, volare 18 m, fluttuare
	\item[\textbf{Tiri Salvez.:}] \resizebox{0.5\linewidth+1.8cm}{!}{\resizebox{0.5\linewidth+1.8cm}{!}{Tempra +30, Riflessi +34, Volontà +30}}
	\item[\textbf{Comp.:}] tutte +20
	\item[\textbf{Immunità:}] al danno delle armi con bonus magico inferiore a +6
	\item[\textbf{Immunità:}] a qualsiasi effetto, danno, condizione non faccia piacere alla Topi
	\item[\textbf{Immunità:}] a qualsiasi magia la Topi non voglia essere influenzata
	\item[\textbf{Immunità:}] a subire a qualsiasi tipo di tiro critico
	\item[\textbf{Sensi:}] Senso tellurico 60, Scurovisione 60 m, Visione del Vero 60 m
	\item[\textbf{Linguaggi:}] tutti
	\item[\textbf{Sfida:}] 0 (10 PX)\smallskip
\end{description}


\begin{center}
	\includegraphics[width=0.4\textwidth]{immagini/mice.png}

	\centering
	\emph{La Topi}
\end{center}


\emph{\textbf{E' La Topi}} La Topi ha +3d6 (oppure +18) ogni volta che deve tirare dei dadi o contare un valore.

\emph{\textbf{Ha ragione!}} Qualsiasi attacco effettuato dalla Topi è considerato magico +6 e non è resistibile o rigenerabile o curabile nelle prime 24 ore.

\smallskip

\textbf{Azioni}\\
\emph{\textbf{Musetto}} ogni bersaglio a scelta di Topi, entro 30 metri, subisce un Musetto. Il bersaglio viene allontanato di 2d6 metri e subisce 3d6 danni\\
\emph{\textbf{Morso topetto} Attacco con Arma da Mischia}: +26 al colpire, portata 0 m, un bersaglio.\\
\emph{Colpisce:} 6 danno perforante.\\
\emph{\textbf{Graffiotto} fino a 4 Attacchi con Arma da Mischia}: colpisce  automaticamente, portata 0 m.\\
\emph{Colpisce:} 1 danno perforante.\\
\emph{\textbf{Appoggia nasino} una creatura}. La Topi appoggia il nasino sulla creatura scelta e questa viene guarita da ogni malattia, condizione o ferita in essere.\\
\emph{\textbf{Arrabbiato:}} la Topi fa quello che vuole (Desiderio illimitato). Costo 1 Reazione.\\
\textbf{Azioni Aggiuntive}\\
La Topi, come Patrono, può fare quante Azioni aggiuntive vuole tra tutte quelle segnate. Può usare Desiderio illimitato una volta a round.\\
\textbf{Ecologia}\\
Ambiente: Ovunque, Mercati\\
Organizzazione: Solitario\\
\textbf{Categoria Tesoro}: Speciale\\
\textbf{Descrizione}\\
Potrebbe essere scambiata per una piccola topina bianca, ma La Topi è molto di più. Furba, intelligente, bellissima adora andare al mercato e comprare borsette.

\mostro{Torciascura}
\noindent
\begin{description}[noitemsep, topsep=0pt, parsep=0pt, partopsep=0pt, leftmargin=0cm, labelwidth=2.2cm]
	\item[\textbf{Taglia/Tipo:}] Media non morto, malvagio
	\item[\textbf{Caratt.:}] \resizebox{0.5\linewidth+1.8cm}{!}{For 3 Des 1 Cos 2 Int 0 Sag -1 Car -2}
	\item[\textbf{Punti Ferita:}] 88,  \textbf{Difesa:} 18,  \textbf{Iniziativa:} +1
	\item[\textbf{Movimento:}] 6 m
	\item[\textbf{Tiri Salvez.:}] \resizebox{0.5\linewidth+1.8cm}{!}{Tempra +6, Riflessi +5, Volontà +3}
	\item[\textbf{Res. Danni:}] da Vuoto; da arma non magica o che non sia argentata
	\item[\textbf{Imm. Danni:}] Veleno
	\item[\textbf{Immunità:}] affaticato, sanguinamento
	\item[\textbf{Sensi:}] Scurovisione, vede nell'oscurità magica
	\item[\textbf{Linguaggi:}] Comprende il Comune, ma non parla
	\item[\textbf{Sfida:}] 4 (1100 PX)\smallskip
\end{description}

\emph{\textbf{Invisibile al buio.}} Un Torciascura è completamente invisibile finché è nell'oscurità\\
\emph{\textbf{Natura Non Morta.}} Torciascura non ha bisogno di aria, cibo, bevande o sonno.\\
\emph{\textbf{Sensibilità alla Luce}}. Mentre è alla luce del sole, Torciascura ha -1d6 ai tiri di attacco\\
\textbf{Multiattacco}\\
\emph{\textbf{Attacco}} Torciascura attacca due volte con la sua torcia oppure esegue Urlo di Tristezza\\
\emph{\textbf{Torcia}} Attacco di mischia, +6 al colpire\\
\emph{\textbf{Colpisce}} 7 (1d6+3) di danni contundenti, lancia l'incantesimo Oscurità sull'obiettivo colpito, durata fino alla distruzione del Torciascura\\
\emph{\textbf{Urlo di Tristezza}} cono di 6 metri. Le creature colpite devono effettuare un Tiro Salvezza su Volontà DC 16 o cadere in una triste disperazione che conferisce -2 al Tiro per Colpire, -2 al danno in mischia.\\
\textbf{Ecologia}\\
Ambiente: Dungeon\\
Organizzazione: Solitario, gruppo 2d4\\
\textbf{Categoria Tesoro}: Speciale\\
\textbf{Descrizione}\\
Un Torciascura era un avventuriero, come voi, morto in preda al terrore dopo che l'ultima torcia si spense. Un Torciascura è un non morto, solitamente umanoide, dall'aspetto vagamente indefinito, che brandisce una torcia che emana pura oscurità. Il suo scopo è uccidere nuovi avventurieri avvolgendoli nell'oscurità eterna.

Solitamente il Torciascura si nasconde nell'oscurità aspettando di toccare l'avversario ed avvolgerlo nella sua maledizione. Una creatura uccisa da un Torciascura torna in vita come Torciascura dopo 1d3 giorni.

Un Torciascura quando viene distrutto lascia a terra la sua torcia. Questa torcia, di pura oscurità può lanciare l'incantesimo Oscurità a tocco tre volte al giorno, fuori dalle mani di un Torciascura se esposta alla luce del sole si distrugge in 2d4 round.

\mostro{Troll}
\noindent
\begin{description}[noitemsep, topsep=0pt, parsep=0pt, partopsep=0pt, leftmargin=0cm, labelwidth=2.2cm]
	\item[\textbf{Taglia/Tipo:}] Grande gigante, malvagio
	\item[\textbf{Caratt.:}] \resizebox{0.5\linewidth+1.8cm}{!}{For 5 Des 1 Cos 5 Int -2 Sag -1 Car -2}
	\item[\textbf{Punti Ferita:}] 110,  \textbf{Difesa:} 19,  \textbf{Iniziativa:} +1
	\item[\textbf{Movimento:}] 9 m
	\item[\textbf{Tiri Salvez.:}] \resizebox{0.5\linewidth+1.8cm}{!}{Tempra +10, Riflessi +6, Volontà +4}
	\item[\textbf{Sensi:}] Scurovisione 18 m
	\item[\textbf{Linguaggi:}] Gigante
	\item[\textbf{Sfida:}] 5 (1800 PX)\smallskip
\end{description}

\emph{\textbf{Olfatto Affinato.}} Il troll ha +1d6 alle prove di Consapevolezza basate sull'olfatto.

\emph{\textbf{Rigenerazione.}} Il troll recupera 10 Punti Ferita all'inizio del suo round. Se il troll subisce danno da acido o da fuoco, questo tratto non funziona all'inizio del prossimo round del troll. Il troll muore solo se inizia il suo round ha meno di -5 Punti Ferita e non può rigenerarsi.

\textbf{Azioni}

\emph{\textbf{Multiattacco.}} Il troll può effettuare tre attacchi: uno con il morso e due con gli artigli.

\emph{\textbf{Artiglio.} Attacco con arma da mischia}: +8 a colpire, portata 1 m, un bersaglio.

\emph{Colpisce:} 12 (2d6 + 5) danni taglienti, 1 danno da Sanguinamento.

\emph{\textbf{Morso.} Attacco con arma da mischia}: +7 a colpire, portata 1 m, un bersaglio.

\emph{Colpisce:} 8 (1d6 + 8) danni perforanti.

\textbf{Ecologia}\\
Ambiente: Montagne Fredde\\
Organizzazione: Solitario o banda (2-4)\\
\textbf{Categoria Tesoro}: B\\
\textbf{Descrizione}\\
I troll possiedono artigli affilati ed incredibili capacità rigenerative che permettono loro di guarire quasi tutte le ferite. Sono gobbi, brutti ma fortissimi: combinata con i loro artigli, la loro forza gli permette di lacerare la carne a mani nude. I troll sono alti circa 3 metri, ma la loro postura li fa apparire più bassi. Un troll adulto pesa circa 500 kg.

L'appetito di un troll e le sue capacità rigenerative lo rendono un combattente indomito, che carica a testa bassa la creatura vivente più vicina ed attacca con tutta la sua furia. Solo il fuoco fa esitare un troll, ma perfino quello che per lui è un pericolo mortale non ferma la sua avanzata. Chi affronta i troll sa di dover localizzare e bruciare qualsiasi sua parte dopo un combattimento, perché perfino dal brandello più piccolo del suo corpo, con il tempo può rinascere un troll completo. Fortunatamente, solo le parti più grandi di un troll, come gli arti, ricrescono in questo modo.

Nonostante la loro ferocia, i troll sono straordinariamente teneri e gentili verso i loro piccoli. I troll femmina lavorano in gruppo, passando molto tempo ad insegnare ai cuccioli come cacciare e difendersi prima di mandarli a cercare un proprio territorio. Un troll maschio vive un'esistenza solitaria, incontrando brevemente le femmine solo per accoppiarsi. Tutti i troll trascorrono il loro tempo a cercare cibo, dato che devono consumarne enormi quantità ogni giorno o muoiono di fame. Per questo, la maggior parte dei troll si crea un proprio territorio di caccia che viene spesso difeso combattendo con i rivali. Simili scontri sono di solito non letali, ma i troll conoscono bene le proprie debolezze, sfruttandole per uccidere l'avversario nei periodi di magra.

E' universalmente conosciuto che i troll possono naturalmente mutare acquisendo per brevi periodi le caratteristiche più peculiari delle creature di cui si nutrono. Non avete idea di quanto può essere buffo un Pegasutroll...

%\addcontentsline{toc}{subsubsection}{U}
\pdfbookmark[3]{U}{U}

\mostro{Uomo Acquatico}
\noindent
\begin{description}[noitemsep, topsep=0pt, parsep=0pt, partopsep=0pt, leftmargin=0cm, labelwidth=2.2cm]
	\item[\textbf{Taglia/Tipo:}] Media umanoide (uomo acquatico), neutrale
	\item[\textbf{Caratt.:}] \resizebox{0.5\linewidth+1.8cm}{!}{For 0 Des 1 Cos 1 Int 0 Sag 0 Car 1}
	\item[\textbf{Punti Ferita:}] 17,  \textbf{Difesa:} 13,  \textbf{Iniziativa:} +1
	\item[\textbf{Movimento:}] 3 m, nuoto 12 m
	\item[\textbf{Tiri Salvez.:}] \resizebox{0.5\linewidth+1.8cm}{!}{Tempra +3, Riflessi +3, Volontà +3}
	\item[\textbf{Comp.:}] Consapevolezza +2
	\item[\textbf{Linguaggi:}] Aquan, Comune
	\item[\textbf{Sfida:}] 1/8 (25 PX)\smallskip
\end{description}

\emph{\textbf{Anfibio.}} L'uomo acquatico può respirare aria e acqua.

\textbf{Azioni}

\emph{\textbf{Lancia.} Attacco con arma da mischia o a Distanza}: +3 a colpire, portata 1 m o gittata 6m, un bersaglio.

\emph{Colpisce:} 3 (1d6) danni perforanti, o 4 (1d8) danni perforanti se usata con due mani per effettuare un attacco da mischia.

\textbf{Ecologia}\\
Ambiente: Oceani temperati\\
Organizzazione: Solitario, pattuglia (2-6), banda (6-10 più un tenete di 3° livello, compagnia (11-60 più 3 tenenti di 3° livello, 2 comandanti di 5° livello, 1 commodoro di 7° livello e 3-12 Calamari\\
\textbf{Categoria Tesoro}: Equipaggiamento da PNG (Tridente, Balestra Leggera con 10 Quadrelli, N)\\
\textbf{Descrizione}\\
Fisicamente, gli Uomini Pesce somigliano ai loro antenati, con fronti espressive, pelle pallida, capelli scuri e occhi porpora. Hanno tre sottili branchie sul collo, ma possono passare per Umani per brevi periodi, se lo desiderano.

\mostro{Uomo Albero (Arborom)}
\noindent
\begin{description}[noitemsep, topsep=0pt, parsep=0pt, partopsep=0pt, leftmargin=0cm, labelwidth=2.2cm]
	\item[\textbf{Taglia/Tipo:}] Enorme pianta, buono
	\item[\textbf{Caratt.:}] \resizebox{0.5\linewidth+1.8cm}{!}{For 6 Des -1 Cos 5 Int 1 Sag 3 Car 1}
	\item[\textbf{Punti Ferita:}] 186,  \textbf{Difesa:} 23,  \textbf{Iniziativa:} +1
	\item[\textbf{Movimento:}] 9 m
	\item[\textbf{Tiri Salvez.:}] \resizebox{0.5\linewidth+1.8cm}{!}{\resizebox{0.5\linewidth+1.8cm}{!}{Tempra +14, Riflessi +8, Volontà +12}}
	\item[\textbf{Res. Danni:}] contundente, perforante
	\item[\textbf{Linguaggi:}] Comune, Druidico, Elfico, Silvano
	\item[\textbf{Sfida:}] 9 (5000 PX)\smallskip
\end{description}

\emph{\textbf{Falso Aspetto.}} Mentre l'uomo albero rimane immobile, è indistinguibile da un normale albero.

\emph{\textbf{Mostro d'Assedio.}} L'uomo albero infligge danni doppi agli oggetti e le strutture.

\textbf{Azioni}

\emph{\textbf{Multiattacco.}} L'uomo albero effettua due attacchi di schianto.

\emph{\textbf{Schianto.} Attacco con arma da mischia}: +11 a colpire, portata 2 m, un bersaglio.

\emph{Colpisce:} 16 (3d6 + 6) danni contundenti.

\emph{\textbf{Sasso.} Attacco con arma a Distanza}: +10 a colpire, gittata 18m, un bersaglio.

\emph{Colpisce:} 28 (4d10 + 6) danni contundenti.

\textbf{Reazione: \emph{Attacco d'opportunità}}: l'uomo albero effettua un attacco di schianto ad una creatura che attraversi o esca dalla sua portata di 2 metri.

\emph{\textbf{Animare Alberi (1/Giorno).}} L'uomo albero anima magicamente uno o due alberi visibili entro 18 metri da lui. Questi alberi hanno le stesse statistiche dell'Arborom, eccetto che hanno punteggio di Intelligenza e Carisma -3, non possono parlare, e hanno solo l'opzione di attacco Schianto. Un albero animato agisce come alleato dell'uomo albero. L'albero resta per 1 giorno o finché muore; finché l'uomo albero muore o si trova più di 36 metri lontano dall'albero, o finché l'uomo albero non effettua una Reazione per ritrasformarlo in un albero inanimato. Poi l'albero prenderà radici, se possibile.

\textbf{Ecologia}\\
Ambiente: Qualsiasi foresta\\
Organizzazione: Solitario o macchia (2-7)\\
\textbf{Categoria Tesoro}: J\\
\textbf{Descrizione}\\
I Arborom sono guardiani delle foreste ed ambasciatori degli alberi. Antichi quanto le foreste stesse, si vedono come genitori e pastori piuttosto che giardinieri: sono lenti e metodici, ma terrificanti quando costretti a combattere per difendere il loro gregge. Anche se raramente cercano la compagnia delle razze dalla vita breve ed hanno un'innata sfiducia verso i cambiamenti, mostrano tolleranza verso chi desidera imparare dai loro lunghi, lenti monologhi, specialmente coloro nei cui occhi leggono il desiderio di proteggere le regioni selvagge. Contro coloro che minacciano le loro foreste, specialmente i boscaioli che raccolgono legna o coloro che vorrebbero disboscare una foresta per costruire una strada o un forte, la rabbia dei Arborom si scatena rapida e devastante. Sono in grado di demolire ciò che gli altri costruiscono: un tratto che li aiuta durante i loro eccessi di furia.

I Arborom sono principalmente creature solitarie, ed un singolo individuo è spesso responsabile di un'intera foresta, ma a volte si raccolgono in gruppi detti boschetti per scambiarsi le ultime notizie e riprodursi.

In tempi di grave pericolo, tutti i boschetti di una regione si uniscono per una riunione della durata di mesi detta concilio, ma simili eventi sono molto rari, e fra i concili passano anche millenni.

Un tipico Arborom è alto 9 metri, con un tronco del diametro di 60 centimetri, e pesa circa 2.250 kg. I Arborom somigliano agli alberi più comuni dei territori dove vivono.

Gli Arborom si dice che siano creati per volere di Efrem.

\mostro{Uomo Magma}
\noindent
\begin{description}[noitemsep, topsep=0pt, parsep=0pt, partopsep=0pt, leftmargin=0cm, labelwidth=2.2cm]
	\item[\textbf{Taglia/Tipo:}] Piccola elementale, caotico
	\item[\textbf{Caratt.:}] \resizebox{0.5\linewidth+1.8cm}{!}{For -2 Des 2 Cos 1 Int -1 Sag 0 Car 0}
	\item[\textbf{Punti Ferita:}] 24,  \textbf{Difesa:} 14,  \textbf{Iniziativa:} +2
	\item[\textbf{Movimento:}] 9 m
	\item[\textbf{Tiri Salvez.:}] \resizebox{0.5\linewidth+1.8cm}{!}{Tempra +3, Riflessi +3, Volontà +3}
	\item[\textbf{Res. Danni:}] da arma non magica\\
	\item[\textbf{Imm. Danni:}] Fuoco
	\item[\textbf{Sensi:}] Scurovisione 18 m
	\item[\textbf{Linguaggi:}] Ignan
	\item[\textbf{Sfida:}] 1/2 (100 PX)\smallskip
\end{description}

\emph{\textbf{Illuminazione Incendiaria.}} Come Azione Immediata, l'uomo magma può accendere o spegnere le sue fiamme. Mentre la fiamma è accesa, l'uomo magma irradia luce intensa in un raggio di 3 metri e luce fioca per 6 metri.

\emph{\textbf{Scoppio Mortale.}} Quando l'uomo magma muore, esplode in uno scoppio di fuoco e magma. Ogni creatura entro 3 metri da esso deve effettuare un Tiro Salvezza di Riflessi DC 12, subendo 7 (2d6) danni da fuoco se fallisce il Tiro Salvezza o la metà di questi danni se lo riesce. Gli oggetti infiammabili che non siano indossati o trasportati e che si trovino nell'area, prendono fuoco.

\textbf{Azioni}

\emph{\textbf{Tocco.} Attacco con arma da mischia}: +4 a colpire, portata 1 m, un bersaglio.

\emph{Colpisce:} 7 (2d6) danni da fuoco. Se il bersaglio è una creatura o un oggetto infiammabile, questi prende fuoco. Fino a che una creatura non usa un'Azione per estinguere la fiamma subisce 3 (1d6) danni da fuoco al termine di ciascun suo round.

\textbf{Ecologia}\\
Ambiente: Qualsiasi terreno (Piano del Fuoco)\\
Organizzazione: Solitario o banda (2-8)\\
\textbf{Categoria Tesoro}: L\\
\textbf{Descrizione}\\
Gli uomini di lava, noti come Ignim, abitano il Piano del Fuoco ma a volte scivolano nel Piano Materiale attraverso crepe elementali. Queste crepe si formano in luoghi di forte calore, come vulcani o fiumi sotterranei di magma, o in aree di intensa magia. Spesso, appiccano involontariamente fuoco agli oggetti infiammabili vicini.

Nonostante non siano coraggiosi, gli Ignim sono temibili nemici per chi non ha resistenza al loro calore intenso. Il loro tocco incenerisce gli abiti e le armi di acciaio rischiano di diventare scorie al contatto. Nel Piano del Fuoco, gli Ignim trovano forza nel numero, popolando insediamenti costellati di laghi di magma e geyser di roccia fusa.

Paranoici e diffidenti, gli Ignim temono gli abitanti più grandi del Piano del Fuoco e sommergono gli intrusi con domande. Se le risposte non soddisfano, cercano di sbarazzarsi delle creature il più rapidamente possibile, anche gettandole in laghi di roccia liquida.

Orgogliosi dei loro laghi di magma, ogni lago ha un diverso scopo: bagni, cottura o relax. Gli Ignim aggiungono minerali e sali ai laghi per adeguarli ai loro scopi. I laghi per cucinare, a volte chiamati "laghi assassini", sono più caldi, mentre quelli per il relax sono di solito più scuri.

Alla maturità, gli Ignim sono alti 1,2 metri e pesano 150 kg grazie alla loro densa composizione.

\mostro{Unicorno}
\noindent
\begin{description}[noitemsep, topsep=0pt, parsep=0pt, partopsep=0pt, leftmargin=0cm, labelwidth=2.2cm]
	\item[\textbf{Taglia/Tipo:}] Grande celestiale, buono
	\item[\textbf{Caratt.:}] \resizebox{0.5\linewidth+1.8cm}{!}{For 4 Des 2 Cos 2 Int 0 Sag 3 Car 3}
	\item[\textbf{Punti Ferita:}] 107,  \textbf{Difesa:} 20,  \textbf{Iniziativa:} +2
	\item[\textbf{Movimento:}] 15 m
	\item[\textbf{Tiri Salvez.:}] \resizebox{0.5\linewidth+1.8cm}{!}{Tempra +7, Riflessi +7, Volontà +8}
	\item[\textbf{Imm. Danni:}] Veleno
	\item[\textbf{Immunità:}] affascinato, paralizzato
	\item[\textbf{Sensi:}] Scurovisione 18 m
	\item[\textbf{Linguaggi:}] Celestiale, Elfico, Silvano, telepatia 18 m
	\item[\textbf{Sfida:}] 5 (1800 PX)\smallskip
\end{description}

\emph{\textbf{Armi Magiche.}} Gli attacchi con armi dell'unicorno sono magici.

\emph{\textbf{Carica.}} Se l'unicorno si muove di almeno 6 metri in linea retta verso il bersaglio e lo colpisce con un attacco di corno durante lo stesso round, il bersaglio subisce 9 (2d8) danni perforanti aggiuntivi. Se il bersaglio è una creatura, deve riuscire un T10iro Salvezza su Tempra DC 15 o cadere prono.

\emph{\textbf{Incantesimi Innati.}} La caratteristica da incantatore innato dell'unicorno è il Carisma (DC 14 per i Tiri Salvezza degli incantesimi). L'unicorno può lanciare in maniera innata i seguenti incantesimi, senza bisogno di componenti:

A volontà: \emph{\hyperlink{Artificio Druidico}{Artificio Druidico}}, \emph{\hyperlink{Passare Senza Tracce}{Passare Senza Tracce}}

1/giorno ciascuno: \emph{\hyperlink{Calmare Emozioni}{Calmare Emozioni}} \emph{\hyperlink{Intralciare}{Intralciare}}

\emph{\textbf{Resistenza alla Magia.}} L'unicorno ha +1d6 ai Tiri Salvezza contro incantesimi e altri effetti magici.

\textbf{Azioni}

\emph{\textbf{Multiattacco.}} L'unicorno effettua due attacchi: uno con gli zoccoli e uno con il corno.

\emph{\textbf{Corno.} Attacco con arma da mischia}: +6 a colpire, portata 1 m, un bersaglio.

\emph{Colpisce:} 8 (1d8 + 4) danni perforanti.

\emph{\textbf{Zoccoli.} Attacco con arma da mischia}: +6 a colpire, portata 1 m, un bersaglio.

\emph{Colpisce:} 11 (2d6 + 4) danni contundenti.

\emph{\textbf{Teletrasporto (1/Giorno).}} L'unicorno può teletrasportare magicamente sé stesso e fino a tre altre creature consenzienti visibili entro 1 metro da esso, insieme a tutto l'equipaggiamento che stanno indossando o trasportando, in un luogo familiare all'unicorno, che si trova ad un massimo di 1,5 chilometri di distanza.

\emph{\textbf{Tocco Guaritore (3/Giorno).}} L'unicorno entra a contatto tramite il corno con un'altra creatura. Il bersaglio recupera magicamente 11 (2d8 + 2) Punti Ferita. Inoltre, il contatto rimuove tutte le malattie e neutralizza tutti i veleni che affliggono il bersaglio.

\textbf{Azioni Aggiuntive}

L'unicorno può effettuare 3 Azioni aggiuntive, scelte tra le opzioni seguenti. Può usare solo un'opzione Aggiuntiva alla volta e solo al termine del round di un'altra creatura. L'unicorno recupera le azioni aggiuntive spese all'inizio del proprio round.

\textbf{Autoguarigione (Costa 3 Azioni).} L'unicorno recupera magicamente 11 (2d8 + 2) Punti Ferita.

\textbf{Scudo Scintillante (Costa 2 Azioni).} L'unicorno crea un campo magico scintillante che circonda lui o un'altra creatura visibile a lui entro 18 metri. Il bersaglio ottiene un bonus di +2 alla Difesa fino al termine del prossimo round dell'unicorno.

\textbf{Zoccoli.} L'unicorno effettua un attacco con gli zoccoli.

\textbf{Ecologia}\\
Ambiente: Foreste Temperate\\
Organizzazione: Solitario, coppia o benedizione (3-6)\\
\textbf{Categoria Tesoro}: Nessuno\\
\textbf{Descrizione}\\
Certo! Ecco un breve riassunto del testo sugli unicorni:

Gli unicorni sono creature intelligenti e solitarie che abitano le foreste, apparendo solo per difendere le loro dimore dal male. Evitano tutte le creature tranne i folletti buoni, le donne umanoidi buone e gli animali nativi. Le coppie di unicorni rimangono insieme per tutta la vita e proteggono le loro foreste, permettendo solo alle creature buone e neutrali di attraversarle.

Il corno dell'unicorno è la fonte dei suoi poteri magici, e le creature malvagie danno grande valore a questi corni per i loro riti oscuri e pozioni di guarigione. In rare occasioni, gli unicorni il cui partner è stato ucciso scelgono giovani donne virtuose come sostituti, permettendo loro di cavalcarli e diventare loro guardiani per tutta la vita.

%\begin{center}
%\includegraphics[width=0.45\textwidth]{immagini/Unicorn.png}
%\end{center}

%\addcontentsline{toc}{subsubsection}{V}
\pdfbookmark[3]{V}{V}

\mostro{Vampiro}
\noindent
\begin{description}[noitemsep, topsep=0pt, parsep=0pt, partopsep=0pt, leftmargin=0cm, labelwidth=2.2cm]
	\item[\textbf{Taglia/Tipo:}] Media non morto (mutaforma), malvagio
	\item[\textbf{Caratt.:}] \resizebox{0.5\linewidth+1.8cm}{!}{For 4 Des 4 Cos 4 Int 3 Sag 2 Car 4}
	\item[\textbf{Punti Ferita:}] 259,  \textbf{Difesa:} 33,  \textbf{Iniziativa:} +4
	\item[\textbf{Movimento:}] 9 m
	\item[\textbf{Tiri Salvez.:}] \resizebox{0.5\linewidth+1.8cm}{!}{\resizebox{0.5\linewidth+1.8cm}{!}{Tempra +17, Riflessi +17, Volontà +15}}
	\item[\textbf{Comp.:}] Furtività +9, Consapevolezza +17
	\item[\textbf{Imm. Danni:}] da Vuoto; Veleno, da arma non magica
	\item[\textbf{Immunità:}] affascinato, assordato, sanguinamento
	\item[\textbf{Sensi:}] Scurovisione 36 m
	\item[\textbf{Linguaggi:}] le lingue che conosceva in vita, Expiran
	\item[\textbf{Sfida:}] 13 (10000 PX)\smallskip
\end{description}

\emph{\textbf{Mutaforma.}} Se il vampiro non è sotto la luce del sole o immerso in acqua corrente, può usare una Azione per trasformarsi in un Minuscolo pipistrello, una nube di foschia Media, o per tornare alla sua vera forma.

Mentre è in forma di pipistrello, il vampiro non può parlare, la sua velocità di movimento è 1 metro e ha velocità di volo 9 metri. Le sue statistiche, a parte la taglia e la velocità, sono immutate. Qualsiasi equipaggiamento stia indossando si trasforma con esso, ma quello che stava trasportando viene fatto cadere a terra. Alla morte ritorna alla sua vera forma.

Mentre è in forma di foschia, il vampiro non può effettuare azioni, parlare o manipolare oggetti. È privo di peso, ha velocità di volo 6 metri, può fluttuare, e può entrare nello spazio di una creatura ostile e fermarsi lì. Inoltre, se in uno spazio vi passa dell'aria, la foschia può fare altrettanto senza stringersi, ma non può attraversare l'acqua. Ha +1d6 ai Tiri Salvezza su Tempra e Riflessi ed è immune a tutti i danni non magici, eccetto i danni subiti dalla luce del
sole.

\emph{\textbf{Debolezze del Vampiro.}} Il vampiro ha i seguenti difetti:

\emph{Danneggiato dall'Acqua Corrente.} Il vampiro subisce 20 danni da acido se termina il suo round all'interno dell'acqua corrente.

\emph{Ipersensibilità alla Luce.} Il vampiro subisce 20 danni da Luce quando inizia il suo round alla luce del sole. Mentre è alla luce del sole, ha -1d6 ai tiri di attacco e le prove di competenza di Base.

\emph{Paletto nel Cuore.} Se un'arma perforante fatta di legno viene conficcata nel cuore del vampiro mentre il vampiro è inabile nel suo luogo di riposo, il vampiro resta paralizzato finché il paletto non viene rimosso.

\emph{Proibizione.} Il vampiro non può entrare in un'abitazione senza invito da parte dei suoi occupanti.

\emph{\textbf{Fuga nella Foschia.}} Quando scende a 0 Punti Ferita al di fuori del suo luogo di riposo, il vampiro si trasforma in una nube di foschia (come per il tratto Mutaforma) invece di cadere privo di sensi, purché non sia esposto alla luce del sole o all'acqua corrente. Se non può trasformarsi, viene distrutto.

Mentre si trova a 0 Punti Ferita in questa forma, non può tornare alla sua forma di vampiro, e deve raggiungere il suo luogo di riposo entro 2 ore o venire distrutto. Una volta raggiunto il suo luogo di riposo, ritorna alla sua forma di vampiro. Resterà quindi paralizzato finché non avrà recuperato almeno 1 punto ferita. Dopo aver trascorso almeno 1 ora nel suo luogo di riposo a 0 Punti Ferita, il vampiro recupererà 1 punto ferita.

\emph{\textbf{Natura Non Morta.}} Il vampiro non ha bisogno di aria.

\emph{\textbf{Resistenza Leggendaria (3/Giorno).}} Se il vampiro fallisce un Tiro Salvezza, può scegliere invece di riuscire.

\emph{\textbf{Rigenerazione.}} Il vampiro recupera 20 Punti Ferita all'inizio del suo round se possiede almeno 1 punto ferita e non è esposto alla luce del sole o l'acqua corrente. Se il vampiro subisce danno da Luce o danno dall'Acqua santa, questo tratto non funziona all'inizio del prossimo round del vampiro.

\emph{\textbf{Scalare come Ragno.}} Il vampiro può scalare superfici difficili, compreso lo stare a testa in giù sul soffitto, senza bisogno di effettuare una prova di competenza.

\textbf{Azioni}

\emph{\textbf{Multiattacco.}} Il vampiro può effettuare due attacchi, ma solo uno di essi può essere un attacco con morso.

\emph{\textbf{Colpo Disarmato (Solo in Forma di Vampiro).} Attacco con arma da mischia}: +12 a colpire, portata 1 m, una creatura.

\emph{Colpisce:} 8 (1d8 + 4) danni contundenti. Invece di infliggere danno, il vampiro può afferrare il bersaglio (DC per fuggire 18).

\emph{\textbf{Morso (Solo in Forma di Pipistrello o Vampiro).} Attacco con arma da mischia}: +11 a colpire, portata 1 m, una creatura consenziente o una creatura afferrata dal vampiro, inabile o intralciata.

\emph{Colpisce:} 7 (1d6 + 4) danni perforanti più 10 (3d6) danni da Vuoto. I Punti Ferita massimi del bersaglio sono ridotti di un ammontare pari al danno da Vuoto subito, e il vampiro recupera un numero di Punti Ferita pari a quell'ammontare, TS Tempra DC 23 per resistere alla perdita di Punti Ferita Massimi. Il bersaglio diviene Affaticato. Il bersaglio muore se questo effetto riduce i suoi Punti Ferita massimi a 0. Un umanoide ucciso in questo modo e poi sepolto nel terreno si rianima la notte seguente come progenie vampirica sotto il controllo del vampiro.

\emph{\textbf{Affascinare.}} Il vampiro prende a bersaglio un umanoide entro 9 metri che può vedere. Se il bersaglio può vedere il vampiro, deve effettuare un Tiro Salvezza di Volontà DC 25 contro questa magia o esserne affascinato. Il bersaglio affascinato considera il vampiro un amico fidato da ascoltare e proteggere. Sebbene il bersaglio non sia sotto il controllo del vampiro, prende le richieste e le azioni del vampiro nel modo più favorevole possibile, ed è un bersaglio consenziente dell'attacco con morso del vampiro.

Ogni volta che il vampiro o i compagni del vampiro fanno qualcosa di nocivo al bersaglio, questi può ripetere il Tiro Salvezza, terminando l'effetto su di sé in caso di successo. Altrimenti, l'effetto persiste 24 ore o finché il vampiro non viene distrutto, si trova su di un piano di esistenza diverso dal bersaglio, o effettua una Reazione per terminare l'effetto.

\emph{\textbf{Figli della Notte (1/Giorno).}} Il vampiro richiama magicamente 2d4 sciami di pipistrelli o ratti, purché il sole non sia sorto. Mentre è all'esterno, il vampiro può richiamare invece 3d6 lupi. Le creature richiamate arrivano in 1d4 round, agendo da alleati del vampiro e obbedendo ai suoi comandi. Le bestie restano per 1 ora, finché il vampiro non muore, o finché non le congeda con un'Azione Immediata.

\textbf{Reazione: \emph{Attacco d'opportunità}}: il vampiro effettua un attacco ad una creatura che attraversi o esca dalla sua portata di 1 metro.

\textbf{Azioni Aggiuntive}

Il vampiro può effettuare 3 Azioni aggiuntive, scelte tra le opzioni seguenti. Può usare solo un'opzione Aggiuntiva alla volta e solo al termine del round di un'altra creatura. Il vampiro recupera all'inizio del proprio round le Azioni aggiuntive che ha speso.

\textbf{Colpo Disarmato.} Il vampiro effettua un colpo disarmato.

\textbf{Morso (Costa 2 Azioni).} Il vampiro effettua un attacco con morso.

\textbf{Muoversi.} Il vampiro si muove del suo movimento senza provocare attacchi di opportunità.

\textbf{Ecologia}
Ambiente: Qualsiasi\\
Organizzazione: Solitario o famiglia (vampiro più 2-8 Progenie)\\
\textbf{Categoria Tesoro}: Equipaggiamento da PNG (Anello della Protezione +2, Fascia della Seduzione +4, Mantello della Resistenza +3)\\
\textbf{Descrizione}\\
I vampiri sono creature umanoidi non morte che si nutrono del sangue dei viventi. Hanno un aspetto molto simile a quando erano in vita, diventando spesso più attraenti, sebbene alcuni appaiano invece duri e ferini.

\mostro{Progenie Vampirica}
\noindent
\begin{description}[noitemsep, topsep=0pt, parsep=0pt, partopsep=0pt, leftmargin=0cm, labelwidth=2.2cm]
	\item[\textbf{Taglia/Tipo:}] Media non morto, malvagio
	\item[\textbf{Movimento:}] 9 m
	\item[\textbf{Tiri Salvez.:}] \resizebox{0.5\linewidth+1.8cm}{!}{Tempra +9, Riflessi +9, Volontà +6}
	\item[\textbf{Caratt.:}] \resizebox{0.5\linewidth+1.8cm}{!}{For 3 Des 3 Cos 3 Int 0 Sag 0 Car 1}
	\item[\textbf{Punti Ferita:}] 126,  \textbf{Difesa:} 23,  \textbf{Iniziativa:} +3
	\item[\textbf{Comp.:}] Furtività +6
	\item[\textbf{Res. Danni:}] da Vuoto; da arma non magica
	\item[\textbf{Sensi:}] Scurovisione 18 m
	\item[\textbf{Linguaggi:}] le lingue che conosceva in vita
	\item[\textbf{Sfida:}] 6 (2300 PX)\smallskip
\end{description}

\emph{\textbf{Debolezze della Progenie Vampirica.}} La Progenie Vampirica ha i seguenti difetti:

\emph{Danneggiato dall'Acqua Corrente.} La Progenie Vampirica subisce 20 danni da acido se termina il suo round all'interno dell'acqua corrente.

\emph{Ipersensibilità alla Luce.} La Progenie Vampirica subisce 20 danni da Luce quando inizia il suo round alla luce del sole. Mentre è alla luce del sole, ha -1d6 ai tiri di attacco e le prove di competenza di Base.

\emph{Paletto nel Cuore.} La Progenie Vampirica è distrutto se un'arma perforante di legno gli viene conficcata nel cuore mentre è inabile all'interno del suo luogo di riposo.

\emph{Proibizione.} La Progenie Vampirica non può entrare in un'abitazione senza invito da parte dei suoi occupanti.

\emph{\textbf{Natura Non Morta.}} La Progenie Vampirica non ha bisogno di aria.

\emph{\textbf{Rigenerazione.}} La Progenie Vampirica recupera 10 Punti Ferita all'inizio del suo round se possiede almeno 1 punto ferita e non è esposto alla luce del sole o l'acqua corrente. Se la Progenie Vampirica subisce danno da Luce o danno dall'Acqua santa, questo tratto non funziona all'inizio del prossimo round del vampiro.

\emph{\textbf{Scalare come Ragno.}} La Progenie Vampirica può scalare superfici difficili, compreso lo stare a testa in giù sul soffitto, senza bisogno di effettuare una prova di competenza.

\textbf{Azioni}

\emph{\textbf{Multiattacco.}} La progenie vampirica può effettuare due attacchi, ma solo uno di essi può essere un attacco con morso.

\emph{\textbf{Artigli.} Attacco con arma da mischia}: +8 a colpire, portata 1 m, una creatura.

\emph{Colpisce:} 8 (2d4 + 3) danni taglienti. Invece di infliggere danno, il vampiro può afferrare il bersaglio (DC per fuggire 13).

\emph{\textbf{Morso.} Attacco con arma da mischia}: +8 a colpire, portata 1 m, una creatura afferrata dal vampiro, inabile o intralciata.

\emph{Colpisce:} 6 (1d6 + 3) danni perforanti più 7 (2d6) danni da Vuoto. I Punti Ferita massimi del bersaglio sono ridotti di un ammontare pari al danno da Vuoto subito, e il vampiro recupera un numero di Punti Ferita pari a quell'ammontare, TS Tempra DC 16 per resistere alla perdita di Punti Ferita massimi. Il bersaglio muore se questo effetto riduce i suoi Punti Ferita massimi a 0. La creatura diventa Affaticata.

\textbf{Reazione: \emph{Attacco d'opportunità}}: la progenie vampirica effettua un attacco ad una creatura che attraversi o esca dalla sua portata di 1 metro.

\textbf{Ecologia}\\
Ambiente: Qualsiasi\\
Organizzazione: Solitario, coppia, gruppo (3-6) o turba (7-12)\\
\textbf{Categoria Tesoro}: M\\
\textbf{Descrizione}\\
Un Vampiro può decidere di creare da una vittima una progenie vampirica anziché farne un vampiro completo solo quando usa la sua capacità creare progenie su una creatura umanoide. Questa decisione deve essere presa appena un vampiro uccide una creatura appropriata usando il morso.

\mostro{Vermi delle carne}
\noindent
\begin{description}[noitemsep, topsep=0pt, parsep=0pt, partopsep=0pt, leftmargin=0cm, labelwidth=2.2cm]
	\item[\textbf{Taglia/Tipo:}] minuscola mostruosità, disallineato
	\item[\textbf{Caratt.:}] \resizebox{0.5\linewidth+1.8cm}{!}{For -4 Des 0 Cos -2 Int -4 Sag 0 Car -4}
	\item[\textbf{Punti Ferita:}] 32,  \textbf{Difesa:} 13,  \textbf{Iniziativa:} +0
	\item[\textbf{Movimento:}] 1 m
	\item[\textbf{Tiri Salvez.:}] \resizebox{0.5\linewidth+1.8cm}{!}{Tempra +3, Riflessi +3, Volontà +3}
	\item[\textbf{Sensi:}] vista tellurica 3 m
	\item[\textbf{Sfida:}] 1 (200 PX)\smallskip
\end{description}

\textbf{Azioni}

\emph{\textbf{Infestare la carne.}} Queste minuscole creature penetrano nella carne esposta senza effettuare Tiro per Colpire purché la carne sia esposta a contatto con loro.

\emph{\textbf{Colpisce.}} Entro 2d4 round i vermi (3d6 creature) della carne scavano nel tessuto dirigendosi verso il cuore. L'infestazione dei vermi causa 1 Punto Ferita di danno a round mentre scavano. Una volta arrivati al cuore ogni round il personaggio deve fare un Tiro Salvezza su Tempra DC 14, con penalità cumulativa di -1 per round. Una volta che il Tiro Salvezza fallisce il personaggio muore.

\emph{\textbf{Debellare i Vermi della carne.}} L'unico modo è usare una fiamma viva (una torcia causa 1d6 di danno ad applicazione od un incantesimo tipo Onda rovente) sulla parte dove i vermi stanno scavando. Ogni applicazione di fuoco può eliminare 3d6 vermi. Una prova di Pronto Soccorso a DC 15 rimuove 1d4 parassiti ma causa 1d4 danni nell'estrazione. Passati i 2d4 round i vermi sono troppo in profondità ed è inutile applicare il fuoco, solo un incantesimo di Cura Malattie, o Guarigione, può debellare completamente l'infestazione.

\textbf{Ecologia}\\
Ambiente: alberi marci, carne putrefatta\\
Organizzazione: gruppi 3d6\\
\textbf{Categoria Tesoro}: Nessuno\\
\textbf{Descrizione}\\
I vermi della carne sono tra i più temuti parassiti dagli avventurieri. Si trovano nei cumuli umidi di foglie o tronchi marci, nei cadaveri in putrefazione, nell acque torbide. Pallidi, viscidi, dotati di affilatissimi denti, lunghi poco più di 4 millimetri penetrano nella carne esposta in maniera facilissima e percepiscono il battito del cuore dove si dirigono. Mentre scavano nelle carni si possono percepire ed anche vedere strisciare sottopelle.

\mostro{Verme Purpureo}
\noindent
\begin{description}[noitemsep, topsep=0pt, parsep=0pt, partopsep=0pt, leftmargin=0cm, labelwidth=2.2cm]
	\item[\textbf{Taglia/Tipo:}] Mastodontica mostruosità, disallineato
	\item[\textbf{Caratt.:}] \resizebox{0.5\linewidth+1.8cm}{!}{For 9 Des -2 Cos 6 Int -5 Sag -1 Car -3}
	\item[\textbf{Punti Ferita:}] 303,  \textbf{Difesa:} 30,  \textbf{Iniziativa:} -2
	\item[\textbf{Movimento:}] 15 m, scavo 9 m
	\item[\textbf{Tiri Salvez.:}] \resizebox{0.5\linewidth+1.8cm}{!}{\resizebox{0.5\linewidth+1.8cm}{!}{Tempra +21, Riflessi +13, Volontà +14}}
	\item[\textbf{Sensi:}] Vista Cieca 9 m, senso tellurico 18 m
	\item[\textbf{Sfida:}] 15 (13000 PX)\smallskip
\end{description}

\emph{\textbf{Scavatore di Tunnel.}} Il verme può scavare attraverso la roccia solida a metà della velocità di scavare e lascia un tunnel di 3 metri di diametro dietro di se.

\textbf{Azioni}

\emph{\textbf{Multiattacco.}} Il verme effettua due attacchi: uno con il morso e uno con il pungiglione.

\emph{\textbf{Morso.} Attacco con arma da mischia}: +13 a colpire, portata 3 m, un bersaglio.

\emph{Colpisce:} 22 (3d8 + 9) danni perforanti. Se il bersaglio è una creatura di taglia Grande, deve riuscire un Tiro Salvezza di Riflessi DC 28 o venire inghiottita dal verme. Mentre è inghiottita, la creatura è accecata e intralciata, ha copertura completa contro gli attacchi e altri effetti provenienti dall'esterno del verme, e subisce 21 (6d6) danni da acido all'inizio di ciascun round del verme.

Se il verme subisce 30 o più danni in un singolo round da una creatura al suo interno, il verme deve riuscire un Tiro Salvezza di Tempra DC 25 al termine del suo round o vomitare tutte le creature inghiottite, che cadono prone in uno spazio entro 3 metri dal verme. Se il verme muore, una creatura inghiottita non risulta più intralciata da esso e può fuggire dal cadavere usando 2 Azioni e uscendo prona.

\emph{\textbf{Pungiglione.} Attacco con arma da mischia}: +13 a colpire, portata 3 m, una creatura.

\emph{Colpisce:} 19 (3d6 + 9) danni perforanti e il bersaglio deve effettuare un Tiro Salvezza di Tempra DC 28, subendo 42 (12d6) danni da veleno o la metà di questi danni se lo riesce.

\emph{\textbf{Avviluppare.} Attacco con arma da mischia}: +12 a colpire, portata 3 m, una creatura. Il verme purpureo si stringe attorno alla creatura. 2 Azioni

\emph{Colpisce:} 30 (8d6 + 9) danni contundenti ed afferrato. Ad ogni round la creatura subisce 15 di danno da stritolamento, TS Tempra DC 24 per liberarsi.

\textbf{Ecologia}\\
Ambiente: Qualsiasi sotterraneo\\
Organizzazione: Solitario\\
\textbf{Categoria Tesoro}: Accidentale\\
\textbf{Descrizione}\\
I vermi purpurei sono giganteschi necrofagi che abitano nelle regioni più profonde del mondo, mangiando qualsiasi materiale organico incontrino. Sono noti per inghiottire le loro prede intere. Non è insolito sentire di un gruppo di avventurieri scomparso all'interno delle fameliche fauci di un verme purpureo, gridando di terrore mentre i suoi membri sparivano uno alla volta.

Mentre vanno in cerca di creature viventi per divorarle, i vermi purpurei ingoiano anche un'enorme quantità di terra e minerali scavando nel sottosuolo. Le interiora di un verme purpureo possono contenere un considerevole numero di gemme e altri oggetti in grado di resistere all'acido corrosivo all'interno del suo esofago. In zone ricche di minerali preziosi, come quelle vicine alle miniere naniche, i tunnel naturali creati dagli scavi dei vermi purpurei sono spesso pieni di un notevole numero di pepite d'oro grezzo.

Un verme purpureo generalmente reclama una grande caverna sotterranea come sua tana, e anche se vi torna per riposare e digerire il cibo, passa la maggior parte del suo tempo in cerca di preda, scavando attraverso l'oscurità senza fine o scivolando lungo tunnel preesistenti alla costante ricerca di cibo per saziare la sua immensa fame. Sebbene quasi privi di intelletto, i vermi purpurei raramente sono stupidi. Sono diffusi come guardiani fra chi riesce a controllarli magicamente o hanno nel loro covo una stanza abbastanza grande da ospitarli.

\mostro{Verme Strisciante Tentacolato}
\noindent
\begin{description}[noitemsep, topsep=0pt, parsep=0pt, partopsep=0pt, leftmargin=0cm, labelwidth=2.2cm]
	\item[\textbf{Taglia/Tipo:}] Grande mostruosità, disallineato
	\item[\textbf{Caratt.:}] \resizebox{0.5\linewidth+1.8cm}{!}{For 4 Des 1 Cos 3 Int -4 Sag 1 Car -3}
	\item[\textbf{Punti Ferita:}] 89,  \textbf{Difesa:} 18,  \textbf{Iniziativa:} +1
	\item[\textbf{Movimento:}] 9 m, scalare 9 m
	\item[\textbf{Tiri Salvez.:}] \resizebox{0.5\linewidth+1.8cm}{!}{Tempra +7, Riflessi +5, Volontà +5}
	\item[\textbf{Sensi:}] Scurovisione 18 m
	\item[\textbf{Sfida:}] 4 (1100 PX)\smallskip
\end{description}

\emph{\textbf{Scalare come Ragno.}} Il Verme Strisciante Tentacolato può scalare superfici difficili, compreso lo stare a testa in giù sul soffitto, senza bisogno di effettuare una prova di competenza.

\textbf{Azioni}

\emph{\textbf{Multiattacco.}} Il Verme Strisciante Tentacolato effettua 3 attacchi, uno con il morso e due con i tentacoli.

\emph{\textbf{Morso.} Attacco con arma da mischia}: +7 a colpire, portata 1 m, un bersaglio.

\emph{Colpisce:} 10 (2d8 + 6) danni perforanti.

\emph{\textbf{Tentacolo.} Attacco con arma da mischia}: +7 a colpire, portata 3 m, una creatura.

\emph{Colpisce:} 1 danno contundente. Il bersaglio deve effettuare un Tiro Salvezza di Tempra DC 18 o rimanere paralizzato fino alla fine del round successivo.

\textbf{Ecologia}\\
Ambiente: Qualsiasi sotterraneo\\
Organizzazione: Solitario, paio, tribù (8-12 +3d6 piccoli)\\
\textbf{Categoria Tesoro}: Accidentale\\
\textbf{Descrizione}\\
Un tipico Verme Strisciante Tentacolato è un anellide lungo quasi 3 metri e pesa sui 400 kilogrammi. Di colore scuro (di varie gradazioni dal blu al verde al marrone) è un largo verme dotato di una possente bocca e lunghi e leggeri tentacoli lungo tutta la testa.

Il Verme Strisciante Tentacolato pur se dotato di corte zampe non cammina ma striscia secernendo un muco appiccicoso che gli permette di arrampicarsi anche su superfici in qualsiasi orientamento.

Sono creature fameliche che non perdono occasione per cacciare e divorare o conservare i cadaveri dove seminare le loro uova. Amano la carne di Nibali e si nutrono di qualsiasi creatura vivente (spesso ratti dato il tipico ambiente delle fogne).

Le origini dei Vermi Striscianti Tentacolato sono piuttosto speculative, alcuni ipotizzano che un incantatore abbia provato, come al solito, fallendo criticamente, a trasformarsi in un Verme Purpureo, altri credono fermamente che i giardini di Shayalia avessero bisogni di maggiore concimazione e così la Patrona trasformò dei normali lombrichi in queste terrificanti creature perché divorassero e digerissero i cadaveri seppelliti.

\mostro{Viverna}
\noindent
\begin{description}[noitemsep, topsep=0pt, parsep=0pt, partopsep=0pt, leftmargin=0cm, labelwidth=2.2cm]
	\item[\textbf{Taglia/Tipo:}] Grande drago, disallineato
	\item[\textbf{Caratt.:}] \resizebox{0.5\linewidth+1.8cm}{!}{For 4 Des 0 Cos 3 Int -3 Sag 1 Car -2}
	\item[\textbf{Punti Ferita:}] 126,  \textbf{Difesa:} 20,  \textbf{Iniziativa:} +0
	\item[\textbf{Movimento:}] 6 m, volo 24 m
	\item[\textbf{Tiri Salvez.:}] \resizebox{0.5\linewidth+1.8cm}{!}{Tempra +9, Riflessi +6, Volontà +7}
	\item[\textbf{Sensi:}] Scurovisione 18 m
	\item[\textbf{Sfida:}] 6 (2300 PX)\smallskip
\end{description}

\textbf{Azioni}

\emph{\textbf{Multiattacco.}} La viverna può effettuare due attacchi: uno con il morso e uno con il pungiglione. Mentre vola, può usare i suoi artigli al posto di uno degli altri attacchi.

\emph{\textbf{Artigli.} Attacco con arma da mischia}: +8 a colpire, portata 1 m, un bersaglio.

\emph{Colpisce:} 13 (2d8 + 4) danni taglienti, 1 danno da Sanguinamento.

\emph{\textbf{Morso.} Attacco con arma da mischia}: +8 a colpire, portata 3 m, una creatura.

\emph{Colpisce:} 11 (2d6 + 4) danni perforanti.

\emph{\textbf{Pungiglione.} Attacco con arma da mischia}: +8 a colpire, portata 3 m, una creatura.

\emph{Colpisce:} 11 (2d6 + 4) danni perforanti. Il bersaglio deve effettuare un Tiro Salvezza di Tempra DC 18, e subire 24 (7d6) danni da veleno se lo fallisce, o la metà di questi danni se lo riesce.

\textbf{Reazione: \emph{Attacco d'opportunità}}: la viverna nero effettua un attacco con Artiglio ad una creatura che attraversi o esca dalla sua portata di 3 metri.

\emph{\textbf{Arrabbiato:}} la Viverna punta la coda in direzione del nemico e genera un cono di 3 metri di veleno. E' possibile eseguire un Tiro Salvezza su Riflessi DC 21 per dimezzare i 7d8 di danno da veleno.

\textbf{Ecologia}\\
Ambiente: Colline temperate o calde\\
Organizzazione: Solitario, coppia o stormo (3-6)\\
\textbf{Categoria Tesoro}: D\\
\textbf{Descrizione}\\
Le viverne sono rettili brutali e violenti imparentati con i draghi. Sono sempre aggressive ed impazienti e preferiscono raggiungere i loro scopi utilizzando la forza. Per questa ragione, i draghi guardano alle viverne con superiorità, considerando questi loro lontani parenti come selvaggi primitivi privi di stile ed intelligenza.

Nella maggior parte dei casi, questa generalizzazione è azzeccata. Anche se non certo di intelletto animale e capace di parola, la maggior parte delle viverne non si cura della diplomazia, preferendo combattere prima e discutere poi, solo se si trovano davanti ad un avversario che non possono sconfiggere o da cui non possono fuggire.

Le viverne sono creature territoriali. Pur cacciando occasionalmente prede più grandi in gruppi più estesi, sono creature solitarie il cui territorio di caccia si estende dai 160 ai 320 km quadrati. È noto che le viverne combattono spesso fra loro fino alla morte per le contese su un territorio ricco di prede.

Seppur costantemente affamate ed inclini ad attaccare, una viverna può essere resa amichevole attraverso un'attenta combinazione di lusinghe, intimidazione, cibo e tesoro, per farne un potente alleato. Spesso servono Giganti e Umanoidi Mostruosi come guardiani come guardiani, ed alcune tribù di Boggard e Lucertoloidi le usano come cavalcature, anche se tali accordi spesso risultano parecchio costosi in termini di cibo ed oro, poiché sono poche le viverne che accettano di servire a lungo creature simili come cavalcature.

Una viverna è lunga circa 4,8 metri e la coda rappresenta da sola circa metà della lunghezza. Una viverna pesa in media 1000 kg.

%\addcontentsline{toc}{subsubsection}{W}
\pdfbookmark[3]{W}{W}

\mostro{Wight}
\noindent
\begin{description}[noitemsep, topsep=0pt, parsep=0pt, partopsep=0pt, leftmargin=0cm, labelwidth=2.2cm]
	\item[\textbf{Taglia/Tipo:}] Media non morto, malvagio
	\item[\textbf{Caratt.:}] \resizebox{0.5\linewidth+1.8cm}{!}{For 2 Des 2 Cos 3 Int 0 Sag 1 Car 2}
	\item[\textbf{Punti Ferita:}] 70,  \textbf{Difesa:} 18,  \textbf{Iniziativa:} +2
	\item[\textbf{Movimento:}] 9 m
	\item[\textbf{Tiri Salvez.:}] \resizebox{0.5\linewidth+1.8cm}{!}{Tempra +6, Riflessi +5, Volontà +4}
	\item[\textbf{Comp.:}] Furtività +4, Consapevolezza +3
	\item[\textbf{Res. Danni:}] da Vuoto; da arma non magica o che non sia argentata
	\item[\textbf{Imm. Danni:}] Veleno
	\item[\textbf{Immunità:}] affaticato, sanguinamento
	\item[\textbf{Sensi:}] Scurovisione 18 m
	\item[\textbf{Linguaggi:}] le lingue che conosceva in vita, Expiran
	\item[\textbf{Sfida:}] 3 (700 PX)\smallskip
\end{description}

\emph{\textbf{Natura Non Morta.}} Il wight non ha bisogno di aria, cibo, bevande o sonno.

\emph{\textbf{Sensibilità alla Luce}}. Mentre è alla luce del sole, il wight ha -1d6 ai tiri di attacco, oltre che alle prove di Consapevolezza basate sulla vista.

\textbf{Azioni}

\emph{\textbf{Multiattacco.}} Il wight può effettuare due attacchi con la spada lunga o due attacchi con l'arco lungo. Può usare Risucchiare Vita al posto di uno dei suoi attacchi con la spada lunga.

\emph{\textbf{Risucchiare Vita.} Attacco con arma da mischia}: +6 a colpire, portata 1 m, una creatura.

\emph{Colpisce:} 5 (1d6 + 2) danni da Vuoto. Il bersaglio deve riuscire un Tiro Salvezza di Tempra DC 14 o vedere i suoi Punti Ferita massimi ridotti di un ammontare pari al danno subito. Il bersaglio diviene Affaticato. Il bersaglio muore se l'effetto riduce i suoi Punti Ferita massimi a 0.

Un umanoide ucciso da questo attacco si rianima 24 ore più tardi come zombi sotto il controllo del wight, a meno che l'umanoide non venga prima riportato in vita o il corpo sia distrutto. Il wight non può controllare più di dodici zombi alla volta.

\emph{\textbf{Spada Lunga.} Attacco con arma da mischia}: +5 a colpire, portata 1 m, un bersaglio.

\emph{Colpisce:} 6 (1d8 + 2) danni taglienti o 7 (1d10 + 2) danni taglienti se usata con due mani.

\emph{\textbf{Arco Lungo.} Attacco con arma a Distanza}: +5 a colpire, gittata 45m, un bersaglio.

\emph{Colpisce:} 6 (1d8 + 2) danni perforanti.

\textbf{Ecologia}\\
Ambiente: qualsiasi\\
Organizzazione: Solitario, coppia, gruppo (3-6) o branco (7-12)\\
\textbf{Categoria Tesoro}: Q\\
\textbf{Descrizione}\\
I wight sono umanoidi risorti come non morti a causa della necromanzia, di una morte violenta o di una personalità estremamente malevola. In alcuni casi, un wight sorge quando uno spirito non morto si lega permanentemente ad un cadavere, spesso quello di un guerriero. Sono appena riconoscibili da chi li conosceva in vita: le loro carni sono corrotte dalla malvagità e dalla non morte, gli occhi ardono d'odio ed i denti divengono quelli di una bestia. In un certo senso, un wight è l'anello di congiunzione tra ghoul e spettri: un cadavere deforme che risucchia energia vitale col tocco.

Essendo non morti, i wight non hanno bisogno di respirare, così a volte si possono trovare sott'acqua, sebbene non siano nuotatori particolarmente abili a meno che non siano originati da creature nuotatrici quali elfi acquatici e marinidi. Sott'acqua i wight preferiscono le caverne dal soffitto basso dove le loro scarse capacità di nuoto non sono una limitazione.

\mostro{Wraith}
\noindent
\begin{description}[noitemsep, topsep=0pt, parsep=0pt, partopsep=0pt, leftmargin=0cm, labelwidth=2.2cm]
	\item[\textbf{Taglia/Tipo:}] Media non morto, malvagio
	\item[\textbf{Caratt.:}] \resizebox{0.5\linewidth+1.8cm}{!}{For -2 Des 3 Cos 3 Int 1 Sag 2 Car 2}
	\item[\textbf{Punti Ferita:}] 108,  \textbf{Difesa:} 21,  \textbf{Iniziativa:} +3
	\item[\textbf{Movimento:}] 0 m, volo 18 m, Fluttuare
	\item[\textbf{Tiri Salvez.:}] \resizebox{0.5\linewidth+1.8cm}{!}{Tempra +8, Riflessi +8, Volontà +7}
	\item[\textbf{Res. Danni:}] Acido, Freddo, Elettricità, Fuoco, Suono; da arma non magica o che non sia argentata
	\item[\textbf{Imm. Danni:}] da Vuoto, Veleno
	\item[\textbf{Immunità:}] affascinato, afferrato, intralciato, paralizzato, pietrificato, prono, affaticato, sanguinamento
	\item[\textbf{Sensi:}] Scurovisione 18 m
	\item[\textbf{Linguaggi:}] le lingue che conosceva in vita, Expiran
	\item[\textbf{Sfida:}] 5 (1800 PX)\smallskip
\end{description}

\emph{\textbf{Movimento Incorporeo.}} Il wraith può attraversare creature e oggetti come fossero terreno difficile. Subisce 5 (1d10) danni da forza se termina il proprio round all'interno di un oggetto.

\emph{\textbf{Natura Non Morta.}} Il wraith non ha bisogno di aria, cibo, bevande o sonno.

\emph{\textbf{Sensibilità alla Luce}}. Mentre è alla luce del sole, il wraith ha -1d6 ai tiri di attacco, oltre che alle prove di Consapevolezza basate sulla vista.

\textbf{Azioni}

\emph{\textbf{Risucchiare Vita.} Attacco con arma da mischia}: +7 a colpire, portata 1 m, una creatura.

\emph{Colpisce:} 21 (4d8 + 3) danni da Vuoto. Il bersaglio deve riuscire un Tiro Salvezza di Tempra DC 16 o vedere i suoi Punti Ferita massimi ridotti di un ammontare pari al danno subito. Il bersaglio diviene Affaticato. Il bersaglio muore se l'effetto riduce i suoi Punti Ferita massimi a 0.

\emph{\textbf{Creare Spettro.}} Il wraith prende a bersaglio un umanoide entro 3 metri da esso e che sia morto da non più di 1 minuto e per cause violente. Lo spirito del bersaglio si anima come spettro nello spazio del suo cadavere e nello spazio più vicino non occupato. Lo spettro è sotto il controllo del wraith. Il wraith non può tenere più di sette spettri alla volta sotto il suo controllo.

\textbf{Reazione: \emph{Attacco d'opportunità}}: il Wraith effettua un attacco di Risucchiare Vita ad una creatura che attraversi o esca dalla sua portata di 1 metro.

\emph{\textbf{Arrabbiato:}} il Wraith canalizza le sue energie negative in una esplosione di Vuoto attorno a se nel raggio di 6 metri. Tutte le creature devono effettuare un Tiro Salvezza su Tempra DC 16 o subito 3d6 di danno da Vuoto, se il Tiro Salvezza riesce sono Rallentate 1/3r.

\textbf{Ecologia}\\
Ambiente: Qualsiasi\\
Organizzazione: Solitario, coppia, gruppo (3-6) o branco (7-12)\\
\textbf{Categoria Tesoro}: Nessuno\\
\textbf{Descrizione}\\
I wraith sono creature nate dal male e dall'oscurità. Detestano la luce e le creature viventi, avendo perduto la maggior parte del legame con la loro vita precedente.

%\addcontentsline{toc}{subsubsection}{X}
\pdfbookmark[3]{X}{X}

\mostro{Xorn}
\noindent
\begin{description}[noitemsep, topsep=0pt, parsep=0pt, partopsep=0pt, leftmargin=0cm, labelwidth=2.2cm]
	\item[\textbf{Taglia/Tipo:}] Media elementale, neutrale
	\item[\textbf{Caratt.:}] \resizebox{0.5\linewidth+1.8cm}{!}{For 3 Des 0 Cos 6 Int 0 Sag 0 Car 0}
	\item[\textbf{Punti Ferita:}] 111,  \textbf{Difesa:} 18,  \textbf{Iniziativa:} +0
	\item[\textbf{Movimento:}] 6 m, scavo 6 m
	\item[\textbf{Tiri Salvez.:}] \resizebox{0.5\linewidth+1.8cm}{!}{Tempra +11, Riflessi +5, Volontà +5}
	\item[\textbf{Comp.:}] Furtività +3, Consapevolezza +6
	\item[\textbf{Res. Danni:}] perforante e tagliente di armi non magiche o che non siano di adamantio
	\item[\textbf{Sensi:}] Scurovisione 18 m, senso tellurico 18 m
	\item[\textbf{Linguaggi:}] Tremun
	\item[\textbf{Sfida:}] 5 (1800 PX)\smallskip
\end{description}

\emph{\textbf{Mimetismo di Pietra.}} Lo xorn ha +1d6 alle prove di Furtività (Nascondersi) effettuate per nascondersi su terreno roccioso.

\emph{\textbf{Scorrere sulla Terra.}} Lo xorn può scavare attraversa la terra e la pietra non magiche e non lavorate. Quando lo fa, lo xorn non disturba il materiale che sposta.

\emph{\textbf{Senso del Tesoro.}} Lo xorn può individuare precisamente, con l'olfatto, la posizione di metalli e pietre preziose, come monete e gemme, entro 18 metri da esso.

\textbf{Azioni}

\emph{\textbf{Multiattacco.}} Lo xorn effettua tre attacchi di artiglio e un attacco di morso.

\emph{\textbf{Artiglio.} Attacco con arma da mischia}: +7 a colpire, portata 1 m, un bersaglio.

\emph{Colpisce:} 6 (1d6 + 3) danni taglienti, 1 danno da Sanguinamento.

\emph{\textbf{Morso.} Attacco con arma da mischia}: +6 a colpire, portata 1 m, un bersaglio.

\emph{Colpisce:} 13 (3d6 + 3) danni perforanti.

\emph{\textbf{Arrabbiato:}} lo Xorn erutta le ultime gemme e pepite mangiate. In un cono di 6 metri tutte le creature devono fare un Tiro Salvezza su Riflessi DC 18 per dimezzare il danno, 3d8 contundente, delle gemme e minerali scagliati (che però non hanno più valore).

\textbf{Ecologia}\\
Ambiente: Qualsiasi (Piano della Terra)\\
Organizzazione: Solitario, coppia o gruppo (3-6)\\
\textbf{Categoria Tesoro}: solo metalli preziosi, gemme e gioielli e gemme magiche\\
\textbf{Descrizione}\\
Strane creature larghe quanto alte, gli xorn hanno poco interesse verso i nativi del Piano Materiale, non fosse per le gemme ed i metalli preziosi che potrebbero avere con sé. Nascosti sotto la superficie del terreno per un tempo che ad un umano potrebbe sembrare lunghissimo, uno xorn può attendere mesi, perfino anni, per la preda ideale, per poi assalire chi porta con sé il suo cibo preferito, come una gemma particolare o un determinato tipo di argento. Gli avventurieri che si addentrano nelle regioni abitate dagli xorn portano spesso con sé piccole pepite di minerali o gemme e cristalli di scarso valore da utilizzare come tributo. Anche se il suo valore è solitamente direttamente proporzionale al suo sapore e all'appetibilità che esso può avere, la maggior parte degli xorn è piuttosto ingorda, e preferisce la quantità alla qualità.

Il tesoro che uno xorn porta con sé o nasconde nella sua tana consiste in uno spuntino che ha conservato per il giorno successivo. Offrire un gioiello o un metallo preziosi particolarmente deliziosi (e costosi) ad uno xorn può cementare un'alleanza temporanea. Dato che gli xorn possono attraversare la roccia con facilità sono ottime guide nelle regioni sotterranee.

Gli xorn non sono molto religiosi, ma quelli fra loro che trovano la fede sono solitamente devoti a Efrem (anche se è raro, se non improbabile, che gli xorn abbiano Compagni Animali, dato che non possono seguirli nella roccia, e scelgono invece il dominio della Terra). Bardi e Devoti xorn non sono sconosciuti: i Bardi scelgono di solito Intrattenere (canto), e gli Devoti hanno invariabilmente la Stirpe Elementale (terra).

%\addcontentsline{toc}{subsubsection}{Z}
\pdfbookmark[3]{Z}{Z}

\mostro{Zombi}
\noindent
\begin{description}[noitemsep, topsep=0pt, parsep=0pt, partopsep=0pt, leftmargin=0cm, labelwidth=2.2cm]
	\item[\textbf{Taglia/Tipo:}] Media non morto, malvagio
	\item[\textbf{Caratt.:}] \resizebox{0.5\linewidth+1.8cm}{!}{For 1 Des -2 Cos 3 Int -4 Sag -2 Car -3}
	\item[\textbf{Punti Ferita:}] 19,  \textbf{Difesa:} 10,  \textbf{Iniziativa:} -2
	\item[\textbf{Movimento:}] 6 m
	\item[\textbf{Tiri Salvez.:}] \resizebox{0.5\linewidth+1.8cm}{!}{Tempra +3, Riflessi +3, Volontà +3}
	\item[\textbf{Imm. Danni:}] Veleno
	\item[\textbf{Immunità:}] sanguinamento
	\item[\textbf{Sensi:}] Scurovisione 18 m
	\item[\textbf{Linguaggi:}] comprende tutte le lingue che parlava in vita ma non può parlare
	\item[\textbf{Sfida:}] 1/4 (50 PX)\smallskip
\end{description}

\emph{\textbf{Natura Non Morta.}} Lo zombi non ha bisogno di aria, cibo, bevande o sonno.

\emph{\textbf{Tempra dei Non Morti.}} Se il danno riduce lo zombi a 0 Punti Ferita, lo zombi deve effettuare un Tiro Salvezza di Tempra DC 5 + il danno subito, a meno che il danno non sia da Luce o un colpo critico. Se riesce, lo zombi scende invece a 1 punto ferita.

\emph{\textbf{Lento come uno Zombi.}} Lo zombie esegue solo due Azioni a round.

\textbf{Azioni}

\emph{\textbf{Schianto.} Attacco con arma da mischia}: +4 a colpire, portata 1 m, un bersaglio.

\emph{Colpisce:} 4 (1d6 + 1) danni contundenti.

\textbf{Ecologia}\\
Ambiente: Qualsiasi\\
Organizzazione: Qualsiasi\\
\textbf{Categoria Tesoro}: Nessuno\\
\textbf{Descrizione}\\
Gli zombi sono i cadaveri animati di creature morte, costretti a muoversi da magie necromantiche come Animare Morti. Anche se gli zombi incontrati di norma sono lenti e robusti, altri possiedono tratti differenti, che permettono loro di diffondere una malattia o di muoversi più rapidi.

Gli zombi sono automi senza mente e non possono fare altro che seguire gli ordini. Se lasciati a loro stessi, attendono immobili o si spostano alla ricerca di creature viventi da massacrare e divorare. Gli zombi attaccano fino alla distruzione, senza curarsi della loro sicurezza.

Sebbene siano in grado di seguire gli ordini, gli zombi vengono spesso lasciati liberi con l'ordine di uccidere tutte le creature viventi. Spesso vengono incontrati in branchi che infestano le terre frequentate dai viventi, in cerca di preda. La maggior parte degli zombi viene creata attraverso \hyperlink{Animare Morti}{Animare Morti}. Simili zombi sono sempre standard, a meno che il creatore lanci anche Velocità o Rimuovi Paralisi per creare Zombi Rapidi o Contagio per creare Zombi Infetti.

\mostro{Zombi Ogre}
\noindent
\begin{description}[noitemsep, topsep=0pt, parsep=0pt, partopsep=0pt, leftmargin=0cm, labelwidth=2.2cm]
	\item[\textbf{Taglia/Tipo:}] Grande non morto, malvagio
	\item[\textbf{Caratt.:}] \resizebox{0.5\linewidth+1.8cm}{!}{For 4 Des -2 Cos 4 Int -4 Sag -2 Car -3}
	\item[\textbf{Punti Ferita:}] 52,  \textbf{Difesa:} 12,  \textbf{Iniziativa:} -2
	\item[\textbf{Movimento:}] 9 m
	\item[\textbf{Tiri Salvez.:}] \resizebox{0.5\linewidth+1.8cm}{!}{Tempra +6, Riflessi +3, Volontà +3}
	\item[\textbf{Imm. Danni:}] Veleno
	\item[\textbf{Immunità:}] sanguinamento
	\item[\textbf{Sensi:}] Scurovisione 18 m
	\item[\textbf{Linguaggi:}] comprende Comune e Gigante ma non può parlare
	\item[\textbf{Sfida:}] 2 (450 PX)\smallskip
\end{description}

\emph{\textbf{Natura Non Morta.}} Lo zombi non ha bisogno di aria, cibo, bevande o sonno.

\emph{\textbf{Tempra dei Non Morti.}} Se il danno riduce lo zombi a 0 Punti Ferita, lo zombi deve effettuare un Tiro Salvezza di Tempra DC 5 + il danno subito, a meno che il danno non sia da Luce o un colpo critico. Se riesce, lo zombi scende invece a 1 punto ferita.

\textbf{Azioni}

\emph{\textbf{Mazza Chiodata.} Attacco con arma da mischia}: +6 a colpire, portata 1 m, un bersaglio.

\emph{Colpisce:} 13 (2d8 + 4) danni contundenti.

\textbf{Categoria Tesoro}: Nessuno

\subsection{Appendice A: Creature Varie}

Questa appendice contiene le statistiche di vari animali, parassiti e
altre creature. Le statistiche sono organizzate in ordine alfabetico.

\mostro{Albero Risvegliato}
\begin{description}[noitemsep, topsep=0pt, parsep=0pt, partopsep=0pt, leftmargin=0cm, labelwidth=2.2cm]
	\item[\textbf{Taglia/Tipo:}] Enorme pianta, disallineato
	\item[\textbf{Caratt.:}] \resizebox{0.5\linewidth+1.8cm}{!}{For 4 Des -2 Cos 2 Int 0 Sag 0 Car -2}
  \item[\textbf{Punti Ferita:}] 51,  \textbf{Difesa:} 12,  \textbf{Iniziativa:} +0
	\item[\textbf{Movimento:}] 6 m
	\item[\textbf{Tiri Salvez.:}] \resizebox{0.5\linewidth+1.8cm}{!}{Tempra +4, Riflessi +3, Volontà +3}
	\item[\textbf{Vul. al Danno:}] Fuoco
	\item[\textbf{Res. al Danno:}] contundente, perforante
	\item[\textbf{Comp.:}] Storia +2
	\item[\textbf{Sensi:}] Scurovisione 18 m
	\item[\textbf{Linguaggi:}] una lingua conosciuta dal suo creatore
	\item[\textbf{Sfida:}] 2 (450 PX)\smallskip
\end{description}

\emph{\textbf{Falso Aspetto.}} Mentre l'albero rimane immobile, è indistinguibile da un normale albero.

\textbf{Azioni}

\emph{\textbf{Schianto.} Attacco con Arma da Mischia}: +6 a colpire, portata 3 m, un bersaglio.

\emph{Colpisce:} 14 (3d6 + 4) danni contundenti.

\mostro{Alce}
\begin{description}[noitemsep, topsep=0pt, parsep=0pt, partopsep=0pt, leftmargin=0cm, labelwidth=2.2cm]
	\item[\textbf{Taglia/Tipo:}] Grande bestia, disallineato
	\item[\textbf{Caratt.:}] \resizebox{0.5\linewidth+1.8cm}{!}{For 3 Des 0 Cos 1 Int -4 Sag 0 Car -2}
	\item[\textbf{Punti Ferita:}] 19,  \textbf{Difesa:} 12,  \textbf{Iniziativa:} +0
	\item[\textbf{Tiri Salvez.:}] \resizebox{0.5\linewidth+1.8cm}{!}{Tempra +3, Riflessi +3, Volontà +3}
	\item[\textbf{Movimento:}] 15 m
	\item[\textbf{Sfida:}] 1/4 (50 PX)\smallskip
\end{description}

\emph{\textbf{Carica.}} Se l'alce si muove di almeno 6 metri diretto verso il bersaglio e lo colpisce con un attacco di rostro durante lo stesso round, il bersaglio subisce 7 (2d6) danni contundenti aggiuntivi. Se il bersaglio è una creatura, deve riuscire un Tiro Salvezza di Tempra DC 13 o cadere prono.

\textbf{Azioni}

\emph{\textbf{Rostro.} Attacco con Arma da Mischia}: +5 a colpire, portata 1 m, un bersaglio.

\emph{Colpisce:} 6 (1d6 + 3) danni contundenti.

\emph{\textbf{Zoccoli.} Attacco con Arma da Mischia}: +5 a colpire, portata 1 m, una creatura prona.

\emph{Colpisce:} 8 (2d4 + 3) danni contundenti.

\mostro{Alce Gigante}
\begin{description}[noitemsep, topsep=0pt, parsep=0pt, partopsep=0pt, leftmargin=0cm, labelwidth=2.2cm]
	\item[\textbf{Taglia/Tipo:}] Enorme bestia, disallineato
	\item[\textbf{Caratt.:}] \resizebox{0.5\linewidth+1.8cm}{!}{For 4 Des 3 Cos 2 Int -2 Sag 2 Car 0}
    \item[\textbf{Tiri Salvez.:}] \resizebox{0.5\linewidth+1.8cm}{!}{Tempra +4, Riflessi +5, Volontà +4}
	\item[\textbf{Punti Ferita:}] 51,  \textbf{Difesa:} 17,  \textbf{Iniziativa:} +3
	\item[\textbf{Movimento:}] 18 m
	\item[\textbf{Sfida:}] 2 (450 PX)\smallskip
\end{description}

\emph{\textbf{Carica.}} Se l'alce si muove di almeno 6 metri diretto verso il bersaglio e lo colpisce con un attacco di rostro durante lo stesso round, il bersaglio subisce 7 (2d6) danni contundenti aggiuntivi. Se il bersaglio è una creatura, deve riuscire un Tiro Salvezza di Tempra DC 14 o cadere prono.

\textbf{Azioni}

\emph{\textbf{Rostro.} Attacco con Arma da Mischia}: +6 a colpire, portata 3 m, un bersaglio.

\emph{Colpisce:} 11 (2d6 + 4) danni perforanti.

\emph{\textbf{Zoccoli.} Attacco con Arma da Mischia}: +6 a colpire, portata 1 m, una creatura prona.

\emph{Colpisce:} 22 (4d4 + 4) danni contundenti.

\mostro{Aquila}
\begin{description}[noitemsep, topsep=0pt, parsep=0pt, partopsep=0pt, leftmargin=0cm, labelwidth=2.2cm]
	\item[\textbf{Taglia/Tipo:}] Piccola bestia, disallineato
	\item[\textbf{Caratt.:}] \resizebox{0.5\linewidth+1.8cm}{!}{For -2 Des 2 Cos 0 Int -4 Sag 2 Car -2}
    \item[\textbf{Tiri Salvez.:}] \resizebox{0.5\linewidth+1.8cm}{!}{Tempra +3, Riflessi +3, Volontà +3}
	\item[\textbf{Punti Ferita:}] 15,  \textbf{Difesa:} 14,  \textbf{Iniziativa:} +2
	\item[\textbf{Movimento:}] 3 m, volo 18 m
	\item[\textbf{Sfida:}] 0 (10 PX)\smallskip
\end{description}

\emph{\textbf{Vista Affinata.}} L'aquila ha +1d6 nelle prove di Consapevolezza basate sulla vista.

\textbf{Azioni}

\emph{\textbf{Speroni.} Attacco con Arma da Mischia}: +4 a colpire, portata 1 m, un bersaglio.

\emph{Colpisce:} 4 (1d4 + 2) danni taglienti.

\mostro{Aquila Gigante}
\begin{description}[noitemsep, topsep=0pt, parsep=0pt, partopsep=0pt, leftmargin=0cm, labelwidth=2.2cm]
	\item[\textbf{Taglia/Tipo:}] Grande bestia, disallineato
	\item[\textbf{Caratt.:}] \resizebox{0.5\linewidth+1.8cm}{!}{For 3 Des 3 Cos 1 Int -1 Sag 2 Car 0}
    \item[\textbf{Tiri Salvez.:}] \resizebox{0.5\linewidth+1.8cm}{!}{Tempra +3, Riflessi +4, Volontà +3}
	\item[\textbf{Punti Ferita:}] 33,  \textbf{Difesa:} 16,  \textbf{Iniziativa:} +3
	\item[\textbf{Movimento:}] 3 m, volo 24 m
	\item[\textbf{Linguaggi:}] Aquila Gigante, comprende il Comune e l'Ictun ma non può parlarli
	\item[\textbf{Sfida:}] 1 (200 PX)\smallskip
\end{description}

\emph{\textbf{Vista Affinata.}} L'aquila ha +1d6 nelle prove di Consapevolezza basate sulla vista.

\textbf{Azioni}

\emph{\textbf{Multiattacco.}} L'aquila effettua due attacchi: uno con il becco e uno con gli speroni.

\emph{\textbf{Becco.} Attacco con Arma da Mischia}: +5 a colpire, portata 1 m, un bersaglio.

\emph{Colpisce:} 6 (1d6 + 3) danni perforanti.

\emph{\textbf{Speroni.} Attacco con Arma da Mischia}: +5 a colpire, portata 1 m, un bersaglio.

\emph{Colpisce:} 10 (2d6 + 3) danni taglienti.

\mostro{Avvoltoio}
\begin{description}[noitemsep, topsep=0pt, parsep=0pt, partopsep=0pt, leftmargin=0cm, labelwidth=2.2cm]
	\item[\textbf{Taglia/Tipo:}] Media bestia, disallineato
	\item[\textbf{Caratt.:}] \resizebox{0.5\linewidth+1.8cm}{!}{For -2 Des 0 Cos 1 Int -4 Sag 1 Car -3}
	\item[\textbf{Punti Ferita:}] 15,  \textbf{Difesa:} 12,  \textbf{Iniziativa:} +0
	\item[\textbf{Tiri Salvez.:}] \resizebox{0.5\linewidth+1.8cm}{!}{Tempra +3, Riflessi +3, Volontà +3}
	\item[\textbf{Movimento:}] 3 m, volo 15 m
	\item[\textbf{Sfida:}] 0 (10 PX)\smallskip
\end{description}

\emph{\textbf{Olfatto e Vista Affinati.}} L'avvoltoio ha +1d6 nelle prove di Consapevolezza basate su olfatto o vista.

\emph{\textbf{Tattiche di Branco.}} L'avvoltoio ha +1d6 al tiro di attacco contro una creatura se almeno uno degli alleati dell'avvoltoio si trova entro 1 metro dalla creatura e quell'alleato non è inabile.

\textbf{Azioni}

\emph{\textbf{Becco.} Attacco con Arma da Mischia}: +3 a colpire, portata 1 m, un bersaglio.

\emph{Colpisce:} 2 (1d4) danni perforanti.

\mostro{Avvoltoio Gigante}
\begin{description}[noitemsep, topsep=0pt, parsep=0pt, partopsep=0pt, leftmargin=0cm, labelwidth=2.2cm]
	\item[\textbf{Taglia/Tipo:}] Grande bestia, disallineato
	\item[\textbf{Caratt.:}] \resizebox{0.5\linewidth+1.8cm}{!}{For 2 Des 0 Cos 2 Int -2 Sag 1 Car -2}
	\item[\textbf{Punti Ferita:}] 15,  \textbf{Difesa:} 12,  \textbf{Iniziativa:} +0
	\item[\textbf{Tiri Salvez.:}] \resizebox{0.5\linewidth+1.8cm}{!}{Tempra +3, Riflessi +3, Volontà +3}
	\item[\textbf{Movimento:}] 3 m, volo 18 m
	\item[\textbf{Sfida:}] 0 (10 PX)\smallskip
\end{description}

\emph{\textbf{Olfatto e Vista Affinati.}} L'avvoltoio ha +1d6 nelle prove di Consapevolezza basate su olfatto o vista.

\emph{\textbf{Tattiche di Branco.}} L'avvoltoio ha +1d6 al tiro di attacco contro una creatura se almeno uno degli alleati dell'avvoltoio si trova entro 1 metro dalla creatura e quell'alleato non è inabile.

\textbf{Azioni}

\emph{\textbf{Multiattacco.}} L'avvoltoio effettua due attacchi: uno con il becco e uno con gli speroni.

\emph{\textbf{Becco.} Attacco con Arma da Mischia}: +4 a colpire, portata 1 m, un bersaglio.

\emph{Colpisce:} 7 (2d4 + 2) danni perforanti.

\emph{\textbf{Speroni.} Attacco con Arma da Mischia}: +4 a colpire, portata 1 m, un bersaglio.

\emph{Colpisce:} 9 (2d6 + 2) danni taglienti.

\mostro{Babbuino}
\begin{description}[noitemsep, topsep=0pt, parsep=0pt, partopsep=0pt, leftmargin=0cm, labelwidth=2.2cm]
	\item[\textbf{Taglia/Tipo:}] Piccola bestia, disallineato
	\item[\textbf{Caratt.:}] \resizebox{0.5\linewidth+1.8cm}{!}{For -1 Des 2 Cos 0 Int -3 Sag 1 Car -2}
	\item[\textbf{Punti Ferita:}] 15,  \textbf{Difesa:} 14,  \textbf{Iniziativa:} +2
	\item[\textbf{Tiri Salvez.:}] \resizebox{0.5\linewidth+1.8cm}{!}{Tempra +3, Riflessi +3, Volontà +3}
	\item[\textbf{Movimento:}] 6 m
	\item[\textbf{Sfida:}] 0 (10 PX)\smallskip
\end{description}

\emph{\textbf{Tattiche di Branco.}} Il babbuino ha +1d6 al tiro di attacco contro una creatura se almeno uno degli alleati del babbuino si trova entro 1 metro dalla creatura e quell'alleato non è inabile.

\textbf{Azioni}

\emph{\textbf{Morso.} Attacco con Arma da Mischia}: +1 a colpire, portata 1 m, un bersaglio.

\emph{Colpisce:} 1 (1d4 - 1) danni perforanti.

\mostro{Balena Assassina (Orca)}
\begin{description}[noitemsep, topsep=0pt, parsep=0pt, partopsep=0pt, leftmargin=0cm, labelwidth=2.2cm]
	\item[\textbf{Taglia/Tipo:}] Enorme bestia, disallineato
	\item[\textbf{Caratt.:}] \resizebox{0.5\linewidth+1.8cm}{!}{For 4 Des 0 Cos 1 Int -4 Sag 1 Car -2}
	\item[\textbf{Punti Ferita:}] 69,  \textbf{Difesa:} 16,  \textbf{Iniziativa:} +0
	\item[\textbf{Tiri Salvez.:}] \resizebox{0.5\linewidth+1.8cm}{!}{Tempra +4, Riflessi +3, Volontà +4}
	\item[\textbf{Movimento:}] 0 m, nuoto 18 m
	\item[\textbf{Sfida:}] 3 (700 PX)\smallskip
\end{description}

\emph{\textbf{Ecolocazione.}} La balena non può usare la vista cieca se assordata.

\emph{\textbf{Trattenere il Fiato.}} La balena può trattenere il fiato per 30 minuti

\emph{\textbf{Udito Affinato.}} La balena ha +1d6 alle prove di Consapevolezza basate sull'udito.

\textbf{Azioni}

\emph{\textbf{Morso.} Attacco con Arma da Mischia}: +6 a colpire, portata 1 m, un bersaglio.

\emph{Colpisce:} 21 (5d6 + 4) danni perforanti.

\mostro{Becco d'Ascia}
\begin{description}[noitemsep, topsep=0pt, parsep=0pt, partopsep=0pt, leftmargin=0cm, labelwidth=2.2cm]
	\item[\textbf{Taglia/Tipo:}] Grande bestia, disallineato
	\item[\textbf{Caratt.:}] \resizebox{0.5\linewidth+1.8cm}{!}{For 2 Des 1 Cos 1 Int -4 Sag 0 Car -3}
	\item[\textbf{Punti Ferita:}] 19,  \textbf{Difesa:} 13,  \textbf{Iniziativa:} +1
	\item[\textbf{Tiri Salvez.:}] \resizebox{0.5\linewidth+1.8cm}{!}{Tempra +3, Riflessi +3, Volontà +3}
	\item[\textbf{Movimento:}] 15 m
	\item[\textbf{Sfida:}] 1/4 (50 PX)\smallskip
\end{description}

\textbf{Azioni}

\emph{\textbf{Becco.} Attacco con Arma da Mischia}: +4 a colpire, portata 1 m, un bersaglio.

\emph{Colpisce:} 6 (1d8 + 2) danni taglienti.

\mostro{Cane della Morte}
\begin{description}[noitemsep, topsep=0pt, parsep=0pt, partopsep=0pt, leftmargin=0cm, labelwidth=2.2cm]
	\item[\textbf{Taglia/Tipo:}] Media mostruosità, malvagio
	\item[\textbf{Caratt.:}] \resizebox{0.5\linewidth+1.8cm}{!}{For 2 Des 2 Cos 2 Int -4 Sag 1 Car -2}
	\item[\textbf{Punti Ferita:}] 33,  \textbf{Difesa:} 15,  \textbf{Iniziativa:} +2
	\item[\textbf{Tiri Salvez.:}] \resizebox{0.5\linewidth+1.8cm}{!}{Tempra +3, Riflessi +3, Volontà +3}
	\item[\textbf{Movimento:}] 12 m
	\item[\textbf{Sfida:}] 1 (200 PX)\smallskip
\end{description}

\emph{\textbf{Bicefalo.}} Il cane ha +1d6 nelle prove di Consapevolezza e nei Tiri Salvezza contro le condizioni accecato, affascinato, assordato, spaventato, stordito o svenuto.

\textbf{Azioni}

\emph{\textbf{Multiattacco.}} Il cane effettua due attacchi di morso.

\emph{\textbf{Morso.} Attacco con Arma da Mischia}: +4 a colpire, portata 1 m, un bersaglio.

\emph{Colpisce:} 5 (1d6 + 2) danni perforanti. Se il bersaglio è una creatura, deve riuscire un Tiro Salvezza di Tempra DC 12 contro la malattia o restare malato finché la malattia non viene curata. Dopo ogni 24 ore, la creatura deve ripetere il Tiro Salvezza, riducendo i suoi Punti Ferita massimi di 5 (1d10) in caso di fallimento. Questa riduzione perdura finché la malattia non viene curata. La creatura muore se la malattia riduce i suoi Punti Ferita massimi a 0.

\mostro{Cane Intermittente}
\begin{description}[noitemsep, topsep=0pt, parsep=0pt, partopsep=0pt, leftmargin=0cm, labelwidth=2.2cm]
	\item[\textbf{Taglia/Tipo:}] Media mostruosità, malvagio
	\item[\textbf{Caratt.:}] \resizebox{0.5\linewidth+1.8cm}{!}{For 1 Des 3 Cos 1 Int 0 Sag 1 Car 0}
	\item[\textbf{Punti Ferita:}] 19,  \textbf{Difesa:} 15,  \textbf{Iniziativa:} +3
	\item[\textbf{Vul. al Danno:}] ferro freddo
	\item[\textbf{Tiri Salvez.:}] \resizebox{0.5\linewidth+1.8cm}{!}{Tempra +3, Riflessi +3, Volontà +3}
	\item[\textbf{Movimento:}] 12 m
	\item[\textbf{Sfida:}] 1/4 (50 PX)\smallskip
\end{description}

\emph{\textbf{Udito e Olfatto Affinato.}} Il cane ha +1d6 nelle prove di Consapevolezza basate su udito o olfatto.

\textbf{Azioni}

\emph{\textbf{Morso.} Attacco con Arma da Mischia}: +3 a colpire, portata 1 m, un bersaglio.

\emph{Colpisce:} 4 (1d6 + 1) danni perforanti.

\emph{\textbf{Teletrasporto (Ricarica 4-6).}} Il cane si teletrasporta magicamente, insieme a qualsiasi cosa stia indossando o trasportando, fino a 12 metri in uno spazio non occupato che possa vedere. Prima o dopo il teletrasporto, il cane può effettuare un attacco di morso.

\mostro{Caprone}
\begin{description}[noitemsep, topsep=0pt, parsep=0pt, partopsep=0pt, leftmargin=0cm, labelwidth=2.2cm]
	\item[\textbf{Taglia/Tipo:}] Media bestia, disallineato
	\item[\textbf{Caratt.:}] \resizebox{0.5\linewidth+1.8cm}{!}{For 1 Des 0 Cos 0 Int -4 Sag 0 Car -3}
	\item[\textbf{Punti Ferita:}] 15,  \textbf{Difesa:} 12,  \textbf{Iniziativa:} +0
	\item[\textbf{Tiri Salvez.:}] \resizebox{0.5\linewidth+1.8cm}{!}{Tempra +3, Riflessi +3, Volontà +3}
	\item[\textbf{Movimento:}] 9 m
	\item[\textbf{Sfida:}] 0 (10 PX)\smallskip
\end{description}

\emph{\textbf{Carica.}} Se il caprone si muove di almeno 6 metri diretto verso il bersaglio e colpisce con un attacco di rostro durante lo stesso round, il bersaglio subisce 2 (1d4) danni contundenti aggiuntivi. Se il bersaglio è una creatura, deve riuscire un Tiro Salvezza di Tempra DC 10 o cadere prona.

\emph{\textbf{Piedi Saldi.}} Il caprone ha +1d6 ai Tiri Salvezza su Tempra e Riflessi effettuati contro effetti che lo farebbero cadere prono.

\textbf{Azioni}

\emph{\textbf{Rostro.} Attacco con Arma da Mischia}: +3 a colpire, portata 1 m, un bersaglio.

\emph{Colpisce:} 3 (1d4 + 1) danni contundenti.

\mostro{Caprone Gigante}
\begin{description}[noitemsep, topsep=0pt, parsep=0pt, partopsep=0pt, leftmargin=0cm, labelwidth=2.2cm]
	\item[\textbf{Taglia/Tipo:}] Media bestia, disallineato
	\item[\textbf{Caratt.:}] \resizebox{0.5\linewidth+1.8cm}{!}{For 3 Des 0 Cos 1 Int -4 Sag 1 Car -2}
	\item[\textbf{Punti Ferita:}] 24,  \textbf{Difesa:} 12,  \textbf{Iniziativa:} +0
	\item[\textbf{Tiri Salvez.:}] \resizebox{0.5\linewidth+1.8cm}{!}{Tempra +3, Riflessi +3, Volontà +3}
	\item[\textbf{Movimento:}] 12 m
	\item[\textbf{Sfida:}] 1/2 (100 PX)\smallskip
\end{description}

\emph{\textbf{Carica.}} Se il caprone si muove di almeno 6 metri diretto verso il bersaglio e colpisce con un attacco di rostro durante lo stesso round, il bersaglio subisce 5 (2d4) danni contundenti aggiuntivi. Se il bersaglio è una creatura, deve riuscire un Tiro Salvezza di Tempra DC 13 o cadere prona.

\emph{\textbf{Piedi Saldi.}} Il caprone ha +1d6 ai Tiri Salvezza su Tempra e Riflessi effettuati contro effetti che lo farebbero cadere prono.

\textbf{Azioni}

\emph{\textbf{Rostro.} Attacco con Arma da Mischia}: +5 a colpire, portata 1 m, un bersaglio.

\emph{Colpisce:} 8 (2d4 + 3) danni contundenti.

\mostro{Cavallo Marino Gigante}
\begin{description}[noitemsep, topsep=0pt, parsep=0pt, partopsep=0pt, leftmargin=0cm, labelwidth=2.2cm]
	\item[\textbf{Taglia/Tipo:}] Grande bestia, disallineato
	\item[\textbf{Caratt.:}] \resizebox{0.5\linewidth+1.8cm}{!}{For 1 Des 2 Cos 0 Int -4 Sag 1 Car -3}
	\item[\textbf{Punti Ferita:}] 24,  \textbf{Difesa:} 14,  \textbf{Iniziativa:} +2
	\item[\textbf{Tiri Salvez.:}] \resizebox{0.5\linewidth+1.8cm}{!}{Tempra +3, Riflessi +3, Volontà +3}
	\item[\textbf{Movimento:}] 0 m, nuoto 12 m
	\item[\textbf{Sfida:}] 1/2 (100 PX)\smallskip
\end{description}

\emph{\textbf{Carica.}} Se il cavallo marino si muove di almeno 6 metri diretto verso il bersaglio e colpisce con un attacco di rostro durante lo stesso round, il bersaglio subisce 7 (2d6) danni contundenti aggiuntivi. Se il bersaglio è una creatura, deve riuscire un Tiro Salvezza su Tempra DC 11 o cadere prona.

\emph{\textbf{Respirare Acqua.}} Il cavallo marino può respirare solo sott'acqua.

\textbf{Azioni}

\emph{\textbf{Rostro.} Attacco con Arma da Mischia}: +3 a colpire, portata 1 m, un bersaglio.

\emph{Colpisce:} 4 (1d6 + 1) danni contundenti.

\mostro{Cervo}
\begin{description}[noitemsep, topsep=0pt, parsep=0pt, partopsep=0pt, leftmargin=0cm, labelwidth=2.2cm]
	\item[\textbf{Taglia/Tipo:}] Media bestia, disallineato
	\item[\textbf{Caratt.:}] \resizebox{0.5\linewidth+1.8cm}{!}{For 0 Des 3 Cos 0 Int -4 Sag 2 Car -3}
	\item[\textbf{Punti Ferita:}] 15,  \textbf{Difesa:} 15,  \textbf{Iniziativa:} +3
	\item[\textbf{Tiri Salvez.:}] \resizebox{0.5\linewidth+1.8cm}{!}{Tempra +3, Riflessi +3, Volontà +3}
	\item[\textbf{Movimento:}] 12 m
	\item[\textbf{Sfida:}] 0 (10 PX)\smallskip
\end{description}

\textbf{Azioni}

\emph{\textbf{Morso.} Attacco con Arma da Mischia}: +2 a colpire, portata 1 m, un bersaglio.

\emph{Colpisce:} 2 (1d4) danni perforanti.

\mostro{Cinghiale}
\begin{description}[noitemsep, topsep=0pt, parsep=0pt, partopsep=0pt, leftmargin=0cm, labelwidth=2.2cm]
	\item[\textbf{Taglia/Tipo:}] Media bestia, disallineato
	\item[\textbf{Caratt.:}] \resizebox{0.5\linewidth+1.8cm}{!}{For 1 Des 0 Cos 1 Int -4 Sag -1 Car -3}
	\item[\textbf{Punti Ferita:}] 19,  \textbf{Difesa:} 12,  \textbf{Iniziativa:} +0
	\item[\textbf{Tiri Salvez.:}] \resizebox{0.5\linewidth+1.8cm}{!}{Tempra +3, Riflessi +3, Volontà +3}
	\item[\textbf{Movimento:}] 12 m
	\item[\textbf{Sfida:}] 1/4 (50 PX)\smallskip
\end{description}

\emph{\textbf{Carica.}} Se il cinghiale si muove di almeno 6 metri diretto verso il bersaglio e colpisce con un attacco di zanna durante lo stesso round, il bersaglio subisce 3 (1d6) danni taglienti aggiuntivi. Se il bersaglio è una creatura, deve riuscire un Tiro Salvezza di Tempra DC 11 o cadere prono.

\emph{\textbf{Implacabile (Ricarica dopo un 1 ora).}} Se il cinghiale subisce 7 danni o meno che lo ridurrebbero a 0 Punti Ferita, scende invece a 1 punto ferita.

\textbf{Azioni}

\emph{\textbf{Zanna.} Attacco con Arma da Mischia}: +3 a colpire, portata 1 m, un bersaglio.

\emph{Colpisce:} 4 (1d6 + 1) danni taglienti.

\mostro{Cinghiale Gigante}
\begin{description}[noitemsep, topsep=0pt, parsep=0pt, partopsep=0pt, leftmargin=0cm, labelwidth=2.2cm]
    \item[\textbf{Taglia/Tipo:}] Grande bestia, disallineato
    \item[\textbf{Caratt.:}] \resizebox{0.5\linewidth+1.8cm}{!}{For 3 Des 0 Cos 3 Int -4 Sag -2 Car -3}
    \item[\textbf{Punti Ferita:}] 52,  \textbf{Difesa:} 14,  \textbf{Iniziativa:} +0
    \item[\textbf{Tiri Salvez.:}] \resizebox{0.5\linewidth+1.8cm}{!}{Tempra +5, Riflessi +3, Volontà +3}
    \item[\textbf{Movimento:}] 12 m
    \item[\textbf{Sfida:}] 2 (450 PX)\smallskip
\end{description}

\emph{\textbf{Carica.}} Se il cinghiale si muove di almeno 6 metri diretto verso il bersaglio e colpisce con un attacco di zanna durante lo stesso round, il bersaglio subisce 7 (2d6) danni taglienti aggiuntivi. Se il bersaglio è una creatura, deve riuscire un Tiro Salvezza di Tempra DC 13 o cadere prono.

\emph{\textbf{Implacabile (Ricarica dopo un 1 ora).}} Se il cinghiale subisce 10 danni o meno che lo ridurrebbero a 0 Punti Ferita, scende invece a 1 punto ferita.

\textbf{Azioni}

\emph{\textbf{Zanna.} Attacco con Arma da Mischia}: +5 a colpire, portata 1 m, un bersaglio.

\emph{Colpisce:} 10 (2d6 + 3) danni taglienti.

\mostro{Coccodrillo}
\begin{description}[noitemsep, topsep=0pt, parsep=0pt, partopsep=0pt, leftmargin=0cm, labelwidth=2.2cm]
    \item[\textbf{Taglia/Tipo:}] Grande bestia, disallineato
    \item[\textbf{Caratt.:}] \resizebox{0.5\linewidth+1.8cm}{!}{For 2 Des 0 Cos 1 Int -4 Sag 0 Car -3}
    \item[\textbf{Punti Ferita:}] 24,  \textbf{Difesa:} 12,  \textbf{Iniziativa:} +0
    \item[\textbf{Tiri Salvez.:}] \resizebox{0.5\linewidth+1.8cm}{!}{Tempra +3, Riflessi +3, Volontà +3}
    \item[\textbf{Movimento:}] 6 m, nuoto 9 m
    \item[\textbf{Sfida:}] 1/2 (100 PX)\smallskip
\end{description}

\emph{\textbf{Trattenere il Fiato.}} Il coccodrillo può trattenere il fiato per 15 minuti.

\textbf{Azioni}

\emph{\textbf{Morso.} Attacco con Arma da Mischia}: +4 a colpire, portata 1 m, una creatura.

\emph{Colpisce:} 7 (1d10 + 2) danni perforanti, e il bersaglio è afferrato (DC 12 per fuggire). Fino al termine dell'afferrare il coccodrillo non può usare il morso contro un altro bersaglio.

\mostro{Coccodrillo Gigante}
\begin{description}[noitemsep, topsep=0pt, parsep=0pt, partopsep=0pt, leftmargin=0cm, labelwidth=2.2cm]
    \item[\textbf{Taglia/Tipo:}] Enorme bestia, disallineato
    \item[\textbf{Caratt.:}] \resizebox{0.5\linewidth+1.8cm}{!}{For 5 Des -1 Cos 3 Int -4 Sag 0 Car -2}
    \item[\textbf{Punti Ferita:}] 108,  \textbf{Difesa:} 17,  \textbf{Iniziativa:} -1
    \item[\textbf{Tiri Salvez.:}] \resizebox{0.5\linewidth+1.8cm}{!}{Tempra +8, Riflessi +4, Volontà +5}
    \item[\textbf{Movimento:}] 9 m, nuoto 15 m
    \item[\textbf{Sfida:}] 5 (1800 PX)\smallskip
\end{description}

\emph{\textbf{Trattenere il Fiato.}} Il coccodrillo può trattenere il fiato per 30 minuti.

\textbf{Azioni}

\emph{\textbf{Multiattacco.}} Il coccodrillo effettua due attacchi: uno con il morso e uno con la coda.

\emph{\textbf{Coda.} Attacco con Arma da Mischia}: +8 a colpire, portata 3 m, un bersaglio non afferrato dal coccodrillo.

\emph{Colpisce:} 14 (2d8 + 5) danni contundenti. Se il bersaglio è una creatura, deve riuscire un Tiro Salvezza di Tempra DC 16 o cadere prono.

\emph{\textbf{Morso.} Attacco con Arma da Mischia}: +8 a colpire, portata 1 m, un bersaglio.

\emph{Colpisce:} 21 (3d10 + 5) danni perforanti, e il bersaglio è afferrato (DC 16 per fuggire). Fino al termine dell'afferrare il coccodrillo non può usare il morso contro un altro bersaglio.

\mostro{Corvo}
\begin{description}[noitemsep, topsep=0pt, parsep=0pt, partopsep=0pt, leftmargin=0cm, labelwidth=2.2cm]
    \item[\textbf{Taglia/Tipo:}] Minuscola bestia, disallineato
    \item[\textbf{Caratt.:}] \resizebox{0.5\linewidth+1.8cm}{!}{For -4 Des 2 Cos -1 Int -4 Sag 1 Car -2}
    \item[\textbf{Punti Ferita:}] 15,  \textbf{Difesa:} 14,  \textbf{Iniziativa:} +2
    \item[\textbf{Tiri Salvez.:}] \resizebox{0.5\linewidth+1.8cm}{!}{Tempra +3, Riflessi +3, Volontà +3}
    \item[\textbf{Movimento:}] 3 m, volo 15 m
    \item[\textbf{Sfida:}] 0(10 PX)\smallskip
\end{description}

\emph{\textbf{Imitazione.}} Il corvo può imitare dei semplici suoni che ha udito, come il sussurro di una persona, il pianto di un bambino o il verso di un animale. Una creatura che ode il suono può identificarlo come imitazione riuscendo una prova Sopravvivenza DC 10.

\textbf{Azioni}

\emph{\textbf{Becco.} Attacco con Arma da Mischia}: +4 a colpire, portata 1 m, un bersaglio.

\emph{Colpisce:} 1 danno perforante.

\mostro{Donnola Gigante}
\begin{description}[noitemsep, topsep=0pt, parsep=0pt, partopsep=0pt, leftmargin=0cm, labelwidth=2.2cm]
    \item[\textbf{Taglia/Tipo:}] Media bestia, disallineato
    \item[\textbf{Caratt.:}] \resizebox{0.5\linewidth+1.8cm}{!}{For 0 Des 3 Cos 0 Int -3 Sag 1 Car -3}
    \item[\textbf{Punti Ferita:}] 17,  \textbf{Difesa:} 15,  \textbf{Iniziativa:} +3
    \item[\textbf{Tiri Salvez.:}] \resizebox{0.5\linewidth+1.8cm}{!}{Tempra +3, Riflessi +3, Volontà +3}
    \item[\textbf{Movimento:}] 12 m
    \item[\textbf{Sfida:}] 1/8 (25 PX)\smallskip
\end{description}

\emph{\textbf{Udito e Olfatto Affinati.}} La donnola ha +1d6 nelle prove di Consapevolezza basate su udito o olfatto.

\textbf{Azioni}

\emph{\textbf{Morso.} Attacco con Arma da Mischia}: +5 a colpire, portata 1 m, un bersaglio.

\emph{Colpisce:} 5 (1d4 + 3) danni perforanti.

\mostro{Elefante}
\begin{description}[noitemsep, topsep=0pt, parsep=0pt, partopsep=0pt, leftmargin=0cm, labelwidth=2.2cm]
    \item[\textbf{Taglia/Tipo:}] Enorme bestia, disallineato
    \item[\textbf{Caratt.:}] \resizebox{0.5\linewidth+1.8cm}{!}{For 6 Des -1 Cos 3 Int -4 Sag 0 Car -2}
    \item[\textbf{Punti Ferita:}] 89,  \textbf{Difesa:} 16,  \textbf{Iniziativa:} -1
    \item[\textbf{Tiri Salvez.:}] \resizebox{0.5\linewidth+1.8cm}{!}{Tempra +7, Riflessi +3, Volontà +4}
    \item[\textbf{Movimento:}] 12 m
    \item[\textbf{Sfida:}] 4 (1100 PX)\smallskip
\end{description}

\emph{\textbf{Carica Travolgente.}} Se l'elefante si muove di almeno 6 metri diretto verso una creatura e la colpisce con un attacco di incornata durante lo stesso round, il bersaglio deve riuscire un Tiro Salvezza su Tempra DC 16 o cadere prono. Se il bersaglio è prono, l'elefante può effettuare un attacco di pestone contro di lui come Azione Immediata.

\textbf{Azioni}

\emph{\textbf{Incornata.} Attacco con Arma da Mischia}: +6 a colpire, portata 1 m, un bersaglio.

\emph{Colpisce:} 19 (3d8 + 6) danni perforanti.

\emph{\textbf{Pestone.} Attacco con Arma da Mischia}: +6 a colpire, portata 1 m, un bersaglio prono.

\emph{Colpisce:} 22 (3d10 + 6) danni contundenti.

\mostro{Falco}
\begin{description}[noitemsep, topsep=0pt, parsep=0pt, partopsep=0pt, leftmargin=0cm, labelwidth=2.2cm]
    \item[\textbf{Taglia/Tipo:}] Minuscola bestia, disallineato
    \item[\textbf{Caratt.:}] \resizebox{0.5\linewidth+1.8cm}{!}{For -3 Des 3 Cos -1 Int -4 Sag 2 Car -2}
    \item[\textbf{Punti Ferita:}] 15,  \textbf{Difesa:} 15,  \textbf{Iniziativa:} +3
    \item[\textbf{Tiri Salvez.:}] \resizebox{0.5\linewidth+1.8cm}{!}{Tempra +3, Riflessi +3, Volontà +3}
    \item[\textbf{Movimento:}] 3 m, volo 18 m
    \item[\textbf{Sfida:}] 0(10 PX)\smallskip
\end{description}

\emph{\textbf{Vista Affinata.}} Il falco ha +1d6 alle prove di Consapevolezza basate sulla vista.

\textbf{Azioni}

\emph{\textbf{Speroni.} Attacco con Arma da Mischia}: +5 a colpire, portata 1 m, un bersaglio.

\emph{Colpisce:} 1 danno tagliente.

\mostro{Falco di Sangue}
\begin{description}[noitemsep, topsep=0pt, parsep=0pt, partopsep=0pt, leftmargin=0cm, labelwidth=2.2cm]
    \item[\textbf{Taglia/Tipo:}] Piccola bestia, disallineato
    \item[\textbf{Caratt.:}] \resizebox{0.5\linewidth+1.8cm}{!}{For -2 Des 2 Cos 0 Int -4 Sag 2 Car -3}
    \item[\textbf{Punti Ferita:}] 17,  \textbf{Difesa:} 14,  \textbf{Iniziativa:} +2
    \item[\textbf{Tiri Salvez.:}] \resizebox{0.5\linewidth+1.8cm}{!}{Tempra +3, Riflessi +3, Volontà +3}
    \item[\textbf{Movimento:}] 3 m, volo 18 m
    \item[\textbf{Sfida:}] 1/8 (25 PX)\smallskip
\end{description}

\emph{\textbf{Tattiche di Branco.}} Il falco ha +1d6 ai tiri di attacco contro una creatura se almeno uno degli alleati del falco si trova entro 1 metro dalla creatura e quell'alleato non è inabile.

\emph{\textbf{Vista Affinata.}} Il falco ha +1d6 alle prove di Consapevolezza basate sulla vista.

\textbf{Azioni}

\emph{\textbf{Becco.} Attacco con Arma da Mischia}: +3 a colpire, portata 1 m, un bersaglio.

\emph{Colpisce:} 4 (1d4 + 2) danni perforanti.

\mostro{Pirana}
\begin{description}[noitemsep, topsep=0pt, parsep=0pt, partopsep=0pt, leftmargin=0cm, labelwidth=2.2cm]
    \item[\textbf{Taglia/Tipo:}] Minuscola bestia, disallineato
    \item[\textbf{Caratt.:}] \resizebox{0.5\linewidth+1.8cm}{!}{For -4 Des 3 Cos -1 Int -5 Sag -2 Car -4}
    \item[\textbf{Punti Ferita:}] 15,  \textbf{Difesa:} 15,  \textbf{Iniziativa:} +3
    \item[\textbf{Tiri Salvez.:}] \resizebox{0.5\linewidth+1.8cm}{!}{Tempra +3, Riflessi +3, Volontà +3}
    \item[\textbf{Movimento:}] 0 m, nuoto 12 m
    \item[\textbf{Sfida:}] 0(10 PX)\smallskip
\end{description}

\emph{\textbf{Frenesia Sanguinaria.}} Il pirana ha +1d6 ai tiri di attacco in mischia contro qualsiasi creatura che non sia al massimo dei Punti Ferita.

\emph{\textbf{Respirare Acqua.}} Il pirana può respirare solo sott'acqua.

\textbf{Azioni}

\emph{\textbf{Morso.} Attacco con Arma da Mischia}: +5 a colpire, portata 1 m, un bersaglio.

\emph{Colpisce:} 1 danno perforante.

\mostro{Gatto}
\begin{description}[noitemsep, topsep=0pt, parsep=0pt, partopsep=0pt, leftmargin=0cm, labelwidth=2.2cm]
    \item[\textbf{Taglia/Tipo:}] Minuscola bestia, disallineato
    \item[\textbf{Caratt.:}] \resizebox{0.5\linewidth+1.8cm}{!}{For -4 Des 2 Cos 0 Int -4 Sag 1 Car -2}
    \item[\textbf{Punti Ferita:}] 15,  \textbf{Difesa:} 14,  \textbf{Iniziativa:} +2
    \item[\textbf{Tiri Salvez.:}] \resizebox{0.5\linewidth+1.8cm}{!}{Tempra +3, Riflessi +3, Volontà +3}
    \item[\textbf{Movimento:}] 12 m, scalata 9 m
    \item[\textbf{Sfida:}] 0(10 PX)\smallskip
\end{description}

\emph{\textbf{Olfatto Affinato.}} Il gatto ha +1d6 alle prove di Consapevolezza basate sull'olfatto.

\textbf{Azioni}

\emph{\textbf{Artigli.} Attacco con Arma da Mischia}: +2 a colpire, portata 1 m, un bersaglio.

\emph{Colpisce:} 1 danno tagliente.

\mostro{Granchio Gigante}
\begin{description}[noitemsep, topsep=0pt, parsep=0pt, partopsep=0pt, leftmargin=0cm, labelwidth=2.2cm]
    \item[\textbf{Taglia/Tipo:}] Media bestia, disallineato
    \item[\textbf{Caratt.:}] \resizebox{0.5\linewidth+1.8cm}{!}{For 1 Des 2 Cos 0 Int -5 Sag -1 Car -4}
    \item[\textbf{Punti Ferita:}] 17,  \textbf{Difesa:} 14,  \textbf{Iniziativa:} +2
    \item[\textbf{Tiri Salvez.:}] \resizebox{0.5\linewidth+1.8cm}{!}{Tempra +3, Riflessi +3, Volontà +3}
    \item[\textbf{Movimento:}] 9 m, nuoto 9 m
    \item[\textbf{Sfida:}] 1/8 (25 PX)\smallskip
\end{description}

\emph{\textbf{Anfibio.}} Il granchio può respirare aria e acqua.

\textbf{Azioni}

\emph{\textbf{Artiglio (Chela).} Attacco con Arma da Mischia}: +3 a colpire, portata 1 m, un bersaglio.

\emph{Colpisce:} 4 (1d6 + 1) danni contundenti e il bersaglio è afferrato (DC 11 per fuggire). Il granchio ha due chele, ciascuna delle quali può afferrare un solo bersaglio.

\mostro{Gufo}
\begin{description}[noitemsep, topsep=0pt, parsep=0pt, partopsep=0pt, leftmargin=0cm, labelwidth=2.2cm]
    \item[\textbf{Taglia/Tipo:}] Minuscola bestia, disallineato
    \item[\textbf{Caratt.:}] \resizebox{0.5\linewidth+1.8cm}{!}{For -4 Des 1 Cos -1 Int -4 Sag 1 Car -2}
    \item[\textbf{Punti Ferita:}] 15,  \textbf{Difesa:} 13,  \textbf{Iniziativa:} +1
    \item[\textbf{Tiri Salvez.:}] \resizebox{0.5\linewidth+1.8cm}{!}{Tempra +3, Riflessi +3, Volontà +3}
    \item[\textbf{Movimento:}] 1 m, volo 18 m
    \item[\textbf{Sfida:}] 0(10 PX)\smallskip
\end{description}

\emph{\textbf{Sorvolare.}} Il gufo non provoca attacchi di opportunità quando vola via dalla portata di un nemico.

\emph{\textbf{Udito e Vista Affinati.}} Il gufo ha +1d6 nelle prove di Consapevolezza basate su udito o vista.

\textbf{Azioni}

\emph{\textbf{Speroni.} Attacco con Arma da Mischia}: +3 a colpire, portata 1 m, un bersaglio.

\emph{Colpisce:} 1 danno tagliente.

\mostro{Gufo Gigante}
\begin{description}[noitemsep, topsep=0pt, parsep=0pt, partopsep=0pt, leftmargin=0cm, labelwidth=2.2cm]
    \item[\textbf{Taglia/Tipo:}] Grande bestia, neutrale
    \item[\textbf{Caratt.:}] \resizebox{0.5\linewidth+1.8cm}{!}{For 1 Des 2 Cos 1 Int -1 Sag 1 Car 0}
    \item[\textbf{Punti Ferita:}] 19,  \textbf{Difesa:} 14,  \textbf{Iniziativa:} +2
    \item[\textbf{Tiri Salvez.:}] \resizebox{0.5\linewidth+1.8cm}{!}{Tempra +3, Riflessi +3, Volontà +3}
    \item[\textbf{Movimento:}] 1 m, volo 18 m
    \item[\textbf{Sfida:}] 1/4 (50 PX)\smallskip
\end{description}

\emph{\textbf{Sorvolare.}} Il gufo non provoca attacchi di opportunità quando vola via dalla portata di un nemico.

\emph{\textbf{Udito e Vista Affinati.}} Il gufo ha +1d6 nelle prove di Consapevolezza basate su udito o vista.

\textbf{Azioni}

\emph{\textbf{Speroni.} Attacco con Arma da Mischia}: +4 a colpire, portata 1 m, un bersaglio.

\emph{Colpisce:} 8 (2d6 + 1) danni perforanti.

\mostro{Iena}
\begin{description}[noitemsep, topsep=0pt, parsep=0pt, partopsep=0pt, leftmargin=0cm, labelwidth=2.2cm]
    \item[\textbf{Taglia/Tipo:}] Media bestia, disallineato
    \item[\textbf{Caratt.:}] \resizebox{0.5\linewidth+1.8cm}{!}{For 0 Des 1 Cos 1 Int -4 Sag 1 Car -3}
    \item[\textbf{Punti Ferita:}] 15,  \textbf{Difesa:} 13,  \textbf{Iniziativa:} +1
    \item[\textbf{Tiri Salvez.:}] \resizebox{0.5\linewidth+1.8cm}{!}{Tempra +3, Riflessi +3, Volontà +3}
    \item[\textbf{Movimento:}] 15 m
    \item[\textbf{Sfida:}] 0(10 PX)\smallskip
\end{description}

\emph{\textbf{Tattiche di Branco.}} La iena ha +1d6 ai tiri di attacco contro una creatura se almeno uno degli alleati della iena si trova entro 1 metro dalla creatura e quell'alleato non è inabile.

\textbf{Azioni}

\emph{\textbf{Morso.} Attacco con Arma da Mischia}: +3 a colpire, portata 1 m, un bersaglio.

\emph{Colpisce:} 3 (1d6) danni perforanti.

\mostro{Iena Gigante}
\begin{description}[noitemsep, topsep=0pt, parsep=0pt, partopsep=0pt, leftmargin=0cm, labelwidth=2.2cm]
    \item[\textbf{Taglia/Tipo:}] Grande bestia, disallineato
    \item[\textbf{Caratt.:}] \resizebox{0.5\linewidth+1.8cm}{!}{For 3 Des 2 Cos 2 Int -4 Sag 1 Car -2}
    \item[\textbf{Punti Ferita:}] 33,  \textbf{Difesa:} 15,  \textbf{Iniziativa:} +2
    \item[\textbf{Tiri Salvez.:}] \resizebox{0.5\linewidth+1.8cm}{!}{Tempra +3, Riflessi +3, Volontà +3}
    \item[\textbf{Movimento:}] 15 m
    \item[\textbf{Sfida:}] 1 (200 PX)\smallskip
\end{description}

\emph{\textbf{Rabbia.}} Quando la iena riduce una creatura a 0 Punti Ferita con un attacco di mischia durante il proprio round, la iena può svolgere una Azione Immediata muoversi fino a metà del suo movimento ed effettuare un attacco di morso.

\textbf{Azioni}

\emph{\textbf{Morso.} Attacco con Arma da Mischia}: +5 a colpire, portata 1 m, un bersaglio.

\emph{Colpisce:} 10 (2d6 + 3) danni perforanti.

\mostro{Leone}
\begin{description}[noitemsep, topsep=0pt, parsep=0pt, partopsep=0pt, leftmargin=0cm, labelwidth=2.2cm]
    \item[\textbf{Taglia/Tipo:}] Grande bestia, disallineato
    \item[\textbf{Caratt.:}] \resizebox{0.5\linewidth+1.8cm}{!}{For 3 Des 2 Cos 1 Int -4 Sag 1 Car -1}
    \item[\textbf{Punti Ferita:}] 33,  \textbf{Difesa:} 15,  \textbf{Iniziativa:} +2
    \item[\textbf{Tiri Salvez.:}] \resizebox{0.5\linewidth+1.8cm}{!}{Tempra +3, Riflessi +3, Volontà +3}
    \item[\textbf{Movimento:}] 15 m
    \item[\textbf{Sfida:}] 1 (200 PX)\smallskip
\end{description}

\emph{\textbf{Balzo.}} Se il leone si muove di almeno 6 metri diretto verso una creatura e la colpisce con un attacco di artiglio durante lo stesso round, il bersaglio deve riuscire un Tiro Salvezza di Tempra DC 13 o cadere prono. Se il bersaglio è prono, il leone può effettuare un attacco di morso come Azione Immediata.

\emph{\textbf{Olfatto Affinato.}} Il leone ha +1d6 alle prove di Consapevolezza basate sull'olfatto.

\emph{\textbf{Salto con Rincorsa.}} Con 3 metri di rincorsa, il leone può saltare in lungo fino a 7 metri.

\emph{\textbf{Tattiche di Branco.}} Il leone ha +1d6 ai tiri di attacco contro una creatura se almeno uno degli alleati del leone si trova entro 1 metro dalla creatura e quell'alleato non è inabile.

\textbf{Azioni}

\emph{\textbf{Artiglio.} Attacco con Arma da Mischia}: +5 a colpire, portata 1 m, un bersaglio.

\emph{Colpisce:} 6 (1d6 + 3) danni taglienti, 1 danno da Sanguinamento.

\emph{\textbf{Morso.} Attacco con Arma da Mischia}: +5 a colpire, portata 1 m, un bersaglio.

\emph{Colpisce:} 7 (1d8 + 3) danni perforanti.

\mostro{Lucertola Gigante}
\begin{description}[noitemsep, topsep=0pt, parsep=0pt, partopsep=0pt, leftmargin=0cm, labelwidth=2.2cm]
    \item[\textbf{Taglia/Tipo:}] Grande bestia, disallineato
    \item[\textbf{Caratt.:}] \resizebox{0.5\linewidth+1.8cm}{!}{For 2 Des 1 Cos 1 Int -4 Sag 0 Car -3}
    \item[\textbf{Punti Ferita:}] 19,  \textbf{Difesa:} 13,  \textbf{Iniziativa:} +1
    \item[\textbf{Tiri Salvez.:}] \resizebox{0.5\linewidth+1.8cm}{!}{Tempra +3, Riflessi +3, Volontà +3}
    \item[\textbf{Movimento:}] 9 m, scalata 9 m
    \item[\textbf{Sfida:}] 1/4 (50 PX)\smallskip
\end{description}

\textbf{Azioni}

\emph{\textbf{Morso.} Attacco con Arma da Mischia}: +4 a colpire, portata 1 m, un bersaglio.

\emph{Colpisce:} 6 (1d8 + 2) danni perforanti.

\textbf{VARIANTE}

Alcune lucertole giganti possiedono uno o entrambi i seguenti tratti.

\emph{\textbf{Scalare come Ragno.}} La lucertola può scalare superfici difficili, compreso lo stare a testa in giù sul soffitto, senza bisogno di effettuare una prova di competenza.

\emph{\textbf{Trattenere il Fiato.}} La lucertola può trattenere il fiato per 15 minuti. (Una lucertola con questo tratto possiede anche velocità di nuoto 9 metri).

\mostro{Lupo}
\begin{description}[noitemsep, topsep=0pt, parsep=0pt, partopsep=0pt, leftmargin=0cm, labelwidth=2.2cm]
    \item[\textbf{Taglia/Tipo:}] Media bestia, disallineato
    \item[\textbf{Caratt.:}] \resizebox{0.5\linewidth+1.8cm}{!}{For 1 Des 2 Cos 1 Int -4 Sag 1 Car -2}
    \item[\textbf{Punti Ferita:}] 19,  \textbf{Difesa:} 14,  \textbf{Iniziativa:} +2
    \item[\textbf{Tiri Salvez.:}] \resizebox{0.5\linewidth+1.8cm}{!}{Tempra +3, Riflessi +3, Volontà +3}
    \item[\textbf{Movimento:}] 12 m
    \item[\textbf{Sfida:}] 1/4 (50 PX)\smallskip
\end{description}

\emph{\textbf{Udito e Olfatto Affinato.}} Il lupo ha +1d6 nelle prove di Consapevolezza basate su udito o olfatto.

\emph{\textbf{Tattiche di Branco.}} Il lupo ha +1d6 ai tiri di attacco contro una creatura se almeno uno degli alleati del lupo si trova entro 1 metro dalla creatura e quell'alleato non è inabile.

\textbf{Azioni}

\emph{\textbf{Morso.} Attacco con Arma da Mischia}: +4 a colpire, portata 1 m, un bersaglio.

\emph{Colpisce:} 7 (2d4 + 2) danni perforanti. Se il bersaglio è una creatura, deve riuscire un Tiro Salvezza di Tempra DC 11 o cadere prona.

\mostro{Dinolupo (Metalupo)}
\begin{description}[noitemsep, topsep=0pt, parsep=0pt, partopsep=0pt, leftmargin=0cm, labelwidth=2.2cm]
    \item[\textbf{Taglia/Tipo:}] Grande bestia, disallineato
    \item[\textbf{Caratt.:}] \resizebox{0.5\linewidth+1.8cm}{!}{For 3 Des 2 Cos 2 Int -2 Sag 1 Car -2}
    \item[\textbf{Punti Ferita:}] 33,  \textbf{Difesa:} 15,  \textbf{Iniziativa:} +2
    \item[\textbf{Tiri Salvez.:}] \resizebox{0.5\linewidth+1.8cm}{!}{Tempra +3, Riflessi +3, Volontà +3}
    \item[\textbf{Movimento:}] 15 m
    \item[\textbf{Sfida:}] 1 (200 PX)\smallskip
\end{description}

\emph{\textbf{Udito e Olfatto Affinato.}} Il lupo ha +1d6 nelle prove di Consapevolezza basate su udito o olfatto.

\emph{\textbf{Tattiche di Branco.}} Il lupo ha +1d6 ai tiri di attacco contro una creatura se almeno uno degli alleati del lupo si trova entro 1 metro dalla creatura e quell'alleato non è inabile.

\textbf{Azioni}

\emph{\textbf{Morso.} Attacco con Arma da Mischia}: +5 a colpire, portata 1 m, un bersaglio.

\emph{Colpisce:} 10 (2d6 + 3) danni perforanti. Se il bersaglio è una creatura, deve riuscire un Tiro Salvezza di Tempra DC 13 o cadere prona.

\mostro{Lupo Invernale}
\begin{description}[noitemsep, topsep=0pt, parsep=0pt, partopsep=0pt, leftmargin=0cm, labelwidth=2.2cm]
    \item[\textbf{Taglia/Tipo:}] Grande mostruosità, malvagio
    \item[\textbf{Caratt.:}] \resizebox{0.5\linewidth+1.8cm}{!}{For 4 Des 1 Cos 2 Int -2 Sag 1 Car -1}
    \item[\textbf{Punti Ferita:}] 70,  \textbf{Difesa:} 17,  \textbf{Iniziativa:} +1
    \item[\textbf{Tiri Salvez.:}] \resizebox{0.5\linewidth+1.8cm}{!}{Tempra +5, Riflessi +4, Volontà +4}
    \item[\textbf{Movimento:}] 15 m
    \item[\textbf{Sfida:}] 3 (700 PX)\smallskip
\end{description}

\emph{\textbf{Camuffamento di Neve.}} Il lupo ha +1d6 alle prove di Furtività (Nascondersi) effettuate per nascondersi su terreno innevato.

\emph{\textbf{Udito e Olfatto Affinato.}} Il lupo ha +1d6 nelle prove di Consapevolezza basate su udito o olfatto.

\emph{\textbf{Tattiche di Branco.}} Il lupo ha +1d6 ai tiri di attacco contro una creatura se almeno uno degli alleati del lupo si trova entro 1 metro dalla creatura e quell'alleato non è inabile.

\textbf{Azioni}

\emph{\textbf{Morso.} Attacco con Arma da Mischia}: +6 a colpire, portata 1 m, un bersaglio.

\emph{Colpisce:} 11 (2d6 + 4) danni perforanti. Se il bersaglio è una creatura, deve riuscire un Tiro Salvezza di Tempra DC 14 o cadere prona.

\emph{\textbf{Soffio Gelido (Ricarica 5-6).}} Il lupo esala un'esplosione di vento gelido in un cono di 5 metri. Ogni creatura in quell'area deve effettuare un Tiro Salvezza di Riflessi DC 15, e subire 18 (4d8) danni da freddo se fallisce il Tiro Salvezza, o la metà di questi danni se lo riesce.

\mostro{Mammut}
\begin{description}[noitemsep, topsep=0pt, parsep=0pt, partopsep=0pt, leftmargin=0cm, labelwidth=2.2cm]
    \item[\textbf{Taglia/Tipo:}] Enorme bestia, disallineato
    \item[\textbf{Caratt.:}] \resizebox{0.5\linewidth+1.8cm}{!}{For 7 Des -1 Cos 5 Int -4 Sag 0 Car -2}
    \item[\textbf{Punti Ferita:}] 129,  \textbf{Difesa:} 19,  \textbf{Iniziativa:} -1
    \item[\textbf{Tiri Salvez.:}] \resizebox{0.5\linewidth+1.8cm}{!}{Tempra +11, Riflessi +5, Volontà +6}
    \item[\textbf{Movimento:}] 12 m
    \item[\textbf{Sfida:}] 6 (2300 PX)\smallskip
\end{description}

\emph{\textbf{Carica Travolgente.}} Se il mammut si muove di almeno 6 metri diretto verso una creatura e la colpisce con un attacco di incornata durante lo stesso round, il bersaglio deve riuscire un Tiro Salvezza su Tempra DC 18 o cadere prono. Se il bersaglio è prono, il mammut può effettuare un attacco di pestone contro di lui come Azione Immediata.

\textbf{Azioni}

\emph{\textbf{Incornata.} Attacco con Arma da Mischia}: +8 a colpire, portata 3 m, un bersaglio.

\emph{Colpisce:} 25 (4d8 + 7) danni perforanti.

\emph{\textbf{Pestone.} Attacco con Arma da Mischia}: +8 a colpire, portata 1 m, una creatura prona.

\emph{Colpisce:} 29 (4d10 + 7) danni contundenti.

\mostro{Mastino}
\begin{description}[noitemsep, topsep=0pt, parsep=0pt, partopsep=0pt, leftmargin=0cm, labelwidth=2.2cm]
    \item[\textbf{Taglia/Tipo:}] Media bestia, disallineato
    \item[\textbf{Caratt.:}] \resizebox{0.5\linewidth+1.8cm}{!}{For 1 Des 2 Cos 1 Int -4 Sag 1 Car -2}
    \item[\textbf{Punti Ferita:}] 17,  \textbf{Difesa:} 14,  \textbf{Iniziativa:} +2
    \item[\textbf{Tiri Salvez.:}] \resizebox{0.5\linewidth+1.8cm}{!}{Tempra +3, Riflessi +3, Volontà +3}
    \item[\textbf{Movimento:}] 12 m
    \item[\textbf{Sfida:}] 1/8 (25 PX)\smallskip
\end{description}

\emph{\textbf{Udito e Olfatto Affinato.}} Il mastino ha +1d6 nelle prove di Consapevolezza basate su udito o olfatto.

\textbf{Azioni}

\emph{\textbf{Morso.} Attacco con Arma da Mischia}: +3 a colpire, portata 1 m, un bersaglio.

\emph{Colpisce:} 4 (1d6 + 1) danni perforanti. Se il bersaglio è una creatura, deve riuscire un Tiro Salvezza di Tempra DC 11 o cadere prono.

\mostro{Orso Bruno}
\begin{description}[noitemsep, topsep=0pt, parsep=0pt, partopsep=0pt, leftmargin=0cm, labelwidth=2.2cm]
    \item[\textbf{Taglia/Tipo:}] Grande bestia, disallineato
    \item[\textbf{Caratt.:}] \resizebox{0.5\linewidth+1.8cm}{!}{For 4 Des 0 Cos 3 Int -4 Sag 1 Car -2}
    \item[\textbf{Punti Ferita:}] 33,  \textbf{Difesa:} 13,  \textbf{Iniziativa:} +0
    \item[\textbf{Tiri Salvez.:}] \resizebox{0.5\linewidth+1.8cm}{!}{Tempra +4, Riflessi +3, Volontà +3}
    \item[\textbf{Movimento:}] 12 m, scalata 9 m
    \item[\textbf{Sfida:}] 1 (200 PX)\smallskip
\end{description}

\emph{\textbf{Olfatto Affinato.}} L'orso ha +1d6 alle prove di Consapevolezza basate sull'olfatto.

\textbf{Azioni}

\emph{\textbf{Multiattacco.}} L'orso effettua due attacchi: uno con il morso e uno con gli artigli.

\emph{\textbf{Artigli.} Attacco con Arma da Mischia}: +5 a colpire, portata 1 m, un bersaglio.

\emph{Colpisce:} 11 (2d6 + 4) danni taglienti.

\emph{\textbf{Morso.} Attacco con Arma da Mischia}: +5 a colpire, portata 1 m, un bersaglio.

\emph{Colpisce:} 8 (1d8 + 4) danni perforanti.

\mostro{Orso Nero}
\begin{description}[noitemsep, topsep=0pt, parsep=0pt, partopsep=0pt, leftmargin=0cm, labelwidth=2.2cm]
    \item[\textbf{Taglia/Tipo:}] Media bestia, disallineato
    \item[\textbf{Caratt.:}] \resizebox{0.5\linewidth+1.8cm}{!}{For 3 Des 0 Cos 2 Int -4 Sag 1 Car -2}
    \item[\textbf{Punti Ferita:}] 24,  \textbf{Difesa:} 12,  \textbf{Iniziativa:} +0
    \item[\textbf{Tiri Salvez.:}] \resizebox{0.5\linewidth+1.8cm}{!}{Tempra +3, Riflessi +3, Volontà +3}
    \item[\textbf{Movimento:}] 12 m, scalata 9 m
    \item[\textbf{Sfida:}] 1/2 (100 PX)\smallskip
\end{description}

\emph{\textbf{Olfatto Affinato.}} L'orso ha +1d6 alle prove di Consapevolezza basate sull'olfatto.

\textbf{Azioni}

\emph{\textbf{Multiattacco.}} L'orso nero effettua due attacchi: uno con il morso e uno con gli artigli.

\emph{\textbf{Artigli.} Attacco con Arma da Mischia}: +4 a colpire, portata 1 m, un bersaglio.

\emph{Colpisce:} 7 (2d4 + 3) danni taglienti, 1 danno da Sanguinamento.

\emph{\textbf{Morso.} Attacco con Arma da Mischia}: +5 a colpire, portata 1 m, un bersaglio.

\emph{Colpisce:} 6 (1d6 + 3) danni perforanti.

\mostro{Orso Polare}
\begin{description}[noitemsep, topsep=0pt, parsep=0pt, partopsep=0pt, leftmargin=0cm, labelwidth=2.2cm]
    \item[\textbf{Taglia/Tipo:}] Grande bestia, disallineato
    \item[\textbf{Caratt.:}] \resizebox{0.5\linewidth+1.8cm}{!}{For 5 Des 0 Cos 3 Int -4 Sag 1 Car -2}
    \item[\textbf{Punti Ferita:}] 52,  \textbf{Difesa:} 14,  \textbf{Iniziativa:} +0
    \item[\textbf{Tiri Salvez.:}] \resizebox{0.5\linewidth+1.8cm}{!}{Tempra +5, Riflessi +3, Volontà +3}
    \item[\textbf{Movimento:}] 12 m, nuoto 9 m
    \item[\textbf{Sfida:}] 2 (450 PX)\smallskip
\end{description}

\emph{\textbf{Olfatto Affinato.}} L'orso ha +1d6 alle prove di Consapevolezza basate sull'olfatto.

\textbf{Azioni}

\emph{\textbf{Multiattacco.}} L'orso effettua due attacchi: uno con il morso e uno con gli artigli.

\emph{\textbf{Artigli.} Attacco con Arma da Mischia}: +5 a colpire, portata 1 m, un bersaglio.

\emph{Colpisce:} 12 (2d6 + 5) danni taglienti.

\emph{\textbf{Morso.} Attacco con Arma da Mischia}: +5 a colpire, portata 1 m, un bersaglio.

\emph{Colpisce:} 9 (1d8 + 5) danni perforanti.

\mostro{Orso Corazzato}
\begin{description}[noitemsep, topsep=0pt, parsep=0pt, partopsep=0pt, leftmargin=0cm, labelwidth=2.2cm]
	\item[\textbf{Taglia/Tipo:}] Enorme bestia, corrotta da Cattalm
	\item[\textbf{Caratt.:}] \resizebox{0.5\linewidth+1.8cm}{!}{For 7 Des 2 Cos 4 Int 1 Sag 1 Car 1}
	\item[\textbf{Punti Ferita:}] 90,  \textbf{Difesa:} 19,  \textbf{Iniziativa:} +2
	\item[\textbf{Tiri Salvez.:}] \resizebox{0.5\linewidth+1.8cm}{!}{Tempra +8, Riflessi +6, Volontà +5}
	\item[\textbf{Movimento:}] 12 m, nuoto 9 m
	\item[\textbf{Sfida:}] 4 (1100 PX)\smallskip
\end{description}

\emph{\textbf{Olfatto Affinato.}} L'orso ha +1d6 alle prove di Consapevolezza basate sull'olfatto.

\textbf{Azioni}

\emph{\textbf{Multiattacco.}} L'orso effettua due attacchi: uno con il morso e uno con gli artigli.

\emph{\textbf{Artigli.} Attacco con Arma da Mischia}: +9 a colpire, portata 2 m, un bersaglio.

\emph{Colpisce:} 17 (3d8 + 7) danni taglienti.

\emph{\textbf{Morso.} Attacco con Arma da Mischia}: +8 a colpire, portata 2 m, un bersaglio.

\emph{Colpisce:} 20 (3d8 + 10) danni perforanti.

\mostro{Pantera}
\begin{description}[noitemsep, topsep=0pt, parsep=0pt, partopsep=0pt, leftmargin=0cm, labelwidth=2.2cm]
    \item[\textbf{Taglia/Tipo:}] Media bestia, disallineato
    \item[\textbf{Caratt.:}] \resizebox{0.5\linewidth+1.8cm}{!}{For 2 Des 2 Cos 0 Int -4 Sag 2 Car -2}
    \item[\textbf{Punti Ferita:}] 19,  \textbf{Difesa:} 14,  \textbf{Iniziativa:} +2
    \item[\textbf{Tiri Salvez.:}] \resizebox{0.5\linewidth+1.8cm}{!}{Tempra +3, Riflessi +3, Volontà +3}
    \item[\textbf{Movimento:}] 15 m, scalata 12 m
    \item[\textbf{Sfida:}] 1/4 (50 PX)\smallskip
\end{description}

\emph{\textbf{Balzo.}} Se la pantera si muove di almeno 6 metri diretta verso una creatura e la colpisce con un attacco di artiglio durante lo stesso round, il bersaglio deve riuscire un Tiro Salvezza di Tempra DC 12 o cadere prono. Se il bersaglio è prono, la pantera può effettuare un attacco di morso contro di esso come Azione Immediata.

\emph{\textbf{Olfatto Affinato.}} La pantera ha +1d6 alle prove di Consapevolezza basate sull'olfatto.

\textbf{Azioni}

\emph{\textbf{Artiglio.} Attacco con Arma da Mischia}: +4 a colpire, portata 1 m, un bersaglio.

\emph{Colpisce:} 4 (1d4 + 2) danni taglienti, 1 danno da Sanguinamento.

\emph{\textbf{Morso.} Attacco con Arma da Mischia}: +4 a colpire, portata 1 m, un bersaglio.

\emph{Colpisce:} 5 (1d6 + 2) danni perforanti.

\mostro{Ragno}
\begin{description}[noitemsep, topsep=0pt, parsep=0pt, partopsep=0pt, leftmargin=0cm, labelwidth=2.2cm]
    \item[\textbf{Taglia/Tipo:}] Minuscola bestia, disallineato
    \item[\textbf{Caratt.:}] \resizebox{0.5\linewidth+1.8cm}{!}{For 2 (-5) Des 2 Cos -1 Int -5 Sag 0 Car -4}
    \item[\textbf{Punti Ferita:}] 15,  \textbf{Difesa:} 14,  \textbf{Iniziativa:} +2
    \item[\textbf{Tiri Salvez.:}] \resizebox{0.5\linewidth+1.8cm}{!}{Tempra +3, Riflessi +3, Volontà +3}
    \item[\textbf{Movimento:}] 6 m, scalata 6 m
    \item[\textbf{Sfida:}] 0(10 PX)\smallskip
\end{description}

\emph{\textbf{Camminare sulla Tela.}} Il ragno ignora le restrizioni al movimento provocate dalle ragnatele.

\emph{\textbf{Scalare come Ragno.}} Il ragno può scalare superfici difficili, compreso lo stare a testa in giù sul soffitto, senza bisogno di effettuare una prova di competenza.

\emph{\textbf{Senso della Tela.}} Mentre è in contatto con una ragnatela, il ragno sa l'esatta posizione di qualsiasi altra creatura in contatto con la stessa ragnatela.

\textbf{Azioni}

\emph{\textbf{Morso.} Attacco con Arma da Mischia}: +4 a colpire, portata 1 m, una creatura.

\emph{Colpisce:} 1 danno perforante e il bersaglio deve riuscire un Tiro Salvezza su Tempra 9 o subire 2 (1d4) danni da veleno.

\mostro{Ragno Fase}
\begin{description}[noitemsep, topsep=0pt, parsep=0pt, partopsep=0pt, leftmargin=0cm, labelwidth=2.2cm]
    \item[\textbf{Taglia/Tipo:}] Grande mostruosità, disallineato
    \item[\textbf{Caratt.:}] \resizebox{0.5\linewidth+1.8cm}{!}{For 2 Des 2 Cos 1 Int -2 Sag 0 Car -2}
    \item[\textbf{Punti Ferita:}] 69,  \textbf{Difesa:} 18,  \textbf{Iniziativa:} +2
    \item[\textbf{Tiri Salvez.:}] \resizebox{0.5\linewidth+1.8cm}{!}{Tempra +4, Riflessi +5, Volontà +3}
    \item[\textbf{Movimento:}] 9 m, scalata 9 m
    \item[\textbf{Sfida:}] 3 (700 PX)\smallskip
\end{description}

\emph{\textbf{Camminare sulla Tela.}} Il ragno ignora le restrizioni al movimento provocate dalle ragnatele.

\emph{\textbf{Passo Etereo.}} Come Reazione, il ragno può magicamente spostarsi dal Piano Materiale al Piano Etereo, o viceversa.

\emph{\textbf{Scalare come Ragno.}} Il ragno può scalare superfici difficili, compreso lo stare a testa in giù sul soffitto, senza bisogno di effettuare una prova di competenza.

\textbf{Azioni}

\emph{\textbf{Morso.} Attacco con Arma da Mischia}: +5 a colpire, portata 1 m, una creatura.

\emph{Colpisce:} 7 (1d10 + 2) danni perforanti e il bersaglio deve effettuare un Tiro Salvezza di Tempra DC 14, e subire 18 (4d8) danni da veleno se fallisce il Tiro Salvezza, o la metà di questo danno se lo riesce. Se il danno da veleno riduce il bersaglio a 0 Punti Ferita, il bersaglio è stabile ma avvelenato per 1 ora, anche dopo aver recuperato i Punti Ferita, e mentre è avvelenato in questo modo resta paralizzato.

\mostro{Ragno Gigante}
\begin{description}[noitemsep, topsep=0pt, parsep=0pt, partopsep=0pt, leftmargin=0cm, labelwidth=2.2cm]
    \item[\textbf{Taglia/Tipo:}] Grande bestia, disallineato
    \item[\textbf{Caratt.:}] \resizebox{0.5\linewidth+1.8cm}{!}{For 2 Des 3 Cos 1 Int -4 Sag 0 Car -3}
    \item[\textbf{Punti Ferita:}] 33,  \textbf{Difesa:} 16,  \textbf{Iniziativa:} +3
    \item[\textbf{Tiri Salvez.:}] \resizebox{0.5\linewidth+1.8cm}{!}{Tempra +3, Riflessi +4, Volontà +3}
    \item[\textbf{Movimento:}] 9 m, scalata 9 m
    \item[\textbf{Sfida:}] 1 (200 PX)\smallskip
\end{description}

\emph{\textbf{Camminare sulla Tela.}} Il ragno ignora le restrizioni al movimento provocate dalle ragnatele.

\emph{\textbf{Scalare come Ragno.}} Il ragno può scalare superfici difficili, compreso lo stare a testa in giù sul soffitto, senza bisogno di effettuare una prova di competenza.

\emph{\textbf{Senso della Tela.}} Mentre è in contatto con una ragnatela, il ragno sa l'esatta posizione di qualsiasi altra creatura in contatto con la stessa ragnatela.

\textbf{Azioni}

\emph{\textbf{Morso.} Attacco con Arma da Mischia}: +5 a colpire, portata 1 m, una creatura.

\emph{Colpisce:} 7 (1d8 + 3) danni perforanti e il bersaglio deve effettuare un Tiro Salvezza di Tempra DC 11, e subire 9

(2d8) danni da veleno se fallisce il Tiro Salvezza, o la metà di questi danni se lo riesce. Se il danno da veleno riduce il bersaglio a 0 Punti Ferita, il bersaglio è stabile ma avvelenato per 1 ora, anche dopo aver recuperato i Punti Ferita, e mentre è avvelenato in questo modo resta paralizzato.

\emph{\textbf{Ragnatela (Ricarica 5-6).} Attacco con Arma a Gittata}: +5 a colpire, gittata 9m, una creatura.

\emph{Colpisce:} Il bersaglio è intralciato dalla ragnatela. Con un'Azione, il bersaglio intralciato può effettuare un Tiro Salvezza Tempra con Forza DC 12 e, in caso di successo, spezzare la tela. La ragnatela può essere anche attaccata e distrutta (CA 10; Punti Ferita 5; vulnerabilità al danno da fuoco; immunità ai danni contundenti e da veleno).

\mostro{Ragno Lupo Gigante}
\begin{description}[noitemsep, topsep=0pt, parsep=0pt, partopsep=0pt, leftmargin=0cm, labelwidth=2.2cm]
    \item[\textbf{Taglia/Tipo:}] Media bestia, disallineato
    \item[\textbf{Caratt.:}] \resizebox{0.5\linewidth+1.8cm}{!}{For 1 Des 3 Cos 1 Int -4 Sag 1 Car -3}
    \item[\textbf{Punti Ferita:}] 19,  \textbf{Difesa:} 15,  \textbf{Iniziativa:} +3
    \item[\textbf{Tiri Salvez.:}] \resizebox{0.5\linewidth+1.8cm}{!}{Tempra +3, Riflessi +3, Volontà +3}
    \item[\textbf{Movimento:}] 12 m, scalata 12 m
    \item[\textbf{Sfida:}] 1/4 (50 PX)\smallskip
\end{description}

\emph{\textbf{Camminare sulla Tela.}} Il ragno ignora le restrizioni al movimento provocate dalle ragnatele.

\emph{\textbf{Scalare come Ragno.}} Il ragno può scalare superfici difficili, compreso lo stare a testa in giù sul soffitto, senza bisogno di effettuare una prova di competenza.

\emph{\textbf{Senso della Tela.}} Mentre è in contatto con una ragnatela, il ragno sa l'esatta posizione di qualsiasi altra creatura in contatto con la stessa ragnatela.

\textbf{Azioni}

\emph{\textbf{Morso.} Attacco con Arma da Mischia}: +4 a colpire, portata 1 m, una creatura.

\emph{Colpisce:} 4 (1d6 + 1) danni perforanti e il bersaglio deve effettuare un Tiro Salvezza di Tempra DC 11, e subire 7 (2d6) danni da veleno se fallisce il Tiro Salvezza, o la metà di questi danni se lo riesce. Se il danno da veleno riduce il bersaglio a 0 Punti Ferita, il bersaglio è stabile ma avvelenato per 1 ora, anche dopo aver recuperato i Punti Ferita, e mentre è avvelenato in questo modo resta paralizzato.

\mostro{Rana}
\begin{description}[noitemsep, topsep=0pt, parsep=0pt, partopsep=0pt, leftmargin=0cm, labelwidth=2.2cm]
    \item[\textbf{Taglia/Tipo:}] Minuscola bestia, disallineato
    \item[\textbf{Caratt.:}] \resizebox{0.5\linewidth+1.8cm}{!}{For -5 Des 1 Cos -1 Int -5 Sag -1 Car -4}
    \item[\textbf{Punti Ferita:}] 15,  \textbf{Difesa:} 13,  \textbf{Iniziativa:} +1
    \item[\textbf{Tiri Salvez.:}] \resizebox{0.5\linewidth+1.8cm}{!}{Tempra +3, Riflessi +3, Volontà +3}
    \item[\textbf{Movimento:}] 6 m, nuoto 6 m
    \item[\textbf{Sfida:}] 0(0 PX)\smallskip
\end{description}

\emph{\textbf{Anfibio.}} La rana può respirare aria e acqua.

\emph{\textbf{Salto da Fermo.}} Una rana può saltare in lungo fino a 3 metri e in alto fino a 1 metro, con o senza la rincorsa.

Una \textbf{rana} è sprovvista di attacchi. Si nutre di piccoli insetti e di solito vive in prossimità di acquitrini, dentro gli alberi o sottoterra.

\mostro{Rana Gigante}
\begin{description}[noitemsep, topsep=0pt, parsep=0pt, partopsep=0pt, leftmargin=0cm, labelwidth=2.2cm]
    \item[\textbf{Taglia/Tipo:}] Media bestia, disallineato
    \item[\textbf{Caratt.:}] \resizebox{0.5\linewidth+1.8cm}{!}{For 1 Des 1 Cos 0 Int -4 Sag 0 Car -4}
    \item[\textbf{Punti Ferita:}] 19,  \textbf{Difesa:} 13,  \textbf{Iniziativa:} +1
    \item[\textbf{Tiri Salvez.:}] \resizebox{0.5\linewidth+1.8cm}{!}{Tempra +3, Riflessi +3, Volontà +3}
    \item[\textbf{Movimento:}] 9 m, nuoto 9 m
    \item[\textbf{Sfida:}] 1/4 (50 PX)\smallskip
\end{description}

\emph{\textbf{Anfibio.}} La rana può respirare aria e acqua.

\emph{\textbf{Salto da Fermo.}} Una rana può saltare in lungo fino a 6 metri e in alto fino a 3 metri, con o senza la rincorsa.

\textbf{Azioni}

\emph{\textbf{Morso.} Attacco con Arma da Mischia}: +3 a colpire, portata 1 m, un bersaglio.

\emph{Colpisce:} 4 (1d6 + 1) danni perforanti e il bersaglio è afferrato (DC 11 per fuggire). Fino al termine dell'afferrare la rana non può usare il morso contro un altro bersaglio.

\emph{\textbf{Inghiottire.}} La rana effettua una attacco di morso contro un bersaglio di taglia Piccola o inferiore che sta afferrando. Se l'attacco colpisce, il bersaglio è inghiottito, e l'afferrare ha termine. Il bersaglio inghiottito è accecato e intralciato, ha copertura completa contro gli attacchi e altri effetti all'esterno della rana, e subisce 5 (2d4) danni da acido all'inizio di ciascun round della rana. La rana può inghiottire solo un bersaglio alla volta. Se la rana muore, una creatura inghiottita non è più intralciata da essa e può uscire dal cadavere utilizzando 1 metro di movimento, uscendo prona.

\mostro{Ratto}
\begin{description}[noitemsep, topsep=0pt, parsep=0pt, partopsep=0pt, leftmargin=0cm, labelwidth=2.2cm]
    \item[\textbf{Taglia/Tipo:}] Minuscola bestia, disallineato
    \item[\textbf{Caratt.:}] \resizebox{0.5\linewidth+1.8cm}{!}{For -4 Des 0 Cos -1 Int -4 Sag 0 Car -3}
    \item[\textbf{Punti Ferita:}] 15,  \textbf{Difesa:} 12,  \textbf{Iniziativa:} +0
    \item[\textbf{Tiri Salvez.:}] \resizebox{0.5\linewidth+1.8cm}{!}{Tempra +3, Riflessi +3, Volontà +3}
    \item[\textbf{Movimento:}] 6 m
    \item[\textbf{Sfida:}] 0(10 PX)\smallskip
\end{description}

\emph{\textbf{Olfatto Affinato.}} Il ratto ha +1d6 alle prove di Consapevolezza basate sull'olfatto.

\textbf{Azioni}

\emph{\textbf{Morso.} Attacco con Arma da Mischia}: +2 a colpire, portata 1 m, un bersaglio.

\emph{Colpisce:} 1 danno perforante.

\mostro{Ratto Gigante}
\begin{description}[noitemsep, topsep=0pt, parsep=0pt, partopsep=0pt, leftmargin=0cm, labelwidth=2.2cm]
    \item[\textbf{Taglia/Tipo:}] Piccola bestia, disallineato
    \item[\textbf{Caratt.:}] \resizebox{0.5\linewidth+1.8cm}{!}{For -2 Des 2 Cos 0 Int -4 Sag 0 Car -3}
    \item[\textbf{Punti Ferita:}] 17,  \textbf{Difesa:} 14,  \textbf{Iniziativa:} +2
    \item[\textbf{Tiri Salvez.:}] \resizebox{0.5\linewidth+1.8cm}{!}{Tempra +3, Riflessi +3, Volontà +3}
    \item[\textbf{Movimento:}] 9 m
    \item[\textbf{Sfida:}] 1/8 (25 PX)\smallskip
\end{description}

\emph{\textbf{Olfatto Affinato.}} Il ratto ha +1d6 alle prove di Consapevolezza basate sull'olfatto.

\emph{\textbf{Tattiche di Branco.}} Il ratto ha +1d6 al tiro di attacco contro una creatura se almeno uno degli alleati del ratto si trova entro 1 metro dalla creatura e quell'alleato non è inabile.

\textbf{Azioni}

\emph{\textbf{Morso.} Attacco con Arma da Mischia}: +4 a colpire, portata 1 m, un bersaglio.

\emph{Colpisce:} 4 (1d4 + 2) danni perforanti.

\medskip

\textbf{VARIANTE: RATTO GIGANTE AMMALATO}
\index[Mostruario]{Ratto Gigante ammalato}\hypertarget{Ratto Gigante ammalato}{}

Alcuni ratti giganti recano una terribile malattia che diffondono tramite il morso. Un ratto gigante ammalato ha grado di sfida 1/8 (25 PX) e la seguente azione invece del suo normale attacco di morso.

\emph{\textbf{Morso.} Attacco con Arma da Mischia}: +4 a colpire, portata 1 m, un bersaglio.

\emph{Colpisce:} 4 (1d4 + 2) danni perforanti. Se il bersaglio è una creatura, deve riuscire un Tiro Salvezza di Tempra DC 10 o contrarre una malattia. Fino a che la malattia non viene curata, TS Tempra DC 12 ogni 24 ore, il bersaglio non può recuperare Punti Ferita eccetto tramite metodi magici e i Punti Ferita massimi del bersaglio diminuiscono di 3 (1d6) ogni 24 ore. Se i Punti Ferita massimi del bersaglio scendono a 0 come risultato della malattia, il bersaglio muore.

\mostro{Rinoceronte lanoso}
\begin{description}[noitemsep, topsep=0pt, parsep=0pt, partopsep=0pt, leftmargin=0cm, labelwidth=2.2cm]
    \item[\textbf{Taglia/Tipo:}] Grande bestia, disallineato
    \item[\textbf{Caratt.:}] \resizebox{0.5\linewidth+1.8cm}{!}{For 5 Des -1 Cos 2 Int -4 Sag 1 Car -2}
    \item[\textbf{Punti Ferita:}] 51,  \textbf{Difesa:} 13,  \textbf{Iniziativa:} -1
    \item[\textbf{Tiri Salvez.:}] \resizebox{0.5\linewidth+1.8cm}{!}{Tempra +4, Riflessi +3, Volontà +3}
    \item[\textbf{Movimento:}] 12 m
    \item[\textbf{Sfida:}] 2 (450 PX)\smallskip
\end{description}

\emph{\textbf{Carica.}} Se il rinoceronte si muove di almeno 6 metri diretto verso un bersaglio e lo colpisce con un attacco di incornata durante lo stesso round, il bersaglio subisce 9 (2d8) danni contundenti aggiuntivi. Se il bersaglio è una creatura, deve riuscire un Tiro Salvezza su Tempra DC 15 o cadere prono.

\textbf{Azioni}

\emph{\textbf{Incornata.} Attacco con Arma da Mischia}: +4 a colpire, portata 1 m, un bersaglio.

\emph{Colpisce:} 14 (2d8 + 5) danni contundenti.

\mostro{Rospo Gigante}
\begin{description}[noitemsep, topsep=0pt, parsep=0pt, partopsep=0pt, leftmargin=0cm, labelwidth=2.2cm]
    \item[\textbf{Taglia/Tipo:}] Grande bestia, disallineato
    \item[\textbf{Caratt.:}] \resizebox{0.5\linewidth+1.8cm}{!}{For 2 Des 1 Cos 1 Int -4 Sag 0 Car -4}
    \item[\textbf{Punti Ferita:}] 33,  \textbf{Difesa:} 14,  \textbf{Iniziativa:} +1
    \item[\textbf{Tiri Salvez.:}] \resizebox{0.5\linewidth+1.8cm}{!}{Tempra +3, Riflessi +3, Volontà +3}
    \item[\textbf{Movimento:}] 6 m, nuoto 12 m
    \item[\textbf{Sfida:}] 1 (200 PX)\smallskip
\end{description}

\emph{\textbf{Anfibio.}} Il rospo può respirare aria e acqua.

\emph{\textbf{Salto da Fermo.}} Un rospo può saltare in lungo fino a 6 metri e in alto fino a 3 metri, con o senza la rincorsa.

\textbf{Azioni}

\emph{\textbf{Morso.} Attacco con Arma da Mischia}: +4 a colpire, portata 1 m, un bersaglio.

\emph{Colpisce:} 7 (1d10 + 2) danni perforanti più 5 (1d10) danni da veleno, e il bersaglio è afferrato (DC 13 per fuggire). Fino al termine dell'afferrare il rospo non può usare il morso contro un altro bersaglio.

\emph{\textbf{Inghiottire.}} Il rospo effettua una attacco di morso contro un bersaglio di taglia Media o inferiore che sta afferrando. Se l'attacco colpisce, il bersaglio è inghiottito, e l'afferrare ha termine. Il bersaglio inghiottito è accecato e intralciato, ha copertura completa contro gli attacchi e altri effetti all'esterno della rana, e subisce 10 (3d6) danni da acido all'inizio di ciascun round del rospo. Il rospo può inghiottire solo un bersaglio alla volta.

Se il rospo muore, una creatura inghiottita non è più intralciata da esso e può uscire dal cadavere utilizzando 1 metro di movimento, uscendo prono.

\emph{Colpisce:} 8 (2d4 + 3) danni contundenti.

\begin{center}
	\includegraphics[width=0.85\linewidth]{immagini/saurovallo2-ai.png}

	\emph{Saurovallo (B.I.C.)}
\end{center}

\mostro{Saurovallo da Galoppo}
\begin{description}[noitemsep, topsep=0pt, parsep=0pt, partopsep=0pt, leftmargin=0cm, labelwidth=2.2cm]
    \item[\textbf{Taglia/Tipo:}] Grande bestia, disallineato
    \item[\textbf{Caratt.:}] \resizebox{0.5\linewidth+1.8cm}{!}{For 3 Des 0 Cos 1 Int -3 Sag 0 Car -2}
    \item[\textbf{Punti Ferita:}] 19,  \textbf{Difesa:} 12,  \textbf{Iniziativa:} +0
    \item[\textbf{Tiri Salvez.:}] \resizebox{0.5\linewidth+1.8cm}{!}{Tempra +3, Riflessi +3, Volontà +3}
    \item[\textbf{Movimento:}] 18 m
    \item[\textbf{Sfida:}] 1/4 (50 PX)\smallskip
\end{description}

\textbf{Azioni}

\emph{\textbf{Zoccoli.} Attacco con Arma da Mischia}: +4 a colpire, portata 1 m, un bersaglio.

\mostro{Saurovallo da Guerra}
\begin{description}[noitemsep, topsep=0pt, parsep=0pt, partopsep=0pt, leftmargin=0cm, labelwidth=2.2cm]
    \item[\textbf{Taglia/Tipo:}] Grande bestia, disallineato
    \item[\textbf{Caratt.:}] \resizebox{0.5\linewidth+1.8cm}{!}{For 4 Des 1 Cos 1 Int -2 Sag 1 Car -2}
    \item[\textbf{Punti Ferita:}] 24,  \textbf{Difesa:} 13,  \textbf{Iniziativa:} +1
    \item[\textbf{Tiri Salvez.:}] \resizebox{0.5\linewidth+1.8cm}{!}{Tempra +3, Riflessi +3, Volontà +3}
    \item[\textbf{Movimento:}] 18 m
    \item[\textbf{Sfida:}] 1/2 (100 PX)\smallskip
\end{description}

\emph{\textbf{Carica Travolgente.}} Se il Saurovallo si muove di almeno 6 metri diretto verso il bersaglio e lo colpisce con un attacco di zoccoli durante lo stesso round, il bersaglio deve riuscire un Tiro Salvezza su Tempra DC 14 o cadere prono. Se il bersaglio è prono, il saurovallo può effettuare un altro attacco di zoccoli contro di lui come Azione Immediata.

\textbf{Azioni}

\emph{\textbf{Zoccoli.} Attacco con Arma da Mischia}: +4 a colpire, portata 1 m, un bersaglio.

\emph{Colpisce:} 11 (2d6 + 4) danni contundenti.


\mostro{Saurovallo da Tiro}
\begin{description}[noitemsep, topsep=0pt, parsep=0pt, partopsep=0pt, leftmargin=0cm, labelwidth=2.2cm]
    \item[\textbf{Taglia/Tipo:}] Grande bestia, disallineato
    \item[\textbf{Caratt.:}] \resizebox{0.5\linewidth+1.8cm}{!}{For 4 Des 0 Cos 1 Int -3 Sag 0 Car -2}
    \item[\textbf{Punti Ferita:}] 19,  \textbf{Difesa:} 12,  \textbf{Iniziativa:} +0
    \item[\textbf{Tiri Salvez.:}] \resizebox{0.5\linewidth+1.8cm}{!}{Tempra +3, Riflessi +3, Volontà +3}
    \item[\textbf{Movimento:}] 12 m
    \item[\textbf{Sfida:}] 1/4 (50 PX)\smallskip
\end{description}

\textbf{Azioni}

\emph{\textbf{Zoccoli.} Attacco con Arma da Mischia}: +4 a colpire, portata 1 m, un bersaglio.

\emph{Colpisce:} 9 (2d4 + 4) danni contundenti.


\begin{giocatore}[Il Saurovallo]\index{Saurovallo}
		La leggenda narra che Calicante appena scese sulla Terra vide i \emph{cavalli} e provò un disgusto incredibile per questi orrendi esseri e con il semplice volere li fece esplodere tutti. Non contento pochi attimi dopo tutti gli \emph{equini} fecero la stessa fine.

		Asini, muli, cavalli, zebre.. solo il cammello ed il dromedario non essendo propriamente equini si salvarono, anche se molti pensano che Calicante semplicemente li stia ignorando...

		Nethergal piuttosto scossa dal fatto che si era perso un utile animale per portare messaggi e cavalcabile per ampie distanze e non avendo il potere per creare una nuova creatura dal nulla, si rivolse ad Efrem ed Orlaith. Chiese ad Efrem di individuare un animale che potesse essere robusto, veloce ed adatto a essere cavalcato, mentre ad Orlaith chiese di inculcargli obbedienza ed il coraggio.

		Efrem sapendo che Torbiorn aveva portato sul pianeta milioni dei suoi amati dinosauri scelse il Parasaurolophus e, con il supporto di Orlaith, lo rese più compatto, piccolo, mansueto, erbivoro: perfetto per essere cavalcato.

		Creò anche poi una versione ridotta, nana nelle misure, che potesse adattarsi a portare le creature di taglia piccola.

		Purtroppo zanzare, cimici e mosche sono rimasti con massimo dispiacere di tutti!
\end{giocatore}

\mostro{Saurovallo nano}
\begin{description}[noitemsep, topsep=0pt, parsep=0pt, partopsep=0pt, leftmargin=0cm, labelwidth=2.2cm]
    \item[\textbf{Taglia/Tipo:}] Media bestia, disallineato
    \item[\textbf{Caratt.:}] \resizebox{0.5\linewidth+1.8cm}{!}{For 2 Des 0 Cos 1 Int -3 Sag 0 Car -2}
    \item[\textbf{Punti Ferita:}] 17,  \textbf{Difesa:} 12,  \textbf{Iniziativa:} +0
    \item[\textbf{Tiri Salvez.:}] \resizebox{0.5\linewidth+1.8cm}{!}{Tempra +3, Riflessi +3, Volontà +3}
    \item[\textbf{Movimento:}] 12 m
    \item[\textbf{Sfida:}] 1/8 (25 PX)\smallskip
\end{description}

\textbf{Azioni}

\emph{\textbf{Zoccoli.} Attacco con Arma da Mischia}: +4 a colpire, portata 1 m, un bersaglio.

\emph{Colpisce:} 7 (2d4 + 2) danni contundenti.

\mostro{Scarabeo di Fuoco Gigante}
\begin{description}[noitemsep, topsep=0pt, parsep=0pt, partopsep=0pt, leftmargin=0cm, labelwidth=2.2cm]
    \item[\textbf{Taglia/Tipo:}] Piccola bestia, disallineato
    \item[\textbf{Caratt.:}] \resizebox{0.5\linewidth+1.8cm}{!}{For -1 Des 0 Cos 1 Int -5 Sag -2 Car -4}
    \item[\textbf{Punti Ferita:}] 15,  \textbf{Difesa:} 12,  \textbf{Iniziativa:} +0
    \item[\textbf{Tiri Salvez.:}] \resizebox{0.5\linewidth+1.8cm}{!}{Tempra +3, Riflessi +3, Volontà +3}
    \item[\textbf{Movimento:}] 9 m
    \item[\textbf{Sfida:}] 0(10 PX)\smallskip
\end{description}

\emph{\textbf{Illuminazione.}} Lo scarabeo irradia luce intensa in un raggio di 3 metri e luce fioca per 6 metri.

\textbf{Azioni}

\emph{\textbf{Morso.} Attacco con Arma da Mischia}: +2 a colpire, portata 1 m, un bersaglio.

\emph{Colpisce:} 2 (1d6 - 1) danni taglienti.

\mostro{Sciacallo}
\begin{description}[noitemsep, topsep=0pt, parsep=0pt, partopsep=0pt, leftmargin=0cm, labelwidth=2.2cm]
    \item[\textbf{Taglia/Tipo:}] Piccola bestia, disallineato
    \item[\textbf{Caratt.:}] \resizebox{0.5\linewidth+1.8cm}{!}{For -1 Des 2 Cos 0 Int -4 Sag 1 Car -2}
    \item[\textbf{Punti Ferita:}] 15,  \textbf{Difesa:} 14,  \textbf{Iniziativa:} +2
    \item[\textbf{Tiri Salvez.:}] \resizebox{0.5\linewidth+1.8cm}{!}{Tempra +3, Riflessi +3, Volontà +3}
    \item[\textbf{Movimento:}] 12 m
    \item[\textbf{Sfida:}] 0 (10 PX)\smallskip
\end{description}

\emph{\textbf{Tattiche di Branco.}} Lo sciacallo ha +1d6 ai tiri di attacco contro una creatura se almeno uno degli alleati dello sciacallo si trova entro 1 metro dalla creatura e quell'alleato non è inabile.

\emph{\textbf{Udito e Olfatto Affinato.}} Lo sciacallo ha +1d6 nelle prove di Consapevolezza basate su udito o olfatto.

\textbf{Azioni}

\emph{\textbf{Morso.} Attacco con Arma da Mischia}: +2 a colpire, portata 1 m, un bersaglio.

\emph{Colpisce:} 1 (1d4 - 1) danni perforanti.

\mostro{Sciame di Centopiedi}
\begin{description}[noitemsep, topsep=0pt, parsep=0pt, partopsep=0pt, leftmargin=0cm, labelwidth=2.2cm]
    \item[\textbf{Taglia/Tipo:}] Media sciame di Minuscole bestie, disallineato
    \item[\textbf{Caratt.:}] \resizebox{0.5\linewidth+1.8cm}{!}{For -4 Des 1 Cos 0 Int -5 Sag -2 Car -5}
    \item[\textbf{Punti Ferita:}] 24,  \textbf{Difesa:} 13,  \textbf{Iniziativa:} +1
    \item[\textbf{Resistenze al danno:}] contundente, perforante, tagliente
    \item[\textbf{Tiri Salvez.:}] \resizebox{0.5\linewidth+1.8cm}{!}{Tempra +3, Riflessi +3, Volontà +3}
    \item[\textbf{Movimento:}] 6 m, scalata 6 m
    \item[\textbf{Sfida:}] 1/2 (100 PX)\smallskip
\end{description}

\emph{\textbf{Sciame.}} Lo sciame può occupare lo spazio di un'altra creatura e viceversa, lo sciame può muoversi attraverso qualsiasi apertura grande abbastanza per un Minuscolo insetto. Lo sciame non può recuperare Punti Ferita né ottenere Punti Ferita temporanei.

\textbf{Azioni}

\emph{\textbf{Morsi.} Attacco con Arma da Mischia}: +4 a colpire, portata 0 m, un bersaglio nello spazio dello sciame.

\emph{Colpisce:} 10 (4d4) danni perforanti, o 5 (2d4) danni perforanti se lo sciame è ha metà o meno dei suoi Punti Ferita. Una creatura ridotta a 0 Punti Ferita da uno sciame di centopiedi ma stabile resta avvelenata per 1 ora, anche dopo aver recuperato i Punti Ferita, e rimane paralizzata dal veleno durante questo periodo.

\mostro{Sciame di Corvi}
\begin{description}[noitemsep, topsep=0pt, parsep=0pt, partopsep=0pt, leftmargin=0cm, labelwidth=2.2cm]
    \item[\textbf{Taglia/Tipo:}] Media sciame di Minuscole bestie, disallineato
    \item[\textbf{Caratt.:}] \resizebox{0.5\linewidth+1.8cm}{!}{For -2 Des 2 Cos -1 Int -4 Sag 1 Car -2}
    \item[\textbf{Punti Ferita:}] 19,  \textbf{Difesa:} 14,  \textbf{Iniziativa:} +2
    \item[\textbf{Resistenze al danno:}] contundente, perforante, tagliente
    \item[\textbf{Tiri Salvez.:}] \resizebox{0.5\linewidth+1.8cm}{!}{Tempra +3, Riflessi +3, Volontà +3}
    \item[\textbf{Movimento:}] 3 m, volo 15 m
    \item[\textbf{Sfida:}] 1/4 (50 PX)\smallskip
\end{description}

\emph{\textbf{Sciame.}} Lo sciame può occupare lo spazio di un'altra creatura e viceversa, lo sciame può muoversi attraverso qualsiasi apertura grande abbastanza per un Minuscolo corvo. Lo sciame non può recuperare Punti Ferita né ottenere Punti Ferita temporanei.

\textbf{Azioni}

\emph{\textbf{Becchi.} Attacco con Arma da Mischia}: +4 a colpire, portata 1 m, un bersaglio nello spazio dello sciame.

\emph{Colpisce:} 7 (2d6) danni perforanti, o 3 (1d6) danni perforanti se lo sciame è ha metà o meno dei suoi Punti Ferita.

\mostro{Sciame di Pirana}
\begin{description}[noitemsep, topsep=0pt, parsep=0pt, partopsep=0pt, leftmargin=0cm, labelwidth=2.2cm]
    \item[\textbf{Taglia/Tipo:}] Media sciame di Minuscole bestie, disallineato
    \item[\textbf{Caratt.:}] \resizebox{0.5\linewidth+1.8cm}{!}{For 1 Des 3 Cos -1 Int -5 Sag -2 Car -4}
    \item[\textbf{Punti Ferita:}] 32,  \textbf{Difesa:} 16,  \textbf{Iniziativa:} +3
    \item[\textbf{Resistenze al danno:}] contundente, perforante, tagliente
    \item[\textbf{Tiri Salvez.:}] \resizebox{0.5\linewidth+1.8cm}{!}{Tempra +3, Riflessi +4, Volontà +3}
    \item[\textbf{Movimento:}] 0 m, nuoto 12 m
    \item[\textbf{Sfida:}] 1 (200 PX)\smallskip
\end{description}

\emph{\textbf{Frenesia Sanguinaria.}} Lo sciame ha +1d6 ai tiri di attacco in mischia contro qualsiasi creatura che non sia al massimo dei Punti Ferita.

\emph{\textbf{Respirare Acqua.}} Lo sciame può respirare solo sott'acqua.

\emph{\textbf{Sciame.}} Lo sciame può occupare lo spazio di un'altra creatura e viceversa, lo sciame può muoversi attraverso qualsiasi apertura grande abbastanza per un Minuscolo pirana. Lo sciame non può recuperare Punti Ferita né ottenere Punti Ferita temporanei.

\textbf{Azioni}

\emph{\textbf{Morsi.} Attacco con Arma da Mischia}: +5 a colpire, portata 0 m, una creatura nello spazio dello sciame.

\emph{Colpisce:} 14 (4d6) danni perforanti, o 7 (2d6) danni perforanti se lo sciame è ha metà o meno dei suoi Punti Ferita.

\mostro{Sciame di Insetti}
\begin{description}[noitemsep, topsep=0pt, parsep=0pt, partopsep=0pt, leftmargin=0cm, labelwidth=2.2cm]
    \item[\textbf{Taglia/Tipo:}] Media sciame di Minuscole bestie, disallineato
    \item[\textbf{Caratt.:}] \resizebox{0.5\linewidth+1.8cm}{!}{For -4 Des 1 Cos 0 Int -5 Sag -2 Car -5}
    \item[\textbf{Punti Ferita:}] 24,  \textbf{Difesa:} 13,  \textbf{Iniziativa:} +1
    \item[\textbf{Resistenze al danno:}] contundente, perforante, tagliente
    \item[\textbf{Tiri Salvez.:}] \resizebox{0.5\linewidth+1.8cm}{!}{Tempra +3, Riflessi +3, Volontà +3}
    \item[\textbf{Movimento:}] 6 m, scalata 6 m
    \item[\textbf{Sfida:}] 1/2 (100 PX)\smallskip
\end{description}

\emph{\textbf{Sciame.}} Lo sciame può occupare lo spazio di un'altra creatura e viceversa, lo sciame può muoversi attraverso qualsiasi apertura grande abbastanza per un Minuscolo insetto. Lo sciame non può recuperare Punti Ferita né ottenere Punti Ferita temporanei.

\textbf{Azioni}

\emph{\textbf{Morsi.} Attacco con Arma da Mischia}: +3 a colpire, portata 0 m, un bersaglio nello spazio dello sciame.

\emph{Colpisce:} 10 (4d4) danni perforanti, o 5 (2d4) danni perforanti se lo sciame è ha metà o meno dei suoi Punti Ferita.

\mostro{Sciame di Pipistrelli}
\begin{description}[noitemsep, topsep=0pt, parsep=0pt, partopsep=0pt, leftmargin=0cm, labelwidth=2.2cm]
    \item[\textbf{Taglia/Tipo:}] Media sciame di Minuscole bestie, disallineato
    \item[\textbf{Caratt.:}] \resizebox{0.5\linewidth+1.8cm}{!}{For -3 Des 2 Cos 0 Int -4 Sag 1 Car -3}
    \item[\textbf{Punti Ferita:}] 19,  \textbf{Difesa:} 14,  \textbf{Iniziativa:} +2
    \item[\textbf{Resistenze al danno:}] contundente, perforante, tagliente
    \item[\textbf{Tiri Salvez.:}] \resizebox{0.5\linewidth+1.8cm}{!}{Tempra +3, Riflessi +3, Volontà +3}
    \item[\textbf{Movimento:}] 0 m, volo 9 m
    \item[\textbf{Sfida:}] 1/4 (50 PX)\smallskip
\end{description}

\emph{\textbf{Ecolocazione.}} Lo sciame non può usare la vista cieca se assordato.

\emph{\textbf{Sciame.}} Lo sciame può occupare lo spazio di un'altra creatura e viceversa, lo sciame può muoversi attraverso qualsiasi apertura grande abbastanza per un Minuscolo pipistrello. Lo sciame non può recuperare Punti Ferita né ottenere Punti Ferita temporanei.

\emph{\textbf{Udito Affinato.}} Lo sciame ha +1d6 alle prove di Consapevolezza basate sull'udito.

\textbf{Azioni}

\emph{\textbf{Morsi.} Attacco con Arma da Mischia}: +4 a colpire, portata 0 m, una creatura nello spazio dello sciame.

\emph{Colpisce:} 5 (2d4) danni perforanti, o 2 (1d4) danni perforanti se lo sciame è ha metà o meno dei suoi Punti Ferita.

\mostro{Sciame di Ragni}
\begin{description}[noitemsep, topsep=0pt, parsep=0pt, partopsep=0pt, leftmargin=0cm, labelwidth=2.2cm]
    \item[\textbf{Taglia/Tipo:}] Media sciame di Minuscole bestie, disallineato
    \item[\textbf{Caratt.:}] \resizebox{0.5\linewidth+1.8cm}{!}{For -4 Des 1 Cos 0 Int -5 Sag -2 Car -5}
    \item[\textbf{Punti Ferita:}] 24,  \textbf{Difesa:} 13,  \textbf{Iniziativa:} +1
    \item[\textbf{Resistenze al danno:}] contundente, perforante, tagliente
    \item[\textbf{Tiri Salvez.:}] \resizebox{0.5\linewidth+1.8cm}{!}{Tempra +3, Riflessi +3, Volontà +3}
    \item[\textbf{Movimento:}] 6 m, scalata 6 m
    \item[\textbf{Sfida:}] 1/2 (100 PX)\smallskip
\end{description}

\emph{\textbf{Camminare sulla Tela.}} Lo sciame ignora le restrizioni al movimento provocate dalle ragnatele.

\emph{\textbf{Scalare come Ragno.}} Lo sciame può scalare superfici difficili, compreso lo stare a testa in giù sul soffitto, senza bisogno di effettuare una prova di competenza.

\emph{\textbf{Senso della Tela.}} Mentre è in contatto con una ragnatela, lo sciame sa l'esatta posizione di qualsiasi altra creatura in contatto con la stessa ragnatela.

\emph{\textbf{Sciame.}} Lo sciame può occupare lo spazio di un'altra creatura e viceversa, lo sciame può muoversi attraverso qualsiasi apertura grande abbastanza per un Minuscolo insetto. Lo sciame non può recuperare Punti Ferita né ottenere Punti Ferita temporanei.

\textbf{Azioni}

\emph{\textbf{Morsi.} Attacco con Arma da Mischia}: +3 a colpire, portata 0 m, un bersaglio nello spazio dello sciame.

\emph{Colpisce:} 10 (4d4) danni perforanti, o 5 (2d4) danni perforanti se lo sciame è ha metà o meno dei suoi Punti Ferita.

\mostro{Sciame di Ratti}
\begin{description}[noitemsep, topsep=0pt, parsep=0pt, partopsep=0pt, leftmargin=0cm, labelwidth=2.2cm]
    \item[\textbf{Taglia/Tipo:}] Media sciame di Minuscole bestie, disallineato
    \item[\textbf{Caratt.:}] \resizebox{0.5\linewidth+1.8cm}{!}{For -1 Des 0 Cos -1 Int -4 Sag 0 Car -4}
    \item[\textbf{Punti Ferita:}] 19,  \textbf{Difesa:} 12,  \textbf{Iniziativa:} +0
    \item[\textbf{Resistenze al danno:}] contundente, perforante, tagliente
    \item[\textbf{Tiri Salvez.:}] \resizebox{0.5\linewidth+1.8cm}{!}{Tempra +3, Riflessi +3, Volontà +3}
    \item[\textbf{Movimento:}] 9 m
    \item[\textbf{Sfida:}] 1/4 (50 PX)\smallskip
\end{description}

\emph{\textbf{Olfatto Affinato.}} Lo sciame ha +1d6 alle prove di Consapevolezza basate sull'olfatto.

\emph{\textbf{Sciame.}} Lo sciame può occupare lo spazio di un'altra creatura e viceversa, lo sciame può muoversi attraverso qualsiasi apertura grande abbastanza per un Minuscolo ratto. Lo sciame non può recuperare Punti Ferita né ottenere Punti Ferita temporanei.

\textbf{Azioni}

\emph{\textbf{Morsi.} Attacco con Arma da Mischia}: +4 a colpire, portata 0 m, un bersaglio nello spazio dello sciame.

\emph{Colpisce:} 7 (2d6) danni perforanti, o 3 (1d6) danni perforanti se lo sciame è ha metà o meno dei suoi Punti Ferita.

\emph{\textbf{Sciame.}} Lo sciame può occupare lo spazio di un'altra creatura e viceversa, lo sciame può muoversi attraverso qualsiasi apertura grande abbastanza per un Minuscolo insetto. Lo sciame non può recuperare Punti Ferita né ottenere Punti Ferita temporanei.

\textbf{Azioni}

\emph{\textbf{Morsi.} Attacco con Arma da Mischia}: +3 a colpire, portata 0 m, un bersaglio nello spazio dello sciame.

\emph{Colpisce:} 10 (4d4) danni perforanti, o 5 (2d4) danni perforanti se lo sciame è ha metà o meno dei suoi Punti Ferita.

\mostro{Sciame di Serpenti Velenosi}
\begin{description}[noitemsep, topsep=0pt, parsep=0pt, partopsep=0pt, leftmargin=0cm, labelwidth=2.2cm]
    \item[\textbf{Taglia/Tipo:}] Media sciame di Minuscole bestie, disallineato
    \item[\textbf{Caratt.:}] \resizebox{0.5\linewidth+1.8cm}{!}{For -1 Des 4 Cos 0 Int -5 Sag 0 Car -4}
    \item[\textbf{Punti Ferita:}] 19,  \textbf{Difesa:} 16,  \textbf{Iniziativa:} +4
    \item[\textbf{Resistenze al danno:}] contundente, perforante, tagliente
    \item[\textbf{Tiri Salvez.:}] \resizebox{0.5\linewidth+1.8cm}{!}{Tempra +3, Riflessi +4, Volontà +3}
    \item[\textbf{Movimento:}] 9 m
    \item[\textbf{Sfida:}] 1/4 (50 PX)\smallskip
\end{description}

\emph{\textbf{Sciame.}} Lo sciame può occupare lo spazio di un'altra creatura e viceversa, lo sciame può muoversi attraverso qualsiasi apertura grande abbastanza per un Minuscolo serpente. Lo sciame non può recuperare Punti Ferita né ottenere Punti Ferita temporanei.

\textbf{Azioni}

\emph{\textbf{Morsi.} Attacco con Arma da Mischia}: +4 a colpire, portata 0 m, una creatura nello spazio dello sciame.

\emph{Colpisce:} 7 (2d6) danni perforanti, o 3 (1d6) danni perforanti se lo sciame è ha metà o meno dei suoi Punti Ferita, e il bersaglio deve effettuare un Tiro Salvezza di Tempra DC 10, e subire 14 (4d6) danni da veleno se fallisce il Tiro Salvezza, o la metà di questi danni se lo riesce.

\mostro{Sciame di Vespe}
\begin{description}[noitemsep, topsep=0pt, parsep=0pt, partopsep=0pt, leftmargin=0cm, labelwidth=2.2cm]
	\item[\textbf{Taglia/Tipo:}] Media sciame di Minuscole bestie, disallineato
	\item[\textbf{Caratt.:}] \resizebox{0.5\linewidth+1.8cm}{!}{For -1 Des 1 Cos 0 Int -5 Sag -2 Car -5}
	\item[\textbf{Punti Ferita:}] 24,  \textbf{Difesa:} 13,  \textbf{Iniziativa:} +1
	\item[\textbf{Tiri Salvez.:}] \resizebox{0.5\linewidth+1.8cm}{!}{Tempra +3, Riflessi +3, Volontà +3}
	\item[\textbf{Movimento:}] 1m, volo 9 m
	\item[\textbf{Sfida:}] 1/2 (100 PX)\smallskip
\end{description}

\emph{\textbf{Sciame.}} Lo sciame può occupare lo spazio di un'altra creatura e viceversa, e lo sciame può muoversi attraverso qualsiasi apertura grande abbastanza per un Minuscolo insetto. Lo sciame non può recuperare Punti Ferita né ottenere Punti Ferita temporanei.

\textbf{Azioni}

\emph{\textbf{Morsi.} Attacco con Arma da Mischia}: +4 a colpire, portata 0 m, un bersaglio nello spazio dello sciame.

\emph{Colpisce:} 10 (4d4) danni perforanti, o 5 (2d4) danni perforanti se lo sciame è ha metà o meno dei suoi Punti Ferita.

\mostro{Scimmia}
\begin{description}[noitemsep, topsep=0pt, parsep=0pt, partopsep=0pt, leftmargin=0cm, labelwidth=2.2cm]
    \item[\textbf{Taglia/Tipo:}] Piccola bestia, disallineato
    \item[\textbf{Caratt.:}] \resizebox{0.5\linewidth+1.8cm}{!}{For -3 Des 2 Cos 0 Int -3 Sag 1 Car -2}
    \item[\textbf{Punti Ferita:}] 19,  \textbf{Difesa:} 14,  \textbf{Iniziativa:} +2
    \item[\textbf{Comp.:}] Acrobatica +6, Consapevolezza +3
    \item[\textbf{Tiri Salvez.:}] \resizebox{0.5\linewidth+1.8cm}{!}{Tempra +3, Riflessi +3, Volontà +3}
    \item[\textbf{Movimento:}] 9 m, scalata 9 m
    \item[\textbf{Sfida:}] 1/4 (50 PX)\smallskip
\end{description}

\textbf{Azioni}

\emph{\textbf{Graffio.} Attacco con arma da Mischia}: +3 a colpire, portata 1 m, un bersaglio.

\emph{Colpisce:} 1 (1d4 - 1) danni da taglio.

\emph{\textbf{Morso.} Attacco con Arma da Mischia}: +2 a colpire, portata 1 metro, un bersaglio.

\emph{Colpisce:} 2 (1d4) danni perforanti.

\mostro{Scimmione}
\begin{description}[noitemsep, topsep=0pt, parsep=0pt, partopsep=0pt, leftmargin=0cm, labelwidth=2.2cm]
    \item[\textbf{Taglia/Tipo:}] Media bestia, disallineato
    \item[\textbf{Caratt.:}] \resizebox{0.5\linewidth+1.8cm}{!}{For 3 Des 2 Cos 2 Int -2 Sag 1 Car -2}
    \item[\textbf{Punti Ferita:}] 24,  \textbf{Difesa:} 14,  \textbf{Iniziativa:} +2
    \item[\textbf{Tiri Salvez.:}] \resizebox{0.5\linewidth+1.8cm}{!}{Tempra +3, Riflessi +3, Volontà +3}
    \item[\textbf{Movimento:}] 9 m, scalata 9 m
    \item[\textbf{Sfida:}] 1/2 (100 PX)\smallskip
\end{description}

\textbf{Azioni}

\emph{\textbf{Multiattacco.}} Lo scimmione effettua due attacchi di pugno.

\emph{\textbf{Pugno.} Attacco con Arma da Mischia}: +5 a colpire, portata 1 m, un bersaglio.

\emph{Colpisce:} 6 (1d6 + 3) danni contundenti.

\emph{\textbf{Sasso.} Attacco con Arma a Gittata}: +5 a colpire, gittata 8m, un bersaglio.

\emph{Colpisce:} 6 (1d6 + 3) danni contundenti.

\mostro{Scimmione Gigante}
\begin{description}[noitemsep, topsep=0pt, parsep=0pt, partopsep=0pt, leftmargin=0cm, labelwidth=2.2cm]
    \item[\textbf{Taglia/Tipo:}] Enorme bestia, disallineato
    \item[\textbf{Caratt.:}] \resizebox{0.5\linewidth+1.8cm}{!}{For 6 Des 2 Cos 4 Int -2 Sag 1 Car -2}
    \item[\textbf{Punti Ferita:}] 146,  \textbf{Difesa:} 23,  \textbf{Iniziativa:} +2
    \item[\textbf{Tiri Salvez.:}] \resizebox{0.5\linewidth+1.8cm}{!}{Tempra +11, Riflessi +9, Volontà +8}
    \item[\textbf{Movimento:}] 12 m, scalata 12 m
    \item[\textbf{Sfida:}] 7 (2900 PX)\smallskip
\end{description}

\textbf{Azioni}

\emph{\textbf{Multiattacco.}} Lo scimmione effettua due attacchi di pugno.

\emph{\textbf{Pugno.} Attacco con Arma da Mischia}: +9 a colpire, portata 3 m, un bersaglio.

\emph{Colpisce:} 22 (3d10 + 6) danni contundenti.

\emph{\textbf{Sasso.} Attacco con Arma a Gittata}: +9 a colpire, gittata 15m, un bersaglio.

\emph{Colpisce:} 30 (7d6 + 6) danni contundenti.

\mostro{Scorpione}
\begin{description}[noitemsep, topsep=0pt, parsep=0pt, partopsep=0pt, leftmargin=0cm, labelwidth=2.2cm]
    \item[\textbf{Taglia/Tipo:}] Minuscola bestia, disallineato
    \item[\textbf{Caratt.:}] \resizebox{0.5\linewidth+1.8cm}{!}{For -4 Des 0 Cos -1 Int -5 Sag -1 Car -4}
    \item[\textbf{Punti Ferita:}] 15,  \textbf{Difesa:} 12,  \textbf{Iniziativa:} +0
    \item[\textbf{Tiri Salvez.:}] \resizebox{0.5\linewidth+1.8cm}{!}{Tempra +3, Riflessi +3, Volontà +3}
    \item[\textbf{Movimento:}] 3 m
    \item[\textbf{Sfida:}] 0(10 PX)\smallskip
\end{description}

\textbf{Azioni}

\emph{\textbf{Pungiglione.} Attacco con Arma da Mischia}: +2 a colpire, portata 1 m, una creatura.

\emph{Colpisce:} 1 danno perforante e il bersaglio deve effettuare un Tiro Salvezza di Tempra DC 9, e subire 4 (1d8) danni da veleno se fallisce il Tiro Salvezza, o la metà di questi danni se lo riesce.

\mostro{Scorpione Gigante}
\begin{description}[noitemsep, topsep=0pt, parsep=0pt, partopsep=0pt, leftmargin=0cm, labelwidth=2.2cm]
    \item[\textbf{Taglia/Tipo:}] Grande bestia, disallineato
    \item[\textbf{Caratt.:}] \resizebox{0.5\linewidth+1.8cm}{!}{For 2 Des 1 Cos 2 Int -5 Sag -1 Car -4}
    \item[\textbf{Punti Ferita:}] 70,  \textbf{Difesa:} 17,  \textbf{Iniziativa:} +1
    \item[\textbf{Tiri Salvez.:}] \resizebox{0.5\linewidth+1.8cm}{!}{Tempra +5, Riflessi +4, Volontà +3}
    \item[\textbf{Movimento:}] 12 m
    \item[\textbf{Sfida:}] 3 (700 PX)\smallskip
\end{description}

\textbf{Azioni}

\emph{\textbf{Multiattacco.}} Lo scorpione effettua tre attacchi: due con gli artigli e uno con il pungiglione.

\emph{\textbf{Artiglio.} Attacco con Arma da Mischia}: +5 a colpire, portata 1 m, un bersaglio.

\emph{Colpisce:} 6 (1d8 + 2) danni contundenti e il bersaglio è afferrato (DC 12 per fuggire). Lo scorpione ha due artigli, ciascuno dei quali può afferrare solo un bersaglio.

\emph{\textbf{Pungiglione.} Attacco con Arma da Mischia}: +5 a colpire, portata 1 m, una creatura.

\emph{Colpisce:} 7 (1d10 + 2) danni perforanti e il bersaglio deve effettuare un Tiro Salvezza di Tempra DC 14, e subire 22 (4d10) danni da veleno se fallisce il Tiro Salvezza, o la metà di questi danni se lo riesce.

\mostro{Serpente Costrittore}
\begin{description}[noitemsep, topsep=0pt, parsep=0pt, partopsep=0pt, leftmargin=0cm, labelwidth=2.2cm]
    \item[\textbf{Taglia/Tipo:}] Grande bestia, disallineato
    \item[\textbf{Caratt.:}] \resizebox{0.5\linewidth+1.8cm}{!}{For 2 Des 2 Cos 1 Int -5 Sag 0 Car -4}
    \item[\textbf{Punti Ferita:}] 19,  \textbf{Difesa:} 14,  \textbf{Iniziativa:} +2
    \item[\textbf{Tiri Salvez.:}] \resizebox{0.5\linewidth+1.8cm}{!}{Tempra +3, Riflessi +3, Volontà +3}
    \item[\textbf{Movimento:}] 9 m, nuoto 9 m
    \item[\textbf{Sfida:}] 1/4 (50 PX)\smallskip
\end{description}

\textbf{Azioni}

\emph{\textbf{Morso.} Attacco con Arma da Mischia}: +4 a colpire, portata 1 m, una creatura.

\emph{Colpisce:} 5 (1d6 + 2) danni perforanti.

\emph{\textbf{Stritolare.} Attacco con Arma da Mischia}: +4 a colpire, portata 1 m, una creatura.

\emph{Colpisce:} 6 (1d8 + 2) danni contundenti, e il bersaglio è afferrato (DC 14 per fuggire). Fino al termine dell'afferrare, la creatura è intralciata, e il serpente non può stritolare un altro bersaglio.

\mostro{Serpente Costrittore Gigante}
\begin{description}[noitemsep, topsep=0pt, parsep=0pt, partopsep=0pt, leftmargin=0cm, labelwidth=2.2cm]
    \item[\textbf{Taglia/Tipo:}] Enorme bestia, disallineato
    \item[\textbf{Caratt.:}] \resizebox{0.5\linewidth+1.8cm}{!}{For 4 Des 2 Cos 1 Int -5 Sag 0 Car -4}
    \item[\textbf{Punti Ferita:}] 51,  \textbf{Difesa:} 16,  \textbf{Iniziativa:} +2
    \item[\textbf{Tiri Salvez.:}] \resizebox{0.5\linewidth+1.8cm}{!}{Tempra +3, Riflessi +4, Volontà +3}
    \item[\textbf{Movimento:}] 9 m, nuoto 9 m
    \item[\textbf{Sfida:}] 2 (450 PX)\smallskip
\end{description}

\textbf{Azioni}

\emph{\textbf{Morso.} Attacco con Arma da Mischia}: +5 a colpire, portata 3 m, una creatura.

\emph{Colpisce:} 11 (2d6 + 4) danni perforanti.

\emph{\textbf{Stritolare.} Attacco con Arma da Mischia}: +5 a colpire, portata 1 m, una creatura.

\emph{Colpisce:} 13 (2d8 + 4) danni contundenti, e il bersaglio è afferrato (DC 16 per fuggire). Fino al termine dell'afferrare, la creatura è intralciata, e il serpente non può stritolare un altro bersaglio.

\mostro{Serpente Velenoso}
\begin{description}[noitemsep, topsep=0pt, parsep=0pt, partopsep=0pt, leftmargin=0cm, labelwidth=2.2cm]
    \item[\textbf{Taglia/Tipo:}] Minuscola bestia, disallineato
    \item[\textbf{Caratt.:}] \resizebox{0.5\linewidth+1.8cm}{!}{For -4 Des 3 Cos 0 Int -5 Sag 0 Car -4}
    \item[\textbf{Punti Ferita:}] 17,  \textbf{Difesa:} 15,  \textbf{Iniziativa:} +3
    \item[\textbf{Tiri Salvez.:}] \resizebox{0.5\linewidth+1.8cm}{!}{Tempra +3, Riflessi +3, Volontà +3}
    \item[\textbf{Movimento:}] 9 m, nuoto 9 m
    \item[\textbf{Sfida:}] 1/8 (25 PX)\smallskip
\end{description}

\textbf{Azioni}

\emph{\textbf{Morso.} Attacco con Arma da Mischia}: +4 a colpire, portata 1 m, un bersaglio.

\emph{Colpisce:} 1 danno perforante e il bersaglio deve effettuare un Tiro Salvezza di Tempra DC 10, e subire 5 (2d4) danni da veleno se fallisce il Tiro Salvezza, o la metà di questi danni se lo riesce.

\mostro{Serpente Velenoso Gigante}
\begin{description}[noitemsep, topsep=0pt, parsep=0pt, partopsep=0pt, leftmargin=0cm, labelwidth=2.2cm]
    \item[\textbf{Taglia/Tipo:}] Media bestia, disallineato
    \item[\textbf{Caratt.:}] \resizebox{0.5\linewidth+1.8cm}{!}{For 0 Des 4 Cos 1 Int -4 Sag 0 Car -4}
    \item[\textbf{Punti Ferita:}] 19,  \textbf{Difesa:} 16,  \textbf{Iniziativa:} +4
    \item[\textbf{Tiri Salvez.:}] \resizebox{0.5\linewidth+1.8cm}{!}{Tempra +3, Riflessi +4, Volontà +3}
    \item[\textbf{Movimento:}] 9 m, nuoto 9 m
    \item[\textbf{Sfida:}] 1/4 (50 PX)\smallskip
\end{description}

\textbf{Azioni}

\emph{\textbf{Morso.} Attacco con Arma da Mischia}: +4 a colpire, portata 3 m, un bersaglio.

\emph{Colpisce:} 6 (1d4 + 4) danni perforanti e il bersaglio deve effettuare un Tiro Salvezza di Tempra DC 11, e subire 10 (3d6) danni da veleno se fallisce il Tiro Salvezza, o la metà di questi danni se lo riesce.

\mostro{Serpente Volante}
\begin{description}[noitemsep, topsep=0pt, parsep=0pt, partopsep=0pt, leftmargin=0cm, labelwidth=2.2cm]
    \item[\textbf{Taglia/Tipo:}] Minuscola bestia, disallineato
    \item[\textbf{Caratt.:}] \resizebox{0.5\linewidth+1.8cm}{!}{For -3 Des 4 Cos 0 Int -4 Sag 1 Car -3}
    \item[\textbf{Punti Ferita:}] 17,  \textbf{Difesa:} 16,  \textbf{Iniziativa:} +4
    \item[\textbf{Tiri Salvez.:}] \resizebox{0.5\linewidth+1.8cm}{!}{Tempra +3, Riflessi +4, Volontà +3}
    \item[\textbf{Movimento:}] 9 m, nuoto 9 m, volo 18 m
    \item[\textbf{Sfida:}] 1/8 (25 PX)\smallskip
\end{description}

\emph{\textbf{Sorvolare.}} Il serpente non provoca attacchi di opportunità quando vola via dalla portata di un nemico.

\textbf{Azioni}

\emph{\textbf{Morso.} Attacco con Arma da Mischia}: +4 a colpire, portata 1 m, un bersaglio.

\emph{Colpisce:} 1 danno perforante più 7 (3d4) danni da veleno.

\mostro{Squalo Cacciatore}
\begin{description}[noitemsep, topsep=0pt, parsep=0pt, partopsep=0pt, leftmargin=0cm, labelwidth=2.2cm]
    \item[\textbf{Taglia/Tipo:}] Grande bestia, disallineato
    \item[\textbf{Caratt.:}] \resizebox{0.5\linewidth+1.8cm}{!}{For 4 Des 1 Cos 2 Int -5 Sag 0 Car -3}
    \item[\textbf{Punti Ferita:}] 51,  \textbf{Difesa:} 15,  \textbf{Iniziativa:} +1
    \item[\textbf{Tiri Salvez.:}] \resizebox{0.5\linewidth+1.8cm}{!}{Tempra +4, Riflessi +3, Volontà +3}
    \item[\textbf{Movimento:}] 0 m, nuoto 12 m
    \item[\textbf{Sfida:}] 2 (450 PX)\smallskip
\end{description}

\emph{\textbf{Frenesia Sanguinaria.}} Lo squalo ha +1d6 ai tiri di attacco in mischia contro qualsiasi creatura che non sia al massimo dei Punti Ferita.

\emph{\textbf{Respirare Acqua.}} Lo squalo può respirare solo sott'acqua.

\textbf{Azioni}

\emph{\textbf{Morso.} Attacco con Arma da Mischia}: +4 a colpire, portata 1 m, un bersaglio.

\emph{Colpisce:} 13 (2d8 + 4) danni perforanti.

\mostro{Squalo Corallino}
\begin{description}[noitemsep, topsep=0pt, parsep=0pt, partopsep=0pt, leftmargin=0cm, labelwidth=2.2cm]
    \item[\textbf{Taglia/Tipo:}] Media bestia, disallineato
    \item[\textbf{Caratt.:}] \resizebox{0.5\linewidth+1.8cm}{!}{For 2 Des 1 Cos 1 Int -5 Sag 0 Car -3}
    \item[\textbf{Punti Ferita:}] 24,  \textbf{Difesa:} 13,  \textbf{Iniziativa:} +1
    \item[\textbf{Tiri Salvez.:}] \resizebox{0.5\linewidth+1.8cm}{!}{Tempra +3, Riflessi +3, Volontà +3}
    \item[\textbf{Movimento:}] 0 m, nuoto 12 m
    \item[\textbf{Sfida:}] 1/2 (100 PX)\smallskip
\end{description}

\emph{\textbf{Respirare Acqua.}} Lo squalo può respirare solo sott'acqua.

\emph{\textbf{Tattiche di Branco.}} Lo squalo ha +1d6 al tiro di attacco contro una creatura se almeno uno degli alleati dello squalo si trova entro 1 metro dalla creatura e quell'alleato non è inabile.

\textbf{Azioni}

\emph{\textbf{Morso.} Attacco con Arma da Mischia}: +4 a colpire, portata 1 m, un bersaglio.

\emph{Colpisce:} 6 (1d8 + 2) danni perforanti.

\mostro{Squalo Gigante}
\begin{description}[noitemsep, topsep=0pt, parsep=0pt, partopsep=0pt, leftmargin=0cm, labelwidth=2.2cm]
    \item[\textbf{Taglia/Tipo:}] Enorme bestia, disallineato
    \item[\textbf{Caratt.:}] \resizebox{0.5\linewidth+1.8cm}{!}{For 6 Des 0 Cos 5 Int -5 Sag 0 Car -3}
    \item[\textbf{Punti Ferita:}] 110,  \textbf{Difesa:} 18,  \textbf{Iniziativa:} +0
    \item[\textbf{Tiri Salvez.:}] \resizebox{0.5\linewidth+1.8cm}{!}{Tempra +10, Riflessi +5, Volontà +5}
    \item[\textbf{Movimento:}] 0 m, nuoto 15 m
    \item[\textbf{Sfida:}] 5 (1800 PX)\smallskip
\end{description}

\emph{\textbf{Frenesia Sanguinaria.}} Lo squalo ha +1d6 ai tiri di attacco

in mischia contro qualsiasi creatura che non sia al massimo dei Punti Ferita.

\emph{\textbf{Respirare Acqua.}} Lo squalo può respirare solo sott'acqua.

\textbf{Azioni}

\emph{\textbf{Morso.} Attacco con Arma da Mischia}: +7 a colpire, portata 1 m, un bersaglio.

\emph{Colpisce:} 22 (3d10 + 6) danni perforanti.

\mostro{Strige}
\begin{description}[noitemsep, topsep=0pt, parsep=0pt, partopsep=0pt, leftmargin=0cm, labelwidth=2.2cm]
    \item[\textbf{Taglia/Tipo:}] Minuscola bestia, disallineato
    \item[\textbf{Caratt.:}] \resizebox{0.5\linewidth+1.8cm}{!}{For -3 Des 3 Cos 0 Int -4 Sag -1 Car -2}
    \item[\textbf{Punti Ferita:}] 17,  \textbf{Difesa:} 15,  \textbf{Iniziativa:} +3
    \item[\textbf{Tiri Salvez.:}] \resizebox{0.5\linewidth+1.8cm}{!}{Tempra +3, Riflessi +3, Volontà +3}
    \item[\textbf{Movimento:}] 3 m, volo 12 m
    \item[\textbf{Sfida:}] 1/8 (25 PX)\smallskip
\end{description}

\textbf{Azioni}

\emph{\textbf{Risucchio di Sangue.} Attacco con Arma da Mischia}: +4 a colpire, portata 1 m, una creatura.

\emph{Colpisce:} 5 (1d4 + 3) danni perforanti e lo strige si attacca al bersaglio. Mentre è attaccato, lo strige non attacca. Invece, all'inizio di ciascun round dello strige, il bersaglio perde 5 (1d4 + 3) Punti Ferita a causa della perdita di sangue.

Lo strige può staccarsi spendendo 1 Azione. Lo fa automaticamente dopo aver risucchiato 10 Punti Ferita dal bersaglio o alla morte del bersaglio. Una creatura, compreso il bersaglio, può usare una Azione per staccare lo strige.

\mostro{Tasso}
\begin{description}[noitemsep, topsep=0pt, parsep=0pt, partopsep=0pt, leftmargin=0cm, labelwidth=2.2cm]
    \item[\textbf{Taglia/Tipo:}] Minuscola bestia, disallineato
    \item[\textbf{Caratt.:}] \resizebox{0.5\linewidth+1.8cm}{!}{For -3 Des 0 Cos 1 Int -4 Sag 1 Car -3}
    \item[\textbf{Punti Ferita:}] 15,  \textbf{Difesa:} 12,  \textbf{Iniziativa:} +0
    \item[\textbf{Tiri Salvez.:}] \resizebox{0.5\linewidth+1.8cm}{!}{Tempra +3, Riflessi +3, Volontà +3}
    \item[\textbf{Movimento:}] 6 m, scavo 1 m
    \item[\textbf{Sfida:}] 0(10 PX)\smallskip
\end{description}

\emph{\textbf{Olfatto Affinato.}} Il tasso ha +1d6 alle prove di Consapevolezza basate sull'olfatto.

\textbf{Azioni}

\emph{\textbf{Morso.} Attacco con Arma da Mischia}: +3 a colpire, portata 1 m, un bersaglio.

\emph{Colpisce:} 1 danno perforante.

\mostro{Tasso Gigante}
\begin{description}[noitemsep, topsep=0pt, parsep=0pt, partopsep=0pt, leftmargin=0cm, labelwidth=2.2cm]
    \item[\textbf{Taglia/Tipo:}] Media bestia, disallineato
    \item[\textbf{Caratt.:}] \resizebox{0.5\linewidth+1.8cm}{!}{For 1 Des 0 Cos 2 Int -4 Sag 1 Car -3}
    \item[\textbf{Punti Ferita:}] 19,  \textbf{Difesa:} 12,  \textbf{Iniziativa:} +0
    \item[\textbf{Tiri Salvez.:}] \resizebox{0.5\linewidth+1.8cm}{!}{Tempra +3, Riflessi +3, Volontà +3}
    \item[\textbf{Movimento:}] 9 m, scavo 3 m
    \item[\textbf{Sfida:}] 1/4 (50 PX)\smallskip
\end{description}

\emph{\textbf{Olfatto Affinato.}} Il tasso ha +1d6 alle prove di Consapevolezza basate sull'olfatto.

\textbf{Azioni}

\emph{\textbf{Multiattacco.}} Il tasso effettua due attacchi: uno con il morso e uno con gli artigli.

\emph{\textbf{Artigli.} Attacco con Arma da Mischia}: +3 a colpire, portata 1 m, un bersaglio.

\emph{Colpisce:} 6 (2d4 + 1) danni taglienti.

\emph{\textbf{Morso.} Attacco con Arma da Mischia}: +3 a colpire, portata 1 m, un bersaglio.

\emph{Colpisce:} 4 (1d6 + 1) danni perforanti.

\mostro{Tigre}
\begin{description}[noitemsep, topsep=0pt, parsep=0pt, partopsep=0pt, leftmargin=0cm, labelwidth=2.2cm]
    \item[\textbf{Taglia/Tipo:}] Grande bestia, disallineato
    \item[\textbf{Caratt.:}] \resizebox{0.5\linewidth+1.8cm}{!}{For 3 Des 2 Cos 2 Int -4 Sag 1 Car -1}
    \item[\textbf{Punti Ferita:}] 33,  \textbf{Difesa:} 15,  \textbf{Iniziativa:} +2
    \item[\textbf{Tiri Salvez.:}] \resizebox{0.5\linewidth+1.8cm}{!}{Tempra +3, Riflessi +3, Volontà +3}
    \item[\textbf{Movimento:}] 12 m
    \item[\textbf{Sfida:}] 1 (200 PX)\smallskip
\end{description}

\emph{\textbf{Balzo.}} Se la tigre si muove di almeno 6 metri diretta verso una creatura e la colpisce con un attacco di artiglio durante lo stesso round, il bersaglio deve riuscire un Tiro Salvezza di Tempra DC 13 o cadere prono. Se il bersaglio è prono, la tigre può effettuare un attacco di morso contro di esso come Azione Immediata.

\emph{\textbf{Olfatto Affinato.}} La tigre ha +1d6 alle prove di Consapevolezza basate sull'olfatto.

\textbf{Azioni}

\emph{\textbf{Artiglio.} Attacco con Arma da Mischia}: +5 a colpire, portata 1 m, un bersaglio.

\emph{Colpisce:} 7 (1d8 + 3) danni taglienti, 1 danno da Sanguinamento.

\emph{\textbf{Morso.} Attacco con Arma da Mischia}: +5 a colpire, portata 1 m, un bersaglio.

\emph{Colpisce:} 8 (1d10 + 3) danni perforanti.

\mostro{Tigre dai Denti a Sciabola}\label{tigrilla}\hypertarget{Smilodonte}{}\hypertarget{Trigrilla}{}\index[Mostruario]{Trigrilla}
\begin{description}[noitemsep, topsep=0pt, parsep=0pt, partopsep=0pt, leftmargin=0cm, labelwidth=2.2cm]
    \item[\textbf{Taglia/Tipo:}] Grande bestia, disallineato
    \item[\textbf{Caratt.:}] \resizebox{0.5\linewidth+1.8cm}{!}{For 4 Des 2 Cos 2 Int -3 Sag 1 Car 0}
    \item[\textbf{Punti Ferita:}] 51,  \textbf{Difesa:} 16,  \textbf{Iniziativa:} +2
    \item[\textbf{Tiri Salvez.:}] \resizebox{0.5\linewidth+1.8cm}{!}{Tempra +4, Riflessi +4, Volontà +3}
    \item[\textbf{Movimento:}] 12 m
    \item[\textbf{Sfida:}] 2 (450 PX)\smallskip
\end{description}

\emph{\textbf{Balzo.}} Se la tigre si muove di almeno 6 metri diretta verso una creatura e la colpisce con un attacco di artiglio durante lo stesso round, il bersaglio deve riuscire un Tiro Salvezza di Tempra DC 16 o cadere prono. Se il bersaglio è prono, la tigre può effettuare un attacco di morso contro di esso come Azione Immediata.

\emph{\textbf{Olfatto Affinato.}} La tigre ha +1d6 alle prove di Consapevolezza basate sull'olfatto.

\textbf{Azioni}

\emph{\textbf{Artiglio.} Attacco con Arma da Mischia}: +6 a colpire, portata 1 m, un bersaglio.

\emph{Colpisce:} 12 (2d6 + 5) danni taglienti, 1 danno da Sanguinamento.

\emph{\textbf{Morso.} Attacco con Arma da Mischia}: +6 a colpire, portata 1 m, un bersaglio.

\emph{Colpisce:} 10 (1d10 + 5) danni perforanti.

\mostro{Vespa Gigante}
\begin{description}[noitemsep, topsep=0pt, parsep=0pt, partopsep=0pt, leftmargin=0cm, labelwidth=2.2cm]
    \item[\textbf{Taglia/Tipo:}] Media bestia, disallineato
    \item[\textbf{Caratt.:}] \resizebox{0.5\linewidth+1.8cm}{!}{For 0 Des 2 Cos 0 Int -5 Sag 0 Car -4}
    \item[\textbf{Punti Ferita:}] 24,  \textbf{Difesa:} 14,  \textbf{Iniziativa:} +2
    \item[\textbf{Tiri Salvez.:}] \resizebox{0.5\linewidth+1.8cm}{!}{Tempra +3, Riflessi +3, Volontà +3}
    \item[\textbf{Movimento:}] 3 m, volo 15 m
    \item[\textbf{Sfida:}] 1/2 (100 PX)\smallskip
\end{description}

\textbf{Azioni}

\emph{\textbf{Pungiglione.} Attacco con Arma da Mischia}: +4 a colpire, portata 1 m, una creatura.

\emph{Colpisce:} 5 (1d6 + 2) danni perforanti e il bersaglio deve effettuare un Tiro Salvezza di Tempra DC 11, e subire 10 (3d6) danni da veleno se fallisce il Tiro Salvezza, o la metà di questi danni se lo riesce. Se il danno da veleno riduce il bersaglio a 0 Punti Ferita, il bersaglio è stabile ma avvelenato per 1 ora, anche dopo aver recuperato i Punti Ferita, e mentre è avvelenato in questo modo resta paralizzato.

\mostro{Worg}
\begin{description}[noitemsep, topsep=0pt, parsep=0pt, partopsep=0pt, leftmargin=0cm, labelwidth=2.2cm]
    \item[\textbf{Taglia/Tipo:}] Grande mostruosità, malvagio
    \item[\textbf{Caratt.:}] \resizebox{0.5\linewidth+1.8cm}{!}{For 3 Des 1 Cos 1 Int -2 Sag 0 Car -1}
    \item[\textbf{Punti Ferita:}] 24,  \textbf{Difesa:} 13,  \textbf{Iniziativa:} +1
    \item[\textbf{Tiri Salvez.:}] \resizebox{0.5\linewidth+1.8cm}{!}{Tempra +3, Riflessi +3, Volontà +3}
    \item[\textbf{Movimento:}] 15 m
    \item[\textbf{Sfida:}] 1/2 (100 PX)\smallskip
\end{description}

\emph{\textbf{Udito e Olfatto Affinato.}} Il worg ha +1d6 nelle prove di Consapevolezza basate su udito o olfatto.

\textbf{Azioni}

\emph{\textbf{Morso.} Attacco con Arma da Mischia}: +5 a colpire, portata 1 m, un bersaglio.

\emph{Colpisce:} 10 (2d6 + 3) danni perforanti. Se il bersaglio è una creatura, deve riuscire un Tiro Salvezza di Tempra DC 13 o cadere prona.

\subsection{Appendice B: Personaggi Non Giocanti}\noindent\rule{\linewidth}{2pt} \index[Mostruario]{Personaggi Non Giocanti}

Questa appendice contiene le statistiche di vari personaggi non giocanti (PNG) umanoidi che gli avventurieri possono incontrare nel corso di una campagna, da infimi popolani a potenti arcimaghi. Queste statistiche possono essere utilizzate per rappresentare PNG umani e non.

Personalizzare i PNG

Esistono molti semplici modi di personalizzare i PNG di questa appendice per l'uso nella tua campagna casalinga.

\emph{\textbf{Cambiare Incantesimi.}} Un modo per personalizzare un PNG incantatore è quello di rimpiazzare uno o più dei suoi incantesimi. Puoi sostituire qualsiasi incantesimo della lista di
incantesimi del PNG con un diverso incantesimo dello stesso livello. Cambiare incantesimi in questo modo non modifica il grado di sfida del PNG.

\textbf{\emph{Cambiare Armi e Armatura}.} Puoi migliorare o peggiorare l'armatura del PNG o aggiungere o cambiare armi. Le modifiche alla Difesa e ai danni possono modificare il grado di sfida del PNG.

\emph{\textbf{Oggetti Magici}}. più potente è un PNG, maggiori le probabilità che possieda uno o più oggetti magici. Un mago, ad esempio, potrebbe avere una bacchetta o un bastone magico, oltre ad una o più pozioni e pergamene. Fornire un PNG di un potente oggetto magico capace di infliggere danni potrebbe modificarne il grado di sfida.

Alcuni oggetti magici di esempio sono descritti più avanti in questo documento.

\textbf{Combattenti}

I combattenti sono individui che si guadagnano da vivere mettendo la loro spada al servizio di un individuo o un ideale.

\mostro{Guardia}
\begin{description}[noitemsep, topsep=0pt, parsep=0pt, partopsep=0pt, leftmargin=0cm, labelwidth=2.2cm]
    \item[\textbf{Taglia/Tipo:}] Media umanoide, qualsiasi Tratto
    \item[\textbf{Caratt.:}] \resizebox{0.5\linewidth+1.8cm}{!}{For 1 Des 1 Cos 1 Int 0 Sag 0 Car 0}
    \item[\textbf{Punti Ferita:}] 24,  \textbf{Difesa:} 13,  \textbf{Iniziativa:} +1
    \item[\textbf{Tiri Salvez.:}] \resizebox{0.5\linewidth+1.8cm}{!}{Tempra +3, Riflessi +3, Volontà +3}
    \item[\textbf{Movimento:}] 9 m
    \item[\textbf{Sfida:}] 1/2 (100 PX)\smallskip
\end{description}

Le guardie comprendono membri della ronda cittadina, sentinelle di una cittadella o città fortificata e le guardie del corpo di nobili e mercanti.

\textbf{Azioni}

\emph{\textbf{Lancia.} Attacco con Arma da Mischia o a Gittata}: +3 a colpire, portata 1 m o gittata 6m, un bersaglio.

\emph{Colpisce:} 4 (1d6 + 1) danni perforanti o 5 (1d8 + 1) danni perforanti se impiegata con due mani per effettuare un attacco da mischia.

\mostro{Veterano}
\begin{description}[noitemsep, topsep=0pt, parsep=0pt, partopsep=0pt, leftmargin=0cm, labelwidth=2.2cm]
    \item[\textbf{Taglia/Tipo:}] Media umanoide, qualsiasi Tratto
    \item[\textbf{Caratt.:}] \resizebox{0.5\linewidth+1.8cm}{!}{For 3 Des 1 Cos 2 Int 0 Sag 0 Car 0}
    \item[\textbf{Punti Ferita:}] 70,  \textbf{Difesa:} 17,  \textbf{Iniziativa:} +1
    \item[\textbf{Comp.:}] Atletica +5
    \item[\textbf{Tiri Salvez.:}] \resizebox{0.5\linewidth+1.8cm}{!}{Tempra +5, Riflessi +4, Volontà +3}
    \item[\textbf{Movimento:}] 9 m
    \item[\textbf{Linguaggi:}] Comune
    \item[\textbf{Sfida:}] 3 (700 PX)\smallskip
\end{description}

Guerrieri sopravvissuti a lungo, guadagnandosi una grande fama di esperti e abili combattenti.

\textbf{Azioni}

\emph{\textbf{Multiattacco.}} Il veterano effettua due attacchi con la spada lunga. Se ha estratto una spada corta, può effettuare anche un attacco con la spada corta.

\emph{\textbf{Spada Lunga.} Attacco con Arma da Mischia}: +5 a colpire, portata 1 m, un bersaglio.

\emph{Colpisce:} 7 (1d8 + 3) danni taglienti, o 8 (1d10 + 3) danni taglienti se usata con due mani.

\emph{\textbf{Spada Corta.} Attacco con Arma da Mischia}: +5 a colpire, portata 1 m, un bersaglio.

\emph{Colpisce:} 6 (1d6 + 3) danni perforanti.

\emph{\textbf{Balestra Pesante.} Attacco con Arma a Gittata}: +3 a colpire, gittata 30m, un bersaglio.

\emph{Colpisce:} 6 (1d10 + 1) danni perforanti.

\mostro{Cavaliere}
\begin{description}[noitemsep, topsep=0pt, parsep=0pt, partopsep=0pt, leftmargin=0cm, labelwidth=2.2cm]
    \item[\textbf{Taglia/Tipo:}] Media umanoide, qualsiasi Tratto
    \item[\textbf{Caratt.:}] \resizebox{0.5\linewidth+1.8cm}{!}{For 3 Des 0 Cos 2 Int 0 Sag 0 Car 2}
    \item[\textbf{Punti Ferita:}] 70,  \textbf{Difesa:} 16,  \textbf{Iniziativa:} +0
    \item[\textbf{Tiri Salvez.:}] \resizebox{0.5\linewidth+1.8cm}{!}{Tempra +5, Riflessi +3, Volontà +3}
    \item[\textbf{Movimento:}] 9 m
    \item[\textbf{Sfida:}] 3 (700 PX)\smallskip
\end{description}

I cavalieri sono combattenti che giurano fedeltà a sovrani, ordini religiosi, e nobili cause. I Tratti del cavaliere determinano fino a che punto è disposto ad onorare il suo giuramento.

\emph{\textbf{Coraggioso.}} Il cavaliere ha +1d6 ai Tiri Salvezza contro l'essere spaventato.

\textbf{Azioni}

\emph{\textbf{Multiattacco.}} Il cavaliere effettua due attacchi da mischia.

\emph{\textbf{Spada Grossa.} Attacco con Arma da Mischia}: +5 a colpire, portata 1 m, un bersaglio.

\emph{Colpisce:} 10 (2d6 + 3) danni taglienti.

\emph{\textbf{Balestra Pesante.} Attacco con Arma a Gittata}: +2 a colpire, gittata 30m, un bersaglio.

\emph{Colpisce:} 5 (1d10) perforanti.

\emph{\textbf{Autorità (Ricarica dopo un 1 ora)}}. Per 1 minuto, il cavaliere può pronunciare un comando speciale o avvertimento ogni qualvolta una creatura non ostile entro 9 metri da lui, e che possa vedere, effettua un tiro di attacco o Tiro Salvezza. La creatura può sommare un d4 al suo tiro purché possa udire e comprendere il cavaliere. Una creatura può beneficiare di un solo dado Autorità alla volta. Questo effetto termina se il cavaliere è inabile.

\textbf{Reazioni}

\emph{\textbf{Parata.}} Il cavaliere può aggiungere 2 alla sua Difesa contro un attacco da mischia che lo colpirebbe. Per farlo, il cavaliere deve vedere l'attaccante e star impugnando un'arma da mischia.

%\begin{center}
%	\includegraphics[width=0.3\textwidth]{immagini/Knight_Death_and_the_Devil_MET_DP159047.png}

%	\emph{Cavaliere, Morte e Diavolo. Albrecht Durer}

%\end{center}

\noindent{\large\textbf{Cittadini}}

\noindent\rule{\linewidth}{2pt} \index[Mostruario]{Cittadini}\hypertarget{Cittadini}{}

In questa categoria rientrano quegli individui che si occupano di mandare avanti il mondo, svolgendo le mansioni necessarie affinché i campi vengano coltivati, le città amministrate, il cibo coltivato e
nuovi territori esplorati.

\mostro{Nobile}
\begin{description}[noitemsep, topsep=0pt, parsep=0pt, partopsep=0pt, leftmargin=0cm, labelwidth=2.2cm]
    \item[\textbf{Taglia/Tipo:}] Media umanoide, qualsiasi Tratto
    \item[\textbf{Caratt.:}] \resizebox{0.5\linewidth+1.8cm}{!}{For 0 Des 1 Cos 0 Int 1 Sag 2 Car 3}
    \item[\textbf{Punti Ferita:}] 17,  \textbf{Difesa:} 13,  \textbf{Iniziativa:} +1
    \item[\textbf{Comp.:}] Percepire Emozioni +4, Ingannare +5
    \item[\textbf{Tiri Salvez.:}] \resizebox{0.5\linewidth+1.8cm}{!}{Tempra +3, Riflessi +3, Volontà +3}
    \item[\textbf{Movimento:}] 9 m
    \item[\textbf{Linguaggi:}] due lingue qualsiasi
    \item[\textbf{Sfida:}] 1/8 (25 PX)\smallskip
\end{description}

I nobili comandano sulla popolazione, in virtù di un diritto di nascita o per le ricchezze accumulate. Tra costoro si annoverano anche i cortigiani che affollano le corti dei ricchi e dei potenti.

\textbf{Azioni}

\emph{\textbf{Stocco.} Attacco con Arma da Mischia}: +3 a colpire, portata 1 m, un bersaglio.

\emph{Colpisce:} 5 (1d8 + 1) danni perforanti.

\textbf{Reazioni}

\emph{\textbf{Parata.}} Il nobile somma 2 alla sua Difesa contro un attacco da mischia che lo colpirebbe. Per farlo, il nobile deve vedere l'attaccante e impugnare un'arma da mischia.

\mostro{Popolano}
\begin{description}[noitemsep, topsep=0pt, parsep=0pt, partopsep=0pt, leftmargin=0cm, labelwidth=2.2cm]
    \item[\textbf{Taglia/Tipo:}] Media umanoide, qualsiasi Tratto
    \item[\textbf{Caratt.:}] \resizebox{0.5\linewidth+1.8cm}{!}{For 0 Des 0 Cos 0 Int 0 Sag 0 Car 0}
    \item[\textbf{Punti Ferita:}] 17,  \textbf{Difesa:} 12,  \textbf{Iniziativa:} +0
    \item[\textbf{Tiri Salvez.:}] \resizebox{0.5\linewidth+1.8cm}{!}{Tempra +3, Riflessi +3, Volontà +3}
    \item[\textbf{Movimento:}] 9 m
    \item[\textbf{Linguaggi:}] Comune
    \item[\textbf{Sfida:}] 1/8 (25 PX)\smallskip
\end{description}

I popolani comprendono contadini, servi, schiavi, servitori, pellegrini, mercanti, artigiani ed eremiti.

\textbf{Azioni}

\emph{\textbf{Randello.} Attacco con Arma da Mischia}: +2 a colpire, portata 1 m, un bersaglio.

\emph{Colpisce:} 2 (1d4) danni contundenti.

\medskip\textbf{Criminali}

I criminali sono individui che vivono al margine della legalità, procurandosi il pane svolgendo attività spesso considerate illecite e immorali.

\mostro{Bandito/Pirata}
\begin{description}[noitemsep, topsep=0pt, parsep=0pt, partopsep=0pt, leftmargin=0cm, labelwidth=2.2cm]
    \item[\textbf{Taglia/Tipo:}] Media umanoide, qualsiasi Tratto
    \item[\textbf{Caratt.:}] \resizebox{0.5\linewidth+1.8cm}{!}{For 0 Des 1 Cos 1 Int 0 Sag 0 Car 0}
    \item[\textbf{Punti Ferita:}] 17,  \textbf{Difesa:} 13,  \textbf{Iniziativa:} +1
    \item[\textbf{Tiri Salvez.:}] \resizebox{0.5\linewidth+1.8cm}{!}{Tempra +3, Riflessi +3, Volontà +3}
    \item[\textbf{Movimento:}] 9 m
    \item[\textbf{Linguaggi:}] Comune
    \item[\textbf{Sfida:}] 1/8 (25 PX)\smallskip
\end{description}

Che siano uomini di strada o di mare (pirati) costoro guadagnano da vivere depredando il prossimo.

\textbf{Azioni}

\emph{\textbf{Scimitarra.} Attacco con Arma da Mischia}: +3 a colpire, portata 1 m, un bersaglio.

\emph{Colpisce:} 4 (1d6 + 1) danni taglienti.

\emph{\textbf{Balestra Leggera.} Attacco con Arma a Gittata}: +3 a colpire, gittata 24m, un bersaglio.

\emph{Colpisce:} 5 (1d8 + 1) danni taglienti.

\mostro{Spia}
\begin{description}[noitemsep, topsep=0pt, parsep=0pt, partopsep=0pt, leftmargin=0cm, labelwidth=2.2cm]
    \item[\textbf{Taglia/Tipo:}] Media umanoide, qualsiasi Tratto
    \item[\textbf{Caratt.:}] \resizebox{0.5\linewidth+1.8cm}{!}{For 0 Des 2 Cos 0 Int 1 Sag 2 Car 3}
    \item[\textbf{Punti Ferita:}] 33,  \textbf{Difesa:} 15,  \textbf{Iniziativa:} +2
    \item[\textbf{Tiri Salvez.:}] \resizebox{0.5\linewidth+1.8cm}{!}{Tempra +3, Riflessi +3, Volontà +3}
    \item[\textbf{Movimento:}] 9 m
    \item[\textbf{Linguaggi:}] due lingue qualsiasi
    \item[\textbf{Sfida:}] 1 (200 PX)\smallskip
\end{description}

Una spia è un individuo addestramento nel reperire segreti per conto di qualcuno, o a volte per rivenderli al miglior offerente.

\emph{\textbf{Attacco Furtivo (1/Turno).}} La spia infligge 7 (2d6) danni aggiuntivi quando colpisce un bersaglio con un attacco con arma e ha +1d6 al tiro di attacco, o quando il bersaglio è entro 1 metro da un alleato dell'assassino che non è inabile e l'assassino non ha -1d6 al tiro di attacco.

\emph{\textbf{Azione Astuta.}} Durante ciascun suo round, la spia può usare una Azione Immediata per effettuare l'azione Disingaggiare, nascondersi o Scattare.

\textbf{Azioni}

\emph{\textbf{Multiattacco.}} La spia effettua due attacchi da mischia.

\emph{\textbf{Spada Corta.} Attacco con Arma da Mischia}: +4 a colpire, portata 1 m, un bersaglio.

\emph{Colpisce:} 5 (1d6 + 2) danni perforanti.

\emph{\textbf{Balestrino.} Attacco con Arma a Gittata}: +4 a colpire, gittata 9m, un bersaglio.

\emph{Colpisce:} 5 (1d6 + 2) danni perforanti.

\mostro{Capitano dei Banditi o Pirata}
\begin{description}[noitemsep, topsep=0pt, parsep=0pt, partopsep=0pt, leftmargin=0cm, labelwidth=2.2cm]
    \item[\textbf{Taglia/Tipo:}] Media umanoide, qualsiasi Tratto
    \item[\textbf{Caratt.:}] \resizebox{0.5\linewidth+1.8cm}{!}{For 2 Des 3 Cos 2 Int 2 Sag 0 Car 2}
    \item[\textbf{Punti Ferita:}] 51,  \textbf{Difesa:} 17,  \textbf{Iniziativa:} +3
    \item[\textbf{Comp.:}] Atletica +4, Ingannare +4
    \item[\textbf{Tiri Salvez.:}] \resizebox{0.5\linewidth+1.8cm}{!}{Tempra +4, Riflessi +5, Volontà +3}
    \item[\textbf{Movimento:}] 9 m
    \item[\textbf{Linguaggi:}]  due lingue qualsiasi
    \item[\textbf{Sfida:}] 2 (450 PX)\smallskip
\end{description}

Che viva in terra o in mare, è un individuo munito di una grande personalità che riesce a tenere in riga la marmaglia che risponde ai suoi ordini.

\textbf{Azioni}

\emph{\textbf{Multiattacco.}} Il capitano effettua tre attacchi da mischia: due con la scimitarra e uno con il pugnale. Oppure il capitano effettua due attacchi a gittata con i pugnali.

\emph{\textbf{Scimitarra.} Attacco con Arma da Mischia}: +5 a colpire, portata 1 m, un bersaglio.

\emph{Colpisce:} 6 (1d6 + 3) danni taglienti.

\emph{\textbf{Pugnale.} Attacco con Arma da Mischia o a Gittata}: +5 a colpire, portata 1 m o gittata 6m, un bersaglio.

\emph{Colpisce:} 5 (1d4 + 3) danni perforanti.

\textbf{Reazioni}

\emph{\textbf{Parata.}} Il capitano somma 2 alla sua Difesa contro un attacco da mischia che lo colpirebbe. Per farlo, il capitano deve vedere l'attaccante e impugnare un'arma da mischia.

\mostro{Assassino}
\begin{description}[noitemsep, topsep=0pt, parsep=0pt, partopsep=0pt, leftmargin=0cm, labelwidth=2.2cm]
    \item[\textbf{Taglia/Tipo:}] Media umanoide, qualsiasi Tratto
    \item[\textbf{Caratt.:}] \resizebox{0.5\linewidth+1.8cm}{!}{For 0 Des 3 Cos 2 Int 1 Sag 0 Car 0}
    \item[\textbf{Punti Ferita:}] 162,  \textbf{Difesa:} 25,  \textbf{Iniziativa:} +3
    \item[\textbf{Comp.:}] Acrobazia +6, Furtività +9, Ingannare +3
    \item[\textbf{Tiri Salvez.:}] \resizebox{0.5\linewidth+1.8cm}{!}{Tempra +10, Riflessi +11, Volontà +8}
    \item[\textbf{Movimento:}] 9 m
    \item[\textbf{Linguaggi:}] Comune più due altre lingue
    \item[\textbf{Sfida:}] 8 (3900 PX)\smallskip
\end{description}

Solitari o membri di una gilda, gli assassini sono pagati per eliminare, spesso in modo silenzioso e discreto, rivali e nemici dei loro datori di lavoro.

\emph{\textbf{Assassinare.}} Durante il suo primo round, l'assassino ha +1d6 ai tiri di attacco contro le creature che non hanno ancora svolto nessun round. Qualsiasi colpo che l'assassino mandi a segno contro una creatura sorpresa, è un colpo critico.

\emph{\textbf{Attacco Furtivo (1/Turno).}} L'assassino infligge 14 (4d6) danni aggiuntivi quando colpisce un bersaglio con un attacco con arma e ha +1d6 al tiro di attacco, o quando il bersaglio è entro 1 metro da un alleato dell'assassino che non è inabile e l'assassino non ha -1d6 al tiro di attacco.

\emph{\textbf{Evasione.}} Se l'assassino è vittima di un effetto che permette di effettuare un Tiro Salvezza di Riflessi per dimezzare i danni, l'assassino non prende danni se riesce il Tiro Salvezza, e solo la metà se lo fallisce.

\textbf{Azioni}

\emph{\textbf{Multiattacco.}} L'assassino effettua due attacchi con le spade corte.

\emph{\textbf{Spada Corta.} Attacco con Arma da Mischia}: +6 a colpire, portata 1 m, un bersaglio.

\emph{Colpisce:} 6 (1d6 + 3) danni perforanti, e il bersaglio deve effettuare un Tiro Salvezza di Tempra DC 19, subendo 24 (7d6) danni da veleno se fallisce il Tiro Salvezza, o la metà di questi danni se lo riesce.

\emph{\textbf{Balestra Leggera.} Attacco con Arma a Gittata}: +6 a colpire, gittata 24m, un bersaglio.

\emph{Colpisce:} 7 (1d8 + 3) danni perforanti, e il bersaglio deve effettuare un Tiro Salvezza di Tempra DC 19, subendo 24 (7d6) danni da veleno se fallisce il Tiro Salvezza, o la metà di questi danni se lo riesce.

\textbf{Reazione: \emph{Attacco d'opportunità}}: l'assassino effettua un attacco con spada corta ad una creatura che attraversi o esca dalla sua portata di 1 metro.

\medskip\textbf{Mago}

Il mago trascorre la vita nello studio e la pratica della magia.

\mostro{Mago Avventuriero}
\begin{description}[noitemsep, topsep=0pt, parsep=0pt, partopsep=0pt, leftmargin=0cm, labelwidth=2.2cm]
    \item[\textbf{Taglia/Tipo:}] Media umanoide, qualsiasi Tratto
    \item[\textbf{Caratt.:}] \resizebox{0.5\linewidth+1.8cm}{!}{For -1 Des 2 Cos 0 Int 2 Sag 1 Car 0}
    \item[\textbf{Punti Ferita:}] 33,  \textbf{Difesa:} 15,  \textbf{Iniziativa:} +3
    \item[\textbf{Comp.:}] Arcana +5, Storia +5
    \item[\textbf{Tiri Salvez.:}] \resizebox{0.5\linewidth+1.8cm}{!}{Tempra +3, Riflessi +3, Volontà +3}
    \item[\textbf{Movimento:}] 9 m
    \item[\textbf{Linguaggi:}] quattro lingue qualsiasi
    \item[\textbf{Sfida:}] 1 (200 PX)\smallskip
\end{description}

Un Mago novizio, che ha superato con successo le sue prime avventure e ha iniziato a stabilire una reputazione come nobile o famigerato avventuriero.

\emph{\textbf{Incantesimi.}} Il mago ha CM 4. La sua abilità da incantatore è l'Intelligenza (+4 al colpire con attacchi con incantesimo). Il Mago ha preparato i seguenti incantesimi:

Trucchetti (a volontà): \emph{\hyperlink{Luce}{Luce}, \hyperlink{Mano Magica}{Mano Magica}, \hyperlink{Stretta Folgorante}{Stretta Folgorante}}

livello 1 (4 slot): \emph{\hyperlink{Charme su Persone}{Charme su Persone},  \hyperlink{Dardo arcano}{Dardo arcano}}

livello 2 (3 slot): \emph{\hyperlink{Blocca Persona}{Blocca Persona}, \hyperlink{Passo Velato}{Passo Velato}}

\textbf{Azioni}

\emph{\textbf{Bastone.} Attacco con Arma da Mischia}: +1 a colpire, portata 1 m, un bersaglio.

\emph{Colpisce:} 3 (1d8 - 1) danni contundenti.

\mostro{Grande Mago}
\begin{description}[noitemsep, topsep=0pt, parsep=0pt, partopsep=0pt, leftmargin=0cm, labelwidth=2.2cm]
    \item[\textbf{Taglia/Tipo:}] Media umanoide, qualsiasi Tratto
    \item[\textbf{Caratt.:}] \resizebox{0.5\linewidth+1.8cm}{!}{For -1 Des 2 Cos 0 Int 3 Sag 1 Car 0}
    \item[\textbf{Punti Ferita:}] 122,  \textbf{Difesa:} 22,  \textbf{Iniziativa:} +3
    \item[\textbf{Comp.:}] Arcana +6, Storia +6
    \item[\textbf{Tiri Salvez.:}] \resizebox{0.5\linewidth+1.8cm}{!}{Tempra +6, Riflessi +8, Volontà +7}
    \item[\textbf{Movimento:}] 9 m
    \item[\textbf{Linguaggi:}] quattro lingue qualsiasi
    \item[\textbf{Sfida:}] 6 (2300 PX)\smallskip
\end{description}

Un Mago che ha stabilito una discreta fama nel territorio e che attira intorno a sé studenti da ogni dove.

\emph{\textbf{Incantesimi.}} Il mago ha CM 9. La sua abilità da incantatore è l'Intelligenza (+8 al colpire con attacchi con incantesimo). Il Mago ha preparato i seguenti incantesimi:

Trucchetti (a volontà): \emph{\hyperlink{Dardo di Fuoco}{Dardo di Fuoco}, \hyperlink{Luce}{Luce}, \hyperlink{Mano Magica}{Mano Magica}, \hyperlink{Prestidigitazione}{Prestidigitazione}}

livello 1 (4 slot): \emph{\hyperlink{Armatura Magica}{Armatura Magica}, \hyperlink{Dardo arcano}{Dardo arcano}, \hyperlink{Individuazione del Magico}{Individuazione del Magico}, \hyperlink{Scudo}{Scudo}}

livello 2 (3 slot): \emph{\hyperlink{Passo Velato}{Passo Velato}, \hyperlink{Suggestione}{Suggestione}}

livello 3 (3 slot): \emph{\hyperlink{Controincantesimo}{Controincantesimo}, \hyperlink{Palla di Fuoco}{Palla di Fuoco}, volare}

livello 4 (3 slot): \emph{\hyperlink{Invisibilità Superiore}{Invisibilità Superiore}, \hyperlink{Tempesta di Ghiaccio}{Tempesta di Ghiaccio}}

livello 5 (1 slot): \emph{\hyperlink{Cono di Freddo}{Cono di Freddo}}

\textbf{Azioni}

\emph{\textbf{Pugnale.} Attacco con Arma da Mischia o a Gittata}: +5 a colpire, portata 1 m o gittata 6m, un bersaglio.

\emph{Colpisce:} 4 (1d4 + 2) danni perforanti.

\textbf{Reazione: \emph{Incantesimo opportunistico}}: il mago reagisce ad un attacco subito lanciando un trucchetto.

\mostro{Arcimago}
\begin{description}[noitemsep, topsep=0pt, parsep=0pt, partopsep=0pt, leftmargin=0cm, labelwidth=2.2cm]
    \item[\textbf{Taglia/Tipo:}] Media umanoide, qualsiasi Tratto
    \item[\textbf{Caratt.:}] \resizebox{0.5\linewidth+1.8cm}{!}{For 0 Des 2 Cos 1 Int 5 Sag 2 Car 3}
    \item[\textbf{Punti Ferita:}] 233,  \textbf{Difesa:} 30,  \textbf{Iniziativa:} +5
    \item[\textbf{Comp.:}] Arcana +13, Storia +13
    \item[\textbf{Tiri Salvez.:}] \resizebox{0.5\linewidth+1.8cm}{!}{\resizebox{0.5\linewidth+1.8cm}{!}{Tempra +13, Riflessi +14, Volontà +14}}
    \item[\textbf{Movimento:}] 9 m
    \item[\textbf{Linguaggi:}] sei lingue qualsiasi
    \item[\textbf{Sfida:}] 12 (8400 PX)\smallskip
\end{description}

Un mago molto potente (e anche molto anziano) che studia i segreti del multiverso.

\emph{\textbf{Incantesimi.}} Il mago ha CM 18. La sua abilità da incantatore è l'Intelligenza (+15 al colpire con attacchi con incantesimo).

L'arcimago può eseguire \emph{\hyperlink{Camuffare Sé Stesso}{Camuffare Sé Stesso}} e \emph{\hyperlink{Invisibilità}{Invisibilità}} a volontà e ha preparato i seguenti incantesimi:

Trucchetti (a volontà): \emph{\hyperlink{Dardo di Fuoco}{Dardo di Fuoco}, \hyperlink{Luce}{Luce}, \hyperlink{Mano Magica}{Mano Magica}. \hyperlink{Prestidigitazione}{Prestidigitazione}, \hyperlink{Stretta Folgorante}{Stretta Folgorante}}

livello 1 (4 slot): \emph{\hyperlink{Armatura Magica}{Armatura Magica}*, \hyperlink{Dardo arcano}{Dardo arcano}, \hyperlink{Identificare}{Identificare}, \hyperlink{Individuazione del Magico}{Individuazione del Magico}}

livello 2 (3 slot): \emph{\hyperlink{Immagine Speculare}{Immagine Speculare}, \hyperlink{Individuazione dei Pensieri}{Individuazione dei Pensieri}, \hyperlink{Passo Velato}{Passo Velato}}

livello 3 (3 slot): \emph{\hyperlink{Controincantesimo}{Controincantesimo}, \hyperlink{Fulmine}{Fulmine}}

livello 4 (3 slot): \emph{\hyperlink{Esilio}{Esilio}, \hyperlink{Pelle di Pietra}{Pelle di Pietra}*, \hyperlink{Scudo di Fuoco}{Scudo di Fuoco}}

livello 5 (3 slot): \emph{\hyperlink{Cono di Freddo}{Cono di Freddo}, \hyperlink{Muro di Forza}{Muro di Forza}, \hyperlink{Scrutare}{Scrutare}}

livello 6 (1 slot): \emph{\hyperlink{Globo di Invulnerabilità}{Globo di Invulnerabilità}}

livello 7 (1 slot): \emph{\hyperlink{Teletrasporto}{Teletrasporto}}

livello 8 (1 slot): \emph{\hyperlink{Scudo Mentale}{Scudo Mentale}*}

livello 9 (1 slot): \emph{\hyperlink{Fermare il Tempo}{Fermare il Tempo}}

L'arcimago esegue questi {*} incantesimi su di sé prima del combattimento.

\textbf{Azioni}

\emph{\textbf{Pugnale.} Attacco con Arma da Mischia o a Gittata}: +6 a colpire, portata 1 m o gittata 6m, un bersaglio.

\emph{Colpisce:} 4 (1d4 + 2) danni perforanti.

\textbf{Reazione: \emph{Incantesimo opportunistico}}: il mago reagisce ad un attacco subito lanciando un incantesimo di 2 livello o meno.

\medskip\textbf{Sacerdoti}

I sacerdoti sono devoti di una divinità o una fede che si prendono cura di impartire gli insegnamenti divini al loro gregge.

\mostro{Cultista}
\begin{description}[noitemsep, topsep=0pt, parsep=0pt, partopsep=0pt, leftmargin=0cm, labelwidth=2.2cm]
    \item[\textbf{Taglia/Tipo:}] Media umanoide, qualsiasi Tratto
    \item[\textbf{Caratt.:}] \resizebox{0.5\linewidth+1.8cm}{!}{For 0 Des 1 Cos 0 Int 0 Sag 0 Car 0}
    \item[\textbf{Punti Ferita:}] 17,  \textbf{Difesa:} 13,  \textbf{Iniziativa:} +1
    \item[\textbf{Comp.:}] Ingannare +2, Religione +2
    \item[\textbf{Tiri Salvez.:}] \resizebox{0.5\linewidth+1.8cm}{!}{Tempra +3, Riflessi +3, Volontà +3}
    \item[\textbf{Movimento:}] 9 m
    \item[\textbf{Linguaggi:}] Comune
    \item[\textbf{Sfida:}] 1/8 (25 PX)\smallskip
\end{description}

I cultisti giurano fedeltà ai poteri oscuri, e nelle loro credenze e pratiche mostrano spesso segni di follia.

\emph{\textbf{Oscura Devozione.}} Il cultista ha +1d6 sui Tiri Salvezza contro l'essere affascinato o spaventato.

\textbf{Azioni}

\emph{\textbf{Scimitarra.} Attacco con Arma da Mischia}: +3 a colpire, portata 1 m, una creatura.

\emph{Colpisce:} 4 (1d6 + 1) danni taglienti.

\mostro{Accolito}
\begin{description}[noitemsep, topsep=0pt, parsep=0pt, partopsep=0pt, leftmargin=0cm, labelwidth=2.2cm]
    \item[\textbf{Taglia/Tipo:}] Media umanoide, qualsiasi Tratto
    \item[\textbf{Caratt.:}] \resizebox{0.5\linewidth+1.8cm}{!}{For 0 Des 0 Cos 0 Int 0 Sag 2 Car 0}
    \item[\textbf{Punti Ferita:}] 19,  \textbf{Difesa:} 12,  \textbf{Iniziativa:} +0
    \item[\textbf{Comp.:}] Pronto Soccorso +4, Religione +2
    \item[\textbf{Tiri Salvez.:}] \resizebox{0.5\linewidth+1.8cm}{!}{Tempra +3, Riflessi +3, Volontà +3}
    \item[\textbf{Movimento:}] 9 m
    \item[\textbf{Linguaggi:}] Comune
    \item[\textbf{Sfida:}] 1/4 (50 PX)\smallskip
\end{description}

Gli accoliti sono membri di grado minore del clero, e di solito rispondono ad un sacerdote di rango superiore. Svolgono diverse funzioni in un tempio e gli viene conferita dalla loro divinità l'abilità di eseguire incantesimi minori.

\emph{\textbf{Incantesimi.}} L'accolito ha CM 1. La sua abilità da incantatore è la Saggezza (+4 al colpire con attacchi con incantesimo). L'accolito ha preparato i seguenti incantesimi:

Trucchetti (a volontà): \emph{\hyperlink{Fiamma Sacra}{Fiamma Sacra}, \hyperlink{Luce}{Luce}, \hyperlink{Taumaturgia}{Taumaturgia}}

livello 1 (3 slot): \emph{\hyperlink{Benedizione}{Benedizione}, \hyperlink{Cura Ferite}{Cura Ferite}, \hyperlink{Santuario}{Santuario}}

\medskip\textbf{Azioni}

\emph{\textbf{Randello.} Attacco con Arma da Mischia}: +2 a colpire, portata 1 m, un bersaglio.

\emph{Colpisce:} 2 (1d4) danni contundenti.

\mostro{Cultista capo}
\begin{description}[noitemsep, topsep=0pt, parsep=0pt, partopsep=0pt, leftmargin=0cm, labelwidth=2.2cm]
    \item[\textbf{Taglia/Tipo:}] Media umanoide, qualsiasi Tratto
    \item[\textbf{Caratt.:}] \resizebox{0.5\linewidth+1.8cm}{!}{For 0 Des 2 Cos 1 Int 0 Sag 1 Car 2}
    \item[\textbf{Punti Ferita:}] 33,  \textbf{Difesa:} 15,  \textbf{Iniziativa:} +2
    \item[\textbf{Comp.:}] Ingannare +4, Religione +2
    \item[\textbf{Tiri Salvez.:}] \resizebox{0.5\linewidth+1.8cm}{!}{Tempra +3, Riflessi +3, Volontà +3}
    \item[\textbf{Movimento:}] 9 m
    \item[\textbf{Linguaggi:}] Comune ed un altra lingua
    \item[\textbf{Sfida:}] 1 (200 PX)\smallskip
\end{description}

Sono i capi di un culto, che usano il proprio carisma e i propri dogmi per influenzare i deboli di volontà.

\emph{\textbf{Incantesimi.}} Il sacerdote ha CM 4. La sua abilità da incantatore è la Saggezza (+3 al colpire con attacchi con incantesimo). Il sacerdote ha preparato i seguenti incantesimi:

Trucchetti (a volontà): \emph{\hyperlink{Fiamma Sacra}{Fiamma Sacra}, \hyperlink{Luce}{Luce}, \hyperlink{Taumaturgia}{Taumaturgia}}

livello 1 (4 slot): \emph{\hyperlink{Comando}{Comando}, \hyperlink{Infliggi Ferite}{Infliggi Ferite}}

livello 2 (3 slot): \emph{\hyperlink{Arma Spirituale}{Arma Spirituale}, \hyperlink{Blocca Persona}{Blocca Persona}}

\emph{\textbf{Oscura Devozione.}} Il cultista ha +1d6 sui Tiri Salvezza contro l'essere affascinato o spaventato.

\textbf{Azioni}

\emph{\textbf{Multiattacco.}} Il fanatico effettua due attacchi da mischia.

\emph{\textbf{Pugnale.} Attacco con Arma da Mischia o a Gittata}: +4 a colpire, portata 1 m o gittata 6m, una creatura.

\emph{Colpisce:} 4 (1d4 + 2) danni perforanti.

\mostro{Gran Sacerdote}
\begin{description}[noitemsep, topsep=0pt, parsep=0pt, partopsep=0pt, leftmargin=0cm, labelwidth=2.2cm]
    \item[\textbf{Taglia/Tipo:}] Media umanoide, qualsiasi Tratto
    \item[\textbf{Caratt.:}] \resizebox{0.5\linewidth+1.8cm}{!}{For 0 Des 0 Cos 1 Int 1 Sag 3 Car 1}
    \item[\textbf{Punti Ferita:}] 51,  \textbf{Difesa:} 14,  \textbf{Iniziativa:} +1
    \item[\textbf{Tiri Salvez.:}] \resizebox{0.5\linewidth+1.8cm}{!}{Tempra +3, Riflessi +3, Volontà +5}
    \item[\textbf{Movimento:}] 7 m
    \item[\textbf{Linguaggi:}] due lingue qualsiasi
    \item[\textbf{Sfida:}] 2 (450 PX)\smallskip
\end{description}

Sono individui al comando di un tempio o altro luogo sacro e che hanno a loro disposizione diversi accoliti.

\emph{\textbf{Eminenza Divina.}} Come Azione Immediata, il sacerdote può spendere uno slot incantesimo per far sì che il suo attacco con arma da mischia infligge 10 (3d6) danni da Luce aggiuntivi. Il beneficio dura fino al termine del round.

\emph{\textbf{Incantesimi.}} Il sacerdote ha CM 6. La sua abilità da incantatore è la Saggezza (+5 al colpire con attacchi con incantesimo). Il sacerdote ha preparato i seguenti incantesimi:

Trucchetti (a volontà): \emph{\hyperlink{Fiamma Sacra}{Fiamma Sacra}, \hyperlink{Luce}{Luce}, \hyperlink{Taumaturgia}{Taumaturgia}}

livello 1 (4 slot): \emph{\hyperlink{Cura Ferite}{Cura Ferite}, \hyperlink{Dardo Tracciante}{Dardo Tracciante}, \hyperlink{Santuario}{Santuario}}

livello 2 (3 slot): \emph{\hyperlink{Arma Spirituale}{Arma Spirituale}, \hyperlink{Ristorare Inferiore}{Ristorare Inferiore}}

livello 3 (2 slot): \emph{\hyperlink{Dissolvi Magie}{Dissolvi Magie}}

\textbf{Azioni}

\emph{\textbf{Mazza.} Attacco con Arma da Mischia}: +4 a colpire, portata 1 m, un bersaglio.

\emph{Colpisce:} 3 (1d6) danni contundenti.

\medskip\textbf{Selvaggi}

Questi individui vivono ai margini della civiltà, a volte entrandovi raramente in contatto. A disagio tra le mura e nelle terre civilizzate, si trovano nel loro ambiente quando possono muoversi tra le terre selvagge.

\mostro{Berserker}
\begin{description}[noitemsep, topsep=0pt, parsep=0pt, partopsep=0pt, leftmargin=0cm, labelwidth=2.2cm]
    \item[\textbf{Taglia/Tipo:}] Media umanoide, qualsiasi Tratto
    \item[\textbf{Caratt.:}] \resizebox{0.5\linewidth+1.8cm}{!}{For 3 Des 1 Cos 3 Int -1 Sag 0 Car -1}
    \item[\textbf{Punti Ferita:}] 52,  \textbf{Difesa:} 15,  \textbf{Iniziativa:} +1
    \item[\textbf{Tiri Salvez.:}] \resizebox{0.5\linewidth+1.8cm}{!}{Tempra +5, Riflessi +3, Volontà +3}
    \item[\textbf{Movimento:}] 9 m
    \item[\textbf{Linguaggi:}] Comune
    \item[\textbf{Sfida:}] 2 (450 PX)\smallskip
\end{description}

Provenienti da terre selvagge, gli imprevedibili berserker si radunano in compagnie di guerra e sono sempre alla ricerca di conflitti in cui combattere.

\emph{\textbf{Incauto.}} All'inizio del suo round, il berserker può ottenere +1d6 su tutti i tiri di attacco con armi da mischia effettuati durante quel round, ma i tiri di attacco contro di esso hanno +1d6 fino all'inizio del suo prossimo round.

\textbf{Azioni}

\emph{\textbf{Ascia Grossa.} Attacco con Arma da Mischia}: +5 a colpire, portata 1 m, un bersaglio.

\emph{Colpisce:} 9 (1d12 + 3) danni taglienti.

\mostro{Combattente Tribale}
\begin{description}[noitemsep, topsep=0pt, parsep=0pt, partopsep=0pt, leftmargin=0cm, labelwidth=2.2cm]
    \item[\textbf{Taglia/Tipo:}] Media umanoide, qualsiasi Tratto
    \item[\textbf{Caratt.:}] \resizebox{0.5\linewidth+1.8cm}{!}{For 1 Des 0 Cos 1 Int -1 Sag 0 Car -1}
    \item[\textbf{Punti Ferita:}] 17,  \textbf{Difesa:} 12,  \textbf{Iniziativa:} +0
    \item[\textbf{Tiri Salvez.:}] \resizebox{0.5\linewidth+1.8cm}{!}{Tempra +3, Riflessi +3, Volontà +3}
    \item[\textbf{Movimento:}] 9 m
    \item[\textbf{Linguaggi:}] Comune
    \item[\textbf{Sfida:}] 1/8 (25 PX)\smallskip
\end{description}

Sono i difensori delle tribù che vivono ai margini della civiltà.

\emph{\textbf{Tattiche di Branco.}} Il combattente tribale ha +1d6 ai tiri di attacco contro una creatura se almeno uno degli alleati del picchiatore si trova entro 1 metro dalla creatura e quell'alleato non è inabile.

\textbf{Azioni}

\emph{\textbf{Lancia.} Attacco con Arma da Mischia o a Gittata}: +3 a colpire, portata 1 m o gittata 6m, un bersaglio.

\emph{Colpisce:} 4 (1d6 + 1) danni perforanti

\mostro{Druido}
\begin{description}[noitemsep, topsep=0pt, parsep=0pt, partopsep=0pt, leftmargin=0cm, labelwidth=2.2cm]
    \item[\textbf{Taglia/Tipo:}] Media umanoide, qualsiasi Tratto
    \item[\textbf{Caratt.:}] \resizebox{0.5\linewidth+1.8cm}{!}{For 0 Des 1 Cos 1 Int 1 Sag 2 Car 0}
    \item[\textbf{Punti Ferita:}] 51,  \textbf{Difesa:} 15,  \textbf{Iniziativa:} +1
    \item[\textbf{Comp.:}] Pronto Soccorso +4, Natura +3, Consapevolezza +4
    \item[\textbf{Tiri Salvez.:}] \resizebox{0.5\linewidth+1.8cm}{!}{Tempra +3, Riflessi +3, Volontà +4}
    \item[\textbf{Movimento:}] 9 m
    \item[\textbf{Linguaggi:}] Druidico più due altre lingue
    \item[\textbf{Sfida:}] 2 (450 PX)\smallskip
\end{description}

I druidi proteggono il mondo naturale dai mostri e dall'avanzare della civiltà. Alcuni sono sciamani tribali che curano i malati, pregano agli spiriti animali e forniscono consigli spirituali. Solitamente sono devoti di Efrem o Shayalia.

\emph{\textbf{Incantesimi.}} Il Druido ha CM 4. La sua abilità da incantatore è la Saggezza (+4 al colpire con attacchi con incantesimo). Il Druido ha preparato i seguenti incantesimi:

Trucchetti (a volontà): \emph{\hyperlink{Artificio Druidico}{Artificio Druidico}, \hyperlink{Randello Incantato}{Randello Incantato}, \hyperlink{Produrre Fiamma}{Produrre Fiamma}}

livello 1 (4 slot): \emph{\hyperlink{Intralciare}{Intralciare}, \hyperlink{Onda Tonante}{Onda Tonante}, \hyperlink{Parlare con gli Animali}{Parlare con gli Animali}, \hyperlink{Passo Veloce}{Passo Veloce}}

livello 2 (3 slot): \emph{\hyperlink{Animale Messaggero}{Animale Messaggero}, \hyperlink{Pelle di Corteccia}{Pelle di Corteccia}}

\textbf{Azioni}

\emph{\textbf{Bastone da Combattimento.} Attacco con Arma da Mischia}: +3 a colpire (+5 a colpire con \emph{\hyperlink{Randello Incantato}{Randello Incantato}}), portata 1 m o gittata 6m, un bersaglio.

\emph{Colpisce:} 3 (1d6) danni contundenti, o 6 (1d8 + 2) danni contundenti con \emph{\hyperlink{Randello Incantato}{Randello Incantato}} o se impugnato con due mani.

\mostro{Esploratore}
\begin{description}[noitemsep, topsep=0pt, parsep=0pt, partopsep=0pt, leftmargin=0cm, labelwidth=2.2cm]
    \item[\textbf{Taglia/Tipo:}] Media umanoide, qualsiasi Tratto
    \item[\textbf{Caratt.:}] \resizebox{0.5\linewidth+1.8cm}{!}{For 0 Des 2 Cos 1 Int 0 Sag 1 Car 0}
    \item[\textbf{Punti Ferita:}] 24,  \textbf{Difesa:} 14,  \textbf{Iniziativa:} +2
    \item[\textbf{Comp.:}] Furtività +6, Natura +4, Consapevolezza +5, Sopravvivenza +5
    \item[\textbf{Tiri Salvez.:}] \resizebox{0.5\linewidth+1.8cm}{!}{Tempra +3, Riflessi +3, Volontà +3}
    \item[\textbf{Movimento:}] 9 m
    \item[\textbf{Linguaggi:}] Comune
    \item[\textbf{Sfida:}] 1/2 (100 PX)\smallskip
\end{description}

Abili cacciatori e battitori di piste.

\emph{\textbf{Olfatto e Vista Affinati.}} L'esploratore ha +1d6 nelle prove di Consapevolezza basate su olfatto o vista.

\textbf{Azioni}

\emph{\textbf{Multiattacco.}} L'esploratore effettua due attacchi da mischia o due attacchi a gittata.

\emph{\textbf{Spada Corta.} Attacco con Arma da Mischia}: +4 a colpire, portata 1 m, un bersaglio.

\emph{Colpisce:} 5 (1d6 + 2) danni perforanti.

\emph{\textbf{Arco Lungo.} Attacco con Arma da Mischia}: +4 a colpire, gittata 45m, un bersaglio.

\emph{Colpisce:} 6 (1d8 + 2) danni perforanti.

}  %chiude \setlength{\parietlength{\parindent}{0cm}{

%} % chiude \small{

\rule{\linewidth}{2pt}

\subsection{Conversioni dalla 5e}\index{Convertire i mostri dalla 5e}

Per convertire i mostri del famoso Gioco di Ruolo:

\begin{itemize}[leftmargin=*] \setlength{\itemsep}{0pt}
\item \textbf{Difesa} uguale a 12+GS+(GS/3)+Destrezza $\pm4$\
\item \textbf{Tiri Salvezza} da GS+Caratteristica $\pm2$\
\item \textbf{Punti Ferita} da (GS+1+(GS/5))*15 + COS*(GS/5) $\pm GS*2$\
\item \textbf{Tiro per Colpire} da 3-18 (oppure GS*0.7+Forza / Destrezza $\pm3$\ )
\item \textbf{DC per Tiri Salvezza} da abilità da 10+GS+(GS/5) + Caratteristica correlata  $\pm2$\
\item \textbf{Iniziativa} uguale a Destrezza o Intelligenza  $\pm2$\
\item \textbf{Danno x Round} circa 5*GS
\end{itemize}

Arrotondate il GS per eccesso nei calcoli.

\end{multicols}

\vfill

\begin{enfasi}

Timeo Danaos et dona ferentes (Temo i Greci anche quando portano doni). (Eneide, Virgilio)

\medskip

Non è morto ciò che può giacere in eterno, e in strani eoni anche la morte può morire. (Il Richiamo di Cthulhu, H.P. Lovecraft)

\end{enfasi}

\pagebreak

\subsection{Template e suggerimenti per i Mostri}

\begin{multicols}{2}

Sono qui proposti dei suggerimenti su come differenziare i mostri e renderli più specifici alla situazione. Un mostro con uno di questi template prenderà alcuni tratti d'aspetto, di capacità e di gioco tipici del template stesso. \\

\textbf{Spettro}

Il template Spettro conferisce alla creatura un aspetto \emph{da spettro}

\textbf{Punti Ferita}: aumenta di 4 per GS

\textbf{Difesa}: aumenta di 4

\textbf{Sensi}: Scurovisione 18 m

\textbf{Resistenza ai danni}: Acido, Freddo, Fuoco, Elettricità, Suono

\textbf{Immunità}: affascinato, spaventato, affaticato, afferrato, paralizzato, pietrificato, veleno, prono, ristretto

\textbf{Movimento Incorporeo}. La creatura può muoversi attraverso creature ed oggetti come se fosse terreno difficile. Subisce 5 (1d10) danni se termina il suo turno
all’interno di un oggetto.

\textbf{Sensibilità alla luce solare}. Mentre è illuminato a luce solare la creatura ha -1d6 ai Tiri per Colpire e alle prove di Consapevolezza.

\textbf{Attacco}: +1 al Tiro per Colpire

\textbf{Danno}: il danno inferto dalla creatura diventa da Vuoto\\

\textbf{Maledetto}

Il template Maledetto conferisce alla creatura un aspetto \emph{corrotto ed oscuro}

\textbf{Punti Ferita}: aumenta di 6 per GS

\textbf{Difesa}: aumenta di 2

\textbf{Resistenza ai danni}: Vuoto, Elettricità

\textbf{Immunità}: affascinato, spaventato

\textbf{Sensi}: Visione Crepuscolare 18 m

\textbf{Attacco}: +2 al Tiro per Colpire

\textbf{Danno}: +1d4 danno da Vuoto\\

\textbf{Bestia Selvaggia}

Il template Bestia Selvaggia conferisce alla creatura un aspetto \emph{più grosso ed aggressivo}

\textbf{Punti Ferita}: aumenta di 6 per GS

\textbf{Difesa}: aumenta di 3

\textbf{Movimento}: +3 m

\textbf{Attacco}: +3 al Tiro per Colpire

\textbf{Danno}: +1d6 danni\\

\textbf{Demoniaco}

Il template Demoniaco conferisce alla creatura un aspetto \emph{da demone}

\textbf{Punti Ferita}: aumenta di 8 per GS

\textbf{Difesa}: aumenta di 6

\textbf{Movimento}: +3 m

\textbf{Sensi}: Scurovisione 18 m

\textbf{Resistenza ai danni}: Freddo, da armi non magiche

\textbf{Resistenza alla Magia}: +2 ai Tiri Salvezza contro Incantesimi

\textbf{Scurovisione}: 18 m

\textbf{Attacco}: +4 al Tiro per Colpire

\textbf{Danno}: +1d6 Sanguinamento\\

\textbf{Scheletro}

Il template Scheletro conferisce alla creatura un aspetto \emph{scheletrico}

\textbf{Punti Ferita}: aumenta di 4 per GS

\textbf{Difesa}: aumenta di 1

\textbf{Sensi}: Scurovisione 18 m

\textbf{Resistenza ai danni}:  perforante, tagliente, Elettricità, Fuoco

\textbf{Immunità ai danni}: Veleno

\textbf{Immunità}: affaticato, sanguinamento

\textbf{Vulnerabilità}: danni contundenti

\textbf{Attacco}: +1 al Tiro per Colpire

\textbf{Danno}: +2\\

\textbf{Infuso di Magia}

Il template Infuso di Magia conferisce alla creatura un aspetto \emph{brillante}

\textbf{Punti Ferita}: aumenta di 4 per GS

\textbf{Difesa}: aumenta di 3

\textbf{Sensi}: Scurovisione 18 m

\textbf{Resistenza ai danni}:  Elettricità, Fuoco, Luce, Vuoto

\textbf{Resistenza alla Magia}: +1d6 ai Tiri Salvezza contro Incantesimi

\textbf{Attacco}: +4 al Tiro per Colpire contro creature con incantesimi attivi o con CM >1

\textbf{Danno}: al posto di fare danno la creatura può provare un \hyperlink{contrastareincantesimi}{contrastare} incantesimi con DC pari al GS+Intelligenza per annullare un incantesimo attivo sull'avversario\\

\textbf{Oozekin}

Il template Oozekin conferisce alla creatura un aspetto \emph{gelatinoso} e fluido

\textbf{Punti Ferita}: aumenta di 6 per GS

\textbf{Difesa}: aumenta di 4

\textbf{Resistenza ai danni}:  perforante, tagliente

\textbf{Movimento}: -1 m

\textbf{Sensi}: Vista Cieca 18 m (cieco oltre questo raggio)

\textbf{Immunità}: accecato, affascinato, assordato, prono

\textbf{Imm. Danni}: Acido, Elettricità, tagliente, da critico

\textbf{Resistenza alla Magia}: +1 ai Tiri Salvezza contro Incantesimi

\textbf{Attacco}: +3

\textbf{Danno}: +1d8 da Acido

\end{multicols}

%\bigskip

%\begin{center}
%\begin{tikzpicture}
%\draw[thick] (0cm, 3cm) rectangle ++(16cm,20cm); % Sposta la cornice di 0 cm a destra e 3 cm in alto
%\node[anchor=north west, inner sep=5pt] at (0.3cm, 23cm) {\textbf{Appunti:}}; % Aggiunge il testo in alto a sinistra della cornice
%\end{tikzpicture}
%\end{center}

%{\scriptsize
%\printindex}
%\end{document}

\pagebreak

\subsection{Lista Mostri per Grado di Sfida}

\begin{multicols}{3}
{\small
\noindent\hyperlink{Aquila}{Aquila}, GS 0 (10 PX)\\
\hyperlink{Avvoltoio}{Avvoltoio}, GS 0 (10 PX)\\
\hyperlink{Babbuino}{Babbuino}, GS 0 (10 PX)\\
\hyperlink{Caprone}{Caprone}, GS 0 (10 PX)\\
\hyperlink{Cervo}{Cervo}, GS 0 (10 PX)\\
\hyperlink{Corvo}{Corvo}, GS 0 (10 PX)\\
%\hyperlink{Donnola}{Donnola}, GS 0 (10 PX)\\
\hyperlink{Falco}{Falco}, GS 0 (10 PX)\\
\hyperlink{Fungo Stridente}{Fungo Stridente}, GS 0 (10 PX)\\
\hyperlink{Gatto}{Gatto}, GS 0 (10 PX)\\
\hyperlink{Gufo}{Gufo}, GS 0 (10 PX)\\
\hyperlink{Iena}{Iena}, GS 0 (10 PX)\\
\hyperlink{Lemure}{Lemure}, GS 0 (10 PX)\\
%\hyperlink{Lucertola}{Lucertola}, GS 0 (10 PX)\\
\hyperlink{Omuncolo}{Omuncolo}, GS 0 (10 PX)\\
\hyperlink{Pirana}{Pirana}, GS 0 (10 PX)\\
\hyperlink{Popolano}{Popolano}, GS 0 (10 PX)\\
\hyperlink{Ragno}{Ragno}, GS 0 (10 PX)\\
\hyperlink{Rana}{Rana}, GS 0 (10 PX)\\
\hyperlink{Ratto}{Ratto}, GS 0 (10 PX)\\
\hyperlink{Scarabeo di Fuoco Gigante}{Scarabeo di Fuoco Gigante}, GS 0 (10 PX)\\
\hyperlink{Sciacallo}{Sciacallo}, GS 0 (10 PX)\\
\hyperlink{Scorpione}{Scorpione}, GS 0 (10 PX)\\
\hyperlink{Tasso}{Tasso}, GS 0 (10 PX)\\
\hyperlink{Topi, La}{Topi, La}, GS 0 (10 PX)\\
\hyperlink{Bandito/Pirata}{Bandito/Pirata}, GS 1/8 (25 PX)\\
\hyperlink{Coboldo}{Coboldo}, GS 1/8 (25 PX)\\
\hyperlink{Cultista}{Cultista}, GS 1/8 (25 PX)\\
\hyperlink{Donnola Gigante}{Donnola Gigante}, GS 1/8 (25 PX)\\
\hyperlink{Falco di Sangue}{Falco di Sangue}, GS 1/8 (25 PX)\\
\hyperlink{Granchio Gigante}{Granchio Gigante}, GS 1/8 (25 PX)\\
\hyperlink{Guardia}{Guardia}, GS 1/8 (25 PX)\\
\hyperlink{Mastino}{Mastino}, GS 1/8 (25 PX)\\
%\hyperlink{Mulo}{Mulo}, GS 1/8 (25 PX)\\andres
\hyperlink{Nobile}{Nobile}, GS 1/8 (25 PX)\\
\hyperlink{Saurovallo nano}{Saurovallo nano}, GS 1/8 (25 PX)\\
\hyperlink{Ratto Gigante}{Ratto Gigante}, GS 1/8 (25 PX)\\
\hyperlink{Serpente Velenoso}{Serpente Velenoso}, GS 1/8 (25 PX)\\
\hyperlink{Serpente Volante}{Serpente Volante}, GS 1/8 (25 PX)\\
\hyperlink{Strige}{Strige}, GS 1/8 (25 PX)\\
\hyperlink{Strige (Uccello Stigeo)}{Strige (Uccello Stigeo)}, GS 1/8 (25 PX)\\
\hyperlink{Uomo Acquatico}{Uomo Acquatico}, GS 1/8 (25 PX)\\
\hyperlink{Accolito}{Accolito}, GS 1/4 (50 PX)\\
\hyperlink{Alce}{Alce}, GS 1/4 (50 PX)\\
\hyperlink{Becco d'Ascia}{Becco d'Ascia}, GS 1/4 (50 PX)\\
\hyperlink{Cane Intermittente}{Cane Intermittente}, GS 1/4 (50 PX)\\
\hyperlink{Saurovallo da Galoppo}{Saurovallo da Galoppo}, GS 1/4 (50 PX)\\
\hyperlink{Saurovallo da Tiro}{Saurovallo da Tiro}, GS 1/4 (50 PX)\\
%\hyperlink{Centopiedi Gigante}{Centopiedi Gigante}, GS 1/4 (50 PX)\\
\hyperlink{Cinghiale}{Cinghiale}, GS 1/4 (50 PX)\\
\hyperlink{Dretch}{Dretch}, GS 1/4 (50 PX)\\
\hyperlink{Fungo Violetto}{Fungo Violetto}, GS 1/4 (50 PX)\\
\hyperlink{Gablin}{Gablin}, GS 1/4 (50 PX)\\
\hyperlink{Grimlock}{Grimlock}, GS 1/4 (50 PX)\\
\hyperlink{Gufo Gigante}{Gufo Gigante}, GS 1/4 (50 PX)\\
\hyperlink{Lucertola Gigante}{Lucertola Gigante}, GS 1/4 (50 PX)\\
\hyperlink{Lupo}{Lupo}, GS 1/4 (50 PX)\\
\hyperlink{Mefito di Vapore}{Mefito di Vapore}, GS 1/4 (50 PX)\\
\hyperlink{Pantera}{Pantera}, GS 1/4 (50 PX)\\
\hyperlink{Pseudodrago}{Pseudodrago}, GS 1/4 (50 PX)\\
\hyperlink{Ragno Lupo Gigante}{Ragno Lupo Gigante}, GS 1/4 (50 PX)\\
\hyperlink{Rana Gigante}{Rana Gigante}, GS 1/4 (50 PX)\\
\hyperlink{Scheletro}{Scheletro}, GS 1/4 (50 PX)\\
\hyperlink{Sciame di Corvi}{Sciame di Corvi}, GS 1/4 (50 PX)\\
\hyperlink{Sciame di Pipistrelli}{Sciame di Pipistrelli}, GS 1/4 (50 PX)\\
\hyperlink{Sciame di Ratti}{Sciame di Ratti}, GS 1/4 (50 PX)\\
\hyperlink{Serpente Costrittore}{Serpente Costrittore}, GS 1/4 (50 PX)\\
\hyperlink{Serpente Velenoso Gigante}{Serpente Velenoso Gigante}, GS 1/4 (50 PX)\\
\hyperlink{Spada Volante}{Spada Volante}, GS 1/4 (50 PX)\\
\hyperlink{Spiritello}{Spiritello}, GS 1/4 (50 PX)\\
\hyperlink{Tasso Gigante}{Tasso Gigante}, GS 1/4 (50 PX)\\
\hyperlink{Zombi}{Zombi}, GS 1/4 (50 PX)\\
\hyperlink{Caprone Gigante}{Caprone Gigante}, GS 1/2 (100 PX)\\
\hyperlink{Saurovallo da Guerra}{Saurovallo da Guerra}, GS 1/2 (100 PX)\\
\hyperlink{Cavallo Marino Gigante}{Cavallo Marino Gigante}, GS 1/2 (100 PX)\\
\hyperlink{Coccodrillo}{Coccodrillo}, GS 1/2 (100 PX)\\
\hyperlink{Cockatrice}{Cockatrice}, GS 1/2 (100 PX)\\
\hyperlink{Esploratore}{Esploratore}, GS 1/2 (100 PX)\\
\hyperlink{Gnoll}{Gnoll}, GS 1/2 (100 PX)\\
\hyperlink{Gnomo delle Profondità}{Gnomo delle Profondità}, GS 1/2 (100 PX)\\
\hyperlink{Hobgoblin}{Hobgoblin}, GS 1/2 (100 PX)\\
\hyperlink{Lucertoloide}{Lucertoloide}, GS 1/2 (100 PX)\\
\hyperlink{Mantoscuro}{Mantoscuro}, GS 1/2 (100 PX)\\
\hyperlink{Mefito di Ghiaccio}{Mefito di Ghiaccio}, GS 1/2 (100 PX)\\
\hyperlink{Mefito di Magma}{Mefito di Magma}, GS 1/2 (100 PX)\\
\hyperlink{Mefito di Polvere}{Mefito di Polvere}, GS 1/2 (100 PX)\\
\hyperlink{Melma Grigia}{Melma Grigia}, GS 1/2 (100 PX)\\
\hyperlink{Monete affamate}{Monete affamate}, GS 1/2 (100 PX)\\
\hyperlink{Ombra}{Ombra}, GS 1/2 (100 PX)\\
\hyperlink{Orchetto}{Orchetto}, GS 1/2 (100 PX)\\
\hyperlink{Orso Nero}{Orso Nero}, GS 1/2 (100 PX)\\
\hyperlink{Rugginofago}{Rugginofago}, GS 1/2 (100 PX)\\
\hyperlink{Sahuagin}{Sahuagin}, GS 1/2 (100 PX)\\
\hyperlink{Satiro}{Satiro}, GS 1/2 (100 PX)\\
\hyperlink{Scheletro di Saurovallo da Guerra}{Scheletro di Saurovallo da Guerra}, GS 1/2 (100 PX)\\
\hyperlink{Sciame di Insetti}{Sciame di Insetti}, GS 1/2 (100 PX)\\
\hyperlink{Sciame di Ragni}{Sciame di Ragni}, GS 1/2 (100 PX)\\
%\hyperlink{Sciame di Scarabei}{Sciame di Scarabei}, GS 1/2 (100 PX)\\
\hyperlink{Sciame di Vespe}{Sciame di Vespe}, GS 1/2 (100 PX)\\
%\hyperlink{Sciami}{Sciami}, GS 1/2 (100 PX)\\
\hyperlink{Scimmione}{Scimmione}, GS 1/2 (100 PX)\\
\hyperlink{Squalo Corallino}{Squalo Corallino}, GS 1/2 (100 PX)\\
\hyperlink{Uomo Magma}{Uomo Magma}, GS 1/2 (100 PX)\\
\hyperlink{Vespa Gigante}{Vespa Gigante}, GS 1/2 (100 PX)\\
\hyperlink{Worg}{Worg}, GS 1/2 (100 PX)\\
\hyperlink{Aquila Gigante}{Aquila Gigante}, GS 1 (200 PX)\\
\hyperlink{Armatura Animata}{Armatura Animata}, GS 1 (200 PX)\\
\hyperlink{Arpia}{Arpia}, GS 1 (200 PX)\\
\hyperlink{Avvoltoio Gigante}{Avvoltoio Gigante}, GS 1 (200 PX)\\
\hyperlink{Bugbear}{Bugbear}, GS 1 (200 PX)\\
\hyperlink{Cane della Morte}{Cane della Morte}, GS 1 (200 PX)\\
\hyperlink{Dinolupo (Metalupo)}{Dinolupo (Metalupo)}, GS 1 (200 PX)\\
\hyperlink{Drago di Ottone Cucciolo}{Drago di Ottone Cucciolo}, GS 1 (200 PX)\\
\hyperlink{Drago di Rame Cucciolo}{Drago di Rame Cucciolo}, GS 1 (200 PX)\\
\hyperlink{Driade}{Driade}, GS 1 (200 PX)\\
\hyperlink{Ghoul}{Ghoul}, GS 1 (200 PX)\\
\hyperlink{Globulo}{Globulo}, GS 1 (200 PX)\\
\hyperlink{Iena Gigante}{Iena Gigante}, GS 1 (200 PX)\\
\hyperlink{Imp}{Imp}, GS 1 (200 PX)\\
\hyperlink{Ippogrifo}{Ippogrifo}, GS 1 (200 PX)\\
\hyperlink{Leone}{Leone}, GS 1 (200 PX)\\
\hyperlink{Mago Avventuriero}{Mago Avventuriero}, GS 1 (200 PX)\\
\hyperlink{Nano Oscuro}{Nano Oscuro}, GS 1 (200 PX)\\
\hyperlink{Orco}{Orco}, GS 1 (100 PX)\\
\hyperlink{Orso Bruno}{Orso Bruno}, GS 1 (200 PX)\\
\hyperlink{Quasit}{Quasit}, GS 1 (200 PX)\\
\hyperlink{Ragno Gigante}{Ragno Gigante}, GS 1 (200 PX)\\
\hyperlink{Rospo Gigante}{Rospo Gigante}, GS 1 (200 PX)\\
\hyperlink{Sciame di Pirana}{Sciame di Pirana}, GS 1 (200 PX)\\
\hyperlink{Spettro}{Spettro}, GS 1 (200 PX)\\
\hyperlink{Spia}{Spia}, GS 1 (200 PX)\\
\hyperlink{Tigre}{Tigre}, GS 1 (200 PX)\\
\hyperlink{Albero Risvegliato}{Albero Risvegliato}, GS 2 (450 PX)\\
\hyperlink{Alce Gigante}{Alce Gigante}, GS 2 (450 PX)\\
\hyperlink{Ameba Paglierina}{Ameba Paglierina}, GS 2 (450 PX)\\
\hyperlink{Ankheg}{Ankheg}, GS 2 (450 PX)\\
\hyperlink{Azer}{Azer}, GS 2 (450 PX)\\
\hyperlink{Berserker}{Berserker}, GS 2 (450 PX)\\
\hyperlink{Blatta Esplosiva}{Blatta Esplosiva}, GS 2 (450 PX)\\
\hyperlink{Capitano dei Banditi o Pirata}{Capitano dei Banditi o Pirata}, GS 2 (450 PX)\\
\hyperlink{Centauro}{Centauro}, GS 2 (450 PX)\\
\hyperlink{Cinghiale Gigante}{Cinghiale Gigante}, GS 2 (450 PX)\\
\hyperlink{Cubo Gelatinoso}{Cubo Gelatinoso}, GS 2 (450 PX)\\
\hyperlink{Diavolo Spinoso}{Diavolo Spinoso}, GS 2 (450 PX)\\
\hyperlink{Drago Bianco Cucciolo}{Drago Bianco Cucciolo}, GS 2 (450 PX)\\
\hyperlink{Drago di Argento Cucciolo}{Drago di Argento Cucciolo}, GS 2 (450 PX)\\
\hyperlink{Drago di Bronzo Cucciolo}{Drago di Bronzo Cucciolo}, GS 2 (450 PX)\\
\hyperlink{Drago Nero Cucciolo}{Drago Nero Cucciolo}, GS 2 (450 PX)\\
\hyperlink{Drago Verde Cucciolo}{Drago Verde Cucciolo}, GS 2 (450 PX)\\
\hyperlink{Druido}{Druido}, GS 2 (450 PX)\\
\hyperlink{Ettercap}{Ettercap}, GS 2 (450 PX)\\
\hyperlink{Fauci Gorgoglianti}{Fauci Gorgoglianti}, GS 2 (450 PX)\\
\hyperlink{Fuoco Fatuo}{Fuoco Fatuo}, GS 2 (450 PX)\\
\hyperlink{Gargoyle}{Gargoyle}, GS 2 (450 PX)\\
\hyperlink{Ghast}{Ghast}, GS 2 (450 PX)\\
\hyperlink{Grick}{Grick}, GS 2 (450 PX)\\
\hyperlink{Grifone}{Grifone}, GS 2 (450 PX)\\
\hyperlink{Megera Marina}{Megera Marina}, GS 2 (450 PX)\\
\hyperlink{Mimic}{Mimic}, GS 2 (450 PX)\\
\hyperlink{Ogre}{Ogre}, GS 2 (450 PX)\\
\hyperlink{Orso Polare}{Orso Polare}, GS 2 (450 PX)\\
\hyperlink{Pegaso}{Pegaso}, GS 2 (450 PX)\\
\hyperlink{Plesiosauro}{Plesiosauro}, GS 2 (450 PX)\\
\hyperlink{Ratto Mannaro}{Ratto Mannaro}, GS 2 (450 PX)\\
\hyperlink{Rinoceronte lanoso}{Rinoceronte lanoso}, GS 2 (450 PX)\\
\hyperlink{Gran Sacerdote}{Gran Sacerdote}, GS 2 (450 PX)\\
\hyperlink{Sciame di Serpenti Velenosi}{Sciame di Serpenti Velenosi}, GS 2 (450 PX)\\
\hyperlink{Serpente Costrittore Gigante}{Serpente Costrittore Gigante}, GS 2 (450 PX)\\
\hyperlink{Sibilante}{Sibilante}, GS 2 (450 PX)\\
\hyperlink{Silku}{Silku}, GS 2 (450 PX)\\
\hyperlink{Squalo Cacciatore}{Squalo Cacciatore}, GS 2 (450 PX)\\
\hyperlink{Tappeto del Soffocamento}{Tappeto del Soffocamento}, GS 2 (450 PX)\\
\hyperlink{Teschio Fiammeggiante}{Teschio Fiammeggiante}, GS 2 (200 PX)\\
\hyperlink{Tigre dai Denti a Sciabola}{Tigre dai Denti a Sciabola}, GS 2 (450 PX)\\
\hyperlink{Zombi Ogre}{Zombi Ogre}, GS 2 (450 PX)\\
\hyperlink{Balena Assassina (Orca)}{Balena Assassina (Orca)}, GS 3 (700 PX)\\
\hyperlink{Basilisco}{Basilisco}, GS 3 (700 PX)\\
\hyperlink{Campione Gablin}{Campione Gablin}, GS 3 (700)\\
\hyperlink{Cavaliere}{Cavaliere}, GS 3 (700 PX)\\
\hyperlink{Destriero dell'Incubo}{Destriero dell'Incubo}, GS 3 (700 PX)\\
\hyperlink{Diavolo Barbuto}{Diavolo Barbuto}, GS 3 (700 PX)\\
\hyperlink{Doppelganger}{Doppelganger}, GS 3 (700 PX)\\
\hyperlink{Drago Blu Cucciolo}{Drago Blu Cucciolo}, GS 3 (700 PX)\\
\hyperlink{Drago d'Oro Cucciolo}{Drago d'Oro Cucciolo}, GS 3 (700 PX)\\
\hyperlink{Lupo Invernale}{Lupo Invernale}, GS 3 (700 PX)\\
\hyperlink{Lupo Mannaro}{Lupo Mannaro}, GS 3 (700 PX)\\
\hyperlink{Manticora}{Manticora}, GS 3 (700 PX)\\
\hyperlink{Megera Verde}{Megera Verde}, GS 3 (700 PX)\\
\hyperlink{Minotauro}{Minotauro}, GS 3 (700 PX)\\
\hyperlink{Mummia}{Mummia}, GS 3 (700 PX)\\
\hyperlink{Orrore Arrampicamuri}{Orrore Arrampicamuri}, GS 3 (700 PX)\\
\hyperlink{Orsogufo}{Orsogufo}, GS 3 (700 PX)\\
\hyperlink{Orsogufo Saggio}{Orsogufo Saggio}, GS 3 (700 PX)\\
\hyperlink{Ragno Fase}{Ragno Fase}, GS 3 (700 PX)\\
\hyperlink{Scheletro Campione}{Scheletro Campione}, GS 3 (700 PX)\\
\hyperlink{Scorpione Gigante}{Scorpione Gigante}, GS 3 (700 PX)\\
\hyperlink{Segugio Infernale}{Segugio Infernale}, GS 3 (700 PX)\\
\hyperlink{Veterano}{Veterano}, GS 3 (700 PX)\\
\hyperlink{Wight}{Wight}, GS 3 (700 PX)\\
\hyperlink{B.O.C.}{B.O.C.}, GS 4 (1100 PX)\\
\hyperlink{Banshee}{Banshee}, GS 4 (1100 PX)\\
\hyperlink{Chuul}{Chuul}, GS 4 (1100 PX)\\
\hyperlink{Cinghiale Mannaro}{Cinghiale Mannaro}, GS 4 (1100 PX)\\
\hyperlink{Couatl}{Couatl}, GS 4 (1100 PX)\\
\hyperlink{Drago Rosso Cucciolo}{Drago Rosso Cucciolo}, GS 4 (1100 PX)\\
\hyperlink{Elefante}{Elefante}, GS 4 (1100 PX)\\
\hyperlink{Ettin}{Ettin}, GS 4 (1100 PX)\\
\hyperlink{Fantasma}{Fantasma}, GS 4 (1100 PX)\\
\hyperlink{Ghoul, putrescente}{Ghoul, putrescente}, GS 4 (1100 PX)\\
\hyperlink{Lamia}{Lamia}, GS 4 (1100 PX)\\
\hyperlink{Maledetto immortale}{Maledetto immortale}, GS 4 (1100 PX)\\
\hyperlink{Protoplasma Nero}{Protoplasma Nero}, GS 4 (1100 PX)\\
\hyperlink{Succube}{Succube}, GS 4 (1100 PX)\\
\hyperlink{Tigre Mannara}{Tigre Mannara}, GS 4 (1100 PX)\\
\hyperlink{Torciascura}{Torciascura}, GS 4 (1100 PX)\\
\hyperlink{Verme Strisciante Tentacolato}{Verme Strisciante Tentacolato}, GS 4 (1100 PX)\\
\hyperlink{Bulette}{Bulette}, GS 5 (1800 PX)\\
\hyperlink{Coccodrillo Gigante}{Coccodrillo Gigante}, GS 5 (1800 PX)\\
\hyperlink{Cumulo Strisciante}{Cumulo Strisciante}, GS 5 (1800 PX)\\
\hyperlink{Elementale del Fuoco Generico}{Elementale del Fuoco Generico}, GS 5 (1800 PX)\\
\hyperlink{Elementale dell'Acqua Generico}{Elementale dell'Acqua Generico}, GS 5 (1800 PX)\\
\hyperlink{Elementale dell'Aria Generico}{Elementale dell'Aria Generico}, GS 5 (1800 PX)\\
\hyperlink{Elementale della Terra Generico}{Elementale della Terra Generico}, GS 5 (1800 PX)\\
\hyperlink{Fustigatore}{Fustigatore}, GS 5 (1800 PX)\\
\hyperlink{Ghoul, Madre}{Ghoul, Madre}, GS 5 (1800 PX)\\
\hyperlink{Gigante delle Colline}{Gigante delle Colline}, GS 5 (1800 PX)\\
\hyperlink{Golem di Carne}{Golem di Carne}, GS 5 (1800 PX)\\
\hyperlink{Gorgone}{Gorgone}, GS 5 (1800 PX)\\
\hyperlink{Megera Notturna}{Megera Notturna}, GS 5 (1800 PX)\\
\hyperlink{Orso Mannaro}{Orso Mannaro}, GS 5 (1800 PX)\\
\hyperlink{Otyugh}{Otyugh}, GS 5 (1800 PX)\\
\hyperlink{Salamandra}{Salamandra}, GS 5 (1800 PX)\\
\hyperlink{Squalo Gigante}{Squalo Gigante}, GS 5 (1800 PX)\\
\hyperlink{Triceratopo}{Triceratopo}, GS 5 (1800 PX)\\
\hyperlink{Troll}{Troll}, GS 5 (1800 PX)\\
\hyperlink{Unicorno}{Unicorno}, GS 5 (1800 PX)\\
\hyperlink{Wraith}{Wraith}, GS 5 (1800 PX)\\
\hyperlink{Xorn}{Xorn}, GS 5 (1800 PX)\\
\hyperlink{Chimera}{Chimera}, GS 6 (2300 PX)\\
\hyperlink{Drago Bianco Giovane}{Drago Bianco Giovane}, GS 6 (2300 PX)\\
\hyperlink{Drago d'Ottone Giovane}{Drago d'Ottone Giovane}, GS 6 (2300 PX)\\
\hyperlink{Drider}{Drider}, GS 6 (2300 PX)\\
\hyperlink{Ghoul, Nero}{Ghoul, Nero}, GS 6 (2300 PX)\\
\hyperlink{Grande Mago}{Grande Mago}, GS 6 (2300 PX)\\
\hyperlink{Mammut}{Mammut}, GS 6 (2300 PX)\\
\hyperlink{Medusa}{Medusa}, GS 6 (2300 PX)\\
\hyperlink{Paladino Gablin}{Paladino Gablin}, GS 6 (2300 PX)\\
\hyperlink{Persecutore Invisibile}{Persecutore Invisibile}, GS 6 (2300 PX)\\
\hyperlink{Progenie Vampirica}{Progenie Vampirica}, GS 6 (1800 PX)\\
\hyperlink{Viverna}{Viverna}, GS 6 (2300 PX)\\
\hyperlink{Vrock}{Vrock}, GS 6 (2300 PX)\\
\hyperlink{Drago di Rame Giovane}{Drago di Rame Giovane}, GS 7 (2900 PX)\\
\hyperlink{Drago Nero Giovane}{Drago Nero Giovane}, GS 7 (2900 PX)\\
\hyperlink{Gigante di Pietra}{Gigante di Pietra}, GS 7 (2900 PX)\\
\hyperlink{Guardiano Protettore}{Guardiano Protettore}, GS 7 (2900 PX)\\
\hyperlink{Oni}{Oni}, GS 7 (2900 PX)\\
\hyperlink{Scimmione Gigante}{Scimmione Gigante}, GS 7 (2900 PX)\\
\hyperlink{Assassino}{Assassino}, GS 8 (3900 PX)\\
\hyperlink{Diavolo delle Catene}{Diavolo delle Catene}, GS 8 (3900 PX)\\
\hyperlink{Drago di Bronzo Giovane}{Drago di Bronzo Giovane}, GS 8 (3900 PX)\\
\hyperlink{Drago Verde Giovane}{Drago Verde Giovane}, GS 8 (3900 PX)\\
\hyperlink{Gigante del Gelo}{Gigante del Gelo}, GS 8 (3900 PX)\\
\hyperlink{Hezrou}{Hezrou}, GS 8 (3900 PX)\\
\hyperlink{Idra}{Idra}, GS 8 (3900 PX)\\
\hyperlink{Manto Assassino}{Manto Assassino}, GS 8 (3900 PX)\\
\hyperlink{Naga Spirituale}{Naga Spirituale}, GS 8 (3900 PX)\\
\hyperlink{Tirannosauro}{Tirannosauro}, GS 8 (3900 PX)\\
\hyperlink{Diavolo d'Ossa}{Diavolo d'Ossa}, GS 9 (5000 PX)\\
\hyperlink{Divora Cervelli}{Divora Cervelli}, GS 9 (5000 PX)\\
\hyperlink{Drago Blu Giovane}{Drago Blu Giovane}, GS 9 (5000 PX)\\
\hyperlink{Drago di Argento Giovane}{Drago di Argento Giovane}, GS 9 (5000 PX)\\
%\hyperlink{Elementale dell'Acqua Maggiore}{Elementale dell'Acqua Maggiore}, GS 9 (5000 PX)\\
\hyperlink{Gigante del Fuoco}{Gigante del Fuoco}, GS 9 (5000 PX)\\
\hyperlink{Gigante delle Nuvole}{Gigante delle Nuvole}, GS 9 (5000 PX)\\
\hyperlink{Glabrezu}{Glabrezu}, GS 9 (5000 PX)\\
\hyperlink{Golem di Argilla}{Golem di Argilla}, GS 9 (5000 PX)\\
\hyperlink{Uomo Albero (Arborom)}{Uomo Albero (Arborom)}, GS 9 (5000 PX)\\
\hyperlink{Aboleth}{Aboleth}, GS 10 (5900 PX)\\
\hyperlink{Angelo Deva}{Angelo Deva}, GS 10 (5900 PX)\\
\hyperlink{Drago d'Oro Giovane}{Drago d'Oro Giovane}, GS 10 (5900 PX)\\
\hyperlink{Drago Rosso Giovane}{Drago Rosso Giovane}, GS 10 (5900 PX)\\
\hyperlink{G.E.C.}{G.E.C.}, GS 10 (5900 PX)\\
\hyperlink{Golem di Pietra}{Golem di Pietra}, GS 10 (5900 PX)\\
\hyperlink{Naga Guardiano}{Naga Guardiano}, GS 10 (5900 PX)\\
\hyperlink{Behir}{Behir}, GS 11 (7200 PX)\\
\hyperlink{Diavolo Cornuto}{Diavolo Cornuto}, GS 11 (7200 PX)\\
\hyperlink{Djinni}{Djinni}, GS 11 (7200 PX)\\
\hyperlink{Efreeti}{Efreeti}, GS 11 (7200 PX)\\
\hyperlink{Ginosfinge}{Ginosfinge}, GS 11 (7200 PX)\\
\hyperlink{Remorhaz}{Remorhaz}, GS 11 (7200 PX)\\
\hyperlink{Arcimago}{Arcimago}, GS 12 (8400 PX)\\
\hyperlink{Erinni}{Erinni}, GS 12 (8400 PX)\\
\hyperlink{Panoptikhan}{Panoptikhan}, GS 12 (8400 PX)\\
\hyperlink{Drago Bianco Adulto}{Drago Bianco Adulto}, GS 13 (10000 PX)\\
\hyperlink{Drago d'Ottone Adulto}{Drago d'Ottone Adulto}, GS 13 (10000 PX)\\
\hyperlink{Gigante delle Tempeste}{Gigante delle Tempeste}, GS 13 (10000 PX)\\
\hyperlink{Nalfeshnee}{Nalfeshnee}, GS 13 (10000 PX)\\
\hyperlink{Rakshasa}{Rakshasa}, GS 13 (10000 PX)\\
\hyperlink{Vampiro}{Vampiro}, GS 13 (10000 PX)\\
\hyperlink{Diavolo del Ghiaccio}{Diavolo del Ghiaccio}, GS 14 (11.500 PX)\\
\hyperlink{Drago di Rame Adulto}{Drago di Rame Adulto}, GS 14 (11.500 PX)\\
\hyperlink{Drago di Bronzo Adulto}{Drago di Bronzo Adulto}, GS 15 (13000 PX)\\
\hyperlink{Drago Verde Adulto}{Drago Verde Adulto}, GS 15 (13000 PX)\\
\hyperlink{Fenice}{Fenice}, GS 15 (13000 PX)\\
\hyperlink{Mummia Sovrana}{Mummia Sovrana}, GS 15 (13000 PX)\\
\hyperlink{Verme Purpureo}{Verme Purpureo}, GS 15 (13000 PX)\\
\hyperlink{Angelo Planetar}{Angelo Planetar}, GS 16 (15000 PX)\\
\hyperlink{Drago Blu Adulto}{Drago Blu Adulto}, GS 16 (15000 PX)\\
\hyperlink{Drago di Argento Adulto}{Drago di Argento Adulto}, GS 16 (15000 PX)\\
\hyperlink{Golem di Ferro}{Golem di Ferro}, GS 16 (15000 PX)\\
\hyperlink{Marilith}{Marilith}, GS 16 (15000 PX)\\
\hyperlink{Androsfinge}{Androsfinge}, GS 17 (18000 PX)\\
\hyperlink{Drago d'Oro Adulto}{Drago d'Oro Adulto}, GS 17 (18000 PX)\\
\hyperlink{Drago Nero Adulto}{Drago Nero Adulto}, GS 17 (18000 PX)\\
\hyperlink{Drago Rosso Adulto}{Drago Rosso Adulto}, GS 17 (18000 PX)\\
\hyperlink{Testuggine Dragona}{Testuggine Dragona}, GS 17 (18000 PX)\\
\hyperlink{Cavaliere Nero}{Cavaliere Nero}, GS 18 (20000 PX)\\
\hyperlink{Balor}{Balor}, GS 19 (22000 PX)\\
\hyperlink{Diavolo della Fossa}{Diavolo della Fossa}, GS 20 (25000 PX)\\
\hyperlink{Drago Bianco Antico}{Drago Bianco Antico}, GS 20 (25000 PX)\\
\hyperlink{Drago di Ottone Antico}{Drago di Ottone Antico}, GS 20 (25000 PX)\\
\hyperlink{Angelo Solar}{Angelo Solar}, GS 21 (33000 PX)\\
\hyperlink{Drago di Rame Antico}{Drago di Rame Antico}, GS 21 (33000 PX)\\
\hyperlink{Drago Nero Antico}{Drago Nero Antico}, GS 21 (33000 PX)\\
\hyperlink{Lich}{Lich}, GS 21 (33000 PX)\\
\hyperlink{Drago di Bronzo Antico}{Drago di Bronzo Antico}, GS 22 (41000 PX)\\
\hyperlink{Drago Verde Antico}{Drago Verde Antico}, GS 22 (41000 PX)\\
\hyperlink{Drago Blu Antico}{Drago Blu Antico}, GS 23 (50000 PX)\\
\hyperlink{Drago di Argento Antico}{Drago di Argento Antico}, GS 23 (50000 PX)\\
\hyperlink{Drago Giallo Antico}{Drago Giallo Antico}, GS 23 (50000 PX)\\
\hyperlink{Kraken}{Kraken}, GS 23 (50000 PX)\\
\hyperlink{Drago d'Oro Antico}{Drago d'Oro Antico}, GS 24 (62000 PX)\\
\hyperlink{Drago Rosso Antico}{Drago Rosso Antico}, GS 24 (62000 PX)\\
\hyperlink{Demogorgone}{Demogorgone}, GS 26 (90000 PX)\\
\hyperlink{Orcus}{Orcus}, GS 26 (90000 PX)\\
\hyperlink{Tàhil}{Tàhil}, GS 30 (155000 PX)\\
\hyperlink{Tarrasque}{Tarrasque}, GS 30 (155000 PX)\\

}

\end{multicols}

\pagebreak

