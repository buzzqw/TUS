\section{Gli Incantesimi}

\begin{multicols}{2}

%\begin{changemargin}{0.3cm}{0.3cm}\begin{tcolorbox}[title = più effetti speciali!]
%Gli incantesimi elencati sono quelli della 5ed più alcune mie proposte ed altre rivisitazioni. Se avete suggerimenti per il Narratore per gestire critici non previsti parlatene con lui! Lo spirito di collaborazione deve essere sempre costruttivo.
%\end{tcolorbox}\end{changemargin}

%\begin{changemargin}{0.3cm}{0.3cm}\begin{narratore}\index{Opzionale - Alternativa al danno da Successo Critico Magico}
%Una alternativa agli effetti del Successo Critico Magico può essere che con incantesimi che causino danno diretto o curino, al posto del dado aggiuntivo si sommi direttamente metà del valore del dado arrotondato per eccesso. Quindi 1d10 diventa 6, 1d8 diventa 5, 1d6 diventa 4, 1d4 diventa 3. Es. Un danno da 1d10 + 1d8 per effetto critico diventa 1d10+5.
%\end{narratore}\end{changemargin}

\subsection{Elenco incantesimi}

\end{multicols}

\begin{multicols}{2}

%\smallskip\noindent\rule{\linewidth}{2pt} \index[Incantesimi]{Aiuto} \hypertarget{Aiuto}{}\medskip\noindent{\textbf{Aiuto}}\pdfbookmark[3]{Aiuto}{Aiuto}\noindent
%\noindent\begin{tabular}{p{2.8cm}|p{\dimexpr 0.5\textwidth - 3cm - 2\tabcolsep - 1pt\relax}}
%	\textbf{Lista di Magia} & Cura, Necromanzia \\
%	\textbf{Livello} & 2, Non Comune \\
%	\textbf{Tempo di Lancio} & 2 Azioni \\
%	\textbf{Gittata} & 9 metri \\
%	\textbf{Componenti} & V, S, M (una sottile striscia di tessuto bianco) \\
%	\textbf{Durata} & 1 ora per Competenza Magica \\
%\end{tabular}

\smallskip\noindent\rule{\linewidth}{2pt} \index[Incantesimi]{Aiuto} \hypertarget{Aiuto}{}\smallskip\noindent{\textbf{Aiuto}}\pdfbookmark[3]{Aiuto}{Aiuto}\noindent
\begin{description}[noitemsep, topsep=0pt, parsep=0pt, partopsep=0pt, leftmargin=0cm, labelwidth=2.8cm]
\item[\textbf{Lista di Magia}]: Cura, Necromanzia
\item[\textbf{Livello}]: 2, Non Comune
\item[\textbf{T. di Lancio}]: 2 Azioni
\item[\textbf{Gittata}]: 9 metri
\item[\textbf{Componenti}]: V, S, M (una sottile striscia di tessuto bianco)
\item[\textbf{Durata}]: 1 ora per CM
\end{description}

Il tuo incantesimo aumenta la robustezza e risolutezza dei tuoi alleati. Scegli fino a tre creature a gittata, per la durata prendono 5 Punti Ferita Temporanei. Non è possibile ricevere più di un incantesimo Aiuto al giorno.

\textbf{Per ogni Successo Critico Magico} ottenuto nella Prova di Magia puoi influenzare una persona in più o aumentare i PF temporanei di 5.

\smallskip\noindent\rule{\linewidth}{2pt} \index[Incantesimi]{Allarme}\hypertarget{Allarme}{}\smallskip\noindent{\textbf{Allarme}}\pdfbookmark[3]{Allarme}{Allarme}
\noindent
\begin{description}[noitemsep, topsep=0pt, parsep=0pt, partopsep=0pt, leftmargin=0cm, labelwidth=2.8cm]
\item[\textbf{Lista di Magia}]: Abiurazione
\item[\textbf{Livello}]: 1, Comune
\item[\textbf{T. di Lancio}]: 1 minuto
\item[\textbf{Gittata}]: 9 metri
\item[\textbf{Componenti}]: V, S, M (una campanella e un pezzo di pregiato filo d'argento)
\item[\textbf{Durata}]: 2 ore per CM (massimo 24 ore)
\end{description}

Predisponi un allarme contro intrusioni indesiderate. Scegli una porta, una finestra o un'area a gittata che non sia più grande di un sfera di 3 metri di raggio. Fino al termine dell'incantesimo, sarai avvertito da un allarme ogni volta che una creatura di taglia Minuscola o superiore entri in contatto o acceda all'area protetta. Quando lanci l'incantesimo, puoi indicare delle creature che non faranno scattare l'allarme Scegli anche se l'allarme è udibile o solo mentale. Un allarme mentale, qualora ti trovi entro 1,5 chilometri dall'area protetta, ti avverte con un rumore nella tua mente. Il rumore è in grado di svegliarti se stai dormendo. Un allarme udibile produce il suono di una campanella per 10 secondi, udibile entro 18 metri.

\textbf{Per ogni Successo Critico Magico} ottenuto nella Prova di Magia la durata aumenta di 2 ore.

\smallskip\noindent\rule{\linewidth}{2pt} \index[Incantesimi]{Allucinazione Mortale}\hypertarget{Allucinazione Mortale}{}\smallskip\noindent{\textbf{Allucinazione Mortale}}\pdfbookmark[3]{Allucinazione Mortale}{Allucinazione Mortale}
\noindent
\begin{description}[noitemsep, topsep=0pt, parsep=0pt, partopsep=0pt, leftmargin=0cm, labelwidth=2.8cm]
\item[\textbf{Lista di Magia}]: Illusione
\item[\textbf{Livello}]: 4, Non Comune
\item[\textbf{T. di Lancio}]: 2 Azioni
\item[\textbf{Gittata}]: 36 metri
\item[\textbf{Componenti}]: V, S
\item[\textbf{Durata}]: Istantanea
\end{description}

Attingi agli incubi di una creatura a gittata e che puoi vedere e crei una manifestazione illusoria delle sue più terrificanti paure, visibile solo per quella creatura. Il bersaglio deve effettuare un Tiro Salvezza su Volontà.

Se fallisce il Tiro Salvezza, il bersaglio è spaventato per 1 minuto e subisce 4d10 di danno.

\textbf{Tiro Salvezza Successo/Fallimento Critico}: In caso di Fallimento Critico il danno raddoppia, in caso di Successo Critico il danno viene ulteriormente dimezzato

\textbf{Per ogni Successo Critico Magico} ottenuto nella Prova di Magia il danno aumenta di 2d10

\smallskip\noindent\rule{\linewidth}{2pt} \index[Incantesimi]{Alterare Sé Stesso}\hypertarget{Alterare Sé Stesso}{}\smallskip\noindent{\textbf{Alterare Sé Stesso}}\pdfbookmark[3]{Alterare Se Stesso}{Alterare Se Stesso}\label{Alter Self}
\noindent
\begin{description}[noitemsep, topsep=0pt, parsep=0pt, partopsep=0pt, leftmargin=0cm, labelwidth=2.8cm, labelsep=0.2cm]
\item[\textbf{Lista di Magia}]: Trasmutazione
\item[\textbf{Livello}]: 2, Non Comune
\item[\textbf{T. di Lancio}]: 2 Azioni
\item[\textbf{Gittata}]: Personale
\item[\textbf{Componenti}]: V, S
\item[\textbf{Durata}]: 1 minuto per CM
\end{description}

Assumi una forma diversa. Quando lanci questo incantesimo, scegli una della seguenti opzioni, l'effetto della quale permane per la durata dell'incantesimo. Per la durata dell'incantesimo puoi terminare un'opzione per ottenere i benefici di un'altra.

\emph{Adattamento Acquatico}. Adatti il tuo corpo a un ambiente acquatico, sviluppando branchie e dita palmate. Puoi respirare sott'acqua e ottieni velocità di nuoto pari alla metà della tua velocità di movimento.

\emph{Armi Naturali}. Sviluppi degli artigli, zanne, spuntoni, corna o una diversa arma naturale a tua scelta. I tuoi colpi senz'armi infliggono 1d6 danni contundenti, perforanti o taglienti, come appropriato all'arma naturale scelta con la quale sei competente. Infine, l'arma naturale è considerata come un arma +1. In questa forma usa la CA per calcolare i Tiri per Colpire.

\emph{Cambio di Aspetto}. Trasformi il tuo aspetto. Decidi il tuo aspetto esteriore, compresa l'altezza, il peso, i lineamenti facciali, il suono della tua voce, la lunghezza dei capelli, il colorito e qualsiasi peculiarità tu desideri. Puoi apparire come membro di un'altra razza, sebbene nessuna delle tue statistiche cambi. Inoltre non puoi apparire come una creatura di taglia diversa dalla tua, e la tua forma base resta la medesima; se sei bipede, non puoi usare questo incantesimo per diventare quadrupede, per esempio.

In qualsiasi momento della durata dell'incantesimo, puoi usare due Azioni per cambiare nuovamente di aspetto in questo modo.

\textbf{Per ogni Successo Critico Magico} ottenuto nella Prova di Magia puoi alterare un altro soggetto o raddoppiare la durata.

\smallskip\noindent\rule{\linewidth}{2pt} \index[Incantesimi]{Amicizia con gli Animali}\hypertarget{Amicizia con gli Animali}{}\smallskip\noindent{\textbf{Amicizia con gli Animali}}\pdfbookmark[3]{Amicizia con gli Animali}{Amicizia con gli Animali}
\noindent
\begin{description}[noitemsep, topsep=0pt, parsep=0pt, partopsep=0pt, leftmargin=0cm, labelwidth=2.8cm]
\item[\textbf{Lista di Magia}]: Animali e Piante
\item[\textbf{Livello}]: 1, Non Comune
\item[\textbf{T. di Lancio}]: 2 Azioni
\item[\textbf{Gittata}]: 9 metri
\item[\textbf{Componenti}]: V, S, M (un pò di cibo)
\item[\textbf{Durata}]: 24 ore
\end{description}

Questo incantesimo ti permette di convincere una bestia naturale che non vuoi arrecargli danno. Scegli una bestia a gittata che puoi vedere. Questa deve vederti e udirti. Se l'Intelligenza della bestia è 2 o più l'incantesimo fallisce. Altrimenti la bestia deve superare un Tiro Salvezza su Volontà o restare affascinata da te per la durata dell'incantesimo. Se tu o uno dei tuoi compagni danneggiate il bersaglio l'incantesimo ha termine.

\textbf{Per ogni Successo Critico Magico} ottenuto nella Prova di Magia puoi agire su di una bestia aggiuntiva.

\smallskip\noindent\rule{\linewidth}{2pt} \index[Incantesimi]{Anatema}\hypertarget{Anatema}{}\smallskip\noindent{\textbf{Anatema}}\pdfbookmark[3]{Anatema}{Anatema}
\noindent
\begin{description}[noitemsep, topsep=0pt, parsep=0pt, partopsep=0pt, leftmargin=0cm, labelwidth=2.8cm]
\item[\textbf{Lista di Magia}]: Ammaliamento
\item[\textbf{Livello}]: 1, Comune
\item[\textbf{T. di Lancio}]: 2 Azioni
\item[\textbf{Gittata}]: 9 metri
\item[\textbf{Componenti}]: V, S, M (una goccia del tuo sangue)
\item[\textbf{Durata}]: 1 minuto
\end{description}

Fino a tre creature di tua scelta che puoi vedere, e che sono a gittata, devono effettuare un Tiro Salvezza su Volontà. Ogni bersaglio che fallisca questo Tiro Salvezza ed effettui un Tiro per Colpire o un Tiro Salvezza prima del termine dell'incantesimo ha un -1 di penalità.

\textbf{Per ogni Successo Critico Magico} ottenuto nella Prova di Magia puoi prendere come bersaglio una creatura aggiuntiva.

\smallskip\noindent\rule{\linewidth}{2pt} \index[Incantesimi]{Animale Messaggero}\hypertarget{Animale Messaggero}{}\smallskip\noindent{\textbf{Animale Messaggero}}\pdfbookmark[3]{Animale Messaggero}{Animale Messaggero}\label{Animal Messenger}
\noindent
\begin{description}[noitemsep, topsep=0pt, parsep=0pt, partopsep=0pt, leftmargin=0cm, labelwidth=2.8cm]
\item[\textbf{Lista di Magia}]: Animali e Piante
\item[\textbf{Livello}]: 2, Comune
\item[\textbf{T. di Lancio}]: 2 Azioni
\item[\textbf{Gittata}]: 9 metri
\item[\textbf{Componenti}]: V, S, M (un poco di cibo)
\item[\textbf{Durata}]: 24 ore
\end{description}

Tramite questo incantesimo, usi un animale per consegnare un messaggio. Scegli una bestia Minuscola a gittata e che puoi vedere, come uno scoiattolo, una ghiandaia o un pipistrello, che abbia un GS inferiore a 0. Specifichi un luogo, che devi aver visitato in passato, e un destinatario che corrisponda a una descrizione generica, come \emph{un uomo o una donna che vesta l'uniforme della guardia cittadina} o \emph{un nano dai capelli rossi che indossa una coppola}. Pronuncia anche un messaggio di massimo venticinque parole. La bestia bersaglio viaggia per la durata dell'incantesimo verso il luogo specificato, coprendo circa 75 chilometri in 24 ore per un messaggero volante, o 40 chilometri per gli altri animali. Quando il messaggero arriva a destinazione, consegna il messaggio alla creatura da te descritta, replicando il suono della tua voce. Il messaggero parla solo a una creatura corrispondente alla descrizione da te fornita. Se il messaggero non riesce a raggiungere la destinazione prima del termine dell'incantesimo, il messaggio è perduto, e la bestia ritorna verso il punto in cui hai lanciato l'incantesimo.

\textbf{Per ogni Successo Critico Magico} ottenuto nella Prova di Magia la durata dell'incantesimo aumenta di 8 ore

\smallskip\noindent\rule{\linewidth}{2pt} \index[Incantesimi]{Animare Morti}\hypertarget{Animare Morti}{}\smallskip\noindent{\textbf{Animare Morti}}\pdfbookmark[3]{Animare Morti}{Animare Morti}
\noindent
\begin{description}[noitemsep, topsep=0pt, parsep=0pt, partopsep=0pt, leftmargin=0cm, labelwidth=2.8cm]
\item[\textbf{Lista di Magia}]: Necromanzia
\item[\textbf{Livello}]: 3, Comune
\item[\textbf{T. di Lancio}]: 1 minuto
\item[\textbf{Gittata}]: 3 metri
\item[\textbf{Componenti}]: V, S, M (una goccia di sangue, un pezzo di carne e un pizzico di polvere d'ossa)
\item[\textbf{Durata}]: Istantanea
\end{description}

Questo incantesimo crea un servitore non morto. Scegli una pila di ossa o un cadavere di un umanoide Medio o Piccolo a gittata. Il tuo incantesimo imbeve il bersaglio di una nefanda parvenza di vita rianimandolo come creatura non morta. Il bersaglio diventa uno scheletro se scegli le ossa o uno zombi se scegli un cadavere. Durante ciascun tuo round, puoi usare un'Azione per comandare mentalmente qualsiasi creatura da te creata con questo incantesimo che si trovi entro 18 metri da te (se controlli più creature, puoi comandarle tutte o solo alcune di loro allo stesso momento, inviando lo stesso comando a tutte). Decidi quale azione la creatura svolgerà e dove si muoverà durante il suo prossimo round oppure inviale un comando generale come quello di stare di guardia a una particolare stanza o corridoio. Se non invii alcun comando, la creatura si limita a difendersi dalle creature ostili. Una volta ricevuto un ordine, la creatura continuerà a svolgerlo fino al suo compimento. La creatura è sotto il tuo controllo per 24 ore, dopodiché smetterà di eseguire i comandi che le impartirai. Per mantenere il controllo sulla creatura per altre 24 ore, devi lanciare di nuovo questo incantesimo su di essa prima del termine dell'attuale periodo di 24 ore. Questo impiego dell'incantesimo riafferma il tuo controllo su di un massimo di quattro creature che hai animato con questo incantesimo, piuttosto che animarne una nuova.

\textbf{Per ogni Successo Critico Magico} ottenuto nella Prova di Magia animi o riaffermi il controllo su due creature non morte. Ciascuna di queste creature deve provenire da un cadavere o pila di ossa differenti.

\smallskip\noindent\rule{\linewidth}{2pt} \index[Incantesimi]{Animare Oggetti}\hypertarget{Animare Oggetti}{}\smallskip\noindent{\textbf{Animare Oggetti}}\pdfbookmark[3]{Animare Oggetti}{Animare Oggetti}\label{Animate Object}
\noindent
\begin{description}[noitemsep, topsep=0pt, parsep=0pt, partopsep=0pt, leftmargin=0cm, labelwidth=2.8cm]
\item[\textbf{Lista di Magia}]: Trasmutazione
\item[\textbf{Livello}]: 5, Comune
\item[\textbf{T. di Lancio}]: 1 minuto
\item[\textbf{Gittata}]: 36 metri
\item[\textbf{Componenti}]: V, S
\item[\textbf{Durata}]: Concentrazione, massimo 1 minuto
\end{description}

Gli oggetti prendono vita al tuo comando. Scegli fino a CM/2 oggetti non magici a gittata e che non siano indossati o trasportati. I bersagli Medi contano come due oggetti, i bersagli Grandi contano come quattro oggetti, i bersagli Enormi contano come otto oggetti. Non puoi animare oggetti di taglia più grossa di Enorme. Ogni bersaglio si anima e diventa una creatura sotto il tuo controllo fino al termine dell'incantesimo o finché non viene ridotto a 0 Punti Ferita.

Ogni bersaglio che si anima fa spuntare le gambe e diventa un Costrutto che utilizza il blocco statistiche di Oggetto Animato, questa creatura è sotto il tuo controllo finché l'incantesimo non termina o finché non viene ridotta a 0 Punti Ferita. Ogni creatura che crei con questo incantesimo è un alleato tuo e dei tuoi alleati. In combattimento, condivide il tuo conteggio di Iniziativa e agisce al tuo comando.

Con un'Azione puoi comandare mentalmente qualsiasi creatura che hai generato con questo incantesimo e che si trovi entro 150 metri da te (se controlli più creature, puoi comandarne solo alcune o tutte allo stesso tempo, impartendo lo stesso comando a ciascuna).
Decidi tu quale azione intraprenderà la creatura e dove si muoverà durante il suo round successivo, o puoi emettere un comando generico, come quello di fare la guardia a una particolare stanza o corridoio.
Se non impartisci comandi, la creatura si limiterà a difendersi dalle creature ostili. Una volta dato un ordine, la creatura continuerà a seguirlo finché non avrà completato il suo compito.

\textbf{Per ogni Successo Critico Magico} ottenuto nella Prova di Magia la durata massima raddoppia.

\medskip

\noindent\rule{\linewidth}{2pt} \index[Mostruario]{Oggetto Animato} \hypertarget{Oggetto Animato}{}\medskip \noindent{\large\textbf{Oggetto Animato}}
\noindent
\begin{description}[noitemsep, topsep=0pt, parsep=0pt, partopsep=0pt, leftmargin=0cm, labelwidth=2.2cm]
	\item[\textbf{Tipo:}] varie, costrutto, disallineato
	\item[\textbf{Caratt.:}] For 3 Des 0 Cos 0 Int -4 Sag -4 Car -5
	\item[\textbf{Punti Ferita:}] 10/20/40,  \textbf{Difesa:} 15,  \textbf{Iniziativa:} +0
	\item[\textbf{Movimento:}] 6 m
	\item[\textbf{Tiri Salvezza:}] Tempra +3, Riflessi +1, Volontà 0
	\item[\textbf{Imm. Danni:}] Veleno, Energia Positiva, Energia Negativa
	\item[\textbf{Immunità:}] accecato, affascinato, assordato, paralizzato, pietrificato, spaventato
	\item[\textbf{Sensi:}] Vista Cieca 18 m (cieco oltre questo raggio)
	\item[\textbf{Sfida:}] nessuno\smallskip
\end{description}

\textbf{Azioni}

\textit{Schianto.} Tiro per Colpire in Mischia Bonus pari al tuo modificatore di attacco con incantesimo, portata 1 m.

\emph{Colpito}: Danni da botta pari a 1d4 + 3 (taglia Media o più piccola), 2d6 + 3 + il tuo modificatore di caratteristica da incantatore (Grande),  2d12 + 3 + il tuo modificatore di caratteristica da incantatore (Enorme).

Quando l'oggetto animato scende a 0 Punti Ferita, ritorna alla sua normale forma di oggetto, e tutti i danni in eccesso vengono inflitti alla sua forma originale.

Se comandi a un oggetto di attaccare, questo può effettuare un singolo attacco da mischia contro una creatura entro 1 metro da esso. Un oggetto potrebbe invece infliggere danni taglienti o perforanti a seconda della sua forma.

\textbf{Per ogni Successo Critico Magico} ottenuto nella Prova di Magia puoi animare due oggetti aggiuntivi.

\smallskip\noindent\rule{\linewidth}{2pt} \index[Incantesimi]{Anti-Individuazione}\hypertarget{Anti-Individuazione}{}\smallskip\noindent{\textbf{Anti-Individuazione}}\pdfbookmark[3]{Anti-Individuazione}{Anti-Individuazione}
\noindent
\begin{description}[noitemsep, topsep=0pt, parsep=0pt, partopsep=0pt, leftmargin=0cm, labelwidth=2.8cm]
	\item[\textbf{Lista di Magia}]: Abiurazione
	\item[\textbf{Livello}]: 3, Non Comune
	\item[\textbf{T. di Lancio}]: 2 Azioni
	\item[\textbf{Gittata}]: Contatto
	\item[\textbf{Componenti}]: V, S, M (un pizzico di polvere di diamante del valore di 25 mo sparsa sul bersaglio, che l'incantesimo consuma)
	\item[\textbf{Durata}]: 8 ore
\end{description}

Per la durata nascondi il bersaglio con cui sei stato in contatto dalla magia di divinazione. Il bersaglio può essere una creatura consenziente o un luogo o un oggetto che occupi uno spazio equivalente ad una sfera di 2 metri di raggio. Il bersaglio non può divenire bersaglio di alcuna magia di divinazione o essere percepito tramite sensi di scrutamento magici.

\smallskip\noindent\rule{\linewidth}{2pt} \index[Incantesimi]{Antipatia/Simpatia}\hypertarget{Antipatia/Simpatia}{}\smallskip\noindent{\textbf{Antipatia/Simpatia}}\pdfbookmark[3]{Antipatia/Simpatia}{Antipatia/Simpatia}
\noindent
\begin{description}[noitemsep, topsep=0pt, parsep=0pt, partopsep=0pt, leftmargin=0cm, labelwidth=2.8cm]
	\item[\textbf{Lista di Magia}]: Ammaliamento
	\item[\textbf{Livello}]: 8, Raro
	\item[\textbf{T. di Lancio}]: 1 ora
	\item[\textbf{Gittata}]: 18 metri
	\item[\textbf{Componenti}]: V, S, M (o un pezzo di allume immerso nell'aceto per l'effetto antipatia o un goccio di miele per l'effetto simpatia)
	\item[\textbf{Durata}]: 10 giorni
\end{description}

Questo incantesimo attrae o repelle delle creature di tua scelta. Prendi un bersaglio a gittata, che sia un oggetto Enorme o più piccolo o una creatura o un'area non più grande di una sfera di 30 metri di raggio. Poi specifica una specie di creature intelligenti, come i draghi rossi, i goblin o i vampiri. Investi il bersaglio di un'aura che attrae o respinge le creature specificate per la durata. Scegli antipatia o simpatia come effetto dell'aura.

\emph{Antipatia}. L'ammaliamento fa sì che le creature del tipo da te indicato provino un forte impulso a lasciare l'area ed evitare il bersaglio. Quando una creatura del genere può vedere il bersaglio o si avvicina entro 18 metri da esso, la creatura deve superare un Tiro Salvezza su Volontà o diventare spaventata. La creatura rimane spaventata finché può vedere il bersaglio o resta entro 18 metri da esso. Mentre è spaventata dal bersaglio, la creatura deve impiegare il suo movimento per muoversi verso il posto sicuro più vicino dal quale non possa più vedere il bersaglio. Se la creatura si muove più di 18 metri lontano dal bersaglio e non può vederlo, la creatura non è più spaventata, ma torna a essere spaventata se torna a vedere il bersaglio o si muove entro 18 metri da esso.

\emph{Simpatia}. L'ammaliamento fa sì che le creature specificate provino un forte impulso ad avvicinarsi al bersaglio se si trovano entro 18 metri da esso o possono vederlo. Quando una simile creatura può vedere il bersaglio o si avvicina entro 18 metri da esso, la creatura deve superare un Tiro Salvezza su Volontà o usare il suo movimento durante ciascun round per entrare nell'area, o muoversi a portata del bersaglio. Quando la creatura l'avrà fatto, non potrà più volontariamente muoversi lontano dal bersaglio. Se il bersaglio danneggia o altrimenti nuoce alla creatura soggetta, questa può effettuare un Tiri Salvezza su Volontà per terminare l'effetto, come descritto di seguito.

\emph{Terminare l'Effetto}. Se una creatura soggetta termina il suo round mentre si trova più lontana di 18 metri dal bersaglio o non può vederlo, la creatura effettua un Tiro Salvezza su Volontà. Se supera il Tiro Salvezza, la creatura non è più soggetta al bersaglio e riconosce la sensazione di ripugnanza o attrazione come magica. Inoltre, una creatura soggetta all'incantesimo, ha diritto a un altro Tiro Salvezza su Volontà ogni 24 ore di durata dell'incantesimo. Una creatura che supera il Tiro Salvezza contro questo effetto è immune a esso per 1 minuto, dopodiché può subirlo nuovamente.

\smallskip\noindent\rule{\linewidth}{2pt} \index[Incantesimi]{Arma Energetica}\hypertarget{Arma Energetica}{}\smallskip\noindent{\textbf{Arma Energetica}}\pdfbookmark[3]{Arma Energetica}{Arma Energetica}
\noindent
\begin{description}[noitemsep, topsep=0pt, parsep=0pt, partopsep=0pt, leftmargin=0cm, labelwidth=2.8cm]
	\item[\textbf{Lista di Magia}]: Aria, Acqua, Terra, Fuoco
	\item[\textbf{Livello}]: 1, Molto Raro
	\item[\textbf{T. di Lancio}]: 1 Azione
	\item[\textbf{Gittata}]: Contatto
	\item[\textbf{Componenti}]: V, S, M (dei capelli di Fata)
	\item[\textbf{Durata}]: 6 round, Concentrazione
\end{description}

Lanci l'incantesimo a contatto di un'arma e questa acquisisce dei poteri a seconda della Lista di Magia dal quale hai lanciato l'incantesimo. L'arma si considera magica, come avesse un bonus di +1.
Se Arma Energetica viene lanciato usando la Lista dell'\emph{Aria} l'arma diviene percorsa da elettricità, in caso di \emph{Acqua} l'arma diventa estremamente fredda, in caso di \emph{Terra} dall'arma sgorga acido, in caso di \emph{Fuoco} questa diventa fiammeggiante. Qualsiasi sia la Lista usata l'effetto è tale che l'arma causa 1d6 di danno aggiuntivo del tipo indicato per colpo andato a segno.
Un arma può avere solo un effetto di Arma Energetica attivo contemporaneamente.

\textbf{Per ogni due Successo Critico Magico ottenuto} nella Prova di Magia il danno aumenta di +1d6.

\smallskip\noindent\rule{\linewidth}{2pt} \index[Incantesimi]{Arma Magica}\hypertarget{Arma Magica}{}\smallskip\noindent{\textbf{Arma Magica}}\pdfbookmark[3]{Arma Magica}{Arma Magica}
\noindent
\begin{description}[noitemsep, topsep=0pt, parsep=0pt, partopsep=0pt, leftmargin=0cm, labelwidth=2.8cm]
	\item[\textbf{Lista di Magia}]: Trasmutazione
	\item[\textbf{Livello}]: 2, Comune
	\item[\textbf{T. di Lancio}]: 1 Azione
	\item[\textbf{Gittata}]: Contatto
	\item[\textbf{Componenti}]: V, S
	\item[\textbf{Durata}]: 10 minuti
\end{description}

Lanci l'incantesimo a contatto di un'arma non magica. Fino al termine dell'incantesimo, l'arma diventa un'arma magica con un bonus di +1 ai Tiri per Colpire e di danno.

\textbf{Per ogni Successo Critico Magico} ottenuto nella Prova di Magia il bonus aumenta a +1.

\smallskip\noindent\rule{\linewidth}{2pt} \index[Incantesimi]{Arma Spirituale}\hypertarget{Arma Spirituale}{}\smallskip\noindent{\textbf{Arma Spirituale}}\pdfbookmark[3]{Arma Spirituale}{Arma Spirituale}
\noindent
\begin{description}[noitemsep, topsep=0pt, parsep=0pt, partopsep=0pt, leftmargin=0cm, labelwidth=2.8cm]
	\item[\textbf{Lista di Magia}]: Invocazione
	\item[\textbf{Livello}]: 2, Comune
	\item[\textbf{T. di Lancio}]: 2 Azioni
	\item[\textbf{Gittata}]: 18 metri
	\item[\textbf{Componenti}]: V, S
	\item[\textbf{Durata}]: 3 minuti, Concentrazione
\end{description}

In un punto nella gittata, crei un'arma spettrale fluttuante, che resta per la durata o finché non lanci di nuovo questo incantesimo. Quando lanci l'incantesimo, puoi effettuare un attacco da incantesimo in mischia contro una creatura entro 1 metro dall'arma con un bonus al colpire pari a Competenza Magica/4. Se colpisci, il bersaglio subisce danni da forza pari a 1d4 + il tuo modificatore di caratteristica per incantesimi da incantatore + Competenza Magica/4. Durante il tuo round, con una Azione, puoi spostare l'arma di 6 metri ed effettuare l'attacco contro una creatura entro 1 metro dall'arma. L'arma può assumere qualsiasi forma tu voglia, magari affine al Patrono. L'arma ha un equivalente bonus magico pari a Competenza Magica/4.

I bonus concessi da Competenza Magica/4 possono essere sostituiti dalla somma dei Tratti in comune con il Patrono/4 se si è un Seguace.

\textbf{Per ogni Successo Critico Magico} ottenuto nella Prova di Magia il danno aumenta di 2.

\smallskip\noindent\rule{\linewidth}{2pt} \index[Incantesimi]{Armatura Magica}\hypertarget{Armatura Magica}{}\smallskip\noindent{\textbf{Armatura Magica}}\pdfbookmark[3]{Armatura Magica}{Armatura Magica}
\noindent
\begin{description}[noitemsep, topsep=0pt, parsep=0pt, partopsep=0pt, leftmargin=0cm, labelwidth=2.8cm]
	\item[\textbf{Lista di Magia}]: Abiurazione
	\item[\textbf{Livello}]: 1, Non Comune
	\item[\textbf{T. di Lancio}]: 2 Azioni
	\item[\textbf{Gittata}]: Contatto
	\item[\textbf{Componenti}]: V, S, M (un pezzo di cuoio lavorato)
	\item[\textbf{Durata}]: 8 ore
\end{description}

Lanci l'incantesimo a contatto di una creatura consenziente che non indossa un'armatura. Una forza magica protettiva circonda il bersaglio fino al termine dell'incantesimo. La Difesa naturale del bersaglio aumenta di 3 + Destrezza + 1/6 Competenza Magica. L'incantesimo termina se il bersaglio indossa un'armatura o interrompe l'incantesimo con un'Azione.

\textbf{Per ogni Successo Critico Magico} ottenuto nella Prova di Magia la Difesa aumenta di 1.

\smallskip\noindent\rule{\linewidth}{2pt} \index[Incantesimi]{Artificio Druidico}\hypertarget{Artificio Druidico}{}\smallskip\noindent{\textbf{Artificio Druidico}}\pdfbookmark[3]{Artificio Druidico}{Artificio Druidico}
\noindent
\begin{description}[noitemsep, topsep=0pt, parsep=0pt, partopsep=0pt, leftmargin=0cm, labelwidth=2.8cm]
	\item[\textbf{Lista di Magia}]: Universale
	\item[\textbf{Livello}]: 0, Non Comune
	\item[\textbf{T. di Lancio}]: 2 Azioni
	\item[\textbf{Gittata}]: 9 metri
	\item[\textbf{Componenti}]: V, S
	\item[\textbf{Durata}]: Istantanea
\end{description}

Sussurrando agli spiriti della natura, crei, a gittata, uno dei seguenti effetti:

\begin{itemize}[leftmargin=*] \setlength{\itemsep}{0pt}
	\item Crei un minuscolo e innocuo effetto sensoriale che predice quale clima ci sarà nel luogo in cui ti trovi per le prossime 24 ore. L'effetto potrebbe manifestarsi come una sfera dorata per i cieli limpidi, una nube per la pioggia, fiocchi di neve per la neve, e così via. L'effetto persiste per 1 round.
	\item Fai immediatamente sbocciare un fiore, un seme o simile pianta.
	\item Crei un istantaneo e innocuo effetto sensoriale, come foglie che cadono, uno sbuffo di vento, il suono di un piccolo animale, o il lieve tanfo di una puzzola. L'effetto deve entrare in una sfera di 1 metro di raggio.
	\item Accendi o spegni istantaneamente una candela, una torcia o un piccolo falò.
\end{itemize}

Questo incantesimo può essere lanciato solo da Seguaci o Devoti di Efrem, Erondil, Gaya, Shayalia.

\smallskip\noindent\rule{\linewidth}{2pt} \index[Incantesimi]{Aura Magica dell'Arcanista}\hypertarget{Aura Magica dell'Arcanista}{}\smallskip\noindent{\textbf{Aura Magica dell'Arcanista}}\pdfbookmark[3]{Aura Magica dell'Arcanista}{Aura Magica dell'Arcanista}
\noindent
\begin{description}[noitemsep, topsep=0pt, parsep=0pt, partopsep=0pt, leftmargin=0cm, labelwidth=2.8cm]
	\item[\textbf{Lista di Magia}]: Illusione
	\item[\textbf{Livello}]: 2, Non Comune
	\item[\textbf{T. di Lancio}]: 2 Azioni
	\item[\textbf{Gittata}]: Contatto
	\item[\textbf{Componenti}]: V, S, M (un piccolo quadretto di seta)
	\item[\textbf{Durata}]: 24 ore
\end{description}

Poni un'illusione su di una creatura od oggetto con cui sei in contatto, così che gli incantesimi di divinazione rivelino false informazioni su di esso. Il bersaglio può essere una creatura consenziente o un oggetto che non sia trasportato o indossato da un'altra creatura. Quando lanci questo incantesimo, scegli uno o entrambi i seguenti effetti. L'effetto permane per la durata. Se esegui questo incantesimo sulla stessa creatura od oggetto ogni giorno per 30 giorni, piazzando ogni volta lo stesso effetto, l'illusione permarrà finché non viene dissolta.

\emph{Aura Falsa}. Cambi il modo in cui il bersaglio risulta a incantesimi ed effetti magici, come individuazione del magico, che individuano le aure magiche. Puoi far apparire magico un oggetto normale, non magico un oggetto magico, o cambiare l'aura magica dell'oggetto così che sembri appartenere a una Liste di Magia di tua scelta. Quando impieghi questo effetto su di un oggetto, puoi far sì che la falsa magia sia apparente a qualsiasi creatura che lo manipoli.

\emph{Mascherare}. Cambi il modo in cui il bersaglio risulta a incantesimi ed effetti magici che individuano il tipo di creatura o Tratti, come l'attivazione dell'incantesimo simbolo. Scegli un tipo di creatura o Tratto, e gli altri incantesimi ed effetti magici considereranno il bersaglio come fosse una creatura di quel tipo o di quel Tratto, e non più di quello originale.

\smallskip\noindent\rule{\linewidth}{2pt} \index[Incantesimi]{Aura Sacra}\hypertarget{Aura Sacra}{}\smallskip\noindent{\textbf{Aura Sacra}}\pdfbookmark[3]{Aura Sacra}{Aura Sacra}
\noindent
\begin{description}[noitemsep, topsep=0pt, parsep=0pt, partopsep=0pt, leftmargin=0cm, labelwidth=2.8cm]
	\item[\textbf{Lista di Magia}]: Abiurazione
	\item[\textbf{Livello}]: 8, Comune
	\item[\textbf{T. di Lancio}]: 2 Azioni
	\item[\textbf{Gittata}]: Personale
	\item[\textbf{Componenti}]: V, S, M (un minuscolo reliquario del valore di almeno 1000 mo contenente una reliquia sacra, come un pezzo di tessuto dell'abito di un Devoto o un frammento di pergamena di un testo religioso)
	\item[\textbf{Durata}]: Concentrazione, 1 minuto
\end{description}

Irradi da te luce divina che si raccoglie in una debole luminosità con raggio di 9 metri intorno a te. Quando lanci l'incantesimo, le creature da te scelte in questo raggio emanano luce fioca con un raggio di 1 metro e hanno +8 a tutti i Tiri Salvezza, mentre le altre creature hanno -8 sui Tiri per Colpire contro di loro fino al termine dell'incantesimo. Inoltre, quando un demone o non morto colpisce una creatura bersaglio con un attacco in mischia, l'aura risplende di una luce intensa e deve superare un Tiro Salvezza su Tempra o restare accecato fino al termine dell'incantesimo.

\smallskip\noindent\rule{\linewidth}{2pt} \index[Incantesimi]{Bacche Benefiche}\hypertarget{Bacche Benefiche}{}\smallskip\noindent{\textbf{Bacche Benefiche}}\pdfbookmark[3]{Bacche Benefiche}{Bacche Benefiche}
\noindent
\begin{description}[noitemsep, topsep=0pt, parsep=0pt, partopsep=0pt, leftmargin=0cm, labelwidth=2.8cm]
	\item[\textbf{Lista di Magia}]: Animali e Piante
	\item[\textbf{Livello}]: 2, Comune
	\item[\textbf{T. di Lancio}]: 2 Azioni
	\item[\textbf{Gittata}]: Contatto
	\item[\textbf{Componenti}]: V, S, M (un rametto di vischio, fino a 8 bacche su cui l'incantesimo agisce)
	\item[\textbf{Durata}]: Istantanea
\end{description}

Incanti fino a 2d4 bacche nella tua mano che vengono infuse di magia per la durata. Una creatura può usare 1 Azione Immediata per mangiare una bacca. Mangiare una bacca ripristina 1 punto ferita e provvede nutrimento, ma non acqua, sufficiente per alimentare una creatura per un giorno. Solo la prima bacca è efficace nel giorno.

Le bacche perdono la loro efficacia se non vengono consumate entro 8 ore dal lancio dell'incantesimo.

\textbf{Per ogni Successo Critico Magico} ottenuto nella Prova di Magia le bacche durano un giorno in più oppure incanti una bacca in più (fino ad un massimo totale di 8).

\smallskip\noindent\rule{\linewidth}{2pt} \index[Incantesimi]{Bagliore Solare}\hypertarget{Bagliore Solare}{}\smallskip\noindent{\textbf{Bagliore Solare}}\pdfbookmark[3]{Bagliore Solare}{Bagliore Solare}\index{Cannone a onde moventi Yamato}
\noindent
\begin{description}[noitemsep, topsep=0pt, parsep=0pt, partopsep=0pt, leftmargin=0cm, labelwidth=2.8cm]
	\item[\textbf{Lista di Magia}]: Invocazione
	\item[\textbf{Livello}]: 6, Non Comune
	\item[\textbf{T. di Lancio}]: 2 Azioni
	\item[\textbf{Gittata}]: Personale (linea di 18 metri)
	\item[\textbf{Componenti}]: V, S, M (una lente di ingrandimento)
	\item[\textbf{Durata}]: Concentrazione, massimo 1 minuto
\end{description}

Una fascio di luce brillante esplode dalla tua mano in una linea larga 1 metro e lunga 18 metri. Ogni creatura sulla linea deve effettuare un Tiro Salvezza su Tempra. Se fallisce il Tiro Salvezza, la creatura subisce 6d8 danni da Luce e rimane accecata fino al tuo prossimo round. Se supera il Tiro Salvezza, subisce la metà dei danni e non è accecata. I non morti e le melme hanno -1d6 su questo Tiro Salvezza. Puoi creare una nuova linea di luminosità spendendo 3 Azioni durante qualsiasi tuo round fino al termine dell'incantesimo.

Per la durata, una particella di luce brillante risplende nella tua mano. Produce luce in un raggio di 9 metri e penombra per ulteriori 9 metri. Questa luce è considerata luce solare.

\textbf{In caso di due Successo Critico Magico ottenuto} l'incantesimo termina dopo il primo raggio ma la linea è larga 6 metri, lunga 108 metri, il danno da Luce diviene 12d8.

\smallskip\noindent\rule{\linewidth}{2pt} \index[Incantesimi]{Banchetto degli Eroi}\hypertarget{Banchetto degli Eroi}{}\smallskip\noindent{\textbf{Banchetto degli Eroi}}\pdfbookmark[3]{Banchetto degli Eroi}{Banchetto degli Eroi}
\noindent
\begin{description}[noitemsep, topsep=0pt, parsep=0pt, partopsep=0pt, leftmargin=0cm, labelwidth=2.8cm]
	\item[\textbf{Lista di Magia}]: Evocazione
	\item[\textbf{Livello}]: 6, Non Comune
	\item[\textbf{T. di Lancio}]: 10 minuti
	\item[\textbf{Gittata}]: 9 metri
	\item[\textbf{Componenti}]: V, S, M (una ciotola incrostata di gemme del valore di almeno 500 mo, che l'incantesimo consuma)
	\item[\textbf{Durata}]: Istantanea
\end{description}

Crei un magnifico banchetto, comprensivo di cibi e bevande prelibate. Il banchetto viene consumato in 1 ora e scompare al termine di questo periodo, ma gli effetti benefici non si faranno sentire fino al termine dell'ora. Fino ad altre dodici creature possono
partecipare al banchetto. Una creatura che partecipi al banchetto ottiene diversi benefici. La creatura viene guarita da tutte le malattie e i veleni non magici. Diventa immune al veleno e all'essere spaventata, ha +1d6 su tutti i Tiri Salvezza su Volontà e Tempra ed acquisisce 2010 Punti Ferita temporanei, questi benefici durano 24 ore.

\textbf{In caso di due Successo Critico Magico ottenuto} nella Prova di Magia la ciotola non è consumata.

\smallskip\noindent\rule{\linewidth}{2pt} \index[Incantesimi]{Barriera Antianimali}\hypertarget{Barriera Antianimali}{}\smallskip\noindent{\textbf{Barriera Antianimali}}\pdfbookmark[3]{Barriera Antianimali}{Barriera Antianimali}
\noindent
\begin{description}[noitemsep, topsep=0pt, parsep=0pt, partopsep=0pt, leftmargin=0cm, labelwidth=2.8cm]
	\item[\textbf{Lista di Magia}]: Animali e Piante
	\item[\textbf{Livello}]: 5, Raro
	\item[\textbf{T. di Lancio}]: 2 Azioni
	\item[\textbf{Gittata}]: Personale
	\item[\textbf{Componenti}]: V, S
	\item[\textbf{Durata}]: 1 turno per CM
\end{description}

Barriera antianimali crea una barriera invisibile in grado di proteggere tutte le creature al suo interno, come se si trovassero dietro un muro, dagli attacchi degli animali, normali e giganti. La barriera, centrata sull'incantatore, ha un diametro di 6 metri.

\textbf{Per ogni Successo Critico Magico} ottenuto nella Prova di Magia raddoppi la durata o allarghi il raggio di 2 metri.

\smallskip\noindent\rule{\linewidth}{2pt} \index[Incantesimi]{Barriera di Lame}\hypertarget{Barriera di Lame}{}\smallskip\noindent{\textbf{Barriera di Lame}}\pdfbookmark[3]{Barriera di Lame}{Barriera di Lame}
\noindent
\begin{description}[noitemsep, topsep=0pt, parsep=0pt, partopsep=0pt, leftmargin=0cm, labelwidth=2.8cm]
	\item[\textbf{Lista di Magia}]: Invocazione
	\item[\textbf{Livello}]: 6, Comune
	\item[\textbf{T. di Lancio}]: 2 Azioni
	\item[\textbf{Gittata}]: 18 metri
	\item[\textbf{Componenti}]: V, S
	\item[\textbf{Durata}]: 10 minuti
\end{description}

Crei un muro verticale di lame rotanti fatte di energia magica, affilate come rasoi. Il muro compare a gittata e resta per la durata. Puoi creare un muro diritto lungo fino a 30 metri, alto 6 metri e spesso 1 metro, o un muro circolare di 18 metri massimo di diametro e spesso 1 metro. Il muro fornisce copertura completa alle creature dietro di esso e il suo spazio è terreno difficile.

Quando una creatura entra per la prima volta in un round nell'area del muro o comincia il suo round lì deve effettuare un Tiro Salvezza su Riflessi. Se la creatura fallisce il Tiro Salvezza subisce 6d10 danni taglienti, o la metà se lo supera.

Un incantatore che è ad una distanza di un metro dalla Barriera di Lame si considera Distratto.

\smallskip\noindent\rule{\linewidth}{2pt} \index[Incantesimi]{Bastoni in Serpenti}\hypertarget{Bastoni in Serpenti}{}\smallskip\noindent{\textbf{Bastoni in Serpenti}}\pdfbookmark[3]{Bastoni in Serpenti}{Bastoni in Serpenti}
\noindent
\begin{description}[noitemsep, topsep=0pt, parsep=0pt, partopsep=0pt, leftmargin=0cm, labelwidth=2.8cm]
	\item[\textbf{Lista di Magia}]: Animali e Piante
	\item[\textbf{Livello}]: 3, Non Comune
	\item[\textbf{T. di Lancio}]: 2 Azioni
	\item[\textbf{Gittata}]: 18 metri
	\item[\textbf{Componenti}]: V, S, M (diversi bastoncini ed una goccia di veleno di serpente)
	\item[\textbf{Durata}]: Concentrazione fino ad 1 minuto
\end{description}

Trasformi 1d4 bastoncini, +1 per ogni volta che hai preso Adepto della Magia, in serpenti velenosi. I serpenti agiscono, nel tuo round, sempre all'unisono e compiono la medesima Azione contro lo stesso avversario.

Questi serpenti, considerati oggetti minuscoli, hanno Difesa 13, 10 Punti Ferita, tutti i Tiri Salvezza a 5. Se scendono sotto 0 Punti Ferita tornano dei bastoncini ma rotti.

Con una Azione puoi comandare i serpenti di attaccare. Esegui un Tiro per Colpire come da attacco con incantesimo in mischia per ogni Serpente contro una creatura entro 1 metro da loro. Ogni serpente che colpisce causa 1 danno da perforazione ed obbliga un Tiro Salvezza su Tempra a DC 14, se il Tiro Salvezza fallisce la creatura subisce 2d4 di danno da veleno o la metà se riesce.

Con una Azione puoi comandare i serpenti di spostarsi fino a 6 metri.

\textbf{Ogni Successo Critico Magico ottenuto} nella Prova di Magia crei un nuovo serpente.

\smallskip\noindent\rule{\linewidth}{2pt} \index[Incantesimi]{Beffa Crudele}\hypertarget{Beffa Crudele}{}\smallskip\noindent{\textbf{Beffa Crudele}}\pdfbookmark[3]{Beffa Crudele}{Beffa Crudele}
\noindent
\begin{description}[noitemsep, topsep=0pt, parsep=0pt, partopsep=0pt, leftmargin=0cm, labelwidth=2.8cm]
	\item[\textbf{Lista di Magia}]: Ammaliamento
	\item[\textbf{Livello}]: 0, Comune
	\item[\textbf{T. di Lancio}]: 1 Azione
	\item[\textbf{Gittata}]: 18 metri
	\item[\textbf{Componenti}]: V
	\item[\textbf{Durata}]: Istantanea
\end{description}

Scateni una serie di insulti avvolti da una subdola malia contro una creatura a gittata e che puoi vedere. Se il bersaglio ti può udire (sebbene non sia necessario che ti comprenda), deve superare un Tiro Salvezza su Volontà o subire 1d4 danni e avere -2 al prossimo Tiro per Colpire che effettuerà prima del termine del suo prossimo round.

Il danno dell'incantesimo aumenta di 1d4 quando raggiungi CM 5, CM 11 e CM 17, ma costa 2 Azioni lanciarlo potenziato e 1 Punti Magia, è altresì necessario avere preso Adepto della Magia un numero di volte pari ai potenziamenti che si vogliono applicare.

\textbf{Ogni 2 Successo Critico Magico ottenuto} nella Prova di Magia influenzi un altra creatura.

\smallskip\noindent\rule{\linewidth}{2pt} \index[Incantesimi]{Benedici Acqua}\hypertarget{Benedici Acqua}{}\smallskip\noindent{\textbf{Benedici Acqua}}\pdfbookmark[3]{Benedici Acqua}{Benedici Acqua}
\noindent
\begin{description}[noitemsep, topsep=0pt, parsep=0pt, partopsep=0pt, leftmargin=0cm, labelwidth=2.8cm]
	\item[\textbf{Lista di Magia}]: Universale
	\item[\textbf{Livello}]: 2, Comune
	\item[\textbf{T. di Lancio}]: 10 Minuti
	\item[\textbf{Gittata}]: Contatto
	\item[\textbf{Componenti}]: V, S, M (25 monete d'oro in offerta alla chiesa che l'incantesimo consuma)
	\item[\textbf{Durata}]: Istantanea
\end{description}

Benedici fino ad un litro di liquido, sufficiente a creare 5 boccette di Acqua santa.

Devi essere un Seguace o Devoto per poter lanciare questo incantesimo.

\textbf{Per ogni Successo Critico Magico} ottenuto nella Prova di Magia benedici un litro di liquido in più.\index{Acqua Benedetta, creare}

\smallskip\noindent\rule{\linewidth}{2pt} \index[Incantesimi]{Benedizione}\hypertarget{Benedizione}{}\smallskip\noindent{\textbf{Benedizione}}\pdfbookmark[3]{Benedizione}{Benedizione}
\noindent
\begin{description}[noitemsep, topsep=0pt, parsep=0pt, partopsep=0pt, leftmargin=0cm, labelwidth=2.8cm]
	\item[\textbf{Lista di Magia}]: Universale
	\item[\textbf{Livello}]: 1, Comune
	\item[\textbf{T. di Lancio}]: 2 Azioni
	\item[\textbf{Gittata}]: 9 metri
	\item[\textbf{Componenti}]: V, S, M (una fialetta di Acqua santa che l'incantesimo consuma)
	\item[\textbf{Durata}]: 1 minuto
\end{description}

Benedici fino a tre creature a gittata, scelte da te. I bersagli guadagnano +1 ai Tiri Salvezza e Tiro per Colpire.

più benedizioni, anche da Patroni diversi non si sommano. Devi essere un Seguace o Devoto per poter lanciare questo incantesimo.

\textbf{Per ogni Successo Critico Magico} ottenuto nella Prova di Magia puoi aggiungere una creatura come bersaglio.

\smallskip\noindent\rule{\linewidth}{2pt} \index[Incantesimi]{Benedizione della Vita}\hypertarget{Benedizione della Vita}{}\smallskip\noindent{\textbf{Benedizione della Vita}}\pdfbookmark[3]{Benedizione della Vita}{Benedizione della Vita}
\noindent
\begin{description}[noitemsep, topsep=0pt, parsep=0pt, partopsep=0pt, leftmargin=0cm, labelwidth=2.8cm]
	\item[\textbf{Lista di Magia}]: Cura
	\item[\textbf{Livello}]: 3, Raro
	\item[\textbf{T. di Lancio}]: 2 Azioni
	\item[\textbf{Gittata}]: 9 metri
	\item[\textbf{Componenti}]: V, S
	\item[\textbf{Durata}]: 1 minuto, Concentrazione
\end{description}

Questo incantesimo conferisce speranza e vitalità. Scegli fino a 6 creature a gittata. Per la durata, ciascun bersaglio ha +2 ai Tiri Salvezza su Volontà e recupera 1 Punto Ferita a round.

\textbf{Se ottieni 2 Successo Critico Magico e sei un Devoto o Seguace di un Patrono buono} ogni round le creature scelte recuperano 2 Punti Ferita in più.

\smallskip\noindent\rule{\linewidth}{2pt} \index[Incantesimi]{Benedizione di Cattalm}\hypertarget{Benedizione di Cattalm}{}\smallskip\noindent{\textbf{Benedizione di Cattalm}}\pdfbookmark[3]{Benedizione di Cattalm}{Benedizione di Cattalm}
\noindent
\begin{description}[noitemsep, topsep=0pt, parsep=0pt, partopsep=0pt, leftmargin=0cm, labelwidth=2.8cm]
	\item[\textbf{Lista di Magia}]: Ammaliamento, Fuoco
	\item[\textbf{Livello}]: 3, Molto Raro
	\item[\textbf{T. di Lancio}]: 2 Azioni
	\item[\textbf{Gittata}]: 18 metri
	\item[\textbf{Componenti}]: V, S, M (uno spruzzo di aceto)
	\item[\textbf{Durata}]: Istantanea
\end{description}

Invochi l'ira di Cattalm sul tuo avversario. La creatura bersaglio subisce 4d6 di danno da fuoco, deve effettuare un Tiro Salvezza su Volontà o subire alla prima successiva prova di competenza oppure Tiro per Colpire o Tiro Salvezza una penalità di -1d6 e l'incantatore aumenta di uno la sua riserva di Punti Fato.

\textbf{Per ogni due Successo Critico Magico ottenuto} nella Prova di Magia puoi influenzare un'altra creatura.

\smallskip\noindent\rule{\linewidth}{2pt} \index[Incantesimi]{Benedizioni di Efrem}\hypertarget{Benedizioni di Efrem}{}\smallskip\noindent{\textbf{Benedizioni di Efrem}}\pdfbookmark[3]{Benedizioni di Efrem}{Benedizioni di Efrem}\label{Animal Shapes}
\noindent
\begin{description}[noitemsep, topsep=0pt, parsep=0pt, partopsep=0pt, leftmargin=0cm, labelwidth=2.8cm]
	\item[\textbf{Lista magia}] : Animali e Piante
	\item[\textbf{Livello}] : 8, Raro
	\item[\textbf{T. di Lancio}] : 2 Azioni
	\item[\textbf{Gittata}] : 9 metri
	\item[\textbf{Componenti}] : V, S, M (ossicini da un cimitero di animali)
	\item[\textbf{Durata}] : 1 ora per CM
\end{description}

Scegli fino a CM creature consenzienti che puoi vedere entro gittata. Ogni bersaglio si trasforma in una Bestia, di taglia piccola, media o grande, di tua scelta che abbia un Grado di Sfida pari o inferiore a 4. Puoi scegliere una forma diversa per ogni bersaglio. Nei round successivi puoi usare 2 Azioni per trasformare di nuovo i bersagli.

La trasformazione segue le regole standard della trasformazione animale.

Le statistiche di gioco di un bersaglio vengono sostituite dalle statistiche della Bestia scelta, ma il bersaglio conserva il suo tipo di creatura; Punti Ferita; Tratti, capacità di comunicare e punteggi di Intelligenza, Saggezza e Carisma. Le azioni del bersaglio sono limitate dall'anatomia della forma Bestia e non può lanciare incantesimi. L'equipaggiamento del bersaglio si fonde nella nuova forma e non può essere usato mentre è in quella forma.

L'incantesimo termina sulla creature se questa perde conoscenza.

\textbf{NOTA}: è necessario essere un Devoto o Seguace di Efrem o Shayalia per poter lanciare questo incantesimo.

\smallskip\noindent\rule{\linewidth}{2pt} \index[Incantesimi]{Benedizione di Ledyal}\hypertarget{Benedizione di Ledyal}{}\smallskip\noindent{\textbf{Benedizione di Ledyal}}\pdfbookmark[3]{Benedizione di Ledyal}{Benedizione di Ledyal}\label{Aura of Vitality}
\noindent
\begin{description}[noitemsep, topsep=0pt, parsep=0pt, partopsep=0pt, leftmargin=0cm, labelwidth=2.8cm]
	\item[\textbf{Lista di Magia}] : Necromanzia
	\item[\textbf{Livello}] : 3, Molto Raro
	\item[\textbf{T. di Lancio}] : 2 Azioni
	\item[\textbf{Gittata}] : Personale
	\item[\textbf{Componenti}] : V, S, M (una goccia del tuo sangue)
	\item[\textbf{Durata}] : Concentrazione, fino a 6 round
\end{description}

Un aura sacra si irradia da te. Qualsiasi creatura che incominci il round entro 9 metri da te viene curato di 1d6 Punti Ferita. Una creatura non viene curata per più di 3 round per incantesimo.

\textbf{NOTA}: è necessario essere un Devoto o Seguace di Ledyal o Sumkjr per poter lanciare questo incantesimo.

\smallskip\noindent\rule{\linewidth}{2pt} \index[Incantesimi]{Benedizione Superiore}\hypertarget{Benedizione Superiore}{}\smallskip\noindent{\textbf{Benedizione Superiore}}\pdfbookmark[3]{Benedizione Superiore}{Benedizione Superiore}
\noindent
\begin{description}[noitemsep, topsep=0pt, parsep=0pt, partopsep=0pt, leftmargin=0cm, labelwidth=2.8cm]
	\item[\textbf{Lista di Magia}]: Invocazione
	\item[\textbf{Livello}]: 2, Non Comune
	\item[\textbf{T. di Lancio}]: 1 Minuto
	\item[\textbf{Gittata}]: 18 metri
	\item[\textbf{Componenti}]: V, S, M (uno spruzzo di Acqua santa e 10 monete d'oro che l'incantesimo consuma)
	\item[\textbf{Durata}]: 1 ora
\end{description}

Benedici una creatura a tua scelta. La creatura entro la durata può aggiungere 1d6 ad un tiro prima di sapere se la prova (TC/TS/Prova) ha avuto successo o meno. Questo bonus può essere usato 2 volte nell'ora. Devi essere un Seguace o Devoto per poter lanciare questo incantesimo.

\textbf{Per ogni Successo Critico Magico} ottenuto nella Prova di Magia puoi aggiungere una creatura come bersaglio o aggiungere un ora alla durata.

\smallskip\noindent\rule{\linewidth}{2pt} \index[Incantesimi]{Benedizione Suprema}\hypertarget{Benedizione Suprema}{}\smallskip\noindent{\textbf{Benedizione Suprema}}\pdfbookmark[3]{Benedizione Suprema}{Benedizione Suprema}
\noindent
\begin{description}[noitemsep, topsep=0pt, parsep=0pt, partopsep=0pt, leftmargin=0cm, labelwidth=2.8cm]
	\item[\textbf{Lista di Magia}]: Invocazione
	\item[\textbf{Livello}]: 3, Raro
	\item[\textbf{T. di Lancio}]: 1 Reazione
	\item[\textbf{Gittata}]: 27 metri
	\item[\textbf{Componenti}]: V, S, M (uno spruzzo di Acqua santa, 25 monete d'oro che l'incantesimo consuma)
	\item[\textbf{Durata}]: Istantanea
\end{description}

Benedici una creatura a tua scelta. La creatura può ritirare due dadi di una singola prova prima di sapere se la prova ha avuto successo o meno. La creatura sceglie se prendere i nuovi tiri ottenuti o tenere i vecchi. Devi essere un Seguace o Devoto per poter lanciare questo incantesimo.

\textbf{Per ogni Successo Critico Magico} ottenuto nella Prova di Magia la creatura prende anche un +1 di bonus alla prova.

\smallskip\noindent\rule{\linewidth}{2pt} \index[Incantesimi]{Blocca Mostri}\hypertarget{Blocca Mostri}{}\smallskip\noindent{\textbf{Blocca Mostri}}\pdfbookmark[3]{Blocca Mostri}{Blocca Mostri}
\noindent
\begin{description}[noitemsep, topsep=0pt, parsep=0pt, partopsep=0pt, leftmargin=0cm, labelwidth=2.8cm]
	\item[\textbf{Lista di Magia}]: Ammaliamento
	\item[\textbf{Livello}]: 5, Comune
	\item[\textbf{T. di Lancio}]: 2 Azioni
	\item[\textbf{Gittata}]: 27 metri
	\item[\textbf{Componenti}]: V, S, M (un piccolo pezzo dritto di ferro)
	\item[\textbf{Durata}]: 1 minuto
\end{description}

Scegli una creatura a gittata e che puoi vedere. Il bersaglio deve superare un Tiro Salvezza su Volontà, o restare paralizzato per la durata. Questo incantesimo non ha effetto su non morti o costrutti. Al termine di ciascun suo round, il bersaglio può effettuare un altro Tiro Salvezza su Volontà. Se lo supera, per quel bersaglio l'incantesimo ha termine.

\textbf{Per ogni Successo Critico Magico} ottenuto nella Prova di Magia puoi aggiungere una creatura come bersaglio purché siano entro 9 metri l'una dall'altra.

\smallskip\noindent\rule{\linewidth}{2pt} \index[Incantesimi]{Blocca Persona}\hypertarget{Blocca Persona}{}\smallskip\noindent{\textbf{Blocca Persona}}\pdfbookmark[3]{Blocca Persona}{Blocca Persona}
\noindent
\begin{description}[noitemsep, topsep=0pt, parsep=0pt, partopsep=0pt, leftmargin=0cm, labelwidth=2.8cm]
	\item[\textbf{Lista di Magia}]: Ammaliamento
	\item[\textbf{Livello}]: 2, Comune
	\item[\textbf{T. di Lancio}]: 2 Azioni
	\item[\textbf{Gittata}]: 18 metri
	\item[\textbf{Componenti}]: V, S, M (un piccolo pezzo dritto di ferro)
	\item[\textbf{Durata}]: 1 minuto
\end{description}

Scegli un umanoide a gittata e che puoi vedere. L'incantesimo non ha effetto su creature con GS 4 o più. Il bersaglio deve superare un Tiro Salvezza su Volontà o restare paralizzato per la durata.

\textbf{Per ogni Successo Critico Magico} ottenuto nella Prova di Magia puoi aggiungere una creatura come bersaglio purché siano entro 9 metri l'una dall'altra.

\smallskip\noindent\rule{\linewidth}{2pt} \index[Incantesimi]{Blocca Persona Avanzato}\hypertarget{Blocca Persona Avanzato}{}\smallskip\noindent{\textbf{Blocca Persona Avanzato}}\pdfbookmark[3]{Blocca Persona Avanzato}{Blocca Persona Avanzato}
\noindent
\begin{description}[noitemsep, topsep=0pt, parsep=0pt, partopsep=0pt, leftmargin=0cm, labelwidth=2.8cm]
	\item[\textbf{Lista di Magia}]: Ammaliamento
	\item[\textbf{Livello}]: 4, Non Comune
	\item[\textbf{T. di Lancio}]: 2 Azioni
	\item[\textbf{Gittata}]: 18 metri, raggio 6 metri
	\item[\textbf{Componenti}]: V, S, M (un piccolo pezzo dritto di argento)
	\item[\textbf{Durata}]: 1 minuto
\end{description}

Blocchi fino a 2d4 creature entro 18 metri da te in un raggio di 6 metri. L'incantesimo non ha effetto su creature con GS 6 o più. I bersagli devono superare un Tiro Salvezza su Volontà o restare paralizzati per la durata, il Tiro Salvezza può essere ripetuto quando subiscono degli attacchi.

\textbf{Per ogni Successo Critico Magico} ottenuto nella Prova di Magia puoi aggiungere 2 punti ai 2d4 tirati.

\smallskip\noindent\rule{\linewidth}{2pt} \index[Incantesimi]{Bocca Magica}\hypertarget{Bocca Magica}{}\smallskip\noindent{\textbf{Bocca Magica}}\pdfbookmark[3]{Bocca Magica}{Bocca Magica}
\noindent
\begin{description}[noitemsep, topsep=0pt, parsep=0pt, partopsep=0pt, leftmargin=0cm, labelwidth=2.8cm]
	\item[\textbf{Lista di Magia}]: Illusione
	\item[\textbf{Livello}]: 2, Comune
	\item[\textbf{T. di Lancio}]: 1 minuto
	\item[\textbf{Gittata}]: 9 metri
	\item[\textbf{Componenti}]: V, S, M (un piccolo pezzo di favo e polvere di giada del valore di almeno 10 mo, che l'incantesimo consuma)
	\item[\textbf{Durata}]: Fino a che dissolto
\end{description}

Impianti un messaggio in un oggetto a gittata, messaggio che viene pronunciato quando si soddisfa la condizione di attivazione. Scegli un oggetto che puoi vedere e che non sia indossato o trasportato da un'altra creatura. Poi pronuncia il messaggio che deve essere di 25 parole o meno ma può essere distribuito in un periodo di massimo 10 minuti. Infine determina la circostanza che attiverà l'incantesimo affinché questo trasmetta il tuo messaggio.

Quando la circostanza si manifesta, una bocca magica appare sull'oggetto e recita il messaggio con la tua voce e allo stesso volume con cui l'hai pronunciato. Se l'oggetto da te scelto ha una bocca o qualcosa che assomiglia a una bocca (per esempio, la bocca di una statua), la bocca magica appare così che le parole sembrino provenire dalla bocca dell'oggetto. Quando lanci questo incantesimo, puoi far sì che l'incantesimo termini dopo aver trasmesso il suo messaggio o che perduri e ripeta il messaggio ogni volta che la condizione si attiva.

La circostanza di attivazione può essere generica o dettagliata quanto desideri, ma deve essere basata su condizioni visibili o udibili che avvengono entro 9 metri dall'oggetto. Per esempio potresti istruire la bocca di parlare quando una qualsiasi creatura si avvicina entro 9 metri dall'oggetto o quando una campanella d'argento suona entro 9 metri da esso.

\smallskip\noindent\rule{\linewidth}{2pt} \index[Incantesimi]{Bolla vitale}\hypertarget{Bolla vitale}{}\smallskip\noindent{\textbf{Bolla vitale}}\pdfbookmark[3]{Bolla vitale}{Bolla vitale}
\noindent
\begin{description}[noitemsep, topsep=0pt, parsep=0pt, partopsep=0pt, leftmargin=0cm, labelwidth=2.8cm]
	\item[\textbf{Lista di Magia}]: Aria, Abiurazione
	\item[\textbf{Livello}]: 4, Non Comune
	\item[\textbf{T. di Lancio}]: 1 minuto
	\item[\textbf{Gittata}]: 9 metri
	\item[\textbf{Componenti}]: V, S, M (polvere di argento e diamante per 100 mo che l'incantesimo consuma)
	\item[\textbf{Durata}]: 1 ora per CM
\end{description}

Puoi creare fino a 6 bolle che circondano le creature da te designate.
La durata totale è di 1 ora per punto in Competenza Magica suddivisa a piacimento tra le creature nelle bolle.
Questa bolla permette ai soggetti di respirare liberamente, anche sott'acqua o nel vuoto, e li rende immuni ai gas e ai vapori nocivi, incluse le malattie e i veleni da inalazione e gli incantesimi come Nebbia Nauseante e Nebbia mortale. La bolla protegge i soggetti dalle temperature estreme (ma non che causino danno ogni round) e dalle pressioni estreme.

Bolla vitale non fornisce protezione dall'energia negativa o positiva (ad esempio sui piani dell'Energia Negativa e Positiva), la capacità di vedere in condizioni di scarsa visibilità (come nel fumo o nella nebbia), né la capacità di muoversi o agire normalmente in condizioni che impediscono il movimento (come sott'acqua).

\smallskip\noindent\rule{\linewidth}{2pt} \index[Incantesimi]{Caduta Piuma}\hypertarget{Caduta Piuma}{}\smallskip\noindent{\textbf{Caduta Piuma}}\pdfbookmark[3]{Caduta Piuma}{Caduta Piuma}\label{cadutapiuma}\hypertarget{cadutapiuma}
\noindent
\begin{description}[noitemsep, topsep=0pt, parsep=0pt, partopsep=0pt, leftmargin=0cm, labelwidth=2.8cm]
	\item[\textbf{Lista di Magia}]: Aria
	\item[\textbf{Livello}]: 1, Comune
	\item[\textbf{T. di Lancio}]: 1 Reazione, che effettui quando tu o una creatura entro 18 metri da te cade
	\item[\textbf{Gittata}]: 18 metri
	\item[\textbf{Componenti}]: V, M (una piccola piuma o un pezzo di piuma)
	\item[\textbf{Durata}]: 1 minuto
\end{description}

Scegli fino a 2 creature a gittata. La velocità di discesa di una creatura che cade diminuisce a 18 metri per round fino al termine dell'incantesimo. Se la creatura atterra prima del termine dell'incantesimo, non subisce danni da caduta e può atterrare sui suoi piedi; per quella creatura l'incantesimo ha termine.

\textbf{Per ogni Successo Critico Magico} ottenuto nella Prova di Magia puoi spostarti lateralmente di 1 metro od influenzare un altra creatura.

\smallskip\noindent\rule{\linewidth}{2pt} \index[Incantesimi]{Calmare Emozioni}\hypertarget{Calmare Emozioni}{}\smallskip\noindent{\textbf{Calmare Emozioni}}\pdfbookmark[3]{Calmare Emozioni}{Calmare Emozioni}
\noindent
\begin{description}[noitemsep, topsep=0pt, parsep=0pt, partopsep=0pt, leftmargin=0cm, labelwidth=2.8cm]
	\item[\textbf{Lista di Magia}]: Ammaliamento
	\item[\textbf{Livello}]: 2, Comune
	\item[\textbf{T. di Lancio}]: 2 Azioni
	\item[\textbf{Gittata}]: 18 metri
	\item[\textbf{Componenti}]: V, S
	\item[\textbf{Durata}]: Concentrazione, massimo 1 minuto
\end{description}

Tenti di sopprimere le forti emozioni in un gruppo di persone. Ogni umanoide in una sfera di 6 metri di raggio centrata su di un punto a gittata da te scelto, deve effettuare un Tiro Salvezza su Volontà; se lo desidera, una creatura può scegliere di fallire questo Tiro Salvezza. Se una creatura fallisce il Tiro Salvezza, scegli uno di questi due effetti.

\emph{Placare}. Puoi sopprimere qualsiasi effetto che renda il bersaglio Affascinato o spaventato. Quando questo incantesimo termina gli effetti soppressi riprendono purché la loro durata non sia nel frattempo esaurita.

\emph{Indifferenza}. Puoi rendere un bersaglio indifferente nei confronti di una creatura di tua scelta, verso la quale è ostile. Questa indifferenza termina se il bersaglio viene attaccato o danneggiato da un incantesimo o se vede uno dei suoi amici venir danneggiato. Quando l'incantesimo termina la creatura diventa di nuovo ostile a meno che il Narratore non determini diversamente.

\smallskip\noindent\rule{\linewidth}{2pt} \index[Incantesimi]{Camminare nell'aria}\hypertarget{Camminare nell'aria}{}\smallskip\noindent{\textbf{Camminare nell'aria}}\pdfbookmark[3]{Camminare nell'aria}{Camminare nell'aria}
\noindent
\begin{description}[noitemsep, topsep=0pt, parsep=0pt, partopsep=0pt, leftmargin=0cm, labelwidth=2.8cm]
	\item[\textbf{Lista di Magia}]: Aria
	\item[\textbf{Livello}]: 4, Non Comune
	\item[\textbf{T. di Lancio}]: 2 Azioni
	\item[\textbf{Gittata}]: 18 metri
	\item[\textbf{Componenti}]: V, S, M (una manciata di fagioli ed Acqua santa)
	\item[\textbf{Durata}]: 1 Turno
\end{description}

Per la durata una creatura da te scelta a gittata, che puoi vedere, e consenziente può camminare nell'aria come se comminasse su solido terreno. Se una creatura è per aria quando l'effetto ha termine, la creatura scende 18 metri per round per un minuto dopo di che cade per la distanza rimanente.

\textbf{Per ogni Successo Critico Magico} ottenuto nella Prova di Magia puoi puoi aggiungere una creatura come bersaglio. Quando lanci l'incantesimo, le creature bersaglio devono trovarsi entro 9 metri l'una dall'altra.

\smallskip\noindent\rule{\linewidth}{2pt} \index[Incantesimi]{Camminare nel Vento}\hypertarget{Camminare nel Vento}{}\smallskip\noindent{\textbf{Camminare nel Vento}}\pdfbookmark[3]{Camminare nel Vento}{Camminare nel Vento}
\noindent
\begin{description}[noitemsep, topsep=0pt, parsep=0pt, partopsep=0pt, leftmargin=0cm, labelwidth=2.8cm]
	\item[\textbf{Lista di Magia}]: Aria
	\item[\textbf{Livello}]: 6, Non Comune
	\item[\textbf{T. di Lancio}]: 1 minuto
	\item[\textbf{Gittata}]: 9 metri
	\item[\textbf{Componenti}]: V, S, M (fuoco e Acqua santa)
	\item[\textbf{Durata}]: 8 ore
\end{description}

Per la durata, tu e fino ad altre dieci creature consenzienti a gittata, che puoi vedere, assumete forma gassosa, diventando nubi. Mentre è in forma di nube una creatura ha velocità di volo 90 metri e ha resistenza ai danni dalle armi non magiche. Ritornare alla forma normale o ritornare a nube richiede 1 minuto, durante il quale la creatura è inabile e non può muoversi. Se una creatura è in forma di nube e sta volando quando l'effetto ha termine, la creatura scende 18 metri per round al minuto finché non atterra, al sicuro. Se non riesce ad atterrare dopo 1 minuto, la creatura cadrà per la distanza rimanente.

\smallskip\noindent\rule{\linewidth}{2pt} \index[Incantesimi]{Camminare sull'Acqua}\hypertarget{Camminare sull'Acqua}{}\smallskip\noindent{\textbf{Camminare sull'Acqua}}\pdfbookmark[3]{Camminare sull'Acqua}{Camminare sull'Acqua}
\noindent
\begin{description}[noitemsep, topsep=0pt, parsep=0pt, partopsep=0pt, leftmargin=0cm, labelwidth=2.8cm]
	\item[\textbf{Lista di Magia}]: Acqua
	\item[\textbf{Livello}]: 3, Comune
	\item[\textbf{T. di Lancio}]: 2 Azioni
	\item[\textbf{Gittata}]: 9 metri
	\item[\textbf{Componenti}]: V, S, M (un pezzo di sughero)
	\item[\textbf{Durata}]: 1 ora
\end{description}

Questo incantesimo conferisce la capacità di muoversi attraverso superfici liquide (come acqua, acido, fango, neve, sabbie mobili o lava) come se fossero innocuo terreno solido (le creature che attraversano la lava fusa possono comunque subire danni dal calore o sciogliersi nell'acido). Fino a dieci creature consenzienti a gittata, e che puoi vedere, ricevono questa capacità per tutta la durata. Se il tuo bersaglio è immerso in un liquido, l'incantesimo riporta il bersaglio in superficie del liquido a una velocità di 9 metri per round.

\smallskip\noindent\rule{\linewidth}{2pt} \index[Incantesimi]{Charme su Persone}\hypertarget{Charme su Persone}{}\smallskip\noindent{\textbf{Charme su Persone}}\pdfbookmark[3]{Charme su Persone}{Charme su Persone}
\noindent
\begin{description}[noitemsep, topsep=0pt, parsep=0pt, partopsep=0pt, leftmargin=0cm, labelwidth=2.8cm]
	\item[\textbf{Lista di Magia}]: Ammaliamento
	\item[\textbf{Livello}]: 1, Comune
	\item[\textbf{T. di Lancio}]: 2 Azioni
	\item[\textbf{Gittata}]: 9 metri
	\item[\textbf{Componenti}]: V, S
	\item[\textbf{Durata}]: 1 ora
\end{description}

Cerchi di affascinare un umanoide a gittata e che puoi vedere. Egli deve effettuare un Tiro Salvezza su Volontà e avrà +1d6 se sta combattendo contro di te o i tuoi alleati. Se fallisce il Tiro Salvezza è Affascinato da te fino al termine dell'incantesimo o finché tu o i tuoi alleati non gli facciate qualcosa di nocivo. La creatura affascinata ti considera un amichevole conoscente. Quando l'incantesimo termina la creatura è consapevole di essere stata affascinata da te. Ogni qual volta la creatura è minacciata da te o da un tuo amico può rifare il Tiro Salvezza con un bonus di +2.

\textbf{Per ogni Successo Critico Magico} ottenuto nella Prova di Magia puoi puoi aggiungere una creatura come bersaglio. Quando lanci l'incantesimo, le creature bersaglio devono trovarsi entro 9 metri l'una dall'altra.

\smallskip\noindent\rule{\linewidth}{2pt} \index[Incantesimi]{Campo Anti-Magia}\hypertarget{Campo Anti-Magia}{}\smallskip\noindent{\textbf{Campo Anti-Magia}}\pdfbookmark[3]{Campo Anti-Magia}{Campo Anti-Magia}
\noindent
\begin{description}[noitemsep, topsep=0pt, parsep=0pt, partopsep=0pt, leftmargin=0cm, labelwidth=2.8cm]
	\item[\textbf{Lista di Magia}]: Abiurazione
	\item[\textbf{Livello}]: 8, Raro
	\item[\textbf{T. di Lancio}]: 2 Azioni
	\item[\textbf{Gittata}]: Personale (sfera di 3 metri di raggio)
	\item[\textbf{Componenti}]: V, S, M (un pizzico di ferro in polvere o lima di ferro)
	\item[\textbf{Durata}]: Concentrazione, massimo 1 ora
\end{description}

Vieni circondato da una sfera invisibile di anti-magia di 3 metri di raggio. Quest'area è separata dall'energia magica che permea la Terra. All'interno della sfera non si possono lanciare incantesimi, le creature richiamate scompaiono e anche gli oggetti magici diventano normali. Fino al termine dell'incantesimo la sfera si muove con te centrata su di te. Gli incantesimi e altri effetti magici, eccetto quelli creati da un artefatto o Patrono, sono soppressi all'interno della sfera né vi possono penetrare. Uno slot speso per lanciare un incantesimo soppresso è consumato. Mentre un effetto è soppresso non funziona ma il tempo che trascorre soppresso è conteggiato per la sua durata.

\medskip

\noindent\emph{Effetti con Bersaglio}. Incantesimi e altri effetti magici, come Dardo arcano e \hyperlink{Charme su Persone}{Charme su Persone}, che prendono come bersaglio una creatura o un oggetto all'interno della sfera non hanno effetto su quel bersaglio.

\emph{Aree di Magia}. L'area di un altro incantesimo o effetto magico, come palla di fuoco, non può estendersi all'interno della sfera. Se la sfera si sovrappone a un'area di magia, la parte di quell'area coperta dalla sfera viene soppressa. Per esempio, le fiamme generate da un muro di fuoco vengono soppresse all'interno della sfera, creando un buco nel muro se la sovrapposizione è sufficientemente grande. Incantesimi. Qualsiasi incantesimo o altro effetto magico attivo su di una creatura od oggetto all'interno della sfera viene soppresso finché la creatura o l'oggetto si trovano all'interno della sfera.

\emph{Oggetti Magici}. Le proprietà e poteri degli oggetti magici vengono soppressi dalla sfera. Per esempio, una spada lunga +1 all'interno della sfera funziona come una spada lunga non magica. Le proprietà e i poteri delle armi magiche vengono soppressi se sono usati contro un bersaglio all'interno della sfera o impugnate da un attaccante dentro la sfera. Se un'arma magica o munizione magica lascia interamente la sfera (per esempio, se tiri una freccia magica o scagli una lancia magica a un bersaglio al di fuori della sfera), la magia dell'oggetto non è più soppressa non appena esce dalla sfera.

\emph{Magia di Viaggio}. Il teletrasporto e il viaggio planare non funzionano all'interno della sfera, che la sfera sia il punto di destinazione o di partenza di questo viaggio magico. All'interno della sfera, un portale verso un altro luogo, mondo, o piano di esistenza, così come uno spazio extradimensionale come quello creato dall'incantesimo trucco della corda, resta chiuso.

\emph{Creature e Oggetti}. All'interno della sfera, una creatura o oggetto evocati o creati dalla magia svaniscono temporaneamente dall'esistenza. La creatura od oggetto riappare istantaneamente una volta che lo spazio occupato da essa non si trova più all'interno della sfera.

\emph{Dissolvi magie}. Gli incantesimi e gli effetti magici come dissolvi magie non hanno effetto sulla sfera. Allo stesso modo le sfere create da altri incantesimi campo antimagia non si annullano vicendevolmente.

\smallskip\noindent\rule{\linewidth}{2pt} \index[Incantesimi]{Camuffare Sé Stesso}\hypertarget{Camuffare Sé Stesso}{}\smallskip\noindent{\textbf{Camuffare Sé Stesso}}\pdfbookmark[3]{Camuffare Se Stesso}{Camuffare Se Stesso}
\noindent
\begin{description}[noitemsep, topsep=0pt, parsep=0pt, partopsep=0pt, leftmargin=0cm, labelwidth=2.8cm]
	\item[\textbf{Lista di Magia}]: Illusione
	\item[\textbf{Livello}]: 1, Comune
	\item[\textbf{T. di Lancio}]: 2 Azioni
	\item[\textbf{Gittata}]: Personale
	\item[\textbf{Componenti}]: V, S
	\item[\textbf{Durata}]: 1 ora
\end{description}

Cambi il tuo aspetto, assieme a quello dei tuoi abiti, armatura, armi e altri oggetti che indossi, fino al termine dell'incantesimo o finché non impieghi un'Azione per interrompere l'incantesimo. Puoi apparire 30 centimetri più basso o più alto, magro, grasso o una via di mezzo. Non puoi modificare la tua conformazione fisica quindi devi adottare una forma che abbia la medesima distribuzione di arti. Per tutto il resto, l'illusione è limitata solo dalla tua fantasia.

I cambi apportati da questo incantesimo non sono in grado di sostenere un'ispezione fisica. Per esempio, se usi questo incantesimo per aggiungere un cappello al tuo abbigliamento, gli oggetti attraversano il cappello, e chiunque lo tocchi non avvertirebbe nulla e finirebbe per toccarti la testa e i capelli. Se usi questo incantesimo per apparire più magro di quello che sei, la mano di una persona che provasse a toccarti rimbalzerebbe su di te, mentre alla vista sembrerebbe fermarsi a mezz'aria. Per distinguere il tuo camuffamento, una creatura può usare 2 Azioni per ispezionare il tuo aspetto e deve superare una prova di Consapevolezza+4 contro la DC del Tiro Salvezza dell'incantesimo.

\smallskip\noindent\rule{\linewidth}{2pt} \index[Incantesimi]{Capanna}\hypertarget{Capanna}{}\smallskip\noindent{\textbf{Capanna}}\pdfbookmark[3]{Capanna}{Capanna}
\noindent
\begin{description}[noitemsep, topsep=0pt, parsep=0pt, partopsep=0pt, leftmargin=0cm, labelwidth=2.8cm]
	\item[\textbf{Lista di Magia}]: Invocazione
	\item[\textbf{Livello}]: 3, Non Comune
	\item[\textbf{T. di Lancio}]: 1 minuto
	\item[\textbf{Gittata}]: Personale (semisfera di 3 metri di raggio)
	\item[\textbf{Componenti}]: V, S, M (una scheggia di diamante del valore di 50 mo che l'incantesimo consuma)
	\item[\textbf{Durata}]: 8 ore
\end{description}

Una mezza sfera di forza immobile del raggio di 3 metri si forma intorno e sopra di te, restando stazionaria per la durata. L'incantesimo termina se lasci l'area. Otto creature di taglia Media o inferiore possono entrare nella cupola insieme a te. L'incantesimo fallisce se l'area include una creatura più grande o più di nove creature. Le creature e gli oggetti all'interno della cupola, quando lanci questo incantesimo, la possono attraversare liberamente. Tutte le altre creature e oggetti devono effettuare un Tiro Salvezza su Tempra o sono impossibilitati dall'attraversarla per quel round. Incantesimi e altri effetti magici possono estendersi oltre la cupola o attraversarla se non sono Trucchetti. L'atmosfera all'interno dello spazio è confortevole e asciutta quale che sia il clima all'esterno.

Fino al termine dell'incantesimo puoi comandare che l'illuminazione interna sia piena, fioca o buio. La cupola è opaca dall'esterno, fornisce copertura media, di qualsiasi colore tu scelga, ma è trasparente dall'interno.

\textbf{Per ogni Successo Critico Magico} ottenuto nella Prova di Magia l'incantesimo dura 2 ore in più.

\smallskip\noindent\rule{\linewidth}{2pt} \index[Incantesimi]{Caratteristica Potenziata}\hypertarget{Caratteristica Potenziata}{}\smallskip\noindent{\textbf{Caratteristica Potenziata}}\pdfbookmark[3]{Caratteristica Potenziata}{Caratteristica Potenziata}
\noindent
\begin{description}[noitemsep, topsep=0pt, parsep=0pt, partopsep=0pt, leftmargin=0cm, labelwidth=2.8cm]
	\item[\textbf{Lista di Magia}]: Trasmutazione
	\item[\textbf{Livello}]: 2, Comune
	\item[\textbf{T. di Lancio}]: 2 Azioni
	\item[\textbf{Gittata}]: Contatto
	\item[\textbf{Componenti}]: V, S, M (pelo o piuma di una bestia)
	\item[\textbf{Durata}]: massimo 10 minuti
\end{description}

Conferisci un potenziamento magico a una creatura con cui sei in contatto. Scegli uno degli effetti seguenti; il bersaglio ottiene quell'effetto fino al termine dell'incantesimo.

\begin{itemize}[leftmargin=*] \setlength{\itemsep}{0pt}
	\item \emph{Astuzia della Volpe}. Il bersaglio ha +1d6 alle Competenze base basate su di Intelligenza e Forza
	\item \emph{Forza del Toro}. Il bersaglio ha +1d6 alle Competenze base basate su Forza e la sua capacità di Ingombro raddoppia.
	\item \emph{Grazia del Energia Luminosa}. Il bersaglio ha +1d6 alle Competenze base basate su Destrezza. Inoltre, qualora non sia inabile, non subisce danni dalle cadute di 6 metri o meno.
	\item \emph{Resistenza della Nutria}. Il bersaglio ha +1d6 alle Competenze base basate su  Costituzione. Ottiene anche 2d6 Punti Ferita temporanei, che vengono persi alla fine dell'incantesimo.
	\item \emph{Saggezza del Dobi}. Il bersaglio ha +1d6 alle Competenze base basate su  Saggezza e +2 alla Consapevolezza.
	\item \emph{Splendore della Topi}. Il bersaglio ha +1d6 alle Competenze base basate su  Carisma.
\end{itemize}

\textbf{Per ogni Successo Critico Magico} ottenuto nella Prova di Magia puoi prendere come bersaglio un'ulteriore creatura

\smallskip\noindent\rule{\linewidth}{2pt} \index[Incantesimi]{Carne in Pietra - Pietra in Carne}\hypertarget{Carne in Pietra - Pietra in Carne}{}\smallskip\noindent{\textbf{Carne in Pietra - Pietra in Carne}}\pdfbookmark[3]{Carne in Pietra}{Carne in Pietra}\hypertarget{Pietra in Carne}{}\index[Incantesimi]{Carne in Pietra}\index[Incantesimi]{Pietra in Carne}\hypertarget{Carbe in Pietra}{}\pdfbookmark[3]{Pietra in Carne}{Pietra in Carne}
\noindent
\begin{description}[noitemsep, topsep=0pt, parsep=0pt, partopsep=0pt, leftmargin=0cm, labelwidth=2.8cm]
	\item[\textbf{Lista di Magia}]: Terra
	\item[\textbf{Livello}]: 6, Non Comune - Raro
	\item[\textbf{T. di Lancio}]: 2 Azioni
	\item[\textbf{Gittata}]: 18 metri
	\item[\textbf{Componenti}]: V, S, M (un pizzico di limo, acqua e terra)
	\item[\textbf{Durata}]: Permanente
\end{description}

Cerchi di trasformare in pietra una creatura a gittata che puoi vedere. Se il corpo del bersaglio è fatto di carne la creatura diventa Rallentata 1/6r e deve effettuare un Tiro Salvezza su Tempra. Se fallisce il Tiro Salvezza diventa invece Rallentata 1/10 minuti e la sua carne comincia a indurirsi.
La creatura che fallisce il Tiro Salvezza iniziale il round dopo deve effettuare un nuovo Tiro Salvezza su Tempra. Se supera il Tiro Salvezza con successo non ci sono ulteriori effetti. Se fallisce questo nuovo Tiro Salvezza viene trasformata in pietra e resta vittima della condizione pietrificato per la durata.

Se supera il primo Tiro Salvezza la creatura non subisce ulteriori effetti.

Se la creatura viene danneggiata fisicamente mentre è pietrificata, soffre di deformità simili ai danni arrecati alla pietra, se ritorna al suo stato originale.

L'incantesimo \emph{Pietra in Carne} fa tornare una creatura di carne purché non sia stata trasformata da più di un anno. L'incantesimo Dissolvi Magia non è in grado di annullarne gli effetti.

\smallskip\noindent\rule{\linewidth}{2pt} \index[Incantesimi]{Cecità/Sordità}\hypertarget{Cecità/Sordità}{}\smallskip\noindent{\textbf{Cecità/Sordità}}\pdfbookmark[3]{Cecita' Sordita'}{Cecita' Sordita'}
\noindent
\begin{description}[noitemsep, topsep=0pt, parsep=0pt, partopsep=0pt, leftmargin=0cm, labelwidth=2.8cm]
	\item[\textbf{Lista di Magia}]: Necromanzia
	\item[\textbf{Livello}]: 2, Comune
	\item[\textbf{T. di Lancio}]: 2 Azioni
	\item[\textbf{Gittata}]: 9 metri
	\item[\textbf{Componenti}]: V
	\item[\textbf{Durata}]: 1 minuto
\end{description}

Puoi accecare o assordare un nemico. Scegli una creatura a gittata e che puoi vedere. Il bersaglio deve effettuare un Tiro Salvezza su Tempra. Se lo fallisce il bersaglio è accecato o assordato (a tua scelta) per la durata.

\textbf{Per ogni due Successo Critico Magico ottenuto} nella Prova di Magia puoi aggiungere un altro bersaglio in gittata. Se fai 3 Successo Critico Magico il bersaglio è influenzato dall'incantesimo per tutto il giorno.

\smallskip\noindent\rule{\linewidth}{2pt} \index[Incantesimi]{Cecità/Sordità Avanzata}\hypertarget{Cecità/Sordità Avanzata}{}\smallskip\noindent{\textbf{Cecità/Sordità Avanzata}}\pdfbookmark[3]{Cecita' Sordita' Avanzata}{Cecita' Sordita' Avanzata}
\noindent
\begin{description}[noitemsep, topsep=0pt, parsep=0pt, partopsep=0pt, leftmargin=0cm, labelwidth=2.8cm]
	\item[\textbf{Lista di Magia}]: Necromanzia
	\item[\textbf{Livello}]: 3, Non Comune
	\item[\textbf{T. di Lancio}]: 2 Azioni
	\item[\textbf{Gittata}]: 36 metri
	\item[\textbf{Componenti}]: V,S,M (del cerume oppure un pezzo di stoffa nera)
	\item[\textbf{Durata}]: 10 minuti
\end{description}

Puoi accecare o assordare un nemico. Scegli una creatura a gittata e che puoi vedere. Il bersaglio deve effettuare un Tiro Salvezza su Tempra. Se lo fallisce, il bersaglio è accecato o assordato (a tua scelta) per la durata.

\textbf{Per ogni Successo Critico Magico} ottenuto nella Prova di Magia puoi prendere come bersaglio una creatura aggiuntiva.

\textbf{Tiro Salvezza Fallimento Critico}: in caso di Fallimento Critico l'effetto è permanente.

\smallskip\noindent\rule{\linewidth}{2pt} \index[Incantesimi]{Celare}\hypertarget{Celare}{}\smallskip\noindent{\textbf{Celare}}\pdfbookmark[3]{Celare}{Celare}
\noindent
\begin{description}[noitemsep, topsep=0pt, parsep=0pt, partopsep=0pt, leftmargin=0cm, labelwidth=2.8cm]
	\item[\textbf{Lista di Magia}]: Trasmutazione
	\item[\textbf{Livello}]: 7, Raro
	\item[\textbf{T. di Lancio}]: 2 Azioni
	\item[\textbf{Gittata}]: Contatto
	\item[\textbf{Componenti}]: V, S, M (una polvere composta da polvere di diamante, smeraldo, rubino e zaffiro del valore di almeno 50000 mo, che l'incantesimo consuma)
	\item[\textbf{Durata}]: Fino a che dissolto
\end{description}

Tramite questo incantesimo, una creatura consenziente o un oggetto può essere nascosto, impossibile da individuare per la durata. Eseguendo questo incantesimo ed entrando in contatto con un bersaglio questo diventa invisibile e non può essere preso come bersaglio dagli incantesimi di divinazione, né percepito da sensori di scrutamento creati da incantesimi di divinazione.

Se il bersaglio è una creatura, cade in uno stato di animazione sospesa. Per lui il tempo cessa di scorrere e non invecchia.

Puoi predisporre una condizione per cui l'incantesimo termini anticipatamente. La condizione può essere qualsiasi cosa tu voglia, ma deve avvenire o essere visibile entro 1,5 chilometri dal bersaglio. Esempi includono \emph{al prossimo giudizio dei Patroni} o \emph{quando il tarrasque si risveglia}. Questo incantesimo termina anche qualora il bersaglio subisca danni.

\smallskip\noindent\rule{\linewidth}{2pt} \index[Incantesimi]{Cerchio d'Invisibilità}\hypertarget{Cerchio d'Invisibilità}{}\smallskip\noindent{\textbf{Cerchio d'Invisibilità}}\pdfbookmark[3]{Cerchio di Invisibilita'}{Cerchio di Invisibilita'}
\noindent
\begin{description}[noitemsep, topsep=0pt, parsep=0pt, partopsep=0pt, leftmargin=0cm, labelwidth=2.8cm]
	\item[\textbf{Lista di Magia}]: Illusione
	\item[\textbf{Livello}]: 3, Non Comune
	\item[\textbf{T. di Lancio}]: 2 Azioni
	\item[\textbf{Gittata}]: Tocco
	\item[\textbf{Componenti}]: V, S, M (un pizzico di polvere di diamante)
	\item[\textbf{Durata}]: 1 minuto per CM
\end{description}

Questo incantesimo funziona come invisibilità, ma ha effetto su tutte le creature toccate dall'incantatore che in seguito restano entro 3 metri da lui. I soggetti che si allontanano oltre il cerchio d'invisibilità tornano immediatamente visibili o se rompono le condizioni per mantenere l'invisibilità.


\smallskip\noindent\rule{\linewidth}{2pt} \index[Incantesimi]{Cerchio Magico}\hypertarget{Cerchio Magico}{}\smallskip\noindent{\textbf{Cerchio Magico}}\pdfbookmark[3]{Cerchio Magico}{Cerchio Magico}
\noindent
\begin{description}[noitemsep, topsep=0pt, parsep=0pt, partopsep=0pt, leftmargin=0cm, labelwidth=2.8cm]
	\item[\textbf{Lista di Magia}]: Abiurazione
	\item[\textbf{Livello}]: 3, Comune
	\item[\textbf{T. di Lancio}]: 1 minuto
	\item[\textbf{Gittata}]: 3 metri
	\item[\textbf{Componenti}]: V, S, M (Acqua santa o argento e ferro in polvere del valore di almeno 100 mo, che l'incantesimo consuma)
	\item[\textbf{Durata}]: 1 ora
\end{description}

Crei un cilindro di energia magica di 3 metri di raggio e alto 6 metri, centrato su di un punto del terreno a gittata e che puoi vedere. Rune luminose compaiono dovunque il cilindro si intersechi con il pavimento o altra superficie.

Scegli uno o più dei seguenti tipi di creature: celestiali, elementali, fatati, demoni o non morti. Il circolo influisce su di una creatura del tipo scelto nei modi seguenti:

\medskip

- La creatura non può entrare consapevolmente nel cilindro tramite alcun mezzo non magico. Se la creatura prova a usare il teletrasporto o il viaggio tra i piani per farlo deve prima superare un Tiro Salvezza su Volontà.

- La creatura ha -1d6 ai Tiri per Colpire contro i bersagli all'interno del cilindro.

- I bersagli all'interno del cilindro non possono essere affascinati, spaventati o posseduti dalla creatura. Quando lanci questo incantesimo, puoi decidere che la magia operi in direzione opposta, impedendo a una creatura del tipo specificato di lasciare il cilindro e proteggendo i bersagli all'esterno.

\textbf{Per ogni Successo Critico Magico} ottenuto nella Prova di Magia puoi aumentare la durata di 1 ora.

\smallskip\noindent\rule{\linewidth}{2pt} \index[Incantesimi]{Cerchio di Morte}\hypertarget{Cerchio di Morte}{}\smallskip\noindent{\textbf{Cerchio di Morte}}\pdfbookmark[3]{Cerchio di Morte}{Cerchio di Morte}
\noindent
\begin{description}[noitemsep, topsep=0pt, parsep=0pt, partopsep=0pt, leftmargin=0cm, labelwidth=2.8cm]
	\item[\textbf{Lista di Magia}]: Invocazione
	\item[\textbf{Livello}]: 6, Molto Raro
	\item[\textbf{T. di Lancio}]: 2 Azioni
	\item[\textbf{Gittata}]: 45 metri
	\item[\textbf{Componenti}]: V, S, M (una perla nera ridotta in polvere del valore di almeno 500 mo)
	\item[\textbf{Durata}]: Istantanea
\end{description}

Una sfera di energia negativa del raggio di 18 metri, erutta in un punto a gittata. Ogni creatura in quell'area deve effettuare un Tiro Salvezza su Tempra. La creatura subisce 8d6 danni da Vuoto se fallisce il Tiro Salvezza, o la metà di questi danni se lo supera.

\textbf{Per ogni Successo Critico Magico} ottenuto nella Prova di Magia il danno aumenta di 4d6.

\textbf{Tiro Salvezza Successo/Fallimento Critico}: In caso di Fallimento Critico il danno raddoppia, in caso di Successo Critico il danno viene ulteriormente dimezzato

\smallskip\noindent\rule{\linewidth}{2pt} \index[Incantesimi]{Cerchio di Teletrasporto}\hypertarget{Cerchio di Teletrasporto}{}\smallskip\noindent{\textbf{Cerchio di Teletrasporto}}\pdfbookmark[3]{Cerchio di Teletrasporto}{Cerchio di Teletrasporto}
\noindent
\begin{description}[noitemsep, topsep=0pt, parsep=0pt, partopsep=0pt, leftmargin=0cm, labelwidth=2.8cm]
	\item[\textbf{Lista di Magia}]: Evocazione
	\item[\textbf{Livello}]: 5, Non Comune
	\item[\textbf{T. di Lancio}]: 1 minuto
	\item[\textbf{Gittata}]: 3 metri
	\item[\textbf{Componenti}]: V, M (gessi e inchiostri rari infusi di gemme preziose del valore di almeno 50 mo, che l'incantesimo consuma)
	\item[\textbf{Durata}]: 1 round
\end{description}

Mentre lanci l'incantesimo, tracci un cerchio di 3 metri di diametro sul pavimento, inscritto con sigilli che collegano il posto in cui ti trovi a un cerchio di teletrasporto permanente di tua scelta, di cui conosci la sequenza dei sigilli e che si trovi sullo stesso piano di esistenza in cui ti trovi tu. Un portale luminoso si apre all'interno del cerchio tracciato da te e resta aperto fino al termine del tuo prossimo round. Qualsiasi creatura che entri nel portale, riappare istantaneamente entro 1 metro dal cerchio di destinazione o nello spazio non occupato più vicino, se non può comparire entro 1 metro da esso.

Molti grandi templi, gilde, e altri luoghi importanti possiedono dei cerchi di teletrasporto permanenti, incisi da qualche parte nelle loro prossimità. Ciascuno di questi cerchi possiede una sequenza di sigilli unica: una serie di rune magiche disposte seguendo una trama precisa.

Quando ottieni la capacità di lanciare questo incantesimo, apprendi le sequenze di sigilli di due destinazioni sul Piano Materiale, determinate dal Narratore. Nel corso delle tue avventure puoi imparare nuove sequenze di sigilli. Puoi mandare a memoria una sequenza di sigilli dopo averla studiata per almeno 1 minuto.

Puoi creare un cerchio di teletrasporto permanente eseguendo questo incantesimo nello stesso luogo ogni giorno per un anno. Non devi usare il cerchio di teletrasporto quando lanci l'incantesimo in questo modo.

\textbf{NOTA}: Teletrasportarsi per più di 500 km ha solo il 5\% di riuscita.

\smallskip\noindent\rule{\linewidth}{2pt} \index[Incantesimi]{Chiaroveggenza}\hypertarget{Chiaroveggenza}{}\smallskip\noindent{\textbf{Chiaroveggenza}}\pdfbookmark[3]{Chiaroveggenza}{Chiaroveggenza}
\noindent
\begin{description}[noitemsep, topsep=0pt, parsep=0pt, partopsep=0pt, leftmargin=0cm, labelwidth=2.8cm]
	\item[\textbf{Lista di Magia}]: Divinazione
	\item[\textbf{Livello}]: 3, Comune
	\item[\textbf{T. di Lancio}]: 10 minuti
	\item[\textbf{Gittata}]: 1,5 chilometri
	\item[\textbf{Componenti}]: V, S, M (un focus del valore di almeno 100 mo, che sia un corno ingioiellato per udire o un occhio di vetro per guardare)
	\item[\textbf{Durata}]: Concentrazione, massimo 10 minuti
\end{description}

Crei un sensore invisibile in un luogo a te familiare e che sia a gittata (un luogo che hai già visitato o visto precedentemente) o in un luogo ovvio ma che non ti è familiare (come dietro una porta o un angolo, o in mezzo un boschetto di alberi). Il sensore rimane sul posto per la durata, e non può essere attaccato né altrimenti vi si può interagire. Quando lanci questo incantesimo, scegli se vedere o udire. Puoi usare il senso scelto tramite il sensore, come ti trovassi nel suo spazio. Con due azioni, puoi passare da udire a sentire e viceversa. Una creatura che può vedere il sensore (una creatura munita di \hyperlink{Vedere l'invisibile}{Vedere l'invisibile} o di visione del vero) lo percepisce come un orbe intangibile e luminoso delle dimensioni del tuo pugno.

\textbf{Per ogni Successo Critico Magico} ottenuto nella Prova di Magia la durata aumenta di 10 minuti o la gittata aumenta di 500m.

\smallskip\noindent\rule{\linewidth}{2pt} \index[Incantesimi]{Chiudi Portale}\hypertarget{Chiudi Portale}{}\smallskip\noindent{\textbf{Chiudi Portale}}\pdfbookmark[3]{Chiudi Portale}{Chiudi Portale}\label{Chiudi Portale}
\noindent
\begin{description}[noitemsep, topsep=0pt, parsep=0pt, partopsep=0pt, leftmargin=0cm, labelwidth=2.8cm]
	\item[\textbf{Lista di Magia}]: Abiurazione
	\item[\textbf{Livello}]: 2, Raro
	\item[\textbf{T. di Lancio}]:  1 Turno
	\item[\textbf{Gittata}]: 18 metri
	\item[\textbf{Componenti}]: V, S
	\item[\textbf{Durata}]: Istantanea
\end{description}

L'incantatore si pone entro distanza da un Portale. Formulato l'incantesimo il Portale se ha una DC inferiore a quella dell'incantatore viene chiuso e scompare.

\textbf{Per ogni Successo Critico Magico} ottenuto nella Prova di Magia la tua DC aumenta di 4.

\textbf{NOTA}: l'incantesimo è Comune per i Devoti di Lynx

\smallskip\noindent\rule{\linewidth}{2pt} \index[Incantesimi]{Colpo Accurato}\hypertarget{Colpo Accurato}{}\smallskip\noindent{\textbf{Colpo Accurato}}\pdfbookmark[3]{Colpo Accurato}{Colpo Accurato}
\noindent
\begin{description}[noitemsep, topsep=0pt, parsep=0pt, partopsep=0pt, leftmargin=0cm, labelwidth=2.8cm]
	\item[\textbf{Lista di Magia}]: Divinazione
	\item[\textbf{Livello}]: 0, Comune
	\item[\textbf{T. di Lancio}]: 2 Azioni
	\item[\textbf{Gittata}]: 9 metri
	\item[\textbf{Componenti}]: S
	\item[\textbf{Durata}]: 1 round
\end{description}

Allunghi la mano e punti il dito verso un bersaglio a gittata. La tua magia ti conferisce una breve comprensione delle difese del bersaglio. Entro la fine del prossimo round ottieni +1d6 al primo Tiro per Colpire contro quel bersaglio.

\textbf{Per ogni Successo Critico Magico} ottenuto il bonus perdura per un round in più.

\smallskip\noindent\rule{\linewidth}{2pt} \index[Incantesimi]{Colpo Accecante}\hypertarget{Colpo Accecante}{}\smallskip\noindent{\textbf{Colpo Accecante}}\pdfbookmark[3]{Colpo Accecante}{Colpo Accecante}
\noindent
\begin{description}[noitemsep, topsep=0pt, parsep=0pt, partopsep=0pt, leftmargin=0cm, labelwidth=2.8cm]
	\item[\textbf{Lista di Magia}]: Invocazione
	\item[\textbf{Livello}]: 3, Raro
	\item[\textbf{T. di Lancio}]: 2 Azioni
	\item[\textbf{Componenti}]: V
	\item[\textbf{Durata}]: 1 minuto
\end{description}

Il bersaglio colpito dal colpo subisce 3d8 danni da Luce e deve superare il Tiro Salvezza su Tempra o diventare Accecato fino al termine dell'incantesimo. Alla fine di ciascuno dei suoi round, il bersaglio accecato ripete il Tiro Salvezza terminando l'incantesimo su se stesso in caso di successo.

\textbf{Per ogni Successo Critico Magico} ottenuto nella Prova di Magia infliggi +2d6 danni da Luce in più.

\smallskip\noindent\rule{\linewidth}{2pt} \index[Incantesimi]{Colpo Infuocato}\hypertarget{Colpo Infuocato}{}\smallskip\noindent{\textbf{Colpo Infuocato}}\pdfbookmark[3]{Colpo Infuocato}{Colpo Infuocato}
\noindent
\begin{description}[noitemsep, topsep=0pt, parsep=0pt, partopsep=0pt, leftmargin=0cm, labelwidth=2.8cm]
	\item[\textbf{Lista di Magia}]: Fuoco
	\item[\textbf{Livello}]: 5, Comune
	\item[\textbf{T. di Lancio}]: 2 Azioni
	\item[\textbf{Gittata}]: 18 metri
	\item[\textbf{Componenti}]: V, S, M (pizzico di zolfo)
	\item[\textbf{Durata}]: Istantanea
\end{description}

Una colonna verticale di fuoco divino scende dal cielo e si abbatte sul luogo da te specificato. Ogni creatura in un cilindro di 3 metri di raggio e alto 12 metri centrato su di un punto a gittata deve effettuare un Tiro Salvezza su Riflessi. Una creatura subisce 8d6 danni da Luce se fallisce il Tiro Salvezza, o la metà di questi danni se lo supera.

\textbf{Per ogni Successo Critico Magico} ottenuto nella Prova di Magia il danno Luce aumenta di 4d6.

\textbf{Tiro Salvezza Successo/Fallimento Critico}: In caso di Fallimento Critico il danno raddoppia, in caso di Successo Critico il danno viene ulteriormente dimezzato

\smallskip\noindent\rule{\linewidth}{2pt} \index[Incantesimi]{Colpo Fiammeggiante}\hypertarget{Colpo Fiammeggiante}{}\smallskip\noindent{\textbf{Colpo Fiammeggiante}}\pdfbookmark[3]{Colpo Fiammeggiante}{Colpo Fiammeggiante}
\noindent
\begin{description}[noitemsep, topsep=0pt, parsep=0pt, partopsep=0pt, leftmargin=0cm, labelwidth=2.8cm]
	\item[\textbf{Lista di Magia}]: Invocazione
	\item[\textbf{Livello}]: 1, Raro
	\item[\textbf{T. di Lancio}]: 1 Azione
	\item[\textbf{Gittata}]: personale
	\item[\textbf{Componenti}]: V
	\item[\textbf{Durata}]: 1 minuto
\end{description}

Tocchi un bersaglio, questo subisce 1d6 danni da Fuoco. Ogni round deve effettuare un Tiro Salvezza su Tempra o subire 1d6 di danno da fuoco, questo effetto termina dopo un minuto oppure quando il Tiro Salvezza riesce.

\textbf{Per ogni Successo Critico Magico} ottenuto nella Prova di Magia infliggi +1d6 danni da Fuoco iniziali.

\smallskip\noindent\rule{\linewidth}{2pt} \index[Incantesimi]{Colpo Luccicante}\hypertarget{Colpo Luccicante}{}\smallskip\noindent{\textbf{Colpo Luccicante}}\pdfbookmark[3]{Colpo Luccicante}{Colpo Luccicante}
\noindent
\begin{description}[noitemsep, topsep=0pt, parsep=0pt, partopsep=0pt, leftmargin=0cm, labelwidth=2.8cm]
	\item[\textbf{Lista di Magia}]: Invocazione
	\item[\textbf{Livello}]: 2, Non Comune
	\item[\textbf{T. di Lancio}]: 1 Azione
	\item[\textbf{Gittata}]: personale
	\item[\textbf{Componenti}]: V
	\item[\textbf{Durata}]: 1 minuto
\end{description}

Tocchi un bersaglio, questo subisce 2d6 danni da Luce e diventa visibile per tutta la durata dell'incantesimo. In aggiunta la creatura emana luce in 1 metro di raggio.

\textbf{Per ogni Successo Critico Magico} ottenuto nella Prova di Magia infliggi 1d6 danni da Luce aggiuntivi.

\smallskip\noindent\rule{\linewidth}{2pt} \index[Incantesimi]{Comando}\hypertarget{Comando}{}\smallskip\noindent{\textbf{Comando}}\pdfbookmark[3]{Comando}{Comando}
\noindent
\begin{description}[noitemsep, topsep=0pt, parsep=0pt, partopsep=0pt, leftmargin=0cm, labelwidth=2.8cm]
	\item[\textbf{Lista di Magia}]: Ammaliamento
	\item[\textbf{Livello}]: 1, Comune
	\item[\textbf{T. di Lancio}]: 2 Azioni
	\item[\textbf{Gittata}]: 18 metri
	\item[\textbf{Componenti}]: V, S
	\item[\textbf{Durata}]: 1 round
\end{description}

Pronunci un comando di una parola verso una creatura a gittata e che puoi vedere ed un gesto. Il bersaglio deve superare un Tiro Salvezza su Volontà o eseguire il comando entro il suo prossimo round. L'incantesimo non ha effetto se il bersaglio è non morto, se non capisce la tua lingua o se il tuo comando gli recherebbe danni.

Sono elencati alcuni tipici comandi e i loro effetti. Puoi dare comandi diversi da quelli descritti qui, in quel caso il Narratore determinerà il comportamento del bersaglio. Se il bersaglio non può eseguire il tuo comando, l'incantesimo ha fine.

\begin{itemize}[leftmargin=*] \setlength{\itemsep}{0pt}
	\item \emph{Avvicinati}. Il bersaglio si muove verso di te per il tragitto più breve e diretto, terminando il suo round se si avvicina a 1 metro da te.
	\item \emph{Fermo}. Il bersaglio non si muove e poi termina il suo round. Una creatura volante resta sul posto, purché le sia possibile. Se deve muoversi per restare in aria, vola la distanza minima necessaria per farlo.
	\item \emph{Getta}. Il bersaglio getta qualsiasi cosa stia tenendo in mano e poi termina il suo round.
	\item \emph{Scappa}. Il bersaglio spende il suo round a muoversi lontano da te con il mezzo più veloce a sua disposizione.
	\item \emph{Striscia}. Il bersaglio si getta prono e poi termina il suo round.
\end{itemize}

\textbf{Per ogni Successo Critico Magico} ottenuto nella Prova di Magia puoi agire su di un'ulteriore creatura. Nel momento in cui lanci l'incantesimo le creature bersaglio devono trovarsi entro 9 metri l'una da l'altra ed eseguono il medesimo comando.

\smallskip\noindent\rule{\linewidth}{2pt} \index[Incantesimi]{Conoscere i Tratti}\hypertarget{Conoscere i Tratti}{}\smallskip\noindent{\textbf{Conoscere i Tratti}}\pdfbookmark[3]{Conoscere i Tratti}{Conoscere i Tratti}
\noindent
\begin{description}[noitemsep, topsep=0pt, parsep=0pt, partopsep=0pt, leftmargin=0cm, labelwidth=2.8cm]
	\item[\textbf{Lista di Magia}]: Divinazione
	\item[\textbf{Livello}]: 1, Leggendario
	\item[\textbf{T. di Lancio}]: 2 Azioni
	\item[\textbf{Gittata}]: Personale
	\item[\textbf{Componenti}]: S
	\item[\textbf{Durata}]: Istantanea
\end{description}

Questo incantesimo permette di conoscere i Tratti di una creatura. Al soggetto è concesso un Tiro Salvezza su Volontà per resistere. Indipendentemente dal risultato del Tiro Salvezza la creatura sa di per certo chi ha formulato l'incantesimo.

\textbf{Per due Successo Critico Magico} ottenuto nella Prova di Magia puoi analizzare un altra creatura.

\smallskip\noindent\rule{\linewidth}{2pt} \index[Incantesimi]{Comprensione dei Linguaggi}\hypertarget{Comprensione dei Linguaggi}{}\smallskip\noindent{\textbf{Comprensione dei Linguaggi}}\pdfbookmark[3]{Comprensione dei Linguaggi}{Comprensione dei Linguaggi}
\noindent
\begin{description}[noitemsep, topsep=0pt, parsep=0pt, partopsep=0pt, leftmargin=0cm, labelwidth=2.8cm]
	\item[\textbf{Lista di Magia}]: Divinazione
	\item[\textbf{Livello}]: 1, Comune
	\item[\textbf{T. di Lancio}]: 2 Azioni
	\item[\textbf{Gittata}]: Personale
	\item[\textbf{Componenti}]: V, S, M (un pizzico di sale e fuliggine)
	\item[\textbf{Durata}]: 1 ora
\end{description}

Per la durata, capisci il significato letterale di qualsiasi linguaggio parlato che ascolti.

\textbf{Per ogni Successo Critico Magico} ottenuto nella Prova di Magia la durata raddoppia. Con tre successi critici sei anche in grado di leggere.

\textbf{NOTA}: se sei un Devoto di Nethergal l'incantesimo dura 2 ore.

\smallskip\noindent\rule{\linewidth}{2pt} \index[Incantesimi]{Comprensione degli Scritti}\hypertarget{Comprensione degli Scritti}{}\smallskip\noindent{\textbf{Comprensione degli Scritti}}\pdfbookmark[3]{Comprensione degli Scritti}{Comprensione degli Scritti}
\noindent
\begin{description}[noitemsep, topsep=0pt, parsep=0pt, partopsep=0pt, leftmargin=0cm, labelwidth=2.8cm]
	\item[\textbf{Lista di Magia}]: Divinazione
	\item[\textbf{Livello}]: 2, Non Comune
	\item[\textbf{T. di Lancio}]: 2 Azioni
	\item[\textbf{Gittata}]: Personale
	\item[\textbf{Componenti}]: V, S, M (un pizzico di argento e inchiostro secco)
	\item[\textbf{Durata}]: 1 ora
\end{description}

Per la durata comprendi il significato letterale di qualsiasi linguaggio scritto non magico che vedi. Devi essere a contatto con la superficie su cui le parole sono scritte. Per leggere una pagina di testo impieghi 1 minuto. Questo incantesimo non decodifica i messaggi segreti in un testo o i glifi, come un sigillo arcano, che non faccia parte di un linguaggio scritto.

\textbf{Per ogni Successo Critico Magico} ottenuto nella Prova di Magia la durata raddoppia.

\textbf{NOTA}: se sei un Devoto di Nethergal l'incantesimo è comune e dura 2 ore.

\smallskip\noindent\rule{\linewidth}{2pt} \index[Incantesimi]{Compulsione}\hypertarget{Compulsione}{}\smallskip\noindent{\textbf{Compulsione}}\pdfbookmark[3]{Compulsione}{Compulsione}
\noindent
\begin{description}[noitemsep, topsep=0pt, parsep=0pt, partopsep=0pt, leftmargin=0cm, labelwidth=2.8cm]
	\item[\textbf{Lista di Magia}]: Ammaliamento
	\item[\textbf{Livello}]: 4, Non Comune
	\item[\textbf{T. di Lancio}]: 2 Azioni
	\item[\textbf{Gittata}]: 9 metri
	\item[\textbf{Componenti}]: V, S
	\item[\textbf{Durata}]: Concentrazione, massimo 1 minuto
\end{description}

Le creature di tua scelta entro la gittata che puoi vedere e che ti possono sentire devono effettuare un Tiro Salvezza su Volontà. Un bersaglio supera automaticamente il Tiro Salvezza se non può essere Affascinato. Fino al termine dell'incantesimo puoi usare un'Azione durante ciascun tuo round per indicare una direzione orizzontale rispetto a te. Ogni bersaglio soggetto all'incantesimo deve usare quanto più possibile del suo movimento, durante il suo prossimo round, per muoversi in quella direzione. Il bersaglio non può effettuare nessun'altra Azione prima di muoversi. Dopo essersi mosso in questo modo il bersaglio può effettuare un altro Tiro Salvezza su Volontà per tentare di terminare l'effetto.

Un bersaglio non può essere obbligato a muoversi dentro un pericolo palesemente letale, come fiamme o crepacci.

\smallskip\noindent\rule{\linewidth}{2pt} \index[Incantesimi]{Comunione}\hypertarget{Comunione}{}\smallskip\noindent{\textbf{Comunione}}\pdfbookmark[3]{Comunione}{Comunione}
\noindent
\begin{description}[noitemsep, topsep=0pt, parsep=0pt, partopsep=0pt, leftmargin=0cm, labelwidth=2.8cm]
	\item[\textbf{Lista di Magia}]: Divinazione
	\item[\textbf{Livello}]: 5, Raro
	\item[\textbf{T. di Lancio}]: 1 minuto
	\item[\textbf{Gittata}]: Personale
	\item[\textbf{Componenti}]: V, S, M (incenso e una fiala di Acqua santa)
	\item[\textbf{Durata}]: 1 minuto
\end{description}

Comunichi con il tuo Patrono e gli poni fino a tre domande a cui si può dare risposta con un sì o un no. Devi porre le domande prima della fine dell'incantesimo. Riceverai la risposta corretta a ciascuna domanda. Le creature divine non sono necessariamente onniscienti, quindi potresti ricevere un \emph{non è chiaro} come risposta a una domanda che riguarda informazioni non pertinenti alle conoscenze del Patrono. Nel caso in cui una risposta di una parola potrebbe essere fuorviante o contraria agli interessi del Patrono, il Narratore potrebbe invece dare una breve frase come risposta.

Se lanci l'incantesimo due o più volte prima che sia sorta la nuova alba c'è una probabilità cumulativa del 25\% che per ogni lancio dopo il primo tu non ottenga alcuna risposta. Il Narratore effettua questo tiro in segreto.

\textbf{NOTA:} è necessario essere almeno Seguace per poter formulare questo incantesimo.

\smallskip\noindent\rule{\linewidth}{2pt} \index[Incantesimi]{Comunione con la Natura}\hypertarget{Comunione con la Natura}{}\smallskip\noindent{\textbf{Comunione con la Natura}}\pdfbookmark[3]{Comunione con la Natura}{Comunione con la Natura}
\noindent
\begin{description}[noitemsep, topsep=0pt, parsep=0pt, partopsep=0pt, leftmargin=0cm, labelwidth=2.8cm]
	\item[\textbf{Lista di Magia}]: Divinazione
	\item[\textbf{Livello}]: 5, Molto Raro
	\item[\textbf{T. di Lancio}]: 1 minuto
	\item[\textbf{Gittata}]: Personale
	\item[\textbf{Componenti}]: V, S
	\item[\textbf{Durata}]: Istantanea
\end{description}

Per un istante diventi tutt'uno con la natura e ottieni informazioni sul territorio circostante. In ambienti esterni, l'incantesimo ti fornisce informazioni sul territorio entro 5 chilometri da te. In grotte e altri ambienti naturali sotterranei, il raggio è limitato a 100 metri. L'incantesimo non funziona nei luoghi in cui la natura è stata soppiantata da costruzioni, come in sotterranei e paesi.

Apprendi immediatamente informazioni su un massimo di tre argomenti a tua scelta su uno dei seguenti soggetti, in relazione all'area:

\begin{itemize}\setlength{\itemsep}{-1pt}
	\item terreno e corpi d'acqua
	\item piante, minerali, animali e popolazioni prevalenti
	\item potenti celestiali, elementali, fatati, demoni o non morti
	\item influenze da altri piani di esistenza
	\item edifici
\end{itemize}

\textbf{Per ogni Successo Critico Magico} ottenuto nella Prova di Magia apprendi un argomento aggiuntivo.

\textbf{NOTA}: Se sei un Devoto di Efrem ottiene sempre almeno un Successo Critico Magico.

\smallskip\noindent\rule{\linewidth}{2pt} \index[Incantesimi]{Confusione}\hypertarget{Confusione}{}\smallskip\noindent{\textbf{Confusione}}\pdfbookmark[3]{Confusione}{Confusione}\hypertarget{incconfusione}{}\label{incconfusione}
\noindent
\begin{description}[noitemsep, topsep=0pt, parsep=0pt, partopsep=0pt, leftmargin=0cm, labelwidth=2.8cm]
	\item[\textbf{Lista di Magia}]: Ammaliamento
	\item[\textbf{Livello}]: 4, Comune
	\item[\textbf{T. di Lancio}]: 2 Azioni
	\item[\textbf{Gittata}]: 27 metri
	\item[\textbf{Componenti}]: V, S, M (tre gusci di noce)
	\item[\textbf{Durata}]: 1 minuto
\end{description}

Questo incantesimo assale e piega la mente delle creature, generando illusioni e provocando azioni incontrollate. Quando lanci questo incantesimo ogni creatura, in una sfera di 3 metri di raggio centrata su di un punto da te scelto entro la gittata, deve superare un Tiro Salvezza su Volontà o subirne gli effetti. Un bersaglio soggetto all'incantesimo non può effettuare reazioni e deve tirare un d10 all'inizio di ciascun suo round per determinare il proprio comportamento per quel round.

Al termine di ciascun suo round, un bersaglio soggetto all'incantesimo può effettuare un Tiro Salvezza su Volontà. Se lo supera per lui l'effetto ha termine.

\textbf{Per ogni Successo Critico Magico} ottenuto nella Prova di Magia il raggio della sfera aumenta di 1 metro.

\medskip

\begin{tabularx}{0.45\textwidth}{lX}
	\hline
	d10 & Comportamento\\
	1 & La creatura usa tutte le sue Azioni per per muoversi in una direzione casuale. Per determinare la direzione tira un d8\\
	2-5 & La creatura non fa nulla per tutto il round\\
	6 & La creatura effettua un attacco contro se stessa e finisce il round\\
	7-8 & La creatura effettua un attacco contro una creatura determinata a caso entro 1 Azione di Movimento. Se è stata colpita il round precedente attaccherà la creatura che l'ha colpito. Fatto l'attacco il round termina.\\
	9-10 & La creatura può agire e muoversi normalmente.
\end{tabularx}

\smallskip\noindent\rule{\linewidth}{2pt} \index[Incantesimi]{Confusione Contagiosa}\hypertarget{Confusione Contagiosa}{}\smallskip\noindent{\textbf{Confusione Contagiosa}}\pdfbookmark[3]{Confusione Contagiosa}{Confusione Contagiosa}
\noindent
\begin{description}[noitemsep, topsep=0pt, parsep=0pt, partopsep=0pt, leftmargin=0cm, labelwidth=2.8cm]
	\item[\textbf{Lista di Magia}]: Ammaliamento
	\item[\textbf{Livello}]: 8, Molto raro
	\item[\textbf{T. di Lancio}]: 10 minuti
	\item[\textbf{Gittata}]: Contatto
	\item[\textbf{Componenti}]: V, S, M (polvere di dente)
	\item[\textbf{Durata}]: 1 minuto
\end{description}

Questo incantesimo assale e piega la mente delle creature, generando illusioni e provocando azioni incontrollate. Una volta formulato questo incantesimo hai poi un minuto per toccare la prima creatura. Questa creatura può fare un Tiro Salvezza su Volontà per annullare gli effetti.

Qualsiasi creatura toccata dalla prima creatura trasmette l'effetto Confusione, con Tiro Salvezza come la prima creatura, la durata dell'effetto su questa creatura sarà di un minuto.

Se l'incantatore non tocca entro un minuto una creatura allora sarà lui stesso vittima dell'incantesimo confusione, senza possibilità di Tiro Salvezza.

\smallskip\noindent\rule{\linewidth}{2pt} \index[Incantesimi]{Cono di Freddo}\hypertarget{Cono di Freddo}{}\smallskip\noindent{\textbf{Cono di Freddo}}\pdfbookmark[3]{Cono di Freddo}{Cono di Freddo}
\noindent
\begin{description}[noitemsep, topsep=0pt, parsep=0pt, partopsep=0pt, leftmargin=0cm, labelwidth=2.8cm]
	\item[\textbf{Lista di Magia}]: Acqua
	\item[\textbf{Livello}]: 5, Comune
	\item[\textbf{T. di Lancio}]: 2 Azioni
	\item[\textbf{Gittata}]: Personale (cono di 18 metri)
	\item[\textbf{Componenti}]: V, S, M (un piccolo cristallo o cono di vetro)
	\item[\textbf{Durata}]: Istantanea
\end{description}

Un'esplosione di aria fredda erutta dalle tue mani. Ogni creatura in un cono di 18 metri deve effettuare un Tiro Salvezza su Tempra. Una creatura subisce 8d8 danni da freddo se fallisce il Tiro Salvezza o la metà di questi danni se lo supera. Una creatura uccisa da questo incantesimo diventa una statua di ghiaccio fino a quando scongela.

\textbf{Per ogni Successo Critico Magico} ottenuto nella Prova di Magia il danno aumenta di 4d8

\textbf{Tiro Salvezza Successo/Fallimento Critico}: In caso di Fallimento Critico il danno raddoppia, in caso di Successo Critico il danno viene ulteriormente dimezzato

\smallskip\noindent\rule{\linewidth}{2pt} \index[Incantesimi]{Conoscenza delle Leggende}\hypertarget{Conoscenza delle Leggende}{}\smallskip\noindent{\textbf{Conoscenza delle Leggende}}\pdfbookmark[3]{Conoscenza delle Leggende}{Conoscenza delle Leggende}
\noindent
\begin{description}[noitemsep, topsep=0pt, parsep=0pt, partopsep=0pt, leftmargin=0cm, labelwidth=2.8cm]
	\item[\textbf{Lista di Magia}]: Divinazione
	\item[\textbf{Livello}]: 5, Comune
	\item[\textbf{T. di Lancio}]: 10 minuti
	\item[\textbf{Gittata}]: Personale
	\item[\textbf{Componenti}]: V, S, M (incenso del valore di almeno 250 mo, che l'incantesimo consuma, e quattro strisce d'avorio del valore di almeno 50 mo)
	\item[\textbf{Durata}]: Istantanea
\end{description}

Nomina o descrivi una persona, luogo od oggetto. L'incantesimo ti porta alla mente un breve riassunto delle conoscenze più importanti sull'argomento da te nominato. Se la cosa da te nominata non ha alcuna rilevanza leggendaria, non ottieni alcuna informazione. Maggiori informazioni hai sull'argomento, più precise e dettagliate saranno le informazioni che riceverai. L'informazione che riceverai sarà accurata, ma celata magari in linguaggio metaforico.

\smallskip\noindent\rule{\linewidth}{2pt} \index[Incantesimi]{Contagio}\hypertarget{Contagio}{}\smallskip\noindent{\textbf{Contagio}}\pdfbookmark[3]{Contagio}{Contagio}
\noindent
\begin{description}[noitemsep, topsep=0pt, parsep=0pt, partopsep=0pt, leftmargin=0cm, labelwidth=2.8cm]
	\item[\textbf{Lista di Magia}]: Necromanzia
	\item[\textbf{Livello}]: 5, Non Comune
	\item[\textbf{T. di Lancio}]: 2 Azioni
	\item[\textbf{Gittata}]: Contatto
	\item[\textbf{Componenti}]: V, S
	\item[\textbf{Durata}]: 7 giorni
\end{description}

Tramite il contatto puoi infliggere malattie. Effettua un attacco da mischia contro una creatura a portata. Se colpisci, infetti la creatura con una malattia a tua scelta tra quelle descritte di seguito. Al termine di ciascun round del bersaglio, esso deve effettuare un Tiro Salvezza su Tempra. Dopo aver fallito tre di questi Tiri Salvezza, gli effetti della malattia permangono per la durata e la creatura non effettua più Tiri Salvezza. Dopo aver superato tre di questi Tiri Salvezza la creatura recupera dalla malattia e l'incantesimo ha termine. Nel mentre che esegue i Tiri Salvezza la creatura subisce gli effetti della malattia.

Dato che questo incantesimo induce nel suo bersaglio una malattia naturale, qualsiasi effetto che rimuova le malattie o migliori gli effetti delle malattie si applica a essa.

\begin{itemize}[leftmargin=*] \setlength{\itemsep}{0pt}
	\item \emph{Carne Putrida}. La pelle della creatura marcisce. La creatura ha -1d6 alle prove di Carisma e ogni danno è raddoppiato.
	\item \emph{Debolezza Accecante}. Il dolore attanaglia la mente della creatura mentre i suoi occhi diventano bianco latte. La creatura ha -1d6 alle prove di Saggezza e ai Tiri Salvezza su Volontà ed è accecata.
	\item \emph{Febbre Lurida}. Una febbre devastante sconvolge il corpo della creatura. La creatura ha -1d6 alle prove di Forza e ai Tiri Salvezza su Tempra e ai Tiri per Colpire che usano la Forza.
	\item \emph{Fitte}. La creatura è sopraffatta dai tremiti. La creatura ha -1d6 alle prove di Destrezza, ai Tiri Salvezza su Riflessi ed ai Tiri per Colpire che usano la Destrezza.
	\item \emph{Fuoco Mentale}. La mente della creatura è preda della febbre. La creatura ha -1d6 alle prove di Intelligenza e ai Tiri Salvezza su Volontà e si comporta come se in combattimento fosse sotto l'effetto dell'incantesimo confusione.
	\item \emph{Morte Melmosa}. La creatura inizia a sanguinare incessantemente. La creatura ha -1d6 alle prove di Costituzione ed ai Tiri Salvezza su Tempra. Inoltre ogni qualvolta la creatura subisce danni è Rallentata 1/1r fino alla fine del suo prossimo round.
\end{itemize}

\smallskip\noindent\rule{\linewidth}{2pt} \index[Incantesimi]{Contingenza}\hypertarget{Contingenza}{}\smallskip\noindent{\textbf{Contingenza}}\pdfbookmark[3]{Contingenza}{Contingenza}
\noindent
\begin{description}[noitemsep, topsep=0pt, parsep=0pt, partopsep=0pt, leftmargin=0cm, labelwidth=2.8cm]
	\item[\textbf{Lista di Magia}]: Invocazione
	\item[\textbf{Livello}]: 6, Non Comune
	\item[\textbf{T. di Lancio}]: 10 minuti
	\item[\textbf{Gittata}]: Personale
	\item[\textbf{Componenti}]: V, S, M (una statuetta raffigurante te stesso scolpita in avorio e decorata con gemme del valore di almeno 1.500 mo)
	\item[\textbf{Durata}]: 10 giorni
\end{description}

Scegli un incantesimo di Livello 4 o più basso che puoi lanciare, che abbia il tempo di lancio di 2 Azioni e che può avere te come bersaglio. Lanci quell'incantesimo (detto incantesimo contingente) come parte del lancio di contingenza, spendendo gli slot incantesimo di entrambi, ma senza che l'incantesimo contingente abbia effetto. Avrà invece effetto quando si avvererà una determinata circostanza. Descrivi questa circostanza mentre lanci i due incantesimi. Per esempio, contingenza lanciato assieme a respirare sott'acqua potrebbe stipulare che respirare sott'acqua entra in azione quando sei immerso nell'acqua o simile liquido.

L'incantesimo contingente ha effetto immediatamente dopo che la circostanza si verifica per la prima volta, che tu lo voglia o no, e poi contingenza termina. L'incantesimo contingente agisce solo su di te, anche se normalmente può prendere come bersaglio anche altri. Puoi usare un solo incantesimo contingenza alla volta. Se lanci di nuovo questo incantesimo, l'effetto di un altro incantesimo contingenza su di te avrà termine. Inoltre, contingenza per te ha termine se la componente materiale non dovesse più trovarsi sulla tua persona.

\textbf{Per ogni Successo Critico Magico} ottenuto nella Prova di Magia la contingenza dura 10 giorni in più.

\smallskip\noindent\rule{\linewidth}{2pt} \index[Incantesimi]{Controincantesimo}\hypertarget{Controincantesimo}{}\smallskip\noindent{\textbf{Controincantesimo}}\pdfbookmark[3]{Controincantesimo}{Controincantesimo}
\noindent
\begin{description}[noitemsep, topsep=0pt, parsep=0pt, partopsep=0pt, leftmargin=0cm, labelwidth=2.8cm]
	\item[\textbf{Lista di Magia}]: Abiurazione
	\item[\textbf{Livello}]: 3, Comune
	\item[\textbf{T. di Lancio}]: 1 Reazione, che effettui quando vedi una creatura/oggetto entro 18 metri manifestare un incantesimo
	\item[\textbf{Gittata}]: 18 metri
	\item[\textbf{Componenti}]: S
	\item[\textbf{Durata}]: Istantanea
\end{description}

Usi una Azione di Reazione per fare una prova di Arcana a DC 13. Se la prova riesce comprendi se puoi annullare l'effetto dell'incantesimo tramite Controincantesimo. L'incantesimo annullato deve essere di Livello 2 o più basso, indipendentemente che sia manifestato da un incantatore od oggetto. Ogni successo Critico magico o potenziamento ottenuto dall'incantesimo originale alza il livello dell'incantesimo di 1.

\textbf{Per ogni due Successo Critico Magico ottenuto} nella Prova di Magia puoi annullare un incantesimo di un livello superiore.

\smallskip\noindent\rule{\linewidth}{2pt} \index[Incantesimi]{Controllare Acqua}\hypertarget{Controllare Acqua}{}\smallskip\noindent{\textbf{Controllare Acqua}}\pdfbookmark[3]{Controllare Acqua}{Controllare Acqua}
\noindent
\begin{description}[noitemsep, topsep=0pt, parsep=0pt, partopsep=0pt, leftmargin=0cm, labelwidth=2.8cm]
	\item[\textbf{Lista di Magia}]: Acqua
	\item[\textbf{Livello}]: 4, Comune
	\item[\textbf{T. di Lancio}]: 2 Azioni
	\item[\textbf{Gittata}]: 90 metri
	\item[\textbf{Componenti}]: V, S, M (un goccio d'acqua e un pizzico di polvere)
	\item[\textbf{Durata}]: Concentrazione, massimo 10 minuti
\end{description}

Fino al termine dell'incantesimo, controlli qualsiasi acqua libera all'interno dell'area che hai scelto fino a un cubo di 30 metri di spigolo. Quando lanci questo incantesimo puoi scegliere qualsiasi tra i seguenti effetti. Come Azione, durante il tuo round, puoi ripetere lo stesso effetto o sceglierne uno diverso.

\medskip

- \emph{Allagamento}. Fai sì che il livello di tutta l'acqua nell'area aumenti fino a 6 metri. Se l'area include una costa, l'acqua inonda la terraferma. Se scegli un'area all'interno di un grosso corpo d'acqua, crei invece un'onda alta 6 metri che viaggia da un lato all'altro dell'area prima di infrangersi. Qualsiasi veicolo di taglia Enorme o inferiore sul percorso dell'onda viene trasportato dall'altro lato. Qualsiasi veicolo di taglia Enorme o inferiore colpito dall'acqua ha una percentuale del 25\% di cappottarsi.

Il livello dell'acqua resta elevato fino al termine dell'incantesimo o finché non scegli un effetto diverso. Se questo effetto ha prodotto un'onda, l'onda si ripete all'inizio del tuo round successivo, finché perdura l'effetto di allagamento.

- \emph{Dividere le Acque}. Fai sì che l'acqua nell'area si sposti a lato per creare un varco. Il varco si estende per l'area dell'incantesimo, e l'acqua divisa forma un muro su entrambi i lati del varco. Il varco resta fino al termine dell'incantesimo o finché non scegli un effetto diverso. L'acqua tornerà poi lentamente a riempire il varco nel corso del round successivo, fino a che non sarà risalita al suo normale livello.

- \emph{Ridirigere il Flusso}. Fai sì che l'acqua corrente nell'area si muova in una direzione a tua scelta, anche se l'acqua deve superare degli ostacoli, risalire muri o dirigersi verso altre direzioni improbabili. L'acqua nell'area si muove secondo le tue indicazioni, ma una volta giunta oltre l'area dell'incantesimo, riprende il suo flusso in base alle condizioni del terreno. L'acqua continua a muoversi nella direzione da te scelta fino al termine dell'incantesimo o finché non scegli un effetto diverso.

- \emph{Turbine}. Questo effetto richiede un corpo d'acqua che copra un quadrato di 15 metri di lato e abbia una profondità di 7 metri. Fai sì che si formi un turbine al centro dell'area. Il turbine produce un vortice largo 1 metro alla base, largo fino a 15 metri in cima e alto 7 metri. Qualsiasi creatura od oggetto nell'acqua e che si trovi entro 7 metri dal vortice viene trascinato 3 metri verso di esso. Una creatura può nuotare per allontanarsi dal vortice effettuando una prova di Nuotare contro la DC del Tiro Salvezza dell'incantesimo.

Quando una creatura entra nel vortice per la prima volta durante un round o inizia lì il suo round, deve effettuare un Tiro Salvezza su Tempra. Se lo fallisce, la creatura subisce 2d8 danni contundenti e viene catturata dal vortice fino al termine dell'incantesimo. Se supera il Tiro Salvezza, la creatura subisce la metà di questi danni, e non è catturata dal vortice. Una creatura catturata dal vortice può usare 3 Azioni per cercare di nuotare via dal vortice come descritto sopra, ma ha -4 alle prove di Nuotare per farlo. La prima volta durante ciascun round in un cui un oggetto entra nel vortice, l'oggetto subisce 2d8 danni contundenti; questo danno viene subito ogni round in cui l'oggetto rimane nel vortice.

\smallskip\noindent\rule{\linewidth}{2pt} \index[Incantesimi]{Controllare Tempo Atmosferico}\hypertarget{Controllare Tempo Atmosferico}{}\smallskip\noindent{\textbf{Controllare Tempo Atmosferico}}\pdfbookmark[3]{Controllare Tempo Atmosferico}{Controllare Tempo Atmosferico}
\noindent
\begin{description}[noitemsep, topsep=0pt, parsep=0pt, partopsep=0pt, leftmargin=0cm, labelwidth=2.8cm]
	\item[\textbf{Lista di Magia}]: Acqua, Aria
	\item[\textbf{Livello}]: 8, Molto Raro
	\item[\textbf{T. di Lancio}]: 10 minuti
	\item[\textbf{Gittata}]: Personale (raggio di 1,5 chilometri)
	\item[\textbf{Componenti}]: V, S, M (incenso bruciato e pò di terra e legno mescolati nell'acqua)
	\item[\textbf{Durata}]: Concentrazione, massimo 8 ore
\end{description}

Per la durata, assumi il controllo del clima entro 7,5 chilometri da te. Per lanciare questo incantesimo devi essere all'esterno. Muoversi in un posto dove non hai la visuale aperta verso il cielo termina l'incantesimo anticipatamente. Quando lanci questo incantesimo, cambia le attuali condizioni climatiche determinate dal Narratore in base alla stagione e la latitudine. Puoi modificare le precipitazioni, la temperatura e il vento. Ci vogliono 1d4 x 10 minuti perché la nuova condizione prenda effetto. Una volta che la condizione avrà preso effetto, potrai cambiarla di nuovo. Quando l'incantesimo termina il clima tornerà gradualmente alla norma.

\medskip

Quando cambi le condizioni climatiche, trova l'attuale condizione sulla seguente tabella e cambiala di uno stadio, verso l'alto o il basso. Quando cambi il vento, puoi cambiarne anche la direzione.

\medskip

\emph{Precipitazione}

- 1 Limpido

- 2 Qualche nuvola

- 3 Coperto o foschia a terra

- 4 Pioggia, grandine o neve

- 5 Pioggia torrenziale, grandinata pesante, tormenta

\medskip

\emph{Temperatura}

- 1 Caldo insopportabile

- 2 Caldo

- 3 Tiepido

- 4 Fresco

- 5 Freddo

- 6 Freddo polare

\medskip

\emph{Vento}

- 1 Calmo

- 2 Vento moderato

- 3 Vento moderato

- 4 Fortunale

- 5 Tempesta

\medskip

\textbf{Per ogni Successo Critico Magico} ottenuto nella Prova di Magia la durata aumenta di 8 ore.

\smallskip\noindent\rule{\linewidth}{2pt} \index[Incantesimi]{Costrizione}\hypertarget{Costrizione}{}\smallskip\noindent{\textbf{Costrizione}}\pdfbookmark[3]{Costrizione}{Costrizione}
\noindent
\begin{description}[noitemsep, topsep=0pt, parsep=0pt, partopsep=0pt, leftmargin=0cm, labelwidth=2.8cm]
	\item[\textbf{Lista di Magia}]: Ammaliamento
	\item[\textbf{Livello}]: 5, Raro
	\item[\textbf{T. di Lancio}]: 1 minuto
	\item[\textbf{Gittata}]: 18 metri
	\item[\textbf{Componenti}]: V
	\item[\textbf{Durata}]: 30 giorni
\end{description}

Imponi un comando magico a una creatura a gittata che puoi vedere, obbligandolo ad adempiere un determinato compito o vietandole di svolgere un'azione o corso d'attività deciso da te. Se la creatura ti può capire, deve superare un Tiro Salvezza su Volontà o restare affascinata da te per la durata. Mentre la creatura è affascinata da te, subisce 3d10 danni ogni volta che agisce in maniera direttamente contraria alle tue istruzioni, ma non più di una volta al giorno. Una creatura che non ti può capire ignora gli effetti di questo incantesimo. Puoi dare qualsiasi comando di tua scelta, tranne un'attività che provocherebbe morte certa. Dovessi tu pronunciare un comando suicida, l'incantesimo avrebbe termine.

Puoi terminare l'incantesimo usando un'Azione. Anche Rimuovi Maledizione, Ristorare superiore o Desiderio vi pongono termine.

\textbf{Se ottieni almeno due Critici Magici} nella Prova di Magia la durata è 1 anno. Se ottieni 3 Critici l'incantesimo dura finché non viene terminato da uno degli incantesimi sopra menzionati.

\smallskip\noindent\rule{\linewidth}{2pt} \index[Incantesimi]{Creare Cibo e Acqua}\hypertarget{Creare Cibo e Acqua}{}\smallskip\noindent{\textbf{Creare Cibo e Acqua}}\pdfbookmark[3]{Creare Cibo e Acqua}{Creare Cibo e Acqua}
\noindent
\begin{description}[noitemsep, topsep=0pt, parsep=0pt, partopsep=0pt, leftmargin=0cm, labelwidth=2.8cm]
	\item[\textbf{Lista di Magia}]: Evocazione
	\item[\textbf{Livello}]: 3, Comune
	\item[\textbf{T. di Lancio}]: 2 Azioni
	\item[\textbf{Gittata}]: 9 metri
	\item[\textbf{Componenti}]: V, S
	\item[\textbf{Durata}]: Istantanea
\end{description}

Crei cibo e acqua in contenitori a gittata, sufficienti a sostenere fino a cinque umanoidi o 2 cavalcature per 24 ore. Il cibo è blando ma nutriente e marcisce dopo 24 ore se non viene consumato, come anche l'acqua.

\textbf{Per ogni Successo Critico Magico} ottenuto nella Prova di Magia crei cibo per altre 3 persone oppure 1 cavalcatura.

\textbf{Nota}: se sei un Seguace di Nihar il cibo è succulento e saporito.

\smallskip\noindent\rule{\linewidth}{2pt} \index[Incantesimi]{Creare Birra}\hypertarget{Creare Birra}{}\smallskip\noindent{\textbf{Creare Birra}}\pdfbookmark[3]{Creare Birra}{Creare Birra}
\noindent
\begin{description}[noitemsep, topsep=0pt, parsep=0pt, partopsep=0pt, leftmargin=0cm, labelwidth=2.8cm]
	\item[\textbf{Lista di Magia}]: Evocazione
	\item[\textbf{Livello}]: 0, Raro
	\item[\textbf{T. di Lancio}]: variabile
	\item[\textbf{Gittata}]: 9 metri
	\item[\textbf{Componenti}]: V, S, M (lievito di birra, malto, acqua)
	\item[\textbf{Durata}]: 1 ora
\end{description}

Crei un boccale di birra, 0.5 litri. La qualità e tipologia di birra dipende dal lievito, malto e acqua usata.
Maggiore è il tempo di lancio dell'incantesimo più viene alta la gradazione alcolica, con un tempo di lancio di due azioni la gradazione è di 4.3, se viene impiegata 1 Azione la birra generata è analcolica, ogni Azione spesa dopo le 2 aumenta la gradazione di 0.3 vol fino ad un massimo di 12.5 vol.
Dopo un ora la birra svanisce, quando consumata dopo un ora terminano anche eventuali effetti alcolici della stessa sulle persone che l'hanno bevuta.

\textbf{Per ogni Successo Critico Magico} ottenuto nella Prova di Magia aumenti di un litro o di un ora la durata.

\smallskip\noindent\rule{\linewidth}{2pt} \index[Incantesimi]{Creare o Distruggere Acqua}\hypertarget{Creare o Distruggere Acqua}{}\smallskip\noindent{\textbf{Creare o Distruggere Acqua}}\pdfbookmark[3]{Creare o Distruggere Acqua}{Creare o Distruggere Acqua}
\noindent
\begin{description}[noitemsep, topsep=0pt, parsep=0pt, partopsep=0pt, leftmargin=0cm, labelwidth=2.8cm]
	\item[\textbf{Lista di Magia}]: Acqua
	\item[\textbf{Livello}]: 1, Comune
	\item[\textbf{T. di Lancio}]: 2 Azioni
	\item[\textbf{Gittata}]: 9 metri
	\item[\textbf{Componenti}]: V, S, M (un goccio d'acqua per creare acqua o qualche granello di sale per distruggerla)
	\item[\textbf{Durata}]: Istantanea
\end{description}

Crei o distruggi l'acqua.

\emph{Creare Acqua}. Crei fino a 40 litri di acqua limpida dalle tue mani che spruzzano fino a 9 metri. In alternativa l'acqua cade come pioggia in una sfera di 3 metri di raggio che si trovi entro la gittata, estinguendo le fiamme esposte nell'area.

L'incantesimo non può essere usato per spegnere fiamme magiche.

\emph{Distruggere Acqua}. Distruggi fino a 40 litri di acqua in un contenitore aperto a gittata. In alternativa, puoi distruggere la nebbia in una sfera di 4 metri di raggio entro la gittata. Usato su una elementale dell'acqua l'incantesimo causa 4d6 di danno con un Tiro Salvezza su Tempra per dimezzare.

\textbf{Per ogni Successo Critico Magico} ottenuto nella Prova di Magia crei o distruggi ulteriori 40 litri d'acqua, o le dimensioni della sfera aumentano di 1 metro di raggio in caso di nebbia.

L'acqua è potabile e disseta se bevuta entro un round dalla creazione.

\smallskip\noindent\rule{\linewidth}{2pt} \index[Incantesimi]{Creare Fossa}\hypertarget{Creare Fossa}{}\smallskip\noindent{\textbf{Creare Fossa}}\pdfbookmark[3]{Creare Fossa}{Creare Fossa}
\noindent
\begin{description}[noitemsep, topsep=0pt, parsep=0pt, partopsep=0pt, leftmargin=0cm, labelwidth=2.8cm]
	\item[\textbf{Lista di Magia}]: Invocazione
	\item[\textbf{Livello}]: 2, Non Comune
	\item[\textbf{T. di Lancio}]: 1 Azione
	\item[\textbf{Gittata}]: 30 metri più 3 metri per livello
	\item[\textbf{Componenti}]: V, S, M (una pala in oro in miniatura del costo di 10 mo)
	\item[\textbf{Durata}]: 1 round più 1 round per CM
\end{description}

Crei una buca extradimensionale di 3 metri per 3 metri con una profondità di 3 metri per ogni due punti di CM fino ad un massimo di 9 metri. Devi creare la fossa su una superficie orizzontale di dimensioni sufficienti. Poiché si estende in un'altra dimensione, la fossa non ha peso e non sposta il materiale sottostante originale.

Qualsiasi creatura che si trova nell'area dove hai evocato la fossa deve effettuare un Tiro Salvezza su Riflessi per saltare in sicurezza nello spazio aperto più vicino. Inoltre, i bordi della fossa sono inclinati, e qualsiasi creatura che termina il suo turno in una casella adiacente alla fossa deve effettuare un Tiro Salvezza su Riflessi con bonus +2 per evitare di caderci dentro.

Le creature che cadono nella fossa subiscono danni da caduta normali. Le pareti rocciose lisce della fossa hanno DC 25 per Scalare. Quando la durata dell'incantesimo termina, le creature dentro la buca si alzano con il fondo della fossa fino a trovarsi sulla superficie nel corso di un singolo round.

\textbf{Per ogni Successo Critico Magico} ottenuto nella Prova di Magia raddoppi la profondità della fossa o la allarghi di 1 metro.

\smallskip\noindent\rule{\linewidth}{2pt} \index[Incantesimi]{Creare Non Morti}\hypertarget{Creare Non Morti}{}\smallskip\noindent{\textbf{Creare Non Morti}}\pdfbookmark[3]{Creare Non Morti}{Creare Non Morti}
\noindent
\begin{description}[noitemsep, topsep=0pt, parsep=0pt, partopsep=0pt, leftmargin=0cm, labelwidth=2.8cm]
	\item[\textbf{Lista di Magia}]: Necromanzia
	\item[\textbf{Livello}]: 6, Non Comune
	\item[\textbf{T. di Lancio}]: 2 Azioni
	\item[\textbf{Gittata}]: 3 metri
	\item[\textbf{Componenti}]: V, S, M (un vaso di terracotta pieno di terra di cimitero, un vaso di terracotta pieno di acqua salmastra e un onice nero del valore di 50 mo per ogni cadavere)
	\item[\textbf{Durata}]: Istantanea
\end{description}

Puoi lanciare questo incantesimo solo di notte. Scegli fino a tre cadaveri di umanoidi Medi o Piccoli a gittata. Ogni cadavere diventa un ghoul sotto il tuo controllo (il Narratore possiede le statistiche di gioco di queste creature). Durante il tuo round, con due Azioni, puoi comandare mentalmente una qualsiasi creatura da te animata con questo incantesimo, se la creatura si trova entro 36 metri da te (se controlli più creature, puoi comandarle tutte o solo una nello stesso momento impartendo lo stesso comando). Decidi tu quale azione effettuerà la creatura e dove si muoverà durante il suo prossimo round, oppure puoi impartire un comando generico, come quello di fare la guardia a una specifica stanza o corridoio. Se non impartisci comandi, le creature si limiteranno a difendersi dalle creature ostili. Una volta ricevuto un comando, la creatura continuerà a eseguirlo finché il compito sarà completo. La creatura è sotto il tuo controllo per 24 ore, dopodiché smetterà di rispondere ai comandi che gli impartisci. Per mantenere il controllo della creatura per altre 24 ore, devi lanciare questo incantesimo sulla creatura prima che l'attuale periodo di 24 ore abbia termine. Questo impiego dell'incantesimo riasserisce il tuo controllo su di un massimo di tre creature che hai animato con questo incantesimo, anziché animarne di nuove.

\textbf{Se ottieni un Critico Magico} nella Prova di Magia puoi rianimare o riasserire il controllo su quattro ghoul. Con due Critici puoi animare o riasserire il controllo su cinque ghoul o due ghast o wight. Con tre Critici puoi animare o riasserire il controllo su sei ghoul, tre ghast o wight, o due mummie.

\smallskip\noindent\rule{\linewidth}{2pt} \index[Incantesimi]{Creazione}\hypertarget{Creazione}{}\smallskip\noindent{\textbf{Creazione}}\pdfbookmark[3]{Creazione}{Creazione}
\noindent
\begin{description}[noitemsep, topsep=0pt, parsep=0pt, partopsep=0pt, leftmargin=0cm, labelwidth=2.8cm]
	\item[\textbf{Lista di Magia}]: Illusione
	\item[\textbf{Livello}]: 5, Raro
	\item[\textbf{T. di Lancio}]: 1 minuto
	\item[\textbf{Gittata}]: 9 metri
	\item[\textbf{Componenti}]: V, S, M (un minuscolo pezzo di materiale dello stesso tipo di oggetto che intendi creare)
	\item[\textbf{Durata}]: Speciale
\end{description}

Afferri pezzi di materia d'ombra dal piano delle Ombre per creare, a gittata, oggetti non viventi di materia vegetale: beni morbidi, corda, legno o qualcosa di simile. Puoi usare questo incantesimo anche per creare oggetti minerali come pietra, cristallo o metallo. L'oggetto creato non può essere più grosso di un cubo di 1 metro di spigolo e l'oggetto deve essere di una forma e materiale che hai già visto in passato.

La durata dipende dal materiale dell'oggetto. Se l'oggetto è composto da più materiali, usa la durata più breve.
\medskip
Tabella Materiale - Durata
\medskip

\begin{tabularx}{0.45\textwidth}{lX}
	\hline
	Materia vegetale &1 giorno\\
	Pietra o cristallo &12 ore\\
	Metalli preziosi &1 ora\\
	Gemme &10 minuti\\
	Adamantio o mithral &1 minuto
\end{tabularx}
\medskip

Usare qualsiasi materiale creato da questo incantesimo come componente materiale di un altro incantesimo farà fallire il nuovo incantesimo.

\textbf{Per ogni Successo Critico Magico} ottenuto nella Prova di Magia il cubo aumenta di 1 metro di spigolo.

\smallskip\noindent\rule{\linewidth}{2pt} \index[Incantesimi]{Crescita di Spuntoni}\hypertarget{Crescita di Spuntoni}{}\smallskip\noindent{\textbf{Crescita di Spuntoni}}\pdfbookmark[3]{Crescita di Spuntoni}{Crescita di Spuntoni}
\noindent
\begin{description}[noitemsep, topsep=0pt, parsep=0pt, partopsep=0pt, leftmargin=0cm, labelwidth=2.8cm]
	\item[\textbf{Lista di Magia}]: Animali e Piante
	\item[\textbf{Livello}]: 2, Comune
	\item[\textbf{T. di Lancio}]: 2 Azioni
	\item[\textbf{Gittata}]: 45 metri
	\item[\textbf{Componenti}]: V, S, M (sette spine affilate o sette ramoscelli, ciascuno di esse appuntito ad un'estremità)
	\item[\textbf{Durata}]: 10 minuti
\end{description}

Il terreno in un raggio di 6 metri centrato su di un punto a gittata si contorce e genera spuntoni e spine molto acuminate. Per la durata, l'area diventa terreno difficile. Quando una creatura entra o si muove all'interno dell'area, subisce 2d4 danni per ogni 1 metro percorsi.
La trasformazione del terreno è talmente ben camuffata da sembrare naturale. Qualsiasi creatura che non abbia visto l'area al momento del lancio dell'incantesimo deve effettuare una prova di Consapevolezza contro la DC del Tiro Salvezza dell'incantesimo, per riconoscere il pericolo rappresentato dal terreno prima di entrarvi.

\smallskip\noindent\rule{\linewidth}{2pt} \index[Incantesimi]{Crescita Vegetale}\hypertarget{Crescita Vegetale}{}\smallskip\noindent{\textbf{Crescita Vegetale}}\pdfbookmark[3]{Crescita Vegetale}{Crescita Vegetale}
\noindent
\begin{description}[noitemsep, topsep=0pt, parsep=0pt, partopsep=0pt, leftmargin=0cm, labelwidth=2.8cm]
	\item[\textbf{Lista di Magia}]: Animali e Piante
	\item[\textbf{Livello}]: 3, Non Comune
	\item[\textbf{T. di Lancio}]: 2 Azioni o 8 ore
	\item[\textbf{Gittata}]: 45 metri
	\item[\textbf{Componenti}]: V, S
	\item[\textbf{Durata}]: Istantanea
\end{description}

Questo incantesimo incanala vitalità nei vegetali entro una specifica area. Esistono due usi possibili per questo incantesimo, che conferiscono benefici immediati o a lungo termine. Se lanci questo incantesimo impiegando 1 Azione, scegli un punto a gittata. Tutte i vegetali normali in un raggio di 30 metri centrato su quel punto diventano densi e folti. Una creatura che attraversa l'area quadruplica il costo del suo movimento.

Puoi escludere dai suoi effetti una o più aree di qualsiasi dimensione all'interno dell'area dell'incantesimo.

Se lanci questo incantesimo nel corso di 8 ore, nutri la terra. Tutti i vegetali in un raggio di 750 metri centrato su di un punto a gittata diventano super produttivi per 1 anno. I vegetali producono il doppio del normale ammontare di cibo al momento del raccolto.

\textbf{Se ottiene due Successo Critico Magico} sortisci gli effetti delle 8 ore di lancio anche se l'incantesimo è stato lanciato con 2 Azioni.

\smallskip\noindent\rule{\linewidth}{2pt} \index[Incantesimi]{CTRLC+CTRLV}\hypertarget{CTRLC+CTRLV}{}\smallskip\noindent{\textbf{CTRLC+CTRLV}}\pdfbookmark[3]{CTRLC+CTRLV}{CTRLC+CTRLV}
\noindent
\begin{description}[noitemsep, topsep=0pt, parsep=0pt, partopsep=0pt, leftmargin=0cm, labelwidth=2.8cm]
	\item[\textbf{Lista di Magia}]: Universale
	\item[\textbf{Livello}]: 1, Molto Raro
	\item[\textbf{T. di Lancio}]: 2 Azioni
	\item[\textbf{Gittata}]: Personale
	\item[\textbf{Componenti}]: V, S, M (tre piccoli cubi di ceramica riportante la lettera C, la lettera V ed il glifo CTRL)
	\item[\textbf{Durata}]: 1 minuto per CM
\end{description}

Questo incantesimo permette di copiare un testo da una sorgente ad ud altra. In caso di sorgente non magica questa può essere un libro, una pergamena, delle rune su una lastra od un bastone. La destinazione che va appoggiata sulla sorgente andrà a copiare i simboli nella forma e dimensione fino alla sua capienza, per un massimo di 1 pagina (di destinazione) al minuto.

Se lo scritto è un incantesimo, quindi su un Tomo o Pergamena, devono essere comunque rispettate le regole e limitazioni previste per la copia di Incantesimi sul Tomo. Questo incantesimo permette di evitare la Prova di Magia in caso di Incantesimo entro un livello superiore al massimo consentito. Copiato un incantesimo questo incantesimo termina.

\smallskip\noindent\rule{\linewidth}{2pt} \index[Incantesimi]{Cuoco Invisibile}\hypertarget{Cuoco Invisibile}{}\smallskip\noindent{\textbf{Cuoco Invisibile}}\pdfbookmark[3]{Cuoco Invisibile}{Cuoco Invisibile}
\noindent
\begin{description}[noitemsep, topsep=0pt, parsep=0pt, partopsep=0pt, leftmargin=0cm, labelwidth=2.8cm]
	\item[\textbf{Lista di Magia}]: Evocazione
	\item[\textbf{Livello}]: 1, Comune
	\item[\textbf{T. di Lancio}]: 2 Azioni
	\item[\textbf{Gittata}]: 18 metri
	\item[\textbf{Componenti}]: V, S, M (un mestolo di legno e qualche goccia di olio di oliva, il cibo che si vuole cucinato)
	\item[\textbf{Durata}]: 2 ore
\end{description}

Questo incantesimo crea una forza quasi invisibile solo delimitata da una leggera aura (di colore a tua scelta) capace e competente nel cucinare. Assieme al cuoco si manifesta anche un set di pentole e padelle nonché stoviglie ed un piccolo fornello da campo.

In base agli ingredienti a disposizione o vegetali commestibili nel raggio di 100 metri (il cuoco non va a caccia) il cuoco cucinerà al meglio degli ingredienti preparando delle ottime vivande fino a 4 persone. L'incantesimo non crea cibo o acqua, questo deve essere a disposizione al momento del lancio dell'incantesimo.

Una volta a disposizione gli ingredienti entro le due ore il cuoco invisibile preparerà da mangiare. E' possibile anche affrettare l'esecuzione ma a discapito della qualità.

Nessuna delle pentole, stoviglie o fuoco può essere usato fuorché dal cuoco invisibile.

\textbf{Se ottiene due Successo Critico Magico} il Cuoco viene evocato con cibo necessario a sfamare 2 persone

\smallskip\noindent\rule{\linewidth}{2pt} \index[Incantesimi]{Cura Ferite}\hypertarget{Cura Ferite}{}\smallskip\noindent{\textbf{Cura Ferite}}\pdfbookmark[3]{Cura Ferite}{Cura Ferite}
\noindent
\begin{description}[noitemsep, topsep=0pt, parsep=0pt, partopsep=0pt, leftmargin=0cm, labelwidth=2.8cm]
	\item[\textbf{Lista di Magia}]: Acqua, Cura
	\item[\textbf{Livello}]: 1, Comune
	\item[\textbf{T. di Lancio}]: 2 Azioni
	\item[\textbf{Gittata}]: Contatto
	\item[\textbf{Componenti}]: V, S
	\item[\textbf{Durata}]: Istantanea
\end{description}

La tua mano si riempie di energia positiva curativa, una creatura che tocchi recupera un numero di Punti Ferita uguale a 1d8 + modificatore di caratteristica per incantesimi. Questo incantesimo se usato su un non morto, Tiro per Colpire con incantesimo a tocco, lo danneggia dello stesso ammontare.

Questo incantesimo se non esplicitato diversamente non può essere usato su animali o piante.

Usando 3 Punti Magia alla formulazione dell'incantesimo curi un ammontare di Punti Ferita pari a 3d8 + 2*modificatore di caratteristica per incantesimi.

Usando 5 Punti Magia alla formulazione dell'incantesimo curi un ammontare di Punti Ferita pari a 5d8 + 3*modificatore di caratteristica per incantesimi.

Spendendo il triplo dei Punti Magia puoi curare fino a 4 creature che si trovino entro 6 metri da te.

\textbf{Per ogni Successo Critico Magico} ottenuto nella Prova di Magia curi 1d8 Punti Ferita in più.

\textbf{NOTA}: Se incantatore e creatura curata sono entrambi Seguaci dello stesso Patrono l'incantesimo cura 1d8 in più.

\textbf{NOTA}: Se incantatore e creatura curata sono entrambi Devoti dello stesso Patrono ogni valore sul dado pari a 1,2,3 sarà considerato 4.

\textbf{NOTA}: l'incantesimo quando lanciato dalla Lista Elementale dell'Acqua non può essere usato con più di 1 Punto Magia.

\smallskip\noindent\rule{\linewidth}{2pt} \index[Incantesimi]{Dardo arcano}\hypertarget{Dardo arcano}{}\smallskip\noindent{\textbf{Dardo arcano}}\pdfbookmark[3]{Dardo arcano}{Dardo arcano}
\noindent
\begin{description}[noitemsep, topsep=0pt, parsep=0pt, partopsep=0pt, leftmargin=0cm, labelwidth=2.8cm]
	\item[\textbf{Lista di Magia}]: Universale
	\item[\textbf{Livello}]: 1, Comune
	\item[\textbf{T. di Lancio}]: 1 Azioni
	\item[\textbf{Gittata}]: 36 metri
	\item[\textbf{Componenti}]: V, S
	\item[\textbf{Durata}]: 1 Turno, Concentrazione
\end{description}

Crei un dardo luminoso di forza magica. Il dardo colpisce una creatura a gittata che puoi vedere, scelta da te. Un dardo infligge 1d4 + 1 danni da forza al suo bersaglio e li puoi dirigere perché colpiscano una o più creature.

Il danno aumenta di 1 ogni due volte che hai preso Adepto della Magia fino ad un massimo di 4 aumenti.

Lanciare uno o più dardi già evocati costa 1 Azione.

Puoi creare un dardo aggiuntivo quando raggiungi CM 3, CM 5, CM 7 e CM 9, ma l'incantesimo costa un Punto Magia addizionale.

Per ogni Azione nel round dedicata al lancio dell'incantesimo oltre la prima manifesti 1 dardo in più.

\textbf{Per ogni Successo Critico Magico} ottenuto nella Prova di Magia l'incantesimo crea un dardo aggiuntivo.

\smallskip\noindent\rule{\linewidth}{2pt} \index[Incantesimi]{Dardo di Fuoco}\hypertarget{Dardo di Fuoco}{}\smallskip\noindent{\textbf{Dardo di Fuoco}}\pdfbookmark[3]{Dardo di Fuoco}{Dardo di Fuoco}
\noindent
\begin{description}[noitemsep, topsep=0pt, parsep=0pt, partopsep=0pt, leftmargin=0cm, labelwidth=2.8cm]
	\item[\textbf{Lista di Magia}]: Fuoco
	\item[\textbf{Livello}]: 1, Comune
	\item[\textbf{T. di Lancio}]: 1 Azione
	\item[\textbf{Gittata}]: 36 metri
	\item[\textbf{Componenti}]: V, S
	\item[\textbf{Durata}]: Istantanea
\end{description}

Scagli una scintilla infuocata a una creatura od oggetto a gittata. Effettua un attacco a distanza con incantesimo contro il bersaglio. Se colpisci il bersaglio subisce 1d10 danni da fuoco. Un oggetto infiammabile colpito da questo incantesimo prende fuoco, se non è indossato o trasportato.

Puoi aumentare il danno dell'incantesimo di 1d8 quando raggiungi CM 5, CM 11 e CM 17 ma costa 2 Azioni lanciarlo potenziato e 2 Punti Magia, è altresì necessario avere preso Adepto della Magia un numero di volte pari ai potenziamenti che si vogliono applicare.

\textbf{Per ogni Successo Critico Magico ottenuto} nella Prova di Magia scagli una scintilla ulteriore.

\smallskip\noindent\rule{\linewidth}{2pt} \index[Incantesimi]{Dardo occulto}\hypertarget{Dardo occulto}{}\smallskip\noindent{\textbf{Dardo occulto}}\pdfbookmark[3]{Dardo occulto}{Dardo occulto}
\noindent
\begin{description}[noitemsep, topsep=0pt, parsep=0pt, partopsep=0pt, leftmargin=0cm, labelwidth=2.8cm]
	\item[\textbf{Lista di Magia}]: Invocazione
	\item[\textbf{Livello}]: 1, Comune
	\item[\textbf{T. di Lancio}]: 1 Azione
	\item[\textbf{Gittata}]: 36 metri
	\item[\textbf{Componenti}]: V, S
	\item[\textbf{Durata}]: Istantanea
\end{description}

Un fascio di energia crepitante si dirige verso una creatura a gittata. Effettua un attacco a distanza con incantesimo contro il bersaglio. Se colpisci, il bersaglio subisce 1d8 danni da forza.

Puoi aumentare il danno dell'incantesimo di 1d8 quando raggiungi CM 5, CM 11 e CM 17 ma costa 2 Azioni lanciarlo potenziato e 2 Punti Magia, è altresì necessario avere preso Adepto della Magia un numero di volte pari ai potenziamenti che si vogliono applicare.

\textbf{Ogni Successo Critico Magico ottenuto} nella Prova di Magia crei un altro fascio di energia.

\smallskip\noindent\rule{\linewidth}{2pt} \index[Incantesimi]{Dardo Tracciante}\hypertarget{Dardo Tracciante}{}\smallskip\noindent{\textbf{Dardo Tracciante}}\pdfbookmark[3]{Dardo Tracciante}{Dardo Tracciante}
\noindent
\begin{description}[noitemsep, topsep=0pt, parsep=0pt, partopsep=0pt, leftmargin=0cm, labelwidth=2.8cm]
	\item[\textbf{Lista di Magia}]: Invocazione
	\item[\textbf{Livello}]: 1, Non Comune
	\item[\textbf{T. di Lancio}]: 2 Azioni
	\item[\textbf{Gittata}]: 36 metri
	\item[\textbf{Componenti}]: V, S
	\item[\textbf{Durata}]: 1 round
\end{description}

Un lampo di luce viaggia verso una creatura a gittata, scelta da te. Effettua un attacco a distanza con incantesimo contro il bersaglio. Se colpisci, il bersaglio subisce 2d6 danni da Luce ed il prossimo Tiro per Colpire effettuato contro di lui prima del termine del tuo prossimo round ha +1d6 al TC, grazie alla mistica luce fioca che continuerà a brillare intorno al bersaglio fino ad allora.

\textbf{Per ogni Successo Critico Magico} ottenuto nella Prova di Magia il danno aumenta di 1d6.

\smallskip\noindent\rule{\linewidth}{2pt} \index[Incantesimi]{Danza Irresistibile}\hypertarget{Danza Irresistibile}{}\smallskip\noindent{\textbf{Danza Irresistibile}}\pdfbookmark[3]{Danza Irresistibile}{Danza Irresistibile}
\noindent
\begin{description}[noitemsep, topsep=0pt, parsep=0pt, partopsep=0pt, leftmargin=0cm, labelwidth=2.8cm]
	\item[\textbf{Lista di Magia}]: Ammaliamento
	\item[\textbf{Livello}]: 8, Leggendario
	\item[\textbf{T. di Lancio}]: 2 Azioni
	\item[\textbf{Gittata}]: 9 metri
	\item[\textbf{Componenti}]: V
	\item[\textbf{Durata}]: 1 minuto
\end{description}

Scegli una creatura a gittata e che puoi vedere. Il bersaglio comincia un comico balletto sul posto: agitando le gambe, battendo i piedi e saltellando per la durata. Le creature che non possono essere affascinate sono immuni a questo incantesimo.

Una creatura che balla deve usare 2 Azioni di Movimento per ballare senza lasciare il suo spazio e ha -1d6 ai Tiri Salvezza su Riflessi e i Tiri per Colpire. Mentre il bersaglio è soggetto a questo incantesimo le altre creature hanno +1d6 ai Tiri per Colpire contro di lui. Spendendo 1 Azione la creatura che balla può effettuare un nuovo Tiro Salvezza su Volontà per recuperare il controllo di se stessa. Se lo supera, l'incantesimo ha fine. Mentre balla si considera Distratto.

\textbf{Se ottieni 2 Successo Critico Magico} la durata aumenta di 1 ora

\smallskip\noindent\rule{\linewidth}{2pt} \index[Incantesimi]{Desiderio}\hypertarget{Desiderio}{}\smallskip\noindent{\textbf{Desiderio}}\pdfbookmark[3]{Desiderio}{Desiderio}
\noindent
\begin{description}[noitemsep, topsep=0pt, parsep=0pt, partopsep=0pt, leftmargin=0cm, labelwidth=2.8cm]
	\item[\textbf{Lista di Magia}]: Evocazione
	\item[\textbf{Livello}]: 9, Leggendario
	\item[\textbf{T. di Lancio}]: 2 Azioni
	\item[\textbf{Gittata}]: Personale
	\item[\textbf{Componenti}]: V,S,M (gemme per 20000 mo)
	\item[\textbf{Durata}]: Istantanea
\end{description}

Desiderio è il più potente incantesimo che una creatura mortale possa lanciare. Semplicemente parlando ad alta voce e consumando le gemme tenute in mano, puoi modificare le stesse fondamenta della realtà a seconda dei tuoi bisogni.

L'uso basilare di questo incantesimo è quello di riprodurre l'effetto di qualsiasi altro incantesimo con livello 8 o meno. Non devi soddisfare nessuno dei requisiti dell'incantesimo, comprese le componenti materiali costose. L'incantesimo ha semplicemente effetto.

In alternativa, puoi creare uno dei seguenti effetti a tua scelta:

\begin{itemize}[leftmargin=*] \setlength{\itemsep}{0pt}
	\item Crei un oggetto del valore massimo di 25000 mo, che non sia un oggetto magico. L'oggetto non può avere dimensioni superiori ai 90 metri in qualsiasi dimensione, e compare in uno spazio non occupato sul terreno.
	\item Permetti fino a venti creature che puoi vedere di recuperare tutti i Punti Ferita, e termini tutti gli effetti su di loro descritti dall'incantesimo ristorare superiore.
	\item Conferisci a un massimo di dieci creature che puoi vedere la resistenza a un tipo di danno a tua scelta per 8 ore.
	\item Conferisci a un massimo di dieci creature che puoi vedere l'immunità a un singolo incantesimo o altro effetto magico per 8 ore. Per esempio, potresti rendere te e tutti tuoi compagni immuni all'attacco risucchia vita del lich.
	\item Annulli un qualsiasi evento recente obbligando a ritirare qualsiasi tiro effettuato nell'ultimo round (compreso il tuo ultimo round). La realtà si rimodella per assecondare il nuovo risultato. Puoi far sì che il nuovo tiro abbia +2d6 o -2d6, puoi scegliere se usare il tiro originale o il nuovo tiro. Potresti anche riuscire a ottenere altro, oltre gli obiettivi negli esempi di cui sopra.
\end{itemize}

\medskip
Definisci i tuoi desideri quanto più possibile al Narratore. Il Narratore ha grande spazio di manovra nel decidere cosa accada in questi casi; maggiore il desiderio, più grosse le probabilità che qualcosa vada storto. L'incantesimo potrebbe semplicemente fallire, l'effetto desiderato manifestarsi solo in parte, oppure potresti subire delle conseguenze impreviste, in base a come hai proferito il desiderio. Lo stress del lanciare questo incantesimo per creare qualsiasi effetto che non sia riprodurre un altro incantesimo ti indebolisce.

Dopo averne retto lo stress, ogni volta che lancerai un incantesimo, fino a che non avrai terminato una notte di riposo, subirai 1d10 danni da Vuoto per livello/2 dell'incantesimo lanciato. Questo danno non può essere ridotto o diminuito in alcun modo. Inoltre, la tua Costituzione scende a -3, se non è già a -3 o meno, per 2d4 giorni.

Per ciascun giorno che trascorri a riposare e non svolgere altro che un'attività leggera, il tuo tempo di recupero rimanente diminuisce di 2 giorni.

Tira 1d100, se fai da 1 a 33\% tu non sarai mai più in grado di lanciare desiderio a causa dello stress sofferto, 34\%-66\% invecchi di 5 anni, 67\%-99\% non succede alcun altro effetto particolare, 100\% recuperi immediatamente lo stress del lancio.

\textbf{In caso di 2 Successi Critici magico ottenuti} non subisci effetti collaterali dal lancio di Desiderio.


\smallskip\noindent\rule{\linewidth}{2pt} \index[Incantesimi]{Desiderio limitato}\hypertarget{Desiderio limitato}{}\smallskip\noindent{\textbf{Desiderio limitato}}\pdfbookmark[3]{Desiderio limitato}{Desiderio limitato}
\noindent
\begin{description}[noitemsep, topsep=0pt, parsep=0pt, partopsep=0pt, leftmargin=0cm, labelwidth=2.8cm]
	\item[\textbf{Lista di Magia}]: Evocazione
	\item[\textbf{Livello}]: 7, Molto Raro
	\item[\textbf{T. di Lancio}]: 2 Azioni
	\item[\textbf{Gittata}]: Personale
	\item[\textbf{Componenti}]: V,S,M (gemme per 1000 mo)
	\item[\textbf{Durata}]: Istantanea
\end{description}

\emph{Desiderio limitato} è un incantesimo estremamente potente e versatile, che permette all'incantatore di realizzare "quasi qualsiasi cosa".

L'uso basilare di questo incantesimo è quello di riprodurre l'effetto di qualsiasi altro incantesimo con livello 7 o meno. Non devi soddisfare nessuno dei requisiti dell'incantesimo, comprese le componenti materiali costose. L'incantesimo ha semplicemente effetto.

In alternativa, puoi creare uno dei seguenti effetti a tua scelta:

\begin{itemize}[leftmargin=*] \setlength{\itemsep}{0pt}
	\item Crei un oggetto del valore massimo di 3000 mo, che non sia un oggetto magico. L'oggetto non può avere dimensioni superiori ai 90 metri in qualsiasi dimensione, e compare in uno spazio non occupato sul terreno.
	\item Permetti fino a 5 creature che puoi vedere di recuperare tutti i Punti Ferita, e termini tutti gli effetti su di loro descritti dall'incantesimo ristorare superiore.
	\item Conferisci a un massimo di 5 creature che puoi vedere la resistenza a un tipo di danno a tua scelta per 8 ore.
	\item Conferisci a un massimo di 4 creature che puoi vedere l'immunità a un singolo incantesimo o altro effetto magico per 8 ore. Per esempio, potresti rendere te e tutti tuoi compagni immuni all'attacco risucchia vita del lich.
\end{itemize}

\medskip
Definisci i tuoi desideri quanto più possibile al Narratore. Il Narratore ha grande spazio di manovra nel decidere cosa accada in questi casi; maggiore il desiderio, più grosse le probabilità che qualcosa vada storto. L'incantesimo potrebbe semplicemente fallire, l'effetto desiderato manifestarsi solo in parte, oppure potresti subire delle conseguenze impreviste, in base a come hai proferito il desiderio.


\smallskip\noindent\rule{\linewidth}{2pt} \index[Incantesimi]{Destriero Fantasma}\hypertarget{Destriero Fantasma}{}\smallskip\noindent{\textbf{Destriero Fantasma}}\pdfbookmark[3]{Destriero Fantasma}{Destriero Fantasma}
\noindent
\begin{description}[noitemsep, topsep=0pt, parsep=0pt, partopsep=0pt, leftmargin=0cm, labelwidth=2.8cm]
	\item[\textbf{Lista di Magia}]: Illusione
	\item[\textbf{Livello}]: 3, Comune
	\item[\textbf{T. di Lancio}]: 1 minuto
	\item[\textbf{Gittata}]: 9 metri
	\item[\textbf{Componenti}]: V, S
	\item[\textbf{Durata}]: 1 ora
\end{description}

Una creatura quasi reale simile a un saurovallo di taglia Grande, appare sul terreno in uno spazio non occupato di tua scelta e a gittata. Decidi tu l'aspetto della creatura, e questa compare equipaggiata di sella, morso e briglia. Qualsiasi equipaggiamento creato dall'incantesimo svanisce in una nuvola di fumo se viene portato a più di 3 metri di distanza dal destriero. Per la durata, tu o una creatura di tua scelta potete cavalcare il destriero. La creatura usa le statistiche del Saurovallo da Galoppo, eccetto che ha velocità 30 metri e può percorrere 15 chilometri in un'ora, o 20 chilometri ad andatura veloce. Quando l'incantesimo termina, il destriero inizia gradualmente a svanire, dando al fantino 1 minuto per smontare di sella. L'incantesimo termina se usi un'Azione per interromperlo o se il destriero subisce danni.

\textbf{Per ogni Successo Critico Magico} ottenuto nella Prova di Magia la durata aumenta di un ora oppure crei una cavalcatura in più.

\smallskip\noindent\rule{\linewidth}{2pt} \index[Incantesimi]{Disco Fluttuante}\hypertarget{Disco Fluttuante}{}\smallskip\noindent{\textbf{Disco Fluttuante}}\pdfbookmark[3]{Disco Fluttuante}{Disco Fluttuante}
\noindent
\begin{description}[noitemsep, topsep=0pt, parsep=0pt, partopsep=0pt, leftmargin=0cm, labelwidth=2.8cm]
	\item[\textbf{Lista di Magia}]: Evocazione
	\item[\textbf{Livello}]: 1, Comune
	\item[\textbf{T. di Lancio}]: 2 Azioni
	\item[\textbf{Gittata}]: 9 metri
	\item[\textbf{Componenti}]: V, S, M (una goccia di mercurio)
	\item[\textbf{Durata}]: 2 ora
\end{description}

Questo incantesimo crea un piano di forza orizzontale leggermente concavo, perfettamente circolare, di 1 metro di diametro e 2,5 centimetri di spessore che fluttua a 1 metro da terra, in uno spazio non occupato di tua scelta a gittata e che puoi vedere. Il disco rimane attivo per la durata, e può sostenere 250 chili o 50 di Ingombro. Se gli viene poggiato sopra un peso superiore, l'incantesimo termina e tutto quello che vi si trova sopra cade a terra. Finché ti trovi entro 6 metri da esso, il disco è immobile. Se ti muovi più di 6 metri lontano da esso, il disco ti segue in modo da rimanere sempre a 6 metri da te. Può muoversi attraverso terreno disomogeneo, su e giù per le scale, pendenze e simili, ma non può superare cambi di altitudine di 3 o più metri. Per esempio, il disco non può attraversare un fossato profondo 3 metri, né potrebbe lasciare il fossato se fosse creato in fondo a esso. Il disco può essere afferrato dall'incantatore e spostato manualmente. Se ti allontani più di 30 metri dal disco (di solito perché non riesce ad aggirare un ostacolo nel seguirti) l'incantesimo termina.

\textbf{Per ogni Successo Critico Magico} ottenuto nella Prova di Magia la durata aumenta di 2 ore.

\smallskip\noindent\rule{\linewidth}{2pt} \index[Incantesimi]{Disintegrazione}\hypertarget{Disintegrazione}{}\smallskip\noindent{\textbf{Disintegrazione}}\pdfbookmark[3]{Disintegrazione}{Disintegrazione}
\noindent
\begin{description}[noitemsep, topsep=0pt, parsep=0pt, partopsep=0pt, leftmargin=0cm, labelwidth=2.8cm]
	\item[\textbf{Lista di Magia}]: Trasmutazione
	\item[\textbf{Livello}]: 6, Non Comune
	\item[\textbf{T. di Lancio}]: 2 Azioni
	\item[\textbf{Gittata}]: 18 metri
	\item[\textbf{Componenti}]: V, S, M (una calamita e un pizzico di polvere)
	\item[\textbf{Durata}]: Istantanea
\end{description}

Un sottile raggio verde parte dal tuo dito puntato verso un bersaglio a gittata e che puoi vedere. Il bersaglio può essere una creatura, un oggetto o una creazione di forza magica, come un muro creato da muro di forza. Una creatura bersaglio di questo incantesimo deve effettuare un Tiro Salvezza su Tempra. Il bersaglio subisce 10d6 + 40 danni da forza se fallisce il Tiro Salvezza, la metà del danno se riesce. Se questo danno riduce il bersaglio a 0 Punti Ferita, questi è disintegrato. Una creatura disintegrata e tutto quello che indossa e trasporta, eccetto gli oggetti magici, viene ridotta a un mucchietto di sottile polvere grigia. La creatura può essere riportata in vita solo tramite l'intervento di un Patrono.

Questo incantesimo disintegra automaticamente gli oggetti non magici o una creazione di forza magica di taglia Grande o più piccola. Se il bersaglio è un oggetto non magico o una creazione di forza di taglia Enorme o più grossa, questo incantesimo disintegra una porzione di essa pari a una sfera di 1 metro di raggio.

\textbf{Per ogni Successo Critico Magico} ottenuto nella Prova di Magia danno aumenta di 5d6.

\textbf{Tiro Salvezza Successo/Fallimento Critico}: In caso di Fallimento Critico il danno raddoppia, in caso di Successo Critico il danno viene ulteriormente dimezzato

\smallskip\noindent\rule{\linewidth}{2pt} \index[Incantesimi]{Dissolvi il Bene e il Male}\hypertarget{Dissolvi il Bene e il Male}{}\smallskip\noindent{\textbf{Dissolvi il Bene e il Male}}\pdfbookmark[3]{Dissolvi il Bene e il Male}{Dissolvi il Bene e il Male}
\noindent
\begin{description}[noitemsep, topsep=0pt, parsep=0pt, partopsep=0pt, leftmargin=0cm, labelwidth=2.8cm]
	\item[\textbf{Lista di Magia}]: Abiurazione
	\item[\textbf{Livello}]: 5, Raro
	\item[\textbf{T. di Lancio}]: 2 Azioni
	\item[\textbf{Gittata}]: Personale
	\item[\textbf{Componenti}]: V, S, M (Acqua santa o argento e ferro in polvere)
	\item[\textbf{Durata}]: Concentrazione, 1 minuto
\end{description}

Un'energia luminosa ti circonda e ti protegge da fatati, non morti e creature originarie di luoghi al di là del Piano Materiale. Per la durata, i celestiali, elementali, fatati, demoni e non morti hanno -1d6 ai Tiri per Colpire contro di te. Puoi terminare l'incantesimo anticipatamente usando una delle seguenti funzioni speciali.

\emph{Spezzare Ammaliamento}. Con un'Azione, puoi entrare in contatto con una creatura affascinata, spaventata o posseduta da un celestiale, elementale, fatato, demoni o non morto. La creatura con cui sei in contatto non è più affascinata, spaventata o posseduta da queste creature.

\emph{Congedo}. Con un'Azione, effettua un attacco da mischia contro un celestiale, elementale, fatato, demone o non morto nella tua portata. Se lo colpisci, puoi cercare di rimandare la creatura al suo piano di origine. La creatura deve superare un Tiro Salvezza su Volontà o venire rispedita sul suo piano nativo (se non vi si trova già). Se non si trovano sul loro piano nativo, i non morti vengono rispediti nel Mondo delle Ombre e i fatati nel Primo Mondo.

\smallskip\noindent\rule{\linewidth}{2pt} \index[Incantesimi]{Dissolvi Magie}\hypertarget{Dissolvi Magie}{}\smallskip\noindent{\textbf{Dissolvi Magie}}\pdfbookmark[3]{Dissolvi Magie}{Dissolvi Magie}\hypertarget{dissolvimagie}{}
\noindent
\begin{description}[noitemsep, topsep=0pt, parsep=0pt, partopsep=0pt, leftmargin=0cm, labelwidth=2.8cm]
	\item[\textbf{Lista di Magia}]: Abiurazione
	\item[\textbf{Livello}]: 3, Comune
	\item[\textbf{T. di Lancio}]: 2 Azioni
	\item[\textbf{Gittata}]: 36 metri
	\item[\textbf{Componenti}]: V, S
	\item[\textbf{Durata}]: Istantanea
\end{description}

Scegli una creatura, oggetto o effetto magico a gittata. Qualsiasi incantesimo di livello 2 o più basso sul bersaglio ha fine.

Se l'incantesimo è tra il 3 ed il 5 livello è necessaria una prova di \hyperlink{contrastareincantesimi}{contrastare incantesimi} (pag. \pageref{contrastareincantesimi}).

Un effetto magico permanente viene soppresso temporaneamente per 10 minuti.

\smallskip\noindent\rule{\linewidth}{2pt} \index[Incantesimi]{Dissolvi Magie Avanzato}\hypertarget{Dissolvi Magie Avanzato}{}\smallskip\noindent{\textbf{Dissolvi Magie Avanzato}}\pdfbookmark[3]{Dissolvi Magie Avanzato}{Dissolvi Magie Avanzato}\hypertarget{dissolvimagieavanzato}{}
\noindent
\begin{description}[noitemsep, topsep=0pt, parsep=0pt, partopsep=0pt, leftmargin=0cm, labelwidth=2.8cm]
	\item[\textbf{Lista di Magia}]: Abiurazione
	\item[\textbf{Livello}]: 5, Raro
	\item[\textbf{T. di Lancio}]: 3 Azioni
	\item[\textbf{Gittata}]: 36 metri
	\item[\textbf{Componenti}]: V, S, M (polvere di diamante per un valore di 200 mo)
	\item[\textbf{Durata}]: Istantanea
\end{description}

Scegli una creatura, oggetto o effetto magico a gittata. Qualsiasi incantesimo di livello 4 o più basso sul bersaglio ha fine.

Se l'incantesimo è di livello maggiore al quarto è necessaria una prova di \hyperlink{contrastareincantesimi}{contrastare incantesimi} (pag. \pageref{contrastareincantesimi}).

Un effetto magico permanente viene soppresso temporaneamente per 10 minuti.

\textbf{Nota}: se si ottengono 3 Successi Critici Magici disperde permanentemente un effetto su un oggetto non artefatto.

\smallskip\noindent\rule{\linewidth}{2pt} \index[Incantesimi]{Distruggere nonmorto}\hypertarget{Distruggere nonmorto}{}\smallskip\noindent{\textbf{Distruggere nonmorto}}\pdfbookmark[3]{Distruggere nonmorto}{Distruggere nonmorto}
\noindent
\begin{description}[noitemsep, topsep=0pt, parsep=0pt, partopsep=0pt, leftmargin=0cm, labelwidth=2.8cm]
	\item[\textbf{Lista di Magia}]: Cura
	\item[\textbf{Livello}]: 3, Non Comune
	\item[\textbf{T. di Lancio}]: 2 Azioni
	\item[\textbf{Gittata}]: 36 metri
	\item[\textbf{Componenti}]: V, S, M (una reliquia di un Devoto di Thaft o Sumkjr)
	\item[\textbf{Durata}]: Istantanea
\end{description}

Scegli un nonmorto entro 36 metri. Un raggio luminoso si propaga dalla tua mano avvolgendo la creatura. Il nonmorto effettua un Tiro Salvezza su Tempra per dimezzare 4d12 danni da energia positiva.

\textbf{Per ogni Successo Critico Magico} ottenuto nella Prova di Magia il danno aumenta di 2d12.

\textbf{Tiro Salvezza Successo/Fallimento Critico}: In caso di Fallimento Critico il danno raddoppia, in caso di Successo Critico il danno viene ulteriormente dimezzato

\smallskip\noindent\rule{\linewidth}{2pt} \index[Incantesimi]{Dito}\hypertarget{Dito}{}\smallskip\noindent{\textbf{Dito}}\pdfbookmark[3]{Dito}{Dito}
\noindent
\begin{description}[noitemsep, topsep=0pt, parsep=0pt, partopsep=0pt, leftmargin=0cm, labelwidth=2.8cm]
	\item[\textbf{Lista di Magia}]: Ammaliamento
	\item[\textbf{Livello}]: 0, Raro
	\item[\textbf{T. di Lancio}]: 1 Azione
	\item[\textbf{Gittata}]: 18 metri
	\item[\textbf{Componenti}]: S
	\item[\textbf{Durata}]: 3 round
\end{description}

Fai il dito (o pernacchia o gesto dell'ombrello) all'avversario che deve poterlo vedere.

Questo deve fare un Tiro Salvezza su Volontà, se riesce non succede nulla.

Se fallisce il Tiro Salvezza in maniera critica, per i prossimi 2 round ha una penalità di 2 ai Tiri per Colpire, TS ed alle prove di Competenza di Base.

Se fallisce il TS di 3 o 4, viene mortificato, fino alla fine del prossimo round ha una penalità di 1 ai Tiri per Colpire e Competenza.

Se fallisce il TS di 2 o 1, è punito, fino alla fine del prossimo round ha una penalità di 1 ai Tiri per Colpire o Difesa (scelta dell'obiettivo).

\textbf{Per ogni Successo Critico Magico} ottenuto nella Prova di Magia puoi influenzare una altra creatura che possa vedere il gesto.

\smallskip\noindent\rule{\linewidth}{2pt} \index[Incantesimi]{Dito della Morte}\hypertarget{Dito della Morte}{}\smallskip\noindent{\textbf{Dito della Morte}}\pdfbookmark[3]{Dito della Morte}{Dito della Morte}
\noindent
\begin{description}[noitemsep, topsep=0pt, parsep=0pt, partopsep=0pt, leftmargin=0cm, labelwidth=2.8cm]
	\item[\textbf{Lista di Magia}]: Necromanzia
	\item[\textbf{Livello}]: 6, Raro
	\item[\textbf{T. di Lancio}]: 2 Azioni
	\item[\textbf{Gittata}]: 18 metri
	\item[\textbf{Componenti}]: V, S
	\item[\textbf{Durata}]: Istantanea
\end{description}

Invii una scarica di energia negativa a una creatura a gittata e che puoi vedere, provocandole profondo dolore. Il bersaglio deve effettuare un Tiro Salvezza su Tempra. Il bersaglio subisce 7d8 + 30 danni da Vuoto se fallisce il Tiro Salvezza, o la metà di questi danni se lo supera.

Un umanoide ucciso da questo incantesimo si rianima come zombi sotto il tuo comando permanente all'inizio del tuo prossimo round, e seguirà i tuoi ordini verbali al meglio delle sue capacità.

\textbf{Per ogni Successo Critico Magico} ottenuto nella Prova di Magia il danno aumenta di 4d8.

\textbf{Tiro Salvezza Successo/Fallimento Critico}: In caso di Fallimento Critico il danno raddoppia, in caso di Successo Critico il danno viene ulteriormente dimezzato

\smallskip\noindent\rule{\linewidth}{2pt} \index[Incantesimi]{Divinazione}\hypertarget{Divinazione}{}\smallskip\noindent{\textbf{Divinazione}}\pdfbookmark[3]{Divinazione}{Divinazione}
\noindent
\begin{description}[noitemsep, topsep=0pt, parsep=0pt, partopsep=0pt, leftmargin=0cm, labelwidth=2.8cm]
	\item[\textbf{Lista di Magia}]: Divinazione
	\item[\textbf{Livello}]: 6, Raro
	\item[\textbf{T. di Lancio}]: 2 Azioni
	\item[\textbf{Gittata}]: Personale
	\item[\textbf{Componenti}]: V, S, M (incenso e un'offerta sacrificale appropriata alla tua religione, il cui valore complessivo sia di 25 mo, che saranno consumati dall'incantesimo)
	\item[\textbf{Durata}]: Istantanea
\end{description}

La tua magia e un'offerta votiva ti mettono in comunicazione con un Patrono o il servitore di un Patrono. Gli puoi porre una singola domanda in merito a uno specifico obiettivo, evento o attività che debba verificarsi entro 7 giorni. Il Narratore dà una risposta veritiera. La replica potrebbe essere una breve frase, una rima criptica o un presagio.

L'incantesimo non tiene conto di ogni possibile circostanza che possa modificare il risultato, come il lancio di ulteriori incantesimi o la perdita o l'arrivo di un alleato.

Se lanci l'incantesimo due o più volte prima di aver riposato almeno 8 ore, c'è una probabilità cumulativa del 25\% che per ogni lancio dopo il primo tu ottenga una lettura erronea. Il Narratore effettua questo tiro in segreto.

\textbf{NOTA}: l'incantesimo deve essere formulato da almeno un Seguace

\smallskip\noindent\rule{\linewidth}{2pt} \index[Incantesimi]{Dominare Bestie}\hypertarget{Dominare Bestie}{}\smallskip\noindent{\textbf{Dominare Bestie}}\pdfbookmark[3]{Dominare Bestie}{Dominare Bestie}
\noindent
\begin{description}[noitemsep, topsep=0pt, parsep=0pt, partopsep=0pt, leftmargin=0cm, labelwidth=2.8cm]
	\item[\textbf{Lista di Magia}]: Ammaliamento, Animali e Piante
	\item[\textbf{Livello}]: 4, Molto Raro - Comune
	\item[\textbf{T. di Lancio}]: 2 Azioni
	\item[\textbf{Gittata}]: 18 metri
	\item[\textbf{Componenti}]: V, S
	\item[\textbf{Durata}]: Concentrazione, massimo 1 minuto
\end{description}

Cerchi di affascinare una bestia a gittata che puoi vedere. Essa deve superare un Tiro Salvezza su Volontà o restare affascinata per la durata, ricevendo +1d6 al tiro se tu o i tuoi alleati la state combattendo.

Mentre la bestia è affascinata, finché voi due vi trovate sullo stesso piano di esistenza mantieni un collegamento telepatico con essa. Puoi usare questo collegamento telepatico per inviare comandi alla creatura mentre sei cosciente (richiede 1 Azione), a cui essa obbedirà al suo meglio. Puoi specificare un corso d'azione semplice e generico, come \emph{Attacca quella creatura}, \emph{Corri laggiù}, o \emph{Prendi quell'oggetto}. Se la creatura completa l'ordine e non riceve ulteriori indicazioni da te, si difenderà e preserverà al meglio delle sue capacità.

Puoi impiegare 2 tue azioni per assumere il totale e preciso controllo del bersaglio. Fino al termine del tuo prossimo round, il bersaglio effettuerà solo le azioni decise da te, e non farà nulla che tu non gli permetta di fare. Durante questo periodo, puoi anche far usare una Azione di Reazione al bersaglio, ma ciò richiede l'uso della tua Reazione.

Ogni volta che il bersaglio subisce danni, effettua un nuovo Tiro Salvezza su Volontà contro l'incantesimo. Se supera il Tiro Salvezza, l'incantesimo termina. La bestia non può avere GS superiore a 4.

\textbf{Per ogni Successo Critico Magico} ottenuto nella Prova di Magia la durata raddoppia fino ad un massimo di 8 ore. Ogni 2 Successi Magici Critici puoi comandare una bestia in più oppure aumenti il GS comandabile di 1.

\smallskip\noindent\rule{\linewidth}{2pt} \index[Incantesimi]{Dominare Mostri}\hypertarget{Dominare Mostri}{}\smallskip\noindent{\textbf{Dominare Mostri}}\pdfbookmark[3]{Dominare Mostri}{Dominare Mostri}
\noindent
\begin{description}[noitemsep, topsep=0pt, parsep=0pt, partopsep=0pt, leftmargin=0cm, labelwidth=2.8cm]
	\item[\textbf{Lista di Magia}]: Ammaliamento
	\item[\textbf{Livello}]: 8, Non Comune
	\item[\textbf{T. di Lancio}]: 2 Azioni
	\item[\textbf{Gittata}]: 18 metri
	\item[\textbf{Componenti}]: V, S
	\item[\textbf{Durata}]: Concentrazione, massimo 1 ora
\end{description}

Cerchi di affascinare una creatura a gittata che puoi vedere. Essa deve superare un Tiro Salvezza su Volontà o restare affascinata per la durata, ricevendo +1d6 al tiro se tu o i tuoi alleati la state combattendo.

Mentre la creatura è affascinata, finché voi due vi trovate sullo stesso piano di esistenza mantieni un collegamento telepatico con essa. Puoi usare questo collegamento telepatico per inviare comandi alla creatura mentre sei cosciente (richiede 1 Azione), a cui essa obbedirà al suo meglio. Puoi specificare un corso d'azione semplice e generico, come \emph{Attacca quella creatura}, \emph{Corri laggiù}, o \emph{Prendi quell'oggetto}. Se la creatura completa l'ordine e non riceve ulteriori indicazioni da te, si difenderà e preserverà al meglio delle sue capacità.

Puoi impiegare due tua Azioni per assumere il totale e preciso controllo del bersaglio. Fino al termine del tuo prossimo round la creatura effettuerà solo le azioni decise da te, e non farà nulla che tu non le permetta di fare. Durante questo periodo, puoi anche far usare una Azione di Reazione alla creatura, ma ciò richiede l'uso della tua Reazione. Ogni volta che il bersaglio subisce danni, effettua un nuovo Tiro Salvezza su Volontà contro l'incantesimo. Se supera il Tiro Salvezza, l'incantesimo termina.

\textbf{Per ogni Successo Critico Magico} ottenuto nella Prova di Magia la durata raddoppia fino ad un massimo di 8 ore.

\smallskip\noindent\rule{\linewidth}{2pt} \index[Incantesimi]{Dominare Persone}\hypertarget{Dominare Persone}{}\smallskip\noindent{\textbf{Dominare Persone}}\pdfbookmark[3]{Dominare Persone}{Dominare Persone}
\noindent
\begin{description}[noitemsep, topsep=0pt, parsep=0pt, partopsep=0pt, leftmargin=0cm, labelwidth=2.8cm]
	\item[\textbf{Lista di Magia}]: Ammaliamento
	\item[\textbf{Livello}]: 5, Non Comune
	\item[\textbf{T. di Lancio}]: 2 Azioni
	\item[\textbf{Gittata}]: 18 metri
	\item[\textbf{Componenti}]: V, S
	\item[\textbf{Durata}]: Concentrazione, massimo 1 minuto
\end{description}

Cerchi di affascinare un umanoide a gittata che puoi vedere. Esso deve superare un Tiro Salvezza su Volontà o restare Affascinato per la durata, ricevendo +1d6 al tiro se tu o i tuoi alleati lo state combattendo.

Mentre il bersaglio è affascinato, finché voi due vi trovate sullo stesso piano di esistenza mantieni un collegamento telepatico con esso. Puoi usare questo collegamento telepatico per inviare comandi al bersaglio mentre sei cosciente (richiede 1 Azione), a cui esso obbedirà al suo meglio. Puoi specificare un corso d'azione semplice e generico, come \emph{Attacca quella creatura}, \emph{Corri laggiù}, o \emph{Prendi quell'oggetto}. Se il bersaglio completa l'ordine e non riceve ulteriori indicazioni da te, si difenderà al meglio delle sue capacità.

Puoi impiegare 2 Azioni per assumere il totale e preciso controllo del bersaglio. Fino al termine del tuo prossimo round, il bersaglio effettuerà solo le azioni decise da te, e non farà nulla che tu non gli permetta di fare. Durante questo periodo, puoi anche far usare una Azione di Reazione al bersaglio, ma ciò richiede l'uso della tua Reazione. Ogni volta che il bersaglio subisce danni effettua un nuovo Tiro Salvezza su Volontà contro l'incantesimo. Se supera il Tiro Salvezza l'incantesimo termina.

\textbf{Per ogni Successo Critico Magico} ottenuto nella Prova di Magia la durata raddoppia fino ad un massimo di 8 ore.

\smallskip\noindent\rule{\linewidth}{2pt} \index[Incantesimi]{Eroismo}\hypertarget{Eroismo}{}\smallskip\noindent{\textbf{Eroismo}}\pdfbookmark[3]{Eroismo}{Eroismo}
\noindent
\begin{description}[noitemsep, topsep=0pt, parsep=0pt, partopsep=0pt, leftmargin=0cm, labelwidth=2.8cm]
	\item[\textbf{Lista di Magia}]: Ammaliamento
	\item[\textbf{Livello}]: 1, Non Comune
	\item[\textbf{T. di Lancio}]: 2 Azioni
	\item[\textbf{Gittata}]: Contatto
	\item[\textbf{Componenti}]: V, S
	\item[\textbf{Durata}]: 1 minuto
\end{description}

Una creatura consenziente con cui sei in contatto vene infusa di coraggio. Fino al termine dell'incantesimo, la creatura è immune all'essere spaventata e all'inizio di ciascun suo round ottiene Punti Ferita temporanei pari al tuo valore di modificatore da incantesimo. Quando l'incantesimo ha termine il bersaglio perde tutti i Punti Ferita temporanei rimasti.

\textbf{Per ogni Successo Critico Magico} ottenuto nella Prova di Magia puoi influenzare un altra creatura.

\smallskip\noindent\rule{\linewidth}{2pt} \index[Incantesimi]{Esilio}\hypertarget{Esilio}{}\smallskip\noindent{\textbf{Esilio}}\pdfbookmark[3]{Esilio}{Esilio}
\noindent
\begin{description}[noitemsep, topsep=0pt, parsep=0pt, partopsep=0pt, leftmargin=0cm, labelwidth=2.8cm]
	\item[\textbf{Lista di Magia}]: Abiurazione
	\item[\textbf{Livello}]: 4, Comune
	\item[\textbf{T. di Lancio}]: 2 Azioni
	\item[\textbf{Gittata}]: 18 metri
	\item[\textbf{Componenti}]: V, S, M (un oggetto disprezzato dal bersaglio)
	\item[\textbf{Durata}]: 1 minuto
\end{description}

Cerchi di spedire una creatura a gittata e che puoi vedere in un altro piano di esistenza. Il bersaglio deve superare un Tiro Salvezza su Volontà o venire esiliato. Se il bersaglio è natio del piano di esistenza in cui ti trovi, esili il bersaglio in un semipiano innocuo. Mentre è lì, il bersaglio è inabile. Il bersaglio rimane lì fino al termine dell'incantesimo, quando riapparirà nello spazio che aveva lasciato o nello spazio non occupato più vicino, se il suo spazio originale adesso è occupato. Se il bersaglio è natio di un diverso piano di esistenza da quello in cui ti trovi, il bersaglio svanisce emettendo un lieve scoppio, tornando al suo piano natio. Se l'incantesimo termina prima che sia trascorso 1 minuto, il bersaglio riappare nello spazio che aveva lasciato o nello spazio non occupato più vicino, se il suo spazio originale è occupato.

\textbf{Per ogni Successo Critico Magico} ottenuto nella Prova di Magia puoi influenzare un altra creatura, oppure la creatura è bandita per una settimana.

\smallskip\noindent\rule{\linewidth}{2pt} \index[Incantesimi]{Esplosione Solare}\hypertarget{Esplosione Solare}{}\smallskip\noindent{\textbf{Esplosione Solare}}\pdfbookmark[3]{Esplosione Solare}{Esplosione Solare}
\noindent
\begin{description}[noitemsep, topsep=0pt, parsep=0pt, partopsep=0pt, leftmargin=0cm, labelwidth=2.8cm]
	\item[\textbf{Lista di Magia}]: Invocazione
	\item[\textbf{Livello}]: 8, Raro
	\item[\textbf{T. di Lancio}]: 2 Azioni
	\item[\textbf{Gittata}]: 45 metri
	\item[\textbf{Componenti}]: V, S, M (fuoco e un pezzo di pietra di sole)
	\item[\textbf{Durata}]: Istantanea
\end{description}

Un'intensa luce solare illumina in un raggio di 18 metri centrato su di un punto a gittata, scelto da te. Tutte le creature all'interno della luce devono effettuare un Tiro Salvezza su Tempra. Se fallisce il Tiro Salvezza, una creatura subisce 12d6 danni da Luce e resta accecata per 1 minuto. Se lo supera, subisce la metà dei danni e non resta accecata dall'incantesimo. Non morti e melme hanno -8 a questo Tiro Salvezza. Una creatura accecata da questo incantesimo effettua un altro Tiro Salvezza su Tempra alla fine di ciascun suo round. Se supera il Tiro Salvezza, non è più accecata.

Nella sua area, questo incantesimo dissolve qualsiasi oscurità generata da un incantesimo.

\textbf{Per ogni Successo Critico Magico} ottenuto nella Prova di Magia il danno aumenta di 6d6.

\textbf{NOTA}: un Devoto di Ljust o Sumkjr ottiene un Successo Critico Magico automatico

\smallskip\noindent\rule{\linewidth}{2pt} \index[Incantesimi]{Estasiare}\hypertarget{Estasiare}{}\smallskip\noindent{\textbf{Estasiare}}\pdfbookmark[3]{Estasiare}{Estasiare}
\noindent
\begin{description}[noitemsep, topsep=0pt, parsep=0pt, partopsep=0pt, leftmargin=0cm, labelwidth=2.8cm]
	\item[\textbf{Lista di Magia}]: Ammaliamento
	\item[\textbf{Livello}]: 2, Comune
	\item[\textbf{T. di Lancio}]: 2 Azioni
	\item[\textbf{Gittata}]: Personale
	\item[\textbf{Componenti}]: V, S
	\item[\textbf{Durata}]: 1 minuto
\end{description}

Intessi una serie di parole svianti, facendo sì che delle creature di tua scelta entro la gittata, che puoi vedere e possano sentirti, effettuino un Tiro Salvezza su Volontà. Qualsiasi creatura che non può restare affascinata supera il Tiro Salvezza automaticamente, e se tu o i tuoi compagni state combattendo una creatura, questa ha +1d6 al Tiro Salvezza. Se fallisce il Tiro Salvezza, il bersaglio ha -1d6 sulle prove di Consapevolezza effettuate per percepire una qualsiasi creatura diversa da te fino al termine dell'incantesimo o finché il bersaglio non può più sentirti.

L'incantesimo termina se sei reso inabile o non puoi più parlare.

\smallskip\noindent\rule{\linewidth}{2pt} \index[Incantesimi]{Evoca Animali}\hypertarget{Evoca Animali}{}\smallskip\noindent{\textbf{Evoca Animali}}\pdfbookmark[3]{Evoca Animali}{Evoca Animali}
\noindent
\begin{description}[noitemsep, topsep=0pt, parsep=0pt, partopsep=0pt, leftmargin=0cm, labelwidth=2.8cm]
	\item[\textbf{Lista di Magia}]: Animali e Piante
	\item[\textbf{Livello}]: 3, Non Comune
	\item[\textbf{T. di Lancio}]: 3 Azioni
	\item[\textbf{Gittata}]: 18 metri
	\item[\textbf{Componenti}]: V, S
	\item[\textbf{Durata}]: 1 Turno
\end{description}

Evochi spiriti magici che assumono l'aspetto di bestie e compaiono in spazi non occupati a gittata e che puoi vedere. Scegli una delle seguenti opzioni per determinare ciò che appare:

\begin{itemize}[leftmargin=*] \setlength{\itemsep}{-1pt}
	\item Una bestia di grado di sfida 2 o inferiore
	\item Due bestie di grado di sfida 1 o inferiore
	\item Quattro bestie di grado di sfida 1/2 o inferiore
	\item Otto bestie di grado di sfida 1/4 o inferiore
\end{itemize}

Ogni bestia è considerata anche magica e sparisce quando scende a 0 Punti Ferita o quando l'incantesimo termina.

Le creature evocate sono amichevoli verso di te e i tuoi compagni e ubbidiscono al meglio delle loro capacità.

\textbf{Per ogni Successo Critico Magico} ottenuto nella Prova di Magia appariranno due bestie in più di grado inferiore o 1 bestia in più di grado superiore a quello inizialmente scelto.

\smallskip\noindent\rule{\linewidth}{2pt} \index[Incantesimi]{Evoca Cavalcatura}\hypertarget{Evoca Cavalcatura}{}\smallskip\noindent{\textbf{Evoca Cavalcatura}}\pdfbookmark[3]{Evoca Cavalcatura}{Evoca Cavalcatura}
\noindent
\begin{description}[noitemsep, topsep=0pt, parsep=0pt, partopsep=0pt, leftmargin=0cm, labelwidth=2.8cm]
	\item[\textbf{Lista di Magia}]: Animali e Piante
	\item[\textbf{Livello}]: 2, Comune
	\item[\textbf{T. di Lancio}]: 10 minuti
	\item[\textbf{Gittata}]: 9 metri
	\item[\textbf{Componenti}]: V, S
	\item[\textbf{Durata}]: 1 ora
\end{description}

Evochi uno spirito che assume la forma di una cavalcatura insolitamente intelligente, forte e leale, stabilendo un legame duraturo con esso. Apparendo in uno spazio a gittata, non occupato, il destriero assume la forma di tua scelta, come quella di un saurovallo da guerra, un saurovallo nano, un cammello, un alce o un mastino (il Narratore potrebbe darti la possibilità di evocare come destrieri anche altri tipi di animali). Il destriero ha le statistiche della forma scelta, sebbene sia di tipo celestiale, fatato o demone (a tua scelta) invece del suo normale tipo. Inoltre, se il tuo destriero ha Intelligenza -3 o meno, la sua Intelligenza diventa -2, e ottiene la capacità di comprendere un linguaggio a tua scelta tra quelli che sei in grado di parlare. Il tuo destriero serve da cavalcatura, sia in combattimento che fuori da esso, e possiedi un legame istintivo con esso, che vi permette di combattere come foste un unico insieme.

Quando il destriero scende a 0 Punti Ferita, scompare, non lasciandosi dietro alcuna forma fisica. puoi congedare il destriero in qualsiasi momento con un'Azione, facendolo sparire. In entrambi i casi, lanciare di nuovo questo incantesimo evoca lo stesso destriero, ripristinato al massimo dei suoi Punti Ferita.

Non puoi avere più di un destriero legato da questo incantesimo alla volta. Con un'Azione, puoi liberare il destriero da questo legame in qualsiasi momento, facendolo sparire.

\textbf{Per ogni Successo Critico Magico} ottenuto nella Prova di Magia l'incantesimo dura 2 ore in più.

\smallskip\noindent\rule{\linewidth}{2pt} \index[Incantesimi]{Evoca Elementale}\hypertarget{Evoca Elementale}{}\smallskip\noindent{\textbf{Evoca Elementale}}\pdfbookmark[3]{Evoca Elementale}{Evoca Elementale}
\noindent
\begin{description}[noitemsep, topsep=0pt, parsep=0pt, partopsep=0pt, leftmargin=0cm, labelwidth=2.8cm]
	\item[\textbf{Lista di Magia}]: Aria, Acqua, Terra, Fuoco
	\item[\textbf{Livello}]: 5, Raro
	\item[\textbf{T. di Lancio}]: 1 minuto
	\item[\textbf{Gittata}]: 27 metri
	\item[\textbf{Componenti}]: V, S, M (incenso bruciato per l'aria, argilla malleabile per la terra, zolfo e fosforo per il fuoco, o acqua e sabbia per l'acqua)
	\item[\textbf{Durata}]: Concentrazione, 1 Turno
\end{description}

Evochi un servitore elementale. Scegli un'area a gittata composta di acqua, aria, fuoco o terra e che riempia una sfera di 1 metro di raggio. Un elementale di grado di sfida 3 o minore appropriato all'area da te scelta compare in uno spazio non occupato entro 3 metri da essa. L'elementale sparisce quando scende a 0 Punti Ferita o l'incantesimo termina.

Ogni Lista di Magia può evocare solo il proprio Elementale specifico

\textbf{Ogni due Successo Critico Magico ottenuto} nella Prova di Magia il grado di sfida dell'elementale evocato aumenta di 1

\smallskip\noindent\rule{\linewidth}{2pt} \index[Incantesimi]{Evoca Elementali Minori}\hypertarget{Evoca Elementali Minori}{}\smallskip\noindent{\textbf{Evoca Elementali Minori}}\pdfbookmark[3]{Evoca Elementali Minori}{Evoca Elementali Minori}
\noindent
\begin{description}[noitemsep, topsep=0pt, parsep=0pt, partopsep=0pt, leftmargin=0cm, labelwidth=2.8cm]
	\item[\textbf{Lista di Magia}]: Aria, Acqua, Terra, Fuoco
	\item[\textbf{Livello}]: 4, Non Comune
	\item[\textbf{T. di Lancio}]: 1 minuto
	\item[\textbf{Gittata}]: 27 metri
	\item[\textbf{Componenti}]: V, S
	\item[\textbf{Durata}]: 1 Turno
\end{description}

Evochi degli elementali che compariranno in spazi non occupati a gittata e che puoi vedere. Scegli una della seguenti opzioni per decidere cosa appare:

\begin{itemize}[leftmargin=*] \setlength{\itemsep}{-1pt}
	\item Un elementale di grado di sfida 2 o meno
	\item Due elementali di grado di sfida 1 o meno
	\item Quattro elementali di grado di sfida 1/2 o meno
	\item Otto elementali di grado di sfida 1/4 o meno
\end{itemize}

Un elementale evocato sparisce quando scende a 0 Punti Ferita o l'incantesimo termina.

Ogni Lista di Magia può evocare solo il proprio Elementale specifico. L'elementale è amichevole verso di te ed i tuoi compagni e ubbidisce al meglio delle sue capacità.

\textbf{Per ogni Successo Critico Magico} ottenuto nella Prova di Magia appariranno due elementale in più di grado inferiore o 1 elementale in più di grado superiore a quello inizialmente scelto.

\smallskip\noindent\rule{\linewidth}{2pt} \index[Incantesimi]{Evocazioni Istantanee}\hypertarget{Evocazioni Istantanee}{}\smallskip\noindent{\textbf{Evocazioni Istantanee}}\pdfbookmark[3]{Evocazioni Istantanee}{Evocazioni Istantanee}
\noindent
\begin{description}[noitemsep, topsep=0pt, parsep=0pt, partopsep=0pt, leftmargin=0cm, labelwidth=2.8cm]
	\item[\textbf{Lista di Magia}]: Evocazione
	\item[\textbf{Livello}]: 6, Raro
	\item[\textbf{T. di Lancio}]: 1 minuto
	\item[\textbf{Gittata}]: Contatto
	\item[\textbf{Componenti}]: V, S, M (uno zaffiro del valore di 1000 mo)
	\item[\textbf{Durata}]: Fino a che dissolto
\end{description}

Entri a contatto con un oggetto del peso di 5 chili o meno e la cui dimensione più grossa non superi i 180 centimetri. L'incantesimo lascia un marchio sulla superficie dell'oggetto e ne incide invisibilmente il nome sullo zaffiro usato come componente materiale. Ogni volta che lanci questo incantesimo, devi usare uno zaffiro diverso.

In qualsiasi momento successivo, puoi usare 2 Azioni per pronunciare il nome dell'oggetto e frantumare lo zaffiro. L'oggetto appare istantaneamente nella tua mano quale che sia la distanza fisica o planare che vi separa, e l'incantesimo ha termine.

Se un'altra creatura sta impugnando o trasportando l'oggetto, frantumare lo zaffiro non trasporterà l'oggetto da te, ma invece apprenderai chi sia la creatura che ne è in possesso e indicativamente dove si trovi in questo momento.

Dissolvi magie, o un effetto simile applicato con successo allo zaffiro, termina l'effetto dell'incantesimo.

\smallskip\noindent\rule{\linewidth}{2pt} \index[Incantesimi]{Fabbricare}\hypertarget{Fabbricare}{}\smallskip\noindent{\textbf{Fabbricare}}\pdfbookmark[3]{Fabbricare}{Fabbricare}
\noindent
\begin{description}[noitemsep, topsep=0pt, parsep=0pt, partopsep=0pt, leftmargin=0cm, labelwidth=2.8cm]
	\item[\textbf{Lista di Magia}]: Trasmutazione
	\item[\textbf{Livello}]: 4, Comune
	\item[\textbf{T. di Lancio}]: 10 minuti
	\item[\textbf{Gittata}]: 36 metri
	\item[\textbf{Componenti}]: V, S
	\item[\textbf{Durata}]: Istantanea
\end{description}

Converti le materie prime in prodotti finiti dello stesso materiale. Per esempio, puoi fabbricare un piccolo ponte di legno da un cumulo di alberi, una corda da un mucchio di canapa, e abiti dal lino o la lana. Scegli le materie prima che puoi vedere a gittata. Puoi fabbricare un oggetto di taglia Grande o inferiore (contenuto in un cubo di 3 metri di spigolo, o otto cubi connessi di 1 metro di spigolo) data una sufficiente quantità di materie prime. Se stai lavorando con il metallo, la pietra o altre sostanze minerali, l'oggetto fabbricato non può essere più grande di taglia Media (contenuto in un singolo cubo di 1 metro di spigolo). La qualità degli oggetti creati da questo incantesimo è commisurata alla qualità delle materie prime.

Tramite questo incantesimo non si possono creare o trasmutare creature od oggetti magici. Inoltre non puoi usarlo per creare oggetti che normalmente richiedono un alto livello di lavorazione, come i gioielli, le armi, il vetro o le armature, a meno che tu non abbia la competenza con il tipo di strumenti da artigiano utilizzati per costruire questi oggetti. In caso di critico nella Prova di Magia si possono processare più volumi o produrre con maggiore qualità.

\smallskip\noindent\rule{\linewidth}{2pt} \index[Incantesimi]{Fatale}\hypertarget{Fatale}{}\smallskip\noindent{\textbf{Fatale}}\pdfbookmark[3]{Fatale}{Fatale}
\noindent
\begin{description}[noitemsep, topsep=0pt, parsep=0pt, partopsep=0pt, leftmargin=0cm, labelwidth=2.8cm]
	\item[\textbf{Lista di Magia}]: Illusione
	\item[\textbf{Livello}]: 9, Raro
	\item[\textbf{T. di Lancio}]: 2 Azioni
	\item[\textbf{Gittata}]: 36 metri
	\item[\textbf{Componenti}]: V, S
	\item[\textbf{Durata}]: 1 minuto
\end{description}

Attingendo alle paure più intime di un gruppo di creature, crei delle creature illusorie nella loro mente, visibili solo a loro. Ogni creatura in una sfera di 9 metri di raggio centrata su di un punto a tua scelta nella gittata, deve effettuare un Tiro Salvezza su Volontà. Se fallisce il Tiro Salvezza, la creatura diventa spaventata per la durata L'illusione affonda nelle paure più intime della creatura, manifestando i suoi incubi peggiori come una implacabile minaccia. Alla fine di ciascun round della creatura spaventata, questa deve superare un Tiro Salvezza su Volontà o subire 4d10 danni da forza. Se supera il Tiro Salvezza, per quella creatura l'incantesimo ha termine.

\textbf{Per due Successo Critico Magico} ottenuto nella Prova di Magia il danno aumenta di 4d10.

\smallskip\noindent\rule{\linewidth}{2pt} \index[Incantesimi]{Favore Divino}\hypertarget{Favore Divino}{}\smallskip\noindent{\textbf{Favore Divino}}\pdfbookmark[3]{Favore Divino}{Favore Divino}
\noindent
\begin{description}[noitemsep, topsep=0pt, parsep=0pt, partopsep=0pt, leftmargin=0cm, labelwidth=2.8cm]
	\item[\textbf{Lista di Magia}]: Invocazione
	\item[\textbf{Livello}]: 1, Non Comune
	\item[\textbf{T. di Lancio}]: 1 Azione
	\item[\textbf{Gittata}]: Personale
	\item[\textbf{Componenti}]: V, S
	\item[\textbf{Durata}]: 1 minuto
\end{description}

Le tue preghiere potenziano te e la tua arma. Fino al termine dell'incantesimo, quando colpisce, la tua arma infligge 1d4 danni da Luce aggiuntivi.

\textbf{Per ogni Successo Critico Magico} ottenuto nella Prova di Magia la tua arma causa +2 danno aggiuntivo da Luce.

\smallskip\noindent\rule{\linewidth}{2pt} \index[Incantesimi]{Ferire}\hypertarget{Ferire}{}\smallskip\noindent{\textbf{Ferire}}\pdfbookmark[3]{Ferire}{Ferire}
\noindent
\begin{description}[noitemsep, topsep=0pt, parsep=0pt, partopsep=0pt, leftmargin=0cm, labelwidth=2.8cm]
	\item[\textbf{Lista di Magia}]: Necromanzia
	\item[\textbf{Livello}]: 6, Non Comune
	\item[\textbf{T. di Lancio}]: 2 Azioni
	\item[\textbf{Gittata}]: 18 metri
	\item[\textbf{Componenti}]: V, S
	\item[\textbf{Durata}]: Istantanea
\end{description}

Scateni una malattia virulenta su di una creatura a gittata che puoi vedere. Il bersaglio deve effettuare un Tiro Salvezza su Tempra. Il bersaglio subisce 12d6 danni da Vuoto se fallisce il Tiro Salvezza, o la metà di questi danni se lo supera. Se il bersaglio fallisce il Tiro Salvezza, i suoi Punti Ferita massimi sono ridotti per 1 ora di un ammontare uguale al danno da Vuoto subito. Qualsiasi effetto che rimuova una malattia permette ai Punti Ferita massimi del personaggio di poter tornare al valore normale prima che trascorra quel tempo.

\textbf{Per ogni Successo Critico Magico} ottenuto nella Prova di Magia causi 6d6 di danno da vuoto aggiuntivo.

\smallskip\noindent\rule{\linewidth}{2pt} \index[Incantesimi]{Fermare il Tempo}\hypertarget{Fermare il Tempo}{}\smallskip\noindent{\textbf{Fermare il Tempo}}\pdfbookmark[3]{Fermare il Tempo}{Fermare il Tempo}
\noindent
\begin{description}[noitemsep, topsep=0pt, parsep=0pt, partopsep=0pt, leftmargin=0cm, labelwidth=2.8cm]
	\item[\textbf{Lista di Magia}]: Trasmutazione
	\item[\textbf{Livello}]: 9, Molto Raro
	\item[\textbf{T. di Lancio}]: 2 Azioni
	\item[\textbf{Gittata}]: Personale
	\item[\textbf{Componenti}]: V
	\item[\textbf{Durata}]: Istantanea
\end{description}

Fermi brevemente il flusso del tempo per tutti, tranne che per te. Il tempo non scorre per le altre creature, mentre tu effettui 1d4 + 1 round di fila, durante i quali puoi effettuare azioni e muoverti come sempre. Questo incantesimo termina se una delle azioni che usi durante questo periodo, o qualsiasi effetto che crei durante questo periodo, ha effetto su di una creatura diversa da te o su di un oggetto indossato o trasportato da qualcuno che non sia tu. Inoltre, l'incantesimo termina se ti muovi in un posto lontano più di 300 metri da quello in cui lo hai lanciato.

\textbf{Per ogni Successo Critico Magico} ottenuto nella Prova di Magia la durata aumenta di 1 round. In caso di due Successo Magico Critico puoi escludere un altra creatura dal fermarsi del tempo.

\smallskip\noindent\rule{\linewidth}{2pt} \index[Incantesimi]{Fiamma Perenne}\hypertarget{Fiamma Perenne}{}\smallskip\noindent{\textbf{Fiamma Perenne}}\pdfbookmark[3]{Fiamma Perenne}{Fiamma Perenne}
\noindent
\begin{description}[noitemsep, topsep=0pt, parsep=0pt, partopsep=0pt, leftmargin=0cm, labelwidth=2.8cm]
	\item[\textbf{Lista di Magia}]: Universale
	\item[\textbf{Livello}]: 2, Leggendario
	\item[\textbf{T. di Lancio}]: 2 Azioni
	\item[\textbf{Gittata}]: Contatto
	\item[\textbf{Componenti}]: V, S, M (polvere di rubino del valore di 150 mo, che l'incantesimo consuma)
	\item[\textbf{Durata}]: 1 giorno
\end{description}

Una luminosità simile a quella prodotta da una torcia si sprigiona da un oggetto con cui sei in contatto. L'effetto sembra quello di una normale fiamma, ma non produce calore né necessita ossigeno. Una fiamma perenne può essere celata o nascosta ma non può essere smorzata né spenta.

\smallskip\noindent\rule{\linewidth}{2pt} \index[Incantesimi]{Fiamma Sacra}\hypertarget{Fiamma Sacra}{}\smallskip\noindent{\textbf{Fiamma Sacra}}\pdfbookmark[3]{Fiamma Sacra}{Fiamma Sacra}
\noindent
\begin{description}[noitemsep, topsep=0pt, parsep=0pt, partopsep=0pt, leftmargin=0cm, labelwidth=2.8cm]
	\item[\textbf{Lista di Magia}]: Universale
	\item[\textbf{Livello}]: 0, Comune
	\item[\textbf{T. di Lancio}]: 1 Azione
	\item[\textbf{Gittata}]: 18 metri
	\item[\textbf{Componenti}]: V, S
	\item[\textbf{Durata}]: Istantanea
\end{description}

Una luminosità simile a quella prodotta da una fiaccola discende su di una creatura a gittata che puoi vedere. Il bersaglio deve superare un Tiro Salvezza su Riflessi o subire 1d8 danni da Luce. Il bersaglio non riceve il beneficio della copertura per questo Tiro Salvezza.

Puoi aumentare il danno dell'incantesimo di 1d8 quando la somma dei Tratti in comune con il Patrono raggiunge 5, 11 e 17, ma costa 2 Azioni lanciarlo potenziato e 1 Punti Magia.

\textbf{Per ogni Successo Critico Magico ottenuto} nella Prova di Magia discende una fiamma in più che deve colpire un bersaglio diverso entro gittata.

\smallskip\noindent\rule{\linewidth}{2pt} \index[Incantesimi]{Fiotto Acido}\hypertarget{Fiotto Acido}{}\smallskip\noindent{\textbf{Fiotto Acido}}\pdfbookmark[3]{Fiotto Acido}{Fiotto Acido}\label{Acid Splash}
\noindent
\begin{description}[noitemsep, topsep=0pt, parsep=0pt, partopsep=0pt, leftmargin=0cm, labelwidth=2.8cm]
	\item[\textbf{Lista di Magia}]: Evocazione
	\item[\textbf{Livello}]: 0, Comune
	\item[\textbf{T. di Lancio}]: 1 Azione
	\item[\textbf{Gittata}]: 18 metri
	\item[\textbf{Componenti}]: V, S
	\item[\textbf{Durata}]: Istantanea
\end{description}

Scagli una bolla di acido. Scegli una creatura a gittata o due creature a gittata che siano entro 1 metro l'una dall'altra. Il bersaglio deve superare un Tiro Salvezza su Riflessi o subire 1d6 danni da acido.

Puoi aumentare il danno dell'incantesimo di 1d8 quando raggiungi CM 5, CM 11 e CM 17, ma costa 2 Azioni lanciarlo potenziato e 1 Punti Magia, è altresì necessario avere preso Adepto della Magia un numero di volte pari ai potenziamenti che si vogliono applicare.

\textbf{Per ogni Successo Critico Magico ottenuto} nella Prova di Magia scagli una bolla di acido in più entro gittata.

\smallskip\noindent\rule{\linewidth}{2pt} \index[Incantesimi]{Folata di Vento}\hypertarget{Folata di Vento}{}\smallskip\noindent{\textbf{Folata di Vento}}\pdfbookmark[3]{Folata di Vento}{Folata di Vento}
\noindent
\begin{description}[noitemsep, topsep=0pt, parsep=0pt, partopsep=0pt, leftmargin=0cm, labelwidth=2.8cm]
	\item[\textbf{Lista di Magia}]: Aria
	\item[\textbf{Livello}]: 2, Comune
	\item[\textbf{T. di Lancio}]: 2 Azioni
	\item[\textbf{Gittata}]: Personale (linea di 18 metri)
	\item[\textbf{Componenti}]: V, S, M (un seme di legume)
	\item[\textbf{Durata}]: Concentrazione, massimo 1 minuto
\end{description}

Una linea di forte vento lunga 18 metri e larga 3 metri esplode partendo da te in una direzione a tua scelta, per la durata dell'incantesimo. Ogni creatura che inizia il suo round dentro la linea deve superare un Tiro Salvezza su Tempra o venire spinta lontano da te di 3 metri, seguendo la direzione della linea.

Qualsiasi creatura sulla linea deve spendere il doppio del movimento per avvicinarsi a te.

La folata disperde gas o vapori, estingue candele, torce e simili fiamme non protette nell'area. Le fiamme protette, come quelle della lanterne, si agitano, e hanno una probabilità del 50\% di estinguersi. Come 1 Azione durante ciascun tuo round, prima del termine dell'incantesimo, puoi cambiare la direzione in cui la linea si proietta da te.

Un arma da lancio che attraversa una folata di vento ha il 50\% di mancare il bersaglio.

\smallskip\noindent\rule{\linewidth}{2pt} \index[Incantesimi]{Forma Eterea}\hypertarget{Forma Eterea}{}\smallskip\noindent{\textbf{Forma Eterea}}\pdfbookmark[3]{Forma Eterea}{Forma Eterea}
\noindent
\begin{description}[noitemsep, topsep=0pt, parsep=0pt, partopsep=0pt, leftmargin=0cm, labelwidth=2.8cm]
	\item[\textbf{Lista di Magia}]: Trasmutazione
	\item[\textbf{Livello}]: 7, Raro
	\item[\textbf{T. di Lancio}]: 2 Azioni
	\item[\textbf{Gittata}]: Personale
	\item[\textbf{Componenti}]: V, S
	\item[\textbf{Durata}]: Massimo 8 ore
\end{description}

Entri nelle regioni di confine del Piano Etereo, nell'area che si sovrappone al tuo piano attuale. Resti sul Confine Etereo per la durata o finché non usi un'Azione per interrompere l'incantesimo. Se ti muovi verso l'alto o il basso, il costo del movimento è raddoppiato, se ti muovi invece orizzontalmente il movimento è raddoppiato per Azione di Movimento. Puoi vedere e udire il piano da cui provieni, ma tutto quello che si trova lì ti appare grigio, e non puoi vedere a più di 18 metri di distanza.

Mentre sei sul Piano Etereo, può interagire solo con altre creature su quel piano. Le creature che non sono sul Piano Etereo non ti possono percepire né interagire con te, a meno che una capacità speciale o la magia gli fornisca la possibilità di farlo.

Ignori tutti gli oggetti e gli effetti che non sono sul Piano etereo, potendo così attraversare gli oggetti che percepisci sul piano da cui provieni. Quando l'incantesimo termina, ritorni immediatamente al piano da cui provieni nel punto che occupi attualmente. Se quando accade occupi lo stesso spazio di un oggetto solido o di una creatura, vieni immediatamente spostato nel più vicino spazio non occupato che puoi occupare e subisci 6 danni da forza per ogni metro di cui vieni spostato (o sua frazione). Questo incantesimo non ha effetto se lo esegui mentre sei già nel Piano Etereo o su di un piano che non vi confina, come uno dei Piani Esterni.

\textbf{Per ogni Successo Critico Magico} ottenuto nella Prova di Magia puoi portare con te un altra creatura.

\smallskip\noindent\rule{\linewidth}{2pt} \index[Incantesimi]{Forma Gassosa}\hypertarget{Forma Gassosa}{}\smallskip\noindent{\textbf{Forma Gassosa}}\pdfbookmark[3]{Forma Gassosa}{Forma Gassosa}
\noindent
\begin{description}[noitemsep, topsep=0pt, parsep=0pt, partopsep=0pt, leftmargin=0cm, labelwidth=2.8cm]
	\item[\textbf{Lista di Magia}]: Trasmutazione
	\item[\textbf{Livello}]: 3, Non Comune
	\item[\textbf{T. di Lancio}]: 2 Azioni
	\item[\textbf{Gittata}]: Contatto
	\item[\textbf{Componenti}]: V, S, M (un pezzo di garza e un filo di fumo)
	\item[\textbf{Durata}]: Concentrazione, massimo 1 ora
\end{description}

Trasformi una creatura consenziente insieme a tutto ciò che sta indossando e trasportando, in una nube vaporosa per la durata. L'incantesimo termina se la creatura scende a 0 Punti Ferita. Le creature incorporee ignorano questo effetto. Mentre è in questa forma, l'unico metodo di movimento del bersaglio è una velocità di volo 3 metri. Il bersaglio può entrare e occupare lo spazio di un'altra creatura. Il bersaglio ha resistenza ai danni non magici, e ha +1d6 ai Tiri Salvezza su Tempra e Riflessi. Il bersaglio può attraversare piccoli buchi, strettoie, e anche semplici fori, sebbene consideri i liquidi come superfici solide. Il bersaglio non può cadere e resta fluttuante nell'aria anche se stordito o altrimenti reso inabile.

Mentre è nella forma di una nube vaporosa, il bersaglio non può parlare né manipolare oggetti, e qualsiasi oggetto stesse indossando o trasportando non può essere gettato, usato o altrimenti impiegato. Il bersaglio non può attaccare né lanciare incantesimi.

\textbf{Per ogni due Successo Critico Magico ottenuto} nella Prova di Magia puoi influenzare una altra creatura.

\smallskip\noindent\rule{\linewidth}{2pt} \index[Incantesimi]{Frantumare}\hypertarget{Frantumare}{}\smallskip\noindent{\textbf{Frantumare}}\pdfbookmark[3]{Frantumare}{Frantumare}
\noindent
\begin{description}[noitemsep, topsep=0pt, parsep=0pt, partopsep=0pt, leftmargin=0cm, labelwidth=2.8cm]
	\item[\textbf{Lista di Magia}]: Invocazione
	\item[\textbf{Livello}]: 2, Comune
	\item[\textbf{T. di Lancio}]: 2 Azioni
	\item[\textbf{Gittata}]: 18 metri
	\item[\textbf{Componenti}]: V, S, M (un frammento di metallo)
	\item[\textbf{Durata}]: Istantanea
\end{description}

Un forte rombo, molto intenso, erutta da un punto a gittata di tua scelta. Ogni creatura in una sfera di 3 metri di raggio centrata su quel punto deve effettuare un Tiro Salvezza su Tempra. Una creatura subisce 3d8 danni da suono se fallisce il Tiro Salvezza, o la metà di questi danni se lo supera. Una creatura composta di materiale inorganico, come pietra, cristallo o metallo, ha -1d6 sul Tiro Salvezza. Un oggetto non magico che non è indossato né trasportato subisce anch'esso danni se si trova nell'area dell'incantesimo.

\textbf{Per ogni Successo Critico Magico} ottenuto nella Prova di Magia il danno aumenta di 2d6.

\textbf{Tiro Salvezza Successo/Fallimento Critico}: In caso di Fallimento Critico il danno raddoppia, in caso di Successo Critico il danno viene ulteriormente dimezzato

\smallskip\noindent\rule{\linewidth}{2pt} \index[Incantesimi]{Freccia Acida di Restser}\hypertarget{Freccia Acida di Restser}{}\smallskip\noindent{\textbf{Freccia Acida di Restser}}\pdfbookmark[3]{Freccia Acida di Restser}{Freccia Acida di Restser}
\noindent
\begin{description}[noitemsep, topsep=0pt, parsep=0pt, partopsep=0pt, leftmargin=0cm, labelwidth=2.8cm]
	\item[\textbf{Lista di Magia}]: Acqua, Terra
	\item[\textbf{Livello}]: 2, Comune
	\item[\textbf{T. di Lancio}]: 2 Azioni
	\item[\textbf{Gittata}]: 27 metri
	\item[\textbf{Componenti}]: V, S, M (una foglia di rabarbaro in polvere e uno stomaco di pitone)
	\item[\textbf{Durata}]: Istantanea
\end{description}

Una freccia verde luminosa saetta verso un bersaglio a gittata ed esplode con uno spruzzo d'acido. Effettua un attacco a distanza con incantesimo contro il bersaglio. Se colpisci, il bersaglio subisce immediatamente 4d4 danni da acido e 2d4 danni da acido al termine del suo prossimo round. Se manchi, la freccia spruzza il bersaglio di acido infliggendo la metà dei danni iniziali e non arrecando danni al termine del prossimo round del bersaglio.

\textbf{Per ogni Successo Critico Magico} ottenuto nella Prova di Magia il danno aumenta di 2d4.

\smallskip\noindent\rule{\linewidth}{2pt} \index[Incantesimi]{Fulmine}\hypertarget{Fulmine}{}\smallskip\noindent{\textbf{Fulmine}}\pdfbookmark[3]{Fulmine}{Fulmine}
\noindent
\begin{description}[noitemsep, topsep=0pt, parsep=0pt, partopsep=0pt, leftmargin=0cm, labelwidth=2.8cm]
	\item[\textbf{Lista di Magia}]: Aria
	\item[\textbf{Livello}]: 3, Comune
	\item[\textbf{T. di Lancio}]: 2 Azioni
	\item[\textbf{Gittata}]: Personale (linea di 30 metri)
	\item[\textbf{Componenti}]: V, S, M (un pezzo di pelliccia e una verga d'ambra, cristallo o vetro)
	\item[\textbf{Durata}]: Istantanea
\end{description}

Esplodi un fulmine che forma una linea lunga 30 metri e larga 1 metro che parte da dove ti trovi in una direzione scelta da te. Ogni creatura sulla linea deve superare un Tiro Salvezza su Riflessi. La creatura subisce 8d6 danni da elettricità se fallisce il Tiro Salvezza, o la metà di questi danni se lo supera.
Il fulmine incendia gli oggetti infiammabili nell'area che non sono indossati o trasportati.

Il fulmine se lanciato contro della pietra dura lavorata rimbalza con un angolo di uscita uguale a quello di entrata (\textbackslash|/) (180-angolo di entrata). Un fulmine lanciato in acqua crea una sfera di 3 metri di raggio di elettricità nel punto in cui entra.

\textbf{Per ogni Successo Critico Magico} ottenuto nella Prova di Magia il danno aumenta di 4d6.

\textbf{Tiro Salvezza Successo/Fallimento Critico}: In caso di Fallimento Critico il danno raddoppia, in caso di Successo Critico il danno viene ulteriormente dimezzato

\smallskip\noindent\rule{\linewidth}{2pt} \index[Incantesimi]{Fulmine a catena}\hypertarget{Fulmine a catena}{}\smallskip\noindent{\textbf{Fulmine a catena}}\pdfbookmark[3]{Fulmine a catena}{Fulmine a catena}
\noindent
\begin{description}[noitemsep, topsep=0pt, parsep=0pt, partopsep=0pt, leftmargin=0cm, labelwidth=2.8cm]
	\item[\textbf{Lista di Magia}]: Aria
	\item[\textbf{Livello}]: 6, Raro
	\item[\textbf{T. di Lancio}]: 2 Azioni
	\item[\textbf{Gittata}]: 45 metri
	\item[\textbf{Componenti}]: V, S, M (un pò di pelliccia; un pezzo d'ambra, vetro o una verga di cristallo; tre spille d'argento)
	\item[\textbf{Durata}]: Istantanea
\end{description}

Crei una saetta di elettricità che colpisce un bersaglio a gittata che puoi vedere scelto da te. Da questo si genera una ulteriore saetta che colpisce il più vicino bersaglio entro 6 metri. Il processo continua finché non sono state colpite 7 bersagli o non c'è più nessun nuovo avversario a distanza. Un bersaglio può essere una creatura o oggetto almeno di taglia piccola e può essere bersaglio di una sola saetta. Un bersaglio deve effettuare un Tiro Salvezza su Riflessi oppure subisce 8d6 danni da elettricità o la metà di questi danni se lo supera.

\textbf{Per ogni Successo Critico Magico} ottenuto nella Prova di Magia la saetta si protende su tre ulteriori bersagli.

\textbf{Tiro Salvezza Successo/Fallimento Critico}: In caso di Fallimento Critico il danno raddoppia, in caso di Successo Critico il danno viene ulteriormente dimezzato

\smallskip\noindent\rule{\linewidth}{2pt} \index[Incantesimi]{Fuorviare}\hypertarget{Fuorviare}{}\smallskip\noindent{\textbf{Fuorviare}}\pdfbookmark[3]{Fuorviare}{Fuorviare}
\noindent
\begin{description}[noitemsep, topsep=0pt, parsep=0pt, partopsep=0pt, leftmargin=0cm, labelwidth=2.8cm]
	\item[\textbf{Lista di Magia}]: Illusione
	\item[\textbf{Livello}]: 5, Non Comune
	\item[\textbf{T. di Lancio}]: 2 Azioni
	\item[\textbf{Gittata}]: Personale
	\item[\textbf{Componenti}]: S
	\item[\textbf{Durata}]: 1 ora
\end{description}

Diventi invisibile nello stesso momento in cui un tuo doppione illusorio compare nel posto in cui ti trovi. Il doppione resta per la durata dell'incantesimo, ma l'invisibilità termina se attacchi o lanci un incantesimo. Puoi usare 2 Azioni per far muovere il doppione illusorio fino al doppio della tua velocità e fargli compiere un gesto, parlare e comportarsi in qualsiasi maniera tu voglia.

Puoi vedere attraverso i suoi occhi e udire tramite le sue orecchie come se fossi nello spazio in cui si trova lui. Durante ciascun tuo round, con un'Azione, puoi passare dall'usare i suoi sensi all'usare i tuoi, o viceversa. Mentre stai usando i suoi sensi, sei accecato e assordato riguardo i tuoi dintorni.

\smallskip\noindent\rule{\linewidth}{2pt} \index[Incantesimi]{Gabbia di Forza}\hypertarget{Gabbia di Forza}{}\smallskip\noindent{\textbf{Gabbia di Forza}}\pdfbookmark[3]{Gabbia di Forza}{Gabbia di Forza}
\noindent
\begin{description}[noitemsep, topsep=0pt, parsep=0pt, partopsep=0pt, leftmargin=0cm, labelwidth=2.8cm]
	\item[\textbf{Lista di Magia}]: Invocazione
	\item[\textbf{Livello}]: 6, Raro
	\item[\textbf{T. di Lancio}]: 2 Azioni
	\item[\textbf{Gittata}]: 30 metri
	\item[\textbf{Componenti}]: V, S, M (polvere di rubino del valore di 1.500 mo)
	\item[\textbf{Durata}]: 1 ora
\end{description}

Una prigione cubica, immobile e invisibile, composta di forza magica compare intorno a un'area a gittata da te scelta. La prigione può essere una gabbia o una scatola solida, a tua scelta. Una prigione nella forma di una gabbia può avere 6 metri di lato ed essere composta da sbarre di 1,5 centimetri separate di 1,5 centimetri tra di loro e fornisce una copertura completa alle creature all'interno. Una prigione a forma di scatola può avere 3 metri di lato, creando una barriera solida che impedisce a qualsiasi materia di attraversarla e bloccando qualsiasi incantesimo lanciato dall'interno o l'esterno dell'area. Quando lanci questo incantesimo, qualsiasi creatura che è completamente all'interno della gabbia deve effettuare un Tiro Salvezza su Riflessi o rimanere intrappolata. Le creature solo parzialmente nell'area della gabbia, o quelle troppo grosse per entrarvi, vengono spinte via dal centro dell'area finché non ne sono completamente fuori.

Una creatura all'interno della gabbia non può lasciarla tramite mezzi non magici. Se la creatura prova a usare il teletrasporto o il viaggio interplanare per lasciare la gabbia, deve prima effettuare un Tiro Salvezza su Volontà. Se lo supera, la creatura può usare quella magia per sfuggire alla gabbia. Se lo fallisce, la creatura non può uscire dalla gabbia e spreca l'uso dell'incantesimo o dell'effetto. La gabbia si estende anche sul Piano Etereo, bloccando così il viaggio etereo.

Questo incantesimo non può essere dissolto da \hyperlink{dissolvimagie}{Dissolvi Magie} ma solo con \hyperlink{dissolvimagieavanzato}{Dissolvi Magie Avanzato}.

\smallskip\noindent\rule{\linewidth}{2pt} \index[Incantesimi]{Giara Magica}\hypertarget{Giara Magica}{}\smallskip\noindent{\textbf{Giara Magica}}\pdfbookmark[3]{Giara Magica}{Giara Magica}
\noindent
\begin{description}[noitemsep, topsep=0pt, parsep=0pt, partopsep=0pt, leftmargin=0cm, labelwidth=2.8cm]
	\item[\textbf{Lista di Magia}]: Necromanzia
	\item[\textbf{Livello}]: 6, Molto Raro
	\item[\textbf{T. di Lancio}]: 1 minuto
	\item[\textbf{Gittata}]: Personale
	\item[\textbf{Componenti}]: V, S, M (una gemma, cristallo, reliquario o qualche altro contenitore ornamentale del valore di almeno 500 mo)
	\item[\textbf{Durata}]: Finché a che dissolto
\end{description}

Il tuo corpo entra in uno stato catatonico mentre la tua anima lo abbandona ed entra nel contenitore da te usato come componente materiale. Mentre la tua anima occupa il contenitore, sei consapevole dei tuoi dintorni come se fossi nello spazio del contenitore. Non puoi muoverti né usare reazioni. L'unica Azione che puoi effettuare è quella di proiettare la tua anima fino a 30 metri di distanza, fuori dal contenitore, ritornando al tuo corpo vivente (e terminando l'incantesimo) o cercando di possedere un corpo umanoide.

Puoi tentare di possedere qualsiasi umanoide entro 30 metri da te e che tu possa vedere (le creature protette da cerchio magico non possono essere possedute). Il bersaglio deve effettuare un Tiro Salvezza su Volontà e, se lo fallisce, la tua anima entra nel corpo del bersaglio, mentre l'anima del bersaglio resta intrappolata nel contenitore. Se lo supera, il bersaglio resiste ai tuoi tentativi di possederlo, e non puoi tentare di possederlo nuovamente prima che siano trascorse 24 ore.

Una volta che possiedi il corpo di una creatura, lo puoi controllare. Le tue statistiche di gioco sono rimpiazzate dalle statistiche della creatura, a eccezione dei tuoi Tratti e dei tuoi punteggi di Intelligenza, Saggezza e Carisma. Mantieni i benefici forniti dalle Abilità. Se il bersaglio possiede delle Abilità non puoi usarne nessuna.

Nel frattempo, l'anima della creatura posseduta può percepire i dintorni del contenitore usando i propri sensi, ma non può muoversi né effettuare alcuna azione.

Mentre possiedi un corpo, puoi usare 2 Azioni per ritornare dal corpo ospite al contenitore, se ti trovi entro 30 metri da esso, riportando l'anima della creatura ospite nel suo corpo. Se il corpo ospite muore mentre sei al suo interno, la creatura muore, e tu devi effettuare un Tiro Salvezza su Volontà contro la tua DC dei Tiri Salvezza degli incantesimi. Se lo superi, ritorni al contenitore, se si trova entro 30 metri da te. Altrimenti, morirai.

Se il contenitore viene distrutto o l'incantesimo termina, la tua anima ritorna immediatamente al tuo corpo. Se il tuo corpo è più di 30 metri lontano o se è morto mentre cerchi di farvi ritorno, morirà anche la tua anima. Se l'anima di un'altra creatura è nel contenitore quando viene distrutto, l'anima della creatura ritorna al suo corpo, se il corpo è vivo e si trova entro 30 metri, altrimenti, la creatura muore. Quando l'incantesimo termina, il contenitore viene distrutto.

\smallskip\noindent\rule{\linewidth}{2pt} \index[Incantesimi]{Glifo di Interdizione}\hypertarget{Glifo di Interdizione}{}\smallskip\noindent{\textbf{Glifo di Interdizione}}\pdfbookmark[3]{Glifo di Interdizione}{Glifo di Interdizione}
\noindent
\begin{description}[noitemsep, topsep=0pt, parsep=0pt, partopsep=0pt, leftmargin=0cm, labelwidth=2.8cm]
	\item[\textbf{Lista di Magia}]: Abiurazione
	\item[\textbf{Livello}]: 3, Comune
	\item[\textbf{T. di Lancio}]: 2 Azioni
	\item[\textbf{Gittata}]: Contatto
	\item[\textbf{Componenti}]: V. S, M (incenso e diamante in polvere del valore di almeno 200 mo, che l'incantesimo consuma)
	\item[\textbf{Durata}]: Fino a che dissolto o attivato
\end{description}

Quando lanci questo incantesimo, inscrivi un glifo che danneggia altre creature su di una superficie (come un tavolo o una sezione di pavimento o muro) o all'interno di un oggetto che può essere chiuso (come un libro, una pergamena o un forziere) per celare il glifo. Se scegli una superficie, il glifo può coprire un'area di superficie non maggiore di 3 metri di diametro. Se scegli un oggetto, quell'oggetto deve restare al suo posto; se l'oggetto viene spostato più di 3 metri dal punto in cui è stato lanciato l'incantesimo, il glifo è spezzato, e l'incantesimo termina senza essere stato attivato.

Il glifo è quasi invisibile e può essere trovato con una prova Consapevolezza contro la DC del Tiro Salvezza dei tuoi incantesimi. Decidi tu cosa attivi il glifo al momento del lancio dell'incantesimo.

Per i glifi inscritti su di una superficie, l'attivazione tipica comprende entrare in contatto o stare sopra il glifo, rimuovere un altro oggetto che copra il glifo, avvicinarsi a una certa distanza dal glifo, o manipolare l'oggetto su cui è inscritto il glifo. Per i glifi inscritti su di un oggetto, l'attivazione tipica comprende aprire l'oggetto, avvicinarsi a una certa distanza dall'oggetto, o vedere o leggere il glifo. Una volta che il glifo è stato attivato, l'incantesimo ha termine.

Puoi definire meglio l'attivazione così che l'incantesimo si attivi solo in determinate circostanze o secondo certe peculiarità fisiche (come l'altezza o il peso), specie di creatura (per esempio, l'interdizione potrebbe agire contro le aberrazioni o gli elfi oscuri), o specifici Tratti. Puoi anche predisporre condizioni per evitare che il glifo venga attivato, come la pronuncia di una parola d'ordine.

Quando inscrivi il glifo scegli rune esplosive o glifo incantesimo.

\medskip

- \emph{Glifo Incantesimo}. Puoi inserire un incantesimo preparato di livello 2 o inferiore nel glifo lanciandolo come parte della creazione del glifo. L'incantesimo deve prendere come bersaglio una singola creatura o un'area. L'incantesimo che viene inserito non ha effetto immediato se lanciato in questo modo. Quando il glifo è attivato, l'incantesimo inserito viene lanciato. Se l'incantesimo ha un bersaglio, prende come bersaglio la creatura che ha attivato il glifo. Se l'incantesimo agisce su di un'area, l'area è incentrata su quella creatura. Se l'incantesimo evoca creature ostili o crea oggetti o trappole nocive, questi appaiono quanto più vicino possibile all'intruso e lo attaccano. Se l'incantesimo richiede concentrazione, questa è mantenuta fino al termine della sua normale durata.

- \emph{Rune Esplosive}. Quando attivato, il glifo erutta energia magica in una sfera di raggio 6 metri centrata sul glifo. La sfera si propaga intorno agli angoli. Ogni creatura nell'area deve effettuare un Tiro Salvezza su Riflessi. Una creatura subisce 5d8 danni da acido, fulmine, fuoco, freddo o suono se fallisce il Tiro Salvezza (a tua scelta quando crei il glifo), o la metà di questi danni se supera il Tiro Salvezza.

Non è possibile avere contemporaneamente più di CM/4 Glifi attivi contemporaneamente.

\textbf{Per ogni Successo Critico Magico} ottenuto nella Prova di Magia il danno del glifo rune esplosive aumenta di 1d8.

\textbf{Per ogni due Successo Critico Magico} ottenuto nella Prova di Magia è possibile inserire un incantesimo di livello superiore nel Glifo Incantesimo.

\smallskip\noindent\rule{\linewidth}{2pt} \index[Incantesimi]{Globo di Invulnerabilità}\hypertarget{Globo di Invulnerabilità}{}\smallskip\noindent{\textbf{Globo di Invulnerabilità}}\pdfbookmark[3]{Globo di Invulnerabilita'}{Globo di Invulnerabilita'}
\noindent
\begin{description}[noitemsep, topsep=0pt, parsep=0pt, partopsep=0pt, leftmargin=0cm, labelwidth=2.8cm]
	\item[\textbf{Lista di Magia}]: Abiurazione
	\item[\textbf{Livello}]: 6, Comune
	\item[\textbf{T. di Lancio}]: 2 Azioni
	\item[\textbf{Gittata}]: Personale (raggio di 3 metri)
	\item[\textbf{Componenti}]: V. S, M (una pallina di vetro o di cristallo che si frantuma quando l'incantesimo termina)
	\item[\textbf{Durata}]: Concentrazione, massimo 1 minuto
\end{description}

Una barriera immobile e lievemente scintillante si erge in un raggio di 3 metri intorno a te e vi rimane per la durata.

Qualsiasi incantesimo di Livello 4 (ad esclusione di risultati superiori grazie a critici magici) o più basso lanciato dall'esterno della barriera non può agire sulle creature o gli oggetti al suo interno. Questi incantesimi vengono soppressi se prendono come bersaglio creature e oggetti all'interno della barriera o coinvolgono l'area su cui è la barriera.

\textbf{Per ogni due Successo Critico Magico ottenuto} nella Prova di Magia puoi bloccare un livello superiore di incantesimo.

\smallskip\noindent\rule{\linewidth}{2pt} \index[Incantesimi]{Goffaggine}\hypertarget{Goffaggine}{}\smallskip\noindent{\textbf{Goffaggine}}\pdfbookmark[3]{Goffaggine}{Goffaggine}
\noindent
\begin{description}[noitemsep, topsep=0pt, parsep=0pt, partopsep=0pt, leftmargin=0cm, labelwidth=2.8cm]
	\item[\textbf{Lista di Magia}]: Trasmutazione
	\item[\textbf{Livello}]: 2, Raro
	\item[\textbf{T. di Lancio}]: 1 Azione
	\item[\textbf{Gittata}]: 9 metri
	\item[\textbf{Componenti}]: V, S
	\item[\textbf{Durata}]: 1 round per CM, Concentrazione
\end{description}

Il bersaglio effettua un Tiro Salvezza su Volontà con bonus Carisma, se fallisce ogni qual volta che effettua una Prova di Competenza, Tiro Salvezza o Tiro per Colpire conta sempre un 1 tirato in più per verificare i fallimenti critici.

\smallskip\noindent\rule{\linewidth}{2pt} \index[Incantesimi]{Gragnola di Ghiande di Kyrin}\hypertarget{Gragnola di Ghiande di Kyrin}{}\smallskip\noindent{\textbf{Gragnola di Ghiande di Kyrin}}\pdfbookmark[3]{Gragnola di Ghiande di Kyrin}{Gragnola di Ghiande di Kyrin}
\noindent
\begin{description}[noitemsep, topsep=0pt, parsep=0pt, partopsep=0pt, leftmargin=0cm, labelwidth=2.8cm]
	\item[\textbf{Lista di Magia}]: Animali e Piante
	\item[\textbf{Livello}]: 2, Non Comune
	\item[\textbf{T. di Lancio}]: 1 Azione
	\item[\textbf{Gittata}]: 50 metri
	\item[\textbf{Componenti}]: V, S, M (9 ghiande che vengono consumate, un pezzo di caucciù)
	\item[\textbf{Durata}]: 1 minuto per CM, Concentrazione
\end{description}

Incanti 9 ghiande di energia magica e queste incominciano a vorticare 30 centimetri sopra la tua spalla.
Ogni round, spendendo 1 Azione, puoi lanciare fino a 5 ghiande contro uno o più bersagli.
Esegui un solo Tiro per Colpire con incantesimi da distanza e confronta il risultato con la Difesa di ogni bersaglio indipendentemente dal numero di ghiande che gli tiri. Ogni ghianda se colpisce fa 1d4 di danni contundenti.

\textbf{Per ogni Successo Critico Magico} ottenuto nella Prova di Magia puoi incantare due ghiande in più.

\smallskip\noindent\rule{\linewidth}{2pt} \index[Incantesimi]{Gragnola di Ghiande Infuocate di Kyrin}\hypertarget{Gragnola di Ghiande Infuocate di Kyrin}{}\smallskip\noindent{\textbf{Gragnola di Ghiande Infuocate di Kyrin}}\pdfbookmark[3]{Gragnola di Ghiande Infuocate di Kyrin}{Gragnola di Ghiande Infuocate di Kyrin}
\noindent
\begin{description}[noitemsep, topsep=0pt, parsep=0pt, partopsep=0pt, leftmargin=0cm, labelwidth=2.8cm]
	\item[\textbf{Lista di Magia}]: Animali e Piante, Fuoco
	\item[\textbf{Livello}]: 3, Raro
	\item[\textbf{T. di Lancio}]: 2 Azione
	\item[\textbf{Gittata}]: 50 metri
	\item[\textbf{Componenti}]: V, S, M (9 ghiande che vengono consumate, un pezzo di caucciù)
	\item[\textbf{Durata}]: 1 minuto per CM, Concentrazione
\end{description}

Incanti 9 ghiande di energia magica e queste incominciano a vorticare 30 centimetri sopra la tua spalla.
Ogni round, spendendo 1 Azione, puoi lanciare fino a 5 ghiande contro uno o più bersagli.
Esegui un solo Tiro per Colpire con incantesimi da distanza e confronta il risultato con la Difesa di ogni bersaglio indipendentemente dal numero di ghiande che gli tiri. Ogni ghianda se colpisce fa 1d4 di danni contundenti + 1d4 di danno da fuoco.

\textbf{Per ogni Successo Critico Magico} ottenuto nella Prova di Magia puoi incantare due ghiande in più.

\smallskip\noindent\rule{\linewidth}{2pt} \index[Incantesimi]{Gragnola di Limoni di Kyrin}\hypertarget{Gragnola di Limoni di Kyrin}{}\smallskip\noindent{\textbf{Gragnola di Limoni di Kyrin}}\pdfbookmark[3]{Gragnola di Limoni di Kyrin}{Gragnola di Limoni di Kyrin}
\noindent
\begin{description}[noitemsep, topsep=0pt, parsep=0pt, partopsep=0pt, leftmargin=0cm, labelwidth=2.8cm]
	\item[\textbf{Lista di Magia}]: Animali e Piante, Terra
	\item[\textbf{Livello}]: 3, Raro
	\item[\textbf{T. di Lancio}]: 2 Azioni
	\item[\textbf{Gittata}]: 30 metri
	\item[\textbf{Componenti}]: V, S, M (almeno 9 gocce di limone, una boccetta)
	\item[\textbf{Durata}]: 1 round per CM, Concentrazione
\end{description}

Incanti una boccetta con dentro almeno 9 gocce di limone.
Ogni round, spendendo 1 Azione, puoi spruzzare fino a 2 gocce di limone, delle 9 totali, contro uno o più bersagli entro 30 metri.
Esegui un solo Tiro per Colpire con incantesimi da distanza e confronta il risultato con la Difesa di ogni bersaglio indipendentemente dal numero di ghiande che gli tiri. Ogni goccia se colpisce fa 1d6+1 danni da acido.

\textbf{Per ogni Successo Critico Magico} ottenuto nella Prova di Magia puoi creare due gocce di limone in più.

\smallskip\noindent\rule{\linewidth}{2pt} \index[Incantesimi]{Gragnola di Marroni di Kyrin}\hypertarget{Gragnola di Marroni di Kyrin}{}\smallskip\noindent{\textbf{Gragnola di Marroni di Kyrin}}\pdfbookmark[3]{Gragnola di Marroni di Kyrin}{Gragnola di Marroni di Kyrin}
\noindent
\begin{description}[noitemsep, topsep=0pt, parsep=0pt, partopsep=0pt, leftmargin=0cm, labelwidth=2.8cm]
	\item[\textbf{Lista di Magia}]: Animali e Piante
	\item[\textbf{Livello}]: 5, Molto Raro
	\item[\textbf{T. di Lancio}]: 1 Azione
	\item[\textbf{Gittata}]: 60 metri
	\item[\textbf{Componenti}]: V, S, M (9 marroni che vengono consumati, un pezzo di caucciù)
	\item[\textbf{Durata}]: 1 minuto per CM, Concentrazione
\end{description}

Incanti 9 marroni di energia magica e queste incominciano a vorticare 60 centimetri sopra la tua spalla.
Ogni round, spendendo 1 Azioni, puoi lanciare fino a 5 marroni contro uno o più bersagli.
Esegui un solo Tiro per Colpire con incantesimi da distanza e confronta il risultato con la Difesa di ogni bersaglio indipendentemente dal numero di ghiande che gli tiri. Ogni ghianda se colpisce fa 2d8+4 di danni contundenti

\textbf{Per ogni Successo Critico Magico} ottenuto nella Prova di Magia puoi incantare due marroni in più.

\smallskip\noindent\rule{\linewidth}{2pt} \index[Incantesimi]{Grido di dolore}\hypertarget{Grido di dolore}{}\smallskip\noindent{\textbf{Grido di dolore}}\pdfbookmark[3]{Grido di dolore}{Grido di dolore}
\noindent
\begin{description}[noitemsep, topsep=0pt, parsep=0pt, partopsep=0pt, leftmargin=0cm, labelwidth=2.8cm]
	\item[\textbf{Lista di Magia}]: Necromanzia
	\item[\textbf{Livello}]: 1, Raro
	\item[\textbf{T. di Lancio}]: 1 Reazione
	\item[\textbf{Gittata}]: personale
	\item[\textbf{Componenti}]: V
	\item[\textbf{Durata}]: Istantanea
\end{description}

Come Azione di Reazione emetti un grido di dolore quando colpito in mischia. La creatura che ti ha colpito deve effettuare un Tiro Salvezza su Tempra o subito 2d4 di danno da Vuoto.

\textbf{Per ogni Successo Critico Magico} ottenuto nella Prova di Magia causi 1d6 di danno in più.

\smallskip\noindent\rule{\linewidth}{2pt} \index[Incantesimi]{Guarigione}\hypertarget{Guarigione}{}\smallskip\noindent{\textbf{Guarigione}}\pdfbookmark[3]{Guarigione}{Guarigione}
\noindent
\begin{description}[noitemsep, topsep=0pt, parsep=0pt, partopsep=0pt, leftmargin=0cm, labelwidth=2.8cm]
	\item[\textbf{Lista di Magia}]: Cura
	\item[\textbf{Livello}]: 6, Raro
	\item[\textbf{T. di Lancio}]: 2 Azioni
	\item[\textbf{Gittata}]: 18 metri
	\item[\textbf{Componenti}]: V, S
	\item[\textbf{Durata}]: Istantanea
\end{description}

Scegli una creatura a gittata e che puoi vedere. un'ondata di energia positiva curativa travolge la creatura, facendole recuperare 70 Punti Ferita. L'incantesimo prova anche a \hyperlink{contrastareincantesimi}{contrastare} a qualsiasi cecità, sordità e malattia (anche magica) che affligga il bersaglio. Questo incantesimo causa 50 Punti Ferita di danno ad un non morto con un Tiro per Colpire con incantesimo a tocco.

\textbf{Per ogni Successo Critico Magico} ottenuto nella Prova di Magia l'ammontare guarito aumenta di 20.

\textbf{NOTA}: Se incantatore e creatura curata sono entrambi \textbf{Seguaci} dello stesso Patrono l'incantesimo cura 90 Punti Ferita.

\textbf{NOTA}: Se incantatore e creatura curata sono entrambi \textbf{Devoti} dello stesso Patrono l'incantesimo riporta a pieno di Punti Ferita.

\smallskip\noindent\rule{\linewidth}{2pt} \index[Incantesimi]{Guarigione di Massa}\hypertarget{Guarigione di Massa}{}\smallskip\noindent{\textbf{Guarigione di Massa}}\pdfbookmark[3]{Guarigione di Massa}{Guarigione di Massa}
\noindent
\begin{description}[noitemsep, topsep=0pt, parsep=0pt, partopsep=0pt, leftmargin=0cm, labelwidth=2.8cm]
	\item[\textbf{Lista di Magia}]: Cura
	\item[\textbf{Livello}]: 9, Leggendario
	\item[\textbf{T. di Lancio}]: 2 Azioni
	\item[\textbf{Gittata}]: 18 metri
	\item[\textbf{Componenti}]: V, S
	\item[\textbf{Durata}]: Istantanea
\end{description}

Un effluvio di energia guaritrice scorre da te verso le creature ferite che ti circondano. Ripristini fino a 700 Punti Ferita, divisi come preferisci tra qualsiasi creatura a gittata e che puoi vedere (con un massimo di 70 Punti Ferita a creatura). Le creature guarite da questo incantesimo sono curate anche di tutte le malattie e da qualsiasi effetto che le renda accecate o assordate. Questo incantesimo può infliggere fino a 120 Punti Ferita di danno ad un non morto. TS su Tempra per annullare l'effetto.

Se l'incantatore e creatura curata sono entrambi \textbf{Seguaci} dello stesso Patrono la cura assegnata aumenta di 20\%

Se l'incantatore e creatura curata sono entrambi \textbf{Devoti} dello stesso Patrono la cura assegnata aumenta di 50\%

\smallskip\noindent\rule{\linewidth}{2pt} \index[Incantesimi]{Guida}\hypertarget{Guida}{}\smallskip\noindent{\textbf{Guida}}\pdfbookmark[3]{Guida}{Guida}
\noindent
\begin{description}[noitemsep, topsep=0pt, parsep=0pt, partopsep=0pt, leftmargin=0cm, labelwidth=2.8cm]
	\item[\textbf{Lista di Magia}]: Divinazione
	\item[\textbf{Livello}]: 0, Comune
	\item[\textbf{T. di Lancio}]: 1 Reazione
	\item[\textbf{Gittata}]: 3 metri
	\item[\textbf{Componenti}]: V, S
	\item[\textbf{Durata}]: 1 Round
\end{description}

Lanci l'incantesimo a contatto di una creatura consenziente. Una volta, prima che l'incantesimo termini, il bersaglio può tirare un d4 e sommare il risultato tirato a una prova di competenza a sua scelta. Può tirare il dado prima o dopo aver effettuato la prova di Competenza. L'incantesimo ha poi termine. Non è possibile lanciare Guida sulla stessa creatura ad intervalli inferiori ad 1 ora.

\smallskip\noindent\rule{\linewidth}{2pt} \index[Incantesimi]{Guscio Anti-Vita}\hypertarget{Guscio Anti-Vita}{}\smallskip\noindent{\textbf{Guscio Anti-Vita}}\pdfbookmark[3]{Guscio Anti-Vita}{Guscio Anti-Vita}
\noindent
\begin{description}[noitemsep, topsep=0pt, parsep=0pt, partopsep=0pt, leftmargin=0cm, labelwidth=2.8cm]
	\item[\textbf{Lista di Magia}]: Animali e Piante
	\item[\textbf{Livello}]: 5, Non Comune
	\item[\textbf{T. di Lancio}]: 2 Azioni
	\item[\textbf{Gittata}]: Personale (raggio di 3 metri)
	\item[\textbf{Componenti}]: V, S
	\item[\textbf{Durata}]: massimo 1 ora
\end{description}

Una barriera luminosa si estende fino a un raggio di 3 metri intorno a te, muovendosi con te e rimanendo centrata su di te, tenendo distanti le creature che non siano non morti o costrutti. La barriera permane per la durata.

La barriera impedisce a una creatura soggetta di attraversarla in alcun modo. Una creatura soggetta può lanciare incantesimi o effettuare attacchi con armi a distanza o con portata attraverso la barriera. Se ti muovi in modo che una creatura soggetta venga forzata ad attraversare la barriera, l'incantesimo termina.

\textbf{Per ogni Successo Critico Magico} ottenuto nella Prova di Magia la durata raddoppia.

\smallskip\noindent\rule{\linewidth}{2pt} \index[Incantesimi]{Identificare}\hypertarget{Identificare}{}\smallskip\noindent{\textbf{Identificare}}\pdfbookmark[3]{Identificare}{Identificare}
\noindent
\begin{description}[noitemsep, topsep=0pt, parsep=0pt, partopsep=0pt, leftmargin=0cm, labelwidth=2.8cm]
	\item[\textbf{Lista di Magia}]: Universale
	\item[\textbf{Livello}]: 1, Comune
	\item[\textbf{T. di Lancio}]: variabile
	\item[\textbf{Gittata}]: Contatto
	\item[\textbf{Componenti}]: V, S, M (una gemma del valore di almeno 10 mo e una piuma di gufo che l'incantesimo consumano)
	\item[\textbf{Durata}]: Variabile
\end{description}

Scegli un oggetto con cui devi restare a contatto per tutto il lancio dell'incantesimo. Se si tratta di un oggetto magico o altro oggetto imbevuto di magia effettua una prova di Arcana, con un bonus di +2d6, come parte di lancio dell'incantesimo.

Se la DC che fatti è superiore a 20 ne apprendi le caratteristiche principali, con DC 25 ne apprendi le proprietà e come usarle e quante cariche abbia, se ne ha.

Apprendi se degli incantesimi stiano agendo sull'oggetto e cosa siano. Se l'oggetto è stato creato da un incantesimo, apprendi quale incantesimo lo abbia creato. Se invece durante l'esecuzione resti a contatto con una creatura, apprendi se degli incantesimi stiano agendo su di essa e quali siano.

La prova di Arcana dura 10 minuti. Con punteggio Arcana 6 dura 5 minuti, con 12 dura 1 minuto, con Arcana 18 è sufficiente 1 Round.

\textbf{Solo se ottieni un Successo Critico Magico} apprendi se l'oggetto è \hyperlink{oggettimaledettiid}{maledetto}.

\smallskip\noindent\rule{\linewidth}{2pt} \index[Incantesimi]{Illusione Minore}\hypertarget{Illusione Minore}{}\smallskip\noindent{\textbf{Illusione Minore}}\pdfbookmark[3]{Illusione Minore}{Illusione Minore}
\noindent
\begin{description}[noitemsep, topsep=0pt, parsep=0pt, partopsep=0pt, leftmargin=0cm, labelwidth=2.8cm]
	\item[\textbf{Lista di Magia}]: Universale
	\item[\textbf{Livello}]: 0, Comune
	\item[\textbf{T. di Lancio}]: 2 Azioni
	\item[\textbf{Gittata}]: 9 metri
	\item[\textbf{Componenti}]: S, M (un pezzo di vello)
	\item[\textbf{Durata}]: 1 minuto
\end{description}

Crei l'immagine di un oggetto o un suono a gittata per la durata dell'incantesimo. L'illusione ha termine se la interrompi con un'Azione o lanci di nuovo questo incantesimo.

Se crei un suono, il suo volume può variare da quello di un bisbiglio a un urlo. Può essere la tua voce, la voce di qualcun altro, il ruggito di un leone, un battito di tamburi, o qualsiasi altro suono tu scelga. Il suono continua incessante per tutta la durata, oppure puoi produrre suoni diversi in momenti diversi prima del termine dell'incantesimo.

Se crei l'immagine di un oggetto (come una sedia, un'impronta fangosa o un piccolo forziere) non può essere più grande di un cubo di 1 metro di spigolo. L'immagine non può produrre suoni, luci, odori o qualsiasi altro effetto sensoriale. L'interazione fisica con l'oggetto lo rivela come illusione, perché le cose lo possono attraversare.

Una creatura che usa 3 Azioni per esaminare il suono o l'immagine può determinare che si tratta di un'illusione con una prova riuscita di Intelligenza (Indagare) contro la DC del Tiro Salvezza del tuo incantesimo. Se una creatura riconosce l'illusione per quello che è, per lei l'illusione sbiadisce.

\smallskip\noindent\rule{\linewidth}{2pt} \index[Incantesimi]{Illusione Programmata}\hypertarget{Illusione Programmata}{}\smallskip\noindent{\textbf{Illusione Programmata}}\pdfbookmark[3]{Illusione Programmata}{Illusione Programmata}
\noindent
\begin{description}[noitemsep, topsep=0pt, parsep=0pt, partopsep=0pt, leftmargin=0cm, labelwidth=2.8cm]
	\item[\textbf{Lista di Magia}]: Illusione
	\item[\textbf{Livello}]: 6, Non Comune
	\item[\textbf{T. di Lancio}]: 2 Azioni
	\item[\textbf{Gittata}]: 36 metri
	\item[\textbf{Componenti}]: V, S, M (un pezzo di vello e polvere di giada del valore di almeno 25 mo)
	\item[\textbf{Durata}]: Fino a che dissolto
\end{description}

Crei, a gittata, l'illusione di un oggetto, creatura o qualche altro fenomeno visibile che si attiva quando viene soddisfatta una specifica condizione. Fino ad allora l'illusione è impercettibile. Non può essere più grande di un cubo di 9 metri di spigolo, e decidi tu quando lanci l'incantesimo, come si comporti l'illusione e che suoni produca. L'esibizione programmata può durare fino a 5 minuti. Quando occorrono le condizioni da te specificate, l'illusione si manifesta e si comporta nel modo da te descritto. Una volta che l'illusione ha terminato la sua esibizione, scompare e rimane dormiente per 10 minuti. Dopo questo periodo, l'illusione può essere attivata di nuovo.

La condizione di attivazione può essere generica o dettagliata quanto vuoi, sebbene debba essere basata su condizioni visibili o udibili che avvengano entro 9 metri dall'area. Per esempio, potresti creare un'illusione di te stesso che appare e avverta chi tenti di aprire una porta munita di trappola, oppure potresti predisporre l'illusione perché si attivi solo quando una creatura pronunci la parola o la frase giusta.

L'interazione fisica con l'immagine la rivela come illusione, dato che le cose le passano attraverso. Una creatura che usi 3 Azioni per esaminare l'immagine può determinare che è un'illusione con una prova riuscita di Intelligenza (Indagare) contro la DC del Tiro Salvezza dell'incantesimo. Se una creatura riconosce l'illusione per quello che è, essa può vedere attraverso l'immagine, e qualsiasi suono prodotto dall'immagine le suona artefatto.

\smallskip\noindent\rule{\linewidth}{2pt} \index[Incantesimi]{Immagine Maggiore}\hypertarget{Immagine Maggiore}{}\smallskip\noindent{\textbf{Immagine Maggiore}}\pdfbookmark[3]{Immagine Maggiore}{Immagine Maggiore}
\noindent
\begin{description}[noitemsep, topsep=0pt, parsep=0pt, partopsep=0pt, leftmargin=0cm, labelwidth=2.8cm]
	\item[\textbf{Lista di Magia}]: Illusione
	\item[\textbf{Livello}]: 3, Comune
	\item[\textbf{T. di Lancio}]: 2 Azioni
	\item[\textbf{Gittata}]: 36 metri
	\item[\textbf{Componenti}]: V, S, M (un pezzo di vello)
	\item[\textbf{Durata}]: Concentrazione, massimo 1 minuto per CM
\end{description}

Crei l'immagine di un oggetto, una creatura o qualche altro fenomeno visibile non più grande di un cubo di 6 metri di spigolo. L'immagine appare in un punto a gittata che puoi vedere e vi rimane per la durata dell'incantesimo. L'immagine sembra completamente reale, e comprende suoni, odori e la temperatura appropriata alla cosa raffigurata. Non puoi generare calore o freddo sufficiente a provocare danni, né un suono abbastanza forte da infliggere danno da suono o assordare una creatura, o un odore che possa far star male una creatura (come il fetore di un troglodita). Finché resti a gittata dell'illusione, puoi usare un'Azione per far muovere l'immagine in qualsiasi altro punto a gittata.

Quando l'immagine cambia posizione, puoi alterarne l'aspetto così che i suoi movimenti appaiano naturali. Per esempio, se crei l'immagine di una creatura e la muovi, puoi alterare l'immagine in modo che sembri camminare. Allo stesso modo, puoi impiegare l'illusione per produrre suoni diversi in momenti diversi, fino a farle portare avanti una conversazione.

L'interazione fisica con l'immagine la rivela come illusione, dato che le cose vi passano attraverso. Una creatura che usa 3 Azioni per esaminare l'immagine può determinare che si tratta di un'illusione con una prova riuscita di Intelligenza (Indagare) contro la DC del Tiro Salvezza del tuo incantesimo. Se una creatura riconosce l'illusione per quello che è, la creatura può vedervi attraverso, e per quella creatura tutte le altre qualità sensoriali svaniscono.

\textbf{Se ottieni un Successo Critico Magico} nella Prova di Magia l'incantesimo dura finché non viene dissolto, senza richiedere la tua concentrazione.

\smallskip\noindent\rule{\linewidth}{2pt} \index[Incantesimi]{Immagine Proiettata}\hypertarget{Immagine Proiettata}{}\smallskip\noindent{\textbf{Immagine Proiettata}}\pdfbookmark[3]{Immagine Proiettata}{Immagine Proiettata}
\noindent
\begin{description}[noitemsep, topsep=0pt, parsep=0pt, partopsep=0pt, leftmargin=0cm, labelwidth=2.8cm]
	\item[\textbf{Lista di Magia}]: Illusione
	\item[\textbf{Livello}]: 7, Non Comune
	\item[\textbf{T. di Lancio}]: 2 Azioni
	\item[\textbf{Gittata}]: 750 chilometri
	\item[\textbf{Componenti}]: V, S, M (una tua piccola riproduzione fatta di materiali del valore almeno di 5 mo)
	\item[\textbf{Durata}]: 1 giorno
\end{description}

Crei una copia illusoria di te stesso che permane per la durata. La copia può apparire in qualsiasi luogo entro la gittata che tu abbia già visto, ignorando qualsiasi ostacolo frapposto. L'illusione riproduce il tuo aspetto e i tuoi rumori ma è intangibile. Se l'illusione subisce danni, scompare, e l'incantesimo ha termine.

Puoi usare 2 Azioni per far muovere questa illusione fino al doppio della tua velocità e farle compiere un gesto, parlare e comportarsi in qualsiasi maniera tu voglia. Imita alla perfezione i tuoi comportamenti.

Puoi vedere attraverso i suoi occhi e udire tramite le sue orecchie come se fossi nello spazio in cui essa si trova. Durante ciascun tuo round, con un'Azione, puoi passare dall'usare i suoi sensi all'usare i tuoi, o viceversa. Mentre stai usando i suoi sensi, sei accecato e assordato riguardo i tuoi dintorni.

L'interazione fisica con l'immagine la rivela come illusione, dato che le cose le passano attraverso. Una creatura che usi 3 Azioni per esaminare l'immagine può determinare che è un'illusione con una prova riuscita di Consapevolezza contro la DC del Tiro Salvezza dell'incantesimo. Se una creatura riconosce l'illusione per quello che è, essa può vedere attraverso l'immagine, e qualsiasi suono prodotto dall'immagine le suona artefatto.

\smallskip\noindent\rule{\linewidth}{2pt} \index[Incantesimi]{Immagine Silenziosa}\hypertarget{Immagine Silenziosa}{}\smallskip\noindent{\textbf{Immagine Silenziosa}}\pdfbookmark[3]{Immagine Silenziosa}{Immagine Silenziosa}
\noindent
\begin{description}[noitemsep, topsep=0pt, parsep=0pt, partopsep=0pt, leftmargin=0cm, labelwidth=2.8cm]
	\item[\textbf{Lista di Magia}]: Illusione
	\item[\textbf{Livello}]: 1, Comune
	\item[\textbf{T. di Lancio}]: 2 Azioni
	\item[\textbf{Gittata}]: 36 metri
	\item[\textbf{Componenti}]: V, S, M (un pezzo di vello)
	\item[\textbf{Durata}]: Concentrazione, massimo 3 minuti per CM
\end{description}

Crei l'immagine di un oggetto, una creatura o qualche altro fenomeno visibile non più grande di un cubo di 3 metri di spigolo. L'immagine appare in un punto a gittata che puoi vedere e resta per la durata dell'incantesimo. L'immagine è puramente visiva; non è accompagnata da suoni, odori o altri effetti sensoriali. Puoi usare un'Azione per far muovere l'immagine in qualsiasi altro punto a gittata. Quando l'immagine cambia posizione, puoi alterarne l'aspetto così che i suoi movimenti appaiano naturali. Per esempio, se crei l'immagine di una creatura e la muovi, puoi alterare l'immagine in modo che sembri camminare.

L'interazione fisica con l'immagine la rivela come illusione, dato che le cose vi passano attraverso. Una creatura che usa 3 Azioni per esaminare l'immagine può determinare che si tratta di un'illusione con una prova di Consapevolezza contro la DC del Tiro Salvezza del tuo incantesimo. Se una creatura riconosce l'illusione per quello che è, la creatura può vedervi attraverso.

\smallskip\noindent\rule{\linewidth}{2pt} \index[Incantesimi]{Immagine Speculare}\hypertarget{Immagine Speculare}{}\smallskip\noindent{\textbf{Immagine Speculare}}\pdfbookmark[3]{Immagine Speculare}{Immagine Speculare}
\noindent
\begin{description}[noitemsep, topsep=0pt, parsep=0pt, partopsep=0pt, leftmargin=0cm, labelwidth=2.8cm]
	\item[\textbf{Lista di Magia}]: Illusione
	\item[\textbf{Livello}]: 2, Comune
	\item[\textbf{T. di Lancio}]: 2 Azioni
	\item[\textbf{Gittata}]: Personale
	\item[\textbf{Componenti}]: V, S
	\item[\textbf{Durata}]: 1 minuto
\end{description}

Nel tuo spazio compaiono 2d4 duplicati illusori di te stesso. Fino al termine dell'incantesimo, i duplicati si muovono con te e imitano le tue azioni, scambiandosi di posto in modo da rendere impossibile determinare quale sia l'immagine reale. Puoi usare 1 Azione per congedare i duplicati illusori.

Ogni volta che una creatura ti prende in realtà ha colpito una immagine illusoria.
Se una creatura fa più attacchi a round può disperdere una immagine per ogni attacco andato a buon fine. Se vieni colpito da un incantesimo ad area tutte le immagini svaniscono.

Una creatura che non può vedere, o si affida a sensi diversi dalla vista (come la vista cieca), o che può distinguere le illusioni come false (come la visione del vero), ignora gli effetti di questo incantesimo.

\textbf{Per ogni Successo Critico Magico} ottenuto nella Prova di Magia crei una immagine duplicata in più fino ad un massimo totale di 8 immagini.

\smallskip\noindent\rule{\linewidth}{2pt} \index[Incantesimi]{Imprigionare}\hypertarget{Imprigionare}{}\smallskip\noindent{\textbf{Imprigionare}}\pdfbookmark[3]{Imprigionare}{Imprigionare}
\noindent
\begin{description}[noitemsep, topsep=0pt, parsep=0pt, partopsep=0pt, leftmargin=0cm, labelwidth=2.8cm]
	\item[\textbf{Lista di Magia}]: Abiurazione
	\item[\textbf{Livello}]: 9, Raro
	\item[\textbf{T. di Lancio}]: 2 Azioni
	\item[\textbf{Gittata}]: 9 metri
	\item[\textbf{Componenti}]: V, S, M (una raffigurazione su vello o una statuetta incisa con le fattezze del bersaglio, e una componente speciale che varia a seconda della versione che scegli dell'incantesimo, del valore di almeno 500 mo per Dado Ferita del bersaglio)
	\item[\textbf{Durata}]: Fino a dissolvimento
\end{description}

Crei dei vincoli magici per bloccare una creatura a gittata e che puoi vedere. Il bersaglio deve superare un Tiro Salvezza su Volontà o essere avvinto dall'incantesimo; se lo supera, è immune all'incantesimo qualora lo lanci di nuovo. Mentre è soggetta a questo incantesimo, la creatura non ha bisogno di respirare, mangiare o bere e non invecchia. Gli incantesimi di divinazione non possono localizzare né percepire il bersaglio.

Quando lanci questo incantesimo, scegli una delle seguenti forme di prigionia.

\begin{itemize}[leftmargin=*] \setlength{\itemsep}{0pt}
	\item \emph{Incatenamento}. Catene pesanti, ben saldate al terreno, tengono il bersaglio ancorato. Il bersaglio è intralciato fino al termine dell'incantesimo, e non può muoversi né essere mosso in alcun modo fino ad allora. La componente speciale per questa versione dell'incantesimo è una catenella di metallo prezioso.
	\item \emph{Isolamento Minimo}. Il bersaglio rimpicciolisce fino a 2,5 centimetri di altezza ed è imprigionato in una gemma o simile oggetto. La luce può attraversare normalmente la gemma (permettendo al bersaglio di vedere all'esterno e ad altre creature di vedere dentro), ma null'altro può attraversarla, neppure tramite teletrasporto o viaggio planare. La gemma non può essere tagliata né infranta finché l'incantesimo rimane in atto. La componente speciale per questa versione dell'incantesimo è una grande gemma trasparente, come il corindone, il diamante o il rubino.
	\item \emph{Prigione Confinata}. L'incantesimo trasporta il bersaglio in un minuscolo semipiano interdetto al teletrasporto e al viaggio planare. Il semipiano può essere un labirinto, una gabbia, una torre, o qualsiasi altra struttura chiusa scelta da te. La componente speciale per questa versione dell'incantesimo è una rappresentazione in miniatura della prigione fatta di giada.
	\item \emph{Sepoltura}. Il bersaglio viene sepolto nelle profondità della terra in una sfera di forza magica grande a sufficienza da contenere il bersaglio. Nulla può attraversare la sfera, né alcuna creatura può teletrasportarsi o usare il viaggio planare per entrarvi o uscire. La componente speciale per questa versione dell'incantesimo è una piccola sfera di mithral.
	\item \emph{Sonno}. Il bersaglio cade addormentato e non può essere risvegliato. La componente speciale per questa versione dell'incantesimo consiste di rare erbe soporifere.
\end{itemize}

\emph{\textbf{Terminare l'incantesimo}}. Durante il lancio dell'incantesimo, in qualsiasi delle sue versioni, puoi specificare una condizione che possa porre fine all'incantesimo e liberare il bersaglio. La condizione può essere tanto specifica o elaborata quanto desideri, ma il Narratore deve concordare che la condizione sia ragionevole e possa avverarsi. Le condizioni possono essere basate sul nome, l'identità o il Patrono di una creatura, ma comunque basate su azioni o qualità percepibili e non su cose intangibili come il livello, le Abilità o i Punti Ferita.

Un incantesimo dissolvi magie può porre fine all'incantesimo solo se lanciato da un personaggio con Competenza Magica almeno 18, che prenda come bersaglio la prigione o la componente materiale usata per crearla.

Puoi usare una particolare componente speciale per creare solo una prigione alla volta. Se lanci l'incantesimo di nuovo usando la stessa componente, il bersaglio del primo lancio dell'incantesimo viene immediatamente liberato dal suo vincolo.

\smallskip\noindent\rule{\linewidth}{2pt} \index[Incantesimi]{Inaridire}\hypertarget{Inaridire}{}\smallskip\noindent{\textbf{Inaridire}}\pdfbookmark[3]{Inaridire}{Inaridire}
\noindent
\begin{description}[noitemsep, topsep=0pt, parsep=0pt, partopsep=0pt, leftmargin=0cm, labelwidth=2.8cm]
	\item[\textbf{Lista di Magia}]: Necromanzia
	\item[\textbf{Livello}]: 4, Non Comune
	\item[\textbf{T. di Lancio}]: 2 Azioni
	\item[\textbf{Gittata}]: 9 metri
	\item[\textbf{Componenti}]: V, S
	\item[\textbf{Durata}]: Istantanea
\end{description}

Energia necromantica avvolge una creatura di tua scelta a gittata e che puoi vedere, deprivandola di linfa e vitalità. Il bersaglio deve effettuare un Tiro Salvezza su Tempra. Se fallisce il Tiro Salvezza il bersaglio subisce 8d8 danni da Vuoto, o la metà di questi danni se supera il Tiro Salvezza. L'incantesimo non ha effetto su non morti o costrutti.

Se il bersaglio è un vegetale non magico che non sia anche una creatura, come un albero o un cespuglio, non effettua alcun Tiro Salvezza, avvizzisce e muore all'istante.

\textbf{Per ogni Successo Critico Magico} ottenuto nella Prova di Magia il danno aumenta di 4d8.

\textbf{Tiro Salvezza Successo/Fallimento Critico}: In caso di Fallimento Critico il danno raddoppia, in caso di Successo Critico il danno viene ulteriormente dimezzato

\smallskip\noindent\rule{\linewidth}{2pt} \index[Incantesimi]{Individuazione del Magico}\hypertarget{Individuazione del Magico}{}\smallskip\noindent{\textbf{Individuazione del Magico}}\pdfbookmark[3]{Individuazione del Magico}{Individuazione del Magico}\index{Occhi della Magia}
\noindent
\begin{description}[noitemsep, topsep=0pt, parsep=0pt, partopsep=0pt, leftmargin=0cm, labelwidth=2.8cm]
	\item[\textbf{Lista di Magia}]: Universale
	\item[\textbf{Livello}]: 1, Comune
	\item[\textbf{T. di Lancio}]: 2 Azioni
	\item[\textbf{Gittata}]: Personale
	\item[\textbf{Componenti}]: V, S
	\item[\textbf{Durata}]: 1d4 +1 round per CM
\end{description}

Per la durata, percepisci la presenza della magia entro 9 metri da te. Puoi usare 1 Azione per vedere una flebile aura che si estende intorno a qualsiasi creatura o oggetto visibile nell'area che rechi magia. Con due Azioni ne apprendi anche la Liste di Magia, se ce l'ha.

L'incantesimo può penetrare la maggior parte delle barriere, ma è bloccato da 30 centimetri di pietra, 2,5 centimetri di metallo comune, un sottile foglio di piombo o 1 metro di legno o terra. Questo incantesimo non permette di vedere cose influenzate dall'incantesimo \hyperlink{Invisibilità}{Invisibilità}.

\textbf{Per ogni Successo Critico Magico} ottenuto nella Prova di Magia la durata aumenta di 3 round.

\smallskip\noindent\rule{\linewidth}{2pt} \index[Incantesimi]{Individuazione delle Malattie e dei Veleni}\hypertarget{Individuazione delle Malattie e dei Veleni}{}\smallskip\noindent{\textbf{Individuazione delle Malattie e dei Veleni}}\pdfbookmark[3]{Individuazione delle Malattie e dei Veleni}{Individuazione delle Malattie e dei Veleni}
\noindent
\begin{description}[noitemsep, topsep=0pt, parsep=0pt, partopsep=0pt, leftmargin=0cm, labelwidth=2.8cm]
	\item[\textbf{Lista di Magia}]: Divinazione
	\item[\textbf{Livello}]: 1, Non Comune
	\item[\textbf{T. di Lancio}]: 2 Azioni
	\item[\textbf{Gittata}]: Personale
	\item[\textbf{Componenti}]: V, S, M (una foglia di basilico)
	\item[\textbf{Durata}]: 1 round per CM
\end{description}

Per la durata, percepisci la presenza e posizione di veleni, creature velenose e malattie entro 9 metri da te. Inoltre riesci a identificare il tipo di veleno, creatura velenosa o malattia. L'incantesimo può penetrare la maggior parte delle barriere, ma è bloccato da 30 centimetri di pietra, 2,5 centimetri di metallo comune, un sottile foglio di piombo o 1 metro di legno o terra.

\textbf{Per ogni Successo Critico Magico} ottenuto nella Prova di Magia durata raddoppia.

\smallskip\noindent\rule{\linewidth}{2pt} \index[Incantesimi]{Individuazione dei Pensieri}\hypertarget{Individuazione dei Pensieri}{}\smallskip\noindent{\textbf{Individuazione dei Pensieri}}\pdfbookmark[3]{Individuazione dei Pensieri}{Individuazione dei Pensieri}
\noindent
\begin{description}[noitemsep, topsep=0pt, parsep=0pt, partopsep=0pt, leftmargin=0cm, labelwidth=2.8cm]
	\item[\textbf{Lista di Magia}]: Divinazione
	\item[\textbf{Livello}]: 2, Raro
	\item[\textbf{T. di Lancio}]: 2 Azioni
	\item[\textbf{Gittata}]: Personale
	\item[\textbf{Componenti}]: V, S, M (un pezzo di rame)
	\item[\textbf{Durata}]: 1 minuto
\end{description}

Per la durata, puoi leggere i pensieri di certe creature. Quando lanci questo incantesimo e con altre due Azioni in ciascun round successivo sino al termine dell'incantesimo, puoi concentrare la tua mente su qualsiasi creatura che tu possa vedere e si trovi entro 9 metri da te. Se la creatura che hai scelto ha un punteggio di Intelligenza -3 o meno o non parla nessun linguaggio, la creatura ignora l'effetto.

Inizialmente, apprendi solo i pensieri di superficie della creatura: quelli più ricorrenti. Con un'Azione, puoi o spostare la tua attenzione sui pensieri di un'altra creatura o tentare di sondare più a fondo la mente della stessa creatura. Se sondi più a fondo, il bersaglio deve effettuare un Tiro Salvezza su Volontà. Se lo fallisce, ottieni una percezione dei suoi ragionamenti (se ve ne sono), del suo stato emotivo, e di ogni cosa abbia prevalenza nei suoi pensieri (come una preoccupazione, l'amore, o l'odio). Se supera il Tiro Salvezza, l'incantesimo termina. A ogni modo, il bersaglio sa che stai sondando la sua mente e, a meno che non sposti la tua attenzione verso la mente di un'altra creatura, nel suo round la creatura può usare la 2 Azioni per effettuare un Tiro Salvezza su Volontà contrapposto; se la vince, l'incantesimo termina.

Le domande poste verbalmente alla creatura bersaglio, ovviamente, modellano il corso dei suoi pensieri, cosicché questo incantesimo risulta particolarmente efficace negli interrogatori.

Puoi anche usare questo incantesimo per individuare la presenza di creature pensanti che non puoi vedere. Quando lanci questo incantesimo o con 2 Azioni nella sua durata, puoi cercare pensieri entro 9 metri da te. L'incantesimo può penetrare le barriere, ma è bloccato da 60 centimetri di pietra, 5 centimetri di metallo che non sia il piombo, o un sottile foglio di piombo. Non puoi individuare una creatura con Intelligenza -3 o meno, o una creatura che non parla alcun linguaggio. Una volta individuata in questo modo la presenza di una creatura, puoi leggerne i pensieri per la durata dell'incantesimo finché resta nella gittata, come descritto sopra, anche se non puoi vederla.
Mentre hai attivo questo incantesimo per il lancio di altri incantesimo risulterai Distratto.

\smallskip\noindent\rule{\linewidth}{2pt} \index[Incantesimi]{Infliggi Ferite}\hypertarget{Infliggi Ferite}{}\smallskip\noindent{\textbf{Infliggi Ferite}}\pdfbookmark[3]{Infliggi Ferite}{Infliggi Ferite}
\noindent
\begin{description}[noitemsep, topsep=0pt, parsep=0pt, partopsep=0pt, leftmargin=0cm, labelwidth=2.8cm]
	\item[\textbf{Lista di Magia}]: Necromanzia
	\item[\textbf{Livello}]: 2, Comune
	\item[\textbf{T. di Lancio}]: 2 Azioni
	\item[\textbf{Gittata}]: Contatto
	\item[\textbf{Componenti}]: V, S
	\item[\textbf{Durata}]: Istantanea
\end{description}

Effettua un attacco in mischia con incantesimo contro una creatura a portata. Se colpisci, il bersaglio subisce 3d10 danni da Vuoto, Tiro Salvezza su Tempra per dimezzare.

\textbf{Per ogni Successo Critico Magico} ottenuto nella Prova di Magia il danno aumenta di 2d8.

\smallskip\noindent\rule{\linewidth}{2pt} \index[Incantesimi]{Ingrandire/Ridurre}\hypertarget{Ingrandire/Ridurre}{}\smallskip\noindent{\textbf{Ingrandire/Ridurre}}\pdfbookmark[3]{Ingrandire/Ridurre}{Ingrandire/Ridurre}
\noindent
\begin{description}[noitemsep, topsep=0pt, parsep=0pt, partopsep=0pt, leftmargin=0cm, labelwidth=2.8cm]
	\item[\textbf{Lista di Magia}]: Trasmutazione
	\item[\textbf{Livello}]: 2, Comune
	\item[\textbf{T. di Lancio}]: 2 Azioni
	\item[\textbf{Gittata}]: 9 metri
	\item[\textbf{Componenti}]: V, S, M (un pizzico di ferro in polvere)
	\item[\textbf{Durata}]: 1 minuto
\end{description}

Fai sì che una creatura od oggetto a gittata e che puoi vedere ingrandisca o rimpicciolisca per la durata dell'incantesimo. Scegli una creatura o un oggetto che non sia né indossato né trasportato. Se il bersaglio non è consenziente, può effettuare un Tiro Salvezza su Tempra, se lo supera, l'incantesimo non ha effetto. Se il bersaglio è una creatura, tutto ciò che sta indossando e trasportando cambia taglia assieme a essa. Qualsiasi oggetto lasciato cadere da una creatura soggetta a questo incantesimo ritorna subito alla sua taglia normale.

\begin{itemize}[leftmargin=*] \setlength{\itemsep}{0pt}
	\item \emph{Ingrandire}. La dimensione del bersaglio raddoppia in tutte le dimensioni, e il suo peso è moltiplicato per otto. Questa crescita aumenta la sua taglia di una categoria: da Media a Grande, per esempio. Se non c'è spazio sufficiente perché il bersaglio raddoppi la sua taglia, la creatura od oggetto assume la taglia più grossa possibile permessagli dallo spazio disponibile. Fino al termine dell'incantesimo, il bersaglio ha +1d6 alle Azioni basate su Forza e ai Tiri Salvezza su Tempra. Le armi del bersaglio crescono per raggiungere la nuova taglia. Mentre queste armi sono ingrandite, gli attacchi del bersaglio con esse faranno una categoria di danno ulteriore.
	\item \emph{Ridurre}. La dimensione del bersaglio si dimezza in tutte le dimensioni, e il suo peso è ridotto a un ottavo. Questa crescita diminuisce la sua taglia di una categoria: da Media a Piccola, per esempio. Fino al termine dell'incantesimo, il bersaglio ha -1d6 alle Azioni basate su Forza e ai Tiri Salvezza su Tempra. Le armi del bersaglio rimpiccioliscono per raggiungere la nuova taglia. Mentre queste armi sono rimpicciolite, gli attacchi del bersaglio con esse faranno una categoria di danno inferiore (ma senza ridurre il danno dell'arma a meno di 1).
\end{itemize}

\textbf{Per ogni due Critici ottenuti} nella Prova di Magia la creatura aumenta di un altra taglia, oppure influenzi un altra creature entro 6 metri dalla prima.

\smallskip\noindent\rule{\linewidth}{2pt} \index[Incantesimi]{Insetto Gigante}\hypertarget{Insetto Gigante}{}\smallskip\noindent{\textbf{Insetto Gigante}}\pdfbookmark[3]{Insetto Gigante}{Insetto Gigante}
\noindent
\begin{description}[noitemsep, topsep=0pt, parsep=0pt, partopsep=0pt, leftmargin=0cm, labelwidth=2.8cm]
	\item[\textbf{Lista di Magia}]: Animali e Piante
	\item[\textbf{Livello}]: 4, Non Comune
	\item[\textbf{T. di Lancio}]: 2 Azioni
	\item[\textbf{Gittata}]: 9 metri
	\item[\textbf{Componenti}]: V, S
	\item[\textbf{Durata}]: 10 minuti
\end{description}

Per la durata dell'incantesimo, trasformi fino a dieci centopiedi, tre ragni, cinque vespe o uno scorpione a gittata, in versioni giganti della loro forma naturale. Un centopiedi diventa un centopiedi gigante, un ragno diventa un ragno gigante, una vespa diventa una vespa gigante e uno scorpione diventa uno scorpione gigante. Ogni creatura obbedisce ai tuoi comandi vocali e, in combattimento, agisce in ciascun round durante il tuo round. Il Narratore possiede le statistiche di queste creature, e sarà sempre Il Narratore a risolvere le loro azioni e i loro movimenti. Una creatura resta nella sua forma gigante per la durata, finché non scende a 0 Punti Ferita, o finché non usi un'Azione per interrompere l'effetto su di essa.

Il Narratore può permetterti di scegliere bersagli differenti. Per esempio, se trasformi un'ape, la sua versione gigante potrebbe avere le stesse statistiche della vespa gigante.

\smallskip\noindent\rule{\linewidth}{2pt} \index[Incantesimi]{Interdizione alla Morte}\hypertarget{Interdizione alla Morte}{}\smallskip\noindent{\textbf{Interdizione alla Morte}}\pdfbookmark[3]{Interdizione alla Morte}{Interdizione alla Morte}
\noindent
\begin{description}[noitemsep, topsep=0pt, parsep=0pt, partopsep=0pt, leftmargin=0cm, labelwidth=2.8cm]
	\item[\textbf{Lista di Magia}]: Necromanzia
	\item[\textbf{Livello}]: 4, Non Comune
	\item[\textbf{T. di Lancio}]: 2 Azioni
	\item[\textbf{Gittata}]: Contatto
	\item[\textbf{Componenti}]: V, S
	\item[\textbf{Durata}]: 1 ora
\end{description}

Lanci l'incantesimo a contatto con una creatura. Conferisci al bersaglio protezione dalla morte. La prima volta che il bersaglio dovesse scendere a 0 Punti Ferita in seguito al danno subito, il bersaglio scende invece a 1 punto ferita e l'incantesimo ha fine. Se l'incantesimo è ancora attivo quando il bersaglio è vittima di un effetto che lo ucciderebbe all'istante senza infliggere danni,quell'effetto viene invece negato sul bersaglio e l'incantesimo ha fine.

\textbf{Per ogni due Successo Critico Magico ottenuto} nella Prova di Magia l'incantesimo protegge una volta in più o protegge un altra creatura.

\smallskip\noindent\rule{\linewidth}{2pt} \index[Incantesimi]{Intermittenza}\hypertarget{Intermittenza}{}\smallskip\noindent{\textbf{Intermittenza}}\pdfbookmark[3]{Intermittenza}{Intermittenza}
\noindent
\begin{description}[noitemsep, topsep=0pt, parsep=0pt, partopsep=0pt, leftmargin=0cm, labelwidth=2.8cm]
	\item[\textbf{Lista di Magia}]: Trasmutazione
	\item[\textbf{Livello}]: 3, Non Comune
	\item[\textbf{T. di Lancio}]: 2 Azioni
	\item[\textbf{Gittata}]: Personale
	\item[\textbf{Componenti}]: V, S
	\item[\textbf{Durata}]: 1 round per CM
\end{description}

Tira un 1d6 alla fine di ciascun tuo round per la durata di questo incantesimo. Se ottieni un numero dispari svanisci dal tuo attuale piano di esistenza e riappari sul Piano Etereo (l'incantesimo fallisce e il lancio è sprecato qualora tu fossi già su quel piano). All'inizio del tuo prossimo round, e quando l'incantesimo termina, qualora tu fossi sul Piano Etereo, ritorni in uno spazio non occupato di tua scelta e che puoi vedere, entro 3 metri dallo spazio da cui sei svanito. Se nessuno spazio non occupato è disponibile entro questa gittata, compari nello spazio non occupato più vicino (determinato casualmente se è disponibile più di uno spazio). Puoi interrompere l'incantesimo con un'Azione.

Mentre sei sul Piano Etereo, puoi vedere e udire il piano da cui provieni, che percepisci in sfumature di grigio, ma non puoi comunque percepire nulla che si trovi a più di 18 metri di distanza. Puoi interagire solo con creature che si trovano sul Piano Etereo. Le creature che non si trovano lì non possono né percepirti né interagire con te, a meno che non siano provviste della capacità di farlo.

\smallskip\noindent\rule{\linewidth}{2pt} \index[Incantesimi]{Intralciare}\hypertarget{Intralciare}{}\smallskip\noindent{\textbf{Intralciare}}\pdfbookmark[3]{Intralciare}{Intralciare}
\noindent
\begin{description}[noitemsep, topsep=0pt, parsep=0pt, partopsep=0pt, leftmargin=0cm, labelwidth=2.8cm]
	\item[\textbf{Lista di Magia}]: Animali e Piante
	\item[\textbf{Livello}]: 1, Comune
	\item[\textbf{T. di Lancio}]: 2 Azioni
	\item[\textbf{Gittata}]: 27 metri
	\item[\textbf{Componenti}]: V, S
	\item[\textbf{Durata}]: 1 minuto
\end{description}

Rampicanti e rami stritolanti spuntano dal terreno in un quadrato di 6 metri di lato a partire da un punto a gittata. Per la durata, questi vegetali trasformano il terreno nell'area in terreno difficile.

Una creatura nell'area nel momento in cui lanci questo incantesimo deve superare un Tiro Salvezza su Tempra o restare intralciata da questi vegetali fino al termine dell'incantesimo. Una creatura \hyperlink{intralciato}{intralciata} (vedi pag. \pageref{intralciato}) dai vegetali può usare due Azioni per effettuare un nuovo Tiro Salvezza. Se la supera, si libera. Quando l'incantesimo ha termine, i vegetali evocati svaniscono.

\smallskip\noindent\rule{\linewidth}{2pt} \index[Incantesimi]{Inversione della Gravità}\hypertarget{Inversione della Gravità}{}\smallskip\noindent{\textbf{Inversione della Gravità}}\pdfbookmark[3]{Inversione della Gravita'}{Inversione della Gravita'}
\noindent
\begin{description}[noitemsep, topsep=0pt, parsep=0pt, partopsep=0pt, leftmargin=0cm, labelwidth=2.8cm]
	\item[\textbf{Lista di Magia}]: Trasmutazione
	\item[\textbf{Livello}]: 7, Raro
	\item[\textbf{T. di Lancio}]: 2 Azioni
	\item[\textbf{Gittata}]: 30 metri
	\item[\textbf{Componenti}]: V, S, M (una calamita e un fil di ferro)
	\item[\textbf{Durata}]: 1 minuto
\end{description}

Questo incantesimo inverte la gravità in un cilindro di raggio 15 metri, alto 30 metri, centrato in un punto a gittata. Quando lanci questo incantesimo, tutte le creature e gli oggetti che non sono in qualche modo ancorati al terreno cadono verso l'alto e raggiungono la cima dell'area. Una creatura può tentare un Tiro Salvezza su Riflessi per afferrare un oggetto fisso a portata, per evitare di cadere in questo modo, in caso lo superi.

Se lungo questa caduta si incontra un oggetto solido (il soffitto), gli oggetti e le creature che cadono vi impattano come accadrebbe durante una normale caduta. Se un oggetto o creatura raggiunge la cima dell'area senza colpire nulla, rimane lì, oscillando lievemente, per la durata.

Al termine della durata, gli oggetti e le creature colpite ricadono verso il basso.

\smallskip\noindent\rule{\linewidth}{2pt} \index[Incantesimi]{Inviare}\hypertarget{Inviare}{}\smallskip\noindent{\textbf{Inviare}}\pdfbookmark[3]{Inviare}{Inviare}
\noindent
\begin{description}[noitemsep, topsep=0pt, parsep=0pt, partopsep=0pt, leftmargin=0cm, labelwidth=2.8cm]
	\item[\textbf{Lista di Magia}]: Invocazione
	\item[\textbf{Livello}]: 3, Comune
	\item[\textbf{T. di Lancio}]: 2 Azioni
	\item[\textbf{Gittata}]: Illimitata
	\item[\textbf{Componenti}]: V, S, M (un piccolo pezzo di cavo di rame)
	\item[\textbf{Durata}]: 1 round
\end{description}

Invii un breve messaggio di 25 parole o meno a una creatura con cui sei familiare. La creatura sente il messaggio nella sua mente, ti riconosce come mittente e può risponderti in modo simile. L'incantesimo permette a creature con un punteggio di Intelligenza almeno di -2 di comprendere il significato del tuo messaggio anche se non comprende la tua lingua.

Puoi inviare il messaggio attraverso qualsiasi distanza e anche su altri piani di esistenza, ma se il bersaglio è su di un piano diverso dal tuo, c'è una probabilità del 5\% che il messaggio non arrivi.

\textbf{Per ogni Successo Critico Magico} ottenuto nella Prova di Magia aumenti di 25 parole il messaggio o di un round la durata.

\textbf{NOTA}: i Devoti di Nethergal ottengono un Successo Critico Magico in automatico al lancio dell'incantesimo

\smallskip\noindent\rule{\linewidth}{2pt} \index[Incantesimi]{Invisibilità}\hypertarget{Invisibilità}{}\smallskip\noindent{\textbf{Invisibilità}}\pdfbookmark[3]{Invisibilita'}{Invisibilita'}
\noindent
\begin{description}[noitemsep, topsep=0pt, parsep=0pt, partopsep=0pt, leftmargin=0cm, labelwidth=2.8cm]
	\item[\textbf{Lista di Magia}]: Illusione
	\item[\textbf{Livello}]: 2, Comune
	\item[\textbf{T. di Lancio}]: 2 Azioni
	\item[\textbf{Gittata}]: Contatto
	\item[\textbf{Componenti}]: V, S, M (un ciglio avvolto nella gomma arabica)
	\item[\textbf{Durata}]: 1 minuto per CM
\end{description}

Lanci l'incantesimo a contatto di una creatura. Il bersaglio diventa invisibile fino alla fine dell'incantesimo. Qualsiasi cosa il bersaglio stia indossando o trasportando diventa invisibile finché resta sul bersaglio. L'incantesimo ha fine per il bersaglio che attacca od esegue un incantesimo.

\textbf{Per ogni Successo Critico Magico} ottenuto nella Prova di Magia puoi scegliere un'ulteriore creatura bersaglio o aumenti 1 minuto la durata.

\smallskip\noindent\rule{\linewidth}{2pt} \index[Incantesimi]{Invisibilità Superiore}\hypertarget{Invisibilità Superiore}{}\smallskip\noindent{\textbf{Invisibilità Superiore}}\pdfbookmark[3]{Invisibilita' Superiore}{Invisibilita' Superiore}
\noindent
\begin{description}[noitemsep, topsep=0pt, parsep=0pt, partopsep=0pt, leftmargin=0cm, labelwidth=2.8cm]
	\item[\textbf{Lista di Magia}]: Illusione
	\item[\textbf{Livello}]: 4, Non Comune
	\item[\textbf{T. di Lancio}]: 2 Azioni
	\item[\textbf{Gittata}]: Contatto
	\item[\textbf{Componenti}]: V, S
	\item[\textbf{Durata}]: 1 minuto
\end{description}

Lanci l'incantesimo a contatto di una creatura. Il bersaglio diventa invisibile fino alla fine dell'incantesimo. Qualsiasi cosa indossata o trasportata dal bersaglio diventa invisibile finché resta addosso al bersaglio.

Eseguire incantesimi o azioni di attacco non fa diventare visibile.

\smallskip\noindent\rule{\linewidth}{2pt} \index[Incantesimi]{Invocare il Fulmine}\hypertarget{Invocare il Fulmine}{}\smallskip\noindent{\textbf{Invocare il Fulmine}}\pdfbookmark[3]{Invocare il Fulmine}{Invocare il Fulmine}
\noindent
\begin{description}[noitemsep, topsep=0pt, parsep=0pt, partopsep=0pt, leftmargin=0cm, labelwidth=2.8cm]
	\item[\textbf{Lista di Magia}]: Aria
	\item[\textbf{Livello}]: 3, Comune
	\item[\textbf{T. di Lancio}]: 1 round
	\item[\textbf{Gittata}]: 36 metri
	\item[\textbf{Componenti}]: V, S
	\item[\textbf{Durata}]: Concentrazione, massimo 10 minuti
\end{description}

Una nube di tempesta compare nella forma di un cilindro alto 3 metri con un raggio di 18 metri, centrato su di un punto che puoi vedere, 30 metri sopra di te. L'incantesimo fallisce automaticamente se non puoi vedere il punto nell'aria dove apparirà la nube di tempesta (per esempio, se sei in una stanza che non può accogliere la nube). Quando lanci l'incantesimo, scegli un punto che puoi vedere entro la gittata. Un fulmine si abbatterà dalla nuvola su quel punto. Ogni creatura entro 1 metro da quel punto deve effettuare un Tiro Salvezza su Riflessi. Una creatura subisce 3d10 danni da elettricità se fallisce il Tiro Salvezza, o la metà di questi danni se lo supera. Durante ciascun tuo round fino al termine dell'incantesimo, puoi usare due Azioni per richiamare un altro fulmine in questo modo, prendendo come bersaglio lo stesso punto o uno diverso.

Se quando lanci questo incantesimo ti trovi all'esterno in condizioni di tempesta, l'incantesimo ti fornisce il controllo della tempesta esistente invece di crearne una nuova. Sotto queste condizioni, il danno dell'incantesimo aumenta di 1d10.

\textbf{Per ogni Successo Critico Magico} ottenuto nella Prova di Magia il danno aumenta di 2d6.

\smallskip\noindent\rule{\linewidth}{2pt} \index[Incantesimi]{Labirinto}\hypertarget{Labirinto}{}\smallskip\noindent{\textbf{Labirinto}}\pdfbookmark[3]{Labirinto}{Labirinto}
\noindent
\begin{description}[noitemsep, topsep=0pt, parsep=0pt, partopsep=0pt, leftmargin=0cm, labelwidth=2.8cm]
	\item[\textbf{Lista di Magia}]: Evocazione
	\item[\textbf{Livello}]: 8, Raro
	\item[\textbf{T. di Lancio}]: 2 Azioni
	\item[\textbf{Gittata}]: 18 metri
	\item[\textbf{Componenti}]: V, S
	\item[\textbf{Durata}]: massimo 10 minuti
\end{description}

Bandisci una creatura a gittata e che puoi vedere in un semipiano labirintico. Il bersaglio rimane lì per la durata dell'incantesimo o finché non fugge dal labirinto. Il bersaglio può impiegare 3 Azioni per tentare di fuggire a partire dal secondo round. Quando lo fa, effettua un Tiro Salvezza su Volontà. Se la supera, fugge, e l'incantesimo termina (un minotauro o un demone goristro riescono automaticamente).

Quando l'incantesimo termina, il bersaglio riappare nello spazio che aveva lasciato o, se quello spazio è occupato nel più vicino spazio non occupato.

\textbf{Per ogni Successo Critico Magico} ottenuto nella Prova di Magia la durata aumenta di 10 minuti. Con due Successo Critico Magico puoi influenzare un altra creatura.

\smallskip\noindent\rule{\linewidth}{2pt} \index[Incantesimi]{Lacrima di Laydel}\hypertarget{Lacrima di Laydel}{}\smallskip\noindent{\textbf{Lacrima di Laydel}}\pdfbookmark[3]{Lacrima di Laydel}{Lacrima di Laydel}
\noindent
\begin{description}[noitemsep, topsep=0pt, parsep=0pt, partopsep=0pt, leftmargin=0cm, labelwidth=2.8cm]
	\item[\textbf{Lista di Magia}]: Invocazione
	\item[\textbf{Livello}]: 2, Molto Raro/Comune
	\item[\textbf{T. di Lancio}]: 2 Azione/1 Azione
	\item[\textbf{Gittata}]: 36 metri
	\item[\textbf{Componenti}]: V, S, M (una lacrima dell'incantatore)
	\item[\textbf{Durata}]: Istantaneo
\end{description}

L'incantatore permea di magia una lacrima che getta contro l'avversario, è necessario un Tiro per Colpire con Incantesimi a distanza.
La creatura subisce 1d6+2d6 di danno, per stabilire il tipo di danno consultare la tabella con i valori del primo d6 tirato.

\medskip

\begin{tabular}{l|l}
	\textbf{1d6}&\textbf{Energia}\\
	\hline
	1 &Fuoco\\
	2 &Elettricità\\
	3 &Freddo\\
	4 &Suono\\
	5 &Vuoto\\
	6 &Forza
\end{tabular}

\medskip

Il danno che l'obiettivo subisce è del tipo di Energia che risulta dal primo d6. Se il primo dado è un 6 ed è 6 anche uno degli altri dadi allora tira nuovamente 1d6 e somma al danno.

Per un Devoto di Laydel questo incantesimo è Comune e ha un tempo di lancio di 1 Azione. Inoltre può continuare a tirare ulteriori d6 di danno purché continui a tirare 6 con quel dado.

\smallskip\noindent\rule{\linewidth}{2pt} \index[Incantesimi]{Lacrima di Ljust}\hypertarget{Lacrima di Ljust}{}\smallskip\noindent{\textbf{Lacrima di Ljust}}\pdfbookmark[3]{Lacrima di Ljust}{Lacrima di Ljust}
\noindent
\begin{description}[noitemsep, topsep=0pt, parsep=0pt, partopsep=0pt, leftmargin=0cm, labelwidth=2.8cm]
	\item[\textbf{Lista di Magia}]: Universale
	\item[\textbf{Livello}]: 0, Non Comune
	\item[\textbf{T. di Lancio}]: 1 Azione
	\item[\textbf{Gittata}]: Personale
	\item[\textbf{Componenti}]: S, M (un piccolo oggetto)
	\item[\textbf{Durata}]: 10 round
\end{description}

L'incantatore permea di magia un piccolo oggetto che incomincia a brillare di luce. La luce si illumina il suo quadretto ed un ulteriore metro attorno, oltre non genera luce fioca. La durata dell'incantesimo è 10 round. L'incantatore può lanciare l'oggetto entro 18 metri e deve rimanere entro questa distanza. Non è possibile lanciare l'incantesimo più volte al giorno di quanti Punti Fato si possiedono.

\smallskip\noindent\rule{\linewidth}{2pt} \index[Incantesimi]{Lama Infuocata}\hypertarget{Lama Infuocata}{}\smallskip\noindent{\textbf{Lama Infuocata}}\pdfbookmark[3]{Lama Infuocata}{Lama Infuocata}
\noindent
\begin{description}[noitemsep, topsep=0pt, parsep=0pt, partopsep=0pt, leftmargin=0cm, labelwidth=2.8cm]
	\item[\textbf{Lista di Magia}]: Fuoco
	\item[\textbf{Livello}]: 2, Comune
	\item[\textbf{T. di Lancio}]: 1 Azione
	\item[\textbf{Gittata}]: Personale
	\item[\textbf{Componenti}]: V, S, M (una foglia di sommacco)
	\item[\textbf{Durata}]: Concentrazione, massimo 10 minuti
\end{description}

Crei nella tua mano una lama infuocata. La lama è simile in dimensioni e forma a una scimitarra e rimane per la durata. Se lasci andare la lama, questa sparisce, ma ne puoi creare un'altra con un'Azione. Puoi usare 2 Azioni per effettuare un attacco in mischia con la lama infuocata. Se colpisci, il bersaglio subisce 3d6 danni da fuoco. La lama infuocata emana luce intensa in un raggio di 3 metri e luce fioca per 6 metri.

\textbf{Per ogni due Critici ottenuti} nella Prova di Magia il danno aumenta di 1d6.

\smallskip\noindent\rule{\linewidth}{2pt} \index[Incantesimi]{Lanciafiamme}\hypertarget{Lanciafiamme}{}\smallskip\noindent{\textbf{Lanciafiamme}}\pdfbookmark[3]{Lanciafiamme}{Lanciafiamme}
\noindent
\begin{description}[noitemsep, topsep=0pt, parsep=0pt, partopsep=0pt, leftmargin=0cm, labelwidth=2.8cm]
	\item[\textbf{Lista di Magia}]: Fuoco
	\item[\textbf{Livello}]: 2, Raro
	\item[\textbf{T. di Lancio}]: 2 Azioni
	\item[\textbf{Gittata}]: Personale
	\item[\textbf{Componenti}]: V, S, M (un tubo in ferro di 30 cm, dei fagioli)
	\item[\textbf{Durata}]: 1 minuto
\end{description}

Una fiammella compare al termine del tubo di metallo che tieni nella mano. La fiamma resta lì per la durata dell'incantesimo e non danneggia né te né il tuo equipaggiamento. La fiamma produce luce intensa nel raggio di 1 metro e luce fioca nel raggio di 1 metro. L'incantesimo termina se lo interrompi con un'Azione o se lo lanci di nuovo.

Con un Tiro per Colpire con incantesimo a distanza e spendendo 1 Azione puoi allungare la fiamma fino a 9 metri per colpire un bersaglio. Se colpisci, il bersaglio subisce 2d6 danni da fuoco, finché mantieni lo stesso bersaglio hai un +2 al colpire con l'attacco successivo.

\textbf{Per ogni Critico ottenuto} nella Prova di Magia il danno aumenta di 1d6.

\smallskip\noindent\rule{\linewidth}{2pt} \index[Incantesimi]{Legame Telepatico}\hypertarget{Legame Telepatico}{}\smallskip\noindent{\textbf{Legame Telepatico}}\pdfbookmark[3]{Legame Telepatico}{Legame Telepatico}
\noindent
\begin{description}[noitemsep, topsep=0pt, parsep=0pt, partopsep=0pt, leftmargin=0cm, labelwidth=2.8cm]
	\item[\textbf{Lista di Magia}]: Divinazione
	\item[\textbf{Livello}]: 5, Raro
	\item[\textbf{T. di Lancio}]: 2 Azioni
	\item[\textbf{Gittata}]: 9 metri
	\item[\textbf{Componenti}]: V, S, M (pezzi di gusci d'uovo da due differenti specie di creature)
	\item[\textbf{Durata}]: 1 ora
\end{description}

Stabilisci un collegamento telepatico tra un massimo di otto creature consenzienti a gittata di tua scelta, collegando psichicamente ciascuna creatura alle altre per la durata dell'incantesimo. Le creature con punteggio di Intelligenza -3 o meno ignorano questo incantesimo. Fino al termine dell'incantesimo, i bersagli possono comunicare telepaticamente tramite questo legame, che condividano o meno un linguaggio comune. La comunicazione è possibile a qualsiasi distanza, ma non può estendersi su differenti piani di esistenza.

\textbf{Per ogni Successo Critico Magico} ottenuto nella Prova di Magia la durata aumenta di 1 ora.

\smallskip\noindent\rule{\linewidth}{2pt} \index[Incantesimi]{Lentezza}\hypertarget{Lentezza}{}\smallskip\noindent{\textbf{Lentezza}}\pdfbookmark[3]{Lentezza}{Lentezza}\hypertarget{lentezza}{}
\noindent
\begin{description}[noitemsep, topsep=0pt, parsep=0pt, partopsep=0pt, leftmargin=0cm, labelwidth=2.8cm]
	\item[\textbf{Lista di Magia}]: Trasmutazione
	\item[\textbf{Livello}]: 3, Non Comune
	\item[\textbf{T. di Lancio}]: 2 Azioni
	\item[\textbf{Gittata}]: 36 metri
	\item[\textbf{Componenti}]: V, S, M (una goccia di melassa)
	\item[\textbf{Durata}]: 1 minuto, Concentrazione
\end{description}

Rallenti il metabolismo di massimo 2 più le volte che hai preso Adepto della Magia creature a tua scelta in un raggio di 3 metri a gittata. Al lancio dell'incantesimo ciascun bersaglio deve superare un Tiro Salvezza su Volontà od eseguire una Azione in meno a round per la durata dell'incantesimo.

Questo incantesimo \hyperlink{contrastareincantesimi}{contrasta ed è contrastato} da \hyperlink{Velocità}{Velocità}.

\textbf{Per ogni Successo Critico Magico} ottenuto nella Prova di Magia puoi influenzare una creatura in più.

\textbf{Tiro Salvezza Fallimento Critico}: In caso di Fallimento Critico si viene rallentati di una ulteriore Azione.

\smallskip\noindent\rule{\linewidth}{2pt} \index[Incantesimi]{Lettura della terra di Kyrin}\hypertarget{Lettura della terra di Kyrin}{}\smallskip\noindent{\textbf{Lettura della terra di Kyrin}}\pdfbookmark[3]{Lettura della terra di Kyrin}{Lettura della terra di Kyrin}
\noindent
\begin{description}[noitemsep, topsep=0pt, parsep=0pt, partopsep=0pt, leftmargin=0cm, labelwidth=2.8cm]
	\item[\textbf{Lista di Magia}]: Terra
	\item[\textbf{Livello}]: 2, Non Comune
	\item[\textbf{T. di Lancio}]: 1 Round
	\item[\textbf{Gittata}]: Personale (raggio di 30 metri)
	\item[\textbf{Componenti}]: V, S
	\item[\textbf{Durata}]: Istantanea
\end{description}

Appoggi le mani sulla terra e formulato l'incantesimo hai una fugace visione dell'ambiente intorno a te nel raggio sferico di 30 metri.
Riesci a percepire la posizione e relativa forma delle creature e delle strutture che poggiano a terra.

\textbf{Per ogni Successo Critico Magico} ottenuto nella Prova di Magia il raggio aumenta di 10 metri.

\smallskip\noindent\rule{\linewidth}{2pt} \index[Incantesimi]{Levitazione}\hypertarget{Levitazione}{}\smallskip\noindent{\textbf{Levitazione}}\pdfbookmark[3]{Levitazione}{Levitazione}
\noindent
\begin{description}[noitemsep, topsep=0pt, parsep=0pt, partopsep=0pt, leftmargin=0cm, labelwidth=2.8cm]
	\item[\textbf{Lista di Magia}]: Aria
	\item[\textbf{Livello}]: 2, Comune
	\item[\textbf{T. di Lancio}]: 2 Azioni
	\item[\textbf{Gittata}]: 18 metri
	\item[\textbf{Componenti}]: V, S, M (o un piccolo laccio di cuoio oppure un pezzo di cavo d'oro piegato a forma di tazza con un lungo stelo alla fine)
	\item[\textbf{Durata}]: 10 minuti
\end{description}

Una creatura o oggetto a gittata che puoi vedere, scelto da te, si alza verticalmente fino a 6 metri e rimane sospeso per la durata dell'incantesimo. L'incantesimo può levitare un bersaglio pesante fino a 250 chili. Una creatura non consenziente che superi un Tiro Salvezza su Tempra ignora l'effetto.

Il bersaglio può muoversi solo spingendo o tirando verso un oggetto fisso o superficie a portata (per esempio un muro o un soffitto). Durante il tuo round puoi cambiare l'altitudine del bersaglio fino a 6 metri in entrambe le direzioni. Se sei tu il bersaglio, ti puoi muovere verso l'alto o il basso come parte del tuo movimento. Altrimenti puoi usare 1 Azione per muovere il bersaglio, che deve rimanere nella gittata dell'incantesimo. Quando l'incantesimo termina, qualora sia ancora in aria, il bersaglio fluttua dolcemente a terra.

Mentre sei sotto l'influenza di questo incantesimo sei considerato Distratto nel lancio di incantesimi.

\textbf{Per ogni Successo Critico Magico} ottenuto nella Prova di Magia puoi spostarti di 1 metro lateralmente o influenzi un altra creatura.

\smallskip\noindent\rule{\linewidth}{2pt} \index[Incantesimi]{Lettura del Magico}\hypertarget{Lettura del Magico}{}\smallskip\noindent{\textbf{Lettura del Magico}}\pdfbookmark[3]{Lettura del Magico}{Lettura del Magico}
\noindent
\begin{description}[noitemsep, topsep=0pt, parsep=0pt, partopsep=0pt, leftmargin=0cm, labelwidth=2.8cm]
	\item[\textbf{Lista di Magia}]: Universale
	\item[\textbf{Livello}]: 1, Comune
	\item[\textbf{T. di Lancio}]: 1 Azione
	\item[\textbf{Gittata}]: Contatto
	\item[\textbf{Componenti}]: V, S, M (un frammento di un oggetto incantato)
	\item[\textbf{Durata}]: 1 minuto, finché usato
\end{description}

L'incantatore conferisce la capacità di leggere una pergamena o una scritta magica ad un bersaglio. Per la durata di 1 minuto o finché usato una volta, quale venga prima, la creatura riesce automaticamente a comprendere una pergamena magica od a lanciare il contenuto della pergamena rispettando i criteri e regole di lancio incantesimi da pergamena.

\textbf{Per ogni Successo Critico Magico} ottenuto nella Prova di Magia puoi leggere o comprendere una pergamena in più.

\smallskip\noindent\rule{\linewidth}{2pt} \index[Incantesimi]{Libertà di Movimento}\hypertarget{Libertà di Movimento}{}\smallskip\noindent{\textbf{Libertà di Movimento}}\pdfbookmark[3]{Liberta' di Movimento}{Liberta' di Movimento}
\noindent
\begin{description}[noitemsep, topsep=0pt, parsep=0pt, partopsep=0pt, leftmargin=0cm, labelwidth=2.8cm]
	\item[\textbf{Lista di Magia}]: Abiurazione
	\item[\textbf{Livello}]: 4, Comune
	\item[\textbf{T. di Lancio}]: 2 Azioni
	\item[\textbf{Gittata}]: Contatto
	\item[\textbf{Componenti}]: V, S, M (una striscia di cuoio, avvolta intorno a un braccio o simile appendice)
	\item[\textbf{Durata}]: 1 ora
\end{description}

Lanci l'incantesimo a contatto di una creatura consenziente. Per la sua durata, il movimento del bersaglio ignora il terreno difficile naturale, mentre gli incantesimi o altri effetti magici non possono ridurre la sua velocità né far sì che il bersaglio sia paralizzato o intralciato se hanno un DC inferiore a quella dell'incantesimo stesso.

Il bersaglio può usare due Azioni per liberarsi automaticamente da qualsiasi restrizione non magica, come manette o una creatura da cui è afferrato. Infine, trovarsi sott'acqua non comporta penalità al movimento o gli attacchi del bersaglio.

\textbf{Per due Successo Critico Magico ottenuto} nella Prova di Magia puoi influenzare un altra creatura.

\smallskip\noindent\rule{\linewidth}{2pt} \index[Incantesimi]{Lingue}\hypertarget{Lingue}{}\smallskip\noindent{\textbf{Lingue}}\pdfbookmark[3]{Lingue}{Lingue}
\noindent
\begin{description}[noitemsep, topsep=0pt, parsep=0pt, partopsep=0pt, leftmargin=0cm, labelwidth=2.8cm]
	\item[\textbf{Lista di Magia}]: Divinazione
	\item[\textbf{Livello}]: 3, Comune
	\item[\textbf{T. di Lancio}]: 2 Azioni
	\item[\textbf{Gittata}]: Contatto
	\item[\textbf{Componenti}]: V, M (un piccolo modello di argilla di una ziggurat)
	\item[\textbf{Durata}]: 1 ora
\end{description}

Questo incantesimo conferisce alla creatura con cui sei stato in contatto al momento del lancio dell'incantesimo la capacità di comprendere qualsiasi linguaggio parlata che ode. Inoltre, quando il bersaglio parla, qualsiasi creatura che conosca almeno un linguaggio e può udire il bersaglio, comprende ciò che dice.

\textbf{Per ogni Successo Critico Magico} ottenuto nella Prova di Magia la durata raddoppia o influenzi un'altra creatura.

\smallskip\noindent\rule{\linewidth}{2pt} \index[Incantesimi]{Localizza Animali e Piante}\hypertarget{Localizza Animali e Piante}{}\smallskip\noindent{\textbf{Localizza Animali e Piante}}\pdfbookmark[3]{Localizza Animali e Piante}{Localizza Animali e Piante}
\noindent
\begin{description}[noitemsep, topsep=0pt, parsep=0pt, partopsep=0pt, leftmargin=0cm, labelwidth=2.8cm]
	\item[\textbf{Lista di Magia}]: Animali e Piante
	\item[\textbf{Livello}]: 2, Non Comune
	\item[\textbf{T. di Lancio}]: 2 Azioni
	\item[\textbf{Gittata}]: Personale
	\item[\textbf{Componenti}]: V, S, M (un pezzo di pelo di un segugio)
	\item[\textbf{Durata}]: Istantanea
\end{description}

Descrivi o nomina uno specifico tipo di bestia o vegetale. Concentrandoti sulla voce della natura nei tuoi dintorni, apprendi la direzione e la distanza dalla più vicina creatura o vegetale di quella specie, se ce ne sono entro 7,5 chilometri.

\textbf{Per ogni Successo Critico Magico} ottenuto nella Prova di Magia aumenti di 1 km l'area controllata

\smallskip\noindent\rule{\linewidth}{2pt} \index[Incantesimi]{Localizza Creatura}\hypertarget{Localizza Creatura}{}\smallskip\noindent{\textbf{Localizza Creatura}}\pdfbookmark[3]{Localizza Creatura}{Localizza Creatura}
\noindent
\begin{description}[noitemsep, topsep=0pt, parsep=0pt, partopsep=0pt, leftmargin=0cm, labelwidth=2.8cm]
	\item[\textbf{Lista di Magia}]: Divinazione
	\item[\textbf{Livello}]: 4, Comune
	\item[\textbf{T. di Lancio}]: 2 Azioni
	\item[\textbf{Gittata}]: Personale
	\item[\textbf{Componenti}]: V, S, M (un pezzo di pelliccia di segugio)
	\item[\textbf{Durata}]: Concentrazione, massimo 1 ora
\end{description}

Descrivi o nomina una creatura che ti è familiare. Percepisci la direzione della posizione della creatura, purché quella creatura si trovi entro 300 metri da te. Se la creatura si muove, conosci anche la direzione del suo movimento.

L'incantesimo può localizzare una specifica creatura a te nota, o la più vicina creatura di una specie (come umano o unicorno), purché tu abbia visto una simile creatura da vicino (entro 9 metri) almeno una volta. Se la creatura che descrivi o nomini ha una forma diversa, per esempio è sotto gli effetti dell'incantesimo metamorfosi, questo incantesimo non sarà in grado di localizzare la creatura.

Questo incantesimo non può localizzare una creatura se un flusso di acqua corrente largo almeno 3 metri blocca un percorso diretto tra te e la creatura.

\textbf{Per ogni Successo Critico Magico} ottenuto nella Prova di Magia aumenta la distanza di altri 300m.

\smallskip\noindent\rule{\linewidth}{2pt} \index[Incantesimi]{Localizza Oggetto}\hypertarget{Localizza Oggetto}{}\smallskip\noindent{\textbf{Localizza Oggetto}}\pdfbookmark[3]{Localizza Oggetto}{Localizza Oggetto}
\noindent
\begin{description}[noitemsep, topsep=0pt, parsep=0pt, partopsep=0pt, leftmargin=0cm, labelwidth=2.8cm]
	\item[\textbf{Lista di Magia}]: Divinazione
	\item[\textbf{Livello}]: 2, Comune
	\item[\textbf{T. di Lancio}]: 2 Azioni
	\item[\textbf{Gittata}]: Personale
	\item[\textbf{Componenti}]: V, S, M (un ramoscello biforcuto)
	\item[\textbf{Durata}]: Concentrazione, massimo 10 minuti
\end{description}

Descrivi o nomina un oggetto che ti è familiare. Percepisci la direzione della posizione dell'oggetto, purché quell'oggetto si trovi entro 300 metri da te. Se l'oggetto si muove, conosci anche la direzione del suo movimento.

L'incantesimo può localizzare uno specifico oggetto a te noto, purché tu lo abbia visto da vicino (entro 9 metri) almeno una volta. In alternativa, l'incantesimo può localizzare l'oggetto più vicino di un particolare tipo, come certi tipi di abbigliamento, gioielleria, mobili, attrezzi o armi.

Questo incantesimo non può localizzare un oggetto se qualsiasi spessore di piombo, anche un foglio sottile, blocca un percorso diretto tra di te e l'oggetto.

\textbf{Per ogni Successo Critico Magico} ottenuto nella Prova di Magia la durata aumenta di 1 ora.

\smallskip\noindent\rule{\linewidth}{2pt} \index[Incantesimi]{Loquacità}\hypertarget{Loquacità}{}\smallskip\noindent{\textbf{Loquacità}}\pdfbookmark[3]{Loquacita'}{Loquacita'}
\noindent
\begin{description}[noitemsep, topsep=0pt, parsep=0pt, partopsep=0pt, leftmargin=0cm, labelwidth=2.8cm]
	\item[\textbf{Lista di Magia}]: Trasmutazione
	\item[\textbf{Livello}]: 8, Raro
	\item[\textbf{T. di Lancio}]: 2 Azioni
	\item[\textbf{Gittata}]: Personale
	\item[\textbf{Componenti}]: V
	\item[\textbf{Durata}]: 1 ora
\end{description}

Fino al termine dell'incantesimo, quando effettui una prova basata sul Carisma puoi rimpiazzare il numero tirato con 15. Inoltre, non importa quello che dici, la magia o l'analisi che determina se stai dicendo la verità indicherà sempre che sei onesto.

\textbf{Per ogni Successo Critico Magico} ottenuto nella Prova di Magia aumenti di 1 ora la durata.

\smallskip\noindent\rule{\linewidth}{2pt} \index[Incantesimi]{Luce}\hypertarget{Luce}{}\smallskip\noindent{\textbf{Luce}}\pdfbookmark[3]{Luce}{Luce}
\noindent
\begin{description}[noitemsep, topsep=0pt, parsep=0pt, partopsep=0pt, leftmargin=0cm, labelwidth=2.8cm]
	\item[\textbf{Lista di Magia}]: Universale
	\item[\textbf{Livello}]: 1, Comune
	\item[\textbf{T. di Lancio}]: 2 Azioni
	\item[\textbf{Gittata}]: Contatto
	\item[\textbf{Componenti}]: V, M (una lucciola o del muschio fosforescente)
	\item[\textbf{Durata}]: 30 minuti di tempo reale di gioco
\end{description}

Lanci l'incantesimo a contatto di un oggetto che non sia più grosso di 3 metri in qualsiasi direzione. Fino al termine dell'incantesimo, l'oggetto irradia una luce intensa in un raggio di 3 metri e penombra per ulteriori 6 metri. La luce può essere di qualsiasi colore tu voglia. Coprire completamente l'oggetto con qualcosa di opaco blocca la luce. Se un oggetto bersaglio è tenuto o indossato da una creatura ostile, quella creatura deve superare un Tiro Salvezza su Riflessi per evitare l'incantesimo. Una creatura colpita dall'incantesimo deve effettuare un Tiro Salvezza su Tempra o rimanere accecato fino alla fine del round successivo. Non puoi avere attivo più di un incantesimo Luce alla volta, un lancio successivo spegne la precedente Luce.

\textbf{Per ogni Critico ottenuto} nella Prova di Magia la durata aumenta di 10 minuti reali.

\smallskip\noindent\rule{\linewidth}{2pt} \index[Incantesimi]{Luce Diurna}\hypertarget{Luce Diurna}{}\smallskip\noindent{\textbf{Luce Diurna}}\pdfbookmark[3]{Luce Diurna}{Luce Diurna}
\noindent
\begin{description}[noitemsep, topsep=0pt, parsep=0pt, partopsep=0pt, leftmargin=0cm, labelwidth=2.8cm]
	\item[\textbf{Lista di Magia}]: Invocazione
	\item[\textbf{Livello}]: 3, Comune
	\item[\textbf{T. di Lancio}]: 2 Azioni
	\item[\textbf{Gittata}]: 18 metri
	\item[\textbf{Componenti}]: V, S
	\item[\textbf{Durata}]: 1 ora di tempo reale di gioco
\end{description}

Una sfera di luce con raggio 6 metri si espande da un punto a tua scelta entro la gittata. La sfera irradia luce intensa e luce fioca per ulteriori 12 metri. Se scegli un punto su di un oggetto che stai reggendo o che non è indossato o trasportato, la luce si irradia dall'oggetto e si muove con esso. Coprire completamente un oggetto con qualcosa di opaco, come un vaso o un elmo, blocca la luce. Se qualsiasi parte dell'area di questo incantesimo si sovrappone con l'area di oscurità creata da un incantesimo di livello 3 o più basso, l'incantesimo che ha creato l'oscurità viene dissolto. La luce creata si considera luce solare.

\textbf{NOTA}: i Devoti di Ljust o Sumkjr prendono +1 ai Tiri Salvezza finché illuminati da questo incantesimo

\smallskip\noindent\rule{\linewidth}{2pt} \index[Incantesimi]{Luci Danzanti}\hypertarget{Luci Danzanti}{}\smallskip\noindent{\textbf{Luci Danzanti}}\pdfbookmark[3]{Luci Danzanti}{Luci Danzanti}
\noindent
\begin{description}[noitemsep, topsep=0pt, parsep=0pt, partopsep=0pt, leftmargin=0cm, labelwidth=2.8cm]
	\item[\textbf{Lista di Magia}]: Invocazione
	\item[\textbf{Livello}]: 1, Non Comune
	\item[\textbf{T. di Lancio}]: 2 Azioni
	\item[\textbf{Gittata}]: 36 metri
	\item[\textbf{Componenti}]: V, S, M (un pezzo di fosforo o legno stregato, o un lombrico)
	\item[\textbf{Durata}]: 10 minuti di tempo reale di gioco
\end{description}

Crei, a gittata, fino a quattro luci delle dimensioni di una torcia, facendole apparire come torce, lanterne o sfere luminose che fluttuano nell'aria per la durata dell'incantesimo. Puoi anche combinare le quattro luci in un'unica forma luminosa vagamente umanoide di taglia Media. Qualsiasi forma scegli, ciascuna luce emette una luce fioca in un raggio di 3 metri. Come 1 Azione di Movimento durante il tuo round, puoi spostare le luci fino a 18 metri in un nuovo punto a gittata.

Una luce deve trovarsi entro 6 metri da un'altra luce creata con questo incantesimo, e le luci svaniscono se eccedono la gittata dell'incantesimo.

\textbf{Per ogni Successo Critico Magico} nella Prova di Magia la durata aumenta di 10 minuti o crei una nuova luce.

\smallskip\noindent\rule{\linewidth}{2pt} \index[Incantesimi]{Luminescenza}\hypertarget{Luminescenza}{}\smallskip\noindent{\textbf{Luminescenza}}\pdfbookmark[3]{Luminescenza}{Luminescenza}
\noindent
\begin{description}[noitemsep, topsep=0pt, parsep=0pt, partopsep=0pt, leftmargin=0cm, labelwidth=2.8cm]
	\item[\textbf{Lista di Magia}]: Invocazione
	\item[\textbf{Livello}]: 1, Non Comune
	\item[\textbf{T. di Lancio}]: 2 Azioni
	\item[\textbf{Gittata}]: 18 metri
	\item[\textbf{Componenti}]: V
	\item[\textbf{Durata}]: 1 minuto di tempo reale di gioco
\end{description}

Tutti gli oggetti in una sfera di 3 metri di raggio a gittata vengono circondati da una luce blu, verde o viola (a tua scelta). Qualsiasi creatura nell'area quando l'incantesimo viene lanciato, viene anch'essa circondata dalla luce se fallisce un Tiro Salvezza su Riflessi. Per la durata dell'incantesimo, gli oggetti e le creature soggette emettono una luce fioca con raggio di 3 metri. Qualsiasi Tiro per Colpire contro una creatura od oggetto soggetto ha +2 se l'attaccante può vederlo e la creatura od oggetto non può beneficiare dell'invisibilità.

\smallskip\noindent\rule{\linewidth}{2pt} \index[Incantesimi]{Onda rovente}\hypertarget{Onda rovente}{}\smallskip\noindent{\textbf{Onda rovente}}\pdfbookmark[3]{Onda rovente}{Onda rovente}
\noindent
\begin{description}[noitemsep, topsep=0pt, parsep=0pt, partopsep=0pt, leftmargin=0cm, labelwidth=2.8cm]
	\item[\textbf{Lista di Magia}]: Fuoco
	\item[\textbf{Livello}]: 1, Comune
	\item[\textbf{T. di Lancio}]: 2 Azioni
	\item[\textbf{Gittata}]: Personale (cono di 3 metri)
	\item[\textbf{Componenti}]: V, S
	\item[\textbf{Durata}]: Istantanea
\end{description}

Tieni le mani chiuse davanti a te, una potente onda rovente si genera da ogni tuo pugno. Ogni creatura in un cono di 3 metri deve effettuare un Tiro Salvezza su Riflessi. Una creatura subisce 1d4 di danno per Competenza Magica, fino ad un massimo di 5d4, danni da fuoco se fallisce il Tiro Salvezza, o la metà se lo supera. Il calore incendia gli oggetti infiammabili nell'area che non siano indossati o trasportati.

\textbf{Per ogni Successo Critico Magico} ottenuto nella Prova di Magia il danno aumenta di 2d4.

\textbf{Tiro Salvezza Successo/Fallimento Critico}: In caso di Fallimento Critico il danno raddoppia, in caso di Successo Critico il danno viene ulteriormente dimezzato

\smallskip\noindent\rule{\linewidth}{2pt} \index[Incantesimi]{Mano Arcana}\hypertarget{Mano Arcana}{}\smallskip\noindent{\textbf{Mano Arcana}}\pdfbookmark[3]{Mano Arcana}{Mano Arcana}
\noindent
\begin{description}[noitemsep, topsep=0pt, parsep=0pt, partopsep=0pt, leftmargin=0cm, labelwidth=2.8cm]
	\item[\textbf{Lista di Magia}]: Invocazione
	\item[\textbf{Livello}]: 5, Non Comune
	\item[\textbf{T. di Lancio}]: 2 Azioni
	\item[\textbf{Gittata}]: 36 metri
	\item[\textbf{Componenti}]: V, S, M (un guscio d'uovo e un guanto di pelle di serpente)
	\item[\textbf{Durata}]: Concentrazione, 1 minuto
\end{description}

Crei una mano Grande, composta di energia trasparente e luminosa, in uno spazio non occupato a gittata e che puoi vedere. La mano permane per la durata dell'incantesimo, e si muove al tuo comando, imitando i movimenti della tua mano.

La mano è un oggetto che ha Difesa 25 e Punti Ferita uguali ai tuoi Punti Ferita massimi. Ha Forza 4 e Destrezza 0. La mano non riempie il suo spazio.
Quando lanci l'incantesimo e come 2 Azioni durante i tuoi round successivi, puoi muovere la mano fino a 18 metri e poi generare uno dei seguenti effetti.

\medskip

- \emph{Mano Afferrante}. La mano cerca di afferrare una creatura di taglia Enorme o più piccola che si trovi entro 1 metro da essa. Per risolvere l'azione di afferrare usi la DC dell'incantesimo contro un Tiro Salvezza su Tempra con bonus Forza dell'avversario.
Chi ha una taglia maggiore guadagna un bonus di +1d6 per taglia differenza.
Mentre la mano tiene afferrato il bersaglio, puoi usare un'Azione per fare stritolare il bersaglio dalla mano. Quando lo fai, il bersaglio subisce danni contundenti pari a 2d6 + 1d6 per taglia di differenza + il tuo modificatore di caratteristica per incantesimo.

- \emph{Mano di Forza}. La mano cerca di spingere una creatura di 1 metro in una direzione a tua scelta. Per risolvere l'azione di afferrare usi la DC dell'incantesimo contro un Tiro Salvezza su Tempra con bonus Forza dell'avversario. Chi ha una taglia maggiore guadagna un bonus di +1d6 per taglia differenza.
Se vinci la contesa, la mano spinge il bersaglio di 1 metro più 1 metro per modificatore di caratteristica per incantesimi. La mano si muove assieme al bersaglio per restare entro 1 metro da lui.

- \emph{Mano Frapposta}. La mano si frappone tra di te e una creatura di tua scelta finché non le dai un comando diverso. La mano si muove di modo da restare tra di te e il bersaglio, fornendoti copertura media contro il bersaglio. Il bersaglio non può muoversi attraverso lo spazio della mano se il suo punteggio di Forza è uguale o inferiore al punteggio di Forza della mano. Se il suo punteggio di Forza è superiore al punteggio di Forza della mano, il bersaglio può muoversi attraverso lo spazio della mano, ma considera quello spazio come fosse terreno difficile.

- \emph{Pugno Serrato}. La mano colpisce una creatura o un oggetto entro 1 metro da essa. Effettua un attacco in mischia con incantesimo usando la mano. Se colpisci, il bersaglio subisce 4d8 danni da forza.

\textbf{Per ogni Successo Critico Magico} ottenuto nella Prova di Magia il danno dell'opzione pugno serrato aumenta di 2d8 e il danno dell'opzione mano afferrante aumenta di 2d6.

\smallskip\noindent\rule{\linewidth}{2pt} \index[Incantesimi]{Mano Magica}\hypertarget{Mano Magica}{}\smallskip\noindent{\textbf{Mano Magica}}\pdfbookmark[3]{Mano Magica}{Mano Magica}
\noindent
\begin{description}[noitemsep, topsep=0pt, parsep=0pt, partopsep=0pt, leftmargin=0cm, labelwidth=2.8cm]
	\item[\textbf{Lista di Magia}]: Evocazione
	\item[\textbf{Livello}]: 0, Comune
	\item[\textbf{T. di Lancio}]: 2 Azioni
	\item[\textbf{Gittata}]: 9 metri
	\item[\textbf{Componenti}]: V, S
	\item[\textbf{Durata}]: 1d4 round +1 per punto di Competenza Magica
\end{description}

Una mano spettrale fluttuante compare in un punto a gittata, scelto da te. La mano resta per la durata dell'incantesimo o finché non viene interrotta con un'Azione. La mano svanisce se si dovesse trovare a più di 9 metri da te o se lanci nuovamente l'incantesimo.

Le Azioni necessarie a muovere ed usare la mano magica sono le stesse che useresti per usare la tua mano. Puoi usare la mano per manipolare un oggetto, aprire una porta o un contenitore non chiusi a chiave, inserire o recuperare un oggetto da un contenitore aperto, o versare fuori i contenuti di una fiala. Puoi muovere la mano di 9 metri ogni volta che la usi. La mano non può attaccare, attivare oggetti magici o trasportare oggetti con Ingombro maggiore di 1.

\textbf{Per ogni Successo Critico Magico} ottenuto nella Prova di Magia l'Ingombro sollevato aumenta di 1 o raddoppi la durata.


\smallskip\noindent\rule{\linewidth}{2pt} \index[Incantesimi]{Marchio Magico}\hypertarget{Marchio Magico}{}\smallskip\noindent{\textbf{Marchio Magico}}\pdfbookmark[3]{Marchio Magico}{Marchio Magico}
\noindent
\begin{description}[noitemsep, topsep=0pt, parsep=0pt, partopsep=0pt, leftmargin=0cm, labelwidth=2.8cm]
	\item[\textbf{Lista di Magia}]: Universale
	\item[\textbf{Livello}]: 0, Comune
	\item[\textbf{T. di Lancio}]: 2 Azioni
	\item[\textbf{Gittata}]: Contatto
	\item[\textbf{Componenti}]: V, S
	\item[\textbf{Durata}]: Permanente
\end{description}

Questo incantesimo permette di iscrivere una personale runa o marchio su un oggetto. La scritta non può essere lunga più di 15 cm. La scritta può essere visibile od invisibile a seconda di come si decide al momento del lancio della magia.
Un incantesimo di Individuazione del Magico o Lettura del magico mostra la scritta se invisibile.
Se la scritta è posta su una creatura questa scompare nel giro di un mese.

\textbf{Per ogni Successo Critico Magico} ottenuto nella Prova di Magia scrivi un logo in più.

\smallskip\noindent\rule{\linewidth}{2pt} \index[Incantesimi]{Messaggio}\hypertarget{Messaggio}{}\smallskip\noindent{\textbf{Messaggio}}\pdfbookmark[3]{Messaggio}{Messaggio}
\noindent
\begin{description}[noitemsep, topsep=0pt, parsep=0pt, partopsep=0pt, leftmargin=0cm, labelwidth=2.8cm]
	\item[\textbf{Lista di Magia}]: Trasmutazione
	\item[\textbf{Livello}]: 0, Comune
	\item[\textbf{T. di Lancio}]: 2 Azioni
	\item[\textbf{Gittata}]: 36 metri
	\item[\textbf{Componenti}]: V, S, M (un piccolo pezzo di cavo di rame)
	\item[\textbf{Durata}]: 1 round
\end{description}

Punti il dito verso una creatura a gittata e sussurri un messaggio breve. Il bersaglio (e solo il bersaglio) ode il messaggio e può replicare con un sussurro che solo tu puoi udire.

Puoi lanciare questo incantesimo anche attraverso oggetti solidi, se sei familiare col bersaglio e sai che questi si trova dietro la barriera. Il silenzio magico, 30 centimetri di pietra, 2,5 centimetri di metallo normale, un sottile foglio di piombo o 1 metro di legno bloccano l'incantesimo. L'incantesimo non deve seguire una linea retta, e può liberamente aggirare gli angoli o attraversare gli spiragli.

\textbf{Per ogni Successo Critico Magico} ottenuto nella Prova di Magia l'incantesimo dura 1 round in più.

\smallskip\noindent\rule{\linewidth}{2pt} \index[Incantesimi]{Metamorfosi}\hypertarget{Metamorfosi}{}\smallskip\noindent{\textbf{Metamorfosi}}\pdfbookmark[3]{Metamorfosi}{Metamorfosi}
\noindent
\begin{description}[noitemsep, topsep=0pt, parsep=0pt, partopsep=0pt, leftmargin=0cm, labelwidth=2.8cm]
	\item[\textbf{Lista di Magia}]: Animali e Piante
	\item[\textbf{Livello}]: 4, Comune
	\item[\textbf{T. di Lancio}]: 2 Azioni
	\item[\textbf{Gittata}]: 18 metri
	\item[\textbf{Componenti}]: V, S, M (un bozzolo di bruco)
	\item[\textbf{Durata}]: 1 ora
\end{description}

Questo incantesimo trasforma una creatura a gittata, che puoi vedere, in una nuova forma. Una creatura non consenziente deve superare un Tiro Salvezza su Volontà per evitare l'effetto. I mutaforma superano automaticamente il Tiro Salvezza. L'incantesimo non ha effetto su di un bersaglio con 0 Punti Ferita.

La trasformazione permane per la durata dell'incantesimo o finché il bersaglio non scende a 0 Punti Ferita o muore. La nuova forma può essere quella di qualsiasi bestia il cui grado di sfida sia la metà del punteggio di Competenza Magica (o somma dei Tratti se Devoto di Shayalia) di chi lancia l'incantesimo. Le statistiche di gioco del bersaglio, compresi i punteggi delle caratteristiche mentali, vengono rimpiazzate dalle statistiche della bestia scelta. Egli mantiene però i suoi Tratti e personalità.

La creatura è limitata nelle azioni che può svolgere dalla natura della sua nuova forma, e non può dialogare, lanciare incantesimi, o effettuare qualsiasi altra azione che richieda mani o di parlare. L'equipaggiamento del bersaglio si fonde nella nuova forma. La creatura non può attivare, usare, impugnare o beneficiare in alcun modo del suo equipaggiamento.

\smallskip\noindent\rule{\linewidth}{2pt} \index[Incantesimi]{Metamorfosi Pura}\hypertarget{Metamorfosi Pura}{}\smallskip\noindent{\textbf{Metamorfosi Pura}}\pdfbookmark[3]{Metamorfosi Pura}{Metamorfosi Pura}
\noindent
\begin{description}[noitemsep, topsep=0pt, parsep=0pt, partopsep=0pt, leftmargin=0cm, labelwidth=2.8cm]
	\item[\textbf{Lista di Magia}]: Animali e Piante
	\item[\textbf{Livello}]: 9, Raro
	\item[\textbf{T. di Lancio}]: 2 Azioni
	\item[\textbf{Gittata}]: 9 metri
	\item[\textbf{Componenti}]: V, S, M (un goccio di mercurio, un mucchietto di gomma arabica, e uno sbuffo di fumo)
	\item[\textbf{Durata}]: 1 ora
\end{description}

Scegli una creatura od oggetto non magico a gittata e che puoi vedere. L'incantesimo non ha effetto su di un bersaglio con 0 Punti Ferita. Trasformi la creatura in una creatura diversa, la creatura in un oggetto, o l'oggetto in una creatura (l'oggetto non deve essere indossato né trasportato da un'altra creatura). La trasformazione permane per la durata dell'incantesimo o finché il bersaglio non scende a 0 Punti Ferita o muore. Se ti concentri su questo incantesimo per l'intera durata, la trasformazione diventa permanente.

I mutaforma ignorano questo incantesimo. Una creatura non consenziente può effettuare un Tiro Salvezza su Volontà e, se lo supera, ignora l'effetto di questo incantesimo.

\begin{itemize}[leftmargin=*] \setlength{\itemsep}{0pt}
	\item \emph{Creatura in Creatura}. Se trasformi una creatura in un'altra specie di creatura, la nuova forma può essere quella di qualsiasi specie tu voglia, il cui grado di sfida sia pari o inferiore al tuo punteggio di Competenza Magica (o somma Tratti in comune se Devoto di Shayalia). Le statistiche di gioco del bersaglio, compresi i punteggi delle caratteristiche mentali, vengono rimpiazzate dalle statistiche della nuova forma. Egli mantiene però il suoi Tratti e personalità.

	Il bersaglio mantiene i medesimi Punti Ferita e ne recupera 1d12 Punti Ferita nella sua nuova forma. Quando ritorna alla sua forma normale, la creatura mantiene i Punti Ferita che ha attualmente. Se arriva a 0, o meno, Punti Ferita nella nuova forma allora torna normale e qualsiasi effetto si ripercuote anche nella forma corrente. La creatura è limitata nelle azioni che può svolgere dalla natura della sua nuova forma, e non può dialogare, lanciare incantesimi, o effettuare qualsiasi altra azione che richieda mani o di parlare, a meno che la nuova forma non sia capace di svolgere queste azioni. L'equipaggiamento del bersaglio si fonde nella nuova forma. La creatura non può attivare, usare, impugnare o beneficiare in alcun modo del suo equipaggiamento.

	\item \emph{Oggetto in Creatura.} Puoi trasformare un oggetto in un qualsiasi tipo di creatura, purché la taglia della creatura non sia maggiore della taglia dell'oggetto e il grado di sfida della creatura sia 9 o meno. La creatura è amichevole verso di te e i tuoi compagni. Essa agisce durante i tuoi round. Decidi tu quali azioni essa compirà e come si muove. Il Narratore possiede le statistiche della creatura e risolverà tutte le sue azioni e i suoi movimenti.
	Se l'incantesimo diventa permanente, perdi il controllo della creatura. A seconda di come l'hai trattata, potrebbe restare amichevole nei tuoi confronti.

	\item \emph{Creatura in Oggetto}. Se trasformi una creatura in un oggetto, essa si trasforma assieme a qualsiasi cosa stia indossando o trasportando. Le statistiche della creatura diventano quelle dell'oggetto, e, dopo che l'incantesimo termina e la creatura ritorna alla sua forma normale, questa non ha più ricordi del tempo trascorso in forma di oggetto.

\end{itemize}

\smallskip\noindent\rule{\linewidth}{2pt} \index[Incantesimi]{Miraggio Arcano}\hypertarget{Miraggio Arcano}{}\smallskip\noindent{\textbf{Miraggio Arcano}}\pdfbookmark[3]{Miraggio Arcano}{Miraggio Arcano}
\noindent
\begin{description}[noitemsep, topsep=0pt, parsep=0pt, partopsep=0pt, leftmargin=0cm, labelwidth=2.8cm]
	\item[\textbf{Lista di Magia}]: Illusione
	\item[\textbf{Livello}]: 7, Raro
	\item[\textbf{T. di Lancio}]: 10 minuti
	\item[\textbf{Gittata}]: Vista
	\item[\textbf{Componenti}]: V, S
	\item[\textbf{Durata}]: 10 giorni
\end{description}

Fai sì che un pezzo di terreno a gittata, in un'area quadrata fino a 1,5 chilometri, appaia, risuoni e odori come qualche altro tipo di terreno. La conformazione generale del terreno rimane tuttavia la stessa. Campi aperti o una strada possono essere trasformati in un acquitrino, colline, un crepaccio o qualche altro tipo di terreno difficile o invalicabile. Un laghetto può essere trasformato in una radura erbosa, un precipizio in una lieve pendenza, un burrone cosparso di rocce in una strada ampia e liscia.

Allo stesso modo, puoi modificare l'aspetto delle strutture, o aggiungerne dove non ve ne sono. L'incantesimo non camuffa, occulta né aggiunge creature.

L'illusione comprende elementi uditivi, visivi, tattili e olfattivi, così da poter trasformare un terreno sgombro in terreno difficile (o viceversa) o impedire altrimenti il movimento nell'area. Qualsiasi pezzo di terreno illusorio (come una pietra o un bastone), che venga rimosso dall'area dell'incantesimo, svanisce immediatamente. Le creature con visione del vero possono vedere oltre l'illusione e distinguere la vera forma del terreno; tuttavia, gli altri elementi dell'illusione rimangono, così, sebbene la creatura sia consapevole della presenza dell'illusione, vi può comunque interagire fisicamente.

\textbf{Con tre Successo Critico Magico ottenuto} nella Prova di Magia la durata è permanente.

\smallskip\noindent\rule{\linewidth}{2pt} \index[Incantesimi]{Modificare Memoria}\hypertarget{Modificare Memoria}{}\smallskip\noindent{\textbf{Modificare Memoria}}\pdfbookmark[3]{Modificare Memoria}{Modificare Memoria}
\noindent
\begin{description}[noitemsep, topsep=0pt, parsep=0pt, partopsep=0pt, leftmargin=0cm, labelwidth=2.8cm]
	\item[\textbf{Lista di Magia}]: Ammaliamento
	\item[\textbf{Livello}]: 5, Molto Raro
	\item[\textbf{T. di Lancio}]: 3 Azioni
	\item[\textbf{Gittata}]: 9 metri
	\item[\textbf{Componenti}]: V, S
	\item[\textbf{Durata}]: Istantanea
\end{description}

Tenti di rimodellare i ricordi di un'altra creatura. Una creatura che puoi vedere deve effettuare un Tiro Salvezza su Volontà. Se la stai combattendo, la creatura ha +1d6 sul Tiro Salvezza. Se fallisce il Tiro Salvezza puoi agire sui ricordi del bersaglio in merito a un evento che abbia vissuto nelle ultime 24 ore e che non sia durato più di 10 minuti.

Puoi eliminare permanentemente tutti i ricordi dell'evento, permettere al bersaglio di ricordare l'evento con perfetta chiarezza e dettagli particolareggiati, modificare il ricordo dei dettagli dell'evento, o creare il ricordo di un altro evento. Devi poter parlare al bersaglio per descrivere il modo in cui i suoi ricordi saranno colpiti, e questi deve essere in grado di comprendere il tuo linguaggio, affinché i ricordi modificati si instaurino nella sua memoria.
I ricordi modificati si instaurano al termine dell'incantesimo.

Una memoria modificata non influisce necessariamente sul comportamento della creatura, in particolare se i suoi ricordi contraddicono le inclinazioni naturali, i Tratti o la fede della creatura. Una memoria modificata in modo illogico, come impiantare il ricordo di quanto la creatura ami immergersi nell'acido, viene rimossa, come fosse un brutto sogno.

Il Narratore può giudicare un ricordo modificato troppo insensato perché abbia alcun effetto su di una creatura. Un incantesimo rimuovi maledizione o ristorare superiore lanciato sul bersaglio ne ripristina i veri ricordi.

\textbf{Con un Successo Critico Magico} ottenuto nella Prova di Magia puoi alterare i ricordi di un bersaglio riguardo un evento svoltosi fino a 7 giorni prima. Con due fino a 30 giorni prima, con tre fino ad 1 anno prima. Con 4 successi Critici Magici in qualsiasi punto nel passato della creatura.

\smallskip\noindent\rule{\linewidth}{2pt} \index[Incantesimi]{Morte Apparente}\hypertarget{Morte Apparente}{}\smallskip\noindent{\textbf{Morte Apparente}}\pdfbookmark[3]{Morte Apparente}{Morte Apparente}
\noindent
\begin{description}[noitemsep, topsep=0pt, parsep=0pt, partopsep=0pt, leftmargin=0cm, labelwidth=2.8cm]
	\item[\textbf{Lista di Magia}]: Necromanzia
	\item[\textbf{Livello}]: 3, Non Comune
	\item[\textbf{T. di Lancio}]: 1 Reazione
	\item[\textbf{Gittata}]: 18 metri
	\item[\textbf{Componenti}]: V
	\item[\textbf{Durata}]: 6 round più 1 round per CM
\end{description}

L'incantatore induce in sé stesso, o in una creatura consenziente, uno stato di paralisi totale che sembra identico alla morte, anche in caso di esame approfondito. Una creatura influenzata sente suoni e odori, ma non si può muovere ed è completamente priva di percezioni tattili; se il corpo viene danneggiato, non percepirà alcun fastidio, né avrà alcuna reazione fisica. Tutto il danno inflitto a una creatura in questo stato viene ridotto della metà; i veleni e gli attacchi che paralizzano o prosciugano la vita non fanno effetto prima della scadenza di questo incantesimo. È necessario un round, terminato l'incantesimo, prima che il corpo possa riprendere le sue normali funzioni vitali.

\textbf{Per ogni Successo Critico Magico} ottenuto nella Prova di Magia raddoppi la durata o influenzi un altra creatura.

\smallskip\noindent\rule{\linewidth}{2pt} \index[Incantesimi]{Movimenti del Ragno}\hypertarget{Movimenti del Ragno}{}\smallskip\noindent{\textbf{Movimenti del Ragno}}\pdfbookmark[3]{Movimenti del Ragno}{Movimenti del Ragno}
\noindent
\begin{description}[noitemsep, topsep=0pt, parsep=0pt, partopsep=0pt, leftmargin=0cm, labelwidth=2.8cm]
	\item[\textbf{Lista di Magia}]: Trasmutazione
	\item[\textbf{Livello}]: 2, Non Comune
	\item[\textbf{T. di Lancio}]: 2 Azioni
	\item[\textbf{Gittata}]: Contatto
	\item[\textbf{Componenti}]: V, S, M (una goccia di bitume e un ragno)
	\item[\textbf{Durata}]: 10 minuti
\end{description}

Lanci l'incantesimo a contatto di una creatura consenziente. Fino al termine dell'incantesimo, la creatura ottiene la capacità di spostarsi verso l'alto, il basso e lungo superfici verticali o stando a testa in giù sul soffitto, tenendo le mani libere. Il bersaglio ottiene anche velocità di scalata pari alla sua velocità di movimento. La creatura soggetta all'incantesimo si considera Distratta nel lancio di altri incantesimi.

\smallskip\noindent\rule{\linewidth}{2pt} \index[Incantesimi]{Muovere il Terreno}\hypertarget{Muovere il Terreno}{}\smallskip\noindent{\textbf{Muovere il Terreno}}\pdfbookmark[3]{Muovere il Terreno}{Muovere il Terreno}
\noindent
\begin{description}[noitemsep, topsep=0pt, parsep=0pt, partopsep=0pt, leftmargin=0cm, labelwidth=2.8cm]
	\item[\textbf{Lista di Magia}]: Terra
	\item[\textbf{Livello}]: 6, Non Comune
	\item[\textbf{T. di Lancio}]: 2 Azioni
	\item[\textbf{Gittata}]: 36 metri
	\item[\textbf{Componenti}]: V, S, M (un badile di ferro e un piccola borsa contenente un misto di tipi di terreno - argilla, concime e sabbia)
	\item[\textbf{Durata}]: Concentrazione, massimo 2 ore
\end{description}

Scegli un'area sul terreno a gittata, non più grande di 12 metri di lato. Per la durata, puoi rimodellare terriccio, sabbia o argilla nell'area in qualsiasi modo tu voglia. Puoi innalzare o abbassare l'altitudine dell'area, creare o riempire un fossato, erigere o abbassare un muro, o formare un pilastro. La portata di questi cambiamenti non può eccedere metà della dimensione più grossa dell'area. Così, se operi su di un quadrato di 12 metri di lato, puoi creare un pilastro alto 6 metri, innalzare o abbassare l'altitudine del terreno di 6 metri, scavare un fossato profondo 6 metri, e così via. Ci vogliono 10 minuti per completare questi mutamenti. Al termine di ogni 10minuti trascorsi a concentrarsi sull'incantesimo, puoi scegliere una nuova area di terreno su cui operare.

Dato che la trasformazione del terreno avviene lentamente, le creature nell'area di solito non possono restare intrappolate o ferite dal movimento del terreno. L'incantesimo non può manipolare la pietra naturale o le costruzioni in pietra. Le rocce e le strutture si muovono per adattarsi al nuovo terreno. Se il modo in cui modelli il terreno renderebbe una struttura instabile, questa potrebbe crollare. Allo stesso modo, questo incantesimo non influenza direttamente la crescita dei vegetali. La terra smossa trasporta con sé qualsiasi vegetale presente.

\smallskip\noindent\rule{\linewidth}{2pt} \index[Incantesimi]{Muro di Forza}\hypertarget{Muro di Forza}{}\smallskip\noindent{\textbf{Muro di Forza}}\pdfbookmark[3]{Muro di Forza}{Muro di Forza}
\noindent
\begin{description}[noitemsep, topsep=0pt, parsep=0pt, partopsep=0pt, leftmargin=0cm, labelwidth=2.8cm]
	\item[\textbf{Lista di Magia}]: Invocazione
	\item[\textbf{Livello}]: 5, Comune
	\item[\textbf{T. di Lancio}]: 2 Azioni
	\item[\textbf{Gittata}]: 36 metri
	\item[\textbf{Componenti}]: V, S, M (un pizzico di polvere prodotta frantumando una gemma trasparente)
	\item[\textbf{Durata}]: 10 minuti
\end{description}

Un invisibile muro di forza si forma in un punto a gittata scelto da te. Il muro appare in qualsiasi orientamento da te desiderato, come una barriera orizzontale o verticale oppure angolata. Può fluttuare nell'aria o appoggiarsi su di una superficie solida. Puoi darle la forma di una cupola semisferica o di una sfera con un raggio massimo di 3 metri, oppure darle l'aspetto di una superficie piana composta da un massimo di dieci pannelli di 3 metri per 3 metri. Ogni pannello deve essere contiguo a un altro pannello. In qualsiasi forma, il muro ha uno spessore di 75 centimetri e resta per tutta la durata dell'incantesimo. Se il muro taglia uno spazio di una creatura, quando compare, la creatura viene spinta da un lato del muro (a tua discrezione). Nulla può attraversare fisicamente il muro, chiunque al di là del muro ha copertura completa. È immune a tutti i danni e non può essere dissolto da dissolvi magie. Tuttavia, il muro è distrutto all'istante dall'incantesimo disintegrazione. Il muro si estende anche sul Piano Etereo, impedendo ai viaggiatori eterei di attraversarlo.

\smallskip\noindent\rule{\linewidth}{2pt} \index[Incantesimi]{Muro di Fuoco}\hypertarget{Muro di Fuoco}{}\smallskip\noindent{\textbf{Muro di Fuoco}}\pdfbookmark[3]{Muro di Fuoco}{Muro di Fuoco}
\noindent
\begin{description}[noitemsep, topsep=0pt, parsep=0pt, partopsep=0pt, leftmargin=0cm, labelwidth=2.8cm]
	\item[\textbf{Lista di Magia}]: Fuoco
	\item[\textbf{Livello}]: 4, Comune
	\item[\textbf{T. di Lancio}]: 2 Azioni
	\item[\textbf{Gittata}]: 36 metri
	\item[\textbf{Componenti}]: V, S, M (un piccolo pezzo di fosforo)
	\item[\textbf{Durata}]: 1 minuto
\end{description}

Crei un muro di fuoco su di una superficie solida a gittata. Puoi creare un muro lungo fino a 18 metri, alto fino a 6 metri e spesso 30 centimetri, o un muro circolare di 6 metri di diametro, 6 metri di altezza e 30 centimetri di spessore. Il muro è opaco e rimane per la durata dell'incantesimo.

Quando il muro appare, ogni creatura nella sua area deve effettuare un Tiro Salvezza su Riflessi. Una creatura subisce 5d8 danni da fuoco se fallisce il Tiro Salvezza, o la metà se lo supera. Un lato del muro, selezionato da te quando lanci questo incantesimo, infligge 5d8 danni da fuoco a ciascuna creatura che termini il suo round entro 3 metri da quel lato o all'interno del muro. Una creatura subisce lo stesso danno quando entra nel muro per la prima volta durante un round. L'altro lato del muro non infligge danni.

\textbf{Per ogni Successo Critico Magico} ottenuto nella Prova di Magia il danno aumenta di 3d6.

\smallskip\noindent\rule{\linewidth}{2pt} \index[Incantesimi]{Muro di Ghiaccio}\hypertarget{Muro di Ghiaccio}{}\smallskip\noindent{\textbf{Muro di Ghiaccio}}\pdfbookmark[3]{Muro di Ghiaccio}{Muro di Ghiaccio}
\noindent
\begin{description}[noitemsep, topsep=0pt, parsep=0pt, partopsep=0pt, leftmargin=0cm, labelwidth=2.8cm]
	\item[\textbf{Lista di Magia}]: Acqua
	\item[\textbf{Livello}]: 6, Comune
	\item[\textbf{T. di Lancio}]: 2 Azioni
	\item[\textbf{Gittata}]: 36 metri
	\item[\textbf{Componenti}]: V, S, M (un piccolo pezzo di quarzo)
	\item[\textbf{Durata}]: 10 minuti
\end{description}

Crei un muro di ghiaccio su di una superficie solida a gittata. Puoi creare una cupola semisferica o una sfera con un raggio massimo di 3 metri, o puoi creare una superficie piana composta di un massimo di dieci panelli quadrati di 3 metri di lato. Ogni pannello deve essere contiguo ad almeno un altro pannello. In ogni forma, il muro è spesso 30 centimetri e rimane per la durata dell'incantesimo.

Se, quando compare, il muro attraversa lo spazio di una creatura, la creatura viene spinta da una parte del muro (a tua scelta) e deve effettuare un Tiro Salvezza su Riflessi. Se fallisce il Tiro Salvezza, la creatura subisce 10d6 danni da freddo, o la metà di questi danni se lo supera.

Il muro è un oggetto che può essere danneggiato e sfondato. Ogni sezione di 3 metri ha Difesa 12 e 30 Punti Ferita, ed è vulnerabile al danno da fuoco. Ridurre una sezione di 3 metri a 0 Punti Ferita la distrugge e lascia nello spazio che era occupato dal muro una brezza di vento gelido. Una creatura che si muova attraverso questa brezza di vento gelido per la prima volta in un round, deve effettuare un Tiro Salvezza su Tempra. Se lo fallisce, la creatura subisce 5d6 danni da freddo, o la metà di questi danni se lo supera.

\textbf{Per ogni Successo Critico Magico} ottenuto nella Prova di Magia il danno aumentano di 5d6.

\smallskip\noindent\rule{\linewidth}{2pt} \index[Incantesimi]{Muro di Pietra}\hypertarget{Muro di Pietra}{}\smallskip\noindent{\textbf{Muro di Pietra}}\pdfbookmark[3]{Muro di Pietra}{Muro di Pietra}
\noindent
\begin{description}[noitemsep, topsep=0pt, parsep=0pt, partopsep=0pt, leftmargin=0cm, labelwidth=2.8cm]
	\item[\textbf{Lista di Magia}]: Invocazione
	\item[\textbf{Livello}]: 5, Comune
	\item[\textbf{T. di Lancio}]: 2 Azioni
	\item[\textbf{Gittata}]: 36 metri
	\item[\textbf{Componenti}]: V, S, M (un piccolo blocco di granito)
	\item[\textbf{Durata}]: 10 minuti
\end{description}

Un muro di pietra solida non magico si forma in un punto a gittata, scelto da te. Il muro è spesso 15 centimetri ed è composto da 10 pannelli di 3 per 3 metri. Ogni pannello deve essere contiguo ad almeno un altro pannello. In alternativa, puoi creare pannelli 3 x 6 metri di soli 7,5 centimetri di spessore.

Se, quando compare, il muro attraversa lo spazio di una creatura, la creatura viene spinta da una parte del muro (a tua scelta). Se la creatura fosse circondata da tutte le parti dal muro (o dal muro e un'altra superficie solida), la creatura può effettuare un Tiro Salvezza su Riflessi. Se lo supera, può usare una Azione di Reazione per muoversi della sua velocità in modo da non essere più intrappolata nel muro.

Il muro può aver qualsiasi forma tu desideri, sebbene non possa occupare lo stesso spazio di una creatura od oggetto. Il muro può anche non essere verticale o poggiare su di un piano. Deve, tuttavia, fondersi con ed essere sostenuto da pietra già esistente. Quindi, puoi usare questo incantesimo per creare un ponte su di un baratro o creare un rampa.

Se crei un muro non verticale del genere, più lungo di 6 metri, devi dimezzare le dimensioni di ciascun pannello per creare dei supporti. Puoi modellare rozzamente la pietra per creare merlature, spalti e così via. Il muro è un oggetto fatto di pietra che può essere danneggiato e sfondato. Ogni pannello ha Difesa 15, Durezza 15 e 15 Punti Ferita ogni 2,5 centimetri di spessore. Ridurre un pannello a 0 Punti Ferita lo distrugge e potrebbe far crollare i pannelli connessi, a discrezione del Narratore. Se mantieni la concentrazione su questo incantesimo per la sua intera durata, il muro diventa permanente e non può essere dissolto. Altrimenti, il muro sparisce al termine dell'incantesimo.

\smallskip\noindent\rule{\linewidth}{2pt} \index[Incantesimi]{Muro Prismatico}\hypertarget{Muro Prismatico}{}\smallskip\noindent{\textbf{Muro Prismatico}}\pdfbookmark[3]{Muro Prismatico}{Muro Prismatico}
\noindent
\begin{description}[noitemsep, topsep=0pt, parsep=0pt, partopsep=0pt, leftmargin=0cm, labelwidth=2.8cm]
	\item[\textbf{Lista di Magia}]: Abiurazione
	\item[\textbf{Livello}]: 9, Raro
	\item[\textbf{T. di Lancio}]: 2 Azioni
	\item[\textbf{Gittata}]: 18 metri
	\item[\textbf{Componenti}]: V, S
	\item[\textbf{Durata}]: 10 minuti
\end{description}

Un piano di luci brillanti e multicolore forma un muro verticale opaco, largo fino a 27 metri, alto 9 metri e spesso 2,5 centimetri, centrato su di un punto a gittata e che puoi vedere. In alternativa, puoi modellare il muro in una sfera, fino a 9 metri di diametro, centrata su di un punto a gittata di tua scelta. Il muro resta fisso sul posto per la durata dell'incantesimo. Se posizioni il muro in modo che attraversi lo spazio occupato da una creatura, l'incantesimo fallisce e lo slot incantesimo sono sprecati. Il muro irradia luce intensa fino a una gittata di 18 metri e luce fioca per 36 metri. Tu e le creature indicate da te al momento del lancio dell'incantesimo potete attraversare e restare vicini al muro senza pericolo. Se un'altra creatura che può vedere il muro si muove entro 6 metri da esso o inizia lì il suo round, deve superare un Tiro Salvezza su Tempra o restare accecata per 1 minuto. Il muro consiste di sette strati, ognuno di un diverso colore. Quando una creatura cerca di immergersi o attraversare il muro, lo fa uno strato alla volta, attraverso tutti gli strati del muro. Mentre si immerge o attraversa ciascuno strato, la creatura deve superare un Tiro Salvezza su Riflessi o subire le proprietà di ciascuno strato, uno alla volta, come descritto di seguito.

Il muro può essere distrutto, uno strato alla volta, in ordine dal rosso al violetto, in un modo specifico per ogni strato. Una volta che uno strato è distrutto, lo sarà per la durata dell'incantesimo. Una verga di cancellazione distrugge un Muro Prismatico, ma un campo anti-magia non ha effetto su di esso.

\begin{itemize}[leftmargin=*] \setlength{\itemsep}{0pt}
	\item \emph{1. Rosso}. Il bersaglio subisce 10d6 danni da fuoco se fallisce il Tiro Salvezza, o la metà di questi danni se lo supera. Finché questo strato esiste, gli attacchi a distanza non magici non possono attraversare il muro. Lo strato può essere distrutto infliggendogli 25 danni da freddo.
	\item \emph{2. Arancio}. Il bersaglio subisce 10d6 danni da acido se fallisce il Tiro Salvezza, o la metà di questi danni se lo supera. Finché questo strato esiste, gli attacchi a distanza magici non possono attraversare il muro. Lo strato può essere distrutto da un forte vento. 3. Giallo. Il bersaglio subisce 10d6 danni da elettricità se fallisce il Tiro Salvezza, o la metà di questi danni se lo supera. Questo strato può essere distrutto infliggendogli 60 danni di forza.
	\item \emph{4. Verde}. Il bersaglio subisce 10d6 danni da veleno se fallisce il Tiro Salvezza, o la metà di questi danni se lo supera. Un incantesimo Passa Porta, o un altro incantesimo di pari livello o più alto che può aprire un portale su di una superficie solida, distrugge questo strato.
	\item \emph{5. Blu}. Il bersaglio subisce 10d6 danni da freddo se fallisce il Tiro Salvezza, o la metà di questi danni se lo supera. Lo strato può essere distrutto infliggendogli almeno 25 danni da fuoco.
	\item \emph{6. Indaco}. Se fallisce il Tiro Salvezza, il bersaglio è intralciato. Deve poi effettuare un Tiro Salvezza su Tempra all'inizio di ciascun suo round. Se supera il Tiro Salvezza tre volte,l'incantesimo termina. Se fallisce il Tiro Salvezza tre volte, viene permanentemente trasformato in pietra e diventa vittima della condizione pietrificato. I successi e i fallimenti non devono essere consecutivi; tieni traccia di entrambi finché il bersaglio non ne ha ottenuti tre dello stesso tipo. Finché questo strato esiste, non si possono lanciare incantesimi attraverso il muro. Lo strato viene distrutto dalla luce intensa emanata dall'incantesimo luce diurna o da un simile incantesimo di livello più alto.
	\item \emph{7. Violetto}. Se fallisce il Tiro Salvezza, il bersaglio è accecato. Deve poi effettuare un Tiro Salvezza su Volontà all'inizio del tuo prossimo round. Se supera il Tiro Salvezza, la cecità termina. Se fallisce il Tiro Salvezza, la creatura viene trasportata su di un altro piano di esistenza a scelta del Narratore e non è più accecata (di solito, una creatura che non è sul suo piano natio, viene esiliata su di esso, mentre le altre creature sono di solito gettate nei piani Astrale o Etereo). Questo strato è distrutto dall'incantesimo dissolvi magie o da un incantesimo simile di pari livello o più alto che possa porre fine a incantesimi ed effetti magici.

\end{itemize}

\smallskip\noindent\rule{\linewidth}{2pt} \index[Incantesimi]{Muro di Spine}\hypertarget{Muro di Spine}{}\smallskip\noindent{\textbf{Muro di Spine}}\pdfbookmark[3]{Muro di Spine}{Muro di Spine}
\noindent
\begin{description}[noitemsep, topsep=0pt, parsep=0pt, partopsep=0pt, leftmargin=0cm, labelwidth=2.8cm]
	\item[\textbf{Lista di Magia}]: Animali e Piante
	\item[\textbf{Livello}]: 6, Non Comune
	\item[\textbf{T. di Lancio}]: 2 Azioni
	\item[\textbf{Gittata}]: 36 metri
	\item[\textbf{Componenti}]: V, S, M (una manciata di spine)
	\item[\textbf{Durata}]: massimo 10 minuti
\end{description}

Crei un muro di cespugli robusti, malleabili e impigliati, ricolmi di spine appuntite. Il muro compare a gittata su di una superficie solida e rimane per la durata dell'incantesimo. Il muro che puoi creare può essere lungo fino a 18 metri, alto fino a 3 metri, e spesso fino a 1 metro o un circolo che abbia un diametro di 6 metri e sia alto fino a 6 metri e spesso 1 metro. Il muro blocca la linea di visuale.

Quando il muro compare, ogni creatura nella sua area deve effettuare un Tiro Salvezza su Riflessi. Se fallisce il Tiro Salvezza, una creatura subisce 7d8 danni perforanti, o la metà di questi danni se lo supera. Una creatura può muoversi attraverso il muro, seppure in maniera lenta e dolorosa. Il terreno coperto dal muro si considera doppiamente difficile. Inoltre, la prima volta che una creatura entra nel muro durante un round o vi termina il suo round dentro, la creatura deve effettuare un Tiro Salvezza su Riflessi. Subisce 7d8 danni taglienti se fallisce il Tiro Salvezza, o la metà di questi danni se lo supera.

\textbf{Per ogni Successo Critico Magico} ottenuto nella Prova di Magia il danno aumenta di 3d8.

\smallskip\noindent\rule{\linewidth}{2pt} \index[Incantesimi]{Muro di Vento}\hypertarget{Muro di Vento}{}\smallskip\noindent{\textbf{Muro di Vento}}\pdfbookmark[3]{Muro di Vento}{Muro di Vento}
\noindent
\begin{description}[noitemsep, topsep=0pt, parsep=0pt, partopsep=0pt, leftmargin=0cm, labelwidth=2.8cm]
	\item[\textbf{Lista di Magia}]: Aria
	\item[\textbf{Livello}]: 3, Non Comune
	\item[\textbf{T. di Lancio}]: 2 Azioni
	\item[\textbf{Gittata}]: 36 metri
	\item[\textbf{Componenti}]: V, S, M (un minuscolo ventaglio e una piuma di origini esotiche)
	\item[\textbf{Durata}]: 1 minuto
\end{description}

Un muro di forte vento si leva dal terreno in un punto a gittata di tua scelta. Puoi creare un muro lungo fino a 15 metri, alto 3 metri e spesso 30 centimetri. Puoi modellare il muro in qualsiasi maniera desideri purché componga un percorso continuo sul terreno. Il muro rimane per la durata dell'incantesimo. Quando il muro appare, ogni creatura all'interno della sua area deve effettuare un Tiro Salvezza su Tempra. Una creatura subisce 3d8 danni contundenti se fallisce il Tiro Salvezza, o la metà di questi danni se lo supera. Il forte vento tiene lontana foschia, fumo e altri gas. Le creature volanti di taglia Piccola o minore non possono attraversare il muro. I materiali leggeri trascinati nel muro volano verso l'alto. Frecce, quadrelli e altre munizioni normali vengono deviati e mancano automaticamente il bersaglio (i macigni scagliati dai giganti e dalle macchine d'assedio, e munizioni simili, ne ignorano invece gli effetti). Le creature in forma gassosa non possono attraversarlo.

\textbf{Per ogni Successo Critico Magico} ottenuto nella Prova di Magia la durata aumenta di 1 minuto.

\smallskip\noindent\rule{\linewidth}{2pt} \index[Incantesimi]{Nube Incendiaria}\hypertarget{Nube Incendiaria}{}\smallskip\noindent{\textbf{Nube Incendiaria}}\pdfbookmark[3]{Nube Incendiaria}{Nube Incendiaria}
\noindent
\begin{description}[noitemsep, topsep=0pt, parsep=0pt, partopsep=0pt, leftmargin=0cm, labelwidth=2.8cm]
	\item[\textbf{Lista di Magia}]: Fuoco
	\item[\textbf{Livello}]: 8, Raro
	\item[\textbf{T. di Lancio}]: 2 Azioni
	\item[\textbf{Gittata}]: 45 metri
	\item[\textbf{Componenti}]: V, S
	\item[\textbf{Durata}]: 1 minuto
\end{description}

Una nube di fumo turbinante attraversata da lapilli incandescenti si forma in una sfera di 6 metri di raggio centrata su di un punto a gittata. La nube si propaga intorno agli angoli ed è in penombra. Rimane per la durata dell'incantesimo o finché un vento di velocità moderata o superiore (almeno 15 chilometri all'ora) non la disperde.

Quando la nube appare, ogni creatura al suo interno deve effettuare un Tiro Salvezza su Riflessi. Una creatura subisce 10d8 danni da fuoco se fallisce il Tiro Salvezza, e la metà di questi danni se lo supera. Una creatura deve effettuare il Tiro Salvezza anche quando entra per la prima volta nell'area o termina lì il suo round.

All'inizio di ciascun tuo round, la nube si muove di 3 metri lontano da te in una direzione a tua scelta.

\smallskip\noindent\rule{\linewidth}{2pt} \index[Incantesimi]{Nebbia Nauseante}\hypertarget{Nebbia Nauseante}{}\smallskip\noindent{\textbf{Nebbia Nauseante}}\pdfbookmark[3]{Nebbia Nauseante}{Nebbia Nauseante}
\noindent
\begin{description}[noitemsep, topsep=0pt, parsep=0pt, partopsep=0pt, leftmargin=0cm, labelwidth=2.8cm]
	\item[\textbf{Lista di Magia}]: Acqua, Aria
	\item[\textbf{Livello}]: 3, Non Comune
	\item[\textbf{T. di Lancio}]: 2 Azioni
	\item[\textbf{Gittata}]: 27 metri
	\item[\textbf{Componenti}]: V, S, M (un uovo marcio o foglie di cavolo puzzolente)
	\item[\textbf{Durata}]: 10 minuti
\end{description}

Crei, in un punto a gittata, una sfera di 6 metri di raggio composta di un gas giallo e nauseabondo. La nebbia si propaga dietro gli angoli e la sua area è in penombra. La nebbia permane nell'aria per la durata. Ogni creatura che si trovi completamente all'interno della nebbia all'inizio del proprio round, deve effettuare un Tiro Salvezza su Tempra contro il veleno. Se il Tiro Salvezza fallisce, la creatura spende 2 Azioni di quel round a vomitare e barcollare. Le creature che non hanno bisogno di respirare o che sono immuni al veleno superano automaticamente il Tiro Salvezza. Un vento moderato (almeno 15 chilometri all'ora) disperde la nebbia dopo 4 round. Un vento forte (almeno 30 chilometri all'ora) lo disperde dopo 1 round.

\smallskip\noindent\rule{\linewidth}{2pt} \index[Incantesimi]{Nebbia mortale}\hypertarget{Nebbia mortale}{}\smallskip\noindent{\textbf{Nebbia mortale}}\pdfbookmark[3]{Nebbia mortale}{Nebbia mortale}
\noindent
\begin{description}[noitemsep, topsep=0pt, parsep=0pt, partopsep=0pt, leftmargin=0cm, labelwidth=2.8cm]
	\item[\textbf{Lista di Magia}]: Acqua
	\item[\textbf{Livello}]: 5, Raro
	\item[\textbf{T. di Lancio}]: 2 Azioni
	\item[\textbf{Gittata}]: 36 metri
	\item[\textbf{Componenti}]: V, S
	\item[\textbf{Durata}]: 10 minuti
\end{description}

Crei una sfera di 6 metri di raggio formata da una nebbia velenosa giallo-verde centrata in un punto a gittata di tua scelta. La nebbia si propaga dietro gli angoli. Rimane per la durata dell'incantesimo o finché un forte vento non disperde la nebbia, terminando l'incantesimo. La sua area è in penombra. Quando una creatura entra nell'area dell'incantesimo per la prima volta in un round o inizia lì il suo round, quella creatura deve effettuare un Tiro Salvezza su Tempra. La creatura subisce 5d8 danni da veleno se fallisce il Tiro Salvezza, o la metà di questi danni se lo supera. Le creature ne sono soggette anche se trattengono il respiro o non hanno bisogno di respirare. La nebbia si allontana di 3 metri da te all'inizio di ogni tuo round, spostandosi lungo la superficie del terreno. I vapori, essendo più pesanti dell'aria, tendono a scendere verso il basso, arrivando addirittura a insinuarsi nelle aperture.

\textbf{Per ogni Successo Critico Magico} ottenuto nella Prova di Magia il danno aumenta di 3d8.

\smallskip\noindent\rule{\linewidth}{2pt} \index[Incantesimi]{Nube di Nebbia}\hypertarget{Nube di Nebbia}{}\smallskip\noindent{\textbf{Nube di Nebbia}}\pdfbookmark[3]{Nube di Nebbia}{Nube di Nebbia}
\noindent
\begin{description}[noitemsep, topsep=0pt, parsep=0pt, partopsep=0pt, leftmargin=0cm, labelwidth=2.8cm]
	\item[\textbf{Lista di Magia}]: Acqua, Aria
	\item[\textbf{Livello}]: 1, Comune
	\item[\textbf{T. di Lancio}]: 2 Azioni
	\item[\textbf{Gittata}]: 36 metri
	\item[\textbf{Componenti}]: V, S
	\item[\textbf{Durata}]: 1 ora
\end{description}

Crei una sfera di foschia del raggio di 6 metri centrata su di un punto a gittata. La sfera si propaga intorno agli angoli, e la sua area è in penombra. Rimane per la durata dell'incantesimo o finché un vento di velocità moderata o superiore (almeno 15 chilometri all'ora) non la disperde.

\textbf{Per ogni Successo Critico Magico} ottenuto nella Prova di Magia il raggio della foschia aumenta di 6 metri.

\smallskip\noindent\rule{\linewidth}{2pt} \index[Incantesimi]{Occhio Arcano}\hypertarget{Occhio Arcano}{}\smallskip\noindent{\textbf{Occhio Arcano}}\pdfbookmark[3]{Occhio Arcano}{Occhio Arcano}
\noindent
\begin{description}[noitemsep, topsep=0pt, parsep=0pt, partopsep=0pt, leftmargin=0cm, labelwidth=2.8cm]
	\item[\textbf{Lista di Magia}]: Divinazione
	\item[\textbf{Livello}]: 4, Comune
	\item[\textbf{T. di Lancio}]: 2 Azioni
	\item[\textbf{Gittata}]: 9 metri
	\item[\textbf{Componenti}]: V, S, M (un pezzo di manto di pipistrello)
	\item[\textbf{Durata}]: Concentrazione, massimo 1 ora
\end{description}

Crei a gittata un occhio magico e invisibile, che fluttua nell'aria per la durata dell'incantesimo.

Ricevi mentalmente le informazioni visive dall'occhio, che ha vista normale e scurovisione fino a 9 metri. L'occhio può guardare in tutte le direzioni. Con un'Azione di Movimento, puoi spostare l'occhio di 9 metri in qualsiasi direzione. Non c'è limite a quanto lontano possa spostarsi l'occhio, ma non può entrare in un altro piano di esistenza. Una barriera solida blocca il movimento dell'occhio, ma questo può attraversare un'apertura di una grandezza minima di 2,5 centimetri di diametro.

\smallskip\noindent\rule{\linewidth}{2pt} \index[Incantesimi]{Onda Tonante}\hypertarget{Onda Tonante}{}\smallskip\noindent{\textbf{Onda Tonante}}\pdfbookmark[3]{Onda Tonante}{Onda Tonante}
\noindent
\begin{description}[noitemsep, topsep=0pt, parsep=0pt, partopsep=0pt, leftmargin=0cm, labelwidth=2.8cm]
	\item[\textbf{Lista di Magia}]: Aria
	\item[\textbf{Livello}]: 1, Comune
	\item[\textbf{T. di Lancio}]: 2 Azioni
	\item[\textbf{Gittata}]: Personale
	\item[\textbf{Componenti}]: V, S
	\item[\textbf{Durata}]: Istantanea
\end{description}

Un'onda di forza tonante si proietta da te. Ogni creatura in una sfera di 2 metri di raggio che origina da te deve effettuare un Tiro Salvezza su Tempra. Se fallisce il Tiro Salvezza una creatura subisce 2d8 danni da suono e viene allontana 3 metri da te. Se supera il Tiro Salvezza, la creatura subisce la metà dei danni e non viene allontanata. Gli oggetti non ancorati che sono totalmente all'interno dell'area vengono spinti 3 metri lontano da te dall'effetto dell'incantesimo. L'incantesimo produce un rimbombo tonante udibile fino a 90 metri.

\textbf{Per ogni Successo Critico Magico} ottenuto nella Prova di Magia il danno aumenta di 1d8.

\smallskip\noindent\rule{\linewidth}{2pt} \index[Incantesimi]{Oscurità}\hypertarget{Oscurità}{}\smallskip\noindent{\textbf{Oscurità}}\pdfbookmark[3]{Oscurita'}{Oscurita'}
\noindent
\begin{description}[noitemsep, topsep=0pt, parsep=0pt, partopsep=0pt, leftmargin=0cm, labelwidth=2.8cm]
	\item[\textbf{Lista di Magia}]: Invocazione
	\item[\textbf{Livello}]: 1, Comune
	\item[\textbf{T. di Lancio}]: 2 Azioni
	\item[\textbf{Gittata}]: 18 metri
	\item[\textbf{Componenti}]: V, M (pelo di pipistrello e un pizzico di bitume o un pezzo di carbone)
	\item[\textbf{Durata}]: 10 minuti
\end{description}

L'oscurità magica si propaga da un punto a gittata, scelto da te, per riempire una sfera di 3 metri di raggio per la durata dell'incantesimo. L'oscurità si propaga intorno agli angoli. Una creatura con scurovisione non può vedere in questa oscurità, e la luce non magica non può illuminarla.

Se il punto che hai scelto è su di un oggetto che stai trasportando o uno che non è indossato o trasportato, l'oscurità emana dall'oggetto e si muove con esso. Coprire completamente la fonte dell'oscurità con un oggetto opaco, come un vaso o un elmo, blocca l'oscurità.

Se qualsiasi parte dell'area di questo incantesimo si sovrappone con l'area di luce creata da un incantesimo con livello 2 o più basso, l'incantesimo che ha creato la luce viene dissolto.

\smallskip\noindent\rule{\linewidth}{2pt} \index[Incantesimi]{Palla di fango di Eithne}\hypertarget{Palla di fango di Eithne}{}\smallskip\noindent{\textbf{Palla di fango di Eithne}}\pdfbookmark[3]{Palla di fango di Eithne}{Palla di fango di Eithne}
\noindent
\begin{description}[noitemsep, topsep=0pt, parsep=0pt, partopsep=0pt, leftmargin=0cm, labelwidth=2.8cm]
	\item[\textbf{Lista di Magia}]: Terra
	\item[\textbf{Livello}]: 1, Non Comune
	\item[\textbf{T. di Lancio}]: 2 Azioni
	\item[\textbf{Gittata}]: 24 metri
	\item[\textbf{Componenti}]: S
	\item[\textbf{Durata}]: Istantanea
\end{description}

L'incantatore mima il gesto di tirare un sasso con una fionda in direzione del bersaglio ed esegue un Tiro per Colpire con incantesimi a distanza.
Se il Tiro per Colpire va a segno il bersaglio subisce 2d6 di danni contundenti e deve effettuare un Tiro Salvezza su Riflessi. Se il tiro salvezza fallisce il movimento del bersaglio diminuisce di 2 metri per 1 minuto.

\textbf{Per ogni Successo Critico Magico} ottenuto nella Prova di Magia scagli un sasso in più.

\begin{changemargin}{0.3cm}{0.3cm}\begin{enfasi}{
			Mi sparpaglio in giro per evitare incantesimi ad area (detta da un giocatore per evitare una Palla di Fuoco)
}\end{enfasi}\end{changemargin}

\smallskip\noindent\rule{\linewidth}{2pt} \index[Incantesimi]{Palla di Fuoco}\hypertarget{Palla di Fuoco}{}\smallskip\noindent{\textbf{Palla di Fuoco}}\pdfbookmark[3]{Palla di Fuoco}{Palla di Fuoco}
\noindent
\begin{description}[noitemsep, topsep=0pt, parsep=0pt, partopsep=0pt, leftmargin=0cm, labelwidth=2.8cm]
	\item[\textbf{Lista di Magia}]: Fuoco
	\item[\textbf{Livello}]: 3, Comune
	\item[\textbf{T. di Lancio}]: 2 Azioni
	\item[\textbf{Gittata}]: 45 metri
	\item[\textbf{Componenti}]: V, S, M (una minuscola palla di guano di pipistrello e zolfo)
	\item[\textbf{Durata}]: Istantanea
\end{description}

Un fascio di luce gialla parte dal tuo dito puntato verso un punto a gittata scelto da te e poi esplode con un boato roboante e si trasforma in lingua di fiamme.

Ogni creatura in una sfera di 6 metri di raggio centrata in quel punto deve effettuare un Tiro Salvezza su Riflessi. Una creatura subisce 8d6 danni da fuoco se fallisce il Tiro Salvezza, o la metà di questi danni se lo supera.

Il fuoco si propaga ed occupa tutto il volume disponibile entro i 6 metri dal punto di esplosione. Il fuoco incendia gli oggetti infiammabili nell'area che non sono indossati o trasportati.

\textbf{Per ogni Successo Critico Magico} ottenuto nella Prova di Magia il danno base aumenta di 4d6.

\textbf{Tiro Salvezza Successo/Fallimento Critico}: In caso di Fallimento Critico il danno raddoppia, in caso di Successo Critico il danno viene ulteriormente dimezzato

\smallskip\noindent\rule{\linewidth}{2pt} \index[Incantesimi]{Palla di Fuoco Ritardata}\hypertarget{Palla di Fuoco Ritardata}{}\smallskip\noindent{\textbf{Palla di Fuoco Ritardata}}\pdfbookmark[3]{Palla di Fuoco Ritardata}{Palla di Fuoco Ritardata}
\noindent
\begin{description}[noitemsep, topsep=0pt, parsep=0pt, partopsep=0pt, leftmargin=0cm, labelwidth=2.8cm]
	\item[\textbf{Lista di Magia}]: Fuoco
	\item[\textbf{Livello}]: 7, Raro
	\item[\textbf{T. di Lancio}]: 2 Azioni
	\item[\textbf{Gittata}]: 45 metri
	\item[\textbf{Componenti}]: V, S, M (una grossa palla di guano di pipistrello e zolfo)
	\item[\textbf{Durata}]: Concentrazione, 1 minuto
\end{description}

Un fascio di luce gialla parte dal tuo dito puntato, per condensarsi per la durata dell'incantesimo nella forma di una pallina luminosa in un punto a gittata, scelto da te. Quando l'incantesimo termina, o perché la tua concentrazione è spezzata o perché decidi tu di porgli fine, la pallina esplode con un boato sommesso e si trasforma in un getto di fiamme che si propaga dietro gli angoli. Ogni creatura in una sfera di 6 metri di raggio centrata in quel punto deve effettuare un Tiro Salvezza su Riflessi. Una creatura subisce danni da fuoco pari al danno totale accumulato se fallisce il Tiro Salvezza, o la metà di questi danni se lo supera. Il danno base dell'incantesimo è 12d6. Se al termine del tuo round la pallina non è ancora detonata, il danno aumenta di 1d6.

Se la pallina luminosa viene toccata prima che l'incantesimo abbia avuto fine la pallina esplode.

Il fuoco danneggia gli oggetti nell'area e incendia gli oggetti infiammabili che non sono indossati o trasportati.

\textbf{Per ogni Successo Critico Magico} ottenuto nella Prova di Magia il danno aumento di 6d6.

\textbf{Tiro Salvezza Successo/Fallimento Critico}: In caso di Fallimento Critico il danno raddoppia, in caso di Successo Critico il danno viene ulteriormente dimezzato.

\smallskip\noindent\rule{\linewidth}{2pt} \index[Incantesimi]{Parlare con gli Animali}\hypertarget{Parlare con gli Animali}{}\smallskip\noindent{\textbf{Parlare con gli Animali}}\pdfbookmark[3]{Parlare con gli Animali}{Parlare con gli Animali}
\noindent
\begin{description}[noitemsep, topsep=0pt, parsep=0pt, partopsep=0pt, leftmargin=0cm, labelwidth=2.8cm]
	\item[\textbf{Lista di Magia}]: Animali e Piante
	\item[\textbf{Livello}]: 1, Comune
	\item[\textbf{T. di Lancio}]: 2 Azioni
	\item[\textbf{Gittata}]: Personale
	\item[\textbf{Componenti}]: V, S
	\item[\textbf{Durata}]: 10 minuti
\end{description}

Per la durata dell'incantesimo, ottieni la capacità di comprendere e comunicare verbalmente con le bestie. Il sapere e la consapevolezza di molte bestie sono limitati dal loro intelletto ma, come minimo, le bestie possono fornirti informazioni riguardo luoghi e mostri nelle vicinanze, compresi quelli che possono percepire o hanno percepito nei giorni passati. A discrezione del Narratore potresti riuscire a convincere una bestia a farti un piccolo favore.

\textbf{Per ogni Successo Critico Magico} ottenuto nella Prova di Magia la durata raddoppia.

\smallskip\noindent\rule{\linewidth}{2pt} \index[Incantesimi]{Parlare con i Morti}\hypertarget{Parlare con i Morti}{}\smallskip\noindent{\textbf{Parlare con i Morti}}\pdfbookmark[3]{Parlare con i Morti}{Parlare con i Morti}
\noindent
\begin{description}[noitemsep, topsep=0pt, parsep=0pt, partopsep=0pt, leftmargin=0cm, labelwidth=2.8cm]
	\item[\textbf{Lista di Magia}]: Necromanzia
	\item[\textbf{Livello}]: 3, Raro
	\item[\textbf{T. di Lancio}]: 2 Azioni
	\item[\textbf{Gittata}]: 3 metri
	\item[\textbf{Componenti}]: V, S, M (incenso acceso)
	\item[\textbf{Durata}]: 10 minuti
\end{description}

Conferisci un'apparenza di vita e Intelligenza a un cadavere a gittata, scelto da te, permettendogli di rispondere alle domande che gli poni. Il cadavere deve avere ancora una bocca e non può essere non morto. L'incantesimo fallisce se il cadavere è già stato bersaglio di questo incantesimo negli ultimi 10 giorni. Fino al termine dell'incantesimo, puoi porre al cadavere fino a cinque domande. Il cadavere conosce solo quello che già sapeva in vita, compresi i linguaggi parlati. Le risposte sono di solito brevi, criptiche o ripetitive, e il cadavere non è sotto nessun obbligo a darti risposte veritiere se gli sei ostile o ti riconosce come suo nemico. Questo incantesimo non riporta l'anima della creatura nel corpo, ma solo lo spirito che lo muove. Di conseguenza, il cadavere non può apprendere nuove informazioni, non capisce nulla di quello che è successo da quando è morto, e non può fare valutazioni su eventi futuri.

\smallskip\noindent\rule{\linewidth}{2pt} \index[Incantesimi]{Parlare con le Creature}\hypertarget{Parlare con le Creature}{}\smallskip\noindent{\textbf{Parlare con le Creature}}\pdfbookmark[3]{Parlare con le Creature}{Parlare con le Creature}
\noindent
\begin{description}[noitemsep, topsep=0pt, parsep=0pt, partopsep=0pt, leftmargin=0cm, labelwidth=2.8cm]
	\item[\textbf{Lista di Magia}]: Animali e Piante, Divinazione
	\item[\textbf{Livello}]: 6, Raro
	\item[\textbf{T. di Lancio}]: 2 Azioni
	\item[\textbf{Gittata}]: 9 metri
	\item[\textbf{Componenti}]: V, S
	\item[\textbf{Durata}]: 1 ora
\end{description}

Questo incantesimo è una versione più potente di parlare con gli animali, che consente di parlare con qualsiasi creatura entro la gittata, indipendentemente dalla sua natura o intelligenza (che deve essere maggiore di -5).

\smallskip\noindent\rule{\linewidth}{2pt} \index[Incantesimi]{Parlare con le Piante}\hypertarget{Parlare con le Piante}{}\smallskip\noindent{\textbf{Parlare con le Piante}}\pdfbookmark[3]{Parlare con le Piante}{Parlare con le Piante}
\noindent
\begin{description}[noitemsep, topsep=0pt, parsep=0pt, partopsep=0pt, leftmargin=0cm, labelwidth=2.8cm]
	\item[\textbf{Lista di Magia}]: Animali e piante
	\item[\textbf{Livello}]: 3, Raro
	\item[\textbf{T. di Lancio}]: 2 Azioni
	\item[\textbf{Gittata}]: Personale (raggio di 9 metri)
	\item[\textbf{Componenti}]: V, S
	\item[\textbf{Durata}]: 10 minuti
\end{description}

Infondi i vegetali entro 9 metri da te di capacità senziente e di limitata mobilità, dandole la capacità di comunicare con te ed eseguire dei semplici comandi. Puoi interrogare i vegetali in merito a eventi avvenuti nell'ultimo giorno nell'area dell'incantesimo, ottenendo informazioni sulle creature di passaggio, il clima e altro. Puoi anche trasformare il terreno difficile prodotto dalla crescita dei vegetali (come cespugli e fitto sottobosco) in terreno ordinario per la durata dell'incantesimo.

Oppure puoi trasformare del terreno normale in cui siano presenti dei vegetali in terreno difficile che rimane per la durata dell'incantesimo facendo sì, per esempio, che rampicanti e rami rallentino gli inseguitori.

A discrezione del Narratore i vegetali potrebbero svolgere anche altri compiti per tuo conto. L'incantesimo non permette ai vegetali di sradicarsi e muoversi, ma possono muovere liberamente rami, steli e gambi. Se una creatura vegetale si trova nell'area, puoi comunicare con essa come se parlaste lo stesso linguaggio, ma non ottieni alcuna capacità magica per influenzarla. Questo incantesimo può far sì che i vegetali creati dall'incantesimo intralciare rilascino una creatura intralciata.

\smallskip\noindent\rule{\linewidth}{2pt} \index[Incantesimi]{Parola Divina}\hypertarget{Parola Divina}{}\smallskip\noindent{\textbf{Parola Divina}}\pdfbookmark[3]{Parola Divina}{Parola Divina}
\noindent
\begin{description}[noitemsep, topsep=0pt, parsep=0pt, partopsep=0pt, leftmargin=0cm, labelwidth=2.8cm]
	\item[\textbf{Lista di Magia}]: Invocazione
	\item[\textbf{Livello}]: 7, Molto Raro
	\item[\textbf{T. di Lancio}]: 2 Azioni
	\item[\textbf{Gittata}]: 9 metri
	\item[\textbf{Componenti}]: V
	\item[\textbf{Durata}]: Istantanea
\end{description}

Pronunci una parola divina, infusa del potere del tuo Patrono. Scegli un qualsiasi numero di creature a gittata e che puoi vedere. Ogni creatura che può udirti deve effettuare un Tiro Salvezza su Volontà. Se fallisce il Tiro Salvezza, la creatura subisce un effetto in base ai suoi attuali Punti Ferita:

\begin{itemize}[leftmargin=*] \setlength{\itemsep}{0pt}
	\item 100 Punti Ferita o meno: assordata per 1 minuto
	\item 40 Punti Ferita o meno: assordata e accecata per 10 minuti
	\item 30 Punti Ferita o meno: accecata, assordata e stordita per 1 ora
	\item 20 Punti Ferita o meno: uccisa all'istante
\end{itemize}

Quali che siano i suoi attuali Punti Ferita, un celestiale, elementale, fatato o demone che fallisca il Tiro Salvezza è obbligato a tornare al suo piano di origine (se non vi si trova già) e non può tornare sul tuo attuale piano prima che siano passate 24 ore, a meno dell'uso dell'incantesimo desiderio.

\smallskip\noindent\rule{\linewidth}{2pt} \index[Incantesimi]{Parola del Potere Stordire}\hypertarget{Parola del Potere Stordire}{}\smallskip\noindent{\textbf{Parola del Potere Stordire}}\pdfbookmark[3]{Parola del Potere Stordire}{Parola del Potere Stordire}
\noindent
\begin{description}[noitemsep, topsep=0pt, parsep=0pt, partopsep=0pt, leftmargin=0cm, labelwidth=2.8cm]
	\item[\textbf{Lista di Magia}]: Ammaliamento
	\item[\textbf{Livello}]: 8, Non Comune
	\item[\textbf{T. di Lancio}]: 2 Azioni
	\item[\textbf{Gittata}]: 18 metri
	\item[\textbf{Componenti}]: V
	\item[\textbf{Durata}]: 1 minuti
\end{description}

Pronunci una parola di potere che può travolgere la mente di una creatura a gittata e che puoi vedere. Se il bersaglio ha 150 Punti Ferita o meno, è stordito per 2d4 round, altrimenti l'incantesimo non ha effetto.

\smallskip\noindent\rule{\linewidth}{2pt} \index[Incantesimi]{Parola del Potere Uccidere}\hypertarget{Parola del Potere Uccidere}{}\smallskip\noindent{\textbf{Parola del Potere Uccidere}}\pdfbookmark[3]{Parola del Potere Uccidere}{Parola del Potere Uccidere}
\noindent
\begin{description}[noitemsep, topsep=0pt, parsep=0pt, partopsep=0pt, leftmargin=0cm, labelwidth=2.8cm]
	\item[\textbf{Lista di Magia}]: Ammaliamento
	\item[\textbf{Livello}]: 9, Raro
	\item[\textbf{T. di Lancio}]: 2 Azioni
	\item[\textbf{Gittata}]: 18 metri
	\item[\textbf{Componenti}]: V
	\item[\textbf{Durata}]: Istantanea
\end{description}

Pronunci una parola di potere che costringe a morire all'istante una creatura a gittata che puoi vedere. Se la creatura che scegli ha 100 Punti Ferita o meno, muore. Altrimenti l'incantesimo non ha effetto.

\smallskip\noindent\rule{\linewidth}{2pt} \index[Incantesimi]{Parola del Ritiro}\hypertarget{Parola del Ritiro}{}\smallskip\noindent{\textbf{Parola del Ritiro}}\pdfbookmark[3]{Parola del Ritiro}{Parola del Ritiro}
\noindent
\begin{description}[noitemsep, topsep=0pt, parsep=0pt, partopsep=0pt, leftmargin=0cm, labelwidth=2.8cm]
	\item[\textbf{Lista di Magia}]: Evocazione
	\item[\textbf{Livello}]: 6, Raro
	\item[\textbf{T. di Lancio}]: 2 Azioni
	\item[\textbf{Gittata}]: 1 metro
	\item[\textbf{Componenti}]: V
	\item[\textbf{Durata}]: Istantanea
\end{description}

Te e fino a cinque creature consenzienti entro 1 metro da te vi teletrasportate istantaneamente in un luogo sicuro indicato precedentemente, detto santuario. Tu e tutte le creature teletrasportate con te, riapparite nello spazio non occupato più vicino al punto indicato quando hai preparato questo santuario (vedi sotto). Se lanci questo incantesimo senza aver prima preparato un santuario, l'incantesimo non ha effetto.

Devi indicare un santuario, che sia dedicato o fortemente collegato al tuo Patrono. Se tenti di lanciare l'incantesimo perché ti porti in un'area che non sia dedicata dal tuo Patrono, l'incantesimo non ha effetto.

\textbf{NOTA}: devi essere un Devoto o Seguace per poter lanciare questo incantesimo.

\smallskip\noindent\rule{\linewidth}{2pt} \index[Incantesimi]{Passa Porta}\hypertarget{Passa Porta}{}\smallskip\noindent{\textbf{Passa Porta}}\pdfbookmark[3]{Passa Porta}{Passa Porta}
\noindent
\begin{description}[noitemsep, topsep=0pt, parsep=0pt, partopsep=0pt, leftmargin=0cm, labelwidth=2.8cm]
	\item[\textbf{Lista di Magia}]: Terra
	\item[\textbf{Livello}]: 5, Non Comune
	\item[\textbf{T. di Lancio}]: 2 Azioni
	\item[\textbf{Gittata}]: 9 metri
	\item[\textbf{Componenti}]: V, S, M (un pizzico di semi di sesamo)
	\item[\textbf{Durata}]: 1 ora
\end{description}

Per la durata dell'incantesimo, compare un passaggio in un punto a gittata che puoi vedere, su di una superficie di legno, muro o pietra (come una parete, un soffitto o un pavimento) scelta da te. Scegli le dimensioni dell'apertura: al massimo larga 1 metro, alta 2 metri e profonda 6 metri. Il passaggio non crea instabilità nella struttura che lo circonda.

Quando l'apertura sparisce, qualsiasi creatura od oggetto ancora nel passaggio creato dall'incantesimo viene espulso al sicuro nello spazio non occupato più vicino alla superficie su cui hai lanciato l'incantesimo.

\textbf{Per ogni Successo Critico Magico} puoi creare un altra porta pur nella durata dell'incantesimo.

\smallskip\noindent\rule{\linewidth}{2pt} \index[Incantesimi]{Passare Senza Tracce}\hypertarget{Passare Senza Tracce}{}\smallskip\noindent{\textbf{Passare Senza Tracce}}\pdfbookmark[3]{Passare Senza Tracce}{Passare Senza Tracce}
\noindent
\begin{description}[noitemsep, topsep=0pt, parsep=0pt, partopsep=0pt, leftmargin=0cm, labelwidth=2.8cm]
	\item[\textbf{Lista di Magia}]: Terra, Animali e Piante
	\item[\textbf{Livello}]: 2, Comune
	\item[\textbf{T. di Lancio}]: 2 Azioni
	\item[\textbf{Gittata}]: Creatura toccata
	\item[\textbf{Componenti}]: V, S, M (ceneri di una foglia di vischio bruciata e un ramoscello di abete rosso)
	\item[\textbf{Durata}]: 1 ora
\end{description}

Per la durata dell'incantesimo la creatura toccata non lascia tracce sul terreno.

\textbf{Per ogni Successo Critico Magico} ottenuto nella Prova di Magia puoi influenzare un altra creatura.

\smallskip\noindent\rule{\linewidth}{2pt} \index[Incantesimi]{Passo Velato}\hypertarget{Passo Velato}{}\smallskip\noindent{\textbf{Passo Velato}}\pdfbookmark[3]{Passo Velato}{Passo Velato}
\noindent
\begin{description}[noitemsep, topsep=0pt, parsep=0pt, partopsep=0pt, leftmargin=0cm, labelwidth=2.8cm]
	\item[\textbf{Lista di Magia}]: Evocazione
	\item[\textbf{Livello}]: 2, Raro
	\item[\textbf{T. di Lancio}]: 1 Azione
	\item[\textbf{Gittata}]: Personale
	\item[\textbf{Componenti}]: V
	\item[\textbf{Durata}]: Istantanea
\end{description}

Avvolto rapidamente da una foschia argentata, ti teletrasporti di massimo 9 metri in uno spazio non occupato che puoi vedere.

\textbf{Se ottieni due Successo Critico Magico ottenuto} nella Prova di Magia puoi scambiarti di posto con una creatura consenziente.

\textbf{NOTA}: se sei un devoto di Lynx l'incantesimo ha tempo di lancio di 1 Azione Immediata e la rarità è Non Comune.

\smallskip\noindent\rule{\linewidth}{2pt} \index[Incantesimi]{Passo Veloce}\hypertarget{Passo Veloce}{}\smallskip\noindent{\textbf{Passo Veloce}}\pdfbookmark[3]{Passo Veloce}{Passo Veloce}
\noindent
\begin{description}[noitemsep, topsep=0pt, parsep=0pt, partopsep=0pt, leftmargin=0cm, labelwidth=2.8cm]
	\item[\textbf{Lista di Magia}]: Trasmutazione
	\item[\textbf{Livello}]: 1, Molto Raro
	\item[\textbf{T. di Lancio}]: 2 Azioni
	\item[\textbf{Gittata}]: Contatto
	\item[\textbf{Componenti}]: V, S, M (una zampa di lepre)
	\item[\textbf{Durata}]: 1 ora
\end{description}

Il movimento di una creatura aumenta di 1 metro fino al termine dell'incantesimo.

\textbf{Per ogni Successo Critico Magico} ottenuto nella Prova di Magia puoi prendere come bersaglio un'ulteriore creatura.

\smallskip\noindent\rule{\linewidth}{2pt} \index[Incantesimi]{Paura}\hypertarget{Paura}{}\smallskip\noindent{\textbf{Paura}}\pdfbookmark[3]{Paura}{Paura}
\noindent
\begin{description}[noitemsep, topsep=0pt, parsep=0pt, partopsep=0pt, leftmargin=0cm, labelwidth=2.8cm]
	\item[\textbf{Lista di Magia}]: Illusione
	\item[\textbf{Livello}]: 3, Non Comune
	\item[\textbf{T. di Lancio}]: 2 Azioni
	\item[\textbf{Gittata}]: Personale (cono di 9 metri)
	\item[\textbf{Componenti}]: V, S, M (una piuma bianca o il cuore di una gallina)
	\item[\textbf{Durata}]: 1 minuto
\end{description}

Proietti un'immagine illusoria delle peggiori paure di una creatura. Ogni creatura in un cono di 9 metri deve superare un Tiro Salvezza su Volontà o far cadere qualsiasi cosa stia impugnando e restare Spaventata per la durata dell'incantesimo.

Mentre è \hyperlink{condizionepaura}{spaventata} da questo incantesimo, una creatura deve, durante ciascun suo round, muoversi lontano da te tramite il tragitto più sicuro, a meno che non abbia spazio per muoversi. Se la creatura termina il suo round in un posto dove non può vederti, può effettuare un Tiro Salvezza su Volontà, se lo supera, l'incantesimo, per quella creatura, ha termine.

\smallskip\noindent\rule{\linewidth}{2pt} \index[Incantesimi]{Pelle di Corteccia}\hypertarget{Pelle di Corteccia}{}\smallskip\noindent{\textbf{Pelle di Corteccia}}\pdfbookmark[3]{Pelle di Corteccia}{Pelle di Corteccia}
\noindent
\begin{description}[noitemsep, topsep=0pt, parsep=0pt, partopsep=0pt, leftmargin=0cm, labelwidth=2.8cm]
	\item[\textbf{Lista di Magia}]: Animali e Piante
	\item[\textbf{Livello}]: 2, Comune
	\item[\textbf{T. di Lancio}]: 2 Azioni
	\item[\textbf{Gittata}]: Contatto
	\item[\textbf{Componenti}]: V, S, M (una manciata di corteccia di quercia)
	\item[\textbf{Durata}]: 1 ora
\end{description}

La pelle del bersaglio con cui sei in contatto al momento del lancio dell'incantesimo diventa ruvida e dall'aspetto simile alla corteccia fino al termine dell'incantesimo e la Difesa naturale del bersaglio aumenta di 1 + 1/6 Competenza Magica indipendentemente dall'armatura che stia indossando.

\smallskip\noindent\rule{\linewidth}{2pt} \index[Incantesimi]{Pelle di Pietra}\hypertarget{Pelle di Pietra}{}\smallskip\noindent{\textbf{Pelle di Pietra}}\pdfbookmark[3]{Pelle di Pietra}{Pelle di Pietra}
\noindent
\begin{description}[noitemsep, topsep=0pt, parsep=0pt, partopsep=0pt, leftmargin=0cm, labelwidth=2.8cm]
	\item[\textbf{Lista di Magia}]: Terra
	\item[\textbf{Livello}]: 4, Non Comune
	\item[\textbf{T. di Lancio}]: 2 Azioni
	\item[\textbf{Gittata}]: Contatto
	\item[\textbf{Componenti}]: V, S, M (polvere di diamante del valore di 100 mo, che l'incantesimo consuma)
	\item[\textbf{Durata}]: 1 ora
\end{description}

Lanci l'incantesimo a contatto di una creatura consenziente, la cui pelle si tramuta in una sostanza dura come la pietra. Tira 1d4+metà del valore di CM, la somma risultante è le volte che un attacco con arma di mischia o distanza viene annullato (indipendentemente che di colpisca o meno).

\textbf{Per ogni Successo Critico Magico} ottenuto nella Prova di Magia aumenti di 2 gli attacchi annullati.

\smallskip\noindent\rule{\linewidth}{2pt} \index[Incantesimi]{Piaga degli Insetti}\hypertarget{Piaga degli Insetti}{}\smallskip\noindent{\textbf{Piaga degli Insetti}}\pdfbookmark[3]{Piaga degli Insetti}{Piaga degli Insetti}
\noindent
\begin{description}[noitemsep, topsep=0pt, parsep=0pt, partopsep=0pt, leftmargin=0cm, labelwidth=2.8cm]
	\item[\textbf{Lista di Magia}]: Animali e Piante
	\item[\textbf{Livello}]: 5, Raro
	\item[\textbf{T. di Lancio}]: 2 Azioni
	\item[\textbf{Gittata}]: 90 metri
	\item[\textbf{Componenti}]: V, S, M (qualche granello di zucchero, qualche chicco di grano, un pò di lardo)
	\item[\textbf{Durata}]: 10 minuti
\end{description}

Uno sciame di locuste affamate riempie una sfera di 6 metri di raggio centrata in un punto a gittata scelto da te. La sfera si propaga intorno agli angoli. La sfera rimane per la durata dell'incantesimo, e la sua area è in penombra. L'area della sfera è terreno difficile.

Quando l'area appare, ogni creatura al suo interno deve effettuare un Tiro Salvezza su Tempra. Una creatura subisce 4d10 danni se fallisce il Tiro Salvezza, o la metà di questi danni se lo supera. Una creatura deve effettuare questo Tiro Salvezza anche quando entra per la prima volta nell'area dell'incantesimo durante un round o se termina il proprio round al suo interno.

\textbf{Per ogni Successo Critico Magico} ottenuto nella Prova di Magia il danno aumenta di 2d10.

\smallskip\noindent\rule{\linewidth}{2pt} \index[Incantesimi]{Pietra in Fango - Fango in Pietra}\hypertarget{Pietra in Fango - Fango in Pietra}{}\smallskip\noindent{\textbf{Pietra in Fango - Fango in Pietra}}\pdfbookmark[3]{Pietra in Fango - Fango in Pietra}{Pietra in Fango - Fango in Pietra}\index[Incantesimi]{Pietra in Fango}\index[Incantesimi]{Fango in Roccia}
\noindent
\begin{description}[noitemsep, topsep=0pt, parsep=0pt, partopsep=0pt, leftmargin=0cm, labelwidth=2.8cm]
	\item[\textbf{Lista di Magia}]: Terra
	\item[\textbf{Livello}]: 5, Non Comune - Molto Raro
	\item[\textbf{T. di Lancio}]: 2 Azioni
	\item[\textbf{Gittata}]: 45 metri
	\item[\textbf{Componenti}]: V, S, M (acqua e argilla)
	\item[\textbf{Durata}]: Istantanea
\end{description}

Questo incantesimo trasforma qualsiasi tipo di roccia naturale in un eguale volume di fango. La pietra magica non viene influenzata dall'incantesimo. L'incantesimo ha effetto fino a 2 cubi di 3x3x3 metri. La profondità del fango creato non può superare i 3 metri. Le creature incapaci di Volare, levitare o allontanarsi in qualche modo dal fango affondano fino alla vita o fino al petto; le creatura sono intralciate ed il terreno diviene doppiamente difficile. Le creature abbastanza grandi da camminare sul fondo della pozza di fango possono guadare l'area come terreno difficile.

Se Pietra in Fango viene lanciato sul soffitto di una caverna o di un tunnel, il fango si riversa sul pavimento e si espande fino a formare una pozza della profondità di 1 metro. Il fango in caduta e la frana che ne segue infliggono 8d6 danni contundenti a chiunque si trovi direttamente sotto l'area se non dimezza i danni con un Tiro Salvezza su Riflessi.

I castelli e i grandi edifici in pietra sono generalmente immuni agli effetti dell'incantesimo, in quanto Trasformare Pietra in Fango non arriva abbastanza in profondità da minare le fondamenta. Tuttavia altri edifici più piccoli spesso poggiano su fondamenta abbastanza superficiali da poter essere danneggiate o perfino distrutte dagli effetti dell'incantesimo.

Il fango rimane finché non viene usato con successo un incantesimo Dissolvi Magie o Fango in Pietra, che ripristina la sua sostanza, ma non necessariamente la sua forma. L'evaporazione naturale trasforma il fango in terreno normale nel giro di diversi giorni a seconda dell'esposizione al sole, al vento ed all'essiccazione naturale.
Se una creatura è nel fango al momento dell'incantesimo Fango in Pietra può effettuare un Tiro Salvezza su Riflessi per liberarsi altrimenti è necessario un Tiro Salvezza Tempra con Forza a DC 26 oppure 30 di danno, per rompere la pietra.

\textbf{Per ogni Successo Critico Magico} ottenuto nella Prova di Magia influenzi un cubo di 3x3x3 metri in più.

\smallskip\noindent\rule{\linewidth}{2pt} \index[Incantesimi]{Pietre Parlanti}\hypertarget{Pietre Parlanti}{}\smallskip\noindent{\textbf{Pietre Parlanti}}\pdfbookmark[3]{Pietre Parlanti}{Pietre Parlanti}
\noindent
\begin{description}[noitemsep, topsep=0pt, parsep=0pt, partopsep=0pt, leftmargin=0cm, labelwidth=2.8cm]
	\item[\textbf{Lista di Magia}]: Terra, Divinazione
	\item[\textbf{Livello}]: 6, Raro
	\item[\textbf{T. di Lancio}]: 2 Azioni
	\item[\textbf{Gittata}]: Tocco
	\item[\textbf{Componenti}]: V, S
	\item[\textbf{Durata}]: 1 turno
\end{description}

L'incantatore acquisisce la capacità di parlare con le pietre, le quali possono dire chi o cosa le abbia toccate e rivelare ciò che nascondono dietro o sotto di loro. Le pietre forniscono descrizioni accurate a richiesta, anche se potrebbero non fornire i dettagli desiderati in certe situazioni. L'incantatore può parlare con pietre sia naturali che lavorate.

\textbf{Per ogni Successo Critico Magico} ottenuto nella Prova di Magia la durata raddoppia oppure può parlare con un altra pietra entro 18 metri dalla prima.



\smallskip\noindent\rule{\linewidth}{2pt} \index[Incantesimi]{Pioggia di Meteore}\hypertarget{Pioggia di Meteore}{}\smallskip\noindent{\textbf{Pioggia di Meteore}}\pdfbookmark[3]{Pioggia di Meteore}{Pioggia di Meteore}\hypertarget{sciamedimeteore}{}
\noindent
\begin{description}[noitemsep, topsep=0pt, parsep=0pt, partopsep=0pt, leftmargin=0cm, labelwidth=2.8cm]
	\item[\textbf{Lista di Magia}]: Fuoco, Terra
	\item[\textbf{Livello}]: 9, Leggendario
	\item[\textbf{T. di Lancio}]: 3 Azioni
	\item[\textbf{Gittata}]: 1,5 chilometri
	\item[\textbf{Componenti}]: V, S
	\item[\textbf{Durata}]: Istantanea
\end{description}

4 meteoriti di fuoco si schiantano a terra in quattro punti differenti a gittata e che puoi vedere. Ogni meteorite colpisce in un raggio di 3 metri. Ogni creatura interessata deve effettuare un Tiro Salvezza su Riflessi. Una creatura subisce 20d6 danni da fuoco e 20d6 danni contundenti se fallisce il Tiro Salvezza, o la metà di
questi danni se lo supera. Una creatura se nell'area di più di un meteorite ne subisce gli effetti solo do uno.

\textbf{Tiro Salvezza Successo/Fallimento Critico}: In caso di Fallimento Critico il danno raddoppia, in caso di Successo Critico il danno viene ulteriormente dimezzato

\textbf{Ogni 3 critici ottenuti} nella Prova di Magia scagli un altro meteorite.

\smallskip\noindent\rule{\linewidth}{2pt} \index[Incantesimi]{Piroesperto}\hypertarget{Piroesperto}{}\smallskip\noindent{\textbf{Piroesperto}}\pdfbookmark[3]{Piroesperto}{Piroesperto}
\noindent
\begin{description}[noitemsep, topsep=0pt, parsep=0pt, partopsep=0pt, leftmargin=0cm, labelwidth=2.8cm]
	\item[\textbf{Lista di Magia}]: Fuoco
	\item[\textbf{Livello}]: 2, Non Comune
	\item[\textbf{T. di Lancio}]: 2 Azioni
	\item[\textbf{Gittata}]: 18 metri
	\item[\textbf{Componenti}]: V, S, M (un fiammifero che viene consumato)
	\item[\textbf{Durata}]: Istantanea
\end{description}

L'incantatore sceglie un'area con un fuoco, di almeno 1 metro di spigolo, entro gittata che sia a lui direttamente visibile. Estinguendo le fiamme può creare fuochi d'artificio o fumo.

\begin{itemize}[leftmargin=*] \setlength{\itemsep}{0pt}
	\item \emph{Fuochi d'Artificio}. Il fuoco bersaglio esplode in uno spettacolo luminoso di fiamme e colori. Ogni creatura entro 3 metri dal bersaglio deve superare un Tiro Salvezza su Tempra oppure diventa accecata fino alla fine round successivo.
	\item \emph{Fumo}. Un fumo nero e denso scaturisce dal fuoco bersaglio e si diffonde in un raggio di 6 metri, muovendosi oltre gli angoli. L'area del fumo è pesantemente oscurata e fornisce copertura media. Il fumo persiste per 1 minuto o finché un vento forte non lo disperde.
\end{itemize}

\smallskip\noindent\rule{\linewidth}{2pt} \index[Incantesimi]{Polvere luccicante}\hypertarget{Polvere luccicante}{}\smallskip\noindent{\textbf{Polvere luccicante}}\pdfbookmark[3]{Polvere luccicante}{Polvere luccicante}
\noindent
\begin{description}[noitemsep, topsep=0pt, parsep=0pt, partopsep=0pt, leftmargin=0cm, labelwidth=2.8cm]
	\item[\textbf{Lista di Magia}]: Fuoco, Aria
	\item[\textbf{Livello}]: 2, Non Comune
	\item[\textbf{T. di Lancio}]: 2 Azioni
	\item[\textbf{Gittata}]: 36 metri
	\item[\textbf{Componenti}]: V, S, M (polvere d'argento)
	\item[\textbf{Durata}]: 1 round per CM
\end{description}

In una sfera di 3 metri di diametro chiunque si trovi viene ricoperto da polvere luccicante e luminosa. La nuvola delinea le creature presenti, anche quelle invisibili e chiunque permanga nell'area deve fare ad inizio round un Tiro Salvezza su Riflessi od essere accecata per il round. La polvere scompare naturalmente dopo la durata o se portata via da un vento anche leggero.

\smallskip\noindent\rule{\linewidth}{2pt} \index[Incantesimi]{Porta Dimensionale}\hypertarget{Porta Dimensionale}{}\smallskip\noindent{\textbf{Porta Dimensionale}}\pdfbookmark[3]{Porta Dimensionale}{Porta Dimensionale}
\noindent
\begin{description}[noitemsep, topsep=0pt, parsep=0pt, partopsep=0pt, leftmargin=0cm, labelwidth=2.8cm]
	\item[\textbf{Lista di Magia}]: Evocazione
	\item[\textbf{Livello}]: 4, Comune
	\item[\textbf{T. di Lancio}]: 2 Azioni
	\item[\textbf{Gittata}]: 150 metri
	\item[\textbf{Componenti}]: V
	\item[\textbf{Durata}]: Istantanea
\end{description}

Ti teletrasporti dalla tua attuale posizione in qualsiasi altro posto a gittata. Arrivi esattamente nel posto desiderato. Può essere un luogo che puoi vedere, uno che puoi visualizzare, o uno che puoi descrivere indicando distanza e direzione, come \emph{30 metri verso il basso} o \emph{90 metri in alto a nordovest con un angolo di 45 gradi}.

Puoi portare con te oggetti il cui peso non ecceda la tua capacità di Ingombro. Puoi portare con te anche una creatura consenziente della tua taglia o più piccola con equipaggiamento fino al limite della sua capacità di carico. La creatura deve essere entro 1 metro da te quando lanci questo incantesimo.

Se dovessi arrivare in un posto già occupato da un oggetto o creatura, tu e la creatura che viaggia con te subite ciascuno 4d6 danni da forza, e l'incantesimo non riesce a teletrasportarvi.

\textbf{Per ogni due Successo Critico Magico ottenuto} nella Prova di Magia puoi portare una ulteriore creatura.

\smallskip\noindent\rule{\linewidth}{2pt} \index[Incantesimi]{Preghiera}\hypertarget{Preghiera}{}\smallskip\noindent{\textbf{Preghiera}}\pdfbookmark[3]{Preghiera}{Preghiera}
\noindent
\begin{description}[noitemsep, topsep=0pt, parsep=0pt, partopsep=0pt, leftmargin=0cm, labelwidth=2.8cm]
	\item[\textbf{Lista di Magia}]: Invocazione
	\item[\textbf{Livello}]: 3, Non Comune
	\item[\textbf{T. di Lancio}]: 2 Azioni
	\item[\textbf{Gittata}]: Personale
	\item[\textbf{Componenti}]: V, S
	\item[\textbf{Durata}]: 1 round per livello, Concentrazione
\end{description}

Intoni un canto al tuo Patrono e ne invochi la benedizione. Le creature entro 9 metri da te prendono un +1 al Tiro per Colpire e Tiro Salvezza per tratto in comune con il Patrono.

Le creature con Tratti diversi dal Patrono prendono un -1 hai Tiri per Colpire e Tiri Salvezza.

\textbf{Nota}: devi essere un Devoto per poter lanciare questo incantesimo, e puoi influenzare un numero di creature pari al tuo punteggio Tratti più alto condiviso con il Patrono.

\smallskip\noindent\rule{\linewidth}{2pt} \index[Incantesimi]{Preghiera di Guarigione}\hypertarget{Preghiera di Guarigione}{}\smallskip\noindent{\textbf{Preghiera di Guarigione}}\pdfbookmark[3]{Preghiera di Guarigione}{Preghiera di Guarigione}
\noindent
\begin{description}[noitemsep, topsep=0pt, parsep=0pt, partopsep=0pt, leftmargin=0cm, labelwidth=2.8cm]
	\item[\textbf{Lista di Magia}]: Cura
	\item[\textbf{Livello}]: 2, Comune
	\item[\textbf{T. di Lancio}]: 10 minuti
	\item[\textbf{Gittata}]: 9 metri
	\item[\textbf{Componenti}]: V
	\item[\textbf{Durata}]: Istantanea
\end{description}

Fino a sei creature a gittata che puoi vedere, scelte da te, recuperano ciascuna Punti Ferita pari a 2d6 + il tuo modificatore di caratteristica per incantesimi. Questo incantesimo causa lo stesso ammontare di danno sui non morti.

\textbf{Per ogni Successo Critico Magico} ottenuto nella Prova di Magia la cura aumenta di 1d8.

\smallskip\noindent\rule{\linewidth}{2pt} \index[Incantesimi]{Presagio}\hypertarget{Presagio}{}\smallskip\noindent{\textbf{Presagio}}\pdfbookmark[3]{Presagio}{Presagio}
\noindent
\begin{description}[noitemsep, topsep=0pt, parsep=0pt, partopsep=0pt, leftmargin=0cm, labelwidth=2.8cm]
	\item[\textbf{Lista di Magia}]: Divinazione
	\item[\textbf{Livello}]: 2, Comune
	\item[\textbf{T. di Lancio}]: 1 minuto
	\item[\textbf{Gittata}]: Personale
	\item[\textbf{Componenti}]: V, S, M (dei bastoncini, ossa o simili oggetti marchiati appositamente e del valore di almeno 25 mo)
	\item[\textbf{Durata}]: Istantanea
\end{description}

Gettando bastoncini intarsiati con gemme, facendo rotolare ossa di drago, impilando carte elaborate o impiegando qualche altro strumento di divinazione, ricevi un presagio da un'entità ultraterrena riguardo il risultato di uno specifico corso di azione che intendi intraprendere nei prossimi 30 minuti. Il Narratore sceglie tra i seguenti presagi:

\medskip

- Prosperità, per i risultati positivi

- Calamità, per i risultati negativi

- Prosperità e calamità, per i risultati sia positivi che negativi

- Nulla, per i risultati che non sono né particolarmente positivi né negativi

L'incantesimo non tiene conto di ogni possibile circostanza che possa modificare il risultato, come il lancio di ulteriori incantesimi o la perdita o l'arrivo di un alleato. Se lanci l'incantesimo due o più volte prima che sia sorto il nuovo sole, c'è una probabilità cumulativa del 25\% che per ogni lancio dopo il primo tu ottenga una lettura erronea. Il Narratore effettua questo tiro in segreto.

\smallskip\noindent\rule{\linewidth}{2pt} \index[Incantesimi]{Prestidigitazione}\hypertarget{Prestidigitazione}{}\smallskip\noindent{\textbf{Prestidigitazione}}\pdfbookmark[3]{Prestidigitazione}{Prestidigitazione}
\noindent
\begin{description}[noitemsep, topsep=0pt, parsep=0pt, partopsep=0pt, leftmargin=0cm, labelwidth=2.8cm]
	\item[\textbf{Lista di Magia}]: Universale
	\item[\textbf{Livello}]: 0, Comune
	\item[\textbf{T. di Lancio}]: 2 Azioni
	\item[\textbf{Gittata}]: 3 metri
	\item[\textbf{Componenti}]: V, S
	\item[\textbf{Durata}]: Massimo 1 ora
\end{description}

Questo incantesimo è un trucco magico minore che gli incantatori novizi impiegano per fare pratica. Crei a gittata uno dei seguenti effetti magici:

\begin{itemize}[leftmargin=*] \setlength{\itemsep}{0pt}
	\item Crei un effetto sensoriale innocuo e istantaneo come una pioggia di scintille, un soffio di vento, una debole nota musicale o uno strano odore.
	\item Illumini o spegni istantaneamente una candela, una torcia o piccolo fuoco da campo.
	\item Ripulisci o insozzi istantaneamente un oggetto non più grosso di 30 cm di lato.
	\item Raffreddi, riscaldi o insapori per 1 ora un cubo di 30 cm di lato di materiale non vivente.
	\item Fai comparire per 1 ora un colore, un piccolo segno o un simbolo su di un oggetto o una superficie.
	\item Crei un ninnolo non magico o un'immagine illusoria che entri nella tua mano e che resta fino al termine del tuo prossimo round.
\end{itemize}

Se lanci questo incantesimo più volte, puoi tenere attivi fino a tre effetti non istantanei alla volta, e puoi interrompere uno di questi effetti con un'Azione.

\textbf{Per ogni Successo Critico Magico} ottenuto nella Prova di Magia puoi attivare un effetto magico in più.

\smallskip\noindent\rule{\linewidth}{2pt} \index[Incantesimi]{Previsione}\hypertarget{Previsione}{}\smallskip\noindent{\textbf{Previsione}}\pdfbookmark[3]{Previsione}{Previsione}
\noindent
\begin{description}[noitemsep, topsep=0pt, parsep=0pt, partopsep=0pt, leftmargin=0cm, labelwidth=2.8cm]
	\item[\textbf{Lista di Magia}]: Divinazione
	\item[\textbf{Livello}]: 9, Non Comune
	\item[\textbf{T. di Lancio}]: 1 minuto
	\item[\textbf{Gittata}]: Contatto
	\item[\textbf{Componenti}]: V, S, M (una piuma di colibrì)
	\item[\textbf{Durata}]: 8 ore
\end{description}

Lanci l'incantesimo a contatto di una creatura consenziente per conferirle una limitata capacità di vedere nell'immediato futuro. Per la durata, il bersaglio non può essere sorpreso e ha +1d6 sui Tiri per Colpire, prove su competenze di base e Tiri Salvezza. Inoltre, sempre per la durata, le altre creature hanno -1d6 sui Tiri per Colpire contro il bersaglio. L'incantesimo ha immediatamente termine se lo lanci di nuovo prima che la sua durata abbia fine.

\smallskip\noindent\rule{\linewidth}{2pt} \index[Incantesimi]{Produrre Fiamma}\hypertarget{Produrre Fiamma}{}\smallskip\noindent{\textbf{Produrre Fiamma}}\pdfbookmark[3]{Produrre Fiamma}{Produrre Fiamma}
\noindent
\begin{description}[noitemsep, topsep=0pt, parsep=0pt, partopsep=0pt, leftmargin=0cm, labelwidth=2.8cm]
	\item[\textbf{Lista di Magia}]: Fuoco
	\item[\textbf{Livello}]: 0, Comune
	\item[\textbf{T. di Lancio}]: 1 Azione
	\item[\textbf{Gittata}]: Personale
	\item[\textbf{Componenti}]: V, S
	\item[\textbf{Durata}]: 10 minuti
\end{description}

Una fiammella compare nella tua mano. La fiammella resta lì per la durata dell'incantesimo e non danneggia né te né il tuo equipaggiamento. La fiamma produce luce fioca nel raggio di 1 metro. L'incantesimo termina se lo interrompi con un'Azione o se lo lanci di nuovo.

Puoi usare la fiamma anche per attaccare, sebbene farlo ponga termine all'incantesimo. Quando lanci questo incantesimo, o con un'Azione in un round successivo, puoi scagliare la fiamma a una creatura entro 9 metri da te. Effettua un attacco a distanza con incantesimo. Se colpisci, il bersaglio subisce 1d8 danni da fuoco.

Il danno dell'incantesimo aumenta di 1d8 quando raggiungi CM 5, CM 11 e CM 17, ma costa 2 Azioni lanciarlo potenziato e 1 Punti Magia, è altresì necessario avere preso Adepto della Magia un numero di volte pari ai potenziamenti che si vogliono applicare.

\textbf{Per ogni Successo Critico Magico} ottenuto nella Prova di Magia puoi attaccare una creatura in più senza terminare l'incantesimo.

\smallskip\noindent\rule{\linewidth}{2pt} \index[Incantesimi]{Profumo di Atherim}\hypertarget{Profumo di Atherim}{}\smallskip\noindent{\textbf{Profumo di Atherim}}\pdfbookmark[3]{Profumo di Atherim}{Profumo di Atherim}\label{Aura of Purity}
\noindent
\begin{description}[noitemsep, topsep=0pt, parsep=0pt, partopsep=0pt, leftmargin=0cm, labelwidth=2.8cm]
	\item[\textbf{Lista di Magia}] : Cura
	\item[\textbf{Livello}] : 4, Raro
	\item[\textbf{T. di Lancio}] : 2 Azioni
	\item[\textbf{Gittata}] : Personale
	\item[\textbf{Componenti}] : V, S, M (acqua benedetta)
	\item[\textbf{Durata}] : Concentrazione, fino a 10 minuti
\end{description}

Un profumo si irradia da te in un raggio di 4 metri per tutta la durata. Mentre si trovano in questo profumo tu e i tuoi alleati avete Resistenza ai danni da veleno e +4 ai Tiri Salvezza per evitare o terminare effetti che includono le condizioni di Accecato, Affascinato, Assordato, Spaventato, Paralizzato, Avvelenato o Stordito.

\textbf{Per ogni Successo Critico Magico} ottenuto nella Prova di Magia allarghi il raggio del profumo di 1 metro.

\smallskip\noindent\rule{\linewidth}{2pt} \index[Incantesimi]{Proibizione}\hypertarget{Proibizione}{}\smallskip\noindent{\textbf{Proibizione}}\pdfbookmark[3]{Proibizione}{Proibizione}
\noindent
\begin{description}[noitemsep, topsep=0pt, parsep=0pt, partopsep=0pt, leftmargin=0cm, labelwidth=2.8cm]
	\item[\textbf{Lista di Magia}]: Abiurazione
	\item[\textbf{Livello}]: 6, Non Comune
	\item[\textbf{T. di Lancio}]: 10 minuti
	\item[\textbf{Gittata}]: Contatto
	\item[\textbf{Componenti}]: V, S, M (uno spruzzo di Acqua santa, incensi rari, e un rubino in polvere del valore di 1000 mo)
	\item[\textbf{Durata}]: 1 giorno
\end{description}

Crei una interdizione al viaggio magico che protegge fino a 4000 metri quadri di pavimento, fino a un'altezza di 9 metri dal suolo. Per la durata dell'incantesimo, le creature non possono teletrasportarsi nell'area o usare passaggi, come quello creato dall'incantesimo portale, per entrare nell'area. L'incantesimo protegge l'area dal viaggio planare, e quindi impedisce alle creature di accedere all'area tramite il Piano Astrale, il Piano Etereo od il Piano delle Ombre.

Inoltre, l'incantesimo danneggia i tipi di creatura scelti da te durante il lancio. Scegli uno o più dei seguenti: celestiali, elementali, fatati, demoni e non morti. Quando una creatura selezionata entra nell'area dell'incantesimo per la prima volta in un round o inizia qui il suo round, la creatura subisce 5d10 danni da Luce o da Vuoto (a tua scelta, quando lanci l'incantesimo).

Quando lanci questo incantesimo, puoi stabilire una parola d'ordine. Una creatura che pronuncia la parola d'ordine mentre entra nell'area dell'incantesimo, non subisce danni da esso.

L'area dell'incantesimo non può sovrapporsi all'area di un altro incantesimo proibizione. Se esegui proibizione ogni giorno per 30 giorni nello stesso posto, l'incantesimo durerà finché non viene dissolto, e le componenti materiali saranno consumate durante l'ultimo lancio.

\smallskip\noindent\rule{\linewidth}{2pt} \index[Incantesimi]{Protezione dall'Energia}\hypertarget{Protezione dall'Energia}{}\smallskip\noindent{\textbf{Protezione dall'Energia}}\pdfbookmark[3]{Protezione dall'Energia}{Protezione dall'Energia}
\noindent
\begin{description}[noitemsep, topsep=0pt, parsep=0pt, partopsep=0pt, leftmargin=0cm, labelwidth=2.8cm]
	\item[\textbf{Lista di Magia}]: Abiurazione
	\item[\textbf{Livello}]: 3, Comune
	\item[\textbf{T. di Lancio}]: 2 Azioni
	\item[\textbf{Gittata}]: Contatto
	\item[\textbf{Componenti}]: V, S
	\item[\textbf{Durata}]: 10 minuti
\end{description}

Lanci l'incantesimo a contatto di una creatura consenziente. Per la durata dell'incantesimo, il bersaglio ha resistenza a un tipo di danno scelto da te: acido, freddo, fuoco, fulmine o suono. Puoi sacrificare tutta la durata dell'incantesimo, terminandolo, per annullare completamente il danno subito da una fonte di energia.

\textbf{Per ogni Successo Critico Magico} ottenuto nella Prova di Magia puoi influenzare un altra persona o aumentare la durata di 10 minuti.

\smallskip\noindent\rule{\linewidth}{2pt} \index[Incantesimi]{Protezione dall'Energia minore}\hypertarget{Protezione dall'Energia minore}{}\smallskip\noindent{\textbf{Protezione dall'Energia minore}}\pdfbookmark[3]{Protezione dall'Energia minore}{Protezione dall'Energia minore}
\noindent
\begin{description}[noitemsep, topsep=0pt, parsep=0pt, partopsep=0pt, leftmargin=0cm, labelwidth=2.8cm]
	\item[\textbf{Lista di Magia}]: Abiurazione
	\item[\textbf{Livello}]: 1, Raro
	\item[\textbf{T. di Lancio}]: 1 Reazione
	\item[\textbf{Gittata}]: Contatto
	\item[\textbf{Componenti}]: V, S
	\item[\textbf{Durata}]: 1 minuto
\end{description}

Lanci l'incantesimo a contatto di una creatura consenziente. Per la durata dell'incantesimo, il bersaglio ha Riduzione al Danno dall'energia scelta pari a 5. Puoi sacrificare tutta la durata dell'incantesimo, terminandolo, per ridurre di 20 il danno subito da una fonte di energia (come se avessi Resistenza al Danno 20 da quella fonte di energia).

\textbf{Per ogni Successo Critico Magico} ottenuto nella Prova di Magia puoi influenzare un altra persona o aumentare la durata di 1 minuto.

\smallskip\noindent\rule{\linewidth}{2pt} \index[Incantesimi]{Protezione dai Veleni}\hypertarget{Protezione dai Veleni}{}\smallskip\noindent{\textbf{Protezione dai Veleni}}\pdfbookmark[3]{Protezione dai Veleni}{Protezione dai Veleni}
\noindent
\begin{description}[noitemsep, topsep=0pt, parsep=0pt, partopsep=0pt, leftmargin=0cm, labelwidth=2.8cm]
	\item[\textbf{Lista di Magia}]: Abiurazione
	\item[\textbf{Livello}]: 2, Non Comune
	\item[\textbf{T. di Lancio}]: 2 Azioni
	\item[\textbf{Gittata}]: Contatto
	\item[\textbf{Componenti}]: V, S
	\item[\textbf{Durata}]: 1 ora
\end{description}

Per la durata dell'incantesimo, il bersaglio ha +1d6 ai Tiri Salvezza contro l'essere avvelenato, e ha resistenza al danno da veleno.

\textbf{In caso di due Successo Critico Magico ottenuto} nella Prova di Magia puoi annullare un veleno in circolo sul bersaglio.

\smallskip\noindent\rule{\linewidth}{2pt} \index[Incantesimi]{Punizione Marchiante}\hypertarget{Punizione Marchiante}{}\smallskip\noindent{\textbf{Punizione Marchiante}}\pdfbookmark[3]{Punizione Marchiante}{Punizione Marchiante}
\noindent
\begin{description}[noitemsep, topsep=0pt, parsep=0pt, partopsep=0pt, leftmargin=0cm, labelwidth=2.8cm]
	\item[\textbf{Lista di Magia}]: Invocazione
	\item[\textbf{Livello}]: 2, Comune
	\item[\textbf{T. di Lancio}]: 1 Azione
	\item[\textbf{Gittata}]: Personale
	\item[\textbf{Componenti}]: V
	\item[\textbf{Durata}]: 1 minuto
\end{description}

La prossima volta che colpisci una creatura con un attacco in mischia con arma nella durata dell'incantesimo, l'arma riluce di un bagliore magico mentre colpisci. L'attacco infligge 1d6 danni da Luce aggiuntivi al bersaglio, che diventa visibile qualora sia invisibile ed emette luce fioca in un raggio di 1 metro. Inoltre il bersaglio non può diventare invisibile fino al termine dell'incantesimo.

\textbf{Per ogni Successo Critico Magico} ottenuto nella Prova di Magia il danno aggiuntivo aumenta di 1d6.

\smallskip\noindent\rule{\linewidth}{2pt} \index[Incantesimi]{Purificare Cibo e Bevande}\hypertarget{Purificare Cibo e Bevande}{}\smallskip\noindent{\textbf{Purificare Cibo e Bevande}}\pdfbookmark[3]{Purificare Cibo e Bevande}{Purificare Cibo e Bevande}
\noindent
\begin{description}[noitemsep, topsep=0pt, parsep=0pt, partopsep=0pt, leftmargin=0cm, labelwidth=2.8cm]
	\item[\textbf{Lista di Magia}]: Animali e Piante
	\item[\textbf{Livello}]: 1, Comune
	\item[\textbf{T. di Lancio}]: 2 Azioni
	\item[\textbf{Gittata}]: 3 metri
	\item[\textbf{Componenti}]: V, S
	\item[\textbf{Durata}]: Istantanea
\end{description}

Tutti i cibi e le bevande non magiche in una sfera di 1 metro di raggio, centrata in un punto a gittata di tua scelta, vengono purificati e liberati da veleni e malattie. Un cibo in decomposizione viene ripulito e reso commestibile.

\smallskip\noindent\rule{\linewidth}{2pt} \index[Incantesimi]{Raggio di Gelo}\hypertarget{Raggio di Gelo}{}\smallskip\noindent{\textbf{Raggio di Gelo}}\pdfbookmark[3]{Raggio di Gelo}{Raggio di Gelo}
\noindent
\begin{description}[noitemsep, topsep=0pt, parsep=0pt, partopsep=0pt, leftmargin=0cm, labelwidth=2.8cm]
	\item[\textbf{Lista di Magia}]: Acqua
	\item[\textbf{Livello}]: 0, Comune
	\item[\textbf{T. di Lancio}]: 1 Azione
	\item[\textbf{Gittata}]: 18 metri
	\item[\textbf{Componenti}]: V, S
	\item[\textbf{Durata}]: Istantanea
\end{description}

Un fascio gelato di luce azzurra colpisce una creatura a gittata. Effettua un attacco a distanza con incantesimo contro il bersaglio. Se colpisci, egli subisce 1d8 danni da freddo, e la sua velocità è ridotta di 3 metri fino all'inizio del tuo prossimo round.

Puoi aumentare il danno dell'incantesimo di 1d8 quando raggiungi CM 5, CM 11 e CM 17, ma costa 2 Azioni lanciarlo potenziato e 1 Punti Magia, è altresì necessario avere preso Adepto della Magia un numero di volte pari ai potenziamenti che si vogliono applicare.

\textbf{Per ogni due Successo Critico Magico ottenuto} nella Prova di Magia crei un fascio gelato aggiuntivo.

\smallskip\noindent\rule{\linewidth}{2pt} \index[Incantesimi]{Raggio di Indebolimento}\hypertarget{Raggio di Indebolimento}{}\smallskip\noindent{\textbf{Raggio di Indebolimento}}\pdfbookmark[3]{Raggio di Indebolimento}{Raggio di Indebolimento}
\noindent
\begin{description}[noitemsep, topsep=0pt, parsep=0pt, partopsep=0pt, leftmargin=0cm, labelwidth=2.8cm]
	\item[\textbf{Lista di Magia}]: Necromanzia
	\item[\textbf{Livello}]: 1, Comune
	\item[\textbf{T. di Lancio}]: 2 Azioni
	\item[\textbf{Gittata}]: 18 metri
	\item[\textbf{Componenti}]: V, S
	\item[\textbf{Durata}]: 1 minuto
\end{description}

Un fascio nero di energia debilitante parte dal tuo dito diretto contro una creatura a gittata. Effettua un attacco a distanza con incantesimo contro il bersaglio. Se colpisci, il bersaglio, quando attacca con un arma che usa la Forza come modificatore, tirerà due volte per il danno prendendo il risultato inferiore fino al termine dell'incantesimo. Non si può essere influenzati da più di un Raggio di Indebolimento al giorno.

\textbf{Per ogni due Successo Critico Magico ottenuto} nella Prova di Magia aumenti di 1 il livello di Affaticamento del bersaglio.

\smallskip\noindent\rule{\linewidth}{2pt} \index[Incantesimi]{Raggio mortale}\hypertarget{Raggio mortale}{}\smallskip\noindent{\textbf{Raggio mortale}}\pdfbookmark[3]{Raggio mortale}{Raggio mortale}
\noindent
\begin{description}[noitemsep, topsep=0pt, parsep=0pt, partopsep=0pt, leftmargin=0cm, labelwidth=2.8cm]
	\item[\textbf{Lista di Magia}]: Necromanzia
	\item[\textbf{Livello}]: 2, Raro
	\item[\textbf{T. di Lancio}]: 2 Azioni
	\item[\textbf{Gittata}]: 6 metri
	\item[\textbf{Componenti}]: V, S, M (del sangue rappreso)
	\item[\textbf{Durata}]: Istantaneo
\end{description}

Un fascio nero di energia crepitante parte dalle tue mani verso una creatura a gittata. Effettua un attacco a distanza con incantesimo contro il bersaglio. Indipendentemente dal fatto se colpisci o meno, il bersaglio deve effettuare un Tiro Salvezza su Tempra.

Se il Tiro per Colpire va a segno la creatura effettua il Tiro Salvezza con -2 di penalità, se il Tiro per Colpire genera un Tiro Critico il Tiro Salvezza viene fatto con -4 di penalità.

Se manchi il Tiro Salvezza viene fatto senza modificatori aggiuntivi.

La condizione Affaticato della creatura aumenta di 1

\textbf{Per ogni tre Successo Critico Magico ottenuto} nella Prova di Magia aumenti di 1 il livello di Affaticamento del bersaglio.

\smallskip\noindent\rule{\linewidth}{2pt} \index[Incantesimi]{Raggio Rovente}\hypertarget{Raggio Rovente}{}\smallskip\noindent{\textbf{Raggio Rovente}}\pdfbookmark[3]{Raggio Rovente}{Raggio Rovente}
\noindent
\begin{description}[noitemsep, topsep=0pt, parsep=0pt, partopsep=0pt, leftmargin=0cm, labelwidth=2.8cm]
	\item[\textbf{Lista di Magia}]: Fuoco
	\item[\textbf{Livello}]: 2, Comune
	\item[\textbf{T. di Lancio}]: 2 Azioni
	\item[\textbf{Gittata}]: 36 metri
	\item[\textbf{Componenti}]: V, S
	\item[\textbf{Durata}]: Istantanea
\end{description}

Crei tre raggi di fuoco e li proietti verso tre bersagli a gittata. Puoi proiettarli contro lo stesso bersaglio o bersagli diversi. Effettua un attacco a distanza con incantesimo per ciascun raggio. Se colpisci, il bersaglio subisce 2d6 danni da fuoco.

\textbf{Per ogni Successo Critico Magico} ottenuto nella Prova di Magia crei un raggio aggiuntivo.

\smallskip\noindent\rule{\linewidth}{2pt} \index[Incantesimi]{Ragnatela}\hypertarget{Ragnatela}{}\smallskip\noindent{\textbf{Ragnatela}}\pdfbookmark[3]{Ragnatela}{Ragnatela}
\noindent
\begin{description}[noitemsep, topsep=0pt, parsep=0pt, partopsep=0pt, leftmargin=0cm, labelwidth=2.8cm]
	\item[\textbf{Lista di Magia}]: Animali e Piante
	\item[\textbf{Livello}]: 2, Comune
	\item[\textbf{T. di Lancio}]: 2 Azioni
	\item[\textbf{Gittata}]: 18 metri
	\item[\textbf{Componenti}]: V, S, M (un pezzo di tela di ragno)
	\item[\textbf{Durata}]: 1 ora
\end{description}

Evochi una spessa massa di tela densa e appiccicosa in un punto a gittata, scelto da te. Per la durata, la ragnatela riempie una sfera di 3 metri di raggio dal punto designato. La ragnatela è terreno difficile e rende quell'area oscurata leggermente.

Se la tela non è ancorate tra due masse solide (come pareti o alberi) o stesa lungo un pavimento, parete o soffitto, la ragnatela evocata crolla su se stessa, e l'incantesimo termina all'inizio del tuo prossimo round. Le tele distese su di una superficie piatta hanno una profondità di 1 metro.

Ogni creatura che inizia il suo round nella ragnatela o che vi entra durante il proprio round deve effettuare un Tiro Salvezza su Riflessi. Se lo fallisce, la creatura è intralciata finché rimane nella ragnatela o finché non si libera.

Una creatura intralciata dalle ragnatele può usare 2 Azioni per effettuare un nuovo Tiro Salvezza. Se lo supera, non è più intralciata.

La ragnatela è infiammabile e se esposta alle fiamme prende fuoco immediatamente e brucia per 2 round causando a ogni creatura dentro la sua area 2d4 di danno da fuoco.

\smallskip\noindent\rule{\linewidth}{2pt} \index[Incantesimi]{Randello Incantato}\hypertarget{Randello Incantato}{}\smallskip\noindent{\textbf{Randello Incantato}}\pdfbookmark[3]{Randello Incantato}{Randello Incantato}
\noindent
\begin{description}[noitemsep, topsep=0pt, parsep=0pt, partopsep=0pt, leftmargin=0cm, labelwidth=2.8cm]
	\item[\textbf{Lista di Magia}]: Animali e Piante
	\item[\textbf{Livello}]: 0, Comune
	\item[\textbf{T. di Lancio}]: 1 Azione
	\item[\textbf{Gittata}]: Contatto
	\item[\textbf{Componenti}]: V, S, M (vischio, una foglia di quadrifoglio, e una randello o bastone da combattimento)
	\item[\textbf{Durata}]: 1 minuto
\end{description}

Il legno di un randello o bastone da combattimento che stai impugnando viene infuso del potere della natura. Per la durata dell'incantesimo, usando quell'arma puoi usare la tua caratteristica da incantatore al posto della Forza per il danno e Tiri per Colpire, il dado di danno dell'arma diventa un d8. L'arma diventa anche magica, se già non lo è. L'incantesimo ha termine se lo lanci di nuovo o se lasci l'arma.

\textbf{Per ogni Successo Critico Magico} ottenuto nella Prova di Magia la durata raddoppia oppure aggiungi un +1 al danno.

\smallskip\noindent\rule{\linewidth}{2pt} \index[Incantesimi]{Reggia Meravigliosa}\hypertarget{Reggia Meravigliosa}{}\smallskip\noindent{\textbf{Reggia Meravigliosa}}\pdfbookmark[3]{Reggia Meravigliosa}{Reggia Meravigliosa}
\noindent
\begin{description}[noitemsep, topsep=0pt, parsep=0pt, partopsep=0pt, leftmargin=0cm, labelwidth=2.8cm]
	\item[\textbf{Lista di Magia}]: Evocazione
	\item[\textbf{Livello}]: 7, Leggendario
	\item[\textbf{T. di Lancio}]: 1 minuto
	\item[\textbf{Gittata}]: 90 metri
	\item[\textbf{Componenti}]: V, S, M (un portale in miniatura scolpito in avorio, un piccolo pezzo di marmo lucido, e un minuscolo cucchiaio d'argento, ciascuno di questi oggetti deve essere almeno del valore di 5 mo)
	\item[\textbf{Durata}]: 24 ore
\end{description}

Entro la gittata, evochi un'abitazione extradimensionale che rimane per la durata dell'incantesimo. Scegli dove è posizionato il suo portone d'ingresso. Il portone d'ingresso emette una lieve luminosità ed è largo 1 metro per 3 metri di altezza. Tu e tutte le creature da te indicate quando hai lanciato l'incantesimo potete entrare nell'abitazione extradimensionale, fino a quando il portone resta aperto. Puoi aprire o chiudere il portone se ti trovi entro 9 metri da esso. Mentre è chiuso, il portone è invisibile.

Oltre il portone si trova un magnifico ingresso, oltre il quale si dipanano numerose stanze. L'atmosfera è pulita, fresca e accogliente. Puoi creare quanti piani desideri, ma lo spazio non può eccedere 50 cubi ognuno di 3 metri di spigolo. Il luogo è ammobiliato e decorato come preferisci. Contiene cibo sufficiente a soddisfare un banchetto di 9 portate per 100 persone. Uno staff di 100 servitori quasi trasparenti è al servizio di chiunque vi faccia ingresso. Sta a te decidere l'aspetto visivo di questi servitori e il loro abbigliamento. Essi obbediscono assolutamente ai tuoi ordini. Ogni servitore può svolgere qualsiasi compito un normale servitore umano possa svolgere, ma non possono attaccare o effettuare alcuna azione che potrebbe arrecare direttamente danno a un'altra creatura. I servitori possono quindi raccogliere oggetti, pulire, riparare, ripiegare vestiti, accendere fuochi, servire cibi, versare vini e così via. I servitori possono recarsi in qualsiasi punto della dimora, ma non possono uscirne. I mobili e gli altri oggetti creati da questo incantesimo diventano fumo quando vengono portati fuori dalla dimora. Quando l'incantesimo termina, qualsiasi creatura all'interno dello spazio extradimensionale viene espulsa nello spazio aperto più vicino all'uscita.

\textbf{Per ogni Successo Critico Magico} ottenuto nella Prova di Magia la durata raddoppia o togli un mese dal conteggio per renderlo permanente.

\textbf{NOTA}: l'incantesimo lanciato per un anno tutti i giorni sempre nello stesso luogo diventa permanente.

\textbf{Per ogni Successo Critico Magico} ottenuto nella Prova di Magia la durata raddoppia o togli un mese dal conteggio per renderlo permanente.

\smallskip\noindent\rule{\linewidth}{2pt} \index[Incantesimi]{Regressione Mentale}\hypertarget{Regressione Mentale}{}\smallskip\noindent{\textbf{Regressione Mentale}}\pdfbookmark[3]{Regressione Mentale}{Regressione Mentale}
\noindent
\begin{description}[noitemsep, topsep=0pt, parsep=0pt, partopsep=0pt, leftmargin=0cm, labelwidth=2.8cm]
	\item[\textbf{Lista di Magia}]: Ammaliamento
	\item[\textbf{Livello}]: 8, Raro
	\item[\textbf{T. di Lancio}]: 2 Azioni
	\item[\textbf{Gittata}]: 45 metri
	\item[\textbf{Componenti}]: V, S, M (una manciata di sfere di argilla, cristallo, vetro o minerali)
	\item[\textbf{Durata}]: Istantanea
\end{description}

Assalti la mente di una creatura a gittata e che puoi vedere, cercando di frammentarne l'intelletto e la personalità. Il bersaglio subisce 4d6 danni e deve effettuare un Tiro Salvezza su Volontà. Se fallisce il Tiro Salvezza, i punteggi di Intelligenza e Carisma della creatura scendono a -4. La creatura non può lanciare incantesimi, attivare oggetti magici, comprendere linguaggi, o comunicare in alcun modo comprensibile. La creatura può, tuttavia, identificare i suoi amici, seguirli e anche proteggerli. Dopo 30 giorni, la creatura può ripetere il Tiro Salvezza contro l'incantesimo. Se lo supera, l'incantesimo ha termine se fallisce l'effetto è permanete.

L'incantesimo può essere terminato entro i 30 giorni da ristorare superiore, guarigione o desiderio.

\smallskip\noindent\rule{\linewidth}{2pt} \index[Incantesimi]{Reincarnazione}\hypertarget{Reincarnazione}{}\smallskip\noindent{\textbf{Reincarnazione}}\pdfbookmark[3]{Reincarnazione}{Reincarnazione}
\noindent
\begin{description}[noitemsep, topsep=0pt, parsep=0pt, partopsep=0pt, leftmargin=0cm, labelwidth=2.8cm]
	\item[\textbf{Lista di Magia}]: Animali e Piante
	\item[\textbf{Livello}]: 5, Raro
	\item[\textbf{T. di Lancio}]: 1 ora
	\item[\textbf{Gittata}]: Contatto
	\item[\textbf{Componenti}]: V, S, M (oli e unguenti rari del valore di almeno 1000 mo, che l'incantesimo consuma)
	\item[\textbf{Durata}]: Istantanea
\end{description}

Entri a contatto con un umanoide morto o un frammento di umanoide morto. Purché la creatura non sia morta da più di 10 giorni, l'incantesimo gli forma un nuovo corpo adulto e poi ne richiama l'anima affinché entri nel corpo. Se l'anima del bersaglio non è libera o consenziente a farlo, l'incantesimo fallisce.

La magia modella un nuovo corpo, che probabilmente provocherà un cambio di razza alla creatura. Il Narratore tira un d10 e consulta la seguente tabella per determinare quale forma assuma la creatura una volta riportata in vita, oppure sarà Il Narratore a scegliere la forma.

\medskip

\begin{tabular}{ll}
	\textbf{d10} &\textbf{Razza}\\
	\toprule
	0 & Lupo/Aquila/Volpe/Lince (tirate 1d4)\\
	1&Nano\\
	2&Elfo\\
	3&Mezzelfo\\
	4&Mezzorco\\
	5&Cinghiale/Tasso/Cane/Ratto (tirate 1d4)\\
	6&Nibali\\
	7&Orso/Gufo/Procione/Gatto (tirate 1d4)\\
	8&Umano\\
	9&Stessa razza precedente
\end{tabular}

\medskip

La creatura reincarnata ricorda la sua vita e le sue esperienze passate (stesso punteggio di CA e CM, Abilità e Competenze). Mantiene le capacità che aveva nella sua forma originale se è in grado di applicarle.

\textbf{Questo incantesimo non non è disponibile se non ai Devoti e Seguaci di Shayalia o Efrem}

\emph{NOTA}: un Devoto o Seguace di Shayalia od Efrem reincarnerà la creatura sempre in un animale, però potendo scegliere il tipo.

Non è possibile reincarnarsi in uno gnomo se non si era prima uno gnomo.

\smallskip\noindent\rule{\linewidth}{2pt} \index[Incantesimi]{Resistenza}\hypertarget{Resistenza}{}\smallskip\noindent{\textbf{Resistenza}}\pdfbookmark[3]{Resistenza}{Resistenza}
\noindent
\begin{description}[noitemsep, topsep=0pt, parsep=0pt, partopsep=0pt, leftmargin=0cm, labelwidth=2.8cm]
	\item[\textbf{Lista di Magia}]: Abiurazione
	\item[\textbf{Livello}]: 0, Comune
	\item[\textbf{T. di Lancio}]: 1 Reazione
	\item[\textbf{Gittata}]: Contatto
	\item[\textbf{Componenti}]: V, S, M (un mantello in miniatura)
	\item[\textbf{Durata}]: Istantanea
\end{description}

Lanci l'incantesimo a contatto con una creatura consenziente. Una volta prima del termine dell'incantesimo, il bersaglio può tirare un 1d4 e sommare il risultato ottenuto a un Tiro Salvezza a sua scelta. Può tirare il dado prima o dopo aver effettuato il Tiro Salvezza. Poi l'incantesimo termina. Non si può ricevere l'incantesimo Resistenza ad intervalli inferiori ad 1 ora.

\textbf{Per ogni Successo Critico Magico} ottenuto nella Prova di Magia puoi fare usufruire del bonus un altra creatura.

\smallskip\noindent\rule{\linewidth}{2pt} \index[Incantesimi]{Respirare Sott'Acqua}\hypertarget{Respirare Sott'Acqua}{}\smallskip\noindent{\textbf{Respirare Sott'Acqua}}\pdfbookmark[3]{Respirare Sott'Acqua}{Respirare Sott'Acqua}
\noindent
\begin{description}[noitemsep, topsep=0pt, parsep=0pt, partopsep=0pt, leftmargin=0cm, labelwidth=2.8cm]
	\item[\textbf{Lista di Magia}]: Acqua, Aria
	\item[\textbf{Livello}]: 3, Comune
	\item[\textbf{T. di Lancio}]: 2 Azioni
	\item[\textbf{Gittata}]: 9 metri
	\item[\textbf{Componenti}]: V, S, M (una cannuccia o una pagliuzza)
	\item[\textbf{Durata}]: 1 Ora per CM
\end{description}

Puoi dividere equamente il totale della durata dell'incantesimo tra CM/2 creature consenzienti.

Questo incantesimo consente  alle creature selezionate entro gittata di respirare sott'acqua per la durata concessa. Le creature soggette mantengono anche il loro normale metodo di respirazione.

\textbf{Per ogni Successo Critico Magico} ottenuto nella Prova di Magia hai un bonus di +2 per la verifica della tua CM.

\smallskip\noindent\rule{\linewidth}{2pt} \index[Incantesimi]{Rigenerazione}\hypertarget{Rigenerazione}{}\smallskip\noindent{\textbf{Rigenerazione}}\pdfbookmark[3]{Rigenerazione}{Rigenerazione}
\noindent
\begin{description}[noitemsep, topsep=0pt, parsep=0pt, partopsep=0pt, leftmargin=0cm, labelwidth=2.8cm]
	\item[\textbf{Lista di Magia}]: Trasmutazione
	\item[\textbf{Livello}]: 7, Leggendario
	\item[\textbf{T. di Lancio}]: 1 minuto
	\item[\textbf{Gittata}]: Contatto
	\item[\textbf{Componenti}]: V, S, M (un rosario e Acqua santa)
	\item[\textbf{Durata}]: 1 ora
\end{description}

Lanci l'incantesimo a contatto di una creatura per stimolare la sua capacità di guarigione naturale. Il bersaglio recupera 4d8 + 15 Punti Ferita. Per la durata dell'incantesimo, il bersaglio recupera 10 Punti Ferita all'inizio di ciascun suo round. Le membra recise del corpo del bersaglio (dita, gambe, code e così via), se ne ha, vengono ripristinate in 2 minuti. Se hai la parte recisa e la tieni appoggiata al moncherino, l'incantesimo fa sì che l'arto si ricucia in 3 round col moncherino.

\textbf{Per ogni Successo Critico Magico} ottenuto nella Prova di Magia raddoppi i Punti Ferita recuperati per round.

\smallskip\noindent\rule{\linewidth}{2pt} \index[Incantesimi]{Rimuovi Malattia}\hypertarget{Rimuovi Malattia}{}\smallskip\noindent{\textbf{Rimuovi Malattia}}
\pdfbookmark[3]{Rimuovi Malattia}{Rimuovi Malattia}\label{rimuovimalattie}\hypertarget{rimuovimalattie}{}
\noindent
\begin{description}[noitemsep, topsep=0pt, parsep=0pt, partopsep=0pt, leftmargin=0cm, labelwidth=2.8cm]
	\item[\textbf{Lista di Magia}]: Cura
	\item[\textbf{Livello}]: 2, Comune
	\item[\textbf{T. di Lancio}]: 1 turno
	\item[\textbf{Gittata}]: Contatto
	\item[\textbf{Componenti}]: V, S
	\item[\textbf{Durata}]: Istantanea
\end{description}

Puoi porre fine a una malattia naturale. In caso di malattie magiche la tua DC di incantesimo deve essere superiore alla DC della malattia.

\textbf{Per ogni Successo Critico Magico} ottenuto nella Prova di Magia puoi guarire una persona in più oppure considerare un +4 per superare la DC della malattia.

\smallskip\noindent\rule{\linewidth}{2pt} \index[Incantesimi]{Rimuovi Maledizione}\hypertarget{Rimuovi Maledizione}{}\smallskip\noindent{\textbf{Rimuovi Maledizione}}\pdfbookmark[3]{Rimuovi Maledizione}{Rimuovi Maledizione}
\noindent
\begin{description}[noitemsep, topsep=0pt, parsep=0pt, partopsep=0pt, leftmargin=0cm, labelwidth=2.8cm]
	\item[\textbf{Lista di Magia}]: Abiurazione
	\item[\textbf{Livello}]: 3, Comune
	\item[\textbf{T. di Lancio}]: 2 Azioni
	\item[\textbf{Gittata}]: Contatto
	\item[\textbf{Componenti}]: V, S
	\item[\textbf{Durata}]: Istantanea
\end{description}

Se l'oggetto o persona è stato maledetto tramite l'incantesimo \hyperlink{Scagliare Maledizione}{Scagliare Maledizione}, o comunque il Narratore decide che l'oggetto ha una maledizione particolare allora la DC di chi lancia Rimuovi Maledizione deve essere superiore a quella della Maledizione.

\textbf{Per ogni Successo Critico Magico} ottenuto nella Prova di Magia puoi guarire una persona in più oppure considerare un +4 per superare la DC della maledizione.

Sia che sia stato sufficiente lanciare l'incantesimo oppure sia stato lanciato con una Prova di Magia, la maledizione resta, ma l'incantesimo permette di rimuovere l'oggetto e gettarlo.


\smallskip\noindent\rule{\linewidth}{2pt} \index[Incantesimi]{Rimuovi Paura}\hypertarget{Rimuovi Paura}{}\smallskip\noindent{\textbf{Rimuovi Paura}}\pdfbookmark[3]{Rimuovi Paura}{Rimuovi Paura}
\noindent
\begin{description}[noitemsep, topsep=0pt, parsep=0pt, partopsep=0pt, leftmargin=0cm, labelwidth=2.8cm]
	\item[\textbf{Lista di Magia}]: Abiurazione
	\item[\textbf{Livello}]: 1, Comune
	\item[\textbf{T. di Lancio}]: 1 Azione
	\item[\textbf{Gittata}]: Tocco
	\item[\textbf{Componenti}]: V, S
	\item[\textbf{Durata}]: 2 turni
\end{description}

Questo incantesimo infonde coraggio nel soggetto e può rimuovere gli effetti della paura indotta magicamente permettendogli di fare un nuovo Tiro Salvezza. Il soggetto toccato riceve un bonus al Tiro Salvezza di +1 per volte che l'incantatore ha preso Adepto della Magia.

\textbf{Per ogni Successo Critico Magico} ottenuto nella Prova di Magia il soggetto prende un +2 al Tiro Salvezza.


\smallskip\noindent\rule{\linewidth}{2pt} \index[Incantesimi]{Rimuovi Veleno}\hypertarget{Rimuovi Veleno}{}\smallskip\noindent{\textbf{Rimuovi Veleno}}\pdfbookmark[3]{Rimuovi Veleno}{Rimuovi Veleno}\label{incrimuoviveleno}\hypertarget{incrimuoviveleno}{}
\noindent
\begin{description}[noitemsep, topsep=0pt, parsep=0pt, partopsep=0pt, leftmargin=0cm, labelwidth=2.8cm]
	\item[\textbf{Lista di Magia}]: Acqua, Cura
	\item[\textbf{Livello}]: 3, Comune
	\item[\textbf{T. di Lancio}]: 1 Round
	\item[\textbf{Gittata}]: Contatto
	\item[\textbf{Componenti}]: V, S
	\item[\textbf{Durata}]: Istantanea
\end{description}

Puoi porre fine ad un veleno naturale. In caso di veleni magici la tua DC di incantesimo deve essere superiore alla DC (o Tiro Salvezza) del veleno.

\textbf{Per ogni Successo Critico Magico} ottenuto nella Prova di Magia aggiungi +4 alla propria DC per capire se ha superato quella dello del veleno.

\smallskip\noindent\rule{\linewidth}{2pt} \index[Incantesimi]{Rinascita}\hypertarget{Rinascita}{}\smallskip\noindent{\textbf{Rinascita}}\pdfbookmark[3]{Rinascita}{Rinascita}
\noindent
\begin{description}[noitemsep, topsep=0pt, parsep=0pt, partopsep=0pt, leftmargin=0cm, labelwidth=2.8cm]
	\item[\textbf{Lista di Magia}]: Cura, Necromanzia
	\item[\textbf{Livello}]: 3, Molto Raro
	\item[\textbf{T. di Lancio}]: 10 Minuti
	\item[\textbf{Gittata}]: Contatto
	\item[\textbf{Componenti}]: V, S, M (diamante del valore di 300 mo, che l'incantesimo consuma)
	\item[\textbf{Durata}]: Istantanea
\end{description}

Una creatura morta nell'ultimo minuto e con cui sei in contatto, ritorna in vita con 1 punto ferita e nessun Punto Magia. Questo incantesimo non può riportare in vita le persone morte di vecchiaia, né può ripristinare le parti del corpo mancanti.

La creatura riportata in vita deve effettuare un Tiro Salvezza su Tempra a DC 15 oppure per il trauma subito non torna in vita, se torna in vita è Affaticato 3.

\textbf{NOTA}: a discrezione del Narratore questo potrebbe essere l'unico incantesimo concesso per riportare in vita una creatura, altrimenti vale la regola che solo un Patrono può riportare in vita.

\smallskip\noindent\rule{\linewidth}{2pt} \index[Incantesimi]{Riparare}\hypertarget{Riparare}{}\smallskip\noindent{\textbf{Riparare}}\pdfbookmark[3]{Riparare}{Riparare}
\noindent
\begin{description}[noitemsep, topsep=0pt, parsep=0pt, partopsep=0pt, leftmargin=0cm, labelwidth=2.8cm]
	\item[\textbf{Lista di Magia}]: Terra
	\item[\textbf{Livello}]: 0, Comune
	\item[\textbf{T. di Lancio}]: 1 minuto
	\item[\textbf{Gittata}]: Contatto
	\item[\textbf{Componenti}]: V, S, M (due calamite)
	\item[\textbf{Durata}]: Istantanea
\end{description}

Questo incantesimo ripara una singola rottura o spaccatura in un oggetto con cui sei a contatto, come una catenella spezzata, due metà di una chiave rotta, un mantello lacerato, o un otre che perde. Purché la rottura o la spaccatura non sia più grande di 30 centimetri in qualsiasi dimensione, sei in grado di ripararle, senza lasciare traccia dei danni subiti. Questo incantesimo può riparare fisicamente un oggetto magico o un costrutto, ma non è in grado di ripristinare le funzioni magiche di questi oggetti.

\smallskip\noindent\rule{\linewidth}{2pt} \index[Incantesimi]{Riposo Inviolato}\hypertarget{Riposo Inviolato}{}\smallskip\noindent{\textbf{Riposo Inviolato}}\pdfbookmark[3]{Riposo Inviolato}{Riposo Inviolato}
\noindent
\begin{description}[noitemsep, topsep=0pt, parsep=0pt, partopsep=0pt, leftmargin=0cm, labelwidth=2.8cm]
	\item[\textbf{Lista di Magia}]: Necromanzia
	\item[\textbf{Livello}]: 2, Non Comune
	\item[\textbf{T. di Lancio}]: 2 Azioni
	\item[\textbf{Gittata}]: Contatto
	\item[\textbf{Componenti}]: V, S, M (un pizzico di sale e un pezzo di rame posto su ciascun occhio del cadavere, che devono restare lì per la durata)
	\item[\textbf{Durata}]: 10 giorni
\end{description}

Entri a contatto con un cadavere o altri resti. Per la durata, il bersaglio è protetto dalla putrefazione e non può diventare non morto.

\textbf{Per ogni Successo Critico Magico} ottenuto nella Prova di Magia raddoppi la durata fino ad un massimo di un anno.

\smallskip\noindent\rule{\linewidth}{2pt} \index[Incantesimi]{Risata Incontenibile}\hypertarget{Risata Incontenibile}{}\smallskip\noindent{\textbf{Risata Incontenibile}}\pdfbookmark[3]{Risata Incontenibile}{Risata Incontenibile}
\noindent
\begin{description}[noitemsep, topsep=0pt, parsep=0pt, partopsep=0pt, leftmargin=0cm, labelwidth=2.8cm]
	\item[\textbf{Lista di Magia}]: Ammaliamento
	\item[\textbf{Livello}]: 1, Non Comune
	\item[\textbf{T. di Lancio}]: 2 Azioni
	\item[\textbf{Gittata}]: 9 metri
	\item[\textbf{Componenti}]: V, S, M (piccole torte e una piuma che viene agitata nell'aria)
	\textbf{Durata}: 1 minuto
\end{description}
Una creatura a gittata di tua scelta e che puoi vedere percepisce tutto come tremendamente ilare e divertente, scoppiando in fragorose risate finché è soggetta a questo incantesimo. Il bersaglio deve superare un Tiro Salvezza su Volontà o cadere prono ed i round successivi perdere 1 Azione a round per ridere. Le creature con un punteggio di Intelligenza -2 o meno, ignorano l'effetto.

Al termine di ciascun suo round e ogni volta che subisce danni, il bersaglio può effettuare un altro Tiro Salvezza su Volontà. Il bersaglio ha +1d6 al Tiro Salvezza se ha subito danni nel round. Se lo supera, l'incantesimo termina.

\smallskip\noindent\rule{\linewidth}{2pt} \index[Incantesimi]{Riscaldare il Metallo}\hypertarget{Riscaldare il Metallo}{}\smallskip\noindent{\textbf{Riscaldare il Metallo}}\pdfbookmark[3]{Riscaldare il Metallo}{Riscaldare il Metallo}
\noindent
\begin{description}[noitemsep, topsep=0pt, parsep=0pt, partopsep=0pt, leftmargin=0cm, labelwidth=2.8cm]
	\item[\textbf{Lista di Magia}]: Fuoco
	\item[\textbf{Livello}]: 2, Non Comune
	\item[\textbf{T. di Lancio}]: 2 Azioni
	\item[\textbf{Gittata}]: 18 metri
	\item[\textbf{Componenti}]: V, S, M (polvere di ferro ed alluminio)
	\item[\textbf{Durata}]: 1 minuto, Concentrazione
\end{description}

Scegli un manufatto di metallo, come un'arma di metallo o un'armatura di metallo media o pesante, a gittata e che puoi vedere. Fai sì che l'oggetto risplenda di rosso per il calore. Qualsiasi creatura in contatto fisico con l'oggetto subisce 1d8 danni da fuoco quando lanci questo incantesimo. Finché mantieni la Concentrazione infliggi nuovamente questo danno nel round.

Se una creatura sta impugnando o indossando l'oggetto e subisce danno da esso la creatura deve superare un Tiro Salvezza su Tempra o gettare l'oggetto se ne è in grado. Se non getta l'oggetto, ha -2 ai Tiri per Colpire e le prove su competenze di base fino all'inizio del suo prossimo round. Se l'oggetto è oltre i 18 metri dall'incantatore l'incantesimo non termina ma smette di essere rovente.

\textbf{Per ogni Successo Critico Magico} ottenuto nella Prova di Magia il danno aumenta di 1d6.

\smallskip\noindent\rule{\linewidth}{2pt} \index[Incantesimi]{Ristorare Inferiore}\hypertarget{Ristorare Inferiore}{}\smallskip\noindent{\textbf{Ristorare Inferiore}}\pdfbookmark[3]{Ristorare Inferiore}{Ristorare Inferiore}
\noindent
\begin{description}[noitemsep, topsep=0pt, parsep=0pt, partopsep=0pt, leftmargin=0cm, labelwidth=2.8cm]
	\item[\textbf{Lista di Magia}]: Cura
	\item[\textbf{Livello}]: 2, Comune
	\item[\textbf{T. di Lancio}]: 2 Azioni
	\item[\textbf{Gittata}]: Contatto
	\item[\textbf{Componenti}]: V, S
	\item[\textbf{Durata}]: Istantanea
\end{description}

Puoi porre fine a una condizione che affligge una creatura con cui sei a contatto. La condizione può essere \textbf{accecato}, \textbf{assordato} o \textbf{paralizzato}. Può ridurre di un grado il livello di \textbf{Affaticamento}. Si recupera 2d6 Punti Ferita Massimi persi, ma non aumentano i Punti Ferita attuali. Puoi recuperare 1 punto di Caratteristica perso non permanentemente.

Non è possibile usufruire di più di un Ristorare Inferiore al giorno.

In caso di condizioni magiche esegui una prova di \hyperlink{contrastareincantesimi}{contrastare} (pag. \pageref{contrastareincantesimi}) con la DC della condizione.

\smallskip\noindent\rule{\linewidth}{2pt} \index[Incantesimi]{Ristorare Superiore}\hypertarget{Ristorare Superiore}{}\smallskip\noindent{\textbf{Ristorare Superiore}}\pdfbookmark[3]{Ristorare Superiore}{Ristorare Superiore}
\noindent
\begin{description}[noitemsep, topsep=0pt, parsep=0pt, partopsep=0pt, leftmargin=0cm, labelwidth=2.8cm]
	\item[\textbf{Lista di Magia}]: Cura
	\item[\textbf{Livello}]: 5, Non Comune
	\item[\textbf{T. di Lancio}]: 2 Azioni
	\item[\textbf{Gittata}]: Contatto
	\item[\textbf{Componenti}]: V, S, M (polvere di diamante del valore di almeno 100 mo, che l'incantesimo consuma)
	\item[\textbf{Durata}]: Istantanea
\end{description}

Imbevi una creatura a contatto di energia positiva curativa per annullare un effetto debilitante, non è possibile usufruire di più di un Ristorare Superiore al giorno:

\begin{itemize}[leftmargin=*] \setlength{\itemsep}{0pt}

	\item Un effetto che ha Affascinato o Dominato il bersaglio.
	\item Fai recuperare 2 punti ad una statistica al bersaglio. Recuperi 1 punto se la perdita era permanente.
	\item I Punti Ferita massimi tornano al valore normale, ma non aumentano i Punti Ferita attuali.
	\item Sei in grado di alleviare di due gradi le condizioni di Affaticamento.
\end{itemize}

Non è possibile usufruire di più di un Ristorare Superiore al giorno.

In caso di condizioni magiche esegui una prova di \hyperlink{contrastareincantesimi}{contrastare} (pag. \pageref{contrastareincantesimi}) con la DC della condizione.

\smallskip\noindent\rule{\linewidth}{2pt} \index[Incantesimi]{Risveglio}\hypertarget{Risveglio}{}\smallskip\noindent{\textbf{Risveglio}}\pdfbookmark[3]{Risveglio}{Risveglio}
\noindent
\begin{description}[noitemsep, topsep=0pt, parsep=0pt, partopsep=0pt, leftmargin=0cm, labelwidth=2.8cm]
	\item[\textbf{Lista di Magia}]: Animali e Piante
	\item[\textbf{Livello}]: 5, Raro
	\item[\textbf{T. di Lancio}]: 8 ore
	\item[\textbf{Gittata}]: Contatto
	\item[\textbf{Componenti}]: V, S, M (un'agata del valore di almeno 1000 mo, che l'incantesimo consuma)
	\item[\textbf{Durata}]: Istantanea
\end{description}

Dopo aver trascorso il tempo di lancio a disegnare tracciati magici con una gemma preziosa, entri a contatto con una bestia o vegetale Enorme o di taglia inferiore. Il bersaglio deve essere privo di punteggio di Intelligenza o avere Intelligenza -3 o meno. Il bersaglio ottiene Intelligenza 0. Il bersaglio ottiene anche la capacità di parlare un linguaggio che conosci. Se il bersaglio è un vegetale, ottiene la capacità di muovere i suoi arti, radici, liane, rampicanti e così via, e ottiene sensi simili a quelli di un umano. Il Narratore sceglierà le statistiche appropriate al tipo di vegetale risvegliato, come le statistiche per il cespuglio risvegliato o l'albero risvegliato.

La bestia o vegetale risvegliato è Affascinato da te per 30 giorni o finché tu o i tuoi compagni non gli arrecherete danno. Quando la condizione Affascinato termina, la creatura risvegliata sceglie se rimanerti amichevole, in base a come l'hai trattata mentre era affascinata.

\textbf{Per ogni Successo Critico Magico} ottenuto nella Prova di Magia aumenti la durata della fascinazione di 30 giorni, fino ad un massimo di 1 anno.

\smallskip\noindent\rule{\linewidth}{2pt} \index[Incantesimi]{Ritirata Rapida}\hypertarget{Ritirata Rapida}{}\smallskip\noindent{\textbf{Ritirata Rapida}}\pdfbookmark[3]{Ritirata Rapida}{Ritirata Rapida}
\noindent
\begin{description}[noitemsep, topsep=0pt, parsep=0pt, partopsep=0pt, leftmargin=0cm, labelwidth=2.8cm]
	\item[\textbf{Lista di Magia}]: Trasmutazione
	\item[\textbf{Livello}]: 1, Non Comune
	\item[\textbf{T. di Lancio}]: 1 Azione
	\item[\textbf{Gittata}]: Personale
	\item[\textbf{Componenti}]: V, S
	\item[\textbf{Durata}]: Concentrazione, 1 minuto
\end{description}

Questo incantesimo ti permette di muoverti a un'andatura incredibile. Quando lanci questo incantesimo il tuo movimento aumenta di 2 metri per Azione di Movimento.

\textbf{Per ogni Successo Critico Magico} ottenuto nella Prova di Magia la durata aumenta di 1 round.

\smallskip\noindent\rule{\linewidth}{2pt} \index[Incantesimi]{Saltare}\hypertarget{Saltare}{}\smallskip\noindent{\textbf{Saltare}}\pdfbookmark[3]{Saltare}{Saltare}
\noindent
\begin{description}[noitemsep, topsep=0pt, parsep=0pt, partopsep=0pt, leftmargin=0cm, labelwidth=2.8cm]
	\item[\textbf{Lista di Magia}]: Aria
	\item[\textbf{Livello}]: 1, Comune
	\item[\textbf{T. di Lancio}]: 2 Azioni
	\item[\textbf{Gittata}]: Contatto
	\item[\textbf{Componenti}]: V, S, M (la zampa posteriore di una cavalletta)
	\item[\textbf{Durata}]: 1 minuto
\end{description}

La distanza di salto della creatura con cui sei in contatto al momento del lancio è triplicata, rispetto al risultato ottenuto e senza limite di lunghezza/altezza, fino al termine dell'incantesimo.

\smallskip\noindent\rule{\linewidth}{2pt} \index[Incantesimi]{Santificare}\hypertarget{Santificare}{}\smallskip\noindent{\textbf{Santificare}}\pdfbookmark[3]{Santificare}{Santificare}
\noindent
\begin{description}[noitemsep, topsep=0pt, parsep=0pt, partopsep=0pt, leftmargin=0cm, labelwidth=2.8cm]
	\item[\textbf{Lista di Magia}]: Invocazione
	\item[\textbf{Livello}]: 5, Raro
	\item[\textbf{T. di Lancio}]: 24 ore
	\item[\textbf{Gittata}]: Contatto
	\item[\textbf{Componenti}]: V, S, M (erbe, oli e incensi del valore di almeno 1000 mo, che l'incantesimo consuma)
	\item[\textbf{Durata}]: Fino a che dissolto
\end{description}

Infondi l'area circostante a un punto con cui sei in contatto del potere del tuo Patrono. L'area può avere un raggio massimo di 18 metri, e l'incantesimo fallisce se include un'area già sotto l'effetto di un incantesimo santificare. L'area soggetta all'incantesimo genera i seguenti effetti.

\emph{Per prima cosa}, celestiali, elementali, fatati, demoni e non morti non possono entrare nell'area, né una simile creatura può affascinare, spaventare o possederne altre al suo interno. Qualsiasi creatura affascinata, spaventata o posseduta da una creatura simile non è più affascinata,spaventata o posseduta dal momento in cui entra in quest'area. Puoi escludere uno o più tipi di queste creature da questo effetto.

\emph{Seconda cosa}, puoi vincolare un effetto ulteriore all'area. Scegli l'effetto dalla lista seguente, o scegline uno presentatoti dal Narratore. Alcuni di questi effetti si applicano alle creature nell'area; puoi decidere se gli effetti si applichino a tutte le creature, le creature Devote o Seguaci di specifica Patrono, o le creature di un tipo specifico, come orchi o troll. Quando una creatura soggetta all'incantesimo entra in quest'area per la prima volta durante un round o inizia il suo round qui, deve effettuare un Tiro Salvezza su Volontà. Se lo supera, la creatura ignora l'effetto aggiuntivo finché non lascia l'area.

\begin{itemize}[leftmargin=*] \setlength{\itemsep}{0pt}
	\item \emph{Coraggio}. Le creature soggette non possono essere spaventate mentre restano in quest'area. Interferenza Extradimensionale. Le creature soggette non possono muoversi o viaggiare usando il teletrasporto o altri mezzi extradimensionali o interplanari.
	\item \emph{Lingue}. Le creature soggette possono comunicare con qualsiasi altra creatura nell'area, anche se non condividono un linguaggio comune.
	\item \emph{Luce Diurna}. Luce intensa riempie l'area. L'oscurità magica creata da incantesimi di più basso livello di quella usata per lanciare questo incantesimo non possono estinguere la luce. La durata in questo caso è di una settimana.
	\item \emph{Oscurità}. L'oscurità riempie l'area. La luce normale, e anche la luce magica creata da incantesimi di più basso livello di quello usato per lanciare questo incantesimo, non possono illuminare l'area.
	\item \emph{Paura}. Le creature soggette sono spaventate mentre restano in quest'area.
	\item \emph{Protezione dall'Energia}. Le creature soggette ricevono resistenza a un tipo di danno a tua scelta (a eccezione dei danni contundenti, perforanti o taglienti), finché restano nell'area.
	\item \emph{Riposo Inviolato}. I corpi morti seppelliti nell'area non possono essere trasformati in non morti.
	\item \emph{Silenzio}. Nessun suono può emanare dall'interno dell'area, e nessun suono può entrarvi.
	\item \emph{Vulnerabilità all'Energia}. Le creature soggette ricevono vulnerabilità a un tipo di danno a tua scelta (a eccezione dei danni contundenti, perforanti o taglienti), finché restano nell'area.
\end{itemize}

\smallskip\noindent\rule{\linewidth}{2pt} \index[Incantesimi]{Santuario}\hypertarget{Santuario}{}\smallskip\noindent{\textbf{Santuario}}\pdfbookmark[3]{Santuario}{Santuario}
\noindent
\begin{description}[noitemsep, topsep=0pt, parsep=0pt, partopsep=0pt, leftmargin=0cm, labelwidth=2.8cm]
	\item[\textbf{Lista di Magia}]: Abiurazione
	\item[\textbf{Livello}]: 1, Comune
	\item[\textbf{T. di Lancio}]: 2 Azioni
	\item[\textbf{Gittata}]: 9 metri
	\item[\textbf{Componenti}]: V, S, M (un piccolo specchio d'argento)
	\item[\textbf{Durata}]: 1 minuto
\end{description}

Proteggi una creatura a gittata dagli attacchi. Fino al termine dell'incantesimo, qualsiasi creatura che prenda come bersaglio la creatura protetta con un attacco o incantesimo dannoso deve prima effettuare un Tiro Salvezza su Volontà. Se fallisce il Tiro Salvezza, l'attaccante deve scegliere un nuovo bersaglio o perdere l'attacco o l'incantesimo. Questo incantesimo non protegge la creatura protetta dagli effetti ad area, come lo scoppio di una palla di fuoco. Se la creatura protetta effettua un attacco o lancia un incantesimo che agisce su creature nemiche, l'incantesimo termina.

\smallskip\noindent\rule{\linewidth}{2pt} \index[Incantesimi]{Santuario Privato}\hypertarget{Santuario Privato}{}\smallskip\noindent{\textbf{Santuario Privato}}\pdfbookmark[3]{Santuario Privato}{Santuario Privato}
\noindent
\begin{description}[noitemsep, topsep=0pt, parsep=0pt, partopsep=0pt, leftmargin=0cm, labelwidth=2.8cm]
	\item[\textbf{Lista di Magia}]: Abiurazione
	\item[\textbf{Livello}]: 4, Molto Raro
	\item[\textbf{T. di Lancio}]: 10 minuti
	\item[\textbf{Gittata}]: 36 metri
	\item[\textbf{Componenti}]: V, S, M (un sottile foglio di piombo, un pezzo di vetro opaco, un batuffolo di cotone o tessuto, e crisolito in polvere)
	\item[\textbf{Durata}]: 24 ore
\end{description}

Proteggi con la magia un'area. L'area è una sfera che può essere piccola fino a 1 metro di raggio o grande fino a 15 metri di raggio. L'incantesimo agisce fino al termine della durata o finché non usi un'Azione per interromperlo.

Quando lanci l'incantesimo, decidi che tipo di protezione questo fornisce, scegliendo una o più delle seguenti proprietà:

\noindent- Il suono non può attraversare il perimetro dell'area protetta.

\begin{itemize}[leftmargin=*] \setlength{\itemsep}{0pt}
	\item Il perimetro dell'area protetta appare buio e nebbioso, impedendo di vedervi attraverso (anche alla scurovisione)
	\item Sensori creati da incantesimi di divinazione non possono apparire all'interno dell'area protetta o attraversare la sua barriera perimetrale
	\item Le creature nell'area non possono essere bersaglio di incantesimi di divinazione
	\item Nulla può teletrasportarsi dentro o fuori dell'area protetta.
	\item All'interno dell'area protetta, il viaggio planare è interdetto.
\end{itemize}

Lanciare questo incantesimo sullo stesso punto ogni giorno per un anno, rende l'effetto permanente.

\textbf{Per ogni Successo Critico Magico} ottenuto nella Prova di Magia puoi aumentare le dimensioni della sfera di 3 metri di raggio oppure aumentare la durata di 12 ore.

\smallskip\noindent\rule{\linewidth}{2pt} \index[Incantesimi]{Scagliare Maledizione}\hypertarget{Scagliare Maledizione}{}\smallskip\noindent{\textbf{Scagliare Maledizione}}\pdfbookmark[3]{Scagliare Maledizione}{Scagliare Maledizione}
\noindent
\begin{description}[noitemsep, topsep=0pt, parsep=0pt, partopsep=0pt, leftmargin=0cm, labelwidth=2.8cm]
	\item[\textbf{Lista di Magia}]: Necromanzia
	\item[\textbf{Livello}]: 3, Non Comune
	\item[\textbf{T. di Lancio}]: 2 Azioni
	\item[\textbf{Gittata}]: Contatto
	\item[\textbf{Componenti}]: V, S
	\item[\textbf{Durata}]: 1 minuto
\end{description}

Una creatura con cui sei a contatto deve superare un Tiro Salvezza su Volontà o restare maledetta per la durata dell'incantesimo. Quando lanci questo incantesimo, scegli la natura della maledizione tra le seguenti opzioni:

\begin{itemize}[leftmargin=*] \setlength{\itemsep}{0pt}
	\item Scegli un punteggio di caratteristica. Mentre è maledetto, il bersaglio ha -1d6 alle prova di competenza base basate su quella caratteristica ed i Tiri Salvezza basati su quella caratteristica.
	\item Mentre è maledetto, il bersaglio ha -1d6 ai Tiri per Colpire e -3 al danno in mischia, contro di te.
	\item Mentre è maledetto, il bersaglio deve effettuare un Tiro Salvezza su Volontà all'inizio di ciascun suo round. Se lo fallisce, spreca 1 Azione di quel suo round senza fare nulla.
	\item Mentre il bersaglio è maledetto, ogni tuo attacco ed incantesimo infliggono 1d8 danni da Vuoto aggiuntivi contro di lui.
\end{itemize}

L'incantesimo rimuovi maledizione (vedi descrizione) termina questo effetto. A discrezione del Narratore, puoi scegliere una maledizione dall'effetto diverso, ma non dovrebbe essere comunque più potente di quelle descritte qui sopra. Il Narratore detiene il giudizio finale sull'effetto di una maledizione.

\textbf{Per ogni Successo Critico Magico} ottenuto nella Prova di Magia raddoppi la durata. Se ottieni 3 successi critici la durata è permanente.

\smallskip\noindent\rule{\linewidth}{2pt} \index[Incantesimi]{Scagliare Maledizione Minore}\hypertarget{Scagliare Maledizione Minore}{}\smallskip\noindent{\textbf{Scagliare Maledizione Minore}}\pdfbookmark[3]{Scagliare Maledizione Minore}{Scagliare Maledizione Minore}
\noindent
\begin{description}[noitemsep, topsep=0pt, parsep=0pt, partopsep=0pt, leftmargin=0cm, labelwidth=2.8cm]
	\item[\textbf{Lista di Magia}]: Universale
	\item[\textbf{Livello}]: 1, Comune
	\item[\textbf{T. di Lancio}]: 2 Azioni
	\item[\textbf{Gittata}]: Contatto
	\item[\textbf{Componenti}]: V, S
	\item[\textbf{Durata}]: 1 minuto
\end{description}

Una creatura con cui sei a contatto deve superare un Tiro Salvezza su Volontà o restare maledetta per la durata dell'incantesimo. Quando lanci questo incantesimo, scegli la natura della maledizione tra le seguenti opzioni:

\begin{itemize}[leftmargin=*] \setlength{\itemsep}{0pt}
	\item Scegli un punteggio di caratteristica. Mentre è maledetto, il bersaglio ha -1 alle prove di competenza di Base e i Tiri Salvezza basati su quel punteggio di caratteristica.
	\item Mentre è maledetto, il bersaglio ha -2 ai Tiri per Colpire contro di te.
	\item Mentre è maledetto, il bersaglio deve effettuare un Tiro Salvezza su Volontà all'inizio di ciascun suo round. Se lo fallisce, spreca 1 Azione di quel suo round senza fare nulla.
\end{itemize}

L'incantesimo \hyperlink{Rimuovi Maledizione}{Rimuovi Maledizione} termina questo effetto. A discrezione del Narratore, puoi scegliere una maledizione dall'effetto diverso, ma non dovrebbe essere comunque più potente di quelle descritte qui sopra. Il Narratore detiene il giudizio finale sull'effetto di una maledizione.

\textbf{Per ogni Successo Critico Magico} ottenuto nella Prova di Magia scegli un altra creatura entro 6 metri dalla prima.

\smallskip\noindent\rule{\linewidth}{2pt} \index[Incantesimi]{Scassinare}\hypertarget{Scassinare}{}\smallskip\noindent{\textbf{Scassinare}}\pdfbookmark[3]{Scassinare}{Scassinare}\label{knock}
\noindent
\begin{description}[noitemsep, topsep=0pt, parsep=0pt, partopsep=0pt, leftmargin=0cm, labelwidth=2.8cm]
	\item[\textbf{Lista di Magia}]: Trasmutazione
	\item[\textbf{Livello}]: 2, Comune
	\item[\textbf{T. di Lancio}]: 2 Azioni
	\item[\textbf{Gittata}]: 18 metri
	\item[\textbf{Componenti}]: V
	\item[\textbf{Durata}]: Istantanea
\end{description}

Scegli un oggetto a gittata e che puoi vedere. L'oggetto può essere una porta, scatola, delle manette, una serratura o un altro oggetto che possieda un metodo comune o magico per prevenirne l'accesso.

Un bersaglio che è chiuso da una serratura comune o che è bloccato o sbarrato viene aperto, sbloccato o liberato. Se l'oggetto ha più serrature, solo una di queste viene aperta.

Se scegli un bersaglio che è tenuto chiuso con Serratura Magica (o simile) esegui una prova di \hyperlink{contrastareincantesimi}{contrastare incantesimi} tra le due DC. Se Scassinare è più efficace di Serratura Magia allora l'incantesimo di chiusura viene annullato, altrimenti Scassinare non ha avuto effetto.

Quando lanci questo incantesimo, un sonoro bussare, udibile fino a 90 metri di distanza, emana dall'oggetto bersaglio.

\textbf{Per ogni Successo Critico Magico} ottenuto nella Prova di Magia puoi aprire un altro lucchetto/serratura entro gittata o aumentare di 4 la DC dell'incantesimo.

\smallskip\noindent\rule{\linewidth}{2pt} \index[Incantesimi]{Schiaffo di Cattalm}\hypertarget{Schiaffo di Cattalm}{}\smallskip\noindent{\textbf{Schiaffo di Cattalm}}\pdfbookmark[3]{Schiaffo di Cattalm}{Schiaffo di Cattalm}
\noindent
\begin{description}[noitemsep, topsep=0pt, parsep=0pt, partopsep=0pt, leftmargin=0cm, labelwidth=2.8cm]
	\item[\textbf{Lista di Magia}]: Evocazione
	\item[\textbf{Livello}]: 1, Non Comune
	\item[\textbf{T. di Lancio}]: 1 Reazione, che puoi effettuare in risposta al danno arrecatoti da una creatura entro 18 metri da te che puoi vedere
	\item[\textbf{Gittata}]: 18 metri
	\item[\textbf{Componenti}]: V, S
	\item[\textbf{Durata}]: Istantanea
\end{description}

Punti il dito, e la creatura che ti ha danneggiato viene schiaffeggiata da fuoco divina. La creatura deve effettuare un Tiro Salvezza su Riflessi e se fallisce subisce 2d10 danni da fuoco o la metà di questi danni se lo supera.

\textbf{Per ogni Successo Critico Magico} ottenuto nella Prova di Magia i danno aumenta di 1d10.



\smallskip\noindent\rule{\linewidth}{2pt} \index[Incantesimi]{Scolpire Pietra}\hypertarget{Scolpire Pietra}{}\smallskip\noindent{\textbf{Scolpire Pietra}}\pdfbookmark[3]{Scolpire Pietra}{Scolpire Pietra}
\noindent
\begin{description}[noitemsep, topsep=0pt, parsep=0pt, partopsep=0pt, leftmargin=0cm, labelwidth=2.8cm]
	\item[\textbf{Lista di Magia}]: Terra
	\item[\textbf{Livello}]: 4, Comune
	\item[\textbf{T. di Lancio}]: 2 Azioni
	\item[\textbf{Gittata}]: Contatto
	\item[\textbf{Componenti}]: V, S, M (argilla malleabile, che deve essere lavorata per ottenere una vaga forma dell'oggetto di pietra)
	\item[\textbf{Durata}]: Istantanea
\end{description}

Scolpisci in qualsiasi forma che si presti ai tuoi scopi un cubo di 3 metri di lato di pietra con cui sei a contatto.

Così, per esempio, potresti scolpire una grossa pietra in un'arma, idolo o feretro, o creare un passaggio attraverso il muro. Potresti anche modellare una porta di pietra o la sua cornice per sigillare la porta. L'oggetto che crei può avere fino a due cardini e un chiavistello, ma è impossibile creare meccanismi più complessi.

\smallskip\noindent\rule{\linewidth}{2pt} \index[Incantesimi]{Scopri il Percorso}\hypertarget{Scopri il Percorso}{}\smallskip\noindent{\textbf{Scopri il Percorso}}\pdfbookmark[3]{Scopri il Percorso}{Scopri il Percorso}
\noindent
\begin{description}[noitemsep, topsep=0pt, parsep=0pt, partopsep=0pt, leftmargin=0cm, labelwidth=2.8cm]
	\item[\textbf{Lista di Magia}]: Divinazione
	\item[\textbf{Livello}]: 6, Non Comune
	\item[\textbf{T. di Lancio}]: 1 minuto
	\item[\textbf{Gittata}]: Personale
	\item[\textbf{Componenti}]: V, S, M (degli attrezzi da divinazione - dei bastoncini d'avorio, ossa, carte, denti o rune incise - del valore di almeno 100 mo e un oggetto dal luogo che desideri trovare)
	\item[\textbf{Durata}]: 1 giorno
\end{description}

Questo incantesimo ti permette di trovare la rotta fisica più breve e diretta verso uno specifico luogo fisso con cui hai familiarità ed è sullo stesso piano di esistenza. Se indichi una destinazione su di un altro piano di esistenza, una destinazione che si muove (come una fortezza mobile) o una destinazione non specifica (come \emph{la tana di un drago verde}), l'incantesimo fallisce.

Per la durata dell'incantesimo, finché sei nello stesso piano di esistenza della destinazione, saprai quanto è distante e in che direzione si trovi. Mentre sei in viaggio verso di essa, ogni volta che ti si presenterà la possibilità di scegliere tra percorsi diversi, determinerai automaticamente qual è la via più breve e la rotta più diretta (ma non necessariamente la più sicura) per raggiungere la destinazione.

\textbf{Per ogni critico} ottenuto nella Prova di Magia l'incantesimo dura 8 ore in più.

\smallskip\noindent\rule{\linewidth}{2pt} \index[Incantesimi]{Scopri Piante}\hypertarget{Scopri Piante}{}\smallskip\noindent{\textbf{Scopri Piante}}\pdfbookmark[3]{Scopri Piante}{Scopri Piante}
\noindent
\begin{description}[noitemsep, topsep=0pt, parsep=0pt, partopsep=0pt, leftmargin=0cm, labelwidth=2.8cm]
	\item[\textbf{Lista di Magia}]: Divinazione
	\item[\textbf{Livello}]: 2, Non Comune
	\item[\textbf{T. di Lancio}]: 2 Azioni
	\item[\textbf{Gittata}]: Personale
	\item[\textbf{Componenti}]: V, S
	\item[\textbf{Durata}]: 1 turno per CM
\end{description}

L'incantatore che lancia questo incantesimo è in grado di trovare una specifica pianta entro un cerchio del diametro di 3 metri per CM centrato sull'incantatore. L'incantatore può concentrarsi su un diverso tipo di pianta ogni round e può muoversi, dal momento che l'area di effetto si sposta con lui.

\textbf{Nota}: per i Devoti di Shayalia l'incantesimo è Comune ed il cerchio ha un diametro di 10 metri per somma Tratti in comune con il Patrono.

\smallskip\noindent\rule{\linewidth}{2pt} \index[Incantesimi]{Scopri Trappole}\hypertarget{Scopri Trappole}{}\smallskip\noindent{\textbf{Scopri Trappole}}\pdfbookmark[3]{Scopri Trappole}{Scopri Trappole}
\noindent
\begin{description}[noitemsep, topsep=0pt, parsep=0pt, partopsep=0pt, leftmargin=0cm, labelwidth=2.8cm]
	\item[\textbf{Lista di Magia}]: Divinazione
	\item[\textbf{Livello}]: 2, Comune
	\item[\textbf{T. di Lancio}]: 2 Azioni
	\item[\textbf{Gittata}]: 36 metri
	\item[\textbf{Componenti}]: V, S
	\item[\textbf{Durata}]: 10 minuti di tempo reale
\end{description}

Per la durata dell'incantesimo avverti la presenza di qualsiasi trappola a gittata che sia nella tua linea di visuale. Una trappola, ai fini di questo incantesimo, comprende qualsiasi cosa che sia in grado di infliggere un effetto improvviso o inaspettato che tu possa considerare dannoso o indesiderabile e che è stato espressamente inteso come tale dal suo creatore. Di conseguenza, l'incantesimo percepirebbe un'area sotto l'incantesimo allarme, un glifo di interdizione, o una botola meccanica, ma non rivelerebbe una debolezza naturale del pavimento, un soffitto instabile o una buca nascosta.

La trappola viene evidenziata alla tua vista con un segno viola.

\smallskip\noindent\rule{\linewidth}{2pt} \index[Incantesimi]{Scrigno Segreto}\hypertarget{Scrigno Segreto}{}\smallskip\noindent{\textbf{Scrigno Segreto}}\pdfbookmark[3]{Scrigno Segreto}{Scrigno Segreto}
\noindent
\begin{description}[noitemsep, topsep=0pt, parsep=0pt, partopsep=0pt, leftmargin=0cm, labelwidth=2.8cm]
	\item[\textbf{Lista di Magia}]: Evocazione
	\item[\textbf{Livello}]: 4, Raro
	\item[\textbf{T. di Lancio}]: 2 Azioni
	\item[\textbf{Gittata}]: Contatto
	\item[\textbf{Componenti}]: V, S, M (un forziere lavorato, 1 metro x 50 cm x 50 cm, costruito con rari materiali del valore di almeno 5000 mo, e una sua replica Minuscola fatta degli stessi materiali e del valore di almeno 50)
	\item[\textbf{Durata}]: Istantanea
\end{description}

Nascondi un forziere e tutti i suoi contenuti sul Piano Etereo. Quando lanci questo incantesimo devi essere in contatto con il forziere e la replica in miniatura che serve da componente materiale. Il forziere può contenere fino a 0,25 metri cubi di materiale non vivente (1 x metro x 50 centimetri x 50 centimetri). Mentre il forziere rimane sul Piano Etereo, puoi usare un'Azione per entrare in contatto con la replica e richiamare il forziere. Esso riapparirà in uno spazio non occupato sul terreno entro 1 metro da te. Puoi rispedire il forziere nel Piano Etereo, usando un'Azione ed entrando in contatto sia col forziere che con la replica.

Dopo 60 giorni, c'è una percentuale cumulativa del 5\% al giorno che l'effetto dell'incantesimo abbia termine.

L'effetto termina se l'incantesimo viene lanciato nuovamente, se la replica del forziere viene distrutta, o se decidi di terminare l'incantesimo con un'Azione. Se l'incantesimo termina e il forziere si trova sul Piano Etereo, viene irrimediabilmente perduto.

\smallskip\noindent\rule{\linewidth}{2pt} \index[Incantesimi]{Scritto Illusorio}\hypertarget{Scritto Illusorio}{}\smallskip\noindent{\textbf{Scritto Illusorio}}\pdfbookmark[3]{Scritto Illusorio}{Scritto Illusorio}
\noindent
\begin{description}[noitemsep, topsep=0pt, parsep=0pt, partopsep=0pt, leftmargin=0cm, labelwidth=2.8cm]
	\item[\textbf{Lista di Magia}]: Illusione
	\item[\textbf{Livello}]: 1, Comune
	\item[\textbf{T. di Lancio}]: 1 minuto
	\item[\textbf{Gittata}]: Contatto
	\item[\textbf{Componenti}]: S, M (un inchiostro a base di piombo del valore di almeno 10 mo, che l'incantesimo consuma)
	\item[\textbf{Durata}]: 10 giorni
\end{description}

Scrivi su di una pergamena, un pezzo di carta o qualche altro materiale adatto a scrivere e lo infondi di una potente illusione che permane per la durata dell'incantesimo.

Per te e qualsiasi creatura da te indicata al lancio dell'incantesimo, la scritta appare normale, con la tua grafia, e trasmette qualsiasi significato volevi comunicare quando hai vergato il testo. Per tutti gli altri, la scritta appare come se fosse redatta in una scrittura ignota o magica, che risulta incomprensibile. In alternativa, puoi far sì che la scritta sembri un messaggio totalmente diverso, in una grafia e linguaggio differente, sebbene debba essere un linguaggio a te conosciuto.

In caso l'incantesimo venisse dissolto, sia la scritta originale che l'illusione svaniscono. Una creatura con visione del vero può leggere il messaggio nascosto.

\smallskip\noindent\rule{\linewidth}{2pt} \index[Incantesimi]{Scrutare}\hypertarget{Scrutare}{}\smallskip\noindent{\textbf{Scrutare}}\pdfbookmark[3]{Scrutare}{Scrutare}
\noindent
\begin{description}[noitemsep, topsep=0pt, parsep=0pt, partopsep=0pt, leftmargin=0cm, labelwidth=2.8cm]
	\item[\textbf{Lista di Magia}]: Divinazione
	\item[\textbf{Livello}]: 5, Raro
	\item[\textbf{T. di Lancio}]: 10 minuti
	\item[\textbf{Gittata}]: Personale
	\item[\textbf{Componenti}]: V, S, M (un focus del valore di almeno 1000 mo, come una sfera di cristallo, un specchio d'argento o una fonte ricolma di Acqua santa)
	\item[\textbf{Durata}]: Concentrazione, massimo 10 minuti
\end{description}

Puoi vedere e udire una particolare creatura a tua scelta che si trovi sul tuo stesso piano di esistenza. Il bersaglio deve effettuare un Tiro Salvezza su Volontà, modificato da quanto bene conosci il bersaglio e la tua connessione fisica a esso. Se il bersaglio sa che stai lanciando l'incantesimo, può fallire volontariamente il Tiro Salvezza, in caso desiderasse essere osservato da
te.

\medskip

\begin{tabular}{ll}
	\toprule
	\textbf{Conoscenza} & \textbf{Mod. al TS}\\
	Ne hai sentito parlare &+4\\
	Hai incontrato il bersaglio &+0\\
	Conosci bene il bersaglio &-4
\end{tabular}

\begin{tabular}{ll}
	\toprule
	\textbf{Connessione} & \textbf{Mod. TS}\\
	Descrizione o immagine &-2\\
	Proprietà o indumento & -4\\
	Parte del corpo (capelli...)&-10
\end{tabular}

\medskip

Se supera il Tiro Salvezza, il bersaglio ignora gli effetti dell'incantesimo, e non potrai usare di nuovo questo incantesimo contro di lui prima che siano passate 24 ore.

Se il Tiro Salvezza fallisce, l'incantesimo crea un sensore invisibile entro 3 metri dal bersaglio. Tramite il sensore puoi udire e vedere come se fossi sul posto. Il sensore si muove assieme al bersaglio, rimanendo entro 3 metri da lui per la durata dell'incantesimo. Una creatura che può vedere oggetti invisibili vede il sensore come una sfera luminosa delle dimensioni all'incirca di un pugno.

Invece di prendere come bersaglio una creatura, puoi scegliere come bersaglio dell'incantesimo un luogo che hai già visto in passato. Quando scegli questa opzione, il sensore compare in quel luogo ma non si muove.

\smallskip\noindent\rule{\linewidth}{2pt} \index[Incantesimi]{Scudo}\hypertarget{Scudo}{}\smallskip\noindent{\textbf{Scudo}}\pdfbookmark[3]{Scudo}{Scudo}
\noindent
\begin{description}[noitemsep, topsep=0pt, parsep=0pt, partopsep=0pt, leftmargin=0cm, labelwidth=2.8cm]
	\item[\textbf{Lista di Magia}]: Universale
	\item[\textbf{Livello}]: 0, Comune
	\item[\textbf{T. di Lancio}]: 1 Azione
	\item[\textbf{Gittata}]: Personale
	\item[\textbf{Componenti}]: V, S
	\item[\textbf{Durata}]: 1 round
\end{description}

Compare una barriera di forza magica invisibile a proteggerti. Fino all'inizio del tuo prossimo round hai un bonus di +1 alla Difesa e non subisci danno dal primo Dardo arcano e Dardo occulto che ti colpisce nel round.

Se lanciato spendendo 1 Punto Magia il bonus alla Difesa aumenta a +2, il tempo di lancio diventa una Reazione. Spendendo 2 Punti Magia oltre alle modifiche precedenti l'obiettivo diventa una qualsiasi creatura entro 6 metri.

\textbf{Per ogni Successo Critico Magico} ottenuto nella Prova di Magia aumenti la durata di 1 round.

\smallskip\noindent\rule{\linewidth}{2pt} \index[Incantesimi]{Scudo di Fuoco}\hypertarget{Scudo di Fuoco}{}\smallskip\noindent{\textbf{Scudo di Fuoco}}\pdfbookmark[3]{Scudo di Fuoco}{Scudo di Fuoco}
\noindent
\begin{description}[noitemsep, topsep=0pt, parsep=0pt, partopsep=0pt, leftmargin=0cm, labelwidth=2.8cm]
	\item[\textbf{Lista di Magia}]: Fuoco, Acqua
	\item[\textbf{Livello}]: 4, Non Comune
	\item[\textbf{T. di Lancio}]: 2 Azioni
	\item[\textbf{Gittata}]: Personale
	\item[\textbf{Componenti}]: V, S, M (un pò di fosforo o una lucciola)
	\item[\textbf{Durata}]: 10 minuti
\end{description}

Fiamme sottili e vaporose avvolgono il tuo corpo per la durata dell'incantesimo, emettendo luce intensa in un raggio di 3 metri e luce fioca per 6 metri. Puoi terminare l'incantesimo in anticipo, usando un'Azione per interromperlo.

Le fiamme ti forniscono uno scudo caldo o uno scudo freddo, a tua scelta. Lo scudo caldo ti conferisce resistenza al danno da freddo, mentre lo scudo freddo ti fornisce resistenza al danno da caldo.

Inoltre, ogni qualvolta una creatura entro 1 metro da te ti colpisce con un attacco in mischia, lo scudo erutta l'elemento. L'attaccante subisce 2d8 danni da fuoco da uno scudo caldo, o 2d8 danni da freddo da uno scudo freddo.

\smallskip\noindent\rule{\linewidth}{2pt} \index[Incantesimi]{Scurovisione}\hypertarget{Scurovisione}{}\smallskip\noindent{\textbf{Scurovisione}}\pdfbookmark[3]{Scurovisione}{Scurovisione}
\noindent
\begin{description}[noitemsep, topsep=0pt, parsep=0pt, partopsep=0pt, leftmargin=0cm, labelwidth=2.8cm]
	\item[\textbf{Lista di Magia}]: Trasmutazione
	\item[\textbf{Livello}]: 2, Comune
	\item[\textbf{T. di Lancio}]: 2 Azioni
	\item[\textbf{Gittata}]: Contatto
	\item[\textbf{Componenti}]: V, S, M (o un pizzico di carota od un mirtillo secco)
	\item[\textbf{Durata}]: 1 ora di tempo reale di gioco
\end{description}

Una creatura consenziente con cui sei in contatto ottiene la capacità di vedere al buio. Per la durata dell'incantesimo, quella creatura ha scurovisione fino a una gittata di 9 metri.

\textbf{Per ogni Successo Critico Magico} ottenuto nella Prova di Magia raddoppi la durata.

\smallskip\noindent\rule{\linewidth}{2pt} \index[Incantesimi]{Segugio Fedele}\hypertarget{Segugio Fedele}{}\smallskip\noindent{\textbf{Segugio Fedele}}\pdfbookmark[3]{Segugio Fedele}{Segugio Fedele}
\noindent
\begin{description}[noitemsep, topsep=0pt, parsep=0pt, partopsep=0pt, leftmargin=0cm, labelwidth=2.8cm]
	\item[\textbf{Lista di Magia}]: Evocazione
	\item[\textbf{Livello}]: 4, Raro
	\item[\textbf{T. di Lancio}]: 2 Azioni
	\item[\textbf{Gittata}]: 9 metri
	\item[\textbf{Componenti}]: V, S, M (un minuscolo fischietto d'argento, un pezzo d'osso ed un filo)
	\item[\textbf{Durata}]: 8 ore
\end{description}

Puoi evocare un cane da guardia fantasma in uno spazio non occupato a gittata che puoi vedere dove rimarrà per la durata dell'incantesimo, finché non viene congedato con un'Azione, o finché non si allontanerà più di 30 metri da te.

Il segugio è invisibile a tutte le creature eccetto che a te e non può essere danneggiato. Quando una creatura di taglia Piccola o superiore si avvicina entro 9 metri da esso senza aver prima pronunciato la parola d'ordine da te specificata quando hai lanciato l'incantesimo, il segugio inizia ad abbaiare a grande volume, ma puoi sentirlo solo tu.

Il segugio vede le creature invisibili e può vedere nel Piano Etereo. Esso ignora le illusioni. All'inizio di ciascun tuo round, il segugio tenta di mordere una creatura entro 1 metro da esso e che ti sia ostile.

Il bonus di attacco del segugio è uguale al tuo modificatore di caratteristica per incantesimi + CM. Se colpisce, infligge 2d8 danni perforanti, ha CM*2 Punti Ferita, Difesa 10+modificatore da incantesimi, Tiri Salvezza pari al tuo modificatore da incantesimi.

\smallskip\noindent\rule{\linewidth}{2pt} \index[Incantesimi]{Sembrare}\hypertarget{Sembrare}{}\smallskip\noindent{\textbf{Sembrare}}\pdfbookmark[3]{Sembrare}{Sembrare}
\noindent
\begin{description}[noitemsep, topsep=0pt, parsep=0pt, partopsep=0pt, leftmargin=0cm, labelwidth=2.8cm]
	\item[\textbf{Lista di Magia}]: Illusione
	\item[\textbf{Livello}]: 5, Non Comune
	\item[\textbf{T. di Lancio}]: 2 Azioni
	\item[\textbf{Gittata}]: 9 metri
	\item[\textbf{Componenti}]: V, S
	\item[\textbf{Durata}]: 8 ore
\end{description}

Questo incantesimo ti permette di cambiare l'aspetto di un qualsiasi numero di creature a gittata e che puoi vedere. Fornisci a ciascun bersaglio un nuovo aspetto illusorio. Una creatura non consenziente può effettuare un Tiro Salvezza su Volontà e, se lo supera, ignora l'incantesimo.

L'incantesimo camuffa l'aspetto fisico oltre che gli abiti, le armature, le armi e l'equipaggiamento. Puoi far sembrare ciascuna creatura 30 centimetri più bassa o più alta, sembrare magra, grassa o una via di mezzo. Non puoi cambiare la conformazione del corpo del bersaglio, e quindi devi scegliere una forma che abbia la stessa distribuzione basilare di arti.

Per tutto il resto, l'illusione è limitata solo dalla tua fantasia. L'incantesimo permane per la sua durata, a meno che tu non usi una Azione per interromperlo prima. I cambi apportati da questo incantesimo non sono in grado di sostenere un'ispezione fisica. Per esempio, se usi questo incantesimo per aggiungere un cappello all'abbigliamento di una creatura, gli oggetti attraversano il cappello, e chiunque lo tocchi non avvertirebbe nulla e finirebbe per toccare la testa e i capelli della creatura.
Se usi questo incantesimo per apparire più magro di quello che sei, la mano di una persona che provasse a toccarti rimbalzerebbe su di te, mentre alla vista sembrerebbe fermarsi a mezz'aria. Una creatura può usare 2 Azioni per ispezionare un bersaglio ed effettuare una prova di Consapevolezza contro la DC del Tiro Salvezza dell'incantesimo, se impiega 3 Azione ha +1d6 di bonus. Se la riesce, capisce che il bersaglio è camuffato.

\smallskip\noindent\rule{\linewidth}{2pt} \index[Incantesimi]{Serratura Magica}\hypertarget{Serratura Magica}{}\smallskip\noindent{\textbf{Serratura Magica}}\pdfbookmark[3]{Serratura Magica}{Serratura Magica}\label{Arcane Lock}
\noindent
\begin{description}[noitemsep, topsep=0pt, parsep=0pt, partopsep=0pt, leftmargin=0cm, labelwidth=2.8cm]
	\item[\textbf{Lista di Magia}]: Abiurazione
	\item[\textbf{Livello}]: 2, Comune
	\item[\textbf{T. di Lancio}]: 2 Azioni
	\item[\textbf{Gittata}]: Contatto
	\item[\textbf{Componenti}]: V, S, M (polvere d'oro del valore di almeno 25 mo, che viene consumata dall'incantesimo)
	\item[\textbf{Durata}]: Fino a che dissolto
\end{description}

Lanci l'incantesimo a contatto di una porta, finestra, portale, forziere o altro ingresso chiuso, e questo diventa chiuso a chiave per la durata. Tu e le creature che hai indicato, quando hai lanciato questo incantesimo, potete aprire l'oggetto normalmente. Puoi anche predisporre una parola d'ordine che, quando pronunciata entro 1 metro dall'oggetto, sopprime l'incantesimo per 1 minuto. Altrimenti l'apertura è invalicabile fino a che non viene distrutta o l''incantesimo è dissolto o soppresso. Lanciare scassinare sull'oggetto sopprime Serratura Magica per 10 minuti.

Mentre è soggetto a questo incantesimo, l'oggetto è più difficile da distruggere o aprire a forza; la DC per romperlo o scassinare una serratura su di esso aumenta di 10.

\textbf{Per ogni Successo Critico Magico} ottenuto nella Prova di Magia puoi influenzare un altra chiusura o aumentare la difficoltà di apertura di 4.

\smallskip\noindent\rule{\linewidth}{2pt} \index[Incantesimi]{Servitore Invisibile}\hypertarget{Servitore Invisibile}{}\smallskip\noindent{\textbf{Servitore Invisibile}}\pdfbookmark[3]{Servitore Invisibile}{Servitore Invisibile}
\noindent
\begin{description}[noitemsep, topsep=0pt, parsep=0pt, partopsep=0pt, leftmargin=0cm, labelwidth=2.8cm]
	\item[\textbf{Lista di Magia}]: Evocazione
	\item[\textbf{Livello}]: 1, Comune
	\item[\textbf{T. di Lancio}]: 2 Azioni
	\item[\textbf{Gittata}]: 18 metri
	\item[\textbf{Componenti}]: V, S, M (un pezzo di corda e un pezzo di legno)
	\item[\textbf{Durata}]: 1 ora
\end{description}

Questo incantesimo crea una forza quasi invisibile solo delimitata da una leggera aura (di colore a tua scelta) che svolge dei semplici compiti al tuo comando, fino al termine dell'incantesimo. Il servitore si forma in uno spazio sul terreno non occupato, entro la gittata. Ha Difesa 10, 1 punto ferita, Forza 0 e non può attaccare. Se scende a 0 Punti Ferita, l'incantesimo ha termine.

Come Azione Immediata, durante ciascun tuo round, puoi comandare mentalmente il servitore di muoversi fino a 3 metri e interagire con un oggetto. Il servitore può svolgere dei semplici compiti alla stregua di un servitore umano, come raccogliere cose, pulire, riparare, piegare abiti, accendere fuochi, servire il cibo e versare il vino. Una volta impartito il comando, il servitore svolgerà il compito al meglio delle sue capacità finché non l'avrà completato, e poi aspetterà il tuo prossimo comando.

Se comandi al servitore di svolgere un compito che lo farà muovere a più di 18 metri da te, l'incantesimo termina.

\smallskip\noindent\rule{\linewidth}{2pt} \index[Incantesimi]{Sfera Congelante}\hypertarget{Sfera Congelante}{}\smallskip\noindent{\textbf{Sfera Congelante}}\pdfbookmark[3]{Sfera Congelante}{Sfera Congelante}
\noindent
\begin{description}[noitemsep, topsep=0pt, parsep=0pt, partopsep=0pt, leftmargin=0cm, labelwidth=2.8cm]
	\item[\textbf{Lista di Magia}]: Acqua
	\item[\textbf{Livello}]: 6, Raro
	\item[\textbf{T. di Lancio}]: 2 Azioni
	\item[\textbf{Gittata}]: 90 metri
	\item[\textbf{Componenti}]: V, S, M (una piccola sfera di cristallo)
	\item[\textbf{Durata}]: Istantanea
\end{description}

Un globo gelido di energia fredda parte dalla punta delle tue dita verso un punto di tua scelta a gittata, dove esplode in una sfera di 18 metri di raggio. Ogni creatura nell'area deve effettuare un Tiro Salvezza su Tempra. Se fallisce il Tiro Salvezza, una creatura subisce 10d6 danni da freddo. Se lo supera, subisce la metà di questi danni.

Se il globo colpisce un corpo d'acqua o un liquido composto principalmente d'acqua (escluse però le creature a base d'acqua), congela il liquido fino a una profondità di 15 centimetri in un'area quadrata di 9 metri di lato. Il ghiaccio dura 1 minuto. Le creature che stavano nuotando sulla superficie dell'acqua congelata restano intrappolate nel ghiaccio. Una creatura intrappolata può usare due azioni per effettuare un nuovo Tiro Salvezza al fine di liberarsi.

Se lo desideri, dopo aver completato l'incantesimo, puoi trattenerti dallo sparare il globo. Un piccolo globo, circa delle dimensioni di una pietra da fionda, freddo al contatto, appare nella tua mano. In qualsiasi momento, tu, o una creatura a cui hai dato il globo, potete lanciare il globo (fino a una gittata di 12 metri). Questo si frantumerà all'impatto, con lo stesso effetto del normale lancio dell'incantesimo. Puoi anche poggiare il globo a terra senza che si frantumi. Dopo 1 minuto, se il globo non è già stato frantumato, esploderà.

\textbf{Per ogni Successo Critico Magico} ottenuto nella Prova di Magia il danno aumenta di 5d6.

\textbf{Tiro Salvezza Successo/Fallimento Critico}: In caso di Fallimento Critico il danno raddoppia, in caso di Successo Critico il danno viene ulteriormente dimezzato

\smallskip\noindent\rule{\linewidth}{2pt} \index[Incantesimi]{Sfera Infuocata}\hypertarget{Sfera Infuocata}{}\smallskip\noindent{\textbf{Sfera Infuocata}}\pdfbookmark[3]{Sfera Infuocata}{Sfera Infuocata}
\noindent
\begin{description}[noitemsep, topsep=0pt, parsep=0pt, partopsep=0pt, leftmargin=0cm, labelwidth=2.8cm]
	\item[\textbf{Lista di Magia}]: Fuoco
	\item[\textbf{Livello}]: 2, Comune
	\item[\textbf{T. di Lancio}]: 2 Azioni
	\item[\textbf{Gittata}]: 18 metri
	\item[\textbf{Componenti}]: V, S, M (un pò di sego, un pizzico di zolfo e una manciata di ferro in polvere)
	\item[\textbf{Durata}]: 1 minuto
\end{description}

Per la durata dell'incantesimo compare una sfera di 1 metro di diametro in uno spazio a gittata, scelto da te. Qualsiasi creatura che termini il suo round entro 1 metro dalla sfera deve effettuare un Tiro Salvezza su Riflessi. La creatura subisce 2d6 danni da fuoco se fallisce il Tiro Salvezza, o la metà di questi danni se lo supera.

Con un'Azione puoi spostare la sfera di 9 metri. Se fai schiantare la sfera contro una creatura, la creatura deve effettuare un Tiro Salvezza contro il danno della sfera, e la sfera smetterà di muoversi per quel round.
Quando muovi la sfera, la puoi spostare oltre barriere alte fino a 1 metro, e farle saltare spazi larghi fino a 3 metri. La sfera incendia gli oggetti infiammabili non indossati o trasportati, e irradia una luce intensa in un raggio di 3 metri e una luce fioca per ulteriori 3 metri.

Mentre hai questo incantesimo attivo sei Distratto nel lancio di altri incantesimi.

\textbf{Per ogni Successo Critico Magico} ottenuto nella Prova di Magia il danno aumenta di 1d8.

\smallskip\noindent\rule{\linewidth}{2pt} \index[Incantesimi]{Sfocatura}\hypertarget{Sfocatura}{}\smallskip\noindent{\textbf{Sfocatura}}\pdfbookmark[3]{Sfocatura}{Sfocatura}
\noindent
\begin{description}[noitemsep, topsep=0pt, parsep=0pt, partopsep=0pt, leftmargin=0cm, labelwidth=2.8cm]
	\item[\textbf{Lista di Magia}]: Illusione
	\item[\textbf{Livello}]: 2, Comune
	\item[\textbf{T. di Lancio}]: 2 Azioni
	\item[\textbf{Gittata}]: Personale
	\item[\textbf{Componenti}]: V
	\item[\textbf{Durata}]: 1 minuto
\end{description}

Il tuo corpo diventa sfocato, indistinto e tremolante a chiunque ti veda. Per la durata dell'incantesimo, tutte le creature hanno ha -1d6 ai Tiri per Colpire contro di te. Gli attaccanti che non si affidano alla vista sono immuni a questo effetto, per esempio se hanno vista cieca o sono in grado di distinguere le illusioni, come per visione del vero.

\smallskip\noindent\rule{\linewidth}{2pt} \index[Incantesimi]{Sguardo Penetrante}\hypertarget{Sguardo Penetrante}{}\smallskip\noindent{\textbf{Sguardo Penetrante}}\pdfbookmark[3]{Sguardo Penetrante}{Sguardo Penetrante}
\noindent
\begin{description}[noitemsep, topsep=0pt, parsep=0pt, partopsep=0pt, leftmargin=0cm, labelwidth=2.8cm]
	\item[\textbf{Lista di Magia}]: Necromanzia
	\item[\textbf{Livello}]: 6, Molto Raro
	\item[\textbf{T. di Lancio}]: 2 Azioni
	\item[\textbf{Gittata}]: Personale
	\item[\textbf{Componenti}]: V, S
	\item[\textbf{Durata}]: Concentrazione, massimo 1 minuto
\end{description}

Per la durata dell'incantesimo i tuoi occhi si tramutano in un vuoto nero infuso di terribile potere. Una creatura a tua scelta entro 18 metri da te e che puoi vedere deve superare un Tiro Salvezza su Volontà o, per la durata, subire uno dei seguenti effetti di tua scelta. Durante ciascun tuo round, fino al termine dell'incantesimo, puoi usare due Azioni per prendere come bersaglio un'altra creatura, ma non puoi prendere di nuovo come bersaglio una creatura che abbia superato un Tiro Salvezza contro questo lancio di sguardo penetrante.

\begin{itemize}[leftmargin=*] \setlength{\itemsep}{0pt}
	\item \emph{Addormentato}. Il bersaglio cade privo di sensi. Si risveglia qualora subisca qualsiasi ammontare di danno o se un'altra creatura usa 2 Azioni per scuoterlo dal sonno.
	\item \emph{Ammalato}. Il bersaglio ha -1d6 ai Tiri per Colpire e le prove su competenze di base. Al termine di ciascun suo round, può effettuare un altro Tiro Salvezza su Volontà. Se lo supera, l'effetto ha termine.
	\item \emph{Impanicato}. Il bersaglio è spaventato da te. Durante ciascun suo round, la creatura spaventata deve effettuare usare due Azioni di Movimento e muoversi lontano da te tramite il tragitto più breve e sicuro possibile, a meno che non abbia spazio per muoversi. Se il bersaglio si muove in un luogo lontano almeno 18 metri da te, dove non ti possa vedere, questo effetto ha termine.
\end{itemize}

\smallskip\noindent\rule{\linewidth}{2pt} \index[Incantesimi]{Silenzio}\hypertarget{Silenzio}{}\smallskip\noindent{\textbf{Silenzio}}\pdfbookmark[3]{Silenzio}{Silenzio}
\noindent
\begin{description}[noitemsep, topsep=0pt, parsep=0pt, partopsep=0pt, leftmargin=0cm, labelwidth=2.8cm]
	\item[\textbf{Lista di Magia}]: Illusione
	\item[\textbf{Livello}]: 2, Comune
	\item[\textbf{T. di Lancio}]: 2 Azioni
	\item[\textbf{Gittata}]: 36 metri
	\item[\textbf{Componenti}]: V, S
	\item[\textbf{Durata}]: 10 minuti
\end{description}

Per la durata dell'incantesimo, nessun suono può essere creato all'interno o attraversare una sfera di 6 metri di raggio centrata su di un punto a gittata, scelto da te. Qualsiasi creatura o oggetto che si trovi completamente all'interno della sfera è immune al danno da suono e le creature che sono completamente al suo interno sono assordate. È estremamente difficile (vedi \hyperlink{magieconimpedimenti}{Tentare magie con impedimenti}, pag. \pageref{magieconimpedimenti} ) lanciare un incantesimo che comprende una componente verbale mentre si è al suo interno. Lanciato su una creatura è concesso un Tiro Salvezza su Volontà per annullarne gli effetti.

\textbf{Per ogni Successo Critico Magico} ottenuto nella Prova di Magia la durata aumenta di 10 minuti.

\textbf{Con due Successo Critico Magico} puoi lanciare l'incantesimo su un oggetto o creatura che diventa il centro della sfera. In caso di creature è concesso un Tiro Salvezza su Volontà.

\smallskip\noindent\rule{\linewidth}{2pt} \index[Incantesimi]{Simbolo}\hypertarget{Simbolo}{}\smallskip\noindent{\textbf{Simbolo}}\pdfbookmark[3]{Simbolo}{Simbolo}
\noindent
\begin{description}[noitemsep, topsep=0pt, parsep=0pt, partopsep=0pt, leftmargin=0cm, labelwidth=2.8cm]
	\item[\textbf{Lista di Magia}]: Abiurazione
	\item[\textbf{Livello}]: 7, Non Comune
	\item[\textbf{T. di Lancio}]: 2 Azioni
	\item[\textbf{Gittata}]: Contatto
	\item[\textbf{Componenti}]: V, S, M (mercurio, fosforo e diamante e opale in polvere con un valore totale di almeno 1000 mo, che l'incantesimo consuma)
	\item[\textbf{Durata}]: Fino a che dissolto o attivato
\end{description}

Quando lanci questo incantesimo, inscrivi un glifo dannoso su di una superficie (come una sezione di pavimento, muro o un tavolo) o all'interno di un oggetto che può essere chiuso per nascondere il glifo (come un libro, una pergamena o un forziere). Se scegli una superficie, il glifo può coprire un'area di superficie non maggiore di 3 metri di diametro. Se scegli un oggetto,quell'oggetto deve restare al suo posto; se l'oggetto viene spostato più di 3 metri dal punto in cui è stato lanciato l'incantesimo, il glifo è spezzato, e l'incantesimo termina senza essere stato attivato.

Il glifo è quasi invisibile e può essere trovato con una prova di Sopravvivenza contro la DC del Tiro Salvezza dei tuoi incantesimi.

Decidi tu cosa attivi il glifo al momento del lancio dell'incantesimo.

Per i glifi inscritti su di una superficie, l'attivazione tipica comprende entrare in contatto o stare sopra il glifo, rimuovere un altro oggetto che copra il glifo, avvicinarsi a una certa distanza dal glifo, o manipolare l'oggetto su cui è inscritto il glifo.

Per i glifi inscritti su di un oggetto, l'attivazione tipica comprende aprire l'oggetto, avvicinarsi a una certa distanza dall'oggetto, o vedere o leggere il glifo.

Puoi definire meglio l'attivazione così che l'incantesimo si attivi solo in determinate circostanze o secondo certe peculiarità fisiche (come l'altezza o il peso) o specie di creatura (per esempio, la protezione potrebbe agire contro le megere o i mutaforma). Puoi anche predisporre condizioni per evitare che il glifo venga attivato, come la pronuncia di una parola d'ordine.

Quando inscrivi il glifo scegli una delle opzioni seguenti come suo effetto. Una volta attivato, il glifo riluce, riempiendo una sfera di 18 metri di raggio di luce fioca per 10 minuti, dopo i quali l'incantesimo termina. Ogni creatura nella sfera quando il glifo si attiva diventa bersaglio del suo effetto, così come una creatura che entri per la prima volta nella sfera durante un round o termine lì il suo round.

\begin{itemize}[leftmargin=*] \setlength{\itemsep}{0pt}
	\item \emph{Demenza}. Ogni bersaglio deve effettuare un Tiro Salvezza su Volontà. Se fallisce il Tiro Salvezza, il bersaglio diventa demente per 1 minuto. Una creatura demente non può effettuare azioni, non comprende quello che gli altri le dicono, non può leggere, e parla solo farfugliando. Il Narratore ne controlla i movimenti, che risultano erratici.
	\item \emph{Discordia}. Ogni bersaglio deve effettuare un Tiro Salvezza su Tempra. Se lo fallisce, il bersaglio inizia a bisticciare e discutere con un'altra creatura per 1 minuto. In questo periodo, è incapace di effettuare qualsiasi comunicazione significativa e ha -1d6 ai Tiri per Colpire e le prove su competenze di base.
	\item \emph{Dolore}. Ogni bersaglio deve effettuare un Tiro Salvezza su Tempra. Se lo fallisce, il bersaglio diventa inabile a causa del dolore lacerante.
	\item \emph{Morte}. Ogni bersaglio deve effettuare un Tiro Salvezza su Tempra, subendo 10d10 danni da Vuoto se lo fallisce, o la metà di questi danni se lo supera.
	\item \emph{Paura}. Ogni bersaglio deve effettuare un Tiro Salvezza su Volontà e, se lo fallisce, restare spaventato per 1 minuto. Mentre è spaventato, il bersaglio getta qualsiasi cosa stesse tenendo e deve muoversi almeno 9 metri lontano dal glifo durante ciascuno suo round, se in grado.
	\item \emph{Sfiducia}. Ogni bersaglio deve effettuare un Tiro Salvezza su Volontà. Se fallisce il Tiro Salvezza, il bersaglio è sopraffatto dalla disperazione per 1 minuto. Durante questo periodo, non può attaccare o prendere come bersaglio nessuna creatura con capacità, incantesimi o altri effetti magici nocivi.
	\item \emph{Sonno}. Ogni bersaglio deve effettuare un Tiro Salvezza su Volontà, e cadere privo di sensi per 10 minuti se lo fallisce. Una creatura si risveglia se subisce danni o se qualcuno usa un'Azione per risvegliarla.
	\item \emph{Stordimento}. Ogni bersaglio deve effettuare un Tiro Salvezza su Volontà, e restare stordito per 1 minuto se lo fallisce.
\end{itemize}

\smallskip\noindent\rule{\linewidth}{2pt} \index[Incantesimi]{Sogno}\hypertarget{Sogno}{}\smallskip\noindent{\textbf{Sogno}}\pdfbookmark[3]{Sogno}{Sogno}
\noindent
\begin{description}[noitemsep, topsep=0pt, parsep=0pt, partopsep=0pt, leftmargin=0cm, labelwidth=2.8cm]
	\item[\textbf{Lista di Magia}]: Illusione
	\item[\textbf{Livello}]: 5, Non Comune
	\item[\textbf{T. di Lancio}]: 2 Azioni
	\item[\textbf{Gittata}]: Speciale
	\item[\textbf{Componenti}]: V, S, M (una manciata di sabbia, una punta di inchiostro, e una penna per scrivere presa da un volatile addormentato)
	\item[\textbf{Durata}]: 8 ore
\end{description}

Questo incantesimo modella i sogni di una creatura. Scegli una creatura a te nota come bersaglio dell'incantesimo. Il bersaglio deve trovarsi sul tuo stesso piano di esistenza. Le creature che non dormono non possono essere soggette a questo incantesimo. Tu o una creatura consenziente con cui sei a contatto entrate in uno stato di trance, agendo da messaggero. Mentre è in trance, il messaggero è consapevole di ciò che lo circonda, ma non può effettuare azioni o muoversi.

Per la durata dell'incantesimo, se il bersaglio è addormentato, il messaggero appare nei sogni del bersaglio e può conversare con lui finché questi rimane addormentato. Il messaggero può anche modellare l'ambiente del sogno, creando terreni, oggetti e altre immagini. Il messaggero può emergere dalla trance in qualsiasi momento, terminando anticipatamente l'effetto dell'incantesimo. Al risveglio, il bersaglio ricorda perfettamente il suo sogno. Se il bersaglio è sveglio quando lanci l'incantesimo, il messaggero ne viene a conoscenza e può porre fine alla trance (e all'incantesimo) o aspettare che il bersaglio si addormenti. A quel punto il messaggero potrà comparire nei sogni del bersaglio.

Puoi fare apparire il messaggero al bersaglio con un aspetto mostruoso e terrificante. Se lo fai, il messaggero può consegnare un messaggio di al massimo dieci parole e poi il bersaglio deve effettuare un Tiro Salvezza su Volontà. Se fallisce il Tiro Salvezza, gli echi della spaventosa mostruosità generano un incubo per la durata del sonno del bersaglio che gli impedisce di ottenere qualsiasi beneficio da quel riposo. Inoltre, quando il bersaglio si sveglia, subisce 3d6 danni.

Se possiedi una ciocca di capelli, delle unghie tagliate, o simile porzione del corpo del bersaglio, egli effettuerà il suo Tiro Salvezza con -1d6.

\smallskip\noindent\rule{\linewidth}{2pt} \index[Incantesimi]{Sonnellino}\hypertarget{Sonnellino}{}\smallskip\noindent{\textbf{Sonnellino}}\pdfbookmark[3]{Sonnellino}{Sonnellino}
\noindent
\begin{description}[noitemsep, topsep=0pt, parsep=0pt, partopsep=0pt, leftmargin=0cm, labelwidth=2.8cm]
	\item[\textbf{Lista di Magia}]: Alterazione
	\item[\textbf{Livello}]: 2, Leggendario
	\item[\textbf{T. di Lancio}]: 1 round
	\item[\textbf{Gittata}]: 6 metri
	\item[\textbf{Componenti}]: V, S, M (una piuma, un pezzo di cotone bianco)
	\item[\textbf{Durata}]: 1 minuto
\end{description}

Questo incantesimo permette all'incantatore di mettere a riposo per 1 ora fino ad 1 creatura per (Competenza Magica/6 + Adepto della Magia). La creatura deve essere consenziente.

Quest'ora di riposo è equivalente a 8 ore di riposo per quanto riguarda il recupero dei Punti Magia e Punti Ferita. L'incantesimo non è usufruibile ad intervalli inferiori alle 35 ore.

\textbf{Per ogni Successo Critico Magico} ottenuto nella Prova di Magia influenzi 1 creatura in più.

\smallskip\noindent\rule{\linewidth}{2pt} \index[Incantesimi]{Sonno}\hypertarget{Sonno}{}\smallskip\noindent{\textbf{Sonno}}\pdfbookmark[3]{Sonno}{Sonno}
\noindent
\begin{description}[noitemsep, topsep=0pt, parsep=0pt, partopsep=0pt, leftmargin=0cm, labelwidth=2.8cm]
	\item[\textbf{Lista di Magia}]: Ammaliamento
	\item[\textbf{Livello}]: 1, Comune
	\item[\textbf{T. di Lancio}]: 2 Azioni
	\item[\textbf{Gittata}]: 27 metri
	\item[\textbf{Componenti}]: V, S, M (un pizzico di sabbia, petali di rosa o un grillo)
	\item[\textbf{Durata}]: 1 minuto
\end{description}

In un raggio di 2 metri le creature devono fare un Tiro Salvezza su Volontà o cadere addormentate. Qualsiasi cosa danneggi le creature o le influenzi causa l'immediato scioglimento del sonno magico. Le creature con Grado di Sfida superiore a 3 non sono influenzate.

\textbf{Per ogni due Successo Critico Magico} ottenuto nella Prova di Magia puoi influenzare creature 1 Grado Sfida più alto.

\smallskip\noindent\rule{\linewidth}{2pt} \index[Incantesimi]{Spada Arcana}\hypertarget{Spada Arcana}{}\smallskip\noindent{\textbf{Spada Arcana}}\pdfbookmark[3]{Spada Arcana}{Spada Arcana}
\noindent
\begin{description}[noitemsep, topsep=0pt, parsep=0pt, partopsep=0pt, leftmargin=0cm, labelwidth=2.8cm]
	\item[\textbf{Lista di Magia}]: Invocazione
	\item[\textbf{Livello}]: 7, Raro
	\item[\textbf{T. di Lancio}]: 2 Azioni
	\item[\textbf{Gittata}]: 18 metri
	\item[\textbf{Componenti}]: V, S, M (una spada di platino in miniatura con l'impugnatura e il pomello di rame e zinco, del valore di 250 mo)
	\item[\textbf{Durata}]: Concentrazione, massimo 1 minuto
\end{description}

Per la durata dell'incantesimo, crei a gittata una fluttuante spada di forza. Quando la spada appare, effettui un attacco in mischia con modificatore CM + modificatore da incantesimo contro un bersaglio scelto da te entro 1 metro dalla spada. Se colpisci, il bersaglio subisce 4d10 danni da forza. Fino al termine dell'incantesimo,puoi usare un'Azione ogni tuo round per muovere la spada di 6 metri in un punto che puoi vedere e ripetere questo attacco contro lo stesso bersaglio o uno differente.

\smallskip\noindent\rule{\linewidth}{2pt} \index[Incantesimi]{Spruzzo Colorato}\hypertarget{Spruzzo Colorato}{}\smallskip\noindent{\textbf{Spruzzo Colorato}}\pdfbookmark[3]{Spruzzo Colorato}{Spruzzo Colorato}
\noindent
\begin{description}[noitemsep, topsep=0pt, parsep=0pt, partopsep=0pt, leftmargin=0cm, labelwidth=2.8cm]
	\item[\textbf{Lista di Magia}]: Illusione
	\item[\textbf{Livello}]: 1, Comune
	\item[\textbf{T. di Lancio}]: 2 Azioni
	\item[\textbf{Gittata}]: Personale (cono di 3 metri)
	\item[\textbf{Componenti}]: V, S, M (un pizzico di polvere o sabbia che sia colorata di rosso, giallo e blu)
	\item[\textbf{Durata}]: 1 round
\end{description}

Uno spruzzo di luci e colori erutta dalla tua mano. Le creature in un cono di 3 metri devono fare un Tiro Salvezza su Volontà.

Se il Tiro Salvezza riesce non si subisce alcun effetto, se fallisce la creatura \hyperlink{confusionecondizione}{confusa} per 1 round.

\textbf{Per ogni Successo Critico Magico} ottenuto nella Prova di Magia la durata aumenta di 1 round.

\smallskip\noindent\rule{\linewidth}{2pt} \index[Incantesimi]{Spruzzo Prismatico}\hypertarget{Spruzzo Prismatico}{}\smallskip\noindent{\textbf{Spruzzo Prismatico}}\pdfbookmark[3]{Spruzzo Prismatico}{Spruzzo Prismatico}
\noindent
\begin{description}[noitemsep, topsep=0pt, parsep=0pt, partopsep=0pt, leftmargin=0cm, labelwidth=2.8cm]
	\item[\textbf{Lista di Magia}]: Invocazione
	\item[\textbf{Livello}]: 7, Raro
	\item[\textbf{T. di Lancio}]: 2 Azioni
	\item[\textbf{Gittata}]: Personale (cono di 18 metri)
	\item[\textbf{Componenti}]: V, S
	\item[\textbf{Durata}]: Istantanea
\end{description}

Otto raggi di luce multicolore partono dalla tua mano. Ogni raggio è di un diverso colore e ha un potere e uno scopo diverso. Ogni creatura in un cono di 18 metri deve effettuare un Tiro Salvezza su Riflessi. Per ogni bersaglio, tirare un d8 per determinare quale sia il colore del raggio che lo ha colpito.

\begin{itemize}[leftmargin=*] \setlength{\itemsep}{0pt}
	\item \emph{1. Rosso}. Il bersaglio subisce 10d6 danni da fuoco se fallisce il Tiro Salvezza, o la metà di questi danni se lo supera.
	\item \emph{2. Arancio}. Il bersaglio subisce 10d6 danni da acido se fallisce il Tiro Salvezza, o la metà di questi danni se lo supera.
	\item \emph{3. Giallo}. Il bersaglio subisce 10d6 danni da elettricità se fallisce il Tiro Salvezza, o la metà di questi danni se lo supera.
	\item \emph{4. Verde}. Il bersaglio subisce 10d6 danni da veleno se fallisce il Tiro Salvezza, o la metà di questi danni se lo supera.
	\item \emph{5. Blu}. Il bersaglio subisce 10d6 danni da freddo se fallisce il Tiro Salvezza, o la metà di questi danni se lo supera.
	\item \emph{6. Indaco}. Se fallisce il Tiro Salvezza, il bersaglio è intralciato. Deve poi effettuare un Tiro Salvezza su Tempra all'inizio di ciascun suo round. Se supera il Tiro Salvezza tre volte, l'incantesimo termina. Se fallisce il Tiro Salvezza tre volte, viene permanentemente trasformato in pietra e diventa vittima della condizione pietrificato. I successi e i fallimenti non devono essere consecutivi; tieni traccia di entrambi finché il bersaglio non ne ha ottenuti tre dello stesso tipo.
	\item \emph{7. Violetto}. Se fallisce il Tiro Salvezza, il bersaglio è accecato. Deve poi effettuare un Tiro Salvezza su Volontà all'inizio del tuo prossimo round. Se supera il Tiro Salvezza, la cecità termina. Se fallisce il Tiro Salvezza, la creatura viene trasportata su di un altro piano di esistenza a scelta del Narratore e non è più accecata (di solito, una creatura che non è sul suo piano natio, viene esiliata su di esso, mentre le altre creature sono di solito portate nei piani Astrale o Etereo).
	\item \emph{8. Speciale}. Il bersaglio è colpito da due raggi. Tira altre due volte, ritirando gli 8.

\end{itemize}

\smallskip\noindent\rule{\linewidth}{2pt} \index[Incantesimi]{Spruzzo Velenoso}\hypertarget{Spruzzo Velenoso}{}\smallskip\noindent{\textbf{Spruzzo Velenoso}}\pdfbookmark[3]{Spruzzo Velenoso}{Spruzzo Velenoso}
\noindent
\begin{description}[noitemsep, topsep=0pt, parsep=0pt, partopsep=0pt, leftmargin=0cm, labelwidth=2.8cm]
	\item[\textbf{Lista di Magia}]: Animali e Piante
	\item[\textbf{Livello}]: 0, Non Comune
	\item[\textbf{T. di Lancio}]: 1 Azione
	\item[\textbf{Gittata}]: 3 metri
	\item[\textbf{Componenti}]: V, S
	\item[\textbf{Durata}]: Istantanea
\end{description}

Stendi la mano verso una creatura a gittata e che puoi vedere, e proietti una nube di gas velenoso dal tuo palmo. La creatura deve superare un Tiro Salvezza su Tempra o subire 1d12 danni da veleno.

Puoi aumentare il danno dell'incantesimo di 1d8 quando raggiungi CM 5, CM 11 e CM 17, ma costa 2 Azioni lanciarlo potenziato e 1 Punti Magia, è altresì necessario avere preso Adepto della Magia un numero di volte pari ai potenziamenti che si vogliono applicare.

\textbf{Per ogni due Successo Critico Magico ottenuto} nella Prova di Magia influenzi un altra creatura entro gittata.


\smallskip\noindent\rule{\linewidth}{2pt} \index[Incantesimi]{Statua}\hypertarget{Statua}{}\smallskip\noindent{\textbf{Statua}}\pdfbookmark[3]{Statua}{Statua}
\noindent
\begin{description}[noitemsep, topsep=0pt, parsep=0pt, partopsep=0pt, leftmargin=0cm, labelwidth=2.8cm]
	\item[\textbf{Lista di Magia}]: Terra, Trasmutazione
	\item[\textbf{Livello}]: 7, Raro
	\item[\textbf{T. di Lancio}]: 2 Azioni
	\item[\textbf{Gittata}]: Tocco
	\item[\textbf{Componenti}]: V, S, M (un frammento di pietra)
	\item[\textbf{Durata}]: 1 ora per livello
\end{description}

Questo incantesimo trasforma l'incantatore o il soggetto consenziente in pietra, insieme a qualsiasi abito o oggetto trasportato. Il soggetto può vedere e percepire suoni e odori, ma non ha bisogno di mangiare o respirare. Il senso del tatto è limitato alle sensazioni percepibili dalla sostanza granitica di cui è composto il corpo del soggetto. Una scheggiatura è paragonabile a un semplice graffio, ma spezzare un braccio della statua equivale a una mutilazione. Il soggetto di statua può tornare allo stato normale e ridiventare di pietra tutte le volte che vuole durante la durata dell'incantesimo. La statua ha durezza 15 ed il doppio dei Punti Ferita della creatura originaria.

\textbf{Per ogni Successo Critico Magico} ottenuto nella Prova di Magia raddoppi la durata o influenzi un altra creatura.

\smallskip\noindent\rule{\linewidth}{2pt} \index[Incantesimi]{Stretta Folgorante}\hypertarget{Stretta Folgorante}{}\smallskip\noindent{\textbf{Stretta Folgorante}}\pdfbookmark[3]{Stretta Folgorante}{Stretta Folgorante}
\noindent
\begin{description}[noitemsep, topsep=0pt, parsep=0pt, partopsep=0pt, leftmargin=0cm, labelwidth=2.8cm]
	\item[\textbf{Lista di Magia}]: Aria
	\item[\textbf{Livello}]: 0, Comune
	\item[\textbf{T. di Lancio}]: 1 Azione
	\item[\textbf{Gittata}]: Contatto
	\item[\textbf{Componenti}]: V, S
	\item[\textbf{Durata}]: Istantanea
\end{description}

Dalle tue mani saettano lampi che infliggono una scossa a una creatura con cui provi a entrare in contatto. Effettua un attacco in mischia con incantesimo contro il bersaglio. Hai +1d6 sul Tiro per Colpire se il bersaglio sta indossando un'armatura fatta di metallo. Se colpisci, il bersaglio subisce 1d8 danni da elettricità, e non può effettuare reazioni fino all'inizio del suo prossimo round.

Puoi aumentare il danno dell'incantesimo di 1d8 quando raggiungi CM 5, CM 11 e CM 17, ma costa 2 Azioni lanciarlo potenziato e 1 Punti Magia, è altresì necessario avere preso Adepto della Magia un numero di volte pari ai potenziamenti che si vogliono applicare.

\textbf{Per ogni Successo Critico Magico ottenuto} l'incantesimo \emph{può saltare} su un altra creatura nemica entro 1 metro da quella iniziale o somma 1d6 di danno aggiuntivo.

\smallskip\noindent\rule{\linewidth}{2pt} \index[Incantesimi]{Succo concentrato di Ribes di Kyrin}\hypertarget{Succo concentrato di Ribes di Kyrin}{}\smallskip\noindent{\textbf{Succo concentrato di Ribes di Kyrin}}\pdfbookmark[3]{Succo concentrato di Ribes di Kyrin}{Succo concentrato di Ribes di Kyrin}
\noindent
\begin{description}[noitemsep, topsep=0pt, parsep=0pt, partopsep=0pt, leftmargin=0cm, labelwidth=2.8cm]
	\item[\textbf{Lista di Magia}]: Animali e Piante, Terra
	\item[\textbf{Livello}]: 2, Non Comune
	\item[\textbf{T. di Lancio}]: 2 Azioni
	\item[\textbf{Gittata}]: 9 metri
	\item[\textbf{Componenti}]: V, M (12 ribes che l'incantesimo consuma)
	\item[\textbf{Durata}]: 1 minuto
\end{description}

Estrai la linfa acida dai ribes e proietti una linea di spruzzo d'acido lunga 9 metri e larga 1 metro in una direzione a tua scelta. Ogni creatura nella linea deve superare un Tiro Salvezza su Riflessi o essere ricoperta di acido per la durata dell'incantesimo o finché una creatura non usa due Azioni per togliere via l'acido da sé o da un'altra creatura. Una creatura coperta dall'acido subisce 2d4 danni da acido all'inizio di ciascuno dei suoi round.

\textbf{Per ogni Successo Critico Magico} ottenuto nella Prova di Magia il danno aumenta di 2d4

\smallskip\noindent\rule{\linewidth}{2pt} \index[Incantesimi]{Suggestione}\hypertarget{Suggestione}{}\smallskip\noindent{\textbf{Suggestione}}\pdfbookmark[3]{Suggestione}{Suggestione}
\noindent
\begin{description}[noitemsep, topsep=0pt, parsep=0pt, partopsep=0pt, leftmargin=0cm, labelwidth=2.8cm]
	\item[\textbf{Lista di Magia}]: Ammaliamento
	\item[\textbf{Livello}]: 2, Raro
	\item[\textbf{T. di Lancio}]: 2 Azioni
	\item[\textbf{Gittata}]: 9 metri
	\item[\textbf{Componenti}]: V, M (la lingua di un serpente e un pezzo di favo o un goccio di olio evo)
	\item[\textbf{Durata}]: 8 ore
\end{description}

Suggerisci un corso di attività (limitato a una o due frasi) e influenzi magicamente una creatura a gittata e che puoi vedere e udire e ti possa capire, scelta da te. Le creature che non possono essere affascinate sono immuni a questo effetto. La suggestione deve essere pronunciata in modo che il corso d'azione suoni ragionevole. Chiedere a una creatura di pugnalarsi,gettarsi su una lancia, darsi fuoco, o fare qualche altro atto palesemente dannoso nega automaticamente gli effetti dell'incantesimo.

Il bersaglio deve effettuare un Tiro Salvezza su Volontà. Se fallisce il Tiro Salvezza, esso segue il corso d'azione da te descritto al meglio delle sue capacità. Il corso d'azione suggerito può proseguire per l'intera durata dell'incantesimo. Se l'attività suggerita può essere completata in un tempo più breve,l'incantesimo ha termine quando il soggetto termina di fare ciò che gli è stato chiesto.

Puoi anche specificare condizioni che attiveranno un'attività speciale per la durata dell'incantesimo. Per esempio, potresti suggerire a un cavaliere di cedere il suo saurovallo da guerra al primo mendicante che incontri. Se la condizione non viene soddisfatta prima del termine dell'incantesimo, l'attività non verrà svolta. Se tu o uno qualsiasi dei tuoi compagni danneggia il bersaglio, l'incantesimo ha termine.

\smallskip\noindent\rule{\linewidth}{2pt} \index[Incantesimi]{Suggestione di Massa}\hypertarget{Suggestione di Massa}{}\smallskip\noindent{\textbf{Suggestione di Massa}}\pdfbookmark[3]{Suggestione di Massa}{Suggestione di Massa}
\noindent
\begin{description}[noitemsep, topsep=0pt, parsep=0pt, partopsep=0pt, leftmargin=0cm, labelwidth=2.8cm]
	\item[\textbf{Lista di Magia}]: Ammaliamento
	\item[\textbf{Livello}]: 6, Molto Raro
	\item[\textbf{T. di Lancio}]: 2 Azioni
	\item[\textbf{Gittata}]: 18 metri
	\item[\textbf{Componenti}]: V, M (la lingua di un serpente e un pezzo di favo o un goccio di olio dolce)
	\item[\textbf{Durata}]: 24 ore
\end{description}

Suggerisci un corso di attività (limitato a una o due frasi) e influenzi magicamente fino a dodici creature a gittata che puoi vedere e udire e ti possano capire, scelte da te. Le creature che non possono essere affascinate sono immuni a questo effetto. La suggestione deve essere pronunciata in modo che il corso d'azione suoni ragionevole. Chiedere a una creatura di pugnalarsi, gettarsi su di una lancia, darsi fuoco, o fare qualche altro atto palesemente dannoso nega automaticamente gli effetti dell'incantesimo.

Ogni bersaglio deve effettuare un Tiro Salvezza su Volontà. Se fallisce il Tiro Salvezza, esso segue il corso d'azione da te descritto al meglio delle sue capacità. Il corso d'azione suggerito può proseguire per l'intera durata dell'incantesimo. Se l'attività suggerita può essere completata in un tempo più breve, l'incantesimo ha termine quando il soggetto termina di fare ciò che gli è stato chiesto.

Puoi anche specificare condizioni che attiveranno un'attività speciale per la durata dell'incantesimo. Per esempio, potresti suggerire a un gruppo di soldati di cedere tutti i loro soldi al primo mendicante che incontrino. Se la condizione non viene soddisfatta prima del termine dell'incantesimo, l'attività non verrà svolta. Se tu o uno qualsiasi dei tuoi compagni danneggia una creatura soggetta a questo incantesimo, per quella creatura l'incantesimo ha termine.

\textbf{Per ogni Successo Critico Magico} ottenuto nella Prova di Magia aggiungi un giorno alla durata.

\smallskip\noindent\rule{\linewidth}{2pt} \index[Incantesimi]{Taumaturgia}\hypertarget{Taumaturgia}{}\smallskip\noindent{\textbf{Taumaturgia}}\pdfbookmark[3]{Taumaturgia}{Taumaturgia}
\noindent
\begin{description}[noitemsep, topsep=0pt, parsep=0pt, partopsep=0pt, leftmargin=0cm, labelwidth=2.8cm]
	\item[\textbf{Lista di Magia}]: Universale
	\item[\textbf{Livello}]: 0, Non Comune
	\item[\textbf{T. di Lancio}]: 2 Azioni
	\item[\textbf{Gittata}]: 9 metri
	\item[\textbf{Componenti}]: V
	\item[\textbf{Durata}]: Massimo 1 minuto
\end{description}

Manifesti a gittata una trucco minore, un segno di potere soprannaturale. Crei a gittata uno dei seguenti effetti magici:

\begin{itemize}[leftmargin=*] \setlength{\itemsep}{0pt}
	\item La tua voce risuona tre volte più forte del normale per 1 minuto.
	\item Fai sì che le fiamme tremolino, si intensifichino, affievoliscano o cambino colore per 1 minuto.
	\item Provochi innocui tremori sul terreno per 1 minuto.
	\item Crei un rumore istantaneo, come un rombo di tuono, il verso di un corvo, o un sussurro inquietante, che origina da un punto a gittata scelto da te.
	\item Fai sì che una porta o una finestra non chiusa a chiave si spalanchi o si chiuda di colpo.
	\item Modifichi l'aspetto dei tuoi occhi per 1 minuto.
\end{itemize}

Se lanci questo incantesimo più volte, puoi tenere attivi fino a tre effetti da un minuto alla volta e puoi interrompere questi effetti con un'Azione.

\textbf{Per ogni Successo Critico Magico} ottenuto nella Prova di Magia puoi manifestare un effetto magico aggiuntivo.

\smallskip\noindent\rule{\linewidth}{2pt} \index[Incantesimi]{Telecinesi}\hypertarget{Telecinesi}{}\smallskip\noindent{\textbf{Telecinesi}}\pdfbookmark[3]{Telecinesi}{Telecinesi}
\noindent
\begin{description}[noitemsep, topsep=0pt, parsep=0pt, partopsep=0pt, leftmargin=0cm, labelwidth=2.8cm]
	\item[\textbf{Lista di Magia}]: Trasmutazione
	\item[\textbf{Livello}]: 5, Non Comune
	\item[\textbf{T. di Lancio}]: 2 Azioni
	\item[\textbf{Gittata}]: 18 metri
	\item[\textbf{Componenti}]: V, S
	\item[\textbf{Durata}]: Concentrazione, massimo 10 minuti
\end{description}

Ottieni la capacità di muovere o manipolare creature o oggetti tramite il pensiero. Quando lanci questo incantesimo e con 2 Azioni durante ciascun round, puoi esercitare la tua volontà su di una creatura od oggetto a gittata e che puoi vedere, provocando l'effetto appropriato tra quelli seguenti. Puoi agire round dopo round sempre sullo stesso bersaglio, o sceglierne uno nuovo ogni volta. Se cambi bersaglio, il bersaglio precedente non è più soggetto all'incantesimo.

\emph{Creatura}. Puoi tentare di muovere una creatura di taglia Enorme o più piccola. Effettua un Tiro Salvezza contrapposto tra la tua Volontà con modificatore la tua caratteristica da incantatore contro un Tiro Salvezza su Tempra. Se vinci la contesa, muovi la creatura di 9 metri in qualsiasi direzione, compreso verso l'alto, ma senza eccedere la gittata dell'incantesimo. Fino al termine del tuo prossimo round, la creatura è intralciata dalla tua presa telecinetica. Una creatura sollevata in alta, resta sospesa a mezz'aria.

Nei round successivi, puoi usare 2 Azioni per tentare di mantenere la tua presa telecinetica sulla creatura ripetendo la contesa.

\emph{Oggetto}. Puoi tentare di muovere un oggetto che pesa fino a 500 chili. Se l'oggetto non è indossato o trasportato, lo sposti automaticamente di 9 metri in qualsiasi direzione, ma senza superare la gittata dell'incantesimo.

Se l'oggetto è indossato o trasportato da una creatura, devi effettuare un Tiro Salvezza contrapposto tra la tua Volontà con modificatore la tua caratteristica da incantatore contro un Tiro Salvezza su Tempra modificato da Forza della creatura che lo trattiene. Se vinci la contesa, trascini via l'oggetto da quella creatura e lo muovi di 9 metri in una qualsiasi direzione, senza però superare la gittata dell'incantesimo.

Puoi esercitare un controllo preciso sugli oggetti tramite la tua presa telecinetica, riuscendo così a manipolare un attrezzo semplice, aprire una porta o un contenitore,inserire o recuperare un oggetto da un contenitore aperto, o versare del materiale in una fiala.

\smallskip\noindent\rule{\linewidth}{2pt} \index[Incantesimi]{Teletrasporto}\hypertarget{Teletrasporto}{}\smallskip\noindent{\textbf{Teletrasporto}}\pdfbookmark[3]{Teletrasporto}{Teletrasporto}
\noindent
\begin{description}[noitemsep, topsep=0pt, parsep=0pt, partopsep=0pt, leftmargin=0cm, labelwidth=2.8cm]
	\item[\textbf{Lista di Magia}]: Evocazione
	\item[\textbf{Livello}]: 7, Comune
	\item[\textbf{T. di Lancio}]: 2 Azioni
	\item[\textbf{Gittata}]: 3 metri
	\item[\textbf{Componenti}]: V
	\item[\textbf{Durata}]: Istantanea
\end{description}

Questo incantesimo teletrasporta istantaneamente te e altre otto creature consenzienti (oppure un singolo oggetto) a gittata e che puoi vedere, scelte da te, in una destinazione di tua scelta. Se il bersaglio è un oggetto, deve poter entrare in una sfera di 2 metri di raggio e non può essere tenuto o trasportato da una creatura non consenziente.

La destinazione che scegli ti deve essere nota, e deve essere sullo stesso piano di esistenza in cui ti trovi. La tua familiarità con la destinazione determina se vi riesce ad arrivare.

Il Narratore tira un d100 e consulta la tabella.

\begin{itemize}[leftmargin=*] \setlength{\itemsep}{0pt}
	\item \emph{Cerchio permanente} indica un cerchio di teletrasporto permanente di cui conosci la sequenza dei sigilli.
	\item \emph{Oggetto associato} indica che possiedi uno oggetto preso negli ultimi sei mesi dalla destinazione desiderata, come il libro della biblioteca di un mago, biancheria della suite reale, o un pezzo di marmo della tomba segreta di un lich.
	\item \emph{Molto familiare} è un luogo in cui sei stato molto spesso, un posto che hai studiato attentamente, o un posto che puoi vedere quando lanci l'incantesimo.
	\emph{Visto casualmente} è un posto che hai visto più di una volta ma con cui non sei molto familiare.
	\item \emph{Visto una volta} è un posto che hai visto una volta sola, magari tramite la magia.\\ \emph{Descrizione} è un luogo la cui posizione e aspetto conosci solo tramite la descrizione di qualcun altro, magari una mappa.
	\item \emph{Falsa destinazione} è un posto che non esiste. Magari hai cercato di scrutare il nascondiglio di un nemico ma hai invece visto un'illusione, oppure stai cercando di teletrasportarti in un posto familiare che non esiste più.
	\item \emph{Sul Bersaglio}. Tu e il tuo gruppo (o l'oggetto bersaglio) apparite dove desideri.
	\item \emph{Fuori Bersaglio}. Tu e il tuo gruppo (o l'oggetto bersaglio) apparite a una distanza casuale dalla destinazione in una direzione casuale. La distanza fuori bersaglio è 1d10 x 1d10 percento della distanza viaggiata. Per esempio, se hai provato a viaggiare per 180 chilometri, atterri fuori bersaglio e tiri 5 e 3 su due d10, allora saresti fuori bersaglio del 15\%, ovvero 27 chilometri. Il Narratore determina la direzione fuori bersaglio casualmente, tirando un d8 e indicando l'1 come nord, il 2 come nordest, il 3 come est e così via seguendo le direzioni della bussola. Se ti stai teletrasportando in una città costiera e finisci 27 chilometri al largo in mare, potresti essere nei guai!
	\item \emph{Area Simile}. Tu e il tuo gruppo (o l'oggetto bersaglio) finite in un'area diversa che è visualmente o tematicamente simile all'area bersaglio. Per esempio, se sei diretto al tuo laboratorio personale, potresti finire nel laboratorio di un altro incantatore o in un negozio di oggetti alchemici che possiede molti degli attrezzi e strumenti del tuo laboratorio. In genere, compari nel luogo simile più vicino, ma dato che l'incantesimo non ha limiti di gittata, potresti finire praticamente dovunque sullo stesso piano.
	\item \emph{Errore}. L'imprevedibile magia dell'incantesimo provoca un viaggio difficile. Ogni creatura teletrasportata (o l'oggetto bersaglio) subisce 3d10 danni da forza, e il Narratore ritira sulla tabella per vedere dove finiscano (possono capitare più errori, che infliggono danni ogni volta).
\end{itemize}

\end{multicols}

\medskip

\noindent\begin{tabularx}{0.95\textwidth}{lllll}
\toprule
d100 &Errore&Area Simile&Fuori Bersaglio&Sul Bersaglio\\
Cerchio permanente&-&-&-&01-100\\
Oggetto Associato&-&-&-&01-100\\
Molto Familiare&01-05&06-13&14-24&25-100\\
Visto per caso&01-33&34-43&44-53&54-100\\
Visto una volta&01-43&44-53&54-73&74-100\\
Descrizione&01-43&44-53&54-73&74-100\\
Falsa Destinazione&01-50&51-100&-&-
\end{tabularx}

\medskip

\begin{multicols}{2}

\smallskip\noindent\rule{\linewidth}{2pt} \index[Incantesimi]{Tempesta di Fuoco}\hypertarget{Tempesta di Fuoco}{}\smallskip\noindent{\textbf{Tempesta di Fuoco}}\pdfbookmark[3]{Tempesta di Fuoco}{Tempesta di Fuoco}
\noindent
\begin{description}[noitemsep, topsep=0pt, parsep=0pt, partopsep=0pt, leftmargin=0cm, labelwidth=2.8cm]
	\item[\textbf{Lista di Magia}]: Fuoco
	\item[\textbf{Livello}]: 7, Raro
	\item[\textbf{T. di Lancio}]: 2 Azioni
	\item[\textbf{Gittata}]: 45 metri
	\item[\textbf{Componenti}]: V, S
	\item[\textbf{Durata}]: Istantanea
\end{description}

Una tempesta composta di fiamme roboanti compare in un punto a gittata, scelto da te. L'area della tempesta consiste di un massimo di dieci cubi di 3 metri di spigolo adiacenti, che puoi disporre come preferisci. Ogni cubo deve avere almeno una faccia adiacente a quella di un altro cubo. Ogni creatura nell'area deve effettuare un Tiro Salvezza su Riflessi. Se lo fallisce subisce 7d10 danni da fuoco, o la metà di questi danni se lo supera. Il fuoco danneggia gli oggetti nell'area e incendia gli oggetti infiammabili che non sono indossati o trasportati. Se lo desideri, la vita vegetale nell'area resta illesa dagli effetti di questo incantesimo.

\textbf{Per ogni Successo Critico Magico} ottenuto nella Prova di Magia aumenti l'area di un cubo di 3 metri di spigolo.

\textbf{Tiro Salvezza Successo/Fallimento Critico}: In caso di Fallimento Critico il danno raddoppia, in caso di Successo Critico il danno viene ulteriormente dimezzato

\smallskip\noindent\rule{\linewidth}{2pt} \index[Incantesimi]{Tempesta di Ghiaccio}\hypertarget{Tempesta di Ghiaccio}{}\smallskip\noindent{\textbf{Tempesta di Ghiaccio}}\pdfbookmark[3]{Tempesta di Ghiaccio}{Tempesta di Ghiaccio}
\noindent
\begin{description}[noitemsep, topsep=0pt, parsep=0pt, partopsep=0pt, leftmargin=0cm, labelwidth=2.8cm]
	\item[\textbf{Lista di Magia}]: Acqua, Aria
	\item[\textbf{Livello}]: 4, Non Comune
	\item[\textbf{T. di Lancio}]: 2 Azioni
	\item[\textbf{Gittata}]: 90 metri
	\item[\textbf{Componenti}]: V, S, M (un pizzico di polvere e alcune gocce d'acqua)
	\item[\textbf{Durata}]: Istantanea
\end{description}

Una grandinata di ghiaccio si abbatte a terra in un cilindro di 6 metri di raggio e 12 metri di altezza centrato su di un punto a gittata. Ogni creatura nel cilindro deve effettuare un Tiro Salvezza su Riflessi. La creatura subisce 2d8 danni contundenti e 4d6 danni da freddo se fallisce il Tiro Salvezza, o la metà se lo supera. La grandine trasforma l'area di effetto della tempesta in terreno difficile fino al termine del tuo prossimo round.

\textbf{Per ogni Successo Critico Magico} ottenuto nella Prova di Magia il danno contundente aumenta di 2d8 e quello da freddo di 2d6.

\textbf{Tiro Salvezza Successo/Fallimento Critico}: In caso di Fallimento Critico il danno raddoppia, in caso di Successo Critico il danno viene ulteriormente dimezzato

\smallskip\noindent\rule{\linewidth}{2pt} \index[Incantesimi]{Tempesta di Nevischio}\hypertarget{Tempesta di Nevischio}{}\smallskip\noindent{\textbf{Tempesta di Nevischio}}\pdfbookmark[3]{Tempesta di Nevischio}{Tempesta di Nevischio}
\noindent
\begin{description}[noitemsep, topsep=0pt, parsep=0pt, partopsep=0pt, leftmargin=0cm, labelwidth=2.8cm]
	\item[\textbf{Lista di Magia}]: Acqua
	\item[\textbf{Livello}]: 3, Molto Raro
	\item[\textbf{T. di Lancio}]: 2 Azioni
	\item[\textbf{Gittata}]: 45 metri
	\item[\textbf{Componenti}]: V, S, M (un pizzico di polvere e qualche goccia d'acqua)
	\item[\textbf{Durata}]: 1 minuto
\end{description}

Fino al termine dell'incantesimo, pioggia gelida e nevischio si abbattono in un cilindro alto 6 metri e del raggio di 12 metri centrato in un punto da te scelto a gittata. L'area è in penombra, mentre le fiamme esposte vengono spente. Il terreno nell'area è coperto di ghiaccio scivoloso, rendendolo terreno difficile. Quando una creatura entra nell'area dell'incantesimo per la prima volta durante un round o inizia il suo round lì, deve effettuare un Tiro Salvezza su Riflessi. Se lo fallisce, cade prona. Se una creatura nell'area dell'incantesimo si sta concentrando, deve superare un Tiro Salvezza su Tempra contro la DC del Tiro Salvezza dell'incantesimo o perdere la concentrazione.

\smallskip\noindent\rule{\linewidth}{2pt} \index[Incantesimi]{Tentacoli Neri}\hypertarget{Tentacoli Neri}{}\smallskip\noindent{\textbf{Tentacoli Neri}}\pdfbookmark[3]{Tentacoli Neri}{Tentacoli Neri}
\noindent
\begin{description}[noitemsep, topsep=0pt, parsep=0pt, partopsep=0pt, leftmargin=0cm, labelwidth=2.8cm]
	\item[\textbf{Lista di Magia}]: Evocazione
	\item[\textbf{Livello}]: 4, Non Comune
	\item[\textbf{T. di Lancio}]: 2 Azioni
	\item[\textbf{Gittata}]: 27 metri
	\item[\textbf{Componenti}]: V, S, M (un pezzo di tentacolo di una piovra gigante o di un calamaro gigante)
	\item[\textbf{Durata}]: 1 minuto
\end{description}

Viscidi tentacoli d'ebano riempiono un quadrato di 6 metri di lato sul terreno, a gittata e che puoi vedere. Per la durata dell'incantesimo, questi tentacoli trasformano l'area in terreno difficile.

Quando una creatura entra nell'area soggetta per la prima volta in un round o comincia qui il suo round, deve superare un Tiro Salvezza su Riflessi o subire 3d6 danni contundenti e rimanere \hyperlink{intralciato}{intralciata} dai tentacoli fino al termine dell'incantesimo. Una creatura intralciata dai tentacoli può usare 2 Azioni per effettuare un nuovo Tiro Salvezza per essere libera in quel round.

\smallskip\noindent\rule{\linewidth}{2pt} \index[Incantesimi]{Terremoto}\hypertarget{Terremoto}{}\smallskip\noindent{\textbf{Terremoto}}\pdfbookmark[3]{Terremoto}{Terremoto}
\noindent
\begin{description}[noitemsep, topsep=0pt, parsep=0pt, partopsep=0pt, leftmargin=0cm, labelwidth=2.8cm]
	\item[\textbf{Lista di Magia}]: Terra
	\item[\textbf{Livello}]: 8, Molto Raro
	\item[\textbf{T. di Lancio}]: 2 Azioni
	\item[\textbf{Gittata}]: 150 metri
	\item[\textbf{Componenti}]: V, S, M (un pizzico di terriccio, un pezzo di pietra e un grumo di argilla)
	\item[\textbf{Durata}]: Concentrazione, massimo 1 minuto
\end{description}

Provochi un disturbo sismico in un punto sul terreno a gittata e che puoi vedere. Per la durata, un intenso tremore scuote il terreno in un cerchio di 30 metri di raggio centrato su quel punto e scuote le creature e le strutture in quell'area che sono a contatto del terreno.Il terreno nell'area diventa terreno difficile. Ogni creatura a terra che si sta concentrando deve effettuare un Tiro Salvezza su Tempra. Se lo fallisce, la sua concentrazione è infranta.

Quando lanci questo incantesimo e alla fine di ogni round che hai speso a concentrarti su di esso, ogni creatura nell'area che si trovi a terra deve effettuare un Tiro Salvezza su Riflessi. Se lo fallisce, la creatura cade prona.

Questo incantesimo ha effetti aggiuntivi a seconda del tipo di terreno nell'area, a discrezione del Narratore. Fenditure. All'inizio del round successivo a quello in cui hai lanciato l'incantesimo si aprono delle fenditure per tutta l'area dell'incantesimo. Un totale di 1d6 fenditure si aprono in punti scelti dal Narratore. Ognuna di esse è profonda 1d10 x 3 metri, larga 3 metri e si estende da un lato dell'area dell'incantesimo all'altro. Una creatura che si trova sul punto in cui si apre una fenditura deve superare un Tiro Salvezza su Riflessi o cadervi dentro. Una creatura che riesca il Tiro Salvezza si sposta sul bordo della fenditura, nel momento in cui questa si apre.

Una fenditura che si apre sotto una struttura la fa crollare immediatamente (vedi sotto). Strutture. Il tremore infligge 50 danni contundenti a qualsiasi struttura in contatto col terreno nell'area quando lanci l'incantesimo e alla fine di ciascuno dei tuoi round fino al termine dell'incantesimo. Se una struttura scende a 0 Punti Ferita, crolla e potrebbe danneggia le creature vicine. Una creatura distante dalla struttura metà della altezza o meno della struttura, deve effettuare un Tiro Salvezza su Riflessi. Se lo fallisce, la creatura subisce 5d6 danni contundenti, cade prona ed è sommersa dalle macerie. Dovrà poi impiegare 2 azioni riuscendo una prova di Atletica DC 20 per liberarsi. Il Narratore può modificare verso l'alto o il basso la DC, a seconda della natura delle macerie. Se supera il Tiro Salvezza, la creatura subisce solo la metà dei danni e non cade né resta sepolta.

\smallskip\noindent\rule{\linewidth}{2pt} \index[Incantesimi]{Terreno Illusorio}\hypertarget{Terreno Illusorio}{}\smallskip\noindent{\textbf{Terreno Illusorio}}\pdfbookmark[3]{Terreno Illusorio}{Terreno Illusorio}
\noindent
\begin{description}[noitemsep, topsep=0pt, parsep=0pt, partopsep=0pt, leftmargin=0cm, labelwidth=2.8cm]
	\item[\textbf{Lista di Magia}]: Illusione
	\item[\textbf{Livello}]: 4, Non Comune
	\item[\textbf{T. di Lancio}]: 10 minuti
	\item[\textbf{Gittata}]: 90 metri
	\item[\textbf{Componenti}]: V, S, M (una pietra, un rametto e un pezzo di pianta verde)
	\item[\textbf{Durata}]: 24 ore
\end{description}

Fai sì che un pezzo di terreno naturale a gittata, in un cubo di 150 metri di spigolo, appaia, risuoni e odori come qualche altro tipo di terreno naturale. Di conseguenza, campi aperti o una strada possono essere trasformati in un acquitrino, colline, un crepaccio o qualche altro tipo di terreno difficile o invalicabile. Un laghetto può essere trasformato in una radura erbosa, un precipizio in una lieve pendenza, un burrone cosparso di rocce in una strada ampia e liscia. Le strutture edificate, l'equipaggiamento e le creature all'interno dell'area non mutano d'aspetto.

Le peculiarità tattili del terreno sono immutate, così che le creature che entrano nell'area è probabile che svelino l'illusione. Se al contatto la differenza non è ovvia, una creatura che esamina con cautela l'illusione può tentare una prova di Consapevolezza contro la DC del Tiro Salvezza dei tuoi incantesimi per dubitare di essa. Una creatura che riconosca l'illusione per quello che è, la percepisce come una vaga immagine sovrapposta al terreno.

\smallskip\noindent\rule{\linewidth}{2pt} \index[Incantesimi]{Tocco Gelido}\hypertarget{Tocco Gelido}{}\smallskip\noindent{\textbf{Tocco Gelido}}\pdfbookmark[3]{Tocco Gelido}{Tocco Gelido}
\noindent
\begin{description}[noitemsep, topsep=0pt, parsep=0pt, partopsep=0pt, leftmargin=0cm, labelwidth=2.8cm]
	\item[\textbf{Lista di Magia}]: Necromanzia
	\item[\textbf{Livello}]: 0, Comune
	\item[\textbf{T. di Lancio}]: 1 Azione
	\item[\textbf{Gittata}]: 36 metri
	\item[\textbf{Componenti}]: V, S
	\item[\textbf{Durata}]: 1 round
\end{description}

Crei una scheletrica mano spettrale nello spazio di una creatura a gittata. Effettua un attacco a distanza con incantesimo contro la creatura, per aggredirla con il gelo della morte. Se colpisci, il bersaglio subisce 1d8 danni da Vuoto, e non può recuperare Punti Ferita fino all'inizio del tuo prossimo round. Fino ad allora, la mano resterà serrata sul bersaglio. Se colpisci un bersaglio non morto, esso avrà anche -1d6 ai Tiri per Colpire contro di te fino alla fine del suo prossimo round.

Puoi aumentare il danno dell'incantesimo di 1d8 quando raggiungi CM 5, CM 11 e CM 17, ma costa 2 Azioni lanciarlo potenziato e 1 Punti Magia, è altresì necessario avere preso Adepto della Magia un numero di volte pari ai potenziamenti che si vogliono applicare.

\textbf{Per ogni due Successo Critico Magico ottenuto} nella Prova di Magia crei una mano scheletrica aggiuntiva che deve attaccare una creatura diversa entro gittata.

\smallskip\noindent\rule{\linewidth}{2pt} \index[Incantesimi]{Tocco Vampirico}\hypertarget{Tocco Vampirico}{}\smallskip\noindent{\textbf{Tocco Vampirico}}\pdfbookmark[3]{Tocco Vampirico}{Tocco Vampirico}
\noindent
\begin{description}[noitemsep, topsep=0pt, parsep=0pt, partopsep=0pt, leftmargin=0cm, labelwidth=2.8cm]
	\item[\textbf{Lista di Magia}]: Necromanzia
	\item[\textbf{Livello}]: 3, Raro
	\item[\textbf{T. di Lancio}]: 2 Azioni
	\item[\textbf{Gittata}]: Personale
	\item[\textbf{Componenti}]: V, S
	\item[\textbf{Durata}]: 1 minuto
\end{description}

Il contatto con la tua mano avvolta dall'ombra può risucchiare la forza vitale altrui per curare le tue ferite. Ogni round puoi effettuare un attacco in mischia, 2 Azioni, con incantesimo contro una creatura a portata. Se colpisci, il bersaglio subisce 3d6 danni da Vuoto e tu recuperi un numero di Punti Ferita pari alla metà del danno da Vuoto che hai inflitto.

Mentre hai questo incantesimo attivo sei considerato Distratto per il lancio di altri incantesimi.

\textbf{Per ogni Successo Critico Magico} ottenuto nella Prova di Magia il danno aumento di 1d8.

\textbf{NOTA}: l'incantesimo è Comune tra i Devoti di Tazher. Un seguace di Tazher sostituisce il d6 di danno con il d8.

\smallskip\noindent\rule{\linewidth}{2pt} \index[Incantesimi]{Trama Ipnotica}\hypertarget{Trama Ipnotica}{}\smallskip\noindent{\textbf{Trama Ipnotica}}\pdfbookmark[3]{Trama Ipnotica}{Trama Ipnotica}
\noindent
\begin{description}[noitemsep, topsep=0pt, parsep=0pt, partopsep=0pt, leftmargin=0cm, labelwidth=2.8cm]
	\item[\textbf{Lista di Magia}]: Illusione
	\item[\textbf{Livello}]: 3, Comune
	\item[\textbf{T. di Lancio}]: 2 Azioni
	\item[\textbf{Gittata}]: 36 metri
	\item[\textbf{Componenti}]: S, M (un bastoncino luminoso di incenso o una fiala di cristallo piena di materiale fosforescente)
	\item[\textbf{Durata}]: 1 minuto
\end{description}

Crei a gittata una trama contorta di colori che si muove nell'aria all'interno di un cubo di 9 metri di spigolo. La trama appare per un momento e poi svanisce. Ogni creatura nell'area che veda la trama deve effettuare un Tiro Salvezza su Volontà. Se fallisce il Tiro Salvezza, una creatura rimane affascinata per la durata. Mentre è affascinata da questo incantesimo, la creatura è inabile e ha velocità 0. L'incantesimo termina per la creatura soggetta, qualora questa subisca danni o se qualcuno usa un'Azione per scuoterla dal suo stato confusionale.

\smallskip\noindent\rule{\linewidth}{2pt} \index[Incantesimi]{Trasformazione}\hypertarget{Trasformazione}{}\smallskip\noindent{\textbf{Trasformazione}}\pdfbookmark[3]{Trasformazione}{Trasformazione}
\noindent
\begin{description}[noitemsep, topsep=0pt, parsep=0pt, partopsep=0pt, leftmargin=0cm, labelwidth=2.8cm]
	\item[\textbf{Lista di Magia}]: Trasmutazione
	\item[\textbf{Livello}]: 9, Raro
	\item[\textbf{T. di Lancio}]: 2 Azioni
	\item[\textbf{Gittata}]: Personale
	\item[\textbf{Componenti}]: V, S, M (un cerchietto di giada del valore di almeno 1.500 mo, che devi poggiare sulla tua testa prima di lanciare l'incantesimo)
	\item[\textbf{Durata}]: 1 ora
\end{description}

Per la durata assumi la forma di una creatura differente. La nuova forma può essere quella di qualsiasi creatura il cui grado di sfida sia pari o inferiore alla tua CM/2+Adepto della magia. La creatura non può essere un costrutto o un non morto e devi averla vista almeno una volta. Ti trasformi in un esemplare medio di quella creatura, uno senza Abilità uniche. Puoi restare nella forma assunta fino al termine dell'incantesimo. Ti ritrasformi automaticamente se cadi privo di sensi, scendi a 0 Punti Ferita o muori. Le tue statistiche di gioco sono rimpiazzate dalle statistiche della creatura scelta, fatta accezione per i tuoi Tratti, e dei tuoi punteggi di Intelligenza, Saggezza e Carisma. Mantieni tutte le tue competenze nelle Abilità e i Tiri Salvezza, oltre a ottenere quelle della creatura. Se la creatura possiede le tue stesse competenze e il bonus indicato nelle sue statistiche è più alto del tuo, usa il bonus della creatura al posto del tuo. Non puoi usare nessuna Azione aggiuntiva o Azione da tana della nuova forma.

Quando ti trasformi mantieni i Punti Ferita. Quando ritorni alla tua forma normale, ritorni al numero di Punti Ferita che avevi prima di trasformarti. Tuttavia, se ti ritrasformi perché sei stato ridotto a 0 Punti Ferita, tutto il danno in eccesso viene riportato alla tua forma originale. A meno che il danno in eccesso non riduca la tua forma normale a 0 Punti Ferita, non cadrai privo di sensi.

Mantieni tutti i benefici di qualsiasi Abilità possedessi, razza, o altra fonte e puoi usarli se la nuova forma è fisicamente capace di farne uso. Tuttavia, non puoi usare nessuno dei tuoi sensi speciali, come la scurovisione, a meno che la nuova forma non possieda anch'essa lo stesso senso. Puoi parlare solo se la creatura è normalmente in grado di parlare.

Quando ti trasformi scegli se il tuo equipaggiamento cade a terra nel tuo spazio, si fonde con la nuova forma o sia indossato da essa. L'equipaggiamento indossato funziona come di norma, ma sta al Narratore decidere se sia comodo per la nuova forma indossare un simile pezzo di equipaggiamento, in base alla taglia e le dimensioni della creatura. Il tuo equipaggiamento non cambia dimensioni né si adatta alla nuova forma, e qualsiasi equipaggiamento che la nuova forma non può indossare deve essere fatto cadere a terra o fondersi con la nuova forma. L'equipaggiamento che si fonde è inefficace.

Nella durata dell'incantesimo, puoi usare due azioni per assumere una forma diversa seguendo le stesse restrizioni e regole della forma originale.

\textbf{NOTA}: devi essere un Devoto di Efrem o Shayalia per imparare questo incantesimo

\smallskip\noindent\rule{\linewidth}{2pt} \index[Incantesimi]{Trasformazione Furiosa di Restser}\hypertarget{Trasformazione Furiosa di Restser}{}\smallskip\noindent{\textbf{Trasformazione Furiosa di Restser}}\pdfbookmark[3]{Trasformazione Furiosa di Restser}{Trasformazione Furiosa di Restser}
\noindent
\begin{description}[noitemsep, topsep=0pt, parsep=0pt, partopsep=0pt, leftmargin=0cm, labelwidth=2.8cm]
	\item[\textbf{Lista di Magia}]: Trasmutazione
	\item[\textbf{Livello}]: 6, Molto Raro
	\item[\textbf{T. di Lancio}]: 2 Azioni
	\item[\textbf{Gittata}]: Personale
	\item[\textbf{Componenti}]: V, S, M (20cc di bevanda alcolica che viene consumata dal lancio dell'incantesimo, un arma magica)
	\item[\textbf{Durata}]: 1 round per CM
\end{description}

Questo incantesimo permette ad un incantatore di convogliare le sue energie magiche per trasformarsi in un potente combattente.

Fino alla fine della durata dell'incantesimo la Competenza Armi dell'incantatore diviene pari alla sua Competenza Magica.

In base all'arma magica che si tiene in mano al momento dell'incantesimo si diviene competente nella Lista d'Armi in cui appartiene quell'arma, se l'arma è presente in più liste sarà l'incantatore a scegliere la lista. L'incantatore acquisisce le capacità di quella Lista d'Armi come se l'avesse scelta un numero di volte pari al doppio delle volte che ha preso Adepto della Magia.

L'incantatore acquisisce 2 Punti Ferita Temporanei per punto di Competenza Magica posseduto.
Il punteggio non modificati delle caratteristiche fisiche (Forza, Destrezza e Costituzione) se inferiori a 2 diventano 2.

Per tutta la durata dell'incantesimo l'incantatore non è più in grado di lanciare incantesimi.

\smallskip\noindent\rule{\linewidth}{2pt} \index[Incantesimi]{Traslazione Arborea}\hypertarget{Traslazione Arborea}{}\smallskip\noindent{\textbf{Traslazione Arborea}}\pdfbookmark[3]{Traslazione Arborea}{Traslazione Arborea}
\noindent
\begin{description}[noitemsep, topsep=0pt, parsep=0pt, partopsep=0pt, leftmargin=0cm, labelwidth=2.8cm]
	\item[\textbf{Lista di Magia}]: Animali e Piante
	\item[\textbf{Livello}]: 5, Raro
	\item[\textbf{T. di Lancio}]: 2 Azioni
	\item[\textbf{Gittata}]: Personale
	\item[\textbf{Componenti}]: V, S
	\item[\textbf{Durata}]: massimo 1 minuto
\end{description}

Ottieni la capacità di entrare in un albero e muoverti dal suo interno all'interno di un altro albero della stessa specie entro 150 metri. Entrambi gli alberi devono essere vivi e almeno della tua stessa taglia. Ogni Azione di Movimento ti permetti di entrare ed uscire da un nuovo albero.

Apprendi istantaneamente la posizione di tutti gli altri alberi della stessa specie entro 150 metri. Quando esci riappari in un punto a tua scelta entro 1 metro dall'albero di destinazione.

Per la durata dell'incantesimo puoi usare questa capacità di trasporto una volta per round. Devi terminare ogni round al di fuori di un albero.

\smallskip\noindent\rule{\linewidth}{2pt} \index[Incantesimi]{Trasporto Vegetale}\hypertarget{Trasporto Vegetale}{}\smallskip\noindent{\textbf{Trasporto Vegetale}}\pdfbookmark[3]{Trasporto Vegetale}{Trasporto Vegetale}
\noindent
\begin{description}[noitemsep, topsep=0pt, parsep=0pt, partopsep=0pt, leftmargin=0cm, labelwidth=2.8cm]
	\item[\textbf{Lista di Magia}]: Animali e Piante
	\item[\textbf{Livello}]: 6, Molto Raro
	\item[\textbf{T. di Lancio}]: 2 Azioni
	\item[\textbf{Gittata}]: 3 metri
	\item[\textbf{Componenti}]: V, S
	\item[\textbf{Durata}]: 1 round
\end{description}

Questo incantesimo crea un legame magico tra un vegetale inanimato di taglia Grande o maggiore a gittata e un altro vegetale, a qualsiasi distanza, sullo stesso piano di esistenza. Devi aver visto o essere entrato in contatto almeno una volta con il vegetale di destinazione. Per la durata dell'incantesimo, qualsiasi creatura può entrare nel vegetale bersaglio e uscire dal vegetale di destinazione usando 1 Azione di Movimento.

\smallskip\noindent\rule{\linewidth}{2pt} \index[Incantesimi]{Trucco della Corda}\hypertarget{Trucco della Corda}{}\smallskip\noindent{\textbf{Trucco della Corda}}\pdfbookmark[3]{Trucco della Corda}{Trucco della Corda}
\noindent
\begin{description}[noitemsep, topsep=0pt, parsep=0pt, partopsep=0pt, leftmargin=0cm, labelwidth=2.8cm]
	\item[\textbf{Lista di Magia}]: Trasmutazione
	\item[\textbf{Livello}]: 2, Comune
	\item[\textbf{T. di Lancio}]: 1 minuto
	\item[\textbf{Gittata}]: Contatto
	\item[\textbf{Componenti}]: V, S, M ( una corda di 18 metri)
	\item[\textbf{Durata}]: 1 ora +1 Turno per CM
\end{description}

Entri a contatto con un pezzo di corda lungo fino a 18 metri. Un'estremità della corda si leva nell'aria finché la corda non pende perpendicolare al terreno. All'estremità opposta della corda, un'entrata invisibile si apre su di uno spazio extradimensionale che resta fino al termine dell'incantesimo

Lo spazio extradimensionale può essere raggiunto arrampicandosi fino alla cima della corda (prova di Arrampicarsi DC 15). Lo spazio può contenere fino a 2 creature di taglia Media o inferiore +1 per ogni volta che si è preso Adepto della Magia. La corda può essere trascinata nello spazio, facendola sparire dalla vista di chi è fuori di esso.

Chi si trova al suo interno o sotto l'ingresso può vedere fuori come se vedesse attraverso una finestra di 1 x 1 metro centrata sulla corda. L'incantesimo di Individuazione del Magico permette di vedere l'apertura. Qualsiasi cosa si trovi nello spazio extradimensionale ne cade fuori al termine dell'incantesimo.

Al termine dell'incantesimo la corda usata scompare.

\textbf{Per ogni Successo Critico Magico} ottenuto nella Prova di Magia la durata aumenta di due ore o può contenere un'altra creatura media o più piccola.

\smallskip\noindent\rule{\linewidth}{2pt} \index[Incantesimi]{Uno con la pietra}\hypertarget{Uno con la pietra}{}\smallskip\noindent{\textbf{Uno con la pietra}}\pdfbookmark[3]{Uno con la pietra}{Uno con la pietra}
\noindent
\begin{description}[noitemsep, topsep=0pt, parsep=0pt, partopsep=0pt, leftmargin=0cm, labelwidth=2.8cm]
	\item[\textbf{Lista di Magia}]: Terra
	\item[\textbf{Livello}]: 3, Comune
	\item[\textbf{T. di Lancio}]: 2 Azioni
	\item[\textbf{Gittata}]: Contatto
	\item[\textbf{Componenti}]: V, S
	\item[\textbf{Durata}]: 8 ore
\end{description}

Entri in un oggetto o superficie di pietra grossa abbastanza da contenere tutto il tuo corpo fondendoti con la pietra assieme a tutto l'equipaggiamento che trasporti per la durata.

Usando il tuo movimento, entri nella pietra in un punto con cui sei in contatto. Non resta nulla della tua presenza che rimanga visibile o altrimenti possa essere individuato da sensi non magici. Mentre sei fuso con la pietra, non puoi vedere ciò che avviene all'esterno, e qualsiasi prova di Consapevolezza che effettui per ascoltare i suoni prodotti fuori da essa è fatta con -1d6. Resti consapevole del passare del tempo e puoi lanciare incantesimi su di te mentre sei fuso con la pietra. Puoi usare il tuo movimento per lasciare la pietra e ricomparire nel punto in cui vi sei entrato, terminando così l'incantesimo. Altrimenti non puoi muoverti.

I danni minori alla pietra non ti danneggiano, ma la sua parziale distruzione o cambio di forma (di modo che tu non vi entri più) ti espellono da essa e ti infliggono 6d6 danni contundenti. La completa distruzione della pietra (o la sua trasmutazione in un'altra sostanza) ti fa espellere e ti infligge 50 danni contundenti. Se vieni espulso, cadi prono in uno spazio non occupato, nel punto più vicino a quello in cui sei entrato nella pietra.

\textbf{Per ogni Successo Critico Magico} ottenuto nella Prova di Magia la durata massima aumenta di 1 ora.

\smallskip\noindent\rule{\linewidth}{2pt} \index[Incantesimi]{Unto}\hypertarget{Unto}{}\smallskip\noindent{\textbf{Unto}}\pdfbookmark[3]{Unto}{Unto}
\noindent
\begin{description}[noitemsep, topsep=0pt, parsep=0pt, partopsep=0pt, leftmargin=0cm, labelwidth=2.8cm]
	\item[\textbf{Lista di Magia}]: Animali e Piante
	\item[\textbf{Livello}]: 1, Comune
	\item[\textbf{T. di Lancio}]: 2 Azioni
	\item[\textbf{Gittata}]: 18 metri
	\item[\textbf{Componenti}]: V, S, M (un pezzo di cotenna di maiale o burro od unto topetto)
	\item[\textbf{Durata}]: 1 minuto
\end{description}

Del grasso scivoloso ricopre il terreno in un quadrato di 3 metri di lato, centrato su di un punto a gittata, e lo trasforma in terreno difficile per la durata dell'incantesimo.

Quando compare il grasso, ciascun bersaglio che si trova in piedi nell'area deve superare un Tiro Salvezza su Riflessi o cadere prono. Una creatura che entra nell'area o termina il suo round lì, deve superare un Tiro Salvezza su Riflessi o cadere prona.

\smallskip\noindent\rule{\linewidth}{2pt} \index[Incantesimi]{Vedere l'invisibile}\hypertarget{Vedere l'invisibile}{}\smallskip\noindent{\textbf{Vedere l'invisibile}}\pdfbookmark[3]{Vedere l'invisibile}{Vedere l'invisibile}
\noindent
\begin{description}[noitemsep, topsep=0pt, parsep=0pt, partopsep=0pt, leftmargin=0cm, labelwidth=2.8cm]
	\item[\textbf{Lista di Magia}]: Divinazione
	\item[\textbf{Livello}]: 2, Comune
	\item[\textbf{T. di Lancio}]: 2 Azioni
	\item[\textbf{Gittata}]: Personale
	\item[\textbf{Componenti}]: V, S, M (un pizzico di talco e una manciata di polvere d'argento)
	\item[\textbf{Durata}]: 1 ora
\end{description}

Per la durata dell'incantesimo, vedi le creature e gli oggetti invisibili come se fossero visibili, e inoltre puoi vedere nel Piano Etereo. Le creature e gli oggetti eterei ti appaiono spettrali e trasparenti.

\smallskip\noindent\rule{\linewidth}{2pt} \index[Incantesimi]{Velocità}\hypertarget{Velocità}{}\smallskip\noindent{\textbf{Velocità}}\pdfbookmark[3]{Velocita'}{Velocita'}
\noindent
\begin{description}[noitemsep, topsep=0pt, parsep=0pt, partopsep=0pt, leftmargin=0cm, labelwidth=2.8cm]
	\item[\textbf{Lista di Magia}]: Trasmutazione
	\item[\textbf{Livello}]: 3, Non Comune
	\item[\textbf{T. di Lancio}]: 2 Azioni
	\item[\textbf{Gittata}]: 9 metri
	\item[\textbf{Componenti}]: V, S, M (una grattata di radice di liquirizia)
	\item[\textbf{Durata}]: 1 minuto
\end{description}

Acceleri il metabolismo di massimo 2 più le volte che hai preso Adepto della Magia creature a tua scelta in un raggio di 3 metri a gittata. Fino al termine dell'incantesimo i bersagli possono eseguire una Azione aggiuntiva di Attacco, senza penalità di multiattacco, o di Movimento. L'Azione aggiuntiva può essere parte di un altra Azione.

Questo incantesimo \hyperlink{contrastareincantesimi}{contrasta ed è contrastato} da \hyperlink{lentezza}{Lentezza}.

Quando l'incantesimo termina, i bersagli sono Rallentati 2/1r mentre sono preda di un'improvvisa sonnolenza.

\textbf{Per ogni Successo Critico Magico} ottenuto nella Prova di Magia puoi influenzare una creatura in più.

\textbf{Per ogni tre Successo Critico Magico ottenuto} nella Prova di Magia puoi aumentare di un ulteriore 1 le Azioni a round ad una creatura.

\smallskip\noindent\rule{\linewidth}{2pt} \index[Incantesimi]{Ventriloquio}\hypertarget{Ventriloquio}{}\smallskip\noindent{\textbf{Ventriloquio}}\pdfbookmark[3]{Ventriloquio}{Ventriloquio}
\noindent
\begin{description}[noitemsep, topsep=0pt, parsep=0pt, partopsep=0pt, leftmargin=0cm, labelwidth=2.8cm]
	\item[\textbf{Lista di Magia}]: Illusione
	\item[\textbf{Livello}]: 1, Comune
	\item[\textbf{T. di Lancio}]: 1 Azione
	\item[\textbf{Gittata}]: 9 metri
	\item[\textbf{Componenti}]: V, M (un pezzo di magnete)
	\item[\textbf{Durata}]: 1 minuto
\end{description}

Puoi far sembrare che la tua voce (o qualsiasi suono che puoi normalmente produrre vocalmente) provenga da un altro luogo. Puoi parlare in qualsiasi lingua tu conosca. Chiunque senta il suono può tentare una prova di Consapevolezza contro la tua DC dell'incantesimo. In caso di successo, riconoscono che è illusorio, ma lo sentono comunque. Puoi interromperlo a volontà per la durata, senza azioni richieste.

\textbf{Per ogni Successo Critico Magico} ottenuto nella Prova di Magia puoi allontanare l'origine della voce di altri 9 metri.

\smallskip\noindent\rule{\linewidth}{2pt} \index[Incantesimi]{Vigilanza e Interdizione}\hypertarget{Vigilanza e Interdizione}{}\smallskip\noindent{\textbf{Vigilanza e Interdizione}}\pdfbookmark[3]{Vigilanza e Interdizione}{Vigilanza e Interdizione}
\noindent
\begin{description}[noitemsep, topsep=0pt, parsep=0pt, partopsep=0pt, leftmargin=0cm, labelwidth=2.8cm]
	\item[\textbf{Lista di Magia}]: Abiurazione
	\item[\textbf{Livello}]: 6, Non Comune
	\item[\textbf{T. di Lancio}]: 10 minuti
	\item[\textbf{Gittata}]: Contatto
	\item[\textbf{Componenti}]: V, S, M (incenso bruciato, un piccolo misurino di zolfo e olio, un laccio legato, un piccolo ammontare di sangue di colosso di terra, e una piccola verga d'argento del valore di almeno 10 mo)
	\item[\textbf{Durata}]: 24 ore
\end{description}

Crei una interdizione che protegge fino a 225 metri quadri di pavimento (un'area quadrata di 15 metri di lato, o cento quadrati di 1 metro di lato o venticinque quadrati di 3 metri di lato). L'area interdetta può essere alta fino a 6 metri, e modellata come preferisci. Puoi interdire diversi piani di una roccaforte dividendo l'area tra di essi, purché tu possa camminare ininterrottamente in ogni area adiacente, mentre lanci l'incantesimo.

Quando lanci questo incantesimo, puoi specificare gli individui che ignorano qualcuno o tutti gli effetti di questo incantesimo. Puoi anche specificare una parola d'ordine che, pronunciata ad alta voce, rende chi la proferisce immune a questi effetti.

Vigilanza e interdizione crea i seguenti effetti all'interno dell'area interdetta.

\begin{itemize}[leftmargin=*] \setlength{\itemsep}{0pt}

	\item \emph{Corridoi}. La nebbia riempie tutti i corridoi interdetti, rendendoli oscurati pesantemente. Inoltre, a ogni intersezione o biforcazione del passaggio che offre una scelta di direzione, c'è una probabilità del 50\% che una creatura, escluso te, creda di stare andando nella direzione opposta a quella che ha scelto.
	\item \emph{Porte}. Tutte le porte nell'area interdetta sono chiuse magicamente, come se fossero sigillate dall'incantesimo Serratura Magica. Inoltre, puoi coprire fino a dieci porte con un'illusione (equivalente della funzione oggetto illusorio dell'incantesimo illusione minore) per farle sembrare delle semplici sezioni di muro.
	\item \emph{Scale}. Ragnatele ricoprono da cima a fondo tutte le scale nell'area interdetta, come per l'incantesimo ragnatela. Questi fili ricrescono in 10 minuti se vengono bruciati o strappati mentre vigilanza e interdizione resta attivo.
\end{itemize}

Altri Incantesimi in Effetto. Puoi piazzare uno dei seguenti effetti magici di tua scelta all'interno dell'area interdetta dell'edificio

\begin{itemize}[leftmargin=*] \setlength{\itemsep}{0pt}
	\item Piazza luci danzanti in quattro corridoi. Puoi indicare un semplice programma che le luci ripeteranno per la durata di vigilanza e interdizione.
	\item Piazza bocca magica in due posti.
	\item Piazza Nebbia Nauseante in due posti. I vapori appaiono nel posto da te indicato; ritornano entro 10 minuti se dispersi dal vento mentre vigilanza e interdizione è ancora attivo.
	\item Piazza una folata di vento costante in un corridoio o stanza.
	\item Piazza una suggestione in un luogo. Seleziona un'area quadrata di 1 metro di lato, e qualsiasi creatura che entra o passa attraverso quell'area riceve mentalmente la suggestione.
\end{itemize}

L'intera area interdetta irradia magia. Un incantesimo dissolvi magie lanciato contro uno specifico effetto, se riesce, rimuove solo quell'effetto Puoi creare una struttura perennemente vigilata e interdetta lanciandovi questo incantesimo ogni giorno per un anno.

\textbf{Se effettui tre critici} la durata è permanente.

\smallskip\noindent\rule{\linewidth}{2pt} \index[Incantesimi]{Vigore}\hypertarget{Vigore}{}\smallskip\noindent{\textbf{Vigore}}\pdfbookmark[3]{Vigore}{Vigore}
\noindent
\begin{description}[noitemsep, topsep=0pt, parsep=0pt, partopsep=0pt, leftmargin=0cm, labelwidth=2.8cm]
	\item[\textbf{Lista di Magia}]: Cura
	\item[\textbf{Livello}]: 4, Raro
	\item[\textbf{T. di Lancio}]: 2 Azioni
	\item[\textbf{Gittata}]: Contatto
	\item[\textbf{Componenti}]: V, S, M (acqua, sale, zucchero)
	\item[\textbf{Durata}]: 1 round per CM
\end{description}

La creatura influenzata da questo incantesimo recupera un livello di Affaticamento, acquisisce 3d6 Punti Ferita Temporanei. Può concentrare le energie per effettuare una Azione di Attacco senza penalità di multiattacco o eseguire una Azione di Movimento in più.

\smallskip\noindent\rule{\linewidth}{2pt} \index[Incantesimi]{Vincolo di Interdizione}\hypertarget{Vincolo di Interdizione}{}\smallskip\noindent{\textbf{Vincolo di Interdizione}}\pdfbookmark[3]{Vincolo di Interdizione}{Vincolo di Interdizione}
\noindent
\begin{description}[noitemsep, topsep=0pt, parsep=0pt, partopsep=0pt, leftmargin=0cm, labelwidth=2.8cm]
	\item[\textbf{Lista di Magia}]: Abiurazione
	\item[\textbf{Livello}]: 2, Comune
	\item[\textbf{T. di Lancio}]: 2 Azioni
	\item[\textbf{Gittata}]: Contatto
	\item[\textbf{Componenti}]: V, S, M (una coppia di anelli di platino del valore di 50 mo l'uno, che tu e il bersaglio dovete indossare per la durata)
	\item[\textbf{Durata}]: 1 ora
\end{description}

Lanci l'incantesimo a contatto di una creatura che vuoi proteggere. Crei una connessione mistica tra di te e il bersaglio fino al termine dell'incantesimo. Finché il bersaglio è entro 18 metri da te, ottiene un bonus di +1 alla Difesa e ai Tiri Salvezza e ha resistenza a tutti i danni. Inoltre, ogni volta che il bersaglio subisce danni, tu ne subisci la stessa quantità. L'incantesimo ha fine se scendi a 0 Punti Ferita o tu e il bersaglio vi allontanate più di 18 metri. Ha fine anche se lo lanci di nuovo sulla stessa creatura su cui è già in atto. Puoi interrompere l'incantesimo con un'Azione.

\smallskip\noindent\rule{\linewidth}{2pt} \index[Incantesimi]{Visione del Vero}\hypertarget{Visione del Vero}{}\smallskip\noindent{\textbf{Visione del Vero}}\pdfbookmark[3]{Visione del Vero}{Visione del Vero}
\noindent
\begin{description}[noitemsep, topsep=0pt, parsep=0pt, partopsep=0pt, leftmargin=0cm, labelwidth=2.8cm]
	\item[\textbf{Lista di Magia}]: Divinazione
	\item[\textbf{Livello}]: 6, Raro
	\item[\textbf{T. di Lancio}]: 2 Azioni
	\item[\textbf{Gittata}]: Contatto
	\item[\textbf{Componenti}]: V, S, M (un unguento per gli occhi che costa 25 mo; fatto di funghi in polvere, zafferano e grasso; viene consumato dall'incantesimo)
	\item[\textbf{Durata}]: 1 ora
\end{description}

Lanci l'incantesimo a contatto di una creatura consenziente. Il bersaglio riceve la capacità di vedere le cose come sono realmente. Per la durata dell'incantesimo, la creatura ha visione del vero, nota porte segrete nascoste dalla magia e può vedere nel Piano Etereo, fino a una gittata di 36 metri. Vedi anche \hyperlink{cap Visione del Vero}{Visione del Vero} pag. \pageref{Visione del Vero}.

\smallskip\noindent\rule{\linewidth}{2pt} \index[Incantesimi]{Vita Falsata}\hypertarget{Vita Falsata}{}\smallskip\noindent{\textbf{Vita Falsata}}\pdfbookmark[3]{Vita Falsata}{Vita Falsata}
\noindent
\begin{description}[noitemsep, topsep=0pt, parsep=0pt, partopsep=0pt, leftmargin=0cm, labelwidth=2.8cm]
	\item[\textbf{Lista di Magia}]: Necromanzia
	\item[\textbf{Livello}]: 1, Comune
	\item[\textbf{T. di Lancio}]: 2 Azioni
	\item[\textbf{Gittata}]: Personale
	\item[\textbf{Componenti}]: V, S, M (un piccolo ammontare di alcool o spirito distillato)
	\item[\textbf{Durata}]: 1 ora
\end{description}

Potenziandoti con una parvenza necromantica di vitalità, ottieni 1d4 + 4 Punti Ferita temporanei per la durata.

\textbf{Per ogni Successo Critico Magico} ottenuto nella Prova di Magia ottieni 5 Punti Ferita temporanei.

\smallskip\noindent\rule{\linewidth}{2pt} \index[Incantesimi]{Viticci Perforanti}\hypertarget{Viticci Perforanti}{}\smallskip\noindent{\textbf{Viticci Perforanti}}\pdfbookmark[3]{Viticci Perforanti}{Viticci Perforanti}
\noindent
\begin{description}[noitemsep, topsep=0pt, parsep=0pt, partopsep=0pt, leftmargin=0cm, labelwidth=2.8cm]
	\item[\textbf{Lista di Magia}]: Ammali e Piante
	\item[\textbf{Livello}]: 0, Non Comune
	\item[\textbf{T. di Lancio}]: 1 Azione
	\item[\textbf{Gittata}]: 9 metri
	\item[\textbf{Componenti}]: V, S
	\item[\textbf{Durata}]: Istantanea
\end{description}

Scateni dal palmo della tua mano 1 viticcio puntuto e spinato. Esegui un Tiro per Colpire a distanza con incantesimi sul bersaglio designato.
Se il Tiro per Colpire ha successo il bersaglio subisce 1d4 Punti Ferita da danno da penetrazione.

Ogni Azione che dedichi in più al lancio dell'incantesimo puoi decidere di manifestare un viticcio aggiuntivo che deve attaccare un bersaglio diverso dagli altri oppure aumentare la portata di 9 metri di un viticcio creato.

Se spendi 1 Punto Magia nel lancio dell'incantesimo il viticcio diventa velenoso e se colpisce causa 2 danni da Veleno aggiuntivi.

\textbf{Per ogni Successo Critico Magico} ottenuto nella Prova di Magia l'incantesimo crea un viticcio aggiuntivo.

\smallskip\noindent\rule{\linewidth}{2pt} \index[Incantesimi]{Volare}\hypertarget{Volare}{}\smallskip\noindent{\textbf{Volare}}\pdfbookmark[3]{Volare}{Volare}
\noindent
\begin{description}[noitemsep, topsep=0pt, parsep=0pt, partopsep=0pt, leftmargin=0cm, labelwidth=2.8cm]
	\item[\textbf{Lista di Magia}]: Aria
	\item[\textbf{Livello}]: 3, Comune
	\item[\textbf{T. di Lancio}]: 2 Azioni
	\item[\textbf{Gittata}]: Contatto
	\item[\textbf{Componenti}]: V, S, M (una piuma dell'ala di qualsiasi volatile)
	\item[\textbf{Durata}]: 10 minuti
\end{description}

Lanci l'incantesimo a contatto di una creatura consenziente. Per la durata dell'incantesimo, il bersaglio ottiene velocità di volo 18 metri. Quando l'incantesimo ha fine, qualora sia ancora in aria, il bersaglio cade, a meno che non riesca a frenare la discesa.

Lanciare un incantesimo mentre si vola è più complesso, si è Distratti se non si riesce in una prova di Volare a DC 11.

\textbf{Per ogni Successo Critico Magico} ottenuto nella Prova di Magia puoi prendere come bersaglio un'ulteriore creatura oppure aumentare la durata di 10 minuti.

\smallskip\noindent\rule{\linewidth}{2pt} \index[Incantesimi]{Scudo Mentale}\hypertarget{Scudo Mentale}{}\smallskip\noindent{\textbf{Scudo Mentale}}\pdfbookmark[3]{Scudo Mentale}{Scudo Mentale}
\noindent
\begin{description}[noitemsep, topsep=0pt, parsep=0pt, partopsep=0pt, leftmargin=0cm, labelwidth=2.8cm]
	\item[\textbf{Lista di Magia}]: Abiurazione
	\item[\textbf{Livello}]: 8, Non Comune
	\item[\textbf{T. di Lancio}]: 2 Azioni
	\item[\textbf{Gittata}]: Contatto
	\item[\textbf{Componenti}]: V, S
	\item[\textbf{Durata}]: 24 ore
\end{description}

Fino al termine dell'incantesimo, una creatura consenziente con cui sei in contatto durante il lancio è immune a qualsiasi effetto che ne percepirebbe le emozioni o leggerebbe i pensieri, incantesimi di divinazione e la condizione Affascinato. l'incantesimo nega anche gli incantesimi desiderio e altri incantesimi o effetti di simili potenza impiegati per
influenzare la mente del bersaglio o per ottenere informazioni su di esso.

\textbf{Per ogni Successo Critico Magico} ottenuto nella Prova di Magia la durata raddoppia. Se ottieni tre critici la durata è permanente.

\smallskip\noindent\rule{\linewidth}{2pt} \index[Incantesimi]{Zona di Verità}\hypertarget{Zona di Verità}{}\smallskip\noindent{\textbf{Zona di Verità}}\pdfbookmark[3]{Zona di Verita'}{Zona di Verita'}
\noindent
\begin{description}[noitemsep, topsep=0pt, parsep=0pt, partopsep=0pt, leftmargin=0cm, labelwidth=2.8cm]
	\item[\textbf{Lista di Magia}]: Ammaliamento
	\item[\textbf{Livello}]: 2, Non Comune
	\item[\textbf{T. di Lancio}]: 2 Azioni
	\item[\textbf{Gittata}]: 18 metri
	\item[\textbf{Componenti}]: V, S
	\item[\textbf{Durata}]: 10 minuti
\end{description}

Crei una zona magica che protegge contro i raggiri in una sfera di 3 metri di raggio centrata su di un punto a gittata di tua scelta. Fino al termine dell'incantesimo, una creatura che entra nell'area dell'incantesimo per la prima volta durante un round, o inizia il suo round al suo interno, deve effettuare un Tiro Salvezza su Volontà. Se fallisce il Tiro Salvezza, la creatura non può pronunciare bugie deliberatamente mentre è nel raggio dell'incantesimo. Sei a conoscenza se una creatura ha superato o fallito il Tiro Salvezza. Una creatura soggetta all'incantesimo ne è consapevole e può quindi evitare di rispondere a domande a cui risponderebbe normalmente con una bugia. Questa creatura può dare risposte elusive purché rimanga entro i confini della verità.

%\vspace{2cm}
%\begin{center}
%\includegraphics[width=0.4\linewidth]{immagini/Bocca_della_Verita.png}
%\medskip
%\emph{La Bocca della Verita', Chiesa di Santa Maria in Cosmedin, Roma}
%\end{center}

\bigskip

%\begin{changemargin}{0.3cm}{0.3cm}\begin{narratore}
%Gli incantesimi elencati sono in parte della 5ed più altre mie proposte e rivisitazioni. Se avete suggerimenti per il Narratore per gestire critici non previsti parlatene con lui! Lo spirito di collaborazione deve essere sempre costruttivo.
%\end{narratore}\end{changemargin}

\end{multicols}

%\vfill

%\begin{center}
%\includegraphics[keepaspectratio,width=0.35\linewidth]{immagini/Goetic_circle_from_The_Lesser_Key_of_Solomon.png}

%\medskip

%\emph{The Circle of Solomon and Triangle of Solomon from The Goetia. L. W. De Laurence}
%\end{center}

\pagebreak

\subsection{Incantesimi per Lista con Livello e Rarita'}\hypertarget{elencoscuole}{}\index{Incantesimi elenco}

A fianco di ogni incantesimo è indicata la Rarità ed il livello dell'incantesimo.

\begin{multicols}{3}

\small{

\setlength{\parindent}{0cm}{

\textbf{Lista dell'Acqua}

\hyperlink{Raggio di Gelo}{Raggio di Gelo}, Comune, 0\\
\hyperlink{Arma Energetica}{Arma Energetica}, Molto Raro, 1\\
\hyperlink{Creare o Distruggere Acqua}{Creare o Distruggere Acqua}, Comune, 1\\
\hyperlink{Cura Ferite}{Cura Ferite}, Comune, 1\\
\hyperlink{Nube di Nebbia}{Nube di Nebbia}, Comune, 1\\
\hyperlink{Freccia Acida di Restser}{Freccia Acida di Restser}, Comune, 2\\
\hyperlink{Camminare sull'Acqua}{Camminare sull'Acqua}, Comune, 3\\
\hyperlink{Nebbia Nauseante}{Nebbia Nauseante}, Non Comune, 3\\
\hyperlink{Respirare Sott'Acqua}{Respirare Sott'Acqua}, Comune, 3\\
\hyperlink{Rimuovi Veleno}{Rimuovi Veleno}, Comune, 3\\
\hyperlink{Tempesta di Nevischio}{Tempesta di Nevischio}, Molto Raro, 3\\
\hyperlink{Controllare Acqua}{Controllare Acqua}, Comune, (4)\\
\hyperlink{Evoca Elementali Minori}{Evoca Elementali Minori}, Non Comune, 4\\
\hyperlink{Scudo di Fuoco}{Scudo di Fuoco}, Non Comune, 4\\
\hyperlink{Tempesta di Ghiaccio}{Tempesta di Ghiaccio}, Non Comune, 4\\
\hyperlink{Cono di Freddo}{Cono di Freddo}, Comune, 5\\
\hyperlink{Evoca Elementale}{Evoca Elementale}, Raro, 5\\
\hyperlink{Nebbia mortale}{Nebbia mortale}, Raro, 5\\
\hyperlink{Muro di Ghiaccio}{Muro di Ghiaccio}, Comune, (6)\\
\hyperlink{Sfera Congelante}{Sfera Congelante}, Raro, 6\\
\hyperlink{Controllare Tempo Atmosferico}{Controllare Tempo Atmosferico}, Raro, 8

\medskip\textbf{Lista dell'Aria}

\hyperlink{Stretta Folgorante}{Stretta Folgorante}, Comune, 0\\
\hyperlink{Arma Energetica}{Arma Energetica}, Molto Raro, 1\\
\hyperlink{Caduta Piuma}{Caduta Piuma}, Comune, 1\\
\hyperlink{Nube di Nebbia}{Nube di Nebbia}, Comune, 1\\
\hyperlink{Onda Tonante}{Onda Tonante}, Comune, 1\\
\hyperlink{Saltare}{Saltare}, Comune, 1\\
\hyperlink{Folata di Vento}{Folata di Vento}, Comune, 2\\
\hyperlink{Levitazione}{Levitazione}, Comune, 2\\
\hyperlink{Polvere luccicante}{Polvere luccicante}, Non Comune, 2\\
\hyperlink{Fulmine}{Fulmine}, Comune, 3\\
\hyperlink{Invocare il Fulmine}{Invocare il Fulmine}, Comune, 3\\
\hyperlink{Muro di Vento}{Muro di Vento}, Non Comune, 3\\
\hyperlink{Nebbia Nauseante}{Nebbia Nauseante}, Non Comune, 3\\
\hyperlink{Respirare Sott'Acqua}{Respirare Sott'Acqua}, Comune, 3\\
\hyperlink{Volare}{Volare}, Comune, 3\\
\hyperlink{Bolla vitale}{Bolla vitale}, Non Comune, 4\\
\hyperlink{Camminare nell'aria}{Camminare nell'aria}, Non Comune, 4\\
\hyperlink{Evoca Elementali Minori}{Evoca Elementali Minori}, Non Comune, 4\\
\hyperlink{Tempesta di Ghiaccio}{Tempesta di Ghiaccio}, Non Comune, 4\\
\hyperlink{Evoca Elementale}{Evoca Elementale}, Raro, 5\\
\hyperlink{Camminare nel Vento}{Camminare nel Vento}, Non Comune, 6\\
\hyperlink{Fulmine a catena}{Fulmine a catena}, Raro, 6\\
\hyperlink{Controllare Tempo Atmosferico}{Controllare Tempo Atmosferico}, Molto Raro, 8\\
\hyperlink{Muro Prismatico}{Muro Prismatico}, Raro, (9)

\medskip\textbf{Lista del Fuoco}

\hyperlink{Produrre Fiamma}{Produrre Fiamma}, Comune, 0\\
\hyperlink{Arma Energetica}{Arma Energetica}, Molto Raro, 1\\
\hyperlink{Dardo di Fuoco}{Dardo di Fuoco}, Comune, 1\\
\hyperlink{Onda rovente}{Onda rovente}, Comune, 1\\
\hyperlink{Lama Infuocata}{Lama Infuocata}, Comune, 2\\
\hyperlink{Lanciafiamme}{Lanciafiamme}, Raro, 2\\
\hyperlink{Piroesperto}{Piroesperto}, Non Comune, 2\\
\hyperlink{Polvere luccicante}{Polvere luccicante}, Non Comune, 2\\
\hyperlink{Raggio Rovente}{Raggio Rovente}, Comune, 2\\
\hyperlink{Riscaldare il Metallo}{Riscaldare il Metallo}, Non Comune, 2\\
\hyperlink{Sfera Infuocata}{Sfera Infuocata}, Comune, 2\\
\hyperlink{Gragnola di Ghiande Infuocate di Kyrin}{Gragnola di Ghiande Infuocate di Kyrin}, Raro, 3\\
\hyperlink{Benedizione di Cattalm}{Benedizione di Cattalm}, Molto Raro, 3\\
\hyperlink{Palla di Fuoco}{Palla di Fuoco}, Comune, 3\\
\hyperlink{Evoca Elementali Minori}{Evoca Elementali Minori}, Non Comune, 4\\
\hyperlink{Muro di Fuoco}{Muro di Fuoco}, Non Comune, 4\\
\hyperlink{Scudo di Fuoco}{Scudo di Fuoco}, Non Comune, 4\\
\hyperlink{Colpo Infuocato}{Colpo Infuocato}, Comune, 5\\
\hyperlink{Evoca Elementale}{Evoca Elementale}, Raro, 5\\
\hyperlink{Palla di Fuoco Ritardata}{Palla di Fuoco Ritardata}, Raro, 7\\
\hyperlink{Tempesta di Fuoco}{Tempesta di Fuoco}, Raro, 7\\
\hyperlink{Nube Incendiaria}{Nube Incendiaria}, Raro, 8\\
\hyperlink{Pioggia di Meteore}{Pioggia di Meteore}, Leggendario, 9

\medskip\textbf{Lista della Terra}

\hyperlink{Riparare}{Riparare}, Comune, 0\\
\hyperlink{Arma Energetica}{Arma Energetica}, Molto Raro, 1\\
\hyperlink{Palla di fango di Eithne}{Palla di fango di Eithne}, Non Comune, 1\\
\hyperlink{Lettura della terra di Kyrin}{Lettura della terra di Kyrin}, Non Comune, 2\\
\hyperlink{Passare Senza Tracce}{Passare Senza Tracce}, Comune, 2\\
\hyperlink{Uno con la pietra}{Uno con la pietra}, Comune, 3\\
\hyperlink{Freccia Acida di Restser}{Freccia Acida di Restser}, Comune, 2\\
\hyperlink{Gragnola di Limoni di Kyrin}{Gragnola di Limoni di Kyrin}, Molto Raro, 2\\
\hyperlink{Succo concentrato di Ribes di Kyrin}{Succo concentrato di Ribes di Kyrin}, Non Comune, 2\\
\hyperlink{Evoca Elementali Minori}{Evoca Elementali Minori}, Non Comune, 4\\
\hyperlink{Pelle di Pietra}{Pelle di Pietra}, Non Comune, 4\\
\hyperlink{Scolpire Pietra}{Scolpire Pietra}, Comune, 4\\
\hyperlink{Evoca Elementale}{Evoca Elementale}, Raro, 5\\
\hyperlink{Muro di Pietra}{Muro di Pietra}, Comune, 5\\
\hyperlink{Passa Porta}{Passa Porta}, Non Comune, 5\\
\hyperlink{Pietra in Fango - Fango in Pietra}{Pietra in Fango - Fango in Pietra}, Non Comune - Molto Raro, 5\\
\hyperlink{Carne in Pietra - Pietra in Carne}{Carne in Pietra - Pietra in Carne}, Non Comune - Raro, 6\\
\hyperlink{Muovere il Terreno}{Muovere il Terreno}, Non Comune, 6\\
\hyperlink{Pietre Parlanti}{Pietre Parlanti}, Raro, 6\\
\hyperlink{Statua}{Statua}, Raro, 7\\
\hyperlink{Terremoto}{Terremoto}, Molto Raro, 8\\
\hyperlink{Pioggia di Meteore}{Pioggia di Meteore}, Leggendario, 9

\medskip\textbf{Abiurazione}

\hyperlink{Resistenza}{Resistenza}, Comune, 0\\
\hyperlink{Allarme}{Allarme}, Comune, 1\\
\hyperlink{Armatura Magica}{Armatura Magica}, Non Comune, 1\\
\hyperlink{Protezione dall'Energia minore}{Protezione dall'Energia minore}, Raro, 1\\
\hyperlink{Santuario}{Santuario}, Comune, 1\\
\hyperlink{Rimuovi Paura}{Rimuovi Paura}, Comune, 1\\
\hyperlink{Chiudi Portale}{Chiudi Portale}, Raro, 2\\
\hyperlink{Protezione dai Veleni}{Protezione dai Veleni}, Non Comune, 2\\
\hyperlink{Serratura Magica}{Serratura Magica}, Comune, 2\\
\hyperlink{Vincolo di Interdizione}{Vincolo di Interdizione}, Comune, 2\\
\hyperlink{Anti-Individuazione}{Anti-Individuazione}, Non Comune, 3\\
\hyperlink{Cerchio Magico}{Cerchio Magico}, Comune, 3\\
\hyperlink{Controincantesimo}{Controincantesimo}, Comune, 3\\
\hyperlink{Dissolvi Magie}{Dissolvi Magie}, Comune, 3\\
\hyperlink{Glifo di Interdizione}{Glifo di Interdizione}, Comune, 3\\
\hyperlink{Protezione dall'Energia}{Protezione dall'Energia}, Comune, 3\\
\hyperlink{Rimuovi Maledizione}{Rimuovi Maledizione}, Comune, 3\\
\hyperlink{Bolla vitale}{Bolla vitale}, Non Comune, 4\\
\hyperlink{Esilio}{Esilio}, Comune, 4\\
\hyperlink{Libertà di Movimento}{Libertà di Movimento}, Comune, 4\\
\hyperlink{Santuario Privato}{Santuario Privato}, Molto Raro, 4\\
\hyperlink{Dissolvi il Bene e il Male}{Dissolvi il Bene e il Male}, Raro, 5\\
\hyperlink{Dissolvi Magie Avanzato}{Dissolvi Magie Avanzato}, Raro, 5\\
\hyperlink{Globo di Invulnerabilità}{Globo di Invulnerabilità}, Comune, 6\\
\hyperlink{Proibizione}{Proibizione}, Non Comune, 6\\
\hyperlink{Vigilanza e Interdizione}{Vigilanza e Interdizione}, Non Comune, 6\\
\hyperlink{Simbolo}{Simbolo}, Non Comune, 7\\
\hyperlink{Aura Sacra}{Aura Sacra}, Comune, 8\\
\hyperlink{Campo Anti-Magia}{Campo Anti-Magia}, Raro, 8\\
\hyperlink{Scudo Mentale}{Scudo Mentale}, Non Comune, 8\\
\hyperlink{Imprigionare}{Imprigionare}, Raro, 9

\medskip\textbf{Animali e Piante}

\hyperlink{Randello Incantato}{Randello Incantato}, Comune, 0\\
\hyperlink{Spruzzo Velenoso}{Spruzzo Velenoso}, Non Comune, 0\\
\hyperlink{Viticci Perforanti}{Viticci Perforanti}, Non Comune, 0\\
\hyperlink{Intralciare}{Intralciare}, Comune, 1\\
\hyperlink{Parlare con gli Animali}{Parlare con gli Animali}, Comune, 1\\
\hyperlink{Purificare Cibo e Bevande}{Purificare Cibo e Bevande}, Comune, 1\\
\hyperlink{Unto}{Unto}, Comune, 1\\
\hyperlink{Amicizia con gli Animali}{Amicizia con gli Animali}, Non Comune, 1\\
\hyperlink{Animale Messaggero}{Animale Messaggero}, Comune, 2\\
\hyperlink{Bacche Benefiche}{Bacche Benefiche}, Comune, 2\\
\hyperlink{Crescita di Spuntoni}{Crescita di Spuntoni}, Comune, 2\\
\hyperlink{Evoca Cavalcatura}{Evoca Cavalcatura}, Comune, 2\\
\hyperlink{Gragnola di Ghiande di Kyrin}{Gragnola di Ghiande di Kyrin}, Non Comune, 2\\
\hyperlink{Gragnola di Limoni di Kyrin}{Gragnola di Limoni di Kyrin}, Molto Raro, 2\\
\hyperlink{Localizza Animali e Piante}{Localizza Animali e Piante}, Non Comune, 2\\
\hyperlink{Movimenti del Ragno}{Movimenti del Ragno}, Non Comune, 2\\
\hyperlink{Passare Senza Tracce}{Passare Senza Tracce}, Comune, 2\\
\hyperlink{Pelle di Corteccia}{Pelle di Corteccia}, Comune, 2\\
\hyperlink{Ragnatela}{Ragnatela}, Comune, 2\\
\hyperlink{Succo concentrato di Ribes di Kyrin}{Succo concentrato di Ribes di Kyrin}, Non Comune, 2\\
\hyperlink{Bastoni in Serpenti}{Bastoni in Serpenti}, Non Comune, 3\\
\hyperlink{Crescita Vegetale}{Crescita Vegetale}, Non Comune, 3\\
\hyperlink{Evoca Animali}{Evoca Animali}, Non Comune, 3\\
\hyperlink{Gragnola di Ghiande Infuocate di Kyrin}{Gragnola di Ghiande Infuocate di Kyrin}, Raro, 3\\
\hyperlink{Parlare con le Piante}{Parlare con le Piante}, Raro, 3\\
\hyperlink{Dominare Bestie}{Dominare Bestie}, Comune, 4\\
\hyperlink{Insetto Gigante}{Insetto Gigante}, Non Comune, 4\\
\hyperlink{Localizza Creatura}{Localizza Creatura}, Comune, 4\\
\hyperlink{Metamorfosi}{Metamorfosi}, Comune, 4\\
\hyperlink{Gragnola di Marroni di Kyrin}{Gragnola di Marroni di Kyrin}, Molto Raro, 5\\
\hyperlink{Guscio Anti-Vita}{Guscio Anti-Vita}, Non Comune, 5\\
\hyperlink{Piaga degli Insetti}{Piaga degli Insetti}, Raro, 5\\
\hyperlink{Reincarnazione}{Reincarnazione}, Raro, 5\\
\hyperlink{Risveglio}{Risveglio}, Raro, 5\\
\hyperlink{Traslazione Arborea}{Traslazione Arborea}, Raro, 5\\
\hyperlink{Barriera Antianimali}{Barriera Antianimali}, Raro, 5\\
\hyperlink{Parlare con le Creature}{Parlare con le Creature}, Raro, 6\\
\hyperlink{Muro di Spine}{Muro di Spine}, Non Comune, 6\\
\hyperlink{Trasporto Vegetale}{Trasporto Vegetale}, Molto Raro, 6\\
\hyperlink{Benedizioni di Efrem}{Benedizioni di Efrem}, Raro, 8\\
\hyperlink{Metamorfosi Pura}{Metamorfosi Pura}, Raro, 9

\medskip\textbf{Ammaliamento}

\hyperlink{Beffa Crudele}{Beffa Crudele}, Comune, 0\\
\hyperlink{Dito}{Dito}, Raro, 0\\
\hyperlink{Charme su Persone}{Charme su Persone}, Comune, 1\\
\hyperlink{Comando}{Comando}, Comune, 1\\
\hyperlink{Eroismo}{Eroismo}, Non Comune, 1\\
\hyperlink{Risata Incontenibile}{Risata Incontenibile}, Non Comune, 1\\
\hyperlink{Sonno}{Sonno}, Comune, 1\\
\hyperlink{Anatema}{Anatema}, Comune, 1\\
\hyperlink{Blocca Persona}{Blocca Persona}, Comune, 2\\
\hyperlink{Calmare Emozioni}{Calmare Emozioni}, Comune, 2\\
\hyperlink{Estasiare}{Estasiare}, Comune, 2\\
\hyperlink{Sonnellino}{Sonnellino}, Leggendario, 2\\
\hyperlink{Suggestione}{Suggestione}, Raro, 2\\
\hyperlink{Zona di Verità}{Zona di Verità}, Non Comune, 2\\
\hyperlink{Benedizione di Cattalm}{Benedizione di Cattalm}, Molto Raro, 3\\
\hyperlink{Blocca Persona Avanzato}{Blocca Persona Avanzato}, Non Comune, 4\\
\hyperlink{Compulsione}{Compulsione}, Non Comune, 4\\
\hyperlink{Confusione}{Confusione}, Comune, 4\\
\hyperlink{Dominare Bestie}{Dominare Bestie}, Molto Raro, 4\\
\hyperlink{Costrizione}{Costrizione}, Raro, 5\\
\hyperlink{Dominare Persone}{Dominare Persone}, Non Comune, 5\\
\hyperlink{Modificare Memoria}{Modificare Memoria}, Molto Raro, 5\\
\hyperlink{Suggestione di Massa}{Suggestione di Massa}, Molto Raro, 6\\
\hyperlink{Antipatia/Simpatia}{Antipatia/Simpatia}, Raro, 8\\
\hyperlink{Confusione Contagiosa}{Confusione Contagiosa}, Molto Raro, 8\\
\hyperlink{Danza Irresistibile}{Danza Irresistibile}, Leggendario, 8\\
\hyperlink{Dominare Mostri}{Dominare Mostri}, Non Comune, 8\\
\hyperlink{Parola del Potere Stordire}{Parola del Potere Stordire}, Non Comune, 8\\
\hyperlink{Regressione Mentale}{Regressione Mentale}, Raro, 8\\
\hyperlink{Parola del Potere Uccidere}{Parola del Potere Uccidere}, Raro, 9

\medskip\textbf{Cura}

\hyperlink{Cura Ferite}{Cura Ferite}, Comune, 1\\
\hyperlink{Preghiera di Guarigione}{Preghiera di Guarigione}, Comune, 2\\
\hyperlink{Ristorare Inferiore}{Ristorare Inferiore}, Comune, 2\\
\hyperlink{Aiuto}{Aiuto}, Non Comune, 2\\
\hyperlink{Rimuovi Malattia}{Rimuovi Malattia}, Comune, 2\\
\hyperlink{Benedizione della Vita}{Benedizione della Vita}, Raro, 3\\
\hyperlink{Distruggere nonmorto}{Distruggere Nonmorto}, Non Comune, 3\\
\hyperlink{Rimuovi Veleno}{Rimuovi Veleno}, Comune, 3\\
\hyperlink{Rinascita}{Rinascita}, Molto Raro, 3\\
\hyperlink{Profumo di Atherim}{Profumo di Atherim}, Molto Raro, 3\\
\hyperlink{Vigore}{Vigore}, Raro, 4\\
\hyperlink{Ristorare Superiore}{Ristorare Superiore}, Non Comune, 5\\
\hyperlink{Guarigione}{Guarigione}, Raro, 6\\
\hyperlink{Rigenerazione}{Rigenerazione}, Leggendario, 7\\
\hyperlink{Guarigione di Massa}{Guarigione di Massa}, Leggendario, 9

\medskip\textbf{Divinazione}

\hyperlink{Colpo Accurato}{Colpo Accurato}, Comune, 0\\
\hyperlink{Comprensione dei Linguaggi}{Comprensione dei Linguaggi}, Comune, 1\\
\hyperlink{Conoscere i Tratti}{Conoscere i Tratti}, Leggendario, 1\\
\hyperlink{Guida}{Guida}, Comune, 1\\
\hyperlink{Comprensione degli Scritti}{Comprensione degli Scritti}, Non Comune, 2\\
\hyperlink{Individuazione dei Pensieri}{Individuazione dei Pensieri}, Raro, 2\\
\hyperlink{Individuazione delle Malattie e dei Veleni}{Individuazione delle Malattie e dei Veleni}, Non Comune, 2\\
\hyperlink{Localizza Oggetto}{Localizza Oggetto}, Comune, 2\\
\hyperlink{Scopri Piante}{Scopri Piante}, Non Comune, 2\\
\hyperlink{Presagio}{Presagio}, Comune, 2\\
\hyperlink{Scopri Trappole}{Scopri Trappole}, Comune, 2\\
\hyperlink{Vedere l'invisibile}{Vedere l'invisibile}, Comune, 2\\
\hyperlink{Lingue}{Lingue}, Comune, 3\\
\hyperlink{Chiaroveggenza}{Chiaroveggenza}, Comune, 3\\
\hyperlink{Occhio Arcano}{Occhio Arcano}, Comune, 4\\
\hyperlink{Comunione}{Comunione}, Raro, 5\\
\hyperlink{Comunione con la Natura}{Comunione con la Natura}, Molto Raro, 5\\
\hyperlink{Conoscenza delle Leggende}{Conoscenza delle Leggende}, Comune, 5\\
\hyperlink{Legame Telepatico}{Legame Telepatico}, Raro, 5\\
\hyperlink{Scrutare}{Scrutare}, Raro, 5\\
\hyperlink{Divinazione}{Divinazione}, Comune, 6\\
\hyperlink{Parlare con le Creature}{Parlare con le Creature}, Raro, 6\\
\hyperlink{Pietre Parlanti}{Pietre Parlanti}, Raro, 6\\
\hyperlink{Scopri il Percorso}{Scopri il Percorso}, Non Comune, 6\\
\hyperlink{Visione del Vero}{Visione del Vero}, Raro, 6\\
\hyperlink{Previsione}{Previsione}, Non Comune, 9

\medskip\textbf{Evocazione}

\hyperlink{Creare Birra}{Creare Birra}, Raro, 0\\
\hyperlink{Fiotto Acido}{Fiotto Acido}, Comune, 0\\
\hyperlink{Mano Magica}{Mano Magica}, Comune, 0\\
\hyperlink{Cuoco Invisibile}{Cuoco Invisibile}, Comune, 1\\
\hyperlink{Disco Fluttuante}{Disco Fluttuante}, Comune, 1\\
\hyperlink{Schiaffo di Cattalm}{Schiaffo di Cattalm}, Non Comune, 1\\
\hyperlink{Servitore Invisibile}{Servitore Invisibile}, Comune, 1\\
\hyperlink{Passo Velato}{Passo Velato}, Non Comune, 2\\
\hyperlink{Creare Cibo e Acqua}{Creare Cibo e Acqua}, Comune, 3\\
\hyperlink{Porta Dimensionale}{Porta Dimensionale}, Comune, 4\\
\hyperlink{Scrigno Segreto}{Scrigno Segreto}, Raro, 4\\
\hyperlink{Segugio Fedele}{Segugio Fedele}, Raro, 4\\
\hyperlink{Tentacoli Neri}{Tentacoli Neri}, Non Comune, 4\\
\hyperlink{Cerchio di Teletrasporto}{Cerchio di Teletrasporto}, Non Comune, 5\\
\hyperlink{Evocazioni Istantanee}{Evocazioni Istantanee}, Raro, 6\\
\hyperlink{Parola del Ritiro}{Parola del Ritiro}, Raro, 6\\
\hyperlink{Desiderio limitato}{Desiderio limitato}, Molto Raro, 7\\
\hyperlink{Reggia Meravigliosa}{Reggia Meravigliosa}, Leggendario, 7\\
\hyperlink{Teletrasporto}{Teletrasporto}, Comune, 7\\
\hyperlink{Labirinto}{Labirinto}, Raro, 8\\
\hyperlink{Desiderio}{Desiderio}, Non Comune, 9

\medskip\textbf{Illusione}

\hyperlink{Camuffare Sé Stesso}{Camuffare Sé Stesso}, Comune, 1\\
\hyperlink{Immagine Silenziosa}{Immagine Silenziosa}, Comune, 1\\
\hyperlink{Scritto Illusorio}{Scritto Illusorio}, Comune, 1\\
\hyperlink{Spruzzo Colorato}{Spruzzo Colorato}, Comune, 1\\
\hyperlink{Ventriloquio}{Ventriloquio}, Comune, 1\\
\hyperlink{Aura Magica dell'Arcanista}{Aura Magica dell'Arcanista}, Non Comune, 2\\
\hyperlink{Bocca Magica}{Bocca Magica}, Comune, 2\\
\hyperlink{Immagine Speculare}{Immagine Speculare}, Comune, 2\\
\hyperlink{Invisibilità}{Invisibilità}, Comune, 2\\
\hyperlink{Lacrima di Laydel}{Lacrima di Laydel}, Molto Raro, 2\\
\hyperlink{Sfocatura}{Sfocatura}, Comune, 2\\
\hyperlink{Silenzio}{Silenzio}, Comune, 2\\
\hyperlink{Cerchio d'Invisibilità}{Cerchio d'Invisibilità}, Non Comune, 3\\
\hyperlink{Destriero Fantasma}{Destriero Fantasma}, Comune, 3\\
\hyperlink{Immagine Maggiore}{Immagine Maggiore}, Comune, 3\\
\hyperlink{Paura}{Paura}, Non Comune, 3\\
\hyperlink{Trama Ipnotica}{Trama Ipnotica}, Comune, 3\\
\hyperlink{Invisibilità Superiore}{Invisibilità Superiore}, Non Comune, 4\\
\hyperlink{Terreno Illusorio}{Terreno Illusorio}, Non Comune, 4\\
\hyperlink{Allucinazione Mortale}{Allucinazione Mortale}, Non Comune, 4\\
\hyperlink{Creazione}{Creazione}, Raro, 5\\
\hyperlink{Fuorviare}{Fuorviare}, Non Comune, 5\\
\hyperlink{Sembrare}{Sembrare}, Non Comune, 5\\
\hyperlink{Sogno}{Sogno}, Non Comune, 5\\
\hyperlink{Illusione Programmata}{Illusione Programmata}, Non Comune, 6\\
\hyperlink{Immagine Proiettata}{Immagine Proiettata}, Non Comune, 7\\
\hyperlink{Miraggio Arcano}{Miraggio Arcano}, Raro, 7\\
\hyperlink{Fatale}{Fatale}, Raro, 9

\medskip\textbf{Invocazione}

\hyperlink{Colpo Fiammeggiante}{Colpo Fiammeggiante}, Raro, 1\\
\hyperlink{Dardo Tracciante}{Dardo Tracciante}, Non Comune, 1\\
\hyperlink{Dardo occulto}{Dardo occulto}, Comune, 1\\
\hyperlink{Favore Divino}{Favore Divino}, Non Comune, 1\\
\hyperlink{Luci Danzanti}{Luci Danzanti}, Non Comune, 1\\
\hyperlink{Luminescenza}{Luminescenza}, Non Comune, 1\\
\hyperlink{Oscurità}{Oscurità}, Comune, 1\\
\hyperlink{Benedizione Superiore}{Benedizione Superiore}, Non Comune, 2\\
\hyperlink{Frantumare}{Frantumare}, Comune, 2\\
\hyperlink{Punizione Marchiante}{Punizione Marchiante}, Comune, 2\\
\hyperlink{Arma Spirituale}{Arma Spirituale}, Comune, 2\\
\hyperlink{Colpo Luccicante}{Colpo Luccicante}, Non Comune, 2\\
\hyperlink{Creare Fossa}{Creare Fossa}, Non Comune, 2\\
\hyperlink{Benedizione Suprema}{Benedizione Suprema}, Raro, 3\\
\hyperlink{Capanna}{Capanna}, Non Comune, 3\\
\hyperlink{Colpo Accecante}{Colpo Accecante}, 3\\
\hyperlink{Inviare}{Inviare}, Comune, 3\\
\hyperlink{Luce Diurna}{Luce Diurna}, Comune, 3\\
\hyperlink{Preghiera}{Preghiera}, Non Comune, 3\\
\hyperlink{Mano Arcana}{Mano Arcana}, Non Comune, 5\\
\hyperlink{Muro di Forza}{Muro di Forza}, Comune, 5\\
\hyperlink{Santificare}{Santificare}, Raro, 5\\
\hyperlink{Bagliore Solare}{Bagliore Solare}, Non Comune, 6\\
\hyperlink{Banchetto degli Eroi}{Banchetto degli Eroi}, Non Comune, 6\\
\hyperlink{Barriera di Lame}{Barriera di Lame}, Comune, 6\\
\hyperlink{Cerchio di Morte}{Cerchio di Morte}, Molto Raro, 6\\
\hyperlink{Contingenza}{Contingenza}, Comune, 6\\
\hyperlink{Parola Divina}{Parola Divina}, Molto Raro, 7\\
\hyperlink{Spada Arcana}{Spada Arcana}, Raro, 7\\
\hyperlink{Spruzzo Prismatico}{Spruzzo Prismatico}, Raro, 7\\
\hyperlink{Esplosione Solare}{Esplosione Solare}, Raro, 8\\
\hyperlink{Gabbia di Forza}{Gabbia di Forza}, Raro, 8\\

\medskip\textbf{Necromanzia}

\hyperlink{Tocco Gelido}{Tocco Gelido}, Comune, 0\\
\hyperlink{Vita Falsata}{Vita Falsata}, Comune, 1\\
\hyperlink{Grido di dolore}{Grido di dolore}, Raro, 1\\
\hyperlink{Raggio di Indebolimento}{Raggio di Indebolimento}, Comune, 1\\
\hyperlink{Cecità/Sordità}{Cecità/Sordità}, Comune, 2\\
\hyperlink{Infliggi Ferite}{Infliggi Ferite}, Comune, 2\\
\hyperlink{Raggio mortale}{Raggio mortale}, Raro, 2\\
\hyperlink{Riposo Inviolato}{Riposo Inviolato}, Non Comune, 2\\
\hyperlink{Aiuto}{Aiuto}, Non Comune, 2\\
\hyperlink{Animare Morti}{Animare Morti}, Comune, 3\\
\hyperlink{Benedizione di Ledyal}{Benedizione di Ledyal}, Molto Raro, 3\\
\hyperlink{Cecità/Sordità Avanzata}{Cecità/Sordità Avanzata}, Non Comune, 3\\
\hyperlink{Parlare con i Morti}{Parlare con i Morti}, Raro, 3\\
\hyperlink{Morte Apparente}{Morte Apparente}, Non Comune, 3\\
\hyperlink{Rinascita}{Rinascita}, Molto Raro, 3\\
\hyperlink{Scagliare Maledizione}{Scagliare Maledizione}, Non Comune, 3\\
\hyperlink{Tocco Vampirico}{Tocco Vampirico}, Raro, 3\\
\hyperlink{Inaridire}{Inaridire}, Non Comune, 4\\
\hyperlink{Interdizione alla Morte}{Interdizione alla Morte}, Non Comune, 4\\
\hyperlink{Contagio}{Contagio}, Non Comune, 5\\
\hyperlink{Creare Non Morti}{Creare Non Morti}, Non Comune, 6\\
\hyperlink{Dito della Morte}{Dito della Morte}, Raro, 6\\
\hyperlink{Ferire}{Ferire}, Non Comune, 6\\
\hyperlink{Giara Magica}{Giara Magica}, Molto Raro, 6\\
\hyperlink{Sguardo Penetrante}{Sguardo Penetrante}, Molto Raro, 6

\medskip\textbf{Trasmutazione}

\hyperlink{Messaggio}{Messaggio}, Comune, 0\\
\hyperlink{Passo Veloce}{Passo Veloce}, Molto Raro, 1\\
\hyperlink{Ritirata Rapida}{Ritirata Rapida}, Non Comune, 1\\
\hyperlink{Alterare Sé Stesso}{Alterare Sé Stesso}, Non Comune, 1\\
\hyperlink{Arma Magica}{Arma Magica}, Comune, 2\\
\hyperlink{Caratteristica Potenziata}{Caratteristica Potenziata}, Comune, 2\\
\hyperlink{Goffaggine}{Goffaggine}, Raro, 2\\
\hyperlink{Ingrandire/Ridurre}{Ingrandire/Ridurre}, Comune, 2\\
\hyperlink{Scassinare}{Scassinare}, Comune, 2\\
\hyperlink{Scurovisione}{Scurovisione}, Comune, 2\\
\hyperlink{Trucco della Corda}{Trucco della Corda}, Comune, 2\\
\hyperlink{Forma Gassosa}{Forma Gassosa}, Non Comune, 3\\
\hyperlink{Intermittenza}{Intermittenza}, Non Comune, 3\\
\hyperlink{lentezza}{Lentezza}, Non Comune, 3\\
\hyperlink{Velocità}{Velocità}, Non Comune, 3\\
\hyperlink{Fabbricare}{Fabbricare}, Comune, 4\\
\hyperlink{Animare Oggetti}{Animare Oggetti}, Comune, 5\\
\hyperlink{Telecinesi}{Telecinesi}, Non Comune, 5\\
\hyperlink{Disintegrazione}{Disintegrazione}, Non Comune, 6\\
\hyperlink{Trasformazione Furiosa di Restser}{Trasformazione Furiosa di Restser}, Molto Raro, 6\\
\hyperlink{Celare}{Celare}, Raro, 7\\
\hyperlink{Forma Eterea}{Forma Eterea}, Raro, 7\\
\hyperlink{Inversione della Gravità}{Inversione della Gravità}, Raro, 7\\
\hyperlink{Statua}{Statua}, Raro, 7\\
\hyperlink{Loquacità}{Loquacità}, Raro, 8\\
\hyperlink{Fermare il Tempo}{Fermare il Tempo}, Molto Raro, 9\\
\hyperlink{Trasformazione}{Trasformazione}, Raro, 9

\medskip\textbf{Universale}

\hyperlink{Fiamma Sacra}{Fiamma Sacra}, Comune, 0\\
\hyperlink{Lacrima di Ljust}{Lacrima di Ljust}, Non Comune, 0\\
\hyperlink{Marchio Magico}{Marchio Magico}, Comune, 0\\
\hyperlink{Prestidigitazione}{Prestidigitazione}, Comune, 0\\
\hyperlink{Scudo}{Scudo}, Comune, 0\\
\hyperlink{Taumaturgia}{Taumaturgia}, Non Comune, 0\\
\hyperlink{Artificio Druidico}{Artificio Druidico}, Non Comune, 0\\
\hyperlink{Benedizione}{Benedizione}, Comune, 1\\
\hyperlink{Dardo arcano}{Dardo arcano}, Comune, 1\\
\hyperlink{Identificare}{Identificare}, Comune, 1\\
\hyperlink{Illusione Minore}{Illusione Minore}, Comune, 1\\
\hyperlink{Individuazione del Magico}{Individuazione del Magico}, Comune, 1\\
\hyperlink{Lettura del Magico}{Lettura del Magico}, Comune, 1\\
\hyperlink{Luce}{Luce}, Comune, 1\\
\hyperlink{Scagliare Maledizione Minore}{Scagliare Maledizione Minore}, Comune, 1\\
\hyperlink{Benedici Acqua}{Benedici Acqua}, Comune, 2\\
\hyperlink{Fiamma Perenne}{Fiamma Perenne}, Leggendario, 2
}}

\end{multicols}

\subsection{Incantesimi per Livello}\hypertarget{elencoinc}{}\index{Incantesimi per livello}

Sono elencati gli incantesimi in ordine per livello e alfabetico. Vedi Capitolo \hyperlink{CPergamene}{Generazione Oggetti Magici} (pag. \pageref{CPergamene}) per generare Tomi casuali.

\begin{multicols}{3}

\small{

\setlength{\parindent}{0cm}{

\textbf{Livello 0 - Trucchetti} \newcounter{inclvzero}

\stepcounter{inclvzero}\hyperlink{Artificio Druidico}{Artificio Druidico}, Non Comune, 0\\
\stepcounter{inclvzero}\hyperlink{Beffa Crudele}{Beffa Crudele}, Comune, 0\\
\stepcounter{inclvzero}\hyperlink{Colpo Accecante}{Colpo Accecante}, 3\\
\stepcounter{inclvzero}\hyperlink{Colpo Accurato}{Colpo Accurato}, Comune, 0\\
\stepcounter{inclvzero}\hyperlink{Creare Birra}{Creare Birra}, Raro, 0\\
\stepcounter{inclvzero}\hyperlink{Dito}{Dito}, Raro, 0\\
\stepcounter{inclvzero}\hyperlink{Fiamma Sacra}{Fiamma Sacra}, Comune, 0\\
\stepcounter{inclvzero}\hyperlink{Fiotto Acido}{Fiotto Acido}, Comune, 0\\
\stepcounter{inclvzero}\hyperlink{Lacrima di Ljust}{Lacrima di Ljust}, Non Comune, 0\\
\stepcounter{inclvzero}\hyperlink{Mano Magica}{Mano Magica}, Comune, 0\\
\stepcounter{inclvzero}\hyperlink{Marchio Magico}{Marchio Magico}, Comune, 0\\
\stepcounter{inclvzero}\hyperlink{Messaggio}{Messaggio}, Comune, 0\\
\stepcounter{inclvzero}\hyperlink{Prestidigitazione}{Prestidigitazione}, Comune, 0\\
\stepcounter{inclvzero}\hyperlink{Produrre Fiamma}{Produrre Fiamma}, Comune, 0\\
\stepcounter{inclvzero}\hyperlink{Raggio di Gelo}{Raggio di Gelo}, Comune, 0\\
\stepcounter{inclvzero}\hyperlink{Randello Incantato}{Randello Incantato}, Comune, 0\\
\stepcounter{inclvzero}\hyperlink{Resistenza}{Resistenza}, Comune, 0\\
\stepcounter{inclvzero}\hyperlink{Riparare}{Riparare}, Comune, 0\\
\stepcounter{inclvzero}\hyperlink{Scudo}{Scudo}, Comune, 0\\
\stepcounter{inclvzero}\hyperlink{Spruzzo Velenoso}{Spruzzo Velenoso}, Non Comune, 0\\
\stepcounter{inclvzero}\hyperlink{Stretta Folgorante}{Stretta Folgorante}, Comune, 0\\
\stepcounter{inclvzero}\hyperlink{Taumaturgia}{Taumaturgia}, Non Comune, 0\\
\stepcounter{inclvzero}\hyperlink{Tocco Gelido}{Tocco Gelido}, Comune, 0\\
\stepcounter{inclvzero}\hyperlink{Viticci Perforanti}{Viticci Perforanti}, Non Comune, 0\\

Totale incantesimi: \theinclvzero\\

\textbf{Livello 1} \newcounter{inclvuno}

\stepcounter{inclvuno}\hyperlink{Allarme}{Allarme}, Comune, 1\\
\stepcounter{inclvuno}\hyperlink{Alterare Sé Stesso}{Alterare Sé Stesso}, Non Comune, 1\\
\stepcounter{inclvuno}\hyperlink{Amicizia con gli Animali}{Amicizia con gli Animali}, Non Comune, 1\\
\stepcounter{inclvuno}\hyperlink{Anatema}{Anatema}, Comune, 1\\
\stepcounter{inclvuno}\hyperlink{Armatura Magica}{Armatura Magica}, Non Comune, 1\\
\stepcounter{inclvuno}\hyperlink{Benedizione}{Benedizione}, Comune, 1\\
\stepcounter{inclvuno}\hyperlink{Caduta Piuma}{Caduta Piuma}, Comune, 1\\
\stepcounter{inclvuno}\hyperlink{Camuffare Sé Stesso}{Camuffare Sé Stesso}, Comune, 1\\
\stepcounter{inclvuno}\hyperlink{Charme su Persone}{Charme su Persone}, Comune, 1\\
\stepcounter{inclvuno}\hyperlink{Colpo Fiammeggiante}{Colpo Fiammeggiante}, Raro, 1\\
\stepcounter{inclvuno}\hyperlink{Comando}{Comando}, Comune, 1\\
\stepcounter{inclvuno}\hyperlink{Comprensione dei Linguaggi}{Comprensione dei Linguaggi}, Comune, 1\\
\stepcounter{inclvuno}\hyperlink{Conoscere i Tratti}{Conoscere i Tratti}, Leggendario, 1\\
\stepcounter{inclvuno}\hyperlink{Creare o Distruggere Acqua}{Creare o Distruggere Acqua}, Comune, 1\\
\stepcounter{inclvuno}\hyperlink{Cuoco Invisibile}{Cuoco Invisibile}, Comune, 1\\
\stepcounter{inclvuno}\hyperlink{Cura Ferite}{Cura Ferite}, Comune, 1\\
\stepcounter{inclvuno}\hyperlink{Dardo arcano}{Dardo arcano}, Comune, 1\\
\stepcounter{inclvuno}\hyperlink{Dardo di Fuoco}{Dardo di Fuoco}, Comune, 1\\
\stepcounter{inclvuno}\hyperlink{Dardo occulto}{Dardo occulto}, Comune, 1\\
\stepcounter{inclvuno}\hyperlink{Dardo Tracciante}{Dardo Tracciante}, Non Comune, 1\\
\stepcounter{inclvuno}\hyperlink{Disco Fluttuante}{Disco Fluttuante}, Comune, 1\\
\stepcounter{inclvuno}\hyperlink{Eroismo}{Eroismo}, Non Comune, 1\\
\stepcounter{inclvuno}\hyperlink{Favore Divino}{Favore Divino}, Non Comune, 1\\
\stepcounter{inclvuno}\hyperlink{Grido di dolore}{Grido di dolore}, Raro, 1\\
\stepcounter{inclvuno}\hyperlink{Guida}{Guida}, Comune, 1\\
\stepcounter{inclvuno}\hyperlink{Identificare}{Identificare}, Comune, 1\\
\stepcounter{inclvuno}\hyperlink{Illusione Minore}{Illusione Minore}, Comune, 1\\
\stepcounter{inclvuno}\hyperlink{Immagine Silenziosa}{Immagine Silenziosa}, Comune, 1\\
\stepcounter{inclvuno}\hyperlink{Individuazione del Magico}{Individuazione del Magico}, Comune, 1\\
\stepcounter{inclvuno}\hyperlink{Intralciare}{Intralciare}, Comune, 1\\
\stepcounter{inclvuno}\hyperlink{Lettura del Magico}{Lettura del Magico}, Comune, 1\\
\stepcounter{inclvuno}\hyperlink{Luce}{Luce}, Comune, 1\\
\stepcounter{inclvuno}\hyperlink{Luci Danzanti}{Luci Danzanti}, Non Comune, 1\\
\stepcounter{inclvuno}\hyperlink{Luminescenza}{Luminescenza}, Non Comune, 1\\
\stepcounter{inclvuno}\hyperlink{Onda rovente}{Onda rovente}, Comune, 1\\
\stepcounter{inclvuno}\hyperlink{Onda Tonante}{Onda Tonante}, Comune, 1\\
\stepcounter{inclvuno}\hyperlink{Oscurità}{Oscurità}, Comune, 1\\
\stepcounter{inclvuno}\hyperlink{Palla di fango di Eithne}{Palla di fango di Eithne}, Non Comune, 1\\
\stepcounter{inclvuno}\hyperlink{Parlare con gli Animali}{Parlare con gli Animali}, Comune, 1\\
\stepcounter{inclvuno}\hyperlink{Passo Veloce}{Passo Veloce}, Molto Raro, 1\\
\stepcounter{inclvuno}\hyperlink{Protezione dall'Energia minore}{Protezione dall'Energia minore}, Raro, 1\\
\stepcounter{inclvuno}\hyperlink{Purificare Cibo e Bevande}{Purificare Cibo e Bevande}, Comune, 1\\
\stepcounter{inclvuno}\hyperlink{Raggio di Indebolimento}{Raggio di Indebolimento}, Comune, 1\\
\stepcounter{inclvuno}\hyperlink{Risata Incontenibile}{Risata Incontenibile}, Non Comune, 1\\
\stepcounter{inclvuno}\hyperlink{Ritirata Rapida}{Ritirata Rapida}, Non Comune, 1\\
\stepcounter{inclvuno}\hyperlink{Saltare}{Saltare}, Comune, 1\\
\stepcounter{inclvuno}\hyperlink{Santuario}{Santuario}, Comune, 1\\
\stepcounter{inclvuno}\hyperlink{Rimuovi Paura}{Rimuovi Paura}, Comune, 1 \\
\stepcounter{inclvuno}\hyperlink{Scagliare Maledizione Minore}{Scagliare Maledizione Minore}, Comune, 1\\
\stepcounter{inclvuno}\hyperlink{Schiaffo di Cattalm}{Schiaffo di Cattalm}, Non Comune, 1\\
\stepcounter{inclvuno}\hyperlink{Scritto Illusorio}{Scritto Illusorio}, Comune, 1\\
\stepcounter{inclvuno}\hyperlink{Servitore Invisibile}{Servitore Invisibile}, Comune, 1\\
\stepcounter{inclvuno}\hyperlink{Sonno}{Sonno}, Comune, 1\\
\stepcounter{inclvuno}\hyperlink{Spruzzo Colorato}{Spruzzo Colorato}, Comune, 1\\
\stepcounter{inclvuno}\hyperlink{Unto}{Unto}, Comune, 1\\
\stepcounter{inclvuno}\hyperlink{Ventriloquio}{Ventriloquio}, Comune, 1\\
\stepcounter{inclvuno}\hyperlink{Vita Falsata}{Vita Falsata}, Comune, 1\\


\medskip Totale incantesimi: \theinclvuno\\

\textbf{Livello 2} \newcounter{inclvdue}

\stepcounter{inclvdue}\hyperlink{Animale Messaggero}{Animale Messaggero}, Comune, 2\\
\stepcounter{inclvdue}\hyperlink{Arma Magica}{Arma Magica}, Comune, 2\\
\stepcounter{inclvdue}\hyperlink{Arma Spirituale}{Arma Spirituale}, Comune, 2\\
\stepcounter{inclvdue}\hyperlink{Aura Magica dell'Arcanista}{Aura Magica dell'Arcanista}, Non Comune, 2\\
\stepcounter{inclvdue}\hyperlink{Bacche Benefiche}{Bacche Benefiche}, Comune, 2\\
\stepcounter{inclvdue}\hyperlink{Benedici Acqua}{Benedici Acqua}, Comune, 2\\
\stepcounter{inclvdue}\hyperlink{Benedizione Superiore}{Benedizione Superiore}, Non Comune, 2\\
\stepcounter{inclvdue}\hyperlink{Blocca Persona}{Blocca Persona}, Comune, 2\\
\stepcounter{inclvdue}\hyperlink{Bocca Magica}{Bocca Magica}, Comune, 2\\
\stepcounter{inclvdue}\hyperlink{Calmare Emozioni}{Calmare Emozioni}, Comune, 2\\
\stepcounter{inclvdue}\hyperlink{Caratteristica Potenziata}{Caratteristica Potenziata}, Comune, 2\\
\stepcounter{inclvdue}\hyperlink{Cecità/Sordità}{Cecità/Sordità}, Comune, 2\\
\stepcounter{inclvdue}\hyperlink{Chiudi Portale}{Chiudi Portale}, Raro, 2\\
\stepcounter{inclvdue}\hyperlink{Colpo Luccicante}{Colpo Luccicante}, Non Comune, 2\\
\stepcounter{inclvdue}\hyperlink{Comprensione degli Scritti}{Comprensione degli Scritti}, Non Comune, 2\\
\stepcounter{inclvdue}\hyperlink{Creare Fossa}{Creare Fossa}, Non Comune, 2\\
\stepcounter{inclvdue}\hyperlink{Crescita di Spuntoni}{Crescita di Spuntoni}, Comune, 2\\
\stepcounter{inclvdue}\hyperlink{Estasiare}{Estasiare}, Comune, 2\\
\stepcounter{inclvdue}\hyperlink{Evoca Cavalcatura}{Evoca Cavalcatura}, Comune, 2\\
\stepcounter{inclvdue}\hyperlink{Fiamma Perenne}{Fiamma Perenne}, Leggendario, 2\\
\stepcounter{inclvdue}\hyperlink{Folata di Vento}{Folata di Vento}, Comune, 2\\
\stepcounter{inclvdue}\hyperlink{Frantumare}{Frantumare}, Comune, 2\\
\stepcounter{inclvdue}\hyperlink{Goffaggine}{Goffaggine}, Raro, 2\\
\stepcounter{inclvdue}\hyperlink{Gragnola di Ghiande di Kyrin}{Gragnola di Ghiande di Kyrin}, Non Comune, 2\\
\stepcounter{inclvdue}\hyperlink{Immagine Speculare}{Immagine Speculare}, Comune, 2\\
\stepcounter{inclvdue}\hyperlink{Individuazione dei Pensieri}{Individuazione dei Pensieri}, Raro, 2\\
\stepcounter{inclvdue}\hyperlink{Individuazione delle Malattie e dei Veleni}{Individuazione delle Malattie e dei Veleni}, Non Comune, 2\\
\stepcounter{inclvdue}\hyperlink{Infliggi Ferite}{Infliggi Ferite}, Comune, 2\\
\stepcounter{inclvdue}\hyperlink{Ingrandire/Ridurre}{Ingrandire/Ridurre}, Comune, 2\\
\stepcounter{inclvdue}\hyperlink{Invisibilità}{Invisibilità}, Comune, 2\\
\stepcounter{inclvdue}\hyperlink{Lacrima di Laydel}{Lacrima di Laydel}, Molto Raro, 2\\
\stepcounter{inclvdue}\hyperlink{Lama Infuocata}{Lama Infuocata}, Comune, 2\\
\stepcounter{inclvdue}\hyperlink{Lanciafiamme}{Lanciafiamme}, Raro, 2\\
\stepcounter{inclvdue}\hyperlink{Lettura della terra di Kyrin}{Lettura della terra di Kyrin}, Non Comune, 2\\
\stepcounter{inclvdue}\hyperlink{Levitazione}{Levitazione}, Comune, 2\\
\stepcounter{inclvdue}\hyperlink{Localizza Animali e Piante}{Localizza Animali e Piante}, Non Comune, 2\\
\stepcounter{inclvdue}\hyperlink{Localizza Oggetto}{Localizza Oggetto}, Comune, 2\\
\stepcounter{inclvdue}\hyperlink{Movimenti del Ragno}{Movimenti del Ragno}, Non Comune, 2\\
\stepcounter{inclvdue}\hyperlink{Passo Velato}{Passo Velato}, Non Comune, 2\\
\stepcounter{inclvdue}\hyperlink{Pelle di Corteccia}{Pelle di Corteccia}, Comune, 2\\
\stepcounter{inclvdue}\hyperlink{Piroesperto}{Piroesperto}, Non Comune, 2\\
\stepcounter{inclvdue}\hyperlink{Preghiera di Guarigione}{Preghiera di Guarigione}, Comune, 2\\
\stepcounter{inclvdue}\hyperlink{Presagio}{Presagio}, Comune, 2\\
\stepcounter{inclvdue}\hyperlink{Protezione dai Veleni}{Protezione dai Veleni}, Non Comune, 2\\
\stepcounter{inclvdue}\hyperlink{Punizione Marchiante}{Punizione Marchiante}, Comune, 2\\
\stepcounter{inclvdue}\hyperlink{Raggio mortale}{Raggio mortale}, Raro, 2\\
\stepcounter{inclvdue}\hyperlink{Raggio Rovente}{Raggio Rovente}, Comune, 2\\
\stepcounter{inclvdue}\hyperlink{Ragnatela}{Ragnatela}, Comune, 2\\
\stepcounter{inclvdue}\hyperlink{Rimuovi Malattia}{Rimuovi Malattia}, Comune, 2\\
\stepcounter{inclvdue}\hyperlink{Riposo Inviolato}{Riposo Inviolato}, Non Comune, 2\\
\stepcounter{inclvdue}\hyperlink{Riscaldare il Metallo}{Riscaldare il Metallo}, Non Comune, 2\\
\stepcounter{inclvdue}\hyperlink{Ristorare Inferiore}{Ristorare Inferiore}, Comune, 2\\
\stepcounter{inclvdue}\hyperlink{Scassinare}{Scassinare}, Comune, 2\\
\stepcounter{inclvdue}\hyperlink{Scopri Piante}{Scopri Piante}, Non Comune, 2\\
\stepcounter{inclvdue}\hyperlink{Scopri Trappole}{Scopri Trappole}, Comune, 2\\
\stepcounter{inclvdue}\hyperlink{Scurovisione}{Scurovisione}, Comune, 2\\
\stepcounter{inclvdue}\hyperlink{Serratura Magica}{Serratura Magica}, Comune, 2\\
\stepcounter{inclvdue}\hyperlink{Sfera Infuocata}{Sfera Infuocata}, Comune, 2\\
\stepcounter{inclvdue}\hyperlink{Sfocatura}{Sfocatura}, Comune, 2\\
\stepcounter{inclvdue}\hyperlink{Silenzio}{Silenzio}, Comune, 2\\
\stepcounter{inclvdue}\hyperlink{Sonnellino}{Sonnellino}, Leggendario, 2\\
\stepcounter{inclvdue}\hyperlink{Suggestione}{Suggestione}, Raro, 2\\
\stepcounter{inclvdue}\hyperlink{Trucco della Corda}{Trucco della Corda}, Comune, 2\\
\stepcounter{inclvdue}\hyperlink{Vedere l'invisibile}{Vedere l'invisibile}, Comune, 2\\
\stepcounter{inclvdue}\hyperlink{Vincolo di Interdizione}{Vincolo di Interdizione}, Comune, 2\\
\stepcounter{inclvdue}\hyperlink{Zona di Verità}{Zona di Verità}, Non Comune, 2\\

\medskip Totale incantesimi: \theinclvdue\\

\textbf{Livello 3}  \newcounter{inclvtre}

\stepcounter{inclvtre}\hyperlink{Animare Morti}{Animare Morti}, Comune, 3\\
\stepcounter{inclvtre}\hyperlink{Anti-Individuazione}{Anti-Individuazione}, Non Comune, 3\\
\stepcounter{inclvtre}\hyperlink{Bastoni in Serpenti}{Bastoni in Serpenti}, Non Comune, 3\\
\stepcounter{inclvtre}\hyperlink{Benedizione di Ledyal}{Benedizione di Ledyal}, Molto Raro, 3\\
\stepcounter{inclvtre}\hyperlink{Benedizione della Vita}{Benedizione della Vita}, Raro, 3\\
\stepcounter{inclvtre}\hyperlink{Benedizione Suprema}{Benedizione Suprema}, Raro, 3\\
\stepcounter{inclvtre}\hyperlink{Camminare sull'Acqua}{Camminare sull'Acqua}, Comune, 3\\
\stepcounter{inclvtre}\hyperlink{Capanna}{Capanna}, Non Comune, 3\\
\stepcounter{inclvtre}\hyperlink{Cecità/Sordità Avanzata}{Cecità/Sordità Avanzata}, Non Comune, 3\\
\stepcounter{inclvtre}\hyperlink{Cerchio Magico}{Cerchio Magico}, Comune, 3\\
\stepcounter{inclvtre}\hyperlink{Cerchio d'Invisibilità}{Cerchio d'Invisibilità}, Non Comune, 3\\
\stepcounter{inclvtre}\hyperlink{Chiaroveggenza}{Chiaroveggenza}, Comune, 3\\
\stepcounter{inclvtre}\hyperlink{Controincantesimo}{Controincantesimo}, Comune, 3\\
\stepcounter{inclvtre}\hyperlink{Creare Cibo e Acqua}{Creare Cibo e Acqua}, Comune, 3\\
\stepcounter{inclvtre}\hyperlink{Crescita Vegetale}{Crescita Vegetale}, Non Comune, 3\\
\stepcounter{inclvtre}\hyperlink{Destriero Fantasma}{Destriero Fantasma}, Comune, 3\\
\stepcounter{inclvtre}\hyperlink{Dissolvi Magie}{Dissolvi Magie}, Comune, 3\\
\stepcounter{inclvtre}\hyperlink{Distruggere nonmorto}{Distruggere Nonmorto}, Non Comune, 3\\
\stepcounter{inclvtre}\hyperlink{Evoca Animali}{Evoca Animali}, Non Comune, 3\\
\stepcounter{inclvtre}\hyperlink{Forma Gassosa}{Forma Gassosa}, Non Comune, 3\\
\stepcounter{inclvtre}\hyperlink{Fulmine}{Fulmine}, Comune, 3\\
\stepcounter{inclvtre}\hyperlink{Glifo di Interdizione}{Glifo di Interdizione}, Comune, 3\\
\stepcounter{inclvtre}\hyperlink{Immagine Maggiore}{Immagine Maggiore}, Comune, 3\\
\stepcounter{inclvtre}\hyperlink{Intermittenza}{Intermittenza}, Non Comune, 3\\
\stepcounter{inclvtre}\hyperlink{Inviare}{Inviare}, Comune, 3\\
\stepcounter{inclvtre}\hyperlink{Invocare il Fulmine}{Invocare il Fulmine}, Comune, 3\\
\stepcounter{inclvtre}\hyperlink{lentezza}{Lentezza}, Non Comune, 3\\
\stepcounter{inclvtre}\hyperlink{Lingue}{Lingue}, Comune, 3\\
\stepcounter{inclvtre}\hyperlink{Luce Diurna}{Luce Diurna}, Comune, 3\\
\stepcounter{inclvtre}\hyperlink{Morte Apparente}{Morte Apparente}, Non Comune, 3\\
\stepcounter{inclvtre}\hyperlink{Muro di Vento}{Muro di Vento}, Non Comune, 3\\
\stepcounter{inclvtre}\hyperlink{Palla di Fuoco}{Palla di Fuoco}, Comune, 3\\
\stepcounter{inclvtre}\hyperlink{Parlare con i Morti}{Parlare con i Morti}, Raro, 3\\
\stepcounter{inclvtre}\hyperlink{Parlare con le Piante}{Parlare con le Piante}, Raro, 3\\
\stepcounter{inclvtre}\hyperlink{Paura}{Paura}, Non Comune, 3\\
\stepcounter{inclvtre}\hyperlink{Preghiera}{Preghiera}, Non Comune, 3\\
\stepcounter{inclvtre}\hyperlink{Protezione dall'Energia}{Protezione dall'Energia}, Comune, 3\\
\stepcounter{inclvtre}\hyperlink{Rimuovi Maledizione}{Rimuovi Maledizione}, Comune, 3\\
\stepcounter{inclvtre}\hyperlink{Scagliare Maledizione}{Scagliare Maledizione}, Non Comune, 3\\
\stepcounter{inclvtre}\hyperlink{Tempesta di Nevischio}{Tempesta di Nevischio}, Molto Raro, 3\\
\stepcounter{inclvtre}\hyperlink{Tocco Vampirico}{Tocco Vampirico}, Raro, 3\\
\stepcounter{inclvtre}\hyperlink{Trama Ipnotica}{Trama Ipnotica}, Comune, 3\\
\stepcounter{inclvtre}\hyperlink{Uno con la pietra}{Uno con la pietra}, Comune, 3\\
\stepcounter{inclvtre}\hyperlink{Velocità}{Velocità}, Non Comune, 3\\
\stepcounter{inclvtre}\hyperlink{Volare}{Volare}, Comune, 3\\

\medskip Totale incantesimi: \theinclvtre\\

\textbf{Livello 4}   \newcounter{inclvquattro}

\stepcounter{inclvquattro}\hyperlink{Allucinazione Mortale}{Allucinazione Mortale}, Non Comune, 4\\
\stepcounter{inclvquattro}\hyperlink{Blocca Persona Avanzato}{Blocca Persona Avanzato}, Non Comune, 4\\
\stepcounter{inclvquattro}\hyperlink{Camminare nell'aria}{Camminare nell'aria}, Non Comune, 4\\
\stepcounter{inclvquattro}\hyperlink{Compulsione}{Compulsione}, Non Comune, 4\\
\stepcounter{inclvquattro}\hyperlink{Confusione}{Confusione}, Comune, 4\\
\stepcounter{inclvquattro}\hyperlink{Controllare Acqua}{Controllare Acqua}, Comune, (4)\\
\stepcounter{inclvquattro}\hyperlink{Dominare Bestie}{Dominare Bestie}, Vario, 4\\
\stepcounter{inclvquattro}\hyperlink{Esilio}{Esilio}, Comune, 4\\
\stepcounter{inclvquattro}\hyperlink{Fabbricare}{Fabbricare}, Comune, 4\\
\stepcounter{inclvquattro}\hyperlink{Inaridire}{Inaridire}, Non Comune, 4\\
\stepcounter{inclvquattro}\hyperlink{Insetto Gigante}{Insetto Gigante}, Non Comune, 4\\
\stepcounter{inclvquattro}\hyperlink{Interdizione alla Morte}{Interdizione alla Morte}, Non Comune, 4\\
\stepcounter{inclvquattro}\hyperlink{Invisibilità Superiore}{Invisibilità Superiore}, Non Comune, 4\\
\stepcounter{inclvquattro}\hyperlink{Libertà di Movimento}{Libertà di Movimento}, Comune, 4\\
\stepcounter{inclvquattro}\hyperlink{Localizza Creatura}{Localizza Creatura}, Comune, 4\\
\stepcounter{inclvquattro}\hyperlink{Metamorfosi}{Metamorfosi}, Comune, 4\\
\stepcounter{inclvquattro}\hyperlink{Muro di Fuoco}{Muro di Fuoco}, Non Comune, 4\\
\stepcounter{inclvquattro}\hyperlink{Occhio Arcano}{Occhio Arcano}, Comune, 4\\
\stepcounter{inclvquattro}\hyperlink{Pelle di Pietra}{Pelle di Pietra}, Non Comune, 4\\
\stepcounter{inclvquattro}\hyperlink{Porta Dimensionale}{Porta Dimensionale}, Comune, 4\\
\stepcounter{inclvquattro}\hyperlink{Profumo di Atherim}{Profumo di Atherim}, Raro, 4\\
\stepcounter{inclvquattro}\hyperlink{Santuario Privato}{Santuario Privato}, Molto Raro, 4\\
\stepcounter{inclvquattro}\hyperlink{Scolpire Pietra}{Scolpire Pietra}, Comune, 4\\
\stepcounter{inclvquattro}\hyperlink{Scrigno Segreto}{Scrigno Segreto}, Raro, 4\\
\stepcounter{inclvquattro}\hyperlink{Segugio Fedele}{Segugio Fedele}, Raro, 4\\
\stepcounter{inclvquattro}\hyperlink{Tentacoli Neri}{Tentacoli Neri}, Non Comune, 4\\
\stepcounter{inclvquattro}\hyperlink{Terreno Illusorio}{Terreno Illusorio}, Non Comune, 4\\
\stepcounter{inclvquattro}\hyperlink{Vigore}{Vigore}, Raro, 4\\

\medskip Totale incantesimi: \theinclvquattro\\

\textbf{Livello 5} \newcounter{inclvcinque}

\stepcounter{inclvcinque}\hyperlink{Animare Oggetti}{Animare Oggetti}, Comune, 5\\
\stepcounter{inclvcinque}\hyperlink{Barriera Antianimali}{Barriera Antianimali}, Raro, 5\\
\stepcounter{inclvcinque}\hyperlink{Cerchio di Teletrasporto}{Cerchio di Teletrasporto}, Non Comune, 5\\
\stepcounter{inclvcinque}\hyperlink{Colpo Infuocato}{Colpo Infuocato}, Comune, 5\\
\stepcounter{inclvcinque}\hyperlink{Comunione}{Comunione}, Raro, 5\\
\stepcounter{inclvcinque}\hyperlink{Comunione con la Natura}{Comunione con la Natura}, Molto Raro, 5\\
\stepcounter{inclvcinque}\hyperlink{Cono di Freddo}{Cono di Freddo}, Comune, 5\\
\stepcounter{inclvcinque}\hyperlink{Conoscenza delle Leggende}{Conoscenza delle Leggende}, Comune, 5\\
\stepcounter{inclvcinque}\hyperlink{Contagio}{Contagio}, Non Comune, 5\\
\stepcounter{inclvcinque}\hyperlink{Costrizione}{Costrizione}, Raro, 5\\
\stepcounter{inclvcinque}\hyperlink{Creazione}{Creazione}, Raro, 5\\
\stepcounter{inclvcinque}\hyperlink{Dissolvi il Bene e il Male}{Dissolvi il Bene e il Male}, Raro, 5\\
\stepcounter{inclvcinque}\hyperlink{Dissolvi Magie Avanzato}{Dissolvi Magie Avanzato}, Raro, 5\\
\stepcounter{inclvcinque}\hyperlink{Dominare Persone}{Dominare Persone}, Non Comune, 5\\
\stepcounter{inclvcinque}\hyperlink{Fuorviare}{Fuorviare}, Non Comune, 5\\
\stepcounter{inclvcinque}\hyperlink{Gragnola di Marroni di Kyrin}{Gragnola di Marroni di Kyrin}, Molto Raro, 5\\
\stepcounter{inclvcinque}\hyperlink{Guscio Anti-Vita}{Guscio Anti-Vita}, Non Comune, 5\\
\stepcounter{inclvcinque}\hyperlink{Legame Telepatico}{Legame Telepatico}, Raro, 5\\
\stepcounter{inclvcinque}\hyperlink{Mano Arcana}{Mano Arcana}, Non Comune, 5\\
\stepcounter{inclvcinque}\hyperlink{Modificare Memoria}{Modificare Memoria}, Molto Raro, 5\\
\stepcounter{inclvcinque}\hyperlink{Muro di Forza}{Muro di Forza}, Comune, 5\\
\stepcounter{inclvcinque}\hyperlink{Muro di Pietra}{Muro di Pietra}, Comune, 5\\
\stepcounter{inclvcinque}\hyperlink{Passa Porta}{Passa Porta}, Non Comune, 5\\
\stepcounter{inclvcinque}\hyperlink{Piaga degli Insetti}{Piaga degli Insetti}, Raro, 5\\
\stepcounter{inclvcinque}\hyperlink{Pietra in Fango - Fango in Pietra}{Pietra in Fango - Fango in Pietra}, Non Comune - Molto Raro, 5\\
\stepcounter{inclvcinque}\hyperlink{Reincarnazione}{Reincarnazione}, Raro, 5\\
\stepcounter{inclvcinque}\hyperlink{Ristorare Superiore}{Ristorare Superiore}, Non Comune, 5\\
\stepcounter{inclvcinque}\hyperlink{Risveglio}{Risveglio}, Raro, 5\\
\stepcounter{inclvcinque}\hyperlink{Santificare}{Santificare}, Raro, 5\\
\stepcounter{inclvcinque}\hyperlink{Scrutare}{Scrutare}, Raro, 5\\
\stepcounter{inclvcinque}\hyperlink{Sembrare}{Sembrare}, Non Comune, 5\\
\stepcounter{inclvcinque}\hyperlink{Sogno}{Sogno}, Non Comune, 5\\
\stepcounter{inclvcinque}\hyperlink{Telecinesi}{Telecinesi}, Non Comune, 5\\
\stepcounter{inclvcinque}\hyperlink{Traslazione Arborea}{Traslazione Arborea}, Raro, 5\\

\medskip Totale incantesimi: \theinclvcinque\\

\textbf{Livello 6} \newcounter{inclvsei}

\stepcounter{inclvsei}\hyperlink{Bagliore Solare}{Bagliore Solare}, Non Comune, 6\\
\stepcounter{inclvsei}\hyperlink{Banchetto degli Eroi}{Banchetto degli Eroi}, Non Comune, 6\\
\stepcounter{inclvsei}\hyperlink{Barriera di Lame}{Barriera di Lame}, Comune, 6\\
\stepcounter{inclvsei}\hyperlink{Camminare nel Vento}{Camminare nel Vento}, Non Comune, 6\\
\stepcounter{inclvsei}\hyperlink{Carne in Pietra - Pietra in Carne}{Carne in Pietra - Pietra in Carne}, Non Comune - Raro, 6\\
\stepcounter{inclvsei}\hyperlink{Cerchio di Morte}{Cerchio di Morte}, Molto Raro, 6\\
\stepcounter{inclvsei}\hyperlink{Contingenza}{Contingenza}, Comune, 6\\
\stepcounter{inclvsei}\hyperlink{Creare Non Morti}{Creare Non Morti}, Non Comune, 6\\
\stepcounter{inclvsei}\hyperlink{Disintegrazione}{Disintegrazione}, Non Comune, 6\\
\stepcounter{inclvsei}\hyperlink{Dito della Morte}{Dito della Morte}, Raro, 6\\
\stepcounter{inclvsei}\hyperlink{Divinazione}{Divinazione}, Comune, 6\\
\stepcounter{inclvsei}\hyperlink{Evocazioni Istantanee}{Evocazioni Istantanee}, Raro, 6\\
\stepcounter{inclvsei}\hyperlink{Ferire}{Ferire}, Non Comune, 6\\
\stepcounter{inclvsei}\hyperlink{Fulmine a catena}{Fulmine a catena}, Raro, 6\\
\stepcounter{inclvsei}\hyperlink{Giara Magica}{Giara Magica}, Molto Raro, 6\\
\stepcounter{inclvsei}\hyperlink{Globo di Invulnerabilità}{Globo di Invulnerabilità}, Comune, 6\\
\stepcounter{inclvsei}\hyperlink{Guarigione}{Guarigione}, Raro, 6\\
\stepcounter{inclvsei}\hyperlink{Illusione Programmata}{Illusione Programmata}, Non Comune, 6\\
\stepcounter{inclvsei}\hyperlink{Muovere il Terreno}{Muovere il Terreno}, Non Comune, 6\\
\stepcounter{inclvsei}\hyperlink{Muro di Ghiaccio}{Muro di Ghiaccio}, Comune, (6)\\
\stepcounter{inclvsei}\hyperlink{Muro di Spine}{Muro di Spine}, Non Comune, 6\\
\stepcounter{inclvsei}\hyperlink{Parlare con le Creature}{Parlare con le Creature}, Raro, 6\\
\stepcounter{inclvsei}\hyperlink{Parola del Ritiro}{Parola del Ritiro}, Raro, 6\\
\stepcounter{inclvsei}\hyperlink{Pietre Parlanti}{Pietre Parlanti}, Raro, 6\\
\stepcounter{inclvsei}\hyperlink{Proibizione}{Proibizione}, Non Comune, 6\\
\stepcounter{inclvsei}\hyperlink{Scopri il Percorso}{Scopri il Percorso}, Non Comune, 6\\
\stepcounter{inclvsei}\hyperlink{Sfera Congelante}{Sfera Congelante}, Raro, 6\\
\stepcounter{inclvsei}\hyperlink{Sguardo Penetrante}{Sguardo Penetrante}, Molto Raro, 6\\
\stepcounter{inclvsei}\hyperlink{Suggestione di Massa}{Suggestione di Massa}, Molto Raro, 6\\
\stepcounter{inclvsei}\hyperlink{Trasformazione Furiosa di Restser}{Trasformazione Furiosa di Restser}, Molto Raro, 6\\
\stepcounter{inclvsei}\hyperlink{Trasporto Vegetale}{Trasporto Vegetale}, Molto Raro, 6\\
\stepcounter{inclvsei}\hyperlink{Vigilanza e Interdizione}{Vigilanza e Interdizione}, Non Comune, 6\\
\stepcounter{inclvsei}\hyperlink{Visione del Vero}{Visione del Vero}, Raro, 6\\

\medskip Totale incantesimi: \theinclvsei\\

\textbf{Livello 7}\newcounter{inclvsette}

\stepcounter{inclvsette}\hyperlink{Celare}{Celare}, Raro, 7\\
\stepcounter{inclvsette}\hyperlink{Desiderio limitato}{Desiderio limitato}, Raro, 7\\
\stepcounter{inclvsette}\hyperlink{Forma Eterea}{Forma Eterea}, Raro, 7\\
\stepcounter{inclvsette}\hyperlink{Immagine Proiettata}{Immagine Proiettata}, Non Comune, 7\\
\stepcounter{inclvsette}\hyperlink{Inversione della Gravità}{Inversione della Gravità}, Raro, 7\\
\stepcounter{inclvsette}\hyperlink{Miraggio Arcano}{Miraggio Arcano}, Raro, 7\\
\stepcounter{inclvsette}\hyperlink{Palla di Fuoco Ritardata}{Palla di Fuoco Ritardata}, Raro, 7\\
\stepcounter{inclvsette}\hyperlink{Parola Divina}{Parola Divina}, Molto Raro, 7\\
\stepcounter{inclvsette}\hyperlink{Reggia Meravigliosa}{Reggia Meravigliosa}, Leggendario, 7\\
\stepcounter{inclvsette}\hyperlink{Rigenerazione}{Rigenerazione}, Leggendario, 7\\
\stepcounter{inclvsette}\hyperlink{Simbolo}{Simbolo}, Non Comune, 7\\
\stepcounter{inclvsette}\hyperlink{Spada Arcana}{Spada Arcana}, Raro, 7\\
\stepcounter{inclvsette}\hyperlink{Spruzzo Prismatico}{Spruzzo Prismatico}, Raro, 7\\
\stepcounter{inclvsette}\hyperlink{Statua}{Statua}, Raro, 7\\
\stepcounter{inclvsette}\hyperlink{Teletrasporto}{Teletrasporto}, Comune, 7\\
\stepcounter{inclvsette}\hyperlink{Tempesta di Fuoco}{Tempesta di Fuoco}, Raro, 7\\

\medskip Totale incantesimi: \theinclvsette\\

\textbf{Livello 8}\newcounter{inclvotto}

\stepcounter{inclvotto}\hyperlink{Antipatia/Simpatia}{Antipatia/Simpatia}, Raro, 8\\
\stepcounter{inclvotto}\hyperlink{Aura Sacra}{Aura Sacra}, Comune, 8\\
\stepcounter{inclvotto}\hyperlink{Benedizioni di Efrem}{Benedizioni di Efrem}, Raro, 8\\
\stepcounter{inclvotto}\hyperlink{Campo Anti-Magia}{Campo Anti-Magia}, Raro, 8\\
\stepcounter{inclvotto}\hyperlink{Confusione Contagiosa}{Confusione Contagiosa}, Molto Raro, 8\\
\stepcounter{inclvotto}\hyperlink{Controllare Tempo Atmosferico}{Controllare Tempo Atmosferico}, Raro, 8\\
\stepcounter{inclvotto}\hyperlink{Danza Irresistibile}{Danza Irresistibile}, Leggendario, 8\\
\stepcounter{inclvotto}\hyperlink{Dominare Mostri}{Dominare Mostri}, Non Comune, 8\\
\stepcounter{inclvotto}\hyperlink{Esplosione Solare}{Esplosione Solare}, Raro, 8\\
\stepcounter{inclvotto}\hyperlink{Gabbia di Forza}{Gabbia di Forza}, Raro, 8\\
\stepcounter{inclvotto}\hyperlink{Labirinto}{Labirinto}, Raro, 8\\
\stepcounter{inclvotto}\hyperlink{Loquacità}{Loquacità}, Raro, 8\\
\stepcounter{inclvotto}\hyperlink{Nube Incendiaria}{Nube Incendiaria}, Raro, 8\\
\stepcounter{inclvotto}\hyperlink{Parola del Potere Stordire}{Parola del Potere Stordire}, Non Comune, 8\\
\stepcounter{inclvotto}\hyperlink{Regressione Mentale}{Regressione Mentale}, Raro, 8\\
\stepcounter{inclvotto}\hyperlink{Scudo Mentale}{Scudo Mentale}, Non Comune, 8\\
\stepcounter{inclvotto}\hyperlink{Terremoto}{Terremoto}, Molto Raro, 8\\

\medskip Totale incantesimi: \theinclvotto\\

\textbf{Livello 9} \newcounter{inclvnove}

\stepcounter{inclvnove}\hyperlink{Desiderio}{Desiderio}, Non Comune, 9\\
\stepcounter{inclvnove}\hyperlink{Fatale}{Fatale}, Raro, 9\\
\stepcounter{inclvnove}\hyperlink{Fermare il Tempo}{Fermare il Tempo}, Molto Raro, 9\\
\stepcounter{inclvnove}\hyperlink{Guarigione di Massa}{Guarigione di Massa}, Leggendario, 9\\
\stepcounter{inclvnove}\hyperlink{Imprigionare}{Imprigionare}, Raro, 9\\
\stepcounter{inclvnove}\hyperlink{Metamorfosi Pura}{Metamorfosi Pura}, Raro, 9\\
\stepcounter{inclvnove}\hyperlink{Muro Prismatico}{Muro Prismatico}, Raro, (9)\\
\stepcounter{inclvnove}\hyperlink{Parola del Potere Uccidere}{Parola del Potere Uccidere}, Raro, 9\\
\stepcounter{inclvnove}\hyperlink{Previsione}{Previsione}, Non Comune, 9\\
\stepcounter{inclvnove}\hyperlink{Trasformazione}{Trasformazione}, Raro, 9\\

\medskip Totale incantesimi: \theinclvnove

}}

\end{multicols}

\vfill

\begin{center}
	\includegraphics[width=0.5\linewidth]{immagini/the-discovery-of-witchcraft.png}

	\emph{"The Discoverie of Witchcraft' by Reginald Scot, 16 secolo }
\end{center}

%\titlespacing*{\subsubsection}{0pt}{*1}{*1}

\pagebreak

