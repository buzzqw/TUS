\section{Oggetti Maledetti}\index{Oggetti Maledetti}

\begin{enfasi}{Quando un empio maledice l'avversario, maledice se stesso. (Siracide)

\medskip

Se maledici una persona ci saranno due fosse. (Proverbio giapponese)}
\end{enfasi}

\begin{multicols}{2}

\label{oggetti-maledetti}

Gli oggetti maledetti sono oggetti magici dotati di un'influenza potenzialmente negativa sul personaggio.

Gli oggetti maledetti non sono quasi mai realizzati intenzionalmente, piuttosto sono il risultato di un lavoro mal riuscito, di artigiani con poca esperienza o della mancanza di componenti adeguati o patti non rispettati con qualche Patrono.

Il Narratore può chiedere una prova di Arcana con una DC pari a 10+giorni impiegati per costruire l'oggetto magico in caso di oggetti particolarmente complessi o se ci siano state situazioni problematiche nella creazione e se la prova fallisce tirate sulla tabella seguente per determinare il tipo di maledizione che l'oggetto possiede.

Una maledizione può manifestarsi anche a seguito dalle influenze negative od emozionali estreme che coinvolgono un oggetto.

\medskip

\textbf{Maledizioni Comuni degli Oggetti}

\medskip

\noindent\begin{tabularx}{\linewidth}{ll}
	\toprule
\rowcolor{gray!20}\textbf{\%} & \textbf{Maledizione}\\
\toprule
01-15 & Inganno\\
\rowcolor{gray!20}16-40 & Effetto o Bersaglio Opposto\\
41-50 & Funzionamento Discontinuo\\
\rowcolor{gray!20}51-65 & Requisito\\
66-90 & Inconveniente\\
\rowcolor{gray!20}91-100& Effetto completamente diverso
\end{tabularx}

\medskip

Gli oggetti maledetti sono \hypertarget{oggettimaledettiid}{\textbf{identificati}}\label{oggettimaledettiid} come qualsiasi altro oggetto magico con una sola eccezione: a meno che non la prova di Arcana per identificare l'oggetto non superi 35 o l'incantesimo \hyperlink{Identificare}{Identificare} sia lanciato con una Prova di Magia ed ottenga un critico magico la maledizione non viene individuata. Se la prova è sotto 35 o senza critico magico tutto quello che viene rivelato è l'originale scopo dell'oggetto magico.

Se si sa che l'oggetto è maledetto, la natura della maledizione può essere determinata usando la DC \hyperlink{identificareom}{standard} per identificare l'oggetto.

\begin{center}
\includegraphics[width=0.75\linewidth]{immagini/vasobasano.png}

\emph{Vaso di Basano. Questo vaso è stato realizzato nella seconda metà del XV secolo ed è realizzato in argento. DC 35}
\end{center}

\begin{narratore}[Maledizioni e perché]
Una maledizione è sempre un \emph{inconveniente} particolare, che non si usa a caso. Ragionate attentamente sugli oggetti maledetti che farete trovare ai personaggi perché vi chiederanno molte informazioni e dovrete essere pronti.

Non c'è bisogno che la maledizione sia eccessiva e limitante può essere benissimo ridicola o particolare, fate in modo che sia caratterizzante. Il personaggio non deve sentirsi (tranne se lo volete) condannato in eterno, sfruttate l'occasione per costruire nuove avventure e spirito di gruppo.
\end{narratore}

\subsection{Rimuovere Oggetti Maledetti}\index{Rimuovere Oggetti Maledetti}

Mentre alcuni oggetti maledetti possono essere semplicemente posati, altri esercitano una forte compulsione sul possessore a tenerli con sé, a qualsiasi costo. Altri riappaiono anche se abbandonati o è impossibile gettarli via.

Questi oggetti possono essere rimossi solo dopo che sul personaggio o l'oggetto viene lanciato l'incantesimo Rimuovi Maledizione.

L'incantesimo \hyperlink{Dissolvi Magie}{Dissolvi Magia} è inutile per rimuovere una maledizione, solo un \hyperlink{Dissolvi Magie Avanzato}{Dissolvi Magie Avanzato} con 3 Successi Magici Critici può essere sufficiente.

Se l'oggetto è stato maledetto tramite l'incantesimo \hyperlink{Scagliare Maledizione}{Scagliare Maledizione}, o comunque il Narratore decide che l'oggetto ha una maledizione particolare allora si deve effettuare una prova di \hyperlink{contrastareincantesimi}{contrasto} (pag. \pageref{contrastareincantesimi})tra chi lancia Rimuovi Maledizione e la DC della maledizione dell'oggetto.

Se la prova di contrasto ha successo allora l'oggetto può essere rimosso nel round successivo e la maledizione rimane e colpisce nuovamente se l'oggetto viene usato/indossato un'altra volta.

Ogni oggetto maledetto ha un proprio metodo per essere distrutto, dall'essere gettato in un vulcano attivo, ad essere colpito dal martello del dio del Tuono (o Patrono...) oppure divorato da un Verme colossale delle sabbie se non colpito dal soffio di un drago rosso e un drago d'oro contemporaneamente...

Se la DC della maledizione non è indicata è sufficiente il lancio dell'incantesimo Rimuovi Maledizioni.

\subsection{Effetti Comuni degli Oggetti Maledetti}

Gli effetti più comuni degli oggetti maledetti sono i seguenti, il Narratore può inventare nuovi effetti particolari per specifici oggetti maledetti.

\subsubsection{Inganno}

Chi utilizza l'oggetto continua a credere che sia ciò che sembra a prima vista, ma in realtà non ha alcun potere, a parte quello di ingannare. Chi lo usa è mentalmente spinto a credere che funzioni e non può essere convinto del contrario se non con l'uso di Rimuovi maledizione

\medskip

\begin{center}
\includegraphics[width=0.70\linewidth]{immagini/mirror.png}

\emph{The mirror in The Myrtles Plantation. DC 28}
\end{center}

\subsubsection{Effetto o Bersaglio Opposto}

Questi oggetti maledetti tendono ad avere dei difetti di funzionamento che in alcuni casi generano effetti diametralmente opposti a quelli desiderati dal loro creatore, mentre in altri casi tendono a colpire chi li utilizza invece di qualcun altro.

La categoria degli oggetti magici dagli effetti opposti include anche le armi che infliggono penalità ai Tiri per Colpire ed ai danni, invece che bonus.

La cosa più interessante è che questi oggetti potrebbero anche non essere uno svantaggio per chi li possiede.

Visto che un personaggio non dovrebbe sapere immediatamente quale sia il bonus od effetto di un oggetto magico, non dovrebbe venire a conoscenza nemmeno della natura della sua maledizione. Una volta che lo verrà a sapere per liberarsi dall'oggetto sarà necessario l'Incantesimo Rimuovi maledizione.

\subsection{Funzionamento Discontinuo}

Gli oggetti discontinui funzionano esattamente come dovrebbero, quando funzionano. Stabilite se l'oggetto è Inaffidabile, Condizionato oppure Incontrollabile.

\medskip
\subsubsection{Inaffidabile}

Ogni volta che l'oggetto viene attivato, c'è una probabilità del 5\% che non funzioni.

\subsubsection{Condizionato}

Questo oggetto funziona solo in determinate situazioni. Per determinare quali siano, scegliete una condizione di attivazione o consultate la tabella poco sotto.

\subsubsection{Incontrollabile}

Un oggetto incontrollabile tende ad attivarsi casualmente. Tirare un d\% ogni giorno. Con un risultato di 01--05 l'oggetto si attiva spontaneamente in un certo momento del giorno.

\medskip

\noindent\begin{tabularx}{\linewidth}{lX}
	\toprule
\rowcolor{gray!20}\textbf{\%} & \textbf{Situazione}\\
\toprule
01-03 & Temperatura sotto lo zero\\
\rowcolor{gray!20}04-05 & Temperatura sopra lo zero\\
06-10 & Durante il giorno\\
\rowcolor{gray!20}11-15 & Durante la notte\\
16-20 & Esposto alla luce solare\\
\rowcolor{gray!20}21-25 & In assenza di luce solare\\
26-34 & Sott'acqua\\
\rowcolor{gray!20}35-37 & Fuori dall'acqua\\
38-45 & Sottoterra\\
\rowcolor{gray!20}46-55 & In superficie\\
56-60 & Entro 3 metri da un tipo di creatura\\
\rowcolor{gray!20}61-64 & Entro 3 metri da una razza o tipo di creatura\\
65-72 & Entro 3 metri da un incantatore\\
\rowcolor{gray!20}73-80 & Entro 3 metri da un Seguace o Devoto di un Patrono specifico\\
81-85 & Nelle mani di un personaggio non incantatore\\
\rowcolor{gray!20}86-90 & Nelle mani di un personaggio incantatore\\
91-95 & Nelle mani di una creatura con particolare Tratto\\
\rowcolor{gray!20}96& Nelle mani di una creatura di un particolare genere\\
97-99 & Nei giorni non sacri o durante particolari ricorrenze astronomiche\\
\rowcolor{gray!20}100 & A più di 150 km da un determinato luogo
\end{tabularx}

\subsection{Requisito}

Alcuni oggetti hanno requisiti molto più difficili da soddisfare perché funzionino. Per far funzionare l'oggetto in questione, potrebbe essere necessario soddisfare una delle seguenti condizioni:

\begin{itemize}[leftmargin=*] \setlength{\itemsep}{0pt}
\item Il personaggio deve mangiare il doppio del normale.
\item Il personaggio deve dormire il doppio del normale.
\item Il personaggio deve compiere almeno una missione specifica.
\item Il personaggio deve sacrificare (distruggere) un valore pari a 100 mo di oggetti o materiali preziosi al giorno.
\item Il personaggio deve giurare lealtà ad un nobile in particolare o alla sua famiglia.
\item Il personaggio deve abbandonare tutti gli altri oggetti magici.
\item Il personaggio deve essere un Seguace o Devoto di uno specifico Patrono
\item Il personaggio deve avere un numero minimo di gradi in una particolare competenza.
\item Il personaggio deve sacrificare parte della propria energia vitale (1 punto di Costituzione permanente) la prima volta che usa l'oggetto.
\item L'oggetto deve essere purificato con l'Acqua santa di uno specifico Patrono ogni giorno.
\item L'oggetto deve essere bagnato in almeno mezzo litro di sangue (animale o umanoide) al giorno.
\item L'oggetto deve essere usato per uccidere una creatura vivente al giorno.
\item L'oggetto deve essere usato almeno una volta al giorno, o smette di funzionare per il suo attuale possessore.
\item Quando viene brandito, l'oggetto deve spillare sangue (solo armi). Non può essere messo da parte o cambiato con un altro oggetto finché non ha messo a segno un colpo.
\end{itemize}

I requisiti dipendono dalla convenienza dell'oggetto che non dovrebbero mai essere determinati a caso. Un oggetto intelligente con un requisito spesso impone il proprio requisito grazie alla sua personalità.

Se il requisito non viene soddisfatto, l'oggetto smette di funzionare. Se invece viene soddisfatto, di solito l'oggetto funziona per un giorno intero prima di dover di nuovo soddisfare il requisito (anche se alcuni requisiti vanno soddisfatti una volta sola, altri una volta al mese e altri ancora in continuazione).

\subsection{Inconveniente}

Gli oggetti che hanno degli inconvenienti hanno solitamente degli effetti positivi su chi li usa, ma hanno anche degli aspetti negativi. Anche se a volte gli inconvenienti vengono alla luce solo quando gli oggetti sono utilizzati (o tenuti in mano, nel caso di oggetti come le armi), di solito rimangono presenti fino a quando il personaggio non si libera dell'oggetto in questione.

A meno che non sia indicato diversamente, gli inconvenienti rimangono attivi per tutto il tempo in cui l'oggetto rimane in possesso del personaggio. La DC dei Tiro Salvezza per evitare questi effetti è pari a 10 + DC della maledizione (se non avete stabilito la difficoltà impostate il Tiro Salvezza, solitamente su Volontà, a DC 25)

\medskip

\begin{narratore}[Creativi ma non punire]
L'elenco è di esempio per poter generare casualmente degli effetti sul possessore dell'oggetto. Prendete spunto e siate creativi!Non fate però che una maledizione renda impossibile giocare il personaggio piuttosto deve essere vissuta come l'occasione per provare, fare, qualcosa di diverso. Non gettate mai un oggetto maledetto a caso nel mucchio dei tesori, pensate sempre cosa potrà accadere e quali conseguenze si genereranno. Un oggetto maledetto richiede sempre un alto livello di attenzione e pianificazione da parte del Narratore\end{narratore}

\medskip

\end{multicols}

\vfill

\begin{center}
\includegraphics[width=0.35\linewidth]{immagini/donnalemb.png}

\emph{Donna di Lemb o Statua della Dea della Morte, 3500 AC. DC 40}
\end{center}

\pagebreak

\textbf{Tabella: Effetti degli oggetti magici maledetti}\index[Tabelle]{Tabella Effetti degli oggetti magici maledetti}

\medskip

%{\small
\noindent\begin{tabularx}{\linewidth}{lX}
	\toprule
\rowcolor{gray!20}\textbf{\%} & \textbf{Inconveniente}\\
\toprule
01-02& I capelli del personaggio crescono di 2,5 cm all'ora.\\
\rowcolor{gray!20}02-04& Le unghie del personaggio crescono di 1 cm ogni 8 ore\\
05-06 & L'altezza del personaggio diminuisce di 5d10 cm \\
\rowcolor{gray!20}07-09 & L'altezza del personaggio aumenta di 5d10 cm \\
10-11 & La temperatura intorno all'oggetto è di 5° C più fredda del normale.\\
\rowcolor{gray!20}12-13 & La temperatura intorno all'oggetto è di 20° C più fredda del normale.\\
14-15 & La temperatura intorno all'oggetto è di 5° C più calda del normale.\\
\rowcolor{gray!20}16-17 & La temperatura intorno all'oggetto è di 20° C più calda del normale.\\
18-20 & Il colore dei capelli del personaggio cambia.\\
\rowcolor{gray!20}21-23 & II colore della pelle del personaggio cambia.\\
24& Il colore dei capelli del personaggio cambia ogni ora\\
\rowcolor{gray!20}25& Il colore della pelle del personaggio cambia ogni ora\\
26& Delle corna come un montone crescono sulla testa del personaggio\\
\rowcolor{gray!20}27& Un palco di corna come un alce crescono sulla testa del personaggio\\
28-29 & II personaggio ora porta un segno distintivo (un tatuaggio, una strana luminescenza ecc.).\\
\rowcolor{gray!20}30-32 & II sesso del Personaggio cambia ogni giorno all'alba.\\
33-34 & La razza o la specie del Personaggio cambiano.\\
\rowcolor{gray!20}35& II PG viene colpito da una Malattia determinata casualmente, che non può essere curata.\\
36-39 & L'oggetto emette costantemente suoni sgradevoli (lamenti, maledizioni, insulti...).\\
\rowcolor{gray!20}40& L'oggetto ha un aspetto ridicolo (colori sgargianti, forma, brilla di un alone rosa ecc.).\\
41& Un unicorno blu, visibile solo con la magia, di dimensioni piccole vola sempre attorno al Personaggio dando consigli inutili e facendo battute stupide.\\
\rowcolor{gray!20}42& Ogni giorno ti prende una improvvisa voglia e capacità di fare l'uncinetto per almeno 1 ora.\\
43-45 & II personaggio diventa estremamente possessivo nei confronti dell'oggetto.\\
\rowcolor{gray!20}46-49 & II personaggio ha una paura incontrollabile di perdere l'oggetto o che venga danneggiato.\\
50& Un Tratto viene sostituito\\
\rowcolor{gray!20}51& Il metabolismo del personaggio cambia e diventa esclusivamente carnivoro\\
52& Il metabolismo del personaggio cambia e diventa esclusivamente vegetariano\\
\rowcolor{gray!20}53-54 & II personaggio deve attaccare la creatura a lui più vicina (probabilità del 5\% ogni giorno).\\
55-57 & II personaggio rimane Stordito per 1d4 round ogni volta che l'oggetto è servito al suo scopo\\
\rowcolor{gray!20}58-60 & Il personaggio diventa sordo\\
61-64 & I Punti Ferita massimi calano di 10 permanentemente (rimanendo con un minimo di 1).\\
\rowcolor{gray!20}65& I Punti Ferita massimi calano di 20 permanentemente (rimanendo con un minimo di 1).\\
66-68 & Il PG acquisisce una Fobia a caso.\\
\rowcolor{gray!20}69-71 & TS su Volontà ogni giorno all'alba con mod. Intelligenza o subisce 1 danno a Intelligenza permanente.\\
72-74 & TS su Volontà ogni giorno all'alba o subisce 1 danno a Saggezza permanente.\\
\rowcolor{gray!20}75-77 & TS su Volontà ogni giorno all'alba con mod. Carisma o subisce 1 danno a Carisma permanente.\\
78-80 & TS su Tempra ogni giorno all'alba con mod. Forza o subisce 1 danno a Forza permanente.\\
\rowcolor{gray!20}81-83 & TS su Tempra ogni giorno all'alba con mod. Destrezza o subisce 1 danno a Destrezza permanente.\\
84-85 & TS su Tempra ogni giorno all'alba o subisce 1 danno a Costituzione permanente.\\
\rowcolor{gray!20}86-89& Il PG incomincia a parlare di se in terza persona.\\
90-92& Saurovalli, cani e gatti domestici diventano ostili.\\
\rowcolor{gray!20}93& Un Patrono farà di tutto per ucciderti.\\
94& Il PG viene teletrasportato a 10d100 Km di distanza ogni giorno all'alba.\\
\rowcolor{gray!20}95& II personaggio viene trasformato in una creatura a caso di una specie specifica (probabilità del 5\% ogni giorno).\\
96& II personaggio viene trasformato in una creatura specifica (probabilità del 5\% ogni giorno).\\
\rowcolor{gray!20}97& II personaggio non può più usare oggetti magici o Incantesimi con livello oltre 5\\
98& II personaggio non può più usare oggetti magici o Incantesimi con livello oltre 3\\
\rowcolor{gray!20}99& II personaggio non può più usare Incantesimi\\
100 & Tira due volte
\end{tabularx}
%}

\pagebreak

\section*{Licantropia}\index{Licantropia}\label{Licantropia}\hypertarget{Licantropia}{}

\begin{multicols}{2}

Le creature mannare sono umanoidi condannati a trasformarsi in animali o in ibridi animali-umanoidi al sorgere della luna piena. Questa maledizione è trasmissibile tramite morso/ferita e procreazione. Una creatura mannara è potenzialmente inconsapevole della sua maledizione fino all'arrivo della prima luna piena.

\subsection{La forma ibrida}\index{La forma ibrida}

Dopo la prima mutazione completa la creatura acquisisce la capacità di trasformarsi a volere nella creatura mannara di forma ibrida.

La forma \textbf{ibrida}, a parte le ovvie trasformazioni fisiche, concede:

\begin{itemize}[leftmargin=*] \setlength{\itemsep}{0pt}
	\item la creatura acquisisce anche il tipo Bestia e Mutaforma
	\item Aumenta di una taglia se il tipo di mannaro è più grande della taglia della creatura originaria
	\item Le Caratteristiche fisiche e Difesa aumentano di 2.
	\item Il Tiro Salvezza su Volontà aumenta di 1.
	\item Può attaccare con artiglio o morso causando 1d6 + Forza. Usa la Lista Armi Accette e Scuri ed è competente nell'arma.
	\item Acquisisce il doppio del livello in Punti Ferita temporanei indipendentemente dall'aumento di Costituzione.
	\item Acquisisce i sensi riportati dalla versione mannara della creatura.
	\item Ottiene +2d6 ad interagire con gli animali della sua forma mannara.
	\item La creatura diventa Vulnerabile all'argento.
\end{itemize}

\subsection{La forma mannara completa}\index{La forma mannara completa}

La forma \textbf{mannara completa} a parte le ovvie ed importanti trasformazioni fisiche, concede e modifica quelle della forma ibrida in questa maniera:

\begin{itemize}[leftmargin=*] \setlength{\itemsep}{0pt}
	\item Le Caratteristiche fisiche e Difesa aumentano di 3.
	\item Il Tiro Salvezza su Volontà aumenta di 2.
	\item Il danno con artiglio o morso causa 1d8 + Forza.
	\item I Punti Ferita temporanei aumentano del quadruplo del livello della creatura.
\end{itemize}

\subsection{Trasformarsi in mannaro}

Al sorgere della luna piena, dalle 22.00 alle 06.00, la creatura si trasforma nella sua versione mannara completa e la mattina non ha ricordi di ciò che ha fatto.

La creatura può resistere con un Tiro Salvezza su Tempra con DC pari a 15 + livello della creatura stessa. La trasformazione impiega 1 minuto e tornare in forma originale lascia Affaticato 2.

Solo dopo essersi trasformato in un mannaro completo sarà possibile attivare la trasformazione in forma ibrida a volere, usando 2 Azioni.

La creatura maledetta dalla licantropia da un altro mannaro tramite ferita acquisisce la capacità di trasformarsi spontaneamente in mannaro completo solo dopo 1 anno dalla prima trasformazione, solo di notte e con luna presente.

Figli di creature mannare hanno un 33\% di possibilità, per genitore mannaro, di essere licantrope naturali e quindi sin dalla nascita poter comandare i loro poteri.

\subsection*{Curare la licantropia}

La licantropia è una maledizione antica e potente e non è facile da rimuovere.
Se ferito da un mannaro deve effettuare un Tiro Salvezza come descritto dalla descrizione del mostro. Se il Tiro Salvezza iniziale fallisce allora è necessario un Rimuovi Maledizione contrastato a DC 21 + livello della creatura stessa.

\end{multicols}

\vfill

\begin{center}
	\includegraphics[width=0.45\linewidth]{immagini/William_Blake_-_Nebuchadnezzar.png}

	\medskip

	\emph{William Blake - Nebuchadnezzar, Tate Museum}
\end{center}

\pagebreak

