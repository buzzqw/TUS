\section{Avventure in Acqua}\index{Avventure in Acqua}

\label{avventure-in-acqua}
\begin{enfasi}{
Guardò il mare e capì fino a che punto era solo, adesso. (Il vecchio e il mare, Ernest Hemingway)}\end{enfasi}

\begin{multicols}{2}

L'acqua permette alle società di esistere, ma può anche distruggerle. La vita non potrebbe esistere senza di essa ma l'acqua può anche uccidere, sia annegando le persone, sia generando alluvioni e tsunami su larga scala.

\textbf{Avventure Acquatiche}

Un'avventura acquatica può aver luogo ovunque l'acqua rappresenti l'elemento principale del territorio: come paludi, fiumi, laghi, stagni, oceani, il Piano dell'Acqua e simili. Le avventure Acquatiche non richiedono che i personaggi abbiano la capacità di respirare sott'acqua, le sfide Acquatiche per avventurieri di basso livello portano ad un'avventura tensione e sensazione di pericolo.

\textbf{Adattarsi agli Ambienti Acquatici}

Le regole per il combattimento sott'acqua si applicano alle creature che non sono native di questo pericoloso ambiente come la maggior parte dei PG. Per avventure Acquatiche prolungate ed esplorazioni particolarmente in profondità, i personaggi necessiteranno dell'uso della magia per proseguire le proprie avventure. Il danno da pressione può essere totalmente evitato tramite incantesimi che offrano una resistenza o dal poter respirare sott'acqua.

\medskip

\begin{center}

%\includegraphics[width=0.4\linewidth]{immagini/avventure_acqua_grey.png}

\includegraphics[width=0.75\linewidth]{immagini/Poe's_Tales_of_Mystery-Rackham-047_grayscale.png}

\emph{Poe's Tales of Mystery-Rackham}
\end{center}

\subsection{Combattimento sott'acqua}\label{combatteresottacqua}\index{Combattimento sott'acqua}\hypertarget{combatteresottacqua}{}
Le creature che vivono sulla terra hanno considerevoli difficoltà a combattere sott'acqua. L'acqua influenza la Difesa di una creatura, i suoi Tiri per Colpire, i danni e il movimento.

\begin{itemize}[leftmargin=*] \setlength{\itemsep}{0pt}
\item
Una creatura sott'acqua perde il bonus di Destrezza alla Difesa.
\item
Una creatura sott'acqua che non sia sotto l'incantesimo \emph{Libertà di movimento} effettua i Tiri per Colpire con un -1d6 e l'avversario si considera che abbia Resistenza al danno contro armi da Taglio e Contundente.

Armi come Tridente, Lancia corta, Spada Corta, Giavellotto non hanno penalità al colpire sott'acqua in mischia.
\item
Muoversi o nuotare in acqua si considera \textbf{\emph{terreno} difficile}.
\end{itemize}

Queste penalità sono valide solo se non si ha un movimento di tipo Nuotare.

\subsubsection{Attacchi a distanza sott'acqua}\index{Attacchi sott'acqua}
Le armi da lancio sono inefficaci sott'acqua, anche quando vengono lanciate da terra. Gli attacchi con le armi a distanza subiscono -1 al danno per ogni 1 metro d'acqua che attraversano.

\subsubsection{Attacchi dalla terraferma}
Quei personaggi che nuotano, galleggiano o attraversano l'acqua in superficie, o guadano un tratto in cui l'acqua è alta almeno fino al petto, godono di copertura media.

Una creatura completamente sommersa dispone di copertura completa contro gli avversari sulla terraferma.

\subsubsection{Effetti magici in acqua}
Gli effetti magici non sono influenzati, tranne quelli che richiedono un Tiro per Colpire (vedi sopra) e gli effetti di fuoco.

Il \textbf{fuoco} non magico (incluso il fuoco dell'alchimista) non brucia sott'acqua. Gli incantesimi o gli effetti magici di fuoco sono inefficaci sott'acqua. Una creatura parzialmente sommersa ha resistenza al fuoco.

\textbf{Lanciare incantesimi sott'acqua}\index{Lanciare incantesimi sott'acqua}

Lanciare incantesimi sott'acqua può essere difficile per chi non ha la capacità di respirare sott'acqua.

Una creatura incapace di respirare sott'acqua spende tre round di trattenere il respiro per lanciare un incantesimo con componente Verbali.

Alcuni incantesimi potrebbero funzionare diversamente sott'acqua, a discrezione del Narratore.

Vedi Capitolo Ambiente per le regole sul \hyperlink{trattenereilfiato}{trattenere il respiro} (pag. \pageref{trattenereilfiato}).

\end{multicols}

\pagebreak

