\section{Sfondare ed Entrare}\index{Sfondare}\index{Entrare}\hypertarget{sfondare}{}\label{sfondarecap}

\begin{changemargin}{0.3cm}{0.3cm}\begin{enfasi}{
Nella vita di un uomo prima o poi arriva un giorno in cui, per andare dove deve andare, se non ci sono porte né finestre, gli tocca sfondare la parete. (Bernard Malamud)

\medskip

Il reato di furto sarà punito con il marchio a fuoco dei ladri, in pieno petto. In caso di reiterazione del reato saranno mozzate prima le orecchie e poi due dita delle mani. (Twoslad, Diritti e Doveri Cittadini)

}\end{enfasi}\end{changemargin}\medskip

\begin{multicols}{2}

\label{sfondare-ed-entrare}

Quando si tenta di spaccare un oggetto le scelte sono due: colpirlo con un'oggetto (arma?) o romperlo con la forza bruta.

\smallskip

\subsection{Le dimensioni contano...}

A seconda delle dimensioni dell'oggetto questo può essere più o meno facile da colpire.

\medskip

\textbf{Tabella: Taglia e Difesa degli Oggetti - Colpire un Oggetto}\index[Tabelle]{Tabella Taglia e Difesa degli Oggetti - Colpire un Oggetto}

\medskip

\begin{tabularx}{0.43\textwidth}{lll}
\textbf{Taglia} & \textbf{Mod. Difesa} & \textbf{Dimensioni}\\
\toprule
Colossale & -8 &18m+\\
Mastodontica & -6 &9-18m\\
Enorme & -4 &4-9m\\
Grande & -2 &2.4-4m\\
Media & +0 &1.2-2.4m\\
Piccola & +2 &60-120cm\\
Minuscola & +4 &30-60cm\\
Minuta & +6 &15-30cm\\
Piccolissima & +8 &5-20cm
\end{tabularx}

\medskip

\textbf{Modificatore Difesa}

Gli oggetti sono più facili da colpire delle creature poiché di solito non si muovono ma molti sono abbastanza resistenti da ignorare del danno ad ogni colpo. La Difesa di un oggetto è pari a 10 + il suo modificatore di Taglia (vedi Tabella: Colpire un Oggetto) + il suo modificatore di Destrezza (caso mai ne avesse uno).

Se si usano 3 Azioni per prendere la mira si colpisce automaticamente con un'arma da mischia.\index{Colpire un oggetto}

\subsection{Spaccare}

Nella Tabella seguente sono indicati materiali ed oggetti con relativa Durezza, Punti Ferita e DC per rompere o sfondare\index{Sfondare}\index{Rompere}

Quando si tenta di rompere o sfondare qualcosa con forza bruta piuttosto che infliggendo danni bisogna effettuare un Tiro Salvezza Tempra con Forza per capire se ci si riesce. Poiché la Durezza non influisce sulla DC per rompere l'oggetto, questo valore dipende più dal modo in cui è costruito l'oggetto che non dal materiale. La DC indicata è per oggetti comuni, un vetro spesso 20 cm non avrà DC 6 per rompersi.

Vedi anche \hyperlink{tabellaporte}{Tabella: Porte}, pag. \pageref{tabellaporte}

\end{multicols}

\textbf{Tabella: Durezza e Punti Ferita oggetti}\index[Tabelle]{Tabella Durezza e Punti Ferita oggetti}\label{durezzaoggetti}\index{Rombere oggetti}\index{Spaccare cose}\index{Rompere bauli}

\noindent\begin{tabularx}{0.98\textwidth}{lllll}
\toprule{}
\textbf{Materiale} &\textbf{Dur.}&\textbf{PF} &\textbf{DC} & \textbf{Oggetti di Esempio}\\
\hline
Carta, Vetro, Stoffa	&0	& 1	&	3&	Fogli di carta, vetro di finestra, stoffa leggera\\
Vetro					&1	& 4 &	6	&Blocco di vetro, tavolo di vetro, vaso pesante\\
Stoffa pesante 			&1	& 4 & 	12	&Armatura di stoffa, giacca pesante, sacco, tenda\\
Legno sottile			&3	& 12 &	14	&Sedia\\
Legno					&5	& 20 &	18	&Baule, tavolo\\
Corda, Cuoio			&2 	& 4	&	19	&Corda di canapa\\
Pietra sottile			&4	& 16 &	20	&Ardesia, mattonelle, baule leggero\\
Struttura in legno		&10	& 40 &	20	&Muro di legno, forziere\\
Cuoio rinforzato		&4	& 16 &	22	&Armatura di cuoio, sella, corda di canapa grossa\\
Acciaio o ferro sottile &5	& 20 &	23	&Corda di seta, scudo d'acciaio, spada corta\\
Acciaio o ferro			&9	& 36 &	26	&Catena, Armatura di metallo, forziere rinforzato\\
Struttura in pietra		&14	& 56 &	35	&Muro di pietra, spada lunga\\
Pietra					&7	& 28 &	35	&Pietra per lastricato, statua\\
Struttura in acciaio o ferro	&18	&90	&45	&Muro di piastre di ferro, spadone a due mani

\end{tabularx}

\begin{multicols}{2}

\subsection{Danneggiare gli oggetti}\index{Danneggiare oggetti}\index{Durezza}

\textbf{Durezza}: rappresenta la resistenza dell'oggetto a essere scalfitto o danneggiato. Quando si calcola il danno ad un oggetto va \textbf{sottratta la Durezza} del materiale prima di applicare il danno.

\textbf{Attacchi di Energia}: quasi tutti gli oggetti hanno Resistenza al danno verso gli attacchi di energia (fuoco, elettricità..), dividete per 2 i danni prima di applicare la Durezza mentre altri oggetti potrebbero essere particolarmente vulnerabili.

Per esempio, il fuoco potrebbe infliggere il doppio del danno a pergamene, stoffa e altri oggetti che bruciano facilmente. Oggetti e creature in cristallo o ceramica potrebbero subire danno doppio (vulnerabilità) contro un attacco sonoro.

Energia Negativa o Positiva non danneggiano gli oggetti, solo le creature viventi o meno.

\textbf{Armi Inefficaci}: Certe armi semplicemente non possono infliggere danni a certi oggetti. Per esempio, un'arma contundente non è in grado di tagliare una corda.
Allo stesso modo è decisamente difficile abbattere una porta o un muro di pietra con la maggior parte delle armi da mischia a meno che non siano specificamente ideate per farlo come picconi e martelli.

\textbf{Immunità}\index{Oggetti e Danno Critici}: Gli oggetti inanimati sono immuni ai Danni Non Letali e ai Danno Critici (ma non all'esplosione del danno). Anche gli oggetti animati, se non considerati come delle creature, hanno queste immunità.\index{Immunità ai critici degli oggetti}

\textbf{Oggetti Danneggiati}: Un oggetto danneggiato rimane pienamente funzionale fino a quando i Punti Ferita non arrivano a 0, e a quel punto è considerato distrutto. Gli oggetti danneggiati (ma non quelli distrutti) possono essere riparati da una Professione Artigiano e alcuni Incantesimi.

\textbf{Tiro Salvezza}: Gli oggetti non magici incustoditi non effettuano mai Tiro Salvezza. Si considera che abbiano fallito i loro Tiro Salvezza se disponibile.

Un oggetto custodito da un personaggio (che lo tenga in mano, lo tocchi o lo indossi) riesce nel Tiro Salvezza se il personaggio riesce nello stesso.\index{Tiro Salvezzo degli oggetti}

\textbf{Gli Oggetti Magici hanno sempre Tiro Salvezza}. Il bonus ai Tiri Salvezza su Tempra, Riflessi o Volontà di un Oggetti Magico è pari a 2 + livello x2 dell'incantesimo più potente che ospitano. Se l'oggetto non ha un incantesimo si considera un bonus di +4 per ogni +1 di bonus posseduto. Gli Oggetti Magici custoditi (indossati) effettuano il Tiro Salvezza solo se il loro possessore fallisce il proprio. Se un effetto influenza specificatamente l'oggetto magico e non chi lo indossa allora è solo l'oggetto magico ad effettuare il Tiro Salvezza.

Un \textbf{oggetto incantato}\index{Danneggiare oggetti incantati} come un arma o armatura ha Durezza, Punti Ferita e DC per romperlo aumentati della metà rispetto all'equivalente non magico.

\textbf{Oggetti animati}: Gli oggetti animati contano come creature per determinarne la Difesa e Punti Ferita (non sono considerati oggetti inanimati).

\medskip

\begin{center}
\includegraphics[width=0.7\linewidth]{immagini/portarinforzata2.png}

\emph{Porta rinforzata}
\end{center}

\subsection{Le Dimensioni contano per Sfondare...}\index{Sfondare}\index{Le Dimensioni contano per Sfondare...}

\label{sfondare}

Creature di Taglia superiore o inferiore a quella Media hanno bonus o penalità dati dalla taglia sulla prova di Forza (TS Tempra con Forza) per sfondare una porta:

\medskip

\textbf{Tabella: Modificatori prova di Forza in base alla propria taglia}\index[Tabelle]{Tabella Modificatori prova di Forza per Sfondare porta}

\medskip

\begin{tabular}{ll|ll}
\textbf{Taglia} & \textbf{Mod.}&\textbf{Taglia} & \textbf{Mod.}\\
\toprule
Piccolissima & -16& Grande & +4\\
Minuta & -12 &Enorme & +8\\
Minuscola & -8& Mastodontica & +12\\
Piccola & -4 & Colossale & +16\\
Media & +0&&
\end{tabular}

\medskip

Un \textbf{piede di porco}\index{Piede di porco} o un \textbf{ariete portatile}\index{Ariete} aumentano la probabilità del personaggio di sfondare una porta di +1d6.

\end{multicols}

\vfill

\begin{center}
\includegraphics[width=0.60\linewidth]{immagini/kitladro.png}

\emph{Kit da furfante}
\end{center}

\pagebreak

