\documentclass[a4paper,12 pt,openany]{book}
\usepackage[utf8]{inputenc}
\usepackage[T1]{fontenc}
\usepackage{amsmath}
\usepackage{amsfonts}
\usepackage{amssymb}
\usepackage{multicol}
\usepackage{graphicx}
\usepackage{xltabular}
\usepackage[a4paper]{geometry}
\geometry{verbose,tmargin=2cm,bmargin=2.5cm,lmargin=1.5cm,rmargin=2cm}
%\usepackage[absolute,overlay]{textpos}
\usepackage[absolute,overlay,showboxes]{textpos}
%\usepackage[width=457.86mm, height=303.28mm]{geometry}
\pagenumbering{gobble}
\begin{document}
	%	 \thispagestyle{empty}
	\center
	
	\begin{textblock*}{19 cm}(1cm,1cm) % larghezza box, coord X, coord Y
		\flushleft
		\textbf{Condizioni}\\
		\footnotesize 	
		
		\begin{multicols}{2}
			
\textbf{Accecato}:\index{Accecato} -2 prove su Forza e Destrezza. 
Si muove a metà della velocita' . Prova di Acrobatica con DC 12 per muoversi più veloci, se fallisci sei Prono.
Chi attacca una creatura per lei invisibile ha un -2d6 al Tiro per Colpire, la creatura invisibile che attacca una creatura che non la vede ha +1d6 al Tiro per Colpire

			
\textbf{Affaticato}\index{Affaticato}: Non pui correre o Caricare e subisce una penalità -1 a Costituzione e Destrezza. Se compie qualsiasi cosa normalmente affaticante diventa Esausto.
Recuperi con 8 ore di riposo. Se non dormi almeno 6 ore alla mattina è affaticato.
			
\textbf{Afferrato}\index{Afferrato}: Un personaggio afferrato non può muoversi, deve usare due Azioni per liberarsi (TS Tempra contrapposto). Può attaccare con armi in mischia se adeguate (difficilmente potrà usare uno spadone, alabarda.. un pugnale o spada corta è più probabile).

\textbf{Annegare/Trattenere il fiato}: \index{Annegare/Trattenere il fiato} Qualsiasi personaggio può trattenere il fiato per un numero di round pari 6 round per il suo punteggio di Costituzione, con un minimo di 3 round. Per ogni Azione compiuta la durata restante diminuisce di 1 round. Trascorso questo periodo di tempo, il personaggio deve effettuare un Tiro Salvezza su Tempra con DC 12 ogni round per continuare a trattenere il fiato. Ogni round, la DC aumenta di 1.

			
\textbf{Assordato}:\index{Assordato}\index{Sordo} Subisce penalità -2 alle prove di Iniziativa, fallisce automaticamente tutte le prove di Consapevolezza basate sul suono e si considera Distratto nel lancio degli incantesimi con componenti almeno verbali.
			
\textbf{Charmato}:\index{Charmato} una creatura charmata tratta il giocatore con un fidato amico ed alleato. Se la creatura viene minacciata o attaccata può fare un nuovo Tiro Salvezza su Volontà con un +2.
			
\textbf{Confuso}: \index{Confuso} Tirate un dado sulla tabella seguente all'inizio di ogni round della creatura confusa ad ogni round per vedere quello che la creatura fa in quel round.
			
\textbf{d100 Comportamento:}
			
01-25 Agisce normalmente
			
26-50 Balbetta
			
51-75 Si infligge 1d8 + Forza con l'arma che tiene in mano
		
76-100 Attacca la creatura più vicina 
			
\textbf{Esausto}:\index{Esausto} Un personaggio esausto si muove a velocità dimezzata e subisce penalità -2 a Costituzione e Destrezza. Dopo 1 ora di completo riposo (o Ristorare Inferiore), un personaggio esausto diventa solo Affaticato. Un personaggio Affaticato diventa Esausto compiendo una azione che normalmente lo affaticherebbe.
			
\textbf{Colpo di Grazia}:\index{Colpo di Grazia} Unica azione nel round. L'attaccante colpisce automaticamente ed infligge un colpo critico. Se il difensore sopravvive, deve superare un Tiro Salvezza su Tempra (DC 10 + danni inflitti) o muore.
			
\textbf{Intralciato}:\index{Intralciato} Un personaggio intralciato ha difficoltà di movimento, ma può comunque provare a muoversi, a meno che i legami che lo intralciano non siano ancorati a un oggetto immobile o impugnati da una forza contrapposta.
Una creatura intralciata può muoversi a velocità dimezzata ma non può Correre o Caricare, e subisce penalità -2 ai Tiri per Colpire e penalità -2 alle prove di Destrezza.
Un personaggio intralciato che cerca di lanciare un incantesimo deve superare la Difficoltà dell'incantesimo di almeno 2.
			
\textbf{Invisibile}:\index{Invisibile} Le creature invisibili non sono percepibili dalla vista.Chi attacca una creatura per lei invisibile ha un -2d6 al Tiro per Colpire, la creatura invisibile che attacca una creatura che non la vede ha +1d6 al Tiro per Colpire
			
\textbf{Paralizzato}: \index{Paralizzato}Un personaggio paralizzato è bloccato sul posto ed è incapace di muoversi od agire. Ha punteggi effettivi di Forza e Destrezza pari a -4, è Indifeso e può compiere azioni esclusivamente mentali.
Una creatura può attraversare una zona occupata da una creatura paralizzata (o morta), che sia un alleato o meno e si considera come terreno difficile.
			
\textbf{Prono}\index{Prono}: chi è prono ha un -1d6 ad attaccare ed un -4 alla Difesa. Alzarsi da prono costa 2 Azioni.
Il personaggio può eseguire una prova di Acrobatica, se è pari o superiore a 15 ed entro 20 ti permette di dimezzare questi malus e costa una Azione alzarsi, se fai 20 o più annulli i malus e costa 1 Azione alzarsi.
			
\textbf{Paura}:\index{Paura} Incantesimi, Oggetti Magici e certe creature possono influenzare i personaggi con paura. In molti casi, il personaggio deve effettuare un Tiro Salvezza su Volontà per resistere agli effetti, e un tiro fallito indica che il personaggio è scosso, spaventato o in preda al panico.
			
\textbf{Scosso}:\index{Scosso}I personaggi che sono scossi subiscono penalità -2 ai Tiri per Colpire, ai Tiri Salvezza e alle prove.
			
\textbf{Spaventato}:\index{Spaventato} I personaggi spaventati sono anche scossi, e inoltre fuggono dalla fonte della loro paura il più velocemente possibile, anche se possono scegliere la direzione di fuga.
I personaggi che non sono in grado di fuggire possono combattere (anche se continuano ad essere scossi).
			
\textbf{In Preda al Panico}:\index{In Preda al Panico} I personaggi in preda al panico sono scossi e spaventati, inoltre, hanno una probabilità del 50\% di far cadere a terra qualsiasi cosa stanno tenendo in mano e di fuggire dalla fonte del loro terrore il più in fretta possibile seguendo un percorso di fuga completamente casuale.
I personaggi in preda al panico prendono anche la condizione Accovacciato se non possono fuggire.
			
\textbf{Trattenere} il fiato: COS*6 round. Ogni Azione -1 round. Incantesimo -3 round. Poi Tiro salvezza su Tempra DC 12 +1 per round.
			
\textbf{Sanguinante}\index{Sanguinante} Il sanguinamento può essere interrotto superando una prova di Pronto Soccorso con DC 15 o con l'uso di un incantesimo che curi ferite.
Se non indicato diversamente il danno massimo da sanguinamento, anche cumulato, è di 5 PF a round.
\end{multicols}
		
	\end{textblock*}
	
	~\newpage
	
	\begin{textblock*}{4cm}(1cm,1cm) % larghezza box, coord X, coord Y
		{\textbf{Punti Fato}\\
			\footnotesize 
			(20-Livello)/5}
	\end{textblock*}	
	
	\begin{textblock*}{4cm}(1cm,2cm) % larghezza box, coord X, coord Y
		{\textbf{Morte}\\
			\footnotesize 
			PF=-10-(COS*3)}
	\end{textblock*}	

\begin{textblock*}{4cm}(1cm,3cm) % larghezza box, coord X, coord Y
\textbf{Copertura - Difesa}
Leggera +2 (>50\%)
Media +4 (<50\%)
Completa +8 (5\%)	
	\end{textblock*}	

\begin{textblock*}{4cm}(1cm,5cm) % larghezza box, coord X, coord Y

\textbf{Colpi Potenti}\\
+2 al danno - 1 CA. MAX CA/4

	\end{textblock*}	

\begin{textblock*}{4cm}(1cm,6.7cm) % larghezza box, coord X, coord Y
\textbf{Maestria del combattimento}\\
+4 Difesa -1d6 al Tiro per Colpire\\
-4 Difesa +1d6 il Tiro per Colpire \\
Non è possibile assegnare in questa maniera più di +-2d6.
	\end{textblock*}	


\begin{textblock*}{4cm}(1cm,11.5cm) % larghezza box, coord X, coord Y
\textbf{Carica}\\
3 Azioni. +1d6 a Tiro per Colpire, -4 alla Difesa,
\end{textblock*}	


\begin{textblock*}{4cm}(1cm,13.6cm) % larghezza box, coord X, coord Y
	\textbf{Attacco di Opportunità}\\
In movimento esce o attraversa la zona di mischia.
Questo attacco è una Reazione che costa una Azione.	
\end{textblock*}	
	
	\begin{textblock*}{15cm}(5.2cm,3cm) % larghezza box, coord X, coord Y
		\footnotesize 	
		\begin{tabular}{lll}
			\textbf{Difficoltà (DC)} & \textbf{Descrizione Difficoltà} & \textbf{ Competenza necessaria} \\
			DC 5                      & Estremamente facile              & Mediocre                        \\
			DC 10                     & Facile                           & Normale                         \\
			DC 15                     & Normale                          & Buona                           \\
			DC 20                     & Difficile                        & Ottimo                          \\
			DC 25                     & Molto difficile                  & Eccellente                      \\
			DC 30                     & Estremamente difficile           & Stupefacente                    \\
			DC 35                     & Quasi impossibile                & Leggendaria                     \\
			DC 40                     & Leggendaria                      & Oltre l'umano                   \\
		\end{tabular}
	\end{textblock*}
	


\begin{textblock*}{10.5cm}(5.2cm,7cm) % larghezza box, coord X, coord Y
\textbf{Azioni per Round}

	\begin{tabular}{ll}
		Eseguire un unico attacco con armi in mischia      & 1\\
		Eseguire due con armi in mischia			       & 2\\
		Eseguire più di due attacchi con armi in mischia  & 3\\
		Scoccare una freccia/dardo                         & 1\\
		Scoccare due frecce/dardo                          & 2\\
		Scoccare più di due frecce/dardo                  & 3\\
		Lanciare un'Incantesimo                            & 2\\
		Eseguire una Azione di Movimento*                  & 1\\
		Scatto   						                   & 1\\
		Alzarsi da prono                                   & 2\\
		Aiutare qualcuno                                   & 2\\
		Scambiare un discorso con qualcuno                 & 2\\
		Scambiare poche battute con qualcuno               & 0\\
		Cercare qualcosa nello zaino di pronto             & 2\\
		Usare qualcosa di appena preso dallo zaino/cintura & 1\\
		Bere una pozione tenuta alla cintura               & 1\\
		Estrarre l'arma (poi rimane estratta)              & 1\\
		Imbracciare lo scudo (poi rimane imbracciato)      & 1\\
		Usare un anello/bacchetta/verga/bastone magico     & 2\\
		Eseguire una prova su una competenza               & 2\\
		Nascondersi										   & 2\\
		Mantenere la concentrazione su un Incantesimo      & 1\\
		Salire o scendere dalla cavalcatura				   & 1\\	
		Azione Immediata                                   & {*}\\
		Azione Reazione                                    & {*}\\
		Bere una pozione tenuta in mano     	           & I\\
		Fare cadere l'arma o lo scudo					   & R\\
		Gettarsi a terra prono							   & R\\	
		Riconoscere un Incantesimo						   & R\\
	\end{tabular}
	
	\end{textblock*}

	\begin{textblock*}{4cm}(16cm,7cm) % larghezza box, coord X, coord Y
\textbf{Riposare 8 ore} \\fa recuperare COS+CA PF, minimo 1.
		\end{textblock*}
	
	
		\begin{textblock*}{4cm}(16cm,9cm) % larghezza box, coord X, coord Y
\textbf{Danni temporanei}\\ Ogni ora si recupera, con un minimo di 1 PF, il proprio valore di Costituzione in PF non letali (danni da stordimento) persi.
	\end{textblock*}


\begin{textblock*}{4cm}(16cm,12.6cm) % larghezza box, coord X, coord Y
\textbf{Difesa Sorpresi}\\NO Scudo, NO Destrezza
\end{textblock*}

\begin{textblock*}{4cm}(16cm,14.2cm) % larghezza box, coord X, coord Y
\textbf{Difesa Tocco}\\ NO Scudo, NO Armatura
\end{textblock*}



\begin{textblock*}{4cm}(16cm,14.2cm) % larghezza box, coord X, coord Y
	\textbf{Difesa Tocco}\\ NO Scudo, NO Armatura
\end{textblock*}


\begin{textblock*}{4cm}(16cm,15.7cm) % larghezza box, coord X, coord Y
\textbf{Tiro Critico}\\
Ogni qual volta hai colpito, tiri un danno aggiuntivo di arma (senza bonus magici o di Abilità o Forza, solo
arma) per ogni due volte che hai tirato 6 nel Tiro per Colpire.
\end{textblock*}

\begin{textblock*}{4cm}(16cm,20.8cm) % larghezza box, coord X, coord Y
\textbf{Esplosione del Danno}\\
Ogni qual volta dal tiro del dado dell’arma ottieni il valore massimo (minimo sul d8) del dado, ritiri il dado e sommi ancora il valore (del solo dado).
\end{textblock*}


\begin{textblock*}{4cm}(1cm,18.4cm) % larghezza box, coord X, coord Y
	\textbf{Attacchi Multipli}\\
La prima azione di attacco non ha malus mentre la seconda azione di attacco ha -5 al colpire cumulativo per attacco
\end{textblock*}


\begin{textblock*}{4cm}(1cm,18.4cm) % larghezza box, coord X, coord Y
	\textbf{Attacchi Multipli}\\
	La prima azione di attacco non ha malus mentre la seconda azione di attacco ha -5 al colpire cumulativo per attacco
\end{textblock*}




\begin{textblock*}{15cm}(5.2cm,1cm) % larghezza box, coord X, coord Y
\textbf{Rompere Oggetti - DC Forza}\\
\begin{tabular}{ll|ll}	
	Corda (2,5 cm di diametro)               & 23&	Porta di legno semplice                 & 13\\
	Porta di legno buona                    & 15&Porta di legno robusta               & 18\\
	Porta di ferro (spessa 5 cm)           & 28&	Catena                                 & 26 \\		
\end{tabular}

\end{textblock*}

%\begin{textblock*}{8cm}(1cm,22.3cm) % larghezza box, coord X, coord Y
%\textbf{Attacchi con armi a spargimento}\\
%1 2 3\\
%4 \textbf{X} 5\\
%6 7 8\\
%X si considera il bersaglio del tiro.\\
%Se il tiro manca di 5 tirate 2d6 ed un d8. 2d6 per determinare lungo la direzione indicata dal d8 a quanti metri è caduto distante dal bersaglio, ovvero contate i metri dal target.
%\end{textblock*}

\begin{textblock*}{8cm}(1cm,22.4cm) % larghezza box, coord X, coord Y
\textbf{Difesa totale}\\
2 Azioni. No Attacco, NO Incantesimi, puoi fare solo una Azione e guadagni un +8 in Difesa. Non causi Attacchi di Opportunità se attraversi la zona di mischia di un avversario.
\end{textblock*}

\begin{textblock*}{8cm}(1cm,25.4cm) % larghezza box, coord X, coord Y
\textbf{Disingaggiare}\\
2 Azioni. Sposti 3 metri. Non causi Attacchi di Opportunità se attraversi la zona di mischia di un avversario
\end{textblock*}



\begin{textblock*}{6.7cm}(9.1cm,22.4cm) % larghezza box, coord X, coord Y
\textbf{Azione di Scatto}\\
x2 Movimento. -1d6 nel Tiro per Colpire, -4 Difesa, Distratto
\end{textblock*}

\begin{textblock*}{6.7cm}(9.1cm,23.9cm) % larghezza box, coord X, coord Y
\textbf{Alzarsi da prono}\\
2 Azioni. -4 Difesa, -4 Iniziativa. Acrobatica , se è pari o superiore a 15 ed entro 20 ti permette di dimezzare questi malus e costa una Azione alzarsi,
se fai 20 o più annulli i malus e costa 1 Azione alzarsi.
\end{textblock*}


	
	~\newpage
	
	
	\begin{textblock*}{19cm}(1cm,1cm) % larghezza box, coord X, coord Y
		\footnotesize 	
		\begin{xltabular}{1\textwidth}{lllXr}
			\textbf{Arma}&\textbf{Costo}&\textbf{Danno} & \textbf{Gittata, Lista, Speciale} & kg\\
			Alabarda& 10 & G/1d10 P/T& \textbf{Lance}, \textbf{Aste}, Controcarica, Arma lunga, ED9 & 4\\
			Arco Corto& 30 & M/1d6 P& 15 metri, \textbf{Arco}, da tiro& 1\\
			Arco Corto Comp.& note*& M/Frecce& 20 metri, \textbf{Arco}, da tiro& 1.5\\
			Arco Lungo& 75 & G/Frecce& 20 metri, \textbf{Arco}, da tiro& 2\\
			Arco Lungo Comp.& note*& G/Frecce& 36 metri, \textbf{Arco}, da tiro& 2.5\\
			Ascia ad una mano& 6  & M/1d6 T& 6 metri, \textbf{Asce}, \textbf{Armi da Tiro}, Versatile& 1\\
			Ascia da battaglia& 10 & M/1d10 T&\textbf{Asce}& 3\\
			Ascia Martello& 16 & M/1d6 T/B& \textbf{Asce}& 3\\
			Balestra una mano& 100& M/Dardi& 12 metri, \textbf{Balestre}, da tiro& 1\\
			Balestra leg.& 35 & P/Dardi& 15 metri, \textbf{Balestre}, \textbf{Armi Semplici}, da tiro& 0.5\\
			Balestra pes.& 50 & G/Dardi& 20 metri \textbf{Balestre}, da tiro& 3\\
			Bastone& 3& M/1d6 B& \textbf{Armi doppie}, \textbf{Armi Semplici}, Arma lunga, Versatile& 2\\
			Bolas& 4& P/1d3 B&6 metri, \textbf{Bloccanti}, intralciato& 0.5\\
			Brandistocco& 10 & M/2d4 P/T& \textbf{Lance}, Controcarica, Arma lunga& 3\\
			Catena chiodata& 25 & G/2d4 P& \textbf{Palle rotanti}, Arma lunga& 4\\
			Falce& 18 & G/2d4 P/T& \textbf{Armi della Morte}, Arma lunga& 3\\
			Falcetto& 6& P/1d6 T& \textbf{Armi della Morte} & 1\\
			Falcione& 75 & M/2d4 T& \textbf{Armi Aggraziate}, \textbf{Lance}, ED7& 2\\
			Falcione in asta& 12 & G/1d10 P/T& \textbf{Lance}, Controcarica, Arma lunga, ED9& 3\\
			Fionda& -& P/1d4 B& 10 metri, \textbf{Archi}, da tiro& 0.5\\
			Flagello& 8& M/1d8 B& \textbf{Armi da Carceriere}, \textbf{Rompi Cranio}& 3\\
			Flagello Doppio& 90 & M/1d10 B& \textbf{Armi doppie}, \textbf{Armi da Carceriere}& 4\\
			Flagello Pesante& 15 & M/1d10 B& \textbf{Armi Doppie}, \textbf{Armi da Carceriere}& 3\\
			Frusta& 1& M/1d3 T& \textbf{Armi da Carceriere}, \textbf{Palle Rotanti}, Arma lunga& 2\\
			Giavellotto& 1& P/1d6P& 12 metri, \textbf{Aste}, \textbf{Armi Semplici}& 1.5\\
			Grande Ascia Doppia& 25 & G/1d12 T& \textbf{Asce}, \textbf{Armi doppie}, Arma lunga& 4\\
			Grosso randello& 2& M/1d8 B&\textbf{Rompi Cranio}& 2\\
			Guanto chiodato& 5& P/1d4 P&\textbf{Armi da Stordimento}& 1\\
			Katana& 300& M/1d10 T& \textbf{Spade}, ED9, Versatile& 1.5\\
			Lancia& 10 & G/1d8 P&\textbf{Lance}, Arma lunga, Controcarica& 3\\
			Lancia corta da fante& 1& M/1d6 P& 6 metri, \textbf{Armi da tiro},\textbf{ Armi Semplici},Versatile & 1.5\\
			Lancia da fante& 2& M/1d8 P&6 metri, \textbf{Lance}, \textbf{Aste}, Arma lunga, Controcarica& 2 \\
			Machete& 10 & M/1d6 T&\textbf{Armi letali} & 1\\
			Manganello& 1& P/1d6 B& \textbf{Armi da stordimento}, non letale& 0.5\\
			Martello da guerra& 5& M/1d8 B/P& 6 metri, \textbf{Rompi Cranio}& 1.5\\
			Mazza Leggera& 3& P/1d6 B/T& \textbf{Armi Leggere}, \textbf{Rompi Cranio}, \textbf{Armi Semplici}, Versatile&1\\
			Mazza Pesante& 5& M/1d8 B/T& \textbf{Rompi Cranio}& 2\\
			Morningstar& 6& M 1d8 B/P&\textbf{Rompi Cranio},\textbf{ Armi Semplici}& 1\\
			Naginata& 8& G/1d10 T&\textbf{Lance}, Arma lunga, ED9& 2\\
			Picca Leggera& 4& M/1d4 P&\textbf{Armi della morte}& 1\\
			Picca Pesante& 8& G/1d6 P&\textbf{Armi della morte}, Arma lunga& 3\\
			Pugnale& 2& P/1d4 P& 6 metri, \textbf{Armi leggere}, \textbf{Armi da tiro}, \textbf{Armi letali}, \textbf{Armi Semplici}, Versatile& 0.5\\
			Pugno/Calcio nudo& 0& P/1d4* B&Versatile& -\\
			Randello& 1& P/1d6 B&\textbf{Rompi Cranio}, \textbf{Armi Semplici}& 0.5\\
			Rete& 8& M-&3 metri, \textbf{Bloccanti}, intralciato& 1\\
			Scimitarra& 15 & M/1d6 T&\textbf{Armi Leggere}, \textbf{Armi Aggraziate}, Versatile& 1.5\\
			Spada a due lame& 100& G/1d8 T& \textbf{Armi Doppie}, \textbf{Spade}, Arma lunga, arma doppia& 3\\
			Spada bastarda& 35 & M/1d10 T&\textbf{Spade}& 2\\
			Spada Corta& 10 & P/1d6 P&\textbf{Armi Leggere}, \textbf{Spade}, Versatile& 1\\
			Spada Lunga& 15 & M/1d8 T&\textbf{Spade}& 1.5\\
			Spadone a due mani& 50 & G/2d6 T&\textbf{Spade}& 3\\
			Stocco& 20 & P/1d6 P& \textbf{Armi Leggere}, \textbf{Armi Aggraziate}, Versatile& 1\\
			Tridente& 15 & M/1d8 P/T& 3 metri, \textbf{Lance}, \textbf{Aste}, \textbf{Armi da tiro}, Arma Lunga, Controcarica& 2\\
			Urgrosh& 18 & M/1d6 T/P& \textbf{Armi Doppie}, \textbf{Lance}, Controcarica, Arma lunga & 3\\
		\end{xltabular}
		
	\end{textblock*}
	
	\begin{textblock*}{13cm}(1cm,26cm) % larghezza box, coord X, coord Y
		
		\footnotesize 	
		
		\begin{tabular}{llll}	
			
			\textbf{Nome Proiettile}   & \textbf{Numero/Costo (mo)} & \textbf{Danno/Tipo} & Peso(kg) \\
			Biglie di Marmo (fionde)   & 15/1 mo                    & 1d4 B               & 0.2      \\
			Dardi da balestra, leggeri & 10/1 mo                    & 1d6 P               & 0.1      \\
			Dardi per balestra pesante & 3/1 mo                     & 1d10 P              & 0.3      \\
			Frecce da caccia           & 20/1 mo                    & 1d6 P               & 0.1      \\
			Frecce da guerra           & 10/1 mo                    & 1d8 P               & 0.2      \\
			Sasso (fionde)             & -                          & 1d2 B               & 0.2      \\
		\end{tabular}
		
	\end{textblock*}
	
		~\newpage
		

	\begin{textblock*}{19cm}(1cm,1cm) % larghezza box, coord X, coord Y

		\begin{tabular}{llllllll}

			\textbf{Armatura} & \textbf{Costo (mo)} & \textbf{Difesa} & \textbf{Prove DES} & \textbf{Prove CM} & \textbf{Tipo} & \textbf{Mov.} & \textbf{Peso (kg)}\\
			Imbottita   		& 5    & 1   & 0  	 & 0  & L   & 0   & 4\\
			Cuoio   			& 10   & 2   & 0   & -1 & L   & 0   & 5\\
			Cuoio rinforzato   	& 25  &3  & 0   & -2 & L   & 0   & 6.5\\
			Giaco di Maglia   	& 15   & 4  & -1  & -3 & M   & 0   & 10\\
			Scaglie				& 50   & 5  & -1  & -4 & M   & 0   & 22.5\\
			Anelli 				& 150  & 6  & -1  & -5 & M   & 0   & 20\\
			Pettorale    		& 200  & 6  & -2  & -5 & M   & 0   & 10\\
			Bande   			& 250  & 7  & -2  & -6 & P   & 0   & 30\\
			Mezza armatura   	& 1200 & 8  & -2  & -7 & P   & 1   & 20\\
			da Campo			& 1400 & 9 & -3  & -7 & P   & 2   & 25\\
			Completa			& 1500 & 10  & -4  & -8 & P   & 3   & 32.5\\
		\end{tabular}
		
\end{textblock*}

		\begin{textblock*}{19cm}(1cm,7.3cm) % larghezza box, coord X, coord Y
	
	\begin{tabular}{lllllll}

		\textbf{Scudi} & \textbf{Costo} & \textbf{Difesa} & \textbf{Penalità TC} & \textbf{Penalità CM} & \textbf{Peso (kg)} & \textbf{Tipo}\\
		Brocchiero					& 5 mo  	& 1 & 0		& 1& 1  & L\\
		Scudo leggero di legno   	& 3 mo  	& 1 & 0		& 2& 2  & L\\
		Scudo leggero di metallo 	& 9  mo  	& 1 & 0		& 3& 3  & L\\
		Scudo medio legno   		& 5 mo   	& 2 & -1	& 4& 3  & M\\
		Scudo medio metallo 		& 12 mo  	& 2 & -1  	& 5& 5  & M\\
		Scudo pesante di legno   	& 7  mo  	& 3 & -2    & 6& 5  & P\\
		Scudo pesante di metallo 	& 20 mo  	& 3 & -2    & 7& 7  & P\\
	\end{tabular}
\end{textblock*}


	\begin{textblock*}{19cm}(1cm,11.7cm) % larghezza box, coord X, coord Y
\textbf{Tempi per indossare e togliere l'armatura}\index{Tabella Tempi per indossare e togliere l'armatura}\\

\begin{tabular}{llll}
	\textbf{Tipo di Armatura}& \textbf{Indossare} & \textbf{Indossare in fretta} & \textbf{Togliere}\\
	Scudo								& 1 azione 	& -     	& 1 azione\\
	Imbottita, Cuoio, Cuoio rinforzata  & 1 minuto	& 3 round  	& - \\
	Giaco di Maglia						& 1 minuto	& 5 round  & 5 round\\
	Scaglie, Anelli, Pettorale, Bande   & 4 minuti 	& 1 minuto{*}  & 1 minuto\\
	Mezza armatura, da Campo, Completa  & 4 minuti{*}{*}& 4 minuti{*}& 1d4+1 minuti\\
\end{tabular}

\end{textblock*}

	\begin{textblock*}{6cm}(1cm,15.5cm) % larghezza box, coord X, coord Y

\begin{tabular}{ll}
	\textbf{Oggetto}&\textbf{Costo}\\
	\textbf{Birra}&\\
	Boccale&4 mr\\
	Caraffa (4 litri)&2 ma\\
	\textbf{Pietanze} &\\
	Banchetto (a persona)&10 mo\\
	Carne, 1 pezzo&3 ma\\
	Formaggio, 1 pezzo&1 ma\\
	Pane (a pagnotta)&2 mr\\
	\textbf{Locanda (al giorno})&\\
	Squallida&7 mr\\
	Povera&1 ma\\
	Modesta&5 ma\\
	Agiata&8 ma\\
	Ricca&2 mo\\
	Aristocratica&4 mo\\
	\textbf{Pasto (al giorno)}&\\
	Squallido&3 mr\\
	Povero&6 mr\\
	Modesto&3 ma\\
	Agiato&5 ma\\
	Ricco&8 ma\\
	Aristocratico&2 mo\\
	\textbf{Vino}&\\
	Buono (bottiglia)&10 mo\\
	Comune (caraffa)&2 ma\\
\end{tabular}

\end{textblock*}

	\begin{textblock*}{12.5cm}(7.5cm,15.5cm) % larghezza box, coord X, coord Y

\begin{tabular}{llll}
	\textbf{Cavalcatura}&\textbf{Costo}&\textbf{Velocità}&\textbf{Kg Trasporto}\\
	
	Asino o Mulo&8 mo&12 m&210 kg\\
	Cammello&50 mo&15 m&240 kg\\
	Cavallo da Galoppo&75 mo&18 m&240 kg\\
	Cavallo da Guerra&400 mo&18 m&270 kg\\
	Cavallo da Tiro&50 mo&12 m&270 kg\\
	Elefante&200 mo&12 m&660 kg\\
	Mastino&25 mo&12 m&97,5 kg\\
	Pony&30 mo&12 m&112,5 kg\\
\end{tabular}

\end{textblock*}

	\begin{textblock*}{12.5cm}(7.5cm,20.5cm) % larghezza box, coord X, coord Y

\begin{tabular}{ll}

	\textbf{Contenitore}&\textbf{Capienza}\\
	Ampolla o Boccale&0,5 litri liquidi\\
	Barile&			160 litri liquidi, 4 cubi di 30 cm\\
	Borsa&			1 cubo di 10 cm/3 kg di oggetti\\
	Bottiglia&		1 litro di liquido\\
	Brocca o Caraffa&4 litri liquidi\\
	Canestro&		2 cubi di 30 cm/20 kg di oggetti\\
	Fiala&			120 ml di liquidi\\
	Forziere&		12 cubi di 30 cm/150 kg di oggetti\\
	Otre&			2 litri liquidi\\
	Sacco&			1 cubo di 30 cm/15 kg di oggetti\\
	Secchio&		12 litri liquidi, 1 cubo di 25 cm\\
	Vaso di Ferro&	4 litri liquidi\\
	Zaino*&			1 cubo di 30 cm/15 kg di oggetti\\
\end{tabular}
\end{textblock*}



	~\newpage
	
		\begin{textblock*}{4cm}(1cm,1cm) % larghezza box, coord X, coord Y
		{\textbf{Competenze}\\
			\footnotesize 
			\textbf{Forza}\\
			Arrampicarsi\\
			Intimidire\\
			Nuotare\\
			Saltare	\\
			\textbf{Destrezza}\\
			Acrobatica\\
			Artista della fuga\\
			Giocoliere\\
			Mani di fata\\
			Muoversi silenziosamente\\
			Nascondersi nelle ombre\\
			Usare corda	\\
			\textbf{Intelligenza}\\
			Arcana\\
			Conoscenza*\\
			Dungeon\\
			Erboristeria\\
			Disattivare congegni\\
			Falsificare\\
			Lingue\\
			Natura magica\\
			Valutare\\
			\textbf{Saggezza}\\
			Cavalcare\\
			Consapevolezza\\
			Gestire animali\\
			Natura\\
			Orientamento\\
			Percepire Emozioni\\
			Pronto soccorso\\
			Religione\\
			Seguire tracce\\
			Sopravvivenza\\
			\textbf{Carisma}\\
			Diplomazia\\
			Intrattenere\\
			Ingannare\\
			Suonare\\
			Tradizioni locali
		}
		
	\end{textblock*}	

	\begin{textblock*}{4cm}(1cm,18.5cm) % larghezza box, coord X, coord Y
\textbf{Riconoscere un incantesimo}\\ DC Arcana pari la Difficoltà dell'incantesimo

	\end{textblock*}	

	\begin{textblock*}{14.5cm}(5.5cm,1cm) % larghezza box, coord X, coord Y
	\textbf{Golen Rules}\\
	
	{\textbf{I 6 esplodono}} - se fai 6, sommi e ritiri\\
	Gli \textbf{1 portano male}, se fai 1 con il dado vale zero\\
	\textbf{Affidarsi alla sorte}. -4 punti di competenza = +1d6\\
\end{textblock*}	

	\begin{textblock*}{7cm}(5.5cm,3.2cm) % larghezza box, coord X, coord Y
	\textbf{Pronto Soccorso}\\
3 Azioni: DC 15 recuperi 1d4 PF\\
+2 TS Tempra Veleno\\
DC 15+3xSanguinamento -1 Sanguinamento
\end{textblock*}	

\begin{textblock*}{7cm}(13cm,3.2cm) % larghezza box, coord X, coord Y
\textbf{Identificare  Pozioni}\\
Erboristeria a DC 12 + fattore di rarità della pianta
\end{textblock*}	

\begin{textblock*}{7cm}(13cm,5cm) % larghezza box, coord X, coord Y
\textbf{Riconoscere oggetto magico}\\
Una prova di Arcana a Difficoltà 25 può dare indicazioni di massima sui poteri e ambiti di utilizzo.
\end{textblock*}	

	\begin{textblock*}{7cm}(5.5cm,5.8cm) % larghezza box, coord X, coord Y
\textbf{Intimidire}\\
2 Azioni e fa la prova su
Intimidire contrapposta al Tiro Salvezza su
Volontà con bonus dato dal Carisma
dell’avversario. Se il Tiro Salvezza fallisce,
l’avversario fino alla fine del round successivo ha
-1d6 al Tiro per Colpire e -2 alla Difesa contro
quell’avversario soltanto.\\
Se la prova di Intimidire fallisce in maniera
critica (l’avversario riesce di 10 o più il Tiro
Salvezza chi ha fatto la prova deve effettuare un
Tiro Salvezza su Volontà con modificatore
Carisma a DC 10+Grado di Sfida dell’avversario
o subire le medesime penalita’ come se fosse
stato intimidito. Se il tiro contrapposto riesce in
maniera critica (l’avversario fallisce di più di 10 il
Tiro Salvezza) la durata dell’effetto permane fino
a fine combattimento.
\end{textblock*}	


\begin{textblock*}{7cm}(13cm,7.5cm) % larghezza box, coord X, coord Y
\textbf{Saltare}\\
\textit{Si hanno penalità dovuta all'Armatura.}

\begin{tabular}{ll}
	\textbf{Salto in Lungo (Distanza)} & DC\\
	1.5 m                              & 5  \\
	3 m                                & 10\\
	5 m                                & 15 \\	
	7 m                                & 20 \\	
	+1,5 m                             & +5	\\
\end{tabular}

\begin{tabular}{ll}
\textbf{Salto in Alto (Altezza)} & DC\\
	0.02 m                           & 4\\
	0.5 m                            & 8\\
	1 m                              & 12\\
	1.5 m                            & 16\\
	+0.5 m                           & +4\\
\end{tabular}

In un \textbf{salto in lungo} la punta più alta del salto è pari ad un 1/4 della lunghezza saltata. Se esegui un salto in lungo di 4 metri a metà salto sei in alto di 1 metro. 

\end{textblock*}	

\begin{textblock*}{7cm}(13cm,17.7cm) % larghezza box, coord X, coord Y
\textbf{Nuotare}\\
Acqua calme DC 10.\\
Acque mosse ha DC 15\\
Acque tempestose DC 20
\end{textblock*}	

\begin{textblock*}{7cm}(13cm,19.7cm) % larghezza box, coord X, coord Y

\begin{tabular}{lll}
\textbf{Fonti di Luce} & Durata&Raggio\\
Torcia& 1 ora & 6m\\
Lanterna & 6 ore & 9m\\
\end{tabular}

\end{textblock*}	

\begin{textblock*}{9cm}(1cm,22.7cm) % larghezza box, coord X, coord Y

\textbf{Arrampicarsi}\\
\textit{Si hanno penalità dovuta all'Armatura.}

\begin{tabular}{ll}
	\textbf{Esempio di Superficie} & \textbf{DC}\\
	Grezza con appigli, mattoni sporgenti&10\\
	Albero, una corda senza nodi&15\\
	Parete liscia con appigli &20\\
	Muro perimetrale pochissimi appigli&25\\
	Parete naturale senza appigli&30\\
	Appoggiare a 2 pareti opposte&-10\\
	Appoggiare a 2 parete angolari&-5\\
	Superficie scivolosa&+5\\
\end{tabular}

\end{textblock*}	


	\begin{textblock*}{8cm}(10.5cm,22.7cm) % larghezza box, coord X, coord Y
\textbf{Sopravvivenza}
\begin{tabular}{ll}
	Se il terreno è molto morbido& DC +5\\
	Se il terreno è morbido& DC +10\\
	Se il terreno è stabile& DC +15\\
	Se il terreno è duro& DC +20\\
	Ogni 3 creature inseguite& DC -1\\
	A seconda della taglia& DC +-8\\
	Ogni 24 ore passate&DC +2\\
	Ogni ora di pioggia&DC +4\\
	Visibilità scarsa&DC +2\\
	Cerca di occultare le traccie&DC +5\\
\end{tabular}\\
\end{textblock*}


\begin{textblock*}{7cm}(5.5cm,16.6cm) % larghezza box, coord X, coord Y
	\textbf{Riconoscere un mostro}\\
	Arcana: Giganti, Costrutti, Spiriti, Mostruosità\\
	Aberrazioni, Draghi\\
	Piani: Elementali\\
	Occulto: Immondi, Spiriti, Non Morti\\
	Religione: Spiriti, Non Morti, Celestiali\\
	Dungeon: Aberrazioni, Mostruosità, Melme, creature sotterranee\\
	Natura: Bestie, Piante, Fatati\\
	DC = Grado di Sfida + 01
\end{textblock*}	


	~\newpage

\begin{textblock*}{5cm}(1cm,1cm) % larghezza box, coord X, coord Y
		\textbf{Prova di Magia}\\
3d6 + Int (o caratteristica indicata dal patrono) + Abilità		
	\end{textblock*}

\begin{textblock*}{5cm}(1cm,3cm) % larghezza box, coord X, coord Y
	\textbf{Distratto}\\
	+5 Difficoltà Incantesimo
\end{textblock*}

\begin{textblock*}{13.5cm}(6.5cm,1cm) % larghezza box, coord X, coord Y
	\textbf{Seguace}\\
2 Tratti comuni con Patrono. +2 alla prova di magia con la Scuola preferita dal Patrono.\\
\end{textblock*}

\begin{textblock*}{13.5cm}(6.5cm,2.7cm) % larghezza box, coord X, coord Y
	\textbf{Devoto}\\
+4 alla Prova di Magia nelle scuole preferite dal Patrono\\
Puo' scegliere la Caratteristica indicata dal Patrono.\\
Puo' usare l'energia preferita dal Patrono.\\
\end{textblock*}
	
\begin{textblock*}{5cm}(1cm,4cm) % larghezza box, coord X, coord Y
\textbf{Incantesimi al giorno}\\
(CM/2)+modificatore di caratteristica da incantatore
\end{textblock*}	
	 


	 

	 	~\newpage


	
\end{document}